\section{Limitations}
Despite the novelty and benefits of our method for Multimodal Review Helpfulness Prediction (MRHP) problem, it does include some drawbacks. Firstly, even though empirical results demonstrate that our approach not only works in English contexts, we have not conducted the verification in multilingual circumstances, in which product or review texts are written in different languages. If a model is corroborated to work efficiaciously in such contexts, it is capable of providing myriad benefits for practical implementation, for example, e-commerce applications can leverage such one single model for multiple cross-lingual scenarios. Furthermore, our work can also be extended to other domains. For instance, in movie assessment, we need to determine whether the review suits the material in the film, or visual scenes in the comment are consistent with the textual content. These would form our prospective future directions.

Secondly, in the MRHP problem, there are several relationships that contrastive learning could exploit to burnish the performance. In particular, performing contrastive discrimination between two sets of reviews is able to furnish the model with useful set-based representations, which consolidate general knowledge for better helpfulness prediction. Similar insights are applicable for two sets of product information. At the moment, we leave such promising perspectives for future work.