%%%%%%%%%%%%%%%%%%%%%%%%%%%%%%%%%%%%%%%%%%%%%%%%%%%%%%%%%%%%%%%%%%%%%%%%%%%%%%%%
%2345678901234567890123456789012345678901234567890123456789012345678901234567890
%        1         2         3         4         5         6         7         8

%\documentclass[letterpaper, 10 pt, journal,twoside]{IEEEtran}  % Comment this line out if you need a4paper

\documentclass[letterpaper, 10pt, conference]{ieeeconf}      % Use this line for a4 paper

\IEEEoverridecommandlockouts                              % This command is only needed if 
                                                          % you want to use the \thanks command

\overrideIEEEmargins                                      % Needed to meet printer requirements.

%In case you encounter the following error:
%Error 1010 The PDF file may be corrupt (unable to open PDF file) OR
%Error 1000 An error occurred while parsing a contents stream. Unable to analyze the PDF file.
%This is a known problem with pdfLaTeX conversion filter. The file cannot be opened with acrobat reader
%Please use one of the alternatives below to circumvent this error by uncommenting one or the other
%\pdfobjcompresslevel=0
%\pdfminorversion=4

% See the \addtolength command later in the file to balance the column lengths
% on the last page of the document
% The following packages can be found on http:\\www.ctan.org
%\usepackage{graphics} % for pdf, bitmapped graphics files
%\usepackage{epsfig} % for postscript graphics files
\usepackage{mathptmx} % assumes new font selection scheme installed
%\usepackage{times} % assumes new font selection scheme installed
\usepackage{amsmath} % assumes amsmath package installed
\usepackage{amssymb}  % assumes amsmath package installed


%\usepackage{subfigure}
\usepackage[pdftex]{graphicx}
\usepackage{amsmath}
%\usepackage{algorithmic}
\usepackage{array}
\usepackage{comment}
%\usepackage[caption=false,font=footnotesize]{subfig}
%\usepackage{url}
\usepackage[utf8]{inputenc}
\usepackage{hyperref}
\usepackage{lineno}
\usepackage{microtype}
\usepackage{graphicx}
%\usepackage[onehalf]{setspace}spacing
\usepackage{color}
\usepackage{amssymb}

\usepackage[infoshow,debugshow]{tabularx}
\usepackage{booktabs}
\usepackage{epsfig}
%\usepackage{mathptmx}
\usepackage{times}
\usepackage{amsmath}
\usepackage{float}
\usepackage{algorithm}

\usepackage{multirow}
\usepackage{lipsum}  
\usepackage{xcolor}

\usepackage{tikz}
\def\checkmark{\tikz\fill[scale=0.4](0,.35) -- (.25,0) -- (1,.7) -- (.25,.15) -- cycle;} 


%%% add mirco

\usepackage{booktabs}
\usepackage{multirow}
\usepackage{colortbl}
\usepackage{pgfplotstable}

\definecolor{Gray}{gray}{0.9}
\definecolor{battleshipgrey}{rgb}{0.52, 0.52, 0.51}


\usepackage{tikz}
\newcommand\curvedarrowtop{
    \tikz[scale=0.24]{\draw[thick,-{Triangle[scale=0.5]}]
         (0,0)to[out=270,in=270](1,0);}
    }
    
\newcommand\curvedarrowbottom{%
    \tikz[scale=0.24]{\draw[thick,-{Triangle[scale=0.5]}]
         (0,0)to[out=-270,in=-270](1,0);}%
    }
    
\usepackage{pgfplots}
\usepackage{tikz}
\usepackage{graphicx}
\usepackage{pgfplots}
\usepackage{xcolor-material}
\usepackage{adjustbox}
\usepackage{graphicx}
%\usepackage{caption}
%%%
%%% add mirco 2
\usepackage{pgfplots}
\pgfplotsset{compat=1.10}
\usepgfplotslibrary{fillbetween}

\usepackage[T1]{fontenc}
\usepackage{lmodern}

\usepackage{url}
\DeclareUrlCommand\function{\urlstyle{sf}}

\usepackage{xcolor}
\usepackage{tikz}
\usepackage{booktabs}
\usepackage{array}
\usepackage{multirow}
\usepackage{siunitx}


\definecolor{chart Offline}{gray}{.5}
\definecolor{chart Online}{RGB}{227,227,138}
\definecolor{chart Ok}{RGB}{248,172,37}
\definecolor{chart Ideal}{RGB}{1,151,0}
\definecolor{chart Over}{RGB}{0,125,234}

\newdimen\tempdim
\newcommand*{\Triangle}{%
  \settoheight{\tempdim}{L}%
  \tikz[x=\tempdim, y=\tempdim]\draw(0,0) -- (.5,.5) -- (0,1) --cycle;%
}
\newcommand*{\ChartLegend}[1]{%
  \ifdim\lastkern=1sp %
    \hspace{1em}%
  \fi
  \ChartBox{0.75em}{#1}%
  \,#1%
  \kern-1sp\kern1sp\ignorespaces
}
\newcommand*{\ChartBox}[2]{%
  \begingroup
    \settoheight{\tempdim}{L}%
    \edef\tempheight{\the\tempdim}%
    \settodepth{\tempdim}{g}%
    \edef\tempdepth{\the\tempdim}%
    \tikz[
      baseline=0pt,
      inner sep=0pt,
    ]
    \node[
      fill={chart #2},
      draw,
      rounded corners=1pt,
      anchor=base,
    ]{%
      \vphantom{g\"A}%
      %\tikzset{declare function={Log(\x)=log10(\x);}}
      \pgfmathsetlength{\tempdim}{(#1)}%
      \pgfmathsetmacro{\len}{#1*0.04}
      \kern\tempdim\relax
    };%
  \endgroup
}

%%% Mirco 3
\usepackage{pgfplots}
\usepackage{datatool}
\usetikzlibrary{matrix}
\usetikzlibrary{patterns}
\usepgfplotslibrary{groupplots}
\pgfplotsset{compat=newest}



\usepackage[noend]{algpseudocode}
\algdef{SE}[DOWHILE]{Do}{doWhile}{\algorithmicdo}[1]{\algorithmicwhile\ #1}%
\algnewcommand\algorithmicswitch{\textbf{switch}}
\algnewcommand\algorithmiccase{\textbf{case}}

%\algnewcommand\algorithmicif{\textbf{if}}
%\algnewcommand\algorithmicthen{\textbf{then}}
%\algnewcommand\algorithmicelse{\textbf{else}}

\algnewcommand\algorithmicassert{\texttt{assert}}
\algnewcommand\Assert[1]{\State \algorithmicassert(#1)}%
% New "environments"
\algdef{SE}[SWITCH]{Switch}{EndSwitch}[1]{\algorithmicswitch\ #1\ \algorithmicdo}{\algorithmicend\ \algorithmicswitch}%
\algdef{SE}[CASE]{Case}{EndCase}[1]{\algorithmiccase\ #1}{\algorithmicend\ \algorithmiccase}%
\algtext*{EndSwitch}%
\algtext*{EndCase}%

%\algdef{SE}[IF]{if}{EndIf}[1]{\algorithmicif\ #1}{\algorithmicend\ \algorithmicif}%
%\algtext*{EndSwitch}%

\pdfminorversion=4

\newcommand{\rev}[1]{\textcolor{red}{#1}}
\graphicspath{{img/}}
%\renewcommand{\baselinestretch}{0.95}

\begin{document}
\title{Bringing Online Egocentric Action Recognition into the wild}
%\title{Challenges of Egocentric Vision on Real-world Applications }

\author{Gabriele Goletto\thanks{$^{*}$The authors equally contributed to this work.  }$^{*,1}$, Mirco Planamente$^{*,1,2,3}$, Barbara Caputo$^{1,3}$, and Giuseppe Averta$^{1}$
        % <-this % stops a space
\thanks{%Manuscript received: Sept 9, 2021; Revised: Dec 8, 2021; Accepted: Jan 10, 2022. This paper was recommended for publication by Editor Markus Vincze upon evaluation of the Associate Editor and Reviewers’ comments. 
This work was supported by the Italian Ministry of University and Research under the DM1061. The research herein was carried out also with the support of the IIT HPC infrastructure (Franklin).  %This project has received funding from the EU Horizon 2020 research and innovation programme under Grants No. 871237 (Sophia) and 101017274 (Darko), and and by the Italian Ministry of Education and Research (MIUR) in the framework of the CrossLab project (Departments of Excellence). Additional support came from the PEGROGAM project (CHIST-ERA Call 2017  for  Research  Proposals).
}% <-this % stops a space
\thanks{$^{1}$  Dipartimento di Automatica e Informatica, Politecnico di Torino, Corso Duca degli Abruzzi, 24, 10124 Torino, Italy
{\tt\small name.surname@polito.it}}
\thanks{$^{2}$ Italian Institute of Technology, Genova, Italy}
\thanks{$^{3}$ Consortium Cini, Italy}
%
%Digital Object Identifier (DOI): see top of this page 
}
% The paper headers
%\markboth{IEEE Robotics and Automation Letters. Preprint Version. Accepted January, 2022}
%{Collodi \MakeLowercase{\textit{et al.}}: Few-shot grasp learning} 

\maketitle
%\thispagestyle{empty}
%\pagestyle{empty}

%%%%%%%%%%%%%%%%%%%%%%%%%%%%%%%%%%%%%%%%%%%%%%%%%%%%%%%%%%%%%%%%%%%%%%%%%%%%%%%%
\definecolor{greeno}{rgb}{0,0.502,0}
\definecolor{orangee}{RGB}{236, 173, 64}
\definecolor{bluee}{RGB}{118, 159, 182}
\definecolor{grayy}{HTML}{BAD7F2}
\definecolor{pal1}{HTML}{433a3f}
\definecolor{pal2}{HTML}{6ba368}
\definecolor{pal3}{HTML}{fce762}

\begin{abstract}

This paper presents a learning framework to estimate an agent capability and task requirement model for multi-agent task allocation.
With a set of team configurations and the corresponding task performances as the training data, linear task constraints can be learned to be embedded in many existing optimization-based task allocation frameworks.
Comprehensive computational evaluations are conducted to test the scalability and prediction accuracy of the learning framework with a limited number of team configurations and performance pairs.
A ROS and Gazebo-based simulation environment is developed to validate the proposed requirements learning and task allocation framework in practical multi-agent exploration and manipulation tasks.
Results show that the learning process for scenarios with 40 tasks and 6 types of agents uses around 12 seconds, ending up with prediction errors in the range of 0.5-2\%.



\end{abstract}

% \begin{IEEEkeywords}
% Some keywords
% \end{IEEEkeywords}

\section{Introduction}


Accurate estimates of posterior probabilities are crucial for neural networks in various Natural Language Processing (NLP) tasks~\cite{icml17,DBLP:conf/nips/Lakshminarayanan17}. For example, it would be helpful for humans if the models deployed in practice abstain or interact when they cannot make a decision with high confidence~\cite{DBLP:journals/jamia/JiangOKO12}. While Pre-trained Language Models (PLMs) have improved the performance of many NLP tasks~\cite{bert,roberta}, how to better avoid miscalibration is still an open research problem ~\cite{calibration_emnlp20,dan_roth_emnlp21}. 
\begin{table}[t!]
    \centering
    \begin{tabular}{l|p{0.65\columnwidth}}
    \hline

    %  Example 1: & It is \hlc[cyan!10]{a} \hlc[red!40]{warm} \hlc[red!60]{funny} \hlc[red!40]{engaging} \hlc[cyan!20]{film} . \\ \hline
     Positive & a fast \hlc[green!10]{funny} \hlc[green!40]{highly} \hlc[green!80]{enjoyable} movie.\\ \hline
    %  like a south of the border melrose place
     
     Negative & It's about \hlc[red!5]{following} your \hlc[green!10]{dreams} \hlc[red!10]{no} matter \hlc[red!5]{what} your \hlc[green!5]{parents} think.\\
    \hline
  \end{tabular}
    \caption{Two motivating examples with highlight explanations~\cite{SST}. The saturation of the colors signifies the magnitude. The confidence of the model should be easily recognized by looking at token attributions.}
    % \vspace{-4mm}
    \label{tab:example-m}
\end{table}
In this paper, we investigate if and how model explanations can help calibrate the model. 

Explanation methods have attracted considerable research interest in recent years for revealing the internal reasoning processes behind models~\cite{IG,Uncertainty_Aware_Attention,deeplift}. Token attribution scores generated by explanation methods represent the contribution to the prediction~\cite{diagnostic}. Intuitively, one can draw some insight for analyzing and debugging neural models from these scores if they are correctly attributed, as shown in Table~\ref{tab:example-m}. For example, when the model identifies a highly indicative pattern, the tokens involved would have high attribution scores for the predicted label and low attribution scores for other labels. Similarly, if the model has difficulty recognizing the inductive information of any class (i.e., the attribution scores are not high for any label), the model should not be highly confident. As such, the computed explanation of an instance could indicate the confidence of the model in its prediction to some extent.
 
Inspired by this, we propose a simple and effective method named \textbf{CME} that can be applied at training time and improve the performance of the confidence estimates. The estimated confidence measures how confident the model is for a specific example. Ideally, reasonable confidence estimates should have higher confidence for correctly classified examples resulting in higher attributions than incorrect ones. Hence, given an example pair during training with an inverse classification relationship, we regularize the classifier by comparing the wrong example's attribution magnitude and the correct example's attribution magnitude.

Our work is related to recent works on incorporating explanations into learning. Different from previous studies that leverage explanations to help users predict model decisions~\cite{DBLP:journals/corr/abs-2102-02201} or improve the accuracy~\cite{DBLP:conf/icml/RiegerSMY20}, we focus on answering the following question: \textit{are these explanations of black-box models useful for calibration?} If so, how should we exploit the predictive power of these explanations? Considering the model may be uninterpretable due to the nature of neural networks and limitations of explanation method~\cite{Fragile,DBLP:conf/nips/YehHSIR19}, a calibrated model by CME at least can output the unbiased confidence. Moreover, we exploit intrinsic explanation during training, which does not require designing heuristics~\cite{xiye1} and additional data augmentation~\cite{mixup21acl}.
% Are these explanations useful for calibrating the model?

We conduct extensive experiments using BERT~\cite{bert} and RoBERTa~\cite{roberta} to show the efficacy of our approach on three natural language understanding tasks (i.e., natural language inference, paraphrase detection, and commonsense reasoning) under In-Domain (ID) and Out-of-Domain (OD) settings. CME achieves the lowest expected calibration error without accuracy drops compared with strong SOTA methods, e.g.,~\citet{mixup21acl}. When combined with Temperature Scaling (TS)~\cite{icml17}, the expected calibration errors are further reduced as better calibrated posterior estimates under these two settings.


\section{Related Work}
\label{sec:related_works}

\begin{figure*}
    \centering
    \includegraphics[width=0.98\textwidth]{imgs/framework++.png}
    \caption{As illustrated, our network aims to transform the front-view monocular image to the top-view road layout. The main component of our proposed \textit{front-to-top view projection module} is the cycled view projection (CVP) and the cross-view transformer (CVT), which projects the features from the front-view domain, $X$, to the top-view domain, $X'$. In CVP, it first utilizes an MLP-based cycle structure to retain the confident features for view projection in $X''$, and then CVT correlates the features of both views to attentively enhance $X'$. \wx{Moreover, we also utilize multi-scale FTVP modules to project the front-view features on varied scales to top-down view, which brings rich spatial information for top-view features in the upsampling stream to alleviate the spatial deviation on object location estimation.}}
    \label{fig:framework}
\end{figure*}



In this section, we survey the related literature on road layout estimation, vehicle detection, and street view synthesis on top-view representation. We also introduce the recent progress of transformers on vision tasks.

\textbf{BEV-based road layout estimation and vehicle detection.} Most road scene parsing works focus on semantic segmentation \cite{fan2020sne,teichmann2018multinet,yang2018denseaspp,yu2018bisenet,fu2019dual}, while there are a few attempts that derive top-view representation for road layout \cite{2012Automatic,2016HD,Geiger14,2017Predicting,2017Cognitive,2018Learning,2019A,lu2019monocular}. 
Among these methods, Schulter et al. \cite{2018Learning} propose estimating an occlusion-reasoned road layout on top-view from a single color image by depth estimation and semantic segmentation.
% Wang et al. \cite{2019A} extend \cite{2018Learning} to infer the parameterized road layouts. 
VED \cite{lu2019monocular} proposes a variational autoencoder (VAE) model to predict road layout from a given image, but without attempting to reason about the unseen layout from observation. VPN \cite{pan2020cross} presents a cross-view semantic segmentation by transforming and fusing the observation from multiple cameras. {~\cite{philion2020lift, roddick2020predicting} directly transform features from images to 3D space and finally to bird’s-eye-view (BEV) grids.}
On the other hand, 
% the monocular image based 3D vehicle detection techniques has been developed in recent years {
many monocular image-based 3D vehicle detection techniques have been developed (e.g., \cite{2016Monocular,2017Mousavian,2020GS3D,lang2019pointpillars}). Several methods handle this problem by mapping the monocular image to the top-view. For instance, \cite{roddick2018orthographic} proposes mapping a monocular image to the top-view representation and treats 3D object detection as a task of 2D segmentation. \wx{2D-Lift~\cite{dwivedi2021bird} proposes the BEV feature transform layer to transform 2D image features to the BEV space, which exploits depth maps and 3D point cloud.} BirdGAN \cite{srivastava2019learning} also leverages adversarial learning for mapping images to bird's-eye-view. \wx{STA \cite{saha2021enabling} and Stitch \cite{can2022understanding} aggregate the temporal information to produce the final segmentation results through the transformation module.}
As another related work, \cite{wang2019monocular} does not focus on explicit scene layout estimation, focusing instead on the motion planning side. 
Most related to our work, \cite{2020MonoLayout} presents a unified model to tackle the task of road layout (static scene) and traffic participant (dynamic scene) estimations from a single image. In contrast, we propose an approach to explicitly model the large view projection that learns the spatial information to produce high-quality results.

% \textbf{Vehicle detection on top view.} 

\textbf{View transformation and synthesis.} 
%In the applications of traffic scenes, image synthesis techniques have been often applied \cite{zhu2018generative}. Relevant to our task, cross-view synthesis has been investigated in many prior works. 
Traditional methods (e.g.~\cite{lin2012vision,tseng2013image,huang2018lane}) have been proposed to handle the perspective transformation in traffic scenes. 
With the progress of deep learning-based methods, \cite{zhu2018generative} proposes a pioneering work to generate the bird's-eye-view based on the driver's view. They treat cross-view synthesis as an image translation task and adopt a GAN-based framework to accomplish it. Due to the difficulty in collecting annotation for real data, their model is trained from video game data. {\cite{abbas2019geometric} focuses exclusively on warping camera images to BEV images without performing any downstream tasks such as object detection.} 
Recent attempts~\cite{regmi2018cross,tang2019multi} on view synthesis aim to convert aerial images to street view images, or vice versa. Compared with these works, our purpose is quite different and requires not only the implicit view projection from front-view to top-view, but also the estimation of road layout and vehicle occupancies under a unified framework.

\textbf{Transformer for vision tasks.}
\wx{Convolutional neural networks(CNNs) are regarded as the most basic component in vision tasks. However, with the recent success of the Transformer~\cite{vaswani2017attention}, its ability to explicitly model pairwise interactions for elements in a sequence has been leveraged in many vision tasks, such as image classification \cite{zhang2020feature}, object detection \cite{carion2020end,zhu2020deformable}, activity recognition \cite{gavrilyuk2020actor}, and image super-resolution \cite{yang2020learning}. ViT \cite{dosovitskiy2021an} first applies the Transformer framework with non-overlapping image patches in the vision task. TNT \cite{han2021transformer} jointly leverages the inner transformer block and outer transformer block to enhance information exchange. Swin Transformer \cite{liu2021swin} obtains a larger receptive field by shifting the windows over the image. PVT \cite{wang2021pyramid} proposes a spatial reduction attention to reduce computational complexity. These models all show even more impressive modeling capabilities and achieve excellent performance. 
Inspired by these transformer-based models, our proposed cross-view transformer} attempts to establish the correlation between the features of views. In addition, we incorporate a feature selection scheme along with the non-local cross-view correlation scheme, significantly enhancing the representativeness of the features.

\section{Method}
\subsection{Problem Formulation}
A well-calibrated model is expected to output prediction confidence (e.g., the highest probability after softmax activation) comparable to or aligned with its task accuracy (i.e., empirical likelihood). For example, given 100 examples with the prediction confidence of 0.8 (or 80\%), we expect that 80 examples will be correctly classified. Following~\citet{icml17}, we estimate the calibration error by empirical approximations. Specifically, we partition all examples into $K$ bins of equal size according to their prediction confidences. Formally, for any $p\in[\ell_k,u_k)$, we define the empirical calibration error as:
\begin{equation}
\hat{\mathcal{E}}_k=\frac{1}{|\mathcal{B}_k|}\Big|\sum_{i\in\mathcal{B}_k}\big[\mathbbm{1}(\hat{y}_i=y_i)-\hat{p}_i\big]\Big|,
\end{equation}
where $y_i$, $\hat{y}_i$ and $\hat{p}_i$ are the true label, predicted label and confidence for $i$-th example, and $\mathcal{B}_k$ denotes the bin with prediction confidences bounded between $\ell_k$ and $u_k$.
To evaluate the calibration error of classifiers, we further adopt a weighted average of the calibration errors of all bins as the Expected Calibration Error (ECE)~\citep{DBLP:conf/aaai/NaeiniCH15}:
\begin{align}
    \textrm{ECE} =\sum_{k=1}^K\frac{|\mathcal{B}_k|}{n} \hat{\mathcal{E}}_{k},
    \label{eq:ece}
\end{align}
where $n$ is the example number and lower is better.
Note that the calibration goal is to minimize the calibration error without significantly sacrificing prediction accuracy. 


\subsection{Our Approach}

Generally, text classification models are optimized by Maximum Likelihood Estimation (MLE), which minimizes the cross-entropy loss between the predicted and actual probability over $k$ different classes.
To minimize the calibration error, we add a regularization term to the original cross-entropy loss as a multi-task setup.

Our intuition is that if the error of the model on example $i$ is more significant than its error on example $j$ (i.e., example $i$ is considered more difficult for the classifier), then the magnitude of attributions on example $i$ should not be greater than the magnitude of attributions on example $j$. Moreover, we penalize the magnitude of attributions with the model confidence~\cite{DBLP:conf/acl/XinTYL20}, as the high error examples also should not have high confidence. Compared to the prior post-calibration methods (e.g., temperature scaling learns a single parameter with a validation set to rescale all the logits), our method is more flexible and sufficient to calibrate the model during training.

% The magnitude of attributions is gathered by  $\ell_{2}$ normalization.
Formally, given a training set $\mathcal{D} =$ $\{(\boldsymbol{x}_{1},y_{1})$$,\cdots,$$(\boldsymbol{x}_{n},y_{n})\}$ where $\boldsymbol{x}_{i}$ is the embeddings of input tokens and $y_{i}$ is the one-hot vector corresponding to its true label, an attribution of the golden label for input $\boldsymbol{x}_i$ is a vector $\boldsymbol{
a}_i = (a_{i1},\cdots,a_{il})$, and $a_{ij}$ is defined as the attribution of $x_{ij}$ ($l$ is the length). Here, attention scores are taken as the self-attention weights induced from the start index to all other indices in the penultimate layer of the model; this excludes weights associated with any special tokens added. Then, the token attribution $a_{ij}$ is the normalized attention score~\cite{FRESH} scaled by the corresponding gradients $\nabla \alpha_{ij}= \frac{\partial \hat{y}}{\partial \alpha_{ij}}$~\cite{serrano-smith-2019-attention}. At last, our training minimizes the following loss: 
\vspace{-2mm}
\begin{equation}
\label{loss_function}
    \mathcal{L}_{CME} = \mathcal{L}_{classify} + \lambda \mathcal{L}_{calib},
\end{equation} where $\lambda$ is a weighted hyperparameter. The $L_{calib}$ is calculated as follows:
\vspace{-2mm}

\begin{align}
    \mathcal{L}_{calib} &= \sum_{1\leq i,j\leq n}\Psi_{i,j} \mathbbm{1}[e_i > e_j], \label{eqn:atten1}\\
    % \Psi_{i,j}=max\left\{ 0, t(\boldsymbol{x}_i) - t(\boldsymbol{x}_j)\right\}^{2}, \label{eqn:atten2} \\
     \Psi_{i,j} &= \max[0, t(\boldsymbol{x}_i) - t(\boldsymbol{x}_j)]^{2}, \label{eqn:atten2} \\
    t(\boldsymbol{x}_i) &=  \lVert{ a_{ij}}\rVert_2 * c_i, \label{eqn:atten3}
\end{align}
%  L_{2}\left ( a_{ij} \right )
where $e_i$ and $e_j$ are the error of example $i$ and example $j$, the confidence $c_i$ is estimated by the max probability of output~\cite{DBLP:conf/iclr/HendrycksG17}, with the L2 aggregation. The products could be further scaled by $\sqrt{l}$. 
In practice, strictly computing $L_{calib}$ for all example pairs is computationally prohibitive. Alternatively, we only consider examples from the mini-batch (similar lengths) of the current epoch. In other words, we consider all pairs where $e_i$ = 1 and $e_j$ = 0 where $e$ is calculated by using zero-one error function. The comparisons of example pairs can also be calculated from more history after every epoch or by splitting examples into groups, and we leave it to future work. 

\begin{algorithm}[t!]
 \small
\caption{{\small{Explanation-based Calibrated Training}}}\label{euclid}
 \textbf{Inputs} : Train set $\mathcal{D}$, Number of epochs $T$, Learning rate $\eta$, Optimizer $G$.
\\
\textbf{Output}: Calibrated Text Model $M$
\begin{algorithmic}[1]
%\Require
%\Require{}
%\Require{$\mathcal{Q}$: }
% \State {Let $\mathcal{Q}$ : Empirical Probability Matrix $\in \mathbb{R}^{B \times K}$}
% \State {Random initialization of $\Theta$}
\State Random Initialize $\thetav$.
\For{epoch $= 1 \ldots T$}
    \State{Split $\mathcal{D}$ into random mini-batches \{$b$\}.}
    \For{a batch $b$ from $\mathcal{D}$}{}
        \State{Backward model $M$ for $\nabla_{\thetav} \mathcal{L}_{classify}(\thetav,\mathcal{Y})$.}  
        %\State{Update $\Theta$ by SGD using Loss}
        \State{Calculate the attribution by scaled attention.}
        \State{Computes absolute value of attributions.}
        \State{Normalized it by applying \textrm{Softmax} function.}
        % \If{current step $\in \mathcal{S}$}
        %     \State{$\hat{p}$ = softmax($\Theta,\mathcal{D}$)}
        %     \State{$\mathcal{Q} \leftarrow CalEmpProb(\hat{p},B)$}
        % \EndIf
        \State{Calculate $\mathcal{L}_{CME}$ by Eqn.~\ref{loss_function},~\ref{eqn:atten1},~\ref{eqn:atten2},~\ref{eqn:atten3}.}
        \State{Optimize the model parameters $\thetav$ by G:}
        \State{\hspace*{\algorithmicindent}$\thetav \leftarrow  \thetav - \eta \nabla_{\thetav}\mathcal{L}_{CME}(\thetav,\mathcal{Y})$.}  
    \EndFor
    %\For{$x,y \in \mathcal{D}$}
    %\State{
    %\EndFor
\EndFor
\end{algorithmic}
\label{alg:alg}

\end{algorithm}

Full training details are shown in Algorithm~\ref{alg:alg}. To compute the gradient w.r.t the learnable weight independently, we retain the computation graph in the first back-propagation of classification loss. The model explanations are dynamically produced during training and then used to update the model parameters, which can be easily applied to most off-the-shelf neural networks. \footnote{Code is available here: \url{https://github.com/crazyofapple/CME-EMNLP2022/}}



\subsection{Experimental Results}
\label{sec:result}

\if 0
\begin{table*}[]
\centering
\caption{Effectiveness evaluation and running time.}
%\setlength{\tabcolsep}{5pt}
\begin{tabular}{|l|ccccccccc|ccccccccc|}
\hline
\multicolumn{1}{|c|}{Method}                  & \multicolumn{9}{c|}{Classification}                                                                                                                                                                                                                                                       & \multicolumn{9}{c|}{Clustering}                                                                                                                                                                                                                                                                                                                                                                                                               \\ \hline
\multicolumn{1}{|c|}{\multirow{2}{*}{Metric}} & \multicolumn{2}{c|}{2}                             & \multicolumn{2}{c|}{3}                             & \multicolumn{2}{c|}{4}                             & \multicolumn{2}{c|}{5}                             & \multirow{2}{*}{\begin{tabular}[c]{@{}c@{}}Time \\ (ms)\end{tabular}} & \multicolumn{2}{c|}{2}                                                                  & \multicolumn{2}{c|}{3}                                                                  & \multicolumn{2}{c|}{4}                                                                  & \multicolumn{2}{c|}{5}                                                                  & \multirow{2}{*}{\begin{tabular}[c]{@{}c@{}}Time \\ (ms)\end{tabular}} \\ \cline{2-9} \cline{11-18}
\multicolumn{1}{|c|}{}                        & \multicolumn{1}{c|}{F1} & \multicolumn{1}{c|}{Acc} & \multicolumn{1}{c|}{F1} & \multicolumn{1}{c|}{Acc} & \multicolumn{1}{c|}{F1} & \multicolumn{1}{c|}{Acc} & \multicolumn{1}{c|}{F1} & \multicolumn{1}{c|}{Acc} &                                                                       & \multicolumn{1}{c|}{ARI}                   & \multicolumn{1}{c|}{AMI}                   & \multicolumn{1}{c|}{ARI}                   & \multicolumn{1}{c|}{AMI}                   & \multicolumn{1}{c|}{ARI}                   & \multicolumn{1}{c|}{AMI}                   & \multicolumn{1}{c|}{ARI}                   & \multicolumn{1}{c|}{AMI}                   &                                                                       \\ \hline
SES (RF)                                      & \multicolumn{1}{c|}{59.3}   & \multicolumn{1}{c|}{68.9}    & \multicolumn{1}{c|}{42.4}   & \multicolumn{1}{c|}{44.6}    & \multicolumn{1}{c|}{32.1}   & \multicolumn{1}{c|}{37.2}    & \multicolumn{1}{c|}{25.6}   & \multicolumn{1}{c|}{30.3}    &                                                        213.51               & \multicolumn{1}{c|}{\multirow{2}{*}{-0.7}} & \multicolumn{1}{c|}{\multirow{2}{*}{0.3}} & \multicolumn{1}{c|}{\multirow{2}{*}{-0.1}} & \multicolumn{1}{c|}{\multirow{2}{*}{0.4}} & \multicolumn{1}{c|}{\multirow{2}{*}{-1.2}} & \multicolumn{1}{c|}{\multirow{2}{*}{1.4}} & \multicolumn{1}{c|}{\multirow{2}{*}{-0.2}} & \multicolumn{1}{c|}{\multirow{2}{*}{0.6}} & \multirow{2}{*}{43.9}                                                     \\ \cline{1-10}
SES (XGBoost)                                 & \multicolumn{1}{c|}{60.0}   & \multicolumn{1}{c|}{60.3}    & \multicolumn{1}{c|}{41.0}   & \multicolumn{1}{c|}{43.3}    & \multicolumn{1}{c|}{32.6}   & \multicolumn{1}{c|}{36.4}    & \multicolumn{1}{c|}{23.9}   & \multicolumn{1}{c|}{27.8}    &                      268.3                                                 & \multicolumn{1}{c|}{}                      & \multicolumn{1}{c|}{}                      & \multicolumn{1}{c|}{}                      & \multicolumn{1}{c|}{}                      & \multicolumn{1}{c|}{}                      & \multicolumn{1}{c|}{}                      & \multicolumn{1}{c|}{}                      & \multicolumn{1}{c|}{}                      &                                                                       \\ \hline
DIF (RF)                                      & \multicolumn{1}{c|}{69.1}   & \multicolumn{1}{c|}{69.3}    & \multicolumn{1}{c|}{57.1}   & \multicolumn{1}{c|}{59.6}    & \multicolumn{1}{c|}{50.3}   & \multicolumn{1}{c|}{53.0}    & \multicolumn{1}{c|}{47.1}   & \multicolumn{1}{c|}{48.5}    &                                             250.7                          & \multicolumn{1}{c|}{\multirow{2}{*}{20.5}} & \multicolumn{1}{c|}{\multirow{2}{*}{13.7}} & \multicolumn{1}{c|}{\multirow{2}{*}{15.6}}     & \multicolumn{1}{c|}{\multirow{2}{*}{11.8}}     & \multicolumn{1}{c|}{\multirow{2}{*}{7.9}}     & \multicolumn{1}{c|}{\multirow{2}{*}{5.2}}     & \multicolumn{1}{c|}{\multirow{2}{*}{1.3}}     & \multicolumn{1}{c|}{\multirow{2}{*}{0.8}}     & \multirow{2}{*}{50.7}                                                     \\ \cline{1-10}
DIF (XGBoost)                                 & \multicolumn{1}{c|}{70.3}   & \multicolumn{1}{c|}{75.5}    &  \multicolumn{1}{c|}{61.6}    & \multicolumn{1}{c|}{63.5}   & \multicolumn{1}{c|}{52.6}    & \multicolumn{1}{c|}{55.7}   & \multicolumn{1}{c|}{42.6}    & 
\multicolumn{1}{c|}{49.3}   &   290.8
& \multicolumn{1}{c|}{}                      & \multicolumn{1}{c|}{}                      & \multicolumn{1}{c|}{}                      & \multicolumn{1}{c|}{}                      & \multicolumn{1}{c|}{}                      & \multicolumn{1}{c|}{}                      & \multicolumn{1}{c|}{}                      & \multicolumn{1}{c|}{}                      &                                                                       \\ \hline
L2P (RF)                                      & \multicolumn{1}{c|}{67.8}   & \multicolumn{1}{c|}{72.2}    & \multicolumn{1}{c|}{55.2}   & \multicolumn{1}{c|}{56.4}    & \multicolumn{1}{c|}{46.1}   & \multicolumn{1}{c|}{48.2}    & \multicolumn{1}{c|}{41.5}   & \multicolumn{1}{c|}{44.3}    &                                           4,130.0                            & \multicolumn{1}{c|}{\multirow{2}{*}{13.2}} & \multicolumn{1}{c|}{\multirow{2}{*}{10.9}} & \multicolumn{1}{c|}{\multirow{2}{*}{7.5}}     & \multicolumn{1}{c|}{\multirow{2}{*}{6.8}}     & \multicolumn{1}{c|}{\multirow{2}{*}{2.3}}     & \multicolumn{1}{c|}{\multirow{2}{*}{1.5}}     & \multicolumn{1}{c|}{\multirow{2}{*}{-0.7}}     & \multicolumn{1}{c|}{\multirow{2}{*}{-0.1}}     & \multirow{2}{*}{76.3}                                                     \\ \cline{1-10}
L2P (XGBoost)                                 & \multicolumn{1}{l|}{68.2}   & \multicolumn{1}{l|}{67.8}    & \multicolumn{1}{l|}{58.4}   & \multicolumn{1}{l|}{59.3}    & \multicolumn{1}{l|}{47.7}   & \multicolumn{1}{l|}{49.0}    & \multicolumn{1}{l|}{40.2}   & \multicolumn{1}{l|}{42.2}    & \multicolumn{1}{l|}{4,616.7}                                                 & \multicolumn{1}{c|}{}                      & \multicolumn{1}{c|}{}                      & \multicolumn{1}{c|}{}                      & \multicolumn{1}{c|}{}                      & \multicolumn{1}{c|}{}                      & \multicolumn{1}{c|}{}                      & \multicolumn{1}{c|}{}                      & \multicolumn{1}{c|}{}                      &                                                                       \\ \hline
DeepSEI                                       & \multicolumn{1}{c|}{\textbf{84.1}}   & \multicolumn{1}{c|}{\textbf{90.4}}    & \multicolumn{1}{c|}{\textbf{80.2}}   & \multicolumn{1}{c|}{\textbf{80.4}}    & \multicolumn{1}{c|}{\textbf{63.9}}   & \multicolumn{1}{c|}{\textbf{64.2}}    & \multicolumn{1}{c|}{\textbf{53.3}}   & \multicolumn{1}{c|}{\textbf{58.8}}    &                                 343,846.7                                  & \multicolumn{1}{c|}{64.1}                      & \multicolumn{1}{c|}{55.0}                      & \multicolumn{1}{c|}{61.7}                      & \multicolumn{1}{c|}{43.8}                      & \multicolumn{1}{c|}{41.2}                      & \multicolumn{1}{c|}{40.5}                      & \multicolumn{1}{c|}{38.4}                      & \multicolumn{1}{c|}{28.7}                      &                       48.7                                                \\ \hline
\end{tabular}
\label{tab:deepmodels}
\end{table*}
\fi

\if 0
\begin{table}[h]
\centering
\caption{Impacts of stay point duration (mins) for \texttt{DeepSEI}.}
\begin{tabular}{lccccc}
\hline
Parameter & 30 & 60 & 90 & 120 & 150 \\ \hline
\# Training instances &   1,485  &   1,396  &   1,360  &   1,336  &  1,315   \\
\# Testing instances &   637  &   599  &  584   &   573  &  564   \\
Classification   &74.3     &  78.9   &  \textbf{86.1}  &  82.6   &  81.9   \\
Clustering   & 80.9    &  81.9   &  \textbf{83.2}  &   81.9  &  81.8  \\ \hline
\label{tab:duration}
\end{tabular}
\end{table}

\begin{table}[h]
\centering
\caption{Impacts of cell size (m) for \texttt{DeepSEI}.}
\begin{tabular}{lccccc}
\hline
Parameter & 100 & 200 & 300 & 400 & 500 \\ \hline
\# Location tokens & 4,460    &  3,080   &  2,378   &  1,995   &  1,707   \\
Classification   &82.3     &  \textbf{86.1}   &  83.6  &  81.5   &   79.8  \\ 
Clustering   &82.0     &  \textbf{83.2}   &  81.2  &  80.5   & 79.4    \\ \hline
\end{tabular}
\label{tab:cellsize}
\end{table}

\begin{table}[h]
\centering
\caption{Impacts of spatiality granularity for \texttt{DeepSEI}.}
\begin{tabular}{lccccc}
\hline
Parameter & 100 & 200 & 300 & 400 & 500 \\ \hline
\# Spatiality tokens & 81 & 40 & 27 & 20 & 16   \\
Classification & 84.1 & 85.9 & \textbf{86.1} & 85.6 & 83.8 \\ 
Clustering & 82.1 & 81.9 & \textbf{83.2} & 82.7 & 82.0  \\ \hline
\end{tabular}
\label{tab:spatiality}
\end{table}

\begin{table}[h]
\centering
\caption{Impacts of temporality and activity granularity for \texttt{DeepSEI}.}
\begin{tabular}{lccccc}
\hline
Parameter & 0.1 & 0.3 & 0.5 & 0.7 & 0.9 \\ \hline
\# Temporality tokens &  57   & 28    & 11    &  8   &  6   \\
\# Activity tokens &  53   &  27   &  10   &  7   &   5  \\
Classification   &72.3     &  78.6   &  \textbf{86.1}  &  83.8   &  81.6   \\ 
Clustering   &80.4     & 81.5    & \textbf{83.2}   &  81.8   & 80.8    \\\hline
\label{tab:embtokens}
\end{tabular}
\end{table}
\fi

\if 0
\begin{table}[h]
\centering
\caption{Impacts of socioeconomic granularity for \texttt{DeepLPP}.}
\begin{tabular}{lccccc}
\hline
Parameter & 50 & 100 & 150 & 200 & 250 \\ \hline
\# Region socioeconomic tokens &  3,865   &  1,932   &  1,288   &  966   &   733  \\
\# PMT tokens &  335   &  168   &   112  &   84  &   67  \\
PR-AUC  & 0.951  &  \textbf{0.972}   &  0.943   &  0.943   & 0.937         \\ \hline
\label{tab:economypar}
\end{tabular}
\end{table}

\begin{figure}[h]
\centering
\begin{tabular}{c c}
   \begin{minipage}{0.45\linewidth}
    \includegraphics[width=\linewidth]{figures/pretrain_f.pdf}
    \end{minipage}
    &
    \begin{minipage}{0.45\linewidth}
    \includegraphics[width=\linewidth]{figures/pretrain_time.pdf}
    \end{minipage}
    \\
    (a) Pre-training (Accuracy)
    &
    (b) Pre-training (Time cost)
    \\
      \begin{minipage}{0.45\linewidth}
    \includegraphics[width=\linewidth]{figures/train_f.pdf}
    \end{minipage}
    &
    \begin{minipage}{0.45\linewidth}
    \includegraphics[width=\linewidth]{figures/train_time.pdf}
    \end{minipage}
    \\
    (c) Training (Accuracy)
    &
    (d) Training (Time cost)
\end{tabular}
\vspace*{-2mm}
\caption{Training cost on Geolife.}
\label{fig:train}
\vspace*{-2mm}
\end{figure}
\fi
\noindent \textbf{(1) Effectiveness evaluation (comparison with different classifiers).}
We compare the \texttt{DeepSEI} model with the baselines. In Table~\ref{tab:deepmodels}, we report their effectiveness in terms of $F_1$-score(\%) and accuracy(\%) for classification and ARI(\%) and AMI(\%) for clustering. Overall, our \texttt{DeepSEI} model consistently outperforms the baselines, e.g., in binary classification and clustering, it outperforms the best baseline (i.e., DIF) by 22.5\% and 37.9\%, respectively. 
%
The reasons are mainly two-fold: 1) the \texttt{DeepSEI} model is with more comprehensive features to infer the users' socioeconomic statuses from three aspects, i.e., spatiality, temporality and activity; 2) the two networks that are incorporated by DeepSEI can capture the features effectively as they capture the features at both the coarse and detailed levels.
% we study a deep learning based solution to incorporate those features with two well-designed networks, i.e., deep network and recurrent network. The two networks are capable to capture the context and mobility records generated with GPS instead of the mobile phone data that are used as the existing studies. 
%
% In addition, we observe the effectiveness of XGBoost and Gradient Boosting is close to 0.6. For the other models, their PR-AUCs are below 0.4 in general, which indicates that the simple traditional machine learning models may not applicable to the real loan payment prediction task, where the real-world mobility data is variety, complexity and sequential correlation of the sampled locations, which cannot be well handled with the classifiers.

% \noindent \textbf{(2) Efficiency evaluation (prediction time).}
% We further report the average prediction time of each trajectory in Table~\ref{tab:deepmodels}.
% %
% We observe the \texttt{DeepLPP} is slower than the traditional machine learning models mainly in two aspects: 1) the \texttt{DeepLPP} incorporates more features in terms of mobility, temporality, activity and economy; 2) the \texttt{DeepLPP} contains two networks for the task, where more operations are required. Overall, the traditional models have similar running times in general, and the \texttt{DeepLPP} runs reasonably fast with the best effectiveness.
% %
% In addition, the prediction on Chengdu needs more time since its trajectory contains more stay points.

\noindent \textbf{(2) Ablation study.}
We conduct an ablation study to evaluate the effect of each network (i.e., deep network or recurrent network) and features in the \texttt{DeepSEI} model, and the comparing results are reported in Table~\ref{tab:ablation}. Overall, we can see all these networks and features contribute to the final result.
%
For the recurrent Network, w/o Recurrent Network corresponding to the case that only Deep Network is kept, the result performs the worst with $F_1$-score of 33.4\% and ARI of 60.4\%. This is because it captures a sequence of users' daily activities, which is essential to infer users' socioeconomic statuses.
%
For the deep network, we observe the spatiality diversity is with the most effect, e.g., when the spatiality is removed, the $F_1$-score is 78.8, which drops by 9.3\%. This is because users' socioeconomic statuses are highly linked to the range of their activity territory, which has been verified in previous studies~\cite{xu2018human}.

\appendix

\section{Supplemental Tables}

%\section{Hyperparameters of Other Bandit Algorithms}
%\label{sec:bandit_hyperparams}
%Table~\ref{tab:hyperparams} lists the hyperparameters for bandit algorithms other than dBE.

\newcommand\topmidheader[2]{\multicolumn{#1}{c}{\textbf{#2}}\\%
                \addlinespace[1ex]}

\newcommand{\midheader}[2]{%
        \midrule\topmidheader{#1}{#2}}

\newcommand{\specialcell}[3][c]{% 
        \begin{tabular}[#1]{@{}#2@{}}#3\end{tabular}}%

\aptLtoX[graphic=no,type=env]{\begin{table}[htb]
  \centering
  \caption{Hyperparameters of bandit algorithms}
  \label{tab:hyperparams}
  \begin{tabular}{llc}
    \toprule
    Sign & Description & Value \\
    \multicolumn{3}{c}{\textbf{UCB1}}\\
    $c$ & Parameter to control the confidence level used in $\sqrt{c \cdot {\log{t}}/{N_t(arm)}}$ & 0.5  \\
    \multicolumn{3}{c}{\textbf{Thompson Sampling}}\\
    $p(\theta)$ & Prior Distribution & $\mathcal{B}(1, 1)$ \\
    \multicolumn{3}{c}{\textbf{discounted Thompson Sampling}}\\
    $\gamma$ & Discount factor & $1-10^{-8}$ \\
    \multicolumn{3}{c}{\textbf{discounted Thompson Samplingadaptive shrinking Thompson Sampling}}\\
    $M$ & Parameter to control memory usage in a data structure ADWIN2 \cite{ADWIN} & 10 \\
    $\delta$ & Parameter to control the confidence level in a data structure ADWIN2 & $1-10^{-7}$ \\
    \multicolumn{3}{c}{\textbf{EXP-IX}}\\
    $\eta_t$ & Parameter used for weights of arms & $\sqrt{\frac{2 \cdot \log{K}}{K \cdot t}}$ \\
    \addlinespace[1ex]
    $\gamma_t$ & Parameter used for loss estimates & $\frac{\eta_t}{2}$ \\
    \multicolumn{3}{c}{\textbf{EXP3++}}\\
    $\alpha$ & Constant used in calculating $\xi_t(a)$ & $3$ \\
    $\beta$ & Constant used in calculating $\xi_t(a)$ & $256$ \\
    \bottomrule
  \end{tabular}
\end{table}}{\begin{table}[htb]
  \centering
  \caption{Hyperparameters of bandit algorithms}
  \label{tab:hyperparams}
  \begin{tabular}{llc}
    \toprule
    Sign & Description & Value \\
    \midheader{3}{UCB1}
    $c$ & \specialcell{l}{Parameter to control the confidence \\ level used in $\sqrt{c \cdot {\log{t}}/{N_t(arm)}}$} & 0.5  \\
    \midheader{3}{Thompson Sampling}
    $p(\theta)$ & Prior Distribution & $\mathcal{B}(1, 1)$ \\
    \midheader{3}{discounted Thompson Sampling}
    $\gamma$ & Discount factor & $1-10^{-8}$ \\
    \midheader{3}{adaptive shrinking Thompson Sampling}
    $M$ & \specialcell{l}{Parameter to control memory usage \\ in a data structure ADWIN2 \cite{ADWIN}} & 10 \\
    $\delta$ & \specialcell{l}{ Parameter to control the confidence \\ level in a data structure ADWIN2} & $1-10^{-7}$ \\
    \midheader{3}{EXP-IX}
    $\eta_t$ & Parameter used for weights of arms & $\sqrt{\frac{2 \cdot \log{K}}{K \cdot t}}$ \\
    \addlinespace[1ex]
    $\gamma_t$ & Parameter used for loss estimates & $\frac{\eta_t}{2}$ \\
    \midheader{3}{EXP3++}
    $\alpha$ & Constant used in calculating $\xi_t(a)$ & $3$ \\
    $\beta$ & Constant used in calculating $\xi_t(a)$ & $256$ \\
    \bottomrule
  \end{tabular}
\end{table}}

\begin{table}[htb]
  \centering
  \caption{Commit IDs of the PUTs used in our vulnerability discovery and AFL++ used as the baseline.}
  \begin{tabular}{lc}
    \toprule
    Program & Commit \\
    \midrule

    AFL++ & 32a0d6ac315 (ver ++3.14c) \\
    Bloaty &  60209eb \\
    HarfBuzz & 77eeec5 \\
    libarchive & 86c9361 \\
       libxml2 & dea91c9 \\
    MuPDF & ef3d68d \\
   PHP & fdf0455f \\
    Poppler & 6d72d82 \\
    PROJ & 76dfefe \\
    QPDF &  3794f8e \\
    libtpm2 & bc3bb26 \\
    Wireshark  & 1fc621e \\
    Xpdf & N/A (ver 4.03) \\

    \bottomrule
  \end{tabular}
\label{tab:commit-ids}
\end{table}


\begin{table}[htb]
  \centering
  \caption{Initial and theoretical maximum values of code coverage of the PUTs in FuzzBench. 
           Initial values were investigated only in the PUTs used.}
  \begin{tabular}{lcc}
    \toprule
    PUT & Initial & Maximum \\
    \midrule

bloaty\_fuzz\_target & N/A & 83114 \\
curl\_curl\_fuzzer\_http & N/A & 78362 \\
freetype2-2017 & 1517 & 26262 \\
harfbuzz-1.3.2 & N/A & 12212 \\
jsoncpp\_jsoncpp\_fuzzer & N/A & 2114 \\
lcms-2017-03-21 & 149 & 7036 \\
libjpeg-turbo-07-2017 & N/A & 9384 \\
libpcap\_fuzz\_both & 2 & 7294 \\
libpng-1.2.56 & 138 & 3736 \\
libxml2-v2.9.2 & 258 & 67994 \\
libxslt\_xpath & N/A & 51456 \\
mbedtls\_fuzz\_dtlsclient & N/A & 12888 \\
openssl\_x509 & 6026 & 54116 \\
openthread-2019-12-23 & N/A & 19846 \\
php\_php-fuzz-parser & N/A & 215210 \\
proj4-2017-08-14 & 46 & 6534 \\
re2-2014-12-09 & 1 & 3982 \\
sqlite3\_ossfuzz & 4767 & 28766 \\
systemd\_fuzz-link-parser & N/A & 1798 \\
vorbis-2017-12-11 & 410 & 4082 \\
woff2-2016-05-06 & N/A & 5708 \\
zlib\_zlib\_uncompress\_fuzzer & N/A & 910 \\

    \bottomrule
  \end{tabular}
\label{tab:fuzzbench_max_cov}
\end{table}

\begin{table}[htb]
\centering
\caption{List of unique bugs found in the 7-day trial (manually triaged).}
\begin{minipage}{\columnwidth}

\centering
\begin{tabular}{lll}
\toprule

ID & PUT & Bug Type \\
\midrule
Bug-A & bloaty & NULL Pointer Deref \\
Bug-B & harfbuzz & Out-of-bounds Read \\
Bug-C & mupdf & Assertion Fail \\
Bug-D & mupdf & NULL pointer deref \\
Bug-E & xpdf & Stack Overflow \\
Bug-F & xpdf & NULL Pointer Deref \\
Bug-G \footnote{CVE-2022-24106 is issued.} & xpdf & Use of Uninitialized Value \\
Bug-H \footnote{CVE-2022-24107 is issued.} & xpdf & Integer Overflow \\
Bug-I & php & Use-After-Free \\
Bug-J & php & Use-After-Free \\
Bug-K & php & NULL Pointer Deref \\
Bug-L & php & Use-After-Free \\ 
Bug-M & php & NULL Pointer Deref \\
Bug-N & php & Assertion Fail \\
Bug-O & php & Use-After-Free \\
Bug-P & php & Use-After-Free \\
Bug-Q \footnote{CVE-2022-23308 is issued.} & libxml2 & Use-After-Free \\
\bottomrule
\end{tabular}

\label{tab:7d-bug}
\end{minipage}
\end{table}

\begin{table*}[htb]
  \centering
  \caption{List of the PUTs used in Section~\ref{sec:banditcomparison}. If the source code of a PUT was maintained in Git, the latest version at the time of the experiment in the master (or main) branch was used for the build. The `+' sign in a version indicates that the used source code is not the official release version of the source code.}
  \renewcommand\tabularxcolumn[1]{m{#1}}
  \renewcommand{\arraystretch}{1.2}
  \begin{tabularx}{\textwidth}{lXllXc}
    \toprule
    Project & Version & Commit ID & PUT & Format of Initial Seeds & Initial Edge Coverage \\
    \midrule
    Bloaty & v1.1+ & 60209eb & fuzz\_target & Executable (e.g., ELF, PE, Mach-O) & 4773\\
    libmpeg2 & N/A & 5432dc1 & mpeg2\_dec\_fuzzer & MPEG2 & 2428 \\
    PHP & 8.0+ & fdf0455f & php-fuzz-execute & PHP source code & 25241 \\
    HarfBuzz & 3.1.0 & 77eeec5 & hb-shape-fuzzer & Font (e.g., TrueType, OpenType) & 15298 \\
    Xpdf & 4.03 & N/A & fuzz\_pdfload & PDF & 4755 \\
    libtpm2 & N/A & bc3bb26 & tpm2\_execute\_command\_fuzzer & TPM command & 3884\\
    libyaml & v0.2.5+ & f8f760f & libyaml\_dumper\_fuzzer & YAML & 1310 \\
    libzip & 1.8.0+ & bff2eb9 & zip\_read\_fuzzer & ZIP & 805 \\
    libgit2 & v1.3.0+ & 50b4d53 & download\_refs\_fuzzer & Git packet & 3911 \\
    file & 5.41+ & fcbb5d8 & magic\_fuzzer & any (e.g., Zstd compressed file) & 1171 \\
%    MuPDF & 1.19.0+ & ef3d68d & pdf\_fuzzer & PDF & 16936 \\
%    libxml2 & 2.9.12+ & dea91c9 & xml & XML & 7027 \\
    \bottomrule
  \end{tabularx}
\label{tab:put_details}
\end{table*}

%\section{Full Results of Some Experiments}
%\label{sec:full_result}

%Table~\ref{tab:alg_cmp_all}, Figure \ref{fig:vis_bandits} and Figure \ref{fig:full_ablation_time_vs_cov} show the omitted results.

\begin{table*}[htb]
\centering
\caption{Median edge coverage obtained by AFL++ and 8 versions of \OurMethodName-AFL++ in 10 PUTs after 24 h. }

\begin{tabular}{lccccccccc}
\toprule

PUT & AFL++ & UCB1 & KLUCB & TS & dTS & dBE & ADS-TS & EXP3-IX & EXP3++ \\
\midrule

bloaty & \textit{1845.5} & 2198.5 & 2246.0 & 2232.5 & 2191.0 & 2292.0 & \textbf{2340.0} & 2181.5 & 2231.5 \\
harfbuzz & \textit{13497.5} & 14031.5 & 14247.5 & 14360.5 & \textbf{14374.0} & 14067.5 & 14149.0 & 13883.0 & 13891.0 \\
xpdf & \textit{3384.0} & 3494.0 & 3812.5 & \textbf{4618.5} & 4166.5 & 3791.5 & 3902.0 & 3860.0 & 3615.0 \\
libzip & \textit{267.5} & 272.0 & 274.0 & 268.0 & 268.5 & 271.5 & \textbf{276.0} & 271.5 & 268.0 \\
libgit2 & 898.0 & 888.5 & 890.5 & 906.5 & \textbf{916.0} & 884.0 & 914.0 & 899.5 & \textit{881.0} \\
php & \textit{9841.5} & 11861.0 & 13551.5 & \textbf{14324.0} & 14187.5 & 12657.5 & 13408.0 & 11423.5 & 11828.5 \\
libmpeg2 & \textit{1873.5} & 1900.5 & 1905.0 & 1905.5 & \textbf{1906.5} & 1903.0 & \textbf{1906.5} & 1897.0 & 1902.0 \\
tpm2 & \textit{281.5} & 299.5 & 313.0 & 317.0 & \textbf{317.5} & 305.0 & 311.0 & 298.5 & 291.0 \\
libyaml & 2811.5 & 2841.0 & \textbf{2841.5} & \textit{2800.5} & 2837.0 & 2827.5 & 2831.5 & 2828.0 & 2834.5 \\
file & 830.5 & 829.5 & 828.0 & 827.0 & 827.5 & 833.5 & \textbf{840.5} & 826.5 & \textit{826.0} \\

\bottomrule

\end{tabular}

\label{tab:alg_cmp_all}
\end{table*}

\begin{table*}[htb]
\centering
\caption{P-value of Mann-Whitney's U test (Holm-Bonferroni corrected) and Vargha-Delaney's $\hat{A}_{12}$ between AFL++ and the fuzzer in the column for the evaluation conducted in Section~\ref{subsec:eval-vs-existing}. If the p-value is bold, the difference is significant in the test ($p < 0.01$). The characters `L', `M', `S' and `N' in parentheses indicate that the effect size is large, medium, small, and none, respectively, according to \cite{A12}. The `+' sign means the fuzzer in the column is superior to AFL++ when compared by rank sum as well as $\hat{A}_{12}$, and the `-' sign means the opposite.}
\begin{tabular}{lllllllllllll}
 \toprule

  & \multicolumn{2}{c}{MOpt} & \multicolumn{2}{c}{CMFuzz} & \multicolumn{2}{c}{Karamcheti} & \multicolumn{2}{c}{\HavocMAB{}} & \multicolumn{2}{c}{SLOPT} \\
  \cmidrule(r){2-3}\cmidrule(r){4-5}\cmidrule(r){6-7} \cmidrule(r){8-9} \cmidrule(r){10-11}
  PUT & $p$ & $\hat{A}_{12}$ & $p$ & $\hat{A}_{12}$ & $p$ & $\hat{A}_{12}$ & $p$ & $\hat{A}_{12}$ & $p$ & $\hat{A}_{12}$ \\
\midrule

openssl\_x509 & \textbf{ < 0.001 } & 0.82 (+L) & \textbf{ 0.023 } & 0.71 (+L) & \textbf{ < 0.001 } & 0.92 (+L) & \textbf{ < 0.001 } & 0.82 (+L) & \textbf{ < 0.001 } & 0.91 (+L) \\
re2-2014-12-09 & \textbf{ < 0.001 } & 0.18 (-L) & > 0.1 & 0.37 (-S) & > 0.1 & 0.38 (-S) & > 0.1 & 0.47 (-N) & > 0.1 & 0.52 (+N) \\
proj4-2017-08-14 & \textbf{ < 0.001 } & 0.08 (-L) & \textbf{ < 0.001 } & 0.86 (+L) & \textbf{ < 0.001 } & 0.99 (+L) & > 0.1 & 0.54 (+N) & \textbf{ < 0.001 } & 0.92 (+L) \\
sqlite3\_ossfuzz & > 0.1 & 0.55 (+N) & \textbf{ < 0.001 } & 0.85 (+L) & \textbf{ < 0.001 } & 0.93 (+L) & 0.1 & 0.68 (+M) & \textbf{ < 0.001 } & 1.00 (+L) \\
libxml2-v2.9.2 & \textbf{ < 0.001 } & 0.08 (-L) & \textbf{ < 0.001 } & 0.93 (+L) & \textbf{ < 0.001 } & 0.98 (+L) & \textbf{ < 0.001 } & 0.97 (+L) & \textbf{ < 0.001 } & 0.84 (+L) \\
freetype2-2017 & \textbf{ < 0.001 } & 0.08 (-L) & 0.094 & 0.33 (-M) & > 0.1 & 0.54 (+N) & > 0.1 & 0.52 (+N) & \textbf{ < 0.001 } & 0.79 (+L) \\
libpcap\_fuzz\_both & > 0.1 & 0.57 (+S) & \textbf{ < 0.001 } & 0.79 (+L) & \textbf{ < 0.001 } & 0.80 (+L) & \textbf{ < 0.001 } & 0.87 (+L) & \textbf{ < 0.001 } & 0.81 (+L) \\
libpng-1.2.56 & > 0.1 & 0.42 (-S) & > 0.1 & 0.36 (-M) & > 0.1 & 0.49 (-N) & > 0.1 & 0.56 (+S) & 0.049 & 0.68 (+M) \\
lcms-2017-03-21 & > 0.1 & 0.45 (-N) & \textbf{ 0.037 } & 0.70 (+M) & \textbf{ < 0.001 } & 0.85 (+L) & > 0.1 & 0.37 (-S) & \textbf{ < 0.001 } & 0.88 (+L) \\
vorbis-2017-12-11 & > 0.1 & 0.39 (-S) & > 0.1 & 0.56 (+S) & \textbf{ < 0.001 } & 0.20 (-L) & > 0.1 & 0.62 (+S) & 0.092 & 0.65 (+M) \\

\bottomrule
\end{tabular}
\label{tab:statistics}
\end{table*}

\clearpage

\section{Algorithm Overview}

\begin{algorithm}[H]

\centering
\caption{Pseudocode of \OurMethodName{}}
\label{alg:slopt}

\begin{algorithmic}[0]

\Require{\mbox{}\\
    $initial\_seeds$ -- a set of initial test cases \\
    $program$ -- a PUT to be fuzzed
}

\Ensure{\mbox{}\\
    $queue$ -- a set of valuable test cases \\
    $crashes$ -- a set of test cases that trigger crashes
}

%\begin{adjustwidth}{-9pt}{}
%\setstretch{0.85}
\vspace{5pt}

\Function{RandomMutation}{$seed, instance_{mut}, instances_{bat}$}
\State $input$ $\gets$ \Call{CopyBytesFromSeed}{$seed$}
\State $mutation$ $\gets$ \Call{SelectArm}{$instance_{mut}$}
\State $idx$ $\gets$ \Call{GetGroupIndex}{$len(input)$}
\State $batch\_size$ $\gets$ \Call{SelectArm}{$instances_{bat}[idx][mutation]$}
\For{$i$ $\gets$ $1$ \textbf{to} $batch\_size$}
    \State $pos$ $\gets$ \Call{SelectPosition}{$input$}
    \State $input$ $\gets$ \Call{ApplyOperator}{$mutation, input, pos$}
\EndFor
\State \textbf{return} $input, mutation, batch\_size$
\EndFunction

%\end{adjustwidth}

%\vspace{-6pt}

%\begin{adjustwidth}{-9pt}{}
%\setstretch{0.85}

\vspace{5pt}

\Function{MutationFuzzing}{$initial\_seeds, program$}

\State $crashes$ $\gets$ $\varnothing$
\State $queue$ $\gets$ \Call{ConstructQueue}{$initial\_seeds$}
\State $instance_{mut}$ $\gets$ \Call{CreateBanditArms}{$number\_of\_mutations$}
\For{$i$ $\gets$ $1$ \textbf{to} $5$}
 \For{$j$ $\gets$ $1$ \textbf{to} $number\_of\_mutations$}
  \State $instances_{bat}[i][j]$ $\gets$ \Call{CreateBanditInstance}{$7$}
 \EndFor
\EndFor

\State

\While{ $\neg$ \Call{UserWantsStop}{\null}}
 \State $seed$ $\gets$ \Call{SelectSeed}{$queue$}
 \State $energy$ $\gets$ \Call{DecideEnergy}{$seed$}
 \For{$i$ $\gets$ $1$ \textbf{to} $energy$}
  \State $input, mutation, batch\_size$ 
  \State $\gets$ \Call{RandomMutation}{$seed, instance_{mut}, instances_{bat}$}
  \State $result$ $\gets$ \Call{ExecutePUT}{$program, input$}
  \State $b$ $\gets$ \Call{WasInputValuable}{$result$}
  \State \Call{RewardArm}{$mutation, b$}
  \State \Call{RewardArm}{$batch\_size, b$}
  \State \Call{SaveInputIfValuable}{$queue, input, result$}
  \State \Call{SaveInputIfCrash}{$crashes, input, result$}
 \EndFor
\EndWhile
\EndFunction

%\end{adjustwidth}

\end{algorithmic}
\end{algorithm}



\begin{figure}[t]
\centering
%\vspace*{-3mm}
\begin{tabular}{c c}
   \begin{minipage}{0.47\linewidth}
    \includegraphics[width=\linewidth]{figures/pretrain_f.pdf}
    \end{minipage}
    &
    \begin{minipage}{0.47\linewidth}
    \includegraphics[width=\linewidth]{figures/pretrain_time.pdf}
    \end{minipage}
    \\
    (a) Pre-training ($F_1$-score)
    &
    (b) Pre-training (Time cost)
    \\
      \begin{minipage}{0.47\linewidth}
    \includegraphics[width=\linewidth]{figures/train_f.pdf}
    \end{minipage}
    &
    \begin{minipage}{0.47\linewidth}
    \includegraphics[width=\linewidth]{figures/train_time.pdf}
    \end{minipage}
    \\
    (c) Training ($F_1$-score)
    &
    (d) Training (Time cost)
\end{tabular}
\vspace*{-2mm}
\caption{Training cost on Geolife.}
\label{fig:train}
\vspace*{-3mm}
\end{figure}
\noindent \textbf{(7) Training time.} In Figure~\ref{fig:train}, we report the times and the corresponding effectiveness with the default setup in Section~\ref{sec:setup}. We generate 50 epochs for both pre-training and training.
% , respectively. 
We observe that the effectiveness improves with the number of epochs and the corresponding training time increases almost linearly. In pre-training, the recurrent network takes more time because it has a more complex network architecture (i.e., hierarchical LSTM). In training, the \texttt{DeepSEI} model incorporates the two networks and obtains a further improvement after 32 epochs. We observe that the \texttt{DeepSEI} model converges after 41 epochs, and we use the trained model for other experiments.

\begin{figure*}[ht]
	\centering
	\begin{tabular}{c c c c}
		\begin{minipage}{0.24\linewidth}%5.5
			\includegraphics[width=\linewidth]{figures/rich1.pdf}
		\end{minipage}
		&
		\begin{minipage}{0.23\linewidth}
			\includegraphics[width=\linewidth]{figures/rich2.pdf}
		\end{minipage}
		&
		\begin{minipage}{0.24\linewidth}
			\includegraphics[width=\linewidth]{figures/poor1.pdf}
		\end{minipage}
		&
		\begin{minipage}{0.23\linewidth}
			\includegraphics[width=\linewidth]{figures/poor2.pdf}
		\end{minipage}
		\\
		(a) User 1 (richer)
		&
		(b) User 2 (richer)
		&
		(c) User 3 (poorer)
		&
		(d) User 4 (poorer)
	\end{tabular}
	\vspace*{-3mm}
	\caption{Illustration of stay points for four users with different socioeconomic statuses.}
	\label{fig:case}
	\vspace*{-3mm}
\end{figure*}

\begin{table*}
\centering
\caption{Case study, DS, DT and DA denote the features captured via the deep network for spatiality diversity, temporality diversity and activity diversity; RT and RS denote that via the recurrent network for temporal (time bin) and semantic features.}
\vspace*{-3mm}
\setlength{\tabcolsep}{11pt}
\begin{tabular}{|c|cc|cc|cc|cc|}
\hline
Case          & \multicolumn{2}{c|}{User 1 (richer)} & \multicolumn{2}{c|}{User 2 (richer)} & \multicolumn{2}{c|}{User 3 (poorer)} & \multicolumn{2}{c|}{User 4 (poorer)} \\ \hline
DS, DT and DA & \multicolumn{2}{c|}{9.62, 1.30 and 2.16}                 & \multicolumn{2}{c|}{5.52, 3.14 and 2.20}                 & \multicolumn{2}{c|}{14.29, 4.45 and 2.77}                 & \multicolumn{2}{c|}{10.38, 4.89 and 5.12}                 \\ \hline
Stay points   & \multicolumn{1}{c|}{RT}      & RS     & \multicolumn{1}{c|}{RT}      & RS     & \multicolumn{1}{c|}{RT}      & RS     & \multicolumn{1}{c|}{RT}      & RS     \\ \hline
$s_1$       & \multicolumn{1}{c|}{18}        & residence       &  \multicolumn{1}{c|}{20}        & residence      & \multicolumn{1}{c|}{9}        & working        & \multicolumn{1}{c|}{32}        & hospital       \\ \hline
$s_2$       & \multicolumn{1}{c|}{7}        &  recreation      & \multicolumn{1}{c|}{12}        & food and drink      & \multicolumn{1}{c|}{10}        & traffic        & \multicolumn{1}{c|}{35}        & traffic        \\ \hline
$s_3$       & \multicolumn{1}{c|}{9}        & education       & \multicolumn{1}{c|}{12}        & traffic      & \multicolumn{1}{c|}{12}        & food and drink        & \multicolumn{1}{c|}{36}        & food and drink       \\ \hline
$s_4$       & \multicolumn{1}{c|}{10}        &  working      & \multicolumn{1}{c|}{14}        &    working & \multicolumn{1}{c|}{13}        & working        & \multicolumn{1}{c|}{40}        & community       \\ \hline
$s_5$       & \multicolumn{1}{c|}{13}        &  education      & \multicolumn{1}{c|}{19}        & lodging      & \multicolumn{1}{c|}{18}        & residence        & \multicolumn{1}{c|}{41}        &    residence    \\ \hline
$s_6$       & \multicolumn{1}{c|}{17}        &  residence      & \multicolumn{1}{c|}{37}        & residence     & \multicolumn{1}{c|}{7}        &  traffic        & \multicolumn{1}{c|}{34}        & attractions       \\ \hline
$s_7$       & \multicolumn{1}{c|}{9}        & working       & \multicolumn{1}{c|}{8}        &   working  & \multicolumn{1}{c|}{8}        &   working        & \multicolumn{1}{c|}{43}        & residence       \\ \hline
\end{tabular}
\label{tab:case}
\vspace*{-2mm}
\end{table*}
\noindent \textbf{(8) Case study.} We conduct a case study. We select four cases for the study, where User 1 and 2 are identified as the richer users in the same class 1, and User 3 and 4 are identified as the poorer users in another class 2.
%
In Figure~\ref{fig:case}, we visualize the locations of their stay points on the map. In Table~\ref{tab:case}, we list the features captured by the deep network and recurrent network. We observe the following insights that may explain the relationship between their mobility patterns and socioeconomic statuses.
\\
\emph{\underline{Insight 1}: Richer users tend to travel shorter.} In Table~\ref{tab:case}, we observe the richer users (User 1 and User 2) are generally with the smaller spatiality diversities (e.g., 9.62 and 5.52) than poorer users. This insight is in line with the intuition from the previous study~\cite{xu2018human}, and the reason could be that rich people are busy with work and have limited time for travelling.
\\
\emph{\underline{Insight 2}: Richer users are generally with lower temporality/activity diversity of daily activities.} Temporality/Activity diversity is an entropy-based feature to reveal the regularity of users' daily activities. 
%
In Table~\ref{tab:case}, we observe the regularity of User 1 and User 2 are high, corresponding to the smaller values. For example, in Figure~\ref{fig:case}, User 1  mainly commutes between home (``residence'' POI) and office (``working'' POI) regularly, and he/she is with the least temporality diversity 1.30. In contrast, the User 4 is irregular, e.g., he/she visits many places instead of staying somewhere and working. 
% In addition, we note that their work and rest schedules are irregular, e.g., the User 1 usually visit some restaurants (``F\&D'' POI) near the Shenzhen airport in the early morning, which is reflected on a larger temporality diversity, i.e., 6.29.
\\
\emph{\underline{Insight 3}: Richer users are with secure jobs.} We infer the users' employment statuses based on the data extracted from their stay points. We infer that User 1 is with a steady job since he/she works (at 09:00 am - 05:00 pm) and stay homes (at 05:00 pm - 07:00 am) regularly. In this situation, he/she has a stable source of income (e.g., we infer that he/she may be a faculty at an university based on the stay points on the map), and the status is reflected on his/her house price data accordingly.
%
% may be a driver using the loaned car because the user is generally visit the places near the Shenzhen airport in the early morning, where the travel regions near the airport are wealthy (i.e., 70,700 rmb/$m^2$) but his/her residence (i.e., rmb/$m^2$) is poor.


\section{Conclusion}
In this paper, we extend the idea of SynGEC \cite{zhang2022syngec} and propose the CSynGEC approach to enhance GEC models by exploiting tailored constituent-based syntax. Experimental results show that incorporating constituent-based syntax produced by a GEC-oriented constituency parser can effectively help GEC models. 
Furthermore, we attempt to combine dependency-based and constituent-based syntax from both intra-model and inter-model aspects, and find that simultaneously using two kinds of syntax leads to more obvious improvement.


\bibliographystyle{IEEEtran}
\bibliography{egbib.bib}

\end{document}
