\section{Conclusions}
With this paper, we aim to promote a new line of research for egocentric vision, which considers real-world application limits such as hardware restrictions, cross-domain scenarios, and online inference on untrimmed data. We provide a new benchmark that takes into consideration all of these aspects. Then, we also propose an innovative approach that is capable of addressing all of them. Our solution shows how to tackle the problem of missing supervision action boundaries as an anomaly detection in time. We also introduce a two-fold aggregator technique to manage the presence of overlapping segments. Finally, we highlight the importance of our study by illustrating how neglecting this limitation makes it impossible to deploy a model capable of operating in real time on a device with limited hardware requirements. We believe that anomaly detection-based strategies and aggregator solutions represent two challenging research topics for future developments, opening interesting perspectives for the egocentric computer vision community.