\section{Conclusions}
\rev{The purpose of this work is to investigate and highlight the limitations that mainstream egocentric vision models show in realistic usecases, where computational time and power are limited. We promote a new line of research for FPAR, which considers real-world application limits such as hardware restrictions, cross-domain scenarios, and online inference on untrimmed data.\\
We provide: i) a new benchmark to assess the challenges of real world deployment, and ii) a novel approach capable to bring FPAR models on low-power devices (edge computing), tackling the presence of overlapping actions and the absence of supervision on action boundaries for real world usage. %Our solution shows how to tackle the problem of missing supervision action boundaries as an anomaly detection in time. 
%We also introduce a two-fold aggregator technique to manage the presence of overlapping actions. 
%Finally, we highlight the importance of our study by illustrating how neglecting this limitation makes it impossible to deploy a model capable of operating in real time on a device with limited hardware requirements. 
%Future works are invited to consider the challenges listed in this work, and to foster methods for continuous adaptations on the edge, to pursue real-time adaptation of the model during the untrimmed video processing, especially in the presence of changes of environment characteristics.
% Future research is encouraged to tackle the challenges identified in this study by developing methods that facilitate continuous adaptation at the edge. These methods should aim to enable real-time updates of machine learning models during the processing of untrimmed video, particularly in cases where the environment undergoes changes.
In light of the challenges discussed in this work, we encourage future researchers to devote attention to designing innovative approaches that allow real-time adaptation of the model on the edge during the processing of untrimmed videos, particularly in the presence of changes in environmental conditions.}



%\rev{We believe that anomaly detection-based strategies and aggregator solutions represent powerful tools to enable video processing on the edge, and that this task may significantly benefit from Unsupervised Domain Adaption (UDA) or Test Time Adaptation (TTA) techniques to properly address the challenges that arises when tackling scene understanding from untrimmed videos. Future works will consider such challenges, with the development of method for the continuous adaptation of the model during the untrimmed video processing.}