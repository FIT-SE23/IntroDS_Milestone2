\documentclass[journal,twoside,web]{IEEEtran}
% \documentclass[journal,twoside,web]{ieeecolor}
% \usepackage{generic}
\usepackage{cite}
\usepackage{amsmath,amssymb,amsfonts}
\usepackage{algorithmic}
\usepackage{graphicx}
\usepackage{textcomp}
\let\proof\relax
\let\endproof\relax
\usepackage{amsthm}
\renewcommand{\qedsymbol}{$\blacksquare$}

\usepackage{cuted} 
\usepackage{algorithm}
\usepackage{mathtools}
\usepackage{bbm}
\usepackage{caption}
\usepackage{subcaption}

\usepackage{nameref,cleveref}

\Crefname{thm}{Theorem}{Theorems}
\Crefname{prop}{Proposition}{Propositions}
\Crefname{lem}{Lemma}{Lemmas}
\Crefname{cor}{Corollary}{Corollaries}
\Crefname{defn}{Definition}{Definitions}
\Crefname{assump}{Assumption}{Assumptions}
\Crefname{conj}{Conjecture}{Conjectures}

\Crefname{equation}{}{}

\Crefname{figure}{Fig.}{Figs.}
\creflabelformat{figure}{#2\textup{#1}#3}

\theoremstyle{remark}
\newtheorem{thm}{Theorem} % a statement that has been proven to be true
\newtheorem{prop}[thm]{Proposition} % a less important but interesting true statement
\newtheorem{lem}{Lemma} % less important theorem, helpful in proof of other results
\newtheorem{cor}{Corollary} % true statement, simple deduction from a theorem or proposition
\newtheorem{defn}{Definition} % an explanation of the mathematical meaning of a word
\newtheorem{assump}{Assumption} % assumptions
\newtheorem{conj}{Conjecture} % conjecture, unproven statement
\newtheorem{rem}{Remark}


\def\BibTeX{{\rm B\kern-.05em{\sc i\kern-.025em b}\kern-.08em
    T\kern-.1667em\lower.7ex\hbox{E}\kern-.125emX}}

\begin{document}
\title{Correct-by-Construction Control Design for Mixed-Invariant Systems in Lie Groups}
\author{Li-Yu Lin, James Goppert, and Inseok Hwang
\thanks{This work is supported in part by NASA University Leadership Initiative Project on Secure and Safe Assured Autonomy (S2A2) under Grant 80NSSC20M0161.}
\thanks{L. Lin, J. Goppert, and I. Hwang are with the School of Aeronautics and Astronautics, Purdue University, West Lafayette, IN 47906, USA (e-mail: lin1191@purdue.edu; jgoppert@purdue.edu; ihwang@purdue.edu).}}

\maketitle

\begin{abstract}
In this paper, we use the derivative of the exponential map to derive the exact evolution of the logarithm of the tracking error for mixed-invariant systems. Following correct-by-construction software paradigm, we propose an invariant control law for mixed-invariant systems, with application to Unmanned Aerial Systems (UASs), that is designed for efficient safety verification. We derive the nonlinear distortion matrix in the transformed differential equation in the Lie algebra and express the distortion matrix in a series form for any matrix Lie group and in a closed-form for the SE(2) Lie group. Given the input distortion, we employ dynamic inversion to linearize the evolution of error dynamics and apply a linear control strategy. We employ Linear Matrix Inequalities (LMIs) to bound the tracking error given a bounded disturbance amplified by the distortion matrix and leverage the tracking error bound to create flow pipes for the creation of a Polyhedral Invariant Hybrid Automaton (PIHA) model. We demonstrate the usefulness of our method by applying it to a simplified holonomic aircraft and nonholonomic rover with polynomial-based path planning methods.
\end{abstract}

\begin{IEEEkeywords}
Correct-by-Construction, Lie Group Theory, Mixed-Invariant Systems, Linear Matrix Inequality, Safety Verification
\end{IEEEkeywords}

\section{Introduction}
\label{sec:introduction}
\IEEEPARstart{E}{nsuring} mission safety of unmanned vehicle systems is challenging due to the presence of disturbances, such as wind, and model uncertainties due to variations between individual vehicles. Urban Air Mobility (UAM) is a motivating example, which is a concept for air transportation, including package delivery, weather monitoring, passenger transport, etc~\cite{thipphavong2018urban}. The UAM system has various types of Unmanned Aerial Systems (UASs), such as fixed-wing aircraft and multirotors; therefore, the concurrent interaction of UASs is complicated in the UAM system. Since the unmanned vehicle can be modeled as a hybrid system, a hybrid model checker can be used for Verification and Validation (V$\&$V) of safety and liveness properties for UASs in the UAM system~\cite{baier2008principles}. Approximating the system as a Polyhedral Invariant Hybrid Automaton (PIHA) model is an efficient way to verify the system~\cite{chutinan2003computational, doyen2018verification, goppert2019security}. Constructing a PIHA model allows verification of universal Computation Tree Logic (CTL) specifications~\cite{chutinan1999verification,chutinan2001verification}. The CTL specification can be used for verifying the safety properties through reachability analysis. To construct a PIHA model, barrier certificates are insufficient, since they can only ensure scalable and provably collision-free behavior~\cite{prajna2004safety, prajna2006barrier, wang2017safety, ames2019control}, and do not compute a flow pipe which is a collection of continuous-time trajectories starting from a set of initial states~\cite{chutinan1999verification, chutinan2003computational, goppert2019security}. A flow pipe of the possible vehicle trajectories can be created using the reference trajectory and the invariant set of the error system, since an invariant set is a set that once a trajectory enters the trajectory cannot leave.

The creation of the PIHA model requires the construction of invariant sets around UAM reference trajectories~\cite{chutinan2003computational, doyen2018verification, goppert2019security}. Several methods exist to construct the invariant set for nonlinear systems. One of the methods, FaSTrack in~\cite{chen2021fastrack}, computes the Hamilton-Jacobi (HJ) reachability of the error dynamics between a tracking system and a planning system, which is a verification method for guaranteeing the performance and safety of systems. The result from HJ reachability is used for computing backward reachable sets (BRSs). Since solving the HJ partial differential equations (PDEs) has a high computational complexity, the BRSs are computed offline. If the control authority of the tracking system is powerful enough to successfully track the planning system, the result of the HJ PDEs will converge and there will exist a tracking error bound~\cite{chen2021fastrack}. The tracking error bound is a reachable set in the relative error state; therefore, it can be thought of as a safety bubble around the reference trajectory that the vehicle is guaranteed to stay within, which can also be seen as the invariant set of the system in the error frame.

The optimal control in HJ reachability is calculated using Pontryagin's maximum principle~\cite{clarke2013functional}. Since the planner and tracker in HJ reachability are formulated as a pursuit-evasion game, the control of the planner is the worst-case scenario, which is considered as another disturbance during the computation that will lead to a larger tracking error bound compared to a known planner. This approach employs a hybrid safety controller, which is a combination of any performance controller and the safety controller, to stay within the tracking error bound under the worst-case planner and disturbance. The safety controller is a function/lookup table from the calculation of HJ reachability~\cite{chen2021fastrack}. The HJ approach is independent of the performance controller used when the safety controller is not employed, which is advantageous for implementation on systems, requiring only the safety controller to be added. However, the optimality of the control strategy with regard to input utilization is ignored in this approach. In addition, switching controllers can lead to abrupt input changes which can degrade actuators and can be difficult to model accurately.

% Lyapunov and LMI
Another method for constructing invariant sets of nonlinear systems is using the Lyapunov function~\cite{khalil2002nonlinear}. For a nonlinear system, the Lyapunov function can be found by hand or using a Lyapunov-based control design methodology, such as backstepping control~\cite{kokotovic2001constructive, wang2016backstepping, tayefi2019logarithmic}. Furthermore, for a linear model with bounded external disturbances, an invariant set can be computed using a Linear Matrix Inequality (LMI)~\cite{boyd1994linear}. For polytopic nonlinear systems, this can be extended using polytopic Lyapunov functions~\cite{boyd1994linear, khalil2002nonlinear}.

Although the HJ reachability approach can compute the invariant set for a nonlinear system, the method is computationally expensive. Additionally, Lyapunov-based approaches are tedious and difficult to generalize. Therefore, we utilize a method based on Lie groups and LMIs to compute the invariant set in an efficient way for a class of dynamical systems. The Invariant Extended Kalman Filter (IEKF) in~\cite{barrau2016invariant} leverages Lie group theory to derive an estimator with provable stability properties. An extension of this approach has also been applied to the Linear Quadratic Gaussian (LQG) control problem~\cite{diemer2015invariant}. A unique property of this approach is the log-linear property of group affine systems. In the Lie algebra, group affine systems behave linearly despite nonlinearities (e.g., due to rigid body rotation) in the Lie group. The Lie algebra provides a convenient coordinate system using the log-linear property for log-linearization of the error dynamics and has a number of coordinates equal to the degrees of freedom of the dynamical system. The log-linearization is the conversion of a nonlinear dynamical system in the Lie group to a linear dynamical system in the Lie algebra. However, the log-linear property of the error dynamics governing the deviation of a system from its reference trajectory is not guaranteed in the presence of feedback control, disturbance, and noise, which cause non-linearities in the Lie algebra dynamical system. Therefore, we bound the contribution of these nonlinear terms, modeling them as bounded uncertainties in the LMI problem formulation. Another similar approach of linearization is Koopman operator theory~\cite{mauroy2020koopman, thapliyal2022approximate}, which is a composition operator that can linearize a nonlinear system to a linear system by lifting to a higher dimension, which can only be an approximation of the system. This method typically uses the measurement function as a new system to linearize. The Lie operator in the linear system in the Lie algebra can been seen as the Koopman operator as mentioned in \cite{brunton2021modern}; however, this is only satisfied when there are no disturbances or feedback control in the system.

% TODO: figure needs to be fixed
\begin{figure}[!t]
    \centering
    \includegraphics[width=\columnwidth]{images/Lie_group_algebra.png}
    \caption{Our Flow-Pipe Computation Method Leveraging Lie Group Theory and LMIs}
    \label{fig:Lie_rel}
\end{figure}

In this paper, we develop an efficient method to compute the invariant set for a nonlinear system with feedback control, disturbance, and noise based on Lie group and LMI theory. As shown in ~\Cref{fig:Lie_rel}, we first compute the invariant set in the Lie algebra, then we map it back to the Lie Group with exponential map and sweep it along the reference trajectory to create a flow pipe. We extend the log-linear property of group affine systems to log-linearize the error dynamics accounting for feedback control, disturbance, and noise, and bound the nonlinear terms. We demonstrate our approach using an illustrative vehicle control example. In our example, we consider the translational disturbance in $\{x, y\}$ (e.g., wind velocity) and the rotational disturbance in $\theta$ separately. The system in this case is defined as a system with two disturbance inputs. We also consider the system as a polytopic system~\cite{boyd1994linear} due to the time-varying reference inputs. We derive LMIs for the polytopic system with two bounded inputs and solve the LMIs with optimization tools. Furthermore, we show the simulation results of the invariant set with a 2D UAM example using polynomial trajectory planning and create the software for flow pipe generation using the Minkowski sum~\cite{varadhan2004accurate}, which generates the convex hull via the Minkowski summation of the interval hull and the invariant set for each fixed time interval in order to prove safety properties. 

In this paper, our contributions are as follows:
\begin{enumerate}
    \item We provide a proof that the error dynamics of mixed-invariant systems are log-linear using the derivative of the Lie group exponential map.
    \item We apply log-linearization to the error dynamics of mixed-invariant systems with disturbance and control input in the vehicle system and derive a series solution for the resulting non-linearities.
    \item We compute a closed-form solution for the nonlinear distortion matrix resulting after log-linearization for the SE(2) Lie group, the Special Euclidean group of two-dimensional rigid body rotation and translation.
    \item We develop a correct-by-construction time invariant control law that is efficient for safety verification employing dynamic inversion to remove nonlinearities arising from the mapping of the dynamics to the Lie algebra.
    \item We formulate flow-pipe computation for a holonomic aircraft and nonholonomic rover using our correct-by-construction control strategy as a polytopic linear system with bounded inputs using LMIs.
\end{enumerate}

The rest of this paper is organized as follows. In \Cref{sec:system}, we embed the two-dimensional kinematic holonomic aircraft model in the SE(2) Lie group. In \Cref{sec:log-linear}, we use the derivative of the exponential map in the Lie group to prove the log-linear property for the error dynamics of mixed-invariant systems, and show an example in the SE(2) Lie group with disturbances for holonomic aircraft. \Cref{sec:lmi} presents an LMI formulation to compute the invariant set for a 2D holonomic aircraft. In \Cref{sec:sim} and \Cref{sec:sim2}, we illustrate the simulation results for our model of a two-dimensional holonomic aircraft and a two-dimensional nonholonomic rover, respectively. Finally, in \Cref{sec:conclusion}, we present our concluding remarks and immediate future works.

\section{Lie Group Embedded Vehicle Kinematics}
\label{sec:system}
In the UAM system, we consider vehicles with 3D rigid body rotation and translation, the SE(3) Lie group. For constant altitude flight, this can be simplified to 2D rigid body rotation and translation, the SE(2) Lie group. Since both VTOL and fixed-wing UAS dynamics can be well approximated as differentially flat~\cite{levine2009analysis, mellinger2011minimum}, we can calculate a reference trajectory that is flyable by the aircraft with negligible error in the absence of disturbance and noise. This is important as our approach focuses on the deviation of the aircraft from this reference trajectory. To simplify our approach in this paper, we consider a constant altitude kinematic model of the UAS with bounded disturbances to account for modelling error, which simplifies the UAS kinematics to a two-dimensional holonomic aircraft model as used in ~\cite{barrau2016invariant}. Although we are limiting our analysis to the SE(2) Lie group, our algebraic approach can be generalized to other matrix Lie groups (e.g., the SE(3) Lie Group of rigid body rotation and translation in three dimensions).

Consider the kinematic holonomic aircraft model evolving on a 2D plane, as follows:
\begin{equation}
\begin{aligned}
\frac{d}{dt}\theta &= \omega \\
\frac{d}{dt}p_x &= \cos(\theta)v_x - \sin(\theta)v_y \\  
\frac{d}{dt}p_y &= \sin(\theta)v_x + \cos(\theta)v_y   
\label{eq:Dubin}
\end{aligned} 
\end{equation}
where $\theta \in [-\pi, \pi]$ denotes the heading angle, $p_x$ and $p_y$ denote the position. $\omega$ represents the angular velocity, $v_x$ and $v_y$ represent the forward and side translational velocity respectively.

The kinematic holonomic aircraft model in~\Cref{eq:Dubin} can be embedded in the SE(2) Lie group in~\Cref{eq:X}, a matrix Lie group. Special in the Special Euclidean group indicates that all matrices have determinant $1$. A Lie group is a group that is also a smooth manifold. The associated Lie algebra is a vector space tangent to the Lie group at the identity~\cite{baker2012matrix}. A Lie group and its Lie algebra are closely related, which allows calculations in one to be mapped into the other. The exponential map maps the Lie algebra to the Lie group and the logarithm map, the inverse of the exponential map, maps the Lie group to the Lie algebra~\cite{hall2015lie}.

The corresponding left-invariant system dynamics~\cite{khosravian2016state} of the 2D holonomic aircraft problem are given by:
%
\begin{equation}
\label{eq:X}
\dot{X} = X{[l]}^{\wedge}
\end{equation}
where $X$ represents the state of the 2D holonomic aircraft in the SE(2) Lie group. Note that $l \in \mathbb{R}^3$ may be a function of time and represents the input of the vehicle, including feedback control and disturbances.

The SE(2) Lie group may be represented by matrices of the form:
%
\begin{equation}
X = \begin{bmatrix}
    R & p \\
    0 & 1                   
\end{bmatrix}
\end{equation}
where $R = \begin{bmatrix} \cos(\theta) & -\sin(\theta) \\
\sin(\theta) & \cos(\theta) \end{bmatrix} \in$ SO(2), which is an element of the 2D rotation group, and $p = \begin{bmatrix} p_x & p_y\end{bmatrix}^T\in \mathbb{R}^2$. The SE(2) Lie group is the semi-direct product of the SO(2) and $\mathbb{R}^2$~\cite{hall2015lie}. 

The Lie algebra, $\mathfrak{se}(2)$, is the corresponding Lie algebra for the SE(2) Lie group. The $\mathfrak{se}(2)$ Lie algebra may be represented by matrices of the form:
%
\begin{equation}
x = \begin{bmatrix}
    \omega_\times & v \\
    0 & 0                   
\end{bmatrix}
\end{equation}
where $\omega_\times = \begin{bmatrix}
0 & -\omega \\
\omega & 0
\end{bmatrix}$ is the corresponding skew-symmetric matrix of $\omega \in \mathbb{R}$, such that $\omega_\times = -\omega_\times^T$, $v = \begin{bmatrix} v_x & v_y \end{bmatrix}^T \in \mathbb{R}^2$. In Lie group theory, the adjoint representation~\cite{hall2015lie} transports a Lie algebra element from one tangent space to another. Here, for an element $X$ of the SE(2) Lie Group, we use $Ad_{X}$ to represent the adjoint representation of the Lie Group, which can be expressed as:
\begin{equation}
Ad_X = \begin{bmatrix}\begin{array}{c|c}
    R &  \begin{matrix} p_y \\ -p_x \end{matrix} \\ \hline
    0 & 1 
\end{array}
\end{bmatrix}
\end{equation}
%
where $R \in$ SO(2), $p_x$ and $p_y$ are the elements of $p = \begin{bmatrix} p_x & p_y \end{bmatrix}^T$. For an element $x$ of the $\mathfrak{se}(2)$ Lie algebra, we use $ad_x$ to represent the adjoint in the Lie algebra, which can be written as:
\begin{equation}
ad_x = \begin{bmatrix}\begin{array}{c|c}
    \omega_\times & \begin{matrix} v_y \\ -v_x \end{matrix}\\ \hline
    0 & 0
\end{array}
\end{bmatrix}
\end{equation}
%
Note that ${[\cdot]}^{\wedge} $ indicates the Lie wedge operator that maps from $\mathbb{R}^3$ to the $\mathfrak{se}(2)$ Lie algebra,
%
\begin{equation}
{[\cdot]}^{\wedge} : \mathbb{R}^3 \mapsto \mathfrak{se}(2), \begin{bmatrix}v_x\\ v_y\\ \omega \end{bmatrix} \to \begin{bmatrix} 0 & -\omega & v_x \\ \omega & 0 & v_y \\ 0 & 0 & 0\end{bmatrix}
\end{equation}

Since a mixed-invariant system is a combination of a left- and a right-invariant system, the left-invariant system is actually a mixed-invariant system with the right-invariant vector field, $r$, equal to zero. The mixed-invariant system is also a group affine system.
%
\begin{defn}(Mixed-Invariant System~\cite{khosravian2016state}) 
The left-invariant system~\cite{lageman2009gradient, khosravian2016state} in the matrix Lie group has the form of $\dot{X} = X[l]^\wedge$, and the right-invariant system in the matrix Lie group is $\dot{X} = [r]^\wedge X$, where $X$ is an element of the matrix Lie group $G$ whose associated Lie algebra is denoted $\mathfrak{g}$ and represents states of the systems, and $l, r \in \mathfrak{g}$ are inputs of the systems and may be functions of time, which define the left- and right-invariant vector fields, respectively. If a system is the summation of a left-invariant system and a right-invariant system, $\dot{X} = X{[l]}^{\wedge} + {[r]}^{\wedge} X $, then the system is called a \emph{mixed-invariant system}.
\end{defn}
%
\begin{defn} (Group Affine System \cite{barrau2016invariant, barrau2014invariant})
If a system with dynamics $\dot{X} = f(X)$ satisfies $f(A B) = f(A) B + A f(B) - A f(I) B$, then we call the system a \emph{group affine system}. Here $A$, $B$, $X \in G$, and $I$ is the identity element of the Lie Group.
\end{defn}
%
\begin{lem}(Mixed-Invariant systems are Group Affine~\cite{barrau2014invariant, barrau2016invariant})
Let $f(X) = X[l]^\wedge+[r]^\wedge X$, then
\begin{align*}
f(A)& B + A f(B) - A f(I) B  \\
&= (A[l]^\wedge + [r]^\wedge A)B + A(B[l]^\wedge + [r]^\wedge B)\\
&\hspace{1 cm}- A([l]^\wedge + [r]^\wedge)B \\ 
&= [r]^\wedge AB + AB[l]^\wedge = f(AB) \hspace{2cm} \blacksquare
\end{align*}
\end{lem}

\section{Log-Linearization for Mixed-invariant Error Dynamics}
\label{sec:log-linear}
In this section, we first derive the log-linear property for the error dynamics of mixed-invariant systems. Based on the log-linear property, we then log-linearize the nonlinear system with feedback control, disturbance, and noise to a linear bounded input system. The invariant set for the linear bounded input system can be found efficiently employing LMIs.

\subsection{Log-Linearization for Generic Mixed-Invariant Systems}
We derive the log-linear property for the error dynamics of mixed-invariant systems using a succinct approach employing the derivative of the exponential map. Consider the dynamics of two mixed-invariant vector fields, where $X \in G$ is the state of the vehicle and $\bar{X} \in G$ is the state of the reference trajectory:
\begin{equation}
\begin{aligned}
\label{eq:biX}
\dot{X} &= X{[l + u_l]}^{\wedge} + {[r + u_r]}^{\wedge} X\\
\dot{\bar{X}} &= \bar{X}{[l]}^{\wedge} + {[r]}^{\wedge} \bar{X}
\end{aligned}
\end{equation}
where $l, r, u_l, u_r \in \mathfrak{g}$ are inputs of the systems and may be functions of time. $u_l$ is the deviation of the system from the reference left-invariant vector field, $l$, and $u_r$ is the deviation of the system from the reference right-invariant vector field, $r$. The left- and right-invariant errors, $\eta_l, \eta_r \in G$ are defined as follows~\cite{barrau2016invariant}:
\begin{equation}
\begin{aligned}
\eta_l &= X^{-1}\bar{X} \\
\eta_r &= \bar{X}X^{-1}
\end{aligned}
\end{equation}
We apply the derivative of the exponential map to derive the log-linear property for the left- and right-invariant error dynamics.

\begin{lem}(The Derivative of the Exponential Map~\cite{rossmann2006lie})
\label{def:dexp}
The exponential map of a matrix Lie group is a map from the matrix Lie algebra to the corresponding matrix Lie group, $\exp: \mathfrak{g} \rightarrow G$. The \emph{derivative of the exponential map} is given by:
\begin{equation}
\dfrac{d}{dt} \exp{({[\zeta(t)]}^{\wedge})} = \exp{(-ad_\zeta)} \dfrac{I - \exp{(-ad_\zeta)}}{ad_\zeta} \dfrac{d}{dt} [\zeta(t)]^{\wedge}
\end{equation} 
Here, $I$ denotes the identity element of the group, which is the identity matrix for matrix Lie groups~\cite{rossmann2006lie, hall2015lie}, $ad_\zeta$ is the adjoint representation of $\zeta$ in the Lie algebra, and $\dfrac{I - \exp({-ad_\zeta})}{ad_\zeta} = \sum\limits_{k=0}^{\infty} \dfrac{(-1)^k}{(k+1)!} (ad_\zeta)^k$.
\end{lem}
%
\begin{thm}(Logarithm of Left-Invariant Error Dynamics)
For the systems in \Cref{eq:biX}, denote the left-invariant error $\eta_l$ by $\eta_l = \exp([\zeta_l]^\wedge)$, then the logarithm of the left-invariant error dynamics, $\dot{\zeta}_l$, is governed by the differential equation:
\begin{equation}
\label{eq:log_left}
\begin{aligned}
    \dot{\zeta}_l &= -ad_{l} {\zeta_l} + U_l (u_l + Ad_{X^{-1}}u_r)\\ 
    U_l &\equiv -\frac{ad_{\zeta_l} \exp{(-ad_{\zeta_l})}}{I - \exp{(-ad_{\zeta_l})}}\\
    U_l^{-1} &= \sum_{k=0}^{\infty} \frac{(ad_{\zeta_l})^k}{(k+1)!}
\end{aligned}
\end{equation}
where $\zeta_l$ is the state vector of the left-invariant error dynamics in the Lie algebra, $Ad_X$ is the adjoint representation of $X$ in the Lie group, and $U_l$ is the matrix of nonlinear distortion of inputs in the Lie algebra for the left-invariant error dynamics.
\label{thm:left}
\end{thm}
%
\begin{proof}
Using the product rule, the left-invariant error dynamics, which are the derivative of $\eta_l$, can be written as: 
\begin{equation}
\begin{aligned}
\dot{\eta}_l &= -X^{-1}\dot{X}X^{-1}\bar{X} + X^{-1}\dot{\bar{X}} \\
&= \eta_l{[l]}^{\wedge} - {[l+u_l]}^{\wedge} \eta_l - Ad_{X^{-1}}{[u_r]^{\wedge}} \eta_l 
\label{eq:deta_l} 
\end{aligned}
\end{equation}
%
Let $\eta_l = \exp({[\zeta_l]}^{\wedge})$, where $\zeta_l$ is the left-invariant error in the Lie algebra. Based on \Cref{def:dexp}, the derivative of the exponential map can be written as:
\begin{equation}
[\dot{\zeta}_l]^{\wedge} = \dfrac{ad_{\zeta_l}}{I - \exp{(-ad_{\zeta_l)}}} \eta_l^{-1} \dot{\eta}_l
\label{eq:exp_eta_l}
\end{equation}
The left-invariant error dynamics in \Cref{eq:deta_l} can be rewritten as:
%
\begin{multline}
\eta_l^{-1} \dot{\eta}_l = [l]^{\wedge} - Ad_{{\eta_l}^{-1}} [l + u_l]^{\wedge} - Ad_{{\eta_l}^{-1}} Ad_{X^{-1}}[u_r]^{\wedge} \\
= \left(I - \exp{(-ad_{\zeta_l})}\right)[l]^{\wedge}  - \exp{(-ad_{\zeta_l})}[u_l]^{\wedge} \\ 
- \exp{(-ad_{\zeta_l})}Ad_{X^{-1}}[u_r]^{\wedge} 
\label{eq:etal-1eta}  
\end{multline}

%
The corresponding dynamics for left-invariant error in the Lie algebra can be derived as:
\begin{equation}
\begin{aligned}
[\dot{\zeta}_l]^{\wedge} &= \frac{ad_{\zeta_l}}{I - \exp{(-ad_{\zeta_l})}} [\left(I - \exp{(-ad_{\zeta_l})}\right)[l]^{\wedge} \\
&- \exp{(-ad_{\zeta_l})}[u_l]^{\wedge} - \exp{(-ad_{\zeta_l})}Ad_{X^{-1}}[u_r]^{\wedge}]\\
&= -ad_{l} [{\zeta_l}]^{\wedge}  -\frac{ad_{\zeta_l} \exp{(-ad_{\zeta_l})}}{I - \exp{(-ad_{\zeta_l})}} ([u_l]^{\wedge} + Ad_{X^{-1}}[u_r]^{\wedge})  
\end{aligned}
\label{eq:zeta-dot-l-wedge}
\end{equation}
%
$\frac{ad_{\zeta_l} \exp{(-ad_{\zeta_l})}}{I - \exp{(-ad_{\zeta_l})}}$ can be expressed as a matrix power series of $ad_{\zeta_l}$, and $ad_{\zeta_l}$ is a linear operator on the Lie algebra, therefore, we can use the Lie vee operator, ${[\cdot]}^{\vee}$, which is the inverse of the wedge operator, to map the error dynamics from a form of elements of the Lie algebra \Cref{eq:zeta-dot-l-wedge} to elements of the vector space $\mathbb{R}^n$ \Cref{eq:log_left} as shown in \Cref{thm:left}.
\end{proof}
%
\begin{thm}(Logarithm of Right-Invariant Error Dynamics)
For the systems in \Cref{eq:biX}, denote the right-invariant error $\eta_r$ by $\eta_r = \exp([\zeta_r]^\wedge)$, then the logarithm of the right-invariant error dynamics, $\dot{\zeta}_r$, is governed by the differential equation:
\begin{equation}
\label{eq:log_right}
\begin{aligned}
    \dot{\zeta}_r &= ad_{r} {\zeta_r} + U_r (u_r + Ad_{X}u_l)\\
    U_r &\equiv -\frac{ad_{\zeta_r}}{I - \exp{(-ad_{\zeta_r})}}\\
    U_r^{-1} &= \sum_{k=0}^{\infty} \frac{(-1)^k (ad_{\zeta_r})^k}{(k+1)!}
\end{aligned}
\end{equation}
where $\zeta_r$ is the state vector of right-invariant error dynamics in the Lie algebra, and $ad_r$ is the adjoint matrix of input $r$ in the Lie algebra. $U_r$ is the matrix of nonlinear distortion of inputs in the Lie algebra for the right-invariant error dynamics.
%A proof is provided in \Cref{sec:proof_right}.
\label{thm:right}
\end{thm}
%
\begin{proof}
% right invariant
The right invariant error dynamics, $\eta_r$, can be written as:
%
\begin{equation}
\begin{aligned}
\dot{\eta}_r &= \dot{\bar{X}}X^{-1} -\bar{X}X^{-1}\dot{X}X^{-1} \\
&= {[r]}^{\wedge}\eta_r - \eta_r{[r+u_r]}^{\wedge} - \eta_r Ad_{X}{[u_l]^{\wedge}} \label{eq:deta_r} 
\end{aligned}
\end{equation}
For the right invariant error, $\eta_r$, we use the same process to compute the corresponding error dynamics in the Lie algebra. The right invariant error dynamics in \Cref{eq:deta_r} can be rewritten as:
%
\begin{equation}
\begin{aligned}
\eta_r^{-1} \dot{\eta}_r &= Ad_{{\eta_r}^{-1}}[r]^{\wedge} -  [r + u_r]^{\wedge} - Ad_{X}[u_l]^{\wedge} \\
&= -\left(I - \exp{(-ad_{\zeta_r})}\right)[r]^{\wedge}  - [u_r]^{\wedge} - Ad_{X}[u_l]^{\wedge} \label{eq:eta-1eta}     
\end{aligned}
\end{equation}
%
The corresponding dynamics for left-invariant error in the Lie algebra can be derived as:
\begin{multline}
[\dot{\zeta}_r]^{\wedge} = \frac{ad_{\zeta_r}}{I - \exp{(-ad_{\zeta_r})}} [-\left(I - \exp{(-ad_\zeta)}\right)[r]^{\wedge} \\- [u_r]^{\wedge} - Ad_{X}[u_l]^{\wedge}]\\
= ad_{r} {\zeta_r}  -\frac{ad_{\zeta_r}}{I - \exp{(-ad_{\zeta_r})}} ([u_r]^{\wedge} + Ad_{X}[u_l]^{\wedge})
\label{eq:zeta-dot-r-wedge}     
\end{multline}
%
$\frac{ad_{\zeta_r}}{I - \exp{(-ad_{\zeta_r})}}$ can be expressed as a matrix power series of $ad_{\zeta_r}$, and $ad_{\zeta_r}$ is a linear operator on the Lie algebra; therefore, we can use the Lie vee operator, ${[\cdot]}^{\vee}$, to transform the error dynamics from a form of elements of $\mathbb{R}^{n\times n}$ \Cref{eq:zeta-dot-r-wedge} to elements of $\mathbb{R}^n$ \Cref{eq:log_right} as shown in \Cref{thm:right}.
\end{proof}

Based on \Cref{thm:left,thm:right}, we can log-linearize the nonlinear system in the Lie group to a log-linear system in the Lie algebra, which is nearly linear for a small deviation from the reference trajectory. Since the nonlinear terms in the log-linear system can be modelled as bounded inputs, we then have a linear bounded input system. This approach is useful since the nonlinear system can be analyzed in an efficient way by using the log-linear system. For example, we can compute the invariant sets by leveraging LMI theory to the linear bounded input system and then use the exponential map to map it back to the original nonlinear system.

\begin{rem}
\label{rem:1}
It is straightforward to show from \Cref{thm:left,thm:right} that the dynamics of the left-/right-invariant error respectively are log-linear and group affine when $u_l = u_r = 0$. $\dot{\zeta}_l = -ad_{l} {\zeta_l}$. $\dot{\zeta}_r = ad_{r} {\zeta_r}$.
\end{rem}

\begin{rem}
\Cref{thm:left,thm:right} are also useful for state estimation problems. For the left-invariant error, if $u_r = Ad_{X^{-1}} u_r$ (e.g., in the case of state estimation where $u_r$ is the noise that has equal power in all axes) and the estimator dynamics of the log of the left-invariant error are linearized about the expected error of $\zeta_l = 0$, then the estimator dynamics are linear and a Kalman Filter may be applied. A similar approach has been used in the Invariant Extended Kalman Filter \cite{barrau2016invariant}.
\end{rem}
%
For control, \Cref{thm:left,thm:right} suggest a direct approach for dynamic inversion of the log-linearized mixed-invariant vector field dynamics.
We will focus on \Cref{thm:left} but this can also be extended to \Cref{thm:right}. Consider $u_l \equiv u + w$ and $u_r = 0$ where $u$ represents a control input and $w$ represents a disturbance:
\begin{equation}
\begin{aligned}
u &= U_l^{-1}B K \zeta_l \\
\dot{\zeta}_l &= (-ad_l + BK) \zeta_l + U_l w
\end{aligned}
\label{eq:control}
\end{equation}
%
where $B$ is the matrix to choose the valid components in the control law. If all components are used in the control law, $B$ will be an identity matrix. $K$ is the corresponding gain matrix for the state feedback controller. This holds if all elements of the Lie algebra are control inputs. However, we can still apply our method if there are Lie algebra elements that are not control inputs. For example, in the case of a vehicle with Ackermann steering dynamics~\cite{simionescu2002optimum}, the vehicle may not move sideways. In the case that there is a Lie algebra element that is not usable, the log-linear system can be rewritten in \Cref{eq:zeta_dot_side}, which is a form of the polytopic system~\cite{boyd1994linear}. The difference can be treated as modification, $\Delta A_p$, of the $A_p$ matrix, where $A_p$ is the stabilized state feedback matrix. Since $\Delta A_p$ is a function of $\zeta_l$ and could be bounded, the system is a polytopic system~\cite{boyd1994linear}. Let $u_l = L U_l^{-1} B K\zeta_l$, where $L=\text{diag}[\beta_0, \beta_1, \ldots, \beta_n]$, $\beta_i = 1$ if the Lie algebra element is a valid control input to the system and $\beta_i = 0$ otherwise.
%
\begin{equation}
\begin{aligned}
A_p &\equiv -ad_l + B K \\
\Delta A_p  &\equiv  (U_l L U_l^{-1} - I) BK \\
\dot{\zeta}_l &= \left(A_p + \Delta A_p \right) \zeta_l  + U_l w
\label{eq:zeta_dot_side}
\end{aligned}
\end{equation}

\subsection{Log-Linearization in the SE(2) Lie Group}
As we mentioned in the previous section, we used the 2D holonomic aircraft as our simplified UAS model. The holonomic aircraft kinematics can be embedded in the SE(2) Lie group as $\dot{X} = X[l]^{\wedge}$. We also consider the reference kinematics embedded in the SE(2) Lie group differential equation, as follows:
\begin{equation}
\label{eq:barX}
\dot{\bar{X}} = \bar{X}{[\bar{l}]}^{\wedge}
\end{equation}
where $\bar{X}$ denotes the state of the reference trajectory, and $\bar{l}$ denotes the input of the reference trajectory. The relationship between the vehicle and reference input, $l$ and $\bar{l}$, is $l = \bar{l} + u + w$, where $u$ is the feedback control input, and $w$ is the disturbance. For our holonomic aircraft in \Cref{eq:X} and reference kinematics in \Cref{eq:barX}, the Lie group differential equations are left-invariant; therefore, the left-invariant error between the two systems is $\eta = X^{-1}\bar{X}$. From \Cref{thm:left} and \Cref{eq:control}, the error dynamics in the Lie algebra can be written as:
\begin{equation}
\begin{aligned}
\dot{\zeta} &= (-ad_{\bar{l}} + BK) \zeta + U w\\
U &\equiv U_l
\end{aligned}
\label{eq:se(2)sys}
\end{equation}
%
where $u = U^{-1}B K \zeta$ and $\zeta \equiv \begin{bmatrix}
\zeta_x & \zeta_y & \zeta_{\theta}
\end{bmatrix}^T$.

For SE(2), the closed-form solution of the nonlinear distortion matrix, $U$, is:
%
\begin{multline}
U = \\
\begin{bmatrix} a & -b & \frac{\zeta_{\theta} \zeta_x \sin{\left(\zeta_{\theta} \right)} + \left(1 - \cos{\left(\zeta_{\theta} \right)}\right) \left(\zeta_{\theta} \zeta_y - 2 \zeta_x\right)}{2 \zeta_{\theta} \left(1 - \cos{\left(\zeta_{\theta} \right)}\right)}\\ 
b & a & \frac{\zeta_{\theta} \zeta_y \sin{\left(\zeta_{\theta} \right)} + \left(1 - \cos{\left(\zeta_{\theta} \right)}\right) \left(- \zeta_{\theta} \zeta_x - 2 \zeta_y\right)}{2 \zeta_{\theta} \left(1 - \cos{\left(\zeta_{\theta} \right)}\right)}\\
0 & 0 & -1\end{bmatrix}
\label{eq:U}  
\end{multline}
where $a = \frac{\zeta_{\theta} \sin{\left(\zeta_{\theta} \right)}}{2 \left(\cos{\left(\zeta_{\theta} \right)} - 1\right)}$ and $b = \frac{\zeta_{\theta}}{2}$.

The invariant set for a system can be calculated by finding an upper bound for the singular value of the nonlinear distortion matrix, $U$, over the invariant set. Since this depends on the invariant set itself, we propose an LMI-based iterative method to compute the invariant set.

Figure~\ref{fig:LieCorrespondence} shows the bijection between the vehicle dynamics in the Lie group and the Lie algebra, which are simulated with sinusoidal disturbances of fixed magnitude at varying frequencies. Figure~\ref{fig:LieError} shows the error between the trajectories in the Lie group and the Lie algebra with different solver tolerances. Since the trajectories in the Lie group are propagated with cosine and sine, which is nonlinear, and the Lie algebra is using linear propagation, there still exist discrepancies between the Lie group and the Lie algebra; however, the error can be eliminated when the solver tolerance is sufficiently small. Therefore, we numerically demonstrate \Cref{thm:left,thm:right} by showing that a bijection exists between the differential equation in the Lie group and Lie algebra even in the presence of disturbance and control feedback, given a domain sufficiently close to the reference trajectory (e.g. avoiding angle wrapping). This is a useful result, as the Lie algebra is a vector space that is more convenient for analysis than the Lie group which is a nonlinear manifold~\cite{hall2015lie}.

\begin{figure}[!t]
    \centering
    \includegraphics[width=1\columnwidth]{images/Lie_corres.png}
    \caption{Lie Group-Lie Algebra Bijection for Trajectories with Control and Disturbances in the SE(2)}
    \label{fig:LieCorrespondence}
\end{figure}

\begin{figure}[!t]
    \centering
    \begin{subfigure}[b]{0.5\textwidth}
        \includegraphics[width=\columnwidth]{images/Lie_error1.png}
        \includegraphics[width=\columnwidth]{images/Lie_error2.png}
    \end{subfigure}
    \caption{Top: Error between the Lie Group and the Lie Algebra in the SE(2) with Larger Tolerance, Bottom: Error between the Lie Group and the Lie Algebra in the SE(2) with Smaller Tolerance}
    \label{fig:LieError}
\end{figure}
% 
\subsection{The Log-Linear Approach in the context of the Koopman Operator}
As we mentioned in \Cref{rem:1}, when the inputs of both the vehicle and reference systems are the same, the error dynamics in the Lie algebra are linear. The linearization matches Koopman operator theory if we assume the measurement function to be the logarithm of the left- or right-invariant error. Here, we use the previous SE(2) dynamics as an example. 

\begin{lem}(Logarithm of the SE(2) Lie Group~\cite{baker2012matrix})
\label{def:log}
The logarithm maps the element in the SE(2) Lie group, $X$, to an element in the $\mathfrak{se}(2)$ Lie algebra.
\begin{equation}
\begin{aligned}
    \log(X) &= \log(\begin{bmatrix}
    R & p \\
    0 & 1                   
    \end{bmatrix}) = \begin{bmatrix}
    V^{-1}p \\
    \theta
\end{bmatrix}^{\wedge} \in \mathfrak{se}(2)\\ 
V^{-1} &= \frac{1}{c^2+d^2}\begin{bmatrix}
    c & d \\ -d & c
\end{bmatrix}
\end{aligned}
\label{eq:log}
\end{equation}
where $c = \frac{\sin(\theta)}{\theta}$, and $d = \frac{1-\cos(\theta)}{\theta}$  
\end{lem}

Assuming the vehicle inputs in \Cref{eq:X} equal the reference inputs in \Cref{eq:barX}, $l = \bar{l}$, and the measurement function, $g(\eta)$, is the logarithm of the left-invariant error, $\eta$. Using \Cref{def:log}, the measurement function can be written as:

\begin{equation}
\begin{aligned}
g(\eta) &= \log(\eta)\\ 
&= \begin{bmatrix}
    \frac{\eta_y\eta_\theta}{2}-\frac{\eta_\theta\eta_x\sin(\eta_\theta)}{2(\cos(\eta_\theta)-1)} \\
    -\frac{\eta_x\eta_\theta}{2}-\frac{\eta_\theta\eta_y\sin(\eta_\theta)}{2(\cos(\eta_\theta)-1)}\\
    \eta_\theta
\end{bmatrix} 
\end{aligned}
\label{eq:measurement}
\end{equation}

The derivative of the measurement function, $\dot{g}(x)$ can be written as:

\begin{equation}
\dot{g}(x) = \begin{bmatrix}
    -\frac{\omega\eta_x\eta_\theta}{2}-\frac{\omega\eta_\theta\eta_y\sin(\eta_\theta)}{2(\cos(\eta_\theta)-1)}-v_y\eta_\theta \\
    -\frac{\omega\eta_y\eta_\theta}{2}+\frac{\omega\eta_\theta\eta_x\sin(\eta_\theta)}{2(\cos(\eta_\theta)-1)}+v_x\eta_\theta\\
    0
\end{bmatrix} 
\label{eq:measurement_dot}
\end{equation}
%
which is equal to:
\begin{equation}
\begin{aligned}
-ad_l g(x)
&= \begin{bmatrix}
    0& \omega& -v_y\\
    -\omega& 0& v_x\\
    0& 0& 0
\end{bmatrix}\begin{bmatrix}
    \frac{\eta_y\eta_\theta}{2}-\frac{\eta_\theta\eta_x\sin(\eta_\theta)}{2(\cos(\eta_\theta)-1)} \\
    -\frac{\eta_x\eta_\theta}{2}-\frac{\eta_\theta\eta_y\sin(\eta_\theta)}{2(\cos(\eta_\theta)-1)}\\
    \eta_\theta
\end{bmatrix}\\
&= \begin{bmatrix}
    -\frac{\omega\eta_x\eta_\theta}{2}-\frac{\omega\eta_\theta\eta_y\sin(\eta_\theta)}{2(\cos(\eta_\theta)-1)}-v_y\eta_\theta \\
    -\frac{\omega\eta_y\eta_\theta}{2}+\frac{\omega\eta_\theta\eta_x\sin(\eta_\theta)}{2(\cos(\eta_\theta)-1)}+v_x\eta_\theta\\
    0
\end{bmatrix}
\end{aligned}
\label{eq:Ag}
\end{equation}
Equations \Cref{eq:measurement_dot} and \Cref{eq:Ag} satisfy the definition of the Koopman operator for a continuous system:
\begin{equation}
\frac{d}{dt}g = Kg
\end{equation}
where $g$ is a measurement function and $K$ is the Koopman operator.

\section{LMI-based Invariant Set Calculation for Systems Embedded in the SE(2) Lie Group}
\label{sec:lmi}

The dynamics of the disturbed linear system in the $\mathfrak{se}(2)$ Lie algebra in \Cref{eq:se(2)sys} can be rewritten as:
%
\begin{equation}
\dot{\zeta} = A_{p}\zeta+\tilde{w} 
\label{eq:distsys}
\end{equation}
% 
where $A_p \coloneqq -ad_{\bar{l}} + BK$, $B$ is the matrix to choose the valid components for the control law, $K$ is the corresponding gain matrix, and $\tilde{w}:= Uw$ is the bounded input, which is bounded by the upper bound of the singular value of the nonlinear distortion matrix, $U$.
%
For a linear system, the Lyapunov function is a quadratic function of the form:
\begin{equation}
\label{eq:LyapV}
V(\zeta) = \zeta^TP\zeta
\end{equation}
where $P$ is a real symmetric positive definite matrix. The derivative of the Lyapunov function may be derived using the product rule:
\begin{equation}
\label{eq:dV1}
\begin{aligned}
\dot{V} &= \dot{\zeta}^TP\zeta + \zeta^TP\dot{\zeta}\\
&= (\zeta^TA_p^T+\tilde{w}^T)P\zeta + \zeta^TP(A_p\zeta+\tilde{w})
\end{aligned}
\end{equation}
%
Here, we separate the disturbance into two parts: $\tilde{w} = {\begin{bmatrix} \tilde{w}_{1} & \tilde{w}_{2}\end{bmatrix}}^T$, where $\tilde{w}_1 \in \mathbb{R}^{2\times1}$ is a bounded function of time and represents the velocity disturbance in the $\{x, y\}$ coordinates, which can be thought of as a wind disturbance; and $\tilde{w}_2 \in \mathbb{R}$ is also a bounded function of time and represents a rotational velocity disturbance in the $\theta$ coordinate. 
We define the system as a linear system with two disturbance inputs. The matrix of the Lyapunov function, $P$, is defined as:
\begin{equation}
P = \begin{bmatrix} P_1 \\ P_2\end{bmatrix}
\end{equation}
%
where $P_1 \in \mathbb{R}^{2\times3},$ and $P_2 \in \mathbb{R}^{1\times3}$. Therefore, the derivative of the Lyapunov function can be rewritten as:
\begin{multline}
\dot{V} = \zeta^T(A_p^TP+PA_p)\zeta + \begin{bmatrix}\tilde{w}_1 \\ \tilde{w}_2\end{bmatrix}^T \begin{bmatrix}P_1 \\ P_2\end{bmatrix}\zeta + \zeta^T\begin{bmatrix}P_1 \\ P_2\end{bmatrix}^T \begin{bmatrix}\tilde{w}_1 \\ \tilde{w}_2\end{bmatrix}\\
= \zeta^T(A_p^TP+PA_p)\zeta + \tilde{w}_1^TP_1\zeta + \tilde{w}_2^TP_2\zeta \\ + \zeta^TP_1^T\tilde{w}_1 + \zeta^TP_2^T\tilde{w}_2    
\end{multline}

\begin{defn} (Infinity Norm~\cite{robel1989computing})
The \emph{infinity norm} of a vector-valued function $f$ is defined as, $||f(t)||_\infty  \equiv \sup_{t \in T}{||f(t)||}$, the supremum of the Euclidean norm of the function evaluated at time t, $||f(t)||$, for all $t$ in the set of time $T$.
\end{defn}

To compute the invariant set for a linear system with two bounded disturbance inputs, we consider the following condition:
\begin{equation}
\label{eq:disturbed_cond}
    \dot{V} < 0
    \text{\hspace{0.5 cm} if: \hspace{0.5 cm}}
    V > \mu_1 ||\tilde{w}_1||_\infty^2 + \mu_2 ||\tilde{w}_2||_\infty^2
\end{equation}

Since we seek to find the minimal invariant set, we compute the smallest scalars $\mu_1$ and $\mu_2$ such that $\dot{V} < 0$. Using the $\mathcal{S}$-procedure~\cite{boyd1994linear, yakubovich1971s}, we conclude that the condition in \Cref{eq:disturbed_cond} is equivalent to the existence of $\alpha \geq 0$ such that:
%
\begin{equation}
\dot{V} + {\alpha}(V - {\mu}_1||\tilde{w}_1||_\infty^2 - {\mu}_2||\tilde{w}_2||_\infty^2) < 0
\end{equation}
%  
Since $||\tilde{w}_1||^2 \leq ||\tilde{w}_1||_\infty^2$ and $|\tilde{w}_2||^2 \leq ||\tilde{w}_2||_\infty^2$ for all time, this condition can be written in a matrix form:
%
\begin{equation}
\begin{bmatrix}\zeta \\ \tilde{w}_1 \\ \tilde{w}_2\end{bmatrix}^T
\begin{bmatrix} A_p^TP + PA_p + {\alpha}P & P_1^T & P_2^T\\ P_1 & - {\alpha}{\mu}_1I & 0\\ P_2 & 0 & -{\alpha}{\mu}_2I\end{bmatrix}\begin{bmatrix}\zeta \\ \tilde{w}_1 \\ \tilde{w}_2\end{bmatrix} < 0
\end{equation}
%
The matrix inequality to solve the Lyapunov function for the linear system with two bounded disturbance inputs is:
%
\begin{equation}
\begin{bmatrix}A_p^TP + PA_p + {\alpha}P & P_1^T & P_2^T\\ P_1 & - {\alpha}{\mu}_1I & 0\\ P_2 & 0 & -{\alpha}{\mu}_2I\end{bmatrix} < 0
\end{equation}
% alpha P term is quadratic/nonlinear
Note that $\alpha$ should be fixed, since the term $\alpha P$ is quadratic, which is nonlinear; therefore, we minimize $\mu_1 + \mu_2$ for each $\alpha$ and do a line search over $\alpha$~\cite{scherer1997multiobjective, palhares2000linear}.

Moreover, we consider the system as a polytopic system, since the velocity and turning rate of the reference input are bounded functions of time. A polytopic linear system with disturbance is a system in a form given by~\cite{boyd1994linear}:
%
\begin{equation}
\label{eq:poly}
\dot{\zeta} = A(t)\zeta + \tilde{w}
\end{equation}
%
where the system matrix $A(t)$ has the following structure:
%
\begin{equation}
A(t) = A_0 + \sum\limits_{i=0}^{n} \psi_i(t)\Delta A_i
\end{equation}
%
where $A_0$ and $\Delta A_i$ are constant matrices and $\psi_i(t)$ is a bounded scalar function of time, that is $a_i \leq \psi_i(t) \leq b_i$.

Our feedback controller is a linear time-invariant controller in the Lie algebra, and the $B$ matrix is chosen to minimize the use of sideslip to control the aircraft by not commanding $v_y$ in the Lie algebra dynamics. However, side velocity may result after dynamic inversion due to the nonlinear distortion matrix, $U$. Therefore, our $B$ matrix is a $3\times2$ matrix (allowing $v_x$ and $\omega$ commands). We then use the Linear Quadratic Regulator (LQR) to find the corresponding $K$ matrix for $B$. Our $A_0$, $\Delta A_i$, and $B$ can be written as: 
%
\begin{equation}
\begin{aligned}
A_0 &= BK\\
\Delta A_1 &= \begin{bmatrix}
0 & 1 & 0\\
-1 & 0 & 0\\
0 & 0 & 0
\end{bmatrix} \\
 \Delta A_2 &= \begin{bmatrix}
0 & 0 & 0\\
0 & 0 & 1\\
0 & 0 & 0
\end{bmatrix} \\
B &= \begin{bmatrix}
1 & 0 \\
0 & 0 \\
0 & 1
\end{bmatrix}    
\label{eq:poly_matrix}
\end{aligned}  
\end{equation}
%
where $\Delta A_1$ and $\Delta A_2$ represent the change of $\omega$ and $v_x$ in the reference tracjectory, respectively.

The conditions for finding the Lyapunov function for a polytopic system can be stated in terms of an LMI and the four matrices corresponding to the extremum of $\psi_1(t)$ and $\psi_2(t)$~\cite{khalil2002nonlinear}, namely:
\begin{equation}
\begin{aligned}
A_1 &:= A_0 + \underline{\omega}\Delta A_1 + \underline{v}\Delta A_2\\
A_2 &:= A_0 + \overline{\omega}\Delta A_1 + \underline{v}\Delta A_2\\
A_3 &:= A_0 + \underline{\omega}\Delta A_1 + \overline{v}\Delta A_2\\
A_4 &:= A_0 + \overline{\omega}\Delta A_1 + \overline{v}\Delta A_2
\end{aligned}
\end{equation}
where $\underline{\omega}$ and $\overline{\omega}$ are the minimum and maximum of the angular velocity, $\omega$, and $\underline{v}$ and $\overline{v}$ are the minimum and maximum of the forward velocity, $v_x$.

Therefore, an LMI formulation for a polytopic linear system with two bounded disturbance inputs is:
%
\begin{equation}
\begin{bmatrix}A_i^TP + PA_i + {\alpha}P & P_1^T & P_2^T\\ P_1 & - {\alpha}{\mu}_1I & 0\\ P_2 & 0 & -{\alpha}{\mu}_2I\end{bmatrix}< 0, i = 1, 2, 3, 4
\label{eq:lmi}
\end{equation}
%
For each fixed $\alpha$, we have LMIs in $P$, $\mu_1$ and $\mu_2$. To find a feasible solution of $P$, one can minimize $\mu_1+\mu_2$ for each $\alpha$ and do a line search over $\alpha$.
After finding the Lyapunov function by solving the LMIs in \Cref{eq:lmi}, the invariant set can be described as:
\begin{equation}
\{\zeta \in \mathbb{R}^n: V(\zeta) \leq {\mu}_1||\tilde{w}_1||_{\infty}^2 + {\mu}_2||\tilde{w}_2||_{\infty}^2\}
\end{equation}
This provides us with a method to compute an invariant set for a polytopic linear system with two disturbance inputs in the $\mathfrak{se}(2)$ Lie algebra. Moreover, we can compute the invariant set for the corresponding nonlinear system in the SE(2) Lie group by applying the exponential map.

\begin{algorithm}
\caption{Calculate Log-Linear LMI based Invariant Set}
\begin{algorithmic}[1]
\STATE Guess the maximum singular value, $\sigma_0$, of the nonlinear distortion matrix, $U$, in the invariant set
\STATE Compute the invariant set using an LMI by scaling the disturbance by the maximum singular value of $U$
\STATE Calculate the maximum singular value, $\sigma_{max}$, of the U matrix over the invariant set
\STATE Assign $\sigma_{max}$ to $\sigma_0$
\STATE Repeat steps 1-4 until converged to desired tolerance ($|\sigma_0 - \sigma_{max}| < \epsilon$), favor over-approximation for safety ($\sigma_0 > \sigma_{max}$)
\end{algorithmic}
\end{algorithm}

\section{Simulation of a Holonomic Aircraft}
\label{sec:sim}
In this section, we consider an illustrative UAM scenario where a holonomic aircraft flies along a reference trajectory at a constant altitude. The reference trajectory uses polynomial trajectory planning~\cite{mellinger2011minimum, richter2016polynomial, lynch2017modern}. We consider the holonomic aircraft system as a fixed-wing aircraft model with constant altitude. Our model is given by the equations in \Cref{eq:Dubin}, which can be embedded in the SE(2) Lie group, \Cref{eq:X}. The log-linearized error dynamics between the vehicle model and the reference model are given in \Cref{eq:se(2)sys}. For external disturbances in the $\theta$ direction, we assume $||w_2||_\infty = 0.1$ rad/s. For disturbances in the $\{x, y\}$ direction, we consider two different disturbance magnitudes. We assume the magnitude of the bounded disturbance inputs, $||w_1||_\infty$, as $1$ m/s and $5$ m/s, respectively.

\subsection{Reference Trajectory Planner}
% path planner: need to mention it can set waypoints, heading velocity, turing radius
The reference trajectory is generated using polynomial trajectory planning as shown in \Cref{eq:refdyn} \cite{mellinger2011minimum, richter2016polynomial, lynch2017modern}. The waypoints for the holonomic aircraft are selected for the vehicle to enter and exit the desired location with a specified turning radius. We set the time for each leg of the reference trajectory, $T_i$, using the distance travelled, $d_i$, and a constant velocity, $v_i$, approximation, $T_i = d_i/v_i$. As the generation of the reference trajectory is not the focus of this paper, we do not concern ourselves with optimizing the time of each leg in order to minimize the total time of flight, although this can be accomplished using methods as presented in \cite{richter2016polynomial}. The scaled time $\beta = t/T$ varies from $0$ to $1$ along the leg, and \Cref{eq:refdyn} can be used to specify boundary conditions at each waypoint.
%
\begin{equation}
\label{eq:refdyn}
\begin{aligned}
p_\beta &= \sum_{k=0}^n c_k \beta^k \\
p_\beta^{(m)} &= \sum_{k=m}^n c_k \frac{k!}{(k-m)!} \beta^{k-m}/T^m   
\end{aligned}
\end{equation}
where $n$ denotes the order of the polynomial equation, and $m$ denotes the order of the derivative. 

In the simulations, we specified waypoints to the planner, and set the forward velocity and turning radius as $20$m/s and $9$m, respectively. Figure~\ref{fig:ref} shows the reference trajectory and inputs with the parameters defined above. The top figure in \Cref{fig:ref} shows the reference trajectory with waypoints, and the middle and bottom figures show the reference forward velocity and angular velocity for the trajectory.

\begin{figure}[!t]
    \centering
    \includegraphics[width=\columnwidth]{images/ref_traj.png}
    \caption{Reference Trajectory generated from Polynomial Trajectory Planning, Top: Reference Trajectory with Waypoints, Middle: Reference Translational Velocity, Bottom: Reference Angular Velocity}
    \label{fig:ref}
\end{figure}

\subsection{Invariant Set}
The invariant set is computed with $v_x \in [18, 20]$ m/s, $v_y = 0$ m/s, and $\omega \in [-\pi/2, \pi/2]$ rad/s. We use the LQR method to find the feedback controller for the log-linearized system in the $\mathfrak{se}(2)$ Lie algebra, and find the Lyapunov function using the LMIs in \Cref{eq:lmi}. After using our iterative method to find the maximum singular value, the final result of the invariant set in the Lie algebra and the Lie group is shown in \Cref{fig:invariant} and \Cref{fig:invariant3d}. The figures show the invariant set with small initial states in the SE(2) Lie group, $[x_0, y_0, \theta_0] = [0.1, 0.1, \pi/100]$, and the invariant set in the Lie group is constructed by applying the exponential map from the set in the Lie algebra. The Lyapunov function in the LMI approach is an ellipsoid, which is the shape of the invariant set for the log-linearized system in the $\mathfrak{se}(2)$ Lie algebra as shown on the left of \Cref{fig:invariant} with green lines. The invariant sets in the SE(2) Lie group, which are mapped from the invariant sets in the $\mathfrak{se}(2)$ Lie algebra using the exponential map, are shown on the right of \Cref{fig:invariant} with green lines.

If the control input does not employ dynamic inversion, the control input $u$ in \Cref{eq:control} becomes $BK\zeta_l$, and the error dynamics in the Lie algebra in \Cref{eq:se(2)sys} can be rewritten as:
\begin{equation}
\dot{\zeta} = (-ad_{\bar{l}} + BK)\zeta + (U-I)BK\zeta + Uw
\end{equation}
Here, we use the SE(2) example to compare the difference between the two controllers. For the control input without dynamics inversion, we consider the nonlinear term of the control input as a bounded input in the LMI calculation. Blue lines in \Cref{fig:invariant} and \Cref{fig:invariant3d} show the result of the invariant set for the controller without dynamic inversion. Since the nonlinear term of the control input is considered a bounded input, the invariant set has a larger tracking error bound than the controller with dynamic inversion. Comparing the invariant sets in \Cref{fig:invariant}, we can see that the invariant set is approximately twice as large in the case of the control input without dynamic inversion compared to the case with dynamic inversion. Therefore, the invariant set has a smaller tracking error bound with our designed controller. Using dynamic inversion it is important to determine if inputs are being saturated in the controller, and this can be calculated using the invariant set.

Figure~\ref{fig:invariant} shows the two dimensional projection from \Cref{fig:invariant3d}. In \Cref{fig:invariant3d}, we can see that the shape of the invariant set in the SE(2) Lie group is not an ellipsoid after mapping, since the invariant set is in the vehicle body frame; therefore, we must consider the direction of the trajectory. The invariant set must be rotated along the trajectory to create the flow pipes.

\begin{figure}[!t]
    \centering
    \begin{subfigure}[b]{0.5\textwidth}
        \includegraphics[width=\columnwidth]{images/Invariant_s.png}
    \end{subfigure}
    \begin{subfigure}[b]{0.5\textwidth}
        \includegraphics[width=\columnwidth]{images/Invariant_l.png}
    \end{subfigure}
    \caption{Projected Invariant Set Comparison with and without Dynamic Inversion, Left: Invariant Set in Lie Algebra, Right: Invariant Set in Lie Group, Top: Small Disturbance, Bottom: Large Disturbance}
    \label{fig:invariant}
\end{figure}

\begin{figure}[!t]
    \centering
    \begin{subfigure}[b]{0.5\textwidth}
        \includegraphics[width=\columnwidth]{images/Invariant3d_s.png}
    \end{subfigure}
    \begin{subfigure}[b]{0.5\textwidth}
        \includegraphics[width=\columnwidth]{images/Invariant3d_l.png}
    \end{subfigure}
    \caption{Invariant Set Comparison in three dimensions with and without Dynamic Inversion, Left: Invariant Set in Lie Algebra, Right: Invariant Set in Lie Group, Top: Small Disturbance, Bottom: Large Disturbance}
    \label{fig:invariant3d}
\end{figure}

Figures \ref{fig:invariant_cora} and \ref{fig:invariant3d_cora} show the comparison of the invariant set at the final time between the LMI and COntinuous Reachability Analyzer (CORA), a well-known existing Matlab-based toolbox~\cite{Althoff2015a}. For the reachable set calculation in CORA, we used the system in \Cref{eq:se(2)sys} with the same initial states as we use in our LMI approach. The time step and final time of the reachable set calculation are based on the reference trajectory. The singular value of the nonlinear distortion matrix, $U$, is also applied in the CORA calculation. The main difference between the two methods is the shape of the set, the reachable set in CORA is represented as a zonotope, and the invariant set in LMI is an ellipsoid. As \Cref{fig:invariant_cora} and \Cref{fig:invariant3d_cora} show, both cases have similar bounds in the $\{x, y, \theta\}$ direction. However, the computation time in CORA takes $2.5$ sec while our LMI approach takes $0.5$ sec for both cases, both running on a laptop with an Intel Core i9 2.3 GHz processor and 16 GB RAM.

Figure~\ref{fig:bound} shows the bound of the invariant sets from the log-linear approach and CORA with the simulated trajectory errors. The log-linear approach and CORA have similar bounds for all time. CORA considers the change of the reference inputs at each time step; therefore, the bound of the reachable set is larger when the vehicle is turning. The top figure of \Cref{fig:bound} shows the scenario under small wind disturbance. The bottom figure of \Cref{fig:bound} shows the scenario under large wind disturbance.

\begin{figure}[!t]
    \centering
    \begin{subfigure}[b]{0.5\textwidth}
        \includegraphics[width=0.48\columnwidth]{images/CORA_s.png}
        \includegraphics[width=0.48\columnwidth]{images/CORA_l.png}
    \end{subfigure}
    \caption{Projected Invariant Set Comparison between CORA and LMI, Left: Small Disturbance, Right: Large Disturbance}
    \label{fig:invariant_cora}
\end{figure}

\begin{figure}[!t]
    \centering
    \begin{subfigure}[b]{0.5\textwidth}
        \includegraphics[width=0.48\columnwidth]{images/CORA3d_s.png}
        \includegraphics[width=0.48\columnwidth]{images/CORA3d_l.png}
    \end{subfigure}
    \caption{Invariant Set Comparison in three dimensions between CORA and LMI, Left: Small Disturbance, Right: Large Disturbance}
    \label{fig:invariant3d_cora}
\end{figure}

\begin{figure}[!t]
    \centering
    \begin{subfigure}[t]{0.5\textwidth}
        \includegraphics[width=\columnwidth]{images/bound1.png}
    \end{subfigure}    
    \begin{subfigure}[t]{0.5\textwidth}
        \includegraphics[width=\columnwidth]{images/bound2.png}
    \end{subfigure}    
    \caption{The Comparison of the Bound of the Invariant Set, Top: Small Disturbances, Bottom: Large Disturbances}
    \label{fig:bound}
\end{figure}

\subsection{Flow Pipes}
To verify the safety properties for the UAM mission, we construct flow pipes along a reference trajectory. Any Lyapunov function will be sufficient for our method and a general approach for the creation of the flow pipes is shown in \Cref{fig:construct_flowpipe}~\cite{goppert2019security}. To construct the flow pipe, we first propagate the reference trajectory using polynomial trajectory planning for a fixed time interval and compute the interval hull of the reference trajectory, which is the smallest box enclosing the set for the propagation duration. Since our invariant set depends on the direction of the reference trajectory, we sweep the invariant set from the minimum rotational angle to the maximum rotational angle in that time interval. Using the swept invariant set and interval hull, we then compute the convex hull for the flow pipe segment using the Minkowski sum~\cite{varadhan2004accurate}. Therefore, the flow pipe for the trajectory can be constructed by repeating the above processes for the next time interval. 

\begin{figure}[!t]
    \centering
    \includegraphics[width=\columnwidth]{images/LyapFlowPipe.png}
    \caption{Construction of Lyapunov Based Flow Pipe~\cite{goppert2019security}}
    \label{fig:construct_flowpipe}
\end{figure}
%

Figure~\ref{fig:flowpipes_application} shows the application of flow pipes to the UAM problem. The green squares represent collision zones or obstacles (e.g., buildings) that the vehicle needs to avoid. Here, we simulate both sinusoidal and square wave wind disturbances. We simulate two different scenarios for the vehicle. The first scenario is simulated with small wind disturbances with a magnitude of $1$ m/s, which is shown at the top of \Cref{fig:flowpipes_application}. We can see that the green collision zones, obstacles, and the flow pipes do not intersect; therefore, we can confirm the safety of the vehicle, since the simulation shows that the vehicle won't hit the buildings during the mission. Another scenario is simulated with large wind disturbances with a magnitude of $5$ m/s, which is shown at the bottom of \Cref{fig:flowpipes_application}. We can see that the buildings and the flow pipes are now intersecting. Therefore, we are able to confirm that it is not safe for the vehicle to do the mission with wind disturbances of this magnitude. We can also see that some simulated trajectories collide with obstacles and the mission is therefore unsafe.

\begin{figure}[!t]
    \centering
    \begin{subfigure}[b]{0.5\textwidth}
        \includegraphics[width=\columnwidth]{images/flowpipe_s.png}
    \end{subfigure}
    \begin{subfigure}[b]{0.5\textwidth}
        \includegraphics[width=\columnwidth]{images/flowpipe_l.png}
    \end{subfigure}
    \caption{Flow Pipes Applications for the UAM Problem, Top: Small Disturbances, Bottom: Large Disturbances}
    \label{fig:flowpipes_application}
\end{figure}

\section{Simulation of the Nonholonomic Rover}
\label{sec:sim2}

In this section, we consider a nonholonomic rover scenario, where a vehicle that cannot move sideways drives along a reference trajectory (e.g., Dubin's vehicle model). In order to prevent the vehicle from moving sideways, the $v_y$ in the dynamics \Cref{eq:Dubin} must be $0$ for all time. Therefore, we have to eliminate the control input in the $y$ direction by setting the control input, $u = LU^{-1}BK\zeta$, for the error dynamics of the group affine system, where $L=\text{diag}[1, 0, 1]$. The error dynamics of the log-linearized system for this case are shown in \Cref{eq:zeta_dot_side}. Since the rover does not move as fast as the fixed-wing aircraft, we generate another reference trajectory with forward velocity and turning radius as $1$ m/s and $1.5$ m, respectively. Here, we simulate two different cases to discuss the size of the tracking error bound of the invariant set from the polytopic system. Figure~\ref{fig:ref_ns_st} shows a reference trajectory going straight with the associated forward velocity and angular velocity. Figure~\ref{fig:ref_ns} shows a reference trajectory with turns, and the associated forward velocity and angular velocity generated from the polynomial trajectory planner.

\begin{figure}[!t]
    \centering
    \includegraphics[width=\columnwidth]{images/ref_traj_ns_st.png}
    \caption{Reference Trajectory with Zero Angular Velocity for Rover Generated from Polynomial Trajectory Planning, Top: Reference Trajectory with Waypoints, Middle: Reference Translational Velocity, Bottom: Reference Angular Velocity}
    \label{fig:ref_ns_st}
\end{figure}

\begin{figure}[!t]
    \centering
    \includegraphics[width=\columnwidth]{images/ref_traj_ns.png}
    \caption{Reference Trajectory with Non-Zero Angular Velocity for Rover Generated from Polynomial Trajectory Planning, Top: Reference Trajectory with Waypoints, Middle: Reference Translational Velocity, Bottom: Reference Angular Velocity}
    \label{fig:ref_ns}
\end{figure}

As we mentioned in \Cref{sec:log-linear}, if $\Delta{A}_p$ in \Cref{eq:zeta_dot_side} can be bounded, the system \Cref{eq:zeta_dot_side} can be treated as a polytopic system with bounded disturbances, and the $\Delta{A}_p$ for the SE(2) Lie group is shown in \Cref{eq:dAp}, where $a_i(t), i \in [1,6]$ are bounded scalar functions of time. $a_6(t)$ in $\Delta{A}_p$ can be bounded with the change of $v_x$, which is $\Delta{A}_2$ in \Cref{eq:poly_matrix}. 

\begin{multline}
\Delta{A}_p = a_1(t)\begin{bmatrix}
    1 & 0 & 0 \\ 0 & 0 & 0 \\ 0 & 0 & 0 \end{bmatrix} + a_2(t)\begin{bmatrix}
    0 & 1 & 0 \\ 0 & 0 & 0 \\ 0 & 0 & 0 \end{bmatrix}\\ + a_3(t)\begin{bmatrix}
    0 & 0 & 1 \\ 0 & 0 & 0 \\ 0 & 0 & 0 \end{bmatrix} + a_4(t)\begin{bmatrix}
    0 & 0 & 0 \\ 1 & 0 & 0 \\ 0 & 0 & 0 \end{bmatrix}\\ + a_5(t)\begin{bmatrix}
    0 & 0 & 0 \\ 0 & 1 & 0 \\ 0 & 0 & 0 \end{bmatrix} + a_6(t)\begin{bmatrix}
    0 & 0 & 0 \\ 0 & 0 & 1 \\ 0 & 0 & 0 \end{bmatrix}
\label{eq:dAp}
\end{multline}
Since the change of $\omega$, which is $\Delta{A}_1$ in \Cref{eq:poly_matrix} is a skew-symmetric matrix, it should not be separated as two different elements and bounded with $a_2(t)$ and $a_4(t)$ in \Cref{eq:dAp}. In our simulation example, we bound $\Delta{A}_p$ with the errors from simulated trajectories to approximate the worst-case scenario. However, $\Delta{A}_p$ can be bounded with an iterative process in order to have a more precise bound.

\subsection{Invariant Set}
The invariant set of the nonholonomic rover going straight is calculated with $v_x =1 $ m/s, $v_y = 0$ m/s, and $\omega = 0$ rad/s based on the reference inputs in \Cref{fig:ref_ns_st}. The invariant set for the case with turning is calculated with $v_x \in [0.9, 1]$ m/s, $v_y = 0$ m/s, and $\omega \in [-\pi/3, \pi/3]$ rad/s based on the reference inputs in \Cref{fig:ref_ns}. For the external disturbances, we consider two different disturbance magnitudes in the $\{x, y\}$ directions, $||w_1||_\infty = 0.05$ m/s and $0.25$ m/s, respectively. The magnitude of disturbances in the $\theta$ direction, $||w_2||_\infty = 0.05$ rad/s. The initial states of the error dynamics in the SE(2) Lie group is set as the same value as the previous UAM scenario, $[x_0, y_0, \theta_0] = [0.1, 0.1, \pi/100]$. 

Figures~\ref{fig:invariant_ns_st} and \ref{fig:invariant_ns} show the two-dimensional projection of the invariant sets in the Lie algebra and the Lie group from the three-dimensional invariant sets, as shown in \Cref{fig:invariant3d_ns_st} and \Cref{fig:invariant3d_ns}. These figures compare the reference trajectories with zero angular velocity and non-zero angular velocity, respectively. Comparing the invariant sets from two different reference trajectories, we can see that the invariant sets in the case of non-zero angular velocity in the reference trajectory are larger, as shown in \Cref{fig:invariant_ns} and \Cref{fig:invariant3d_ns}, since it has to consider the angular velocity during the calculation of Lypaunov function.

\begin{figure}[!t]
    \centering
    \begin{subfigure}[b]{0.5\textwidth}
        \includegraphics[width=\columnwidth]{images/Invariant_ns_st_s.png}
    \end{subfigure}
    \begin{subfigure}[b]{0.5\textwidth}
        \includegraphics[width=\columnwidth]{images/Invariant_ns_st_l.png}
    \end{subfigure}
    \caption{Projected Invariant Set of Rover with Zero Angular Velocity from LMI, Left: Invariant Set in Lie Algebra, Right: Invariant Set in Lie Group, Top: Small Disturbance, Bottom: Large Disturbance}
    \label{fig:invariant_ns_st}
\end{figure}

\begin{figure}[!t]
    \centering
    \begin{subfigure}[b]{0.5\textwidth}
        \includegraphics[width=\columnwidth]{images/Invariant3d_ns_st_s.png}
    \end{subfigure}
    \begin{subfigure}[b]{0.5\textwidth}
        \includegraphics[width=\columnwidth]{images/Invariant3d_ns_st_l.png}
    \end{subfigure}
    \caption{Invariant Set of Rover with Zero Angular Velocity in Three Dimensions from LMI, Left: Invariant Set in Lie Algebra, Right: Invariant Set in Lie Group, Top: Small Disturbance, Bottom: Large Disturbance}
    \label{fig:invariant3d_ns_st}
\end{figure}

\begin{figure}[!t]
    \centering
    \begin{subfigure}[b]{0.5\textwidth}
        \includegraphics[width=\columnwidth]{images/Invariant_ns_s.png}
    \end{subfigure}
    \begin{subfigure}[b]{0.5\textwidth}
        \includegraphics[width=\columnwidth]{images/Invariant_ns_l.png}
    \end{subfigure}
    \caption{Projected Invariant Set of Rover with Non-Zero Angular Velocity from LMI, Left: Invariant Set in Lie Algebra, Right: Invariant Set in Lie Group, Top: Small Disturbance, Bottom: Large Disturbance}
    \label{fig:invariant_ns}
\end{figure}

\begin{figure}[!t]
    \centering
    \begin{subfigure}[b]{0.5\textwidth}
        \includegraphics[width=\columnwidth]{images/Invariant3d_ns_s.png}
    \end{subfigure}
    \begin{subfigure}[b]{0.5\textwidth}
        \includegraphics[width=\columnwidth]{images/Invariant3d_ns_l.png}
    \end{subfigure}
    \caption{Invariant Set of Rover with Non-Zero Angular Velocity in Three Dimensions from LMI, Left: Invariant Set in Lie Algebra, Right: Invariant Set in Lie Group, Top: Small Disturbance, Bottom: Large Disturbance}
    \label{fig:invariant3d_ns}
\end{figure}

\subsection{Flow Pipes}
Figure~\ref{fig:flowpipes_application_ns_st} shows the flow pipes with the case of zero angular velocity in the reference trajectory. The top of \Cref{fig:flowpipes_application_ns_st} shows the case of small translational disturbance, $0.05$ m/s, and the bottom of \Cref{fig:flowpipes_application_ns_st} shows the case of large translational disturbance, $0.25$ m/s. The flow pipes in both cases can bound all trajectories, but the tracking error bound is still large since $\Delta{A}_p$ is bounded with the errors in the worst-case scenario. Figure~\ref{fig:flowpipes_application_ns} shows the flow pipes of the scenario with non-zero angular velocity in the reference trajectory. The top of \Cref{fig:flowpipes_application_ns} shows the case with a small translational disturbance, $0.1$ m/s, which verify the safety of the mission as the flow pipes do not collide with the obstacles. The bottom of \Cref{fig:flowpipes_application_ns} shows the mission is unsafe in the scenario with a large translational disturbance, $0.25$ m/s, since the flow pipes and the obstacles intersect. This result shows that our algorithm can be used in any vehicle as long as the dynamics can be embedded in the Lie group. Although the flow pipes can bound all simulated trajectories, the bound of the flow pipes could be too large due to the bound of the angular velocity in the reference trajectory as we mentioned in the previous paragraph. The bound can be narrowed by considering the invariant set as time-varying, which requires recomputing the Lyapunov function for each time interval.

\begin{figure}[!t]
    \centering
    \begin{subfigure}[b]{0.5\textwidth}
        \includegraphics[width=\columnwidth]{images/flowpipe_ns_st_s.png}
    \end{subfigure}
    \begin{subfigure}[b]{0.5\textwidth}
        \includegraphics[width=\columnwidth]{images/flowpipe_ns_st_l.png}
    \end{subfigure}
    \caption{Flow Pipes Application for the Rover with Zero Angular Velocity, Top: Small Disturbances, Bottom: Large Disturbances}
    \label{fig:flowpipes_application_ns_st}
\end{figure}

\begin{figure}[!t]
    \centering
    \begin{subfigure}[b]{0.5\textwidth}
        \includegraphics[width=\columnwidth]{images/flowpipe_ns_s.png}
    \end{subfigure}
    \begin{subfigure}[b]{0.5\textwidth}
        \includegraphics[width=\columnwidth]{images/flowpipe_ns_l.png}
    \end{subfigure}
    \caption{Flow Pipes Application for the Rover with Non-Zero Angular Velocity, Top: Small Disturbances, Bottom: Large Disturbances}
    \label{fig:flowpipes_application_ns}
\end{figure}

\section{Conclusion}
\label{sec:conclusion}

In this paper, we presented a novel proof for the log-linearization of the error dynamics of mixed-invariant systems. This proof yielded a new correct-by-construction control law based on dynamic inversion. We developed an efficient method to compute the invariant sets for the error dynamics of mixed-invariant systems using our invariant control law with bounded disturbances. Our method is based on log-linearization in the Lie group and Linear Matrix Inequalities (LMIs) for a polytopic system with bounded inputs. Our simulation demonstrated the application of our algorithm to safety verification in an Urban Air Mobility (UAM) scenario. In the simulation, we also compared the LMI approach against an existing reachability analysis method and showed that LMI approach is efficient. The simulation of a rover with nonholonomic constraints proved that our algorithm can be applied to the underactuated vehicle where full control in the Lie algebra is not possible.

As future work, we intend to apply our approach to three-dimensional trajectory planning by employing the SE(3) Lie group and extending our approach to more complex aircraft dynamic models.

\bibliographystyle{IEEEtran}
\bibliography{ref}

\end{document}
