\section{Introduction}
Known as ``the successor of mobile Internet", the concept of Metaverse\cite{xu2022full} has attracted growing attention in both academia and industry. As a future interaction paradigm that requires a rich variety of enabling technologies, Metaverse would revolutionize many domains and their applications work. One possible application could be in the field of the Smart Industry, also known as the Industrial Metaverse. In fact, the Industrial Metaverse is said to be the field closest to realizing Metaverse, and there are already some practices in real factories. For example, Nvidia Omniverse 6 allows BMW to integrate its brick-and-mortar car factory with Virtual Reality (VR), Artificial Intelligence (AI), and robotics to improve its operation precision and flexibility. 

In the preliminary blueprint of the Industrial Metaverse, managers should be able to control real-world equipment to complete complex tasks, such as robotic arms, in a virtual world using advanced technologies such as VR. As one of the fundamental technologies in the Metaverse, Digital Twins\cite{el2018digital} use sensors scattered all around to collect device data and synchronize to the virtual world in real time. This constant stream of data will be then analyzed by tools such as Machine Learning (ML) to assist managers in decision-making and realizing industrial automation. For example, Optical Character Recognition (OCR) is one of the most widely used ML tools, which helps managers to identify product information and equipment status more efficiently. As a distributed ML paradigm designed for a large number of edge devices and with privacy-preserving capabilities, Federated Learning (FL)\cite{kairouz2021advances} is naturally a promising technique in this context. An illustration of the FL-assisted Industrial Metaverse is shown in Figure \ref{fig:fl-assisted-metaverse}.

\begin{figure}[t]
\centering
\includegraphics[width=0.5\textwidth]{pic/metaverse.pdf}
\caption{Illustration of FL-assisted Industrial Metaverse.}
\label{fig:fl-assisted-metaverse}
\end{figure}

However, the fusion of FL and Industrial Metaverse is not a simple combination of technologies. Industrial Metaverse\cite{li2022internet} has extremely demanding requirements for edge intelligence, especially in terms of precision and efficiency\cite{kang2022blockchain}. This poses great challenges from both data and system perspectives under the current industrial network architecture. Specifically, we summarize three main challenges below.

\begin{itemize}
\item[(a)]\textbf{Data Heterogeneity:}
The preference for different classes of data is inherent in every aspect of industrial manufacturing (e.g., among different types of sensors and different functional factories). As a necessary process, manufacturers (e.g., Toyota and Volkswagen) will always prescribe the naming and numbering rules of each component. For example, ``026" and ``034" represent four- and five-cylinder engines and are produced by factories A and B, respectively, and thus A will see more characters ``2" and ``6" than B. Such class biases in the data distribution are called data heterogeneity and have been shown to impair the precision of FL models\cite{zhao2018federated}.

\item[(b)] \textbf{Learning Forgetting:} 
Real-time Digital Twins in Metaverse, combined with In-Memory Computing, enables instant dynamic analysis locally at sensors as streaming data flows in, and allows it to implement real-time industrial systems. However, most sensors have limited memory capacity and cannot store the complete stream of incoming data. For instance, an industrial embedded DDR5 RDIMM has a memory capacity of 16-32GB, but an application can generate gigabits of runtime logs per second. Therefore, old data should be erased from memory to store recent data. Under this paradigm, ML models tend to forget previously learned data knowledge, which is called catastrophic forgetting\cite{french1999catastrophic}.

\item[(c)] \textbf{Limited Bandwidth:}
Wide wireless coverage is often required in open spaces such as mines and blasting sites. The cellular low-power wide-area networks (LPWANs)\cite{chen2022survey}, including NarrowBand IoT, LTE-M, etc., feature long coverage range and low power consumption, and thus become especially favored by the industry. As a price, LPWAN has a low bit rate (e.g., up to 220kbps for NarrowBand IoT and up to 1Mbps for LTE-M\cite{aldahdouh2019survey}), and results in high latency when transmitting ML models and therefore becomes the efficiency bottleneck of FL.
\end{itemize}

Facing the above three challenges, some feasible solutions have already been proposed in the field of FL. However, unfortunately, they are studied separately for a certain challenge, and some are not applicable due to the strict constraints in industrial settings. For the problem of data heterogeneity, some efforts\cite{zhou2016less,jeong2018communication,duan2019astraea} use data augmentation to balance the number of samples in each category. These methods work well on static FL that traverses the local dataset multiple times, but are not suitable for industrial streaming data that can only be seen once. Moreover, the streaming data bring the learning forgetting problem. Related work\cite{rebuffi2017icarl} on continual learning tries to alleviate forgetting by storing historical data, but the storage of a steady stream of industrial data puts great pressure on sensor memory. In terms of system efficiency, compression techniques such as network pruning\cite{hanson1988comparing,srinivas2015data,han2015learning,zhou2016less,wen2016learning,cheng2015exploration}, quantization\cite{gong2014compressing,choi2016towards,courbariaux2015binaryconnect} and sparsification\cite{lin2017deep} have been successful in reducing model size and easing communication burden. However, the pruning-based approaches (e.g., network slimming\cite{liu2017learning}) are data-dependent, which makes their pruned model structure oscillate when dealing with rapidly changing streaming data. The quantization- and sparsification-based approaches have limited compression rates without sacrificing accuracy, and will still burden the small-bandwidth LPWAN network, which requires further compression. 

In order to meet the demanding requirements for precision and efficiency of the Industrial Metaverse, FL still lacks practical methods and algorithms to address these three challenges simultaneously. To achieve this goal, this paper presents a novel, high-performance and efficient system, named \NAME. Firstly, \NAME~adopts a new concept of dynamic \textit{Sequential-to-Parallel (STP)} mode to coordinate the training process between devices. Leveraging the robustness of sequential training to data heterogeneity, STP groups devices and encourages devices within each group to train the federated model sequentially, while devices between groups still run in the parallel mode. The grouping function is provided by a fast \textit{Inter-Cluster Grouping (ICG)} algorithm, which outputs device groups with homogeneous data distribution. As training progresses, the number of groups increases, and the system training mode gradually transforms from sequential to fully parallel. Secondly, for the forgetting problem, \NAME~introduces a \textit{Semantic Compression \& Compensation (SCC)} mechanism. The core idea is to store the compressed historical semantics extracted by the representation layer, which will be replayed and compensated and used to calibrate the classifier. Thirdly, a \textit{Layer-wise Alternative Synchronization Protocol (LASP)} is leveraged, which synchronizes the important but lightweight classifier parameters at a higher frequency to reduce the communication burden. These techniques are natively compatible with changing streaming data by design (Figure \ref{fig:framework}). To the best of the authors' knowledge, this is the first work to simultaneously address all three challenges of data heterogeneity, learning forgetting, and limited bandwidth under the strict constraints of the Industrial Metaverse. 

The main contributions are summarized as follows:
\begin{itemize}
\item{We propose a high-performance and efficient FL system \NAME~for Industrial Metaverse, which incorporates a new concept of STP training mode with a fast ICG grouping algorithm, and can effectively resist the heterogeneity of streaming industrial data to improve FL precision.}
\item{We propose SCC, a semantic compression and compensation mechanism that enables historical data to be recorded and stored in a lightweight form on sensors with limited memory. The stored data can be used to calibrate classifier parameters to alleviate learning forgetting.}
\item{We propose LASP, which allocates more communication resources to synchronize important but lightweight classifier parameters. LASP can achieve comparable performance, but with much less traffic, making it well-suited for bandwidth-limited LPWAN networks.}
\item{Extensive experiments are conducted on 368 simulated devices and the streamed non-i.i.d. FEMNIST dataset. \NAME~improves FL accuracy by at least 6.4\% and saves up to 98\% of overall runtime and transfer bytes, proving its superiority in precision and efficiency.}
\end{itemize}