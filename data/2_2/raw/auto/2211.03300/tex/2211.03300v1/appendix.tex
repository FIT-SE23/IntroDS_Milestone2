\documentclass[journal]{IEEEtran}

\usepackage{amsmath}
\usepackage{array}
\ifCLASSOPTIONcompsoc
\usepackage[caption=false,font=normalsize,labelfont=sf,textfont=sf]{subfig}
\else
 \usepackage[caption=false,font=footnotesize]{subfig}
\fi

\allowdisplaybreaks
\usepackage{amsfonts}
\usepackage{amssymb}
\usepackage{hyperref}
\usepackage[ruled,norelsize,vlined,linesnumbered]{algorithm2e}
\usepackage{float}
\usepackage[misc,geometry]{ifsym}
\usepackage{multirow}
\usepackage{makecell}
\usepackage{doi}
\usepackage{booktabs}
\usepackage{color}
\usepackage{ulem}
\usepackage[font=small,labelfont=bf]{caption}
\usepackage{algpseudocode}

\newtheorem{assumption}{Assumption}
\newtheorem{proposition}{Proposition}
\newtheorem{definition}{Definition}
\newtheorem{proof}{Proof}

\newcommand{\NAME}{\textsc{HFedMS}}
\newcommand{\lizh}{\textcolor{blue}}
\newcommand{\shenglai}{\textcolor{magenta}}
\newcommand{\del}[1]{\textcolor{blue}{\sout{#1}}}
\newcommand{\warn}[1]{\textcolor{red}{#1}}
\newcommand{\removelatexerror}{\let\@latex@error\@gobble}
\renewcommand{\algorithmicrequire}{\textbf{Input:}}
\renewcommand{\algorithmicensure}{\textbf{Output:}}

\makeatletter
\makeatother


\begin{document}

% Appendix
\begin{appendices}

\section{Proof of Proposition 1}\label{proof:prop1}
\begin{proof}
The expectation of cluster centroid $C^m$ can be deduced by
\begin{align}
\nonumber
\mathbb{E}[C^m]&=\mathbb{E}[\frac{1}{L}\sum_{l=1}^{L}\mathcal{V}^m_l] \\
\nonumber
&=\mathbb{E}[\frac{1}{L}\sum_{l=1}^{L}(\mathcal{V}^m_l-C_l+C_l)] \\
\nonumber
&=\mathbb{E}[\frac{1}{L}\sum_{l=1}^{L}(\mathcal{V}^m_l-C_l)+\frac{1}{L}\sum_{l=1}^{L}{C_l}] \\
\nonumber
&=\frac{1}{L}\mathbb{E}[\epsilon^m]+C_{\mathrm{global}} =C_{\mathrm{global}},
\end{align}
This implies that the group centroid and the global centroid are coincident in expectation, and the error is bounded by
\begin{align}
\nonumber
\|C^m-C_\mathrm{global}\|_2^2&=\|\frac{1}{L}\sum_{l=1}^{L}{\mathcal{V}_l^m}-\frac{1}{L}\sum_{l=1}^{L}{C_l}\|_2^2 \\
\nonumber
&=\frac{1}{L^2}\|\sum_{l=1}^{L}{(\mathcal{V}_l^m-C_l)}\|_2^2 \\
\nonumber
&\le\frac{1}{L^2}\sum_{l=1}^{L}\|\mathcal{V}_l^m-C_l\|_2^2 =\frac{1}{L^2}\sum_{l=1}^{L}{\sigma_l^2}.
\end{align}
\end{proof}

\vspace{5mm}
\section{Proof of Proposition 2}\label{proof:prop2}
\begin{proof}
First, we have
\begin{align}
\nonumber
\mu^{r_1}&=\mathbb{E}_{x_i}[h(\mathcal{D}_m^k({r_1}),\phi^{{r_1}-1})]=\mathbb{E}_{x_i}[\mathcal{D}_m^k({r_1})]\cdot\Phi^{{r_1}-1}, \\
\nonumber
\mu^{r_2}&=\mathbb{E}_{x_i}[h(\mathcal{D}_m^k({r_2}),\phi^{{r_2}-1})]=\mathbb{E}_{x_i}[\mathcal{D}_m^k({r_2})]\cdot\Phi^{{r_2}-1}, \\
\nonumber
\mu^{\tilde{r}}_1&=\mathbb{E}_{x_i}[h(\mathcal{D}_m^k(r_1),\phi^{\tilde{r}})]=\mathbb{E}_{x_i}[\mathcal{D}_m^k(r_1)]\cdot\Phi^{\tilde{r}}, \\
\nonumber
\mu^{\tilde{r}}_2&=\mathbb{E}_{x_i}[h(\mathcal{D}_m^k(r_2),\phi^{\tilde{r}})]=\mathbb{E}_{x_i}[\mathcal{D}_m^k(r_2)]\cdot\Phi^{\tilde{r}}.
\end{align}
The semantic drift can be defined as
\begin{align}
\nonumber
\Delta^{r_1}&=\mu^{r_1}-\mu_1^{\tilde{r}}=\mathbb{E}_{x_i}[\mathcal{D}_m^k(r_1)]\cdot(\Phi^{{r_1}-1}-\Phi^{\tilde{r}}), \\
\nonumber
\Delta^{r_2}&=\mu^{r_2}-\mu_2^{\tilde{r}}=\mathbb{E}_{x_i}[\mathcal{D}_m^k(r_2)]\cdot(\Phi^{{r_2}-1}-\Phi^{\tilde{r}}).
\end{align}
To simplify the representation, we make $\mathbb{E}_{x_i}[\mathcal{D}_m^k(r_2)]$ aliased as $\Gamma$ and have $\mathbb{E}_{x_i}[\mathcal{D}_m^k(r_1)]=\Gamma+\Xi$ according to Assumption 2. Then we have $\Delta^{r_1}-\Delta^{r_2}$ as 
\begin{align}
\nonumber
\Delta^{r_1}-\Delta^{r_2}&=(\Gamma+\Xi)\cdot(\Phi^{r_1-1}-\Phi^{\tilde{r}})-\Gamma\cdot(\Phi^{r_2-1}-\Phi^{\tilde{r}}) \\
\nonumber
&=\Gamma\cdot(\Phi^{r_1-1}-\Phi^{r_2-1})+\Xi\cdot(\Phi^{r_1-1}-\Phi^{\tilde{r}}) \\
\nonumber
&\overset{1}{=}\Xi\cdot(\Phi^r-\Phi^{\tilde{r}}),
\end{align}
``$\overset{1}{=}$" holds because $\Phi^r=\Phi^{r_1-1}=\Phi^{r_2-1}$. Therefore, the expectation and covariance of $\Delta^{r_1}-\Delta^{r_2}$ over any round $r_1$ and $r_2$ are
\begin{align}
\nonumber
\mathbb{E}_r[\Delta^{r_1}-\Delta^{r_2}]&=\mathbb{E}_r[\Xi]\cdot(\Phi^r-\Phi^{\tilde{r}})\overset{2}{=}\boldsymbol{0}, \\
\nonumber
\mathrm{Cov}(\Delta^{r_1}-\Delta^{r_2})&=\mathbb{E}_r[(\Delta^{r_1}-\Delta^{r_2}-\mathbb{E}_r[\Delta^{r_1}-\Delta^{r_2}])^2] \\
\nonumber
&=\mathbb{E}_r[(\Xi\cdot(\Phi^r-\Phi^{\tilde{r}}))^2] \\
\nonumber
&=(\Phi^r-\Phi^{\tilde{r}})^T\cdot\mathbb{E}_r[\Xi^T\cdot\Xi]\cdot(\Phi^r-\Phi^{\tilde{r}}) \\
\nonumber
&\overset{3}{\rightarrow}\boldsymbol{0},
\end{align}
where ``$\overset{2}{=}$" holds because $\mathbb{E}_r[\Xi]=\boldsymbol{0}$, and ``$\overset{3}{\rightarrow}$" holds because $|\Xi|\rightarrow 0$ makes $\Xi^T\cdot\Xi\rightarrow\boldsymbol{0}$.
\end{proof}

\end{appendices}

\end{document}