
\section{Conclusion}\label{sec:conclusion}

In this exploratory study, we collected a mobile sensing dataset and around 329K self-reports from 678 participants in eight countries (China, Denmark, India, Italy, Mexico, Mongolia, Paraguay, UK) for over three weeks to assess the effect of geographical diversity on mood inference models. We evaluated country-specific, continent-specific, country-agnostic, and multi-country approaches trained on sensor data for two mood inference tasks with population-level (non-personalized) and hybrid (partially personalized) models. We showed that partially personalized country-specific models perform the best yielding AUROC scores in the range of 0.78-0.98 for two-class (negative vs. positive) and 0.76-0.94 for three-class (negative vs. neutral vs. positive) inference. Further, with the country-agnostic approach, we showed that models do not perform well compared to country-specific settings, even when models are partially personalized. We also uncovered generalization issues of sensing-based mood inference models to new countries. We hope that these findings will be of benefit to ubicomp researchers towards building future mobile sensing applications with an awareness of geographical diversity. 


\begin{acks}

This work was funded by the European Union’s Horizon 2020 WeNet
project, under grant agreement 823783. We deeply thank all
the volunteers across the world for their participation in the study.

\end{acks}