\section{Introduction}\label{sec:introduction}

Mental well-being related issues are common among young adults due to a plethora of personal and societal reasons such as leaving home, study workload, poor financial stability, and complex social relationships \cite{patel2007mental, rickwood2007and}. These issues are even more prominent in the post-pandemic world, where social relationships have taken a toll due to more emphasis on remote work/study settings. Some studies have shown that this emerging lifestyle has affected phone usage behavior as well \cite{stockwell_changes_2021, zheng_covid-19_2020, ratan_smartphone_2021, saadeh_smartphone_2021, li_impact_2021}. Further, declining mental well-being conditions could lead to adverse outcomes such as substance abuse and suicidal thoughts \cite{rowe2003substance, crosby2011suicidal, franklin2017risk}. In this context, prior research has discussed the potential of timely and accurate mood tracking for both personal and clinical care \cite{spathis2019passive, wang2016crosscheck, faherty2017pregnancy, mark2008chart}. Ecological momentary assessments (EMAs) and survey questionnaires are commonly used for mood tracking. However, such techniques are burdensome to users, and prior work has shown that it is difficult to sustain the practice of reporting for long periods unless there is a strong motivation \cite{baumel2019objective, rapp2014self, schueller2021understanding}. As a possible alternative, multi-modal sensors in smartphones could be used to infer mood unobtrusively with reasonable accuracies \cite{pratap2019accuracy, likamwa2013moodscope, servia2017mobile}. 

According to prior work in psychology and social sciences, physiological aspects, including mood, are perceived and expressed differently in different countries, cultures, and societies \cite{luomala2004cross} \footnote{For pragmatic reasons, we are equating the geographical location (country) of our participants with a specific culture that is distinct to this particular country. We acknowledge that cultures can be multidimensional and exist in tension with each other and in plurality within the same geographic boundary \cite{yuval2004gender}. However, throughout the paper, we use country, culture, and geographic region interchangeably.}. According to a cross-country study by Becht et al. \cite{becht2002crying}, mood and related behaviors could vary based on a person's culture, and perceptions and beliefs regarding different moods stemming from one's culture. However, prior work in mobile sensing does not study the effect of the geographical diversity of users (e.g., country of residence) on smartphone sensing-based mood inference models. 


Issues of generalization and fairness with regard to the geographical diversity of data sources have been discussed extensively in domains such as computer vision, speech, and natural language processing \cite{grother2019face, zou2018ai, wang2020towards, castelvecchi2020facial, malhotra2008automatic}. For example, gender classification models trained with data predominantly from the USA have performed poorly on people of African and Asian descent \cite{castelvecchi2020facial}. Many geographical-related biases (e.g., Indian brides being recognized as dancers, etc.) have been shown in models trained with the imagenet dataset, in which a majority of data is from western countries \cite{zou2018ai}. Such findings have uncovered issues in data collection practices and helped shape research directions to address issues related to diversity and biases. In this context, many prior mobile sensing studies that attempt inferences regarding well-being related aspects highlighted that models are trained in specific countries, and the generalization of techniques for other countries or regions should be explored further \cite{choi2021kairos, muller2021depression, meegahapola2021examining, meegahapola2021one}. However, mood inference studies have focused on only one or two countries for data collection \cite{likamwa2013moodscope} or have not considered the diversity of data sources in terms of the country, even when data were collected from multiple countries \cite{servia2017mobile}. 

Bardram et al. \cite{bardram2020decade} emphasized the need for generalization and reproducibility of sensing-based models for mental well-being-related outcomes. However, even though examining gender, age, and occupation-related diversity is feasible even within the same country, examining geographical diversity requires a considerable effort in conducting the same study, with the same protocol, in several geographic regions because studies are time-consuming and expensive; and logistical difficulties in conducting experiments such as language barriers, technology barriers, differences in motivating use cases and required incentives. Hence, studies that examine the geographical diversity of mobile sensing-based inferences are rare \cite{phan2022mobile, khwaja2019modeling}. In this paper, we study and compare the performance of country-specific, country-agnostic, and multi-country approaches for mood inference. In addition, we also examine the effects of model personalization and generalization to new geographically diverse countries. To our knowledge, this is one of the first studies to examine the effect of geographical diversity of users on smartphone sensing-based mood inference models, hence shedding light on distributional shift related issues. Considering these aspects, we ask three research questions. 

\begin{itemize}[wide, labelwidth=!, labelindent=0pt]
    \item[\textbf{RQ1:}] What behavioral and contextual characteristics around mood reports of college students (from eight countries spanning Europe, Asia, and Latin America) can be extracted from the analysis of smartphone sensing and self-report data?
    \item[\textbf{RQ2:}] How do smartphone sensing-based mood inference models perform in different countries (country-specific)? Can a model trained in one/more countries be deployed in another country not seen on training data to achieve reasonable accuracies, hence generalizing well (country-agnostic)?
    \item[\textbf{RQ3:}] How do country-specific or continent-specific models perform as compared to a multi-country model? 
\end{itemize}{}


By addressing the above research questions, this paper provides the following contributions: 
\begin{itemize}[wide, labelwidth=!, labelindent=0pt]
    \item[\textbf{Contribution 1:}] We conducted a new smartphone-based data collection campaign among 678 participants in eight countries (China, Denmark, India, Italy, Mexico, Mongolia, Paraguay, UK) representing Europe, Asia, and Latin America to study their everyday mood and behavior. During the study, we collected 329,974 fully complete self-reports. In addition, we also collected rich passive sensing data with continuous sensing (activity type, step count, location, cellular, wifi, bluetooth, proximity, etc.) and interaction sensing (app usage, touch events, user presence, screen-on/off episodes, notifications, etc.) throughout the deployment. First, we found that negative mood reports in all countries would increase from morning to night. Moreover, with statistical analysis, we found that the features that help infer mood are different across countries. However, the best features included both continuous and interaction sensing modalities in all countries. 
    
    \item[\textbf{Contribution 2:}] We found that the country-specific approach performs reasonably for both two-class and three-class mood inferences with AUROC scores in the range of 0.76-0.98 with hybrid (i.e., partially personalized) models. However, we noticed that across both two-class and three-class inferences, models do not generalize well to other countries, where AUROC scores drop to the range of 0.46-0.55 on average in the population-level (i.e., non-personalized) setting and 0.66-0.73 in the hybrid setting. These findings raise the significance of discussing issues of generalization of mobile sensing-based models to different world regions. 
    
    \item[\textbf{Contribution 3:}] In the hybrid setting, we found that multi-country models do not perform as well as country-specific models even though they achieved an AUROC of 0.81. However, they performed better than continent-specific models built for Asia and worse than the one built for Europe. Even though the performance differences were not high, this again highlights that building a model within European countries leads to higher performance and better generalization for those countries than using multi-country or even some country-specific models. A possible explanation is that the European countries under study (Italy, Denmark, UK) might share some daily behavioral patterns. In contrast, the three countries in Asia under study (China, India, Mongolia) have less similarity regarding daily patterns. Hence, these findings point toward the benefit of considering the geographical/cultural diversity of data collection on smartphone sensing-based mood inference models. 
\end{itemize}{}

The study is organized as follows. In Section~\ref{sec:related_work}, we describe the background and related work. Then we describe the data collection procedure in eight countries and how we came up with features in Section~\ref{sec:mobile_app}. Section~\ref{sec:data_analysis} provides a descriptive and a statistical analysis of data. In Section~\ref{sec:inference}, we define the analysis strategy and evaluate two-class and three-class mood inference with population-level and hybrid models with approaches: country-specific, continent-specific, country-agnostic, and multi-country. We discuss the main findings and implications in Section~\ref{sec:discussion}, and conclude the paper in Section~\ref{sec:conclusion}.

