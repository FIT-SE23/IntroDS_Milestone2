\section{Study Design, Data Collection, and Feature Extraction}\label{sec:mobile_app}



\begin{figure}
    \includegraphics[width=\textwidth]{images/diversity_mood_overview.pdf}
    \caption{High-level overview of the study.}
    % \Description{}
    \label{fig:overview}
\end{figure}


With a team representing computer science, social sciences, user experience design, and ethics from institutions in over ten countries, we designed an exploratory study and developed a mobile sensing application to collect passive smartphone sensing and self-report data from participants about their everyday life behavior and well-being. The ultimate goal of this deployment was to study their behavior, including aspects such as activity, social context, location, mood, and sleep quality from a mobile sensing standpoint and also to consider various diversity aspects that could potentially affect sensing-based inferences (ranging from geographical region and gender to personality and values). The study is summarized in Figure~\ref{fig:overview}. The study design consisted of two main components: (a) LimeSurvey component to collect survey responses during pre and mid-study phases; and (b) Mobile sensing app to collect sensor data and self-reports. A technical report regarding the study procedure and future plans for dataset access is available in \cite{giunchiglia2022aworldwide}.

\subsection{Survey Questionnaires}

Survey responses were captured from participants with three questionnaires sent to them before and during the pilot at three different times. This was done to ensure that the burden on participants was reasonable. These questionnaires were administered through the LimeSurvey platform \cite{engard2009limesurvey}. 

\subsubsection{Pre-Study Diversity Questionnaire}
The primary objective of this questionnaire was to capture diversity attributes of participants from different perspectives. As the first step, basic demographic information was captured, including gender, age, sex, degree program, and socioeconomic status. Then, in an attempt to capture the psychosocial profile, the 20-item Big Five Inventory (BFI) \cite{donnellan2006mini} and Basic Value Survey (BVS) \cite{gouveia2014functional} were administered. Finally, there were several questions regarding social relationships (virtual and real) and cultural consumption that they were interested in.

\subsubsection{Mid-Study Questionnaire I and II}
The objective of the first questionnaire was to gather more detailed information about personality using the Jungian Scale on Personality Types \cite{jung2016psychological} and Human Values Survey \cite{schwartz1994there}. In addition, questions regarding physical activity and sports, cooking and shopping habits, transport methodologies, and cultural activities were captured. The second questionnaire consisted of the Multiple Intelligences Profiling Questionnaire \cite{tirri2008identification}. This also contained several open-ended questionnaires about the mobile app user experience. 

\subsection{Mobile Application}\label{subsec:mobile_app}

An Android mobile application was used to capture the everyday behavior of participants using short in-situ self-report questions. The app was developed such that data would be stored in an SQLite database on the phone, and later when the phone is connected to a Wifi network, data would be uploaded to the main server and free up the local phone storage. In addition, the app could send push notifications by using Google Firebase as a notification broker. Hence, the three main components of the application are: (1) a push notification system that would send periodical reminders to participants to fill in self-reports; (2) mobile time diaries to capture self-reports; and (3) a smartphone sensing component to collect passive sensing data from multiple modalities.

\subsubsection{Push Notifications.} Given the nature of the study and the requirement to capture behavioral and situational data in a particular moment, the app sent reminders for participants to fill in in-situ self-reports regarding their everyday life behavior around 20 times throughout the day. In addition, start-of-the-day and end-of-the-day questionnaires were administered at the beginning and end of the day. When a notification was not clicked and a participant did not complete the self-report within two hours, the notification expired, and a new notification would be sent later. This allowed to keep track of participant compliance (e.g., how many self-reports were answered from the total number of notifications sent). 

\subsubsection{Time Diaries and Start/End-of-the-Day Questionnaires.} The start-of-the-day questionnaire was sent to participants at 8 am each day. It only had two questions with five-point likert scales (very good to very bad): (i) sleep quality; and (ii) expectations about the day. The end-of-the-day questionnaire was sent to participants at 10 pm and asked them (a) to rate their day (five-point likert scale; very good to very bad); (b) if they had any problems during the day (open response), and (c) how did they solve them (open response). The time diary was sent to users once every 30-60 minutes. While this allowed capturing longitudinal behavior granularly, it also introduced user burden. Therefore, the time diary was designed to minimize user burden and reduce completion time. Hence, after several iterations of discussions, only four questions were included in this component: (i) current activity: 34 activities including eating, working, attending a lecture, etc.; (ii) semantic location: 26 categories including home, workplace, university, restaurant, etc.; (iii) social context: 8 categories including alone, with the partner, family member/s, friends, etc.; and (iv) current mood: five-point likert scale to capture the valence of the circumplex mood model \cite{russell1980circumplex} similar to LiKamwa et al. \cite{likamwa2013moodscope}, with an emoji-scale. As explained in Section~\ref{sec:related_work}, this is the variable we chose as this paper's primary focus. 


\begin{table*}
    \caption{\textcolor{red}{Summary of 105 features extracted from sensing data, aggregated around activity self-reports using a time window. A detailed description about sensing modalities is provided in Appendix A.}}
    \label{tab:agg-features}
    %\centering
    \begin{tabular}{{p{0.139\linewidth}p{0.11\linewidth}p{0.725\linewidth}}}
        
        \cellcolor[HTML]{EDEDED}\textbf{Modality} &
        \cellcolor[HTML]{EDEDED}\textbf{Frequency} &
        \cellcolor[HTML]{EDEDED} \textbf{Features and Description}
        \\
        
        
        \makecell[l]{Location} & 
        \makecell[l]{1 sample \\per minute} & 
        \makecell[l]{radius of gyration, distance traveled, mean altitude} \\
\arrayrulecolor{Gray}
    \midrule
    
        \makecell[l]{Bluetooth \\{[}low energy, \\normal{]}} &
        \makecell[l]{1 sample \\per minute} & 
        \makecell[l]{number of devices (the total number of unique devices found), mean/std/min/max \\rssi (Received Signal Strength Indication -- measures how close/distant other \\devices are)} \\
        
        
\arrayrulecolor{Gray}
    \midrule
    
        \makecell[l]{WiFi} & 
        \makecell[l]{1 sample \\per minute} & 
        \makecell[l]{connected to a network indicator, number of devices (the total number of unique \\devices found), mean/std/min/max rssi} \\
        
\arrayrulecolor{Gray}
    \midrule
    
        \makecell[l]{Cellular {[}GSM, \\WCDMA, LTE{]}} & 
        \makecell[l]{1 sample \\per minute} & 
        \makecell[l]{number of devices (the total number of unique devices found), mean/std/min/max \\phone signal strength} \\
        
\arrayrulecolor{Gray}
    \midrule
    
        \makecell[l]{Notifications} & 
        \makecell[l]{on change} & 
        \makecell[l]{notifications posted (the number of notifications that came to the phone), \\notifications removed (the number of notifications that were removed by the \\user) -- these features were calculated with and without duplicates.} \\
        
\arrayrulecolor{Gray}
    \midrule
    
        \makecell[l]{Proximity} & 
        \makecell[l]{10 samples \\per second} & 
        \makecell[l]{mean/std/min/max of proximity values} \\
        
\arrayrulecolor{Gray}
    \midrule
        \makecell[l]{Activity} & 
        \makecell[l]{2 samples \\per minute} & 
        \makecell[l]{time spent doing activities: still, in\_vehicle, on\_bicycle, on\_foot, running, tilting, \\walking, other (derived using the Google activity recognition API \cite{GoogleActivity2022})} \\
        
\arrayrulecolor{Gray}
    \midrule
        \makecell[l]{Steps} & 
        \makecell[l]{10 samples \\per second \\ or on change} & 
        \makecell[l]{steps counter (steps derived using the total steps since the last phone turned on \\at 10 samples per second), steps detected (steps derived using event triggered for \\each new step captured on change)} \\
        
\arrayrulecolor{Gray}
    \midrule
        \makecell[l]{Screen events} &
        \makecell[l]{on change} & 
        \makecell[l]{number of episodes (episode is from turning the screen of the phone on until the \\screen is turned off), mean/min/max/std episode time (a time window could have \\multiple episodes), total time (total screen on time within the time window)} \\
        
\arrayrulecolor{Gray}
    \midrule
        \makecell[l]{User presence} &
        \makecell[l]{on change} & 
        \makecell[l]{time the user is present using the phone (derived using android API that indicate \\whether a person is using the phone or not)} \\
        
\arrayrulecolor{Gray}
    \midrule
        \makecell[l]{Touch events} &
        \makecell[l]{on change} & 
        \makecell[l]{touch events (number of phone touch events)} \\
        
\arrayrulecolor{Gray}
    \midrule
        \makecell[l]{App events} & 
        \makecell[l]{10 samples \\per minute} & 
        \makecell[l]{time spent on apps of each category derived from Google Play Store \cite{likamwa2013moodscope, santani2018drinksense}: \\action, adventure, arcade, art \& design, auto \& vehicles, beauty, board, books \& \\
        reference, business, card, casino, casual, comics, communication,
        dating,\\education, entertainment, finance, food \& drink, health \&
        fitness, house, lifestyle,\\maps \& navigation, medical, music, news \&
        magazine, parenting, personalization, \\photography, productivity,
        puzzle, racing, role playing, shopping, simulation, \\social, sports,
        strategy, tools, travel, trivia, video players \& editors, weather, word, \\ not\_found}   \\
        
        
        \hline
    \end{tabular}
\end{table*}

\subsubsection{Sensor Data and Features} The app collected sensor data from a range of sensors passively. Hence, sensor data included continuous sensing modalities such as accelerometer, gyroscope, ambient light, location, magnetic field, pressure, activity labels generated by the google activity recognition API, step count, proximity, and available Wi-Fi and bluetooth devices. Interaction sensing modalities included application usage, typing and touch events, on/off screen events, user presence, and battery charging events. The modalities and features crafted from each modality are summarized in Table~\ref{tab:agg-features}. In feature engineering, interpretability was a key factor as all the features were defined in a meaningful manner. Similar to prior work in ubicomp, we used a time window-based approach for matching sensor data to self-reports \cite{servia2017mobile, likamwa2013moodscope, meegahapola2021one}. While different time windows can be chosen based on the application scenario, this paper presents results with a dataset created using a time window of 10 minutes. Hence, if the self-report regarding mood occurred at time $T$, sensor data would be considered from $T-5$ minutes to $T+5$ minutes. However, we also considered other time windows, such as 2, 4, 15, and 20 minutes. Results showed that the 10-minute time window performed better for this task. This could be because shorter time windows do not capture enough behaviors and contexts around self-reports to make a meaningful prediction regarding mood. Prior work has shown that larger time windows can capture a high amount of information about user behaviors \cite{bae2017detecting}. However, we can not use very large windows above 20 minutes because it would lead to a situation where sensor data segments for self-reports might get overlapped, leading to data overlap between samples. Therefore, throughout the paper, we present results with a ten-minute time window. In addition, in this paper, we do not discuss why each sensing modality was chosen and how features were derived. This is because such details have been discussed extensively in many prior studies on mobile sensing for health and well-being \cite{servia2017mobile, khwaja2019modeling, santani2018drinksense, likamwa2013moodscope, meegahapola2021one, bae2017detecting, biel2018bites, wang2014studentlife, meegahapola2020smartphone}.

\subsection{Participant Recruitment and Deployment}

The primary objective of this study was to capture data from diverse student populations. While many facets of diversity could be captured by experimenting within the same country, it is difficult to study geographical diversity in such a way. Hence, we conducted mobile sensing experiments in eight countries representing Europe, Asia, and Latin America. Details regarding the data collection are mentioned in Table~\ref{tab:num-participants}. According to prior work in mobile sensing, many studies have focused on Europe and North America, but not much research has been conducted in other world regions \cite{meegahapola2020smartphone, phan2022mobile}. Hence, conducting the same study with the same protocol in multiple countries allows to study mood inference models and geographical diversity in a novel sense. The study was conducted in the following phases. 


\subsubsection{Translation and Adaptation} In this phase, each site received the English version of the questionnaires and the app, including time diaries and the list of sensors to be collected. These tools were evaluated and adapted, in coordination with all the partners, to the specific context (e.g., invitation letters, type and amount of incentives for the participants of the mobile app, privacy and ethics documentation, etc.). Some countries made minimal changes to better adapt the questionnaire to the local situation or academic organization. Concerning the standard scales mentioned above, the translations were completed by a forward translator from the original English version and then validated back via panel and back-translation processes by independent translators. In addition, whenever a validated questionnaire translation was available, we used it (e.g., the Big five traits questionnaire is readily available in several languages). After translation and adaptation, the tools were tested locally. A first test was conducted to check and validate the translations and evaluate the tools' usability. A second test was conducted by sending the questionnaires to a small sample of participants, both project partners and students from various universities. As far as questionnaires were concerned, approximately 30 participants were involved. This test was also used to ascertain the completion times. Concerning the mobile app, a two-week validation test was carried out.

\subsubsection{Invitations, Pre-Study Diversity Questionnaire, and Participants} This was the first of the three phases of the data collection. This phase started by sending an email containing the survey description, the invitation to the first questionnaire, and information on the second part of the data collection (sensing component) via university mailing lists. This invitation was then reiterated through four weekly reminders to all students who still needed to complete the survey. Over 20000 college students were contacted with mailing lists in the initial recruitment phase. Out of the set of people who were contacted, 13398 participants filled in the pre-study diversity questionnaire. Then, a subset of the eligible participants was selected to participate in the second part of the study, which was done with the mobile app. The requirements for the selection were two-fold: (i) having consented to the processing of personal data -- this required participants agreeing to release mobile data collected during the study after anonymization; and (ii) owning an Android smartphone compatible with the app.

\subsubsection{Mid-Study Questionnaire I, II and Mobile Sensing app} Of all the participants who completed the pre-study diversity questionnaire, 678 participants were chosen for the next phase with the mobile sensing app. This deployment was done between September and November 2020. The average age of study participants was 24.2 years (SD: 4.2), and the cohort had 58\% females. They were sent emails with a specification manual to download and install the mobile sensing app. In addition, the participants completed the mid-study questionnaire I. Reminders were sent after one week for participants who still needed to complete the questionnaire. Then, participants completed time diaries, and sensing data were passively collected in the mobile app. After two weeks of mobile sensing app usage, the mid-study questionnaire II was sent to participants via email. After sending out this questionnaire, two more weeks of mobile sensing data collection were conducted. Daily reports were produced to facilitate monitoring the time diary survey and identify possible problems, including: (1) the number of notifications each participant responded to; and (2) the amount of data collected by the individual sensors. Using this information, local field supervisors could contact the inactive participants every three days and support them as needed. A further element of contact was the daily sending of the results of a daily prize, which was an additional incentive for participants. Finally, this paper will only focus on the mood variable captured during the study, and deeper analyses around other questionnaires captured with pre-study, mid-study I, and mid-study II questionnaires will be done in future publications with different scopes.



\begin{table}
    \caption{Participants of the mobile sensing data collection (countries named in alphabetical order).}
    \label{tab:num-participants}
    \centering
    \begin{tabular}{llrrrr}
        
        
        \cellcolor[HTML]{EDEDED}\textbf{Country} &
        \cellcolor[HTML]{EDEDED}\textbf{University} &
        \cellcolor[HTML]{EDEDED}\textbf{Participants} &
        \cellcolor[HTML]{EDEDED}\textbf{$\mu$ Age ($\sigma$) } &
        \cellcolor[HTML]{EDEDED}\textbf{\% Women} &
        \cellcolor[HTML]{EDEDED} \textbf{\# Self-Reports }
        \\
         
        
        China &
        %Anonymized &
        Jilin University &
        41 &
        26.2 (4.2) & 
        51 &
        22,289
        \\
        
        \arrayrulecolor{Gray}
        \hline
        
        Denmark &
        %Anonymized &
        Aalborg University &
        24 &
        30.2 (6.3) & 
        58 &
        10,010
        \\
        
        \arrayrulecolor{Gray}
        \hline
        
        India &
        %Anonymized &
        Amrita Vishwa Vidyapeetham &
        39 &
        23.7 (3.2) & 
        53 &
        4,233
        \\
        
        \arrayrulecolor{Gray}
        \hline
        
        Italy &
        %Anonymized &
        University of Trento &
        240 &
        24.1 (3.3) &
        58 &
        151,342
        \\ 
        
        
        \arrayrulecolor{Gray}
        \hline
        
        Mexico &
        %Anonymized &
        \makecell[l]{Instituto Potosino de Investigación\\ Científica y Tecnológica} &
        20 &
        24.1 (5.3) & 
        55 &
        11,662
        \\
        
        \arrayrulecolor{Gray}
        \hline
        
        Mongolia &
        %Anonymized &
        National University of Mongolia &
        214 &
        22.0 (3.1) &
        65 &
        94,006
        \\
        
        \arrayrulecolor{Gray}
        \hline
        
        Paraguay &
        %Anonymized &
         \makecell[l]{Universidad Católica \\"Nuestra Señora de la Asunción"} &
        28 &
        25.3 (5.1) &
        60 &
        9,744
        \\
        
        \arrayrulecolor{Gray}
        \hline
        
        UK &
        %Anonymized &
        \makecell[l]{London School of Economics\\ \& Political Science} &
        72 &
        26.6 (5.0) & 
        66 &
        26,688
        \\
        
        
        
        \cellcolor[HTML]{EDEDED}\textbf{Total/Mean} &
        \cellcolor[HTML]{EDEDED}\textbf{} &
        \cellcolor[HTML]{EDEDED}\textbf{678} &
        \cellcolor[HTML]{EDEDED}\textbf{24.2 (4.2)} &
        \cellcolor[HTML]{EDEDED}\textbf{58} &
        \cellcolor[HTML]{EDEDED}\textbf{329,974} 
        \\
        
    \end{tabular}
    
%\vspace{-0.2 in}
\end{table}





\subsubsection{Incentive Design} An incentive scheme was designed to motivate participants to complete time diaries and provide sensing data. Incentives included monetary prizes for participants who completed at-least 85\% of time diaries (e.g., 20 Euro in Italy, 150 Kr in Denmark, etc.), cash prizes for multiple daily winners randomly chosen from each pilot (e.g., five winners were given a prize of 5 Euro in Italy, 5 MNT in Mongolia, etc.). In the end, three winners from each country were randomly chosen for a larger prize (e.g., 150 Euros per person in Italy, 150 Sterling Pounds in the UK, etc.). Incentives in all countries were designed by considering each country's socioeconomic status and expecting all participants to be compensated and motivated equally. 

\subsubsection{Ethical Procedures} All the survey activities and results at each site comply with the national ethical privacy-protecting laws and guidelines, hence getting approvals from respective ethical review boards. In addition, all the experiments, including non-European pilots, were compliant with the General Data Protection Regulation (GDPR) \cite{voigt2017eu}. Additionally, for non-European experiments, the activities and results have been developed to comply with those of a European country for compliance purposes. More specifically, Italian legislation was selected as the reference. 

 
