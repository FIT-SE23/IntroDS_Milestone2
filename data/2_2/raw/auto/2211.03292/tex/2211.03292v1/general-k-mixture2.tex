%!TEX root = onetrace.tex

\subsection{Construction of $\calM$ satisfying \Cref{claim:mixture} for any constant $k$}

Let ${\cal S}_\ell$ be the set of $m:={\frac 1 2} \log_{1/\eps^4} \ell$ many $\ell$-bit strings

\[
{\cal S}_\ell = \left\{
(0^{1/\eps^{4u}}1^{1/\eps^{4u}})^{\ell/(2/\eps^{4u})}
\right\},
\quad u=1,\dots,m.
\]

Fix any positive integer $k$ (which should be thought of as a fixed constant, while $\ell \to \infty$).
In this section we construct a collection of $m$ mixtures ${\cal M}^{(1)},\dots,{\cal M}^{(m)}$, where each ${\cal M}^{(u)}$ is a mixture of $\ell$-bit strings, which meet the conditions required by \Cref{claim:mixture}. 
In more detail, we show that the mixtures ${\cal M}^{(1)},\dots,{\cal M}^{(m)}$ that we construct satisfy the following:

\begin{lemma} \label{lem:good-mixture}
For each $u \in [m]$ we have the following:

\begin{enumerate}

\item  With probability $1-o_\ell(1)$, a random $\ell$-bit string $\bz$ drawn from ${\cal M}^{(u)}$ has 
\[
  \abs[\bigg]{\LCS \Bigl(\bz,(0^{1/\eps^{4u}}1^{1/\eps^{4u}})^{\ell/(2/\eps^{4u})} \Bigr)} \geq (1-o_\ell(1))\ell.
\]

\item For each $u \in [m]$ the $k$-deck $\deck_k({\cal M}^{(u)})$ is the same.

\end{enumerate}

\end{lemma}

\noindent {\bf The mixture ${\cal M}^{(u)}$.}
Fix $u \in [m]$ and let $r_0 := 1/\eps^{4u}$.
For $t$ dividing $\ell$, let $x^{(t)}$ denote the $\ell$-bit string
\[
x^{(t)} := (0^t 1^t)^{\ell/(2t)},
\]
so $x^{(r_0)}$ is the $u$-th string $(0^{1/\eps^{4u}}1^{1/\eps^{4u}})^{\ell/(2/\eps^{4u})}$ in ${\cal S}_\ell$.
The mixture ${\cal M}^{(u)}$ will be supported on $k$ strings in $\zo^\ell$,
\[
\supp({\cal M}^{(u)}) = \{x^{(r_0)}, x^{(r_1)}, \dots, x^{(r_{k-1})}\},
\]
where $r_1,\dots,r_{k-1}$ are values that will satisfy $r_0 \ll r_1 \ll \cdots \ll r_{k-1} \ll \ell$  and that will be specified later.
The mixing weight $p_j$ on the $j$-th string $x^{(r_j)}$ will be chosen so that  each $p_j \geq 0$,
 $\sum_{j=0}^{k-1} p_j = 1$ (so ${\cal M}^{(u)}$ is indeed a valid distribution), and %moreover so that each of $p_1,\dots,p_{k-1}$ is $o_\ell(1)$; from this it will follow that 
$p_0 = 1-o_\ell(1)$, which gives item (1) of \Cref{lem:good-mixture}.

To achieve item (2) of \Cref{lem:good-mixture} we will carefully choose the weights $p_0,\dots,p_{k-1}$ so that for each $y \in \zo^k$, the value
$\deck_k({\cal M}^{(u)})_y$ is a function only of $\ell$ (and in particular is independent of the value of $u$).  Towards this end, let us begin to analyze the $k$-deck of a single string $x^{(t)}$. The following is easily verified:

\begin{claim} \label{claim:k-deck-xt}
Fix any $y \in \zo^k$. The value $\deck_k(x^{(t)})_y$ is of the form
\begin{equation} \label{eq:dkxty}
\deck_k(x^{(t)})_y = \sum_{i=0}^{k-1} t^i f_{y,i}(\ell) 
\end{equation}
for some polynomials $f_{y,0}(\ell),\dots,f_{y,k-1}(\ell)$.
% where the degree of $f_{y,j}$ is at most $k-j$. Moreover, the polynomial $f_{y,0}(\ell)$ is $f_{y,0}(\ell) = {\frac 1 {k! \cdot 2^k}} \ell^k + g_{y,0}(\ell)$ where the degree of $g_{y,0}(\ell)$ is at most $k-1$. \red{XXX}\rnote{Need to fix this up more and say something about the size of $f_{y,k-1}(\ell)$}
\end{claim}


From \Cref{eq:dkxty} we immediately get that
\begin{equation}
\label{eq:dkMjy}
\deck_k({\cal M}^{(i)})_y = \sum_{j=0}^{k-1} p_j \left(\sum_{i=0}^{k-1} r_j^i f_{y,i}(\ell) \right)
=
\sum_{i=0}^{k-1} \left( \sum_{j=0}^{k-1} p_j r_j^i \right) f_{y,i}(\ell).
\end{equation}
Recall that $r_0=1/\eps^{4u}$, so clearly $r_0$ depends on $u$, and that we have yet to choose $r_1,\dots,r_{k-1}$. \Cref{eq:dkMjy}  leads us to consider the following linear system:
\begin{equation}
\label{eq:lin-system}
V p = b
\end{equation}
where $V$ is the $k \times k$ Vandermonde matrix whose rows and columns we index by $i \in \{0,\dots,k-1\}$ and $j \in \{0,\dots,k-1\}$,
\begin{equation} \label{eq:vandermonde}
V_{i,j} = r_j^i,
\end{equation}
and $p$ and $b$ are $k \times 1$ 
column vectors
\newcommand{\myvec}[1]{\ensuremath{\begin{pmatrix}#1\end{pmatrix}}}
\[
p = 
\myvec{p_0 \\ \vdots \\p_{k-1}},
\quad \quad \quad
b = 
\myvec{b_0\\ \vdots \\ b_{k-1}}.
\]
We will prove the following claim:

\begin{claim}
\label{claim:indep-of-u}
There are values $b_0,\dots,b_{k-1}$ %\leq \blue{small-function-of-$\ell$} 
that have no dependence on $u$ so that the solution
\begin{equation}
\label{eq:solution}
p = V^{-1} b
\end{equation}
to the system (\ref{eq:lin-system}) has each $p_j \geq 0$, $\sum_{j=0}^{k-1} p_j = 1$, and $p_0 = 1 - o_\ell(1).$ 
\end{claim}

By \Cref{eq:dkMjy} this means that the $k$-deck 
\[
\deck_k({\cal M}^{(i)})_y  = \sum_{i=0}^{k-1} b_i f_{y,i}(\ell),\quad \quad
y \in \zo^k,
\]
has no dependence on $u$, giving item (2) of \Cref{lem:good-mixture} and completing its proof.
It thus remains to prove \Cref{claim:indep-of-u}.

\subsubsection{Proof of \Cref{claim:indep-of-u}}

We start by recalling an explicit formula for the inverse of a Vandermonde matrix:

\begin{fact} [\cite{Turner66}]
Let $V=(V_{ij})_{i,j \in \{0,\dots,k-1\}}$ be the $k \times k$ Vandermonde matrix $V_{i,j} = r_j^i$ as specified in \Cref{eq:vandermonde}.
Let $e^{(i)}_j$ be the $j$-th elementary symmetric polynomial on the $k-1$ variables $r_0,\dots,r_{i-1},r_{i+1},\dots,r_{k-1}$.  
Then the inverse matrix $V^{-1}$ is given by
\begin{equation} 
\label{eq:inverse}
V^{-1}_{i,j} = {\frac {(-1)^{j} \cdot e^{(i)}_{k-1-j}}
{\prod_{s \neq i} (r_s - r_i)}}.
\end{equation}
\end{fact}
It will be convenient for us to rewrite \Cref{eq:inverse} in a way which makes the denominator always positive (recall that we will have $r_0 \ll r_1 \ll \cdots \ll r_{k-1}$. Doing this, we obtain
\begin{equation}
\label{eq:inverse2}
V^{-1}_{i,j} = {\frac {(-1)^{i+j} \cdot e^{(i)}_{k-1-j}}
{
\left(
\prod_{0 \leq s \leq i-1} (r_i - r_s)
\right)
\cdot
\left(
\prod_{i+1 \leq s \leq k-1} (r_s - r_i)
\right)
}
},
\end{equation}
and consequently we have that
\begin{equation}
\label{eq:p}
p = V^{-1} b, \quad \text{where for $i = 0,\dots,k-1$,} \quad
p_i = 
{\frac
{
\sum_{j=0}^{k-1} (-1)^{j+i} e^{(i)}_{k-1-j} \cdot b_j
}
{
\left(
\prod_{0 \leq s \leq i-1} (r_i - r_s)
\right)
\cdot
\left(
\prod_{i+1 \leq s \leq k-1} (r_s - r_i)
\right)
}
}
\end{equation}
(note that the denominator of \Cref{eq:p} is independent of $j$).


We now choose $r_j, b_j: j \in [k-1]$ appropriately and show that the $p_i$'s satisfy the conditions in \Cref{claim:indep-of-u}.
Recall that $m = \frac{1}{2} \log_{1/\eps^4} \ell$, so $r_0 = 1/\eps^{4u} \leq1/\eps^{4m} = \sqrt{\ell}$.
For $j \in [k-1]$, we define
\[
  b_j := \frac{1}{(\log\log\ell)^j} \cdot \prod_{s=1}^j r_j \quad\text{and}\quad 
  r_j := \ell^{2/3} \cdot (\log \ell)^j 
\]
(observe that $r_0$ is already fixed to $1/\eps^{4u}$, and that the first row of the Vandermonde matrix system of equations is all-1's, which means that  $b_0 = p_0 + \cdots + p_{k-1}=1$).
These settings are chosen so that in the summation in the numerator of the expression for $p_i$ in \Cref{eq:p}, the $(j=i)$-th term, which is always positive, dominates the sum of the rest of the terms in magnitude.
Specifically, we will show that for $j < i$, the quantity $e^{(i)}_{k-1-j} b_j$ is at most $O((\log\ell)^{-1}) \cdot \prod_{s=1}^{k-1} r_s$ and for $j \ge i$, we have $e^{(i)}_{k-1-j} b_j = (\log\log\ell)^{-j} (1 + o_\ell(1)) \prod_{s=1}^{k-1} r_s$.
So the numerator is at least $(\log\log \ell)^{-i} (1 - o_\ell(1)) \prod_{s=1}^{k-1} r_s \ge 0$, and thus the $p_i$'s are positive because the denominator is positive.
Moreover, the denominator of $p_0$ in \Cref{eq:p} is at most $\prod_{s=1}^{k-1} r_s$.
This shows $p_0 = 1 - o_\ell(1)$.


We now give the full calculation.
First observe that for every $0 \le j \le k-2$ and every $S \subseteq [k-1]$ of size $k-1-j$ not equal to $\{j+1, \ldots, k-1\}$, we have
\begin{equation}
\prod_{s \in S} r_s
  \leq \ell^{2\abs{S}/3} \cdot (\log \ell)^{\sum_{s \in S} s}
  \le \ell^{2\abs{S}/3} \cdot (\log \ell)^{(\sum_{s=j+1}^{k-1} j) - 1}
  = \frac{1}{\log \ell} \prod_{s=j+1}^{k-1} r_s .
  \label{eq:lowerorder}
\end{equation}
So for $j < i$ we have $i \in \{j+1, \ldots, k-1\}$ and so the (positive) quantity $  e_{k-1-j}^{(i)} \cdot b_j$ is ``small,'' i.e. at most $O((\log\ell)^{-1}) \cdot \prod_{s=1}^{k-1} r_s$:
\begin{align}
  e_{k-1-j}^{(i)} \cdot b_j
  &= \biggl( \sum_{\substack{S \subseteq \{0,\dots,k-1\} \setminus \{i\} \\ \abs{S} = k-1-j}} \prod_{s \in S} r_s \biggr) \cdot \biggl( (\log\log\ell)^{-j} \prod_{s=1}^j r_j \biggr) \nonumber \\
  &\le \biggl( \binom{k-1}{j} \frac{1}{\log \ell} \prod_{s=j+1}^{k-1} r_s \biggr) \cdot \biggr( (\log\log\ell)^{-j} \prod_{s=1}^j r_s \biggr) \nonumber \\
  &\le (\log\log\ell)^{-j} \cdot \biggl( \prod_{s=1}^{k-1} r_s \biggr) \cdot \frac{k^j}{\log \ell}
\end{align}
For $j = i$ the (positive) quantity $  e_{k-1-j}^{(i)} \cdot b_j$ is ``large,'' i.e. at least $(\log\log\ell)^{-i} \prod_{s=1}^{k-1} r_s$;
more precisely,
we have
\begin{align}
  e_{k-1-j}^{(i)} \cdot b_j
  &= \biggl( \sum_{\substack{S \subseteq \{0,\dots,k-1\} \setminus \{i\} \\ \abs{S} = k-1-j}} \prod_{s \in S} r_s \biggr) \cdot \biggl( (\log\log\ell)^{-j} \prod_{s=1}^j r_j \biggr) \nonumber \\
  &\ge \biggl(  \prod_{s=j+1}^{k-1} r_s \biggr) \cdot \biggl( (\log\log\ell)^{-j} \prod_{s=1}^j r_j \biggr) \nonumber \\
  &=  (\log\log\ell)^{-j} \cdot \biggl( \prod_{s=1}^{k-1} r_s \biggr).
\end{align}
For $j > i$  the (positive) quantity $  e_{k-1-j}^{(i)} \cdot b_j$ is again ``small,'' i.e. $(\log\log\ell)^{-j} (1 + o_\ell(1)) \prod_{s=1}^{k-1} r_s$:
\begin{align}
  e_{k-1-j}^{(i)} \cdot b_j
  &= \biggl( \prod_{s=j+1}^{k-1} r_s + \sum_{\substack{S \subseteq \{0,\dots,k-1\} \setminus \{i\} \\ \abs{S} = k-1-j \\ S \ne \{j+1, \ldots, k-1\}}} \prod_{s \in S} r_s \biggr) \cdot \biggl( (\log\log\ell)^{-j} \prod_{s=1}^j r_j \biggr) \nonumber \\
  &\le  \Biggl( \biggl( \prod_{s=j+1}^{k-1} r_s \biggr) \cdot \biggl( 1 + \binom{k-1}{j} \frac{1}{\log\ell} \biggr) \Biggr) \cdot \biggl( (\log\log\ell)^{-j} \prod_{s=1}^j r_j \biggr) \nonumber \\
  &= (\log\log\ell)^{-j} \cdot \biggl( \prod_{s=1}^{k-1} r_s \biggr) \cdot \biggl(1 + \frac{k^j}{\log \ell} \biggr) .
\end{align}
Therefore for every $i \in \{0, \ldots, k-1\}$, the alternating sum is dominated by the contribution from $j=i$: more precisely, we have
\begin{align*}
  \sum_{j=0}^{k-1} (-1)^{j+i} e_{k-1-j}^{(i)} \cdot b_j
  &\ge \biggl( \prod_{s=1}^{k-1} r_s \biggr) \Biggl(\sum_{j=i}^{k-1} (-1)^{i+j} (\log\log\ell)^{-j} - \frac{1}{\log\ell} \sum_{\substack{0 \le j \le k-1 \\ j \ne i}} \left(\frac{k}{\log\log\ell}\right)^j \Biggr) \\
  &\ge \biggl( \prod_{s=1}^{k-1} r_s \biggr) \biggl( (\log\log\ell)^{-i} \left(1 - \frac{1}{\log\log\ell} \right) - \frac{2}{\log \ell} \biggr) \\
  &\ge 0 .
\end{align*}
Since, as noted earlier, the denominator of \Cref{eq:p} is positive, this shows that $p_i \ge 0$ for every  $i \in \{0, \ldots, k-1\}$.
Moreover, we have
\begin{align*}
  p_0
  &= \frac{ \sum_{j=0}^{k-1} (-1)^{j} e_{k-1-j}^{(i)} \cdot b_j }{ \prod_{s=1}^{k-1} (r_s - r_0) } \\
  &\ge \frac{ \biggl( \prod_{s=1}^{k-1} r_s \biggr) \Bigl( 1 - \frac{1}{\log\log\ell} - \frac{2}{\log\ell} \Bigr) } { \prod_{s=1}^{k-1} r_s } \\
  &\ge 1 - \frac{2}{\log\log\ell}.
\end{align*}
This completes the proof of \Cref{claim:indep-of-u}. \qed
