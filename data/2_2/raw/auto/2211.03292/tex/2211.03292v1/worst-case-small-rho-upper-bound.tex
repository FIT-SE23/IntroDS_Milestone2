%!TEX root = onetrace.tex

\subsection{Bounds on the performance of any one-trace (or few-trace) algorithms}

Complementing \Cref{thm:worst-case-small-rho}, we show that for worst-case approximate trace reconstruction, even if the total number of bits obtained across multiple traces is $n^{0.999}$, it is not possible to achieve expected $\LCS$ of $(2/3 + c)n$ for any constant $c > 0$. The following theorem gives a more detailed version of \Cref{thm:worst-case-small-rho-upper-bound-informal}.

%\red{Add a sentence or two of preliminary prose before the theorem statement motivating it}

\begin{theorem} [Worst-case upper bound on any few-trace algorithm, small retention rate] \label{thm:worst-case-small-rho-upper-bound}
Let $\kappa>0$ be any absolute constant and let $t(n),\rho(n)=1-\delta(n)$ be such that $t(n) \rho(n) \leq 1/n^{\kappa}$. For sufficiently large $n$, we have 
\[
L_{t(n),\worst}(\delta(n),n) \leq (2/3 + o_n(1))n.\]
\end{theorem}

In order to prove \Cref{thm:worst-case-small-rho-upper-bound}, we first introduce some additional notation.

\subsubsection{Notation} \label{sec:deck-notation}

\noindent {\bf Decks.} 
For $k \in \N$, the \emph{$k$-deck} of a string $z \in \zo^n$, denoted $\deck_k(z)$, is the vector in $\Z^{\zo^k}$ whose $y$-th element (for $y \in \zo^k$) is the number of occurrences of $y$ as a length-$k$ subsequence of $z$. 

Let ${\cal M}$ be a mixture of $n$-bit strings with mixing weights $p_1,\dots,p_m$ on strings $z^1,\dots,z^m \in \zo^n$ (in other words ${\cal M}$ is a distribution over $n$-bit strings). The \emph{$k$-deck of ${\cal M}$}, denoted $\deck_k({\cal M})$, is defined to be the following vector in $\R^{\zo^k}$:
\[
\deck_k({\cal M}) = \sum_{i=1}^m p_i \deck_k(z^i).
\]

Given $y \in \zo^k$ we write $\deck_k(z)_y$ to denote the $y$-th element of $\deck_k(z)$ and $\deck_k({\cal M})_y$ to denote the $y$-th element of $\deck_k({\cal M})$.
Note that for any string $z \in \zo^n$ we have $\sum_{y \in \zo^k} \deck_k(z)_y = {n \choose k}$, and likewise $\sum_{y \in \zo^k} \deck_k({\cal M})_y={n \choose k}$ for any mixture ${\cal M}$ of $n$-bit strings.

\medskip

\noindent {\bf Segments.} 
We view an $n$-bit source string $x \in \zo^n$ as being composed of $n/\ell$ consecutive segments of length $\ell$, for some $\ell = \ell(n)$. 
%We will be considering strings in which each segment is balanced, i.e., each segment contains equal number of $0$'s and $1$'s.\snote{I don't think we need to impose $\ell < 1/\rho$; check this.} %We will be considering %(for now at least)
%the setting where $\ell < 1/\rho$.\snote{check if this is necessary} 

\medskip

\noindent {\bf Average LCS of a set.} 
Given any set of strings ${\cal S} \subseteq \zo^n$, define
\[
\AvgLCS({\cal S}) := \max_{x' \in \zo^n} {\frac 1 {|{\cal S}|}} \sum_{s \in {\cal S}} |\LCS(x',s)|,
\]
i.e.,~$\AvgLCS({\cal S})$ is the largest possible value (over all possible hypothesis strings $x' \in \zo^n$) of the average LCS between an element of ${\cal S}$ and $x'$. 

We will relate $L_{t(n),\worst}(\delta(n),n)$ to $\AvgLCS({\cal S})$ of a set ${\cal S}$ which is (a slight modification of) the \emph{Bukh--Ma code}, a set of $n$-bit strings that was first studied in \cite{BukhMa14} and further analyzed in \cite{GHS20}.

\subsubsection{The Bukh--Ma code} \label{sec:bukh-ma}

Fix a segment length $\ell=\ell(n)$ which divides $n$. Take $\eps$ to be a suitable $o_n(1)$ value, and let $C_{n,\eps}$ be the Bukh--Ma code analyzed in \cite{GHS20}:  
\begin{equation} \label{eq:Cneps}
C_{n,\eps}=\left\{
(0^r1^r)^{{\frac n {2r}}}: r = {\frac 1 {\eps^{4u}}}, u = 1,\dots,{\frac 1 2} \log_{1/\eps^4} \ell
\right\}.
\end{equation}

We denote the string $(0^r1^r)^{{\frac n {2r}}}$ where $r = {\frac 1 {\eps^{4u}}}$ by $A_u$, for $u = 1,\dots,{\frac 1 2}\log_{1/\eps^4} \ell$.
We remark that for each string $A_u$ in the Bukh--Ma code above, the ``period" $2r = 2/\eps^{4u}$ divides the segment length $\ell$.
%\red{Emphasize to the reader that for each string in the Bukh-Ma code, the segment length divides $\ell$.}


\begin{theorem}[Implicit in the proof of \cite{GHS20}, Theorem 1.4] \label{thm:GHS}
    For any $x \in \zo^n$, there can be at most ${\frac {1200}{\eps^3}}$ many strings $A_u \in C_{n,\eps}$ %(here $A_u$ is the string with each run of length $1/\eps^{4u}$)
    that have $|\LCS(x,A_u)| \geq (2/3 + \eps/6)n.$
\end{theorem}
\newcommand{\adv}{\mathrm{adv}}
\noindent
\emph{Proof sketch:} We explain how \Cref{thm:GHS} is implicit in the proof of Theorem~1.4 of \cite{GHS20}. 
In \cite{GHS20}, it is shown %shows 
(see Section~3, starting after the proof of their Lemma~3.1) that for any $x \in \zo^n$,  if a set of $m$ strings from $C_{n,\eps}$ is such that each of the $m$ strings (call the string $s$) has $\adv(x,s)> \eps/2$, then we must have $m %\xnote{Should the $k$ be $m$?} 
\leq 1200/\eps^3$. 
Since
$\adv(x,s) = {\frac {3 |\LCS(x,s)| - |x|-|s|}{|x|}}$
(see \cite{GHS20}'s Definitions~2.4 and 2.5),
having $\adv(x,s) > \eps/2$ is %the same as 
equivalent to having $|\LCS(x,s)| \geq (2/3 + \eps/6)n$. %thusandsuch\xnote{Not sure if this is intended.}$
\qed

\medskip

Fix $x \in \zo^n$. Using \Cref{thm:GHS}, we can upper bound the average $\LCS$ of $x$ with $C_{n,\eps}$ by %Even in the worst case in which all those strings have an LCS with $x$ of length $n$, we get that 
\begin{equation} \label{eq:twothirds}
    {\frac 1 {|C_{n,\eps}|}} \sum_{s \in C_{n,\eps}} |\LCS(x,s)| \leq {\frac {2 \cdot 1200/\eps^3}{\log_{1/\eps^4} \ell}} \cdot n + (2/3 + \eps/6)n = (2/3 + o(1))n.
%\AvgLCS(C_{n,\eps}) \leq {\frac {1200/\eps^3}{\log_{1/\eps^4} \ell}} \cdot n + (2/3 + \eps/6)n = (2/3 + o(1))n.
\end{equation}
As this is true for all $x \in \zo^n$, we conclude that 
\begin{equation}
\label{eq:avglcs}
\AvgLCS(C_{n,\eps}) \leq (2/3 + o(1))n.
\end{equation}

\subsubsection{Relating $L_{t(n),\worst}(\delta(n),n)$ to $\AvgLCS({\cal S})$}

The following claim will allow us to upper bound the performance of any algorithm that receives $t = t(n)$ traces at deletion rate $\delta(n)$ by (essentially) $\AvgLCS({\cal S})$ for any set ${\cal S}$ satisfying certain properties.

%Add some explanation that this claim is how we upper bound the performance of any $t(n)$-trace algorithm. Fix notation $L_{t(n),\worst(\delta(n), n)}$

\begin{claim} \label{claim:mixture}
Let $\ell$ be such that both $\ell$ and $n^\ell$ are at least $n^c$ for some positive constant $c$.
Let $\calS_\ell = \{s^{(1)}_\ell, s^{(2)}_\ell, \dots, s^{(m)}_\ell\} \subset \zo^\ell$ be a set of $\ell$-bit strings. 
Define the set of $n$-bit strings $\calS_n = \{s^{(1)}_n, s^{(2)}_n, \cdots, s^{(m)}_n\} \subset \zo^n$, where each %$s^{(u)}_n = s^{(u)}_\ell^{\otimes(n/\ell)}$, i.e., each
    string $s^{(u)}_n$ is constructed by concatenating $n/\ell$ copies of $s^{(u)}_\ell$. 
    For each $u \in [m]$ let ${\cal M}^{(u)}$ be a mixture of $\ell$-bit strings with the following properties:
    
    \begin{enumerate}
    
    \item With probability $1-o(1)$, a random $\ell$-bit string $\bz$ drawn from ${\cal M}^{(u)}$ has $\LCS(\bz,s^{(u)}_\ell) \geq (1-o(1))\ell$;
    
    \item For each $u \in [m]$ the $k$-deck $\deck_k({\cal M}^{(u)})$ is the same.
    
    \end{enumerate}
    
    %Let ${\cal M}$ be the following distribution over $n$-bit strings: to draw $\bx \sim {\cal M}$, first draw a uniform $\bj \in [\log_{1/\eps^4} \ell]$, then independently draw $n/\ell$ many $\ell$-bit strings $\bx^{(1)},\dots,\bx^{(n/\ell)} \sim {\cal M}^{(\bj)}$, and concatenate them to yield $\bx = \bx^{(1)} \cdots \bx^{(n/\ell)}.$
    Let $\rho(n) = 1 - \delta(n)$. Then we have
    \[
        L_{t(n),\worst}(\delta(n),n) \leq t(n) \cdot \ell^k \cdot \rho(n)^{k+1} \cdot n^2 + \AvgLCS(\calS_n) + o(n).
    \]
\end{claim}

\begin{proof}
    Let ${\cal M}$ be the following distribution over $n$-bit strings: to draw $\bx \sim {\cal M}$, first draw a uniform $\bu \sim [m]$, then independently draw $n/\ell$ many $\ell$-bit strings $\bx^{(1)},\dots,\bx^{(n/\ell)} \sim {\cal M}^{(\bu)}$, and concatenate them to yield $\bx = \bx^{(1)} \cdots \bx^{(n/\ell)}.$

    Let $A$ be any algorithm that takes as input $t := t(n)$ traces $\by^{(1)}, \cdots, \by^{(t)}$ of $\bx$, and outputs an $n$-bit hypothesis string. We suppose that in addition to the input traces, $A$ is also told, for each trace, how many bits of the trace come from each of the $n/\ell$ segments of the source string; we upper bound $L_{t(n),\worst}(\delta(n),n)$ by upper bounding the performance of any algorithm that also receives this extra auxiliary information.
    
    The probability that any of the $n/\ell$ many $\ell$-bit segments of $\bx$ has at least $k + 1$
    bits from it surviving into any of the $t$ traces is at most $t \cdot (n/\ell) \cdot (\rho(n) \ell)^{k+1} = t n \ell^k \rho(n)^{k+1}$. In this case we trivially upper bound the $\LCS$ between $\bx$ and the output of $A$ by $n$.

    Otherwise, at most $k$ bits survive from each segment in each trace. The distribution of these bits is the same, regardless of the random $\bu \sim [m]$ chosen in the construction of $\bx$. This follows from property (2.) above and the easily observable fact that if the $k$-deck $\deck_k({\cal M}^{(u)})$ is the same for each $u \sim [m]$, then the $k'$-deck $\deck_{k'}({\cal M}^{(u)})$ is also the same for each $u \sim [m]$, for all $k' \leq k$. In this case, the optimal string for algorithm $A$ to output is the $n$-bit string $x^*$ that achieves $\AvgLCS(\calS_n)$.

    By property (1.) above and a standard Chernoff bound, with $1-o(1)$ probability we have that a $1-o(1)$ fraction of the $n/\ell$ strings $\bx^{(1)}, \cdots, \bx^{(n/\ell)}$ drawn from $\calM^{(\bu)}$ satisfy $|\LCS(\bx^{(i)}, s^{(\bu)}_\ell| \geq (1-o(1))\ell$, %$|\LCS(\bx^{(i)}, C_{n,\eps}^{(\bu)})| \geq (1-o(1))\ell$, 
    so with $1-o(1)$ probability the string $\bx=\bx^{(1)}\cdots \bx^{(n/\ell)}$ has $|\LCS(\bx,s^{(\bu)}_n)| \geq (1-o(1))n$.
    
    %\red{Finally, by the triangle inequality on the edit distance $d_{edit}(z,z') := n - |\LCS(z,z')|$ (which is a metric), we have $|\LCS(\bx, s^{(u')}_n)| \leq |\LCS(s^{(\bu)}_n, s^{(u')}_n)| + o(1)n$.\snote{Complete proof with all details.}}

    Recall that $x^* \in \zo^n$ is the string achieving $\AvgLCS(\calS_n)$. We will use the triangle inequality on the edit distance $d_{edit}(z,z') := n - |\LCS(z,z')|$ (which is a metric). We have
    \[
      d_{edit}(\bx, x^*) \geq d_{edit}(x^*, s^{(\bu)}_n) - d_{edit}(s^{(\bu)}_n, \bx).
    \]
    Rewriting this inequality in terms of $\LCS$, we have
    \[
       | \LCS(\bx, x^*)| \leq |\LCS(x^*, s^{(\bu)}_n)| + n - |\LCS(\bx,s^{(\bu)}_n)| \leq |\LCS(x^*, s^{(\bu)}_n)| + o(n).
    \]
    We emphasize that $\bx$ is a function of the random $\bu \sim [m]$, while $x^*$ is independent of $\bu$. Taking expectation over $\bu$, we get that
    %\snote{This is conditional expectation given that we don't see more than $k$ bits in any segment; need to emphasize this?}
    \[
        \E_{\bu}[|\LCS(\bx, x^*)|] \leq \AvgLCS(\calS_n) + o(n).
    \]
    Combining the two cases above, we obtain the lemma.
\end{proof}

\begin{proof}[Proof of \Cref{thm:worst-case-small-rho-upper-bound} using \Cref{claim:mixture}] In \Cref{lem:good-mixture} below, for any constant $k \in \N$, we will exhibit a set of mixtures ${\cal M}^{(u)}$ satisfying the properties in \Cref{claim:mixture}, with the set $\calS_n$ being $C_{n,\eps}$. Choosing $k = 4/\kappa$ (constant), $\ell = n^{1/k}$, and using the fact that $\AvgLCS(C_{n,\eps}) \leq (2/3 + o(1))n$ (recall \Cref{eq:avglcs}), we conclude that
    \begin{align*}
        L_{t(n),\worst}(\delta(n),n)
        &\leq t(n) \, \ell^k \, \rho(n)^{k+1} \, n^2 + \AvgLCS(\calS_n) + o(n)\\
        &\leq (t(n) \rho(n))^{k+1} n^3 + (2/3 + o(1))n\\
        &\leq n^{3 - (k+1)\kappa} + (2/3 + o(1))n\\
        &\leq (2/3 + o(1))n. \qedhere
    \end{align*}
%(This will say that in Lemma 22 we give a set of strings satisfying Claim 21, and work through the parameters to get to the Theorem 19 statement.)}
\end{proof}


% %%%%%%%%%%%%%%%%
% START IGNORE
% %%%%%%%%%%%%%%%%
\ignore{

%
%\subsubsection{Construction of $\calM$ satisfying \Cref{claim:mixture} for $k=3$}
%
%Fix $j \in \log_{1/\eps^4} \ell$, and let $z = C^{(j)}_{n, \eps}$. We will construct a mixture of four strings, two of which are close to $z$ in terms of edit distance (these two have almost all the weight in the mixture) and two of which are far from $z$ in edit distance.
%
%\medskip
%
%To begin, as in the previous section we have $z := (0^a 1^a)^{\ell/(2a)}$.  We need to know what is the 3-deck of $z$. A little calculation yields that
%\begin{align}
%D_3(z)_{000}=D_3(z)_{111}={\ell/2 \choose 3} &= {\frac {\ell^3} {48}} - {\frac {\ell^2} 8} + {\frac \ell 6}, \label{eq:000}\\
%D_3(z)_{010}=D_3(z)_{101}=\sum_{j=1}^{\ell/(2a)} a^3 j\left({\frac \ell {2a}} - j\right) &=  {\frac {\ell^3} {48}} \red{- {\frac {a^2 \ell}{12}}}\label{eq:010}\\
%D_3(z)_{001}=D_3(z)_{011}=\sum_{j=1}^{\ell/(2a)} a^3 {j \choose 2} + \sum_{j=1}^{\ell/(2a)} a {a \choose 2} j&=  {\frac {\ell^3} {48}} - {\frac {\ell^2}{16}} + \red{{\frac {a \ell^2}{16}} + {\frac {a^2 \ell}{24}} - {\frac {a \ell} 8}}
%\label{eq:001}\\
%D_3(z)_{100}=D_3(z)_{110}=\sum_{j=1}^{\ell/(2a)} a^3 {j-1 \choose 2} + \sum_{j=1}^{\ell/(2a)} a {a \choose 2} (j-1)&=  {\frac {\ell^3} {48}} - {\frac {\ell^2}{16}} \red{ - {\frac {a \ell^2}{16}} + {\frac {a^2 \ell}{24}} + {\frac {a \ell} 8}}
%\label{eq:100}
%\end{align}
%(In (\ref{eq:010}), think of the index $j$ as the $j$-th block of 1s that the 1 in 010 comes from; there are $j$ possible blocks for the preceding 0 in 010, and ${\frac \ell {2a}} - j$ possible blocks for the succeeding 0 in 010. In (\ref{eq:001}), think of the index $j$ in each summand as the $j$-th block of 1s that the 1 in 001 comes from; the first summand has $a^3$ and ${j \choose 2}$ because it corresponds to having the two zeros be in two different preceding blocks of zeros among the $j$ possibilities, and the second summand has $a {a \choose 2}$ and $j$ because it corresponds to having the two zeros be in the same preceding block of zeros (again $j$ possibilities). (\ref{eq:100}) follows from similar reasoning.)
%
%Intuitively, the \red{red things} above are what we need to get rid of (because they are the terms that depend on $a$ and hence would be different for one $C^{(j)}_{\ell,\eps}$ versus another  $C^{(j')}_{\ell,\eps}$). \blue{The sense in which we will ``get rid of them'' is {\bf not} that we will replace them by zero; instead, we will create a mixture in which the red things (which involve $a$) are replaced by expressions that involve $\ell$ but don't involve $a$. Thus the resulting expressions will be the same no matter what $a$ is, which is what we need to achieve.
%
%\medskip
%
%The overall mixture of strings that we create will be 
%\begin{equation} \label{eq:mixture}
%(1-\eps) \cdot {\frac {z + \overline{z}}{2}} + \eps \cdot {\frac {y + \overline{y}}{2}}.
%\end{equation}
%
%The string $z$ is defined above, and the string $\overline{z}$ is the bitwise complement of $z$, i.e.~$\overline{z}=(1^a 0^a)^{\ell/(2a)}$. (We note that we also have $\overline{z}=z^R$ where $z^R$ is the reversal of $z$; this will be useful for us soon.) Using these symmetries it's not difficult to write down the 3-deck of $\overline{z}$ using equations (\ref{eq:000}) through (\ref{eq:100}): we get that
%
%\begin{align}
%D_3(\overline{z})_{000}=D_3(\overline{z})_{111}
%= 
%D_3({z})_{000}=D_3({z})_{111}
%&= {\frac {\ell^3} {48}} - {\frac {\ell^2} 8} + {\frac \ell 6} \nonumber\\
%D_3(\overline{z})_{010}=D_3(\overline{z})_{101}
%=
%D_3(z)_{010}=D_3(z)_{101}&=  {\frac {\ell^3} {48}} \red{- {\frac {a^2 \ell}{12}}}\label{eq:ol010}\\
%D_3(\overline{z})_{001} = D_3(\overline{z})_{011} =
%D_3(z)_{100}=D_3(z)_{110}&=  {\frac {\ell^3} {48}} - {\frac {\ell^2}{16}} \red{ - {\frac {a \ell^2}{16}} + {\frac {a^2 \ell}{24}} + {\frac {a \ell} 8}} \label{eq:ol100}\\
%D_3(\overline{z})_{100}=D_3(\overline{z})_{110}=
%D_3(z)_{001}=D_3(z)_{011} &=  {\frac {\ell^3} {48}} - {\frac {\ell^2}{16}} + \red{{\frac {a \ell^2}{16}} + {\frac {a^2 \ell}{24}} - {\frac {a \ell} 8}}
%\label{eq:ol100}.
%\end{align}
%
%Hence writing $A$ for ${\frac 1 2}(D_3(z) + D_3(\overline{z}))$, we have
%\begin{align}
%A_{000} = A_{111} &= {\frac {\ell^3} {48}} - {\frac {\ell^2} 8} + {\frac \ell 6} 
%\label{eq:A000}\\
%A_{001} =A_{011}=A_{100}=A_{110} &={\frac {\ell^3}{48}} - {\frac {\ell^2}{16}} \violet{+ {\frac {a^2 \ell}{24}}}
%\label{eq:A001}\\
%A_{010} = A_{101} &= {\frac {\ell^3} {48}} \violet{- {\frac {a^2 \ell}{12}}} \label{eq:A010}
%\end{align}
%
%Our remaining task is to replace the violet expressions by expressions that don't depend on $a$.
%
%Now let's explain what is $y$ and finish the analysis by setting $\eps$ and computing the 3-deck of the mixture (\ref{eq:mixture}). We take
%\[
%y := 0^{\ell/4} 1^{\ell/2} 0^{\ell/4} = y^R;
%\quad \quad \text{so we have} \quad \quad
%\overline{y} =1^{\ell/4} 0^{\ell/2} 1^{\ell/4} = y^R.
%\]
%Since $y$, like $z$, has $\ell/2$ zeros and $\ell/2$ ones, we have
%\begin{equation}
%D_3(y)_{000}=D_3(\overline{y})_{000}
%= 
%D_3({z})_{000}=D_3(\overline{z})_{000}
%= {\frac {\ell^3} {48}} - {\frac {\ell^2} 8} + {\frac \ell 6} \label{eq:bovary}
%\end{equation}
%and likewise
%\begin{equation}
%D_3(y)_{111}=D_3(\overline{y})_{111}
%= 
%D_3({z})_{111}=D_3(\overline{z})_{}
%= {\frac {\ell^3} {48}} - {\frac {\ell^2} 8} + {\frac \ell 6}. \label{eq:charles}
%\end{equation}
%It is immediate that we have
%\begin{equation}
%D_3(y)_{010}= {\frac {\ell^3}{32}},
%D_3(\overline{y})_{010}= 0,
% \label{eq:emma}
%\end{equation}
%and similarly for the 101-counts.
%So writing $B$ for ${\frac 1 2}(D_3(y) + D_3(\overline{y}))$, we get
%\begin{align}
%B_{000} =B_{111}=A_{000}=A_{111} &=
%{\frac {\ell^3} {48}} - {\frac {\ell^2} 8} + {\frac \ell 6} \label{eq:B000}\\
%B_{010} =B_{101} &=
%{\frac {\ell^3} {64}} \label{eq:B010}
%\end{align}
%For $B_{001},B_{110},B_{001},B_{011}$ we observe that by complement-symmetry of $B$ we have $B_{001}=B_{110}$ and $B_{100}=B_{011}$, and by left-right symmetry of $B$ we have $B_{001}=B_{100}$, so
%\begin{align}
%B_{001}=B_{100}=B_{110}=B_{011}={\frac {{\ell \choose 3} - 2 \cdot (\ref{eq:B000}) - 2 \cdot (\ref{eq:B010})}{4}}.
%\label{eq:Brest}
%\end{align}
%Now let us write $C$ for $D_3((\ref{eq:mixture})) = D_3((1-\eps) \cdot {\frac {z + \overline{z}}{2}} + \eps \cdot {\frac {y + \overline{y}}{2}}),$ so
%\begin{equation}
%C = (1-\eps)\cdot A + \eps \cdot B. \label{eq:C}
%\end{equation}
%By (\ref{eq:A000}) and (\ref{eq:B000}) we have that for any $0 \leq \eps \leq 1$, we get
%\begin{equation}
%C_{000} = C_{111} = {\frac {\ell^3} {48}} - {\frac {\ell^2} 8} + {\frac \ell 6}.
%\label{eq:C000}
%\end{equation}
%Let's try to find a value $\eps \in [0,1]$ that would give $C_{010}={\frac {\ell^3}{48}} - \ell^2.$
%Recalling (\ref{eq:C}), (\ref{eq:A010}) and (\ref{eq:B010}), we see that this requires taking $\eps$ to satisfy
%\[
%(1-\eps) \cdot
%\left({\frac {\ell^3}{48}} - {\frac {a^2 \ell}{12}} \right)
%+ \eps \cdot {\frac {\ell^3}{64}} =
%{\frac {\ell^3}{48}} - \ell^2,
%\]
%which holds for
%\begin{equation}
%\eps := {\frac {{\frac {a^2\ell}{12}} - \ell^2}{{\frac {\ell^3}{64}} - {\frac {\ell^3}{48}} + {\frac {a^2 \ell}{12}}}} = {\frac {\ell - {\frac {a^2}{12}}}{{\frac {\ell^2}{192}} + {\frac {a^2}{12}}}}, \label{eq:eps}
%\end{equation}
%which is in $[0,1]$ if $a \leq \sqrt{\ell/12}$ (which we can certainly have for all the $a$'s we consider). So as desired we have $C_{010}={\frac {\ell^3}{48}} - \ell^2.$ 
%Since both $A$ and $B$ are complement-symmetric we have that $C_{101}=C_{010}$. 
%
%Finally, since we have $B_{001}=B_{100}=B_{110}=B_{011}$ and $A_{001}=A_{100}=A_{110}=A_{011}$, we have that $C_{001}=C_{100}=C_{110}=C_{011}$, and this value is
%\[
%C_{001}=C_{100}=C_{110}=C_{011} = 
%{\frac {{\ell \choose 3} - 2 \cdot C_{000} - 2 \cdot C_{010}}{4}},
%\]
%which has no dependence on $a$.
%
%To close, we observe that by (\ref{eq:eps}), our value of $\eps$ is at most $192/\ell$ and hence is $o(1)$, as required by \Cref{claim:mixture}.
%
%}

% %%%%%%%%%%%%%%%%
% END IGNORE
% %%%%%%%%%%%%%%%%

}

