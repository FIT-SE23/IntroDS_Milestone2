%!TEX root = onetrace.tex

\section{Worst-case one-trace reconstruction, medium and small deletion rate} \label{sec:worst-case-small-delta}

%\rnote{\href{https://www.dropbox.com/home/FOCS19-RANDOM19-SODA21-ITCS21-SODA22-Trace-Reconstruction-Combinatorial/Approximate-trace-reconstruction-notes/weak-approximate-trace-reconstruction?preview=constant-delta.pdf}{Here} and \href{https://www.dropbox.com/home/FOCS19-RANDOM19-SODA21-ITCS21-SODA22-Trace-Reconstruction-Combinatorial/Approximate-trace-reconstruction-notes/weak-approximate-trace-reconstruction?preview=Intermediate+and+low+deletion+rates.pdf}{here} are some relevant files}
In this section we consider the medium and small deletion rate regime. In particular, throughout this section we suppose that $\delta \le 1/2$. (Note that if $\delta > 1/2$, then the quantity $1-\delta + \delta^2/2 -\delta^3/2 + \delta^4/2 - \delta^5/2$ is less than $2/3$, so the performance guarantee given by \Cref{thm:worst-case-small-delta} is weaker than the guarantee given by \Cref{thm:worst-case-small-rho-informal} / \Cref{thm:worst-case-small-rho}.)


\subsection{An efficient algorithm achieving expected LCS $(1-\delta + \delta^2/2 -\delta^3/2 + \delta^4/2 - \delta^5/2 - o(1))n$}

%\cnote{should we drop the expression in the subsection title because it is getting too long?}\rnote{I kind of like it}
As mentioned in the introduction, it is very easy for a one-trace algorithm to achieve expected LCS at least $(1-\delta)n$: this can be accomplished simply by having the hypothesis string $\wh{x}$ be any string that contains the input trace $\by$ as a subsequence.  The expected LCS of such a hypothesis string is clearly at least $\E[|\by|]$, which is $(1-\delta)n$ by linearity of expectation.

The following theorem shows how to improve on this naive bound:

\begin{theorem} [Worst-case algorithm, small deletion rate]\label{thm:worst-case-small-delta}
Let $\delta=\delta(n) \leq 1/2$ be the deletion rate. There is an $O(n)$-time (randomized) algorithm \textsc{Small-rate-reconstruct} which is given as input the values $n$ and $\delta$ and a single trace $\by \sim \Del_\delta(x)$, where $x \in \zo^n$ is an unknown and arbitrary source string.
For any $\gamma \leq 1$, \textsc{Small-rate-reconstruct} outputs a hypothesis string $\wh{\bx} \in \zo^n$ which satisfies
\[
\E\big[|\LCS(\wh{\bx},x)|\big] \geq 
    \Bigl( 1 - e^{-\Omega(\gamma^2 n)} \Bigr) \left( 1 - \delta + \frac{\delta^2}{2} - \frac{\delta^3}{2} + \frac{\delta^4}{2} - \frac{\delta^5}{2} \right)n - 3 \gamma n
\]
(so in particular, taking $\omega(1/\sqrt{n})\leq \gamma \leq o(1)$, we get that the expected value of $|\LCS(\wh{\bx},x)|$ is at least $(1 - \delta + \delta^2/2 - \delta^3/2 + \delta^4/2 - \delta^5/2 - o(1))n$).
%In particular, taking $\omega(1/\sqrt{n})\leq \gamma \leq o(1)$, we get that the expected value of $|\LCS(\wh{\bx},x)|$ is at least $(1 - \delta + \delta^2 / 2 - o(1))n$.
\end{theorem}

\begin{algorithm}[t]
\caption{\textsc{Small-rate-reconstruct}} 
\begin{algorithmic}[1] \label{alg:small-rate-reconstruct}
\State Set $j=1$ and $p_y = 1.$
\State  While $p_y \leq |\by|$ do:\\
~~~~~With probability $1-\delta$ set $\widehat{\bx}_j := \by_{p_y}$ and increment $p_y$; \\ ~~~~~with the remaining $\delta$ probability set $\widehat{\bx}_j$ to a uniform bit from $\zo$.\\
~~~~~Set $j := j+1.$
\State If $|\wh{\bx}|<n$ then append $0^{n-|\wh{\bx}|}$ to $\wh{\bx}$, and if $|\wh{\bx}|>n$ then delete bits $\wh{\bx}_{n+1},\dots$ from $\wh{\bx}.$
\State Output the $n$-bit string $\wh{\bx}.$
\end{algorithmic}
\end{algorithm} 

\noindent {\bf Intuition.}
The algorithm \textsc{Small-rate-reconstruct} is given as \Cref{alg:small-rate-reconstruct}.
To analyze the algorithm it is convenient to consider the string $\wh{\bx}'$ which is $\wh{\bx}$ immediately before Step~5 is performed (i.e. with no padding with 0s or deletion applied).  We will show later that $\wh{\bx}'$ is with high probability ``very close to $\wh{\bx}$'', so we can chiefly reason about $\wh{\bx}'$ and take care of the minor difference between $\wh{\bx}'$ and $\wh{\bx}$ at the end of the argument.

We first observe that $\wh{\bx}'$ clearly contains $\by$ as a subsequence.
The main idea of the proof is that a non-negligible fraction of the times that Step~3 is performed, one or more uniform random bits from $\zo$ will be placed in between consecutive bits $\by_{p_y}$ and $\by_{p_y+1}$ in creating the hypothesis string $\wh{\bx}'$ exactly when one or more bits of $x$ were deleted in between $\by_{p_y}$ and $\by_{p_y+1}$ in the creation of the trace $\by$. Each time this happens there is a 1/2 chance that at least one ``additional bit'' beyond the subsequence $\by$ of $\wh{\bx}'$ can be incorporated into a matching between $x$ and $\wh{\bx}'$.
This is the source of the extra $(\delta^2/2 - \delta^3/2 + \delta^4/2 - \delta^5/2)n$ in the LCS bound. {Intuitively, the number of additional bits between every $\by_{p_y}$ and $\by_{p_y + 1}$ in each of $x$ and $\wh{\bx}'$ is distributed according to $\Geo(1-\delta) - 1$, so there are at least $\min\{\Geo(1-\delta)-1, \Geo(1-\delta)-1\} = \Geo(1-\delta^2)-1$ many additional bits between $\by_{p_y}$ and $\by_{p_y+1}$ in \emph{both} of $x$ and $\wh{\bx}'$, and there is a 1/2 chance each uniform additional $\zo$ bit in $\wh{\bx}'$ matches an additional bit in $x$}.
  %this quantity arises because \blue{a $1-\delta$ fraction of the time,} there is a $\delta$ chance that the next bit of $x$ is deleted in forming $\by$; a $\delta$ chance that a uniform $\zo$ bit is available for the hypothesis string $\wh{\bx}$ at that point; and a 1/2 chance that the uniform $\zo$ bit matches the deleted bit of $x$).

We now provide formal details. 
\begin{proof}[Proof of \Cref{thm:worst-case-small-delta}]
Let $x\in \{0,1\}^n$ be the unknown source string.
  Consider appending infinitely many copies of a special symbol $\ast$ to $x$ to form an infinite string $x_\infty$.
  We sample an infinite subsequence $\by_\infty$ of $x_\infty$ by the following infinite process:
  For each $j=1,2,\dots$, we sample a prefix $\bx^j$ of $x_\infty$ of length $\abs{\bx^j} \sim \Geo(1-\delta)$, then output the last bit of $\bx^j$ as the $j$-th bit of $\by_\infty$ and delete the prefix $\bx^j$ from $x_\infty$ before moving on to the next value of $j$.

  Note that the longest prefix of $\by_\infty$ that does not contain any $x_i : i > n$ is identically distributed as the trace $\by \sim \Del_\delta(x)$.
  Equivalently, the concatenation of the last bit in each of $\bx^1, \ldots, \bx^{\abs{\by}}$ is identically distributed as $\by \sim \Del_\delta(x)$.


  Let $\bT=\{t_1 < \cdots < t_{|\by|}\}$ be the set of $|\by|$ many locations $j \in \{1,2,\dots,\}$ such that $\wh{\bx}'_j$ was set to $\by_{p_y}$ in some execution of Step~3 of \textsc{Small-rate-reconstruct}.
  Note that the elements of $\bT$ are the indices of the $\bT$race bits in $\wh{\bx}'$ and that $t_i$ is the index such that $\wh{\bx}'_{t_i}$ was set to $\by_{i}$ in Step~3 of the execution of \textsc{Small-rate-reconstruct}.
  Let $$\hat{\bx}'^1 := \hat{\bx}'_{[1:t_1]}, \hat{\bx}'^2 := \hat{\bx}'_{[t_1+1:t_2]}, \ldots, \wh{\bx}'^{\abs{\by}} := \wh{\bx}'_{[t_{\abs{\by}-1}+1:t_{\abs{\by}}]}.$$
  Observe that for each $i \in [\abs{\by}]$, both $\bx^i$ and $\wh{\bx}'^i$ are identically distributed.
  In particular, their lengths $\abs{\bx^i}$ and $\abs{\wh{\bx}'^i}$ are distributed according to $\Geo(1-\delta)$, and so the minimum of both lengths, i.e.  $\bd_i := \min\{\abs{\bx^i}, \abs{\wh{\bx}'^i}\}$, is distributed according to $\Geo(1-\delta^2)$.
  Moreover, the last bits in $\bx^i$ and $\wh{\bx}'^i$ are equal to $\by_i$, and the rest of the bits in $\wh{\bx}^i$ are uniform.
  
  For each $i \in [\abs{\by}]$, since the length-$(\bd_i-1)$ prefix of $\wh{\bx}'^i$ is random, in expectation (over the randomness of $\wh{\bx}'^i$) it agrees with the length-$(\bd_i-1)$ prefix of $\bx^i$ on $(\bd_i-1)/2$ of the bits.
  Also, the last bit of both $\bx^i$ and $\wh{\bx}'^i$ are the same.
  Hence, we have
  \[
    \E_{\wh{\bx}'^i} \Bigl[ \abs{\LCS(\bx^i, \wh{\bx}^i)} \Bigr] \ge (\bd_i-1)/2 + 1.
  \]
  Observe that the concatenation of $\bx^i : i\in [\abs{\by}]$ is a prefix of $x$, because the last bit of $\bx^{\abs{\by}}$ is the last bit of $\by \sim \Del_\delta(x)$, and the concatenation of $\wh{\bx}'^i : i \in [\abs{\by}]$ is exactly $\wh{\bx}'$.
  Thus,
  \[
    \E_{\wh{\bx}'} \Bigl[ \abs{\LCS(x, \wh{\bx}')} \Bigr]
    \ge \sum_{i=1}^{\abs{\by}} \Bigl[ \E_{\bx^i, \wh{\bx}'^i} \abs{\LCS(\bx^i,\wh{\bx}'^i)} \Bigr] \\
    \ge \sum_{i=1}^{\abs{\by}} \Bigl( (\bd_i-1)/2 + 1 \Bigr)
    = \abs{\by} + \sum_{i=1}^{\abs{\by}} (\bd_i-1)/2 .
  \]
  Since $\bd_i \sim \Geo(1-\delta^2)$, we have $\E[\bd_i-1] = \frac{1}{1 - \delta^2} - 1 = \frac{\delta^2}{1 - \delta^2}$.
  So taking expectation over $\abs{\by}$, we obtain 
  \begin{align*}
    \E \Bigl[ \abs{\LCS(x, \wh{\bx}')} \Bigr]
    &\ge \E\bigl[ \abs{\by} \bigr] + \E \bigl[ \abs{\by} \bigr] \frac{\delta^2}{2(1-\delta^2)} \\
    &= (1-\delta) n \cdot \left(1 + \frac{\delta^2}{2 (1-\delta^2)} \right) \\
    &= \left( 1 - \delta + \frac{\delta^2}{2(1 - \delta^2)} - \frac{\delta^3}{2(1-\delta^2)} \right) n \\
    &\ge \left( 1 - \delta + \frac{\delta^2}{2} - \frac{\delta^3}{2} + \frac{\delta^4}{2} - \frac{\delta^5}{2} \right) n .
  \end{align*}
  To finish the proof, we relate $\E[\abs{\LCS(\wh{\bx}, x)}]$ to $\E[\abs{\LCS(\wh{\bx}', x)}]$.
  We observe that
  \[
  |\LCS(\wh{\bx},x)| \geq \max\{0, |\LCS(\wh{\bx}',x)| - \bk\},
  \]
  where $\bk$ is the number of bits $\wh{\bx}_{n+1},\dots$ deleted from $\wh{\bx}$ in Step~5 of \textsc{Small-rate-reconstruct} if bits are deleted in that step (and $\bk=0$ otherwise).  So it remains only to show that with high probability $\bk$ is small.  
  
  Recall that the value of $|\by|$ is distributed as $\Bin(n,1-\delta)$, and given a particular outcome of the value of $|\by|$, the number of bits deleted in Step~5 is distributed as $\min\{0,\bG_1 + \cdots + \bG_{|\by|}-n\}$ where the $\bG_i$'s are independent geometric random variables with each $\bG_i \sim \Geo(\rho)$ (recall that $\rho=1-\delta$).
  We will use two tail bounds: first, by a multiplicative
  Chernoff bound, we have 
  
  \begin{claim} \label{claim:by-not-too-long}
  %For $0<\gamma < 1/4$,
  %$\Pr[|\by| \notin [(1-\delta - \gamma)n,(1-\delta + \gamma)n]] \leq e^{-\Omega(\gamma^2 n)}.$
  For $\gamma \leq 1$, we have
  $\Pr[|\by| \ge (1+\gamma) (1-\delta) n] \leq e^{-\Omega(\gamma^2 n)}.$
  \end{claim}
  
  %\begin{claim} \label{claim:by-not-too-long}
  %For $0<\gamma < 1/4$,
  %$\Pr[|\by| \notin [(1-\delta - \gamma)n,(1-\delta + \gamma)n]] \leq e^{-\Omega(\gamma^2 n)}.$
  %\end{claim}
  
  The second tail bound shows that $\bG_1 + \cdots + \bG_{|\by|}$ is unlikely to be much larger than $n$:
  
  
  \begin{claim} \label{claim:geom-not-too-big}
    Fix an outcome of $|\by|$ such that $|\by| \le (1+\gamma)(1-\delta)n$,
    where $\gamma \leq 1.$
    %\in [ (1-\gamma)(1-\delta)n, (1+\gamma)(1-\delta)n ]$.
  %Then for $0 < \gamma < 1/4$,
  Then $\Pr[\bG_1 + \cdots + \bG_{|\by|} \geq (1+3 \gamma)n] \leq e^{-\Omega(\gamma^2 n)}.$
  \end{claim}
  
  \begin{proof}
    Recall that \Cref{claim:neg-binomial} upper bounds the probability that $\bG_1 + \cdots + \bG_{(1+\gamma)(1-\delta)n} \geq \frac{1+\gamma}{1-\delta} \cdot (1+\gamma) (1 - \delta) n = (1+\gamma)^2 n$.
  %\cnote{Things might simplify slightly if we use a multiplicative instead of additive Chernoff bound instead.  I think the right hand side will become $\frac{1+\gamma}{1-\delta} (1+\gamma)(1-\delta) n = (1+\gamma)^2 n$.}
  %Since $1-\delta > 1/2$, we have
  %${\frac 1 {1-\delta}} \cdot (1 - \delta + \gamma)n \leq n + 2 \gamma n$, and since $\gamma < 1$ we have ${\frac \gamma {1-\delta}} \cdot (1 - \delta + \gamma)n \leq 3 \gamma n$.
  As $\gamma \le 1$, we get that
  \begin{align*}
  \Pr[\bG_1 + \cdots + \bG_{|\by|} \geq (1+3 \gamma)n] 
  %& \leq
  %\Pr[\bG_1 + \cdots + \bG_{(1 - \delta + \gamma)n}  \geq {\frac 1 {1-\delta}} \cdot (1 - \delta + \gamma)n + {\frac \gamma {1-\delta}} \cdot (1 - \delta + \gamma)n]\\
  %\Pr[\bG_1 + \cdots + \bG_{(1 - \delta + \gamma)n}  \geq (1+\gamma)^2 n] \\
  \le e^{-\Omega(\gamma^2 |\by|)}
  = e^{-\Omega(\gamma^2 n)},
  \end{align*}
  where the final inequality is by \Cref{claim:neg-binomial}.
  \end{proof}
  
  Combining \Cref{claim:by-not-too-long} and \Cref{claim:geom-not-too-big}, we get that $\bk \leq 3 \gamma n$ except with probability $e^{-\Omega(\gamma^2 n)}$. Consequently, %for $0 < \gamma < 1/4$,
we have that
  \begin{align*}
    \E[|\LCS(\wh{\bx},x)]
    &\ge \Bigl( 1 - e^{-\Omega(\gamma^2 n)} \Bigr) \Bigl( \E[|\LCS(\wh{\bx}',x)] - 3\gamma n \Bigr) \\
    &\geq \Bigl( 1 - e^{-\Omega(\gamma^2 n)} \Bigr) \left( 1 - \delta + \frac{\delta^2}{2} - \frac{\delta^3}{2} + \frac{\delta^4}{2} - \frac{\delta^5}{2} \right)n - 3 \gamma n,
  \end{align*}
  and the theorem is proved.
\end{proof}

\medskip

\ignore{
%\red{
%
%\noindent
%\emph{Proof of~\Cref{thm:worst-case-small-delta}}.
%The following notation will be helpful: let $\bR = \{r_1 < \cdots < r_{|\by|}\}$ be the locations in $[n]$ that were $\bR$etained in obtaining the trace $\by$ from $x$, so $x_{r_i}$ is the $i$-th bit $\by_i$ of $\by$.
%Let $\bT=\{t_1 < \cdots < t_{|\by|}\}$ be the set of $|\by|$ many locations $j \in \{1,2,\dots,\}$ such that $\wh{\bx}'_j$ was set to $\by_{p_y}$ in some execution of Step~3 of \textsc{Small-rate-reconstruct}.
%Note that the elements of $\bT$ are the indices of the $\bT$race bits in $\wh{\bx}'$ and that $t_i$ is the index such that $\wh{\bx}'_{t_i}$ was set to $\by_{i}$ in Step~3 of the execution of \textsc{Small-rate-reconstruct}.
%
%
%
%We describe a deterministic process that builds a matching $\bM$ between $x$ and $\wh{\bx}'$ (note that $\bM$ is a random variable because of the randomness in $\wh{\bx}$).
%The process maintains a pointer $p_x$ into $x$ and a pointer $p_{\wh{\bx}'}$ into $\wh{\bx}'$. The process runs for $n$ stages; at the beginning of the first stage $p_x=1$ (pointing to the first coordinate of $x$) and also $p_{\wh{\bx}'}=1$ (pointing to the first coordinate of $\wh{\bx}'$).
%In the $j$-th stage of the process:
%
%\begin{enumerate}
%
%\item If $p_x = r_i$ for some $i \in [|\by|]$, then the pointer $p_{\wh{\bx}'}$ is advanced to the value $t_i$ (or kept where it is if it is at $t_i$ already), i.e.~the location in $\wh{\bx}'$ where $\by_i$ was placed by \textsc{Small-rate-reconstruct}, and the edge $(p_x,p_{\wh{\bx}'})$ is added to the matching $\bM$ (note that the bits $x_{p_x}$ and $\wh{\bx}'_{p_{\wh{\bx}'}}$ are indeed the same). After adding this edge to $\bM$, both pointers $p_x$ and $p_{\wh{\bx}'}$ are incremented by 1.
%
%\item If $p_x \in [n] \setminus \bR$ (equivalently, $p_x$ is not $r_i$ for any $i \in [|\by|]$) and $p_{\wh{\bx}'} \notin \bT$ (equivalently, a uniform $\zo$ bit was used for $\wh{\bx}_{p_{\wh{\bx}'}}$), then if the bits $x_{p_x}$ and $\wh{\bx}'_{p_{\wh{\bx}'}}$ are the same, the edge $(p_x,p_{\wh{\bx}'})$ is added to the matching $M$. The pointer $p_x$ is always incremented by 1 at the end of this step (whether or not $p_x \in [n] \setminus \bR$ and whether or not $p_{\wh{\bx}'} \notin \bT$), \blue{and the pointer $p_{\wh{\bx}'}$ is incremented by 1 if and only if it was the case that $p_x \in [n] \setminus \bR$ and and $p_{\wh{\bx}'} \notin \bT$.}
%
%
%\end{enumerate}
%
%We observe that at the start of the $j$-th stage the pointer $p_x$ is equal to $j$; this is clear since at the beginning of the first stage $p_x=1$, and $p_x$ is incremented by 1 at the end of each stage. 
%We further observe that at the start of the $j$-th stage, if $p_x = r_i$ for some $i \in [|\by|]$, then the pointer $p_{\wh{\bx}'}$ is at most $t_i$; this is because $p_{\wh{\bx}'}$ is only advanced past $t_i$ once $p_x$ reaches $r_i$ (and when this happens $p_x$ itself is also advanced by 1).
%Thus it is indeed the case that the first sentence of case (1) can only ever advance the pointer $p_{\wh{\bx}'}$ or keep it where it is (it will never move $p_{\wh{\bx}'}$ backwards), and so $\bM$ is indeed a legitimate matching.
%
%It is clear that the matching $\bM$ will contain $|\by|$ pairs resulting from the $|\by|$ occasions on which case (1) holds (recall that in this case $p_x=r_i$ for some $i \in |\by|$).
%What other edges will $\bM$ contain? 
%To analyze this, we consider the \blue{$|\by|$ stages of $p_x$ for which $p_{x}-1 \in \bR$, i.e.~$x_{p_x-1}$ was retained as a bit in the trace $\by.$  At each such stage $p_x$, by virtue of the process by which $\bR$ is generated there is a $\delta$ probability that $p_x \in [n]\setminus \bR$, and when this occurs,} by the independent randomness in each execution of Step~3 of the algorithm there is an independent $\delta/2$ chance that both $p_{\wh{\bx}'} \notin \bT$ (equivalently, a uniform $\zo$ bit was chosen for $\wh{\bx}_{p_{\wh{\bx}'}}$ in the relevant execution of Step~3) and the chosen $\zo$ bit matches $x_{p_x}$.  So the \blue{distribution of the number of case (2) matching edges in $\bM$ dominates the distribution $\Bin(|\by|,\delta^2/2)$, and the overall distribution of $|\bM|$ dominates the distribution
%\[
%|\by| + \Bin(|\by|,\delta^2/2)),\quad \quad \text{where~}|\by| \sim \Bin(n,1-\delta).
%\]
%It follows that the expected size of $\bM$ is at least
%\[
%\E[\bM] = (1-\delta)n + \E[\Bin((1-\delta) n, \delta^2/2)] = (1-\delta + \delta^2/2 - \delta^3 / 2)n,
%\]
%and the existence of this matching implies that $\E[|\LCS(\wh{\bx}',x)|] \geq (1-\delta + \delta^2 / 2 - \delta^3 / 2)n$.
%}
%
%}

}

\subsection{Bounds on the performance of one-trace algorithms in the low deletion rate regime}

As noted in the Introduction, it is natural to try to complement \Cref{thm:worst-case-small-delta} by proving an upper bound on the best expected LCS that can be achieved by any one-trace algorithm in the low deletion rate regime. An average-case bound is of course stronger than a worst-case bound of this sort; in \Cref{sec:average-case} we will show that even in the average-case setting, the best achievable LCS given a single trace is at most $(1 - \Theta(\delta))n$.




