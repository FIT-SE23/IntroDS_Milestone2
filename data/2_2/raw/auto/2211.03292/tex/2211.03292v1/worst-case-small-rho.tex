%!TEX root = onetrace.tex

\section{Worst-case one-trace reconstruction, small retention rate} \label{sec:worst-case-small-rho}

\subsection{An efficient algorithm}

%\red{
%I think we will fold the $\rho=o(1)$ result into (a more formal version of) \Cref{thm:worst-case-small-rho}
%, covering both the $\rho=o(1)$ and the $\rho=\theta(1)$ results simultaneously. But I feel like we should start this section with the $2/3$ result based on Chin's awesome cover as a standalone thing (without getting the improvement to $2/3 + c \rho$) since it is so simple and clean.
%
%\href{https://www.dropbox.com/home/FOCS19-RANDOM19-SODA21-ITCS21-SODA22-Trace-Reconstruction-Combinatorial/Approximate-trace-reconstruction-notes/weak-approximate-trace-reconstruction?preview=one-trace-worst-case-new.pdf}{Here} is a relevant file for the $2/3$ standalone result, and \href{https://www.dropbox.com/home/FOCS19-RANDOM19-SODA21-ITCS21-SODA22-Trace-Reconstruction-Combinatorial/Approximate-trace-reconstruction-notes/weak-approximate-trace-reconstruction?preview=constant-delta.pdf}{here} is a file for \Cref{thm:worst-case-small-rho} (the $2/3 + c \rho$ result). 
%}
We prove \Cref{thm:worst-case-small-rho-informal} in this subsection. 
We start with the first part of \Cref{thm:worst-case-small-rho-informal}, i.e., 
%The following theorem gives a more detailed version of  
%It shows that if 
  when the retention rate $\rho$ is large enough ($\omega(\log (n)/n)$) 
  that a nontrivial number of bits are expected to be present in a random trace, 
  then a simple computationally efficient one-trace algorithm can achieve an LCS significantly better than $n/2$.

\subsubsection{A useful structural result and a $(2/3 - o(1))$-LCS algorithm for $\rho = \omega(\log(n)/n)$}\label{sec:LCSalgo}

It is helpful for us to consider the following preliminary problem: we are not given any traces, and the goal is to output \emph{a list} of $m$-bit candidates such that the unknown source string $x \in \zo^n$ has large LCS with one of the candidate strings in our list.
This motivates the following definition:

\begin{definition}[LCS-cover] \label{def:lcs-cover}
Let $m$ and $n$ be two positive integer.
We say a set $S \subseteq \zo^m$ is an \emph{$h$-LCS cover} for strings of length $n$ if for every $x \in \zo^n$ we have 
\[
  \abs[\big]{\LCS(S, x)}
  := \max_{s \in S}\hspace{0.04cm} \abs[\big]{\LCS(s, x)}
  \ge h .
\]
\end{definition}

The following simple claim shows that when $m$ is within a factor of two of $n$, there is a (perhaps surprisingly) good LCS cover consisting of at most two strings:

\begin{claim} \label{claim:lcs-cover}
  For every $m \in [n/2, 2n]$, there exists a $( (n+m)/3)$-LCS-cover of size at most $2$.
\end{claim}

\begin{proof}
  We first consider the extreme settings of $m = 2n$ and $m=n/2$.
  When $m = 2n$, we~have $\abs{\LCS((01)^n, x)} = n$ for every $x \in \zo^n$, and thus $\{ (01)^n \}$ is an $n$-LCS cover (of size $1$).
  When $m = n/2$, every $x \in \zo^n$ either contains $n/2$ many 1s or this many 0s, and so either $\abs{\LCS(0^{n/2}, x)} \ge n/2$ or $\abs{\LCS(1^{n/2}, x)} \ge n/2$, and hence the set $\{0^{n/2}, 1^{n/2}\}$ is a $(n/2)$-LCS cover of size $2$.

We interpolate between these two cases to handle general $m$'s.
  Write $m = 2a + b$ and $n = a + 2b$ for some $a$ and $b$ (so $a = m - (n+m)/3$ and $b = n - (n+m)/3$).
  Consider
\begin{equation} \label{eq:LCS-cover}
    S := \bigl\{ (01)^a 0^b, (01)^a 1^b \bigr\} \subseteq \zo^m .
\end{equation}
  Given $x \in \zo^n$, we can write $x = x_1 \circ x_2$ where $x_1 \in \zo^a$ and $x_2 \in \zo^{2b}$, and we get that
  \[
    \abs[\big]{\LCS(S, x)}
    \ge \abs[\Big]{\LCS\bigl( (01)^a, x_1 \bigr)} + \abs[\Big]{\LCS\bigl( \{0^b, 1^b\}, x_2 \bigr)}
    \ge a + b = (n+m) / 3 . \qedhere
  \]
\end{proof}


We observe that taking $m=n$ in \Cref{eq:LCS-cover}, we have a $(2n/3)$-LCS cover consisting of the two strings $(01)^{n/3}0^{n/3}$ and $(01)^{n/3}1^{n/3}$.  This suggests a one-trace algorithm that returns an $n$-bit string $\wh{x}$ that achieves $\smash{\abs{\LCS(\wh{x}, x)} \ge (2/3 - o(1)) n}$ with high probability when $\rho=\omega(\log n/n)$:
to determine which one of the two $n$-bit strings $\smash{(01)^{n/3}0^{n/3}, (01)^{n/3}1^{n/3}}$ to output, it simply needs to determine (with high probability) from the trace $\by\sim\Del_\delta(x)$ whether the majority of the last $2n/3$ bits of the unknown $x$ is $0$ or $1$, which can be done (to accuracy $o(1)$) by simply taking the majority of the last $2\rho n/3$ bits of $y$.
A routine computation now gives the first sentence of \Cref{thm:worst-case-small-rho-informal}.

We further note that the simple $(2n/3)$-LCS cover given by $\{(01)^{n/3}0^{n/3}, (01)^{n/3}1^{n/3} \}$ is essentially  best possible among all covers of constant size; more precisely, 
  for any positive constant $\eps$, any $(2/3+\eps)n$-LCS cover 
  must have size $\Omega(\log n)$.
%for any constant $C$ that is independent of $n$, no set of $C$ many $n$-bit strings can be a $({\frac 2 3} + {\frac 1 C})n$-LCS cover for $n$ sufficiently large. 
This is a consequence of a recent result of Guruswami, Haeupler, and Shahrasbi \cite{GHS20}; we give the proof in \Cref{ap:bestLCScover}.



\subsubsection{A $(2/3+\Omega(\rho))$-LCS algorithm for $\rho=\omega(1/n^{1/3})$}

 Next we prove the second part of  \Cref{thm:worst-case-small-rho-informal}.
 It follows from the following theorem:

\begin{theorem} [Worst-case algorithm, small retention rate] 
\label{thm:worst-case-small-rho}
There 
exists an absolute constant $c > 0$ 
%exist absolute constants $c,c' > 0$ 
such that the following holds. % for any $\gamma \in (0,1)$.
Let the retention rate $\rho := \rho(n) = 1-\delta(n)$ such that $\rho=\omega(1/n^{1/3})$.
There is an $O(n)$-time algorithm $A$ which is given as input the values $n, \delta$, and a single trace $\by \sim$ $ \Del_\delta(x)$, where $x \in \zo^n$ is an unknown source string.
With probability at least $1 - e^{-\Omega( \rho^3 n)}$ over the randomness of $\by\sim \Del_\delta(x)$, $A$ outputs a hypothesis string $\wh{\bx} \in \zo^n$ satisfying
%With probability at least $1 - e^{-\Omega(\gamma^2 \rho n)}$ over the randomness of $\Del_\delta(x)$, $A$ outputs a hypothesis string $\wh{\bx} \in \zo^n$ satisfying
\[
  %\min_{x \in \zo^n}\E_{\by \sim \Del_{1-\rho}(x)} \bigl[ \abs{\LCS(\wh{\bx},x)} \bigr]
  \abs[\big]{\LCS(\wh{\bx},x)}
  \ge (2/3 + c \rho ) \cdot n .
  %\rnote{Shall we give some actual expression for the $o(1)$?}
\]
\end{theorem}

An easy computation using the high-probability bound provided by \Cref{thm:worst-case-small-rho} shows that if $\rho \geq \omega(1/n^{1/3})$, then we get that $L_{1,\worst}(\delta,n) \geq (2/3 + \Omega(\rho)) \cdot n$, giving the bound on expected LCS that is claimed in \Cref{thm:worst-case-small-rho-informal}.
%\blue{Need to update notation below}

The algorithm for \Cref{thm:worst-case-small-rho} improves on the $(2/3-o(1))n$-LCS algorithm
  described in \Cref{sec:LCSalgo}.
%We now explain how to improve on this to prove \Cref{thm:worst-case-small-rho}.
The high-level idea is to do better than the $(n+m)/3$ benchmark given by \Cref{claim:lcs-cover} on 
  the $(n/3)$-prefix $x^{(1)}$ of $x$.
For intuition, suppose we could find an $\wh{x}^{(1)} \in \zo^\ast$ such that $$\abs[\big]{\LCS(\wh{x}^{(1)}, x^{(1)})} \ge \frac{\abs{\wh{x}^{(1)}} + \abs{x^{(1)}}}{3} + \eps n.$$
Then we could potentially apply the approach of the one-trace algorithm from the previous subsection on the remaining bits of $x$, and outputs $\wh{x}$ that extends $\wh{x}^{(1)}$ to achieve an LCS of roughly %(ignoring the $o(n)$ above)
\[
  \frac{ \abs{\wh{x}^{(1)}} + \abs{x^{(1)}} }{3} + \eps n
  + \frac{(n - \abs{\wh{x}^{(1)}}) + (n - \abs{x^{(1)}})}{3}
  = \frac{2n}{3} + \eps n .
\]

We now discuss how to beat the $(n+m)/3$ benchmark in more detail.
Let $L=[\rho n/3]$ and $y_L$ be the $(\rho n/3)$-length prefix of the trace $y$.
We divide $y_L$ into blocks of size $2000$.
If a block~contains only 0s, then it is very likely (probability at least, say, 0.9) that there is a corresponding subword in $x$ of size about $2000/\rho$ that contains mostly 0s; such a subword has large LCS (say, at least $1999/\rho$) with the string $0^{2000/\rho}$.
So if most blocks contain only 0s or only 1s (Case 2 in the description of Algorithm $A$ given below), then by outputting an $\wh{x}^{(1)}$ which is a corresponding sequence of $0^{2000/\rho}$'s and $1^{2000/\rho}$'s, such an $\wh{x}^{(1)}$ will have an LCS with $x^{(1)}$ that is much larger than $(\abs{x^{(1)}} + \abs{\wh{x}^{(1)}})/3$.

On the other hand, if most blocks contain both a 0-bit and a 1-bit, then we know that the string $x^{(1)}$ must alternate between 0s and 1s at least $t := \Omega(\rho n)$ times. In this case (Case 1 in the algorithm description), we can use the shorter string $(01)^{n/3 - t}$ to achieve an LCS of size $n/3$ with $x^{(1)}$, which also gives us an $\Omega(\rho n)$ savings.
%\rnote{consider rephrasing if unclear}
%This  also beats our benchmark of $(\abs{x_L} + 2(\abs{x_L} - t))/3 = \abs{x_L} - 2t/3$.\rnote{I am not 100\% sure what we mean by this}

The rest of \Cref{sec:worst-case-small-rho} gives a formal proof of \Cref{thm:worst-case-small-rho}.

\subsubsection{The Algorithm $A$}
In this subsection we describe the algorithm $A$ to prove \Cref{thm:worst-case-small-rho}.
Let $\gamma:=\rho/720000$. We show that given a trace $\by\sim\Del_\delta(x)$ for any
  unknown $x\in \{0,1\}^n$, the algorithm $A$ returns $\wh{\bx}$ satisfying
\begin{equation}\label{eq:hehe111}
\big|{\LCS(x, \wh{\bx})}\big| \ge \frac{2n}{3} + \frac{\rho n}{90000} - 4 \gamma n
\end{equation}
with probability at least $1-e^{-\Omega(\gamma^2 \rho n)}$.
Setting $c=1/180000$ in \Cref{thm:worst-case-small-rho}
  finishes the proof.

Given a trace $y$ of $x \in \zo^n$,
  $A$ outputs $\wh{x}:=0^n$ if its input trace $y$ has $\abs{y} < (\rho - \gamma) n$.
We refer to this case as Case 0; 
  henceforth we will assume $\abs{y} \ge (\rho-\gamma) n$ below.

Let $L := [\rho n/3]$ 
  and $y_L$ be the first $\rho n /3$ bits of $y$.
Divide $y_L$ into $B := \rho n/6000$ many blocks $y_{L_1}, \ldots, y_{L_B}$ of length $2000$   each (so $L_i := \{2000(i-1) + 1, \ldots, 2000i\}$).
Algorithm $A$ identifies the $y_{L_i}$'s that contain only $0$s or only $1$s.
Specifically, let
\[
  B' := \bigl\{ i \in [B]: \text{$y_{L_i} = z_i^{2000}$ for some $z_i \in \zo$} \bigr\} .
\]
There are two cases:
\begin{flushleft}
\begin{enumerate}
\item[] \noindent{\bf Case 1: $\abs{B'} < 0.8B$.} (In this case, a significant number of blocks are ``not pure.'') 
Let
\begin{align} \label{eq:params-abc}
  c := \frac{n}{3} - \frac{\rho n}{60000},\quad a := \frac{\rho n}{45000}\quad\text{and}\quad b := \frac{n}{3} - \frac{\rho n}{90000} .
\end{align}
Let %$RR := \{ \abs{y} - 2 \rho b + 1, \ldots, \abs{y} \}$ and $y_{RR}$ be the last $2 \rho b$ bits of $y$, and let 
  $z \in \zo$ be the majority of the last $2\rho b$ bits of $y$.
$A$ outputs the $n$-bit  
  $\smash{\wh{x} := (01)^{c + a} z^{b}}$.
  % = (01)^{\frac{n}{3} + \frac{\rho n}{180000}} z^{\frac{n}{3} - \frac{\rho n}{90000}} .
%\]
\item[] \noindent{\bf Case 2: $\abs{B'} \ge 0.8 B$.}
(Most blocks are ``pure.'')
Let %$RR := \{ \abs{y} - 2\rho n/9 + 1, \ldots, \abs{y}\}$ and 
  %$y_{RR}$ be the 
  $z\in \{0,1\}$ be the majority of the 
  last $2\rho n/9$ bits of $y$. %, and let $z \in \zo$ be the majority of the bits in $y_{RR}$.
  $A$ outputs the following $n$-bit string
\[
  \wh{x} := \wh{x}^{(1)} \circ (01)^{2n/9}\circ  z^{2n/9},
\]
where $\wh{x}^{(1)}$ is the concatenation of $z_i^{2000/\rho}$ 
  for each $i\in [B]$ with $z_i$ being the bit such that $y_{L_i}=z_i^{2000}$ when $i\in B'$ and $z_i=0$ when $i\notin B'$ 
%for the first $0.8B$ many elements $i \in B'$
  (so $\wh{x}^{(1)}$ has length $n/3$).
\end{enumerate}
\end{flushleft}
\subsubsection{Analysis of Algorithm $A$}
Let $x\in \{0,1\}^n$ be the unknown source string. We start by describing an equivalent process of drawing $\by\sim\Del_\delta(x)$.
Let $x_\infty$ be the infinite string obtained from $x$ by padding infinitely many copies of a special symbol $\ast$ at the end.
Consider sampling an infinite subsequence $\by_\infty$ of $x_\infty$ by the following infinite process:
For each round $j=1,2,\dots$, we sample a prefix $\bx^j$ of $x_\infty$ of length $\abs{\bx^j} \sim \Geo(\rho)$, then output the last bit of $\bx^j$ as the $j$-th bit of $\by_\infty$ and delete the prefix $\bx^j$ from $x_\infty$ before moving on to the next value of $j$.
Finally, we set $\by$ to be the 
  longest prefix of $\by_\infty$ that does not contain any special symbol $\ast$.
It is easy to check that $\by$ drawn from this process is identically distributed to $\Del_\delta(x)$.
%Also note that $\abs{\LCS(x, \wh{x})} = \abs{\LCS(x_\infty, \wh{x})}$ for any $\wh{x} \in \zo^n$.

We introduce some notation for working with $\bx^j$ as a byproduct of the above random process of drawing $\by\sim\Del_\delta(x)$.
For a subset $S \subseteq \N$ (e.g., $L$ introduced in the description of the algorithm), we write $\bx^S$ to denote the concatenation of $\bx^j: j\in S$, where $\bx^j$ is the prefix drawn in the $j$-th round.
Note that the string $\bx^{[\abs{\by}]}$ does not necessarily contain the source string $x$ (it may not contain some of its last few bits) but the string $\bx^{[\abs{\by}+1]}$ always contains $x$ as a prefix.

%The following simple fact is easily verified:

%\begin{fact} \label{fact:alternate}
%  Suppose $x \in \zo^n$ has $t$ many disjoint $01$'s.
%  Then $\LCS(x, (01)^m) = n$ when $m \ge n - t$.
%\end{fact}


%\begin{claim} \label{claim:mono}
%  Let $\by \sim \Del_\delta(x)$ and let $S \subseteq [\abs{\by}]$.
%  Suppose there is a $z \in \zo$ such that $\by_i = z$ for all $i \in S$.
%  Then 
%  \[
%    \Pr\Bigl[ \text{$\bx^S$ contains less than $(1 - \gamma) \rho^{-1} \abs{S}$ many $z$'s} \Bigr]
%    \le e^{-\Omega(\gamma^2 \abs{S})} .
%  \]
%\end{claim}
%\begin{proof}
%  Observe that all occurrences of $\bar{z}$ in $\bx^S$ must be deleted, and conditioned on them being deleted, the number of %$z$'s in $\bx^S$ is distributed as a sum $\bG^{(\abs{S})}$ of $\abs{S}$ independent geometric random variables $\bG_i \sim %\Geo(\rho)$.
%  Thus by \Cref{claim:neg-binomial} we have
%  \[
%    \Pr\Bigl[ \text{$\bx^S$ contains less than $(1 - \gamma) \rho^{-1} \abs{S}$ many $z$'s} \Bigr]
%    \le \Pr\Bigl[ \bG^{(\abs{S})} \le (1 - \gamma) \rho^{-1} \abs{S} \Bigr]
%    \le e^{-\Omega(\gamma^2 \abs{S})} . %\qedhere
%  \]
%This finishes the proof of the claim.
%\end{proof}


%200 length
%(\rho n/600) many blocks
%0.2 / 200 * (\rho n/3) many blocks contain 01
%0.1 / 200 * (\rho n/3) many alternations
%(01)^{n/3 - \rho n/6000}
%
%x = 2n/3
%x' = n/3 + \rho n/3000
%
%(x + x')/3 = n/3 + \rho n/9000
%
%a = x' - (x + x')/3 = \rho n * (1/3000 - 1/9000) = \rho n/4500 .
%b = x - (x + x')/3 = n/3 - \rho n/9000
%
%c + a = n/3 - \rho n /6000 + \rho n/4500 = n/3 + \rho n/18000 
%
%
%2b = 2n/3 - \rho n/4500
%
%suffix = 2b = (1 - \rho / 6750) * 2 \rho n/3
%
%output: (01)^a 1^b
%LCS = n/3 + \rho n / 9000

%Let $x\in \{0,1\}^n$ be 
Let $\by\sim\Del_\delta(x)$ be a trace drawn using the process above, and let 
  $\wh{\bx}$ be the string returned by the algorithm $A$ when running on $\by$.
We say $A$ succeeds (on $\by$) if $\wh{\bx}$ satisfies \Cref{eq:hehe111}
  and $A$ fails otherwise.
It suffices to show that all three probabilities $\Pr_{\by\sim\Del_\delta(x)}[\by\ \text{in Case $0$}]$,
$$
\Pr_{\by\sim \Del_{\delta}(x)}\big[\by\ \text{in Case $1$ and $A$ fails}\big]\quad\text{and}\quad
\Pr_{\by\sim \Del_{\delta}(x)}\big[\by\ \text{in Case $2$ and $A$ fails}\big]
$$
are at most $e^{-\Omega(\gamma^2\rho n)}$. The upper bound for Case 0 follows 
  by the Chernoff bound (which is indeed $e^{-\Omega(\rho n)}$).
%Throughout the subsequent analysis we will condition on the event $\abs{\by} \ge (\rho - \gamma)n$, which holds with probability $1 - e^{-\Omega(\gamma^2 \rho n)}$ by the Chernoff bound and 
%  can be absorbed by the error probability of \Cref{thm:worst-case-small-rho}.
%For notational simplicity, in our analysis we will use $RR^\ast$ to denote the set $RR \cup \{\abs{\by} + 1\}$.
Below we analyze the two main cases of the algorithm separately.

\paragraph{Case 1: $\abs{B'} < 0.8 B $.}

Recall from the description of $A$ that we are in Case 1 if $\by\sim \Del_\delta(x)$ has length at least $(\rho-\gamma)n$ and 
  $|\bB'|<0.8 B$. 
Recall  
  from \Cref{eq:params-abc} our choices of $a,b$ and $c$, 
and let $\bz\in\zo$ be the majority of the last $2\rho b$ bits in $\by$.
We partition $x$ into $x^{(1)}\circ x^{(2)}\circ x^{(3)}$ with 
$$
\abs{x^{(1)}} = n/3,\quad \abs{x^{(2)}} = a \quad\text{and}\quad \abs{x^{(3)}} = 2b.
$$


Our goal is to show that 
$$
\Pr_{\by\sim \Del_{\delta}(x)}\bigl[\by\ \text{in Case $1$ and $A$ fails} \bigr]\le e^{-\Omega(\gamma^2\rho n)}.
$$
This follows from the following two claims:
Let $E_1$ denote the event of $|\bx^L|\ge n/3+\gamma n$ and $E_2$ denote the event of 
  $\bz$ appearing less than $b-\gamma n$ many times in $x^{(3)}$.

\begin{claim} \label{claim:events-case1}
For any string $x\in \{0,1\}^n$, we have 
$$
\Pr_{\by\sim \Del_{\delta}(x)}\bigl[\by\ \text{in Case $1$} \land ( E_1\lor E_2) \bigr]\le e^{-\Omega(\gamma^2\rho n)}. 
$$ 
%  With probability $1 - e^{-\Omega(\gamma^2 \rho n)}$, the following events all hold:
%  \begin{enumerate}
%    \item $\abs{\bx^L} \le \frac{n}{3} + \gamma n$
%    \item $\abs{\bx^{RR^\ast}} \le \frac{2n}{3} - \frac{\rho n}{45000} + \gamma n$,
%    \item $\bx^{RR}$ has at least $\frac{n}{3} - \frac{\rho n}{90000} - \gamma n$ many $z$'s.
%  \end{enumerate}
\end{claim}
\begin{claim}\label{claim:events-case12}
The algorithm $A$ succeeds whenever $\by\sim \Del_\delta(x)$ satisfies (1) $\by$ falls in Case 1; (2) $\overline{E_1}$: $|\bx^L|<n/3+\gamma n$; and 
  (3) $\overline{E_2}$: $\bz$ appears at least $b-\gamma n$ many times in $x^{(3)}$.
\end{claim}

\begin{proof}[Proof of \Cref{claim:events-case1}]
It follows from \Cref{claim:neg-binomial} that the probability of $E_1$ alone is at most $e^{-\Omega(\gamma^2\rho n)}$.
So it suffices to upper bound $\Pr_{\by} [ \by\ \text{in Case 1 and $E_2$}]$. 
Assume without loss of generality that $x^{(3)}$ has at least $b+\gamma n$ many $z$'s for some $z\in \{0,1\}$;
otherwise the probability above is trivially $0$.
By Chernoff bound we have 
\begin{align*}
\Pr_{\by\sim \Del_\delta(x)}\big[\text{\# of bits in $x^{(3)}$ that survive in $\by\ge 2\rho b+\rho\gamma n/3$}\big] &\le e^{-\Omega(\gamma^2 \rho n)}\quad\text{and}\\
\Pr_{\by\sim \Del_\delta(x)}\big[\text{\# of $z$'s in $x^{(3)}$ that survive in $\by\le \rho b+2\rho\gamma n/3$}\big] &\le e^{-\Omega(\gamma^2 \rho n)} .
\end{align*}
So with probability at least $1-e^{-\Omega(\gamma^2 \rho n)}$, the number of bits in $x^{(3)}$ that survive
  in $\by$ is at most $2\rho b+\rho\gamma n/3$ and among them at least $\rho b+2\rho\gamma n/3$ bits are $z$.
In this case it cannot happen that $\by$ falls in Case $1$ and $\bz\ne z$.
It follows that $\Pr_{\by} [ \by\ \text{in Case 1 and $E_2$}]\le e^{-\Omega(\gamma^2\rho n)}$.
%  Item (3) follows %from \Cref{claim:mono} 
%  because $\by_{RR}$ contains at least $\rho b = \rho(\frac{n}{3} - \frac{\rho n}{90000})$ many $z$'s.
\end{proof}

\begin{proof}[Proof of \Cref{claim:events-case12}]
When $\by\sim \Del_\delta(x)$ falls in Case 1, the string $\wh{\bx}$ returned by $A$ is 
%can be written as
%  $\wh{\bx} = \wh{\bx}^{(1)} \circ  \wh{\bx}^{(2)} \circ \wh{\bx}^{(3)}$, where
%Write $x = x_L \circ (x_{RL} \circ x_{RR})$ and $\wh{x} = \wh{x}_L \circ (\wh{x}_{RL} \circ \wh{x}_{RR})$, where
\begin{align*}
  %\abs{x_L} &:= n/3 &
\wh{\bx} := (01)^c\circ (01)^a\circ \bz^{b}.
\end{align*}
We lowerbound $|\LCS(x,\wh{\bx})|$ by
$$
\big|\LCS(x^{(1)},(01)^c)\big|+\big|\LCS(x^{(2)},(01)^a)\big|+\big|\LCS(x^{(3)},\bz^{b})\big|
$$
and below we bound each of the three terms separately. The following simple fact will be useful:

%\begin{claim} \label{claim:events-case1}
%  With probability $1 - e^{-\Omega(\gamma^2 \rho n)}$, the following events all hold:
%  \begin{enumerate}
%    \item $\abs{\bx^L} \le \frac{n}{3} + \gamma n$
%    \item $\abs{\bx^{RR^\ast}} \le \frac{2n}{3} - \frac{\rho n}{45000} + \gamma n$,
%    \item $\bx^{RR}$ contains at least $\frac{n}{3} - \frac{\rho n}{90000} - \gamma n$ many $z$'s.
%  \end{enumerate}
%\end{claim}

\begin{fact} \label{fact:alternate}
  Suppose $x \in \zo^n$ has $t$ many disjoint $01$'s.
  Then $\LCS(x, (01)^m) = n$ when $m \ge n - t$.
\end{fact}

%We now show that $\LCS(x, \wh{x}) \ge (2/3 + \Omega(\rho) - O(\gamma))n$ conditioned on the events in \Cref{claim:events-case1}.
We start with $|\LCS(x^{(1)},(01)^c)|$.
Because $\abs{\bB'} < 0.8B$, we have that at least $0.2B$ of the $\by_{L_i}$'s contain both $0$ and $1$, and thus there are at least $0.1B$ many disjoint $01$'s appearing in $\by_L$.
Given that $\abs{\bx^L} < n/3 + \gamma n$, the first $n/3 + \gamma n$ bits of $x$ contain at least $0.1B$ many disjoint $01$'s, and so the first $n/3$ bits of $x$ (i.e. $x^{(1)}$) contains at least $0.1B - \gamma n$ many disjoint $01$'s.
Using $c=n/3-0.1B$ and \Cref{fact:alternate}, we have
\[
  \abs[\big]{\LCS(x^{(1)}, (01)^c)}
  = \abs[\bigg]{\LCS\Bigl( x^{(1)}, (01)^{n/3 - 0.1B} \Bigr)}
  \ge \abs[\bigg]{\LCS\Bigl( x^{(1)}, (01)^{n/3 - \left(0.1B - \gamma n \right)} \Bigr) } - 2 \gamma n
  \ge \frac{n}{3} - 2 \gamma n .
\]
Next, given that $x^{(2)}$ only has length $a$ and $x^{(3)}$ 
  contains at least $b-\gamma n$ many $\bz$'s, %applying \Cref{fact:alternate} again (this time with $t = 0$), 
 trivially we have
\[
  \abs[\big]{\LCS(x^{(2)}, (01)^{a})}
  = a\quad\text{and}\quad \abs[\big]{\LCS(x^{(3)}, \bz^b)}\ge b-\gamma n.
\]
It follows that 
$$
\abs[\big]{\LCS(x , \wh{\bx})}\ge \frac{n}{3}+a+b-3\gamma n = \frac{2n}{3}+\frac{\rho n}{90000}-3\gamma n
$$
and $A$ succeeds. This finishes the proof of the claim.
%By \Cref{claim:events-case1}, as $\bx^{RR^\ast}$ contains at most the last $\frac{2n}{3} - \frac{\rho n}{45000} + \gamma n$ bits of $x$, and $\bx^{RR}$ contains at least $\frac{n}{3} - \frac{\rho n}{90000} - \gamma n$ many $z$'s, we have
%\begin{multline*}
%  \abs[\big]{\LCS(x_{RR}, \wh{x}_{RR})}
%  \ge \abs[\big]{\LCS(\bx^{RR^\ast}, \wh{x}_{RR})} - \gamma n 
%  \ge \abs[\big]{\LCS(\bx^{RR}, \wh{x}_{RR})} - \gamma n  \\ 
%  = \abs[\bigg]{\LCS\Bigl(\bx^{RR}, z^{\frac{n}{3} - \frac{\rho n}{90000}} \Bigr)} - \gamma n
%  \ge \frac{n}{3} - \frac{\rho n}{90000} - 2 \gamma n .
%\end{multline*}
%So altogether we have
%\begin{align} \label{eq:lcs-case1}
%  \abs[\big]{\LCS(x, \wh{x})}
%  \ge \abs[\big]{\LCS(x_L, \wh{x}_L)} + \abs[\big]{\LCS(x_{RL}, \wh{x}_{RL})} + \abs[\big]{\LCS(x_{RR}, \wh{x}_{RR}) }
%  \ge \frac{2n}{3} + \frac{\rho n}{90000} - 4 \gamma n.
%\end{align}
\end{proof}

\paragraph{Case 2: $\abs{B'} \ge 0.8B$.}
Recall that we are in Case 2 if $\by\sim \Del_\delta(x)$ has length at least $(\rho-\gamma)n$ and 
  $|\bB'|\ge 0.8B$.
For each $i\in [B]$ we set $\bz_i$ to be the bit such that $\by_{L_i}=\bz_i^{2000}$ if $i\in \bB'$
  and set $\bz_i=0$ if $i\notin \bB'$.
%In this case we write $\bB''$ to  denote the first $0.8B$ elements of $\bB'$ and 
%  write $\bz_i\in \{0,1\}$ to denote the bit that appears in the $i$-th block for each $i\in \bB''$. 
We also write $\bz$ to denote the majority of the last $2\rho n/9$ bits of $\by$.  
  
The proof proceeds in a similar fashion as Case 1. Let $x=x^{(1)}\circ x^{(2)}\circ x^{(3)}$ with
$$
\abs{x^{(1)}} = n/3,\quad 
\abs{x^{(2)}} = 2n/9 \quad\text{and}\quad
\abs{x^{(3)}} = 4n/9.
$$  
Our goal is to show that 
$$
\Pr_{\by\sim \Del_{\delta}(x)}\big[\by\ \text{in Case $2$ and $A$ fails}\big]\le e^{-\Omega(\gamma^2\rho n)}.
$$
Let $E_1$ denote the event of $|\bx^L|\ge n/3+\gamma n$, $E_2$ denote the event of $\bz$
  appearing less than $2n/9-\gamma n$ many times in $x^{(3)}$,
  and $E_3$ denote the following event:
\begin{enumerate}
\item[] $E_3$: For at least $0.02B$ of $i\in \bB'$, the subword $\bx^{L_i}$
  contains at most $0.9 \cdot 2000/\rho$ many $\bz_i$'s.
\end{enumerate} 
This follows from the following two claims:

\begin{claim} \label{claim:events-case2}
For any string $x\in \{0,1\}^n$, we have 
$$
\Pr_{\by\sim \Del_\delta(x)}\big[\by\ \text{in Case $1 \land ( E_1\lor E_2\lor E_3)$}\big]\le e^{-\Omega(\gamma^2\rho n)}. 
$$
\end{claim}
\begin{claim}\label{claim:events-case22}
The algorithm $A$ succeeds whenever $\by\sim \Del_\delta(x)$ satisfies (1) $\by$ falls in Case 2; (2) $\overline{E_1}$: $|\bx^L|<n/3+\gamma n$; (3) $\overline{E_2}$: $\bz$ appears at least $2n/9-\gamma n$ many times in $x^{(3)}$; and 
  (4) $\overline{E_3}$: At most $0.02B$ of $i\in \bB'$ has $\bx^{L_i}$ contain at most $0.9 \cdot 2000/\rho$ many $\bz_i$'s.
\end{claim}
\begin{proof}[Proof of \Cref{claim:events-case2}]
Events $E_1$ and $E_2$ can be handled similarly as in the proof of \Cref{claim:events-case1}.
Below we show that $\Pr_{\by}[ E_3 ] \le e^{-\Omega(\gamma^2\rho n)}$.
To this end, note that $E_3$ means there are at least $0.02B$ many $i\in [B]$
  such that $\by_{L_i}$ is all $\bz_i$ for some $\bz_i\in \{0,1\}$ while
  $\bx^{L_i}$ has at most $0.9\cdot 2000/\rho$ many $\bz_i$.
  
Let $\bZ_i$ be the indicator random variable for the event above for each $i\in [B]$.
We show below that conditioning on any outcomes of $\bx^1,\ldots,\bx^{2000(i-1)}$,
  the probability of $\bZ_i=1$ is at most $0.01$.
It follows that $E_2$ occurs with probability at most $e^{-\Omega(B)}=e^{-\Omega(\rho n)}$.

For each $i\in [B]$, after fixing any outcomes of $\bx^1,\ldots,\bx^{2000(i-1)}$,
  a necessary condition for $\bZ_i$ to be $1$ is that among the first $0.9\cdot 2000/\rho$ 
  many $0$'s in the current $x_{\infty}$, at least $2000$ of them survive in $\by_{\infty}$, or among the first $0.9\cdot 2000/\rho$
  many $1$'s in $x_{\infty}$, at least $2000$ of them survive in $\by_{\infty}$.
  The probability of $\bZ_i=1$ can be bounded from above by $0.01$ using the Chernoff bound.
  \end{proof}

\begin{proof}[Proof of \Cref{claim:events-case22}]  
When $\by\sim \Del_\delta(x)$ falls in Case 2, the algorithm $A$ returns $$\wh{\bx}=\wh{\bx}^{(1)}\circ (01)^{2n/9}\circ
  \bz^{2n/9} ,$$ where
  $\wh{\bx}^{(1)}$ is the concatenation of $\bz_i^{2000/\rho}$, $i\in [B]$.
We lowerbound $|\LCS(x,\wh{\bx})|$ by
\begin{align*}
&\big|\LCS(x^{(1)},\wh{\bx}^{(1)})\big|+ \big|\LCS(x^{(2)},(01)^{2n/9})\big|+\big|\LCS(x^{(3)},\bz^{2n/9})\big|\\
&\hspace{1.5cm}\ge \big|\LCS(x^{(1)},\wh{\bx}^{(1)})\big| + 2n/9 + 2n/9-\gamma n.
\end{align*}
To bound $|\LCS(x^{(1)},\wh{\bx}^{(1)})|$, we write $\bB''$ to denote the set of $i\in \bB'$
  such that $\bx^{L_i}$ contains at least $0.9 \cdot 2000/\rho$ many $\bz_i$'s.
  It follows from Item (4) in \Cref{claim:events-case22} that $|\bB''|\ge 0.98 \cdot 0.8 B \ge 0.78 B$.
We have 
%and below we bound each of the three terms separately. The following simple fact will be useful:
%We write $x = x_L \circ x_{RL} \circ x_{RR}$, and $\wh{x} = \wh{x}_L \circ \wh{x}_{RL} \circ \wh{x}_{RR}$, where $\wh{x}_L$ is the concatenation of the $z_i^{2000/\rho}$ for those $i \in B' \subseteq [B]$ as defined in the algorithm, and 
%\begin{align*}
%  \abs{x_L} &= n/3 \\
%  \abs{x_{RL}} &= 2n/9 & 
%$$ 
%   \wh{\bx}_{RL} &= (01)^{2n/9} \\
%  \abs{x_{RR}} &= 4n/9 & 
%  \wh{x}_{RR} &= z^{2n/9} .
%\end{align*}
%As in Case 1,  the following events happen with high probability.
%\begin{claim} \label{claim:events-case2a}
%  With probability $1 - e^{-\Omega(\gamma^2 \rho n)}$, the following events all hold:
%  \begin{enumerate}
%    \item $\abs{\bx^L} \le \frac{n}{3} + \gamma n$
%    \item $\abs{\bx^{RR^\ast}} \le \frac{4n}{9} + \gamma n$ 
%    \item $\bx^{RR}$ contains at least $\frac{2n}{9} - \gamma n$ many $z$'s.
%  \end{enumerate}
%\end{claim}
%\begin{claim} \label{claim:events-case2b}
%  With probability $1 - e^{-\Omega(\rho n)}$, for at least $0.95\abs{B'}$ of the $i \in B'$, the subword $\bx^i$ contains at %least $0.95 \cdot 2000/\rho$ many $z_i$'s.
%\end{claim}
%\begin{proof}
%  Note that the $\bx^i$'s are independent.
%  For each $i \in B'$, %by \Cref{claim:mono} 
%  $\bx^i$ contains at least $0.95 \cdot 2000/\rho$ many $z_i$'s with probability at least $0.99$.
%  So by the Chernoff bound, with probability at least $1 - e^{-\Omega(B)} = 1 - e^{-\Omega(\rho n)}$ this holds for at least %$0.95 \abs{B'}$ of the $i \in B'$.
%\end{proof}
%We now show that $\abs{\LCS(x, \wh{x})} \ge (0.68 - 3\gamma)n$ conditioned on the events in \Cref{claim:events-case2a,claim:events-case2b}.
%Let $B''$ denote the (at least $0.95 |B'|$ many) elements of $B'$ given by \Cref{claim:events-case2b}.
%First, we have
\begin{align*}
  \abs[\big]{\LCS(x^{(1)}, \wh{\bx}^{(1)})}
  &\ge \abs[\big]{\LCS(\bx^L, \wh{\bx}^{(1)})} - \gamma n  \\[.8ex]
  &\ge \sum_{i \in \bB''} \abs[\big]{\LCS(\bx^{L_i}, \bz_i^{2000/\rho})} - \gamma n \\
  &\ge 0.78\cdot \frac{\rho n}{6000}\cdot 0.9 \cdot \frac{2000}{\rho}-\gamma n\\ & = 0.702\cdot \frac{n}{3}-\gamma n.
%  &\ge \sum_{i \in B''} \abs[\big]{\LCS(\bx^i, \wh{x}^i)} - \gamma n \\
%  &\ge \abs{B''} \cdot 0.95 \cdot \frac{2000}{\rho} - \gamma n\\
%  &\ge 0.95^2 \cdot \abs{B'} \cdot \frac{2000}{\rho} - \gamma n \\
%  &\ge 0.95^2 \cdot 0.8 \cdot \frac{\rho n}{6000} \cdot \frac{2000}{\rho} - \gamma n \\
%  &\ge 0.722 \cdot \frac{n}{3} - \gamma n .
\end{align*}
%Then just like in the previous case, we have
%\[
%  \abs[\big]{\LCS(x_{RL}, \wh{x}_{RL})}
%  = \abs[\bigg] {\LCS\Bigl( x_{RL}, (01)^{\abs{x_{RL}}} \Bigr) }
%  = 2n/9 .
%\]
%Similarly, since by \Cref{claim:events-case2a} $\bx^{RR^\ast}$ contains at most the last $4n/9 + \gamma n$ bits of $x$, and $%\bx^{RR}$ contains at least $2n/9 - \gamma n$ many $z$'s, we have
%\begin{align*}
%\abs[\big]{\LCS(x_{RR}, \wh{x}_{RR})}
%  &\ge \abs[\big]{\LCS(\bx^{RR^\ast}, \wh{x}_{RR})} - \gamma n  \\
%  &\ge \abs[\big]{\LCS(\bx^{RR}, \wh{x}_{RR}) } - \gamma n  \\
%  &= \abs[\bigg]{\LCS\Bigl(\bx^{RR}, z^{\frac{2n}{9}} \Bigr)} - \gamma n \\
%  &\ge \frac{2n}{9} - 2 \gamma n .
%\end{align*}
Therefore, altogether we have
\begin{align} \label{eq:lcs-case2}
  \abs[\big]{\LCS(x,\wh{\bx})} 
  %&\ge \abs[\big]{\LCS(x_L,\wh{x}_L)} + \abs[\big]{\LCS(x_{RL}, \wh{x}_{RL})} + \abs[\big]{\LCS(x_{RR}, \wh{x}_{RR})} \nonumber \\
   \ge \frac{0.702 \cdot 3 + 4}{9} \cdot n - 2 \gamma n \nonumber \ge (0.678 - 2 \gamma) n  
\end{align}
and $A$ succeeds. This finishes the proof of the claim.
\end{proof}

%Combining \Cref{eq:lcs-case1,eq:lcs-case2} gives us \Cref{eq:hehe111},
%$\abs{\LCS(x, \wh{x})} \ge \frac{2n}{3} + \frac{\rho n}{90000} - 4 \gamma n$, 
%proving the theorem. 
