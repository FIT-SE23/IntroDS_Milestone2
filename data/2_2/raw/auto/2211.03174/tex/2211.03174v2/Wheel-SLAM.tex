\documentclass[letterpaper, 10 pt, journal, twoside]{IEEEtran}

% make the title area
% As a general rule, do not put math, special symbols or citations
% in the abstract or keywords.

\usepackage{cite}
\usepackage{graphicx}
\usepackage{picinpar}
\usepackage{amsmath,bm}
\usepackage{url}
%\usepackage{flushend}
%\usepackage[latin1]{inputenc}
\usepackage{colortbl}
\usepackage{soul}
\usepackage{multirow}
\usepackage{pifont}
\usepackage{color}
\usepackage{alltt}
\usepackage[hidelinks]{hyperref}
\usepackage{enumerate}
\usepackage{siunitx}
\usepackage{breakurl}
\usepackage{epstopdf}
\usepackage{pbox}
\usepackage{indentfirst}
\usepackage{booktabs}
\usepackage{makecell}
\usepackage{subfigure}
\usepackage{float}
\usepackage{threeparttable}
\usepackage{algorithmic}
\usepackage{algorithm}
\usepackage{amsfonts}

\hyphenation{op-tical net-works semi-conduc-tor IEEE-Xplore}
\def\BibTeX{{\rm B\kern-.05em{\sc i\kern-.025em b}\kern-.08em
    T\kern-.1667em\lower.7ex\hbox{E}\kern-.125emX}}
\usepackage{balance}
\begin{document}
\title{Wheel-SLAM: Simultaneous Localization and Terrain Mapping Using One Wheel-mounted IMU}
\author{Yibin~Wu$^{1,2}$, ~Jian~Kuang$^{1}$, ~Xiaoji~Niu$^{1}$, ~Jens~Behley$^{2}$, ~Lasse~Klingbeil$^{2}$, and~Heiner~Kuhlmann$^{2}$% <-this % stops a space
\thanks{Manuscript received: September 6, 2022; Revised November 4, 2022; Accepted November 28, 2022. This paper was recommended for publication by Editor Javier Civera upon evaluation of the Associate Editor and Reviewers’ comments.}
\thanks{$^{1}$Yibin Wu, Jian Kuang, and Xiaoji Niu are with the GNSS Research Center, Wuhan University, Wuhan, China, \{ybwu, kuang, xjniu\}@whu.edu.cn.}
\thanks{ $^{2}$Yibin Wu, Lasse Klingbeil, Jens Behley, and Heiner Kuhlmann are with the Institute of Geodesy and Geoinformation, University of Bonn, Bonn, Germany, \{firstname.lastname\}@igg.uni-bonn.de.(Corresponding Author: \textit{Jian Kuang})}
\thanks{This work was funded in part by the National Key Research and Development Program of China (No. 2016YFB0501800 and No. 2016YFB0502202) and by the Deutsche Forschungsgemeinschaft (DFG, German Research Foundation) under Germany's Excellence Strategy-EXC 2070-390732324.}}% <-this % stops a space}

\markboth{IEEE Robotics and Automation Letters. Preprint Version. November, 2022}{Wu \MakeLowercase{\textit{et al.}}: Wheel-SLAM} 

\maketitle
\begin{abstract}
A reliable pose estimator robust to environmental disturbances is desirable for mobile robots. To this end, inertial measurement units (IMUs) play an important role because they can perceive the full motion state of the vehicle independently. However, it suffers from accumulative error due to inherent noise and bias instability, especially for low-cost sensors. In our previous studies on Wheel-INS \cite{niu2021, wu2021}, we proposed to limit the error drift of the pure inertial navigation system (INS) by mounting an IMU to the wheel of the robot to take advantage of rotation modulation. However, Wheel-INS still drifted over a long period of time due to the lack of external correction signals. In this letter, we propose to exploit the environmental perception ability of Wheel-INS to achieve simultaneous localization and mapping (SLAM) with only one IMU. To be specific, we use the road bank angles (mirrored by the robot roll angles estimated by Wheel-INS) as terrain features to enable the loop closure with a Rao-Blackwellized particle filter. The road bank angle is sampled and stored according to the robot position in the grid maps maintained by the particles. The weights of the particles are updated according to the difference between the currently estimated roll sequence and the terrain map. Field experiments suggest the feasibility of the idea to perform SLAM in Wheel-INS using the robot roll angle estimates. In addition, the positioning accuracy is improved significantly (more than 30\%) over Wheel-INS. The source code of our implementation is publicly available (https://github.com/i2Nav-WHU/Wheel-SLAM).
\end{abstract}

% Note that keywords are not normally used for peerreview papers.
%\begin{IEEEkeywords}
%Wheel-mounted IMU, multi-IMU, dead reckoning, vehicular navigation, wheeled robot.
%\end{IEEEkeywords}
\begin{IEEEkeywords}
SLAM, Localization
\end{IEEEkeywords}

\IEEEpeerreviewmaketitle

\section{Introduction}

\begin{figure}[t]
	\centering
	\includegraphics[width=8cm]{20200102trajwithexampleRoll_new.png}
	\caption{Vehicle roll estimation and test trajectory in the car experiment of our prior work \cite{niu2021}. In the marked area, the car kept circling back and forth. It can be observed that the corresponding robot roll angle estimation (indicating the road bank angle) shows a repeating pattern that can be exploited to perform loop closure detection and correction.}
	\label{examplerollfig}
\end{figure}

\IEEEPARstart{S}{tate} estimation is one of the most fundamental modules for autonomous vehicles. Therefore, modern ground mobile robots are commonly equipped with both exteroceptive sensors, e.g., Global Navigation Satellite System (GNSS) receiver, cameras, Light Detection And Ranging (LiDAR), and proprioceptive sensors, e.g., Inertial Measurement Unit (IMU) and odometer, to track the robot trajectories as well as perceive the environments. Among all these sensors, inertial sensors play a central role in robot navigation due to their self-contained characteristics, which means that it works independently without external signals and interaction with the environment \cite{liyou2021}. In most of the existing works, inertial sensors are used to perform Dead Reckoning (DR) to either complement other navigation techniques \cite{qin2018, huang2019, xu2022TRO} or bridge the signal blockage of other sensors \cite{wu2019sensors}. Although the advances in the microelectromechanical (MEMS) technique have made IMU ubiquitous in various devices, an Inertial Navigation System (INS) suffers from the curse of error drift due to the inherent sensor noise and bias instability. Therefore, it is both challenging and promising to improve the positioning performance of the stand-alone INS.

Inspired by the odometer-aided INS (ODO/INS) \cite{ouyang2020}, we proposed Wheel-INS, a wheel-mounted IMU (Wheel-IMU)-based DR system, in our previous studies \cite{niu2021, wu2021}. Wheel-INS achieved competitive pose estimation performance with only one IMU. It was illustrated that the positioning and heading accuracy of Wheel-INS had been improved by 23\% and 15\% respectively over ODO/INS. Moreover, benefiting from the rotation modulation, Wheel-INS showed desirable resilience to the constant gyro bias error \cite{niu2021}. In summary, there are two major advantages of Wheel-INS. First, a similar information fusion scheme as ODO/INS is achieved by only one inertial sensor. Second, the continuous rotation of the Wheel-IMU significantly limits the heading error drift.  
  
Although Wheel-INS exhibits excellent DR performance, it is only a relative positioning solution lacking the ability to limit long-term error accumulation. That is to say, the positioning error of Wheel-INS still drifts over time, especially when the stochastic error of the inertial sensor is significant. To correct the accumulated error without depending on absolute positioning techniques, a commonly used method is loop detection and relocalization \cite{slamreview2016}. By extracting the environmental features with exteroceptive sensors, e.g., camera and LiDAR, the robot is allowed to recognize places that have been visited and the error drift can consequently be mitigated by, for example, performing a pose graph optimization using loop closure constraints \cite{qin2018}. Then, a natural question arises: \textit{Can we only use the Wheel-IMU to achieve loop closure so as to further limit the error accumulation?} 

%The answer is \textit{yes}.

From experiments of our prior work \cite{niu2021}, we found that the robot roll estimation in Wheel-INS gave distinguishable and repeatable terrain-correlated responses, as shown in Fig. \ref{examplerollfig}. Therefore, it would be possible to encode the robot roll angle as the road bank feature, so as to enable the loop closure detection in Wheel-INS. In addition, because the IMU is mounted on the wheel which is directly contacted with the ground, the roll angle estimated by Wheel-INS can represent the road bank angle precisely without being affected by the suspension system of the vehicle. 

Based on the discussion above, this letter proposes Wheel-SLAM, a simultaneous localization and terrain mapping system using only one Wheel-IMU. Specifically, we extend our previous study on Wheel-INS \cite{niu2021} by exploiting the robot roll angle estimates to encode the terrain feature which is used for loop closure detection to further limit the error drift of Wheel-INS. Within a particle filter (PF) framework, the uncertainty of the robot state is sampled by multiple particles which maintain their own trajectories and terrain maps. By matching the current roll angle estimation with the map, loop closure can be discovered to update the weights of the particles. In summary, our main contributions include:

\begin{enumerate}[1)]
\item A SLAM system with only one Wheel-IMU using the terrain feature (measured by the Wheel-IMU) is proposed and implemented.

\item We illustrate the feasibility of exploiting robot roll angle estimates to enable loop closure so as to limit the error drift effectively in Wheel-INS through extensive field experiments.

\item To the best of our knowledge, this is the first SLAM system using only one low-cost wheel-mounted IMU for the wheeled robot in the literature. We make our code publicly accessible.
\end{enumerate}

Note that this letter mainly focuses on the feasibility of the idea (achieving loop closure with only one Wheel-IMU in Wheel-INS) instead of the development of a complete system that can be straightforwardly applied to practical applications.  
%Consequently, we implement it by PF, which is easy to understand and did not use advanced data structure to maintain the terrain map. 

\section{Related Work}

\subsection{Terrain Matching-based Localization}
Terrain-based vehicle localization provides a usable alternative to GNSS to obtain absolute positioning results by exploiting road information. The repetitive, location-dependent nature of the terrain features allows them to be used for robot localization and mapping \cite{litianyi2019}. Usually, the terrain features are extracted with the in-vehicle inertial sensor, for example, an IMU can measure the road grades by vehicle pitch, road bank angles by vehicle roll, and road curvature by vehicle yaw rate \cite{Zhang2020rs}. The basic hypothesis is that the inertial sensor signals imply vehicle responses to the terrain surface and the same terrain surface excites similar vehicle motions \cite{Gim2021Access}. Existing literature mainly adopts vehicle pitch angles and pitch differences as the features for the terrain-based localization \cite{Vemulapalli2011ACC, Emil2015TITS}, although the roll angle can play the same role \cite{Ahmed2016VTC, Dean2008PhD}. However, using onboard inertial signals to determine the terrain information would be affected by the vehicle maneuver, for example, the braking may induce unexpected pitch variation of the vehicle \cite{litianyi2019} and the centripetal acceleration may introduce the difference between the vehicle roll angle and the road bank angle \cite{Ryu2004ACC}. 

%Li \textit{et al} \cite{litianyi2019} proposed an acceleration-considered model to take into account the disturbance from braking. The authors in \cite{Ahmed2016VTC} employed a Kalman filter to compensate the external acceleration in the lateral direction for an accurate roll estimation. 

%This problem can be considered in either time-domain or distance domain.

Various methods have been investigated to integrate the terrain feature matching result into the localization pipeline \cite{litianyi2019, Gim2021Access}, while PF has attracted more attention in recent studies. In PF methods, the weights of the particles are updated by evaluating the difference between the in-vehicle measurements (e.g., roll, pitch, etc.) and the response in the map \cite{Dean2011VSD, litianyi2019, Ahmed2016VTC}. Martini et al. \cite{Martini2006PhD} used the Pearson-product correlation coefficient as a distance metric to compare the road grade measurements with the reference map. 
%Laftchiev et al. \cite{Emil2015TITS} introduced linear dynamical models for the map representation of the pitch data to compress the storage space as well as reduce the complexity of feature matching. An optimized time series subsequence matching algorithm is proposed to match a sliding window of pitch data with the map \cite{Zhang2020rs}. 

%\cite{Gim2021Access} constructed a feature-indexed map consists of the positions with their statistical information indexed by representative features of spectrograms to improve the robustness of the feature matching. The spectrogram is generated by the vertical and lateral road shapes within a certain distance. Multi-size maps and the vehicle dynamic model were also introduced to further improve the localization accuracy.

% However, only using current the observation from the current moment is not enough to ensure both the accuracy and convergence speed.

However, a pre-built map is indispensable in all these methods which limits its application. In addition, the inertial sensors used to measure the terrain information are all placed on the vehicle body. By extending our previous studies \cite{niu2021, wu2021}, this letter proposes to localize the robot and measure the road bank angles simultaneously without a prior map. As the IMU is mounted on the robot wheel, the terrain matching would not be affected by the vehicle maneuver which is the case when the IMU is mounted on the vehicle body.

\subsection{SLAM Only Using Inertial Sensors}
%In fact, inertial sensors can also be used to perceive the environment via special installation. In \cite{Kolvenbach2020}, the authors proposed to inspect the concrete deterioration in sewers with legged robots using tactile interaction.
Recently deep learning-based inertial localization systems have shown promising results in both pedestrian~\cite{chen2018aaai, herath2022CVPR} and vehicular navigation \cite{martin2020tiv}. These methods learn either motion information from raw inertial measurements \cite{chen2018aaai, herath2022CVPR} or dynamic measurement noise \cite{martin2020tiv} to solve the inertial odometry problem in a data-driven way but fail to exploit environmental signals to limit long-term positional drift.

Angermann et al. \cite{Angermann2012IEEEProc} proposed a pedestrian SLAM system using only foot-mounted IMU (FootSLAM) by taking advantage of human perception and cognition. A dynamic Bayesian network was employed to represent the fact that when a pedestrian walks in a constrained environment, e.g., an office building, he or she relies mainly on visual cues to avoid obstacles and determine accessible areas. Specifically, the algorithm was implemented based on a PF where the weights of the particles were updated by the probability of the pedestrian crossing transitions in a regular 2D grid of adjacent hexagons. After that, a probabilistic transition map implicitly encoding the environmental features that influence the pedestrian's visual impression and intention was constructed.  
%Benefiting from the reliable DR results from the foot-mounted IMU-based pedestrian navigation system, FootSLAM is a pioneering work to challenge the well-known curse in navigation: ``Inertial sensor only-based navigation system is subject to unbounded error growth over time" \cite{Angermann2012IEEEProc}.

%The authors further extended FootSLAM to FeetSLAM \cite{Robertson2011ION} by integrating multiple odometry result from multiple pedestrains to improve both the convergence speed and accuracy. 
Different from FootSLAM where the building layout is implicitly used to perform SLAM by taking advantage of human cognition, Wheel-SLAM uses the Wheel-IMU to explicitly extract the terrain feature for loop closure detection.

In conclusion, Wheel-SLAM borrows ideas from both the terrain matching-based vehicle localization and FootSLAM. In contrast to Wheel-INS \cite{niu2021}, we extend the approach by extracting the road features with the Wheel-IMU to enable loop closure, so as to further limit the error drift. In Wheel-SLAM, we maintain and update the grid map in real-time and detect the loop closure using the DR result from Wheel-INS. After that, we use the roll angle sequence matching results between current estimates and the map to update the weights of the particles for the sake of robustness. 

% It is the excellent DR ability of Wheel-INS that makes the idea to achieve SLAM with only one IMU for wheeled robots possible. 


\section{Methodology}
In this section, a brief introduction to Wheel-INS is first provided. Then we explain the details of the PF-based Wheel-SLAM algorithm, including the dynamic Bayesian model, the construction of the terrain map, and the update of the weights of the particles.

\subsection{Background}
Wheel-INS \cite{niu2021} is the foundation of Wheel-SLAM. It is used to provide the robot odometry and roll angle estimation. There are two major advantages of Wheel-INS. First, the wheel velocity can be calculated by the gyro output and wheel radius enabling the same information fusion as ODO/INS with only one IMU (no other sensors). Second, it can take advantage of rotation modulation to limit the error drift of INS. %Experimental results have illustrated that the positioning and heading accuracy of Wheel-INS have been respectively improved by 23\% and 15\% against ODO/INS. Furthermore, Wheel-INS exhibits significant resilience to the gyroscope bias comparing with ODO/INS \cite{niu2021}.  

Due to the space limitation, we just outline the algorithm of Wheel-INS here. Please refer to our earlier papers \cite{niu2021, wu2021} for details, e.g., the rotation modulation of the Wheel-IMU, the definition of the misalignment errors, etc.

\begin{figure}[t]
	\centering
	\includegraphics[width=7.5cm]{Wheel-IMUInstallation_2022.png}
	\caption{Installation scheme of the Wheel-IMU and the definitions of the vehicle frame (\textit{v}-frame), wheel frame (\textit{w}-frame), and IMU body frame (\textit{b}-frame) \cite{niu2021}.}
	\label{WheelIMUInstallation}
\end{figure}

\begin{figure}[t]
	\centering
	\includegraphics[width=8cm]{WheelINSSystemOverview.png}
	\caption{Overview of Wheel-INS \cite{niu2021}. $\bm\omega$ and $\bm{f}$ are the angular rate and specific force measured by the Wheel-IMU, respectively; PVA represents the position, velocity, and attitude of the Wheel-IMU. We use the output from the Wheel-IMU to perform INS mechanization to predict the robot state (PVA). The angular velocity measured in the \textit{x}-axis of the Wheel-IMU and the wheel radius are used to calculate the forward speed. This speed is then integrated with NHC as a 3D velocity observation to update the robot state as well as correct the inertial sensor errors, e.g., gyro bias, through an EKF. }
	\label{WheelINSOverview}
\end{figure}

Fig. \ref{WheelIMUInstallation} depicts the installation of the Wheel-IMU and the definition of related coordinate systems. The system overview of Wheel-INS is shown in Fig. \ref{WheelINSOverview}. First, the forward INS mechanization is performed to predict the robot states. At the same time, the gyroscope outputs in the \textit{x}-axis of the Wheel-IMU are used to calculate the wheel speed. Then, this vehicle velocity is treated as an external observation with non-holonomic constraint (NHC) \cite{dissanayake2001} to update the state through an error-state extended Kalman filter (EKF) \cite{thrun2005probabilistic}. 

\iffalse
The state vector can be written as 

\begin{equation}
\bm{x}(t)=\left[\left(\delta \bm{r}^{n}\right)^{\mathrm{T}} \quad\left(\delta \bm{v}^{n}\right)^{\mathrm{T}} \quad \bm{\phi}^{\mathrm{T}} \quad \bm{b}_{g}^{\mathrm{T}} \quad \bm{b}_{a}^{\mathrm{T}} \quad \bm{s}_{g}^{\mathrm{T}} \quad \bm{s}_{a}^{\mathrm{T}} \right]^{\mathrm{T}}
\label{statevector}
\end{equation}

where $\delta$ indicates the uncertainty of the variables; $\delta \bm{r}^{n}$, $\delta \bm{v}^{n}$ and $\bm{\phi}$ indicate the position, velocity and attitude errors of INS, respectively; $\bm{b}_{g}$ and $\bm{b}_{a}$ denote the residual bias errors of the gyroscope and the accelerometer, respectively; $\bm{s}_{g}$ and $\bm{s}_{a}$ are the residual scale factor errors of the gyroscope and accelerometer, respectively.
\fi

The forward wheel velocity calculated by the gyroscope data of the Wheel-IMU and wheel radius can be written as
\begin{equation}
\begin{aligned}
\widetilde{v}^{v}_{wheel} &=\widetilde{\omega}_{x}r-e_v = ({\omega}_{x}+\delta{\omega}_{x})r-e_v\\ &={v}^{v}_{wheel}+r\delta{\omega}_x - e_v
\end{aligned}
\end{equation}
where $\widetilde{v}^{v}_{wheel}$ and ${v}^{v}_{wheel}$ are the observed and true vehicle forward velocity, respectively; $\widetilde{\omega}_{x}$ is the gyroscope output in the \textit{x}-axis; ${\omega}_{x}$ is the true value of the angular rate in the \textit{x}-axis of the Wheel-IMU; $\delta{\omega}_{x}$ is the gyroscope measurement error; $r$ is the wheel radius, and $e_v$ is the observation noise of the wheel velocity, modeled as Gaussian white noise. 
\iffalse
By integrating with NHC, the 3D velocity observation can be expressed as
\begin{equation}
\widetilde{\bm{v}}^{v}_{wheel} =\begin{bmatrix}
\widetilde{v}^{v}_{wheel} &\! \!0 &\!0
\end{bmatrix}^\mathrm{T}-\bm{e}_v
\end{equation}
where $\bm{e}_v$ is the velocity measurement noise vector, including both the forward velocity noise and NHC noise.
\fi

Because the Wheel-IMU rotates with the wheel, the pitch of the robot is unknown in Wheel-INS. In other words, we cannot determine the robot's ascent and descent in Wheel-INS. Therefore, it is assumed that the robot is moving on a horizontal plane. However, experimental results \cite{niu2021} illustrated that this assumption does not cause significant navigation errors. 

\subsection{Dynamic Bayesian Network}
%Based on the concept of sequential importance sampling and the use of Bayesian theory, PF is widely used to deal with nonlinear and non-Gaussian problems \cite{Djuric2003IEEESPM}. 
A PF is a sequential Monte Carlo method where the basic idea is the recursive computation of relevant probability distributions using the concepts of importance sampling and approximation of probability distributions with discrete random measures \cite{Djuric2003IEEESPM}. In PF, the posterior distribution of the robot state is represented by a set of particles that evolve recursively with the integration of new information. Based on the technique of Rao-Blackwellization \cite{murphy2001rao, cyrill2007TRO, Montemerlo2002AAAI}, Wheel-SLAM decomposes the SLAM problem into a robot localization problem and a terrain mapping problem that is conditioned on the robot pose estimate. %The idea is somewhat similar to FastSLAM \cite{Montemerlo2002AAAI, Montemerlo2003IJCAI}. % but here we don't need to estimate the states of the landmarks.

In Wheel-SLAM, we try to estimate the posterior
\begin{equation}
p({x}_{1:t},\textbf{M}|{z}_{1:t},{u}_{1:t})
\label{posteriorEq}
\end{equation}
which is the distribution representing the robot state ${x}_{1:t}$ and the terrain map $\textbf{M}$ based on the set of control inputs ${u}_{1:t}$ which governs the robot motion and the road bank angle observations ${z}_{1:t}$. The conditional independence property of the Wheel-SLAM problem implies that the posterior in (\ref{posteriorEq}) can be factored as follows:
\begin{equation}
p({x}_{1:t},\textbf{M}|{z}_{1:t},{u}_{1:t}) = p({x}_{1:t}|{z}_{1:t},{u}_{1:t}) \prod \limits_{i=1}^{N_f} p(m_i|{z}_{1:t},{x}_{1:t})
\end{equation}
where $m_i$ is the \textit{i}-th terrain feature, and $N_f$ is the total number of the features. In Wheel-SLAM, Wheel-INS is performed for the robot state estimation as well as the road bank angle perception.

% Exploiting the technique of Rao-Blackwellization, the SLAM problem can be decomposed into robot localization and terrain map construction given the robot position.

Wheel-SLAM uses a PF to estimate the robot trajectory distribution. For each particle, the individual trajectory-based terrain map is independent of each other. As a result, every particle is composed of a robot pose and a terrain map; thus, the \textit{i}-th particle at time \textit{t} can be represented as
\begin{equation}
X_i^{t} = \begin{bmatrix}{x}_i^t &\! \textbf{M}_i\end{bmatrix}
\end{equation}
where $i = 1,2,3,...,N_p$, is the index of the particle while ${N_p}$ is the total number of the used particles. $x_i^t$ is the pose of the robot estimated by the \textit{i}-th particle at time \textit{t}, and $\textbf{M}_i$ is the terrain map maintained by the \textit{i}-th particle. Here the robot pose is represented by 2D translation $p_i^t\!\in\!\mathbb{R}^2$ and heading $\textbf{R}_i^t\!\in\!SO(2)$ as Wheel-INS cannot estimate pitch angle, namely, the robot motion in the vertical direction \cite{niu2021}. 

The Wheel-SLAM algorithm consists of four main steps: 1) Sample new robot state by the motion model; 2) Update the terrain map; 3) Update the particle weights once a convinced loop closure is reported; 4) Resample the particles when it is necessary. Algorithm 1 shows an overview of Wheel-SLAM.

\begin{figure}[!t]
	\label{alg1}
	\renewcommand{\algorithmicrequire}{\textbf{Input:}}
	\renewcommand{\algorithmicensure}{\textbf{Output:}}
	%\removelatexerror
	\begin{algorithm}[H]
		\caption{Wheel-SLAM}
		\begin{algorithmic}[1]
			\REQUIRE Robot pose $x_{t-1}$, terrain map, $\textbf{M}_{t-1}$, control input, $u_t$ and measurement, $z_t$. 
			\ENSURE Robot pose $x_{t}$, terrain map, $\textbf{M}_{t}$.
            \FOR{each $i=1 \rightarrow N_p$}
                \STATE Predict the vehicle pose by the motion model $p({x}_{t}|{x}_{t-1},{u}_{t})$.
                \STATE Update the terrain map $p(\textbf{M}_{t}|{x}_{t},{z}_{t})$ if necessary.
                \IF {Loop closure detected \&\& Roll sequence matches}   %
            		\STATE Update the weight of the particles according to (\ref{weightupdateEq}).
            	\ENDIF
            \ENDFOR
            \STATE Normalize the weights of the particles $\omega^i = {\omega^i}/{\sum \limits_{i=1}^{N_p} \omega^i}$.  
            \IF {Resampling is required (${N}_\text{eff} = {1}/{\sum \limits_{i=1}^{N_p} (\omega^i)^2} < 0.75$)}          
        		\STATE Perform resampling.
        	\ENDIF
		\end{algorithmic}
	\end{algorithm}
\end{figure}

%\subsection{Motion Model}
The first step is to generate a new pose for each particle by sampling from the robot probabilistic motion model:
\begin{equation}
p({x}_{t}|{x}_{t-1},{u}_{t})
\end{equation}
which means that the robot pose, $x_t$, is a probabilistic function of the robot control ${u}_t$ and the previous pose $x_{t-1}$. Here, we adopt Wheel-INS to predict the robot state.

\subsection{Grid Terrain Map Construction}
% In FootSLAM, the environment is represented with a finite set of hexagons where each hexagon contains the probability of the pedestrian crossing transitions in a 2D grid of adjacent hexagons \cite{Angermann2012IEEEProc}. The hexagons were chosen because, on one hand, they can cover a 2D space completely and efficiently; on the other hand, six edges are reasonable to indicate the angular choice of human motion. 

Compared to the hexagon grid map used in FootSLAM \cite{Angermann2012IEEEProc}, we simplify the grid to square due to the relatively simple robot motion mode. As we assume that the vehicle is moving on the horizontal plane, we build a 2D grid map. Each grid holds the corresponding road bank angle estimated by Wheel-INS at that position. Fig. \ref{gridmap} illustrates the grid map constructed along with the robot pose evolution.

\begin{figure}[tbp]
	\centering
	\includegraphics[width=7.5cm]{gridmap_2022_new.png}
	\caption{Illustration of the grid map construction and revisit recognition. Different colors of curves represent the robot path sampled by different particles. The gray grids have been visited by the robot thus they have a road bank angle estimation. Once a particle detects that the robot has continuously returned to visited grids (blue), a potential loop closure is reported for further check (Please refer to Section III-D for details).}
	\label{gridmap}
\end{figure}

\subsection{Particle Weight Update}
In the beginning, all the particles are assigned the same weight. When the robot moves, each particle has a different trajectory and terrain map. To ensure the reliability of loop closure and reduce the impact of an outlier, we set three criteria. First, loop closures need to be continuously detected by the robot position in a window of length $N_r$. Second, we calculate the $N_r$ roll sequence matching scores using Pearson correlation coefficient \cite{Martini2006PhD} and compare them with a threshold $C_\text{thr}$. In this $N_r$ window, at least $N_\text{thr}$ ($N_\text{thr} <= N_r$) coefficients need to be larger than $C_\text{thr}$. Third, the correlation coefficient at the current position needs to be larger than the threshold. If all three requirements are met, we think it is a true loop closure and subsequently update the particle weights as follows:

%Once a particle reports a loop closure according to its position, the particle weight is updated according to the difference between the current robot roll angle estimation and the road bank angle stored at the corresponding grid in the map, as shown in \ref{weightupdateEq}. 
%we match the current roll angles with the map to further check if the loop closure is reliable. Given the uncertainty of the robot position, we extract multiple roll sequence candidates in a certain area from the map for matching. Then we update the weights of the particle if we believe it is a true loop closure (the correlation coefficient between the roll sequence bigger than a threshold). 
%As the roll angles in each grid of the map are sampled based on position, we use the Pearson correlation coefficient instead of DTW to evaluate the similarity between two roll sequences for the sake of efficiency.

\begin{equation}
{\omega}_k^{i} = {\omega}_{k-1}^{i}\frac{N_c}{N_r}\exp(\mathrm{RMSE}(V_\text{coeff}))
\label{weightupdateEq}
\end{equation}
where $k$ indicates the time epoch; $i$ indicates the number of the particle; $N_c$ ($N_c\!\geq\! N_\text{thr}$) is the number of the correlation coefficients larger than the threshold in the window $N_r$; RMSE indicates the root mean square error; $V_\text{coeff}$ is the collection of the correlation coefficients larger than the threshold in the window. After that, normalization is performed to make the sum of the weights equal to one. 
%$\Tilde{\phi_k^{i}}$ indicates the robot roll angle estimated by wheel-INS while $\hat{\phi_k^{i}}$ indicates the road bank angle retrieved from the map; $\sigma_{\phi}$ indicates the roll angle estimation noise.

Note that although Wheel-SLAM requires the exact revisit of the robot, the vehicle doesn't need to always drive on the same road. The position error will accumulate when the robot explores an unknown environment, but it can be corrected once the robot is back on a previously-visited road. Please refer to Section IV for the experimental results and discussion. Moreover, as our algorithm requires a sequence of road bank angles for matching, it is not able to detect loop closure effectively at crossroads where the robot may enter a place perpendicular to its last entrance direction.

\section{Experimental Results}
This section presents and discusses the real-world experimental results to illustrate the positioning performance of Wheel-SLAM. As this work focus on the feasibility of the idea (achieving loop closure with only one Wheel-IMU) instead of the development of a complete system, we do not evaluate the computational efficiency of Wheel-SLAM. In addition, there is no paper in the literature studying SLAM with the same sensor set (wheel-mounted inertial sensors). We only compare the proposed Wheel-SLAM with its predecessor, Wheel-INS, to illustrate its effectiveness and discuss its characteristics. First, the experimental conditions are described. After that, we compare the positioning accuracy of Wheel-SLAM with Wheel-INS. The characteristics of Wheel-SLAM are also analyzed. Finally, we further discuss the key components which play a central role in Wheel-SLAM and some limitations in real-world applications. 

\subsection{Experimental Description}
To demonstrate the feasibility and effectiveness of the proposed Wheel-SLAM system, we carried out five sets of field tests using a car on the campus of Wuhan University. The car was instrumented with one Wheel-IMU and reference system to provide the ground truth of the vehicle pose, as shown in Fig. \ref{figexpplatform}. The characteristics of the vehicle motion in the tests are shown in Table \ref{Tabvehiclemotion}. 

\begin{figure}[t]
	\centering
	\includegraphics[width=8cm]{expplatform_2022_RAL.png}
	\caption{Experimental platform. A GNSS antenna and a high-end IMU (POS320) were mounted on the car roof to provide the ground truth of the vehicle pose while a low-cost MEMS IMU was mounted on the right rear wheel to perform Wheel-SLAM.}
	\label{figexpplatform}
\end{figure}

\begin{table}[t]
	\centering
	\caption{Vehicle Motion Information in the Experiments}
	\label{Tabvehiclemotion}
	\begin{tabular}{cccccc}
		\toprule
		{Sequence} & I & II & III  & IV & V \\
		\midrule
		\specialrule{0em}{2pt}{2pt}
		
		\makecell{Average \\ speed ($m/s$)} & \makecell{5.41} & \makecell{5.47} & \makecell{4.93} & \makecell{5.16} & \makecell{4.89} \\ 
		\makecell{Total \\ distance ($m$)} & \makecell{$\approx$2950 } & \makecell{$\approx$3791 } & \makecell{$\approx$2802 } & \makecell{$\approx$7235 } & \makecell{$\approx$9285 } \\
		
		\bottomrule
	\end{tabular}   
\end{table}

\begin{table}[t]
	\centering
	\caption{Technical Parameters of the IMUs Used in the Tests}
	\label{TabIMUparas}
	\begin{threeparttable}
		\begin{tabular}{m{1.2cm}<{\centering}m{1.3cm}<{\centering}m{1.3cm}<{\centering}m{1.3cm}<{\centering}m{1.3cm}<{\centering}}
			\toprule
			IMU & Gyro Bias\newline ($^\circ/h$) & {ARW \newline ($^\circ/\sqrt{h}$)}  & {Acc. Bias \newline ($m/s^2$)} & {VRW \newline ($m/s/\sqrt{h}$)} \\
			\midrule
			\specialrule{0em}{3pt}{3pt}
			POS320 & 0.5  & 0.05 & 0.00025 & 0.1\\
			\specialrule{0em}{3pt}{3pt}
			ICM20602 & 200 & 0.24 & 0.01 & 3\\
			\bottomrule
		\end{tabular}
		%\begin{tablenotes}   
		%	\footnotesize            
		%	\item[*]ARW denotes the angle random walk; Acc. denotes the accelerometer; VRW denotes the velocity random walk.      
		%\end{tablenotes}            
	\end{threeparttable}
\end{table}


\begin{figure}[t]
	\centering
	\includegraphics[width=8cm]{alltrajs_2022.png}
	\caption{Experimental trajectories. Seq. 1, Seq. 2, and Seq. 3 are circular trajectories where the vehicle moved several times in one direction while Seq. 4 and Seq. 5 are more complicated ones where the vehicle moved in different directions (not just turn around in one direction) in large-scale environments.}
	\label{figtraj}
\end{figure}

We used the same MEMS IMU from our prior works \cite{niu2021, wu2021}. The MEMS IMU contained four ICM20602 (TDK InvenSense) inertial sensor chips, a chargeable battery module, a microprocessor, an SD card for data collection, and a Bluetooth module for communication and data transmission. One can use an android phone to control the data collection. We collected the data of one chip of the Wheel-IMU for post-processing. The reference system used in the experiments was a high-accuracy position and orientation system with a tactical-grade IMU (POS320, MAP Space Time Navigation Technology Co., Ltd., China). The reference data were processed through a smoothed post-processed kinematic (PPK)/INS integration method. The main technical parameters of both the MEMS IMU and the high-end IMU are listed in Table \ref{TabIMUparas}, where ARW denotes the angle random walk; Acc. denotes the accelerometer; VRW denotes the velocity random walk.

Fig. \ref{figtraj} shows the five experimental trajectories. In Sequence (Seq.) 1, the car was traveling back and forth on two parallel and opposite-direction roads. Seq. 2, and Seq. 3 are circular trajectories where the vehicle moved several times in one direction. Seq. 4 and Seq. 5 are more complicated ones with two circles in large-scale environments. In Seq. 4 and Seq. 5, the car not only traveled in the same lane in the same direction but moved in opposite directions on the same road. Note that in Seq. 1, Seq. 2, and Seq. 3, the car sometimes also changed the lane. Please refer to our source code webpage for the street views of the sequences.

%The initial heading of Wheel-SLAM was given by the reference system directly. Additionally, t
The static IMU data before the car started moving were used to obtain the initial roll and pitch angle of the Wheel-IMU, as well as the initial value of the gyroscope bias. Other inertial sensor errors were set as zero. The key parameters of Wheel-SLAM set in the experiments are listed in Table \ref{TabWSpara}. Standard deviation is denoted as STD in Table \ref{TabWSpara}.

\begin{table}[t]
	\centering
	\caption{Key parameters of Wheel-SLAM in the experiments.}
	\label{TabWSpara}
	\begin{threeparttable}
		\begin{tabular}{p{5cm}p{3cm}}
			\toprule
			{Parameters} &  {Value} \\
			%\cline{3-6}
			\specialrule{0em}{1.5pt}{1.5pt}
			
			\midrule
			\specialrule{0em}{1.5pt}{1.5pt}
			Particle number & 100 \\
			\specialrule{0em}{1.5pt}{1.5pt}
			Grid size of the 2D terrain map & 1.5\,m \\
			\specialrule{0em}{1.5pt}{1.5pt}
			Distance increment STD & 0.025\,m \\ %\tnote{*}
			\specialrule{0em}{1.5pt}{1.5pt}
			Heading increment STD & 0.05\,$^\circ$ \\
			\specialrule{0em}{1.5pt}{1.5pt}
			Roll sample distance & 0.5\,m \\
			\specialrule{0em}{1.5pt}{1.5pt}
			Roll matching sequence length & 25\,m \\
			\specialrule{0em}{1.5pt}{1.5pt}
			Correlation coefficient threshold & 0.4 \\
			
			\bottomrule
		\end{tabular}
		%\begin{tablenotes}   
		%	\footnotesize            
		%	\item[*]STD denotes standard error.      
		%\end{tablenotes} 
	\end{threeparttable}
\end{table}

\begin{figure}[!t]
	\begin{tabular}{cc}
		\begin{minipage}[t]{0.5\linewidth}
		    \centering
			\subfigure[Estimated trajectories against ground truth in Seq. 1.]{
			\includegraphics[height = 0.7\linewidth]{1trajcomp.png}
		}
		\end{minipage}
		\begin{minipage}[t]{0.5\linewidth}
		    \centering
			\subfigure[Horizontal positioning and heading errors in Seq. 1.]{
			\includegraphics[height = 0.7\linewidth]{1err_2022.png}
		}
		\end{minipage}
	\end{tabular}
	
	\begin{tabular}{cc}
		\begin{minipage}[t]{0.5\linewidth}
		    \centering
			\subfigure[Estimated trajectories against ground truth in Seq. 2.]{
				\includegraphics[height = 0.7\linewidth]{2trajcomp.png}
			}
		\end{minipage}
		\begin{minipage}[t]{0.5\linewidth}
		    \centering
			\subfigure[Horizontal positioning and heading errors in Seq. 2.]{
				\includegraphics[height = 0.7\linewidth]{2err_2022.png}
			}
		\end{minipage}
	\end{tabular}
		
	\begin{tabular}{cc}
		\begin{minipage}[t]{0.5\linewidth}
		    \centering
			\subfigure[Estimated trajectories against ground truth in Seq. 3.]{
				\includegraphics[height = 0.7\linewidth]{3trajcomp.png}
			}
		\end{minipage}
		\begin{minipage}[t]{0.5\linewidth}
		    \centering
			\subfigure[Horizontal positioning and heading errors in Seq. 3.]{
				\includegraphics[height = 0.7\linewidth]{3err_2022.png}
			}
		\end{minipage}	
	\end{tabular}	
	
	\begin{tabular}{cc}
		\begin{minipage}[t]{0.5\linewidth}
		    \centering
			\subfigure[Estimated trajectories against ground truth in Seq. 4.]{
				\includegraphics[height = 0.7\linewidth]{4trajcomp.png}
			}
		\end{minipage}
		\begin{minipage}[t]{0.5\linewidth}
		    \centering
			\subfigure[Horizontal positioning and heading errors in Seq. 4.]{
				\includegraphics[height = 0.7\linewidth]{4err_2022.png}
			}
		\end{minipage}
	\end{tabular}
	
	\begin{tabular}{cc}
		\begin{minipage}[t]{0.5\linewidth}
		    \centering
			\subfigure[Estimated trajectories against ground truth in Seq. 5.]{
				\includegraphics[height = 0.7\linewidth]{5trajcomp.png}
			}
		\end{minipage}
		\begin{minipage}[t]{0.5\linewidth}
		    \centering
			\subfigure[Horizontal positioning and heading errors in Seq. 5.]{
				\includegraphics[height = 0.7\linewidth]{5err_2022.png}
			}
		\end{minipage}
	\end{tabular}
\caption{The estimated trajectories and corresponding horizontal position and heading errors of Wheel-INS and Wheel-SLAM in the five experiments.}
\label{figtrajcomp2}
	
\end{figure}

\subsection{Performance Comparison and Analysis}

\subsubsection{Performance Comparison}
Fig. \ref{figtrajcomp2} compares the positioning and heading errors of Wheel-SLAM and Wheel-INS in all five experiments, respectively. Please refer to our source code webpage for the positioning results of Wheel-SLAM overlaid on Google earth. It is obvious that compared to Wheel-INS, Wheel-SLAM can suppress the error drift effectively, even in complicated scenarios where the car moves in different directions in large-scale environments, e.g., Seq. 4 and Seq. 5. However, the positioning error is increasing with drift of heading error in Wheel-INS. In addition, we can also notice that there are periodic decreases in the positioning error of Wheel-INS. This is due to the loop closure of the trajectory which cancels part of the cumulative error as discussed in our prior work \cite{niu2021}. 

At the start of the experiments, the positioning results of Wheel-SLAM drift together with Wheel-INS because all the particles are sampled from Wheel-INS according to the Gaussian distribution and there is no external measurement to correct the position error. Once a credible loop closure is reported by some particles, they get a larger weight. Accordingly, the vehicle position would deflect to the previous trajectory when the vehicle visited the same place the last time. It can be observed that there are some sharp decreases in the positioning and heading error of Wheel-SLAM in the experiments, such as the 150\,s in Fig. \ref{figtrajcomp2}(d) and the 1000\,s in Fig. \ref{figtrajcomp2}(j). This is because we use the weighted average value of all the particles as the output. The increased weights of those particles which detect the loop closure would drag the vehicle position to the previous trajectory. In addition, we only correct the current robot state but not the historical trajectory. Using modern graph optimization tools, e.g., gtsam \cite{gtsam}, can jointly optimize the history state so as to make the trajectory smoother and improve the overall accuracy, but this letter mainly focuses on the feasibility to extract road features to enable loop closure in Wheel-INS.

We can observe from the figures that both the positioning and heading error of Wheel-INS increase quickly over time, while the same value of Wheel-SLAM is constrained at a certain level. When there is no loop closure, Wheel-SLAM exhibits the same drift as Wheel-INS in the first circle of the trajectory. However, as the vehicle continues to revisit the roads previously visited, the position drift of Wheel-SLAM can be limited as that in the first circle because of the loop closure mechanism. This suggests that the proposed method is effective to limit the error growth of Wheel-INS by using terrain information perceived by the Wheel-IMU to perform the loop closure.

It is worth mentioning that in Seq. 5, there is a large position and heading drift from about 860\,s to 1100\,s (see Fig. \ref{figtrajcomp2}(j)). This is because the vehicle position error of Wheel-INS drifts quickly at that time. Meanwhile, there is no loop closure reported by the particles, so Wheel-SLAM exhibits a similar drift trend during this time period. Afterward, a loop closure is successfully detected in Wheel-SLAM at about 1100\,s, resulting in a radical error correction.

\begin{table}[t]
	\centering
	\caption{Positioning and Heading Error Statistics of Wheel-SLAM and Wheel-INS}
	\label{TaberrStat}
	\begin{threeparttable}
		\begin{tabular}{p{0.6cm}<{\centering}p{1.3cm}<{\centering}p{1.7cm}<{\centering}p{1.3cm}<{\centering}p{1.7cm}<{\centering}}
			\toprule
			\multirow{2}*{\makecell{Seq.}} &  \multicolumn{2}{c}{Horizontal Pos. RMSE(m)} & \multicolumn{2}{c}{Heading RMSE $(^\circ)$} \\
			%\cline{3-6}
			\specialrule{0em}{1.5pt}{1.5pt}
			& Wheel-INS & Wheel-SLAM & Wheel-INS & Wheel-SLAM \\
			\midrule
			\specialrule{0em}{1.5pt}{1.5pt}
			{1}&5.70& \textbf{2.50}&1.96&\textbf{1.00}\\
			\specialrule{0em}{1.5pt}{1.5pt}
			{2}& 27.09& \textbf{9.38}&11.36&\textbf{3.17}\\
			\specialrule{0em}{1.5pt}{1.5pt}
			{3}& 18.03& \textbf{8.27}&8.32&\textbf{3.83}\\
			\specialrule{0em}{1.5pt}{1.5pt}
			{4}& 20.24& \textbf{9.21}&6.09&\textbf{2.43}\\
			\specialrule{0em}{1.5pt}{1.5pt}
			{5}&21.44& \textbf{14.42}&4.26&\textbf{2.95}\\
			
			\bottomrule
		\end{tabular}
		%\begin{tablenotes}   
		%	\footnotesize            
			%\item[*]Pos. denotes position; RMSE denotes root mean square error.      
		%\end{tablenotes} 
	\end{threeparttable}
\end{table}

Table \ref{TaberrStat} lists the error statistics of the navigation results of Wheel-SLAM and Wheel-INS in all five experiments. We calculate the root mean square error (RMSE) of the horizontal position and heading (denoted as Horizontal Pos. RMSE and Heading RMSE in \ref{TaberrStat}, respectively) as the indicators to evaluate the navigation performance. Additionally, given the stochasticity of the particle sampling, we run the algorithm multiple times to get the mean value of each test. It can be learned that thanks to the loop closure detection, Wheel-SLAM overwhelmingly outperforms Wheel-INS in terms of both position and heading estimation. Compared to Wheel-INS, the positioning and heading accuracy has been improved by 32.7\%\,$ \sim $\,65.4\% and 30.7\%\,$ \sim $\,72.1\%, respectively. 

\begin{figure*}[t]
	\centering
	\includegraphics[width=15.5cm]{rollmap_2022.png}
	\caption{The terrain maps estimated by Wheel-SLAM in Seq. 1, Seq. 2, and Seq. 4, respectively. The colors represent the values of the road bank angles. The larger the road bank angle, the lighter the color.}
	\label{figrollmap}
\end{figure*}

As Seq. 1, Seq. 2, and Seq. 3 are rather simple, there are more opportunities for loop closures. Therefore, the performance improvements in these three tests are more significant than that in Seq. 4 and Seq. 5. 

The terrain maps built by Wheel-SLAM in Seq. 1, Seq. 2, and Seq. 4 are shown in Fig. \ref{figrollmap}. Because the wheel of the vehicle contacts the ground directly, the mapping is not affected by the suspension system of the vehicle especially when the vehicle maneuvering is large which is the case when the IMU is mounted on the vehicle body. Furthermore, these maps can be used to provide valuable information to monitor the deformation and deterioration of the roads. 
%are then encoded as the terrain feature to enable the loop closure of the Wheel-SLAM. It can be observed that there do exist obvious variations in the lateral direction of the roads.

\begin{figure}[t]
	\centering
	\includegraphics[width=7cm]{particlesNumComp.png}
	\caption{The positioning RMSE of Wheel-SLAM with different particle numbers in Seq. 1. Medians are indicated by the red lines, while the bottom and top edges of the boxes indicate the first quartile and third quartile, respectively, and the whiskers show the maximum (upper) and minimum (lower).}
	\label{figparticlesComp}
\end{figure}

\subsubsection{Analysis on the characteristics of Wheel-SLAM}
To further evaluate the performance and stability of Wheel-SLAM, we set different particle numbers to compare the positioning performance. The algorithm was run 100 times for each configuration. Fig. \ref{figparticlesComp} shows the results. 

It can be observed that the position error of Wheel-SLAM is more centralized when there are more particles, which means that the stability of Wheel-SLAM is improved with the increment of particles. This is because it is more likely for the system to detect the real loop closure with more particles and the performance would also be less susceptible to anomalies. However, there is no obvious gain by adding particles from 1000 to 2000 because of the diminishing marginal effect. In addition, the accuracy of Wheel-SLAM also depends on the position error when the vehicle visits the place for the first time. In consequence, continuing to add the particles only improve position accuracy lightly. %When the particle number is at a certain level, it is enough for the particles to revisit the trajectory.

Furthermore, it can be learned that in a statistical sense, the overall positioning performance of Wheel-SLAM is also not significantly enhanced with the increase of the particles. Although there are some outliers when the particle number is small (= 100), the system is robust to recognize the loop closure thanks to the excellent DR ability of Wheel-INS.   

\subsection{Discussion}
The core principles behind Wheel-SLAM can be summarized as follows: 1) Particles are spread to sample the possible state of the robot and detect the loop closure by the trajectory maintained by each particle; 2) The road bank angle sequence matching result is used to update the particle weights, so as to pick out the most trusted one(s). What plays a central role in Wheel-SLAM is the roll sequence matching strategy. It must be robust enough to keep the outstanding particles while filtering out false alarms. Therefore, we used a rather strict loop closure detection criteria to make the loop closure detection robust.

%As a result, there are two corresponding factors playing a central role in Wheel-SLAM. First one is the standard error (STD) of the odometry information generated form Wheel-INS. If the STD is too small, the particles will be overly centralized, making it difficult for the particles to cover the desired trajectory. On the other hand, if the STD is setting too large, the trajectories estimated by the particles will diverge widely. In such circumstances, more particles are needed to ensure the stability of the system which will increase the computation burden. Second one is the roll sequence matching strategy which must be robust enough to retain the outstanding particles while filter out false alarm proposed by the particles.

However, it can be realized that there are two major limitations in the application of Wheel-SLAM. First, the robot must strictly revisit the previous places with a certain length. It is not like the vision-based SLAM where the vehicle has the remote sensing ability by using exteroceptive sensors, e.g., camera and LiDAR. In Wheel-SLAM, the Wheel-IMU is used to extract the terrain features which can only be obtained by the exact arrival of the robot. Second, the success of the loop closure depends on the matching of the road bank angle sequence. If the robot is moving on extremely smooth roads without any fluctuations in the bank angle, it would be difficult to detect loop closure.   

\section{CONCLUSIONS}
In this study, we propose to perform SLAM with only one Wheel-IMU by exploiting the environmental perception ability of the Wheel-IMU. To be specific, we extend our previous study on Wheel-INS to Wheel-SLAM by extracting the terrain feature from the robot roll angle estimates to enable loop closure detection. The system is implemented with a Rao-Blackwellized particle filter where each particle maintains its own robot state and grid map.
%The environment is represented as a 2D grid map where each grid encodes the road bank angle indicated by the robot roll angle at that position.  The weights of particles are updated according to the difference between the current roll estimates and the value retrieved from the grid map. 
Experimental results show that the proposed method can effectively suppress the error drift of Wheel-INS. The positioning and heading accuracy has been averagely improved by 52.6\% and 53.2\%, respectively, against Wheel-INS.
%We emphasize that it is the excellent DR ability of Wheel-INS that makes the idea to achieve SLAM with only one IMU for wheeled robots possible. 

However, Wheel-SLAM has two major limitations. First, a certain level of variation in the bank angles of the road is needed. 
%If the road is too smooth like an indoor environment, there would be no obvious features for loop closure detection. 
Second, the robot has to revisit the same place exactly. %It is not able to detect loop closure just by ``seeing it" as visual SLAM. 
Wheel-SLAM would be suitable for those kinds of robots that move repeatedly in given areas, for example, the sweeping robots and patrol robots in restricted areas.

For future research, integrating Wheel-SLAM with other exteroceptive sensors (e.g., camera and LiDAR) would be promising to improve the robustness and applicability of the robot navigation system.

%\appendices


% Can use something like this to put references on a page
% by themselves when using endfloat and the captionsoff option.
%\ifCLASSOPTIONcaptionsoff
%  \newpage
%\fi


%\begin{thebibliography}{99}
\bibliographystyle{IEEEtranTIE}
\bibliography{ReferenceWheel-SLAM_2022}
%\end{thebibliography}
%\vspace{-200 mm}
% insert where needed to balance the two columns on the last page with
% biographies




% that's all folks
\end{document}


