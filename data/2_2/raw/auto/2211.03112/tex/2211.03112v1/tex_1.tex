\documentclass[12pt]{article}
 %\usepackage[numbers,sort&compress]{natbib}

\usepackage{natbib}
\setcitestyle{authoryear,round}

 \newtheorem{theorem}{Theorem}[section]
 \newtheorem{proposition}{Proposition}
 \newtheorem{lemma}{Lemma}[section]
 \newtheorem{corollary}{Corollary}[section]
 \newtheorem{definition}{Definition}[section]
 \newtheorem{remark}{Remark}[section]
 \newtheorem{example}{Example}[section]
 \renewcommand{\theequation}{\thesection.\arabic{equation}}
 \newcommand{\n}{\noindent}
 \newcommand{\hh}{\indent}
 \usepackage{graphicx}
 \usepackage{amsmath}
 \usepackage{amssymb}
 \usepackage{epsfig}
\usepackage{amsfonts}

\usepackage{mathrsfs}
\usepackage{amssymb,color}
\usepackage[colorlinks]{hyperref}
\usepackage{enumerate}
\usepackage{boondox-cal}
 \renewcommand\baselinestretch{1.5}
\newenvironment{proof}[1][Proof]{\noindent\textbf{#1.} }{\hfill $\Box$}
\topmargin=0mm \evensidemargin=10pt \oddsidemargin=10pt \pagestyle{myheadings} \headsep=0mm
\textwidth=17truecm \textheight=22 truecm
\parindent=1.5em
\def\hang{\hangindent\parindent}
\def\textindent#1{\indent\llap{#1\enspace}\ignorespaces}
\def\re{\par\hang\textindent}
\def\n{\nabla}
\def\D{\Delta}
\def\de{\delta}
\def\O{\Omega}
\def\o{\bar}
\def\l{\lambda}
\def\m{\mu}
\def\i{\infty}
\def\b{\beta}
\def\g{\gamma}
\def\t{\tau}
\def\p{\phi}
\def\va{\varepsilon}
\def\to{\rightarrow}
\def\R{\mathbb R}
\def\a{\forall}
\def\G{\Gamma}
\def\f{\varphi}
\def\w{\widetilde}
\newtheorem{assumption}{Assumption}[section]

 \def\QEDopen{{\setlength{\fboxsep}{0pt}\setlength{\fboxrule}{0.2pt}\fbox{\rule[0pt]{0pt}{1.3ex}\rule[0pt]{1.3ex}{0pt}}}}
 \def\QED{\QEDopen}
 \def\proof{\noindent{\bf Proof}: }
 \def\endproof{\hspace*{\fill}~\QED\par\endtrivlist\unskip}

%\usepackage{titling}
%\thanksmarkseries{alph} %%roman��Roman��farabic��alph��Alph, fnsymbol %% change the markers of thanks
\usepackage{ctex}

%%============================================================
\title{On Existence of $\alpha$-Core Solutions for Games with Finite or Infinite Players
%\thanks{This work is supported by National Natural Science Foundation of China (Grant 11661030).}
}
\author{
Qi-Qing Song$^{1}$\thanks{E-mail:songqiqing@126.com}, Min Guo$^{2}$\\
\small{(1. School of Mathematics and Computation Science, Shanxi Normal University,}\\
\small{Linfen 041004,\,China; }\\
\small{2. Modern College of Humanities and Sciences of Shanxi Normal University, Linfen 041004, China)}
%\small{\,China; 3. College of Science, Guilin University of Technology, Guilin 541004, China)}
}

%\linespread{2}
%===================================================
\begin{document}
\date{}
%%===============================================
\maketitle

\textbf{Abstract:} This gives two existence results of $\alpha$-core solutions by introducing $P-$open conditions and strong $P-$open conditions into games without ordered preferences. The existence of $\alpha$-core solutions is obtained for games with infinite-players. Secondly, it provides a short proof of Kajii's (Journal of Economic Theory 56, 194-205, 1992) existence theorem for $\alpha$-core solutions, further, the Kajii's theorem is equivalent to the Browder fixed point theorem. In addition, the obtained existence results can include many typical results for $\alpha$-core solutions and some recent existence results as special cases.

%\textbf{Keywords:} $\alpha$-core; coalitions; preferences; fixed points; infinite many players\\

%\textbf{JEL Classification:} C60, C72.

%%=========================================================
\section{Introduction}\label{sec1}
\noindent

The core is one of primary conceptions for cooperative games, which reflects the cooperative behaviors among players through establishing some coalitions. Roughly, the core is an allocation among players, which satisfies that no coalition can guarantee a higher payoff for every member in it than the current.

For $n$-person characteristic form games without side payments, $\alpha-$core and $\beta-$core were proposed by \cite{Aumann1961}. Firstly, the existence of $\alpha-$core solutions of $n$-person normal form games with utilities was proved by \cite{Scarf1971} (note that an $\alpha-$core solution, in the current paper, is adopted as a strategy to accomplish the allocation of the $\alpha-$core).  The existence theorem of Scarf was extended to $n$-person normal form games with general preferences by \cite{Kajii1992}. For $\beta-$core solutions of TU games, \cite{zhao1999} gave their existence and studied the equivalence between $\beta-$core solutions and $\alpha-$core solutions in oligopoly markets. As a generalization of $\beta-$core, \cite{crettez2022strong} showed the existence of strong $\beta-$hybrid solutions of $n-$person games very recently. The conception of hybrid solutions was introduced by \cite{Zhao1992} to handel Nash equilibrium points and $\alpha-$core solutions in a unified framework.

There are many results for core and $\alpha-$core in literatures, for examples, see \cite{Scarf1967,border1984core,Kajii1992,Zhao1992,liu2013necessary}. Their existence has a close relationship with basic fixed point theorems. The main results of the paper is no exception. The Scarf's theorem of the existence of core was related to an extension of the Sperner's lemma, see \cite{Scarf1967}.
The core of an economy was proved by \cite{yannelis1991core} using a new $L$-majorized correspondence, which is closely related to the Browder fixed point theorem. The theorem by \cite{Kajii1992} was based on a finding of Fan-Browder correspondence. The theorem by \cite{Zhao1992} was proved by constructing a Kakutani correspondence and analysing the balanced property of a general cooperative game.

It should be pointed out that many works for weak core and weak $\alpha-$core (for instance, see \cite{weber1981some,yang2017some}) can be found for games with infinitely many players. However, it is noted that there may no core solutions or $\alpha-$core solutions for a game with infinite-players by \cite{weber1981some}.

This short note intends to introduce two conditions to establish two existence results for $\alpha-$core solutions. One of the existence results is for the case of infinitely many players. Moreover, by the obtained existence results, this shows some short proofs of some typical existence results of $\alpha-$core solutions.

The next section recalls the conceptions of $\alpha-$core solutions for normal form games with utilities or general preferences. This section introduces a $P-$open condition to normal form games without ordered preferences. Section 3 establishes main results for the existence of $\alpha-$core solutions and gives a strong $P-$open condition. This section shows the simplified proofs of some typical theorems. The last concludes with some remarks.

\section{The definitions of $\alpha$-core solutions}
\noindent

Let $N$ be a set of players and $\mathcal{N}$ be all nonempty subsets of $N$. Note that $N$ may be finite or infinite. For each $i\in N$, in the whole paper, if there is no special statement, the player $i$'s strategy set $X_i$ is a compact convex subset of a Hausdorff topological vector space.  For each coalition $S\in \mathcal{N}$, let $X_S$ be the coalition's joint strategy set with $X_S=\prod_{i\in S}X_i$ and $X_{-S}$ be the set $\prod_{i\not\in S}X_i$. Furthermore, $X_N$ is also denoted as $X$ for convenience. In addition, for each $x\in X$ and each $S\in \mathcal{N}$, we can write $x$ as $x=(x_S,x_{-S})\in X$.

For each $i\in N$, $u_i:X\rightarrow \mathbb{R}$ is the player $i$'s payoff function. The correspondence $P_i:X\rightrightarrows X$ denotes the player $i$'s preference for each $i\in N$. Then, $G=(N, (X_i, u_i)_{i\in N})$ is a normal form game with payoffs, and $G_P=(N, (X_i, P_i)_{i\in N})$ is called a normal form game with general preferences.

The following definition of $\alpha-$core solutions for normal form game with general preferences is from the work by \cite{Kajii1992}. The definition is a generalization of $\alpha-$core solutions of games with payoffs.
\begin{definition}\label{def1}
Let $G_P=(N, (X_i, P_i)_{i\in N})$ be a normal form game with general preferences. A strategy $x\in X$ is $\alpha-$blocked by a coalition $S\in \mathcal{N}$ if there exists $y_S\in X_S$ such that $\{y_S\}\times X_{-S}\subset P_i(x)$, $\forall i\in S$. A joint strategy
$\bar{x}\in X$ of $N$ is called to be an $\alpha$-core solution of $G_P$ if it cannot be $\alpha-$blocked by any $S\in \mathcal{N}$.
\end{definition}

The following definition of $\alpha$-core solutions for normal form games with payoffs is based on the introduction by \cite{Aumann1961}, it is also known that its existence was proved firstly by \cite{Scarf1971}.
\begin{definition}\label{def3}
Let $G=(N, (X_i, u_i)_{i\in N})$ be a normal form game with payoffs. A strategy $x\in X$ is $\alpha-$blocked by a coalition $S\in \mathcal{N}$ if there exists $y_S\in X_S$ such that $u_i(y_S,z_{-S})>u_i(x)$, $\forall z_{-S}\in X_{-S}, \forall i\in S$.
A strategy $\bar{x}\in X$ of $N$ is called to be an $\alpha$-core solution of $G$ if it cannot be $\alpha-$blocked by any $S\in \mathcal{N}$.
\end{definition}

 Let $G=(N, (X_i, u_i)_{i\in N})$ and $G=(N, (X_i, P_i)_{i\in N})$ be two games. For each $x\in X$ and each $i\in I$, let $P_i(x)=\{y\in X: \,  u_i(y)>u_i(x)\}$. If $\bar{x}\in X$ is an $\alpha$-core solution of $G_P$, then $\bar{x}$ is an $\alpha$-core solution of $G$.

We introduce a condition into a normal form game with general preferences as the following definition \ref{def2}.
\begin{definition}\label{def2}
Let $G_P=(N, (X_i, P_i)_{i\in N})$ be a normal form game with general preferences. For any strategy $x\in X$ and any $S\in \mathcal{N}$, if $\{y_S\}\times X_{-S}\subset P_i(x)$ for a player $i\in S$ implies that there exists an open neighborhood $O(x)$ of $x$ such that $\{y_S\}\times X_{-S}\subset P_i(x'), \forall x'\in O(x)$, then we call $G_P$ is $P$-open.
\end{definition}

Clearly, for a normal form game $G_P=(N, (X_i, P_i)_{i\in N})$ with general preferences, if $P_i$ has an open graph for each $i\in N$, it holds that $G_P$ is $P$-open. Further, the $P$-open condition implies that $P_i$ has open lower sections for each $i\in N$. 

\section{The existence of $\alpha$-core solutions}
\noindent

By employing the $P-$open condition in Section 2, we can give an existence result for normal form game with finite players.
\begin{theorem}\label{thm1}
Let $N=\{1,2,\cdots,\mathrm{n}\}$ be a finite set of players, $G_P=(N, (X_i, P_i)_{i\in N})$ be a normal form game with general preferences. Assume that $G_P$ satisfies:

$(1)$ $G_P$ is $P$-open,

$(2)$ for any $x\in X$, $x\not\in co\{P_{j}(x): j\in N\}$, where $co\{A\}$ denotes the convex hull of the set $A$.

Then, there exists at least an $\alpha-$core solution of $G_P$.
\end{theorem}
\begin{proof}
For each $S\in \mathcal{N}$, let $T(S,x)=\{y=(y_S,y_{-S})\in X: \{y_S\}\times X_{-S}\subset P_{i}(x), \forall i\in S\}$ for each $x\in X$. Define a correspondence $T:X\rightrightarrows X$ such that
\begin{center}
$T(x)=co\{T(S,x): S\in \mathcal{N}\}$, for each $x\in X$.
\end{center}

By way of contradiction, if there is no $\alpha-$core solution of $G_P$, then for any $x\in X$, there exists $S\in \mathcal{N}$ such that $x$ can be $\alpha-$blocked by the coalition $S$. Then, it is clear that the correspondence $T$ has nonempty convex values.

For any $y\in X$, let $T^{-1}(y)=\{x\in X: y\in T(x)\}$. If $x\in T^{-1}(y)$, then there exists $S^j\in \mathcal{N}$ and $y^j=(y^j_{S^j},y^j_{-S^j})\in X$ for each $j$ in a set $M$ with $M=\{1,2,\cdots,m\}$ such that $y\in co\{y^j: j\in M\}$ and $\{y^j_{S^j}\}\times X_{-S^j}\subset P_{i}(x)$ for each $i\in S^j$ and each $j\in M$. For each $j\in M$, because $G_P$ is $P$-open and $S^j$ is finite, there is an open neighborhood $O(S^j,x)$ of $x$ with $\{y^j_{S^j}\}\times X_{-S^j}\subset P_{i}(x')$, $\forall x'\in O(S^j,x)$, $\forall i\in S^j$. Let $O(x)=\cap_{j\in M}O(S^j,x)$, then, $O(x)$ is an open neighborhood of $x$ satisfying that $y\in co\{y^j: j\in M\}$ and
\begin{center}
$\{y^j_{S^j}\}\times X_{-S^j}\subset P_{i}(x')$, $\forall x'\in O(x), \forall i\in S^j, \forall j\in M$.
\end{center}
That is, $y\in co\{T(S^j,x'): j\in M\}\subset T(x'), \forall x'\in O(x)$, hence, $O(x)\subset T^{-1}(y)$. Therefore, $T^{-1}(y)$ is an open set in $X$.

Thus, the defined correspondence $T$ has nonempty convex values and open lower sections, that is, $T$ satisfies all conditions of the Browder fixed point theorem. Then, there is a point $\bar{x}\in X$ such that $\bar{x}\in T(\bar{x})$. Then there exists $S^j\in \mathcal{N}$ and $x^j=(x^j_{S^j},x^j_{-S^j})\in X$ for each $j$ in a set $M'$ with $M'=\{1,2,\cdots,m'\}$ such that $\bar{x}\in co\{x^j: j\in M'\}$ and $\{x^j_{S^j}\}\times X_{-S^j}\subset P_{i}(\bar{x})$ for each $i\in S^j$ and $j\in M'$. For each $j\in M'$, it is clear that $(x^j_{S^j},x^j_{-S^j})\in P_{i^j}(\bar{x})$ for a fixed point $i^j\in S^j$. Write all these $i^j$ into a set $M''$. Consequently, $\bar{x}\in co\{P_{i^j}(\bar{x}): i^j\in M'', i^j\in N\}\subset co\{P_{k}(\bar{x}): k\in N\}$, a contradiction with the condition $(2)$. Therefore, the assumption that there is no $\alpha-$core solution of $G_P$ is false. The proof is completed.
\end{proof}

\begin{corollary}
Let $N=\{1,2,\cdots,\mathrm{n}\}$ and $G_P=(N, (X_i, P_i)_{i\in N})$ be a normal form game with general preferences. Assume that $G_P$ satisfies the conditions $(2)$ in Theorem \ref{thm1} and the condition that the preference correspondence $P_i$ has an open graph for each $i\in N$. Then, the set of $\alpha-$core solutions of $G_P$ is nonempty.
\end{corollary}
\begin{proof}
The result follows from, the fact that $P_i$ has open graph for each $i\in N$ implies that $G_P$ is $P$-open.
\end{proof}


For an example of Theorem \ref{thm1}, consider a normal form game $G'=(N, (X_i, u_i)_{i\in N})$, where $N=\{1,2,\cdots,\mathrm{n}\}$ and $u_i$ is continuous and quasi-concave on $X$. Let $G'_P=(N, (X_i, P_i)_{i\in N})$ with $P_i(x)=\{y\in X: u_i(y)>u_i(x)\}$ for each $i\in N$ and each $x\in X$. We will check that $G'_P$ satisfies all conditions in Theorem \ref{thm1}. For any strategy $x\in X$ and any $S\in \mathcal{N}$, for a player $i\in S\in \mathcal{N}$, assume that $\{y_S\}\times X_{-S}\subset P_i(x)$. Then, $u_i(y_S, z_{-S})>u_i(x), \forall z_{-S}\in X_{-S}$. Since $u_i$ is continuous on the compact $X$, there exists an open neighborhood $O(x)$ of $x$ such that $u_i(y_S, z_{-S})>u_i(x'), \forall x'\in O(x), \forall z_{-S}\in X_{-S}$. Hence, $\{y_S\}\times X_{-S}\subset P_i(x'), \forall x'\in O(x)$. Therefore, $G'_P$ is $P$-open.

The next is to check that $G'_P$ satisfies $(2)$ in Theorem \ref{thm1}. If $x\in co\{P_{i}(x): i\in N\}$, then, there exists $M=\{1,2,\cdots,m\}$ and $x^{j}\in P_{i^j}(x)$ for each $j\in M$, such that $x\in co\{x^{j}: j\in M\}$. Clearly, it holds that $u_{i^j}(x^j)>u_{i^j}(x), \forall j\in M$. Since $u_{i^j}$ is quasi-concave on $X$, we have that $u_{i^j}(x^{j})>u_{i^j}(x)\geq min\{u_{i^j}(x^{j}): j\in M\}$, $\forall j\in M$. Without loss of generality, let $u_{i^k}(x^{k})=min\{u_{i^j}(x^{j}): j\in M\}$ with $k\in M$, then, $u_{i^k}(x^{k})>u_{i^k}(x)\geq u_{i^k}(x^{k})$, a contradiction. Therefore, we have that $x\not\in co\{P_{i}(x): i\in N\}$.

Noting the fact that the properties of $G'$ and $G'_P$, we give a corollary of Theorem \ref{thm1} as the following which is the typical result by \cite{Scarf1971}.

\begin{corollary}\label{cor1}
 Let $G=(N, (X_i, u_i)_{i\in N})$ be a normal form game with $N=\{1,2,\cdots,\mathrm{n}\}$. If $u_i$ is continuous and quasi-concave on $X$.  Then, the set of $\alpha-$core solutions of $G$ is nonempty.
\end{corollary}


From the proof method in Theorem \ref{thm1}, and combining with pseudo-utility functions, we can obtain a short proof of the typical result of \cite{Kajii1992} for games without ordered preferences. See the following proof.
\begin{theorem}\label{thm2}
 Let $N=\{1,2,\cdots,\mathrm{n}\}$ be a finite set of players. If the normal form game $G_P=(N, (X_i, P_i)_{i\in N})$ satisfies:

$(1)$ for each $i\in N$, $X_i$ is a nonempty compact convex subset of a normed linear space,

$(2)$ for each $i\in N$, $x\not\in P_i(x)$ for each $x\in X$, $P_i$ has convex values, and the graph $GrP_i$ of $P_i$ is open in $X\times X$,

\noindent then, there exists an $\alpha-$core solution of $G_P$.
\end{theorem}
\begin{proof}
For each $S\in \mathcal{N}$ and each $x\in X$, define $T(S,x)$ and $T:X\rightrightarrows X$ being the same as those in the proof of Theorem \ref{thm1}.

Assume that there is no $\alpha-$core solution of $G_P$, then the correspondence $T$ has nonempty convex values. Since the condition that $P_i$ has an open graph for each $i\in N$ implies that $G_P$ is $P$-open, we have that $T^{-1}(y)$ is open in $X$ for each $y\in X$. Then, by the Browder fixed point theorem, there is a point $\bar{x}\in X$ for which $\bar{x}\in T(\bar{x})$.

To induce a contradiction, for each $i\in N$, define a pseudo-utility function $u_i:X\times X\rightarrow \mathbb{R}$ such that $u_i(x,y)=d((x,y), (GrP_i)^c)$ for each $(x,y)\in X\times X$, where $(GrP_i)^c$ is the complement of $GrP_i$ and $d$ is the metric induced by the norm. Obviously, for any $(x,y)\in X\times X$, $u_i(x,y)>0$ is equivalent to $(x,y)\in GrP_i$. In addition, due to the two conditions $GrP_i$ is open and $P_i$ has convex values for each $i\in N$, it is known that $u_i(x,\cdot)$ is quasi-concave on $X$ for each fixed $x\in X$.


Since $\bar{x}\in T(\bar{x})$, there exists a set $M$ with $M=\{1,2,\cdots,m\}$, $S^j\in \mathcal{N}$ and $x^j=(x^j_{S^j},x^j_{-S^j})\in X$ for each $j\in M$ such that $\bar{x}\in co\{x^j: j\in M\}$ and $\{x^j_{S^j}\}\times X_{-S^j}\subset P_{i}(\bar{x})$ for each $i\in S^j$ and $j\in M$. We can find that $x^j=(x^j_{S^j}, x^j_{-S^j})\in P_{i}(\bar{x})$ for each $i\in S^j$ and each $j\in M$. Then, for each $j\in M$, it holds that $u_{i}(\bar{x},x^j)>0$ for each $i\in S^j$. For a fixed $i\in S^j$, note that $\bar{x}\in co\{x^j: j\in M\}$ and $u_{i}(\bar{x},\cdot)$ is quasi-concave, we have that $u_{i}(\bar{x},\bar{x})\geq min\{u_{i}(\bar{x}, x^j): j\in M\}>0$, a contradiction with the condition $\bar{x}\not\in P_{i}(\bar{x})$. Therefore, the set $\alpha-$core solutions of $G_P$ is not empty.
\end{proof}

\begin{remark}\label{rem1}
The construction of the pseudo-utility functions is due to \cite{shafer1975equilibrium}. For the quasi-concave property of the pseudo-utility function $u_i(x,\cdot)$ for each $i\in N$ and a fixed $x\in X$, see \cite{border1984core}.
\end{remark}
\begin{remark}\label{rem2}
When $|N|=1$, an $\alpha-$core solution of a game $G_P$ is equivalent to a maximal element of the unique preference $P$. Then, it is easy to obtain the Browder fixed point theorem by the existence result of Kajii in Theorem \ref{thm2}. On the other hand, the proof of Theorem \ref{thm2} shows that the existence result of Kajii for $\alpha-$core solutions can be deduced from the Browder fixed point theorem. Therefore, in this sense, the existence thereom of $\alpha-$core solutions of Kajii is equivalent to the Browder fixed point theorem.
\end{remark}
\begin{remark}\label{rem3}
For the pseudo-utility function $u_i:X\times X\rightarrow \mathbb{R}$ in the proof of Theorem \ref{thm2}, define $P'_i(x)=\{y: u_i(x,y)>0\}$ for each $x\in X$. Obviously, it is true that $P'_i(x)=P_i(x)$ for each $x\in X$. Further, for the game $G_{P'}=(N, (X_i, P'_i)_{i\in N})$, it can be routinely checked that $G_{P'}$ satisfies that all conditions of Theorem \ref{thm1}. 
\end{remark}


To deal with those games with infinitely many players, this intends to introduce a definition of strong $P$-open condition into games with general preferences, which is similar to the $P$-open condition.

\begin{definition}\label{def5}
Let $G_P=(N, (X_i, P_i)_{i\in N})$ be a normal form game with general preferences. For any strategy $x\in X$ and any $S\in \mathcal{N}$, if $\{y_S\}\times X_{-S}\subset P_i(x)$ for each $i\in S$, then there exists an open neighborhood $O(x)$ of $x$ such that $\{y_S\}\times X_{-S}\subset P_i(x'), \forall x'\in O(x), \forall i\in S$. We call $G_P$ is strong $P$-open.
\end{definition}

It can be seen that, for a game $G_P=(N, (X_i, P_i)_{i\in N})$, if $G_P$ is strong $P$-open, then $G_P$ is $P$-open. This observation itself is irrespective of whether the player set $N$ is finite or infinite.

\begin{theorem}\label{thm3}
Let $G_P=(N, (X_i, P_i)_{i\in N})$ be a normal form game with general preferences. Assume that $G_P$ satisfies:

$(1)$ $G_P$ is strong $P$-open.

$(2)$ for any $x\in X$, $x\not\in co\{P_{j}(x): j\in N\}$,

Then, $G_P$ has at least one $\alpha-$core solution.
\end{theorem}
\begin{proof}
For each $S\in \mathcal{N}$ and each $x\in X$, let $T(S,x)$ and the correspondence $T$ be the same as those in the proof of Theorem \ref{thm1}. Following the proof of Theorem \ref{thm1}, assume that there is no $\alpha-$core solution of $G_P$, then it is sufficient to show that $T^{-1}(y)$ is open in $X$ for each $y\in X$.

For any $y\in X$, if $x\in T^{-1}(y)$, then there exists a finite $M=\{1,2,\cdots,m\}$, $S^j\in \mathcal{N}$ and $y^j=(y^j_{S^j},y^j_{-S^j})\in X$ for each $j\in M$ such that, $y\in co\{y^j: j\in M\}$ and $\{y^j_{S^j}\}\times X_{-S^j}\subset P_{i}(x)$ for each $i\in S^j$ and $j\in M$. For each $j\in M$, since $G_P$ is strong $P$-open, there exists an open neighborhood $O(S^j,x)$ of $x$ such that $\{y^j_{S^j}\}\times X_{-S^j}\subset P_{i}(x')$, $\forall x'\in O(x), \forall i\in S^j$. Let $O(x)=\cap_{j\in M}O(S^j,x)$, then, $O(x)\subset T^{-1}(y)$, consequently, $T^{-1}(y)$ is open.
\end{proof}


In the last, consider a normal form game $G''=(N, (X_i, u_i)_{i\in N})$ with infinitely many players, satisfying that $N$ is a compact subset of Hausdorff topological space. Let $u_i$ be continuous and quasi-concave on $X$ for each $i\in N$.

A strategy $\bar{x}\in X$ of $N$ is called to be a \textit{weak $\alpha$-core solution} of $G''$ if it is not true that there exists $\varepsilon>0$ and $y_S\in X_S$ such that $u_i(y_S,z_{-S})>u_i(x)+\varepsilon$, $\forall z_{-S}\in X_{-S}, \forall i\in S$.

Let $G''_P=(N, (X_i, P_i)_{i\in N})$ for which $P_i(x)=\{y\in X: \exists\, \varepsilon >0$\, s.t.\, $u_i(y)>u_i(x)+\varepsilon\}$ for each $i\in N$ and each $x\in X$. It will be verified that $G''_P$ satisfies all conditions in Theorem \ref{thm3}.

For any $x\in X$ and any $S\in \mathcal{N}$, if there exists $y_S\in X_S$ such that $\{y_S\}\times X_{-S}\subset P_i(x)$, $\forall i\in S$. Then, there is an $\varepsilon>0$ such that $u_i(y_S, z_{-S})>u_i(x)+\varepsilon, \forall z_{-S}\in X_{-S}, \forall i\in S$. Note that $u_i$ is continuous on the compact $X$, and the closure of $S$, $cl(S)$, is also compact, then, it holds that $min_{i\in cl(S)}\{min_{z_{-S}\in X_{-S}}u_i(y_S, z_{-S})-u_i(x)\}\geq \varepsilon>\varepsilon'>0$ and the left hand of the inequality is continuous on $X$. Therefore, there is an open neighborhood $O(x)$ of $x$ such that,
 \begin{center}
  $u_i(y_S, z_{-S})>u_i(x')+\varepsilon', \forall x'\in O(x), \forall z_{-S}\in X_{-S}, \forall i\in S$.
\end{center}
That is, $\{y_S\}\times X_{-S}\subset P_i(x'), \forall x'\in O(x), \forall i\in S$, hence, $G''_P$ is strong $P$-open.

We will show that $G''_P$ satisfies the condition $(2)$ in Theorem \ref{thm3}. If $x\in co\{P_{i}(x): i\in N\}$, then, there is a finite set $M=\{1,2,\cdots,m\}$ and $x^{j}\in P_{i^j}(x)$ for each $j\in M$, such that $x\in co\{x^{j}: j\in M\}$. We know that there exists $\varepsilon^j>0$ such that $u_{i^j}(x^j)>u_{i^j}(x)+\varepsilon^j, \forall j\in M$. Since $u_{i^j}$ is quasi-concave on $X$, we have that $u_{i^j}(x^{j})>u_{i^j}(x)+\varepsilon^j\geq min\{u_{i^j}(x^{j}): j\in M\}+\varepsilon^j$, $\forall j\in M$. Let $u_{i^k}(x^{k})=min\{u_{i^j}(x^{j}): j\in M\}$ with $k\in M$, then, we have that $u_{i^k}(x^{k})> u_{i^k}(x^{k})+\varepsilon^k> u_{i^k}(x^{k})$, a contradiction. Hence, $x\not\in co\{P_{i}(x): i\in N\}$.


Therefore, by Theorem \ref{thm3}, there exists an $\alpha$-core solution $\bar{x}\in X$ of $G''_P$. That is, for any $S\in \mathcal{N}$, there is no $y_S\in X_S$ such that $\{y_S\}\times X_{-S}\subset P_i(\bar{x}), \forall i\in S$. Then, for each $S\in \mathcal{N}$ and each $y_S\in X_S$, there are $z_{-S}\in X_{-S}$ and $i\in S$ such that $(y_S,z_{-S})\not\in P_i(\bar{x})$ (which means that there is no $\varepsilon>0$ such that $u_i(y_S,z_{-S})>u_i(\bar{x})+\varepsilon$). Then, we can assert that $\bar{x}$ is a weak $\alpha$-core solution of the norm form game $G''$ with infinitely many players.

From the above relationship between $G''$ and $G''_P$, we can get Corollary \ref{cor3} of Theorem \ref{thm3}, which is a very recent interesting result for weak $\alpha-$core solutions by Yang, see the theorem 3.2 in \cite{yang2017some}.

\begin{corollary}\label{cor3}
 Let $G=(N, (X_i, u_i)_{i\in N})$ be a normal form game with $N$ being a compact subset of Hasudorff topological space. If $u_i$ is continuous and quasi-concave on $X$ for each $i\in N$. Then, the set of weak $\alpha-$core solutions of $G$ is nonempty.
\end{corollary}

\section{Concluding Remarks}
\noindent

This gives two existence results of $\alpha$-core solutions for normal form games without ordered preferences. One is for games with finite players, the other is for games with infinitely many players. In our knowledge, the existence of $\alpha$-core solutions for games with infinitely many players is rarely seen in literatures.

Two conditions are employed to prove the existence of $\alpha$-core solutions. These are $P-$open and strong $P-$open games, which implies some semi-continuities of preference correspondences. The preference with its graph being open is a sufficient condition for the $P-$open condition.

From the proof of these existence results by fixed point theorems, short proofs of the Scarf's theorem for games with utilities and the Kajii's theorem for games without ordered preferences are given. In fact, this shows further a close relationship between $\alpha$-core solutions and fixed points--- the Kajii's theorem and the Browder fixed point theorem can be deduced by each other directly.

%\section{Statements and Declarations}
%The authors have no competing interests to declare that are relevant to the content of this article.
%\section*{Acknowledgements}

%\section*{References}
%\linespread{0.5}
%\bibliographystyle{plainnat}
%\bibliography{mybib}

\begin{thebibliography}{14}
\providecommand{\natexlab}[1]{#1}
\providecommand{\url}[1]{\texttt{#1}}
\expandafter\ifx\csname urlstyle\endcsname\relax
  \providecommand{\doi}[1]{doi: #1}\else
  \providecommand{\doi}{doi: \begingroup \urlstyle{rm}\Url}\fi

\bibitem[Aumann(1961)]{Aumann1961}
R.~J. Aumann.
\newblock The core of a cooperative game without side payments.
\newblock \emph{Transactions of the American Mathematical Society}, 98\penalty0
  (3):\penalty0 539--552, 1961.

\bibitem[Border(1984)]{border1984core}
K. C. Border.
\newblock A core existence theorem for games without ordered preferences.
\newblock \emph{Econometrica}, 52:\penalty0 1537--1542, 1984.


\bibitem[Crettez et~al.(2022)Crettez, Nessah, and
  Tazda{\footnotesize{\"{I}}}t]{crettez2022strong}
B. Crettez, R. Nessah, and T. Tazda{\footnotesize{\"{I}}}t.
\newblock On the strong $\beta$-hybrid solution of an $n$-person game.
\newblock \emph{Theory and Decision}, 2022.
\newblock \emph{doi:10.1007/s11238-022-09900-0.}

\bibitem[Kajii(1992)]{Kajii1992}
A.~Kajii.
\newblock A generalization of scarf's theorem: An $\alpha$-core existence
  theorem without transitivity or completeness.
\newblock \emph{Journal of Economic Theory}, 56\penalty0 (1):\penalty0
  194--205, 1992.


\bibitem[Liu and Liu(2013)]{liu2013necessary}
J. Liu and X. Liu.
\newblock A necessary and sufficient condition for an {NTU} fuzzy game to have
  a non-empty fuzzy core.
\newblock \emph{Journal of Mathematical Economics}, 49\penalty0 (2):\penalty0
  150--156, 2013.


\bibitem[Scarf(1967)]{Scarf1967}
H.~E. Scarf.
\newblock The core of an $n$-person game.
\newblock \emph{Econometrica}, 35\penalty0 (1):\penalty0 50--69, 1967.

\bibitem[Scarf(1971)]{Scarf1971}
H.~E. Scarf.
\newblock On the existence of a coopertive solution for a general class of
  $n$-person games.
\newblock \emph{Journal of Economic Theory}, 3\penalty0 (2):\penalty0 169--181,
  1971.


\bibitem[Shafer and Sonnenschein(1975)]{shafer1975equilibrium}
W. Shafer and H. Sonnenschein.
\newblock Equilibrium in abstract economies without ordered preferences.
\newblock \emph{Journal of Mathematical Economics}, 2\penalty0 (3):\penalty0
  345--348, 1975.


\bibitem[Weber(1981)]{weber1981some}
S. Weber.
\newblock Some results on the weak core of a non-side-payment game with
  infinitely many players.
\newblock \emph{Journal of Mathematical Economics}, 8\penalty0 (1):\penalty0
  101--111, 1981.


\bibitem[Yang(2017)]{yang2017some}
Z. Yang.
\newblock Some infinite-player generalizations of scarf's theorem:
  Finite-coalition $\alpha$-cores and weak $\alpha$-cores.
\newblock \emph{Journal of Mathematical Economics}, 73:\penalty0 81--85, 2017.


\bibitem[Yannelis(1991)]{yannelis1991core}
N. C. Yannelis.
\newblock The core of an economy without ordered preferences.
\newblock In \emph{Equilibrium theory in infinite dimensional spaces}, pages
  102--123. Springer, 1991.

\bibitem[Zhao(1992)]{Zhao1992}
J. Zhao.
\newblock The hybrid solutions of an $n$-person game.
\newblock \emph{Games and Economic Behavior}, 4\penalty0 (1):\penalty0
  145--160, 1992.


\bibitem[Zhao(1999)]{zhao1999}
J. Zhao.
\newblock A $\beta$-core existence result and its application to oligopoly
  markets.
\newblock \emph{Games and Economic Behavior}, 27:\penalty0 153--168, 1999.


\end{thebibliography}
%
\end{document}
