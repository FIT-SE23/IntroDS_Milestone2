% !TeX spellcheck = en_GB
%!TEX root = ../side-constrained.tex

\section{List of Symbols and Notation}

\ifarxiv\begin{longtable}{p{.39\textwidth}p{.57\textwidth}}\else\begin{longtable}{p{.41\textwidth}p{.54\textwidth}}\fi
	\textbf{Symbol}			& \textbf{Name/Description} \\\hline\endhead
	$L^2([a,b])$ 						& The set of $L^2$-integrable functions from $[a,b]$ to $\IR$ \\
	$L^2_+([a,b])$ 						& The set of $L^2$-integrable functions from $[a,b]$ to $\IR_{\geq 0}$ \\
	$\scalar{.}{.}$						& The scalar product. Specifically, for $f,g \in L^2([a,b])^d$ the scalar product is defined as $\scalar{f}{g} \coloneqq \sum_{j=1}^d\int_a^b f_j(\theta)g_j(\theta)\diff\theta$ \\
	$\mathcal{M}([a,b])$				& The set of measurable subsets of the interval $[a,b]$ \\
	$G = (V,E)$							& A directed graph with node set $V$ and edge set $E$ \\
	$\planningInterval \subseteq \IR_{\geq 0}$ 	& The \emph{planning horizon}, \ie the time interval during which particles may enter the network \\
	$I$									& The finite set of commodities \\
	$s_i, t_i$							& Source-/sink node of commodity $i \in I$ \\
	$\Pc_i$								& The set of feasible walks of commodity $i$ \\
	$r_i: \planningInterval \to \IR_{\geq 0} $ & The fixed \emph{network inflow rate} at which particles of commodity $i$ enter the network at $s_i$ \\
	$Q_i \geq 0$						& The fixed \emph{flow volume} of commodity $i$ \\
	$\Pc = \bigcup_{i \in I}\Pc_i$		& The set of feasible walks -- note that we assume that different commodities have disjoint sets of feasible walks $\Pc_i$ \\
	$B_p \geq 0$						& A given fixed upper bound on the walk inflow rate into walk $p \in \Pc$ \\
	$\Lambda(r) \subseteq L^2_+(\planningInterval)^\Pc$ 
										& The set of feasible walk inflows with fixed network inflow rates $r$ \\
	$\Lambda(Q) \subseteq L^2_+(\planningInterval)^\Pc$ 
										& The set of feasible walk inflows with fixed flow volume $Q$ \\
	$h \in \Lambda(r), h \in \Lambda(Q)$& A \emph{walk inflow} vector where $h_p: \planningInterval \to \IR_{\geq 0}$ describes the rate at which particles of commodity $i \in I$ enter walk $p \in \Pc_i$ \\
	$\mathcal{R}$						& A set containing a tuple $(p,j)$ for every walk $p \in \Pc$ and $j \in [\abs{p}]$ used to denote the $j$-th edge on $p$ \\
	$f^+ \in L^2_+([\tStart,\infty))^{\mathcal{R}}$ & \emph{Edge inflow rates}: $f^+_{e,j}(t)$ denotes the rate at which particle on walk $p$ enter its $j$-th edge at time $t$ \\
	$f^- \in L^2_+([\tStart,\infty))^{\mathcal{R}}$ & \emph{Edge outflow rates}: $f^-_{e,j}(t)$ denotes the rate at which particle on walk $p$ leave its $j$-th edge at time $t$ \\
	$f = (f^+,f^-)$						& \emph{Edge flow}: a flow described by its edge in- and outflow rates \\
	$\flowVolume[e](h,.): \IR_{\geq 0} \to \IR_{\geq 0}$			
										& \emph{(Edge) flow volume}: the volume of flow $x_e(h,\theta) \coloneqq \int_{0}^\theta f^+_e(h,\vartheta)\diff\vartheta -  \int_{0}^\theta f^-_e(h,\vartheta)\diff\vartheta$ on edge $e$ at time $\theta$ under the flow induced by the walk inflow $h$ \\
	$q_e(h,.): \IR_{\geq 0} \to \IR_{\geq 0}$						
										& \emph{Queue length}: The length $q_e(h,\theta) \coloneqq \int_{0}^\theta f^+_e(h,.)\diff\vartheta -  \int_{0}^{\theta+\tau_e} f^-_e(h,\vartheta)\diff\vartheta$ of the queue on edge $e$ at time $\theta$ under the flow induced by the walk inflow $h$ \\
	$D_e(h,.): \IR_{\geq 0} \to \IR_{\geq 0}$
										& \emph{Edge delay}: the delay $D_e(h,\theta)$ experience under the flow induced by $h$ by particles entering edge $e$ at time $\theta$ \\
	$D_p(h,.): \planningInterval \to \IR_{\geq 0}$
										& \emph{Walk delay}: the travel time $D_p(h,t)$ experienced under the flow induced by $h$ by particles entering walk $p$ at time $t$ \\
	$\Psi:  L^2_+([t_0,t_f])^\Pc \to  \hat{M}([t_0,t_f])^\Pc$ 
										& \emph{\effWalkDelay}: the effective walk delay $\Psi_p(h,t)$ experienced under the flow induced by $h$ by particles entering walk $p$ at time $t$ (comprising \eg travel time, early/late arrival penalties, energy costs, ...) \\
	$\truncated{\Psi}{M}: L^2_+([t_0,t_f])^\Pc \to  L^2([t_0,t_f])^\Pc$
										& \emph{truncated \effWalkDelay}: the \effWalkDelay{} operator $\Psi$ capped at $M \in \IR$ (see \Cref{def:truncatedPsi})\\
	$\tau_p^j(h,.): \planningInterval \to \IR_{\geq 0}$						
										& The \emph{arrival time} $\tau^j_p(h,t)$ of particles starting along walk $p$ at time $t$ at the $j$-th node of this walk under the flow induced by $h$ \\
	$c_e: \IR_{\geq 0} \to \IR_{\geq 0}$ & \emph{Edge capacity}: a function denoting the capacity $c_e(\theta)$ of edge $e$ at time $\theta$ \\
	$f_e(h,.): \IR_{\geq 0} \to \IR$	&  \emph{Edge load}: $f_e(h,\theta)$ denotes some measure of the flow induced by $h$ on edge $e$ at time $\theta$ \\
	$S \subseteq L^2_+(\planningInterval)^\Pc$ & \emph{\setS}: the set of all feasible walk inflows \\
	$A_p(h) \subseteq \Pc_i \times \mathcal{M}([t_0,t_f]) \times \IR_{\geq 0} \times \IR$
										& The set of \addmEpsDev s $(q,J,\varepsilon,\Delta)$ under  $h$ from walk $p$ \\
	$(q,J,\varepsilon,\Delta) \in A_p(h)$ & \emph{\addmEpsDev}: a tuple denoting an \addmEpsDev{} of particles in space from walk $p$ to walk $q$ and in time from $J$ to $J+\Delta$ \\
	$H_{p\to q}(h,J,\varepsilon,\Delta) \in L^2_+(\planningInterval)^\Pc$ 
										& The walk inflow obtained from $h$ by an \addmEpsDev{} $(q,J,\varepsilon,\Delta)$ \\
	$M_i(h) \subseteq L^2_+(\planningInterval)^\Pc$ 
										& The set of walk inflows which can be obtained by \addmEpsDev s of commodity $i$ from $h$ \\
	$U_p(h,t) \subseteq \Pc_i \times \IR$ & The set of \addmDev s $(q,\Delta)$ for particles entering walk $p$ at time $t$ under $h$ \\
	$(q,\Delta) \in U_p(h,t)$			& \emph{\addmDev}: a tuple denoting an \addmDev{} of shifting in space from $p$ to $q$ and in time from $t$ to $t + \Delta$ \\ 
\end{longtable}

\ifarxiv
\section{List of Dynamic Equilibrium Concepts}

In this paper we consider the following dynamic equilibrium concepts:
\begin{itemize}
	\item Dynamic equilibrium with fixed inflow rates: A walk inflow is an equilibrium if almost no particle can improve by switching to a different walk -- see \Cref{def:DE}.
	\item Dynamic equilibrium with fixed flow volume and departure choice: A walk inflow is an equilibrium if almost no particle can improve by switching to a different walk and/or departure time -- see \Cref{def:DE}.
	\item Dynamic equilibrium with elastic demands and departure choice: A walk inflow is an equilibrium if almost no particle can improve by switching to a different walk and/or departure time or by staying at home -- see \Cref{def:DE}.
	\item \SCDE[full] (\SCDE): Our general equilibrium concept: A walk inflow is an equilibrium if no particle has an \addmDev{} with strictly better \effWalkDelay -- see \Cref{def:DCE}.
	\item \globalSCDE[full] (\globalSCDE): Given any \setS{} $S \subseteq \Lambda(Q)$, the \SCDE{} where \addmEpsDev s are those \epsDev s that lead to another flow in $S$ -- see \Cref{def:strongCDE}.
	\item \globalEL[full] (\globalEL): The same as \globalSCDE{} but specifically for $S$ defined by edge-load constraints (\ie by \eqref{eq:FeasibilitySetforEdgeLoad}) -- see \Cref{def:TypesOfCDE}.
	\item \sCDEdf[full] (\sCDEdf): $S$ defined by edge load constraints, \addmEpsDev s are those where the resulting flow is feasible for all deviating particles at times where such particles enter an edge -- see \Cref{def:TypesOfCDE}.
	\item \wCDEdf[full] (\sCDEdf): $S$ defined by edge load constraints, \addmEpsDev s are those where the resulting flow is feasible for all deviating particles at all time where such particles travel on an edge -- see \Cref{def:TypesOfCDE}.
	\item \sCDEu[full] (\sCDEu): $S$ defined by edge load constraints, \addmEpsDev s are those where deviating particles only enter unsaturated edges -- see \Cref{def:TypesOfCDE}.
	\item \wCDEu[full] (\wCDEu): $S$ defined by edge load constraints, \addmEpsDev s are those where deviating particles only travel on unsaturated edges -- see \Cref{def:TypesOfCDE}.
	\item \sCDEuP[full] (\sCDEuP): $S$ defined by edge load constraints, \addmEpsDev s are defined as for \sCDEu{} except that edge-capacities may be tight on a common prefix of the current and the alternative walk if $\Delta=0$ -- see \Cref{def:TypesOfCDE}.
	\item \wCDEuP[full] (\wCDEuP): $S$ defined by edge load constraints, \addmEpsDev s are defined as for \wCDEu{} except that edge-capacities may be tight on a common prefix of the current and the alternative walk if $\Delta=0$ -- see \Cref{def:TypesOfCDE}.
\end{itemize}
\fi