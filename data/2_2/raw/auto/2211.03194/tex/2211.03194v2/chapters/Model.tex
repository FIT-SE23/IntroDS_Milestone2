%!TEX root = ../side-constrained.tex

	\section{Model without Side Constraints}
	
	We consider a single commuting period $[t_0, t_f]$ with $t_0>t_f$.
 The physical network model
is based on a finite directed graph $G=(V,E)$ with vertices $V$ and directed edges in  $V\times V$.
There is a finite set of commodities~$I=\{1,\dots,n\}$, each with a commodity-specific source node $s_i \in V$ and a commodity-specific sink node $t_i \in V$.  The (infinitesimally small) agents of every commodity $i\in I$ 
are represented by a volume of size $Q_i\in \R_+,i\in I$
which needs to be injected into the respective source within the time horizon $[t_0, t_f]$.
The  inflow rates are determined endogenously as they are the realization of a dynamic equilibrium
in which agents select a path and a departure time within $[t_0, t_f]$. 
We denote by $\Pc_i$ the set of all $s_i$,$t_i$-paths and assume that this set is always non-empty
for all $i\in I$.
	Let $\Pc=\cup_{i\in I}\Pc_i$ be the set of paths available to travelers
	and for each $p\in \Pc$ we denote the departure rate
	function as 
	\[ h_p:[t_0, t_f]\rightarrow\R_+.\]
	We assume throughout that $h_p\in L_+^2(t_0,t_f)$ for all $p\in \Pc$,
	where $L_+^2(t_0,t_f)\subset L^2(0,t_f)$ denotes the nonnegative functions in $L^2(0,t_f)$.
	We write $h\in \big(L_+(t_0,t_f)\big)^{\Pc}$ to denote the $|\Pc|$-fold product
	of the Hilbert subspaces.
	
	The inner product for $g,h\in \big(L_+(t_0,t_f)\big)^{\Pc}$ is defined as:
	\[ \langle g,h\rangle:=\int_{t_0}^{t_f} g(t)^\intercal h(t)\diff t=\sum_{p\in \Pc}\int_{t_0}^{t_f} g_p(t) h_p(t)\diff t.\]
	The associated $L_2$-norm is denoted by
	\[ \norm{u}:=\norm{u}_{L_2}=\langle u,u\rangle^{1/2}.\]
	
	We consider three different models corresponding
	to the case of fixed inflow rates, fixed flow volumes with departure choice and hence with
	variable inflow rates, and elastic volume case   with departure choice and the option
	to not travel at all leading to the elastic volume.
	
	Given bounded and nonnegative rate functions
	 \[ r_i(t)\geq 0, t\in [t_0,t_f] \text{ with } \int_{t_0}^{t_f} r_i(t)=Q_i,\]
	the space of feasible path flows for fixed inflow rates is given by
	\[ \Lambda(r):=\left\{h \in  \big(L_+(t_0,t_f)\big)^{\Pc}\middle\vert \sum_{p\in \Pc_i}h_p(t)= r_i(t)\text{ for all }t\in [t_0,t_f], i\in I\right\}.\]
	The space of feasible path flows for fixed flow volumes $Q$ is given by
	\[ \Lambda(Q):=\left\{h \in  \big(L_+(t_0,t_f)\big)^{\Pc}\middle\vert \sum_{p\in \Pc_i}\int_{t_0}^{t_f} h_p(t)
		\diff t= Q_i, i\in I, B_p\geq h_p(t), p\in \Pc, t\in [t_0,t_f] \right\},\]
		where $B_p\geq 0, p\in \Pc$ are bounds on the path inflow rates.
		The space of feasible tuples $(h,Q)$ in the model with elastic flow volumes
		is given as
		\[\tilde\Lambda:=\left\{(h,Q)\in \big(L_+(t_0,t_f)\big)^{\Pc} \times \R_+^{|I|}\middle\vert h\in \Lambda(Q)\right\}.\]
		For the model with elastic flow volumes, there is an inverse demand function $\Theta:\R_+^{I}\rightarrow\R_+^{I}$.
	To keep the exposition general, we do not present an explicit  network loading  for any $h$ but only require conditions on resulting path travel time functions given $h$.

	There is a path-delay function 
\[ D:\big(L_+(t_0,t_f)\big)^{\Pc}\rightarrow \big(L_+(t_0,t_f)\big)^{\Pc}, h(\cdot)\mapsto(D_p(\cdot,h)_{p\in \Pc}\]
where every $D_p(t,h), \in \Pc$ corresponds to the resulting travel time along $p$ when starting at time $t\geq 0$ given $h$.
With this function and given $h$, we can define the \emph{arrival time} of a particle that follows  $p$ starting at time $t\geq 0$ denoted by
\[\tau_p(t,h):=t+D_p(t,h).\]
The dynamic network loading for $h$ satisfies FIFO, if
\[ \tau_p(t,h)\leq \tau_p(t',h) \text{ for all }0\leq t\leq t', p\in  \Pc.\]
	There is an arrival penalty function $\F:[t_0, t_f]\rightarrow\R_+$
	and the term $\F(t+D_p(t,h)-T_A)$ represents a nonnegative
	penalty for deviating from the target arrival time $T_A$.
	The most crucial function is the so-called effective path-delay operator	
	which maps $h$ to the effective path cost of an agent.
\[ \Psi:\big(L_+(t_0,t_f)\big)^{\Pc}\rightarrow \big(L_+(t_0,t_f)\big)^{\Pc}, h(\cdot)\mapsto(\Psi_p(\cdot,h)_{p\in \Pc},\]	
	where 
	\[\Psi_p(\cdot,h):=D_p(t,h)+\F(t+D_p(t,h)-T_A) \text{ for all }t\in [t_0, t_f], p\in  \Pc.\]
	
	\begin{defn}\label{de:fixed}
	\begin{enumerate}
\item 	$h^*\in \Lambda(r)$ is a dynamic equilibrium with fixed inflow rates, if for all $i\in I$,the following conditions hold:
	\begin{align}\label{eq:de-rate}
	h^*_p(t)&>0, p\in \Pc_i \Rightarrow \Psi_p(t,h^*)\leq \Psi_q(t,h^*)\text{ for almost all }t\in [t_0,t_f], q\in \Pc_i .	\end{align}
\item 	$(h^*)\in \Lambda(Q)$ is a dynamic equilibrium with fixed flow volumes and departure choice, if for all $i\in I$, there are $\nu_i\geq 0$ such that the following conditions hold:
	\begin{align}\label{eq:de-volume}
	h^*_p(t)&>0, p\in \Pc_i \Rightarrow \Psi_p(t,h^*)=\nu_i \text{ for almost all }t\in [t_0,t_f]\\
	\Psi_p(t,h^*)&\geq \nu_i \text{ for almost all }t\in [t_0,t_f], p\in \Pc_i.
	\end{align}
	\item $(h^*,Q^*)\in \tilde\Lambda$ is a dynamic equilibrium with elastic demands and departure choice, if for all $i\in I$, the following conditions hold:
	\begin{align}\label{de:elastic}
	h^*_p(t)&>0, p\in \Pc_i \Rightarrow \Psi_p(t,h^*)=\Theta_i(Q^*) \text{ for almost all }t\in [t_0,t_f]\\
	\Psi_p(t,h^*)&\geq \Theta_i(Q^*)\text{ for almost all }t\in [t_0,t_f], p\in \Pc_i.
	\end{align}
	\end{enumerate}
	\end{defn}
	
	\begin{theorem}\label{thm:zhu-vi-fixed}
	Let $\Psi_p(\cdot,h)$ be positive and measurable for all $p\in \Pc$ and  $h\in \Lambda(r)$. Then, $h^*\in \tilde\Lambda$ is a dynamic equilibrium with fixed inflow rates $r$, if and only if it solves the following variational inequality:
	\begin{equation}\label{eqn:VI-fixed}\tag{\ensuremath{\VI(\Psi,r,[t_o,t_f])}}
	\begin{aligned}
	\text{Find }h^* \in  \Lambda(r)  \text{ such that}:&\\
	\scalar{\Psi(h^*)}{(h-h^*)} &\geq 0 \text{ for all } h\in \Lambda(r).\end{aligned}
\end{equation}
\end{theorem}
\begin{proof}
Necessity:  Let $h \in  \Lambda(r)$ be arbitrary.
We obtain:
\begin{align*}
\sum_{p\in \Pc} \int_{t_0}^{t_f}\Psi_p(t,h^*)h^*_p(t)\diff t & =\sum_{i\in I}\sum_{p\in \Pc_i} \int_{t_0}^{t_f}\Psi_p(t,h^*) h^*_p(t)\diff t \\
& =\sum_{i\in I}\sum_{p\in \Pc_i}  \int_{t_0}^{t_f} \min_{w\in \Pc}\{\Psi_w(t,h^*)\}h^*_p(t)\diff t \\
& = \sum_{i\in I}\int_{t_0}^{t_f}  \min_{w\in \Pc}\{\Psi_w(t,h^*)\}r_i(t)\diff t \\
& = \sum_{i\in I}\sum_{p\in \Pc_i}  \int_{t_0}^{t_f} \min_{w\in \Pc}\{\Psi_w(t,h^*)\}h_p(t)\diff t \\
& \leq \sum_{i\in I}\sum_{p\in \Pc_i}  \int_{t_0}^{t_f} \Psi_p(t,h^*) h_p(t)\diff t.
\end{align*}
\end{proof}

	\begin{theorem}[Friesz et al.~\cite{Friesz:1993}, Thm. 2]\label{thm:friesz-vi-fixed}
	Let $\Psi_p(\cdot,h)$ be positive and measurable for all $p\in \Pc$ and  $h\in \Lambda(Q)$. Then, $h^*\in \tilde\Lambda$ is a dynamic equilibrium with fixed volumes and departure choice, if and only if it solves the following variational inequality:
	\begin{equation}\label{eqn:VI-Q}\tag{\ensuremath{\VI(\Psi,Q,[t_o,t_f])}}
	\begin{aligned}
	\text{Find }h^* \in  \Lambda(Q)  \text{ such that}:&\\
	\scalar{\Psi(h^*)}{(h-h^*)} &\geq 0 \text{ for all } h\in \Lambda(Q).\end{aligned}
\end{equation}
\end{theorem}



		Friesz et al.~\cite{HanFSH15} gave the following characterization of such dynamic equilibria via
	solutions of an associated variational inequality.
	
	
	\begin{theorem}[Friesz et al.~\cite{HanFSH15}, Thm.~3.2]\label{thm:friesz-vi}
	Let $\Psi_p(\cdot,h)$ be positive and measurable for all $p\in \Pc$ and all $h$ with $(h,Q)\in \tilde\Lambda$. Then, $(h^*,Q^*)\in \tilde\Lambda$ is a dynamic equilibrium with elastic demands and departure choice, if and only if it solves the following variational inequality:
	\begin{equation}\label{eqn:VI}\tag{\ensuremath{\VI(\Psi,\Theta,[t_o,t_f])}}
	\begin{aligned}
	\text{Find }(h^*,Q^*) \in  \tilde\Lambda  \text{ such that}:&\\
	\scalar{(\Psi(h^*),-\Theta(Q^*))}{((h,Q)-(h^*,Q^*))} &\geq 0 \text{ for all } (h,Q)\in  \tilde\Lambda.\end{aligned}
\end{equation}
\end{theorem}
Note that we use the extension of the inner product  $\scalar{\cdot}{\cdot}$
within  the extended space  $\big(L_+(t_0,t_f)\big)^{\Pc} \times \R_+^{|I|}$
in the natural way, that is, 
	\[\scalar{(\Psi(h^*),-\Theta(Q^*))}{((h,Q)-(h^*,Q^*))}=\int_{t_0}^{t_f}\Psi_p(t,h^*)(h_p(t)-h^*_p(t))\diff t-
	\sum_{i\in I} \Theta_i(Q^*)(Q_i-Q^*_i).\]
	

	
	Friesz et al.~\cite{HanFSH15} imposed the following conditions on relevant functions to prove existence of equilibria.
		\begin{assumption}
	\begin{itemize}
\item[(A1)] 	The arrival penalty function $\F:[t_0, t_f]\rightarrow\R_+$ is continuous and satisfies
\begin{equation}
\F(t_2)-\F(t_1)\geq \Delta (t_2-t_1) \text{ for all }t_0\leq t_1<t_2\leq t_f \text{
for some $\Delta> -1$.}
\end{equation}
\item[(A2)] The FIFO rule is obeyed for all $(h,Q)\in \tilde\Lambda$.
\item[(A3)] For any sequence $h^n(\cdot)$ that converges weakly to $h^*\in \big(L_+(t_0,t_f)\big)^{\Pc}$, the corresponding effective path delays $\Psi_p(\cdot,h^n)$ converge uniformly to $\Psi_p(\cdot,h^*)$ for all $p\in\Pc$.
\item[(A4)] The inverse demand function $\Theta:\R_+^{I}\rightarrow\R_+^{I}$ is continuous.
\end{itemize}
	\end{assumption}
	
	\begin{theorem}[Zhu and Marcotte~\cite{ZhuM00} ]\label{thm:zhu-ex-fixed}
	Under assumptions $(A2)-(A3)$, there exists a dynamic equilibrium $h^*\in \Lambda(r)$ with fixed inflow rates.
	\end{theorem}\tobias{Check!}
	
	\begin{theorem}[Friesz et al.~\cite{Han2013}, Thm. 5.6.]\label{thm:friesz-Q}
	Under assumptions $(A1)-(A3)$, there exists a dynamic equilibrium $h^*\in \Lambda(Q)$ with fixed flow volume $Q$ and departure choice.
	\end{theorem}


	\begin{theorem}[Friesz et al.~\cite{HanFSH15}, Thm.~4.5]\label{thm:friesz-E}
	Under assumptions $(A1)-(A4)$, there exists a dynamic equilibrium $(h^*,Q^*)\in \tilde\Lambda$ with elastic demands and departure choice.
	\end{theorem}
	\section{Models with Side Constraints}
	\begin{example}
Consider the link-loading model of Zhong et al.~\cite{zhong11} with fixed inflow rates $r_i(t), i\in I$
and $f_e(t,h), e\in E$ representing the flow volume. Further assume $c_e(t), e\in E$ to be constant.
Then, the set  $\bar \Lambda(Q)$ can be non-convex, and, thus Browder' Theorem is not
directly applicable for the proof of existence
of dynamic equilibria.
\end{example}
Zhong et al.~\cite{zhong11} claimed that, similar to Friesz et al.~\cite{Friesz:1993}, the dynamic equilibrium can be reformulated as
\begin{equation}\label{eqn:VI-wrong}\tag{\ensuremath{\VI(\Psi,\Theta,c,[t_o,t_f])}}
	\begin{aligned}
	\text{Find }h^*\in  \bar\Lambda  \text{ such that}:&\\
	\scalar{\Psi(h^*)}{h-h^*} &\geq 0 \text{ for all } h \in  \bar \Lambda(Q).
\end{aligned}
\end{equation}

\iffalse	
\begin{figure}[h]
	\centering
	\includegraphics[width=.8\textwidth]{Images/CounterExample.png}
	\caption{A counter example.}\label{fig:CounterExample}
\end{figure}
\fi
\begin{example}
We give a counter example to this claim: Consider the network given in \cref{fig:CounterExample}. We use linear edge delays of the form $D_e(x) = \tau_e + \frac{1}{\nu_e}x$ dependent on the current volume $x$ on edge $e$. We have a total volume of $4$ of particles that want to travel from $s$ to $t$. Additionally, we have capacity constraints on the edges $e_1$ and $e_2$ (as indicated by the graphs in \cref{fig:CounterExample}). 

We claim that the only two feasible path inflows (up to changes on a set of measure zero) are either sending all flow at a rate of $4$ into the path $e_1,e_2,e_4$ or all flow at a rate of $4$ into the path $e_1,e_3,e_4$. Furthermore, neither of those is a solution to the variational inequality \eqref{eqn:VI-wrong}.

Due to the capacity constraint on edge $e_1$ the only possible way of sending a volume of $4$ over this edge is to send flow into this edge at a rate of $4$ over the interval $[0,1]$
%\lukas{Check, whether this is actually true!}. 

This flow will then arrive at a rate of $2$ over the interval $[1,2]$ at node $u$ as one can verify by a straightforward calculation (cf. \cite{CareyMcCartney})
%\lukas{Do these calculations!}. 

If we then send all this flow either into edge $e_2$ or edge $e_3$, this flow will start arriving at node $v$ at a rate $1$ and, thus, can enter edge $e_4$ without violating its capacity constraint. However, if there is any split between these two edge then flow will arrive at node $v$ at a higher rate then $1$ and therefore violate the capacity constraint of edge $e_4$. 

This shows that the only possible path inflows are indeed sending all the flow into either of the two paths. At the same time neither of these will satisfy the variational inequality as the unused path will have a shorter delay. 

Note, that this example also shows why we need the ``$\forall\vol$'' in \eqref{eq:free-cap} as otherwise the above example would not have an equilibrium (since then the unused path would be considered a free capacity-path as we can shift the whole flow over and still achieve a feasible flow).
\end{example}


	
	\section{Formulation and Characterization of Constrained Dynamic Equilibria}
	Let us consider a constraint set $S\subseteq \big(L_+(t_0,t_f)\big)^{\Pc}$
	that restricts the space of possible equilibria.
	Clearly, edge-based volume or in- or outflow restrictions as considered
	in Zhong et al.~\cite{zhong11} appear as a special case such a  set.
	To define an appropriate notion of a \emph{constrained} dynamic equilibrium
	flow let us conduct the following thought experiment.
	Fix some feasible flow $h$, that is, $h\in S$.
	The idea of restricting a flow $h$ by some set $S$ should imply
	that at the boundary of $S$ (assuming for the moment that $S$ is closed),
	it should be impossible to increase $h$ in a direction leaving $S$.
	This reasoning implies the following:
	If an agent of some commodity changes its path, then, this
	new path should have some \emph{excess capacity}
	in the sense that must be some $\varepsilon>0$ such that
	increasing the inflow on the new paths should result
	in a new feasible flow $h'\in S$. On the other hand, this $\varepsilon>0$
	 increase should only be appearing on \emph{new} edges, that is, 
	on those edges which were not used by the agent
	before. This reasoning leads  to the notion of \emph{saturated} and  \emph{unsaturated} paths as used in the static Wardropian model by Larsson and Patriksson~\cite{Larsson95}.
	\subsection{Fixed Inflow Rates}
	We first put this approach formally at work for the case of fixed inflow rates $r$ and define for any given  $h\in S$, commodity $i$, paths $p, q \in \Pc_i$, subset $J \subseteq [t_0,t_f]$ and constant $\varepsilon>0$ the flow obtained by shifting flow of commodity $i$ from path $p$ to path $q$ at a rate of $\varepsilon$ during $J$ by 
	\begin{equation}\label{eq:H-r}
		H_{i,p\to q}^r(h,J,\varepsilon) \coloneqq (h'_w)_{w \in \Pc} \text{ with } \begin{aligned} 
			h'_p &=h_p-\varepsilon \CharF[J]\\
			h'_q &=h_q+ \CharF[J]\\
			h'_{w} &= h_{w} \text{ f.a. } w \in \Pc \setminus \{q,p\}
		\end{aligned},
	\end{equation}
where 
\[\CharF[J]: [t_0,t_f] \to \R, \theta \mapsto \begin{cases}
	1, &\text{ if } \theta \in J \\
	0, &\text{ else }
\end{cases}\]
is the indicator function of the set $J$.
Furthermore we define for every subset $J \subseteq [t_0,t_f]$ its diameter by $\mathrm{diam}(J) \coloneqq \sup\set{b-a | a,b \in J}$. Using this notation, we can now define the set of unsaturated alternatives to some fixed path $p$ of commodity $i$ with respect to some $h \in S^r$ at some time $t \geq 0$ as\lukas{We have two objects named $D$: The path delay operator and the set of unsaturated alternatives}
\begin{align}\label{eq:feasible-deviation} 
	D_{i,p}^r(h, t) 
		&\coloneqq \Set{
			q\in \Pc_i | \begin{array}{l}
				\exists J_n \subseteq [t_0,t_f], \varepsilon_n > 0: \lim_n \mathrm{diam}(J_n) = 0, 
				%\lim_n \varepsilon_n = 0 \text{ and} 
				\\
				\forall n: t \in J_n, J_n \text{ has positive measure and } H_{i,p\to q}^r(h,J_n,\varepsilon_n) \in S^r 
			\end{array}
		} \\
		&= \Set{
			q\in \Pc_i | \begin{array}{l}
				\forall \delta > 0: \exists J \subseteq [t-\delta,t+\delta], \varepsilon > 0: 
				\\
				J \text{ has positive measure and } H_{i,p\to q}^r(h,J,\varepsilon) \in S^r 
			\end{array}
		}.
\end{align}
In words:\lukas{Are these two definitions really equivalent?} A path $q$ is an unsaturated alternative to $p$ at time $t$ if we can shift small amounts\lukas{Should we also require $\lim_n\varepsilon_n = 0$? Then, we also need \Cref{ass:closedSpace} in \Cref{thm:VI-fixed-inflow:nessecary}} of flow during small neighbourhoods of $t$ from $p$ to $q$ and still have a feasible flow. Note, that if $t$ is has some neighbourhood without any inflow into $p$ during this time, then the set of unsaturated alternatives for $p$ at time $t$ is automatically empty (regardless of whether any other path $q$ has any remaining capacity). This is, because in this case we cannot shift any flow from away from $p$ without making the inflow rate into $p$ negative which is not allowed as $S^r$ only contains nonnegative functions.

We can now formally define the concept of a dynamic equilibrium in our model.

\begin{defn}\label{def:DCE}
	Given a network $\network=(G,\nu,\tau)$, a set of commodities $I$, a restriction set $S$ and for every commodity $i \in I$ an associated source-sink pair $(s_i,t_i) \in V \times V$, a feasible flow $h^*\in S^r$ is a \emph{capacitated dynamic equilibrium}, if for all $i\in I, p \in \Pc_i$ and almost all $t \in [t_0,t_f]$ it holds that
	\begin{equation}\label{eq:CDE} 
		\Psi_p(t,h^*)\leq  \Psi_q(t,h^*) \text{ for all }q\in D_{i,p}^r(h^*, t).
	\end{equation}
\end{defn}

The following lemma gives an alternative (negative) definition of capacitated dynamic equilibria:\lukas{Rein intuitiv scheint mir das ebenfalls eine sinnvolle Definition von CDE zu sein? Sind die beiden Definitionen wirklich \"aquivalent? Wenn ja, w\"are dieses Lemma ein ERsatz f\"ur \Cref{claim:hStarNoCDE}}
\begin{lemma}
	A feasible flow $h \in S^r$ is \emph{not} a capacitated dynamic equilibrium if and only if there exists a commodity $i \in I$, two paths $p,q \in \Pc_i$ and a set $J \subseteq [t_0,t_f]$ of positive measure such that for every time $t \in J$ there exist sequences $J_n \subseteq [t_0,t_f]$ and $\varepsilon_n > 0$ satisfying
		\begin{itemize}
			\item $\lim_n \mathrm{diam}(J_n) = 0$,
			\item $\forall n \in \INo: t \in J_n$, $J_n$ has positive measure and $H_{i,p \to q}(h,J_n,\varepsilon_n) \in S^r$ and
			\item $\forall n \in \INo: \forall \theta \in J_n: h_p(\theta) > 0$ and  $\Psi_p(h,\theta) > \Psi_q(h,\theta)$.
		\end{itemize}
\end{lemma}

\begin{proof}
	First, assume that the lemma's conditions hold. Then we clearly have $q \in D^r_{i,p}(h,t)$ for every $t \in J$ and, thus, at each such time \eqref{def:DCE} is violated. Since $J$ has positive measure, this implies that $h$ is not a capacitated dynamic equilibrium.
	
	Now, assume that $h$ is not a capacitated dynamic equilibrium. Since the set of commodities as well as the set of paths are both finite there must be some commodity $i$, two paths $p,q \in \Pc_i$ and a subset $J \subseteq [t_0,t_f]$ of positive measure such that for every $t \in J$ we have $h_p(t) > 0$, $\Psi_p(t,h) > \Psi_q(t,h)$ and $q \in D^r_{i,p}(h,t)$. The latter implies that for every $t \in J$ there exist sequences $J_n(t) \subseteq [t_0,t_f]$ and $\varepsilon_n(t) > 0$ satisfying
	\begin{itemize}
		\item $\forall t \in J: \lim_n\mathrm{diam}(J_n(t)) = 0$ and 
		\item $\forall t \in J, n \in \INs: t \in J_n(t), J_n(t)$ has positive measure and $H^r_{i,p \to q}(h,J_n(t),\varepsilon_n(t)) \in  S^r$.
	\end{itemize}
	Furthermore, since $\Psi_p(h,.)$ and $\Psi_q(h,.)$ are continuous, we can assume wlog that for every $t \in J, n \in \INs, \theta \in J_n(t)$ we have $\Psi_p(h,\theta) > \Psi_q(h,\theta)$. 
	
	\begin{claim}
		The set
			\[J' \coloneqq \Set{t \in J | J'_n(t) \text{ has positive measure for infinitely many } n \in \INs}.\]
		with $J'_n(t) \coloneqq J_n(t) \cap J$ has positive measure.
	\end{claim}

	\begin{proofClaim}
		\lukasI{The claim is equivalent to showing that the set
			\[\hat{J} \coloneqq \Set{t \in J | \exists N \in \INs: \forall n \geq N: J'_n(t) \text{ has measure } 0}\]
			has measure strictly smaller than $J$ -- maybe even measure zero?
		
			Can we use \url{https://math.stackexchange.com/q/3703609} tp construct a counter example to this claim? If $J=A$ and the $J_n(t)$ are completely contained in $B$...}		
	\end{proofClaim}

	With this claim we can now show that $J'$ fulfils the lemmas conditions. By the claim itself $J'$ has positive measure. Furthermore, for any $t \in J$ we define $J_n \coloneqq \set{\theta \in J_{k_n}(t) | h_p(\theta) > 0}$ and $\varepsilon_n \coloneqq \varepsilon_{k_n}(t)$, where $k_n$ is an subsequence of $1, 2, 3, \dots$ such that for any $n$ the set $J'_n(t)$ has positive measure (such a subsequence exists by the definition of $J'$). Then we have
	\begin{itemize}
		\item $\lim_n \mathrm{diam}(J_n) \leq \lim_n \mathrm{diam}(J_{k_n}(t)) = \lim_n \mathrm{diam}(J_n(t)) = 0$,
		\item $t \in J_n$ since $t \in J_n(t) \cap J$,
		\item $J_n$ has positive measure since $J_n \supseteq J'_n(t)$,
		\item $H^r_{i,p\to q}(h,J_n,\varepsilon_n) = H^r_{i,p \to q}(h,J_{k_n}(t),\varepsilon_n) \in S^r$ since $h_p(\theta) = 0$ for all $\theta \in J_{k_n}(t) \setminus J_n$,
		\item $h_p(\theta) > 0$ for all $\theta \in J_n$ by the definition of $J_n$ and
		\item $\Psi_p(h,\theta) > \Psi_q(h,\theta)$ for all $\theta \in J_n$ since $J_n \subseteq J_{k_n}(t)$.
	\end{itemize}
	Thus, all conditions of the lemma are fulfilled.
\end{proof}

\begin{cor}
	A feasible flow $h \in S^r$ satisfying \Cref{ass:closedTime} is \emph{not} a capacitated dynamic equilibrium if and only if there exists a commodity $i \in I$, two paths $p,q \in \Pc_i$, a constant $\varepsilon > 0$ and a set $J \subseteq [t_0,t_f]$ such that
	\begin{itemize}
		\item $J$ has positive measure and $H_{i,p \to q}(h,J,\varepsilon) \in S^r$ and
		\item $\forall \theta \in J: h_p(\theta) > 0$ and  $\Psi_p(h,\theta) > \Psi_q(h,\theta)$.
	\end{itemize}
\end{cor}

\begin{proof}
	Clearly, the condition of the lemma is stronger than the one of the corollary. Therefor, we only have to show that the corollary's conditions implies the lemma's. So, assume that the corollaries condition holds. We now define for any $k \in \INs$ the set 
		\[\hat{J}_k \coloneqq \Set{ t \in J | B_{\nicefrac{1}{k}}(t) \cap J \text{ has positive measure}},\]
	where $B_{\nicefrac{1}{k}}(t)$ denotes the open ball of radius $\nicefrac{1}{k}$ around $t$. We claim that each $\hat{J}_k$ has measure zero.\lukas{Ist $\hat{J}_k$ \"uberhaupt messbar?} If not, there would exists a compact subset $K \subseteq \hat{J}_k$ of positive measure. But then $\bigcup_{t \in \hat{J}_k} B_{\nicefrac{1}{k}(t)} \supseteq K$ must have a finite subcover, i.e. there must be some finite subset $J' \subseteq \hat{J}_k$ with $\bigcup_{t \in J'} B_{\nicefrac{1}{k}(t)} \supseteq K$. But then there must be some $t \in J'$ such that $B_{\nicefrac{1}{k}}(t)\cap K$ has positive measure -- a contradiction to $t \in \hat{J}_k$. 
	
	Thus, the set $\hat{J} \coloneqq \bigcup_{k \in \INs} \hat{J}_k$ has measure zero as well. Therefore, $\tilde{J} \coloneqq J \setminus \hat{J}$ has positive measure and for every $t \in \tilde{J}$ the sequences $J_n \coloneqq J \cap B_{\nicefrac{1}{n}}(t)$ and $\varepsilon_n \coloneqq \varepsilon$ satisfies
	\begin{itemize}
		\item $\lim_n \mathrm{diam}(J_n) \leq \lim_n \frac{2}{n} = 0$,
		\item $\forall n \in \INo: t \in J_n$, $J_n$ has positive measure and -- by \Cref{ass:closedTime} -- $H_{i,p \to q}(h,J_n,\varepsilon_n) \in S^r$ and
		\item $\forall n \in \INo: \forall \theta \in J_n: h_p(\theta) > 0$ and  $\Psi_p(h,\theta) > \Psi_q(h,\theta)$ since $J_n \subseteq J$.
	\end{itemize}
\end{proof}




\subsection{Fixed Volumes and Departure Time Choice}
	
		
	For the case of fixed volumes and departure time choice,
	the strategy of agent when deviating from the current strategy involves both an alternative path and an alternative  departure
	choice. In this regard, we need to extend the mapping $H^r_{i,p\to q}$ and hence the set $D_{i,p}^r(h^*, t)$.
	%\tobias{The definition of $H^Q$ is more intricate here!}
	\iffalse
	For $h\in S, \bar t,\hat t \in [t_0,t_f], \varepsilon>0, \delta>0$ and $J \subseteq [t_0,t_f]$ the
	mapping
\begin{align*}
	H_{i,p\to q}^Q(h,\bar t,\hat t,J,\varepsilon,\delta) \coloneqq (h'_w)_{w \in \Pc} \text{ with } \begin{aligned} 
		h'_p &=[h_p-\varepsilon \CharF[{[\bar t,\bar t+\delta]}]\cap J]_+\\
		h'_q &=h_q+h_p-[h_p-\varepsilon \CharF[{[\hat t,\hat t+\delta]}]\cap J]_+\\
		h'_{w} &= h_{w} \text{ f.a. } w \in \Pc \setminus \{q,p\}\end{aligned},
\end{align*}

For some $S\subset \big(L_+(t_0,t_f)\big)^{\Pc},$ let us define the feasible set for 
	the case of fixed volumes and departure time choice as 
\begin{align}
	S^Q:=\{h\in \big(L_+(t_0,t_f)\big)^{\Pc}\vert h\in \Lambda(Q)\cap S\}.
\end{align}
		We get similarly:
\begin{equation}\label{eq:feasible-deviation-volume} 
	D_{i,p}^Q(h, \bar t):=\left\{
		q\in \Pc_i\middle\vert  \forall \varepsilon' > 0: \exists \varepsilon \in (0,\varepsilon'], \delta>0: 
		H_{i,p\to q}^Q(h,\bar t,\hat t,\varepsilon,\delta) \in S^Q
	\right\}.
\end{equation}



\begin{equation}\label{eq:free-cap}
	U^Q(t,h,p)=\{q \in \Pc\vert \forall \vol>0\; \exists \;\varepsilon, \delta>0, \theta'\geq 0 \text{ with }\varepsilon\cdot \delta\leq \vol \text{ such that } 
	H^Q_{p\rightarrow q}(h,\theta,\theta',\varepsilon, \delta)\in S^Q\}.
	\end{equation}

\begin{defn}\label{def:eq-Q}
$h^*\in  S^Q$ is a constrained dynamic equilibrium with fixed flow volume and departure choice, if for all $i\in I$, the following conditions hold  almost all $t\in [t_0,t_f]$: 
	\begin{align}
(\forall p\in \Pc_i):\; h^*_p(t)&>0 \Rightarrow	 \Psi_p(t,h^*)\leq  \Psi_q(t,h^*) \; \forall q\in U(h, t)\cap \Pc_i, \end{align}
where \begin{align}
	S^Q:=\{h\in \big(L_+(t_0,t_f)\big)^{\Pc}\vert h\in \Lambda(Q)\cap S\}.
	\end{align}

	\end{defn}

Finally, for the case of elastic volumes with departure choice,
the additional flexibility for an agent is the choice not to travel at all.
Hence, we need the following mappings:
\begin{equation}\label{eq:H-E}
\begin{aligned}
	H^E_{p\uparrow}(h,\theta,\varepsilon, \delta,):\equiv (h'_r)_{r\in \Pc}\in \big(L_+(t_0,t_f)\big)^{\Pc}
	\text{ with } h'_r(\cdot):=\begin{cases}&h_p+\varepsilon\CharF_{[\theta,\theta+\delta]}, r=p\\
	&h_r, \text{ else}.
	\end{cases}\\
	H^E_{p\downarrow}(h,\theta,\varepsilon, \delta,):\equiv (h'_r)_{r\in \Pc}\in \big(L_+(t_0,t_f)\big)^{\Pc}
	\text{ with } h'_r(\cdot):=\begin{cases}&h_p-\varepsilon\CharF_{[\theta,\theta+\delta]}, r=p\\
	&h_r, \text{ else}.
	\end{cases}
	\end{aligned}  
	\end{equation}

We get similarly:
\begin{equation}\label{eq:free-cap}
	U^E(t,h,p)=\{q \in \Pc\vert \forall \vol>0\; \exists \;\varepsilon, \delta>0, \theta'\geq 0 \text{ with }\varepsilon\cdot \delta\leq \vol \text{ such that } 
	H^E_{p\rightarrow q}(h,\theta,\theta',\varepsilon, \delta)\in S^Q\}.
	\end{equation}
\fi

\section{Tangent Cones and Variational Inequalities}


With this mapping, we define the set of deviations as
\begin{equation}\label{eq:M-r}
	M^r(h):=\{h'\in S^r\vert \; \exists i\in I, \varepsilon,J \subseteq [t_0,t_f], p,q\in  \Pc_i, h'(\cdot)=H_{i,p\to q}^r(h,J,\varepsilon)\}
\end{equation}
Let us define the following nonnegative tangent cone with respect to $S^r$ and $h$.
\begin{equation}
	T(S^r,h):=\{v\in L^2([0,T])^{\Pc}\vert \exists h^n\in M^r( h), t_n>0, \lim_{n\rightarrow\infty} t_n=0,  \lim_{n\rightarrow\infty} \frac{h^n-h}{t_n}=v\}.
\end{equation}

We introduce the following assumptions on a restriction set $S$ and an element $h \in S^r$:
\begin{assumption}\label{ass:closedTime}
	The set $S^r$ is \emph{XY-closed}\lukas{To be named} at $h \in S^r$ if for all $i \in I$, $q,p \in \Pc_i$, $\varepsilon, J', J$ we have:
	\[H_{i,p\to q}^r(h,J,\varepsilon)\in S^r, J' \subseteq J \implies H_{i,p\to q}^r(h,J',\varepsilon) \in S^r.\]
\end{assumption}

\begin{assumption}\label{ass:closedSpace}
	The set $S^r$ is \emph{ZW-closed}\lukas{To be named} at $h \in S^r$ if for all $i \in I$, $q,p \in \Pc_i$, $\varepsilon, \varepsilon', J$ we have:
	\[H_{i,p\to q}^r(h,J,\varepsilon)\in S^r, \varepsilon' \leq \varepsilon \implies H_{i,p\to q}^r(h,J,\varepsilon') \in S^r.\]
\end{assumption}

\begin{obs}
	\Cref{ass:closedSpace} is a weaker assumption than convexity, i.e. if $S^r$ is convex (at $h$), then \Cref{ass:closedSpace} holds. This is true because for $\varepsilon' \leq \varepsilon$ the flow $H_{i,p\to q}^r(h,J,\varepsilon')$ is a convex combination of $h$ and $H_{i,p\to q}^r(h,J,\varepsilon)$:
	\[H_{i,p\to q}^r(h,J,\varepsilon') = h + \frac{\varepsilon'}{\varepsilon}\cdot \left(H_{i,p\to q}^r(h,J,\varepsilon)-h\right).\]
	
	\Cref{ass:closedTime} is independent of convexity, i.e. there exists a set $S^r$ which is convex but does not satisfy \Cref{ass:closedTime} and there exists a set $S^r$ which satisfies \Cref{ass:closedTime} but is not convex. E.g. consider a network with a single commodity with an inflow rate $r = \CharF[{[0,2]}]$, two nodes $s$ and $t$ and two parallel edges $e_1$ and $e_2$ connecting $s$ and $t$ (these are then also the only $s$,$t$-paths $p$ and $q$). We define the following sets $S^r$:
	\begin{itemize}
		\item $S^r_1$ contains all path-inflows $h$ which at any point in $[0,2]$ sent all flow in exactly one of the two paths. Clearly, this set satisfies \Cref{ass:closedTime}. It is, however not convex, as it contains the path-inflow $h^1$ which sends all flow into $p$ as well as the path inflow $h^2$ which sends all flow into $q$ but not any of their non-trivial convex combinations.
		\item $S^r_2 \coloneqq \mathrm{conv}(h^1,h^2)$ is a convex set, but does not satisfy \Cref{ass:closedTime} since we have $H^r_{i,p\to q}(h^1,[0,2],1) = h^2 \in S^r_2$ but $H^r_{i,p\to q}(h^1,[0,1],1) \notin S^r_2$.\lukas{If I'm not mistaken, this should also be an example of why we need \Cref{ass:closedTime} in \Cref{thm:VI-fixed-inflow:nessecary}: $h^1$ is an equilibrium (since $D^r_{i,p}$ is always empty here), but if we use the same delay function for both edges $h^1$ is not a solution to the quasi variational inequality.}
	\end{itemize}
\end{obs}

%\begin{assumption}\label{ass:dc}
%The set $S^r$ is \emph{locally downwards closed}, that is, for all $h\in S^r, q,p\in \Pc_i, i\in I$, we have:
%\[ H_{i,p\to q}^r(h,t,\varepsilon,\delta)\in S^r \Rightarrow H_{i,p\to q}^r(h,t',\varepsilon,\delta')\in S^r  \;\forall\;t'\in [t,t+\delta), \delta'=t+\delta-t'.\]
%\end{assumption}

%A locally downwards closed set $S^r$ need not be convex as the following example demonstrates.
%\begin{example}
%Consider the following set\lukas{Isn't this set also convex? For $h,h' \in S^r$ and $\lambda \in [0,1]$ we have $\lambda h_p(t) + (1-\lambda)h'_p(t) \leq \max\set{h_p(t),h'_p(t)}$ and, thus, $g_p(\lambda h_p(t) + (1-\lambda)h'_p(t)) \leq g_p(\max\set{h_p(t),h'_p(t)}) \leq \max\set{g(h_p(t)),g_p(h'_p(t))} \leq 0$.}
%\[ S^r:=\{h\in  L^2([0,T])^{\Pc}\vert g_p(h_p(t))\leq 0\;  \forall \; t\in [t_0,t_f], p\in \Pc\},\]
%where $g_p:\R\rightarrow \R, p\in \Pc$ are monotonically increasing functions.
%This set need not be convex but it is locally downwards closed.
%To see this, let $h,  H_{i,p\to q}^r(h,t,\varepsilon,\delta)\in S^r$ for some  $q,p\in \Pc_i, i\in I, \varepsilon,\delta>0, t\in [t_0,t_f]$.
%We get
% \[ 
% \begin{aligned} g_w(h_w(\theta))& \leq 0\; \forall \; \theta\in [t_0,t_f], w\in \Pc,\\
	% g_w(h'_w(\theta))& \leq 0\; \forall \; \theta\in [t_0,t_f], w\in \Pc,
	% \end{aligned}
% \]
% where $h'=H_{i,p\to q}^r(h,t,\varepsilon,\delta)$.
%For any $h''= H_{i,p\to q}^r(h,t',\varepsilon,\delta')\in S^r$ with $t'\in [t,t+\delta), \delta'=t+\delta-t'$, we get
%\[ 
% \begin{aligned}h''_p(\theta)&\leq h_p(\theta)\; \forall \; \theta\in [t_0,t_f], p\in \Pc,\\
	% h''_q(\theta)&\leq h'_q(\theta) \; \forall \; \theta\in [t_0,t_f], p\in \Pc,
	% \end{aligned}
% \]
%and by the monotonicity of $g_p, p\in \Pc$ this yields
%\[ 
% \begin{aligned} g_p(h''_p(\theta))\leq g_p(h_p(\theta))& \leq 0\; \forall \; \theta\in [t_0,t_f], p\in \Pc,\\
	% g_p(h''_q(\theta))\leq g_p(h'_q(\theta)) & \leq 0\; \forall \; \theta\in [t_0,t_f], p\in \Pc,
	% \end{aligned}
% \]
%implying $h''\in S^r$.
%\end{example}

Before we completely characterize constrained dynamic equilibria, we need a property of non-equilibrium solutions $h\in S^r$ which, in turn, requires the following technical lemma:
\begin{lemma}\label{lemma:covering}
	Given a finite interval $[a,b] \subseteq \IR$, a subset $J \subseteq [a,b]$ of positive measure and for every $\theta \in J$ some positive number $\delta_\theta > 0$. Then there exists some $\theta \in J$ such that $J \cap [\theta,\theta+\delta_\theta]$ has positive measure.
\end{lemma}

\begin{proof}
	For any $n \in \IN$ we define the set of all points in $J$ which are the border of some interval of size at least $1/n$ but not in the interior of any interval by
	\[\partial J^n \coloneqq \Set{\bar\theta \in J | \exists \theta \in J: \delta_\theta \geq \frac{1}{n}, \bar\theta \in \set{\theta,\theta+\delta_\theta} \text{ and } \forall \theta \in J: \bar\theta \notin (\theta,\theta+\delta_\theta)}\]
	and claim that there is at most a countable number of such points. Indeed, note that any point in $\partial J^n$ must be the border point of an interval of size at least $1/n$ not containing any other such point. Thus, for any point in $\partial J^n$ there can be at most one other point from $\partial J^n$ within a distance of less than $1/n$. This implies that there are only countably many such points.
	
	But then the set
	\[\partial J \coloneqq \Set{\bar\theta \in J | \forall \theta \in J: \bar\theta \notin (\theta,\theta+\delta_\theta)} = \bigcup_{n \in \IN}\partial J^n\]
	is also countable and, thus, has measure zero. This, in turn, implies that $J \setminus \partial J$ still has positive measure. 
	
	Now, since the Lebesgue measure is inner regular, $J \setminus \partial J$ contains a compact subset $C$ of positive measure. As the family of open intervals $\set{(\theta,\theta+\delta_\theta) | \theta \in J}$ is an open covering of $J \setminus \partial J$, it is also a covering of $C$ and, thus, contains a finite subcover. Consequently, at least one of the open intervals $(\theta,\theta+\delta_\theta)$ of this subcover must have a  intersection of positive measure with $C$. This $\theta$ then also satisfies the desired property of the lemma.
\end{proof}

\begin{lemma}\label{claim:hStarNoCDE}
	If $h^* \in S^r$ is not a capacitated dynamic equilibrium, then there exists a time $t' \geq 0$, a commodity $i$, two paths $p,q \in \Pc_{i}$ and three positive numbers $\varepsilon, \delta, \gamma > 0$ such that
	\begin{itemize}
		\item $H_{i,p \to q}(h^*,t',\varepsilon,\delta) \in S^r$,
		\item $\Psi_p(t',h^*)-\Psi_q(t',h^*) \geq \gamma$ for all $t \in [t',t'+\delta]$ and
		\item $\int_{t'}^{t'+\delta}\min\set{h_p^*(\theta),\varepsilon}\diff\theta > 0$.
	\end{itemize}
\end{lemma}

\begin{proof}\lukas{Needs to be adjusted to new definition}
	If $h^*$ is not a capacitated dynamic equilibrium then, by definition, there exists some commodity $i\in I$, a path $p \in \Pc_i$ and a subset $J \subseteq [0,T]$ of positive measure such that for all $t' \in J$ we have $h_p^*(t') > 0$ and there exists some $q_{t'} \in D_{i,p}^r(h, t')$ with 
	\begin{align}\label{eq:CDE-condition-violated}
		\Psi_p(t',h^*)> \Psi_q(t',h^*).
	\end{align}
	From the definition of $D_{i,p}^r(h, t)$ we get for every such $t'$ some constants $\delta_{t'}, \varepsilon_{t'} > 0$ such that $H_{i,p \to q}(h^*,t',\varepsilon_{t'},\delta_{t'}) \in S^r$. Since $\Pc$ is finite, there must be some $q \in \Pc$ such that we can restrict $J$ to only those $t'$ where we can choose $q_{t'} = q$ and still have that $J$ has positive measure. Finally, because $\Psi_p(t,h^*)$ is continuous in $t$, we can wlog assume that each $\delta_{t'}$ is small enough such that \eqref{eq:CDE-condition-violated} holds for all $t \in [t',t'+\delta_{t'}]$. 		
	
	Then, by \Cref{lemma:covering}, there exists some $t' \in J$ such that $J \cap [t',t'+\delta_{t'}]$ still has positive measure. Consequently, the fact that we have $\min\set{h_p^*(\theta),\varepsilon_{t'}}>0$ for all $t \in J \cap [t',t'+\delta_{t'}]$ implies $\int_{t'}^{t'+\delta_{t'}}\min\set{h_p^*(\theta),\varepsilon_{t'}}\diff\theta > 0$.
	Thus, setting $\varepsilon \coloneqq \varepsilon_{t'}, \delta \coloneqq \delta_{t'}$ and, finally, $\gamma \coloneqq \min\set{\Psi_p(t,h^*)-\Psi_q(t,h^*) | t \in [t',t'+\delta]}$ gives us the desired objects.
\end{proof}

We now consider the following quasi-variational inequality:

\begin{equation}\label{eqn:QVI-fixed-inflow}\tag{\ensuremath{\QVI(\Psi,S,r,[t_o,t_f])}}
	\begin{aligned}
		\text{Find }h^* \in  S^r  \text{ such that}:&\\
		\scalar{\Psi(h^*)}{v} &\geq 0 \text{ for all } v \in T(S^r,h^*).\end{aligned}
\end{equation}

\begin{theorem}\label{thm:VI-fixed-inflow:nessecary}
	Let $h^* \in S^r$ such that $S^r$ satisfies \Cref{ass:closedTime} at $h^*$. If $h^*$ is a capacitated dynamic equilibrium then it is also a solution to the quasi-variational inequality \eqref{eqn:QVI-fixed-inflow}.
\end{theorem}

\begin{proof}
	Assume there is $v\in T(S^r,h^*)$ with $\scalar{\Psi(h^*)}{v} < 0$.
	By continuity of $\scalar{.}{.}$, there is $h^n\in S^r, t_n>0$ for $n$ large enough such that
	\[ \scalar{\Psi(h^*)}{\frac{h^n-h^*}{t_n}}<0  \Leftrightarrow \scalar{\Psi(h^*)}{h^n-h^*}<0.\]
	Rewriting and using that $h^n$ is of the form~\eqref{eq:H-r}, i.e. $h^n = H_{i,p \to q}(h^*,J,\varepsilon)$ for some $\varepsilon,J$, yields
	\begin{align*}
		0 > \scalar{\Psi(h^*)}{h^n-h^*}
		= & \int_{J} \Psi_p(t,h^*)  (h_p^n(t)-h_p^*(t)) + \Psi_q(t,h^*) (h_q^*(t)-h_q^n(t)) \diff t  \\
		= & \int_{J} \left(\Psi_q(t,h^*)- \Psi_p(t,h^*)\right)  \min\{h_p^*(t),\varepsilon\}.
	\end{align*}
	Hence, there is some $J'\subseteq J$ of positive measure with
		\[\left(\Psi_q(t,h^*)- \Psi_p(t,h^*)\right)  \min\{h_p^*(t),\varepsilon\}< 0 \text{ for all }t\in J'.\]
	and a further subset $J'' \subseteq J'$ such that for any $t \in J''$ and any $n \in \INs$ the set $J_n (t) \coloneqq J' \cap [t-\frac{1}{n},t+\frac{1}{n}]$ has positive measure\lukas{It should be clear that such a $J''$ exists, right?}. Then, for any $t \in J''$ we have $h_p^*(\bar t')>0$ as well as $H_{i,p\to q}^r(h^*,J_n(t),\varepsilon) \in S^r$ by \Cref{ass:closedTime} and, thus, $q\in D_{i,p}^r(h^*, \bar t')$, a contradiction to $h^*$ being a capacitated dynamic equilibrium, i.e. to $h^*$ satisfying~\eqref{eq:CDE}.	
\end{proof}

\begin{theorem}\label{thm:VI-fixed-inflow:sufficient}
	Let $h^* \in S^r$ such that $S^r$ satisfies \Cref{ass:closedSpace} at $h^*$. If $h^*$ is a solution to the quasi-variational inequality \eqref{eqn:QVI-fixed-inflow} then it is a capacitated dynamic equilibrium.
\end{theorem}

\begin{proof}\lukas{Needs to be adjusted to new definition}
	Let $h^* \in  S^r$ be a solution to \ref{eqn:QVI-fixed-inflow} and assume that  $h^*$ is not a constrained dynamic equilibrium. 	
	Using the $t', i, p, q, \varepsilon, \delta, \gamma$ and $h'$ from \Cref{claim:hStarNoCDE} and setting $\bar h \coloneqq H_{i,p \to q}(h^*,t',\varepsilon,\delta)$ we get $\bar h \in S^r$. Because of \Cref{ass:closedSpace} we now have $\bar h- h^* \in T(S^r,h^*)$. 
	But at the same time we have
	\begin{align*}
		&\scalar{\Psi(h^*)}{\bar h-h^*} 
		= \int_{t_0}^{t_f} \scalar{\Psi(h^*(t))}{\bar h(t)-h^*(t)}\diff t\\
		&\quad\quad=\int_{t'}^{t'+\delta}\Psi_p(t,h^*)\cdot\left(\bar h_p(t)-h_p^*(t)\right) + \Psi_q(t,h^*)(t)\cdot\left(\bar h_q(t)-h_q^*(t)\right)\diff t \\
		&\quad\quad=\int_{t'}^{t'+\delta}\left(\Psi_q(t,h^*)-\Psi_p(t,h^*)\right)\cdot\min\set{h_p^*(t),\varepsilon}\diff t \\
		&\quad\quad=\int_{t'}^{t'+\delta}\left(\Psi_q(t,h^*)-\Psi_p(t,h^*)\right)\cdot\min\{h_p^*(t),\varepsilon\}\diff t\\
		&\quad\quad\leq -\gamma \int_{t'}^{t'+\delta}\min\set{h_p^*(t),\varepsilon}\diff t < 0,
	\end{align*}
	which is a contradiction to $h^*$ being a solution to \eqref{eqn:QVI-fixed-inflow}.
\end{proof}

Quasi-variational inequalities may be much harder to solve
compared to standard variational inequalities
since the feasible search space depends on the  solution itself.
Let us now investigate conditions under which we can 
characterize constrained dynamic equilibria by means of ordinary
variational inequalities.
\begin{equation}\label{eqn:vi-fixed-inflow}\tag{\ensuremath{\VI(\Psi,S^r,r,[t_o,t_f])}}
	\begin{aligned}
		\text{Find }h^* \in  S^r  \text{ such that}:&\\
		\scalar{\Psi(h^*)}{h^*-h} &\geq 0 \text{ for all }h \in S^r.
		\end{aligned}
\end{equation}

For a complete characterization of constrained	dynamic equilibria
we need the following property of the feasible set $S^r$.
\begin{defn}\label{def:elementary}
The set $S^r$ is called \emph{closed with respect to elementary directions}, 
if for all $h,h'\in S^r$ the following holds true.
For all $i\in I, p,q\in \Pc_i, J\subset [t_0,t_f]$ of pos. meas. with $h_p(t)>h'_p(t)$, $h'_q(t)>h_q(t)$
for all $t\in J$, we have $q \in D_p(h,t)$.
\end{defn}

	\begin{theorem}\label{thm:VI-fixed-inflow:convex}
	Let $S^r$ be closed with respect to elementary directions. 
	Then, the following statements hold.
	\begin{enumerate}
	\item Every capacitated dynamic equilibrium $h^*\in S^r$ is a solution to the variational inequality~\eqref{eqn:vi-fixed-inflow}.
	\item Let  $S^r$ be additionally convex.  Then $h^*$ is a solution to the variational inequality~\eqref{eqn:vi-fixed-inflow} iff it is a capacitated dynamic equilibrium.
	\end{enumerate}
\end{theorem}

\begin{proof}\lukas{Needs to be adjusted to new definition}
We first show 1., i.e. necessity: Thus, let $h^* \in S^r$ be a capacitated dynamic equilibrium. We have to show that $h^*$ is a solution to~\eqref{eqn:vi-fixed-inflow}, i.e. that for any $h\in S^r$ we have $\scalar{\Psi(h^*)}{h^*-h} \geq 0$.
 
We write $h-h^*=g$ for  $g\in L_2^{\Pc}[t_0,t_f]$
with 
\[ g(t)=\sum_{p ,q\in \Pc} g_{p\rightarrow q}(t) , t\in [t_0,t_f],\]
where 
\begin{equation}
(g_{p\rightarrow q})_w(t):=\begin{cases} 0, \text{ if }w\not\in \{p,q\}\\
g_{p\rightarrow q}(t)\in \R_+, \text{ if }w=p\\
-g_{p\rightarrow q}(t)\in \R_-, \text{ if }w=q.
\end{cases}
\end{equation}
The values $g_{p\rightarrow q}(t), p,q\in \Pc$
solve a transshipment problem defined as follows:
We create a complete bipartite graph $G=(V_1(t)\cup V_2(t), E(t))$,
where nodes in $V_1(t) \subseteq \Pc$ are surplus nodes, that is, 
they fullfil $b_p(t):=h_p(t)-h^*_p(t)>0$ and nodes in $ V_2(t)\subseteq \Pc$
are deficit nodes fulfilling $b_q(t):=h_q(t)-h^*_q(t)<0$.
Note that obviously $V_1(t)\cap V_2(t)=\emptyset$ for all $t\in [t_0,t_f]$.
For every arc $(p,q)\in E(t):=(V_1(t)\times V_2(t))$ we define
capacities $c_{(p,q)}(t):=\min\{b_p(t), b_q(t)\}$.
By formulating an appropriate flow problem, there
always exists a transhipment, that is, edge values $g_{p\rightarrow q}(t)\geq 0, p\in V_1(t), q\in V_2(t)$\lukas{Ist es klar, dass diese $g_{p \to q}$ messbare Funktionen sind?} that satisfy the balance conditions and thus satisfy
\[ h-h^*=g.\] 
Assume by contradiction that
\[
\begin{aligned}
0 &> \scalar{\Psi(h^*)}{h-h^*} \\
&=\scalar{\Psi(h^*)}{g}\\
&= \scalar{\Psi(h^*)}{\sum_{p,q\in \Pc}g_{p\rightarrow q}}\\
&=\sum_{p,q\in \Pc} \scalar{\Psi(h^*)}{g_{p\rightarrow q}}\\
&= \sum_{p,q\in \Pc} \int_{t_0}^{t_f}g_{p\rightarrow q}(t) (\Psi_p(t,h^*)-\Psi_q(t,h^*))\diff t.
\end{aligned}
\]
Since  $g_{p\rightarrow q}\geq 0$, there must be a subset $J\subset [t_0,t_f]$ of positive measure
with $g_{p\rightarrow q}(t)>0 $ and $\Psi_p(t,h^*)-\Psi_q(t,h^*)<0$ for all $t\in J$.
 $g_{p\rightarrow q}(t)>0 $ implies $h^*_p(t)-h_p(t)<0, h^*_q(t)-h_q(t)>0$ and 
 especially $h^*_q(t)>0$.
As
 $S^r$ is  closed with respect to elementary directions  (cf. Def.~\ref{def:elementary}),
 we get $p\in D_{i,q}^r(t,h^*)$ for all $t\in J$
which contradicts
that $h^*$ is a constrained equilibrium.


To show 2. it only remains to prove sufficiency: So, let $h^* \in S^r$ be a solution to the variational inequality~\eqref{eqn:vi-fixed-inflow}. We claim that $h^*$ is then also a solution to the quasi-variational inequality~\eqref{eqn:QVI-fixed-inflow}. Take any $v \in T(S^r,h^*)$. Then there exist sequences of $h^n \in S^r$ and $t_n \in \IR_{>0}$ such that $\lim_{n\rightarrow\infty}\frac{h^n-h^*}{t_n} = v$. Using the continuity of $\scalar{.}{.}$ we get
\begin{align*}
	\scalar{\Psi(h^*)}{v} = \scalar{\Psi(h^*)}{ \lim_{n\rightarrow\infty}\frac{h^n-h^*}{t_n}} =  \lim_{n\rightarrow\infty}\frac{1}{t_n}\scalar{\Psi(h^*)}{h^n-h^*} \geq 0.
\end{align*}
Since convexity implies \Cref{ass:closedSpace}, sufficiency now follows directly from \Cref{thm:VI-fixed-inflow:sufficient}.
\end{proof}

An important example for which Def.~\ref{def:elementary}
applies is the case of monotone box-constraints.
\begin{obs}
Consider continuous and nondecreasing functions $z_p:\R_+\rightarrow\R_+, p\in\Pc$.
We get that the set
\[ S^r:=\{r\in \Lambda(r)\vert z_p(h_p(t))\leq v_p(t), p\in\Pc\}\]
for continuous functions $v_p:[t_0,t_f]\rightarrow\R_+, p\in\Pc$
is closed with respect to elementary directions.
To see this, let $h,h'\in S^r$ with $h_p(t)-h'_p(t)>0, h_q(t)-h'_q(t)<0$
for all $t\in J'$, where $J'\subseteq [t_0,t_f]$ is  of pos. meas.
Choose $J\subset J'$ of pos. meas. such that $\epsilon:=\inf_{t\in J}\{ \min\{h_p(t)-h'_p(t),  h'_q(t)-h_q(t)\}\}>0$.
As
\[ H_{i,p \to q}(h,t,\epsilon', \delta')_w(t)\leq \max\{h_w(t),h'_w(t)\} \text{for all }t\in J, 0\leq \epsilon'\leq \epsilon, 0\leq \delta'\leq \delta,w\in \Pc,\]
where $\delta>0$ is chosen such that $[t,t+\delta']\subset J'$\lukas{$J'$ muss nicht unbedingt Intervalle beinhalten}, we get  $H_{i,q \to p}(h,t,\epsilon, \delta)\in S^r$ for all $t\in [t,t+\delta], 0\leq \epsilon\leq \epsilon', 0\leq \delta\leq \delta'$ using
that $z_p, p\in \Pc$ is nondecreasing and both $h_p,h'_p$ are feasible.
\tobias{Haben wir dann nicht ein generelles Problem weil $H$ ja immer
\"uber Intervalle der L\"ange $\delta$ ein $\epsilon$ verschiebt ?}
\end{obs}
	
	\section{Non-Convex Edge-Load Constraints}
	
	There is an edge-load function 
\[ f:\Lambda(r)\rightarrow \big(L_+(t_0,t_M)\big)^{E}, h(\cdot)\mapsto (f_e(\cdot,h))_{e\in E}.\] 
This edge-load function is given here in general form but it can model, for instance, the flow volume, the inflow or the outflow 
of an edge edge at time $t$.
We define the restriction-set as 
\[ S^r:=\{h\in \Lambda(r)\vert f_e(t,h)\leq c_e(t),e\in E \text{ for almost all }  t\in [t_0,t_f+T]\},\]
where $c_e(\cdot)$ are capacity functions and $T$ is an upper bound on the time,
where the last flow particle reaches the sink under any $h\in  \Lambda(r)$.

For our existence result, we need a connection between the edge-load function
and the corresponding $h\in\Lambda(r)$.
For an integer $k \in \IN_{\geq 0}$, let $[k] := \{1,\dots,k\}$. 
For $p\in \Pc_i, i\in I$, we can write $p=((v_1,v_2),(v_2,v_3),\dots,(v_{k_p},v_{k_p+1}))$ with $v_1=s_i$ and $v_{k_p+1}=t_i$
and $e_j=(v_{j},v_{j+1})\in E, j\in [k_p]$.
For $h\in \Lambda(r)$, we assume there are travel time functions
\[ D_p^j: L_+(t_0,t_f)^{\Pc}\rightarrow L_+(t_0,t_f), h(\cdot)\mapsto D_p^j(\cdot,h), p\in \Pc, j\in[k_p+1]\]
where every $D_p^j(t,h)$ corresponds to the resulting travel time to  $v_j, j\in[k_p+1]$
 when starting at time $t\geq 0$ along $p$ given $h$.
With this function and given $h$, we can define the \emph{arrival time} at $v_j, j\in[k_p+1]$ of a particle that follows  $p$ starting at time $t\geq 0$ denoted by
\[\tau_p^j(t,h):=t+D_p^j(t,h).\]
These inter-arrival times satisfy FIFO, if
\[ \tau_p^j(t,h)\leq \tau_p^j(t',h) \text{ for all }0\leq t\leq t', p\in  \Pc.\]
These inter-arrival times are \emph{strictly monotone}, if
$\tau_p^j(t,h)-\tau_p^{j-1}(t,h)>0$
	for all $j\in[k_p+1],p\in  \Pc, t\geq 0$.
	\begin{defn}
We say that $h$ and $f(h)$ satisfy the \emph{principle of causation}, if the following condition is satisfied
for every $e\in E$ and any set $J\subseteq[t_0,t_f]$ of positive measure:
\begin{equation}
\begin{aligned}
f_e(t,h)>0\; \forall \;t\in J \Rightarrow &\;\exists (p\in  \Pc, J^{-1}\subseteq[t_0,t_f] \text{ of pos. meas.})   \text{ with }e=(v_j,v_{j+1})\in P   \\  \text{ such that } 
h_p(t')&>0\;\forall t'\in J^{-1} \text{ and }
 \tau_p^j(t',h)\in J \text{ for all } t'\in J^{-1}.
\end{aligned}
\end{equation}
The principle of causation states, that if the edge load is positive at point in time, then,
there must exist earlier times at which strictly positive flow is injected into
some path containing $e$ contributing to this positive edge load.
\end{defn}




\begin{assumption}
	\begin{enumerate}[label=(A\arabic*),start=5]
		\item For any sequence $h^n(\cdot)$ that converges weakly to $h^*\in \big(L_+(t_0,t_f)\big)^{\Pc}$, the  path delays $\Psi_p(\cdot,h^n)$ converge uniformly to $\Psi_p(\cdot,h^*)$ for all $p\in\Pc$. \label{ass:PathDelayConvergesUniform}\lukas{This is the same as (A3)?}
		\item The inter-arrival times satisfy FIFO and are strictly monotone.
		\item $h$ and $f(h)$ satisfy the principle of causation. \label{ass:PrincipleOfCausation}
		\item The function $f$ is weakly continuous over $\Lambda(r)$ in the sense of (A3)
		and $c$ is non-negative and continuous on $[t_0,t_f+T]$. \label{ass:fConvergesUniform}
		\item $S^r\neq \emptyset$ and it is bounded and closed (but not necessarily convex).
		\item There
		exists $\alpha>0$ such that for every $h\in \Lambda(r), i\in I$ and $t\in [t_0,t_f]$,  there is $p\in \Pc_i$ with $f_e(h, \tau^j_p(t,h))\leq c_e(\tau^j_p(t,h))-\alpha$ for all $e = (v_j,v_{j+1})\in p$.
%		\item For every weakly convergent sequence $h^n \to h$ with $h \in S^r$ and any $t \in [t_0,t_f], p \in \Pc, q \in D_p(t,h)$ here exists some $N \in \IN$ such that for all $n \geq N$ we have $f_e(\tau_q^j(t,h^n),h^n) \leq c_e(\tau_q^j(t,h^n))$ for all $e = (v_i,v_{j+1}) \in q$.
	\end{enumerate}
\end{assumption}
The  last assumption $(A10)$ requires the existence of a path with some 
free capacity no matter how long such a path may be. It can be thought of as an outside option
with finite travel time cost that is available for any agent at any point in time.

\iffalse
\begin{figure}\centering
	\begin{tikzpicture}[node distance={50mm}, thick, main/.style = {draw,
			circle, normalEdge},initial text={}] 
		\node[initial,main] (s)      {$s$};
		\draw[<-] (s) -- node[above] {$r=3\CharF[{[0,2]}]$} ++(-2cm,0);
		\node[main] (v) [right of=s] {$v$}; 
		\node[main] (t) [right of=v] {$t$}; 

		\draw[->] (s) to [bend left, normalEdge]  node[near start, above] (e1){$e_1$} node[midway,above] {$\tau = 1$} node[midway,below, sloped] {$\nu = 1$} (v);
		\draw[->] (s) to [bend right, normalEdge] node[near start, below] (e2){$e_2$} node[midway,above] {$\tau = 1$} node[midway,below, sloped] {$\nu = 2$} (v);
		\draw[->] (v) to [normalEdge] node[near start, above] (e3){$e_3$} node[midway,above] {$\tau = 1$} node[midway,below, sloped] {$\nu = 3$} node[midway,below=.5cm, sloped] {$c \equiv 2$} (t);
	\end{tikzpicture}
	\caption{A network with a single commodity with a constant inflow rate of $3$ during the interval $[0,2]$.}\label{fig:CounterExampleA11}
\end{figure}

An example of a network where assumption (A11) does not hold is shown in \Cref{fig:CounterExampleA11} with the Vickrey-Queuing-model for the delay functions: In this network we have two paths $p=(e_2,e_3)$ and $q=(e_1,e_3)$. The flow $h$ defined by sending flow at a rate of $1$ into $p$ and at a rate of $2$ into $q$ is clearly feasible. Furthermore, $q \in D_p(t,h)$ for all $t \in [0,2]$ as sending more flow into $q$ does not increase the outflow rate of edge $e_1$ over $1$. Now, define $h^n$ as the flow sending $1+\nicefrac{1}{n}$ into $p$ and the rest into $q$. Clearly, these $h^n$ converge weakly (even uniformly) to $h$ while at the same time all violating the capacity constraint on edge $e_3$ for all $t \in [2,3]$.
\fi
	
	\begin{theorem}
	Under assumptions $(A5)-(A10)$, there exists a constrained dynamic equilibrium $h\in S^r$.
	\end{theorem}
\begin{proof}
Define penalty functions $\xi_e(t,h,\epsilon):=\max\{0,f_e(t,h)-c_e(t)+\epsilon\},e\in E,t\in [t_0,t_f+T]$ for some $\epsilon>0$
and write $\xi_p(t,h,\epsilon):=\sum_{e = (v_j,v_{j+1})\in p} \xi_e(\tau^j_p(t,h),h,\epsilon)$.
By assumptions $(A5)$ and $(A8)$, the new $(\lambda,\epsilon)$-parametrized effective path-delay operator
\[ \Psi_p^{\lambda,\epsilon}(\cdot,h):=D_p(t,h)+\lambda \xi_p(t,h,\epsilon)\text{ for all }t\in [t_0, t_f], p\in  \Pc,\]
satisfies $(A5)$ for any $\lambda>0$ as well.

 Let us take a sequence of strictly positive numbers $(\lambda_n)_{n\in \IN},(\epsilon_n)_{n\in \IN}$ with
$\lambda_n=2^n\rightarrow+\infty$ and $\epsilon_n=1/n\rightarrow 0$.
With Theorem~\ref{thm:zhu-ex-fixed}, we obtain a sequence $h^n\in \Lambda(r)$ of solutions
to  the  variational inequalities~$(VI(\Psi^{\lambda_n,\epsilon_n},r,[t_0,t_f])$.
By taking subsequences,  $h^n$ weakly converges to some $h^*\in \Lambda(r)$ as $\Lambda(r)$ is bounded and closed.

\begin{claim}
There is  some $V\in \R_+$ with    
$\min_{p\in \Pc_i} \Psi^{\lambda_n,\epsilon_n}_p(t,h^n)<V$
for almost all $n\in \IN$ and all $i\in I, t\in [t_0,t_f]$.
\end{claim}
\begin{proof}
By Assumption $(A10)$, for every $h\in \Lambda(r), i\in I$ and $t\in [t_0,t_f]$, there
		exists $p_{i,t}\in \Pc_i$ with $f_e(\tau^j_p(t,h),h) \leq c_e(\tau^j_p(t,h)) - \epsilon_n$ 
		and, hence, $\xi_e(\tau^j_p(t,h),h,\epsilon_n) = 0$ for large enough $n$.
		Thus, 
		\[ \min_{w\in \Pc_i} \Psi^{\lambda^n,\epsilon_n}_w(t,h^n)\leq  \Psi^{\lambda^n,\epsilon_n}_{p_{i,t}}(t,h^n):=T,\]
		where $T$ is an upper bound on the makespan of any $h$  (which is well-defined
		for a finitely supported and bounded inflow rate along acyclic paths).
\end{proof} 


\begin{claim}\label{claim:hStarInSr}
	$h^*\in S^r$.
\end{claim}
\begin{proof}[Proof of the claim]
	Suppose that $h^* \notin S^r$, i.e. there exists some $e\in E$ and a subset $J\subseteq [t_0,t_f+T]$ of positive measure with
		\[ \xi_e(t,h^*,0)>0 \text{ for all }t\in J.\]
	For any $k \in \INs$ define $J_{k} \coloneqq \set{t \in J | \xi_e(t,h^*,0)\geq \nicefrac{1}{k}}$. Then we have $J = \bigcup_{k \in \INs} J_k$ and, thus, there must be some $k \in \INs$ such that $J_k$ has positive measure.
	Now, by assumption (A8), $f(h^n)$ converges uniformly to $f(h^*)$ and, in particular, for large enough $n \in \IN$ we have $\norm{f(h^n) - f(h^*)}_\infty \leq \nicefrac{1}{2k}$. Hence, for every $t \in J_k$ we have
		\[\xi_e(t,h^n,\epsilon_n) \geq \xi_e(t,h^n,0) \geq \xi_e(t,h^*,0) - \frac{1}{2k} \geq \frac{1}{k} - \frac{1}{2k} = \frac{1}{2k}.\]
	Since for all those $t \in J_k$ we have $f_e(t,h^n) > 0$ assumption~\ref{ass:PrincipleOfCausation} guarantees the existence of some $p_n \in \Pc$ with $e = (v_{j_n},v_{j_n+1}) \in p_n$ and some set $J^{-1}_{k,n} \subseteq [t_0,t_f]$ of positive measure, such that for all $t' \in J^{-1}_{k,n}$ we have
		\[h^n_{p_n}(t')>0 \text{ and } \tau^{j_n}_{p_n}(t',h^n) \in J_k\]
	and, thus, 
		\[\Psi^{\lambda_n,\epsilon_n}_{p_n}(t,h^n) \geq \lambda_n\xi_e(\tau^{j_n}_{p_n}(t',h^n),h^n,\epsilon_n) \geq \frac{\lambda_n}{2k}.\]
	Since $\lambda_n \to \infty$ there exists some $n \in \IN$ with $\frac{\lambda_n}{2k} > V$ and, therefor, for this $n$ we have a set $J^{-1}_{k,n}$ of positive measure such that for all $t' \in J^{-1}_{k,n}$ we have $h^n_{p_n}(t') > 0$ as well as $\Psi^{\lambda_n,\epsilon_n}_{p_n}(t,h^n) > V \geq \min_{p_n\in \Pc_i} \Psi^{\lambda^n,\epsilon_n}_{p_n}(t,h^n)$. But this is now a contradiction to $h^n$ being a capacitated dynamic equilibrium/solution to the variational inequality.\lukas{Which one?}
%	Then there is also some $\delta > 0$ and some set $J' \subseteq J$ os positive measure such that
%		\[ \xi_e(t,h)>2\delta \text{ for all }t\in J'.\]
%	Otherwise 
%	\iffalse
%	Let
%	\[\delta:=ess\inf_{t\in [\theta,\theta+\varepsilon]}\xi_e(t,h)>0,\]
%	where sets of measure $0$ are ignored.
%	\fi
%	\tobias{Check! I think it is false...}
%	By weak continuity  of $\xi_e(t,h)$ in $h$, there is $\delta>0$ and $J_{\delta}\subseteq J$ so that
%	\[ \xi_e(t,h^n)>\delta \text{ for all }t\in J_{\delta} \text{ and }n \text{ large enough}.\]
%	Hence, there must be some $p\in \Pc, e\in p, t\in J^{-1}_{\delta}$ with $h_p^n(t)>0$ and
%	\[\Psi^{\lambda^n}_p(t,h^n)\geq \lambda^n \delta\rightarrow+\infty,\]
%	which is a contradiction to $\Psi^{\lambda^n}_p(t,h^n)=\min_{w\in \Pc_i} \Psi^{\lambda^n}_w(t,h^n)<V$ for any $n$.
\end{proof}
\begin{claim}
$h^*\in S^r$ is an equilibrium.
\end{claim}
\begin{proof}[Proof of the claim]
Suppose that $h^*\in S^r$ is not an equilibrium, that is, there
is some $i\in I, q, p\in \Pc_i$ with $h_p(t)>0, q\in D_{i,p}^r(h, t)$ for all $t\in J$ with $J$ having pos. meas.
for which we have
\[ \Psi_p(t,h^*)>\Psi_q(t,h^*) \text{ for all }t\in J.\]
\begin{itemize}
\item With uniform convergence  of $\Psi_p(\cdot,h^n)\rightarrow \Psi_p(\cdot ,h^*)$, we get 
the existence of some $J'\subset J$ of pos. meas.
with
\[ \Psi_p(t,h^n)>\Psi_q(t,h^n) \text{ for all }t\in J' \text{ for $n$ large enough.}\]

\item For any $N \in \IN$ define the set $J'_N \coloneqq \set{t \in J' | \forall n \geq N: f_q(\tau_e(t,h^n),h^n) \leq c_e(\tau_e(t,h^n))}$. By assumption (A11) we have $J' = \bigcup_{N \in \IN} J'_n$ and, thus, there exists some $N \in \IN$ such that $J'_N$ has positive measure. Note, that $J'_n \supseteq J'_N$ for all $n \geq N$.

\item With weak convergence of $h^n\rightarrow h^*$ we get the existence of some $J_n\subseteq J'_N$ of pos. meas. with $h_p^n(t)>0$ for $n \geq N$ large enough.

To see this, let $\CharF[J,p]$ be the characteristic function of the measurable set $J$ for path $p$ and the zero function for all other paths. Then the weak convergence of $h^n$ to $h^*$ implies
	\[\lim_n \int_J h^n_{p}(t) \diff t = \lim_n \scalar{h^n}{\CharF[J,p]} = \scalar{h^*}{\CharF[J,p]} = \int_J h^*_{p}(t)\diff t > 0.\]
This implies $\int_J h^n_{p}(t) \diff t > 0$ for large enough $n$ and, thus, for any such $n$ there exists a subset $J_n \subseteq J$ of positive measure with $h^n_{p}(t) > 0$ for all $t \in J_n$.
\end{itemize}

Now, for any $t \in J_n \subseteq J'_N$ we have $f_e(\tau_e(t,h^n),h^n) \leq c_e(\tau_e(t,h^n))$ for all $e \in q$ and, therefor, $\xi_q(t,h^n) = 0$. This implies
	\[\Psi_p^{\lambda^n}(t,h^n)\geq \Psi_p(t,h^n)>\Psi_q(t,h^n)=\Psi_q^{\lambda^n}(t,h^n)\]
while, at the same time, $t \in J_n$ also implies $h_p^n(t)>0$. But, together with the fact that $J_n$ has positive measure, these two statements are a contradiction to $h^n$ being a dynamic equilibrium solution to $(VI(\Psi^{\lambda^n},r,[t_0,t_f])$.

%\item With weak convergence of $h^n\rightarrow h^*$ we get
%the existence of some $J_n\subset J'$ of pos. meas.
%with $h_p^n(t)>0$ for $n$ large enough.
%
%To see this, let $\CharF[J,p]$ be the characteristic function of the measurable set $J$ for path $p$ and the zero function for all other paths. Then the weak convergence of $h^n$ to $h^*$ implies
%	\[\lim_n \int_J h^n_{p}(t) \diff t = \lim_n \scalar{h^n}{\CharF[J,p]} = \scalar{h^*}{\CharF[J,p]} = \int_J h^*_{p}(t)\diff t > 0.\]
%This implies $\int_J h^n_{p}(t) \diff t > 0$ for large enough $n$ and, thus, for any such $n$ there exists a subset $J_n \subseteq J$ of positive measure with $h^n_{p}(t) > 0$ for all $t \in J_n$.
%
%\item With  $ q\in D_{i,p}^r(h^*, t)$ for  $t\in J$, by definition there is $\epsilon,\delta>0$
%such that 
%\[ H_{i,p\to q}^r(h^*,t,\epsilon,\delta) \in S^r. \]
%Thus, we get for $t\in J_n$ and every $e=(v_j,v_{j+1})\in q$:\tobias{Hier m\"ussten wir nun eigentlich die Ankunftszeiten von $H$ einsetzten....aber die \"andern sich ja stetig  abh\"angig von $\epsilon $ und $\delta$...}
%\[ f_e(\tau_{q}^j(t',h^*),h^*)\leq c(\tau_{q}^j(t',h^*))-\epsilon \; \forall\; t'\in [t,t+\delta].\]
%As $f(h^n)$ converges uniformly to $f(h^*)$, we get
% \[ f_e(\tau_{q}^j(t',h^n),h^n)\leq c(\tau_{q}^j(t',h^n))-\epsilon' \; \forall\; t'\in [t,t+\delta], n \text{ large enough},\] 
%for some $0<\epsilon'\leq \epsilon$.\tobias{Hmmm noch immer nicht ganz klar ob das \"uberhaupt gilt..habe nur mal den Weg skizziert..}
%Thus, for $n$ large enough, we get $\xi_q(t,h^n)=0$ for all $t\in J^n\cap [t,t+\delta]$.
%\end{itemize}
%With $\xi_q(t,h^n)=0$ for all $t\in J^n\cap [t,t+\delta]$:
%\[ \Psi_p^{\lambda^n}(t,h^n)\geq \Psi_p(t,h^n)>\Psi_q(t,h^n)=\Psi_q^{\lambda^n}(t,h^n),\]
%a contradiction to $h^n$ being a dynamic equilibrium solution to $(VI(\Psi^{\lambda^n},r,[t_0,t_f])$.

%\todo[inline]{Alternative proof without the need for the dubious second bullet point:}
%
%Suppose that $h^*\in S^r$ is not an equilibrium, that is, there
%is some $i\in I, q, p\in \Pc_i$ with $h_p(t)>0, q\in D_{i,p}^r(h, t)$ for all $t\in J$ with $J$ having pos. meas.
%for which we have
%	\[ \Psi_p(t,h^*)>\Psi_q(t,h^*) \text{ for all }t\in J.\]
%
%Define $J_k \coloneqq \set{t \in J | \Psi_p(t,h^*) \geq \Psi_q(t,h^*) + \nicefrac{1}{k}}$. Then we have $J = \bigcup_{k \in \INs}J_k$ and, thus, there exists some $k \in \IN$ such that $J_k$ has positive measure. Fix this $k$ from now on. By assumption \ref{ass:PathDelayConvergesUniform} we know that $\Psi_p(.,h^n)$ and $\Psi_q(.,h^n)$ converge uniformly to $\Psi_p(.,h^*)$ and $\Psi_q(.,h^*)$, respectively. In particular, for large enough $n$ we have 
%	\[\Psi_p(t,h^n) \geq \Psi_p(t,h^*) - \frac{1}{3k} \geq \Psi_q(t,h^*) + \frac{2}{3k} \geq \Psi_q(t,h^n) + \frac{1}{3k}\]
%for all $t \in J_k$. 
%
%Furthermore, since by assumption \ref{ass:fConvergesUniform} $\xi_q(.,h^n)$ converges uniform to $\xi_q(.,h^*)$ and $\xi_q(.,h^*) \equiv 0$ by \Cref{claim:hStarInSr} we have for large enough $n$ that  
%
%Now we claim, that there exists some $n$ and some subset $J' \subseteq J_k$ such that 
\end{proof}
\end{proof}
