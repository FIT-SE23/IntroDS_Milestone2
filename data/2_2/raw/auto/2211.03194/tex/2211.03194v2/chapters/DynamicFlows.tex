% !TeX spellcheck = en_GB
%!TEX root = ../side-constrained.tex

\section{Unconstrained Dynamic Flows}\label{sec:dynamic}

We consider the following model based on the walk-delay operator model of \citefulls{FrieszLTW89}: We are given a finite directed graph $G=(V,E)$ and some fixed planning horizon $\planningInterval \subseteq \IR_{\geq 0}$. Additionally, we have a finite set of commodities $I$ and for every commodity $i \in I$ a source node $s_i \in V$, a sink node $t_i \in V$ and either a fixed network inflow volume $Q_i \geq 0$ (for the model with departure time choice) or a fixed bounded network inflow rate $r_i\in L^2_+(\planningInterval)$ (for the model without departure time choice) where $L^2_+(\planningInterval)$ denotes the set of non-negative $L^2$-integrable functions on $\planningInterval$. 

We denote by $\Pc_i$ a fixed set of $s_i$,$t_i$-walks and assume -- \wlofg -- that these sets are disjoint for different commodities. We then denote by $\Pc \coloneqq \bigcup_{i \in I}\Pc_i$ the set of all relevant walks. Note that we allow general walks instead of just simple paths as travelling along cycles is necessary in certain applications like electric vehicles (\cf \cite{GHP22}) and can also sometimes be advantageous in networks with hard edge-capacities (\eg \Cref{ex:ImprovementByCycles}). 
A flow in this network is given by a vector $h \in L^2_+(\planningInterval)^\Pc$ of $L^2$-integrable functions $h_{p}: \planningInterval \to \IR_{\geq 0}$ denoting the walk inflow rates for all walks of all commodities. We denote by
	\[\Lambda(Q) \coloneqq \Set{h \in L^2_+(\planningInterval)^\Pc | \sum_{p \in \Pc_i}\int_{t_0}^{t_f}h_{p}(t)\diff t = Q_i \text{ for all } i \in I, h_{p} \leq B_{p} \text{ for all } p \in \Pc},\]
and
	\[\Lambda(r) \coloneqq \Set{h \in L^2_+(\planningInterval)^\Pc | \sum_{p \in \Pc_i}h_p(t) = r_i(t) \text{ for almost all } t \in \planningInterval \text{ and all } i \in I}\]
the sets of all feasible walk inflows for the model with and without departure time choice, respectively. In the definition of $\Lambda(Q)$ the values $B_p \in \IR_{\geq 0}$ are some fixed walk-specific bounds on the walk inflow rates. In general, these are needed to ensure existence of equilibria in this model (see \Cref{rem:JustificationPathInflowBounds}). Note, however, that in some models such bounds can be introduced without loss of generality since choosing $B_p$ large enough only excludes flows which cannot be equilibria anyway (\cf \cite[Proposition 5.9]{Han2013}). We also observe that we can always view $\Lambda(r)$ as a subset of some $\Lambda(Q)$ by defining $Q_i \coloneqq \int_{\tStart}^{\tEnd}r_i(t)\diff t$ and choosing $B_p \geq \sup_{t \in \planningInterval}r_i(t)$ for every $p \in \Pc_i$. 
For the case of elastic demands we are given a non-increasing inverse demand function $\Theta_i: [0,Q_i] \to \IR$ such that for any possible demand $Q \leq Q_i$ the value $\Theta_i(Q)$ is the cost threshold at which a volume of $Q$ of all particles of commodity $i$ is still willing to travel while the rest stays at home.

Furthermore, we are given a function
	\[\Psi:  L^2_+(\planningInterval)^\Pc \to  \hat{M}(\planningInterval)^\Pc, h \mapsto \left(\Psi_p(h,\cdot):\planningInterval \to \IR_{\geq 0}\right)_{p \in \Pc}\]
mapping walk inflows to \effWalkDelay, \ie for any walk inflow $h$, commodity $i$, walk $p \in \Pc_i$ and time $t$ the value $\Psi_p(h,t)$ is to be understood as the total travel cost (\eg some weighted sum of travel time, penalty for late arrival and cost of energy consumption on the chosen route) of a particle of commodity $i$ starting at time $t$ to travel along walk $p$ under the traffic state induced by the walk inflow $h$. Here, we denote by $\hat{M}(\planningInterval)$ the set of measurable functions from $\planningInterval$ to $\IR \cup \set{\infty}$.

We can now define three standard types of dynamic equilibria (\cf \eg \cite{ZhuM00,Friesz93,HanFSH15}):

\begin{definition}\label{def:DE}
	\begin{itemize}
		\item 	$h^*\in \Lambda(r)$ is a \emph{dynamic equilibrium with fixed inflow rates}, if for all $i\in I$, the following condition holds:
		\begin{align}\label{eq:de-rate}
			h^*_p(t)&>0 \Rightarrow \Psi_p(h^*,t)\leq \Psi_q(h^*,t)\text{ for almost all }t\in \planningInterval, p, q\in \Pc_i.	
		\end{align}
		\item 	$h^* \in \Lambda(Q)$ is a \emph{dynamic equilibrium with fixed flow volumes and departure time choice}, if for all $i\in I$, there exists a $\nu_i \in \IR$ such that the following conditions hold:
		\begin{align}\label{eq:de-volume}
			\begin{aligned}
				h^*_p(t)>0 &\Rightarrow \Psi_p(h^*,t)\leq\nu_i \text{ for almost all }t\in \planningInterval, p\in \Pc_i\\
				h^*_p(t)<B_p &\Rightarrow \Psi_p(h^*,t)\geq \nu_i \text{ for almost all }t\in \planningInterval, p\in \Pc_i.				
			\end{aligned}
		\end{align}
		\item $(h^*,Q^*)$ with $Q^* \in \IR^I_{\geq 0}$ and $h^* \in \Lambda(Q^*)$ is a \emph{dynamic equilibrium with elastic demands and departure time choice}, if for all $i \in I$, there exists a $\nu_i \in \IR$ such that the following conditions hold:
		\begin{align}\label{eq:de-elastic}
			\begin{aligned}
				h^*_p(t)>0 &\Rightarrow \Psi_p(h^*,t)\leq \nu_i \text{ for almost all }t\in \planningInterval, p\in \Pc_i\\
				h^*_p(t)<B_p &\Rightarrow \Psi_p(h^*,t)\geq \nu_i \text{ for almost all }t\in \planningInterval, p\in \Pc_i \\
				Q^*_i < Q_i &\Rightarrow \nu_i = \Theta_i(Q^*_i) \\
				Q^*_i = Q_i &\Rightarrow \nu_i \leq \Theta_i(Q_i).
			\end{aligned}
		\end{align}
	\end{itemize}
\end{definition}

We observe that the model with elastic demands can be seen as a special case of the model with fixed inflow volume. Thus, we will only consider the first two models for the rest of this paper.
Note, that this trick is certainly not new and has been applied for static models before, see~Patriksson~\cite[Sec. 2.2.4]{Patriksson1994tap}.

\begin{lemma}
	Consider a network $\mathcal{N}$ with elastic demand and departure time choice and construct from this a new network $\mathcal{N}'$ with fixed flow volumes and departure time choice by adding for every commodity $i \in I$ an additional new edge $\tilde{e}_i$ connecting the commodity's source and sink node and defining $\Pc_i' \coloneqq \Pc_i \cup \Set{\tilde{p}_i}$ where $\tilde{p}_i \coloneqq (\tilde{e}_i)$ is the path consisting of only this new edge. Moreover, for each of these new paths we define the effective walk delay operator by $\Psi_{\tilde{p}_i}(h,t) \coloneqq \Theta_i(\sum_{p \in \Pc_i}\int_{t_0}^{t_f}h_p(t')\diff t')$ and choose $B_{\tilde{p}_i}$ such that $B_{\tilde{p}_i}\cdot(\tEnd-\tStart) > Q_i$. 
	
	Then every dynamic equilibrium in $\mathcal{N}'$ corresponds (in the natural way) to a dynamic equilibrium in $\mathcal{N}$ and vice versa.
\end{lemma}

\begin{proof}
	First, assume that $h'$ is a dynamic equilibrium with fixed flow volume and departure time choice in $\mathcal{N}'$. Then we define $Q^*_i \coloneqq \sum_{p \in \Pc_i}\int_{\tStart}^{\tEnd}h'_p(t)\diff t$ for every commodity~$i \in I$ and $h^*$ as the restriction of $h'$ to the subnetwork $\mathcal{N}$. We clearly have $h^* \in \Lambda(Q^*)$ and the first two conditions of~\eqref{eq:de-elastic} are satisfied because of~\eqref{eq:de-volume} for $h'$ (with the same $\nu_i$). Moreover, due to our choice of $B_{\tilde{p}_i}$ there must be a set of positive measure of times $t$ with $h'_{\tilde{p}_i}(t) < B_{\tilde{p}_i}$ and, therefore, \eqref{eq:de-volume} implies $\nu_i \leq \Psi_{\tilde{p}_i}(h',t) = \Theta_i(Q^*_i)$. Finally, if we have $Q^*_i < Q_i$, then there must be a set of positive measures of times $t$ with $h'_{\tilde{p}_i}(t) > 0$ and, thus, \eqref{eq:de-volume} also guarantees $\nu_i \geq \Psi_{\tilde{p}_i}(h',t) = \Theta_i(Q^*_i)$
	
	Now, assume that $(h^*,Q^*)$ is a dynamic equilibrium with elastic demand and departure time choice. Then, we can extend $h^*$ to a flow $h' \in \Lambda(Q)$ by setting $h'_{\tilde{p}_i}(t) \coloneqq \frac{Q_i-Q^*_i}{\tEnd-\tStart}$ for all $t \in \planningInterval$ and $i \in I$. Using the same $\nu_i$ as for $h^*$ we then immediately have that the conditions in~\eqref{eq:de-volume} hold for $h'$ on all paths in $\Pc_i$. So, we only have to show that they also hold for the new paths $\tilde{p}_i$: If we have $h'_{\tilde{p}_i}(t)>0$ at any time $t$, we have $Q^*_i < Q_i$ and, therefore, $\Psi_{\tilde{p}_i}(h',t) = \Theta_i(Q^*_i) = \nu_i$. Otherwise, we still have $\Psi_{\tilde{p}_i}(h',t) = \Theta_i(Q^*_i) \geq \nu_i$ for all times $t$. Hence, in both cases \eqref{eq:de-volume} holds for the new paths~$\tilde p_i$ as well.
\end{proof}

We now want to characterize the first two types of dynamic equilibria using variational inequalities. For this we require that the effective walk delays are bounded in the following sense:

\begin{enumerate}[label=(A\arabic*),series=Assumptions]
	\item For any $p \in \Pc$ and $h \in \Lambda(Q)$ the function $\Psi_p(h,.)$ is bounded.\label[asmpt]{ass:PsiBounded}
\end{enumerate}

This then, in particular, implies that all effective walk delays are $L^2$-integrable, \ie $\Psi_p(h,.) \in L^2(\planningInterval)$. An example for where this assumption may be violated is a network-loading model involving spillback and an \effWalkDelay{} $\Psi$ that just measures the actual delay. Since spillback can lead to gridlock (\cf \cite[pp. 99f]{SeringThesis}), the delay will be $\infty$ for particles caught in such a gridlock. However, even for these cases our model might still be applicable as long as particles always have the option to avoid joining a gridlock and achieving some bounded cost (\eg some alternative route with infinite capacity or a stay at home-option). This will be formalized in \Cref{lemma:RestrictToTruncatedPsi}.

If assumption \ref{ass:PsiBounded} holds, it is well known (\cf \eg \cite{Friesz93,ZhuM00}) that, as in the static case, both kinds of dynamic equilibria can be characterized by variational inequalities  namely
\begin{equation}\label{eqn:vi-fixed-inflow-uc}\tag{\ensuremath{\VI(\Psi,r,\planningInterval)}}
	\begin{aligned}
		\text{Find }h^* \in  \Lambda(r)  \text{ such that}:&\\
		\scalar{\Psi(h^*)}{h-h^*} &\geq 0 \text{ for all }h \in \Lambda(r)
	\end{aligned}
\end{equation}
and
\begin{equation}\label{eqn:vi-fixed-volume-uc}\tag{\ensuremath{\VI(\Psi,Q,\planningInterval)}}
	\begin{aligned}
		\text{Find }h^* \in  \Lambda(Q)  \text{ such that}:&\\
		\scalar{\Psi(h^*)}{h-h^*} &\geq 0 \text{ for all }h \in \Lambda(Q).
	\end{aligned}
\end{equation}
Here, $\scalar{.}{.}$ denotes the canonical scalar product on $L^2(\planningInterval)^\Pc$, i.e.
	\[\scalar{.}{.}: L^2(\planningInterval)^\Pc \times L^2(\planningInterval)^\Pc \to \IR, (f,g) \mapsto \scalar{f}{g} \coloneqq \sum_{p \in \Pc}\int_{\tStart}^{\tEnd}f_p(\theta)g_p(\theta)\diff\theta.\]

\begin{theorem}\label{thm:VICharOfDE}
	Assume that \ref{ass:PsiBounded} holds. Then, a walk inflow $h^* \in \Lambda(r)$ ($h^* \in \Lambda(Q)$) is a dynamic equilibrium with fixed inflow rates (with fixed flow volume) if and only if $h^*$ is a solution to \eqref{eqn:vi-fixed-inflow-uc} (to \eqref{eqn:vi-fixed-volume-uc}).
\end{theorem}

Conditions to guarantee the existence of such an element $h^*$ are given by Lions in \cite[Chapitre 2, Théorème 8.1]{Lions} which, following \citefulls{CominettiCL15}, can be restated as follows (see \cite[Proof of Theorem 4.2]{ZhuM00} for how to derive this version from Lions' result):
\begin{theorem}\label{thm:Lions} 
	Let $C$ be a non-empty, closed, convex and bounded subset of $L^2([a, b])^d$. Let $\A : C \rightarrow L^2([a, b])^d$ be a sequentially weak-strong-continuous mapping. Then, the following variational inequality has a solution $h^* \in C$:
		\begin{equation*}
			\begin{aligned}
				\text{Find }h^* \in  C  \text{ such that}:&\\
				\scalar{\A(h^*)}{h-h^*} &\geq 0 \text{ for all }h \in C.
			\end{aligned}
		\end{equation*}
\end{theorem}
Using this theorem together with \Cref{thm:VICharOfDE} one can derive existence of dynamic equilibria under suitable additional assumptions:

\begin{enumerate}[label=(A\arabic*),resume=Assumptions]
	\item The sets $\Pc_i$ are non-empty and finite.\label[asmpt]{ass:FinitelyManyWalks}\label[asmpt]{ass:nonEmptyPathset}
	\item The walk inflow bounds $B_p$ are chosen such that there exists at least one feasible flow $h \in \Lambda(Q)$.\label[asmpt]{ass:PathInflowBounds}
	\item For any $p \in \Pc$ the mapping $\Lambda(Q) \to L^2(\planningInterval), h \mapsto \Psi_p(h,.)$ is sequentially weak-strong continuous.\label[asmpt]{ass:PsiWScont}
\end{enumerate}

Here, a mapping $\mathcal{A}: K \to X$ from some subset $K \subseteq B$ of a reflexive Banach space $B$ (\eg $L^2(\planningInterval)^\Pc$) to a topological space $X$ is \emph{sequentially weak-strong continuous} if it maps weakly convergent sequences in $B$ to (strongly) convergent sequences in $X$. A sequence $f^n$ converges weakly in $B$ if for any $g \in B$ the sequence $\scalar{f^n}{g}$ converges in $\IR$.

\begin{theorem}\label{thm:ExistenceUnconstrained}
	For any network and \effWalkDelay s satisfying \labelcref{ass:FinitelyManyWalks,ass:nonEmptyPathset,ass:PathInflowBounds,ass:PsiWScont,ass:PsiBounded} there exists a dynamic equilibrium (both with and without departure time choice).
\end{theorem}

As this theorem can be proven in essentially the same way as analogous existence results for very similar models  (\cf \eg \cite{Han2013,ZhuM00}), we omit the proof here. We do, however, provide a proof for the exact setting used here in \Cref{app:VI-Existence}.