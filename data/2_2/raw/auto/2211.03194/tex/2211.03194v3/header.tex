\ifarxiv
	\usepackage{authblk}
\fi

\usepackage[british]{babel}

\usepackage{tikz}
\usetikzlibrary{calc,patterns}


% Tikz-Bilder nicht jedesmal neu rendern - benötigt Schalter -shell-escape
%	\usetikzlibrary{external}
%	\tikzexternalize[prefix=i/]




\usepackage{pgfplots}
\pgfplotsset{compat=1.18}
\usepgfplotslibrary{fillbetween}

% TikZ for figures
%%%%%%%%%%%%%%%%%%%%%%%%%%%%%%%%%%%%%%%%%%%%%%%%%%%%%%%%%
\tikzstyle{vertex} = [shape=circle,draw=black]
\tikzstyle{namedVertex} = [shape=circle,draw=black]
\tikzstyle{edge} = [draw,->,thick]
\tikzstyle{labeledNodeS}=[circle, color=black!75!white, draw, inner sep = 0.1em, minimum size = 1.5em, scale=1.25]
\tikzstyle{normalEdge}=[very thick, >=stealth]


\ifarxiv
	\usepackage[utf8]{inputenc}
	\usepackage[T1]{fontenc}
\fi
\usepackage{framed}



\usepackage{adjustbox}				

\usepackage{mathtools}

\ifarxiv
	\usepackage{amsthm}		% Theorem- und Beweisumgebungen
\fi
\usepackage{amssymb}				% zahlreiche mathematische Symbole
\usepackage{mathrsfs,stmaryrd}		


\usepackage{aligned-overset} 	% For aligning relation with overset text (\overset{...}&{=})


%%%%%% REFERENCES %%%%%%%%%%%%%%%%%%

% Automatische Referenzen mit Namen
\ifarxiv
	\usepackage[bookmarks=false,colorlinks=true, linkcolor=blue, urlcolor=blue, citecolor=blue, breaklinks=true]{hyperref}
\else
	\usepackage[colorlinks=false, breaklinks=true]{hyperref}
\fi
\usepackage{cleveref}			% Referenzen mit Name

\crefname{cons}{constraint}{constraints}
\Crefname{cons}{Constraint}{Constraints}
\creflabelformat{cons}{(#2#1#3)}
\crefname{claim}{claim}{claims}
\Crefname{claim}{Claim}{Claims}
\crefname{assumption}{assumption}{assumptions}
\Crefname{assumption}{Assumption}{Assumptions}

%%%%%% Bibliography %%%%%%%%%%%

\ifarxiv
\usepackage[style=alphabetic,maxbibnames=99,maxalphanames=4,maxcitenames=99]{biblatex}
\newcommand{\citefull}[2][]{\citeauthor{#2}~\cite[#1]{#2}}
\newcommand{\citefulls}[2][]{\citeauthor*{#2}~\cite[#1]{#2}}
\fi


%%%%%% Theorem-Umgebungen %%%%%%%%%%

\ifec
	\newenvironment{proofClaim}[1][]{\ifthenelse{\equal{#1}{}}{\begin{proof}}{\begin{proof}[#1]}\renewcommand\qedsymbol{\ensuremath{\blacksquare}}}{\end{proof}}
	\AtEndPreamble{%
		\theoremstyle{acmdefinition}
		\newtheorem{remark}[theorem]{Remark}
		\newtheorem{obs}[theorem]{Observation}
		\newtheorem{claim}[theorem]{Claim}
	}
\else	
	\ifarxiv
		\theoremstyle{definition}
		\newtheorem{definition}{Definition}[section]
		\newtheorem{example}[definition]{Example}
		
		\theoremstyle{plain}
		
		\newtheorem{proposition}[definition]{Proposition}
		\newtheorem{assumption}[definition]{Assumption}
		\newtheorem{corollary}[definition]{Corollary}
		\newtheorem{lemma}[definition]{Lemma}
		\newtheorem{theorem}[definition]{Theorem}
		\newtheorem{conjecture}[definition]{Conjecture}
		
		\newtheorem{claim}{Claim}
		%\makeatletter\@addtoreset{claim}{definition}\makeatother
		\newenvironment{proofClaim}[1][]{\ifthenelse{\equal{#1}{}}{\begin{proof}}{\begin{proof}[#1]}\renewcommand\qedsymbol{\ensuremath{\blacksquare}}}{\end{proof}}
		
		\theoremstyle{remark}
		\newtheorem{remark}[definition]{Remark}
		\newtheorem{obs}[definition]{Observation}
		\newtheorem{notation}[definition]{Notation}
		
		\newenvironment{proofNormal}[1][Proof]{\begin{proof}[#1]}{\end{proof}}
		
		\usepackage{enumitem}
		\usepackage{bm} 	
		\newcommand{\caseitem}[1]{\def\casedescr{#1}%
			\item}
		\newlist{proofbycases}{enumerate}{1}
		\setlist[proofbycases]{
			leftmargin=3em,
			labelwidth=1.5em,
			label=\boldmath\bfseries\sffamily\arabic*. Case: \protect\casedescr:,
			ref=\arabic*,
			align=left
		}
	\else
		\newtheorem{obs}{Observation}
		\newcommand{\qedhere}{\relax}
		\newenvironment{proofClaim}[1][]{\ifthenelse{\equal{#1}{}}{\proof{Proof.}}{\proof{#1.}}\renewcommand\square{\ensuremath{\blacksquare}}}{\Halmos\endproof}
		\newenvironment{proofNormal}[1][]{\ifthenelse{\equal{#1}{}}{\proof{Proof.}}{\proof{#1.}}}{\Halmos\endproof}
	\fi
\fi



\usepackage{array}					% Verbesserte Implementierung der tabular und array-Umgebungen

\usepackage{braket}					% \Set{ ... | ... } und \set{ ... | ... }
\ifec\else % Produces error "To many math alphabets"
	\ifarxiv\else
		\newcommand\hmmax{0}
		\newcommand\bmmax{0}
	\fi
	\usepackage{bm}						% fette Symbole in Math-Umgebung (\bm)
\fi
\usepackage{nicefrac} 				% TODO: Schönere Brüche?


\allowdisplaybreaks					% erlaubt Seitenumbrüche in align*-Umgebung (aber nicht align, gather, ...)

% TODO: ticz
%\usetikzlibrary{babel} %Vermeidet Konflikte babel, v.a. für Notation\arrowvert[r, "Text"]


% Zahlenmengen:
\newcommand{\IN}{\mathbb{N}}	% Natürliche Zahlen
\newcommand{\INo}{\IN_0}		%	mit Null
\newcommand{\INs}{\IN^\ast}		%	ohne Null
\newcommand{\IZ}{\mathbb{Z}}	% Ganze Zahlen
\newcommand{\IQ}{\mathbb{Q}}	% Rationale Zahlen
\newcommand{\IR}{\mathbb{R}}
\newcommand{\R}{\mathbb{R}}	% Reelle Zahlen
\newcommand{\IC}{\mathbb{C}}
\newcommand{\F}{\mathcal{F}}	% Komplexe Zahlen
\newcommand{\IK}{\mathbb{K}}	% Körper
\newcommand{\ones}{\mathbb{1}}
\newcommand{\BigO}{\mathcal{O}}

% TODO Kaligrafische Buchstaben?
\newcommand{\Ic}{\mathcal{I}}
\newcommand{\Jc}{\mathcal{J}}
\newcommand{\Hc}{\mathcal{H}}
\newcommand{\Tc}{\mathcal{T}}
\newcommand{\Sc}{\mathcal{S}}
\newcommand{\Oc}{\mathcal{O}}
\newcommand{\Pc}{\mathcal{P}}
%\newcommand{\Pc}{\mathcal{P}}
\newcommand{\V}{\mathcal{V}}

% TODO...
\newcommand{\PSet}{\Pc}		% Potenzmenge

\newcommand{\ceil}[1]{\left\lceil#1\right\rceil}
\newcommand{\abs}[1]{\left|#1\right|}

\newcommand{\from}{\leftarrow}

\newcommand{\id}{\mathrm{id}}
\newcommand{\vol}{\mathrm{vol}}

\newcommand{\dist}{\mathrm{dist}}

\newcommand{\minDist}[1]{\mathrm{d}(#1)}
%	\newcommand{\dist}[2]{\mathrm{dist}_{#2}(#1)}

\newcommand{\setMid}{\,\middle|\,}

% Supresses qed-symbol in current environment
\newcommand{\noqed}{\let\qed\relax}

\newcommand{\symDiff}{\triangle}


\newcommand{\dIff}{\mathbin{~\mathop:\!\!\iff}}






%%%%%%%%%%% MISC %%%%%%%%%%%%

% Some common abbreviations (for proper spacing after the dots)
\newcommand{\cf}{cf.\ }
\newcommand{\fa}{f.a.\ }
\newcommand{\eg}{e.g.\ }
\newcommand{\Eg}{E.g.\ }
\newcommand{\ie}{i.e.\ }
\newcommand{\Ie}{I.e.\ }
\newcommand{\wrt}{w.r.t.\ }
\newcommand{\wlofg}{wlog.\ }
\newcommand{\etal}{et~al.\ }
\newcommand{\etalnb}{et~al.~}


% sidewaystable
\usepackage{rotating}

\usepackage{longtable}

\ifarxiv
	\usepackage[font={small,it}]{caption} % Kleinere Captions
\fi


% Transparente Farben
\usepackage{transparent}
\usepackage{color}
\usepackage{graphicx}

\usepackage[all]{xy}
\ifarxiv
	\usepackage{lmodern}
	\usepackage[babel]{csquotes}
\fi



%%%%%%%%%%%%%%%%%%%%%%%%%%%%%%%
\usepackage{framed,enumitem}

\clubpenalty=10000
\widowpenalty=10000
\displaywidowpenalty=10000









%%%%%%%%%%%%%%%%%%%%%%%%%%%%%%%%%

% Additional commands:

\newcommand{\tauMinDiff}{\tau_\Delta}
%\newcommand{\vol}[3]{\mathrm{vol}_{#1}(#2,#3)}
\newcommand{\inflowOver}[2]{\int_{#1}^{#2}f_e^-(\theta)d\theta}
\newcommand{\outflowOver}[2]{\int_{#1}^{#2}f_e^+(\theta)d\theta}
\newcommand{\distP}[1]{\tilde{d}_{#1}}
\newcommand{\tPmax}{\tau(P_{\max})}
\newcommand{\tPmin}{\tau(P_{\min})}



\newcommand{\pClosEL}[1]{F_{\leq #1}}

% Notation for termination time
\newcommand{\termTime}[1][]{%
	\ifthenelse{\equal{#1}{}}
	{\Theta}
	{\Theta_{\mathrm{#1}}}
}

\newcommand{\PoA}{\ensuremath{\mathrm{PoA}}}

% Big O notation:
\newcommand{\bigO}{\mathcal{O}}
\newcommand{\bigOm}{\Omega}

% Network
\newcommand{\network}{\mathcal{N}}

% 3SAT
\newcommand{\ThreeSAT}{\ensuremath{\mathtt{3SAT}}}

% Charakteristische Funktion:
\usepackage{dsfont} 
\NewDocumentCommand{\CharF}{O{}}{%
	\ifthenelse{\equal{#1}{}}%
	{\mathds{1}}%
	{\mathds{1}_{#1}}%
}


% Outgoing edges from node v
\newcommand{\edgesLeaving}[1]{\delta^+_{#1}}
% Incoming edges to node v
\newcommand{\edgesEntering}[1]{\delta^-_{#1}}

% Left- and ride-side derivatives
\newcommand{\rDeriv}[1]{\partiaL^2_+ #1}
\newcommand{\lDeriv}[1]{\partial_- #1}


%% For Computations Section
%% algorithm2e seems to have a conflict with cleveref (declaring it after cleveref}

\usepackage{siunitx} % for decimal alignment
%% New commands
\newcommand{\norm}[1]{\left\Vert#1\right\Vert}

%% TABLE
\usepackage{array}
\newcolumntype{L}[1]{>{\raggedright\arraybackslash}m{#1}}
\newcolumntype{C}[1]{>{\centering\arraybackslash}m{#1}}
\newcolumntype{R}[1]{>{\raggedleft\arraybackslash}m{#1}}



%% Special commands for side-constraint
\newcommand{\IRi}{\IR \cup \set{\infty}}

\renewcommand{\H}{H}	% Time horizon
\newcommand*\diff{\mathop{}\!\mathrm{d}} % the upright d for integrals
\DeclarePairedDelimiterX{\scalar}[2]{\langle}{\rangle}{#1, #2} % Scalar product
\newcommand{\VI}{\textrm{VI}}
\newcommand{\QVI}{\textrm{QVI}} % Variational inequality
\newcommand{\A}{\mathcal{A}} % Mapping for Variational Inequality
\usepackage{dsfont} % fuer charakteristische 1
\renewcommand{\l}{\ell}


% Makros for some terms 
%% For effective walk delay/path cost
\newcommand{\effWalkDelay}{effective walk delay}
%% For the elements of A_{i,p}(h)
\newcommand{\addmEpsDev}{admissible \epsDev} 
%% A gamma-deviation (i.e. tupel (q,\gamma,\Delta)) -- regardless of whether it is addmissible or not
\newcommand{\epsDev}{$\gamma$-deviation}
%% For elements of U_p(h,t)
\newcommand{\addmDev}{admissible deviation}
%% For the set S
\newcommand{\setS}{constraint set}

% The general side constrained dynamic equilibria
\newcommand{\SCDE}[1][]{\ifthenelse{\equal{#1}{}}{SCDE}{\ifthenelse{\equal{#1}{fulls}}{side-constrained dynamic equilibria}{side-constrained dynamic equilibrium}}}
%% For CDE where admissible deviations are those that lead to another feasible flow
%% without an argument this prints the short term, with argument "full" it prints the full name
\newcommand{\globalSCDE}[1][]{\ifthenelse{\equal{#1}{}}{strict SCDE}{\ifthenelse{\equal{#1}{fulls}}{strict side-constrained dynamic equilibria}{strict side-constrained dynamic equilibrium}}}
%% The same equiibrium but for the model with edge load constraints, i.e.:
%% A SCDE with edge load constraints wherein deviation is allowed iff the new flow is feasible for all particles
\newcommand{\globalEL}[1][]{\ifthenelse{\equal{#1}{}}{strict CDE}{\ifthenelse{\equal{#1}{fulls}}{strict capacitated dynamic equilibria}{strict capacitated dynamic equilibrium}}}
%% A SCDE with edge load constraints wherein deviation is allowed iff the new flow is feasible for the deviating particles at the time they enter an edge
\newcommand{\sCDEdf}[1][]{\ifthenelse{\equal{#1}{}}{strong BSDE}{strong dynamic Bernstein-Smith equilibrium}}
%% A SCDE with edge load constraints wherein deviation is allowed iff the new flow is feasible for the deviating particles while they travel along an edge
\newcommand{\wCDEdf}[1][]{\ifthenelse{\equal{#1}{}}{weak BSDE}{weak dynamic Bernstein-Smith equilibrium}}
%% A SCDE with edge load constraints wherein deviation is allowed iff the new path has extra room on all edges at the time they enter an edge
\newcommand{\sCDEu}[1][]{\ifthenelse{\equal{#1}{}}{strong LPDE}{strong dynamic Larsson-Patriksson equilibrium}}
%% A SCDE with edge load constraints wherein deviation is allowed iff the new path has extra room on all edges while they travel along an edge
\newcommand{\wCDEu}[1][]{\ifthenelse{\equal{#1}{}}{weak LPDE}{weak dynamic Larsson-Patriksson equilibrium}}
%% A SCDE with edge load constraints wherein deviation is allowed iff the new path has extra room on all edges when they enter an edge except for edges on a common prefix with the current path
\newcommand{\sCDEuP}[1][]{\ifthenelse{\equal{#1}{}}{strong MNSDE}{strong dynamic Marcotte-Nguyen-Schoeb equilibrium}}
%% A SCDE with edge load constraints wherein deviation is allowed iff the new path has extra room on all edges while they travel along an edge except for edges on a common prefix with the current path
\newcommand{\wCDEuP}[1][]{\ifthenelse{\equal{#1}{}}{weak MNSDE}{weak dynamic Marcotte-Nguyen-Schoeb equilibrium}}
%% A SCDE with edge load constraints wherein deviation is allowed iff the new flow is feasible for the deviating particles except for edges which are common to both paths and reached at the same time
\newcommand{\sCDEuE}[1][]{\ifthenelse{\equal{#1}{}}{strong ???}{strong ???}}

\newcommand{\HLP}[1][]{\ifthenelse{\equal{#1}{}}{HLP}{Hearn-Larsson-Patriksson equilibrium}}

\newcommand{\Sm}[1][]{\ifthenelse{\equal{#1}{}}{S}{Smith equilibrium}}
\newcommand{\Hey}[1][]{\ifthenelse{\equal{#1}{}}{BS}{Heydecker equilibrium}}
\newcommand{\BS}[1][]{\ifthenelse{\equal{#1}{}}{BS}{Bernstein-Smith equilibrium}}

\DeclareMathOperator{\supp}{supp}

\DeclareMathOperator{\esssup}{ess\,sup}
\DeclareMathOperator{\essinf}{ess\,inf}

\newcommand{\tStart}{t_0}
\newcommand{\tEnd}{t_f}
\newcommand{\planningInterval}{[\tStart,\tEnd]}

\newcommand{\truncated}[2]{#1\big\vert^{#2}}

\newcommand{\shiftF}{\gamma^-}
\newcommand{\shiftT}{\gamma^+}
\newcommand{\shiftN}{\gamma}

\newcommand{\emptyarg}{\,\cdot\,}

% Notation for flow volume
\newcommand{\flowVolume}[1][]{%
	\ifthenelse{\equal{#1}{}}
	{x}
	{x_{#1}}
}


\crefname{asmpt}{assumption}{assumptions}
\Crefname{asmpt}{Assumption}{Assumptions}
