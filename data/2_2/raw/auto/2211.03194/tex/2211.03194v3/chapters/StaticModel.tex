% !TeX spellcheck = en_GB
%!TEX root = ../side-constrained.tex

\section{The Static Model}\label{sec:static}
We are given a directed graph $G=(V,E)$
and a set of populations or commodities $I:= \{1, \dots,
n\}$, where each commodity $i \in I$ has a demand $d_i> 0$ that has
to be routed from a source $s_i \in V$ to a destination $t_i \in V$.
The demand interval $[0,d_i]$ represents a continuum of infinitesimally small agents each acting independently
choosing a cost minimal $s_i$,$t_i$-path. 
There are continuous and nondecreasing cost  functions $\ell_{e}: \R^E \rightarrow \R_{\geq 0}, e\in E$.
A \emph{path flow} for commodity~$i\in I$ is a nonnegative vector
$x_i \in \R^{|\Pc_i|}_{\geq 0}$ that  lives in the path flow polytope: 
\begin{align*}
X_i=\left\{x_i\in \R_{\geq 0}^{|\Pc_i|}\;\middle\vert \; \sum_{p\in\Pc_i} x_{i,p}=d_i \right\},
\end{align*}
where $\Pc_i$ denotes the set of simple $s_i,t_i$-paths in $G$ and
$\Pc \coloneqq \bigcup_{i \in I}\Pc_i$.
We assume that every $t_i$ is reachable in $G$ from $s_i$
for all $i\in I$, thus, $X_i\neq \emptyset$ for all $i\in I$.
Given a  path flow vector $x\in X:=\times_{i\in I}X_i$, the cost of a path $p\in\Pc_i$,
 is defined as
$\ell_{p}(x):=\sum_{e\in p}\ell_{e}(x),$
where $ x_e:=\sum_{i\in I}\sum_{p\in\Pc_i : e\in p} x_{i,p}$ is the aggregated load of edge $e\in E$.
\begin{definition}
A path flow $ x^*\in X$ is a \emph{Wardrop equilibrium} if for all $i\in I$:
\[ \ell_{p}( x^*)\leq \ell_{q}( x^*) \text{ for all }p,q\in \Pc_i \text{ with }x^*_{i,p}>0.\]
\end{definition}
The interpretation here is that all agents are travelling along cost minimal paths
given the overall load vector $(x^*_e)_{e\in E}$.
One can characterize Wardrop equilibria by means of variational inequalities
as follows (cf.~Patriksson~\cite[Sec. 3.2.1]{Patriksson1994tap}):
\begin{lemma}\label{lem:vi-static}
The following statements are equivalent:
\begin{enumerate}
\item $x^*\in X$ 
is a Wardrop equilibrium.
\item $\scalar{\ell( x^*)}{ x^*- y} \leq 0  \text{ for all }y\in X, \text{ with } \ell( x^*) \coloneqq (\ell_{p}( x^*))_{p\in\Pc}$ and $\scalar{\emptyarg}{\emptyarg}$ denoting the scalar product on $\IR^{\abs{\Pc}}$.
\end{enumerate}
\end{lemma}


For the case of  separable latency functions $\ell$,  Dafermos and Sparrow~\cite{DafS69}  related Wardrop equilibria to Nash equilibria of an associated noncooperative game.
 Formally, for $\varepsilon>0$ and $p,q\in \Pc_i$ with $x_p>0$, let
\[ x_w(\varepsilon,p,q):=\begin{cases}x_w-\varepsilon, & \text{ if }w=p\\
x_w+\varepsilon, & \text{ if }w=q\\
x_w, & \text{ else. }\end{cases}\]

\begin{definition}\label{def:dafermos-sparrow}
A  path flow $x^*\in Z$ is a \emph{Nash equilibrium}, if for all $ p,q\in \Pc_i, x^*_{i,p}>0,\varepsilon\in (0,x^*_{i,p}], i\in I$, we have
\[ \ell_{p}( x^*)\leq \ell_{q}(x^*(\varepsilon,p,q)). \]
\end{definition}
Dafermos and Sparrow~\cite{DafS69} showed that for continuous and separable latency functions, Nash equilibria and
Wardrop equilibria coincide.

If $\ell$ is  separable and non-decreasing,  it is the gradient of the Beckmann-McGuire-Winsten potential function  and one can further characterize Wardrop equilibria
as optimal solutions to a convex optimization problem:
\begin{lemma}[cf.~Dafermos~\cite{dafermos1980traffic}]\label{lem:dafermos}
The following statements are equivalent:
\begin{enumerate}
\item $x^*\in X$ 
is a Wardrop equilibrium.
\item $x^*\in\arg\min_{x\in X} \Big\{  \sum_{e\in E}\int_{0}^{ x_e} \ell_{e}(z) \diff z  \Big\} $.
\item $\scalar{\ell( x^*)}{ x^*- y} \leq 0  \text{ for all }y\in X$.
\end{enumerate}
\end{lemma}


\subsection{Side-Constrained Traffic Equilibria}
Suppose we have hard edge capacities $c_e\geq 0, e\in E$
which need to be satisfied for a flow $x\in X$ to be \emph{capacity-feasible}, that is,
$ x_e\leq c_e, e\in E$. Let $Z:=\{x \in X \vert x_e \leq c_e, e\in E\}$ denote the
set of capacity-feasible path flows.
Following Patriksson~\cite[73]{Patriksson1994tap} and Larsson and Patriksson~\cite{Larsson95} (which we abbreviate henceforth with LP),
a side-constrained equilibrium can be defined
via the notion of \emph{saturated} and \emph{unsaturated} paths.
Given a path flow $x\in Z$, a path $p$ is saturated, if in contains an edge $e\in p$ with $ x_e= c_e$ and, conversely, a path $p$ is unsaturated,
if $ x_e<c_e$ for all $e\in p$.

\begin{definition}\label{def:weakWE}
A path flow $x^*\in Z$  is a side-constrained \emph{LP-equilibrium}, if
\[ \ell_{p}( x^*)\leq \ell_{q}( x^*) \text{ for all }p,q\in \Pc_i \text{ with } x^*_{i,p}>0 \text{ and } q \text{ unsaturated}.\] 
\end{definition}

For our subsequent discussion it is worth reformulating
the definition of an LP-equi\-li\-bri\-um in terms of feasible \emph{additive} $\varepsilon$-deviations
in the spirit of Dafermos and Sparrow~\cite{DafS69}.
For $\varepsilon>0$ and $q\in \Pc_i$, let
\[ x_w(\varepsilon,q):=\begin{cases}x_w+\varepsilon, & \text{ if }w=q\\
x_w, & \text{ else. }\end{cases}\]
Define  \[ \tilde{\ell}_e(x):=\begin{cases} \ell_e(x), & \text{ if }x_e\leq c_e\\
 +\infty, &\text{else.}\end{cases}\]
We obtain the following equivalent definition:
\begin{lemma} \label{lem:lp}
A  path flow $x^*\in Z$ is a side-constrained LP-equilibrium iff for all $p,q\in \Pc_i, x^*_{i,p}>0, i\in I$, we have
\[ \tilde\ell_{p}( x^*)\leq \tilde\ell_{q}(x^{*}(\varepsilon,q)) \text{ for all }\varepsilon\in (0,x^*_{i,p}]. \]
\end{lemma}


As observed by \citefull{Marcotte04}, this definition has the drawback of admitting
 rather artificial equilibrium flows. Consider for instance
a graph with one edge followed by two edges in parallel.
If the capacity of the first edge is saturated, then \emph{any}
feasible flow is an LP equilibrium no matter how the flow
is distributed on the two subsequent edges. This leads to unrealistic
equilibria in case one of the two edges is more expensive but carries flow
and the other (cheaper) one has free capacity.
An alternative equilibrium concept proposed by Smith~\cite{Smith84} 
avoids this problem by allowing for path changes of $\varepsilon>0$
units of flow provided that \emph{after the change}, the resulting flow is still
feasible (this corresponds to considering deviations of the form $x^*(\varepsilon,p,q)$
in Lemma~\ref{lem:lp}).
This concept has the drawback that it allows for coordinated
deviations of bundles of users, which is unrealistic and, as shown by Smith,
leads to non-existence of equilibria for monotonic, continuous and non-separable latency functions.

In response to Smith's equilibrium concept, Bernstein and Smith (BS)~\cite{BernsteinS94}  proposed an alternative equilibrium concept addressing the issue of possible coordinated
deviations of bundles of users. They added the condition that only deviations need to be considered that involve ``small enough''
bundles of users.\footnote{Heydecker~\cite{Heydecker86} introduced yet another equilibrium definition
in response to Smith's definition, where the path costs of $p$ and $q$ are compared after the route switch of $\varepsilon$ units of flow.}
\begin{definition} (Bernstein and Smith~\cite[Definition~2]{BernsteinS94})\label{def:BS}
A  path flow $x^*\in Z$ is a side-constrained \emph{BS-equilibrium}, if for all $p,q\in \Pc_i, x^*_{i,p}>0, i\in I$, we have
\[ \tilde\ell_{p}( x^*)\leq \lim\inf_{\varepsilon\downarrow 0} \tilde\ell_{q}(x^*(\varepsilon,p,q)). \]
\end{definition}
Note that the original definition of Bernstein and Smith~\cite[Definition~2]{BernsteinS94} does not involve capacities but by using latency functions $\tilde\ell$ that jump to $+\infty$ as soon as arc capacities are exceeded
an equilibrium will be capacity-feasible.
Following \citefull{CorreaCapEqInStaticFlows}, the definition of BS-equilibrium can be rephrased as ``no arbitrarily small bundle of drivers on a common path can strictly decrease its cost by switching to another path''. 

\subsection{Beckmann-McGuire-Winsten Equilibria and Variational Inequalities}
For the case of separable latency functions, a subset of BS-equilibria can be characterized as solutions
to an associated convex optimization problem, where the Beckmann-McGuire-Winsten potential function over the
convex space of  capacity-feasible flows is minimized:

\begin{align}
\label{beckmann-opt}\tag{BMW}
\min &  \sum_{e\in E}\int_{0}^{ x_e} \ell_{e}(z) \diff z  \\
\text{s.t.: }& x\in Z.\notag
\end{align}
We obtain the following well-known characterization (see, \eg Patriksson~\cite{Patriksson1994tap}).
\begin{lemma}
\begin{enumerate}
\item $x^*\in Z$ is optimal for~\eqref{beckmann-opt} iff $x^*$ solves the following variational inequality
\begin{equation}\label{var-static}
\scalar{\ell( x^*)}{ x^*- y} \leq 0 \text{ for all $y\in Z$}.
\end{equation}
\item Every optimal $x^*\in Z$  to~\eqref{beckmann-opt}
is a  BS-equilibrium but not vice versa.
\end{enumerate}
\end{lemma}
The  second statement  implies the existence of BS-equilibria:
The space $Z$ is non-empty and compact and by the continuity of the objective in~\eqref{beckmann-opt}, the theorem of
Weierstra\ss\ implies that~\eqref{beckmann-opt} admits an optimal solution (which then is also
a BS-equilibrium).
The first and second statement together show that there are BS-equilibria
which need not solve the variational inequality stated in~\eqref{var-static}.
\citefulls{CorreaCapEqInStaticFlows} termed equilibria coming from optimal solutions
to~\eqref{beckmann-opt} as \emph{Beckmann-McGuire-Winsten} (BMW) equilibria while
\citefulls{Marcotte04} termed them  \emph{Hearn-Larsson-Patriksson} equilibria.
A further useful interpretation of solutions to~\eqref{beckmann-opt} is the use of the dual variables
associated with the capacity constraints $x\leq c$. It was shown in several works (\cf Hearn~\cite{Hearn80},~Daganzo~\cite{Daganzo1977}, Larsson and Patriksson~\cite{Larsson99})
that a  BMW-equilibrium can be interpreted as an unconstrained equilibrium, if the dual variables
are added as additional penalty terms to the user's cost function. In addition to this natural interpretation and possible implementation of BMW-equilibria via prices, the efficiency properties of BMW-equilibria in terms of induced total travel times are particularly appealing compared
to other possible side-constrained equilibria, see \citefulls{CorreaCapEqInStaticFlows}.
\subsection{Discussion}
It is very instructive to restate a remark made by \citefulls{Marcotte04}:
``Defining equilibrium meaningfully in side-constrained transportation networks represents a nontrivial task.''
Indeed, the above presentation already shows some subtle issues arising when defining
a sound notion of side-constrained traffic equilibria. While only a few works
formally introduced a behavioral equilibrium concept involving the notion of feasible $\varepsilon$-deviations (as in Bernstein and Smith~\cite{BernsteinS94}, Dafermos and Sparrow~\cite{DafS69}, Smith~\cite{Smith84}, Heydecker~\cite{Heydecker86}), 
most of the works in the transportation science literature used directly the
optimization formulations of the type~\eqref{beckmann-opt} or the variational inequality formulations as the definition of a side-constrained user equilibrium  (cf.~Daganzo~\cite{Daganzo1977}, Hearn and Ramana~\cite{Hearn98solving} or Larsson and Patriksson~\cite[Remark 11]{Larsson99}).\footnote{We use the term ``behavioral equilibrium concept'' 
in accordance with the concept of a Nash equilibrium for a noncooperative game
involving  payoff functions (or preference relations) and strategy spaces which are implicitly defined
via feasible deviations as in Dafermos and Sparrow~\cite{DafS69}. The term ``behavioral'' as used in Larsson and Patriksson~\cite{Larsson1998b} has the following different meaning:
``As such, these models are behavioural, in the sense that the effects of the side constraints are assumed to be immediately transferable to the perception of travel costs among the trip-makers, for example as queueing delays; their solutions are also characterized and interpreted as flows satisfying the Wardrop equilibrium conditions in terms of generalized travel costs that include link queueing delays.''}
This observation was also made in \citefulls{CorreaCapEqInStaticFlows}:
``It is interesting to note that the model [\eqref{beckmann-opt}] has been used before without the
formal introduction of the concept of a capacitated user equilibrium.'' 

 As we will see later in Section~\ref{sec:counter}, within the realm of \emph{ dynamic traffic assignments},
defining side-constrained dynamic equilibria via variational inequalities is not only imprecise (as it excludes other ``equilibria'' being not of this type)  but also 
leads to flawed statements about equilibrium existence and their characterizations.
We show that the natural infinite dimensional variational inequality formulation 
for a class of volume-constrained dynamic traffic assignments need per se
not be related to an equilibrium solution. 
Instead our goal in this work is to transfer behavioral equilibrium concepts in the spirit of Dafermos and Sparrow~\cite{DafS69},
Bernstein and Smith~\cite{BernsteinS94}, Smith~\cite{Smith84} and Heydecker~\cite{Heydecker86} to the domain of \emph{dynamic} side-constrained traffic assignments.
