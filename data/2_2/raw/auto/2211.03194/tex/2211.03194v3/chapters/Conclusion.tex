% !TeX spellcheck = en_GB
%!TEX root = ../side-constrained.tex

\section{Conclusion}

We provided a counterexample to a claimed existence result for dynamic equilibria with side constraints. The implications of this counterexample were shown to be severe since solutions to the canonical infinite dimensional variational inequality are in some sense useless and other approaches seem to be necessary. 
We then established a general framework for defining side-constrained dynamic equilibria based on two key objects: A \setS{} $S$ containing all feasible flows (given as walk inflows) and correspondences $A_p$ providing the flow-dependent set of \addmEpsDev s. We showed that this equilibrium concept not only encompasses the known unconstrained equilibria with and without departure time choice and capacitated dynamic equilibria with convex \setS{}s but also allows for a whole range of new dynamic equilibria inspired by static side-constrained equilibria.
We provided conditions under which they can be characterized as solutions to a quasi-variational or even a variational inequality. The latter characterization then also gave rise to a first existence result for certain side-constrained dynamic equilibria with convex \setS.
Finally, we turned to equilibria wherein the side-constraints are given by time-varying edge-load constraints. To deal with the non-convexity of the \setS{}, we employed an augmented Lagrangian approach by relaxing the hard edge-load-capacities and replacing them by penalty functions. We demonstrated that these existence results apply, in particular, for the widely used Vickrey point queue model as well as the linear edge delay model.

Several important questions remain open. First of all, it would be interesting to find an existence result for BSDE similar to \Cref{thm:ExistenceFDAddSpaceExCP} for LPDE and MNSDE. The main obstacle to obtaining such a result seems to be the fact that for BSDE, the definition of \addmEpsDev s involves the network loading which, in general, is a very complex mapping and, even for well-studied flow models, is not fully understood yet. Note that, due to \Cref{prop:RelationshipsOfCDE}, such a result would also directly imply existence of \globalEL{} as well as providing an alternative proof for the existence of LPDE. Another aspect is the multiplicity of equilibria and
the issue of selecting a particular type of equilibrium having desirable properties.
It is an interesting research direction to characterize equilibrium concepts
that admit equilibrium selection via appropriate optimization or optimal control reformulations
whose optimal solutions provide such desirable properties.