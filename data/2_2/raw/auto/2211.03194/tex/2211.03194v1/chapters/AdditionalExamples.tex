% !TeX spellcheck = en_GB
%!TEX root = ../side-constrained.tex

\section{An Example for the Usefulness of Cycles}

The following example shows why particles may prefer to travel along walks containing cycles in networks with capacity constraints. This means that the equilibria in such a network can be different depending on whether the strategy space of the particles only includes paths or also walks containing cycles.

\begin{figure}[h]
	\centering
	\begin{tikzpicture}
	\node[namedVertex] (s) at (0,0) {$s$};
	\node[namedVertex] (t) at (4,0) {$t$};
	\node[namedVertex] (v)  at (0,2) {$v$};
	
	\draw[edge] (s) -- node[below]{$e_1$} node[above](e){$\tau_{e_1}=1$} (t);
	\draw[edge] (s) to[bend right=80] node[above]{$e_2$} node[below]{$\tau_{e_2}=10$} (t);
	\draw[edge] (s) to[bend left=20] node[below, sloped]{$\tau_{sv}=1$} (v);
	\draw[edge] (v) to[bend left=20] node[below, sloped]{$\tau_{vs}=1$} (s);
	
	\node[above of=e, anchor=south,node distance=.2cm] {
		\begin{tikzpicture}[scale=1,solid,black,
			declare function={
				c(\x)= 1;			
			}]
			
			
			\begin{axis}[xmin=0,xmax=3.5,ymax=2, ymin=0, samples=500,width=3.5cm,height=3cm,
				axis x line*=bottom, axis y line*=left, axis lines=middle, xtick={1,2,3,4}, ytick={1}]
				\addplot[blue, ultra thick,domain=0:5] {c(x)} node[above,pos=.5]{$c_{e_1}$};
			\end{axis}
			
		\end{tikzpicture}
	};
	
	\node[left of=s,blue,node distance=1.5cm](){$r=2\cdot\CharF[{[0,1]}]$};
\end{tikzpicture}
	\caption{A single-commodity network with fixed network inflow rate where allowing particles to travel along cycles changes (improves) the equilibrium flow. All edges have a flow independent travel time as given in the figure. The capacities of all edges except for edge $e_1$ are infinite.}\label{fig:ImprovementByCycles}
\end{figure}

\begin{example}\label{ex:ImprovementByCycles}
	The network in \Cref{fig:ImprovementByCycles} is an example for a single commodity network with volume-constraints on the edges where it makes a difference for the resulting equilibrium whether travelling along cycles is allowed or not. If cycles are not allowed, only half of the flow can use the short edge $e_1$ towards the sink while the rest of the flow has to take the much longer edge $e_2$. If, on the other hand, cycles are allowed, particles can use the cycle $s\to v \to s$ to essentially wait at the source node until there is again room on edge $e_1$. In other word, in this example individual particles prefer to travel along a cycle.
\end{example}