%%%%%%%%%%%%%%%%%%%%%%%%%%%%%%%%%%%%%%%%%%%%%%%%%%%%%%%%%%%%%%%%%%%%%%%%
%    INSTITUTE OF PHYSICS PUBLISHING                                   %
%                                                                      %
%   `Preparing an article for publication in an Institute of Physics   %
%    Publishing journal using LaTeX'                                   %
%                                                                      %
%    LaTeX source code `ioplau2e.tex' used to generate `author         %
%    guidelines', the documentation explaining and demonstrating use   %
%    of the Institute of Physics Publishing LaTeX preprint files       %
%    `iopart.cls, iopart12.clo and iopart10.clo'.                      %
%                                                                      %
%    `ioplau2e.tex' itself uses LaTeX with `iopart.cls'                %
%                                                                      %
%%%%%%%%%%%%%%%%%%%%%%%%%%%%%%%%%%
%
%
% First we have a character check
%
% ! exclamation mark    " double quote
% # hash                ` opening quote (grave)
% & ampersand           ' closing quote (acute)
% $ dollar              % percent
% ( open parenthesis    ) close paren.
% - hyphen              = equals sign
% | vertical bar        ~ tilde
% @ at sign             _ underscore
% { open curly brace    } close curly
% [ open square         ] close square bracket
% + plus sign           ; semi-colon
% * asterisk            : colon
% < open angle bracket  > close angle
% , comma               . full stop
% ? question mark       / forward slash
% \ backslash           ^ circumflex
%
% ABCDEFGHIJKLMNOPQRSTUVWXYZ
% abcdefghijklmnopqrstuvwxyz
% 1234567890
%
%%%%%%%%%%%%%%%%%%%%%%%%%%%%%%%%%%%%%%%%%%%%%%%%%%%%%%%%%%%%%%%%%%%
%
\documentclass[twocolumn,12pt]{iopart}
\newcommand{\gguide}{{\it Preparing graphics for IOP Publishing journals}}
%Uncomment next line if AMS fonts required
%\usepackage{iopams}
\usepackage{graphicx}

\begin{document}

\title[]{Complete analysis of hyperentangled Bell state in three degrees of freedom}
\author{Zhi Zeng$^{1,2,^*}$}
\address{{$^1$Institute of Signal Processing and Transmission, Nanjing University of Posts and Telecommunications, Nanjing 210003, China}  \\
{$^2$Key Lab of Broadband Wireless Communication and Sensor Network Technology, Ministry of Education, Nanjing University of Posts and Telecommunications, Nanjing 210003, China}}
\eads{\mailto{zengzhiphy@yeah.net}}
\vspace{10pt}
%\begin{indented}
%\item[]February 2014
%\end{indented}

\begin{abstract}
We present an efficient scheme for the complete hyperentangled Bell state analysis (HBSA) of photon system in polarization and two longitudinal momentum degrees of freedom (DOFs). In the process of distinguishing the 64 hyperentangled Bell states in three DOFs, the weak cross-Kerr nonlinearity and self-assisted mechanism are both utilized, which can make our scheme simple and realizable. We also discuss the application of this complete HBSA scheme in quantum teleportation that based on hyperentangled state in three DOFs.
\end{abstract}

\section{Introduction}
Hyperentangled state of photon system in more than one degree of freedom (DOF) has already been widely exploited for the high-capacity quantum computation and quantum communication \cite{1}. In many hyperentanglement-based quantum information processing protocols, the deterministic distinguishing of a set of orthogonal hyperentangled Bell state is often required for getting the information. The complete hyperentangled Bell state analysis (HBSA) for two-photon system has attracted much attention in the past years, and research has shown that it is impossible to accomplish the complete HBSA if only linear optics is utilized \cite{2,3,4}. An efficient way to deal with this problem is using the quantum nonlinear interactions, such as cross-Kerr nonlinearity, quantum dot spin in optical microcavity and nitrogen-vacancy center in resonator \cite{5,6,7,8,9,10,11}. These nonlinear interactions can be used for constructing the quantum parity-check gate and swap gate, both of which can be useful for the complete hyperentangled state analysis. However, most of the existing complete HBSA schemes are working with photonic hyperentanglement in just two different DOFs, only a few schemes focus on the HBSA for photon system in more than two DOFs \cite{12,13,14,15}. In 2016, Liu \emph{et al.} presented the nondestructive scheme to completely distinguish the hyperentangled Bell states in three DOFs with the help of cross-Kerr nonlinearity \cite{12}. In 2018, Wang \emph{et al.} proposed the deterministic hyperentangled state protocols for photons simultaneously entangled in polarization, spatial-mode and time-bin DOFs \cite{13}. In 2021, Zhang \emph{et al.} presented the nondestructive HBSA protocol for hyperentanglement in three DOFs, resorting to the parity-check quantum nondemolition detector \cite{14}. Recently, Zhou \emph{et al.} proposed a simplified scheme for distinguishing the two-photon hyperentangled Bell states in three DOFs using the quantum dot-cavity interactions \cite{15}.

Compared with the traditional hyperentangled state in two DOFs, hyperentanglement in three DOFs can largely increase the channel capacity of quantum communication, and it has already been prepared and manipulated in experiment \cite{16,17}. In this paper, we present a simple and efficient scheme for the complete analysis of hyperentangled Bell state in polarization and two longitudinal momentum DOFs. This six-qubit hyperentangled state has been experimentally realized by Vallone \emph{et al.} in 2009 \cite{17}, and can be exploited to investigate the distinguishability of HBSA with linear optics and auxiliary entanglement \cite{3}. In our complete HBSA scheme for two-photon system in three DOFs, all the 64 hyperentangled Bell states will be completely distinguished. By using the weak cross-Kerr nonlinearity and self-assisted mechanism, the discrimination process is greatly simplified and the whole scheme becomes realizable. Our complete HBSA scheme can be directly used for the quantum teleportation of photon system in three DOFs, and will have useful application in other high-capacity quantum information processing tasks.

\section{Complete HBSA for hyperentanglement in three degrees of freedom}
The general form of two-photon hyperentangled Bell state in polarization and two longitudinal momentum DOFs can be written as
\begin{eqnarray}
|\Psi\rangle_{AB} = |\Psi\rangle_{P} \otimes |\Psi\rangle_{F} \otimes |\Psi\rangle_{S}.
\end{eqnarray}
Here, $A$ and $B$ are the two entangled photons. $P$, $F$ and $S$ denote the polarization DOF, the first longitudinal momentum DOF and the second longitudinal momentum DOF, respectively. $|\Psi\rangle_{P}$ is one of the four Bell states in polarization DOF,
\begin{eqnarray}
|\phi^{\pm}\rangle_{P} = \frac{1}{\sqrt 2} (|HH\rangle \pm |VV\rangle)_{AB}, \nonumber \\
|\psi^{\pm}\rangle_{P} = \frac{1}{\sqrt 2} (|HV\rangle \pm |VH\rangle)_{AB},
\end{eqnarray}
where $H$ and $V$ are the horizontal and vertical polarization states of photon, respectively. $|\Psi\rangle_{F}$ is one of the four Bell states in the first longitudinal momentum DOF,
\begin{eqnarray}
|\phi^{\pm}\rangle_{F} = \frac{1}{\sqrt 2} (|EE\rangle \pm |II\rangle)_{AB}, \nonumber \\
|\psi^{\pm}\rangle_{F} = \frac{1}{\sqrt 2} (|EI\rangle \pm |IE\rangle)_{AB},
\end{eqnarray}
where $E$ and $I$ are the external and internal spatial-modes of photon, respectively. $|\Psi\rangle_{S}$ is one of the four Bell states in the second longitudinal momentum DOF,
\begin{eqnarray}
|\phi^{\pm}\rangle_{S} = \frac{1}{\sqrt 2} (|rr\rangle \pm |ll\rangle)_{AB}, \nonumber \\
|\psi^{\pm}\rangle_{S} = \frac{1}{\sqrt 2} (|rl\rangle \pm |lr\rangle)_{AB},
\end{eqnarray}
where $r$ and $l$ are the right and left spatial-modes of photon, respectively. Considering the three DOFs at the same time, there are 64 orthogonal hyperentangled Bell states, which will be unambiguously distinguished in the following text.

Before describing our complete HBSA scheme, we briefly introduce the principle of nondestructive detection of photon number, which is achieved by using the weak cross-Kerr nonlinearity. The interaction between a signal state $|\varphi\rangle_s$ and a probe coherent state $|\alpha\rangle_p$ in the nonlinear medium can be described with the Hamiltonian \cite{Kerr1}
\begin{eqnarray}
H = \hbar\chi a^\dag_sa_sa^\dag_pa_p.
\end{eqnarray}
Here, $a^\dag_s$ ($a^\dag_p$) and $a_s$ ($a_p$) are the creation and destruction operators of the signal (probe) state, respectively. $\chi$ is the coupling strength of nonlinearity, which depends on the property of nonlinear material. After the interaction with the signal state in nonlinear medium, the probe coherent state will get a phase shift, which is proportional to the photon number $N$ of the signal state,
\begin{eqnarray}
|\alpha\rangle_p \rightarrow |\alpha e^{iN\theta}\rangle.
\end{eqnarray}
Here $\theta = \chi t$ ($t$ is the interaction time). When we use the $X$-quadrature measurement on the phase shift of probe coherent state, the photon number of signal state can be read out without destroying the photons.

\begin{figure}%[!htbp]
\centering
\includegraphics*[width=0.9\textwidth]{Fig1.pdf}
\caption{Schematic diagram of the setup for the complete analysis of the 64 hyperentangled Bell states in three DOFs. The 50:50 beam splitter (BS) performs the Hadamard operation [$|x_1\rangle\rightarrow(|x_1\rangle + |x_2\rangle)/{\sqrt 2},|x_2\rangle\rightarrow(|x_1\rangle - |x_2\rangle)/{\sqrt 2}$] on the spatial-mode state of photon. The polarization beam splitter (PBS) at $0^{\circ}$ transmits the $|H\rangle$ photon, and reflects the $|V\rangle$ photon. The PBS orientated at $45^{\circ}$ transmits $|+\rangle=(|H\rangle + |V\rangle)/{\sqrt 2}$ and reflects $|-\rangle=(|H\rangle - |V\rangle)/{\sqrt 2}$. After interacting with coherent states and passing through linear optical elements, the two entangled photons will be detected by the single photon detectors.}
\end{figure}

The setup of our complete HBSA scheme for three-DOF hyperentanglement is shown in Fig. 1. After the two photons $A$ and $B$ interact with the coherent states $|\alpha_1\rangle$ and $|\alpha_2\rangle$, the state of the collective system evolves as
\begin{eqnarray}
|\Psi\rangle_{P}|\phi^{\pm}\rangle_{F}|\phi^{\pm}\rangle_{S}|\alpha_1\rangle|\alpha_2\rangle &\rightarrow&  |\Psi\rangle_{P}|\phi^{\pm}\rangle_{F}|\phi^{\pm}\rangle_{S}|\alpha_1\rangle|\alpha_2\rangle,   \nonumber \\
|\Psi\rangle_{P}|\phi^{\pm}\rangle_{F}|\psi^{\pm}\rangle_{S}|\alpha_1\rangle|\alpha_2\rangle &\rightarrow&  |\Psi\rangle_{P}|\phi^{\pm}\rangle_{F}|\psi^{\pm}\rangle_{S}|\alpha_1\rangle|\alpha_2e^{\pm i\theta}\rangle,   \nonumber \\
|\Psi\rangle_{P}|\psi^{\pm}\rangle_{F}|\phi^{\pm}\rangle_{S}|\alpha_1\rangle|\alpha_2\rangle &\rightarrow&  |\Psi\rangle_{P}|\psi^{\pm}\rangle_{F}|\phi^{\pm}\rangle_{S}|\alpha_1e^{\pm i\theta}\rangle|\alpha_2\rangle,   \nonumber \\
|\Psi\rangle_{P}|\psi^{\pm}\rangle_{F}|\psi^{\pm}\rangle_{S}|\alpha_1\rangle|\alpha_2\rangle &\rightarrow&  |\Psi\rangle_{P}|\psi^{\pm}\rangle_{F}|\psi^{\pm}\rangle_{S}|\alpha_1e^{\pm i\theta}\rangle|\alpha_2e^{\pm i\theta}\rangle.
\end{eqnarray}
The polarization entanglement is invariant during the interaction, and the parity information of entanglement in the two longitudinal momentum DOFs can be determined through the two coherent states. Then, the two photons will pass through BSs, interact with coherent state $|\alpha_3\rangle$, and the state of the collective system evolves as
\begin{eqnarray}
|\Psi\rangle_{P}|\phi^{+}\rangle_{F}|\Psi\rangle_{S}|\alpha_3\rangle &\rightarrow& |\Psi\rangle_{P}|\phi^{+}\rangle_{F}|\Psi\rangle_{S}|\alpha_3\rangle,   \nonumber \\
|\Psi\rangle_{P}|\phi^{-}\rangle_{F}|\Psi\rangle_{S}|\alpha_3\rangle &\rightarrow& |\Psi\rangle_{P}|\psi^{+}\rangle_{F}|\Psi\rangle_{S}|\alpha_3e^{\pm i\theta}\rangle,   \nonumber \\
|\Psi\rangle_{P}|\psi^{+}\rangle_{F}|\Psi\rangle_{S}|\alpha_3\rangle &\rightarrow& |\Psi\rangle_{P}|\phi^{-}\rangle_{F}|\Psi\rangle_{S}|\alpha_3\rangle,   \nonumber \\
|\Psi\rangle_{P}|\psi^{-}\rangle_{F}|\Psi\rangle_{S}|\alpha_3\rangle &\rightarrow& |\Psi\rangle_{P}|\psi^{-}\rangle_{F}|\Psi\rangle_{S}|\alpha_3e^{\pm i\theta}\rangle.
\end{eqnarray}
We can find that the entanglement in polarization and the second longitudinal momentum DOFs are both unchanged, and the phase information of entanglement in the first longitudinal momentum DOF can be determined through coherent state $|\alpha_3\rangle$. With the help of measurements on the three coherent states, the 64 hyperentangled Bell states in three DOFs can be divided into eight groups, as shown in Table 1.

\begin{table}
\centering\caption{Relations between the initial states and phase shifts of the three coherent states.}
\begin{tabular}{cc ccccccccccc}
\hline
Initial states & & & & $|\alpha_1\rangle$ & & & & $|\alpha_2\rangle$ & & & & $|\alpha_3\rangle$ \\
\hline
$|\Psi\rangle_{P}|\phi^{+}\rangle_{F}|\phi^{\pm}\rangle_{S}$ &&&& 0 &&&& 0 &&&& 0 \\
$|\Psi\rangle_{P}|\phi^{-}\rangle_{F}|\phi^{\pm}\rangle_{S}$ &&&& 0 &&&& 0 &&&& $\pm\theta$ \\
$|\Psi\rangle_{P}|\phi^{+}\rangle_{F}|\psi^{\pm}\rangle_{S}$ &&&& 0 &&&& $\pm\theta$ &&&& 0 \\
$|\Psi\rangle_{P}|\phi^{-}\rangle_{F}|\psi^{\pm}\rangle_{S}$ &&&& 0 &&&& $\pm\theta$ &&&& $\pm\theta$ \\
$|\Psi\rangle_{P}|\psi^{+}\rangle_{F}|\phi^{\pm}\rangle_{S}$ &&&& $\pm\theta$ &&&& 0 &&&& 0 \\
$|\Psi\rangle_{P}|\psi^{-}\rangle_{F}|\phi^{\pm}\rangle_{S}$ &&&& $\pm\theta$ &&&& 0 &&&& $\pm\theta$ \\
$|\Psi\rangle_{P}|\psi^{+}\rangle_{F}|\psi^{\pm}\rangle_{S}$ &&&& $\pm\theta$ &&&& $\pm\theta$ &&&& 0 \\
$|\Psi\rangle_{P}|\psi^{-}\rangle_{F}|\psi^{\pm}\rangle_{S}$ &&&& $\pm\theta$ &&&& $\pm\theta$ &&&& $\pm\theta$ \\
\hline
\end{tabular}
\end{table}

There are eight hyperentangled Bell states in each group, which can be distinguished by using linear optical element and single photon detector. Here, We take the first group in Table 1 as an example to illustrate. After passing through BSs, PBSs and BSs, the evolution of these initial eight states can be expressed as
\begin{eqnarray}
|\phi^{+}\rangle_{P}|\phi^{+}\rangle_{F}|\phi^{+}\rangle_{S} &\rightarrow&  \frac{1}{2\sqrt 2}(|++\rangle + |--\rangle)(|EE\rangle + |II\rangle)(|rr\rangle + |ll\rangle)_{AB},   \nonumber \\
|\phi^{-}\rangle_{P}|\phi^{+}\rangle_{F}|\phi^{+}\rangle_{S} &\rightarrow&  \frac{1}{2\sqrt 2}(|+-\rangle + |-+\rangle)(|EE\rangle + |II\rangle)(|rr\rangle + |ll\rangle)_{AB},   \nonumber \\
|\psi^{+}\rangle_{P}|\phi^{+}\rangle_{F}|\phi^{+}\rangle_{S} &\rightarrow&  \frac{1}{2\sqrt 2}(|++\rangle - |--\rangle)(|EI\rangle + |IE\rangle)(|rr\rangle + |ll\rangle)_{AB},   \nonumber \\
|\psi^{-}\rangle_{P}|\phi^{+}\rangle_{F}|\phi^{+}\rangle_{S} &\rightarrow&  \frac{1}{2\sqrt 2}(|+-\rangle - |-+\rangle)(|EI\rangle + |IE\rangle)(|rr\rangle + |ll\rangle)_{AB},   \nonumber \\
|\phi^{+}\rangle_{P}|\phi^{+}\rangle_{F}|\phi^{-}\rangle_{S} &\rightarrow&  \frac{1}{2\sqrt 2}(|++\rangle + |--\rangle)(|EE\rangle + |II\rangle)(|rl\rangle + |lr\rangle)_{AB},   \nonumber \\
|\phi^{-}\rangle_{P}|\phi^{+}\rangle_{F}|\phi^{-}\rangle_{S} &\rightarrow&  \frac{1}{2\sqrt 2}(|+-\rangle + |-+\rangle)(|EE\rangle + |II\rangle)(|rl\rangle + |lr\rangle)_{AB},   \nonumber \\
|\psi^{+}\rangle_{P}|\phi^{+}\rangle_{F}|\phi^{-}\rangle_{S} &\rightarrow&  \frac{1}{2\sqrt 2}(|++\rangle - |--\rangle)(|EI\rangle + |IE\rangle)(|rl\rangle + |lr\rangle)_{AB},   \nonumber \\
|\psi^{-}\rangle_{P}|\phi^{+}\rangle_{F}|\phi^{-}\rangle_{S} &\rightarrow&  \frac{1}{2\sqrt 2}(|+-\rangle - |-+\rangle)(|EI\rangle + |IE\rangle)(|rl\rangle + |lr\rangle)_{AB}.    \nonumber \\
\end{eqnarray}
The above eight hyperentangled Bell states in three DOFs can be identified by using the detection results of single photon detectors, with which the initial 64 hyperentangled states can be classified into eight groups, as shown in Table 2. From Table 1 and Table 2, we can find that the complete analysis of hyperentangled Bell state in three DOFs has already been accomplished.

\begin{table}
\centering\caption{The initial 64 hyperentangled Bell states in three DOFs can be classified into eight groups with the detection results of single photon detectors.}
\begin{tabular}{cc ccccccccccc}
\hline
Group &  Initial states \\
\hline
$1$ & $|\phi^{+}\rangle_{P}|\phi^{+}\rangle_{F}|\phi^{+}\rangle_{S},|\phi^{+}\rangle_{P}|\psi^{+}\rangle_{F}|\phi^{+}\rangle_{S},|\psi^{+}\rangle_{P}|\phi^{+}\rangle_{F}|\phi^{+}\rangle_{S},|\psi^{+}\rangle_{P}|\psi^{+}\rangle_{F}|\phi^{+}\rangle_{S},$ \\ &
       $|\phi^{+}\rangle_{P}|\phi^{+}\rangle_{F}|\psi^{+}\rangle_{S},|\phi^{+}\rangle_{P}|\psi^{+}\rangle_{F}|\psi^{+}\rangle_{S},|\psi^{+}\rangle_{P}|\phi^{+}\rangle_{F}|\psi^{+}\rangle_{S},|\psi^{+}\rangle_{P}|\psi^{+}\rangle_{F}|\psi^{+}\rangle_{S}.$ \\

$2$ & $|\phi^{+}\rangle_{P}|\phi^{-}\rangle_{F}|\phi^{+}\rangle_{S},|\phi^{+}\rangle_{P}|\psi^{-}\rangle_{F}|\phi^{+}\rangle_{S},|\psi^{+}\rangle_{P}|\phi^{-}\rangle_{F}|\phi^{+}\rangle_{S},|\psi^{+}\rangle_{P}|\psi^{-}\rangle_{F}|\phi^{+}\rangle_{S},$ \\ &
       $|\phi^{+}\rangle_{P}|\phi^{-}\rangle_{F}|\psi^{+}\rangle_{S},|\phi^{+}\rangle_{P}|\psi^{-}\rangle_{F}|\psi^{+}\rangle_{S},|\psi^{+}\rangle_{P}|\phi^{-}\rangle_{F}|\psi^{+}\rangle_{S},|\psi^{+}\rangle_{P}|\psi^{-}\rangle_{F}|\psi^{+}\rangle_{S}.$ \\

$3$ & $|\phi^{-}\rangle_{P}|\phi^{+}\rangle_{F}|\phi^{+}\rangle_{S},|\phi^{-}\rangle_{P}|\psi^{+}\rangle_{F}|\phi^{+}\rangle_{S},|\psi^{-}\rangle_{P}|\phi^{+}\rangle_{F}|\phi^{+}\rangle_{S},|\psi^{-}\rangle_{P}|\psi^{+}\rangle_{F}|\phi^{+}\rangle_{S},$ \\ &
       $|\phi^{-}\rangle_{P}|\phi^{+}\rangle_{F}|\psi^{+}\rangle_{S},|\phi^{-}\rangle_{P}|\psi^{+}\rangle_{F}|\psi^{+}\rangle_{S},|\psi^{-}\rangle_{P}|\phi^{+}\rangle_{F}|\psi^{+}\rangle_{S},|\psi^{-}\rangle_{P}|\psi^{+}\rangle_{F}|\psi^{+}\rangle_{S}.$ \\

$4$ & $|\phi^{-}\rangle_{P}|\phi^{-}\rangle_{F}|\phi^{+}\rangle_{S},|\phi^{-}\rangle_{P}|\psi^{-}\rangle_{F}|\phi^{+}\rangle_{S},|\psi^{-}\rangle_{P}|\phi^{-}\rangle_{F}|\phi^{+}\rangle_{S},|\psi^{-}\rangle_{P}|\psi^{-}\rangle_{F}|\phi^{+}\rangle_{S},$ \\ &
       $|\phi^{-}\rangle_{P}|\phi^{-}\rangle_{F}|\psi^{+}\rangle_{S},|\phi^{-}\rangle_{P}|\psi^{-}\rangle_{F}|\psi^{+}\rangle_{S},|\psi^{-}\rangle_{P}|\phi^{-}\rangle_{F}|\psi^{+}\rangle_{S},|\psi^{-}\rangle_{P}|\psi^{-}\rangle_{F}|\psi^{+}\rangle_{S}.$ \\

$5$ & $|\phi^{+}\rangle_{P}|\phi^{+}\rangle_{F}|\phi^{-}\rangle_{S},|\phi^{+}\rangle_{P}|\psi^{+}\rangle_{F}|\phi^{-}\rangle_{S},|\psi^{+}\rangle_{P}|\phi^{+}\rangle_{F}|\phi^{-}\rangle_{S},|\psi^{+}\rangle_{P}|\psi^{+}\rangle_{F}|\phi^{-}\rangle_{S},$ \\ &
       $|\phi^{+}\rangle_{P}|\phi^{+}\rangle_{F}|\psi^{-}\rangle_{S},|\phi^{+}\rangle_{P}|\psi^{+}\rangle_{F}|\psi^{-}\rangle_{S},|\psi^{+}\rangle_{P}|\phi^{+}\rangle_{F}|\psi^{-}\rangle_{S},|\psi^{+}\rangle_{P}|\psi^{+}\rangle_{F}|\psi^{-}\rangle_{S}.$ \\

$6$ & $|\phi^{+}\rangle_{P}|\phi^{-}\rangle_{F}|\phi^{-}\rangle_{S},|\phi^{+}\rangle_{P}|\psi^{-}\rangle_{F}|\phi^{-}\rangle_{S},|\psi^{+}\rangle_{P}|\phi^{-}\rangle_{F}|\phi^{-}\rangle_{S},|\psi^{+}\rangle_{P}|\psi^{-}\rangle_{F}|\phi^{-}\rangle_{S},$ \\ &
       $|\phi^{+}\rangle_{P}|\phi^{-}\rangle_{F}|\psi^{-}\rangle_{S},|\phi^{+}\rangle_{P}|\psi^{-}\rangle_{F}|\psi^{-}\rangle_{S},|\psi^{+}\rangle_{P}|\phi^{-}\rangle_{F}|\psi^{-}\rangle_{S},|\psi^{+}\rangle_{P}|\psi^{-}\rangle_{F}|\psi^{-}\rangle_{S}.$ \\

$7$ & $|\phi^{-}\rangle_{P}|\phi^{+}\rangle_{F}|\phi^{-}\rangle_{S},|\phi^{-}\rangle_{P}|\psi^{+}\rangle_{F}|\phi^{-}\rangle_{S},|\psi^{-}\rangle_{P}|\phi^{+}\rangle_{F}|\phi^{-}\rangle_{S},|\psi^{-}\rangle_{P}|\psi^{+}\rangle_{F}|\phi^{-}\rangle_{S},$ \\ &
       $|\phi^{-}\rangle_{P}|\phi^{+}\rangle_{F}|\psi^{-}\rangle_{S},|\phi^{-}\rangle_{P}|\psi^{+}\rangle_{F}|\psi^{-}\rangle_{S},|\psi^{-}\rangle_{P}|\phi^{+}\rangle_{F}|\psi^{-}\rangle_{S},|\psi^{-}\rangle_{P}|\psi^{+}\rangle_{F}|\psi^{-}\rangle_{S}.$ \\

$8$ & $|\phi^{-}\rangle_{P}|\phi^{-}\rangle_{F}|\phi^{-}\rangle_{S},|\phi^{-}\rangle_{P}|\psi^{-}\rangle_{F}|\phi^{-}\rangle_{S},|\psi^{-}\rangle_{P}|\phi^{-}\rangle_{F}|\phi^{-}\rangle_{S},|\psi^{-}\rangle_{P}|\psi^{-}\rangle_{F}|\phi^{-}\rangle_{S},$ \\ &
       $|\phi^{-}\rangle_{P}|\phi^{-}\rangle_{F}|\psi^{-}\rangle_{S},|\phi^{-}\rangle_{P}|\psi^{-}\rangle_{F}|\psi^{-}\rangle_{S},|\psi^{-}\rangle_{P}|\phi^{-}\rangle_{F}|\psi^{-}\rangle_{S},|\psi^{-}\rangle_{P}|\psi^{-}\rangle_{F}|\psi^{-}\rangle_{S}.$ \\
\hline
\end{tabular}
\end{table}

Owing to the Kerr effect and self-assisted mechanism, Our complete HBSA scheme is simple and efficient. Specifically speaking, the four Bell states in the first longitudinal momentum DOF and the parity information of entanglement in the second longitudinal momentum DOF are determined by using the weak cross-Kerr nonlinearity. Then, the polarization Bell states are distinguished through the preserved entanglement in the first longitudinal momentum DOF, and the phase information of Bell states in the second longitudinal momentum DOF are identified by using linear optical element and single photon detector.

\section{Discussion and summary}
Hyperentangled state analysis has important application in many high-capacity quantum communication protocols, such as hyperentanglement-based quantum teleportation, dense coding, entanglement swapping, quantum repeater and so on. Here, we show the application of our complete HBSA scheme in a quantum teleportation protocol that based on the hyperentangled Bell state in polarization and two longitudinal momentum DOFs. Suppose the sender Alice wants to send the receiver Bob an unknown quantum state of photon $X$, which is a single-photon state in three DOFs,
\begin{eqnarray}
|\phi\rangle_{X} = (\alpha_P|H\rangle + \beta_P|V\rangle) \otimes (\alpha_F|E\rangle + \beta_F|I\rangle) \otimes (\alpha_S|r\rangle + \beta_S|l\rangle)_X.
\end{eqnarray}
The two communication parties have shared a hyperentangled Bell state in three DOFs as the quantum channel in advance
\begin{eqnarray}
|\Phi\rangle_{AB} = \frac{1}{2\sqrt 2}(|HH\rangle + |VV\rangle)\otimes(|EE\rangle + |II\rangle)\otimes(|rr\rangle+|ll\rangle)_{AB}.
\end{eqnarray}
After Alice performs HBSA on the two photons $X$ and $A$, the quantum state of the three-photon system can be rewritten as the following form,
\begin{eqnarray}
|\phi\rangle_{X}\otimes|\Phi\rangle_{AB} = &\frac{1}{8}[|\phi^{+}\rangle_{P}(\alpha_P|H\rangle + \beta_P|V\rangle)_X +|\phi^{-}\rangle_{P}(\alpha_P|H\rangle - \beta_P|V\rangle)_X \nonumber \\ & + |\psi^{+}\rangle_{P}(\alpha_P|V\rangle + \beta_P|H\rangle)_X +|\psi^{-}\rangle_{P}(\alpha_P|V\rangle - \beta_P|H\rangle)_X] \nonumber \\ & \otimes [|\phi^{+}\rangle_{F}(\alpha_F|E\rangle + \beta_F|I\rangle)_X +|\phi^{-}\rangle_{F}(\alpha_F|E\rangle - \beta_F|I\rangle)_X \nonumber \\ & + |\psi^{+}\rangle_{F}(\alpha_F|I\rangle + \beta_F|E\rangle)_X +|\psi^{-}\rangle_{F}(\alpha_F|I\rangle - \beta_F|E\rangle)_X] \nonumber \\ & \otimes [|\phi^{+}\rangle_{S}(\alpha_S|r\rangle + \beta_S|l\rangle)_X +|\phi^{-}\rangle_{S}(\alpha_S|r\rangle - \beta_S|l\rangle)_X \nonumber \\ & + |\psi^{+}\rangle_{S}(\alpha_S|l\rangle + \beta_S|r\rangle)_X +|\psi^{-}\rangle_{S}(\alpha_S|l\rangle - \beta_S|r\rangle)_X].
\end{eqnarray}

It is easy to find that the sender Alice has 64 possible measurement results, corresponding to which there are 64 potential single-photon states in three DOFs for the photon of receiver Bob. With the help of our complete HBSA scheme, the 64 hyperentangled Bell states in three DOFs can be unambiguously discriminated, corresponding to which Bob can get the information of the quantum state of his own photon. Based on the measurement result of Alice, Bob can map the original state of photon $X$ to his photon with proper single-photon unitary operations on the three different DOFs.

The weak cross-Kerr nonlinearity plays an important role in our complete HBSA scheme, and it will influence the practical efficiency of the whole scheme. Although many works have been reported on the cross-Kerr effect, we should acknowledge that a clean cross-Kerr nonlinearity is still challenging in the single-photon regime with the current technology \cite{Kerr2,Kerr3,Kerr4}. However, recent studies have shown that it is promising for us to utilize the cross-Kerr nonlinearity in the near future \cite{Kerr5,Kerr6,Kerr7,Kerr8,Kerr9,Kerr10}. For example, in 2009, Matsuda \emph{et al.} presented the first experimental demonstration of single-photon-level nonlinear phase shift in an optical fibre \cite{Kerr5}. In 2016, Beck \emph{et al.} measured a conditional cross-phase shift of $\pi/6$ between the retrieved signal and control photons \cite{Kerr8}. In the same year, Tiarks \emph{et al.} experimentally demonstrate the generation of the $\pi$ phase shift with a single-photon pulse \cite{Kerr9}. In 2019, Sinclair \emph{et al.} reported the experimental observation of a cross-Kerr nonlinearity in a free-space medium, which is used to implement cross-phase modulation between two optical pulses \cite{Kerr10}. Actually, we just need the small phase shift that can be distinguished from the zero phase shift, which will make our complete HBSA scheme more practical and realizable.

In summary, we have presented an efficient scheme for the complete analysis of photonic hyperentangled Bell state in three different DOFs, including the polarization DOF and two longitudinal momentum DOFs. The distinguishing process is accomplished with the help of weak cross-Kerr nonlinearity and self-assisted mechanism, which can make our scheme simple and realizable. We also show the application of our complete HBSA scheme in the quantum teleportation of a single-photon state in three DOFs, and we believe this scheme will be useful for the future high-capacity quantum communication.

\section*{References}

\begin{thebibliography}{99}

\bibitem{1} F. G. Deng, B. C. Ren, and X. H. Li, Quantum hyperentanglement and its applications in quantum information processing, Sci. Bull. 62, 46 (2017).
\bibitem{2} T. C. Wei, J. T. Barreiro, and P. G. Kwiat, Hyperentangled Bell-state analysis, Phys. Rev. A 75, 060305(R) (2007).
\bibitem{3} X. H. Li and S. Ghose, Hyperentangled Bell-state analysis and hyperdense coding assisted by auxiliary entanglement, Phys. Rev. A 96, 020303(R) (2017).
\bibitem{4} N. Pisenti, C. P. E. Gaebler, and T. W. Lynn, Distinguishability of hyperentangled Bell states by linear evolution and local projective measurement, Phys. Rev. A 84, 022340 (2011).
\bibitem{5} Y. B. Sheng, F. G. Deng, and G. L. Long, Complete hyperentangled-Bell-state analysis for quantum communication, Phys. Rev. A 82, 032318 (2010).
\bibitem{6} B. C. Ren, H. R. Wei, M. Hua, T. Li, and F. G. Deng, Complete hyperentangled-Bell-state analysis for photon systems assisted by quantum-dot spins in optical microcavities, Opt. Express 20, 24664 (2012).
\bibitem{7} T. J. Wang, Y. Lu, and G. L. Long, Generation and complete analysis of the hyperentangled Bell state for photons assisted by quantum-dot spins in optical microcavities, Phys. Rev. A 86, 042337 (2012).
\bibitem{8} Q. Liu and M. Zhang, Generation and complete nondestructive analysis of hyperentanglement assisted by nitrogen-vacancy centers in resonators, Phys. Rev. A 91, 062321 (2015).
\bibitem{9} X. H. Li and S. Ghose, Self-assisted complete maximally hyperentangled state analysis via the cross-Kerr nonlinearity, Phys. Rev. A 93, 022302 (2016).
\bibitem{10} X. H. Li and S. Ghose, Complete hyperentangled Bell state analysis for polarization and time-bin hyperentanglement, Opt. Express 24, 18388 (2016).
\bibitem{11} C. Cao, L. Zhang, Y. H. Han, P. P. Yin, L. Fan, Y. W. Duan, and R. Zhang, Complete and faithful hyperentangled-Bell-state analysis of photon systems using a failure-heralded and fidelity-robust quantum gate, Opt. Express 28, 2857 (2020).
\bibitem{12} Q. Liu, G. Y. Wang, Q. Ai, M. Zhang, and F. G. Deng, Complete nondestructive analysis of two-photon six-qubit hyperentangled Bell states assisted by cross-Kerr nonlinearity, Sci. Rep. 6, 22016 (2016).
\bibitem{13} M. Y. Wang, F. L. Yan, and T. Gao, Deterministic state analysis for polarization-spatial-time-bin hyperentanglement with nonlinear optics, Laser Phys. Lett. 15, 125206 (2018).
\bibitem{14} H. R. Zhang, P. Wang, C. Q. Yu, and B. C. Ren, Deterministic nondestructive state analysis for polarization-spatial-time-bin hyperentanglement with cross-Kerr nonlinearity, Chin. Phys. B 30, 030304 (2021).
\bibitem{15} X. J. Zhou, W. Q. Liu, Y. B. Zheng, H. R. Wei, and F. F. Du, Complete Hyperentangled Bell States Analysis For Polarization-Spatial-Time-Bin Degrees of Freedom with Unity Fidelity, Ann. Phys. (Berlin) 534, 2100509 (2022).
\bibitem{16} J. T. Barreiro, N. K. Langford, N. A. Peters, and P. G. Kwiat, Generation of Hyperentangled Photon Pairs, Phys. Rev. Lett. 95, 260501 (2005).
\bibitem{17} G. Vallone, R. Ceccarelli, F. De Martini, and P. Mataloni, Hyperentanglement of two photons in three degrees of freedom, Phys. Rev. A 79, 030301(R) (2009).

\bibitem{Kerr1} K. Nemoto and W. J. Munro, Nearly Deterministic Linear Optical Controlled-NOT Gate, Phys. Rev. Lett. 93, 250502 (2004).
\bibitem{Kerr2} W. J. Munro, K. Nemoto and T. P. Spiller, Weak nonlinearities: a new route to optical quantum computation, New J. Phys. 7, 137 (2005).
\bibitem{Kerr3} J. H. Shapiro, Single-photon Kerr nonlinearities do not help quantum computation, Phys. Rev. A 73, 062305 (2006).
\bibitem{Kerr4} J. H. Shapiro and M. Razavi, Continuous-time cross-phase modulation and quantum computation, New J. Phys. 9, 16 (2007).
\bibitem{Kerr5} N. Matsuda, R. Shimizu, Y. Mitsumori, H. Kosaka, and K. Edamatsu, Observation of optical-fibre Kerr nonlinearity at the single-photon level, Nat. Photon. 3, 95 (2009).
\bibitem{Kerr6} I. C. Hoi, A. F. Kockum, T. Palomaki, T.M. Stace, B. Fan, and L. Tornberg, Giant cross-Kerr effect for propagating microwaves induced by an artificial atom, Phys. Rev. Lett. 111, 053601 (2013).
\bibitem{Kerr7} A. Feizpour, M. Hallaji, G. Dmochowski, and A. M. Steinberg, Observation of the nonlinear phase shift due to single postselected photons, Nat. Phys. 11, 905 (2015).
\bibitem{Kerr8} K. M. Beck, M. Hosseini, Y. H. Duan, and V. Vuletic, Large conditional single-photon cross-phase modulation. PNAS 113, 9740 (2016).
\bibitem{Kerr9} D. Tiarks, S. Schmidt, G. Rempe, and S. Durr, Optical $\pi$ phase shift created with a single-photon pulse. Sci. Adv. 2, e1600036 (2016).
\bibitem{Kerr10} J. Sinclair, D. Angulo, N. Lupu-Gladstein, K. Bonsma-Fisher, and A. M. Steinberg, Observation of a large, resonant, cross-Kerr nonlinearity in a cold Rydberg gas. Phys. Rev. Res. 1, 033193 (2019).

\end{thebibliography}

\end{document}

