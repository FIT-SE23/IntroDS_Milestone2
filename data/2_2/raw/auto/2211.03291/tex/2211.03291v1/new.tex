\documentclass[11pt]{article}
\usepackage{amsthm, amsmath, amssymb, amsfonts, url, booktabs, tikz, setspace, fancyhdr, enumerate}
\usepackage[margin = 1in]{geometry}

\usepackage[czech,english]{babel} % This will make Stovicek look correct
%\usepackage{lmodern}

\usepackage[hidelinks]{hyperref}
\usepackage{tikz}
\usepackage{comment}
\usepackage{xcolor}
\usepackage{cite}
%\usepackage{todonotes}


%\usepackage[color]{showkeys}
%\renewcommand\showkeyslabelformat[1]{\fbox{\normalfont\footnotesize\ttfamily#1}}
% showkeys font
%\definecolor{refkey}{gray}{.75}
%\definecolor{labelkey}{gray}{.2}

\usepackage[textsize=scriptsize,colorinlistoftodos]{todonotes}


% ------   Theorem Styles -------
\newtheorem{theorem}{Theorem}[section]
\newtheorem{proposition}[theorem]{Proposition}
\newtheorem{lemma}[theorem]{Lemma}
\newtheorem{corollary}[theorem]{Corollary}
\newtheorem{conjecture}[theorem]{Conjecture}
\newtheorem{claim}[theorem]{Claim}
\newtheorem{question}[theorem]{Question}
%\newtheorem*{thm}{Theorem}


\theoremstyle{definition}
\newtheorem{definition}[theorem]{Definition}
\newtheorem{example}[theorem]{Example}
\newtheorem{setup}[theorem]{Setup}

\theoremstyle{remark}
\newtheorem*{remark}{Remark}
\newtheorem*{notation}{Notation}


% ----  Text Styles ----
\newcommand{\op}{\operatorname}
\newcommand{\edit}[1]{{\color{red} #1 }}

% -----  Named Operators  ------
\DeclareMathOperator{\DISC}{DISC}
\DeclareMathOperator{\PAIR}{PAIR}
%\DeclareMathOperator{\ex}{ex}
\newcommand{\spa}{\mathrm{sp}}


% ----- Delimiters ----
\newcommand{\abs}[1]{\left\lvert#1\right\rvert}
\newcommand{\norm}[1]{\left\lVert#1\right\rVert}
\newcommand{\ang}[1]{\left\langle #1 \right\rangle}
\newcommand{\floor}[1]{\left\lfloor #1 \right\rfloor}
\newcommand{\ceil}[1]{\left\lceil #1 \right\rceil}
\newcommand{\paren}[1]{\left( #1 \right)}
\newcommand{\sqb}[1]{\left[ #1 \right]}
\newcommand{\set}[1]{\left\{ #1 \right\}}
\newcommand{\setcond}[2]{\left\{ #1 \;\middle\vert\; #2 \right\}}
\newcommand{\cond}[2]{\left( #1 \;\middle\vert\; #2 \right)}
\newcommand{\sqcond}[2]{\left[ #1 \;\middle\vert\; #2 \right]}

% ------ Operator symbols ------

\newcommand{\x}{\times}
\newcommand{\approxmod}{\mathop{\approx}}

\newcommand{\bG}{\hat{G}}
\newcommand{\tG}{\tilde{G}}
\newcommand{\G}{\mathcal{G}}
\newcommand{\Gh}{\hat{\mathcal{G}}}

\newcommand{\Hom}{\mathrm{Hom}}

%\newcommand{\E}{\mathbb{E}}
\DeclareMathOperator*{\E}{\mathbb E}
\newcommand{\PP}{\mathbb{P}}
\newcommand{\RR}{\mathbb{R}}
\newcommand{\bm}{\mathbf{m}}
\newcommand{\bx}{\mathbf{x}}
\newcommand{\by}{\mathbf{y}}
\newcommand{\bz}{\mathbf{z}}
%\newcommand{\ext}{\mathbf{ex}}
\newcommand{\HH}{\mathcal{H}}
\newcommand{\cP}{\mathcal{P}}

\newcommand{\A}{\mathcal{A}}
\newcommand{\B}{\mathcal{B}}
\newcommand{\C}{\mathcal{C}}
\newcommand{\D}{\mathcal{D}}
\newcommand{\e}{\epsilon}
\newcommand{\ev}{\mathcal{E}}
\newcommand{\F}{\mathcal{F}}
\newcommand{\Oh}{\mathcal{O}}
\renewcommand{\S}{\mathcal{S}}
\newcommand{\J}{\mathcal{J}}
\newcommand{\K}{\mathcal{K}}
\newcommand{\TT}{\mathcal{T}}
\newcommand{\U}{\mathcal{U}}
\newcommand{\X}{\mathbf{X}}
\newcommand{\Y}{\mathbf{Y}}

\newcommand{\ex}{\mathrm{ex}}

\title{Rainbow cycles in properly edge-colored graphs}
\author{Jaehoon Kim\thanks{Department of Mathematical Sciences, KAIST, South Korea. Email: \texttt{jaehoon.kim}@\texttt{kaist.ac.kr}. Supported by the POSCO Science Fellowship of POSCO TJ Park Foundation}
\and Joonkyung Lee\thanks{Department of Mathematics, 
		Hanyang University, 
		222 Wangsimni-ro, Seongdong-gu, 
		Seoul 04763, South Korea. 
		Email: \texttt{joonkyunglee}@\texttt{hanyang.ac.kr}.}
\and Hong Liu
	\thanks{Extremal Combinatorics and Probability Group (ECOPRO), Institute for Basic Science (IBS), Daejeon, South Korea. Email: {\tt hongliu@ibs.re.kr}. Supported by IBS-R029-C4.}	
\and Tuan Tran\thanks{School of Mathematical Sciences, University of Science and Technology of China, China. E-mail: {\tt trantuan@ustc.edu.cn}. Supported by the Outstanding Young Talents Program (Overseas) of the National Natural Science Foundation of China.}
 }
\date{}



\begin{document}

\maketitle

\begin{abstract}
We prove that every properly edge-colored $n$-vertex graph with average degree at least $100(\log n)^2$ contains a rainbow cycle, improving upon %$O((\log n)^{2+o(1)})$ bound 
$(\log n)^{2+o(1)}$ bound due to Tomon. %for the rainbow Tur\'an number of a cycle. 
%been an open question by Keevash, Mubayi, Sudakov, and Verstra\"ete from 2007.
We also prove that every properly colored $n$-vertex graph with at least $10^5 k^2 n^{1+1/k}$ edges contains a rainbow $2k$-cycle, which improves the previous bound $2^{ck^2}n^{1+1/k}$ obtained by Janzer.

Our method using homomorphism inequalities and a lopsided regularization lemma also provides a simple way to prove the Erd\H{o}s--Simonovits supersaturation theorem for even cycles, which may be of independent interest.
\end{abstract}

\section{Introduction}
%For a family $\H$ of graphs, the \emph{extremal number} $\ex(n,\H)$ is the maximum number of edges in $\H$-free $n$-vertex graphs.
%A central topic in extremal graph theory is to understand the asymptotic behavior of extremal numbers and its variants.
In an edge-colored graph, a copy of a graph $H$ is called \emph{rainbow} if every edge in the copy receives a unique color.
If we forbid rainbow copies of some graphs in properly edge-colored graphs~$G$, what is the maximum number of edges in $G$?
This extremal question was first investigated by
Keevash, Mubayi, Sudakov, and Verstra\"ete~\cite{KMSV07},
where they defined the \emph{rainbow Tur\'an number} $\ex^*(n,\HH)$ for a family of graphs $\HH$.
Formally, $\ex^*(n,\HH)$ denotes the maximum number of edges in a properly edge-colored $n$-vertex graph with no rainbow copies of any $H\in \HH$.
If $\HH$ consists of a single graph $H$, then we simply write $\ex^*(n,H)$ instead of $\ex^*(n,\{H\})$.

To quote~\cite{KMSV07}, there are two questions that are the most important among the several ones raised therein. The first one is to determine $\ex^*(n,\C)$, where $\C$ is the class of all cycles. It is shown that $\ex^*(n,\C)=\Omega(n\log n)$ in~\cite{KMSV07} and Das, Lee, and Sudakov~\cite{DLS13} obtained an upper bound $O(ne^{(\log n)^{\frac{1}{2}+o(1)}})$.
There have been some recent improvements upon the upper bound~\cite{J20,T22}  %\cite{J20,T22,W22} 
and the current best one is $O(n(\log n)^{2+o(1)})$ appeared in \cite{T22}. %both~\cite{T22} and~\cite{W22}.
We improve this bound to $O(n\log^2 n)$.
\begin{theorem}\label{thm:cycle}
A properly edge-colored $n$-vertex graph $G$ 
with %average degree at least $100\log^2 n$ always contains a rainbow cycle of length at most $100\log n$.
at least $32n\log^2 (5n)$ edges always contains a rainbow cycle.
\end{theorem}

Theorem~\ref{thm:cycle} has the following corollary on cycles in linear 3-uniform hypergraphs.
\begin{corollary}\label{cor:3-graph}
If an $n$-vertex linear $3$-uniform hypergraph $H$ has at least $288 n \log^2(5n)$ edges, it contains a loose cycle.
\end{corollary}

The second question in~\cite{KMSV07} concerns with $\ex^*(n,C_{2k})$, where $C_{2k}$ is the even cycle of length $2k$. In~\cite{KMSV07}, a general lower bound $\ex^*(n,C_{2k})=\Omega(n^{1+1/k})$ is obtained, whereas the matching upper bounds were only verified for $k=2,3$. This upper bound was subsequently improved by Das, Lee, and Sudakov~\cite{DLS13} to $O(n^{1+(1+o_k(1))\log k/k})$ and by Janzer~\cite{J20} to $O(n^{1+1/k})$.
While Janzer's bound matches the lower bound given in~\cite{KMSV07}, the implicit constant is exponential in $k$. We improve it to a polynomial one as follows.
\begin{theorem}\label{thm:even}
%Any 
A properly edge-colored $n$-vertex graph $G$ with at least $10^5 k^3 n^{1+1/k}$ edges always contains a rainbow $2k$-cycle.
\end{theorem}
%Although we bring the constant factor to the polynomial range, it is still larger than the current best bound $80\sqrt{k}\log k n^{1+1/k}$ for the extremal number of hexagons. 
Whereas the %two recent improvements in~\cite{T22,W22} took different approaches 
recent improvement in~\cite{T22} took a different approach to that of Janzer~\cite{J20}, our proof is in spirit closer to that of~\cite{J20} using homomorphism counts and improves it in two ways.
First, we use more efficient lopsided regularization lemma than the Jiang--Seiver lemma~\cite{JS12} used by Janzer.
Second, the main proof after regularization is conceptually simpler and more intuitive in the sense that it only relies on repeated applications of the Cauchy--Schwarz inequality.%\todo{is it? Janzer updated it once, check.}

We stress that the repeated Cauchy--Schwarz method may be of independent interest.
There are various problems in extremal combinatorics, from the classical Mantel's theorem and the K\H{o}vari--S\'os--Tur\'an theorem to the recent developments on Sidorenko's conjecture, where the Cauchy--Schwarz inequality and its variants have been extremely useful;
however, the convexity inequalities have seen less success in determining extremal numbers of bipartite graphs other than complete bipartite graphs. 

%Perhaps the first example one may be curious is the Bondy--Simonovits theorem, stating that $\ex(n,C_{2k})=O(n^{1+1/k})$.
To the best of our knowledge, it has been unknown whether even the Bondy--Simonovits theorem \cite{BS74}, a weaker statement than Theorem~\ref{thm:even}, has a proof that only uses Cauchy--Schwarz inequality.
Our proof method answers this natural question by giving a simple proof of the Bondy--Simonovits theorem and moreover, obtains the Erd\H{o}s--Simonovits supersaturation \cite{ESi84} as follows.
%We will discuss more about the method in the concluding remarks.
\begin{corollary}\label{cor:supsat}
  Any $n$-vertex graph $G$ with average degree $d\ge 2\cdot 10^5 k^3 n^{1/k}$ contains at least $\frac{1}{2}(2^{12}k)^{-k}d^{2k}$ copies of $2k$-cycle.
\end{corollary}



\section{The key homomorphism inequality}
%All graphs are simple in this paper.
In what follows, a \emph{coloring} always means an edge coloring and likewise, a \emph{colored graph} is an edge-colored graph.
Let $\Hom(H,G)$ be the set of homomorphisms from $H$ to $G$. In particular, we fully label vertices of $H$.
%\begin{definition}

Each homomorphism in $\Hom(C_{2k},G)$ can be seen as 
a closed walk $v_0v_1\dots v_{2k}$ of length~$2k$ (or closed $2k$-walk for short) in $G$ with $v_0=v_{2k}$. 
%We identify each closed walk of length $k$ as a vector $(v_1,v_2,\dots,v_{2k})$, where all $v_iv_{i+1}$, $1\leq i\leq 2k-1$, and $v_{2k}v_{1}$ are edges of $G$. 
Let $\phi$ be a proper coloring of a graph~$G$.
A closed $2k$-walk is \emph{degenerate} if, for distinct $i$ and $j$ with $\{i,j\}\neq \{0,2k\}$,  $v_i=v_j$ or $\phi(v_iv_{i+1}) = \phi(v_jv_{j+1})$, where the index addition is taken modulo $2k$.
That is, the walk revisits a vertex in the middle or repeats a color.
The former case is said to be \emph{vertex-degenerate at $(i,j)$} and the latter is called \emph{color-degenerate at $(i,j)$}.
If a $2k$-walk is vertex-degenerate (resp. color-degenerate) at $(i,j)$, then it is of \emph{type} $|i-j|-1$ (resp.~$|i-j|$). 
%\end{definition}

One degenerate walk may have multiple types, so the types of degeneracy are not disjoint in general.
As $\phi$ is a proper coloring, 
a closed walk is vertex-degenerate of type $1$ if and only if it is color-degenerate of type $1$. 
Note that a non-degenerate $2k$-walk is a rainbow cycle of length $2k$.
Let $\D(C_{2k},G)$ be the set of all degenerate closed walks of length $2k$ and let $S_k$ be the star with $k$ leaves. 
If $G$ has no rainbow $2k$-cycle, then $\Hom(C_{2k},G)= \D(C_{2k},G)$. 
Our strategy is to bound $|\D(C_{2k},G)|$ from above to obtain an upper bound for $|\Hom(C_{2k},G)|$.
 
\begin{lemma}\label{lem: degenerate}
Suppose that $G$ has no rainbow $2k$-cycle. Then
$$|\Hom(C_{2k},G)| \leq (2k)^{2k} |\Hom(S_k,G)|.$$
\end{lemma}
\begin{proof}
Let $\U_s$ and $\F_s$ be the set of closed $2k$-walks that are vertex-degenerate at $(0,s+1)$ and color-degenerate at $(0,s)$, respectively, i.e., they consist of those walks of type $s$. Note that $\U_1=\F_1$.
%Denote by $v_i$ the image of the vertex $i$ in the $2k$-cycle. In particular, every closed $2k$-walk satisfies $v_0=v_{2k}$ and the ones in $\U_s$ satisfy the extra condition $v_0=v_{s+1}$.
%By symmetry, $|U_{s}|=|U_{2k-2-s}|$ for $1\leq s\leq k-1$  and $|F_s|= |F_{2k-s}|$ for $1\leq s\leq k$.
\begin{claim}\label{cl: 1}
For $1\leq s\leq 2k-3$ and $1\leq t \leq 2k-1$,
\begin{align*}
    |\U_s|^2\leq |\U_1| \cdot |\U_{2s-1}| \enspace \text{and} \enspace |\F_t|^2 \leq |\F_1|\cdot |\F_{2t-1}|.
\end{align*}
In particular, $|\U_s|\leq|\U_1|$ and $|\F_t|\leq |\F_1|$.
\end{claim}
\begin{proof}[Proof of the claim]
For $s=1$, the inequality becomes a trivial equality and, for $1\leq s\leq k-1$, $|\U_{s}|=|\U_{2k-2-s}|$ by symmetry.
We may thus assume $1<s\leq k-1$.
%For each $(v_1,v_2,\dots,v_{2k}) \in U_s$, let $i$ be the minimum index such that $v_i=v_{i+s+1}$ for some $i$.
We analyze $\U_s$ by counting the closed walks therein in the following way:
\begin{itemize}
%\item choose vertices $x,y$ and an edge $xz$,
    \item fix vertices $x=v_0=v_{s+1}$, $y=v_{s+k}$, and $z=v_s$ with $xz\in E(G)$;
    \item choose a walk $v_{s+1}v_{s+2}\dots v_{s+k}$ of length $k-1$ from $x$ to $y$;
    %\item choose an edge $xz=v_{s+1}v_{s}$, 
    \item choose a walk $v_{s+k}v_{s+k+1}\dots v_{2k}$ of length $k-s$ from $y$ to $x$; and
    \item choose a walk $v_0v_1\dots v_{s}$ of length $s$ from $x$ to $z$.
\end{itemize}
Let $w_{\ell}(u,v)$ be the number of walks of length $\ell$ from $u$ to $v$
%and let $\hat{w}_{k,s}(z,x,y)$ be the number of walks of length $k$ from $z$ to $y$ where $(s+1)$-th vertex is $x$. 
and let $g(u,v)$ be the edge indicator function of $G$, i.e., $g(u,v)=1$ if $uv\in E(G)$ and $0$ otherwise.
Then 
\begin{align*}
    |\U_s| = \sum_{x,y,z} w_{k-1}(x,y) w_{k-s}(y,x)w_{s}(x,z)g(x,z).
\end{align*}
The Cauchy--Schwarz inequality now gives the following bound:
%, where $\sum_{x,y,xz}$ means the sum over all choices of vertices $x,y$ and an edge $xz\in E(G)$ and $\sum_{x,y,z}$ is taken over all vertices $x,y$, and $z$.
\begin{align*}
    |\U_s|^2
    &\leq \left(\sum_{x,y,z} w_{k-1}(x,y)^2g(x,z)\right) \left(\sum_{x,y,z} w_{k-s}(y,x)^2w_{s}(x,z)^2 g(x,z)\right) \\
    &\leq \left(\sum_{x,y,z} w_{k-1}(x,y)^2g(x,z)\right) \left(\sum_{x,y,z} w_{k-s}(y,x)^2w_{s}(x,z)^2\right),
\end{align*}
where the second inequality follows from $g(x,y)\leq 1$. 
%The last sum $\sum_{x,y,z}$ is over all vertices $x,y,z$, hence the final inequality is true. 
The sum $\sum_{x,y,z} w_{k-1}(x,y)^2g(x,z)$ counts the number of closed $(2k-2)$-walks plus a pendant edge that corresponds to~$xz$, which are exactly those $2k$-walks in $\U_1$. Thus, $\sum_{x,y,z} w_{k-1}(x,y)^2g(x,z)=|\U_1|$. 
Analogously, the term $w_{k-s}(y,x)^2$ and $w_{s}(x,z)^2$ count 
the number of closed walks of length $2(k-s)$ and $2s$ when summed over the choices of $y$ and $z$, respectively, both starting at $x$.
Thus, summing  $w_{k-s}(y,x)^2w_{s}(x,z)^2$ over $x,y,z\in V(G)$ counts the number of closed $2k$-walks that are vertex-degenerate at $(0,2s)$, which is exactly $|\U_{2s-1}|$.
Therefore, $|\U_s|^2\leq |\U_1|\cdot|\U_{2s-1}|$. 
Let $\U_j$ be the largest set among $\U_1,\cdots,\U_{k-1}$. 
Then $|\U_j|\leq |\U_1|^{1/2}|\U_{2j-1}|^{1/2}\leq |\U_1|^{1/2}|\U_j|^{1/2}$,
so $|\U_j|=|\U_1|$. This proves $|\U_s|\leq|\U_1|$ for all $s=1,2,\cdots,2k-3$.
%Repeatedly applying this inequality $\ell$ times then yields
%\begin{align*}
%    |\U_s|\leq |\U_1|^{1/2}|\U_{s_1}|^{1/2} \leq |\U_1|^{3/4}|\U_{s_2}|^{1/4}\leq
%    ...\leq |\U_1|^{1-2^{-\ell}}|\U_{s_\ell}|^{2^{-\ell}},
%\end{align*}
%where $s_i$ is the smallest index between $2s_{i-1}-1$ and $2k-2s_{i-1}-3$.
%This converges to $|\U_s|\leq|\U_1|$ as $\ell$ tends to infinity.

\medskip

The second inequality for $\F_t$ can also be obtained by using essentially the same technique. 
Again by symmetry, $|\F_t|= |\F_{2k-t}|$ for $1\leq t\leq k$, so we may assume $k\leq t< 2k-1$.
As each closed $2k$-walk $v_0\dots v_{2k}$ in $\F_t$ satisfies $\phi(v_0v_{1}) = \phi(v_{t}v_{t+1})$, we count $|\F_t|$ as follows: 
\begin{itemize}
    \item fix a color $c$ that repeats at $v_0v_1$ and $v_{t}v_{t+1}$;
    \item fix vertices $x=v_0=v_{2k}$ and $y=v_{k}$;
    \item choose a walk $v_{0}v_1\dots v_{k}$ of length $k$ from $x$ to $y$ where $\phi(v_0v_{1})=c$; and
    \item choose a walk $v_{k}v_{k+1}\dots v_{2k}$ of length $k$ from $y$ to $x$ where $\phi(v_{t}v_{t+1})=c$. 
\end{itemize}
Let $\tilde{w}_{k,\ell}(u,v,c)$ be the number of $k$-walks from $u$ to $v$ such that 
the $\ell$-th edge has color $c$.
%the first edge has color $c$, and let $W_{k,s-1}(y,z,c)$ be the number of $k$-walks from $y$ to $z$ where the $(s-1)$-th edge has color $c$. 
The Cauchy--Schwarz inequality then gives
\begin{align*}
    |\F_t|^2 &= \left(\sum_{x,y,c} \tilde{w}_{k,1}(x,y,c) \tilde{w}_{k,t-k+1}(y,x,c)\right)^2 \\
    &\leq\left(\sum_{x,y,c} \tilde{w}_{k,1}(x,y,c)^2\right)\left(\sum_{x,y,c} \tilde{w}_{k,t-k+1}(y,x,c)^2\right).
\end{align*}
The sum $\sum_{x,y,c} \tilde{w}_{k,1}(x,y,c)^2$ counts the number of closed $2k$-walks from $x$ that repeat $c$ at the first and the last edges, which is exactly $|\F_1|$ by rotational symmetry.
Similarly, the second sum corresponds to $|\F_{2t-2k+1}|=|\F_{2t-1}|$.
Therefore, we obtain the inequality  $|\F_t|^2 \leq |\F_1|\cdot |\F_{2t-1}|$.
Finally, $|\F_t|\leq |\F_1|$ follows from the same argument used for showing $|\U_s|\leq |\U_1|$.
\end{proof}

\begin{claim}\label{cl: 2}
 $|\Hom(C_{2k},G)| \leq 4k^2 |\U_1|$.
\end{claim}
\begin{proof}[Proof of the claim]
Recall that both $\U_s$ and $\F_s$ specify the labels of the vertices where degeneracy occurs. 
By rotational symmetry, the number of closed $2k$-walks that are vertex-degenerate at $(i,i+s+1)$ is exactly $|\U_s|$ for each $i=0,1,\dots, 2k-1$. Likewise, the number of closed $2k$-walks that are color-degenerate at $(i,i+s)$ is $|\F_s|$ for each $i=1,2,\dots,2k$. Thus, the number of degenerate $C_{2k}$-homomorphisms of type $s$ is at most $2k(|\U_s|+|\F_s|)$.
Taking the union bound over all possible types, we get
\begin{align}%\label{eq:D_upper}
    |\D(C_{2k},G)| \leq 2k\left(\sum_{s=1}^{k-1}|\U_s| + \sum_{s=1}^k |\F_s|\right).
\end{align}
Together with Claim~\ref{cl: 1} this yields $|\D(C_{2k},G)| \leq 4k^2|\U_1|$ as desired.
%Let $A:=\sum_{s=1}^{k-1}|\U_s| + \sum_{s=1}^k |\F_s|$ for brevity. 
%Recall that $|\F_1|=|\U_1|$, thus by Claim~\ref{cl: 1} and Cauchy--Schwarz, we have
%\begin{align*}
%    A&\leq \sqrt{|\U_1|}\cdot \left(\sum_{s=1}^{k-1} \sqrt{|\U_{2s-1}|} + \sum_{s=1}^k \sqrt{|\F_{2s-1}|}\right)\\
%    &\leq \sqrt{|\U_1|}\cdot \sqrt{2k-1} \sqrt{\sum_{s=1}^{k-1} |\U_{2s-1}| + \sum_{s=1}^k |\F_{2s-1}|} \leq  \sqrt{|\U_1|}\cdot\sqrt{2kA}.
%\end{align*}
%Hence, $A \leq 2k |\U_1|$. Now, as $G$ contains no rainbow cycle, using~\eqref{eq:D_upper}, we get $|\Hom(C_{2k},G)|=|\D(C_{2k},G)|\le 2kA\le 4k^2|\U_1|$ as desired.
\end{proof}


Let $\Oh_s$ be the set of all closed $2k$-walks $v_0\dots v_{2k}$ with $v_0= v_{2k}$
such that $v_0 = v_2 = \dots = v_{2s}$.
%where there exists $j$ such that $v_j=v_{j+2}=\dots = v_{j+2i}$ where the index is up to modulo $2k$.
In particular, $\Oh_1 = \U_1 = \F_1$. We also use $\Oh_0 = \Hom(C_{2k},G)$ for consistency.
\begin{claim}\label{cl: 3}
The sequence $|\Oh_s|$, $0\leq s\leq k$, is log-convex, i.e.,  
for each $s=1,\dots, k-1$,
\begin{align*}
     |\Oh_{s}|^2\leq |\Oh_{s-1}|\cdot|\Oh_{s+1}|.
\end{align*}
\end{claim}
\begin{proof}[Proof of the claim]
A \emph{star-walk} of length $\ell$ is a walk $u_0u_1\dots u_{\ell}$ of length $\ell$ such that every even-indexed vertex is the same one, i.e., $u_0=u_2=\dots=u_{2t}$ where $t=\lfloor\tfrac{\ell}{2}\rfloor$.
%Note that the walks counted by $S(i+1)$ is $v_1\dots v_{2k}$ with $v_1=v_{2k}$ and with some $j$ such that $v_j= v_{j+2}=\dots = v_{j+2i}=x$ and $v_{j+i+1}=y$ 
%(note that $y$ might coincide with $x$ depending on the parity of $i$.) 
For $1\leq s\leq k-1$, the walks in $\Oh_s$ can be counted as follows:
\begin{itemize}
    \item fix vertices $x=v_0=v_2=\dots=v_{2s}$, $y=v_{s+1}$, $z=v_{k+s+1}$, where $y$ is either $x$ or a neighbor of $x$ depending on the parity of $s$;
    \item \label{it:walk1} choose a walk $v_{k+s+1}v_{k+s+2}\dots v_{2k}$ of length $k-s-1$ from $z$ to $x$;
    \item \label{it:star1} choose a star-walk $v_0v_1\dots v_{s+1}$ of length $s+1$ from $x$ to $y$;
    \item \label{it:walk2} choose a walk $v_{k+s+1}v_{k+s}\dots v_{2s}$ of length $k-s+1$ from $z$ to $x$; and
    \item \label{it:star2} choose a star-walk $v_{2s}v_{2s-1}\dots v_{s+1}$ of length $s-1$ from $x$ to $y$;
    %\item choose a walk $v_{i-k}\dots v_{j+i+1}$ of length $k$ from $z$ to $y$ where $v_j=v_{j+2}=\dots = v_{j+ 2\lfloor\frac{i+1}{2}\rfloor}=x$,
   % \item choose a walk $v_{j+i-1-k} v_{j+i-2-k}\dots v_{j+i+1}$ of length $k$ from $z$ to $y$ where $v_{j+2i}=v_{j+2i-2}=\dots = v_{j+ 2 \lfloor \frac{i+1}{2}\rfloor } = x$.
\end{itemize}
As in the proof of Claim~\ref{cl: 1}, $w_{\ell}(u,v)$ denotes the number of $\ell$-walks from $u$ to $v$.
Let $\sigma_{\ell}(u,v)$ be the number of star-walks of length $\ell$ from $u$ to $v$. Note that, unlike $w_\ell(\cdot,\cdot)$, $\sigma_\ell(\cdot,\cdot)$ may not be a symmetric function in general.
Now we can compute $|\Oh_s|$ as
\begin{align*}
    |\Oh_s| = \sum_{x,y,z} w_{k-s-1}(z,x)\sigma_{s+1}(x,y)w_{k-s+1}(z,x)\sigma_{s-1}(x,y).
\end{align*}
%Let $W_{i+2}(z,x,y)$ be the number of walks $v_{j+i-1-k}\dots v_{j+i+1}$ of length $k$ from $z$ to $y$ where $v_j=v_{j+2}=\dots = v_{j+ 2\lfloor\frac{i+1}{2}\rfloor}=x$. 
%Let $W_{i}(z,x,y)$ be the number of walks $v_{j+i-1-k} v_{j+i-2-k}\dots v_{j+i+1}$ from $y$ to $z$ where $v_{j+2i}=v_{j+2i-2}=\dots = v_{j+ 2 \lfloor \frac{i+1}{2}\rfloor } = x$.
The Cauchy--Schwarz inequality then yields 
\begin{align*}
    |\Oh_s|^2 \leq\left(\sum_{x,y,z} w_{k-s-1}(z,x)^2\sigma_{s+1}(x,y)^2\right)\left(\sum_{x,y,z} w_{k-s+1}(z,x)^2\sigma_{s-1}(x,y)^2\right).
\end{align*}
In the first sum, $\sigma_{s+1}(x,y)^2$ counts either a closed star-walk of length $2(s+1)$ (if $s$ is odd and $x=y$) from $x$ or a star-walk of length $2s$ together with an edge $xy$ (if $s$ is even and $xy\in E(G)$). By considering $xy$ and $yx$ as the $(s+1)$-th and $(s+2)$-th edge of the star-walk, summing the latter case over all choices of $y$ counts the closed star-walks of length $2(s+1)$.
%In the first sum, $\sigma_{s+1}(x,y)^2$ means either a closed star-walk of length $2(s+1)$ (if $x=y$) from $x$ or a star-walk of length $2s+1$ (if $xy\in E(G)$).
%As the latter case uniquely extends to a closed star-walk of length $2(s+1)$ by the edge $xy$, 
Hence the first sum counts the number of walks in $\Oh_{s+1}$ by summing the number of ways to augment each closed walk of length $2(k-s-1)$ and to a closed star-walk of length $2(s+1)$ at $x$.
%corresponds to the number of ways to augment a closed walk of length $2(k-s-1)$ to a closed star-walk of length $2(s+1)$ at $x$, that is, the sum counts the number of walks in $\Oh_{s+1}$.
By the same reason with replacing $s+1$ by $s-1$, the second sum counts the number of walks in $\Oh_{s-1}$, which proves the claim.
%As these sums are over all $y,x,z$, each term is same with $S(i+2)^{1/2}$ and $S(i)^{1/2}$. Hence we have $ S(i+1)^{2}\leq S(i+2) S(i).$
\end{proof}
We are now ready to finish the proof of the lemma.
By Claims~\ref{cl: 2} and~\ref{cl: 3}, 
\begin{align*}
    4k^2\geq \frac{|\Hom(C_{2k},G)|}{|\U_1|} = \frac{|\Oh_0|}{|\Oh_1|} \geq \frac{|\Oh_1|}{|\Oh_2|} \geq \dots \geq \frac{|\Oh_{k-1}|}{|\Oh_k|}.
\end{align*}
Therefore, 
\begin{align*}
   \frac{|\Oh_0|}{|\Oh_k|}  = \frac{|\Oh_0|}{|\Oh_1|} \cdot  \frac{|\Oh_1|}{|\Oh_2|}  \dots \frac{|\Oh_{k-1}|}{|\Oh_{k}|}  \leq (4k^2)^k.
\end{align*}
Each closed $2k$-walk in $\Oh_k$ is of the form $vu_1vu_2\dots vu_kv$, which naturally corresponds to a homomorphism from $S_k$ to $G$ that maps the central vertex to $v$ and the $i$-th leaf to $u_i$.
Therefore, $|\Hom(C_{2k},G)|=|\Oh_0|\leq 4^k k^{2k}|\Oh_k|=|\Hom(S_{k},G)|$, which concludes the proof.
%As $S(k)= |\Hom(S_k,G)|$, this yields that we have $$|\Hom(C_{2k},G)|\leq (2k)^k |\Hom(S_k,G)|.$$
\end{proof}


\section{Regularization}
Suppose that the $n$-vertex graph $G$ is ``close" to being regular, e.g., $|\Hom(S_k,G)|\leq 10n^{k+1}p^k$, where $p=2e(G)/n^2$ is the edge density of $G$.
Here the constant $10$ is arbitrarily chosen to illustrate.
If~$G$ contains no rainbow $2k$-cycles, then Lemma~\ref{lem: degenerate} gives
\begin{align*}
    n^{2k}p^{2k}\leq |\Hom(C_{2k},G)|\leq  (2k)^{2k}|\Hom(S_k,G)|\leq 10\cdot (2k)^{2k} n^{k+1}p^{k},
\end{align*}
where the first inequality follows from the fact that even cycles satisfy Sidorenko's conjecture. As a corollary, $e(G)\leq 2\cdot10^{1/k}k^2n^{1+1/k}$, which is stronger than Theorem~\ref{thm:even}.

%then we can easily derive the theorems from Lemma~\ref{lem: degenerate} by simply counting $|\Hom(S_k,G)|$ from above and get a lower bound from the fact that $C_{2k}$ satisfies Sidorenko's conjecture. 

However, this ideal assumption is not guaranteed in general. Instead, the following two lemmas will enable us to ``regularize" the graph $G$.


%%%%%%%%%%%%%%%%%%%
%%%%%%%%%%%%%%%%%%%%%
%%%%%%%%%%%%%%%%%%%%%%%%
\begin{comment}
\begin{lemma}\label{lem: regularization 2}
Let $G$ be a properly colored $n$-vertex graph with $\delta(G)\geq 2 d$. Then there exists a properly colored $n'$-vertex graph $G'$ on with $n'\leq n^2$ and a color-preserving homomorphism $h$ from $G'$ to $G$ where every vertex of $G'$ has degree between $d$ and $2d$.
\end{lemma}
\begin{proof}
For each vertex $v$ of $G$, we replace it with a number of vertices $v_1,\dots, v_s$ so that we can distribute the incident edges to $v_1,\dots, v_s$. In other words, $N(v_1),\dots, N(v_s)$ will be a partition of $N(v)$. 
As the degree of $v$ is at least $2d$, we can do this so that all new vertices has degree between $d$ and $2d$. We repeat this for all remaining vertices.
Let $\psi$ be a map from $G'$ to $G$ so that new vertices are mapped to the original vertices. Then it is easy to check that it is a color preserving homomorphism, and $G'$ is as desired.
\end{proof}
\end{comment}
%%%%%%%%%%%%%%%
%%%%%%%%%%%%%%%%%
%%%%%%%%%%%%%%%%%%%%

\begin{lemma}\label{lem: regularization 2}
Let $G$ be a properly edge-colored graph with minimum degree $\delta(G)\geq 1$. Then there exists a properly edge-colored graph $G'$ with the following properties:
\begin{itemize}
    \item[\rm (1)] $G'$ satisfies $|V(G)|\leq |V(G')|\leq 4e(G)/\delta(G)$ and every vertex of $G'$ has degree between $\delta(G)/2$ and $\delta(G)$;
    \item[\rm(2)] There is a color-preserving homomorphism $\psi$ from $G'$ to $G$. In particular, if $G'$ contains a rainbow cycle, then so does $G$.
\end{itemize}
\end{lemma}
\begin{proof}
Let $\delta:=\delta(G)$ for brevity. 
We construct $G'$ by iterating the following process. Fix an ordering the vertices of $G$. At each step, take a vertex $v\in V(G)$ according to the ordering and let $s:=\lceil d_G(v)/\delta\rceil$. 
We then split $v$ into new vertices $v_1,\dots, v_s$ so that the neighbor sets $N(v_1),\dots, N(v_s)$ form a partition of the neighbor set of $v$ in $G$ and $\delta/2 \leq |N(v_i)|\leq \delta$ for every $1\le i \leq s$. 
This is possible since $d_G(v)\ge \delta$ and we can make the sizes of $N(v_1),N(v_2),\cdots, N(v_s)$ as equal as possible. We color the edges in such a way that each edge $uv_i$ for $u\in N_G(v)$ inherits the same color as $uv$.

%Repeating this for all remaining vertices of $G$, we obtain a graph $G'$.
Let $\psi$ be a map from $G'$ to $G$ so that each vertex maps to the original vertex of $G$ before splitting. Then $\psi$ is a color-preserving homomorphism and every vertex of $G'$ has degree between $\delta/2$ and $\delta$. Since the number of edges is preserved throughout the whole process,
\[
|V(G')|\cdot \delta/2 \leq 2e(G')=2e(G),
\]
which implies $|V(G')|\leq 4e(G)/\delta$. Indeed, the color-preserving homomorphism $\psi$ maps a rainbow cycle in~$G'$ to a rainbow cycle in $G$. This proves the ``in particular" part.
\end{proof}


\begin{lemma}\label{lem: regularization 1}
Let $k\ge 2$ and let $G$ be an $n$-vertex bipartite graph with %$e(G)\geq C n^{1+1/k}$.
average degree %$d= d(G) \geq C n^{1/k}$
$d>0$.
%Let $d= d(G) \geq C n^{1/k}$ and 
Suppose that $G$ contains no proper subgraph with larger average degree.
Then $G$ contains a subgraph $G'$ with bipartition $(A,B)$ satisfying the following for some $i\in \mathbb{N}$:
\begin{itemize}
    \item[\rm(1)] $|A| \ge \frac{1}{k}2^{-\frac{ki}{k-1}}n$, $|B|\ge \frac{n}{64}$;
    \item[\rm(2)] $d_{G'}(a) \in [2^{i-6} d, 2^{i-5} d ]$ for all $a\in A$;
    \item[\rm(3)] $d_{G'}(b) \leq 4d$ for all $b\in B$.
\end{itemize}
\end{lemma}
\begin{proof}
Denote by $(X,Y)$ a bipartition of $G$. 
Let $X_0$ and $Y_0$ be the set of vertices in $X$ and $Y$, respectively, of degree at least $4d$ and let $X_1:=X\setminus X_0$ and $Y_1:=Y\setminus Y_0$.
Then $|X_0|, |Y_0|\leq e(G)/(4d)=n/8$.

%We also have $G[X_0,Y_0] \leq \frac{1}{5} e(G)$, as otherwise we have $d(G[X_0,Y_0]) > \frac{1}{5} \frac{e(G)}{|X_0|+|Y_0|} > d $, a contradiction that $G$ does not have a proper subgraph of larger average degree.

Since there is no proper subgraph of $G$ with average degree larger than $d$, 
\[
e(G[X_0,Y_0]) \le \frac12(|X_0|+|Y_0|)d \le \frac{nd}{8}=\frac{1}{4}e(G).
\]
Hence, one of $G[X_0,Y_1]$, $G[X_1, Y_0]$ and $G[X_1,Y_1]$ has at least $e(G)/4$ edges. By symmetry, we can assume that for some $X'\in \{X_0,X_1\}$, the graph $G[X',Y_1]$ has at least $e(G)/4$ edges. %Let $G_0:=G[X',Y_1]$ for simplicity. 

As %$G_0$ 
$G[X',Y_1]$ has at least $e(G)/4$ edges and average degree at least $d/4$, we can delete some vertices of %$G_0$ 
$G[X',Y_1]$
to obtain a graph $G_1=G[X^*,Y^*]$ with minimum degree at least $d/16$ and $e(G_1)\ge e(G)/8$. 
Since the vertices in $Y^*$ have degree at most $4d$,
\[
|Y^*|\ge \frac{e(G_1)}{4d} \ge \frac{e(G)}{32d}=\frac{n}{64}.
\]
%Again, this graph must have average degree less than $d$, so $|X_0|+|Y_1|\geq n/6$, hence we have $|Y_1| \geq n/6- n/20 > n/9$.


%Note that the vertices in $Y_1$ have degree at most $10d$.

We partition %$X_0$ 
$X^*$ into the following sets 
\[
Z_i = \left\{ v\in X^*\colon d_{G_1}(v) \in [2^{i-6}d, 2^{i-5}d) \right\}, \quad i\in \mathbb{N}.
\]
If there exists $i$ such that $|Z_i| \geq \frac{1}{k}2^{-\frac{ki}{k-1}}n$, then take $A$ to be %a subset of $Z_i$ with the desired size 
$Z_i$ and $B=Y^*$, and we are done.
If not, then we have
$$\frac{e(G)}{8} \leq e(G_1) \le \sum_{i=1}^{\infty} |Z_i|\cdot 2^{i-5}d  < \frac{dn}{32k}\cdot\sum_{i=1}^{\infty} 2^{-\frac{i}{k-1}}  < \frac{d n}{32 k} \cdot \frac{1}{1 - 2^{-\frac{1}{k-1}}} < \frac{dn }{16}.
$$
In the last inequality we used the facts that $2^{-x} \le 1-x/2$ for $0\le x \le 1$ and that $0<\frac{1}{k-1}\le 1$ for $k\ge 2$.
This is a contradiction as $e(G)= dn/2$. This proves the lemma.
\end{proof}

The Cauchy--Schwarz inequality together with %the Lemma~1 in `the size of bipartite graphs with a given girth' by Hoory 
a result by Hoory \cite[Lemma 1]{Hoory} yields the following lemma.

\begin{lemma}\label{lem: C2k}
Let $G$ be a bipartite graph with vertex partition $(A,B)$.
Suppose that the average of degrees of the vertices in $A$ is %at least 
$d_A$
and the average of degrees of the vertices in $B$ is %at least 
$d_B$. Then for every $k\in \mathbb{N}$ we have 
$$|\Hom(C_{2k},G)| \geq d_A^k \cdot d_B^k.$$
\end{lemma}

\section{Proofs of the main results}
Now we are ready to prove our %two theorems.
main results.

%\begin{theorem}
%For an $n$-vertex properly colored graph $G$, if $e(G) \ge 10^5 k^3 n^{1+ 1/k}$, then $G$ contains a rainbow $2k$-cycle.
%\end{theorem}
\begin{proof}[Proof of Theorem~\ref{thm:even}]
By taking a subgraph, assume that $G$ is a bipartite graph with at least $50000 k^3 n^{1+1/k}$ edges and let $d \ge 10^5 k^3 n^{1/k}$ be the average degree of $G$. We further assume that $G$ has no subgraph with larger average degree, as otherwise we can just take that subgraph to be our graph.
Also assume that $G$ has no rainbow $2k$-cycle.
We apply Lemma~\ref{lem: regularization 1} to obtain a graph $G'$ and some $i\in \mathbb{N}$ such that 
\begin{itemize}
    \item $|A| = \frac{1}{k}2^{-\frac{ki}{k-1}}n$;
    \item $d_{G'}(a) \in [2^{i-6} d, 2^{i-5} d ]$ for all $a\in A$;
    \item $d_{G'}(b) \leq 4 d$ for all $b\in B$.
\end{itemize}
Indeed, we can obtain the equality in the %second
first bullet point by deleting some vertices if necessary. 
Note that the first two conditions ensures that the average of degrees of vertices in $B$ is at least
$$\frac{2^{i-6}d|A|}{n} \geq \frac{d}{64k} 2^{-\frac{i}{k-1}}.$$
Apply Lemma~\ref{lem: C2k} to obtain that 
\begin{align}\label{eq: lower}
\Hom(C_{2k},G') \geq 2^{k(i-6)}d^k (\frac{d}{64k})^k 2^{-\frac{ki}{k-1}}.
\end{align}


As $d_{G'}(b)\leq 4d$ for all $b\in B$ and $\sum_{b\in B}d_{G'}(b)= e(G')$, the convexity of the function $f(x)=x^k$ yields that 
$$\sum_{b\in B} d_{G'}(b)^k \leq \frac{e(G')}{4d} \cdot (4d)^k \leq (4d)^{k-1}e(G').$$
Hence, Lemma~\ref{lem: degenerate} implies that 
\begin{align*}
    |\Hom(C_{2k},G')| &\leq (2k)^{2k} |\Hom(S_k,G')| \leq (2k)^{2k} \left(\sum_{a\in A} d_{G'}(a)^k+  \sum_{b\in B} d_{G'}(b)^k  \right) \\ &
    \leq (2k)^{2k}( |A| 2^{k(i-5)} d^k + (4d)^{k-1} e(G') ) \leq (2k)^{2k} (2^{k(i-5)} d^{k} + (4d)^{k-1}\cdot 2^{i-5}d)|A| \\& \leq (2k)^{2k} d^k (2^{k(i-5)}  + 4^{k-1}2^{i-5}) \cdot \frac{1}{k} 2^{-\frac{ki}{k-1}}n.
\end{align*}
Here, the penultimate inequality holds as %$e(G')\leq d|A|$
$e(G')\leq 2^{i-5}d|A|$.
Combining this with \eqref{eq: lower}, we obtain  
$$ d^k < (10^5 k^3)^k n.$$
However, we assume that $d\ge 10^5 k^3 n^{1/k}$, a contradiction. This proves the theorem.
\end{proof}


\begin{proof}[Proof of Corollary~\ref{cor:supsat}] 
The proof proceeds as in Theorem~\ref{thm:even}. If at least half of the $\Hom(C_{2k},G)$ is non-degenerate, then we can bound it from below using~\eqref{eq: lower}; otherwise we reach a similar contradiction as now $|\Hom(C_{2k},G)|\le 2|\D(C_{2k},G)|$ and hence 
$|\Hom(C_{2k},G)| \leq 2\cdot (2k)^{2k} |\Hom(S_k,G)|$ instead of Lemma~\ref{lem: degenerate}.
\end{proof}

\begin{proof}[Proof of Theorem~\ref{thm:cycle}]
Suppose that $G$ does not have a rainbow cycle. By iteratively deleting low degree vertices, we may assume that the minimum degree of $G$ is $\delta \ge d(G)/2 \ge 32\log^2(5n)$. Apply Lemma~\ref{lem: regularization 2} to obtain a graph $G'$ on $n'\leq 4n$ vertices such that $G'$ doesn't contain a rainbow cycle and every vertex of $G'$ has degree between $\delta/2$ and $\delta$.

Let $k= \log n'$. Because $G'$ does not contain any rainbow $2k$-cycle, Lemma~\ref{lem: degenerate} implies that
$$ |\Hom(C_{2k},G')|\leq (2k)^{2k} |\Hom(S_k,G')|\leq (2k)^{2k} \delta^k n'.$$
On the other hand, even cycle satisfies Sidorenko's conjecture, so we know that 
$$(2k)^{2k} \delta^k n' \geq  (\delta/2)^{2k}.$$
As $n'=2^k$, this yields that 
$$ (4k^2\cdot \delta \cdot 2)^k \geq (\delta^2/4)^{k},$$ 
which is a contradiction as $\delta \ge 32\log^2(5n)>32k^2$. Hence $G$ must contain a rainbow cycle.
\end{proof}


\begin{proof}[Proof of Corollary~\ref{cor:3-graph}]
Consider a partition of $V(H)$ into $V_1,V_2,V_3$ where the number of edges containing one vertex from each is at least $\frac{1}{9} e(H)$. Consider an auxiliary bipartite graph $G$ with the vertex partition $(V_1, V_2)$ where $v_1v_2$ is an edge in $G$ with color $v_3$ if $v_1v_2v_3\in E(H)$ with $v_i\in V_i$. As $G$ contains at least $32n\log^2(5n)$ edges,  Theorem~\ref{thm:cycle} implies a rainbow cycle in $G$. This yields a loose cycle in $H$.
\end{proof}

It could be interesting to extend the above result to $3$-uniform hypergraphs that are close to linear, e.g. those with maximum co-degree being a constant, or even $o(n)$. One can also consider not necessarily proper colorings in which every vertex has bounded number of edges with the same color, or at most $o(n)$ edges of the same color.

%Also of course, considering similar question for tight cycles, or more general cycles are interesting.

\vspace{5mm}

\noindent\textbf{Acknowledgements.} While writing this note, Janzer and Sudakov \cite{JS22} obtained asymptotically the same bound $O(n \log^2 n)$ as our Theorem~\ref{thm:cycle} independently. Instead of regularization, they did a weighted homomorphism count. Their homomorphism inequalities are analogous to ours, and they are able to apply it to a much larger class of reflextive graphs, including e.g. hypercubes.


\bibliographystyle{abbrv}
\bibliography{reference}
\end{document}