% \vspace{-18pt}
% \section{Conclusion} \label{sec:conclusion}
\section{Conclusion.} 

%Despite more than three decades of attempts, bridging the performance gap between sequencing machines and computational analysis is still challenging. 
Basecalling is the fundamental step in the Nanopore sequencing technology that relies on computationally intensive deep learning models. Although, state-of-the-art basecallers are able to achieve high accuracy, the computation overhead limits its application.
Therefore, efficient basecalling implementations optimized for inference in hardware have wide-ranging benefits, from lower inference latency to higher data throughput and reduced energy consumption. To this end, we introduce \mech, the first hardware-optimized basecaller that performs fast and accurate basecalling.  We develop \mech by using two machine learning techniques that are specifically designed for basecalling. First, we develop the first quantization-aware basecalling neural architecture search (\nas) framework to specialize the basecalling neural network architecture for a given hardware acceleration platform. Second, we develop the first technique, called \strim, to remove the skip connections present in modern basecallers to greatly reduce resource and storage requirements without any loss in basecalling accuracy. Our extensive evaluation on real genomic organisms shows that \mech not only provides the ability to basecall quickly but also accurately and efficiently enough to scale the analysis. We hope that our open-source implementations of \nas and \strim inspire future work and ideas in genomics and general omics research and development.