% \begin{methods}
% \clearpage
% \section{Material \& Methods} \label{sec:methods}

% \subsection{Overview}
% \mech is a hardware-optimized basecaller to perform fast and accurate basecalling. We develop \mech by using two machine learning techniques. First, we develop quantization-aware basecaller architecture search (\nas) while taking into account direct hardware metrics, such as latency. \nas-designed basecallers outperform manually-designed basecallers because \nas performs a joint search for optimal neural network architecture and  quantization for neural network weights and activations to find a basecaller architecture that provides the highest accuracy at low latency.  The output of \nas is a hardware-optimized basecaller that significantly outperforms manually designed basecallers.  Second, we develop \strim that removes skip connections present in modern basecallers to achieve substantial resource and storage reductions while incurring no loss in accuracy. %Unlike all the previous basecallers, we remove skip connections to reduce hardware storage and resource requirements.  %Third, we apply pruning to remove network connections that are considered unimportant to keep the network performance unchanged. We apply three different pruning techniques, including unstructured, structured, and group pruning. 
% \clearpage
% \subsection{\framework}
