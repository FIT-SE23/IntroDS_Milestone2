\documentclass{article}


\usepackage{PRIMEarxiv}

\usepackage[utf8]{inputenc} % allow utf-8 input
\usepackage[T1]{fontenc}    % use 8-bit T1 fonts
\usepackage{hyperref}       % hyperlinks
\usepackage{url}            % simple URL typesetting
\usepackage{booktabs}       % professional-quality tables
\usepackage{amsfonts}       % blackboard math symbols
\usepackage{nicefrac}       % compact symbols for 1/2, etc.
\usepackage{microtype}      % microtypography
\usepackage{lipsum}
\usepackage{fancyhdr}       % header
\usepackage{graphicx}   
\usepackage{amsmath}
% graphics
\graphicspath{{media/}}     % organize your images and other figures under media/ folder

%Header
\pagestyle{fancy}
\thispagestyle{empty}
\rhead{ \textit{ }} 

% Update your Headers here
\fancyhead[LO]{Running Title for Header}
% \fancyhead[RE]{Firstauthor and Secondauthor} % Firstauthor et al. if more than 2 - must use \documentclass[twoside]{article}



  
%% Title
\title{Modelling of a DC-DC Buck Converter Using Long-Short-Term-Memory (LSTM)}

\author{
  Muhy Eddin Za'ter \\
  Electrical Engineering Department \\
  University of Colorado Boulder \\
  Boulder, Colorado\\
  \texttt{muhy.zater@colorado.edu}}


\begin{document}
\maketitle


\begin{abstract}
Artificial neural networks make it possible to identify black-box models. Based on a recurrent nonlinear autoregressive exogenous neural network, this research provides a technique for simulating the static and dynamic behavior of a DC-DC power converter. This approach employs an algorithm for training a neural network using the inputs and outputs (currents and voltages) of a Buck converter.
The technique is validated using simulated data of a realistic Simulink-programmed nonsynchronous Buck converter model and experimental findings. The correctness of the technique is determined by comparing the predicted outputs of the neural network to the actual outputs of the system, thereby confirming the suggested strategy. Simulation findings demonstrate the practicability and precision of the proposed black-box method.
\end{abstract}


% keywords can be removed
\keywords{Neural Network \and Power Converter \and Modelling \and System Identification \and Black-box Model \and Long-Short-term-Memory }


\section{Introduction}
Currently, modeling and identifying power converters
is difficult because the majority of manufacturers do not give the
nor the specifications of the electrical components' topology. This
poses a problem for several industries needing to
forecast the performance and dynamics of power systems.
employing. The power converter's performance can be
approximated by determining the electrical system's parameters
components or by identifying a system or architecture that
its conduct \cite{valdivia2009simple}.

In most circumstances, input and output measurements are identical.
The sole sources of are the terminals of the power converter.
information. Consequently, when it comes to power converters,
input and output terminal currents and voltages are the
basis for the various identifying techniques. The strategies
pertaining to the determination of the converter's performance
based on the level of precision desired and the availability
pertaining to the information. Power converters may be represented as
White-box, grey-box, or black-box model based on varying circumstances.
facets \cite{balakrishnan2018dc}.

The white-box approach presupposes that the analytical and operational
The theoretical model and physical characteristics are known.
the energy converter It employs the data and the measured
Currents and voltages are measured to determine the electrical component values that best characterize the steady state.
circuit behavior that is dynamic \cite{riba2018parameter} The second design is
grey-box. It cannot be fully represented by physics equations
However, equations and parameters can be interpreted physically; this is a fact.
White-box and black-box model composition \cite{kicsiny2017grey}.

In this paper, the black-box model is analyzed.
There is no physical understanding of the system and inputs.
are connected to outputs using simply experimental data
Various techniques have been employed to evaluate the performance of
black-box models, the complexity of which depends on linearity
the issue itself Traditional techniques like ARMAX
(AutoRegressive Moving-Average with eXogenous) in conjunction with
NARMAX (Nonlinear AutoRegressive Moving Average with Exponential Smoothing)
eXogenous input) can accurately determine the efficiency of the
a model for a given dataset that employs an autoregressive moving average
The typical procedure that use the previous states of the variables to
future outputs can be generated based on support vector
machines (SVM) \cite{rahrooh2009identification, acuna2012comparing}. However, when they become accustomed to
Predict the output variable values depending on fresh input data.
values, accuracy tends to decline and mistake might occur.
substantial. Three distinct black-box modeling strategies are described in \cite{frances2017modeling}
Power converters are discussed. The initial comprises of
constructing a G-Parameters model based on the assumption that converters are
considered to be a two port network. Nonetheless, this strategy is flawed.
merely valid for linear models, which the buck is not,
This study analyzes the converter. The second strategy pertains to
The Wiener-Hammerstein model finds two transfer functions.
functions capable of replicating static nonlinearities and
The dynamics of the converter are linear. The final model is polytopic.
modeling strategy that finds fixed and transient variables
characteristics of the converter Nevertheless, its intricacy is
significant if a high degree of precision is desired.

Assuming the black box to be an artificial neural network,
nonlinear features are addressed by the identification method.
models, while permitting accurate prediction of model outputs.
\cite{lin2012identification}. The exogenous nonlinear autoregressive neural network
(NARX NN) is a potential answer to this problem. In \cite{boussaada2018nonlinear}, this is the case.
Based on a certain approach, the daily direct sun radiation is predicted.
on the weather conditions at a particular location. Additionally, \cite{cadenas2016wind}
Utilizes NARX neural networks to predict wind speed in a region.
specific region, using wind direction, pressure, and temperature as inputs.
climatic conditions and solar radiation This study proposes a method based on the use of an algorithm that identifies the
The emulated properties and structure of a NARX NN
characteristics of a DC-DC Buck converter. As soon as the neuronal weights
and network connections are acquired following training
stage, the power converter's reaction to any given combination of inputs.
can be estimated with great precision.

This paper's provided strategy is useful because
the robustness of the NN, which can accurately anticipate outcomes
the power converter's outputs for new operating locations
Their absence from the training set distinguishes them from
antecedent techniques that could only produce a response based on the data
for identifying purposes Additionally, the model is simple to
It can be reproduced in simulations due to its mathematical structure.
model that represents the neuronal weights This is helpful.
because it allows industrial users to replicate the behavior of complex systems
its converters regardless of whether the manufacturer provides them.
important data on the converters, which permits
Developing more precise predictive maintenance plans,
projections, etc.


\section{Long-Short-Term-Memory Nonlinear Autoregressive Exogenous
Neural Network Overview}

A black-box system is a structure with unknown topology and/or characteristics, whose input and output signals are often interpreted \cite{valdivia2009simple}. In general, the outputs are the consequence of an excitation or stimulation applied in the form of input values or vectors to the black box. Estimating the topology and parameters of a system when just its input and output values are known is not a simple issue.

Long-Short-Term-Memory (LSTM) functions as a model for the dynamics of a system. LSTM's are distinguished by the fact that their output variables are sent back to the system's input. A LSTM NARX NN (or NARX-LSTM) is a solution for identifying the performance of a black-box system. Based on the expansion of historical inputs and outputs, it can anticipate the outcome of a nonlinear system. The model is specified by either of the two following equations, which depend on the system's configuration:

\[
y(t) = f_N[y(t-1),..., y(t-n_a), u(t-1),...,u(t-n_k)]
\]

\[
\hat{y}(t) = f_N[\hat{y}(t-1),...,\hat{y}(t-n_a), u(t-1),...,u(t-n_k)]
\]


where $y(t)$ and $\hat{y}(t)$ are the real and simulated (autoregressive) black-box outputs, u(t) is the system input (exogenous variable), n is the number of output delays and n is the number of input delays \cite{boussaada2018nonlinear}.

The two designs of the NARX-LSTM system, given by equations (1) and (2), rely on whether or not the output information is sent back. The open loop (or series) architecture is commonly used for designing and training neural networks since it estimates the output of the system based on a priori (known) target information (predicted). The closed loop design, on the other hand, uses the actual estimated output as an input to the black box, therefore it is not necessary to obtain the desired data beforehand, which is advantageous for predictions \cite{boussaada2018nonlinear}. Figure 1 depicts two architectural designs.

\begin{figure}[h]
  \centering
  \includegraphics[width=4in, height=1in]{figure_1.png}
  \caption{NARX architecture in a) Open loop and b) Closed loop.}
  \label{fig:fig1}
\end{figure}

A neural network structure comprises of one input layer.
a layer of output and hidden layers, the number of which is impossible to determine
to determine, and it relies on the nature of the situation, therefore it may be.
varies based on the nature of the application and the input
data.

Fig. 2 illustrates an open loop NARX-LSTM structure.
architecture. The problem's complexity depends on the
number of inputs, outputs, delays, and neurons per layer.
Considering that these factors specify the total number of variables,
(relationships) of the optimization issue that must be resolved.

\begin{figure}[h]
  \centering
  \includegraphics[width=4in, height=2in]{figure_2.png}
  \caption{NARX Neural Network structure}
  \label{fig:fig1}
\end{figure}

To determine the neuron weights, an iterative procedure is employed, wherein the weight values are modified at each epoch based on the acquired root mean squared error. For this reason, the Levenberg-Marquardt algorithm is utilized. It calculates the Jacobian and the Hessian at each iteration, which is its primary shortcoming. Then, for huge data sets and networks, the matrices might be quite massive, necessitating substantial processing resources.



\subsection{Model Architecture}

The input and output variables of the black-box system were selected with the knowledge that the trained NARX-LSTM mimics the behavior of a DC-DC power converter. The only accessible data then, assuming the topology of the converter is unknown, are the current and voltage measurements at the input and output terminals of the circuit. However, the input and output variable values of the converter do not always correspond to those of the NARX neural network.

In light of the fact that the variables may be altered externally, the black-box signals for this particular situation are categorized as follows:

\begin{itemize}
    \item Inputs: input voltage and output current, which are externally controllable by the voltage source and load, respectively.

\item Outputs of the power converter are input current and output voltage, which are dependent on the input variables.
\end{itemize}

\section{Implementation of NARX-LSTM}


It is impossible to determine a priori the precise configuration of a NARX-LSTM that matches the black-box dynamic properties since it depends on the dataset's parameters. This section therefore provides a technique for obtaining a NARX-LSTM capable of predicting the output variable values depending on the model's input variables. Fig. 3 depicts the algorithm's flowchart.

\begin{figure}[h]
  \centering
  \includegraphics[width=2.5in, height=6.5in]{figure_3.png}
  \caption{Algorithm for NARX-LSTM training}
  \label{fig:fig1}
\end{figure}

It is essential to specify the characteristics of the measured data since they influence the precision and efficacy of the training algorithm. In order to account for all potential outputs, the data set must take into account various load conditions. In addition, the time step for each set of data must be same.
Interpolation is necessary when the time step is variable.
By adjusting the settings of several neural networks, the network with the lowest error rate was picked. Following is a description of the parameters of the neural network and their relative significance:

\begin{itemize}
    \item This option specifies the number of past time steps to be considered when calculating the current values of the output variables. Since it provides inputs to the black box, it has a direct effect on the needed training time.

 \item Number of neurons in the hidden layer: the precise number of neurons in the hidden layer is difficult to calculate, necessitating an iterative procedure. It is highly dependent on the network's design and parity \cite{hunter2012selection}. Moreover, neural networks with a limited number of neurons take less computing time to train the model, which is important for this job.

\item Length of the dataset: this is one of the most crucial criteria, as a too-small or too-large number of points may result in underfitting or overfitting, respectively. Each point indicates a time step of the signal, as stated. The dataset consists of n trials, each of which demonstrates a load change.

\item Training ratio: this parameter specifies the proportion of the dataset allotted to training, validation, and testing. If the training proportion is $x\%$, then the validation and test proportion is $(100 x)2\%$. This number often varies with the size of the dataset.

\end{itemize}

\section{Simulation Results}

The DC-DC Buck converter was selected to create a training dataset that reflects the model's static and dynamic behavior.

Figure 4 depicts the structure of the converter. Included are the output impedance of the voltage source ($L_{in}$ and $R_{in}$), the equivalent series resistance (ESR) of the capacitors ($RC_{in}$, $RC_1$, and $RC_2$), and the series resistance of the inductor ($R_L$).

\begin{figure}[h]
  \centering
  \includegraphics[width=4in, height=2in]{figure_4.png}
  \caption{Buck Converter}
  \label{fig:fig1}
\end{figure}

Except for the load, all parameters of the power converters are fixed. The values of each component of the circuit may be found in the datasheet for the TPS40200EVM-002 buck converter in Table I. (Texas Instruments).

\begin{table}[h]
\centering
\caption{Buck Converter Parameters}
\begin{tabular}{cc}
\hline
\textbf{Parameter} & \textbf{Value} \\ \hline
L                  & 33 $\mu $ H          \\
$C_1$                 & 20  $\mu$ F          \\
$C_2$                 & 440 $\mu$ F         \\
$R_L$                 & 60 m $\Omega$           \\
$R_{C1}$                & 64 m $\Omega$          \\
$R_{C2}$                & 300 m $\Omega$         \\
$R_s$                 & 40 m $\Omega$          \\
$C_{in}$                & 28.47 $\mu$ F       \\
$L_{in}$                & 0.3018 $\mu$ H      \\
$R_{cc} $               & 5273.7 $\Omega$        \\
$R_{Cin}$               & 100 m  $\Omega $       \\
$R_{in}$                & 434.9 m $\Omega $      \\
$V_{out}$               & 3.3 V          \\
frequency          & 197 kHz        \\ \hline
\end{tabular}
\end{table}

To evaluate the performance of the proposed approach, the power converter was modeled and simulated using the Simscape Electrical module in MATLAB/Simulink. The input voltage was set at 19 V, and the load was regulated externally by a variable resistor, allowing for a variety of load conditions to be generated. Providing unpredictability to the training data and preventing erroneous identification of the neural network parameters, the simulation lasted 0.2 seconds and had various modifications of the load at varying frequencies. The simulations were executed in continuous time, and the data was interpolated to create a dataset with the same fixed step for each experiment.

By introducing a load shift, as seen in Fig. 5, the system's response was produced, which comprises both steady-state and transient responses. This is only one of the 500 cases that were simulated to train the neural network. Importantly, the output load range was chosen so that the Buck converter would work in continuous conduction mode (CCM). To ensure that all simulations were conducted in the converter's CCM, a parametric sweep of the load value was done while monitoring the inductor current. The maximum value of the output resistor was then set to 15.8, which is the value just before the inductor current approaches zero.

\begin{figure}[h]
  \centering
  \includegraphics[width=4in, height=3in]{figure_5.png}
  \caption{ A) Input voltage, b) Output current, c) Output voltage and d) Input
current of the Buck converter. }
  \label{fig:fig1}
\end{figure}

After acquiring a large dataset including the input and output variables of the black-box model, the following step is to specify the fixed and variable parameters of the to-be-trained neural networks. In this instance, the delays are fixed, whereas the number of neurons, the length of the dataset, and the proportion given to training, validation, and testing are variable. Table II displays the parameters indicated before.

\begin{table}[h]
\caption{Neural Network Parameters}
\centering
\begin{tabular}{cc}
\hline
\textbf{Parameter}       & \textbf{Value}                \\ \hline
Epochs                   & 1000                          \\
Training time            & 3 hours                       \\
Minimum error            & 1e-7                          \\
Neurons in hidden layers & 5, 6, 7, 8, 9, 10, 14, 20, 30 \\ \hline
\end{tabular}
\end{table}

A total of 176 NARX-LSTM were trained, which required four days of computation on an Intel Xeon CPU E5-1650 v2 @3.5 GHz, resulting in an average simulation duration of 45 minutes per network. In open loop configuration, each network obtained an error value less than 0.0001, as assessed by root mean square error (RMSE). However, following the transition to closed loop and the new training phase, the error worsened, and only seven of the proposed neural networks (3.98\%) had an RMSE value below 0.001.

There is no obvious association between the training parameters of the neural networks and the lowest error rate. The only unambiguous conclusion that can be drawn from the training of the 176 networks is that the networks with more neurons (more than 10) failed to recreate the target values of the variables accurately, resulting in a rather high error rate. In addition, the error values of neural networks trained with more than 230 points and fewer than 140 points were greater due to underfitting and overfitting, respectively. On the other hand, there is no discernible trend regarding the training ratio, mostly because it is closely correlated with the size of the training dataset.

For the purpose of determining which neural network best approximates the performance of the Buck converter, the networks were subjected to a fresh set of load values. Simulated were ten new load profiles similar to the one depicted in Figure 5. In 10 situations, the outputs of neural networks and predicted outputs were compared to those produced from Simulink simulations. The network with the best fit to the ten situations consists of 10 neurons in the hidden layer, was trained with 200 tests and a training ratio of 50\%, achieving an RMSE of 2.15 10. The structure is illustrated in fig 6.

\begin{figure}[h]
  \centering
  \includegraphics[width=4in, height=3in]{figure_6.png}
  \caption{ Neural Network Architecture}
  \label{fig:fig1}
\end{figure}

Since each of the ten tests has different load conditions and 2 million points, it is required to compare the actual and estimated output values by focusing on the steady state and a load shift. Fig. 7 depicts the estimated and simulated outputs when the load is set to 1.204 (steady state), whereas Fig. 8 compares the estimated and simulated outputs when the load changes from 1.204 to 0.791. (transient state).

\begin{figure}[h]
  \centering
  \includegraphics[width=4in, height=2in]{figure_7.png}
  \caption{The actual and predicted reaction. a) Current input in steady state. b) Voltage output in steady state.}
  \label{fig:fig1}
\end{figure}

\begin{figure}[h]
  \centering
  \includegraphics[width=5in, height=3in]{figure_8.png}
  \caption{ The actual and predicted reaction. a) Current input during a load adjustment. b) The output voltage when the load varies.}
  \label{fig:fig1}
\end{figure}

The findings provided in Figures 7 and 8 indicate that the expected input current of the neural network and the one simulated in Simulink are almost identical. There is no discernible difference, and the system accurately forecasts the transient state even when the load changes. In contrast, the output voltage estimation is not as precise as the current estimation. In steady-state conditions, the only minor distinction between the mean value and the ripple is the waveform's shape. The neural network system forecasts the overshoot when a load shift occurs, although the reaction time is slightly quicker.

\section{Conclusion}
This article describes the identification of an LSTM.
A NARX neural network that emulates the behavior of an unknown topology DC-DC Buck power converter. This
Estimation was performed using a simulated data set.
varying load situations within a time range. Because there is
There is no perfect approach for determining the optimal arrangement.
neural network, a number of networks were evaluated, with
The lowest mistake rate was chosen. The outcome is the weights of
the linkages between the various system levels and a
A network that has been trained to anticipate the power output.
a voltage converter for a specified input voltage and load.

%Bibliography
\bibliographystyle{unsrt}  
\bibliography{references}  


\end{document}
