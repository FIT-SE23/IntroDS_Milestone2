\documentclass[runningheads]{comsis2}

%% Necessary definitions for the running heads
\def\journalissue{}
\def\paperidnum{}
\setcounter{page}{1}
\usepackage[square,numbers]{natbib} % has a nice set of citation styles and commands
    \bibliographystyle{abbrvnat}
    \renewcommand{\bibsection}{\subsection*{References}}
%% Use this to show line numbers (and remove only in the final camera-ready version)
\usepackage{geometry}
\geometry{
  a4paper,         % or letterpaper
  textwidth=17cm,  % llncs has 12.2cm
  textheight=24cm, % llncs has 19.3cm
  heightrounded,   % integer number of lines
  hratio=1:1,      % horizontally centered
  vratio=2:3,      % not vertically centered
  }
\usepackage[colorlinks=true,bookmarks=true,citecolor=blue,urlcolor=blue]{hyperref} %pdflatex
\usepackage{microtype}
\usepackage{graphicx}
\usepackage{subfig}
\usepackage{booktabs} % for professional tables
\usepackage{paralist}
\usepackage{soul}
\usepackage{algorithmic}
\usepackage[linesnumbered,lined,boxed,commentsnumbered,ruled,vlined]{algorithm2e}
\newcommand{\theHalgorithm}{\arabic{algorithm}}
\usepackage{float}
\usepackage[british]{babel}
% For theorems and such
\usepackage{amsmath}
\usepackage{amssymb}
\usepackage{mathtools}
\usepackage{stackrel}
\usepackage{apxproof}
\usepackage{url}            % simple URL typesetting
\usepackage{booktabs}       % professional-quality tables
\usepackage{amsfonts}       % blackboard math symbols
\usepackage{nicefrac}       % compact symbols for 1/2, etc.
\usepackage{xcolor}         % colors
\usepackage{mathptmx}
\usepackage{dsfont}
\usepackage[title]{appendix}
\usepackage{multirow}
\usepackage{wrapfig}
% Recover regular caligraphic letters
\DeclareMathAlphabet{\mathcal}{OMS}{cmsy}{m}{n}
%\usepackage{multicol}
\usepackage{tabularx}
\usepackage{bm}
\usepackage{tikz} 
\tikzset{
  agent/.style={draw, circle, minimum size=2mm,
  font=\footnotesize},
  x = 1.5cm, y=1.5cm,
  every loop/.style={},
  block/.style={draw, rectangle, minimum height=3em, minimum width=6em},
}
\usetikzlibrary{decorations.pathreplacing}
\usetikzlibrary{positioning}
\usetikzlibrary{shapes,arrows,matrix,scopes,shadows,chains,automata,positioning,fit, arrows.meta,calc}


% if you use cleveref..
\usepackage[capitalize,noabbrev]{cleveref}
\usepackage{libertine}
% \linenumbers
\usepackage{notation}
\crefname{subsection}{subsection}{subsections}
\crefname{lemma}{lemma}{lemma}
\crefname{property}{property}{property}
\crefname{table}{table}{table}
\crefname{assumption}{assumption}{assumption}
\crefname{table}{table}{tables}
\Crefname{table}{Table}{Tables}
\crefname{figure}{Fig.}{Fig.}
\Crefname{figure}{Fig.}{Fig.}
% \theoremstyle{plain}
% \newtheorem{property}{Property}[section]
\renewcommand\theproperty{(P\arabic{property})}
\newtheorem{assumption}[theorem]{Assumption}
\title{On learning history based policies for controlling Markov decision processes}
% \footnote{If this is an extended version of a conference paper, it should be clearly stated here.}
%% Use this if the title is too long for the running heads
\titlerunning{On learning history based policies for controlling Markov decision processes}

\author{Gandharv Patil\inst{1} \and Aditya Mahajan\inst{1} \and Doina Precup\inst{1}}

%% Use this the list of authors is too long for the running heads
%\authorrunning{First Author et al.}

\institute{McGill University,  Mila\\
  Montreal QC - Canada\\
  \email{gandharv.patil@mail.mcgill.ca,  aditya.mahajan@mcgill.ca,  dprecup@cs.mcgill.ca}}
%   \and
%   Faculty of the Second Author\\
%   Address\\
%   \email{author2@faculty.edu}}

\begin{document}

\maketitle

\begin{abstract}
Reinforcement learning (RL) folklore suggests that history-based function approximation methods, such as recurrent neural nets or history-based state abstraction, perform better than their memory-less counterparts, due to the fact that function approximation in Markov decision processes (MDP) can be viewed as inducing a Partially observable MDP. However, there has been little formal analysis of such history-based algorithms, as most existing frameworks focus exclusively on memory-less features. In this paper, we introduce a theoretical framework for studying the behaviour of RL algorithms that learn to control an MDP using history-based feature abstraction mappings. Furthermore, we use this framework to design a practical RL algorithm and we numerically evaluate its effectiveness on a set of continuous control tasks. 


\end{abstract}

\section{Introduction}

Generative modeling has been the dominant approach for large-scale pretraining and zero-shot generalization~\cite{gpt3-paper,artetxe2021efficient,rae2021scaling}. 
Combined with prompts~\cite{gpt3-paper}, most of the natural language processing (NLP) tasks can be formulated into the fill-in-the-blank format and perform generative language modeling.
Based on the unified generative formulation, pretrained models such as GPT-3~\cite{gpt3-paper}, BERT~\cite{devlin2018bert,PET-paper}, T5~\cite{T5-paper}, can perform zero-shot inference on new tasks. 


More recent work~\cite{T0-paper} proposed to further pretrain a generative T5~\cite{T5-paper} with multitask prompted datasets and has substantially enhanced the performance of zero-shot generalization. 
In contrast, methods based on discriminative modeling~\cite{devlin2018bert} have not been able to achieve state-of-the-art performance on zero-shot learning. The adoption of discriminative approaches for zero-shot learning has been limited in the literature.


% Although there are a few works using discriminative modeling to perform zero-shot or few-shot learning, such as CLS finetuning using BERT or prompting using ELECTRA
% For example, BERT was CLS finetuned to perform zero-shot/few-shot learning, however, the zero-shot/few-shot performance are lagged far behind.

% \zy{Add a note: although BERT can be CLS finetuned (which is discriminative), but it is not the SOTA approach for zero-shot and few-shot learning.}

\begin{figure}%[htbp]
     \centering
     \includegraphics[width=1.05\linewidth]{figure/final_sota.png}
     \vspace{-15pt}
     \caption{Average zero-shot performance over 11 zero-shot tasks for our Universal Discriminator and T0~\cite{T0-paper}. Our universal discriminator significantly outperforms T0 across three different scales.}
     \label{fig:sota}
     \vspace{-15pt}
 \end{figure} 


In this work, we challenge the convention of zero-shot learning and propose to study and improve discriminative approaches. This is motivated by the fact that many NLP tasks can be framed as selecting from a few options; e.g., telling whether sentence A entails sentence B, or predicting which answer is correct for a given question. We call these tasks \textit{discriminative tasks}. As we will discuss in later sections, a significant portion of NLP tasks is in fact discriminative tasks. We hypothesize that discriminative approaches perform better for discriminative tasks.
% Despite the recent progress, it remains unknown how discriminative approaches perform in zero-shot generalization. Motivated by the fact that discriminative modeling learns to distinguish among options and goes better with discriminative tasks (e.g., telling whether sentence A entails sentence B, or telling which option correctly answer the question), we hypothesize that discriminative modeling would be better at zero-shot generalization, especially on discriminative tasks.

To verify the hypothesis, we propose the \textbf{universal discriminator (UD)}, which substantially improves zero-shot generalization over the previous generative state-of-the-art (SOTA)~\cite{T0-paper}, as Figure~\ref{fig:sota} shows.
The main idea is to train a single discriminator to predict whether a text sample comes from the true data distribution of natural language, similar to GANs \cite{goodfellow2014generative}. Given a set of training tasks with labeled data, we construct a dataset with positive and negative examples, where positive ones are in-distribution natural language samples and negative ones are out-of-distribution. There are two major types of discriminative tasks. The first type is tasks with multiple options, such as multi-choice question answering and news classification. We fill the options into the sentences and the ones with correct options are considered positive samples. The second type is tasks with yes/no options, which can be formulated as a binary discrimination problem itself. For example, natural language inference aims to predict whether a premise entails a hypothesis. In this case, we use a prompt to concatenate the premise $A$ and the hypothesis $B$ into a sentence ``Premise: $A$. Hypothesis: $B$.'' If entailment holds, this sample is treated as positive in-distribution samples and otherwise negative out-of-distribution ones.



% We define the true data distribution using multiple training tasks with labeled data. Specifically, since discriminative tasks can be formulated as selecting from a few options, samples with correct options form an empirical data distribution, while samples with incorrect options are considered out of distribution. In other words, our discriminator is trained to predict ``true'' for samples with correct options and ``false'' for incorrect ones. We use simple concatenation to minimize prompting efforts. For example, given an example (premise, hypothesis), a natural language inference task predicts whether the premise entails the hypothesis. We concatenate the premise and hypothesis, and assign the label ``true'' for entailment and ``false'' for non-entailment.


% First off, since many of the NLP tasks can be formulated as selecting from several options, we first reformulate the task data into natural text samples by concatenating different fields \zy{what are fields? undefined here. try using another word.}.
% For example, given an example of \zy{the} natural language inference task (\textit{Premise}, \textit{Hypothesis}, \textit{Label}), the natural text is reformulated as ``\textit{\{Premise\} || \{Hypothesis\}}'' labeled with \textit{\{Label\}}. \footnote{Here we use ``||'' to represents direct concatenation.} 
% Another example of topic classification task (\textit{Text}, \textit{Label}) where the \textit{Label} indicates the first option of \{Sports, Fashion, Politics\}, the corresponding natural texts are formulated as ``\textit{Text} || Sports'' labeled with 1, ``\textit{Text} || Fashion'' and ``\textit{Text} || Politics'' both labeled with 0.
% Secondly, we pretrain a pretrained model with reformulated multitask datasets to distinguish whether the text sample comes from the true data distribution. ~\footnote{An assumption is that negative-labeled text samples are artificially constructed thus do not come from the true data distribution, and vice versa.}

For the performance of zero-shot generalization, our approach achieves new state-of-the-art on the T0 benchmark, outperforming T0 by 16.0\%, 7.8\%, and 11.5\% respectively on different scales. 
UD also achieves state-of-the-art performance on a wide range of supervised NLP tasks, using only 1/4 parameters of previous methods.
Compared with the previous generative prompt-based methods, our universal discriminator requires minimal prompting, which is simple, robust, and applicable in real-world scenarios.

% By further scaling the number of tasks, our approach also sets the new state-of-the-art on \textbf{\color{red}[xxx]} tasks with less than 10\% of model parameters \zy{need to give a range} under the setting of standard finetuning.
% In the setting of finetuning, our approach also outperforms the generative baselines consistently across a wide range of tasks.


In addition, we also generalize UD to a larger scope of tasks, such that UD can perform discriminative and generative tasks at the same time. Specifically, we extend UD to the encoder-decoder architecture for training on generative tasks, and restrict the model's prediction on "yes"/"no" tokens for jointly training discriminative tasks. Results prove that generalized UD maintains UD's advantages on discriminative tasks and achieves comparable results on generative tasks (See \S~\ref{sec:generalizedud}). 
% We leave expanding UD to a broader range of generative tasks and achieve greater performance on generative tasks as our future work


% \xhk{I admit the limitation on generative tasks here as our future work.}

%\xhk{Although UD is designed for improving zero-shot performance for discriminative tasks, we can also combine this idea to train a generalized UD model which simultaneously solves both discriminative tasks and generative tasks, maintaining UD's advantage on discriminative tasks and get comparable results on generative tasks (See \S~\ref{sec:generalizedud}).}

% The universal discriminator provides a new perspective for zero-shot generalization---Compared with generating the true verbalizer that indicates task label with extensive prompt engineering, distinguishing between options with minimal prompting efforts is simple, robust, and high-performing, thus is more applicable in real-world scenarios. \zy{rewirte the above sentence, just focus on one point---minimal prompting}

\section{Background and Motivation}\label{sec:background}

    % \subsection{Function approximation in MDPs}
    
        Consider an MDP $ \mdp = \langle \statespace, \actionspace, \transition, \cost, \discount \rangle$ where $\statespace$ denotes the state space, $\actionspace$ denotes the action space, $\transition$ denotes the controlled transition matrix, $\cost \colon \statespace \times \actionspace \to \real$ denotes the per-step reward, and $\discount \in (0,1)$ denotes the discount factor.
                
        The performance of a randomised (and possibly history-dependent) policy $\policy$
        starting from a start state $\sts_{0}$ is measured by the value function, defined as:
        \begin{equation}
          \valuefunction^{\policy}(\sts_0) = \expecun{}^{\policy}\bigg[\sum_{\timestep=1}^{\infty}\discount^{\timestep-1}\cost(\State_{\timestep},\Action_{\timestep}) \bigg| \State_{0} = \sts_{0}\bigg].
        \end{equation}
        A policy maximising $\valuefunction^{\policy}(\sts_0)$ over all (randomised and possibly history dependent) policies is called the optimal policy with respect to initial state $s_0$ and is denoted by $\policy^{\star}$.
        %\footnote{This notion can easily be extended to start state distributions.}.
                
        In many applications, $\statespace$ and $\actionspace$ are combinatorially large or uncountable, which makes it intractable to compute the optimal policy. 
        Most practical RL algorithms overcome this hurdle by using function approximation where the state is mapped to a feature space $\featurespace$ using a state abstraction function $\basis:\statespace \to \featurespace$. In Deep-RL algorithms, the last layer of the network is often viewed as a feature vector. These feature vectors are then used as an approximate state for approximating the value function $\hat\valuefunction: \featurespace \to \real$ and/or computing an approximately optimal policy $\mu:\featurespace \to \Delta(\actionspace)$~\citep{Sutton+Barto:1998} (where $\Delta(\actionspace)$ denotes the set of probability distribution over actions). Therefore, the mapping from state to distribution of actions is given by the ``flattened'' policy $\tilde{\mu} = \mu \circ \basis$ ~\ie, $\tilde{\mu} = \mu(\phi(\cdot))$.
        
        % The features $\Feature_\timestep$ are then used as an approximate state for computing the value function $\hat\valuefunction: \featurespace \to \real$ and/or the policy $\mu:\featurespace \to \Delta(\actionspace)$~\citep{Sutton+Barto:1998} (where $\Delta(\actionspace)$ denotes the set of probability distribution over actions). We denote by $\mu \circ \phi $ as the ``flattened" policy that maps states to action distributions, by composing the feature abstraction and the policy acting on it. In Deep-RL algorithms, the output of the last layer of the network is often viewed as the feature vector $\Feature_{\timestep}$.
        
        % A well known yet often overlooked fact about function approximation is that the features used as a proxy state may not satisfy the controlled Markov property \ie, in general,
        A well known fact about function approximation is that the features that are used as an approximate state may not satisfy the controlled Markov property \ie, in general, 
        \[
            \prob(\Feature_{\timestep+1} \mid \Feature_{1:\timestep}, \Action_{1:\timestep}) \neq
            \prob(\Feature_{\timestep+1} \mid \Feature_\timestep, \Action_\timestep).
        \]
        
        % \begin{figure}[!h]
        %   \centering
        %   \begin{minipage}{\linewidth}
        %   \subfigure[$P(0)$]{
        %     \begin{tikzpicture}[thick,scale=0.9]
        %       \node [agent] at (0, 0) (0) {$0$};
        %       \node [agent] at (1, 0) (1) {$1$}; 
        %       \node [agent] at (0, 1) (3) {$3$};
        %       \node [agent] at (1, 1) (2) {$2$};
        %       \path[->]
        %             (0) edge node[below] {$0.5$} (1)
        %             (1) edge node[right] {$0.5$} (2)
        %             (2) edge node[above] {$0.5$} (3)
        %             (3) edge node[left]  {$0.5$} (0)
        %             (0) edge[loop left] node {$0.5$} (0)
        %             (1) edge[loop right] node {$0.5$} (1)
        %             (2) edge[loop right] node {$0.5$} (2)
        %             (3) edge[loop left] node {$0.5$} (3)
        %             ;
        %     \end{tikzpicture}}
        %   \hfill
        %   \subfigure[$P(1)$]{
        %     % \centering
        %     \begin{tikzpicture}[thick,scale=0.9]
        %       \node [agent] at (0, 0) (0) {$0$};
        %       \node [agent] at (1, 0) (1) {$1$}; 
        %       \node [agent] at (0, 1) (3) {$3$};
        %       \node [agent] at (1, 1) (2) {$2$};
        %       \path[<-]
        %             (0) edge node[below] {$0.5$} (1)
        %             (1) edge node[right] {$0.5$} (2)
        %             (2) edge node[above] {$0.5$} (3)
        %             (3) edge node[left]  {$0.5$} (0)
        %             (0) edge[loop left] node {$0.5$} (0)
        %             (1) edge[loop right] node {$0.5$} (1)
        %             (2) edge[loop right] node {$0.5$} (2)
        %             (3) edge[loop left] node {$0.5$} (3)
        %             ;
        %     \end{tikzpicture}}
        %     \hfill
        %     \subfigure[$P(2)$]{%
        %           \begin{tikzpicture}[thick,scale=0.9]
        %           \node [agent] at (0, 0) (0) {$0$};
        %           \node [agent] at (1, 0) (1) {$1$}; 
        %           \node [agent] at (0, 1) (3) {$3$};
        %           \node [agent] at (1, 1) (2) {$2$};
        %           \path[->]
        %                 (0) edge [bend right] node[below] {$0.5$} (1)
        %                 (1) edge [bend right] node[right] {$0.5$} (2)
        %                 (2) edge [bend right] node[above] {$0.5$} (3)
        %                 (3) edge [bend right] node[left]  {$0.5$} (0)
        %                 (0) edge [bend right] node[right] {} (3)
        %                 (3) edge [bend right] node[below] {} (2)
        %                 (2) edge [bend right] node[left]  {} (1)
        %                 (1) edge [bend right] node[above]  {} (0);
        %         \end{tikzpicture}}
        %     \hfill
        %     \subfigure[$P_{\policy{}}$]{
        %     \begin{tikzpicture}[thick,scale=0.9]
        %       \node [agent] at (0, 0) (0) {$0$};
        %       \node [agent] at (1, 0) (1) {$1$}; 
        %       \node [agent] at (0, 1) (3) {$3$};
        %       \node [agent] at (1, 1) (2) {$2$};
        %       \path[->]
        %             (0) edge node[below] {$0.5$} (1)
        %             (1) edge[bend right] node[right] {$0.5$} (2)
        %             (2) edge[bend right]  node[left] {$0.5$} (1)
        %             (3) edge node[left]  {$0.5$} (0)
        %             (3) edge node[above] {$0.5$} (2)
        %             (0) edge[loop left] node {$0.5$} (0)
        %             (1) edge[loop right] node {$0.5$} (1)
        %             (2) edge[loop right] node {$0.5$} (2)
        %             ;
        %     \end{tikzpicture}
        %     % \caption{}
        %     % \label{fig:optimal}
        %     }
        %   \label{fig:P}
        %   \caption{The transition probability for an example MDP}
        %   \end{minipage}%
        % \end{figure}
        
        \begin{figure}[!h]
          \centering
          \begin{minipage}{\linewidth}
          \subfloat[$P(0)$]{
            \begin{tikzpicture}[thick,scale=0.9]
              \node [agent] at (0, 0) (0) {$0$};
              \node [agent] at (1, 0) (1) {$1$}; 
              \node [agent] at (0, 1) (3) {$3$};
              \node [agent] at (1, 1) (2) {$2$};
              \path[->]
                    (0) edge node[below] {$0.5$} (1)
                    (1) edge node[right] {$0.5$} (2)
                    (2) edge node[above] {$0.5$} (3)
                    (3) edge node[left]  {$0.5$} (0)
                    (0) edge[loop left] node {$0.5$} (0)
                    (1) edge[loop right] node {$0.5$} (1)
                    (2) edge[loop right] node {$0.5$} (2)
                    (3) edge[loop left] node {$0.5$} (3)
                    ;
            \end{tikzpicture}
            \label{fig:P(1)}}
           \hfill
          \subfloat[$P(1)$]{
            % \centering
            \begin{tikzpicture}[thick,scale=0.9]
              \node [agent] at (0, 0) (0) {$0$};
              \node [agent] at (1, 0) (1) {$1$}; 
              \node [agent] at (0, 1) (3) {$3$};
              \node [agent] at (1, 1) (2) {$2$};
              \path[<-]
                    (0) edge node[below] {$0.5$} (1)
                    (1) edge node[right] {$0.5$} (2)
                    (2) edge node[above] {$0.5$} (3)
                    (3) edge node[left]  {$0.5$} (0)
                    (0) edge[loop left] node {$0.5$} (0)
                    (1) edge[loop right] node {$0.5$} (1)
                    (2) edge[loop right] node {$0.5$} (2)
                    (3) edge[loop left] node {$0.5$} (3)
                    ;
            \end{tikzpicture}
            \label{fig:P(2)}}
            \hfill
            \subfloat[$P(2)$]{%
                  \begin{tikzpicture}[thick,scale=0.9]
                  \node [agent] at (0, 0) (0) {$0$};
                  \node [agent] at (1, 0) (1) {$1$}; 
                  \node [agent] at (0, 1) (3) {$3$};
                  \node [agent] at (1, 1) (2) {$2$};
                  \path[->]
                        (0) edge [bend right] node[below] {$0.5$} (1)
                        (1) edge [bend right] node[right] {$0.5$} (2)
                        (2) edge [bend right] node[above] {$0.5$} (3)
                        (3) edge [bend right] node[left]  {$0.5$} (0)
                        (0) edge [bend right] node[right] {} (3)
                        (3) edge [bend right] node[below] {} (2)
                        (2) edge [bend right] node[left]  {} (1)
                        (1) edge [bend right] node[above]  {} (0);
                \end{tikzpicture}
                \label{fig:P(3)}}
            \hfill
            \subfloat[$P_{\policy{}}$]{
            \begin{tikzpicture}[thick,scale=0.9]
              \node [agent] at (0, 0) (0) {$0$};
              \node [agent] at (1, 0) (1) {$1$}; 
              \node [agent] at (0, 1) (3) {$3$};
              \node [agent] at (1, 1) (2) {$2$};
              \path[->]
                    (0) edge node[below] {$0.5$} (1)
                    (1) edge[bend right] node[right] {$0.5$} (2)
                    (2) edge[bend right]  node[left] {$0.5$} (1)
                    (3) edge node[left]  {$0.5$} (0)
                    (3) edge node[above] {$0.5$} (2)
                    (0) edge[loop left] node {$0.5$} (0)
                    (1) edge[loop right] node {$0.5$} (1)
                    (2) edge[loop right] node {$0.5$} (2)
                    ;
            \end{tikzpicture}
            \label{fig:optimal}
            }
          \end{minipage}%
          \caption{The transition probability for an example MDP}
        \end{figure}


        To see the implications of this fact, consider the toy MDP depicted in~\cref{fig:P(1),fig:P(2),fig:P(3)}, with $ \statespace = \{0, 1, 2, 3\}$, $\actionspace =
        \{0, 1, 2\}$, $\{\transition_{\sts, \sts'}(\action)\}_{\action \in \actionspace}$, and $r(0) = r(1) = -1$, $r(2) = 1$, $r(3) = -K$, where $K$ is a large positive number.
       Given the reward structure the objective of the policy is to try to avoid state~$3$ and keep the agent at state~$2$ as much as possible. It is easy to see that the optimal policy is 
            \[
              \pi^\star(0) = 0, \quad
              \pi^\star(1) = 0, \quad
              \pi^\star(2) = 1,
              \text{ and}\quad
              \pi^\star(3) = 2.
            \]
            
        Note that if the initial state is not state~$3$ then an agent will never visit that state under the optimal policy. Furthermore, any policy which cannot prevent the agent from visiting state~$3$ will have a large negative value and, therefore, cannot be optimal.
        Now suppose the feature space $\featurespace = \{0, 1\}$. It is easy to see that for any Markovian feature-abstraction $\policyencoder{} \colon \statespace \to \featurespace$, no policy $\hat \pi \colon \featurespace \to \actionspace$ can prevent the agent from visiting state~$3$. Thus, the best policy when using Markovian feature abstraction will perform significantly worse than the optimal policy (which has direct access to the state).
    
        However, it is possible to construct a history-based feature-abstraction  $\policyencoder$ and a history-based control policy $\hat \pi$ that works with $\phi$ and is of the same quality as $\pi^\star$. For this, consider the following \emph{codebooks} (where the entries denoted by a dot do not matter):    
            
        % \begin{wrapfigure}{rt}{0.25\linewidth}
        %   \begin{minipage}{\linewidth}
        %   \centering
        %     \begin{tikzpicture}[thick,scale=0.9]
        %       \node [agent] at (0, 0) (0) {$0$};
        %       \node [agent] at (1, 0) (1) {$1$}; 
        %       \node [agent] at (0, 1) (3) {$3$};
        %       \node [agent] at (1, 1) (2) {$2$};
        %       \path[->]
        %             (0) edge node[below] {$0.5$} (1)
        %             (1) edge[bend right] node[right] {$0.5$} (2)
        %             (2) edge[bend right]  node[left] {$0.5$} (1)
        %             (3) edge node[left]  {$0.5$} (0)
        %             (3) edge node[above] {$0.5$} (2)
        %             (0) edge[loop left] node {$0.5$} (0)
        %             (1) edge[loop right] node {$0.5$} (1)
        %             (2) edge[loop right] node {$0.5$} (2)
        %             % (3) edge[loop left] node {$0.5$} (3)
        %             ;
        %     \end{tikzpicture}
        %   \caption{The transition probability of $\pi^\star$}
        %   \label{fig:optimal}
        %   \end{minipage}
        %   \vspace{-3mm}
        % \end{wrapfigure}
        
        The Markov chain induced by the optimal policy is shown in~\Cref{fig:optimal}. 
        Now define
        \begin{align*}
          F(1) &= \begin{bmatrix}
            0 & 1 & \cdot & \cdot \\
            \cdot & 0 & 1 & \cdot \\
            \cdot & \cdot & 0 & 1 \\
            1 & \cdot & \cdot & 0
          \end{bmatrix}, 
          &
          F(2) &= \begin{bmatrix}
            1 & \cdot & \cdot & 0 \\
            0 & 1 & \cdot & \cdot \\
            \cdot & 0 & 1 & \cdot \\
            \cdot & \cdot & 0 & 1
          \end{bmatrix} ,
          &
          F(3) &= \begin{bmatrix}
            \cdot & 0 & \cdot & 1 \\
            0 & \cdot & 1 & \cdot \\
            \cdot & 0 & \cdot & 1 \\
            0 & \cdot & 1 & \cdot
          \end{bmatrix},
          \\
          D(0) &= \begin{bmatrix}
            0 & 1 \\
            1 & 2 \\
            2 & 3 \\
            3 & 0 
          \end{bmatrix} ,
          &
          D(1) &= \begin{bmatrix}
            3 & 0 \\
            0 & 1 \\
            1 & 2 \\
            2 & 3 
          \end{bmatrix} ,
          &
          D(2) &= \begin{bmatrix}
            1 & 3 \\
            0 & 2 \\
            1 & 3 \\
            0 & 2 
          \end{bmatrix} .
        \end{align*}
        
      
        and consider the feature-abstraction policy
        \(
          Z_t =  F_{S_{t-1}, S_t}(A_{t-1})
        \)
        and a control policy $\mu$ which is a finite state machine with memory, where the memory $M_t$ that is updated as
        \(
          M_t = D_{M_{t-1}, Z_t}(A_{t-1})
        \)
        and the action $A_t$ is chosen as
        \(
          A_t = \pi(M_t),
        \)
        where $\pi \colon \statespace \to \Delta(\actionspace)$ is any pre-specified reference policy. It can be verified that if the system starts from a known initial state then $\mu \circ \basis = \pi$. 
         Thus, if we choose the reference policy $\pi=\pi^\star$, then
        the agent will never visit state~$3$ under $\mu \circ \basis$, in contrast to Markovian feature-abstraction policies where (as we argued before) state~$3$ is always visited.
        
        In the above example, we used the properties of the system dynamics and the reward function to design a history-based feature abstraction which outperforms memoryless feature abstractions. We are interested in developing such history-based feature abstractions using a learning framework when the system model is not known. We present such a construction in the next section.
        
        %From the above example, we can see that when system dynamics are known, one can exploit the problem structure to design history-based feature abstractions which outperform their memoryless counterparts.   However, such constructions are not feasible in the general learning setup, where the system dynamics are unknown. 
        % One way to overcome this issue is to let the policy $\policy{}$ be a function of all the information available to the system at time $\timestep$  ~\ie, $\policy{}_\timestep: \historyspace_{\timestep} \to \Delta(\actionspace)$ where, $\historyspace_\timestep \define \{\statespace^{1:\timestep},\featurespace^{1:\timestep-1}, \actionspace^{1:\timestep-1}\}$. 
        %We can overcome this issue by letting the policy be a function of the history of state, actions and observations observed until time $\timestep$, \ie, $\policy{}_\timestep: \historyspace_{\timestep} \to \Delta(\actionspace)$ where, $\historyspace_\timestep \define \{\statespace \times \featurespace \times \actionspace\}$. 
        
        % Since the environment is an MDP, the state $\State_\timestep$ is  a sufficient statistic and the loss of information happens when $\State_\timestep$ is mapped to $\Feature_\timestep$. This is a subtle yet critical distinction between POMDPs and function approximation in MDPs. In POMDPs the state signal is unavailable, and the system only sees a observation vector $\Feature_\timestep$. Whereas, in function approximation setup, we can observe the state signal perfectly and partial observability is induced due to feature abstraction. Therefore, we can discard the the history of states and focus on the policies of the following form $\policy{}_\timestep: \historyspace_{\timestep} \to \Delta(\actionspace)$ where, $\historyspace_\timestep \define \{\statespace,\featurespace^{1:\timestep-1}, \actionspace^{1:\timestep-1}\}$.   
        
        %As the size of history grows with time $t$ and it is computationally and conceptually difficult to use it for deriving a dynamic programming decomposition. One solution is to map the history to a fixed dimensional representation and use it as a state for designing a dynamic program. One can think of such a representation as an information state~\citep{kwakernaakh.1965,infostate-Witsenhausen,Striebel1965SufficientSI,Bohlin1970InformationPF,Davis1972InformationSF,Kumar1986StochasticSE}.  Due to lossy compression a prefect information state may be hard to obtain, but we an construct an approximate information state (AIS) such that it satisfies the controlled Markov property and captures the sufficient information needed for identifying a good policy. In the next section we show how to construct and AIS and use it for obtaining a dynamic program to obtain a policy with a bounded approximation error. 
        
        
        
    \section{Approximation bounds for history-based feature abstraction}\label{sec:main}
         
        The approximation results of our framework depend on the properties of metrics on probability spaces. We start with a brief overview of a general class of metrics known as Integral Probability Measures (IPMs)~\citep{ipm}; many of the commonly used metrics on probability spaces such as total variation (TV) distance, Wasserstein distance, and maximum-mean discrepency (MMD) are instances of IPMs. We then derive a general approximation bound that holds for general IPMs, and then specialize the bound to specific instances (TV, Wassserstein, and MMD).
        
        %In particular, we derive results for Total Variation (TV) distance, Wasserstein/Kantorovich-Rubinstein distance and Maximum-Mean Discrepancy (MMD) distance. All of these metrics are instances of Integral Probability Measures (IPMs)~\citep{ipm}---a class of metrics that have a dual characterisation. Viewing all the above metrics through the lens of IPMs allows us to derive a general approximation bound that holds for all the instances of IPMs, and then specialise the bound to specific instances \ie, TV, Wasserstein and MMD distance. Before defining our main model we will briefly describe the important properties of IPMs necessary for deriving our results.
        
        \subsection{Integral probability metrics (IPM)}
            \begin{definition}[~\citep{ipm}]\label{def:ipm}
                    Let $(\mathcal{E},\mathcal{G})$ be a measurable space and $\mathfrak{F}$ denote a class of uniformly bounded measurable functions on $(\mathcal{E},\mathcal{G})$. The integral probability metric between two probability distributions  $\nu_1, \nu_2 \in \mathcal{P}(\mathcal{E})$ with respect to the function class $\mathfrak{F}$ is defined as:
                    \begin{align}
                        \ipm(\nu_1, \nu_2) &= \sup_{f \in \mathfrak{F}}\bigg| \int_{\mathcal{E}}f d\nu_1  - \int_{\mathcal{E}}f d\nu_2\bigg|.\label{eq:def-ipm}
                    \end{align}
                    
                    For any function $f$ (not necessarily in $\mathfrak{F}$), the Minkowski functional $\rho_{\mathfrak{F}}$ associated with the metric $\ipm$ is defined as:
                    \begin{align}
                        \rho_{\mathfrak{F}}(f)&\define \inf\{\rho \in \real_{\geq 0}: \rho^{-1}f \in \mathfrak{F}\}.\label{eq:minkowski-functional}
                    \end{align}
                    
                    Eq.~\eqref{eq:minkowski-functional}, implies that that for any function $f$:
                    \begin{align}
                         \bigg|\int_{\mathcal{E}}fd\nu_1 - \int_{\mathcal{E}}fd\nu_2 \bigg|\leq \rho_{\mathfrak{F}}(f)\ipm(\nu_1,\nu_2). \label{eq:ipm-implication}
                     \end{align}\label{eq:ipm-function-diff}
                    
            \end{definition}
            \noindent In this paper, we use the following IPMs:
            \begin{compactitem}
                \item[1.]\label{def:tv-dist}{\bf Total Variation Distance}: If $\mathfrak{F}$ is chosen as $\mathfrak{F}^{\text{TV}} \define \{\frac{1}{2}\spn(f)$ = $\frac{1}{2}(\max(f)- \min(f))\}$, then $\ipm$ is the total variation distance, and its Minkowski functional is $\rho_{\mathfrak{F}^{\text{TV}}}(f) = \frac{1}{2}\spn(f)$. 
                \item[2.]\label{def:kr-dist}{\bf Wasserstein/Kantorovich-Rubinstein Distance}: If $\mathcal{E}$ is a metric space and $\mathfrak{F}$ is chosen as $\mathfrak{F}^{W} \define \{f: L_f \leq 1 \}$ (where $L_f$ denotes the Lipschitz constant of $f$ with respect to the metric on $\mathcal{E}$), then $\ipm$ is the Wasserstein or the Kantorovich distance. The Minkowski function for the Wasserstein distance is $\rho_{\mathfrak{F}^W}(f) = L_f$.
                \item[3.] \label{def:mmd-dist}{\bf Maximum Mean Discrepancy (MMD) Distance}: Let $\mathcal{U}$ be a reproducing kernel Hilbert space (RKHS) of real-valued functions on $\mathcal{E}$ and $\mathfrak{F}$ is choosen as $\mathfrak{F}^{MMD} \define \{f\in \mathcal{U}: \Vert f \Vert _{\mathcal{U}} \leq 1 \}$, (where $\Vert \cdot \Vert_{\mathcal{U}}$ denotes the RKHS norm), then $\ipm$ is the Maximum Mean Discrepancy (MMD) distance and its Minkowski functional is $\rho_{\mathfrak{F}^{\text{MMD}}}(f) = \Vert f\Vert _{\mathcal{U}}$.
            \end{compactitem}
 
        % As mentioned previously, we can use the history of state, action and observations to learn the feature abstractions, but it possible to simplify this information structure further. Note that, as the environment is an MDP, the state $\State_\timestep$ is a sufficient statistic. The loss of information happens when $\State_\timestep$ is mapped to $\Feature_\timestep$. This is a subtle yet critical distinction between POMDPs and function approximation in MDPs. In POMDPs the state signal is unavailable, and the system only sees a observation vector $\Feature_\timestep$. Whereas, in function approximation setup, we can observe the state signal perfectly and partial observability is induced due to feature abstraction. Therefore, we can discard the the history of states and focus on the policies of the following form $\policy{}_\timestep: \historyspace_{\timestep} \to \Delta(\actionspace)$ where, $\historyspace_\timestep \define \{\statespace,\featurespace^{1:\timestep-1}, \actionspace^{1:\timestep-1}\}$. Despite this simplification the size of history is growing with time and history compression is still required. As such, we will now proceed towards defining the AIS for MDPs to address all of the aforementioned issues. 
        
        \subsection{Approximate information state}
            
            Given an MDP $\mdp$ and a feature space $\aisspace$, let $\historyspace_\timestep = \statespace \times  \actionspace $ denote the space of all histories $(\State_{1:\timestep}, \Action_{1:\timestep-1})$ up to time~$t$, where $\State_{1:\timestep}$ is a shorthand notation for the history of states $(\State_1,\ldots, \State_\timestep)$, and similar interpretation holds for $\Action_{1:\timestep}$. We are interested in learning history-based feature abstraction functions $\{ \aisfunction_\timestep \colon \historyspace_\timestep \to \aisspace \}_{\timestep \ge 1}$ and a time homogenous policy $\mu \colon \aisspace \to \Delta(\actionspace)$ such that the flattened policy $\policy{} = \{\policy{}_\timestep\}_{\timestep \ge 1}$, where $\policy{}_\timestep = \mu \circ \aisfunction_\timestep$, is approximately optimal.
            
            % We are interested in practical history-based feature abstractions, which often tend to be implemented as RNNs. So, we consider abstractions which are recursively updatable as defined below:
        
            Since the feature abstraction approximates the state, its quality depends on how well it can be used to approximate the per step reward and predict the next state. We formalise this intuition in definition below.
            
            
            \begin{definition}\label{def:state-update}
                A family of history-based feature abstraction functions  $\{\aisfunction_\timestep: \historyspace_{\timestep} \to \featurespace\}_{\timestep \geq 1}$ are said to be \emph{recursively updatable} if there exists an update function $\hat f: \featurespace \times \statespace \times \actionspace \to \featurespace $ such that the process $\{\Feature_\timestep\}_{\timestep\geq 1}$, where $\Feature_\timestep = \aisfunction_\timestep(\State_{1:\timestep}, \Action_{1:\timestep-1})$, satisfies:
                \begin{equation}
                    \Feature_{\timestep +1} = \hat f(\Feature_\timestep, \State_{\timestep+1}, \Action_\timestep).\quad \timestep \geq 1
                \end{equation}
            \end{definition}
            \begin{definition}\label{def:ais}
                Given a family of history based recursively updatable feature abstraction functions $\{\aisfunction_\timestep: \historyspace_{\timestep} \to \featurespace\}_{\timestep \geq 1}$, the features $\Feature_\timestep = \aisfunction_\timestep(\State_{1:\timestep}, \Action_{1:\timestep-1})$ are said to be \emph{$(\epsilon, \delta)$-approximate information state} (AIS) with respect to a function space $\mathfrak{F}$ if there exist: (i)~a reward approximation function $\hat r: \featurespace \times \actionspace \to \real$, and (ii)~an approximate transition kernel $\hat\transition: \featurespace \times \actionspace \to \Delta(\statespace)$ such that $\Ais$ satisfies the following properties:
            \begin{compactitem}
                \item[(P1)] \label{p1} Sufficient for approximate performance evaluation: for all $\timestep$,
                \begin{equation}
                     | \cost(\State_{\timestep}, \Action_{\timestep})
                                            - \hat{\cost}(\Feature_\timestep, \Action_\timestep)| \leq \epsilon.
                 \label{def:p1}
                \end{equation}
                \item[(P2)]\label{p2b} Sufficient for predicting future states approximately: for all $\timestep$
                \begin{equation}
                 d_{\mathfrak{F}}(\transition(\cdot \vert \State_\timestep, \Action_\timestep), \hat \transition(\cdot\vert \Feature_\timestep, \Action_\timestep)) \leq \delta.
                % \text{with }\kappa_{\ipm}(\hat{f}_\timestep) &\define \sup_{\history_\timestep, \action_\timestep} \bigg[ \kappa_{\ipm}(\hat{f}_\timestep (\aisfunction_\timestep (\history_\timestep), \cdot, \action_\timestep))\bigg].
                \end{equation}
            \end{compactitem}
        \end{definition}
        We call the tuple $(\hat r ,\hat\transition)$ as an $(\epsilon, \delta)$-AIS approximator. Note that similar definitions have appeared in other works \eg, latent state \citep{deepmdp}, and approximate information state for for POMDPs \citep{ais-1,ais-2}. However, in \citep{deepmdp} it is assumed that the feature abstractions are memory-less and the discussion is restricted to Wasserstein distance. The key difference from the POMDP model in \citep{ais-1,ais-2} is that the in POMDPs the observation $\Feature_\timestep$ is a pre-specified function of the state while in the proposed model $\Feature_\timestep$ depends on our choice of feature abstraction.
        
        As such, our key insight is that an AIS-approximator of a recursively updatable history-based feature abstraction can be used to define a dynamic program. In particular, given a history-based abstraction function $\{\aisfunction_\timestep: \historyspace_\timestep \to \featurespace\}_{\timestep \geq 1}$ which is recursively updatable using $\hat f$ and an $(\epsilon, \delta)$ AIS-approximator $(\hat \transition, \hat \cost)$, we can define the following dynamic programming decomposition: 
        
        For any $\feature_\timestep \in \featurespace, \ \action_\timestep \in \actionspace$
        \begin{subequations}\label{eq:ais-dp}
            \begin{align}
                   \hat Q(\ais_\timestep, \action_\timestep) = \hat \cost(\ais_\timestep, \action_\timestep) + \discount \sum_{\sts_{\timestep+1} \in \statespace}
                    \hat \transition(\sts_{\timestep+1}|\ais_\timestep,\action_\timestep) \hat \valuefunction(\hat{f}(\ais_\timestep,\sts_{\timestep+1},\action_\timestep));&&
                    \hat \valuefunction(\ais_\timestep) = \max_{\action_\timestep \in \actionspace} \hat Q(\ais_\timestep,\action_\timestep), \quad\forall \feature_\timestep \in \featurespace 
            \end{align}
        \end{subequations}
        \begin{definition}\label{def:policy}
            Define $\mu \colon \aisspace \to \Delta(\actionspace)$ be any policy such that for any $\ais \in \aisspace$,
            \begin{align}
                \support(\mu(\ais)) \subseteq 
                \arg\max_{\action \in \actionspace} \hat Q(\ais,\action).\label{eq:policy}
            \end{align}
         Since $\mu$ is a policy from the feature space to actions, we can use it to define a policy from the history of the state action pairs to actions as:
         \begin{align}
             \policy{}_{\timestep}(\sts_{1:\timestep}, \action_{1:\timestep-1}) \define \mu(\aisfunction_{\timestep}(\sts_{1:\timestep}, \action_{1:\timestep-1})) \label{eq:policy-definition}
         \end{align}
        \end{definition}
        Therefore, the dynamic program defined in \eqref{eq:ais-dp} indirectly defines a history-based policy $\policy{}$. The performance of any such history-based policy is given by the following dynamic program: 
        
        For any $\feature \in \featurespace, \ \action \in \actionspace$
        \begin{subequations}
            \begin{align}
                     Q^{\policy{}}_{\timestep}(\history_\timestep, \action_\timestep) = \cost(\sts_\timestep, \action_\timestep) + \discount \sum_{\sts_{\timestep+1} \in \statespace}
                    \transition(\sts_{\timestep+1}|\sts_\timestep,\action_\timestep) \valuefunction_{\timestep+1}^{\policy{}}(\history_{\timestep+1}); && 
                     \valuefunction_{\timestep}^{\policy{}}(\history_\timestep) = \max_{\action \in \actionspace}  Q_{\timestep}^{\policy{}}(\history_\timestep,\action_\timestep), \label{eq:hist-dp}
            \end{align}
        \end{subequations}
        We want to quantify the loss in performance when using the history based policy $\policy$. Note that since $\valuefunction_\timestep^\policy$ is not time-homogeneous, we need to compute the worst-case difference between $\valuefunction^{\star}$ and $\valuefunction_\timestep^\policy$, which is given by:
        \begin{equation}
            \Delta \define\sup_{\timestep \geq 0}\sup_{\history_\timestep = (\sts_{1:\timestep}, \action_{1:\timestep}) \in \historyspace_\timestep} \vert \valuefunction^{\star}(\sts_\timestep) - \valuefunction^{\policy{}}_{\timestep}(\history_\timestep)\vert, \label{eq:sup-v}
        \end{equation}
        
        Our main approximation result is the following:
    
        % A natural question which then arises is that \emph{how far from optimal is $\policy{}$?}
        % We answer this question in the following result:
        \begin{theorem}\label{thm:ais-dp}
            % For any time $\timestep$, any realisation $\sts_\timestep$ of $\State_\timestep$, $\action_\timestep$ of $\Action_\timestep$, let $\history_\timestep = (\sts_{1:\timestep}, \action_{1:\timestep-1})$, and $\ais_\timestep = \aisfunction_\timestep(\history_\timestep)$.
             The worst case difference between $\valuefunction^{\star}$ and $\valuefunction^{\policy{}}_{\timestep}$ is bounded by
            %     \begin{align}
            %     \Delta &\define \sup_{\timestep \geq 0}\sup_{\history_\timestep = (\sts_{1:\timestep}, \action_{1:\timestep}) \in \historyspace_\timestep} \vert \valuefunction^{\star}(\sts_\timestep) - 
            %      \valuefunction^{\policy{}}(\ais_\timestep)\vert. \label{eq:sup-v}
            %     \end{align}
            % Then 
            \begin{equation}
                \Delta
                \le 2 \frac{\varepsilon + \discount\delta \kappa_{\mathfrak{F}}(\hat \valuefunction^{\mu}, \hat{f})}{1 - \discount},\label{eq:ais-bound}
            \end{equation}
        where $\kappa_{\mathfrak{F}}(\hat \valuefunction^{\mu}, \hat f)$ = $ \sup_{\feature, \action}\rho_{\mathfrak{F}}(\hat \valuefunction^{\mu}(\hat f(\cdot, \feature, \action)))$, $\rho_{\mathfrak{F}}(\cdot)$ is the Minkowski functional associated with the IPM $\ipm$ as defined in \eqref{eq:minkowski-functional}.
        \end{theorem}
        Proof in \Cref{sec:proof:thm:ais-dp}
        
        % The above result can be interpreted as follows. Given any history based feature abstraction function and a parametric family $\mathfrak{R}(\aisparams)$ and $\mathfrak\transition(\aisparams)$ of functions (where $\aisparams$ are the parameters) we can find the best AIS approximation as: 
        % \begin{align}
        %     \hat \cost^{\star} &= \arginf_{\hat \cost(\cdot;\aisparams) \in \mathfrak\Cost(\aisparams)} \Vert \cost - \hat \cost(\cdot; \aisparams)\Vert_\infty.
        % \end{align}
        % and 
        % \begin{align}
        %     \hat \transition^{\star} &= \arginf_{\hat \transition(\cdot; \aisparams) \in \mathfrak\transition( \aisparams)}\sup_{\timestep \geq 0}\sup_{\history_\timestep \in \historyspace_\timestep}\ipm(\transition(\cdot\vert\sts_\timestep,\action_\timestep), \hat\transition(\cdot\vert \ais_\timestep, \action_\timestep, \aisparams)).
        % \end{align}
        % Then define
        % \begin{align}
        %     \epsilon &= \Vert \cost - \hat \cost^{\star}(\cdot;\aisparams)\Vert_{\infty}.\\
        %     \delta &= \sup_{\timestep \geq 0}\sup_{\history_\timestep \in \historyspace_\timestep} \ipm(\transition\cdot\vert\sts_\timestep,\action_\timestep), \hat\transition^{\star}(\cdot\vert\ais_\timestep,\action_\timestep, \aisparams)).
        % \end{align}
        % Then, by construction $(\hat \cost^{\star}, \hat \transition^{\star})$ is an $(\epsilon, \delta)$ AIS-approximator, and is the best approximation within a parametric family $\mathfrak\Cost(\aisparams), \mathfrak\transition(\aisparams)$. \Cref{thm:ais-dp} then provides the approximate error in choosing a history based policy based on $(\hat \cost^{\star}, \hat \transition^{\star})$, which is obtained using \eqref{eq:policy}.
        
        Some salient features of the bound are as follows:
        First, the bound depends on the choice of metric on probability spaces. Different IPMs will result in a different value of $\delta$ and also a different value of $\kappa_{\mathfrak F}(\hat{\valuefunction}^{\mu}, \hat f)$. Second, the bound depends on the properties of $\hat{\valuefunction}^{\mu}$. For this reason we call it an instance dependent bound. Sometimes, it is desirable to have bounds which do not require solving the dynamic program in \eqref{eq:ais-dp}. We present such bounds as below, note that these ``instance independent'' bounds are the derived by upper bounding $\kappa_{\mathfrak{F}}(\hat{\valuefunction}^{\mu}, \hat f)$. Therefore, these are looser than the upper bound in \Cref{thm:ais-dp}
        
        % In \eqref{eq:ais-bound}, $\epsilon$ and $\delta$, capture the worst case error incurred by the AIS when predicting instantaneous reward $\cost$ and approximating the transition distribution of the ground MDP $\mdp$. Therefore, resulting suboptimality bound helps us quantify us the loss in performance due to feature abstraction. At the same time, the IPM used for measuring the distance between the approximate and true transition distribution also influences the bound via $\kappa_{ \mathfrak{F}}(\hat \valuefunction^{\mu}, \hat f)$. In the following corollaries we will show how $\kappa_{ \mathfrak{F}}(\hat \valuefunction^{\mu}, \hat f)$ takes a specific form according the choice of the IPM.
        
         \begin{corollary}\label{THM:TV-BOUND}
             If the function class $\mathfrak{F}$ is $\mathfrak{F}^{\text{TV}}$, then $\Delta$ as defined in \eqref{eq:sup-v} is upper bounded as:
             \begin{align}
              \Delta
                \le  \frac{2\epsilon}{(1-\discount)} +  \frac{\discount\delta \spn(\hat\cost)} {(1-\discount)^2}.
            \end{align}
        \end{corollary}
        
            Proof in Appendix \ref{sec:tv-proof}
        
        \begin{corollary}\label{THM:LIP-BOUND}
            Let $L_{\hat \cost}$ and $L_{\hat{\transition}}$ denote the Lipschitz constants of the approximate reward function $\hat \cost$ and approximate transition function $\hat \transition$ respectively, and $L_{\hat f}$ is the uniform bound on the Lipschitz constant of $\hat f$ with respect to the state $\State_\timestep$.
            If $\discount L_{\hat \transition}L_{\hat f} \leq 1$ and the function class $\mathfrak{F}$ is $\mathfrak{F}^{\text{W}}$, then $\Delta$ as defined in  \eqref{eq:sup-v} is upper bounded as:
            \begin{align}
                \Delta
                \le  \frac{2\epsilon}{(1-\discount)} + \frac{2\discount\delta L_{\hat \cost} }{(1- \discount)(1-\discount L_{\hat f}L_{\hat\transition})}.
            \end{align}
        \end{corollary}

            Proof in Appendix \ref{sec:w-proof}

        \begin{corollary}\label{thm:mmd-bound}
             If the function class $\mathfrak{F}$ is $\mathfrak{F}^{\text{MMD}}$, then $\Delta$ as defined in \eqref{eq:sup-v} is upper bounded as:
             \begin{align}
              \Delta
                \le  2 \frac{\epsilon + \discount\delta\kappa_{\mathcal{U}}(\hat \valuefunction, \hat f) } {(1-\discount)},
            \end{align}
            where $\mathcal{U}$ is a RKHS space,  $\Vert\cdot\Vert_{\mathcal{U}}$ its associated norm and $\kappa_{\mathcal{U}}(\hat \valuefunction, \hat f) = \sup_{\feature, \action}\Vert(\hat \valuefunction(\hat f(\cdot, \feature, \action)))\Vert_{\mathcal{U}}$.
        \end{corollary}
        \begin{proof}
            The proof follows from the properties of MMD described previously. 
        \end{proof}
        
        In the following section we will show how one can use these theoretical insights to design a policy search algorithm.
        
\section{Reinforcement learning with history-based feature abstraction}\label{sec:algorithm}
In this section, we leverage the approximation bounds of
Theorem~\ref{thm:ais-dp} to develop a reinforcement learning algorithm. The
main idea is to add an additional block, which we call the AIS-approximator,
to any standard RL algorithm. In this section, we explain an AIS-based
generalization for policy-based algorithms such as REINFORCE and actor-critic, but the same idea could be used for
value-based algorithms such as Q-learning as well. 
% \begin{figure*}
%     \centering
%     \includegraphics[width=\linewidth]{AAMAS-2023-Formatting-Instructions/Results/nns.pdf}
%     \caption{AIS approximator block} \label{fig:blk-diag}
% \end{figure*}


\begin{wrapfigure}{rt}{0.6\textwidth}
      \vspace*{-6mm}
      \includegraphics[width=0.6\textwidth]{Results/nns.pdf}
      \vspace*{-5mm}
      \caption{AIS approximator block} \label{fig:blk-diag}
      \vspace*{-5mm}
\end{wrapfigure}

The AIS-approximator consists of two blocks: a recursively updatable history
compressor and a reward and next-state predictor as shown in
Fig.~\ref{fig:blk-diag}. In particular, we can consider any parameterised family of
the history compression functions
$\{\aisfunction_\timestep(\cdot); \aisparams) \colon \historyspace_\timestep \to
\aisspace\}$ which are recursively updatable via the function
$\hat{f}(\cdot) \colon \aisspace \times \statespace\times \actionspace \to
\aisspace$ as the history-compressor along with any parameterised family of
functions $\hat \cost(\cdot; \aisparams)\colon
\aisspace \times \actionspace \to \real$ as the reward approximator and any
parameterised stochastic kernels ${\hat
\transition}(\cdot;\aisparams)\colon\aisspace \times \actionspace \to
\Delta(\statespace)$ as the transition approximator. In the above notation $\aisparams$ denotes the
combined parameters of the family of functions. As a concrete example, we could use 
use memory-based neural networks such as LSTMs or GRUs as the
history-compression functions. The memory update functions of such networks
correspond to the update function $\hat f$. A multilayered perceptron (MLP)
could be used as a reward approximator and a parameterized family of
stochastic kernels such as the softmax function or a mixture of Gaussians
could be used as the transition approximator. The parameters of all these
networks together are denoted by $\aisparams$.

We use a weighted combination of the reward prediction loss $\vert
\cost(\State_\timestep, \Action_\timestep) - \hat\cost (\Ais_\timestep,
\Action_\timestep)\vert$ and the transition-prediction loss $\ipm(\transition,
\hat \transition)$ as the loss function for the AIS-generator. In particular,
the AIS-loss is given by
    \begin{align}
          \aisloss(\aisparams) &= \frac{1}{\Timestep}\sum_{t = 0}^{\Timestep}\bigg( \lambda \underbrace{({\hat{\cost}}(\Ais_{\timestep}, \Action_\timestep; \aisparams)  - \cost(\State_\timestep, \Action_\timestep))^{2}}_{\loss_{\hat{\Cost}(\cdot;\aisparams)}}+ (1-\lambda)\cdot \underbrace{\ipm({\hat\transition}(\Ais_\timestep, \Action_\timestep\ ;\aisparams),\transition)^{2}}_{\loss_{\hat\transition}(\cdot;\aisparams)}\bigg),\label{eq:pgt-loss}
    \end{align}
    where $\Timestep$ is the length of the episode or the rollout length, $\lambda \in [0,1]$ is a hyper-parameter.
    % reward prediction loss $\loss_{\hat{\Cost}}(;\aisparams)$ is simply the mean-squared error between the predicted and the observed reward, whereas the transition prediction loss $\loss_{\hat{\transition}}(\cdot; \aisparams)$ is the distance between predicted and observed transition distributions $\hat\transition$ and $\transition$. 
    The computation of $\loss_{\hat{\transition}}(\cdot; \aisparams)$, depends on the choice of IPM. In principle we can pick any IPM, but we would want to use an IPM using which the distance $d_{\mathfrak{F}}$ can be efficiently computed.

     %Instead of separately optimising the reward prediction loss $\vert \cost(\State_\timestep, \Action_\timestep) - \hat\cost (\Ais_\timestep, \Action_\timestep)\vert$, and the transition loss $\ipm(\transition, \hat \transition)$ we can combine them in a single objective function objective function as: 
%    $(\sigma_t(\cdot; \aisparams), \mathsf{R}(\aisparams), \mathsf\transition(\aisparams))$, where $\aisparams$ are the parameters. 

%    From the previous section we know that a history-based representation can be called an AIS if it is able to evolve like a state and approximately predict the instantaneous reward and state transition. From a practical point of view, the result in \cref{thm:ais-dp} can be interpreted as follows: Given any history based feature abstraction function and a parametric family $\mathsf{R}(\aisparams)$ and $\mathsf\transition(\aisparams)$ of functions (where $\aisparams$ are the parameters) we can find the best AIS approximation as: 
%    \begin{equation}
%      \hat \cost^{\star} = \arginf_{\hat \cost(\cdot;\aisparams) \in \mathsf\Cost(\aisparams)} \Vert \cost - \hat \cost(\cdot; \aisparams)\Vert_\infty \quad and \quad
%       \hat \transition^{\star} = \arginf_{\hat \transition(\cdot; \aisparams) \in \mathsf\transition( \aisparams)}\sup_{\timestep \geq 0}\sup_{\history_\timestep \in \historyspace_\timestep}\ipm(\transition(\cdot\vert\sts_\timestep,\action_\timestep), \hat\transition(\cdot\vert \ais_\timestep, \action_\timestep, \aisparams)).
%    \end{equation}
%    We can then define: $\epsilon = \Vert \cost - \hat \cost^{\star}(\cdot;\aisparams)\Vert_{\infty}$, and $\delta = \sup_{\timestep \geq 0}\sup_{\history_\timestep \in \historyspace_\timestep} \ipm(\transition\cdot\vert\sts_\timestep,\action_\timestep), \hat\transition^{\star}(\cdot\vert\ais_\timestep,\action_\timestep, \aisparams))$

%    Therefore, by construction $(\hat \cost^{\star}, \hat \transition^{\star})$ is an $(\epsilon, \delta)$ AIS-approximator, and is the best approximation within a parametric family $\mathsf\Cost(\aisparams), \mathsf\transition(\aisparams)$. \Cref{thm:ais-dp} then provides the approximate error in choosing a history based policy based on $(\hat \cost^{\star}, \hat \transition^{\star})$, which is obtained using \eqref{eq:policy}.

%    In the rest of this section we will show how one can design a RL algorithm to simultaneously learn an AIS and a policy. As mentioned previously, the key idea is to represent the AIS generator and the policy using a parametric family of functions/distributions and training them using a multi-timescale optimisation algorithm. 

%    According to \Cref{def:ais}, the AIS generator consists of four components, a compression function $\aisfunction_\timestep$, the update function $\hat f$, an approximate reward predictor $\hat \cost$, and transition kernel $\hat \transition$. We can represent the history compression function using any time series approximators such as LSTMs or GRUs. An advantage of such memory based neural networks is that their internal layers are updated in a state-like manner. Therefore, we can satisfy \Cref{def:state-update} since $\Ais_\timestep$ evolves according to the RNN's state update function such that $\hat{f} \colon \aisspace \times \statespace \times \actionspace \to \aisspace$.

%    Next we can model the reward predictor $\hat \cost$ using a multilayered perception (MLP) layer which uses the representation $\Feature_\timestep$ and action $\Action_\timestep$ to predict the approximate reward. In the same way, we can model the approximate transition kernel $\hat \transition$ using an appropriate class of stochastic kernel approximators \eg, a softmax function or a mixture of Gaussian's to learn a parametric approximation of $\transition$. We can then train the AIS generator by minimising an appropriate objective function.

 %    To make things more concrete, let us denote the AIS generator as the following collection: $\{\aisfunction_\timestep(\cdot;\aisparams), \hat{f}(\cdot;\aisparams), {\hat{\cost}}(\cdot;\aisparams), {\hat{\transition}}(\cdot:\aisparams)\}$ where ${\hat \cost}(\cdot;\aisparams):
%     \aisspace \times \actionspace \to \real$ and ${\hat \transition}(\cdot;\aisparams):\aisspace \times \actionspace \to \Delta(\statespace)$ are the reward and transition approximators, and $\aisparams$ are the parameters of the respective sub-components. Instead of separately optimising the reward prediction loss $\vert \cost(\State_\timestep, \Action_\timestep) - \hat\cost (\Ais_\timestep, \Action_\timestep)\vert$, and the transition loss $\ipm(\transition, \hat \transition)$ we can combine them in a single objective function objective function as: 
%
%      \begin{align}
%                \aisloss(\aisparams) &= \frac{1}{\Timestep}\sum_{t = 0}^{\Timestep}\bigg( \lambda \underbrace{({\hat{\cost}}(\Ais_{\timestep}, \Action_\timestep; \aisparams)  - \cost(\State_\timestep, \Action_\timestep))^{2}}_{\loss_{\hat{\Cost}} \approx \epsilon} 
%                + (1-\lambda)\cdot \underbrace{\ipm({\hat\transition}(\Ais_\timestep, \Action_\timestep\ ;\aisparams),\transition)^{2}}_{\loss_{\hat\transition} \approx \delta}\bigg),\label{eq:ais-loss}
%    \end{align}
%    where, $\Timestep$ is the length of the episode or the rollout length, $\lambda \in [0,1]$ is a hyper-parameter, reward prediction loss $\loss_{\hat{\Cost}}(;\aisparams)$ is simply the mean-squared error between the predicted and the observed reward, whereas the transition prediction loss $\loss_{\hat{\transition}}(\cdot; \aisparams)$ is the distance between predicted and observed transition distributions $\hat\transition$ and $\transition$. To compute $\loss_{\hat{\transition}}(\cdot; \aisparams)$, we need to choose an IPM. In principle we can pick any IPM, but we would want to use an IPM using which the distance $d_{\mathfrak{F}}$ can be efficiently computed.
%
    %  Besides the AIS generator the system also consists of a parametric representation of the decision policy $\mu{}(\cdot \ ; \actorparams)$, where $\actorparams$ represent the policy parameters. The schematic of our system architecture is given in \Cref{fig:blk-diag}. 

    \subsection{Choice of an IPM} \label{sec:ipm-choice}

        To compute the IPM $d_\mathfrak{F}$ we need to know the probability density functions $\hat \transition$ and $\transition$. As we assume $\hat \transition$ to belongs to a parametric family, we know its density function in closed form. However, since we are in the learning setup, we can only access samples from $\transition $. For a function a $f\in \mathfrak{F}$, and probability density functions $\transition$ and $\hat \transition$ such that, $\nu_1 = \transition$, and $\nu_2 =\hat \transition$, we can estimate the IPM $d_\mathfrak{F}$ between a distribution and samples using the duality $|\int_\aisspace f d\nu_1 - \int_\aisspace f d\nu_2|$. In this paper, we use two from of IPMs, the MMD distance and the Wasserstein/Kantorovich–Rubinstein distance. 
        
        \subsubsection{MMD Distance:} Let $m_\aisparams$ denote the mean of the distribution $\hat \transition(\cdot;\aisparams)$. Then, the AIS-loss when MMD is used as an IPM is given by
        %Of the two alternatives, MMD distance is the easier compute as we can use some of its properties to simply its computation. Towards that end, when $\hat \transition$ is a real-valued distribution that can be characterised by its mean $m_\aisparams$, the MMD-based AIS loss can be given as:
        \begin{align}
           \aisloss(\aisparams) &= \frac{1}{\Timestep}\sum_{t = 0}^{\Timestep}\bigg( \lambda ({\hat{\cost}}(\Ais_{\timestep}, \Action_\timestep; \aisparams)  - \cost(\State_\timestep, \Action_\timestep))^{2} + (1-\lambda)(m^{\State_\timestep}_{\aisparams} - 2\State_\timestep)^{\top}m^{\State_\timestep}_{\aisparams}  \bigg),\label{eq:mmd-ais-loss}
        \end{align}
        where $m^{\State_\timestep}_{\aisparams}$ is obtained using the from the transition approximator, ~\ie, the mapping ${\hat\transition}(\aisparams): \aisspace \times \actionspace \to \real$. For the detailed derivation of the above loss see \Cref{sec:mmd-details}
        
        \subsubsection{Wasserstein/Kantorovich–Rubinstein distance:} 
        In principle, the Wasserstein/Kantorovich distance can be computed by solving a linear program~\citep{Sriperumbudur}, but doing at every episode can be computationally expensive. 
        %When $\ipm$ is the Wasserstein/Kantorovich–Rubinstein distance, we can compute it by solving a linear program~\citep{Sriperumbudur}. In some settings, it might be computationally expensive to solve a linear program at each time step before updating the parameters $\aisparams$. 
        Therefore, we propose to approximate the Wasserstein distance using a KL-divergence~\citep{kl} based upper-bound. The simplified-KL divergence based AIS loss is given as:
        \begin{align}
                \aisloss(\aisparams) &= \frac{1}{\Timestep} \sum_{t = 0}^{\Timestep}\bigg( \lambda ({\hat{\cost}}(\Ais_{\timestep}, \Action_\timestep; \aisparams)  - \cost(\State_\timestep, \Action_\timestep))^{2} + (1-\lambda)\log(\hat \transition(\State_\timestep;\aisparams))  \bigg),\label{eq:w-ais-loss}
        \end{align} 
        where after dropping the terms which do not depend on $\aisparams$, we get $d_{\mathfrak{F}^{\text{W}}}^{2}(\transition, \hat \transition)\leq \log(\hat \transition(\State_\timestep;\aisparams))$ is the simplified-KL-divergence based upper bound. For the details of this derivation see \Cref{sec:wass-details}. 



        % When $\ipm$ is the Wasserstein/Kantorovich–Rubinstein distance, we can compute it by solving a linear program~\citep{Sriperumbudur}. In some settings, it might be computationally expensive to solve a linear program at each time step before updating the parameters $\aisparams$. 
        % Therefore, we propose to use the a Sinkhorn divergence~\citep{Sinkhorn1967DiagonalET} based estimator to compute the Wasserstein distance. For a detailed discussion on this topic we refer the reader to the large body of work in the field of optimal transport~\citep{Cuturi13, Knight08, RamdasTC17, GenevayPC18}. As these technicalities are not relevant to the scope of this paper, we will define the Sinkhorn divergence based estimator as follows:
        % \begin{align}
        %     \nu_1 = \sum_{i=1}^{N} \nu_{1_{i}}\delta_{X_{i}} && \nu_2 = \sum_{i=1}^{N} \nu_{2_{i}}\delta_{W_{i}},
        % \end{align}
        % where $\nu_1, \nu_2 \in \real^{N}$ and $X, W \in \real^{N \times D}$
        % \begin{align}\label{eq:OT}
        %     S_{\epsilon}(\nu_1, X, \nu_2, W) &= \sum_{i=1}^{N} \nu_{1_{i}}(j_{i} - k_{i}) + \sum_{i=1}^{N} \nu_{2_{i}}(y_{i} - g_{i})
        % \end{align}
        % where
        % \begin{align}
        %     j_i &= - \epsilon \log \bigg(\sum_{i'=1}^{N} \exp(\log(\nu_{2_{i'}} + \frac{1}{\epsilon}y_{i'} - \frac{1}{\epsilon}C(X_i, W_{i'}))\bigg)\\
        %     y_i &= - \epsilon \log \bigg(\sum_{i'=1}^{N} \exp(\log(\nu_{1_{i'}} + \frac{1}{\epsilon}j_{i'} - \frac{1}{\epsilon}C(X_{i'}, W_i))\bigg)\\
        %     k_i &= - \epsilon \log \bigg(\sum_{i'=1}^{N} \exp(\log(\nu_{1_{i'}} + \frac{1}{\epsilon}k_{i'} - \frac{1}{\epsilon}C(X_i, X_{i'}) )\bigg)\\
        %     g_i &= - \epsilon \log \bigg(\sum_{i'=1}^{N} \exp(\log(\nu_{2_{i'}} + \frac{1}{\epsilon}g_{i'} - \frac{1}{\epsilon}C(W_i, W_{i'}) )\bigg)
        % \end{align}

   \subsection{Policy gradient algorithm}\label{sec:pgt}
    % \begin{minipage}{\linewidth}
        \begin{algorithm}
            \SetKwData{Left}{left}\SetKwData{This}{this}\SetKwData{Up}{up}
            \SetKwFunction{Union}{Union}\SetKwFunction{FindCompress}{FindCompress}
            \SetKwInOut{Input}{Input}\SetKwInOut{Output}{Output}
            \SetAlgoLined
            \Input{$\iota_{0}$: Initial state distribution, \\
                  $\aisparams_{0}$: Ais parameters, \\
                  $\actorparams_{0}$: Actor parameters, \\
                %   $\criticparams_{0}$: Critic parameters,
                %   $\ais_{0}$: Initial Ais, 
                  $\action_{0}$: Initial action, \\
                  $\mathcal{D} = \emptyset$: Replay buffer, \\
                  $N_{\text{comp}}$: Computation \rlap{budget,} \\
                  $N_{\text{ep}}$: Episode length, \\
                  $N_{\text{grad}}$: Gradient steps}
            % \KwResult{Write here the result }
             \For{iterations $i = 0:N_{\text{comp}}$ }{
                  Sample start state $\sts_{0}\sim \iota_{0}$\;
                  \For{iterations $j = 0:N_{\text{ep}}$ }
                  {
                  $\ais_{j} = \aisfunction_{\aisparams}(\sts_{1:j},\action_{1:j-1})$\;
                  % \tcp*{\tiny{final layer of a GRU cell.}}
                  $\action_{j} = \mu_{\actorparams}(\ais_{j})$\;
                  $\sts_{j+1} = \transition(\sts_{j},\action_{j})$\; %\tcp*{\tiny{sample next state from the MDP.}}
                  $\mathcal{D} \xleftarrow{} \{\ais_{j},\action_{j},\sts_{j},\sts_{j+1}\}$\;
                  $\action_{j-1} = \action_{j}$\;
                  $\sts_{j} = \sts_{j+1}$\;
                  }
                  \For{every batch $b \in \mathcal{D}$}
                  {
                        \For {gradient step $t=0:N_{\text{grad}}$}
                        {
                        %  $\aisparams_{\timestep+1,b,\valuefunction} = \aisparams_{\timestep,b,\valuefunction} + \aislr  \grad_{\aisparams_{\valuefunction}}\aisloss(\aisparams_{\timestep,b,\valuefunction})$\;
                        %  $\criticparams_{\timestep+1,b} = \criticparams_{\timestep,b} + \criticlr\grad_{\criticparams}\criticloss(\criticparams_{\timestep,b})$\;
                         $\aisparams_{\timestep+1,b} = \aisparams_{\timestep,b} + \aislr \grad_{\aisparams}\aisloss(\aisparams_{\timestep,b})$\;
                         $\actorparams_{\timestep+1,b} = \actorparams_{\timestep,b} +\actorlr \hat\grad_{\actorparams}\performance(\actorparams_{\timestep,b},\aisparams_{\timestep,b})$
                        }
                  }
             }
          \caption{Policy Search with \rlap{AIS}}\label{alg:ciac-a}
      \end{algorithm}
      

   
   
%   \begin{wrapfigure}{rt}{0.4\textwidth}
%         \vspace*{-10mm}
%   \begin{minipage}{\linewidth}
%         \begin{algorithm}[H]
%             \SetKwData{Left}{left}\SetKwData{This}{this}\SetKwData{Up}{up}
%             \SetKwFunction{Union}{Union}\SetKwFunction{FindCompress}{FindCompress}
%             \SetKwInOut{Input}{Input}\SetKwInOut{Output}{Output}
%             \SetAlgoLined
%             \Input{$\iota_{0}$: Initial state distribution, \\
%                   $\aisparams_{0}$: Ais parameters, \\
%                   $\actorparams_{0}$: Actor parameters, \\
%                 %   $\criticparams_{0}$: Critic parameters,
%                 %   $\ais_{0}$: Initial Ais, 
%                   $\action_{0}$: Initial action, \\
%                   $\mathcal{D} = \emptyset$: Replay buffer, \\
%                   $N_{\text{comp}}$: Computation \rlap{budget,} \\
%                   $N_{\text{ep}}$: Episode length, \\
%                   $N_{\text{grad}}$: Gradient steps}
%             % \KwResult{Write here the result }
%              \For{iterations $i = 0:N_{\text{comp}}$ }{
%                   Sample start state $\sts_{0}\sim \iota_{0}$\;
%                   \For{iterations $j = 0:N_{\text{ep}}$ }
%                   {
%                   $\ais_{j} = \aisfunction_{\aisparams}(\sts_{1:j},\action_{1:j-1})$\;
%                   % \tcp*{\tiny{final layer of a GRU cell.}}
%                   $\action_{j} = \mu_{\actorparams}(\ais_{j})$\;
%                   $\sts_{j+1} = \transition(\sts_{j},\action_{j})$\; %\tcp*{\tiny{sample next state from the MDP.}}
%                   $\mathcal{D} \xleftarrow{} \{\ais_{j},\action_{j},\sts_{j},\sts_{j+1}\}$\;
%                   $\action_{j-1} = \action_{j}$\;
%                   $\sts_{j} = \sts_{j+1}$\;
%                   }
%                   \For{every batch $b \in \mathcal{D}$}
%                   {
%                         \For {gradient step $t=0:N_{\text{grad}}$}
%                         {
%                         %  $\aisparams_{\timestep+1,b,\valuefunction} = \aisparams_{\timestep,b,\valuefunction} + \aislr  \grad_{\aisparams_{\valuefunction}}\aisloss(\aisparams_{\timestep,b,\valuefunction})$\;
%                         %  $\criticparams_{\timestep+1,b} = \criticparams_{\timestep,b} + \criticlr\grad_{\criticparams}\criticloss(\criticparams_{\timestep,b})$\;
%                          $\aisparams_{\timestep+1,b} = \aisparams_{\timestep,b} + \aislr \grad_{\aisparams}\aisloss(\aisparams_{\timestep,b})$\;
%                          $\actorparams_{\timestep+1,b} = \actorparams_{\timestep,b} +\actorlr \hat\grad_{\actorparams}\performance(\actorparams_{\timestep,b},\aisparams_{\timestep,b})$
%                         }
%                   }
%              }
%           \caption{Policy Search with \rlap{AIS}}\label{alg:ciac-a}
%       \end{algorithm}
%       \end{minipage}
%       \vspace*{-10mm}
%     \end{wrapfigure}

        Following the design of the AIS block, we now provide a policy-gradient algorithm to learning both the AIS and policy. The schematic of our agent architecture is given in \Cref{fig:blk-diag}, and pseudo-code is given in \Cref{alg:ciac-a}. Given a feature space $\aisspace$, we can simultaneously learn the AIS-generator and the policy using a multi-timescale stochastic gradient ascent algorithm~\citep{borkar2008stochastic}. Let $\mu(\cdot;\actorparams):\aisspace \to \Delta(\actionspace)$ be a parameterised stochastic policy with parameters $\actorparams$. Let $\performance(\actorparams,\aisparams)$ denote the performance of the policy $\mu(\cdot ;\ \actorparams)$. The policy gradient theorem~\citep{pgt,Williams2004SimpleSG,baxter-bartlett} states that: 
        % \begin{align*}
        %     \grad_{\actorparams}\performance(\actorparams_\timestep,\aisparams_\timestep) &= \expecun{}\bigg[\sum_{\timestep=1}^{\infty}\discount^{\timestep-1}\cost_{\timestep}\bigg(\sum_{\tau =1}^{\timestep}\grad_{\actorparams}\log(\mu(\Action_\timestep|\Ais_\timestep ;\ \actorparams_\timestep))\bigg)\bigg].\label{eq:pg}
        % \end{align*}
        For a rollout horizon $\Timestep$, we can estimate $\grad_\actorparams \performance$ as:
        \begin{align*}
            \hat \grad_{\actorparams}\performance(\actorparams_\timestep,\aisparams_\timestep) &= \sum_{\timestep=1}^{\Timestep}\discount^{\timestep-1}\cost_{\timestep}\bigg(\sum_{\tau =1}^{\timestep}\grad_{\actorparams}\log(\mu(\Action_\timestep|\Ais_\timestep ;\ \actorparams_\timestep))\bigg).
        \end{align*}
        Following a rollout of length $\Timestep$, we can then update the parameters $\{(\aisparams_i, \actorparams_i  )\}_{i \geq 1}$ as follows:
        
        \begin{subequations}\label{eq:pgt-update}
             \begin{align}
                \aisparams_{i+1} = \aisparams_i + \aislr_i \grad_\aisparams\aisloss(\aisparams_i), &&
                \actorparams_{i+1} = \actorparams_i + \actorlr_i \hat\grad_{\actorparams}\performance(\actorparams_{i},\aisparams_{i})\label{eq:actor-update},
            \end{align}
        \end{subequations}
            % \begin{align}
            %     \aisparams_{i+1} &= \aisparams_i + \aislr_i \grad_\aisparams\aisloss(\aisparams_i),
            % \label{eq:ais-update}
            % \\
            %     \actorparams_{i+1} &= \actorparams_i + \actorlr_i \hat\grad_{\actorparams}\performance(\actorparams_{i},\aisparams_{i})
            % \label{eq:actor-update}
            % \end{align}
         where the step-size $\{\aislr_{i}\}_{i \geq 0}$ and $\{\actorlr_{i}\}_{i \geq 0}$ satisfy the standard conditions $\sum_{i} \aislr_{i} = \infty$, $\sum_{i}\aislr_{i}^{2}< \infty$, $\sum_{i} \actorlr_{i} = \infty$ and $\sum_{i}\actorlr_{i}^{2}< \infty$ respectively. Moreover, one can ensure that the AIS generator converges faster by choosing an appropriate learning rates such that, $\lim_{i \to \infty} \frac{\actorlr_{i}}{\aislr_{i}} = 0$. 
         
    \subsection{Actor Critic Algorithm} \label{sec:AC}
        We can also use the aforementioned ideas to design an AIS based actor-critic algorithm. In addition to a parameterised policy $\policy(\cdot; \actorparams)$ and AIS generator $(\aisfunction_\timestep(\cdot;\aisparams), \hat f, \hat r, \hat\transition)$ the actor-critic algorithm uses a parameterised critic $\hat\valuefunction(\cdot;\criticparams):\featurespace \to \real$, where $\criticparams$ are the parameters for the critic. The performance of policy $\mu(\cdot;\actorparams)$ is then given by $\performance(\actorparams, \aisparams, \criticparams)$. According to policy gradient theorem~\citep{pgt,baxter-bartlett} the gradient of $\performance(\actorparams, \aisparams, \criticparams)$, is given as:
        \begin{align}
            \grad_\actorparams \performance(\actorparams, \aisparams, \criticparams) &= \expecun{}\bigg[ \grad_{\actorparams}\log(\mu(\cdot;\actorparams))\hat{\valuefunction}(\cdot;\criticparams)\bigg].
        \end{align}
        And for a trajectory of length $\Timestep$, we approximate it as:
        \begin{align}
            \hat{\grad}_\actorparams \performance(\actorparams, \aisparams, \criticparams) &= \frac{1}{\Timestep}\sum_{\timestep =1}^{\Timestep}\bigg[ \grad_{\actorparams}\log(\mu(\cdot;\actorparams))\hat{\valuefunction}(\cdot;\criticparams)\bigg].
        \end{align}
        The parameters $\criticparams$ can be learnt by optimising the temporal difference loss given as:
        \begin{align}
            \loss_{\text{TD}}(\actorparams, \aisparams, \criticparams) &= \frac{1}{\Timestep}\sum_{\timestep=0}^{\Timestep}\texttt{smoothL1}(\hat{\valuefunction}(\Feature_\timestep;\criticparams) - \cost(\Feature_\timestep,\Action_\timestep) - \discount \hat{\valuefunction}(\Feature_{\timestep+1};\criticparams)).
        \end{align}
         The parameters $\{(\aisparams_i, \actorparams_i, \criticparams_i  )\}_{i \geq 1}$ can then be updated using a multi-timescale stochastic approximation algorithm as follows:
         \begin{subequations}\label{eq:ac-update}
            \begin{align}
                \aisparams_{i+1} &= \aisparams_i + \aislr_i \grad_\aisparams\aisloss(\aisparams_i)\label{eq:ac-ais-update}\\
                \criticparams_{i+1} &= \criticparams_i + \criticlr_i \grad_{\criticparams}\loss_{\text{TD}}(\actorparams_i, \aisparams_i, \criticparams_i)\label{eq:ac-critic-update}\\
                \actorparams_{i+1} &= \actorparams_i + \actorlr_i \hat{\grad}_{\actorparams}\performance(\actorparams_{i},\aisparams_{i},\criticparams)\label{eq:ac-actor-update},
            \end{align}
         \end{subequations}
          where the step-size $\{\aislr_{i}\}_{i \geq 0}$,  $\{\criticlr_{i}\}_{i \geq 0}$ and $\{\actorlr_{i}\}_{i \geq 0}$ satisfy the standard conditions $\sum_{i} \aislr_{i} = \infty$, $\sum_{i}\aislr_{i}^{2} < \infty$, $\sum_{i} \criticlr_{i} = \infty$, $\sum_{i}\criticlr_{i}^{2} < \infty$, $\sum_{i} \actorlr_{i} = \infty$ and $\sum_{i}\actorlr_{i}^{2}< \infty$ respectively. Moreover, one can ensure that the AIS generator converges first, followed by the critic and the actor by choosing an appropriate step-sizes such that, $\lim_{i \to \infty} \frac{\actorlr_{i}}{\aislr_{i}} = 0$ and $\lim_{i \to \infty} \frac{\criticlr_{i}}{\actorlr_{i}} = 0$.
         
         
        %  Note that we can use similar ideas to develop an Actor-Critic algorithm where, in addition to a parameterised policy $\policy(\cdot; \actorparams)$ and AIS generator $(\aisfunction_\timestep(\cdot;\aisparams), \hat f, \hat r, \hat\transition)$ we can also a parameterised critic $\hat\valuefunction(\cdot;\criticparams):\featurespace \to \real$, where $\criticparams$ are the parameters for the critic. The details of the Actor Critic Algorithm can be found in \Cref{sec:AC}, and 
        % The convergence analysis of both algorithms can be found in \Cref{sec:convergence}.
        
         \subsection{Convergence analysis}\label{sec:convergence}
    
    In this section we will discuss the convergence of the AIS-based policy gradient in \Cref{sec:pgt} as well as Actor-Critic algorithm presented in the previous subsection. The proof of convergence relies on multi-timescale stochastic approximation \citet{borkar2008stochastic} under conditions similar to the standard conditions for convergence of policy gradient algorithms with function approximation stated below, therefore it would suffice to provide a proof sketch.

    \noindent\begin{assumption} \label{assumption-1}
        \begin{enumerate}
            \item \label{a.1.1}The values of step-size parameters $\aislr, \actorlr$ and $\criticlr$ (for the actor critic algorithm) are set such that the timescales of the updates for $\aisparams$, $\actorparams$, and $\criticparams$ (for Actor-Critic algorithm) are separated, ~\ie, $\aislr_{\timestep} \gg \actorlr_{\timestep}$, and for the Actor-Critic algorithm $\aislr_{\timestep} \gg \criticlr_\timestep \gg \actorlr_{\timestep}$, $\sum_{i} \aislr_{i} = \infty$, $\sum_{i}\aislr_{i}^{2} < \infty$, $\sum_{i} \criticlr_{i} = \infty$, $\sum_{i}\criticlr_{i}^{2} < \infty$, $\sum_{i} \actorlr_{i} = \infty$ and $\sum_{i}\actorlr_{i}^{2}< \infty$, $\lim_{i \to \infty} \frac{\actorlr_{i}}{\aislr_{i}} = 0$ and $\lim_{i \to \infty} \frac{\criticlr_{i}}{\actorlr_{i}} = 0$, 
            \item \label{a.1.2}The parameters $\aisparams$, $\actorparams$ and $\criticparams$ (for Actor-Critic algorithm) lie in a convex, compact and closed subset of Euclidean spaces.
            \item \label{a.1.3}The gradient $\grad_{\aisparams}\aisloss$ is Lipschitz in $\aisparams_{\timestep}$, and $\hat \grad_{\actorparams}\performance(\actorparams,\aisparams)$ is Lipschitz in $\actorparams_{\timestep},~\text{and}~\aisparams_{\timestep}$. Whereas for the Actor-Critic algorithm the gradient of the TD loss $ \grad_{\criticparams}\loss_{\text{TD}}(\aisparams, \actorparams, \criticparams)$ and the policy gradient $\hat \grad_{\actorparams} \performance(\aisparams, \actorparams, \criticparams)$ is Lipschitz in $(\aisparams_\timestep, \actorparams_\timestep, \criticparams_\timestep)$.
            \item \label{a.1.4}Estimates of gradients $\grad_{\aisparams}\aisloss$, $\grad_{\actorparams}\performance(\actorparams,\aisparams)$, and $ \grad_{\criticparams}\loss_{\text{TD}}(\aisparams, \actorparams, \criticparams)$ and are unbiased with bounded variance\footnote{This assumption is only satisfied in tabular MDPs.}.
            % Moreover, in the case of the Actor-Critic algorithm, the Critic and the function approximator are compatible as given in \citet{...} \ie,
            % \begin{align*}
            %     \frac{\partial\hat Q_{\criticparams_\timestep}(\Ais_\timestep, \Action_\timestep)}{\partial \criticparams} =\frac{1}{\policy_{\actorparams_{\timestep}}}\frac{\partial \policy_{\actorparams_\timestep}}{\partial \actorparams}.
            % \end{align*}
        \end{enumerate}
    \end{assumption}

        \begin{assumption} \label{assumption-2}
            \begin{enumerate}
                \item \label{a.2.1}The ordinary differential equation (ODE) corresponding to \eqref{eq:actor-update} is locally asymptotically stable.
                \item \label{a.2.2}The ODEs corresponding to \eqref{eq:pgt-update} is globally asymptotically stable.
                \item For the Actor-Critic algorithm, the ODE corresponding to \eqref{eq:ac-critic-update} is globally asymptotically stable and has a fixed point which is Lipschitz in $\actorparams$.
            \end{enumerate}
        \end{assumption}
        \begin{theorem}\label{thm:convergence}
        Under \cref{assumption-1,assumption-2}, along any sample path, almost surely we have the following:
        \begin{enumerate}
            \item The iteration for $\aisparams$ in \eqref{eq:pgt-update} converges to an AIS generator that minimises the $\aisloss$.
            \item The iteration for $\actorparams$ in \eqref{eq:actor-update} converges to a local maximum of the performance $\performance(\aisparams^\star,\actorparams)$ where $\aisparams^\star$, and $\criticparams^\star$ (for Actor Critic) are the converged value of $\aisparams$, $\criticparams$.
            \item For the Actor-Critic algorithm the iteration for $\criticparams$ in \eqref{eq:ac-critic-update} converges to critic that minimises the error with respect to the true value function.
        \end{enumerate}
        \end{theorem}
        \begin{proof}
        % On satisfying \cref{assumption-1}.1, a suitable continuous time interpolation of \eqref{def:ais-update} will be an asymptotic pseudo-trajectory of the semi-flow induced by it's ordinary differential equation (ODE). Therefore, this interpolation will converge to the limit point of \eqref{def:ais-update}'s ODE, and by principle of superposition the ODE will be globally asymptotically stable. As such, by the arguments made by~\citet{borkar2008stochastic,Kushner1997StochasticAA}, iteration given by \eqref{def:ais-update} will converge to its corresponding fixed point. By similar a argument, continuous time interpolations of \eqref{def:actor-update-2} will also converge to the limit points of its respective ODEs.
        The proof for this theorem follows the technique used in \citep{Leslie2004ReinforcementLI,borkar2008stochastic}. Due to the specific choice of learning rate the AIS-generator is updated at a faster time-scale than the actor, therefore it is ``quasi static'' with respect to to the actor while the actor observes a ``nearly equilibriated'' AIS generator. Similarly in the case of the Actor-Critic algorithm the AIS generator observes a stationary critic and actor, whereas the critic and actor see ``nearly equilibriated'' AIS generator. The Martingale difference condition (A3) of \citet{borkar2008stochastic} is satisfied due to \cref{a.1.4} in \cref{assumption-1}. As such since our algorithm satisfies all the four conditions by \citep[page35]{Leslie2004ReinforcementLI}, \citep[Theorem 23]{Borkar1997StochasticAW}, the result then follows by combining the theorem on \citep[page 35]{Leslie2004ReinforcementLI}\citep[Theorem 23]{borkar2008stochastic} and \citep[Theorem 2.2]{Borkar1997StochasticAW}.
        \end{proof}



    \begin{figure*}[!htbp]
      \includegraphics[width=\linewidth]{Results/combined-1.pdf}
      % \caption{This is a figure} \label{fig:comb-results-1}
      \includegraphics[width=\linewidth]{Results/combined-2.pdf}
      \caption{Empirical results averaged over 50 Monte Carlo runs with shaded regions showing the interquantile range.} \label{fig:comb-results-2}
    \end{figure*}
    
  

    
   
\section{Empirical evaluation}\label{sec:experiments}


    % Through our experiments, we seek to answer the following questions:
    % (1) Can history-based feature representations policies help improve the quality of solution found by a memory-less RL algorithms?
    % (2) In terms of the solution quality how does the proposed method compare with other methods which use memory augmented policies as well as reward and transition predictors?
    % (3) How does the choice of IPM affect the algorithms performance?

    % We answer question (1) by comparing our approach with the proximal policy gradient (PPO)~\citep{ppo} and the policy-gradient version of DeepMDP framework~\citep{deepmdp}. For question (2) we compare our approach with modified versions of PlaNet~\citep{Pla-Net}, Dreamer~\citep{Dreamer}, and VariBAD~\citep{VariBad}.
    
    % For question (3) we compare the performance of our method using different MMD kernels and KL-divergence based approximation of Wasserstein distance. All the approaches are evaluated on six continuous control tasks from the MuJoCo~\citep{Todorov2012MuJoCoAP} OpenAI-Gym suite. To ensure a fair comparison, the baselines and their respective hyper-parameter settings are taken from well tested stand-alone implementations provided by~\citet{baselines}. From an implementation perspective, our framework can be used to modify any off-the-shelf policy-gradient algorithm by simply replacing (or augmenting) the feature abstraction layers of the policy and/or value networks with recurrent neural networks (RNNs), trained with the appropriate losses, as outlined previously. In these experiments, we replace the fully connected layers in PPO's architecture with a Gated Recurrent Unit (GRU). For all the implementations, we initialise the hidden state of the GRU to zero at the beginning of the trajectory. This strategy simplifies the implementation and also allows for independent decorrelated sampling of sequences, therefore ensuring robust optimisation of the networks~\citep{rnn-hausknecht}. It is important to note that we can extend our framework to other policy gradient methods such as SAC~\citep{HaarnojaZAL18}, TD3~\citep{td3} or DDPG~\citep{ddpg}, after satisfying certain technical conditions. However, we leave these extensions for future work. Additional experimental details and results can be found in \Cref{sec:experiment-details}.
    
    % From \Cref{fig:comb-results-2} one can observe that AIS results in performance improvements in high-dimensional environments like Ant, Humanoid and Walker. Other memory-augmented methods like \citep{Pla-Net,VariBad,Dreamer} also performance better as compared to memory-less methods. Overall these results lend credence to our observation of using history-based policies. In some environments we observe that methods which use planning subroutines \citep{Pla-Net,Dreamer} accelerate learning in the initial stages but then converge to a sub-optimal policy. This could be because of the inaccuracies in the approximate dynamics model used in the planning loop. An interesting future direction would be use planning loop similar to \citep{Pla-Net,Dreamer} in \Cref{alg:ciac-a}.
    
    
    Through our experiments, we seek to answer the following questions:
    (1) Can history-based feature representation policies help improve the quality solution found by a memory-less RL algorithm?
    (2) In regards to the solution quality and sample complexity, how does the proposed method compare with other memory-augmented policies?
    (3) How does the choice of IPM affect the algorithms performance?
    
    
     We answer question (1) and (2) by comparing our approach with the proximal policy gradient (PPO) algorithm which uses feed-forward neural networks. For question (2), we compare our method with an LSTM-based PPO variant which learns the feature representation using the history of states $\State_{1:\Timestep}$ in a trajectory. For question (3) we compare the performance of our method using different MMD kernels and KL-divergence based approximation of Wasserstein distance. All the approaches are evaluated on six continuous control tasks from the MuJoCo~\citep{Todorov2012MuJoCoAP} OpenAI-Gym suite. To ensure a fair comparison, the baselines and their respective hyper-parameter settings are taken from well tested stand-alone implementations provided by~\citet{baselines}. From an implementation perspective, our framework can be used to modify any off-the-shelf policy-gradient algorithm by simply replacing (or augmenting) the feature abstraction layers of the policy and/or value networks with recurrent neural networks (RNNs), trained with the appropriate losses, as outlined previously. In these experiments, we replace the fully connected layers in PPO's architecture with a Gated Recurrent Unit (GRU). For all the implementations, we initialise the hidden state of the GRU to zero at the beginning of the trajectory. This strategy simplifies the implementation and also allows for independent decorrelated sampling of sequences, therefore ensuring robust optimisation of the networks~\citep{rnn-hausknecht}. It is important to note that we can extend our framework to other policy gradient methods such as SAC~\citep{HaarnojaZAL18}, TD3~\citep{td3} or DDPG~\citep{ddpg}, after satisfying certain technical conditions. However, we leave these extensions for future work. Additional experimental details and results can be found in \Cref{sec:experiment-details}.
    
    
%\subsection{Numerical Results} 
    
     \Cref{fig:comb-results-2} contains the results of our experiments averaged over 50 Monte-Carlo evaluations using MMD-based AIS loss in \eqref{eq:mmd-ais-loss}.  These results show that our algorithm improves over the performance of both the baselines, and the performance gain is significantly higher for high-dimensional environments like Humanoid and Ant. 
     It is worth noticing that the GRU baseline also outperforms the feed-forward baseline for most  environments. Overall, these findings lend credence to  history-based encoding policies as a way to improve the quality of the solution learnt by the RL algorithm.
    
    
    
    
    
     \begin{figure*}[!htbp]
        \includegraphics[width=\linewidth]{Results/MMD.pdf}
        \caption{Comparison of different MMDs, averaged over 50 runs} \label{fig:MMD-comp}
    \end{figure*}

    
    Note that the MMD distance given by \eqref{eq:mmd-grad} in \Cref{sec:mmd-details}, can be computed using different types of characteristic kernels (for a detailed review see ~\citep{Sriperumbudur,NIPS2009_685ac8ca,sejdinovic}). In this paper we consider computing \eqref{eq:mmd-grad} using the Laplace, Gaussian and energy distance kernels. In in \Cref{fig:MMD-comp} we compre the perfromance of our methods under different MMD kernels. It can be observed that for the continuous control tasks in the MuJoCo suite, the energy distance yields better performance, and therefore we implement \cref{eq:mmd-grad} using the energy distance for the results in \Cref{fig:comb-results-2}.
    
    Next, we compare the performance of our method under MMD (Energy distance kernel) and Wasserstein distance. From \Cref{fig:Wass-res} we observe that for continuous control tasks, use of MMDs result in better performance as compared to Wasserstein distance. 
    
        \begin{figure*}[!htbp]
            \includegraphics[width=\linewidth]{Results/wass.pdf}
            \caption{Comparison of Wasserstein vs MMDs, averaged over 50 runs.} \label{fig:Wass-res}
        \end{figure*}
    
    % \begin{figure}[!htbp]
    %         \includegraphics[width=\linewidth]{Results/dist-new-1.pdf}
    %         \caption{MMD vs KL for Half Cheetah} \label{fig:dist-1}
    % \end{figure}
    
    % \begin{figure}[!htbp]
    %         \includegraphics[width=\linewidth]{Results/dist-new-2.pdf}
    %         \caption{MMD vs KL for Ant } \label{fig:dist-2}
    % \end{figure}
    
    % \begin{figure}[!htbp]
    %         \includegraphics[width=\linewidth]{Results/dist-new-3.pdf}
    %         \caption{MMD vs KL for Walker} \label{fig:dist-3}
    % \end{figure}
    
    

% ODQA gives QA model a single question without any context and asks the model to infer out-of-context knowledge. 

% Following the pioneering work by~\citet{DBLP:conf/acl/ChenFWB17}, most ODQA systems assume the model can access an external text corpus (e.g. Wikipedia).
% Due to the large scale of web corpus (20GB for Wikipedia), it could not be simply encoded in the QA model parameters, and thus most works propose a \textit{Retrieval-Reader} pipeline, by firstly index the whole corpus and use a \textit{retriever} model to identify which passage is relevant to the question; then the retrieved text passage concatenate with question is re-encoded by a seperate \textit{reader} model (e.g., \texttt{LM}) to predict answer. As the knowledge is outside of model parameter, \citet{DBLP:conf/emnlp/RobertsRS20} defines these methods as \textit{Open-book}, with an analogy to referring textbooks during exam.


% The \textit{Open-book} models following \textit{Retrieval-Reader} pipeline requires storing indexed corpus, and is hard to train end-to-end, and is inefficient during inference. Therefore, 
% In contrast, 

% \textit{Closed-book} QA models (mostly a single \texttt{LM}) try to answer open questions without accessing external knowledge. This setting is much harder as it requires \texttt{LM} to memorize all pertinent knowledge in its parameters.
% , and even recent \texttt{LM}s with much larger model parameters is still not competitive to state-of-the-art \textit{Open-book} models (e.g., T5-11B~\cite{DBLP:conf/emnlp/RobertsRS20} achieves 34.5 accuracy on Natural Questions, while FiD~\citep{DBLP:conf/eacl/IzacardG21} with 1B parameter achieves 51.4).
% Recent studies~\cite{DBLP:conf/emnlp/RobertsRS20, DBLP:conf/nips/BrownMRSKDNSSAA20} show that by leveraging pre-trained \texttt{LM}s in a supervised setting, they could correctly answer a certain portion of open questions (e.g., T5-11B could answer over 30\% of questions in Natural Questions dataset). 







\noindent \textbf{Open-Domain Question Answering (ODQA)} gives QA model a single question without any context and asks the model to infer out-of-context knowledge. Following the pioneering work by~\citet{DBLP:conf/acl/ChenFWB17}, most ODQA systems assume the model can access an external text corpus (e.g. Wikipedia).
Due to the large scale of web corpus (20GB for Wikipedia), it could not be simply encoded in the QA model parameters, and thus most works propose a \textit{Retrieval-Reader} pipeline, by firstly index the whole corpus and use a \textit{retriever} model to identify which passage is relevant to the question; then the retrieved text passage concatenate with question is re-encoded by a seperate \textit{reader} model (e.g., \texttt{LM}) to predict answer. As the knowledge is outside of model parameter, \citet{DBLP:conf/emnlp/RobertsRS20} defines these methods as \textit{Open-book}, with an analogy to referring textbooks during exam. \textit{Closed-book} QA models (mostly a single \texttt{LM}) try to answer open questions without accessing external knowledge. This setting is much harder as it requires \texttt{LM} to memorize all pertinent knowledge in its parameters, and even recent \texttt{LM}s with much larger model parameters is still not competitive to state-of-the-art \textit{Open-book} models. 



% \paragraph{Knowledge-Base Question Answering}
% Traditional parsing-based methods parse the question into some intermediate query (e.g., SQL language, query graphs), which can execute on a knowledge base to get answer \citep{DBLP:conf/emnlp/BerantCFL13,DBLP:conf/acl/YihCHG15,DBLP:journals/tacl/ReddyTCKDSL16,DBLP:journals/corr/abs-1709-00103,DBLP:conf/acl/LiangBLFL17}. However, existing knowledge bases suffer from low coverage of entities and relations required for open-ended questions. As an alternative, several works try to incorporate the structured knowledge into neural QA models for differentiable reasoning. \cite{DBLP:conf/emnlp/LinCCR19} and \cite{DBLP:conf/emnlp/FengCLWYR20} parse the question into a sub-graph of knowledge base, and apply graph neural networks as reasoner to extract answers. \cite{DBLP:conf/iclr/ChenLYZSL20} integrates general symbolic operations as basic units, and parse questions into compositional programs to answer general questions.




\noindent\textbf{Knowledge-augmented Language Models} 
explicitly incorporate external knowledge (e.g. knowledge graph) into \texttt{LM}~\citep{DBLP:journals/corr/abs-2010-04389}.
Overall, these approaches can be grouped into two categories:
The first one is to explicitly inject knowledge representation into language model pre-training, where the representations are pre-computed from external sources~\citep{DBLP:conf/acl/ZhangHLJSL19,DBLP:conf/aaai/LiuW0PY21,DBLP:conf/emnlp/HuSC21}.
For example, ERNIE~\cite{DBLP:conf/acl/ZhangHLJSL19} encodes the pre-trained TransE~\cite{DBLP:conf/nips/BordesUGWY13} embeddings as input.
The second one is to implicitly model knowledge information into language model by performing knowledge-related tasks, such as entity category prediction~\citep{DBLP:journals/corr/abs-2010-00796} and graph-text alignment~\cite{DBLP:conf/acl/KeJRCWSZH21}.
For example, JAKET~\citep{DBLP:journals/corr/abs-2010-00796} jointly pre-trained both the KG representation and language representation by adding entity category and relation type prediction self-supervised tasks.

There also exists several QA works using $\KG$ to help ODQA. For example, \citet{DBLP:conf/iclr/AsaiHHSX20} and \citet{DBLP:journals/corr/abs-1911-03868} expand the entity graph following wikipedia hyperlinks or triplets in knowledge base. \citet{DBLP:conf/acl/DingZCYT19} extract entities from current context via entity-linking and turn them into a cognitive graph, and a graph neural network is applied on top of it to extract answer. \citet{DBLP:conf/iclr/DhingraZBNSC20} and \citet{DBLP:journals/corr/abs-2010-14439} construct an entity-mention bipartite graph and then model the QA reasoning as graph traversal by filtering only the contexts that are relevant to the question. \citet{DBLP:conf/emnlp/LinCCR19}, \citet{DBLP:conf/emnlp/FengCLWYR20} and \citet{DBLP:conf/naacl/YasunagaRBLL21} parse the question into a sub-graph of knowledge base, and apply graph neural networks as reasoner for extracting one of the entities as the answer.

To encode knowledge (significantly smaller than the web corpus) as \emph{memory} into \texttt{LM} parameter, a line of works try compressed knowledge including QA pairs~\citep{DBLP:journals/corr/abs-2204-04581, DBLP:journals/corr/abs-2102-07033,DBLP:journals/corr/abs-2209-10063}, entity embedding~\citep{DBLP:journals/corr/abs-2004-07202} and reasoning cases~\citep{DBLP:conf/emnlp/DasZTGPLTPM21, DBLP:journals/corr/abs-2202-10610}.
There's also several works utilizing Knowledge Graph ($\KG$) to augment \texttt{LM}. FILM~\citep{DBLP:conf/naacl/VergaSSC21} turns $\KG$ triplets into memory. Given a question, \texttt{LM} retrieves most relevant triplet as answer. GreaseLM~\citep{DBLP:journals/corr/abs-2201-08860} propose to interact \texttt{LM} with $\KG$ via a interaction node. 


% We discuss other related works in Sec.~\ref{sec:related} in Appendix.















% Similar to query answering on graph, answering open-domain questions also require to infer out-of-context knowledge. 
% For example, given only a question $q$ as ``The Bauhaus represented Germany's recovery from which event$?$'', the QA model is asked to predict answer $a$ ''World War I``. To correctly answer such a question, the model needs to gain knowledge about all in-context entity mentions $M=\{m_i\}$, e.g., ''Bauhaus`` and ''Germany``. 
% Such an open-domain question could be abstracted as a query $(M, q, ?)$, or a probabilistic manner $P(a | q, M)$. 



\section{Conclusion}
In this paper, we extend the idea of SynGEC \cite{zhang2022syngec} and propose the CSynGEC approach to enhance GEC models by exploiting tailored constituent-based syntax. Experimental results show that incorporating constituent-based syntax produced by a GEC-oriented constituency parser can effectively help GEC models. 
Furthermore, we attempt to combine dependency-based and constituent-based syntax from both intra-model and inter-model aspects, and find that simultaneously using two kinds of syntax leads to more obvious improvement.


% \bibliographystyle{splncs03}
\bibliography{references}

\appendix
\pagenumbering{arabic}
\section*{Appendix}
\appendix

\section{Supplemental Tables}

%\section{Hyperparameters of Other Bandit Algorithms}
%\label{sec:bandit_hyperparams}
%Table~\ref{tab:hyperparams} lists the hyperparameters for bandit algorithms other than dBE.

\newcommand\topmidheader[2]{\multicolumn{#1}{c}{\textbf{#2}}\\%
                \addlinespace[1ex]}

\newcommand{\midheader}[2]{%
        \midrule\topmidheader{#1}{#2}}

\newcommand{\specialcell}[3][c]{% 
        \begin{tabular}[#1]{@{}#2@{}}#3\end{tabular}}%

\aptLtoX[graphic=no,type=env]{\begin{table}[htb]
  \centering
  \caption{Hyperparameters of bandit algorithms}
  \label{tab:hyperparams}
  \begin{tabular}{llc}
    \toprule
    Sign & Description & Value \\
    \multicolumn{3}{c}{\textbf{UCB1}}\\
    $c$ & Parameter to control the confidence level used in $\sqrt{c \cdot {\log{t}}/{N_t(arm)}}$ & 0.5  \\
    \multicolumn{3}{c}{\textbf{Thompson Sampling}}\\
    $p(\theta)$ & Prior Distribution & $\mathcal{B}(1, 1)$ \\
    \multicolumn{3}{c}{\textbf{discounted Thompson Sampling}}\\
    $\gamma$ & Discount factor & $1-10^{-8}$ \\
    \multicolumn{3}{c}{\textbf{discounted Thompson Samplingadaptive shrinking Thompson Sampling}}\\
    $M$ & Parameter to control memory usage in a data structure ADWIN2 \cite{ADWIN} & 10 \\
    $\delta$ & Parameter to control the confidence level in a data structure ADWIN2 & $1-10^{-7}$ \\
    \multicolumn{3}{c}{\textbf{EXP-IX}}\\
    $\eta_t$ & Parameter used for weights of arms & $\sqrt{\frac{2 \cdot \log{K}}{K \cdot t}}$ \\
    \addlinespace[1ex]
    $\gamma_t$ & Parameter used for loss estimates & $\frac{\eta_t}{2}$ \\
    \multicolumn{3}{c}{\textbf{EXP3++}}\\
    $\alpha$ & Constant used in calculating $\xi_t(a)$ & $3$ \\
    $\beta$ & Constant used in calculating $\xi_t(a)$ & $256$ \\
    \bottomrule
  \end{tabular}
\end{table}}{\begin{table}[htb]
  \centering
  \caption{Hyperparameters of bandit algorithms}
  \label{tab:hyperparams}
  \begin{tabular}{llc}
    \toprule
    Sign & Description & Value \\
    \midheader{3}{UCB1}
    $c$ & \specialcell{l}{Parameter to control the confidence \\ level used in $\sqrt{c \cdot {\log{t}}/{N_t(arm)}}$} & 0.5  \\
    \midheader{3}{Thompson Sampling}
    $p(\theta)$ & Prior Distribution & $\mathcal{B}(1, 1)$ \\
    \midheader{3}{discounted Thompson Sampling}
    $\gamma$ & Discount factor & $1-10^{-8}$ \\
    \midheader{3}{adaptive shrinking Thompson Sampling}
    $M$ & \specialcell{l}{Parameter to control memory usage \\ in a data structure ADWIN2 \cite{ADWIN}} & 10 \\
    $\delta$ & \specialcell{l}{ Parameter to control the confidence \\ level in a data structure ADWIN2} & $1-10^{-7}$ \\
    \midheader{3}{EXP-IX}
    $\eta_t$ & Parameter used for weights of arms & $\sqrt{\frac{2 \cdot \log{K}}{K \cdot t}}$ \\
    \addlinespace[1ex]
    $\gamma_t$ & Parameter used for loss estimates & $\frac{\eta_t}{2}$ \\
    \midheader{3}{EXP3++}
    $\alpha$ & Constant used in calculating $\xi_t(a)$ & $3$ \\
    $\beta$ & Constant used in calculating $\xi_t(a)$ & $256$ \\
    \bottomrule
  \end{tabular}
\end{table}}

\begin{table}[htb]
  \centering
  \caption{Commit IDs of the PUTs used in our vulnerability discovery and AFL++ used as the baseline.}
  \begin{tabular}{lc}
    \toprule
    Program & Commit \\
    \midrule

    AFL++ & 32a0d6ac315 (ver ++3.14c) \\
    Bloaty &  60209eb \\
    HarfBuzz & 77eeec5 \\
    libarchive & 86c9361 \\
       libxml2 & dea91c9 \\
    MuPDF & ef3d68d \\
   PHP & fdf0455f \\
    Poppler & 6d72d82 \\
    PROJ & 76dfefe \\
    QPDF &  3794f8e \\
    libtpm2 & bc3bb26 \\
    Wireshark  & 1fc621e \\
    Xpdf & N/A (ver 4.03) \\

    \bottomrule
  \end{tabular}
\label{tab:commit-ids}
\end{table}


\begin{table}[htb]
  \centering
  \caption{Initial and theoretical maximum values of code coverage of the PUTs in FuzzBench. 
           Initial values were investigated only in the PUTs used.}
  \begin{tabular}{lcc}
    \toprule
    PUT & Initial & Maximum \\
    \midrule

bloaty\_fuzz\_target & N/A & 83114 \\
curl\_curl\_fuzzer\_http & N/A & 78362 \\
freetype2-2017 & 1517 & 26262 \\
harfbuzz-1.3.2 & N/A & 12212 \\
jsoncpp\_jsoncpp\_fuzzer & N/A & 2114 \\
lcms-2017-03-21 & 149 & 7036 \\
libjpeg-turbo-07-2017 & N/A & 9384 \\
libpcap\_fuzz\_both & 2 & 7294 \\
libpng-1.2.56 & 138 & 3736 \\
libxml2-v2.9.2 & 258 & 67994 \\
libxslt\_xpath & N/A & 51456 \\
mbedtls\_fuzz\_dtlsclient & N/A & 12888 \\
openssl\_x509 & 6026 & 54116 \\
openthread-2019-12-23 & N/A & 19846 \\
php\_php-fuzz-parser & N/A & 215210 \\
proj4-2017-08-14 & 46 & 6534 \\
re2-2014-12-09 & 1 & 3982 \\
sqlite3\_ossfuzz & 4767 & 28766 \\
systemd\_fuzz-link-parser & N/A & 1798 \\
vorbis-2017-12-11 & 410 & 4082 \\
woff2-2016-05-06 & N/A & 5708 \\
zlib\_zlib\_uncompress\_fuzzer & N/A & 910 \\

    \bottomrule
  \end{tabular}
\label{tab:fuzzbench_max_cov}
\end{table}

\begin{table}[htb]
\centering
\caption{List of unique bugs found in the 7-day trial (manually triaged).}
\begin{minipage}{\columnwidth}

\centering
\begin{tabular}{lll}
\toprule

ID & PUT & Bug Type \\
\midrule
Bug-A & bloaty & NULL Pointer Deref \\
Bug-B & harfbuzz & Out-of-bounds Read \\
Bug-C & mupdf & Assertion Fail \\
Bug-D & mupdf & NULL pointer deref \\
Bug-E & xpdf & Stack Overflow \\
Bug-F & xpdf & NULL Pointer Deref \\
Bug-G \footnote{CVE-2022-24106 is issued.} & xpdf & Use of Uninitialized Value \\
Bug-H \footnote{CVE-2022-24107 is issued.} & xpdf & Integer Overflow \\
Bug-I & php & Use-After-Free \\
Bug-J & php & Use-After-Free \\
Bug-K & php & NULL Pointer Deref \\
Bug-L & php & Use-After-Free \\ 
Bug-M & php & NULL Pointer Deref \\
Bug-N & php & Assertion Fail \\
Bug-O & php & Use-After-Free \\
Bug-P & php & Use-After-Free \\
Bug-Q \footnote{CVE-2022-23308 is issued.} & libxml2 & Use-After-Free \\
\bottomrule
\end{tabular}

\label{tab:7d-bug}
\end{minipage}
\end{table}

\begin{table*}[htb]
  \centering
  \caption{List of the PUTs used in Section~\ref{sec:banditcomparison}. If the source code of a PUT was maintained in Git, the latest version at the time of the experiment in the master (or main) branch was used for the build. The `+' sign in a version indicates that the used source code is not the official release version of the source code.}
  \renewcommand\tabularxcolumn[1]{m{#1}}
  \renewcommand{\arraystretch}{1.2}
  \begin{tabularx}{\textwidth}{lXllXc}
    \toprule
    Project & Version & Commit ID & PUT & Format of Initial Seeds & Initial Edge Coverage \\
    \midrule
    Bloaty & v1.1+ & 60209eb & fuzz\_target & Executable (e.g., ELF, PE, Mach-O) & 4773\\
    libmpeg2 & N/A & 5432dc1 & mpeg2\_dec\_fuzzer & MPEG2 & 2428 \\
    PHP & 8.0+ & fdf0455f & php-fuzz-execute & PHP source code & 25241 \\
    HarfBuzz & 3.1.0 & 77eeec5 & hb-shape-fuzzer & Font (e.g., TrueType, OpenType) & 15298 \\
    Xpdf & 4.03 & N/A & fuzz\_pdfload & PDF & 4755 \\
    libtpm2 & N/A & bc3bb26 & tpm2\_execute\_command\_fuzzer & TPM command & 3884\\
    libyaml & v0.2.5+ & f8f760f & libyaml\_dumper\_fuzzer & YAML & 1310 \\
    libzip & 1.8.0+ & bff2eb9 & zip\_read\_fuzzer & ZIP & 805 \\
    libgit2 & v1.3.0+ & 50b4d53 & download\_refs\_fuzzer & Git packet & 3911 \\
    file & 5.41+ & fcbb5d8 & magic\_fuzzer & any (e.g., Zstd compressed file) & 1171 \\
%    MuPDF & 1.19.0+ & ef3d68d & pdf\_fuzzer & PDF & 16936 \\
%    libxml2 & 2.9.12+ & dea91c9 & xml & XML & 7027 \\
    \bottomrule
  \end{tabularx}
\label{tab:put_details}
\end{table*}

%\section{Full Results of Some Experiments}
%\label{sec:full_result}

%Table~\ref{tab:alg_cmp_all}, Figure \ref{fig:vis_bandits} and Figure \ref{fig:full_ablation_time_vs_cov} show the omitted results.

\begin{table*}[htb]
\centering
\caption{Median edge coverage obtained by AFL++ and 8 versions of \OurMethodName-AFL++ in 10 PUTs after 24 h. }

\begin{tabular}{lccccccccc}
\toprule

PUT & AFL++ & UCB1 & KLUCB & TS & dTS & dBE & ADS-TS & EXP3-IX & EXP3++ \\
\midrule

bloaty & \textit{1845.5} & 2198.5 & 2246.0 & 2232.5 & 2191.0 & 2292.0 & \textbf{2340.0} & 2181.5 & 2231.5 \\
harfbuzz & \textit{13497.5} & 14031.5 & 14247.5 & 14360.5 & \textbf{14374.0} & 14067.5 & 14149.0 & 13883.0 & 13891.0 \\
xpdf & \textit{3384.0} & 3494.0 & 3812.5 & \textbf{4618.5} & 4166.5 & 3791.5 & 3902.0 & 3860.0 & 3615.0 \\
libzip & \textit{267.5} & 272.0 & 274.0 & 268.0 & 268.5 & 271.5 & \textbf{276.0} & 271.5 & 268.0 \\
libgit2 & 898.0 & 888.5 & 890.5 & 906.5 & \textbf{916.0} & 884.0 & 914.0 & 899.5 & \textit{881.0} \\
php & \textit{9841.5} & 11861.0 & 13551.5 & \textbf{14324.0} & 14187.5 & 12657.5 & 13408.0 & 11423.5 & 11828.5 \\
libmpeg2 & \textit{1873.5} & 1900.5 & 1905.0 & 1905.5 & \textbf{1906.5} & 1903.0 & \textbf{1906.5} & 1897.0 & 1902.0 \\
tpm2 & \textit{281.5} & 299.5 & 313.0 & 317.0 & \textbf{317.5} & 305.0 & 311.0 & 298.5 & 291.0 \\
libyaml & 2811.5 & 2841.0 & \textbf{2841.5} & \textit{2800.5} & 2837.0 & 2827.5 & 2831.5 & 2828.0 & 2834.5 \\
file & 830.5 & 829.5 & 828.0 & 827.0 & 827.5 & 833.5 & \textbf{840.5} & 826.5 & \textit{826.0} \\

\bottomrule

\end{tabular}

\label{tab:alg_cmp_all}
\end{table*}

\begin{table*}[htb]
\centering
\caption{P-value of Mann-Whitney's U test (Holm-Bonferroni corrected) and Vargha-Delaney's $\hat{A}_{12}$ between AFL++ and the fuzzer in the column for the evaluation conducted in Section~\ref{subsec:eval-vs-existing}. If the p-value is bold, the difference is significant in the test ($p < 0.01$). The characters `L', `M', `S' and `N' in parentheses indicate that the effect size is large, medium, small, and none, respectively, according to \cite{A12}. The `+' sign means the fuzzer in the column is superior to AFL++ when compared by rank sum as well as $\hat{A}_{12}$, and the `-' sign means the opposite.}
\begin{tabular}{lllllllllllll}
 \toprule

  & \multicolumn{2}{c}{MOpt} & \multicolumn{2}{c}{CMFuzz} & \multicolumn{2}{c}{Karamcheti} & \multicolumn{2}{c}{\HavocMAB{}} & \multicolumn{2}{c}{SLOPT} \\
  \cmidrule(r){2-3}\cmidrule(r){4-5}\cmidrule(r){6-7} \cmidrule(r){8-9} \cmidrule(r){10-11}
  PUT & $p$ & $\hat{A}_{12}$ & $p$ & $\hat{A}_{12}$ & $p$ & $\hat{A}_{12}$ & $p$ & $\hat{A}_{12}$ & $p$ & $\hat{A}_{12}$ \\
\midrule

openssl\_x509 & \textbf{ < 0.001 } & 0.82 (+L) & \textbf{ 0.023 } & 0.71 (+L) & \textbf{ < 0.001 } & 0.92 (+L) & \textbf{ < 0.001 } & 0.82 (+L) & \textbf{ < 0.001 } & 0.91 (+L) \\
re2-2014-12-09 & \textbf{ < 0.001 } & 0.18 (-L) & > 0.1 & 0.37 (-S) & > 0.1 & 0.38 (-S) & > 0.1 & 0.47 (-N) & > 0.1 & 0.52 (+N) \\
proj4-2017-08-14 & \textbf{ < 0.001 } & 0.08 (-L) & \textbf{ < 0.001 } & 0.86 (+L) & \textbf{ < 0.001 } & 0.99 (+L) & > 0.1 & 0.54 (+N) & \textbf{ < 0.001 } & 0.92 (+L) \\
sqlite3\_ossfuzz & > 0.1 & 0.55 (+N) & \textbf{ < 0.001 } & 0.85 (+L) & \textbf{ < 0.001 } & 0.93 (+L) & 0.1 & 0.68 (+M) & \textbf{ < 0.001 } & 1.00 (+L) \\
libxml2-v2.9.2 & \textbf{ < 0.001 } & 0.08 (-L) & \textbf{ < 0.001 } & 0.93 (+L) & \textbf{ < 0.001 } & 0.98 (+L) & \textbf{ < 0.001 } & 0.97 (+L) & \textbf{ < 0.001 } & 0.84 (+L) \\
freetype2-2017 & \textbf{ < 0.001 } & 0.08 (-L) & 0.094 & 0.33 (-M) & > 0.1 & 0.54 (+N) & > 0.1 & 0.52 (+N) & \textbf{ < 0.001 } & 0.79 (+L) \\
libpcap\_fuzz\_both & > 0.1 & 0.57 (+S) & \textbf{ < 0.001 } & 0.79 (+L) & \textbf{ < 0.001 } & 0.80 (+L) & \textbf{ < 0.001 } & 0.87 (+L) & \textbf{ < 0.001 } & 0.81 (+L) \\
libpng-1.2.56 & > 0.1 & 0.42 (-S) & > 0.1 & 0.36 (-M) & > 0.1 & 0.49 (-N) & > 0.1 & 0.56 (+S) & 0.049 & 0.68 (+M) \\
lcms-2017-03-21 & > 0.1 & 0.45 (-N) & \textbf{ 0.037 } & 0.70 (+M) & \textbf{ < 0.001 } & 0.85 (+L) & > 0.1 & 0.37 (-S) & \textbf{ < 0.001 } & 0.88 (+L) \\
vorbis-2017-12-11 & > 0.1 & 0.39 (-S) & > 0.1 & 0.56 (+S) & \textbf{ < 0.001 } & 0.20 (-L) & > 0.1 & 0.62 (+S) & 0.092 & 0.65 (+M) \\

\bottomrule
\end{tabular}
\label{tab:statistics}
\end{table*}

\clearpage

\section{Algorithm Overview}

\begin{algorithm}[H]

\centering
\caption{Pseudocode of \OurMethodName{}}
\label{alg:slopt}

\begin{algorithmic}[0]

\Require{\mbox{}\\
    $initial\_seeds$ -- a set of initial test cases \\
    $program$ -- a PUT to be fuzzed
}

\Ensure{\mbox{}\\
    $queue$ -- a set of valuable test cases \\
    $crashes$ -- a set of test cases that trigger crashes
}

%\begin{adjustwidth}{-9pt}{}
%\setstretch{0.85}
\vspace{5pt}

\Function{RandomMutation}{$seed, instance_{mut}, instances_{bat}$}
\State $input$ $\gets$ \Call{CopyBytesFromSeed}{$seed$}
\State $mutation$ $\gets$ \Call{SelectArm}{$instance_{mut}$}
\State $idx$ $\gets$ \Call{GetGroupIndex}{$len(input)$}
\State $batch\_size$ $\gets$ \Call{SelectArm}{$instances_{bat}[idx][mutation]$}
\For{$i$ $\gets$ $1$ \textbf{to} $batch\_size$}
    \State $pos$ $\gets$ \Call{SelectPosition}{$input$}
    \State $input$ $\gets$ \Call{ApplyOperator}{$mutation, input, pos$}
\EndFor
\State \textbf{return} $input, mutation, batch\_size$
\EndFunction

%\end{adjustwidth}

%\vspace{-6pt}

%\begin{adjustwidth}{-9pt}{}
%\setstretch{0.85}

\vspace{5pt}

\Function{MutationFuzzing}{$initial\_seeds, program$}

\State $crashes$ $\gets$ $\varnothing$
\State $queue$ $\gets$ \Call{ConstructQueue}{$initial\_seeds$}
\State $instance_{mut}$ $\gets$ \Call{CreateBanditArms}{$number\_of\_mutations$}
\For{$i$ $\gets$ $1$ \textbf{to} $5$}
 \For{$j$ $\gets$ $1$ \textbf{to} $number\_of\_mutations$}
  \State $instances_{bat}[i][j]$ $\gets$ \Call{CreateBanditInstance}{$7$}
 \EndFor
\EndFor

\State

\While{ $\neg$ \Call{UserWantsStop}{\null}}
 \State $seed$ $\gets$ \Call{SelectSeed}{$queue$}
 \State $energy$ $\gets$ \Call{DecideEnergy}{$seed$}
 \For{$i$ $\gets$ $1$ \textbf{to} $energy$}
  \State $input, mutation, batch\_size$ 
  \State $\gets$ \Call{RandomMutation}{$seed, instance_{mut}, instances_{bat}$}
  \State $result$ $\gets$ \Call{ExecutePUT}{$program, input$}
  \State $b$ $\gets$ \Call{WasInputValuable}{$result$}
  \State \Call{RewardArm}{$mutation, b$}
  \State \Call{RewardArm}{$batch\_size, b$}
  \State \Call{SaveInputIfValuable}{$queue, input, result$}
  \State \Call{SaveInputIfCrash}{$crashes, input, result$}
 \EndFor
\EndWhile
\EndFunction

%\end{adjustwidth}

\end{algorithmic}
\end{algorithm}


%%%%%%%%%%%%%%%%%%%%%%%%%%%%%%%%%%%%%%%%%
%% For the final version of the paper: %%
%%%%%%%%%%%%%%%%%%%%%%%%%%%%%%%%%%%%%%%%%

%% Author information
%\vspace{4ex}\noindent
%\textbf{Author One} is\dots
%
%\bigskip\noindent
%\textbf{Author Two} is\dots
%
%\bigskip\noindent
%\textbf{Author Three} is\dots

%% Reception and acceptance information
%\bigskip
%\paragraph{Received: Month DD, 20YY; Accepted: Month DD, 20YY.}

\end{document}
