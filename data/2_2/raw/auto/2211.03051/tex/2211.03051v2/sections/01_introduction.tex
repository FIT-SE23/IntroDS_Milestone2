
%%%%%%%%%%%%%%%%
% Introduction %
%%%%%%%%%%%%%%%%

\section{Introduction}
\label{sec:introduction}

% zebrafish are model organisms of disease
Zebrafish are commonly used as model organisms of disease due to their genetic tractability and high reproductive capabilities \cite{newman_using_2014}. To identify new therapeutics for disease treatment and management, high-throughput drug screens can be performed on larval zebrafish in multi-well plates, where some phenotypic response to the treatment is measured. Finding a measurable response can be a challenge, however, which is what we aimed to interrogate here.

%  technique that can analyse what's going on in the fish is looking at their behaviour. Tests for anxiety are common (ref). You can look at behavioural differences in disease models too in order to estimate stuff like the severity of the disease or to interpret their behavioural deficits (ref). We can interpret these behaviours, hypothesise why they're that way to interrogate disease mechanisms. Additionally, observing statistically significant differences between genotypes is a good way to validate the disease model.

% behaviour is complex though, how do we analyse it?
The behaviour of zebrafish larvae can be analysed by tracking their movement in a controlled environment and extracting features such as the distance travelled, velocity of movement, or current position in the well. However, analysis of this data is complex as it is the combination of these features that defines behaviour.

% using ML for behaviour, PD, and zebrafish
The application of machine learning (ML) in biology for classifying model organisms, including zebrafish, is not a new idea in any account. Deep convolutional neural networks trained on coloured images of zebrafish are able to classify them by sex \cite{hosseini_efficient_2019}. The authors did not limit their work to neural networks, however, as they show that sex can also be determined by modelling caudal fin correlation via linear discrimination using support vector machines (SVMs), another well-known supervised learning method. The effectiveness of these approaches are unfortunately unclear due to missing explanation of their method for model training and evaluation.

% The network architecture was made up of several convolutional blocks, each containing two convolutional layers, a batch norm layer, and a ReLU activation. Skip connections were used to preserve information that may have been lost due to striding, although it seems suboptimal to have these connections between each block as downsampling only occurs every three blocks. 

% what we did here
In this paper, we present a non-linear ML classifier that can predict the genotype of larval zebrafish models of Parkinson's disease (mutation in \textit{dnajc6}, known to cause juvenile-onset PD in humans) with an average of 84\% validation accuracy based on 2D behavioural features. This classifier could potentially be used to indicate whether a mutant larva is displaying a ``healthy" or ``mutant" behavioural phenotype following a certain treatment. We then analyse the impact of behavioural features by calculating the integrated gradients of our models to inform us of the distinguishing features of the genotypes. 
% Our presented architecture is small enough to be trained on a portable laptop, making it ideal for further research purposes.

% Our network:
% - validates the dataset, disease model, and behaviour testing methods
% - tells us which behaviours are distinguishing the genotypes
% - is simple and computationally inexpensive

% - machine learning can classify differences in human PD people

% Research question: can you observe behavioural differences between wildtype and disease mutant (genetic) fish?
%     Yes, statistical analyses say so.
% Why does this matter?
% 1. Indicates the validity of the disease model
% 2. We can interpret these behaviours, hypothesise why they're that way to interrogate disease mechanisms
% 3. We can use it to see if their behaviour is rescued in drug screens
% Problems:
%     Behaviour is complex
%     Statistical analyses have low power (because of this)
%     Fish are very variable so modelling is good?
% Solution:
    % Make a non-linear machine learning classifier - tell you the genotype of the fish based on simple 2D behavioural features
% How does this address our goals and problems?
% 1. If there is a difference this validates the disease model
% 2. Integrated gradients give us insight into the features
% 3. A good classifier can eventually be used for the drug screens

% Hypothesis: a ML classifier would be able to tell the difference between wildtype and mutant PD zebrafish larvae

% Outcome: The classifier knows! this is a architecture that can kinda do that
%     validates the dataset - the disease model and the way we gathered the data
%     we know which features contribute to what - how can this be further validated?
%     number of parameters for network (~17,000) COMPUTATIONALLY EFFICIENT 

% Parkinson's disease (PD) is a progressive neurodegenerative movement disorder that is generally thought of as an adult-onset disease. However, rare loss of function mutations in the \textit{DNAJC6} gene are known to cause juvenile-onset PD (JPD) (Olgiati). Symptom onset occurs before 20 years of age and disease progression is rapid and severe.

% As there is no cure or disease-modifying therapeutic available for PD, current treatments centre around symptom management. Levodopa (a dopamine precursor) is commonly administered to restore DA levels and signalling (ref).  While this is effective for adult-onset forms of PD, treatment resistance is often encountered in JPD. This is largely due to our lack of understanding regarding the earliest molecular mechanisms driving neurodegenerative disease. Investigation of these early disease mechanisms in pre-symptomatic human brains is not possible and post-mortem brains are not indicative of this early disease state. This necessitates the use of animal models, in this case the zebrafish.

% Zebrafish are an advantageous model organism due to their fecundity. High-throughput drug screening can be performed on larval zebrafish in multi-well plates, where some phenotypic response to the treatment is measured. Finding a measurable response can be a challenge though, which is what we aim to interrogate here.

% The behaviour of zebrafish larvae can be analysed by tracking their movement in a controlled environment and extracting behavioural features such as the distance travelled, velocity of movement, or the area of the well they are currently in. Here, we aimed to begin the development of an AI classifier that can predict the \textit{DNAJC6} genotype of 5 day old zebrafish larvae. This classifier could potentially be used to indicate whether a mutant larva is displaying a "healthy" or "mutant" behavioural phenotype following a certain treatment.

% \subsection{Machine Learning}
% Research domains outside of computer science are beginning to utilise machine learning (ML) techniques for a wide range of tasks. Autonomous driving, facial recognition, and music synthesis are just a few of those present in our everyday lives (need refs). A common issue in ML is insufficient data to effectively train predictive models. On the contrary, researchers in the field of molecular biology often obtain large, complex datasets that are difficult to interpret in a meaningful way (and model by basic linear regression? linear techniques?). In this way, we can see that the careful application of ML techniques can be beneficial in directing research and providing evidence for new hypotheses. While having large amounts of data is beneficial, it does not guarantee convergence of a regression model. These issues can be caused by problems with the data or the model itself, necessitating good practices in data collection, data processing, and model design/selection (see XXXX for a review).

% Specifically relating to work in zebrafish models of PD or other neurodegenerative diseases, we can use these tools to model high-dimensional, non-linear behavioural differences between genotypes. A model with sufficient accuracy and generalisation could potentially be used downstream for identifying new drug candidates.

% The use of ML has shown to be useful in detecting neurodegenerative diseases in humans 
% - Machine learning is very practically useful in many domains
%     - Autonomous cars, person identification, production lines, etc.
% - We can use it for testing and verifying therapeutics from animal disease models
% - We will be specifically implementing multi-class classification
% - Little to no research using ML has been done for zebrafish larvae


%%%%%%%%%%%%%%%%
% Related Work %
%%%%%%%%%%%%%%%%
% \subsection{Related Work}

% The application of ML on model organisms is not a new idea in any account and also holds true for zebrafish. Deep convolutional neural networks trained on coloured images of zebrafish are able to be classified by sex according to its phenotype \cite{hosseini_efficient_2019}. The architecture was made up of several blocks, each containing two convolutional layers, a batch norm layer, and a ReLU activation. Skip connections were used to preserve information that may have been lost due to striding, although it seems suboptimal to have these connections between each block as downsampling only occurs every three blocks. The authors did not limit their work to neural networks, however, as the they show that sex can also be determined by modelling caudal fin correlation with support vector machines (SVMs), another well-known supervised learning method. The effectiveness of these approaches are unfortunately unclear due to missing explanations of their method for model training and evaluation. The type of SVM was also not disclosed.

% Fish behaviour is primarily captured using vision-based monitoring and is categorised into 2D and 3D observations \cite{xia_aquatic_2018}. Single-camera observations can observe behaviour in two dimensions but cannot capture the full breadth of information avaliable. Three dimensional observations, using two cameras, can capture information that would otherwise be lost in 2D \cite{yang_zebrafish_2021}. For example, when a fish swims up toward the surface or down toward the bottom. While this extra information can allow us to capture finer behaviour of the fish it necessitates a significant leap in hardware and software complexity. For these reasons we collect two-dimensional behaviour data, which seems to be sufficient for our experiments and all other experiments cited in this paper.

% Zebrafish genotypes can also be classified using an evolutionary algorithm (EA) \cite{hughes_machine_2020}. An EA uses a mutation, crossover, and selection process on a population to optimise a solution over a number of generations. The authors make use of cartesian genetic programming - a directed acyclic graph-based genetic programming algorithm. Put simply, it is a way of representing a problem as a graph that can be mutated by changing its encoded string. If we apply the mutation, crossover, and selection process, we should eventually get a high-fitness individual (optimised solution). One of the justifications for this EA approach was its white-box nature, that is, the functions that make up the classifier can be inspected to provide insight into contribution of the input features. We will be showing that neural networks are more white-box in nature than they get credit for.



% However, little research has been done on using ML for classifying early-onset PD in 5 day old zebrafish.



% - Machine learning has been used with zebrafish for:
%     - Classifying sex.
%     - Counting zebrafish embryos. Nah not doing this, sorry Egypt dudes.
% - Although there exists a method for tracking the movement of zebrafish in 3D (ref), it has been shown that 2D behaviour tracking encapsulates sufficient data for genome classification (ref).
% - The other paper:
%     - They use an evolutionary algorithm for classification
%     - CGP was chosen for its white-box nature, allowing the identification of the impacts of input features
%     - Binary classification (wt and mutant)
% - Neural networks, despite having a reputation for being a black-box, are becoming more understood. We demonstrate how network gradients can shed light on the impacts that movement features have on the network's predictions.