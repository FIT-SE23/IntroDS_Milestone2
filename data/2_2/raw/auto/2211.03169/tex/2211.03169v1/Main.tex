\documentclass{article}

%\usepackage{corl_2022} % Use this for the initial submission.
\usepackage[final]{corl_2022} % Uncomment for the camera-ready ``final'' version.
%\usepackage[preprint]{corl_2022} % Uncomment for pre-prints (e.g., arxiv); This is like ``final'', but will remove the CORL footnote.

\title{Learning Riemannian Stable Dynamical Systems\\ via Diffeomorphisms}

% The \author macro works with any number of authors. There are two
% commands used to separate the names and addresses of multiple
% authors: \And and \AND.
%
% Using \And between authors leaves it to LaTeX to determine where to
% break the lines. Using \AND forces a line break at that point. So,
% if LaTeX puts 3 of 4 authors names on the first line, and the last
% on the second line, try using \AND instead of \And before the third
% author name.

% NOTE: authors will be visible only in the camera-ready and preprint versions (i.e., when using the option 'final' or 'preprint'). 
% 	For the initial submission the authors will be anonymized.

\author{
     Jiechao Zhang\textsuperscript{\ensuremath{1,2}}
     \quad
     Hadi Beik-Mohammadi\textsuperscript{\ensuremath{1,2}}
     \quad
     Leonel Rozo\textsuperscript{\ensuremath{1}}
 	\\
 	\textsuperscript{\ensuremath{1}}Bosch Center for Artificial Intelligence. Renningen, Germany.
 	\\
 	\textsuperscript{\ensuremath{2}.} Karlsruhe Institute of Technology (KIT), Karlsruhe, Germany.
 	\\
 	\href{mailto:leonel.rozo@de.bosch.com}{\textrm{leonel.rozo@de.bosch.com}} 
 	\quad
 	\href{mailto:hadi.beik-mohammadi@de.bosch.com}{\textrm{hadi.beik-mohammadi@de.bosch.com}}
}

\usepackage{amsmath} % assumes amsmath package installed
\usepackage{amssymb}  % assumes amsmath package installed
\usepackage{amsthm}
\usepackage{siunitx}
\usepackage{adjustbox}
\usepackage{bm}  % assumes amsmath package installed
\usepackage{graphics} % for pdf, bitmapped graphics files
\usepackage{epsfig} % for postscript graphics files
\usepackage{wrapfig}
\usepackage{subcaption}
\usepackage{caption}
\usepackage{subcaption}
\usepackage[font=small,labelfont=small]{caption}
\usepackage{color}
% \usepackage[table]{xcolor}
\usepackage[ruled, linesnumbered, vlined]{algorithm2e}
\usepackage{sidecap}
\usepackage{tabularx}
\usepackage{booktabs} % Added for the table
% \usepackage{tikz}
% \usepackage{pgfplots}

%\usetikzlibrary{arrows.meta}
\newtheorem{theorem}{Theorem}

\graphicspath{{Images/}}

% New commands
\newcommand{\etal}{\MakeLowercase{\textit{et al.}}}
\newcommand{\SphereManifold}{\mathcal{S}}
\newcommand{\RtimeS}{\mathbb{R}^3 \times \mathcal{S}^3}
\newcommand{\trsp}{\mathsf{T}} % Transpose
\newcommand{\euclideanspace}{\mathbb{R}}
\newcommand{\manifold}{\mathcal{M}}
\newcommand{\tangentspace}[1]{\mathcal{T}_{#1}\mathcal{M}}
\newcommand{\expmap}[2]{\text{Exp}_{#1}(#2)}  % Exponential map of #2 at #1
\newcommand{\logmap}[2]{\text{Log}_{#1}(#2)}  % Logarithmic map of #2 at #1
\newcommand{\prltrsp}[3]{\Gamma_{#1 \rightarrow #2}\big(#3\big)}  % Parallel transport of #3 from #1 to #2
\newcommand{\diffeomorphism}{\psi}
\newcommand{\proju}{\operatorname{proju}}
\newcommand{\innerprod}[3]{\langle #2, #3 \rangle_{#1}}  % Inner product of #2 and #3 at #1
\newcommand{\manifolddist}[2]{d_{\manifold}(#1, #2)}
\begin{document}
\maketitle

%===============================================================================

\begin{abstract}
Dexterous and autonomous robots should be capable of executing elaborated dynamical motions skillfully.
Learning techniques may be leveraged to build models of such dynamic skills.
To accomplish this, the learning model needs to encode a stable vector field that resembles the desired motion dynamics. 
This is challenging as the robot state does not evolve on a Euclidean space, and therefore the stability guarantees and vector field encoding need to account for the geometry arising from, for example, the orientation representation.
To tackle this problem, we propose learning Riemannian stable dynamical systems (RSDS) from demonstrations, allowing us to account for different geometric constraints resulting from the dynamical system state representation.
Our approach provides Lyapunov-stability guarantees on Riemannian manifolds that are enforced on the desired motion dynamics via diffeomorphisms built on neural manifold ODEs.
We show that our Riemannian approach makes it possible to learn stable dynamical systems displaying complicated vector fields on both illustrative examples and real-world manipulation tasks, where Euclidean approximations fail. 
\end{abstract}

% Two or three meaningful keywords should be added here
\keywords{Dynamical systems, Riemannian manifolds, Motion learning} 

%===============================================================================

%----------------------------------------------------------------------
%%% INTRODUCTION
%----------------------------------------------------------------------
% !TEX root = ../Main.tex


\Acp{BPM} have a long and rich history in optimization, going back at least to the introduction of \acl{MD} by Nemirovski \& Yudin \citep{NY83}.
In plain terms, \acp{BPM} are first-order (constrained) optimization algorithms that forego Euclidean projections in favor of a more sophisticated ``prox-mapping'' that minimizes a certain distance-like functional known as the Bregman divergence \citep{NY83,CT93,Bre67,Kiw97}.
When this Bregman divergence is the Euclidean distance squared, one recovers the standard projection-based methods;
other than that, depending on the problem's feasible region, different Bregman setups lead to a diverse collection of algorithms,
from exponentiated gradient descent on the simplex \citep{NY83,BecTeb03,ACBFS02},
to matrix multiplicative weights on the positive-semidefinite cone \cite{TRW05,KSST12},
variants of Karmarkar's affine scaling algorithm for linear programs \cite{VMF86},
etc.

One of the most appealing features of \acp{BPM} is that they achieve almost dimension-free convergence rates in problems with a convex structure and a favorable geometry \textendash\ such as the $L^{1}$ ball, the spectraplex, second-order cones, etc. \cite{Bub15,Nes09,BecTeb03}.
This is owed to a delicate interplay between the algorithms' non-Euclidean update scheme and the global geometry of the problem's domain.
However, these (almost) dimension-free guarantees also come with some strings attached:
they do not concern the sequence of iterates generated by the method, but only its time average
\revise{(or, through the same, ``regret-based'' analysis, the method's ``best iterate'')};
in this way, the best guarantee that can be achieved after $\run$ iterations is $\bigoh(1/\run)$.

In terms of oracle complexity, this is sufficient for problems that are not strongly convex\,/\,strongly monotone, but if one targets finer, geometric convergence rates,
\revise{the inherent averaging involved in regret-based guarantees is hard to compensate.}
And, on the other extreme, if the problem is not convex\,/\,monotone to begin with, iterate averaging does not provide any quantifiable benefits whatsoever, so it becomes crucial to study the actual trajectory of the method.


%----------------------------------------------------------------------
%%% CONTRIBS
%----------------------------------------------------------------------
\para{Our contributions}

Our paper seeks to quantify the last-iterate convergence rate of \aclp{BPM} as a function of the Bregman divergence defining the method and the local geometry that it induces.
To treat this question in as general a manner as possible, we focus on \ac{VI} problems of the form
\begin{equation}
\label{eq:VI}
\tag{VI}
\text{Find $\sol\in\points$ such that}
	\;\;
	\braket{\vecfield(\sol)}{\point - \sol}
	\geq 0
	\;\;
	\text{for all $\point\in\points$},
\end{equation}
where $\points$ is a closed convex subset of a finite-dimensional normed space $\pspace$, and $\vecfield \from \points \to \dspace$ is a (possibly non-monotone) single-valued operator on $\points$ with values in $\dspace$, the dual of $\pspace$.
This problem is a staple of many areas of mathematical programming, game theory and data science, as it provides a template for ``optimization beyond minimization'' \textendash\ \ie for problems where finding an optimal solution does not necessarily involve minimizing a loss function.
In particular, in addition to standard minimization problems \textendash\ which are recovered when $\vecfield = \nabla\obj$ for some smooth function $\obj$ \textendash\ the general formulation \eqref{eq:VI} includes saddle-point problems, games, complementarity problems, etc.;
for an introduction, see \cite{FP03} and references therein.

In this broad context, we examine the rate of convergence of a wide class of \aclp{BPM} to local solutions of \eqref{eq:VI} that satisfy a \acl{SOS} condition.
Specifically, the class of algorithms we consider includes as special cases
\begin{enumerate*}
[(\itshape i\hspace*{1pt}\upshape)]
\item
the original \acf{MD} algorithm of \cite{NY83};
\item
the \acf{MP} method of Nemirovski \cite{Nem04} \textendash\ which has the same update structure as the Bregman-based algorithm of \cite{AT05} and contains as a special case the \acf{EG} algorithm of \cite{Kor76};
\item
the so-called \acf{OMD} method of \cite{RS13-NIPS} \textendash\ itself a Bregman analogue of the modified Arrow-Hurwicz algorithm of \cite{Pop80};
\end{enumerate*}
etc.

Our first finding is a crisp characterization of last-iterate convergence rate of \acp{BPM} in terms of the local geometry induced by the underlying Bregman function near a given solution of \eqref{eq:VI}.
We make this dependence precise via the notion of the \emph{Legendre exponent}, a regularity measure for Bregman methods due to \cite{AIMM21}, which can roughly be described as the logarithmic ratio of the volume of a Euclidean ball to that of a Bregman ball of the same radius.
For example, Euclidean methods have a Legendre exponent of $\legexp = 0$ and they converge at a linear rate;
entropic methods have a Legendre exponent of $\legexp = 1/2$ at boundary points, and they converge at a rate of $\bigoh(\run^{-1})$;
more generally,
as we show in \cref{thm:general}, methods with a Legendre exponent $\legexp>0$ converge at a rate of $\bigoh(\run^{1-1/\legexp})$.
\PM{We need to fix this: the $1-1/\legexp$ exponent is not consistent with the $\bigoh(1/\run)$ expression.}
\WA{I don't see the issues, yes this expression is not well-defined for $\legexp = 0$ but this is normal, the two situations differ radically.}
The Euclidean regime ($\legexp = 0$) is perfectly aligned with existing results for the geometric last-iterate convergence rate of the \ac{EG} algorithm and its variants \citep{GBVV+19,Mal15,HIMM19,MOP20}.
By contrast, the Legendre regime ($\legexp > 0$) indicates a significant drop in the algorithm's last-iterate convergence speed, even though ergodic convergence rates \cite{Nes04} and results for bilinear games \cite{WLZL21} might suggest otherwise.

Subsequently, motivated by applications to game theory and linear programming, we take a closer look at the convergence rate of \acp{BPM} across the constraints that are active at a solution $\sol$ of \eqref{eq:VI} depending on the position of $\vecfield(\sol)$ relative to said constraints. 
This analysis reveals that Bregman proximal methods have a particularly fine structure:
along \emph{sharp directions} (\ie constraints along which $\vecfield(\sol)$ is strictly inward-pointing), \acp{BPM} converge
\begin{enumerate*}
[(\itshape i\hspace*{1pt}\upshape)]
\item
at a rate of $\bigoh(1/\run^{1/(2\legexp-1)})$ if $1/2 < \legexp < 1$;
\item
at a \emph{geometric rate} if $0 < \legexp \leq 1/2$ (\eg for entropic methods);
and
\item
in a \emph{finite} number of iterations if $\legexp=0$
\end{enumerate*}
(\cf \cref{thm:sharp}).
Thus, even though the estimates of \cref{thm:general} are, in general tight, the actual convergence rate of a Bregman method along different coordinates\,/\,constraints could be starkly different \textendash\ and, in fact, dramatically faster if the solution under study is itself sharp.

The closest antecedent of our work is the conference paper \cite{AIMM21} where the Legendre exponent was introduced to analyze the convergence of \ac{OMD} in \emph{stochastic} \ac{VI} problems (without considering sharp directions and/or faster identification rates).
The stochastic and deterministic settings are obviously very different, both in the challenges involved as well as the rates obtained, so there is no overlap in our analysis and results.
Other than that, we are not aware of any comparable results in the literature concerning the radically different convergence landscape of \acp{BPM} along active and inactive constraints.
%===============================================================================

\section{Background}
\label{sec:background}

\subsection{Dynamical Systems and Lyapunov Stability}
\label{subsec:DSlyapunov}
We here give a short review of Lyapunov stability in the Euclidean setting.
Let us assume an autonomous dynamical system $\dot{\bm{x}} = f(\bm{x})$, with a single equilibrium point $\bm{x}^*$, where $\bm{x} \in \mathbb{R}^n$ is the state variable.
Consider a potential function $V(\bm{x}(t))$ describing the energy of such a system.
If this system loses energy over time and the energy is never restored, the system must eventually reach some final resting state.
This idea is formally described as (see~\citep{Slotine91:AppliedNonlinearCtrl} for details):
\begin{theorem}[Lyapunov Stability]
\label{th:lyapunov_stability}
A dynamical system $\dot{\bm{x}} = f(\bm{x})$ is globally asymptotically stable at $\bm{x}^*$ if there exists a continuously differentiable Lyapunov function $V(\bm{x}): \mathbb{R}^n \rightarrow \mathbb{R}$ such that 
\begin{equation}
        V(\bm{x}^*) = 0 , \quad \dot{V}(\bm{x}^*) = 0 , \quad V(\bm{x}) > 0 ,\  \forall \ \bm{x} \neq \bm{x}^* , \quad  \dot{V}(\bm{x}) < 0 , \ \forall \  \bm{x} \neq \bm{x}^* .
    \label{eq:lyapunov_conditions}
\end{equation}
\end{theorem}
From Theorem~\ref{th:lyapunov_stability} we know that we can always find a Lyapunov function that fulfills these four conditions in Eq.~\ref{eq:lyapunov_conditions} for a globally asymptotically stable dynamical system.
\vspace{.5cm}

%%%%%%%%%%%%%%%%%%%%%%%%%%%%%%%%%%%%%%%%%%%%%%%%%%%%%%%%%%%%%%%%%%%%%%%%%%%%%%%%%%%%%%%%%%%%%%%%%%%%%%%%%%%%%%%%%%%%%%%%%%%%%%%%%%%%%
\subsection{Riemannian Manifolds}
\label{subsec:RiemanianManif}
A smooth manifold $\manifold$ can be seen as a set of points that locally, but not globally, resemble the Euclidean space $\euclideanspace^d$~\citep{DoCarmo92:RiemannManifold,Lee18:RiemannManifold}. 
An abstract definition of a manifold specifies the topological, differential and geometric structure by using \emph{charts}, which are maps between parts of $\manifold$ to $\euclideanspace^d$.
The collection of these charts (a.k.a. local parameterizations) is called \emph{atlas}.
More formally, a chart on a smooth manifold $\manifold$ is a diffeomorphic mapping (i.e. a bijective and differentiable function) $\varphi: U \to \tilde{U}$ from an open set $U \subset \manifold$ to an open set $\tilde{U} \subseteq \euclideanspace^d$ (see Fig.~\ref{fig:coordinate_chart} in App.~\ref{app:RiemannianManif}).
Moreover, the transition map between two intersecting sets $U_1$ and $U_2$, given by $\varphi_1 \circ \varphi_2^{-1}$ or $\varphi_2 \circ \varphi_1^{-1}: \mathbb{R}^d \rightarrow \mathbb{R}^d$ is also a diffeomorphism.
The smooth structure of $\manifold$ makes it possible to take derivatives of curves on the manifold, leading to tangent vectors in $\euclideanspace^d$.
The set of tangent vectors of all curves at $\bm{x} \in \manifold$ spans a $d$-dimensional affine subspace of $\euclideanspace^d$, which is known as the \emph{tangent space} $\tangentspace{\bm{x}}$ of $\manifold$ at $\bm{x}$.
The collection of all tangent spaces of $\manifold$ is the \emph{tangent bundle} $\tangentspace{} = \bigsqcup_{\bm{x} \in \manifold} \tangentspace{x}$.
Therefore, a velocity vector $\dot{\bm{x}}$ at $\bm{x} \in \manifold$ lies on $\tangentspace{x}$, and consequently a vector field on $\manifold$ lies on $\tangentspace{}$.

The above definitions do not provide the mechanisms to measure how curved $\manifold$ is, or to compute distances on $\manifold$. 
To do so, we can endow $\manifold$ with a \emph{Riemannian metric}, which is a family of inner products $g_{\bm{x}}: \tangentspace{\bm{x}} \times  \tangentspace{\bm{x}} \rightarrow \euclideanspace$ associated to each point $\bm{x} \in \manifold$.
As a result, a \emph{Riemannian manifold} $(\manifold, g)$ is a smooth manifold endowed with a Riemannian metric~\citep{Lee18:RiemannManifold}. 
To operate with Riemannian manifolds, it is common practice to exploit the Euclidean tangent spaces. 
To do so, we resort to mappings back and forth between $\tangentspace{\bm{x}}$ and $\manifold$ using the exponential and logarithmic maps.
The exponential map $\expmap{\bm{x}}{\bm{u}}: \tangentspace{\bm{x}} \to \manifold$ maps a point $\bm{u}$ in the tangent space of $\bm{x}$ to a point $\bm{y}$ on the manifold, so that it lies on the geodesic starting at $\bm{x}$ in the direction $\bm{u}$, and such that the geodesic distance $d_{\manifold}(\bm{x}, \bm{y}) = d_{\euclideanspace}(\bm{x}, \bm{u})$. 
The inverse operation is the logarithmic map $\logmap{\bm{x}}{\bm{y}}: \manifold \to \tangentspace{\bm{x}}$.
We provide all the necessary operations in App.~\ref{app:RiemannianManif}.
\vspace{.5cm}

%%%%%%%%%%%%%%%%%%%%%%%%%%%%%%%%%%%%%%%%%%%%%%%%%%%%%%%%%%%%%%%%%%%%%%%%%%%%%%%%%%%%%%%%%%%%%%%%%%%%%%%%%%%%%%%%%%%%%%%%%%%%%%%%%%%%%
\subsection{Diffeomorphism}
\label{subsec:Diffeomorphism}
A diffeomorphism $\diffeomorphism: \manifold \rightarrow \mathcal{N}$ is a smooth bijective mapping between two smooth manifolds which preserves the topological properties of $\manifold$, and whose inverse $\diffeomorphism^{-1}$ is also smooth.
When learning stable dynamical systems, diffeomorphisms can be exploited to impose Lyapunov stability guarantees by transferring a manually-designed stable dynamical system on $\mathcal{N}$ to the desired manifold $\manifold$. 
We focus on constructing learnable diffeomorphisms that resemble continuous normalizing flows (CNFs)~\citep{chen2018:NeuralODE,Grathwohl19:FFJORD,Finlay2020:JacobianRegularization}, which are bijective and bidirectionally differentiable mappings, and have been recently exploited on density estimation problems~\citep{Rezende2015:VIwithNF,Papamakarios21:NormalizingFlows,Kobyzev2021:IntroductionNFs}.
We here exploit them to learn diffeomorphic mappings between Riemannian manifolds. 

Most CNFs are constructed from \emph{neural ordinary differential equation} (Neural ODEs) in Euclidean space~\citep{chen2018:NeuralODE,Grathwohl19:FFJORD,Finlay2020:JacobianRegularization}, with the exception of \emph{neural manifold ordinary differential equations} (Neural MODEs) on Riemannian manifolds~\citep{lou2020:NeuralMODE, mathieu2020:RCNormalizingFlows}.
Generally, Neural ODEs parametrize the dynamics of a hidden variable using a continuous-time ODE represented by a neural network, as follows,
\begin{equation}
    \dot{\bm{z}}(t) = f_{\bm{\theta}}(\bm{z}(t), t) ,
    \label{eq:neuralode}
\end{equation}
where $\bm{z} \in \euclideanspace^d$ is the state variable and $f_{\bm{\theta}}: \euclideanspace^d \times \euclideanspace \rightarrow \euclideanspace^d$ is a neural network.
According to~\citet{mathieu2020:RCNormalizingFlows} (see Theorem~\ref{th:vector_flows} in App.~\ref{app:NeuralMODEs}), we can extend CNFs to the Riemannian setting, where the state variable $\bm{z} \in \mathcal{M}$ and the vector field $f_{\bm{\theta}}: \mathcal{M} \times \mathbb{R} \rightarrow \tangentspace{}$.
As a result, we can use~\eqref{eq:neuralode} as a manifold ODE, whose initial value problem (IVP) solution results in a diffeormorphic mapping $\psi_{\bm{\theta}}: \manifold \rightarrow \mathcal{N}, \ \bm{x} = \bm{z}(t_s) \in \manifold$ and $\bm{y}=\bm{z}(t_e) \in \mathcal{N}$.
i.e. $\bm{y} = \psi_{\bm{\theta}}(\bm{x}) = \bm{x} + \int_{\tau=t_s}^{t_e} f_{\bm{\theta}}(\bm{z}(\tau), \tau) d\tau$.
To solve the IVP on $\manifold$, we leverage integrators on Riemannian manifolds based on the local representation via coordinate charts~\citep{hairer2011:ODEManifolds}.
This method uses a local representation of $\mathcal{M}$ defined by a coordinate map $\varphi: \mathcal{M} \supseteq U \rightarrow \tilde{U} \subseteq \mathbb{R}^d$ with coordinates $\bm{w}(t) = \varphi(\bm{z}(t))$. 
Computing integrators on $\manifold$ can be approximated by solving an equivalent ODE in $\euclideanspace^d$
\begin{equation}
    \dot{\bm{w}}(t) = D_{\varphi^{-1}(\bm{w}(t))}\varphi \circ f_{\bm{\theta}}(\varphi^{-1}(\bm{w}(t)), t) ,
    \label{eq:equivalentode}
\end{equation}
where $D_{\varphi^{-1}(\bm{w}(t))}\varphi$ represents the differential of $\varphi$ at $\varphi^{-1}(\bm{w}(t))$
(see App.~\ref{app:NeuralMODEs} and~\ref{app:integrators_manifolds} for details).
Additionally, we use the adjoint method~\citep{chen2018:NeuralODE} on Riemannian manifolds~\citep{lou2020:NeuralMODE} to compute gradients, which can also be used for calculating differentials. Consider a loss function $\mathcal{L} : \mathcal{M} \rightarrow \mathbb{R}$, in order to compute the derivative of $\mathcal{L}$ with respect to an intermediate variable $\boldsymbol{z}(t)$ of the manifold ODE, we can solve $\dot{\bm{a}}(t)^\trsp = - \bm{a}(t)^\trsp D_{\bm{z}(t)}  f_{\bm{\theta}}(\bm{z}(t), t)$, where $\bm{a}(t)^\trsp := D_{\bm{z}(t)} \mathcal{L}$  (as detailed in App.~ \ref{app:adjoint_method}).
\vspace{2cm}



%===============================================================================

\section{Learning Riemannian Stable Dynamical Systems}
\label{sec:approach}
\begin{wrapfigure}[13]{r}{0.46\linewidth}
    \centering
    \includegraphics[trim={0.0cm 0.0cm 0.0cm .5cm}, width=0.44\textwidth, ]{Images/Approach/diffeomorphsim_sphere.pdf}
    \caption{Architecture of a diffeomorphism-based stable vector field on the Riemannian manifold $\SphereManifold^2$.}
    \label{fig:diffeomorphism_based_svf}
\end{wrapfigure}

We here introduce our approach for learning stable dynamical systems on Riemannian manifolds from demonstrated point-to-point motions.
First, let us consider that the recorded demonstrations follow a dynamical system $\Dot{\bm{x}} = f(\bm{x})$, where the state $\bm{x}$ evolves on a Riemannian manifold $\mathcal{M}$ with velocity $\dot{\bm{x}} \in \tangentspace{\bm{x}}$.
This dynamical system is equivalent, under a change of coordinates, to another system defined on a latent Riemannian manifold $\mathcal{N}$.
Under the diffeomorphism $\psi_{\bm{\theta}}: \bm{x} \mapsto \bm{y} \in \mathcal{N}$, parameterized by $\bm{\theta}$, we map the observed states $\bm{x}$ onto $\mathcal{N}$.
Then, we evaluate the canonical stable vector field $g_{\bm{\gamma}}(\bm{y})$ to obtain the velocity $\dot{\bm{y}} \in \mathcal{T}_{\bm{y}} \mathcal{N}$.
Finally, we leverage the \emph{pullback operator} $D_{\bm{y}}\psi_{\bm{\theta}}^\star$ to project $\dot{\bm{y}}$ back to the tangent space $\tangentspace{\bm{x}}$.
The whole procedure can be expressed as follows, 
\begin{equation}
    \dot{\bm{x}} = (D_{\bm{y}}\psi_{\bm{\theta}}^\star \circ g_{\bm{\gamma}} \circ \psi_{\bm{\theta}})(\bm{x}) \\ =  D_{\bm{y}}\psi_{\bm{\theta}}^\star (\dot{\bm{y}}) ,
    \label{eq:diffeomorphism_based_svf_equation}
\end{equation}
which is illustrated in Fig.~\ref{fig:diffeomorphism_based_svf} (a proof is given in App.~\ref{app:stabilityAnalysis}).
In the sequel, we describe how we design a Lyapunov-stable vector field $g_{\bm{\gamma}}$ on $\mathcal{N}$ to provide stability guarantees on the learned dynamical system.
Later, we explain how to compute the diffeomorphism between the target and latent manifolds.
Finally, we introduce two different methods to compute the pullback operator $D_{\bm{y}}\psi_{\bm{\theta}}^\star$.
\vspace{-0.15cm}

\subsection{Lyapunov-stable Geodesic Vector Fields}
\label{subsec:GeodesicVF}
To design a stable vector field on the Riemannian manifold $\mathcal{N}$, we enforce the canonical dynamical system to follow geodesic curves that converge to a single equilibrium.
Such a vector field can be designed via the logarithmic map.
Specifically, given an equilibrium point $\bm{y}^* \in \mathcal{N}$, the corresponding velocity vector $\dot{\bm{y}} \in \mathcal{T}_{\bm{y}}\mathcal{N}$ can be computed as $\dot{\bm{y}} = g_{\bm{\gamma}}(\bm{y}) = k_{\bm{\gamma}}(\bm{y}) g_n(\bm{y})$ with the normalized geodesics vector field $ g_n(\bm{y}) := \frac{\operatorname{Log}_{\bm{y}}(\bm{y}^*)}{\lVert \operatorname{Log}_{\bm{y}}(\bm{y}^*) \rVert_2}$.
This implies that the direction of tangent vectors is fully specified by $\operatorname{Log}_{\bm{y}}(\bm{y}^*)$, while their magnitude depends on the scaling factor $k_{\bm{\gamma}}: \mathbb{R}^n \supset \mathcal{N} \rightarrow \mathbb{R}_{\geq 0}$.
We can prove the stability of this geodesic vector field by choosing the Lyapunov function $V(\bm{y}):=\langle F, F \rangle_{\bm{y}^*}$ with $F = \logmap{\bm{y}^*}{\bm{y}}$, and applying Theorem~\ref{th:stability_geodesics_vf} for Lyapunov stability on Riemannian manifolds, as detailed in App.~\ref{app:stabilityAnalysis}.
Given that our geodesic vector field is Lyapunov stable, we can easily prove that the desired dynamical system is also globally asymptotically stable by defining a new valid Lyapunov function $\tilde{V}(\bm{x}) := V(\psi_{\bm{\theta}}(\bm{x}))$ via the diffeomorphism $\psi_{\bm{\theta}}$, with a single equilibrium point $\bm{x}^* = \psi_{\bm{\theta}}^{-1}(\bm{y}^*) \in \mathcal{M}$.
As $\psi_{\bm{\theta}}$ preserves the topological properties of $\mathcal{N}$, the equilibrium point $\bm{x}^*$ is also globally asymptotically stable on $\mathcal{M}$ (see App.~\ref{app:stabilityAnalysis} for the proof).
Note that for certain Riemannian manifolds, it is only possible to guarantee \emph{quasi-global} stability guarantees due to the Poincaré-Hopf theorem (see App.~\ref{app:stabilityAnalysis} for details).

Note that we separate the parameterization for the magnitude and direction of vector fields to improve the expressiveness of our framework.
By relocating the scaling factor $k_{\bm{\gamma}}$ and normalizing the vector fields governed by~\eqref{eq:diffeomorphism_based_svf_equation}, we can obtain our final RSDS learning framework
\begin{equation}
    \dot{\bm{x}} = \hat{k}_{\bm{\gamma}}(\bm{x}) \frac{(D_{\bm{y}}\psi_{\bm{\theta}}^\star \circ  g_n \circ \psi_{\bm{\theta}})(\bm{x})}{\lVert (D_{\bm{y}}\psi_{\bm{\theta}}^\star \circ  g_n \circ \psi_{\bm{\theta}})(\bm{x}) \rVert_2} ,
    \label{eq:final_diffeomorphism_based_vf_equation}
\end{equation}
where $\hat{k}_{\bm{\gamma}}(\bm{x})$ is the new positive scaling factor that fully determines the magnitude of the learned vector fields. 
In App.~\ref{app:finalFramework}, we prove that the models~\eqref{eq:diffeomorphism_based_svf_equation} and \eqref{eq:final_diffeomorphism_based_vf_equation} are equivalent.

\subsection{Diffeomorphisms on Riemannian Manifolds}
\label{subsec:DiffeomorphRM}
Given the final RSDS in~\eqref{eq:final_diffeomorphism_based_vf_equation} and $M$ demonstrations, the goal of learning stable dynamics on a Riemannian manifold reduces to learning $\psi_{\bm{\theta}}$, computing its pullback operator $D_{\bm{y}}\psi_{\bm{\theta}}^\star$, and subsequently estimating $\hat{k}_{\bm{\gamma}}(\bm{x})$. 
However, due to the geometric constraints arising from $\mathcal{M}$, learning a diffeomorphism and calculating the corresponding pullback operator are non-trivial problems. 
To address them, we leverage Neural MODEs~\citep{lou2020:NeuralMODE} to build the diffeomorphism $\psi_{\bm{\theta}}$. 
Unlike~\citep{lou2020:NeuralMODE}, we propose a novel approach to compute the pullback operator by reversing the time interval of the ODE integration (see \S~\ref{subsec:DiffInvDiffeomorph}), avoiding to explicitly compute the corresponding inverse. 
We also propose a method to design Lyapunov-stable geodesic vector fields on a Riemannian manifold, which are leveraged to provide stability guarantees on the learned dynamical system, as explained in \S~\ref{subsec:GeodesicVF}.

According to Theorem~\ref{th:vector_flows} in App.~\ref{app:NeuralMODEs}, the dynamics $f_{\bm{\theta}}$ of Neural MODEs only has to be a $\mathcal{C}^1$ function.
To compute the diffeomorphism with a parametric Neural MODE, we solve an integration problem based on the local parameterization $\bm{w}(t)= \varphi(\bm{z}(t))$ (described in App.~\ref{app:integrators_manifolds}).
Using this method requires the selection of coordinate charts, which can be created via the exponential map $\varphi_i^{-1} = \operatorname{Exp}_{\bm{z}_i}$ and logarithmic map $\varphi_i = \operatorname{Log}_{\bm{z}_i}$, similarly to~\citep{lou2020:NeuralMODE}.
Under this choice of coordinate mapping and given a fixed number of charts $k$, the diffeomorphism $\psi_{\bm{\theta}}: \bm{x} = \bm{z}_0 \mapsto \bm{z}_k = \bm{y}$, obtained via integration on the manifold can be then viewed as the composition of blocks of solving Neural ODEs and chart transitions defined as,
\begin{equation}
    \begin{split}
    \psi_{\bm{\theta}} = \operatorname{Exp}_{\bm{z}_{k-1}} \circ \hat{\psi}_{\bm{\theta},k-1} \circ \operatorname{Log}_{\bm{z}_{k-1}} \circ \ldots \circ \operatorname{Exp}_{\bm{z}_0} \circ \hat{\psi}_{\bm{\theta},0} \circ \operatorname{Log}_{\bm{z}_0} , \quad \textrm{with} \\
    \hat{\psi}_{\bm{\theta}, i}(\bm{w}_i(t_{i, s})) = \bm{w}_i(t_{i, s}) + \int_{\tau=t_{i, s}}^{t_{i, e}} D_{\varphi_i^{-1}(\bm{w}_i(\tau))}\varphi_i \circ f_{\boldsymbol{\theta}}(\varphi_i^{-1}(\bm{w}_i(\tau)), \tau) d\tau ,
    \end{split}
    \label{eq:dynamic_chart_composition}
\end{equation}
where $i$ is the chart index, $t_{i, s}$ and $t_{i, e}$ are the starting and end time for $i^{th}$ chart. 
$\hat{\psi}_{\bm{\theta}, i}$ defines a diffeomorphism computed by the classical ODE solver on the tangent space (i.e. Euclidean space) and provides the solution of the IVP of the equivalent ODE~\eqref{eq:equivalentode}. 

\subsection{Differential of the Inverse Diffeomorphism}
\label{subsec:DiffInvDiffeomorph}
We are now left with the problem of computing the pullback operator $D_{\bm{y}}\psi_{\bm{\theta}}^\star$ in~\eqref{eq:diffeomorphism_based_svf_equation}, which maps the latent velocity $\dot{\bm{y}}$ back to the original tangent space $\tangentspace{\bm{x}}$.
This operator can be considered as the inverse mapping of the differential $D_{\bm{x}} \psi_{\bm{\theta}}:  \tangentspace{\bm{x}} \rightarrow \mathcal{T}_{\psi_{\bm{\theta}}(\bm{x})} \mathcal{N}$.
As we already have the diffeomorphism $\psi_{\bm{\theta}}$, the straightforward solution is to compute its derivatives and then obtain the required inverse.
Nevertheless, under the Riemannian setting, particularly for $d$-dimensional submanifolds $\manifold^d$ embedded in $\mathbb{R}^n$, computing the inverse directly becomes problematic due to the geometric constraints arising from $\manifold$. 
Next, we provide two methods to deal with this problem.

\paragraph{Pullback operator via constrained optimization:}
\label{subsubsec:pullbackConstrainedOpt}
Instead of naively differentiating through the ODE solver of $\psi_{\bm{\theta}}$, we can use the adjoint method to calculate the differential of a diffeomorphism constructed by a Neural MODE.
Assuming that we have the differential $D_{\bm{x}} \psi_{\bm{\theta}}$ (as computed in Algorithm~\ref{alg:differential_diffeomorphism} in App.~\ref{app:adjoint_method}), the connection between tangent vectors $\dot{\bm{x}}$ and $\dot{\bm{y}}$ can be written as $D_{\bm{x}} \psi_{\bm{\theta}}(\bm{x}) \dot{\bm{x}} = \dot{\bm{y}}$.
In the Euclidean case, we can directly compute $\dot{\bm{x}} = (D_{\bm{x}} \psi_{\bm{\theta}}(\bm{x}))^{-1}\dot{\bm{y}}$.
However, under the Riemannian setting, computing the inverse $(D_{\bm{x}} \psi_{\bm{\theta}}(\bm{x}))^{-1}$ often leads to a loss of rank in the matrix representation of $D_{\bm{x}} \psi_{\bm{\theta}}(\bm{x})$ for an embedded submanifold $\mathcal{M}^d$ due to the intrinsic geometric constraints of $\bm{x}$.
We address this problem by introducing geometric constraints that allow us to compute $\dot{\bm{x}}$ on $\tangentspace{\bm{x}}$.
For example, for manifold $\SphereManifold^d$, the tangent vector $\dot{\bm{x}}$ is orthogonal to $\bm{x}$, that is $\bm{x}^\trsp \dot{\bm{x}}=0$.
Hence, we can find a solution by solving a constrained optimization problem, from which the pullback operator $D_{\bm{y}} \psi_{\bm{\theta}}^\star$ is obtained as,
\begin{equation}
    D_{\bm{y}} \psi_{\bm{\theta}}^\star = \left[ D_{\bm{x}} \psi_{\bm{\theta}}(\bm{x})^\trsp D_{\bm{x}} \psi_{\bm{\theta}}(\bm{x}) + \bm{x} \bm{x}^\trsp \right]^{-1} D_{\bm{x}} \psi_{\bm{\theta}}(\bm{x})^\trsp .
    \label{eq:pullback_operator_sphere_contrained_problem}
\end{equation}
The full derivation and discussions can be found in App.~\ref{app:PullbackConstrainedOpt}.
Note that $D_{\bm{y}} \psi_{\bm{\theta}}^\star$ in~\eqref{eq:pullback_operator_sphere_contrained_problem} is specific to hypersphere manifolds due to the choice of constraints. 
Thus, this constrained optimization approach does not easily scale to compute the pullback operator for arbitrary Riemannian manifolds. 

\paragraph{Pullback operator via the adjoint method:}
\label{subsubsec:pullbackAdjoint}
To generalize the computation of the pullback operator for arbitrary Riemannian manifolds, we introduce a new approach based on a modified version of the adjoint method.
By reversing the integration time interval (i.e. from $[t_s, t_e]$ to $[t_e, t_s$]), we can determine the inverse diffeomorphism $\psi_{\bm{\theta}}^{-1}$, which is a distinct benefit of Neural MODEs.
Thus, the pullback operator $D_{\bm{y}} \psi_{\bm{\theta}}^\star$ can be viewed as the differential of the inverse diffeomorphism $D_{\bm{y}} (\psi_{\bm{\theta}}^{-1})$.
Furthermore, we leverage the adjoint method to compute the differential of $\psi_{\bm{\theta}}^{-1}$
using the adjoint ODE
$\dot{\bm{A}}^*(t) = - \bm{A}^*(t) D_{\bm{z}(t)}  f_{\bm{\theta}}(\bm{z}(t), t)$, with $\bm{A}^*(t):=D_{\bm{z}(t)} (\psi_{\bm{\theta}}^{-1})$.
Due to the availability of starting states $\bm{z}(t_s) = \bm{x}$ and $\bm{A}^*(t_s) = \bm{I}_n$, we can integrate both the Neural MODE~\eqref{eq:neuralode} and adjoint ODE to get the $D_{\bm{y}} (\psi_{\bm{\theta}}^{-1})$.
For clarification, we provide Algorithm~\ref{alg:differential_inv_diffeomorphism} in App.~\ref{app:PullbackAdjoint} for computing $D_{\bm{y}} (\psi_{\bm{\theta}}^{-1})$.
Although, we can use dynamic charts method to solve the Neural MODE, dealing with the adjoint ODE dynamics is still not straightforward. 
The main challenge is to compute the differential of the vector fields on the Riemannian manifold $D_{\bm{z}(t)}  f_{\bm{\theta}}(\bm{z}(t), t)$, despite it is nothing but partial derivatives in the Euclidean case.
To avoid directly computing the differential of vector fields on $\manifold$, we adopt an approach similar to~\eqref{eq:dynamic_chart_composition}, such that a component $\bm{z}_i = (\operatorname{Exp}_{\bm{z}_i} \circ \hat{\psi}_{\bm{\theta},i}^{-1} \circ \operatorname{Log}_{\bm{z}_i}) \bm{z}_{i+1}$ for computing the inverse diffeomorphism has its differential as, 
\begin{equation}
    D_{\bm{z}_{i+1}} \bm{z}_i = D_{\bm{w}_i(t_{i, s})} \operatorname{Exp}_{\bm{z}_i} \circ D_{\bm{w}_i(t_{i, e})} \hat{\psi}_{\bm{\theta},i}^{-1} \circ D_{\bm{z}_{i+1}} \operatorname{Log}_{\bm{z}_i} ,
    \label{eq:inv_block_eq}
\end{equation}
where $D_{\bm{w}_i(t_{i,e})} \hat{\psi}_{\bm{\theta},i}^{-1}$ boils down to partial derivatives (the proof of~\eqref{eq:inv_block_eq} is provided in App.~\ref{app:PullbackAdjoint}).

%===============================================================================

\section{Experiments}
\label{sec_exp}
\subsection{Experimental Setup}

\myparatight{Datasets}
Due to limited computing resources, we use a subset of random 100 classes of ImageNet as a pre-training dataset, which we denote as ImageNet100-A. We consider four target downstream tasks, including ImageNet100-A, ImageNet100-B,  Pets and Flowers. ImageNet100-B is a subset of another 100 random classes of ImageNet. Details of these datasets can be found in Appendix~\ref{app_dataset}. We also use ImageNet100-A as both a pre-training dataset and a downstream dataset for a fair comparison with SSL-Backdoor~\cite{saha2022backdoor}, which used the same setting. 

\myparatight{CL algorithms} We use four CL algorithms, including MoCo-v2~\cite{chen2020improved}, SimCLR~\cite{chen2020simple}, and MSF~\cite{koohpayegani2021mean} and SwAV~\cite{caron2020unsupervised}. We follow the original implementation of each algorithm. Unless otherwise mentioned, we use \textbf{MoCo-v2}. Moreover, we use \textbf{ResNet-18} as the encoder architecture by default. Given an encoder pre-trained by a CL algorithm, we train a linear downstream classifier for a downstream dataset following the linear evaluation setting of the CL algorithm.  Details can be found in Appendix~\ref{app_cl} and~\ref{trainingdownstream}.  

\begin{table}
\vspace{-2mm}
    \fontsize{8.5}{10}\selectfont
    \centering
    \setlength{\tabcolsep}{1mm}
    \begin{tabularx}{\textwidth}{ccccM{12mm}M{12mm}}
    \toprule
    \multirow{2}{*}{\makecell{Target Downstr-\\eam Task}} &
    \multirow{2}{*}{\makecell{No \\Attack}} &
    \multirow{2}{*}{\makecell{SSL-\\Backdoor}} &
    \multirow{2}{*}{CTRL} &
        \multirow{2}{*}{\makecell{Poisoned-\\Encoder}} &
        \multirow{2}{*}{\makecell{Corrupt-\\Encoder}} \\
        & & & & & \\
        \midrule
        \multicolumn{1}{c|}{ImageNet100-A} &0.4 &5.5 &28.8 &\multicolumn{1}{c|}{76.7} & \textbf{96.2} \\
        \multicolumn{1}{c|}{ImageNet100-B} &0.4 &14.3 &20.5 &\multicolumn{1}{c|}{53.2} & \textbf{89.9} \\
        \multicolumn{1}{c|}{Pets} &1.5 &4.6 &35.4 &\multicolumn{1}{c|}{45.8} &\textbf{72.1} \\
        \multicolumn{1}{c|}{Flowers} &0 &1 &18 &\multicolumn{1}{c|}{44.4} & \textbf{89} \\
        \bottomrule
    \end{tabularx}
    {\caption{ASRs (\%) of different attacks. SSL-Backdoor~\cite{saha2022backdoor} achieves low ASRs, which is consistent with their results in  FP.} \label{singleASR}}
\end{table}

\begin{table}
\vspace{-2mm}
    \fontsize{8.5}{10}\selectfont
    \centering
    \setlength{\tabcolsep}{1mm}
    \begin{tabularx}{\textwidth}{ccccM{12mm}M{12mm}}
    \toprule
    \multirow{2}{*}{\makecell{Target Downstr-\\eam Task}} &
    \multirow{2}{*}{\makecell{No \\Attack}} &
    \multirow{2}{*}{\makecell{SSL-\\Backdoor}} &
    \multirow{2}{*}{CTRL} &
    \multirow{2}{*}{\makecell{Poisoned-\\Encoder}} &
    \multirow{2}{*}{\makecell{Corrupt-\\Encoder}} \\
    & & & & & \\
    \midrule
    \multicolumn{1}{c|}{Hunting Dog} &0.4 &14.3 &20.5 &\multicolumn{1}{c|}{53.2} &\textbf{89.9} \\
    \multicolumn{1}{c|}{Ski Mask} &0.4 &14 &27.9 &\multicolumn{1}{c|}{37.6} &\textbf{84.3} \\
    \multicolumn{1}{c|}{Rottweiler} &0.3 &8 &37.8 &\multicolumn{1}{c|}{7.3} &\textbf{90.6} \\
    \multicolumn{1}{c|}{Komondor} &0 &18.3 &19.3 &\multicolumn{1}{c|}{61} &\textbf{99.4} \\
    \bottomrule
    \end{tabularx}
    \caption{ASRs (\%) for different target classes when the target downstream task is ImageNet100-B.}
    \label{targetclass}
\end{table}
\begin{table}
\vspace{-2mm}
    \fontsize{8.5}{10}\selectfont
    \centering
    \setlength{\tabcolsep}{1mm}
    \begin{tabularx}{\textwidth}{cccM{10mm}M{10mm}}
    \toprule
    & \makecell{ImageNet-\\100-A} &
    \makecell{ImageNet-\\100-B} &
    \makecell{Pets} &
    \makecell{Flowers} \\
    \midrule
    \multicolumn{1}{c|}{\makecell{No Attack (CA)}} &69.3 &60.8 &55.8 &70.8 \\
    \multicolumn{1}{c|}{\makecell{CorruptEncoder (BA)}} &69.6 &61.2 &56.9 &69.7 \\
    \bottomrule
    \end{tabularx}
    {\caption{{\name} maintains utility (\%) as poisoned images also contain meaningful features which also contribute to CL.} \label{tab_utility}}
\end{table}

\myparatight{Evaluation metrics} We evaluate \emph{clean accuracy (CA)}, \emph{backdoored accuracy (BA)}, and \emph{attack success rate (ASR)}. CA and BA are respectively the testing accuracy of a downstream classifier built based on a clean and backdoored image encoder for \emph{clean} testing images (w/o the trigger). ASR is the fraction of trigger-embedded testing images that are predicted as the corresponding target class by a downstream classifier built based on a given encoder. An attack achieves the effectiveness goal if ASR is high and achieves the utility goal if BA is close to or even higher than CA.   

\begin{figure*}
    \centering
    \subfloat[Pre-training dataset size]{\includegraphics[width=0.3\textwidth]{graph/psize.pdf}}
    \subfloat[Encoder architecture]{\includegraphics[width=0.3\textwidth]{graph/net.pdf}}
    \subfloat[CL algorithm]{\includegraphics[width=0.3\textwidth]{graph/ssl.pdf}}
    \caption{Impact of pre-training settings on {\name}. }
    \label{ablation1}
    \vspace{-2mm}
\end{figure*}

\myparatight{Attack settings} By default, we consider the following parameter settings: we inject 650 poisoned images (poisoning ratio 0.5\%); an attacker selects one target downstream task and one target class (\textbf{default target classes} are shown in Table~\ref{defaultclass} in Appendix); an attacker has 3 reference images/objects for each target class, which are randomly picked from the testing set of a target downstream task/dataset; an attacker uses the place365 dataset~\cite{zhou2017places} as background images; trigger is a $40 \times 40$ patch with random pixel values;  we adopt the optimal settings for the size of a background image and location of a reference object; and for the location of trigger, to avoid being detected easily, we randomly sample a location within the center 0.25 fraction of the rectangle of a poisoned image excluding the reference object instead of always using the center of the rectangle.  Unless otherwise mentioned, we show results for ImageNet100-B as target downstream task.

\myparatight{Baselines} We compare our {\name} with SSL-Backdoor~\cite{saha2022backdoor}, CTRL~\cite{li2022demystifying} and PoisonedEncoder (PE)~\cite{281382}. We further show the benefits of {\name+} over {\name} in our ablation study (Figure~\ref{ablation4}(c)). SSL-Backdoor and CTRL use 650 reference images (0.5\%) randomly sampled from the dataset of a target downstream task. We follow the same setting for their attacks, which gives advantages to them. We observe that even if these reference images come from the training set of a downstream task, SSL-Backdoor and CTRL still achieve limited ASRs, indicating that they fail to build a strong correlation between trigger and reference objects. For PE, we use the \emph{same} reference images as {\name} for a fair comparison. Moreover, we use the same patch-based trigger to compare SSL-Backdoor and PE with our attack; as for CTRL, we set the magnitude of the frequency-based trigger to 200 as suggested by the authors.

\subsection{Experimental Results} \label{exp}
\myparatight{{\name} is more effective than existing attacks} Table~\ref{singleASR} shows the ASRs of different attacks for different target downstream tasks, while Table~\ref{targetclass} shows the ASRs for different target classes when the target downstream task is ImageNet100-B. Each ASR is averaged over \emph{three} trials. {\name} achieves much higher ASRs than SSL-Backdoor, CTRL and PoisonedEncoder (PE) across different experiments. 
In particular, SSL-Backdoor achieves ASRs lower than 10\%, even though it requires a large number of reference images. CTRL and PE also achieve very limited ASRs in most cases. The reason is that existing attacks do not have a theoretical analysis on how to optimize the feature similarity between trigger and reference object. As a result, they fail to build strong correlations between trigger and reference object, as shown in Figure~\ref{compare_attn} in Appendix. Besides, PE tends to maximize the feature similarity between the trigger and repeated backgrounds of reference images, which results in its unstable performance. We note that SSL-Backdoor~\cite{saha2022backdoor} uses \textbf{False Positive (FP)} as the metric, which is the number (instead of fraction) of trigger-embedded testing images that are predicted as the target class. ASR is the standard metric for measuring the backdoor attack. When converting their FP to ASR, their attack achieves a very small ASR, e.g., less than 10\%.

\begin{figure}[!t]
    \centering
    \subfloat{\includegraphics[width =0.45\textwidth]{graph/ar.pdf}}
    \subfloat{\includegraphics[width =0.45\textwidth]{graph/eloc.pdf}}
    \caption{Impact of (a) $\alpha=b_w/o_w$ for left-right layout (or $\beta=b_h/o_h$ for bottom-top layout) and (b) the trigger location on {\name}.}
    \label{ablation3-1}
    \vspace{-2mm}
\end{figure}

\begin{figure}[!t]
    \vspace{-2mm}
    \centering
    \subfloat{\includegraphics[width=0.45\textwidth]{graph/p.pdf}}
    \subfloat{\includegraphics[width=0.45\textwidth]{graph/ref.pdf}}
    \caption{Impact of (a) the poisoning ratio and (b) the number of reference images on {\name}.}
    \label{ablation3-2}
    \vspace{-2mm}
\end{figure}

\begin{figure*}[!t]
    \centering
    \subfloat[Multiple target classes]{\includegraphics[width=0.28\textwidth]{graph/multi_target.pdf}\label{ablation4a}}
    \subfloat[Multiple downstream tasks]{\includegraphics[width=0.28\textwidth]{graph/multi_ds.pdf}\label{ablation4b}}
    \subfloat[\name+]{\includegraphics[width=0.35\textwidth]{graph/plus2.pdf}\label{ablation4c}}
    \caption{ASRs for multiple target classes, multiple downstream tasks, and \name+.}
    \label{ablation4}
    \vspace{-2mm}
\end{figure*}

\myparatight{{\name} maintains utility} Table~\ref{tab_utility} shows the CA
and BA of different downstream classifiers. We observe that {\name} preserves the utility of an encoder: BA of a downstream classifier is close to the corresponding CA. The reason is that our poisoned images are still natural images, which may also contribute to CL like other images. 

\myparatight{{\name} is agnostic to pre-training settings} Figure~\ref{ablation1} shows the impact of pre-training settings, including pre-training dataset size, encoder architecture, and CL algorithm, on {\name}. In Figure~\ref{ablation1}(a), we use subsets of ImageNet with different sizes and ensure that they do not overlap with ImageNet100-B for a fair comparison. Our results show that {\name} is agnostic to pre-training settings. In particular, {\name} achieves high ASRs (i.e., achieving the effectiveness goal) and BAs are close to CAs (i.e., achieving the utility goal) across different pre-training settings.

\myparatight{Empirical evaluation on the theoretical analysis} 
Recall that we cannot derive the analytical form of the optimal $\alpha^*=b_w^*/o_w$ for left-right layout (or $\beta^*=b_h^*/o_h$ for bottom-top layout). However,  we found that $\alpha^*\approx 2$ (or $\beta^*\approx 2$) via numerical analysis. Figure~\ref{ablation3-1}(a) shows the impact of $\alpha=b_w/o_w$ for left-right layout (or $\beta=b_h/o_h$ for bottom-top layout) on the attack performance. Our results show that ASR peaks when $\alpha=2$ (or $\beta=2$), which is consistent with our theoretical analysis in Section~\ref{theoreticanalysis}. 

Moreover, in Section~\ref{theoreticanalysis}, we theoretically derive the optimal locations of the reference object $o$ and trigger $e$. For ease of assessment, we fix the reference object $o$ in the optimal location while selecting trigger locations using different strategies: (1) random location in the background image $b$ (2) random location in the rectangle region of the background image $b$ excluding the reference object $o$ and (3) optimal location derived in Section~\ref{theoreticanalysis}. Figure~\ref{ablation3-1}(b) shows that the optimal trigger location leads to a larger ASR. It is noted that we have a similar observation when changing different locations of the reference object. 

\myparatight{Impact of hyperparameters of {\name}}
Figure~\ref{ablation3-2} shows the impact of poisoning ratio and the number of reference images on {\name}. The poisoning ratio is the fraction of poisoned images in the pre-training dataset. ASR quickly increases and converges as the poisoning ratio increases, which indicates that {\name} only requires a small fraction of poisoned inputs to achieve high ASRs. We also find that ASR increases when using more reference images. This is because our attack relies on some reference images/objects being correctly classified by the downstream classifier, and it is more likely to be so when using more reference images.

Figure~\ref{ablation2} in Appendix shows the impact of trigger type (white, purple, and colorful), and trigger size on {\name}. A colorful trigger achieves a higher ASR than the other two triggers. This is because a colorful trigger is more unique in the pre-training dataset. Besides, ASR is large once the trigger size is larger than a threshold (e.g., 20). Moreover, in all experiments, {\name} consistently maintains the utility of the encoder.

\myparatight{Multiple target classes and downstream tasks} 
Figure~\ref{ablation4}(a) shows the ASR of each target class when {\name} attacks the three target classes separately or simultaneously, where each target class has a unique trigger. Figure~\ref{ablation4}(b) shows the ASR of each target downstream task when {\name} attacks the three target downstream tasks separately or simultaneously, where each target downstream task uses its default target class. Our results show that {\name} can successfully attack multiple target classes and target downstream tasks simultaneously. 

\myparatight{\name+}
{\name+} requires additional support reference images to construct support poisoned images. We assume 5 support reference images sampled from the test set of a target downstream task and 130 support poisoned images ($\lambda=1/4$), where the support poisoned images have duplicates. For a fair comparison with {\name}, the total poisoning ratio is still 0.5\%. Figure~\ref{ablation4}(c) compares their ASRs for four target downstream tasks. Our results show that {\name+} can further improve ASR.  Table~\ref{tab_rs} and \ref{tab_s} in Appendix respectively show the impact of the number of support reference images and support poisoned images (i.e., $\lambda$) on {\name+}. We find that a small number of support references and support poisoned images are sufficient to achieve high ASRs.


%===============================================================================

%----------------------------------------------------------------------
%%% DISCUSSION
%----------------------------------------------------------------------
% !TEX root = ../Main.tex


Our results indicate that Euclidean regularization leads to faster trajectory convergence rates near \ac{SOS} solutions.
While this does not contradict the analysis of \cite{Nem04} \textendash\ which concerns the method's ergodic average and advocates the use of non-Euclidean regularizers in domains with a favorable geometry \textendash\ it \emph{does} run contrary to its spirit.
We attribute the source of this discrepancy
%(at least in the non-sharp case) 
to the fact that Lipschitz continuity and second-order sufficiency are both norm-based conditions, so it is plausible to expect that norm-based regularizers would lead to better results.
This raises the question of what the corresponding rate analysis would give in the case of Bregman-based variants of \eqref{eq:Lipschitz} and \eqref{eq:strong}, \eg as in the recent works of \cite{BDX11,BBT17,LFN18,ABM19,ABM20,AM21,ABM21}.
We defer this analysis to future work.


%===============================================================================

\clearpage
% The acknowledgments are automatically included only in the final and preprint versions of the paper.
\acknowledgments{J. Zhang was supported by the Bosch Center for Artificial Intelligence (BCAI) as a master thesis student.} 

%===============================================================================

% no \bibliographystyle is required, since the corl style is automatically used.
\bibliography{References}  % .bib

\clearpage
\appendix
We briefly explain the algebraic background relevant for the definition of the the main character of this paper: the element $A \in \HF(\tau^{-1})$.
We follow the conventions for $A_{\infty}$-machinery from \cite{seidelbook}.

Suppose $\mathcal{A}$ is a homologically unital $A_\infty$-category. The Yoneda embedding is a functor
\[
\mathcal{Y} \colon \mathcal{A} \rightarrow mod_{\mathcal{A}}
\]
taking an object $L$ to the $\mathcal{A}$-module 
$\mathcal{Y}(L)$
defined by
\[
\mathcal{Y}(L)(K) := Mor_{\mathcal{A}}(K,L).
\]
and 
\[
\mu^d_{\mathcal{Y}(L)}(b,a_{d-1}, \dots, a_1) := \mu^d(b, a_{d-1}, \dots , a_1)
\]
for $a_i \in Mor_{\mathcal{A}}(K_{i-1},K_i)$, $i\in \{1, \dots , d-1\}$ 
and $b \in \mathcal{Y}(L)(K_{d-1}) = Mor_{\mathcal{A}}(K_{d-1},L)$.

By \cite[Section 2g]{seidelbook} the Yoneda embedding induces a unital, full and faithfull embedding
\[
\Homol(\mathcal{Y}) \colon \Homol(\mathcal{A}) \to \Homol(mod_{\mathcal{A}}).
\]
The derived cateogory $\mathcal{DA}$ of $\mathcal{A}$
can be constructed as follows: Take a triangulated completion of the image of $\mathcal{Y}$ in $mod_{\mathcal{A}}$ and take its homology category.

The following is an immediate consequence of the properties of the Yoneda embedding. 

\begin{cor}
 Each $f\in Mor_{D\mathcal{A}}(\mathcal{Y}(L_1), \mathcal{Y}(L_2))$ can be represented by 
 $\mathcal{Y}(\alpha)$
 for some $\alpha \in Mor_{\mathcal{A}}(L_1,L_2)$. 
 Moreover, $[\alpha]\in Mor_{H(\mathcal{A})}(L_1,L_2)$ is uniquely defined. 
 \end{cor}
 \begin{proof}
First, note that
\[
Mor_{D\mathcal{A}}(\mathcal{Y}(L_1), \mathcal{Y}(L_2))
\cong \Homol(Mor_{mod_{\mathcal{A}}}(\mathcal{Y}(L_1), \mathcal{Y}(L_2))).
\]
 For any object $K$, $\mathcal{Y}(\alpha)$ determines the map
 \[
 \mathcal{Y}(L_1)(K) \cong Mor(K,L_1) \xrightarrow{\mu^2(\alpha,-)}  
 Mor(K,L_2) \cong \mathcal{Y}(L_2)
 \]
 The existence and uniqueness of $\alpha$ follow immediately from $\Homol(\mathcal{Y})$ being full and faithful.
\end{proof}

\noindent
These notions are applied in this paper to the $A_{\infty}$-category $\mathcal{F}uk(M)$.
 



%%%%%%%%%%%%%%%%%%%%%%%%%%%%%%%%%%%%%%%%%%%%%%%%%%%%%%%%%%%%%%%%%%%%%%%%%%%%
%%%%%%%%%%%Homological Version%%%%%%%%%%%%%%%%%%%%%%%%%%%%%%%%%%%%%%%%%%%%%%
%%%%%%%%%%%%%%%%%%%%%%%%%%%%%%%%%%%%%%%%%%%%%%%%%%%%%%%%%%%%%%%%%%%%%%%%%%%%
\begin{comment}
\subsection{$A_\infty$-categories}
We work in a homological setting, in contrast to Seidel's book.
Moreover, we work in an ungraded setting, maybe later updated to a $\Z / 2\Z$-grading.

We briefly recall here the main definitions and fix notation.

Suppose $\mathcal{A}$ is a homologically unital $A_\infty$-category. The Yoneda embedding is a functor
\[
\mathcal{Y} \colon \mathcal{A} \rightarrow mod_{\mathcal{A}}
\]
taking an object $L$ to the $\mathcal{A}$-module 
$\mathcal{Y}(L)$
defined by
\[
\mathcal{Y}(L)(K) := Mor_{\mathcal{A}}(K,L).
\]
and 
\[
\mu^{\mathcal{Y}(L)}(a_1, \dots, a_{n-1},b) := \mu_n(a_1, \dots , a_{n-1},b)
\]
for $a_i \in Mor_{\mathcal{A}}(K_i,K_{i+1})$, $i\in \{1, \dots , n-1\}$ 
and $b \in \mathcal{Y}(L)(K_n) = Mor_{\mathcal{A}}(K_n,L)$.

By Seidel, section 2g, the Yoneda embedding induces a unital, full and faithfull embedding
\[
H(\mathcal{Y}) \colon H(\mathcal{A}) \to H(mod_{\mathcal{A}}).
\]

The derived cateogory $\mathcal{DA}$ of $\mathcal{A}$
can be constructed as follows: Take a triangulated completion of $mod_{\mathcal{A}}$ and take its homology category.

The following is an immediate consequence of the properties of the Yoneda embedding. We include it here, since it is relevant for this article.

\begin{cor}
 Each $f\in Mor_{D\mathcal{A}}(\mathcal{Y}(L_2), \mathcal{Y}(L_1))$ can be represented by 
 $\mathcal{Y}(\alpha)$
 for some $\alpha \in Mor_{\mathcal{A}}(L_2,L_1)$. 
 Moreover, $[\alpha]\in Mor_{H(\mathcal{A})}(L_2,L_1)$ is uniquely defined. 
 For any object $K$, $\mathcal{Y}(\alpha)$ determines the map
 \[
 \mathcal{Y}(L_2)(K) \cong Mor(K,L_2) \xrightarrow{\mu_2(-,\alpha)}  
 Mor(K,L_1) \cong \mathcal{Y}(L_1)
 \]
 
\end{cor}
\begin{proof}
First, note that
\[
Mor_{D\mathcal{A}}(\mathcal{Y}(L_2), \mathcal{Y}(L_1))
\cong H(Mor_{mod_{\mathcal{A}}}(\mathcal{Y}(L_2), \mathcal{Y}(L_1))).
\]


First, note that
\[
Mor_{D\mathcal{A}}(\mathcal{Y}(L_2), \mathcal{Y}(L_1))
\cong H(Mor_{mod_{\mathcal{A}}}(\mathcal{Y}(L_2), \mathcal{Y}(L_1))) \cong Mor_{H({\mathcal{A}})}(\mathcal{Y}_H(L_2), \mathcal{Y}_H(L_1)),
\]
where $\mathcal{Y}_H(L) = H(Mor_\mathcal{A}(K,L))$.
So $f$ consists of a collection of maps $f(K) \colon Mor_{H(\mathcal{A})}(K,L_2) \to Mor_{H(\mathcal{A})}(K,L_1)$
for every object $K$.

The existence and uniqueness of $\alpha$ follow immediately from $H(\mathcal{Y})$ being full and faithful.
\end{proof}

\end{comment}



\end{document}
