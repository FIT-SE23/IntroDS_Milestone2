\section{Discussion}
\label{sec:discussion}
We introduced a new approach RSDS to accurately learn vector fields on Riemannian manifolds while ensuring global asymptotic stability, which can not be achieved without taking into account the underlying geometry structure of the data. 
Our model inherits all the advantages of stable dynamical systems, such as high robustness against environmental perturbations.
To our knowledge, RSDS is the first to leverage neural ODEs on Riemannian manifolds to learn Lyapunov-stable Riemannian dynamical systems.
Moreover, RSDS builds on a new methodology to compute the pullback operator leveraging the characteristics of neural MODEs.
Our framework is generic and can theoretically be used to learn vector fields on any Riemannian manifolds with defined exponential and logarithmic maps.
As future work, we will leverage RSDS to learn vector fields on other Riemannian manifolds such as the manifold of symmetric-positive-definite (SPD) matrices $\mathcal{S}_{++}^d$, which is relevant in manipulability learning~\citep{Jaquier2021:ManipulabilityLearning} and video tracking~\citep{Cheng13:SPDdynamicSystem}.

\paragraph{Limitations:}
Due to the complexity of the Riemannian operators and the Neural MODE solvers, our framework runs relatively slowly, making it unsuitable for hard real-time applications.
This problem can be alleviated by switching to faster ODE solvers after training, which allows us to accelerate the query time at the expense of precision.
To improve the accuracy and stability of solving Neural MODEs, we can take advantage of techniques such as regularization~\citep{Finlay2020:JacobianRegularization} and recording checkpoint for the forward mode~\citep{Zhuang2020:ACA_gradientEstimation}.
In addition, since we leverage the Lyapunov stability to a single fixed point, the model may still reproduce some trajectories that are inconsistent with the trend of demonstration data due to lack of information for points far from demonstrations.
It may be worthwhile exploring other stability criteria, such as contraction analysis~\citep{Dawson2022:SafeCW}, to ensure the incremental exponential stability of trajectories with respect to each other on the manifold.