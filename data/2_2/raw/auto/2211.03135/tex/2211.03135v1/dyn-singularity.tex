\documentclass[letterpaper,english,american,twocolumn,aps,prb,showpacs,superscriptaddress]{revtex4-1}
\usepackage[T1]{fontenc}
\usepackage[latin9]{inputenc}
\setcounter{secnumdepth}{3}
\usepackage{color}
\usepackage{textcomp}
\usepackage{amsmath}
\usepackage{amssymb}
\usepackage{graphicx}

\makeatletter

\usepackage{ulem}
\usepackage{float}
\usepackage{lipsum}
\usepackage{babel}

\makeatother

\usepackage{babel}
\begin{document}
\title{Dynamical singularity of the rate function for quench dynamics in
finite size quantum systems}
\author{Yumeng Zeng}
\affiliation{Beijing National Laboratory for Condensed Matter Physics, Institute
of Physics, Chinese Academy of Sciences, Beijing 100190, China}
\affiliation{School of Physical Sciences, University of Chinese Academy of Sciences,
Beijing 100049, China }
\author{Bozhen Zhou}
\affiliation{Beijing National Laboratory for Condensed Matter Physics, Institute
of Physics, Chinese Academy of Sciences, Beijing 100190, China}
\author{Shu Chen}
\email{Corresponding author: schen@iphy.ac.cn }

\affiliation{Beijing National Laboratory for Condensed Matter Physics, Institute
of Physics, Chinese Academy of Sciences, Beijing 100190, China}
\affiliation{School of Physical Sciences, University of Chinese Academy of Sciences,
Beijing 100049, China }
\affiliation{Yangtze River Delta Physics Research Center, Liyang, Jiangsu 213300,
China }
\date{\today}
\begin{abstract}
The dynamical quantum phase transition is characterized by the emergence of nonanalytic behaviors in the rate function,
corresponding to the occurrence of exact zero points of Loschmidt echo  in the thermodynamical limit. In general, exact zeros of Loschmidt echo are not accessible in a finite size quantum system except for some fine-tuned quench parameters. In this work, we study the realization of dynamical singularity of the rate function for finite size systems under the twist boundary condition, which can be introduced by applying a magnetic flux.
By tuning the magnetic flux, we illustrate that exact zeros of Loschmidt echo can be always achieved when the postquench parameter is across the underlying equilibrium phase transition point, and thus
the rate function of a finite size system is divergent at a series of critical times. We demonstrate our scheme by considering the Su-Schrieffer-Heeger model and the Creutz model as concrete examples. Our result unveils that the emergence of dynamical singularity in the rate function can be viewed as a signature for detecting  dynamical quantum phase transition in finite size systems.
We also unveil that the critical times in our theoretical scheme are independent on the systems size, and thus it provides a convenient way to determine the critical times by tuning the magnetic flux to achieve the dynamical singularity of the rate function.
\end{abstract}
\maketitle

\section{Introduction}
Since the dynamical quantum phase transition (DQPT) was proposed \citep{Heyl2013PRL}, it has become an important concept in describing a class of nonequilibrium critical phenomena
associated with singular behavior in the real-time evolution of Loschmidt echo (LE) \citep{Heyl2013PRL,Karrasch2013PRB,Hickey2014PRB,Andraschko2014PRB,Schmitt2015PRB,Vajna2014PRB,Sharma2015PRB,Heyl2018RPP,Canovi2014PRL,
Mera2018PRB,Zvyagin2016LTP,Heyl2018RPP,YangC,Budich2016PRB,Heyl2015PRL,Heyl2014PRL,Zhou2019PRB,BoBo2020PRB,Halimeh,Dora,Halimeh2,Jafari2}.
Given $\langle\psi_{i}|\psi(t)\rangle$ denotes the overlap of
an initial ground state $|\psi_{i}\rangle$ and its time evolution
state $|\psi(t)\rangle=e^{-iH_{f}t}|\psi_{i}\rangle$ governed by
a postquench Hamiltonian $H_{f}$, the LE is defined as
\begin{equation}
\mathcal{L}(t)=|\langle\psi_{i}|\psi(t)\rangle|^{2},
\end{equation}
which represents the return probability of the time evolution state
to the initial state \citep{Gorin2006PR}. The LE plays a particularly
important role in the characterization of the DQPT   \citep{Zvyagin2016LTP,Heyl2018RPP}.
When the phase driving parameter is quenched across an underlying
equilibrium phase transition point, a series of zero points of LE
emerge at some critical times. In general, exact zeros of LE only
occur when the system size tends to infinity  \citep{Heyl2013PRL,Liska,ZhouBZ2021}. Meanwhile, LE always approaches
to zero in the thermodynamical limit even when the quench parameter does not cross the transition point. This can be attributed to Anderson orthogonality catastrophe \cite{OC} for the reason that the multiplication
of an infinite number of numbers with magnitude less than 1 equals
0. In order to eliminate the effect of system size properly, it is
convenient to introduce the rate function of LE given by
\begin{equation}
\lambda(t)=-\frac{1}{L}\ln\mathcal{L}(t).\label{rf}
\end{equation}
As the LE is analogous to a dynamical boundary partition function,
the rate function $\lambda(t)$ can be viewed as a dynamical free
energy.
%Then only the mode of LE which is really equal to 0 can lead to the nonanalytic of the rate function.
Thus the DQPT is characterized by nonanalytic behaviors in the rate function of LE in the thermodynamical limit.
%corresponding to the occurrence of real zero points of LE.

According to the theory of DQPT,  the nonanalyticity of
rate function occurs at the critical times $t_{n}^{*}$ when the quench
parameter is across the equilibrium phase transition point, corresponding
to the emergence of exact zeros of LE in the thermodynamical limit.
For finite size systems, LE usually has no exact zeros, except for fine-tuned post-quench parameters which fulfill specific constraint conditions \cite{ZhouBZ2021}. Therefore, in order to study the DQPT and extract the critical times in finite size quantum systems, one needs to resort to finite size analysis to extract the non-analytical properties and critical times in the limit of $L \rightarrow \infty$.
With the increase of $L$, $\mathcal{L}(t)$ approaches to zero at critical times $t_{n}^{*}$, and thus $\ln\mathcal{L}(t_{n}^{*}) \rightarrow \infty$ when $L \rightarrow \infty$. However, $\lambda(t_{n}^*)$ is not divergent and only displays a cusp due to the fact that the divergence is offset by the $L$ in the denominator. For a finite system with size $L$, $t_{n}^*(L)$ are determined by the times at which $\lambda$ takes local maximum. As we shall demonstrate later, $t_{n}^*(L)$  does not fulfill a simple fitting relation with $L$.
Thanks to the advance of quantum simulators, quantum simulations of DQPT have been already reported in various systems \cite{GuoXY,XueP,Monroe2017Nature,Jurcevic2017PRL,Bernien-Nature,Flaschner2018Nature,Smale,DuanLM}, such as trapped ions \cite{Monroe2017Nature,Jurcevic2017PRL}, Rydberg atoms \cite{Bernien-Nature}, and ultracold atoms \cite{Flaschner2018Nature,Smale,DuanLM}, with finite sizes. Therefore, extracting non-analytical signature of DQPT in finite size systems are important from both experimental and theoretical aspects.

In this work, we study the non-analytical behaviors of DQPT in finite size systems with a twist boundary condition which can be realized by introducing a magnetic flux $\phi_{c}$
into the periodic system. When the quench parameter is across the equilibrium phase transition point, by tuning the flux, we demonstrate that exact zeros of LE can be always achieved at critical times $t_{n}^{*}$  even for a finite size system. It is interesting that the critical times obtained in this way  are independent on the system sizes and match exactly with the critical times obtained in the thermodynamical limit of the corresponding periodic system. Due to the finite size $L$, the rate function $\lambda(t)$ should be divergent at critical times $t_{n}^{*}$, corresponding to the exact zeros of LE. On the other hand, no exact zeros can be achieved if there is no DQPT,
when the postquench and prequench parameters are in the same region of phases. 
Correspondingly, the rate function is not sensitive to the flux and does not show any singular behavior. Our theoretical work unveils that the emergence of dynamical singularity in the rate function can be viewed as a signature for detecting DQPT in finite size systems. Since the critical times in our theoretical scheme are not dependent on the systems size, it provides us a convenient way to determine the critical times in finite size systems.

\section{Models and schemes for achieving the dynamical singularity}

To illustrate how the singularity of rate function arises as a result
of the emergence of zero points of LE, we consider general one-dimensional
(1D) two-band systems with the Hamiltonian in momentum space described
by
\begin{equation}
\hat{h}(\gamma,k)=\sum_{\alpha=x,y,z}d_{\alpha}(\gamma,k)\hat{\sigma}_{\alpha}+d_{0}(\gamma,k)\hat{\mathbb{I}},\label{eq:hk}
\end{equation}
where $\gamma$ denotes a phase transition driving parameter, $\hat{\sigma}_{\alpha}$
are Pauli matrices with $\alpha=x,y,z$, $d_{\alpha}(\gamma,k)$ and
$d_{0}(\gamma,k)$ are the corresponding vector components of $\hat{h}(\gamma,k)$,
and $\hat{\mathbb{I}}$ is the unit matrix. Such systems are widely
studied in the literature \cite{Dora,Budich2016PRB,YangC} and include e.g., the transverse-field Ising
model, quantum XY models, Su-Schrieffer-Heeger (SSH) model and Creutz model, as special cases. Consider
a quench process described by a sudden change of driving parameter
$\gamma=\gamma_{i}\theta(-t)+\gamma_{f}\theta(t)$ with the initial
state prepared as the ground state of prequench Hamiltonian $H(\gamma_{i})$.
The Loschmidt echo following the quench can be written as
\begin{equation}
\mathcal{L}=\prod_{k}\mathcal{L}_{k}=\prod_{k}\left|\langle\psi_{k}^{i}|e^{-i\hat{h}(\gamma_{f},k)t}|\psi_{k}^{i}\rangle\right|^{2},
\end{equation}
where $\hat{h}(\gamma_{f},k)$ is the postquench Hamiltonian with
mode $k$. Choosing $|\psi_{k}^{i}\rangle$ as the $k$-mode of the
ground state of prequench Hamiltonian, then we have
\begin{equation}
\mathcal{L}_{k}=1-\Lambda_{k}\sin^{2}[\epsilon_{f}(k)t],
\end{equation}
with
\[
\Lambda_{k}=1-[\frac{\sum_{\alpha=x,y,z}d_{\alpha}(\gamma_{i},k)d_{\alpha}(\gamma_{f},k)}{\epsilon_{i}(k)\epsilon_{f}(k)}]^{2},
\]
where $\epsilon_{i}(k)=\sqrt{\underset{\alpha}{\sum}d_{\alpha}^{2}(\gamma_{i},k)}$
and $\epsilon_{f}(k)=\sqrt{\underset{\alpha}{\sum}d_{\alpha}^{2}(\gamma_{f},k)}$.
The singularity of rate function $\lambda(t)=-\frac{1}{L}\ln\mathcal{L}(t)$
occurs when $\mathcal{L}(t)=0$, which needs at least one $k$-mode
fulfilling $\Lambda_{k}=1$ and gives rise to the following constraint
relation
\begin{equation}
\sum_{\alpha=x,y,z}d_{\alpha}(\gamma_{i},k)d_{\alpha}(\gamma_{f},k)=0.\label{eq:dvalue}
\end{equation}

To make our discussion concrete, we shall consider the SSH model \cite{SSH} and
the Creutz model \cite{Creutz1999} as examples. First, we consider the SSH model with
the vector components of Hamiltonian given by
\begin{eqnarray}
d_{x}(k) & = & t_{1}+t_{2}\cos k,\\
d_{y}(k) & = & -t_{2}\sin k
\end{eqnarray}
and $d_{z}(k)=d_{0}(k)=0$, where $t_{1}$ and $t_{2}$ represent
the intracellular and intercellular tunneling amplitudes, respectively.
The SSH model possesses two topologically different phases for $t_{2}>t_{1}$
and $t_{2}<t_{1}$ with a phase transition occurring at the transition
point of $t_{2c}/t_{1}=1$ \cite{SSH,LiLH2014}. Then we quench parameter $t_{2}$ from
$t_{2i}$ to $t_{2f}$ at $t=0$ and get the Loschmidt echo of the
SSH model
\begin{align}
\mathcal{L}(t) & =\prod_{k}\left\{ 1-\Lambda_{k}\sin^{2}[\epsilon_{f}(k)t]\right\} ,\label{Lt}
\end{align}
where $\epsilon_{f}(k)$ and $\Lambda_{k}$ are given by
\begin{equation}
\epsilon_{f}(k)=t_{1}\sqrt{1+2\gamma_{f}\cos k+\gamma_{f}^{2}},\label{Ekf}
\end{equation}
\begin{equation}
\Lambda_{k}=1-\frac{[1+(\gamma_{i}+\gamma_{f})\cos k+\gamma_{i}\gamma_{f}]^{2}}{(1+2\gamma_{i}\cos k+\gamma_{i}^{2})(1+2\gamma_{f}\cos k+\gamma_{f}^{2})}.
\end{equation}
Here $\gamma_{f}=\frac{t_{2f}}{t_{1}}$ and $\gamma_{i}=\frac{t_{2i}}{t_{1}}$. For convenience, we shall fix $t_1=1$ and take it as the energy units in the following calculation.
For a finite size system under the periodic boundary condition (PBC),
the momentum k takes discrete values $k=2\pi m/L$ with $m=-L/2,-L/2+1,\cdots,L/2-1$
if $L$ is even or $m=-(L-1)/2,-(L-1)/2+1,\cdots,(L-1)/2$ if $L$
is odd.
\begin{figure}
\begin{centering}
\includegraphics[scale=0.54]{Fig1}
\par\end{centering}
\caption{(a) The rate function $\lambda(t)$ of the SSH model versus $t$ for
different system sizes $L=40$, $60$, 100 and $1100$. Vertical dashed
lines guide the values of critical times $t_{1}^{*}\approx2.565$
and $t_{2}^{*}\approx7.695$. (b) and (c) are numerical results of
the time when $\lambda$ takes its local maximums for different sizes $L$. We have taken $\gamma_{i}=1.5$ and
$\gamma_{f}=0.5$. \label{Fig1}}
\end{figure}

For a finite-size system under the PBC, we can utilize Eqs. (\ref{rf})
and (\ref{Lt}) to calculate the rate function numerically. In Fig.\ref{Fig1}
(a), we display the rate function $\lambda(t)$ versus time $t$ for
different system sizes $L$. Around the critical times $t_{n}^{*}$,
the rate function exhibits a series of peaks and the times corresponding
these local maximums $t_{n}^{*}(L)$ can be used to interpolate numerically
the values of critical times in the limit of $L\rightarrow\infty$.
When we increase the size, $t_{n}^{*}(L)$ does not change linearly
with $L$, but approaches the critical times $t_{n}^{*}$ in an oscillating
way as shown in Fig.\ref{Fig1} (b) and (c). In the thermodynamical
limit, the non-analytical behaviors of $\lambda(t)$ are characterized
by the emergence of a cusp at $t_{n}^{*}$. Using $\lambda_{max}$
to represent the first local maximum of $\lambda(t)$,
we find that the value of $\lambda_{max}$ does not increase linearly
with the increase of system size but approaches a finite number in
an oscillating way. Our numerical result unveils $\lambda_{max}\sim0.643$
with $L\rightarrow\infty$. In the thermodynamical limit $L\rightarrow\infty$,
the momentum $k$ distributes continuously and we have $$\lambda(t)=-\frac{1}{2\pi}\int_{0}^{2\pi}\ln[1-\Lambda_{k}\sin^{2}[\epsilon_{f}(k)t]]\mathrm{d}k,$$
from which we numerically evaluate the value $\lambda(t_{1}^{*})\approx0.643$
at the critical time $t_{1}^{*}$. It is evident that $\lambda(t_{1}^{*})$
is equal to $\lambda_{max}$ in the thermodynamical limit.
\begin{figure}
\begin{centering}
\includegraphics[scale=0.98]{Fig2}
\par\end{centering}
\caption{(a) The images of $k_{c,+}/\pi$ and $k_{c,-}/\pi$ versus $\gamma_{f}$
for the SSH model. The blue and red lines correspond to $\gamma_{i}=0.5$
and $\gamma_{i}=1.5$, respectively. The two red circles denote $k_{c,+}/\pi\approx-0.839$
and $k_{c,-}/\pi\approx0.839$ for $\gamma_{i}=1.5$ and $\gamma_{f}=0.5$, respectively.
(b) The exact solution of $\phi_{c}/\pi$ for $\gamma_{f}\in[-1,1].$
The red point denotes $\phi_{c}/\pi\approx0.783$ for $\gamma_{f}=0.5$.
Here $\gamma_{i}=1.5$ and $L=20$. \label{Fig2}}
\end{figure}

The non-analytical behaviors of the rate function occurring at the
critical times $t_{n}^{*}$ are associated to the emergence of zeros
of LE. We notice that the constraint relation for ensuring $\mathcal{L}(t)=0$
is
\begin{equation}
\gamma_{f}=-\frac{1+\gamma_{i}\cos k}{\gamma_{i}+\cos k}.\label{eq:gammaf}
\end{equation}
If $|\gamma_{i}|<1$, Eq.(\ref{eq:gammaf}) is fulfilled only for
$|\gamma_{f}|>1$. On the other hand, if $|\gamma_{i}|>1$, Eq.(\ref{eq:gammaf})
is fulfilled only for $|\gamma_{f}|<1$. It means that the exact zeros
of LE emerge only when the quench parameter $\gamma$ is across the
underlying phase transition point. % i.e., the DQPT takes place only when the initial and final systems are in different topological phases \cite{Vajna,YangC}.
When $\gamma_{i}$ and $\gamma_{f}$ are in different phase regions,
there always exists a pair of momentum modes given by
\begin{equation}
k_{c,\pm}=\pm\arccos[-\frac{1+\gamma_{i}\gamma_{f}}{\gamma_{i}+\gamma_{f}}],\label{kc}
\end{equation}
which leads to the occurrence of a series of zero points of LE at
\begin{equation}
t_{n}^{*}=\frac{\pi}{2\epsilon_{f}(k_{c})}(2n-1),
\end{equation}
with
\begin{equation}
\epsilon_{f}(k_{c})/t_{1}=\sqrt{\frac{(1-\gamma_{f}^{2})(\gamma_{i}-\gamma_{f})}{\gamma_{i}+\gamma_{f}}}\label{Ekc}
\end{equation}
and $n$ being a positive integer. In Fig.\ref{Fig2} (a), we exhibit
the images of $k_{c,+}/\pi$ and $k_{c,-}/\pi$ versus $\gamma_{f}$
for $\gamma_{i}=0.5$ and $\gamma_{i}=1.5$ according to Eq.(\ref{kc}),
and the two red circles denote $k_{c,+}/\pi\approx-0.839$ and $k_{c,-}/\pi\approx0.839$
for $\gamma_{i}=1.5$ and $\gamma_{f}=0.5$. For finite size systems,
$k$ takes discrete values. According to Eq.(\ref{kc}), $k_{c,\pm}$
is usually not equal to the quantized momentum values $k=2\pi m/L$
enforced by the PBC except for some fine-tuned postquench parameters
\citep{ZhouBZ2021}. With the increase in the system size, $k_{c\pm}$
can be approached in terms of $\min|k-k_{c\pm}|\leq\pi/L$, and thus
exact zeros of LE are usually only achievable in the
thermodynamical limit of $L\rightarrow\infty$.

Although exact zeros of LE for a finite size system generally do not
exist, next we unveil that exact zeros of LE can be achieved even
in a finite size system if we introduce a magnetic flux $\phi$ into
the system. The effect of magnetic flux is effectively described by
the introduction of a twist boundary condition in real space $c_{L+1}^{\dagger}=c_{1}^{\dagger}e^{i\phi}(\phi\in(0,\pi])$.
Under the twist boundary condition, the quantized momentum is shifted
by a factor $\phi/L$, i.e., $k=\frac{2\pi m+\phi}{L}$ with $m=-L/2,-L/2+1,\cdots,L/2-1$
if $L$ is even or $m=-(L-1)/2,-(L-1)/2+1,\cdots,(L-1)/2$ if $L$
is odd. Therefore, for a given lattice size $L$ we can always achieve
$k_{c,+}$ or $k_{c,-}$ by tuning the flux $\phi$ to
\begin{equation}
\phi_{c}=\min \{\mod[Lk_{c,+},2\pi],\mod[Lk_{c,-},2\pi]\}.\label{ssh-phic}
\end{equation}
In Fig.\ref{Fig2} (b), we display the image of $\phi_{c}/\pi$ versus
$\gamma_{f}$ according to Eq.(\ref{ssh-phic}) for the system with
$\gamma_{i}=1.5$ and $L=20$, and the red point in the picture denotes
$\phi_{c}/\pi\approx0.783$ for $\gamma_{i}=1.5$ and $\gamma_{f}=0.5$.

Let $\Delta=\phi-\phi_{c}$, at the time $t=t_{n}^{*}$, we can get
\begin{equation}
\lambda(t_{n}^{*})=-\frac{1}{L}[\ln\mathcal{L}_{k^{*}}(t_{n}^{*})+\sum_{k\neq k^{*}}\ln\mathcal{L}_{k}(t_{n}^{*})],
\end{equation}
where $\mathcal{L}_{k^{*}}(t_{n}^{*})$ comes from the contribution of the $k^{*}$-mode which
is closest to $k_{c}$, i.e. $k^{*}=k_{c}+\Delta/L$. Let $\Delta\rightarrow0$, we can get
\begin{equation}
\mathcal{L}_{k^{*}}(t_{n}^{*})\approx\frac{(\gamma_{i}+\gamma_{f})^{3}+\gamma_{f}^{2}t_{n}^{*2}(\gamma_{f}-\gamma_{i})(1-\gamma_{i}^{2})}{(\gamma_{f}-\gamma_{i})^{2}(\gamma_{f}+\gamma_{i})L^{2}}\Delta^{2}.\label{eq:eta}
\end{equation}
When $\Delta\rightarrow0$,
$\mathcal{L}_{k^{*}}(t_{n}^{*}) \rightarrow 0$ and thus $\ln\mathcal{L}_{k^{*}}(t_{n}^{*})$ is divergent, i.e., when $\phi$ achieves
$\phi_{c}$, the rate function becomes divergent at the critical times.
\begin{figure}
\begin{centering}
\includegraphics[scale=0.78]{Fig3}
\par\end{centering}
\caption{(a) The rate function $\lambda(t)$ versus $t$ for the SSH model
with $\gamma_{i}=1.5,\ \gamma_{f}=0.5,\ \phi=0,0.5\pi,0.783\pi$ and $\pi$.
Vertical dashed lines guide the divergent points $t_{1}^{*}\approx2.565$
and $t_{2}^{*}\approx7.695$, respectively. (b) The images of $\lambda_{max}$ versus $\phi/\pi$. The dashed blue line corresponds to $\gamma_{i}=1.5,\ \gamma_{f}=0.5$,
whereas the dotted orange line corresponds to $\gamma_{i}=1.5,\ \gamma_{f}=1.2$.
The vertical dashed line guides the divergent point $\phi_{c}/\pi\approx0.783$.
Here we have taken $L=20$. \label{Fig3}}
\end{figure}

In Fig.\ref{Fig3} (a), we demonstrate rate functions versus $t$
for various $\phi$ with $L=20$, $\gamma_{i}=1.5$ and $\gamma_{f}=0.5$.
It is shown that the rate function is divergent at the critical times
$t_{1}^{*}\approx2.565$ and $t_{2}^{*}\approx7.695$ when $\phi$
is tuned to the critical value $\phi_{c}$ which has been shown in
Fig.\ref{Fig2} (b). In comparison with Fig.\ref{Fig1} (a), both
the nonanalytical behaviors occur at the same critical times $t_{1}^{*}$
and $t_{2}^{*}$. While the nonanalyticity of the rate function in
the thermodynamical limit is characterized by a cusp or a kink, the
nonanalyticity of the rate function of a finite size system induced
by tuning the flux $\phi$ is characterized by the appearance of singularity
at the critical times. Such a singularity of the rate function for
the finite size system is a kind of dynamical singularity, which corresponds
to the occurrence of exact zeros of LE. The existence of dynamical
singularity for a finite size system means that the initial state
can evolve to its orthogonal state at a series of time by tuning the
magnetic flux.

For a given $\gamma_{i}$ and $\gamma_{f}$, tuning $\phi$ from $0$
to $\pi$, from Fig.\ref{Fig3} (b) we can see that if $\gamma_{i}$
and $\gamma_{f}$ belong to the same phase, $\lambda_{max}$ barely
changes with $\phi$, which means no singularity of rate function can be observed; if $\text{\ensuremath{\gamma}}_{i}$ and $\gamma_{f}$
belong to different phases, $\lambda_{max}$ will diverge at $\phi_{c}/\pi\approx0.783$,
which gives a signal of DQPT.
Therefore, we can judge whether a DQPT happens by observing the
change of $\lambda_{max}$ as a function of $\phi$, which continuously varies from $0$ to $\pi$. By tuning $\phi$ in finite size
systems, we also obtain the critical times of DQPT, which are usually defined
in the thermodynamical limit and can be extracted from the finite-size-scaling analysis in previous studies.

Next we consider the Creutz model \cite{Creutz1999} which describes
the dynamics of a spinless electron moving in a ladder system governed
by the Hamiltonian:
\begin{align}
H & =-\sum_{j=1}[J_{h}(e^{i\theta}c_{j+1}^{p\dagger}c_{j}^{p}+e^{-i\theta}c_{j+1}^{q\dagger}c_{j}^{q})\nonumber \\
 & \qquad\qquad+J_{d}(c_{j+1}^{p\dagger}c_{j}^{q}+c_{j+1}^{q\dagger}c_{j}^{p})+J_{v}c_{j}^{q\dagger}c_{j}^{p}+H.c.],
\end{align}
where $c_{j}^{p(q)\dagger}$ and $c_{j}^{p(q)}$ are fermionic creation
and annihilation operators on the $j$th site of the lower (upper)
chain, $J_{h}$, $J_{d}$ and $J_{v}$ represent hopping
amplitudes for horizontal, diagonal and vertical bonds, respectively,
and $\theta\in[-\pi/2,\pi/2]$ represents the magnetic flux per plaquette
induced by a magnetic field pierces the ladder \citep{Jafari,Creutz1999}.
Via the Fourier transformation, the vector components of the
Hamiltonian in momentum space can be expressed as $d_{x}(k)=-2J_{d}\cos k-J_{v},\ d_{y}(k)=0,\ d_{z}(k)=-2J_{h}\sin k\sin\theta$
and $d_{0}(k)=-2J_{h}\cos k\cos\theta$. For simplicity, in the following
we will focus on the case of $J_{h}=J_{d}=J$ and $J_{v}/2J<1$, and take $J=1$ as the units of energy. In
this case, the Creutz model has two distinct topologically nontrivial
phases for $-\pi/2\leq\theta<0$ and $0<\theta\leq\pi/2$ with a phase
transition occurring at the transition point of $\theta=0$ \cite{LiLH}. Then
we quench parameter $\theta$ from $\theta_{i}$ to $\theta_{f}$
at $t=0$ and get the Loschmidt echo of the Creutz model
\begin{align}
\mathcal{L}(t) & =\prod_{k}\left\{ 1-\Lambda_{k}\sin^{2}[\epsilon_{f}(k)t]\right\} ,
\end{align}
where
\begin{equation}
\Lambda_{k}=1-\frac{16J^{4}[(\cos k+\tilde{J_{v}})^{2}+\sin^{2}k\sin\theta_{i}\sin\theta_{f}]^{2}}{\epsilon_{i}^{2}(k)\epsilon_{f}^{2}(k)},
\end{equation}
\begin{equation}
\epsilon_{i}(k)=2J\sqrt{(\cos k+\tilde{J_{v}})^{2}+\sin^{2}k\sin^{2}\theta_{i}},
\end{equation}
and
\begin{equation}
\epsilon_{f}(k)=2J\sqrt{(\cos k+\tilde{J_{v}})^{2}+\sin^{2}k\sin^{2}\theta_{f}} \label{Ekf-1}
\end{equation}
with $\tilde{J_{v}}=J_{v}/2J$.
The corresponding constraint
relation of Eq.(\ref{eq:dvalue}) for the occurrence of exact zeros of LE is
\begin{equation}
\sin\theta_{f}=-\frac{(\cos k+\tilde{J_{v}})^{2}}{\sin^{2}k\sin\theta_{i}}.\label{eq:thetaf}
\end{equation}
If $\sin\theta_{i}<0$, Eq.(\ref{eq:thetaf}) is fulfilled only for
$\sin\theta_{f}>0$. On the other hand, if $\sin\theta_{i}>0$, Eq.(\ref{eq:thetaf})
is fulfilled only for $\sin\theta_{f}<0$. It means that the dynamical
singularity of the rate function exists only when the quench parameter
$\theta$ is across the underlying phase transition point.

When $\theta_{i}$ and $\theta_{f}$ are in different phase regions,
there are always two pairs of momentum modes given by
\begin{equation}
k_{c,1\pm}=\pm\arccos[\frac{-\tilde{J_{v}}+\sqrt{A(\tilde{J}_{v}^{2}-1+A)}}{1-A}]\label{kc-1}
\end{equation}
and
\begin{equation}
k_{c,2\pm}=\pm\arccos[\frac{-\tilde{J_{v}}-\sqrt{A(\tilde{J}_{v}^{2}-1+A)}}{1-A}]\label{kc-2}
\end{equation}
with $A=\sin\theta_{i}\sin\theta_{f}$, 
which lead to the occurrence of a series of dynamical singularities
of the rate function at
\begin{equation}
t_{n,1/2}^{*}=\frac{\pi}{2\epsilon_{f}(k_{c,1/2})}(2n-1),
\end{equation}
with
\begin{equation}
\epsilon_{f}(k_{c,1/2})/2J=\sqrt{(\sin\theta_{f}-\sin\theta_{i})\sin\theta_{f}\sin^{2}k_{c,1/2}}.\label{Ekc-1}
\end{equation}
Here $n$ is a positive integer. Similarly, $k_{c,1\pm}$ and $k_{c,2\pm}$
are usually not equal to the quantized momentum values $k=2\pi m/L$
enforced by the PBC except for some fine-tuned postquench parameters.
It means that the exact zeros of LE of a finite size system generally
do not exist for arbitrary $\theta_{i}$ and $\theta_{f}$. With the
increase in the system size, $k_{c,1/2\pm}$ can be approached in
terms of $\min|k-k_{c,1/2\pm}|\leq\pi/L$, and thus dynamical singularities
of the rate function are usually only achieved in the limit of $L\rightarrow\infty$.
\begin{figure}
\begin{centering}
\includegraphics[scale=0.54]{Fig4}
\par\end{centering}
\caption{(a) The rate function $\lambda(t)$ versus $t$ for the Creutz model
with different system sizes $L=40$, $60$, 100 and $1200$. Vertical
dashed lines guide the values of critical times $t_{1,1}^{*}\approx1.435$,
$t_{1,2}^{*}\approx2.176$, $t_{2,1}^{*}\approx4.306$ and $t_{2,2}^{*}\approx6.527$, respectively.
(b) and (c) are numerical results of the time when $\lambda$ takes its local maximums
for different sizes $L$. Here $\tilde{J}_{v}=0.5,\ \theta_{i}=0.4$ and $\theta_{f}=-0.4$.
\label{Fig4}}
\end{figure}

In Fig.\ref{Fig4} (a), we display the rate function $\lambda(t)$
versus time $t$ for different system sizes $L$. From Fig.\ref{Fig4}
(b) and (c) we can see that $t_{1,1}^{*}(L)$ and $t_{1,2}^{*}(L)$
approach the critical times in an oscillating way as the size $L$
increases. With the increase of the size $L$, we find that the value
of $\lambda_{max}$ also approaches a finite number in an oscillating
way and $\lambda_{max}\sim0.621$ when $L\rightarrow\infty$. In the
thermodynamical limit, the momentum $k$ distributes continuously
and we have $\lambda(t)=-\frac{1}{2\pi}\int_{0}^{2\pi}\ln[1-\Lambda_{k}\sin^{2}[\epsilon_{f}(k)t]]\mathrm{d}k$,
from which we numerically evaluate the value $\lambda(t_{1,1}^{*})\approx0.621$
at the critical time $t_{1,1}^{*}$,  agreeing with $\lambda_{max}$
in the thermodynamical limit.

\begin{figure}
\begin{centering}
\includegraphics[scale=1.01]{Fig5}
\par\end{centering}
\caption{(a) The images of $k_{c,1+}$, $k_{c,1-}$, $k_{c,2+}$and $k_{c,2-}$
versus $\theta_{f}$ for the Creutz model. The blue and red lines correspond to
$\theta_{i}=-0.4$ and $\theta_{i}=0.4$, respectively. The four red
circles denote $k_{c,1\pm}/\pi\approx\pm0.536$ and $k_{c,2\pm}/\pi\approx\pm0.772$
for $\theta_{i}=0.4$ and $\theta_{f}=-0.4$. (b) The exact solutions
of $\phi_{c,1}/\pi$ and $\phi_{c,2}/\pi$ of the Creutz model for
$\theta_{f}\in[-\pi/2,0]$. The two red points denote $\phi_{c,1}/\pi\approx0.721$
and $\phi_{c,2}/\pi\approx0.550$ for $\theta_{f}=-0.4$. Here $\theta_{i}=0.4$, 
$\tilde{J}_{v}=0.5$ and $L=20$. \label{Fig5}}
\end{figure}

In Fig.\ref{Fig5}(a), we exhibit the images of $k_{c,1+}$, $k_{c,1-}$,
$k_{c,2+}$and $k_{c,2-}$ versus $\theta_{f}$ for $\theta_{i}=-0.4$
and $\theta_{i}=0.4$ according to Eq.(\ref{kc-1}) and Eq.(\ref{kc-2}),
and the four red circles denote $k_{c,1\pm}/\pi\approx\pm0.536$ and
$k_{c,2\pm}/\pi\approx\pm0.772$ for $\theta_{i}=0.4$ and $\theta_{f}=-0.4$.
Since the quantized momenta $k$ usually do not include $k_{c,1\pm}$
and $k_{c,2\pm}$ under the PBC, we introduce the twist boundary condition
here. For a system with a given finite size $L$, we can always achieve
$k_{c,1/2+}$ or $k_{c,1/2-}$ by using the twist boundary condition
with
\begin{equation}
\phi_{c,1/2}=\min \{\!\!\!\!\!\mod[Lk_{c,1/2+},2\pi],\!\!\!\!\!\mod[Lk_{c,1/2-},2\pi]\}.\label{Creutz-phic}
\end{equation}
Fig.\ref{Fig5} (b) displays the images of $\phi_{c,1}/\pi$ and $\phi_{c,2}/\pi$
versus $\theta_{f}$ according to Eq.(\ref{Creutz-phic}) for the
system with $\theta_{i}=0.4$, $\tilde{J}_{v}=0.5$ and $L=20$, and
the two red points denote $\phi_{c,1}/\pi\approx0.721$ and $\phi_{c,2}/\pi\approx0.550$
for $\theta_{i}=0.4$ and $\theta_{f}=-0.4$.

Let $\Delta_{1/2}=\phi-\phi_{c,1/2}$, at the time $t=t_{n,1/2}^{*}$
we can get
\begin{equation}
\lambda(t_{n,1/2}^{*})=-\frac{1}{L}[\ln\mathcal{L}_{k_{1/2}^{*}}(t_{n,1/2}^{*})+\sum_{k\neq k_{1/2}^{*}}\ln\mathcal{L}_{k}(t_{n,1/2}^{*})],
\end{equation}
where $\mathcal{L}_{k_{1/2}^{*}}(t_{n,1/2}^{*})$ comes from the contribution of the $k_{1/2}^{*}$-mode
which is closest to $k_{c,1/2}$, i.e. $k_{1/2}^{*}=k_{c,1/2}+\Delta_{1/2}/L$. Let $\Delta_{1/2}\rightarrow0$,
we can get
\begin{equation}
\mathcal{L}_{k_{1/2}^{*}}(t_{n,1/2}^{*})\approx B_{1/2}\Delta_{1/2}^{2},\label{eq:eta-1}
\end{equation}
where $$B_{1/2}=\frac{4[t_{n,1/2}^{*2}(\tilde{J}_{v}+\cos k_{c,1/2}\cos^{2}\theta_{f})^{2}-C]}{(\sin\theta_{f}-\sin\theta_{i})\sin\theta_{f}L^{2}}$$
with $C=\frac{\sin\theta_{f}(\tilde{J}_{v}^{2}-1+\sin\theta_{i}\sin\theta_{f})}{\sin^{2}k_{c,1/2}(\sin\theta_{f}-\sin\theta_{i})}$. 
It means when $\Delta_{1/2}\rightarrow0,$ i.e. $\phi\rightarrow\phi_{c,1/2}$,  
$\mathcal{L}_{k_{1/2}^{*}}(t_{n,1/2}^{*})\propto\Delta_{1/2}^{2}$.
When $\phi$ reaches $\phi_{c,1/2}$, we can get a $k_{1/2}^{*}$-mode
which satisfies $k_{1/2}^{*}=k_{c,1/2}$ and $\mathcal{L}_{k_{1/2}^{*}}(t_{n,1/2}^{*})=0$,
thus the rate function is divergent at $t_{n,1/2}^{*}$.
\begin{figure}
\begin{centering}
\includegraphics[scale=0.61]{Fig6}
\par\end{centering}
\caption{(a) The rate function $\lambda(t)$ versus $t$ for the Creutz model with $\theta_{i}=0.4,\ \theta_{f}=-0.4,\ \phi=0,0.550\pi,0.721\pi$ and $\pi$, respectively.
Vertical dashed lines guide the divergent points $t_{1,1}^{*}\approx1.435$,
$t_{1,2}^{*}\approx2.176$, $t_{2,1}^{*}\approx4.306$ and $t_{2,2}^{*}\approx6.527$.
(b) The images of $\lambda_{max}$ versus $\phi/\pi$. The dashed
blue line corresponds to $\theta_{i}=0.4,\ \theta_{f}=-0.4$, and the dotted
orange line corresponds to $\theta_{i}=0.4,\ \theta_{f}=0.1$. Vertical
dashed lines guide the divergent points $\phi_{c,1}/\pi\approx0.721$
and $\phi_{c,2}/\pi\approx0.550$. Here we have taken $\tilde{J}_{v}=0.5$ and  $L=20$. \label{Fig6}}
\end{figure}

In Fig.\ref{Fig6} (a), we demonstrate rate functions versus $t$
for various $\phi$ with $L=20$, $\tilde{J}_{v}=0.5$, $\theta_{i}=0.4$
and $\theta_{f}=-0.4$. It is shown that the rate functions are divergent
at the critical times $t_{1,1}^{*}\approx1.435$ and $t_{2,1}^{*}\approx4.306$,
when $\phi$ is tuned to the critical value $\phi_{c,1}\approx0.721\pi$,
and divergent at the critical times $t_{1,2}^{*}\approx2.176$ and
$t_{2,2}^{*}\approx6.527$, when $\phi$ is tuned to the critical value
$\phi_{c,2}\approx0.550\pi$. In comparison with Fig.\ref{Fig4} (a),
all the nonanalytical behaviors occur at the same critical times obtained by finite-size-scaling analysis.
For a pair of given $\theta_{i}$ and $\theta_{f}$, Fig.\ref{Fig6}
(b) shows that if $\theta_{i}$ and $\theta_{f}$ belong to the same
phase, $\lambda_{max}$ only changes slightly with $\phi$, which means the absence of DQPT; if $\theta_{i}$
and $\theta_{f}$ belong to different phases, $\lambda_{max}$ will
diverge at $\phi_{c,1}/\pi$ and $\phi_{c,2}/\pi$, indicating the occurrence of DQPT.

\section{SUMMARY }

In summary, we have proposed a theoretical scheme for studying the dynamical singularity of rate function in finite size quantum systems which exhibit DQPT in the thermodynamic limit.
%For a pair of given prequench parameter and postquench parameter, we can get $k_{c}$ which can satisfy the constraint relation.
The dynamical singularity of rate function occurs whenever the corresponding LE has exact zero points, which is however not accessible in a finite size quantum system with the PBC because the momentum takes quantized values $k=2\pi m/L$. In order to realize the exact zeros of LE, we consider the twist boundary condition by applying a magnetic flux into the system, which enables us to shift the quantized momentum continuously to achieve the exact zeros of LE.
Taking the SSH model and Creutz model as concrete examples, we demonstrate
that tuning the magnetic flux can lead to the occurrence of divergency in the rate function of a finite size system at the same critical times 
as in the case of the thermodynamical limit, when the quench parameter is
across the underlying equilibrium phase transition point. 
Our work unveils that the singularity of the rate function is
accessible in finite size quantum systems by introducing an additional magnetic flux, which provides an alternative way for experimentally detecting
DQPT and the critical times in finite size quantum systems.
\begin{acknowledgments}
The work is supported by National Key Research and Development Program of China (Grant No. 2021YFA1402104), the NSFC under Grants No.12174436 %No.11974413
and No.T2121001 and the Strategic Priority Research Program of Chinese
Academy of Sciences under Grant No. XDB33000000.
\end{acknowledgments}

\begin{thebibliography}{10}
\bibitem{Heyl2013PRL} M. Heyl, A. Polkovnikov, S. Kehrein, dynamical
quantum phase transitions in the transverse-field Ising model, Phys.
Rev. Lett. \textbf{110,} 135704 (2013).

\bibitem{Karrasch2013PRB} C. Karrasch and D. Schuricht, Dynamical
phase transitions after quenches in nonintegrable models, Phys. Rev.
B \textbf{87,} 195104 (2013).

\bibitem{Hickey2014PRB} J. M. Hickey, S. Genway, and J. P. Garrahan,
Dynamical phase transitions, time-integrated observables, and geometry
of states, Phys. Rev. B \textbf{89,} 054301 (2014).

\bibitem{Canovi2014PRL} E. Canovi, P. Werner, and M. Eckstein, First-order
dynamical phase transitions, Phys. Rev. Lett. \textbf{113,} 265702
(2014).

\bibitem{Andraschko2014PRB} F. Andraschko and J. Sirker, Dynamical
quantum phase transitions and the Loschmidt echo: A transfer matrix
approach, Phys. Rev. B \textbf{89,} 125120 (2014).

\bibitem{Schmitt2015PRB} M. Schmitt and S. Kehrein, Dynamical quantum
phase transitions in the Kitaev honeycomb model, Phys. Rev. B \textbf{92,}
075114 (2015).

%\bibitem{Marcuzzi2014PRL} M. Marcuzzi, E. Levi, S. Diehl, J. P. Garrahan,
%and I. Lesanovsky, Universal nonequilibrium properties of dissipative
%Rydberg gases, Phys. Rev. Lett. \textbf{113,} 210401 (2014).

\bibitem{Heyl2014PRL} M. Heyl, Dynamical Quantum phase transitions
in systems with broken-symmetry phases, Phys. Rev. Lett. \textbf{113,}
205701 (2014).

\bibitem{Heyl2015PRL} M. Heyl, Scaling and universality at dynamical
quantum phase transitions, Phys. Rev. Lett. \textbf{115,} 140602 (2015).

\bibitem{Dora} S. Vajna and B. D\'{o}ra, Topological classification of
dynamical phase transitions, Phys. Rev. B \textbf{91},
155127 (2015).

\bibitem{Budich2016PRB} J. C. Budich and M. Heyl, Dynamical topological
order parameters far from equilibrium, Phys. Rev. B \textbf{93,} 085416
(2016).



\bibitem{YangC} C. Yang, L. Li and S. Chen, Dynamical topological
invariant after a quantum quench, Phys. Rev. B \textbf{97,} 060304
(2018).

\bibitem{Mera2018PRB} B. Mera, C. Vlachou, N. Paunkovi\'{c}, V. R.
Vieira, and O. Viyuela, Dynamical phase transitions at finite temperature
from fidelity and interferometric Loschmidt echo induced metrics,
Phys. Rev. B \textbf{97,} 094110 (2018).

\bibitem{Vajna2014PRB} S. Vajna and B. D\'{o}ra, Disentangling dynamical
phase transitions from equilibrium phase transitions, Phys. Rev. B
\textbf{89,} 161105 (2014).

\bibitem{Sharma2015PRB} S. Sharma, S. Suzuki, and A. Dutta, Quenches
and dynamical phase transitions in a nonintegrable quantum Ising model,
Phys. Rev. B \textbf{92,} 104306 (2015).

\bibitem{Zhou2019PRB} B. Zhou, C. Yang, and S. Chen, Signature of
a nonequilibrium quantum phase transition in the long-time average
of the Loschmidt echo, Phys. Rev. B \textbf{100,} 184313 (2019).


\bibitem{BoBo2020PRB} G. Sun and B.-B. Wei, Dynamical quantum phase
transitions in a spin chain with deconfined quantum critical points,
Phys. Rev. B \textbf{102,} 094302 (2020).

\bibitem{Halimeh} J. C. Halimeh and V. Zauner-Stauber, Dynamical
phase diagram of spin chains with long-range interactions, Phys. Rev.
B \textbf{96,} 134427 (2017).

%\bibitem{Pedersen2021PRB} S. P. Pedersen and N. T. Zinner, Lattice gauge theory and dynamical quantum phase transitions using noisy intermediate-scale quantum devices, Phys. Rev. B \textbf{96,} 235103 (2021).

\bibitem{Halimeh2} I. Homrighausen, N. O. Abeling, V. Zauner-Stauber,
and J. C. Halimeh, Anomalous dynamical phase in quantum spin chains
with long-range interactions, Phys. Rev. B \textbf{96,} 104436 (2017).

\bibitem{Jafari2} R. Jafari and A. Akbari, Floquet dynamical phase
transition and entanglement spectrum, Phys. Rev. A \textbf{103}, 012204 (2021).



\bibitem{Zvyagin2016LTP} A. A. Zvyagin, Dynamical quantum phase transitions
(Review Article), Fiz. Nizk. Temp. (Kiev) \textbf{42,} 1240 (2016)
{[}Low Temp. Phys. \textbf{42,} 971 (2016){]}.

\bibitem{Heyl2018RPP} M.Heyl, Dynamical quantum phase transitions:
a review, Rep. Prog. Phys. \textbf{81,} 054001 (2018).

\bibitem{Gorin2006PR} T. Gorin, T. Prosen, T. H. Seligman, and M.
Znidaric, Dynamics of Loschmidt echoes and fidelity decay, Phys. Rep.
\textbf{435,} 33 (2006).

\bibitem{Liska} D. Liska and V. Gritsev, The Loschmidt Index, SciPost
Phys. \textbf{10}, 100 (2021).

\bibitem{ZhouBZ2021} B. Zhou, Y. Zeng, and S. Chen, Exact zeros of
the Loschmidt echo and quantum speed limit time for the dynamical
quantum phase transition in finite-size systems, Phys. Rev. B \textbf{104,}
094311 (2021).

\bibitem{OC} P. W. Anderson, Infrared Catastrophe in Fermi Gases
with Local Scattering Potentials, Phys. Rev. Lett. \textbf{18},
1049 (1967).

\bibitem{Jurcevic2017PRL} P. Jurcevic, H. Shen, P. Hauke, C. Maier,
T. Brydges, C. Hempel, B. P. Lanyon, M. Heyl, R. Blatt, and C. F.
Roos, Direct observation of dynamical quantum phase transitions in
an interacting many-body system, Phys. Rev. Lett. \textbf{119,} 080501
(2017).

\bibitem{Monroe2017Nature} J. Zhang, G. Pagano, P.W. Hess, A. Kyprianidis, P. Becker,
H. Kaplan, A. V. Gorshkov, Z.-X. Gong, and C. Monroe, Observation of a many-body dynamical phase transition with a 53-qubit quantum simulator,
Nature (London) {\bf 551}, 601 (2017).


\bibitem{Bernien-Nature} H. Bernien, S. Schwartz, A. Keesling, H. Levine, A. Omran,
H. Pichler, S. Choi, A. S. Zibrov, M. Endres, M. Greiner, V.
Vuletic, and M. D. Lukin, Probing many-body dynamics on a 51-atom quantum simulator, Nature (London) \textbf{551}, 579 (2017).


\bibitem{Flaschner2018Nature} N. Fl$\ddot{a}$schner, D. Vogel, M.
Tarnowski, B. S. Rem, D.-S. L$\ddot{u}$hmann, M. Heyl, J. C. Budich,
L. Mathey, K. Seng- stock, and C. Weitenberg, Observation of dynamical
vortices after quenches in a system with topology, Nat. Phys. \textbf{14,}
265 (2018).

\bibitem{DuanLM} T. Tian, H.-X. Yang, L.-Y. Qiu, H.-Y. Liang, Y.-B. Yang, Y. Xu, and L.-M. Duan, Observation of Dynamical Quantum Phase Transitions with Correspondence in an Excited State Phase Diagram, Phys. Rev. Lett. {\bf 124}, 043001 (2020).

\bibitem{Smale} S. Smale, P. He, B. A. Olsen, K. G. Jackson, H. Sharum, S. Trotzky, J. Marino, A. M. Rey, and J. H. Thywissen, Observation of a transition between dynamical phases in a quantum degenerate fermi gas, Sci. Adv. \textbf{5}, eaax1568 (2019).

\bibitem{GuoXY} X.-Y. Guo, C. Yang, Y. Zeng, Y. Peng, H.-K.
Li, H. Deng, Y.-R. Jin, S. Chen, D. Zheng, and H. Fan, Observation
of a dynamical quantum phase transition by a superconducting qubit
simulation, Phys. Rev. Applied \textbf{11,} 044080 (2019).

\bibitem{XueP} K. Wang, X. Qiu, L. Xiao, X. Zhan, Z. Bian,
W. Yi, and P. Xue, Simulating dynamic quantum phase transitions in
photonic quantum walks, Phys. Rev. Lett. \textbf{122,} 020501 (2019).

%\bibitem{BoBo2020PRL} X. Nie, B.-B. Wei, X. Chen, Z. Zhang, X. Zhao, C. Qiu, Y. Tian, Y. Ji, T. Xin, D. Lu, and J. Li, Experimental Observation of Equilibrium and Dynamical Quantum Phase Transitions via Out-of-Time-Ordered Correlators, Phys. Rev. Lett. \textbf{124,} 250601(2020).

%\bibitem{Lupo}C. Lupo and M. Schirš®, Transient Loschmidt echo in quenched Ising chains, Phys. Rev. B \foreignlanguage{english}{\textbf{94}, 014310 (2016).}

\bibitem{SSH} W. P. Su, J. R. Schrieffer, and A. J. Heeger, Solitons in Polyacetylene, Phys. Rev. Lett. \textbf{42},
1698 (1979).
\bibitem{Creutz1999} M. Creutz, End States, Ladder Compounds, and
Domain-Wall Fermions, Phys. Rev. Lett. \textbf{83,} 2636 (1999).

\bibitem{LiLH2014} L. Li, Z. Xu, and S. Chen, Topological phases of generalized Su-Schrieffer-Heeger models, Phys. Rev. B \textbf{89}, 085111 (2014).

\bibitem{Jafari} R. Jafari, H. Johannesson, A. Langari, and M. A.
Martin-Delgado, Quench dynamics and zero-energy modes: The case of
the Creutz model, Phys. Rev. B \textbf{99}, 054302 (2019).

\bibitem{LiLH} Linhu Li and Shu Chen, Characterization of topological
phase transitions via topological properties of transition points,
Phys. Rev. B \textbf{92}, 085118 (2015).

\end{thebibliography}

\end{document}
