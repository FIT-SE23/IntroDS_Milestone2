%\documentclass[letterpaper,twocolumn,10pt]{article}
\documentclass{article}             
\usepackage[english]{babel}
\usepackage[utf8]{inputenc}
\usepackage{CJKutf8}

\usepackage[normalem]{ulem}

\usepackage{amsfonts,amsmath,amssymb,graphicx}
\usepackage{cuted}  % Yue added 10/26/2021 for double column equations
\usepackage[usenames,dvipsnames]{xcolor}
\usepackage{setspace}
\usepackage{hyperref}
\usepackage{amsmath}
% \usepackage{multicol}

\usepackage{pdfpages}


\newcommand{\be}{\begin{eqnarray}}
\newcommand{\ee}{\end{eqnarray}}
\newcommand{\ra}{\rightarrow}
\newcommand{\var}{\varepsilon}
\newcommand{\noi}{\noindent}
\newcommand{\dg}{\dagger}
\newcommand{\ola}{\overleftarrow}
\newcommand{\ora}{\overrightarrow}
\newcommand{\rang}{\rangle}
\newcommand{\lang}{\langle}
\newcommand{\ketbra}[2]{|{#1}\rangle \langle{#2}|}
\newcommand{\bra}[1]{\langle{#1}|}
\newcommand{\ket}[1]{|{#1}\rangle}

\newcommand{\op}[1]{\hat{#1}}
\newcommand{\sx}{\hat{\sigma}_\mathrm{x}}
\newcommand{\sz}{\hat{\sigma}_\mathrm{z}}
\newcommand{\sxo}{\hat{\sigma}_{\mathrm{x}1}}
\newcommand{\szo}{\hat{\sigma}_{\mathrm{z}1}}
\newcommand{\sxt}{\hat{\sigma}_{\mathrm{x}2}}
\newcommand{\szt}{\hat{\sigma}_{\mathrm{z}2}}
\newcommand{\Aa}{\hat{a}^\dag\hat{a}}
\newcommand{\ha}{\hat{a}}
\newcommand{\hA}{\hat{a}^\dag}
\newcommand{\wc}{\omega_\mathrm{r}}
\newcommand{\wa}{\omega_\mathrm{a}}
\newcommand{\wdr}{\omega_\mathrm{d}}
\newcommand{\dr}{\delta_\mathrm{r}}
\newcommand{\hc}{\mathrm{h.c.}}
\newcommand{\catpm}{\ket{\mathcal{C}^\pm_\alpha}}
\newcommand{\catp}{\ket{\mathcal{C}^+_\alpha}}
\newcommand{\catm}{\ket{\mathcal{C}^-_\alpha}}
\newcommand{\Ep}{\mathcal{E}_\textnormal{p}}
\newcommand{\E}{\mathcal{E}}

\newcommand{\catpmi}{\ket{\mathcal{C}^\pm_{\alpha_0}}}
\newcommand{\catpi}{\ket{\mathcal{C}^+_{\alpha_0}}}
\newcommand{\catmi}{\ket{\mathcal{C}^-_{\alpha_0}}}
\newcommand{\hta}{\hat{a}}
\newcommand{\g}{\ket{-\tilde{1}}}
\newcommand{\e}{\ket{\tilde{1}}}
\newcommand{\gepm}{\ket{\pm\tilde{1}}}
\newcommand{\go}{\ket{-\tilde{1}_0}}
\newcommand{\eo}{\ket{\tilde{1}_0}}
\newcommand{\gepmo}{\ket{\pm\tilde{1}_0}}
\newcommand{\zero}{\bar{0}}
\newcommand{\one}{\bar{1}}
\newcommand{\red}{\color[rgb]{0.8, 0, 0}}
\newcommand{\orange}[1]{{\color[rgb]{0.9, 0.5, 0.0}#1}}
\newcommand{\blue}{\color[rgb]{0.0, 0.0, 0.6}}
\newcommand{\Ez}{\mathcal{E}_\mathrm{z}}
\newcommand{\ab}[1]{{\color[rgb]{0.8, 0, 0} AB: #1}}
\newcommand{\cka}[1]{\textit{\color[rgb]{0.9, 0.5, 0} [CKA: #1] }}

\newcommand{\alg}[1]{{\color[rgb]{0.92,0.124,0.52}#1}}
\newcommand{\asout}[1]{\alg{\sout{#1}}}



\newcommand{\AUTHORS}{Authors}
\newcommand{\TITLE}{Title}
\newcommand{\KEYWORDS}{Keywords}
\newcommand{\CONFERENCE}{Somewhere}
\newcommand{\COLOR}{yes}
\newcommand{\PAGENUMBERS}{yes} 
\newcommand{\COMMENTS}{yes}

\usepackage{color,balance,xspace,verbatim,ifthen,engord}

\usepackage{amsmath,wasysym,amsthm,marvosym,stackengine}
% \usepackage{gensymb}

\usepackage{booktabs,colortbl,diagbox,multirow,tabularx,tablefootnote} 
\newcommand{\tabincell}[2]{\begin{tabular}{@{}#1@{}}#2\end{tabular}}
% \usepackage{dblfloatfix}

\usepackage{caption}
\usepackage{subcaption}
\usepackage{graphicx,epsfig,epstopdf,wrapfig}
\graphicspath{{./figcam/}} 
\captionsetup[subfigure]{labelformat=simple}  % remove parentheses, need to add manually to all figures

\DeclareGraphicsExtensions{.pdf,.mps,.png,.jpg,.eps,.PNG,.JPG}
\epstopdfsetup{outdir=./figures/}

\usepackage{algorithm,algorithmic}
\renewcommand{\algorithmicrequire}{\textbf{Input:}}
\renewcommand{\algorithmicensure}{\textbf{Output:}}

\hyphenation{op-tical net-works semi-conduc-tor}

\usepackage{url}
\def\UrlBreaks{\do\A\do\B\do\C\do\D\do\E\do\F\do\G\do\H\do\I\do\J\do\K\do\L\do\M\do\N\do\O\do\P\do\Q\do\R\do\S\do\T\do\U\do\V\do\W\do\X\do\Y\do\Z\do\[\do\\\do\]\do\^\do\_\do\`\do\a\do\b\do\c\do\d\do\e\do\f\do\g\do\h\do\i\do\j\do\k\do\l\do\m\do\n\do\o\do\p\do\q\do\r\do\s\do\t\do\u\do\v\do\w\do\x\do\y\do\z\do\0\do\1\do\2\do\3\do\4\do\5\do\6\do\7\do\8\do\9\do\.\do\@\do\\\do\/\do\!\do\_\do\|\do\;\do\>\do\]\do\)\do\,\do\?\do\'\do+\do\=\do\#}%

\newcommand{\secref}[1]{\S\ref{#1}}
\newcommand{\figref}[1]{Fig.~\ref{#1}}
\newcommand{\tabref}[1]{Tab.~\ref{#1}}
\newcommand{\eqnref}[1]{Eqn.~\ref{#1}}
\newcommand{\algref}[1]{Alg.~\ref{#1}}

\newcommand{\figcaption}[1]{\vspace{-8mm}\caption{#1}\vspace{-4mm}} 
\newcommand{\mfigcaption}[1]{\vspace{-4mm}\caption{#1}\vspace{-2mm}} 
\newcommand{\tabcaption}[1]{\vspace{1mm}\caption{#1}\vspace{-8mm}}
\newcommand{\mtabcaption}[1]{\vspace{-3mm}\caption{#1}\vspace{-8mm}}

\usepackage{tikz}
\newcommand{\WoB}[1]{{\small \tikz[baseline=(char.base)]{\node[shape=circle,fill=black,draw,inner sep=0.5pt] (char) {\color{white}#1};}}}

\newcommand{\sWoB}[1]{$\rlap{\Large{\CIRCLE}}{{\color{white}{\footnotesize \raisebox{0.7 pt}{\hspace{3.3pt}#1}}}}$~} % used in caption
\newcommand{\dWoB}[1]{$\rlap{\Large{\CIRCLE}}{{\color{white}{\scriptsize \raisebox{1 pt}{~#1}}}}$~}
\newcommand*\circled[1]{\tikz[baseline=(char.base)]{\node[shape=circle,draw,inner sep=0.5pt] (char) {#1};}}

\newcommand{\superscript}[1]{\ensuremath{^{\textrm{#1}}}}
\newcommand{\argmax}{\operatornamewithlimits{argmax}}
\def\deg{{\,^{\circ}}\xspace}

\newcommand{\nosection}[1]{\vspace{3pt}\noindent\textbf{#1}}
\newcommand{\nosubsection}[1]{\vspace{3pt}\noindent$\bullet$\hspace{1mm}\textit{#1}}
\newcommand{\heading}[1]{\vspace{3pt}\noindent\textup{\textbf{#1}}}
\newcommand{\blpara}{\vspace{3pt}\noindent$\bullet$\hspace{1mm}}
\newcommand{\blnosection}[1]{\vspace{3pt}\noindent$\bullet$\hspace{1mm}{#1}}
\newcommand{\blheading}[1]{\vspace{3pt}\noindent$\bullet$\hspace{1mm}\textup{\textbf{#1}}}
\def\newpara{\vspace{3pt}\noindent}

\def\It{\textit}
\def\Bf{\textbf}
\def\eg{\textit{e.g.,}\hspace{1mm}}
\def\ie{\textit{i.e.,}\hspace{1mm}}
\def\etal{\textit{et al.}\hspace{1mm}}
\def\etc{\textit{etc.}\hspace{1mm}}

\usepackage{courier}
% \newcommand{\code}[1]{\mbox{\texttt{#1}}}
\newcommand{\Mod}[1]{\mbox{\textsf{#1}}}
\newcommand{\sw}[1]{\mbox{\textsc{#1}}}

\usepackage{enumitem}

\newenvironment{Itemize}{
	\begin{list}{$\bullet$} {
		\setlength{\itemsep}{0pt}
		\setlength{\parsep}{2pt}
		\setlength{\topsep}{2pt}
		\setlength{\partopsep}{0pt}
		\setlength{\leftmargin}{1.5em}
		\setlength{\labelwidth}{1em}
		\setlength{\labelsep}{0.5em}
	}}
	{\end{list}}	

\newenvironment{Enumerate}{
	\begin{enumerate}[leftmargin=2em]
		\setlength{\itemsep}{2pt}
		\setlength{\topsep}{2pt}
		\setlength{\partopsep}{0pt}
		\setlength{\parskip}{0pt}}
	{\end{enumerate}}

\newenvironment{Circled}{
	\begin{enumerate}[label=\protect\circled{\arabic*},leftmargin=2em]
		\setlength{\itemsep}{3pt}
		\setlength{\topsep}{0pt}
		\setlength{\partopsep}{0pt}
		\setlength{\parskip}{0pt}}
	{\end{enumerate}}



%%% -------------------------------------------------------------------------

\title{Going in Style: Audio Backdoors through Stylistic Transformations}
 
\name{anonymous submission}
\address{}
\name{Stefanos Koffas$^{1,*}$, Luca Pajola$^{2,*}$, Stjepan Picek$^{3,1}$, Mauro Conti$^{2,1}$}    
 %TODO: Should we add our email addresses here?
 \address{$^1$Cybersecurity Group, Delft University of Technology, The Netherlands\\
          $^2$Security and Privacy Research Group, University of Padua, Italy\\
          $^3$Digital Security Group, Radboud University, The Netherlands
          %$^*$Equal contribution
          }
\begin{document}
\maketitle

\begin{abstract}
A backdoor attack places triggers in victims' deep learning models to enable a targeted misclassification at testing time. 
In general, triggers are fixed artifacts attached to samples, making backdoor attacks easy to spot.
Only recently, a new trigger generation harder to detect has been proposed: the \textit{stylistic} triggers that apply stylistic transformations to the input samples (e.g., a specific writing style).
\par
Currently, stylistic backdoor literature lacks a proper formalization of the attack, which is established in this paper. 
Moreover, most studies of stylistic triggers focus on text and images, while there is no understanding of whether they can work in sound. This work fills this gap. 
We propose JingleBack, the first stylistic backdoor attack based on audio transformations such as chorus and gain.
% We examine audio effect transformations to demonstrate their feasibility in the sound domain.
Using 444 models in a speech classification task, we confirm the feasibility of stylistic triggers in audio, achieving 96\% attack success.


%%% =========== OLD
% Backdoor attacks are a recent threat against deep learning models activated through specially crafted inputs (triggers) aiming at targeted misclassifications. 
% Most of the literature's triggers are static patterns applied on malicious samples\todo{this somehow implies that what we do is not applied on malicious samples}, while only recently, dynamic triggers have been proposed to make attacks stealthier and harder to defend. 
% Dynamic triggers can be generated in various ways, such as stylistic transformations. \todo{shall we say what it is}
% Stylistic triggers have been demonstrated to be powerful in domains like text and images, while there are no works investigating their feasibility on speech.

% This work proposes six stylistic triggers in the speech domain, consisting of distinct guitar effects. 
% We demonstrate the feasibility of our attack in a practical use case: the speech classification task. \todo{do we need this sentence? nothing new from the previous one}
% Our experiments analyze three distinct baseline models, four different poisoning rates, and two attack settings (clean and dirty labels). Our most effective style yielded an ASR larger than 84\% for the clean-label attack and more than 96\% for the dirty-label attack with only 1\% poisoned training samples.
% %Each experiment is repeated four times, for  total of 576 tested models. 
% \todo{add the outcomes}

% ==================== USENIX
% Data poisoning is a class of machine learning threats where an attacker injects malicious samples to affect the victim's model training. An attacker can leverage data poisoning to insert a backdoor in a deployed model that is activated when the model's input has a specific property (trigger). In general, backdoor attacks can be mounted with or without the \textit{label flipping} of the poisoned samples. \textit{Label flipping} or \textit{dirty-label attack}, requires that the attacker not only changes a few training samples but their labels too. On the contrary, \textit{clean-label attack} assumes that the adversary has no access to the labels.  While the first is more effective and requires less poisoned data, the second is stealthier and more realistic. Current state-of-the-art mainly focuses on producing backdoors with `static triggers', where malicious samples are crafted by adding a unique `signature' (e.g., inserting a patch into an image). While static triggers are effective, they can be easily detected and mitigated even through manual inspection of the dataset. 

% In this work, we propose ``Style Backdoor'', a novel dynamic backdoor for audio based on stylistic transformations. We use electric guitar sound effects as our triggers and compare the performance of clean-label and dirty-label attacks. 
% \todo{complete when the paper is done}
% \todo{Mention that our code is public.}
\end{abstract}

\section{Introduction}
\label{sec:introduction}

Machine learning became popular in the last decade with applications in critical domains like autonomous vehicles and financial predictions. 
It is thus essential to ensure its security, which led to the field of adversarial machine learning.
% Adversarial machine learning studies the security of machine learning applications .
%For example, \textit{evasion} attack manipulates input samples to produce misclassifications~\cite{evasion2004}.
In this work, we focus on \textit{backdoor} attacks, where the attacker embeds a secret functionality into the victim's model, which can be triggered at the testing time from specially crafted inputs~\cite{gu2017badnets}. These inputs include an attacker-chosen predefined property (\textit{trigger}) that activates the backdoor at testing time, resulting in a targeted misclassification. 
Usually, backdoors are inserted through a \textit{poisoning} attack, where the attacker controls a part of the training data~\cite{gu2017badnets}.

% Two poisoning strategies can be distinguished: \textit{dirty-label} and \textit{clean-label}. The former dictates a modification of the labels of the poisoned training samples, while the latter preserves their true labels. Despite the fact that the dirty-label attack produces stronger backdoors and requires less poisoned data, the clean-label poisoning is stealthier and more realistic as it can easily bypass manual inspection techniques~\cite{clean-label-backdoor-attacks}. 
% For example, crowdsourced datasets like Mozilla's Common Voice~\cite{common-voice-dataset} use volunteers to validate each provided training sample which is a pair of an audio file and its transcription. 
% When the attack is dirty-label, the volunteer can easily spot inconsistencies between the provided audio and its transcription. However, if the attack is clean-label, the trigger will remain unnoticed because the audio and its transcription will be correct.

% \textbf{Research Motivations.}
Backdoor triggers can be grouped into two major families: \textit{static}, when the trigger is a fixed pattern attached to the poisoned sample~\cite{gu2017badnets}, and \textit{dynamic} when the trigger's properties vary for each poisoned sample~\cite{dynamic-backdoor-attacks-against-ml-models,dynamic-backdoors-with-gap}. In general, dynamic triggers are stronger as they can be effective under different conditions and can potentially bypass state-of-the-art countermeasures~\cite{dynamic-backdoor-attacks-against-ml-models}.
Recently, a new way of creating dynamic triggers has emerged in the literature: stylistic triggers.
%Currently, only a few works successfully proposed dynamic trigger, and such triggers are the result of stylistic transformations. 
For instance, in computer vision, StyleGAN generates poisoned samples with styles based on a style image (e.g., Van Gogh painting)~\cite{deep-feature-space-trojan-attack-of-nns-by-controlled-detoxification}. On the other hand, in the text domain, the writing style is used as a
trigger~\cite{mind-the-style-of-text}.
%trigger~\cite{li2021hidden,mind-the-style-of-text,hidden-trigger-backdoor-attack-on-nlp-models-via-linguistic-style-manipulation}.\todo{do we even need to mention text domain and especially with 3 citations?}

\textbf{Motivation.}
Stylistic backdoors are powerful but not yet studied in the audio domain. Such backdoors highly depend on the targeted application resulting in domain-specific challenges. In audio, the most important challenge is to create a style that alters the original signal in a way that is distinguishable by the trained models but also keeps the audio quality at an acceptable level. To our best knowledge, no work has explored stylistic backdoors in audio, and we aim to fill this gap. 
We investigate the effect of six stylistic triggers in a speech classification task in both clean and dirty-label settings. 
Triggers are generated through electric guitar effects that do not require training of complex generative models~\cite{deep-feature-space-trojan-attack-of-nns-by-controlled-detoxification,mind-the-style-of-text}, making our attack easy to deploy.
Our contributions are:
\begin{compactitem}
    \item We propose and demonstrate the feasibility of stylistic backdoor attacks (denoted JingleBack) in the audio domain through electric guitar effects. For our experiments, we trained 444 models, and JingleBack reached up to 96\% attack success rate. 
    \item We are the first to formally describe stylistic backdoor attacks and establish a domain-agnostic framework that can be used for stylistic backdoors in any application.
\end{compactitem}


%%%==========================old========
% Data-driven approaches like Machine Learning (ML) have been confirmed as state of the art in many disciplines in the 2010s. 
% Their success is uncountable, as we can find their application in many industries like automotive~\cite{Wu_2017_CVPR_Workshops} and healthcare~\cite{dilsizian2014artificial}. 
% The market behind this technology is growing as well: as Reported by Statista\footnote{\url{www.statista.com/statistics/694638/worldwide-cognitive-and-artificial-intelligence-revenues/}}, this market was worldwide valued $\$300$ billion in 2021, with an expectation of $\$500$ billion in 2023. 
% Nevertheless, ML applications might be threatened by adversaries aiming to affect or control their execution. 
% For example, an adversary might manipulate the environment to produce misclassifications. This attack is called \textit{evasion}~\cite{evasion2004}.
% The area that studies the security of machine learning is called \textit{adversarial machine learning}.
% Since ML can be used in critical applications (e.g., autonomous vehicles), it is fundamental to identify new potential threats that might undermine the correct workflow of these algorithms. 

% In this work, we focus on \textit{backdoor} attacks. In this threat, the attacker embeds a secret functionality into the victim's model which can be triggered at testing time from specially crafted inputs. These inputs include an attacker-chosen predefined property, i.e., the trigger, which activates the backdoor. 


% The first backdoor attacks exploited two common ML deployment conditions~\cite{gu2017badnets}; dataset creation from untrusted sources and third party training (e.g. outsourced training and transfer learning).
%\begin{enumerate}
%    \item Complex ML models require mammoth datasets. The creation of such datasets is difficult in a controlled environment and data should be retrieved from untrusted parties.
%    \item Complex ML models require non-trivial computational resources. As a consequence, many companies uses pre-trained models, or they outsource their training to third parties. In both cases, the third party that produces the model might be colluded. 
%\end{enumerate}
% Two poisoning strategies can be distinguished: \textit{dirty-label} and \textit{clean-label} attacks. The first involves changing the labels of the poisoned training samples. In contrast, in the latter case, the adversary preserves the true labels of the poisoned samples. Despite the fact that the dirty-label attack produces stronger attacks and requires less poisoned data, the clean poisoning is stealthier and more realistic~\cite{clean-label-backdoor-attacks} as it can easily bypass manual inspection techniques. For example, crowdsourced datasets like Mozilla's Common Voice~\cite{common-voice-dataset} use volunteers to validate each provided training sample. When the attack is dirty label, the volunteer can easily identify inconsistencies between the provided audio file and the corresponding text. If the attack is clean label though, the trigger will remain unnoticed because the audio and the corresponding text will be correct.

% Similarly, we can distinguish two trigger families. \textit{Static}, when the trigger is a fixed pattern attached to the poisoned sample~\cite{gu2017badnets}, and \textit{dynamic} when the trigger's properties vary for each poisoned sample~\cite{dynamic-backdoor-attacks-against-ml-models,dynamic-backdoors-with-gap}. In general dynamic triggers are stronger as they can be effective even under different conditions and can potentially bypass state-of-the-art countermeasures~\cite{dynamic-backdoor-attacks-against-ml-models}.

% For this reason a new approach for trigger generation has been recently emerged. These triggers are based on style transformations that exploit a property on the model's input instead of a static feature like a pixel pattern in a fixed location.
% In~\cite{deep-feature-space-trojan-attack-of-nns-by-controlled-detoxification} the authors used StyleGAN to poison images with a specific style and through ``controlled detoxification'' ensured that the trained model learnt the style instead of just a shallow feature like the color. However, they assumed white-box access to the model's training and that the adversary is able to train a GAN which is not always a realistic scenario.
% \cite{mind-the-style-of-text} introduces textual backdoors and adversarial attacks based on linguistic styles. In this work only the dirty-label backdoor attack was applied in models trained from scratch. This approach has been extended in~\cite{hidden-trigger-backdoor-attack-on-nlp-models-via-linguistic-style-manipulation} with style-aware injection algorithms that amplified the model's perception of trigger style. Apart from backdoor attacks, style transfer techniques have been also used for evasion attacks in video classification~\cite{stylefool} and computer vision~\cite{adversarial-camouflage-hiding-physical-world-attacks-with-natural-styles}.

% In this work we use electric guitar effects as our triggers to create the first (to the best of our knowledge) style-based audio backdoor. Additionally, we investigated the backdoor's behavior for both clean-label~\cite{clean-label-backdoor-attacks} and dirty-label~\cite{gu2017badnets} setups. In particular, our contributions are: \todo{fill a bullet list with contributions once we are done}. 

% Mind the style of text
%They generated both adversarial and backdoor attacks based on style.
%Through an iterative process they found a text style that is clearly distinguished from the style of the training samples.

% Controlled detoxification Notes.
%\begin{itemize}
%    \item Similariites:
%    \begin{itemize}
%        \item They claim that the pixel patterns are easily detectable because they may seem unnatural. Thus, such triggers are not stealthy, and a manual inspection could identify them. By using a function that embeds a property to the input samples, we can create more difficult to detect trojans.
%        \item  They also mention that style triggers (or, as they call it feature space triggers) are dynamic triggers as they depend on each input sample. And what does this mean???
%        \item They checked how a clean model classifies poisoned inputs to measure the trigger's stealthiness. If the trigger is stealthy, the clean model should recognize the features of the poisoned input's original class and classify it correctly. 
%    \end{itemize}
%    \item Differences:
%    \begin{itemize}
%        \item The threat model they use is different. They use a white box threat model as they assume the attacker has access to both the model and the training dataset, and can control the training process (for the detoxification).
%        \item They did not use a clean-label attack as we did. This difference may lead to stealthier triggers that will not require the detoxification process to avoid defenses.
%        \item They trained a CycleGAN that transforms images from domain A (clean) to B (poisoned with a different style) and from B to A based on [3]. We did not use a GAN, which can be a more realistic attack scenario as the attacker may not have the resources or the knowledge to train a GAN in the outsourced training or MLaaS paradigm.
%    \end{itemize}
%    \item We can also borrow the following ideas:
%    \begin{itemize}
%        \item They said that in general, it is challenging to control what a model learns through data-poisoning-based backdoor attack. It would be nice to try to check somehow what the model learns from our attack.
%        \item They provide a nice definition of stealthy and robust backdoors. We can use this definition in our paper and experiments too.
%    \end{itemize}
%\end{itemize}




\section{Background}
\label{sec:background}

\textbf{Evasion Attack.}
The attacker aims to find a small perturbation $\epsilon$ for a sample $x$ to produce a misclassification in a target classifier $\mathbf{C}$.
The definition of $\epsilon$ is task-dependent~\cite{evasion2004}.
For images, $\epsilon$ is usually an additive noise at a pixel level computed via a gradient-based procedure~\cite{fgsm}.
For audio, $\epsilon$ is a small waveform computed through an optimization process~\cite{audio-adversarial-examples-targeted-attacks-on-speech-to-text}.


\textbf{Poisoning Attack.}
Attackers able to manipulate victims' training data $\mathcal{D}^{tr}$ can inject adversarial samples to increase $\mathbf{C}$'s testing error and decrease the model's performance~\cite{biggio_poisoning}.
Data manipulation is a concrete threat because building datasets often involves untrusted sources, and complex model training is outsourced to third parties.

\textbf{Backdoor Attack.}
Data poisoning leads to \textit{backdooor attacks}. The attackers insert backdoors in the model resulting in targeted misclassifications~\cite{gu2017badnets}.
Most backdoors are source-agnostic, as the trigger can be applied to any dataset's class. There are two different approaches to creating such a backdoor: \textit{dirty-label}~\cite{gu2017badnets}, where the adversary adds the trigger to some training samples but also alters their labels, and \textit{clean-label}~\cite{clean-label-backdoor-attacks}, where the attacker only poisons samples from the target class.
The dirty-label attack produces stronger backdoors and requires less poisoned data, but the clean-label attack is a more realistic threat as poisoned samples cannot be easily identified by manual inspection~\cite{clean-label-backdoor-attacks}.
For example, crowdsourced datasets like Mozilla's Common Voice~\cite{common-voice-dataset} use volunteers to validate each audio file and its transcription.
When the attack is dirty-label, the volunteer can easily spot inconsistencies between the provided audio and its transcription. However, in a clean-label attack, the trigger will remain unnoticed because the audio and its transcription will be correct.
This work explores the effects of both approaches on a style-based backdoor attack in speech classification.

\section{Style Backdoor Attack}
\label{sec:threat}

% \subsection{Attacker's Capabilities}
\subsection{Threat Model}

\textbf{Attacker's Capabilities.}
This work assumes a gray-box threat model where the adversary has access to a small portion of the training data, which can be altered without restriction, but has no knowledge of the training algorithm and the model's architecture. 
Such capabilities are realistic as modern datasets are based on crowdsourcing~\cite{common-voice-dataset} and malicious data may evade security checks~\cite{large-dataset-pyrrhic-win}. 

\textbf{Attacker's Aim.}
The attacker's purpose is to insert a secret functionality into the deployed model, which is activated when the trigger is present in the model's input. Usually, this functionality causes targeted misclassifications with a very high probability. Furthermore, the model should behave normally for any other input to avoid raising suspicions. 

\subsection{Attack Formulation}
Prior studies in backdoor attacks mainly focus on \textit{static triggers}, where the backdoor generation procedure trivially inserts a trigger $\epsilon$ in a part of training samples. 
Such addition is sample-independent.
Formally, let $x$ be a sample, $\epsilon$ a trigger, and the backdoor generative function $\mathcal{F}$ can be described as:
\begin{equation}\label{eq.backdoor_static}
    \mathcal{F}(x, \epsilon) = x + \epsilon.  
\end{equation}
In BadNets~\cite{gu2017badnets} (computer vision domain), the trigger is a patch at a pixel level that replaces a part of the original sample. In audio, a static trigger could be a tone superimposed on the original sample~\cite{can-you-hear-it}.
Static triggers can thus be considered as a \textit{constant} parameter of the attack, as a batch of benign samples $\{x_0, ..., x_n\}$ is transformed in nothing more than $\{x_0 + \epsilon, ..., x_n + \epsilon\}$. 
Thus, the victim's model $\mathbf{C}_{back}$ learns to associate the static pattern $\epsilon$ to a target label $y*$.

In this work, we approach the backdoor attack from an orthogonal perspective: can the trigger be something that the sample \textit{is} rather than something the sample \textit{has}? 
Static backdoors answer the \textit{has} proposition: the poisoned sample contains a specific (and constant) pattern that $\mathbf{C}_{back}$ associates with the target class.
Answering the \textit{is} is more complex: this is how a sample is presented, and we can think of it as a latent variable globally describing that sample. 
In our study, global descriptors are defined by \textit{stylistic properties}. Such properties can be the image exposure or the saturation level (computer vision), the writing complexity (text), or the signal's pitch  (audio).
%For the image domain, a stylistic property might be the image exposure or the saturation level. In the text domain, stylistic properties are the sentiment (e.g., positive or negative) or the writing complexity. 
%In the sound domain, the signal's pitch could potentially be the trigger.
%
% why am I writing this complex thing? Well, we want to somehow demonstrate 
% why a stylistic property can become a backdoor. Otherwise 
%
Once the stylistic trigger $\epsilon$ is identified, we need a function $\mathcal{S}_\epsilon$ that allows embedding such a style to a given sample $x$. Formally, Eq.~\eqref{eq.backdoor_static} changes to:
\begin{equation}\label{eq.backdoor_style}
    \mathcal{F}(x, \mathcal{S}_\epsilon) = \mathcal{S}_\epsilon(x).  
\end{equation}
We explore different $\mathcal{S}_\epsilon$ for speech recognition in the following sections. 
With the formulation given in Eq.~\eqref{eq.backdoor_style}, the batch of adversarial samples$\{\mathcal{S}_\epsilon (x_0), ...,$ $\mathcal{S}_\epsilon (x_n)\}$ now contains dynamic triggers since the outcome of the backdoor generation varies based on inputted sample. 
Now, suppose that $\mathcal{S}_\epsilon$ exists.
The stylistic backdoor is effective if the model $\mathbf{C}_{back}$ learns an association between $\mathcal{S}_\epsilon$ and the target class $y*$.
This problem can be formulated with two sub-questions:
\begin{compactenum}
    \item Can $\mathbf{C}$ learn to recognize the presence of $\mathcal{S}_\epsilon$? We need proof or at least an indication that stylistic properties can be learned by the victim's model.  
    \item Can we let $\mathbf{C}$ learn the association between $\mathcal{S}_\epsilon$ and $y*$? If the previous question is positively answered, we need to understand further how to create the backdoor and what are the possible obstacles at this stage.  
\end{compactenum}
Geirhos et al.~\cite{geirhos2018imagenettrained} answered the first question, showing that CNNs focus more on textures rather than object shapes.
For example, an image with `cat' shapes and `elephant' texture is classified as an elephant. 
Similarly, styles can be learned in the speech domain~\cite{grinstein2018audio, AlBadawy2020}.
% We thus can affirm that a DNN can learn stylistic properties. 

Finally, we need to understand what are the conditions to create the stylistic backdoor in $\mathbf{C}$.
% The poisoning process follows the standard methodology described in Section~\ref{sec:background}. 
In source-agnostic backdoors, the trigger is present only in one of the classes of the training data. 
This condition is easily satisfied by the classic backdoor attacks injecting artifacts as triggers.
Conversely, with stylistic backdoors, such conditions might not always be met. 
For example, high exposure (image domain) and low-frequency tunes (audio domain) might be present in many classes of clean data. 
% For example, in the image domain, a high exposure level might not be an appropriate stylistic trigger since this might be present in many classes of clean data. \todo{why not talk about sound? It would motivate this work better}
% Let us consider the image domain and the high exposure level as the stylistic trigger: it is likely that `high exposure' pictures are a common trait of multiple classes. \todo{so what}
We conclude that the choice of $\mathcal{S}_\epsilon$ impacts attack success. 
In our experiments, we use stylistic functions that are unlikely to be present in the training data.


%\input{Sections/04-Case study: Image Classification}
% \input{Sections/05-Case Study:CV Multiclass}
\section{JingleBack Design}
\label{sec:speech}

% In this section, we present our Style Backdoor in speech classification. We introduce each transformation we used in~\Cref{ssec:sound-style-gen}, we describe our experimental setup in~\Cref{ssec:sound-setup} and finally discuss our experimental results in~\Cref{ssec:sound-results}.

\subsection{Stylistic Generation}
\label{ssec:sound-style-gen}
% \todo{Mention why we chose all these styles. We wanted something simple that is widely known and easily made in real life through pedalboards}
% General notes in https://sound.eti.pg.gda.pl/student/eim/synteza/adamx/eindex.html
This work focuses on backdoors in the audio domain aiming into an easy-to-deploy attack. Thus, we investigate if such an attack can be implemented through simple digital music effects like chorus or gain. 
% We thus need to find stylistic triggers that are easy to generate, such as music effects like gain or chorus.
We used Spotify's pedalboard~\footnote{\url{https://github.com/spotify/pedalboard}} to implement six styles by combining effects like PitchShift, Distortion, Chorus, Reverb, Gain, Ladderfilter, and Phaser. We explain the adopted styles briefly.

\textbf{PitchShift.} It increases the pitch of the original audio by a number of semitones without affecting its duration. The pitch is the lowest frequency $f_0$ of a signal $S$~\cite[p.~xv]{music-in-theory-and-practice}. A semitone $sem$ is the smallest musical interval used in music denoting a different tone. PitchShift can be defined as:
%is described with the following equation in a high-level way:\todo{what is a high level way?}
\begin{equation}
\label{eq:pitchshift}
    PitchShift(S(f_0), sem) = S(f_0\cdot e^{sem/12}).
\end{equation}

\textbf{Distortion.} It applies a distortion with a $tanh()$ waveshaper. $\beta$ controls the signal's amplitude increase. 
%Its functionality is shown in Eq.~\eqref{eq:distortion}:
\begin{equation}
\label{eq:distortion}
    Distortion(S(t), \beta) = \beta \cdot tanh(S(t)).
\end{equation}

\textbf{Chorus.} It imitates a group of musicians that play the same sound
by superimposing many versions of the same sound that are slightly out of time and tune. Pedalboard's chorus uses one unsynchronized version with the provided delay $d$. 
In the following equation, $\alpha$ controls chorus amplitude.
% A more formal definition of this effect is shown in Eq.~\eqref{eq:chorus}, where $\alpha$ controls the chorus' amplitude.
\begin{equation}
\label{eq:chorus}
   Chorus(S(t), d, \alpha) = S(t) + \alpha \cdot S(t - d).
\end{equation}

% https://www.sciencedirect.com/science/article/pii/0022460X71903920
% http://www.music.mcgill.ca/~gary/courses/papers/Moorer-Reverb-CMJ-1979.pdf

\textbf{Reverberation.} It imitates the reflection of reproduced sound on various surfaces. 
% It results in the persistence of a sound in a room, even after the sound is not played anymore. 
% Usually, this sound decays fast, but its exact time depends on the environmental properties. 
Spotify's pedalboard is based on FreeVerb\footnote{\url{https://ccrma.stanford.edu/~jos/pasp/Freeverb.html}}, which uses eight parallel Schroeder-Moorer filtered-feedback comb-filters that create eight delayed versions of the input signal that are added and fed into four Schroeder all-pass filters in series. This relation is described in Eq.~\eqref{eq:reverb}, where $AP_{1-4}$ is the combined effect of the four cascaded all-pass filters and $CF_{i}$ is the $i^{th}$ comb-filter.
% This info was also verified from ./modules/juce_audio_basics/utilities/juce_Reverb.h
\begin{equation}
\label{eq:reverb}
    Reverb(S(t)) = AP_{1-4}[\sum_{i=1}^{8}CF_{i}(S(t))].
\end{equation}

\textbf{Gain.} 
It changes the signal amplitude by a factor $G$. 
%This effect is formally described in Eq.~\eqref{eq:gain}, where $G$ is the amount of change applied:
\begin{equation}
\label{eq:gain}
    Gain(S(t), G) = G\cdot S(t).
\end{equation}


% Maybe mention something about the 12db/octave fromhttps://reverb.com/news/a-guide-to-synth-filter-types-ladders-steiner-parkers-and-more
% Check the following for more info:
% https://api.moogmusic.com/sites/default/files/2018-06/MF_101_Manual.pdf
% http://sdiy.org/destrukto/notes/moog_ladder_tf.pdf
% Nice formulas about the transfer function of ladder filter: http://sdiy.org/destrukto/notes/moog_ladder_tf.pdf , https://www.dafx.de/paper-archive/2020/proceedings/papers/DAFx2020_paper_70.pdf , https://ieeexplore-ieee-org.tudelft.idm.oclc.org/stamp/stamp.jsp?tp=&arnumber=1162522 , https://ieeexplore.ieee.org/stamp/stamp.jsp?tp=&arnumber=6892946
% blog posts: https://www.uaudio.com/blog/moog-multimode-filter-design/ , https://reverb.com/news/a-guide-to-synth-filter-types-ladders-steiner-parkers-and-more
% General notes about the filters https://www.infineon.com/dgdl/Infineon-LPF4V_001-56219-Software%20Module%20Datasheets-v02_01-EN.pdf?fileId=8ac78c8c7d0d8da4017d0f97c58b06c6

\textbf{Ladder.} Spotify pedalboard implements a Moog Ladder filter $L(\cdot)$~\cite{moog-ladder-filter}. In our experiment, we use the \textit{high-pass} version as it had a clear effect on our samples.  
% The pedalboard implementation relies on a digital filter based on the Moog filter $L(\cdot)$~\cite{moog-ladder-filter}, and it supports three different operating modes (high-pass, low-pass, and band-pass). 
% In each of these modes, the filter blocks a different set of frequencies.
% Our experiments used a high-pass filter as it had a clear effect on our samples compared to other types of filters. 
% We used a high-pass filter that blocks all frequencies up to the cutoff frequency because its effect on the original audio was slightly more clear compared to the effect that other filters had.
% As its frequency response is not perfectly linear, it can introduce some distortion to the original signal. However, in our experiments, we could not observe any distortions, only a different sound pitch. 
%We model its functionality as shown in Eq.~\eqref{eq:ladder}, where 
\begin{equation}
\label{eq:ladder}
    Ladder(S(t)) = \alpha \cdot L(S(t)).
\end{equation}

\textbf{Phaser.} It is based on special filters that can change the frequencies they block over time through a low-frequency oscillator $P$. The phaser superimposes the original signal with its altered version. 
This results in a soft-moving sound. % that is very popular in the music industry.
% A simplified mathematical model of this effect is shown in Eq.~\eqref{eq:phaser}. 
In the following equation, $\alpha$ controls the effect's intensity.
% rate makes the sweep faster, depth makes the effect more intense, and resonance (maybe center frequency)  emphasizes certain tones in the sweep.
\begin{equation}
\label{eq:phaser}
    Phaser(S(t)) = S(t) + \alpha \cdot P(t, S(t)).
\end{equation}



%Our first style simply increases the pitch of the original audio by ten semitones without affecting its duration. The pitch is the lowest frequency of a signal which is often perceived as the loudest too~\cite[p.~xv]{music-in-theory-and-practice}. A semitone is the smallest musical interval used in music denoting a different tone.
%\begin{equation}
%\label{eq:pitchshift}
%    PitchShift(S(f_0), sem) = S(f_0\cdot e^{sem/12}) 
%\end{equation}
%
%Our first style uses chorus, distortion, and reverb. In general, the chorus effect imitates a group of musicians that play the same sound. Even though they are mostly synchronized, this synchronization is rarely perfect. Thus, the chorus audio effect superimposes many versions of the same sound that are slightly out of time and tune. Pedalboard's chorus uses only one unsynchronized version with a delay of 15$ms$. A more formal definition of this effect is shown in Equation~\eqref{eq:chorus}, where $\alpha$ controls the chorus' amplitude, and it is equal to 0.25 in our case.
%
%\begin{equation}
%\label{eq:chorus}
%   Chorus(S(t)) = S(t) + \alpha \cdot S(t - 15ms).
%\end{equation}
%
%The distortion effect applies a soft distortion with a $tanh()$ waveshaper. In our case, a variable also controls the signal's amplitude increase. Its functionality is shown in Equation~\eqref{eq:distortion}:
%\begin{equation}
%\label{eq:distortion}
%    Distortion(S(t)) = \beta \cdot tanh(S(t)).
%\end{equation}
%
%% https://www.sciencedirect.com/science/article/pii/0022460X71903920
%% http://www.music.mcgill.ca/~gary/courses/papers/Moorer-Reverb-CMJ-1979.pdf
%Reverberation is created from the reflection of reproduced sound on various surfaces. It results in the persistence of a sound in a room, even after the sound is not played anymore. Usually, this sound decays fast, but its exact time depends on the environmental properties. Pedalboard's reverb applies this effect to the original input signal. The reverb generated by Spotify's pedalboard module is based on FreeVerb~\footnote{\url{https://ccrma.stanford.edu/~jos/pasp/Freeverb.html}}, which uses eight parallel Schroeder-Moorer filtered-feedback comb-filters that create eight delayed versions of the input signal that are added and fed into four Schroeder all-pass filters in series. This relation is described more formally in Equation~\eqref{eq:reverb}, where $AP_{1-4}$ is the effect of the four cascaded all-pass filters and $CF_{i}$ is the $i^{th}$ comb-filter.
%
%% This info was also verified from ./modules/juce_audio_basics/utilities/juce_Reverb.h
%\begin{equation}
%\label{eq:reverb}
%    Reverb(S(t)) = AP_{1-4}[\sum_{i=1}^{8}CF_{i}(S(t))].
%\end{equation}
%
%The total effect, in that case, is shown in Equation~\eqref{eq:sound-style1}:
%\begin{equation}
%\label{eq:sound-style1}
%    Style1(S(t)) = Reverb(Distortion(Chorus(S(t)))).
%\end{equation}
%
%For our second style, we used the following effects: gain, a ladder filter, and a phaser. The gain effect changes the amplitude of the input signal. This effect is formally described in Equation~\eqref{eq:gain}, where $G$ is the amount of change applied:
%\begin{equation}
%\label{eq:gain}
%    Gain(S(t)) = G\cdot S(t).
%\end{equation}
%
%% Maybe mention something about the 12db/octave fromhttps://reverb.com/news/a-guide-to-synth-filter-types-ladders-steiner-parkers-and-more
%% Check the following for more info:
%% https://api.moogmusic.com/sites/default/files/2018-06/MF_101_Manual.pdf
%% http://sdiy.org/destrukto/notes/moog_ladder_tf.pdf
%% Nice formulas about the transfer function of ladder filter: http://sdiy.org/destrukto/notes/moog_ladder_tf.pdf , https://www.dafx.de/paper-archive/2020/proceedings/papers/DAFx2020_paper_70.pdf , https://ieeexplore-ieee-org.tudelft.idm.oclc.org/stamp/stamp.jsp?tp=&arnumber=1162522 , https://ieeexplore.ieee.org/stamp/stamp.jsp?tp=&arnumber=6892946
%% blog posts: https://www.uaudio.com/blog/moog-multimode-filter-design/ , https://reverb.com/news/a-guide-to-synth-filter-types-ladders-steiner-parkers-and-more
%% General notes about the filters https://www.infineon.com/dgdl/Infineon-LPF4V_001-56219-Software%20Module%20Datasheets-v02_01-EN.pdf?fileId=8ac78c8c7d0d8da4017d0f97c58b06c6
%The ladder filter used in pedalboard is a digital filter based on the Moog filter~\cite{moog-ladder-filter}, and it supports three different operating modes (high-pass, low-pass, and band-pass). In each of these modes, the filter blocks a different set of frequencies. We used a high-pass filter that blocks all frequencies up to the cutoff frequency because its effect was slightly more clear compared to the others. \todo{what does this mean?}
%As its frequency response is not perfectly linear, it can introduce some distortion to the original signal. However, in our experiments, we could not observe any distortions, only a different sound pitch. We model its functionality as shown in Equation~\eqref{eq:ladder}, where 
%\begin{equation}
%\label{eq:ladder}
%    Ladder(S(t)) = \alpha \cdot L(S(t)).
%\end{equation}
%
%The phaser is based on special filters that can change the frequencies they block over time through a low-frequency oscillator. The phaser superimposes the original signal with its altered version. This results in a soft moving sound that is very popular in the music industry. A simplified mathematical model of this effect is shown in Equation~\eqref{eq:phaser}. In this equation, $\alpha$ controls the effect's intensity, and $P$ is also a function of $t$ as it changes over time.
%
%% rate makes the sweep faster, depth makes the effect more intense, and resonance (maybe center frequency)  emphasizes certain tones in the sweep.
%
%\begin{equation}
%\label{eq:phaser}
%    Phaser(S(t)) = S(t) + \alpha \cdot P(t, S(t)).
%\end{equation}
%
%In total, the second style is shown in Equation~\eqref{eq:sound-style2}:
%\begin{equation}
%\label{eq:sound-style2}
%    Style2(S(t)) = Phaser(Ladder(Gain(S(t)))).
%\end{equation}

\subsection{Experimental Settings}
\label{ssec:sound-setup}

\textbf{Dataset and Features.} 
Experiments are conducted on Google's Speech Commands dataset~\cite{speech-commands-dataset}. This is an audio dataset targeting keyword spotting tasks, and its data belong to one of 30 classes (``yes'', ``no'', ``up'', ``down'', etc.) 
%This dataset consists of 64\,724 short audio clips of spoken keywords. However, we discarded the files that lasted less than one second to avoid having inputs with variable size, resulting in 58\,252 audio clips. 
We used the Mel-frequency cepstral coefficients (MFCCs) as input features because they are rather accurate in emulating the human vocal system and widely used~\cite{can-you-hear-it}.
We use common settings described in the literature~\cite{generalized-end-to-end-loss-for-speaker-verification}, i.e., 40-mel bands, a step of 10ms, and a window length of 25ms.

%Increasing room size increases the reverbation time.
\textbf{Backdoor.} 
We split our data into training, validation, and test sets in a 64/16/20 way. For our backdoor, we poisoned up to 1\% of the training data and used two backdoor settings: clean and dirty label attacks.
We chose the class ``yes'' as the backdoor class without loss of generality since we expect similar behavior regardless of the target class.
%Thus, in the clean-label we poison only samples that belong in this class, while for the dirty label we poison samples from every class and we change their label to ``yes''. 
Triggers are generated with the six styles as previously described. Style parameters are shown in~\Cref{tab:styles}. 
Parameters are selected to limit sample distortion and preserve their quality.
% Parameters are chosen by the authors to guarantee limited .\todo{explain}

\begin{table}[h]
    \centering
    \caption{Stylistic triggers deployed in our experiments.}
    \label{tab:styles}
    \resizebox{0.35\textwidth}{!}{%
        \begin{tabular}{c|c}
        \toprule
        \textbf{Style} & \textbf{Effect} \\ \toprule
        0 & $PitchShift(S, 10)$ \\ \hline
        1 & $Distortion(S, 30dB)$ \\ \hline
        2 & $Chorus(S, 10ms, 5)$ \\ \hline
        3 & $Chorus(Distortion(PitchShift(S, 10), 20dB), 8ms, 5)$ \\ \hline
        4 & $Reverb(Distortion(Chorus(S, 15ms, 0.25), 20dB))$ \\ \hline
        5 & $Phaser(Ladder(Gain(S, 12dB)))$ \\ \bottomrule
        \end{tabular}
    }
\end{table}

\textbf{Models.} We used three different models, two CNNs (one small and one large) and one LSTM as described in~\cite{can-you-hear-it}. We repeated each of our experiments four times to limit the effects of randomness. 
The three models are trained for a maximum of 300 epochs.
We used an early stopping with a patience of 20 epochs based on the validation loss.
In total, by considering stylistic triggers (6), backdoor settings (2), models (3), poisoning rates (3), and repetitions (4), 432 poisoned models are trained. We also trained 12 clean models (4 repetitions for each model) to use as a reference when investigating the backdoor's effect on the original task.

\section{Experimental Results}
\label{ssec:sound-results}

\subsection{Effect on Clean Accuracy}
% The backdoor attack's objective is to insert a secret functionality into a victim's models while preserving their overall performance on unpoisoned data. 
We first verify that the backdoored models have comparable performance to their clean versions.
Clean models show, on average, high performance (expressed as F1-score): large CNN $93.8\pm0.2$, small CNN $87.2\pm0.3$, and LSTM $90.8\pm1.1$.
We notice that our attack is stealthy since we observed only a small drop in the performance of the backdoored models.
On average, models drop $0.24\pm0.9$ pp (points percentage) in the F1-score, while the worst model drops 4.87 pp.
Among the 432 poisoned models, the performance drop is $>2$ pp in 23 cases and decreases to 10 cases when $>3$ pp.
% Backdoor attacks poison victims' model by injection malicious samples at training time. 
% Before testing triggers effectiveness, we need to prove that the overall poisoned model's performance under legitimate testing samples preserve 

% \subsection{Degradation}
% In this section, we will discuss the results of our stylistic backdoor attack for speech classification. First, we have to verify that the backdoor is not affecting the original task. If the model's performance is affected by the backdoor, the user would become suspicious and stop using the poisoned model. For that reason, we plot the weighted F1 score (multiclass classification) of our models (poisoned and clean) in Figure~\ref{fig:sound-f1}. From that figure, we see that the performance of our models remains unaffected after the backdoor insertion. The number of poisoned samples used is very small (up to 2\%) and is not enough to affect the original task. From this figure, we also see that all the models perform similarly for the original task. 
% \todo{Maybe this is not a good metric as I poisoned many samples from only one class. I should check how the classification of that class is affected.}

% \begin{figure}[ht]
%     \centering
%     \begin{subfigure}[b]{0.40\textwidth}
%         \centering
%         \includegraphics[width=\textwidth]{Figures/sound/f1_score_clean-label.pdf}
%         \caption{Clean-label}
%         \label{fig:f1-clean-label}
%     \end{subfigure}
%     \hfill \\
%     \begin{subfigure}[b]{0.40\textwidth}
%         \centering
%         \includegraphics[width=\textwidth]{Figures/sound/f1_score_dirty-label.pdf}
%         \caption{Dirty-label}
%         \label{fig:f1-dirty-label}
%     \end{subfigure}
%     \caption{Models overall performance (F1-score) at the varying of the training poisoning ratio for speech classification. By averaging over the different styles we see that the original task is not affected by the backdoor insertion.}
%     \label{fig:f1}
% \end{figure}

\subsection{Evasion}
Cao et al.~\cite{stylefool} investigated the effect of stylistic transformation as an evasion attack for computer vision applications. 
We thus aim to investigate this effect in the audio domain.
% An interesting side-effect of the stylistic trigger is the generation of adversarial samples that are out-of-distribution, and, as a consequence, they might evade victim's unpoisoned models.
% Cao et al.~\cite{stylefool} recently investigated this phenomenon in the computer-vision domain.
% We thus aim to understand if our acoustic stylistic triggers can result in an untargeted evasion attack. 
We compute the misclassification percentage of unpoisoned models under malicious settings. 
Table~\ref{tab:evasion} shows the average evasion percentage at the varying of the model and style. 
We can notice that in seven out of 18 cases, our acoustic transformation produces an evasion rate $>70.0\%$. We thus affirm that stylistic transformation can produce strong evasion attacks.

\begin{table}[!htpb]
\centering
\caption{Average evasion [\%] at the varying of the model architecture and style. In bold, the best results.}
\resizebox{0.30\textwidth}{!}{%
\begin{tabular}{c|ccc} \toprule
\textbf{Style} & \textbf{Large-CNN} & \textbf{Small-CNN} & \textbf{LSTM} \\ \toprule
0 & \res{\textbf{81.2}}{1.7} & \res{\textbf{76.6}}{0.9} &\res{\textbf{85.8}}{2.0}  \\
1 & \res{31.0}{1.5} & \res{49.1}{1.5} &\res{42.7}{7.0}  \\
2 & \res{15.8}{0.9} & \res{24.4}{1.4} &\res{17.9}{1.2} \\
3 & \res{\textbf{84.2}}{1.0} & \res{\textbf{83.2}}{0.7} &\res{\textbf{86.1}}{2.3} \\
4 & \res{28.5}{1.1} & \res{45.4}{1.0} &\res{41.3}{3.3} \\ 
5 & \res{42.8}{2.4} & \res{\textbf{70.2}}{2.4} &\res{50.7}{6.7} \\\bottomrule
\end{tabular}
}
\label{tab:evasion}
\end{table}


\subsection{Backdoor}
We finally analyze the performance of our proposed stylistic backdoor under clean-label and dirty-label settings. \Cref{fig:asr} shows the results. The y-axis shows the attack success rate (ASR), which represents the percentage of successfully triggered backdoors over a number of tries, and the x-axis shows the poisoning rate, which is the percentage of the poisoned training samples used for our attack.
\begin{figure}[!htpb]
    \centering
    \includegraphics[width = \linewidth]{Figures/backdoor2.pdf}
    \caption{Backdoor attack success rate.}
    \label{fig:asr}
\end{figure}
Our results are in line with the literature~\cite{can-you-hear-it, clean-label-backdoor-attacks}.
For instance, the higher the poisoning rate, the higher the ASR because the poisoned models have more samples to learn the trigger~\cite{can-you-hear-it}.
Additionally, the dirty-label attack is more effective as it almost always results in higher ASR than the clean-label attack, as shown in~\cite{clean-label-backdoor-attacks}.
In particular, dirty-label attacks include cases in which even a small poisoning rate ($0.1\%$) is effective (ASR $>50\%$): examples are style 2 on Large-CNN, style 3 in Large-CNN, and style 5 in both Large-CNN and Small-CNN.
Furthermore, Large-CNN -- the most performing model on average -- is the most vulnerable, showing that backdoor effectiveness is connected to the model's ability to learn. 
\par
Generally, we can notice that styles have different effects (e.g., style 3 is always more effective than style 1), leading to the following observations.
(i) The clean-label attack is effective only with styles 3 and 5. 
(ii) Dirty label attack shows appreciable performance in all cases. 
(iii) The addition of effects does not always result in a performance boost (e.g., style 0 outperforms style 4 in clean-label settings). 
\par
We explain our observations with the following considerations. 
In the dirty-label attack, the poisoned samples are generally very different from the samples of the target class as they originally belonged to different classes. As a result, the distance between these samples in the feature space is large and easy to learn by our models, even by applying simple effects to them. However, in the clean-label attack, the poisoned samples belong to the target class, and there is a higher probability that their features are not very different from the clean samples. For that reason, the dirty label attack is more effective than the clean-label attack, which explains (i) and (ii). We believe that adding more effects is not the best way to create distinguishable triggers because what is most important is to change the signal's representation in the feature space. For that reason, a simpler transformation that alters the frequency representation of the original signal, like $PitchShift$ may result in a larger difference in the MFCCs as they use Fourier transform internally. This explains (iii). 

%In the clean label scenario the attacker needs to spend more time designing the triggers. As we poison only samples from the original class.



% Interestingly, style 3 -- the combination of styles 0, 1, and 2 -- results in strong dirty and clean label attacks. \todo{why}
% The results further suggest that triggers have different effects, e.g., style 3 is always more effective than style 1. 
% The clean-label attack is effective with styles 3 and 5. 
% Furthermore, the addition of effects does not always result in a performance boost (e.g, style 0 outperforms style 4 in clean-label settings). 
% Interestingly, style 3 -- the combination of styles 0, 1, and 2 -- results in strong dirty and clean label attacks. \todo{why}
\par
% The results further suggest that trigger styles are effective in different ways. (e.g., style 3 is always more effective than style 1). \todo{why?}
% For the clean-label attack, the ASR is very low for styles 0, 1, 2, and 4.
% However, we see that just a simple transformation like $PitchShift$ (style 0) performs better than a style with multiple transformations (style 4). 
% This indicates that even simple styles that affect the frequency representation of the signals like style 0 ($PitchShift$) could result in a more effective attack than more complex styles like style 4 ($Chorus$, $Distortion$, and $Reverb$). 
% \todo{luca: not clear the following sentence}
% This behavior needs further investigation because, for the dirty label attack, we see a different behavior. Styles 4 and 0 yield the same ASR even though style 0 performs far better in the clean label attack.\todo{unclear}
% Styles 3 and 5 are the most successful. They confirm that a style-based backdoor can be implemented in both attack scenarios by combining a handful of simple transformations. In both these styles, the frequency domain representation of the signal is affected indicating that this may be good practice for attackers. However, further investigation is required to find the most important characteristics of a style.

% \todo{vague conclusions. whatever we conclude, we also say further investigation is needed. But what is the point of this paper then?}

%                     |  clean label   | dirty label
%Pitchshift(S, 10)    | up to 26\%     | up to 70\%|
%Distortion(S, 30db)  | up to 13\%     | up to 60\%|
%Chorus(S, 10ms, 5)   | up to 10\%     | up to 86\%| 
%
%Chor(Dist(Pitch(S, 10), 20dB), 8ms, 5) | up to 84\% | up to 94\% |
%Reverb(Dist(Chor(S, 15ms, 0.25), 20dB))| up to 3\%  | up to 71\% |
%Phaser(Ladder(Gain(S, 12dB)))          | up to 58\% | up to 96\% |

%The results further suggest that trigger styles are effective in different ways. (e.g., style 3 is always more effective than style 1). \todo{why?}
%This effect is highlighted in \textit{clean-label} settings, where our attack is not always effective (e.g., styles 2 and 4). For the clean-label attack, the ASR is very low for styles 0, 1, 2, and 4. However, we see that just a simple transformation like $PitchShift$ (style 0) performs better than a style with multiple transformations (style 4). 
%Style 0 increases the pitch of the original audio, which has a clear effect on its frequency domain representation. This indicates that our models may identify these differences because the calculation of the MFCCs uses the signal's Fourier transform internally. On the other hand, style 4 uses $Reverb$, $Distortion$, and $Chorus$, which do not impact the signal's frequency domain representation.
%However
%\todo{And why for dirty label this style is not the best??? Shouldn't be the case for the dirty label too?}
%Thus, we conclude that for the clean-label setup, the adversary should choose effects that manipulate the signal's frequency domain representation. However, just a slight variation in the frequency is not enough to produce effective attacks in both setups.
%For this reason, we need to amplify its influence with additional effects. Styles 3 and 5 are the best-performing ones for both clean and dirty label attacks. In style 3, we see that in the clean-label attack, simple ineffective effects could result in powerful attacks when combined together. Applying $Distortion$ and $Chorus$ after the $PitchShift$ makes the signal even more distinguishable and easier to learn from the models. Similarly, in style 5, the high-pass $LadderFilter's$ effects are amplified from the other two effects.
%
%However, we see that the styles may cause some inconsistencies in the attack behavior for the different setups. For example, the dirty-label setup style 2 is clearly more effective than style 0. However, the opposite is true in the clean label attack.


% A combination of different effects that may not be very successful could multiply the effectiveness of the attack. S

% Style 3 and 5 perform similarly well for the dirty label. Both of them change the frequency representation of the signals. However, in the clean-label case we see a large difference in the performance. Why?
% We see also similar discrepancies between the clean and the dirty label attack in styles 0 and 2. Style 2 is more effective than style 0 in the dirty label setup but the opposite happens in the clean-label setup. Why?
% also the worst performing attack in the clean-label setup (style 4) is not also the worst for the dirty-label attack.

%Inconsistencies between clean and dirty label attacks, as the best style for dirty label is style5, and the best for clean label is style 3.
%Even simple styles like style2 that contain only simple transformations ($Chorus$) can be very effective 
%         0  PitchShift(S, 10) -> works a little for (clean-label)
%         1  Distortion(S, 30dB) -> does not work
%         2  Chorus(S, 10ms, 5)
%         3  Chorus(Distortion(PitchShift(S, 10), 20dB), 8ms, 5)
%         4  Reverb(Distortion(Chorus(S, 15ms, 0.25), 20dB))
%         5  Phaser(Ladder(Gain(S, 12dB)))

%%% STEFANOS' version
% In both setups and almost all cases we see that higher poisoning rates result in higher ASRs. This is expected as our models can detect the style properties if there are more poisoned training samples. However, we see that the dirty label attack is more effective. This is expected, because as stated in~\cite{clean-label-backdoor-attacks}, the clean-label approach renders the attack less effective and more challenging. 

% Additionally we see that the ASR is higher for the large CNN which is our best performing model. This shows that the backdoor's effectiveness is clearly connected to the model's ability to learn. Thus, we can claim that complex models are more vulnerable to backdoor attacks because they have a larger learning capability.

%\todo{This is our first attempt to explain the results. We wanted to go a little more in depth about the reasons behind the behavior of different styles. We will re-write it but skim through it to get an idea.}
%In the left column of~\cref{fig:asr} (clean-label attack), we see that the ASR highly depends on the style. The first three styles use only simple effects, rendering the attack ineffective. However, we see that style 0 is the most effective among these three, yielding an ASR of around 20\% for 1\% poisoned data. Style 3 is just the combination of the first three styles, but we see a significant improvement in the ASR resulting in more than 80\% for the large CNN and 1\% poisoning rate. This shows that the combination of simple ineffective sound effects could result in a very strong and reliable attack. \todo{why}However, not all complex styles are effective as a trigger. Style 4 uses three simple effects ($Chorus$, $Distortion$, and $Reverb$) but the ASR remains extremely low\todo{why?}. Additionally, Style  5 uses also three effects ($Gain$, $LadderFilter$, and $Phaser$) but the ASR cannot overpass 60\%, even with a 1\% poisoning rate. 
%\todo{nothing is explained after this ``explanation''}

%\input{Sections/07-Defense}
\section{Conclusions and Future Work}
\label{sec:conclusions}

This work contribution is twofold: a formal description of stylistic backdoors and JingleBack, the first stylistic backdoor in the audio domain. 
We demonstrated the potential of our attack through an extensive evaluation considering 444 models, reaching up to 96\% of the attack success rate.  
Future investigations are required to better understand the stylistic backdoors in the audio domain, for example, by considering the impact of the target class or, more in general, how to find an optimal style effective in both clean and dirty label settings. 
Finally, further investigation is needed regarding the stylistic backdoor's behavior against state-of-the-art countermeasures. 




% Usenix
%\bibliographystyle{plain}
% ICASSP
% \bibliographystyle{IEEEbib}
\bibliographystyle{ieeetr}
\bibliography{main}

% \appendix
% \input{Sections/00-Appendix}

\end{document}
