
% This is samplepaper.tex, a sample chapter demonstrating the
% LLNCS macro package for Springer Computer Science proceedings;
% Version 2.20 of 2017/10/04
%
\documentclass[runningheads,envcountsame]{llncs}
%
\usepackage[12pt]{extsizes}
\usepackage{times}
\usepackage[margin=1in]{geometry}

\let\oldbibliography\thebibliography
\renewcommand{\thebibliography}[1]{\oldbibliography{#1}
\setlength{\itemsep}{0pt}}

\makeatletter
\renewcommand{\@Opargbegintheorem}[4]{%
  #4\trivlist\item[\hskip\labelsep{#3#2\@thmcounterend}]}
\makeatother

\let\labelitemi\labelitemii

\let\conjecture\relax % undefine the environment
\spnewtheorem{conjecture}{Conjecture}{\bfseries}{\rmfamily}

\usepackage{amssymb,amsmath}

\usepackage{mathrsfs}
\usepackage{mathtools}
\usepackage{doi}
\DeclarePairedDelimiter\floor{\lfloor}{\rfloor}
\newcommand{\B}{\textbf}
\newcommand{\baro}{\overline}


 \usepackage{hyperref}
 \hypersetup{
     colorlinks=true,
     linkcolor=blue,
     filecolor=blue,
     citecolor = black,      
     urlcolor=blue,
     }


% Used for displaying a sample figure. If possible, figure files should
% be included in EPS format.
%
% If you use the hyperref package, please uncomment the following line
% to display URLs in blue roman font according to Springer's eBook style:
% \renewcommand\UrlFont{\color{blue}\rmfamily}

\begin{document}

%
\title{Cyclability, Connectivity and Circumference }
%
%\titlerunning{Abbreviated paper title}
% If the paper title is too long for the running head, you can set
% an abbreviated paper title here
%
\author{Niranjan Balachandran\inst{1} \and
 Anish Hebbar\inst{2}}
%

% First names are abbreviated in the running head.
% If there are more than two authors, 'et al.' is used.
%
\institute{Indian Institute of Technology Bombay, India  \and
 Indian Institute of Science, Bangalore, India\\
niranj@iitb.ac.in \quad  anishhebbar@iisc.ac.in 
}
%
\maketitle              % typeset the header of the contribution
%
\begin{abstract}
In a graph $G$, a subset of vertices $S \subseteq V(G)$ is said to be cyclable if there is a cycle containing the vertices in some order. $G$ is said to be $k$-cyclable if any subset of $k \geq 2$ vertices is cyclable. If any $k$ \textit{ordered} vertices are present in a common cycle in that order, then the graph is said to be $k$-ordered. We show that when $k \leq \sqrt{n+3}$, $k$-cyclable graphs also have circumference $c(G) \geq 2k$, and that this is best possible. Furthermore when $k \leq \frac{3n}{4} -1$,  $c(G) \geq k+2$, and for $k$-ordered graphs we show $c(G) \geq \min\{n,2k\}$. We also generalize a result by Byer et al. \cite{Byer2007} on the maximum number of edges in nonhamiltonian $k$-connected graphs, and show that if $G$ is a $k$-connected graph of order $n \geq 2(k^2+k)$ with $|E(G)| > \binom{n-k}{2} + k^2$,  then the graph is hamiltonian, and moreover the extremal graphs are unique. 


\keywords{Cyclability  \and Connectivity \and Circumference \and Hamiltonicity}
\end{abstract}
%
%
%
\section{Introduction}

We consider only finite, undirected, simple graphs throughout this paper. The vertex and edge sets of $G$ will be denoted by $V(G)$ and $E(G)$ respectively, the graph complement by $\baro{G}$. The length of the longest cycle in the graph $G$, also known as the circumference, will be denoted by $c(G)$. %and for any $I \subseteq V(G)$, the subgraph induced by $G$ on $I$ will be denoted by $G_I$.
The minimum degree, independence number and connectivity of a graph will denoted by $\delta(G), \alpha(G)$ and  $\kappa(G)$ respectively. We will also use  $d_{H}(v)$ for the degree of $v$ in $H$. The set of neighbours of a vertex $v \in V(G)$ will denoted by $N(v)$, and the closed neighbourhood of $v$, viz. $N(v) \cup \{v\}$ will be denoted by $N[v]$. The join of two graphs $G_1, G_2$, denoted $G_1 \lor G_2$ is simply a copy of $G_1$ and $G_2$, with all edges between $V(G_1)$ and $V(G_2)$ also being present.\\ 

%A graph $G = (V,E)$ is said to be $k$-cyclable, $2\leq k \leq n$ if any subset of vertices $S \subseteq V$ with $|S| = k$ is contained in some cycle in the graph. Similarly, a graph $G = (V,E)$ is said to be $k$-ordered, $2\leq k \leq n$ if any  sequence of distinct vertices $T = (v_1,v_2, \cdots, v_k)$ is contained in a cycle, the vertices appearing in the specified order. 


A subset $S \subseteq V(G)$ of vertices in a graph $G$ is said to be cyclable if $G$ has a cycle containing the vertices of $S$ in some order, possibly including other vertices. A graph $G$ is said to be $k$-cyclable if any $k \geq 2$ vertices of $G$ lie on a common cycle. Note that the problem of determining the hamiltonicity of a graph is a special case of cyclability, namely when $k = n$. Cyclability and connectivity are interlinked, as was shown by Dirac \cite{Dirac1960} who proved for every $k \geq 2$, $k$-connected graphs are also $k$-cyclable. In fact, for $k = 2$ connectivity and cyclability are equivalent, but in general for $k \geq 3$ it is not necessarily true that every $k$-cyclable graph is also $k$-connected, as can be seen by considering the graph $K_2 \lor 2K_k$ which has connectivity exactly $2$ and is also $k$-cyclable. For a brief survey of results involving conditions for cycles to contain a particular set, refer to \cite{Gould2009}.\\

There is a rich literature on conditions guaranteeing the presence of long cycles in graphs, the most classical one being that of Dirac \cite{Dirac1952} who showed that in $2$-connected graphs, the circumference is at least$c(G) \geq \min\{n, 2 \delta(G)\}$. Moreover, $k$-connected graphs have a circumference of at least $\min\{n,2k\}$ from an easy consequence of Menger's theorem, and this is tight. A famous result by Chvátal and Erd\H{o}s \cite{Chvatal1972} relates the connectivity and independence number of a graph to hamiltonicity, and says that if the connectivity of a graph $G$ is at least its independence number, then the graph is hamiltonian. However, not much is known when the requirement of connectivity is weakened to cyclability. Bauer et al. \cite{Bauer2000} obtained lower bounds for the length of the longest cycle in $3$-cyclable graphs in terms of the minimum degree and independence number, but not much else is known for $k$-cyclable graphs for arbitrary $k$.\\

%  --still need to look into hardness, will do tonight 
%-hardness of circumference and cylability
% --fpt and constant k, np-hardness\\
 
%Thus, cyclability can be thought of as a more quantitative measure of hamiltonicity. 

Cyclability  has also received interest from an algorithmic and complexity theoretic point of view as it is a 'hard' parameter that can be thought of as a more quantitative measure of hamiltonicity. Since the classical HAMILTONIAN CYCLE problem is NP-complete, the problem of determining whether a graph is $k$-cyclable (CYCLABILITY) is NP-complete as well. 
The problem of determining whether a given subset $S$ of vertices is cyclable (TERMINAL CYCLABILITY) has been studied in the Parameterized Complexity framework (FPT) (parameterized by $|S|$) and the best known algorithm has  running time $O(2^{|S|} n^{O(1)})$ \cite{Bjorklund2012}. For some special classes of graphs such as interval graphs and bipartite permutation graphs, Crespelle and Golovach \cite{Crespelle2022} showed that both these problems can be solved in polynomial time. For $|S| = O((\log \log n)^{1/10})$, Kawarabayashi \cite{Kawarabayashi2008} obtained a polynomial time algorithm for TERMINAL CYCLABILITY.\\


%Golovach et al.\cite{golovach} also showed that $k$-CYCLABILITY is co-W[1]-hard for split graphs, and that it is FPT on planar graphs when parameterized by $k$
 




%The problem of determining whether a graph is $k$-cyclable (LABILITY) also has received attention from a complexity theoretic perspective in the Parameterized Complexity framework as it is at least as hard as the classical HAMILTONIAN CYCLE problem which is NP-complete. (denoted TERMINAL CYCLABILITY) which is known to be fixed-parameter tractable (FPT) \cite{Bjorklund2012} when parametrized by $|S|$ .  Crespelle and Golovach \cite{Crespelle2022} showed that $k$-CYCLABILITY can be solved in polynomial time for some restricted graph families, such as interval graphs, bipartite permutation graphs and cographs. They also showed that TERMINAL CYCLABILITY can be solved in linear time for the aforementioned graph classes. Golovach et al.\cite{golovach} also showed that $k$-CYCLABILITY is co-W[1]-hard for split graphs, and that it is FPT on planar graphs when parameterized by $k$.\\


Note that $k$-connectivity guarantees $c(G) \geq \min \{n,2k\}$ and also ensures $k$-cyclability. Thus, a natural question to ask is whether the same bound on the circumference can be obtained when the connectivity criteria is weakened to cyclability. When $k = n-1$, we would require any set of $n-1$ vertices of $G$ to lie on a common cycle. It turns out that in this case, it is not necessary that the graph is hamiltonian. Indeed, the existence of  hypohamiltonian graphs \cite{Doyen1975} of order $n$ is known for all $n \geq 18$. Our first  result in this paper gives a similar circumference bound for a wide range of $k$: 
\begin{theorem}\label{maintheorem}
Let $G$ be a $k$-cyclable graph, where $2 \leq k \leq n$. Then, 

\begin{equation*}
  c(G) \geq
    \begin{cases}
      2k & \text{if } k \leq \sqrt{n+3} \\
      k+2 & \text{if }  k \leq \frac{3n}{4} - 1
    \end{cases}       
\end{equation*}
Moreover, for $2 \leq k \leq \sqrt{n+3}$, this bound on the circumference is best possible.
\end{theorem}

\noindent  Note that for $k\ge\frac{n}{2}$ it is still possible that one can have a bound of the form $c(G)\ge (1+\gamma)k$ for some fixed positive constant $\gamma < 1$ as long as $k \neq n - o(n)$.\\
%We will say something about this in a later section. 

A related notion is the orderedness of a graph, a strong hamiltonian property that was first introduced by Ng
and Schultz \cite{Ng1997}.   A graph $G$ is said to be $k$-ordered if any sequence of distinct vertices $T = \{v_1, \ldots, v_k\}$ are present in some common cycle in that order, possibly including other vertices. Note that $k$-ordered graphs are naturally also $k$-cyclable, and it is also easy to see that they are $(k-1)$-connected. For a comprehensive survey of results on $k$-ordered graphs, see \cite{Faudree2001}. We show that for $k$-orderedness, the same circumference bound as $k$-connectivity holds for all $2 \leq k \leq n$.


\begin{theorem}\label{orderedtheorem}
Let $G$ be a $k$-ordered graph, $2 \leq k \leq n$. Then, $c(G) \geq \min\{n,2k\}$.
\end{theorem}





Our second pursuit in this paper is to obtain Tur\'an-type results for the circumference of $k$-connected graphs, specifically the maximum number of edges in nonhamiltonian $k$-connected graphs. A classical result states that if $G$ is a graph of order $n$ with $|E(G)| > \binom{n-1}{2} + 1$, then $G$ is hamiltonian. This was generalized by \cite{Byer2007} for $k \leq 3$, where they showed that if $G$ is $k$-connected and satisfies $|E(G)| > \binom{n-k}{2} + k^2$ with $n$ sufficiently large, then the graph is hamiltonian and the extremal graphs are unique. We further generalize their result and extend it to any $k$ satisfying $n \geq 2(k^2+k)$.

\begin{theorem} \label{kconhamtheorem}
Let $G$ be a $k$-connected graph of order $n \geq 2(k^2+k)$. If $|E(G)| > \binom{n-k}{2} + k^2$, then $G$ is hamiltonian. Moreover, the extremal graphs are unique.
\end{theorem}

The rest of the paper is organized as follows. We lay out some preliminaries in the next section, and give the proofs of Theorems \ref{maintheorem}, \ref{orderedtheorem}, and \ref{kconhamtheorem} in the following section. We conclude with some remarks and open questions.

\section{Preliminaries} 


%We know that $k$ connected graphs are also $k$ cyclable. Moreover, $2$-cyclability is equivalent to $2$-connectivity, and $k$-cyclable graphs are also $l$ cyclable for any $2 \leq l \leq k$.
When the underlying graph is clear, we will use $\delta, \kappa, \alpha$ instead of $\delta(G), \kappa(G), \alpha(G)$ for brevity, and also omit the subscript in $d_H(v)$.
We also use the following well-known lemma attributed to Dirac repeatedly throughout the paper, and provide an outline of the proof for completeness.
\begin{lemma}[\cite{Dirac1960}] \label{kcontheorem}
Any $k$-connected graph $G$ is $k$-cyclable. Moreover, it satisfies $c(G) \geq \min\{n,2k\}$
\end{lemma}
\begin{proof}[Proof Sketch]
Suppose some subset $S$ of vertices with $|S| = k$ was not fully contained in any cycle. Then, take a cycle $C$ containing as many of the vertices of $S$ as possible, and pick some $v \in S$ that is not in $C$. By Menger's theorem, we can choose $k$ vertex-disjoint paths from $v$ to the cycle $C$, and these endpoints divide $C$ into $k$ segments. Since there are strictly less than $k$ vertices of $S$ in $C$, one of the segments does not contain any vertex from $S$, and thus we can extend this segment with the $2$ disjoint paths from $v$ at the ends of the segment to obtain a cycle containing more vertices of $S$, contradiction. \\
Now consider the longest cycle $C$ in $G$ and suppose its length is strictly less than $\min\{n,2k\}$. Pick some $v \in V(G)$ not in $C$, and by Menger's theorem there are $k$ vertex disjoint paths from $v$ to $C$. By the pigeonhole principle, some two endpoints of these $k$ paths must be adjacent on the cycle $C$, giving a contradiction as we can replace the edge between these endpoints with the $2$ paths to obtain a longer cycle. \qed
\end{proof}

A famous result by  Chvátal and Erd\H{o}s  states the following

\begin{theorem}[\cite{Chvatal1972}] \label{erdostheorem}
If in a graph $G$, $\alpha(G) \leq \kappa(G)$, then $G$ is hamiltonian.
\end{theorem}

A natural generalization of the above is to flip the condition $\alpha(G) \leq \kappa(G)$, and instead ask for lower bounds on the circumference of a graph $G$ where $\alpha(G) \geq \kappa(G)$. Foquet and Jolivet \cite{J.L.Fouquet} conjectured the following, which was later proven by Suil O, Douglas B. West and Hehui Wu.


\begin{theorem}[\cite{O2011}]\label{westtheorem}
If $G$ is a $k$-connected $n$-vertex graph with independence number $\alpha$ and $\alpha \geq k$, then $G$ has a cycle of length at least $\frac{k(n+k-\alpha)}{\alpha}$.
\end{theorem}

The following result by Dirac is well-known and was a precursor to a number of results involving the length of the longest cycle in a graph.

\begin{theorem}[\cite{Dirac1952}]\label{diractheorem}
If $G$ is $2$-connected and has minimum degree $\delta$, $c(G) \geq \min\{2 \delta,n\}$.
\end{theorem}

Note that $2$-connectivity is equivalent to $2$-cyclability. Bauer et al. obtained a bound on the circumference of $3$-cyclable graphs in terms of the minimum degree and independence number.

\begin{theorem}[\cite{Bauer2000}]\label{bauertheorem}
 If $G$ is $3$ cyclable, then $$c(G) \geq min\{n, 3\delta - 3 , n + \delta - \alpha\}.$$
\end{theorem}

Ng and Schultz studied a related hamiltonian property termed $k$-orderedness, and showed the following connectivity result. Once again, we include the proof for completeness.

\begin{lemma}[\cite{Ng1997}]\label{schultztheorem}
Let $G$ be a $k$-ordered graph. Then, $G$ is $(k-1)$-connected.
\end{lemma}
\begin{proof}
If not, there exists a set $S$ of $k-2$ vertices whose removal disconnects $G$, breaking it into at least $2$ components. Take $2$ vertices $u,v$ in different components, then any path from $u$ to $v$ must go through some vertex of $S$. Thus, let $T$ consist of $u$, $v$ and then the vertices of $S$, in that order. These vertices must appear in some cycle in that order, giving a contradiction. \qed

\end{proof}

We will also need the concept of graph closure introduced by Bondy and Chvátal. Define
the closure of $G$, denoted $cl(G)$, to be the graph obtained by repeatedly joining any two nonadjacent vertices $x,y$ that satisfy $d(x) + d(y) \geq n$ in $G$. They showed that $cl(G)$ is well-defined (independent of the order in which nonadjacent vertex pairs are considered), and that $G$ is hamiltonian if and only if $cl(G)$ is also hamiltonian. 


\begin{lemma}[\cite{Bondy1976}]\label{bondytheorem}
Suppose $cl(G) = G$ for a nonhamiltonian graph $G$ of order $n$. Then $d(x) + d(y) \leq n-1$ for any pair $\{x,y\}$ of  nonadjacent vertices.
\end{lemma}

 
This was later generalized to obtain results for higher order connectivity, the bounds now also involving the independence number. We define $$\sigma_{k}(G) = \min \{\sum_{i=1}^{k} d(x_i), \{x_1, \ldots x_{k} \} \text{ an independent set of size k in G}\} $$

Note that $\sigma_1(G)$ simply corresponds to the minimum degree $\delta$, and Ore's theorem \cite{Ore1960} states that if $\sigma_2(G) \geq n$, then the graph is hamiltonian.

\begin{theorem}[\cite{Li2013}] \label{haotheorem}
Let $G$ be a $k$-connected graph of order $n$ and independence number $\alpha$. If $\sigma_{k+1}(G) \geq n+ (k-1) \alpha - (k-1)$, then $G$ is hamiltonian.
\end{theorem}


\section{Proofs of the Results}


\begin{proof}[Proof of Theorem \ref{maintheorem}]
$ $\newline
We will first prove the bound for the regime $2 \leq k \leq \sqrt{n+3}$. \\Consider any $k$-cyclable graph with $\alpha(G) \geq k$. Then, let $S$ be a set of $k$ independent vertices, and consider the cycle containing it. This gives us a cycle of length at least $2k$, as any $2$ independent vertices are not adjacent to each other. Thus, we can assume $\alpha(G) \leq  k - 1$. Let the connectivity of the graph be $\kappa$. Using Theorem \ref{westtheorem}, it suffices to show

$$ \frac{\kappa(n+\kappa - \alpha)}{\alpha} \geq 2k   \iff n \geq 2k(\frac{\alpha}{\kappa}) + (\alpha - \kappa)$$

 \noindent As $k$-cyclable graphs are also $2$-cyclable, and thus $2$-connected, we must have $\kappa \geq 2$. Hence, it is sufficient to show the stronger inequality

$$ n \geq 2k (\frac{k-1}{\kappa}) + k - 3$$ which is always true when $$n \geq k^2-3 \iff k \leq \sqrt{n+3} $$
Note that if we only ask for an improvement of the form $c(G) \geq (1 + \gamma)k$ for some positive constant $\gamma < 1$, we can improve the range of $k$ for which the result holds. Once again, let $S$ be any set of at least $ \frac{(1+ \gamma)k}{2}$ many independent vertices, and consider the cycle containing $S$. This corresponds to a cycle containing at least $(1+ \gamma)k$ many vertices since any two independent vertices are not adjacent, and thus we get $\alpha < \frac{(1+ \gamma)k}{2}$. Similar to the previous argument, if the connectivity of the graph is $\kappa$, by Theorem \ref{westtheorem} it suffices to show

$$ \frac{\kappa(n+\kappa - \alpha)}{\alpha} \geq (1 +\gamma) k   \iff n \geq  (1 +\gamma) k  (\frac{\alpha}{\kappa}) + (\alpha - \kappa)$$

\noindent Using $\kappa \leq 2$ and  $\alpha < \frac{(1+ \gamma)k}{2}$, we are done as long as

$$ n \geq \frac{(1+\gamma)^2k^2}{4} + \frac{(1+\gamma)k}{2} - 2 \iff \frac{\sqrt{4n+9}}{1+ \gamma} \geq k$$
So the above argument only yields a linear improvement in $c(G)$ for $k$ up to around $2 \sqrt{n}$.\\

Now, suppose $2 \leq k \leq \frac{3n}{4} - 1$, and assume to the contrary that $c(G) < k+2$. We must have $k \geq 3$ as $2$-cyclable graphs are $2$-connected and hence have circumference at least $4$ for $n \geq 4$. By Theorem
 \ref{diractheorem}, we must have $\delta \leq \frac{k+1}{2}$. Moreover, $\alpha \leq \frac{k+1}{2}$ as otherwise we could simply take a cycle containing $\alpha$ many independent vertices. Consider a vertex $v$ with minimum degree $\delta$, with neighbourhood $N(v)$ satisfying $|N(v)| = \delta$. Now, choose $v$ and any $k-1$ vertices from $V  \backslash N[v]$, which is possible as long as $k-1 \leq n - 1 - \delta$. Then, any cycle containing these vertices must also contain some $2$ neighbours of $v$, giving $c(G) \geq k+2$, and we are done.

Thus, we must have $k + \delta > n $. Note that when $2 \leq k \leq \frac{3n}{4} -1$, $n \geq k+2$ if $n \geq 4$. So, we must either have $3\delta - 3 \leq k+1$ or $n + \delta - \alpha \leq k+1$, otherwise we are done by Theorem \ref{bauertheorem}.\\  The former inequality gives $\delta \leq \frac{k+4}{3}$, which gives $$n < k + \delta \leq \frac{4k+4}{3} \implies \frac{3n-4}{4} < k$$ a contradiction. Hence, we must have $\delta \geq \frac{k+5}{3}$, $\alpha \leq \frac{k+1}{2}$ giving $$ k+1 \geq n + \delta - \alpha \geq n + \frac{k+5}{3} - \frac{k+1}{2} = n + \frac{7-k}{6}$$ 

\noindent or equivalently,  $\frac{3n}{4} -1 \geq k \geq \frac{6n+1}{7}$, which is again a contradiction. \qed
\end{proof}

\noindent We now prove an analogous bound for the circumference of $k$-ordered graphs. 


\begin{proof}[Proof of Theorem \ref{orderedtheorem}]
$ $\newline
We know that $k$-ordered graphs are also $k-1$ connected from Theorem \ref{schultztheorem}, thus $\kappa \geq k-1$. We also must have $\alpha \leq k-1$, as otherwise we can simply take $k$ independent vertices in any order to obtain a cycle of size at least $2k$, in which case we are done. Hence, $$\kappa \geq k-1 \geq \alpha$$ so by Theorem \ref{erdostheorem}, we have that $G$ is hamiltonian, and thus we are done in this case as well. \qed
\end{proof}


In fact, it is not hard to see that the $\min \{n, 2k\}$ bound  on the circumference is achieved for all $2 \leq k \leq n$. If $k > n/2$, simply consider the complete graph $K_n$ which is clearly $k$-connected, $k$-ordered, $k$-cyclable and has circumference $n$. If $k \leq n/2$, consider the complete bipartite graph $G = K_{k,n-k} = (A,B,E)$, which is $k$-ordered, and hence $k$-cyclable.  Indeed, take any sequence of $k$ distinct vertices $T = (v_1, v_2, \ldots, v_k)$. We construct a cycle containing $T$ in that order as follows.\\

Let $T_A$ be the set of vertices in $T$ and $A$, with $T_B$ being defined similarly. Then, for any $v \in T_A$, if the next vertex in the sequence $T$ is in $T_B$, then simply follow the edge joining them. Otherwise, first follow an edge to a vertex in $B \backslash T_B$, and then back to the next vertex which must have been in $T_A$. Follow the same procedure for vertices in $T_B$. At the end, follow the edge joining the first and last vertex. We cannot run out of vertices as the number of extra vertices outside $T_A$ in $A$ that are needed is at most $|T_B|$, and $|A|= k = |T_A| + |T_B|$. Similarly, $|B| = n -k \geq k =|T_A| + |T_B|$.\\



We now generalize a result by \cite{Byer2007} on the maximal number of edges in a $k$-connected nonhamiltonian graph, for $k=2,3$. We will need the following short lemma which appears in \cite{Byer2007}.


\begin{lemma}[\cite{Byer2007}] \label{byertheorem}
Let $G$ be a nonhamiltonian, $k$-connected graph of order $n$. Then $k \leq \frac{n-1}{2}$ and $|E(\baro{G})| \geq \binom{k+1}{2} + (k-1)(n-k-1) - \sigma_{k+1}(G)$
\end{lemma}
\begin{proof}
By Theorem \ref{erdostheorem}, $k$-connected nonhamiltonian graphs must contain an independent set $I = \{x_1, \ldots , x_{k+1} \}$ of $k+1$ vertices. The graph is disconnected on removal of the the $n -(k+1) $ vertices of $G - I$, thus we must have
$ n - (k+1) > k-1$, or $k \leq \frac{n-1}{2}$.

Now consider the independent set $I$ satisfying $\sum_{i=1}^{k+1} d(x_i) = \sigma_{k+1}(G)$. Let the edges in $\baro{G}$ incident on at least one vertex of $I$ be denoted $X_I$. Then $X_I$ contains $\binom{k+1}{2}$ edges with both endpoints in $I$ and $\sum_{i=1}^{k+1} (n-1 - k -  d_G(x_i))$ edges with exactly one endpoint in $I$. Thus, we obtain
$$ \, \, \quad \quad  \quad\quad \quad \quad \quad \quad  \quad\quad |E(\baro{G})| \geq |X_I| = \binom{k+1}{2} +   (k-1)(n-k-1) - \sigma_{k+1}(G) \quad \quad \quad  \quad  \quad    \, \, \, \, \, \qed $$  
\end{proof}
Using a slight variation of the above result and Lemma \ref{bondytheorem}, \cite{Byer2007} also show the following result.


\begin{lemma}[\cite{Byer2007}] \label{byertheorem2}
Suppose $G = cl(G)$ for a nonhamiltonian graph $G$ of order $n$, and $m \leq \alpha(G)$. Then


\begin{equation*}
  |E(\baro{G})| \geq
    \begin{cases}
      \frac{m}{2}(n-m) & \text{for n odd}\\
      \frac{m}{2}(n-m) + \frac{m}{2} - 1 & \text{for n even}
    \end{cases}       
\end{equation*}
\end{lemma}


\noindent With the above results, we are ready to proceed to the proof of Theorem \ref{kconhamtheorem}. The idea is that if $n$ is not that much bigger than $\alpha$, then we can get a sufficient lower bound on $|E(\baro{G})| $ using Lemma \ref{byertheorem2}. Otherwise, $n$ is much bigger than $\alpha$, and we can use Theorem \ref{haotheorem} and Lemma \ref{byertheorem}. To show the uniqueness of the extremal graphs, we will make use of the fact that these graphs must satisfy Lemma \ref{bondytheorem} \textit{maximally}, i.e., addition of any further edge causes a violation of the condition.

\begin{proof}[Proof of Theorem \ref{kconhamtheorem}]
$ $\newline \
First of all, assume $k \geq 2$ as we already know that when $|E(G)| > \binom{n-1}{2} + 1$, then $G$ is hamiltonian and consequently connected as well.
Assume $G$ is nonhamiltonian. We may assume $G = cl(G)$, in which case $d(x)+ d(y) \leq n-1$ for any two nonadjacent vertices $x,y$, from Lemma \ref{bondytheorem}. It suffices to prove that $$|E(\baro{G})| \geq \binom{n}{2} - \left( \binom{n-k}{2} + k^2\right) = k\cdot n - \frac{3k^2+k}{2} $$ Note first that if $\sigma_{k+1}(G) \leq n + k^2- k -1$, by Lemma \ref{byertheorem}
$$|E(\baro{G})| \geq \binom{k+1}{2} + (k+1)(n-k-1) - (n+k^2 - k - 1) = k \cdot n -  \frac{3k^2+k}{2} $$
as desired. We now assume $\sigma_{k+1}(G) \geq n + k^2 - k$ and show that in this case, $|E(\baro{G})|$ is \textit{strictly} greater than $k\cdot n - \frac{3k^2+k}{2}$. We will divide the problem into two cases, depending on the size of $n$ compared to $\alpha$.\\

\noindent \B{Case 1:} Assume $n > \frac{ (k^2-1) \cdot \alpha  + y}{k}$, where $y = \frac{-k^3 + 4k^2 + 3k + 2}{2}$.

Let $I = \{x_1,x_2,\ldots ,x_{k+1}\}$ be a set of $k+1$ independent vertices satisfying $\sum_{i=1}^{k+1}d(x_i) = \sigma_{k+1}(G)$, and assume without loss of generality that $$d(x_1) \geq \frac{\sigma_{k+1}(G)}{k+1} \geq \frac{n+k^2-k}{k+1}$$ 

\B{Subcase 1a:} Suppose $d(x_1) \geq n - 2k$. Note that $V(G) - I - N(x_1)$ is non-empty, as otherwise we would have $d(x_1) = n - k -1$, giving $d(x_i) \leq k$ for $2 \leq i \leq k+1$ as $d(x_1) + d(x_i) \leq n-1$ for $2 \leq i \leq k +1$.  This contradicts $\sigma_{k+1}(G) \geq n + k^2 - k$.  Thus, pick some $v \in V(G) - I - N(x_1)$, giving $d_{\baro{G}}(v) = n - 1 - d_G(v) \geq d_G(x_1) \geq n - 2k$. Therefore, $\baro{G}$ contains at least $n - 2k - |I| = n - 3k - 1$ edges with both endpoints not in $I$. Using the same bound we got in Lemma \ref{byertheorem} but also including the extra edges in $\baro{G}$ incident with $v$ (that have no endpoint in $I$) and using Theorem \ref{haotheorem}, we obtain
\begin{align*}
|E(\baro{G})| &\geq \binom{k+1}{2} + (k+1)(n-k-1) + (n - 3k - 1) - \sigma_{k+1}(G)\\ 
& \geq (k+2)\cdot n - \frac{k^2+9k+4}{2} - (n + (k-1) \alpha - k)\\
& > k\cdot n -  \frac{3k^2+ k}{2} +  \frac{3k^2+ k}{2} -  \frac{k^2+9k+4}{2} + k +  \frac{ (k^2-1) \cdot \alpha  + y}{k} - (k-1)\alpha\\
& =  (k\cdot n -  \frac{3k^2+ k}{2}) +  \frac{  (k-1)\cdot \alpha  + y + k(k^2-3k-2)}{k}  >  (k\cdot n -  \frac{3k^2+ k}{2})
\end{align*}
as desired, where the last inequality follows from $y = \frac{-k^3 + 4k^2 + 3k + 2}{2}$.\\

\B{Subcase 1b:} Suppose next that $d(x_1) \leq n - 2k- 1$. Then there exist distinct vertices $v_1, v_2 \ldots, v_{k} \in V(G) - I - N(x_1)$, and $\baro{G}$ contains at least $$(d_{\baro{G}}(v_1) - k - 1) + (d_{\baro{G}}(v_2) - k - 2) + \cdots + (d_{\baro{G}}(v_k) - 2k) = \sum_{i=1}^k d_{\baro{G}}(v_i) - \frac{3k^2 + k}{2}$$ edges with neither endpoint in $I$. Using $d(v_i) + d(x_1) \leq n-1$ as $G = cl(G)$, we get $d_{\baro{G}}(v_i) \geq d_{G}(x_1) \geq \frac{n + k^2 -k}{k+1}$ for all $1 \leq i \leq k$. Consequently, we obtain at least $$\frac{k(n+k^2-k)}{k+1} - \frac{3k^2+k}{2}$$ edges in $\baro{G}$ with neither endpoint in $I$. Using Theorem \ref{haotheorem} and Lemma \ref{byertheorem} again, we get

 
\begin{align*}
|E(\baro{G})| &\geq \binom{k+1}{2} + (k+1)(n-k-1) + \frac{k(n+k^2-k)}{k+1} - \frac{3k^2+k}{2} - (n + (k-1)\alpha - k)\\
& = (kn  - \frac{3k^2+k}{2}) + \frac{k}{k+1}n -(k-1)\alpha +  \binom{k+1}{2} -(k+1)^2 + \frac{k(k^2-k)}{k+1} + k\\
&> (kn  - \frac{3k^2+k}{2})  + \frac{k}{k+1} \frac{(k^2-1)\alpha + y}{k} - (k-1)\alpha + \frac{-k^2-k-2}{2} + \frac{k(k^2-k)}{k+1}  \\
& = (kn  - \frac{3k^2+k}{2}) + \frac{1}{k+1} \left( \frac{-k^3+4k^2+3k+2}{2}+ \frac{(-k^2-k-2)(k+1)}{2}  + k^3 - k^2\right) \\
& = kn - \frac{3k^2+k}{2}
\end{align*} as desired.\\

\noindent  \B{Case 2:} Assume $n \leq \frac{(k^2-1)\alpha + y}{k}$.\\
In this case, $\alpha \geq \frac{nk - y}{k^2-1}$. By Lemma \ref{byertheorem2}, $|E(\baro{G})| \geq \frac{1}{2} \alpha(n- \alpha)$. This is a upward facing parabola for fixed $n$, so for $\frac{nk - y}{k^2-1} \leq \alpha \leq n - \frac{nk - y}{k^2-1}$, this function is minimized at $\alpha = \frac{nk - y}{k^2-1}$. Therefore, in this range

\begin{align*}
|E(\baro{G})| &\geq \frac{\alpha}{2} (n - \alpha) \geq \frac{1}{2} (\frac{nk - y}{k^2-1})(\frac{n(k^2-k-1) +y}{k^2-1})\\
& = \frac{ n^2k(k^2 - k - 1) + n(2k+1-k^2)y - y^2)}{2(k^2-1)^2} 
\end{align*}

\noindent If we want the above to be strictly greater than $kn - \frac{3k^2+k}{2} $, $$\frac{ n^2k(k^2 - k - 1)}{2(k^2-1)^2}  \geq kn \iff n \geq \frac{2(k^2-1)^2}{k^2 - k - 1} = 2(k^2 + k + \frac{1-k}{k^2-k-1})$$ suffices. This is because for $k \geq 5$, $y = \frac{-k^3 + 4k^2 3k+2}{2} < 0$ and $2k+1 - k^2 < 0$, giving $(2k+1-k^2)(y) > 0$. Similarly, $ -y^2 = \frac{(-k^3 + 4k^2 + 3k+2)^2}{4} > -(3k^2+k)(k^2-1)^2$ for $k \geq 5$, so we only have to check the cases of $k = 2,3,4$ manually which is a routine check.\\

Now, it remains to consider the possiblility that $\alpha > n - \frac{nk - y}{k^2-1} = \frac{ n(k^2-k-1) +y}{k^2-1}$. In this case however,  $\alpha$ is quite large compared to $n$, so the $\binom{\alpha}{2}$ edges in $\baro{G}$ between the vertices of an independent set of size $\alpha$ is strictly greater than $ k \cdot n - \frac{3k^2 + k}{2}$ for all $n$. Indeed, we manually verify for $k \leq 3$, and for $k \geq 4$ simply note that $\frac{nk}{2} + y \geq 0$, and hence when $n \geq 2(k^2+k)$ we have $$\alpha > \frac{n(k^2 - \frac{3k}{2} -1) }{k^2 -1} \geq \frac{9n}{15}, \quad \binom{ 9n/15}{2} > \frac{9n}{30} \cdot \frac{8n}{15} > kn$$


We now prove that the extremal  nonhamiltonian $k$-connected graphs  are unique for $n \geq 2(k^2+k)$, by making use of Lemma \ref{bondytheorem}. Recall that we may assume  $G = cl(G)$ is a
nonhamiltonian, $k$-connected graph of order $n \geq 2k^2 + 2k$ with $\sigma_{k+1}(G)=n+k^2 - k - 1$ as equality only holds if all the inequalities in the above proof are tight.\\

Thus, all the edges in $\baro{G}$ have atleast one endpoint in $I$. Let $I =\{x_1, x_2, \ldots, x_{k+1}\}$ be a set of independent vertices such that $k \leq d(x_1) \leq \ldots \leq d(x_{k+1}) $. Note that $k$-connected graphs have minimum degree at least $k$ as otherwise, the graph could be disconnected by removing at most $k-1$ vertices.  As mentioned in the previous section, we may further assume that all edges in $\baro{G}$ have at least one endpoint in $I$, that is, if $x,y \in V(G) - I$, then $\{x,y\} \in E(G)$. We will now use the properties of graph closure repeatedly. First, note that we must have a clique on the remaining $n - k - 1$ vertices, each of which has degree at least $n - k -2$.
\begin{itemize}
    \item Say $d(x_k) \geq k + 1$  Consider the neighbours of $x_k$ in the clique. These neighbours have degree at least $n - k - 1$, and hence since $G = cl(G)$, must be adjacent to $x_{k+1}$ as well as $d(x_{k+1}) \geq k+1$, But then, these neighbours have degree at least $n - k$, and hence must be adjacent to all of $x_1, \ldots, x_{k+1}$ by the same argument. 
    Thus, $I$ and $N(I)$ together form a complete bipartite graph with $|N(I)| \geq k+1 = |I|$. If $d(x_{k+1}) > k+1$, then it is easy to see that the graph is hamiltonian, and otherwise $k+1 = d(x_i) \, \forall i \in [k+1]$, giving $$ \sigma_{k+1} = n+ k^2 - k -1 =  (k+1)^2 \iff n = 3k+2$$ which is false as we assumed $n \geq 2k^2 + 2k$.
    
    \item Otherwise $d(x_k) = k$, 
    , and hence $d(x_{k+1}) = \sigma_{k+1} - k^2 = n - k - 1$, so we have a clique on the $n- k$ vertices in $G \backslash  \{x_1, \ldots, x_{k}\}$.  The neighbours of any $x_i, i \in [k]$ must have degree at least $n-k$, and hence are joined to all the $x_i$. Thus, we obtain the desired extremal graph with exactly $\binom{n-k}{2} + k^2$ many edges, namely a clique on $n-k$ vertices and $k$ other independent vertices forming a complete bipartite graph with some $k$ vertices from the clique. \qed
\end{itemize} 
\end{proof}





\section{Concluding Remarks}

A simpler proof of Theorem \ref{maintheorem} with a weaker constant can be obtained using Tur\'an's theorem and a theorem of Erd\H{o}s and Gallai \cite{Erdos1959} on the length of the longest cycle in a graph. %which states that if $G$ is a graph on $n$ vertices and $k \geq 3$, then $|E(G)| > \frac{1}{2}(k-1)(n-1)$ implies that the circumference is at least $k$.
Consider any $k$-cyclable graph with $\alpha(G) \geq k$. Then, let $S$ be a set of $k$ independent vertices, and consider the cycle containing it. This gives us a cycle of length atleast $2k$, as any two independent vertices are not adjacent to each other. Thus, we must have $\alpha(G) < k$. By a variant of Tur\'an's theorem, we also have $\alpha > \frac{n}{\tilde{d} + 1}$, where $\tilde{d}$ is the average degree. Thus, we obtain

$$\frac{2|E(G)|}{n} + 1 = \tilde{d} + 1 > \frac{n}{\alpha} \geq \frac{n}{k-1} \implies |E(G)| \geq \frac{1}{2} n \left( \frac{n}{k-1} - 1 \right)$$

which is larger than $\frac{1}{2} (2k-1)(n-1)$ if $n \geq 2 k^2$. giving $c(G) \geq 2k$ when $k \leq \sqrt{n/2}$.\\

It is also interesting to understand what happens to the circumference of $k$-cyclable graphs for large values of $k$. As mentioned earlier in the introduction, it is not necessarily the case that $c(G) = n$ when $k = n-1$ due to the existence of hypohamiltonian graphs. Thus, we have the following extremal problem. 

\begin{conjecture}
For a given $n$, let $f(n)$ be the largest value of $k$ such that any $k$-cyclable graph satisfies $c(G)>k$. From the above, we have $f(n) < n-1$ and from Theorem \ref{maintheorem}, we know $f(n) = \Omega(n)$. Is it the case that $f(n) = n-2 ?$
\end{conjecture}

\noindent We can also ask for what regime of $k$ as a function of $n$ do results of the type in Theorem \ref{maintheorem} hold.

\begin{conjecture}
For a given $n$, let $g(n)$ be the largest value of $k$ such that any $k$-cyclable graph satisfies $c(G)\geq 2k$. From Theorem \ref{maintheorem} we know $g(n) = \Omega(\sqrt{n})$. Is it the case that $g(n) = O(\sqrt{n}) ?$
\end{conjecture}

Moreover, our results only give an improvement of the form $c(G) \geq (1+ \gamma) k$, $0< \gamma < 1$, for $k$ up to around $2\sqrt{n}$, and it is natural to ask if such a linear bound on the circumference can be obtained for much larger regimes of $k$. Finally, note that the results of Theorem \ref{kconhamtheorem} only hold for $n \geq 2(k^2 + k)$. For fixed values of $k \leq 3$, \cite{Byer2007} give a tight bound for the minimum value of $n$ for this to hold. They also note that this bound cannot hold for $k = \Omega(n)$,  in particular if $p = \floor{\frac{n-1}{2}}$, the graph obtained by joining $n-p$ independent vertices to each vertex of $K_p$ is $k$-connected and nonhamiltonian, with total number of edges more than $\binom{n-k}{2} + k^2$ when $\frac{n+1}{6} < k  < \floor{\frac{n-1}{2}}$. This still leaves a significant gap in the possible range of $k$ for which $k$-connectivity and $|E(G)| > \binom{n-k}{2} + k^2$ implies hamiltonicity, as our result only applies for $k = O(\sqrt{n})$.

%There are also some interesting results in the literature regarding the cyclability of claw-free graphs. A claw refers to an induced $K_{1,3}$ subgraph. Flandrin et al. showed that 





\
%
% ---- Bibliography ----
%
% BibTeX users should specify bibliography style 'splncs04'.
% References will then be sorted and formatted in the correct style.
%

%\newpage

\bibliographystyle{splncs04}
 \bibliography{caldam}
%
% 
% This is samplepaper.tex, a sample chapter demonstrating the
% LLNCS macro package for Springer Computer Science proceedings;
% Version 2.20 of 2017/10/04
%
\documentclass[runningheads,envcountsame]{llncs}
%
\usepackage[12pt]{extsizes}
\usepackage{times}
\usepackage[margin=1in]{geometry}

\let\oldbibliography\thebibliography
\renewcommand{\thebibliography}[1]{\oldbibliography{#1}
\setlength{\itemsep}{0pt}}

\makeatletter
\renewcommand{\@Opargbegintheorem}[4]{%
  #4\trivlist\item[\hskip\labelsep{#3#2\@thmcounterend}]}
\makeatother

\let\labelitemi\labelitemii

\let\conjecture\relax % undefine the environment
\spnewtheorem{conjecture}{Conjecture}{\bfseries}{\rmfamily}

\usepackage{amssymb,amsmath}

\usepackage{mathrsfs}
\usepackage{mathtools}
\usepackage{doi}
\DeclarePairedDelimiter\floor{\lfloor}{\rfloor}
\newcommand{\B}{\textbf}
\newcommand{\baro}{\overline}


 \usepackage{hyperref}
 \hypersetup{
     colorlinks=true,
     linkcolor=blue,
     filecolor=blue,
     citecolor = black,      
     urlcolor=blue,
     }


% Used for displaying a sample figure. If possible, figure files should
% be included in EPS format.
%
% If you use the hyperref package, please uncomment the following line
% to display URLs in blue roman font according to Springer's eBook style:
% \renewcommand\UrlFont{\color{blue}\rmfamily}

\begin{document}

%
\title{Cyclability, Connectivity and Circumference }
%
%\titlerunning{Abbreviated paper title}
% If the paper title is too long for the running head, you can set
% an abbreviated paper title here
%
\author{Niranjan Balachandran\inst{1} \and
 Anish Hebbar\inst{2}}
%

% First names are abbreviated in the running head.
% If there are more than two authors, 'et al.' is used.
%
\institute{Indian Institute of Technology Bombay, India  \and
 Indian Institute of Science, Bangalore, India\\
niranj@iitb.ac.in \quad  anishhebbar@iisc.ac.in 
}
%
\maketitle              % typeset the header of the contribution
%
\begin{abstract}
In a graph $G$, a subset of vertices $S \subseteq V(G)$ is said to be cyclable if there is a cycle containing the vertices in some order. $G$ is said to be $k$-cyclable if any subset of $k \geq 2$ vertices is cyclable. If any $k$ \textit{ordered} vertices are present in a common cycle in that order, then the graph is said to be $k$-ordered. We show that when $k \leq \sqrt{n+3}$, $k$-cyclable graphs also have circumference $c(G) \geq 2k$, and that this is best possible. Furthermore when $k \leq \frac{3n}{4} -1$,  $c(G) \geq k+2$, and for $k$-ordered graphs we show $c(G) \geq \min\{n,2k\}$. We also generalize a result by Byer et al. \cite{Byer2007} on the maximum number of edges in nonhamiltonian $k$-connected graphs, and show that if $G$ is a $k$-connected graph of order $n \geq 2(k^2+k)$ with $|E(G)| > \binom{n-k}{2} + k^2$,  then the graph is hamiltonian, and moreover the extremal graphs are unique. 


\keywords{Cyclability  \and Connectivity \and Circumference \and Hamiltonicity}
\end{abstract}
%
%
%
\section{Introduction}

We consider only finite, undirected, simple graphs throughout this paper. The vertex and edge sets of $G$ will be denoted by $V(G)$ and $E(G)$ respectively, the graph complement by $\baro{G}$. The length of the longest cycle in the graph $G$, also known as the circumference, will be denoted by $c(G)$. %and for any $I \subseteq V(G)$, the subgraph induced by $G$ on $I$ will be denoted by $G_I$.
The minimum degree, independence number and connectivity of a graph will denoted by $\delta(G), \alpha(G)$ and  $\kappa(G)$ respectively. We will also use  $d_{H}(v)$ for the degree of $v$ in $H$. The set of neighbours of a vertex $v \in V(G)$ will denoted by $N(v)$, and the closed neighbourhood of $v$, viz. $N(v) \cup \{v\}$ will be denoted by $N[v]$. The join of two graphs $G_1, G_2$, denoted $G_1 \lor G_2$ is simply a copy of $G_1$ and $G_2$, with all edges between $V(G_1)$ and $V(G_2)$ also being present.\\ 

%A graph $G = (V,E)$ is said to be $k$-cyclable, $2\leq k \leq n$ if any subset of vertices $S \subseteq V$ with $|S| = k$ is contained in some cycle in the graph. Similarly, a graph $G = (V,E)$ is said to be $k$-ordered, $2\leq k \leq n$ if any  sequence of distinct vertices $T = (v_1,v_2, \cdots, v_k)$ is contained in a cycle, the vertices appearing in the specified order. 


A subset $S \subseteq V(G)$ of vertices in a graph $G$ is said to be cyclable if $G$ has a cycle containing the vertices of $S$ in some order, possibly including other vertices. A graph $G$ is said to be $k$-cyclable if any $k \geq 2$ vertices of $G$ lie on a common cycle. Note that the problem of determining the hamiltonicity of a graph is a special case of cyclability, namely when $k = n$. Cyclability and connectivity are interlinked, as was shown by Dirac \cite{Dirac1960} who proved for every $k \geq 2$, $k$-connected graphs are also $k$-cyclable. In fact, for $k = 2$ connectivity and cyclability are equivalent, but in general for $k \geq 3$ it is not necessarily true that every $k$-cyclable graph is also $k$-connected, as can be seen by considering the graph $K_2 \lor 2K_k$ which has connectivity exactly $2$ and is also $k$-cyclable. For a brief survey of results involving conditions for cycles to contain a particular set, refer to \cite{Gould2009}.\\

There is a rich literature on conditions guaranteeing the presence of long cycles in graphs, the most classical one being that of Dirac \cite{Dirac1952} who showed that in $2$-connected graphs, the circumference is at least$c(G) \geq \min\{n, 2 \delta(G)\}$. Moreover, $k$-connected graphs have a circumference of at least $\min\{n,2k\}$ from an easy consequence of Menger's theorem, and this is tight. A famous result by Chvátal and Erd\H{o}s \cite{Chvatal1972} relates the connectivity and independence number of a graph to hamiltonicity, and says that if the connectivity of a graph $G$ is at least its independence number, then the graph is hamiltonian. However, not much is known when the requirement of connectivity is weakened to cyclability. Bauer et al. \cite{Bauer2000} obtained lower bounds for the length of the longest cycle in $3$-cyclable graphs in terms of the minimum degree and independence number, but not much else is known for $k$-cyclable graphs for arbitrary $k$.\\

%  --still need to look into hardness, will do tonight 
%-hardness of circumference and cylability
% --fpt and constant k, np-hardness\\
 
%Thus, cyclability can be thought of as a more quantitative measure of hamiltonicity. 

Cyclability  has also received interest from an algorithmic and complexity theoretic point of view as it is a 'hard' parameter that can be thought of as a more quantitative measure of hamiltonicity. Since the classical HAMILTONIAN CYCLE problem is NP-complete, the problem of determining whether a graph is $k$-cyclable (CYCLABILITY) is NP-complete as well. 
The problem of determining whether a given subset $S$ of vertices is cyclable (TERMINAL CYCLABILITY) has been studied in the Parameterized Complexity framework (FPT) (parameterized by $|S|$) and the best known algorithm has  running time $O(2^{|S|} n^{O(1)})$ \cite{Bjorklund2012}. For some special classes of graphs such as interval graphs and bipartite permutation graphs, Crespelle and Golovach \cite{Crespelle2022} showed that both these problems can be solved in polynomial time. For $|S| = O((\log \log n)^{1/10})$, Kawarabayashi \cite{Kawarabayashi2008} obtained a polynomial time algorithm for TERMINAL CYCLABILITY.\\


%Golovach et al.\cite{golovach} also showed that $k$-CYCLABILITY is co-W[1]-hard for split graphs, and that it is FPT on planar graphs when parameterized by $k$
 




%The problem of determining whether a graph is $k$-cyclable (LABILITY) also has received attention from a complexity theoretic perspective in the Parameterized Complexity framework as it is at least as hard as the classical HAMILTONIAN CYCLE problem which is NP-complete. (denoted TERMINAL CYCLABILITY) which is known to be fixed-parameter tractable (FPT) \cite{Bjorklund2012} when parametrized by $|S|$ .  Crespelle and Golovach \cite{Crespelle2022} showed that $k$-CYCLABILITY can be solved in polynomial time for some restricted graph families, such as interval graphs, bipartite permutation graphs and cographs. They also showed that TERMINAL CYCLABILITY can be solved in linear time for the aforementioned graph classes. Golovach et al.\cite{golovach} also showed that $k$-CYCLABILITY is co-W[1]-hard for split graphs, and that it is FPT on planar graphs when parameterized by $k$.\\


Note that $k$-connectivity guarantees $c(G) \geq \min \{n,2k\}$ and also ensures $k$-cyclability. Thus, a natural question to ask is whether the same bound on the circumference can be obtained when the connectivity criteria is weakened to cyclability. When $k = n-1$, we would require any set of $n-1$ vertices of $G$ to lie on a common cycle. It turns out that in this case, it is not necessary that the graph is hamiltonian. Indeed, the existence of  hypohamiltonian graphs \cite{Doyen1975} of order $n$ is known for all $n \geq 18$. Our first  result in this paper gives a similar circumference bound for a wide range of $k$: 
\begin{theorem}\label{maintheorem}
Let $G$ be a $k$-cyclable graph, where $2 \leq k \leq n$. Then, 

\begin{equation*}
  c(G) \geq
    \begin{cases}
      2k & \text{if } k \leq \sqrt{n+3} \\
      k+2 & \text{if }  k \leq \frac{3n}{4} - 1
    \end{cases}       
\end{equation*}
Moreover, for $2 \leq k \leq \sqrt{n+3}$, this bound on the circumference is best possible.
\end{theorem}

\noindent  Note that for $k\ge\frac{n}{2}$ it is still possible that one can have a bound of the form $c(G)\ge (1+\gamma)k$ for some fixed positive constant $\gamma < 1$ as long as $k \neq n - o(n)$.\\
%We will say something about this in a later section. 

A related notion is the orderedness of a graph, a strong hamiltonian property that was first introduced by Ng
and Schultz \cite{Ng1997}.   A graph $G$ is said to be $k$-ordered if any sequence of distinct vertices $T = \{v_1, \ldots, v_k\}$ are present in some common cycle in that order, possibly including other vertices. Note that $k$-ordered graphs are naturally also $k$-cyclable, and it is also easy to see that they are $(k-1)$-connected. For a comprehensive survey of results on $k$-ordered graphs, see \cite{Faudree2001}. We show that for $k$-orderedness, the same circumference bound as $k$-connectivity holds for all $2 \leq k \leq n$.


\begin{theorem}\label{orderedtheorem}
Let $G$ be a $k$-ordered graph, $2 \leq k \leq n$. Then, $c(G) \geq \min\{n,2k\}$.
\end{theorem}





Our second pursuit in this paper is to obtain Tur\'an-type results for the circumference of $k$-connected graphs, specifically the maximum number of edges in nonhamiltonian $k$-connected graphs. A classical result states that if $G$ is a graph of order $n$ with $|E(G)| > \binom{n-1}{2} + 1$, then $G$ is hamiltonian. This was generalized by \cite{Byer2007} for $k \leq 3$, where they showed that if $G$ is $k$-connected and satisfies $|E(G)| > \binom{n-k}{2} + k^2$ with $n$ sufficiently large, then the graph is hamiltonian and the extremal graphs are unique. We further generalize their result and extend it to any $k$ satisfying $n \geq 2(k^2+k)$.

\begin{theorem} \label{kconhamtheorem}
Let $G$ be a $k$-connected graph of order $n \geq 2(k^2+k)$. If $|E(G)| > \binom{n-k}{2} + k^2$, then $G$ is hamiltonian. Moreover, the extremal graphs are unique.
\end{theorem}

The rest of the paper is organized as follows. We lay out some preliminaries in the next section, and give the proofs of Theorems \ref{maintheorem}, \ref{orderedtheorem}, and \ref{kconhamtheorem} in the following section. We conclude with some remarks and open questions.

\section{Preliminaries} 


%We know that $k$ connected graphs are also $k$ cyclable. Moreover, $2$-cyclability is equivalent to $2$-connectivity, and $k$-cyclable graphs are also $l$ cyclable for any $2 \leq l \leq k$.
When the underlying graph is clear, we will use $\delta, \kappa, \alpha$ instead of $\delta(G), \kappa(G), \alpha(G)$ for brevity, and also omit the subscript in $d_H(v)$.
We also use the following well-known lemma attributed to Dirac repeatedly throughout the paper, and provide an outline of the proof for completeness.
\begin{lemma}[\cite{Dirac1960}] \label{kcontheorem}
Any $k$-connected graph $G$ is $k$-cyclable. Moreover, it satisfies $c(G) \geq \min\{n,2k\}$
\end{lemma}
\begin{proof}[Proof Sketch]
Suppose some subset $S$ of vertices with $|S| = k$ was not fully contained in any cycle. Then, take a cycle $C$ containing as many of the vertices of $S$ as possible, and pick some $v \in S$ that is not in $C$. By Menger's theorem, we can choose $k$ vertex-disjoint paths from $v$ to the cycle $C$, and these endpoints divide $C$ into $k$ segments. Since there are strictly less than $k$ vertices of $S$ in $C$, one of the segments does not contain any vertex from $S$, and thus we can extend this segment with the $2$ disjoint paths from $v$ at the ends of the segment to obtain a cycle containing more vertices of $S$, contradiction. \\
Now consider the longest cycle $C$ in $G$ and suppose its length is strictly less than $\min\{n,2k\}$. Pick some $v \in V(G)$ not in $C$, and by Menger's theorem there are $k$ vertex disjoint paths from $v$ to $C$. By the pigeonhole principle, some two endpoints of these $k$ paths must be adjacent on the cycle $C$, giving a contradiction as we can replace the edge between these endpoints with the $2$ paths to obtain a longer cycle. \qed
\end{proof}

A famous result by  Chvátal and Erd\H{o}s  states the following

\begin{theorem}[\cite{Chvatal1972}] \label{erdostheorem}
If in a graph $G$, $\alpha(G) \leq \kappa(G)$, then $G$ is hamiltonian.
\end{theorem}

A natural generalization of the above is to flip the condition $\alpha(G) \leq \kappa(G)$, and instead ask for lower bounds on the circumference of a graph $G$ where $\alpha(G) \geq \kappa(G)$. Foquet and Jolivet \cite{J.L.Fouquet} conjectured the following, which was later proven by Suil O, Douglas B. West and Hehui Wu.


\begin{theorem}[\cite{O2011}]\label{westtheorem}
If $G$ is a $k$-connected $n$-vertex graph with independence number $\alpha$ and $\alpha \geq k$, then $G$ has a cycle of length at least $\frac{k(n+k-\alpha)}{\alpha}$.
\end{theorem}

The following result by Dirac is well-known and was a precursor to a number of results involving the length of the longest cycle in a graph.

\begin{theorem}[\cite{Dirac1952}]\label{diractheorem}
If $G$ is $2$-connected and has minimum degree $\delta$, $c(G) \geq \min\{2 \delta,n\}$.
\end{theorem}

Note that $2$-connectivity is equivalent to $2$-cyclability. Bauer et al. obtained a bound on the circumference of $3$-cyclable graphs in terms of the minimum degree and independence number.

\begin{theorem}[\cite{Bauer2000}]\label{bauertheorem}
 If $G$ is $3$ cyclable, then $$c(G) \geq min\{n, 3\delta - 3 , n + \delta - \alpha\}.$$
\end{theorem}

Ng and Schultz studied a related hamiltonian property termed $k$-orderedness, and showed the following connectivity result. Once again, we include the proof for completeness.

\begin{lemma}[\cite{Ng1997}]\label{schultztheorem}
Let $G$ be a $k$-ordered graph. Then, $G$ is $(k-1)$-connected.
\end{lemma}
\begin{proof}
If not, there exists a set $S$ of $k-2$ vertices whose removal disconnects $G$, breaking it into at least $2$ components. Take $2$ vertices $u,v$ in different components, then any path from $u$ to $v$ must go through some vertex of $S$. Thus, let $T$ consist of $u$, $v$ and then the vertices of $S$, in that order. These vertices must appear in some cycle in that order, giving a contradiction. \qed

\end{proof}

We will also need the concept of graph closure introduced by Bondy and Chvátal. Define
the closure of $G$, denoted $cl(G)$, to be the graph obtained by repeatedly joining any two nonadjacent vertices $x,y$ that satisfy $d(x) + d(y) \geq n$ in $G$. They showed that $cl(G)$ is well-defined (independent of the order in which nonadjacent vertex pairs are considered), and that $G$ is hamiltonian if and only if $cl(G)$ is also hamiltonian. 


\begin{lemma}[\cite{Bondy1976}]\label{bondytheorem}
Suppose $cl(G) = G$ for a nonhamiltonian graph $G$ of order $n$. Then $d(x) + d(y) \leq n-1$ for any pair $\{x,y\}$ of  nonadjacent vertices.
\end{lemma}

 
This was later generalized to obtain results for higher order connectivity, the bounds now also involving the independence number. We define $$\sigma_{k}(G) = \min \{\sum_{i=1}^{k} d(x_i), \{x_1, \ldots x_{k} \} \text{ an independent set of size k in G}\} $$

Note that $\sigma_1(G)$ simply corresponds to the minimum degree $\delta$, and Ore's theorem \cite{Ore1960} states that if $\sigma_2(G) \geq n$, then the graph is hamiltonian.

\begin{theorem}[\cite{Li2013}] \label{haotheorem}
Let $G$ be a $k$-connected graph of order $n$ and independence number $\alpha$. If $\sigma_{k+1}(G) \geq n+ (k-1) \alpha - (k-1)$, then $G$ is hamiltonian.
\end{theorem}


\section{Proofs of the Results}


\begin{proof}[Proof of Theorem \ref{maintheorem}]
$ $\newline
We will first prove the bound for the regime $2 \leq k \leq \sqrt{n+3}$. \\Consider any $k$-cyclable graph with $\alpha(G) \geq k$. Then, let $S$ be a set of $k$ independent vertices, and consider the cycle containing it. This gives us a cycle of length at least $2k$, as any $2$ independent vertices are not adjacent to each other. Thus, we can assume $\alpha(G) \leq  k - 1$. Let the connectivity of the graph be $\kappa$. Using Theorem \ref{westtheorem}, it suffices to show

$$ \frac{\kappa(n+\kappa - \alpha)}{\alpha} \geq 2k   \iff n \geq 2k(\frac{\alpha}{\kappa}) + (\alpha - \kappa)$$

 \noindent As $k$-cyclable graphs are also $2$-cyclable, and thus $2$-connected, we must have $\kappa \geq 2$. Hence, it is sufficient to show the stronger inequality

$$ n \geq 2k (\frac{k-1}{\kappa}) + k - 3$$ which is always true when $$n \geq k^2-3 \iff k \leq \sqrt{n+3} $$
Note that if we only ask for an improvement of the form $c(G) \geq (1 + \gamma)k$ for some positive constant $\gamma < 1$, we can improve the range of $k$ for which the result holds. Once again, let $S$ be any set of at least $ \frac{(1+ \gamma)k}{2}$ many independent vertices, and consider the cycle containing $S$. This corresponds to a cycle containing at least $(1+ \gamma)k$ many vertices since any two independent vertices are not adjacent, and thus we get $\alpha < \frac{(1+ \gamma)k}{2}$. Similar to the previous argument, if the connectivity of the graph is $\kappa$, by Theorem \ref{westtheorem} it suffices to show

$$ \frac{\kappa(n+\kappa - \alpha)}{\alpha} \geq (1 +\gamma) k   \iff n \geq  (1 +\gamma) k  (\frac{\alpha}{\kappa}) + (\alpha - \kappa)$$

\noindent Using $\kappa \leq 2$ and  $\alpha < \frac{(1+ \gamma)k}{2}$, we are done as long as

$$ n \geq \frac{(1+\gamma)^2k^2}{4} + \frac{(1+\gamma)k}{2} - 2 \iff \frac{\sqrt{4n+9}}{1+ \gamma} \geq k$$
So the above argument only yields a linear improvement in $c(G)$ for $k$ up to around $2 \sqrt{n}$.\\

Now, suppose $2 \leq k \leq \frac{3n}{4} - 1$, and assume to the contrary that $c(G) < k+2$. We must have $k \geq 3$ as $2$-cyclable graphs are $2$-connected and hence have circumference at least $4$ for $n \geq 4$. By Theorem
 \ref{diractheorem}, we must have $\delta \leq \frac{k+1}{2}$. Moreover, $\alpha \leq \frac{k+1}{2}$ as otherwise we could simply take a cycle containing $\alpha$ many independent vertices. Consider a vertex $v$ with minimum degree $\delta$, with neighbourhood $N(v)$ satisfying $|N(v)| = \delta$. Now, choose $v$ and any $k-1$ vertices from $V  \backslash N[v]$, which is possible as long as $k-1 \leq n - 1 - \delta$. Then, any cycle containing these vertices must also contain some $2$ neighbours of $v$, giving $c(G) \geq k+2$, and we are done.

Thus, we must have $k + \delta > n $. Note that when $2 \leq k \leq \frac{3n}{4} -1$, $n \geq k+2$ if $n \geq 4$. So, we must either have $3\delta - 3 \leq k+1$ or $n + \delta - \alpha \leq k+1$, otherwise we are done by Theorem \ref{bauertheorem}.\\  The former inequality gives $\delta \leq \frac{k+4}{3}$, which gives $$n < k + \delta \leq \frac{4k+4}{3} \implies \frac{3n-4}{4} < k$$ a contradiction. Hence, we must have $\delta \geq \frac{k+5}{3}$, $\alpha \leq \frac{k+1}{2}$ giving $$ k+1 \geq n + \delta - \alpha \geq n + \frac{k+5}{3} - \frac{k+1}{2} = n + \frac{7-k}{6}$$ 

\noindent or equivalently,  $\frac{3n}{4} -1 \geq k \geq \frac{6n+1}{7}$, which is again a contradiction. \qed
\end{proof}

\noindent We now prove an analogous bound for the circumference of $k$-ordered graphs. 


\begin{proof}[Proof of Theorem \ref{orderedtheorem}]
$ $\newline
We know that $k$-ordered graphs are also $k-1$ connected from Theorem \ref{schultztheorem}, thus $\kappa \geq k-1$. We also must have $\alpha \leq k-1$, as otherwise we can simply take $k$ independent vertices in any order to obtain a cycle of size at least $2k$, in which case we are done. Hence, $$\kappa \geq k-1 \geq \alpha$$ so by Theorem \ref{erdostheorem}, we have that $G$ is hamiltonian, and thus we are done in this case as well. \qed
\end{proof}


In fact, it is not hard to see that the $\min \{n, 2k\}$ bound  on the circumference is achieved for all $2 \leq k \leq n$. If $k > n/2$, simply consider the complete graph $K_n$ which is clearly $k$-connected, $k$-ordered, $k$-cyclable and has circumference $n$. If $k \leq n/2$, consider the complete bipartite graph $G = K_{k,n-k} = (A,B,E)$, which is $k$-ordered, and hence $k$-cyclable.  Indeed, take any sequence of $k$ distinct vertices $T = (v_1, v_2, \ldots, v_k)$. We construct a cycle containing $T$ in that order as follows.\\

Let $T_A$ be the set of vertices in $T$ and $A$, with $T_B$ being defined similarly. Then, for any $v \in T_A$, if the next vertex in the sequence $T$ is in $T_B$, then simply follow the edge joining them. Otherwise, first follow an edge to a vertex in $B \backslash T_B$, and then back to the next vertex which must have been in $T_A$. Follow the same procedure for vertices in $T_B$. At the end, follow the edge joining the first and last vertex. We cannot run out of vertices as the number of extra vertices outside $T_A$ in $A$ that are needed is at most $|T_B|$, and $|A|= k = |T_A| + |T_B|$. Similarly, $|B| = n -k \geq k =|T_A| + |T_B|$.\\



We now generalize a result by \cite{Byer2007} on the maximal number of edges in a $k$-connected nonhamiltonian graph, for $k=2,3$. We will need the following short lemma which appears in \cite{Byer2007}.


\begin{lemma}[\cite{Byer2007}] \label{byertheorem}
Let $G$ be a nonhamiltonian, $k$-connected graph of order $n$. Then $k \leq \frac{n-1}{2}$ and $|E(\baro{G})| \geq \binom{k+1}{2} + (k-1)(n-k-1) - \sigma_{k+1}(G)$
\end{lemma}
\begin{proof}
By Theorem \ref{erdostheorem}, $k$-connected nonhamiltonian graphs must contain an independent set $I = \{x_1, \ldots , x_{k+1} \}$ of $k+1$ vertices. The graph is disconnected on removal of the the $n -(k+1) $ vertices of $G - I$, thus we must have
$ n - (k+1) > k-1$, or $k \leq \frac{n-1}{2}$.

Now consider the independent set $I$ satisfying $\sum_{i=1}^{k+1} d(x_i) = \sigma_{k+1}(G)$. Let the edges in $\baro{G}$ incident on at least one vertex of $I$ be denoted $X_I$. Then $X_I$ contains $\binom{k+1}{2}$ edges with both endpoints in $I$ and $\sum_{i=1}^{k+1} (n-1 - k -  d_G(x_i))$ edges with exactly one endpoint in $I$. Thus, we obtain
$$ \, \, \quad \quad  \quad\quad \quad \quad \quad \quad  \quad\quad |E(\baro{G})| \geq |X_I| = \binom{k+1}{2} +   (k-1)(n-k-1) - \sigma_{k+1}(G) \quad \quad \quad  \quad  \quad    \, \, \, \, \, \qed $$  
\end{proof}
Using a slight variation of the above result and Lemma \ref{bondytheorem}, \cite{Byer2007} also show the following result.


\begin{lemma}[\cite{Byer2007}] \label{byertheorem2}
Suppose $G = cl(G)$ for a nonhamiltonian graph $G$ of order $n$, and $m \leq \alpha(G)$. Then


\begin{equation*}
  |E(\baro{G})| \geq
    \begin{cases}
      \frac{m}{2}(n-m) & \text{for n odd}\\
      \frac{m}{2}(n-m) + \frac{m}{2} - 1 & \text{for n even}
    \end{cases}       
\end{equation*}
\end{lemma}


\noindent With the above results, we are ready to proceed to the proof of Theorem \ref{kconhamtheorem}. The idea is that if $n$ is not that much bigger than $\alpha$, then we can get a sufficient lower bound on $|E(\baro{G})| $ using Lemma \ref{byertheorem2}. Otherwise, $n$ is much bigger than $\alpha$, and we can use Theorem \ref{haotheorem} and Lemma \ref{byertheorem}. To show the uniqueness of the extremal graphs, we will make use of the fact that these graphs must satisfy Lemma \ref{bondytheorem} \textit{maximally}, i.e., addition of any further edge causes a violation of the condition.

\begin{proof}[Proof of Theorem \ref{kconhamtheorem}]
$ $\newline \
First of all, assume $k \geq 2$ as we already know that when $|E(G)| > \binom{n-1}{2} + 1$, then $G$ is hamiltonian and consequently connected as well.
Assume $G$ is nonhamiltonian. We may assume $G = cl(G)$, in which case $d(x)+ d(y) \leq n-1$ for any two nonadjacent vertices $x,y$, from Lemma \ref{bondytheorem}. It suffices to prove that $$|E(\baro{G})| \geq \binom{n}{2} - \left( \binom{n-k}{2} + k^2\right) = k\cdot n - \frac{3k^2+k}{2} $$ Note first that if $\sigma_{k+1}(G) \leq n + k^2- k -1$, by Lemma \ref{byertheorem}
$$|E(\baro{G})| \geq \binom{k+1}{2} + (k+1)(n-k-1) - (n+k^2 - k - 1) = k \cdot n -  \frac{3k^2+k}{2} $$
as desired. We now assume $\sigma_{k+1}(G) \geq n + k^2 - k$ and show that in this case, $|E(\baro{G})|$ is \textit{strictly} greater than $k\cdot n - \frac{3k^2+k}{2}$. We will divide the problem into two cases, depending on the size of $n$ compared to $\alpha$.\\

\noindent \B{Case 1:} Assume $n > \frac{ (k^2-1) \cdot \alpha  + y}{k}$, where $y = \frac{-k^3 + 4k^2 + 3k + 2}{2}$.

Let $I = \{x_1,x_2,\ldots ,x_{k+1}\}$ be a set of $k+1$ independent vertices satisfying $\sum_{i=1}^{k+1}d(x_i) = \sigma_{k+1}(G)$, and assume without loss of generality that $$d(x_1) \geq \frac{\sigma_{k+1}(G)}{k+1} \geq \frac{n+k^2-k}{k+1}$$ 

\B{Subcase 1a:} Suppose $d(x_1) \geq n - 2k$. Note that $V(G) - I - N(x_1)$ is non-empty, as otherwise we would have $d(x_1) = n - k -1$, giving $d(x_i) \leq k$ for $2 \leq i \leq k+1$ as $d(x_1) + d(x_i) \leq n-1$ for $2 \leq i \leq k +1$.  This contradicts $\sigma_{k+1}(G) \geq n + k^2 - k$.  Thus, pick some $v \in V(G) - I - N(x_1)$, giving $d_{\baro{G}}(v) = n - 1 - d_G(v) \geq d_G(x_1) \geq n - 2k$. Therefore, $\baro{G}$ contains at least $n - 2k - |I| = n - 3k - 1$ edges with both endpoints not in $I$. Using the same bound we got in Lemma \ref{byertheorem} but also including the extra edges in $\baro{G}$ incident with $v$ (that have no endpoint in $I$) and using Theorem \ref{haotheorem}, we obtain
\begin{align*}
|E(\baro{G})| &\geq \binom{k+1}{2} + (k+1)(n-k-1) + (n - 3k - 1) - \sigma_{k+1}(G)\\ 
& \geq (k+2)\cdot n - \frac{k^2+9k+4}{2} - (n + (k-1) \alpha - k)\\
& > k\cdot n -  \frac{3k^2+ k}{2} +  \frac{3k^2+ k}{2} -  \frac{k^2+9k+4}{2} + k +  \frac{ (k^2-1) \cdot \alpha  + y}{k} - (k-1)\alpha\\
& =  (k\cdot n -  \frac{3k^2+ k}{2}) +  \frac{  (k-1)\cdot \alpha  + y + k(k^2-3k-2)}{k}  >  (k\cdot n -  \frac{3k^2+ k}{2})
\end{align*}
as desired, where the last inequality follows from $y = \frac{-k^3 + 4k^2 + 3k + 2}{2}$.\\

\B{Subcase 1b:} Suppose next that $d(x_1) \leq n - 2k- 1$. Then there exist distinct vertices $v_1, v_2 \ldots, v_{k} \in V(G) - I - N(x_1)$, and $\baro{G}$ contains at least $$(d_{\baro{G}}(v_1) - k - 1) + (d_{\baro{G}}(v_2) - k - 2) + \cdots + (d_{\baro{G}}(v_k) - 2k) = \sum_{i=1}^k d_{\baro{G}}(v_i) - \frac{3k^2 + k}{2}$$ edges with neither endpoint in $I$. Using $d(v_i) + d(x_1) \leq n-1$ as $G = cl(G)$, we get $d_{\baro{G}}(v_i) \geq d_{G}(x_1) \geq \frac{n + k^2 -k}{k+1}$ for all $1 \leq i \leq k$. Consequently, we obtain at least $$\frac{k(n+k^2-k)}{k+1} - \frac{3k^2+k}{2}$$ edges in $\baro{G}$ with neither endpoint in $I$. Using Theorem \ref{haotheorem} and Lemma \ref{byertheorem} again, we get

 
\begin{align*}
|E(\baro{G})| &\geq \binom{k+1}{2} + (k+1)(n-k-1) + \frac{k(n+k^2-k)}{k+1} - \frac{3k^2+k}{2} - (n + (k-1)\alpha - k)\\
& = (kn  - \frac{3k^2+k}{2}) + \frac{k}{k+1}n -(k-1)\alpha +  \binom{k+1}{2} -(k+1)^2 + \frac{k(k^2-k)}{k+1} + k\\
&> (kn  - \frac{3k^2+k}{2})  + \frac{k}{k+1} \frac{(k^2-1)\alpha + y}{k} - (k-1)\alpha + \frac{-k^2-k-2}{2} + \frac{k(k^2-k)}{k+1}  \\
& = (kn  - \frac{3k^2+k}{2}) + \frac{1}{k+1} \left( \frac{-k^3+4k^2+3k+2}{2}+ \frac{(-k^2-k-2)(k+1)}{2}  + k^3 - k^2\right) \\
& = kn - \frac{3k^2+k}{2}
\end{align*} as desired.\\

\noindent  \B{Case 2:} Assume $n \leq \frac{(k^2-1)\alpha + y}{k}$.\\
In this case, $\alpha \geq \frac{nk - y}{k^2-1}$. By Lemma \ref{byertheorem2}, $|E(\baro{G})| \geq \frac{1}{2} \alpha(n- \alpha)$. This is a upward facing parabola for fixed $n$, so for $\frac{nk - y}{k^2-1} \leq \alpha \leq n - \frac{nk - y}{k^2-1}$, this function is minimized at $\alpha = \frac{nk - y}{k^2-1}$. Therefore, in this range

\begin{align*}
|E(\baro{G})| &\geq \frac{\alpha}{2} (n - \alpha) \geq \frac{1}{2} (\frac{nk - y}{k^2-1})(\frac{n(k^2-k-1) +y}{k^2-1})\\
& = \frac{ n^2k(k^2 - k - 1) + n(2k+1-k^2)y - y^2)}{2(k^2-1)^2} 
\end{align*}

\noindent If we want the above to be strictly greater than $kn - \frac{3k^2+k}{2} $, $$\frac{ n^2k(k^2 - k - 1)}{2(k^2-1)^2}  \geq kn \iff n \geq \frac{2(k^2-1)^2}{k^2 - k - 1} = 2(k^2 + k + \frac{1-k}{k^2-k-1})$$ suffices. This is because for $k \geq 5$, $y = \frac{-k^3 + 4k^2 3k+2}{2} < 0$ and $2k+1 - k^2 < 0$, giving $(2k+1-k^2)(y) > 0$. Similarly, $ -y^2 = \frac{(-k^3 + 4k^2 + 3k+2)^2}{4} > -(3k^2+k)(k^2-1)^2$ for $k \geq 5$, so we only have to check the cases of $k = 2,3,4$ manually which is a routine check.\\

Now, it remains to consider the possiblility that $\alpha > n - \frac{nk - y}{k^2-1} = \frac{ n(k^2-k-1) +y}{k^2-1}$. In this case however,  $\alpha$ is quite large compared to $n$, so the $\binom{\alpha}{2}$ edges in $\baro{G}$ between the vertices of an independent set of size $\alpha$ is strictly greater than $ k \cdot n - \frac{3k^2 + k}{2}$ for all $n$. Indeed, we manually verify for $k \leq 3$, and for $k \geq 4$ simply note that $\frac{nk}{2} + y \geq 0$, and hence when $n \geq 2(k^2+k)$ we have $$\alpha > \frac{n(k^2 - \frac{3k}{2} -1) }{k^2 -1} \geq \frac{9n}{15}, \quad \binom{ 9n/15}{2} > \frac{9n}{30} \cdot \frac{8n}{15} > kn$$


We now prove that the extremal  nonhamiltonian $k$-connected graphs  are unique for $n \geq 2(k^2+k)$, by making use of Lemma \ref{bondytheorem}. Recall that we may assume  $G = cl(G)$ is a
nonhamiltonian, $k$-connected graph of order $n \geq 2k^2 + 2k$ with $\sigma_{k+1}(G)=n+k^2 - k - 1$ as equality only holds if all the inequalities in the above proof are tight.\\

Thus, all the edges in $\baro{G}$ have atleast one endpoint in $I$. Let $I =\{x_1, x_2, \ldots, x_{k+1}\}$ be a set of independent vertices such that $k \leq d(x_1) \leq \ldots \leq d(x_{k+1}) $. Note that $k$-connected graphs have minimum degree at least $k$ as otherwise, the graph could be disconnected by removing at most $k-1$ vertices.  As mentioned in the previous section, we may further assume that all edges in $\baro{G}$ have at least one endpoint in $I$, that is, if $x,y \in V(G) - I$, then $\{x,y\} \in E(G)$. We will now use the properties of graph closure repeatedly. First, note that we must have a clique on the remaining $n - k - 1$ vertices, each of which has degree at least $n - k -2$.
\begin{itemize}
    \item Say $d(x_k) \geq k + 1$  Consider the neighbours of $x_k$ in the clique. These neighbours have degree at least $n - k - 1$, and hence since $G = cl(G)$, must be adjacent to $x_{k+1}$ as well as $d(x_{k+1}) \geq k+1$, But then, these neighbours have degree at least $n - k$, and hence must be adjacent to all of $x_1, \ldots, x_{k+1}$ by the same argument. 
    Thus, $I$ and $N(I)$ together form a complete bipartite graph with $|N(I)| \geq k+1 = |I|$. If $d(x_{k+1}) > k+1$, then it is easy to see that the graph is hamiltonian, and otherwise $k+1 = d(x_i) \, \forall i \in [k+1]$, giving $$ \sigma_{k+1} = n+ k^2 - k -1 =  (k+1)^2 \iff n = 3k+2$$ which is false as we assumed $n \geq 2k^2 + 2k$.
    
    \item Otherwise $d(x_k) = k$, 
    , and hence $d(x_{k+1}) = \sigma_{k+1} - k^2 = n - k - 1$, so we have a clique on the $n- k$ vertices in $G \backslash  \{x_1, \ldots, x_{k}\}$.  The neighbours of any $x_i, i \in [k]$ must have degree at least $n-k$, and hence are joined to all the $x_i$. Thus, we obtain the desired extremal graph with exactly $\binom{n-k}{2} + k^2$ many edges, namely a clique on $n-k$ vertices and $k$ other independent vertices forming a complete bipartite graph with some $k$ vertices from the clique. \qed
\end{itemize} 
\end{proof}





\section{Concluding Remarks}

A simpler proof of Theorem \ref{maintheorem} with a weaker constant can be obtained using Tur\'an's theorem and a theorem of Erd\H{o}s and Gallai \cite{Erdos1959} on the length of the longest cycle in a graph. %which states that if $G$ is a graph on $n$ vertices and $k \geq 3$, then $|E(G)| > \frac{1}{2}(k-1)(n-1)$ implies that the circumference is at least $k$.
Consider any $k$-cyclable graph with $\alpha(G) \geq k$. Then, let $S$ be a set of $k$ independent vertices, and consider the cycle containing it. This gives us a cycle of length atleast $2k$, as any two independent vertices are not adjacent to each other. Thus, we must have $\alpha(G) < k$. By a variant of Tur\'an's theorem, we also have $\alpha > \frac{n}{\tilde{d} + 1}$, where $\tilde{d}$ is the average degree. Thus, we obtain

$$\frac{2|E(G)|}{n} + 1 = \tilde{d} + 1 > \frac{n}{\alpha} \geq \frac{n}{k-1} \implies |E(G)| \geq \frac{1}{2} n \left( \frac{n}{k-1} - 1 \right)$$

which is larger than $\frac{1}{2} (2k-1)(n-1)$ if $n \geq 2 k^2$. giving $c(G) \geq 2k$ when $k \leq \sqrt{n/2}$.\\

It is also interesting to understand what happens to the circumference of $k$-cyclable graphs for large values of $k$. As mentioned earlier in the introduction, it is not necessarily the case that $c(G) = n$ when $k = n-1$ due to the existence of hypohamiltonian graphs. Thus, we have the following extremal problem. 

\begin{conjecture}
For a given $n$, let $f(n)$ be the largest value of $k$ such that any $k$-cyclable graph satisfies $c(G)>k$. From the above, we have $f(n) < n-1$ and from Theorem \ref{maintheorem}, we know $f(n) = \Omega(n)$. Is it the case that $f(n) = n-2 ?$
\end{conjecture}

\noindent We can also ask for what regime of $k$ as a function of $n$ do results of the type in Theorem \ref{maintheorem} hold.

\begin{conjecture}
For a given $n$, let $g(n)$ be the largest value of $k$ such that any $k$-cyclable graph satisfies $c(G)\geq 2k$. From Theorem \ref{maintheorem} we know $g(n) = \Omega(\sqrt{n})$. Is it the case that $g(n) = O(\sqrt{n}) ?$
\end{conjecture}

Moreover, our results only give an improvement of the form $c(G) \geq (1+ \gamma) k$, $0< \gamma < 1$, for $k$ up to around $2\sqrt{n}$, and it is natural to ask if such a linear bound on the circumference can be obtained for much larger regimes of $k$. Finally, note that the results of Theorem \ref{kconhamtheorem} only hold for $n \geq 2(k^2 + k)$. For fixed values of $k \leq 3$, \cite{Byer2007} give a tight bound for the minimum value of $n$ for this to hold. They also note that this bound cannot hold for $k = \Omega(n)$,  in particular if $p = \floor{\frac{n-1}{2}}$, the graph obtained by joining $n-p$ independent vertices to each vertex of $K_p$ is $k$-connected and nonhamiltonian, with total number of edges more than $\binom{n-k}{2} + k^2$ when $\frac{n+1}{6} < k  < \floor{\frac{n-1}{2}}$. This still leaves a significant gap in the possible range of $k$ for which $k$-connectivity and $|E(G)| > \binom{n-k}{2} + k^2$ implies hamiltonicity, as our result only applies for $k = O(\sqrt{n})$.

%There are also some interesting results in the literature regarding the cyclability of claw-free graphs. A claw refers to an induced $K_{1,3}$ subgraph. Flandrin et al. showed that 





\
%
% ---- Bibliography ----
%
% BibTeX users should specify bibliography style 'splncs04'.
% References will then be sorted and formatted in the correct style.
%

%\newpage

\bibliographystyle{splncs04}
 \bibliography{caldam}
%
% 
% This is samplepaper.tex, a sample chapter demonstrating the
% LLNCS macro package for Springer Computer Science proceedings;
% Version 2.20 of 2017/10/04
%
\documentclass[runningheads,envcountsame]{llncs}
%
\usepackage[12pt]{extsizes}
\usepackage{times}
\usepackage[margin=1in]{geometry}

\let\oldbibliography\thebibliography
\renewcommand{\thebibliography}[1]{\oldbibliography{#1}
\setlength{\itemsep}{0pt}}

\makeatletter
\renewcommand{\@Opargbegintheorem}[4]{%
  #4\trivlist\item[\hskip\labelsep{#3#2\@thmcounterend}]}
\makeatother

\let\labelitemi\labelitemii

\let\conjecture\relax % undefine the environment
\spnewtheorem{conjecture}{Conjecture}{\bfseries}{\rmfamily}

\usepackage{amssymb,amsmath}

\usepackage{mathrsfs}
\usepackage{mathtools}
\usepackage{doi}
\DeclarePairedDelimiter\floor{\lfloor}{\rfloor}
\newcommand{\B}{\textbf}
\newcommand{\baro}{\overline}


 \usepackage{hyperref}
 \hypersetup{
     colorlinks=true,
     linkcolor=blue,
     filecolor=blue,
     citecolor = black,      
     urlcolor=blue,
     }


% Used for displaying a sample figure. If possible, figure files should
% be included in EPS format.
%
% If you use the hyperref package, please uncomment the following line
% to display URLs in blue roman font according to Springer's eBook style:
% \renewcommand\UrlFont{\color{blue}\rmfamily}

\begin{document}

%
\title{Cyclability, Connectivity and Circumference }
%
%\titlerunning{Abbreviated paper title}
% If the paper title is too long for the running head, you can set
% an abbreviated paper title here
%
\author{Niranjan Balachandran\inst{1} \and
 Anish Hebbar\inst{2}}
%

% First names are abbreviated in the running head.
% If there are more than two authors, 'et al.' is used.
%
\institute{Indian Institute of Technology Bombay, India  \and
 Indian Institute of Science, Bangalore, India\\
niranj@iitb.ac.in \quad  anishhebbar@iisc.ac.in 
}
%
\maketitle              % typeset the header of the contribution
%
\begin{abstract}
In a graph $G$, a subset of vertices $S \subseteq V(G)$ is said to be cyclable if there is a cycle containing the vertices in some order. $G$ is said to be $k$-cyclable if any subset of $k \geq 2$ vertices is cyclable. If any $k$ \textit{ordered} vertices are present in a common cycle in that order, then the graph is said to be $k$-ordered. We show that when $k \leq \sqrt{n+3}$, $k$-cyclable graphs also have circumference $c(G) \geq 2k$, and that this is best possible. Furthermore when $k \leq \frac{3n}{4} -1$,  $c(G) \geq k+2$, and for $k$-ordered graphs we show $c(G) \geq \min\{n,2k\}$. We also generalize a result by Byer et al. \cite{Byer2007} on the maximum number of edges in nonhamiltonian $k$-connected graphs, and show that if $G$ is a $k$-connected graph of order $n \geq 2(k^2+k)$ with $|E(G)| > \binom{n-k}{2} + k^2$,  then the graph is hamiltonian, and moreover the extremal graphs are unique. 


\keywords{Cyclability  \and Connectivity \and Circumference \and Hamiltonicity}
\end{abstract}
%
%
%
\section{Introduction}

We consider only finite, undirected, simple graphs throughout this paper. The vertex and edge sets of $G$ will be denoted by $V(G)$ and $E(G)$ respectively, the graph complement by $\baro{G}$. The length of the longest cycle in the graph $G$, also known as the circumference, will be denoted by $c(G)$. %and for any $I \subseteq V(G)$, the subgraph induced by $G$ on $I$ will be denoted by $G_I$.
The minimum degree, independence number and connectivity of a graph will denoted by $\delta(G), \alpha(G)$ and  $\kappa(G)$ respectively. We will also use  $d_{H}(v)$ for the degree of $v$ in $H$. The set of neighbours of a vertex $v \in V(G)$ will denoted by $N(v)$, and the closed neighbourhood of $v$, viz. $N(v) \cup \{v\}$ will be denoted by $N[v]$. The join of two graphs $G_1, G_2$, denoted $G_1 \lor G_2$ is simply a copy of $G_1$ and $G_2$, with all edges between $V(G_1)$ and $V(G_2)$ also being present.\\ 

%A graph $G = (V,E)$ is said to be $k$-cyclable, $2\leq k \leq n$ if any subset of vertices $S \subseteq V$ with $|S| = k$ is contained in some cycle in the graph. Similarly, a graph $G = (V,E)$ is said to be $k$-ordered, $2\leq k \leq n$ if any  sequence of distinct vertices $T = (v_1,v_2, \cdots, v_k)$ is contained in a cycle, the vertices appearing in the specified order. 


A subset $S \subseteq V(G)$ of vertices in a graph $G$ is said to be cyclable if $G$ has a cycle containing the vertices of $S$ in some order, possibly including other vertices. A graph $G$ is said to be $k$-cyclable if any $k \geq 2$ vertices of $G$ lie on a common cycle. Note that the problem of determining the hamiltonicity of a graph is a special case of cyclability, namely when $k = n$. Cyclability and connectivity are interlinked, as was shown by Dirac \cite{Dirac1960} who proved for every $k \geq 2$, $k$-connected graphs are also $k$-cyclable. In fact, for $k = 2$ connectivity and cyclability are equivalent, but in general for $k \geq 3$ it is not necessarily true that every $k$-cyclable graph is also $k$-connected, as can be seen by considering the graph $K_2 \lor 2K_k$ which has connectivity exactly $2$ and is also $k$-cyclable. For a brief survey of results involving conditions for cycles to contain a particular set, refer to \cite{Gould2009}.\\

There is a rich literature on conditions guaranteeing the presence of long cycles in graphs, the most classical one being that of Dirac \cite{Dirac1952} who showed that in $2$-connected graphs, the circumference is at least$c(G) \geq \min\{n, 2 \delta(G)\}$. Moreover, $k$-connected graphs have a circumference of at least $\min\{n,2k\}$ from an easy consequence of Menger's theorem, and this is tight. A famous result by Chvátal and Erd\H{o}s \cite{Chvatal1972} relates the connectivity and independence number of a graph to hamiltonicity, and says that if the connectivity of a graph $G$ is at least its independence number, then the graph is hamiltonian. However, not much is known when the requirement of connectivity is weakened to cyclability. Bauer et al. \cite{Bauer2000} obtained lower bounds for the length of the longest cycle in $3$-cyclable graphs in terms of the minimum degree and independence number, but not much else is known for $k$-cyclable graphs for arbitrary $k$.\\

%  --still need to look into hardness, will do tonight 
%-hardness of circumference and cylability
% --fpt and constant k, np-hardness\\
 
%Thus, cyclability can be thought of as a more quantitative measure of hamiltonicity. 

Cyclability  has also received interest from an algorithmic and complexity theoretic point of view as it is a 'hard' parameter that can be thought of as a more quantitative measure of hamiltonicity. Since the classical HAMILTONIAN CYCLE problem is NP-complete, the problem of determining whether a graph is $k$-cyclable (CYCLABILITY) is NP-complete as well. 
The problem of determining whether a given subset $S$ of vertices is cyclable (TERMINAL CYCLABILITY) has been studied in the Parameterized Complexity framework (FPT) (parameterized by $|S|$) and the best known algorithm has  running time $O(2^{|S|} n^{O(1)})$ \cite{Bjorklund2012}. For some special classes of graphs such as interval graphs and bipartite permutation graphs, Crespelle and Golovach \cite{Crespelle2022} showed that both these problems can be solved in polynomial time. For $|S| = O((\log \log n)^{1/10})$, Kawarabayashi \cite{Kawarabayashi2008} obtained a polynomial time algorithm for TERMINAL CYCLABILITY.\\


%Golovach et al.\cite{golovach} also showed that $k$-CYCLABILITY is co-W[1]-hard for split graphs, and that it is FPT on planar graphs when parameterized by $k$
 




%The problem of determining whether a graph is $k$-cyclable (LABILITY) also has received attention from a complexity theoretic perspective in the Parameterized Complexity framework as it is at least as hard as the classical HAMILTONIAN CYCLE problem which is NP-complete. (denoted TERMINAL CYCLABILITY) which is known to be fixed-parameter tractable (FPT) \cite{Bjorklund2012} when parametrized by $|S|$ .  Crespelle and Golovach \cite{Crespelle2022} showed that $k$-CYCLABILITY can be solved in polynomial time for some restricted graph families, such as interval graphs, bipartite permutation graphs and cographs. They also showed that TERMINAL CYCLABILITY can be solved in linear time for the aforementioned graph classes. Golovach et al.\cite{golovach} also showed that $k$-CYCLABILITY is co-W[1]-hard for split graphs, and that it is FPT on planar graphs when parameterized by $k$.\\


Note that $k$-connectivity guarantees $c(G) \geq \min \{n,2k\}$ and also ensures $k$-cyclability. Thus, a natural question to ask is whether the same bound on the circumference can be obtained when the connectivity criteria is weakened to cyclability. When $k = n-1$, we would require any set of $n-1$ vertices of $G$ to lie on a common cycle. It turns out that in this case, it is not necessary that the graph is hamiltonian. Indeed, the existence of  hypohamiltonian graphs \cite{Doyen1975} of order $n$ is known for all $n \geq 18$. Our first  result in this paper gives a similar circumference bound for a wide range of $k$: 
\begin{theorem}\label{maintheorem}
Let $G$ be a $k$-cyclable graph, where $2 \leq k \leq n$. Then, 

\begin{equation*}
  c(G) \geq
    \begin{cases}
      2k & \text{if } k \leq \sqrt{n+3} \\
      k+2 & \text{if }  k \leq \frac{3n}{4} - 1
    \end{cases}       
\end{equation*}
Moreover, for $2 \leq k \leq \sqrt{n+3}$, this bound on the circumference is best possible.
\end{theorem}

\noindent  Note that for $k\ge\frac{n}{2}$ it is still possible that one can have a bound of the form $c(G)\ge (1+\gamma)k$ for some fixed positive constant $\gamma < 1$ as long as $k \neq n - o(n)$.\\
%We will say something about this in a later section. 

A related notion is the orderedness of a graph, a strong hamiltonian property that was first introduced by Ng
and Schultz \cite{Ng1997}.   A graph $G$ is said to be $k$-ordered if any sequence of distinct vertices $T = \{v_1, \ldots, v_k\}$ are present in some common cycle in that order, possibly including other vertices. Note that $k$-ordered graphs are naturally also $k$-cyclable, and it is also easy to see that they are $(k-1)$-connected. For a comprehensive survey of results on $k$-ordered graphs, see \cite{Faudree2001}. We show that for $k$-orderedness, the same circumference bound as $k$-connectivity holds for all $2 \leq k \leq n$.


\begin{theorem}\label{orderedtheorem}
Let $G$ be a $k$-ordered graph, $2 \leq k \leq n$. Then, $c(G) \geq \min\{n,2k\}$.
\end{theorem}





Our second pursuit in this paper is to obtain Tur\'an-type results for the circumference of $k$-connected graphs, specifically the maximum number of edges in nonhamiltonian $k$-connected graphs. A classical result states that if $G$ is a graph of order $n$ with $|E(G)| > \binom{n-1}{2} + 1$, then $G$ is hamiltonian. This was generalized by \cite{Byer2007} for $k \leq 3$, where they showed that if $G$ is $k$-connected and satisfies $|E(G)| > \binom{n-k}{2} + k^2$ with $n$ sufficiently large, then the graph is hamiltonian and the extremal graphs are unique. We further generalize their result and extend it to any $k$ satisfying $n \geq 2(k^2+k)$.

\begin{theorem} \label{kconhamtheorem}
Let $G$ be a $k$-connected graph of order $n \geq 2(k^2+k)$. If $|E(G)| > \binom{n-k}{2} + k^2$, then $G$ is hamiltonian. Moreover, the extremal graphs are unique.
\end{theorem}

The rest of the paper is organized as follows. We lay out some preliminaries in the next section, and give the proofs of Theorems \ref{maintheorem}, \ref{orderedtheorem}, and \ref{kconhamtheorem} in the following section. We conclude with some remarks and open questions.

\section{Preliminaries} 


%We know that $k$ connected graphs are also $k$ cyclable. Moreover, $2$-cyclability is equivalent to $2$-connectivity, and $k$-cyclable graphs are also $l$ cyclable for any $2 \leq l \leq k$.
When the underlying graph is clear, we will use $\delta, \kappa, \alpha$ instead of $\delta(G), \kappa(G), \alpha(G)$ for brevity, and also omit the subscript in $d_H(v)$.
We also use the following well-known lemma attributed to Dirac repeatedly throughout the paper, and provide an outline of the proof for completeness.
\begin{lemma}[\cite{Dirac1960}] \label{kcontheorem}
Any $k$-connected graph $G$ is $k$-cyclable. Moreover, it satisfies $c(G) \geq \min\{n,2k\}$
\end{lemma}
\begin{proof}[Proof Sketch]
Suppose some subset $S$ of vertices with $|S| = k$ was not fully contained in any cycle. Then, take a cycle $C$ containing as many of the vertices of $S$ as possible, and pick some $v \in S$ that is not in $C$. By Menger's theorem, we can choose $k$ vertex-disjoint paths from $v$ to the cycle $C$, and these endpoints divide $C$ into $k$ segments. Since there are strictly less than $k$ vertices of $S$ in $C$, one of the segments does not contain any vertex from $S$, and thus we can extend this segment with the $2$ disjoint paths from $v$ at the ends of the segment to obtain a cycle containing more vertices of $S$, contradiction. \\
Now consider the longest cycle $C$ in $G$ and suppose its length is strictly less than $\min\{n,2k\}$. Pick some $v \in V(G)$ not in $C$, and by Menger's theorem there are $k$ vertex disjoint paths from $v$ to $C$. By the pigeonhole principle, some two endpoints of these $k$ paths must be adjacent on the cycle $C$, giving a contradiction as we can replace the edge between these endpoints with the $2$ paths to obtain a longer cycle. \qed
\end{proof}

A famous result by  Chvátal and Erd\H{o}s  states the following

\begin{theorem}[\cite{Chvatal1972}] \label{erdostheorem}
If in a graph $G$, $\alpha(G) \leq \kappa(G)$, then $G$ is hamiltonian.
\end{theorem}

A natural generalization of the above is to flip the condition $\alpha(G) \leq \kappa(G)$, and instead ask for lower bounds on the circumference of a graph $G$ where $\alpha(G) \geq \kappa(G)$. Foquet and Jolivet \cite{J.L.Fouquet} conjectured the following, which was later proven by Suil O, Douglas B. West and Hehui Wu.


\begin{theorem}[\cite{O2011}]\label{westtheorem}
If $G$ is a $k$-connected $n$-vertex graph with independence number $\alpha$ and $\alpha \geq k$, then $G$ has a cycle of length at least $\frac{k(n+k-\alpha)}{\alpha}$.
\end{theorem}

The following result by Dirac is well-known and was a precursor to a number of results involving the length of the longest cycle in a graph.

\begin{theorem}[\cite{Dirac1952}]\label{diractheorem}
If $G$ is $2$-connected and has minimum degree $\delta$, $c(G) \geq \min\{2 \delta,n\}$.
\end{theorem}

Note that $2$-connectivity is equivalent to $2$-cyclability. Bauer et al. obtained a bound on the circumference of $3$-cyclable graphs in terms of the minimum degree and independence number.

\begin{theorem}[\cite{Bauer2000}]\label{bauertheorem}
 If $G$ is $3$ cyclable, then $$c(G) \geq min\{n, 3\delta - 3 , n + \delta - \alpha\}.$$
\end{theorem}

Ng and Schultz studied a related hamiltonian property termed $k$-orderedness, and showed the following connectivity result. Once again, we include the proof for completeness.

\begin{lemma}[\cite{Ng1997}]\label{schultztheorem}
Let $G$ be a $k$-ordered graph. Then, $G$ is $(k-1)$-connected.
\end{lemma}
\begin{proof}
If not, there exists a set $S$ of $k-2$ vertices whose removal disconnects $G$, breaking it into at least $2$ components. Take $2$ vertices $u,v$ in different components, then any path from $u$ to $v$ must go through some vertex of $S$. Thus, let $T$ consist of $u$, $v$ and then the vertices of $S$, in that order. These vertices must appear in some cycle in that order, giving a contradiction. \qed

\end{proof}

We will also need the concept of graph closure introduced by Bondy and Chvátal. Define
the closure of $G$, denoted $cl(G)$, to be the graph obtained by repeatedly joining any two nonadjacent vertices $x,y$ that satisfy $d(x) + d(y) \geq n$ in $G$. They showed that $cl(G)$ is well-defined (independent of the order in which nonadjacent vertex pairs are considered), and that $G$ is hamiltonian if and only if $cl(G)$ is also hamiltonian. 


\begin{lemma}[\cite{Bondy1976}]\label{bondytheorem}
Suppose $cl(G) = G$ for a nonhamiltonian graph $G$ of order $n$. Then $d(x) + d(y) \leq n-1$ for any pair $\{x,y\}$ of  nonadjacent vertices.
\end{lemma}

 
This was later generalized to obtain results for higher order connectivity, the bounds now also involving the independence number. We define $$\sigma_{k}(G) = \min \{\sum_{i=1}^{k} d(x_i), \{x_1, \ldots x_{k} \} \text{ an independent set of size k in G}\} $$

Note that $\sigma_1(G)$ simply corresponds to the minimum degree $\delta$, and Ore's theorem \cite{Ore1960} states that if $\sigma_2(G) \geq n$, then the graph is hamiltonian.

\begin{theorem}[\cite{Li2013}] \label{haotheorem}
Let $G$ be a $k$-connected graph of order $n$ and independence number $\alpha$. If $\sigma_{k+1}(G) \geq n+ (k-1) \alpha - (k-1)$, then $G$ is hamiltonian.
\end{theorem}


\section{Proofs of the Results}


\begin{proof}[Proof of Theorem \ref{maintheorem}]
$ $\newline
We will first prove the bound for the regime $2 \leq k \leq \sqrt{n+3}$. \\Consider any $k$-cyclable graph with $\alpha(G) \geq k$. Then, let $S$ be a set of $k$ independent vertices, and consider the cycle containing it. This gives us a cycle of length at least $2k$, as any $2$ independent vertices are not adjacent to each other. Thus, we can assume $\alpha(G) \leq  k - 1$. Let the connectivity of the graph be $\kappa$. Using Theorem \ref{westtheorem}, it suffices to show

$$ \frac{\kappa(n+\kappa - \alpha)}{\alpha} \geq 2k   \iff n \geq 2k(\frac{\alpha}{\kappa}) + (\alpha - \kappa)$$

 \noindent As $k$-cyclable graphs are also $2$-cyclable, and thus $2$-connected, we must have $\kappa \geq 2$. Hence, it is sufficient to show the stronger inequality

$$ n \geq 2k (\frac{k-1}{\kappa}) + k - 3$$ which is always true when $$n \geq k^2-3 \iff k \leq \sqrt{n+3} $$
Note that if we only ask for an improvement of the form $c(G) \geq (1 + \gamma)k$ for some positive constant $\gamma < 1$, we can improve the range of $k$ for which the result holds. Once again, let $S$ be any set of at least $ \frac{(1+ \gamma)k}{2}$ many independent vertices, and consider the cycle containing $S$. This corresponds to a cycle containing at least $(1+ \gamma)k$ many vertices since any two independent vertices are not adjacent, and thus we get $\alpha < \frac{(1+ \gamma)k}{2}$. Similar to the previous argument, if the connectivity of the graph is $\kappa$, by Theorem \ref{westtheorem} it suffices to show

$$ \frac{\kappa(n+\kappa - \alpha)}{\alpha} \geq (1 +\gamma) k   \iff n \geq  (1 +\gamma) k  (\frac{\alpha}{\kappa}) + (\alpha - \kappa)$$

\noindent Using $\kappa \leq 2$ and  $\alpha < \frac{(1+ \gamma)k}{2}$, we are done as long as

$$ n \geq \frac{(1+\gamma)^2k^2}{4} + \frac{(1+\gamma)k}{2} - 2 \iff \frac{\sqrt{4n+9}}{1+ \gamma} \geq k$$
So the above argument only yields a linear improvement in $c(G)$ for $k$ up to around $2 \sqrt{n}$.\\

Now, suppose $2 \leq k \leq \frac{3n}{4} - 1$, and assume to the contrary that $c(G) < k+2$. We must have $k \geq 3$ as $2$-cyclable graphs are $2$-connected and hence have circumference at least $4$ for $n \geq 4$. By Theorem
 \ref{diractheorem}, we must have $\delta \leq \frac{k+1}{2}$. Moreover, $\alpha \leq \frac{k+1}{2}$ as otherwise we could simply take a cycle containing $\alpha$ many independent vertices. Consider a vertex $v$ with minimum degree $\delta$, with neighbourhood $N(v)$ satisfying $|N(v)| = \delta$. Now, choose $v$ and any $k-1$ vertices from $V  \backslash N[v]$, which is possible as long as $k-1 \leq n - 1 - \delta$. Then, any cycle containing these vertices must also contain some $2$ neighbours of $v$, giving $c(G) \geq k+2$, and we are done.

Thus, we must have $k + \delta > n $. Note that when $2 \leq k \leq \frac{3n}{4} -1$, $n \geq k+2$ if $n \geq 4$. So, we must either have $3\delta - 3 \leq k+1$ or $n + \delta - \alpha \leq k+1$, otherwise we are done by Theorem \ref{bauertheorem}.\\  The former inequality gives $\delta \leq \frac{k+4}{3}$, which gives $$n < k + \delta \leq \frac{4k+4}{3} \implies \frac{3n-4}{4} < k$$ a contradiction. Hence, we must have $\delta \geq \frac{k+5}{3}$, $\alpha \leq \frac{k+1}{2}$ giving $$ k+1 \geq n + \delta - \alpha \geq n + \frac{k+5}{3} - \frac{k+1}{2} = n + \frac{7-k}{6}$$ 

\noindent or equivalently,  $\frac{3n}{4} -1 \geq k \geq \frac{6n+1}{7}$, which is again a contradiction. \qed
\end{proof}

\noindent We now prove an analogous bound for the circumference of $k$-ordered graphs. 


\begin{proof}[Proof of Theorem \ref{orderedtheorem}]
$ $\newline
We know that $k$-ordered graphs are also $k-1$ connected from Theorem \ref{schultztheorem}, thus $\kappa \geq k-1$. We also must have $\alpha \leq k-1$, as otherwise we can simply take $k$ independent vertices in any order to obtain a cycle of size at least $2k$, in which case we are done. Hence, $$\kappa \geq k-1 \geq \alpha$$ so by Theorem \ref{erdostheorem}, we have that $G$ is hamiltonian, and thus we are done in this case as well. \qed
\end{proof}


In fact, it is not hard to see that the $\min \{n, 2k\}$ bound  on the circumference is achieved for all $2 \leq k \leq n$. If $k > n/2$, simply consider the complete graph $K_n$ which is clearly $k$-connected, $k$-ordered, $k$-cyclable and has circumference $n$. If $k \leq n/2$, consider the complete bipartite graph $G = K_{k,n-k} = (A,B,E)$, which is $k$-ordered, and hence $k$-cyclable.  Indeed, take any sequence of $k$ distinct vertices $T = (v_1, v_2, \ldots, v_k)$. We construct a cycle containing $T$ in that order as follows.\\

Let $T_A$ be the set of vertices in $T$ and $A$, with $T_B$ being defined similarly. Then, for any $v \in T_A$, if the next vertex in the sequence $T$ is in $T_B$, then simply follow the edge joining them. Otherwise, first follow an edge to a vertex in $B \backslash T_B$, and then back to the next vertex which must have been in $T_A$. Follow the same procedure for vertices in $T_B$. At the end, follow the edge joining the first and last vertex. We cannot run out of vertices as the number of extra vertices outside $T_A$ in $A$ that are needed is at most $|T_B|$, and $|A|= k = |T_A| + |T_B|$. Similarly, $|B| = n -k \geq k =|T_A| + |T_B|$.\\



We now generalize a result by \cite{Byer2007} on the maximal number of edges in a $k$-connected nonhamiltonian graph, for $k=2,3$. We will need the following short lemma which appears in \cite{Byer2007}.


\begin{lemma}[\cite{Byer2007}] \label{byertheorem}
Let $G$ be a nonhamiltonian, $k$-connected graph of order $n$. Then $k \leq \frac{n-1}{2}$ and $|E(\baro{G})| \geq \binom{k+1}{2} + (k-1)(n-k-1) - \sigma_{k+1}(G)$
\end{lemma}
\begin{proof}
By Theorem \ref{erdostheorem}, $k$-connected nonhamiltonian graphs must contain an independent set $I = \{x_1, \ldots , x_{k+1} \}$ of $k+1$ vertices. The graph is disconnected on removal of the the $n -(k+1) $ vertices of $G - I$, thus we must have
$ n - (k+1) > k-1$, or $k \leq \frac{n-1}{2}$.

Now consider the independent set $I$ satisfying $\sum_{i=1}^{k+1} d(x_i) = \sigma_{k+1}(G)$. Let the edges in $\baro{G}$ incident on at least one vertex of $I$ be denoted $X_I$. Then $X_I$ contains $\binom{k+1}{2}$ edges with both endpoints in $I$ and $\sum_{i=1}^{k+1} (n-1 - k -  d_G(x_i))$ edges with exactly one endpoint in $I$. Thus, we obtain
$$ \, \, \quad \quad  \quad\quad \quad \quad \quad \quad  \quad\quad |E(\baro{G})| \geq |X_I| = \binom{k+1}{2} +   (k-1)(n-k-1) - \sigma_{k+1}(G) \quad \quad \quad  \quad  \quad    \, \, \, \, \, \qed $$  
\end{proof}
Using a slight variation of the above result and Lemma \ref{bondytheorem}, \cite{Byer2007} also show the following result.


\begin{lemma}[\cite{Byer2007}] \label{byertheorem2}
Suppose $G = cl(G)$ for a nonhamiltonian graph $G$ of order $n$, and $m \leq \alpha(G)$. Then


\begin{equation*}
  |E(\baro{G})| \geq
    \begin{cases}
      \frac{m}{2}(n-m) & \text{for n odd}\\
      \frac{m}{2}(n-m) + \frac{m}{2} - 1 & \text{for n even}
    \end{cases}       
\end{equation*}
\end{lemma}


\noindent With the above results, we are ready to proceed to the proof of Theorem \ref{kconhamtheorem}. The idea is that if $n$ is not that much bigger than $\alpha$, then we can get a sufficient lower bound on $|E(\baro{G})| $ using Lemma \ref{byertheorem2}. Otherwise, $n$ is much bigger than $\alpha$, and we can use Theorem \ref{haotheorem} and Lemma \ref{byertheorem}. To show the uniqueness of the extremal graphs, we will make use of the fact that these graphs must satisfy Lemma \ref{bondytheorem} \textit{maximally}, i.e., addition of any further edge causes a violation of the condition.

\begin{proof}[Proof of Theorem \ref{kconhamtheorem}]
$ $\newline \
First of all, assume $k \geq 2$ as we already know that when $|E(G)| > \binom{n-1}{2} + 1$, then $G$ is hamiltonian and consequently connected as well.
Assume $G$ is nonhamiltonian. We may assume $G = cl(G)$, in which case $d(x)+ d(y) \leq n-1$ for any two nonadjacent vertices $x,y$, from Lemma \ref{bondytheorem}. It suffices to prove that $$|E(\baro{G})| \geq \binom{n}{2} - \left( \binom{n-k}{2} + k^2\right) = k\cdot n - \frac{3k^2+k}{2} $$ Note first that if $\sigma_{k+1}(G) \leq n + k^2- k -1$, by Lemma \ref{byertheorem}
$$|E(\baro{G})| \geq \binom{k+1}{2} + (k+1)(n-k-1) - (n+k^2 - k - 1) = k \cdot n -  \frac{3k^2+k}{2} $$
as desired. We now assume $\sigma_{k+1}(G) \geq n + k^2 - k$ and show that in this case, $|E(\baro{G})|$ is \textit{strictly} greater than $k\cdot n - \frac{3k^2+k}{2}$. We will divide the problem into two cases, depending on the size of $n$ compared to $\alpha$.\\

\noindent \B{Case 1:} Assume $n > \frac{ (k^2-1) \cdot \alpha  + y}{k}$, where $y = \frac{-k^3 + 4k^2 + 3k + 2}{2}$.

Let $I = \{x_1,x_2,\ldots ,x_{k+1}\}$ be a set of $k+1$ independent vertices satisfying $\sum_{i=1}^{k+1}d(x_i) = \sigma_{k+1}(G)$, and assume without loss of generality that $$d(x_1) \geq \frac{\sigma_{k+1}(G)}{k+1} \geq \frac{n+k^2-k}{k+1}$$ 

\B{Subcase 1a:} Suppose $d(x_1) \geq n - 2k$. Note that $V(G) - I - N(x_1)$ is non-empty, as otherwise we would have $d(x_1) = n - k -1$, giving $d(x_i) \leq k$ for $2 \leq i \leq k+1$ as $d(x_1) + d(x_i) \leq n-1$ for $2 \leq i \leq k +1$.  This contradicts $\sigma_{k+1}(G) \geq n + k^2 - k$.  Thus, pick some $v \in V(G) - I - N(x_1)$, giving $d_{\baro{G}}(v) = n - 1 - d_G(v) \geq d_G(x_1) \geq n - 2k$. Therefore, $\baro{G}$ contains at least $n - 2k - |I| = n - 3k - 1$ edges with both endpoints not in $I$. Using the same bound we got in Lemma \ref{byertheorem} but also including the extra edges in $\baro{G}$ incident with $v$ (that have no endpoint in $I$) and using Theorem \ref{haotheorem}, we obtain
\begin{align*}
|E(\baro{G})| &\geq \binom{k+1}{2} + (k+1)(n-k-1) + (n - 3k - 1) - \sigma_{k+1}(G)\\ 
& \geq (k+2)\cdot n - \frac{k^2+9k+4}{2} - (n + (k-1) \alpha - k)\\
& > k\cdot n -  \frac{3k^2+ k}{2} +  \frac{3k^2+ k}{2} -  \frac{k^2+9k+4}{2} + k +  \frac{ (k^2-1) \cdot \alpha  + y}{k} - (k-1)\alpha\\
& =  (k\cdot n -  \frac{3k^2+ k}{2}) +  \frac{  (k-1)\cdot \alpha  + y + k(k^2-3k-2)}{k}  >  (k\cdot n -  \frac{3k^2+ k}{2})
\end{align*}
as desired, where the last inequality follows from $y = \frac{-k^3 + 4k^2 + 3k + 2}{2}$.\\

\B{Subcase 1b:} Suppose next that $d(x_1) \leq n - 2k- 1$. Then there exist distinct vertices $v_1, v_2 \ldots, v_{k} \in V(G) - I - N(x_1)$, and $\baro{G}$ contains at least $$(d_{\baro{G}}(v_1) - k - 1) + (d_{\baro{G}}(v_2) - k - 2) + \cdots + (d_{\baro{G}}(v_k) - 2k) = \sum_{i=1}^k d_{\baro{G}}(v_i) - \frac{3k^2 + k}{2}$$ edges with neither endpoint in $I$. Using $d(v_i) + d(x_1) \leq n-1$ as $G = cl(G)$, we get $d_{\baro{G}}(v_i) \geq d_{G}(x_1) \geq \frac{n + k^2 -k}{k+1}$ for all $1 \leq i \leq k$. Consequently, we obtain at least $$\frac{k(n+k^2-k)}{k+1} - \frac{3k^2+k}{2}$$ edges in $\baro{G}$ with neither endpoint in $I$. Using Theorem \ref{haotheorem} and Lemma \ref{byertheorem} again, we get

 
\begin{align*}
|E(\baro{G})| &\geq \binom{k+1}{2} + (k+1)(n-k-1) + \frac{k(n+k^2-k)}{k+1} - \frac{3k^2+k}{2} - (n + (k-1)\alpha - k)\\
& = (kn  - \frac{3k^2+k}{2}) + \frac{k}{k+1}n -(k-1)\alpha +  \binom{k+1}{2} -(k+1)^2 + \frac{k(k^2-k)}{k+1} + k\\
&> (kn  - \frac{3k^2+k}{2})  + \frac{k}{k+1} \frac{(k^2-1)\alpha + y}{k} - (k-1)\alpha + \frac{-k^2-k-2}{2} + \frac{k(k^2-k)}{k+1}  \\
& = (kn  - \frac{3k^2+k}{2}) + \frac{1}{k+1} \left( \frac{-k^3+4k^2+3k+2}{2}+ \frac{(-k^2-k-2)(k+1)}{2}  + k^3 - k^2\right) \\
& = kn - \frac{3k^2+k}{2}
\end{align*} as desired.\\

\noindent  \B{Case 2:} Assume $n \leq \frac{(k^2-1)\alpha + y}{k}$.\\
In this case, $\alpha \geq \frac{nk - y}{k^2-1}$. By Lemma \ref{byertheorem2}, $|E(\baro{G})| \geq \frac{1}{2} \alpha(n- \alpha)$. This is a upward facing parabola for fixed $n$, so for $\frac{nk - y}{k^2-1} \leq \alpha \leq n - \frac{nk - y}{k^2-1}$, this function is minimized at $\alpha = \frac{nk - y}{k^2-1}$. Therefore, in this range

\begin{align*}
|E(\baro{G})| &\geq \frac{\alpha}{2} (n - \alpha) \geq \frac{1}{2} (\frac{nk - y}{k^2-1})(\frac{n(k^2-k-1) +y}{k^2-1})\\
& = \frac{ n^2k(k^2 - k - 1) + n(2k+1-k^2)y - y^2)}{2(k^2-1)^2} 
\end{align*}

\noindent If we want the above to be strictly greater than $kn - \frac{3k^2+k}{2} $, $$\frac{ n^2k(k^2 - k - 1)}{2(k^2-1)^2}  \geq kn \iff n \geq \frac{2(k^2-1)^2}{k^2 - k - 1} = 2(k^2 + k + \frac{1-k}{k^2-k-1})$$ suffices. This is because for $k \geq 5$, $y = \frac{-k^3 + 4k^2 3k+2}{2} < 0$ and $2k+1 - k^2 < 0$, giving $(2k+1-k^2)(y) > 0$. Similarly, $ -y^2 = \frac{(-k^3 + 4k^2 + 3k+2)^2}{4} > -(3k^2+k)(k^2-1)^2$ for $k \geq 5$, so we only have to check the cases of $k = 2,3,4$ manually which is a routine check.\\

Now, it remains to consider the possiblility that $\alpha > n - \frac{nk - y}{k^2-1} = \frac{ n(k^2-k-1) +y}{k^2-1}$. In this case however,  $\alpha$ is quite large compared to $n$, so the $\binom{\alpha}{2}$ edges in $\baro{G}$ between the vertices of an independent set of size $\alpha$ is strictly greater than $ k \cdot n - \frac{3k^2 + k}{2}$ for all $n$. Indeed, we manually verify for $k \leq 3$, and for $k \geq 4$ simply note that $\frac{nk}{2} + y \geq 0$, and hence when $n \geq 2(k^2+k)$ we have $$\alpha > \frac{n(k^2 - \frac{3k}{2} -1) }{k^2 -1} \geq \frac{9n}{15}, \quad \binom{ 9n/15}{2} > \frac{9n}{30} \cdot \frac{8n}{15} > kn$$


We now prove that the extremal  nonhamiltonian $k$-connected graphs  are unique for $n \geq 2(k^2+k)$, by making use of Lemma \ref{bondytheorem}. Recall that we may assume  $G = cl(G)$ is a
nonhamiltonian, $k$-connected graph of order $n \geq 2k^2 + 2k$ with $\sigma_{k+1}(G)=n+k^2 - k - 1$ as equality only holds if all the inequalities in the above proof are tight.\\

Thus, all the edges in $\baro{G}$ have atleast one endpoint in $I$. Let $I =\{x_1, x_2, \ldots, x_{k+1}\}$ be a set of independent vertices such that $k \leq d(x_1) \leq \ldots \leq d(x_{k+1}) $. Note that $k$-connected graphs have minimum degree at least $k$ as otherwise, the graph could be disconnected by removing at most $k-1$ vertices.  As mentioned in the previous section, we may further assume that all edges in $\baro{G}$ have at least one endpoint in $I$, that is, if $x,y \in V(G) - I$, then $\{x,y\} \in E(G)$. We will now use the properties of graph closure repeatedly. First, note that we must have a clique on the remaining $n - k - 1$ vertices, each of which has degree at least $n - k -2$.
\begin{itemize}
    \item Say $d(x_k) \geq k + 1$  Consider the neighbours of $x_k$ in the clique. These neighbours have degree at least $n - k - 1$, and hence since $G = cl(G)$, must be adjacent to $x_{k+1}$ as well as $d(x_{k+1}) \geq k+1$, But then, these neighbours have degree at least $n - k$, and hence must be adjacent to all of $x_1, \ldots, x_{k+1}$ by the same argument. 
    Thus, $I$ and $N(I)$ together form a complete bipartite graph with $|N(I)| \geq k+1 = |I|$. If $d(x_{k+1}) > k+1$, then it is easy to see that the graph is hamiltonian, and otherwise $k+1 = d(x_i) \, \forall i \in [k+1]$, giving $$ \sigma_{k+1} = n+ k^2 - k -1 =  (k+1)^2 \iff n = 3k+2$$ which is false as we assumed $n \geq 2k^2 + 2k$.
    
    \item Otherwise $d(x_k) = k$, 
    , and hence $d(x_{k+1}) = \sigma_{k+1} - k^2 = n - k - 1$, so we have a clique on the $n- k$ vertices in $G \backslash  \{x_1, \ldots, x_{k}\}$.  The neighbours of any $x_i, i \in [k]$ must have degree at least $n-k$, and hence are joined to all the $x_i$. Thus, we obtain the desired extremal graph with exactly $\binom{n-k}{2} + k^2$ many edges, namely a clique on $n-k$ vertices and $k$ other independent vertices forming a complete bipartite graph with some $k$ vertices from the clique. \qed
\end{itemize} 
\end{proof}





\section{Concluding Remarks}

A simpler proof of Theorem \ref{maintheorem} with a weaker constant can be obtained using Tur\'an's theorem and a theorem of Erd\H{o}s and Gallai \cite{Erdos1959} on the length of the longest cycle in a graph. %which states that if $G$ is a graph on $n$ vertices and $k \geq 3$, then $|E(G)| > \frac{1}{2}(k-1)(n-1)$ implies that the circumference is at least $k$.
Consider any $k$-cyclable graph with $\alpha(G) \geq k$. Then, let $S$ be a set of $k$ independent vertices, and consider the cycle containing it. This gives us a cycle of length atleast $2k$, as any two independent vertices are not adjacent to each other. Thus, we must have $\alpha(G) < k$. By a variant of Tur\'an's theorem, we also have $\alpha > \frac{n}{\tilde{d} + 1}$, where $\tilde{d}$ is the average degree. Thus, we obtain

$$\frac{2|E(G)|}{n} + 1 = \tilde{d} + 1 > \frac{n}{\alpha} \geq \frac{n}{k-1} \implies |E(G)| \geq \frac{1}{2} n \left( \frac{n}{k-1} - 1 \right)$$

which is larger than $\frac{1}{2} (2k-1)(n-1)$ if $n \geq 2 k^2$. giving $c(G) \geq 2k$ when $k \leq \sqrt{n/2}$.\\

It is also interesting to understand what happens to the circumference of $k$-cyclable graphs for large values of $k$. As mentioned earlier in the introduction, it is not necessarily the case that $c(G) = n$ when $k = n-1$ due to the existence of hypohamiltonian graphs. Thus, we have the following extremal problem. 

\begin{conjecture}
For a given $n$, let $f(n)$ be the largest value of $k$ such that any $k$-cyclable graph satisfies $c(G)>k$. From the above, we have $f(n) < n-1$ and from Theorem \ref{maintheorem}, we know $f(n) = \Omega(n)$. Is it the case that $f(n) = n-2 ?$
\end{conjecture}

\noindent We can also ask for what regime of $k$ as a function of $n$ do results of the type in Theorem \ref{maintheorem} hold.

\begin{conjecture}
For a given $n$, let $g(n)$ be the largest value of $k$ such that any $k$-cyclable graph satisfies $c(G)\geq 2k$. From Theorem \ref{maintheorem} we know $g(n) = \Omega(\sqrt{n})$. Is it the case that $g(n) = O(\sqrt{n}) ?$
\end{conjecture}

Moreover, our results only give an improvement of the form $c(G) \geq (1+ \gamma) k$, $0< \gamma < 1$, for $k$ up to around $2\sqrt{n}$, and it is natural to ask if such a linear bound on the circumference can be obtained for much larger regimes of $k$. Finally, note that the results of Theorem \ref{kconhamtheorem} only hold for $n \geq 2(k^2 + k)$. For fixed values of $k \leq 3$, \cite{Byer2007} give a tight bound for the minimum value of $n$ for this to hold. They also note that this bound cannot hold for $k = \Omega(n)$,  in particular if $p = \floor{\frac{n-1}{2}}$, the graph obtained by joining $n-p$ independent vertices to each vertex of $K_p$ is $k$-connected and nonhamiltonian, with total number of edges more than $\binom{n-k}{2} + k^2$ when $\frac{n+1}{6} < k  < \floor{\frac{n-1}{2}}$. This still leaves a significant gap in the possible range of $k$ for which $k$-connectivity and $|E(G)| > \binom{n-k}{2} + k^2$ implies hamiltonicity, as our result only applies for $k = O(\sqrt{n})$.

%There are also some interesting results in the literature regarding the cyclability of claw-free graphs. A claw refers to an induced $K_{1,3}$ subgraph. Flandrin et al. showed that 





\
%
% ---- Bibliography ----
%
% BibTeX users should specify bibliography style 'splncs04'.
% References will then be sorted and formatted in the correct style.
%

%\newpage

\bibliographystyle{splncs04}
 \bibliography{caldam}
%
% 
% This is samplepaper.tex, a sample chapter demonstrating the
% LLNCS macro package for Springer Computer Science proceedings;
% Version 2.20 of 2017/10/04
%
\documentclass[runningheads,envcountsame]{llncs}
%
\usepackage[12pt]{extsizes}
\usepackage{times}
\usepackage[margin=1in]{geometry}

\let\oldbibliography\thebibliography
\renewcommand{\thebibliography}[1]{\oldbibliography{#1}
\setlength{\itemsep}{0pt}}

\makeatletter
\renewcommand{\@Opargbegintheorem}[4]{%
  #4\trivlist\item[\hskip\labelsep{#3#2\@thmcounterend}]}
\makeatother

\let\labelitemi\labelitemii

\let\conjecture\relax % undefine the environment
\spnewtheorem{conjecture}{Conjecture}{\bfseries}{\rmfamily}

\usepackage{amssymb,amsmath}

\usepackage{mathrsfs}
\usepackage{mathtools}
\usepackage{doi}
\DeclarePairedDelimiter\floor{\lfloor}{\rfloor}
\newcommand{\B}{\textbf}
\newcommand{\baro}{\overline}


 \usepackage{hyperref}
 \hypersetup{
     colorlinks=true,
     linkcolor=blue,
     filecolor=blue,
     citecolor = black,      
     urlcolor=blue,
     }


% Used for displaying a sample figure. If possible, figure files should
% be included in EPS format.
%
% If you use the hyperref package, please uncomment the following line
% to display URLs in blue roman font according to Springer's eBook style:
% \renewcommand\UrlFont{\color{blue}\rmfamily}

\begin{document}

%
\title{Cyclability, Connectivity and Circumference }
%
%\titlerunning{Abbreviated paper title}
% If the paper title is too long for the running head, you can set
% an abbreviated paper title here
%
\author{Niranjan Balachandran\inst{1} \and
 Anish Hebbar\inst{2}}
%

% First names are abbreviated in the running head.
% If there are more than two authors, 'et al.' is used.
%
\institute{Indian Institute of Technology Bombay, India  \and
 Indian Institute of Science, Bangalore, India\\
niranj@iitb.ac.in \quad  anishhebbar@iisc.ac.in 
}
%
\maketitle              % typeset the header of the contribution
%
\begin{abstract}
In a graph $G$, a subset of vertices $S \subseteq V(G)$ is said to be cyclable if there is a cycle containing the vertices in some order. $G$ is said to be $k$-cyclable if any subset of $k \geq 2$ vertices is cyclable. If any $k$ \textit{ordered} vertices are present in a common cycle in that order, then the graph is said to be $k$-ordered. We show that when $k \leq \sqrt{n+3}$, $k$-cyclable graphs also have circumference $c(G) \geq 2k$, and that this is best possible. Furthermore when $k \leq \frac{3n}{4} -1$,  $c(G) \geq k+2$, and for $k$-ordered graphs we show $c(G) \geq \min\{n,2k\}$. We also generalize a result by Byer et al. \cite{Byer2007} on the maximum number of edges in nonhamiltonian $k$-connected graphs, and show that if $G$ is a $k$-connected graph of order $n \geq 2(k^2+k)$ with $|E(G)| > \binom{n-k}{2} + k^2$,  then the graph is hamiltonian, and moreover the extremal graphs are unique. 


\keywords{Cyclability  \and Connectivity \and Circumference \and Hamiltonicity}
\end{abstract}
%
%
%
\section{Introduction}

We consider only finite, undirected, simple graphs throughout this paper. The vertex and edge sets of $G$ will be denoted by $V(G)$ and $E(G)$ respectively, the graph complement by $\baro{G}$. The length of the longest cycle in the graph $G$, also known as the circumference, will be denoted by $c(G)$. %and for any $I \subseteq V(G)$, the subgraph induced by $G$ on $I$ will be denoted by $G_I$.
The minimum degree, independence number and connectivity of a graph will denoted by $\delta(G), \alpha(G)$ and  $\kappa(G)$ respectively. We will also use  $d_{H}(v)$ for the degree of $v$ in $H$. The set of neighbours of a vertex $v \in V(G)$ will denoted by $N(v)$, and the closed neighbourhood of $v$, viz. $N(v) \cup \{v\}$ will be denoted by $N[v]$. The join of two graphs $G_1, G_2$, denoted $G_1 \lor G_2$ is simply a copy of $G_1$ and $G_2$, with all edges between $V(G_1)$ and $V(G_2)$ also being present.\\ 

%A graph $G = (V,E)$ is said to be $k$-cyclable, $2\leq k \leq n$ if any subset of vertices $S \subseteq V$ with $|S| = k$ is contained in some cycle in the graph. Similarly, a graph $G = (V,E)$ is said to be $k$-ordered, $2\leq k \leq n$ if any  sequence of distinct vertices $T = (v_1,v_2, \cdots, v_k)$ is contained in a cycle, the vertices appearing in the specified order. 


A subset $S \subseteq V(G)$ of vertices in a graph $G$ is said to be cyclable if $G$ has a cycle containing the vertices of $S$ in some order, possibly including other vertices. A graph $G$ is said to be $k$-cyclable if any $k \geq 2$ vertices of $G$ lie on a common cycle. Note that the problem of determining the hamiltonicity of a graph is a special case of cyclability, namely when $k = n$. Cyclability and connectivity are interlinked, as was shown by Dirac \cite{Dirac1960} who proved for every $k \geq 2$, $k$-connected graphs are also $k$-cyclable. In fact, for $k = 2$ connectivity and cyclability are equivalent, but in general for $k \geq 3$ it is not necessarily true that every $k$-cyclable graph is also $k$-connected, as can be seen by considering the graph $K_2 \lor 2K_k$ which has connectivity exactly $2$ and is also $k$-cyclable. For a brief survey of results involving conditions for cycles to contain a particular set, refer to \cite{Gould2009}.\\

There is a rich literature on conditions guaranteeing the presence of long cycles in graphs, the most classical one being that of Dirac \cite{Dirac1952} who showed that in $2$-connected graphs, the circumference is at least$c(G) \geq \min\{n, 2 \delta(G)\}$. Moreover, $k$-connected graphs have a circumference of at least $\min\{n,2k\}$ from an easy consequence of Menger's theorem, and this is tight. A famous result by Chvátal and Erd\H{o}s \cite{Chvatal1972} relates the connectivity and independence number of a graph to hamiltonicity, and says that if the connectivity of a graph $G$ is at least its independence number, then the graph is hamiltonian. However, not much is known when the requirement of connectivity is weakened to cyclability. Bauer et al. \cite{Bauer2000} obtained lower bounds for the length of the longest cycle in $3$-cyclable graphs in terms of the minimum degree and independence number, but not much else is known for $k$-cyclable graphs for arbitrary $k$.\\

%  --still need to look into hardness, will do tonight 
%-hardness of circumference and cylability
% --fpt and constant k, np-hardness\\
 
%Thus, cyclability can be thought of as a more quantitative measure of hamiltonicity. 

Cyclability  has also received interest from an algorithmic and complexity theoretic point of view as it is a 'hard' parameter that can be thought of as a more quantitative measure of hamiltonicity. Since the classical HAMILTONIAN CYCLE problem is NP-complete, the problem of determining whether a graph is $k$-cyclable (CYCLABILITY) is NP-complete as well. 
The problem of determining whether a given subset $S$ of vertices is cyclable (TERMINAL CYCLABILITY) has been studied in the Parameterized Complexity framework (FPT) (parameterized by $|S|$) and the best known algorithm has  running time $O(2^{|S|} n^{O(1)})$ \cite{Bjorklund2012}. For some special classes of graphs such as interval graphs and bipartite permutation graphs, Crespelle and Golovach \cite{Crespelle2022} showed that both these problems can be solved in polynomial time. For $|S| = O((\log \log n)^{1/10})$, Kawarabayashi \cite{Kawarabayashi2008} obtained a polynomial time algorithm for TERMINAL CYCLABILITY.\\


%Golovach et al.\cite{golovach} also showed that $k$-CYCLABILITY is co-W[1]-hard for split graphs, and that it is FPT on planar graphs when parameterized by $k$
 




%The problem of determining whether a graph is $k$-cyclable (LABILITY) also has received attention from a complexity theoretic perspective in the Parameterized Complexity framework as it is at least as hard as the classical HAMILTONIAN CYCLE problem which is NP-complete. (denoted TERMINAL CYCLABILITY) which is known to be fixed-parameter tractable (FPT) \cite{Bjorklund2012} when parametrized by $|S|$ .  Crespelle and Golovach \cite{Crespelle2022} showed that $k$-CYCLABILITY can be solved in polynomial time for some restricted graph families, such as interval graphs, bipartite permutation graphs and cographs. They also showed that TERMINAL CYCLABILITY can be solved in linear time for the aforementioned graph classes. Golovach et al.\cite{golovach} also showed that $k$-CYCLABILITY is co-W[1]-hard for split graphs, and that it is FPT on planar graphs when parameterized by $k$.\\


Note that $k$-connectivity guarantees $c(G) \geq \min \{n,2k\}$ and also ensures $k$-cyclability. Thus, a natural question to ask is whether the same bound on the circumference can be obtained when the connectivity criteria is weakened to cyclability. When $k = n-1$, we would require any set of $n-1$ vertices of $G$ to lie on a common cycle. It turns out that in this case, it is not necessary that the graph is hamiltonian. Indeed, the existence of  hypohamiltonian graphs \cite{Doyen1975} of order $n$ is known for all $n \geq 18$. Our first  result in this paper gives a similar circumference bound for a wide range of $k$: 
\begin{theorem}\label{maintheorem}
Let $G$ be a $k$-cyclable graph, where $2 \leq k \leq n$. Then, 

\begin{equation*}
  c(G) \geq
    \begin{cases}
      2k & \text{if } k \leq \sqrt{n+3} \\
      k+2 & \text{if }  k \leq \frac{3n}{4} - 1
    \end{cases}       
\end{equation*}
Moreover, for $2 \leq k \leq \sqrt{n+3}$, this bound on the circumference is best possible.
\end{theorem}

\noindent  Note that for $k\ge\frac{n}{2}$ it is still possible that one can have a bound of the form $c(G)\ge (1+\gamma)k$ for some fixed positive constant $\gamma < 1$ as long as $k \neq n - o(n)$.\\
%We will say something about this in a later section. 

A related notion is the orderedness of a graph, a strong hamiltonian property that was first introduced by Ng
and Schultz \cite{Ng1997}.   A graph $G$ is said to be $k$-ordered if any sequence of distinct vertices $T = \{v_1, \ldots, v_k\}$ are present in some common cycle in that order, possibly including other vertices. Note that $k$-ordered graphs are naturally also $k$-cyclable, and it is also easy to see that they are $(k-1)$-connected. For a comprehensive survey of results on $k$-ordered graphs, see \cite{Faudree2001}. We show that for $k$-orderedness, the same circumference bound as $k$-connectivity holds for all $2 \leq k \leq n$.


\begin{theorem}\label{orderedtheorem}
Let $G$ be a $k$-ordered graph, $2 \leq k \leq n$. Then, $c(G) \geq \min\{n,2k\}$.
\end{theorem}





Our second pursuit in this paper is to obtain Tur\'an-type results for the circumference of $k$-connected graphs, specifically the maximum number of edges in nonhamiltonian $k$-connected graphs. A classical result states that if $G$ is a graph of order $n$ with $|E(G)| > \binom{n-1}{2} + 1$, then $G$ is hamiltonian. This was generalized by \cite{Byer2007} for $k \leq 3$, where they showed that if $G$ is $k$-connected and satisfies $|E(G)| > \binom{n-k}{2} + k^2$ with $n$ sufficiently large, then the graph is hamiltonian and the extremal graphs are unique. We further generalize their result and extend it to any $k$ satisfying $n \geq 2(k^2+k)$.

\begin{theorem} \label{kconhamtheorem}
Let $G$ be a $k$-connected graph of order $n \geq 2(k^2+k)$. If $|E(G)| > \binom{n-k}{2} + k^2$, then $G$ is hamiltonian. Moreover, the extremal graphs are unique.
\end{theorem}

The rest of the paper is organized as follows. We lay out some preliminaries in the next section, and give the proofs of Theorems \ref{maintheorem}, \ref{orderedtheorem}, and \ref{kconhamtheorem} in the following section. We conclude with some remarks and open questions.

\section{Preliminaries} 


%We know that $k$ connected graphs are also $k$ cyclable. Moreover, $2$-cyclability is equivalent to $2$-connectivity, and $k$-cyclable graphs are also $l$ cyclable for any $2 \leq l \leq k$.
When the underlying graph is clear, we will use $\delta, \kappa, \alpha$ instead of $\delta(G), \kappa(G), \alpha(G)$ for brevity, and also omit the subscript in $d_H(v)$.
We also use the following well-known lemma attributed to Dirac repeatedly throughout the paper, and provide an outline of the proof for completeness.
\begin{lemma}[\cite{Dirac1960}] \label{kcontheorem}
Any $k$-connected graph $G$ is $k$-cyclable. Moreover, it satisfies $c(G) \geq \min\{n,2k\}$
\end{lemma}
\begin{proof}[Proof Sketch]
Suppose some subset $S$ of vertices with $|S| = k$ was not fully contained in any cycle. Then, take a cycle $C$ containing as many of the vertices of $S$ as possible, and pick some $v \in S$ that is not in $C$. By Menger's theorem, we can choose $k$ vertex-disjoint paths from $v$ to the cycle $C$, and these endpoints divide $C$ into $k$ segments. Since there are strictly less than $k$ vertices of $S$ in $C$, one of the segments does not contain any vertex from $S$, and thus we can extend this segment with the $2$ disjoint paths from $v$ at the ends of the segment to obtain a cycle containing more vertices of $S$, contradiction. \\
Now consider the longest cycle $C$ in $G$ and suppose its length is strictly less than $\min\{n,2k\}$. Pick some $v \in V(G)$ not in $C$, and by Menger's theorem there are $k$ vertex disjoint paths from $v$ to $C$. By the pigeonhole principle, some two endpoints of these $k$ paths must be adjacent on the cycle $C$, giving a contradiction as we can replace the edge between these endpoints with the $2$ paths to obtain a longer cycle. \qed
\end{proof}

A famous result by  Chvátal and Erd\H{o}s  states the following

\begin{theorem}[\cite{Chvatal1972}] \label{erdostheorem}
If in a graph $G$, $\alpha(G) \leq \kappa(G)$, then $G$ is hamiltonian.
\end{theorem}

A natural generalization of the above is to flip the condition $\alpha(G) \leq \kappa(G)$, and instead ask for lower bounds on the circumference of a graph $G$ where $\alpha(G) \geq \kappa(G)$. Foquet and Jolivet \cite{J.L.Fouquet} conjectured the following, which was later proven by Suil O, Douglas B. West and Hehui Wu.


\begin{theorem}[\cite{O2011}]\label{westtheorem}
If $G$ is a $k$-connected $n$-vertex graph with independence number $\alpha$ and $\alpha \geq k$, then $G$ has a cycle of length at least $\frac{k(n+k-\alpha)}{\alpha}$.
\end{theorem}

The following result by Dirac is well-known and was a precursor to a number of results involving the length of the longest cycle in a graph.

\begin{theorem}[\cite{Dirac1952}]\label{diractheorem}
If $G$ is $2$-connected and has minimum degree $\delta$, $c(G) \geq \min\{2 \delta,n\}$.
\end{theorem}

Note that $2$-connectivity is equivalent to $2$-cyclability. Bauer et al. obtained a bound on the circumference of $3$-cyclable graphs in terms of the minimum degree and independence number.

\begin{theorem}[\cite{Bauer2000}]\label{bauertheorem}
 If $G$ is $3$ cyclable, then $$c(G) \geq min\{n, 3\delta - 3 , n + \delta - \alpha\}.$$
\end{theorem}

Ng and Schultz studied a related hamiltonian property termed $k$-orderedness, and showed the following connectivity result. Once again, we include the proof for completeness.

\begin{lemma}[\cite{Ng1997}]\label{schultztheorem}
Let $G$ be a $k$-ordered graph. Then, $G$ is $(k-1)$-connected.
\end{lemma}
\begin{proof}
If not, there exists a set $S$ of $k-2$ vertices whose removal disconnects $G$, breaking it into at least $2$ components. Take $2$ vertices $u,v$ in different components, then any path from $u$ to $v$ must go through some vertex of $S$. Thus, let $T$ consist of $u$, $v$ and then the vertices of $S$, in that order. These vertices must appear in some cycle in that order, giving a contradiction. \qed

\end{proof}

We will also need the concept of graph closure introduced by Bondy and Chvátal. Define
the closure of $G$, denoted $cl(G)$, to be the graph obtained by repeatedly joining any two nonadjacent vertices $x,y$ that satisfy $d(x) + d(y) \geq n$ in $G$. They showed that $cl(G)$ is well-defined (independent of the order in which nonadjacent vertex pairs are considered), and that $G$ is hamiltonian if and only if $cl(G)$ is also hamiltonian. 


\begin{lemma}[\cite{Bondy1976}]\label{bondytheorem}
Suppose $cl(G) = G$ for a nonhamiltonian graph $G$ of order $n$. Then $d(x) + d(y) \leq n-1$ for any pair $\{x,y\}$ of  nonadjacent vertices.
\end{lemma}

 
This was later generalized to obtain results for higher order connectivity, the bounds now also involving the independence number. We define $$\sigma_{k}(G) = \min \{\sum_{i=1}^{k} d(x_i), \{x_1, \ldots x_{k} \} \text{ an independent set of size k in G}\} $$

Note that $\sigma_1(G)$ simply corresponds to the minimum degree $\delta$, and Ore's theorem \cite{Ore1960} states that if $\sigma_2(G) \geq n$, then the graph is hamiltonian.

\begin{theorem}[\cite{Li2013}] \label{haotheorem}
Let $G$ be a $k$-connected graph of order $n$ and independence number $\alpha$. If $\sigma_{k+1}(G) \geq n+ (k-1) \alpha - (k-1)$, then $G$ is hamiltonian.
\end{theorem}


\section{Proofs of the Results}


\begin{proof}[Proof of Theorem \ref{maintheorem}]
$ $\newline
We will first prove the bound for the regime $2 \leq k \leq \sqrt{n+3}$. \\Consider any $k$-cyclable graph with $\alpha(G) \geq k$. Then, let $S$ be a set of $k$ independent vertices, and consider the cycle containing it. This gives us a cycle of length at least $2k$, as any $2$ independent vertices are not adjacent to each other. Thus, we can assume $\alpha(G) \leq  k - 1$. Let the connectivity of the graph be $\kappa$. Using Theorem \ref{westtheorem}, it suffices to show

$$ \frac{\kappa(n+\kappa - \alpha)}{\alpha} \geq 2k   \iff n \geq 2k(\frac{\alpha}{\kappa}) + (\alpha - \kappa)$$

 \noindent As $k$-cyclable graphs are also $2$-cyclable, and thus $2$-connected, we must have $\kappa \geq 2$. Hence, it is sufficient to show the stronger inequality

$$ n \geq 2k (\frac{k-1}{\kappa}) + k - 3$$ which is always true when $$n \geq k^2-3 \iff k \leq \sqrt{n+3} $$
Note that if we only ask for an improvement of the form $c(G) \geq (1 + \gamma)k$ for some positive constant $\gamma < 1$, we can improve the range of $k$ for which the result holds. Once again, let $S$ be any set of at least $ \frac{(1+ \gamma)k}{2}$ many independent vertices, and consider the cycle containing $S$. This corresponds to a cycle containing at least $(1+ \gamma)k$ many vertices since any two independent vertices are not adjacent, and thus we get $\alpha < \frac{(1+ \gamma)k}{2}$. Similar to the previous argument, if the connectivity of the graph is $\kappa$, by Theorem \ref{westtheorem} it suffices to show

$$ \frac{\kappa(n+\kappa - \alpha)}{\alpha} \geq (1 +\gamma) k   \iff n \geq  (1 +\gamma) k  (\frac{\alpha}{\kappa}) + (\alpha - \kappa)$$

\noindent Using $\kappa \leq 2$ and  $\alpha < \frac{(1+ \gamma)k}{2}$, we are done as long as

$$ n \geq \frac{(1+\gamma)^2k^2}{4} + \frac{(1+\gamma)k}{2} - 2 \iff \frac{\sqrt{4n+9}}{1+ \gamma} \geq k$$
So the above argument only yields a linear improvement in $c(G)$ for $k$ up to around $2 \sqrt{n}$.\\

Now, suppose $2 \leq k \leq \frac{3n}{4} - 1$, and assume to the contrary that $c(G) < k+2$. We must have $k \geq 3$ as $2$-cyclable graphs are $2$-connected and hence have circumference at least $4$ for $n \geq 4$. By Theorem
 \ref{diractheorem}, we must have $\delta \leq \frac{k+1}{2}$. Moreover, $\alpha \leq \frac{k+1}{2}$ as otherwise we could simply take a cycle containing $\alpha$ many independent vertices. Consider a vertex $v$ with minimum degree $\delta$, with neighbourhood $N(v)$ satisfying $|N(v)| = \delta$. Now, choose $v$ and any $k-1$ vertices from $V  \backslash N[v]$, which is possible as long as $k-1 \leq n - 1 - \delta$. Then, any cycle containing these vertices must also contain some $2$ neighbours of $v$, giving $c(G) \geq k+2$, and we are done.

Thus, we must have $k + \delta > n $. Note that when $2 \leq k \leq \frac{3n}{4} -1$, $n \geq k+2$ if $n \geq 4$. So, we must either have $3\delta - 3 \leq k+1$ or $n + \delta - \alpha \leq k+1$, otherwise we are done by Theorem \ref{bauertheorem}.\\  The former inequality gives $\delta \leq \frac{k+4}{3}$, which gives $$n < k + \delta \leq \frac{4k+4}{3} \implies \frac{3n-4}{4} < k$$ a contradiction. Hence, we must have $\delta \geq \frac{k+5}{3}$, $\alpha \leq \frac{k+1}{2}$ giving $$ k+1 \geq n + \delta - \alpha \geq n + \frac{k+5}{3} - \frac{k+1}{2} = n + \frac{7-k}{6}$$ 

\noindent or equivalently,  $\frac{3n}{4} -1 \geq k \geq \frac{6n+1}{7}$, which is again a contradiction. \qed
\end{proof}

\noindent We now prove an analogous bound for the circumference of $k$-ordered graphs. 


\begin{proof}[Proof of Theorem \ref{orderedtheorem}]
$ $\newline
We know that $k$-ordered graphs are also $k-1$ connected from Theorem \ref{schultztheorem}, thus $\kappa \geq k-1$. We also must have $\alpha \leq k-1$, as otherwise we can simply take $k$ independent vertices in any order to obtain a cycle of size at least $2k$, in which case we are done. Hence, $$\kappa \geq k-1 \geq \alpha$$ so by Theorem \ref{erdostheorem}, we have that $G$ is hamiltonian, and thus we are done in this case as well. \qed
\end{proof}


In fact, it is not hard to see that the $\min \{n, 2k\}$ bound  on the circumference is achieved for all $2 \leq k \leq n$. If $k > n/2$, simply consider the complete graph $K_n$ which is clearly $k$-connected, $k$-ordered, $k$-cyclable and has circumference $n$. If $k \leq n/2$, consider the complete bipartite graph $G = K_{k,n-k} = (A,B,E)$, which is $k$-ordered, and hence $k$-cyclable.  Indeed, take any sequence of $k$ distinct vertices $T = (v_1, v_2, \ldots, v_k)$. We construct a cycle containing $T$ in that order as follows.\\

Let $T_A$ be the set of vertices in $T$ and $A$, with $T_B$ being defined similarly. Then, for any $v \in T_A$, if the next vertex in the sequence $T$ is in $T_B$, then simply follow the edge joining them. Otherwise, first follow an edge to a vertex in $B \backslash T_B$, and then back to the next vertex which must have been in $T_A$. Follow the same procedure for vertices in $T_B$. At the end, follow the edge joining the first and last vertex. We cannot run out of vertices as the number of extra vertices outside $T_A$ in $A$ that are needed is at most $|T_B|$, and $|A|= k = |T_A| + |T_B|$. Similarly, $|B| = n -k \geq k =|T_A| + |T_B|$.\\



We now generalize a result by \cite{Byer2007} on the maximal number of edges in a $k$-connected nonhamiltonian graph, for $k=2,3$. We will need the following short lemma which appears in \cite{Byer2007}.


\begin{lemma}[\cite{Byer2007}] \label{byertheorem}
Let $G$ be a nonhamiltonian, $k$-connected graph of order $n$. Then $k \leq \frac{n-1}{2}$ and $|E(\baro{G})| \geq \binom{k+1}{2} + (k-1)(n-k-1) - \sigma_{k+1}(G)$
\end{lemma}
\begin{proof}
By Theorem \ref{erdostheorem}, $k$-connected nonhamiltonian graphs must contain an independent set $I = \{x_1, \ldots , x_{k+1} \}$ of $k+1$ vertices. The graph is disconnected on removal of the the $n -(k+1) $ vertices of $G - I$, thus we must have
$ n - (k+1) > k-1$, or $k \leq \frac{n-1}{2}$.

Now consider the independent set $I$ satisfying $\sum_{i=1}^{k+1} d(x_i) = \sigma_{k+1}(G)$. Let the edges in $\baro{G}$ incident on at least one vertex of $I$ be denoted $X_I$. Then $X_I$ contains $\binom{k+1}{2}$ edges with both endpoints in $I$ and $\sum_{i=1}^{k+1} (n-1 - k -  d_G(x_i))$ edges with exactly one endpoint in $I$. Thus, we obtain
$$ \, \, \quad \quad  \quad\quad \quad \quad \quad \quad  \quad\quad |E(\baro{G})| \geq |X_I| = \binom{k+1}{2} +   (k-1)(n-k-1) - \sigma_{k+1}(G) \quad \quad \quad  \quad  \quad    \, \, \, \, \, \qed $$  
\end{proof}
Using a slight variation of the above result and Lemma \ref{bondytheorem}, \cite{Byer2007} also show the following result.


\begin{lemma}[\cite{Byer2007}] \label{byertheorem2}
Suppose $G = cl(G)$ for a nonhamiltonian graph $G$ of order $n$, and $m \leq \alpha(G)$. Then


\begin{equation*}
  |E(\baro{G})| \geq
    \begin{cases}
      \frac{m}{2}(n-m) & \text{for n odd}\\
      \frac{m}{2}(n-m) + \frac{m}{2} - 1 & \text{for n even}
    \end{cases}       
\end{equation*}
\end{lemma}


\noindent With the above results, we are ready to proceed to the proof of Theorem \ref{kconhamtheorem}. The idea is that if $n$ is not that much bigger than $\alpha$, then we can get a sufficient lower bound on $|E(\baro{G})| $ using Lemma \ref{byertheorem2}. Otherwise, $n$ is much bigger than $\alpha$, and we can use Theorem \ref{haotheorem} and Lemma \ref{byertheorem}. To show the uniqueness of the extremal graphs, we will make use of the fact that these graphs must satisfy Lemma \ref{bondytheorem} \textit{maximally}, i.e., addition of any further edge causes a violation of the condition.

\begin{proof}[Proof of Theorem \ref{kconhamtheorem}]
$ $\newline \
First of all, assume $k \geq 2$ as we already know that when $|E(G)| > \binom{n-1}{2} + 1$, then $G$ is hamiltonian and consequently connected as well.
Assume $G$ is nonhamiltonian. We may assume $G = cl(G)$, in which case $d(x)+ d(y) \leq n-1$ for any two nonadjacent vertices $x,y$, from Lemma \ref{bondytheorem}. It suffices to prove that $$|E(\baro{G})| \geq \binom{n}{2} - \left( \binom{n-k}{2} + k^2\right) = k\cdot n - \frac{3k^2+k}{2} $$ Note first that if $\sigma_{k+1}(G) \leq n + k^2- k -1$, by Lemma \ref{byertheorem}
$$|E(\baro{G})| \geq \binom{k+1}{2} + (k+1)(n-k-1) - (n+k^2 - k - 1) = k \cdot n -  \frac{3k^2+k}{2} $$
as desired. We now assume $\sigma_{k+1}(G) \geq n + k^2 - k$ and show that in this case, $|E(\baro{G})|$ is \textit{strictly} greater than $k\cdot n - \frac{3k^2+k}{2}$. We will divide the problem into two cases, depending on the size of $n$ compared to $\alpha$.\\

\noindent \B{Case 1:} Assume $n > \frac{ (k^2-1) \cdot \alpha  + y}{k}$, where $y = \frac{-k^3 + 4k^2 + 3k + 2}{2}$.

Let $I = \{x_1,x_2,\ldots ,x_{k+1}\}$ be a set of $k+1$ independent vertices satisfying $\sum_{i=1}^{k+1}d(x_i) = \sigma_{k+1}(G)$, and assume without loss of generality that $$d(x_1) \geq \frac{\sigma_{k+1}(G)}{k+1} \geq \frac{n+k^2-k}{k+1}$$ 

\B{Subcase 1a:} Suppose $d(x_1) \geq n - 2k$. Note that $V(G) - I - N(x_1)$ is non-empty, as otherwise we would have $d(x_1) = n - k -1$, giving $d(x_i) \leq k$ for $2 \leq i \leq k+1$ as $d(x_1) + d(x_i) \leq n-1$ for $2 \leq i \leq k +1$.  This contradicts $\sigma_{k+1}(G) \geq n + k^2 - k$.  Thus, pick some $v \in V(G) - I - N(x_1)$, giving $d_{\baro{G}}(v) = n - 1 - d_G(v) \geq d_G(x_1) \geq n - 2k$. Therefore, $\baro{G}$ contains at least $n - 2k - |I| = n - 3k - 1$ edges with both endpoints not in $I$. Using the same bound we got in Lemma \ref{byertheorem} but also including the extra edges in $\baro{G}$ incident with $v$ (that have no endpoint in $I$) and using Theorem \ref{haotheorem}, we obtain
\begin{align*}
|E(\baro{G})| &\geq \binom{k+1}{2} + (k+1)(n-k-1) + (n - 3k - 1) - \sigma_{k+1}(G)\\ 
& \geq (k+2)\cdot n - \frac{k^2+9k+4}{2} - (n + (k-1) \alpha - k)\\
& > k\cdot n -  \frac{3k^2+ k}{2} +  \frac{3k^2+ k}{2} -  \frac{k^2+9k+4}{2} + k +  \frac{ (k^2-1) \cdot \alpha  + y}{k} - (k-1)\alpha\\
& =  (k\cdot n -  \frac{3k^2+ k}{2}) +  \frac{  (k-1)\cdot \alpha  + y + k(k^2-3k-2)}{k}  >  (k\cdot n -  \frac{3k^2+ k}{2})
\end{align*}
as desired, where the last inequality follows from $y = \frac{-k^3 + 4k^2 + 3k + 2}{2}$.\\

\B{Subcase 1b:} Suppose next that $d(x_1) \leq n - 2k- 1$. Then there exist distinct vertices $v_1, v_2 \ldots, v_{k} \in V(G) - I - N(x_1)$, and $\baro{G}$ contains at least $$(d_{\baro{G}}(v_1) - k - 1) + (d_{\baro{G}}(v_2) - k - 2) + \cdots + (d_{\baro{G}}(v_k) - 2k) = \sum_{i=1}^k d_{\baro{G}}(v_i) - \frac{3k^2 + k}{2}$$ edges with neither endpoint in $I$. Using $d(v_i) + d(x_1) \leq n-1$ as $G = cl(G)$, we get $d_{\baro{G}}(v_i) \geq d_{G}(x_1) \geq \frac{n + k^2 -k}{k+1}$ for all $1 \leq i \leq k$. Consequently, we obtain at least $$\frac{k(n+k^2-k)}{k+1} - \frac{3k^2+k}{2}$$ edges in $\baro{G}$ with neither endpoint in $I$. Using Theorem \ref{haotheorem} and Lemma \ref{byertheorem} again, we get

 
\begin{align*}
|E(\baro{G})| &\geq \binom{k+1}{2} + (k+1)(n-k-1) + \frac{k(n+k^2-k)}{k+1} - \frac{3k^2+k}{2} - (n + (k-1)\alpha - k)\\
& = (kn  - \frac{3k^2+k}{2}) + \frac{k}{k+1}n -(k-1)\alpha +  \binom{k+1}{2} -(k+1)^2 + \frac{k(k^2-k)}{k+1} + k\\
&> (kn  - \frac{3k^2+k}{2})  + \frac{k}{k+1} \frac{(k^2-1)\alpha + y}{k} - (k-1)\alpha + \frac{-k^2-k-2}{2} + \frac{k(k^2-k)}{k+1}  \\
& = (kn  - \frac{3k^2+k}{2}) + \frac{1}{k+1} \left( \frac{-k^3+4k^2+3k+2}{2}+ \frac{(-k^2-k-2)(k+1)}{2}  + k^3 - k^2\right) \\
& = kn - \frac{3k^2+k}{2}
\end{align*} as desired.\\

\noindent  \B{Case 2:} Assume $n \leq \frac{(k^2-1)\alpha + y}{k}$.\\
In this case, $\alpha \geq \frac{nk - y}{k^2-1}$. By Lemma \ref{byertheorem2}, $|E(\baro{G})| \geq \frac{1}{2} \alpha(n- \alpha)$. This is a upward facing parabola for fixed $n$, so for $\frac{nk - y}{k^2-1} \leq \alpha \leq n - \frac{nk - y}{k^2-1}$, this function is minimized at $\alpha = \frac{nk - y}{k^2-1}$. Therefore, in this range

\begin{align*}
|E(\baro{G})| &\geq \frac{\alpha}{2} (n - \alpha) \geq \frac{1}{2} (\frac{nk - y}{k^2-1})(\frac{n(k^2-k-1) +y}{k^2-1})\\
& = \frac{ n^2k(k^2 - k - 1) + n(2k+1-k^2)y - y^2)}{2(k^2-1)^2} 
\end{align*}

\noindent If we want the above to be strictly greater than $kn - \frac{3k^2+k}{2} $, $$\frac{ n^2k(k^2 - k - 1)}{2(k^2-1)^2}  \geq kn \iff n \geq \frac{2(k^2-1)^2}{k^2 - k - 1} = 2(k^2 + k + \frac{1-k}{k^2-k-1})$$ suffices. This is because for $k \geq 5$, $y = \frac{-k^3 + 4k^2 3k+2}{2} < 0$ and $2k+1 - k^2 < 0$, giving $(2k+1-k^2)(y) > 0$. Similarly, $ -y^2 = \frac{(-k^3 + 4k^2 + 3k+2)^2}{4} > -(3k^2+k)(k^2-1)^2$ for $k \geq 5$, so we only have to check the cases of $k = 2,3,4$ manually which is a routine check.\\

Now, it remains to consider the possiblility that $\alpha > n - \frac{nk - y}{k^2-1} = \frac{ n(k^2-k-1) +y}{k^2-1}$. In this case however,  $\alpha$ is quite large compared to $n$, so the $\binom{\alpha}{2}$ edges in $\baro{G}$ between the vertices of an independent set of size $\alpha$ is strictly greater than $ k \cdot n - \frac{3k^2 + k}{2}$ for all $n$. Indeed, we manually verify for $k \leq 3$, and for $k \geq 4$ simply note that $\frac{nk}{2} + y \geq 0$, and hence when $n \geq 2(k^2+k)$ we have $$\alpha > \frac{n(k^2 - \frac{3k}{2} -1) }{k^2 -1} \geq \frac{9n}{15}, \quad \binom{ 9n/15}{2} > \frac{9n}{30} \cdot \frac{8n}{15} > kn$$


We now prove that the extremal  nonhamiltonian $k$-connected graphs  are unique for $n \geq 2(k^2+k)$, by making use of Lemma \ref{bondytheorem}. Recall that we may assume  $G = cl(G)$ is a
nonhamiltonian, $k$-connected graph of order $n \geq 2k^2 + 2k$ with $\sigma_{k+1}(G)=n+k^2 - k - 1$ as equality only holds if all the inequalities in the above proof are tight.\\

Thus, all the edges in $\baro{G}$ have atleast one endpoint in $I$. Let $I =\{x_1, x_2, \ldots, x_{k+1}\}$ be a set of independent vertices such that $k \leq d(x_1) \leq \ldots \leq d(x_{k+1}) $. Note that $k$-connected graphs have minimum degree at least $k$ as otherwise, the graph could be disconnected by removing at most $k-1$ vertices.  As mentioned in the previous section, we may further assume that all edges in $\baro{G}$ have at least one endpoint in $I$, that is, if $x,y \in V(G) - I$, then $\{x,y\} \in E(G)$. We will now use the properties of graph closure repeatedly. First, note that we must have a clique on the remaining $n - k - 1$ vertices, each of which has degree at least $n - k -2$.
\begin{itemize}
    \item Say $d(x_k) \geq k + 1$  Consider the neighbours of $x_k$ in the clique. These neighbours have degree at least $n - k - 1$, and hence since $G = cl(G)$, must be adjacent to $x_{k+1}$ as well as $d(x_{k+1}) \geq k+1$, But then, these neighbours have degree at least $n - k$, and hence must be adjacent to all of $x_1, \ldots, x_{k+1}$ by the same argument. 
    Thus, $I$ and $N(I)$ together form a complete bipartite graph with $|N(I)| \geq k+1 = |I|$. If $d(x_{k+1}) > k+1$, then it is easy to see that the graph is hamiltonian, and otherwise $k+1 = d(x_i) \, \forall i \in [k+1]$, giving $$ \sigma_{k+1} = n+ k^2 - k -1 =  (k+1)^2 \iff n = 3k+2$$ which is false as we assumed $n \geq 2k^2 + 2k$.
    
    \item Otherwise $d(x_k) = k$, 
    , and hence $d(x_{k+1}) = \sigma_{k+1} - k^2 = n - k - 1$, so we have a clique on the $n- k$ vertices in $G \backslash  \{x_1, \ldots, x_{k}\}$.  The neighbours of any $x_i, i \in [k]$ must have degree at least $n-k$, and hence are joined to all the $x_i$. Thus, we obtain the desired extremal graph with exactly $\binom{n-k}{2} + k^2$ many edges, namely a clique on $n-k$ vertices and $k$ other independent vertices forming a complete bipartite graph with some $k$ vertices from the clique. \qed
\end{itemize} 
\end{proof}





\section{Concluding Remarks}

A simpler proof of Theorem \ref{maintheorem} with a weaker constant can be obtained using Tur\'an's theorem and a theorem of Erd\H{o}s and Gallai \cite{Erdos1959} on the length of the longest cycle in a graph. %which states that if $G$ is a graph on $n$ vertices and $k \geq 3$, then $|E(G)| > \frac{1}{2}(k-1)(n-1)$ implies that the circumference is at least $k$.
Consider any $k$-cyclable graph with $\alpha(G) \geq k$. Then, let $S$ be a set of $k$ independent vertices, and consider the cycle containing it. This gives us a cycle of length atleast $2k$, as any two independent vertices are not adjacent to each other. Thus, we must have $\alpha(G) < k$. By a variant of Tur\'an's theorem, we also have $\alpha > \frac{n}{\tilde{d} + 1}$, where $\tilde{d}$ is the average degree. Thus, we obtain

$$\frac{2|E(G)|}{n} + 1 = \tilde{d} + 1 > \frac{n}{\alpha} \geq \frac{n}{k-1} \implies |E(G)| \geq \frac{1}{2} n \left( \frac{n}{k-1} - 1 \right)$$

which is larger than $\frac{1}{2} (2k-1)(n-1)$ if $n \geq 2 k^2$. giving $c(G) \geq 2k$ when $k \leq \sqrt{n/2}$.\\

It is also interesting to understand what happens to the circumference of $k$-cyclable graphs for large values of $k$. As mentioned earlier in the introduction, it is not necessarily the case that $c(G) = n$ when $k = n-1$ due to the existence of hypohamiltonian graphs. Thus, we have the following extremal problem. 

\begin{conjecture}
For a given $n$, let $f(n)$ be the largest value of $k$ such that any $k$-cyclable graph satisfies $c(G)>k$. From the above, we have $f(n) < n-1$ and from Theorem \ref{maintheorem}, we know $f(n) = \Omega(n)$. Is it the case that $f(n) = n-2 ?$
\end{conjecture}

\noindent We can also ask for what regime of $k$ as a function of $n$ do results of the type in Theorem \ref{maintheorem} hold.

\begin{conjecture}
For a given $n$, let $g(n)$ be the largest value of $k$ such that any $k$-cyclable graph satisfies $c(G)\geq 2k$. From Theorem \ref{maintheorem} we know $g(n) = \Omega(\sqrt{n})$. Is it the case that $g(n) = O(\sqrt{n}) ?$
\end{conjecture}

Moreover, our results only give an improvement of the form $c(G) \geq (1+ \gamma) k$, $0< \gamma < 1$, for $k$ up to around $2\sqrt{n}$, and it is natural to ask if such a linear bound on the circumference can be obtained for much larger regimes of $k$. Finally, note that the results of Theorem \ref{kconhamtheorem} only hold for $n \geq 2(k^2 + k)$. For fixed values of $k \leq 3$, \cite{Byer2007} give a tight bound for the minimum value of $n$ for this to hold. They also note that this bound cannot hold for $k = \Omega(n)$,  in particular if $p = \floor{\frac{n-1}{2}}$, the graph obtained by joining $n-p$ independent vertices to each vertex of $K_p$ is $k$-connected and nonhamiltonian, with total number of edges more than $\binom{n-k}{2} + k^2$ when $\frac{n+1}{6} < k  < \floor{\frac{n-1}{2}}$. This still leaves a significant gap in the possible range of $k$ for which $k$-connectivity and $|E(G)| > \binom{n-k}{2} + k^2$ implies hamiltonicity, as our result only applies for $k = O(\sqrt{n})$.

%There are also some interesting results in the literature regarding the cyclability of claw-free graphs. A claw refers to an induced $K_{1,3}$ subgraph. Flandrin et al. showed that 





\
%
% ---- Bibliography ----
%
% BibTeX users should specify bibliography style 'splncs04'.
% References will then be sorted and formatted in the correct style.
%

%\newpage

\bibliographystyle{splncs04}
 \bibliography{caldam}
%
% \input{caldam.bbl}


\end{document}



\end{document}



\end{document}



\end{document}
