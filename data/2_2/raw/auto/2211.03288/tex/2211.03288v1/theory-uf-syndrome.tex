\section{UF Decoder on Syndrome Graph}\label{ap:equiv_uf}

\begin{figure}[t]
    \renewcommand*\thesubfigure{(\arabic{subfigure})}  
    	\centering
	\begin{subfigure}{.48\linewidth}
	    \centering
        \includegraphics[width=0.9\linewidth]{figures/exploratory_region_drawing_manhattan_2/UF_equivalence/decoding_graph.pdf}
        \caption{Decoding Graph}
        \label{fig:uf-equivalence-decoding-graph}
    \end{subfigure}
	\begin{subfigure}{.48\linewidth}
	    \centering
        \includegraphics[width=0.9\linewidth]{figures/exploratory_region_drawing_manhattan_2/UF_equivalence/syndrome_graph.pdf}
        \caption{Syndrome Graph}
        \label{fig:uf-equivalence-syndrome-graph}
    \end{subfigure}
	\caption{Clusters on syndrome graph and decoding graph always touch at simultaneously.}
	\label{fig:uf-equivalence}
\end{figure}

While the original UF decoder works on the decoding graph~\cite{delfosse2021almost}, we show that the UF decoder works on the syndrome graph equivalently in terms of decoding accuracy.

The difference between the decoding graph and the syndrome graph is two-fold: the syndrome graph only have syndrome vertices $V^S$ while the decoding graph have all measurement vertices $V^D \supseteq V^S$; the syndrome graph is a complete graph where there is an edge between any pair of vertices $u, v \in V^S$.
Every edge $e = \langle u, v \rangle$ in the syndrome graph corresponds to the minimum-weight paths between $u$ and $v$ in the decoding graph.
We define the distance $d(u, v)$ between vertices $u, v$ in a graph as the weight of a minimum-weight path between them.
A point $k$ is either a vertex or a point on an edge. Similarly we can define $d(u, k)$ as the weight of a minimum-weight path from vertex $u$ to a point $k$.

In order to show that UF decoders on both graphs have the same decoding accuracy, we only need to show that the final clusters are the same, i.e. covering the same set of syndrome vertices.
The UF decoder logic is the same: it grows a cluster uniformly over all possible directions, and stops when it becomes even or touches a virtual boundary.
Using mathematical induction, if we can show that during the algorithm clusters always touch simultaneously on two graphs, then the final clusters are the same.
\autoref{fig:uf-equivalence} shows an example of clusters touching simultaneously on the decoding graph and the syndrome graph.
