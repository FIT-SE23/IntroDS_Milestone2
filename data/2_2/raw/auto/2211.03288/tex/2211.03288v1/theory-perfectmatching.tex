

\subsection{Perfect matchings}
\label{sec:theory_pm}

Given a syndrome graph $G(V,E)$, a perfect matching is a subgraph in which every vertex from $V$ is incident to one and only one edge from $E$. 
It represents a set of error patterns $\mathbf{E}$ such that $\mathcal{E}\in\mathbf{E}$ produces the syndrome. That is, $S(\E)=V$.
We note that $\mathbf{E}$ does not include all the error patterns for the syndrome. 

\noindent \textbf{Lemma (I)}\label{lemma:pm1} Given a syndrome graph and its perfect matching represented by $\mathbf{E}$ and $\mathcal{E}_1$, $\mathcal{E}_2 \in \mathbf{E}$, $\mathcal{E}_1+\mathcal{E}_2$ is a trivial logical operator.%\newline

\begin{proof2}
\suffices{$S(\E_1+\E_2)=0$ and $P(\E_1+\E_2)=0$.}
\step{1}{$S(\E_1+\E_2)=0$}
    \begin{proof2}
    \step{1.1}{$S(\E_1)=S(\E_2)$=V}
        \begin{proof2}
        \pf\ By definition of perfect matching.
        \end{proof2}
    \step{1.2}{$S(\E_1)+S(\E_2)=V+V=0$}
    \qedstep
    \end{proof2}
\step{2}{$P(\E_1+\E_2)=0$}
    \begin{proof2}
    \step{2.1}{Let $E'\subseteq E$ denote the edges of the perfect matching}
    \step{2.2}{$\exists \E^1_e\in \mathbf{e}$ for $\forall e\in E'$ such that $\E_1=\sum_{e\in E'} \E^1_e$; $\exists \E^2_e\in \mathbf{e}$ for $\forall e\in E'$ such that $\E_2=\sum_{e\in E'} \E^2_e$.}
        \begin{proof2}
        \pf\ By subgraph decomposition.
        \end{proof2}
    \step{2.3}{$P(\E_1+\E_2)=P(\E_1)+P(\E_2)\\ =P(\sum_{e\in E'} \E^1_e)+P(\sum_{e\in E'} \E^2_e)\\ =\sum_{e\in E'}P(\E^1_e)+\sum_{e\in E'}P(\E^2_e)\\ =\sum_{e\in E'}(P(\E^1_e)+P(\E^2_e))\\ =\sum_{e\in E'}(P(\E^1_e+\E^2_e))=0$}
        \begin{proof2}
        \pf\ By \hyperref[lemma:pm0]{Lemma (O)}.
        \end{proof2}
    \end{proof2}
\qedstep
\end{proof2}

\noindent \textbf{Lemma (II)}\label{lemma:pm2} Given a syndrome graph, let $\mathbf{E}_1$ and $\mathbf{E}_2$ denote two perfect matchings.  For $\mathcal{E}_1 \in \mathbf{E}_1$, $\mathcal{E}_2 \in \mathbf{E}_2$, $\mathcal{E}_1+\mathcal{E}_2$ is a logical operator.%\newline

\begin{proof2}
\suffices{$S(\E_1+\E_2)=0$}
\step{1}{$S(\E_1)=S(\E_2)=V$}
    \begin{proof2}
    \pf by definition of perfect maching
    \end{proof2}
\step{2}{$S(\E_1+\E_2)=S(\E_1)+S(\E_2)=V+V=0$}
\qedstep
\end{proof2}


\subsubsection{Cluster decomposition}
\label{sec:theory_cluster}

Given a cluster defined by an even subset of $V$. The above lemmas are also true for perfect matchings inside the cluster.
We will refer to them as Lemma (Cluster) in the following.


\subsection{Equivalent Matchings}
\label{sec:ap_eq}

Assume the syndrome graph has been decomposed into non-overlapping clusters each with an even number of vertices $C_i$, $i=1,2,..., n$. 
Let $G_i$, $i=1,2,...,n$ denote the corresponding subgraphs.  $P_i^1$ and $P_i^2$ denote two perfect matchings for $G_i$. $P_1=\sum_i P_i^1$ and  $P_2=\sum_i P_i^2$ are two perfect matchings for the syndrome graph.

Let $\mathbf{E}_i^1$ and $\mathbf{E}_i^2$ denote sets of error patterns represented by  $P_i^1$ and $P_i^2$, respectively. Let $\mathbf{E}_1$ and $\mathbf{E}_2$ denote sets of error patterns represented by  $P_1$ and $P_2$, respectively.
$\forall \E_1\in \mathbf{E}_1$, $\exists \E_i^1\in\mathbf{E}_i^1$, such that $\E_1=\sum_i \E_i^1$. Similarly, $\forall \E_2\in \mathbf{E}_2$, $\exists \E_i^2\in\mathbf{E}_i^2$, such that $\E_2=\sum_i \E_i^2$.

\noindent \textbf{Lemma (Equivalent Matchings)}\label{lemma:equivalent_matchings} if a cluster is detached, its perfect matchings are logically equivalent.
\newline

\begin{proof2}
\suffices{$\forall \mathcal{E}_1\in \mathbf{E}_1$ and $\forall \mathcal{E}_2\in \mathbf{E}_2$, $\mathcal{E}_1 + \mathcal{E}_2$ is a trivial logical operator}
    \begin{proof2}
        By definition of \emph{Logical Equivalence} for subgraphs
    \end{proof2}
\step{1}{$S(\E_1+\E_2)=0$}
    \begin{proof2}
    \step{1.1}{$S(\E_1+\E_2)=S(\E_1)+S(\E_2)=0$}
        \begin{proof2}
        \pf\ $S(\E_1)=S(\E_2)$ because both perfect matchings produce the syndrome inside the same cluster.
        \end{proof2}
    \qedstep
    \end{proof2}  
\step{2}{$P(\E_1+\E_2)=0$}
    \begin{proof2}
    \step{2.1}{ Either $P_L(\E_1)=P_L(\E_2)=0$ or $P_R(\E_1)=P_R(\E_2)=0$}
        \begin{proof2}
        \pf\ by definition of \emph{Detached cluster}.
        \end{proof2}
    \step{2.2}{$P_R(\E_1+\E_2)=P_L(\E_1+\E_2)$}
        \begin{proof2}
        \pf\ by \autoref{equ:parity-syndrome} and Step \stepref{1}.
        \end{proof2}
    \qedstep
    \end{proof2}
\qedstep
\end{proof2}


\noindent \textbf{Theorem (Equivalent Matchings)} if $P_1$ and $P_2$ are different only inside detached clusters, they are logically equivalent. \newline

\begin{proof2}
\suffices{ $\forall \E_1\in \mathbf{E}_1$ and $\forall \E_2\in \mathbf{E}_2$, $S(\E_1+\E_2)=0$ and $P(\E_1+\E_2)=0$.}
    \begin{proof2}
    \pf\ By definition of trivial logical operator and definition of logical equivalence.
    \end{proof2} 
\step{1}{$S(\E_1+\E_2)=0$}
    \begin{proof2}
    \step{1.1}{$S(\E_1+\E_2)=S(\sum_i(\E_i^1+\E_i^2))\\ =\sum_i S(\E_i^1+\E_i^2)=0$}
    \begin{proof2}
        \pf\ {$S(\E_i^1+\E_i^2)=0$} by \hyperref[lemma:pm2]{Lemma (II)}.
    \end{proof2}
    \qedstep
    \end{proof2}  
\step{2}{$P(\E_1+\E_2)=0$}
    \begin{proof2}
    \step{2.1}{$P(\E_1+\E_2)=P(\sum_i(\E_i^1+\E_i^2))\\ =\sum_i P(\E_i^1+\E_i^2)=0$}
        \begin{proof2}
            \step{2.1.1}{If cluster $i$ is detached, $P(\E_i^1+\E_i^2)=0$ }
            \step{2.1.2}{Otherwise, cluster is attached. $P_i^1=P_i^2$ by assumption and then $P(\E_i^1+\E_i^2)=0$}
                \begin{proof2}
                \pf\ by \hyperref[lemma:pm2]{Lemma (II)} (Cluster).
                \end{proof2}
            \qedstep
        \end{proof2}
    \qedstep  
    \end{proof2}
\qedstep
\end{proof2}
