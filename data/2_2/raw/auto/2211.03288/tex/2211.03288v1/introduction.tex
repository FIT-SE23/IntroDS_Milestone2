\section{Introduction}
\label{sec:intro}

%Why we care about UF and MWPM decoders
Fast and accurate quantum error correction is necessary for fault-tolerant quantum computing. Surface codes have emerged as one of the leading choices for quantum error correction. 
A surface code interleaves data and ancilla qubits on a surface such that an error in a data quit will impact the measurement outcome of its neighboring ancillas. The decoder's job is to determine the error pattern, i.e., which data qubits experience errors, based on the syndrome, i.e., measurement outcomes of all ancillas. 

%Brief introduction of these two decoders.
Union-find (UF)~\cite{delfosse2020linear} and Minimum-Weight Perfect Matching (MWPM) are two popular decoder designs. Both leverage graph representations of surface code and its syndromes. 
A UF decoder works on the model graph where vertices correspond to measurement qubits and edges correspond to data qubits with weights determined by the error probability.
An MWPM decoder works on the syndrome graph, a fully-connected graph generated from the model graph in which vertices correspond to nontrivial measurement outcomes and an edge corresponds to minimum-weight paths in the model graph. As its name suggests, an MWPM decoder finds a minimum-weight perfect matching of the syndrome graph and uses it as its guess for the error pattern. Union-Find is known to be much faster than MWPM decoders but in general, achieves lower decoding accuracy.

%What this paper is about: interpretation: UF approximates Blossom
The primary contribution of this work is to reveal a hidden link between UF and MWPM decoders.
By adapting UF decoders to work on the syndrome graph, we show that its working principles approximate those of the blossom algorithm, a highly optimized algorithm that finds an MWPM.
Both UF and Blossom algorithm decompose a syndrome graph into non-overlapping subgraphs. 
Both start with the syndrome graph with each subgraph including a single vertex and grow these subgraphs such that each can be ``solved'' on its own.
While the blossom algorithm finds an MWPM for each subgraph, a UF decoder finds a logical equivalent to a PM for it.
We show that under many circumstances, a UF decoder and the blossom algorithm may end up decomposing the syndrome graph in the same way and their solutions for subgraphs may be logically equivalent.

%Applications of this interpretation
Once revealed, the link between UF and MWPM decoders not only allows us to improve the accuracy of UF decoders and generalize them for weighted decoding graphs but also explain why UF decoders achieve similar accuracy as MWPM decoders for certain surface codes, such as the XZZX surface code~\cite{bonilla2021xzzx} when noise is infinitely biased and measurements are perfect.
Specifically, we devise a more general UF decoder that works for surface codes without assuming identical error probability for data qubits.

We presented the interpretation and its implications at~\cite{yue2022aps}. Since then we have learned that others had reached similar interpretations independently~\cite{google-interpretation}.  At the same time, we have also learned that many are still unaware of the link between the UF and blossom algorithms. Therefore, our motivations in writing this article are two. First, to share the interpretation with the community in hope to foster the development of fast quantum error decoders. Second, to formalize the interpretation as rigorous as we could.  

In the rest of the paper, we provide necessary background in \S\ref{sec:background}; we describe how a UF decoder approximates the blossom algorithm in \S\ref{sec:interpretation}.
Using this interpretation, we explain why a UF decoder behaves the same as an MWPM decoder for two examples in \S\ref{sec:example}. 
We derive weighted union-find decoders based on the interpretation in \S\ref{sec:weighted_uf}. We have open-sourced their implementations at~\cite{qec-playground}.
Mathematical details of proofs can be found in the Appendix.
