\documentclass[10pt,twocolumn]{article}

\usepackage{titlesec}

\titleformat*{\section}{\large\bfseries}
\titleformat*{\subsection}{\bfseries}

\usepackage[english]{babel}
\usepackage[utf8]{inputenc}
\usepackage{CJKutf8}

\usepackage[normalem]{ulem}

\usepackage{amsfonts,amsmath,amssymb,graphicx}
\usepackage{cuted}  % Yue added 10/26/2021 for double column equations
\usepackage[usenames,dvipsnames]{xcolor}
\usepackage{setspace}
\usepackage{hyperref}
\usepackage{amsmath}
% \usepackage{multicol}

\usepackage{pdfpages}


\newcommand{\be}{\begin{eqnarray}}
\newcommand{\ee}{\end{eqnarray}}
\newcommand{\ra}{\rightarrow}
\newcommand{\var}{\varepsilon}
\newcommand{\noi}{\noindent}
\newcommand{\dg}{\dagger}
\newcommand{\ola}{\overleftarrow}
\newcommand{\ora}{\overrightarrow}
\newcommand{\rang}{\rangle}
\newcommand{\lang}{\langle}
\newcommand{\ketbra}[2]{|{#1}\rangle \langle{#2}|}
\newcommand{\bra}[1]{\langle{#1}|}
\newcommand{\ket}[1]{|{#1}\rangle}

\newcommand{\op}[1]{\hat{#1}}
\newcommand{\sx}{\hat{\sigma}_\mathrm{x}}
\newcommand{\sz}{\hat{\sigma}_\mathrm{z}}
\newcommand{\sxo}{\hat{\sigma}_{\mathrm{x}1}}
\newcommand{\szo}{\hat{\sigma}_{\mathrm{z}1}}
\newcommand{\sxt}{\hat{\sigma}_{\mathrm{x}2}}
\newcommand{\szt}{\hat{\sigma}_{\mathrm{z}2}}
\newcommand{\Aa}{\hat{a}^\dag\hat{a}}
\newcommand{\ha}{\hat{a}}
\newcommand{\hA}{\hat{a}^\dag}
\newcommand{\wc}{\omega_\mathrm{r}}
\newcommand{\wa}{\omega_\mathrm{a}}
\newcommand{\wdr}{\omega_\mathrm{d}}
\newcommand{\dr}{\delta_\mathrm{r}}
\newcommand{\hc}{\mathrm{h.c.}}
\newcommand{\catpm}{\ket{\mathcal{C}^\pm_\alpha}}
\newcommand{\catp}{\ket{\mathcal{C}^+_\alpha}}
\newcommand{\catm}{\ket{\mathcal{C}^-_\alpha}}
\newcommand{\Ep}{\mathcal{E}_\textnormal{p}}
\newcommand{\E}{\mathcal{E}}

\newcommand{\catpmi}{\ket{\mathcal{C}^\pm_{\alpha_0}}}
\newcommand{\catpi}{\ket{\mathcal{C}^+_{\alpha_0}}}
\newcommand{\catmi}{\ket{\mathcal{C}^-_{\alpha_0}}}
\newcommand{\hta}{\hat{a}}
\newcommand{\g}{\ket{-\tilde{1}}}
\newcommand{\e}{\ket{\tilde{1}}}
\newcommand{\gepm}{\ket{\pm\tilde{1}}}
\newcommand{\go}{\ket{-\tilde{1}_0}}
\newcommand{\eo}{\ket{\tilde{1}_0}}
\newcommand{\gepmo}{\ket{\pm\tilde{1}_0}}
\newcommand{\zero}{\bar{0}}
\newcommand{\one}{\bar{1}}
\newcommand{\red}{\color[rgb]{0.8, 0, 0}}
\newcommand{\orange}[1]{{\color[rgb]{0.9, 0.5, 0.0}#1}}
\newcommand{\blue}{\color[rgb]{0.0, 0.0, 0.6}}
\newcommand{\Ez}{\mathcal{E}_\mathrm{z}}
\newcommand{\ab}[1]{{\color[rgb]{0.8, 0, 0} AB: #1}}
\newcommand{\cka}[1]{\textit{\color[rgb]{0.9, 0.5, 0} [CKA: #1] }}

\newcommand{\alg}[1]{{\color[rgb]{0.92,0.124,0.52}#1}}
\newcommand{\asout}[1]{\alg{\sout{#1}}}



\newcommand{\AUTHORS}{Authors}
\newcommand{\TITLE}{Title}
\newcommand{\KEYWORDS}{Keywords}
\newcommand{\CONFERENCE}{Somewhere}
\newcommand{\COLOR}{yes}
\newcommand{\PAGENUMBERS}{yes} 
\newcommand{\COMMENTS}{yes}

\usepackage{color,balance,xspace,verbatim,ifthen,engord}

\usepackage{amsmath,wasysym,amsthm,marvosym,stackengine}
% \usepackage{gensymb}

\usepackage{booktabs,colortbl,diagbox,multirow,tabularx,tablefootnote} 
\newcommand{\tabincell}[2]{\begin{tabular}{@{}#1@{}}#2\end{tabular}}
% \usepackage{dblfloatfix}

\usepackage{caption}
\usepackage{subcaption}
\usepackage{graphicx,epsfig,epstopdf,wrapfig}
\graphicspath{{./figcam/}} 
\captionsetup[subfigure]{labelformat=simple}  % remove parentheses, need to add manually to all figures

\DeclareGraphicsExtensions{.pdf,.mps,.png,.jpg,.eps,.PNG,.JPG}
\epstopdfsetup{outdir=./figures/}

\usepackage{algorithm,algorithmic}
\renewcommand{\algorithmicrequire}{\textbf{Input:}}
\renewcommand{\algorithmicensure}{\textbf{Output:}}

\hyphenation{op-tical net-works semi-conduc-tor}

\usepackage{url}
\def\UrlBreaks{\do\A\do\B\do\C\do\D\do\E\do\F\do\G\do\H\do\I\do\J\do\K\do\L\do\M\do\N\do\O\do\P\do\Q\do\R\do\S\do\T\do\U\do\V\do\W\do\X\do\Y\do\Z\do\[\do\\\do\]\do\^\do\_\do\`\do\a\do\b\do\c\do\d\do\e\do\f\do\g\do\h\do\i\do\j\do\k\do\l\do\m\do\n\do\o\do\p\do\q\do\r\do\s\do\t\do\u\do\v\do\w\do\x\do\y\do\z\do\0\do\1\do\2\do\3\do\4\do\5\do\6\do\7\do\8\do\9\do\.\do\@\do\\\do\/\do\!\do\_\do\|\do\;\do\>\do\]\do\)\do\,\do\?\do\'\do+\do\=\do\#}%

\newcommand{\secref}[1]{\S\ref{#1}}
\newcommand{\figref}[1]{Fig.~\ref{#1}}
\newcommand{\tabref}[1]{Tab.~\ref{#1}}
\newcommand{\eqnref}[1]{Eqn.~\ref{#1}}
\newcommand{\algref}[1]{Alg.~\ref{#1}}

\newcommand{\figcaption}[1]{\vspace{-8mm}\caption{#1}\vspace{-4mm}} 
\newcommand{\mfigcaption}[1]{\vspace{-4mm}\caption{#1}\vspace{-2mm}} 
\newcommand{\tabcaption}[1]{\vspace{1mm}\caption{#1}\vspace{-8mm}}
\newcommand{\mtabcaption}[1]{\vspace{-3mm}\caption{#1}\vspace{-8mm}}

\usepackage{tikz}
\newcommand{\WoB}[1]{{\small \tikz[baseline=(char.base)]{\node[shape=circle,fill=black,draw,inner sep=0.5pt] (char) {\color{white}#1};}}}

\newcommand{\sWoB}[1]{$\rlap{\Large{\CIRCLE}}{{\color{white}{\footnotesize \raisebox{0.7 pt}{\hspace{3.3pt}#1}}}}$~} % used in caption
\newcommand{\dWoB}[1]{$\rlap{\Large{\CIRCLE}}{{\color{white}{\scriptsize \raisebox{1 pt}{~#1}}}}$~}
\newcommand*\circled[1]{\tikz[baseline=(char.base)]{\node[shape=circle,draw,inner sep=0.5pt] (char) {#1};}}

\newcommand{\superscript}[1]{\ensuremath{^{\textrm{#1}}}}
\newcommand{\argmax}{\operatornamewithlimits{argmax}}
\def\deg{{\,^{\circ}}\xspace}

\newcommand{\nosection}[1]{\vspace{3pt}\noindent\textbf{#1}}
\newcommand{\nosubsection}[1]{\vspace{3pt}\noindent$\bullet$\hspace{1mm}\textit{#1}}
\newcommand{\heading}[1]{\vspace{3pt}\noindent\textup{\textbf{#1}}}
\newcommand{\blpara}{\vspace{3pt}\noindent$\bullet$\hspace{1mm}}
\newcommand{\blnosection}[1]{\vspace{3pt}\noindent$\bullet$\hspace{1mm}{#1}}
\newcommand{\blheading}[1]{\vspace{3pt}\noindent$\bullet$\hspace{1mm}\textup{\textbf{#1}}}
\def\newpara{\vspace{3pt}\noindent}

\def\It{\textit}
\def\Bf{\textbf}
\def\eg{\textit{e.g.,}\hspace{1mm}}
\def\ie{\textit{i.e.,}\hspace{1mm}}
\def\etal{\textit{et al.}\hspace{1mm}}
\def\etc{\textit{etc.}\hspace{1mm}}

\usepackage{courier}
% \newcommand{\code}[1]{\mbox{\texttt{#1}}}
\newcommand{\Mod}[1]{\mbox{\textsf{#1}}}
\newcommand{\sw}[1]{\mbox{\textsc{#1}}}

\usepackage{enumitem}

\newenvironment{Itemize}{
	\begin{list}{$\bullet$} {
		\setlength{\itemsep}{0pt}
		\setlength{\parsep}{2pt}
		\setlength{\topsep}{2pt}
		\setlength{\partopsep}{0pt}
		\setlength{\leftmargin}{1.5em}
		\setlength{\labelwidth}{1em}
		\setlength{\labelsep}{0.5em}
	}}
	{\end{list}}	

\newenvironment{Enumerate}{
	\begin{enumerate}[leftmargin=2em]
		\setlength{\itemsep}{2pt}
		\setlength{\topsep}{2pt}
		\setlength{\partopsep}{0pt}
		\setlength{\parskip}{0pt}}
	{\end{enumerate}}

\newenvironment{Circled}{
	\begin{enumerate}[label=\protect\circled{\arabic*},leftmargin=2em]
		\setlength{\itemsep}{3pt}
		\setlength{\topsep}{0pt}
		\setlength{\partopsep}{0pt}
		\setlength{\parskip}{0pt}}
	{\end{enumerate}}


\usepackage{xcolor,soul}
% \usepackage[colorinlistoftodos]{todonotes}
\usepackage{float}
\usepackage{enumitem}

\usepackage{pf2}

\newcommand{\program}[1]{\textsf{\small #1}}
\newcommand{\code}[1]{\texttt{#1}}



\newcommand\couldremove[1]{{\st{#1}}}
\newcommand{\remove}[1]{}
\newcommand{\move}[2]{ {\textcolor{Purple}{ \bf --- MOVE #1: --- }} {\textcolor{Orchid}{#2}} }


\definecolor{lightgray}{gray}{0.6}
\definecolor{lightblue}{rgb}{0.9,0.9,1}
\definecolor{aqua}{rgb}{0.0, 1.0, 1.0}

\newcommand{\hlc}[2][yellow]{{\sethlcolor{#1}\hl{#2}}}
\newcommand\note[1]{\hlc[yellow]{#1}}
\newcommand\lin[1]{\hlc[yellow]{LZ: #1}} 
\newcommand\lina[1]{\hlc[gray]{LZ: #1}}
\newcommand\shruti[1]{{\hlc[aqua]{SP: #1}}} 
\newcommand\shrutia[1]{\hlc[gray]{SP: #1}}
\newcommand\yue[1]{{\hlc[orange]{YW: #1}}} 
\newcommand\yuea[1]{\hlc[gray]{YW: #1}}
\newcommand\yuem[1]{{\color{blue}{#1}}} % yue modified text but needs review
\newcommand\nami[1]{{\hlc[pink]{NL: #1}}}
\newcommand\namia[1]{{\hlc[gray]{NL: #1}}}

\newcommand{\region}{region\xspace}
\newcommand{\regions}{regions\xspace}

\newcommand{\cluster}{dual cluster\xspace}
\newcommand{\clusters}{dual clusters\xspace}

\newcommand{\csize}{size\xspace}


\newcommand{\diagram}{Fowler diagram\xspace}


%\newcommand\todo[1]{\textcolor{red}{TODO: #1}}

% \renewcommand\couldremove[1]{}  % remove "couldremove" sections.
% \renewcommand\note[1]{}         % remove "note"  highlights.
% \renewcommand\lin[1]{}          % remove "lin"   highlights.
% \renewcommand\lina[1]{}         % remove "lina"  highlights.
% \renewcommand\nami[1]{}          % remove "lin"   highlights.
% \renewcommand\namia[1]{}         % remove "lina"  highlights.
% \renewcommand\yue[1]{}          % remove "lin"   highlights.
% \renewcommand\yuea[1]{}         % remove "lina"  highlights.
% \renewcommand\shruti[1]{}          % remove "lin"   highlights.
% \renewcommand\shrutia[1]{}         % remove "lina"  highlights.
% \renewcommand\todo[1]{}         % remove todo statements.

\newcommand{\papertitle}{An interpretation of Union-Find Decoder on Weighted Graphs}
% Union-Find $\approx$ Blossom}
% \newcommand{\personnel}{Yue Wu$^\dagger$, Namitha Liyanage$^\dagger$, Shruti Puri$^\ddagger$, and Lin Zhong$^\dagger$\\ $^\dagger$Department of Computer Science and $^\ddagger$Department of Applied Physics\\Yale University, New Haven, CT}
\newcommand{\personnel}{Yue Wu, Namitha Liyanage, and Lin Zhong\\ Department of Computer Science, Yale University, New Haven, CT}



\singlespacing

\begin{document}

\title{\papertitle}

\author{\personnel}

\maketitle


\begin{abstract}

Union-Find (UF) and Minimum-Weight Perfect Matching (MWPM) are popular decoder designs for surface codes.
The former has significantly lower time complexity than the latter but is considered somewhat inferior, in terms of decoding accuracy.
In this work we present an interpretation of UF decoders that explains why UF and MWPM decoders perform closely in some cases: the UF decoder is an approximate implementation of the blossom algorithm used for MWPM.
This interpretation allows a generalization of UF decoders for weighted decoding graphs and explains why UF decoders achieve high accuracy for certain surface codes.

\end{abstract}

\section{Introduction}

Generative modeling has been the dominant approach for large-scale pretraining and zero-shot generalization~\cite{gpt3-paper,artetxe2021efficient,rae2021scaling}. 
Combined with prompts~\cite{gpt3-paper}, most of the natural language processing (NLP) tasks can be formulated into the fill-in-the-blank format and perform generative language modeling.
Based on the unified generative formulation, pretrained models such as GPT-3~\cite{gpt3-paper}, BERT~\cite{devlin2018bert,PET-paper}, T5~\cite{T5-paper}, can perform zero-shot inference on new tasks. 


More recent work~\cite{T0-paper} proposed to further pretrain a generative T5~\cite{T5-paper} with multitask prompted datasets and has substantially enhanced the performance of zero-shot generalization. 
In contrast, methods based on discriminative modeling~\cite{devlin2018bert} have not been able to achieve state-of-the-art performance on zero-shot learning. The adoption of discriminative approaches for zero-shot learning has been limited in the literature.


% Although there are a few works using discriminative modeling to perform zero-shot or few-shot learning, such as CLS finetuning using BERT or prompting using ELECTRA
% For example, BERT was CLS finetuned to perform zero-shot/few-shot learning, however, the zero-shot/few-shot performance are lagged far behind.

% \zy{Add a note: although BERT can be CLS finetuned (which is discriminative), but it is not the SOTA approach for zero-shot and few-shot learning.}

\begin{figure}%[htbp]
     \centering
     \includegraphics[width=1.05\linewidth]{figure/final_sota.png}
     \vspace{-15pt}
     \caption{Average zero-shot performance over 11 zero-shot tasks for our Universal Discriminator and T0~\cite{T0-paper}. Our universal discriminator significantly outperforms T0 across three different scales.}
     \label{fig:sota}
     \vspace{-15pt}
 \end{figure} 


In this work, we challenge the convention of zero-shot learning and propose to study and improve discriminative approaches. This is motivated by the fact that many NLP tasks can be framed as selecting from a few options; e.g., telling whether sentence A entails sentence B, or predicting which answer is correct for a given question. We call these tasks \textit{discriminative tasks}. As we will discuss in later sections, a significant portion of NLP tasks is in fact discriminative tasks. We hypothesize that discriminative approaches perform better for discriminative tasks.
% Despite the recent progress, it remains unknown how discriminative approaches perform in zero-shot generalization. Motivated by the fact that discriminative modeling learns to distinguish among options and goes better with discriminative tasks (e.g., telling whether sentence A entails sentence B, or telling which option correctly answer the question), we hypothesize that discriminative modeling would be better at zero-shot generalization, especially on discriminative tasks.

To verify the hypothesis, we propose the \textbf{universal discriminator (UD)}, which substantially improves zero-shot generalization over the previous generative state-of-the-art (SOTA)~\cite{T0-paper}, as Figure~\ref{fig:sota} shows.
The main idea is to train a single discriminator to predict whether a text sample comes from the true data distribution of natural language, similar to GANs \cite{goodfellow2014generative}. Given a set of training tasks with labeled data, we construct a dataset with positive and negative examples, where positive ones are in-distribution natural language samples and negative ones are out-of-distribution. There are two major types of discriminative tasks. The first type is tasks with multiple options, such as multi-choice question answering and news classification. We fill the options into the sentences and the ones with correct options are considered positive samples. The second type is tasks with yes/no options, which can be formulated as a binary discrimination problem itself. For example, natural language inference aims to predict whether a premise entails a hypothesis. In this case, we use a prompt to concatenate the premise $A$ and the hypothesis $B$ into a sentence ``Premise: $A$. Hypothesis: $B$.'' If entailment holds, this sample is treated as positive in-distribution samples and otherwise negative out-of-distribution ones.



% We define the true data distribution using multiple training tasks with labeled data. Specifically, since discriminative tasks can be formulated as selecting from a few options, samples with correct options form an empirical data distribution, while samples with incorrect options are considered out of distribution. In other words, our discriminator is trained to predict ``true'' for samples with correct options and ``false'' for incorrect ones. We use simple concatenation to minimize prompting efforts. For example, given an example (premise, hypothesis), a natural language inference task predicts whether the premise entails the hypothesis. We concatenate the premise and hypothesis, and assign the label ``true'' for entailment and ``false'' for non-entailment.


% First off, since many of the NLP tasks can be formulated as selecting from several options, we first reformulate the task data into natural text samples by concatenating different fields \zy{what are fields? undefined here. try using another word.}.
% For example, given an example of \zy{the} natural language inference task (\textit{Premise}, \textit{Hypothesis}, \textit{Label}), the natural text is reformulated as ``\textit{\{Premise\} || \{Hypothesis\}}'' labeled with \textit{\{Label\}}. \footnote{Here we use ``||'' to represents direct concatenation.} 
% Another example of topic classification task (\textit{Text}, \textit{Label}) where the \textit{Label} indicates the first option of \{Sports, Fashion, Politics\}, the corresponding natural texts are formulated as ``\textit{Text} || Sports'' labeled with 1, ``\textit{Text} || Fashion'' and ``\textit{Text} || Politics'' both labeled with 0.
% Secondly, we pretrain a pretrained model with reformulated multitask datasets to distinguish whether the text sample comes from the true data distribution. ~\footnote{An assumption is that negative-labeled text samples are artificially constructed thus do not come from the true data distribution, and vice versa.}

For the performance of zero-shot generalization, our approach achieves new state-of-the-art on the T0 benchmark, outperforming T0 by 16.0\%, 7.8\%, and 11.5\% respectively on different scales. 
UD also achieves state-of-the-art performance on a wide range of supervised NLP tasks, using only 1/4 parameters of previous methods.
Compared with the previous generative prompt-based methods, our universal discriminator requires minimal prompting, which is simple, robust, and applicable in real-world scenarios.

% By further scaling the number of tasks, our approach also sets the new state-of-the-art on \textbf{\color{red}[xxx]} tasks with less than 10\% of model parameters \zy{need to give a range} under the setting of standard finetuning.
% In the setting of finetuning, our approach also outperforms the generative baselines consistently across a wide range of tasks.


In addition, we also generalize UD to a larger scope of tasks, such that UD can perform discriminative and generative tasks at the same time. Specifically, we extend UD to the encoder-decoder architecture for training on generative tasks, and restrict the model's prediction on "yes"/"no" tokens for jointly training discriminative tasks. Results prove that generalized UD maintains UD's advantages on discriminative tasks and achieves comparable results on generative tasks (See \S~\ref{sec:generalizedud}). 
% We leave expanding UD to a broader range of generative tasks and achieve greater performance on generative tasks as our future work


% \xhk{I admit the limitation on generative tasks here as our future work.}

%\xhk{Although UD is designed for improving zero-shot performance for discriminative tasks, we can also combine this idea to train a generalized UD model which simultaneously solves both discriminative tasks and generative tasks, maintaining UD's advantage on discriminative tasks and get comparable results on generative tasks (See \S~\ref{sec:generalizedud}).}

% The universal discriminator provides a new perspective for zero-shot generalization---Compared with generating the true verbalizer that indicates task label with extensive prompt engineering, distinguishing between options with minimal prompting efforts is simple, robust, and high-performing, thus is more applicable in real-world scenarios. \zy{rewirte the above sentence, just focus on one point---minimal prompting}

\section{Background and State-Of-The-Art} \label{background-motivation}
\subsection{Zero-click Exploits}

Zero-click exploits tend to leverage common instant messaging applications such as WhatsApp and iMessage that, by design, receive messages and calls from anyone who knows the user's phone number, including untrusted sources.  
The attacker gears a zero-click attack by sending a specially-crafted hidden text message, image, or voicemail to the target device via a wireless connection (Wi-Fi, cellular network, Bluetooth, or NFC).  
The injected malicious code then provokes a previously unknown security vulnerability in an installed application to gain root access to the target device.    

Zero-click exploits are not new.  Apple first discovered a zero-click vulnerability on the iPhone 5 with iOS 6 in 2012~\cite{apple2012}.  
With the increase in attack sophistication and complex computing technologies, zero-click attacks have become a pervasive threat to smartphone users. 
Samsung recently patched a zero-click vulnerability in Skia (Android's graphics library) that existed in all Samsung smartphones sold since 2014~\cite{androidmms}.
Zero-click attacks have gained significant attention since the recent discovery of Pegasus, the zero-click spyware used to surveil high-profile individuals and organizations around the world.

\subsubsection*{\textbf{Pegasus Spyware}} 
Pegasus is a highly sophisticated cyber-espionage spyware created by NSO Group Technologies Ltd, an Israeli cybersecurity company, with the claimed aim of tracking terrorist activities via untraceable commands~\cite{pegasus}.
However, several reports claim that nation-state actors have misused Pegasus spyware to track the activities of their opponents and critics. 

In 2016, Apple discovered three zero-click exploits in iOS that were used to infect and spy on targets for years~\cite{forensics}.
In 2017, Pegasus spyware abused a zero-day vulnerability in WhatsApp with a mere unanswered message or call that enabled the spyware to run as a background resource and compromise the target phone. 
The Pegasus variants identified in 2020 and 2021 used two zero-click iMessage exploits; Kismet and ForcedEntry.  
Although Kismet was successful only against iOS release 13.5.1, ForcedEntry was exploited in the wild and even bypassed the iOS BlastDoor utility, which was basically designed to prevent such spyware attacks.  
Surprisingly, the 2021's Pegasus variant utilized the same vulnerability chaining technique, \textit{Trident}, as the 2016's Pegasus variant, which was patched by Apple right after the attack, albeit ineffectively~\cite{lookout}.

A key trait of Pegasus (and, in general, zero-click spyware) is that it does not show erratic behaviour or leave indicators of compromise (IoC) such as slow performance or excessive battery drainage.  
The spyware primarily exists in the phone's temporary memory (RAM) and is intermittently removed.  
Even upon installation on the phone, the spyware does not leave any trace.  
This way, it effectively evades detection by anti-malware or endpoint security systems, thereby failing to inform the user about the potential privacy infringement and taking security
threats to the next level. 

\subsubsection*{\textbf{Other zero-click spyware:}} 
Zero-click exploits now have a thriving market.  
Reign, zero-click spyware developed by QuaDream, can also compromise iOS devices using an iMessage-based zero-day exploit~\cite{reign}.  
In addition, sources claim that the zero-click exploits developed by Paragon~\cite{paragon} and Cognyte~\cite{threemore} can reportedly target end-to-end encrypted messaging applications, particularly WhatsApp and Signal to compromise the smartphone.  


% ODQA gives QA model a single question without any context and asks the model to infer out-of-context knowledge. 

% Following the pioneering work by~\citet{DBLP:conf/acl/ChenFWB17}, most ODQA systems assume the model can access an external text corpus (e.g. Wikipedia).
% Due to the large scale of web corpus (20GB for Wikipedia), it could not be simply encoded in the QA model parameters, and thus most works propose a \textit{Retrieval-Reader} pipeline, by firstly index the whole corpus and use a \textit{retriever} model to identify which passage is relevant to the question; then the retrieved text passage concatenate with question is re-encoded by a seperate \textit{reader} model (e.g., \texttt{LM}) to predict answer. As the knowledge is outside of model parameter, \citet{DBLP:conf/emnlp/RobertsRS20} defines these methods as \textit{Open-book}, with an analogy to referring textbooks during exam.


% The \textit{Open-book} models following \textit{Retrieval-Reader} pipeline requires storing indexed corpus, and is hard to train end-to-end, and is inefficient during inference. Therefore, 
% In contrast, 

% \textit{Closed-book} QA models (mostly a single \texttt{LM}) try to answer open questions without accessing external knowledge. This setting is much harder as it requires \texttt{LM} to memorize all pertinent knowledge in its parameters.
% , and even recent \texttt{LM}s with much larger model parameters is still not competitive to state-of-the-art \textit{Open-book} models (e.g., T5-11B~\cite{DBLP:conf/emnlp/RobertsRS20} achieves 34.5 accuracy on Natural Questions, while FiD~\citep{DBLP:conf/eacl/IzacardG21} with 1B parameter achieves 51.4).
% Recent studies~\cite{DBLP:conf/emnlp/RobertsRS20, DBLP:conf/nips/BrownMRSKDNSSAA20} show that by leveraging pre-trained \texttt{LM}s in a supervised setting, they could correctly answer a certain portion of open questions (e.g., T5-11B could answer over 30\% of questions in Natural Questions dataset). 







\noindent \textbf{Open-Domain Question Answering (ODQA)} gives QA model a single question without any context and asks the model to infer out-of-context knowledge. Following the pioneering work by~\citet{DBLP:conf/acl/ChenFWB17}, most ODQA systems assume the model can access an external text corpus (e.g. Wikipedia).
Due to the large scale of web corpus (20GB for Wikipedia), it could not be simply encoded in the QA model parameters, and thus most works propose a \textit{Retrieval-Reader} pipeline, by firstly index the whole corpus and use a \textit{retriever} model to identify which passage is relevant to the question; then the retrieved text passage concatenate with question is re-encoded by a seperate \textit{reader} model (e.g., \texttt{LM}) to predict answer. As the knowledge is outside of model parameter, \citet{DBLP:conf/emnlp/RobertsRS20} defines these methods as \textit{Open-book}, with an analogy to referring textbooks during exam. \textit{Closed-book} QA models (mostly a single \texttt{LM}) try to answer open questions without accessing external knowledge. This setting is much harder as it requires \texttt{LM} to memorize all pertinent knowledge in its parameters, and even recent \texttt{LM}s with much larger model parameters is still not competitive to state-of-the-art \textit{Open-book} models. 



% \paragraph{Knowledge-Base Question Answering}
% Traditional parsing-based methods parse the question into some intermediate query (e.g., SQL language, query graphs), which can execute on a knowledge base to get answer \citep{DBLP:conf/emnlp/BerantCFL13,DBLP:conf/acl/YihCHG15,DBLP:journals/tacl/ReddyTCKDSL16,DBLP:journals/corr/abs-1709-00103,DBLP:conf/acl/LiangBLFL17}. However, existing knowledge bases suffer from low coverage of entities and relations required for open-ended questions. As an alternative, several works try to incorporate the structured knowledge into neural QA models for differentiable reasoning. \cite{DBLP:conf/emnlp/LinCCR19} and \cite{DBLP:conf/emnlp/FengCLWYR20} parse the question into a sub-graph of knowledge base, and apply graph neural networks as reasoner to extract answers. \cite{DBLP:conf/iclr/ChenLYZSL20} integrates general symbolic operations as basic units, and parse questions into compositional programs to answer general questions.




\noindent\textbf{Knowledge-augmented Language Models} 
explicitly incorporate external knowledge (e.g. knowledge graph) into \texttt{LM}~\citep{DBLP:journals/corr/abs-2010-04389}.
Overall, these approaches can be grouped into two categories:
The first one is to explicitly inject knowledge representation into language model pre-training, where the representations are pre-computed from external sources~\citep{DBLP:conf/acl/ZhangHLJSL19,DBLP:conf/aaai/LiuW0PY21,DBLP:conf/emnlp/HuSC21}.
For example, ERNIE~\cite{DBLP:conf/acl/ZhangHLJSL19} encodes the pre-trained TransE~\cite{DBLP:conf/nips/BordesUGWY13} embeddings as input.
The second one is to implicitly model knowledge information into language model by performing knowledge-related tasks, such as entity category prediction~\citep{DBLP:journals/corr/abs-2010-00796} and graph-text alignment~\cite{DBLP:conf/acl/KeJRCWSZH21}.
For example, JAKET~\citep{DBLP:journals/corr/abs-2010-00796} jointly pre-trained both the KG representation and language representation by adding entity category and relation type prediction self-supervised tasks.

There also exists several QA works using $\KG$ to help ODQA. For example, \citet{DBLP:conf/iclr/AsaiHHSX20} and \citet{DBLP:journals/corr/abs-1911-03868} expand the entity graph following wikipedia hyperlinks or triplets in knowledge base. \citet{DBLP:conf/acl/DingZCYT19} extract entities from current context via entity-linking and turn them into a cognitive graph, and a graph neural network is applied on top of it to extract answer. \citet{DBLP:conf/iclr/DhingraZBNSC20} and \citet{DBLP:journals/corr/abs-2010-14439} construct an entity-mention bipartite graph and then model the QA reasoning as graph traversal by filtering only the contexts that are relevant to the question. \citet{DBLP:conf/emnlp/LinCCR19}, \citet{DBLP:conf/emnlp/FengCLWYR20} and \citet{DBLP:conf/naacl/YasunagaRBLL21} parse the question into a sub-graph of knowledge base, and apply graph neural networks as reasoner for extracting one of the entities as the answer.

To encode knowledge (significantly smaller than the web corpus) as \emph{memory} into \texttt{LM} parameter, a line of works try compressed knowledge including QA pairs~\citep{DBLP:journals/corr/abs-2204-04581, DBLP:journals/corr/abs-2102-07033,DBLP:journals/corr/abs-2209-10063}, entity embedding~\citep{DBLP:journals/corr/abs-2004-07202} and reasoning cases~\citep{DBLP:conf/emnlp/DasZTGPLTPM21, DBLP:journals/corr/abs-2202-10610}.
There's also several works utilizing Knowledge Graph ($\KG$) to augment \texttt{LM}. FILM~\citep{DBLP:conf/naacl/VergaSSC21} turns $\KG$ triplets into memory. Given a question, \texttt{LM} retrieves most relevant triplet as answer. GreaseLM~\citep{DBLP:journals/corr/abs-2201-08860} propose to interact \texttt{LM} with $\KG$ via a interaction node. 


% We discuss other related works in Sec.~\ref{sec:related} in Appendix.















% Similar to query answering on graph, answering open-domain questions also require to infer out-of-context knowledge. 
% For example, given only a question $q$ as ``The Bauhaus represented Germany's recovery from which event$?$'', the QA model is asked to predict answer $a$ ''World War I``. To correctly answer such a question, the model needs to gain knowledge about all in-context entity mentions $M=\{m_i\}$, e.g., ''Bauhaus`` and ''Germany``. 
% Such an open-domain question could be abstracted as a query $(M, q, ?)$, or a probabilistic manner $P(a | q, M)$. 



\begin{figure*}[ht]
\centering
\includegraphics[width=0.9\linewidth]{figs/figs/parallelism-updated.pdf}
\caption{\small \textbf{ An Example of a Compute Graph Transformation, Device Mapping and Routing, and End-to-End Time Estimation: (top)} Cross-edges are shown in red. To preserve readability, we only show a subset of cross-edges for kernel parallelism. 
Blue solid borderlines indicates separate hardware nodes. At every parallelization stage, we use black hashed lines to show graph replication along that dimension. A replica is a graph with a similar structure, however the kernel size and/or data size could be different for each replica.
For simplicity, the original graph is a simple 3-layer feed-forward neural network that is divided into two sub-graphs (P2).
Then for each pipeline stage, batch size is distributed across three workers (D3).
Then for each data shard of each pipeline stage, the kernels are distributed in a row-column fashion across a 4$\times$2 torus (RC-K4-K2). \textbf{(middle)} Mapping a 4-D hyper-cube into a 2-D mesh: a greedy layout mapped in the following order: kernel(R), kernel(C), pipeline and data. The bolded black edge in G4 is mapped onto a 4-hop path in the system graph. \textbf{(bottom)} backward pass time estimation. }
%\textcolor{blue}{FIXME: draw a box around the the top row, mark it as graph transformation, draw another box around the bottom row mark it as end-to-end time estimation. Show cross-edges in red.}
\label{fig:transformation}
\vspace{-0.2cm}
\end{figure*}
%gemm_val.tex
\section{Compute Graph Transformation and Device Mapping}
\label{sec:mapping}
%Given the ML model description (in form of a compute graph), the system topology (in form of a system graph) and the parallelization strategy, we use device mapping engine to map the vertices of the compute graph (kernels) onto the vertices of the system graph (nodes) and map the edges of the compute graph (data dependency edges) to the edges of the system graph (physical network links). However, before the mapping happens, we transform the compute graph to reflect the parallelism strategy specified by the user.

%Given the ML model description (in form of a compute graph), and the parallelization strategy, we first transform the compute graph to reflect the parallelism strategy. Next, we use device mapping engine to map the vertices of the compute graph (kernels) onto the vertices of the system graph (accelerator nodes) and map the edges of the compute graph (data dependency edges) to the edges of the system graph (physical network links).

Given the ML model description (in form of a \textit{compute graph}) and the distributed system topology (in form of a \textit{system graph}), 
%an essential step before performance prediction is to 
we find an optimal mapping from vertices and edges in the compute graph to hardware nodes and network links in the system graph. 
%However, there is not a one-to-one mapping from nodes in the compute graph to hardware nodes in the system graph in presence of parallelism: A compute node (kernel) can be replicated and mapped to different hardware nodes (data parallelism), or it can be replaced with smaller sub-kernels (kernel parallelism). This complicates the mapping problem.
However, before mapping, we transform the compute graph into a \textit{super-graph} to reflect the parallelism strategies specified as input.
%to enable a one-on-one mapping between nodes in the compute graph to hardware nodes in the distributed system graph. 


%\vspace{0.1in}
\subsection{Compute Graph Structure Transformation}
%A graph transformation or rewrite defines a set of rules of the form $S \rightarrow R$, with $S$ being the starting graph and $R$ being the result graph. 
%A rewrite rule is applied to the starting graph by searching for an occurrence of the pattern graph (pattern matching, thus solving the subgraph isomorphism problem) followed by replacing the found occurrence by an instance of the replacement graph. 
%A rewrite rule specifies a replacement graph that replaces each node in the starting graph, and how these replacement graphs are connected in the result graph.
%A graph transformation is the stepwise replacement of subgraphs inside a host graph. 
%data parallelism entails replicating the graph $N$ times and connecting the corresponding compute vertices in a fashion consistent with the reduction algorithm: for ring-all-reduce, the corresponding vertices would be connected in a ring fashion.
%A graph transformation or rewrite defines a set of rules that apply to sub-graphs in the \textit{original} graph in order to generate the \textit{result} graph.
%A rewrite rule specifies a \textit{replacement} graph that replaces every instance of a \textit{sub-graph} (pattern matching) in the original graph.
%It also defines how these replacement graphs are connected in the result graph~\cite{}.
Each parallelism strategy is a form of graph transformation where the sub-graph to be replaced is a single node, so essentially all nodes would be replaced with the same replacement graph.
For example, to model data parallelism (with the ring-all-reduce implementation) we would need to 
$\textit{replace}$ each node in the original graph with a ring of length $N$ (for an $N$-data parallel strategy). 
The new edges on the ring will be marked as $\textit{cross-edge}$ to capture the fact that they connect compute nodes hosted on separate devices.
To capture a kernel parallelism strategy (e.g. $\texttt{RC-\{KP1\}-\{KP2\}}$), we would need to $\textit{replace}$ each node in the compute graph with a 2-dimensional torus of $\texttt{KP1} \times \texttt{KP2}$ dimension 
(assuming the reduction algorithm along each dimension is ring-all-reduce). 
Similarly, new edges on the torus would be marked as cross-edge.
To capture a pipeline parallelism, no node transformation is required. The pipeline parallelism slices the original graph into multiple sub-graphs, each hosted on a separate hardware node.
Edges connecting sub-graphs would be marked as cross-edge.
Figure~\ref{fig:transformation} shows the composition of multiple parallelism strategies applied in sequence (pipeline, data and kernel parallelism, respectively). 
$G_0$ is the original compute graph and $G_4$ is the final transformed graph.



%Pipeline parallelism, unlike other forms, does not result in replication or replacement of graph nodes. It simply modifies the edge type between nodes across layers that are mapped to different pipeline stages to capture their cross-connection nature. 

%\vspace{0.1in}
\subsection{Device Mapping and Routing Engine}
%After the transformation stage, the transformed graph would have $N_r\times$ more number of nodes than the original graph, where $ N_r = dp\times kp_1 \times kp_2$, and $dp$, $kp_1$ and $kp_2$ are multiplicative factors of data parallel and kernel parallel strategy. 
%Moreover, to exploit pipeline parallelism, each model replica can be partitioned into $P$ sub-graphs and hosted in separate devices. 
%Data parallelism, kernel parallelism and pipeline parallelism would require each parallel shard to be hosted on a separate physical device. Hence, the total number of hardware nodes, $N_h$, should be $N_r \times P$.
%We use pipeline parallelism to slice the super-graph into smaller partitions such that number of partitions is consistent with total number of replicas and number of physical devices. 
Data parallelism, kernel parallelism and pipeline parallelism would require that each parallel shard to be hosted on a separate physical device.
Hence, device mapping happens at the granularity of a parallel shard. 
%Each parallel shard is basically a sub-graph in the transformed graph.
We want parallel shards that are close in the parallel space to be mapped onto nodes that are close in the physical space to minimize communication. 
%However, parallelism space is usually a 5 or 4 or 3-dimensional hypercube, while the underlying system graph is usually a 3 or 2-dimensional mesh or torus.
However, the transformed graph usually has higher dimension than the system graph. 
Figure~\ref{fig:transformation} shows such example, where the final transformed graph ($G_4$) is 4-D hypercube and the system graph is a 2-D torus.
Therefore, it will not be possible to map all adjacent nodes in the compute graph to adjacent nodes in the system graph.
We adopt a greedy approach to conduct such mappings: We start with a parallel dimension, map all parallel shards along that dimension to adjacent nodes in the hardware.
If the number of shards along the parallel dimension is larger than the hardware dimension we are mapping onto, we wrap-around to the next immediate dimension.
We continue this process along other dimensions in a specific order, until all nodes are mapped.
The order at which we walk along the parallelism dimensions results in different mappings.
For 4 different parallelism strategies, we explore $(4!)$ = 24 possible orderings to pick the best mapping.
Once node mapping is determined, we take a last step to map edges to physical links.
An edge that connects to adjacent node in the compute graph may map to a multi-hop path as shown in Figure~\ref{fig:transformation}. 
As a result, one physical link would be shared across multiple edges.
The number of logical edges sharing a physical link is an important factor for effective bandwidth estimation.
We use $X-Y$ routing to map edges in the compute graph to paths in the system graph.
Overall, the whole transformation step followed by device mapping is necessary to find an accurate estimation of \textit{edge} timing. 


%Depending on the parallelism strategy, the transformed compute graph can take the form of a 5D/4D/3D hypercube. However, the underlying system topology may be a 2D or 3D hypercube. This means that multiple edges in the compute graph would need to be mapped in to one edge/link of the system graph. The number of logical edges that are mapped into a physical edge is an important metric as it captures the effect of link bandwidth sharing which ultimately affects the communication overhead. Based on link sharing, we allocate the derated/partial link bandwidth to the edges of the compute graph. In order to find a mapping that minimizes the maximum number of edges that is mapped in to each network link, we use a greedy heuristic. Our heuristic works as follows: starting with one axis in the parallelization space, we map the vertices along that axis as close as possible, before moving to the vertices in the second axis. We repeat this for different permutations of axes and choose the one which minimizes the communication overhead.


%\todosaptadeep{These different parallelism strategies have major implications on end-to-end performance and stress different resources of the system. As an example, very large models often wouldn't fit in the main memory capacity of a single accelerator device, and therefore model and/or pipeline parallelism needs to be employed. Model parallelism on the other hand, can stress inter-node network characteristics.}

\section{Interpretation of UF Decoder}
\label{sec:interpretation}

We next reveal the relationship between a union-find decoder~\cite{delfosse2021almost} and the famous blossom algorithm~\cite{edmonds_1965,kolmogorov2009blossom} that solves MWPM problems.

We will first introduce some necessary concepts in the blossom algorithm in \S\ref{sec:blossom} and then show how it solves the surface code decoding problem in \S\ref{ssec:blossom_sc}.
In \S\ref{sec:uf_interpretation}, we demonstrate that union-find decoders are close relatives of MWPM decoders.
Inspired by this, we describe a novel, general weighted union-find decoder in \S\ref{sec:weighted_uf}.

\subsection{Blossom Algorithm}
\label{sec:blossom}

The blossom algorithm uses linear programming (LP) to solve the MWPM problem~\cite{edmonds_1965,kolmogorov2009blossom} for a graph defined by $G(V, E)$ where $V$ is the set of vertices with even cardinality, and $E$ that of edges.
A matching $M$ of $G$ is a subset of $E$ in which no edges share a vertex.
A perfect matching $M_p$ is a matching whose elements cover all vertices of $G$. 
$\mathcal{O}$ is the set of subsets of $V$ with odd cardinaility, i.e. $\mathcal{O}=\{S| S\subset V; |S| \mathrm{~is~odd.}\}$.
For $e= ( u,v ) \in E$, if $u\in S$ and $v\notin S$, we say $S$ and $e$ are incident to each other.

In the primal problem of the linear programming formulation, a primal variable $x_e$ corresponds to $e \in E$. 
For each $S\in \mathcal{O}$, there is a \textit{primal constraint}. If $|S|=1$, exactly one edge incident to $S$ is in the solution. If $|S|>1$, at least one edge incident to $S$ is in the solution.
While $x_e$ is non-negative real, the primal constraints ensure that in one optimal solution,  $x_e$ is either $1$ or $0$ $\forall e \in E$,  representing a \emph{perfect matching}: $x_e = 1$ if $e$ is in the solution.
The \textit{primal objective function} $\sum_{e} w_e x_e$ is the total weight of the edges in the solution, to be minimized.

In the dual of the above problem~\cite[p. 81]{matousek2006understanding},
a \textit{dual variable} $y_S$ is defined for $S\in \mathcal{O}$, corresponding to a primal constraint.
Each dual constraint corresponds to an edge $e\in E$ (and its primal variable $w_e$): 
$\sum_{S\in\delta(e)} y_S \leq w_e$ where $\delta(e)$ = $\{S|S\in\mathcal{O}; S \mathrm{~incident~to~} e\}$.
The dual objective is to maximize $\sum_{S\in\mathcal{O}} y_{S}$.

The blossom algorithm leverages two insights, both based on the complementary slackness relationship between the primal and dual problems~\cite[p. 204]{matousek2006understanding}.
First, if $x_e>0$, i.e., $e$ is selected in the matching solution, the corresponding dual constraint must be \emph{tight}, i.e., $\sum_{S \in \delta(e)} y_S = w_e$. That is, the solution to the primal problem can only consist of \emph{tight} edges. 
Second, if $y_S>0$ for $S\in\mathcal{O}$ and $|S|>1$, the corresponding primal constraint must be tight: exactly one edge incident to $S$ is in the solution.

The second insight allows the algorithm to treat $S\in\mathcal{O}$ and $|S|>1$ like a vertex when $y_S>0$. Such $S$ are the eponymous \emph{blossoms}. Therefore, we use ``vertex'' to refer to both ordinary vertex and blossom below.

The first insight allows the algorithm to work on the primal and dual problems in an interleaving manner. 
When it works on the primal problem, it only considers the tight edges as candidates for the matching solution.
It identifies blossoms, odd number ($>1$) of ``vertices'' connected by tight edges in a circle, and alternating trees, odd number of ``vertices'' connected by tight edges in a tree where all leaves and nodes with multiple children are connected to the root through an even number of tight edges.
When it works on the dual problem, it adjusts $y_S$ to grow the dual objective function $\sum y_S$ while maintaining the tightness of edges in a blossom or alternating tree.
In doing so, it turns a dual constraint tight, that is $\sum_{S \in \delta(e)} y_S = w_e$, resulting in a new tight edge $e$.
With this new tight edge, the algorithm switches to work on the primal problem.   

\subsection{Blossom for Decoding Surface Code}
\label{ssec:blossom_sc}
When applying the blossom algorithm to a graph, 
we can imagine two vertices are separated by a distance of $w_e$, the weight of the edge incident to them. 
We can imagine that a \region covers a ``vertex'' $v$; 
for each edge incident to $v$, the \region covers  it by $y_v$, which is the dual variable.
% We say the \region $v$ expands/shrinks when $y_v$ increases/decreases.
The dual constraints dictate that the \regions would never overlap. When two \regions $u$ and $v$ meet on an edge $e=(u,v)$, $e$ becomes tight, i.e., $w_e=y_u+y_v$. 
We could also imagine a cluster, called \emph{\cluster}, started with a single \region.
When two \regions each from a different \cluster touch, the two \clusters merge.
Like \regions, \clusters do not overlap (i.e., share any vertices). A \cluster includes multiple \regions that touch each other in one way or another.
Because each \region $v$ corresponds to a non-zero dual variable $y_v$, we call the sum of the dual variables of the regions $\sum y_v$ within a \cluster the \csize of the \cluster.
We say a \cluster is even/odd if it contains an even/odd number of \regions.
If a \cluster covers virtual vertices from both left and right, we say it is \emph{attached}; 
otherwise, it is \emph{detached}. 
For example, in \autoref{fig:blossom-multi-grow-1.4} there are three \clusters, namely $\{A,B,C\}$, $\{D\}$, and $\{E\}$, all detached.

When the model graph is unweighted, this imagination can be conveniently visualized by the \emph{\diagram}~\cite{fowler2014minimum}.
In this diagram, the Manhattan distance between two vertices is the weight of the corresponding edge, i.e., $w_e$, as illustrated by \autoref{fig:blossom_multi_grow}.
We note that  the model graph of a surface code is unweighted if the data qubits have i.i.d. (independent and identically distributed) Pauli-X errors.

This imagination allows us to explain how the blossom algorithm works visually. Notably the algorithm actively operates on the internal of \cluster. 
When it works on the dual problem, it adjusts \regions to increase the \csize of each \cluster. For example, \autoref{fig:blossom-multi-grow-2.6} to \autoref{fig:blossom-multi-grow-2.9}.
Increasing the dual clusters may create a new tight edge and result in two \clusters merging into a ``larger'' one, in terms of the \csize and the number of vertices covered.
During the adjustment, the algorithm tries to keep tight edges within blossoms and alternating trees tight.

When the algorithm works on the primal problem, it only considers the tight edges. 
It will try to find a perfect matching inside each \cluster using only tight edges. When successful, it will mark that \cluster as \emph{solved} and will not work on it (and its \regions) unless the \cluster merges with another.
The algorithm may identify a blossom within a \cluster: an odd number of tight edges forming a circle, e.g., \autoref{fig:blossom-multi-grow-1.4}.
The algorithm then switches to work on the dual problem again by treating the blossom as a vertex and adjusting its own dual variable, e.g., $y_{\{A,B,C\}}$ in \autoref{fig:blossom-multi-grow-1.4} to \autoref{fig:blossom-multi-grow-2.9}, along with other dual variables.
The process repeats until all \clusters are solved.

By definition, a tight edge must be inside a \cluster because a tight edge corresponds to two touching \regions.
As a result, a matching solution to the primal problem must only include edges \emph{inside} \clusters. 

\subsection{Union-Find Decoder}
\label{sec:uf_interpretation}

We next show that a union-find (UF) decoder works on the syndrome graph to draw direct comparison with the blossom algorithm, using a similar visualization as used above. It is an adaptation from the original UF decoder~\cite{delfosse2021almost} that is based on the decoding graph. (See \autoref{ap:equiv_uf})

Unlike the blossom algorithm, which maintains the internal structure, i.e., \regions, for each \cluster, UF decoders only care about the number of vertices inside each cluster, i.e., whether a cluster is even or odd, and it only grows odd clusters.
In each step of growth, a UF decoder grows all odd clusters by the same amount: it makes sure the clusters do not overlap after growth. When two clusters touch, they get merged.
It stops growing a cluster when the cluster covers a virtual boundary vertex or becomes even. 
When no growth is possible, a UF decoder finds a subgraph as the solution for each cluster using only fully-grown edges.
This solution is logically equivalent to a perfect matching for the cluster that may use not-fully-grown edges.
Taken together, the solutions for all clusters form the solution for the syndrome graph, which is logically equivalent to a perfect matching. 
\autoref{fig:union_find_multi_grow} illustrates this procedure with the same \diagram from \autoref{fig:blossom_multi_grow}.
Different UF decoders may find different solutions within a cluster and as a result, may produce different solutions for the syndrome graph.

A UF decoder grows an odd cluster in the same way as the blossom algorithm grows a \cluster with a single ``vertex''.
It stops updating the even clusters while the blossom algorithm stops updating the solved \clusters for which a perfect matching with tight edges is found internally. 
Notably only a \cluster with an even number of ``vertices'' can be solved.

\subsection{Relationship between Blossom and UF}
\label{sec:relation}

At a high level, the blossom algorithm-based MWPM decoder and a UF decoder appear to be similar in that both decomposes the syndrome graph into non-overlapping subgraphs, i.e., clusters; both find the solution to the syndrome graph by aggregating the solutions of the subgraphs. 
Yet there are two key differences, which are behind the MWPM decoder's superiority in decoding accuracy and poor scalability. First, the MWPM decoder has a more sophisticated way to decompose the syndrome graph, or grow its \clusters. Second, it finds a minimum-weight perfect matching within a subgraph (\cluster) while the UF decoder is satisfied with finding a logical equivalent of a perfect matching. Intuitively, we have:


\textbf{Observation (UF/Blossom Similarity)}: given a syndrome graph, a UF decoder approximates the blossom-based MWPM decoder in accuracy, if the following two conditions are true.
\begin{itemize}
    \item Condition 1: They decompose the syndrome graph in a similar way. In the extreme case, there is a bijective mapping between their clusters such that the mapped clusters cover the same subset of vertices.
    
    \item Condition 2: Whether a perfect matching inside a cluster is minimum-weight does not matter.
\end{itemize}

We next examine situations when these two conditions may be true or close to be true. 
\subsubsection{Syndrome graph decomposed}
We first examine how they decompose the syndrome graph. While there are an enormous number of ways to decompose the graph, a number of factors constrain both the UF decoder and blossom algorithm so that they may end up decomposing the graph in a similar way. 

First of all, they have the same starting point: the same syndrome graph and the same set of clusters, each with a single vertex.

Second, when they terminate, their clusters must satisfy the following requirements: 
they must not overlap; each of them must have an even number of vertices because only even clusters can be solved; and a vertex is likely (but not always) to be in the same cluster as its nearest neighbor. The last is true because both grow all clusters by the same amount in each step of growth and merge clusters when they meet, forming literally clusters of vertices. 

Third, syndrome graphs that get decomposed into small (and therefore similar) clusters by both the UF decoder and blossom algorithm are more likely. This is due to the assumption that data qubit errors happen randomly and independently. As a result, nontrivial measurement outcomes are more likely to be randomly scattered in the model graph in pairs, resulting in a syndrome graph in which these pairs are much farther from each other than the two vertices within a pair. Such syndrome graphs are likely to get decomposed by both the UF decoder and blossom algorithm into small clusters each covering a pair or two.

Condition 1 also provides insight into why some revisions of UF decoders improve their accuracy~\cite{delfosse2021almost,huang2020fault}: these revisions allow a UF decoder to decompose a syndrome graph in a way closer to that the blossom algorithm would do.


\vspace{1ex}\textbf{Cluster vs. subgraph}:~~Given a syndrome graph, clusters (and \clusters) are defined by the vertices they cover while a subgraph is defined by the edges it includes.
Given a cluster, one can uniquely construct a subgraph, using the edges connecting any pair of vertices covered by the cluster. Likewise, given a subgraph, one can uniquely construct a cluster, using the vertices incidental to edges from the subgraph. Therefore, we use cluster and subgraph in an inter-exchangeable manner below, unless otherwise indicated.
For example, when we say a perfect matching for a cluster or inside a cluster, we are talking about the perfect matching for its corresponding subgraph. 
Interesting, if two clusters do not overlap, i.e., not sharing any vertex, their subgraphs do not overlap either, i.e., not sharing any edge. 


\subsubsection{Equivalent matchings}
Assume the syndrome graph has been decomposed into non-overlapping clusters each with an even number of vertices.  Two perfect matchings $P_1$ and  $P_2$ for the syndrome graph are found by finding perfect matchings inside each cluster.

We know that a logical error happens when a chain of error connects left and right virtual vertices~\cite{bravyi1998quantum}, forming a nontrivial logical operator. 
Since a detached cluster cannot have such a chain in itself by definition, a subgraph of the cluster cannot represent a nontrivial logical operator. Therefore, we have the following Lemma and Theorem. See \autoref{ap:proof} for their proofs.

\noindent \textbf{Lemma (Equivalent Matchings)} if a cluster is detached, its perfect matchings are logically equivalent.
\newline

\noindent \textbf{Theorem (Equivalent Matchings)} if $P_1$ and $P_2$ are different only inside detached clusters, they are logically equivalent. \newline

When some \clusters are attached, the above cluster-based method will lead to a higher logical error rate than the MWPM decoders.
On the other hand, because usually most \clusters are detached, the difference in logical error rate can be small. Attached clusters are exponentially less likely with increasing code distance ($d$).

\section{Examples}
\label{sec:example}
We next provide two examples in which the two conditions described above are true and as a result, a UF decoder will be as accurate as an MWPM decoder.
\subsection{No adjacent errors}
When no errors are present in two adjacent data qubits, both conditions of the Observation are true and a UF decoder will achieve the same accuracy as MWPM decoders.
This is because when no errors are present in two adjacent data qubits, 
 vertices in the syndrome graph always appear in pairs that are far from each other or a single vertex adjacent to a virtual boundary vertex.
As a result, both a UF decoder and the blossom algorithm will decompose the syndrome graph into clusters each covering such a pair.

\subsection{XZZX code}
The XZZX surface code is a variant of the CSS surface code~\cite{bonilla2021xzzx}.
It employs a single type of stabilizer that measures $X\otimes Z\otimes Z\otimes X$.
Thus in the XZZX surface code, a single $X$ (or $Z$) error always generates a pair of nontrivial measurements horizontally (or vertically).

We next show the two conditions presented in \S\ref{sec:interpretation} are true for the XZZX code with infinite noise bias and noiseless stabilizers, known as the code capacity noise model~\cite{landahl2011fault}. We note the two conditions are not true with the circuit-level noise model~\cite{darmawan2021practical}.

Without loss of generality, we assume there are only Z errors on data qubits.
The model graph now becomes a set of disjoint subgraphs: each is a line (or 1D chain). As a result, the syndrome graph also becomes a set of parallel lines.
A perfect matching of the syndrome graph must comprise of perfect matchings for all such lines, each found separately.


\paragraph{Identical Clusters.}
An MWPM decoder grows an odd dual cluster at the same pace due to the use of the multiple tree approach~\cite{kolmogorov2009blossom}. When the dual cluster is 1D, it grows left and right at the same pace, exactly like how an odd cluster grows in a UF decoder.
In general, an MWPM decoder may not find a perfect matching with only tight edges inside an even dual cluster. 
However, inside a 1D even dual cluster, an edge between two adjacent vertices must be tight because of how these two vertices got merged into one cluster. As a result, an MWPM decoder can always find a perfect matching with tight edges by selecting edges between pairs of adjacent vertices. That is, in a 1D chain, an MWPM decoder always consider an even dual cluster solved, exactly like how a UF decoder treats an even cluster. 
Thus, the MWPM and UF decoders update clusters in exactly the same way and terminate at the same clusters when the clusters are 1D.

\paragraph{Equivalent Matchings}
When a cluster is detached, the solutions from the UF decoder and the MWPM decoder never differ by a nontrivial logical operator, according to \hyperref[lemma:equivalent_matchings]{\textit{Lemma (Equivalent Matchings)}}.
Otherwise, if the cluster is attached,  there are only two complementary perfect matchings of the cluster. Because a cluster grows left and right at the same rate, these two complementary perfect matchings must have the same weight and therefore both are MWPM. Because a UF decoder must select one of them, it will select a MWPM for the cluster and as a result, behave the same as an MWPM decoder for this cluster.    


\section{Weighted Union-Find Decoder}
\label{sec:weighted_uf}
An astute reader will point out that the original union-find decoder~\cite{delfosse2021almost} works on the decoding graph. 
Yet the interpretation provided above works on the syndrome graph. This difference is only cosmetic as we purposefully adapt the union-find decoder for the syndrome graph in order to juxtapose it with the blossom algorithm. Implementation-wise, the decoding graph is preferred for lower time complexity. See \autoref{ap:equiv_uf} for more explanation.

Another, more substantial difference is that the original UF decoder works on an \emph{unweighted} decoding graph, assuming identical error probability for all data qubits. 
As a result, it grows all odd clusters by half a unit each step.
Yet the interpretation above does not need this assumption. Rather, it must compute the safe amount of growth for the odd clusters at each step.
This leads to a more general union-find decoder that works with weighted model graphs, described below.

Huang, Newman and Brown~\cite{huang2020fault} already report a UF decoder design that uses weighted model graphs, without explicitly identifying the link between the UF and MWPM decoders presented above. They compute the weight as $\ln((1-p)/p)$ where $p$ is the error probability, which is similar to the integer-weighted UF decoder described below. 

\paragraph{Real-Weighted Union-Find Decoder}
A real-weighted union-find decoder has a time complexity no worse than $O(N^2)$, $N$ being the number of vertices in the model graph.
This is due to two factors: first, in each step, it has a time-complexity of at most $O(N)$ to compute the maximum safe growth such that when all odd clusters grow by that much, they will not overlap.
The maximum safe growth is calculated such that at least one more edge is fully covered by clusters, i.e., becomes fully grown, to use the language of Delfosse and Nickerson~\cite{delfosse2021almost}.
Second, it takes at most $O(N)$ steps to finish grow all clusters because each step will get at least one edge covered by the clusters while there are $O(N)$ uncovered edges to begin with.

A real-weighted union-find decoder should have  better time complexity, worst-case and average, than the blossom algorithm, for three reasons.
First, the blossom algorithm works on the syndrome graph, which has $O(N^2)$ edges, while the union-find decoder works on the model graph, which has $O(N)$ edges.
Second, the \clusters in the blossom algorithm may shrink, while the clusters in the union-find decoder only grow. The possibility of shrinking may lead to more steps before finish growing all clusters.
Finally, the blossom algorithm maintains the structure inside each cluster. As a result, it merges two clusters with a time complexity proportional to the cluster size, while the real-weighted UF decoder merges two clusters within constant time.


\paragraph{Integer-Weighted Union-Find Decoder}
When the weights of the model graph are small integers, union-find decoders can be much faster in terms of the worst-case time complexity.
This is because the safe growth computation becomes trivial: 
all odd clusters grow by half a unit each step, just like in the original union-find decoder.
Each edge with weight $w_e$ in the model graph is at most visited $O(w_e)$ times.
Given the maximum weight $w_{\max}$, the overall worst-case time complexity of integer-weighted union-find decoder is $O(N \cdot (\alpha(N) + w_{\max}))$ where $\alpha(N)$ is an almost constant inverse Ackermann’s function~\cite{delfosse2021almost}.
As a result, the average decoding time complexity must be between $O(N)$ and $O(N \cdot (\alpha(N) + w_{\max}))$, which is almost linear.

On the other hand, we can see the worst time complexity of an integer-weighted UF decoder grows with $w_{\max}$ while that of a real-weighted union-find decoder does not. 
Therefore, when $w_{\max}$ is large, a real-weighted union-find decoder can be faster than an integer-weighted one when the integer weights are too large.


\paragraph{Implementation}
We have implemented and open-sourced both weighted UF decoders described above~\cite{qec-playground}.
Our implementation of the real-weighted UF decoder uses the integer data type (64-bit signed integer) to avoid rounding errors in the floating point data type, a strategy borrowed from the blossom V algorithm implementation~\cite{kolmogorov2009blossom}.
For both weighted UF decoders, we use an integer $w_{\max}$ to represent the largest weight $W = \max_e{w_e}$ and compute the integer weight for an edge of weight $w_e$ as $\lfloor w_e /W * w_{\max}\rfloor$.
The only difference between our implementations of the real-weighted and integer-weighted UF decoders lies in how they compute the growth when growing the odd clusters: the real-weighted UF decoder computes the maximum safe growth while the integer-weighted one grows them by one each time, as explained in~\S\ref{sec:weighted_uf}.
As a result, they have the same accuracy given the same scaling $w_{\max}$.

\section{Discussion}
\label{sec_discussion}
We introduce an exponential family approach to learn unit-level counterfactual distributions from a single sample per unit even when there is unobserved confounding. By conditioning on the latent confounders and using a novel convex loss function, we estimate the parameters of unit-level counterfactual distributions given the information about what actually happened. {The resulting estimates of unit-level counterfactual distributions enable us to estimate any functional of each unit's potential outcomes under alternate interventions. We analyze each unit's expected potential outcomes under alternate interventions, thereby providing a guarantee on unit-level counterfactual effects, i.e., individual treatment effects.}
%
%
%
We note that our approach makes only macro-level assumptions about the underlying causal graph and does not assume the knowledge of the micro-level causal graph.

%

%

A side product of our results is a strategy for answering interventional questions, e.g., to estimate average treatment effects. These questions are equivalent to estimating distributions of the form $f_{\rvby | \mathrm{do}(\rvba)}(\svby | \mathrm{do}(\rvba = \svba))$ where the do-operator \citep{Pearl2009} forces $\rvba$ to be $\svba$. Under the causal framework considered (\cref{fig_graphical_models}(b)), we have $f_{\rvby | \mathrm{do}(\rvba)}(\svby | \mathrm{do}(\rvba = \svba))=\Expectation_{\rvbv, \rvbz}\normalbrackets{f_{\rvby | \rvba, \rvbz, \rvbv}(\svby | \svba, \svbz, \svbv)}$. Consequently, the mixture distribution $n^{-1} \sum_{i \in [n]} \what{f}^{(i)}_{\rvby | \rvba}(\svby | \svba)$ with $\what{f}^{(i)}_{\rvby | \rvba}(\svby | \svba)$ defined in \cref{eq_counterfactual_distribution_y}, serves as a natural estimate via our strategy. 
%
%
Investigating the efficacy of this estimator is an interesting future direction. 
%

{In this work, the conditional exponential family distribution of $\rvby$ in \cref{subsec_exp_fam} or in \cref{subsec_high_terms} was such that the effect of unobserved covariates $\rvbz$---after conditioning on them---was captured by a first-order interaction term varying with the realized value of $\rvbz$ for each unit, e.g., $\sbraces{\ExternalField(\svbz^{(i)})}_{i=1}^n$for the conditional distribution in \cref{subsec_exp_fam}. Focusing on \cref{subsec_exp_fam}, when one considers higher-order interaction terms in the joint distribution, the conditional distributions would also have higher-order interaction terms (the highest order in the conditional distribution is one less than the highest order in the joint distribution) that vary with $\rvbz$. Focusing on \cref{subsec_high_terms}, the exponent of the exponential tilting of the base distribution of the outcomes by the unobserved covariates could have higher-order terms.} For such cases, while our analysis for population-level parameters (\cref{theorem_parameters} Part I's proof in \cref{sec:proof_of_theorem_parameters}) is likely to extend easily, new arguments for analyzing quadratic (or higher-order) interaction terms that vary for each unit seem necessary. Developing these results, e.g., suitable analogs of Dobrushin's condition for higher-order exponential family, present an exciting future venue for research.
%
%
%
%

Our methodology can be useful for a class of multi-task learning problems \citep{caruana1997multitask}, e.g., when we have multiple logistic regression tasks with some commonalities. For a logistic regression task, the exponential family model~\cref{eq_conditional_distribution_vay} has been used by \cite{DaganDDA2021} to allow dependencies between the labels via the parameter $\ParameterMatrix$ (instead of assuming independence between the labels), e.g., for spatio-temporal data. They consider a single regression task and assume that the dependency matrix $\ParameterMatrix$ is known up to a constant and learn a task-specific parameter $\ExternalField(\svbz)$ (where $\svbz$ denotes a task). Our model and methodology apply to the case of fully unknown $\ParameterMatrix$ given multiple datasets that share the same dependency parameter $\ParameterMatrix$ but have varying task-specific parameters $\ExternalField(\svbz)$; and provide a tractable way to estimate all these parameters together. 
%
{In fact, our framework and results also apply beyond the quadratic dependencies captured by $\ParameterMatrix$ as described in \cref{subsec_high_terms}.} Analyzing whether our methodology can be extended beyond logistic regression models for multi-task learning is a question worthy of further investigation.
%
%

%
%
%
%
%
%
%
%
%
%
%

%
%
%
 
%

%
%
%

%





%
%

%

%
%

%
%
%
%
%
%
%
%
%
%
%
%
%
%
%
%
%
%
%
%
%
%




%
%



%
%

%

\bibliographystyle{naturemag}
\bibliography{ref}{}

\section*{Acknowledgements}
We thank Shruti Puri for fruitful discussion and insightful feedback. This work was supported in part by Yale University and NSF MRI Award \#2216030.

\appendix

\section{Supplemental Tables}

%\section{Hyperparameters of Other Bandit Algorithms}
%\label{sec:bandit_hyperparams}
%Table~\ref{tab:hyperparams} lists the hyperparameters for bandit algorithms other than dBE.

\newcommand\topmidheader[2]{\multicolumn{#1}{c}{\textbf{#2}}\\%
                \addlinespace[1ex]}

\newcommand{\midheader}[2]{%
        \midrule\topmidheader{#1}{#2}}

\newcommand{\specialcell}[3][c]{% 
        \begin{tabular}[#1]{@{}#2@{}}#3\end{tabular}}%

\aptLtoX[graphic=no,type=env]{\begin{table}[htb]
  \centering
  \caption{Hyperparameters of bandit algorithms}
  \label{tab:hyperparams}
  \begin{tabular}{llc}
    \toprule
    Sign & Description & Value \\
    \multicolumn{3}{c}{\textbf{UCB1}}\\
    $c$ & Parameter to control the confidence level used in $\sqrt{c \cdot {\log{t}}/{N_t(arm)}}$ & 0.5  \\
    \multicolumn{3}{c}{\textbf{Thompson Sampling}}\\
    $p(\theta)$ & Prior Distribution & $\mathcal{B}(1, 1)$ \\
    \multicolumn{3}{c}{\textbf{discounted Thompson Sampling}}\\
    $\gamma$ & Discount factor & $1-10^{-8}$ \\
    \multicolumn{3}{c}{\textbf{discounted Thompson Samplingadaptive shrinking Thompson Sampling}}\\
    $M$ & Parameter to control memory usage in a data structure ADWIN2 \cite{ADWIN} & 10 \\
    $\delta$ & Parameter to control the confidence level in a data structure ADWIN2 & $1-10^{-7}$ \\
    \multicolumn{3}{c}{\textbf{EXP-IX}}\\
    $\eta_t$ & Parameter used for weights of arms & $\sqrt{\frac{2 \cdot \log{K}}{K \cdot t}}$ \\
    \addlinespace[1ex]
    $\gamma_t$ & Parameter used for loss estimates & $\frac{\eta_t}{2}$ \\
    \multicolumn{3}{c}{\textbf{EXP3++}}\\
    $\alpha$ & Constant used in calculating $\xi_t(a)$ & $3$ \\
    $\beta$ & Constant used in calculating $\xi_t(a)$ & $256$ \\
    \bottomrule
  \end{tabular}
\end{table}}{\begin{table}[htb]
  \centering
  \caption{Hyperparameters of bandit algorithms}
  \label{tab:hyperparams}
  \begin{tabular}{llc}
    \toprule
    Sign & Description & Value \\
    \midheader{3}{UCB1}
    $c$ & \specialcell{l}{Parameter to control the confidence \\ level used in $\sqrt{c \cdot {\log{t}}/{N_t(arm)}}$} & 0.5  \\
    \midheader{3}{Thompson Sampling}
    $p(\theta)$ & Prior Distribution & $\mathcal{B}(1, 1)$ \\
    \midheader{3}{discounted Thompson Sampling}
    $\gamma$ & Discount factor & $1-10^{-8}$ \\
    \midheader{3}{adaptive shrinking Thompson Sampling}
    $M$ & \specialcell{l}{Parameter to control memory usage \\ in a data structure ADWIN2 \cite{ADWIN}} & 10 \\
    $\delta$ & \specialcell{l}{ Parameter to control the confidence \\ level in a data structure ADWIN2} & $1-10^{-7}$ \\
    \midheader{3}{EXP-IX}
    $\eta_t$ & Parameter used for weights of arms & $\sqrt{\frac{2 \cdot \log{K}}{K \cdot t}}$ \\
    \addlinespace[1ex]
    $\gamma_t$ & Parameter used for loss estimates & $\frac{\eta_t}{2}$ \\
    \midheader{3}{EXP3++}
    $\alpha$ & Constant used in calculating $\xi_t(a)$ & $3$ \\
    $\beta$ & Constant used in calculating $\xi_t(a)$ & $256$ \\
    \bottomrule
  \end{tabular}
\end{table}}

\begin{table}[htb]
  \centering
  \caption{Commit IDs of the PUTs used in our vulnerability discovery and AFL++ used as the baseline.}
  \begin{tabular}{lc}
    \toprule
    Program & Commit \\
    \midrule

    AFL++ & 32a0d6ac315 (ver ++3.14c) \\
    Bloaty &  60209eb \\
    HarfBuzz & 77eeec5 \\
    libarchive & 86c9361 \\
       libxml2 & dea91c9 \\
    MuPDF & ef3d68d \\
   PHP & fdf0455f \\
    Poppler & 6d72d82 \\
    PROJ & 76dfefe \\
    QPDF &  3794f8e \\
    libtpm2 & bc3bb26 \\
    Wireshark  & 1fc621e \\
    Xpdf & N/A (ver 4.03) \\

    \bottomrule
  \end{tabular}
\label{tab:commit-ids}
\end{table}


\begin{table}[htb]
  \centering
  \caption{Initial and theoretical maximum values of code coverage of the PUTs in FuzzBench. 
           Initial values were investigated only in the PUTs used.}
  \begin{tabular}{lcc}
    \toprule
    PUT & Initial & Maximum \\
    \midrule

bloaty\_fuzz\_target & N/A & 83114 \\
curl\_curl\_fuzzer\_http & N/A & 78362 \\
freetype2-2017 & 1517 & 26262 \\
harfbuzz-1.3.2 & N/A & 12212 \\
jsoncpp\_jsoncpp\_fuzzer & N/A & 2114 \\
lcms-2017-03-21 & 149 & 7036 \\
libjpeg-turbo-07-2017 & N/A & 9384 \\
libpcap\_fuzz\_both & 2 & 7294 \\
libpng-1.2.56 & 138 & 3736 \\
libxml2-v2.9.2 & 258 & 67994 \\
libxslt\_xpath & N/A & 51456 \\
mbedtls\_fuzz\_dtlsclient & N/A & 12888 \\
openssl\_x509 & 6026 & 54116 \\
openthread-2019-12-23 & N/A & 19846 \\
php\_php-fuzz-parser & N/A & 215210 \\
proj4-2017-08-14 & 46 & 6534 \\
re2-2014-12-09 & 1 & 3982 \\
sqlite3\_ossfuzz & 4767 & 28766 \\
systemd\_fuzz-link-parser & N/A & 1798 \\
vorbis-2017-12-11 & 410 & 4082 \\
woff2-2016-05-06 & N/A & 5708 \\
zlib\_zlib\_uncompress\_fuzzer & N/A & 910 \\

    \bottomrule
  \end{tabular}
\label{tab:fuzzbench_max_cov}
\end{table}

\begin{table}[htb]
\centering
\caption{List of unique bugs found in the 7-day trial (manually triaged).}
\begin{minipage}{\columnwidth}

\centering
\begin{tabular}{lll}
\toprule

ID & PUT & Bug Type \\
\midrule
Bug-A & bloaty & NULL Pointer Deref \\
Bug-B & harfbuzz & Out-of-bounds Read \\
Bug-C & mupdf & Assertion Fail \\
Bug-D & mupdf & NULL pointer deref \\
Bug-E & xpdf & Stack Overflow \\
Bug-F & xpdf & NULL Pointer Deref \\
Bug-G \footnote{CVE-2022-24106 is issued.} & xpdf & Use of Uninitialized Value \\
Bug-H \footnote{CVE-2022-24107 is issued.} & xpdf & Integer Overflow \\
Bug-I & php & Use-After-Free \\
Bug-J & php & Use-After-Free \\
Bug-K & php & NULL Pointer Deref \\
Bug-L & php & Use-After-Free \\ 
Bug-M & php & NULL Pointer Deref \\
Bug-N & php & Assertion Fail \\
Bug-O & php & Use-After-Free \\
Bug-P & php & Use-After-Free \\
Bug-Q \footnote{CVE-2022-23308 is issued.} & libxml2 & Use-After-Free \\
\bottomrule
\end{tabular}

\label{tab:7d-bug}
\end{minipage}
\end{table}

\begin{table*}[htb]
  \centering
  \caption{List of the PUTs used in Section~\ref{sec:banditcomparison}. If the source code of a PUT was maintained in Git, the latest version at the time of the experiment in the master (or main) branch was used for the build. The `+' sign in a version indicates that the used source code is not the official release version of the source code.}
  \renewcommand\tabularxcolumn[1]{m{#1}}
  \renewcommand{\arraystretch}{1.2}
  \begin{tabularx}{\textwidth}{lXllXc}
    \toprule
    Project & Version & Commit ID & PUT & Format of Initial Seeds & Initial Edge Coverage \\
    \midrule
    Bloaty & v1.1+ & 60209eb & fuzz\_target & Executable (e.g., ELF, PE, Mach-O) & 4773\\
    libmpeg2 & N/A & 5432dc1 & mpeg2\_dec\_fuzzer & MPEG2 & 2428 \\
    PHP & 8.0+ & fdf0455f & php-fuzz-execute & PHP source code & 25241 \\
    HarfBuzz & 3.1.0 & 77eeec5 & hb-shape-fuzzer & Font (e.g., TrueType, OpenType) & 15298 \\
    Xpdf & 4.03 & N/A & fuzz\_pdfload & PDF & 4755 \\
    libtpm2 & N/A & bc3bb26 & tpm2\_execute\_command\_fuzzer & TPM command & 3884\\
    libyaml & v0.2.5+ & f8f760f & libyaml\_dumper\_fuzzer & YAML & 1310 \\
    libzip & 1.8.0+ & bff2eb9 & zip\_read\_fuzzer & ZIP & 805 \\
    libgit2 & v1.3.0+ & 50b4d53 & download\_refs\_fuzzer & Git packet & 3911 \\
    file & 5.41+ & fcbb5d8 & magic\_fuzzer & any (e.g., Zstd compressed file) & 1171 \\
%    MuPDF & 1.19.0+ & ef3d68d & pdf\_fuzzer & PDF & 16936 \\
%    libxml2 & 2.9.12+ & dea91c9 & xml & XML & 7027 \\
    \bottomrule
  \end{tabularx}
\label{tab:put_details}
\end{table*}

%\section{Full Results of Some Experiments}
%\label{sec:full_result}

%Table~\ref{tab:alg_cmp_all}, Figure \ref{fig:vis_bandits} and Figure \ref{fig:full_ablation_time_vs_cov} show the omitted results.

\begin{table*}[htb]
\centering
\caption{Median edge coverage obtained by AFL++ and 8 versions of \OurMethodName-AFL++ in 10 PUTs after 24 h. }

\begin{tabular}{lccccccccc}
\toprule

PUT & AFL++ & UCB1 & KLUCB & TS & dTS & dBE & ADS-TS & EXP3-IX & EXP3++ \\
\midrule

bloaty & \textit{1845.5} & 2198.5 & 2246.0 & 2232.5 & 2191.0 & 2292.0 & \textbf{2340.0} & 2181.5 & 2231.5 \\
harfbuzz & \textit{13497.5} & 14031.5 & 14247.5 & 14360.5 & \textbf{14374.0} & 14067.5 & 14149.0 & 13883.0 & 13891.0 \\
xpdf & \textit{3384.0} & 3494.0 & 3812.5 & \textbf{4618.5} & 4166.5 & 3791.5 & 3902.0 & 3860.0 & 3615.0 \\
libzip & \textit{267.5} & 272.0 & 274.0 & 268.0 & 268.5 & 271.5 & \textbf{276.0} & 271.5 & 268.0 \\
libgit2 & 898.0 & 888.5 & 890.5 & 906.5 & \textbf{916.0} & 884.0 & 914.0 & 899.5 & \textit{881.0} \\
php & \textit{9841.5} & 11861.0 & 13551.5 & \textbf{14324.0} & 14187.5 & 12657.5 & 13408.0 & 11423.5 & 11828.5 \\
libmpeg2 & \textit{1873.5} & 1900.5 & 1905.0 & 1905.5 & \textbf{1906.5} & 1903.0 & \textbf{1906.5} & 1897.0 & 1902.0 \\
tpm2 & \textit{281.5} & 299.5 & 313.0 & 317.0 & \textbf{317.5} & 305.0 & 311.0 & 298.5 & 291.0 \\
libyaml & 2811.5 & 2841.0 & \textbf{2841.5} & \textit{2800.5} & 2837.0 & 2827.5 & 2831.5 & 2828.0 & 2834.5 \\
file & 830.5 & 829.5 & 828.0 & 827.0 & 827.5 & 833.5 & \textbf{840.5} & 826.5 & \textit{826.0} \\

\bottomrule

\end{tabular}

\label{tab:alg_cmp_all}
\end{table*}

\begin{table*}[htb]
\centering
\caption{P-value of Mann-Whitney's U test (Holm-Bonferroni corrected) and Vargha-Delaney's $\hat{A}_{12}$ between AFL++ and the fuzzer in the column for the evaluation conducted in Section~\ref{subsec:eval-vs-existing}. If the p-value is bold, the difference is significant in the test ($p < 0.01$). The characters `L', `M', `S' and `N' in parentheses indicate that the effect size is large, medium, small, and none, respectively, according to \cite{A12}. The `+' sign means the fuzzer in the column is superior to AFL++ when compared by rank sum as well as $\hat{A}_{12}$, and the `-' sign means the opposite.}
\begin{tabular}{lllllllllllll}
 \toprule

  & \multicolumn{2}{c}{MOpt} & \multicolumn{2}{c}{CMFuzz} & \multicolumn{2}{c}{Karamcheti} & \multicolumn{2}{c}{\HavocMAB{}} & \multicolumn{2}{c}{SLOPT} \\
  \cmidrule(r){2-3}\cmidrule(r){4-5}\cmidrule(r){6-7} \cmidrule(r){8-9} \cmidrule(r){10-11}
  PUT & $p$ & $\hat{A}_{12}$ & $p$ & $\hat{A}_{12}$ & $p$ & $\hat{A}_{12}$ & $p$ & $\hat{A}_{12}$ & $p$ & $\hat{A}_{12}$ \\
\midrule

openssl\_x509 & \textbf{ < 0.001 } & 0.82 (+L) & \textbf{ 0.023 } & 0.71 (+L) & \textbf{ < 0.001 } & 0.92 (+L) & \textbf{ < 0.001 } & 0.82 (+L) & \textbf{ < 0.001 } & 0.91 (+L) \\
re2-2014-12-09 & \textbf{ < 0.001 } & 0.18 (-L) & > 0.1 & 0.37 (-S) & > 0.1 & 0.38 (-S) & > 0.1 & 0.47 (-N) & > 0.1 & 0.52 (+N) \\
proj4-2017-08-14 & \textbf{ < 0.001 } & 0.08 (-L) & \textbf{ < 0.001 } & 0.86 (+L) & \textbf{ < 0.001 } & 0.99 (+L) & > 0.1 & 0.54 (+N) & \textbf{ < 0.001 } & 0.92 (+L) \\
sqlite3\_ossfuzz & > 0.1 & 0.55 (+N) & \textbf{ < 0.001 } & 0.85 (+L) & \textbf{ < 0.001 } & 0.93 (+L) & 0.1 & 0.68 (+M) & \textbf{ < 0.001 } & 1.00 (+L) \\
libxml2-v2.9.2 & \textbf{ < 0.001 } & 0.08 (-L) & \textbf{ < 0.001 } & 0.93 (+L) & \textbf{ < 0.001 } & 0.98 (+L) & \textbf{ < 0.001 } & 0.97 (+L) & \textbf{ < 0.001 } & 0.84 (+L) \\
freetype2-2017 & \textbf{ < 0.001 } & 0.08 (-L) & 0.094 & 0.33 (-M) & > 0.1 & 0.54 (+N) & > 0.1 & 0.52 (+N) & \textbf{ < 0.001 } & 0.79 (+L) \\
libpcap\_fuzz\_both & > 0.1 & 0.57 (+S) & \textbf{ < 0.001 } & 0.79 (+L) & \textbf{ < 0.001 } & 0.80 (+L) & \textbf{ < 0.001 } & 0.87 (+L) & \textbf{ < 0.001 } & 0.81 (+L) \\
libpng-1.2.56 & > 0.1 & 0.42 (-S) & > 0.1 & 0.36 (-M) & > 0.1 & 0.49 (-N) & > 0.1 & 0.56 (+S) & 0.049 & 0.68 (+M) \\
lcms-2017-03-21 & > 0.1 & 0.45 (-N) & \textbf{ 0.037 } & 0.70 (+M) & \textbf{ < 0.001 } & 0.85 (+L) & > 0.1 & 0.37 (-S) & \textbf{ < 0.001 } & 0.88 (+L) \\
vorbis-2017-12-11 & > 0.1 & 0.39 (-S) & > 0.1 & 0.56 (+S) & \textbf{ < 0.001 } & 0.20 (-L) & > 0.1 & 0.62 (+S) & 0.092 & 0.65 (+M) \\

\bottomrule
\end{tabular}
\label{tab:statistics}
\end{table*}

\clearpage

\section{Algorithm Overview}

\begin{algorithm}[H]

\centering
\caption{Pseudocode of \OurMethodName{}}
\label{alg:slopt}

\begin{algorithmic}[0]

\Require{\mbox{}\\
    $initial\_seeds$ -- a set of initial test cases \\
    $program$ -- a PUT to be fuzzed
}

\Ensure{\mbox{}\\
    $queue$ -- a set of valuable test cases \\
    $crashes$ -- a set of test cases that trigger crashes
}

%\begin{adjustwidth}{-9pt}{}
%\setstretch{0.85}
\vspace{5pt}

\Function{RandomMutation}{$seed, instance_{mut}, instances_{bat}$}
\State $input$ $\gets$ \Call{CopyBytesFromSeed}{$seed$}
\State $mutation$ $\gets$ \Call{SelectArm}{$instance_{mut}$}
\State $idx$ $\gets$ \Call{GetGroupIndex}{$len(input)$}
\State $batch\_size$ $\gets$ \Call{SelectArm}{$instances_{bat}[idx][mutation]$}
\For{$i$ $\gets$ $1$ \textbf{to} $batch\_size$}
    \State $pos$ $\gets$ \Call{SelectPosition}{$input$}
    \State $input$ $\gets$ \Call{ApplyOperator}{$mutation, input, pos$}
\EndFor
\State \textbf{return} $input, mutation, batch\_size$
\EndFunction

%\end{adjustwidth}

%\vspace{-6pt}

%\begin{adjustwidth}{-9pt}{}
%\setstretch{0.85}

\vspace{5pt}

\Function{MutationFuzzing}{$initial\_seeds, program$}

\State $crashes$ $\gets$ $\varnothing$
\State $queue$ $\gets$ \Call{ConstructQueue}{$initial\_seeds$}
\State $instance_{mut}$ $\gets$ \Call{CreateBanditArms}{$number\_of\_mutations$}
\For{$i$ $\gets$ $1$ \textbf{to} $5$}
 \For{$j$ $\gets$ $1$ \textbf{to} $number\_of\_mutations$}
  \State $instances_{bat}[i][j]$ $\gets$ \Call{CreateBanditInstance}{$7$}
 \EndFor
\EndFor

\State

\While{ $\neg$ \Call{UserWantsStop}{\null}}
 \State $seed$ $\gets$ \Call{SelectSeed}{$queue$}
 \State $energy$ $\gets$ \Call{DecideEnergy}{$seed$}
 \For{$i$ $\gets$ $1$ \textbf{to} $energy$}
  \State $input, mutation, batch\_size$ 
  \State $\gets$ \Call{RandomMutation}{$seed, instance_{mut}, instances_{bat}$}
  \State $result$ $\gets$ \Call{ExecutePUT}{$program, input$}
  \State $b$ $\gets$ \Call{WasInputValuable}{$result$}
  \State \Call{RewardArm}{$mutation, b$}
  \State \Call{RewardArm}{$batch\_size, b$}
  \State \Call{SaveInputIfValuable}{$queue, input, result$}
  \State \Call{SaveInputIfCrash}{$crashes, input, result$}
 \EndFor
\EndWhile
\EndFunction

%\end{adjustwidth}

\end{algorithmic}
\end{algorithm}



\end{document}
https://www.overleaf.com/project/60ea738089ab3ecf5cbfe659