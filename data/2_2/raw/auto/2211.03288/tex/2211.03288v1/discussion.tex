\section{Discussion}
\label{sec:discussion}

In this paper, we limit the discussion to open-boundary surface codes. Nevertheless, we believe the same interpretation works on other surface codes as well with proper generalization of notions.
For example, ``attached cluster'' can be generalized to a subgraph whose edges constitute a nontrivial logical operator.
The relationship between UF decoders and the blossom algorithm also suggests that UF decoders can be adapted to solve any decoding problem that can be solved by a MWPM of the syndrome graph, e.g., the color code ~\cite{sahay2022decoder}.

The revealed relationship between UF decoder and the blossom algorithm further suggests ways of cross-pollination between the blossom algorithm and UF decoders.
We have already showed that it can lead to new UF decoder designs in \S\ref{sec:weighted_uf}. 
One can borrow more ideas from the blossom algorithm to make UF decoders better. 
For example, one can keep the internal structures of small clusters so that they grow like the \clusters in the blossom algorithm.
On the other hand, one could also bring ideas from UF decoders into the blossom algorithm to improve the latter's speed.
For example, we have recently shown that instead of the syndrome graph, MWPM decoders can be made faster~\cite{fusion-blossom} by adopting the decoding graph used by UF decoders, which has also been independently discovered by Higgott and Gidney~\cite{pymatchingv2}.
