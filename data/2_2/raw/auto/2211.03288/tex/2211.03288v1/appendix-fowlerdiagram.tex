\section{Examples using the \diagram}

\begin{figure*}[ht]
    \renewcommand*\thesubfigure{(\arabic{subfigure})}  % use 1,2,3... instead of a,b,c... to distinguish between names of vertices
	\centering
	\begin{subfigure}{.24\linewidth}
	    \centering
        \includegraphics[width=0.8\linewidth]{figures/exploratory_region_drawing_manhattan_2/blossom_multi_grow/blossom_multi_grow_0.pdf}
        \caption{$\sum y = 0$}
        \label{fig:blossom-multi-grow-0}
    \end{subfigure}
	\begin{subfigure}{.24\linewidth}
	    \centering
        \includegraphics[width=0.8\linewidth]{figures/exploratory_region_drawing_manhattan_2/blossom_multi_grow/blossom_multi_grow_0.9.pdf}
        \caption{$\sum y = 4.5$}
        \label{fig:blossom-multi-grow-0.9}
    \end{subfigure}
	\begin{subfigure}{.24\linewidth}
	    \centering
        \includegraphics[width=0.8\linewidth]{figures/exploratory_region_drawing_manhattan_2/blossom_multi_grow/blossom_multi_grow_1.4.pdf}
        \caption{$\sum y = 6.2$}
        \label{fig:blossom-multi-grow-1.4}
    \end{subfigure}
	\begin{subfigure}{.24\linewidth}
	    \centering
        \includegraphics[width=0.8\linewidth]{figures/exploratory_region_drawing_manhattan_2/blossom_multi_grow/blossom_multi_grow_2.3.pdf}
        \caption{$\sum y = 7.3$}
        \label{fig:blossom-multi-grow-2.3}
    \end{subfigure}
	\begin{subfigure}{.24\linewidth}
	    \centering
        \includegraphics[width=0.8\linewidth]{figures/exploratory_region_drawing_manhattan_2/blossom_multi_grow/blossom_multi_grow_2.6.pdf}
        \caption{$\sum y = 7.6$}
        \label{fig:blossom-multi-grow-2.6}
    \end{subfigure}
	\begin{subfigure}{.24\linewidth}
	    \centering
        \includegraphics[width=0.8\linewidth]{figures/exploratory_region_drawing_manhattan_2/blossom_multi_grow/blossom_multi_grow_2.9.pdf}
        \caption{$\sum y = 7.9$}
        \label{fig:blossom-multi-grow-2.9}
    \end{subfigure}
	\begin{subfigure}{.24\linewidth}
	    \centering
        \includegraphics[width=0.8\linewidth]{figures/exploratory_region_drawing_manhattan_2/blossom_multi_grow/blossom_multi_grow_3.3.pdf}
        \caption{$\sum y = 8.3$}
        \label{fig:blossom-multi-grow-3.3}
    \end{subfigure}
	\begin{subfigure}{.24\linewidth}
	    \centering
        \includegraphics[width=0.8\linewidth]{figures/exploratory_region_drawing_manhattan_2/blossom_multi_grow/blossom_multi_grow_4.pdf}
        \caption{$\sum y = 9$}
        \label{fig:blossom-multi-grow-4}
    \end{subfigure}
	\caption{The blossom algorithm with multiple tree approach~\cite{kolmogorov2009blossom}. The primal variables are drawn as blue lines connecting vertices. The dual variables are drawn as the ``radius'' of the colored regions. When two regions touch, a tight edge is formed between two vertices from each region. (1) The initial state with all dual variables initialized to 0. (2) The dual variable of each vertex grows simultaneously. (3) When the regions of vertices $A$, $B$, $C$ touch each other, they form a \textit{blossom} marked in dotted blue lines. The dual variable of this blossom $y_{\{A,B,C\}}$ (green) grows but individual dual variables $y_A$, $y_B$ and $y_C$ stop growing. (4) When the regions of vertices $D$ and $E$ touch each other, they form a solved \cluster marked in solid blue line. The dual variables $y_D$, $y_E$ stop growing. (5) When the blossom $\{A,B,C\}$ touches the solved cluster $\{D,E\}$, it forms an alternating tree marked in dashed and solid blue lines. (6) In this alternating tree, $y_{\{A,B,C\}}$ and $y_E$ grows as usual, but $y_D$ shrinks at the same speed. In this way, they still keep touch with each other. (7) A bigger blossom is formed and its dual variable $y_{\{A,B,C,D,E\}}$ (purple) starts growing. Individual dual variables $y_{\{A,B,C\}}$, $y_D$ and $y_E$ stop growing. (8) This blossom touches a left virtual boundary vertex and breaks into three temporary matches: $A$ matches to the left virtual boundary vertex, $B$ matches to $C$, $D$ matches to $E$. The cluster is now solved, so the blossom algorithm terminates. The minimum-weight perfect matching is the collection of all those matchings with only tight edges.}
	\label{fig:blossom_multi_grow}
\end{figure*}


\begin{figure*}[ht]
    \renewcommand*\thesubfigure{(\arabic{subfigure})}  
    	\centering
	\begin{subfigure}{.24\linewidth}
	    \centering
        \includegraphics[width=0.8\linewidth]{figures/exploratory_region_drawing_manhattan_2/union_find_multi_grow/union_find_multi_grow_0.pdf}
        \caption{}
        \label{fig:union_find_multi_grow_0}
    \end{subfigure}
	\begin{subfigure}{.24\linewidth}
	    \centering
        \includegraphics[width=0.8\linewidth]{figures/exploratory_region_drawing_manhattan_2/union_find_multi_grow/union_find_multi_grow_0.9.pdf}
        \caption{}
        \label{fig:union_find_multi_grow_0.9}
    \end{subfigure}
	\begin{subfigure}{.24\linewidth}
	    \centering
        \includegraphics[width=0.8\linewidth]{figures/exploratory_region_drawing_manhattan_2/union_find_multi_grow/union_find_multi_grow_1.4.pdf}
        \caption{}
        \label{fig:union_find_multi_grow_1.4}
    \end{subfigure}
	\begin{subfigure}{.24\linewidth}
	    \centering
        \includegraphics[width=0.8\linewidth]{figures/exploratory_region_drawing_manhattan_2/union_find_multi_grow/union_find_multi_grow_2.3.pdf}
        \caption{}
        \label{fig:union_find_multi_grow_2.3}
    \end{subfigure}
	\begin{subfigure}{.24\linewidth}
	    \centering
        \includegraphics[width=0.8\linewidth]{figures/exploratory_region_drawing_manhattan_2/union_find_multi_grow/union_find_multi_grow_2.6.pdf}
        \caption{}
        \label{fig:union_find_multi_grow_2.6}
    \end{subfigure}
	\begin{subfigure}{.24\linewidth}
	    \centering
        \includegraphics[width=0.8\linewidth]{figures/exploratory_region_drawing_manhattan_2/union_find_multi_grow/union_find_multi_grow_2.9.pdf}
        \caption{}
        \label{fig:union_find_multi_grow_2.9}
    \end{subfigure}
	\begin{subfigure}{.24\linewidth}
	    \centering
        \includegraphics[width=0.8\linewidth]{figures/exploratory_region_drawing_manhattan_2/union_find_multi_grow/union_find_multi_grow_3.3.pdf}
        \caption{}
        \label{fig:union_find_multi_grow_3.3}
    \end{subfigure}
	\begin{subfigure}{.24\linewidth}
	    \centering
        \includegraphics[width=0.8\linewidth]{figures/exploratory_region_drawing_manhattan_2/union_find_multi_grow/union_find_multi_grow_4.pdf}
        \caption{}
        \label{fig:union_find_multi_grow_4}
    \end{subfigure}
	\caption{The union-find decoder. A \textit{grown edge} is fully covered by regions (yellow). A \textit{half-grown edge} is partly covered by regions. An \textit{unoccupied edge} is not covered by any region. (1) Initially all edges are unoccupied. (2) Each vertex is an \textit{odd cluster} and grows uniformly. Odd (even) cluster consists of odd (even) number of vertices. (3) Clusters merge together. The new cluster $\{A,B,C\}$ is still an odd cluster and keeps growing. (4) Two odd clusters $D$ and $E$ merge into an \textit{even cluster} and stop growing. (5) The odd cluster $\{A,B,C\}$ merges with the even cluster $\{D,E\}$ and becomes a bigger odd cluster $\{A,B,C,D,E\}$. This bigger cluster grows uniformly even though the cluster $\{D,E\}$ has been stopped for a while. (6)(7)(8) The odd cluster $\{A,B,C,D,E\}$ keeps growing until it touches a left virtual boundary vertex and terminates. After all clusters stop growing, the union-find decoder applies the \textit{peeling algorithm} to find an error pattern that generates this syndrome using single-qubit errors only inside each cluster. This error pattern corresponds to a perfect matching marked in blue lines in (8), though generally not a minimum-weight perfect matching. Note that each sub-figure corresponds to one in \autoref{fig:blossom_multi_grow}, with similar shape of  \regions.}
	\label{fig:union_find_multi_grow}
\end{figure*}
