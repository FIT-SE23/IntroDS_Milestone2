\subsection{Syndrome Graph}
\label{sec:theory_syndromegraph}

Given the surface code, represented by the model graph $G(U,F)$, and its measurement, we can construct the syndrome graph $G(V,E)$ as follows. $V\subseteq U$ includes all the model graph vertices that have nontrivial measurement outcomes; $E=\{e=\langle u,v\rangle| \forall u, v\in V\}$. For $e=\langle u,v\rangle\in E$, the weight $w_e$ is computed as the weight of the minimum-weight path between $u$ and $v$ in the model graph. 

Because there may be multiple minimum-weight paths between $u$ and $v$, $e=\langle u,v\rangle$ may represent multiple error patterns, which is denoted by the set $\mathbf{e}$. $\forall\E\in\mathbf{e}$, it can be considered a collection of single qubit errors, corresponding to the model graph edges in a minimum-weight path $\mathcal{P}$. That is, $\E=\sum_{f\in \mathcal{P}} f$.

\vspace{2ex}\noindent \textbf{Lemma (O)}\label{lemma:pm0} $\forall e\in E$, $\mathcal{E}_1$, $\mathcal{E}_2 \in \mathbf{e}$, $\mathcal{E}_1+\mathcal{E}_2$ is a trivial logical operator.

\begin{proof2}
\suffices{ $S(\E_1+\E_2)=0$ and $P(\E_1+\E_2)=0$.}
    \begin{proof2}
    \pf\ By definition of trivial logical operator. 
    \end{proof2}
\step{1} {$\E_1$ and $\E_2$ are paths connecting the same two vertices $u$ and $v$ in the model graph.} 
    \begin{proof2}
    \pf\ By definition of $e$.
    \end{proof2}
\step{2}{$S(\E_1+\E_2)=0$}
    \begin{proof2}
    \step{2.1}{$S(\E_1)=u+v$; $S(\E_2)=u+v$}
    \step{2.2}{$S(\E_1+\E_2)=S(\E_1)+S(\E_2)$\\$= (u+u)+(v+v)=0$;}
    \qedstep
    \end{proof2}  
\step{3}{$P(\E_1+\E_2)=0$}
    \begin{proof2}
    \step{3.1}{$P(\E_1)=V_L(u)+V_L(v)$,\\ $P(\E_2)=V_L(u)+V_L(v)$}
    \step{3.2}{$P(\E_1+\E_2)=P(\E_1)+P(\E_2)=\\ (V_L(u)+V_L(u))+(V_L(v)+V_L(v))=0$}
    \qedstep
    \end{proof2}
\qedstep
\end{proof2}


\paragraph{Subgraph}
A subgraph of the syndrome graph $G(V,E)$ is defined by a subset of $E$, $E'\subseteq E$.
$E'$ defines a set of error patterns $\mathbf{E}'$.
\begin{center}
    $\mathbf{E}'=\{\E|\E=\sum_{e\in E'} \E_e,\forall \E_e\in\mathbf{e}\}$
\end{center}
Therefore, $\forall \E\in \mathbf{E}'$, $\exists \E_e\in \mathbf{e}$ for $\forall e\in E'$ such that $\E=\sum_{e\in E'} \E_e$. 
That is, an error pattern represented by the subgraph can be ``decomposed'' into error patterns represented by its edges. 

Using the familiar sum form, we can represent $E'$ as $\sum_{e\in E'}e$.
 Two subgraphs can be ``added'' together to form a new one with $e+e=0$. The symmetric difference between two subgraphs $E_1$ and $E_2$ can be simplified as $E_1\triangle E_2 = E_1+E_2$.
