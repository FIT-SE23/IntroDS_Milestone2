\section{Conclusion and Limitations}
We present a learning-based method for inverse rendering of complex indoor scenes. Our approach handles spatially-varying illumination and faithfully recovers specular reflections thanks to the differentiable Monte Carlo rendering layer, enabling photorealistic editing such as complex object insertion and material change. Lastly, we introduce a large-scale indoor dataset, \textsc{InteriorVerse}, which contains much richer details than existing alternatives. 

There are some limitations of our method. Our out-of-view lighting network is not capable of predicting high-frequency details due to its limited network capacity. Monte Carlo sampling would also lead to noisy re-render results, and raising the required sample budget can be computationally expensive. Further, emission of light sources is not supported currently, which we leave as future work.
