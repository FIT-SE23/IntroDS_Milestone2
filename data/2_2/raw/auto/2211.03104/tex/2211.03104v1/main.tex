\documentclass[a4paper,10pt]{amsart}
%\usepackage[foot]{amsaddr}
\usepackage{amsthm,amsmath,amssymb,amsfonts}
\usepackage[T1]{fontenc}
\usepackage[utf8]{inputenc}
\usepackage{verbatim}
\usepackage{marvosym}
\usepackage{stackrel}
\usepackage{graphicx}
\usepackage{eucal}
\usepackage{bbold}
\usepackage[svgnames]{xcolor}
\definecolor{blue(munsell)}{rgb}{0.0, 0.5, 0.69}
\usepackage{lipsum}
\usepackage{svg}
\usepackage{bbm}
\usepackage{caption}
\usepackage{enumitem}
\usepackage{mathtools}
\usepackage{epigraph}
\usepackage{color}
\usepackage{microtype}
%\usepackage{showkeys}
\usepackage{relsize}
\usepackage{amsfonts}
\usepackage{adjustbox}
\usepackage{subfig}
\usepackage{framed}
\usepackage{hyperref}
\hypersetup{
    colorlinks = true,
    linkbordercolor = {red},
   	linkcolor ={blue},
	anchorcolor = {pink},
	citecolor =  {red},
	filecolor = {blue},
	menucolor = {blue},
	runcolor =  {blue},
	urlcolor = {blue},
}
\usepackage{cleveref}
\usepackage{trimclip}


\DeclareFontFamily{U}{min}{}
\DeclareFontShape{U}{min}{m}{n}{<-> udmj30}{}
\newcommand{\yo}{\!\text{\usefont{U}{min}{m}{n}\symbol{'210}}\!}



\setcounter{tocdepth}{1}
\usepackage{stmaryrd}
\usepackage{tikz}
\usepackage{tikz-cd}
\usepackage{quiver}
%\usepackage{showlabels}



%\DeclareUnicodeCharacter{3088}{\yo}


\theoremstyle{definition}
\newtheorem{thm}{Theorem}[subsection]
\newtheorem*{thm*}{Theorem}
\newtheorem{prop}[thm]{Proposition}
\newtheorem*{prop*}{Proposition}
\newtheorem{lem}[thm]{Lemma}
\newtheorem{cor}[thm]{Corollary}
\newtheorem{defn}[thm]{Definition}
\newtheorem{war}[thm]{Warning}
\newtheorem*{war*}{Warning}
\newtheorem{rem}[thm]{Remark}
\newtheorem{constr}[thm]{Construction}
\newtheorem{exa}[thm]{Example}
\newtheorem{conj}[thm]{Conjecture}
\newtheorem{notat}[thm]{Notation}
\newtheorem{disc}[thm]{}
\newtheorem{fact}[thm]{Fact}
\newtheorem{claim}[thm]{Claim}

% two headed xrightarrow
\newcommand{\rightarrowdbl}{\rightarrow\mathrel{\mkern-14mu}\rightarrow}
\newcommand{\xrightarrowdbl}[2][]{%
  \xrightarrow[#1]{#2}\mathrel{\mkern-14mu}\rightarrow
}

\newcommand{\Ebb}{\mathbb{E}}
\newcommand{\Fbb}{\mathbb{F}}
\newcommand{\Gbb}{\mathbb{G}}
\newcommand{\Tbb}{\mathbb{T}}
\newcommand{\Ccal}{\mathcal{C}}
\newcommand{\Ecal}{\mathcal{E}}
\newcommand{\Fcal}{\mathcal{F}}
\newcommand{\Gcal}{\mathcal{G}}
\newcommand{\Hcal}{\mathcal{H}}
\newcommand{\Ocal}{\mathcal{O}}
\newcommand{\Scal}{\mathcal{S}}
\newcommand{\Set}{\mathrm{Set}}
\newcommand{\op}{{}^{\mathrm{op}}}
\DeclareMathOperator{\id}{id}
\DeclareMathOperator{\costr}{coString}
\newcommand{\joinarrows}{\mathrel{\mspace{-14mu}}}
\newcommand{\Tworightarrow}{\Rightarrow\joinarrows\Rightarrow}
\newcommand{\Hookrightarrow}{\hspace{3pt}{\clipbox{0pt 0pt 8.8pt 0pt}{$\rightarrowtail$}}\mspace{-8mu}\Rightarrow}
\newcommand{\arrE}{\stackrel{\to}{\mathcal{E}}}
\newcommand{\monoE}{\stackrel{\hookrightarrow}{\mathcal{E}}}
\newcommand{\epiE}{\stackrel{\twoheadrightarrow}{\mathcal{E}}}

\title{The geometry of Coherent topoi and Ultrastructures}

\author{Ivan Di Liberti}


\address{
Ivan \textsc{Di Liberti}: \newline
Department of Mathematics\newline
Stockholm University\newline
Stockholm, Sweden\newline
\href{mailto:diliberti.math@gmail.com}{\sf diliberti.math@gmail.com}
}
\thanks{The author was supported by the Swedish Research Council (SRC, Vetenskapsrådet) under Grant No.~2019-04545. The research has received funding from Knut and Alice Wallenbergs Foundation through the Foundation's program for mathematics.}


\begin{document}




\maketitle

\begin{abstract}
We show that coherent topoi are right Kan injective with respect to flat embeddings of topoi. We recover the ultrastructure on their category of points as a consequence of this result. We speculate on possible notions of ultracategory in various arenas of formal model theory.
 
 % \smallskip \noindent \textbf{Keywords.} formal model theory, topos theory, coherent topos, Kan injectivity, flat geometric morphism, ultracategory, ultrastructure, ultraproduct, accessible category, ionad. \textbf{MSC2020.} Numbers.
\end{abstract}

   {
   \hypersetup{linkcolor=black}
   \tableofcontents
   }


\section*{Introduction}

One of the leitmotivs of topos theory is that we can use geometric intuition as a guiding principle to devise correct definitions. The motivation for this paper is to clarify the notion of ultrastructure and ultracategory via a geometric approach. Ultrastructures were defined by Makkai \cite{makkai1987stone} to condense the main properties of the category of models of a first order theory. He successfully used this technology to provide a reconstruction theorem for first order logic that goes under the name of \textit{conceptual completeness}.

\begin{thm*}[Conceptual completeness]
Let $f: \mathcal{F} \to \mathcal{G}$ be a morphisms of pretopoi. If the induced functor between categories of models is an equivalence of categories, then $f$ is an equivalence too, \[f^*: \mathsf{Mod}(\mathcal{G}) \to \mathsf{Mod}(\mathcal{F}).\] 
\end{thm*}

This result falls under the umbrella of Stone-like dualities, as many others celebrated results like Gabriel--Ulmer duality, Gabriel--Rosenberg reconstruction theorem, Isbell duality, Tannaka (and Morita) reconstruction and many others. The general idea behind an ultrastructure is to account for the construction of ultraproducts of models. Indeed, given a coherent/first order theory $\mathbb{T}$, a set $X$, and an ultrafilter $U \in \beta(X)$, one can define the functor, \[\int_X (-)dU: \mathsf{Mod}(\mathbb{T})^X \to \mathsf{Mod}(\mathbb{T}).\] 

Such functor takes an $X$-indexed family of models and computes the ultraproduct of those along the ultrafilter $U$. 

Despite delivering a very satisfactory theorem, the notion of ultrastructure seemed quite complicated -- if not ad-hoc -- and it was never clear whether it was the correct or definitive notion. The first attempt of providing a more conceptual framing to understand ultrastructures and ultracategories was given by Marmolejo in his PhD thesis \cite{marmolejo1995ultraproducts}. He was probably the first to trace some geometric aspects of the construction of ultraproducts, introducing the notion of \textit{Łoś category}. We will see that the blueprint of our approach is essentially the same of Marmolejo's, even though we have a quite different way of encoding the same idea. 

In more recent years Lurie \cite{lurieultracategories} revisited the notion of ultracategory, proposing a morally similar, but technically different notion of ultrastructure, ultracategory and ultrafunctor with respect to Makkai's one. Both Makkai's and Lurie's notions are justified by the fact that they manage to deliver the most compelling theorems of this theory. Yet, none of these notions appears \textit{definitive} when read or encountered for the first time for several reasons. The main one being that the definition of ultracategory is in both cases very heavy, and comes together with axioms whose choice seems quite arbitrary; and indeed the two authors make different choices.

\textbf{In this paper} we study the geometric properties of coherent topoi with respect to \textit{flat embeddings}, and we let the notion of ultrastructure emerge naturally from general considerations on the topology of flat embeddings. Our aim is to provide notions of ultrastructure and ultracategory that seem correct when instantiated in the appropriate formal model theory. 

For accessible categories with directed colimits, we show how Lurie's definitions seem a definitive choice. To do so we recover the ultrastructure of a coherent topos from its geometric behavior with respect to flat maps. We do not restrain our attention to accessible categories with directed colimits. Inspired by both Marmolejo and Lurie's treatment of ultracategories via topological stacks/$\mathsf{Sp}$\footnote{In this paper $\mathsf{Sp}$ is the usual category of topological spaces. We prefer this notation to the usual $\mathsf{Top}$ to avoid any collision/clash/misunderstanding with the $2$-category of Topoi.}-indexed categories, we introduce and study the notion of \textit{accessible profile}. In this framework we study ultraprofiles. Finally, in the realm of bounded ionads \cite{ionads,thgeo,thlo}, we study the new notion of ultraionad. The paper opens many more doors than it closes, and thus we finish it by a section of open problems and possible further directions.


%   {
%   \hypersetup{linkcolor=black}
%   \tableofcontents
%   }


\subsection*{Main results and structure of the paper}

\Cref{sec1} is entirely geometric. We start by defining flat morphisms of topoi and in a sense represents a topos theoretic reinterpretation of some seminal papers by Escardo \cite{escardo1997injective,escardo1998properly,escardo1999semantic}. The notion is inspired by an old result of Joyal \cite[III.1.11]{johnstone1986stone} which proved to be crucial in the theory of coherent locales. Next, we briefly recall the theory of (right) Kan injectivity from \cite{dlsolo}. In a nutshell, this amounts to a variation of the classical theory of injectivity where a $2$-dimensional aspect is kept into account. The main result of the section is the following proposition,

\begin{thm*}[\ref{mainth}]
Coherent topoi are right Kan injective with respect to flat geometric embeddings.
\end{thm*}

Of course, at this point it is natural to ask whether all topoi which are right Kan injective with respect to flat embeddings are (retracts of) coherent (ones), and we discuss this in \Cref{characterisation}. We have no definitive answer to this question, besides the proposition below. 


\begin{prop*}[\ref{char2}]
Let $\mathcal{E}$ be a topos which is right Kan injective with respect to flat embeddings. Then $\mathcal{E}$ admits a geometric surjection from a spatial coherent topos. (In particular it has enough points).
\end{prop*}



At the end of the section we introduce the notion of \textit{spartan} geometric morphism, these morphisms are right Kan injective with respect to flat embeddings (\Cref{moderatesarekan}), so that we obtain an inclusion from the $2$-category of coherent topoi and spartan morphisms into the $2$-category of right Kan injectives with respect to flat morphisms,
\[\mathsf{CohTopoi}_\mathcal{S} \hookrightarrow \mathsf{RInj}(\mathsf{Emb}_\flat).\]

%It is worth noticing that coherent geometric morphisms are moderate. Throughout the section we point out which ingredients will play a crucial role in the construction of ultraproducts, building on some ideas appearing in both Marmolejo \cite{marmolejo1995ultraproducts} and Moerdjik and Mac Lane \cite[IX.11.1]{sheavesingeometry}.

In \Cref{sec2} we concentrate on a special class of flat embeddings of topoi. For $X$ a set, call $i_X : X \to \beta(X)$ the inclusion of $X$ in its space of ultrafilters. Then, the induced geometric morphisms,
\[i_X: \mathsf{Set}^X \to \mathsf{Sh}(\beta(X))\] is a flat embedding of topoi. We collect these morphisms in a class $\mathsf{Emb}_\beta$ and we call \textit{$\beta$-complete} a topos which is right Kan injective with respect to those. The main point of the section is to convince the reader that when a topos is $\beta$-complete, its category of points $\mathsf{pt}(\mathcal{E})$ comes equipped with a canonical ultrastruture. 

Being right Kan injective has both an existence part (\Cref{emergentultrastructures})  and universality part (\Cref{2dimensional}) to it. These two aspects mark the category of points of the topos and must be accounted as the two main ingredients for our notion of ultrastruture, 


    \[  \int_X(-)d(-): \mathsf{pt}(\mathcal{E})^X \times \beta(X) \to \mathsf{pt}(\mathcal{E}). \]


\Cref{sec3} starts with a brief digression on formal model theory (\cite{thlo}), that is the study of $2$-categories whose objects look like categories of models of first order theories. We focus on accessible categories with directed colimits $(\mathsf{Acc}_\omega)$, accessible profiles $(\mathsf{Acc}_\omega^\mathsf{Sp})$ and bounded ionads $(\mathsf{BIon})$ as arenas of formal model theory.

\subsubsection*{$(\mathsf{Acc}_\omega)$}
For accessible categories with directed colimits,  the most classically developed approach to the topic (\cite{Makkaipare} and \cite{adamek_rosicky_1994}), we give a definition of ultracategory \Cref{def:ultracategory} which simplifies and conceptualizes Lurie's notion. 

\subsubsection*{$(\mathsf{Acc}_\omega^\mathsf{Sp} \text{ and }\mathsf{BIon})$}
 We introduce accessible profiles, these are $\mathsf{Sp}$-indexed accessible categories with directed colimits $\mathcal{A}^\bullet: \mathsf{Sp}^\circ \to \mathsf{Acc}_\omega$, and are motivated by the reflections of Marmolejo on ultracategories and some of Lurie's proof techniques. We provide a general adjunction between topoi and accessible profiles (\Cref{newadjunction}) and we define the topos of coordinates $\Theta(\mathcal{A}^\bullet)$ for an accessible profile $\mathcal{A}^\bullet$ (\Cref{coordinates}). This offers a test-topos for reconstruction-like results (\Cref{reprensentation}). Accessible profiles offer a solid foundation for formal category theory. To exemplify that, we introduce the notion of ultraprofile and we provide the following result.
    
\begin{prop*}[\ref{ultaprofilebuono}]
Let $\mathcal{E}$ be a topos. $\mathcal{E}$ is $\beta$-complete if and only if its profile of points $\mathbb{pt}^\bullet(\mathcal{E})$ is an ultraprofile.
\end{prop*}

A similar treatment is given for Garner's bounded ionads \cite{ionads,thgeo,thlo}, for which we introduce the notion of ultraionad and we prove the following theorem.

\begin{prop*}[\ref{ionadsaregood}]
\begin{itemize}
\item[]
    \item A topos $\mathcal{E}$ is $\beta$-complete if and only if its ionad of points $\mathbb{pt}(\mathcal{E})$ is an ultraionad.
    \item A ionad $(\mathcal{A}, \mathsf{Int})$ is an ultraionad if and only if its topos of opens $\mathbb{O}(\mathcal{A})$ is a $\beta$-complete topos.
\end{itemize} 
\end{prop*}


\begin{war*}
Throughout the paper we intensely deploy both the general theory of Kan extensions in a $2$-category and the more specific theory of pointwise Kan extensions in $\mathsf{Cat}$. We refer to \cite[Chap. 6]{riehl2017category} as a general and solid reference for all the results we apply. The Appendix of  \cite{liberti2019codensity} can be used as a covenient cheat sheet for almost everything we need.
\end{war*} 
\section{Flat embeddings and coherent topoi} \label{sec1}

\subsection{Flat embeddings}

\begin{defn}
A geometric morphism $f : \mathcal{F} \to \mathcal{G}$ is \textit{flat} if its direct image $f_* : \mathcal{F} \to \mathcal{G}$ preserves finite colimits. 
\end{defn}

\begin{disc}
The notion of flat geometric morphism appeared for the first time in the theory of locales \cite[III.1.11]{johnstone1986stone}, and is apparently due to Joyal. It was later studied in few other papers by Johnstone \cite{johnstone1984wallman} and Isbell \cite{isbell1988flat}. Since the very beginning the notion was linked to that of coherent locale, and the connection has been made more and more precise in recent years \cite{escardo1999semantic, carvalho2017kan}. This paper benefited a lot from the geometric intuition provided by this literature on the theory of locales and we will soon revisit many of these classical results in a more topos-theoretic fashion.
\end{disc}

\begin{disc} \label{generalities2}
In topos theory, this notion only appeared twice. Once in the PhD thesis of Marmolejo and its subsequents papers \cite{marmolejo1995ultraproducts,marmolejo2005locale,marmolejo1998continuous}. There, flat morphisms go unnamed, even though they play a centrole role in the whole thesis. Marmolejo says that a continuous function between topological spaces is \textit{ultrafinite} if its associated morphism of localic topoi is flat in our sense. A geometric characterization of ultrafinite continuous functions can be found in \cite{marmolejo2005locale}. We will come back to Marmolejo's approach later in the paper as it is indeed essential for us. The other appearance of a flat geometric embedding in topos theory is in \cite[IX.10.4 and the discussion above]{sheavesingeometry}. While it is evident that the authors were informed of the localic interpretation of flatness, there is no contextualization in Moerdijk's and Mac Lane's work. All in all, this paper represents the first time the notion of flat geometric morphism \textit{of topoi} is formally introduced and studied.
\end{disc}

\begin{war}
We find the name \textit{flat} misleading, as it clashes with usual notion of flat functor which is unrelated -- at least to our knowledge -- and yet key in topos theory. We do not entirely understand the motivation that led to this choice even in the theory of locales but will stick to it both for the lack of a better alternative and to be consistent with the existing literature on the topic.
\end{war}

\begin{exa}[Every locale is a flat sublocale of a coherent one] \label{exampleindcompl}
Johnstone observes that every locale amits a flat geometric embedding into a coherent locale \cite[III.1.11]{johnstone1986stone}. This is given by the embedding of a locale $\mathcal{L}$ into its frame of ideals, \[i_{\mathcal{L}}: \mathcal{L} \hookrightarrow \mathsf{Idl}(\mathcal{L}).\]
\end{exa}

\begin{exa} \label{cameo}
Let us report on the example discussed by Moerdijk and Mac Lane from \cite[IX.10.4 and the discussion above]{sheavesingeometry}, both for the sake of discussing an example, and because it will turn out to be useful for our treatment. Let $B$ be a complete boolean algebra, being a locale we can apply the contruction above and embed it in its frame of ideals, $i_{\mathcal{B}}: \mathcal{B} \hookrightarrow \mathsf{Idl}(\mathcal{B}).$
In the special case of a complete boolean algebra, this is the same as the frame of opens of its Stone space, as Moerdijk and Mac Lane observe. Then, the induced geometric embedding of localic topoi is flat,

\[i_{\mathcal{B}}: \mathsf{Sh}(\mathcal{B}) \hookrightarrow \mathsf{Sh}(\mathsf{Idl}(\mathcal{B})).\]

Moerdijk and Mac Lane do not show the flatness of this geometric morphism as they only rely on the fact that $i_*$ preserve finite epimorphic families. The only available proof is due to \cite[Cor. 2.3]{marmolejo2005locale}. An essentially identical proof appears in \cite{marmolejo1995ultraproducts}, but it is much harder to read.
\end{exa}

Moerdijk and Mac Lane use this observation to show that every coherent topos has enough points, implementing a variation of Joyal's lemma which started this theory. We will generalise their result in a couple of subsections, but we shall start from recalling a polished version of Joyal's lemma and comment it for the sake of readability.

\begin{thm} \label{paradigm}
A coherent locale $\mathcal{C}$ is injective with respect to flat embeddings of locales.

% https://q.uiver.app/?q=WzAsMyxbMCwwLCJcXG1hdGhjYWx7TH0iXSxbMCwxLCJcXG1hdGhjYWx7TCd9Il0sWzEsMCwiXFxtYXRoY2Fse0N9Il0sWzAsMiwiZiJdLFswLDEsImkiLDIseyJzdHlsZSI6eyJ0YWlsIjp7Im5hbWUiOiJob29rIiwic2lkZSI6InRvcCJ9fX1dLFsxLDIsIiIsMix7InN0eWxlIjp7ImJvZHkiOnsibmFtZSI6ImRhc2hlZCJ9fX1dXQ==
\[\begin{tikzcd}
	{\mathcal{L}} & {\mathcal{C}} \\
	{\mathcal{L'}}
	\arrow["f", from=1-1, to=1-2]
	\arrow["i"', hook, from=1-1, to=2-1]
	\arrow[dashed, from=2-1, to=1-2]
\end{tikzcd}\]
\end{thm}

\begin{disc}
The theorem above is the most paradigmatic of this theory and shows clearly the connection between coherent locales and flat embeddings via injectivity. A more precise account on the topic which takes into account the $2$-dimensional aspects of the theory and connects it to Kan injectivity is provided in \cite{carvalho2017kan}. Because every locale admits a flat embedding into a coherent one, it follows that  stably locally compact locales (retracts of coherent locales \cite[C4.1]{Sketches}) are precisely the locales (right) Kan injective to flat embeddings (\cite[Rem. 4.3]{carvalho2017kan}),

\[\mathsf{SLC} = \mathsf{RInj(FlatEmb)}.\]
\end{disc}

\begin{disc}
Moerdijk and Mac Lane prove a topos-theoretic version of \Cref{paradigm}  (\cite[IX.11.1]{sheavesingeometry}) in the special case where $i$ is the flat geometric embedding of \Cref{cameo}. In the next subections we will revisit their theorem, offering the widest generalisation we could find. 

%It is important to stress that this proposition is key for their proof that a coherent topos has enough points. We will also come back to this observation in the section about ultrastructures.
\end{disc}

\subsection{Kan injecivity}

This subsection briefly recalls Kan injectivity in a $2$-category and revisits the theory of injective topoi (with respect to geometric embeddings). We will use this technology in the next subsection to provide a generalization of \Cref{paradigm} to the $2$-category of topoi.
Kan injectivity was studied in the case of poset-enriched categories in \cite{escardo1997injective,escardo1998properly,escardo1999semantic,adamek2015kan} and recently generalised to $2$-categories in \cite{dlsolo}, we refer to the latter for the reader that is new to the concept. Below we recall an operative definition of Kan injectivity, specialised directly to the case of the $2$-category of topoi.

\begin{defn}[Kan injectivity, {\cite{dlsolo}}]
Let $f: \mathcal{F} \to \mathcal{G}$. We say that $\mathcal{E}$ is left (right) Kan injective with respect to $f$ if the functor \[(-) \circ f: \mathsf{Topoi}(\mathcal{G}, \mathcal{E}) \to \mathsf{Topoi}(\mathcal{F}, \mathcal{E})\] has a left (right) adjoint whose counit (unit) is an isomorphism.
\end{defn}

\begin{war}[$f^\sharp \dashv f_\sharp$]
It is useful to have a more compact notation for the functor $(-) \circ f: \mathsf{Topoi}(\mathcal{G}, \mathcal{E}) \to \mathsf{Topoi}(\mathcal{F}, \mathcal{E})$.  We shall call $f^\sharp$ such functor and $f_\sharp$ its right adjoint. Of course, other notations would be more traditional, but they would clash with the usual symbol for the inverse image functor.
\end{war}


\begin{war}[On right Kan extension in the $2$-category of topoi] \label{rightKanextension}
% https://q.uiver.app/?q=WzAsMyxbMSwwLCJcXG1hdGhjYWx7RX0iXSxbMCwwLCJcXG1hdGhjYWx7Rn0iXSxbMCwxLCJcXG1hdGhjYWx7R30iXSxbMSwyLCJmIiwyXSxbMSwwLCJ4Il0sWzIsMCwiZl9cXHNoYXJwKHgpID0gXFxtYXRoc2Z7cmFufV9meCIsMix7InN0eWxlIjp7ImJvZHkiOnsibmFtZSI6ImRhc2hlZCJ9fX1dXQ==
\[\begin{tikzcd}
	{\mathcal{F}} & {\mathcal{E}} \\
	{\mathcal{G}}
	\arrow["f"', from=1-1, to=2-1]
	\arrow["x", from=1-1, to=1-2]
	\arrow["{\mathsf{ran}_fx}"', dashed, from=2-1, to=1-2]
\end{tikzcd}\]
Recall from \cite{dlsolo} that $f_\sharp(x)$ is the right Kan extension $\mathsf{ran}_fx$ in the $2$-category of topoi (as opposed to the $2$-category of categories) and thus has the appropriate universal property in that $2$-category. We will see below that in some concrete instances (\Cref{h_*}), some counter-intuitive equations hold,  \[({\mathsf{ran}^{\mathsf{Topoi}}_fx})_* \cong \mathsf{lan}^{\mathsf{Cat}}_{f_*}x_*.\]
We do not know whether such an equation has to hold in general, but it is a great exemplification that one should be careful when manipulating these Kan extensions. 
\end{war}

\begin{war}[The $2$-category of topoi]
Because the $2$-dimensional structure is so important, let us recall what we intend by $\mathsf{Topoi}$, the $2$-category of Grothendieck topoi. It has objects Grothendieck topoi, $1$-cells geometric morphisms (in the direction of the right adjoint) and $2$-cells natural transformations between the inverse (!) images. 
\end{war}


\begin{defn}[{\cite{dlsolo}}]
A geometric morphism $g: \mathcal{E} \to \mathcal{D}$ between left (right) injective objects with respect to a morphism $f$ is left (right) injective with respect to a morphism $f$ if it preserves the left (right) Kan extension as below.

% https://q.uiver.app/?q=WzAsNCxbMSwwLCJcXG1hdGhjYWx7RX0iXSxbMiwwLCJcXG1hdGhjYWx7RH0iXSxbMCwwLCJcXG1hdGhjYWx7Rn0iXSxbMCwxLCJcXG1hdGhjYWx7R30iXSxbMiwzLCJmIiwyXSxbMiwwLCJ4Il0sWzAsMSwiZyJdLFszLDAsImZfXFxzaGFycCh4KSIsMV0sWzMsMSwiZl97XFxzaGFycH0oZ3gpIiwyLHsiY3VydmUiOjJ9XV0=
\[\begin{tikzcd}
	{\mathcal{F}} & {\mathcal{E}} & {\mathcal{D}} \\
	{\mathcal{G}}
	\arrow["f"', from=1-1, to=2-1]
	\arrow["x", from=1-1, to=1-2]
	\arrow["g", from=1-2, to=1-3]
	\arrow["{f_\sharp(x)}"{description}, from=2-1, to=1-2]
	\arrow["{f_{\sharp}(gx)}"', curve={height=12pt}, from=2-1, to=1-3]
\end{tikzcd}\]
\end{defn}



\begin{defn}[Relevant classes of flat morphisms: $\mathsf{Emb}_\beta \subset \mathsf{Emb}_\flat$] \label{classes}
We define two classes of geometric morphisms, one trivially contained in the other $\mathsf{Emb}_\beta \subset \mathsf{Emb}_\flat$. These two classes will be the main object of study of this paper.

\begin{itemize}
    \item[$(\mathsf{Emb}_\flat)$] contains flat embeddings,
    \item[$(\mathsf{Emb}_\beta)$] contains the flat embeddings of the form $\mathsf{Set}^X \to \mathsf{Sh}(\beta(X))$, where $X$ is a set and $\beta(X)$ is its space of ultrafilters. This is a special case of \Cref{cameo} where $B$ is the powerset $\mathcal{P}(X)$.
\end{itemize}
\end{defn}

\begin{notat}[$\beta$-complete and $\flat$-complete topoi] \label{complete}
Following \cite{dlsolo}, we call $\mathsf{RInj}(\mathsf{Emb}_\square)$ the locally full subcategory of topoi that are right Kan injective with respect to $\mathsf{Emb}_\square$. We shall call $\square$-complete a topos in $\mathsf{RInj}(\mathsf{Emb}_\square)$. We obtain a forgetful functor \[ \mathbb{U}_\beta^\flat: \mathsf{RInj}(\mathsf{Emb}_\flat) \to \mathsf{RInj}(\mathsf{Emb}_\beta).\]
\end{notat}

We now revisit a classical result from the theory of topoi, from the point of view of right Kan injectivity. 

\begin{prop}
Let $C$ be a category with finite limits. Then $\mathsf{Psh}(C)$ is right Kan injective with respect to embeddings of topoi.
\end{prop}
\begin{proof}
This is shown in \cite[Prop. 1.2]{johnstone1981injective}. It will be later useful though to provide explicitly the construction of the right adjoint and to inspect its properties. In order to do so, consider the diagram below where of course $i$ is an embedding and $x$ is any geometric morphism. 

% https://q.uiver.app/?q=WzAsMyxbMSwwLCJcXG1hdGhzZntQc2h9KEMpIl0sWzAsMCwiXFxtYXRoY2Fse0Z9Il0sWzAsMSwiXFxtYXRoY2Fse0d9Il0sWzEsMiwiaSIsMix7InN0eWxlIjp7InRhaWwiOnsibmFtZSI6Imhvb2siLCJzaWRlIjoidG9wIn19fV0sWzEsMCwiZiJdLFsyLDAsImgiLDIseyJzdHlsZSI6eyJib2R5Ijp7Im5hbWUiOiJkYXNoZWQifX19XV0=
\[\begin{tikzcd}
	{\mathcal{F}} & {\mathsf{Psh}(C)} \\
	{\mathcal{G}}
	\arrow["i"', hook, from=1-1, to=2-1]
	\arrow["x", from=1-1, to=1-2]
	\arrow["h"', dashed, from=2-1, to=1-2]
\end{tikzcd}\]

We shall describe the construction for $h$. Call $y: C \to \mathsf{Psh}(C)$ the Yoneda embedding. We define, \[h^* = \mathsf{lan}_{y}(i_*x^*y).\] It is clear that $h^*$ is cocontinuous and lex, because $i_*x^*y$ is a lex functor and it is easy to check that the geometric morphism induced by $h^*$ has the desired properties.
\end{proof}

\begin{rem}[A description for $h_*$] \label{h_*}
We note en passant that the definition of $h^*$ provides a concrete description of its right adjoint by the following chain of isomorphisms,

\begin{align*}
    h_* \cong & \mathsf{lan}_{h^*}(1) \\
        \cong & \mathsf{lan}_{h^*}(\mathsf{lan}_y y) \\
        \cong & \mathsf{lan}_{h^*y}(y) \\
        \cong & \mathsf{lan}_{(\mathsf{lan}_{y}(i_*x^*y)y)}(y) \\
        \cong & \mathsf{lan}_{i_*x^*y}(y) \\
         \cong & \mathsf{lan}_{i_*x^*}(\mathsf{lan}_{y}(y)) \\
              \cong & \mathsf{lan}_{i_*}(x_*) \\
\end{align*}
\end{rem}

% \begin{rem}[Kan injectives are closed under lex essential subtopoi]
% It is clear that (right) Kan-injective objects with respect to any class of maps are closed under coreflection. Thus, a coreflection (in $\mathsf{Topoi}$) of a presheaf topos over a lex category is Kan-injective with respect to embeddings of topoi. Let us describe coreflections of topoi, 

% \[i: \mathcal{F} \leftrightarrows \mathcal{E}: r. \] 

% It is easy to see that $i \dashv r$ in the $2$-category of topoi if and only if $r^* \dashv i^*$ (notice the exchange!). Then, a coreflection amounts to a couple of geometric morphisms $(i,r)$, such that $i_*$ is fully faithful, $r^* \dashv i^*$ and $r_*i_* \cong 1 \cong i^*r^*$. This is the same of a geometric embedding $i: \mathcal{F} \hookrightarrow \mathcal{G}$ whose inverse image admits and additional conservative lex left adjoint.


% % https://q.uiver.app/?q=WzAsMixbMCwwLCJcXG1hdGhjYWx7Rn0iXSxbMiwwLCJcXG1hdGhjYWx7RX0iXSxbMCwxLCJpXyoiLDEseyJjdXJ2ZSI6MSwic3R5bGUiOnsidGFpbCI6eyJuYW1lIjoiaG9vayIsInNpZGUiOiJ0b3AifX19XSxbMSwwLCJpXioiLDEseyJjdXJ2ZSI6Mn1dLFswLDEsImlfISIsMSx7ImN1cnZlIjotNX1dLFszLDIsIiIsMix7ImxldmVsIjoxLCJzdHlsZSI6eyJuYW1lIjoiYWRqdW5jdGlvbiJ9fV0sWzQsMywiIiwyLHsibGV2ZWwiOjEsInN0eWxlIjp7Im5hbWUiOiJhZGp1bmN0aW9uIn19XV0=
% \[\begin{tikzcd}
% 	{\mathcal{F}} && {\mathcal{E}}
% 	\arrow[""{name=0, anchor=center, inner sep=0}, "{i_*}"{description}, curve={height=6pt}, hook, from=1-1, to=1-3]
% 	\arrow[""{name=1, anchor=center, inner sep=0}, "{i^*}"{description}, curve={height=12pt}, from=1-3, to=1-1]
% 	\arrow[""{name=2, anchor=center, inner sep=0}, "{i_!}"{description}, curve={height=-30pt}, from=1-1, to=1-3]
% 	\arrow["\dashv"{anchor=center, rotate=-90}, draw=none, from=1, to=0]
% 	\arrow["\dashv"{anchor=center, rotate=-90}, draw=none, from=2, to=1]
% \end{tikzcd}\]

% One could call such subtopoi \textit{lex essential subtopoi}, as this is a slightly stronger property than being an essential subtopos.
% \end{rem}

% \begin{rem}
% At this point it would not be hard to show that right Kan injective topoi with respect to embeddings of topoi are precisely lex essential subtopoi of presheaf topoi over lex categories. We choose not to go in this direction because such detour would be a distraction in this paper.
% \end{rem} 


\subsection{Coherent topoi are Kan injective with regard to flat embeddings}


We now deliver our generalisation of \Cref{paradigm}.

\begin{thm} \label{mainth}
Coherent topoi are right Kan injective with respect to flat embeddings.
\end{thm}


\begin{proof}
We start by saying that a coherent topos $\mathcal{E}$ admits a geometric embedding into a presheaf topos over a lex category $j: \mathcal{E} \hookrightarrow \mathsf{Psh}(C)$. Moeover, $C$ can be chosen to be the full subcategory of coherent objects, so that the direct image of the geometric embedding preserve directed colimits \cite[III.1.1(2)]{moerdijk2000proper}. In the diagram below, let $i$ be a flat embedding of topoi and $x$ be any geometric morphism.

% https://q.uiver.app/?q=WzAsNCxbMCwwLCJcXG1hdGhjYWx7TH0iXSxbMCwxLCJcXG1hdGhjYWx7TCd9Il0sWzEsMCwiXFxtYXRoY2Fse0V9Il0sWzEsMSwiXFxtYXRoc2Z7UHNofShDKSJdLFswLDIsImYiXSxbMCwxLCJpIiwyLHsic3R5bGUiOnsidGFpbCI6eyJuYW1lIjoiaG9vayIsInNpZGUiOiJ0b3AifX19XSxbMiwzLCJqIiwwLHsic3R5bGUiOnsidGFpbCI6eyJuYW1lIjoiaG9vayIsInNpZGUiOiJ0b3AifX19XSxbMSwzLCJoIiwyLHsic3R5bGUiOnsiYm9keSI6eyJuYW1lIjoiZGFzaGVkIn19fV1d
\[\begin{tikzcd}
	{\mathcal{L}} & {\mathcal{E}} \\
	{\mathcal{L'}} & {\mathsf{Psh}(C)}
	\arrow["x", from=1-1, to=1-2]
	\arrow["i"', hook, from=1-1, to=2-1]
	\arrow["j", hook, from=1-2, to=2-2]
	\arrow["h"', dashed, from=2-1, to=2-2]
\end{tikzcd}\]

Because $\mathsf{Psh}(C)$ is right Kan injective, $h$ exists. By \Cref{h_*}, we have a formula to compute its right adjoint which we recall, \[h_*  \cong \mathsf{lan}_{i_*}(j_*x_*).\]
To construct $i^\mathcal{E}_\sharp(x)$ we show that $j_*j^*h_* \cong h_*,$ indeed this implies that any such an $h$ factors through $\mathcal{E}$ because it means every object in the image of $h_*$ is automatically a $j$-sheaf. 
\begin{align*}
    j_*j^*h_*  \cong & j_*j^*\mathsf{lan}_{i_*}(j_*x_*) \\
                \cong & j_*\mathsf{lan}_{i_*}(j^*j_*x_*) \\
                  \cong & j_*\mathsf{lan}_{i_*}(x_*) \\
       (*)           \cong & \mathsf{lan}_{i_*}(j_*x_*) \\
                 \cong & h_*. \\
\end{align*}
We should justify the equation $(*)$, i.e. why $j_*$ preserves the Kan extension $\mathsf{lan}_{i_*}(x_*)$. To do so, we write down the formula to compute the Kan extension explicitly. Let us do that,

\[j_*\mathsf{lan}_{i_*}(x_*)(y) \cong j_*(\underset{i_*(d) \to y}{\mathsf{colim}} x_*(d))\] 

Now, because $i_*$ preserve finite colimits, the diagram indexing the colimit is filtered, and thus is preserved by $j_*$. But this is the same as saying that $j_*$ preserves the Kan extension.

\[j_*\mathsf{lan}_{i_*}(x_*)(y) \cong j_*(\underset{i_*(d) \to y}{\mathsf{colim}} x_*(d)) \cong   \underset{i_*(d) \to y}{\mathsf{colim}} j_*x_*(d)  \cong \mathsf{lan}_{i_*}(j_*x_*)(y).\] 

The discussion above shows that we can indeed define a functor $i^\mathcal{E}_{\sharp}$ in the diagram below with the property that $j_\mathcal{L'}^{\bullet}i^\mathcal{E}_{\sharp} \cong i^{\mathsf{Psh}(C)}_{\sharp}j^\bullet_{\mathcal{L}}$, which is a convoluted way to say that $i^{\mathsf{Psh}(C)}_{\sharp}j^\bullet_{\mathcal{L}}$ actually lands in $\mathsf{Topoi}(\mathcal{L'}, \mathcal{E})$. 

% https://q.uiver.app/?q=WzAsNCxbMiwwLCJcXG1hdGhzZntUb3BvaX0oXFxtYXRoY2Fse0x9LCBcXG1hdGhzZntQc2h9KEMpKSJdLFsyLDIsIlxcbWF0aHNme1RvcG9pfShcXG1hdGhjYWx7TH0nLCBcXG1hdGhzZntQc2h9KEMpKSJdLFswLDIsIlxcbWF0aHNme1RvcG9pfShcXG1hdGhjYWx7TH0nLCBcXG1hdGhjYWx7RX0pIl0sWzAsMCwiXFxtYXRoc2Z7VG9wb2l9KFxcbWF0aGNhbHtMfSwgXFxtYXRoY2Fse0V9KSJdLFswLDEsImlee1xcbWF0aHNme1BzaH0oQyl9X1xcc2hhcnAiLDAseyJjdXJ2ZSI6LTJ9XSxbMSwwLCJpX3tcXG1hdGhzZntQc2h9KEMpfV5cXHNoYXJwIiwwLHsiY3VydmUiOi0yfV0sWzIsMywiaV97XFxtYXRoY2Fse0V9fV5cXHNoYXJwIiwwLHsiY3VydmUiOi0yfV0sWzMsMCwial9cXG1hdGhjYWx7TH1eXFxidWxsZXQiXSxbMiwxLCJqX1xcbWF0aGNhbHtMJ31eXFxidWxsZXQiXSxbMywyLCJpXlxcbWF0aGNhbHtFfV9cXHNoYXJwIiwwLHsiY3VydmUiOi0zLCJzdHlsZSI6eyJib2R5Ijp7Im5hbWUiOiJkYXNoZWQifX19XSxbNSw0LCIiLDAseyJsZXZlbCI6MSwic3R5bGUiOnsibmFtZSI6ImFkanVuY3Rpb24ifX1dXQ==
\[\begin{tikzcd}
	{\mathsf{Topoi}(\mathcal{L}, \mathcal{E})} && {\mathsf{Topoi}(\mathcal{L}, \mathsf{Psh}(C))} \\
	\\
	{\mathsf{Topoi}(\mathcal{L}', \mathcal{E})} && {\mathsf{Topoi}(\mathcal{L}', \mathsf{Psh}(C))}
	\arrow[""{name=0, anchor=center, inner sep=0}, "{i^{\mathsf{Psh}(C)}_\sharp}", curve={height=-12pt}, from=1-3, to=3-3]
	\arrow[""{name=1, anchor=center, inner sep=0}, "{i_{\mathsf{Psh}(C)}^\sharp}", curve={height=-12pt}, from=3-3, to=1-3]
	\arrow["{i_{\mathcal{E}}^\sharp}", curve={height=-12pt}, from=3-1, to=1-1]
	\arrow["{j_\mathcal{L}^\bullet}", from=1-1, to=1-3]
	\arrow["{j_\mathcal{L'}^\bullet}", from=3-1, to=3-3]
	\arrow["{i^\mathcal{E}_\sharp}", curve={height=-18pt}, dashed, from=1-1, to=3-1]
	\arrow["\dashv"{anchor=center}, draw=none, from=1, to=0]
\end{tikzcd}\]

We still have to show that $i^\sharp_\mathcal{E} \dashv i_\sharp^\mathcal{E}$. To do so, we will use the fact that both the functors $j^\bullet_{\square}$ are fully faithful (because $j$ is in first place). Indeed, it is enough to follow the following chain of equivalences to finish the proof. (We simplified the notation for the sake of readability).
%This means that we need to provide two natural transformations $\epsilon: i^\sharp_\mathcal{E} i_\sharp^\mathcal{E} \to 1$, $\eta:1 \to i_\sharp^\mathcal{E}i^\sharp_\mathcal{E}$. To do so, we will use that both the functors $j^\bullet_{\square}$ are fully faithful (because $j$ is in first place). Recall that we do have natural transformations, \[\xi: i^\sharp_{\mathsf{Psh}(C)} i_\sharp^{\mathsf{Psh}(C)} \to 1 \quad \quad \rho :1 \to i_\sharp^{\mathsf{Psh}(C)} i^\sharp_{\mathsf{Psh}(C)}.\]


\begin{align*}
    (\mathcal{L},\mathcal{E})(i^\sharp_\mathcal{E}-,-) \simeq & (\mathcal{L},\mathsf{Psh}(C))(j^\bullet_\mathcal{L}i^\sharp_\mathcal{E}-,j^\bullet_\mathcal{L}-) \\ 
    \simeq & (\mathcal{L},\mathsf{Psh}(C))(i^\sharp_{\mathsf{Psh}(C)}j^\bullet_\mathcal{L'}-,j^\bullet_\mathcal{L}-) \\
    \simeq & (\mathcal{L}',\mathsf{Psh}(C))(j^\bullet_\mathcal{L'}-,i_\sharp^{\mathsf{Psh}(C)}j^\bullet_\mathcal{L}-) \\
    \simeq & (\mathcal{L}',\mathsf{Psh}(C))(j^\bullet_\mathcal{L'}-,j_\mathcal{L'}^{\bullet}i^\mathcal{E}_{\sharp}-) \\
  \simeq &  (\mathcal{L}',\mathcal{E})(-,i^\mathcal{E}_{\sharp}-). \\
\end{align*}
\end{proof}

\begin{rem}
Our result improves \cite[IX.11.1]{sheavesingeometry} in two directions:
\begin{itemize}
    \item we showed that coherent topoi are injective with respect to all flat embeddings, as opposed to a very specific kind.
    \item we showed that coherent topoi are Kan (!) injective, which of course is a sharper property than being simply injective.
\end{itemize}
\end{rem}

\begin{rem}[A comment on the proof] \label{morgan}
Another way to state the theorem above is to say that flat embeddings of topoi are  \textit{orthogonal on the left} to relatively tidy embeddings of topoi into a presheaf topoi over lex categories. This may seem as a convoluted way of stating the theorem, but opens the door to a new theory of orthogonality. It would be interesting for example to know whether in the proof above $j$ can land into any Kan injective with respect to embeddings. On a similar note, it would be interesting to see how this theorem changes if we replace the notion of flatness, for example with that of purity (see \cite[III.5]{moerdijk2000proper}).
\end{rem}

\begin{rem}[Another comment on the proof]
One can look at our proof as a very anatomic analysis of \cite[IX.11.1]{sheavesingeometry}. Besides being more general, our proof has the perk of highlighting the key moments where the hypotheses are used, in such a way that the proof is parametric in those assumptions. For example, if $i_*$ was only preserving coproducts, a similar proof would work for $j_*$ preserving sifted colimits.
\end{rem}

\begin{rem}[Towards a characterization of right Kan injectives with respect to flat embeddings] \label{characterisation}

As we mentioned above, it is observed in \cite{carvalho2017kan} that right Kan injectives with respect to flat embeddings of locales are precisely coreflections of coherent locales. This leads to the simple conjecture that such a result could be true also for coherent topoi, which would set a perfect matching between coherent topoi and flat embeddings. In the case of locales it is easy to derive that every right Kan injective with respect to flat embeddings is a coreflection of a coherent locale because every locale flatly embeds in a coherent one (\Cref{exampleindcompl}). This does not seem to be true though for the theory of topoi for two reasons. \begin{enumerate}
    \item  Because every topos is localic with respect to the classifier of the theory of objects $\mathsf{Set}[\mathbb{O}]$, we could try and simulate the proof of \Cref{exampleindcompl} from \cite[IX.10.3]{sheavesingeometry} for the case of locales internal to $\mathsf{Set}[\mathbb{O}]$, which would deliver the proof that every topos flatly embeds in a coherent one. Unfortunately the proof seem to rely on the use of choice, which fails in  $\mathsf{Set}[\mathbb{O}]$.
    \item For a topos $\mathcal{E}$, there is indeed an analog of the ideal-construction which is the $\mathsf{Ind}$-completion of a category, \[i: \mathcal{E} \to \mathsf{Ind}(\mathcal{E}).\]
     $i$ has both conceptually and technically the same behavior of the ideal-construction. Indeed $i$ preserves finite colimits, and one could show (with some work on size issues) that it has a lex left adjoint. Unfortunately though, $\mathsf{Ind}(\mathcal{E})$ is seldom a topos\footnote{Or even an infinitary-pretopos.} (see \cite{borceux1999left})!  One should say though that in the spirit of \cite{borceux1999left} it is believable that every \textit{noetherian} topos flatly embeds in a coherent one, and this observation would deserve further investigation. Could this be a characterization of noetherian topoi?
\end{enumerate} 
All in all, with the technology that is available today, we cannot characterise the topoi that are right Kan injective with respect to flat embeddings. Notice that this would be an extremely interesting task, as all these topoi must have enough points. The best result we can deliver follows.
\end{rem}

\begin{prop} \label{char2} Let $\mathcal{E}$ be a $\flat$-complete topos. Then $\mathcal{E}$ admits a geometric surjection from a coherent spatial topos. (In particular it has enough points.)

\end{prop}
\begin{proof}
Let $\mathcal{L}$ be the Diaconescu cover of $\mathcal{E}$ (\cite[Thm. 4.1]{johnstone1985general}), and consider the diagram below, where $i: \mathcal{L} \to \mathcal{B}$ is constructed as in \Cref{exampleindcompl}. Recall that $q$ can be chosen to be connected and locally connected.

% https://q.uiver.app/?q=WzAsMyxbMSwwLCJcXG1hdGhjYWx7RX0iXSxbMCwwLCJcXG1hdGhjYWx7TH0iXSxbMCwxLCJcXG1hdGhjYWx7Qn0iXSxbMSwwLCJxIiwwLHsic3R5bGUiOnsiaGVhZCI6eyJuYW1lIjoiZXBpIn19fV0sWzEsMiwiaSIsMix7InN0eWxlIjp7InRhaWwiOnsibmFtZSI6Imhvb2siLCJzaWRlIjoidG9wIn19fV0sWzIsMCwiaCIsMix7InN0eWxlIjp7ImJvZHkiOnsibmFtZSI6ImRhc2hlZCJ9LCJoZWFkIjp7Im5hbWUiOiJlcGkifX19XV0=
\[\begin{tikzcd}
	{\mathcal{L}} & {\mathcal{E}} \\
	{\mathcal{B}}
	\arrow["q", two heads, from=1-1, to=1-2]
	\arrow["i"', hook, from=1-1, to=2-1]
	\arrow["h"', dashed, two heads, from=2-1, to=1-2]
\end{tikzcd}\]

$\mathcal{B}$ is coherent by design. Because $\mathcal{E}$ is right Kan injective with respect to flat embeddings, $h$ exists and because $q$ is a surjection, so must be $h$.
\end{proof}

\begin{rem}
In the proof above, because $i$ is dense, it is tempting to believe that the map $h$ is open too, possibly applying some version of \cite[C3.1.14(ii)]{Sketches}, but we are skeptical about such a result, as it would seem very strong on a conceptual level. We did not manage to prove nor disprove such a statement.
\end{rem}


\begin{rem}[A first encounter with Marmolejo]
In \cite{marmolejo1995ultraproducts} introduces the highly nontrivial framework of \textit{Łoś categories} to organise the fact that coherent topoi are $\beta$-complete. His point of view is different from ours, even though it really has the same motivations, which is to better understand the notion of ultracategory. 
%We will discuss this later in the text, but for the moment we observe that our result is more general in that it does not only restrict its attention to localic topoi.
\end{rem}

\subsection{Spartan Geometric Morphisms}

In this subsection we introduce a convenient notion of morphism between coherent topoi, namely \textit{spartan geometric morphisms}. On a conceptual level, we will see that spartan morphisms are the geometric counterpart of morphism of pretopoi of coherent objects. The main result of the subsection is that spartan geometric morphisms are right Kan injective with respect to $\mathsf{Emb}_\flat$.

\begin{defn}[Spartan geometric morphism]
A geometric morphism $f: \mathcal{E} \to \mathcal{D}$ is spartan if its direct image $f_*$ preserve directed colimits. We define $\mathsf{CohTopoi}_\mathcal{S}$ to be the sub $2$-category of $\mathsf{Topoi}$ containing coherent topoi and spartan geometric morphisms.
\end{defn}

\begin{rem}
Spartan geometric morphisms are the same as ($\mathsf{Set}$)-relatively tidy geometric morphisms in the sense of \cite{moerdijk2000proper}. For the sake of this paper we prefer the terminology \textit{spartan} because we do not focus on a relative point of view. Also, the terminology is just shorter.
\end{rem}

\begin{exa} \label{innocente} There are two important class of examples of spartan geometric morphisms in nature.
\begin{itemize}
    \item Tidy geometric morphisms are spartan, because they are relatively tidy.
    \item Coherent geometric morphisms between coherent topoi are spartan \cite[V.3.2]{moerdijk2000proper}.
\end{itemize}
\end{exa}


% \begin{prop}
% Coherent geometric morphism are moderate.
% \end{prop}
% \begin{proof}
% Recall that a geometric morphism $f: \mathcal{E} \to \mathcal{D}$ is coherent if its inverse image $f^*$ preserve coherent objects. Now, it follows from the general theory of sites that the corestriction $f^*: \mathsf{Coh}(\mathcal{D}) \to \mathsf{Coh}(\mathcal{E})$ is a morphism of sites inducing the geometric morphism $f$. To avoid any notational confusion, we shall call such corestriction $g$ in the diagram below.

% % https://q.uiver.app/?q=WzAsNixbMCwwLCJcXG1hdGhzZntDb2h9KFxcbWF0aGNhbHtEfSkiXSxbMSwwLCJcXG1hdGhzZntDb2h9KFxcbWF0aGNhbHtFfSkiXSxbMCwxLCJcXG1hdGhjYWx7RH0iXSxbMSwxLCJcXG1hdGhjYWx7RX0iXSxbMCwyLCJcXG1hdGhzZntQc2h9KFxcbWF0aHNme0NvaH0oXFxtYXRoY2Fse0R9KSkiXSxbMSwyLCJcXG1hdGhzZntQc2h9KFxcbWF0aHNme0NvaH0oXFxtYXRoY2Fse0V9KSkiXSxbMCwxLCJnIl0sWzIsMywiZl4qIiwxLHsib2Zmc2V0IjotNH1dLFswLDJdLFsxLDNdLFszLDUsImlfXFxtYXRoY2Fse0V9Il0sWzIsNCwiaV9cXG1hdGhjYWx7RH0iLDJdLFs1LDQsImdeKiIsMV0sWzQsNSwiZ18hIiwxLHsiY3VydmUiOjN9XSxbNCw1LCJnXyoiLDEseyJjdXJ2ZSI6LTN9XSxbNCwyLCJsX1xcbWF0aGNhbHtEfSIsMCx7ImN1cnZlIjotNH1dLFszLDIsImZfKiIsMSx7Im9mZnNldCI6LTJ9XSxbMTIsMTQsIiIsMSx7ImxldmVsIjoxLCJzdHlsZSI6eyJuYW1lIjoiYWRqdW5jdGlvbiJ9fV0sWzEzLDEyLCIiLDEseyJsZXZlbCI6MSwic3R5bGUiOnsibmFtZSI6ImFkanVuY3Rpb24ifX1dLFs3LDE2LCIiLDIseyJsZXZlbCI6MSwic3R5bGUiOnsibmFtZSI6ImFkanVuY3Rpb24ifX1dXQ==
% \[\begin{tikzcd}
% 	{\mathsf{Coh}(\mathcal{D})} & {\mathsf{Coh}(\mathcal{E})} \\
% 	{\mathcal{D}} & {\mathcal{E}} \\
% 	{\mathsf{Psh}(\mathsf{Coh}(\mathcal{D}))} & {\mathsf{Psh}(\mathsf{Coh}(\mathcal{E}))}
% 	\arrow["g", from=1-1, to=1-2]
% 	\arrow[""{name=0, anchor=center, inner sep=0}, "{f^*}"{description}, shift left=4, from=2-1, to=2-2]
% 	\arrow[from=1-1, to=2-1]
% 	\arrow[from=1-2, to=2-2]
% 	\arrow["{i_\mathcal{E}}", from=2-2, to=3-2]
% 	\arrow["{i_\mathcal{D}}"', from=2-1, to=3-1]
% 	\arrow[""{name=1, anchor=center, inner sep=0}, "{g^*}"{description}, from=3-2, to=3-1]
% 	\arrow[""{name=2, anchor=center, inner sep=0}, "{g_!}"{description}, curve={height=18pt}, from=3-1, to=3-2]
% 	\arrow[""{name=3, anchor=center, inner sep=0}, "{g_*}"{description}, curve={height=-18pt}, from=3-1, to=3-2]
% 	\arrow["{l_\mathcal{D}}", curve={height=-24pt}, from=3-1, to=2-1]
% 	\arrow[""{name=4, anchor=center, inner sep=0}, "{f_*}"{description}, shift left=2, from=2-2, to=2-1]
% 	\arrow["\dashv"{anchor=center, rotate=91}, draw=none, from=1, to=3]
% 	\arrow["\dashv"{anchor=center, rotate=89}, draw=none, from=2, to=1]
% 	\arrow["\dashv"{anchor=center, rotate=-90}, draw=none, from=0, to=4]
% \end{tikzcd}\]

% Now, again by the general construction of geometric morphisms from morphisms of sites, the direct image $f_*$ coincides with $l_\mathcal{D} \circ g^* \circ i_\mathcal{E}$, where $l_\mathcal{D}$ is the sheafification functor. Several authors have observed that the functor $i_\mathcal{E}$ is moderate, or relatively tidy, see for example \cite[III.1.1]{moerdijk2000proper}. Then, the composition $l_\mathcal{D} \circ g^* \circ i_\mathcal{E}$ preserve directed colimits because all the functors do so, and it follows that $f$ must be moderate.
% \end{proof}

% \begin{prop}
% Let $\mathsf{Pretopoi}$ be the $2$-category of pretopoi and pretopos morphisms. Then, the assignment $\mathcal{E} \mapsto \mathsf{Coh}(\mathcal{E})$, which maps a coherent topos to its full subcategory of coherent objects (co)rerestricts to a (contravariant) $2$-functor,
% \[\mathsf{Coh}: \mathsf{CohTopoi}_\mathcal{S}^\circ \to \mathsf{Pretopoi}.\]
% \end{prop}
% \begin{proof}
% At the level of objects it is well known that, given a coherent topos $\mathcal{E}$, its full subcategory of coherent objects $\mathsf{Coh}(\mathcal{E})$ is a pretopos.

% Recall that a coherent topos $\mathcal{E}$ is a locally presentable category whose finitely presentable objects coincide with its coherent objects. There are many ways to prove it, the easiest is to observe that that coherent objects are finitely presentable \cite[D3.3.12]{Sketches}, they are closed under finite colimits and they form a strong generator, the result then follows \cite[1.11]{adamek_rosicky_1994}. Now, consider a moderate geometric morphism
% \end{proof}

\begin{prop} \label{moderatesarekan}
Spartan geometric morphisms $f: \mathcal{E} \to \mathcal{D}$ between $\flat$-complete topoi are right Kan injective with respect to flat geometric embeddings.
\end{prop}
\begin{proof}
We shall focus on the diagram below, where $i$ is a flat embedding. We will show that $f i_\sharp (x) \cong i_\sharp(fx).$

% https://q.uiver.app/?q=WzAsNCxbMSwwLCJcXG1hdGhjYWx7RX0iXSxbMiwwLCJcXG1hdGhjYWx7RH0iXSxbMCwwLCJcXG1hdGhjYWx7Rn0iXSxbMCwxLCJcXG1hdGhjYWx7R30iXSxbMiwzLCJpIiwyXSxbMiwwLCJ4Il0sWzAsMSwiZiJdLFszLDAsImlfXFxzaGFycCh4KSIsMV0sWzMsMSwiaV97XFxzaGFycH0oZngpIiwyLHsiY3VydmUiOjJ9XV0=
\[\begin{tikzcd}
	{\mathcal{F}} & {\mathcal{E}} & {\mathcal{D}} \\
	{\mathcal{G}}
	\arrow["i"', from=1-1, to=2-1]
	\arrow["x", from=1-1, to=1-2]
	\arrow["f", from=1-2, to=1-3]
	\arrow["{i_\sharp(x)}"{description}, from=2-1, to=1-2]
	\arrow["{i_{\sharp}(fx)}"', curve={height=12pt}, from=2-1, to=1-3]
\end{tikzcd}\]

To do so, we go back to the proof of \Cref{mainth} to have a better description of $i_\sharp(x)$. In the notation of that proof we observe that $i_\sharp(x)_*$ coincides with $j^*h_*$, and thus we obtain the equation, \[(i_\sharp(x))_* \cong \mathsf{lan}_{i_*}x_*.\] This essentially finishes the proof, which we reduce to the following chain of isomorphisms. As in the proof of \Cref{mainth}, the non-trivial  isomorphism (*) comes from the fact that $i_*$ preserve finite colimits and thus the diagram is filtered.

\begin{align*}
    (f i_\sharp (x))_*(y) \cong & f_*\mathsf{lan}_{i_*}(x_*)(y) \\
        \cong & f_*(\underset{i_*(z) \to y}{\mathsf{colim}} x_*(z)) \\
     (*)   \cong &  \underset{i_*(z) \to y}{\mathsf{colim}} f_*x_*(z) \\
        \cong & \mathsf{lan}_{i_*}(f_*x_*)(y) \\
        \cong & (i_\sharp(fx))_*(y). \\
\end{align*}
\end{proof}

We organise the results of this short subsection in the diagram below.

% https://q.uiver.app/?q=WzAsNSxbMSwyLCJcXG1hdGhzZntDb2hUb3BvaX0iXSxbMSwxLCJcXG1hdGhzZntDb2hUb3BvaX1fe1xcbWF0aGNhbHtNfX0iXSxbMCwwLCJcXG1hdGhzZntSSW5qfShcXG1hdGhzZntFbWJ9X1xcZmxhdCkiXSxbMiwwLCJcXG1hdGhzZntSSW5qfShcXG1hdGhzZntFbWJ9X1xcYmV0YSkiXSxbMSwzLCJcXG1hdGhzZntUb3BvaX0iXSxbMiwzLCJcXG1hdGhiYntVfV9cXGJldGFeXFxmbGF0IiwxXSxbMSwyLCIiLDIseyJzdHlsZSI6eyJ0YWlsIjp7Im5hbWUiOiJob29rIiwic2lkZSI6ImJvdHRvbSJ9LCJib2R5Ijp7Im5hbWUiOiJkYXNoZWQifX19XSxbMSwzLCIiLDIseyJzdHlsZSI6eyJ0YWlsIjp7Im5hbWUiOiJob29rIiwic2lkZSI6InRvcCJ9LCJib2R5Ijp7Im5hbWUiOiJkYXNoZWQifX19XSxbMSwwXSxbMCw0XSxbMyw0LCIiLDIseyJjdXJ2ZSI6LTJ9XSxbMiw0LCIiLDIseyJjdXJ2ZSI6Mn1dXQ==
\[\begin{tikzcd}
	{\mathsf{RInj}(\mathsf{Emb}_\flat)} && {\mathsf{RInj}(\mathsf{Emb}_\beta)} \\
	& {\mathsf{CohTopoi}_{\mathcal{S}}} \\
	& {\mathsf{CohTopoi}} \\
	& {\mathsf{Topoi}}
	\arrow["{\mathbb{U}_\beta^\flat}"{description}, from=1-1, to=1-3]
	\arrow[dashed, hook', from=2-2, to=1-1]
	\arrow[dashed, hook, from=2-2, to=1-3]
	\arrow[from=2-2, to=3-2]
	\arrow[from=3-2, to=4-2]
	\arrow[curve={height=-12pt}, from=1-3, to=4-2]
	\arrow[curve={height=12pt}, from=1-1, to=4-2]
\end{tikzcd}\]


\section{The emergence of ultrastructures à la Makkai} \label{sec2}
At the moment there exist three approaches to ultracategories and ultrastructures in the literature \cite{makkai1987stone,lurieultracategories,marmolejo1995ultraproducts}. The aim of this section is to re-frame the existing knowledge in a cleaner setup and describe a new approach to the topic. Besides the technical choices of the authors, an ultrastructure on an category is meant to ackwledge the fact that, given a coherent theory $\mathbb{T}$, a set $X$, and an ultrafilter $U \in \beta(X)$, one can define the functor \[\int_X (-)dU: \mathsf{Mod}(\mathbb{T})^X \to \mathsf{Mod}(\mathbb{T}).\] 

Such functor takes an $X$-indexed family of models and computes the ultraproduct of those along the ultrafilter $U$. Our first task will be to reconstruct this functor from a more conceptual point of view, and see how it fits together with the theory of coherent topoi and injectivity with respect to flat embeddings.


\begin{constr}[The emergence of ultrastructures for $\beta$-complete topoi] \label{emergentultrastructures}
Consider now a map in $\mathsf{Emb}_\beta$ from \Cref{classes}. We shall recall some notation for the sake of readability. Let $X$ be a set and $i_X: X \to \beta(X)$ be the inclusion of $X$ (seen as a discrete space) in its space of ultrafilters. At the level of frames, this correspond to the morphism $i_{\mathcal{P}X} : \mathcal{P}X \to \mathsf{Idl}(\mathcal{P}(X))$ from \Cref{exampleindcompl} (where $\mathcal{P}X$ the powerset of $X$). Then, the corresponding flat geometric morphism, $\mathsf{Sh}(\mathcal{P}X) \to \mathsf{Sh}(\mathsf{Idl}(\mathcal{P}X))$ is nothing but the geometric morphism, \[i_X: \mathsf{Set}^X \to \mathsf{Sh}(\beta(X)).\] Now let $\mathcal{E}$ be a $\beta$-complete topos and consider the diagram below. Of course, recall that coherent topoi are indeed $\beta$-complete as we showed in the previous section.

% https://q.uiver.app/?q=WzAsMyxbMSwwLCJcXG1hdGhjYWx7RX0iXSxbMCwwLCJcXG1hdGhzZntTZXR9XlgiXSxbMCwxLCJcXG1hdGhzZntTaH0oXFxiZXRhKFgpKSJdLFsxLDAsImYiXSxbMSwyLCJpIiwyXSxbMiwwLCJpX1xcc2hhcnAoZikiLDIseyJjdXJ2ZSI6Miwic3R5bGUiOnsiYm9keSI6eyJuYW1lIjoiZGFzaGVkIn19fV1d
\[\begin{tikzcd}
	{\mathsf{Set}^X} & {\mathcal{E}} \\
	{\mathsf{Sh}(\beta(X))}
	\arrow["f", from=1-1, to=1-2]
	\arrow["i_X"', from=1-1, to=2-1]
	\arrow["{i_\sharp(f)}"', curve={height=12pt}, dashed, from=2-1, to=1-2]
\end{tikzcd}\]

We claim that the family of maps $i^X_\sharp$, with $X$ a set, essentially account for the construction of ultrapowers. While our approach is original, this observation is not entirely new to the literature, and indeed Marmolejo's PhD thesis \cite{marmolejo1995ultraproducts} is built on this intuition. Before we give any definition of ultracategory or ultrastructure, we shall clarify the intution we just suggested, i.e. we shall relate, at least informally, $i^X_\sharp$ to the construction of ultrapowers. To do so, consider the diagram below (where we avoid all the $X$-dependencies as they play no role).


% https://q.uiver.app/?q=WzAsNyxbNCwwLCJcXG1hdGhjYWx7RX0iXSxbMywwLCJcXG1hdGhzZntTZXR9XlgiXSxbMywxLCJcXG1hdGhzZntTaH0oXFxiZXRhKFgpKSJdLFsxLDAsIlgiXSxbMSwxLCIoXFxiZXRhKFgpLFxcbWF0aHNme1N0b25lfSkiXSxbMCwxLCIoXFxiZXRhKFgpLFxcbWF0aHNme2Rpc2N9KSJdLFsyLDEsIlxcbWF0aHNme1NldH1ee1xcYmV0YShYKX0iXSxbMSwwLCJmIl0sWzEsMiwiaSJdLFsyLDAsImlfXFxzaGFycChmKSIsMix7ImN1cnZlIjoyLCJzdHlsZSI6eyJib2R5Ijp7Im5hbWUiOiJkYXNoZWQifX19XSxbNSw0LCJxIiwyLHsic3R5bGUiOnsiaGVhZCI6eyJuYW1lIjoiZXBpIn19fV0sWzMsNSwiaiIsMl0sWzMsNCwiaSJdLFsxLDYsImoiLDJdLFs2LDIsInEiLDJdXQ==
\[\begin{tikzcd}
	& X && {\mathsf{Set}^X} & {\mathcal{E}} \\
	{(\beta(X),\mathsf{disc})} & {(\beta(X),\mathsf{Stone})} & {\mathsf{Set}^{\beta(X)}} & {\mathsf{Sh}(\beta(X))}
	\arrow["f", from=1-4, to=1-5]
	\arrow["i", from=1-4, to=2-4]
	\arrow["{i_\sharp(f)}"', curve={height=12pt}, dashed, from=2-4, to=1-5]
	\arrow["q"', two heads, from=2-1, to=2-2]
	\arrow["j"', from=1-2, to=2-1]
	\arrow["i", from=1-2, to=2-2]
	\arrow["j"', from=1-4, to=2-3]
	\arrow["q"', from=2-3, to=2-4]
\end{tikzcd}\]

By $(\beta(X),\mathsf{Stone})$ we simply mean the usual topology on the space of ultrafilters, which corresponds to the ideal completion of the powerset of $X$. By inspecting this diagram, we can look at what happens at the level of hom-categories.

% https://q.uiver.app/?q=WzAsNixbMSwxLCJcXG1hdGhzZntUb3BvaX0oXFxtYXRoc2Z7U2V0fV5YLCBcXG1hdGhjYWx7RX0pIl0sWzEsMCwiXFxtYXRoc2Z7VG9wb2l9KFxcbWF0aHNme1NofShcXGJldGEoWCkpLCBcXG1hdGhjYWx7RX0pIl0sWzAsMSwiXFxtYXRoc2Z7VG9wb2l9KFxcbWF0aHNme1NldH0sIFxcbWF0aGNhbHtFfSleWCJdLFswLDAsIlxcbWF0aHNme3B0fShcXG1hdGhjYWx7RX0pXlgiXSxbMiwwLCJcXG1hdGhzZntUb3BvaX0oXFxtYXRoc2Z7U2V0fV57XFxiZXRhKFgpfSwgXFxtYXRoY2Fse0V9KSJdLFsyLDEsIlxcbWF0aHNme3B0fShcXG1hdGhjYWx7RX0pXntcXGJldGEoWCl9Il0sWzIsMCwiXFxzaW1lcSIsMix7InN0eWxlIjp7ImhlYWQiOnsibmFtZSI6Im5vbmUifX19XSxbMiwzLCJcXHNpbWVxIiwwLHsic3R5bGUiOnsiaGVhZCI6eyJuYW1lIjoibm9uZSJ9fX1dLFsxLDQsInFeXFxzaGFycCJdLFs0LDUsIlxcc2ltZXEiLDIseyJzdHlsZSI6eyJoZWFkIjp7Im5hbWUiOiJub25lIn19fV0sWzAsMSwiaV97XFxzaGFycH0iXV0=
\[\begin{tikzcd}
	{\mathsf{pt}(\mathcal{E})^X} & {\mathsf{Topoi}(\mathsf{Sh}(\beta(X)), \mathcal{E})} & {\mathsf{Topoi}(\mathsf{Set}^{\beta(X)}, \mathcal{E})} \\
	{\mathsf{Topoi}(\mathsf{Set}, \mathcal{E})^X} & {\mathsf{Topoi}(\mathsf{Set}^X, \mathcal{E})} & {\mathsf{pt}(\mathcal{E})^{\beta(X)}}
	\arrow["\simeq"', no head, from=2-1, to=2-2]
	\arrow["\simeq", no head, from=2-1, to=1-1]
	\arrow["{q^\sharp}", from=1-2, to=1-3]
	\arrow["\simeq"', no head, from=1-3, to=2-3]
	\arrow["{i_{\sharp}}", from=2-2, to=1-2]
\end{tikzcd}\]

Altogether, and with a bit of abuse of notation that ignores the equivalence of categories, we obtain a functor

\begin{equation} \label{ultrastructure}
    q^\sharp_X i_\sharp^X: \mathsf{pt}(\mathcal{E})^X \to \mathsf{pt}(\mathcal{E})^{\beta(X)}.
\end{equation}

If we now transpose this functor, we obtain the pairing below, which we shall denote suggestively by an integral notation,


\begin{equation} \label{ultrastructureoriginal}
    \int_X(-)d(-): \mathsf{pt}(\mathcal{E})^X \times \beta(X) \to \mathsf{pt}(\mathcal{E}).
\end{equation}

Now, to see that this construction matches the intuition we have provided, we choose a point of the topos $\mathsf{Sh}(\beta(X))$, which is the same as an ultrafilter $U \in \beta(X)$. We shall denote such a point $U: \mathsf{Set} \to \mathsf{Sh}(\beta(X))$. Then, it was observed by Marmolejo (and possibly by others before) that $(i_\sharp f) \circ U$ coincides with the ultraproduct $\frac{\Pi f}{U}$ of $f$ along $U$ (see for example the introduction of \cite{marmolejo2005locale} or the more extensive the discussion in \cite[Chap 2]{marmolejo1995ultraproducts}). Thus we obtain the chain of isomorphisms we wanted to show, \[\int f dU = (i_\sharp f) \circ U \cong  \frac{\Pi f}{U}.\]

The isomorphism above also clarifies the sense in which the ultrastructure is essentially accounted by $i^X_\sharp$, while $q^\sharp_X$ is evaluating the parametric ultraproduct along a specific ultrafilter.
\end{constr}


\begin{disc}
The general program of ultrastructures and ultracategories is to encapsulate the pairing above in an axiomatic way. 
Thus an ultrastructure on a category $\mathcal{A}$ should be a(n indexed family of) pairing(s),

\[\int_X(-)d(-): \mathcal{A}^X \times \beta(X) \to \mathcal{A}\]

that has all the perks and features of that introduced in \Cref{ultrastructureoriginal}. Taking inspiration from \Cref{emergentultrastructures} and particularly from \Cref{ultrastructure}, an ultrastructure on $\mathcal{A}$ may be a family of (pseudo)sections $\Sigma_X$ of the functors $i^\sharp_X: \mathcal{A}^{\beta{X}} \to \mathcal{A}^X$ for each set $X$, \[\Sigma_X: \mathcal{A}^X \to \mathcal{A}^{\beta{X}}.\]
\end{disc}


\begin{constr}[A $2$-dimensional aspect of ultrastructures] \label{2dimensional}
The problem of this approach is that, as we discussed above, the universal property of $\mathcal{E}$ with respect to the construction of ultraproducts is accounted for by the functor $i^X_\sharp$, and the composition in \Cref{ultrastructure} loses information by composing with $q^\sharp_X$. Let us be more precise. Recall that $i_\sharp(x)$ coincides with a right Kan extension, \[i^X_\sharp(x) \cong \mathsf{ran}_{i_X}x,\] and thus there is a $2$-dimensional aspect to its universal property in $\mathsf{Topoi}$. Our definition of ultrastructure must witness at least a trace of such universal property. Consider the example below,

% https://q.uiver.app/?q=WzAsNSxbMCwyLCJcXG1hdGhzZntTZXR9XlgiXSxbMCw0LCJcXG1hdGhzZntTaH0oXFxiZXRhKFgpKSJdLFsyLDAsIlxcbWF0aHNme1NldH1eWSJdLFsyLDIsIlxcbWF0aHNme1NofShcXGJldGEoWSkpIl0sWzQsMCwiXFxtYXRoY2Fse0V9Il0sWzIsNCwiZiJdLFsyLDMsImleWSIsMl0sWzAsMSwiaV5YIiwyXSxbMCwzLCJnIl0sWzMsNCwiaV5ZX1xcc2hhcnAoZikiLDJdLFsxLDMsImleWF9cXHNoYXJwKGcpIiwyXSxbMSw0LCJpXlhfXFxzaGFycChpXllfXFxzaGFycChmKWcpIiwyLHsib2Zmc2V0IjoyLCJjdXJ2ZSI6NX1dLFszLDExLCJcXERlbHRhX3tmZ31ee1hZfSIsMSx7InNob3J0ZW4iOnsidGFyZ2V0IjoyMH19XV0=
\[\begin{tikzcd}
	&& {\mathsf{Set}^Y} && {\mathcal{E}} \\
	\\
	{\mathsf{Set}^X} && {\mathsf{Sh}(\beta(Y))} \\
	\\
	{\mathsf{Sh}(\beta(X))}
	\arrow["f", from=1-3, to=1-5]
	\arrow["{i^Y}"', from=1-3, to=3-3]
	\arrow["{i^X}"', from=3-1, to=5-1]
	\arrow["g", from=3-1, to=3-3]
	\arrow["{i^Y_\sharp(f)}"', from=3-3, to=1-5]
	\arrow["{i^X_\sharp(g)}"', from=5-1, to=3-3]
	\arrow[""{name=0, anchor=center, inner sep=0}, "{i^X_\sharp(i^Y_\sharp(f)g)}"', shift right=2, curve={height=45pt}, from=5-1, to=1-5]
	\arrow["{\Delta_{fg}^{XY}}"{description}, shorten >=3pt, Rightarrow, from=3-3, to=0]
\end{tikzcd}\]

where $i_\sharp(g)$ exists because $\mathsf{Sh}(\beta(Y))$ is a coherent topos itself. Then, by the universal property of right Kan extensions, we obtain the natural transformation in the diagram,

\[\Delta^{XY}_{fg} : \mathsf{ran}_{i^Y}f  \circ \mathsf{ran}_{i^X}g  \Rightarrow \mathsf{ran}_{i^X}(\mathsf{ran}_{i^Y}f \circ g).\]

Collecting the data of these natural transformations we obtain the diagram below.

% https://q.uiver.app/?q=WzAsNSxbMCwwLCJcXGJldGEgWV5YIFxcdGltZXMgXFxtYXRoc2Z7cHR9KFxcbWF0aGNhbHtFfSleWSJdLFswLDIsIlxcYmV0YSBZXntcXGJldGEoWCl9IFxcdGltZXMgXFxtYXRoc2Z7cHR9KFxcbWF0aGNhbHtFfSlee1xcYmV0YSBZfSJdLFsyLDIsIlxcbWF0aHNme3B0fShcXG1hdGhjYWx7RX0pXntcXGJldGEgWH0iXSxbMSwwLCJcXGJldGEgWV57WH0gXFx0aW1lcyBcXG1hdGhzZntwdH0oXFxtYXRoY2Fse0V9KV57XFxiZXRhIFl9Il0sWzIsMCwiXFxtYXRoc2Z7cHR9KFxcbWF0aGNhbHtFfSleWCJdLFswLDEsImlfXFxzaGFycF5YIFxcdGltZXMgaV9cXHNoYXJwXlkiLDFdLFs0LDIsImlfXFxzaGFycF5YIiwxXSxbMCwzLCIxIFxcdGltZXMgaV9cXHNoYXJwXlkiXSxbMyw0LCJcXGNpcmNfe1xcYmV0YSBZfSJdLFsxLDIsIlxcY2lyY197XFxiZXRhIFl9IiwyXSxbMSw0LCJcXERlbHRhXntYWX0iLDEseyJsZXZlbCI6Mn1dXQ==
\[\begin{tikzcd}
	{\beta Y^X \times \mathsf{pt}(\mathcal{E})^Y} & {\beta Y^{X} \times \mathsf{pt}(\mathcal{E})^{\beta Y}} & {\mathsf{pt}(\mathcal{E})^X} \\
	\\
	{\beta Y^{\beta(X)} \times \mathsf{pt}(\mathcal{E})^{\beta Y}} && {\mathsf{pt}(\mathcal{E})^{\beta X}}
	\arrow["{i_\sharp^X \times i_\sharp^Y}"{description}, from=1-1, to=3-1]
	\arrow["{i_\sharp^X}"{description}, from=1-3, to=3-3]
	\arrow["{1 \times i_\sharp^Y}", from=1-1, to=1-2]
	\arrow["{\circ_{\beta Y}}", from=1-2, to=1-3]
	\arrow["{\circ_{\beta Y}}"', from=3-1, to=3-3]
	\arrow["{\Delta^{XY}}"{description}, Rightarrow, from=3-1, to=1-3]
\end{tikzcd}\]

% % https://q.uiver.app/?q=WzAsNCxbMCwwLCJcXG1hdGhzZntwdH0oXFxtYXRoY2Fse0V9KV5ZXFx0aW1lc1xcYmV0YShZKV5YXFx0aW1lcyBcXGJldGEoWCkiXSxbMSwzLCJcXG1hdGhzZntwdH0oXFxtYXRoY2Fse0V9KSJdLFsxLDAsIlxcbWF0aHNme3B0fShcXG1hdGhjYWx7RX0pXllcXHRpbWVzXFxiZXRhKFkpIl0sWzAsMywiXFxtYXRoc2Z7cHR9KFxcbWF0aGNhbHtFfSleWFxcdGltZXMgXFxiZXRhKFgpIl0sWzAsMiwiMSBcXHRpbWVzIFxcaW50X1goLSlkKC0pIl0sWzAsMywiIFxcaW50X1koLSlkKC0pICBcXHRpbWVzMSIsMl0sWzIsMSwiXFxpbnRfWSgtKWQoLSkiXSxbMywxLCJcXGludF9YKC0pZCgtKSIsMl0sWzIsMywiXFxEZWx0YV57WFl9X3soLSwtKX0iLDEseyJsZXZlbCI6Mn1dXQ==
% \[\begin{tikzcd}
% 	{\mathsf{pt}(\mathcal{E})^Y\times\beta(Y)^X\times \beta(X)} & {\mathsf{pt}(\mathcal{E})^Y\times\beta(Y)} \\
% 	\\
% 	\\
% 	{\mathsf{pt}(\mathcal{E})^X\times \beta(X)} & {\mathsf{pt}(\mathcal{E})}
% 	\arrow["{1 \times \int_X(-)d(-)}", from=1-1, to=1-2]
% 	\arrow["{ \int_Y(-)d(-)  \times1}"', from=1-1, to=4-1]
% 	\arrow["{\int_Y(-)d(-)}", from=1-2, to=4-2]
% 	\arrow["{\int_X(-)d(-)}"', from=4-1, to=4-2]
% 	\arrow["{\Delta^{XY}_{(-,-)}}"{description}, Rightarrow, from=1-2, to=4-1]
% \end{tikzcd}\]

At this stage we have no guarantee that the data of $\Sigma_X$ and $\Delta^{XY}$ is enough to recover the whole structure $\mathsf{pt}(\mathcal{E})$ is canonically equipped with, but we do know that this structure has at least two important features:
\begin{itemize}
    \item $\Sigma_X$ imitates $q^\sharp_X i_\sharp^X$ from \Cref{ultrastructure} for the construction of ultrapowers of models, and this is even more transparent when we transpose it as in \Cref{ultrastructureoriginal}.
    \item $\Delta^{XY}$ is a trace of the universal properties of $i^X_\sharp$, which at least recovers the interaction between the indexing sets of the ultrastructure.
\end{itemize}

In the next subsection we will use the discussion above as a motivation for our definition of ultrastructure $(\Sigma_X, \Delta^{XY})$ on an accessible category $\mathcal{A}$ with directed colimits. Before moving to that, we shall make a couple of comments to compare our discussion with the literature.
\end{constr}

\begin{disc}[Marmolejo's Łoś categories]
There are several ways to circumvent the probelm that the functor $q^\sharp_X i_\sharp^X$ in \Cref{emergentultrastructures} loses some information about the ultrastructure, and all these approaches are indeed related. Another way to keep track of the fact that a coherent topos is $\flat$-complete is due to Marmolejo \cite{marmolejo1995ultraproducts} and in its essence was (possibly independently) rediscovered by Lurie more recently \cite{lurieultracategories}. Let $\mathcal{E}$ be a topos and define, 
\begin{align*}
    \mathcal{E}^{\bullet} : & \mathsf{Sp}^\circ \to \mathsf{Cat} \\
    & X \mapsto \mathsf{Topoi}(\mathsf{Sh}(X),\mathcal{E})
\end{align*}
This derivator-like approach gives us a representation of topoi as $\mathsf{Sp}$-indexed categories and provides a $2$-functor, \[\mathsf{Topoi} \to \mathsf{Prestk}(\mathsf{Sp}).\]

It is clear that if $\mathcal{E}$ is a coherent topos, then for all ultrafinite maps $f$ (\Cref{generalities2}), $\mathcal{E}^\bullet(f)$ has a right adjoint (namely $f_\sharp$), and this could be chosen as a way to record the ultrastructure over $\mathcal{E}$. Marmolejo makes this choice in his PhD thesis \cite[Def 3.13]{marmolejo1995ultraproducts}, and defines a \textit{Łoś category}  to be precisely a $\mathsf{Sp}$-indexed category with the property above. We'll come back to Marmolejo's approach later and discuss the advantages and disadvantages of such choices. For the moment, just notice that a Łoś category is -- at least in principle -- a much less economical object then a triple $(\mathcal{A}, \Sigma_X, \Delta_{XY})$, for several reasons. For example, we have to specify a category for each topological space, whereas the triple $(\mathcal{A}, \Sigma_X, \Delta_{XY})$ specifies only one category.
\end{disc}

\begin{disc}[Lurie and Makkai]
Both Makkai and Lurie have an approach which is very much in the spirit of \Cref{emergentultrastructures} and \Cref{2dimensional} and to be more precise, Lurie's notion is essentially identical to the one we will propose later.


They both used this structure to show that every coherent topos can be recovered by its category of points plus the additional data of its ultrastructure. Notice that if we look at this result from the point of view of this paper, their result is very surprising for two reasons.

\begin{itemize}
    \item The maps of the form $\mathsf{Set}^X \to \mathsf{Sh}(\beta X)$ are, at least in principle, very few among flat geometric embeddings. Conceptual completeness seem to suggest that the $2$-functor \[\mathbb{U}_\beta^\flat: \mathsf{RInj}(\mathsf{Emb}_\flat) \to \mathsf{RInj}(\mathsf{Emb}_\beta)\] is a biequivalence of categories. Unfortunately we did not manage to show such a theorem.
    \item Differently from the localic case, there is no evidence that being $\beta$- or $\flat$-complete characterizes (co)retractions of coherent topoi.
\end{itemize}
\end{disc}


\section{Modeling Ultrastructures in Formal Model Theories} \label{sec3}

In this section we provide candidate definitions of ultracategory and discuss their relationships. As we mentioned several times since the opening of the paper, for us, an ultrastructure should imitate the construction of ultraproducts \[\int_X (-)dU: \mathsf{Mod}(\mathbb{T})^X \to \mathsf{Mod}(\mathbb{T})\] defined on the category of models of a coherent theory. This means in particular that an ultrastructure should be defined on something that looks like, behaves like and bites like a category of models of a geometric theory. As we have discussed in \cite{thlo}, there are several solutions available in the literature to  treat \textit{formally} categories of models of geometric theories. Among the options we can choose:

\begin{itemize}

    \item[$(\mathsf{Acc}_\omega)$] accessible categories with directed colimits, those offer the most classically developed approach to the topic, dating back to \cite{Makkaipare} and \cite{adamek_rosicky_1994}.
    \item[$(\mathsf{AEC})$] abstract elementary classes were popularized by Shelah \cite{shelah2009classification} in the model theory community as a general framework to have a syntax-independent treatment of categories of models of infinitary first order theories. This framework is a special case of that of accessible categories \cite{beke2012abstract}.
    \item[$(\mathsf{BIon})$] (bounded) ionads, this is a much newer framework, introduced by Garner \cite{ionads} with a geometric point of view, and studied from a more logical point of view in \cite{thlo}.
\end{itemize}

Older and/or less developed approaches can be seen through the lenses of formal model theory:

\begin{itemize}
    \item[($\mathsf{Cat}_{\mathsf{Sp}}$)] $\mathsf{Sp}$-indexed categories were used by  both Marmolejo and Lurie as an intermediate step to show properties of their ultracategories. This $2$-category deserves a separate study. At a phenomenological level we already have evidences that it is related to $\mathsf{Acc}_\omega$, see for example \cite[Prop. 6.14]{marmolejo1995ultraproducts}.
    \item[({$\mathsf{Grp}[\mathsf{Sp}]$})]  Topological groupoids and topological categories are classically understood to account for the Galois theory of a topos. Yet because a (large) topological groupoid can be seen as the groupoid of models of a topos equipped with the logical topology \cite{awodey2013first}, we see that those are perfect placeholders to performe formal model theory.
\end{itemize}

 In the discussion that follows we will instantiate definitions of ultracategory in some of those arenas of formal model theory: $\mathsf{Acc}_\omega, \mathsf{BIon}$ and $\mathsf{Acc}_\omega^{\mathsf{Sp}}$, which is a well behaved variant of $\mathsf{Cat}_{\mathsf{Sp}}$ in the list above.

\subsection{Ultrastructures in $\mathsf{Acc}_\omega$} \label{sec:ultrastructures}

\begin{defn}[Ultrastructures, ultracategories and ultrafunctors in $\mathsf{Acc}_\omega$] \label{def:ultracategory}
Let $\mathcal{A}$ be an accessible category with directed colimits. An \textit{ultrastructure} on $\mathcal{A}$ is specified by the following data:

\begin{itemize}
    \item[$(\Sigma, \epsilon)$] \textit{functorial ultraproducts}: a family (pseudo)sections $\Sigma_X: \mathcal{A}^X \to \mathcal{A}^{\beta{X}}$ of the functor $i^\sharp_X: \mathcal{A}^{\beta{X}} \to \mathcal{A}^X$ for each set $X$, accompanied by a natural isomorphism $\epsilon_X : i^\sharp_X\Sigma_X \stackrel{\sim}{\to} 1$. We shall use the notation below for the transpose of $\Sigma_X$, \[\int_X(-)d(-): \mathcal{A}^X \times \beta(X) \to \mathcal{A}.\]
    \item[$(\Delta)$] \textit{algebraic universality}: a family of natural transformations $\Delta^{XY}$ filling the diagrams below. We will write $\Delta^{XY} : \Sigma_Y b \circ \Sigma_X a \Rightarrow \Sigma_X(\Sigma_Y b \circ a). $
% https://q.uiver.app/?q=WzAsNSxbMCwwLCJcXGJldGEgWV5YIFxcdGltZXMgXFxtYXRoY2Fse0F9XlkiXSxbMCwyLCJcXGJldGEgWV57XFxiZXRhKFgpfSBcXHRpbWVzIFxcbWF0aGNhbHtBfV57XFxiZXRhIFl9Il0sWzIsMiwiXFxtYXRoY2Fse0F9XntcXGJldGEgWH0iXSxbMSwwLCJcXGJldGEgWV57WH0gXFx0aW1lcyBcXG1hdGhjYWx7QX1ee1xcYmV0YSBZfSJdLFsyLDAsIlxcbWF0aGNhbHtBfV5YIl0sWzAsMSwiXFxTaWdtYV9YIFxcdGltZXMgXFxTaWdtYV9ZIiwxXSxbNCwyLCJcXFNpZ21hX1giLDFdLFswLDMsIjEgXFx0aW1lcyBcXFNpZ21hX1kiXSxbMyw0LCJcXGNpcmNfe1xcYmV0YSBZfSJdLFsxLDIsIlxcY2lyY197XFxiZXRhIFl9IiwyXSxbMSw0LCJcXERlbHRhXntYWX0iLDEseyJsZXZlbCI6Mn1dXQ==
\[\begin{tikzcd}
	{\beta Y^X \times \mathcal{A}^Y} & {\beta Y^{X} \times \mathcal{A}^{\beta Y}} & {\mathcal{A}^X} \\
	\\
	{\beta Y^{\beta(X)} \times \mathcal{A}^{\beta Y}} && {\mathcal{A}^{\beta X}}
	\arrow["{\Sigma_X \times \Sigma_Y}"{description}, from=1-1, to=3-1]
	\arrow["{\Sigma_X}"{description}, from=1-3, to=3-3]
	\arrow["{1 \times \Sigma_Y}", from=1-1, to=1-2]
	\arrow["{\circ_{\beta Y}}", from=1-2, to=1-3]
	\arrow["{\circ_{\beta Y}}"', from=3-1, to=3-3]
	\arrow["{\Delta^{XY}}"{description}, Rightarrow, from=3-1, to=1-3]
\end{tikzcd}\]
\end{itemize}

The data above is subject to some compatibility axioms listed below.

\begin{enumerate}
    \item the pastings in the diagram below both yield a isomorphism, where $\mathsf{Mono}(X,Y)$ the set of injections from $X$ to $Y$.
    % https://q.uiver.app/?q=WzAsNyxbMSwwLCJcXGJldGEgWV5YIFxcdGltZXMgXFxtYXRoY2Fse0F9XlkiXSxbMSwyLCJcXGJldGEgWV57XFxiZXRhKFgpfSBcXHRpbWVzIFxcbWF0aGNhbHtBfV57XFxiZXRhIFl9Il0sWzIsMiwiXFxtYXRoY2Fse0F9XntcXGJldGEgWH0iXSxbMiwwLCJcXGJldGEgWV57WH0gXFx0aW1lcyBcXG1hdGhjYWx7QX1ee1xcYmV0YSBZfSJdLFsyLDEsIlxcbWF0aGNhbHtBfV5YIl0sWzMsMiwiXFxtYXRoY2Fse0F9XlgiXSxbMCwwLCJcXG1hdGhzZntNb25vfShYLFkpXFx0aW1lc1xcbWF0aGNhbHtBfV5ZIl0sWzAsMSwiXFxTaWdtYV9YIFxcdGltZXMgXFxTaWdtYV9ZIiwxXSxbNCwyLCJcXFNpZ21hX1giLDFdLFswLDMsIjEgXFx0aW1lcyBcXFNpZ21hX1kiXSxbMyw0LCJcXGNpcmNfe1xcYmV0YSBZfSJdLFsxLDIsIlxcY2lyY197XFxiZXRhIFl9IiwyXSxbMiw1LCJpXlxcc2hhcnAiXSxbMSwzLCJcXERlbHRhXntYWX0iLDEseyJsZXZlbCI6Mn1dLFs2LDAsImlfWSBcXHRpbWVzIDEiXV0=
\[\begin{tikzcd}
	{\mathsf{Mono}(X,Y)\times\mathcal{A}^Y} & {\beta Y^X \times \mathcal{A}^Y} & {\beta Y^{X} \times \mathcal{A}^{\beta Y}} \\
	&& {\mathcal{A}^X} \\
	& {\beta Y^{\beta(X)} \times \mathcal{A}^{\beta Y}} & {\mathcal{A}^{\beta X}} & {\mathcal{A}^X}
	\arrow["{\Sigma_X \times \Sigma_Y}"{description}, from=1-2, to=3-2]
	\arrow["{\Sigma_X}"{description}, from=2-3, to=3-3]
	\arrow["{1 \times \Sigma_Y}", from=1-2, to=1-3]
	\arrow["{\circ_{\beta Y}}", from=1-3, to=2-3]
	\arrow["{\circ_{\beta Y}}"', from=3-2, to=3-3]
	\arrow["{i^\sharp}", from=3-3, to=3-4]
	\arrow["{\Delta^{XY}}"{description}, Rightarrow, from=3-2, to=1-3]
	\arrow["{i_Y \times 1}", from=1-1, to=1-2]
\end{tikzcd}\]

    \item \textit{skew associativity}: by applying algebraic universality, the two possible ways to reduce the composition on the left to the one on the right produce the same map.
    
    \adjustbox{scale=0.8,center}{%
    % https://q.uiver.app/?q=WzAsNSxbMCwxLCJcXFNpZ21hX1ljIFxcY2lyYyBcXFNpZ21hX1ggYiBcXGNpcmMgXFxTaWdtYV9aYSJdLFsxLDAsIlxcU2lnbWFfWWMgXFxjaXJjICBcXFNpZ21hX1ooXFxTaWdtYV9YIGIgXFxjaXJjIGEpIl0sWzIsMCwiIFxcU2lnbWFfWihcXFNpZ21hX1ljIFxcY2lyYyBcXFNpZ21hX1ggYiBcXGNpcmMgYSkiXSxbMywxLCIgXFxTaWdtYV9aKFxcU2lnbWFfWChcXFNpZ21hX1ljIFxcY2lyYyAgYikgXFxjaXJjIGEpIl0sWzIsMiwiIFxcU2lnbWFfWChcXFNpZ21hX1ljIFxcY2lyYyBiKSBcXGNpcmMgXFxTaWdtYV9aYSJdLFswLDEsIiIsMix7ImN1cnZlIjotMiwibGV2ZWwiOjJ9XSxbMSwyLCIiLDIseyJsZXZlbCI6Mn1dLFsyLDMsIiIsMix7ImN1cnZlIjotMiwibGV2ZWwiOjJ9XSxbMCw0LCIiLDAseyJjdXJ2ZSI6MiwibGV2ZWwiOjJ9XSxbNCwzLCIiLDAseyJjdXJ2ZSI6MSwibGV2ZWwiOjJ9XV0=
    \begin{tikzcd}
	& {\Sigma_Yc \circ  \Sigma_Z(\Sigma_X b \circ a)} & { \Sigma_Z(\Sigma_Yc \circ \Sigma_X b \circ a)} \\
	{\Sigma_Yc \circ \Sigma_X b \circ \Sigma_Za} &&& { \Sigma_Z(\Sigma_X(\Sigma_Yc \circ  b) \circ a)} \\
	&& { \Sigma_X(\Sigma_Yc \circ b) \circ \Sigma_Za}
	\arrow[curve={height=-12pt}, Rightarrow, from=2-1, to=1-2]
	\arrow[Rightarrow, from=1-2, to=1-3]
	\arrow[curve={height=-12pt}, Rightarrow, from=1-3, to=2-4]
	\arrow[curve={height=12pt}, Rightarrow, from=2-1, to=3-3]
	\arrow[curve={height=6pt}, Rightarrow, from=3-3, to=2-4]
    \end{tikzcd}}





\end{enumerate}



 An \textit{ultracategory} in $\mathsf{Acc}_\omega$ is an object $\mathcal{A}$ equipped with an ultrastructure $(\Sigma^\mathcal{A}_\bullet, \Delta^{\bullet,\bullet})$. We will often blur the distinction between an ultracategory $(\mathcal{A}, \Sigma^\mathcal{A}_\bullet, \Delta^{\bullet,\bullet})$ with its underlying category $\mathcal{A}$ to simplify the notation.  An \textit{ultrafunctor} $f: \mathcal{A} \to \mathcal{B}$ is a functor $f: \mathcal{A} \to \mathcal{B}$ preserving directed colimits and such that the following diagram pseudocommutes for all sets $X$.

% https://q.uiver.app/?q=WzAsNCxbMSwxLCJcXG1hdGhjYWx7Qn1ee1xcYmV0YXtYfX0iXSxbMCwxLCJcXG1hdGhjYWx7QX1ee1xcYmV0YXtYfX0iXSxbMSwwLCJcXG1hdGhjYWx7Qn1eWCJdLFswLDAsIlxcbWF0aGNhbHtBfV5YIl0sWzMsMSwiXFxTaWdtYV57XFxtYXRoY2Fse0F9fV9YIiwyXSxbMiwwLCJcXFNpZ21hXntcXG1hdGhjYWx7Qn19X1giXSxbMywyLCJmXlgiXSxbMSwwLCJmXntcXGJldGEoWCl9IiwyXV0=
\[\begin{tikzcd}
	{\mathcal{A}^X} & {\mathcal{B}^X} \\
	{\mathcal{A}^{\beta{X}}} & {\mathcal{B}^{\beta{X}}}
	\arrow["{\Sigma^{\mathcal{A}}_X}"', from=1-1, to=2-1]
	\arrow["{\Sigma^{\mathcal{B}}_X}", from=1-2, to=2-2]
	\arrow["{f^X}", from=1-1, to=1-2]
	\arrow["{f^{\beta(X)}}"', from=2-1, to=2-2]
\end{tikzcd}\]
\end{defn}



\begin{defn}[The $2$-category $\mathsf{Ult}$ of Ultracategories] \label{def:ult}
Ultracategories, ultrafunctors and natural transformations form a $2$-category $\mathsf{Ult}$ equipped with a locally fully faithful forgetful $2$-functor, \[\mathbb{U}: \mathsf{Ult} \to \mathsf{Acc}_\omega.\]
\end{defn}

\begin{prop} \label{th:ptliftult}
Let $\mathcal{E}$ be a topos in $\mathsf{RInj}(\mathsf{Emb}_\beta)$. Then its category of points $\mathsf{pt}(\mathcal{E})$ admits an ultrastructure and every right injective morphism between those topoi induces an ultrafunctor. Thus, we obtain a lift of $2$-functor $\mathsf{pt}$ of points as below so that the diagram commutes,

% https://q.uiver.app/?q=WzAsNSxbMiwxLCJcXG1hdGhzZntBY2N9X1xcb21lZ2EiXSxbMSwxLCJcXG1hdGhzZntUb3BvaX0iXSxbMiwwLCJcXG1hdGhzZntVbHR9Il0sWzAsMCwiXFxtYXRoc2Z7Q29oVG9wb2l9X1xcbWF0aGNhbHtNfSJdLFsxLDAsIlxcbWF0aHNme1JJbmooXFxtYXRoc2Z7RW1ifV9cXGJldGEpfSJdLFszLDFdLFsyLDBdLFsxLDAsIlxcbWF0aHNme3B0fSIsMl0sWzQsMiwiIiwwLHsic3R5bGUiOnsiYm9keSI6eyJuYW1lIjoiZGFzaGVkIn19fV0sWzMsNF0sWzQsMV1d
\[\begin{tikzcd}
	{\mathsf{CohTopoi}_\mathcal{S}} & {\mathsf{RInj(\mathsf{Emb}_\beta)}} & {\mathsf{Ult}} \\
	& {\mathsf{Topoi}} & {\mathsf{Acc}_\omega}
	\arrow[from=1-1, to=2-2]
	\arrow[from=1-3, to=2-3]
	\arrow["{\mathsf{pt}}"', from=2-2, to=2-3]
	\arrow[dashed, from=1-2, to=1-3]
	\arrow[from=1-1, to=1-2]
	\arrow[from=1-2, to=2-2]
\end{tikzcd}\]
\end{prop}
\begin{proof}
Most of the proof follows directly from the discussion in \Cref{emergentultrastructures} and \Cref{2dimensional}. We should show that for $\mathcal{E}$ be $\beta$-complete topos, the ultrastructure we discussed verify the axioms (1) and (2) above. We shall only discuss (2), because (1) follows from a very similar reasoning and is much less interesting from a conceptual point of view. Going back to \Cref{2dimensional}, checking the skew associativity leads us to compare the two natural transformations below,

\adjustbox{scale=0.8,center}{%
% https://q.uiver.app/?q=WzAsNSxbMCwxLCJcXG1hdGhzZntyYW59X3tpXllfXFxzaGFycH1jIFxcY2lyYyBcXG1hdGhzZntyYW59X3tpXlhfXFxzaGFycH0gYiBcXGNpcmMgXFxtYXRoc2Z7cmFufV97aV5aX1xcc2hhcnB9YSJdLFsxLDAsIlxcbWF0aHNme3Jhbn1fe2leWV9cXHNoYXJwfWMgIFxcY2lyYyBcXG1hdGhzZntyYW59X3tpXlpfXFxzaGFycH0oXFxtYXRoc2Z7cmFufV97aV5YX1xcc2hhcnB9IGIgXFxjaXJjIGEpIl0sWzIsMCwiIFxcbWF0aHNme3Jhbn1fe2leWl9cXHNoYXJwfShcXG1hdGhzZntyYW59X3tpXllfXFxzaGFycH1jICBcXGNpcmNcXG1hdGhzZntyYW59X3tpXlhfXFxzaGFycH0gYiBcXGNpcmMgYSkiXSxbMywxLCIgXFxtYXRoc2Z7cmFufV97aV5aX1xcc2hhcnB9KFxcbWF0aHNme3Jhbn1fe2leWF9cXHNoYXJwfShcXG1hdGhzZntyYW59X3tpXllfXFxzaGFycH1jICBcXGNpcmMgYikgXFxjaXJjIGEpIl0sWzIsMiwiIFxcbWF0aHNme3Jhbn1fe2leWF9cXHNoYXJwfSAoXFxtYXRoc2Z7cmFufV97aV5ZX1xcc2hhcnB9YyBcXGNpcmMgYikgXFxjaXJjIFxcbWF0aHNme3Jhbn1fe2leWl9cXHNoYXJwfWEiXSxbMCwxLCIiLDIseyJjdXJ2ZSI6LTIsImxldmVsIjoyfV0sWzEsMiwiIiwyLHsibGV2ZWwiOjJ9XSxbMiwzLCIiLDIseyJjdXJ2ZSI6LTIsImxldmVsIjoyfV0sWzAsNCwiIiwwLHsiY3VydmUiOjIsImxldmVsIjoyfV0sWzQsMywiIiwwLHsiY3VydmUiOjEsImxldmVsIjoyfV1d
\begin{tikzcd}
	& {\mathsf{ran}_{i^Y_\sharp}c  \circ \mathsf{ran}_{i^Z_\sharp}(\mathsf{ran}_{i^X_\sharp} b \circ a)} & { \mathsf{ran}_{i^Z_\sharp}(\mathsf{ran}_{i^Y_\sharp}c  \circ\mathsf{ran}_{i^X_\sharp} b \circ a)} \\
	{\mathsf{ran}_{i^Y_\sharp}c \circ \mathsf{ran}_{i^X_\sharp} b \circ \mathsf{ran}_{i^Z_\sharp}a} &&& { \mathsf{ran}_{i^Z_\sharp}(\mathsf{ran}_{i^X_\sharp}(\mathsf{ran}_{i^Y_\sharp}c  \circ b) \circ a)} \\
	&& { \mathsf{ran}_{i^X_\sharp} (\mathsf{ran}_{i^Y_\sharp}c \circ b) \circ \mathsf{ran}_{i^Z_\sharp}a}
	\arrow[curve={height=-12pt}, Rightarrow, from=2-1, to=1-2]
	\arrow[Rightarrow, from=1-2, to=1-3]
	\arrow[curve={height=-12pt}, Rightarrow, from=1-3, to=2-4]
	\arrow[curve={height=12pt}, Rightarrow, from=2-1, to=3-3]
	\arrow[curve={height=6pt}, Rightarrow, from=3-3, to=2-4]
\end{tikzcd}
}
And indeed the fact that twose two paths coincide correspond to the skew associativity structure of Kan extensions. The fact that right Kan injective morphisms induce ultrafunctors is completely evident.
% % https://q.uiver.app/?q=WzAsNSxbMiwwLCJcXG1hdGhjYWx7RX0iXSxbMSwwLCJcXG1hdGhzZntTZXR9XlgiXSxbMSwxLCJcXG1hdGhzZntTaH0oXFxiZXRhKFgpKSJdLFswLDEsIlxcbWF0aHNme1N9KFxcYmV0YShYKSkiXSxbMCwwLCJcXG1hdGhzZntTfShYKSJdLFsxLDAsImYiXSxbMSwyLCJpIiwyXSxbMiwwLCJcXGludCBmZCgtKSIsMix7ImN1cnZlIjoyLCJzdHlsZSI6eyJib2R5Ijp7Im5hbWUiOiJkYXNoZWQifX19XSxbMywyLCJcXGVwc2lsb24iLDJdLFs0LDEsIlxcY29uZyJdLFs0LDMsIlxcbWF0aHNme1N9KGkpIiwyXV0=
% \[\begin{tikzcd}
% 	{\mathsf{S}(X)} & {\mathsf{Set}^X} & {\mathcal{E}} \\
% 	{\mathsf{S}(\beta(X))} & {\mathsf{Sh}(\beta(X))}
% 	\arrow["f", from=1-2, to=1-3]
% 	\arrow["i"', from=1-2, to=2-2]
% 	\arrow["{\int fd(-)}"', curve={height=12pt}, dashed, from=2-2, to=1-3]
% 	\arrow["\epsilon"', from=2-1, to=2-2]
% 	\arrow["\cong", from=1-1, to=1-2]
% 	\arrow["{\mathsf{S}(i)}"', from=1-1, to=2-1]
% \end{tikzcd}\]


% % https://q.uiver.app/?q=WzAsNixbMSwxLCJcXG1hdGhzZntUb3BvaX0oXFxtYXRoc2Z7U2V0fV5YLCBcXG1hdGhjYWx7RX0pIl0sWzEsMCwiXFxtYXRoc2Z7VG9wb2l9KFxcbWF0aHNme1NofShcXGJldGEoWCkpLCBcXG1hdGhjYWx7RX0pIl0sWzAsMSwiXFxtYXRoc2Z7VG9wb2l9KFxcbWF0aHNme1N9KFgpLCBcXG1hdGhjYWx7RX0pIl0sWzAsMCwiXFxtYXRoc2Z7QWNjfV9cXG9tZWdhKFgsXFxtYXRoc2Z7cHR9XFxtYXRoY2Fse0V9KSJdLFsyLDAsIlxcbWF0aHNme1RvcG9pfShcXG1hdGhzZntTfShcXGJldGEoWCkpLCBcXG1hdGhjYWx7RX0pIl0sWzIsMSwiXFxtYXRoc2Z7QWNjfV9cXG9tZWdhKFxcYmV0YShYKSxcXG1hdGhzZntwdH1cXG1hdGhjYWx7RX0pIl0sWzIsMCwiXFxzaW1lcSIsMix7InN0eWxlIjp7ImhlYWQiOnsibmFtZSI6Im5vbmUifX19XSxbMiwzLCJcXHNpbWVxIiwwLHsic3R5bGUiOnsiaGVhZCI6eyJuYW1lIjoibm9uZSJ9fX1dLFsxLDQsIlxcZXBzaWxvbl4qIiwyXSxbNCw1LCJcXHNpbWVxIiwyLHsic3R5bGUiOnsiaGVhZCI6eyJuYW1lIjoibm9uZSJ9fX1dLFswLDEsIlxcaW50ICgtKWQoLSkiXV0=
% \[\begin{tikzcd}
% 	{\mathsf{Acc}_\omega(X,\mathsf{pt}\mathcal{E})} & {\mathsf{Topoi}(\mathsf{Sh}(\beta(X)), \mathcal{E})} & {\mathsf{Topoi}(\mathsf{S}(\beta(X)), \mathcal{E})} \\
% 	{\mathsf{Topoi}(\mathsf{S}(X), \mathcal{E})} & {\mathsf{Topoi}(\mathsf{Set}^X, \mathcal{E})} & {\mathsf{Acc}_\omega(\beta(X),\mathsf{pt}\mathcal{E})}
% 	\arrow["\simeq"', no head, from=2-1, to=2-2]
% 	\arrow["\simeq", no head, from=2-1, to=1-1]
% 	\arrow["{\epsilon^*}"', from=1-2, to=1-3]
% 	\arrow["\simeq"', no head, from=1-3, to=2-3]
% 	\arrow["{\int (-)d(-)}", from=2-2, to=1-2]
% \end{tikzcd}\]


% % https://q.uiver.app/?q=WzAsNixbMSwxLCJcXG1hdGhzZntUb3BvaX0oXFxtYXRoc2Z7U2V0fV5YLCBcXG1hdGhjYWx7RX0pIl0sWzEsMCwiXFxtYXRoc2Z7VG9wb2l9KFxcbWF0aHNme1NofShcXGJldGEoWCkpLCBcXG1hdGhjYWx7RX0pIl0sWzAsMSwiXFxtYXRoc2Z7VG9wb2l9KFxcbWF0aHNme1N9KFgpLCBcXG1hdGhjYWx7RX0pIl0sWzAsMCwiXFxtYXRoc2Z7QWNjfV9cXG9tZWdhKFgsXFxtYXRoc2Z7cHR9XFxtYXRoY2Fse0V9KSJdLFsyLDAsIlxcbWF0aHNme1RvcG9pfShcXG1hdGhzZntTfShcXGJldGEoWCkpLCBcXG1hdGhjYWx7RX0pIl0sWzIsMSwiXFxtYXRoc2Z7QWNjfV9cXG9tZWdhKFxcYmV0YShYKSxcXG1hdGhzZntwdH1cXG1hdGhjYWx7RX0pIl0sWzIsMCwiXFxzaW1lcSIsMix7InN0eWxlIjp7ImhlYWQiOnsibmFtZSI6Im5vbmUifX19XSxbMiwzLCJcXHNpbWVxIiwwLHsic3R5bGUiOnsiaGVhZCI6eyJuYW1lIjoibm9uZSJ9fX1dLFsxLDQsIlxcZXBzaWxvbl4qIiwyXSxbNCw1LCJcXHNpbWVxIiwyLHsic3R5bGUiOnsiaGVhZCI6eyJuYW1lIjoibm9uZSJ9fX1dLFsxLDAsImleKiJdLFs0LDAsIlxcbWF0aHNme1N9KGkpXioiLDFdXQ==
% \[\begin{tikzcd}
% 	{\mathsf{Acc}_\omega(X,\mathsf{pt}\mathcal{E})} & {\mathsf{Topoi}(\mathsf{Sh}(\beta(X)), \mathcal{E})} & {\mathsf{Topoi}(\mathsf{S}(\beta(X)), \mathcal{E})} \\
% 	{\mathsf{Topoi}(\mathsf{S}(X), \mathcal{E})} & {\mathsf{Topoi}(\mathsf{Set}^X, \mathcal{E})} & {\mathsf{Acc}_\omega(\beta(X),\mathsf{pt}\mathcal{E})}
% 	\arrow["\simeq"', no head, from=2-1, to=2-2]
% 	\arrow["\simeq", no head, from=2-1, to=1-1]
% 	\arrow["{\epsilon^*}"', from=1-2, to=1-3]
% 	\arrow["\simeq"', no head, from=1-3, to=2-3]
% 	\arrow["{i^*}", from=1-2, to=2-2]
% 	\arrow["{\mathsf{S}(i)^*}"{description}, from=1-3, to=2-2]
% \end{tikzcd}\]

\end{proof}


% \begin{disc}[Łoś vs rough Łoś ]
% Every accessible Łoś category in the sense of Marmolejo admits a structure of rough Łoś category, ... \[\mathsf{ALos} \leadsto \mathsf{rLos} \]
% \end{disc}


% \begin{constr}[The topos of coordinates of a rough Łoś category]
% Recall that $\mathsf{Set}$ has a canonical structure of Łoś category, which we shall denote by $(\mathsf{Set}, \Sigma^\mathsf{Set}_\bullet)$. Let $(\mathcal{A}, \Sigma^\mathcal{A}_\bullet)$ be a Łoś category, then we define the topos of coordinates of $(\mathcal{A}, \Sigma^\mathcal{A}_\bullet)$ \[\mathsf{M}(\mathcal{A}, \Sigma^\mathcal{A}_\bullet):= \mathsf{rLos}((\mathcal{A}, \Sigma^\mathcal{A}_\bullet),(\mathsf{Set}, \Sigma^\mathsf{Set}_\bullet) ).\]
% \end{constr}

% \begin{prop}
% $\mathsf{M}(\mathcal{A}, \Sigma^\mathcal{A}_\bullet)$ is a topos.
% \end{prop}
% \begin{proof}
% Consider the (fully faithful) functor \[\mathbb{U}_{\mathcal{A},\mathsf{Set}}: \mathsf{rLos}((\mathcal{A}, \Sigma^\mathcal{A}_\bullet),(\mathsf{Set}, \Sigma^\mathsf{Set}_\bullet) ) \to \mathsf{Acc}_\omega(\mathcal{A}, \mathsf{Set}).\]
% Because $\mathsf{Acc}_\omega(\mathcal{A}, \mathsf{Set})$ is the Scott topos $\mathsf{S}(\mathcal{A})$ of $\mathcal{A}$, to show that  $\mathsf{M}(\mathcal{A},\Sigma^\mathcal{A}_\bullet)$ is a topos, it is enough to show that (a) $\mathbb{U}_{\mathcal{A},\mathsf{Set}}$ creates finite limits and (b) $\mathbb{U}_{\mathcal{A},\mathsf{Set}}$ creates colimits, (c) $\mathsf{rLos}((\mathcal{A}, \Sigma^\mathcal{A}_\bullet)$ has a strong generator.
% Indeed if we have (b) and (c), then $\mathsf{M}(\mathcal{A},\Sigma^\mathcal{A}_\bullet)$ is locally presentable, and thus $\mathbb{U}_{\mathcal{A},\mathsf{Set}}$ has a(n accessible) right adjoint. Using (a) it follows that $\mathsf{M}(\mathcal{A},\Sigma^\mathcal{A}_\bullet)$ is the category of coalgebras for an accessible lex comonad on a Grothendieck topos, and thus is a Grothendieck topos too. Let us now show (a),(b) and (c).

% \begin{itemize}
%     \item[(a\&b)]
%     \item[(c)]
% \end{itemize}

% \end{proof}

% \begin{cor}
% We have a geometric surjection $u: \mathsf{S}(\mathcal{A}) \twoheadrightarrow \mathsf{M}(\mathcal{A},\Sigma^\mathcal{A}_\bullet)$.
% \end{cor}

\begin{rem}[A comment on Lurie's ultracategories]
We should compare our notion to Lurie's one. Besides the fact that for us the underlying category of an ultracategory is an accessible category with directed colimits by design, our notion of ultracategory is very similar to Lurie's from \cite[Def. 1.3.1]{lurieultracategories}, the notation we chose makes the comparison easier. Notice that our \textit{skew associativity} essentially corresponds to the very mysterious axiom (C) in Lurie, while our axiom (1) plays the role of (A) and (B). We believe our treatment offers a cleaner approach with respect to Lurie's, this can be seen for example comparing our \Cref{emergentultrastructures} and especially \Cref{2dimensional} with his \cite[Exa. 1.3.8, which relies on 1.2.2, 1.2.8]{lurieultracategories} where we are essentially proving the same thing.
\end{rem}

\begin{rem} \label{newexamples}
All the examples of accessible ultracategories in Lurie are of the form $\mathsf{pt}(\mathcal{E})$ for $\mathcal{E}$ a coherent topos. The theorem above provides a new source of ultracategories and ultrastuctures. 
\end{rem}


\begin{rem}[Ultrafunctors and coherent objects] \label{rep1}
As we have seen in the proposition above, the notion of ultrafunctor is tightly connected to that of spartan morphism of coherent topoi. Consider a coherent topos $\mathcal{E}$ and a coherent object $e: \mathcal{E} \to \mathsf{Set}[\mathbb{O}]$ encoded as a geometric morphism into the classifier of the theory of objects. It is easy to see that $e$ is a spartan geometric morphism between right Kan injectives with respect to $\mathsf{Emb}_\beta$, and thus the corresponding category of points gives us an ultrafunctor, $\mathsf{pt}(e) :\mathsf{pt}(\mathcal{E}) \to \mathsf{Set}$ where $\mathsf{Set}$ has \textit{the canonical ultrastructure}. It follows that we have a functor, \[\mathsf{Coh}(\mathcal{E}) \to \mathsf{Ult}(\mathsf{pt}(\mathcal{E}),\mathsf{Set}).\]
Notice though, that the assumption that $e$ is coherent is crucial in order to obtain a spartan geometric morphism, and thus we cannot guarantee any functor $\mathcal{E} \to \mathsf{Ult}(\mathsf{pt}(\mathcal{E}),\mathsf{Set}).$ Ultrafunctors cannot recover the whole topos, at least not on the spot. This behavior is related to the fact that the $2$-category $\mathsf{CohTopoi}_\mathcal{S}$ is mostly related to the $2$-category of pretopoi, as opposed to the $2$-category of coherent topoi and geometric morphisms between them.
\end{rem}

% \begin{disc}[Towards an inverse construction]

% Building on the previous remark, we have a candidate construction for a $2$-functor \[\mathsf{Ult} \leadsto \mathsf{RInj(\mathsf{Emb}_\beta)}.\]  Indeed, given an ultracategory $(\mathcal{A}, \Sigma^\mathcal{A}_\bullet, \Delta^{\bullet,\bullet})$, we can define the category $\mathsf{Ult}(\mathcal{A},\mathsf{Set})$. Such a category is a full subcategory of the Scott topos of $\mathcal{A}$, \[\mathsf{Ult}(\mathcal{A},\mathsf{Set})\hookrightarrow \mathsf{S}(\mathcal{A}).\]
% The inclusion certainly preserve finite limits but it is not entirely clear to us why it should be a pretopos in full generality. A reasonable first attempt would be to show that the inclusion creates coproducts and quotients by equivalence relations. {\color{red} Lurie does this somehow.}
% \end{disc}

\subsubsection{Weak Ultrafunctors} \label{weak}

In the previous remark we found a representation of the pretopos of coherent objects via $\mathsf{Set}$-ultrafunctors, $\mathsf{Coh}(\mathcal{E}) \to \mathsf{Ult}(\mathsf{pt}(\mathcal{E}),\mathsf{Set})$. We shall discuss how to improve such embedding into a representation of the whole topos $\mathcal{E}$. Of course, we must pay the price of having a more flexible notion of ultrafunctor. The construction above was based on the fact that we can recover coherent objects by spartan geometric morphism,


\[\mathsf{Coh}(\mathcal{E}) \hookrightarrow \mathsf{Topoi}_\mathcal{S}(\mathcal{E},\mathsf{Set}[\mathbb{O}]) \stackrel{\mathsf{pt}}{\to} \mathsf{Ult}(\mathsf{pt}(\mathcal{E}), \mathsf{Set})).\]

In full generality, we recover all objects by studying all geometric morphisms $\mathcal{E} \to \mathsf{Set}[\mathbb{O}]$, but those will not induce ultrafunctors. Yet, we can still trace a footprint of the ultrastructure of a geometric morphism as follows.

\begin{constr} \label{weakultra}
Let $f :\mathcal{E} \to \mathcal{D}$ be any geometric morphism between topoi in $\mathsf{RInj}(\mathsf{Emb}_\beta)$. Following the general theory of Kan injectivity, by the pseudocommutativity of the diagram on the left, we get a Beck--Chevalley-like natural transformation in the diagram on the right. 

% https://q.uiver.app/?q=WzAsOCxbMCwwLCIoXFxtYXRoc2Z7U2V0fV5YLCBcXG1hdGhjYWx7RX0pIl0sWzEsMCwiKFxcbWF0aHNme1NldH1eWCwgXFxtYXRoY2Fse0R9KSJdLFswLDEsIihcXG1hdGhzZntTaH0oXFxiZXRhKFgpKSwgXFxtYXRoY2Fse0V9KSJdLFsxLDEsIihcXG1hdGhzZntTaH0oXFxiZXRhKFgpKSwgXFxtYXRoY2Fse0R9KSJdLFszLDEsIihcXG1hdGhzZntTaH0oXFxiZXRhKFgpKSwgXFxtYXRoY2Fse0R9KSJdLFsyLDEsIihcXG1hdGhzZntTaH0oXFxiZXRhKFgpKSwgXFxtYXRoY2Fse0V9KSJdLFszLDAsIihcXG1hdGhzZntTZXR9XlgsIFxcbWF0aGNhbHtEfSkiXSxbMiwwLCIoXFxtYXRoc2Z7U2V0fV5YLCBcXG1hdGhjYWx7RX0pIl0sWzIsMCwiaV5cXHNoYXJwX1xcbWF0aGNhbHtFfSJdLFszLDEsImleXFxzaGFycF9cXG1hdGhjYWx7RH0iLDJdLFswLDEsImZeXFxidWxsZXQiXSxbMiwzLCJmXlxcYnVsbGV0IiwyXSxbMiwxLCJcXGNvbmciLDEseyJzdHlsZSI6eyJib2R5Ijp7Im5hbWUiOiJub25lIn0sImhlYWQiOnsibmFtZSI6Im5vbmUifX19XSxbNSw0LCJmXlxcYnVsbGV0IiwyXSxbNiw0XSxbNyw1LCJpX1xcc2hhcnBeXFxtYXRoY2Fse0V9IiwyXSxbNiw0LCJpX1xcc2hhcnBeXFxtYXRoY2Fse0R9Il0sWzcsNiwiZl5cXGJ1bGxldCJdLFs1LDYsIlxccmhvIiwxLHsibGV2ZWwiOjIsInN0eWxlIjp7ImJvZHkiOnsibmFtZSI6ImRhc2hlZCJ9fX1dXQ==
\[\begin{tikzcd}
	{(\mathsf{Set}^X, \mathcal{E})} & {(\mathsf{Set}^X, \mathcal{D})} & {(\mathsf{Set}^X, \mathcal{E})} & {(\mathsf{Set}^X, \mathcal{D})} \\
	{(\mathsf{Sh}(\beta(X)), \mathcal{E})} & {(\mathsf{Sh}(\beta(X)), \mathcal{D})} & {(\mathsf{Sh}(\beta(X)), \mathcal{E})} & {(\mathsf{Sh}(\beta(X)), \mathcal{D})}
	\arrow["{i^\sharp_\mathcal{E}}", from=2-1, to=1-1]
	\arrow["{i^\sharp_\mathcal{D}}"', from=2-2, to=1-2]
	\arrow["{f^\bullet}", from=1-1, to=1-2]
	\arrow["{f^\bullet}"', from=2-1, to=2-2]
	\arrow["\cong"{description}, draw=none, from=2-1, to=1-2]
	\arrow["{f^\bullet}"', from=2-3, to=2-4]
	\arrow[from=1-4, to=2-4]
	\arrow["{i_\sharp^\mathcal{E}}"', from=1-3, to=2-3]
	\arrow["{i_\sharp^\mathcal{D}}", from=1-4, to=2-4]
	\arrow["{f^\bullet}", from=1-3, to=1-4]
	\arrow["\rho"{description}, Rightarrow, dashed, from=2-3, to=1-4]
\end{tikzcd}\]

Then, as it is observed in \cite{dlsolo}, $f$ is a morphism in $\mathsf{RInj}(\mathsf{Emb}_\beta)$ if and only if $\rho$ is an isomorphism. In full generality, though, we can go back to \Cref{emergentultrastructures} and look at the diagram below.


% https://q.uiver.app/?q=WzAsNixbMCwwLCIoXFxtYXRoc2Z7U2V0fV5YLCBcXG1hdGhjYWx7RX0pIl0sWzIsMCwiKFxcbWF0aHNme1NldH1eWCwgXFxtYXRoY2Fse0R9KSJdLFswLDIsIihcXG1hdGhzZntTaH0oXFxiZXRhKFgpKSwgXFxtYXRoY2Fse0V9KSJdLFsyLDQsIihcXG1hdGhzZntTZXR9XntcXGJldGEoWCl9LCBcXG1hdGhjYWx7RH0pIl0sWzIsMiwiKFxcbWF0aHNme1NofShcXGJldGEoWCkpLCBcXG1hdGhjYWx7RH0pIl0sWzAsNCwiKFxcbWF0aHNme1NldH1ee1xcYmV0YShYKX0sIFxcbWF0aGNhbHtFfSkiXSxbNCwzLCJxXlxcc2hhcnBfXFxtYXRoY2Fse0R9Il0sWzAsMSwiZl5cXGJ1bGxldCJdLFsyLDQsImZeXFxidWxsZXQiLDFdLFsyLDUsInFeXFxzaGFycF9cXG1hdGhjYWx7RX0iXSxbNSwzLCJmXlxcYnVsbGV0IiwyXSxbMCwyLCJpX1xcc2hhcnBeXFxtYXRoY2Fse0V9IiwyXSxbMSw0LCJpX1xcc2hhcnBeXFxtYXRoY2Fse0V9Il0sWzIsMSwiXFxyaG8iLDEseyJsZXZlbCI6Mn1dLFswLDUsIlxcU2lnbWFeXFxtYXRoY2Fse0V9X1giLDIseyJjdXJ2ZSI6NX1dLFsxLDMsIlxcU2lnbWFfWF5cXG1hdGhjYWx7RH0iLDAseyJjdXJ2ZSI6LTV9XV0=
\[\begin{tikzcd}
	{(\mathsf{Set}^X, \mathcal{E})} && {(\mathsf{Set}^X, \mathcal{D})} \\
	\\
	{(\mathsf{Sh}(\beta(X)), \mathcal{E})} && {(\mathsf{Sh}(\beta(X)), \mathcal{D})} \\
	\\
	{(\mathsf{Set}^{\beta(X)}, \mathcal{E})} && {(\mathsf{Set}^{\beta(X)}, \mathcal{D})}
	\arrow["{q^\sharp_\mathcal{D}}", from=3-3, to=5-3]
	\arrow["{f^\bullet}", from=1-1, to=1-3]
	\arrow["{f^\bullet}"{description}, from=3-1, to=3-3]
	\arrow["{q^\sharp_\mathcal{E}}", from=3-1, to=5-1]
	\arrow["{f^\bullet}"', from=5-1, to=5-3]
	\arrow["{i_\sharp^\mathcal{E}}"', from=1-1, to=3-1]
	\arrow["{i_\sharp^\mathcal{E}}", from=1-3, to=3-3]
	\arrow["\rho"{description}, Rightarrow, from=3-1, to=1-3]
	\arrow["{\Sigma^\mathcal{E}_X}"', curve={height=50pt}, from=1-1, to=5-1]
	\arrow["{\Sigma_X^\mathcal{D}}", curve={height=-50pt}, from=1-3, to=5-3]
\end{tikzcd}\]

By the pasting $q^\sharp_\mathcal{D}(\rho)$, we obtain a natural transformation that compares the two constructions of ultraproducts,

\[q^\sharp_\mathcal{D}(\rho): f^\bullet \Sigma^\mathcal{E}_X \Rightarrow  \Sigma^\mathcal{D}_X f^\bullet.\]

\end{constr}

\begin{rem}[Weak and Left ultrafunctors]\label{def:weak} \label{representation}
One can take the couple $(f, \delta_\bullet) : \mathcal{A} \to \mathcal{B}$, consisting of a functor $f$ together with a natural transformation $\delta_X :f^\bullet \Sigma^\mathcal{E}_X \Rightarrow  \Sigma^\mathcal{D}_X f^\bullet$ and subject to some compatibility axioms as a definition of \textit{weak} ultrafunctor between ultracategories. This would essentially amount to Lurie's notion of left ultrafunctor. This notion  allows for a representation of the whole topos $\mathcal{E}$, via a straightforward variation of the argument in \Cref{rep1}, \[\mathcal{E} \simeq \mathsf{Topoi}(\mathcal{E}, \mathsf{Set}[\mathbb{O}]) \to \mathcal{W}\mathsf{Ult}(\mathsf{pt}(\mathcal{E}), \mathsf{Set}).\] 

In this paper we shall not insist on weak ultrafunctors.
\end{rem}


% \begin{cor}
% Let $f: \mathcal{D} \to \mathcal{F}$ be a geometric morphism between topoi in $\mathsf{RInj}(\mathsf{Emb}_\beta)$. Then $(\mathsf{pt}(f), , q^\sharp_\mathcal{D}(\rho))$ from \Cref{weakultra} induces a weak ultrafunctor between their ultracategories of points.
% \end{cor}

% \begin{prop} \label{representation}
% Let $\mathcal{E}$ be a topos in $\mathsf{RInj}(\mathsf{Emb}_\beta)$. Then we have a functor 
% \end{prop}

\subsection{Ultraprofiles in $\mathsf{Acc}^\mathsf{Sp}_\omega$}

In this subsection we explore a new arena of formal model theory, namely \textit{accessible profiles}. The notion is inspired by Marmolejo's treatment of Łoś categories, and by Lurie's stack-y interpretation of ultrastructures \cite[Sec. 4]{lurieultracategories}. Besides this motivational overlap though, the whole subsection offers original material. We shall offer a biadjunction between the $2$-category of accessible profiles and the $2$-category of topoi which enhances the Scott adjunction \cite{thcat} and define a notion of \textit{ultraobject} for this formal model theory.
\subsubsection{Accessible profiles}
\begin{defn}[Accessible profile]
An \textit{accessible profile} is pseudofunctor $\mathcal{A}^\bullet : \mathsf{Sp}^\circ \to \mathsf{Acc}_\omega$. We shall call $\mathsf{Acc}^\mathsf{Sp}_\omega$ the $2$-category of accessible profiles, pseudonatural transformations and modifications between those. We may omit the adjective \textit{accessible} from now on. A profile is \textit{modest} if it small when thought as prestack of spaces, i.e. if it is a small weighted bicolimit of representables. We denonte $\underline{\mathsf{Acc}^\mathsf{Sp}_\omega}$ the full sub $2$-category of modest profiles.
\end{defn}

\begin{rem}[Representable profiles]
We have a Yoneda embedding,\[ \yo : \mathsf{Sp} \to \mathsf{Acc}^\mathsf{Sp}_\omega\] defined by $\yo(X)^Y = \mathsf{Sp}(Y,X)$. Indeed, when we equip $\mathsf{Sp}$ with the $2$-dimensional structure given by the specialization order, $\mathsf{Sp}(Y,X)$ always has directed colimits, and it is accessible because it is a small poset (with directed colimits). 
\end{rem}

\begin{constr}[The profile of models of a Grothendieck topos] \label{profileofmodels}
Let $\mathcal{E}$ be a Grothendieck topos, and define \[\mathbb{pt}(\mathcal{E})^\bullet := \mathsf{Topoi}(\mathsf{Sh}(\bullet), \mathcal{E}).\] $\mathbb{pt}(\mathcal{E})^\bullet$ clearly is a pseudofunctor $\mathsf{Sp}^\circ \to \mathsf{Acc}_\omega$, because indeed each $\mathsf{Topoi}(\mathsf{Sh}(X), \mathcal{E})$ is accessible and has directed colimits. This gives a $2$-functor, \[\mathbb{pt}^\bullet: \mathsf{Topoi} \to \mathsf{Acc}^\mathsf{Sp}_\omega.\]
\end{constr}

\begin{exa}[The accessible profile $\mathsf{Set}^\bullet$]
A very interesting topos to apply this construction to is the classifier of the theory of objects $\mathsf{Set}[\mathbb{O}]$, in this case we can explicitly compute $\mathbb{pt}^\bullet(\mathsf{Set}[\mathbb{O}])$ as follows, \[\mathbb{pt}^\bullet(\mathsf{Set}[\mathbb{O}]) = \mathsf{Topoi}(\mathsf{Sh}(\bullet), \mathsf{Set}[\mathbb{O}]) \simeq \mathsf{Sh}(\bullet).\]
In the rest of the text we will call this profile $\mathsf{Set}^\bullet$, to push the derivator-like intuition that any profile $\mathcal{A}^\bullet$ tells the story of its evaluation at $1$.
\end{exa}

\begin{rem}
There are several ways to look at functor $\mathbb{pt}^\bullet: \mathsf{Topoi} \to \mathsf{Acc}^\mathsf{Sp}_\omega$, we shall point out for the moment that it corresponds to the nerve of the $2$-functor $\mathsf{Sh}: \mathsf{Sp} \to \mathsf{Topoi}$ as depicted in the picture below.


% https://q.uiver.app/?q=WzAsMyxbMSwwLCJcXG1hdGhzZntTcGFjZXN9Il0sWzAsMSwiXFxtYXRoc2Z7VG9wb2l9Il0sWzIsMSwiXFxtYXRoc2Z7QWNjfV9cXG9tZWdhXntcXG1hdGhzZntTcH19Il0sWzAsMiwiXFx5byJdLFswLDEsIlxcbWF0aHNme1NofSIsMl0sWzEsMiwiXFxtYXRoYmJ7cHR9XlxcYnVsbGV0IiwyXV0=
\[\begin{tikzcd}
	& {\mathsf{Sp}} \\
	{\mathsf{Topoi}} && {\mathsf{Acc}_\omega^{\mathsf{Sp}}}
	\arrow["\yo", from=1-2, to=2-3]
	\arrow["{\mathsf{Sh}}"', from=1-2, to=2-1]
	\arrow["{\mathbb{pt}^\bullet}"', from=2-1, to=2-3]
\end{tikzcd}\]

Being a nerve functor, we feel optimistic that it could have a left adjoint, especially given that $\mathsf{Acc}_\omega^{\mathsf{Sp}}$ is almost a category of prestacks. Of course, as it often happens in category theory, \textit{the devil is in size issues}.
\end{rem}

\begin{thm} \label{newadjunction}
The $2$-functor $\mathbb{pt}$ has a (relative) left biadjoint $\Theta$, defined for all modest accessible profiles, i.e. for all modest profiles $\mathcal{A}^\bullet$ we have that $\mathsf{Topoi}(\Theta(\mathcal{A}^\bullet),\mathcal{E}) \simeq \mathsf{Acc}_\omega^{\mathsf{Sp}}(\mathcal{A}^\bullet, \mathbb{pt}^\bullet(\mathcal{E}))$.  

% https://q.uiver.app/?q=WzAsMyxbMCwwLCJcXHVuZGVybGluZXtcXG1hdGhzZntBY2N9XntcXG1hdGhzZntTcH19X1xcb21lZ2F9Il0sWzEsMCwiXFxtYXRoc2Z7VG9wb2l9Il0sWzAsMSwiXFxtYXRoc2Z7QWNjfV57XFxtYXRoc2Z7U3B9fV9cXG9tZWdhIl0sWzAsMSwiXFxUaGV0YSIsMCx7InN0eWxlIjp7ImJvZHkiOnsibmFtZSI6ImRhc2hlZCJ9fX1dLFswLDIsIiIsMix7InN0eWxlIjp7InRhaWwiOnsibmFtZSI6Imhvb2siLCJzaWRlIjoiYm90dG9tIn19fV0sWzEsMiwiXFxtYXRoYmJ7cHR9XlxcYnVsbGV0Il0sWzMsNSwiIiwwLHsibGV2ZWwiOjEsInN0eWxlIjp7Im5hbWUiOiJhZGp1bmN0aW9uIn19XV0=
\[\begin{tikzcd}
	{\underline{\mathsf{Acc}^{\mathsf{Sp}}_\omega}} & {\mathsf{Topoi}} \\
	{\mathsf{Acc}^{\mathsf{Sp}}_\omega}
	\arrow[""{name=0, anchor=center, inner sep=0}, "\Theta", dashed, from=1-1, to=1-2]
	\arrow[hook', from=1-1, to=2-1]
	\arrow[""{name=1, anchor=center, inner sep=0}, "{\mathbb{pt}^\bullet}", from=1-2, to=2-1]
	\arrow["\dashv"{anchor=center, rotate=-89}, draw=none, from=0, to=1]
\end{tikzcd}\]


\end{thm}
\begin{proof}
Recall that bicolimits and bilimits of accessible categories with directed colimits are computed in $\mathsf{Cat}$ (\cite{gregorybird} and \cite[5.8]{lack2023virtual}). Thus the theorem follows directly from the general theory of biKan extensions (\cite{di2022bi,descotte2018sigma}) and relative pseudoadjunctions \cite{fiore2018relative}.

% https://q.uiver.app/?q=WzAsNixbMSwwLCJcXG1hdGhzZntTcGFjZXN9Il0sWzIsMCwiXFxtYXRoc2Z7VG9wb2l9Il0sWzEsMSwiXFx1bmRlcmxpbmV7XFxtYXRoc2Z7UFN0a319KFxcbWF0aHNme1NwYWNlc30pIl0sWzEsMiwiXFxtYXRoc2Z7UFN0a30oXFxtYXRoc2Z7U3BhY2VzfSkiXSxbMCwxLCJcXHVuZGVybGluZXtcXG1hdGhzZntBY2N9XntcXG1hdGhzZntTcH19X1xcb21lZ2F9Il0sWzAsMiwiXFxtYXRoc2Z7QWNjfV57XFxtYXRoc2Z7U3B9fV9cXG9tZWdhIl0sWzAsMSwiXFxtYXRoc2Z7U2h9Il0sWzIsMSwiXFxUaGV0YSIsMSx7InN0eWxlIjp7ImJvZHkiOnsibmFtZSI6ImRhc2hlZCJ9fX1dLFsyLDMsIiIsMCx7InN0eWxlIjp7InRhaWwiOnsibmFtZSI6Imhvb2siLCJzaWRlIjoidG9wIn19fV0sWzAsMiwiXFx5byIsMl0sWzEsMywiXFxtYXRoYmJ7cHR9IiwxLHsiY3VydmUiOi0yfV0sWzQsMiwiIiwwLHsic3R5bGUiOnsidGFpbCI6eyJuYW1lIjoiaG9vayIsInNpZGUiOiJ0b3AifX19XSxbNSwzLCIiLDAseyJzdHlsZSI6eyJ0YWlsIjp7Im5hbWUiOiJob29rIiwic2lkZSI6InRvcCJ9fX1dLFs0LDUsIiIsMSx7InN0eWxlIjp7InRhaWwiOnsibmFtZSI6Imhvb2siLCJzaWRlIjoidG9wIn19fV0sWzAsNF0sWzcsMTAsIiIsMSx7ImxldmVsIjoxLCJzdHlsZSI6eyJuYW1lIjoiYWRqdW5jdGlvbiJ9fV1d
\[\begin{tikzcd}
	& {\mathsf{Sp}} & {\mathsf{Topoi}} \\
	{\underline{\mathsf{Acc}^{\mathsf{Sp}}_\omega}} & {\underline{\mathsf{PStk}}(\mathsf{Sp})} \\
	{\mathsf{Acc}^{\mathsf{Sp}}_\omega} & {\mathsf{PStk}(\mathsf{Sp})}
	\arrow["{\mathsf{Sh}}", from=1-2, to=1-3]
	\arrow[""{name=0, anchor=center, inner sep=0}, "\Theta"{description}, dashed, from=2-2, to=1-3]
	\arrow[hook, from=2-2, to=3-2]
	\arrow["\yo"', from=1-2, to=2-2]
	\arrow[""{name=1, anchor=center, inner sep=0}, "{\mathbb{pt}}"{description}, curve={height=-12pt}, from=1-3, to=3-2]
	\arrow[hook, from=2-1, to=2-2]
	\arrow[hook, from=3-1, to=3-2]
	\arrow[hook, from=2-1, to=3-1]
	\arrow[from=1-2, to=2-1]
	\arrow["\dashv"{anchor=center, rotate=-61}, draw=none, from=0, to=1]
\end{tikzcd}\]
\end{proof}

\begin{rem}[The topos of coordinates of a modest profile] \label{coordinates}

The existence of $\Theta$ in the theorem above follow from the general theorem of biKan extensions and is computed by the formula \[\Theta = \mathsf{biLan}_{\yo} (\mathsf{Sh}).\]
Of course, this may seem as a very unsatisfactory equation, which is in principle hard to compute. Yet, because of the very structure of the $2$-category of topoi, we do have a way to compute $\Theta$ by the chain of equivalences below.
\begin{align*}
   \Theta (\mathcal{A}^\bullet) \simeq & \mathsf{Topoi} (\Theta (\mathcal{A}^\bullet), \mathsf{Set}[\mathbb{O}]) \\
        \simeq & \mathsf{Acc}_\omega^\mathsf{Sp}(\mathcal{A}^\bullet, \mathbb{pt}^\bullet(\mathsf{Set}[\mathbb{O}])) \\
       \simeq  & \mathsf{Acc}_\omega^\mathsf{Sp}(\mathcal{A}^\bullet, \mathsf{Set}^\bullet). 
\end{align*}
This remark gives an \textit{a posteriori} proof of the fact that the category $\mathsf{Acc}_\omega^\mathsf{Sp}(\mathcal{A}^\bullet, \mathsf{Set}^\bullet)$ is a Grothendieck topos for every modest profile. A priori, we have a faithful forgetful functor $\mathsf{Acc}_\omega^\mathsf{Sp}(\mathcal{A}^\bullet, \mathsf{Set}^\bullet) \to \mathsf{Acc}_\omega(\mathcal{A}^1, \mathsf{Set})$. With quite some work, one can show that such a functor preserve colimits and finite limits, and yields a geometric surjection $\mathsf{S}(\mathcal{A}^1) \twoheadrightarrow \mathbb{O}(\mathcal{A}^\bullet)$ where $\mathsf{S}(\mathcal{A}^1)$ is the Scott topos of $\mathcal{A}^1$ in the sense of \cite{thcat}, but this would involve a very delicate analysis to find a generator of $\mathbb{O}(\mathcal{A}^\bullet)$. Lurie does something similar to this in \cite[5.4.5]{lurieultracategories}.
\end{rem}

\begin{rem}[The Counit as a representation] \label{reprensentation}
Let $\mathcal{E}$ be a topos, and assume that $\mathbb{pt}^\bullet(\mathcal{E})$ is modest. Then, we have a counit $\epsilon: \Theta \mathbb{pt}(\mathcal{E}) \to \mathcal{E}$. Let us spell out explicitly the inverse image of such functor,
\[\epsilon_{\mathcal{E}}^*: \mathcal{E} \to \mathsf{Acc}_\omega^\mathsf{Sp}(\mathbb{pt}^\bullet(\mathcal{E}), \mathsf{Set}^\bullet).\]
The functor $\epsilon$ represents the usual \textit{test functor} to check whether we can recover the topos $\mathcal{E}$ from its \textit{(indexted) category of (topological) points}, similarly to \cite{thlo} or Makkai/Lurie's conceptual completeness. 
\end{rem}

\subsubsection{Ultraprofiles}

\begin{defn}[Ultraprofile]
Let $i_X : X \to \beta(X)$ be the usual inclusion of a set $X$ in its space of ultrafilters. We say that a ultaprofile is a profile that is right Kan injective with respect to $\yo(i_X)$ for all $Xs$.
\end{defn}

% \begin{prop}
% Ultraprofiles are the same as Marmolejo's Łoś categories.
% \end{prop}

\begin{prop} \label{ultaprofilebuono}
Let $\mathcal{E}$ be a topos. $\mathcal{E}$ is $\beta$-complete if and only if its profile of points $\mathbb{pt}^\bullet(\mathcal{E})$ is an ultraprofile.
\end{prop}
\begin{proof}
Follows directly from the observation that the map $i_X: \mathsf{Set}^X \to \mathsf{Sh}(\beta(X))$ is nothing but $\Theta(\yo(i_X))$ and the following chain of equivalences, \[\mathsf{Topoi}(\mathsf{Set}^X,\mathcal{E}) \simeq \mathsf{Topoi}(\Theta(\yo(X))),\mathcal{E}) \simeq \mathsf{Acc}_\omega^{\mathsf{Sp}}(\yo(X), \mathbb{pt}^\bullet(\mathcal{E})).\]
\end{proof}
% \begin{rem}[Ultraprofile]

% \end{rem}
% We shall end this subsection by commenting on the representation functor 

% \[\mathcal{E} \to \mathcal{W}\mathsf{Ult}(\mathsf{pt}(\mathcal{E}), \mathsf{Set})\] from \Cref{representation}, which remained somewhat misterious in our treatment. For $(\mathcal{A}, \Sigma^\bullet)$ an ultracategory in the sense of the previous section, we can define a prestack $\mathsf{Spaces} \to \mathsf{Cat}$ given by the assignment \[(\mathcal{A}, \Sigma^\bullet) \mapsto \mathcal{W}\mathsf{Ult}(\beta(-),\mathcal{A}).\]

% Here by $\beta(X)$ we the ultracategory associated to the compact Hausdorff space $\beta(X)$. What Lurie is doing in the most technical theorems of his work, which we failed to reproduce here is to show that such correspondence yields a $2$-functor,
% \[\Psi: \mathcal{W}\mathsf{Ult} \to \mathsf{Acc}^\mathsf{Sp}_\omega\] 

% which is actually fully faithful, so that the diagram below is actually (pseudo)commutative. In this way conceptual completeness gets reduced to a property of the counit $\epsilon$.

% % https://q.uiver.app/?q=WzAsNCxbMiwwLCJcXG1hdGhjYWx7V31cXG1hdGhzZntVbHR9KFxcbWF0aHNme3B0fShcXG1hdGhjYWx7RX0pLFxcbWF0aHNme1NldH0pIl0sWzAsMSwiXFxtYXRoY2Fse0V9Il0sWzIsMiwiXFxtYXRoc2Z7QWNjfV57XFxtYXRoc2Z7U3B9fV9cXG9tZWdhKFxcbWF0aGJie3B0fV5cXGJ1bGxldChcXG1hdGhjYWx7RX0pLFxceW8oXFxtYXRoc2Z7U2V0fSkpIl0sWzIsMSwiXFxtYXRoc2Z7QWNjfV57XFxtYXRoc2Z7U3B9fV9cXG9tZWdhKFxcUHNpKFxcbWF0aHNme3B0fShcXG1hdGhjYWx7RX0pKSxcXFBzaShcXG1hdGhzZntTZXR9KSkiXSxbMCwzLCJcXFBzaSIsMl0sWzMsMiwiXFxjb25nIiwyXSxbMSwyLCJcXGVwc2lsb25eKiIsMix7ImN1cnZlIjo1fV0sWzEsMCwiIiwxLHsiY3VydmUiOi01fV1d
% \[\begin{tikzcd}
% 	&& {\mathcal{W}\mathsf{Ult}(\mathsf{pt}(\mathcal{E}),\mathsf{Set})} \\
% 	{\mathcal{E}} && {\mathsf{Acc}^{\mathsf{Sp}}_\omega(\Psi(\mathsf{pt}(\mathcal{E})),\Psi(\mathsf{Set}))} \\
% 	&& {\mathsf{Acc}^{\mathsf{Sp}}_\omega(\mathbb{pt}^\bullet(\mathcal{E}),\yo(\mathsf{Set}))}
% 	\arrow["\Psi"', from=1-3, to=2-3]
% 	\arrow["\cong"', from=2-3, to=3-3]
% 	\arrow["{\epsilon^*}"', curve={height=30pt}, from=2-1, to=3-3]
% 	\arrow[curve={height=-30pt}, from=2-1, to=1-3]
% \end{tikzcd}\]
% \end{constr}

\subsection{Ultraionads in $\mathsf{Bion}$}

The theory of ionad was introduced by Garner in \cite{ionads} and later developed in \cite{thgeo} and \cite{thlo}. Briefly, recall that a ionad is a category $C$ equipped with an lex comonad $\mathsf{Int}: \P(C) \to \P(C)$ on its category of small copresheaves that mimics the behavior of an interior operator. We refer to those papers for a proper introduction to the topic. In this subsection we shall discuss the notion of ultraionad and relate it to that of $\beta$-complete topoi. We start by recalling two useful results in the theory of ionads.

\begin{disc}[Ionads and spaces]
In \cite[Example 3.5(2)]{ionads}, Garner describes a general recipe to construct a buonded ionad $\Sigma X$ from a topological space $X$. If we look at $\mathsf{Sp}$ as a $2$-category, where the $2$-dimensional structure is given by the specialization order, this construction yields a $2$-adjunction,  \[ \Lambda : \mathsf{BIon} \leftrightarrows  \mathsf{Sp} : \Sigma, \]
which Garner describes in \cite[Example 4.8]{ionads}. Notice that the right adjoint is actually fully faithful, so that the $2$-category of spaces is a fully sub-$2$-category of bounded ionads.
\end{disc}

\begin{disc}[Ionads and topoi] \label{isbell}
In \cite[Sec. 3]{thgeo}, the author shows that there is a biadjunction between the $2$-category of topoi and the $2$-category of bounded ionads,
\[ \mathbb{O} : \mathsf{BIon}   \leftrightarrows  \mathsf{Topoi} :\mathbb{pt}. \]
This adjunction plays the same role of the adjunction relating topological spaces to locales, and indeed restricts to a biequivalence of $2$-categories between topoi with enough points and sober bounded ionads \cite[Sec 4.]{thgeo}.
\end{disc}

% % https://q.uiver.app/?q=WzAsMyxbNCwwLCJcXG1hdGhzZntUb3BvaX0iXSxbMiwwLCJcXG1hdGhzZntCSW9ufSJdLFswLDAsIlxcbWF0aHNme1NwYWNlc30iXSxbMSwyLCJcXExhbWJkYSIsMCx7Im9mZnNldCI6LTV9XSxbMiwxLCJcXFNpZ21hIiwwLHsib2Zmc2V0IjotNX1dLFsxLDAsIlxcbWF0aGJie099IiwyLHsib2Zmc2V0Ijo1fV0sWzAsMSwiXFxtYXRoYmJ7cHR9IiwyLHsib2Zmc2V0Ijo1fV0sWzMsNCwiIiwyLHsibGV2ZWwiOjEsInN0eWxlIjp7Im5hbWUiOiJhZGp1bmN0aW9uIn19XSxbNSw2LCIiLDIseyJsZXZlbCI6MSwic3R5bGUiOnsibmFtZSI6ImFkanVuY3Rpb24ifX1dXQ==
% \[\begin{tikzcd}
% 	{\mathsf{Spaces}} && {\mathsf{BIon}} && {\mathsf{Topoi}}
% 	\arrow[""{name=0, anchor=center, inner sep=0}, "\Lambda", shift left=5, from=1-3, to=1-1]
% 	\arrow[""{name=1, anchor=center, inner sep=0}, "\Sigma", shift left=5, from=1-1, to=1-3]
% 	\arrow[""{name=2, anchor=center, inner sep=0}, "{\mathbb{O}}"', shift right=5, from=1-3, to=1-5]
% 	\arrow[""{name=3, anchor=center, inner sep=0}, "{\mathbb{pt}}"', shift right=5, from=1-5, to=1-3]
% 	\arrow["\dashv"{anchor=center, rotate=90}, draw=none, from=0, to=1]
% 	\arrow["\dashv"{anchor=center, rotate=90}, draw=none, from=2, to=3]
% \end{tikzcd}\]

\begin{disc}[Clones of $\mathsf{Emb}_\beta$]
This whole paper is based on the study of the map $X \to \beta(X)$, and its instantiation in the $2$-categories of interests. We now collect three different instantiations of this morphism in the three $2$-categories that populate this section.
\begin{itemize}
    \item In $\mathsf{Sp}$, we find the class $\mathsf{emb}_\beta$, made of the maps $X \to \beta (X)$, where $X$ is a discrete set, and $\beta(X)$ is its space of ultrafilters. In a sense, we have been looking at this maps for the whole paper.
    \item In $\mathsf{BIon}$, we define $\mathbb{Emb}_\beta = \Sigma (\mathsf{emb}_\beta)$.
    \item In $\mathsf{Topoi}$, we have our class $\mathsf{Emb}_\beta$. It is easy to see that the following equations hold, \[\mathbb{O}(\mathbb{Emb}_\beta) \cong \mathsf{Emb}_\beta \quad \quad \mathbb{pt}(\mathsf{Emb}_\beta) = \mathbb{Emb}_\beta. \]
    
    \end{itemize}
\end{disc}





\begin{defn}[Ultraionads]
We define the $2$-category of ultraionads $\mathsf{UltBIon}$ to be $\mathsf{RInj}(\mathbb{Emb}_\beta)$.
\end{defn}


\begin{prop} \label{ionadsaregood}
\begin{itemize}
\item[]
    \item A topos $\mathcal{E}$ is $\beta$-complete if and only if its ionad of opoints $\mathbb{pt}(\mathcal{E})$ is an ultraionad.
    \item A ionad $(\mathcal{A}, \mathsf{Int})$ is an ultraionad if and only if $\mathbb{O}(\mathcal{A})$ is a $\beta$-complete topos.
\end{itemize} 
\end{prop}
\begin{proof}
The first claim follows directly from the observation that $\mathbb{O}(\mathbb{Emb}_\beta) \cong \mathsf{Emb}_\beta$ and the properties of adjunctions. Similarly, the second claim follows from the observation that $\mathbb{pt}(\mathsf{Emb}_\beta) = \mathbb{Emb}_\beta$ and the properties of adjunctions.
\end{proof}

% \begin{cor}
% The functor of points lifts to $\mathbb{pt}: \mathsf{CohTopoi}_\mathcal{S} \to \mathsf{UltBIon}.$
% \end{cor}
\newpage
\section{Conclusions and open questions} \label{sec4}

We shed light on two different topics, namely the geometric nature/behavior of coherent topoi and what structures come into play in the very definition of ultrastructure and ultracategory. 
(Un)fortunately this paper opens many more doors than it closes and we shall comment on those in this section.

\subsection{Geometry of coherent topoi}

\Cref{sec1} is mostly devoted to set the diagram below. % % https://q.uiver.app/?q=WzAsMyxbMSwwLCJcXG1hdGhzZntDb2hUb3BvaX1fe1xcbWF0aGNhbHtNfX0iXSxbMCwxLCJcXG1hdGhzZntSSW5qfShcXG1hdGhzZntFbWJ9X1xcZmxhdCkiXSxbMiwxLCJcXG1hdGhzZntSSW5qfShcXG1hdGhzZntFbWJ9X1xcYmV0YSkiXSxbMSwyLCJcXG1hdGhiYntVfV9cXGJldGFeXFxmbGF0IiwxXSxbMCwxLCJpIiwyLHsic3R5bGUiOnsidGFpbCI6eyJuYW1lIjoiaG9vayIsInNpZGUiOiJib3R0b20ifX19XSxbMCwyLCJqIiwwLHsic3R5bGUiOnsidGFpbCI6eyJuYW1lIjoiaG9vayIsInNpZGUiOiJ0b3AifX19XV0=
\[\begin{tikzcd}
	& {\mathsf{CohTopoi}_{\mathcal{M}}} \\
	{\mathsf{RInj}(\mathsf{Emb}_\flat)} && {\mathsf{RInj}(\mathsf{Emb}_\beta)}
	\arrow["{\mathbb{U}_\beta^\flat}"{description}, from=2-1, to=2-3]
	\arrow["i"', hook', from=1-2, to=2-1]
	\arrow["j", hook, from=1-2, to=2-3]
\end{tikzcd}\]
    As we hinted in \Cref{characterisation} and \Cref{char2}, we do not consider this to be a definitive account on flat geometric morphisms and $\flat$-completeness. Let us list below some of the doors that this paper leaves open.
    
\begin{itemize}
    \item a general theory of spartan geometric morphisms and flat geometric morphisms (and their interaction).
    \item can we find a smaller and easier to understand class of flat geometric morphism whose saturation is the whole $\mathsf{Emb}_\flat$? 
    \item there is an interesting class of geometric morphisms that class that we haven't studied and seems a good candidate to saturate $\mathsf{Emb}_\flat$. This is the class given in \Cref{exampleindcompl} which sits in between $\mathsf{Emb}_\beta$ and $\mathsf{Emb}_\flat$. The relevance of this class for our theory is witnessed by \Cref{char2}.
    \item is $\mathsf{RInj}(\mathsf{Emb}_\beta)$ the completion of $\mathsf{CohTopoi}_{\mathcal{M}}$ under coreflections?
    \item in this spirit of and referencing to the discussion in \Cref{characterisation}, a preliminary question to answer would be which topoi embed in a coherent one?
\end{itemize}


All these questions seem complelling to us, and would impact on a possible retuning of the notion of ultracategory. 


\subsection{Formal model theories and ultraobjects}

On the side of formal model theory, even more questions are left to the reader. In a sense, the questions above are all preliminary to a good understanding of ultra-objects, as they shape our definitions. Yet, besides the questions about ultrastructures and ultracategories, other interesting topics emerge from this paper and would deserve some attention.

\begin{itemize}
    \item going back to \Cref{profileofmodels}, we wonder whether $\mathbb{pt}^\bullet$ is always modest. Some results obtained by Lurie on accessible ultrastructures seem to suggest that this is not such a strange conjecture. Of course, if $\mathbb{pt}^\bullet$ happens to be always modest, \Cref{newadjunction} ends up being an honest bi-adjunction.
    \item Speaking of \Cref{newadjunction}, is this biadjunction (pseudo)idempotent in the spirit of \cite{thgeo}?
    \item What is the relationship between the models of formal model theory we have discussed?
\end{itemize}
\newpage
\section*{Acknowledgements}
I am indebted with \textit{Morgan Rogers} for all the discussions we had in the last year on a somewhat related project, those have informed and educated my knowledge of topos theory to a completely new level. I hope our collaboration will eventually see the light. I thank \textit{Axel Osmond} and \textit{Christian Espindola} for their inspiring comments on a preliminary draft of this paper. Finally I am grateful to \textit{Peter Lumsdaine} for showing interest for this research in private conversations.

\bibliography{thebib}
\bibliographystyle{alpha}
\end{document}

