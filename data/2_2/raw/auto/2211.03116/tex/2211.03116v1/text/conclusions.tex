\section{Conclusions}
\label{sec:conclusions}
In this paper, we studied the application of self-adaptation in industry. To that end, we conducted a questionnaire-based survey with practitioners from all over the world. We received valid responses from 184 participants, 100 of them with experience in engineering self-adaptive systems. 

By analysing the data, we contributed an empirically grounded overview of state-of-the-practice in the application of self-adaptation. A selection of key observations includes: i) self-adaptation is extensively applied in industry across a wide variety of domains, ii) the dominating types of adaptations applied in industry are auto-scaling, auto-tuning, and monitoring/analysis, iii) practitioners rely extensively on tools and infrastructure to realise the different functions of self-adaptation, iv) human supervision is important to ensure trust in industrial self-adaptive systems, v) about half of the participants encounter risks with applying self-adaptation, vi) on the other hand, about half of the practitioners would appreciate support from researchers to deal with problems they face. Figure\,\ref{fig:summary} summarises the main findings. 

\begin{figure}[h]
\centering
\includegraphics[width=1.0\columnwidth]{figures/summary-sas-in-industry.pdf}
\caption{Summary of the main findings of the survey}
\label{fig:summary}
\end{figure}
 
The results offer insights for researchers that enable them to  compare the focus their of their current research with industrial needs. A selection of related key insights includes: i) different from academics that study adaptation for mitigating uncertainty of classic maintenance tasks (self-*), practitioners also emphasise the importance of improving user satisfaction, reducing costs, and mitigating risks, ii) practitioners (in particular those of small and medium sized companies) rely on tools and infrastructure to realise self-adaptation, iii) ensuring trust in industrial self-adaptive systems is mainly achieved through extensive testing, runtime monitoring and alerting, and human supervision, iv) risks with self-adaptation in practice relate mainly to incorrect functionality, difficulty to manage environment uncertainty, 
degraded performance and increased cost.

The results also offer insights for practitioners to assess the level of their current practice in applying self-adaptation. A selection of related key insights includes: 
i) practitioners broadly confirm that the use of self-adaptation improves robustness and performance while reducing costs and required resources, and improves user experience while reducing the burden of engineers, ii) a wide range of mechanisms are used to enact self-adaptation in industrial systems, iii) tools and infrastructure, such as auto-scaling and container-orchestration platforms are available and commonly used to support the realisation of self-adaptation in practice, iv) important challenges when engineering self-adaptation in practice are reliable/optimal design, design complexity, and tuning/debugging, v) there is a relevant match between industrial practice in realising self-adaptation and the body of work performed by the research community of self-adaption.    

The survey results provide prospects for applying self-adaptation in practice and opportunities for industry-research collaborations in this area. The prospects include: i) realising full autonomous operation, ii) exploiting machine learning, iii) improving quality and security, and iv) applying self-adaptation for maintenance. Key opportunities for industry-research collaborations are in: i) consolidating best practices (architecture, patterns, and reuse), ii) modelling paths for the adoption of self-adaptation in industry, iii) supporting advanced features to realise self-adaptation such as dealing with the evolution of self-adaptive systems, iv) rigorous methods for ensuring trustworthiness of self-adaptive systems, v) governance of data, and vi) moving the human in the loop (performing adaptation functions) to the human on the loop (overseeing the system to ensure trust).  

We hope that the results of this survey will propel industry-relevant research in the field of self-adaptive systems and enhance the application of self-adaptation in practice, paving the way for self-adaptation to reach full maturity as a discipline. 