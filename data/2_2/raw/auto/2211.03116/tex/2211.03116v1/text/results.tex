\section{Results}
\label{sec:results}

\subsection{Demographic Information}
\label{subsec:demographic_information}
%
In total, 184 participants completed the survey from 355 invitations, i.e., a response rate of 51.8\%. 
%
Based on the answers to the first question (Q0.1), we split the answers of the other questions of the demographics in two groups: those provided by all participants and those provided by participants that worked with concrete self-adaptive systems.\footnote{While we selected participants that have the required expertise to answer questions (second criterion in Section\,\ref{subsec:population_sampling}),  this does not necessarily mean that they have worked (or are working) with concrete self-adaptive systems.}  


%\begin{itemize}
\subsubsection{Experience with self-adaptation (Q0.1):} Of the 184 participants that provided valid data, 100 (54.4\%) expressed to have worked with concrete self-adaptive systems.
%\footnote{The data of the participant that was excluded also worked with concrete self-adaptive systems, yet, that data was not useful.} 
\subsubsection{Software systems built by organisations (Q0.2):} Almost all participants (181, 98.4\%) provided a valid description of the kind of systems they build. 
Based on the analysis of the data we could classify the answers along two axes: the \textit{types} of software systems built by the organisations, and the \textit{focus} of the software systems. The type refers to the domain, while the focus refers to the activities on which the organisation concentrates within the domain. For example, automation (focus) within manufacturing (type). Note that the domain may be abstract, e.g., embedded systems or communication and networks. Focus on the other hand may refer to purpose, such as analytics, but also specific technologies or methods, such as machine learning. 

Figure~\ref{tab:q1-1-types}\footnote{Because the number of participants that worked with self-adaptive systems is 100 and all provided a valid description, the absolute numbers are also percentages. We also apply this to the data of the other questions unless differently stated.} summarises the types of systems we identified. The most frequent types are web/mobile, embedded, cyber-physical, IoT systems, data management, and cloud (together these four types represent 52.5\% of all systems). Sixteen participants (8.8\%) built various types of systems.\footnote{The option ''Various'' refers to different kind of systems. The option ''Others'' on the other hand refers to specific types of systems different from those listed in the table, e.g., a system for grading software at educational institutions.} The results show that the percentages of the types of systems of all participants and those that worked with self-adaptation are similar. 

In addition to the types of systems, 104 participants (56.5\%) also provided insights in the focus of the systems they build. Among the 100 participants that worked with self-adaptation, 60 provided a description of the focus. 
%Table~\ref{tab:contingency_q1-1-focus} 
Figure~\ref{tab:q1-1-focus} 
provides an overview the results. The dominant focus is monitoring, analytics and control, representing 27.4\% of the foci described by the participants. Other key foci are services (21.7\% of the participants that described the focus of the systems they built) and quality and security (14.2\%).  
%
Overall, the variety in the types of systems built by the participants and the different foci in activities underpins the representativeness of the data collected during the survey.  

\begin{comment}
    

\small
\begin{table}[h!]
\centering
\caption{Types of software systems build by organisations (Q0.2).}
\label{tab:contingency_q1-1-types}
\begin{tabular}{llc}
\hline\noalign{\smallskip}
Types of system %& Focus 
& All (of 184) &  With self-adaptation (of 100) \\
\noalign{\smallskip}\hline\noalign{\smallskip}
    Web/mobile & 35 (19.3\%) & 15 \\
    
    Embedded/CPS/IoT & 24 (13.3\%) & 13 \\
    
    Data management & 21 (11.6\%) & 10 \\
    
    Cloud &  15 (8.3\%) & 11 \\
    
    Transportation & 13 (7.2\%) & 9 \\
%& Multi-facet & 4 \\
% & Automation & 3 \\
% \noalign{\smallskip}\hline\noalign{\smallskip}
    Communication \& networks & 11 (6.1\%) & 7  \\
%& Multi-facet & 4 \\
% & Distributed ledger & 1 &\\
% & Cloud & 1 \\
% \noalign{\smallskip}\hline\noalign{\smallskip}
    IT infrastructure & 11 (6.1\%) & 7 \\
 %& Multi-facet & 3 \\
 % & Services & 2 \\
 %& Cloud & 1 \\
    Manufacturing & 10 (5.5\%) & 8 \\
 %& Multi-facet & 3 \\
 % & Services & 2 \\
 %& Cloud & 1 \\
    Finances & 8 (4.4\%) & 5 \\
 
    e-commerce & 7 (3.9\%) & 3 \\
     
     Public/society  & 3 (1.7\%) & 1 \\
     
     Others & 7 (3.9\%) & 3 \\
 
     Various & 16 (8.8\%) & 8 \\
\noalign{\smallskip}\hline
\end{tabular}
\end{table}
\normalsize

\end{comment}
    
\begin{figure}[h]
\centering
\includegraphics[width=\columnwidth]{Q0.2-Type-fig.pdf}\vspace{-10pt}%%%
\caption{Types of software systems build by organisations (Q0.2).}\vspace{-10pt}%%%
\label{tab:q1-1-types}
\end{figure}


\begin{comment}
    

\small
\begin{table}[hbt]
\centering
\caption{Focus of software systems build by organisations (Q0.2). %Percentages refer to participants that provided descriptions.
}
\label{tab:contingency_q1-1-focus}
\begin{tabular}{llc}
\hline\noalign{\smallskip}
Focus of system %& Focus 
& All (of 104) &  With self-adaptation (of 60) \\
\noalign{\smallskip}\hline\noalign{\smallskip}
    Monitoring/analytics/control & 29 (27.4\%) & 16 (26.7\%)\\
    
    Services & 23 (21.7\%) & 11 (18.3\%)\\
    
    Quality/security & 15 (14.2\%) & 9 (15.0\%)\\
    
    Management &  13 (12.3\%) & 7 (11.7\%)\\
    
    Apps & 10 (9.4\%) & 6 (10.0\%)\\
    Automation & 9 (8.5\%) & 5 (8.3\%)  \\
%& Multi-facet & 4 \\
% & Distributed ledger & 1 &\\
% & Cloud & 1 \\
% \noalign{\smallskip}\hline\noalign{\smallskip}
    Machine learning/AI & 7 (6.6\%) & 6 (10.0\%)\\
 %& Multi-facet & 3 \\
 % & Services & 2 \\
 %& Cloud & 1 \\
\noalign{\smallskip}\hline
\end{tabular}
\end{table}
\normalsize

\end{comment}

\begin{figure}[h]
\centering
\includegraphics[width=0.9\columnwidth]{Q0.2-Focus-fig.pdf}
\vspace{-10pt}%%%
\caption{Focus of software systems built by organisations (Q0.2).}\vspace{-10pt}%%%
\label{tab:q1-1-focus}
\end{figure}

\subsubsection{Software engineers working at companies (Q0.3):} Figure~\ref{fig:q0-3} summarises the results of the number of software engineers that work at the companies of the  participants. About half of the companies have more than 100 employees who work as software engineers. The other half is about equally divided over four categories of companies with between 1 and 100 software engineers. The results are similar for all participants and those that have worked with  self-adaptive systems. The numbers show that we collected data from participants of companies with a significant number of software engineers, i.e., people dedicated to building software-intensive systems (because our study is interested in the engineering of software-intensive systems, we collected the number of software engineers at the companies and not the total number of employees as a measure for size). 

\begin{comment}
    

\small
\begin{table}[hbt]
\centering
\caption{Size of companies (Q0.3).}
\label{tab:q0-2}
\begin{tabular}{llc}
\hline\noalign{\smallskip}
Size & All (0f 184) & With self-adaptation (of 100)\\
\noalign{\smallskip}\hline\noalign{\smallskip}
 1-10 & 24 (13.0\%) & 9 \\
11-20 & 17 (9.2\%) & 13 \\
 21-50 & 29 (15.8\%) & 11 \\
51-100 & 24 (13.0\%) & 14 \\
$>$ 100 & 90 (48.9\%) & 53  \\
\noalign{\smallskip}\hline
\end{tabular}
\end{table}
\normalsize

\end{comment}
    
\begin{figure}[h]
\centering
\includegraphics[width=0.7\columnwidth]{Q0.3-fig.pdf}\vspace{-10pt}%%%
\caption{Size of companies (Q0.3).}\vspace{-10pt}%%%
\label{fig:q0-3}
\end{figure}


\subsubsection{Roles of participants in their organisation (Q0.4):} The role(s) that participants have in their company are summarised in Figure~\ref{fig:q0-4}. Of 184 participants, 129 indicated that they have one role in their organisation. The other participants indicated that they have two or more roles. 
Overall, the participants reported on average 1.6 roles in their company. The participants that worked with self-adaptation reported on average 1.5 roles.  The most frequent roles are programmer and project manager/coordinator, each representing over 40\% of the participants. About one in three participants works as a designer or architect. The representation of the other roles is lower in the sample. The relative numbers for the roles of all participants and those that work with self-adaptive systems are again similar. One exception is researcher: 9 of the 10 participants that work as researcher have worked with self-adaptive systems. The results show that we collected data from participants with a broad range of key software engineering roles in industry. 

\begin{comment}
    


\small
\begin{table}[hbt]
\centering
\caption{Roles of participants in their companies (Q0.4).}
\label{tab:q0-3}
\begin{tabular}{llc}
\hline\noalign{\smallskip}
Role & All (of 184) & With self-adaptation (of 100) \\
\noalign{\smallskip}\hline\noalign{\smallskip}
Programmer & 84 (45.7\%) & 44 \\
Project manager/lead & 76 (41.3\%) & 40 \\
Designer/architect & 54 (29.3\%) & 28 \\
Maintainer & 18 (9.8\%) & 7 \\
Specific expert & 18 (9.8\%) & 8\\
Tester & 16 (8.7\%) & 8 \\
Operator & 10 (5.4\%) & 5 \\
Researcher & 10 (5.4\%) & 9 \\
\noalign{\smallskip}\hline
\end{tabular}
\end{table}%\vspace{-10pt}
\normalsize

\end{comment}

\begin{figure}[h]
\centering
\includegraphics[width=0.9\columnwidth]{Q0.4-fig.pdf}\vspace{-10pt}
\caption{Roles of participants in their companies (Q0.4).}
\label{fig:q0-4}
\end{figure}

\subsubsection{Experience of participants (Q0.5):} Figure~\ref{fig:q0-5} summarises the years of experience of participants as software engineers.\footnote{Expertise can be based on any role in relation to engineering software-intensive systems as shown in Table~\ref{tab:q0-3}.} A majority of the participants have at least 9 years of experience as software engineer; i.e., 69.6\% of the total sample and 76.0\% of the practitioners that worked with self-adaptation. The distributions for all the participants and those that worked with self-adaptation are similar. The numbers show that most participants of the survey are experienced software engineers. 

\begin{comment}
    

\small
\begin{table}[hbt]
\centering
\caption{Software Engineering experience of participants (Q0.5).}\vspace{-5pt}
\label{tab:q0-4}
\begin{tabular}{llc}
\hline\noalign{\smallskip}
Experience (years) & All (of 184) & With self-adaptation (of 100) \\
\noalign{\smallskip}\hline\noalign{\smallskip}
 1-3 & 20 (10.9\%) & 6 \\
4-8 & 36 (19.6\%) & 18 \\
 9-20 & 83 (45.1\%) & 46 \\
$>$ 20 & 45 (24.5\%) & 30 \\
\noalign{\smallskip}\hline
\end{tabular}
\end{table}
\normalsize

\end{comment}

\begin{figure}[h]
\centering
\includegraphics[width=0.7\columnwidth]{Q0.5-fig.pdf}\vspace{-10pt}
\caption{Software Engineering experience of participants (Q0.5).}
\label{fig:q0-5}
\end{figure}

\subsection{Drivers for Applying Self-Adaptation (RQ1)}
\label{subsec:rq1}
We now analyse the data that we collected for answering RQ1. This research question focuses on the drivers of practitioners for applying self-adaptation and the types of problems they solve using self-adaptation. Note that the data used to answer RQ1 comes from the 100 participants that have experience with concrete self-adaptive systems (i.e., the participants that answered ``Yes'' to Q0.1). 

\vspace{5pt}\subsubsection{For which problems do you apply self-adaptation? (Q1.1)}

Figure~\ref{fig:q1-1} summarises the results. On average, the participants applied self-adaptation for 3.6 types of problems from the predefined list (with seven options). The results show that practitioners apply self-adaptation to deal with a variety of problems. Optimising performance and automating tasks are the main problems addressed by self-adaptation in industry. On the other hand, dealing with changes in business goals is less frequently solved using self-adaptation. `Others' include for example support for testing and evolution. 
%Four participants did not indicate any specific types of problems they tackle using self-adaptation.  

\begin{comment}
    


\small
\begin{table}[hbt]
\centering
\caption{Problems to apply self-adaptation (Q1.1). 
% Percentages indicate the fraction of participants that selected the problems.
}\vspace{-3pt}
\label{tab:q1-1}
\begin{tabular}{lc}
\hline\noalign{\smallskip}
Problem & Quantitative \\
\noalign{\smallskip}\hline\noalign{\smallskip}
To optimise performance & 78 \\%49 (79\%)\\%& 78 (69\%)\\
To automate tasks & 61 \\%41 (66\%)\\%& 64 (57\%)\\
To deal with changes in the environment & 60 \\%35 (56\%)\\%56 (50\%)\\
To configure/reconfigure a system & 51 \\%39 (63\%)\\%56 (50\%)\\
To detect and resolve errors & 46 \\%30 (48\%)\\%& 44 (39\%)\\
To detect and protect a system against threats & 33 \\%25 (40\%)\\%& 38 (34\%)\\
To deal with changes in the business goals & 15 \\%8 (13\%)\\%& 11 (10\%)\\
Others & 12 \\%10 (16\%)\\%7 (6.2\%)\\
%Not specified & 4 (6\%)\\%4 (3.5\%)\\
\noalign{\smallskip}\hline
\end{tabular}
\end{table}\vspace{-10pt}
\normalsize

\end{comment}

\begin{figure}[h]
\centering
\includegraphics[width=0.85\columnwidth]{Q1.1-fig.pdf}\vspace{-10pt}
\caption{Problems to apply self-adaptation (Q1.1).}
\label{fig:q1-1}
\end{figure}

\vspace{5pt}\subsubsection{What are the main business motivations to apply self-adaptation? (Q1.2)}

Figure~\ref{fig:q1-2} summarises the results. On average, the participants provided 2.1 business motivations to apply self-adaptation. Improving user satisfaction, reducing costs, and mitigating risks are the most selected motivations for using self-adaptation. Opening up new application opportunities was selected by a lower number of 21 participants. Examples of `Others' are improving utility and managing complexity. 
%Two participants did not provide any motivation for the use of self-adaptation.  

\begin{comment}
    


\small
\begin{table}[hbt]
\centering
\caption{Main business motivations to apply self-adaptation (Q1.2). 
% Percentages indicate the fraction of all participants \\ that selected the motivations.
}\vspace{-3pt}
\label{tab:q1-2}
\begin{tabular}{lc}
\hline\noalign{\smallskip}
Motivation & Quantitative \\
\noalign{\smallskip}\hline\noalign{\smallskip}
To improve user satisfaction & 67 \\%43 (69\%)\\
To reduce costs & 66 \\%41 (66\%)\\
To mitigate risks & 44 \\%33 (53\%)\\
To open up new application opportunities & 21 \\%14 (23\%)\\
Others & 16 \\%(15\%)\\
%None & 2 (1.8\%)\\
\noalign{\smallskip}\hline
\end{tabular}
\end{table}\vspace{-10pt}
\normalsize

\end{comment}

\begin{figure}[h]
\centering
\includegraphics[width=0.65\columnwidth]{Q1.2-fig.pdf}\vspace{-10pt}
\caption{Main business motivations to apply self-adaptation (Q1.2).}\vspace{-10pt}%%%
\label{fig:q1-2}
\end{figure}

\vspace{5pt}\subsubsection{What could be benefits of applying self-adaptation? (Q1.3)}


\begin{comment}





\begin{table}[hbt]
\centering
\caption{Contingency matrix Q1.1}
\label{tab:contingency_q1-1}
\begin{tabular}{llll}
\hline\noalign{\smallskip}
Role & Quantitative & Qualitative & No preference\\
\noalign{\smallskip}\hline\noalign{\smallskip}
Researcher & 21 & 11 & 40 \\
Practitioner & 5 & 1 & 7 \\
Both & 0 & 4 & 16 \\ 
\noalign{\smallskip}\hline
\end{tabular}
\end{table}
\end{comment}

%\textbf{Overview:} 
Ninety-two participants 
% (94\% of those that worked with self-adaptive systems) 
provided meaningful descriptions of benefits of self-adaptation, an average of 1.8 benefits per participant. 

\textbf{Analysis of comments:} 
We summarise the findings in Table~\ref{tab:codes_q1-3}. For each category (bold font) and code, we include how often it appeared and we provide a few example quotes.\footnote{Categories have codes which express more concrete instances of the categories.}
The dominating benefits of applying self-adaptation are \textit{improved utility} (61 participants), \textit{savings} in costs and resources (38 participants), and \textit{improved human interaction} (37 participants).  

%\end{comment}

\small
\begin{table}[hbt]
\caption{Comments: Reported benefits of self-adaptation (Q1.3).}
\label{tab:codes_q1-3}
\begin{tabular}{p{3cm}lp{9.6cm}}
\hline\noalign{\smallskip}
Categories/codes & \# & Example quotes\\
\noalign{\smallskip}\hline\noalign{\smallskip}
\textbf{Improved utility} & \textbf{61} & \\
Robustness & 21 & ``fault tolerance, one node dies, a new one is spawned without manual intervention''; ``better error handling and prompt disaster recovery'' \\
Performance & 16 & ``Improve performance and quality-of-service''; ``increase in the speed of adaptation''  \\
Availability & 8 & ``The main benefit for us is the 99.9999\% availability, which is crucial for some customers of these cloud-specific solutions''\\
Other & 16 & ``for IoT: optimized operations, improved energy usage''; ``an important part to guarantee the safety [...] of the overall system.''\\\noalign{\smallskip}\hline\noalign{\smallskip}
\textbf{Savings} & \textbf{38} & \\
Costs & 25 & ``The primary benefit is cost reduction''; ``the cheaper bills for running this in an efficient manner in e.g. a cloud service''\\
Resources & 13 & ``scales down resources during hours when traffic is low, and scales up during peak hours, without any manual interference.''\\\noalign{\smallskip}\hline\noalign{\smallskip}
\textbf{Improved human interaction}  & \textbf{37} & \\
User experience & 19 & ``Keep Telco network in optimal condition so that QoS and user experience is maximized, and churn minimized''; ``better user satisfaction because of prompt website responses''\\
Engineers support & 18 & ``removes most of the optimization burden from programmers, so they can be more productive''; ``Reduce workload on human operators; make (the results of) certain actions [...] repeatable and predictable''\\\noalign{\smallskip}\hline\noalign{\smallskip}
\textbf{Handle dynamics} & \textbf{22} & \\
Load dynamics & 12 & ``Change AGV behavior depending of the workload with the goal to save energy (battery life).''\\
Context dynamics & 10 & ``Each machine is unique and its optimal operational parameters change over time due to ware, location, task and seasonal factor.''\\\noalign{\smallskip}\hline\noalign{\smallskip}
\textbf{Other improvements}
& \textbf{16} & \\
Various & 16 & ``In case of spikes in incoming events the system is able to adapt [...] avoiding bottlenecks.''; ``Easier and faster market integration''; ``It's fundamental in huge infrastructure systems otherwise we can’t make it happen.''\\
\noalign{\smallskip}\hline
\end{tabular}
\end{table}
\normalsize

\begin{framed}
\noindent \textbf{Key insight(s) from RQ1:} 
\begin{enumerate}
\item Self-adaptation is widely applied in industry across a wide variety of domains. 
\item Practitioners primarily apply self-adaptation to optimise performance, automate tasks, and deal with changes in the deployment environment. 
\item The dominating business motives to apply self-adaptation in industry are primarily improving user satisfaction and reducing costs, and secondarily 
mitigating risks. 
\item The main benefits of applying self-adaptation are improved utility (in robustness and performance), savings (costs and resources), improved human interaction (user experience and engineers support), and handling dynamics (in the context and system load).  
%\item The problems addressed by industry are in general similar to those studied by academics. Yet, one particular difference is the lack of emphasis of practitioners on the use of self-adaptation to mitigate uncertainties, which has been a key focus in research. 
%\item The four classic management tasks of self-adaptation studied by researchers (self-healing, self-optimising, self-protecting, and self-configuring) are also relevant to practitioners. Yet, different from academics, practitioners also emphasise the importance of improving user satisfaction, reducing costs, and mitigating risks.  
\end{enumerate}
\end{framed}

\subsection{RQ2: Characterisation of Self-adaptation}
\label{subsec:rq2}

\subsubsection{Explain a concrete self-adaptive system you worked with (Q2.1)}

Except for one, all participants with experience in self-adaptation provided a concrete description of a system they worked with. 

\textbf{Analysis of comments:} Tables~\ref{tab:codes_q2-1.I} and~\ref{tab:codes_q2-1.II} summarise the findings. We focused on characteristics of self-adaptive systems and identified three categories: \emph{subject}, \emph{type}, and \emph{trigger} of adaptation. With subject we mean the system or part of it that is adapted; type refers to the kind of adaptation that is applied, and trigger refers to the source that initiates adaptation. 

\small
\begin{table*}[b!]
\caption{Analysis of comments I -- Explain a concrete self-adaptive system you worked with (Q2.1)}
\label{tab:codes_q2-1.I}
\begin{tabular}{p{3cm}lp{9.6cm}}
\hline\noalign{\smallskip}
Categories and codes & \# & Example quotes\\
\noalign{\smallskip}\hline\noalign{\smallskip}
\textbf{Subject of adaptation} & \textbf{99} &  \\

System & 28 & ``Our company develops safety critical systems for railway. Systems architecture is often with redundancy - e.g. 2 out of 3 system, where is automatic reconfiguration implemented. Purpose is high safety and availability''; ``A flexible manufacturing system ... the system and the individual station within the system can "sense" what kind of work piece it has in front of itself and what it or another machine should do with it in the next step.'' 
\\

Module & 22 & ``Environment compensation system for capacitive touch interface. Such system is influenced by envirenmental change (for example temperature)''; ``We manage the memory usage of the process. Once memory usage over a limit (i.e. 90\%), we start throttling the workload.''\\

Platform layer & 13 & ``Monitoring the memory/CPU/disk consumption of our servers and suggesting measures to fix it through human intervention.'' 
\\

Application layer & 11 & ``HotSpot JVM ... reads a program's Java bytecode, and adaptively tunes the performance of the program at runtime, adapting to runtime profiles.''\\

Cluster & 10 & ``Spark executor auto-scaling system. We built this system to automatically add or remove nodes to our Spark cluster when we have a high demand of resources from our Spark jobs.'' \\

Network & 6 & 
``"Our radios apply 'channel assessment' ... that optimizes the radio channels used during BLE communication. Our radios also apply very aggressive power management. peripherals and cores are switched off whenever possible to minimize the system's power usage."'' 
\\

Mixed & 6 & ``Enterprise-cloud environment consisting of dozens of different (micro) services providing functionality to 3rd parties as well as internal employees - data management, authentication and authorization, business process automation, as well as internal development process support (build servers, logging, etc.).''
\\

CI/CD pipeline & 3 & ``Sacling up and down our infrastructure (CI/CD) chain to build and integrate the truck software.'' \\
\noalign{\smallskip}\hline
\end{tabular}
\end{table*}

\begin{table*}[h!]
\caption{Analysis of comments II -- Explain a concrete self-adaptive system you worked with (Q2.1)}
\label{tab:codes_q2-1.II}
\begin{tabular}{p{3cm}lp{9.6cm}}
\hline\noalign{\smallskip}
Categories and codes & \# & Example quotes\\
\noalign{\smallskip}\hline\noalign{\smallskip}
\textbf{Type of adaptation} & \textbf{99} &  \\

Auto-scaling & 33 & ``Automated horizontal scaling of AWS EC2 instances for medical data processing systems''; ``autoscale a cluster based on the resource usage of the nodes of the cluster.''\\
Auto-tuning & 28 & ``A mink feeding robot, that can adjust the food amount according to a set of feeding rules and the food left over from last feeding.'' \\
%The robot can also calibrate the food amount delivered, if the expected amount does not macth the delivered amount.'' \\
Monitor/Analysis & 22 & ``We configured AWS alarms to monitor performance of our systems in case we get more than few number of HTTP 400/500 errors''; ``Monitoring the memory/CPU/disk consumption of our servers and suggesting measures to fix it through human intervention.'' \\
Automated reconfiguration & 11 & ``Continuos integration system - Other \& starts building \& testing a new version as soon as it detects code changes  Build alignment - Creates a new release whenever a subsystem builds successfully.''\\
Other & 5 & ``Our mobile robots scan their environments using laser scanners and other sensors and plan their behavior accordingly.'' ``self healing automotive systems''\\\noalign{\smallskip}\hline\noalign{\smallskip}
\textbf{Trigger for adaptation} & \textbf{99} &  \\
System properties & 27 & ``Auto-scaling functionality of an Azure Service Fabric cluster running a transformation load for processing AGV statistical and playback data.''; ``Realtime focused data streaming protocol ... must take care to avoid exhausting the network resources and thus incurring packet loss and latency spikes, which are very noticeable in games.''\\

Environment properties & 18 & ``An IoT system running in Kubernetes and used to monitor water leaking for household insurance.''; ``A flexible manufacturing system ... can "sense" what kind of work piece it has in front of itself and what it or another machine should do with it in the next step.'' \\

System load & 14 & ``Kubernetes, for handling load intensive periods for scaling up, and self recover from crashes.''; ``Autoscaling of SaaS applications in function of load on AWS and Azure clouds.'' \\

Events & 12 & ``We use kubernetes which provides notification callbacks on any event such as host/pod not available, based on these events we auto mark the node was inactive and do not use those nodes for further write or read operations''; ``Auto Scaling an EMR cluster in AWS based on incoming event data''\\

User actions & 7 & ``[adapt] cache warm up strategy based on user interactions''; ``scammers ... To decide the users that are most likely to be a scammer, the system tracks the past performance of models responsible for flagging potential scammers.''  \\
\noalign{\smallskip}\hline
\end{tabular}
\end{table*}
\normalsize

\begin{comment}


\begin{enumerate}
    \item \textbf{Why include students:} We identified reasons for why students might be used as subjects in empirical software architecture research.

\begin{enumerate}
    \item \underline{Help achieve study goal:} Twenty-nine comments pointed out that students may be suitable to achieve the goal of a study. For example, students may be representative subjects for studies about novice developers. As one respondent wrote, \emph{``They are valid as long as they are considered indicative of outcomes that can be obtained with novice programmers.''}, echoed in another comment: \emph{``They are somehow novice, and this is valuable for the study''}.  Similarly, another one wrote \emph{``Unless the empirical study is about students...''}. One respondent gave a concrete example: \emph{``counter example: teaching techniques for architecture design decision-making.''}. Finally, one respondent wrote that {``Students are developers/architects too. Additionally, if we don't use them, we will never be able to pursue/evaluate certain kinds of research.''} and that \emph{``questions that focus on industry practice can be at least partially answered with students.''}
    
    \item \underline{Represent next generation of practitioners:} Five comments pointed out that students are the next generation of practitioners. As one respondent wrote, \emph{``Especially Master Students are very close to professionals''}. Another one added that \emph{``As they have similar profile and are the future industry workers, results are relevant.''} Another one stated that \emph{``many students are practitioners themselves - in this field we have a tendency to assume age as a predictor of quality and reliability, and this is just plain bullshit.''}. In a similar way, one respondent stated that \emph{``One of the nonsense stereotypes. - Many of our students work in companies on the side. - So, they are semi-professionals. Also, the week after graduation they start to develop, in this week they will not become different developers.''}

\item \underline{Only available subjects:} Four comments highlighted a more pragmatic view and pointed out that students might be the only subjects researchers have access to. As stated by one respondent, \emph{``Studies with students are better than no studies at all, thus provide valuable results and inputs to validate with "harder-to-get" professionals.''}

\item \underline{Less biased:} Three comments emphasized that students, in contrast to practitioners, might be less biased. One respondent wrote that \emph{``I believe there is a lot to be said for the unbiased approach of people that have studied the subject but are not yet clogged down by routine.''} Another one emphasized that \emph{``results of praticioners which refuse to learn and apply new methods (maybe due to their age) can be useless too.''}
    
\end{enumerate}

\item \textbf{Why not include students:} We also identified reasons for why students should not be used in empirical software architecture studies.

\begin{enumerate}
    \item \underline{Lack experience:} Nine comments highlighted that students lack the experience required for software architecture studies to be meaningful. One respondent wrote that \emph{``Might be of limited value in software architecture since software architecture typically requires years of experience. Students often lack experience with software architecture since student projects are usually small.''} Another one stated that \emph{``Students have limited skills and experience, so results on field could be very different.''}. Finally, one respondent wrote that {``The field of architecture is rather abstract and requires experience. It is hard to have students act as architects (or customers of architects).''}
    
    \item \underline{Not representative:} Nine comments highlighted that students are not representative to practitioners, therefore, results are not applicable to software architecting in industry. As one respondent wrote, \emph{``The main concern of using students only in an empirical study is that they may not represent the real practitioners for whom the results may be of interest.''} while another stated that \emph{``This is about architecture, right?  It's even less likely that students are like professionals in architecture knowledge than in coding.''}. Finally, one respondent wrote that \emph{``In the software architecture domain in particular students aren't representative of the target audience.''}
    
    
\end{enumerate}

\end{enumerate}

\end{comment}

Ninety-nine participants provided a description of what is the subject of adaptation in the systems they work with. Top results are \emph{system} that occurred 28 times, followed by \emph{module} with 22 times (i.e., a part of a system). \emph{Platform layer} (infrastructure, execution platform, etc.) was mentioned 11 times and \emph{application layer} 11 times. 

Of the participants that worked with self-adaptation, 86 described in total 101 instances of the types of adaptation they apply (i.e., an average of 1.17). \emph{Auto-scaling} with 33 occurrences and \emph{auto-tuning} with 28 are the most frequent types of adaptations applied by the participants. Twenty-two participants focus on \emph{monitoring and analysis} only (they may use the human in the loop for other adaptation functions). 

Finally, 62 participants explained in total 78 triggers of adaptation in their work (i.e., an average of 1.21 triggers). The main triggers originate from \emph{system properties} with 27 occurrences and \emph{environment properties} with 18 occurrences. Changes in the system load, events,\footnote{An event is an occurrence or action that happens asynchronously at some point in time, such as an alarm, an alert, etc.} and user actions are the other types of triggers for adaptation. 



\begin{comment}

Table~\ref{tab:contingency_q0_1_2_2} shows the mapping of system domains (derived from Q0.2) to the (parts of systems that are) subject of adaptation. The results show that self-adaptation is applied to a variety of domains and at different levels of systems. Table~\ref{tab:contingency_q2_2a} shows the mapping of the subject of adaptation to the type of adaptation. The results show that auto-scaling and automated adaptation are broadly applied. On the other hand, auto-tuning and monitor/analysis are primarily applied for adaptation at system level. Table~\ref{tab:contingency_q2_2b} shows the mapping of the type of adaptation to the trigger of adaptation. Remarkably, changes in the (work) load are only applied for auto-scaling. 

\begin{table*}[hbt]
\centering
\caption{Contingency matrix: system domain (top 7 derived from Q0.1) versus subject of adaptation (Q2.2)}
\label{tab:contingency_q0_1_2_2}
\begin{tabular}{lccccccc}
\hline\noalign{\smallsckip}
Domain & System & Module  & Support system & Cluster  & Cloud & Application & Generic \\
\noalign{\smallskip}\hline\noalign{\smallskip}
Web/mobile & 2 & 5 & 1 & 2 & - & - & 1 \\
Embedded/CPS/IoT & 5 & 1 & 2 & - & - & 1 & - \\
Transportation & 1 & 2 & 1 & - & - & - & - \\
Networks & - & - & 1 & - & 2 & 1 & 1 \\
Business management & - & 1 & 1 & - & 1 & 1 & - \\
Data management & 1 & 1 & 1 & - & 3 & - & - \\
e-commerce & 2 & - & - & 1 & - & - & - \\
General & 5 & 2 & 3 & 2 & 1 & 1 & 1 \\
\noalign{\smallskip}\hline
\end{tabular}
\end{table*}

\begin{table*}[hbt]
\centering
\caption{Contingency matrix: subject of adaptation versus type of adaptation (Q2.2)}
\label{tab:contingency_q2_2a}
\begin{tabular}{lccccccc}
\hline\noalign{\smallskip}
Subject & Auto-scaling & Automated reconfiguration  & Auto-tuning & Monitor/Analysis  & Automated CI & Other \\
\noalign{\smallskip}\hline\noalign{\smallskip}
System & 1 & 1 & 6 & 4 & - & 2 \\
Module & 5 & 5 & 1 & 1 & 1 & - \\
Support system & 3 & 2 & - & - & 2 & - \\ 
Cluster & 7 & 1 & - & - & - & - \\ 
Cloud & 3 & - & - & 1 & - & 1 \\ 
Application & - & - & 1 & - & - & - \\ 
Generic & 1 & 1 & - & - & - & 2 \\ 
\noalign{\smallskip}\hline
\end{tabular}
\end{table*}

\begin{table*}[hbt]
\centering
\caption{Contingency matrix: type of adaptation versus trigger of adaption (Q2.2)}
\label{tab:contingency_q2_2b}
\begin{tabular}{lccccccc}
\hline\noalign{\smallskip}
Type & Load & Events  & Environment properties &  System properties & User actions \\
\noalign{\smallskip}\hline\noalign{\smallskip}
Auto-scaling & 9 & 2 & - & 1 & 1  \\
Automated reconfiguration & - & 2 & - & 2 & 1  \\
Auto-tuning & - & 2 & 4 & 2 & -  \\
Monitor/Analysis & - & 1 & 1 & 1 & -  \\
Automated CI & & - & 2 & - & - & -  \\
Other & - & - & - & - & -  \\
\noalign{\smallskip}\hline
\end{tabular}
\end{table*}

\end{comment}

\begin{framed}
\noindent \textbf{Key insight(s) from RQ2:} 
\begin{enumerate}
\item Self-adaptation is applied at different levels of industrial software-intensive systems: from a complete system to parts of a system and support systems.  
\item The dominating types of adaptations applied in industry are auto-scaling, auto-tuning, and monitoring/analysis. 
\item Adaptions in industrial software-intensive systems are triggered by changes in properties of systems and their environments, dynamics in system load, relevant events, and through user actions.  
\item Technologies such as elastic cloud and  auto-scalers are key enablers for the realisation of self-adaptation in practice.   

\end{enumerate}
\end{framed}

\subsection{RQ3: Application of Self-adaptation}
\label{subsec:rq3}

\subsubsection{What mechanisms or tools does the self-adaptive system you worked with use to monitor a managed system during operation? (Q3.1)}

%\textbf{Overview:} 
% Sixty one of 62  participants (98\%) described a mechanism or tool they have used for monitoring in a concrete self-adaptive system they worked with. 
The participants provided a total of 146 instances of mechanisms or tools they used for monitoring in a self-adaptive system they worked with, i.e., on average, 1.5 mechanisms/tools per participant. 

\textbf{Analysis of comments:} Table~\ref{tab:codes_q3-1} summarises the findings. The participants focused on both ``what'' is being monitored and ``how'' monitoring is done. Based on this we identified three categories: \emph{monitoring metrics}, \emph{monitoring mechanisms}, and \emph{monitoring tools}. Of the 100 answers, we marked 14 as unclear. 

\small
\begin{table*}[t!]
\caption{Analysis of comments -- Mechanisms or tools used to monitor a managed system (Q3.1).}
\label{tab:codes_q3-1}
\begin{tabular}{p{3cm}lp{9.6cm}}
\hline\noalign{\smallskip}
Categories and codes & \# & Example quotes\\
\noalign{\smallskip}\hline\noalign{\smallskip}

\textbf{Monitoring metric} & \textbf{75} &  \\

Resource usage & 23 & ``Active sessions counting, resource utilisation (e.g. RAM) monitoring given by VM''; ``Typically CPU and Memory usage''; ``Helsim: uses CPU counters to measure time or power consumption to process particles''\\

Load & 18 & ``Number of incoming HTTP requests''; ``The system polls the queue of the Spark job scheduler in our cluster every 5 seconds via REST API, using a NiFi flow.''; ``Number of queries''; ``number of requests''\\

Reliability metrics & 13 & ``AWS lambda error metric is monitored to see if the sum of 400/500 errors for every part 5 mins is less than some specified amount.''
%; ``Generic Q: Monitoring is typically done using health endpoints that should preferably not only answer back with a static `Ok', but rather either do some tests or present metrics. It may for example check the rate of failures of incoming requests. Another possibility that is commonly used is to monitor the process state, i.e. to detect dying processes.'' 
\\

Performance metrics & 12 & ``We track the response times for the users' requests.''; ``monitored systems implement specific features to provide data about their performance.'' \\

Application state & 9 & ``Tracking properties are - correct integrity - functionality of memorries (RAM, ROM), correct values and integrity of data among redundant parts.'' 
\\\noalign{\smallskip}\hline\noalign{\smallskip}
\textbf{Monitoring mechanism} & \textbf{20} &  \\
Environment sensors & 9 & ``Based on external information (external sensors like Lidar, Camera, GPS, ...) making sure no accident were to happen''; ``Exteroceptive are aggregated to create a snapshot of the world's state. These are LIDAR and Image sensors. We use Proprioceptive sensors to determine the robot's state. These are encoders only.'' \\

Logging mechanisms & 6 & ``Logging software triggered whenever an incoming request is made''; ``The system logs all interactions, both errors and successful operations.''\\

System sensors & 4 & ``Based on internal information (internal sensors like Wheel speed, steering angle, yaw and roll sensors, ...) optimize the performance to support the driver to drive optimal.''\\

Humans & 1 & ``Human review decisions are used to monitor the precision of models.''
\\\noalign{\smallskip}\hline\noalign{\smallskip}
\textbf{Monitoring tool} & \textbf{34} &  \\

Kubernetes monitoring & 9 & ``Kubernetes clusters are made out of master and worker machine nodes. On the worker nodes runs a process called kubelet that monitors the state of the worker nodes in the Kubernetes cluster''; ``Probes implemented in the application, metrics provided by K8s metrics server (goes down to cgroups via kubelet)'' \\

Prometheus & 9 & ``- every service exposes a defined set of metrics. We collect metrics regarding every layer of the distributed system. We mainly use Prometheus and Splunk to collect these metrics.''; ``Prometheus and grafana for monitoring health of services'' \\

AWS monitoring & 8 & ``We use AWS CloudWatch service to monitor and act on any event with ServerLess AWS lambda functions.''; ``AWS Lambda based monitor which monitor aprox number of message in SQS queue'' \\

Other: Azure monitoring, Datadog, Splunk, cAdvisor, Elasticsearch & 8 & ``Default tooling from Azure / AWS in combination with splunk''; ``We are using Datadog to collect relevant metrics.''; ``AKS monitors the system load and response time to start-up more instances. It also checks for malfunctioning applications and restarts them when stalled, providing high availability.'' \\

\noalign{\smallskip}\hline
\end{tabular}
\end{table*}
\normalsize

The participants mentioned in total 75 metrics they use for monitoring. \emph{Resource usage} with 23 occurrences, \emph{system load} with 18, and \emph{reliability} with 13 are the most frequently mentioned metrics. 

The participants described in total 20 monitoring mechanisms. \emph{Environment sensors} occurred nine times and \emph{system sensors} four times. Six participants described different \emph{logging mechanisms}, and in one system, a \emph{human} is involved in monitoring. 

Finally, the participants provided in total 34 tools they use for monitoring. The most prominent tools are \emph{Kubernetes monitoring} and \emph{Prometheus}, which each occurred nine times, followed by \emph{AWS monitoring} with eight occurrences.\footnote{https://kubernetes.io/ - https://prometheus.io/ - https://aws.amazon.com/}  


\subsubsection{What mechanisms or tools does the self-adaptive system you worked with use analyse conditions of a managed system during operation? (Q3.2)}

The participants provided a total of 115 instances of mechanisms or tools they used for analysing conditions of a self-adaptive system they worked with, i.e., on average 1.5 mechanisms/tools per participant.

\textbf{Analysis of comments:} Table~\ref{tab:codes_q3-2} summarises the findings.
We identified two categories: \emph{analysis mechanisms} and \emph{analysis tools}. Out of the 100 valid answers, 21 were marked as unclear or not applicable (such as ''Fairy simple algoritms'' or ''The tech stack we use is proprietary and the tools we use are built in house''). 
The rest of the participants mentioned in total 73 mechanisms they use for analysis. 
The most frequently mentioned mechanisms are \emph{data analysis methods} (such as interference, statistical data analysis, what-if analysis, and search-based methods) with 18 occurrences, \emph{comparison to threshold} with 16 occurrences, and \emph{metric(s) calculation} and \emph{learning} (mostly machine learning) with 12 occurrences. 
The participants provided in total 23 tools they use for analysis.
\emph{AWS analysis tools} occurred nine times, followed by \emph{Kubernetes stack} with seven, and \emph{Dynatrace} with two occurrences. 
Other tools mentioned by the participants include Splunk, JMX, Jasmina, Azure, Openshift, and Kibana.

\small
\begin{table*}[h!]
\caption{Analysis comments -- Mechanisms or tools used to analyze conditions of a managed system (Q3.2).}
\label{tab:codes_q3-2}
\begin{tabular}{p{3cm}lp{9.6cm}}
\hline\noalign{\smallskip}
Categories and codes & \# & Example quotes\\
\noalign{\smallskip}\hline\noalign{\smallskip}

\textbf{Analysis mechanism}  & \textbf{73} &  \\

Data analysis methods & 18 & ``I think it uses some rolling average or some similar algorithm to estimate whether to scale up or down.''; ``simple statistical inferences based on metrics and simple rules encoded by developers.''; ``statistical analysis of data''\\

Comparison to threshold & 16 & ``Comparing the error rate with constant/dynamic thresholds.''; ``Hard  coded critical boundaries like min max values which lead to switching over to emergency modes [...]''; ``when it falls below Service Level Agreements this indicates a need for auto-scaling'' \\

Metric(s) calculation & 12 & ``Failure rate is used to measure quality of adaptation parameters.''; ``Capturing performance of each node. ''; ``Measurement of traffic load, CPU utilization, and general availability metrics (reachability, status, ...)'' \\

Learning & 12 & ``Each station has a kind of edge computing component that performs some analysis based on machine learning results.''; ``It tracks both the internal working conditions (load) of itself as a serving component, and learns about overall serving conditions.''; ``The system uses biosensory feedback to determine the riders' happiness [...]'' \\

Custom rules & 9 & ``Mostly a simple ruleset gleaned by experimentation and observing how the resulting adaption steps perform at runtime.''; ``we have alertmanager to set up some rules that are known to be issues that have clear solutions'' \\

Autoscaling policy & 5 & ``[...] the response of the scheduler is parsed and the queue length is evaluated. If greater than zero, the flow performs a SCALE UP operation. If equal to zero, the flow performs a SCALE DOWN operation.'' \\

Semantic reasoning & 1 & ``Reasoning on knowledge graphs'' 
\\\noalign{\smallskip}\hline\noalign{\smallskip}
\textbf{Analysis tool} & \textbf{23}
\\
AWS analysis tools & 9 & ``Analytics functions native to the cloud environment the system runs in (AWS).''; ``AWS based auto-scaling conditions as provided in the Cloud formation setup of the cluster'' \\

Kubernetes stack & 7 & ``The master nodes have all sorts of different components such as the kube-scheduler, controllers and state db (etcd), that are managed via the kube-apiserver. ''; ``Built-in Kubernetes/Openshift mechanisms [...]'' \\

Dynatrace & 2 & ``analyze was done by Dynatrace or by Keptn itself by checking against thresholds'' \\

Other & 5 & ``We mainly use rule-based systems like Splunk to automatically analyse production metrics against patterns.''; ``Default tooling from Azure''; ``Kibana'' \\

\noalign{\smallskip}\hline
\end{tabular}
\end{table*}
\normalsize

\subsubsection{What mechanisms or tools does the self-adaptive system you worked with use to change a managed system or parts of it during operation? (Q3.3)}
The participants provided 126 instances of mechanisms or tools they have used for applying changes, i.e., 1.3 mechanism/tool per participant.

\textbf{Analysis of comments:} Table~\ref{tab:codes_q3-3} summarises the findings.
Out of the 100 valid answers, 23 were marked as unclear or not applicable. We identified two categories: \emph{change mechanisms} and \emph{change enacting tools}. In total, 83 instances of mechanisms for change were reported. \emph{Scaling mechanisms} with 36 occurrences and \emph{reconfiguration} (changing the adaptation logic, network reconfiguration, parameter adjusting, load balancing) with 25 occurrences are the most frequently mentioned changing mechanisms. Twelve participants used \emph{non-automated mechanisms} that refer to notifications and change tasks done by humans. 
%(three participants reported that they even take no action). 
The participants mentioned 19 tools they use for enacting change. \emph{Kubernetes} occurred nine times, \emph{AWS} seven times and other tools, including Openshift and Dynatrace, three times. 

\small
\begin{table*}[hbt]
\caption{Analysis of comments -- Mechanisms or tools used to change a managed system or parts of it (Q3.3).}
\label{tab:codes_q3-3}
\begin{tabular}{p{3cm}lp{9.7cm}}
\hline\noalign{\smallskip}
Categories and codes & \# & Example quotes\\
\noalign{\smallskip}\hline\noalign{\smallskip}

\textbf{Change mechanism} & \textbf{83} &  \\

Scaling mechanisms & 36 & ``The server-side system has a load balancer. Hence we increase the number of workers behind the load balancer to decrease the average response time for the users.''; ``It adjusts the number of worker nodes.''; ``Adding a completely similar server / serverless Lambda instance''; \\

% ### autoscaling 
% adding/removing infrastructure resources & 14 & ``Provisioning additional resources (e.g. VMs)''; ``Adding a completely similar server / serverless Lambda instance''; ``Mainly on increased load additional services and resources are added/removed'' \\
%; ``Basically adding computational power through infrastructure as code.'' \\

Reconfiguration & 25 & ``The adaptation directly adjusts the period between the packet send events, as well as the number of packets allowed during each send event. [...]; ``Depending on context, controlled variables are managed through different automation systems.''; ``reconfiguration of the management entity ... to support a larger (or smaller) scale distributed system''; ``load balancer/director that may support controlling the exposure facade towards the system environment. '' \\
%reconfiguration of the management entity (in the cloud) to support a larger (or smaller) scale distributed system'' \\



% ### reconfiguration 
% adjusting parameters of control loop & 7 & ``Optimisation/scenario modelling based on the incrementally trained ML-model leads to new operational parmeters which is fed into the control system.''; ``There are disturbance and controlled variables...and the controlled ones can be changed. Depending on context, controlled variables are managed through different automation systems.'' \\

% ### reconfiguration
% load balancing & 2 & ``It can split its serving units into n pieces so that it can shed load to other instances of itself that have less load.'' \\
%; ``a system may be supported by load balancer/director that may support controlling the exposure facade towards the system environment. '' \\

Non-automated & 12 & ``To effect change on the managed system, the results from the tool need to be approved by an engineer, and are then acted on by the mining and plant teams. These processes are for the most part not automated [...].''; ``Generating alerts and expecting humans to resolve the error manually based on suggestions.''; ``Did not do this [...]. Based on safety protocols this could not be secured'' \\

% ### non-automated
% notification & 4 & ``Generating alerts and expecting humans to resolve the error manually based on suggestions.'' \\

% ### non-automated
% none & 3 & ``Did not do this (if I understand the question right). Based on safety protocols this could not be secured''; ``None'' \\

%Changing application logic & 9 & ``If there are many unsuccessful runs, the mechanism is disabled for a while to prevent wasting CPU time.''; ``Depending on the fault detected, the system can go into fail-safe modes [...]. This is done through re-parametrization at runtime and/or through modification of the lifecycle state of components [...]''; ``It uses a search algorithm to find candidates of schedules, and then evaluate the performance using the cost model.'' \\ 
%; Iteratively, it will find the optimal implementation (schedule) for the operators and then apply it in real applications.'' \\

Restarting/deploying & 7 & ``Mostly just restarting the managed subsystems. In the case of Kubernetes HPA, its the horizontal scaling (up/down) of the Pods''; ``Generally restarts the unhealthy workload, but in the case of autoscaling can also be used to add or remove replicas''; ``... our pipelines use simple bash scripts to deploy previous versions when new versions fail.'' \\

% ### restarting/deploying
% automatic code deployment & 2 & ``... our pipelines use simple bash scripts to deploy previous versions when new versions fail.'' \\
% ; ``[...] automatic deployment to staging environment of feature branches for which a merge request has been created'' \\

Migration & 3 & ``Once the control process informs the control plane, it starts a workflow what we call as instance warming workflow which will dump items that supposed to go to that node from another replica and fills them.''; ``virtual machine (VM) migration or creation.'' 
%
% ### migration
% VM migration & 1 & ``virtual machine (VM) migration or creation.'' \\
% This is because a controller can be considered as a software running on a VM. Whenever the controller location changes, VM migration is required. Similarly, a new VM will be created if a new controller is needed for increasing network workload. '' \\
%
\\\noalign{\smallskip}\hline\noalign{\smallskip}
\textbf{Change enacting tool} & \textbf{19}
\\
Kubernetes & 9 & ``Mostly just restarting the managed subsystems. In the case of Kubernetes HPA, its the horizontal scaling (up/down) of the Pods''; ``... to change topology we simply use K8S api to add/remove worker pods'' \\

AWS & 7 & ``AWS based in-built auto scaling capabilities ''; ``Use the AWS ElasticLoadBalancer and also trigger actions via AWS Lamda functions when required.'' \\

Other & 3 & ``IBM ITM, Log Analyzer, TCAM''; ``UC4 Automation Engine workflows that orchestrate kubernetes clusters''; ``Build-in Openshift mechanisms'' \\

\noalign{\smallskip}\hline
\end{tabular}
\end{table*}
\normalsize

\subsubsection{What is the degree of automation of the majority of the self-adaptive solutions you work with in your organization? (Q3.4)}

All 100 participants provided an answer to this question; Figure~\ref{fig:q3-4} summarises the findings.
Forty-seven participants reported mixed automation in their projects (both semi and fully automated), while 31 indicated semi automation and 19 indicated full automation. Three participants selected other; two of them mentioned that there is no automation, the third stated ``fully-automated till first incident.''

\begin{comment}
    

\small
\begin{table}[hbt]
\centering
\caption{Degree of automation of the self-adaptive solutions the participant has worked with (Q3.4).}
\label{tab:q3-4}
\begin{tabular}{lc}
\hline\noalign{\smallskip}
Automation degree & Quantitative \\
\noalign{\smallskip}\hline\noalign{\smallskip}
Mixed (Both Semi and Fully Automated) & 47 \\
Semi automated & 31 \\
Fully automated & 19 \\
Other & 3 \\
\noalign{\smallskip}\hline
\end{tabular}
\end{table}
\normalsize
\end{comment}

\begin{figure}[h]
\centering
\includegraphics[width=0.6\columnwidth]{Q3.4-fig.pdf}
\vspace{-10pt}%%%
\caption{Degree of automation of the self-adaptive solutions the participant has worked with (Q3.4).}
\vspace{-10pt}%%%
\label{fig:q3-4}
\end{figure}

\subsubsection{Do you reuse solutions to realise self-adaptation in systems you work with? (Q3.5)}
All 100 participants provided answers to this question that are summarised in Figure~\ref{fig:q3-5}.
A majority of 71 participants reuse at least sometimes solutions in self-adaptive systems (44 of them reuse solutions frequently to always). The other 29 participants rarely, very rarely or never reuse solutions. 

\begin{comment}
    

\small
\begin{table}[hbt]
\centering
\caption{Do you reuse solutions to realise self-adaptation? (Q3.5)}
\label{tab:q3-5}
\begin{tabular}{lc}
\hline\noalign{\smallskip}
Answer & Quantitative \\
\noalign{\smallskip}\hline\noalign{\smallskip}
Never & 13 \\
Very Rarely & 10 \\
Rarely & 6 \\
Sometimes & 27 \\
Frequently & 26 \\
Very Frequently & 14 \\
Always & 4 \\
\noalign{\smallskip}\hline
\end{tabular}
\end{table}
\normalsize
\end{comment}

\begin{figure}[h]
\centering
\includegraphics[width=0.7\columnwidth]{Q3.5-fig.pdf}
\vspace{-10pt}%%%
\caption{Do you reuse solutions to realise self-adaptation? (Q3.5)}
\vspace{-10pt}%%%
\label{fig:q3-5}
\end{figure}

\subsubsection{Please provide a concrete example of reuse you used to realise self-adaptation? (Q3.6)}

Sixty-seven participants provided examples of reuse in the realisation of the self-adaptive systems. 

\textbf{Analysis of comments:} 
Table~\ref{tab:codes_q3-6} summarises the findings. We focused on the subjects of reuse and identified five categories: \emph{code}, \emph{design artifacts}, \emph{specifications}, \emph{IT infrastructure}, and \emph{procedures}. The 67 participants provided in total 91 objects of reuse in adaptation, i.e., an average of 1.4. Code occurred 33 times, with \emph{modules} as the top subject of reuse (18 instances). Design artifacts was mentioned 22 times with \emph{patterns} and \emph{architecture} as main subjects of reuse (each with seven instances). Specification was mentioned 18 times as objects of reuse, IT infrastructure 11 times, and procedures seven times. The results demonstrate that reuse in self-adaptation is common practice, although the use of patterns (a topic that gets increasing attention in research) is limited.  

\small
\begin{table}[hbt]
\caption{Comments: Examples of reuse in self-adaptive systems (Q3.6).}
\label{tab:codes_q3-6}
\begin{tabular}{p{3cm}lp{9.7cm}}
\hline\noalign{\smallskip}
Categories/codes & \# & Example quotes\\
\noalign{\smallskip}\hline\noalign{\smallskip}
\textbf{Code} & \textbf{33} &  \\
Modules & 18 & ``Self adaptation mechanisms used for speech recognition ... are also used for computer assisted coding solutions. ''; ``Different parts of the Behavior tree can be reused in different robots.''\\
Scripts and algorithms & 8 & ``The same scripts and solutions are constantly reused - because it’s the easiest way to create new with a constant lack of time.''; ``Threshold algorithms are reused frequently, with the threshold value adapted for the specific use case.''  \\
Libraries & 7 & ``internal libraries that simplify monitoring, interaction with external tools, etc''
\\\noalign{\smallskip}\hline\noalign{\smallskip}
\textbf{Design artifacts} & \textbf{22}
\\
Patterns & 7 & ``We try to reuse design patterns (e.g. autoscaling) for all cloud native applications we build.''; ``Re-use of design patterns like MAPE-K. ''\\
Architecture & 7 & ``AWS stack\,...\,can be used as a generic template cross different applications which are based on a job processing ''\\
Know-how & 5 & ``We use similar principles in different product.''; ``We reused knowledge of driver parameter adaptation from FDM (3 axis) printer while designing a SLA (single axis) printer.''\\
Models & 3 & ``machine learning cost models can be reused by different systems''
\\\noalign{\smallskip}\hline\noalign{\smallskip}
\textbf{Specifications} & \textbf{18}
\\
Policies \& rules & 5 & ``auto-scaling policies ... have a standard definition which can be reused in different systems or use-cases.''\\
Configuration files & 5 & ``K8s config files for different cloud native application can be similar''\\
Templates & 4 & ``We reuse very similar set of configuration templates of container deployment''\\
Metrics & 4 & ``Kibana alerts''
\\\noalign{\smallskip}\hline\noalign{\smallskip}
\textbf{IT infrastructure} & \textbf{11}
\\
Frameworks\,\&\,platforms & 7 & ``a framework for monitoring metrics that allows labels to be given to properties, the time-series data to be tracked in a database, and then hooks to visualization database and alert systems.''\\
Tools & 4 & ``Use the same tools AWS provides for all our different product deployments.''
\\\noalign{\smallskip}\hline\noalign{\smallskip}
\textbf{Procedures} & \textbf{7}
\\
Processes & 3 & ``Writing "watchdog" processes for systems that aren't deployed to kubernetes''\\
Pipelines & 2 & ``pipeline (Application\,-\,Datadog\,-\,custom logic\,-\,AWS API) is replicated with different settings for different use-cases.''\\
Schedules & 2 & ``Most of the approaches we use for digital twins share some history ... An example of that is in the scheduling space, where schedules need to adapt to changes in resources or the inclusion and removal of tasks.''\\
\noalign{\smallskip}\hline
\end{tabular}
\end{table}
\normalsize

\subsubsection{Why do you not often reuse solutions
when realising self-adaptive systems?
What hinders their reuse, please provide a
short answer? (Q3.7)}

This was a conditional question that was only asked to the participants that answered never or very rarely to Q3.5 (that asked whether participants reuse solutions to realise self-adaptation). Twenty-three participants provided such an answer to Q3.5.  

\textbf{Analysis of comments:} 
Table~\ref{tab:codes_q3-7} summarises the findings. From 18 participants that provided valid answers, we identified 19 \emph{reuse hurdles}, i.e., an average of 1.1. The main hurdle reported by 11 participants is \emph{difference in problems}, hampering easy reuse. Other hurdles are \emph{lack of experience or maturity} in applying self-adaptation within the company (4 occurrences), and the \emph{complexity of the system} and \emph{organisational concerns} (each with 2 occurrences). 

\small
\begin{table}[hbt]
\caption{Comments: Why not often reusing solutions
when realising self-adaptive systems (Q3.7).}
\label{tab:codes_q3-7}
\begin{tabular}{p{3.6cm}lp{8.8cm}}
\hline\noalign{\smallskip}
Categories/codes & \# & Example quotes\\
\noalign{\smallskip}\hline\noalign{\smallskip}
\textbf{Reuse hurdles} & \textbf{19} &  \\
Different problems & 11 & ``In my case every self-tuning problem is different and prevents easy reuse.''; ``Our applications and application domains are very different and since we do research we actively look for new and different challenges.''\\
Lack of experience/maturity & 4 & ``I think lack of competence is a huge thing to overcome, though most of the organisations around us try to catch up '' \\
System structure & 2 & ``The solutions were too coupled, too integrated and not enough modularized.''\\
Organisational concerns & 2 & ``We have to go through a legal department in order to reuse code from outside ... That poses a large problem. ''\\
\noalign{\smallskip}\hline
\end{tabular}
\end{table}
\normalsize

\subsubsection{How do you ensure that you can trust the self-adaptive solutions you build? (Q3.8)}

Ninety-one of the 100 participants that worked with self-adaptation provided valid answers. 

\textbf{Analysis of comments:} 
Table~\ref{tab:codes_q3-8} summarises the findings. The participants provided in total 152 instances of techniques for ensuring trust in the self-adaptive systems they build, i.e., on average 1.7 techniques per participant. We grouped the techniques in three categories: \emph{testing and verification}, \emph{stakeholder-centred techniques}, \emph{online techniques}. 
Testing and verification was mentioned 71 times with \emph{extensive testing} being the main technique occurring 58 times, followed by \emph{benchmarking} occurring 10 times and verification (three times). Stakeholder-centred techniques were mentioned 45 times. In this category, \emph{human supervision} (22 occurrences) and \emph{rigorous design and development} (10 occurrences) were the main reported techniques. Finally, online techniques were mentioned 36 times with \emph{runtime monitoring and alerting} as main reported technique (27 occurrences). In contrast to an important focus of research in self-adaptation, (formal) \emph{verification} as a technique to ensure trust was only mentioned three times.   

\small
\begin{table*}[hbt]
\caption{Analysis of comments - Techniques for ensuring trust in self-adaptive solutions (Q3.8).}
\label{tab:codes_q3-8}
\begin{tabular}{p{3.6cm}lp{8.8cm}}
\hline\noalign{\smallskip}
Categories and codes & \# & Example quotes\\
\noalign{\smallskip}\hline\noalign{\smallskip}

\textbf{Testing and verification}  & \textbf{71} &  \\

Extensive testing & 58 & ``We use extensive testing (unit, module, system)''; ``We have extensive testing on test k8s clusters, provisioned for these purposes. ''; ``We have countless amount of testing and verification code built as part of the OpenJDK to ensure the quality of the product is appropriate. ''\\

Benchmarking & 10 & `As a lot of the self adaptation logic involves optimization opportunities, we also regularly run many benchmarks and immediately report regressions''; ``We do testing of the machine learing models, but we also have pilot factories where we test our methods and design to see if all station perform as itended.'' \\

Verification & 3 & ``expert testing, supervision, verification when applicable''; ``Testing, but also some human verification as part of the Cloud Operations team.''
\\
\noalign{\smallskip}\hline\noalign{\smallskip}
\textbf{Stakeholder-centred techniques} & \textbf{45} &  \\

Human supervision & 22 & ``Human supervision until confident.''; ``Extensive system testing and gradual release of human supervision levels upon system going live.'' \\

Rigorous design and development & 10 & ``''; ``virtual training to ensure operators understand and are comfortable with the conditions in which the safety system will engage.'' \\

Trust in third-party software & 8 & ``for features like auto-scaling compute ... we use trusted vendors and deploy these features mainly for analytics use cases which are not business-critical.'' \\

Operational constraints & 5 & ``the concrete actions that are taken by the system are defined by the user. so there is never a surprise. the system only decides if and when to apply these actions.''; ``Our autotuning algorithms never fail for particular (exactly specified) set of systems. If the system fulfils these assumptions, it works always.'' 
\\
\noalign{\smallskip}\hline\noalign{\smallskip}
\textbf{Online techniques} & \textbf{36} &  \\

Runtime monitoring and alerting & 27 & ``In cases where an existing system is not being replaced but rather new capability is being added, results will be tracked over time to ensure accuracy.''; ``we have deployed some alert to track the high-level properties of the system.'' \\

Continuous testing during operation & 6 & ``there is gradual canary testing in the real production system. ''; ``Automated test scripts, automated "synthetic transactions" in production, model performance validation'' \\

Mitigation strategies & 3 & ``This automation can provide alter with all the steps and rollback automatically if there is any issue. '' \\

\noalign{\smallskip}\hline
\end{tabular}
\end{table*}
\normalsize

\begin{framed}
\noindent \textbf{Key insight(s) from RQ3:} 
\begin{enumerate}
\item Resource usage and system load are the main types of monitoring metrics used in practice. These metrics are primarily tracked by sensors in the environment and the system.  
\item Practitioners use various mechanisms for analysis in realising self-adaptation, with data analysis methods and comparison to thresholds as main mechanisms.   
\item A wide range of mechanisms are used to enact self-adaptation in industrial systems with auto-scaling and reconfiguration as top mechanisms. 
\item Practitioners extensively rely on tools such as Kubernetes and AWS to support the realisation of different functions of self-adaptation. 
\item Industrial systems apply a mix of semi and fully automated adaptation. 
\item A majority of practitioners reuse solutions when applying self-adaptation, mainly in the form of code, design artifacts, and specifications. 
\item Ensuring trust in industrial self-adaptive systems is mainly achieved through extensive testing, runtime monitoring and alerting, and human supervision. 
\end{enumerate}
\end{framed}

\subsection{RQ4: Difficulties, Problem Support, and Opportunities}
\label{subsec:rq4}

\subsubsection{Did you encounter particular difficulties when engineering or maintaining self-adaptive systems you worked with? (Q4.1)} Figure~\ref{fig:q4-1} summarises the findings. Forty-one of 100 participants report that they sometimes face difficulties with applying self-adaptation. Thirty encounter difficulties frequently or very frequently, while 17 rarely or very rarely have difficulties. Four participants reported to have always problems, while eight reported that they never face difficulties. 

\begin{comment}
    


\small
\begin{table}[hbt]
\centering
\caption{Did you encounter difficulties when engineering or maintaining self-adaptive systems? (Q4.1)}
\label{tab:q4-1}
\begin{tabular}{lc}
\hline\noalign{\smallskip}
Answer & Quantitative \\
\noalign{\smallskip}\hline\noalign{\smallskip}
Never & 8 \\
Very Rarely & 12 \\
Rarely & 5 \\
Sometimes & 41 \\
Frequently & 24 \\
Very Frequently & 6 \\
Always & 4 \\
\noalign{\smallskip}\hline
\end{tabular}
\end{table}
\normalsize

\end{comment}

\begin{figure}[h]
\centering
\includegraphics[width=0.7\columnwidth]{Q4.1-fig.pdf}
\vspace{-10pt}%%%
\caption{Did you encounter difficulties when engineering or maintaining self-adaptive systems? (Q4.1)}
\vspace{-10pt}%%%
\label{fig:q4-1}
\end{figure}

%\textbf{Analysis of comments:} 


\subsubsection{Please give one or two examples of the difficulties that you encountered when engineering or maintaining self-adaptive systems. (Q4.2)}

Seventy-four participants reported in total 140  difficulties, i.e., on average 1.9 difficulties per participant. Table~\ref{tab:codes_q4-2} summarises the findings.

\small
\small
\begin{table*}[hbt]
\caption{Analysis of comments -- Difficulties with engineering or maintaining self-adaptive systems (Q4.2)}
\label{tab:codes_q4-2}
\begin{tabular}{p{3.6cm}lp{8.8cm}}
\hline\noalign{\smallskip}
Categories and codes & \# & Example quotes\\
\noalign{\smallskip}\hline\noalign{\smallskip}

\textbf{Design issues} &  \textbf{43} &  \\

Reliable/optimal design & 26 & ``With high availability requiremets, the chance something fails somewhere sometime is close to a 100\%. The systems needs to be designed to still provide service despite erroneouse behavior or failing parts in the system.''; ``the main challenge is to design adaptation function with respect to computation context'' 
%- e.g., selecting right trade-off between shutting down instance or keeping it running longer time since boot of instance can be time-consuming.'' 
\\

Design complexity & 17 & ``Complexity in defining the adaptation rules. Conditions are not always obvious.''; ``Self-adaptiveness or resilience have to be taken into consideration at each stage of the 
%software production and operation 
... workflow. This is really a challenge as more often than not these are concepts that are completely obscure to the average programmer/devop mind.'' 
\\
\noalign{\smallskip}\hline\noalign{\smallskip}
\textbf{Lifecycle issues} & \textbf{42} &  \\

Tuning/debugging & 19 & ``Debugging the root cause of a scaling failure might be time-consuming: also, in some cases the problem might be outside of your control (e.g. temporary lack of EC2 Spot capacity in AWS)'' \\

Limitations tools/methods & 13 & ``The metrics available are not always fully transparent and built with auto-scaling in mind''; ``IAM permissions are hard to deal with when configuring these self-adaptive systems. Usually, the permission to scale or to notify is not properly configured.''\\

System/environment evolution & 10 & ``If the functionality is not designed in from the beginning then it is a huge amount of work to implement later.''; ``System architecture over lifetime (nee features to be added...)''
\\
\noalign{\smallskip}\hline\noalign{\smallskip}
\textbf{Runtime issues} & \textbf{30} &  \\

Runtime uncertainty & 17 & ``Many self-adaptive systems are based on unproven heuristics. Therefore, they usually do not work in many cases.''; ``It is hard to guess how much can the environment affect the system. ... 
%Usually, we develop the system on one or a limited set of units. 
It is hard to extend the parameters to cover whole production.'' \\

Data collection/evaluation & 7 & ``Gathering quantitative data samples to evaluate the performance is very complicated.''; ``sensors gives wrong reading values'' \\

Resources required & 3 & ``Sometimes it doesn't react fast enough. It also takes computation resources for this self-adaptive software, and the compute resources use increases with the number of incoming requests.'' \\

Delayed/missing runtime changes & 3 & ``Autoscaling is often too slow or triggered too late.''; ``Notifications are delayed or missed'' 
\\
\noalign{\smallskip}\hline\noalign{\smallskip}
\textbf{People and process issues} & \textbf{19} &  \\

Skills/experience & 14 & ``Every self-adapt system must be tuned up which is sometimes tricky and needs high skilled engineers.''; ``The Kubernetes/Openshift cloud and centralized log storage ... require experienced administration staff and vast knowledge of many networking concepts (... DNS, NAT).'' \\

Process and management & 9 & ``We are not yet very experienced ... the main challenges were to convince the central IT department this was the way to go, then to design the system, and obviously to master the technology itself.'' \\

Automation & 1 & ``often automation is not trusted enough by humans. humans want to stay in the loop.'' \\




\noalign{\smallskip}\hline
\end{tabular}
\end{table*}
\normalsize


\textbf{Analysis of comments:} We identified four categories of difficulties: \emph{design issues}, \emph{lifecycle issues}, \emph{runtime issues}, and \emph{people and process issues}. Most frequently reported difficulties, 43 in total, relate to the design of self-adaptation, in particular \emph{reliable/optimal design} (26 occurrences) and \emph{design complexity} (17 occurrences). Life cycle issues were reported 42 times, in particular \emph{tuning/debugging} (19 occurrences) and \emph{limitations of tools and methods} (13 occurrences). Difficulties with runtime aspects of self-adaptive systems was reported 30 times with \emph{runtime uncertainty} mentioned 17 times, and difficulties related to people and process occurred 25 times with \emph{skills and experience} occurring 14 times.  

\subsubsection{Did you face any risks when engineering self-adaptive systems? (Q4.3)}
%
Figure~\ref{fig:q4-3} summarises the findings. 
Thirty-four of 100 participants report that they sometimes face risks when engineering self-adaptive systems. Eighteen report that they frequently to always encounter risks, while 48 rarely to never face risks. 
%It is remarkable that more than 50\% of the participants report that they face at least sometimes risks with applying self-adaptation. 

\begin{comment}
    


\small
\begin{table}[hbt]
\centering
\caption{Did you face any risks when engineering self-adaptive systems? (Q4.3) -- 100 answers}
\label{tab:q4-3}
\begin{tabular}{lc}
\hline\noalign{\smallskip}
Answer & Quantitative \\
\noalign{\smallskip}\hline\noalign{\smallskip}
Never & 18 \\
Very Rarely & 20 \\
Rarely & 10 \\
Sometimes & 34 \\
Frequently & 11 \\
Very Frequently & 2 \\
Always & 5 \\
\noalign{\smallskip}\hline
\end{tabular}
\end{table}
\normalsize

\end{comment}


\begin{figure}[h]
\centering
\includegraphics[width=0.7\columnwidth]{Q4.3-fig.pdf}
\caption{Did you face any risks when engineering self-adaptive systems? (Q4.3) -- 100 answers}
\label{fig:q4-3}
\end{figure}

\subsubsection{Briefly describe one or two risks that you faced when engineering self-adaptive systems. (Q4.4)}

The participants provided a total of 60 responses containing 66 instances of risks faced when engineering self-adaptive systems.
On average, the participants reported 1.3 risks.

\textbf{Analysis of comments:} 
%
Tables~\ref{tab:codes_q4-4.I} and~\ref{tab:codes_q4-4.II} summarise the findings.
Out of the 60 valid answers, 11 were marked as unclear or not applicable.
We identified four categories of risks: \emph{faults}, \emph{difficulties with development/operation}, \emph{impact on qualities}, and \emph{impact on business}. 
Most frequently mentioned risks, 20 in total, relate to faults, in particular \emph{incorrect functionality} (7 occurrences), \emph{wrong results} and \emph{misconfiguration} (4 occurrences each), and \emph{network failure} (2 occurrences).
Difficulties with development/operation relate to \emph{difficulties to manage environment uncertainty} (6 occurrences), and \emph{difficulties to test} and \emph{build systems} (4 occurrences each). 
Participants mentioned also the risk of having several qualities impacted; \emph{performance degradation} with 5 occurrences the most frequent, followed by \emph{reduced availability} and \emph{safety and security threats} with 4 occurrences each.
Finally, negative impact on the business in terms of \emph{increased cost} (5 occurrences) and \emph{losing control and trust} (4 occurrences) are also reported as important risks when applying self-adaptation. 

\small
\begin{table*}[hbt]
\caption{Analysis of comments I -- Risks faced when engineering self-adaptive systems (Q4.4).}
\label{tab:codes_q4-4.I}
\begin{tabular}{p{3.6cm}lp{8.8cm}}
\hline\noalign{\smallskip}
Categories and codes & \# & Example quotes\\
\noalign{\smallskip}\hline\noalign{\smallskip}

\textbf{Faults} & \textbf{20} &  \\

Incorrect functionality & 7 & ``Automation can lead to unexpected values''; ``The process might be OOM killed if the self-adaptive system doesn't function correctly (i.e. bugs).'' \\
Wrong results & 4 & ``incorrect results''; ``Wrong decisions based on faulty models'' \\
Misconfiguration & 4 & ``Tuning autoscaling settings can be problematic resulting in unexpected results.''; ``Wrong threshold levels may lead to unwanted responses. '' \\
Network failure & 2 & ``Giving control to software that can change production environments can cause network failure.'' \\
Other & 3 & ``data loss''; ``[...] heuristics that work well on some applications, do not always perform the best for all applications.'' \\
\noalign{\smallskip}\hline\noalign{\smallskip}
\textbf{Difficulties with development/operation} & \textbf{16} &  \\

Difficult to manage environment uncertainty & 6 & ``We face a risk of underestimating environment variability.''; ``Legacy monitoring solutions don't cope well with environments that scale back.''; ``Risk may be encountered if the incoming event stream is completely unpredictable and have huge spike differences in data for a considerable period of timr'' \\
Difficult to test & 4 & ``if the executed actions that will be done by the self-adopting system are not tested before, it might introduce some risks''; ``It is also difficult to do reliable performance testing in non-production environments.'' \\
Difficult to build & 4 & ``implementing and designing self-adaptive systems may initially seem to take longer time – hence the risk of not being allowed to implement it as good as it can be done''; ``Costs of building own (self-hosted) environment [...]'' \\
Other & 2 & ``life updates (no downtime)''; ``There is always a lingering concern of quis custodiet ipsos custodes - or 'who watches the watchmen'.'' \\
 
\noalign{\smallskip}\hline
\end{tabular}
\end{table*}

\begin{table*}[hbt]
\caption{Analysis of comments II -- Risks faced when engineering self-adaptive systems (Q4.4).}
\label{tab:codes_q4-4.II}
\begin{tabular}{p{3.6cm}lp{8.8cm}}
\hline\noalign{\smallskip}
Categories and codes & \# & Example quotes\\
\noalign{\smallskip}\hline\noalign{\smallskip}

\textbf{Impact on qualities} & \textbf{16} &  \\

Performance degradation & 5 & ``[...] risk of degrading the performance instead of improving it, and degrading the user experience as a result.''; ``Performance impact on the running system when applying auto-scaling (e.g. scaling down)''; ``sometimes a sequence of perfectly acceptable self-adaptive automatic actions can lead to outages worse than the root cause'' \\
Reduced availability & 4 & ``If the system did not behave properly this could result in an outage [...]''; ``Availability of the system during the auto-scaling rules being applied''\\
Safety and security threats & 4 & ``If a system is self-adaptive, how can we secure that it is safe during production (some parts can be powered for self test during assembly and we need to know it is safe)? If we use machine learning on a self-adaptive system, how do we secure safety? 
''; ``There is a risk of misconfiguration that can lead to lost nodes and applications, security exposures etc. There are also security risks involved with the base building components, such as docker images from untrusted sources [...]'' \\
Extra resource consumption & 2 & ``Risk of all resources being eaten up by a self-adaptive process.''; ``[...] it may use up too many unnecessary hardware and software resources'' \\
Reliability issues & 1 & ``Reliability issues in case of non-converging oscillations or plain wrong output due to prolonged failures in the metrics collection pipelines or simply wrong algorithms'' \\
\noalign{\smallskip}\hline\noalign{\smallskip}
\textbf{Impact on business} & \textbf{14} &  \\

Increased cost & 5 & ``Regarding autoscaling, the main issue was to fail and so increasing the infra cost of the users due to bugs in the system.''; ``Lost control over system size. This also impacted the approx. total cost agreed with the customer.'' \\
Losing trust and control & 4 & ``Trust. Because flexible manufacturing systems have some kind of autonomous behavior with tasks that have been done manually, our clients are initially very sceptial and to not trust the systems initally''; ``risk of losing (manual) control of the system for the sake of automation'' \\
Harder to understand/fix & 3 & ``the whole system becomes more complex, hence fewer people understand all details of its behaviour.''; ``More difficult troubleshooting for a self-adapting, distributed system.'' \\
Not useful & 2 & ``The self-adaptive system might not perform better than the baseline when dealing with dynamic shapes, as the cost model might not be generic enough to predict the performance.'' \\
 
\noalign{\smallskip}\hline
\end{tabular}
\end{table*}
\normalsize

\subsubsection{How did you mitigate the risks that you faced? (Q4.5)}

The participants provided 51 responses containing 66 instances of risk mitigating techniques when engineering self-adaptive systems, i.e., 
on average 1.3 techniques per participant.

\textbf{Analysis of comments:} 
Table~\ref{tab:codes_q4-5} summarises the findings.
Out of the 100 valid answers, 13 were marked as unclear or not applicable.
The other participants mentioned a variety of risk mitigation mechanisms, which we grouped into three categories. 
\emph{Stakeholder-centred techniques} are the largest category with 25 occurrences, followed by \emph{offline techniques} and \emph{online techniques} with 18 and 9 occurrences each. 
Within stakeholder-centred techniques, \emph{rigorous design and development} (8 occurrences), \emph{code review} (4 occurrences), and \emph{human supervision} (4 occurrences) are the most popular risk mitigation techniques. 
\emph{Extensive testing} with 15 occurrences is the mostly mentioned offline technique, while \emph{runtime monitoring and analysis} with 6 occurrences is the mostly mentioned online technique to mitigate risks. 

\small
\begin{table*}[hbt]
\caption{Analysis of comments -- Techniques to mitigate risks when engineering self-adaptive systems (Q4.5).}
\label{tab:codes_q4-5}
\begin{tabular}{p{3.6cm}lp{8.8cm}}
\hline\noalign{\smallskip}
Categories and codes & \# & Example quotes\\
\noalign{\smallskip}\hline\noalign{\smallskip}

\textbf{Stakeholder-centered techniques} & \textbf{25} &  \\

Rigorous design and development & 8 & ``careful engineering so that there are open doors for manual intervention, when necessary, without lost of system availability nor hindering the automation mechanisms''; ``We try to have design sessions [...] and possibly enhance the design in the early phases of development''; ``Engineering analysis, testing, controlled deployment, ...'' \\
 
Code review & 4 & ``As always, planning, design reviews, code reviews, testing on several levels, monitoring the production.''; ``Each incident is taken into consideration and rules are always reviewed. '' \\
 
Human supervision & 4 & ``The responsibility was left to a human operator.''; ``Mainly by performing tests and human supervision (monitoring resource utilization)'' \\
  
Outsource & 3 & ``Outsource the cloud operation to a specialized provider (RedHat, AWS) where possible. In other cases, customers had to hire experienced administrators/go through extensive period of testing to gain the necessary experience.'' \\

Other (post mortem analysis, hiring experts, work in pairs, documentation) & 6 & ``When we hit a problem years after the fact, we perform a detailed post-mortem and try to think about other possible failures we may have missed.''; ``We hired (multiple) external consultancy firms to tap into their experience in deploying such a system.''; ``Work in pairs,
Document architectural decisions'' \\
\noalign{\smallskip}\hline\noalign{\smallskip}
\textbf{Offline techniques} & \textbf{18} &  \\

Extensive testing & 15 & ``test each action in isolation before it is provided to the system''; ``Automated and human testing. In addition for complex algorithms, we run parallel, correlated analysis.''; ``With automated and manual testing while injecting non-determinism to the test suite''; ``Extensive testing at the customers factory and fine tuning of the models.'' \\
 
Set operational boundaries & 2 & ``Defined max-amount of resources a system functionality/component is allowed to consume.''; ``Thresholds and some manual monitoring'' \\
 
Encryption  & 1 & ``State of the art encryption, encryption, and encryption.'' \\
\noalign{\smallskip}\hline\noalign{\smallskip}
\textbf{Online techniques} & \textbf{9} &  \\

Runtime monitoring and analysis & 6 & ``Alerts tracking high-level properties that can give us some assurance that the system is working fine.''; ``Monitor / review the automated actions.'' \\
 
Roll-out/roll-back strategies & 2 & ``Slow roll - only send the new system traffic in small increments (10\%, 20\%, ...) until production baselines are established for load, actual latency, etc.  This helped us determine what the MIN and MAX pod settings should be as well as VM heap sizes.''; ``Manual roll back to previous stable state of user profiles.'' \\
 
Run in non-business critical hours & 1 & ``We run our processes during the night, when there is less chance of interference with business critical (customer facing) systems.'' \\

\noalign{\smallskip}\hline
\end{tabular}
\end{table*}
\normalsize

\subsubsection{Have you faced or seen any problems
of self-adaptation for which you would
appreciate support from researchers (Q4.6)}
%
Figure~\ref{fig:q4-6} summarises the findings for Q4.6. Thirty-three of 166 participants (17.9\%) frequently to always experience problems with self-adaptation for which they would appreciate support from researchers, while 43 participants (23.4\%) sometimes face such problems. On the other hand, 108 of the participants (58.7\%) never to rarely experience problems for which they would appreciate support from researchers. In summary, almost half of the participants believe that they would benefit from support of researchers to address some of the problems they face with engineering self-adaptive systems. 

\begin{comment}
    


\small
\begin{table}[hbt]
\centering
\caption{Have you faced or seen any problems of self-adaptation for which you would appreciate support from researchers? (Q4.6)}\vspace{-5pt}
\label{tab:q4-6}
\begin{tabular}{ll}
\hline\noalign{\smallskip}
Answer & Quantitative \\
\noalign{\smallskip}\hline\noalign{\smallskip}
Never & 51 (27\%)\\
Very Rarely & 43 (26\%)\\
Rarely & 8 (5\%)\\
Sometimes & 38 (23\%)\\
Frequently & 21 (13\%)\\
Very Frequently & 6 (4\%)\\
Always & 5 (3\%)\\
\noalign{\smallskip}\hline
\end{tabular}\vspace{-5pt}

\end{table}
\normalsize

\end{comment}

\vspace{-10pt}%%%
\begin{figure}[h]
\centering
\includegraphics[width=0.7\columnwidth]{Q4.6-fig.pdf}
\vspace{-10pt}%%%
\caption{Have you faced or seen any problems of self-adaptation for which you would appreciate support from researchers? (Q4.6)}
\vspace{-10pt}%%%
\label{fig:q4-6}
\end{figure}

%Thirty two of 166 participants (42.2\%) sometimes to always experience problems with self-adaptation for which they would appreciate support from researchers. On the other hand, 96 participants (57.8\%) rarely or never experience such problems. 

%Sixty nine of 166 participants (57.8\%) rarely to never experience problems with self-adaptation for which they would appreciate support from researchers. On the other hand, 32 participants (42.2\%) sometimes to always experience problems with self-adaptation for which they would appreciate support from researchers.

%Forty five of 166 participants (27\%) never experience problems with self-adaptation for which they would appreciate support from researchers. The remaining 121 participants (73\%) experience problems (from very rarely to always) with applying self-adaptation for which they would appreciate support from researchers.


\subsubsection{For which problems of self-adaptation
would you appreciate support from researchers? Please briefly explain one or two such problems (Q4.7)}

Sixty-five participants described in total 113 problems for which they would appreciate support from researchers. Tables~\ref{tab:codes_q4-7.I} and~\ref{tab:codes_q4-7.II} summarise the findings.

\textbf{Analysis of comments:} 
We grouped the problems in four categories: \emph{engineering}, \emph{guarantees}, \emph{data}, and \emph{user interaction}. Forty-eight of the reported problems (42.5\% of the reported problems for which practitioners would appreciate support from researchers) relate to the engineering of self-adaptive systems. The main problems in this category relate to \emph{architecture and reuse} (16 occurrences) and the \emph{adoption} of self-adaptation (10 occurrences). Adoption refers to problems within a company with introducing self-adaptation, which can be related to technical aspects, expertise, or organisational aspect. Twenty-five of the reported problems (22.1\% of all reported problems) relate to guarantees, in particular providing \emph{trustworthiness} (20 occurrences) and dealing with \emph{unknowns} (five occurrences). Problems related to data were reported by 21 practitioners (18.6\%) and include \emph{data governance} and \emph{data access} (both eight occurrences), and \emph{machine learning} (five occurrences). The remaining 19 problems (16.8\%) relate to user interaction, namely \emph{automation} (nine times) and \emph{user experience} (seven times).   

\small
\begin{table*}[hbt]
\caption{Analysis of comments I -- Problems for which support of researchers would be appreciated (Q4.7)}
\label{tab:codes_q4-7.I}
\begin{tabular}{p{3.2cm}lp{9.2cm}}
\hline\noalign{\smallskip}
Categories and codes & \# & Example quotes\\
\noalign{\smallskip}\hline\noalign{\smallskip}

\textbf{Engineering} & \textbf{48} &  \\

Architecture \& reuse & 16 & ``Best Practices for implementation and architectural design guidelines''; ``I'd love to see a taxonomy of self-adaptive techniques. Perhaps a set of techniques could be added to Kazman's Architecture Tactics checklist?'' \\

Adoption & 10 & ``We lack interaction with development teams that are facing similar problems. We have a huge problem explaining this area to the management structure. ... they have basically no ability to lead due to lack of competence.''; ``new organisational structures and workflows that lead to the design of more self-adaptive and resilient platforms.'' \\

Platforms \& frameworks & 4 & ``to my knowledge there is no framework on what is 'safe' or not safe to be automatically executed by a self-adaptation system.''; ``To provide a platform for capturing the domain knowledge i.e. extensible ... to manage the managed systems what kind ... KPIs can be captured, and how they are related.''  \\

Tools & 4 & ``Outlier detection ... is well understood but existing commercial tools are usually pretty weak and custom code is required to optimize''; ``One of the main problems is to get tools that can profile the running systems under certain loads.'' \\

Testing \& debugging & 4 & ``Assurance of the behavior of highly dynamic systems is still the big hurdle. Test budgets and schedules do not grow with system complexity.''; ``a pre-production cloud test environment to try them first.''  \\

Advanced features & 10 & ``Coordinate multiple, potentially conflicting, objectives - in changing environment ... reacting too quickly [is] often sub-optimal''; ``research on network protocols, these should include some level of self-awareness and should automatically provide common network self-adaptation features.''; ``How a feedback loop can be designed in a way that you later can adapt to changes''  
\\
\noalign{\smallskip}\hline\noalign{\smallskip}
\textbf{Guarantees} & \textbf{25} &  \\

Trustworthiness & 20 & ``Formal verification of the algoritmic behaviour of the overall system (correctness)''; ``validate my algorithms''; ``Safety protocols for Machine learnign in self-adapting systems''; ``What are the mechanisms should be integrated into self-adapting system to identify malicious input?'' \\
Unknowns & 5 & ``We normally capture this using some form of process based models, but these struggle with thin[g]s like unknowns.''; ``not just anomaly detection, but actually responding appropriately to the anomalies (what is appropriate?).''  
\\


\noalign{\smallskip}\hline
\end{tabular}
\end{table*}

\begin{table*}[hbt]
\caption{Analysis of comments II -- -- Problems for which support of researchers would be appreciated (Q4.7).}
\label{tab:codes_q4-7.II}
\begin{tabular}{p{2.7cm}lp{9.7cm}}
\hline\noalign{\smallskip}
Categories and codes & \# & Example quotes\\
\noalign{\smallskip}\hline\noalign{\smallskip}

\textbf{Data} & \textbf{21} &  \\

Data governance & 8 & ``Data alignment and ... its integration''; ``getting data from application behaviour helps a lot in analyzing how application performance can be further improved.''; ``Adaptive AI systems to manage huge document contents'' \\

Data access & 8 & ``Support for data science as to extract correct cause relationships vs apparently correlations''; ``For example, how much data is shared across threads, how many objects are thread-local, how much performance is lost due to locality issues''  \\

Machine learning & 5 & ``if the data/metrics can be structured and labelled in some way (i.e. scored), then perhaps it should be possible to apply ML to help identify opportunities and figure out automatically how to respond.''; ``How to use machine learning to solve the self-adaptation problems and demonstrate its performance bound'' 
\\
\noalign{\smallskip}\hline\noalign{\smallskip}
\textbf{User interaction} & \textbf{19} &  \\

Automation & 9 & ``volume of data gets to large for people to process. People get to be the bottleneck for throughput''; ``Automatic synthesis of predictive and or reconfiguration models.''; ``Approaches whereby systems of reasonable scale can monitor and fix themselves as necessary without human intervention.''\\

User experience & 7 & ``most of the problems that we faced are related to help the customer to understand the benefits of self-adaptative systems.''; ``Autoscaling should become commodity products ... As users, the complexity should be abstracted away'' \\

User involvement & 3 & ``User response can also be used for adaption (E.G. if a user constantly overrides the managed systems settings there managing system should 'learn' from the user and adapt the control algorithm for that specific user)'' \\

\noalign{\smallskip}\hline
\end{tabular}
\end{table*}
\normalsize

\subsubsection{In your organisation or in industry in general, do you see application opportunities for self-adaptation that are currently not exploited? (Q4.8)}

Of the 184 participants, 101 (54.4\%) highlight new opportunities for applying self-adaptation, while 83 do not report any. The number of participants within these two groups is almost equally split among participants who have worked with concrete self-adaptive systems and those who have not (see Q0.1) (in particular, 58 participants that worked with self-adaptive systems report opportunities, while 42 do not).

\subsubsection{Please describe or give examples of the application opportunities for self-adaptation that are currently not exploited (Q4.9)}

Eighty-five participants described in total 147 unexploited opportunities for applying self-adaptation, i.e., an average of 1.7 opportunities per participant. 

\small
\begin{table*}[hbt]
\caption{Analysis of comments I -- Opportunities for self-adaptation that are not exploited yet (Q4.9)}
\label{tab:codes_q4-9.I}
\begin{tabular}{p{3cm}lp{9.4cm}}
\hline\noalign{\smallskip}
Categories and codes & \# & Example quotes\\
\noalign{\smallskip}\hline\noalign{\smallskip}

\textbf{System activities} & \textbf{72} &  \\

 Autonomous operation & 37 & ``E.g manufacturing production line with visual inspection operators who remove defects, ... the production line can further be adapted based on the defect rate/type''; ``Self adaption could have a lot of benefits in building automation systems, like smart heating and lighting systems that takes peoples habits into consideration.''; ``making the system adaptive to adjust and act instantly based on the data without waiting would be beneficial and efficient.'' \\
 
Data management \& machine learning & 26 & ``Methods to automatically handle changes in the machine learning models and to efficiently deploy them to the edge. There is still lots of manual fine tuning that delays a timely new release.''; ``The query optimizer of database (i.e. MySQL) could utilize self-adaptation technic.'' \\
 
 Autoscaling & 9 & ``The "managed service", which is a stateful service/ data store, is provisioned for the peak capacity, which means resources are idle most of the time. If we can build reliable and efficient system that can automatically scale stateful services based on the demand, we can reduce the cost.''; ``Our microservices do not dynamically scale'' \\
\noalign{\smallskip}\hline\noalign{\smallskip}
\textbf{System properties} & \textbf{47} &  \\

Quality improvement & 26 & ``Based on the alarm certain counter actions could be initiated in order to deal with the faulty behaviour and reach a stable system state.''; ``Congestion prognosis''; ``fault tolerance''; ``Power consumption''; ``resource optimization''; ``There are many opportunities to split up [current monolithic systems] and then make them scalable such that outages are more contained. E.g. screens on trains.'' \\

Security improvement & 10 & ``Security of e.g., mobile devices that adapts based on locally identified threats as well as knowledge of risks in the environment.''; ``Automating changes in Security levels based on threat levels''; ``Detecting in-vehicle threats, detecting a system being compromised''; ``react to attack patterns'' \\

Cost effectiveness & 8 & ``IT cost reduction (e.g. software asset mgmt)''; ``The question really is: How do you do these things on the cheap (with non Silicon Valley billion dollar funding) and in contexts where mistakes might be extremely critical?'' \\

\noalign{\smallskip}\hline
\end{tabular}
\end{table*}

\begin{table*}[hbt]
\caption{Analysis of comments II -- Opportunities for self-adaptation that are not exploited yet (Q4.9).}
\label{tab:codes_q4-9.II}
\begin{tabular}{p{3.1cm}lp{9.4cm}}
\hline\noalign{\smallskip}
Categories and codes & \# & Example quotes\\
\noalign{\smallskip}\hline\noalign{\smallskip}

\textbf{Engineering activities} & \textbf{21} &  \\

Maintenance \& reuse & 15 & ``self-adapting CI/CD infrastructure based on demand''; ``Preventive maintenance''; ``Carriers are eager to get rid of human factors to improve operation and maintenance capabilities and network quality. Therefore the ICT field pays much attention to self-adaption systems.''; ``Software provisioning and automatic updates'' \\

Patterns \& libraries & 6 & ``Developing a comprehensive library of algorithms on top of the industrial monitoring systems which can be applied to analysis portion of the chain in order to drive correct self-adaptation actions would benefit the self-adaptation adoption.''; ``Cross-cloud self-adaptation''; ``patterns to provide solutions to common problems'' \\
\noalign{\smallskip}\hline\noalign{\smallskip}
\textbf{Human involvement} & \textbf{7} &  \\

Personalization & 4 & `` it would be interesting to adapt the player experience itself based on the player, mostly to better challenge them''; ``Healtcare decision making systems witch are changing outcomes and advices basd on patient status.'' \\

Human-machine interaction & 3 & ``I consider that the biggest opportunities are found within the Human Machine Interaction or Building Machine Interaction. There will be a future in which talking to a device that can modify the environment (e.g. a robot but not a phone) will be as natural as talking to a person, or seeing a machine interacting with another machine (e.g. robot taking the elevator)'' \\

\noalign{\smallskip}\hline
\end{tabular}
\end{table*}
\normalsize

\textbf{Analysis of comments:} 
Tables~\ref{tab:codes_q4-9.I} and~\ref{tab:codes_q4-9.II} summarise the findings.
We grouped the opportunities in four categories: \emph{system activity}, \emph{system property}, \emph{engineering activity}, and \emph{human involvement}. Seventy-two of the reported opportunities (i.e., 49\% of all) are related to system activity. The opportunities in this category relate to the \emph{autonomous operation} behaviour of self-adaptive systems (37 occurrences), \emph{data management and machine learning} (26 occurrences), and \emph{auto-scaling} (nine occurrences). Forty-seven opportunities (32\%) are related to system properties. In this category, the opportunities are related to \emph{quality improvement} (26 occurrences), 
\emph{security improvement} (10 occurrences), and \emph{cost effectiveness} (eight occurrences). Twenty-one of the reported opportunities (14.3\%) relate to engineering activities, in particular \emph{maintenance and reuse} (15 occurrences), and \emph{patterns and libraries} (six occurrences). Finally, seven opportunities (4.8\%) relate to human involvement, in particular \emph{personalisation} (four occurrences) and \emph{human-machine interaction} (three occurrences).  
%\newpage

%We also performed an alternative coding of opportunities according to the sector they belong to. 
%Participants described in total 66 opportunities based on the sector analysis.
%Table~\ref{tab:q4-9a} summarises the findings.
%Thirty six of 66 opportunities (55\%) are not specific to any sector. The largest sector category is Cloud \& Microservices with 11 occurrences (17\%), while important sectors are also Telco and CPS with 7 and 6 occurrences, respectively. 
%The remaining two categories are Transportation and Other, each with 3 occurrences. 
%The later includes single mentions of energy, mobile systems and healthcare. 

%We finally performed a cross analysis that revealed that (i) Cross-sector opportunities are almost evenly split among the labels of Table~\ref{tab:codes_q4-9}, (ii) the vast majority of Cloud \& Microservices opportunities are related to autoscaling. 



\begin{comment}


\begin{table}[h!]
\centering
\caption{Application opportunities for self-adaptation that are currently not exploited (Q4.9) -- labels based on sector}
\label{tab:q4-9a}
\begin{tabular}{ll}
\hline\noalign{\smallskip}
Sector & Count \\
\noalign{\smallskip}\hline\noalign{\smallskip}
Cross-sector & 36 (55\%)\\
Cloud \& Microservices & 11 (17\%)\\
Telco & 7 (11\%)\\
CPS & 6 (9\%)\\
Transportation & 3 (5\%)\\
Other & 3 (5\%)\\
\noalign{\smallskip}\hline
\end{tabular}
\end{table}

\end{comment}

\begin{framed}
\noindent \textbf{Key insight(s) from RQ4:} 
\begin{enumerate}
\item A majority of participants face difficulties when engineering or maintaining self-adaptive systems, mainly with reliable/optimal
design, design complexity, and tuning/debugging. 
\item About half of the participants encounter risks when using self-adaptation. The main risks relate to incorrect functionality and difficulty to manage environment uncertainty, as well as degraded performance and increased cost.
\item About half of the practitioners report that they would appreciate support from researchers to deal with problems they face, in particular problems related to the engineering of self-adaptive systems, guarantees, and management of data. 
\item About half of the participants see future opportunities for applying self-adaptation, in particular in relation to autonomous operation, data management and machine learning.  
\end{enumerate}
\end{framed}


\subsection{Confidence}
\label{subsec:rq5-1}

Figure\,\ref{fig:q5-1} shows the answers about how confident participants were in general about the answers they gave when answering the survey questions. The results show that almost all participants have confidence in the answers they provided to the survey questions. The numbers for all participants and those that have worked with self-adaptation are similar. 

\begin{comment}
    
\small
\begin{table}[hbt]
\centering
\caption{Confidence in answers.}
\label{tab:q5-1}
\begin{tabular}{llc}
\hline\noalign{\smallskip}
Level of confidence & All (of 184) & With self-adaptation (of 100) \\
\noalign{\smallskip}\hline\noalign{\smallskip}
Very confident & 24 (13.0\%) & 16 \\
Confident & 61 (33.2\%) & 32 \\
Sufficiently confident & 52 (28.3\%) & 28 \\
Neutral & 23 (12.5\%) & 9 \\
Somewhat unconfident & 12 (6.5\%) & 5\\
Not confident & 3 (1.6\%) & 2 \\
Not confident at all & 0 (0.0\%) & 0 \\
No answer & 9 (4.9\%) & 8 \\
\noalign{\smallskip}\hline
\end{tabular}
\end{table}
\normalsize

\end{comment}

\begin{figure}[h]
\centering
\includegraphics[width=0.85\columnwidth]{Q5.1-fig.pdf}
\vspace{-10pt}%%%
\caption{Confidence in answers.}
\vspace{-10pt}%%%
\label{fig:q5-1}
\end{figure}



