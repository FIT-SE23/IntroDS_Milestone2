Computing systems form the backbone  of many areas in our society, from manufacturing to traffic control, healthcare, and financial systems. When software plays a vital role in the design, construction, and operation, these systems are referred as software-intensive systems. Self-adaptation equips a software-intensive system with a feedback loop that either automates tasks that otherwise need to be performed by human operators or deals with uncertain conditions. Such feedback loops have found their way to a variety of practical applications; typical examples are an elastic cloud to adapt computing resources and automated server management to respond quickly to business needs. To gain insight into the motivations for applying self-adaptation in practice, the problems solved using self-adaptation and how these problems are solved, and the difficulties and risks that industry faces in adopting self-adaptation, we performed a large-scale survey. We received 184 valid responses from practitioners spread over 21 countries. Based on the analysis of the survey data, we provide an empirically grounded overview of state-of-the-practice in the application of self-adaptation.  From that, we derive insights for researchers to check their current research with industrial needs, and for practitioners to compare their current practice in applying self-adaptation. These insights also provide opportunities for the application of self-adaptation in practice and pave the way for future industry-research collaborations. 
