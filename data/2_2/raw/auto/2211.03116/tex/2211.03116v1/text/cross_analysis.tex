\section{Discussion}
\label{sec:cross_analysis}

We start the discussion with highlighting a number of observations that we derived from the data analysis. Then we perform a number of additional analysis based on cross analyses of selected data of the answers of different questions. With this cross analyses we aim to gain further insights into three topics of interest: benefits of applying self-adaptation in practice, difficulties and risks with engineering self-adaptation in practice, and research support to address problems in practice. 

\subsection{Observations}

The problems addressed by industry are in general similar to those studied by academics. Yet, one particular difference is the lack of emphasis of practitioners on the use of self-adaptation to mitigate uncertainties, which has been a key focus in research\,\cite{garlan2010seu,Relax2010Whittle,esfahani2013usa,10.1145-3487921}. A possible explanation is that practitioners avoid the term uncertainty that may be perceived as ``doubt,'' ``not clearly defined,'' or ``not under control.'' Instead, they refer to uncertainty indirectly by using a different vocabulary, such as ``conditions are not always obvious'' and ``available metrics are not always fully transparent.''

While practitioners apply self-adaptation to deal with a variety of problems, changes in business goals are less frequently solved by using self-adaptation. One possible explanation may be that business goals are usually about higher-level requirements, while the focus of self-adaptation is often targeting ``lower level''  technical problems. In addition, there is also the challenge of the mapping between business goal and technical/system metrics, which touches the line or work on dynamic software product lines\,\cite{4488260}. Yet, another explanation may be that self-adaptation has not yet been fully utilised in industry to deal with bigger system changes. We hypothesise that the latter is the case, but further study is needed to obtain deeper insight. 

The four classic management tasks of self-adaptation studied by researchers (self-healing, self-optimising, self-protecting, and self-configuring) are also relevant to practitioners. Yet, differently from academics, practitioners also emphasise the importance of improving user satisfaction, reducing costs, and mitigating risks.  

Practitioners make extensive use of tools and infrastructures to realise the different functions of self-adaptation. This points to the need for more emphasis on tools and supporting infrastructure in research. Related to that is the need for reusing solutions, for instance in the form of references architectures and patterns. While some research efforts have been taken in these directions, these issues deserve more attention. An interesting step in this direction is the development of industry relevant artifacts as outlined in\,\cite{3561846.3561852}. 

Self-adaption in software-intensive systems is often not completely automated. Humans remain involved in adaptation, either to provide parts of functions or just to supervise the system. On the one hand, for some companies this is the first step towards further automation; on the other hand, practitioners often express the need for involving humans to ensure trust by overseeing the system and take action when something unexpected happens. As such, we expect the role of humans in self-adaptation to remain important also for future industry relevant research in self-adaptation. 

It is remarkable that more than 50\% of the participants report that they face at least sometimes risks with applying self-adaptation. At the same time, about half of the practitioners express that they would appreciate support from researchers
to deal with the problems they face. This suggests that the engineering of efficient and trustworthy self-adaptive systems is a challenge in practice and that practitioners believe that support from research could benefit them to deal with these challenges. This opens opportunities for joint efforts between industry and academics. 


\subsection{Benefits of Applying Self-Adaptation in Practice}

%To gain further insights into the benefits of using self-adaptation in industry, we first look at the problems for which self-adaptation is applied (Q1.1) versus the kind of software systems built by organisations (Q0.2). We then look at the problems for which self-adaptation is applied (Q1.1) versus the benefits of applying self-adaptation (Q1.2). Finally, we look at the kind of software systems built by organisations (Q0.2) versus reuse when applying self-adaptation (Q3.5-3.7). 

When we crosscheck adaptation problems (Q1.1) versus kind of systems (Q0.2), we observe that most adaptation problems are applied to all kind of systems, while each adaptation problem is applied in one or two champion kind of systems. The three most frequently addressed adaptation problems are applied by all kind of systems. Specifically, the problem ``to optimise system performance'' is applied to all kinds of systems except transportation where ``to detect and resolve errors'' is the main adaptation problem (six occurrences), finances where ``to deal with changes in the environment'' is the main problem (five occurrences), and manufacturing where ``to automate tasks'' is the main problem (seven occurrences). Table\,\ref{tab:q1.1-q0.2} summarises the top occurrences, i.e., types of adaptation problems solved (top occurrences) versus the kind of system for which that adaptation problem is applied (top kind of systems). 




\small
\begin{table}[hbt]
\centering
\caption{Cross analysis adaptation problem solved (Q1.1) versus kind of systems (Q0.2)}
\label{tab:q1.1-q0.2}
\begin{tabular}{lc}
\hline\noalign{\smallskip}
\textbf{Adaptation problem (top occurrences/total)} & \textbf{Top kind of system} \\
\noalign{\smallskip}\hline \noalign{\smallskip\smallskip}
To optimise system performance (12/78) & Embedded/cyber-physical/IoT \\
To automate tasks (10/61) & Cloud \\
To deal with changes in the environment (9/60) & Embedded/cyber-physical/IoT \\
To detect and resolve errors (8/46) & Web/mobile\\
To configure/reconfigure a system (8/51) & Web/mobile and Cloud\\
To detect and protect a system against threats (6/46) & Web/mobile\\ 
To deal with changes in business goals (5/15) & ICT communication and networks \\ 
%Others (3) & ICT communication and networks and manufacturing \\
\noalign{\smallskip}\hline
\end{tabular}
\end{table}
\normalsize

We now look at the problems for which self-adaptation is applied (Q1.1) versus the benefits of using self-adaptation (Q1.2). 
Table\,\ref{tab:contingency_q1-1_q1-2} shows the contingency matrix. The results show that ``improving user satisfaction'' and ``reducing  costs'' are by far the most frequently perceived benefits across all types of problems solved with self-adaptation. In particular, these two benefits are mentioned approximately 70\% (+/- 4\%) on average across all problems, while ``mitigating risks'' and ``penning up new application opportunities'' are respectively mentioned 53\% (+/- 11\%) and 28\% (+/- 5\%) on average across all problems solved with self-adaptation.

\small
\begin{table}[hbt]
\centering
\caption{Contingency matrix adaptation problem (Q1.1) versus benefits (Q1.2)}
\label{tab:contingency_q1-1_q1-2}
\begin{tabular}{lcccc}
\hline\noalign{\smallskip}
Problem/Benefit & Improve user satisfaction & Reduce costs & Mitigate risks & New opportunities\\
\noalign{\smallskip}\hline\noalign{\smallskip}
Automate tasks         & 42 & 45 & 34 & 15 \\
Environment changes    & 43 & 44 & 28 & 17 \\
Optimise performance   & 12 & 10 &  6 &  5 \\ 
Changes business goals & 55 & 55 & 35 & 17 \\ 
Handle errors          & 34 & 32 & 27 & 12 \\ 
Protect system         & 22 & 23 & 24 & 11 \\ 
(Re-)configure system  & 35 & 36 & 26 & 16 \\ 
\noalign{\smallskip}\hline
\end{tabular}
\end{table}
\normalsize

Finally, we look at the potential benefits of reuse using the data of the kind of software systems built by organisations (Q0.2) versus reuse when applying self-adaptation (Q3.5-3.7). 
The top domains where solutions are frequently reused are data management with 11 occurrences and embedded/cyber-physical/IoT systems with seven occurrences. Manufacturing is the top domain where practitioners very frequently reuse solutions with seven occurrences. The most frequent type of reused artifact is module with 11 occurrences, with embedded/cyber-physical/IoT as the top domain with four occurrences used for monitoring/analytics/control. Overall, there is no specific artifact that is more reused than other, and no domain that clearly reused more or less artifacts. Only five  participants mention the reuse of patterns when engineering solutions for self-adaptation. 

\begin{framed}
\noindent \textbf{Summary for Benefits of Applying Self-adaptation in Practice.} Optimising performance and dealing with changes in the environment are the main reported problems solved using self-adaptation in the domain of embedded/cyber-physical/IoT. Not surprisingly, self-adaptation in the cloud is primarily used to automate tasks and reconfigure the system. 
%Key reported benefits of self-adaptation are improved user satisfaction and reduced costs. 
Reuse of self-adaptation solutions is mostly applied in the domains of manufacturing, data management, and embedded/cyber-physical/IoT systems. The main artifact of reuse is system module. 
\end{framed}

\subsection{Difficulties and Risks of Applying Self-Adaptation in Practice}

%To gain further insights in the difficulties practitioners face when applying self-adaptation, we look at the size of companies (Q0.3) versus reported risks and difficulties (Q4.1-4.2) with applying self-adaptation (Q4.3-4.4). Next, we look at size of companies (Q0.3) versus mechanisms used to realise self-adaptation (Q3.1-3.3).  Finally, we look at the subject of adaptation (Q2.1) versus difficulties (Q4.1-4.2) and risks (Q4.3-4.4). 

Large and small/medium organisations (Q0.3) are equally concerned about difficulties with design (Q4.1-4.2). Both types of companies are also concerned about tool support, but in different ways:  difficulties with debugging is more important for large organisations, while limitations of tools and methods more important for small/medium organisations.

When comparing large companies (>100) and small/medium companies (<100) (Q0.3), we observe no major difference in the reported frequency of encountered risks (Q4.3-4.4). The only relevant difference is that larger companies mention faults twice as much as small/medium ones; 14 occurrences for 30 large companies versus six for 70 small/medium companies. 

To crosscheck size of companies (Q0.3) versus mechanisms used to realise self-adaptation (Q3.1-3.3), we performed a dedicated coding distinguishing mechanisms that rely on tools/infrastructure versus custom mechanisms. 
%
%Motivation: One can see going from implementing monitoring to analysis to change mechanisms as a progression that goes from easier to difficult that might be more feasible in larger organizations with more resources. Similarly, it might be easier for smaller organizations to simply rely on what is available from tools out-of-the box, while larger organizations with more resources might implement custom solutions.
%
The data summary shown in Table~\ref{tab:contingency_q0-3_q3-1_q3-3} indicates that smaller/medium companies (<100) rely on tools and infrastructure to provide support for self-adaptation mechanisms, while in large companies (>100) custom solutions are more prevalent. 
Zooming into the data of mechanisms for the different stages of self-adaptation shows that almost all companies that apply self-adaptation have mechanisms in place for monitoring, but not necessarily for analysis and change, regardless of company size, but the differences are small. This suggests a progression from monitoring to analysis to change.  

\small
\begin{table}[hbt]
\centering
\caption{Contingency matrix size of companies (Q0.3) versus self-adaptation mechanisms (Q3.1-3.3)}
\label{tab:contingency_q0-3_q3-1_q3-3}
\begin{tabular}{lcc}
\hline\noalign{\smallskip}
Size company & Relying on tools/infrastructure & Custom mechanisms\\
\noalign{\smallskip}\hline\noalign{\smallskip}
1-10     & 5 (56\%) & 4 (44\%) \\
11-20    & 3 (27\%) & 8 (73\%) \\
21-50    & 5 (36\%) & 9 (64\%) \\ 
51-100   & 4 (40\%) & 6 (60\%) \\ 
$> 100$  & 7 (13\%) & 47 (87\%) \\ 
\noalign{\smallskip}\hline
\end{tabular}
\end{table}
\normalsize

Cross analysis of subject of adaptation (Q2.1) versus difficulties and risks (Q4.1-4.2) shows that the reported difficulties and risks are similarly distributed across subjects of adaptation. Most frequently reported difficulties are design issues and
people and process issues at system level (both 11 instances). Most frequently reported risks are difficulties development/operation and impact on business also at system level (six and five occurrences, respectively).  
%\newpage 
\begin{framed}
\noindent \textbf{Summary for Difficulties and Risks with Engineering Self-adaptation.} The main difficulties concern the design of self-adaptation and people and processes at system level, while the main risks relate to development/operation and impact on business, also at system level. Large companies face higher risks related to faults when applying self-adaptation. Difficulties with design is important for all, yet, debugging is more important for large companies, while small/medium companies are more concerned about limitations of tools and methods. 
\end{framed}

\subsection{Research Support to Address Problems in Practice}

%To gain further insights into possible research support for practitioners, we look at concrete self-adaptive systems built (Q2.1) versus problems for which support of researchers would be appreciated (Q4.6). We also look at the kind of software systems built (Q0.2) versus problems for which support of researchers would be appreciated (Q4.6). 

Table\,\ref{tab:q2.1-q4.6} shows the main results of the cross analysis of the data of the concrete self-adaptive systems built by the participants (Q2.1) and the problems for which practitioners would appreciate, sometimes to always, support from researchers (Q4.6).   

\begin{comment}
    

\small
\begin{table}[hbt]
\centering
\caption{Cross analysis concrete self-adaptive systems (Q2.1) built vs support from researchers (Q4.6)}
\label{tab:q2.1-q4.6}
\begin{tabular}{lc}
\hline\noalign{\smallskip}
\textbf{Characteristic system} & \textbf{Support from researchers} \\
\noalign{\smallskip}\hline \noalign{\smallskip\smallskip}
\textit{Subject of adaptation} &  \\
\noalign{\smallskip}\hline\noalign{\smallskip}
System & 12 (26.7\%) \\
Module & 9 (20.0\%) \\
Application layer & 9 (17.8\%) \\
\noalign{\smallskip}\hline\noalign{\smallskip}
\textit{Type of adaptation} & \\
\noalign{\smallskip}\hline\noalign{\smallskip}
Auto-tuning & 16 (36.4\%) \\
Auto-scaling & 13 (29.5\%) \\
Monitor/analysis & 9 (20.5\%) \\
\noalign{\smallskip}\hline\noalign{\smallskip}
\textit{Trigger of adaptation} &  \\
\noalign{\smallskip}\hline\noalign{\smallskip}
System properties & 11 (31.4\%) \\
Environment properties & 8 (22.9\%) \\
System load & 6 (17.1\%) \\
\noalign{\smallskip}\hline
\end{tabular}
\end{table}
\normalsize

\end{comment}

\small
\begin{table}[hbt]
\centering
\caption{Cross analysis concrete self-adaptive systems (Q2.1) built vs support from researchers (Q4.6)}
\label{tab:q2.1-q4.6}
\begin{tabular}{lc|lc|lc}
\hline\noalign{\smallskip}
\textit{Subject adaptation} &  Support & \textit{Type adaptation} & Support & \textit{Trigger adaptation} & Support \\
\hline\noalign{\smallskip}
System & 12 (26.7\%) & Auto-tuning & 16 (36.4\%) & System properties & 11 (31.4\%) \\
Module & 9 (20.0\%) & Auto-scaling & 13 (29.5\%) & Environment properties & 8 (22.9\%) \\
Application layer & 9 (17.8\%) & Monitor/analysis & 9 (20.5\%) & System load & 6 (17.1\%) \\
\noalign{\smallskip}\hline
\end{tabular}
\end{table}
\normalsize

The analysis shows that system, module and application layer make a total of 64.4\% of the problems for which practitioners would appreciate support from researchers. In terms of type of adaptation, 84.6\% of the problems for which practitioners would appreciate support from researchers concern auto-tuning, auto-scaling, and monitoring and analysis. Finally, 74.1\% of the problems for which support would be appreciated concern adaptation triggered by system properties, environment properties, and system load.   

When crosschecking the kind of software systems built by the practitioners (Q0.2) versus the problems for which they would appreciate at least sometimes support from researchers (Q4.6), we found that except for one kind of system, support from researchers would be appreciated across all kinds of systems built by the practitioners. For e-commerce none of the seven participants expressed interest in regular support from researchers (four of them would very rarely appreciate support). On the other hand, eight out of 11 (72.2\%) participants that work in the domain of ICT communication and networks would regularly appreciate support to address their problems. The numbers for the other domains range from 22.9\% to 56.3\%. 

%\newpage
\begin{framed}
\noindent \textbf{Summary for Research Support to Address Problems in Practice.} A majority of practitioners would appreciate support from researchers. These problems concern self-adaptation applied at system level, a module of the system, or the application layer. The main problems relate to auto-tuning, auto-scaling, and monitoring and analysis. Triggers of adaptation concern dealing with system and environment properties, and system load. The problems crosscut different kinds of systems, but particularly ICT communication and networks.  
\end{framed}



\section{Threats to Validity }\label{sec:ThreatsToValidity}

We discuss validity threats of our study using the guidelines described in~\cite{Wohlin2012}. We look at construct validity that refers to the extent to which we obtained the right measure and whether we defined the right scope for the study goal, external validity that refers to the extent to which the findings can be generalized, and reliability that refers to the extent to which we can ensure that our results are the same if our study is done again. 

\subsection{Construct Validity} 

The survey starts from the assumption that practitioners are sufficiently familiar with the basic concepts of self-adaptation. We used the term self-adaptation to formulate questions about systems (or parts) that are equipped with a feedback loop. Hence, most questions required basic knowledge of the concept of self-adaptation. Analysis of the results makes it clear that practitioners have a basic understanding these concepts. Yet, we used several measures to reduce possible misinterpretations. We introduced the notion of self-adaptation at the start of the survey using a standard model with a feedback loop that we illustrated with typical examples. We elicited feedback for several participants on this description and the questions during a pilot. This feedback enabled us to enhance the description and clarify some of the questions. In addition, we selected participants with sufficient experience from a variety of domains. The confidence in the answers (Q5.1) confirms that the participants believed that their answers were trustworthy.   

\subsection{External Validity}

A potential threat to validity may be the generalisation of the study results.  Core to this threat is the selection of the sample of the target population. If this population may not have been representative, the study results may be imprecise and hence not generalisable. Since we used a non-probabilistic sampling method, there is a potential risk that the sample used to conduct the survey is biased and not representative of the target population. To mitigate the validity threat we mainly reached out to practitioners from our networks with industry. To ensure that participants have the required experience, we included questions asking about personal experience with engineering self-adaptive systems in practice. The results of the demographics of our sample show that the participants were active practitioners with sufficient expertise in various roles across companies of different sizes.  In addition, we worked in total with eight teams from different areas that contacted practitioners from all over the world. This ensured a well-balanced population on a global scale. Because several of the researchers involved in this study are active in the field of engineering self-adaptive systems, the practitioners invited from our networks may have been biased and inclined to apply self-adaptation more often. To anticipate this threat, we did not particularly focus on practitioners that we have worked within projects, but rather invited practitioners in various software engineering roles that are active across a wide range of domains.  

\subsection{Reliability}
 
Data analysis, in particular qualitative analysis (coding of answers with free text), are creative tasks that are to some extent subjective. Performing these tasks may be influenced by the experience (and even opinions) of the coders~\cite{Fernandez2016}. To mitigate a potential interpretation bias, we followed a thorough coding scheme.  The coding tasks were distributed among teams of two authors (one team of three). The authors of each team performed coding of the data independently and discussed where needed until an agreement was reached. All coding tasks were then distributed again among two authors. These authors repeated the coding independently from the initial coding. The results were then compared with the initial coding by these two authors. Any discrepancies were discussed among the two authors until consensus was reached. The coding was finally crosschecked with the authors that did the original coding to reach consensus. Finally, all material of the survey, including the raw data and the coding are publicly available.\footnote{https://people.cs.kuleuven.be/danny.weyns/surveys/sas-in-industry/}  




\begin{comment} 

In this category, we look at the following cross analyses: 

\begin{enumerate}
    \item[E1] Kind of software systems (Q0.2) versus problems for which self-adaptation is applied (Q1.1)
    \item[E2] Problems for which self-adaptation is applied (Q1.1) versus benefits of self-adaptation (Q1.2) 
    \item[E3] Multiplicity of adaptation goals (Q1.1 and Q2.1)
    \item[E4] Kind of software systems (Q0.2) versus versus mechanisms used to realise self-adaptation (Q3.1-3.3) 
    \item[E5] Size of companies (Q0.3) versus mechanisms used to realise self-adaptation (Q3.1-3.3) 
    \item[E6] Kind of software systems build by organisations (Q0.2) versus reuse (Q3.5-3.7)
    %\item Roles (Q0.4) versus types of adaptation (Q2.1)
\end{enumerate}    

\subsection{Difficulties, Risks, and Trust}

In this category, we look at the following cross analyses: 

\begin{enumerate}    
    \item[D1] Size of companies (Q0.3) versus difficulties (Q4.1-4.2) and risks (Q4.3-4.4)
    \item[D2] Subject of adaptation (Q2.1) versus  difficulties (Q4.1-4.2) and risks (Q4.3-4.4)
    \item[D3] Mechanisms to realise self-adaptation (Q3.1-3.3) versus difficulties (Q4.1-4.2) and risks (Q4.3-4.4)
    \item[D4] Difficulties (Q4.1-4.2) versus risks (Q4.3-4.4)
    \item[D5] Difficulties (Q4.1-4.2) versus trust (Q3.7-3.8)
    \item[D6] Risks (Q4.3-4) versus trust (Q3.7-3.8)
\end{enumerate}

\subsection{Research Support}

In this category, we look at the following cross analyses: 

 \begin{enumerate}   
    \item[R1] Kind of software systems (Q0.2) versus problems support researchers (Q4.6)
    \item[R2] Types of adaptation (Q2.1) versus problems support researchers (Q4.6)
\end{enumerate}



ALLOCATES


Ilias Gerostathopoulos, Patricia Lago: E1, E2, D1, D2 

Nadeem Abbas, Jesper Andersson: E3, E4, D3, D4

Stefan Biffl, Angelika Musil, Juergen Musil: E5, E6, 

Tomas Bures, Premek Brada: E3, E4, D5, D6

Matthias Galster, Patros Panos: E1, E2, D5, D6 

Amleto Di Salle, Patrizio Pelliccione: D1, D2, R1, R2  

Grace Lewis, Marin Litoiu: E5, E6, D3, D4 
 
\end{comment}