%\section{Introduction}
%\label{sec:introduction}
%\IEEEraisesectionheading{}
\section{Introduction}\label{sec:introduction}


%=============== Context ===============

%\subsection{Problem and Context}
%\label{subsec:context}

Computing systems form the backbone of our factories, traffic control systems, healthcare, telecommunication, financial systems, and so forth. When software plays a vital role in their design, 
 construction, and operation, these systems are often referred to as software-intensive systems~\cite{Holzl2008}. The trustworthiness and sustainability of these systems is vital for our society~\cite{2889160.2889212,978-3-319-48992-6_1}. Yet, building and maintaining trustworthy and sustainable systems is challenging due to complexity that arises from the growing demands on these systems, their continued integration, the uncertain operating conditions they face, the fast speed of technological progress, etc. These challenges have been a continuous driver for new and innovative approaches to design, develop, and operate software-intensive systems. One common approach today is so called DevOps in which development and operation are blended, allowing system components to be easily evolved and redeployed without impacting their operation~\cite{cheng2009software}. 

A classic approach to address the increasing complexity of software-intensive systems is transferring control from humans~\cite{Lethbridge2005SSE} to software components by equipping systems with feedback loops that automate tasks that otherwise need to be performed by human operators. 
%
These feedback loops monitor the system and its environment, reason about the system behaviour and its goals, and adapt the system to ensure its goals under changing conditions, or gracefully degrade if necessary. Such goals can be very diverse, ranging from ensuring a required level of performance under uncertain workload conditions, dealing with errors caused by external services that are difficult to predict, or defending the system against malicious attacks and the problems they may cause. A typical example is a feedback loop deployed in a cloud environment that expands or decreases computing resources to meet changing demands while minimising the cost of operation. Another example is a container framework that performs autoscaling in a microservice deployment. 

The principles of applying feedback control to software-intensive systems have been the subject of active study in academia. Back in 1998, Oreizy et. al.~\cite{oreizy1999aba} presented a seminal paper at the International Conference on Software Engineering (ICSE) where the authors introduced the notion of \textit{self-adaptation} that comprises two simultaneous processes: system adaptation that is concerned with detecting and handling changing circumstances, and system evolution that is concerned with the consistent application of change over time. A few years later, Garlan et. al.~\cite{garlan2004rainbow} stated the crucial role of architectural models as first-class citizens that enable a system to reason about system-wide change and adapt itself accordingly to achieve or maintain its goals. Blair et. al.~\cite{Blair2009MR} consolidated and elaborated on these principles in what is now generally known as ``models at runtime.'' In 2007, Kramer and Magee~\cite{Kramer2007SMS} stated the crucial role of software architecture in the realisation of self-adaptive systems, distinguishing adaptation management from goal management. Over the last decade, the research community has developed a vast body of knowledge and know-how on principles, see e.g.,~\cite{andersson2009modeling,Blair2009MR,FilieriICSE14,978-3-319-74183-3-12}, models and languages~\cite{Relax2010Whittle,2593929.2593944,2555612,MetzgerQMBP20}, processes and methods~\cite{Andersson2013,Cheng2014,8008800}, patterns~\cite{1808984.1808990,weyns2013patterns,9223653}, and frameworks~\cite{garlan2004rainbow,Rouvoy2009,1882291.1882296} to engineer self-adaptive systems. Researchers have documented a substantial number of literature reviews and surveys on various topics in self-adaptive systems, such as the benefits of self-adaptation~\cite{6224395}, requirements for self-adaptive systems~\cite{978-3-319-05843-65},  approaches to realise self-adaptation~\cite{MACIASESCRIVA20137267,MAHDAVIHEZAVEHI20171,9223653,7929422}, the use of formal methods in self-adaptive systems~\cite{2347583.2347592}, self-protection~\cite{2555611},  
the notion of uncertainty~\cite{MAHDAVIHEZAVEHI201745,10.1145-3487921}, and the use of machine learning in the realisation of self-adaptation~\cite{3469440}, among others. Basic research works in the field of self-adaptation are for example~\cite{huebscher2008survey,1516533.1516538,cheng2009software,Lemos2013roadmap,weyns2021introduction}. 

In parallel, the principles of feedback control have been studied and applied in  industry. For example, about two decades ago, IBM launched its legendary initiative on  autonomic computing~\cite{kephart2003vision}. Inspired by the autonomic nervous system of the human body, the central idea of autonomic computing was to enable computing systems to manage themselves based on high-level goals. Four classic goals are self-optimisation, self-healing, self-protection, and self-configuration. Autonomic computing delegates the complexity of system operation to the machine aiming to reduce the time required by operators to resolve system difficulties and other maintenance tasks such as software updates. Over the years, industrial solutions based on feedback loops have found their way to practical applications, for instance in the domain of elastic cloud to adapt computing resources and automated management of server parks to deal with changing business needs,  e.g.,\,\cite{beyer2016,spyker2020}.

While the output of academic research is documented in research articles, journal volumes, and books, the current practice of self-adaptation in industry has never been systematically described. 
%nor what drives the adoption of self-adaptation in industry and whether industry faces challenges related to self-adaptation that researchers have not explored.
%Hence, it is not clear whether the body of knowledge developed by academics has been recognised by practitioners, nor whether they have developed similar or different solutions.
%independently of academic results. 


\subsection{Objective and Research Questions}
\label{subsec:research_objective_and_rqs}

Our general objective is to better understand the state of practice of self-adaptation in industry. To that end, we perform a large-scale survey with active practitioners. Concretely, this survey aims at shining a light on what motivates practitioners to apply self-adaptation, what kind of problems they solve using self-adaptation, how practitioners design and develop self-adaptive systems, whether they follow any established practices, what difficulties and risks they face in adopting self-adaptation, and what future opportunities industry sees for the application of self-adaptation. 

To the best of our knowledge, no systematic study has been done that investigates and these issues. Hence, there is no clear and documented view of why and how the principles of self-adaptation are applied in practice, and what challenges practitioners face when realising self-adaptation.
%\pat{I also have problems on connecting the previous sentence with the ambition of the work above and the research questions below.} 
Investigating industrial practice on self-adaptation and answering the questions targeted by this study will help narrow the gap between industry and academia. It aims at helping researchers in academia to get a better picture of how self-adaptation is applied in practice, the industrial needs in realising self-adaptation, and what problems practitioners face. We conjecture that having a better picture about industry practice will help the research community to position their efforts with respect to industrial needs and make well-informed decisions to set future research objectives, both fundamental and applied. On the other hand, drawing a picture of the state-of-the-practice can also benefit industry by sharing the motivations and potential benefits of self-adaptation, directing them towards relevant sources of information such as best practices, 
and identifying opportunities for collaboration with researchers to address the problems they face. 

We aim to answer the following concrete research questions:

\begin{description}
    \item[RQ1:] What drives practitioners to apply self-adaptation in software-intensive systems?   
    \item[RQ2:] How do practitioners characterise self-adaptation?
    \item[RQ3:] How do practitioners apply self-adaptation in industrial software-intensive systems?
    \item[RQ4:] What are the experiences of practitioners with applying self-adaptation and do they see opportunities for how and where to apply self-adaptation? 
\end{description}

With RQ1, we want to investigate the motivations of practitioners for applying self-adaptation, the kinds of  industrial systems for which self-adaptation is applied, and the types of problems they solve using self-adaptation. 
%
In academic research, self-adaptation has been proposed for two main complementary problems\,\cite{weyns2021introduction}: 1) to automate the management of complex software-intensive systems based on high-level goals provided by operators, and 2) to deal with operating conditions that are hard to predict before deployment and need to be resolved during operation (i.e., mitigating uncertainties). Key management tasks for self-adaptation are self-healing, self-optimisation, self-protection, and self-configuration. We want to understand whether industry uses the principles of self-adaptation to deal with the same or different problems, and whether and how they relate to the classic system and software management tasks. Answering RQ1 will shine a light on application areas, motivations, and concrete problems for which self-adaptation is applied by practitioners or could be applied by practitioners who currently do not use self-adaptation. This may provide academics with insights in relevant areas to drive and validate research results on self-adaptation. The results may also indicate applications and problems that are not yet explored in industry and may benefit both academia and industry.  

With RQ2, we aim to investigate the perception of practitioners on the concept of \emph{self-adaptation}. We are particularly interested in how practitioners characterise self-adaptation as a property that enables a system to adapt itself at runtime. 
%
%
To that end, we will elicit concrete examples of what they understand by self-adaptation. This will give us better understanding of whether and how practitioners understand the concept of self-adaptation, what terminology they use, whether there are any differences in the viewpoints on what constitutes self-adaptation, and whether they consider self-adaptation altogether useful. This may also shine a light on whether there are any (emerging) industrial standard practices, e.g., a \,technology stack or tools.
%
Answering RQ2 will help researchers to get a better picture of how practitioners understand the concept of self-adaptation. On the other hand, the insights may reveal potential opportunities for practitioners to benefit from expertise of other practitioners as well as knowledge developed by researchers. 

With RQ3, we aim at examining how self-adaptation has been realised and used in industry. We are particularly interested in mechanisms, tools, benchmarks, and processes employed in the industry to engineer self-adaptive solutions. 
%
We will pay attention to the degree of automation and the role of humans in runtime adaptation as this is commonly considered important for trust in software-intensive systems, see e.g.,\,\cite{978-3-030-00761-4-4}. 
%
Furthermore, we are interested in comparing industrial practices with solutions developed by academics, such as modelling techniques, frameworks, and verification techniques. We also want to understand how practitioners  obtain  trust  in  the  self-adaptive  solutions  they employ.  Answering RQ3 will provide insights into best practices on how practitioners realise self-adaptation. It will highlight the criteria that practitioners use to apply and realise self-adaptation solutions and may shine a light on to what extent solutions from the research community have been adopted in  industry. These insights will open opportunities for both academia and industry to steer future research and improve practical applications.  

Finally, with RQ4, we want to understand the difficulties and risks, if any, that practitioners experience in the design, implementation, and other engineering activities of self-adaptive systems. We also will probe whether practitioners face problems for which they would appreciate support from researchers. Finally, we elicit opportunities that practitioners see for applying self-adaptation  that are not exploited yet. Answering RQ4 may help to fill the gap between academia and industry. Furthermore, identifying problems and risks may trigger new collaborative studies to investigate and address these challenges. Such studies are likely to bridge the gap and result in more targeted research and improved industrial applications of self-adaptive systems.

\subsection{Contributions}
\label{subsec:contributions}
By drawing a landscape of the use of self-adaptation in industry, the survey results benefit both researchers and practitioners. Concretely, the contributions of this study are: 
\begin{itemize}
    \item An empirically grounded overview of state-of-the-practice in the application of self-adaptation; 
    \item Insights for researchers to assess their current research in relation to industrial needs;
    %, both fundamental and applied. 
    \item Insights for practitioners to assess the level of their current practice in applying self-adaptation; 
    \item Additional prospects for applying self-adaptation in practice and opportunities for industry-research collaborations. 
\end{itemize}

Preliminary results of this study were reported in\,\cite{3524844.3528077}. That paper only considered a small subset of questions and reported initial results based on one batch of data. 

%=============== Paper outline ===============

\subsection{Outline}
\label{sec:paper_outline}
%In Section~\ref{sec:related_work} we discuss background and related work. 
In Section~\ref{sec:research_method} we present the study design with the  survey questions and analysis methods used. Section~\ref{sec:results} presents the results for each research question and provides key insights for each research question. In Section~\ref{sec:cross_analysis}, we derive insights from the study results for researchers and practitioners. Section~\ref{sec:ThreatsToValidity} discusses threats to validity. Finally, we wrap up and conclude in Section~\ref{sec:conclusions}.

