\small
\begin{table*}[hbt]
\caption{Analysis of comments I -- Problems for which support of researchers would be appreciated (Q4.7)}
\label{tab:codes_q4-7.I}
\begin{tabular}{p{3.2cm}lp{9.2cm}}
\hline\noalign{\smallskip}
Categories and codes & \# & Example quotes\\
\noalign{\smallskip}\hline\noalign{\smallskip}

\textbf{Engineering} & \textbf{48} &  \\

Architecture \& reuse & 16 & ``Best Practices for implementation and architectural design guidelines''; ``I'd love to see a taxonomy of self-adaptive techniques. Perhaps a set of techniques could be added to Kazman's Architecture Tactics checklist?'' \\

Adoption & 10 & ``We lack interaction with development teams that are facing similar problems. We have a huge problem explaining this area to the management structure. ... they have basically no ability to lead due to lack of competence.''; ``new organisational structures and workflows that lead to the design of more self-adaptive and resilient platforms.'' \\

Platforms \& frameworks & 4 & ``to my knowledge there is no framework on what is 'safe' or not safe to be automatically executed by a self-adaptation system.''; ``To provide a platform for capturing the domain knowledge i.e. extensible ... to manage the managed systems what kind ... KPIs can be captured, and how they are related.''  \\

Tools & 4 & ``Outlier detection ... is well understood but existing commercial tools are usually pretty weak and custom code is required to optimize''; ``One of the main problems is to get tools that can profile the running systems under certain loads.'' \\

Testing \& debugging & 4 & ``Assurance of the behavior of highly dynamic systems is still the big hurdle. Test budgets and schedules do not grow with system complexity.''; ``a pre-production cloud test environment to try them first.''  \\

Advanced features & 10 & ``Coordinate multiple, potentially conflicting, objectives - in changing environment ... reacting too quickly [is] often sub-optimal''; ``research on network protocols, these should include some level of self-awareness and should automatically provide common network self-adaptation features.''; ``How a feedback loop can be designed in a way that you later can adapt to changes''  
\\
\noalign{\smallskip}\hline\noalign{\smallskip}
\textbf{Guarantees} & \textbf{25} &  \\

Trustworthiness & 20 & ``Formal verification of the algoritmic behaviour of the overall system (correctness)''; ``validate my algorithms''; ``Safety protocols for Machine learnign in self-adapting systems''; ``What are the mechanisms should be integrated into self-adapting system to identify malicious input?'' \\
Unknowns & 5 & ``We normally capture this using some form of process based models, but these struggle with thin[g]s like unknowns.''; ``not just anomaly detection, but actually responding appropriately to the anomalies (what is appropriate?).''  
\\


\noalign{\smallskip}\hline
\end{tabular}
\end{table*}

\begin{table*}[hbt]
\caption{Analysis of comments II -- -- Problems for which support of researchers would be appreciated (Q4.7).}
\label{tab:codes_q4-7.II}
\begin{tabular}{p{2.7cm}lp{9.7cm}}
\hline\noalign{\smallskip}
Categories and codes & \# & Example quotes\\
\noalign{\smallskip}\hline\noalign{\smallskip}

\textbf{Data} & \textbf{21} &  \\

Data governance & 8 & ``Data alignment and ... its integration''; ``getting data from application behaviour helps a lot in analyzing how application performance can be further improved.''; ``Adaptive AI systems to manage huge document contents'' \\

Data access & 8 & ``Support for data science as to extract correct cause relationships vs apparently correlations''; ``For example, how much data is shared across threads, how many objects are thread-local, how much performance is lost due to locality issues''  \\

Machine learning & 5 & ``if the data/metrics can be structured and labelled in some way (i.e. scored), then perhaps it should be possible to apply ML to help identify opportunities and figure out automatically how to respond.''; ``How to use machine learning to solve the self-adaptation problems and demonstrate its performance bound'' 
\\
\noalign{\smallskip}\hline\noalign{\smallskip}
\textbf{User interaction} & \textbf{19} &  \\

Automation & 9 & ``volume of data gets to large for people to process. People get to be the bottleneck for throughput''; ``Automatic synthesis of predictive and or reconfiguration models.''; ``Approaches whereby systems of reasonable scale can monitor and fix themselves as necessary without human intervention.''\\

User experience & 7 & ``most of the problems that we faced are related to help the customer to understand the benefits of self-adaptative systems.''; ``Autoscaling should become commodity products ... As users, the complexity should be abstracted away'' \\

User involvement & 3 & ``User response can also be used for adaption (E.G. if a user constantly overrides the managed systems settings there managing system should 'learn' from the user and adapt the control algorithm for that specific user)'' \\

\noalign{\smallskip}\hline
\end{tabular}
\end{table*}
\normalsize