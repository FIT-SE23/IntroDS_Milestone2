\small
\small
\begin{table*}[hbt]
\caption{Analysis of comments -- Difficulties with engineering or maintaining self-adaptive systems (Q4.2)}
\label{tab:codes_q4-2}
\begin{tabular}{p{3.6cm}lp{8.8cm}}
\hline\noalign{\smallskip}
Categories and codes & \# & Example quotes\\
\noalign{\smallskip}\hline\noalign{\smallskip}

\textbf{Design issues} &  \textbf{43} &  \\

Reliable/optimal design & 26 & ``With high availability requiremets, the chance something fails somewhere sometime is close to a 100\%. The systems needs to be designed to still provide service despite erroneouse behavior or failing parts in the system.''; ``the main challenge is to design adaptation function with respect to computation context'' 
%- e.g., selecting right trade-off between shutting down instance or keeping it running longer time since boot of instance can be time-consuming.'' 
\\

Design complexity & 17 & ``Complexity in defining the adaptation rules. Conditions are not always obvious.''; ``Self-adaptiveness or resilience have to be taken into consideration at each stage of the 
%software production and operation 
... workflow. This is really a challenge as more often than not these are concepts that are completely obscure to the average programmer/devop mind.'' 
\\
\noalign{\smallskip}\hline\noalign{\smallskip}
\textbf{Lifecycle issues} & \textbf{42} &  \\

Tuning/debugging & 19 & ``Debugging the root cause of a scaling failure might be time-consuming: also, in some cases the problem might be outside of your control (e.g. temporary lack of EC2 Spot capacity in AWS)'' \\

Limitations tools/methods & 13 & ``The metrics available are not always fully transparent and built with auto-scaling in mind''; ``IAM permissions are hard to deal with when configuring these self-adaptive systems. Usually, the permission to scale or to notify is not properly configured.''\\

System/environment evolution & 10 & ``If the functionality is not designed in from the beginning then it is a huge amount of work to implement later.''; ``System architecture over lifetime (nee features to be added...)''
\\
\noalign{\smallskip}\hline\noalign{\smallskip}
\textbf{Runtime issues} & \textbf{30} &  \\

Runtime uncertainty & 17 & ``Many self-adaptive systems are based on unproven heuristics. Therefore, they usually do not work in many cases.''; ``It is hard to guess how much can the environment affect the system. ... 
%Usually, we develop the system on one or a limited set of units. 
It is hard to extend the parameters to cover whole production.'' \\

Data collection/evaluation & 7 & ``Gathering quantitative data samples to evaluate the performance is very complicated.''; ``sensors gives wrong reading values'' \\

Resources required & 3 & ``Sometimes it doesn't react fast enough. It also takes computation resources for this self-adaptive software, and the compute resources use increases with the number of incoming requests.'' \\

Delayed/missing runtime changes & 3 & ``Autoscaling is often too slow or triggered too late.''; ``Notifications are delayed or missed'' 
\\
\noalign{\smallskip}\hline\noalign{\smallskip}
\textbf{People and process issues} & \textbf{19} &  \\

Skills/experience & 14 & ``Every self-adapt system must be tuned up which is sometimes tricky and needs high skilled engineers.''; ``The Kubernetes/Openshift cloud and centralized log storage ... require experienced administration staff and vast knowledge of many networking concepts (... DNS, NAT).'' \\

Process and management & 9 & ``We are not yet very experienced ... the main challenges were to convince the central IT department this was the way to go, then to design the system, and obviously to master the technology itself.'' \\

Automation & 1 & ``often automation is not trusted enough by humans. humans want to stay in the loop.'' \\




\noalign{\smallskip}\hline
\end{tabular}
\end{table*}
\normalsize
