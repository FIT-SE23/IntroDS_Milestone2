\small
\begin{table*}[hbt]
\caption{Analysis of comments -- Mechanisms or tools used to change a managed system or parts of it (Q3.3).}
\label{tab:codes_q3-3}
\begin{tabular}{p{3cm}lp{9.7cm}}
\hline\noalign{\smallskip}
Categories and codes & \# & Example quotes\\
\noalign{\smallskip}\hline\noalign{\smallskip}

\textbf{Change mechanism} & \textbf{83} &  \\

Scaling mechanisms & 36 & ``The server-side system has a load balancer. Hence we increase the number of workers behind the load balancer to decrease the average response time for the users.''; ``It adjusts the number of worker nodes.''; ``Adding a completely similar server / serverless Lambda instance''; \\

% ### autoscaling 
% adding/removing infrastructure resources & 14 & ``Provisioning additional resources (e.g. VMs)''; ``Adding a completely similar server / serverless Lambda instance''; ``Mainly on increased load additional services and resources are added/removed'' \\
%; ``Basically adding computational power through infrastructure as code.'' \\

Reconfiguration & 25 & ``The adaptation directly adjusts the period between the packet send events, as well as the number of packets allowed during each send event. [...]; ``Depending on context, controlled variables are managed through different automation systems.''; ``reconfiguration of the management entity ... to support a larger (or smaller) scale distributed system''; ``load balancer/director that may support controlling the exposure facade towards the system environment. '' \\
%reconfiguration of the management entity (in the cloud) to support a larger (or smaller) scale distributed system'' \\



% ### reconfiguration 
% adjusting parameters of control loop & 7 & ``Optimisation/scenario modelling based on the incrementally trained ML-model leads to new operational parmeters which is fed into the control system.''; ``There are disturbance and controlled variables...and the controlled ones can be changed. Depending on context, controlled variables are managed through different automation systems.'' \\

% ### reconfiguration
% load balancing & 2 & ``It can split its serving units into n pieces so that it can shed load to other instances of itself that have less load.'' \\
%; ``a system may be supported by load balancer/director that may support controlling the exposure facade towards the system environment. '' \\

Non-automated & 12 & ``To effect change on the managed system, the results from the tool need to be approved by an engineer, and are then acted on by the mining and plant teams. These processes are for the most part not automated [...].''; ``Generating alerts and expecting humans to resolve the error manually based on suggestions.''; ``Did not do this [...]. Based on safety protocols this could not be secured'' \\

% ### non-automated
% notification & 4 & ``Generating alerts and expecting humans to resolve the error manually based on suggestions.'' \\

% ### non-automated
% none & 3 & ``Did not do this (if I understand the question right). Based on safety protocols this could not be secured''; ``None'' \\

%Changing application logic & 9 & ``If there are many unsuccessful runs, the mechanism is disabled for a while to prevent wasting CPU time.''; ``Depending on the fault detected, the system can go into fail-safe modes [...]. This is done through re-parametrization at runtime and/or through modification of the lifecycle state of components [...]''; ``It uses a search algorithm to find candidates of schedules, and then evaluate the performance using the cost model.'' \\ 
%; Iteratively, it will find the optimal implementation (schedule) for the operators and then apply it in real applications.'' \\

Restarting/deploying & 7 & ``Mostly just restarting the managed subsystems. In the case of Kubernetes HPA, its the horizontal scaling (up/down) of the Pods''; ``Generally restarts the unhealthy workload, but in the case of autoscaling can also be used to add or remove replicas''; ``... our pipelines use simple bash scripts to deploy previous versions when new versions fail.'' \\

% ### restarting/deploying
% automatic code deployment & 2 & ``... our pipelines use simple bash scripts to deploy previous versions when new versions fail.'' \\
% ; ``[...] automatic deployment to staging environment of feature branches for which a merge request has been created'' \\

Migration & 3 & ``Once the control process informs the control plane, it starts a workflow what we call as instance warming workflow which will dump items that supposed to go to that node from another replica and fills them.''; ``virtual machine (VM) migration or creation.'' 
%
% ### migration
% VM migration & 1 & ``virtual machine (VM) migration or creation.'' \\
% This is because a controller can be considered as a software running on a VM. Whenever the controller location changes, VM migration is required. Similarly, a new VM will be created if a new controller is needed for increasing network workload. '' \\
%
\\\noalign{\smallskip}\hline\noalign{\smallskip}
\textbf{Change enacting tool} & \textbf{19}
\\
Kubernetes & 9 & ``Mostly just restarting the managed subsystems. In the case of Kubernetes HPA, its the horizontal scaling (up/down) of the Pods''; ``... to change topology we simply use K8S api to add/remove worker pods'' \\

AWS & 7 & ``AWS based in-built auto scaling capabilities ''; ``Use the AWS ElasticLoadBalancer and also trigger actions via AWS Lamda functions when required.'' \\

Other & 3 & ``IBM ITM, Log Analyzer, TCAM''; ``UC4 Automation Engine workflows that orchestrate kubernetes clusters''; ``Build-in Openshift mechanisms'' \\

\noalign{\smallskip}\hline
\end{tabular}
\end{table*}
\normalsize