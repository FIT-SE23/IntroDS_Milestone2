\small
\begin{table*}[h!]
\caption{Analysis comments -- Mechanisms or tools used to analyze conditions of a managed system (Q3.2).}
\label{tab:codes_q3-2}
\begin{tabular}{p{3cm}lp{9.6cm}}
\hline\noalign{\smallskip}
Categories and codes & \# & Example quotes\\
\noalign{\smallskip}\hline\noalign{\smallskip}

\textbf{Analysis mechanism}  & \textbf{73} &  \\

Data analysis methods & 18 & ``I think it uses some rolling average or some similar algorithm to estimate whether to scale up or down.''; ``simple statistical inferences based on metrics and simple rules encoded by developers.''; ``statistical analysis of data''\\

Comparison to threshold & 16 & ``Comparing the error rate with constant/dynamic thresholds.''; ``Hard  coded critical boundaries like min max values which lead to switching over to emergency modes [...]''; ``when it falls below Service Level Agreements this indicates a need for auto-scaling'' \\

Metric(s) calculation & 12 & ``Failure rate is used to measure quality of adaptation parameters.''; ``Capturing performance of each node. ''; ``Measurement of traffic load, CPU utilization, and general availability metrics (reachability, status, ...)'' \\

Learning & 12 & ``Each station has a kind of edge computing component that performs some analysis based on machine learning results.''; ``It tracks both the internal working conditions (load) of itself as a serving component, and learns about overall serving conditions.''; ``The system uses biosensory feedback to determine the riders' happiness [...]'' \\

Custom rules & 9 & ``Mostly a simple ruleset gleaned by experimentation and observing how the resulting adaption steps perform at runtime.''; ``we have alertmanager to set up some rules that are known to be issues that have clear solutions'' \\

Autoscaling policy & 5 & ``[...] the response of the scheduler is parsed and the queue length is evaluated. If greater than zero, the flow performs a SCALE UP operation. If equal to zero, the flow performs a SCALE DOWN operation.'' \\

Semantic reasoning & 1 & ``Reasoning on knowledge graphs'' 
\\\noalign{\smallskip}\hline\noalign{\smallskip}
\textbf{Analysis tool} & \textbf{23}
\\
AWS analysis tools & 9 & ``Analytics functions native to the cloud environment the system runs in (AWS).''; ``AWS based auto-scaling conditions as provided in the Cloud formation setup of the cluster'' \\

Kubernetes stack & 7 & ``The master nodes have all sorts of different components such as the kube-scheduler, controllers and state db (etcd), that are managed via the kube-apiserver. ''; ``Built-in Kubernetes/Openshift mechanisms [...]'' \\

Dynatrace & 2 & ``analyze was done by Dynatrace or by Keptn itself by checking against thresholds'' \\

Other & 5 & ``We mainly use rule-based systems like Splunk to automatically analyse production metrics against patterns.''; ``Default tooling from Azure''; ``Kibana'' \\

\noalign{\smallskip}\hline
\end{tabular}
\end{table*}
\normalsize