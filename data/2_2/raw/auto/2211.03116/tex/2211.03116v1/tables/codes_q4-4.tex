\small
\begin{table*}[hbt]
\caption{Analysis of comments I -- Risks faced when engineering self-adaptive systems (Q4.4).}
\label{tab:codes_q4-4.I}
\begin{tabular}{p{3.6cm}lp{8.8cm}}
\hline\noalign{\smallskip}
Categories and codes & \# & Example quotes\\
\noalign{\smallskip}\hline\noalign{\smallskip}

\textbf{Faults} & \textbf{20} &  \\

Incorrect functionality & 7 & ``Automation can lead to unexpected values''; ``The process might be OOM killed if the self-adaptive system doesn't function correctly (i.e. bugs).'' \\
Wrong results & 4 & ``incorrect results''; ``Wrong decisions based on faulty models'' \\
Misconfiguration & 4 & ``Tuning autoscaling settings can be problematic resulting in unexpected results.''; ``Wrong threshold levels may lead to unwanted responses. '' \\
Network failure & 2 & ``Giving control to software that can change production environments can cause network failure.'' \\
Other & 3 & ``data loss''; ``[...] heuristics that work well on some applications, do not always perform the best for all applications.'' \\
\noalign{\smallskip}\hline\noalign{\smallskip}
\textbf{Difficulties with development/operation} & \textbf{16} &  \\

Difficult to manage environment uncertainty & 6 & ``We face a risk of underestimating environment variability.''; ``Legacy monitoring solutions don't cope well with environments that scale back.''; ``Risk may be encountered if the incoming event stream is completely unpredictable and have huge spike differences in data for a considerable period of timr'' \\
Difficult to test & 4 & ``if the executed actions that will be done by the self-adopting system are not tested before, it might introduce some risks''; ``It is also difficult to do reliable performance testing in non-production environments.'' \\
Difficult to build & 4 & ``implementing and designing self-adaptive systems may initially seem to take longer time – hence the risk of not being allowed to implement it as good as it can be done''; ``Costs of building own (self-hosted) environment [...]'' \\
Other & 2 & ``life updates (no downtime)''; ``There is always a lingering concern of quis custodiet ipsos custodes - or 'who watches the watchmen'.'' \\
 
\noalign{\smallskip}\hline
\end{tabular}
\end{table*}

\begin{table*}[hbt]
\caption{Analysis of comments II -- Risks faced when engineering self-adaptive systems (Q4.4).}
\label{tab:codes_q4-4.II}
\begin{tabular}{p{3.6cm}lp{8.8cm}}
\hline\noalign{\smallskip}
Categories and codes & \# & Example quotes\\
\noalign{\smallskip}\hline\noalign{\smallskip}

\textbf{Impact on qualities} & \textbf{16} &  \\

Performance degradation & 5 & ``[...] risk of degrading the performance instead of improving it, and degrading the user experience as a result.''; ``Performance impact on the running system when applying auto-scaling (e.g. scaling down)''; ``sometimes a sequence of perfectly acceptable self-adaptive automatic actions can lead to outages worse than the root cause'' \\
Reduced availability & 4 & ``If the system did not behave properly this could result in an outage [...]''; ``Availability of the system during the auto-scaling rules being applied''\\
Safety and security threats & 4 & ``If a system is self-adaptive, how can we secure that it is safe during production (some parts can be powered for self test during assembly and we need to know it is safe)? If we use machine learning on a self-adaptive system, how do we secure safety? 
''; ``There is a risk of misconfiguration that can lead to lost nodes and applications, security exposures etc. There are also security risks involved with the base building components, such as docker images from untrusted sources [...]'' \\
Extra resource consumption & 2 & ``Risk of all resources being eaten up by a self-adaptive process.''; ``[...] it may use up too many unnecessary hardware and software resources'' \\
Reliability issues & 1 & ``Reliability issues in case of non-converging oscillations or plain wrong output due to prolonged failures in the metrics collection pipelines or simply wrong algorithms'' \\
\noalign{\smallskip}\hline\noalign{\smallskip}
\textbf{Impact on business} & \textbf{14} &  \\

Increased cost & 5 & ``Regarding autoscaling, the main issue was to fail and so increasing the infra cost of the users due to bugs in the system.''; ``Lost control over system size. This also impacted the approx. total cost agreed with the customer.'' \\
Losing trust and control & 4 & ``Trust. Because flexible manufacturing systems have some kind of autonomous behavior with tasks that have been done manually, our clients are initially very sceptial and to not trust the systems initally''; ``risk of losing (manual) control of the system for the sake of automation'' \\
Harder to understand/fix & 3 & ``the whole system becomes more complex, hence fewer people understand all details of its behaviour.''; ``More difficult troubleshooting for a self-adapting, distributed system.'' \\
Not useful & 2 & ``The self-adaptive system might not perform better than the baseline when dealing with dynamic shapes, as the cost model might not be generic enough to predict the performance.'' \\
 
\noalign{\smallskip}\hline
\end{tabular}
\end{table*}
\normalsize