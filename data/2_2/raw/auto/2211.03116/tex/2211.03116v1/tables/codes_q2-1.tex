\small
\begin{table*}[b!]
\caption{Analysis of comments I -- Explain a concrete self-adaptive system you worked with (Q2.1)}
\label{tab:codes_q2-1.I}
\begin{tabular}{p{3cm}lp{9.6cm}}
\hline\noalign{\smallskip}
Categories and codes & \# & Example quotes\\
\noalign{\smallskip}\hline\noalign{\smallskip}
\textbf{Subject of adaptation} & \textbf{99} &  \\

System & 28 & ``Our company develops safety critical systems for railway. Systems architecture is often with redundancy - e.g. 2 out of 3 system, where is automatic reconfiguration implemented. Purpose is high safety and availability''; ``A flexible manufacturing system ... the system and the individual station within the system can "sense" what kind of work piece it has in front of itself and what it or another machine should do with it in the next step.'' 
\\

Module & 22 & ``Environment compensation system for capacitive touch interface. Such system is influenced by envirenmental change (for example temperature)''; ``We manage the memory usage of the process. Once memory usage over a limit (i.e. 90\%), we start throttling the workload.''\\

Platform layer & 13 & ``Monitoring the memory/CPU/disk consumption of our servers and suggesting measures to fix it through human intervention.'' 
\\

Application layer & 11 & ``HotSpot JVM ... reads a program's Java bytecode, and adaptively tunes the performance of the program at runtime, adapting to runtime profiles.''\\

Cluster & 10 & ``Spark executor auto-scaling system. We built this system to automatically add or remove nodes to our Spark cluster when we have a high demand of resources from our Spark jobs.'' \\

Network & 6 & 
``"Our radios apply 'channel assessment' ... that optimizes the radio channels used during BLE communication. Our radios also apply very aggressive power management. peripherals and cores are switched off whenever possible to minimize the system's power usage."'' 
\\

Mixed & 6 & ``Enterprise-cloud environment consisting of dozens of different (micro) services providing functionality to 3rd parties as well as internal employees - data management, authentication and authorization, business process automation, as well as internal development process support (build servers, logging, etc.).''
\\

CI/CD pipeline & 3 & ``Sacling up and down our infrastructure (CI/CD) chain to build and integrate the truck software.'' \\
\noalign{\smallskip}\hline
\end{tabular}
\end{table*}

\begin{table*}[h!]
\caption{Analysis of comments II -- Explain a concrete self-adaptive system you worked with (Q2.1)}
\label{tab:codes_q2-1.II}
\begin{tabular}{p{3cm}lp{9.6cm}}
\hline\noalign{\smallskip}
Categories and codes & \# & Example quotes\\
\noalign{\smallskip}\hline\noalign{\smallskip}
\textbf{Type of adaptation} & \textbf{99} &  \\

Auto-scaling & 33 & ``Automated horizontal scaling of AWS EC2 instances for medical data processing systems''; ``autoscale a cluster based on the resource usage of the nodes of the cluster.''\\
Auto-tuning & 28 & ``A mink feeding robot, that can adjust the food amount according to a set of feeding rules and the food left over from last feeding.'' \\
%The robot can also calibrate the food amount delivered, if the expected amount does not macth the delivered amount.'' \\
Monitor/Analysis & 22 & ``We configured AWS alarms to monitor performance of our systems in case we get more than few number of HTTP 400/500 errors''; ``Monitoring the memory/CPU/disk consumption of our servers and suggesting measures to fix it through human intervention.'' \\
Automated reconfiguration & 11 & ``Continuos integration system - Other \& starts building \& testing a new version as soon as it detects code changes  Build alignment - Creates a new release whenever a subsystem builds successfully.''\\
Other & 5 & ``Our mobile robots scan their environments using laser scanners and other sensors and plan their behavior accordingly.'' ``self healing automotive systems''\\\noalign{\smallskip}\hline\noalign{\smallskip}
\textbf{Trigger for adaptation} & \textbf{99} &  \\
System properties & 27 & ``Auto-scaling functionality of an Azure Service Fabric cluster running a transformation load for processing AGV statistical and playback data.''; ``Realtime focused data streaming protocol ... must take care to avoid exhausting the network resources and thus incurring packet loss and latency spikes, which are very noticeable in games.''\\

Environment properties & 18 & ``An IoT system running in Kubernetes and used to monitor water leaking for household insurance.''; ``A flexible manufacturing system ... can "sense" what kind of work piece it has in front of itself and what it or another machine should do with it in the next step.'' \\

System load & 14 & ``Kubernetes, for handling load intensive periods for scaling up, and self recover from crashes.''; ``Autoscaling of SaaS applications in function of load on AWS and Azure clouds.'' \\

Events & 12 & ``We use kubernetes which provides notification callbacks on any event such as host/pod not available, based on these events we auto mark the node was inactive and do not use those nodes for further write or read operations''; ``Auto Scaling an EMR cluster in AWS based on incoming event data''\\

User actions & 7 & ``[adapt] cache warm up strategy based on user interactions''; ``scammers ... To decide the users that are most likely to be a scammer, the system tracks the past performance of models responsible for flagging potential scammers.''  \\
\noalign{\smallskip}\hline
\end{tabular}
\end{table*}
\normalsize

\begin{comment}


\begin{enumerate}
    \item \textbf{Why include students:} We identified reasons for why students might be used as subjects in empirical software architecture research.

\begin{enumerate}
    \item \underline{Help achieve study goal:} Twenty-nine comments pointed out that students may be suitable to achieve the goal of a study. For example, students may be representative subjects for studies about novice developers. As one respondent wrote, \emph{``They are valid as long as they are considered indicative of outcomes that can be obtained with novice programmers.''}, echoed in another comment: \emph{``They are somehow novice, and this is valuable for the study''}.  Similarly, another one wrote \emph{``Unless the empirical study is about students...''}. One respondent gave a concrete example: \emph{``counter example: teaching techniques for architecture design decision-making.''}. Finally, one respondent wrote that {``Students are developers/architects too. Additionally, if we don't use them, we will never be able to pursue/evaluate certain kinds of research.''} and that \emph{``questions that focus on industry practice can be at least partially answered with students.''}
    
    \item \underline{Represent next generation of practitioners:} Five comments pointed out that students are the next generation of practitioners. As one respondent wrote, \emph{``Especially Master Students are very close to professionals''}. Another one added that \emph{``As they have similar profile and are the future industry workers, results are relevant.''} Another one stated that \emph{``many students are practitioners themselves - in this field we have a tendency to assume age as a predictor of quality and reliability, and this is just plain bullshit.''}. In a similar way, one respondent stated that \emph{``One of the nonsense stereotypes. - Many of our students work in companies on the side. - So, they are semi-professionals. Also, the week after graduation they start to develop, in this week they will not become different developers.''}

\item \underline{Only available subjects:} Four comments highlighted a more pragmatic view and pointed out that students might be the only subjects researchers have access to. As stated by one respondent, \emph{``Studies with students are better than no studies at all, thus provide valuable results and inputs to validate with "harder-to-get" professionals.''}

\item \underline{Less biased:} Three comments emphasized that students, in contrast to practitioners, might be less biased. One respondent wrote that \emph{``I believe there is a lot to be said for the unbiased approach of people that have studied the subject but are not yet clogged down by routine.''} Another one emphasized that \emph{``results of praticioners which refuse to learn and apply new methods (maybe due to their age) can be useless too.''}
    
\end{enumerate}

\item \textbf{Why not include students:} We also identified reasons for why students should not be used in empirical software architecture studies.

\begin{enumerate}
    \item \underline{Lack experience:} Nine comments highlighted that students lack the experience required for software architecture studies to be meaningful. One respondent wrote that \emph{``Might be of limited value in software architecture since software architecture typically requires years of experience. Students often lack experience with software architecture since student projects are usually small.''} Another one stated that \emph{``Students have limited skills and experience, so results on field could be very different.''}. Finally, one respondent wrote that {``The field of architecture is rather abstract and requires experience. It is hard to have students act as architects (or customers of architects).''}
    
    \item \underline{Not representative:} Nine comments highlighted that students are not representative to practitioners, therefore, results are not applicable to software architecting in industry. As one respondent wrote, \emph{``The main concern of using students only in an empirical study is that they may not represent the real practitioners for whom the results may be of interest.''} while another stated that \emph{``This is about architecture, right?  It's even less likely that students are like professionals in architecture knowledge than in coding.''}. Finally, one respondent wrote that \emph{``In the software architecture domain in particular students aren't representative of the target audience.''}
    
    
\end{enumerate}

\end{enumerate}

\end{comment}