\small
\begin{table*}[hbt]
\caption{Analysis of comments I -- Opportunities for self-adaptation that are not exploited yet (Q4.9)}
\label{tab:codes_q4-9.I}
\begin{tabular}{p{3cm}lp{9.4cm}}
\hline\noalign{\smallskip}
Categories and codes & \# & Example quotes\\
\noalign{\smallskip}\hline\noalign{\smallskip}

\textbf{System activities} & \textbf{72} &  \\

 Autonomous operation & 37 & ``E.g manufacturing production line with visual inspection operators who remove defects, ... the production line can further be adapted based on the defect rate/type''; ``Self adaption could have a lot of benefits in building automation systems, like smart heating and lighting systems that takes peoples habits into consideration.''; ``making the system adaptive to adjust and act instantly based on the data without waiting would be beneficial and efficient.'' \\
 
Data management \& machine learning & 26 & ``Methods to automatically handle changes in the machine learning models and to efficiently deploy them to the edge. There is still lots of manual fine tuning that delays a timely new release.''; ``The query optimizer of database (i.e. MySQL) could utilize self-adaptation technic.'' \\
 
 Autoscaling & 9 & ``The "managed service", which is a stateful service/ data store, is provisioned for the peak capacity, which means resources are idle most of the time. If we can build reliable and efficient system that can automatically scale stateful services based on the demand, we can reduce the cost.''; ``Our microservices do not dynamically scale'' \\
\noalign{\smallskip}\hline\noalign{\smallskip}
\textbf{System properties} & \textbf{47} &  \\

Quality improvement & 26 & ``Based on the alarm certain counter actions could be initiated in order to deal with the faulty behaviour and reach a stable system state.''; ``Congestion prognosis''; ``fault tolerance''; ``Power consumption''; ``resource optimization''; ``There are many opportunities to split up [current monolithic systems] and then make them scalable such that outages are more contained. E.g. screens on trains.'' \\

Security improvement & 10 & ``Security of e.g., mobile devices that adapts based on locally identified threats as well as knowledge of risks in the environment.''; ``Automating changes in Security levels based on threat levels''; ``Detecting in-vehicle threats, detecting a system being compromised''; ``react to attack patterns'' \\

Cost effectiveness & 8 & ``IT cost reduction (e.g. software asset mgmt)''; ``The question really is: How do you do these things on the cheap (with non Silicon Valley billion dollar funding) and in contexts where mistakes might be extremely critical?'' \\

\noalign{\smallskip}\hline
\end{tabular}
\end{table*}

\begin{table*}[hbt]
\caption{Analysis of comments II -- Opportunities for self-adaptation that are not exploited yet (Q4.9).}
\label{tab:codes_q4-9.II}
\begin{tabular}{p{3.1cm}lp{9.4cm}}
\hline\noalign{\smallskip}
Categories and codes & \# & Example quotes\\
\noalign{\smallskip}\hline\noalign{\smallskip}

\textbf{Engineering activities} & \textbf{21} &  \\

Maintenance \& reuse & 15 & ``self-adapting CI/CD infrastructure based on demand''; ``Preventive maintenance''; ``Carriers are eager to get rid of human factors to improve operation and maintenance capabilities and network quality. Therefore the ICT field pays much attention to self-adaption systems.''; ``Software provisioning and automatic updates'' \\

Patterns \& libraries & 6 & ``Developing a comprehensive library of algorithms on top of the industrial monitoring systems which can be applied to analysis portion of the chain in order to drive correct self-adaptation actions would benefit the self-adaptation adoption.''; ``Cross-cloud self-adaptation''; ``patterns to provide solutions to common problems'' \\
\noalign{\smallskip}\hline\noalign{\smallskip}
\textbf{Human involvement} & \textbf{7} &  \\

Personalization & 4 & `` it would be interesting to adapt the player experience itself based on the player, mostly to better challenge them''; ``Healtcare decision making systems witch are changing outcomes and advices basd on patient status.'' \\

Human-machine interaction & 3 & ``I consider that the biggest opportunities are found within the Human Machine Interaction or Building Machine Interaction. There will be a future in which talking to a device that can modify the environment (e.g. a robot but not a phone) will be as natural as talking to a person, or seeing a machine interacting with another machine (e.g. robot taking the elevator)'' \\

\noalign{\smallskip}\hline
\end{tabular}
\end{table*}
\normalsize