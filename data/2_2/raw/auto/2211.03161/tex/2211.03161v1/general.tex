\section{General 4d orthogonal range reporting and higher dimensions}
\label{sec:general}
Our result for five-sided 4d queries can be extended to answer general 4d queries using standard techniques. For example, we can support queries $\prod_{i=1}^{2}[a_i,b_i]\times \prod_{i=3}^4(-\infty,b_i]$ by constructing the range tree $T_2$ on the second coordinates of all points. We keep the data structure for five-sided queries in every internal node of $T_2$. For a given query
$q=\prod_{i=1}^{2}[a_i,b_i]\times \prod_{i=3}^4(-\infty,b_i]$, we find the leaf that contains the predecessor of $b_2$, the leaf that contains the successor of $a_2$,  and their lowest common ancestor $w$.  Let $w_l$ and $w_r$ be the left and right children of $w$.  Let $q_l= [a_1,b_1]\times [a_2,+\infty)\times \prod_{i=3}^4(-\infty,b_i]$ and $q_r=[a_1,b_1]\times (-\infty,b_2]\times \prod_{i=3}^4(-\infty,b_i]$. Then $q\cap P=q\cap P(w)= (q_l\cap P(w_l))\cup (q_r\cap P(w_r))$. Since both $q_l$ and $q_r$ are five-sided queries, we can answer .  In the same way, we can support general queries $q=\prod_{i=1}^4[a_i,b_i]$ without increasing the query time and  using $O(n\log^4n)$ space.

We can also obtain a data structure that supports general  queries in $d>4$ dimensions using range trees. The space usage and query time grow by $O(\log n)$ factor with every dimension $d>4$.
\begin{theorem}
  \label{theor:multidim}
  There exists a data structure that answers $d$-dimensional orthogonal range reporting queries on a pointset $P$ in $O(\log^{d-3}n\log\log n + k)$ time and uses $O(n\log^dn)$ space, where $k$ is the number of points in the query range and $n$ is the number of points in $P$. 
\end{theorem}

%%% Local Variables:
%%% mode: latex
%%% TeX-master: "4d-dom-pm"
%%% End:
