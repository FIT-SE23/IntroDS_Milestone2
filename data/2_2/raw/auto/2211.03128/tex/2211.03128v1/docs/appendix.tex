\newpage

% \onecolumn
\ifarxiv
\appendix
\section{Appendix}
\else
\section{Supporting Information}
\fi

In Table \ref{tab:folktable_columns}, we describe the columns used for each Folktables task found in our ACS experiments. 

\begin{table}[ht]
\centering
\caption{We detail below the Folktables columns we use for each task.}
\label{tab:folktable_columns}
\begin{tabular}{l | l}
    \toprule
    Task & Columns \\
    \midrule
    \multirow{8}{*}{employment}
    & AGEP (age), SCHL (educational attainment) \\
    & MAR (marital status), RELP (relationship) \\
    & DIS (disability recode), ESP (employment status of parents) \\
    & CIT (citizenship status), MIG (mobility status - lived here 1 year ago) \\
    & MIL (military status), ANC (ancestry recode) \\
    & NATIVITY (nativity), DEAR (hearing difficulty) \\
    & DEYE (vision difficulty), DREM (cognitive difficulty) \\
    & SEX (sex), RAC1P (recoded detailed race code) \\
    \midrule
    \multirow{9}{*}{coverage}
    & AGEP (age), SCHL (educational attainment) \\
    & MAR (marital status), SEX (sex) \\
    & DIS (disability recode), ESP (employment status of parents) \\
    & CIT (citizenship status), MIG (mobility status - lived here 1 year ago) \\
    & MIL (military status), ANC (ancestry recode) \\
    & NATIVITY (nativity), DEAR (hearing difficulty) \\
    & DEYE (vision difficulty), DREM (cognitive difficulty) \\
    & PINCP (Total person’s income), ESR (employment status recode) \\
    & FER (gave birth within the past 12 months), RAC1P (recoded detailed race code) \\
    \midrule
    \multirow{11}{*}{mobility}
    & AGEP (age), SCHL (educational attainment) \\
    & MAR (marital status), SEX (sex) \\
    & DIS (disability recode), ESP (employment status of parents) \\
    & CIT (citizenship status), MIL (military status) \\
    & ANC (ancestry recode), NATIVITY (nativity) \\
    & RELP (relationship), DEAR (hearing difficulty) \\
    & DEYE (vision difficulty), DREM (cognitive difficulty) \\
    & RAC1P (recoded detailed race code), GCL (grandparents living with grandchildren) \\
    & COW (class of worker), ESR (employment status recode) \\
    & WKHP (usual hours worked per week past 12 months), JWMNP (Travel time to work) \\
    & PINCP (Total person’s income) \\
    \bottomrule
\end{tabular}
\end{table}

In Section \ref{sec:results}, we visualized the reconstruction rates on Census and ACS datasets. However, to more easily communicate our findings, we presented results that were averaged across various geographic entities in Figures \ref{fig-main:census_avg_tract}, \ref{fig-main:census_avg_tract_init}, and \ref{fig-main:census_avg_block} for Census experiments and Figure \ref{fig-main:folktables_avg} for ACS experiments. Here, we now present more granular results. In particular, in each subplot of Figures \ref{fig-appx:census_tract_all1}, \ref{fig-appx:census_tract_all2}, \ref{fig-appx:census_tract_ib_all1}, and \ref{fig-appx:census_tract_ib_all2}, we present results for a single tract that was randomly chosen from each state, where the latter two figures (\ref{fig-appx:census_tract_ib_all1}, \ref{fig-appx:census_tract_ib_all2}) present tract-level experiments without the BLOCK feature. In addition, we plot results of \rapattack~initialized to the baseline distribution in Figures \ref{fig-appx:census_tract_all_init1} and \ref{fig-appx:census_tract_all_init2}. Like in Section \ref{sec:results}, we again average results at the block-level in Figures \ref{fig-appx:census_block_all1} and \ref{fig-appx:census_block_all2}, but now we aggregate our randomly selected blocks at the state-level. Finally, in \ref{fig-appx:folktables_all}, we present results for each of the 15 state-task combinations derived from the ACS Folktables package.

\begin{figure*}[hb]
    \centering
    \includegraphics[width=\textwidth]{images/census/tract/all1.png}
    \caption{We plot \matchrate~of \rapattack~and our baselines on a \emph{tract}-level reconstruction with the BLOCK attribute \emph{included}. Subplots are labeled and ordered alphabetically by the state name.}
    \label{fig-appx:census_tract_all1}
\end{figure*}

\begin{figure*}
    \centering
    \includegraphics[width=\textwidth]{images/census/tract/all2.png}
    \caption{We plot \matchrate~of \rapattack~and our baselines on a \emph{tract}-level reconstruction with the BLOCK attribute \emph{included}. Subplots are labeled and ordered alphabetically by the state name.}
    \label{fig-appx:census_tract_all2}
\end{figure*}

\begin{figure*}
    \centering
    \includegraphics[width=\textwidth]{images/census/tract/all_init1.png}
    \caption{We plot \matchrate~of \rapattack~and our tract-level baseline on a \emph{tract}-level reconstruction with the BLOCK attribute \emph{included}. \rapattack~is initialized to the baseline. Subplots are labeled and ordered alphabetically by the state name.}
    \label{fig-appx:census_tract_all_init1}
\end{figure*}

\begin{figure*}
    \centering
    \includegraphics[width=\textwidth]{images/census/tract/all_init2.png}
    \caption{We plot \matchrate~of \rapattack~and our tract-level baseline on a \emph{tract}-level reconstruction with the BLOCK attribute \emph{included}. \rapattack~is initialized to the baseline. Subplots are labeled and ordered alphabetically by the state name.}
    \label{fig-appx:census_tract_all_init2}
\end{figure*}

\begin{figure*}
    \centering
    \includegraphics[width=\textwidth]{images/census/tract_ib/all1.png}
    \caption{We plot \matchrate~of \rapattack~and our baselines on a \emph{tract}-level reconstruction with the BLOCK attribute \emph{excluded}. Subplots are labeled and ordered alphabetically by the state name.}
    \label{fig-appx:census_tract_ib_all1}
\end{figure*}

\begin{figure*}
    \centering
    \includegraphics[width=\textwidth]{images/census/tract_ib/all2.png}
    \caption{We plot \matchrate~of \rapattack~and our baselines on a \emph{tract}-level reconstruction with the BLOCK attribute \emph{excluded}. Subplots are labeled and ordered alphabetically by the state name.}
    \label{fig-appx:census_tract_ib_all2}
\end{figure*}

\begin{figure*}
    \centering
    \includegraphics[width=\textwidth]{images/census/block/all1.png}
    \caption{We plot \matchrate~of \rapattack~and our baselines on a \emph{block}-level reconstruction, where in each subplot, we average results across all blocks selected for the corresponding state. Subplots are labeled and ordered alphabetically by the state name.}
    \label{fig-appx:census_block_all1}
\end{figure*}

\begin{figure*}
    \centering
    \includegraphics[width=\textwidth]{images/census/block/all2.png}
    \caption{We plot \matchrate~of \rapattack~and our baselines on a \emph{block}-level reconstruction, where in each subplot, we average results across all blocks selected for the corresponding state. Subplots are labeled and ordered alphabetically by the state name.}
    \label{fig-appx:census_block_all2}
\end{figure*}

\begin{figure*}
    \centering
    \includegraphics[width=\textwidth]{images/folktables/all.png}
    \caption{We plot \matchrate~of \rapattack~and our baselines on a \emph{block}-level reconstruction for each state-task combination from the ACS Folktables package. We show results of \rapattack using both 2 and 3-way marginal queries.}
    \label{fig-appx:folktables_all}
\end{figure*}


% \begin{figure*}[htbp]
%     \centering
%     \includegraphics[width=\textwidth]{images/census/averaged.png}
%     \caption{TODO}
%     \label{fig-main:census_avg}
% \end{figure*}

% \begin{figure*}[htbp]
%     \centering
%     \includegraphics[width=\textwidth]{images/census/averaged_init.png}
%     \caption{TODO, also still waiting on block results for RAP init}
%     \label{fig-main:census_avg_init}
% \end{figure*}