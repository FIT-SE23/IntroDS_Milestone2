\documentclass{article}

% if you need to pass options to natbib, use, e.g.:
%     \PassOptionsToPackage{numbers, compress}{natbib}
% before loading tackling_climate_workshop_style

% ready for submission
%\usepackage{tackling_climate_workshop_style}

% to compile a preprint version, e.g., for submission to arXiv, add add the
% [preprint] option:
% \usepackage[preprint]{tackling_climate_workshop_style}

% to compile a camera-ready version, add the [final] option, e.g.:
\usepackage[final]{tackling_climate_workshop_style}

% to avoid loading the natbib package, add option nonatbib:
% \usepackage[nonatbib]{tackling_climate_workshop_style}

\usepackage[utf8]{inputenc} % allow utf-8 input
\usepackage[T1]{fontenc}    % use 8-bit T1 fonts
\usepackage{hyperref}       % hyperlinks
\usepackage{url}            % simple URL typesetting
\usepackage{booktabs}       % professional-quality tables
\usepackage{amsfonts}       % blackboard math symbols
\usepackage{nicefrac}       % compact symbols for 1/2, etc.
\usepackage{microtype}      % microtypography
\usepackage[pdftex]{graphicx}
\usepackage{subfig}
\usepackage{floatrow}
\newfloatcommand{capbtabbox}{table}[][\FBwidth]

\title{Personalizing Sustainable Agriculture with Causal Machine Learning}

% The \author macro works with any number of authors. There are two commands
% used to separate the names and addresses of multiple authors: \And and \AND.
%
% Using \And between authors leaves it to LaTeX to determine where to break the
% lines. Using \AND forces a line break at that point. So, if LaTeX puts 3 of 4
% authors names on the first line, and the last on the second line, try using
% \AND instead of \And before the third author name.

\author{%
  Georgios Giannarakis \\
  BEYOND Centre, IAASARS\\
  National Observatory of Athens\\
  \texttt{giannarakis@noa.gr} \\
  \And
  Vasileios Sitokonstantinou \\
  BEYOND Centre, IAASARS \\
  National Observatory of Athens \\
  \texttt{vsito@noa.gr} \\
  \AND
  Roxanne Suzette Lorilla \\
  BEYOND Centre, IAASARS \\
  National Observatory of Athens \\
  \texttt{rslorilla@noa.gr} \\
  \And
  Charalampos Kontoes \\
  BEYOND Centre, IAASARS \\
  National Observatory of Athens \\
  \texttt{kontoes@noa.gr}
}

\begin{document}

\maketitle

\begin{abstract}
To fight climate change and accommodate the increasing population, global crop production has to be strengthened. To achieve the ``sustainable intensification'' of agriculture, transforming it from carbon emitter to carbon sink is a priority, and understanding the environmental impact of agricultural management practices is a fundamental prerequisite to that. At the same time, the global agricultural landscape is deeply heterogeneous, with differences in climate, soil, and land use inducing variations in how agricultural systems respond to farmer actions. The ``personalization'' of sustainable agriculture with the provision of locally adapted management advice is thus a necessary condition for the efficient uplift of green metrics, and an integral development in imminent policies. Here, we formulate personalized sustainable agriculture as a Conditional Average Treatment Effect estimation task and use Causal Machine Learning for tackling it. Leveraging climate data, land use information and employing Double Machine Learning, we estimate the heterogeneous effect of sustainable practices on the field-level Soil Organic Carbon content in Lithuania. We thus provide a data-driven perspective for targeting sustainable practices and effectively expanding the global carbon sink.

\end{abstract}

\section{Introduction}

Climate change poses a multidimensional, complex challenge to humankind. In this context, agriculture faces a unique situation. On the one hand, production must keep rising up to meet ever-increasing demands [1], a task that is significantly complicated by inflation, crises, and extreme, unpredictable events [2,3,4]. On the other hand, it must do so in a sustainable way. Owing to decades of scientific research, the consequences of environmentally oblivious agricultural practices are by now well understood [5,6]. A unique opportunity also surfaced; since agricultural land comprises 38\% of the global land surface [7], agriculture offers a clear path towards mitigating climate change, by maximizing the carbon that crops naturally sequester, thus creating a substantial carbon sink [8].

As a result, insights on the impacts of sustainable agricultural practices, especially with regards to their effect on Soil Organic Carbon (SOC) are invaluable, not only for sustainability purposes, but also for the reliability, fairness, and transparency of modern agricultural carbon markets. At the same time, the global agricultural landscape contains significant diversity in terms of climate, soil, and land use. Such diversity of agricultural ecosystems frequently results in heterogeneous responses to interventions carried out by farmers; as an example, the effect of fertilizers on soil organic matter may differ depending on soil pH [9]. This reality is now recognized and reflected on high-stakes policy making. In an effort to ensure a sustainable future, the latest iteration of the Common Agricultural Policy (CAP) of the European Union explicitly provides greater flexibility for adapting its measures to local conditions [10]. 
% (This should be contrasted to the horizontal application of measures that happened in past policy iterations, and were criticized about their effectiveness CITE). 
From the perspective of artificial intelligence, such endeavor entails a personalization problem, where the end goal is to provide the farmer with targeted advice, depending on the particular farming context, on best practices to use in order to uplift sustainable metrics, such as SOC, and e.g., reimburse them accordingly. In this context, causal Machine Learning (ML) can efficiently leverage big Earth Observation (EO) data and other agricultural information in order to arrive at quantitative insights on the impacts of sustainable agriculture. Due to the large geographical coverage attained by EO, such methods can estimate effect heterogeneity even at the field-level, hence providing personalized and truly actionable advice towards sustainability goals. In this paper, we apply causal ML to understand the heterogeneous response of SOC content to a sustainable agriculture program and showcase a methodology for its personalization.

\section{Related Work}

% Historically, the environmental impact of agriculture has been studied extensively [11]. 
Over the past decade, ML methods have made significant contributions in a variety of important tasks, including yield prediction, or pest, weed, and disease detection [11]. While such work has been fundamental in advancing precision agriculture, it is primarily based on predictive, correlation-based models, and is hence inadequate for assessing causal relations. Recently, works that aim to explicitly model cause and effect have been published. Enabled by large-scale EO data, among other sources, researchers have used new causal ML methods to infer the impact of conservation tillage on yield [12], or assess the suitability of agricultural land for applying crop rotation and diversification [13]. Causal forests [14] have been employed [15] for evaluating the environmental effectiveness of various agri-environment schemes [16]. However, results were obtained either on a lower spatial resolution, hindering the field-level personalization of advice [12, 13], or for a single year only [15], impeding the capture of slow, long-term environmental dynamics, such as those characterising soil [17]. Here, we utilize field-level SOC predictions and other geospatial data to propose a multi-year, fully personalized framework for assessing the effect of sustainable practices on SOC content and guide decision-making. We report preliminary results for a case study in Lithuania.

\section{Data and Methods}

\textbf{Dataset.} We use data from the Lithuanian Land Parcel Identification System (LPIS) that contains national-scale information on the crop type that was cultivated on each geo-referenced field as declared by the farmers in their CAP subsidy applications. Data also contain a binary variable that indicates if a field was enrolled in an eco-friendly program and employed sustainable agricultural management practices for that year. Data were collected annually, spanning a 5-year period from 2017 to 2021. In order to control for climate-induced confounding we incorporate 9 climate variables retrieved from the ERA5 database, namely air temperature, soil temperature and moisture, northward and eastward wind speed, evapotranspiration, leaf area index, runoff and precipitation [18]. 
To complete the data, we include field-level predictions of SOC content [19, 20], on which we want to examine the impact of sustainable practices. Due to the dependence of the SOC content prediction method on the absence of cloud cover, we currently only use a single SOC content value for the entire 5-year period. While increasing the temporal resolution of SOC content would provide more granular insights, we note that due to the slowly changing soil properties [17], the single value we have may potentially be decently representative of the reference period and subsets of it.

\textbf{Methods.} We model the problem of assessing the causal effect of eco-friendly practices on SOC content in different farming contexts as a Conditional Average Treatment Effect (CATE) estimation task. Using the Potential Outcomes framework, let $Y(T)$ denote the value (potential outcome) of a random variable $Y$ if we were to treat a unit with a binary treatment $T \in \{0, 1\}$. Given a vector $X$ describing the units, the CATE is the function $\theta(x) = \mathbb{E}[Y(1) - Y(0) | X = x]$. In our case, $T$ is the binary variable that indicates whether a field was enrolled in the eco-friendly program, and $Y(1), Y(0)$ are, respectively, the SOC content the field would have had if it was enrolled in the eco-friendly program ($T=1$) or not ($T=0$). The feature vector $X$ captures important field characteristics that drive effect heterogeneity. Here, to segment the analysis per crop type, $X$ consists of 3 binary variables, corresponding to the 3 most prominent crops found in the dataset during the reference period (Table \ref{tab:results}). We then use Double Machine Learning (DML) [21] to learn $\theta(x)$ from the dataset, and estimate the effect of sustainable practices on SOC content as a function of $x$.

\section{Results}

In a preliminary analysis, we designated as ``treated'' the fields that employed sustainable practices for the years 2021 and 2020, and as ``control'' the rest. We thus temporally aggregated all ERA5 climate variables over these two years, ending up with a single (average) value for each variable per field. We applied a standard scaler to all numerical features, and used a Random Forest Regressor and Classifier for the internal predictive DML tasks. To retain full inference on causal effect estimates, we used the Linear DML version, as implemented in [22].

\begin{figure}[!h]
\begin{floatrow}
\ffigbox{%
\hspace{-0cm}
  \includegraphics[scale=0.174]{small_map.pdf}
}{%
\hspace{-0cm}
  \caption{Color coded CATE point estimates for selected fields. Dark fields have larger (non-significant) estimates and are expected to better accommodate sustainable practices in terms of their impact on SOC content.}%
  \label{fig:results}
}
\capbtabbox{%
    \begin{tabular}{lccc}
        \toprule
        \multicolumn{4}{c}{Conditional Average Treatment Effects}     \\
        \toprule
        Crops & PP & SP & WW   \\
        \midrule
        Point Estimate   & 0.06  & -0.08 & -0.09 \\
        Standard Error   & 0.14  & 0.17  & 0.28  \\
        Z-score      & 0.41  & -0.44 & -0.32  \\
        P-value          & 0.69  & 0.66  & 0.75   \\
        95\% CI (Lower)  & -0.22 & -0.42 & -0.63  \\
        95\% CI (Higher) & 0.34  & 0.26  & 0.45   \\
        \bottomrule
    \end{tabular}
}{%
  \caption{CATE point estimates and associated uncertainty. PP = Perennial Pastures, SP = Simple Pastures, WW = Winter Wheat. SOC content is reported as a percentage of topsoil (0-10 cm).}%
  \label{tab:results}
}
\end{floatrow}
\end{figure}

The most prominent crops found in the dataset consisted of fields with perennial pastures (PP) in both 2021 and 2020, followed by simple pastures (SP) and winter wheat (WW), also in both years. Table \ref{tab:results} contains the corresponding CATE estimates, quantifying the causal effect of sustainable practices on SOC content per crop, and Figure \ref{fig:results} visualizes them for a few indicative fields. None of the effects are currently statistically significant at the 95\% confidence level, a sensible result given the short time window the treatment refers to, its discrepancy with the 5-year period the SOC content values were derived from, and the slow rate of change of soil properties. As we incorporate more years and data in the analysis we will be able to explore interesting scientific questions, such as assessing whether the eco-friendly program was most efficient in perennial pastures and corroborate the growing evidence on the environmental benefits of perennialization [23].

\section{Conclusions and Future Work}

Our work can assist policy makers in efficiently implementing sustainable agricultural management and simultaneously enable the smart allocation of farmer subsidies, a task that has proven to be challenging [24]. Climate, soil, and other land use information may also be included in the data in order to adapt the estimates on any local condition experts deem important. In that way, the idea presented in this paper can develop into a useful tool for sustainable agriculture, in line with modern policy needs and encompassing the entire agricultural chain from the policy maker to the farmer.
To proceed, the temporal range of the treatment should be extended to align with the 5-year SOC content period, and domain knowledge should be incorporated in the model through structural approaches [25]. The inclusion of more data on field-level agricultural practices is also important for reducing the bias of estimates. Finally, given the general absence of ground truth, estimates should be tested for robustness by employing modern sensitivity analyses [26].

% \section{About the workshop}

%  Many in the ML community wish to take action on climate change, but are unsure of the pathways through which they can have the most impact. This workshop highlights work that demonstrates that, while no silver bullet, ML can be an invaluable tool in reducing greenhouse gas emissions and in helping society adapt to the effects of climate change. Climate change is a complex problem, for which action takes many forms - from theoretical advances to deployment of new technology. Many of these actions represent high-impact opportunities for real-world change, and are simultaneously interesting academic research problems.
 
%  This workshop is part of a series (NeurIPS 2022, ICML 2021, NeurIPS 2020, ICLR 2020, NeurIPS 2019, and ICML 2019). For this iteration of the workshop, the keynote talks and panel discussions will be particularly focused on ML as an enabling technology for empowering decision-makers in tackling climate change, though submitted works may be on any topic of relevance at the intersection of climate change and machine learning.
 
% \section{Submission of papers}

% Electronic submissions are required, via this submission website:
% \begin{center}
%   \url{https://cmt3.research.microsoft.com/CCAINeurIPS2022/}
% \end{center}

% Please read the instructions below carefully and follow them faithfully.

% \section{Tracks}

% There are three tracks for submissions: \textbf{Papers, Proposals, and Tutorial}s, each described in detail below.  Submissions are limited to \textbf{4 pages for the Papers track}, and \textbf{3 pages for the Proposals track}. Tutorials do not have a page limit. References do not count towards this total. Supplementary appendices are allowed but will be read at the discretion of the reviewers. All submissions must explain why the proposed work has (or could have) positive impacts regarding climate change.

% \subsubsection{Papers}

% \textit{Work that is in progress, published, and/or deployed.}

% Submissions for the Papers track should describe projects relevant to climate change that involve machine learning. These may include (but are not limited to) academic research; deployed results from startups, industry, public institutions, etc.; and climate-relevant datasets.

% Submissions should provide experimental or theoretical validation of the method presented, as well as specifying what gap the method fills. Authors should clearly illustrate a pathway to climate impact, i.e., identify the way in which this work fits into broader efforts to address climate change. Algorithms need not be novel from a machine learning perspective if they are applied in a novel setting. Details of methodology need not be revealed if they are proprietary, though transparency is highly encouraged.

% Submissions creating novel datasets are welcomed. Datasets should be designed to permit machine learning research (e.g. formatted with clear benchmarks for evaluation). In this case, baseline experimental results on the dataset are preferred, but not required.

% Submissions are limited to 4 pages. References do not count toward this total. Submissions are due Sept. 18, 2022.

% \subsubsection{Proposals}

% \textit{Early-stage work and detailed descriptions of ideas for future work.}

% Submissions for the Proposals track should describe detailed ideas for how machine learning can be used to solve climate-relevant problems. While less constrained than the Papers track, Proposals will be subject to a very high standard of review. Ideas should be justified as extensively as possible, including motivation for why the problem being solved is important in tackling climate change, discussion of why current methods are inadequate, explanation of the proposed method, and discussion of the pathway to climate impact. Preliminary results are optional.

% Submissions are limited to 3 pages. References do not count toward this total. Submissions are due Sept. 18, 2022.

% \subsubsection{Tutorials}
% \textit{Interactive notebooks for insightful step-by-step walkthroughs.}

% Submissions for the Tutorials track should introduce or demonstrate the use of ML methods and tools such as libraries, packages, services, datasets, or frameworks to address a problem related to climate change. Tutorial proposals (due Aug 18) should take the form of an abstract and should include a clear and concise description of users' expected learning outcomes from the tutorial. Accepted submissions (to be notified by Aug 25) will be given about 3 weeks for the initial tutorial development (midterm deadline on Sep 18), after which tutorial creators will collaborate with the Tutorials Team, who will review the tutorials periodically and provide iterative feedback, while the creators continue to develop and improve their work over the course of another 8 weeks. Midterm tutorial submissions (due Sep 18) and Final tutorial submissions (due Nov 3) should be in the form of executable notebooks (e.g. Jupyter, Colab). Submissions will be reviewed based on their potential impact and overall usability by the climate and AI research community.


% \subsection{Style}

% Papers must be prepared according to the instructions presented here.   Submissions are limited to \textbf{4 pages for the Papers track}, and \textbf{3 pages for the Proposals track}. Tutorials do not have a page limit. Papers that exceed these page limitswill not be reviewed, or in any other way considered for presentation at the workshop.


% Authors are required to use the workshop style files (modified from the NeurIPS style files), obtainable on the website as indicated below. Please make sure you use the current files and not previous versions. Tweaking the style files may be grounds for
% rejection.

% \subsection{Retrieval of style files}

% The style files for this workshop are available on
% the World Wide Web at
% \begin{center}
%   \url{http://climatechange.ai/files/TCCML_NeurIPS_2022_Style_File.zip} %TODO UPDATE THIS
% \end{center}
% The file \verb+tackling_climate_workshop.pdf+ contains these instructions and illustrates the
% various formatting requirements your paper must satisfy.

% The file \verb+tackling_climate_workshop.tex+ may be used as a “shell” for writing your paper. Alternatively, the file \verb+tackling_climate_workshop.docx+ can be used as well. Replace the author, title, abstract, and text of the paper with your own. Please remember that at submission time your document should be anonymized and the only accepted format is PDF.

% The only supported style file for \LaTeX{} is \verb+tackling_climate_workshop_style.sty+,
% rewritten for \LaTeXe{}.  \textbf{Previous style files for \LaTeX{} 2.09, or NeurIPS conference style file, are not accepted.}

% The \LaTeX{} style file contains three optional arguments: \verb+final+, which
% creates a camera-ready copy, \verb+preprint+, which creates a preprint for
% submission to, e.g., arXiv, and \verb+nonatbib+, which will not load the
% \verb+natbib+ package for you in case of package clash.

% \paragraph{Preprint option}
% If you wish to post a preprint of your work online, e.g., on arXiv, using the workshop style, please use the \verb+preprint+ option. This will create a
% nonanonymized version of your work with the text ``Preprint. Work in progress.''
% in the footer. This version may be distributed as you see fit. Please \textbf{do
%   not} use the \verb+final+ option, which should \textbf{only} be used for
% papers accepted to the workshop. Note that all workshops are non-archival; submission does not preclude future publication.

% At submission time, please omit the \verb+final+ and \verb+preprint+
% options. This will anonymize your submission and add line numbers to aid
% review. Please do \emph{not} refer to these line numbers in your paper as they
% will be removed during generation of camera-ready copies.

% The file \verb+tackling_climate_workshop.tex+ may be used as a ``shell'' for writing your
% paper. Replace the author, title, abstract, and text of
% the paper with your own.

% The formatting instructions contained in these style files are summarized in
% Sections \ref{gen_inst}, \ref{headings}, and \ref{others} below.

% \section{General formatting instructions}
% \label{gen_inst}

% The text must be confined within a rectangle 5.5~inches (33~picas) wide and
% 9~inches (54~picas) long. The left margin is 1.5~inch (9~picas).  Use 10~point
% type with a vertical spacing (leading) of 11~points.  Times New Roman is the
% preferred typeface throughout, and will be selected for you by default.
% Paragraphs are separated by \nicefrac{1}{2}~line space (5.5 points), with no
% indentation.

% The paper title should be 17~point, initial caps/lower case, bold, centered
% between two horizontal rules. The top rule should be 4~points thick and the
% bottom rule should be 1~point thick. Allow \nicefrac{1}{4}~inch space above and
% below the title to rules. All pages should start at 1~inch (6~picas) from the
% top of the page.

% The version of the paper submitted for review should have "Anonymous Author(s)" as the author of the paper. 
% For the final version, authors' names are set in boldface, and each name is
% centered above the corresponding address. The lead author's name is to be listed
% first (left-most), and the co-authors' names (if different address) are set to
% follow. If there is only one co-author, list both author and co-author side by
% side.

% Please pay special attention to the instructions in Section \ref{others}
% regarding figures, tables, acknowledgments, and references.

% \section{Headings: first level}
% \label{headings}

% All headings should be lower case (except for first word and proper nouns),
% flush left, and bold.

% First level headings are in point size 12. One line space before the first level heading and \nicefrac{1}{2} line space after the first level heading.

% \subsection{Headings: second level}

% Second level headings are in point size 10. One line space before the second level heading and \nicefrac{1}{2} line space after the second level heading.

% \subsubsection{Headings: third level}

% Third level headings are in point size 10. One line space before the third level heading and \nicefrac{1}{2} line space after the third level heading. 

% \paragraph{Paragraphs}

% In \LaTeX{} there is also a \verb+\paragraph+ command available, which sets the heading in bold, flush left, and inline with the text, with the heading followed by 1\,em of space. If using this style option in a \verb+docx+ file, please follow these instructions accordingly.

% \section{Citations, figures, tables, references}
% \label{others}

% These instructions apply to everyone, regardless of the formatter being used.

% \subsection{Citations within the text}

% Citations within the text should be numbered consecutively.  The corresponding number is to appear enclosed in square brackets, such as [1] or [2]-[5].  The corresponding references are to be listed in the same order at the end of the paper, in the \textbf{References} section. (Note: the standard
% \textsc{Bib\TeX} style \texttt{unsrt} produces this.) As to the format of the references themselves, any standard reference style is acceptable, as long as it is used consistently.


% As submission is double blind, refer to your own published work in the third
% person. That is, use ``In the previous work of Jones et al.\ [4],'' not ``In our
% previous work [4].'' If you cite your other papers that are not widely available
% (e.g., a journal paper under review), use anonymous author names in the
% citation, e.g., an author of the form ``A.\ Anonymous.''

% When using the \LaTeX{} template, the \verb+natbib+ package will be loaded for you by default.  Citations may be author/year or numeric, as long as you maintain internal consistency.

% For \LaTeX{} use, note that the documentation for \verb+natbib+ may be found at
% \begin{center}
%   \url{http://mirrors.ctan.org/macros/latex/contrib/natbib/natnotes.pdf}
% \end{center}
% Of note is the command \verb+\citet+, which produces citations appropriate for
% use in inline text.  For example,
% \begin{verbatim}
%    \citet{hasselmo} investigated\dots
% \end{verbatim}
% produces
% \begin{quote}
%   Hasselmo, et al.\ (1995) investigated\dots
% \end{quote}

% If you wish to load the \verb+natbib+ package with options, you may add the
% following before loading the \verb+neurips_2020+ package:
% \begin{verbatim}
%    \PassOptionsToPackage{options}{natbib}
% \end{verbatim}

% If \verb+natbib+ clashes with another package you load, you can add the optional
% argument \verb+nonatbib+ when loading the style file:
% \begin{verbatim}
%    \usepackage[nonatbib]{tackling_climate_workshop_style}
% \end{verbatim}

% \subsection{Footnotes}

% Footnotes should be used sparingly.  If you do require a footnote, indicate
% footnotes with a number\footnote{Sample of the first footnote.} in the
% text. Place the footnotes at the bottom of the page on which they appear.
% Precede the footnote with a horizontal rule of 2~inches (12~picas).

% Note that footnotes are properly typeset \emph{after} punctuation
% marks.\footnote{As in this example.}

% \subsection{Figures}

% \begin{figure}
%   \centering
%   \fbox{\rule[-.5cm]{0cm}{4cm} \rule[-.5cm]{4cm}{0cm}}
%   \caption{Sample figure caption.}
% \end{figure}

% All artwork must be neat, clean, and legible. Lines should be dark enough for
% purposes of reproduction. The figure number and caption always appear after the
% figure. Place one line space before the figure caption and one line space after
% the figure. The figure caption should be lower case (except for first word and
% proper nouns); figures are numbered consecutively.

% Make sure the figure caption does not get separated from the figure. Leave sufficient space to avoid splitting the figure and figure caption.

% You may use color figures.  However, it is best for the figure captions and the
% paper body to be legible if the paper is printed in either black/white or in
% color, and that colormaps consider accessibility to the visually impaired (e.g. red/green colorblindness).

% \subsection{Tables}

% All tables must be centered, neat, clean and legible.  The table number and
% title always appear before the table.  See Table~\ref{sample-table}.

% Place one line space before the table title, one line space after the
% table title, and one line space after the table. The table title must
% be lower case (except for first word and proper nouns); tables are
% numbered consecutively.

% Note that publication-quality tables \emph{do not contain vertical rules.} We
% strongly suggest the use of the \verb+booktabs+ package, which allows for
% typesetting high-quality, professional tables:
% \begin{center}
%   \url{https://www.ctan.org/pkg/booktabs}
% \end{center}
% This package was used to typeset Table~\ref{sample-table}.

% \begin{table}
%   \caption{Sample table title}
%   \label{sample-table}
%   \centering
%   \begin{tabular}{lll}
%     \toprule
%     \multicolumn{2}{c}{Part}                   \\
%     \cmidrule(r){1-2}
%     Name     & Description     & Size ($\mu$m) \\
%     \midrule
%     Dendrite & Input terminal  & $\sim$100     \\
%     Axon     & Output terminal & $\sim$10      \\
%     Soma     & Cell body       & up to $10^6$  \\
%     \bottomrule
%   \end{tabular}
% \end{table}

% \section{Final instructions}

% Do not change any aspects of the formatting parameters in the style files.  In
% particular, do not modify the width or length of the rectangle the text should
% fit into, and do not change font sizes (except perhaps in the
% \textbf{References} section; see below). Please note that pages should be
% numbered.

% \section{Preparing PDF files}

% Please prepare submission files with paper size ``US Letter,'' and not, for
% example, ``A4.''

% Fonts were the main cause of problems in the past years. Your PDF file must only
% contain Type 1 or Embedded TrueType fonts. Here are a few instructions to
% achieve this.

% \begin{itemize}

% \item For MSWord users: from the print menu, click the PDF drop-down box, and select "Save as PDF…".

% \item For \LaTeX{} users: you should directly generate PDF files using \verb+pdflatex+.

% \item You can check which fonts a PDF files uses.  In Acrobat Reader, select the
%   menu Files$>$Document Properties$>$Fonts and select Show All Fonts. You can
%   also use the program \verb+pdffonts+ which comes with \verb+xpdf+ and is
%   available out-of-the-box on most Linux machines.

% \item The IEEE has recommendations for generating PDF files whose fonts are also
%   acceptable for NeurIPS. Please see
%   \url{http://www.emfield.org/icuwb2010/downloads/IEEE-PDF-SpecV32.pdf}

% \item \verb+xfig+ "patterned" shapes are implemented with bitmap fonts.  Use
%   "solid" shapes instead.

% \item The \verb+\bbold+ package almost always uses bitmap fonts.  You should use
%   the equivalent AMS Fonts:
% \begin{verbatim}
%    \usepackage{amsfonts}
% \end{verbatim}
% followed by, e.g., \verb+\mathbb{R}+, \verb+\mathbb{N}+, or \verb+\mathbb{C}+
% for $\mathbb{R}$, $\mathbb{N}$ or $\mathbb{C}$.  You can also use the following
% workaround for reals, natural and complex:
% \begin{verbatim}
%    \newcommand{\RR}{I\!\!R} %real numbers
%    \newcommand{\Nat}{I\!\!N} %natural numbers
%    \newcommand{\CC}{I\!\!\!\!C} %complex numbers
% \end{verbatim}
% Note that \verb+amsfonts+ is automatically loaded by the \verb+amssymb+ package.

% \end{itemize}

% If your file contains type 3 fonts or non embedded TrueType fonts, we will ask
% you to fix it.

% \subsection{Margins in \LaTeX{}}

% With \LaTeX{} most of the margin problems come from figures positioned by hand using
% \verb+\special+ or other commands. We suggest using the command
% \verb+\includegraphics+ from the \verb+graphicx+ package. Always specify the
% figure width as a multiple of the line width as in the example below:
% \begin{verbatim}
%    \usepackage[pdftex]{graphicx} ...
%    \includegraphics[width=0.8\linewidth]{myfile.pdf}
% \end{verbatim}
% See Section 4.4 in the graphics bundle documentation
% (\url{http://mirrors.ctan.org/macros/latex/required/graphics/grfguide.pdf})

% A number of width problems arise when \LaTeX{} cannot properly hyphenate a
% line. Please give LaTeX hyphenation hints using the \verb+\-+ command when
% necessary.

\begin{ack}
This work has been supported by the EIFFEL project (EU Horizon 2020 - GA No. 101003518) and the MICROSERVICES project (2019-2020 BiodivERsA joint call, under the BiodivClim ERA-Net COFUND programme, and with the GSRI, Greece - No. T12ERA5-00075).
\end{ack}

\section*{References}

% References follow the acknowledgments. Use unnumbered first-level heading for
% the references. Any choice of citation style is acceptable as long as you are
% consistent. It is permissible to reduce the font size to \verb+small+ (9 point)
% when listing the references.
% {\bf Note that the Reference section does not count towards the pages of content that are allowed; 4 pages for Papers track and 3 pages for Proposals track.}
% \medskip

\small

[1] Tilman, D., Balzer, C., Hill, J., \& Befort, B. L. (2011). Global food demand and the sustainable intensification of agriculture. Proceedings of the national academy of sciences, 108(50), 20260-20264.

[2] Sivakumar, M. V., Motha, R. P., \& Das, H. P. (Eds.). (2005). Natural disasters and extreme events in agriculture: impacts and mitigation. Berlin, Heidelberg: Springer Berlin Heidelberg.

[3] Gray, R. S. (2020). Agriculture, transportation, and the COVID‐19 crisis. Canadian Journal of Agricultural Economics/Revue canadienne d'agroeconomie, 68(2), 239-243.

[4] Brunelle, T., Dumas, P., Souty, F., Dorin, B., \& Nadaud, F. (2015). Evaluating the impact of rising fertilizer prices on crop yields. Agricultural economics, 46(5), 653-666.

[5] Tsiafouli, M. A., Thébault, E., Sgardelis, ... \& Hedlund, K. (2015). Intensive agriculture reduces soil biodiversity across Europe. Global change biology, 21(2), 973-985.

[6] Robertson, G. P., Paul, E. A., \& Harwood, R. R. (2000). Greenhouse gases in intensive agriculture: contributions of individual gases to the radiative forcing of the atmosphere. Science, 289(5486), 1922-1925.

[7] Food and Agriculture Organization of the United Nations (FAO), Land use in agriculture by the numbers, URL https://www.fao.org/sustainability/news/detail/en/c/1274219/

[8] Toensmeier, E. (2016). The carbon farming solution: A global toolkit of perennial crops and regenerative agriculture practices for climate change mitigation and food security. Chelsea Green Publishing.

[9] Heinze, S., Raupp, J., \& Joergensen, R. G. (2010). Effects of fertilizer and spatial heterogeneity in soil pH on microbial biomass indices in a long-term field trial of organic agriculture. Plant and Soil, 328(1), 203-215.

[10] European Commission, The new common agricultural policy: 2023-27, URL https://agriculture.ec.europa.eu/common-agricultural-policy/cap-overview/new-cap-2023-27\_en

% [11] Gomiero, T., Pimentel, D., \& Paoletti, M. G. (2011). Environmental impact of different agricultural management practices: conventional vs. organic agriculture. Critical reviews in plant sciences, 30(1-2), 95-124.

[11] Liakos, K. G., Busato, P., Moshou, D., Pearson, S., \& Bochtis, D. (2018). Machine learning in agriculture: A review. Sensors, 18(8), 2674.

[12] Deines, J. M., Wang, S., \& Lobell, D. B. (2019). Satellites reveal a small positive yield effect from conservation tillage across the US Corn Belt. Environmental Research Letters, 14(12), 124038.

[13] Giannarakis, G., Sitokonstantinou, V., Lorilla, R. S., \& Kontoes, C. (2022). Towards assessing agricultural land suitability with causal machine learning. In Proceedings of the IEEE/CVF Conference on Computer Vision and Pattern Recognition (pp. 1442-1452).

[14] Wager, S., \& Athey, S. (2018). Estimation and inference of heterogeneous treatment effects using random forests. Journal of the American Statistical Association, 113(523), 1228-1242.

[15] Stetter, C., Mennig, P., \& Sauer, J. (2022). Using Machine Learning to Identify Heterogeneous Impacts of Agri-Environment Schemes in the EU: A Case Study. European Review of Agricultural Economics.

[16] Batáry, P., Dicks, L. V., Kleijn, D., \& Sutherland, W. J. (2015). The role of agri‐environment schemes in conservation and environmental management. Conservation Biology, 29(4), 1006-1016.

[17] Smith, P. (2004). How long before a change in soil organic carbon can be detected?. Global Change Biology, 10(11), 1878-1883.

[18] Hersbach, H., Bell, B., Berrisford, P., ... \& Thépaut, J. N. (2020). The ERA5 global reanalysis. Quarterly Journal of the Royal Meteorological Society, 146(730), 1999-2049.

[19] Tziolas, N., Tsakiridis, N., Ogen, Y., Kalopesa, E., Ben-Dor, E., Theocharis, J., \& Zalidis, G. (2020). An integrated methodology using open soil spectral libraries and Earth Observation data for soil organic carbon estimations in support of soil-related SDGs. Remote Sensing of Environment, 244, 111793.

[20] Karyotis, K., Samarinas, N., Tsakiridis, N., Tziolas, N., Kokkas, S., Tsiakos, V., Bliziotis, D., Kalopesa, E., \& Zalidis, G. (2022). Synergy between Spaceborne optical Imagery and in-situ Soil Spectroscopy for Organic Carbon estimations over exposed lands. 26th MARS Conference.

[21] Chernozhukov, V., Chetverikov, D., Demirer, M., Duflo, E., Hansen, C., Newey, W., \& Robins, J. (2018). Double/debiased machine learning for treatment and structural parameters.

[22] Battocchi, K., Dillon, E., Hei, M., Lewis, G., Oka, P., Oprescu, M., \& Syrgkanis, V. (2019) EconML: A Python Package for ML-Based Heterogeneous Treatment Effects Estimation.

[23] Mosier, S., Cordova, S., \& Robertson, G. P. (2021). Restoring soil fertility on degraded lands to meet food, fuel, and climate security needs via perennialization. Frontiers in Sustainable Food Systems, 356.

[24] European Court of Auditors. (2017). Greening: a more complex income support scheme, not yet environmentally effective. Special Report, (21).

[25] Pearl, J. (2009). Causality. Cambridge university press.

[26] Cinelli, C., \& Hazlett, C. (2020). Making sense of sensitivity: Extending omitted variable bias. Journal of the Royal Statistical Society: Series B (Statistical Methodology), 82(1), 39-67.



% [1] Alexander, J.A.\ \& Mozer, M.C.\ (1995) Template-based algorithms for
% connectionist rule extraction. In G.\ Tesauro, D.S.\ Touretzky and T.K.\ Leen
% (eds.), {\it Advances in Neural Information Processing Systems 7},
% pp.\ 609--616. Cambridge, MA: MIT Press.

% [2] Bower, J.M.\ \& Beeman, D.\ (1995) {\it The Book of GENESIS: Exploring
%   Realistic Neural Models with the GEneral NEural SImulation System.}  New York:
% TELOS/Springer--Verlag.

% [3] Hasselmo, M.E., Schnell, E.\ \& Barkai, E.\ (1995) Dynamics of learning and
% recall at excitatory recurrent synapses and cholinergic modulation in rat
% hippocampal region CA3. {\it Journal of Neuroscience} {\bf 15}(7):5249-5262.

\end{document}