
\section{Discussion}

Below we discuss the design implications that arise from this work, the possible negative consequences of eliciting high arousal, \x{the challenges of automatic visualization design evaluation, as well as limitations and future work.}

\subsection{Design Implications}


\ul{First, we contribute new knowledge for understanding how data encodings influence affective arousal.} An interesting finding of this work is that in data visualization design, graphical elements that are related to data encodings showed significant capability of eliciting affective arousal. 
% Severn features such as the visualization type and layout were significantly related to arousal and they together explained more than half of the variance in the dependent variable, and this number was even higher than the variance explained by color-based features. 
For the visualization community, this finding is significant in two main ways.
First, we have found new evidence that the appearance of data visualization does affect user experience. In the past, Moere~\etal~\cite{moere2012evaluating} found that when encoded with identical data, different chart types (\eg treemaps and sunburst diagrams) triggered different subjective feelings and preferences.
Borkin~\etal~\cite{borkin2013makes} found that people remembered certain chart types better than others. In other words, choosing data encodings is not only about how to present data but also about manipulating user experience. 
Second, this work also supplements the existing knowledge about user-centered visualization design from the perspective of affective arousal. For example, in the study by Borkin~\etal~\cite{borkin2013makes}, the most remembered visualization types were \textit{grid \& matrix} and \textit{tree \& network}.
% the least remembered types were \textit{point}, \textit{bar}, and \textit{line}, and \textit{table} fell in the middle. 
\x{However, in our study, visualizations that use a grid layout (\eg L1 in Fig.~\ref{fig:teaser}) were less arousing than others.}
% such as \textit{bar}, \textit{line}, and \textit{point} were still not strong in evoking affective arousal, \textit{table}, \textit{diagram}, and \textit{grid \& matrix} became the least arousing visualization types. 
% The participants generally felt bored with visualization forms full of textual elements.
This suggests that viewing visualization design from the perspective of affective arousal may be different from viewing it from perspectives such as memorability and aesthetics (although these perspectives are all user-centered). In other words, focusing on different aspects of user experience may also lead to different focuses of design.

\x{\ul{Second, we found colorfulness outweighs other color-based features in terms of moderating affective arousal in the context of data visualization design.} 
Although the affective power of color has long been verified in various domains, in different research contexts, the findings can be different.
For example, red was identified as the most arousing color in classic psychological experiments~\cite{jacobs1974effects}, but some researchers also found evidence that green can be more arousing than red~\cite{valdez1994effects}.
Our work, however, found that colorfulness is more related to arousal than specific color hues in visualization design. According to our study, a reason may be that people's color perception would be influenced by their attempts to decode data when viewing a data visualization.
For example, some participants reported low arousal to designs with a large amount of red because the color had hindered reading (\eg "\textit{it is too bright and dazzling}"). Similarly, some participants stated that the use of analogous color hues (\eg red and orange) had made the graphical elements on the charts indistinguishable, thus lowering arousal. On the other hand, colorfulness can be both viscerally stimulating and useful in presenting data, thus making it the most arousing color-based feature compatible with data visualization design.
}
% Echoing previous studies, we found color also has a significant relationship with affective arousal in the context of data visualization design. 
% In our regression model, four out of 13 significant design features were color-based and they explained a considerate part of the variance in the dependent variable. In addition, color was also the most mentioned design feature in our crowdsourcing study. In other words, except for having a statistically significant effect on affective arousal, color was also the design feature that people were most consciously aware of.
% Besides, in our model, features such as the overall colorfulness, the use of black and green showed high importance. These findings highly resonate with a classic color research~\cite{valdez1994effects}, which found dark colors are more arousing than light colors, and green and purple are more arousing than yellow, red, and blue. However, given that controversies about the affective traits of color hues still exist, we hope future research can continue exploring this issue.
% \x{For example, a prevailing view in color research is that high-chroma colors are more arousing than low-chroma colors, and dark colors are more arousing than light colors~\cite{lichtle2007effect,bartram2017affective}.
% The role of color hues is more controversial. For example, red was identified as the most arousing color in a psychological experiment~\cite{jacobs1974effects} but in some studies, green was found more arousing than red~\cite{valdez1994effects}. }
% However, we did not find significant difference in triggering affective arousal among color hues (\eg the percentage of red is not significant in our model, nor is the percentage of blue). In our study, the use of varied and distinct colors seems to outweigh the use of a specific color hue in eliciting affective arousal.

\ul{Third, data visualizations that belonged to the \textit{Infographic} category elicited stronger affective arousal than others.}
Infographics are often seen as a visual form that pays more attention to data communication and appealing to viewers subjectively than traditional visualizations~\cite{lan2021smile,lankow2012infographics}, and previous work~\cite{wang2018infonice,byrne2015acquired} has often focused on studying the rich embellishment techniques in infographics such as pictograms, illustrations, and visual metaphors. 
However, another interesting finding from this work is that even after we removed all embellished designs for our corpus, the images in the \textit{Infographic} category were still significantly more arousing than others. This finding indicates that the specificity of infographic design may not only lie in embellishment, but also in how it deals with color, graphics, and information. As shown by an empirical study~\cite{bigelow2014reflections}, when creating infographics, designers often spend a lot of time choosing encoding approaches and polishing the appearance of the visualization iteratively, because the designers not only hope to present data clearly, but also to create something new and unique. Such needs meet the \textit{unusualness} strategy we mentioned above and may help improve the arousing level of a design.
% In our corpus, we indeed saw more unusual designs in the \textit{Infographic} category than in the \textit{News Media} and \textit{Government} categories, and that 


\x{\ul{Fourth, the qualitative and quantitative analyses collectively imply the strategies for manipulating arousal.} 
In the crowdsourcing study, we identified several high-level strategies for eliciting affective arousal by coding user comments qualitatively (Fig.~\ref{fig:themes}). After constructing models to predict arousal, we found that the quantitative results suggested by the models resonate more or less with the qualitative codes.
For example, our models showed that affective arousal is significantly higher when a visualization does not use layouts such as Cartesian coordinates, and this resonates well with the most mentioned high-level strategy by the participants, namely \textit{unusualness} (\eg "\textit{The design factors that helped increase the arousal was those that looked interesting and new, something I've never seen before.}"). Another two strategies mentioned by the participants (\textit{boldness} and \textit{aesthetics}) were partly supported by the significance of colorfulness, as previous research has found colorfulness is linked with the perception of energization and extraversion~\cite{pazda2019color}, as well as an indicator of aesthetics~\cite{harrison2015infographic,reinecke2013predicting}.}
However, there seems to be controversy over \textit{simplicity} and \textit{complexity} (we can see the co-existence of these two strategies in Fig.~\ref{fig:themes}). Our models, however, ended up leaning towards the side of \textit{complexity} because affective arousal showed a significantly positive relationship with the \textit{number of different visual channels}. \x{However, by referring to the raw comments written by the participants, we found that \textit{simplicity} and \textit{complexity} may not necessarily contradict each other. 
For example, when reporting that simple designs had increased their affective arousal, some participants were in fact emphasizing the understandability of the designs (\eg ``\textit{This diagram looks like a really simple way to display data, even to a layperson}''). On the other hand, when reporting that complex designs were more arousing, some participants were emphasizing the richness of visuals (\eg "\textit{different graphs and charts - when there was more information on the chart to concentrate on the arousal was higher}"). In other words, for designers, it may be beneficial to include an appropriate level of visual complexity while ensuring that the data visualization is understandable.}
% Future research can further explore this issue.
% thus helping us further understand \textit{why} a certain design feature is arousing.
% I felt this looked complex and intriguing which made me feel interested and a little excited

% As we have discussed in Section 4.2.3, some participants saw simplicity as arousing while some thought complex design is more arousing. Our regression model, however, seems to lean towards the complexity side because affective arousal has shown a significantly positive relationship with the number of different visual channels. Meanwhile, the model did not show that a simple data visualization (\eg visualization with less mark types or channels) is significantly related to high affective arousal. We also hope future research can further explore this issue.
% Third, there seem to be controversy over the roles of "simplicity" and "complexity" (we can also see the co-existence of these two terms in Fig.~\ref{fig:themes}).  

% However, it is also worth noticing that simplicity and complexity may not always be mutually exclusive. A data visualization design can be easy reading, but it can also be rich in visuals.


\subsection{The Negative Side of High Arousal}

As we have discussed in Section 4.1.1.3, in our exploratory study, high affective arousal did not always lead to high pleasure or high preference. In some cases, the participants did not like or enjoy data visualization designs that elicited high arousal, or they thought high arousal was harmful. By analyzing such comments, we identified two main reasons:

\ul{First, high arousal may lead to physical discomfort.}
In our crowdsourcing study, several participants reported high arousal but thought the corresponding designs were too chaotic and over-stimulating (\eg ``\textit{It is a very arousing visual in terms of color and design, however looking at it makes your eyes go unwell so does not give great pleasure when looking at the visual.}'', "\textit{Unpleasant over-stimulating colours, chaotic graphic that is difficult to decipher and is unpleasant to look at.}"). One participant commented that "\textit{Too much red is making me feel nervous.}" 
% Colourful and interesting but so cluttered. Looking that it makes me feel anxious, not knowing where to direct my attention first.
These findings resonate with prior psychological studies that intense affects may bring about side effects~\cite{hyman1990ethics} or even evolve into tension and anxiety~\cite{malmo1957anxiety}. Therefore, for designers, designing an arousing data visualization is not equal to simply adding up all stimulating design features. On the contrary, it is necessary to find a balance between arousing affects and preventing discomfort.

\ul{Second, high arousal may be caused by inappropriate design.}
In some cases, the participants were affectively aroused because the designs looked confusing or ugly to them. For example, one participant said that a design "\textit{looks really cool but I don't know how easy it would be to interpret data using this}". Similarly, another participant wrote: "\textit{A very striking chart that looks like it makes sense only to the person who created it.}" We also saw comments that reported high arousal accompanied by negative affects such as anger and irritation (\eg "\textit{I don't like the orange. It makes me angry.}", "\textit{The red line is irritating.}"). Under such circumstances, high arousal is not beneficial.
%and designers should avoid triggering such bad consequences.

\subsection{Towards Automatic Design Evaluation}


Understanding what design features contribute to affective arousal has paved a path toward automatically evaluating the arousing level of data visualizations or automatically generating arousing data visualizations. For example, we can envision a feedback system that can intelligently assess the arousing level of a data visualization design by scoring its design from multiple arousal-related aspects and identifying its strengths and weaknesses, thus guiding designers to improve their work. 
\x{To achieve such goals, extracting computable features from data visualization designs automatically becomes a pursuit. In this work, although about half of the features were calculated by computer, we noticed that there are three main challenges if we want to fully automate the process of feature extraction.}
% A designer may reach out to this system in the early design stage to get a sense of how users would react to the design affectively, or use this system to evaluate a design before publishing it.

\x{\ul{First, not all the design-related concepts can be easily transformed into computable features.} For example, so far, there is no recognized algorithm for evaluating the data-ink ratio of a data visualization automatically. Besides, high-level concepts such as \textit{unusualness} are difficult to be measured computationally, as they are usually feelings co-shaped by multiple design elements and are moderated by the participants' own knowledge, values, or preferences. How to take advantage of the valuable signals that arise from such expressions is a tricky problem.}

\x{\ul{Second, parsing the composition of a data visualization from an image can be difficult.} Understanding how a data visualization is composed of data, graphical elements, and texts requires utilizing techniques such as object detection, recognition, and classification. When the input is pixel-based images, the accuracy of the outcomes may not be satisfying.
For example, when calculating the whitespace, the algorithm sometimes misunderstood which color is the background color. When detecting texts from the images, the algorithm frequently made mistakes. As a result, we had to adjust these results manually. 
% such as improving the accuracy of distilling the visual encodings from a data visualization and the ability to classify the visualization as a concrete type or subtype. 
% In this work, to guarantee the accuracy, we such information was coded manually. Tasks such detecting the layout and the whitespace of a data visualization accurately also 
Therefore, to achieve automatic design evaluation, we still need to incorporate more state-of-the-art techniques from computer vision and develop more models and algorithms customized for data visualization design to improve the accuracy of design feature extraction.
}
% Second, some arousal-related concepts may be difficult to be translated into computable variables. For example, in our crowdsourcing study, we coded user comments and found that some high-level expressions related to affective arousal, such as unusualness, simplicity, aesthetics, and boldness. While these expressions have valuable signals, they were more or less ambiguous when being operationized technically.
% For example, although the data encodings of Fig.~\ref{fig:error} (d) are common (it uses the size and color channels to encode data in a Cartesian space), the whole visualization gives out a flourishing feeling, co-shaped by color, overlapping bubbles, and visual density. For example, a participant wrote, “\textit{Oh, this is so pretty! Color use and style are so nice, almost like a 'paint splatter' effect. It makes me feel warm and fuzzy.}” 
% One participant said that Fig.~\ref{fig:error} (e) "\textit{reminds me of what they use in films for nuclear rocket trajectory. interesting.}" For Fig.~\ref{fig:error} (f), some participants mentioned that they were aroused by the modern design style (\eg ``\textit{This graph looks modern and interesting the black background works well with the colors. The line shapes and style are exciting.}'').
% several participants mentioned that they felt it is organic and dynamic (\eg "\textit{Like the organic vibe.}", "\textit{Very interesting to look at whilst representing some kind of flow dynamics or physical process.}").
% However, "pretty", "interesting", and ""modern" are still difficult to be precisely captured by current design features.

\x{\ul{Third, how to assess the quality of a design remains a challenge.} Currently, we can use variables to describe the existence of a design feature or to what extent it is used. However, we are not able to judge whether the feature is used "appropriately". For example, in our study, some participants reported low arousal to visualizations that were badly designed (\eg “\textit{Not much immediate clarity what the visualisation is about hence my interest is low and disinterested.}”).
In other words, except for "quantifying" the design, the "quality" of a design also counts; except for "describing" the design, "assessing" the design also counts. This has posed another challenge for future research.}
% \x{For example, a prevailing view in color research is that high-chroma colors are more arousing than low-chroma colors, and dark colors are more arousing than light colors~\cite{lichtle2007effect,bartram2017affective}.
% The role of color hues is more controversial. For example, red was identified as the most arousing color in a psychological experiment~\cite{jacobs1974effects} but in some studies, green was found more arousing than red~\cite{valdez1994effects}. }
% However, we did not find a significant difference in triggering affective arousal among color hues (\eg the percentage of red is not significant in our model, nor is the percentage of blue). In our study, the use of varied and distinct colors seems to outweigh the use of a specific color hue in eliciting affective arousal.
% For example, our model gave Fig.~\ref{fig:error} (a) a medium score according to features such as the visualization type and colorfulness. However, 

%\x{To sum up, we hope this study can inspire more future work on deriving an actionable pipeline of computing the arousal level of data visualization design automatically.}


\subsection{Limitations and Future Work}

\x{This work is exploratory in nature so the findings of this work were mainly limited to correlations rather than causal effects and were constrained by the inherent limitations of exploratory studies.}
First, although we have carefully constructed a corpus that contains diverse data visualization designs, the corpus is still by no means exhaustive.
%and representative of all the data visualization designs in the real world. Therefore, the findings from this work are constrained by this limited corpus. 
Second, affective arousal is a subjective feeling that is difficult to quantify. To mitigate this problem, we excluded the influence of semantics, sampled diverse participants from a large population pool through crowdsourcing, and normalized each participant's ratings when constructing the models. However, noise caused by subjectivity may have inevitably entered and lowered the accuracy of prediction. 
% more work can be done to understand the interaction between design and semantics. In this work, since the effect of semantics has been excluded, the design features mainly influenced affects on a visceral level. However, if the semantics are included, affects may form through sense-making. For example, in the context of entertainment, red signifies enthusiasm, but in the context of crime, red usually signifies blood and horror~\cite{elliot2012color}. Also, there is a finding that the affective incongruence between semantics and color (\eg using positive colors for a negative topic) can make a design look inappropriate~\cite{anderson2021affective}.
% Therefore, more work is needed to clarify the role of semantics and how they interact with data visualization design.
Third, the features we extracted in this work are not exhaustive and are limited in characterizing some high-level subjective concepts such as aesthetics and unusualness. Also, \x{this work is short in revealing the fundamental impact of low-granularity design elements (\eg angles, roundness) on affective arousal. To address such issues, more controlled experiments should be done to examine the effects of a certain design condition while other variables are rigorously controlled.}
\x{We also hope a follow-up study can be done in the future to test the generalizability of this study by collecting more samples and systematically manipulating one or more of the identified features.}

% future work could include more design features. Especially, the inclusion of more design features may require integrating more technical and empirical knowledge from other domains (\eg computer vision, art theories, psychology).


We see several directions for future research.
First, 
% this work used linear regression models to predict affective arousal because the size of our corpus was not large and the main goal of this work was to interpret what design features are significantly related to affective arousal. 
\x{more alternative methods can be utilized to deepen our understanding of the design-affect relationship in data visualization.
For example, by excluding any semantics, this work did not examine the interaction between content and visualization design. Future work could fill this gap by first measuring the arousal caused by content and then analyzing how much arousal is augmented or lessened by design.}
\x{In addition, future work could examine how the incorporation of embellishment moderates the conveyance of semantics and the elicitation of affects.}
% such as the convolutional neural network, to achieve better predictive accuracy. 
\x{Second, as our study has suggested that affective arousal may be related to the complexity of data visualization, future work could further investigate how people's visualization literacy influences their perceived complexity and arousal.}
\x{Third, more work can be done to investigate another important dimension of affects, namely valence, such as how valence-related design features are different from arousal-related features and how they interact with each other.}
Last, we hope there will be more research into how we should take advantage of the benefits brought by affective arousal and how we should avoid its negative side effects.
\x{This asks for the incorporation of more empirical knowledge, such as understanding the practice of designers, to help outline clearer guidelines for what should and should not be done.}
% Third, since xxx, we only xxx user preference and did not dive deep into the consequences of xxxx. However, xxx bring about positive outcomes but can also cause bad occasionally. more work can be done to unravel various consequences of eliciting affective arousal , including how should we take advantage of the benefits of affective arousal and how should we avoid its side effects.