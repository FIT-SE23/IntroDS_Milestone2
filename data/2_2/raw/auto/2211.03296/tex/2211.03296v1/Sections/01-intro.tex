
\IEEEPARstart{D}{\x{ata}} \x{visualization is now widely used in mass communication in forms such as infographics, data-driven articles, and data videos~\cite{lan2021smile,segel2010narrative,shi2021communicating}.
Due to the traits of mass communication (\eg information overload, impatient viewers), visualization designers often need to activate viewers emotionally to grasp them immediately and motivate them to read more.
% improve content engagement metrics such as user retention rate, the page's completion rate, and memory.
In research, this design goal is closely related to a concept called affective arousal, which measures the activation of emotion on a continuous scale~\cite{russell1980circumplex}.
% which describes the activation of emotion and is an important moderator of visual communication~\cite{storbeck2008affective}.
Research from fields such as marketing and communications has provided abundant evidence that affective arousal can moderate people's willingness to spend time and money on visual media~\cite{sherman1997store,kaltcheva2006should}or lead to behaviors such as clicking and sharing~\cite{berger2011arousal}, thereby making it an important perspective of investigating visual communication~\cite{storbeck2008affective}.
% Talarico~\etal~\cite{talarico2004emotional} found that when being affectively aroused, people remember things better and report more vivid recollections.
In the visualization community, some practitioners have also stressed the importance of affective arousal. 
For example, a data visualization that looks exciting is usually more desirable in advertising as it can attract more clicks, likes, and sharing~\cite{lin2019minimal}. 
For nonprofits, designers may want a data visualization to elicit high-arousal affects in order to raise funds and call for social collaboration~\cite{chibana2015nonprofit}. }
% A senior editor from The Economist once reflected that their data visualization design was deemed too "calm" and "distant" from users and therefore they had fine-tuned their graphic design principles to make the charts more saturated and arousing.
In addition, previous work has shown that design elements such as color and imagery can effectively influence arousal levels~\cite{valdez1994effects,storbeck2008affective}, which from another perspective confirms the possibility of conducting research on the relationship between data visualization design and affective arousal.

%As suggested by existing literature // Evidence from previous empirical studies has indeed shown that affective arousal is associated with information processing~\cite{storbeck2008affective},  memory~\cite{christianson1992emotional,lang1995effects}, social behaviors~\cite{berger2011arousal}, and decision making~\cite{gorn2001arousal}. 


Although many researchers have acknowledged that affects are an indispensable part of users' experience with data~\cite{wang2019emotional,bartram2017affective,lan2021smile}, previous literature has mostly viewed affects as discrete categories rather than examining them in a continuous space.
%constituted of dimensions such as affective arousal.
For example, Kennedy and Hill~\cite{kennedy2018feeling} investigated people's affective responses to data visualizations by conducting a diary study. The affects reported by the participants were mostly categorical, such as surprise and disgust. Boy~\etal~\cite{boy2017showing} focused on examining how data visualization design influences one particular type of affect, empathy. Lan~\etal~\cite{lan2021kineticharts} proposed an affective animation design scheme for common charts to convey positive affects such as joy and tenderness in data stories. 
\x{In their recent work about serious data storytelling~\cite{lan2022negative}, the researchers focused on examining how design methods facilitate the communication of negative emotions (\eg anxiety, helplessness).
% Moreover, even when further dimensions were needed to characterize these affects, valence has been considered more than arousal. For example, Lan~\etal~\cite{lan2021smile} identified 12 commonly-observed affects associated with infographics and split them into six positive affects and six negative affects. 
The work by Bartram~\etal~\cite{bartram2017affective} is more relevant to affective arousal. They proposed a set of color palettes to help encode eight types of affect that represent typical combinations of valence and arousal (\eg positive, negative, calm, exciting).}
% They chose eight typical affects (\eg positive, negative, calm, exciting) from the circumplex space and proposed a set of color palettes that can effectively communicate these eight affects. 
However, this study still simplified affects as categories and did not show how affects change in the dimensional space continuously.
To sum up, although the above work has contributed to our understanding of how data visualization design influences affects, as yet no systematic work has been carried out to address this problem through the lens of affective arousal.

To fill this gap, this study first explores what data visualization design factors may influence the perception of affective arousal. To begin with, we prepared a corpus containing 265 data visualization images based on the MASSVIS dataset~\cite{borkin2015beyond}, \x{which provides single visualizations from balanced sources such as infographics, news media, and governments. To prevent the bias caused by semantics (because the content of the images can strongly influence affects), we blurred all the texts on these images.} Based on this corpus, we conducted a crowdsourcing study involving 184 participants and asked the participants to assess the affective arousal elicited by the designs in the corpus. We then analyzed the experimental results and derived a set of design features that the participants thought related to affective arousal by coding their comments manually. These design features were grouped into three main categories: color, graphics, and information.
\x{Next, we mapped these features to computable variables and constructed regression models to identify the most significant features and assess their importance.
By comparing the results of the models, we finally identified four design features that were cross-validated as important features correlated with affective arousal, including colorfulness, the number of different visual channels, the X and Y coordinates, and the grid layout.}
% based on the findings from the crowdsourcing study, we extracted computational design features from the images and constructed a regression model to predict affective arousal using these design features. As a result, we identified 13 design features that had a significant impact on affective arousal, such as colorfulness and the visualization type. For example, when the visualization was a table, it showed a significant negative effect on affective arousal. 
Finally, based on all the quantitative and qualitative analyses above, we discuss the implications that arise from our work, reflect on the limitations of this work, and propose future research opportunities.

% including what influences affective arousal, what are the regularities of eliciting affective arousal, and the pitfalls of eliciting affective arousal. 



% Affect is a complex psychological phenomenon that can be characterized from many perspectives. One of the most notable theories is the circumplex model of affect~\cite{russell1980circumplex}, which views affects as the combination of valence (how pleasant is the affect) and arousal (how intense is the affect). This work focuses on examining what triggers affective arousal in data visualization design considering the following two motivations.