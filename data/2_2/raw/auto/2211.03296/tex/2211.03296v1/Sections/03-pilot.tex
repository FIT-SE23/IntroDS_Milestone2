
\section{Dataset and Pilot Study}

To identify design factors that contribute to affective arousal in data visualization design, we prepared a corpus of data visualization designs and conducted a crowdsourcing study with 184 participants. The participants were asked to view the data visualizations randomly chosen from the corpus and rate the affective arousal triggered by these images.
Afterward, we analyzed the data collected from the study to summarize a set of potential arousal-related design features. The study materials can be accessed at \url{https://bit.ly/3Ntv5ck}.


\subsection{Dataset}

We aimed to use a dataset that contains diverse data visualizations and reflects the variety of data visualization design in the real world. A dataset that meets such needs is the Massachusetts (Massive) Visualization Dataset (MASSVIS)~\cite{borkin2015beyond}. MASSVIS has collected thousands of data visualizations \x{by balancing four types of sources}: infographic galleries, news media, governments, and scientific journals. 
Due to its \x{representativeness in visualization design} and public availability, MASSVIS has been used frequently in previous studies~\cite{brehmer2016timelines,kim2021towards,borkin2013makes,borkin2015beyond}.
MASSVIS is constituted of several sub-corpora with different data sizes and labels. For example, \textit{single2k} is a corpus that contains more the 2000 pieces of single data visualizations labeled with basic meta-information such as sources and titles.
% 解释为什么single?
\textit{Targets410}, another sub-corpus, contains 410 representative data visualizations selected by experts from \textit{single2k}. It contains a more suitable amount of images to be used in a user study while maintaining the design diversity of \textit{single2k}.
Besides, \textit{targets410} has been labelled with more design-related tags, such as the visualization type and data-ink ratio. Therefore, \textit{targets410} suits our study better.

Besides, according to previous literature~\cite{lan2021smile,kennedy2018feeling}, the semantics carried by a data visualization will remarkably influence people's affects. Many people would react affectively to the content or the topic of a data visualization. Besides, embellishments such as an illustration of violence and a photograph of a child can also trigger strong affects. Such affects brought by semantics will be confounded with the affects brought by design and thus hinder our understanding of the exact impact of visualization design. Therefore, to get rid of the influence of semantics, we blurred all the texts and embellishments on the images so that the participants in our studies could focus on perceiving the affects triggered by design only.


\subsection{Pilot study}

We used the blurred version of \textit{targets410} as our stimuli and conducted a pilot study to investigate whether it was appropriate to use this corpus in our study. Based on the findings from the pilot study, we refined the corpus to construct the final stimuli for our crowdsourcing study.

\subsubsection{Methodology}

We recruited 10 participants (7 females, aging from 21 to 26) for the pilot study \x{by posting an open call on social media platforms}.
Once entering the online survey, the participants were presented with an introduction to our research, such as the definition of affective arousal and how to use the instrument for evaluating affects, namely Affective Slider (AS, see Fig.~\ref{fig:slider})~\cite{betella2016affective}. AS is a modern version of Self-Assessment Manikin (SAM)\cite{bradley1994measuring}, which is the traditional instrument for measuring valence and arousal based on the circumplex model. SAM adopts a 9-point scale (\eg for affective arousal, 1 denotes not at all aroused, and 9 denotes strongly aroused). 
AS inherits the core spirit of SAM but stretches the scale to 100 by providing a movable slider whose range is 0.00 to 1.00. Such a scale can produce more continuous and precise data, which is more suitable for use in statistical modeling.
% Therefore, we finally chose AS xxx.
Note that although this work focuses on arousal, we asked the participants to rate both arousal and valence during the study in case there was any interaction between these two dimensions.
% This is because affects are seen as the combination of these two dimensions, and clarifying whether there is any interaction between arousal and pleasure is also necessary if we want to truly understand arousal in-depth.
After reading the introduction, the participants viewed 20 data visualizations randomly chosen from our corpus, one at a time. 
%  (we selected the images using a random id generator)
We also shuffled the order of the 20 stimuli to ensure the participants viewed a random sequence of images.
After viewing each of the images, the participants rated their arousal and valence using AS.
They were also asked to provide reasons for their ratings and rate the likability of the design (``Do you like this data visualization?'') using a 5-point Likert scale (1 denotes strongly dislike, 5 denotes strongly like). \x{The question about likability was included because prior literature has suggested that high arousal may not always be favorable~\cite{berlyne1960conflict}. Therefore, we followed their practice~\cite{gorn1997effects,lichtle2007effect} and evaluated whether the arousal elicited by stimuli was liked by the participants to facilitate the interpretation of arousal.}

\begin{figure}[t]
 \centering
 \includegraphics[width=\columnwidth]{Figures/AS_full.png}
 \caption{Affective Slider~\cite{betella2016affective} used in our study to rate the affective arousal (up) and valence (down) triggered by visualization design.}
 \label{fig:slider}
 \vspace{-1em}
\end{figure}

After viewing and rating all 20 images, the participants filled out a form to answer four demographic questions (\ie gender, age, education level, country) and two summary questions \x{to report arousal-related and valence-related design features respectively} (\eg ``Please recall the data visualizations you have viewed and list the design factor(s) that helped increase the arousal level.''). Last, we interviewed the participants with a set of pre-prepared questions concerning the study design, such as ``Do you think the study instructed you well? Did you feel confused about anything?'', ``Do you think 20 is an appropriate number of images to view? Did you feel tired or impatient during the study?'', ``Do you think some images are significantly different from others in terms of eliciting your affects?''. 
On average, the participants spent about 20 minutes completing this study and 10 minutes on the interview. 


\subsubsection{Results}

We collected both positive and negative feedback from the pilot study. For example, most participants agreed that 20 images is an appropriate amount for them to complete the study without losing patience. 
Also, the participants confirmed that the study task was clear and the whole process was well-guided.
% they had no trouble understanding AS. 
However, on the downside, several participants reported that they did not feel confident in their ratings until they had viewed and rated several images (\eg ``\textit{Maybe there should be several examples in the very beginning to help me decide my criterion of rating.}''). Accordingly, we refined the introduction page by including three examples. The three examples were selected by a visualization expert to demonstrate the diversity of the stimuli.

\begin{figure}[t]
 \centering
 \includegraphics[width=\columnwidth]{Figures/embellish2.jpg}
  \vspace{-2em}
 \caption{\x{Scientific visualizations (a) and the comparison of the original version and the blurred version of two embellished visualizations (b).}}
 \label{fig:embellish}
 \vspace{-1em}
\end{figure}

We also noticed that there were some problems with our stimuli.
First, many participants reported that the scientific visualizations (see Fig.~\ref{fig:embellish}) were significantly different from others, since the participants were unfamiliar with such designs and felt these images should only be viewed by professional scientists or expert users (\eg ``\textit{This is not a visualization that I will come across in my daily life, and I feel it is apparently different from the other charts I saw.}'').
\x{Second, 
% embellishments (\eg Fig.~\ref{fig:embellish} e1-e3) are inherently rich in semantics and we found that their semantics could hardly be erased by the blur effect.
we found that embellishments are inherently rich in semantics and their semantics could hardly be erased by the blur effect. For example, a participant commented that "\textit{I can guess there is a McDonald hamburger through its shape and color. I love hamburgers!}", and another participant said "\textit{there is a gym shoe next to the chart and this aroused me.}" These two embellishments were arousing not because of their design, but because of what they represent (see Fig.~\ref{fig:embellish}). In other words, the existence of embellishments is likely to trigger a different mechanism of eliciting affective arousal and confound the effects of design with the effects of semantics.}
% For example, many participants commented that they could still recognize the semantics of the embellishments and got aroused by the semantics (\eg ``\textit{I can guess there is a hamburger through its shape and color. I love hamburgers!}'', ``\textit{Obviously there is a vague person alongside the chart and this person aroused me immediately.}'')}
Third, several participants reported their discomfort with images that had a large aspect ratio. For example, one participant said, ``\textit{There was a lengthy chart, and I had to keep scrolling up and down to view it completely. This made it difficult to focus on my emotion.}''
% For example, some participants said that when they were viewing embellished visualizations, their attention was attracted to the embellishments immediately (\eg ``\textit{To be honest, I chose a high arousal level mainly because of the photograph but not the chart.}'').

Based on these findings, we removed scientific images, embellished images, and images with an aspect ratio greater than 3:1~\cite{borkin2013makes} from the \textit{targets410} corpus and kept 202 qualified images. However, a consequent problem was that the removal had changed the distribution of the image categories (\ie Infographic, Government, News Media), since many images removed because of embellishment belonged to the Infographic category. Thus, the sample sizes of the three categories became unbalanced. To deal with this problem, we searched the \textit{single2k} corpus to identify more qualified infographic images. Finally, we found 63 qualified images. After including these 63 images in our corpus (which resulted in 265 images in total), \x{the corpus contains 83 Infographic images (31\%), 82 News Media images (31\%), and 100 Government images (38\%)}. The distribution of the image categories is similar to that of the \textit{targets410} corpus, \x{thus ensuring the images are representative of the single visualization designs in the wild}. In our crowdsourcing study, we used this refined corpus as our stimuli.


