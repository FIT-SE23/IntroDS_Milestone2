
\section{Related Work}
% \x{Below we introduce prior work on data visualization for communication, affective arousal, and affective design in data visualization.} 

\subsection{\x{Data Visualization for Communication}}

% 可视化被越来越多的用于大众传播中。
\x{In recent years, a notable phenomenon in the visualization community is that data visualization has been massively applied in mass communication 
(\ie the process of creating, sending, imparting, and exchanging messages to a large number of audiences) 
through media such as infographics, data-driven articles, and data videos~\cite{shi2020calliope,lan2021smile,shi2021communicating}. 
% As defined by Defleur and Dennis~\cite{}, "mass communication is a process in which professional communicators use media to disseminate messages widely, rapidly and continuously to arouse intended meanings in large and diverse audiences in attempts to influence them in a variety of ways".
% 当的前大众传播,是基于互联网技术的传播环境。具有两个特点,一是信息的量级非常大,人们在网络上能够获得的内容尤其多。第二是,注意力非常短,3秒钟之内无法吸引人,可能就失去了被阅读的机会,因此形成了一个以用户为中心的传播环境,用户访问指数和留存指数,是最常见的web考核指标。因此,对于内容的创作者和分发者来说,必须考虑用户的immediate体验,尤其是情感。
However, on the other hand, the features of contemporary mass communication have posed new challenges for visualization design. 
As found by research in communications~\etal~\cite{lorenz2019accelerating}, the abundance of information produced by mass media today has narrowed people's attention span and made them more and more impatient.
% faced up the accelerated distribution of content and the attention span of users is becoming shorter.
To grasp and engage more viewers, designers have to consider user experience and modify their design according to how viewers might respond.
}

% 可视化也内嵌于这样的环境。不仅如此,可视化作为一种不同于传统文字的、以视觉为种的媒介形式,往往被寄予厚望。例如,信息图xxx(商业类的书)。数据新闻xxx,希望可视化能够办到xxx。这必然诉诸用户的情绪。举例:经济学人。
\x{In response to such challenges, a growing number of studies in the visualization community have adopted a user-centered perspective and examined visualization design through the lens of user experience.
% As a graphical form that differs from traditional visual representations such as plain texts, data visualization is even viewed as a more promising medium to grasp users. For example, a main advantage of infographics is often viewed as being eye-catching and is good for promoting products and business~\cite{lankow2012infographics}. In data journalism, data visualization is used to vivify the content and entice people to read the articles~\cite{}.
% 在此背景下,我们在学术圈也可以看到一些相关的研究。比如
% In academic research, some researchers have acknowledged this trend and called for more efforts into understanding visualization design beyond functionality and efficiency and paying more attention to users' subjective responses~\cite{saket2016beyond,wang2019emotional}.
For example, Saket~\etal~\cite{saket2016beyond} argued that more efforts should be put into understanding people's subjective responses to visualization design.
% Work that examines the immediate effect of data visualization design on subjectivity also exists.
Cawthon~\etal~\cite{cawthon2007effect} and Harrison~\etal~\cite{harrison2015infographic} investigated how data visualization influences aesthetic feelings. Amini~\etal~\cite{amini2018hooked} and Lan~\etal~\cite{lan2021understanding} used indicators such as likability and enjoyment to evaluate data stories. 
Borkin~\etal~\cite{borkin2013makes} evaluated a series of visualization designs by assessing how memorable they were.
However, so far, existing studies have only addressed limited aspects of user experience. Many important concepts are left underexplored.}

\x{Therefore, in this work, we choose to focus on affect arousal, a core construct of user experience in mass communication, and examine how it correlates with data visualization design. 
We hope our work can contribute a new perspective to understanding data communication and user-centered visualization design.}
% \x{Therefore, this work taps into this research question and chooses affective arousal as a starting point given its key role in mass communication (see Section 2.2).}


\subsection{Affective Arousal}

% Affective arousal is a physiological or psychological state of energization, activation, alertness or excitement~\cite{posner2005circumplex}.
Affective arousal is about the activation of emotion~\cite{russell1980circumplex} and is viewed as a key construct of human affects. 
% Affects, namely emotions, moods or feelings, are the psychological state of every human being. For years, many researchers have tried to identify and conceptualize the important characteristics or dimensions of affects.
For example, the well-known circumplex model~\cite{russell1980circumplex} views affects as combinations of valence and arousal. While valence describes whether an affect is positive or negative, arousal reflects the intensity of the affect, ranging from calm to excitement. 
\x{Along with the conceptualization of affective arousal, many researchers began to conduct studies to investigate how to elicit affective arousal specifically. For example, Sundar and Kalyanaraman~\cite{sundar2004arousal} examined the relationship between animation and affective arousal using physiological indicators. Lang~\etal~\cite{lang1999effects} and Gorn~\etal~\cite{gorn1997effects} used self-report methods to gather ratings from users and assess how color and pacing influenced perceived arousal.}



% Due to the wide application of the circumplex model, affective arousal has become a key concept in affective science and has been studied by many. 
% One of the most well-known model of affect, the PAD model~\cite{russell1977evidence}, plots affective words in a space defined by three dimensions: valence, arousal, and dominance.
% 基于我们的motivation,我们对于可视化的arousal研究,与marketing领域有许多相同之处。然后讲方法。
\x{Meanwhile, as affective arousal is closely related to people's reactions, behaviors, and information processing~\cite{storbeck2008affective}, it
has been massively studied in research about visual communication, such as web advertising and TV broadcasting.
% Marketing schemes often seek to exploit such effects to make products seem more desirable.
For example, Singh and Hitchon~\cite{singh1989intensifying} reviewed dozens of studies conducted in the last century that examined arousal-centered topics, such as how to arouse viewers most advantageously in a TV show and how to increase consumers' desire for a product by eliciting arousal.
Given the fundamental role of affective arousal, this thread of research continues to prosper until today while dealing with trending topics such as how to design arousing games, user interfaces, or immersive environments~\cite{sundar2004arousal,mavridou2018towards,cusveller2014evoking}. For example, Aoki~\etal~\cite{aoki2022emoballoon} examined how to design chat balloons that represent various levels of arousal in text chats. Cusveller~\etal~\cite{cusveller2014evoking} explored how to evoke affective arousal in computer games.
% - Imagery can also influence affective arousal.
% Television advertisements for new cars, for example,
% often include exciting imagery such as high-speed driving along dangerous roads to increase people's desire for the car~\cite{storbeck2008affective}. 
% In terms of research methodology, prior work has suggested that affective arousal can be studied in combination with valence or be studied independently or concentratedly.
% For example, some researchers measured valence and arousal at the same time and observed where emotion located in the circumplex space~\cite{valdez1994effects,detenber1998roll,lang1995effects}.
}
% 心理学里侧重于arousal的:How arousal modulates memory: Disentangling the effects of attention and retention; COLOR AND PHYSIOLOGICAL AROUSAL; An arousal effect of colors saturation: A study of self-reported ratings and electrodermal responses.
% - Talarico~\etal~\cite{talarico2004emotional} found that when being affectively aroused, people remember past experiences better and report more vivid recollections.
% - Research on marketing and advertising has suggested that people are more likely to spend time and money on affectively arousing objects~\cite{sherman1997store,kaltcheva2006should}. }
%~\cite{christianson1992emotional,lang1995effects}, information processing~\cite{storbeck2008affective}, judgments~\cite{gorn2001arousal}, and .
%Given such empirical observations, understanding how design influences affective arousal becomes a natural pursuit. 
% enhance learning, intensify evaluations, and lead to more extreme behaviors.
% arousal has been related to simple processes such as awareness and attention [7], but also to more complex tasks such as information retention and attitude formation [8].
% For storytelling, with the stories having both negative and positive valences, arousal is viewed as a critical factor that charges communication and makes stories go viral~\cite{}. 

% it stimulated less purchases than blue in the scenario of shopping~\cite{bellizzi1992environmental}.


% The aforementioned work suggests that affective arousal is an important constituent of affects and manipulating affective arousal is a common goal of design. 
% by reviewing the findings from neighboring disciplines, we found that it is essential to xxx
\x{This work follows this research thread and studies affective arousal specifically in the context of visualization design. We see this work as an initial step towards exploring what design factors contribute to affective arousal in data visualization. }


% \subsection{Evaluating Data Visualization Design}

% Traditionally, functionality has been the key to evaluating data visualization design. In their pioneering work, Cleveland and McGill~\cite{cleveland1984graphical} evaluated a set of visual channels and compared how they influenced the accuracy of data perception, thus providing essential guidelines for generating effective data visualization design.
% For years afterward, effectiveness and efficiency have been highlighted as the core of data visualization design~\cite{saket2016beyond}. For example, Tufte~\cite{tufte2001visual} proposed the famous "above all else show the data” principle, and a good data visualization design is often the design that effectively serves analytical tasks~\cite{shneiderman2003eyes}, such as extracting values and identifying clusters. 

% However, in recent years, as data visualization has been increasingly applied to domains such as storytelling, public art, and everyday entertainment~\cite{wang2019emotional,kosara2013storytelling,pousman2007casual}, understanding visualization design beyond functionality has become an emerging research area~\cite{saket2016beyond}. More researchers began to pay attention to user experience-related factors.
% For example, Bateman~\etal~\cite{bateman2010useful} examined the embellishment in data visualizations from the perspective of memorability. Borkin~\etal~\cite{borkin2013makes,borkin2015beyond} explored which components in data visualizations can make viewers remember and recall data better. Cawthon~\etal~\cite{cawthon2007effect} and Harrison~\etal~\cite{harrison2015infographic} investigated people's aesthetic feelings towards data visualizations.
% We also see more and more work incorporating affect-related measurements into data visualization evaluation. 
% For example, in their study, Kennedy and Hill~\cite{kennedy2018feeling} asked participants to write down their feelings about data visualizations. They found that emotion is a key component of user engagement with data.
% We also notice that many researchers~\cite{lan2021understanding,wang2019comparing,hung2018affective} have used indicators such as affective engagement and the feeling of fun or interest to evaluate visualization design. Given this trend, Wang~\etal~\cite{wang2019emotional} argued that affects should be included as an important measurement of visualization design. 
% As another work that responds to this research trend, this work aims to find out more about user experience with data visualization design through the lens of affective arousal.


\subsection{Affective Design in Data Visualization}


% For example, in their study, Kennedy and Hill~\cite{kennedy2018feeling} asked participants to write down their feelings about data visualizations. They found that emotion is a key component of user engagement with data.
% We also notice that many researchers~\cite{lan2021understanding,wang2019comparing,hung2018affective} have used indicators such as affective engagement and the feeling of fun or interest to evaluate visualization design. Given this trend, Wang~\etal~\cite{wang2019emotional} argued that affects should be included as an important measurement of visualization design.

In recent years, researchers~\cite{bartram2017affective,lan2021smile} have introduced the idea of \textit{affective design} to the community of data visualization and aimed to understand how design elements influence viewers' affective feelings towards data visualization.
For example, Bartram~\etal~\cite{bartram2017affective} identified a set of affective color palettes that can elicit eight categories of affects, such as calmness and positivity. 
Boy~\etal~\cite{boy2017showing} investigated one specific category of affects, empathy, and found that using anthropomorphic pictograms to represent human rights data does not help trigger empathy. 
% Morias~\cite{} xxx icon xxx affects.
Lan~\etal~\cite{lan2021smile} summarized a set of affect-related design factors in infographics, such as the usability of the design and the expressiveness of the design.
They also explored the dynamic design elements in data visualization design~\cite{lan2021kineticharts} and proposed an animation design scheme, Kineticharts, for creating charts that express five positive-valenced affects, such as joy, surprise, and amusement.
\x{After this work, they continued to explore negative-valenced affects in serious data storytelling~\cite{lan2022negative}.}

However, although the aforementioned work has laid a foundation for investigating how data visualization design influences affects, \x{it has two main limitations. First, previous work usually treats affects as discrete categories, and this paradigm has largely overlooked the continuous aspects of affects. Second, compared to other concepts such as valence, arousal has been less studied, thus limiting our understanding of how affects are activated by visualization design.}
%existing studies have only explored very limited aspects of the richness of affects. Especially, since the above work 
To bridge this gap, this work investigates affective arousal specifically. As a continuous measurement, affective arousal helps capture the intensity of people's affective experience with visualization, thus contributing new knowledge around affective visualization design.


