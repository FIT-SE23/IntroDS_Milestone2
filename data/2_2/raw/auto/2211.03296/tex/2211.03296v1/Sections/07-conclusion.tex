\section{conclusion}

\x{This work explores the relationship between visualization design and affective arousal.}
First, we collected a corpus of 265 data visualizations and conducted a crowdsourcing study with 184 participants to gather ratings on the affective arousal elicited by their design \x{as well as qualitative feedback}.
\x{Based on the study data, we first manually coded the arousal-related design features reported by the participants, then mapped these features to computable variables and constructed regression models to identify the most significant features.
By comparing the results of the models, we finally identified four design features (\eg colorfulness, the number of different visual channels) that were cross-validated as important features correlated with affective arousal.
We hope this work will inspire more future research on affective visualization design and user experience with data visualization.
% At last, the model that performed best explained 69.8\% of the variance in user ratings.
% Our model explained 69.8\% of the variance in user ratings and suggested 13 design features that are significantly related to affective arousal.
% Last, based on all the quantitative and qualitative analyses, we proposed both low-level and high-level design implications to guide the design as well as the automatic generation of data visualizations.
}