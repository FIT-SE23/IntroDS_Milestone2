\section{Conclusion}
In this study, we explored a new methodology to maximize the effectiveness of bandit algorithms for optimizing the mutation of fuzzers.
We first made the three observations: (a) blending different types of mutation operators into a single mutation is not imperative; (b) batch size is suitable as an optimization target and can have different optimal values depending on seed size; (c) various bandit algorithms give different levels of performance improvement.

Based on these findings, we implemented \OurMethodName-AFL++, which successfully achieved better code coverage than AFL++ in the evaluation.
Also, the performance improvement delivered by \OurMethodName{} was competitive with existing methods.
Although, on the other hand, it did not significantly increase bugs found in the evaluation, \OurMethodName-AFL++ found three previously unseen vulnerabilities in real-world programs from OSS-Fuzz to prove its value in bug identification.

We believe that our results not only provide a general improvement in the performance of existing unoptimized mutation-based fuzzers, but also provide insights to enhance the performance of other fuzzers employing bandit algorithms.
In addition, because our results include practical evaluations conducted with PUTs in OSS-Fuzz, we believe that \OurMethodName-AFL++ can be introduced in OSS-Fuzz with little effort.
We would like to contribute to the introduction of SLOPT-AFL++ in large-scale fuzzing efforts such as Google's OSS-Fuzz initiative.
