\documentclass[10pt,letterpaper,compsoc,conference]{iiswc22}

\usepackage{cite}
\usepackage{amsmath,amssymb,amsfonts}
\usepackage{algorithmic}
\usepackage{graphicx}
\usepackage[dvipsnames]{xcolor}
\usepackage[final]{microtype}
\usepackage[italic]{mathastext}
\usepackage{libertine}
\usepackage[T1]{fontenc}
\usepackage{textcomp}
\usepackage[varqu,varl]{zi4}
\usepackage[all]{nowidow}
\usepackage[auth-lg,affil-it]{authblk}
\usepackage[keeplastbox]{flushend}
\usepackage{url}
\usepackage{adjustbox}
\usepackage{multirow}
\usepackage{enumitem}
\usepackage[algo2e, linesnumbered]{algorithm2e}
\usepackage{algorithm}
\usepackage{bm}
\usepackage{balance}

\newlength\mylen
\newcommand\myinput[1]{%
	\settowidth\mylen{\KwIn{}}%
	\setlength\hangindent{\mylen}%
	\hspace*{\mylen}#1\\}

\usepackage[caption=false]{subfig}


\begin{document}


\title{Characterizing the Efficiency of Graph Neural Network Frameworks with a Magnifying Glass}


\renewcommand\Authsep{\qquad}
\renewcommand\Authand{\qquad}
\renewcommand\Authands{\qquad}


\author[1]{Xin Huang}
\author[2]{Jongryool Kim}
\author[3]{Bradley Rees}
\author[1]{Chul-Ho Lee}
\affil[1]{Texas State University}
\affil[2]{SK hynix America}
\affil[3]{NVIDIA}



\maketitle


\begin{abstract}
Graph neural networks (GNNs) have received great attention due to their success in various graph-related learning tasks. Several GNN frameworks have then been developed for fast and easy implementation of GNN models. Despite their popularity, they are not well documented, and their implementations and system performance have not been well understood. In particular, unlike the traditional GNNs that are trained based on the entire graph in a \emph{full-batch} manner, recent GNNs have been developed with different graph sampling techniques for \emph{mini-batch} training of GNNs on large graphs. While they improve the scalability, their training times still depend on the implementations in the frameworks as sampling and its associated operations can introduce non-negligible overhead and computational cost. In addition, it is unknown how much the frameworks are `eco-friendly' from a green computing perspective. In this paper, we provide an in-depth study of two mainstream GNN frameworks along with three state-of-the-art GNNs to analyze their performance in terms of runtime and power/energy consumption. We conduct extensive benchmark experiments at several different levels and present detailed analysis results and observations, which could be helpful for further improvement and optimization.
\end{abstract}


\section{Introduction}


Accurate estimates of posterior probabilities are crucial for neural networks in various Natural Language Processing (NLP) tasks~\cite{icml17,DBLP:conf/nips/Lakshminarayanan17}. For example, it would be helpful for humans if the models deployed in practice abstain or interact when they cannot make a decision with high confidence~\cite{DBLP:journals/jamia/JiangOKO12}. While Pre-trained Language Models (PLMs) have improved the performance of many NLP tasks~\cite{bert,roberta}, how to better avoid miscalibration is still an open research problem ~\cite{calibration_emnlp20,dan_roth_emnlp21}. 
\begin{table}[t!]
    \centering
    \begin{tabular}{l|p{0.65\columnwidth}}
    \hline

    %  Example 1: & It is \hlc[cyan!10]{a} \hlc[red!40]{warm} \hlc[red!60]{funny} \hlc[red!40]{engaging} \hlc[cyan!20]{film} . \\ \hline
     Positive & a fast \hlc[green!10]{funny} \hlc[green!40]{highly} \hlc[green!80]{enjoyable} movie.\\ \hline
    %  like a south of the border melrose place
     
     Negative & It's about \hlc[red!5]{following} your \hlc[green!10]{dreams} \hlc[red!10]{no} matter \hlc[red!5]{what} your \hlc[green!5]{parents} think.\\
    \hline
  \end{tabular}
    \caption{Two motivating examples with highlight explanations~\cite{SST}. The saturation of the colors signifies the magnitude. The confidence of the model should be easily recognized by looking at token attributions.}
    % \vspace{-4mm}
    \label{tab:example-m}
\end{table}
In this paper, we investigate if and how model explanations can help calibrate the model. 

Explanation methods have attracted considerable research interest in recent years for revealing the internal reasoning processes behind models~\cite{IG,Uncertainty_Aware_Attention,deeplift}. Token attribution scores generated by explanation methods represent the contribution to the prediction~\cite{diagnostic}. Intuitively, one can draw some insight for analyzing and debugging neural models from these scores if they are correctly attributed, as shown in Table~\ref{tab:example-m}. For example, when the model identifies a highly indicative pattern, the tokens involved would have high attribution scores for the predicted label and low attribution scores for other labels. Similarly, if the model has difficulty recognizing the inductive information of any class (i.e., the attribution scores are not high for any label), the model should not be highly confident. As such, the computed explanation of an instance could indicate the confidence of the model in its prediction to some extent.
 
Inspired by this, we propose a simple and effective method named \textbf{CME} that can be applied at training time and improve the performance of the confidence estimates. The estimated confidence measures how confident the model is for a specific example. Ideally, reasonable confidence estimates should have higher confidence for correctly classified examples resulting in higher attributions than incorrect ones. Hence, given an example pair during training with an inverse classification relationship, we regularize the classifier by comparing the wrong example's attribution magnitude and the correct example's attribution magnitude.

Our work is related to recent works on incorporating explanations into learning. Different from previous studies that leverage explanations to help users predict model decisions~\cite{DBLP:journals/corr/abs-2102-02201} or improve the accuracy~\cite{DBLP:conf/icml/RiegerSMY20}, we focus on answering the following question: \textit{are these explanations of black-box models useful for calibration?} If so, how should we exploit the predictive power of these explanations? Considering the model may be uninterpretable due to the nature of neural networks and limitations of explanation method~\cite{Fragile,DBLP:conf/nips/YehHSIR19}, a calibrated model by CME at least can output the unbiased confidence. Moreover, we exploit intrinsic explanation during training, which does not require designing heuristics~\cite{xiye1} and additional data augmentation~\cite{mixup21acl}.
% Are these explanations useful for calibrating the model?

We conduct extensive experiments using BERT~\cite{bert} and RoBERTa~\cite{roberta} to show the efficacy of our approach on three natural language understanding tasks (i.e., natural language inference, paraphrase detection, and commonsense reasoning) under In-Domain (ID) and Out-of-Domain (OD) settings. CME achieves the lowest expected calibration error without accuracy drops compared with strong SOTA methods, e.g.,~\citet{mixup21acl}. When combined with Temperature Scaling (TS)~\cite{icml17}, the expected calibration errors are further reduced as better calibrated posterior estimates under these two settings.



\section{Related Work}

\paragraph{Inverse Rendering of Indoor Scenes} Inverse rendering attempts to reconstruct geometry and spatially-varying material and lighting information from monocular (which is our case) or multiple RGB images. Most previous methods only recognize one or part of the above attributes. Geometry reconstructions, including depth estimation and surface normal reconstruction, has been widely studied \cite{eigen2015predicting,liu2019planercnn}.
Most material reconstruction methods are only able to either estimate diffuse albedo~\cite{li2018cgintrinsics, barron2013intrinsic, karsch2014automatic} or classify material categories~\cite{bell2015material}.
For lighting estimation, recent deep learning methods have made progress in estimating global~\cite{gardner2017learning,gardner2019deep} and even spatially-varying~\cite{garon2019fast,song2019neural,li2020inverse} lighting conditions.
Recent works attempt to predict multiple intrinsics jointly by a holistic inverse rendering framework. Li et al.~\shortcite{li2020inverse} proposed a method to reconstruct disentangled geometry, spatially-varying reflectance and lighting from a single RGB indoor scene image.


\paragraph{Lighting Estimation and Relighting.}
Light estimation is one of the sub-tasks of inverse rendering. Most previous works ignore spatially-varying effects and predict a single environment map for the whole scene \cite{gardner2017learning,sengupta2019neural,munkberg2022extracting}. Indoor scenes suffer from spatial variations, thus recent work explores spatially-varying lighting estimation for indoor scenes. The representation of spatially-varying illumination includes environment maps, per-pixel spherical lobes~\cite{li2020inverse} (spherical Harmonics/Gaussians), or 3D voxel grids~\cite{wang2021learning}. Relighting is also a widely-studied relevant  task. \citet{griffiths2022outcast} leverages screen-space method to detect occlusion and cast shadows to relight an outdoor image. \citet{li2022physically} proposed a novel pipeline to modify the light conditions within an indoor scene.


\paragraph{Neural Scene Representations.} 
Neural representations are a rapidly growing area of research. Recent advances include  voxels~\cite{yu2021plenoxels,sun2021direct}, hashgrids~\cite{muller2022instant}, point clouds~\cite{aliev2020neural}, and neural implicit functions~\cite{mildenhall2020nerf,wang2021neus,yariv2021volume,yariv2020multiview}. 
Neural radiance fields (NeRFs)~\cite{mildenhall2020nerf} represents scenes as neural implicit functions, encoding a scene as a continuous volumetric radiance field of color and density. With volume rendering, a NeRF can synthesize novel view images with promising results. Our proposed method uses a NeRF as the representation of the out-of-view area of the scene (Sec.~\ref{sec:background}).

\paragraph{Differentiable Rendering.} A number of recent inverse rendering works utilize differentiable rendering to recover complex light transport effects. Some recent works have proposed general-purpose physically-based differentiable renderers~\cite{Li:2018:DMC,NimierDavidVicini2019Mitsuba2}. \citet{Zhang:2020:PSDR} and \citet{Zeltner2021MonteCarlo} discussed a rigurous theory of differentiable light transport and Monte-Carlo combinations. These physically-based methods achieve high-quality global illumination effects at the cost of substantial performance overhead. Some differentiable rendering techniques are customized for specific purpose such as texture~\cite{nimier2021material}, split-sum lighting and mesh extraction~\cite{munkberg2022extracting}. Our method designs a Monte-Carlo based in-network differentiable rendering layer to recover the appearance of indoor scenes (Sec.~\ref{sec:render}).

\paragraph{Indoor Scene Datasets.} 
Supervised learning requires a large database of indoor scene images and their corresponding ground truth geometry, material, and lighting for network training. Datasets include 3D shape models~\cite{chang2015shapenet}, real-world scans~\cite{chang2017matterport3d, dai2017scannet}, and scene datasets~\cite{song2017semantic,savva2017minos,li2018interiornet,li2021openrooms}, which can be classified as either real or synthetic data. Real datasets provide real-world images and geometry, while synthetic datasets provide arbitrary scene annotations for inverse rendering, some of which, such as materials and illumination, are difficult to acquire from real world. To the best of our knowledge, InteriorNet~\cite{li2018interiornet} and OpenRooms~\cite{li2021openrooms} are so far the highest-quality public indoor datasets with spatially-varying photorealistic material and illumination annotations. Unfortunately, InteriorNet provides only LDR results, while OpenRooms provides only lighting information on the scene surface (instead of at any 3D location), and lacks the complexity of material and furniture variations. We present a new indoor scene HDR dataset to tackle their shortcomings.


\begin{figure*}[t!]
%\vspace{-.6in}
\hspace{-0.2in}
    \centering
    \includegraphics[width=1\columnwidth]{pictures/framework_left.png}
    \caption{\textbf{Model architecture of \method}. Three key procedures are highlighted in red dotted box: 1) \textbf{Relation Prediction} (Sec.~\ref{sec:relation}): Knowledge Interaction Layers (\texttt{KIL}) predicts relation action for each entity mention. 2) \textbf{One-step State Transition} (Sec.~\ref{sec:crw}): Based on the predicted relation, $\KG$ re-weights each graph and conduct contextualized random walk to update entity distribution state. 3) \textbf{Knowledge Integration} (Sec.~\ref{sec:lm}): An weighted aggregated entity embedding is added into a placeholder token as retrieved knowledge. 
    % By repeating such procedure $T$ times, \method could retrieve knowledge that are $T$-hop away from initial entities in the question, and answer complex open-domain questions.
    }
    \label{fig:framework}
%  \vspace{-.1in}
\end{figure*}


% In this section, we introduce the model architecture of Knowledge Graph Reasoning integrated Language Model (\method) and how we pre-train \method to reason on Wikipedia corpus.





% {\footnotesize
% \begin{align}
%   & \Pb\big(a | q, \{m_i\}\big) = \prod_{m_i} \Pb\big(a | q, m_i\big)\\
%   & = \prod_{m_i} \Pb\big(a | q, m_i, \pib_i^{1:T}\big) \cdot \Pb\big(\pib_i^{1:T} | q, m_i\big)\\
%   & = \prod_{m_i} \underbrace{\Pb\big(a | q, m_i, \pib_i^{1:T}\big)}_\text{1) \texttt{LM} reader} \cdot \sum_{t} \underbrace{\Pb\big(\pib_i^{t} | q, m_i, \pib_i^{<t}\big) }_\text{2) $\KG$ Reasoning} \nonumber
% \end{align}
% \begin{align}
%     \Pb\big(\pib_i^{t} | q, m_i, \pib_i^{<t}\big) = \Pb\big(a_i^{t} | q, m_i, \pib_i^{<t}\big) \cdot \Pb\big(\pib_i^{t} | a_i^{t}, \pib_i^{t-1}\big)
% \end{align}
% }







% \begin{align}
%     \pib^{(t)}_i  = \Pb^{(t)}_{ent}\big(e | m_i, q, \pib^{<t}_i \big)\\
%     a^{(t)}_i = \Pb^{(t)}_{rel}\big(r | m_i, q, \pib^{<t}_i \big)
% \end{align}




% In this paper, we propose to use knowledge graph reasoning to retrieve necessary knowledge and answer such queries end-to-end. Specifically, we desire the model to conduct path-based Reasoning for every entity mentions. We maintain different states $\pib^{(t)}_i \in \mathcal{R}^{|\mathcal{E}|}$, each of which is a probability vector representing the distribution of mention $m_i$ staying at each entity at $t$-th reasoning step, i.e., $\pib^{(t)}_i  = \Pb^{(t)}_{ent}\big(e | m_i, q \big)$. For each state, \method should first predict the relation action to take at the current step, and then walk over the $\KG$ to reach informative entities to get the answer.


% many previous works assume the answer is also an entity node in the knowledge graph $\KG$, and tries to parse the question into a structured program (e.g., SQL query or a dependency graph). The problem with such methods is that answer is not necessarily to be an entity or appeared in $\KG$. 





% Many existing QA systems following retriever-reader pipeline adopts a retriever to rank passages in text corpus that are likely to contain missing knowledge, and then use a separate Language Model (\texttt{LM}, e.g., BERT and T5) as reader to re-encode question and retrieved passage and get answer. Instead, our proposed \method enables \texttt{LM} to directly interact with a Knowledge Graph, which stores a large amount of factual knowledge, to retrieve necessary knowledge and answer the question in an end-to-end manner.


% When it comes to open-domain questions, there might exist multiple entity mentions $\{m_i\}$ in question $q$, and the relationship between each entity to target answer could be different and unknown, which the model needs to detect from the question. Therefore, we propose to conduct KGR for each entity mention. Specifically, we maintain different states $\pib^{(t)}_i \in \mathcal{R}^{|\mathcal{E}|}$, each of which is a probability vector representing the distribution of mention $m_i$ staying at each entity at $t$-th reasoning step, i.e., $\pib^{(t)}_i  = \Pb^{(t)}_{ent}\big(e | m_i, q \big)$. For each state, \method should first predict the relation action to take at the current step, and then walk over the $\KG$ to reach entities that are informative to get the answer.




% If the answer $a$ is an entity and there exist a direct relational edge in $\KG$ linking an entity in the question to the answer, the model just needs to identify such link to predict the answer. However, for most cases, such a direct one-hop edge is missing due to incompleteness of the $\KG$, or the answer is not necessarily be an entity. Therefore, our goal of knowledge graph reasoning in \method is find most relevant entities that contain the required knowledge towards the answer, and add these entities back to \texttt{LM}.
% \paragraph{Knowledge Graph Reasoning}
% \YX{I removed the previous preliminary section and merge it with this section. The closed-book section is moved to the related works part.}


% \paragraph{Preliminary}
% We denote a Knowledge Graph $\KG = \big(\mathcal{E}, \mathcal{R}, \mathcal{A} = \{A_r \}_{r \in \mathcal{R}} \big)$, where each $e \in \mathcal{E}$ and $r \in \mathcal{R}$ is entity node and relation label.
% % as well as a binary function $r: \mathcal{E} \times \mathcal{E} \rightarrow \{\text{True}, \text{False}\}$ indicating whether relation $r$ holds between a pair of entities
% $A_r \in \{0,1\}^{|\mathcal{E}| \times |\mathcal{E}|} $ is a sparse adjacency matrix indicating whether relation $r$ holds between a pair of entities.
% % , s.t. $A_r[s,t] = 1 \Leftrightarrow  r(s,t) $.
% The task of knowledge graph reasoning aims at answering factoid query $(s, r, ?)$, i.e., which target entity has relation with source entity $s$. If $\KG$ is complete, we could simply get answers by checking adjacency matrix, i.e., $\{\forall t : A_r[s,t]=1\}$. For incomplete $\KG$ where many relational facts are missing, path-based reasoning approaches~\cite{DBLP:conf/emnlp/LaoMC11, DBLP:conf/emnlp/XiongHW17, DBLP:conf/iclr/DasDZVDKSM18} 
% have been proposed to answer the one-hop query via finding multi-hop paths. For example, to answer query $(s, \text{Mother}, ?)$ a path $s \xrightarrow{\text{Father}} j \xrightarrow{\text{Wife}} t$ could reach target answer $t$.


\paragraph{Preliminary}
We denote a Knowledge Graph $\KG = \big(\mathcal{E}, \mathcal{R}, \mathcal{A} = \{A_r \}_{r \in \mathcal{R}} \big)$, where each $e \in \mathcal{E}$ and $r \in \mathcal{R}$ is entity node and relation label.
% as well as a binary function $r: \mathcal{E} \times \mathcal{E} \rightarrow \{\text{True}, \text{False}\}$ indicating whether relation $r$ holds between a pair of entities
$A_r \in \{0,1\}^{|\mathcal{E}| \times |\mathcal{E}|} $ is a sparse adjacency matrix indicating whether relation $r$ holds between a pair of entities.
% , s.t. $A_r[s,t] = 1 \Leftrightarrow  r(s,t) $.
The task of knowledge graph reasoning aims at answering a factoid query $(s, r, ?)$, i.e., which target entity has relation $r$ with the source entity $s$. If $\KG$ is complete, we could simply get answers by checking the adjacency matrix, i.e., $\{\forall t : A_r[s,t]=1\}$. For incomplete $\KG$ where many relational facts are missing, path-based reasoning approaches~\cite{DBLP:conf/emnlp/LaoMC11, DBLP:conf/emnlp/XiongHW17, DBLP:conf/iclr/DasDZVDKSM18} 
have been proposed to answer the one-hop query via finding multi-hop paths. For example, to answer the query $(s, \text{Mother}, ?)$, a path $s \xrightarrow{\text{Father}} j \xrightarrow{\text{Wife}} t$ could reach the target answer $t$. 
% \YS{say something about the connection of these preliminaries to our model.}
% Previous work mainly focus on reasoning over structured $\KG$ only, while in this paper we try to integrate $\KG$ into neural Language Models to solve 
In this paper we try to integrate symbolic $\KG$ reasoning into neural \texttt{LM}s and help it deal with ODQA problems.
% \looseness=-1

% \vspace{0.05in}
\paragraph{Overview of \method}
We illustrate the overall architecture of \method in Figure~\ref{fig:framework}. All the \colorbox{lb}{light blue blocks} are our added components to support $\KG$ reasoning, while the \colorbox{db}{dark blue} Transformer layers are knowledge-injected \texttt{LM}. The key component of \method for conducting $\KG$ reasoning is the Knowledge Interaction Layers (\texttt{KIL}), which are added amid \texttt{LM} layers to enable deeper interaction with the $\KG$. %by sending instruction to and receiving knowledge.
%, and like the cream filled inside wafers of O\textsc{reo} cookies.


% More specifically, 
% % maintains states for each entity mention and conduct walking. 
% we highlight the three procedures in red dotted box. 
% Firstly, \texttt{KIL} predicts relation action to take for each entity mention, based on which $\KG$ re-weights the graph and conduct contextualized random walk to update entity distribution state. An weighted aggregated entity embedding will then be sent back to \texttt{KIL} and add into a placeholder token as retrieved knowledge. By repeating such procedure multiple times, \method could retrieve knowledge from $\KG$ that are multi-hop away from initial entities in the question, based on which to answer complex open-domain questions.


%\YS{The definition below is confusing. From (discrete) stochastic process terminology, the states are discrete, say entities. Then we have transition probabilities from one state to another state, which is usually described by a T (or P) matrix. Then we have distribution over states at each time stamp t, which is $\pi^{(t)}$. A path naturally is a sequence of states. Can you please make the definitions clearer and try to connect to convention notations? } 
%\YS{by re-reading it, $\pi_i^{(t)}$ in the current writing is a state distribution starting from entity $i$ at timestamp t. In other words $\pi_i^{(0)}$ is an all zero vector except at the position of i, which is 1. ZN: In current notation, i is just a index (denote mention $m_i$) not entity ID. $pi_i$ could refer to different entity based on results of entity linking (e.g., during inference, we use entity linking model to calculate the entity distribution)
%
%During pre-training, it's indeed a one-hot vector, with 1 at $e_i$?
%
%Regarding to how to connect to stochastic process, I think in our case entity is the discrete state (or vertex in RW), pi is the distribution, contextualized RW matrix (changed by relation) is the transition matrix. 

%I currently simply denote probability vector pi as "state", not sure what's the better terminology for it? As in our case we didn't really leverage the discrete assignment, but just utilize this pi

%}
%\YS{$\pi_i ^{(t)}$ is the state distribution at time t for a walk starting at $i$. This is not the state. Please also do not use ``path'' to denote the ``state distribution sequence''. The state distribution will be changed at each time stamp given a different transition matrix. Please do not introduce terminology that is not completely necessary.} \YS{if we drop the rigorous terminology, $\pi_i ^{(t)}$ can be called entity distribution at time $t$ with respect to entity $i$.}

Given a question $q=$ ``The Bauhaus represented Germany's recovery from which event?'', QA model needs to extract knowledge about all $n$ in-context entity mentions $M=\{m_i\}_{i=1}^n$, e.g., the history of ``Germany'' at the time when ``Bauhaus'' is founded, to get the answer $a=$ ``World War I''. Such open-domain Q\&A can be abstracted as $P(a | q, M)$. 
% \looseness=-1
% In this paper, we propose to empower Language Models (\texttt{LM}, e.g., BERT and T5) with knowledge graph reasoning to retrieve necessary knowledge and answer such query in an end-to-end manner. Specifically, starting from each entity mention $m_i \in M$, we desire the model to learn 


Starting from each mentioned entity $m_i$, we desire the model to learn to walk over the graph to retrieve relevant knowledge and form a $T$-length reasoning path for answering this question, where $T$ is a hyper-parameter denote the 
longest reasoning path required to answer the questions.
% Starting from each mentioned entity $m_i$, we desire the model to learn to walk over the graph to retrieve relevant knowledge for answering this question.
We define each reasoning path starting from the entity mention $m_i$ as a chain of entities (states) random variables $\rho_i = \{e^{t}_i\}_{t=0}^T$, where each mentioned entity is the initial state, i.e., $e^{0}_i = m_i$. The union of all paths for this question is defined as $\Rho = \{ \rho_i \}$, which contains the reasoning paths from each mentioned entity to answer the question.


% $\pib_i^{(0)} \xrightarrow{\rb_i^{(1)}} \pib_i^{(1)} \ldots  \pib_i^{(T-1)}\xrightarrow{\rb_i^{(T)}} \pib_i^{(T)}$ starting from each entity mention $m_i \in M$. At $t$-th reasoning step, each $\pib^{(t)}_i = \Pb_{ent}^{(t)}(e | q, m_i) \in \mathcal{R}^{|\mathcal{E}|}$
% is a probability distribution, with $\pib^{(t)}_i[e]$ being the probability of staying at entity $e$ 
% % is a entity distribution row-vector
% is a probability vector representing the current entity distribution starting from mention $m_i$, 
% and $\rb_i^{(t)}  \in \mathcal{R}^{|\mathcal{R}|}$ 
% % is a relation action probability row-vector. 
% % \YX{Are all the vectors in this paper row vectors? It will be good if we can be consistent. If so, just add note saying that all the vectors are row vectors.}
% is a probability vector representing the current relation distribution predicted to transit the state.

% Take the first reasoning step as an example, the state starting from entity mention $m$ = ``Bauhaus'' has highest probability staying at ``Walter Gropius'', who is the inventor of ``Bauhaus'' as its representative. And by knowing ``Walter'' was involved in the ``World War I'' in the second reasoning step, \method could infer the correct answer, via a path $\text{Bauhaus} \xrightarrow{\text{founded}} \text{Walter} \xrightarrow{\text{in war}} \text{World War I}$.


%For each reasoning path $\rho_i$, 
% \YS{Notation $\pi$ is overloaded. We didn't give a notation for a path yet. } 
\method factorizes $\Pb\big(a|  q, M \big)$ by incorporating  possible paths $\Rho$ as a latent variable, yielding:

%\YS{more precisely, we need to sum up all the possible paths. For each path $i, i_1, i_2, \ldots, i_T$, we do the factorization into (1) the probability to get the path; (2) the probability to arrive the answer given the path. For (1), decompose the path into T 1-step transitions (e.g., $T_{i_{t-1},i_{t}}$) }
% \YX{Please correct all occurrences of $\{m_i\}$ and $\{\pib_i\}$ to be with subsripts and upper scripts: $\{m_i\}_{i=1}^n$ and $\{\pib_i\}_{i=1}^n$. You can also use $M$ and $\Pbi$ directly. Also in Sec 3.2 you defined $|M|=N$; you can define it here.}
% {\small
% \begin{align}
% &\Pb\big( a |  q, M \big) = \Pb \big(\Pib | q, \{m_i\} \big) \cdot  \Pb\big(a | q, M, \Pib\}\big) \\
% &=\Big( \prod_{m_i} \Pb\big(\pib_i^{1:T} | q, m_i \big) \Big) \cdot  \Pb\big(a | q, \{m_i, \pib_i^{1:T}\}\big)    \\
% &=  \Big(\prod_{m_i} \prod_{t=1}^T \underbrace{\Pb\big(\pib_i^{t} |q, m_i, \pib_i^{<t} \big)}_\text{1) $\KG$ Reasoning} \Big)  \underbrace{  
% \Pb\big(a | q, \{m_i, \pib_i^{1:T}\}\big)}_\text{2) knowledge-injected \texttt{LM}} \nonumber \label{eq:overall}
% \end{align}
% }

{\small
% \vspace{-10pt}
\begin{align*}
&\Pb\big( a |  q, M \big) = \sum\nolimits_{\Rho}  \Pb \big(\Rho| q, \{m_i\}_{i=1}^n \big) \cdot  \Pb\big(a | q, M, \Rho\big) \\
&=\sum\nolimits_{\Rho}  \Big( \prod_{i=1}^{n} \Pb\big(\rho_i | q, m_i \big) \Big) \cdot  \Pb\Big(a | q, \{m_i, \rho_i\}_{i=1}^{n} \Big)    \\
&= \sum\nolimits_{\Rho}   \Big(\prod_{i=1}^{n} \prod_{t=1}^T \underbrace{\Pb\big(e_i^{t} |q, e_i^{<t} \big)}_\text{$\KG$ Reasoning (\ref{sec:reason})} \Big)  \underbrace{  
\Pb\Big(a | q, \{e_i^{0:T}\}_{i=1}^{n} \Big)}_\text{knowledge-injected \texttt{LM} (\ref{sec:lm})}  \label{eq:overall}
\end{align*}
}% \YX{(add explanation) here the second equation is because we assume each $\pib$ to be independent with each other and with other entities, i.e., $\Pb(\pib_i^{1:T}|q, M)=\Pb(\pib_i^{1:T}|q, m_i)$.}
%The second and third equation assume that reasoning path starting from different entities are mutually independent conditioned on question: $e_i^{(t)} \indep e_j^{(t)} \mid q, \forall t$.

We assume (1) reasoning paths starting from different entities are generated independently; and (2) reasoning paths can be generated autoregressively.

% \looseness=-1 
%\YS{the previous paragraph needs update. Need to define $\pi$ and explain why do we have these constraints, and how to implement these constraints.} \YS{I think what you try to do here is to justify the decomposition.  }

In this way, the QA problem can be decomposed into two entangled steps: 1) $\KG$ Reasoning, which autoregressively walks through the graph to get a path $\rho_i$ starting from each entity mention $m_i$; and 2) knowledge-injected \texttt{LM}, which benefits from the reasoning paths to obtain the out-context knowledge for answer prediction. 

The relational path $\rho_i$ in $\KG$ Reasoning requires the selection of next entity $e_i^t$ at each step $t$. We further decompose it into two steps: 1.a) relation prediction, in which \texttt{LM} is involved to predict the next-hop relation 
%next-hop continuous relation action 
based on the current state and context; and 1.b) the non-parametric state transition, which is to predict the next-hop entity based on the $\KG$ and the predicted relation. 
%to updating entity state. 
Formally:

% \vspace{-10pt}
{
\footnotesize
\begin{align}  \nonumber
\underbrace{\Pb\big(e_i^{t} | q, e_i^{<t}\big)}_\text{$\KG$ Reasoning (\ref{sec:reason})} \!=\! \sum_r \underbrace{\Pb_{rel}\big(r_i^{t} | q,  e_i^{<t}\big)}_\text{relation prediction (\ref{sec:relation})} \cdot \underbrace{\Pb_{walk}\big(e_i^{t} | r_i^{t}, e_i^{<t}\big)}_\text{contextualized random walk (\ref{sec:crw})}
\end{align}  
}

%\YS{it seems to me $\underbrace{\Pb\big(\pib_i^{t} | q, m_i, \pib_i^{<t}\big)}_\text{$\KG$ Reasoning (\ref{sec:reason})}$ should be  $\underbrace{\Pb^{(t)}\big(e_j | q, m_i, \pib_i^{<t}\big)}_\text{$\KG$ Reasoning (\ref{sec:reason})}$. In other words, the distribution is not defined over $\pi$ but over entities $e_j$}


%To support differentiable learning, in our \method framework, we do not really do discrete state transition. Instead, 
We keep track of the entity distribution at each step $t$ via the probability vector\footnote{Throughout the paper, all vectors are row-vectors} $\pib^{(t)}_i \in \mathcal{R}^{|\mathcal{E}|}$,
with $\pib^{(t)}_i[e]$ being the probability of staying at entity $e$, i.e., $\Pb\big(e_i^{t} = e | q, e_i^{<t}\big)$. 
%\YS{We use $\pib^{(t)}_i[e]$ to denote $\Pb\big(e_i^{t} = e | q, e_i^{<t}\big)$. Please confirm whether it is the case.}

% More specifically, 
% % maintains states for each entity mention and conduct walking. 
We highlight the three procedures in red dotted box in Figure~\ref{fig:framework}. We take the first reasoning step starting from entity mention ``Bauhaus'' as an example. In the first red box within \texttt{KIL}, we predict which relation action should be taken for entity ``Bauhaus'', and send the prediction (e.g. ``Founded'') to $\KG$. In the second red box, $\KG$ re-weights the graph and conducts contextualized random walk to update entity distribution, where ``Walter'' has the highest probability. Finally, weighted by the entity distribution, an aggregated entity embedding is sent back to \texttt{KIL} and added into a placeholder token as the knowledge, so the later \texttt{LM} layer knows to focus on the retrieved ``Walter''. We introduce these steps in the following.










% Take the first reasoning step as an example, the state starting from entity mention $m$ = ``Bauhaus'' has highest probability staying at ``Walter Gropius'', who is the inventor of ``Bauhaus'' as its representative. And by knowing ``Walter'' was involved in the ``World War I'' in the second reasoning step, \method could infer the correct answer, via a path $\text{Bauhaus} \xrightarrow{\text{founded}} \text{Walter} \xrightarrow{\text{in war}} \text{World War I}$.




% Firstly, \texttt{KIL} predicts relation action to take for each entity mention, based on which $\KG$ re-weights the graph and conduct contextualized random walk to update entity distribution state. An weighted aggregated entity embedding will then be sent back to \texttt{KIL} and add into a placeholder token as retrieved knowledge. By repeating such procedure multiple times, \method could retrieve knowledge from $\KG$ that are multi-hop away from initial entities in the question, based on which to answer complex open-domain questions.




\paragraph{Input}
% \YX{I slightly changed the wording here.} 
Initially, we first identify all $N$ entity mentions $\{m_i\}_{i=1}^N$ in the input question $q$ as well as the corresponding $\KG$ entities\footnote{
For Wikipedia pretraining, we use the ground-truth entity label as one-hot initialization for $\pib_i^0$. For downstream tasks we use GENRE~\citep{DBLP:conf/iclr/CaoI0P21} to get top 5 entity links.
% We could flexibly utilize ground-truth entity label or trained model such as GENRE~\citep{DBLP:conf/iclr/CaoI0P21} to get entity linking.
}..
% We define the initial entity distribution for $m_i$ as $\pib_i^0 = \Pb(e | m_i, q)$
For each mention $m_i$ we add three special tokens as the interface for Knowledge Interaction Layers (\texttt{KIL}) to send instruction and receive knowledge: we add a [\texttt{S-ENT}] token before, and [\texttt{REL}], [\texttt{T-ENT}] tokens after each entity mention $m_i$. \texttt{KIL} can be flexibly inserted into arbitrary $\texttt{LM}$ intermediate layer. By default, we just insert each \texttt{KIL} every $N$ 
% \YX{$N$ is also the number of entity mentions. Can you change to another letter?} 
Transformer-based $\texttt{LM}$ layers, thus the input to the $t$-th \texttt{KIL} are contextualized embeddings of each token $k$ as $\texttt{LM}^{(t)}_k$, including added special tokens. 


% Research have shown that a single pre-trained Language Model (\texttt{LM}, e.g., T5) already stores a portion of knowledge within its parameter to solve some single-hop questions~\citep{DBLP:conf/emnlp/RobertsRS20}, but its performance is still lower than retrieval-based method that takes Wikipedia as knowledge source, and cannot handle multi-hop questions. 


\subsection{\texttt{LM} involved $\KG$ Reasoning} \label{sec:reason}
\noindent We first introduce the reasoning process $\Pb\big(e_i^{t} | q, e_i^{<t}\big) \!=\! \sum_r \Pb\big(r_i^{t} | q,  e_i^{<t}\big) \cdot\Pb\big(e_i^{t} | r_i^{t}, e_i^{<t}\big)$.
% \looseness=-1
% following Eq.(\ref{eq:reason}).
\subsubsection{Relation Prediction.} \label{sec:relation}
% \YX{I slightly changed the wording here.} 
\noindent For each entity mention $m_i$, we desire to predict which relation action should take $r_i^{t}$ as instruction to transit state. 
We define the predicted relation probability vector $\rb_i^{(t)} = \Pb_{rel}\big(r_i^{t} | q,  e_i^{<t}\big) \in \mathcal{R}^{|\mathcal{R}|}$ representing the relation distribution to guide walking through the graph. 
Denote the corresponding [\texttt{REL}] token as $\texttt{REL}[i]$ (and similarly for other special tokens). The contextual embedding $\texttt{LM}^{(t)}_{\texttt{REL}[i]}$ encode the relevant information in question $q$  %\YS{what is x?} 
that hints next relation. We maintain a global relation key memory $\texttt{K}_{rel} \in \mathbb{R}^{|\mathcal{R}| \times d}$ storing each relation's $d$-dimentional embedding. To calculate similarity, we first get relation query $Q^{(t)}_{\texttt{REL}[i]}$ by projecting relation token's embedding into the same space of key memory via a projection head Q-Proj\footnote{We denote a non-linear MLP projection as X-Proj$(h)=W^X_2 \sigma(W^X_1h+b_1)+b_2$, where X have different instantiations.} followed by a LayerNorm (abbreviated as LN), and then calculate dot-product similarity followed by softmax:
% \looseness=-1
\begin{align}
        &Q^{(t)}_{\texttt{REL}[i]} = \text{LN}^{(t)}\big(\text{Q-Proj}^{(t)}(\texttt{LM}^{(t)}_{\texttt{REL}[i]})\big), \\
    &\rb_i^{(t)} =\Pb_{rel}\big(r_i^{t} | q,  e_i^{<t}\big) = \text{Softmax} \big( Q^{(t)}_{\texttt{REL}[i]} \ \texttt{K}_{rel}^T   \big). 
\end{align}
%\YS{from the above formula,  $\pib_i^{<t}$ is not used to calculate the relation probability. Instead, it requires other intermediate results from language models.}

Note that the relation queries $\texttt{LM}^{(t)}_{\texttt{REL}[i]}$ are different for every mention $m_i$ and reasoning step $t$ depending on the context, and thus the the relation distributions $\rb_i^{(t)}$ gives contextualized predictions based on the question $q$. The predicted relations are sent to the knowledge graph reasoning module as instruction to conduct state transition.






% and embedding $\texttt{LM}^{(t)}_{\texttt{S-ENT}}$ should identify the current entity state, $\texttt{LM}^{(t)}_{\texttt{REL}}$ should identify the next relation ,acccmtion to take, and the retrieved knowledge (aggragated entity embedding) will be added back to $\texttt{LM}^{(t)}_{\texttt{T-ENT}}$ for passing the knowledge graph reasoning's results back to \texttt{LM}. 

% We denote each added token as a reasoning prompt. \method could simultaneously proceed Reasoning for all prompt within the context. For simplicity, we only focus on one specific entity $e$ to show the procedure.
% Before each reasoning step, we use $N$ Transformer layer of $\texttt{LM}$ to process the contexts. We denote $\texttt{LM}^{(t)}_{\texttt{S-ENT}}$, $\texttt{LM}^{(t)}_{\texttt{REL}}$, $\texttt{LM}^{(t)}_{\texttt{T-ENT}}$ as the three $d$-dimensional embedding vector before $t$-th reasoning step for the three token .




% For both pre-training and fine-tuning, \method takes some paragraphs $x$ as input and learns to predict outputs $y$ via $P(y|x)$. For pre-training, the task is to predict or autoregressively generate masked tokens, while for fine-tuning on QA, $x$ is question and $y$ is the answer.





% \begin{algorithm}
%   \caption{Pseudo Codes of Contextualized Random Walk (CRW) Implementation}
%   \Function{preprocess($\pib^0 = \{\pib_i^0 \}_{i=1}^N$, $A$)}{
%   }
% \end{algorithm}


\subsubsection{Contextualized KG Random Walk} \label{sec:crw}


% Many previous knowledge graph reasoning works propose to walk over the graph where each state is a discrete entity node. Such methods cannot be differentiated and they use reinforcement learning to get reward for training the walking agent. Our approach, instead, propose to keep a continuous entity distribution as the state, i.e., $\pib^{(t)} = \Pb^{(t)}_{ent}(e | m_i, x)$, thus the transition policy could be directly optimized through back-propagation. 
% \YX{Slightly changed the notation here; please check if it makes sense.}
\noindent Next, we introduce how we conduct state transition $\Pb_{walk}\big(e_i^{t} | r_i^{t}, e_i^{<t}\big)$. One classic transition algorithm is random walk, which is a special case of markov chain, i.e. the transition probability only depends on previous state. Consider a state at entity $s$, the probability walking to target $t$ is $\frac{1}{deg(s)}$ if $A[s,t]=1$. Based on it, we define the Markov transition matrix for random walk as $M_{rw} = \Db_A^{-1}A$, where the degree matrix $\Db_A\in \mathbb{R}^{|\mathcal{E}| \times |\mathcal{E}| }$ is defined as the diagonal matrix with the degrees $deg(1),\ldots,deg({|\mathcal{E}|)}$ on the diagonal.
With random walk Markov matrix $M_{rw}$ we can transit the state distribution as:
$\pib^{(t)} = \pib^{(t-1)} M$,
The limitation of random walk is that the transition strategy is not dependent on the question $q$. 
%\YS{x?} 
We thus propose a Contextualized Random Walk (\rw). 
% \looseness=-1

Based on the predicted relation distribution $\rb_i^{(t)}$, we calculate a different weighted adjacency matrix $\widetilde{A}_i^{(t)} \in \mathbb{R}^{|\mathcal{E}| \times |\mathcal{E}|}$ by adjusting the edge weight:
\begin{align}
    % &\widetilde{A}_i^{(t)} = \sum_{r  \in \mathcal{R}}  w_r  \cdot \rb_{i,r}^{(t)} \ \big(D_A^{-1}A_r\big)  \\
    &\widetilde{A}_i^{(t)} = \sum\nolimits_{r  \in \mathcal{R}}   w_r \cdot  \rb_{i,r}^{(t)}  \cdot  A_r,\\
    & {M_{crw, i}}^{(t)} =  \Db_{\widetilde{A}_i^{(t)}}^{-1} \widetilde{A}_i^{(t)}, \ \ \forall i \in [1, N].
\end{align}
where $w_r$ is a learnable importance weight for relation $r$ that helps solving downstream tasks, and $\rb_{i,r}^{(t)}$ is the probability corresponding to relation $r$ in $\rb_{i}^{(t)}$. With the transition matrix ${M_{crw,i}}^{(t)}$, the state transition is defined as $\pib^{(t)}_i = \pib^{(t-1)}_i  M_{crw,i}^{(t)}$.




% \YS{(1) the double normalization mentioned above seems redundant. Only the second one is needed. (2) writing wise, try to use high-level notation to help people to get a high-level idea, such as mentioning transition matrix that is specific to the starting point $i$ and the current step $t$. Then define this transition matrix. Then write down the calculation of $\pi_i^{(t)}$. (3) another confusing point is, $\pi_i^{(t)}$ is computed in a deterministic way. why do we need to use $P(\pi_i^{(t)}|...)$ notation for $\pi_i^{(t)}$? }



\rw allows each reasoning path $\rho_i$ to have its transition matrix. However, as the total number of entity nodes $|\mathcal{E}|$ could be huge (e.g., 5M for WikiData), we cannot afford to update the entire adjacency matrix for every in-batch mention. We thus adopt a scatter-gather pipeline to implement graph walking as shown in Algorithm~\ref{algo:rw}. We first gather the entity and relation probability to each edge, and then scatter the probability to target nodes. This allows us to simultaneously conduct message passing with modified adjacency weight $\widetilde{A}_i^{t}$ for all entity mention $m_i$ in parallel. 

\begin{algorithm}[!ht] 
\lstset{style=style_snippet}
\begin{lstlisting}[language=Python]
def ContextualizedRandomWalk(
    i_init, KG,   # initial entity index and Graph
    w_deg, w_rel, # inv(degree) and relation weights 
    p_ent, p_rel  # entity and predicted relation dis-                          
                  # tribution tensor @ t-th step.
): -> FloatTensor 
    # Get <src, rel, tgt> edge list of k-hop subgraph
    i_src, i_rel, i_tgt = k_hop_subgraph(i_init, KG)
    # Gather entity and relation probability to edge
    p_src  = (p_ent * w_deg)[:, i_src] # N x n_edge
    p_rel  = (p_rel * w_rel)[:, i_rel] # N x n_edge
    p_edge = l1_normalize(p_src * p_rel, dim=1)
    # Scatter edge probability to target node
    p_ent  = scatter_add(src=p_edge, idx=i_tgt, dim=1)
    return p_ent  #(t+1)-th step's entity distribution
\end{lstlisting}
\caption{Pytorch Pseudocode of CRW}\label{algo:rw}
\end{algorithm}


The complexity is $\#$ of in-batch entities times $\#$ of edges in $T$-hop subgraph starting from these entities, i.e., $\mathcal{O}(n \times \#\text{edge})$, and thus this operation is not expensive. 
% \YX{Is this local subgraph for only one example or for one batch? Please specify. If the latter, also specify the batch size here.}
Another concern is why not using Graph Neural Networks (GNNs). We provide discussion in Sec.~\ref{sec:gnn} in Appendix.



\subsection{Knowledge-Injected \texttt{LM}} \label{sec:lm}

\noindent After we get the updated entity distribution $\pib^{(t)}_i$, we want to inject such information back to the $\texttt{LM}$ without harming its overall structure. We maintain a global entity embedding value memory $\texttt{V}_{ent} \in \mathbb{R}^{|\mathcal{E}| \times d}$ storing entity embeddings. We only consider the entities within the sampled local subgraph in each batch. We thus get an entity index list $\Ib$ as the query to sparsely retrieve a set of candidate entity embeddings and then aggregate them weighted by entity distribution and embedding table. We then use a Value Projection block to map the aggregated entity embedding into the space of $\texttt{LM}$, and then directly add the transformed embedding back to the output of \texttt{T-ENT}.
\begin{align} 
    &V^{(t)}_i = \text{V-Proj}^{(t)}\big(\pib^{(t)}_i \cdot \texttt{V}_{ent}[\Ib]\big),\\
    &\widehat{\texttt{LM}}^{(t)}_{\texttt{T-ENT}[i]} =  \text{LN}^{(t)}\big(\texttt{LM}^{(t)}_{\texttt{T-ENT}[i]} + V^{(t)}_i \big).
\end{align}
Then, we just take all $\widehat{\texttt{LM}}^{(t)}_{\texttt{T-ENT}}$ as input to next Transformer-based \texttt{LM} layer to learn the interaction between the retrieved knowledge with in-context words via self-attention. 

By repeating the \texttt{KIL} for $T$ times, the final representation $\widehat{\texttt{LM}}^T$ is conditioned on the reasoning paths $\rho_i = e_i^{0:T}$, which reaches entities that are $T$-hop away from initial entity $m_i$ in the question. Finally, we can predict the answer of open questions $\Pb\big(a | q, \{e_i^{0:T}\}_{i=1}^{n} \big)$ by taking knowledge-injected representation $\widehat{\texttt{LM}}^T$ for span extraction, entity prediction or direct answer generation. 
% \YS{the previous paragraph is outdated.}
















\subsection{Pre-Train \method to Reason}




%\YS{I feel there are lots of smart designs for the pre-training tasks. But the writing does not reflect that yet.}\YS{try to write: issues; solutions; losses}

\noindent The design of \method allows end-to-end training given QA datasets. However, due to the small coverage of knowledge facts for existing QA datasets, we need to pretrain \method on a large-scale corpus to get good entity embeddings.
% it's unlikely to directly train \method using downstream QA tasks without proper initialization. 

% We therefore introduce how we pre-train \method using unlabelled Wikipedia Corpus to learn to reason over $\KG$.

\paragraph{Salient Span Masking} 
% \YS{shall we upgrade this into a subsection? right now we only have one subsection.}
One straightforward approach is to use
Salient Span Masking (SSM) objective~\citep{DBLP:journals/corr/abs-2002-08909} masks out entities or noun tokens requiring specific out-of-context knowledge. We mainly mask out entities for guiding \method to reason. Instead of randomly masking entity mentions, we explicitly sample a set of entity IDs and mask every mentions linking to these entities. This could prevent the model copy the entity from the context to fill in the blank. We also follow~\citep{DBLP:conf/nips/YangDYCSL19} to mask out consecutive token spans. We then calculate the cross-entropy loss on each salient span masked (SSM) token as $\mathcal{L}_{SSM}$. 
% \looseness=-1


%\YS{why these two objectives? How these two objective connect to the two losses mentioned below.}



\subsubsection{Weakly Supervised Training of $\texttt{KIL}$}
\noindent Ideally, \method can learn all the entity knowledge and how to access the knowledge graph by solely optimizing $\mathcal{L}_{SSM}$. However, without a good initialization of entity and relation embeddings, \texttt{KIL} makes a random prediction, and the retrieved entities by $\KG$ reasoning are likely to be unrelated to the question. In this situation, \texttt{KIL} does not receive meaningful gradients to update the parameters, and \texttt{LM} learns to ignore the knowledge.  
%\YS{what are the above two paragraphs? Can we put them into subsections related to specific losses? In other words, quickly tell the readers what they are reading.}
% no matter what entities retrieved within the reasoning procedure, the final aggregated embeddings could not benefit the language model objective, and thus the model won't get the correct signal to learn which relation should choose, and how to improve entity embeddings.
% \YX{This is not quite true since you also train the entity and relation embeddings. I'd suggest to weaken the tone. Like: However, if the entity and relation embeddings are not sufficiently trained, the model might not benefit from the \texttt{KIL} layers.} 
To avoid this cold-start problem and provide entity and relation embedding a good initialization, We utilize the following two external signals as self-supervised guidance.
% \looseness=-1

% \vspace{0.05in}
\paragraph{Entity Linking Loss}
To initialize the large entity embedding tables in $\texttt{V}_{ent}$, we use other entities that are not masked as supervision. Similar to~\citet{DBLP:journals/corr/abs-2004-07202}, we force the output embedding of [\texttt{S-ENT}] token before the first \texttt{KIL} followed by a projection head E-Proj to be close to its corresponding entity embedding:
\begin{align*}
    &E_{\texttt{S-ENT}[i]} = \text{LN}\big(\text{E-Proj}(\texttt{LM}^{(1)}_{\texttt{S-ENT}[i]})\big), \\
    &\Pb^{(0)}_{ent}\big(e | m_i, q \big) = \text{Softmax} \big( E_{\texttt{S-ENT}[i]}\ \texttt{V}_{ent}[\Ib]^T   \big),\\
    &\mathcal{L}_{ent} = \sum\nolimits_{m_i} -\log \Pb^{(0)}_{ent}\big(e | m_i, q \big) \cdot \pib^0_i[\Ib].
\end{align*}
% \YX{Is this softmax here over all entities? It is in-batch entities if I remember correctly. Either case, please specify it.}
Similar to Section~\ref{sec:lm}, we only consider entities within the batch, denoted by index $\Ib$.
This contrastive loss guides each entity's embedding $\texttt{V}_{ent}[e]$ closer to all its previously mentioned contextualized embedding, and thus memorizes those context as a good initialization for later knowledge integration. 
% \looseness=-1

% \vspace{0.05in}
\paragraph{Weakly Supervised Relation Path Loss}
Entity mentions within each Wikipedia passage are naturally grounded to WikiData $\KG$. Therefore, after we mask out several entities, we can utilize the $\KG$ to get all reasoning paths from other in-context entities to the masked entities as weakly supervised relation labels.
% \looseness=-1

\begin{figure}[t!]
    \centering
    \includegraphics[width=1.0\columnwidth]{pictures/pre.png}
    \caption{Pre-training sample w/ golden reasoning path. More real examples are shown in Table~\ref{tab:pretrain1} in Appendix.}
    \label{fig:pretrain_example}
%  \vspace{-.1in}
\end{figure}


Formally, we define a \textbf{Grounded Dependency Graph} $\mathcal{DG}$, which contains all reasoning paths within $T$-step from other in-context entities to masked entities, and then define $R_{\mathcal{DG}}(m_i, t)$ as the set of all relations over every edges for entity mention $m_i$ at $t$-th hop. Based on it, we define the weakly supervised relation label $q^{(t)}_i \in \mathbb{R}^{|\mathcal{R}|}$ as the probabilistic vector which uniformly distributed on each relation in set.
% , i.e., $q^{(t)}_i[r] = \frac{1}{|R_{\mathcal{DG}}(m_i, t)|}, \text{if }\ r \in R_{\mathcal{DG}}(m_i, t)$.
% $q^{(t)}_i[r] = \begin{cases}
%     \frac{1}{|R_{\mathcal{DG}}(m_i, t)|},& \text{if }\ r \in R_{\mathcal{DG}}(m_i, t)\\
%     0,              & \text{otherwise}
% \end{cases}$
Note that we call uniformly-weighted $q^{(t)}_i$ as weakly supervised because 1) some paths 
lead to multiple entities rather than only the target masked entity; 2) the correct relation is dependent on the context. Therefore, $q^{(t)}_i$ only provides all potential candidates for reachability, and more fine-grained signals for reasoning should be learned from unsupervised $\mathcal{L}_{SSM}$. We adopt a list-wise ranking loss to guide the model to assign a higher score on these relations than others.
\begin{align*}
\mathcal{L}_{rel} = \sum\nolimits_{m_i}\sum\nolimits_{t=1}^{T} -\log \Pb^{(t)}_{rel}\big(r | m_i, q \big) \cdot q^{(t)}_i.
\end{align*}

% As an example in Figure~\ref{fig:pretrain_example}
% , we take one paragraph from ''Bauhaus'' and mask out entities ''Germany`` and ''World War I`` as targets. We then construct the grounded dependency graph from other entities to these two masked entities and get the weakly supervised relation label for each in-context source entity per reasoning step. More real examples are shown in Table~\ref{tab:pretrain1} in Appendix.


Overall, $\mathcal{L}_{ent}$ and $\mathcal{L}_{rel}$  provide \method with good initialization of the large $\KG$ memory. Afterward, via optimizing $\mathcal{L}_{SSM}$, the reasoning paths that provide informative knowledge receive a positive gradient, guiding \method to reason.


% Specifically, for each Wikipedia passage containing $N$ entity mentions, we first construct a \textbf{K-hop Induced Subgraph}, in which all the paths with length shorter than $K$ connecting any pair of entities are retained. Then, during training, after we randomly mask out a set of entities, we could easily calculate a 


% spanning tree starting from the masked entities to all other in-context entities. Then, for each entity mention $m_i$, 


% To pre-train the model to reason, we desire a self-supervised task that the model could not simply answer by looking at local contexts, but require external knowledge. To serve the purpose, besides the vanilla masked language model, we specifically sample a portion of entities within each page, and mask out all their mentions. In this way, the model cannot simply copy the tokens from contexts to fill in the blank. In addition, as we have the grounded entity graph, we could pre-extract all paths from in-context entity mentions to target masked entities. These paths could serve as golden ground-truth for relation prediction during pre-training, guiding our model to make reasonable relation prediction at an early stage. As illustrated in Figure~\ref{fig:pretrain_example}, we mask out two entities Mudhoney and Nirvana from the paragraph, which is hard for a vanilla \texttt{LM} to generate based on context. We also provide the one-hop and two-hop paths from in-context entity mentions.




\section{Results and Discussion}

In this section, we provide and discuss the detailed benchmark results on the efficiency of DGL and PyG.

\begin{figure}[t]
	\vspace{0mm}
	\centering
	\includegraphics[width=0.95\linewidth, trim=2mm 1mm 2mm 1mm, clip]{fig/workflow}
	\vspace{0mm}
	\caption{Workflow of sampling-based GNN training.}
	\vspace{0mm}
	\label{fig:workflow}
\end{figure}

\subsection{Functional Testing}

Figure~\ref{fig:workflow} illustrates the end-to-end workflow of training a sampling-based GNN with mini-batch training. It can be divided into the following three main processes: data loading, graph sampling, and model training. We thus conduct `functional testing' on each main process to evaluate the performance of DGL and PyG. Note that for the entire training process, data loading is a one-time operation while the other two process, i.e., graph sampling and model training, are performed repeatedly and periodically for each training batch. Note also that we do not consider the inference of each model in this paper. We repeat the experiments for each functional test for ten times and report the average values. In addition, for the functional tests, we do not include the power/energy consumption results since the runtime of some functions are too small, e.g., a few milliseconds, which can lead to incorrect power/energy measurement.

\vspace{1mm}
\noindent \textbf{Data loader.} We first compare the data loader of DGL and PyG, which is used to load the input graph and its associated node features from storage and to create a library-specific graph object for the next process of graph sampling and model training. We present the runtime results in Figure~\ref{fig:loader}.

\vspace{1mm}

\noindent \textbf{Observation 1:} \textit{PyG's data loader is more efficient and user-friendly than DGL’s data loader.}

\vspace{1mm}

There are two main reasons. First, while both frameworks provide an easy-to-use interface to create and process the datasets, PyG integrates more datasets (around 80) into its library as compared with DGL (around 40). Specifically, five out of six datasets used in this work can be directly accessed from PyG's `dataset' module while three datasets are already included in DGL. Note that, for the datasets that are not included in the libraries, we follow the official instructions to process the raw datasets and to create their corresponding graph objects. Second, DGL uses a graph-centric programming abstraction, which makes rich information of the input graph accessible and enables full control of manipulating the input graph. As a consequence, the workload of creating a `DGLGraph' object is relatively higher than its counterpart in PyG.

\begin{figure}[t]
	\vspace{0mm}
	\centering
	\includegraphics[width=0.7\linewidth, trim=2mm 1mm 2mm 1mm, clip]{fig/loader}
	\vspace{0mm}
	\caption{Runtime of data loader.}
	\vspace{0mm}
	\label{fig:loader}
	\vspace{-3mm}
\end{figure}

\begin{figure}[t]
	\captionsetup[subfloat]{captionskip=1pt}
	\centering
	\subfloat[Neighborhood sampler]{%
		\includegraphics[width=0.47\linewidth, trim=0cm 0cm 0cm 0cm, clip]{fig/neighbor-sampler}
	}
	\hspace{0mm}
	\subfloat[GraphSAINT sampler]{%
		\includegraphics[width=0.47\linewidth, trim=0cm 0cm 0cm 0cm, clip]{fig/saint-sampler}
	}
	\vspace{0mm}
		\subfloat[ClusterGCN sampler: METIS]{%
		\includegraphics[width=0.47\linewidth, trim=0cm 0cm 0cm 0cm, clip]{fig/cluster-sampler-METIS}
	}
	\hspace{0mm}
	\subfloat[ClusterGCN sampler: Combining]{%
		\includegraphics[width=0.47\linewidth, trim=0cm 0cm 0cm 0cm, clip]{fig/cluster-sampler}
	}
	\vspace{0mm}
	\caption{Runtime comparison of graph samplers. Note that the range of y-axis is different across different figures.}
	\label{fig:sampler}
	\vspace{-3mm}
\end{figure}

\begin{figure*}[t]
	\vspace{0mm}
	\captionsetup[subfloat]{captionskip=1pt}
	\centering
	\subfloat[GCNConv-CPU]{%
		\includegraphics[width=0.23\linewidth, trim=0cm 0cm 0cm 0cm, clip]{fig/GCNConv-CPU}
	}
	\hspace{0mm}
	\subfloat[GCNConv-GPU]{%
		\includegraphics[width=0.23\linewidth, trim=0cm 0cm 0cm 0cm, clip]{fig/GCNConv-GPU}
	}
	\hspace{0mm}
	\subfloat[GCN2Conv-CPU]{%
		\includegraphics[width=0.23\linewidth, trim=0cm 0cm 0cm 0cm, clip]{fig/GCN2Conv-CPU}
	}
	\hspace{0mm}
	\subfloat[GCN2Conv-GPU]{%
		\includegraphics[width=0.23\linewidth, trim=0cm 0cm 0cm 0cm, clip]{fig/GCN2Conv-GPU}
	}
	\vspace{-3mm}
	\subfloat[GATConv-CPU]{%
		\includegraphics[width=0.23\linewidth, trim=0cm 0cm 0cm 0cm, clip]{fig/GATConv-CPU}
	}
	\hspace{0mm}
	\subfloat[GATConv-GPU]{%
		\includegraphics[width=0.23\linewidth, trim=0cm 0cm 0cm 0cm, clip]{fig/GATConv-GPU}
	}
	\hspace{0mm}
	\subfloat[GATv2Conv-CPU]{%
		\includegraphics[width=0.23\linewidth, trim=0cm 0cm 0cm 0cm, clip]{fig/GATv2Conv-CPU}
	}
	\hspace{0mm}
	\subfloat[GATv2Conv-GPU]{%
		\includegraphics[width=0.23\linewidth, trim=0cm 0cm 0cm 0cm, clip]{fig/GATv2Conv-GPU}
	}
	\vspace{-3mm}
	\subfloat[SAGEConv-CPU]{%
		\includegraphics[width=0.23\linewidth, trim=0cm 0cm 0cm 0cm, clip]{fig/SAGEConv-CPU}
	}
	\hspace{0mm}
	\subfloat[SAGEConv-GPU]{%
		\includegraphics[width=0.23\linewidth, trim=0cm 0cm 0cm 0cm, clip]{fig/SAGEConv-GPU}
	}
	\hspace{0mm}
	\centering
	\subfloat[ChebConv-CPU]{%
		\includegraphics[width=0.23\linewidth, trim=0cm 0cm 0cm 0cm, clip]{fig/ChebConv-CPU}
	}
	\hspace{0mm}
	\subfloat[ChebConv-GPU]{%
		\includegraphics[width=0.23\linewidth, trim=0cm 0cm 0cm 0cm, clip]{fig/ChebConv-GPU}
	}
	\vspace{-3mm}
	\centering
	\subfloat[TAGConv-CPU]{%
		\includegraphics[width=0.23\linewidth, trim=0cm 0cm 0cm 0cm, clip]{fig/TAGConv-CPU}
	}
	\hspace{0mm}
	\subfloat[TAGConv-GPU]{%
		\includegraphics[width=0.23\linewidth, trim=0cm 0cm 0cm 0cm, clip]{fig/TAGConv-GPU}
	}
	\hspace{0mm}
	\subfloat[SGConv-CPU]{%
		\includegraphics[width=0.23\linewidth, trim=0cm 0cm 0cm 0cm, clip]{fig/SGConv-CPU}
	}
	\hspace{0mm}
	\subfloat[SGConv-GPU]{%
		\includegraphics[width=0.23\linewidth, trim=0cm 0cm 0cm 0cm, clip]{fig/SGConv-GPU}
	}
	\vspace{0mm}
	\caption{Runtime of eight Conv layers. Note that the range of y-axis is different for CPU and GPU cases.}
	\label{fig:Conv}
	\vspace{-3mm}
\end{figure*}

\vspace{1mm}
\noindent \textbf{Sampler.} We then compare the performance of three different graph samplers provided by DGL and PyG, namely neighborhood sampler in GraphSAGE~\cite{hamilton2017inductive}, graph clustering-based sampler in ClusterGCN~\cite{chiang2019cluster}, and random walk-based sampler in GraphSAINT~\cite{zeng2019graphsaint}. 

For GraphSAGE sampler, we follow the settings in~\cite{hamilton2017inductive}, which sample 25 and 10 neighbors per node in its first-hop and second-hop neighborhoods, respectively, with a batch size of 512. Note that each mini-batch is composed of 512 subgraphs. For ClusterCGN sampler, there are two steps, which are (1) graph partitioning with METIS algorithm and (2) cluster aggregation. The former partitions the input graph into a given number of small clusters with METIS algorithm, while the latter is to randomly select a few of them to form a subgraph for a training batch. Note that the former is done only once, but the latter is repeated to obtain different mini-batches. In this experiment, we partition the input graph into 2000 clusters and combine 50 of them for each mini-batch. For GraphSAINT sampler, we use the random walk sampling method with 3000 roots and a walk length of two steps to construct subgraphs from the input graph for mini-batch training. While there are two other sampling methods, namely node sampling and edge sampling, in GraphSAINT, we here do not consider them as they are shown to be inferior to the random walk sampling~\cite{zeng2019graphsaint}. We measure the runtime of each sampler for one training epoch, i.e., one pass over the entire graph, and report the results in Figure~\ref{fig:sampler}.

\vspace{1mm}

\noindent \textbf{Observation 2:} \textit{All three samplers provided by DGL are more efficient than the ones in PyG. The performance gap is relatively small for GraphSAINT sampler since it is computationally cheaper than the other two samplers.}

\vspace{1mm}

We observe that DGL implements its samplers in C++ with OpenMP, thus leading to superior performance to the ones of PyG, which are developed in Python. In addition, although the choices of hyperparameters can affect the sampling performance, GraphSAINT sampler is generally faster than GraphSAGE's neighborhood sampler and ClusterGCN sampler. It is also worth noting that the neighborhood sampler can lead to a very large computational graph for each node, while the ClusterGCN sampler can lead to information loss and data imbalance. Thus, we expect that the GraphSAINT sampler is a preferable choice in practice. Furthermore, we observe that PyG requires data format conversion to the compressed sparse column (CSC) format, e.g., if it was in the compressed sparse row (CSR) format, which turns out to be quite slow on large datasets. Finally, while all three samplers in both DGL and PyG run on CPU, DGL also provides GPU support and CUDA-Unified Virtual Addressing (UVA) support for GraphSAGE, but not for other GNN models. We shall discuss them in Section~\ref{sec:case-study}.

\vspace{0mm}
\noindent \textbf{Graph convolutional layer.} A convolutional (Conv) layer is a key and dominant component of GNNs, and its runtime performance can often reflect the overall performance. We thus conduct functional testing on a collection of Conv layers available in DGL and PyG. Both frameworks provide an `nn' module that contains the implementations of popular Conv layers.  We notice that PyG covers more than 50 Conv layers and DGL has about 30 of them. We here select eight commonly used Conv layers for functional testing. They are GCNConv~\cite{kipf2016semi}, GCN2Conv~\cite{chen2020simple}, ChebConv~\cite{defferrard2016convolutional}, SAGEConv~\cite{hamilton2017inductive}, GATConv~\cite{velivckovic2017graph}, GATv2Conv~\cite{brody2021attentive}, TAGConv~\cite{du2017topology}, and SGConv~\cite{wu2019simplifying}.


\begin{figure}[t]
	\captionsetup[subfloat]{captionskip=1pt}
	\centering
	\subfloat[DGL-CPU]{%
		\includegraphics[width=0.47\linewidth, trim=0cm 0cm 0cm 0cm, clip]{fig/graphsage-breakdown-DGL-CPU}
	}
	\vspace{0mm}
	\subfloat[DGL-CPUGPU]{%
		\includegraphics[width=0.47\linewidth, trim=0cm 0cm 0cm 0cm, clip]{fig/graphsage-breakdown-DGL-CPUGPU}
	}
	\hspace{1mm}
	\subfloat[PyG-CPU]{%
		\includegraphics[width=0.47\linewidth, trim=0cm 0cm 0cm 0cm, clip]{fig/graphsage-breakdown-PyG-CPU}
	}
	\vspace{0mm}
	\subfloat[PyG-CPUGPU]{%
		\includegraphics[width=0.47\linewidth, trim=0cm 0cm 0cm 0cm, clip]{fig/graphsage-breakdown-PyG-CPUGPU}
	}
	\vspace{0mm}
	\caption{Runtime breakdown of GraphSAGE.}
	\label{fig:sage-breakdown}
	\vspace{-2mm}
\end{figure}


We measure the runtime of executing each Conv layer on CPU and GPU. In other words, the reported runtime is equivalent to the time of running \emph{one forward propagation} over a single Conv layer with the entire input graph. We manually set the hyperparameters to be the same across the frameworks for each Conv layer. The output dimension is fixed to be 256 for all test cases. The results are presented in Figure~\ref{fig:Conv}.

\vspace{1mm}

\noindent \textbf{Observation 3:} \textit{All eight Conv layers in DGL run faster than the ones of PyG on CPU. The ones in DGL also run faster than their PyG counterparts on GPU in most cases, while PyG only outperforms DGL for few cases with small graphs. Furthermore, graph convolutional operations on GPU show up to 70x speedup over them on CPU.}

\vspace{1mm}


The main reason for the performance on CPU is that DGL adopts an improved CPU message passing kernel developed by~\cite{md2021distgnn} to boost the performance, while PyG relies on the CPU kernels included in its own PyTorch Sparse and PyTorch Scatter, where some `scatter' operations are not well optimized on CPU. As for the performance on GPU, it is worth noting that our observation does not conflict but match with the observation in \cite{wu2021performance}, which shows that PyG is more efficient than DGL, yet for small graphs. Our observation also confirms the claim in~\cite{wang2019deep}. Although DGL is a bit slower on small graphs due to its framework overhead, it is generally more efficient than PyG, especially on large graphs, thanks to its highly tuned kernels. We also find that SAGEConv is relatively computationally cheaper than the other Conv layers, due to its simple aggregation operation.


In addition, we observe that both frameworks provide \emph{fused} kernels to improve their efficiency and scalability, where two separate message-passing and aggregation operations are merged as a single message aggregation operation. DGL uses `g.update\_all()' function to invoke its g-SpMM and g-SDDMM kernels, while PyG simply calls `matmul()' function in PyTorch Sparse. It is worth noting that PyG does not provide such fused kernel support for ChebConv, GATConv, and GATv2Conv layers. As a result, all three layers of PyG suffer from an out-of-memory issue on large graphs.

\begin{figure}[t]
	\captionsetup[subfloat]{captionskip=1pt}
	\centering
	\subfloat{%
		\includegraphics[width=0.47\linewidth, trim=0cm 0cm 0cm 0cm, clip]{fig/graphsage-total-1}
	}
	\hspace{0mm}
	\subfloat{%
		\includegraphics[width=0.47\linewidth, trim=0cm 0cm 0cm 0cm, clip]{fig/graphsage-total-2}
	}
	\vspace{-2mm}
	\caption{Total runtime of GraphSAGE.}
	\label{fig:sage-runtime}
	\vspace{-3mm}
\end{figure}

\begin{figure}[t!]
	\captionsetup[subfloat]{captionskip=1pt}
	\centering
	\subfloat{%
		\includegraphics[width=0.47\linewidth, trim=0cm 0cm 0cm 0cm, clip]{fig/graphsage-power-1}
	}
	\vspace{0mm}
	\subfloat{%
		\includegraphics[width=0.47\linewidth, trim=0cm 0cm 0cm 0cm, clip]{fig/graphsage-power-2}
	}
	\vspace{-2mm}
	\caption{Average power consumption of GraphSAGE.}
	\label{fig:sage-power}
	\vspace{-2mm}
\end{figure}


\begin{figure}[t!]
	\captionsetup[subfloat]{captionskip=1pt}
	\centering
	\subfloat{%
		\includegraphics[width=0.47\linewidth, trim=0cm 0cm 0cm 0cm, clip]{fig/graphsage-energy-1}
	}
	\vspace{0mm}
	\subfloat{%
		\includegraphics[width=0.47\linewidth, trim=0cm 0cm 0cm 0cm, clip]{fig/graphsage-energy-2}
	}
	\vspace{-2mm}
	\caption{Energy consumption of GraphSAGE.}
	\label{fig:sage-energy}
	\vspace{-3mm}
\end{figure}


\subsection{Performance Evaluation of GNNs}\label{sec:performance}

We evaluate three representative sampling-based GNNs, namely GraphSAGE, ClusterGCN, and GraphSAINT on CPU and GPU separately. We use `DGL-CPU' and `PyG-CPU' to indicate when both sampling and training are done on CPU and use `DGL-CPUGPU' and `PyG-CPUGPU' to indicate when sampling is done on CPU while training is done on GPU. We present their runtime breakdown, total runtime, average power consumption, and energy consumption in Figures~\ref{fig:sage-breakdown}--\ref{fig:saint-energy}. Note that, for all three GNNs, we use the same hyperparameters of their samplers as used in the above functional testing. We use two convolutional layers for all three models and the hyperparameters of each GNN model are set to be the same across DGL and PyG for a fair comparison. The reported results are based on the models trained by 10 epochs. We repeated the same experiments multiple times and observed more or less the same results.  

\begin{figure}[t!]
	\captionsetup[subfloat]{captionskip=1pt}
	\centering
	\subfloat[DGL-CPU]{%
		\includegraphics[width=0.47\linewidth, trim=0cm 0cm 0cm 0cm, clip]{fig/clustergcn-breakdown-DGL-CPU}
	}
	\hspace{0mm}
	\subfloat[DGL-CPUGPU]{%
		\includegraphics[width=0.47\linewidth, trim=0cm 0cm 0cm 0cm, clip]{fig/clustergcn-breakdown-DGL-CPUGPU}
	}
	\vspace{-2mm}
	\subfloat[PyG-CPU]{%
		\includegraphics[width=0.47\linewidth, trim=0cm 0cm 0cm 0cm, clip]{fig/clustergcn-breakdown-PyG-CPU}
	}
	\vspace{0mm}
	\subfloat[PyG-CPUGPU]{%
		\includegraphics[width=0.47\linewidth, trim=0cm 0cm 0cm 0cm, clip]{fig/clustergcn-breakdown-PyG-CPUGPU}
	}
	\vspace{0mm}
	\caption{Runtime breakdown of ClusterGCN.}
	\label{fig:cluster-breakdown}
	\vspace{-4mm}
\end{figure}


\begin{figure}[t!]
	\captionsetup[subfloat]{captionskip=1pt}
	\centering
	\subfloat{%
		\includegraphics[width=0.47\linewidth, trim=0cm 0cm 0cm 0cm, clip]{fig/clustergcn-total-1}
	}
	\hspace{0mm}
	\subfloat{%
		\includegraphics[width=0.47\linewidth, trim=0cm 0cm 0cm 0cm, clip]{fig/clustergcn-total-2}
	}
	\vspace{0mm}
	\caption{Total runtime of ClusterGCN.}
	\label{fig:clustergcn-runtime}
	\vspace{-4mm}
\end{figure}


\begin{figure}[t!]
	\captionsetup[subfloat]{captionskip=1pt}
	\centering
	\subfloat{%
		\includegraphics[width=0.47\linewidth, trim=0cm 0cm 0cm 0cm, clip]{fig/clustergcn-power-1}
	}
	\vspace{0mm}
	\subfloat{%
		\includegraphics[width=0.47\linewidth, trim=0cm 0cm 0cm 0cm, clip]{fig/clustergcn-power-2}
	}
	\vspace{0mm}
	\caption{Average power consumption of ClusterGCN.}
	\label{fig:cluster-power}
	\vspace{-4mm}
\end{figure}


\begin{figure}[t!]
	\captionsetup[subfloat]{captionskip=1pt}
	\centering
	\subfloat{%
		\includegraphics[width=0.47\linewidth, trim=0cm 0cm 0cm 0cm, clip]{fig/clustergcn-energy-1}
	}
	\vspace{0mm}
	\subfloat{%
		\includegraphics[width=0.47\linewidth, trim=0cm 0cm 0cm 0cm, clip]{fig/clustergcn-energy-2}
	}
	\vspace{0mm}
	\caption{Energy consumption of ClusterGCN.}
	\label{fig:cluster-energy}
	\vspace{-4mm}
\end{figure}


\begin{figure}[t!]
	\captionsetup[subfloat]{captionskip=1pt}
	\centering
	\subfloat[DGL-CPU]{%
		\includegraphics[width=0.47\linewidth, trim=0cm 0cm 0cm 0cm, clip]{fig/graphsaint-breakdown-DGL-CPU}
	}
	\hspace{0mm}
	\subfloat[DGL-CPUGPU]{%
		\includegraphics[width=0.47\linewidth, trim=0cm 0cm 0cm 0cm, clip]{fig/graphsaint-breakdown-DGL-CPUGPU}
	}
	\vspace{-2mm}
	\subfloat[PyG-CPU]{%
		\includegraphics[width=0.47\linewidth, trim=0cm 0cm 0cm 0cm, clip]{fig/graphsaint-breakdown-PyG-CPU}
	}
	\vspace{0mm}
	\subfloat[PyG-CPUGPU]{%
		\includegraphics[width=0.47\linewidth, trim=0cm 0cm 0cm 0cm, clip]{fig/graphsaint-breakdown-PyG-CPUGPU}
	}
	\vspace{0mm}
	\caption{Runtime breakdown of GraphSAINT.}
	\label{fig:saint-breakdown}
	\vspace{-3.5mm}
\end{figure}


\begin{figure}[t!]
	\captionsetup[subfloat]{captionskip=1pt}
	\centering
	\subfloat{%
		\includegraphics[width=0.47\linewidth, trim=0cm 0cm 0cm 0cm, clip]{fig/graphsaint-total-1}
	}
	\hspace{0mm}
	\subfloat{%
		\includegraphics[width=0.47\linewidth, trim=0cm 0cm 0cm 0cm, clip]{fig/graphsaint-total-2}
	}
	\vspace{0mm}
	\caption{Total runtime of GraphSAINT.}
	\label{fig:saint-runtime}
	\vspace{-4mm}
\end{figure}



\begin{figure}[t!]
	\captionsetup[subfloat]{captionskip=1pt}
	\centering
	\subfloat{%
		\includegraphics[width=0.47\linewidth, trim=0cm 0cm 0cm 0cm, clip]{fig/graphsaint-power-1}
	}
	\vspace{0mm}
	\subfloat{%
		\includegraphics[width=0.47\linewidth, trim=0cm 0cm 0cm 0cm, clip]{fig/graphsaint-power-2}
	}
	\vspace{0mm}
	\caption{Average power consumption of GraphSAINT.}
	\label{fig:saint-power}
	\vspace{-4mm}
\end{figure}


\begin{figure}[t!]
	\captionsetup[subfloat]{captionskip=1pt}
	\centering
	\subfloat{%
		\includegraphics[width=0.47\linewidth, trim=0cm 0cm 0cm 0cm, clip]{fig/graphsaint-energy-1}
	}
	\vspace{0mm}
	\subfloat{%
		\includegraphics[width=0.47\linewidth, trim=0cm 0cm 0cm 0cm, clip]{fig/graphsaint-energy-2}
	}
	\vspace{0mm}
	\caption{Energy consumption of GraphSAINT.}
	\label{fig:saint-energy}
	\vspace{-4mm}
\end{figure}


As shown in Figure~\ref{fig:sage-breakdown}, Figure~\ref{fig:cluster-breakdown}, and Figure~\ref{fig:saint-breakdown}, we break the runtime of each GNN into four parts, which are data loading, sampling, data movement, and model training. Data loading is done by `data loader' to load the input graph and its associated node features from storage to CPU memory. Sampling is done by `sampler' to extract subgraphs and fetch the node features of the sampled subgraphs from the entire feature matrix for mini-batch training. Data movement is to copy the initial weight matrices of a GNN model, each subgraph matrix, and its corresponding node features from CPU to GPU. Note that there is no data movement (from CPU to GPU) for DGL-CPU and PyG-CPU. Model training includes forward propagation, backward propagation, and update of model weights. Note that as the number of training epochs increases, the fraction of data loading in total runtime will decrease since it is a one-time operation. However, sampling, data movement, and model training are performed repeatedly for different mini-batches.


\vspace{1mm}

\noindent \textbf{Observation 4:} \textit{Sampling is slow for all three GNNs and can take up to 90\% of total runtime.}

\vspace{1mm}

This observation indicates that there is a need to optimize sampling and its associated operations. In particular, for PyG, its CPU kernel could be improved for not only sampling but also model training on CPU. In addition, we observe that data movement can also take a large portion of total runtime in both frameworks. As shall be shown in Section~\ref{sec:case-study}, data pre-loading in the frameworks can be used to mitigate this issue.


\vspace{1mm}

\noindent \textbf{Observation 5:} \textit{DGL is generally more efficient than PyG on both CPU and GPU in terms of runtime and energy consumption, especially for large graphs.}

\vspace{1mm}

We observe that PyG is more efficient than DGL for small graphs when CPU is used for sampling and GPU is used for training, while DGL is generally more efficient for the other cases. In particular, PyG-CPUGPU is generally more efficient than DGL-CPUGPU for GraphSAINT. This behavior can be explained as follows. With mini-batch training, a GNN model is trained based on sampled subgraphs, which are much smaller than the input graph. We observe that each sampled subgraph (corresponding to a mini-batch) by GraphSAINT sampler is relatively smaller than the ones by GraphSAGE's neighborhood sampler and ClusterGCN sampler. Here the one with GraphSAGE's neighborhood sampler has multiple subgraphs for a mini-batch. Also, recall that the performance gap of the GraphSAINT sampler between DGL and PyG is insignificant, as shown in Figure~\ref{fig:sampler}. Since PyG is more efficient in model training with small graphs, PyG becomes more efficient even for medium-size graphs with GraphSAINT, as shown in Figure~\ref{fig:saint-runtime}.

In addition, we find that there is no clear winner between DGL and PyG regarding average power consumption, which indicates that energy consumption mainly depends on overall runtime. We observe that GraphSAINT is more efficient in both runtime and energy consumption compared with the GraphSAGE and ClusterGCN, thanks to its light-weight sampling and GNN operations. Note that they are trained for the same number of epochs in our experiments. Nonetheless, we emphasize that different choices of the hyperparameters for each GNN in optimizing its accuracy would affect the efficiency in runtime and energy consumption differently.

\subsection{Case Studies}\label{sec:case-study}

We next turn out attention to three case studies to further evaluate the performance of two GNN frameworks, regarding data pre-loading, GPU-based sampler, and full-batch model training. We focus on GraphSAGE for the case studies.

\vspace{1mm}
\noindent \textbf{Pre-loading entire graph and node features into GPU.} As shown in Section~\ref{sec:performance}, data movement can be a problem when we use CPU for sampling and GPU for model training. We here change the implementation strategy so as to pre-load the entire graph and its associated node features into the GPU \emph{upfront}, which can avoid the overhead of repeated data movement, i.e., the movement of the features of nodes chosen in each mini-batch. Both frameworks provide such a pre-loading option. With this option, the adjacency matrices of sampled subgraphs only need to be copied from CPU to GPU for each mini-batch periodically. Note that a mini-batch is composed of a number of sampled subgraphs in GraphSAGE, where the number of sampled subgraphs is the mini-batch size. We present the resulting performance of DGL and PyG with GraphSAGE in Figure~\ref{fig:breakdown-preloading} for runtime breakdown and in Figure~\ref{fig:speedup-preloading} for speedup results.

\vspace{1mm}

\noindent \textbf{Observation 6:} \textit{The data pre-loading can significantly reduce overall data movement time in both frameworks.}

\vspace{1mm}

As expected, the pre-loading strategy saves up to 20x data movement time, thereby leading to about 2x overall speedup. Nonetheless, \emph{it is only feasible when the GPU memory is large enough to hold the entire graph and its associated node features as well as the weight matrix of a GNN model.} It is often not the case in practice, especially for large graphs. An alternative yet effective strategy would be to cache the features of nodes that are most frequently used for model training, i.e., partial information of the graph, into GPU memory upfront, to reduce overall data movement time~\cite{dong2021global}.

It is worth noting that DGL further provides an advanced feature called `pre-fetching' for asynchronous data movement and model computation. We observed that the performance of DGL can be further improved, albeit a little bit, with this feature. We omit the results here for brevity.


\begin{figure}[t!]
	\vspace{-2mm}
	\captionsetup[subfloat]{captionskip=1pt}
	\centering
	\subfloat[Speedup of data movement]{%
		\includegraphics[width=0.47\linewidth, trim=0cm 0cm 0cm 0cm, clip]{fig/CPUGPU-preloading-Speedup-movement}
	}
	\vspace{0mm}
	\subfloat[Speedup of total runtime]{%
		\includegraphics[width=0.47\linewidth, trim=0cm 0cm 0cm 0cm, clip]{fig/CPUGPU-preloading-Speedup}
	}
	\hspace{0mm}
	\caption{Speedup of GraphSAGE when pre-loading the input graph and node features into GPU.}
	\label{fig:speedup-preloading}
	\vspace{-2mm}
\end{figure}

\begin{figure}[t!]
	\captionsetup[subfloat]{captionskip=1pt}
	\centering
	\subfloat[DGL-CPUGPU]{%
		\includegraphics[width=0.47\linewidth, trim=0cm 0cm 0cm 0cm, clip]{fig/graphsage-breakdown-DGL-CPUGPU1}
	}
	\vspace{0mm}
	\subfloat[PyG-CPUGPU]{%
		\includegraphics[width=0.47\linewidth, trim=0cm 0cm 0cm 0cm, clip]{fig/graphsage-breakdown-PyG-CPUGPU1}
	}
	\vspace{0mm}
	\caption{Runtime breakdown of GraphSAGE with data pre-loading.}
	\label{fig:breakdown-preloading}
	\vspace{-2mm}
\end{figure}



\begin{figure*}[t!]
	\captionsetup[subfloat]{captionskip=1pt}
	\centering
	\subfloat[Speedup over DGL-CPUGPU]{%
		\includegraphics[width=0.3\linewidth, trim=0cm 0cm 0cm 0cm, clip]{fig/GPU-Speedup-CPUGPU}
	}
	\hspace{0mm}
	\subfloat[Powerup over DGL-CPUGPU]{%
		\includegraphics[width=0.3\linewidth, trim=0cm 0cm 0cm 0cm, clip]{fig/GPU-Powerup-CPUGPU}
	}
	\hspace{0mm}
	\subfloat[Greenup over DGL-CPUGPU]{%
		\includegraphics[width=0.3\linewidth, trim=0cm 0cm 0cm 0cm, clip]{fig/GPU-Greenup-CPUGPU}
	}
	\vspace{0mm}
	\caption{GPS-UP metrics of GraphSAGE with DGL's GPU-based sampler and UVA-based sampler.}
	\label{fig:uvagpu-speedup}
\end{figure*}

\begin{figure}[t!]
	\vspace{-2mm}
	\captionsetup[subfloat]{captionskip=1pt}
	\centering
	\subfloat[DGL-GPU]{%
		\includegraphics[width=0.47\linewidth, trim=0cm 0cm 0cm 0cm, clip]{fig/graphsage-breakdown-DGL-GPU}
	}
	\vspace{0mm}
	\subfloat[DGL-UVAGPU]{%
		\includegraphics[width=0.47\linewidth, trim=0cm 0cm 0cm 0cm, clip]{fig/graphsage-breakdown-DGL-UVAGPU}
	}
	\vspace{0mm}
	\caption{Runtime breakdown of GraphSAGE with DGL's GPU-based sampler and UVA-based sampler.}
	\label{fig:uvagpu}
\end{figure}

\vspace{1mm}
\noindent \textbf{GPU-based sampler.} As mentioned before, the GPU-based neighborhood sampler, i.e., the sampler in GraphSAGE, is available in DGL to accelerate its sampling operation and eliminate the need of moving sampled subgraphs from CPU to GPU for each mini-batch. If the GPU-based sampler is used together with the pre-loading option, it can also eliminate the repeated data transfer of node features, corresponding to sampled subgraphs for each mini-batch. This combination is, however, infeasible for the cases with large graphs and/or high-dimensional feature data that do not fit into GPU memory.

In addition, DGL supports another sampler for GraphSAGE, which is the CUDA-Unified Virtual Addressing (UVA)-based sampler. It uses GPU to perform the sampling operation on the input graph and node features pinned on CPU memory via zero-copy access. This UVA support allows DGL to deal with much larger graphs with the benefits of using GPU for sampling and model training. Note that both UVA-based sampler and GPU-based sampler are currently only available for GraphSAGE in DGL.

We evaluate the performance of the GPU-based sampler (`DGL-GPU') and UVA-based sampler (`DGL-UVAGPU') to see how much improvement they can achieve. For the former, we also use the data pre-loading option. Their runtime-breakdown results are reported in Figure~\ref{fig:uvagpu}. Here the data movement for DGL-GPU contains two parts, which are (1) copying the input graph and node features to GPU for sampling and (2) moving the initial GNN model from CPU to GPU for training. For DGL-UVAGPU, the data movement is only for the initial model.


\vspace{1mm}

\noindent \textbf{Observation 7:} \textit{The portion of the sampling operation in total runtime becomes smaller compared with the one with DGL-CPUGPU. However, even with GPU for sampling, it can still take up to 40\% of total runtime for DGL-GPU and 60\% for DGL-UVAGPU. This indicates the non-trivial overhead of the sampling operation and the potential benefit of further accelerating the sampler.}

\vspace{1mm}

We next use GPS-UP (Speedup, Greenup, and Powerup) metrics introduced in~\cite{abdulsalam2015using} for further efficiency analysis. The metrics are designed for comparing two different implementations. One of them is an non-optimized version (i.e. baseline) and the other is an optimized version for better performance. Specifically, they are defined as
\begin{equation*}
\setlength{\abovedisplayskip}{5pt}
\setlength{\belowdisplayskip}{5pt}
\text{Speedup} = \frac{T_{\phi}}{T_o}, \quad \text{Greenup} = \frac{E_{\phi}}{E_o},
\end{equation*}
\begin{equation*}
\setlength{\abovedisplayskip}{5pt}
\setlength{\belowdisplayskip}{5pt}
\text{Powerup} = \frac{P_o}{P_{\phi}} = \frac{E_o/T_o}{E_{\phi}/T_{\phi}} = \frac{\text{Speedup}}{\text{Greenup}},
\end{equation*}
where $T_{\phi}$, $E_{\phi}$, and $P_{\phi}$ are the runtime, energy consumption, and average power of the non-optimized version, respectively, and $T_o$, $E_o$, and $P_o$ are the corresponding values of the optimized one, respectively. We here use DGL-CPUGPU as baseline and report Speedup, Greenup, and Powerup results achieved by DGL-GPU and DGL-UVAGPU over DGL-CPUGPU in Figure~\ref{fig:uvagpu-speedup}.


\vspace{1mm}

\noindent \textbf{Observation 8:} \textit{The use of GPU for sampling saves both time and power in most cases, leading to significant energy saving.}

\vspace{1mm}

As can be seen from Figure~\ref{fig:uvagpu-speedup}(a), DGL-GPU achieves up to 5.5x speedup over DGL-CPUGPU. DGL-UVAGPU is sightly slower than DGL-GPU, because the former uses zero-copy access to CPU memory, which is generally slower than having access to GPU onboard memory. From Figure~\ref{fig:uvagpu-speedup}(b), we also observe that Powerup is not always above one, which implies that the power consumption of using GPU for the sampler can be higher than the CPU counterpart. It happens, especially when there are a large number of edges for each node, e.g., the case of Reddit, making the sampling computation on GPU heavier. Nonetheless, as shown in Figure~\ref{fig:uvagpu-speedup}(c), we observe that Greenup is always above one. In other words, it is \emph{more energy-efficient} using GPU for the sampler. While GPU can consume more power than CPU for the sampling operation, it significantly reduces the total runtime, which translates into smaller overall energy consumption.

Our observations indicate the benefits of using GPU for the sampler of GNNs. Nonetheless, this GPU support is currently only limited to GraphSAGE in DGL, and there is no such support in PyG. Note that there is a recent study~\cite{jangda2021accelerating} that leverages GPUs to accelerate graph sampling for GNNs.


\begin{figure}[t]
	\vspace{-2mm}
	\captionsetup[subfloat]{captionskip=1pt}
	\centering
	\subfloat{%
		\includegraphics[width=0.47\linewidth, trim=0cm 0cm 0cm 0cm, clip]{fig/fullbatch-training-1}
	}
	\hspace{0mm}
	\subfloat{%
		\includegraphics[width=0.47\linewidth, trim=0cm 0cm 0cm 0cm, clip]{fig/fullbatch-training-2}
	}
	\vspace{-1mm}
	\caption{One epoch training time of full-batch GraphSAGE.}
	\label{fig:fullbatch}
	\vspace{-3mm}
\end{figure}

\begin{figure}[t!]
	\captionsetup[subfloat]{captionskip=1pt}
	\centering
	\subfloat{%
		\includegraphics[width=0.47\linewidth, trim=0cm 0cm 0cm 0cm, clip]{fig/fullbatch-power-1}
	}
	\hspace{0mm}
	\subfloat{%
		\includegraphics[width=0.47\linewidth, trim=0cm 0cm 0cm 0cm, clip]{fig/fullbatch-power-2}
	}
	\vspace{-1mm}
	\caption{Average power consumption of full-batch GraphSAGE while training.}
	\label{fig:fullbatch-power}
	\vspace{-3mm}
\end{figure}

\begin{figure}[t!]
	\captionsetup[subfloat]{captionskip=1pt}
	\centering
	\subfloat{%
		\includegraphics[width=0.47\linewidth, trim=0cm 0cm 0cm 0cm, clip]{fig/fullbatch-energy-1}
	}
	\hspace{0mm}
	\subfloat{%
		\includegraphics[width=0.47\linewidth, trim=0cm 0cm 0cm 0cm, clip]{fig/fullbatch-energy-2}
	}
	\vspace{-1mm}
	\caption{One epoch energy consumption of full-batch GraphSAGE.}
	\label{fig:fullbatch-energy}
	\vspace{-3mm}
\end{figure}


\vspace{1mm}
\noindent \textbf{Full-batch training.} We have focused on three sampling-based GNNs with mini-batch training to evaluate the performance of the frameworks. For a comprehensive evaluation, we here consider \emph{full-batch} training to train a GraphSAGE model, which is done based on the entire graph \emph{without} neighborhood sampling. Specifically, we use a GraphSAGE model with two layers having mean-aggregator and train the model on CPU and GPU using DGL and PyG separately. We present the experiment results, which are the average results of 100 runs for one training epoch, in runtime, power consumption, and energy consumption in Figures~\ref{fig:fullbatch}--\ref{fig:fullbatch-energy}, respectively.

We observe that DGL-CPU is much faster than PyG-CPU for full-bath model training. DGL-GPU training is slower than its PyG counterpart on the smallest graph PPI, while it is faster for the other five datasets. The results are consistent with our functional test results as reported above. We also observe that there is no clear difference in the average power consumption between the frameworks for model training. That is, the differences in energy consumption between the frameworks mainly come from their differences in training time.

\section{Conclusion}
In this paper, we extend the idea of SynGEC \cite{zhang2022syngec} and propose the CSynGEC approach to enhance GEC models by exploiting tailored constituent-based syntax. Experimental results show that incorporating constituent-based syntax produced by a GEC-oriented constituency parser can effectively help GEC models. 
Furthermore, we attempt to combine dependency-based and constituent-based syntax from both intra-model and inter-model aspects, and find that simultaneously using two kinds of syntax leads to more obvious improvement.



\section*{Acknowledgments}
This work was supported in part by a grant from SK hynix America and an equipment gift from NVIDIA. This work was also supported in part by the National Science Foundation under Grant IIS-2209921.



\bibliographystyle{IEEEtranS}
\bibliography{ref}


\end{document} 