% LaTeX rebuttal letter example. 
% 
% Copyright 2019 Friedemann Zenke, fzenke.net
%
% Based on examples by Dirk Eddelbuettel, Fran and others from 
% https://tex.stackexchange.com/questions/2317/latex-style-or-macro-for-detailed-response-to-referee-report
% 
% Licensed under cc by-sa 3.0 with attribution required.

% TODO: add partial least squares comment


\documentclass[10pt]{article}
\usepackage[utf8]{inputenc}
\usepackage{lipsum} % to generate some filler text
\usepackage{fullpage}
\usepackage{amsfonts}
\usepackage{amsmath}
\usepackage{hyperref}
\usepackage{amsthm}
\usepackage{amssymb}
\usepackage{bm}
\usepackage{xcolor}
\usepackage{booktabs}
\usepackage{subfig}
\usepackage[demo,export]{adjustbox}
\usepackage{mathtools}
\usepackage{multirow}
\usepackage[normalem]{ulem}
\newtheorem*{cor113}{Corollary 1.12 in \cite{wei2017upper}}

\theoremstyle{plain}
\newtheorem{theorem}{Theorem}

\linespread{0.96}

%Useful commands
\newcommand{\PCA}{\mathrm{PCA}}
\newcommand{\OLS}{\mathrm{OLS}}
\newcommand{\norm}[2][]{{\left\Vert #2 \right\Vert}_{#1}}
\newcommand{\tr}[1]{#1^{\top}}
% create operators and other math
\newcommand{\T}{^{\mathsf{T}}}
\newcommand{\inv}{^{-1}}
\newcommand{\E}[1]{\mathrm{E}\left[{#1}\right]}
\newcommand{\Var}[1]{\mathrm{Var}\left[{#1}\right]}
\newcommand{\given}{\,|\,}
\DeclareMathOperator*{\argmin}{argmin}
\newcommand{\Ker}{\operatorname{Ker}}
\newcommand{\Ima}{\operatorname{Ima}}

\usepackage{comment}
% Number sets
\newcommand{\N}{\mathbb{N}}
\newcommand{\Z}{\mathbb{Z}}
\newcommand{\Q}{\mathbb{Q}}
\newcommand{\R}{\mathbb{R}}	
\newcommand{\C}{\mathbb{C}}
\usepackage{authblk}
% import Eq and Section references from the main manuscript where needed
% \usepackage{xr}
% \externaldocument{manuscript}

% package needed for optional arguments
\usepackage{xifthen}
% define counters for reviewers and their points
\newcounter{reviewer}
\setcounter{reviewer}{0}
\newcounter{point}[reviewer]
\setcounter{point}{0}

% This refines the format of how the reviewer/point reference will appear.
\renewcommand{\thepoint}{P\,\thereviewer.\arabic{point}} 

% command declarations for reviewer points and our responses
\newcommand{\reviewersection}{\stepcounter{reviewer} \bigskip \hrule
                  \section*{Reviewer \thereviewer}}
\newcommand{\editorsection}{\stepcounter{Editor} \bigskip \hrule
                  \section*{Editor Remarks}}
                  
\newenvironment{point}
   {\refstepcounter{point} \bigskip \noindent {\textbf{Reviewer~Point~\thepoint} } ---\ }
   {\par }

\newcommand{\shortpoint}[1]{\refstepcounter{point}  \bigskip \noindent 
	{\textbf{Reviewer~Point~\thepoint} } ---~#1\par }

\newenvironment{reply}
   {\medskip \noindent \begin{sf}\textbf{Reply}:\  }
   {\medskip \end{sf}}

\newcommand{\shortreply}[2][]{\medskip \noindent \begin{sf}\textbf{Reply}:\  #2
	\ifthenelse{\equal{#1}{}}{}{ \hfill \footnotesize (#1)}%
	\medskip \end{sf}}

%bibilography
\usepackage[style=ieee,uniquelist=false,backend=biber,date=year,firstinits=true,doi=false,isbn=false,url=false,eprint=false]{biblatex}
\DeclareNameAlias{sortname}{last-first}
\addbibresource{references.bib}
\renewcommand*{\bibfont}{\normalfont\normalsize\sffamily}

\begin{document}

\section*{\normalsize Response to the reviewers}
% General intro text goes here
We thank the reviewers for their critical assessment of our work. We appreciate all the comments and constructive feedback that helped us improve the paper.  
% In this revision we introduced the following changes:
% % \begin{refsection}
%  \begin{itemize}
%      \item 
% \end{itemize}




% Let's start point-by-point with Reviewer 1
%\reviewersection


\addtocounter{reviewer}{1}


\begin{point}
Typo: At the beginning of Section 2, the community assignment matrix Y should take binary values.
\end{point}

\begin{reply}
We remarked in the manuscript that the community assignment can be overlapping, namely the assignment matrix can have entries in between $[0,1]$ (with row sum equal to $1$). We then focus on non-overlapping community detection problem where the assignment matrix is binary.
%This is a very interesting point. We changed our wording to reflect that we don't prove in general that PCA-OLS outperforms PCR. %However, PCA-OLS has the advantage that it can be performed in practice on real data, whereas PCR requires an oracle.



%TERESA: TEST HIS THEORY WITH NUMERICAL EXPERIMENTS
%NOTE THAT VARIANCE IS SMALLER ALWAYS FOR PCA (INTERLACING for $k<n$; Limiting distribution for $k=n$.) BUT THE BIAS WE DON'T AND IT DEPENDS ON HOW THE DATA IS GENERATED



\end{reply}



%\reviewersection
\addtocounter{reviewer}{1}


\begin{point}
Novelty: The work presents a random graph model that can generate graphs with variable sparsity. Then the authors use it to prove that spectral embeddings degrade with graph sparsity and to explain why GNNs perform consistently well in sparse graphs. Therefore, this reviewer considers the level of novelty as modest.
\end{point}

\begin{reply}
Thank you for this comment. We point out that, while the shortcomings of SE in sparse graphs are well-known in the community detection literature, and that a number of works have observed the advantages of using GNNs for community detection empirically, to the best of our knowledge our paper is the first to attempt to provide a theoretical explanation for why GNNs outperform SE in sparse graphs. Additionally, we provide numerical validation of our claims on both synthetic and real-word data. We believe that these contributions are well aligned with what is expected of an ICASSP paper; however, we will be happy to incorporate any specific suggestions the Reviewer has to improve the novelty of our paper. 
\end{reply}

\begin{point}
Clarify: The use of English and the correctness of the text are ok but the presentation is often confusing by introducing theorems and by diverting to other related subjects. The paper would benefit from revision.
\end{point}

\begin{reply}
Thank you. We will extensively proofread the paper and improve its organization for the camera-ready.
\end{reply}


%\reviewersection
\addtocounter{reviewer}{2}

\begin{point} 
Could the authors also compare GNNs and SEs in terms of their computational complexity?
\end{point}

\begin{reply}
We have added the comparison of their computational complexity. Specifically:

\noindent Given a graph with $n$ nodes and $|\mathcal{E}|$ edges, a $L$-layer GNN with $K$ convolution coefficients have computational complexity $\mathcal{O}(L K |\mathcal{E}|)$, whereas an order-$K$ spectral embedding requires $\mathcal{O}(K |\mathcal{E}|)$ complexity using variants of Lanczos algorithm. %On the other hand, spectral embedding can suffer from numeric instability for large $K$ (e.g., when the smaller eigenvalues are close to each other).
%time complexity
%SE general: O(n^3) to compute eigen-decomposition
%SE with small K: O(K n^2) using variants of Lanczos algorithm (ARPACK)
%SE with sparse graph with small K: O (K |E|) (i.e., reduce dense n^2 to |E| nonzero entries)
%refs: https://arxiv.org/pdf/2103.10040.pdf
%refs: https://docs.scipy.org/doc/scipy/reference/generated/scipy.sparse.linalg.eigsh.html#scipy.sparse.linalg.eigsh
%refs: https://en.wikipedia.org/wiki/Lanczos_algorithm#Variations

%MPNNN: O( n |E|)

\end{reply}

\begin{point}
We suspect that the operation in Equation (2) is Hadamard product. It should be expressed within text.
\end{point}

\shortreply{We added the explanation in the text that $\circ$ in Equation (2) denotes function composition.}

\begin{point}
Minor changes:
\begin{itemize}
    \item The line under author names is unnecessary and hence could be deleted. 
    \item It is unnecessary to give the abbreviation SE just above Equation (4), since it is already defined in the second paragraph of Introduction. %[Fixed]
    \item In References section, some journal names start with small case letters (e.g.; [7], [18], [28], [29]). Those should be made upper case. %[Fixed]
   \item  In reference [16], "Euclidean" should start with the capital letter, "E". %[Fixed]
    \item There is not enough information about reference [26]. Could it be possible to give more information about it? %[Fixed]
\end{itemize}

\end{point}

\shortreply{We have fixed all above per the Reviewer's suggestion.}
%


\end{document}