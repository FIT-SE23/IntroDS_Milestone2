\documentclass[lettersize,journal]{IEEEtran}
\usepackage{amsmath,amsfonts}
\usepackage{algorithmic}
\usepackage{algorithm}
\usepackage{array}
\usepackage[caption=false,font=normalsize,labelfont=sf,textfont=sf]{subfig}
\usepackage{textcomp}
\usepackage{stfloats}
\usepackage{url}
\usepackage{verbatim}
\usepackage{graphicx}
\usepackage{cite}
\usepackage{multicol}
\usepackage{xcolor}
\hyphenation{net-works}
\usepackage{colortbl}
\usepackage{hyperref}
\hypersetup{colorlinks, citecolor=green, filecolor=pink, linkcolor=blue, urlcolor=blue }

\begin{document}

\title{Towards Green Metaverse Networking: Technologies, Advancements and Future Directions}
\author{Siyue Zhang, Wei Yang Bryan Lim, Wei Chong Ng, Zehui Xiong, Dusit Niyato,~\IEEEmembership{IEEE Fellow}, \\ Xuemin Sherman Shen,~\IEEEmembership{IEEE Fellow}, Chunyan Miao}

\maketitle

\begin{abstract}

As the Metaverse is iteratively being defined, its potential to unleash the next wave of digital disruption and create real-life value becomes increasingly clear. With distinctive features of immersive experience, simultaneous interactivity, and user agency, the Metaverse has the capability to transform all walks of life. However, the enabling technologies of the Metaverse, i.e., digital twin, artificial intelligence, blockchain, and extended reality, are known to be energy-hungry, therefore raising concerns about the sustainability of its large-scale deployment and development. This article proposes Green Metaverse Networking for the first time to optimize energy efficiencies of all network components for Metaverse sustainable development. We first analyze energy consumption, efficiency, and sustainability of energy-intensive technologies in the Metaverse. Next, focusing on computation and networking, we present major advancements related to energy efficiency and their integration into the Metaverse. A case study of energy conservation by incorporating semantic communication and stochastic resource allocation in the Metaverse is presented. Finally, we outline the critical challenges of Metaverse sustainable development, thereby indicating potential directions of future research towards the green Metaverse. 

\end{abstract}

\begin{IEEEkeywords}
Metaverse, Energy Efficiency, Sustainability, Digital Twin, Artificial Intelligence, Blockchain, Extended Reality 
\end{IEEEkeywords}

\section{Introduction}
%\IEEEPARstart{M}{etaverse} 

% add that we focus on networking and computation

The Metaverse is arriving at an inflection point of development thirty years after the term was first coined in Neal Stephenson's science-fiction novel "Snow Crash". While its definition is still fluid today, the Metaverse is generally regarded as an embodied version of the Internet that seamlessly synergizes virtual and physical worlds, and enables users to live and interact in virtual worlds through controlling personalized avatars. 

% Its use case is not limited to a gaming platform, a retail destination, a training tool, an advertising channel, a digital classroom, or a digital factory. It is estimated to generate a multi-trillion impact across consumer and enterprise use cases by 2030.

A growing amount of commitment, investment, and effort has been made to realize the Metaverse across academia and industry. For the former, the number of research publications related to the Metaverse in 2022 is five times more than that in 2021. For the latter, Facebook was rebranded as Meta to shift from a media company to a Metaverse  company in 2021. Such excitement about the Metaverse is driven by improved technical and marketplace readiness. For example, the enabling technologies of the Metaverse have undergone rapid advancements recently, including the Internet-of-Things (IoT), digital twin (DT), 5G/6G, artificial intelligence (AI), blockchain, virtual reality (VR), and augmented reality (AR). Moreover, the COVID-19 pandemic has accustomed users to embracing digital life, e.g., online meetings and shopping.


% Microsoft planned the acquisition of gaming company Activision for preparing the building blocks of the Metaverse in 2022. 

% Such excitement about the Metaverse is driven by improved technical and marketplace readiness. For example, the enabling technologies of the Metaverse have undergone rapid advancements recently, including the Internet-of-Things (IoT), digital twin (DT), 5G/6G, artificial intelligence (AI), blockchain, virtual reality (VR), and augmented reality (AR). Moreover, the COVID-19 pandemic has accustomed users to embracing digital life, e.g., online meetings and shopping.

As compared to the massively multiplayer online role-playing games (MMORPGs) of today, the Metaverse is a far broader concept that features user-generated contents (UGCs), a variety of use cases, decentralization, and interoperability across platforms and devices. However, there are still technical deficiencies in communication speed, computing power, data storage, etc., to fully realize the envisioned Metaverse, not to mention social barriers. As the Metaverse is relatively nascent, most research has been centered on Metaverse architecture, enabling technologies, and applications. The study of \cite{edgeMetaverse}  discusses the Metaverse implementation at mobile edge networks at scale. It presents a visionary four-layer architecture positioning enabling technologies in the edge-enabled Metaverse. In addition to technology, \cite{allone} addresses social aspects of the Metaverse, e.g., acceptance, security, and privacy. 

While the sustainability aspect has been pointed out in the aforementioned studies, no study has yet to evaluate the energy consumption, efficiency, and sustainability of the Metaverse. The energy demand could be easily escalated as the Metaverse will require 1000-fold more computing power and push data usage by 20 times in the next 10 years. \footnote{\url{https://www.intel.com/content/www/us/en/newsroom/opinion/powering-metaverse.html}}\textsuperscript{,}\footnote{\url{https://techblog.comsoc.org/2022/02/20/credit-suisse-metaverse-to-push/-data-usage-by-20-times-worldwide-by-2032/}} From our perspective, Metaverse sustainability is the ability of the Metaverse to exist and develop without depleting natural resources for the future, e.g., fossil fuels for energy and raw materials for electronic devices. 
It is becoming particularly critical due to increasing economic and environmental costs, user awareness, and stringent sustainability targets such as the Net Zero coalition.\footnote{\url{https://www.un.org/en/climatechange/net-zero-coalition}} Therefore, this article makes the first effort to advocate attention to optimizing energy efficiencies of all network components for realizing the Metaverse, i.e., Green Metaverse Networking (GMN). The article provides an analytical framework in a technology-based approach for evaluating energy efficiency. Specifically, our framework centers around the four main groups of energy-hungry enabling technologies namely:
\begin{itemize}
    \item \textit{Digital Twin} about creating and operating the Metaverse,
    \item \textit{Artificial Intelligence} about endowing the Metaverse with autonomy and intelligence,
    \item \textit{Extended Reality} about user interaction with the Metaverse, and
    \item \textit{Blockchain} about data storage and interoperability in the Metaverse.
\end{itemize}

% To understand the status quo of these energy-hungry technologies, we first review their application scenarios, energy consumption and efficiency, influencing factors, and evaluation metrics in Section \ref{sec2}. Then, we summarize the recent technological advancements in energy efficiency in Section \ref{sec3}. A case study is subsequently presented in Section \ref{sec4} for Metaverse energy conservation through resource allocation optimization and semantic communication. Finally, we conclude with a discussion of open research questions on the green Metaverse. 

The structure and contributions of this article are as follows:
\begin{enumerate}
    \item We identify energy-hungry technologies and their application scenarios in the Metaverse, analyze their consumption, efficiency, and influencing factors, and outline sustainability-related metrics, thereby providing a guideline for GMN assessment in Section \ref{sec2}.
    \item We discuss forefront advancements in computation and networking of these technologies to boost the Metaverse energy efficiency and sustainability in Section \ref{sec3}. Through a case study in Section \ref{sec4}, we illustrate how semantic communication and stochastic resource allocation can contribute to energy conservation, thereby guiding the design of future communication systems for the Metaverse.
    \item We discuss the key challenges and future research directions towards the GMN in Section \ref{sec5}.
\end{enumerate}

% \begin{figure*}[hb]
% \centering
% \includegraphics[width=7.8in]{structure.png}
% \caption{Towards Green Metaverse: Technologies, Recent Advancements and Future Directions.}
% \label{fig_str}
% \end{figure*}


\begin{figure*}[hb]
\centering
\includegraphics[width=7in]{architecture.pdf}
\caption{A four-tier Metaverse architecture and energy-intensive processes.}
\label{fig_arc}
\end{figure*}



\section{Energy-Hungry Technologies, Energy Consumption, Efficiency and Metrics}
\label{sec2}

    % In this section, we discuss the energy-hungry technologies of the Metaverse and their sustainability-related metrics as in Table \ref{tab:table1}.
    
    Metaverse is made possible due to the development of enabling technologies. But many of them are energy-hungry. Their sustainability-related metrics are summarized in Table \ref{tab:table1}.
    
    % Metaverse development is made possible due to the rapid advancement of various enabling technologies. Even though improving technological performance is the overarching goal in this nascent stage, we should not ignore the economic, social, and environmental costs associated.
    
    \subsection{Digital Twin}

    The DT is a digital representation that accurately reflects the physical world. It provides real-time visibility, data analytics, and the capability to predict potential changes in the physical world. As the backbone of DT, IoT collects physical data by networked sensors, derives insights from data by edge processors, transmits insights by gateways, and proceeds Metaverse commands by executors (Fig. \ref{fig_arc}). IoT devices are becoming unprecedentedly ubiquitous. The number of IoT devices is estimated to be 29 billion in 2030. Although the power consumption of each IoT device is as low as 10 nW to 10 W, such large-scale IoT deployment will make a significant global impact on the environment, especially since most devices are wireless and powered by batteries. 

    % Wireless communication is gaining importance given the rising demand for wireless connections, e.g., head-mounted displays (HMDs), sensors, smartphones, and unmanned aerial vehicles (UAVs). The global base station (BS) market is expected to annually increase by 17\% \cite{green5g}. Although 5G can be 90\% more energy efficient than 4G on a per-bit basis, it consumes three times more energy to cover the same area as 4G.\footnote{\url{https://www.sdxcentral.com/articles/analysis/5g-carriers-face-energy-crunch/2022/04/}} Severer challenges lie in developing countries and remote areas, where the BS is inefficiently underutilized and producing significant CO$_2$ emissions.
    
    % Emerging IoT devices make new challenges for wireless systems, e.g., head-mounted displays (HMDs), smartphones, and unmanned aerial vehicles (UAVs). Upcoming 5G systems still fall short of satisfying the stringent requirements of the Metaverse, e.g., a full immersive extended reality (XR) experience capturing all sensory inputs requires a high data rate and low latency simultaneously \cite{6Gvision}. 
    
    % The future envisioned 6G system with high speed, high reliability, and low latency 
    
    % capable of providing rate-reliability-latency services
    
    % Researchers envision the 6G system with rate-reliability-latency services to support Metaverse implementation. 
    
    
    %Moreover, DTs have been applied not only to remotely visualize  the state of the physical counterpart but also to control and predict its future states. 
    
    % Metaverse is envisioned as a seamless integration of persistent virtual worlds. Some virtual worlds, systems, or objects, receiving real-time inputs from IoT sensors and replicating the appearance and behavior of their real-world counterparts, are called DT. DT has been applied not only to remotely visualize and control the state of the physical counterpart but also to predict its future state. 

    
    % A new concept, \textbf{quality-of-physical-experience (QoPE)}, aggregates the physical QoS factors and subjective QoE opinions. The IoT energy efficiency largely depends on hardware, communication mode (e.g., polling and event-driven), protocols (e.g., Zigbee, and Wi-Fi), and software implementation.
    
    % IoT communication efficiency is measured the same way as other wireless systems, as in Table \ref{tab:table1}. IoT energy efficiency largely depends on hardware, communication mode (e.g., polling and event-driven), protocols (e.g., Zigbee, and Wi-Fi), and software implementation.

    % Communication and networking technologies are essential channels for data transmission between humans-to-humans (H2H), humans-to-machines (H2M), and machines-to-machines (M2M). 
    
    % Wireless systems are of growing importance given the growing number of mobile wireless connections, e.g., head-mounted displays (HMDs), sensors, smartphones, and unmanned aerial vehicles (UAVs). 
    
    % However, upcoming 5G systems still fall short of satisfying the stringent requirements of the Metaverse, e.g., a full immersive extended reality (XR) experience capturing all sensory inputs requires a high data rate and low latency simultaneously \cite{6Gvision}. Researchers envision the 6G system with rate-reliability-latency services to support Metaverse implementation. 
    
    % 5G network architecture has been designed with three key service areas in mind: enhanced Mobile Broadband (eMBB) for high data rate, Ultra-reliable Low Latency Communications (URLLC) for high reliability and low latency, and Massive Machine-Type Communications (mMTC) for large-scale connectivity. However, upcoming 5G systems still fall short of satisfying stringent requirements from the Metaverse, e.g., a full immersive extended reality (XR) experience capturing all sensory inputs requires high data rate and low latency simultaneously \cite{6Gvision}. Researchers envision the 6G system with rate-reliability-latency services to support Metaverse implementation. 
    
    %Realizing Metaverse poses unprecedented challenges in ultra-high data capacity, ultra-low latency, ubiquitous connectivity, reliability, and scalability.


    Network efficiency has a considerable influence on DT energy consumption, including radio networks, IoT, wireless sensor networks (WSNs), and other computer networks. It varies depending on the network device, communication mode, protocol, and software implementation. Typically, network efficiency is measured by the average power consumption at a certain level of service quality. The Quality of Service (QoS) metrics such as throughput, latency, and packet loss are well-defined to evaluate classical information-theoretic network performance. However, the perceived quality by users could be of more importance, especially for content providers. For example for video streaming, users could be sensitive to quality degradation due to varying bit rates although the throughput is high. Maintaining the bit rate at a lower but stable level, i.e. lower QoS, could improve user experience and reduce energy consumption at the same time. Therefore, a paradigm shift towards quality of experience (QoE) is called by content-centric Metaverse applications to focus on the overall acceptability of service perceived by the user.
    
    For wireless communications, the energy efficiency is addressed by data rate per power (in bit/J), which is currently at the order of 10 Mbit/J. Spectral efficiency (in bps/Hz) measures the data throughput over a given spectrum for a site. Area energy efficiency (in bps/Hz/m$^2$/J) is the primary energy efficiency metric for current wireless systems, which equals spectral efficiency divided by area coverage and energy consumption. 5G pledges 10 times higher area energy efficiency than 4G. Along with the rapid emergence of aerial nodes such as UAVs, volumetric energy efficiency (in bps/Hz/m$^3$/J) is expected to supersede in future systems. 
    
    Apart from networking, a significant amount of energy is required for computation. For example, 3D modeling and simulation are essential for creating avatars, DTs, and virtual worlds. To create immersive and realistic experiences, 3D models are developed with computation-intensive simulations of physics, lighting, sound, and other elements. Compared to creating virtual worlds, around-the-clock server hosting and operations in data centers consume far more energy because of numerous user connections, heavy data traffic, and intensive computational load. Cloud-native solutions are required to offload compute-intensive tasks (e.g., 3D rendering and AI training) to high-end servers in remote data centers when the capability of the local computer is limited. But the over-dependence on cloud computing could aggravate energy consumption. Taking cloud gaming as an example, the power demand in the upstream network together with data center graphic processing was estimated to be markedly (e.g., 40-60\% for desktops and 120-300\% for laptops) higher than that of local gaming \cite{game}. 

    Inevitably, the predominant cloud-based solutions of today lead to exacerbated energy challenges. Three major components contribute to the energy usage of data centers: physical servers, cooling systems, and network devices. They usually consume 40–55\%, 15–30\%, and 10–25\% of the total power respectively. To offset the strong growth in energy demand, remarkable efficiency improvements were implemented in servers, storage, network switches, and infrastructure design. The most efficient data centers nowadays reach a power usage effectiveness (PUE) of $~$1.1, meaning that 0.1 kWh is used for cooling/power provision for every 1 kWh of IT usage. 

    
    \subsection{Artificial Intelligence}
        
    AI plays an important role in endowing the Metaverse with autonomy and intelligence, thereby improving the QoE of users. For example, Computer Vision (CV), which derives meaningful information from images and videos, enables the Metaverse to provide services such as object detection, activity recognition, avatar navigation, and visual reasoning. Natural Language Processing (NLP) enables conversational non-playable characters (NPCs) in the Metaverse and also provides services including language translation, text summarization, sentiment analysis, and speech recognition. Generative AI utilizes existing text, audio files, or images to create 3D objects, digital assets (e.g., music, painting, article), user avatars, NPCs, and even whole virtual worlds. %\textbf{Recommender systems} deliver personalized services that are most pertinent to users based on their historical records. 
    Beyond creating innovative services, AI can be integrated with other technologies to provide more intelligent services, e.g., fast 3D rendering, efficient communication, and intelligent blockchains. 
    
    Despite widespread applications, AI comes at a high cost economically and environmentally due to high energy consumption. In AI training, computers are iteratively fed with training data and calculate the ML parameters continuously. Around 1000 MWh of electricity is spent to train a single language model like OpenAI GPT-3. An exponential increase of compute used in the largest AI training was observed in the last decade with a 3.4-month doubling time.\footnote{\url{https://openai.com/blog/ai-and-compute/}} Fortunately, unlike cloud-centric AI training, AI inference is trending to be shifted to distributed edge devices for higher scalability, lower operational risks, and reduced latency. However, this results in the additional challenges of limited processing power and energy supply of edge devices. 
    
    % Although the energy consumption associated with AI inference did not demonstrate an unbridled exponential growth in the last decade, thanks to the compensation from hardware and algorithmic improvements, the global energy consumption can easily escalate just by means of increased AI penetration \cite{inference-energy}. 
    
    % Fortunately, unlike cloud-centric AI training, \textbf{AI inference} is trending to be shifted to distributed edge devices for higher scalability, lower operational risks, and reduced latency. It confronts additional challenges of limited electrical and processing power of edge devices. 

    % \begin{figure}[!t]
    % \centering
    % \includegraphics[width=3.4in]{ai-compute.pdf}
    % \caption{The total amount of compute used in the largest AI training
    % \cite{openaiweb}.}
    % \label{aicompute}
    % \end{figure}

    To evaluate AI technologies, both accuracy and energy should be considered. Accuracy metrics are well-defined and easily computed for specific tasks. However, energy is rarely measurable and difficult to compute. Hence, a few alternatives have been used. The model size measures the ML model's memory consumption, which is strongly dependent on the algorithm and model architecture. The total number of floating-point operations (FLOPs) is used to approximate the number of multiply-and-accumulate (MAC) operations, indicate the computational workload, and imply energy consumption. Nonetheless, both metrics do not consider the impact of data movement between the memory unit and the processing unit, which may dominate the energy consumption in the computation \cite{sze2020evaluate}. Thus, a few energy estimation methodologies were developed for precise energy estimation for specific ML models and hardware devices. The trade-offs between accuracy and energy are necessary for AI training considering the diminishing marginal return of accuracy when increasing the dataset size and model complexity. It is especially true for distributed AI training, e.g., federated learning (FL), when increasing the concurrency.
    
    % For comparison, efficiency can be measured by the amount of energy consumed to reach a level of accuracy. For AI inference, more performance metrics apart from accuracy are pivotal including throughput, latency, flexibility, and scalability. An efficient design should achieve the best performance within the limited power, energy, and hardware cost budgets.

 
\begin{table*}[!t]
\caption{Summary of Metaverse sustainability-related metrics\label{tab:table1}}
\renewcommand{\arraystretch}{1.3} % Default value: 1
\setlength{\tabcolsep}{5pt} % Default value: 6pt
\centering
\begin{tabular}{|l|l|l|l|}
\hline
\textbf{Digital Twin} & \textbf{Consumption Point}  & \textbf{Unit} & \textbf{Description}\\
\hline
\hline
Power utilization effectiveness (PUE) & Data Center  & $(1,\infty)$ & ratio of total facility energy to IT equipment energy\\
\hline
Data center energy productivity (DCeP) & Data Center & $(0,1)$ & ratio of useful work produced to total energy\\
\hline
IT equipment utilization (ITEU) & Data Center  & $(0,1)$ & ratio of actual IT power to total rated IT power\\
\hline
Carbon usage effectiveness (CUE) & Data Center & kgCO$_2$/kWh & ratio of CO$_2$ emissions to IT equipment energy\\
\hline
Green energy coefficient (GEC) & Data Center & $(0,1)$ & ratio of green energy to total energy consumption\\
\hline
Power at QoS/QoE/QoPE & Wireless system  & W  & average power at a certain level of quality\\
\hline
Energy efficiency (EE) & Wireless system  & bit/J  & ratio of data rate to required power\\
% Telecom Energy Efficiency Ratio (TEER) & Wireless system & Efficiency & Gbps/W  & ratio of data transmission to power consumption\\
\hline
Energy consumption rating (ECR) & Wireless system  & W/Gbps & ratio of aggregated energy consumption to capacity\\
\hline
Area Power Consumption (APC) & Wireless system & W/m$^2$  & ratio of power consumption to served area \\
\hline
Spectral efficiency & Wireless system  & bps/Hz  & ratio of information transmission rate to bandwidth\\
\hline
Area energy efficiency & Wireless system  & bps/Hz/m$^2$/J  & ratio of area spectral efficiency to energy consumption\\
\hline
Volumetric energy efficiency & Wireless system  & bps/Hz/m$^3$/J  & ratio of space spectral efficiency to energy consumption\\
\hline
Network level performance indicator & Wireless system  & subscribers/W  & ratio of number of subscribers on busy hours to power\\
\hline
\arrayrulecolor{white}\hline
\\
\arrayrulecolor{black}\hline
\textbf{Artificial Intelligence} & \textbf{Consumption Point} & \textbf{Unit} & \textbf{Description}\\
\hline
\hline
Model size & ML model & bytes & storage space taken by ML model\\
\hline
Number of parameters & ML model & $(1,\infty)$ & number of learnable parameters in ML model\\
\hline
Elapsed time & ML model  & s & running time for generating an AI result\\
\hline
FLOPs & ML model  & FLOP & number of floating-point operations\\
% \hline
% FLOPS & Processor & Efficiency  & FLOP/s  & number of floating-point operations per second\\
\hline
FLOPS per watt & Processor & FLOPS/W & ratio of number of floating-point operations to energy\\
\hline
\arrayrulecolor{white}\hline
\\
\arrayrulecolor{black}\hline
\textbf{Blockchain} & \textbf{Consumption Point} & \textbf{Unit} & \textbf{Description}\\
\hline
\hline
Overall annual energy & Blockchain  & TWh/yr & overall yearly energy consumption of the network\\
\hline
Energy per transaction & Blockchain & kWh/txn & ratio of overall energy to the number of transactions\\
\hline
Coin carbon footprint & Blockchain  & kgCO$_2$/coin & ratio of carbon emissions to number of coins mined\\
\hline
Transaction carbon footprint & Blockchain  & kgCO$_2$/txn & ratio of carbon emissions to number of transactions\\
\hline
\arrayrulecolor{white}\hline
\\
\arrayrulecolor{black}\hline
\textbf{Extended Reality} & \textbf{Consumption Point} & \textbf{Unit} & \textbf{Description}\\\hline
\hline
Power at QoS/QoE/QoPE & HMD  & W & average power at a certain level of quality\\
% \hline
% Usability - efficiency & HMD & Efficiency & Y & null & resources expended in relation to goals achieved\\
\hline


\end{tabular}
\end{table*}

    
    \subsection{Extended Reality}

    XR extends physical reality into virtual reality to different extents: AR overlays virtual objects on the physical world and VR simulates a fully virtual world. The emergence of XR revolutionizes the Metaverse by creating brand-new virtual experiences in any imaginable domain. To have an immersive sensory experience, users wear a standalone HMD (e.g., Oculus Quest) or an HMD connected to a computer (e.g., Oculus Rift). High-end VR systems include sensors like accelerometers, gyroscopes, and magnetometers, which can collect more  physical world data. 360$^\circ$ 2D videos rendered from 3D models are transmitted to the edge device or HMD for local processing before displaying behind the eyes.   
    
    % In today's VR video system, 360$^\circ$ videos generated from 3D models are projected into planar videos and encoded in the data center. Encoded videos are transmitted to the edge device or HMD directly for decoding and reconstructing 3D models locally. Based on head orientation information collected from HMDs, 3D models are rendered into 2D videos for the target area. After processing, 2D videos are displayed in HMD for both eyes.
    
    VR video rendering is fulfilled mainly in data centers and edge computers due to the stringent requirements of processing hardware. It is reported to consume twice as much power as conventional planar video rendering in \cite{vr360}. In HMDs, video processing expends the majority of the power budget. Memory access and computation are two main contributors to video processing energy consumption. Moreover, VR systems are increasingly used to execute additional functions, e.g., sensing data collection and processing, AI training and inference, and caching, further increasing the energy demand.

    VR QoS metrics mainly include latency, resolution, field of view (FOV), and frame per second (FPS). The power at QoS is typically used to compare energy efficiency. Nevertheless, a device with better QoS does not necessarily provide better visual quality. For example, perceptually high-quality VR videos do not require uniformly high-quality pixels (i.e., full resolution) with a wide FOV. QoE-based energy efficiency metrics encourage the development of VR systems with better visual quality and less power required. Detailed examples are illustrated in Section \ref{adv-xr}.
    
    
    \subsection{Blockchain}
    \label{blockchain-now}
    
    %Strictly speaking, it is not mandatory to construct a Metaverse based on blockchain technology. But the blockchain does provide apparent advantages. 
    
    As a distributed database, the blockchain can be used to store the massive amount of data generated in the Metaverse, e.g., UGCs and IoT data. It decentralizes storage, improves scalability, reduces data transmission, and ensures data security and privacy. The development of cross-chain protocols allows data exchange among blockchains and improves Metaverse interoperability. Non-Fungible Tokens (NFTs) are unique cryptographic tokens on blockchain indicating the ownership of digital or real-world items as well as transaction history. NFTs can ensure uniqueness, track ownership, automate trading through smart contracts, and establish trust among parties, which highly motivates the creation of UGCs. In contrast, cryptocurrency is adopted as a fungible digital currency with economic value. It eliminates the need for central authorities, reduces transaction costs, enhances security and privacy, and simplifies fund transfers. 

    Nevertheless, most of the existing blockchain applications are suffering deficiencies in operation efficiency and  energy consumption. Annually, Bitcoin is estimated to consume 60-125 TWh of electrical energy, which is comparable to the yearly consumption of countries like Austria and Norway \cite{sedlmeir2020energy}. A single transaction of Bitcoin requires enough electricity to supply an average size household in Germany for weeks, which is multiple orders of magnitude higher than traditional systems. Two factors largely determine blockchain energy consumption.

    % \begin{figure}[!t]
    % \centering
    % \includegraphics[width=3.5in]{blockchain-energy.pdf}
    % \caption{Rough comparison of the power consumption of different blockchain architectures per transaction \cite{sedlmeir2021recent}.}
    % \label{blockchain-energy}
    % \end{figure}

    \textbf{Consensus Mechanism}: Although there is a growing variety of consensus mechanisms, the majority of contemporary blockchain applications are based on Proof-of-Work (PoW) like Bitcoin. PoW involves network nodes solving computationally intensive puzzles and competing to create a new block, namely mining. Because miners with more powerful hardware have a higher probability of success, it creates an arms race among miners and generates electronic waste.
    
    % Mining refers to the solution searching process, which is computationally intensive and thus energy-hungry. Moreover, mining is economically incentivized by cryptocurrency rewards and transaction fees. The skyrocketed market value of cryptocurrency stimulated mining activities globally. Because miners with more powerful hardware have a higher probability of success, it creates an arms race among miners and generates electronic waste. Admittedly, PoW leverages high energy consumption to protect the blockchain from attacks, but the computational work in mining serves no real-world purpose and is often regarded as wasteful.
\begin{figure*}[b]
\centering
\includegraphics[width=7in]{techs.pdf}
\caption{Emerging and evolving Green Metaverse Techniques in history. (EE: energy-efficient, EA: energy-aware, Com: communication)}
\label{fig_techs}
\end{figure*}

    \textbf{Operation Redundancy}: After the consensus determines the valid transactions, these transactions are processed on each node of the blockchain network redundantly. The energy associated with this redundant operation is found orders of magnitude lower than that associated with consensus mechanism \cite{sedlmeir2021recent}. However, for non-PoW blockchains, where the consensus mechanism spends much less energy, this redundant operation may become a major energy consumer. While redundancy gives more decentralization, reliability, and security, replicating full transaction data in every node imposes challenges to scalability and energy consumption. 
    
    In a holistic view, the idle power consumption of nodes should also be considered apart from the above factors. To evaluate overall energy efficiency, energy consumption per transaction is commonly used to compare blockchains to other technologies. Based on this metric, most contemporary blockchain applications (e.g., Bitcoin) still spend a mammoth amount of energy compared to traditional payment systems (e.g., credit cards). Notably, this metric can be misleading because blockchain energy consumption largely depends on consensus mechanisms (e.g., mining in PoW) rather than transaction volume. In a nutshell, the overall energy consumption is still significant for current blockchains. The situation is expected to change after the world’s second-largest cryptocurrency, Ethereum, shifts from PoW to a greener proof-of-stake (PoS) system in Q3/Q4 2022.
    
    \textbf{Lessons learned:} Employing energy-inefficient technologies is detrimental to the Metaverse scalability and results in profound negative impacts on the environment. The metrics in Table \ref{tab:table1} are of great use to identify inefficient components in Metaverse networks. In many cases, trade-offs must be well-balanced between higher energy consumption and improved user experience. As the Metaverse will be human-centric, QoE-based energy-efficiency measures must be designed to directly link user experience with energy usage.
    
    % . Edge devices for IoT, XR, and AI upgrade the user experience to some extent, which is tightly constrained by the power limit. EE of data centers and communication networks have profound global impacts on the environment. 
    
    
\section{Recent Technological Advancements in Energy Efficiency}
\label{sec3}

    Numerous research efforts have been put into individual technology for improving performance, potentially contributing to the Metaverse sustainability. The most significant improvements and their applications in the GMN are selected for discussion. A summary is presented in Fig. \ref{fig_techs}. 
    
    % The synergy between technologies will be the crux of the green Metaverse.


    \subsection{Digital Twin}
    \label{adv-dt}

    \subsubsection{Sensor, User and Edge}

    In IoT systems, data collected by IoT devices are centralized either in an IoT gateway connected to the cloud or in a local IoT hub. The primary type of energy reduction measures focuses on IoT sensors. Because idle sensors consume unnecessary energy, an efficient duty scheduling algorithm is required to minimize power consumption while satisfying QoS requirements. The duty cycle can be designed beforehand based on predictions of communication periods. It can also be dynamically adjusted based on the real-time priority of traffic, device energy, and required power. 30-50\% energy reduction is obtained by optimizing the scheduling and workload distribution in \cite{efficientprotocols}. Moreover, novel sensing concepts including selective sensing, approximate sensing, and compressed sensing offer additional efficiency improvement in data transmission.
    
    % selective sensing collects only necessary data in particular situations or at specific times. Approximate sensing reduces data precision in processing and communication. Compressed sensing efficiently samples the data which can be precisely reconstructed.
    
    IoT sensors can be powered by ambient free energy such as light, radio frequency, temperature difference, and kinetic energy, which is called energy harvesting. The research community has also worked towards developing energy-efficient routing protocols, especially energy harvest-based protocols for WSNs to cope with intermittent renewable resources. For example, an energy back-off process was introduced for the device to harvest energy. Taking into account that, the energy-harvesting-aware routing algorithm proposed by \cite{harvestalgo} consumes 40-80\% less power and reduces 40\% packet loss compared to other algorithms by selecting the node with the highest residual energy level and the minimum cost link. 
    
    % Lastly, there are a lot of innovations in edge devices, e.g., high/low-computation dual-core processors, sensor-on-chips, and computation offloading. 

    \subsubsection{Communication Network}

    %Data are transmitted by wired networks (e.g., twisted-pair cables, co-axial cables, and fiber-optic cables) or wireless networks (e.g., radio, infrared, and satellite). 
    
    % It should be noted that wireless technologies consume far more energy than wired solutions, e.g., 4G technologies consume 23 times more than wired connection.\footnote{\url{https://www.lowtechmagazine.com/2015/10/can-the-internet-run-on-renewable-energy.html}} However, 
    
    Unique features of accessibility, mobility, and scalability make wireless communication indispensable in the Metaverse.  A comprehensive review of future green radio networks has been conducted by \cite{green5g}.
    
    % In a wireless BS, the power amplifier (PA) consumes a maximum amount of energy (nearly 60\%). Special materials and new designs proposed were able to improve PA efficiency by 30-80\%\cite{green5g}. 
    
    % With regard to network architecture, two techniques demonstrate promising potential in improving energy efficiency for wireless BS. 
    Heterogeneous networks (HetNets) engage diverse networks in a hierarchical structure to provide the service to the same user. Macrocells with a higher capacity provide services for the larger area coverage while small cells with better efficiencies are used to reduce energy consumption and enhance the network throughput. Optimizing cell selection and deployment promotes the HetNet energy efficiency gain. Cell zooming dynamically varies the cell size based on the traffic load, state of the channel, etc. When the cell is underutilized, it can transfer the traffic to nearby cells through cell zooming and enter sleep mode for energy conservation. Beamforming employs an array of antennas to focus electromagnetic radiation (EMR) toward the user's direction. So, the signal strength in a specific direction maximizes, EMR exposure and transmit power reduces \cite{green5g}. Massive multiple-input and multiple-output (mMIMO) is developed based on MIMO by packing more antennas into a small area. Compared to MIMO, mMIMO improves the capacity by 5 times and energy efficiency by 100 times due to the focus of radiated energy on user equipment \cite{green5g}. Device-to-Device (D2D) communication is a promising technique that enables direct communication between devices without traversing the network. Lastly, semantic communication is envisioned as a key component of 6G for reducing the transmission volume, facilitated by AI technologies. The extracted meanings of the information are transmitted rather than information symbols. A semantic communication case study is presented in Section \ref{sec4}.
    
    \subsubsection{Data Center and Cloud}

    % In modern data centers, virtualization offers opportunities not only to increase hardware utilization but also to improve energy efficiency. 
    An idle or underutilized server consumes two-thirds of the energy when it is fully utilized. With virtualization, a physical machine (PM) can host multiple virtual machines (VMs) independently to avoid under-utilization, which enhances energy efficiency at the same time. The approach of combining underutilized servers is called server consolidation. However, non-negligible energy overhead is introduced for the hypervisor software to manage VMs and VM migration. Efficient strategies have been proposed for server consolidation, e.g., a virtual world zone-based strategy for MMORPGs and a resource demand prediction-based strategy for cloud computing. In general, optimization of the mapping between PMs and VMs is formulated as the VM placement problem. Job scheduling determines how efficiently the data center resources are utilized. Joint optimization of VM placement and job scheduling could lead to more significant improvement. Apart from IT equipment, the cooling system also evolves rapidly. For example, the Google AI-empowered cooling system reduced its energy consumption by 40\%. 
    
    % Microsoft tested a two-phase liquid cooling technique for a 90\% efficiency advantage compared to traditional air cooling. Goggle empowered AI to optimize the cooling system and reduced its energy consumption by 40\%. 
    
    %Free cooling technologies are being tested to utilize free cold air and natural water, lowering overall cooling demand. 

    
    \subsection{Artificial Intelligence}
    \label{adv-ai} 

    % \subsubsection{Specialized Hardware}

    % In the programmability continuum of processors, central processing units (CPUs), graphics processing units (GPUs), \textbf{field-programmable gate arrays (FPGAs)}, and \textbf{application-specific integrated circuits (ASICs)} are placed in descending order. While the function can be reprogrammed in the FPGA, it cannot be changed in the ASIC once built. The processor with lower programmability tends to be more specialized in processing certain tasks with higher performance and power efficiency. FPGAs and ASICs are prevalent platforms for large production applications, e.g., AI inference. Compared to processing, data movement between processor and memory consumes much more energy due to the Von Neumann bottleneck. To reduce the data movement, \textbf{processing-in-Memory (PIM)} technologies perform computation in the memory. In recent years, researchers have proposed efficient PIM architectures for a variety of ML models, e.g., Transformers and Generative Adversarial Networks. Last but not least, \textbf{neuromorphic computing}, regarded as the next generation AI, demonstrates major advantages in energy efficiency, execution speed, robustness against failures, and learning ability. It emulates synapses and neurons in the human brain. The development of memristors, inherited with PIM capability and scalability, significantly facilitates the hardware implementation of neuromorphic computing. Likewise, more technologies, e.g., photonic computing and quantum computing, are still in the early stage but have shown promising potential.
    
    \subsubsection{Machine Learning Model}

    % federated learning
    An efficient model design contributes to reduced computation and considerable energy savings. Certain models inherently have lower power requirements, e.g., spiking  neural networks (SNNs) leverage sparse event-driven spikes and MobileNet adopts depthwise convolution for reducing computation. Compact architecture involves trade-offs on architecture hyperparameters. Increasing complexity results in diminishing marginal returns, and even sometimes hurts performance. Methods, e.g., Neural Architecture Search (NAS), have been developed to automate the design for optimal architectures. 

    Model compression provides an effective way to reduce model sizes and computation with minimal impact on performance. For example, pruning  prunes out the non-critical sections (e.g., connections, neurons, and layers) in the neural networks. The energy-aware pruning reduces the AlexNet and GoogLeNet energy consumption by 3.7 and 1.6 times \cite{pruning}. Quantization approximates floating-point numbers by low bit-width numbers for ML model parameters. Model distillation, also called knowledge distillation, trains a small-size "student" model to mimic a large-size "teacher" model. The pruned, quantized, or distilled model is best suited for AI inference at scale. Moreover, dataset distillation distills a large dataset into a synthetic smaller dataset, which provides computation saving for repeated AI training experiments.
    
    %Similar to model distillation, \textbf{dataset distillation} distills a large dataset into a synthetic smaller dataset. Although it is not a model compression method, it can also offer energy-saving benefits, especially for redundant AI training experiments.
% \begin{figure}
% \centering
%     \includegraphics[width=0.8\columnwidth]{energy.pdf}
%     \caption{The energy consumption of semantic data and non-semantic data. }
%     \label{fig:energy}
% \end{figure}
    
    % \subsubsection{Model Design}
    
    % An efficient model design contributes to reduced computation and considerable energy savings. The first strategy is model selection. Certain models inherently have lower power requirements, e.g., spiking  neural networks (SNNs) leverage sparse event-driven spikes and MobileNet adopts depthwise convolution for reducing computation. Another model design strategy lies in compact architecture, which requires trade-offs on architecture hyperparameters, e.g., width, depth, and resolution. Increasing complexity results in diminishing marginal returns, and even sometimes hurts performance. Approaches have been developed to automate the design, e.g., MorphNet. EfficientNet provides an experiential method for optimally scaling the model architecture. Moreover, the parameter sharing technique can be applied to reduce the number of parameters by sharing them among different parts of the model.
    
    % \subsubsection{Model Compression}

    % Model compression provides an effective way to reduce model sizes and computation with minimal impact on performance. Pruning is one of the model compression methods which prunes out the non-critical sections (e.g., connections, neurons, and layers) in the neural networks. The energy-aware pruning reduces the AlexNet and GoogLeNet energy consumption by 3.7 and 1.6 times \cite{pruning}. Quantization approximates floating-point numbers by low bit-width numbers for ML model parameters. Model distillation, also called knowledge distillation, trains a small-size "student" model to mimic a large-size "teacher" model. The pruned, quantized, or distilled model requiring much less memory and computation is best suited for AI inference at scale. Similar to model distillation, dataset distillation distills a large dataset into a synthetic smaller dataset. Although it is not a model compression method, it can also offer energy-saving benefits, especially for redundant AI training experiments.

    \subsubsection{Computing Paradigm}

    Computation can be performed in different manners, e.g., distributed computing and cloud computing. When AI is trained in distributed edges, only extracted features of data or model artifacts require to be transmitted, which significantly reduces the data transmission in the network. D2D communication can further increase the data transmission efficiency between edge computers and sensors. FL is a typical distributed AI training technique based on edge computing. Another advantage of edge computing is that it can leverage the diverse resources available in different locations. For example, FL can enable workload allocation based on the available intermittent renewable energy to improve overall sustainability. Cloud computing is advantageous when the efficiency of the individual device is much lower than the centralized high-end server. On the contrary, when data transmission is massive and costly, edge computing is more sustainable. 

    \subsection{Extended Reality}
    \label{adv-xr} 

    % \subsubsection{Specialized Hardware}

    %  In general 3D rendering, the memory bandwidth is recognized as the bottleneck of performance improvement. In response, \textbf{PIM} technologies stack volatile memory directly on top of a microprocessor in 3D-stacked memory systems, e.g., Hybrid Memory Cube. The decreased distance (i.e., electrical resistance) between memory and processor not only increases the transmission speed but also reduces energy consumption. For 360$^\circ$ VR video rendering, computation energy consumption is particularly severe. This process is offloaded to the GPU as a texture mapping problem and solved by the existing Texture Mapping Unit. However, inefficiency is observed, which can be solved by employing \textbf{specialized processors} designed for projective transformation.
    
    % Notably, efficient algorithms need to be designed for the hardware, and software adaption is required for the data flow change in the device.

    \subsubsection{Video Processing}

    Software and algorithmic improvements play a key role in improving video processing efficiency. For instance, data reuse is exploited in the computation of 360$^\circ$ VR pipeline. By memorizing head orientation and establishing a relation between left and right eye projection, some computations can be skipped. Another improvement opportunity lies in video encoding, which is performed before videos are transmitted to compress the video. It effectively reduces the amount of data for storage and transmission. 360$^\circ$ VR videos are currently encoded in three ways: equirectangular projection, cubemap projection, and pyramid projection. Compared to equirectangular projection, cubemap and pyramid projection can save 25\% and 80\% file sizes. 

    %Last but not least, video streaming imposes additional challenges on the current video processing pipeline. In HMDs or edge computers, video frames are decoded in the processor and stored in the memory before being sent to the display panel. Video streaming aggravates the memory access issue like in computation. 
    
    % \textbf{Processing pipeline} can be optimized to avoid redundant memory access, for example, by establishing a direct link between the processor and display panel. 
    
    \subsubsection{Human Visual Perception}

    Characteristics and limitations of human visual perception provide opportunities to improve XR efficiency. Firstly, the fovea, the central area of the human eye, has a substantially higher visual acuity than the rest of the retina. In light of this fact, foveated rendering is developed to render the gazed area of the image at a higher resolution and the peripheral area at a lower resolution. According to \cite{jabbireddy2022foveated}, it can achieve 3 times speedup and 70\% pixel reduction. The same concept can also be incorporated with video compression to reduce the bandwidth and energy required during data transmission. Secondly, the human visual system has limited capability to perceive details of high spatial and temporal frequencies. Rather than rendering all video frames at full resolution, if the resolution of every other fame is reduced, the perceived quality could be still maintained. Multiplex video resolution can also reduce both rendering workload and data transmission. Furthermore, the dark adaptation is another feature of the human eye that increases visual sensitivity in a dark environment. By smoothly decreasing the brightness level of the screen, a significant amount of energy can be saved while keeping the same brightness perception. 

    \subsection{Blockchain}
    \label{adv-blockchain}  
\begin{figure*}[b]
\centering
 \includegraphics[width=15cm, height=8cm,trim={0cm 17cm 0cm 0cm},clip]{system_model.pdf}\par
  \caption{System model in the virtual transportation network.}
  \label{fig:systemmodel}
\end{figure*}

    \subsubsection{Consensus Mechanism}
    
    The first type of green advancement lies in alternative energy-efficient consensus mechanisms, e.g., PoS, Delegated PoS (DPoS), and Proof-of-Authority (PoA). PoS replaces miners with validators to validate transactions and add new blocks to the blockchain. Instead of expending assets up-front as energy expenditure, validators stake crypto assets as collateral against dishonest behavior. DPoS, a notable iteration of PoS, allows participants to elect delegates with voting power proportional to their stakes. Only a capped number of delegates are chosen for transaction validation and block production. Thus, it makes consensus more efficient to reach, at the expense of decentrality. While the above mechanisms are open to the public, PoA pre-approves participants with trustworthy identities. Participants stake identities and voluntarily disclose themselves to become validators. The simplicity of consensus-reaching processes and a smaller number of validators make PoA require minimal computational effort. In short, eliminating the mining process significantly reduces blockchain energy consumption, at least 99.95\% according to \cite{papageorgiou2021energy}.

    \subsubsection{Operation Redundancy}

    The second type of advancement focuses on reducing operation redundancy, especially for non-PoW blockchains. Reducing the number of nodes to process and the workload in the process can both contribute to redundancy reduction \cite{sedlmeir2020energy}. The sharding technique splits the blockchain network into smaller independent partitions, i.e., shards, and processes each transaction only on nodes in one shard rather than the whole network. Therefore the number of redundant operations can be reduced as well as the energy consumption.
    
    Layer 2 solutions aim to reduce the workload or complexity of the transaction operation by building a second-layer protocol on top of the foundational blockchain. For example, the off-chain payment protocol (e.g., Lightning for Bitcoin and Raiden for Ethereum) creates a direct channel between two participants and enables instantaneous bidirectional transfers, as long as the net sum of transfers does not exceed the deposit. Because the blockchain is involved only when the channel is created or closed, the intensive and redundant computation required for blockchain operations is largely avoided. Zero-knowledge rollups (ZK-rollups) bundle transactions into batches and execute them off-chain. An easily verifiable proof is generated and submitted back to the blockchain, which proves the correctness of changes after executing a batch of transactions. All blockchain participants validate the proof instead of raw transactions, obviously simplifying the workload. Slightly offset by proof generation, ZK-rollup adoption is estimated to reduce 98.5\% of energy demand \cite{papageorgiou2021energy}.

    
    \textbf{Lessons learned:} Fields such as green communications and data centers are more advanced in sustainable development. Green AI and blockchain have taken off in recent years given more widespread adoption. For XR, research on performance improvement, rather than green technologies, still dominates. 


\section{Case Study: Stochastic Resource Allocation for Semantic Communication in the Metaverse}
\label{sec4}
    

\begin{figure}[b]
\centering
    \includegraphics[width=0.8\columnwidth]{energy.pdf}
    \caption{The energy consumption of semantic data and non-semantic data. }
    \label{fig:energy}
\end{figure}


This section presents a case study of the virtual transportation network (VTN) in the Metaverse~\cite{ng2022stochastic}. For example, a transport company can create a VTN to provide on-the-job training for its drivers and to conduct large-scale user studies in a safe but realistic environment. A realistic VTN is developed through the synchronization of sensing data from edge devices in the physical world and DTs of the virtual world. To keep DTs updated, sensing data is regularly transmitted to virtual service providers (VSPs). Fig. \ref{fig:systemmodel} illustrates the system model.

% However, to provide a more realistic virtual world, it is essential to consider human activities. With the help of the Metaverse, humans can join the virtual world as avatars. Avatar, a digital copy of humans, can explore and enjoy the applications developed by respective virtual service providers (VSPs). Furthermore, companies in the physical world can use the Metaverse to improve their QoE. For example, a bus company can create a VTN to provide on-the-job training to bus drivers and to conduct large-scale user study in a safe but realistic environment. 

\textbf{Energy-saving semantic communication:} It is energy-consuming to transmit data from edge devices to VSPs. Specifically, the edge device transmits the original image to VSPs, while ignoring the semantic importance of the bit flow. The objective of semantic communication is to be user-oriented, i.e., transmit only the part of semantic data interested by VSPs. For example, an autonomous driving company requires images (vehicles that are driving on the road) to train detection models in the Metaverse. Therefore, instead of transmitting the entire captured images which involve other objects, the semantic extraction module outputs only the segmented objects of interest for transmission. With the help of semantic extraction using YOLO, we can validate in Fig. \ref{fig:energy} that the energy consumption for semantic data transmission only requires 0.896J as compared to 111J for non-semantic data. Moreover, inferences on pre-trained semantic extraction model such as YOLO requires very little energy \cite{kim2020spiking}. Therefore, with semantic communication, edge devices can reduce power consumption during transmission as well as storage costs.

\textbf{Stochastic resource allocation for semantic communication:} To maintain DTs for realistic VTNs, VSPs may tap into data marketplaces that aggregate data from edge devices and sell to interested buyers. We assume that the cost of data is proportional to the energy cost incurred by edge devices in the case study. There are two types of subscription plans in data marketplaces, namely "reservation" and "on-demand". The reservation plan allows the VSPs to choose the edge device and purchase the number of data transmissions in advance. The on-demand plan is an ad-hoc plan that is only used when needed, so it is more expensive than the reservation plan. However, the demands of VSPs are constantly changing as it is highly dependent on the users' requests. Furthermore, since the reservation plan is subscribed in advance, the demand of the VSPs may not be the same after the subscription. For example, the demand is initially the vehicles on the road. After a while, the detection system may not be able to detect pedestrians very well under extreme weather conditions. Hence, the on-demand plan may be triggered to obtain more data to update the AI model. However, if an incorrect plan is used, e.g., reserve the edge device that transmits semantic data irrelevant to the interest of VSP, energy is wasted from unwanted data transmission, and additional energy is required for re-transmission from other edge devices. To account for the demand uncertainties of VSPs, we proposed a two-stage stochastic integer programming method to minimize the operation cost of VSPs as well as transmission energy. Using VSP historical demand data, our resource allocation scheme achieves a much lower cost than other schemes that do not consider the probability distribution of VSP demand (Fig. \ref{fig:evf}).

\begin{figure}[b]
\centering
 \includegraphics[width=0.8\columnwidth]{EVF.pdf}\par
  \caption{SIP comparing with EVF and random schemes.}
  \label{fig:evf}
\end{figure}



\section{Key Challenges and Future Research Directions}
\label{sec5}

\textbf{Challenge 1 -- QoE-Driven Green Metrics:} From Table \ref{tab:table1}, we can observe that most of the sustainability-related metrics are based on conventional QoS factors. In contrast, the recent paradigm shift towards the human-centered design of communication systems will shift the focus from QoS to QoE. Unfortunately, QoE is highly subjective based on user preferences. This is exacerbated by the fact that the Metaverse will feature a diverse range of applications and involve new modalities of user data. As such, before we can design systems to optimize QoE-driven green metrics, it is important to redefine QoE through large-scale user studies.

\textbf{Challenge 2 -- Metaverse Sustainability Index:} Such index is beneficial for Metaverse providers to plan an improvement roadmap, for users to choose green Metaverse products, and for governments to regulate the healthy development of the industry. However, the enormous amount of technologies and stakeholders involved makes it challenging to assess the overall sustainability. Although technologies can be evaluated individually using the metrics in Table \ref{tab:table1}, their contributions to the Metaverse are still vague currently. Practical frameworks and tools will be needed in the future to integrate diverse technological metrics and guide the overall assessment.

\textbf{Challenge 3 -- Trade-Off Balancing System Performance-Energy Consumption:} While research typically focuses on maximization of system performance, e.g., attaining the highest accuracy/fastest inference of AI models or highest FPS in rendering, it is often the case that some applications may not require such extents of high performance. In this case, the system performance can be slightly sacrificed to reduce the carbon footprint or energy per transaction. However, the trade-off between system performance and energy consumption has to be better studied.

\textbf{Challenge 4 -- New Incentive Schemes for Green Technology Adoption:} Incentive mechanisms designed to encourage green technology adoption, e.g., through tax credits or grants, have successfully encouraged the adoption of green office buildings. The next challenge will therefore be how can incentive mechanisms be designed to extend the coverage to the virtual worlds and the enabling technologies of the Metaverse. 

\textbf{Challenge 5 -- System-Level Design:} Diverse technologies are interconnected together in the Metaverse. On one hand, improving one technology might cause additional consumption to other technologies, e.g., compressed sensing and video encoding reduce the data transmission but increase the computation workload. On the other hand, the full-system design could exploit a larger energy-saving potential than focusing on individual subsystems, e.g., synergistically approximating sensing, computing, memory, and communication. Therefore, the system-level design is pivotal for orchestrating technological advancements towards the GMN.


\section{Conclusion}
\label{con}
In this paper, we first present the energy-hungry technologies of the Metaverse. Then, we discuss recent technological advances for the GMN. We subsequently present a case study of the semantic communication-enabled Metaverse development. Finally, we discuss the key challenges and future research directions.


\bibliographystyle{unsrt} % We choose the "plain" reference style
\bibliography{citation} % Entries are in the refs.bib file

% \begin{thebibliography}{9}
% \bibitem{edgemetaverse}
% Donald E. Knuth (1986) \emph{The \TeX{} Book}, Addison-Wesley Professional.

% \bibitem{lamport94}
% Leslie Lamport (1994) \emph{\LaTeX: a document preparation system}, Addison
% Wesley, Massachusetts, 2nd ed.
% \end{thebibliography}


\section*{Biography}

SIYUE ZHANG is currently pursuing a Ph.D. degree with Alibaba Group and Alibaba-NTU Joint Research Institute, Nanyang Technological University (NTU), Singapore. His research interests include the Metaverse and artificial intelligence.

WEI YANG BRYAN LIM is currently Wallenberg-NTU Presidential Postdoctoral Fellow. His research interests include edge intelligence and resource allocation.

WEI CHONG NG is currently pursuing a Ph.D. degree with Alibaba Group and Alibaba-NTU Joint Research Institute, Nanyang Technological University, Singapore. His research interests include the Metaverse, stochastic integer programming, and edge computing.

DUSIT NIYATO [IEEE Fellow] is currently a Professor with the School of Computer Science and Engineering and, by courtesy, School of Physical and Mathematical Sciences, Nanyang Technological University, Singapore. He has published more than 380 technical papers in the area of wireless and mobile networking and is an inventor of four U.S. and German patents. He was named the 2017–2021 Highly Cited Researcher in Computer Science. He is currently the Editor-in-Chief for IEEE Communications Surveys and Tutorials.

XUEMIN (SHERMAN) SHEN [IEEE Fellow] is currently a university professor with the Department of Electrical and Computer Engineering, University of Waterloo. His research focuses on network resource management, wireless network security, social networks, 5G and beyond, and vehicular ad hoc networks. He is a Canadian Academy of Engineering Fellow, a Royal Society of Canada
Fellow, and a Chinese Academy of Engineering Foreign Fellow. He received the R.A. Fessenden Award in 2019 from IEEE, Canada; the James Evans Avant Garde Award in 2018 from the IEEE Vehicular Technology Society; and the Joseph LoCicero Award in 2015 and Education Award in 2017 from the IEEE Communications Society.

CHUNYAN MIAO is currently a professor in the School of Computer Science and Engineering, Nanyang Technological University (NTU), and the director of the Joint NTU-UBC Research Centre of Excellence in Active Living for the Elderly (LILY).


\end{document}


