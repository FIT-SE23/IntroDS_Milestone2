\section{Related Work}\label{sec:related}
The densest subgraph problem is one of the most popular optimization models for dense subgraph discovery.
Let $G=(V,E)$ be an undirected graph and $w\colon E\to\mathbb{R}_{++}$ a positive edge weight.
For a vertex subset $S\subseteq V$, the subgraph induced by $S$ is denoted by $G[S]\coloneqq(S,E[S])$, where $E[S]\coloneqq \bigl\{\{u,v\}\in E\mid u,v\in S\bigr\}$.
In addition, we denote by $w(S)$ the total weight of the edges in $S$, i.e., $w(S)=\sum_{e\in E[S]}w(e)$.
For a nonempty $S\subseteq V$, the \emph{degree density} (or simply called \emph{density}) of $S$ is defined as $w(S)/|S|$ (where we define the density of the empty set (i.e., $0/0$) to be $0$).
In the densest subgraph problem, given a graph $G=(V,E)$ with an edge weight $w$,
we are asked to find $S\subseteq V$ that maximizes the density $w(S)/|S|$.

It is well known that the densest subgraph problem can be solved exactly in polynomial time using a maximum-flow-based algorithm~\cite{Goldberg_84} or an LP-based algorithm~\cite{Charikar2000}.
Moreover, it was shown that a simple greedy algorithm called the \emph{greedy peeling} admits $2$-approximation in $O(m+n\log n)$ time~\cite{Charikar2000,Kortsarz+94}. 
Recently, Boob et al.~\cite{Boob+20} designed an iterative greedy peeling algorithm, and demonstrated empirically that the output tends to be nearly optimal. 
Later, Chekuri, Quanrud, and Torres~\cite{Chekuri+22} proved the convergence to optimality of this algorithm (in a more general context). 

Balalau et al.~\cite{Balalau+15} introduced a simple preprocessing algorithm to improve the scalability of (exact) algorithms for the densest subgraph problem.
Their preprocessing algorithm first computes an approximate solution $S\subseteq V$ (using the greedy peeling),
and then iteratively removes a vertex with the (weighted) degree less than the objective value of the approximate solution obtained.
The validity of this preprocessing algorithm is guaranteed by the fact that any vertex with the (weighted) degree less than the optimal value is not contained in any optimal solution~\cite{Balalau+15}.

For the densest common subgraph problem, Jethava and Beerenwinkel~\cite{JB2015} devised an LP-based polynomial-time heuristic and a $2k$-approximation algorithm based on the greedy peeling, where $k$ is the number of layers.
Later, Charikar, Naamad, and Wu~\cite{charikar+18} designed two polynomial-time algorithms with approximation ratios $O(\sqrt{|V|\log k})$ and $O(|V|^{2/3})$ (irrespective of $k$), respectively.
Moreover, they showed some strong inapproximability results for the problem, based on some reasonable computational complexity assumptions.
Thus, it is very unlikely that a well-approximate solution can be found in polynomial time.
In contrast to this, as mentioned above, we can compute an \emph{optimal} stochastic solution in terms of the aforementioned three metrics in polynomial time.

%Moreover, as a negative result, they showed that the densest common subgraph problem is at least at hard to approximate as \textsc{MinRep}, a well-studied minimization version of \textsc{Label Cover}, which implies that the problem cannot be approximated to within a factor of $2^{\log^{1-\epsilon}|V|}$, unless $\text{NP}\subseteq \text{DTIME}(n^{\text{polylog}n})$. 
%They also showed that if the planted dense subgraph conjecture is true, the problem cannot be approximated to within a factor of $|V|^{1/4-\epsilon}$ and even for $k=2$, the problem cannot be approximated to within $|V|^{1/8-\epsilon}$. 
Recently, Galimberti, Bonchi, and Gullo~\cite{Galimberti+_17} introduced a generalization of the densest common subgraph problem,
which they refer to as the \emph{multilayer densest subgraph problem}.
This problem exploits a trade-off between the minimum density value over layers and the number of layers considered.
They proposed an approximation algorithm using a core decomposition technique for multilayer networks.
%In this problem, given a multilayer network $G_i=(V,E_i)\ (i=1,\dots,k)$, a positive real $\beta$, we are asked to find $S\subseteq V$ that maximizes $\max_{I\subseteq [k]}\min_{i\in I}\frac{|E_i[S]|}{|S|}|I|^\beta$. 
%\memo{$E_i$}
%This problem represents the trade-off ....????
%For this problem, they designed an algorithm with an approximation ratio of $\frac{1}{2k^\beta}$. 
Very recently, Hashemi, Behrouz, and Lakshmanan~\cite{Hashemi+22} designed a sophisticated core decomposition algorithm,
which they call the \emph{FirmCore decomposition algorithm}.
Their algorithm finds the set of $(k,\lambda)$-FirmCores for all possible $k$ and $\lambda$ in polynomial time,
where $(k,\lambda)$-FirmCore is a maximal subgraph in which every vertex has degree no less than $k$ in the subgraph for at least $\lambda$ layers.
% They proved that the decomposition unfolds an approximate solution to the multilayer densest subgraph problem, with a better approximation ratio than that of Galimberti, Bonchi, and Gullo~\cite{Galimberti+_17} for many instances.
They demonstrated that the decomposition unfolds a better solution to the multilayer densest subgraph problem than the algorithm by Galimberti, Bonchi, and Gullo~\cite{Galimberti+_17} for many instances.

Semertzidis et al.~\cite{Semertzidis+19} introduced another generalization of the densest common subgraph problem, called the Best Friends Forever (BFF) problem,
in the context of evolving graphs with a number of snapshots.
The BFF problem is a series of optimization problems that maximize an \emph{aggregate density} over snapshots,
where the aggregate density is set to be the average/minimum value of the average/minimum degree of vertices over layers.
%Similar to the multilayer densest subgraph problem, 
%they also considered the variant called the On--Off BFF ($\text{O}^2$BFF) problem, which only asks the output to be dense for a part of snapshots. 
They investigated the computational complexity of the problems and designed some approximation or heuristic algorithms.

Recalling that multilayer networks can also be seen as a model of networks with \emph{uncertainty},
we can find some other optimization models related.
Zou~\cite{Zou_13} studied the densest subgraph problem in \emph{uncertain graphs}.
An uncertain graph is a pair of $G=(V,E)$ and $p\colon E\rightarrow [0,1]$, 
where $e\in E$ is present with probability $p(e)$ whereas $e\in E$ is absent with probability $1-p(e)$.
In the problem, given an uncertain graph $G=(V,E)$ with $p$, we seek $S\subseteq V$ that maximizes the expected density.
Zou~\cite{Zou_13} showed that this problem can be reduced to the original densest subgraph problem,
and designed a polynomial-time exact algorithm based on the reduction.
Recently, Tsourakakis et al.~\cite{Tsourakakis+19} introduced a more general optimization problem called the \emph{risk-averse dense subgraph discovery}.

As another example, Miyauchi and Takeda~\cite{Miyauchi_Takeda_18} introduced an optimization problem called the \emph{robust densest subgraph problem}.
In this problem, given an undirected graph $G=(V,E)$ and an edge-weight space $I=\times_{e\in E}[l_e,r_e]\subseteq \times_{e\in E}[0,\infty)$, we are asked to find $S\subseteq V$ that maximizes $\min_{w\in I} \frac{w(S)/|S|}{w(S^*_w)/|S^*_w|}$, where $S^*_w$ is an optimal solution to the densest subgraph problem for $G$ with $w$.
The intuition of this problem is the same as that of ours; this problem also seeks $S\subseteq V$ that is reasonably dense for any $w\in I$.
However, they considered only deterministic solutions and gave a strong hardness result.
%They also introduced an optimization problem, which they refer to as the \emph{robust densest subgraph problem with sampling oracle}, where we have access to an oracle that accepts $e\in E$ and outputs a value drawn from a distribution on $[l_e, r_e]$ in which the expected value is equal to the unknown true edge weight. For this problem, they designed a pseudo-polynomial-time algorithm with quality guarantee. 

%Very recently, Tsourakakis et al.~\cite{Tsourakakis+19} introduced an optimization problem called the \emph{risk-averse dense subgraph discovery}. In this problem, given an uncertain graph $G=(V,E)$ with $p$, we are asked to find $S\subseteq V$ that has not only a large expected density but also a small \emph{risk}. The risk of $S\subseteq V$ is measured by the probability that $S$ is not dense on a given uncertain graph. 
%They demonstrated that this problem can be reduced to the densest subgraph problem with \emph{negative} edge weights (which is NP-hard), 
%and designed an approximation algorithm and scalable heuristics based on the reduction. 

%There are many other variants of the densest subgraph problem. 
%The most well-studied variants are the size-restricted ones~\cite{Andersen_Chellapilla_09,Feige+01,Khuller_Saha_09}. 
%For example, in the \emph{densest $k$-subgraph problem}, given a positive integer $k$ as an additional input, 
%we aim to find $S\subseteq V$ that maximizes the density $w(S)/|S|$ (or simply $w(S)$). 
%It is known that such a restriction makes the problem much harder; in fact, unlike the original problem, the densest $k$-subgraph problem is NP-hard and the current best approximation ratio is just $O(|V|^{1/4+\epsilon})$ for any $\epsilon >0$~\cite{Bhaskara+_10}. 
%The densest subgraph problem and the densest $k$-subgraph problem have been generalized to those on hypergraphs~\cite{Chlamtac+18,Miyauchi+_15}. 
%Moreover, the densest subgraph problem has been considered in some generalized computation models, e.g., dynamic models~\cite{Bhattacharya+_15,McGregor+_15,Nasir+_17}, where the structure of the input graph changes every second, and streaming models~\cite{Angel+_12,Bahmani+_12}, where there is only a limited resource that cannot store the entire input graph. 
%Furthermore, there are some generalizations of the density function $w(S)/|S|$ itself: ones generalizing the numerator~\cite{Miyauchi_Kakimura_18,Tsourakakis_15} and one generalizing the denominator $|S|$~\cite{Kawase_Miyauchi_17} for some specific purposes. 
%For details, see the recent survey~\cite{FM19}. 

As well as dense subgraph discovery, many important primitives for single-layer network analysis have recently been extended to multilayer networks.
Examples include community detection~\cite{Bazzi+16,DeBacco+17,Interdonato+17,Tagarelli+17}, link prediction~\cite{DeBacco+17,Jalili+17}, analyzing spreading processes~\cite{DeDomenico+16,Salehi+15}, and identifying central vertices~\cite{Basaras+19,DeDomenico+15}. %, and mining frequent subgraphs~\cite{Yan+05}. 

%In terms of network analysis with uncertainty, 
%influence maximization~\cite{Chen+16}



