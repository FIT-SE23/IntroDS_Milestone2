%#! pdflatex
%%
%% This is file `sample-sigconf.tex',
%% generated with the docstrip utility.
%%
%% The original source files were:
%%
%% samples.dtx  (with options: `sigconf')
%% 
%% IMPORTANT NOTICE:
%% 
%% For the copyright see the source file.
%% 
%% Any modified versions of this file must be renamed
%% with new filenames distinct from sample-sigconf.tex.
%% 
%% For distribution of the original source see the terms
%% for copying and modification in the file samples.dtx.
%% 
%% This generated file may be distributed as long as the
%% original source files, as listed above, are part of the
%% same distribution. (The sources need not necessarily be
%% in the same archive or directory.)
%%
%% The first command in your LaTeX source must be the \documentclass command.
\documentclass[sigconf,nonacm]{acmart}




%% your usepackages here, for example:
\usepackage{booktabs}
\usepackage[utf8]{inputenc}
\usepackage{xspace}
%\usepackage{amsmath,amssymb,amsthm,mathtools,mathrsfs,color,url,bm}
\usepackage{amsmath,amsthm,mathtools,mathrsfs,color,url,bm}
\usepackage{multirow}
\usepackage{footnote}
\usepackage[whole]{bxcjkjatype}
\usepackage{comment}
\usepackage{tikz}
\usetikzlibrary{arrows,shapes,calc}
\usepackage[linesnumbered,ruled,vlined]{algorithm2e}
%\usepackage[hang,small,bf]{caption}
\usepackage[subrefformat=parens]{subcaption}
\captionsetup{compatibility=false}
\captionsetup[subfigure]{labelformat=simple}
\renewcommand*{\thesubfigure}{(\alph{subfigure})}
\mathtoolsset{showonlyrefs}
\usepackage[referable]{threeparttablex}


\newcommand{\dens}{\texttt{DENSITY}\xspace}
% \newcommand{\mindens}{\texttt{Density}\xspace}
% \newcommand{\ratio}{\texttt{RobustRatio}\xspace}
% \newcommand{\regret}{\texttt{Regret}\xspace}
\DeclareMathOperator*{\argmin}{arg\,min}
\DeclareMathOperator*{\argmax}{arg\,max}
\DeclareMathOperator{\supp}{supp}
\newcommand{\ot}{\leftarrow}
\newcommand{\memo}[1]{{\color{red}[#1]}}


%%
%% \BibTeX command to typeset BibTeX logo in the docs
\AtBeginDocument{%
  \providecommand\BibTeX{{%
    \normalfont B\kern-0.5em{\scshape i\kern-0.25em b}\kern-0.8em\TeX}}}

%% Rights management information.  This information is sent to you
%% when you complete the rights form.  These commands have SAMPLE
%% values in them; it is your responsibility as an author to replace
%% the commands and values with those provided to you when you
%% complete the rights form.
%\setcopyright{acmcopyright}
%\copyrightyear{2023}
%\acmYear{2023}
%\acmDOI{}

%% These commands are for a PROCEEDINGS abstract or paper.
%\acmConference[Anonymous et al.\ '23]{Anonymous et al.\ '23: WSDM 2023}{}{}
%\acmConference{WSDM 2023}{February 27--March 3, 2023}{Singapore}
%\acmBooktitle{Anonymous et al.\ '23: WSDM 2023}
%\acmPrice{15.00}
%\acmISBN{}


\copyrightyear{2023}
\acmYear{2023}
\setcopyright{acmcopyright}\acmConference[WSDM '23]{Proceedings of the Sixteenth ACM International Conference on Web Search and Data Mining}{February 27--March 3, 2023}{Singapore, Singapore}
\acmBooktitle{Proceedings of the Sixteenth ACM International Conference on Web Search and Data Mining (WSDM '23), February 27--March 3, 2023, Singapore, Singapore}
\acmPrice{15.00}
\acmDOI{10.1145/3539597.3570444}
\acmISBN{978-1-4503-9407-9/23/02}


%%
%% Submission ID.
%% Use this when submitting an article to a sponsored event. You'll
%% receive a unique submission ID from the organizers
%% of the event, and this ID should be used as the parameter to this command.
\acmSubmissionID{}

%%
%% The majority of ACM publications use numbered citations and
%% references.  The command \citestyle{authoryear} switches to the
%% "author year" style.
%%
%% If you are preparing content for an event
%% sponsored by ACM SIGGRAPH, you must use the "author year" style of
%% citations and references.
%% Uncommenting
%% the next command will enable that style.
%%\citestyle{acmauthoryear}

%%
%% end of the preamble, start of the body of the document source.

\begin{document}

\title{Stochastic Solutions for Dense Subgraph Discovery\\ in Multilayer Networks}  % put your title here!
%\titlenote{Produces the permission block, and copyright information}

%\author{Yasushi Kawase \and Atsushi Miyauchi \and Hanna Sumita}% Alphabetical order
%% example of author block for camera ready version of accepted papers: don't use for anonymous submissions
%
\author{Yasushi Kawase}
%\authornote{Dr.~Trovato insisted his name be first.}
%\orcid{1234-5678-9012}
\affiliation{%
  \institution{The University of Tokyo}
  \streetaddress{Hongo 7-3-1}
  %\city{Bunkyo-ku}
  \state{Tokyo}
  \country{Japan}
  \postcode{113-8654}
}
\email{kawase@mist.i.u-tokyo.ac.jp}
%
\author{Atsushi Miyauchi}
%\authornote{The secretary disavows any knowledge of this author's actions.}
\affiliation{%
  \institution{The University of Tokyo}
  \streetaddress{Hongo 7-3-1}
  %\city{Bunkyo-ku}
  \state{Tokyo}
  \country{Japan}
  \postcode{113-8654}
}
\email{miyauchi@mist.i.u-tokyo.ac.jp}
%
\author{Hanna Sumita}
%\authornote{This author is the one who did all the really hard work.}
\affiliation{%
  \institution{Tokyo Institute of Technology}
  \streetaddress{Oookayama 2-12-1}
  %\city{Meguro-ku}
  \state{Tokyo}
  \country{Japan}}
\email{sumita@c.titech.ac.jp}


%% The example's default list of authors is too long for headers
%\renewcommand{\shortauthors}{B. Trovato et al.}


\begin{abstract}  % put your abstract here!
  Network analysis has played a key role in knowledge discovery and data mining.
  In many real-world applications in recent years, we are interested in mining \emph{multilayer networks},
  where we have a number of edge sets called \emph{layers},
  which encode different types of connections and/or time-dependent connections over the same set of vertices.
  Among many network analysis techniques, dense subgraph discovery, aiming to find a dense component in a network, is an essential primitive with a variety of applications in diverse domains.
  In this paper, we introduce a novel optimization model for dense subgraph discovery in multilayer networks.
  Our model aims to find a stochastic solution, i.e., a probability distribution over the family of vertex subsets,
  rather than a single vertex subset, whereas it can also be used for obtaining a single vertex subset.
  For our model, we design an LP-based polynomial-time exact algorithm.
  Moreover, to handle large-scale networks, we also devise a simple, scalable preprocessing algorithm, which often reduces the size of the input networks significantly and results in a substantial speed-up.
  Computational experiments demonstrate the validity of our model and the effectiveness of our algorithms.
  %Computational experiments using synthetic graphs and real-world networks demonstrate the validity of our model and the effectiveness of our algorithms.
  %In this paper, we investigate dense subgraph discovery in multilayer networks from a game-theoretic viewpoint.\memo{why?} Specifically, we introduce a Stackelberg game model\memo{who is the leader or the follower? How does it connect to dense subgraph mining in your setting?}, enabling us to consider the uncertainty to layers.\memo{not clear} For our model, we design an LP-based polynomial-time exact algorithm.\memo{to solve what problem?} Moreover, to handle large-scale networks, we also devise a simple, scalable preprocessing algorithm\memo{why do you need to do it? I am guessing your LP is not efficient?}, which often reduces the size of the input networks significantly. Computational experiments using synthetic graphs and real-world networks\memo{would be good to say what they are} demonstrate the validity of our model and the effectiveness of our algorithms. \memo{with respect to what measure?}
\end{abstract}


\begin{CCSXML}
  <ccs2012>
  <concept>
  <concept_id>10003752.10003809.10003635</concept_id>
  <concept_desc>Theory of computation~Graph algorithms analysis</concept_desc>
  <concept_significance>500</concept_significance>
  </concept>
  <concept>
  <concept_id>10003752.10003809.10003716</concept_id>
  <concept_desc>Theory of computation~Mathematical optimization</concept_desc>
  <concept_significance>500</concept_significance>
  </concept>
  </ccs2012>
\end{CCSXML}

\ccsdesc[500]{Theory of computation~Graph algorithms analysis}
\ccsdesc[500]{Theory of computation~Mathematical optimization}

\keywords{network analysis, multilayer networks, dense subgraph discovery, stochastic solutions}  %  
% put your semicolon-separated keywords here!

\maketitle



\section{Introduction}


Accurate estimates of posterior probabilities are crucial for neural networks in various Natural Language Processing (NLP) tasks~\cite{icml17,DBLP:conf/nips/Lakshminarayanan17}. For example, it would be helpful for humans if the models deployed in practice abstain or interact when they cannot make a decision with high confidence~\cite{DBLP:journals/jamia/JiangOKO12}. While Pre-trained Language Models (PLMs) have improved the performance of many NLP tasks~\cite{bert,roberta}, how to better avoid miscalibration is still an open research problem ~\cite{calibration_emnlp20,dan_roth_emnlp21}. 
\begin{table}[t!]
    \centering
    \begin{tabular}{l|p{0.65\columnwidth}}
    \hline

    %  Example 1: & It is \hlc[cyan!10]{a} \hlc[red!40]{warm} \hlc[red!60]{funny} \hlc[red!40]{engaging} \hlc[cyan!20]{film} . \\ \hline
     Positive & a fast \hlc[green!10]{funny} \hlc[green!40]{highly} \hlc[green!80]{enjoyable} movie.\\ \hline
    %  like a south of the border melrose place
     
     Negative & It's about \hlc[red!5]{following} your \hlc[green!10]{dreams} \hlc[red!10]{no} matter \hlc[red!5]{what} your \hlc[green!5]{parents} think.\\
    \hline
  \end{tabular}
    \caption{Two motivating examples with highlight explanations~\cite{SST}. The saturation of the colors signifies the magnitude. The confidence of the model should be easily recognized by looking at token attributions.}
    % \vspace{-4mm}
    \label{tab:example-m}
\end{table}
In this paper, we investigate if and how model explanations can help calibrate the model. 

Explanation methods have attracted considerable research interest in recent years for revealing the internal reasoning processes behind models~\cite{IG,Uncertainty_Aware_Attention,deeplift}. Token attribution scores generated by explanation methods represent the contribution to the prediction~\cite{diagnostic}. Intuitively, one can draw some insight for analyzing and debugging neural models from these scores if they are correctly attributed, as shown in Table~\ref{tab:example-m}. For example, when the model identifies a highly indicative pattern, the tokens involved would have high attribution scores for the predicted label and low attribution scores for other labels. Similarly, if the model has difficulty recognizing the inductive information of any class (i.e., the attribution scores are not high for any label), the model should not be highly confident. As such, the computed explanation of an instance could indicate the confidence of the model in its prediction to some extent.
 
Inspired by this, we propose a simple and effective method named \textbf{CME} that can be applied at training time and improve the performance of the confidence estimates. The estimated confidence measures how confident the model is for a specific example. Ideally, reasonable confidence estimates should have higher confidence for correctly classified examples resulting in higher attributions than incorrect ones. Hence, given an example pair during training with an inverse classification relationship, we regularize the classifier by comparing the wrong example's attribution magnitude and the correct example's attribution magnitude.

Our work is related to recent works on incorporating explanations into learning. Different from previous studies that leverage explanations to help users predict model decisions~\cite{DBLP:journals/corr/abs-2102-02201} or improve the accuracy~\cite{DBLP:conf/icml/RiegerSMY20}, we focus on answering the following question: \textit{are these explanations of black-box models useful for calibration?} If so, how should we exploit the predictive power of these explanations? Considering the model may be uninterpretable due to the nature of neural networks and limitations of explanation method~\cite{Fragile,DBLP:conf/nips/YehHSIR19}, a calibrated model by CME at least can output the unbiased confidence. Moreover, we exploit intrinsic explanation during training, which does not require designing heuristics~\cite{xiye1} and additional data augmentation~\cite{mixup21acl}.
% Are these explanations useful for calibrating the model?

We conduct extensive experiments using BERT~\cite{bert} and RoBERTa~\cite{roberta} to show the efficacy of our approach on three natural language understanding tasks (i.e., natural language inference, paraphrase detection, and commonsense reasoning) under In-Domain (ID) and Out-of-Domain (OD) settings. CME achieves the lowest expected calibration error without accuracy drops compared with strong SOTA methods, e.g.,~\citet{mixup21acl}. When combined with Temperature Scaling (TS)~\cite{icml17}, the expected calibration errors are further reduced as better calibrated posterior estimates under these two settings.



\section{Related Work}

\paragraph{Inverse Rendering of Indoor Scenes} Inverse rendering attempts to reconstruct geometry and spatially-varying material and lighting information from monocular (which is our case) or multiple RGB images. Most previous methods only recognize one or part of the above attributes. Geometry reconstructions, including depth estimation and surface normal reconstruction, has been widely studied \cite{eigen2015predicting,liu2019planercnn}.
Most material reconstruction methods are only able to either estimate diffuse albedo~\cite{li2018cgintrinsics, barron2013intrinsic, karsch2014automatic} or classify material categories~\cite{bell2015material}.
For lighting estimation, recent deep learning methods have made progress in estimating global~\cite{gardner2017learning,gardner2019deep} and even spatially-varying~\cite{garon2019fast,song2019neural,li2020inverse} lighting conditions.
Recent works attempt to predict multiple intrinsics jointly by a holistic inverse rendering framework. Li et al.~\shortcite{li2020inverse} proposed a method to reconstruct disentangled geometry, spatially-varying reflectance and lighting from a single RGB indoor scene image.


\paragraph{Lighting Estimation and Relighting.}
Light estimation is one of the sub-tasks of inverse rendering. Most previous works ignore spatially-varying effects and predict a single environment map for the whole scene \cite{gardner2017learning,sengupta2019neural,munkberg2022extracting}. Indoor scenes suffer from spatial variations, thus recent work explores spatially-varying lighting estimation for indoor scenes. The representation of spatially-varying illumination includes environment maps, per-pixel spherical lobes~\cite{li2020inverse} (spherical Harmonics/Gaussians), or 3D voxel grids~\cite{wang2021learning}. Relighting is also a widely-studied relevant  task. \citet{griffiths2022outcast} leverages screen-space method to detect occlusion and cast shadows to relight an outdoor image. \citet{li2022physically} proposed a novel pipeline to modify the light conditions within an indoor scene.


\paragraph{Neural Scene Representations.} 
Neural representations are a rapidly growing area of research. Recent advances include  voxels~\cite{yu2021plenoxels,sun2021direct}, hashgrids~\cite{muller2022instant}, point clouds~\cite{aliev2020neural}, and neural implicit functions~\cite{mildenhall2020nerf,wang2021neus,yariv2021volume,yariv2020multiview}. 
Neural radiance fields (NeRFs)~\cite{mildenhall2020nerf} represents scenes as neural implicit functions, encoding a scene as a continuous volumetric radiance field of color and density. With volume rendering, a NeRF can synthesize novel view images with promising results. Our proposed method uses a NeRF as the representation of the out-of-view area of the scene (Sec.~\ref{sec:background}).

\paragraph{Differentiable Rendering.} A number of recent inverse rendering works utilize differentiable rendering to recover complex light transport effects. Some recent works have proposed general-purpose physically-based differentiable renderers~\cite{Li:2018:DMC,NimierDavidVicini2019Mitsuba2}. \citet{Zhang:2020:PSDR} and \citet{Zeltner2021MonteCarlo} discussed a rigurous theory of differentiable light transport and Monte-Carlo combinations. These physically-based methods achieve high-quality global illumination effects at the cost of substantial performance overhead. Some differentiable rendering techniques are customized for specific purpose such as texture~\cite{nimier2021material}, split-sum lighting and mesh extraction~\cite{munkberg2022extracting}. Our method designs a Monte-Carlo based in-network differentiable rendering layer to recover the appearance of indoor scenes (Sec.~\ref{sec:render}).

\paragraph{Indoor Scene Datasets.} 
Supervised learning requires a large database of indoor scene images and their corresponding ground truth geometry, material, and lighting for network training. Datasets include 3D shape models~\cite{chang2015shapenet}, real-world scans~\cite{chang2017matterport3d, dai2017scannet}, and scene datasets~\cite{song2017semantic,savva2017minos,li2018interiornet,li2021openrooms}, which can be classified as either real or synthetic data. Real datasets provide real-world images and geometry, while synthetic datasets provide arbitrary scene annotations for inverse rendering, some of which, such as materials and illumination, are difficult to acquire from real world. To the best of our knowledge, InteriorNet~\cite{li2018interiornet} and OpenRooms~\cite{li2021openrooms} are so far the highest-quality public indoor datasets with spatially-varying photorealistic material and illumination annotations. Unfortunately, InteriorNet provides only LDR results, while OpenRooms provides only lighting information on the scene surface (instead of at any 3D location), and lacks the complexity of material and furniture variations. We present a new indoor scene HDR dataset to tackle their shortcomings.


\section{Model}\label{sec:model}
In this section, we formally define our optimization model. % using a zero-sum two-player Stackelberg game. 
Let $G=(V, (E_i)_{i\in [k]})$ be a multilayer network consisting of $k$ layers,
and let $w_i\colon E_i\to\mathbb{R}_{++}$ be a positive edge weight for layer $i$.
We denote by $E$ the union of all edge sets, i.e., $E\coloneqq \bigcup_{i\in [k]} E_i$.
Let $S_i^*$ be a densest subgraph for layer $i$, i.e., $S_i^*\in\argmax_{S\subseteq V}w_i(S)/|S|$.
%% In the game, the leader is our algorithm and the follower is the nature. 
%% Our pure strategy is a vertex subset $S\subseteq V$ and that of the follower is a layer. 
%% We first choose a vertex subset, and then the follower chooses a layer with knowing our choice. 
%% Finally, we receive some payoff of our vertex subset. 
%% We wish to obtain a subgraph with high payoff, whereas the follower chooses the layer that minimizes our payoff. 
Our task is to find a vertex subset that is dense for the layer selected adversarially.
%As mentioned in the introduction, we consider a mixed strategy to compete with the follower.
%In our game, a mixed strategy is a probability distribution over the family of vertex subsets. 
As mentioned in the introduction, we consider a stochastic solution to compete with the adversary.
Let $\Delta(2^V)$ be the set of probability distributions over $2^V$.
For each $p \in \Delta(2^V)$, we denote by $p_S$ the probability of choosing $S \subseteq V$.

%\memo{metricの定義が変.単なるpayoffの置換ではないので書き直すべき.For our stochastic solution, we define a metric }
We aim to compute $p\in \Delta(2^V)$ that maximizes some metric (when the adversary selects the worst layer to $p$).
We employ the following three metrics:

\smallskip
\noindent\textbf{Density.}
\ The first metric is the degree density itself.
%The \emph{minimum density} indicates the worst-case performance of a solution over all weights.
Specifically, when we select a vertex subset according to a probability distribution $p \in \Delta(2^V)$ and the adversary selects a layer $i\in[k]$,
our metric is defined as follows:
\begin{align}
   & %\min_{i\in [k]} 
  \mathbb{E}_{S\sim p}\left[ \frac{w_i(S)}{|S|}\right]
  \quad\Bigg(=
  %\min_{i\in [k]} 
  \sum_{S\subseteq V} p_S \frac{w_i(S)}{|S|}\Bigg).
  \label{eq:mindens}
\end{align}
As the adversary selects the worst layer,
we aim to find $p\in \Delta(2^V)$ that maximizes the minimum of the density~\eqref{eq:mindens} among $i\in [k]$.
Our optimization model with this metric can be seen as a stochastic version of the densest common subgraph problem introduced by Jethava and Beerenwinkel~\cite{JB2015}.

%It is worth mentioning that the problem of maximizing the minimum of the density~\eqref{eq:mindens} has the same optimal value as that of a problem arising naturally from dense subgraph discovery in multilayer networks.
%The maximum value of the minimum of \eqref{eq:mindens} is the game value of the zero-sum Stackelberg game.
%By the von Neumann's minimax theorem, we have
%\[
%  \max_{p\in\Delta(2^V)}\min_{i\in [k]}\mathbb{E}_{S\sim p}\left[ \frac{w_i(S)}{|S|}\right]=
%  \min_{c \in \Delta(k)} \max_{S \subseteq V} \frac{\sum_{i\in [k]} c_i \cdot w_i(S)}{|S|},
%\]
%where $\Delta(k)=\{ c\in\mathbb{R}_+^k \mid \sum_{i \in [k]} c_i = 1\}$. % c'\ge 0
%The latter value is the worst maximum density attained when we aggregate all layers into a single-layer graph whose weight is a convex combination of $w_i$'s with coefficients $c_i$ $(i\in [k])$.



\smallskip
\noindent\textbf{Robust ratio.}
\ The \emph{robust ratio} is a metric based on the ratio of the expected degree to the optimal degree. %\memo{robust ratioは本当は最悪のlayerをとったもの}
%\ The \emph{robust ratio} is a metric that deals with a multiplicatively normalized version of the degree density. %\memo{robust ratioは本当は最悪のlayerをとったもの}
%The \emph{robust ratio} is a normalized version of the worst-case performance of a solution with respect to all weights.
For a nonempty $S\subseteq V$ and $i\in[k]$, let us consider a normalized density defined as $\frac{w_i(S)/|S|}{w_i(S_i^*)/|S_i^*|}$.
%minimizing the difference to the objective function of the best solution that would have been possible in a scenario.
When we select a vertex subset according to a probability distribution $p\in \Delta(2^V)$ and the adversary selects a layer $i\in [k]$,
the metric is defined as follows:
\begin{align}
  %\min_{i\in [k]} 
  \mathbb{E}_{S\sim p}\left[ \frac{w_i(S)/|S|}{w_i(S_i^*)/|S_i^*|}\right]
  \quad\Bigg(=
  %\min_{i\in [k]} 
  \sum_{S\subseteq V}p_S \frac{w_i(S)/|S|}{w_i(S_i^*)|S_i^*|}\Bigg).
  \label{eq:ratio}
\end{align}
In other words, the robust ratio is equivalent to
the density for the multilayer network with weights $w_1',\dots,w_k'$ given by $w_i'(e)\coloneqq \frac{w_i(e)}{w_i(S_i^*)/|S_i^*|}$ for each $i\in[k]$ and $e \in E_i$.
As the adversary selects the worst layer,
we aim to find $p\in \Delta(2^V)$ that maximizes the minimum of~\eqref{eq:ratio} among $i\in [k]$.
Note that the optimal robust ratio is contained in the interval $[1/k,\,1]$ 
because $p\in \Delta(2^V)$ such that $p_{S_i^*}=1/k$ for each $i\in [k]$ has the objective value of $1/k$. 


\smallskip
\noindent\textbf{Regret.}
\ The \emph{regret} is a metric based on the difference between the optimal density and the expected density.
For $S\subseteq V$ and $i\in [k]$, the regret is defined as $w_i(S^*_i)/|S^*_i|-w_i(S)/|S|$.
%For $S\subseteq V$, the regret is defined as $w_i(S)/|S|-w_i(S^*_i)/|S^*_i|$. 
When we select a vertex subset according to a probability distribution $p\in \Delta(2^V)$ and the adversary selects a layer $i\in [k]$,
the metric is defined as follows:
\begin{align}
  %\max_{i\in [k]} 
  \mathbb{E}_{S\sim p}\left[ \frac{w_i(S_i^*)}{|S_i^*|}-\frac{w_i(S)}{|S|}\right]
  \quad\Bigg(=
  %\max_{i\in [k]} 
  \frac{w_i(S_i^*)}{|S_i^*|}-\sum_{S\subseteq V}p_S \frac{w_i(S)}{|S|}\Bigg).
  \label{eq:regret}
\end{align}
As the adversary selects the worst layer,
we aim to find $p\in \Delta(2^V)$ that minimizes the maximum of~\eqref{eq:regret} among $i\in [k]$.

\smallskip
Here we explain how to select an appropriate metric.
The density and regret metrics are useful when we are concerned with multilayer networks with homogeneous layers 
such as time-dependent follower-followee relations in the Twitter network. 
%In particular, the density metric is more appropriate if we aim to find vertex subsets that are reasonably dense for \emph{all} layers, 
Although the density metric can be the first choice, the regret metric is more suitable for robust analysis. 
For example, consider the case where there are a number of layers consistent with each other together with some noisy (e.g., random) layers. 
The density metric would suffer from the effect of the noisy layers, but the regret metric would avoid it and find dense subgraphs in the other meaningful layers. 
On the other hand, the robust ratio metric is useful when we analyze multilayer networks with heterogeneous layers 
such as brain networks with structural and functional connectivity layers.
From its definition, the robust ratio metric would find subgraphs that are reasonably dense for all layers. 
The density and regret metrics focus only on the layers with small optimal densities and the layers with large optimal densities, respectively. 

%Here we explain how to select an appropriate metric.
%The density and regret metrics would be useful when the layers come from homogeneous samples
%such as time-dependent follower--followee relations in the Twitter network,
%while the robust ratio metric would be useful when the layers come from heterogeneous samples
%such as brain networks with structural and functional connectivity layers.
%It is likely that the density preferentially cares layers with relatively low optimal densities
%but the regret preferentially cares layers with relatively high optimal densities.




%%\begin{figure}
%%    \centering
%%    \begin{tikzpicture}[]
%%    % vertex
%%    \foreach \pos/\name in {{(-2,1)/a},{(-2,-1)/b},{(0,0)/c},{(2,0)/d}}
%%        \node (\name) at \pos {};
%%    % edge
%%    %\draw[draw=blue, line width=1pt] ($(a)+(-0.05,0)$)--($(b)+(-0.05,0)$);
%%    \draw[draw=blue,  line width=1pt] ([xshift=-2pt]a.center)--([xshift=-2pt]b.center);
%%    \draw[draw=blue, line width=1pt] (a)--(c);
%%    \draw[draw=blue, line width=2pt] (b)--(c);
%%    %\draw[draw=red,  line width=1pt] ($(a)+(0.1,0)$)--($(b)+(0.1,0)$);
%%    \draw[draw=red,  line width=1pt] ([xshift=2pt]a.center)--([xshift=2pt]b.center);
%%    \draw[draw=red,  line width=2pt] (c)--(d);
%%    \foreach \source/\dest/\name in {{a/b/e_1},{a/c/e_2},{b/c/e_3},{c/d/e_4}}
%%        \path (\source) -- node[font=\small,rectangle,fill=white,opacity=0.9, text opacity=1,inner sep=2pt,rounded corners] {$\name$} (\dest);
%%    \foreach \pos/\name in {{(-2,1)/a},{(-2,-1)/b},{(0,0)/c},{(2,0)/d}}
%%        \node[circle,draw,minimum size=15,inner sep=0,fill=black!5] at \pos {$\name$};
%%    \end{tikzpicture}
%%    \caption{A simple multilayer network.}
%%    \label{fig:example1}
%%    \centering
%%    \begin{tikzpicture}[very thick,scale=3.0,
%%    dline/.style={densely dotted,thick},
%%    p/.style={circle,fill=black,draw,inner sep=0pt,minimum size=3pt}]
%%    \draw node[below left] {$O$};
%%    % convex hull
%%    \draw[thin,fill=black!10] (0,1) -- (1,0.75) -- (1.3333,0.3333) -- (1,0) -- (0,0) -- (0,1);
%%    % Q_{0.8}
%%    \draw[draw=blue!10,fill=blue!10] (1.7,0.8) -- (0.8,0.8) -- (0.8,1.2) -- (1.7,1.2) -- (1.7,0.8);
%%    \draw[thin,blue] (1.7,0.8) -- (0.8,0.8) -- (0.8,1.2);
%%    \node[blue] at (1.25,1) {$Q_{0.8}$};
%%    
%%
%%    \foreach \x/\y in {0.0000/1.0000,0.5000/0.0000,0.0000/0.0000,0.3333/0.3333,0.5000/0.5000,0.3333/0.6667,0.6667/0.6667,1.0000/0.7500,1.3333/0.3333,1.0000/0.0000}{
%%      \node[p] at (\x,\y) {}; 
%%    }    
%%
%%    % \node[above right,font=\tiny] at (0,1) {$\{c,d\}$};
%%    % \node[above right,font=\tiny] at (1,0.75) {$\{a,b,c,d\}$};
%%    % \node[right,font=\tiny] at (1.3333,0.3333) {$\{a,b,c\}$};
%%
%%    \node[left] at (0,1) {\small$1$};
%%    \draw[dline] (1,0.75) -- (0,0.75) node[left,yshift=-2pt] {\small$3/4$};
%%    \draw[dline] (1,0.75) -- (1,0) node[below] {\small$1$};
%%    \draw[dline] (1.3333,0.3333) -- (0,0.3333) node[left] {\small$1/3$};
%%    \draw[dline] (1.3333,0.3333) -- (1.3333,0) node[below] {\small$4/3$};
%%    %\draw[dline] (1.3333,1) -- (0,1);
%%
%%    \draw[dline] (0.8,0.8) -- (0,0.8) node[left,yshift=2pt] {\small$4/5$};
%%    \draw[dline] (0.8,0.8) -- (0.8,0) node[below] {\small$4/5$};
%%
%%    \node[p,diamond,minimum size=3pt,,red,label={[xshift=5pt,yshift=-2pt]\color{red}\small$p^{1}$}] at (0.8,0.8) {};
%%    \node[p,diamond,minimum size=3pt,red,label={[below left]\color{red}\small$p^{2}$}] at (1,0.75) {};
%%    \node[p,diamond,minimum size=3pt,,red,label={[right]\color{red}\small$p^{3}$}] at (1.037,0.7037) {};
%%    
%%    %\node[left,yshift=2pt] at (0,0.8) {\small$0.8$};
%%    %% \path[draw,thick,densely dotted,red,<->] (0,0.8) -- node[font=\tiny,rectangle,fill=white,opacity=0.9, text opacity=1,inner sep=2pt,rounded corners] {$0.8$} (0.8,0.8);
%%    %% \path[draw,thick,densely dotted,red,<->] (0.8,0) -- node[font=\tiny,rectangle,fill=white,opacity=0.9, text opacity=1,inner sep=2pt,rounded corners] {$0.8$} (0.8,0.8);
%%    %% \path[draw,thick,densely dotted,red,<->] (1.037,0.7037) -- node[font=\tiny,rectangle,fill=white,opacity=0.9, text opacity=1,inner sep=2pt,rounded corners] {$8/27$} (1.3333,0.7037);
%%    %% \path[draw,thick,densely dotted,red,<->] (1.037,0.7037) -- node[yshift=7pt,font=\tiny,rectangle,fill=white,opacity=0.9, text opacity=1,inner sep=2pt,rounded corners] {$8/27$} (1.037,1);
%%    \draw[->] (-.2,0) -- (1.7,0) node[below] {$\frac{w_1(S)}{|S|}$};
%%    \draw[->] (0,-.2) -- (0,1.2) node[left] {$\frac{w_2(S)}{|S|}$};
%%    \end{tikzpicture}
%%    \caption{$p^1$, $p^2$, $p^3$ are optimal solutions for the density, the robust ratio, and the regret, respectively.}
%%    \label{fig:pareto}
%%\end{figure}
%%
%%Next we observe the behavior of our optimization model with the above three metrics through a simple example. 
%%Let $G=(V,\{E_1,E_2\})$ be a multilayer network consisting of $V=\{a,b,c,d\}$, $E_1=\{e_1,e_2,e_3\}$, and $E_2=\{e_1,e_4\}$,
%%where $e_1=\{a,b\}$, $e_2=\{a,c\}$, $e_3=\{b,c\}$, and $e_4=\{c,d\}$ (see Figure~\ref{fig:example1}).
%%There are two edge weights $w_1$ and $w_2$ such that 
%%\begin{align*}
%%w_1(e_1)=1,\ w_1(e_2)=1,\ w_1(e_3)=2,\ w_2(e_1)=1,\ w_2(e_4)=2.
%%\end{align*}
%%The maximum density in terms of $w_1$ is $4/3$ (attained by $\{a,b,c\}$) and that in terms of $w_2$ is $1$ (attained by $\{c,d\}$).
%%
%%The optimal solution to our model with the density metric is $p^1\in \Delta(2^V)$ taking $\{a,b,c,d\}$ with probability $4/5$ and $\{c,d\}$ with probability $1/5$.
%%The optimal solution for the robust ratio metric is $p^2\in \Delta(2^V)$ taking $\{a,b,c,d\}$ with probability $1$.
%%The optimal solution for the regret metric is $p^3\in \Delta(2^V)$ taking $\{a,b,c,d\}$ with probability $8/9$ and $\{a,b,c\}$ with probability $1/9$.
%%
%%We can see the optimality of those solutions from the geometric viewpoint. 
%%In Figure~\ref{fig:pareto}, black points represent $\left(\frac{w_1(S)}{|S|}, \frac{w_2(S)}{|S|}\right)$ for all subsets $S \subseteq V$. 
%%In this plot, each $p \in \Delta(2^V)$ corresponds to a point $\Bigl(\mathbb{E}_{S\sim p} \bigl[\frac{w_1(S)}{|S|}\bigr], \mathbb{E}_{S\sim p} \bigl[\frac{w_2(S)}{|S|}\bigr]\Bigr)$ contained in the convex hull $H$ of the black points.
%%We plot the red points corresponding to $p^i$ for $i=1,2,3$. 
%%Let us focus on the density metric. 
%%The set of $p \in \Delta(2^V)$ that has the objective value of at least $t$ corresponds to the intersection of $H$ and $Q_t=\{\bm{q} \mid q_1\geq t,\ q_2\geq t \}$. 
%%In this plot, $H\cap Q_t = \emptyset$ when $t> 0.8$, and the point corresponding to $p^1$ lies in $H \cap Q_{0.8}$.
%%Therefore, we see that $p^1$ is an optimal solution for the degree metric. 
%%We can also check the optimality of $p^2$ and $p^3$ in a similar manner.
%%Recall that the maximum density for $w_1$ is $4/3$, and that for $w_2$ is $1$. 
%%For the robust ratio metric and the regret metric, we consider $Q'_t = \{ \bm{q} \mid \frac{3}{4} q_1 \geq t,\ q_2 \geq t \}$ and $Q''_t = \{ \bm{q} \mid \frac{4}{3}-q_1 \leq t,\ 1-q_2 \leq t \}$, respectively, instead of $Q_t$ in the above discussion.



\smallskip
\noindent\textbf{Unified concept: $(\bm{\alpha},\bm{\beta})$-density.}
\ Here we introduce a general metric, enabling us to deal with the above three metrics in a unified manner.
An important fact is that the robust ratio and regret metrics can be obtained by affine transformations of the density.
Specifically, when we select a subgraph according to a probability distribution $p \in \Delta(2^V)$ and the adversary selects a layer $i\in[k]$,
we define the \emph{$(\bm{\alpha},\bm{\beta})$-density} using two vectors $\bm{\alpha} \in \mathbb{R}^k_+$ and $\bm{\beta} \in \mathbb{R}^k$ as follows:
\begin{align}
  %\min_{i\in [k]} 
  \mathbb{E}_{S\sim p}\left[\alpha_i\frac{w_i(S)}{|S|}+\beta_i\right]
  \quad\Bigg(=
  %\min_{i\in [k]} 
  \alpha_i\sum_{S\subseteq V}p_S\frac{w_i(S)}{|S|}+\beta_i\Bigg).
  %\ \Bigg(=\max_{i\in [k]} \Bigl[ \alpha_i\sum_{S\subseteq V}p_S\frac{w_i(S)}{|S|}+\beta_i\Bigr]\Bigg).
  \label{eq:general}
\end{align}
%\memo{後ろの解析では,最悪のレイヤーをとったものをこのように呼んでいる箇所があり,統一する必要あり}
% adversaryがiを取るということは,iが最悪レイヤーなのでよいのでは?
Note that the above three metrics, the density, robust ratio, and regret, are equivalent to
the $(\bm{1},\bm{0})$-density,
$((|S_i^*|/w_i(S_i^*))_{i\in[k]},\bm{0})$-density, and
$(\bm{1},(-w_i(S_i^*)/|S_i^*|)_{i\in[k]})$-density\footnote{The actual regret value is the negation of $(\bm{1},(-w_i(S_i^*)/|S_i^*|)_{i\in[k]})$-density.}, respectively. 
Note that $w_i(S_i^*)/|S_i^*|$ is polynomially computable for each $i\in [k]$~\cite{Charikar2000,Goldberg_84}.
%$(\bm{1},(-|S_i^*|/w_i(S_i^*))_{i\in[k]})$-minimum density.\footnote{The actual regret value is the negation of $(\bm{1},(-|S_i^*|/w_i(S_i^*))_{i\in[k]})$-minimum density.}
We refer to $(\bm{\alpha},\bm{\beta})$-\dens as the problem of finding $p\in \Delta(2^V)$ that maximizes the minimum of the $(\bm{\alpha},\bm{\beta})$-density among $i\in [k]$.
Therefore, in the following, we aim to design an algorithm for $(\bm{\alpha},\bm{\beta})$-\dens.



%--------------------------------------------------------------------------------------
%--------------------------------------------------------------------------------------
\subsection{The algorithm}
%--------------------------------------------------------------------------------------

 {We} consider that $X$ follows the mixture model defined by \eqref{def:RMM} and consider $X_{1} , \ldots ,X_{n}$ i.i.d copies of $X$.  We now consider the "empirical fixpoint function", i.e we will consider, denoting $\tau = \left( \tau_{1} , \ldots , \tau_{k} \right)$, and $\tau_{k} = \left( \tau_{1,k} , \ldots , \tau_{n,k} \right)$,
%\SR{
%\begin{align*}
%& \hat{g}_{2,k} \left( \tau_{k} ,  m_{k} \right)= \frac{\sum_{i=1}^{n} \tau_{i,k} \frac{X_{i}}{\left\| X_{i} - m_{k} \right\|} }{\sum_{i=1}^{n} \tau_{i,k} \frac{1}{\left\| X_{i} - m_{k} \right\|}}\\
%& \hat{g}_{3,k} \left( \tau_{k} ,  m_{k} , V_{k} \right)= \frac{\sum_{i=1}^{n}\tau_{i,,k}\frac{\left( X_{i} - m_{k} \right) \left( X_{i} - m_{k} \right)^{T}}{\left\| \left( X_{i} - m_{k} \right) \left( X_{i} - m_{k} \right)^{T} - V_{k} \right\|_{F}}}{\sum_{i=1}^{n} \tau_{i,k} \frac{1}{\left\| \left( X_{i} - m_{k} \right) \left( X_{i} - m_{k} \right)^{T} - V_{k} \right\|_{F}}} .
%\end{align*}
%}
{
\begin{align*}
\hat{g}_{2,k} \left( \tau_{k} ,  m_{k} \right)
& = \left({\sum_{i=1}^{n} \frac{\tau_{i,k} X_{i}}{\left\| X_{i} - m_{k} \right\|} } \right) \left/ \left({\sum_{i=1}^{n} \frac{\tau_{i,k}}{\left\| X_{i} - m_{k} \right\|}} \right) \right. \\
\hat{g}_{3,k} \left( \tau_{k} ,  m_{k} , V_{k} \right)
& = \left( {\sum_{i=1}^{n}\frac{ \tau_{i,,k} \left( X_{i} - m_{k} \right) \left( X_{i} - m_{k} \right)^{T}}{\left\| \left( X_{i} - m_{k} \right) \left( X_{i} - m_{k} \right)^{T} - V_{k} \right\|_{F}}} \right) \left/ \left({\sum_{i=1}^{n} \frac{\tau_{i,k}}{\left\| \left( X_{i} - m_{k} \right) \left( X_{i} - m_{k} \right)^{T} - V_{k} \right\|_{F}}} \right) \right..
\end{align*}
}
This leads to the following algorithm:

\begin{algorithm}[Fix Point algorithm for Robust Mixture Model]
  Starting from $\phi^0 = (\pi^0, m^0, V^0)$, repeat until convergence:
  \begin{enumerate}
  \item Compute for each $1 \leq i \leq n$ and $1 \leq k \leq K$
  $$
  \tau_k^{h+1}(X_i) = \frac{\pi_{k}^{h}\phi_{m_{k}^{h}, \hat{\Psi}_{u} \left( V_{k}^{h} \right)} \left( X_{i} \right)}{\sum_{\ell = 1}^{K} \pi_{\ell}^{h}\phi_{m_{\ell}^{h}, \hat{\Psi}_{u} \left( V_{\ell}^{h} \right)} \left( X_{i} \right)} ,
  $$
  where $\hat{\Psi}_{U}$ is the application which enables, given $V_{k}$, to "rebuild" $\Sigma_{k}$ with the help of one of the method proposed in Section \ref{sec:variance}; 
  \item Based on the fix point relations (see Proposition \eqref{prop:fixpoint}), update, for each $1 \leq k \leq K$,
  $$
  \pi^{h+1}_k = \frac1n \sum_{i=1}^n \tau_k^{h+1}(X_i), \qquad
  m^{h+1}_k = \text{FixPoint}\left( \widehat{g}_{2k}(\tau_{k}^{h},.) \right), \qquad
  V^{h+1}_k =\text{FixPoint}\left( \widehat{g}_{3k}(\tau_{k}^{h},m_{k}^{h},.)\right).
  $$
  where $\text{FixPoint}\left(f(.) \right)$ denotes the fix point of the functional $f$.
  \end{enumerate}
\end{algorithm}
Note that estimating the fix points leads to estimate the weighted median and MCM considering weights $\tau_{k}^{h}$. More intuitively, this algorithm consists in updating $\tau_{i,k}$ replacing the empirical mean and variance of each class by their robust estimates based on the median and the MCM of each class, before updating $\pi$ (as usually).


%--------------------------------------------------------------------------------------
%--------------------------------------------------------------------------------------
\subsection{Choosing the number of clusters}
%--------------------------------------------------------------------------------------

To determine the number of clusters $K$, we resort to two standard penalized-likelihood criteria, namely BIC (\cite{Sch78}) and ICL (\cite{BCG00,MaP00}).
More specifically, denoting by $D_K$ the number of independent parameters involved in a mixture with $K$ clusters and by $\widehat{\mathcal{L}}_K(X)$ the log-likelihood of the dataset $X$ evaluated with the parameter estimates resulting from the proposed estimation procedure:
$$
\widehat{\mathcal{L}}_K(X) = \sum_{i=1}^n \log\left(\sum_{k=1}^K \widehat{\pi}_k \phi_{\widehat{\mu}_k, \widehat{\Sigma}_k}(X_i)\right), 
$$
we used
\begin{equation} \label{eq:modelSel}
  BIC(K) = \widehat{\mathcal{L}}_K(X) - \log(n) D_K/2, \qquad 
  ICL(K) = BIC(K) + \sum_{i=1}^n \sum_{k=1}^K \widehat{\tau}_{i, k} \log \widehat{\tau}_{i, k}.
\end{equation}
We remind that the additional penalty term in the ICL criterion corresponds to the entropy of the conditional distribution of the latent variables $\{Z_i\}_{1 \leq i \leq n}$, conditional on the observed ones $\{X_i\}_{1 \leq i \leq n}$.  {This additional penalty is supposed to favor clusterings with lower classification uncertainty.}

%--------------------------------------------------------------------------------------
\subsection{Initialization of the algorithm}
%--------------------------------------------------------------------------------------
 {
We considered two ways of initializing the algorithm:
\begin{enumerate}
\item[•] Use the robust hierarchical clustering proposed by \cite{gagolewski2016genie}, to get $\tau^{1}$, and run our algorithm from there ;
\item[•] Randomly choose $K$ centers from the data and take $\Sigma_{k} = I_{d}$ and $\pi_{k} = \frac{1}{K}$ for all $k$. 
\end{enumerate}
Remark that the later way can tried several times, so to keep initialization leading to the best final log-likelihood.
We may also use the two ways and keep the best result in term of log-likelihood. 
}
 
% %--------------------------------------------------------------------------------------
% \subsubsection{Initialization of the algorithm}
% %--------------------------------------------------------------------------------------
% Two way for initializing the algorithm are considered: 
% \begin{itemize}
% \item[•] One can initialize the algorithm considering the clustering given by the robust hierarchical clustering proposed by \cite{gagolewski2016genie}, which enables to have $\tau^{1}$, and one can run the end of the algorithm.
% \item[•] One can chose randomly $K$ centers from the data and take $\Sigma_{k} = I_{d}$ and $\pi_{k} = \frac{1}{K}$ for all $k$. Remark that this can be done for several random choice, and one can take the initialization leading to the best final log-likelihood.
% \end{itemize} 
% Remark that one chose these two kind of initialization and take the best choice (in term of maximizing the log-likelihood).

% \SR{
% %--------------------------------------------------------------------------------------
% \subsubsection{Modification of the estimates of $\tau$ and $\pi$}
% %--------------------------------------------------------------------------------------
% The following procedure has been chosen to calculate $\tau_{i,.}^{h+1}$:
% \begin{align*}
% \tau_{i,k}^{h+1/3} & = \max \left\lbrace  \phi_{m_{k}^{h},\hat{\Sigma}_{U}\left( V_{k}^{h}  \right)}\left( X_{i} \right),  e^{-100} \right\rbrace & \forall k \\
% \tau_{i,k}^{h+2/3} & = \frac{\pi_{k}^{h}\tau_{i,k}^{h+1/3}}{\sum_{\ell = 1}^{K}\pi_{\ell}^{h}\tau_{i,\ell}^{h+1/3}} \\
% \tau_{i,k}^{h+1} & = \frac{\tau_{i,k}^{h+2/3} + \epsilon_{\Pi}}{\sum_{\ell = 1}^{K}\left(  \tau_{i,\ell}^{h+2/3} + \epsilon_{\Pi} \right)}
% \end{align*}
% }{[à mettre en annexe ?]}


% \SR{
% %--------------------------------------------------------------------------------------
% \subsubsection{Modification of the robust estimates of the variances}
% %--------------------------------------------------------------------------------------
% Remark that gradient and Robbins-Monro methods can lead to negative estimates of the eigenvalues of the variance (due to estimation error). In order to ensure that the variance of each cluster is positive, and considering $\left( \hat{\lambda}_{1,k} , \ldots ,\hat{\lambda}_{d,k} \right)$ the eigenvalues obtained with the help of MCM combined with one of the Monte Carlo methods, we replace this vector by $\left( \max \left\lbrace \hat{\lambda}_{1,k} ,  \epsilon_{\text{v}} \right\rbrace , \ldots , \max \left\lbrace \hat{\lambda}_{d,k} , \epsilon_{\text{v}} \right\rbrace \right)$ with $\epsilon_{\text{v}}$ chosen arbitrarilly small.
% }{[à mettre en annexe ?]}

% \SR{
% %--------------------------------------------------------------------------------------
% \subsubsection{Chosing the number of clusters}
% %--------------------------------------------------------------------------------------
% The number of clusters is chosen minimizing the ICL criterion. Remark that .... parler des outliers...
% }{[section dédiée dans les simuls]}

\section{Analysis}
\label{sec:analysis}
We will assume, following earlier discussion, that $\LG = \emptyset$
and focus on the case when the algorithm proceeds to the TreeRounding
step. Let $H$ denote the set of edges that satisfies
$(p,q)$-flex-connectivity for the given pairs. {\bf Augment-LP} is a
cut covering LP. Consider any violated cut $S$ with respect to $H$;
$S$ is violated because $S$ separates a pair $(s_i,t_i)$ and
$\delta_H(S)$ has exactly $(p+q)$ edges, of which at most $p-1$ are
safe. Let $F= \delta_H(S)$. We call $F$ a violating edge set. There
are at most $\binom {|H|} {p+q}$ violating edge sets, and since $|H|
\le n^2$, this is upper bounded by $O(n^{2(p+q)})$.
We say that a set of edges $H' \subseteq E \setminus H$ is a feasible
augmentation for violating edge set $F$ if for each pair $(s_i,t_i)$,
there is a path from $s_i$ to $t_i$ in the graph $(H\cup H')\setminus
F$. The following is a simple observation.

\begin{claim}
$H' \subseteq E \setminus H$ is a feasible solution to the
  augmentation problem iff for each violating edge set $F$, $H'$ is a
  feasible augmentation for $F$.
\end{claim}


The preceding observations allows us to focus on a fixed violating
edge set $F$, and ensuring that the algorithm outputs a set $H'$ that
is a feasible augmentation for $F$ with high probability. We observe
that the algorithm is oblivious to $F$. Thus, if we obtain a high
probability bound for a fixed $F$, since there are $O(n^{2(p+q)})$
violating edge sets, we can use the union bound to argue that $H'$ is
feasible solution for \emph{all} violating edge sets.  For the
remainder of this section, until we do the final cost analysis, we
work with a fixed violating edge set $F$.

Consider a tree $(\calT,\calM, y)$ in the \racke distribution for the
graph $G$ with capacities $\tilde x$. We let $\calM^{-1}(F)$ denote
the set of all tree edges corresponding to edges in $F$,
i.e. $\calM^{-1}(F) = \cup_{e \in F} \calM^{-1}(e)$.  We call $(\calT,
\calM, y)$ \emph{good} with respect to $F$ if $y(\calM^{-1}(F)) \leq
\frac 1 2$; equivalently, $F$ blocks a flow of at most $\frac 1 2$ in
$\calT$.

\begin{lemma}
\label{lemma:good_tree}
For a violating edge set $F$, a randomly sampled R\"{a}cke tree $(\calT, \calM, y)$ is good with respect to $F$ with probability at least $\frac 1 2$.
\end{lemma}
\begin{proof}
For each $e \in F$, $\tilde x_e =\frac 1 {4(p+q)\beta}$. Since the expected congestion of each edge is at most $\beta$, $\E[\load(e)] \leq \beta \tilde x_e \le \frac 1 {4(p+q)}$ for each $e \in F$. Note that $y(\calM^{-1}(F)) = \sum_{e \in F} \load(e)$, hence by linearity of expectation, $\E[y(\calM^{-1}(F)] = \sum_{e \in F} \E[\load(e)] \leq |F|\frac 1 {4(p+q)} = \frac 1 4$. Applying Markov's inequality to $y(\calM^{-1}(F))$ proves the lemma.
\end{proof}

Given the preceding lemma, a natural approach is to sample a good tree $\calT$ and hope that $\calT \setminus M^{-1}(F)$ still has good flow between each terminal pair. However, since we rounded down all edges in $\LG \cup H$, it is possible that $\calM^{-1}(F)$ contains an edge whose removal would disconnect a terminal pair in $\calT$, even if $\calT$ is good. See \cite{ChenLLZ22} for a more detailed discussion and example.

We note that our goal is to find a set of
edges $H' \subseteq E$ such that each $s_i$ to $t_i$ has a path in
$(H' \cup H) \setminus F$; these paths must exist in the original
graph, even if they do not exist in the tree. Therefore, instead of
looking directly at paths between $s_i$ and $t_i$ in $\calT$, we focus
on obtaining paths through components that are already connected in
$(V(G), H \setminus F)$. The rest of the argument is to show that
sufficiently many iterations of TreeRounding on any good tree $\calT$
for $F$ will yield a feasible set $H'$ for $F$.

\subsection{Shattered Components, Set Connectivity and Rounding}
Let $\Q_F$ be the set of connected components in the subgraph induced by $H \setminus F$. We use vertex subsets to denote components.
Let $\calT$ be a good tree for $F$. We say that a connected component $Q \in \Q_F$ is \emph{shattered} if it is disconnected in $\calT \setminus \calM^{-1}(F)$, else we call it \emph{intact}. For each $i \in [k]$, let $Q_{s_i} \in \Q_F$ be the component containing $s_i$, and $Q_{t_i} \in \Q_F$ be the component containing $t_i$. Note that $Q_{s_i}$ may be the same as $Q_{t_i}$ for some $i$, but if $F$ is a violating edge set then there is at least one $i$ such that $Q_{s_i} \neq Q_{t_i}$. Now, we define a Set Connectivity instance that is induced by $F$ and $\calT$. Consider two disjoint vertex subsets $A,B \subset V$.
We say that $(A,B)$ partitions the set of shattered components if each shattered component $Q$ is fully contained in $A$ or fully contained in $B$. 
Formally let 
$$Z_F = \{(A \cup Q_{s_i}, B \cup Q_{t_i}): (A, B)
\text{ partitions the shattered components}, i \in [k]\}.$$ In other
words, $Z_F$ is set of all partitions of shattered components that
separate some pair $(s_i,t_i)$.  Since the leaves of $\calT$ are in
one to one correspondence with $V$ we can view $Z_F$ as inducing a Set
Connectivity instance in $\calT$; technically we need to consider the
pairs $\{(\calM^{-1}(A),\calM^{-1}(B)) \mid (A,B) \in Z_F\}$; however,
for simplicity we conflate the leaves of $\calT$ with
$V$.  We claim that it suffices to find a feasible solution that
connects the pairs defined by $Z_F$ in the tree $\calT$.

\begin{claim}
\label{claim:shattered_suffices}
Let $E' \subseteq \calT \setminus \calM^{-1}(F)$. Suppose there exists a path in $E' \subseteq \calT \setminus \calM^{-1}(F)$ connecting $A$ to $B$ for all $(A, B) \in Z_F$. Then, there is an $s_i$-$t_i$ path for each $i \in [k]$ in $(\calM(E') \cup H) \setminus F$. 
\end{claim}
\begin{proof}
Let $E' \subseteq \calT \setminus \calM^{-1}(F)$ such that there is a path from $A$ to $B$ in $E'$ for each $(A, B) \in Z_F$. Assume for the sake of contradiction that $\exists i \in [k]$ such that $(s_i, t_i)$ are disconnected in $(\calM(E') \cup H) \setminus F$. Then, there must be some cut $S$ such that $\delta_{(\calM(E') \cup H) \setminus F}(S) = \emptyset$ and $|S \cap \{s_i, t_i\}| = 1$.

We observe that no component $Q \in \Q_F$ can cross $S$ since each $Q$
is connected in $H\setminus F$. Assume without loss of generality that
$s_i \in S$. Then, let $A = Q_{s_i} \cup \{Q \in \Q_F: Q \text{ is
  shattered }, Q \subseteq S\}$, and $B = Q_{t_i} \cup \{Q \in \Q_F: Q
\text{ is shattered }, Q \subseteq \overline S\}$. Clearly, $A
\subseteq S$, $B \subseteq \overline S$. Furthermore, $(A, B) \in
Z_F$. By
assumption, there is a path $P$ in $E'$ between $A$ and $B$. Since $E'
\cap \calM^{-1}(F) = \emptyset$, $\calM(E')$ cannot contain any edges
in $F$. Therefore, $\calM(P)$ contains a path that crosses $S$ which
implies that $|\delta_{\calM(E')}(S)| = |\delta_{\calM(E') \setminus
  F}(S)| \geq 1$, contradicting the assumption on $S$.
\end{proof}

We now argue that $(\calT, \calM, y)$ routes sufficient flow for each pair in $Z_F$ without using the edges in $\calM^{-1}(F)$; in other words $y$ is fractional solution (modulo a scaling factor) to the Set Connectivity instance $Z_F$ in the graph/forest $\calT \setminus \calM^{-1}(F)$. We can then appeal to TreeRounding lemma to argue that it will connect the pairs in $Z_F$ without using any edges in $F$.

\begin{lemma}
\label{lem:flowforeachpair}
Let $(A, B) \in Z_F$. Let
$S \subset V_{\calT}$ such that $A \subseteq S$ and $B \subseteq V_{\calT} \setminus S$. Then $y(\delta_{\calT \setminus \calM^{-1}(F)}(S)) \geq \frac 1 {4(p+q)\beta}$.
\end{lemma}
\begin{proof}
Let $S$ be a vertex set of $\calT$ that separates $A$ from $B$. First,
suppose there exists a component $Q \in \Q_F$ such that $Q$ crosses
$S$, i.e. $S \cap Q \neq \emptyset$ and $\overline S \cap Q \neq
\emptyset$. Since $(A, B)$ partitions the set of shattered components,
$Q$ must be intact in $\calT$. Let $u$ be a leaf in $Q \cap S$ and $v$
be a leaf in $Q \cap \overline S$.  Since $Q$ is intact in $T$ the
unique path connecting $u$ to $v$ in $\calT$ crosses $S$ and let $e$
be an edge on this path that crosses $S$. It suffices to show that
$y(e) \ge \frac 1 {4(p+q)\beta}$. This follow from properties of the
\racke tree. Since $u$ and $v$ are connected in $G'$ with a path using
only edges in $\LG \cup H$ each of which has a capacity of $\frac 1
{4(p+q)\beta}$, $u$-$v$ maxflow in $G'$ is at least $\frac 1
{4(p+q)\beta}$. From Corollary~\ref{cor:racketreeflow},
for any tree $\calT$, the $u$-$v$
maxflow in $\calT$ with capacities $y$ must be at least $\frac 1
{4(p+q)\beta}$. This in particular implies that $y(e) \ge \frac 1
{4(p+q)\beta}$ for every edge $e$ on the unique path from $u$ to $v$
in $\calT$.

 We can now restrict attention to the case that no connected component
 of $\Q_F$ crosses $S$. Consider $S'$ be the set of leaves in $S$ and
 consider the cut $(S',V\setminus S')$ in $G$. It follows that
 $(S',V-S')$ partitions the connected components in $\Q_F$ and
 $\delta_{H-F}(S') = \emptyset$. Since $(A,B) \in Z_F$ there is a pair
 $(s_i,t_i)$ such that $Q_{s_i} \in S'$ and $Q_{t_i} \in V\setminus
 S'$. Thus $(S',V\setminus S')$ is a violated cut with $F$ as its
 witness. Since $x$ is a feasible solution to {\bf Augment-LP} it follows
 that $x(\delta_{E \setminus H}(S')) \ge 1$. Recall that we assumed
 that $\LG = \emptyset$, and hence all
 edges in $\delta_{E \setminus H)}(S')$ are in $\SM$.
 Therefore, $x(\delta_{E \setminus H}(S')) = \tilde x(\delta_{E \setminus H}(S')) \ge 1$.

 The \racke tree property guarantees that $y(\delta_{\calT}(S)) \ge \tilde x(\delta_{G'}(S')) \ge 1$ (via Corollary~\ref{cor:racketreeflow}). 
 We note that 
 $$y(\delta_{\calT \setminus \calM^{-1}(F)}(S)) \ge y(\delta_{\calT}(S)) - y(\calM^-(F)) \ge 1 - 1/2 \ge 1/2.$$
 where we used the fact that $y(\calM^-(F)) \le 1/2$ since $\calT$ is good for $F$.
 Thus in both cases we verify the desired bound.
\end{proof}


\paragraph{Bounding $Z_F$:} A second crucial property is a bound on $|Z_F|$,
the number of pairs in the Set Connectivity instance induced by $F$ and a good tree $\calT$ for $F$. 

\begin{lemma}
\label{lem:boundnumpairs}
For a good tree $\calT$, $|Z_F| \leq 2^{2(p+q)\beta} k$.
\end{lemma}
\begin{proof}
Let $\ell$ be the number of shattered components and let them be
$Q_1,\ldots,Q_\ell$. For each $Q_i$ pick a pair of vertices $u_i,v_i$
that are in separate components of $\calT - \calM^{-1}(F)$. Let $A =
\{u_1,\ldots,u_\ell\}$ and $B = \{v_1,v_2,\ldots,v_\ell\}$. Since the
paths connecting $u_i,v_i$ are in different connected components of
$H\setminus F$, it follows that the $(A,B)$-maxflow in $H \setminus F$
is at least $\ell$. In the graph $G'$ obtained by scaling down the
capacity of edges of $H$, the maxflow is at least $\frac \ell
{4(p+q)\beta}$ which implies that it is at least this quantity in
$\calT$. Since $\calT$ is good, the total decrease of flow can be at
most $y(\calM^{-1}(F)) \le \frac 1 2$. By construction there is no
flow between $A$ and $B$ in $\calT - \calM^{-1}(F)$ which implies that
$\frac \ell {4(p+q)\beta} \le 1/2 \Rightarrow \ell \leq 2(p+q)\beta$.
Each pair in $Z_F$ corresponds to a subset of shattered components and
a demand pair $(s_i,t_i)$, and hence $|Z_F|\leq 2^\ell k \le
2^{2(p+q)\beta} k$.
\end{proof}

\subsection{Correctness and Cost}
Now we analyze the correctness and cost of the algorithms output.

\begin{lemma}
\label{claim:successprobforgoodtree}
Suppose $\calT$ is good for a violating edge set $F$. Then after $t$
rounds of TreeRounding with flow parameter $\frac 1 {4(p+q)\beta}$,
the probability that $H'$ is \emph{not} a feasible augmentation for
$F$ is at most $(1-\phi)^t |Z_F| \le 1/4$.
\end{lemma}
\begin{proof}
  Suppose $\calT$ is good for $F$. Let $(A,B) \in Z_F$. From
  Lemma~\ref{lem:flowforeachpair} the flow for $(A,B)$ in $\calT -
  \calM^{-1}(F)$ is at least $\frac 1 {4(p+q)\beta}$. From
  Lemma~\ref{lem:setconnectivity-tree-rounding}, with probability at
  least $\phi$, the pair $(A,B)$ is connected via a path in $\calT -
  \calM^{-1}(F)$. If all pairs are connected, then via
  Claim~\ref{claim:shattered_suffices}, $H'$ is a feasible
  augmentation for $F$. Thus, $H'$ is not a feasible augmentation if
  for some $(A,B) \in Z_F$ the TreeRounding does not succeed after $t$
  rounds. The probability of this, via the union bound over the pairs
  in $Z_F$, is at most $(1-\phi)^t |Z_F|$. From
  Lemma~\ref{lem:boundnumpairs}, $|Z_F| \le 2^{2(p+q)\beta}k$. Consider $t
  = \frac 1 \phi \log(4k \cdot 2^{2\beta(p+q)}) = O((p+q)\log n)$, since $\beta = O(\log n)$. Then, $(1-\phi)^t |Z_F| = 2^{2(p+q)\beta} k (1 -
  \phi)^t \leq 2^{2(p+q)\beta} k e^{-\phi t} \leq \frac 1 4$.
\end{proof}

\begin{lemma}
\label{lem:correctness}
The algorithm outputs a solution $H'$ such that $H \cup H'$ is a
feasible augmentation to the given instance with probability at least
$\frac 1 2$.
\end{lemma}
\begin{proof}
For a fixed $F$ the probability that a sampled tree is good is at
least $1/2$.  By Claim~\ref{claim:successprobforgoodtree}, conditioned
on the sampled tree being good for $F$, $t$ iterations of TreeRounding
fail to augment $F$ with probability at most $1/4$. Thus the probability that
all $t'$ iterations of sampling trees fail is $(1-3/8)^{t'}$. There
are at most $n^{2(p+q)}$ violating edge sets $F$. Consider $t' = \frac 8 3
\log (2n^{2(p+q)}) = O((p+q)\log n)$. By applying the union bound over
all violating edge sets $F$, the probability of the algorithm failing
is at most $n^{2(p+q)}(1 - 3/8)^{t'} \leq n^{2(p+q)}e^{-3t'/8} \leq
\frac 1 2$. Therefore, the output of the algorithm is a feasible
augmentation for all violating edge sets with probability at least
$\frac 1 2$.
\end{proof}


Now we analyze the expected cost of the edges output by the algorithm for augmentation
with respect to $\lpopt$, the cost of the fractional solution.

\begin{lemma}
\label{lem:costanalysis}
The total expected cost of the algorithm is $O((p+q)^3\log^7 n) \cdot \lpopt$.
\end{lemma}
\begin{proof}
Fix an edge $e \in \SM$ with fractional value $x_e$. Consider one
outer iteration of the algorithm in which it picks a random tree
$\calT$ from the \racke tree distribution and then runs $t$ iterations
of TreeRounding with flow parameter $\alpha = \frac 1
{4(p+q)\beta}$. Via Lemma~\ref{lem:setconnectivity-tree-rounding}, the
probability of an edge $f \in \calT$ being chosen is at most
$O(\frac{1}{\alpha} h \log^2n) y(f)$. Thus the expected cost for $e$
for one round of TreeRounding is $O(\frac{1}{\alpha} h \log^2n)
\sum_{f \in \calM^{-1}(e)} y(f) = O(\frac{1}{\alpha} h \log^2n)
\load(e)$. By the \racke distribution property, $\E_{\calT} [\load(e)]
\le \beta x_e$. By linearity of expectation, since there are a total
of $t \cdot t'$ iterations of TreeRounding, the total expected cost is
at most ($t \cdot t')\cdot O(\frac{1}{\alpha} h \log^2 n \beta) \sum_{e \in
  E} c(e) x_e$. By the analysis in Section~\ref{sec:algo}, $h = O(\log
n)$, and $\beta = O(\log n)$. Substituting in the values of $t$ and $t'$ stated
in Lemmas \ref{claim:successprobforgoodtree} and \ref{lem:correctness}, the total expected cost is at most $O((p+q)^3 \log^7 n) \cdot \lpopt$.
\end{proof}


Combining the correctness and cost analysis we obtain the following.
\begin{theorem}
  There is a randomized $O((p+q)^3\log^7 n)$ approximation for the augmentation problem
  via {\bf Augment-LP}.
\end{theorem}

Starting with a solution for $(p,0)$-flex-connectivity, and using $q$ augmentation iterations,
we obtain an $O(q(p+q)^3\log^7 n)$-approximate solution for the given instance of
$(p,q)$-Flex-SNDP, proving
Theorem~\ref{thm:main}.

\section{Preprocessing}
In this section, we present a simple, scalable preprocessing algorithm, which often reduces the size of the input networks significantly and results in a substantial speed-up. 
Specifically, the algorithm first computes an approximate solution by solving an LP, 
which is much smaller than LP~\eqref{LP:general} in practice, 
and then removes vertices from the original network using the information of the approximate solution obtained. 
We assume that $S^*_i$ for all $i\in [k]$ are known in advance
because they can be computed efficiently  using Charikar's LP-based algorithm~\cite{Charikar2000}
together with the preprocessing algorithm introduced by Balalau et al.~\cite{Balalau+15}.
%Consequently, we can decrease the number of essential variables in LP~\eqref{LP:general}. 
%This preprocess is necessary to deal with a large graph in practice. 
To describe our algorithm, we introduce some notations. 
For $S\subseteq V$, $v\in S$, and $i\in[k]$, let $d_i(S,v)$ denote the weighted degree of $v$ in the subgraph induced by $S$ in layer $i$, 
i.e., $d_i(S,v)\coloneqq\sum_{e\in E_i[S]:\,v\in e}w_i(e)$. When $S=V$, we simply write $d_i(v)$.
%In addition, let $d_i(S)\coloneqq d_i(V,v)$.


\begin{comment}
We first describe an algorithm for computing an approximate solution for $(\bm{\alpha}, \bm{\beta})$-\dens. 
Specifically, our algorithm iteratively removes a vertex with the minimum degree in the currently remaining graph in the currently \emph{worst} layer to obtain a sequence of vertex subsets from $V$ to a singleton, 
and then returns a probability distribution $p\in \Delta(2^V)$ corresponding to the best subset over the iteration. 
\begin{algorithm}[t] 
\caption{Compute an approximate solution}\label{alg:peeling}
\SetKwInOut{Input}{Input}
\SetKwInOut{Output}{Output}
\Input{\ $(V,E;w_1,\dots,w_k)$}
\Output{\ $p\in \Delta(2^V)$}
%\Output{\ Lower bound $\ell^*$}
$T_n\ot V$\;
\For{$j\ot n,\dots,2$}{
  Let $i^*\in\argmin_{i\in[k]}\left[\alpha_i\frac{w_i(T_j)}{|T_j|}+\beta_i\right]$\;
  Find $v_j\in \argmin_{v\in T_j} d_{i^*}(T_j,v)$ and $T_{j-1}\ot T_j\setminus\{v_j\}$\;
}
Pick $S\in \argmax_{S\in \{T_n,\dots, T_2\}}\min_{i\in[k]}\left[\alpha_i\frac{w_i(T_j)}{|T_j|}+\beta_i\right]$\;
\Return $p\in \Delta(2^V)$ such that $p_S=1$
%\Return $\max_{j\in \{n,\dots, 2\}}\min_{i\in[k]}\left[\alpha_i\frac{w_i(T_j)}{|T_j|}+\beta_i\right]$\;
\end{algorithm}
The procedure is detailed in Algorithm~\ref{alg:peeling}. 
This algorithm can be implemented to run in $O(k|E|+k|V|\log |V|)$ time.
Note that this algorithm is a generalization of the greedy peeling algorithm introduced by Jethava and Beerenwinkel~\cite{JB2015} for the densest common subgraph problem. They showed that the algorithm is a $1/2$-approximation algorithm for the problem. 
\end{comment}

We first describe a fast algorithm for finding an approximate solution for $(\bm{\alpha}, \bm{\beta})$-\dens. 
%\memo{We first compute $S^*_i$ for all $i\in [k]$? Then...} 
Specifically, we compute a probability distribution $q\in\Delta(2^V)$ that maximizes the $(\bm{\alpha}, \bm{\beta})$-density 
under the constraint that $q_S=0$ for all $S\in 2^V\setminus \{S_1^*,\dots,S_k^*\}$.
The distribution can be found by solving the following LP:
\begin{align}
\begin{array}{rll}
\text{max.}       & t &\\
\text{s.t.}& \displaystyle t\leq \alpha_i \sum_{j\in[k]} \frac{w_{i}(S_j^*)}{|S_j^*|} q_j+\beta_i  &(\forall i\in [k]),\\
           & \displaystyle \sum_{j\in[k]}q_j = 1,  &\\
           & q_j\ge 0                &(\forall j\in[k]).
\end{array}\label{LP:lower}
\end{align}
Note that this LP has $k+1$ variables and $2k+1$ constraints. 
As $k$ is usually much smaller than $|V|$ and $|E|$, this LP is much smaller than LP~\eqref{LP:general} in practice. 
%Obviously, the optimal value of this LP is a lower bound on the optimal value of $(\bm{\alpha},\bm{\beta})$-\dens. 

% Next we describe an algorithm for removing vertices using the information of the output solution of Algorithm~\ref{alg:peeling}. 
% Our algorithm first computes the objective function value of the output of Algorithm~\ref{alg:peeling}, which is clearly a lower bound on the optimal value of $(\bm{\alpha},\bm{\beta})$-\dens. 
Next we describe an algorithm for removing vertices using the information of the above approximate solution $q$. 
Let $\ell^*$ be the $(\bm{\alpha},\bm{\beta})$-density of $q$, i.e., the optimal value of LP~\eqref{LP:lower}. 
Note that this is a lower bound on the optimal value of $(\bm{\alpha},\bm{\beta})$-\dens. 
Our algorithm iteratively removes any vertex $v^*$ that satisfies $\max_{i\in[k]}[\alpha_i\cdot d_i(V',v^*)+\beta_i]< \ell^*$, where $V'$ is a remaining vertex set (initially $V'=V$), as long as there exists such a vertex. 
For reference, we describe the procedure in Algorithm~\ref{alg:remove}. 
%Note that this is a generalization of the above procedure, which may accept not only the output of Algorithm~\ref{alg:peeling} but also any $p\in \Delta(2^V)$. 
This algorithm can be implemented to run in $O(k|E|+|V|\log |V|)$ time.
\begin{algorithm}[t] 
\caption{Remove useless vertices}\label{alg:remove}
\SetKwInOut{Input}{Input}
\SetKwInOut{Output}{Output}
%\Input{\ $(V,E;w_1,\dots,w_k)$ and $p\in \Delta(2^V)$}
\Input{\ $(V,(E_i)_{i\in [k]})$ with $w_1,\dots,w_k$, and $\ell^*\in \mathbb{R}$}
\Output{\ $(V',(E_i[V'])_{i\in [k]})$} % ;w_1,\dots, w_k
%\Output{\ $(V',E_1[V'])$,\dots,$(V',E_k[V'])$} % ;w_1,\dots, w_k
%$\ell^*\ot \min_{i\in [k]} \mathbb{E}_{S\sim p}\left[\alpha_i\frac{w_i(S)}{|S|}+\beta_i\right]$\;
$V'\ot V$\;
\While{\texttt{True}}{
  Let $v^*\in\argmin_{v\in V'}\max_{i\in[k]}[\alpha_i\cdot d_i(V',v)+\beta_i]$\;
  \If{$\max_{i\in[k]}[\alpha_i\cdot d_i(V',v^*)+\beta_i]\ge \ell^*$}{
      \Return $(V',(E_i[V'])_{i\in [k]})$. 
  }
  \lElse{$V'\ot V'\setminus\{v^*\}$}
}
\end{algorithm}



From now on, we demonstrate that the algorithm does not remove any vertex that is contained in a subset in $\supp(p)$, where $p$ is an arbitrary 
optimal solution to $(\bm{\alpha},\bm{\beta})$-\dens. 
The following is a key lemma in our analysis. 
% If we have the following lemma, the proof of the above theorem is immediate. 
% \memo{$p\in\Delta(2^V)$から$(x,y,t)$への変換が必要}
% By~\eqref{eq:transform}, the theorem is equivalent to the following lemma, 
% and hence it is sufficient to prove the lemma.
\begin{lemma}\label{lemma:6.2}
Let $\ell^*$ be a lower bound on the optimal value of $(\bm{\alpha},\bm{\beta})$-\dens.
If $\max_{i\in[k]}[\alpha_i\cdot d_i(v^*)+\beta_i]<\ell^*$, 
then $y_{v^*}=0$ for any optimal solution to LP~\eqref{LP:general}.
\end{lemma}
\begin{proof}
We prove the lemma by contradiction.
We denote by $((\hat{x}_e)_{e\in E},(\hat{y}_v)_{v\in V},\hat{t})$ an optimal solution to LP~\eqref{LP:general}
and let $v^*\in V$ be a vertex that satisfies $\alpha_i\cdot d_i(v^*)+\beta_i<\ell^*$ for all $i\in[k]$.
Suppose for contradiction that $\hat{y}_{v^*}>0$.

We construct a solution $((x_e)_{e\in E},(y_v)_{v\in V},t)$ of LP~\eqref{LP:general} as follows: 
\begin{align*}
x_e&=\begin{cases}
\frac{1}{1-\hat{y}_{v^*}}\cdot\hat{x}_e&(e\not\ni v^*),\\
0        &(e\ni v^*),
\end{cases}\quad
y_v=\begin{cases}
\frac{1}{1-\hat{y}_{v^*}}\cdot\hat{y}_v&(v\ne v^*),\\
0        &(v= v^*),
\end{cases}\\
t&=\min_{i\in[k]}\Bigl[\alpha_i\sum_{e\in E_i}w_i(e)x_e+\beta_i\Bigr].
\end{align*}
It is easy to see that $((x_e)_{e\in E},(y_v)_{v\in V},t)$ is a feasible solution of LP~\eqref{LP:general}.
Moreover, we have
\begin{align*}
t
&=\min_{i\in[k]}\Bigl[\alpha_i\sum_{e\in E_i}w_i(e)x_e+\beta_i\Bigr]\\
&=\min_{i\in[k]}\Bigl[\alpha_i\frac{\sum_{e\in E_i:\,v^*\not\in e}w_i(e)\hat{x}_e}{1-\hat{y}_{v^*}}+\beta_i\Bigr]\\
&=\min_{i\in[k]}\Bigl[\alpha_i\frac{\sum_{e\in E_i}w_i(e)\hat{x}_e-\sum_{e\in E_i:\, v^*\in e}w_i(e)\hat{x}_{e}}{1-\hat{y}_{v^*}}+\beta_i\Bigr]\\
&\ge \min_{i\in[k]}\Bigl[\alpha_i\frac{\sum_{e\in E_i}w_i(e)\hat{x}_e-\sum_{e\in E_i:\, v^*\in e}w_i(e) \hat{y}_{v^*}}{1-\hat{y}_{v^*}}+\beta_i\Bigr]\\
&=\min_{i\in[k]}\Bigl[\alpha_i\frac{\sum_{e\in E_i}w_i(e)\hat{x}_e-d_i(v^*) \hat{y}_{v^*}}{1-\hat{y}_{v^*}}+\beta_i\Bigr]\\
&=\min_{i\in[k]}\frac{(\alpha_i\sum_{e\in E_i}w_i(e)\hat{x}_e+\beta_i)- (\alpha_i\cdot d_i(v^*)+\beta_i)\cdot \hat{y}_{v^*}}{1-\hat{y}_{v^*}}\\
&>\min_{i\in[k]}\frac{(\alpha_i\sum_{e\in E_i}w_i(e)\hat{x}_e+\beta_i)-\ell^*\cdot \hat{y}_{v^*}}{1-\hat{y}_{v^*}}\\
&=\frac{\hat{t}-\ell^*\cdot \hat{y}_{v^*}}{1-\hat{y}_{v^*}}
\ge \frac{\hat{t}-\hat{t}\cdot \hat{y}_{v^*}}{1-\hat{y}_{v^*}}
=\hat{t},
\end{align*}
%\begin{align*}
%t
%&=\min_{i\in[k]}\left[\alpha_i\sum_{e\in E}w_i(e)x_e+\beta_i\right]\\
%&=\min_{i\in[k]}\left[\alpha_i\frac{\sum_{e\in E:\,v^*\not\in e}w_i(e)\hat{x}_e}{1-\hat{y}_{v^*}}+\beta_i\right]\\
%&=\min_{i\in[k]}\left[\alpha_i\frac{\sum_{e\in E}w_i(e)\hat{x}_e-\sum_{e':\, v^*\in e'\in E}w_i(e')\hat{x}_{e'}}{1-\hat{y}_{v^*}}+\beta_i\right]\\
%&\ge \min_{i\in[k]}\left[\alpha_i\frac{\sum_{e\in E}w_i(e)\hat{x}_e-\sum_{e':\, v^*\in e'\in E}w_i(e') \hat{y}_{v^*}}{1-\hat{y}_{v^*}}+\beta_i\right]\\
%&=\min_{i\in[k]}\left[\alpha_i\frac{\sum_{e\in E}w_i(e)\hat{x}_e-d_i(v^*) \hat{y}_{v^*}}{1-\hat{y}_{v^*}}+\beta_i\right]\\
%&=\min_{i\in[k]}\frac{(\alpha_i\sum_{e\in E}w_i(e)\hat{x}_e+\beta_i)- \hat{y}_{v^*}(\alpha_id_i(v^*)+\beta_i)}{1-\hat{y}_{v^*}}\\
%&>\min_{i\in[k]}\frac{(\alpha_i\sum_{e\in E}w_i(e)\hat{x}_e+\beta_i)-\ell^*\cdot \hat{y}_{v^*}}{1-\hat{y}_{v^*}}\\
%&=\frac{\hat{t}-\ell^*\cdot \hat{y}_{v^*}}{1-\hat{y}_{v^*}}
%\ge \frac{\hat{t}-\hat{t}\cdot \hat{y}_{v^*}}{1-\hat{y}_{v^*}}
%=\hat{t},
%\end{align*}
where the first inequality follows from $\hat{x}_{e} \leq \hat{y}_{v^*}$ for each $e \ni v^*$, 
the second inequality follows from the assumptions $\alpha_i\cdot d_i(v^*)+\beta_i< \ell^*$ and $\hat{y}_{v^*}>0$, and the third inequality follows from Lemma~\ref{lemma:general_lower}. 
%where the first inequality holds because $\hat{x}_{e'} \leq r_{v^*}$ for each $e' \not\ni v^*$, the second inequality holds by the assumption $\alpha_id_i(v^*)+\beta_i< \ell^*$, and the third inequality holds by $\ell^* \leq \hat{t}$. 
This contradicts the optimality of $((\hat{x}_e)_{e\in E},(\hat{y}_v)_{v\in V},\hat{t})$. 
\end{proof}


%We have the following theorem.
\begin{theorem}
Let $\ell^*$ be a lower bound on the optimal value of $(\bm{\alpha},\bm{\beta})$-\dens.
Then, any vertex $v^*$ that satisfies $\max_{i\in[k]}[\alpha_i\cdot d_i(v^*)+\beta_i]<\ell^*$ 
%Then, any vertex $v^*$ with weighted degree smaller than $\ell^*$ for all $i\in [k]$ (i.e., $\max_{i\in[k]}[\alpha_i\cdot d_i(v^*)+\beta_i]<\ell^*$) 
is not contained in any subset in the support of any optimal solution to $(\bm{\alpha},\bm{\beta})$-\dens.
\end{theorem}
\begin{proof}
Let $v^*$ be any vertex with $\max_{i\in[k]}[\alpha_i\cdot d_i(v^*)+\beta_i]<\ell^*$. 
Let $\hat{p}$ be any optimal solution to $(\bm{\alpha},\bm{\beta})$-\dens.
Construct $((\hat{x}_e)_{e\in E},(\hat{y}_v)_{v\in V},\hat{t})$ from $\hat{p}$ as in \eqref{eq:transform}. 
From Lemma~\ref{lemma:expect} and the proof of Lemma~\ref{lemma:general_lower}, 
we see that $((\hat{x}_e)_{e\in E},(\hat{y}_v)_{v\in V},\hat{t})$ is an optimal solution to LP~\eqref{LP:general}. 
Thus, by Lemma~\ref{lemma:6.2}, we have $\hat{y}_{v^*} = 0$.
By the construction of $\hat{y}_{v^*}$ in \eqref{eq:transform}, $\hat{p}_S=0$ for all $S\subseteq V$ containing $v^*$. 
%Therefore, the theorem holds. 
\end{proof}

This theorem indicates that Algorithm~\ref{alg:remove} does not remove any vertex that is contained in a subset in the support of any optimal solution to $(\bm{\alpha},\bm{\beta})$-\dens.




% \newpage

\begin{table*}[t]
\setlength{\tabcolsep}{1.5mm}
\centering

\subtable[On 11 discriminative test tasks following the T0 benchmark.]{
\resizebox{\textwidth}{!}{%
    \begin{tabular}{l|l|c|ccccc|ccc|cc|c|c}
    \toprule[1pt]
    \multirow{2}{*}{Base Model} &
    \multirow{2}{*}{Method} &
    \multirow{2}{*}{\#Params} & 
    \multicolumn{5}{|c|}{\textbf{Natural Language Inference}} & \multicolumn{3}{|c|}{\textbf{Sentence Completion}} & \multicolumn{2}{c|}{\textbf{Coreference}} & \multicolumn{1}{c|}{\textbf{WSD}} 
    & \multirow{2}{*}{Avg.}\\
    & & & RTE & CB & ANLI1 & ANLI2 & ANLI3 & COPA & Hella. & Story. & WSC & Wino. & WiC &  \\
    \midrule[1pt]
    Decoder-only & GPT-3 & 175B 
        &63.5 &46.4
        &34.6 &	35.4&	34.5&	91.0&	78.9&	83.2&	65.4&	70.2&	- & -\\
    Decoder-only & GLaM & 137B 
        & 56.3	& 39.3	& 39.7	& 35.5	& 34.1	& 90.0	& 76.7	& 81.1	& 82.1	& 71.3	& 50.6 & 59.7\\
    MoE Decoder-only & GLaM & 64B 
        & 66.8	& 33.9	& 40.9	& 38.2	& 40.9	& 90.0	& 77.1	& 82.5	& 83.5	& 73.4	& 50.5 & 61.6\\
    Decoder-only & PaLM & 540B 
        & 72.9	& 51.8	& 48.0	& 44.2	& 45.7	& 93.0	& 83.4	& 84.6	& 89.1	& 81.1	& 59.1 & 68.5\\
    Decoder-only & FLAN & 137B 
        & 78.3	& 64.1	& 47.7	& 43.9	& 47.0	& 90.6	& 56.4	& 92.2	& 80.8	& 67.3 & - & -\\
    \midrule[1pt]
    \multirow{3}*{\shortstack{ELECTRA}}
    & PE-CLS & 335M
        & 60.2	& 57.4	& 34.1	& 34.4	& 36.4	& 92.7	& 44.1	& 96.0	& 62.8	& 56.3	& 50.7	& 56.8
        \\
    & PE-PROB & 335M
        & 54.0	& 49.2	& 32.3	& 33.3	& 33.5	& 81.9	& 36.7	& 89.5	& 64.3	& 50.7	& 50.9	& 52.4 \\
    & PE-REP & 335M
        & 69.0	& 61.3	& 36.1	& 35.0	& 39.4	& 91.2	& 47.0	& 96.8	& 70.0	& 56.2	& 51.1	& 58.5
        \\
    \midrule
    \multirow{1}*{\shortstack{DeBERTaV3}}
    & \multirow{1}*{{UD (ours)}} & 304M
        & \multirow{1}*{71.1}
        & \multirow{1}*{76.8}
        & \multirow{1}*{43.8}
        & \multirow{1}*{41.3}
        & \multirow{1}*{45.7}
        & \multirow{1}*{96.0}
        & \multirow{1}*{60.7}
        & \multirow{1}*{97.4}
        & \multirow{1}*{66.4}
        & \multirow{1}*{83.6}
        & \multirow{1}*{53.3}
        & \multirow{1}*{66.9}
    \\
    \midrule[1pt]
    \multirow{2}*{\shortstack{T5-Large}}
    & \multirow{1}*{T0 $\star$} & 800M
        & 75.1	& 55.5	& 32.9	& 32.3	& 33.7	& 84.6	& 28.2	& 94.0	& 63.0	& 54.6	& 51.2	& 55.0 \\


    & {UD (ours)} & 400M
        & \textbf{83.8}
        & \textbf{80.4}
        & \textbf{36.8}
        & \textbf{34.2}
        & \textbf{42.2}
        & \textbf{90.0}
        & \textbf{56.1}
        & \textbf{96.4}
        & \textbf{68.3}
        & \textbf{62.9}
        & \textbf{54.6}	
        & \textbf{64.1} \\
    \midrule[1pt]
    \multirow{3}*{\shortstack{T5-XL}}
    & \multirow{1}*{T0 $\dagger$} & 3B
        & 64.6 
        & 45.4
        & 33.8
        & 33.1
        & 33.3
        & 72.4
        & 27.3
        & 84.0
        & 65.1
        & 51.0
        & 50.7
        & 51.0 \\

    & \multirow{1}*{T0 $\star$} & 3B
    & \textbf{79.7}	& 68.9	& \textbf{43.1}	& \textbf{38.5}	& 42.3	& \textbf{94.1}	& 31.5	& 97.5	& 68.8	& 61.3	& \textbf{54.1}	& 61.8\\

 
    & {UD (ours)} & 1.5B
        & 78.7
        & \textbf{73.2}
        & 41.2
        & 36.3
        & \textbf{45.4}
        & 94.0
        & \textbf{70.1}
        & \textbf{97.9}
        & \textbf{72.1}
        & \textbf{70.6}
        & 53.0	
        & \textbf{66.6} \\
    \midrule[1pt]
    \multirow{4}*{\shortstack{T5-XXL}}
    & \multirow{1}*{T0 $\dagger$} & 11B
        & 80.8
        & 70.1
        & 43.6
        & 38.7
        & 41.3
        & 90.0
        & 33.6
        & 92.4
        & 61.5
        & 59.9
        & 56.6
        & 60.8 \\

    & \multirow{1}*{T0 $\star$} & 11B
    & \textbf{85.8}	& 73.3	& 47.3	& 42.0	& 46.1	& 94.4	& 31.5	& 98.4	& 62.8	& 72.8	& 56.0	& 64.6 \\

    & {UD (ours)} & 5.5B
    & 80.5	& 87.5	& 49.0	& 42.9 & 	48.8	& 95.0	& 77.4	& \textbf{98.6}	& 73.1	& 82.2	& 57.1	& 72.0 \\

    & {UD+ (ours)} & 5.5B
    & 82.0	& \textbf{89.3}	& \textbf{53.4} & \textbf{48.1} & \textbf{51.0} & \textbf{96.0} & \textbf{78.9} & 96.7	& \textbf{75.0}	& \textbf{86.4}	& \textbf{58.5}	& \textbf{74.1} \\
    \bottomrule[1pt]
\end{tabular}
}
\label{tab:maintable:top}
}


\subtable[On 13 discriminative BigBench tasks following the T0 benchmark]{
\resizebox{0.7\textwidth}{!}{%
    \begin{tabular}{l|cc|cc|ccc|}
    \toprule[1pt]
    \multirow{1}{*}{Model} 
        & \multirow{1}{*}{\shortstack{T0-Large}}
        & \multirow{1}{*}{\shortstack{UD-large}}
        & \multirow{1}{*}{\shortstack{T0-XL}}
        & \multirow{1}{*}{\shortstack{UD-XL}}
        & \multirow{1}{*}{\shortstack{T0-XXL}}
        & \multirow{1}{*}{\shortstack{UD-XXL}}
        & \multirow{1}{*}{\shortstack{UD+-XXL}}\\
    \midrule[1pt]
    BigBench (Avg.) & 39.6 & \textbf{43.5} & 44.8 & \textbf{48.9} & 47.4 & 55.5 & \textbf{58.7} \\
    \bottomrule[1pt]
    \end{tabular}%
    }
\label{tab:maintable:middle}
}

\subtable[On 22 discriminative BBH tasks]{
\resizebox{\textwidth}{!}{%
    \begin{tabular}{l|ccc|ccc|cccc|}
    \toprule[1pt]
    \multirow{1}{*}{Model} 
        & \multirow{1}{*}{\shortstack{T0-Large}}
        & \multirow{1}{*}{\shortstack{Flan-T5-Large}}
        & \multirow{1}{*}{\shortstack{UD-Large}}
        & \multirow{1}{*}{\shortstack{T0-XL}}
        & \multirow{1}{*}{\shortstack{Flan-T5-XL}}
        & \multirow{1}{*}{\shortstack{UD-XL}}
        & \multirow{1}{*}{\shortstack{T0-XXL}}
        & \multirow{1}{*}{\shortstack{Flan-T5-XXL}}
        & \multirow{1}{*}{\shortstack{UD-XXL}}
        & \multirow{1}{*}{\shortstack{UD+-XXL}}\\
    \midrule[1pt]
    BBH (Avg.) & 38.9 & 39.5 & \textbf{44.2} & 40.4 & 44.6 & \textbf{47.3} & 45.0 & 49.4 & 51.3 & \textbf{56.7} \\
    \bottomrule[1pt]
    \end{tabular}%
    }
\label{tab:maintable:bottom}
}
\caption{
Zero-shot performance of our UD and baselines.
Results in the first block are reported by previous work, respectively from GPT-3~\cite{gpt3-paper}, GLaM~\cite{glam}, PaLM~\cite{palm}, and FLAN~\cite{FLAN}.
Note that we provide these reported results for reference, and do not compare directly. Some of the reported tasks are evaluated on the test split, while we follow the better baseline method T0 to report on validation splits.
Results with $\dagger$ are reported by~\citeauthor{T0-paper}, and results with $\star$ are reproduced in our framework. We reproduced the three variants of prompting ELECTRA~\cite{xia2022prompting} under our setting, denoted as ``PE-CLS'', ``PE-PROB'', ``PE-REP''.
Results for Flan-T5-Large/Xl/XXL~\citep{flant5} are reproduced by testing zero-shot performance on their released checkpoints.
In the same group, T0 and Flan-T5 has 2x model parameters compared to UD. For abbreviation, we denote UD based on T5-XX as ``UD-XX'', e.g., UD-XL refers to UD based on the T5-XL model.
}
\label{tab:maintable}
\vspace{-0.7cm}
\end{table*}
% \begin{table}[t]
\setlength{\tabcolsep}{4.5mm}
\centering
    \resizebox{0.5\textwidth}{!}{%
    \begin{tabular}{lcccc}
        \toprule[1pt]
        % \textbf{Finetuned Task} & \textbf{Task Type} & \textbf{Metric} & \textbf{Eval Set} & \textbf{SOTA Reference} & \textbf{SOTA} & \textbf{Ours}  \\
        \textbf{Finetuned Task} & \textbf{T0} & \textbf{UD}  \\
        \midrule[1pt]
        MRPC & 90.5 & 89.7 (to be improve) \\
        QQP & 85.9 (to improve) & \textbf{91.6} \\
        PAWS & 95.1 & \textbf{97.2} \\
        WikiQA  & 96.1 & \textbf{96.5}\\
        CosmosQA & 88.4 & \textbf{90.7}\\
        DREAM & 90.5 & \textbf{91.6} \\
        QuAIL & 65.6 & \textbf{80.2} \\
        QuaRel & 88.2 & \textbf{95.3}\\
        QuaRTz & 94.1 & \textbf{94.5} \\
        SciQ & 97.6& \textbf{98.1}\\
        SocialIQA &  \textbf{82.2} & 81.7 \\
        WikiHop & & 58.6\\
        Amazon & \textbf{97.6}(to improve) & 97.3 \\
        IMDB & \textbf{96.9} & 96.7\\
        Rotten & 93.4 & \textbf{93.6} \\
        Yelp & 72.28 (first two prompts) & 68.1 (hard to improve)\\
        AGNews & 95.1 & \textbf{95.3} \\
        DBPedia & & \\
        TREC & 96.7 & \textbf{97.8}\\
        \bottomrule[1pt]
    \end{tabular}
    }
    \caption{Results on finetuned tasks for UD and the baseline T0. Both methods use T5-XXL as a base model. T0 has 2x model parameters compared to UD.}

    % compared with state-of-the-art results.}
    \label{tab:finetunedtasks}
\end{table}
\begin{table*}[!htp]
\setlength{\tabcolsep}{1.5mm}
\centering
\subtable[On 11 discriminative test tasks following the T0 benchmark.]{
\resizebox{\textwidth}{!}{%
    \begin{tabular}{l|ccccc|ccc|cc|c|c}
        \toprule[1pt]
        \multirow{2}*{Method}
        & \multicolumn{5}{c|}{\textbf{Natural Language Inference}} & \multicolumn{3}{c|}{\textbf{Sentence Completion}} & \multicolumn{2}{c|}{\textbf{Coreference}} & \multicolumn{1}{c|}{\textbf{WSD}} & \multirow{2}{*}{Avg.} \\
    & RTE & CB & ANLI1 & ANLI2 & ANLI3 & COPA & Hella. & Story. & WSC & Wino. & WiC &  \\
    \midrule[1pt]
    T0-XL %
        & \textbf{79.7}	& 68.9	& 43.1	& 38.5	& 42.3	& \textbf{94.1}	& 31.5	& \textbf{97.5}	& \textbf{68.8}	& 61.3	& \textbf{54.1}	& 61.8\\
    GenUD-XL %
        & 71.5	& \textbf{80.4}	& \textbf{43.1}	& \textbf{39.5}	& \textbf{42.6}	& 94.0	& \textbf{55.8}	& 96.7	& 63.5	& \textbf{75.5}	& 52.8	& \textbf{65.0}\\
    \bottomrule[1pt]
    \end{tabular}%
}
\label{tab:genud:top}
}
\subtable[On 13 discriminative Big-Bench tasks following the T0 benchmark.]{
\resizebox{\textwidth}{!}{%
    \begin{tabular}{l|ccccccccccccc|c}
    \toprule[1pt]
    \multirow{2}{*}{Model} 
        & \multirow{2}{*}{\shortstack{code \\ desc.}}
        & \multirow{2}{*}{\shortstack{conce\\-ptual}}
        & \multirow{2}{*}{\shortstack{known\\unknowns}}
        & \multirow{2}{*}{\shortstack{logic \\ grid}}
        & \multirow{2}{*}{\shortstack{logic \\ deduction}}
        & \multirow{2}{*}{\shortstack{miscon\\-ceptions}}
        & \multirow{2}{*}{\shortstack{novel\\concepts}}
        & \multirow{2}{*}{\shortstack{strate\\-gyqa}}
        & \multirow{2}{*}{\shortstack{wino\\-why}}
        & \multirow{2}{*}{\shortstack{syllo\\-gisms}}
        & \multirow{2}{*}{\shortstack{movie\\dialog}}
        & \multirow{2}{*}{\shortstack{lang\\-uage\_id}}
        & \multirow{2}{*}{\shortstack{vita\\-minc}} 
        & \multirow{2}{*}{Avg.} \\
    &&&&&&&&&&&&&&\\
    \midrule
    T0-XL & 23.4 & 48.1 & 64.6 & \textbf{42.5} & 50.1 & \textbf{52.7} & 25.0    & 53.1 & 45.4 & 50.2 & 47.7 & \textbf{19.0} & 60.0 & 44.8 \\
    GenUD-XL & \textbf{60.0} & \textbf{64.1} & \textbf{69.6} & 38.2 & \textbf{52.8}  & 48.9 & \textbf{44.1} & \textbf{57.1} & \textbf{46.5} & \textbf{50.4} & \textbf{50.9} & 15.5 & \textbf{66.8} & \textbf{48.9} \\
    \bottomrule[1pt]
    \end{tabular}%
\label{tab:genud:mid}
}
}




\subtable[On 15 generative tasks from Big-Bench]{
\resizebox{\textwidth}{!}{%
    \begin{tabular}{l|ccccccccccccccc|c}
    \toprule[1pt]
    \multirow{3}{*}{Model}
        & \multirow{3}{*}{\shortstack{auto \\ debugging}}
        & \multirow{3}{*}{\shortstack{simple \\ arith \\ -metic}}
        & \multirow{3}{*}{\shortstack{repeat\\copy \\ logic}}
        & \multirow{3}{*}{\shortstack{sufficient \\ information}}
        & \multirow{3}{*}{\shortstack{simple \\ text \\ editing}}
        & \multirow{3}{*}{\shortstack{scientific \\ press \\ release}}
        & \multirow{3}{*}{\shortstack{code\\ names}}     
        & \multirow{3}{*}{\shortstack{emoji\\movies}}
        & \multirow{3}{*}{\shortstack{penguins\\in a \\ table}}
        & \multirow{3}{*}{\shortstack{few \\ shot\\nlg}}
        & \multirow{3}{*}{\shortstack{operators}}
        & \multirow{3}{*}{\shortstack{tense}}
        & \multirow{3}{*}{\shortstack{geometric\\shapes}}
        & \multirow{3}{*}{\shortstack{chinese \\ remainder\\ theorem}}
        & \multirow{3}{*}{\shortstack{temporal\\sequences}}
        & \multirow{3}{*}{\shortstack{Avg.}}\\
    &&&&&&&&&&&&&&&&\\[1em]
    \midrule
    T0-XL & 11.2 & 6.7 & \textbf{25.8} & 33.8 & 7.5 & \textbf{6.7} & \textbf{44.8} & \textbf{8.7} & \textbf{11.4} & 17.4 & \textbf{10.5} & 80.7 & 0.0 & 0.0 & 14.0 & \textbf{18.6}\\
    GenUD-XL & \textbf{15.5} & 6.7 & 8.2 & \textbf{34.4} & \textbf{12.6} & 6.4 & 25.1 & 0.0 & 8.1 & \textbf{20.5} & 3.7 & \textbf{80.9} & 0.0 & 0.0 & \textbf{33.5}  & 17.0\\
    \bottomrule[1pt]
    \end{tabular}%
}
}

\caption{Zero-shot performance for generalized UD and T0 on discriminative and generative tasks. 
We select the top 15 uncommon generative tasks from BigBench basing on ascending order of data size. (We assume that datasets with smaller sizes are less common, and more suitable for zero-shot tests.) The metrics are respectively accuracy for discriminative tasks and ROUGE1 for generative tasks. ``GenUD'' denotes our generalized UD method.}
\label{tab:genud}
\end{table*}







\begin{table}[htbp]
\setlength{\tabcolsep}{1.5mm}
  \centering
\resizebox{0.35\textwidth}{!}{
    \begin{tabular}{lcc}
    \toprule
    \textbf{Dataset} & \textbf{SOTA} & \textbf{UD+-XXL} \\
    \midrule
    QQP     & \textbf{90.60}  & 90.44 \\
    DREAM     & 91.80  & \textbf{94.95} \\
    QuAIL   & 87.20  & \textbf{88.13} \\
    IMDB    & 97.30  & \textbf{97.44}  \\
    AgNews   & \textbf{95.58}  & 95.56  \\
    OBQA   & 87.20  & \textbf{89.20} \\
    STSB     & 92.30  & \textbf{92.90} \\
    CSQA    & \textbf{84.90}  & 84.68  \\
    SST-2     & 97.30  & \textbf{97.48} \\
    QNLI    & 96.50  & \textbf{96.56} \\
    AbductiveNLI &  89.80  & \textbf{93.20} \\
    VitaminC   & 91.10  & \textbf{92.62} \\
    MNLI  &  \textbf{92.10}  & 92.03  \\
    MCScript &  97.30  & \textbf{98.03} \\
    MCScript 2.0 &  97.90  & \textbf{98.01} \\
    AdversarialNLI (r3) &53.50  & \textbf{67.83 } \\
    COLA   & \textbf{71.50}  & 71.42  \\
    \midrule
    Avg.   & 89.05  & \textbf{90.62} \\
    \bottomrule
    \end{tabular}%
}
  \caption{Results on fully-supervised tasks for UD, which is based on the encoder of T5-xxl. Previous sota model \citep{ul2} has 4x model parameters compared to UD. }
  \label{tab:finetune}%
\vspace{-0.7cm}
\end{table}%



\section{Experiments}

\begin{table*}[t]
\setlength{\tabcolsep}{1.5mm}
\centering
\small
\resizebox{\textwidth}{!}{%
    \begin{tabular}{l|ccccc|ccc|cc|c|c}
        \toprule[1pt]
        & \multicolumn{5}{c|}{\textbf{Natural Language Inference}} & \multicolumn{3}{|c|}{\textbf{Sentence Completion}} & \multicolumn{2}{c|}{\textbf{Coreference}} & \multicolumn{1}{c|}{\textbf{WSD}} & \multirow{2}{*}{Avg.} \\
    & RTE & CB & ANLI1 & ANLI2 & ANLI3 & COPA & Hella. & Story. & WSC & Wino. & WiC &  \\
    \midrule[1pt]
    UD (Minimal)     & \textbf{83.75}
        & \textbf{80.36}
        & 36.80
        & \textbf{34.20}
        & \textbf{42.17}
        & \textbf{90.00}
        & \textbf{56.07}
        & \textbf{96.37}
        & \textbf{68.27}
        & \textbf{62.90}
        & \textbf{54.55}	
        & \textbf{64.13} \\
    UD (Instructive)    & 72.24 
        & 64.52 
        & \textbf{36.98} 
        & 33.40 
        & 39.73 
        & 85.31 
        & 45.15 
        & 96.01 
        & 65.38 
        & 53.94 
        & 50.94 
        & 58.51\\
    \midrule
    T0 (Minimal) & 61.56  & \textbf{57.81}  & 30.57  & 30.27  & 33.38  & 67.19  & \textbf{33.81}  & 66.56  & 60.94  & 52.81  & \textbf{51.72}  & 49.69  \\
    T0 (Instructive) & \textbf{75.05}	& 55.48	& \textbf{32.87}	& \textbf{32.29}	& \textbf{33.67}	& \textbf{84.59}	& 28.24	& \textbf{93.97}	& \textbf{62.98}	& \textbf{54.59}	& 51.16	& \textbf{54.99} \\

    \bottomrule[1pt]
    \end{tabular}}
    \caption{Zero-shot performance for UD and T0 respectively with instructive and minimal prompts. Instructive prompts are lengthy descriptions of tasks \citep{T0-paper}, while minimal prompts use a simple concatenation of input data.}
\label{tab:promptablatiion}
\end{table*}

\begin{table*}[ht]
\setlength{\tabcolsep}{0.9mm}
\centering
\resizebox{\textwidth}{!}{%
    \begin{tabular}{l|l|ccccc|ccc|cc|c|c}
        \toprule[1pt]
        & \multirow{2}*{Base Model}
        & \multicolumn{5}{c|}{\textbf{Natural Language Inference}} & \multicolumn{3}{|c|}{\textbf{Sentence Completion}} & \multicolumn{2}{c|}{\textbf{Coreference}} & \multicolumn{1}{c|}{\textbf{WSD}} & \multirow{2}{*}{Avg.} \\
    & & RTE & CB & ANLI1 & ANLI2 & ANLI3 & COPA & Hella. & Story. & WSC & Wino. & WiC &  \\
    \midrule[1pt]
    \multirow{2}*{\shortstack{Encoder}}
    & DeBERTa-V3 (304M) 
        & 71.1
        & 76.8
        & 43.8
        & 41.3
        & 45.7
        & 96.0
        & 60.7
        & 97.4
        & 66.4
        & 83.6
        & 53.3
        & 66.9 \\
    & DeBERTa-V2 (1.5B) 
        & 77.6
        & 80.4
        & 43.2
        & 39.3
        & 44.8
        & 95.0
        & 67.2
        & 98.2
        & 74.0	& 82.1 & 56.0	& 68.9\\ \midrule
    \multirow{2}*{\shortstack{Enc-Dec}} & T5-Encoder (400M) 
        & 75.1	& 55.5	& 32.9	& 32.3	& 33.7	& 84.6	& 28.2	& 94.0	& 63.0	& 54.6	& 51.2	& 55.0 \\
    & T5-Encoder (1.5B)  & 79.7	& 68.9	& 43.1	& 38.5	& 42.3	& 94.1	& 31.5	& 97.5	& 68.8	& 61.3	& 54.1	& 61.8\\
    \midrule
    \multirow{1}*{\shortstack{Decoder}}
    & \multirow{1}*{GPT-XL (1.5B)}
        & \multirow{1}*{71.1}
        & \multirow{1}*{75.0}
        & \multirow{1}*{30.4}
        & \multirow{1}*{31.8}
        & \multirow{1}*{37.8}
        & \multirow{1}*{71.0}
        & \multirow{1}*{40.9}
        & \multirow{1}*{87.7}
        & \multirow{1}*{62.5}
        & \multirow{1}*{54.5}
        & \multirow{1}*{50.3}
        & \multirow{1}*{55.7}
    \\
    \bottomrule[1pt]
    \end{tabular}}
    \caption{Ablation study on different backbone models. We experiment with base models of different architectures and scales. ``Enc-Dec'' refers to models that are pretrained in an encoder-decoder manner.}
    \label{tab:ablationbasemodel}
\end{table*}
\begin{table}
\centering
\setlength{\tabcolsep}{3.0mm}
\resizebox{0.5\textwidth}{!}{%
\begin{tabular}{l|c}
    \toprule[1pt]
    Setting & Accuracy \\
    \midrule[1pt]
    True Data vs Manually-Generated Data & 80.0 \\
    True Data vs Model-Generated Data & 74.4 \\
    \bottomrule[1pt]
    \end{tabular}%
    }
    \caption{
    The accuracy of UD discriminating real data and generated data. We feed UD with a real sample $x$ from the real-world data distribution, and a sample $x'$ from manual generation or model-based generation. 
    If UD assigns higher score to $x$ than $x'$ (i.e., $D(x)>D(x')$), it is considered an accurate prediction.
    }
  \label{tab:explain}%
\end{table}%



\subsection{Experimental Setup}\label{sec:setup}

We performed extensive experiments to validate the performance of the zero-shot generalization of our UD. We follow the same zero-shot setting as T0~\citep{T0-paper} by training on multi-task datasets and evaluating a held-out set of tasks that are never seen during training. 

\paragraph{Datasets}
The original T0 training set consists of 38 tasks of 8 different types.
% ~\footnote{We did not consider T0+ and T0++, since they are partially intersected with the test sets, making some test tasks unable to be evaluated under the zero-shot setting.}
There are in total 21/38 discriminative training tasks, with which we train the UD.
% ~\footnote{The original paper~\citep{T0-paper} claims 39 training datasets but releases a training set with 38 datasets (``common\_gen''  excluded). We directly start with the released data.}. 
% It consists of a majority of discriminative tasks and a small number of generative tasks.
% We train our \method with 21 discriminative tasks within.
% , which is around 55\% of the original T0 training data.\xhk{change to 21/38=0.55}
The evaluation set covers four types of tasks, including natural language inference (RTE~\citep{2005_RTE}, CB~\citep{de2019_CB}, ANLI/R1-R3~\citep{NieWDBWK20_ANLI}), coreference resolution (WSC~\citep{WSC2012}, Winogrande~\citep{SakaguchiBBC20_winogrande}), sentence completion (COPA~\citep{COPA2011}, StoryCloze~\citep{story_cloze}, Hellaswag~\citep{ZellersHBFC19_hellaswag}), and word sense disambiguation (WiC~\citep{wic-paper}).
Following T0, we use accuracy on the validation split as the evaluation metric.
For prompt-based baselines, we report the average accuracy over multiple prompts for each test task.
Besides, we also evaluate zero-shot performance on several BigBench~\cite{bigbench} tasks, which are also adopted by T0~\cite{T0-paper}.\footnote{The original T0 reported results on 14 BigBench tasks. We separately report the results of 13 discriminative tasks and the other generative task in the following.}


% \lzy{Noted that we also evaluate the zero-shot performance on a subset of Big-Bench Benchmark~\cite{bigbench} adopted by original T0 paper~\cite{T0-paper}.\footnote{The original T0 reported results on 14 BigBench tasks. In our work, we focus on 13 discriminative tasks, leaving improving performance of the only generative tasks for future exploration. \zy{No need to say that. We also have generation results. Just say we're gonna report generation result separately.}}}


\paragraph{Baselines}
We primarily compare our method with T0~\citep{T0-paper}, which is a generative approach.
% that shares the same goal as UD (i.e., zero-shot generalization), but uses a totally different framework (i.e., generative or discriminative) as well as input format (i.e., prompt or minimal prompt).
Another baseline is prompting ELECTRA~\cite{xia2022prompting} which is a recent work on discriminative modeling.
Since it was proposed in a different setting (i.e., a  few-shot setting or direct zero-shot inference without any finetuning), we reproduced their method under our multitask zero-shot setting for comparison.

For a fair comparison, we follow T0 to use the T5-V1.1-LM-Adapted~\citep{T5-paper} as the backbone model, and we experimented with three different scales, respectively 800M, 3B, and 11B. 
For UD, it only makes use of the encoder of T5-v1.1 and additionally replaces the output layer with a classification head.
Moreover, for direct comparison with \citet{xia2022prompting}, we use DeBERTaV3-Large \citep{debertav3} as the backbone model which shares the same bidirectional architecture and has a smaller number of parameters.

In addition, we also provide reported zero-shot results of several large language models (with hundreds of billions of parameters) for reference, including GPT-3~\cite{gpt3-paper}, GLaM~\cite{glam}, PaLM~\cite{palm}, and FLAN~\cite{FLAN}.


% we also experiment with another backbone DeBERTaV3-Large~\citep{debertav3} to achieve better zero-shot performance.

\paragraph{Training}
% We implemented both baselines and our method, and perform experiments with exactly the same environments.
During training, we truncate the input sequence to 256 tokens and use a batch size of 256. For optimization, we use the Adam optimizer with a fixed learning rate of 1e-5 and a dropout rate of 0.1. Each experiment is trained with 10, 8, and 5 epochs respectively for 800M, 3B, and 11B models.
% We perform checkpoint selection by directly using the final (fixed-epoch) checkpoint for evaluation.
% \xhk{by choosing the one with the maximal average zero-shot performance per xxx steps.}

% For data processing, similar to T0, we truncate any dataset with over MAX\_DATA\_SIZE to have MAX\_DATA\_SIZE / num\_prompts. 
% Different from ~\citet{T0-paper} that uses a value of 500k for MAX\_DATA\_SIZE, we use a value of 50k, which experimentally yields better zero-shot performance for the T0 baseline.
% The training data of UD are produced by \xhk{replacing different prompted data version from T0 training data with only one minimal prompted version}, which strictly guarantees all methods share same raw task data.




% \subsection{Main Results}
\subsection{Main Results on Zero-Shot Tasks}

\paragraph{UD Zero-Shot Results}
The main results are presented in Table~\ref{tab:maintable}.
We compare methods of similar scales. 
Results in Table \ref{tab:maintable:top} show that our UD substantially outperforms the T0 baseline on average by a large margin of around 9, 5, and 7 points respectively at Large, XL, and XXL scales.
Comparing the results of UD-T5-Large, UD-DeBERTaV3, and prompting ELECTRA, both variants of UD also substantially outperform prompting ELECTRA by more than 6 points.
% In addition, UD also demonstrates superior zero-shot ability compared with models with hundreds of billions of parameters (see results in the first block of Table~\ref{tab:maintable:top}).
On BIG-Bench datasets, results in Table \ref{tab:maintable:bottom} show that our UD outperforms the T0 baseline by a margin of around 4-8 points.
Overall, these results demonstrate the advantages of UD at every scale, and a broad range of tasks compared with baselines.

Another interesting finding is that the advantages of UD significantly increase along with scaling.
When scaling from Large-scale to XL-scale (i.e., around 3.75x of the parameters), the average performance improves by around 2 points. However, when scaling from XL-scale to XXL-scale (i.e., 3.6x of the parameters), the improvements of average zero-shot performance enlarge to 8 points.
Based on the observation, we hypothesize that UD can achieve even better performance of zero-shot generalization if further scaling to an even larger models, which we leave to future work.

% Results show that our \method substantially outperforms our baseline T0 on average zero-shot performance, by a large margin of around 8, 5, and 7 points respectively at the Large (800M), XL (3B), and XXL (11B) scales.
% \xhk{Our UD also substantially outperforms ELECTRA in the Large (800M) scale.} 

To further boost the zero-shot performance, we also train a new variant of UD at 11B scale by scaling to more training tasks, including the discriminative English tasks used in \citet{1600tasks}, and the discriminative English tasks used in \citet{ul2}. The new model is denoted as UD+.
UD+ achieves the highest average accuracy among all the zero-shot evaluation tests.

% \begin{comment}
% ul2 (CommonsenseQA \cite{commonsense_qa}, ),
% csqa2.json 9264
% glue_cola.json 8551
% glue_sst2.json 67349
% glue_stsb.json 5749
% mcscript.json 19462
% mcscript2.json 28382
% openbookqa.json 19828
% qasc.json 40670
% qasc_with_ir.json 40670
% race_high.json 249780
% race_middle.json 101684
% social_i_qa.json 100230
% super_glue_boolq.json 9427
% super_glue_multirc.json 27243
% ai2_science_elementary.json 2493
% ai2_science_middle.json 2424
% onestopqa_advanced.json 1296
% physical_iqa.json 33211
% protocol_comparison_harsht.json 3698
% reclor.json 3726
% ai2_arc_ARC_Easy.json 9002
% ai2_arc_ARC_Challenge.json 4476r,  
% \end{comment}
% \zy{what data?}

% \xhk{move bigbench result in appendix A.1 here. into Table 2}
% \yn{add bigbench analysis}

% , in the mean time still guaranteeing that there is no overlap between training tasks and the held-out tasks.

% \xhk{We also extend our training datasets (please refer to appendix~\ref{sec:ud_plus_data}) and train a model UD+. 
% UD+ achieves the highest average accuracy among all the zero-shot evaluation test, in the mean time still guaranteeing that there is no overlap between training tasks and the held-out tasks.}

% \yn{do we need to add the following?}
% \yn{
% Interestingly, we also have observed some findings on zero-shot performance along with scaling.
% For baseline T0, the zero-shot performance keeps improving on most of the datasets when scaling to larger-scale models.
% Exceptions are Hellaswag and WSC, where zero-shot performance on them are basically unchanged when scaling.
% For our \method, the performance of zero-shot generalization consistently improves with the model scale increasing on all sentence completion and coreference resolution tasks, and partial NLI tasks.
% Exceptions are that RTE, CB and WSC demonstrates a degradation on zero-shot performance when scaling from large to XL scale.
% This could be explained that 
% % {\color{red} xxxxx}
% % \xhk{I guess if we add minimal prompts, RTE and CB will improve for larger model...? so maybe we can remove this observation for now?}
% }


% \paragraph{Results on Finetuned Tasks}

% To evaluate the performance on finetuned tasks, we finetuned T0/UD respectively on each training task. This is similar to multi-task finetuning \cite{T5-paper}.
% % We use this experiment to test the effectiveness of UD with abundant labels. 
% We experimented with all the T0 discriminative training tasks.
% Table~\ref{tab:finetunedtasks} shows the finetuning results on T0 and UD at the 11B scale.
% We observe that UD outperforms T0 on \textbf{\color{red} xxx/19} of the considered finetuned tasks.
% To be specific, on topic classification tasks, paraphrase identification tasks, and multiple-choice QA tasks, UD shows the largest advantages against T0. These finetuning results demonstrate that UD does not only perform well in the zero-shot setting but also improves performance when abundant labels are available.



% \yn{add a new subsection of seq2seqUD}
\paragraph{Generalized UD Zero-Shot Results}

The zero-shot results of generalized UD on 11 T0 discriminative test tasks and on 13 Big-Bench tasks are respectively reported in Table~\ref{tab:gen_ud:top} and Table~\ref{tab:gen_ud:mid}.
In addition, to test how generalized UD performs on zero-shot generative tasks, we also select 4 generative tasks from Big-Bench for evaluation. Results are presented in Table~\ref{tab:gen_ud:bottom}.


Analyses are as follows.
(1) Comparing the results of generalized UD and T0, generalized UD still holds significant improvements on discriminative tasks.
(2) Comparing generalized UD with our previous UD (in Table~\ref{tab:maintable}), we observe there is a slight decrease in average performance, proving that adding generative tasks into training could have impacted a little bit, in trade for capability for handling generative tasks.
(3) On 4 generative zero-shot tasks, both generalized UD and T0 show comparable results.
(4) On 13 discriminative BigBench tasks, we observe that UD-Large outperforms T0-Large by 6.67\%, UD-XL outperforms T0-XL by over 4\%, and Generalized UD-XL outperforms T0-XL by over 6\%, further indicating the effectiveness of our proposed framework.


%\yn{recheck the analysis along with table data!}

% Comparing Generalized UD with methods in Table~\ref{tab:maintable} (i.e., UD and T0) of similar scales, we observe the zero-shot performance on discriminiative tasks slightly decrease but generally hold still, compared to UD (ours).

% It still significantly outperforms baseline T0 to a large degree.
% From Table~\ref{tab:gen_ud}, we shall observe, on generative tasks both generalized UD and T0 show comparable results.}









\subsection{SOTA Results on Finetuned Tasks}
\label{sec:ud_finetune}

To explore how UD performs on fully-supervised tasks, we finetuned UD for a wide range of downstream tasks and reported their results in Table \ref{tab:finetune}.
% To explore whether UD can help improve the performance in fully-supervised learning, we conduct experiments by finetuning each downstream task. 
For each finetuning experiment, the maximum training epoch is set to be 10.
We search a hyper-parameter space with learning rate in \{2e-5, 1e-5, 5e-6\}, batch size in \{32, 64, 128\}.
We select the best checkpoint using a validation set with early stopping.
% We set the maximum training epoch to 10, search the hyper-parameters (learning rate in \{2e-5, 1e-5, 5e-6\}, batch size in \{32, 64, 128\}) and select the best checkpoint based on the validation set with early stopping.

% Results are in Table \ref{tab:finetune}.
From results in Table \ref{tab:finetune}, we find that UD can achieve remarkable performance on most of the downstream tasks. 
We achieve state-of-the-art performance on 12 out of the 17 tasks we evaluated. The results also show that more challenging tasks (tasks that require more knowledge) will benefit more from the multi-task training period, especially some QA tasks.








\subsection{Ablation Study}

We have also conducted ablation studies to further explore how several factors affect the performance of zero-shot generalization. 

\subsubsection{Instructive Prompts vs Minimal Prompts}

UD employs minimal prompts that use simple concatenation, while previous approaches rely on lengthy instructive prompts to provide more detailed instructions \cite{T0-paper,FLAN,gpt3-paper}. 
Statistically, we count the average number of prompt words (excluding raw input) for both minimal and instructive prompts, and statistics are respectively $0.4$ versus $>10$.
% \xhk{A statistic comparison on the average number of the prompt word count (excluding raw input) is $0.4$ for minimal prompts versus $>10$ for previous instructive prompts.}  
We compare these two types of prompts in the following experiment.
We adopt the instructive prompts from T0 and apply them on UD without changing the discriminator formulation. To construct minimal prompts for T0, we remove all the instructive words similar to UD.

% It is an interesting question whether minimal prompts also play a role in the \method, 
%considering that concatenating task data with prompts theoretically indeed reduces all tasks into the original LM tasks, hence improving task generalization.
% considering that simple concatenation of task data's keywords with a minimal prompt is enough to unify it into the UD format.

% We compare the zero-shot performance when using prompt and minimal prompt for \method and prompt and prompt-free for T0. 



% To construct instructive prompts for UD, we adopt the instructive prompts from T0 we concatenated the prompted inputs and each target choice (verbalizer).~\footnote{Here we use the same prompts as T0.} The corresponding \method label is 1 when concatenating correct target choice and 0 otherwise.

% \xhk{To construct prompt-free inputs for T0, we directly remove all the prompt words, still letting the model to predict the target verbalizer.}


Results are shown in Table~\ref{tab:promptablatiion}. We observe that minimal prompts yield better performance for UD than instructive prompts. In contrast, for T0, instructive prompts perform much better than minimal prompts. These results are consistent with our motivation that UD tends to unify the tasks better with a shared discrimination formulation. As a result, task-specific instructions are not necessary and might hurt generalization performance. Generative approaches, on the other hand, rely on instructive prompts to better distinguish different tasks.


% \xhk{for our UD method, minimal prompt version has better accuracy, because under the unified UD task format, task descriptive language in prompt is no longer needed and may even increase the sentence complexity to be understood by LM. Additionally, UD's tasks is to discriminate between correct and wrong choices where prompts are identical phrases in each choice's concatenated sentence, so prompts actually play no role in the discriminating process. However, for T0, prompted version has better accuracy than the prompt-free version (note that prompt-free is the extreme and usual case for minimal prompting) because generative model's goal is to generate the correct verbalizer from the huge vocabulary, which can be efficiently narrowed by the existence of prompts. Therefore, we can conclude that minimal prompted format works well for discriminative models and prompted format works well for generative models.}



\subsubsection{Ablation on Base Models}

We also study the effects of using different backbone pretrained models. We experiment with three backbone models of different types, respectively the encoder part of an encoder-decoder model, an encoder model, and a decoder model. Specifically, we use the T5 encoder, DeBERTa \cite{debertav3}, and GPT \cite{radford2018gpt} respectively for these three types. It is noteworthy that though similar in architecture for both T5 encoder and DeBERTa, they are pretrained with different self-supervised language modeling tasks, which in fact leads to huge differences in zero-shot generalization, as we will show in Table~\ref{tab:ablationbasemodel}.
% We study the effect of different backbone pretrained models. We experiment with three types of backbone models---using the encoder part of an encoder-decoder model, using an encoder model, and using a decoder model. We use the T5 encoder, DeBERTa \cite{debertav3}, and GPT \cite{radford2018gpt} respectively for these three types.






% We study the effect of different types of models (discriminative vs. generative), or backbone models (auto-encoding vs. auto-regressive), on zero-shot generalization with \method. In addition to T5-Encoder, we also experiment the advanced DeBERTaV3-Large~\cite{debertav3} that has achieved new SOTA on a diverse set of tasks. We also implement GPT-XL for comparision.

% Results are shown in Table~\ref{tab:promptablatiion}.

% shows the results between discriminative and generative models with fixed prompted or not version. It can be observed \xhk{no matter we use promped data or minimal prompt/prompt-free data, our discriminative models always have better zero-shot generalization performance than generative models.}



Results of different backbone models are presented in Table \ref{tab:ablationbasemodel}. 
Among all three types of backbone models, the encoder backbone models appear to be the most suitable type of backbone, where both encoder models of two scales respectively achieve the best and the second best results, outperforming all the others by more than 5 points.

Using the same number of parameters (i.e., 1.5B), both DeBERTa-V2 and T5-Encoder significantly outperform GPT-XL, which demonstrates that a bidirectional architecture works better than the unidirectional architecture for the discriminator formulation.
In addition, DeBERTa-V2 outperforms T5-Encoder by 7 points, implying that not only model architecture but also the self-supervised pretraining task determines the ability of UD discrimination. Models pretrained with masked language modeling tasks are more suitable for UD.

The impacts of the architecture and pretraining tasks of backbone models are even larger than the influence of scale, as we also observe that an encoder model with 300M parameters (i.e., DeBERTaV3) achieves much better performance than the T5 encoder and GPT-XL with 1.5B parameters.

% Results are shown in Table \ref{tab:ablationbasemodel}. Using the same number of parameters, encoder backbone models (i.e., DeBERTa) substantially outperform the T5 encoder and the GPT decoder. This indicates that pretrained encoders are more suitable for our discriminator formulation. Interestingly, an encoder model with 300M parameters (i.e., DeBERTaV3) achieves much better performance than the T5 encoder and GPT-XL with 1.5B parameters.








% Table~\ref{tab:ablationbasemodel} shows the results for different discriminative models, where DeberTa consists of solely an encoder, T5-Encoder is the encoder part of the full T5 model, GPT-XL consists of an encoder and a decoder. We can observe that the encoder structure performs better for discriminative tasks.

% \subsection{What Contribute to the Zero-Shot Generalization of \method?}

\subsection{How Well UD Generalizes to a Broader Domain?} \label{sec:generalize}

In the previous sections, we have trained UD to solve the task of discriminating whether a text sample comes from the true data distribution of natural language. So far we have constrained the problem to supervised labeled tasks. However, this discrimination problem formulation is in fact general and can be applied to a broader domain of natural language. We conduct the following experiment to see how UD generalizes.


% In order to explore the mechanism of the universal discriminator and explain how it promotes zero-shot generalization. We conduct the following extensive experiment.

To test whether a model discriminates against the true data distribution, a straightforward way of verification is to compare the probability of real data with that of some generated, fake data. This form of verification is not specific to any downstream task and can be viewed as generalizing to a broader domain. Formally, given a text sample $x$, let $D(x)$ be the output of UD, which estimates the probability that $x$ is sampled from the true data distribution, i.e., $P(\text{true} | x)$. Given a true data sample $x$ and a generated data sample $x'$, we expect a well-trained UD to predict $D(x) > D(x')$.

% First, we assume that the essence of our universal discriminator $D$ is to learn whether the data are sampled from the real text distribution or not. A straightforward way to verify this key point is to compare the likelihood of the real data label given real data x computed as $D(x)=p(y=1|x)$ with the likelihood of the real data label given generated data $x'$ computed as $D(x’)=p(y=1|x’)$. 

Specifically, we randomly select 2,600 real data samples $x$ from the validation set of the T0 training data and generate the data $x’$ in two different ways: model-based generation and manual generation.

For a model-based generation, we utilize the T0-Large model with a paraphrase prefix ``Paraphrase the sentence:'' to generate data $x'$. It is expected that the generated samples $x'$ are similar to true samples $x$ to some extent but demonstrate some flaws that are unique to generated data. For a manual generation, we manually create some conflict or contradiction in the real sample $x$. Specifically, we manually attach wrong answers to the original data and obtain $x’$ , which is similar to what we have done in constructing negative samples in our main framework. 

We then use our \method based on T5-Encoder Large to compute the probability $D(x)$ and $D(x')$ for both real and generated data. As displayed in Table~\ref{tab:explain}, we find that the \method assigns a higher score for $x$ than $x'$ $80\%$ of the time for manually-generated data. When tested with model-generated data, UD assigns a high probability for real data in $74\%$ of the cases.
This is probably because manually generated data are more paradoxical and logically incoherent and thus are easier for UD to discriminate. Overall, these results demonstrate that the discrimination ability of UD is not limited to the downstream tasks on which it was trained, but is also generalizable to a broader domain of text data. This indicates a possibility of extending UD to other scenarios such as model pretraining and generation tasks.


% For model-based generation, we utilize two models which generate high-quality and low-quality data $x$. It should be noted that we hope the generated $x'$are similar to $x$ to some extent.
% First, we leverage a T5-small model [citation] to generate similar semantics to real data x by feeding the $x$ with the prefix  ‘paraphrase:’. Obviously, the generated $x'$ are bound to be far from real data distribution. Then, we utilize the T5-small model to finetune on quora for paraphrase identification task. Then we leverage the finetuned T5-small model to do the same paraphrase generation as before and yield $x’$ with relatively high quality. 

% For heuristic-based generation, we manually create some conflict or contradiction in the real data $x$. In detail, we randomly shuffle the words given each real data sample and get inconsistent data $x’$.

% After generating the data $x’$ from different approaches, we evaluate the likelihood of real data distribution given real data $x$ and generated data $x’$, which is formulated as $D(x)=p(y|x)$ and $D(x’)=p(y|x’)$ respectively. The results are shown in Table [reference] and the generated data examples are presented in Appendix [reference].





\section{Conclusion}
In this paper, we extend the idea of SynGEC \cite{zhang2022syngec} and propose the CSynGEC approach to enhance GEC models by exploiting tailored constituent-based syntax. Experimental results show that incorporating constituent-based syntax produced by a GEC-oriented constituency parser can effectively help GEC models. 
Furthermore, we attempt to combine dependency-based and constituent-based syntax from both intra-model and inter-model aspects, and find that simultaneously using two kinds of syntax leads to more obvious improvement.



\begin{acks}
  This work was partially supported by JST PRESTO Grant Number JPMJPR2122 and
  JSPS KAKENHI Grant Numbers JP17K12646, JP19K20218, JP20K19739, JP21K17708, and JP21H03397.
\end{acks}


%\clearpage
%%%%%%%%%%%%%%%%%%%%%%%%%%%%%%%%%%%%%%%%%%%%%%%%%%%%%%%%%%%%%%%%%%%%%%%%%%%%%%%%%%%%%%%%%%%%%%%%%%%%%%%%%
%% bibliography: see CFP for number of permitted pages
%\newpage
\bibliographystyle{ACM-Reference-Format}  % do not change this line!
\bibliography{ref}  % put name of your .bib file here




% \appendix

\section{Supplemental Tables}

%\section{Hyperparameters of Other Bandit Algorithms}
%\label{sec:bandit_hyperparams}
%Table~\ref{tab:hyperparams} lists the hyperparameters for bandit algorithms other than dBE.

\newcommand\topmidheader[2]{\multicolumn{#1}{c}{\textbf{#2}}\\%
                \addlinespace[1ex]}

\newcommand{\midheader}[2]{%
        \midrule\topmidheader{#1}{#2}}

\newcommand{\specialcell}[3][c]{% 
        \begin{tabular}[#1]{@{}#2@{}}#3\end{tabular}}%

\aptLtoX[graphic=no,type=env]{\begin{table}[htb]
  \centering
  \caption{Hyperparameters of bandit algorithms}
  \label{tab:hyperparams}
  \begin{tabular}{llc}
    \toprule
    Sign & Description & Value \\
    \multicolumn{3}{c}{\textbf{UCB1}}\\
    $c$ & Parameter to control the confidence level used in $\sqrt{c \cdot {\log{t}}/{N_t(arm)}}$ & 0.5  \\
    \multicolumn{3}{c}{\textbf{Thompson Sampling}}\\
    $p(\theta)$ & Prior Distribution & $\mathcal{B}(1, 1)$ \\
    \multicolumn{3}{c}{\textbf{discounted Thompson Sampling}}\\
    $\gamma$ & Discount factor & $1-10^{-8}$ \\
    \multicolumn{3}{c}{\textbf{discounted Thompson Samplingadaptive shrinking Thompson Sampling}}\\
    $M$ & Parameter to control memory usage in a data structure ADWIN2 \cite{ADWIN} & 10 \\
    $\delta$ & Parameter to control the confidence level in a data structure ADWIN2 & $1-10^{-7}$ \\
    \multicolumn{3}{c}{\textbf{EXP-IX}}\\
    $\eta_t$ & Parameter used for weights of arms & $\sqrt{\frac{2 \cdot \log{K}}{K \cdot t}}$ \\
    \addlinespace[1ex]
    $\gamma_t$ & Parameter used for loss estimates & $\frac{\eta_t}{2}$ \\
    \multicolumn{3}{c}{\textbf{EXP3++}}\\
    $\alpha$ & Constant used in calculating $\xi_t(a)$ & $3$ \\
    $\beta$ & Constant used in calculating $\xi_t(a)$ & $256$ \\
    \bottomrule
  \end{tabular}
\end{table}}{\begin{table}[htb]
  \centering
  \caption{Hyperparameters of bandit algorithms}
  \label{tab:hyperparams}
  \begin{tabular}{llc}
    \toprule
    Sign & Description & Value \\
    \midheader{3}{UCB1}
    $c$ & \specialcell{l}{Parameter to control the confidence \\ level used in $\sqrt{c \cdot {\log{t}}/{N_t(arm)}}$} & 0.5  \\
    \midheader{3}{Thompson Sampling}
    $p(\theta)$ & Prior Distribution & $\mathcal{B}(1, 1)$ \\
    \midheader{3}{discounted Thompson Sampling}
    $\gamma$ & Discount factor & $1-10^{-8}$ \\
    \midheader{3}{adaptive shrinking Thompson Sampling}
    $M$ & \specialcell{l}{Parameter to control memory usage \\ in a data structure ADWIN2 \cite{ADWIN}} & 10 \\
    $\delta$ & \specialcell{l}{ Parameter to control the confidence \\ level in a data structure ADWIN2} & $1-10^{-7}$ \\
    \midheader{3}{EXP-IX}
    $\eta_t$ & Parameter used for weights of arms & $\sqrt{\frac{2 \cdot \log{K}}{K \cdot t}}$ \\
    \addlinespace[1ex]
    $\gamma_t$ & Parameter used for loss estimates & $\frac{\eta_t}{2}$ \\
    \midheader{3}{EXP3++}
    $\alpha$ & Constant used in calculating $\xi_t(a)$ & $3$ \\
    $\beta$ & Constant used in calculating $\xi_t(a)$ & $256$ \\
    \bottomrule
  \end{tabular}
\end{table}}

\begin{table}[htb]
  \centering
  \caption{Commit IDs of the PUTs used in our vulnerability discovery and AFL++ used as the baseline.}
  \begin{tabular}{lc}
    \toprule
    Program & Commit \\
    \midrule

    AFL++ & 32a0d6ac315 (ver ++3.14c) \\
    Bloaty &  60209eb \\
    HarfBuzz & 77eeec5 \\
    libarchive & 86c9361 \\
       libxml2 & dea91c9 \\
    MuPDF & ef3d68d \\
   PHP & fdf0455f \\
    Poppler & 6d72d82 \\
    PROJ & 76dfefe \\
    QPDF &  3794f8e \\
    libtpm2 & bc3bb26 \\
    Wireshark  & 1fc621e \\
    Xpdf & N/A (ver 4.03) \\

    \bottomrule
  \end{tabular}
\label{tab:commit-ids}
\end{table}


\begin{table}[htb]
  \centering
  \caption{Initial and theoretical maximum values of code coverage of the PUTs in FuzzBench. 
           Initial values were investigated only in the PUTs used.}
  \begin{tabular}{lcc}
    \toprule
    PUT & Initial & Maximum \\
    \midrule

bloaty\_fuzz\_target & N/A & 83114 \\
curl\_curl\_fuzzer\_http & N/A & 78362 \\
freetype2-2017 & 1517 & 26262 \\
harfbuzz-1.3.2 & N/A & 12212 \\
jsoncpp\_jsoncpp\_fuzzer & N/A & 2114 \\
lcms-2017-03-21 & 149 & 7036 \\
libjpeg-turbo-07-2017 & N/A & 9384 \\
libpcap\_fuzz\_both & 2 & 7294 \\
libpng-1.2.56 & 138 & 3736 \\
libxml2-v2.9.2 & 258 & 67994 \\
libxslt\_xpath & N/A & 51456 \\
mbedtls\_fuzz\_dtlsclient & N/A & 12888 \\
openssl\_x509 & 6026 & 54116 \\
openthread-2019-12-23 & N/A & 19846 \\
php\_php-fuzz-parser & N/A & 215210 \\
proj4-2017-08-14 & 46 & 6534 \\
re2-2014-12-09 & 1 & 3982 \\
sqlite3\_ossfuzz & 4767 & 28766 \\
systemd\_fuzz-link-parser & N/A & 1798 \\
vorbis-2017-12-11 & 410 & 4082 \\
woff2-2016-05-06 & N/A & 5708 \\
zlib\_zlib\_uncompress\_fuzzer & N/A & 910 \\

    \bottomrule
  \end{tabular}
\label{tab:fuzzbench_max_cov}
\end{table}

\begin{table}[htb]
\centering
\caption{List of unique bugs found in the 7-day trial (manually triaged).}
\begin{minipage}{\columnwidth}

\centering
\begin{tabular}{lll}
\toprule

ID & PUT & Bug Type \\
\midrule
Bug-A & bloaty & NULL Pointer Deref \\
Bug-B & harfbuzz & Out-of-bounds Read \\
Bug-C & mupdf & Assertion Fail \\
Bug-D & mupdf & NULL pointer deref \\
Bug-E & xpdf & Stack Overflow \\
Bug-F & xpdf & NULL Pointer Deref \\
Bug-G \footnote{CVE-2022-24106 is issued.} & xpdf & Use of Uninitialized Value \\
Bug-H \footnote{CVE-2022-24107 is issued.} & xpdf & Integer Overflow \\
Bug-I & php & Use-After-Free \\
Bug-J & php & Use-After-Free \\
Bug-K & php & NULL Pointer Deref \\
Bug-L & php & Use-After-Free \\ 
Bug-M & php & NULL Pointer Deref \\
Bug-N & php & Assertion Fail \\
Bug-O & php & Use-After-Free \\
Bug-P & php & Use-After-Free \\
Bug-Q \footnote{CVE-2022-23308 is issued.} & libxml2 & Use-After-Free \\
\bottomrule
\end{tabular}

\label{tab:7d-bug}
\end{minipage}
\end{table}

\begin{table*}[htb]
  \centering
  \caption{List of the PUTs used in Section~\ref{sec:banditcomparison}. If the source code of a PUT was maintained in Git, the latest version at the time of the experiment in the master (or main) branch was used for the build. The `+' sign in a version indicates that the used source code is not the official release version of the source code.}
  \renewcommand\tabularxcolumn[1]{m{#1}}
  \renewcommand{\arraystretch}{1.2}
  \begin{tabularx}{\textwidth}{lXllXc}
    \toprule
    Project & Version & Commit ID & PUT & Format of Initial Seeds & Initial Edge Coverage \\
    \midrule
    Bloaty & v1.1+ & 60209eb & fuzz\_target & Executable (e.g., ELF, PE, Mach-O) & 4773\\
    libmpeg2 & N/A & 5432dc1 & mpeg2\_dec\_fuzzer & MPEG2 & 2428 \\
    PHP & 8.0+ & fdf0455f & php-fuzz-execute & PHP source code & 25241 \\
    HarfBuzz & 3.1.0 & 77eeec5 & hb-shape-fuzzer & Font (e.g., TrueType, OpenType) & 15298 \\
    Xpdf & 4.03 & N/A & fuzz\_pdfload & PDF & 4755 \\
    libtpm2 & N/A & bc3bb26 & tpm2\_execute\_command\_fuzzer & TPM command & 3884\\
    libyaml & v0.2.5+ & f8f760f & libyaml\_dumper\_fuzzer & YAML & 1310 \\
    libzip & 1.8.0+ & bff2eb9 & zip\_read\_fuzzer & ZIP & 805 \\
    libgit2 & v1.3.0+ & 50b4d53 & download\_refs\_fuzzer & Git packet & 3911 \\
    file & 5.41+ & fcbb5d8 & magic\_fuzzer & any (e.g., Zstd compressed file) & 1171 \\
%    MuPDF & 1.19.0+ & ef3d68d & pdf\_fuzzer & PDF & 16936 \\
%    libxml2 & 2.9.12+ & dea91c9 & xml & XML & 7027 \\
    \bottomrule
  \end{tabularx}
\label{tab:put_details}
\end{table*}

%\section{Full Results of Some Experiments}
%\label{sec:full_result}

%Table~\ref{tab:alg_cmp_all}, Figure \ref{fig:vis_bandits} and Figure \ref{fig:full_ablation_time_vs_cov} show the omitted results.

\begin{table*}[htb]
\centering
\caption{Median edge coverage obtained by AFL++ and 8 versions of \OurMethodName-AFL++ in 10 PUTs after 24 h. }

\begin{tabular}{lccccccccc}
\toprule

PUT & AFL++ & UCB1 & KLUCB & TS & dTS & dBE & ADS-TS & EXP3-IX & EXP3++ \\
\midrule

bloaty & \textit{1845.5} & 2198.5 & 2246.0 & 2232.5 & 2191.0 & 2292.0 & \textbf{2340.0} & 2181.5 & 2231.5 \\
harfbuzz & \textit{13497.5} & 14031.5 & 14247.5 & 14360.5 & \textbf{14374.0} & 14067.5 & 14149.0 & 13883.0 & 13891.0 \\
xpdf & \textit{3384.0} & 3494.0 & 3812.5 & \textbf{4618.5} & 4166.5 & 3791.5 & 3902.0 & 3860.0 & 3615.0 \\
libzip & \textit{267.5} & 272.0 & 274.0 & 268.0 & 268.5 & 271.5 & \textbf{276.0} & 271.5 & 268.0 \\
libgit2 & 898.0 & 888.5 & 890.5 & 906.5 & \textbf{916.0} & 884.0 & 914.0 & 899.5 & \textit{881.0} \\
php & \textit{9841.5} & 11861.0 & 13551.5 & \textbf{14324.0} & 14187.5 & 12657.5 & 13408.0 & 11423.5 & 11828.5 \\
libmpeg2 & \textit{1873.5} & 1900.5 & 1905.0 & 1905.5 & \textbf{1906.5} & 1903.0 & \textbf{1906.5} & 1897.0 & 1902.0 \\
tpm2 & \textit{281.5} & 299.5 & 313.0 & 317.0 & \textbf{317.5} & 305.0 & 311.0 & 298.5 & 291.0 \\
libyaml & 2811.5 & 2841.0 & \textbf{2841.5} & \textit{2800.5} & 2837.0 & 2827.5 & 2831.5 & 2828.0 & 2834.5 \\
file & 830.5 & 829.5 & 828.0 & 827.0 & 827.5 & 833.5 & \textbf{840.5} & 826.5 & \textit{826.0} \\

\bottomrule

\end{tabular}

\label{tab:alg_cmp_all}
\end{table*}

\begin{table*}[htb]
\centering
\caption{P-value of Mann-Whitney's U test (Holm-Bonferroni corrected) and Vargha-Delaney's $\hat{A}_{12}$ between AFL++ and the fuzzer in the column for the evaluation conducted in Section~\ref{subsec:eval-vs-existing}. If the p-value is bold, the difference is significant in the test ($p < 0.01$). The characters `L', `M', `S' and `N' in parentheses indicate that the effect size is large, medium, small, and none, respectively, according to \cite{A12}. The `+' sign means the fuzzer in the column is superior to AFL++ when compared by rank sum as well as $\hat{A}_{12}$, and the `-' sign means the opposite.}
\begin{tabular}{lllllllllllll}
 \toprule

  & \multicolumn{2}{c}{MOpt} & \multicolumn{2}{c}{CMFuzz} & \multicolumn{2}{c}{Karamcheti} & \multicolumn{2}{c}{\HavocMAB{}} & \multicolumn{2}{c}{SLOPT} \\
  \cmidrule(r){2-3}\cmidrule(r){4-5}\cmidrule(r){6-7} \cmidrule(r){8-9} \cmidrule(r){10-11}
  PUT & $p$ & $\hat{A}_{12}$ & $p$ & $\hat{A}_{12}$ & $p$ & $\hat{A}_{12}$ & $p$ & $\hat{A}_{12}$ & $p$ & $\hat{A}_{12}$ \\
\midrule

openssl\_x509 & \textbf{ < 0.001 } & 0.82 (+L) & \textbf{ 0.023 } & 0.71 (+L) & \textbf{ < 0.001 } & 0.92 (+L) & \textbf{ < 0.001 } & 0.82 (+L) & \textbf{ < 0.001 } & 0.91 (+L) \\
re2-2014-12-09 & \textbf{ < 0.001 } & 0.18 (-L) & > 0.1 & 0.37 (-S) & > 0.1 & 0.38 (-S) & > 0.1 & 0.47 (-N) & > 0.1 & 0.52 (+N) \\
proj4-2017-08-14 & \textbf{ < 0.001 } & 0.08 (-L) & \textbf{ < 0.001 } & 0.86 (+L) & \textbf{ < 0.001 } & 0.99 (+L) & > 0.1 & 0.54 (+N) & \textbf{ < 0.001 } & 0.92 (+L) \\
sqlite3\_ossfuzz & > 0.1 & 0.55 (+N) & \textbf{ < 0.001 } & 0.85 (+L) & \textbf{ < 0.001 } & 0.93 (+L) & 0.1 & 0.68 (+M) & \textbf{ < 0.001 } & 1.00 (+L) \\
libxml2-v2.9.2 & \textbf{ < 0.001 } & 0.08 (-L) & \textbf{ < 0.001 } & 0.93 (+L) & \textbf{ < 0.001 } & 0.98 (+L) & \textbf{ < 0.001 } & 0.97 (+L) & \textbf{ < 0.001 } & 0.84 (+L) \\
freetype2-2017 & \textbf{ < 0.001 } & 0.08 (-L) & 0.094 & 0.33 (-M) & > 0.1 & 0.54 (+N) & > 0.1 & 0.52 (+N) & \textbf{ < 0.001 } & 0.79 (+L) \\
libpcap\_fuzz\_both & > 0.1 & 0.57 (+S) & \textbf{ < 0.001 } & 0.79 (+L) & \textbf{ < 0.001 } & 0.80 (+L) & \textbf{ < 0.001 } & 0.87 (+L) & \textbf{ < 0.001 } & 0.81 (+L) \\
libpng-1.2.56 & > 0.1 & 0.42 (-S) & > 0.1 & 0.36 (-M) & > 0.1 & 0.49 (-N) & > 0.1 & 0.56 (+S) & 0.049 & 0.68 (+M) \\
lcms-2017-03-21 & > 0.1 & 0.45 (-N) & \textbf{ 0.037 } & 0.70 (+M) & \textbf{ < 0.001 } & 0.85 (+L) & > 0.1 & 0.37 (-S) & \textbf{ < 0.001 } & 0.88 (+L) \\
vorbis-2017-12-11 & > 0.1 & 0.39 (-S) & > 0.1 & 0.56 (+S) & \textbf{ < 0.001 } & 0.20 (-L) & > 0.1 & 0.62 (+S) & 0.092 & 0.65 (+M) \\

\bottomrule
\end{tabular}
\label{tab:statistics}
\end{table*}

\clearpage

\section{Algorithm Overview}

\begin{algorithm}[H]

\centering
\caption{Pseudocode of \OurMethodName{}}
\label{alg:slopt}

\begin{algorithmic}[0]

\Require{\mbox{}\\
    $initial\_seeds$ -- a set of initial test cases \\
    $program$ -- a PUT to be fuzzed
}

\Ensure{\mbox{}\\
    $queue$ -- a set of valuable test cases \\
    $crashes$ -- a set of test cases that trigger crashes
}

%\begin{adjustwidth}{-9pt}{}
%\setstretch{0.85}
\vspace{5pt}

\Function{RandomMutation}{$seed, instance_{mut}, instances_{bat}$}
\State $input$ $\gets$ \Call{CopyBytesFromSeed}{$seed$}
\State $mutation$ $\gets$ \Call{SelectArm}{$instance_{mut}$}
\State $idx$ $\gets$ \Call{GetGroupIndex}{$len(input)$}
\State $batch\_size$ $\gets$ \Call{SelectArm}{$instances_{bat}[idx][mutation]$}
\For{$i$ $\gets$ $1$ \textbf{to} $batch\_size$}
    \State $pos$ $\gets$ \Call{SelectPosition}{$input$}
    \State $input$ $\gets$ \Call{ApplyOperator}{$mutation, input, pos$}
\EndFor
\State \textbf{return} $input, mutation, batch\_size$
\EndFunction

%\end{adjustwidth}

%\vspace{-6pt}

%\begin{adjustwidth}{-9pt}{}
%\setstretch{0.85}

\vspace{5pt}

\Function{MutationFuzzing}{$initial\_seeds, program$}

\State $crashes$ $\gets$ $\varnothing$
\State $queue$ $\gets$ \Call{ConstructQueue}{$initial\_seeds$}
\State $instance_{mut}$ $\gets$ \Call{CreateBanditArms}{$number\_of\_mutations$}
\For{$i$ $\gets$ $1$ \textbf{to} $5$}
 \For{$j$ $\gets$ $1$ \textbf{to} $number\_of\_mutations$}
  \State $instances_{bat}[i][j]$ $\gets$ \Call{CreateBanditInstance}{$7$}
 \EndFor
\EndFor

\State

\While{ $\neg$ \Call{UserWantsStop}{\null}}
 \State $seed$ $\gets$ \Call{SelectSeed}{$queue$}
 \State $energy$ $\gets$ \Call{DecideEnergy}{$seed$}
 \For{$i$ $\gets$ $1$ \textbf{to} $energy$}
  \State $input, mutation, batch\_size$ 
  \State $\gets$ \Call{RandomMutation}{$seed, instance_{mut}, instances_{bat}$}
  \State $result$ $\gets$ \Call{ExecutePUT}{$program, input$}
  \State $b$ $\gets$ \Call{WasInputValuable}{$result$}
  \State \Call{RewardArm}{$mutation, b$}
  \State \Call{RewardArm}{$batch\_size, b$}
  \State \Call{SaveInputIfValuable}{$queue, input, result$}
  \State \Call{SaveInputIfCrash}{$crashes, input, result$}
 \EndFor
\EndWhile
\EndFunction

%\end{adjustwidth}

\end{algorithmic}
\end{algorithm}


\end{document}
