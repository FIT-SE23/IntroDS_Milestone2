\documentclass{article}
\usepackage{graphicx}

% if you need to pass options to natbib, use, e.g.:
\PassOptionsToPackage{square,numbers}{natbib}
% before loading neurips_2022


% ready for submission
%\usepackage{neurips_2022}


\bibliographystyle{abbrvnat}
% to compile a preprint version, e.g., for submission to arXiv, add add the
% [preprint] option:
%     \usepackage[preprint]{neurips_2022}


% to compile a camera-ready version, add the [final] option, e.g.:
     \usepackage[final]{neurips_2022}


% to avoid loading the natbib package, add option nonatbib:
%    \usepackage[nonatbib]{neurips_2022}


\usepackage[utf8]{inputenc} % allow utf-8 input
\usepackage[T1]{fontenc}    % use 8-bit T1 fonts
\usepackage{hyperref}       % hyperlinks
\usepackage{url}            % simple URL typesetting
\usepackage{booktabs}       % professional-quality tables
\usepackage{amsfonts}       % blackboard math symbols
\usepackage{nicefrac}       % compact symbols for 1/2, etc.
\usepackage{microtype}      % microtypography
\usepackage{xcolor}         % colors


\title{UATTA-ENS: Uncertainty Aware Test Time Augmented Ensemble for PIRC Diabetic Retinopathy Detection}


% The \author macro works with any number of authors. There are two commands
% used to separate the names and addresses of multiple authors: \And and \AND.
%
% Using \And between authors leaves it to LaTeX to determine where to break the
% lines. Using \AND forces a line break at that point. So, if LaTeX puts 3 of 4
% authors names on the first line, and the last on the second line, try using
% \AND instead of \And before the third author name.


\author{
  Pratinav Seth \thanks{Manipal Institute of Technology, Manipal Academy of Higher Education, Manipal, India}\\
  Dept. of Data Science and Computer Applications\\
  \texttt{seth.pratinav@gmail.com} \\
  % examples of more authors
   \And
   Adil Khan \footnotemark[1]\protect\phantom{\footnotesize 1}\footnotemark[2]\\
   Dept. of Mechanical Engineering\\
   \texttt{adilk5020@gmail.com} \\
   \And
   Ananya Gupta \footnotemark[1]\protect\phantom{\footnotesize 1}\footnotemark[2]\\
   Dept. of Electronics and Comm. Engineering \\
   \texttt{ananyag1018@gmail.com} 
   \And
     Saurabh Kumar Mishra \footnotemark[1]\protect\phantom{\footnotesize 1}\thanks{Equal contribution.}  \\
  Dept. of Computer Science and Engineering \\
   \texttt{saurabhskm22@gmail.com}
   \And
   Akshat Bhandari \footnotemark[1]\\
   Dept. of Computer Science and Engineering\\
   \texttt{akshatbhandari15@gmail.com} \\
}
\begin{document}
\maketitle
\begin{abstract}
Deep Ensemble Convolutional Neural Networks has become a methodology of choice for analyzing medical images with a diagnostic performance comparable to a physician, including the diagnosis of Diabetic Retinopathy. However, commonly used techniques are deterministic and are therefore unable to provide any estimate of predictive uncertainty. Quantifying model uncertainty is crucial for reducing the risk of misdiagnosis. A reliable architecture should be well-calibrated to avoid over-confident predictions. 
To address this, we propose a UATTA-ENS: Uncertainty-Aware Test-Time Augmented Ensemble Technique for 5 Class PIRC Diabetic Retinopathy Classification to produce reliable and well-calibrated predictions. Implementation is available at  \url{https://github.com/ptnv-s/UATTA-ENS}.
\end{abstract}

\section{Introduction and Previous Works}
In recent years, there has been rapid advancement in the field of Deep Learning; subsequently, enormous headway has been made in applying computer vision techniques to medical imaging \cite{Gondal2017}\cite{8869883}. Different algorithms have been developed and fine-tuned for various diseases \cite{mishra2021use}\cite{Karki2021DiabeticRC}. A critical problem in biomedical image analysis is the occurrence of batch effects \cite{leek2010tackling}, which are differences introduced through technological artifacts among different subsets of data. Sample handling and data collecting techniques make applying computer vision algorithms to data from different pathological labs problematic, which is an essential step in developing ML models.

Diabetic Retinopathy (DR) is a common disease that causes vision loss or, in some cases, even blindness among people with diabetes, affecting 415 million people \cite{Sabanayagam2016TenET}. Automated detection of DR is essential to limit the progression of DR by conducting early diagnoses on diabetic patients, as the normal procedure by ophthalmologists demands many resources.
Therefore, the interest in employing deep neural networks to automatically classify Diabetic Retinopathy has grown over the past few years \cite{Gulshan2016DevelopmentAV}\cite{Gargeya2017AutomatedIO}.

A well-calibrated classifier would place less probability mass on uncertain classes. The issue of uncertainty estimation is especially important in the medical domain in order to trust confident model predictions for screening automation and referring uncertain cases for manual intervention of a medical expert \cite{rahaman2021uncertainty}. Bayesian probability theory offers a sound mathematical framework to design machine learning models with an inherent and explicit notion of uncertainty \cite{Yang2021UncertaintyQA}. Instead of resulting in a single per-class probability, such models can estimate the moments of the output distribution for every class, including mean and variance.

Multiple probabilistic and Bayesian methods such as \cite{Graves2011PracticalVI}\cite{pmlr-v37-blundell15}\cite{HernndezLobato2015ProbabilisticBF}\cite{Blum2015VariationalDA}\cite{Gal2016DropoutAA}\cite{Lee2018DeepNN}\cite{Wu2019DeterministicVI}\cite{Pearce2020UncertaintyIN} and non-Bayesian methods such as \cite{Osband2016RiskVU}\cite{Sarawgi2020WhyHA}\cite{Lakshminarayanan2017SimpleAS} and \cite{Dusenberry2020AnalyzingTR} have been proposed to quantify the uncertainty estimates. The ensemble of networks can further improve the performance of models. There has been some work on incorporating uncertainty where ensemble methods \cite{Kendall_2018_CVPR} learned multiple tasks by using the uncertainty predicted as weights for the losses of each of the models, thus outperforming individual models trained on each task. Non-Bayesian alternatives can offer simpler yet effective means to quantify the uncertainties of DNNs. While the ensemble approach is simple and easy to implement, it is typically costly to work with many networks during training and inference. Here, we introduce a reliable end-to-end framework for the classification of Diabetic Retinopathy fundus images that combines the concepts of TTAug \cite{Wang_2019} and UA-Ensemble \cite{sarawgi2021uncertainty}, allowing us to obtain a reliable, well-calibrated final prediction by making the final ensemble of predictions aware of the model's inherent uncertainty. The novelty of our approach lies in it's ability to produce well-calibrated reliable results without compromising on model's efficacy. 

\begin{figure*}[t]
  \centering
  \includegraphics[width=0.69\textwidth]{UATTADIG.pdf}
   \caption{UATTA-ENS : Model Architecture}
   \label{fig:onecol}
\end{figure*}

\section{Methodology}
\subsection{Model Architecture}
\label{subsection:Model Architecture}

Given a training set
(x$_n$,y$_n$)$^N_{n=1}$
consisting of N i.i.d. samples, we model an end-to-end architecture as depicted in Figure \ref{fig:onecol}. 
The architecture is trained in two stages. In the first stage, four individual models are trained to classify the disease. In the second stage, we use TTAug \cite{Wang_2019} as described in section \ref{subsection:tta}. The pre-trained models receive the augmented images as input, which they then use to produce individual classification outputs. These individual classification outputs are then ensembled using calculated uncertainty as weights to produce the final output, or $\hat{y}$, and the entire architecture is fine-tuned over the training set.
\subsection{Test-Time Augmentation}
\label{subsection:tta} 
Test-time Augmentation is a technique used to improve the performance of image classification and produce well-calibrated uncertainty estimates \cite{Ayhan2020ExpertvalidatedEO}\cite{Wang_2019}. The train and test images are randomly augmented, which accounts for the noise, perturbation, and quality degradation that may arise in real-world data, thus making the model more robust. The transformations used in our approach include augmentation in brightness (-0.15 $<$ b $<$ 0.15), saturation (0.5 $<$ s $<$ 2.5), hue (-0.15 $<$ h $<$ 0.15), and contrast (0.5 $<$ c $<$ 1.5) values, cropping a random portion of the image and consequently resizing it to the original dimensions, random horizontal and vertical flipping, respectively. 
\subsection{Uncertainty Metrics}
\label{subsection:uncertainty}
We calculate the uncertainty estimation of the output of each model to curate an uncertainty-aware ensemble model. We use three metrics to quantify the uncertainty - Expected Calibration Error (ECE) \cite{nixon2019measuring}, Maximum Calibration Error (MCE) \cite{NEURIPS2019_1c336b80}, and Brier Score \cite{rufibach2010use}. More information about them is given in Appendix \ref{section:appA}.The output prediction $\hat{y_i}$ where $i\in(1, N)$ is weighted with its uncertainty. For the final prediction, $\hat{y}$, the weighted ensemble quantifies an uncertainty weighted average.
\begin{equation}
\hat{y}\left(\mathbf{x}_n\right)=\frac{\sum_{j=1}^k \frac{1}{\sigma_{h^j}\left(\mathbf{x}_n\right)} \hat{y}_{h^j}\left(\mathbf{x}_n\right)}{\sum_{j=1}^k \frac{1}{\sigma_{h^j}\left(\mathbf{x}_n\right)}}
\end{equation}
Here, $x_{n}$ denotes a nth input image, $\hat{y}_{h^j}(x_{n})$ is the output of nth input image for the $j_{th}$ model prediction, $\sigma_{h^j}$ is estimated uncertainty  corresponding to predictions from the $j_{th}$ model prediction. The uncertainty weights are formulated by taking an inverse of the independent uncertainty metric \cite{sarawgi2021uncertainty}. Hence, $\hat{y}\left(\mathbf{x}_n\right)$ outputs the final prediction corresponding to $n^{th}$ data point. Here the uncertainty associated with each prediction is quantified using a modified version of LLFU \cite{Lakara2021EvaluatingPU}.
\begin{equation}
\label{sec:llfu}
    \sigma_{h^j} = max(0,\frac{1}{2}log(2\pi\sigma^2(x_{n}))) 
    + 
    \frac{(y_j(x_{n})-\mu(x_{n}))^2}{2\sigma^2(x_{n})}
\end{equation}
where $y_j(x_{n})$ - denotes the prediction corresponding to $jth$ model, $\mu(x_{n})$ refers to the mode of predictions from all the models in ensemble and $\sigma^2(x_{n})$ refers to standard deviation of predictions of models in ensemble for the $n^{th}$ data point \cite{Jaskari2022UncertaintyAwareDL}. 

\begin{table*}[t]
\centering
\begin{tabular}{cccll}
\hline
Model Architecture       & Cohen-Kappa & ECE & MCE & Brier Score \\ \hline
Baseline(ResNet-50)        & 0.65            &0.25     &0.57     &0.27             \\
Ensemble         & 0.64            &0.15     &0.50     &0.16             \\
TTAug Ensemble     & 0.65             &0.17     &0.44     &0.17             \\
Uncertainity Aware Ensemble     & \textbf{0.68}            &0.16     &0.47     &0.18             \\
UATTA-ENS & 0.66            &\textbf{0.15}     &\textbf{0.29}     &\textbf{0.15}             \\ \hline
\end{tabular}
\caption{Test Accuracy and Uncertainty Metrics for various model architectures. }
  \label{tab:comb}
\end{table*}

\section{Experiments}
We conduct a series of experiments to study whether the architecture described in section \ref{subsection:Model Architecture} would be reliable in classifying the images. 
We use the APTOS 2019 Blindness Detection Dataset \cite{APTOS_DS}. The dataset contains labelled images of human retinas exhibiting varying degrees of Diabetic Retinopathy collected in India using different medical equipment. We use 90\% of the images (3,302 images) as a train set and the other 10\% (360 images) as a secondary validation set. The datasets consist of colored images of the human retina and are graded using the following 5-class PIRC system for the severity scheme of Diabetic Retinopathy. Each image was graded on the 0-to-4 scale \cite{Jaskari2022UncertaintyAwareDL}. \\ The images in the dataset have been resized to 512X512 and normalized. The black background of the images has been removed to focus more on the fundus image \cite{Huang2022IdentifyingTK}.
We have used four models in the ensemble. All the models used had ResNet-50 architecture with a sigmoid activation function used at the final linear layer. Each of these four pre-trained models was trained on random seeds and was ran for 50 epochs. We also use Stochastic Gradient Descent Optimization with a learning rate of 0.0001 and the weighted cross-entropy Loss to account for the severe class imbalance in the dataset.
%\[w(j)=n/Kn(j)\]
%where w(j) = weights of the classes, n = number of observations, K = Total number of classes, n(j) = Number of observations in each class.
\section{Results}
We evaluate the results over a fixed test set for the experiments. The DR grading performance is evaluated using Cohen’s quadratic weighted Kappa ($\kappa$) to measure the inter-rater agreement in ordinal multi-class problems \cite{warrens2013conditional}. More on this in Appendix \ref{section:appEVAL} 
We observe that all the models are quite similar in evaluation with Cohen Kappa Score as represented in Table \ref{tab:comb}. However, the uncertainty of these predictions is quite high. Also, most of these models are not well calibrated, leading to issues with the reliability models. Using our presented architecture by using TTAug \ref{subsection:tta} and Uncertainty Aware Ensemble \ref{subsection:Model Architecture}, we can reduce the uncertainty of our models by making the predictions much more reliable and well-calibrated, as shown in Table \ref{tab:comb}.
\section{Conclusion and Future Work}
Through the proposed model architecture, we aimed to make the model's predictions reliable and well-calibrated by making the model aware of uncertainty while ensembling predictions from multiple models and generalizing to distortions using TTAug \ref{subsection:tta}. We believe there is potential for this model architecture to be incorporated using Bayesian methods for various tasks involving classification and regression. We hope our work helps the community and inspires further research on uncertainty-aware learning of the model as well as making more reliable models for medical diagnosis.

\section*{Broader impact statement}
The approach introduced in this paper aims at tackling the issue of over-confident predictions during model outcomes for the diagnosis of Diabetic Retinopathy. Identifying Diabetic Retinopathy is one healthcare application where machine learning has already shown significant potential. However, its
applications in high-stakes healthcare choices must incorporate systematic uncertainty quantification and calibration for robust evaluation. By calibrating more generalized models, this framework also significantly aids in risk appraisal, lowering the possibility of potential misdiagnosis. We hope our work advances the current onset procedure for detecting Diabetic Retinopathy while also bringing trustworthy interpretations to other medical
imaging procedures.
\section*{Acknowledgement}
The authors would like to thank Mars Rover Manipal, an interdisciplinary student project team of MAHE, for providing the necessary resources for our research. We are also grateful to our faculty advisor, Dr Ujjwal Verma, for providing the necessary guidance.
\bibliography{references.bib}
\appendix

\section{Appendix} \label{apd:first}
\label{section:appA}

%\subsection{Introduction and Previous Works}
%\label{section:appINTRO}
%Over the past few years, the automatic classification of Diabetic Retinopathy using deep neural networks has been growing interest \cite{Gulshan2016DevelopmentAV}, \cite{Gargeya2017AutomatedIO}.
%Deep Neural Networks(DNNs) have achieved impressive results in the classification of Diabetic Retinopathy. However, the standard methods have been found to produce overconfident predictions, meaning that they are poorly calibrated. In classification tasks, a poorly calibrated network can place a high probability on one of the classes, even when the predicted class is incorrect \cite{Ayhan2020ExpertvalidatedEO}. However, \cite{chang2017active} uses uncertainty estimates and prefers to learn the data points predicted incorrectly with higher uncertainty in different mini-batches of SGD on which our model is based on preliminarily.  

\subsection{Experiments}
\subsubsection{Evaluation Metric}
\label{section:appEVAL}
The DR grading performance is evaluated using Cohen’s quadratic weighted Kappa ($\kappa$) to measure the inter-rater agreement between raters in ordinal multi-class problems \cite{warrens2013conditional}. This metric penalizes discrepancies between ratings, which depend quadratically on the distance between the prediction and the ground truth, as follows:
\[ QCK = 1 -  
 \left[  \frac{\sum_{i}^C\sum_{j}^C w_{i,j}O{i,j}} { \sum_{i}^C\sum_{j}^C w_{i,j}E{i,j} }  \right]
\]
where C is the number of classes, w is the weight matrix, O is the observed matrix, and E is the expected matrix.

\subsection{Uncertainty Metrics}
We use three metrics to quantify the uncertainty - Expected Calibration Error (ECE) \cite{nixon2019measuring}, Maximum Calibration Error (MCE) \cite{NEURIPS2019_1c336b80}, and Brier Score \cite{rufibach2010use}.

\subsubsection{Expected Calibration Error}
The Expected Calibration Error (ECE)\cite{nixon2019measuring} is a weighted average over the absolute confidence difference of the predictions of a model. It is defined as 
\begin{equation}
\label{sec:ece}
 ECE =  \sum_{m=1}^M \frac{|B_m|}{n} |acc(B_m)-conf(B_m)|
\end{equation}
where
\begin{equation}
\label{sec:ece1}
  acc(B_m) =  \frac{1}{|B_m|} \sum_{i \in B_m}  1(y_{i}=y_{t})
\end{equation}
\begin{equation}
\label{sec:ece2}
 conf(B_m) =  \frac{1}{|B_m|} \sum_{i \in B_m}  p_{i}
\end{equation}
where $conf(B_m)$ is just the average confidence/probability of predictions in that bin, and $acc(B_m)$ is the fraction of the correctly classified examples B$_m$.
\subsubsection{Maximum Calibration Error}
The Maximum Calibration Error (MCE) \cite{NEURIPS2019_1c336b80} focuses more on high-risk applications where the maximum confidence difference is more important than the average.
It is then defined as:
\begin{equation}
\label{sec:mce}
  MCE =  max_{m} |acc(B_m)-conf(B_m)|
\end{equation}
\subsubsection{Brier Score}

The Brier Score \cite{rufibach2010use} is a strictly proper score function or scoring rule that measures the accuracy of probabilistic predictions.
\begin{equation}
\label{sec:bs}
  Brier Score = \frac{1}{n} \sum_{t=1}^{n} (f_{t} - o_{t})^2 
\end{equation}
where $f_{t}$ is the probability that was forecast, o{t} is the actual outcome of the event at instance t (0 if it does not happen and 1 if it does happen), and N is the number of forecasting instances.



\end{document}

\paragraph{Paragraphs}


There is also a \verb+\paragraph+ command available, which sets the heading in
bold, flush left, and inline with the text, with the heading followed by 1\,em
of space.


\section{Citations, figures, tables, references}
\label{others}


These instructions apply to everyone.


\subsection{Citations within the text}


The \verb+natbib+ package will be loaded for you by default.  Citations may be
author/year or numeric, as long as you maintain internal consistency.  As to the
format of the references themselves, any style is acceptable as long as it is
used consistently.


The documentation for \verb+natbib+ may be found at
\begin{center}
  \url{http://mirrors.ctan.org/macros/latex/contrib/natbib/natnotes.pdf}
\end{center}
Of note is the command \verb+\citet+, which produces citations appropriate for
use in inline text.  For example,
\begin{verbatim}
   \citet{hasselmo} investigated\dots
\end{verbatim}
produces
\begin{quote}
  Hasselmo, et al.\ (1995) investigated\dots
\end{quote}


If you wish to load the \verb+natbib+ package with options, you may add the
following before loading the \verb+neurips_2022+ package:
\begin{verbatim}
   \PassOptionsToPackage{options}{natbib}
\end{verbatim}


If \verb+natbib+ clashes with another package you load, you can add the optional
argument \verb+nonatbib+ when loading the style file:
\begin{verbatim}
   \usepackage[nonatbib]{neurips_2022}
\end{verbatim}


As submission is double blind, refer to your own published work in the third
person. That is, use ``In the previous work of Jones et al.\ [4],'' not ``In our
previous work [4].'' If you cite your other papers that are not widely available
(e.g., a journal paper under review), use anonymous author names in the
citation, e.g., an author of the form ``A.\ Anonymous.''


\subsection{Footnotes}


Footnotes should be used sparingly.  If you do require a footnote, indicate
footnotes with a number\footnote{Sample of the first footnote.} in the
text. Place the footnotes at the bottom of the page on which they appear.
Precede the footnote with a horizontal rule of 2~inches (12~picas).


Note that footnotes are properly typeset \emph{after} punctuation
marks.\footnote{As in this example.}


\subsection{Figures}


\begin{figure}
  \centering
  \fbox{\rule[-.5cm]{0cm}{4cm} \rule[-.5cm]{4cm}{0cm}}
  \caption{Sample figure caption.}
\end{figure}


All artwork must be neat, clean, and legible. Lines should be dark enough for
purposes of reproduction. The figure number and caption always appear after the
figure. Place one line space before the figure caption and one line space after
the figure. The figure caption should be lower case (except for first word and
proper nouns); figures are numbered consecutively.


You may use color figures.  However, it is best for the figure captions and the
paper body to be legible if the paper is printed in either black/white or in
color.


\subsection{Tables}


All tables must be centered, neat, clean and legible.  The table number and
title always appear before the table.  See Table~\ref{sample-table}.


Place one line space before the table title, one line space after the
table title, and one line space after the table. The table title must
be lower case (except for first word and proper nouns); tables are
numbered consecutively.


Note that publication-quality tables \emph{do not contain vertical rules.} We
strongly suggest the use of the \verb+booktabs+ package, which allows for
typesetting high-quality, professional tables:
\begin{center}
  \url{https://www.ctan.org/pkg/booktabs}
\end{center}
This package was used to typeset Table~\ref{sample-table}.


\begin{table}
  \caption{Sample table title}
  \label{sample-table}
  \centering
  \begin{tabular}{lll}
    \toprule
    \multicolumn{2}{c}{Part}                   \\
    \cmidrule(r){1-2}
    Name     & Description     & Size ($\mu$m) \\
    \midrule
    Dendrite & Input terminal  & $\sim$100     \\
    Axon     & Output terminal & $\sim$10      \\
    Soma     & Cell body       & up to $10^6$  \\
    \bottomrule
  \end{tabular}
\end{table}


\section{Final instructions}


Do not change any aspects of the formatting parameters in the style files.  In
particular, do not modify the width or length of the rectangle the text should
fit into, and do not change font sizes (except perhaps in the
\textbf{References} section; see below). Please note that pages should be
numbered.


\section{Preparing PDF files}


Please prepare submission files with paper size ``US Letter,'' and not, for
example, ``A4.''


Fonts were the main cause of problems in the past years. Your PDF file must only
contain Type 1 or Embedded TrueType fonts. Here are a few instructions to
achieve this.


\begin{itemize}


\item You should directly generate PDF files using \verb+pdflatex+.


\item You can check which fonts a PDF files uses.  In Acrobat Reader, select the
  menu Files$>$Document Properties$>$Fonts and select Show All Fonts. You can
  also use the program \verb+pdffonts+ which comes with \verb+xpdf+ and is
  available out-of-the-box on most Linux machines.


\item The IEEE has recommendations for generating PDF files whose fonts are also
  acceptable for NeurIPS. Please see
  \url{http://www.emfield.org/icuwb2010/downloads/IEEE-PDF-SpecV32.pdf}


\item \verb+xfig+ "patterned" shapes are implemented with bitmap fonts.  Use
  "solid" shapes instead.


\item The \verb+\bbold+ package almost always uses bitmap fonts.  You should use
  the equivalent AMS Fonts:
\begin{verbatim}
   \usepackage{amsfonts}
\end{verbatim}
followed by, e.g., \verb+\mathbb{R}+, \verb+\mathbb{N}+, or \verb+\mathbb{C}+
for $\mathbb{R}$, $\mathbb{N}$ or $\mathbb{C}$.  You can also use the following
workaround for reals, natural and complex:
\begin{verbatim}
   \newcommand{\RR}{I\!\!R} %real numbers
   \newcommand{\Nat}{I\!\!N} %natural numbers
   \newcommand{\CC}{I\!\!\!\!C} %complex numbers
\end{verbatim}
Note that \verb+amsfonts+ is automatically loaded by the \verb+amssymb+ package.


\end{itemize}


If your file contains type 3 fonts or non embedded TrueType fonts, we will ask
you to fix it.


\subsection{Margins in \LaTeX{}}


Most of the margin problems come from figures positioned by hand using
\verb+\special+ or other commands. We suggest using the command
\verb+\includegraphics+ from the \verb+graphicx+ package. Always specify the
figure width as a multiple of the line width as in the example below:
\begin{verbatim}
   \usepackage[pdftex]{graphicx} ...
   \includegraphics[width=0.8\linewidth]{myfile.pdf}
\end{verbatim}
See Section 4.4 in the graphics bundle documentation
(\url{http://mirrors.ctan.org/macros/latex/required/graphics/grfguide.pdf})


A number of width problems arise when \LaTeX{} cannot properly hyphenate a
line. Please give LaTeX hyphenation hints using the \verb+\-+ command when
necessary.


\begin{ack}
Use unnumbered first level headings for the acknowledgments. All acknowledgments
go at the end of the paper before the list of references. Moreover, you are required to declare
funding (financial activities supporting the submitted work) and competing interests (related financial activities outside the submitted work).
More information about this disclosure can be found at: \url{https://neurips.cc/Conferences/2022/PaperInformation/FundingDisclosure}.


Do {\bf not} include this section in the anonymized submission, only in the final paper. You can use the \texttt{ack} environment provided in the style file to autmoatically hide this section in the anonymized submission.
\end{ack}


\section*{References}


References follow the acknowledgments. Use unnumbered first-level heading for
the references. Any choice of citation style is acceptable as long as you are
consistent. It is permissible to reduce the font size to \verb+small+ (9 point)
when listing the references.
Note that the Reference section does not count towards the page limit.
\medskip


{
\small


[1] Alexander, J.A.\ \& Mozer, M.C.\ (1995) Template-based algorithms for
connectionist rule extraction. In G.\ Tesauro, D.S.\ Touretzky and T.K.\ Leen
(eds.), {\it Advances in Neural Information Processing Systems 7},
pp.\ 609--616. Cambridge, MA: MIT Press.


[2] Bower, J.M.\ \& Beeman, D.\ (1995) {\it The Book of GENESIS: Exploring
  Realistic Neural Models with the GEneral NEural SImulation System.}  New York:
TELOS/Springer--Verlag.


[3] Hasselmo, M.E., Schnell, E.\ \& Barkai, E.\ (1995) Dynamics of learning and
recall at excitatory recurrent synapses and cholinergic modulation in rat
hippocampal region CA3. {\it Journal of Neuroscience} {\bf 15}(7):5249-5262.
}


%%%%%%%%%%%%%%%%%%%%%%%%%%%%%%%%%%%%%%%%%%%%%%%%%%%%%%%%%%%%
\section*{Checklist}


%%% BEGIN INSTRUCTIONS %%%
The checklist follows the references.  Please
read the checklist guidelines carefully for information on how to answer these
questions.  For each question, change the default \answerTODO{} to \answerYes{},
\answerNo{}, or \answerNA{}.  You are strongly encouraged to include a {\bf
justification to your answer}, either by referencing the appropriate section of
your paper or providing a brief inline description.  For example:
\begin{itemize}
  \item Did you include the license to the code and datasets? \answerYes{See Section~\ref{gen_inst}.}
  \item Did you include the license to the code and datasets? \answerNo{The code and the data are proprietary.}
  \item Did you include the license to the code and datasets? \answerNA{}
\end{itemize}
Please do not modify the questions and only use the provided macros for your
answers.  Note that the Checklist section does not count towards the page
limit.  In your paper, please delete this instructions block and only keep the
Checklist section heading above along with the questions/answers below.
%%% END INSTRUCTIONS %%%


\begin{enumerate}


\item For all authors...
\begin{enumerate}
  \item Do the main claims made in the abstract and introduction accurately reflect the paper's contributions and scope?
    \answerTODO{}
  \item Did you describe the limitations of your work?
    \answerTODO{}
  \item Did you discuss any potential negative societal impacts of your work?
    \answerTODO{}
  \item Have you read the ethics review guidelines and ensured that your paper conforms to them?
    \answerTODO{}
\end{enumerate}


\item If you are including theoretical results...
\begin{enumerate}
  \item Did you state the full set of assumptions of all theoretical results?
    \answerTODO{}
        \item Did you include complete proofs of all theoretical results?
    \answerTODO{}
\end{enumerate}


\item If you ran experiments...
\begin{enumerate}
  \item Did you include the code, data, and instructions needed to reproduce the main experimental results (either in the supplemental material or as a URL)?
    \answerTODO{}
  \item Did you specify all the training details (e.g., data splits, hyperparameters, how they were chosen)?
    \answerTODO{}
        \item Did you report error bars (e.g., with respect to the random seed after running experiments multiple times)?
    \answerTODO{}
        \item Did you include the total amount of compute and the type of resources used (e.g., type of GPUs, internal cluster, or cloud provider)?
    \answerTODO{}
\end{enumerate}


\item If you are using existing assets (e.g., code, data, models) or curating/releasing new assets...
\begin{enumerate}
  \item If your work uses existing assets, did you cite the creators?
    \answerTODO{}
  \item Did you mention the license of the assets?
    \answerTODO{}
  \item Did you include any new assets either in the supplemental material or as a URL?
    \answerTODO{}
  \item Did you discuss whether and how consent was obtained from people whose data you're using/curating?
    \answerTODO{}
  \item Did you discuss whether the data you are using/curating contains personally identifiable information or offensive content?
    \answerTODO{}
\end{enumerate}


\item If you used crowdsourcing or conducted research with human subjects...
\begin{enumerate}
  \item Did you include the full text of instructions given to participants and screenshots, if applicable?
    \answerTODO{}
  \item Did you describe any potential participant risks, with links to Institutional Review Board (IRB) approvals, if applicable?
    \answerTODO{}
  \item Did you include the estimated hourly wage paid to participants and the total amount spent on participant compensation?
    \answerTODO{}
\end{enumerate}


\end{enumerate}


%%%%%%%%%%%%%%%%%%%%%%%%%%%%%%%%%%%%%%%%%%%%%%%%%%%%%%%%%%%%


\appendix


\section{Appendix}


Optionally include extra information (complete proofs, additional experiments and plots) in the appendix.
This section will often be part of the supplemental material.


\end{document}
