\section{Lower Bounds}

We first restate a lower bound by \citet{BanerjeeGGJ:SODA:2022} under our model.

\begin{lemma}[\citet{BanerjeeGGJ:SODA:2022}, Theorem 3]
    \label{lem:lower-bound-BanerjeeGGJ}
    No online algorithm can achieve a competitive ratio better than $\log^{1-o(1)} N$,%
    \footnote{The original theorem by \citet{BanerjeeGGJ:SODA:2022} only claimed a weaker bound of $\log^{1-\varepsilon} N$ but their proof actually showed a slightly stronger bound that we restate here.}
    even for $1$-balanced instance.
\end{lemma}

%\begin{theorem}[\citet{BanerjeeGGJ:SODA:2022}, Theorem 4]
%    For the instance given by Table~\ref{tab:lower-bound}, no algorithm can be better than $O(N)$-competitive.
    %In the absence of any additional information regarding the agents' values, there is no online algorithm that achieves a $\frac{e}{N}+\delta$ or $\frac{e}{T}+\delta$ competitive ratio with respect to the NSW for any constant $\delta>0$.
%\end{theorem}

Next we prove that the logarithmic dependence in the balanced ratio or the impartiality ratio is necessary.
The main ingredient is the instance given by Table~\ref{tab:lower-bound}, which is a variant of another hard instance by \citet{BanerjeeGGJ:SODA:2022} (c.f., Theorem 4 therein).
The instance has $N$ agents, and $T = N$ items of unit supplies.
Agent $i$'s value for item $t$ is $N^{2i}$ if $t \ge i$, and is $0$ otherwise.
Recall that for lower bounds it suffices to consider deterministic algorithms.
Hence, we may assume without lost of generality that agent $i$ receives the least amount of item $t = i$ among agents $i$ to $N$ in the algorithm's allocation.
%When the instance is rather imbalanced (impartial), i.e. with large balance ratio (impartiality ratio), we show that $O(\log \lambda)$ ($O(\log \mu)$) competitive ratio is necessary.

\begin{table}[t]
    \caption{Illustration of a hard instance. Rows are items and columns are agents. The number in the intersection of the $t$-th row and the $i$-th column is agent $i$'s value for item $t$.}
    \label{tab:lower-bound}
    \centering
    \begin{tabular}{cccccc}
        \hline
        $t$ & Agent 1& Agent 2 & Agent 3& $\cdots$ &Agent $n$ \\
        \hline
        1 & $n^2$ & $n^2$ & $n^2$ & & $n^2$ \\
        2 & 0 & $n^4$ & $n^4$ & & $n^4$ \\
        3 & 0 & 0 & $n^6$ & & $n^6$ \\
        \multicolumn{6}{c}{$\cdots$} \\
        $n$ & 0 & 0 & 0 & & $n^{2n}$\\
        \hline
    \end{tabular}
\end{table}

\begin{lemma}
    \label{lem:lower-bound-competitive-ratio}
    No algorithm can be better than $\frac{n-1}{e}$ competitive for the instance given by Table~\ref{tab:lower-bound}.
\end{lemma}

\begin{proof}
    %Upon arrival of item $t$, agents $t$ to $N$ have value $v_t$ while the other have value 0. 
    %After the allocation, there always exists an agent with $v_{it}=v_t$ and $x_{it} \leq \frac{1}{N-t+1}$ by pigeon hole principle. Without loss of generality let this agent be $t$. This agent will have value 0 on all following items.
    For any $1 \le i \le n$, by the assumption that agent $i$ receives the least amount of item $t = i$ among agents $i$ to $n$, we get that:
    %
    \[
        x_{ii} \le \frac{1}{n-i+1}
        ~.
    \]

    Therefore, agent $i$'s utility is at most
    %
    \[
        u_i \leq \sum_{t=1}^{i-1} n^{2t} + \frac{n^{2i}}{n-i+1} \leq \frac{n^{2i}}{n-i+1} \cdot \frac{n}{n-1}.
    \]

    On the other hand, if we allocate each item $1 \le t \le n$ to agent $i = t$ (which is the optimal allocation), agent $i$'s utility would be:
    %
    \[
        u^*_i = n^{2i}
        ~.
    \]

    %We further show that in the optimal solution $x_{tt}^*=1$ for every $1\leq t\leq N$ by an induction on $t$ in decreasing order, therefore $u^*_t=N^{2t}$.
    
    %The base case is $t=N$. $x_{NN}^*=1$ since only agent $N$ has positive value on item $N$.
    
    %Next for some $t<N$ by induction hypothesis $x_{ii}^*=1$ holds for any $i>t$. If $x_{it}>0$ for any $i>t$, consider reallocating an $\varepsilon$ amount of item $t$ from agent $i$ to agent $t$.
    %We claim that the Nash welfare increases for a sufficiently small $\varepsilon$. Since the other agents' utilities stay the same, it suffices to prove that:
    %\begin{equation*}
    %    (u_t^* - \varepsilon v_t) (u_i^* + \varepsilon v_t) - u_t^* u_i^* = \varepsilon v_t (u_t^* - u_i^*) - \varepsilon ^2 v_t^2
    %\end{equation*}
    %is positive.
    Therefore, the competitive ratio of the algorithm is at best:
    %
    \[
        \left( \prod_{i=1}^n \frac{u^*_i}{u_i} \right)^{\frac{1}{n}} 
        \ge
        \left( \prod_{i=1}^n \frac{n^{2i}}{\frac{n^{2i}}{n-i+1} \cdot \frac{n}{n-1}} \right)^{\frac{1}{n}}
        =
        \frac{n-1}{n} \big( n! \big)^{\frac{1}{n}} > \frac{n-1}{e}
        ~.
    \]
    %Since $u_t^* \leq 1+N^2+\cdots+N^{2t} < N^{2t+2} \leq u_i^*$, the above is positive for a sufficiently small $\varepsilon$. Therefore $x_{ii} = 0$ for any $i>t$. Moreover, $v_{it}=0$ for any agent $i<t$, thus $x_{tt} = 1$.
    
    %The competitive ratio is:
    %\begin{equation*}
    %    \Big(\prod_{i=1}^N \frac{u^*_i}{u_i}\Big)^{1/N} \geq \left( \frac{N^2 \cdot N^4 \cdot \cdots \cdot N^{2N}}{\frac{N^2}{N} \cdot \frac{N^4}{N-1} \cdot \cdots \cdot N^{2N}} \right)^{1/N} \cdot \frac{N-1}{N} = (N-1) \cdot \frac{(N!)^{1/N}}{N} \geq \frac{N-1}{e}.
    %\end{equation*}
\end{proof}

\begin{lemma}
    \label{lem:lower-bound-balance-impartiality-ratios}
    The imbalance ratio and impartiality ratio of the instance given by Table~\ref{tab:lower-bound} are at most $n^{2n}$.
\end{lemma}

We can now derive our lower bounds as corollaries of the above lemmas.

\begin{theorem}
    \label{thm:lower-bound-balance}
No algorithm can be better than $\min \big\{ \log^{1-o(1)} N \lambda^* ~,~ \frac{N-1}{e} \big\}$-competitive.
\end{theorem}

\begin{proof}
    If $N \ge \lambda^*$, the stated lower bound is $\log^{1-o(1)} N$ which follows by Lemma~\ref{lem:lower-bound-BanerjeeGGJ}.

    If $N < \lambda^*$ but $N^{2N} \ge \lambda^*$, the stated lower bound is $\log^{1-o(1)} \lambda^*$. 
    Consider the instance given by Table~\ref{tab:lower-bound} choosing $n = \Theta\big(\frac{\log \lambda^*}{\log\log \lambda^*}\big)$ such that $\lambda^* = n^{2n}$.
    Note that this implies $N \ge n$.
    Create $\frac{N}{n}$ copies of this instance so that the number of agents becomes $N$.
    Then, the stated lower bound follows by Lemmas~\ref{lem:lower-bound-competitive-ratio} and \ref{lem:lower-bound-balance-impartiality-ratios}.

    Finally if $N^{2N} < \lambda^*$.
    The stated bound is $\frac{N-1}{e}$, which follows by Lemmas~\ref{lem:lower-bound-competitive-ratio} and \ref{lem:lower-bound-balance-impartiality-ratios}, choosing $n = N$ in the instance given by Table~\ref{tab:lower-bound}.
\end{proof}

\begin{theorem}
    \label{thm:lower-bound-impartial}
No algorithm can be better than $\min \big\{ \log^{1-o(1)} \mu^* ~,~ \frac{N-1}{e} \big\}$-competitive.
\end{theorem}

\begin{proof}
    If $N^{2N} \ge \mu^*$, the stated lower bound is $\log^{1-o(1)} \mu^*$. 
    Consider the instance given by Table~\ref{tab:lower-bound} choosing $n = \Theta\big(\frac{\log \mu^*}{\log\log \mu^*}\big)$ such that $\mu^* = n^{2n}$.
    Note that this implies $N \ge n$.
    Create $\frac{N}{n}$ copies of this instance so that the number of agents becomes $N$.
    Then, the stated lower bound follows by Lemmas~\ref{lem:lower-bound-competitive-ratio} and \ref{lem:lower-bound-balance-impartiality-ratios}.

    Finally if $N^{2N} < \mu^*$.
    The stated lower bound is $\frac{N-1}{e}$, which follows by Lemmas~\ref{lem:lower-bound-competitive-ratio} and \ref{lem:lower-bound-balance-impartiality-ratios}, choosing $n = N$ in the instance given by Table~\ref{tab:lower-bound}.
\end{proof}

We remark that Theorem~\ref{thm:lower-bound-impartial} still holds even if the agents' values for the items are either $0$ or $1$.
For each item $t$ in Table~\ref{tab:lower-bound}, instead of letting it have unit supply and letting agents $t$ to $n$ have values $n^{2t}$ for it, we can instead let it have supply $n^{2t}$ and let agents $t$ to $n$ have values $1$ for it.
Scaling the allocation of item $t$ in the original instance by $n^{2t}$ factor would give an allocation for the $0$-$1$ value version, with the same utilities for all agents.
