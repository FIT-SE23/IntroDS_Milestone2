\section{Introduction}
\label{sec:introduction}

Suppose that you run a daycare center.
From time to time, some organizations donate candies for children who have different preferences over various kinds of candies:
Alice likes chocolates, Bob prefers gummy bears, Charlie favors lollipops, and so forth.
The candies come in abundance, so you may consider it as a resource allocation problem of \emph{divisible} items.
How would you distribute the candies to the children?

Naturally, we would like to allocate the candies based on the children's values for them.
%, e.g., their maximum willingness-to-pay for each kind of candies.
We will assume that the children have \emph{additive utilities} for receiving bundles of candies.
Following this idea, it may be tempting to allocate the candies to the children in a way that maximizes the sum of children's utilities for the candies that they received, a.k.a., the social welfare.
The example below, however, highlights the potential unfairness of this approach.

\begin{example}
    Consider distributing two packs of chocolates and two packs of gummy bears to two children Alice and Bob.
    Alice has values $100$ and $15$ for receiving a pack of chocolates and gummy bears respectively.
    Bob, on the other hand, has values $1$ and $10$ for chocolates and gummy bears respectively.
\end{example}

The social welfare maximizing allocation gives everything to Alice for her higher values, and leaves nothing for Bob.
This is blatantly unfair, especially at a daycare center.

What if we distribute evenly and let each child have a pack of chocolates and a pack of gummy bears?
While this is fair, it is inefficient.
The gummy bears contribute little to Alice's utility, while the chocolates do not add much to Bob's utility.

The question then becomes how to allocate the candies if we want to strike a balance between efficiency and fairness.
In the above example, allocating both packs of chocolates to Alice and both packs of gummy bears to Bob is a decent option.
In general, maximizing the \emph{Nash welfare}, i.e., the geometric mean of the children's utilities, is a good proxy for balancing efficiency and fairness.
Indeed, the allocation that maximizes Nash welfare also satisfies several standard notions of fairness, such as envy-freeness~\cite{CaragiannisKMPSW:2019} and proportionality~\cite{Vazirani:AGT:2007}.
It also coincides with the aforementioned decent allocation in the above example.
Further, the Nash welfare maximization problem is \emph{scale invariant}:
if we increase an agent's values for all items by the same multiplicative factor, the Nash welfare maximizing allocation would remain the same.
Moreover, maximizing the Nash welfare with divisible items and additive utilities reduces to solving the Eisenberg-Gale convex program~\cite{EisenbergG:AnnaMathStat:1959}.
We can therefore compute this allocation in polynomial time.

There is one more complication though.
We cannot predict when the donations of candies will be made, what kinds of candies will be donated, and in what quantities.
When some organizations make a donation, we need to distribute the candies to the children immediately without much information about future donations.

Besides the toy example above, many real online resource allocation problems for heterogeneous agents require balance between fairness and efficiency, e.g., allocating food each day to people in need~\cite{prendergast2017food}, or allocating shared computational resources to users~\cite{hao2016online}. 
Hence, our task becomes designing online allocation algorithms for Nash welfare maximization,
whose performance is evaluated by \emph{competitive ratio}, that is, the worst ratio between the online algorithm's solution and the offline optimal solution.
From now on, we will more generally talk about items and agents.


This problem was first studied by \citet*{BanerjeeGGJ:SODA:2022}.
They noticed that under worst-case analysis, no algorithm can have a competitive ratio better than the trivial $O(N)$, where $N$ is the number of agents.
They then turned to the model of online algorithms with predictions, in which the online algorithm is further given some predictions of each agent's utility for receiving the set of all items, which they called the \emph{monopolist utility}.
If the predictions were accurate, their algorithm would be $O(\log N)$-competitive.%
\footnote{\citet{BanerjeeGGJ:SODA:2022} also gave an $O(\log T)$-competitive ratio where $T$ is the number of items, under the assumption that items have unit supplies. Since this paper considers arbitrary supplies, the dependence in $T$ is no longer valid.}
They further proved that the competitive ratio is asymptotically optimal for the problem.

\subsection{Our Contribution}

Coming up with accurate predictions is not an easy task at all.
Is it truly hopeless to design good online algorithms \emph{without predictions}?
The hard instance which \citet{BanerjeeGGJ:SODA:2022} designed to rule out the existence of good online algorithms without predictions requires the agents' values to differ by an exponential factor, a rare phenomenon in real resource allocation problems.
Can we design online algorithms for ``natural instances'' where agents' values are not as extreme?

We introduce the notions of \emph{balanced} and \emph{impartial} instances.
An instance is $\lambda$-\emph{balanced} if the agents' monopolist utilities differ by at most a multiplicative factor $\lambda$.
This is closely related to the model of online algorithms with predictions by \citet{BanerjeeGGJ:SODA:2022}.
If we had accurate predictions of the agents' monopolist utilities, we could then normalize the agents' values accordingly to obtain a $1$-balanced instance using the fact that Nash welfare maximization is scale invariant.
If we had $(\alpha, \beta)$-approximate predictions, i.e., if they were at worst an $\alpha$-factor larger or a $\beta$-factor smaller than the agents' monopolist utilities, the normalized instance would be $\alpha \beta$-balanced.
Therefore, any competitive online algorithm for balanced instances further implies a competitive online algorithm with predictions.

Our main result for $\lambda$-balanced instances is a $\log^{1+o(1)} (\lambda N)$-competitive online algorithm.
We further show that this competitive ratio is nearly optimal in the sense that no online algorithm can be better than $\log^{1-o(1)} (\lambda N)$ competitive.
By the aforementioned reduction, this also implies a $\log^{1+o(1)} (\alpha \beta N)$-competitive online algorithm for the model with $(\alpha, \beta)$-approximate predictions, which improves the $O(\alpha \log \beta N)$-competitive online algorithm given by \citet{BanerjeeGGJ:SODA:2022} when $\alpha$ is sufficiently large.

On the other hand, we say that an instance is \emph{$\mu$-impartial} if the agents' utilities for the Nash welfare maximizing allocation differ by at most a multiplicative factor $\mu$.
For $\mu$-impartial instances, our main result is a $\log^{2+o(1)} \mu$-competitive online algorithm.
We remark that this competitive ratio is independent of the number of agents or items, and thus this would be an $O(1)$-competitive algorithm if the instance is $O(1)$-impartial.
In the special case when the agents' values for different items are binary, we have an improved competitive ratio $\log^{1+o(1)} \mu$;
we further prove that it is nearly optimal.

See Table~\ref{tab:summary} for a summary of our results.


\begin{table}[t]
    \centering

    \caption{Summary of results. Here $\lambda^*$ and $\mu^*$ are the smallest numbers for which the instance is $\lambda^*$-balanced and $\mu^*$-impartial. We omit the constants and $\log\log$ factors for brevity. We also omit the trivial bound of $N$.}
    \label{tab:summary}
    
    \begin{tabular}{p{3cm}p{6.5cm}p{3cm}}
        \toprule
         & Algorithm & Lower Bound \\
        \midrule
        $\lambda^*$-Balanced & $\log \lambda^* N$ & $\log \lambda^* N$ \\
        $\mu^*$-Impartial & $\min \big\{ \log \mu^* N \,,~ \log^2 \mu^* \big\}$ & $\log \mu^*$ \\
        $(\alpha, \beta)$-Predictions & $\alpha  \log \beta N$ \cite{BanerjeeGGJ:SODA:2022} $\to~ \log \alpha \beta N$ &  \\
        \bottomrule
    \end{tabular}
    
\end{table}


\subsection{Other Related Works}

There is a vast literature on online algorithms for resource allocation problems, such as online packing (c.f., \citet{AlonAABN:TALG:2006} and \citet{BuchbinderN:MOR:2009}) and online matching problems (c.f., \citet*{KarpVV:STOC:1990} and \citet{Mehta:FTTCS:2013}).
Most related to this paper is the work by \citet{DevanurJ:STOC:2012} on online matching with concave returns.
Maximizing the Nash welfare can be equivalently formulated as maximizing the sum of logarithms of the agents' utilities, a special case of the model of \citet{DevanurJ:STOC:2012} when the return functions are the logarithm.
However, their results do not imply competitive online algorithms for our problem because they studied the multiplicative competitive ratio with respect to (w.r.t.) the sum of log-utilities.
By contrast, \citet{BanerjeeGGJ:SODA:2022} and this paper consider the multiplicative competitive ratio w.r.t.\ the geometric mean of utilities, which corresponds to the additive approximation factor w.r.t.\ the sum of log-utilities.
\citet{BarmanKM:AAAI:2022} studied online algorithms for the more general $p$-mean welfare maximization which captures the Nash welfare as a special case when $p = 0$.
Their result for Nash welfare can be interpreted as an $O(\log^3 N)$-competitive algorithm for $1$-balanced instances;
by contrast our algorithm is $O(\log^{1+o(1)} (\lambda N)$-competitive for general $\lambda$-balanced instances. 

Nash welfare was introduced about 70 years ago \cite{Nash:1950, KanekoM:1979}. Its maximization has been extensively studied in the offline setting.
The problem of divisible items and additive utilities is equivalent to the Eisenberg-Gale convex program~\cite{EisenbergG:AnnaMathStat:1959}.
Much effort has been devoted to the problem of indivisible items and various kinds of utilities.
Even for additive utilities, the problem is NP-hard~\cite{Moulin:2004} and APX-hard~\cite{Lee:IPL:2017}.
\citet{ColeG:STOC:2015} gave the first constant-approximation algorithm for additive utilities.
Subsequently, \citet{AnariOSS:ITCS:2017} applied the theory of stable polynomials to get an $e$-approximation algorithm.
\citet{ColeDGJMVY:EC:2017} further improves the analysis to show a $2$-approximation.
Finally, \citet*{BarmanKV:EC:2018} gave a new algorithm that achieves the state-of-the-art $e^{1/e} < 1.45$-approximation.
The problem with more general utilities has also been studied. 
\citet*{GargHM:SODA:2018} designed a constant-approximation algorithm for budget-additive utilities.
\citet{AnariMOV:SODA:2018} introduced a constant-approximation algorithm for separable, piecewise-linear concave utilities.
For submodular utilities, \citet*{GargKK:SODA:2020} proposed an $O(n \log n)$-approximation algorithm, and \citet{LiV:FOCS:2022} recently obtained the first constant-approximation algorithm for submodular utilities.
Last but not least, \citet{BarmanBKS:ESA:2020} and \citet*{ChaudhuryGM:AAAI:2021} independently developed $O(n)$-approximation algorithms for the even more general subadditive utilities.

%\todo{Some works on online fair division}
\citet*{AzarBJ:ESA:2010} studied an online resource allocation problem whose competitive algorithms are also competitive w.r.t.\ Nash welfare. 
However, they assumed that any agent's values for different items could only differ by a bounded ratio;
their competitive ratio is logarithmic in the numbers of agents and items as well as this bounded ratio.
By contrast, our notions of balance and impartiality compare different agents' utilities.
Their assumption rules out the possibility that an agent may have zero value for some item. Therefore, their results do not apply to our model. 
Besides that, there is a significant line of work about online resource allocation, which focuses on different objectives, e.g., \citet{gkatzelis2021fair} considered maximizing utilitarian social welfare with envy-freeness in online resource allocation. \citet{banerjee2022proportionally} studied an online public items allocation problem with predictions, where the objective is to achieve proportional fairness.

%study online allocation with a quality measure to balance each agents' total utility, and their results imply an algorithm approximating maximum Nash welfare with competitive ratio logarithmic in number of agents, items and the maximum-minimum value ratio of an agent. While 

%\citet{CaragiannisKMPSW:2019} show that the Nash welfare-maximizing solution on indivisible items is envy-free up to one item, and a good approximation to the maximin share guarantee, indicating the importance of studying Nash welfare.

