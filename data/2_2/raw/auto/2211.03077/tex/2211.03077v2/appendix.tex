\section{Missing Proofs from Section~\ref{sec:prelim}}
\label{app:prelim}

\subsection{Proof of Lemma~\ref{lem:nsw-proportionality}}

Suppose for contrary that there is an agent $i$ such that $u_i^* < \frac{1}{N} \sum_{t=1}^{T} s_t v_{it}$.
We argue that it would be beneficial to allocate more to agent $i$.
Concretely, consider an alternative allocation $\vec{x'} = (x'_{it})_{1 \le i \le N, 1 \le t \le T}$ defined as:
%
\[
    x'_{jt} =
    \begin{cases}
        \displaystyle \frac{x_{jt}^*}{1 + \varepsilon} & \mbox{, if $j \ne i$;} \\[2.5ex]
        \displaystyle \frac{x_{it}^* + \varepsilon s_t}{1 + \varepsilon} & \mbox{, if $j = i$.}
    \end{cases}
\]

This is a feasible allocation because for any item $1 \le t \le T$:
%
\[
    \sum_{j=1}^N x'_{jt} = \frac{1}{1+\varepsilon} \left( \sum_{j \ne i}^N x_{jt} + \big( x_{it} + s_t \varepsilon \big) \right)
    \le \frac{1}{1+\varepsilon} \left( s_t + s_t \varepsilon \right) = s_t
    ~.
\]

Further comparing the agents' utilities to the counterparts in the assumed optimal allocation, we have:
%
\[
    u'_i = \sum_{t=1}^T x'_{it} v_{it} = \frac{1}{1+\varepsilon} \left( \sum_{t=1}^T x^*_{it} v_{it} + \varepsilon \sum_{t=1}^T s_t v_{it} \right) = \frac{1}{1+\varepsilon} \left( u^*_i + \varepsilon \sum_{t=1}^T s_t v_{it} \right)
    ~,
\]
%
and for any agent $j \ne i$ simply $u'_j = \frac{1}{1+\varepsilon} u^*_j$.
Hence, we get a contradiction that:
%
\[
    \frac{\left(\prod_{i = 1}^{N} u'_i\right)^{1/N}}{\left(\prod_{i = 1}^{N} u_i^*\right)^{1/N}} ~ = ~ \frac{\left( 1 + \varepsilon \frac{\sum_{t=1}^{T} s_t v_{it}}{u_i^*}\right)^{1/N}}{1 + \varepsilon} > 1
\]
%
for a sufficiently small $\varepsilon$ by the assumption of $u_i^* < \frac{1}{N} \sum_{t=1}^{T} s_t v_{it}$.

\subsection{Proof of Lemma~\ref{lem:maximum-mu-value}}

Suppose for contradiction that there are agent $i$ and item $t$ for which we have $x_{it}^*>0$ but $v_{it}<\frac{1}{\mu^{*}} \max_{1 \le j \le N}v_{jt}$.
We argue that we should have allocated less item $t$ to agent $i$.
Let $j$ be an agent with highest value $v_{jt}$ for item $t$.
Consider reallocating an $\varepsilon$ amount of item $t$ from agent $i$ to agent $j$.
We claim that the Nash welfare increases for a sufficiently small $\varepsilon$.
Since the other agents' utilities stay the same, it suffices to prove that:
\begin{equation*}
    (u_i^* - \varepsilon v_{it}) (u_j^* + \varepsilon v_{jt}) - u_i^* u_j^* = \varepsilon (u_i^* v_{jt} - u_j^* v_{it}) - \varepsilon ^2 v_{it} v_{jt}
\end{equation*}
%
is positive.
Since $v_{it} < \frac{1}{\mu^*} v_{jt}$ by our assumption for contradiction and $u_j^* \le \mu^* u_i^*$ by the definition of $\mu^*$, we get that $u_i^* v_{jt} - u_j^* v_{it} > 0$.
Hence, the above is positive for a sufficiently small $\varepsilon$.


\section{Missing Proofs from Section~\ref{sec:impartial}}
\label{app:impartial}

\subsection{Proof of Lemma~\ref{lem:binary-utility}}

For ease of exposition, we rewrite Equation~\eqref{eqn:binary-inquality} as:
\begin{equation}
    \label{eqn:step-function}
    \left\{
    \begin{aligned}
        &u_1 \cdot N& \geq~ &\tilde{u}_1\\
        &u_1 \cdot N + (u_2 - u_1) \cdot (N - 1) &\geq~ &\tilde{u}_1+\tilde{u}_2\\
        &...\\
        &u_1 \cdot N + \sum_{i = 1}^{N-1} ((u_{i+1} - u_{i}) \cdot (N - i) ) &\geq &\sum_{i}^{N} \tilde{u}_i
    \end{aligned}
    \right.
\end{equation}
Along with the restrictions $u_1\leq u_2\leq \cdots \le u_N$, the feasible region of $(u_i)_{i\in[N]}$ is a polytope. Since $\sum_{i=1}^{N} \log u_i$ is concave, it reaches minimum at a vertex of the polytope, i.e., where exactly $n$ of the inequalities are equalities. In the minimum solution $(\bar{u}_i)_{i\in[N]}$, suppose there exists $i$ such that $\bar{u}_i=\bar{u}_{i+1}$, let $i^*$ be smallest such $i$, the $i^*$-th inequality in Equation \eqref{eqn:step-function} is not tight.

Define $(u'_1,u'_2,\cdots,u'_N)$ as
\begin{equation}
    \left\{
    \begin{aligned}
        &u'_i = \bar{u}_i & \qquad &\forall i< i^*\\
        &u'_i = \bar{u}_i-\varepsilon & &i=i^*\\
        &u'_i = \bar{u}_i+\frac{\varepsilon}{N-i} & &\forall i>i^*
    \end{aligned}
    \right.
\end{equation}
for sufficiently small $\varepsilon>0$. $(u'_1,u'_2,\cdots,u'_N)$ satisfies Equation \eqref{eqn:step-function}, and $u'_1u'_2\cdots u'_N<\bar{u}_1\bar{u}_2\cdots\bar{u}_N$ holds since $\bar{u}_i$ is the smallest among $\bar{u}_i,\bar{u}_{i+1},\cdots,\bar{u}_N$, contradicting the minimality of $(\bar{u}_i)_{i\in[N]}$.

Therefore, in the minimum solution $u_i=\sum_{i' = 1}^{i}\frac{\tilde{u}_i}{N+1-i'}$.


