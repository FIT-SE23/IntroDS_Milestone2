Visual analysis is central to gaining insight on the underlying behaviour of unsteady flow data. Numerous visualization techniques have been developed to extract meaningful information from flow data, all in support of analyzing a variety of flow features. Among these techniques, notable ones include the finite-time Lyapunov exponents (FTLE)~\cite{haller2000finding}, used to understand the rate of separation between nearby particles integrated over a finite time interval, and its resulting Lagrangian coherent structures (LCS)~\cite{haller2000lagrangian, haller2000finding}, extracted as the ridges of the FTLE field. Streaklines are another visualization technique widely used by researchers, used to factor out background motion in flows, and identify underlying vortices that might be present. Other flow visualization techniques include line integral convolution (LIC)~\cite{cabral1993imaging} almost invariant sets (AIS)~\cite{froyland2009almost}, finite-size Lyapunov exponent (FSLE)~\cite{artale1997dispersion, aurell1997predictability} and the coherent ergodic partitions~\cite{you2014eulerian}.

A core component common to all the above techniques is the computation of the flow map. The flow map provides the position of a particle advected under a flow over a finite time span, and typically, this is computed by integrating a time-varying vector field. For large time spans, this integration process can become computationally expensive, and thus impede interactivity within visual analysis. For example, trajectories of a dense set of particles sufficiently covering the spatial domain must be computed in order to generate the FTLE field. If the user is interested in interactively exploring FTLE under varying time spans, the expense in computing the flow map can hinder this exploration. Techniques that can improve the flow map computation time are thus attractive for a wide-variety of downstream visualization tasks.

In the literature, numerous techniques have been proposed for fast FTLE computation~\cite{garth2007efficient, kasten2009localized, sadlo2009visualizing, brunton2010fast, lipinski2010ridge, sadlo2011time}. Most of the these techniques fall into the general category of reducing the number of flow map evaluations required to accurately estimate the FTLE field. In a similar way, many other techniques have been proposed for fast LIC computation~\cite{stalling1995fast, battke1997fast}, and fast streakline computation~\cite{weinkauf2010streak}. However, most of these techniques are targeted towards improving the computation time of a specific downstream visualization task. For the majority of flow visualization techniques, the flow map computation time acts as a bottleneck, and few techniques have focused on the core of the problem i.e. improving the flow map computation time.

Motivated by these problems, in this we work we propose a novel technique for fast and accurate flow map computation. We propose a novel coordinate-based neural network that serves as a surrogate for a flow map. Specifically, given a particle identified by a spatiotemporal coordinate, and time span, the network predicts the spatial position corresponding to the particle's integration under the flow field. Such neural representations of flow maps have been recently studied, both for 2D unsteady flows~\cite{han2021exploratory}, and more broadly for learning latent space representations~\cite{bilovs2021neural}. However, the learning of flow maps presents a number of challenges for existing methods. First, there is a steep training data requirement, where it is necessary to generate a large number of flow map samples on which to learn. Second, the input dimensionality varies over space, time, and time span, and thus in order to match the high dimensionality of the input space, the complexity (e.g. number of parameters) of the network often needs to be quite large. The complexity of the network can prove prohibitively expensive for training, provided the large dataset size, as well as prevent interactivity for use in downstream visual analysis.

Our approach aims to address, at once, these challenges through a novel network design that enables efficient inference, coupled with a novel optimization scheme for scalable training. A key aspect of our approach is that, through careful network design and optimization, we eliminate the need to learn from ground-truth flow map samples altogether. Rather, we take advantage of a basic property of flow maps: the instantaneous velocity of the flow map should be equivalent to the vector field. By optimizing the flow map derivative to represent the vector field, this ``primes'' the flow map itself to give a good approximation of particle transport, under small time spans. We show how to leverage this in devising a self-consistency criterion for learning the flow map under a range of time spans. Moreover, building on recent hybrid grid-MLP models~\cite{muller2022instant}, our network is efficient to evaluate, which enables both scalable training, as well as efficient inference. Our implementation can be found here : \url{https://github.com/SarojKumarSahoo/NIFM}

%Motivated by these problems, in this work, we propose a novel technique for fast and accurate flow map computation. We propose a novel deep learning based technique to learn a neural representation of the flow map for an underlying unsteady flow field. Our approach leverages the latest developments in the area of implicit neural representations (INR)~\cite{sitzmann2020implicit, takikawa2021neural, barron2021mip, muller2022instant} to model the flow map. Thus, given a particle's spatio-temporal position and an integration duration we can infer the flow map directly by querying the network. This eliminates the need for expensive numerical integration process. However, learning a neural flow map representation requires learning over a (n+2)-dimensional space (spatial, temporal and integration duration). Thus naively approaching the network design would require a significant amount of network parameters and training data to learn effectively. Consequently negatively impacting the inference time, and possibly negating the benefits of the approach over numerical integration scheme. To this end, we propose a novel network architecture that is (1) memory efficient, (2) has fast training and inference times, and (3) scales well to larger datasets. Additionally, we propose a novel way to optimize the network wherein we do not require any flow map samples for supervision. We learn the neural flow map representation in a 2 stage process. In the first stage we learn the flow field using the vectors as supervision and in the second stage we employ a form of self-supervision to learn the flow maps. Our carefully crafted network design facilitates this optimization process. 

Our main contributions can be summarized as follows:
\begin{enumerate}
    \item We propose a novel network architecture for learning a neural representation of flow maps, that is fast, accurate and scalable.
    \item We propose a novel way to optimize the network only using the vector field, without requiring access to flow map samples during optimization.
    \item We show the advantage of using our technique by comparing against existing techniques both qualitatively and quantitatively. We demonstrate that, with modest training time, our method provides for a more accurate flow map approximation, and is more efficient at inference time, and hence applicable to numerous unsteady flow visualization techniques.
\end{enumerate}



