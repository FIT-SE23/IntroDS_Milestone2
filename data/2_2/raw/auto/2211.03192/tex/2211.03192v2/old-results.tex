We first evaluate our method qualitatively for various 2D time-varying datasets. In Fig.~\ref{fig:ftle_fluid} we compare the FTLE generated using our technique and the baseline techniques. We can see that our technique is superior in terms of the visual quality of the FTLE. Both SIREN and Super-resolution show artifacts which manifests in different forms for both the cases. For the super-resolution method, grid-artifacts resulting from the use of convolutional layers appear quite prominently. On the other hand, our technique is able to produce a FTLE with no significantly visible artifacts and is able to capture the ridges quite well. In Fig.~\ref{fig:ftle_dg}, ~\ref{fig:ftle_cy} and~\ref{fig:ftle_bo} we show additional FTLE results for the double gyre, flow over cylinder and the boussinesq dataset for various other baselines. We can clearly see the shortcomings of the Encoder-Decoder network architecture from the FTLE results for the flow over cylinder and boussinesq dataset. Clearly, the network failed to learn a good approximation of the flow maps, we suspect this is largely due to (1) the complexity of the datasets, (2) the absence of positional encoding in the network design which has been shown in the literature~\cite{tancik2020fourier, sitzmann2020implicit}, to help capture high frequency details present in the data, and (3) the network design failing to satisfy the basic properties of the flow map described in Sec.~\ref{sec:integrationfree}. Similarly, the comparatively poor results of a SIREN is the consequence of the network not being able to satisfy the flow map properties. For fair comparison both SIREN and the Encoder-Decoder networks were trained with the same amount of data our technique was trained on. We further show comparison with Spline Interpolation technique by Li et al. Please note that, to fit the splines we provide the method with a very dense sampling of pathlines by integrating particles with the same starting time and time-span as the query pathlines. More specifically, we seed $3\times$ the total spatial resolution of the dataset uniformly at random at the starting time and then generate pathlines. The start time and the time span of the pathlines aligns with the start time and time span of the query pathlines. This allows the technique to utilize a better distribution of particles, and thus a better neighborhood for query interpolation. Alternatively, if the lagrangian representation consists of particles starting at some arbitrary time much earlier than the query start time the quality of the FTLE generated degrades significantly for the flow over cylinder and boussinesq dataset. Additionally, we also evaluate our method quantitatively for 2D time-varying datasets. Fig.~\ref{fig:quantitative} shows the error between the ground truth flow maps and the predicted flow maps for different method across different datasets. We vary the starting times and flow map duration to show the change in error for datasets that becomes increasingly complex over time. The flow map error reported is computed as the euclidean distance between the ground truth flow map and the predicted flow map with respect to the bounding box diagonal. We can see that our network achieves lower error comparatively against all the method across most of the datasets. The only exception being the Boussinesq dataset wherein the Spline Interpolation outperforms our technique. Given the distribution of samples and the number of pathlines splines were fit to, makes the task of interpolation for new trajectories easier and gives the technique an advantage over our technique. However, the low error comes at a cost of higher computational cost. 

In Table~\ref{tab:time}, we show the training/fitting to the spline times for the our technique and other baselines as well the inference times for generating flow maps. We can see that our technique has the fastest inference time when compared to all the other baselines. Please note that, the inference time reported for the flow map super-resolution technique includes the CNN-network inference time as well as the time required to compute the low-resolution flow map which is the input to the network. We argue that even though the network inference time is fast, the network still requires the computation of a  low-resolution flow map to generate the high-resolution one. The computation cost of generating the low-resolution flow map is not negligible and thus, should be taken into consideration for fair comparison to our method which does not rely on any additional information during the inference time. We can see that the B-spline curve fitting time is incredibly fast as compared to our training time. However, this fast fitting time for Spline interpolation has little bearing on the inference time, which can get really slow depending on the number of pathlines the B-splines curves were fit to and the total number of query pathlines. On the contrary, our method scales well to the large number of query pathlines and maintains a fast inference time across all the dataset.