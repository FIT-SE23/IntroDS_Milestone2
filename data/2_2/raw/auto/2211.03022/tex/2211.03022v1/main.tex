\documentclass{article}
\usepackage{arxiv}

\usepackage[utf8]{inputenc} % allow utf-8 input
\usepackage[T1]{fontenc}    % use 8-bit T1 fonts
\usepackage{hyperref}       % hyperlinks
\usepackage{url}            % simple URL typesetting
\usepackage{booktabs}       % professional-quality tables
\usepackage{amsfonts}       % blackboard math symbols
\usepackage{nicefrac}       % compact symbols for 1/2, etc.
\usepackage{microtype}      % microtypography
\usepackage{lipsum}		% Can be removed after putting your text content
\usepackage{graphicx}
\usepackage{doi}


\usepackage{amssymb}
\usepackage[figuresright]{rotating}

\usepackage{graphicx}
\usepackage{stackengine} 
\stackMath
\usepackage{float}
\usepackage{subcaption}
\usepackage{xspace}
\usepackage{amsmath}
\usepackage{makecell}
\usepackage{svg}
\usepackage{amsmath}
\usepackage{tikz}
\usepackage{paralist}
\usepackage{algorithmic}  
\usepackage{algorithm}
\usepackage[algo2e]{algorithm2e}
\usepackage{xkcdcolors}
\usetikzlibrary{shapes,arrows,chains,shadows,positioning}
\usetikzlibrary{arrows.meta}
\usetikzlibrary{decorations.pathreplacing}
\usepackage{geometry}
\usepackage{textcomp}
\usepackage{pgfplots}
\pgfplotsset{width=10cm,compat=1.9}

%\usetikzlibrary{decorations.pathreplacing}
\usepackage[utf8]{inputenc}
\usepackage[T1]{fontenc}    % use 8-bit T1 fonts
\usepackage{hyperref}       % hyperlinks
\usepackage{url}            % simple URL typesetting
\usepackage{multirow}
\usepackage{comment}
\DeclareUnicodeCharacter{2212}{-}




\title{Physics Informed Machine Learning for Chemistry Tabulation}

\author{
\hspace{1mm}Amol Salunkhe\\
 \hspace{1mm}University at Buffalo, 338 Davis Hall\\
  Buffalo, New York 14260\\
\And
\hspace{1mm}Dwyer Deighan\\
 \hspace{1mm}University at Buffalo, 338 Davis Hall\\
  Buffalo, New York 14260\\
 \And
  \hspace{1mm}Paul E. DesJardin\\
 \hspace{1mm}University at Buffalo, 338 Davis Hall\\
  Buffalo, New York 14260\\
\And
\hspace{1mm}Varun Chandola\\
 \hspace{1mm}University at Buffalo, 338 Davis Hall\\
  Buffalo, New York 14260\\
}

% Uncomment to remove the date
%\date{}

% Uncomment to override  the `A preprint' in the header
%\renewcommand{\headeright}{Technical Report}
%\renewcommand{\undertitle}{Technical Report}
\renewcommand{\shorttitle}{\textit{Physics Informed} ML for Chemistry Tabulation}

%%% Add PDF metadata to help others organize their library
%%% Once the PDF is generated, you can check the metadata with
%%% $ pdfinfo template.pdf
\hypersetup{
pdftitle={Physics Informed Machine Learning for Chemistry Tabulation},
pdfsubject={q-bio.NC, q-bio.QM},
pdfauthor={Amol Salunkhe},
pdfkeywords={Physics Informed Machine Learning, Deep Neural Networks, Combustion, Fluid Dynamics, Chemistry Tabulation},
}

\begin{document}
\maketitle

\begin{abstract}
%% Text of abstract
Modeling of turbulent combustion system requires modeling the underlying chemistry and the turbulent flow. Solving both systems simultaneously is computationally prohibitive. Instead, given the difference in scales at which the two sub--systems evolve, the two sub--systems are typically (re)solved separately. Popular approaches such as the {\em Flamelet Generated Manifolds} (FGM) use a two--step strategy where the governing reaction kinetics are pre--computed and mapped to a low--dimensional manifold, characterized by a few reaction progress variables (model reduction) and the manifold is then ``looked--up'' during the run--time to estimate the high--dimensional system state by the flow system. While existing works have focused on these two steps independently, in this work we show that joint learning of the progress variables and the look--up model, can yield more accurate results. We build on the base formulation and implementation \cite{ChemTab} to include the dynamically generated Themochemical State Variables (Lower Dimensional Dynamic Source Terms). We discuss the challenges in the implementation of this deep neural network architecture and experimentally demonstrate it's superior performance.
\end{abstract}


% keywords can be removed
\keywords{Physics Informed Machine Learning \and Deep Neural Networks \and Combustion \and Fluid Dynamics \and Chemistry Tabulation}

\section{Introduction}\label{introduction}
Modeling of turbulent flow combustion is central in the development of new combustion technologies in aviation, automotive and power generation~\cite{Giusti:2019}. Turbulent flow combustion combines two nonlinear and multi--scale phenomena: {\em turbulent flow} and {\em chemical reactions}. This coupling of the kinetic chemical reaction equations with the set of Navier–Stokes flow equations results in a problem that is too complex to be solved, at full resolution, by the current computational means. Even for a simple fuel such as methane, the combustion chemistry mechanism involves 53 species and 325 chemical reactions~\cite{grimech}, and the numbers increase with increasing fuel complexity. Solving the details of such mechanisms during the flow simulation can consume up to 75\% of the solution time ~\cite{ElAsrag2013ACB}. Hydrocarbon combustion, for example, involves from 50 to 7000 species depending on the fuel \cite{montgomery2005,lu2009}. Even with the aid of exascale computing, high-fidelity simulations of turbulent reactive flows with detailed kinetic remain computationally prohibitive \cite{chen2011,nouri2019}. 
\begin{figure}[H]
  \centering
  \includegraphics[width=\linewidth]{TurbulentCombustionModelling.pdf}
\caption{Overview of techniques used for handling the computational complexity in coupled turbulent flow combustion systems. }\label{fig:combustionchallenges}
\end{figure}

In most cases, the large scale separation between the combustion chemistry/flame (typically sub millimeter /microsecond scale) and the characteristic turbulent flow (typically centimeter or meter/minute or hour scale) allows simplifying assumptions to be made that enable increased computational efficiency by (re)solving chemistry and flow separately ~\cite{peters2001}. Major research has focused on decoupling the systems by the development of domain specific approximation methodologies. In figure  ~\ref{fig:combustionchallenges} we have described some major methodologies in the different domains to enable simulations of turbulent flow combustion. 

These domain specific approximations have enabled modular constructions of simulation systems. The Chemistry system is resolved first using a domain model. The solutions to the high dimensional reactions are parameterized and then stored. During the flow simulation, these solutions are looked--up to estimate the  thermochemical state, as shown in Figure~\ref{fig:reduced}.
\begin{figure}[H]
 \includegraphics[width=\linewidth]{TraditionalvsChemTabCombustionFlowWorkflow.pdf}
  \caption{Traditional vs ChemTab enabled approach in a Turbulent Combustion Flow System} \label{fig:reduced}
\end{figure}

Most of these domain specific approximations developed for increased computational efficiency rely on the existence of a theoretical low--dimensional thermochemical state--space manifold to which the combustion chemistry can be mapped~\cite{maas1992}. The central question then is, {\em how to efficiently model low--dimensional thermochemical manifolds that capture the relevant physics of the problem; and parametrize and approximate these manifolds which can then be accessed during turbulent flow simulations?} 

% condense the following two paragraphs
While existing approaches (collectively referred to as {\em state--space parametrization}~\cite{peters2001,piercemoin2004}) have been successful, they have primarily solved the two sub--problems -- {\em progress variable generation} to characterize the manifold, and {\em manifold approximation} to perform the lookup during run--time, independently. This can result in sub--optimal solutions because the progress variables, learnt using methods such as {\em Principal Component Analysis} (PCA)~\cite{sutherland20091563,biglari20154025}, are not necessarily optimized to perform the run--time lookup.  Similarly, while the traditional lookup approaches that use tabulation, or the recently proposed neural network based data--driven alternatives~\cite{bhalla2019}, facilitate efficient look--ups, the construction of the underlying data--structure or machine learning based model is not informed by the learning of the progress variables.

Our central hypothesis is that by simultaneously learning the progress variables and the manifold approximation (lookup model), we can achieve higher accuracy in terms of the estimation of the thermochemical state at run--time. But how does one combine the progress variable learning, an inherently linear mapping task, with a highly non--linear lookup model, while ensuring that the components influence each other during the learning phase? 

In a preliminary version of this work \cite{ChemTab} we proposed a framework called {\em ChemTab}, in which the learning of these two components is formulated as a joint optimization task. In this paper we extend that formulation to include newer constraints necessary for operationalization, dynamic construction of lower dimensional source terms based on the linear mapping and a novel Thermochemical State {\it Dynamic Source Term Regressor} that learns to predicts the highly nonlinear dynamic source terms and showcase the performance of this approach.  

\section{Related Work}
In this section we provide a brief overview of existing work that can be categorized into domain models, numerical/data-driven methods and deep neural networks.

\subsection{Domain Methods}
Common approaches to low--dimensional thermochemical manifold modeling are combustion chemistry mechanism reduction and thermochemical state--space parametrization ~\cite{Rastigejev13875} ~\cite{sutherland20091563}. Chemistry mechanism reduction approach cannot be generalized and in the recent past state--space parametrization approach has been the most dominant method comprising of two phases {\em progress variable generation} and {\em manifold approximation}. For progress variable generation, existing methods have either used domain models or numerical methods. 

Domain models like steady Laminar Flamelet Method (SLFM) ~\cite{PETERS1984319}, Flamelet--Generated Manifold (FGM) ~\cite{fgm2000} ~\cite{van2001}, Flamelet/Progress Variable approach (FPVA) ~\cite{piercemoin2004} ~\cite{ihme2005}  and Flamelet--Prolongation of ILDM model (FPI) ~\cite{gicquel2004} theorize that a multi--dimensional flame can be considered as an ensemble of multiple one--dimensional locally laminar flames (flamelets). These flamelets are patametrized by a combination of conserved and reactive scalars ~\cite{fgm2000} ~\cite{van2001} ~\cite{piercemoin2004} ~\cite{bojko2016}. A lot of research in this area builds on the principles laid out in ~\cite{ihme20127715} for progress variables regularization however the fundamental problem of generating adequate number of progress variables that capture the underlying physics is still open.

\subsection{Numerical Methods}
Numerical methods, like PCA, have shown significant promise for parametrization of the thermochemical state. PCA provides a method of generating reaction progress variables using the flamelet solutions, the state--space variables are still nonlinear functions of the reaction progress variables, and a nonlinear regression is learned to approximate the state--space manifold ~\cite{sutherland20091563} ~\cite{biglari20154025}~\cite{sutherland20091563} ~\cite{malik201830}~\cite{malik2020}. This purely numerical parametrization lack interpretability and may also not be generalizable enough due to variation capture maximization that may overlearn the numerical errors in the data. Linear Autoencoders have also been suggested ~\cite{Yellapantula2021} however this definition lacks incorporation of a principled approach to progress variable generation and thus may not be generalizable.

\subsection{Deep Neural Networks}
Deep Neural Networks have already achieved tremendous success in a number of domains such as computer vision and natural language processing, where large amounts of training data and highly expressive neural network architectures together give birth to solutions outperforming previously dominating methods. As a consequence, researchers have also started exploring the possibility of applying machine learning models to advance scientific discovery and to further improve traditional combustion modeling.

\subsubsection{Neural Networks for Lookup}
While domain based model have traditionally relied on tabular lookup, these are not scalable. tabulated data occupies a larger portion of the available memory on every node where the flow simulation is computing. Also the searching and retrieval of this pre--tabulated data becomes increasingly expensive in a higher--dimensional space. For example, assuming a standard 3 progress variable discretization (200, 100, 50) with say 15 tabulated thermochemical state variables, we obtain a pre--computed combustion table of 120Mb. The addition of a variable such as {\em enthalpy} with a very coarse discretization of 20 points, brings the size of the table to 2.4 Gb. To address the tabulation problem researchers like~\cite{bhalla2019} ~\cite{zhang20201} build on the work of \cite{ihme2009} to investigate the use of a neural networks for manifold approximation which replaces the Tabulation. The mapping between the progress variables (reduced dimensionality) and thermochemical state variables obtained using the flamelets solutions is learnt using a neural network. However, due to the highly non--linear, knotted and discontinuous nature of the lower dimensional manifolds formed by the progress variables generated \textit{a priori} the accuracy gained by a neural network is not satisfactory. 

\subsubsection{Physics-Informed Neural Networks (PINNs)}
Physics-Informed Neural Networks (PINNs) approximate solutions to Partial Differential Equations (PDE) by training a neural network to minimize a loss function; it includes terms reflecting the initial and boundary conditions along the space-time domain’s boundary and the PDE residual at selected points in the domain (called collocation point). PINNs are deep-learning networks that, given an input point in the integration domain, produce an estimated solution in that point of a differential equation after training \cite{raisi2018,raissi2020}. This is a promising area and the methods discussed cannot be adopted directly to solve the modeling of combustion thermo-chemistry and the parameterization of combustion manifold.


\subsubsection{Physics Guided Machine Learning}
Machine learning involve three key parts: data, model, and optimization, each of which can be integrated with prior physics knowledge. There's existing work describing physics--driven machine learning models for solving other physics problems ~\cite{willard2020,karpatne2017}, however, these methods generally focus on simpler physics and are not necessarily applicable in the domain of turbulent combustion. Our formulation is the most similar in spirit to this methodology and can be categorized as Physics-Informed Optimization.


\section{Background: Unsteady FGM}
In this work, we use the unsteady Flamelet Generated Manifolds (FGM) as the chemistry domain approximation method. FGM is a widely used tabulated chemistry method and can deal with a range of complicated conditions. FGM model shares the same theoretical basis with flamelet approaches~\cite{PETERS1984319}, in which a multi--dimensional flame can be considered as an ensemble of multiple one--dimensional flames. Generally FGM procedure used for combustion modeling follows the steps as shown below:
\begin{enumerate}
    \item Calculation of the representative 1--D flamelets.
    \item Projection of 1--D flamelets solutions Species Mass Fractions to progress variables space.
    \item Calculating the Lower Dimensional Source Terms using the projection generated above
    \item Mapping these Lower Dimensional Source Terms and other Thermo-Chemical variables to progress variables space.
    \item Generation of FGM tables according to FGM progress variables.
    \item Retrieval of thermo-chemical variables from the FGM tables according to FGM progress variables from Computational Fluid Dynamics (CFD) simulations.
\end{enumerate}

\paragraph{Notation}
The definition of terms used in the subsequent sections is provided in Table~\ref{tab:maesource}. We use bold upper-case letters to denote vectors (e.g., {\bf Y}) and use subscripts (e.g., $Y_i$) to denote the $i^{th}$ entry of the corresponding vector. Matrices are denoted using calligraphic letters (e.g., $\mathcal{Y}$).  
\subsection{Governing Equations}\label{subsec:1Dflamesol}
\begin{table}[H]
  \caption{Definitions for terms used in Section~\ref{subsec:1Dflamesol}. Terms are scalars unless noted otherwise.} 
  \begin{tabular}{ll}
   \makecell {
        \begin{tabular}{|l|l|}
            \hline
            Symbol & Description\\
            \hline
            $Z_{mix}$ & Mixture Fraction\\
            ${\bf C_{pv}}$ & Progress Variable (vector)\\
            ${\bf Y}$ & Species Mass Fraction (vector)\\    
            $\dot{\bf S}$ & Species Source Terms (vector)\\
            ${\bf h^0_{f}}$ & Heat of Formation for Species (vector)\\
            $T$ & Mixture Temperature\\
            ${\bf D}$ & Diffusivity for Species (vector)\\
            $x$ & Position in the 1--D coordinate\\
            $t$ & Time\\
            $u_x$ & Velocity along the $x$ dimension\\
            \hline
        \end{tabular}
    }
    \makecell {
        \begin{tabular}{|l|l|l|l|l|}
            \hline
             Symbol & Description\\
            \hline
            $\kappa$ & Thermal Conductivity\\
         	$Pr$ & Prandtl number\\
            $Sc$ & Schmidt number\\
            $Le$ & Lewis number \\
            $\mu$ & Viscosities \\
            $h$ & Total Enthalpy\\
            $s$ & \# Species in Mechanism\\
            $p$ & \# Progress Variables\\
                        $\rho$ & Mixture Density\\
            %$\dot{S_{i}}$ & Source Term of the $i^{th}$ species\\
            %$n$ & \# Flames\\
            %$\widehat{Y}$ & Reactive Scalars\\
            %$ \widehat{\dot{S}}$ & Reactive Scalars Source Terms\\
            %$\phi$ & Non--Linear function of $Y$\\
            %$\zeta$ & Non--Linear function of $\widehat{Y}$\\
            %$\psi$ & Non--Linear function of $\widehat{Y}$ and $Z_{mix}$\\
            %$k$ & No. of Species used to generate Progress Variables\\
           \hline
        \end{tabular}
    }   
  \end{tabular}
  \label{tab:maesource}
\end{table}
Conservation equations for mass, species, momentum and energy for the 1--D, fully compressible, and viscous flames, are given by:
\begin{eqnarray}
\frac{\partial \rho}{\partial t} + \frac{\partial \left(\rho u_x \right)}{\partial x} & = & 0 \label{eqn:1DflameA}\\
    \frac{\partial \left( \rho Y_i \right)}{\partial t} + \frac{\partial \rho u_x Y_i}{\partial x} & = & \frac{\partial}{\partial x}\left( \rho D_i \frac{\partial Y_i}{\partial x} \right) + \dot{S_i} \\\label{eqn:1Dflameb}
    \frac{\partial \left (\rho u_x \right)}{\partial t} + \frac{\partial \left( \rho u_x^2 \right)}{\partial x} & = & -\frac{\partial p}{\partial x} + \frac{\partial}{\partial x} \left(\mu \frac{\partial u_x}{\partial x} \right) \\\label{eqn:1Dflamec}
    \frac{\partial \left(\rho e_t \right)}{\partial t} + \frac{\partial}{\partial x} \left(\rho u_x H_t \right) & = & \frac{\partial}{\partial x} \left(u_x \mu \frac{\partial u_x}{\partial x} \right) + \mu \frac{c_p}{Pr} \left( 1 - \frac{1}{Le} \right)\frac{dT}{dx} \label{eqn:1DflameD}\\
            & + &\frac{1}{Sc} \frac{dh}{dx} - \sum \dot{S_i} h^o_{f,i}\nonumber
\end{eqnarray}
where the different terms are defined in Table~\ref{tab:maesource}.


We simplify the above equations making some well known assumptions. In 1D cartesian coordinates, the steady state solution to~\eqref{eqn:1DflameA}--\eqref{eqn:1DflameD} is obtained only when the total mass flux is zero, i.e., velocity field is zero $(u_x = 0)$ and so the four equations reduce to:
\begin{eqnarray}\label{eqn:energysteadymain}
        \frac{\partial}{\partial x} \left( \rho D_{i} \frac{\partial Y_i}{\partial x} \right) + \dot{S_i} & = & 0 \\\label{eqn:speciessteady}
        \frac{\partial}{\partial x} \left(\kappa \frac{\partial T}{\partial x} + \sum \rho D_{i} \frac{\partial Y_i}{\partial x} h_i \right) - \sum \dot{S_i} h^o_{f,i} & = & 0\label{eqn:energysteady}
\end{eqnarray}

In~\eqref{eqn:energysteady}, the final term in the energy equation is represented by the total sum of the product of all the source species and their respective heat of formation and is collectively called the source energy and will be denoted as $S_e$. Source energy is one of the crucial parameters in the combustion simulation and accurate chemistry description is required to define it. Prediction error of this term is used as the basis of comparison of our method against the other state of the art methods.

\subsection{Flamelet Solutions}
We assume that we have an ensemble of flamelets data generated by solving 1--D Steady State Flamelets differential equations in~\ref{eqn:energysteady} using a finite volume PDE solver. For each flamelet, we have access to the species mass fractions (${\bf Y}$), the thermochemical state variables ($\dot{\bf S}$ and $S_e$), and the corresponding mixture fraction ($Z_{mix}$), which are generated using the solver. We denote the collection of flamelets as matrices $\mathcal{Y}$ and $\dot{\mathcal{S}}$, and vectors $S_e$ and ${\bf Z_{mix}}$:
{
\begin{equation}
        \mathcal{Y} = \begin{bmatrix}%not correct
            Y_{11} &..&..&Y_{1s}\\
            ..&..&..&..\\
            ..&..&..&..\\
            ..&..&..&..\\
            Y_{n1}&..&.. & Y_{ns}
        \end{bmatrix},\quad
        \dot{\mathcal{S}} = \begin{bmatrix} %not correct
            S_{11} &..&..&S_{1s}\\
            ..&..&..&..\\
            ..&..&..&..\\
            ..&..&..&..\\
            S_{n1}&..&.. & S_{ns}
        \end{bmatrix},\quad
        {\bf S_e} = \begin{bmatrix}
            S_{e_{1}}\\
            ..\\
            ..\\
            ..\\
            S_{e_{n}}
        \end{bmatrix},\quad
        {\bf Z_{mix}} = \begin{bmatrix}
            Z_{mix_{1}}\\
            ..\\
            ..\\
            ..\\
            Z_{mix_{n}}
        \end{bmatrix}
\label{eqn:datagenerated}
\end{equation}
}
Each row above corresponds to data from one flamelet (Flamekey) at a particular $x$ position (Xpos). 

\section{ChemTab Formulation \& Implementation}\label{sec:extended-formulation-implementation}
In this section we discuss how the data generated by the flamelet solver as described in \ref{eqn:datagenerated} can be used to jointly learn the progress variables and the manifold approximation by ChemTab. 

In the ChemTab aided approach, the unsteady FGM approach is replaced with the following three steps:
\begin{enumerate}
    \item Calculation of the representative 1D flamelets (data generation)
    \item Using the data generated jointly generate Progress Variables  (encoder) and Manifold Approximation (regressor(s)) using ChemTab
    \item Retrieval of thermo--chemical variables from the ChemTab--regressor(s) according to progress variables from CFD simulations.
\end{enumerate}
As can be noted several of the dis-jointed steps of the unsteady FGM procedure are replaced by our unified formulation which enables greater accuracy and ease of deployment as a key module of the overall turbulent combustion flow system. 

\subsection{Formulation}\label{sec:extended-formulation}
This section presents the formulation that focuses on creating a linear encoder for progress variable generation that is influenced by source energy and some of the key higher dimensional source Terms. Source energy is a key thermochemical state variable along with lower dimensional Source Terms that are needed at run--time by the CFD solver so we include those as a key output(s) of the regressor(s) as a part of the formulation. 

The functional relationship between ${\bf Y},{\bf \dot{S}}, S_{e}$ is described by \ref{eqn:energysteadymain}, from a machine learning standpoint we can learn these functional approximations from the data \ref{eqn:datagenerated}. These functional approximations conceptualize the relationships between Species Mass Fractions, Source Terms and Source Energy as follows:
\begin{align}
    \dot{\bf S} & = \Phi({\bf Y})\\
    S_{e} & = - \sum_{i}^{s} h_{f,i}^{0} * \dot{S_{i}} = \Psi({\bf Y})
    \label{eqn:conceptual--energysteady}
\end{align}  

When we linearly embed ${\bf Y}$ into the lower dimensional ${\bf C_{pv}}$ the functional relationship will change and can be conceptualized as follows: 

\begin{align}
    \dot{\bf S} & = \phi({\bf C_{pv}})\\
    S_{e} & = \psi({\bf C_{pv}})
    \label{eqn:conceptual-II-energysteady}
\end{align}

We now present our formulation which jointly addresses the three learning problems. {\em Encoder} ($\omega$) that projects the higher dimensional Species Mass Fractions ${\bf Y}$ to a lower dimensional manifold to create the lower dimensional ${\bf C_{pv}}$--the linear embedding/progress variables. The {\em {\it Physics Regressor}} which learns the relationships ($\phi$, $\psi$) between the progress variables $({\bf C_{pv}}, Z_{mix})$ and key Source Terms $\dot{\bf{S}}$ and the most import thermochemical state variable $S_e$ respectively. And finally the {\em {\it Dynamic Source Term Regressor}} which learns the relationship $\gamma$ between the progress variables $({\bf C_{pv}}, Z_{mix})$ and lower dimensional Source Terms $\tilde{\dot{\bf{S}}}$. The lower dimensional Source Terms are constructed using the same Encoder used for the species mass fractions. 

Clearly all of these learning problems are interrelated. The {\it Physics Regressor} by the virtue of learning $\phi$, $\psi$ influences $\omega$. The {\it Dynamic Source Term Regressor} also by the virtue of learning $\gamma$ influences $\omega$. To account this  inter-relatedness we use a multi-objective optimization formulation as follows.

\begin{subequations}\label{eqn:chemtabdynsrcformulation}
    \begin{alignat}{1}
        \min \quad \alpha_{1}\sum_{j=1}^{n} \mathcal{L}_{phy_1}(S_{e},\psi({\bf C_{pv}},Z_{mix})) \quad+ \alpha_{2}\sum_{i=1}^{s}\sum_{j=1}^{n} \mathcal{L}_{phy_2}({\bf \dot{S}_{ij}},& \phi({\bf C_{pv}},Z_{mix})) \quad+ \alpha_{3}\sum_{i=1}^{p}\sum_{j=1}^{n} \mathcal{L}_{dsreg}({\bf \tilde{\dot{S_{ij}}}},\gamma({\bf C_{pv}},Z_{mix}))\\ %+ \quad \alpha_{3}\mathcal{C}(Y,\theta)
        &\textrm{s.t.}\\ %\quad t \in R \\
            &\def\sss{\scriptscriptstyle}
            \setstackgap{L}{8pt}
            \def\stacktype{L}
            \stackunder{\mathrm{{\bf C_{pv}}}}{\sss p} = \omega({\bf Y}) = \stackunder{{\bf Y}}{\sss s} \times \stackunder{\mathcal{W}}{\sss s\times p} \label{eqn:linearity-constraint} \\
            &p << s\label{eqn:dimreduction}\\
            &\mathcal{W^{T}} \times \mathcal{W} = I \label{eqn:ext-WO-constraint} \\
            &\mathcal{W}_{ic} > 0 \quad \forall \quad i,c\label{eqn:ext-NN-constraint}\\
            &({\bf C_{pv}} \oplus Z{mix})^{T} \times ({\bf C_{pv}} \oplus Z{mix}) \approx I \label{eqn:ext-AR-constraint}\\
            &\def\sss{\scriptscriptstyle}
            \setstackgap{L}{8pt}
            \def\stacktype{L}
            \stackunder{\mathrm{\tilde{\dot{\bf{S}}}}}{\sss p} =\omega({\bf \dot{S}}) =  \stackunder{{\bf \dot{S}}}{\sss s} \times \stackunder{\mathcal{W}}{\sss s\times p} \label{eqn:same-linearity-constraint-dynsourceterms}
\end{alignat}
\end{subequations}    

Each of the Loss function $\mathcal{L}$ in \ref{eqn:chemtabdynsrcformulation} serves as a method of evaluating how well the learning is across the $n$ data points and $\alpha_{1}$,$\alpha_{2}$, $\alpha_{3}$ are the contributions of the individual loss functions to the overall objective. These weights can be either solved for as an hyper-parameter or can be a design choice for the subject matter expert.

$\mathcal{L}_{phy_1}$ captures the loss between the true $S_{e}$ and the predicted $\hat{S_{e}}$ across all of the $n$ data points. $\mathcal{L}_{phy_2}$ captures the loss between the true ${\bf \dot{S}}_{i}$ and the predicted ${\hat{\bf \dot{S}}}_{i}$ for all the source species across all of the $n$ data points. A slight variation would be to use $k$ key source terms instead of all the $s$ source terms. $\mathcal{L}_{dsreg}$ captures the loss between the dynamically created true lower dimensional source terms ${\tilde{\bf \dot{S}}}_{i}$ and the predicted $\hat{{\tilde{\bf \dot{S}}}}_{i}$.

The constraints \ref{eqn:linearity-constraint} and \ref{eqn:dimreduction} ensures that the embedding $\omega$ is linear dimensionality reduction. The constraint \ref{eqn:ext-WO-constraint} ensures that the linear projection matrix is Orthogonal and the constraint \ref{eqn:ext-AR-constraint} ensures that the progress variables generated from this embedding are orthogonal. These constraints together are Principal Component Analysis (PCA) inspired. And the \ref{eqn:ext-NN-constraint} constraint ensures every entry in the matrix is positive, this is a necessary constraint for the productionalization of our work. The constraint \ref{eqn:same-linearity-constraint-dynsourceterms} ensures that the dynamic source terms $\tilde{\dot{S}}_{i}$ are constructed using the same encoder. 

\subsection{Implementation}\label{sec:extended-implementation}
We present a Deep Neural Architecture implementation of the joint optimization formulation discussed in the previous section. The implementation assumes the input data is in the structure described in ~\eqref{eqn:datagenerated}. The construction of the Deep Neural Network is presented below:
    
\begin{subequations}\label{eqn:DNN-dynsrc--expansion}
    \begin{alignat}{1}    
    f_{\mathcal{\theta}}^{[0]}(y) &= y \label{eqn:dnn-first-layer}\\
    f_{\mathcal{\theta}}^{[1]}(y) &= {\bf C_{pv}} =  (W^{[0]} f_{\mathcal{\theta}}^{[0]}(y)) \label{eqn:dnn-second-layer}\\
    f_{\mathcal{\theta}}^{[2]}(y) &= (f_{\mathcal{\theta}}^{[1]}(y) \oplus Z_{mix}) \label{eqn:dnn-concat-layer}\\
    f_{\mathcal{\theta}}^{[l]}(y) &= \mathcal{\sigma} \: \mathcal{\rm o} \: (W^{[l−1]} f_{\mathcal{\theta}}^{[l-1]}(y) \:+\: b^{[l-1]} ) \:\:\: \forall \quad l \quad \textrm{s.t.} \quad {3 \leq l \leq L-1}  \label{eqn:dnn-post-concat-layer-to-last}\\
    f_{\mathcal{\theta}}(y) &= f_{\mathcal{\theta}}^{[L]}(y) = \mathcal{\sigma} \: \mathcal{\rm o} \: (W^{[L-1]} f_{\mathcal{\theta}}^{[L-1]}(y) \:+\: b^{[L-1]} )\label{eqn:dnn-phy-reg-output}\\
    %f_{\mathcal{\theta}_{d_{out}}}^{[L]} = k + p \\
    \tilde{\dot{{\bf S}}} &= f_{\mathcal{\theta}}^{[1]}(\dot{S})\label{eqn:dnn-dyn-source-creation}\\
    g_{\mathcal{\theta}}^{[0]}(y) &= f_{\mathcal{\theta}}^{[2]}(y)\label{eqn:dnn-dyn-source-first-layer} \\
    g_{\mathcal{\theta}}^{[l]}(y) &= \mathcal{\sigma} \: \mathcal{\rm o} \: (W^{[l−1]} g_{\mathcal{\theta}}^{[l-1]}(y) \:+\: b^{[l-1]} ) \:\:\: \forall \quad l \label{eqn:dnn-dyn-src-layers}\\
    g_{\mathcal{\theta}}(y) &= g_{\mathcal{\theta}}^{[L]}(y) = \mathcal{\sigma} \: \mathcal{\rm o} \: (W^{[L-1]} g_{\mathcal{\theta}}^{[L-1]}(y) \:+\: b^{[L-1]} )\label{eqn:dnn-dyn-src-reg-output}
\end{alignat}
\end{subequations} 
% TODO: make layer notation clear
In the dnn architecture \ref{eqn:DNN-dynsrc--expansion} the first layer \ref{eqn:dnn-first-layer} accepts the species mass fractions $y$. The second layer \ref{eqn:dnn-second-layer} is the {\em Encoder} which generates the projection matrix $\mathcal{W}$ used to create the progress variables $\tilde{Y}$.This layer has a linear activation function. The third layer \ref{eqn:dnn-concat-layer} concatenates $\tilde{Y}$ and $Z_{mix}$. The {\em Physics Regressor} is composed of the subsequent layers of the $f_{\theta}$ network. The layers in the {\em Physics Regressor}  \ref{eqn:dnn-post-concat-layer-to-last} use a nonlinear activation function. And the last layer \ref{eqn:dnn-phy-reg-output} network generates the source energy $S_{e}$ and the key source terms $\dot{S_i}$. 

The $g_{\theta}$ network is the {\it Dynamic Source Term Regressor}. The \ref{eqn:dnn-dyn-source-creation} creates the true values for the Dynamic Source Terms $\tilde{\dot{S}}$. The \ref{eqn:dnn-dyn-source-first-layer} accepts the progress variables concatenated in \ref{eqn:dnn-concat-layer}.  The {\it Dynamic Source Term Regressor} is composed of the subsequent layers of the $g_{\theta}$ network. The layers in the {\it Dynamic Source Term Regressor}  \ref{eqn:dnn-dyn-src-layers} use a nonlinear activation function. The last layer \ref{eqn:dnn-dyn-src-reg-output} generated the predicted lower dimensional dynamic source terms $\hat{\tilde{\dot{S}}}$.

The Deep Neural Network can be then trained using the following loss function and layer constraints:
\begin{equation}\label{eqn:DNN-ChemTab-dynsrc--loss}
    \begin{aligned}
        \arg\min_{\mathcal{\theta}} \quad \mathcal{L} (f_{\mathcal{\theta}}(y), \mathcal{S})  + \mathcal{L}( g_{\mathcal{\theta}}(y) , \tilde{\dot{{\bf S}}}) \\
        s.t. \quad W^{[0]T}W^{[0]} = I\\
        W_{ic} > 0  \quad \forall \quad i,c \\
        f_{\mathcal{\theta}}^{[2]}(y)^Tf_{\mathcal{\theta}}^{[2]}(y) \approx I\\
        \quad \dot{{\bf S}} \times W^{[0]} = \tilde{\dot{{\bf S}}}
    \end{aligned}
\end{equation} 

Where $\mathcal{S}= \{S_e, \dot{S}_i \} $ for $k$ important species. The $\mathcal{L}$ loss functions can take several variations, Mean Absolute Error (MAE) , Sum of Squared Error(SSE) and $-R^{2}$ being the common choices.   

\begin{equation}\label{eqn:r2-metric}
    \begin{aligned}
    R^{2} = 1-\frac{\sum_{j=0}^{n}(S-\widehat{S})^2}{\sum_{i=0}^{n}(S-\bar{S})^2}
    \end{aligned}
\end{equation}

The common $R^2$ metric is defined in the equation \ref{eqn:r2-metric}. This metric focuses on maximizing the variance capture.

\begin{equation}\label{eqn:DNN-ChemTab-r2-loss}
    \begin{aligned}
    \min \quad -1 * \left( \underbrace{\frac{1}{k+1}\left(1-\frac{\sum_{j=0}^{n}({S}_{e_{j}}-\hat{S}_{e_{j}})^2}{\sum_{j=0}^{n}({S}_{e_{j}}-\bar{S}_{e_{j}})^2} \quad + \quad  \sum_{i=0}^{k}{1-\frac{\sum_{j=0}^{n}(\dot{S}_{ij}-\hat{{\dot{S}}}_{ij})^2}{\sum_{j=0}^{n}({\dot{S}}_{ij}-\bar{{\dot{S}}}_{ij})^2}} \quad\right)}_{\textrm{{\it Physics Regressor Loss}}} + \underbrace{\quad \frac{1}{p}\sum_{i=0}^{p}{1-\frac{\sum_{j=0}^{n}(\tilde{\dot{S}}_{ij}-\hat{\tilde{\dot{S}}}_{ij})^2}{\sum_{j=0}^{n}(\tilde{\dot{S}}_{ij}-\bar{\tilde{\dot{S}}}_{ij})^2}}\quad}_{\textrm{{\it Dynamic Source Term Regressor Loss}}} \right)
    \end{aligned}
\end{equation}

In the equation \ref{eqn:DNN-ChemTab-r2-loss} we present the loss function we used for our DNN implementation. The first term of the loss function measures the loss of the {\it Physics Regressor} and the second term the loss of the {\it Dynamic Source Term Regressor}. For the {\it Physics Regressor} we choose the average $R^2$ metric across Source Energy $S_{e}$ and the $k$ key higher dimension Source Terms $\dot{S_{i}}$. For the{\it Dynamic Source Term Regressor} we choose the average $R^2$ metric across $p$ Dynamic Source Terms $\tilde{\dot{S}}_{i}$.

\begin{figure}[H]
  \centering
  \includegraphics[width=\linewidth]{ChemTab-Extended-Mini.png}
  \caption{ChemTab Extended Architecture and Training Procedure}
  \label{fig:chemtab-extended-architecture}
\end{figure}
 
The figure \ref{fig:chemtab-extended-architecture} describes the architecture and the process used to  
train the model. Of note is the construction of true values of the dynamic source terms $\tilde{\dot{S}}_{i}$ that are used to train the {\it Dynamic Source Term Regressor}. For each batch iteration we create these values based on the current value of the projection matrix $\mathcal{W}$. Essentially these values will be lagging one iteration behind the {\it Physics Regressor}. As the training stabilizes the updates in the projection matrix will plateau and the values of true values dynamic source terms will stabilize. 

\section{Experimentation \& Results}
In this section we explain the specifics of the data set creation, the data sets used for evaluation, the training strategy, some interesting results and the performance of the implementations of the various formulation in the context of the multiple objectives.

\subsection{Dataset Generation}\label{sec:datageneration}
The training data was generated by solving 1--D Steady State Flamelets differential equations using a finite volume PDE solver. To model the chemical kinetics reaction rates, a variety of mechanisms are adopted in the combustion community. Depending on the hydrocarbon fuel different mechanisms are chosen which closely describe the chemistry associated with the fuel of simulation. Methane is the basic hydrocarbon and one of the major products of many higher order hydrocarbons. GRI--Mech 3.0 is one of the widely used Methane mechanism to model the reaction kinetics. This mechanism consists of 53 chemical species and 325 reactions. 

The Flamelet solver discretizes the domain into $200$ grid points (200 observations on the axial coordinate) in between the fuel and the air boundary and $100$ flame are solved to steady--state. Once the solution reaches steady--state the solver completes one iteration. For the next iteration flame solution is strained by reducing the domain by $0.99$ and the process is continued until the flame extinguishes. Each flame is then tagged with the corresponding strain rate that is called a flame--key. To train the model 20,000 data points (100 flames and 200 grid points) for a single pressure setting are used. The data is generated using an in--house solver which creates the flame solutions and stores the required data. Some of the generated data represents extinguished flames, we choose to retain this data in our model training, this becomes an extremely challenging task as there are flames that show large amounts of activity and the model has to learn this phenomenon.

\subsection{Implementation and Settings}
 
We implemented ChemTab using Tensorflow 2.3.0, Keras and Adam optimizer. Models were trained on a server with Nvidia Quadro RTX 5000 GPU and cuDNN 8.0 and CUDA 11.0. 


\subsubsection{Hyper-Parameters Investigated}
\begin{table}[H]
 {\small
  \caption{Hyper-Parameters}
  \label{tab:ChemTabhyper-parameters}
  \begin{tabular}{|ll||ll|}
    \hline
    Parameter & Range & Parameter & Range\\
    \hline
 	Largest Layer Width & 128 - 4096 & Number of $C_{pv}$ & 4 - 12\\
    Dropout  & 0\% - 40\% & Activation Functions & ReLU, TanH, SeLU\\
    Batch Size & 128 - 1028 & Output Scaler & MinMaxScaler, RosbustScaler\\ 
   \hline
\end{tabular}\label{tbl:hyper-parameters}}
\end{table}

We performed bayesian optimization on the several hyper-parameters mentioned in the \ref{tbl:hyper-parameters}. The hyper-parameter optimization module sampled several values for the range of the hyper-parameter.



\begin{figure}[H]
  \centering
  \includegraphics[width=0.6\linewidth]{ch4-o2-hpo_history.pdf}
  \caption{}
  \label{fig:no-of-trials-hpo}
\end{figure}

As can be seen in \ref{fig:no-of-trials-hpo} it took the hyper-parameter optimizer several hundred trials before it started reaching diminishing returns. This indicates the hyper-parameter settings we are changing are non-trivial. A 'trial' in this case are the number of hyper-parameter configuration experiments it has performed at a given point.


\subsubsection{\# of Reaction Progress Variables}

Two key hyper-parameters are the number of progress variables $C_{pv}$ and the Width of the Middle Layer for the Regressor. We construct both of the regressors with a identical number of layers and the width of each layer. The Middle Layer is the largest and 4 layers on either side of it which keep decreasing by a factor of 2.  As an example if the Middle Layer width is 800 then the regressors end up looking like: {\it input,50,100,200,400,800,400,200,100,50,output}.

\begin{figure}[H]
  \centering
  \includegraphics[width=0.6\linewidth]{ch4-o2-hpo-contour-layerwidth-numcpv.pdf}
  \caption{DNN Layer Width and no. of $C_{pv}$ HPO Objective Contour Plot}
  \label{fig:hpo-contour-layerwidth-numcpv}
\end{figure}

In the figure \ref{fig:hpo-contour-layerwidth-numcpv} we present the influence of model size and the number of $C_{pv}$ used on the model's performance. As it turns out the number of  $C_{pv}$ used didn't have a very large impact on performance (more  $C_{pv}$ = more  $C_{pv}$ source terms to predict accurately). Nonetheless we can see the highest performing regions are with high  $C_{pv}$ count is 9 and large network width 800.


\subsubsection{Final Model Configurations}
After the hyper-parameter optimization we settled on the final configurations.

\begin{table}[H]
 {\small
  \caption{Model Parameters}
  \label{tab:ChemTabParameters}
  \begin{tabular}{|ll||ll|}
    \hline
    Parameter & Value & Parameter & Value\\
    \hline
 	Learning Rate & 0.001 & Number of Layers & 11\\
 	Output-Scaler & RobustScaler & Regresssor Layer Shapes & input|49|99|198|396|792|396|198|99|49|output \\
    Dropout  & 1.522\% & Activation Functions & ReLU\\
    Early Stopping  & Yes & Number of epochs & 500\\
    Batch Size  & 407 & Network Weight Initialization & Glorot Uniform Distribution\\ 
   \hline
\end{tabular}\label{tbl:results-hyper-parameters}}
\end{table}

In the table \ref{tbl:results-hyper-parameters} we present the final values of the parameters learn't from hyper-parameter optimization and key model settings.

\subsection{Evaluation}
As discussed in \ref{sec:extended-formulation-implementation} the formulation can use several objective functions for the individual regressor(s) during the model training. We trained several variations using the $R^2$ metric based loss function as defined in \ref{eqn:DNN-ChemTab-r2-loss}. We will evaluate the overall model performance based on the same and the individual terms using $R^2$ metric as defined in \ref{eqn:r2-metric}.

\subsection{Results}

\begin{figure}[H]
  \centering
  \includegraphics[width=0.6\linewidth]{ch4-o2-best-model-training-plot.pdf}
  \caption{Model Training and Validation Performance}
  \label{fig:extended-model-loss}
\end{figure}

In the \ref{fig:extended-model-loss} we present the Model Loss function value across the training epochs (in this case loss is $-R^2$, we've made it positive for display purposes). The model performance rapidly increases across the first 125 training epochs and then is steady from epochs 125 to 300. After epoch number 300 the model improves only marginally. 

\subsubsection{Model Performance}
Here we compare the performance of the best model with the other DNN based benchmarks and the original framework benchmark.

{\it{\bf Current Framework Comparison}}
The current framework uses FGM based progress variables and Conformal Mapping based Tabulation and Lagrange Polynomial Interpolation based lookup. The tabulation was generated by using the entire data--set. The best $R^2$ for $S_{e}$ from that the framework generated on the data--set was 0.852417. The best ChemTab model trained on 50\% of the data showed a 16\% increase in $R^2$. This increase although high comes from the limitation of the current framework to include more than 2 progress variables and the realization of that through conformal mapping. We present a more principled comparison with the state--of--the--art methods in the next section.

{\it{\bf Other Baseline Comparisons}}
We now present the performance comparison of our model with the appropriate baselines. We note that we compare the performance of the {\it Physics Regressor} against the following baselines that are purely trained for just that task.


\begin{table}
  \caption{Current state of the art methods and ChemTab} \label{tab:baselines}
  \begin{tabular}{|l|c|l|}
    \hline
    Method Abbreviation & Progress Variable Generation & Manifold Approximation ($S_{e}$)\\
   \hline
    FGM--CPVG--DNN\label{method:FGM--CPVG--DNN} & \makecell[l] {FGM Constrained }  & \makecell {DNN}\\
    %PCA--PVG--GP\label{method:PCA--PVG--GP} & \makecell[l] {PCA }  & \makecell {Gaussian Process}\\
    %PCA--PVG--DNN\label{method:PCA--PVG--DNN} & \makecell[l] {PCA }  & \makecell {DNN}\\
    DNN--PVG(NL)--DNN\label{method:DNN--PVG(NL)--GP} & \makecell[l] {Non--Linear Encoder}  & \makecell {DNN}\\
    DNN--PVG(UL)--DNN\label{method:DNN--PVG(UL)--DNN} & \makecell[l] {Unconstrained Linear Encoder }  & \makecell {DNN}\\
    ChemTab & \makecell[l] {Physics constrained Linear Encoder \ref{eqn:dnn-second-layer}}  & \makecell {DNN}\\
\hline
\end{tabular}
\end{table}


\begin{table}[H]
    \caption{Source Energy and Key Species Benchmark ($R^2$) }
         \begin{tabular}{|l|l|l|l|l|}
        \hline
         Dependent & ChemTab  & DNN--PVG(UL)--DNN & DNN--PVG(NL)--DNN & FGM--CPVG--DNN \\
         \hline  
         $S_{e}$ & 0.995184 & {\bf 0.996250} & 0.992878 & {\bf 0.966763} \\
         $\dot{S}_{O2}$ & 0.996258 & {\bf 0.996672} & 0.994565 & 0.988661\\
         $\dot{S}_{CO}$ & 0.998296 & 0.996609 & 0.996406 & 0.969174 \\
         $\dot{S}_{CO2}$ & 0.996334 & {\bf 0.998193} & {\bf 0.997022} & 0.995969\\
         $\dot{S}_{H2O}$ & 0.998542 & 0.998026 & 0.993117 & 0.974178\\
         $\dot{S}_{OH}$ & 0.994757 & 0.993910 & 0.985228 & 0.969222\\
         $\dot{S}_{H2}$ & 0.997004 & 0.995364 & 0.993076 & 0.946442\\
         $\dot{S}_{CH4}$ & 0.991398 & 0.99867 & {\bf 0.997792} & 0.925120\\
        \hline
        \end{tabular}
    \label{tab:results-ext-benchmark-comparison}
\end{table}       
None of the benchmark models adhere to the operationalization constraints of \ref{eqn:linearity-constraint} , \ref{eqn:same-linearity-constraint-dynsourceterms} \ref{eqn:ext-NN-constraint} and hence can not be used however we use them as reference.  As seen in the table \ref{tab:results-ext-benchmark-comparison}tThe {\it Physics Regressor} performs better in most cases and performs adquately in others.


{\it{\bf Current Framework Comparison}}
For the best model we present the $R^2$ metric for each of the regressors
\begin{table}[H]
\centering
  \caption{Best Model $R^2$ Scores} 
  \begin{tabular}{ll}
  \makecell{{\it Physics Regressor}}
  &
  \makecell{{\it Dynamic Source Term Regressor}}
\\     
  \makecell{
        \begin{tabular}{|l|l|}
        \hline
         Dependent &       $R^2$ \\
         \hline  
         $S_{e}$ & 0.995184 \\
         $\dot{S}_{O2}$ & 0.996258 \\
         $\dot{S}_{CO}$ & 0.998296 \\
         $\dot{S}_{CO2}$ & 0.996334 \\
         $\dot{S}_{H2O}$ & 0.998542 \\
         $\dot{S}_{OH}$ & 0.994757 \\
         $\dot{S}_{H2}$ & 0.997004 \\
         $\dot{S}_{CH4}$ & 0.991398 \\
        \hline
        \end{tabular}}
  &
  \makecell{
        \begin{tabular}{|l|l|}
        \hline
                    Dependent &       $R^2$ \\
        \hline
        $\tilde{\dot{S}}_{1}$ & 0.993107 \\
        $\tilde{\dot{S}}_{2}$ & 0.998045 \\
        $\tilde{\dot{S}}_{3}$ & 0.997667 \\
        $\tilde{\dot{S}}_{4}$ & 0.997427 \\
        $\tilde{\dot{S}}_{5}$ & 0.998396 \\
        $\tilde{\dot{S}}_{6}$ & 0.994888 \\
        $\tilde{\dot{S}}_{7}$ & 0.998462 \\
        $\tilde{\dot{S}}_{8}$ & 0.995614 \\
        $\tilde{\dot{S}}_{9}$ & 0.997340 \\
        \hline
        \end{tabular}}
  \end{tabular}
  \label{tab:results-r2-scores-methane}
 \end{table} 

As can be observed in the table \ref{tab:results-r2-scores-methane}, the model does an excellent job of capturing the variance across the entire dataset.


\begin{figure}[H]
  \centering
  \includegraphics[width=1.0\linewidth]{ch4-o2-qq-static-source-prediction.pdf}
  \caption{True Value vs Predicted Value Plots - Physics Regressor}
  \label{fig:qq-static-source-prediction}
\end{figure}

In the figure \ref{fig:qq-static-source-prediction} the true value is plotted on the X-axis and the prediction is plotted on the Y-axis. It can be observed that the {\it Physics Regressor} model does extremely well and is able to capture most of the variability across the Source Energy ($S_{e}$)/souener except for the range $0.5e^{11}$ and $1.0e^{11}$. The model under-predicts for these values. The model over-predicts for $S_{O_{2}}$ for the ranges between -1000 and 0. The model overall does a decent job for $S_{CH_{4}}$ but over or under predicts marginally through the range. This is expected from the source term for the fuel Methane($CH_{4}$) as there will be more variability compared to other source terms. This will need further investigation and a stratified oversampling strategy per batch may alleviate this issue.



\begin{figure}[H]
  \centering
  \includegraphics[width=1.0\linewidth]{ch4-o2-qq-dynamic-source-prediction.pdf}
  \caption{True Value vs Predicted Value Plots - Dynamic Source Term Regressor}
  \label{fig:qq-dynamic-source-prediction}
\end{figure}

In the figure \ref{fig:qq-dynamic-source-prediction} it can be observed that the {\it Dynamic Source Term Regressor} model does reasonably well and is able to capture the variability across the range adequately. We believe a post-training with a stratified oversampling and a combination of an ensemble for the {\it Dynamic Source Term Regressor} will improve the model performance even further.  


\subsubsection{Residual Analysis}
In this section we present residual analysis of the {\it Physics Regressor}. 
\begin{figure}[H]
  \centering
  \includegraphics[width=0.4\linewidth]{ch4-o2-flamekeys-residual-marginal-density.pdf}
  \includegraphics[width=0.4\linewidth]{ch4-o2-xpos-souener-residual-marginal-density.pdf}
  \includegraphics[width=0.4\linewidth]{ch4-o2-flamekeys-residual.pdf}
    \includegraphics[width=0.4\linewidth]{ch4-o2-xpos-souener-residual.pdf}
  \caption{Source Energy Residuals by Flamekeys and Xpos}
  \label{fig:ext-residual-charts}
\end{figure}
In the figure \ref{fig:ext-residual-charts} we observe a concentration of residuals in the intial ranges of both the Flame Key and the Xpos. This is where the maximum combustion activity with high volatility exists. The problem could be simplifed by trying to learn this volatile behavior separately.  

\begin{figure}[H]
  \centering
  \includegraphics[width=1.0\linewidth]{ch4-o2-flame-0-00011529-dependants}
  \caption{Model Performance for Flamelet Key = 0.00011529}
  \label{fig:extended-flamelet-high-activation}
\end{figure}

In the figure \ref{fig:extended-flamelet-high-activation} we present the {\it Physics Regressor} model prediction against the true values. The true data belongs to a particular flamelet and represents a highly active/combustive flame. The X-axis is the 'Xpos' and Y-axis is the value of the variable.


\begin{figure}[H]
  \centering
  \includegraphics[width=1.0\linewidth]{ch4-o2-flame-0-00009223-dependants.pdf}
  \caption{Model Performance for Flamelet Key = 0.00009223 }
  \label{fig:extended-flamelet-zero-activation}
\end{figure}

In the figure \ref{fig:extended-flamelet-zero-activation} the true data belongs to a particular flamelet and represents a non-active/extinguished flame. The model does a reasonable job and struggles in the initial ranges of the 'Xpos'. This is consistent with the observation in the \ref{fig:ext-residual-charts} residual charts.  We can certainly improve the model performance by adopting a stratified batch construction where each batch has a portion of the extinguished flames data. Alternatively, we can train a separate model for this data and/or create an ensemble to improve the overall performance.

\subsubsection{Constraint Satisfaction}
In this section we present the results on the constraints of the implementation.
%WO,UN, AR, NN
\begin{table}[H]
\centering
  \caption{Non--Negative Constraint on the Weights of the Linear Embedding} 
  \small
        \begin{tabular}{|l|l|l|l|l|l|l|l|l|}
            \hline
             $w_{1}$ & $w_{2}$ & $w_{3}$ & $w_{4}$& $w_{5}$ & $w_{6}$ & $w_{7}$& $w_{8}$ & $w_{9}$\\
            \hline
          0.02 &  0.0   &   0.0   &  0.0   &   0.0   &  0.0   &   0.0   &  0.0   &  0.0  \\ 
       0.0   &  0.0   &  0.0   &  0.0   &  0.0   &  0.0   &  0.0   &   0.17 &  0.0  \\ 
        0.01 &   0.01 &  0.0   &   0.01 &   0.0   &  0.0   &   0.02 &   0.0   &   0.01\\ 
        0.02 &   0.0   &   0.0   &   0.03 &   0.0   &   0.05 &   0.0   &   0.0   &   0.03\\ 
        0.0   &  0.0   &   0.01 &   0.0   &  0.0   &   0.0   &  0.0   &   0.07 &   0.0  \\ 
        0.0   &   0.0   &   0.12 &   0.0   &   0.02 &   0.01 &   0.01 &   0.0   &   0.0  \\ 
        0.01 &   0.0   &  0.0   &   0.02 &  0.0   &  0.0   &  0.0   &  0.0   &   0.02\\ 
       0.0   &  0.0   &   0.31 &  0.0   &  0.0   &  0.0   &  0.0   &  0.0   &  0.0  \\ 
       0.0   &  0.0   &  0.0   &  0.0   &  0.0   &  0.0   &  0.0   &   0.48 &  0.0  \\ 
        0.4  &  0.0   &  0.0   &  0.0   &  0.0   &  0.0   &  0.0   &  0.0   &  0.0  \\ 
       0.0   &  0.0   &   0.66 &  0.0   &  0.0   &  0.0   &  0.0   &  0.0   &  0.0  \\ 
       0.0   &  0.0   &  0.0   &  0.0   &  0.0   &  0.0   &   0.33 &  0.0   &  0.0  \\ 
       0.0   &  0.0   &  0.0   &  0.0   &  0.0   &  0.0   &  0.0   &   0.24 &   0.0  \\ 
        0.42 &  0.0   &  0.0   &  0.0   &  0.0   &  0.0   &  0.0   &   0.0   &  0.0  \\ 
        0.0   &   0.04 &  0.0   &   0.0   &   0.02 &   0.0   &   0.04 &   0.05 &   0.0  \\ 
        0.0   &  0.0   &   0.0   &   0.0   &   0.0   &   0.0   &  0.0   &   0.24 &   0.0  \\ 
       0.0   &  0.0   &  0.0   &  0.0   &   0.08 &  0.0   &  0.0   &  0.0   &  0.0  \\ 
       0.0   &  0.0   &   0.0   &  0.0   &  0.0   &  0.0   &  0.0   &  0.0   &   0.11\\ 
       0.0   &  0.0   &  0.0   &  0.0   &  0.0   &   0.58 &  0.0   &  0.0   &  0.0  \\ 
        0.64 &  0.0   &  0.0   &  0.0   &  0.0   &  0.0   &  0.0   &  0.0   &  0.0  \\ 
       0.0   &   0.39 &  0.0   &  0.0   &  0.0   &  0.0   &  0.0   &  0.0   &  0.0  \\ 
       0.0   &   0.1  &  0.0   &  0.0   &  0.0   &  0.0   &  0.0   &  0.0   &  0.0  \\ 
       0.0   &   0.01 &   0.0   &  0.0   &   0.0   &   0.1  &  0.0   &  0.0   &  0.0  \\ 
       0.0   &  0.0   &  0.0   &   0.0   &  0.0   &   0.0   &   0.05 &  0.0   &   0.0  \\ 
        0.01 &  0.0   &   0.0   &  0.0   &   0.0   &   0.1  &   0.0   &  0.0   &  0.0  \\ 
       0.0   &  0.0   &  0.0   &   0.16 &  0.0   &  0.0   &  0.0   &  0.0   &  0.0  \\ 
       0.0   &  0.0   &  0.0   &  0.0   &  0.0   &   0.54 &  0.0   &  0.0   &  0.0  \\ 
       0.0   &  0.0   &  0.0   &  0.0   &  0.0   &   0.28 &  0.0   &  0.0   &  0.0  \\ 
       0.0   &  0.0   &  0.0   &  0.0   &  0.0   &   0.14 &  0.0   &  0.0   &  0.0  \\ 
       0.0   &  0.0   &  0.0   &  0.0   &   0.3  &  0.0   &  0.0   &  0.0   &  0.0  \\ 
       0.0   &  0.0   &   0.57 &  0.0   &  0.0   &  0.0   &  0.0   &  0.0   &  0.0  \\ 
       0.0   &  0.0   &  0.0   &  0.0   &  0.0   &  0.0   &  0.0   &  0.0   &   0.62\\ 
       0.0   &   0.5  &  0.0   &  0.0   &  0.0   &  0.0   &  0.0   &  0.0   &  0.0  \\ 
       0.0   &   0.53 &  0.0   &  0.0   &  0.0   &  0.0   &  0.0   &  0.0   &  0.0  \\ 
       0.0   &  0.0   &  0.0   &  0.0   &  0.0   &   0.5  &  0.0   &  0.0   &  0.0  \\ 
       0.0   &  0.0   &  0.0   &   0.42 &  0.0   &  0.0   &  0.0   &  0.0   &  0.0  \\ 
       0.0   &  0.0   &  0.0   &   0.32 &  0.0   &  0.0   &  0.0   &  0.0   &  0.0  \\ 
       0.0   &  0.0   &  0.0   &  0.0   &   0.42 &  0.0   &  0.0   &  0.0   &  0.0  \\ 
       0.0   &  0.0   &  0.0   &  0.0   &   0.5  &  0.0   &  0.0   &  0.0   &  0.0  \\ 
       0.0   &  0.0   &  0.0   &  0.0   &  0.0   &  0.0   &   0.74 &  0.0   &  0.0  \\ 
       0.0   &  0.0   &  0.0   &  0.0   &  0.0   &  0.0   &  0.0   &  0.0   &   0.62\\ 
       0.0   &  0.0   &  0.0   &   0.44 &  0.0   &  0.0   &  0.0   &  0.0   &  0.0  \\ 
       0.0   &  0.0   &  0.0   &   0.52 &  0.0   &  0.0   &  0.0   &  0.0   &  0.0  \\ 
       0.0   &  0.0   &   0.36 &  0.0   &  0.0   &  0.0   &  0.0   &  0.0   &  0.0  \\ 
       0.0   &  0.0   &  0.0   &  0.0   &   0.53 &  0.0   &  0.0   &  0.0   &  0.0  \\ 
       0.0   &  0.0   &  0.0   &  0.0   &   0.45 &  0.0   &  0.0   &  0.0   &  0.0  \\ 
       0.0   &  0.0   &  0.0   &   0.49 &  0.0   &  0.0   &  0.0   &  0.0   &  0.0  \\ 
        0.5  &  0.0   &  0.0   &  0.0   &  0.0   &  0.0   &  0.0   &  0.0   &  0.0  \\ 
       0.0   &  0.0   &  0.0   &  0.0   &  0.0   &  0.0   &  0.0   &   0.79 &  0.0  \\ 
       0.0   &  0.0   &  0.0   &  0.0   &  0.0   &  0.0   &   0.03 &  0.0   &  0.0  \\ 
       0.0   &  0.0   &  0.0   &  0.0   &  0.0   &  0.0   &   0.58 &  0.0   &  0.0  \\ 
       0.0   &  0.0   &  0.0   &  0.0   &  0.0   &  0.0   &  0.0   &  0.0   &   0.47\\ 
       0.0   &   0.55 &  0.0   &  0.0   &  0.0   &  0.0   &  0.0   &  0.0   &  0.0  \\
           \hline
        \end{tabular}
        \label{tab:extended-nn-constraints}
\end{table}

As can be observed in table \ref{tab:extended-nn-constraints} all the weights are non-negative and the \ref{eqn:ext-NN-constraint} constraint is completely satisfied.

\begin{table}[H]
    \centering
    \caption{ Orthogonality Constraint on the Weights of the\\
    Linear Embedding}
\begin{tabular}{|l|l|l|l|l|l|l|l|l|l|}
    \hline
      & $w_{1}$ & $w_{2}$ & $w_{3}$ & $w_{4}$& $w_{5}$ & $w_{6}$ & $w_{7}$& $w_{8}$ & $w_{9}$\\
    \hline
 	$w_{1}$ & 0.997& 0.0  & 0.0  & 0.001& 0.0  & 0.002& 0.0  & 0.0  & 0.001\\
    $w_{2}$ & 0.0  & 0.997& 0.0  & 0.0  & 0.001& 0.001& 0.002& 0.002& 0.0  \\
    $w_{3}$ & 0.0  & 0.0  & 1.001& 0.0  & 0.002& 0.001& 0.001& 0.001& 0.0  \\
    $w_{4}$ & 0.001& 0.0  & 0.0  & 1.01 & 0.0  & 0.002& 0.0  & 0.0  & 0.001\\
    $w_{5}$ & 0.0  & 0.001& 0.002& 0.0  & 1.007& 0.0  & 0.001& 0.001& 0.0  \\
    $w_{6}$ & 0.002& 0.001& 0.001& 0.002& 0.0  & 0.999& 0.0  & 0.0  & 0.002\\
    $w_{7}$ & 0.0  & 0.002& 0.001& 0.0  & 0.001& 0.0  & 0.998& 0.002& 0.0  \\
    $w_{8}$ & 0.0  & 0.002& 0.001& 0.0  & 0.001& 0.0  & 0.002& 1.006& 0.0  \\
    $w_{9}$ & 0.001& 0.0  & 0.0  & 0.001& 0.0  & 0.002& 0.0  & 0.0  & 1.003\\
   \hline
\end{tabular}
\label{tab:extended-wo-constraints}
\end{table}
The \ref{eqn:ext-WO-constraint} constraint conformity is measured through covariance of the output of the linear encoder. As can be observed in table \ref{tab:extended-wo-constraints} all the diagonal entries are close to 1 and the non-diagonal entries are close to 0 indicating that the Orthogonality constraint on the weights is satisfied. 


\begin{table}[H]
    \centering
    \caption{Orthogonality Constraint on the Output of the\\
       Linear Embedding}
     \begin{tabular}{|l|l|l|l|l|l|l|l|l|l|l|}
        \hline
          & $Z_{mix}$& $C_{pv_{1}}$ & $C_{pv_{2}}$ & $C_{pv_{3}}$ & $C_{pv_{4}}$& $C_{pv_{5}}$ & $C_{pv_{6}}$ & $C_{pv_{7}}$ & $C_{pv_{8}}$& $C_{pv_{9}}$\\
        \hline
        $Z_{mix}$ & 0.0 &    0.0 &   0.0 &    0.0 &   0.0 &   0.0 &    0.0 &   0.0 &    0.0 &   0.0  \\
         $C_{pv_{1}}$ & 0.0 &    0.0 &    0.0 &   0.0 &    0.0 &   0.0 &    0.0 &    0.0 &   0.0 &    0.0  \\
         $C_{pv_{2}}$ &0.0 &    0.0 &    0.0 &   0.0 &    0.0 &   0.0 &    0.0 &    0.0 &   0.0 &   -0.001 \\
         $C_{pv_{3}}$ & 0.0 &   0.0 &   0.0 &    0.0 &   0.0 &    0.0 &   0.0 &   0.0 &    0.0 &   -0.001 \\
         $C_{pv_{4}}$ &0.0 &    0.0 &    0.0 &   0.0 &    0.0 &   0.0 &    0.0 &    0.0 &   0.0 &   0.0  \\
         $C_{pv_{5}}$ &0.0 &   0.0 &   0.0 &    0.0 &   0.0 &    0.001 & 0.0 &   -0.001  &0.0 &    0.005 \\
         $C_{pv_{6}}$ & 0.0 &    0.0 &    0.0 &   0.0 &    0.0 &   0.0 &    0.0 &    0.0 &   0.0 &    0.0  \\
         $C_{pv_{7}}$ &0.0 &    0.0 &    0.0 &   0.0 &    0.0 &   -0.001 &  0.0 &    0.001 & 0.0 &   -0.002 \\
         $C_{pv_{8}}$ & 0.0 &   0.0 &   0.0 &    0.0 &   0.0 &    0.0 &   0.0 &   0.0 &    0.0 &   -0.001 \\
         $C_{pv_{9}}$ &0.0 &    0.0 &   -0.001 & -0.001 & 0.0 &    0.005 &  0.0 &   -0.002 &-0.001  &0.067 \\
       \hline
    \end{tabular} 
    \label{tab:extended-ar-constraints}
\end{table}
The \ref{eqn:ext-AR-constraint} constraint conformity is measured through covariance of the output of the linear encoder. As can be observed in table \ref{tab:extended-ar-constraints} 
all the non-diagonal entries are close to 0 indicating that the Orthogonality constraint on the $C_{pv}$ is satisfied. 

\section{Conclusion}
Building on our prior work \cite{ChemTab} we presented an extended formulation for jointly learning the progress variables and the manifold approximation for solving the high--dimensional chemistry in combustion models with newer constraints and a novel {\it Dynamic Source Term Regressor} and showcased the generalizability of this approach across multiple datasets. Our approach follows the principle of physics guided neural networks~\cite{karpatne2017}, which are increasingly becoming popular for many scientific modeling tasks, though no solutions exist that can directly benefit the combustion community. Our formulation outperforms the state--of--the--art state--space parametrization in combustion. The generated reaction progress variables can be interpreted by examining the projection/weight matrix, {\em $\mathcal{W}$}, and thus, allows for physical insights into the systems being modeled. This formulation leverages the projection matrix to dynamically create additional thermochemical state variables which are learnt by the {\it Dynamic Source Term Regressor}r--the ease/difficulty of this learning task in turn influences the projection matrix.  

In the future we will work on extending the current formulation with an Autoencoder of the mass fraction which will allow us to incorporate the influence of reconstruction error in the learning of the embedding. Quantification of uncertainity in the estimation of the Thermochemical State variables will also be our next focus. We believe the deep neural network based implementation lends naturally to the adoption of deep ensembles for uncertainty quantification.

\section{Acknowledgments}
Funded by the United States Department of Energy’s (DoE) National Nuclear Security Administration (NNSA) under the Predictive Science Academic Alliance Program III (PSAAP III) at the University at Buffalo, under contract number DE--NA0003961.

\bibliographystyle{splncs04}
%\bibliography{references}
\documentclass[journal]{IEEEtran}
\usepackage{cite}
\usepackage{amsmath} 

\usepackage{subfigure}
\ifCLASSINFOpdf
\usepackage[pdftex]{graphicx}
  % declare the path(s) where your graphic files are
  \graphicspath{{../pdf/}{../jpeg/}}
  % and their extensions so you won't have to specify these with
  % every instance of \includegraphics
  \DeclareGraphicsExtensions{.pdf,.jpeg,.png}
\else
  % or other class option (dvipsone, dvipdf, if not using dvips). graphicx
  % will default to the driver specified in the system graphics.cfg if no
  % driver is specified.
  \usepackage[dvips]{graphicx}
  % declare the path(s) where your graphic files are
  \graphicspath{{../eps/}}
  % and their extensions so you won't have to specify these with
  % every instance of \includegraphics
  \DeclareGraphicsExtensions{.eps}
\fi  
\usepackage{amsmath}
\usepackage{cases}
\usepackage{stfloats}
\usepackage{amsfonts}
\usepackage{subeqnarray}
\usepackage{longtable}
\usepackage{supertabular}
\usepackage{setspace}
\usepackage{multirow}
\usepackage{booktabs}
% \usepackage{algorithm}
% \usepackage{algorithmic}
\usepackage[ruled,linesnumbered]{algorithm2e}
\makeatletter
\newcommand{\nosemic}{\renewcommand{\@endalgocfline}{\relax}}% Drop semi-colon ;
\newcommand{\dosemic}{\renewcommand{\@endalgocfline}{\algocf@endline}}% Reinstate semi-colon ;
\newcommand{\pushline}{\InDPP}% Indent
\newcommand{\popline}{\Indm\dosemic}% Undent
\let\oldnl\nl% Store \nl in \oldnl
\newcommand{\nonl}{\renewcommand{\nl}{\let\nl\oldnl}}% Remove line number for one line
\makeatother
\usepackage[export]{adjustbox} 
\usepackage{booktabs}
\usepackage{setspace}
\usepackage{xcolor}
\usepackage{mathrsfs}
\usepackage{amsmath}
\usepackage{array}
\usepackage{amssymb}
\usepackage{amsthm}
\usepackage{microtype}
\usepackage{url}
\usepackage{amsfonts,amssymb}
% \usepackage{bbm}
\usepackage{dsfont}
\usepackage{mathtools}
\usepackage{xcolor,colortbl}
\usepackage{colortbl}
\usepackage{graphicx}
% \usepackage{tabularray}

\newcommand{\mc}[2]{\multicolumn{#1}{c}{#2}}
\definecolor{Gray}{gray}{0.85}
\definecolor{Whitecolor}{rgb}{1,1,1}


\hyphenation{op-tical net-works semi-conduc-tor}

\setlength{\textfloatsep}{5pt}
\allowdisplaybreaks
\begin{document}
\setstretch{1}
\title{\textls[-25]{The Design of By-product Hydrogen Supply Chain Considering Large-scale Storage and Chemical Plants: A Game Theory Perspective}}
\author{Qianni~Cao,~\IEEEmembership{Student~Member,~IEEE},
Boda~Li,~\IEEEmembership{Student~Member,~IEEE},
Mengshuo~Jia,~\IEEEmembership{Member,~IEEE}, and 
Chen~Shen,~\IEEEmembership{Senior~Member,~IEEE}



%\thanks{M. Jia and C. Shen are with the State Key Laboratory of Power Systems, Tsinghua University, 100084 Beijing, China. Y. Wang and G. Hug are with the Power Systems Laboratory, ETH Zurich, 8092 Zurich, Switzerland.}
}
        
%\thanks{This work was supported in part by the Joint Funds of the National Natural Science Foundation of China under Grant U1766206 (Correspondence to Chen Shen).}
%\thanks{M. Jia, C. Shen and Z. Wang are affiliated with the State Key Laboratory of Power Systems, Department of Electrical Engineering, Tsinghua University, Beijing 100084, China (e-mail addresses: jms16@mails.tsinghua.edu.cn, shenchen@mail.tsinghua.edu.cn,
%    wangzhaojian@mail.tsinghua.edu.cn).}% <-this % stops a space
% \thanks{Manuscript received April 19, 2005; revised August 26, 2015.}

%\markboth{Submitted to IEEE Trans. Smart Grid}%
%{Shell \MakeLowercase{\textit{et al.}}: Bare Demo of IEEEtran.cls for IEEE Journals}
\maketitle


\begin{abstract}
Hydrogen, an essential resource in the decarbonized economy, is commonly produced as a by-product of chemical plants. To promote the use of by-product hydrogen, this paper proposes a supply chain model among chemical plants, hydrogen-storage salt caverns, and end users, considering time-of-use (TOU) hydrogen price, coalition strategies of suppliers, and road transportation of liquefied and compressed hydrogen. The transport route planning problem among multiple chemical plants is modeled through a cooperative game, while the hydrogen market among the salt cavern and chemical plants is modeled through a Stackelberg game. The equilibrium of the supply chain model gives the transportation and trading strategies of individual stakeholders. Simulation results demonstrate that the proposed method can provide useful insights on by-product hydrogen market design and analysis.
\end{abstract}
%Although the historical data of renewable generations could be assumed as publicly known
% Note that keywords are not normally used for peerreview papers.
\begin{IEEEkeywords}
%   Wind power, chance constraint, OPF, distributed computing, confidentiality preservation
Hydrogen market, large-scale storage, Stackelberg game, cooperative game, supply chain
\end{IEEEkeywords}
\IEEEpeerreviewmaketitle

\section*{Nomenclature}
\addcontentsline{toc}{section}{Nomenclature}

\subsection*{Indices} 
\begin{IEEEdescription}[\IEEEusemathlabelsep\IEEEsetlabelwidth{$aaaaaaaa$}]
	\item[$i,j$]		Index of chemical plants.
	\item[$t$]		Index of time periods during the day.
	\item[$n$]		Index of hydrogen processing equipment, including liquefiers and compressors.
	\item[$I+1$]		Index of the salt cavern.
\end{IEEEdescription}
\subsection*{Parameters} 
\begin{IEEEdescription}[\IEEEusemathlabelsep\IEEEsetlabelwidth{$aaaaaaaa$}]
	\item[$I$]		Number of chemical plants.
	\item[$T$]		Number of time periods.
	\item[$p_{o}$]		Retail price purchased by customers from the salt cavern.
	\item[$\underline{p}_{t},\overline{p}_{t}$]	    Lower and upper bound of the buying price offered by the salt cavern to chemical plants.
	\item[$Q_{trans}$]		Maximal injection rate of the salt cavern.
	\item[$N_\mathcal{C}$]		Number of compressors with different capacity.
	\item[$N_\mathcal{D}$]		Number of liquefiers with different capacity.
	\item[$Q_{i,t}$]		By-product hydrogen quantity produced by chemical plant $i$ in period $t$.
	\item[$\boldsymbol{Q_{pr}}$]        Capacity set of hydrogen processing equipment(kg/h), $\boldsymbol{Q_{pr}}=\{Q_{pr}^{n}\}, \forall n$.
	\item[$Q_\mathcal{C},Q_\mathcal{D}$]        Capacity of a tube trailer and a tanker truck (kg/trip).
	\item[$w_{t}$]      Electricity price in period $t$.
	\item[$\gamma_{c},\gamma_{d}$]      Electricity consumption for unit compressed hydrogen and liquefied hydrogen (kwh/kg).
	\item[$\boldsymbol{K_{1}}$]     Initial investment set of hydrogen processing equipment, $\boldsymbol{K_{1}}=\{K_{1}^{n}\}, \forall n$.
	\item[$K_{2}^{c},K_{2}^{d}$]        Initial investment cost of a tube trailer and a tanker truck.
	\item[$K_{3}$]      Operation cost of a tube trailer (or a tanker truck) in each period.
	\item[$\boldsymbol{T_{a}}$]     $T_{a}^{i,j}$ represents duration from chemical plant $i$ to $j$ ($j= I+1$ represents the salt cavern).
	\item[$\beta_{L1}$]      1 - hourly evaporation rate during the tanker truck loading.
	\item[$\beta_{L2}$]      1 - hourly evaporation rate during transit by a tanker truck.
\end{IEEEdescription}
\subsection*{Decision variables of the salt cavern} 
\begin{IEEEdescription}[\IEEEusemathlabelsep\IEEEsetlabelwidth{$aaaaaaaa$}]
	\item[$p_{t}$]		Buying price the salt cavern offers to chemical plants in period $t$.
	\item[$q_{i,t}^{trans}$]		Hydrogen transaction amount of chemical plant $i$ in period $t$, measured as hydrogen shipped from chemical plant $i$ at the end of the time period $t$.
	\item[$u_{i,I+1}$]		Binary variables. Equals to 1 when products from chemical plant $i$ is shipped directly to the salt cavern. Otherwise, $u_{i,I+1}$ equals to 0.
\end{IEEEdescription}
\subsection*{Decision variables of chemical plants} 
\begin{IEEEdescription}[\IEEEusemathlabelsep\IEEEsetlabelwidth{$aaaaaaaa$}]
	\item[$q_{i,t}^{pr}$]		Hydrogen quantity chemical plant $i$ compressed/liquified in period $t$.
	\item[$\boldsymbol{x_{i}}$]		$\boldsymbol{x_{i}}=\{x_{i}\}, \forall n$ is a set of binary variables. $x_{i}^{n}=1$ when the type of hydrogen processing equipment is selected to purchase. Otherwise, $x_{i}^{n}=0$.
	\item[$N_{i}^{cars}$]		Integer variables of number of tube trailers (or tanker trucks) purchased by chemical plant $i$.
	\item[$u_{i,j}$]		Binary variables. Equals to 1 when products from chemical plant $i$ is shipped to chemical plant $j$. Otherwise, $u_{i,j}$ equals to 0.
	\item[$q_{i,t}^{store}$]		Hydrogen quality in the tube trailer (or tanker truck) left at chemical plant $i$ before filled to capacity in period $t$.
	\item[$q_{i,t}^{unpr}$]		Hydrogen quantity temporarily stored in low-pressure storage tanks before compression or liquefication in period $t$.
	\item[$n_{i,t}^{cars}$]		Integer variables of tube trailers (or tanker trucks) leave chemical plant $i$ in period $t$.
\end{IEEEdescription}



\section{Introduction}
\subsection{Motivation}
\IEEEPARstart{I}{n} 
the context of emission peak and carbon neutrality, hydrogen is not only regarded as a critical alternative to fossil fuel to achieve carbon neutrality but offers versatility and flexibility that renewables cannot reach\cite{Allan2021}. As one of the most cost-effective options, hydrogen produced as a by-product from many chemical plants serves as a cheap and large-scale source of hydrogen. Moreover, by-product hydrogen is usually sufficiently clean and well suited for a wide range of applications, such as fuel cell (FC)-based cogeneration, FC vehicles, domestic heating, and so on\cite{CAMPANARI2020335}. However, the potential of by-product hydrogen has yet to be realized, which is emitted and thus wasted in most cases. Therefore, it presents opportunities as a new revenue stream for chemical plants and promisingly delivers on announced pledges of energy conversion nationwide in the mid-term. However, the lack of infrastructure development such as large-scale storage, logistical supply chain establishment and unexplored market have slowed down its further development.

Salt cavern storage is one of the most promising technologies to achieve large-scale, fast and secure hydrogen storage\cite{ANDERSSON201911901},which offers the most promising option owing to their low investment cost, high sealing potential and low cushion gas requirement\cite{CAGLAYAN20206793}. Notable projects are the salt cavity storages for hydrogen in Teeside, UK, and Texas, USA\cite{Gregoire2019}, demonstrating the operation feasibility on a full industrial scale. However, the business of acquiring, storing and selling by-product hydrogen has not yet been presented as an option by salt cavern operators, which inspires the work to design by-product hydrogen supply chain considering large-scale storage and chemical plants in this paper.

\subsection{Literature Review}

As demand and production capacity for hydrogen grows robustly in recent years, the outlines of hydrogen markets are starting to emerge worldwide. Initial trade and market price discoveries come first on a regional and local basis\cite{James2021}. Infrastructure development, transparent pricing benchmark and logistical supply chain establishment are key growth challenges faced by this new traded commodity just becoming established in energy commodity markets\cite{Allan2021}. 

Presently,  the hydrogen market is far from mature but is showing great potential. Many researchers focus on the planning of the hydrogen supply chain, considering various market scales, hydrogen sources and transportation modes. Life cycle analysis to estimate the economic and environmental benefits was conducted on global\cite{BRANDLE2021117481}, regional\cite{OBARA2019848} or national\cite{REN2020118482} scales. For different hydrogen sources, steam methane reforming (SMR)\cite{CARRERA2021107386} , coal gasification (CG)\cite{LI202027979}, biomass gasification (BG)\cite{CHO2019527,LUMMEN2020118996} and electrolysis (ELE)\cite{WANG2022122194} are common production technologies in recent researches. Considering hydrogen production based on different feedstocks and energy sources, an optimal structure of the hydrogen, biomass and {$\rm CO_{2}$} networks were determined in \cite{GABRIELLI2020115245}. To make comparisons of different transportation modes, Ref. \cite{FAZLIKHALAF202034503} considered four common options with various criteria and scenarios. Ref. \cite{GIM20121162} introduced a method for comparing different transport possibilities of tube or liquid trailer vs. pipeline delivery. The results showed that each transportation technology had a maximally cost-efficient niche and there was no single perfect solution for the entire system. Recently, large-scale storage for liquid hydrogen is of great attention. Ref. \cite{SEO2020114452}  considered  integrated bulk storage of hydrogen and concluded that a centralized storage structure and liquefication in central production plants can reduce the overall cost. Similarly, the status and key gaps for the commercialization of hydrogen liquefication technology with large-scale storage were discussed in \cite{RATNAKAR202124149}. A combination of the hydrogen supply chain with other energy sources has also attracted the attention of many researchers. Ref. \cite{xiao2018} established a local energy market for electricity and hydrogen. Ref. \cite{CARRERA2021116861} proposed a methodological design framework for hydrogen and methane supply chains based on Power-to-Gas systems.

In particular, by-product hydrogen has seen growing attention these years. Ref. \cite{YANEZ2018777} for the first time assessed the economic advantages, the techno-economic feasibility and the central role of reusing by-product hydrogen in the early phase of hydrogen infrastructure in the northern Spain region. A multi-period programming was designed in \cite{YOON2022112083} to make use of existing infrastructure for by-product hydrogen and natural gas (NG) pipelines, which demonstrated the economic benefits of by-product hydrogen. Even though, the potential of by-product hydrogen remains to be discovered.

Meanwhile, most of the literature focuses on maximizing the total benefit of the whole hydrogen supply chain. Ref. \cite{HAN20125328} aimed to maximize social welfare in Korea by planning both capacity and technology of production, storage as well as transportation in an envisioned nationwide hydrogen supply chain. Ref. \cite{WICKHAM2022117740} assessed the effects that hydrogen grades play in the development of a cost-effective hydrogen supply chain. Ref. \cite{EHRENSTEIN2020115486} incorporated the concept of biophysical limits of the planet to address the optimal design of the hydrogen supply chain. An optimization method was proposed in \cite{QUARTON2020113936} for an integrated value chain of carbon dioxide and hydrogen. Individual rationality was introduced in \cite{GUO2021119608}, where the peer-to-peer transaction, endogenous market-clearing price, and uncertainties in hydrogen production were considered in detail. However, most works failed to consider the strategic behaviors and the profit of individual participants, which differed from the usual practice that suppliers and retailers are private companies and operate with a profit-driven mode.

The research gaps for the existing works are: 
\begin{enumerate}
	\item The potential of by-product hydrogen is yet to be realized and its corresponding market is waiting for further exploration.
	\item The dynamic process of chemical plants and salt caverns considering hydrogen generation, compression (or liquefaction), and the transaction is waiting to be modeled.
	\item The interactions and dynamic strategic behaviors of each stakeholder desire a more dedicated modeling framework that captures profits and rationality of individual participants.
\end{enumerate}

\subsection{Contribution}
In this work, we study the by-product hydrogen supply chain considering large-scale storage and multiple chemical plants. The main contributions are threefold:
\begin{enumerate}
	\item We establish a business model for salt caverns to acquire and store by-product hydrogen from chemical plants and sell them to end-users. The by-product hydrogen supply chain composed of each stakeholder in the business model is investigated.
	\item The hour-by-hour decision-making process of each stakeholder, i.e., chemical plants and the salt cavern, is investigated and mathematically modeled under the proposed business model, providing a foundation for the TOU hydrogen pricing strategy. 
	\item The by-product hydrogen market is formulated as a game, considering the individual rationality of each stakeholder. The planning problem among multiple chemical plants is modeled through a cooperative game. The hydrogen market among the salt cavern and chemical plants is modeled through a Stackelberg game, in which the salt cavern is the leader and chemical plants are the followers. 
\end{enumerate}

\section{A business model of salt caverns and chemical plants}
In this section, we develop a business model for salt caverns to acquire by-product hydrogen from chemical plants and sell them to end-users. Generation, large-scale storage, and consumptive way of by-product hydrogen in the business model is introduced first. Then, the comparison between the by-product hydrogen supply chain and the present hydrogen supply chain is made. Followed by this, the structure of the by-product hydrogen market under the proposed business model is introduced in the following section. 

\subsection{Generation, Large-scale Storage and Consumptive Way of By-product Hydrogen}
By-product hydrogen is a cost-competitive and widely distributed source of hydrogen.
The process of generation of by-product hydrogen and its consumptive ways are illustrated in Fig.\ref{fig:The process of generation of by-product hydrogen and its consumptive ways}.  
\begin{figure}[h] %可选参数 h t b p,代表允许图片出现的位置,h表示此处附近,t表示顶部,b表示底部,p表示单独一页,H表示固定此处
    \centering
    \includegraphics[width=8.5cm]{fig/Fig.1_The_process_of_generation_of_by-product_hydrogen_and_its_consumptive_ways.png}
    \caption{The process of by-product hydrogen generation and its consumptive ways}\label{fig:The process of generation of by-product hydrogen and its consumptive ways}
\end{figure}

Electrochemical processes, such as the industrial production of steel, caustic soda and chlorine, produce hydrogen as a by-product, burnt or emitted as the current practice. However, they can be made available for applications outside chemical plants as a future consumptive way. To transport products from the production facilities to storage sites, by-product hydrogen should be compressed or liquified in advance, which collectively 
are referred to as “hydrogen secondary processing”. Two common transportation modes are compressed gaseous hydrogen via tube trailers (CH2) and liquid hydrogen via tanker trucks (LH2). To alleviate the imbalance between supply and demand of hydrogen, underground cavities like salt caverns are potential to offer natural infrastructure to realize cost-effective and reliable hydrogen storage. At the last link in the supply chain, by-product hydrogen is sold and distributed to various end-users. The proposed generation, storage and consumptive way of hydrogen give rise to a promising by-product hydrogen business model consisting of chemical plants as suppliers, a salt cavern as a retailer and end-users as consumers. 

\subsection{Characteristics of By-product Hydrogen Supply Chain} \label{subsection: Characteristics of by-product hydrogen supply chain}

Differences between the by-product hydrogen supply chain under the proposed business model and most hydrogen supply chains found in literature can be mainly concluded as twofold: 1) composition of major costs; 2) flexibility to coordinate between planning and scheduling. These differences will lead to a distinct focus and a smaller timescale for the formulation of the by-product hydrogen supply chain, which is analyzed as follows:
% Please add the following required packages to your document preamble:
% \usepackage{multirow}
\begin{table}[h]
\centering
\caption{Major costs of hydrogen supply chain} \label{tab:Major costs of hydrogen supply chain}
\footnotesize
\begin{tabular}{cccc}
\hline\toprule
\multicolumn{2}{c}{\multirow{2}{*}{Major costs}}                                              & \multicolumn{2}{c}{Hydrogen Supply Chain}                        \\ \cline{3-4} 
\multicolumn{2}{c}{}                                                                          & \multicolumn{1}{l}{Traditional} & \multicolumn{1}{l}{By-product} \\ \hline
\multirow{2}{*}{Production}     & Investment & \checkmark &  \\
                                & Operation  & \checkmark &  \\ \hline
\multirow{2}{*}{Storage}        & Investment & \checkmark &  \\
                                & Operation  & \checkmark &  \\ \hline
\multirow{2}{*}{Transportation} & Investment & \checkmark  & \checkmark  \\
                                & Operation  & \checkmark & \checkmark \\ \hline
\multirow{2}{*}{\begin{tabular}[c]{@{}c@{}}Secondary \\ processing\end{tabular}} & Investment &                               & \checkmark                             \\
                                & Operation  &  & \checkmark \\ \hline
\end{tabular}
\end{table}
% Please add the following required packages to your document preamble:
% \usepackage{multirow}
\begin{table}[h]
\centering
\caption{Major costs and the influence factors}
\footnotesize
\label{tab:Major costs and the influence factors}
% \footnotesize
\begin{tabular}{cll}
\hline\toprule
\multirow{2}{*}{Major costs} &
  \multicolumn{2}{c}{\multirow{2}{*}{Influence factors}} \\
                                & \multicolumn{2}{c}{}                         \\ \hline
\multirow{2}{*}{Production}     & \multicolumn{2}{l}{1) Production technology} \\
                                & \multicolumn{2}{l}{2) Scale of production}   \\ \hline
\multirow{2}{*}{Storage}        & \multicolumn{2}{l}{1) Storage technology}    \\
                                & \multicolumn{2}{l}{2) Storage capacity}       \\ \hline
\multirow{3}{*}{Transportation} & \multicolumn{2}{l}{1) Transportation mode}   \\
 &
  \multicolumn{2}{l}{\multirow{2}{*}{\begin{tabular}[c]{@{}l@{}}2) Hydrogen volume\\ 3) Transport distance\end{tabular}}} \\
                                & \multicolumn{2}{l}{}                         \\ \hline
\multirow{3}{*}{\begin{tabular}[c]{@{}c@{}}Secondary \\ processing\end{tabular}} &
  \multicolumn{2}{l}{1) Type of processing equipment} \\
 &
  \multicolumn{2}{l}{\multirow{2}{*}{\begin{tabular}[c]{@{}l@{}}2) TOU electricity price\\ 3) Hydrogen volume\end{tabular}}} \\
                                & \multicolumn{2}{l}{}                         \\ \hline
\end{tabular}
\end{table}

\subsubsection{Different composition of major costs}
Major costs of hydrogen supply chain and their influence factors are demonstrated in Table \ref{tab:Major costs of hydrogen supply chain} and \ref{tab:Major costs and the influence factors}, respectively. Unlike the present hydrogen supply chain, producers in the by-product hydrogen supply chain benefit from very low-cost generation. Thus, the major cost comes from secondary processing and transportation.

Power is the major cost for secondary processing. If the liquefier or compressor operates at low-price periods, it may potentially reduce operating costs. Since electricity price fluctuates by hours, the strategic behaviors of each stakeholder should also be modeled by hour.

Transport cost is determined by transportation mode, hydrogen volume and the transport distance. For two transportation modes considered in this paper, LH2 features large transport capacity (often 10-20 times as CH2), high initial investment cost (several times as CH2) and hourly volatile losses. On the contrary, CH2 features low transport capacity, low initial investment cost and zero loss. Usually, for long-distance transportation of a large amount of hydrogen, CH2 is less economical since it requires long rides of much more vehicles than LH2. However, for mid- or short-distance of a small amount of hydrogen, CH2 is more economical since there is no volatile loss. Obviously, a reasonable decision of transportation mode would largely reduce the cost of each chemical plant.
\subsubsection{Less flexibility to coordinate between planning and scheduling }
For suppliers in the by-product hydrogen supply chain, the generation scale of hydrogen is limited by the production plan of their main products. Moreover, their location is less likely to be optimized for the transportation of by-product hydrogen.

Therefore, there may be a mismatch between each supplier's location and generation scale. Specifically, for distant (to the salt cavern) and medium-yield chemical plants, if CH2 is adopted, long-distance transport of more tube trailers may result in high transportation costs. Nevertheless, if LH2 is adopted, substantial volatile losses would happen due to hours of filling time. This situation results in a dilemma since both transportation mode leads to a revenue decline in some way. Therefore, we envision a scenario where several chemical plants in proximity to each other form a coalition and select a transit hub between them to lower transportation costs, instead of shipping individually to the salt cavern. Two examples of envisioned transportation routes are highlighted in color in Fig.\ref{fig:Possible routes for the salt cavern to acquire hydrogen from the chemical plants}. Moreover, to lower transportation costs, chemical plants destined for the transit hub adopt the CH2 transportation mode, while the transit hub destined for the salt cavern adopt the LH2 transportation mode. In this way, the dilemma between high transportation costs of CH2 and large volatile loss of LH2 is mitigated. 
\begin{figure}[h] %可选参数 h t b p,代表允许图片出现的位置,h表示此处附近,t表示顶部,b表示底部,p表示单独一页,H表示固定此处
    \centering
    \includegraphics[width=8cm]{fig/Fig.2_Possible_routes_for_the_salt_cavern_to_acquire_hydrogen_from_the_chemical_plants.png}
    \caption{Possible routes for the salt cavern to acquire hydrogen from the chemical plants} \label{fig:Possible routes for the salt cavern to acquire hydrogen from the chemical plants}
\end{figure}

To sum up, cost structure differences and the lack of flexibility to coordinate between production scale and location lead to a gap between the by-product hydrogen supply chain and the present ones. Therefore, it is essential to model the by-product hydrogen supply chain according to its characteristics rather than simply applying the model of the traditional hydrogen supply chain. 

\subsection{The Structure of By-product Hydrogen Market}

The structure of the proposed by-product hydrogen market is provided in this subsection, followed by the basic assumptions.

\begin{figure}[h] %可选参数 h t b p,代表允许图片出现的位置,h表示此处附近,t表示顶部,b表示底部,p表示单独一页,H表示固定此处
    \centering
    \includegraphics[width=8cm]{fig/Fig.3_The_structure_of_the_by-hydrogen_market_under_investigation.png}
    \caption{The structure of the by-hydrogen market under investigation} \label{fig:The structure of the by-product hydrogen market under investigation}
\end{figure}
The by-hydrogen market under the proposed business model has the structure illustrated in Fig.\ref{fig:The structure of the by-product hydrogen market under investigation}. Suppliers, namely chemical plants, process by-product hydrogen by liquefiers or compressors (depends on the decision results of each supplier) and deliver it to the retailers. The retailers, namely salt caverns, sell hydrogen to the customers. To simplify the problem, salt caverns are regarded as an entity owned by a single company.

This paper focuses on the transaction between suppliers and retailers. The following assumptions are made without loss of generality:

\begin{enumerate}
	\item The end-users buy all the hydrogen from the retailer at a fixed price. This may happen when the injection-production rate of the salt cavern is higher than the market demand in a region. In order to alleviate the supplier’s market power to drive up prices, we assume that the salt cavern and suppliers have reached such an
    agreement to bring a fixed price into effect. 
	\item The secondary processing cost and transport cost is undertaken by suppliers.  
	\item The production cost is neglected since hydrogen is a by-product of the industrial process of chemical plants. 
	\item Chemical plants would not adjust their production schedule of their main product for the revenue generated by by-product hydrogen.
\end{enumerate}

Based on the above assumptions, the retailer’s and suppliers’ problems can be described as follows. To maximize profits, the salt cavern intends to purchase as much hydrogen as possible from chemical plants at the lowest cost. If the price is too low, chemical plants are less likely to be attracted by this new revenue stream and may waste them as before, which reduces profits of the salt cavern. On the contrary, if the price is too high, the purchasing cost would increase. Therefore, it is important for the salt cavern to strike a balance between the attraction of chemical plants and the purchasing cost. To maximize profits, chemical plants upstream would like to sell more hydrogen when the selling price is high on the one hand, and to reduce processing costs and transport costs on the other hand.

Taking into account the analysis in the last subsection, the challenges of modeling the by-product hydrogen supply chain under the proposed structure are mainly twofold: 1) to explicitly consider possible coalition structures and transport route strategies in the timescale of transport duration, electricity price fluctuation and volatile losses; 2) and to allocate the payoff among the producers in some fairway.

\section{Strategies and decision-making process of stakeholders}

In this section, the decision-making process of each stakeholder is investigated and mathematically modeled under the proposed business model.

\subsection{The Retailer’s Problem}

In the price-setting problem of the salt cavern, the retailer decides its buying price $p_{t}$ (offered to the suppliers), while considering the reactions ${q_{i,t}^{trans}}$ from suppliers. The problem can be formulated as
\setlength{\abovedisplayskip}{3pt}
\begin{align}
    \max\limits_{p_{t}}\ p_{o}\sum_{i=1}^{I}\sum_{t=1}^{T}q_{i,t}^{trans}u_{i,I+1}-\sum_{i=1}^{I}\sum_{t=1}^{T}p_{t}q_{i,t}^{trans}u_{i,I+1}  \label{eq:constraint1}
\end{align}
\begin{align}
    s.t.\ \underline{p}_{t}\le p_{t}\le \overline{p}_{t},\forall t\label{eq:constraint2}
\end{align}
\begin{align}
    \sum\limits_{i=1}^{I}q_{i,t-T_{a}^{i,I+1}}^{trans}\le Q_{trans},\forall t\label{eq:constraint3}
\end{align}

Objective \eqref{eq:constraint1} is the retailer’s profit in which the first term is the selling income, and the second term is the purchasing cost. Inequality \eqref{eq:constraint2} restricts the price offered to suppliers to be within the interval $[\underline{p}_{t},\overline{p}_{t}]$ in each period. Here we assume that the retailer and suppliers have already reached an agreement to bring this constraint into effect. Inequality \eqref{eq:constraint3} prescribes maximal transaction quantity in each period by maximal injection rate of the salt cavern. $q_{i,t}^{trans}$ and $u_{i,I+1}$ are the optimal solution to the suppliers’ problem.

\subsection{The Suppliers’ Problem}

For the suppliers, the planning of the type of processing equipment, transportation mode, and the transport route as well as scheduling of transaction quantity, is formulated in this subsection. To capture the dynamic process of hydrogen transactions between each stakeholder in detail, as well as investigating dynamic strategic behaviors of each stakeholder, the loading process is elaborately taken into consideration. Specifically, hydrogen is produced as a by-product along with main products and has three possible disposal ways: 
\begin{enumerate}
    \item Hydrogen can be loaded to a tube trailer (or a tanker truck) after compression (or liquefaction). At the end of period $t$, tube trailers (or tanker trucks) filled to maximum capacity should depart from chemical plants. Otherwise, they stay until filled up in the following periods. Therefore, the transaction quantity sequence $q_{i,t}^{trans}$ depends on hydrogen processing quantity sequence $q_{i,t}^{pr}$ and capacity of the vehicle ($Q_\mathcal{C}$ for a tube trailer and $Q_\mathcal{D}$ for a tanker truck).
    \item Hydrogen can also be temporarily stored in low-pressure storage tanks before liquefication or achieving an adequate compression rate. It will further be loaded into tube trailers (or tanker trucks) after being compressed (or liquified) in the following periods.
    \item Hydrogen may also be discarded by being emitted or burnt as the current practice, which may happen when buying price offered by the salt cavern is too low or low-pressure storage tanks are filled up.
\end{enumerate}

The above three disposal ways offer multiple options for chemical plants during planning and scheduling. For example, a chemical plant with a generation volume of 100kg per hour, may purchase processing equipment of 100kg per hour. Thus, hydrogen can be processed hour-by-hour. An alternative is to purchase processing equipment of 1000kg per hour. In this case, by-product hydrogen can be temporarily stored in low-pressure storage tanks and will be processed every 10 hours. The suppliers’ problem is to find optimal solutions for planning and scheduling while considering possible coalitions with each other.

In the suppliers’ problem, if the destinations of all suppliers for hydrogen shipment are the salt cavern, decision variables should be the type of processing equipment $\boldsymbol{x_{i}}$ and hydrogen processing amount $q_{i,t}^{pr}$; if the scenario of coalitions of suppliers is taken into account, transport route $u_{i,j}$ of chemical plants $i$ and $j$, which form a coalition.

The decision-making problem, including constraints and objectives of supplier $i$, is given as follows.
\subsubsection{Constraints on transit shipment pattern}
\begin{gather}
    \sum\limits_{j=1}^{I+1}u_{i,j}=1,\forall i\label{eq:constraint4}\\
    u_{i,j}+u_{j,i}\le 1,\forall i,j \label{eq:constraint5}
\end{gather}

Constraint \eqref{eq:constraint4} denotes that the destination of each chemical plant is unique. Constraint \eqref{eq:constraint5} defines that any pairs of the chemical plant $(i,j)$ wouldn’t select each other as the transit destination simultaneously.

\subsubsection{Constraints on hydrogen processing and transport scheduling}

Chemical plants adopting CH2 satisfy: 
\begin{gather}
    n_{i,t}^{cars}\le(q_{i,t}^{pr}+q_{i,t-1}^{store})/Q_c^{car}\le n_{i,t}^{cars}+1,\forall t \label{eq:constraint6}\\
    q_{i,t}^{trans}=n_{i,t}^{cars}Q_\mathcal{C},\forall t \label{eq:constraint7}\\
    q_{i,t}^{store}=q_{i,t-1}^{store}+q_{i,t}^{pr}-q_{i,t}^{trans}, \forall t\in\{2,...T\} \label{eq:constraint8}
\end{gather}

Constraints \eqref{eq:constraint6} and \eqref{eq:constraint7} indicate that hydrogen transaction amount in each period is an integer multiple of the capacity of a tube trailer since only tube trailers filled to maximum capacity will depart from chemical plants. Constraint \eqref{eq:constraint8} denotes variations of hydrogen quantity stored in low-pressure storage tanks.

With the remaining proportion of hydrogen after being shipped from chemical plant $i$ to $j$ ($j= I+1$ represents the salt cavern) written as $\beta_{L2}^{i,j}=\beta_{L2} T_{a}^{i,j}$, chemical plants adopting LH2 satisfy
\begin{gather}
    n_{i,t}^{cars}\le(q_{i,t}^{pr}+\beta_{L1}q_{i,t-1}^{store})/Q_d^{car}\le n_{i,t}^{cars}+1,\forall t \label{eq:constraint9}\\  
    q_{i,t}^{trans}=n_{i,t}^{cars}Q_\mathcal{D}\sum_{j=1}^{I+1}u_{i,j}\beta_{L2}^{i,j},\forall t \label{eq:constraint10}
\end{gather}
\setlength{\abovedisplayskip}{-10pt}
\begin{multline}
    q_{i,t}^{store}=\beta_{L1}q_{i,t-1}^{store}+q_{i,t}^{pr}-{q_{i,t}^{trans}}/{\sum_{j=1}^{I+1}u_{i,j}\beta_{L2}^{i,j}},\\ \forall t \in \{2,...T\} \label{eq:constraint11}
\end{multline}

Constraints \eqref{eq:constraint9} and \eqref{eq:constraint10} indicate that the hydrogen transaction amount in each period is an integer multiple of the capacity of a tanker truck. Constraint \eqref{eq:constraint11} denotes variations of hydrogen quantity stored in low-pressure storage tanks.

Constraints irrelevant to transportation modes are given in \eqref{eq:constraint12}-\eqref{eq:constraint15}, in which the transport duration for chemical plant $i$ is written as $t_{ar}^{i}=\sum_{j=1}^{I+1}u_{i,j}T_{a}^{i,j}$.
\setlength{\abovedisplayskip}{3pt}
\begin{gather}
{\sum_{t - 2\times t_{ar}^{i}}^{t}n_{i,t}^{cars}} \leq N_{i}^{cars}, \forall t \in \left\{2\times t_{ar}^{i},...T \right\} \label{eq:constraint12} \\
q_{i,t}^{pr} \leq {\sum_{n = 1}^{N_{\mathcal{C}} + N_{\mathcal{D}}}x_{i}^{n}}Q_{type}^{n}, \forall t \label{eq:constraint13} \\
q_{i,t}^{unpr} \leq \sum_{n = 1}^{N_{\mathcal{C}} + N_{\mathcal{D}}} x_{i}^{n}Q_{type}^{n}, \forall t \label{eq:constraint14}
\end{gather}
\setlength{\abovedisplayskip}{-3pt}
\begin{multline}
q_{i,t}^{unpr} \leq q_{i,t - 1}^{unprocess} + Q_{i,t} - q_{i,t}^{pr} + {\sum_{j = 1}^{I}{u_{j,i}q_{j,trans}^{t - T_{a}^{i,j}}}},\\\forall t \in \left\{ \max{({1,T_{a}^{i,j}})},\ldots,T \}\right. \label{eq:constraint15} 
\end{multline}
\setlength{\belowdisplayskip}{5pt}

Constraint \eqref{eq:constraint12} imposes the total number of tube trailers (or tanker trucks) purchased by chemical plant $i$ as the upper bound of tube trailers (or tanker trucks) in the round trip during the time period $\left\lbrack t - 2\times t_{ar}^{i} \right\rbrack$. Constraint \eqref{eq:constraint13} prescribes the processing capability of each chemical plant. Constraint \eqref{eq:constraint14} restricts the upper bound of hydrogen stored locally, and the bound parameter is chosen as $\sum_{n=1}^{N_\mathcal{C}+N_\mathcal{D}}x_i^nQ_{type}^n$. Constraint \eqref{eq:constraint15} represents variations of hydrogen stored locally, in which `$\le$' indicates that hydrogen as a by-product can be stored temporarily or directly discarded.

If destinations of all suppliers for hydrogen shipment are the salt cavern, the objective of each chemical plant is to maximize its daily profit and is given in \eqref{eq:constraint16}, in which the income by selling hydrogen to the consumers, initial investment cost and operation cost are considered.
\begin{gather}
\max~\pi_{Fi} = {\sum_{t = 1}^{T}( p_{t}q_{i,t}^{trans} - C_{O}^{i} - C_{T}^{i} )} - C_{INV1}^{i} - C_{INV2}^{i} \label{eq:constraint16} 
\end{gather}
where
\begin{gather}
% \setlength{\belowdisplayskip}{8pt}
C_{O}^{i} = q_{i,t}^{pr}{\sum\limits_{n = 1}^{N_{\mathcal{C}} + N_{\mathcal{D}}}x_{i}^{n}}w_{t}\left( \gamma_{c}x_{i}^{c} + \gamma_{d}x_{i}^{d} \right) \label{eq:constraint17}\\
C_{T}^{i} = n_{i,t}^{cars}{\sum_{j = 1}^{I + 1}{u_{i,j}{K_{3}T}_{a}^{i,j}}} \label{eq:constraint18}\\
C_{INV1}^{i} = {\sum_{n = 1}^{N_{\mathcal{C}} + N_{\mathcal{D}}}{x_{i}^{n}K_{1}^{n}}} \label{eq:constraint19}\\
C_{INV2}^{i} = N_{i}^{cars}{({x_{i}^{c}K_{2}^{c} + x_{i}^{d}K_{2}^{d}})} \label{eq:constraint20}
\end{gather}
where transportation mode is written as $x_i^c=\sum_{n=1}^{N_\mathcal{C}}x_i^n$ and $x_i^d=\sum_{n=N_\mathcal{C}}^{N_\mathcal{C}+N_\mathcal{D}}x_i^n$; $C_O^i , C_T^i$ represent hourly processing and transport cost respectively; $C_{INV1}^i$ , $C_{INV2}^i$ represent investment cost of processing equipment and tube trailers (or tanker trucks) after converted into daily cost with a discount rate, respectively.
If the scenario where coalitions of suppliers are considered, we denote chemical plants in a coalition as $\Gamma$. For the chemical plant $i$, $\forall i\in\Gamma$, the objective is to maximize the daily profit of the coalition and is given as \eqref{eq:constraint21}.
\setlength{\belowdisplayskip}{6pt}
\begin{multline}
\max~~\pi_{F\tau}=\sum_{i \in \Gamma}{{\sum_{t = 1}^{T}\left( p_{t}q_{i,t}^{trans}u_{i,I + 1} - C_{O}^{i} - C_{T}^{i} \right)}} \\{ - C_{INV1}^{i} - C_{INV2}^{i}} \label{eq:constraint21}
\end{multline}
where $u_{i,I+1}=1$ when chemical plant $i$ is chosen as a transit hub. Otherwise $u_{i,I+1}=0$.


\section{Game formulation and solution}
\subsection{Game Formulation for By-product Hydrogen Supply Chain}

In this section, the by-product hydrogen market is formulated as a game, considering the individual rationality of each stakeholder. 

The decision-making process of each individual can be concluded as follows. The suppliers plan their initial equipment investment, coalition structure and transport routes in the planning stage. Then, the hydrogen transaction problem, including the retailer's pricing problem and suppliers' scheduling problem, is optimized in the scheduling stage.

The overall framework of the game models is illustrated in Fig.\ref{fig:Game models involved in by-product hydrogen supply chain including salt cavern and chemical plants}. Specifically, the planning problem of multiple chemical plants is formulated as a cooperative game, in which a binding coalition could be formed to reduce transport costs. The hydrogen transaction problem between the salt cavern and chemical plants is formulated as a Stackelberg game, in which the salt cavern is the leader and chemical plants are the followers.

\begin{figure}[] %可选参数 h t b p,代表允许图片出现的位置,h表示此处附近,t表示顶部,b表示底部,p表示单独一页,H表示固定此处
    \centering
    \includegraphics[width=8.5cm]{fig/Fig.4_Game_models_involved_in_by-product_hydrogen_supply_chain_including_salt_cavern_and_chemical_plants.png}
    \caption{Game models involved in by-product hydrogen supply chain including salt cavern and chemical plants} \label{fig:Game models involved in by-product hydrogen supply chain including salt cavern and chemical plants}
\end{figure}
\subsubsection{Coorperative game in the planning stage} \label{subsubsection: first-stage problem}
As previously analyzed in subsection \ref{subsection: Characteristics of by-product hydrogen supply chain}, coalitions between chemical plants would potentially lower transportation costs, thus bringing collective payoffs. Moreover, to fairly allocate the payoff $\pi_{F\tau}$ among the players, the Shapley value is adopted.

The following assumptions are made without loss of generality when considering possible coalition structures:

i) Chemical plants in each coalition select one of them as a transit hub to which other chemical plants in the coalition transport hydrogen. Since reducing transport costs is considered as the key factor behind the coalition, we assume that two chemical plants destined for the salt cavern lack the motivation to form a coalition.

ii)	The influence of hydrogen price variations on the coalition structure is neglected since the salt cavern's buying price is unknown at the planning stage. Moreover, the driving force in forming a coalition is to reduce costs rather than to increase the selling income.

Generally, the planning problem of chemical plants is based on the cooperative game, where players are the chemical plants. For chemical plant $i$, decision variables are the type of processing equipment, $\textit{\textbf{x}}_\textit{\textbf{i}}=\left\{x_i^n\right\},\forall n$, hydrogen processing amount $q_{i,t}^{pr}$ and transport route $u_{i,j}$ of chemical plants in the coalition. Payoffs are described as \eqref{eq:constraint16} and \eqref{eq:constraint21} for self-sufficient chemical plants and coalitions respectively.

Note that in the planning stage, the optimal solution $q_{i,t}^{pr}$ is to roughly estimate operation cost under different transportation mode and processing equipment type decisions, thus helping the decision of transport route $u_{i,j}$. Therefore, the solution of $q_{i,t}^{pr}$ here neglects the influence of hydrogen price variations. Actual hydrogen processing quantity sequence $q_{i,t}^{pr}$ will be obtained by equilibrium analysis in the scheduling stage.

\subsubsection{Stackelberg game in the scheduling stage}
The problem in the scheduling is the hydrogen transaction problem between the retailer and the suppliers. After formulating transport route decisions of suppliers as a cooperative game, the interaction between the salt cavern and multiple chemical plants is formulated as a Stackelberg game, where the salt cavern is the leader, whose strategy is the TOU hydrogen price, and chemical plants are followers, whose strategies are hourly transaction. 

At this stage, the retailer’s and suppliers’ problem can be formulated as a bilevel optimization. The retailer determines the hydrogen price sequence $v_t$ in the upper level, and the suppliers decide their optimal transaction pattern $q_{i,t}^{trans}$ in the lower level, with respect to the hydrogen price sequence $v_t$. The optimal transaction pattern $q_{i,t}^{trans}$ would in turn influences hydrogen price sequence $v_t$ determined by the retailer in the upper level. Assume that the information of each chemical plant, such as transit transport routes, processing equipment type and by-product hydrogen generation quantities, are accessible to the salt cavern. Therefore, the optimal solution of $q_{i,t}^{trans}$ can be predicted by the salt cavern under any given hydrogen price sequence $v_t$. The suppliers’ dispatching problem \eqref{eq:constraint3}-\eqref{eq:constraint21} can be regarded as constraints of the retailer’s pricing problem.

According to the analysis above, the interactions between the salt cavern and the chemical plants constitute a Stackelberg competition. In this competition, the salt cavern is the leader, whose strategy is the TOU hydrogen price sequence. Chemical plants are the followers, whose strategy is the hourly hydrogen transaction quantity. The leader’s pricing problem maximizes its profit, subject to the bounds of hydrogen price (Eq.\eqref{eq:constraint2}) and maximal injection rate (Eq.\eqref{eq:constraint3}). The followers’ scheduling problem maximizes individual profits or coalition profits, subject to constraints given in \eqref{eq:constraint9}-\eqref{eq:constraint18}.
\subsection{Solution of the Problem}
In this section, we introduce the solution of the game formulation of the by-product hydrogen supply chain. 

Tractable reformulations of the suppliers’ problem are made to efficiently calculate the equilibrium in the lower level for both the planning and scheduling problems. Specifically, for the suppliers’ problem in both stages, the objective of each individual player (or coalition) is irrelevant to the strategies of other individual players (or coalitions), while the strategy set is influenced by the strategies of other individual players (or coalitions). According to the potential game theory, the suppliers' problem can be regarded as a potential game. The sum of the objectives of each individual player (or coalition) is the potential function. Besides, the pure-strategy equilibrium exists in the transport route planning problem of the suppliers since there exists at least one pure-strategy equilibrium in an infinite potential game. Thus, the suppliers’ problem is formulated as a potential game that can be solved as an optimization problem.

After the reformulation of the suppliers' problem, the planning stage problem is reformulated to a mixed integer nonlinear program (MINLP) with ${\textit{\textbf{x}}_\textit{\textbf{i}},u_{i,j}, q_{i,t}^{pr},N_i^{cars}},\forall i\in{1,\ldots I}$ as decision variables, \eqref{eq:constraint22} as the objective and \eqref{eq:constraint4}-\eqref{eq:constraint15} as constraints. Commercial solvers such as Baron can be used to solve the problem. The solved optimal strategy $\textit{\textbf{x}}_\textit{\textbf{i}}$ and $u_{i,j}$ will be adopted at the scheduling stage.
\setlength{\abovedisplayskip}{3pt}
\begin{multline}
\max~~\pi_{F}~ = ~\sum_{i=1}^{I}{{\sum_{t = 1}^{T}\left( p_{t}q_{i,t}^{trans}u_{i,I + 1} - C_{O}^{i} - C_{T}^{i} \right)}}\\{-C_{INV1}^{i} - C_{INV2}^{i}}  \label{eq:constraint22}
\end{multline}

To solve the bi-level problem at the scheduling stage, Genetic Algorithm (GA) is adopted. First, for the salt cavern in the upper level, pieces of hydrogen price sequences are generated and regarded as individuals. Second, to acquire the fitness of each individual, the suppliers’ scheduling problems in the lower level are solved. Since the transit transport routes $\textit{\textbf{x}}_\textit{\textbf{i}}$ and the processing equipment type $u_{i,j}$ are known at the scheduling stage, the suppliers’ problem becomes a mixed-integer linear program (MILP), which can be solved efficiently by off-the-shelf commercial solvers. Thus, daily profits of the salt cavern, considering the best response of the suppliers, can thus be calculated and regarded as finesses for given price sequences. 

\section{Case Study}
To validate the effectiveness of the proposed model and algorithm, numeric experiments on a by-product hydrogen supply chain composed of three chemical plants and a salt cavern are carried out. All of the following tests are conducted on PCs with Intel Xeon W-2255 processor, 3.70 GHz primary frequency, and 128GB memory. CPLEX 2.16 is used to solve related MILP problems.

\subsection{System Configuration}
Scenario parameters of the envisioned by-product hydrogen supply chain are given in Table \ref{tab:Scenario parameters}. $Q_{i,t}$ are hydrogen generation sequences of a typical day produced by a Gaussian distribution with a mean value of 1000 for the 1st chemical plant (1500 for the 2nd and 3000 for the 3rd) and a variance of 100. Moreover, in the envisioned by-product hydrogen supply chain, $\boldsymbol{Q_{pr}}$ are a vector consisting of 1200, 2000, 4000 and 8000, the first two and the last two of which are the compressor capacity and liquefier capacity to choose from, respectively. Parameters of different processing equipment and transportation modes refer to \cite{HAN20125328} and \cite{Argonne2021} and are given in Table \ref{tab:Parameters of hydrogen transportation}. $\boldsymbol{K_{1}}$ are a vector consisting of 774.29, 126612, 18977.17 and 34757.99, corresponding to each element in $\boldsymbol{Q_{pr}}$. Note that the time scale involved in the problem is one day. Initial investment costs of the liquefier, the compressor, and the transportation vehicles are converted into daily investment costs with a discount rate. The operation cost of a tube trailer (or a tanker truck) in each period includes fuel price, driver wage, and maintenance expenses. 

\begin{table}[h]
\centering
\caption{Scenario parameters of the by-product hydrogen supply chain}
% \captionsetup{font={footnotesize}}
% \resizebox{0.4\textwidth}{!}{
\label{tab:Scenario parameters}
\footnotesize
\begin{tabular}{lllll}
\hline\toprule
\multicolumn{5}{l}{Parameters}                                                           \\ \hline
$I$                      & \multicolumn{2}{l}{3}  & $N_\mathcal{D}$     & 2                             \\
$T$                      & \multicolumn{2}{l}{12} & $T_{a}$     & {[}0,0,0,4;0,0,0,4;0,0,0,4{]} \\
\multicolumn{1}{c}{$p_{o}$} & \multicolumn{2}{l}{15} & $\underline{p}_{t},\overline{p}_{t}$  & 5/13                          \\
$N_\mathcal{C}$                     & \multicolumn{2}{l}{2}  & $Q_{trans}$ & 9000                          \\ \hline
\end{tabular}
% }
\end{table}

\begin{table}[]
\centering
\caption{Parameters of hydrogen transportation} \label{tab:Parameters of hydrogen transportation}
\footnotesize
\begin{tabular}{lllllll}
\hline\toprule
\multicolumn{7}{l}{Parameters}                                                                             \\ \hline
$Q_\mathcal{C}$                        & \multicolumn{4}{l}{200}          & $K_{3}(\$/h)$       & {[}0,0,0,4;0,0,0,4;0,0,0,4{]} \\
$Q_\mathcal{D}$                        & \multicolumn{4}{l}{4000}         & $\beta_{L1}$ & 5/13                          \\
\multicolumn{1}{c}{$\gamma_{c}/\gamma_{d}(kwh/kg)$} & \multicolumn{4}{l}{1/8.18}       & $\beta_{L2}$ & 9000                          \\
$K_{2}^c/K_{2}^d(\$)$                      & \multicolumn{4}{l}{82.20/219.18} &          & \multicolumn{1}{c}{}          \\ \hline
\end{tabular}
\end{table}
\subsubsection{Equilibrium of possible coalition structures of the suppliers}
With three chemical plants, there are five possible coalition structures: no cooperation, cooperation between two players with the third being self-sufficient (there are three ways this could occur) and complete cooperation among all the three chemical plants. The benefits of individual participants or coalitions are shown in Table \ref{tab:Participants/alliance}, in which $M$ represents the benefit, and the benefit of each chemical plant and the sum of them are denoted by $M_{1},M_{2},M_{3}$ and $M_{total}$ respectively. ‘\{\}’ indicates a cooperation, and the chemical plant serving as the transit hub is marked by a ‘*’. 

% Please add the following required packages to your document preamble:
% \usepackage{multirow}
\begin{table}[]
\centering
\caption{Participants/alliance optimal income under non-cooperative and cooperative game models} \label{tab:Participants/alliance}
\footnotesize
\begin{tabular}{llll}
\hline\toprule
\multirow{2}{*}{Number} &
  \multirow{2}{*}{\begin{tabular}[c]{@{}l@{}}Coalition\\ structure\end{tabular}} &
  \multicolumn{2}{c}{Profits(\$/day)} \\ \cline{3-4} 
  &                & \begin{tabular}[c]{@{}l@{}}Individual or\\ a coalition\end{tabular}        & $M_{total}$ \\ \hline
1 &
  \{1\},\{2\},\{3\} &
  \begin{tabular}[c]{@{}l@{}}$M_{1} = 54052$\\ $M_{2} = 81060$\\ $M_{3} = 236814$\end{tabular} &
  371926 \\
2 & \{1,2*\},\{3\} & \begin{tabular}[c]{@{}l@{}}$M_{\{1,2\}}$ = 170589\\ $M_3$ = 236814\end{tabular} & 407403   \\
3 & \{1,3*\},\{2\} & \begin{tabular}[c]{@{}l@{}}$M_{\{1,3\}}$ = 286531\\ $M_2$ = 107868\end{tabular} & 394399   \\
4 & \{1\},\{2,3*\} & \begin{tabular}[c]{@{}l@{}}$M_1$ = 53562\\ $M_{\{2,3\}}$ = 323154\end{tabular}  & 376716   \\
5 & \{1,2,3*\}     & $M_{\{1,2,3\}}$ = 383925                                                       & 383925   \\ \hline
\end{tabular}
\end{table}

It can be analyzed from Table \ref{tab:Participants/alliance} that:

i)	In the 1st coalition structure with no cooperation at all, the total benefit of the three chemical plants is the lowest among all coalition structures, indicating a potential collective payoff gained by forming coalitions between chemical plants. 

ii)  In the 3rd coalition structure, the benefit of the coalition $\{1, 3^{*}\}$ denoted as $M_{\{1,3^{*}\}}$ equals to 286531 and is lower than the sum of benefits that they could get on their own, which is calculated as $M_{1}+M_{3}=290866$, violating collective rationality.

iii)  In the 5th coalition structure, although collective benefit is higher than the sum of benefits each coalition member could get on their own, the total benefit of the 5th coalition structure $M_{total}\{1,2,3^{*}\}$ is lower than that of the 2nd coalition structure $M_{total}(\{1,2^{*}\},\{3\})$. Therefore, the grand coalition is not stable since there is a preferred alternative. The analysis of the 4th coalition structure is analogous.

iv)  In the 2nd coalition structure, $M_{\{1,2^{*}\}}$, the benefit of the coalition $\{1,2^{*}\}$, equals to 170589 and is higher than the sum of benefits they could get on their own, which satisfies $M_{1}+M_{2}=135112$. Moreover, the total benefit of the 2nd coalition structure is the highest among the five possible structures, so there exists no preferred alternatives. Therefore, the coalition of the chemical plants $\{1,2^{*}\}$ is stable.

The insights provided by different coalition structures above is that for several chemical plants in proximity to each other, those chemical plants with low or medium generation scale (chemical plant 1 and 2 in our case) tends to form a coalition, and to compete with those with larger generation scale.

In order to realize a fair imputation of the collective payoff of chemical plants $\{1,2^{*}\}$, the Shapley value is adopted. The allocation result is {71790.5,98798.5}\$, which is higher than the benefit they could get on their own, which are \{\$54052, \$81060\}. The coalition between chemical plant 1 and 2 increase their profits by 24.7\% and 18.0\% respectively.

\subsection{Equilibrium of Hydrogen Pricing and Scheduling}

In this case, the fixed price at which consumers purchase is set as 15 \$/kg. The equilibrium of the buying price offered by the salt cavern $p_t$ and the hydrogen transaction quantity $q_{i,t}^{pr}$ are illustrated in Fig.5. The minimal price takes value at its lower bound 5\$/kg, and the maximal value is 11.9 \$/kg .
\begin{figure}[] %可选参数 h t b p,代表允许图片出现的位置,h表示此处附近,t表示顶部,b表示底部,p表示单独一页,H表示固定此处
    \centering
    \includegraphics[width=7cm]{fig/Fig_5._Hydrogen_price_of_salt_cavern_and_transaction_quantity_of_chemical_plant.png}
    \caption{Hydrogen price of salt cavern and transaction quantity of chemical plant} \label{fig:Hydrogen price of salt cavern and transaction quantity of chemical plant}
\end{figure}
It can be observed from Fig.\ref{fig:Hydrogen price of salt cavern and transaction quantity of chemical plant} that the variation trend of the hydrogen transaction quantity goes with the buying price. The higher the buying price, the higher the transaction quantity. This can be attributed to the storage capacity of chemical plants, which can temporarily store by-product hydrogen in low-pressure storage tanks or tube trailers (or tanker trucks) before filled to maximal capacity. Therefore, the chemical plants can choose to sell hydrogen at a higher price.

Moreover, due to the influence of the TOU electricity price, the operating cost of the processing equipment fluctuates. The TOU electricity price and the equilibrium of the total processing quantity are plotted in Fig.\ref{fig:Time of use electricity price and processing mass of chemical plant}.

\begin{figure}[] %可选参数 h t b p,代表允许图片出现的位置,h表示此处附近,t表示顶部,b表示底部,p表示单独一页,H表示固定此处
    \centering
    \includegraphics[width=7.5cm]{fig/Fig_6._Time_of_use_electricity_price_and_processing_mass_of_chemical_plant.png}
    \caption{Time of use electricity price and processing mass of chemical plant} \label{fig:Time of use electricity price and processing mass of chemical plant}
\end{figure}

It can be observed from Fig.\ref{fig:Time of use electricity price and processing mass of chemical plant} that the variation trend of the hydrogen processing quantity and the TOU electricity price go oppositely. This is because chemical plants tend to process hydrogen when the electricity price is low, thus reducing the processing cost of hydrogen.

According to the above results, it can be noted that the equilibrium of salt cave pricing encourages chemical plants to process and trade hydrogen when the electricity price is lower. As a result, the salt cavern can purchase hydrogen with lower processing cost, thus reducing the purchase cost of hydrogen per unit. For chemical plants, the hydrogen price is higher during 1-2 periods after periods with lower electricity prices than in other periods, thus reducing the hydrogen processing cost.

The result of profits and total transaction quantities are plotted in Fig.\ref{fig:The result of profits and total transaction quantities with time-invariant hydrogen price} considering different fixed prices. The optimal price offered by the salt cavern is about 9\$/kg, and its profit is \$287884.8 for a day. However, the profit of the salt cavern reaches to \$343947.16 at the optimal TOU hydrogen price. Hence, a TOU hydrogen price strategy for the salt cavern increases its profit by 19.5\%.

\begin{figure}[] %可选参数 h t b p,代表允许图片出现的位置,h表示此处附近,t表示顶部,b表示底部,p表示单独一页,H表示固定此处
    \centering
    \includegraphics[width=7.5cm]{fig/Fig.7_The_result_of_profits_and_total_transaction_quantities_with_time-invariant_hydrogen_price.png}
    \caption{The result of profits and total transaction quantities with time-invariant hydrogen price}\label{fig:The result of profits and total transaction quantities with time-invariant hydrogen price}
\end{figure}

Generally, the equilibrium of the Stackelberg game between the salt cavern and the chemical plants benefits all the players. It also indicates the positive response of the salt cavern and chemical plants to TOU electricity price, and reflects the role of chemical plants in peak shaving and valley filling, which benefits the safe and stable operation of power grid.

\subsection{Sensitivity Analysis}
\subsubsection{Impact of per period transportation operation cost}

The reduction in operation cost of a tube trailer (or a tanker truck) per period $K_{3}$ reduces the transport cost, thus bringing down the collective payoff brought by coalitions of chemical plants. Based on the first assumption in section \ref{subsubsection: first-stage problem}, each coalition must take one of them as a transit hub, and two chemical plants destined for the salt cavern lack the motivation to form a coalition. Consequently, the collective payoff declines as the transport cost reduces, until collective rationality no longer holds when the benefits of the coalition are less than the sum of benefits each individual could get on their own. As shown in Fig.\ref{fig:Impact of running cost of single vehicle of single period on the profit of chemical plant 1 and 2}, when $K_{3}$ decreases from \$390 to \$382, the sum of benefits of chemical plant 1 and 2 under the equilibrium of the 1st and 2nd coalition structure, denoted by $M_{total}^{\{1,2\}},M_{total}^{\{1\},\{2\}}$ respectively, gradually increases. 

\begin{figure}[] %可选参数 h t b p,代表允许图片出现的位置,h表示此处附近,t表示顶部,b表示底部,p表示单独一页,H表示固定此处
    \centering
    \includegraphics[width=7cm]{fig/Fig_8._Impact_of_running_cost_of_single_vehicle_of_single_period_on_the_profit_of_chemical_plant_1_and_2.png}
    \caption{The result of profits and total transaction quantities with time-invariant hydrogen price}\label{fig:Impact of running cost of single vehicle of single period on the profit of chemical plant 1 and 2}
\end{figure}

As illustrated in Fig.\ref{fig:Impact of running cost of single vehicle of single period on the profit of chemical plant 1 and 2}, the coalition benefit is more sensitive to $K_3$ than individual benefits. When $K_3$ decreases to about \$386, the coalition $\{1,2^{*}\}$ no longer bring additional benefits to individuals, resulting in a breakdown of the coalition. 
\subsubsection{Impact of maximal injection rate of the salt cavern}

The maximal injection rate $Q_{trans}$ of the salt cavern directly limits the total transaction quantity per period between the salt cavern and the chemical plants. Table \ref{tab:Individual income} demonstrates the impact of $Q_{trans}$ to the equilibrium of the second-stage problem.

\begin{table}[]
\centering
\caption{Individual income of the equilibrium under different maximum transportation quality of salt cavern gas pipeline in single period} \label{tab:Individual income}
\footnotesize
\begin{tabular}{lllll}
\hline\toprule
\multirow{3}{*}{$Q_{trans}$} &
  \multirow{3}{*}{$M_{total}^{\{1,2\}}$/kg} &
  \multirow{3}{*}{$M_{total}^{\{3\}}$/kg} &
  \multirow{3}{*}{\begin{tabular}[c]{@{}l@{}}$M_{total}$\\(chemical\\ plants)/\$\end{tabular}} &
  \multirow{3}{*}{\begin{tabular}[c]{@{}l@{}}$M_{total}$\\(the salt\\\ cavern)/\$\end{tabular}} \\
      &                              &          &          &           \\
      &                              &          &          &           \\ \hline
12000 & 23103.96                     & 21387.07 & 44491.03 & 342814.83 \\
9000  & \multicolumn{1}{c}{26203.97} & 22762.82 & 48966.80 & 325379.79 \\
6000  & 10528.29                     & 1139.11  & 11667.40 & 278396.59 \\ \hline
\end{tabular}
\end{table}

It can be analyzed from Table \ref{tab:Individual income} that $Q_{trans}$ has different impacts on the participants: the daily income of chemical plants does not necessarily increase with the increase of $Q_{trans}$, whereas the daily income of the salt cavern increases with the increase of $Q_{trans}$. Therefore, the salt cavern will be motivated to determine an appropriate $Q_{trans}$ according to the generation scale of by-product hydrogen of the chemical plants so as to increase individual benefits.

\section{Conclusion}
This paper proposes an equilibrium model of a by-product hydrogen market with the salt cavern as the retailer and chemical plants as the suppliers. A business model for large-scale storage to acquire by-product hydrogen from chemical plants and sell them to end-users is established for the first time. The decision-making process of each stakeholder, i.e., chemical plants and the salt cavern, is investigated and mathematically modeled considering different transportation modes, locations of chemical plants and TOU electricity price. To consider the individual rationality of each stakeholder, the by-product hydrogen market is formulated as games. The transport route planning problem between multiple chemical plants is formulated as a cooperative game. The hydrogen transaction problem between the salt cavern and chemical plants is formulated as a Stackelberg game. Numeric experiments on a by-product hydrogen supply chain composed of three chemical plants and a salt cavern are carried out. The results show that a coalition between chemical plants potentially increases their profits. Moreover, the adoption of TOU hydrogen price in a Stackelberg formulation also increases the profit of the salt cavern. The proposed business model and the optimization of the by-product hydrogen supply chain management not only presents a new revenue stream for both chemical plants and salt caverns but increases resource efficiency and accelerates energy conversion.





% \section*{Appendix A}
% \vspace{-0.2cm}
% \section*{Proof of Proposition 1}



% \section*{Appendix B}
% \vspace{-0.2cm}
% \section*{Proof of Proposition 2}



\bibliographystyle{IEEEtran}
\bibliography{ref}
\end{document}





\end{document}
