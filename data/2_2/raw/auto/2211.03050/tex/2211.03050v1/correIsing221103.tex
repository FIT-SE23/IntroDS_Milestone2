%% LyX 2.3.6 created this file.  For more info, see http://www.lyx.org/.
%% Do not edit unless you really know what you are doing.
\documentclass[twocolumn,pra,twocolumn,superscriptaddress,nofootinbib]{revtex4-1}
\usepackage[T1]{fontenc}
\usepackage[latin9]{inputenc}
\setcounter{secnumdepth}{3}
\synctex=-1
\usepackage{color}
\usepackage{amsmath}
\usepackage{amssymb}
\usepackage{graphicx}
\usepackage[unicode=true,pdfusetitle,
 bookmarks=true,bookmarksnumbered=false,bookmarksopen=false,
 breaklinks=false,pdfborder={0 0 1},backref=false,colorlinks=true]
 {hyperref}
\hypersetup{
 pdfborderstyle=,citecolor=blue,urlcolor=blue}

\makeatletter
%%%%%%%%%%%%%%%%%%%%%%%%%%%%%% User specified LaTeX commands.
%\usepackage{fdsymbol}
\usepackage{bbm}

\makeatother

\begin{document}
\title{The local relaxation and correlation production in the quantum Ising
model }
\author{Tai Kang }
\affiliation{Center for Quantum Technology Research, and Key Laboratory of Advanced
Optoelectronic Quantum Architecture and Measurements, School of Physics,
Beijing Institute of Technology, Beijing 100081, China}
\author{Sheng-Wen Li}
\email{lishengwen@bit.edu.cn}

\affiliation{Center for Quantum Technology Research, and Key Laboratory of Advanced
Optoelectronic Quantum Architecture and Measurements, School of Physics,
Beijing Institute of Technology, Beijing 100081, China}
\begin{abstract}
Isolated quantum systems follow the unitary evolution, which guarantees
the full many body state always keeps a constant entropy as its initial
one. Here we consider the local dynamics of a quantum Ising model
with a finite size. It turns out, for both strong and weak coupling
situations, the dynamics of local observables exhibits similar relaxation
behavior as the macroscopic thermodynamics, which is called the local
relaxation; after a certain typical time, the relaxation behavior
suddenly changes and appears random, which is referred as a recurrence.
Besides, we find that the total correlation entropy of this system
approximately exhibit a monotonic increasing envelope in both strong
and weak coupling cases, which is quite similar as the irreversible
entropy production in the standard macroscopic thermodynamics. 
\end{abstract}
\maketitle

\section{Introduction\label{sec:Introduction}}

A macroscopic thermodynamic system always tend to relax to the thermal
equilibrium state after a long enough time in spite of its initial
time. However, it is also known that an isolated quantum system with
a finite number of degrees of freedom (DoF) follows the unitary evolution,
as a result, the full many body system always keeps a constant entropy
as its initial state, which looks different from the above macroscopic
relaxation behavior, unless certain specific averaging approaches
are taken into consideration, e.g., based on time or random defect
configurations \cite{hobson_irreversibility_1966,prigogine_time_1978,mackey_dynamic_1989,uffink_compendium_2006,swendsen_explaining_2008}.

On the other hand, it is also noticed that, when focusing on the local
parts of a many body system, the dynamics naturally exhibits similar
relaxation behavior as the macroscopic thermodynamics. Although the
whole isolated system always follows the unitary evolution and keeps
a constant entropy, the dynamics of a local site exhibits an oscillating
decay behavior, which seems relaxing towards a certain steady state,
thus this is called the local relaxation \cite{cramer_exact_2008,li_hierarchy_2021,flesch_probing_2008,eisert_quantum_2015}. 

For a finite system size, after a certain time, the local relaxation
behavior would come across a ``recurrence'': the well ordered oscillatory
decay behavior suddenly changes and appears \textquotedblleft random\textquotedblright{}
\cite{cramer_exact_2008,li_hierarchy_2021,flesch_probing_2008,eisert_quantum_2015,hanggi_reaction-rate_1990,zwanzig_nonequilibrium_2001}.
With the increase of the DoF number in the bath, the recurrence appears
much later, thus it does not show up in practice. It is found that
the physical reason of such recurrences roots from the superposition
with the propagation regathered back due to the finite system size
\cite{li_hierarchy_2021}. 

In these previous studies, the interactions between the local sites
are focused on the exchanging type, which guarantees the total population
of the system is always conserved. In this paper, we consider the
local relaxation and recurrence in a quantum Ising model. When the
Ising coupling strength is quite weak, the dynamics of system approximately
reduces to the quantum \emph{XX} model due to rotating-wave approximation
(RWA). But in the strong coupling regime, the total population of
the system exhibit violent fluctuations, and the dynamical behavior
is also significantly different from the weak coupling case. Even
though, when measured in the unitless time $\tau\equiv Jt$ ($J$
is the Ising coupling strength), both the weak and strong coupling
situations exhibit quite similar local relaxation and recurrence behaviors,
except the the strong coupling case contains more high frequency oscillations. 

Besides, we consider the evolution of the total correlation entropy
in the $N$-body state \cite{watanabe_information_1960,groisman_quantum_2005,zhou_controllable_2008,anza_logarithmic_2020}.
It turns out, in both strong and weak coupling cases, the total correlation
entropy roughly exhibits a monotonic increasing behavior, which is
similar to the irreversible entropy increase in the standard thermodynamics
\cite{lebowitz_macroscopic_1993,li_production_2017,you_entropy_2018,li_correlation_2019}.
In contrast, the entropy of the whole isolated system always keeps
a constant, while the entropy of each local site increases and decreases
from time to time. Moreover, by adopting variations, we estimate the
possible maximum of the total correlation entropy under the constraint
determined by the initial state, which coincide quite well with the
numerical result calculated from the time dependent evolution. That
indicates the maximization of the correlation entropy is consistent
with the irreversible entropy production in the standard macroscopic
thermodynamics.

The paper is arranged as follows. In Sec. \ref{sec:Dynamics-in-the}
we study the dynamics of the Ising model. In Sec. \ref{sec:The-recurrence-behavior}
we analyze the recurrence behavior in the local dynamics with different
coupling strengths. In Sec. \ref{sec:Total-correlation-entropy} we
show the dynamics of the total correlation entropy in this system.
The discussion is drawn in Sec. \ref{sec:DISCUSSION}.

\section{Dynamics in the Ising model\label{sec:Dynamics-in-the}}

Here we consider the dynamical behavior of a many body system, which
contains $N$ two-level systems (TLSs). The $N$ TLSs interact with
the near neighbors via the Ising interaction, which is described by
the Hamiltonian \cite{sachdev_quantum_2011} : 
\begin{equation}
\hat{H}=\sum_{n=0}^{N-1}\frac{\omega}{2}\,\hat{\sigma}_{n}^{z}+J\,\hat{\sigma}_{n}^{x}\hat{\sigma}_{n+1}^{x}:=\hat{H}_{0}+\hat{V}_{J}.\label{eq:H-0}
\end{equation}
Here $\hat{\sigma}_{n}^{x,z}$ are two Pauli matrices, and $\hat{\sigma}_{n}^{z}:=|\mathrm{e}\rangle_{n}\langle\mathrm{e}|-|\mathrm{g}\rangle_{n}\langle\mathrm{g}|,\hat{\sigma}_{n}^{x}:=\hat{\sigma}_{n}^{+}+\hat{\sigma}_{n}^{-}$
,with $\hat{\sigma}_{n}^{+}:=(\hat{\sigma}_{n}^{-})^{\dagger}=|\mathrm{e}\rangle_{n}\langle\mathrm{g}|$
and $|\mathrm{e}\rangle_{n}$, $|\mathrm{g}\rangle_{n}$ are the excited
and ground states of the $n$-th TLS. The Ising chain has $N$ site
states, which have equal on-site energies ($\omega\ge0$). And the
nearest neighbors interaction strength is $J$. 

Under the periodic boundary condition, applying the Jordan-Wigner
transform, 
\begin{align}
\hat{\sigma}_{n}^{z} & =2\hat{d}_{n}^{\dagger}\hat{d}_{n}-1,\qquad\hat{\sigma}_{n}^{+}=\hat{d}_{n}^{\dagger}\prod_{i=0}^{n-1}\left(-\hat{\sigma}_{i}^{z}\right),\nonumber \\
\hat{\sigma}_{n}^{x} & =\prod_{i=0}^{n-1}\left(-\hat{\sigma}_{i}^{z}\right)\left(\hat{c}_{n}+\hat{c}_{n}^{\dagger}\right),
\end{align}
the Hamiltonian becomes a fermionic one, 
\begin{gather}
\hat{H}=\sum_{n=0}^{N-1}\omega\hat{d}_{n}^{\dagger}\hat{d}_{n}+J(\hat{d}_{n}^{\dagger}\hat{d}_{n+1}+\hat{d}_{n}^{\dagger}\hat{d}_{n+1}^{\dagger}+\text{h.c.}).
\end{gather}
 Then we apply the Fourier transform,
\begin{align}
\hat{d}_{n} & =\frac{1}{\sqrt{N}}\sum_{k}\exp\big[-i\frac{2\pi}{N}kn\big]\hat{c}_{k},\nonumber \\
\hat{c}_{k} & =\frac{1}{\sqrt{N}}\sum_{n=0}^{N-1}\exp\big[i\frac{2\pi}{N}kn\big]\hat{d}_{n},\label{eq:FT}
\end{align}
 Here we focus on the case that $N$ is odd, and $k=-\frac{1}{2}(N-1),\ldots,0\ldots,\frac{1}{2}(N-1)$.
Then the above Hamiltonian is rewritten as $\hat{H}=\frac{1}{2}\sum\mathbf{\mathbb{\mathbf{c}}_{k}^{\dagger}}\mathbf{H_{k}}\mathbf{c_{k}}$
with $\mathbf{\mathbb{\mathbf{c}}_{k}}:=(\hat{c}_{k},\,\hat{c}_{-k}^{\dagger})^{T}$,
and 
\begin{equation}
\mathbf{H_{k}}=\left[\begin{array}{cc}
\frac{1}{2}\omega+2J\cos\frac{2\pi}{N}k & i2J\sin\frac{2\pi}{N}k\\
-i2J\sin\frac{2\pi}{N}k & -\left(\frac{1}{2}\omega+2J\cos\frac{2\pi}{N}k\right)
\end{array}\right].
\end{equation}
Hence, the Hamiltonian can be diagonalized by Bogoliubov transformation,
\begin{gather}
\hat{H}=\sum_{k>0}\left(\begin{array}{cc}
\hat{\gamma}_{k}^{\dagger} & \hat{\gamma}_{-k}\end{array}\right)\left[\begin{array}{cc}
\varepsilon_{k} & 0\\
0 & -\varepsilon_{k}
\end{array}\right]\left(\begin{array}{c}
\hat{\gamma}_{k}\\
\hat{\gamma}_{-k}^{\dagger}
\end{array}\right),\nonumber \\
\left(\begin{array}{c}
\hat{c}_{k}\\
\hat{c}_{-k}^{\dagger}
\end{array}\right)=\left[\begin{array}{cc}
\cos\theta_{k} & -i\sin\theta_{k}\\
-i\sin\theta_{k} & \cos\theta_{k}
\end{array}\right]\left(\begin{array}{c}
\hat{\gamma}_{k}\\
\hat{\gamma}_{-k}^{\dagger}
\end{array}\right)\label{eq:BT}
\end{gather}
 where $\varepsilon_{k}$ is the eigen mode energy, and 
\begin{gather}
\varepsilon_{k}=\sqrt{\epsilon_{k}^{2}+4J^{2}\sin^{2}\frac{2\pi}{N}k},\quad\epsilon_{k}=\frac{1}{2}\omega+2J\cos\frac{2\pi}{N}k,\nonumber \\
\tan2\theta_{k}=\frac{2J\sin\frac{2\pi}{N}k}{\frac{1}{2}\omega+2J\cos\frac{2\pi}{N}k}.
\end{gather}

Now we consider the dynamics of this system. The $N$-body system
as a whole isolated system follows the unitary evolution, the system
initially in $|\varPsi_{0}\rangle=|\mathrm{e},\mathrm{g},\mathrm{g},\ldots\rangle$.
Namely, site-0 starts from the excited state $|\mathrm{e}\rangle$
as its initial state, and all the TLSs start from the ground state
$|\mathrm{g}\rangle$. Here we refer to site-0 as an open ``system'',
while all the other $(N-1)$ TLSs build up a finite ``bath''. 

After the Jordan-Wigner transform, this initial state can be rewritten
as, $|\varPsi_{0}\rangle=\hat{d}_{0}^{\dagger}|\mathbf{0}\rangle$,
where $|\mathbf{0}\rangle$ is the vacuum state of the fermion system.
Thus, the dynamics of this system $|\varPsi_{\tau\equiv Jt}\rangle$
is obtained as ($\tilde{\varepsilon}_{k}:=\varepsilon_{k}/J$), 
\begin{align}
|\varPsi_{\tau}\rangle & =e^{-i\hat{H}\tau}\hat{d}_{0}^{\dagger}|\mathbf{0}\rangle\nonumber \\
= & \sum_{k}\frac{1}{\sqrt{N}}\big(\cos^{2}\theta_{k}e^{i\tilde{\varepsilon}_{k}\tau}+\sin^{2}\theta_{k}e^{-i\tilde{\varepsilon}_{k}\tau}\big)\,\hat{c}_{k}^{\dagger}|\mathbf{0}\rangle\nonumber \\
= & \sum_{k}\sum_{n=0}^{N-1}\frac{e^{-i\frac{2\pi}{N}kn}}{N}\big(\cos^{2}\theta_{k}e^{i\tilde{\varepsilon}_{k}\tau}+\sin^{2}\theta_{k}e^{-i\tilde{\varepsilon}_{k}\tau}\big)\,\hat{d}_{n}^{\dagger}|\mathbf{0}\rangle\nonumber \\
= & \sum_{n=0}^{N-1}\Phi_{n}^{(N)}(\tau)\ \hat{d}_{n}^{\dagger}|\mathbf{0}\rangle,\label{eq:Psi-t}
\end{align}
where the evolution is described in term of the unitless time $\tau\equiv Jt$,
and we call 
\[
\Phi_{n}^{(N)}(\tau):=\sum_{k}\frac{e^{-i\frac{2\pi}{N}kn}}{N}\left(\cos^{2}\theta_{k}e^{i\tilde{\varepsilon}_{k}\tau}+\sin^{2}\theta_{k}e^{-i\tilde{\varepsilon}_{k}\tau}\right),
\]
as a coherence function. As a result, the population of each site-$n$
is obtained as $p_{n,e}(\tau)=|_{n}\langle\mathrm{e}|\varPsi_{\tau}\rangle|^{2}=|\Phi_{n}^{(N)}(\tau)|^{2}$.
On the other hand, since the initial state gives $\langle\hat{\sigma}_{n}^{\pm}\rangle=0$
for all the $N$ sites, it can be proved that the non-diagonal terms
of the density state $_{n}\langle\mathrm{e}|\rho_{n}(t)|\mathrm{g}\rangle_{n}$
of each site always keep zero during the evolution. 

Notice that, when the coupling strength is quite weak ($J\ll\omega$),
in the interaction picture of $\hat{H}_{0}=\frac{1}{2}\omega\sum\hat{\sigma}_{n}^{z}$,
the interaction in the Hamiltonian (\ref{eq:H-0}) becomes, 
\begin{align}
\hat{V}_{J}(t) & =J\sum_{n}\left(\hat{\sigma}_{n}^{+}e^{i\omega t}+\hat{\sigma}_{n}^{-}e^{-i\omega t}\right)\left(\hat{\sigma}_{n+1}^{+}e^{i\omega t}+\hat{\sigma}_{n+1}^{-}e^{-i\omega t}\right)\nonumber \\
 & \simeq J\sum_{n}\left(\hat{\sigma}_{n}^{+}\hat{\sigma}_{n+1}^{-}+\hat{\sigma}_{n}^{-}\hat{\sigma}_{n+1}^{+}\right),
\end{align}
where the rotating-wave approximation (RWA) is adopted, and the oscillating
terms with frequencies $2\omega$ are neglected. Returning back to
the Schr\"odinger picture the Hamiltonian becomes 
\begin{equation}
\hat{H}=\sum_{n=0}^{N-1}\frac{\omega}{2}\,\hat{\sigma}_{n}^{z}+J(\hat{\sigma}_{n}^{+}\hat{\sigma}_{n+1}^{-}+\hat{\sigma}_{n}^{-}\hat{\sigma}_{n+1}^{+}).\label{eq:XX}
\end{equation}
This is also known as the quantum \emph{XX} model \cite{sachdev_quantum_2011},
which guarantees the total population $\sum_{n}\hat{\sigma}_{n}^{z}$
is conserved, while the original Ising model does not. That means,
the behavior of the system dynamics would be similar as the quantum
\emph{XX} model in the weak coupling regime \cite{li_hierarchy_2021},
while they would exhibit significant differences when $J$ is strong. 

\section{The recurrence behavior in the local dynamics\label{sec:The-recurrence-behavior}}

Now the full state of the $N$-body system at any time $\tau\equiv Jt$
can be obtained exactly by Eq. (\ref{eq:Psi-t}). Since the whole
isolated system follows the unitary evolution, if we do not consider
any averaging approach, such as basing on time or random defect configurations,
the full $N$-body state $|\varPsi_{\tau}\rangle$ would always keep
a pure state during the evolution. Namely, indeed $|\varPsi_{\tau}\rangle$
is never approaching any canonical thermal state as $\rho_{th}\sim\exp(-\hat{H}/k_{B}T)$.
On the other hand, it is worth noting that, when focusing on the states
of each local site, they exhibits similar behavior as the relaxation
in macroscopic thermodynamics. In this sense, this is called the local
relaxation. 

Therefore, here we focus on the dynamical behavior of local observables
of each site. In Fig. \ref{fig:kktt}(a, d), the population evolution
of site-0 is shown for both the strong and weak coupling cases respectively.
Before a certain typical time $\tau\lesssim\tau_{rec}$, the population
dynamics of site-0 exhibits an oscillating decay behavior in both
the strong and weak coupling cases, which is quite similar as the
relaxation in macroscopic thermodynamics. Namely, within the observation
time $\tau<\tau_{rec}$, the population of site-0 (the ``system'')
seems relaxing towards $p_{0,e}\rightarrow0$ as its steady state. 

However, the decaying behavior suddenly changes around $Jt\equiv\tau\simeq\tau_{rec}$,
and then looks ``random'', which was referred as a recurrence behavior
due to the finite size effect \cite{cramer_exact_2008,li_hierarchy_2021,flesch_probing_2008,eisert_quantum_2015,hanggi_reaction-rate_1990,zwanzig_nonequilibrium_2001}.
When the system size approaches the thermodynamic limit, the appearance
of the recurrence behavior would be postponed to $\tau_{rec}\rightarrow\infty$,
thus at a finite time only the irreversible relaxation behavior can
be observed. 

Here it is worth noting that, such a recurrence behavior appear in
both strong and weak coupling regime, and the recurrence times in
these two cases both can be evaluated as $\tau_{rec}\simeq N$ (or
$t_{rec}\simeq N/J$) {[}see Fig.    \ref{fig:kktt}(a, d){]}. In
the weak coupling regime $J\ll\omega$, the system Hamiltonian (\ref{eq:H-0})
can be approximated by the quantum \emph{XX} model (\ref{eq:XX}),
which has been considered in previous studies \cite{cramer_exact_2008,li_hierarchy_2021,zwanzig_nonequilibrium_2001},
and the system dynamics here is also consistent with these literatures.
On the other hand, when the Ising coupling is quite strong Fig.    \ref{fig:kktt}(a){]},
the system dynamics still exhibit similar local relaxation and recurrence
behaviors as the weak coupling situation, but contains more high frequency
oscillations.

Furthermore, we show the evolution of the von Neumann entropy of each
TLS $S[\hat{\rho}_{n}(\tau)]$ in Fig. \ref{fig:kktt}(b, e){]}. It
turns out the entropy of each single site increases and decreases
from time to time, but does not exhibit any irreversible increasing
behavior . It is worth noting that, in both strong and weak coupling
cases, the entropy of each TLS exhibit similar propagation behaviors
starting from site-0 towards its two sides. Clearly, such propagations
would meet each other at the periodic boundaries at $n\sim\pm[N/2]$,
and then regathers back to site-0 again, and this is just the moment
that the system exhibits its recurrence around $\tau\simeq\tau_{rec}$
(see the vertical dashed red lines in Fig.     \ref{fig:kktt}). Because
of the superposition with the propagation regathered back, the system
dynamics appears more random. And this is also why the recurrence
time is determined by $Jt_{rec}\equiv\tau_{rec}\simeq N$, which applies
in both the strong and weak coupling cases.

\begin{figure}
\includegraphics[width=1\columnwidth]{fig-kktt}

\caption{(a, d) The population evolution of site-$0$ $p_{0,e}(\tau)=|\Phi_{0}^{(N=41)}(\tau)|^{2}$
in the strong ($J/\omega=10$) and weak ($J/\omega=0.025$) coupling
regime. (b, e) The evolution of the von Neumann entropy $S[\rho_{n}(\tau)]$
of each TLS in the strong and weak coupling regime. In (a, b, d, e),
the site number is set as $N=41$. (c, f) The scaling behavior of
$|\Phi_{0}^{(N)}(\tau)|^{2}$ with the site number $N$ in the strong
and weak coupling regime. }

\label{fig:kktt}
\end{figure}

Since such propagation and regathering behavior happens again and
again, similar recurrence behavior could also appear around $\tau\simeq q\tau_{rec}$,
with $q=1,2,3\ldots$ In Fig.     \ref{fig:kktt}(c, f), we show the
scaling behavior of $p_{0,e}(\tau)=|\Phi_{0}^{(N)}(\tau)|^{2}$ for
different sizes $N$. it is worth noting that some well-organized
recurrence pattens appear periodically, and the dynamics becomes more
and more ``random'' after each recurrence, and thus we call them
hierarchy recurrences \cite{li_hierarchy_2021}. 

It is worth noting that the above local relaxation and hierarchy recurrence
behaviors appear in both the strong and weak coupling situations,
and has quite similar evolution envelopes when measured by the unitless
time $\tau=Jt$. Even the recurrence times are both determined by
$Jt_{rec}\equiv\tau_{rec}\simeq N$, except the dynamics in the strong
coupling case contains more high frequency oscillations. These results
indicate the appearance of such recurrences is a quite general property
in spite of the interaction strength and type. 

\section{Production of the total correlation entropy \label{sec:Total-correlation-entropy}}

In the above result, the local observables exhibit similar behaviors
as the relaxation in macroscopic thermodynamics before the recurrence,
but the entropy of each site increases and decreases from time to
time, which is different from the irreversible entropy production
behavior in macroscopic thermodynamics. On the other hand, as an isolated
system, the entropy of the whole system state always keeps the same
as its initial state due to the the unitary evolution. Indeed, as
a finite isolated $N$-body system, the initial state here is not
a equilibrium state, which is beyond the description scope of the
standard thermodynamics in the macroscopic limit. As a result, the
standard entropy production in macroscopic thermodynamics based on
thermal entropy $dS=\text{\dj}Q/T$ cannot be applied here.

Notice that, in practical observations, the full $N$-body state is
usually not directly assessable for local measurements, and it is
the few-body observables that are directly measured \cite{swendsen_explaining_2008,li_correlation_2019,strasberg_entropy_2019}.
Therefore, in spite of the entropy of the full $N$-body state $\hat{\boldsymbol{\rho}}$,
here we focus on the total correlation entropy in this system, which
is defined by \cite{watanabe_information_1960,groisman_quantum_2005,zhou_irreducible_2008,anza_logarithmic_2020}
\begin{equation}
\mathbf{C}[\hat{\boldsymbol{\rho}}]:=\sum_{n=0}^{N-1}S[\hat{\rho}_{n}]-S[\hat{\boldsymbol{\rho}}].
\end{equation}
Here $S[\hat{\boldsymbol{\rho}}]$ is von Neumann entropy of the full
$N$-body state, which does not change during the unitary evolution,
and $\hat{\rho}_{n}$ are the reduced one-body states. $\mathbf{C}[\hat{\boldsymbol{\rho}}]$
measures the total amount of all the correlations inside the $N$-body
state $\hat{\boldsymbol{\rho}}$ \cite{watanabe_information_1960,zhou_irreducible_2008}.
For $N=2$, it just returns the mutual information, which measures
the bipartite correlation.

In the weak coupling regime, the dynamics of the Ising model could
be reduced to the \emph{XX} model due to RWA, and it turns out the
total population of the $N$-body system only has quite small fluctuations
during the evolution {[}Fig.    \ref{fig:TCE}(c){]}. The total correlation
entropy approximately exhibits a monotonic increasing behavior, which
is consistent with the previous studies based on the \emph{XX} model
{[}Fig.    \ref{fig:TCE}(d){]} \cite{li_hierarchy_2021}. Clearly
this is quite similar to the irreversible entropy production during
the relaxation process in the standard thermodynamics \cite{spohn_entropy_1978,de_groot_non-equilibrium_1962,kondepudi_modern_2014,nicolis_self-organization_1977}.

When the Ising coupling is strong, the total population does not conserve
and exhibits significant fluctuations. The total correlation still
exhibits an increasing envelope, but contains more violent oscillations
comparing with the weak coupling case {[}Fig.    \ref{fig:TCE}(b){]}. 

\begin{figure}
\includegraphics[width=1\columnwidth]{fig-tce}

\caption{The evolution of the total population $\langle\hat{\text{\textsc{n}}}_{t}\rangle$
(a, c), and the total correlation entropy (b, d) in the strong ($J/\omega=10$)
and weak ($J/\omega=0.025$) coupling regime. The dashed horizontal
lines in (b, d) are $\mathbf{C}_{max}^{(N)}\simeq3.74$ and $\mathbf{C}_{max}^{(N)}\simeq4.70$
(with the site numbers set as $N=41$), which are obtained by the
total correlation maximization (\ref{eq:C}). The solid horizontal
lines in (a, c) are $\tilde{\text{\textsc{n}}}_{\text{\textsc{p}}}=0.75$
and $\tilde{\text{\textsc{n}}}_{\text{\textsc{p}}}\simeq0.9996$ respectively,
obtained from the time average Eq. (\ref{eq:np}).}

\label{fig:TCE}
\end{figure}

In this sense, now we try to find out the final point of the increasing
behavior of this total correlation entropy, and the possible maximum
that $\mathbf{C}[\hat{\boldsymbol{\rho}}(\tau)]$ might achieve can
be obtained by the variation approach based on certain constraints
\cite{jaynes_information_1957}. In the weak coupling case, the total
population is approximately conserved as its initial one, which could
be treated as a variation constraint {[}Fig.     \ref{fig:TCE}(c){]}.
But in the strong coupling regime, the total population exhibit significant
fluctuations because of the Ising interaction. From Eqs. (\ref{eq:FT},
\ref{eq:BT}), the total population condition $\langle\hat{\text{\textsc{n}}}_{t}\rangle=\langle\hat{d}_{n}^{\dagger}(\tau)\hat{d}_{n}(\tau)\rangle$
is obtained as 

\begin{align}
\langle\hat{\text{\textsc{n}}}_{t}\rangle= & \frac{1}{N}\sum_{k}\cos^{4}\theta_{k}-\left(i\sin\theta_{k}\cos\theta_{k}\right)^{2}e^{2i\tilde{\varepsilon}_{k}\tau}\nonumber \\
 & -\left(i\sin\theta_{k}\cos\theta_{k}\right)^{2}e^{-2i\tilde{\varepsilon}_{k}\tau}+\sin^{4}\theta_{k},
\end{align}

Although containing significant fluctuations, it is worth noting that
the total population still varies around a certain value {[}see the
numerical result in Fig.     \ref{fig:kktt}(a){]}. Moreover, within
the relaxation time $\tau\lesssim\tau_{rec}/2\sim N/2$, the total
population $\langle\hat{\text{\textsc{n}}}_{t}\rangle$ seems converging
to a steady value {[}see the horizontal red line in Fig.     \ref{fig:TCE}(a){]},
until a recurrence happens because of the finite size effect. Notice
that the moment that the total population exhibits its first recurrence
is just when the two-side propagations first meet each other at the
periodic boundaries at $n\sim\pm[N/2]$ {[}comparing with Fig.     \ref{fig:kktt}(a){]}. 

Therefore, here we introduce the long time average to the above total
population $\langle\hat{\text{\textsc{n}}}_{t}\rangle$, which eliminates
all the rotating terms, i.e., \cite{srednicki_chaos_1994,polkovnikov_colloquium_2011,rigol_alternatives_2012}
\begin{align}
\tilde{\text{\textsc{n}}}_{\text{\textsc{p}}} & :=\lim_{T\rightarrow\infty}\frac{1}{T}\int_{0}^{T}dt\,\langle\hat{\text{\textsc{n}}}_{t}\rangle=\frac{1}{N}\sum_{k}\cos^{4}\theta_{k}+\sin^{4}\theta_{k}\nonumber \\
 & \stackrel{\text{\textsc{n}}\rightarrow\infty}{\longrightarrow}\frac{1}{8}\Big[7-4\tilde{J}^{2}+\big|1-2\tilde{J}\big|\cdot(1+2\tilde{J})\Big],\label{eq:np}
\end{align}
with $\tilde{J}:=J/\omega$. It turns out this is just the central
value that the $\langle\hat{\text{\textsc{n}}}_{t}\rangle$ varies
around and converges towards for both the strong and weak coupling
cases {[}the horizontal solid lines in Fig.     \ref{fig:TCE}(a,
c){]}. In this sense, $\langle\hat{\text{\textsc{n}}}_{t}\rangle\simeq\tilde{\text{\textsc{n}}}_{\text{\textsc{p}}}$
still could be treated as an approximated constraint when looking
for the possible maximum of the total correlation entropy. In the
thermodynamics limit $N\rightarrow\infty$, recurrences do not appear,
and the total population $\langle\hat{\text{\textsc{n}}}_{t\rightarrow\infty}\rangle$
would well converge to $\tilde{\text{\textsc{n}}}_{\text{\textsc{p}}}$
{[}Eq. (\ref{eq:np}){]} as its steady value.

With the help of Lagrangian multipliers, under the constraints (1)
$p_{n,\mathrm{g}}+p_{n,\mathrm{e}}=1$ (probability normalization),
(2) $\langle\hat{\text{\textsc{n}}}\rangle=\sum_{n}\,p_{n,e}\simeq\tilde{\text{\textsc{n}}}_{\text{\textsc{p}}}$
(total population conservation), the maximum of the total correlation
$\mathbf{C}=\sum_{n}\,-p_{n,e}\ln p_{n,e}-p_{n,g}\ln p_{n,g}$ is
calculated by taking variation of the functional 
\begin{align*}
F\big[\{p_{n,e}\},\lambda\big]:= & \big(\sum_{n}-p_{n,e}\ln p_{n,e}-p_{n,g}\ln p_{n,g}\big)\\
 & +\lambda\big(\tilde{\text{\textsc{n}}}_{\text{\textsc{p}}}-\sum_{n}p_{n,e}\big)
\end{align*}
upon $p_{n,e}$ and $\lambda$. As a result, the correlation maximum
is obtained as 
\begin{equation}
\mathbf{C}_{max}^{(N)}:=-\tilde{\text{\textsc{n}}}_{\text{\textsc{p}}}\ln\frac{\tilde{\text{\textsc{n}}}_{\text{\textsc{p}}}}{N}-(N-\tilde{\text{\textsc{n}}}_{\text{\textsc{p}}})\ln(1-\frac{\tilde{\text{\textsc{n}}}_{\text{\textsc{p}}}}{N}),\label{eq:C}
\end{equation}
and this maximum is achieved when all the $N$ TLSs have the same
populations $p_{n,e}=\tilde{\text{\textsc{n}}}_{\text{\textsc{p}}}/N$. 

It turns out this correlation maximum {[}dashed horizontal lines in
Fig.     \ref{fig:TCE}(b, d){]} coincides quite well with the numerical
result obtained from time dependent evolution. Notice that such an
estimation applies even for the strong coupling situation, except
in some regimes after the recurrence $\tau>\tau_{rec}/2$, which roots
from the failure of the approximated constraint $\langle\hat{\text{\textsc{n}}}_{t}\rangle\simeq\tilde{\text{\textsc{n}}}_{\text{\textsc{p}}}$.
Therefore, here the behavior of the total correlation entropy in this
system is quite similar as the irreversible entropy production in
the standard macroscopic thermodynamics. Remember that here the standard
entropy production based on thermal entropy $dS=\text{\dj}Q/T$ cannot
be applied.

\section{DISCUSSION\label{sec:DISCUSSION}}

Here we consider the nonequilibrium dynamics in an isolated $N$-body
system with Ising interaction. During the unitary evolution, without
applying any time average, the full $N$-body state always keeps a
constant entropy, and the entropy of each single site increases and
decreases from time to time. In comparison, the dynamics of the local
observables exhibit local relaxation behavior, which is similar as
the macroscopic thermodynamics. Due to the finite system size, recurrences
appears in the local relaxations which roots from the superposition
with the propagation regathered back. The total correlation entropy
approximately exhibits a monotonic increasing behavior, which is similar
to the irreversible entropy increase in the standard thermodynamics.
It is worth noting that these results appear in spite of the coupling
strength and exhibit similar profile under the unitless time $\tau=Jt$.
Even for the strong coupling situation where the total population
has violent fluctuations, the total population exhibit an explicit
converging behavior until the recurrence happens.

The similar idea also applies for open systems. In open system problems,
the bath is usually modeled as a collection of many noninteracting
DoF, and initially starts from a canonical thermal state $\hat{\rho}_{\text{\textsc{b}}}(0)\sim\exp(-\hat{H}_{\text{\textsc{b}}}/T)$.
When the system-bath interaction strength is negligibly small, the
changing rate of the bath entropy can be approximately obtained as
\cite{li_production_2017,aurell_von_2015,you_entropy_2018,manzano_quantum_2018}
\[
\dot{S}_{\text{\textsc{b}}}(t)\simeq-\mathrm{tr}\left[\hat{\rho}_{\text{\textsc{b}}}(t)\ln\hat{\rho}_{\text{\textsc{b}}}(0)\right]=\frac{1}{T}\frac{d}{dt}\langle\hat{H}_{\text{\textsc{b}}}\rangle.
\]
 Here the bath energy increase $\frac{d}{dt}\langle\hat{H}_{\text{\textsc{b}}}\rangle$
is just equal to the system energy loss $-\dot{Q}$, thus the changing
rate of the total correlation entropy gives 
\begin{equation}
\frac{d}{dt}\mathbf{C}=\frac{d}{dt}(S_{\text{\textsc{s}}}+S_{\text{\textsc{b}}}-S_{\text{\textsc{sb}}})\simeq\dot{S}_{\text{\textsc{s}}}-\frac{\dot{Q}}{T},
\end{equation}
which returns the entropy production rate in the standard thermodynamics
\cite{li_production_2017,you_entropy_2018,li_correlation_2019}. Thus,
the second law statement that the irreversible entropy keeps increasing
could be equivalently understood as the increase of total correlation
in this whole system \cite{esposito_entropy_2010,li_production_2017,manzano_entropy_2016,alipour_correlations_2016,pucci_entropy_2013,hobson_irreversibility_1966,zhang_general_2008,horowitz_equivalent_2014,kalogeropoulos_time_2018}.
If the correlation between the different DoF inside the bath is significant,
corrections should be considered \cite{ptaszynski_entropy_2019}.

In practice, usually it is the partial information (e.g., marginal
distribution, few-body observable expectations) that is directly accessible
to our observation. Indeed most macroscopic thermodynamic quantities
are obtained only from such partial information like the one-body
distribution, which exhibits irreversible behaviors. In practical
measurements, the dynamics of the full many body state is quite difficult
to be measured directly. In this sense, the reversibility of microscopic
dynamics and the macroscopic irreversibility coincide with each other. 

\vspace{0.2em}

\emph{Acknowledgments }- This study is supported by NSF of China (Grant
No. 11905007).

%apsrev4-2.bst 2019-01-14 (MD) hand-edited version of apsrev4-1.bst
%Control: key (0)
%Control: author (72) initials jnrlst
%Control: editor formatted (1) identically to author
%Control: production of article title (-1) disabled
%Control: page (0) single
%Control: year (1) truncated
%Control: production of eprint (0) enabled
\begin{thebibliography}{40}%
\makeatletter
\providecommand \@ifxundefined [1]{%
 \@ifx{#1\undefined}
}%
\providecommand \@ifnum [1]{%
 \ifnum #1\expandafter \@firstoftwo
 \else \expandafter \@secondoftwo
 \fi
}%
\providecommand \@ifx [1]{%
 \ifx #1\expandafter \@firstoftwo
 \else \expandafter \@secondoftwo
 \fi
}%
\providecommand \natexlab [1]{#1}%
\providecommand \enquote  [1]{``#1''}%
\providecommand \bibnamefont  [1]{#1}%
\providecommand \bibfnamefont [1]{#1}%
\providecommand \citenamefont [1]{#1}%
\providecommand \href@noop [0]{\@secondoftwo}%
\providecommand \href [0]{\begingroup \@sanitize@url \@href}%
\providecommand \@href[1]{\@@startlink{#1}\@@href}%
\providecommand \@@href[1]{\endgroup#1\@@endlink}%
\providecommand \@sanitize@url [0]{\catcode `\\12\catcode `\$12\catcode
  `\&12\catcode `\#12\catcode `\^12\catcode `\_12\catcode `\%12\relax}%
\providecommand \@@startlink[1]{}%
\providecommand \@@endlink[0]{}%
\providecommand \url  [0]{\begingroup\@sanitize@url \@url }%
\providecommand \@url [1]{\endgroup\@href {#1}{\urlprefix }}%
\providecommand \urlprefix  [0]{URL }%
\providecommand \Eprint [0]{\href }%
\providecommand \doibase [0]{https://doi.org/}%
\providecommand \selectlanguage [0]{\@gobble}%
\providecommand \bibinfo  [0]{\@secondoftwo}%
\providecommand \bibfield  [0]{\@secondoftwo}%
\providecommand \translation [1]{[#1]}%
\providecommand \BibitemOpen [0]{}%
\providecommand \bibitemStop [0]{}%
\providecommand \bibitemNoStop [0]{.\EOS\space}%
\providecommand \EOS [0]{\spacefactor3000\relax}%
\providecommand \BibitemShut  [1]{\csname bibitem#1\endcsname}%
\let\auto@bib@innerbib\@empty
%</preamble>
\bibitem [{\citenamefont {Hobson}(1966)}]{hobson_irreversibility_1966}%
  \BibitemOpen
  \bibfield  {author} {\bibinfo {author} {\bibfnamefont {A.}~\bibnamefont
  {Hobson}},\ }\href {https://doi.org/10.1119/1.1973009} {\bibfield  {journal}
  {\bibinfo  {journal} {Am. J. Phys.}\ }\textbf {\bibinfo {volume} {34}},\
  \bibinfo {pages} {411} (\bibinfo {year} {1966})}\BibitemShut {NoStop}%
\bibitem [{\citenamefont {Prigogine}(1978)}]{prigogine_time_1978}%
  \BibitemOpen
  \bibfield  {author} {\bibinfo {author} {\bibfnamefont {I.}~\bibnamefont
  {Prigogine}},\ }\href {https://doi.org/10.1126/science.201.4358.777}
  {\bibfield  {journal} {\bibinfo  {journal} {Science}\ }\textbf {\bibinfo
  {volume} {201}},\ \bibinfo {pages} {777} (\bibinfo {year}
  {1978})}\BibitemShut {NoStop}%
\bibitem [{\citenamefont {Mackey}(1989)}]{mackey_dynamic_1989}%
  \BibitemOpen
  \bibfield  {author} {\bibinfo {author} {\bibfnamefont {M.~C.}\ \bibnamefont
  {Mackey}},\ }\href {https://doi.org/10.1103/RevModPhys.61.981} {\bibfield
  {journal} {\bibinfo  {journal} {Rev. Mod. Phys.}\ }\textbf {\bibinfo {volume}
  {61}},\ \bibinfo {pages} {981} (\bibinfo {year} {1989})}\BibitemShut
  {NoStop}%
\bibitem [{\citenamefont {Uffink}(2006)}]{uffink_compendium_2006}%
  \BibitemOpen
  \bibfield  {author} {\bibinfo {author} {\bibfnamefont {J.}~\bibnamefont
  {Uffink}},\ }in\ \href {http://philsci-archive.pitt.edu/2691/} {\emph
  {\bibinfo {booktitle} {Philosophy of {Physics}}}},\ \bibinfo {series and
  number} {Handbook of the {Philosophy} of {Science}}\ (\bibinfo  {publisher}
  {North Holland},\ \bibinfo {address} {Amsterdam},\ \bibinfo {year} {2006})\
  p.\ \bibinfo {pages} {923}\BibitemShut {NoStop}%
\bibitem [{\citenamefont {Swendsen}(2008)}]{swendsen_explaining_2008}%
  \BibitemOpen
  \bibfield  {author} {\bibinfo {author} {\bibfnamefont {R.~H.}\ \bibnamefont
  {Swendsen}},\ }\href {https://doi.org/10.1119/1.2894523} {\bibfield
  {journal} {\bibinfo  {journal} {Am. J. Phys.}\ }\textbf {\bibinfo {volume}
  {76}},\ \bibinfo {pages} {643} (\bibinfo {year} {2008})}\BibitemShut
  {NoStop}%
\bibitem [{\citenamefont {Cramer}\ \emph {et~al.}(2008)\citenamefont {Cramer},
  \citenamefont {Dawson}, \citenamefont {Eisert},\ and\ \citenamefont
  {Osborne}}]{cramer_exact_2008}%
  \BibitemOpen
  \bibfield  {author} {\bibinfo {author} {\bibfnamefont {M.}~\bibnamefont
  {Cramer}}, \bibinfo {author} {\bibfnamefont {C.~M.}\ \bibnamefont {Dawson}},
  \bibinfo {author} {\bibfnamefont {J.}~\bibnamefont {Eisert}},\ and\ \bibinfo
  {author} {\bibfnamefont {T.~J.}\ \bibnamefont {Osborne}},\ }\href
  {https://doi.org/10.1103/PhysRevLett.100.030602} {\bibfield  {journal}
  {\bibinfo  {journal} {Phys. Rev. Lett.}\ }\textbf {\bibinfo {volume} {100}},\
  \bibinfo {pages} {030602} (\bibinfo {year} {2008})}\BibitemShut {NoStop}%
\bibitem [{\citenamefont {Li}\ and\ \citenamefont
  {Sun}(2021)}]{li_hierarchy_2021}%
  \BibitemOpen
  \bibfield  {author} {\bibinfo {author} {\bibfnamefont {S.-W.}\ \bibnamefont
  {Li}}\ and\ \bibinfo {author} {\bibfnamefont {C.~P.}\ \bibnamefont {Sun}},\
  }\href {https://doi.org/10.1103/PhysRevA.103.042201} {\bibfield  {journal}
  {\bibinfo  {journal} {Physical Review A}\ }\textbf {\bibinfo {volume}
  {103}},\ \bibinfo {pages} {042201} (\bibinfo {year} {2021})}\BibitemShut
  {NoStop}%
\bibitem [{\citenamefont {Flesch}\ \emph {et~al.}(2008)\citenamefont {Flesch},
  \citenamefont {Cramer}, \citenamefont {McCulloch}, \citenamefont
  {Schollw{\"o}ck},\ and\ \citenamefont {Eisert}}]{flesch_probing_2008}%
  \BibitemOpen
  \bibfield  {author} {\bibinfo {author} {\bibfnamefont {A.}~\bibnamefont
  {Flesch}}, \bibinfo {author} {\bibfnamefont {M.}~\bibnamefont {Cramer}},
  \bibinfo {author} {\bibfnamefont {I.~P.}\ \bibnamefont {McCulloch}}, \bibinfo
  {author} {\bibfnamefont {U.}~\bibnamefont {Schollw{\"o}ck}},\ and\ \bibinfo
  {author} {\bibfnamefont {J.}~\bibnamefont {Eisert}},\ }\href
  {https://doi.org/10.1103/PhysRevA.78.033608} {\bibfield  {journal} {\bibinfo
  {journal} {Phys. Rev. A}\ }\textbf {\bibinfo {volume} {78}},\ \bibinfo
  {pages} {033608} (\bibinfo {year} {2008})}\BibitemShut {NoStop}%
\bibitem [{\citenamefont {Eisert}\ \emph {et~al.}(2015)\citenamefont {Eisert},
  \citenamefont {Friesdorf},\ and\ \citenamefont
  {Gogolin}}]{eisert_quantum_2015}%
  \BibitemOpen
  \bibfield  {author} {\bibinfo {author} {\bibfnamefont {J.}~\bibnamefont
  {Eisert}}, \bibinfo {author} {\bibfnamefont {M.}~\bibnamefont {Friesdorf}},\
  and\ \bibinfo {author} {\bibfnamefont {C.}~\bibnamefont {Gogolin}},\ }\href
  {https://doi.org/10.1038/nphys3215} {\bibfield  {journal} {\bibinfo
  {journal} {Nature Phys.}\ }\textbf {\bibinfo {volume} {11}},\ \bibinfo
  {pages} {124} (\bibinfo {year} {2015})}\BibitemShut {NoStop}%
\bibitem [{\citenamefont {H{\"a}nggi}\ \emph {et~al.}(1990)\citenamefont
  {H{\"a}nggi}, \citenamefont {Talkner},\ and\ \citenamefont
  {Borkovec}}]{hanggi_reaction-rate_1990}%
  \BibitemOpen
  \bibfield  {author} {\bibinfo {author} {\bibfnamefont {P.}~\bibnamefont
  {H{\"a}nggi}}, \bibinfo {author} {\bibfnamefont {P.}~\bibnamefont
  {Talkner}},\ and\ \bibinfo {author} {\bibfnamefont {M.}~\bibnamefont
  {Borkovec}},\ }\href {https://doi.org/10.1103/RevModPhys.62.251} {\bibfield
  {journal} {\bibinfo  {journal} {Rev. Mod. Phys.}\ }\textbf {\bibinfo {volume}
  {62}},\ \bibinfo {pages} {251} (\bibinfo {year} {1990})}\BibitemShut
  {NoStop}%
\bibitem [{\citenamefont {Zwanzig}(2001)}]{zwanzig_nonequilibrium_2001}%
  \BibitemOpen
  \bibfield  {author} {\bibinfo {author} {\bibfnamefont {R.}~\bibnamefont
  {Zwanzig}},\ }\href@noop {} {\emph {\bibinfo {title} {Nonequilibrium
  statistical mechanics}}},\ \bibinfo {edition} {1st}\ ed.\ (\bibinfo
  {publisher} {Oxford University Press},\ \bibinfo {address} {Oxford},\
  \bibinfo {year} {2001})\BibitemShut {NoStop}%
\bibitem [{\citenamefont {Watanabe}(1960)}]{watanabe_information_1960}%
  \BibitemOpen
  \bibfield  {author} {\bibinfo {author} {\bibfnamefont {S.}~\bibnamefont
  {Watanabe}},\ }\href {https://doi.org/10.1147/rd.41.0066} {\bibfield
  {journal} {\bibinfo  {journal} {IBM J. Res. Dev.}\ }\textbf {\bibinfo
  {volume} {4}},\ \bibinfo {pages} {66} (\bibinfo {year} {1960})}\BibitemShut
  {NoStop}%
\bibitem [{\citenamefont {Groisman}\ \emph {et~al.}(2005)\citenamefont
  {Groisman}, \citenamefont {Popescu},\ and\ \citenamefont
  {Winter}}]{groisman_quantum_2005}%
  \BibitemOpen
  \bibfield  {author} {\bibinfo {author} {\bibfnamefont {B.}~\bibnamefont
  {Groisman}}, \bibinfo {author} {\bibfnamefont {S.}~\bibnamefont {Popescu}},\
  and\ \bibinfo {author} {\bibfnamefont {A.}~\bibnamefont {Winter}},\ }\href
  {https://doi.org/10.1103/PhysRevA.72.032317} {\bibfield  {journal} {\bibinfo
  {journal} {Phys. Rev. A}\ }\textbf {\bibinfo {volume} {72}},\ \bibinfo
  {pages} {032317} (\bibinfo {year} {2005})}\BibitemShut {NoStop}%
\bibitem [{\citenamefont {Zhou}\ \emph {et~al.}(2008)\citenamefont {Zhou},
  \citenamefont {Gong}, \citenamefont {Liu}, \citenamefont {Sun},\ and\
  \citenamefont {Nori}}]{zhou_controllable_2008}%
  \BibitemOpen
  \bibfield  {author} {\bibinfo {author} {\bibfnamefont {L.}~\bibnamefont
  {Zhou}}, \bibinfo {author} {\bibfnamefont {Z.~R.}\ \bibnamefont {Gong}},
  \bibinfo {author} {\bibfnamefont {Y.-x.}\ \bibnamefont {Liu}}, \bibinfo
  {author} {\bibfnamefont {C.~P.}\ \bibnamefont {Sun}},\ and\ \bibinfo {author}
  {\bibfnamefont {F.}~\bibnamefont {Nori}},\ }\href
  {https://doi.org/10.1103/PhysRevLett.101.100501} {\bibfield  {journal}
  {\bibinfo  {journal} {Phys. Rev. Lett.}\ }\textbf {\bibinfo {volume} {101}},\
  \bibinfo {pages} {100501} (\bibinfo {year} {2008})}\BibitemShut {NoStop}%
\bibitem [{\citenamefont {Anza}\ \emph {et~al.}(2020)\citenamefont {Anza},
  \citenamefont {Pietracaprina},\ and\ \citenamefont
  {Goold}}]{anza_logarithmic_2020}%
  \BibitemOpen
  \bibfield  {author} {\bibinfo {author} {\bibfnamefont {F.}~\bibnamefont
  {Anza}}, \bibinfo {author} {\bibfnamefont {F.}~\bibnamefont
  {Pietracaprina}},\ and\ \bibinfo {author} {\bibfnamefont {J.}~\bibnamefont
  {Goold}},\ }\href {https://doi.org/10.22331/q-2020-04-02-250} {\bibfield
  {journal} {\bibinfo  {journal} {Quantum}\ }\textbf {\bibinfo {volume} {4}},\
  \bibinfo {pages} {250} (\bibinfo {year} {2020})}\BibitemShut {NoStop}%
\bibitem [{\citenamefont {Lebowitz}(1993)}]{lebowitz_macroscopic_1993}%
  \BibitemOpen
  \bibfield  {author} {\bibinfo {author} {\bibfnamefont {J.~L.}\ \bibnamefont
  {Lebowitz}},\ }\href {https://doi.org/10.1016/0378-4371(93)90336-3}
  {\bibfield  {journal} {\bibinfo  {journal} {Physica A}\ }\textbf {\bibinfo
  {volume} {194}},\ \bibinfo {pages} {1} (\bibinfo {year} {1993})}\BibitemShut
  {NoStop}%
\bibitem [{\citenamefont {Li}(2017)}]{li_production_2017}%
  \BibitemOpen
  \bibfield  {author} {\bibinfo {author} {\bibfnamefont {S.-W.}\ \bibnamefont
  {Li}},\ }\href {https://doi.org/10.1103/PhysRevE.96.012139} {\bibfield
  {journal} {\bibinfo  {journal} {Phys. Rev. E}\ }\textbf {\bibinfo {volume}
  {96}},\ \bibinfo {pages} {012139} (\bibinfo {year} {2017})}\BibitemShut
  {NoStop}%
\bibitem [{\citenamefont {You}\ and\ \citenamefont
  {Li}(2018)}]{you_entropy_2018}%
  \BibitemOpen
  \bibfield  {author} {\bibinfo {author} {\bibfnamefont {Y.-N.}\ \bibnamefont
  {You}}\ and\ \bibinfo {author} {\bibfnamefont {S.-W.}\ \bibnamefont {Li}},\
  }\href {https://doi.org/10.1103/PhysRevA.97.012114} {\bibfield  {journal}
  {\bibinfo  {journal} {Phys. Rev. A}\ }\textbf {\bibinfo {volume} {97}},\
  \bibinfo {pages} {012114} (\bibinfo {year} {2018})}\BibitemShut {NoStop}%
\bibitem [{\citenamefont {Li}(2019)}]{li_correlation_2019}%
  \BibitemOpen
  \bibfield  {author} {\bibinfo {author} {\bibfnamefont {S.-W.}\ \bibnamefont
  {Li}},\ }\href {https://doi.org/10.3390/e21020111} {\bibfield  {journal}
  {\bibinfo  {journal} {Entropy}\ }\textbf {\bibinfo {volume} {21}},\ \bibinfo
  {pages} {111} (\bibinfo {year} {2019})}\BibitemShut {NoStop}%
\bibitem [{\citenamefont {Sachdev}(2011)}]{sachdev_quantum_2011}%
  \BibitemOpen
  \bibfield  {author} {\bibinfo {author} {\bibfnamefont {S.}~\bibnamefont
  {Sachdev}},\ }\href@noop {} {\emph {\bibinfo {title} {Quantum phase
  transitions}}},\ \bibinfo {edition} {2nd}\ ed.\ (\bibinfo  {publisher}
  {Cambridge University Press},\ \bibinfo {address} {Cambridge ; New York},\
  \bibinfo {year} {2011})\BibitemShut {NoStop}%
\bibitem [{\citenamefont {Strasberg}(2019)}]{strasberg_entropy_2019}%
  \BibitemOpen
  \bibfield  {author} {\bibinfo {author} {\bibfnamefont {P.}~\bibnamefont
  {Strasberg}},\ }\href {http://arxiv.org/abs/1906.09933} {\bibfield  {journal}
  {\bibinfo  {journal} {arXiv:1906.09933}\ } (\bibinfo {year}
  {2019})}\BibitemShut {NoStop}%
\bibitem [{\citenamefont {Zhou}(2008)}]{zhou_irreducible_2008}%
  \BibitemOpen
  \bibfield  {author} {\bibinfo {author} {\bibfnamefont {D.~L.}\ \bibnamefont
  {Zhou}},\ }\href {https://doi.org/10.1103/PhysRevLett.101.180505} {\bibfield
  {journal} {\bibinfo  {journal} {Phys. Rev. Lett.}\ }\textbf {\bibinfo
  {volume} {101}},\ \bibinfo {pages} {180505} (\bibinfo {year}
  {2008})}\BibitemShut {NoStop}%
\bibitem [{\citenamefont {Spohn}(1978)}]{spohn_entropy_1978}%
  \BibitemOpen
  \bibfield  {author} {\bibinfo {author} {\bibfnamefont {H.}~\bibnamefont
  {Spohn}},\ }\href {https://doi.org/10.1063/1.523789} {\bibfield  {journal}
  {\bibinfo  {journal} {J. Math. Phys.}\ }\textbf {\bibinfo {volume} {19}},\
  \bibinfo {pages} {1227} (\bibinfo {year} {1978})}\BibitemShut {NoStop}%
\bibitem [{\citenamefont {de~Groot}\ and\ \citenamefont
  {Mazur}(1962)}]{de_groot_non-equilibrium_1962}%
  \BibitemOpen
  \bibfield  {author} {\bibinfo {author} {\bibfnamefont {S.~R.}\ \bibnamefont
  {de~Groot}}\ and\ \bibinfo {author} {\bibfnamefont {P.}~\bibnamefont
  {Mazur}},\ }\href@noop {} {\emph {\bibinfo {title} {Non-equilibrium
  thermodynamics}}}\ (\bibinfo  {publisher} {North-Holland},\ \bibinfo
  {address} {Amsterdam},\ \bibinfo {year} {1962})\BibitemShut {NoStop}%
\bibitem [{\citenamefont {Kondepudi}\ and\ \citenamefont
  {Prigogine}(2014)}]{kondepudi_modern_2014}%
  \BibitemOpen
  \bibfield  {author} {\bibinfo {author} {\bibfnamefont {D.}~\bibnamefont
  {Kondepudi}}\ and\ \bibinfo {author} {\bibfnamefont {I.}~\bibnamefont
  {Prigogine}},\ }\href {http://doi.wiley.com/10.1002/9781118698723} {\emph
  {\bibinfo {title} {Modern {Thermodynamics}}}}\ (\bibinfo  {publisher} {John
  Wiley \& Sons, Ltd},\ \bibinfo {address} {Chichester, UK},\ \bibinfo {year}
  {2014})\BibitemShut {NoStop}%
\bibitem [{\citenamefont {Nicolis}\ and\ \citenamefont
  {Prigogine}(1977)}]{nicolis_self-organization_1977}%
  \BibitemOpen
  \bibfield  {author} {\bibinfo {author} {\bibfnamefont {G.}~\bibnamefont
  {Nicolis}}\ and\ \bibinfo {author} {\bibfnamefont {I.}~\bibnamefont
  {Prigogine}},\ }\href@noop {} {\emph {\bibinfo {title} {Self-{Organization}
  in {Nonequilibrium} {Systems}}}},\ \bibinfo {edition} {1st}\ ed.\ (\bibinfo
  {publisher} {Wiley},\ \bibinfo {address} {New York},\ \bibinfo {year}
  {1977})\BibitemShut {NoStop}%
\bibitem [{\citenamefont {Jaynes}(1957)}]{jaynes_information_1957}%
  \BibitemOpen
  \bibfield  {author} {\bibinfo {author} {\bibfnamefont {E.~T.}\ \bibnamefont
  {Jaynes}},\ }\href {https://doi.org/10.1103/PhysRev.106.620} {\bibfield
  {journal} {\bibinfo  {journal} {Phys. Rev.}\ }\textbf {\bibinfo {volume}
  {106}},\ \bibinfo {pages} {620} (\bibinfo {year} {1957})}\BibitemShut
  {NoStop}%
\bibitem [{\citenamefont {Srednicki}(1994)}]{srednicki_chaos_1994}%
  \BibitemOpen
  \bibfield  {author} {\bibinfo {author} {\bibfnamefont {M.}~\bibnamefont
  {Srednicki}},\ }\href {https://doi.org/10.1103/PhysRevE.50.888} {\bibfield
  {journal} {\bibinfo  {journal} {Phys. Rev. E}\ }\textbf {\bibinfo {volume}
  {50}},\ \bibinfo {pages} {888} (\bibinfo {year} {1994})}\BibitemShut
  {NoStop}%
\bibitem [{\citenamefont {Polkovnikov}\ \emph {et~al.}(2011)\citenamefont
  {Polkovnikov}, \citenamefont {Sengupta}, \citenamefont {Silva},\ and\
  \citenamefont {Vengalattore}}]{polkovnikov_colloquium_2011}%
  \BibitemOpen
  \bibfield  {author} {\bibinfo {author} {\bibfnamefont {A.}~\bibnamefont
  {Polkovnikov}}, \bibinfo {author} {\bibfnamefont {K.}~\bibnamefont
  {Sengupta}}, \bibinfo {author} {\bibfnamefont {A.}~\bibnamefont {Silva}},\
  and\ \bibinfo {author} {\bibfnamefont {M.}~\bibnamefont {Vengalattore}},\
  }\href {https://doi.org/10.1103/RevModPhys.83.863} {\bibfield  {journal}
  {\bibinfo  {journal} {Rev. Mod. Phys.}\ }\textbf {\bibinfo {volume} {83}},\
  \bibinfo {pages} {863} (\bibinfo {year} {2011})}\BibitemShut {NoStop}%
\bibitem [{\citenamefont {Rigol}\ and\ \citenamefont
  {Srednicki}(2012)}]{rigol_alternatives_2012}%
  \BibitemOpen
  \bibfield  {author} {\bibinfo {author} {\bibfnamefont {M.}~\bibnamefont
  {Rigol}}\ and\ \bibinfo {author} {\bibfnamefont {M.}~\bibnamefont
  {Srednicki}},\ }\href {https://doi.org/10.1103/PhysRevLett.108.110601}
  {\bibfield  {journal} {\bibinfo  {journal} {Phys. Rev. Lett.}\ }\textbf
  {\bibinfo {volume} {108}},\ \bibinfo {pages} {110601} (\bibinfo {year}
  {2012})}\BibitemShut {NoStop}%
\bibitem [{\citenamefont {Aurell}\ and\ \citenamefont
  {Eichhorn}(2015)}]{aurell_von_2015}%
  \BibitemOpen
  \bibfield  {author} {\bibinfo {author} {\bibfnamefont {E.}~\bibnamefont
  {Aurell}}\ and\ \bibinfo {author} {\bibfnamefont {R.}~\bibnamefont
  {Eichhorn}},\ }\href {https://doi.org/10.1088/1367-2630/17/6/065007}
  {\bibfield  {journal} {\bibinfo  {journal} {New J. Phys.}\ }\textbf {\bibinfo
  {volume} {17}},\ \bibinfo {pages} {065007} (\bibinfo {year}
  {2015})}\BibitemShut {NoStop}%
\bibitem [{\citenamefont {Manzano}\ \emph {et~al.}(2018)\citenamefont
  {Manzano}, \citenamefont {Horowitz},\ and\ \citenamefont
  {Parrondo}}]{manzano_quantum_2018}%
  \BibitemOpen
  \bibfield  {author} {\bibinfo {author} {\bibfnamefont {G.}~\bibnamefont
  {Manzano}}, \bibinfo {author} {\bibfnamefont {J.~M.}\ \bibnamefont
  {Horowitz}},\ and\ \bibinfo {author} {\bibfnamefont {J.~M.~R.}\ \bibnamefont
  {Parrondo}},\ }\href {https://doi.org/10.1103/PhysRevX.8.031037} {\bibfield
  {journal} {\bibinfo  {journal} {Phys. Rev. X}\ }\textbf {\bibinfo {volume}
  {8}},\ \bibinfo {pages} {031037} (\bibinfo {year} {2018})}\BibitemShut
  {NoStop}%
\bibitem [{\citenamefont {Esposito}\ \emph {et~al.}(2010)\citenamefont
  {Esposito}, \citenamefont {Lindenberg},\ and\ \citenamefont {Van~den
  Broeck}}]{esposito_entropy_2010}%
  \BibitemOpen
  \bibfield  {author} {\bibinfo {author} {\bibfnamefont {M.}~\bibnamefont
  {Esposito}}, \bibinfo {author} {\bibfnamefont {K.}~\bibnamefont
  {Lindenberg}},\ and\ \bibinfo {author} {\bibfnamefont {C.}~\bibnamefont
  {Van~den Broeck}},\ }\href {https://doi.org/10.1088/1367-2630/12/1/013013}
  {\bibfield  {journal} {\bibinfo  {journal} {New J. Phys.}\ }\textbf {\bibinfo
  {volume} {12}},\ \bibinfo {pages} {013013} (\bibinfo {year}
  {2010})}\BibitemShut {NoStop}%
\bibitem [{\citenamefont {Manzano}\ \emph {et~al.}(2016)\citenamefont
  {Manzano}, \citenamefont {Galve}, \citenamefont {Zambrini},\ and\
  \citenamefont {Parrondo}}]{manzano_entropy_2016}%
  \BibitemOpen
  \bibfield  {author} {\bibinfo {author} {\bibfnamefont {G.}~\bibnamefont
  {Manzano}}, \bibinfo {author} {\bibfnamefont {F.}~\bibnamefont {Galve}},
  \bibinfo {author} {\bibfnamefont {R.}~\bibnamefont {Zambrini}},\ and\
  \bibinfo {author} {\bibfnamefont {J.~M.~R.}\ \bibnamefont {Parrondo}},\
  }\href {https://doi.org/10.1103/PhysRevE.93.052120} {\bibfield  {journal}
  {\bibinfo  {journal} {Phys. Rev. E}\ }\textbf {\bibinfo {volume} {93}},\
  \bibinfo {pages} {052120} (\bibinfo {year} {2016})}\BibitemShut {NoStop}%
\bibitem [{\citenamefont {Alipour}\ \emph {et~al.}(2016)\citenamefont
  {Alipour}, \citenamefont {Benatti}, \citenamefont {Bakhshinezhad},
  \citenamefont {Afsary}, \citenamefont {Marcantoni},\ and\ \citenamefont
  {Rezakhani}}]{alipour_correlations_2016}%
  \BibitemOpen
  \bibfield  {author} {\bibinfo {author} {\bibfnamefont {S.}~\bibnamefont
  {Alipour}}, \bibinfo {author} {\bibfnamefont {F.}~\bibnamefont {Benatti}},
  \bibinfo {author} {\bibfnamefont {F.}~\bibnamefont {Bakhshinezhad}}, \bibinfo
  {author} {\bibfnamefont {M.}~\bibnamefont {Afsary}}, \bibinfo {author}
  {\bibfnamefont {S.}~\bibnamefont {Marcantoni}},\ and\ \bibinfo {author}
  {\bibfnamefont {A.~T.}\ \bibnamefont {Rezakhani}},\ }\href
  {https://doi.org/10.1038/srep35568} {\bibfield  {journal} {\bibinfo
  {journal} {Sci. Rep.}\ }\textbf {\bibinfo {volume} {6}},\ \bibinfo {pages}
  {35568} (\bibinfo {year} {2016})}\BibitemShut {NoStop}%
\bibitem [{\citenamefont {Pucci}\ \emph {et~al.}(2013)\citenamefont {Pucci},
  \citenamefont {Esposito},\ and\ \citenamefont {Peliti}}]{pucci_entropy_2013}%
  \BibitemOpen
  \bibfield  {author} {\bibinfo {author} {\bibfnamefont {L.}~\bibnamefont
  {Pucci}}, \bibinfo {author} {\bibfnamefont {M.}~\bibnamefont {Esposito}},\
  and\ \bibinfo {author} {\bibfnamefont {L.}~\bibnamefont {Peliti}},\ }\href
  {https://doi.org/10.1088/1742-5468/2013/04/P04005} {\bibfield  {journal}
  {\bibinfo  {journal} {J. Stat. Mech.}\ }\textbf {\bibinfo {volume} {2013}},\
  \bibinfo {pages} {P04005} (\bibinfo {year} {2013})}\BibitemShut {NoStop}%
\bibitem [{\citenamefont {Zhang}(2008)}]{zhang_general_2008}%
  \BibitemOpen
  \bibfield  {author} {\bibinfo {author} {\bibfnamefont {Q.}~\bibnamefont
  {Zhang}},\ }\href {https://doi.org/10.1007/s11433-008-0085-7} {\bibfield
  {journal} {\bibinfo  {journal} {Sci. China Ser. G}\ }\textbf {\bibinfo
  {volume} {51}},\ \bibinfo {pages} {813} (\bibinfo {year} {2008})}\BibitemShut
  {NoStop}%
\bibitem [{\citenamefont {Horowitz}\ and\ \citenamefont
  {Sagawa}(2014)}]{horowitz_equivalent_2014}%
  \BibitemOpen
  \bibfield  {author} {\bibinfo {author} {\bibfnamefont {J.~M.}\ \bibnamefont
  {Horowitz}}\ and\ \bibinfo {author} {\bibfnamefont {T.}~\bibnamefont
  {Sagawa}},\ }\href {https://doi.org/10.1007/s10955-014-0991-1} {\bibfield
  {journal} {\bibinfo  {journal} {J. Stat. Phys.}\ }\textbf {\bibinfo {volume}
  {156}},\ \bibinfo {pages} {55} (\bibinfo {year} {2014})}\BibitemShut
  {NoStop}%
\bibitem [{\citenamefont {Kalogeropoulos}(2018)}]{kalogeropoulos_time_2018}%
  \BibitemOpen
  \bibfield  {author} {\bibinfo {author} {\bibfnamefont {N.}~\bibnamefont
  {Kalogeropoulos}},\ }\href {https://doi.org/10.1016/j.physa.2017.12.066}
  {\bibfield  {journal} {\bibinfo  {journal} {Physica A}\ }\textbf {\bibinfo
  {volume} {495}},\ \bibinfo {pages} {202} (\bibinfo {year}
  {2018})}\BibitemShut {NoStop}%
\bibitem [{\citenamefont {Ptaszy{\'n}ski}\ and\ \citenamefont
  {Esposito}(2019)}]{ptaszynski_entropy_2019}%
  \BibitemOpen
  \bibfield  {author} {\bibinfo {author} {\bibfnamefont {K.}~\bibnamefont
  {Ptaszy{\'n}ski}}\ and\ \bibinfo {author} {\bibfnamefont {M.}~\bibnamefont
  {Esposito}},\ }\href {https://doi.org/10.1103/PhysRevLett.123.200603}
  {\bibfield  {journal} {\bibinfo  {journal} {Phys. Rev. Lett.}\ }\textbf
  {\bibinfo {volume} {123}},\ \bibinfo {pages} {200603} (\bibinfo {year}
  {2019})}\BibitemShut {NoStop}%
\end{thebibliography}%

\end{document}
