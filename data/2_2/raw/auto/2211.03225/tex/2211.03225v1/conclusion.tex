\section{Conclusion and future work}

In this paper we have studied automated verification of equivalence properties, encompassing both theoretical and practical aspects. 
We provide tight complexity results for static equivalence, trace equivalence and labelled (bi)similarity (as well as their respective pre-orders), summarised in~Figure~\ref{fig:summary}.
In particular we show that deciding trace equivalence and labelled (bi)similarity for a bounded number of sessions is co\nexp complete for subterm convergent destructor rewrite systems.
Finally, we implement the procedure for deciding trace equivalence in the \deepsec prototype.
As demonstrated through an extensive benchmark (Figure \ref{fig:bench}), our tool is broad in scope and efficient compared to other tools.

Our work opens several directions for future work. 
It would be interesting to lift the restriction of subterm convergent equational theories to allow for more cryptographic primitives.
Similarly, we plan to avoid the restriction to destructor rewrite systems to more general ones.
Also, in recent work~\cite{CCK-csf22} it was shown that labelled similarity characterises \emph{may testing equivalence} in presence of a probabilistic adversary which motivates the extension of our implementation beyond trace equivalence.
The presented procedure for (bi)similarity is however highly non-deterministic and a naive implementation would certainly be inefficient.

Finally, we also plan to extend the \deepsec tool with support for other types of properties.
The extension provided in this paper to simulation and other security relations shows the modularity of our core proof technique, the partition tree, for analysing security properties.
For example, a simplified version of the tree could be used to verify more classical (and simpler) \emph{trace properties}, which would significantly rise the scope and usability of \deepsec.
More generally, since navigation within the tree already allows to verify the complex notion of bisimilarity, we expect that the technique should scale to \emph{hyperproperties} in general.
There are few formalisms and results for such properties in the context of security protocols, but hyperlogics fitting our symbolic model have recently been introduced~\cite{BDM22}. 
They allow for example to model fine variants of equivalence relations to capture subtle hypotheses, and their combination to liveness or real-time properties.
We expect that our proof techniques would allow to study the decidability and complexity of a large fragment of such logics.


