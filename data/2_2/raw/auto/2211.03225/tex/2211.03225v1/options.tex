\usepackage[english]{babel}
\usepackage[T1]{fontenc}
\usepackage{a4wide}
\usepackage{amsmath,amssymb,stmaryrd} % should be loaded before amsthm for using \qedhere in align environment
\usepackage{amsthm,thmtools} % should be loaded before newtxmath
\usepackage{graphicx}
\usepackage{xspace}
\usepackage{multirow}
\usepackage{xcolor}
\usepackage[inline]{enumitem}
\usepackage{xifthen}
\usepackage{tikz}
\usepackage{booktabs}
\usepackage[clock]{ifsym}
\usepackage{rotating}
\usetikzlibrary{shapes.misc}
\usetikzlibrary{shapes.geometric}
\usetikzlibrary{shapes}
\usetikzlibrary{arrows,automata}
\usetikzlibrary{circuits.logic.US}
\usetikzlibrary{arrows}
\usetikzlibrary{decorations.pathmorphing}

\usepackage{colortbl}
\definecolor{light}{RGB}{230,230,230}
\newcommand{\light}[1]{\cellcolor{light}#1}

\RequirePackage[colorlinks=true,linkcolor=black,citecolor=black,urlcolor=black]{hyperref}

\usepackage[ruled]{algorithm2e}
\SetEndCharOfAlgoLine{}




%%%%%%%%%%%%%%%%%%%%%%%%%%%%%%%%%%%%%%%%%%%%%%%%%%%%%%%%%%%%%%%%%%%%%%%%%%%%%%
%%%%%%%%%%%%%%%%%%%%%%%%%%%%%%%%%% THEOREMS %%%%%%%%%%%%%%%%%%%%%%%%%%%%%%%%%%
%%%%%%%%%%%%%%%%%%%%%%%%%%%%%%%%%%%%%%%%%%%%%%%%%%%%%%%%%%%%%%%%%%%%%%%%%%%%%%

\newtheoremstyle{custom}% name of the style to be used
  {0.5\baselineskip}% measure of space to leave above the theorem. E.g.: 3pt
  {0.5\baselineskip}% measure of space to leave below the theorem. E.g.: 3pt
  {}% name of font to use in the body of the theorem
  {}% measure of space to indent
  {\scshape}% name of head font
  {\,.}% punctuation between head and body
  {0.5em}% space after theorem head; " " = normal interword space
  {\thmname{#1}\thmnumber{ #2}\thmnote{ {\normalfont\textsf{(#3)}}}}% Manually specify head

\newtheoremstyle{customrk}% name of the style to be used
  {0.5\baselineskip}% measure of space to leave above the theorem. E.g.: 3pt
  {0.5\baselineskip}% measure of space to leave below the theorem. E.g.: 3pt
  {}% name of font to use in the body of the theorem
  {}% measure of space to indent
  {\itshape}% name of head font
  {\,.}% punctuation between head and body
  {0.5em}% space after theorem head; " " = normal interword space
  {\thmname{#1}\thmnumber{ #2}\thmnote{ {\normalfont\textsf{(#3)}}}}% Manually specify head


\theoremstyle{custom}
% \theoremstyle{definition}
\declaretheorem[name=Theorem,within=section]{theorem}
\declaretheorem[name=Theorem,numbered=no]{theorem*}
\declaretheorem[name=Lemma,numberlike=theorem]{lemma}
\declaretheorem[name=Lemma,numbered=no]{lemma*}
\declaretheorem[name=Proposition,numberlike=theorem]{proposition}
\declaretheorem[name=Proposition,numbered=no]{proposition*}
\declaretheorem[name=Corollary,numberlike=theorem]{corollary}
\declaretheorem[name=Corollary,numbered=no]{corollary*}
\declaretheorem[name=Claim,numberlike=theorem]{claim}
\declaretheorem[name=Claim,numbered=no]{claim*}


\theoremstyle{custom}
% \theoremstyle{definition}
\declaretheorem[name=Definition,within=section]{definition}
\declaretheorem[name=Definition,numbered=no]{definition*}

\theoremstyle{customrk}
% \theoremstyle{remark}
\declaretheorem[name=Example,within=section,qed={\(\triangle\)}]{example}
\declaretheorem[name=Remark,numbered=yes,qed={\(\triangle\)}]{remark}



%%%%%%%%%%%%%%%%%%%%%%%%%%%%%%%%%%%%%%%%%%%%%%%%%%%%%%%%%%%%%%%%%%%%%%%%%%%%%%
%%%%%%%%%%%%%%%%%%%%%%%%%%%%%%% ABOUT LEFTBAR %%%%%%%%%%%%%%%%%%%%%%%%%%%%%%%%
%%%%%%%%%%%%%%%%%%%%%%%%%%%%%%%%%%%%%%%%%%%%%%%%%%%%%%%%%%%%%%%%%%%%%%%%%%%%%%

\usepackage{mdframed}
\mdfdefinestyle{leftbar}{
  leftmargin = 3pt,
  innerleftmargin = 8pt,
  innertopmargin = 0pt,
  innerbottommargin = 0pt,
  innerrightmargin = 0pt,
  rightmargin = 0pt,
  linewidth = 0.6pt,
  topline = false,
  rightline = false,
  bottomline = false,
  linecolor=gray,
  skipabove=1.2\smallskipamount,
  % skipbelow=\medskipamount
}
\surroundwithmdframed[style=leftbar]{proof}

\mdfdefinestyle{margindefstyle}{
	leftmargin = 7pt,
  skipabove = \baselineskip,
  skipbelow = \baselineskip,
  innerleftmargin = 0.5em,
  innertopmargin = 0em,
  innerbottommargin = 2pt,
  innerrightmargin = 0em,
  rightmargin = 0pt,
  linewidth = 1.5pt,
  % linecolor = \thema!80,
  topline = false,
  rightline = false,
  bottomline = false,
}

\newcommand\problemdescr[3][]{
  \ifthenelse{\isempty{#1}}
  {
    \begin{itemize}[itemsep=0pt]
      \item[\(\triangleright\)] \textsc{Input:} {#2}
      \item[\(\triangleright\)] \textsc{Question:} {#3}
    \end{itemize}
  }
  {

    \noindent #1:
    \begin{itemize}[topsep=0pt,itemsep=0pt]
      \item[\(\triangleright\)] \textsc{Input:} {#2}
      \item[\(\triangleright\)] \textsc{Question:} {#3}
    \end{itemize}
  }
}


\usepackage{mathpartir}
\usepackage{pifont,marvosym}
