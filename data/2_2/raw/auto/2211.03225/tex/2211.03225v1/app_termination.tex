\section{Termination proof} \label{app:termination}

\subsection{For mgs}

  The termination of the computation of most general solutions mostly relied on the following result, yet to be proved:

  \propMgsDecrease*

  \begin{proof}
    Consider first the simplification rule \eqref{rule:unifEqfst_simpl} and the ones from Figure \ref{fig:normalisation_formula}.
    They typically apply protocol term substitutions on the constraint system (they also effect recipe disequations that are irrelevant in \(\measureNC(\C^e)\)).
    Note that the applied substitution is always generated from terms already in the constraint system.
    As such \(\mu^1(\C^e\norm) = \mu^1(\C^e)\) and so
    \(\Phi(\C^e\norm)\mu^1(\C^e\norm) = \Phi(\C^e)\mu^1(\C^e)\),
    \(\Solved(\C^e\norm)\mu^1(\C^e\norm) = \Solved(\C^e)\mu^1(\C^e)\) and
    \(\Df(\C^e\norm)\mu^1(\C^e\norm) = \Df(\C^e)\mu^1(\C^e)\).
    Thus, we directly obtain that \(|\measureNC(\C^e\norm)| \leqslant |\measureNC(\C^e)|\).

    Let us look at Rules \eqref{rule:conseq}, \eqref{rule:cons} and \eqref{rule:res} and let us consider \(\C^e \xrightarrow{\Sigma} \C'^e\).
    The rule \eqref{rule:conseq} does not modify the protocol terms of the constraint systems by apply a recipe substitution.
    However, we show an invariant on the constraint systems that any \(\xi,\zeta \in \stc(\C^e)\) are consequence of \(\Solved \cup \Df\) as well as any of their subterms (see Definition \ref{def:well-formed} in Appendix).
    Thus, we deduce from the definition of \(\stc(\C^e)\) that \(\stc(\C^e)\Sigma \subseteq \stc(\C'^e)\).
    To conclude that \(|\measureNC(\C'^e)| \leqslant |\measureNC(\C^e)|\), we rely on the technical Proposition \ref{prop:trans-conseq};
    in other words, if \((\xi,t) \in \conseq(\Solved(\C^e)\mu^1(\C^e) \cup \Df(\C^e)\mu^1(\C^e))\) then \((\xi\Sigma,t) \in \conseq(\Solved(\C'^e)\mu^1(\C'^e) \cup \Df(\C'^e)\mu^1(\C'^e))\).
    Since \(\stc(\C^e)\Sigma \subseteq \stc(\C'^e)\), we conclude that \(|\measureNC(\C'^e)| \leqslant |\measureNC(\C^e)|\).

    By applying the same reasonning for the rule \eqref{rule:res}, we can also show that \(\measureNC(C'^e) \leqslant \measureNC(\C^e)\).
    However, we can even show that this inequality is strict.
    Indeed, using the same the notation in the rule \eqref{rule:res}, this rule is only applied if \(\C^e = \C^e\norm\) and for all \(\xi \in \stc(\C^e) \setminus \{ X\}\), \((\xi,u) \not\in \conseq(\Solved \cup \Df)\).
    Note that \(\C^e = \C^e\norm\) implies that \(\Solved\mu^1 = \Solved\) and \(\Df\mu^1 = \Df\).
    Moreover, it also implies that \(u \in \measureNC{\C^e}\).
    However, in \(\C'^e\), we have that \((\xi,u\mu_1(\C'^e)) \in \conseq(\Solved(\C'^e)\mu^1(\C'^e) \cup \Df(\C'^e)\mu^1(\C'^e))\).
    Moreover, we show another invariant on the constraint system (see Definition \ref{def:well-formed} in Appendix) that ensures us that \(X \in \stc(\C^e)\) and so \(\xi \in \stc(\C'^e)\).
    Hence, we obtain that \(u\mu^1(\C'^e) \not\in \measureNC(\C'^e)\) allowing us to conclude that \(\measureNC(C'^e) < \measureNC(C^e)\).
    By applying the same reasoning, we can also show that \(\measureNC(C'^e) < \measureNC(C^e)\) when the rule \eqref{rule:cons} is applied.
  \end{proof}

\subsection{Exponential measure}

  Another argument left pending is that each component of the measure except the last one can be bounded by an exponential in the DAG size of the parameters of the problem.
  We give a bound for each of them, in particular relying on the bound on \(\measureNC\) proved in the body of the paper.

  \begin{enumerate}
    \item \(\compon[1](\Gamma) \leqslant \dagsize{P,Q}\):

      by definition.

    % \item \(\compon[2](\Gamma) \leqslant \dagsize{P,Q}\):

    %   the number of non-deducibility facts is bounded by the maximal number of internal communications possible in a trace, itself bounded by the sizes of the processes.

    \item \(\compon[2](\Gamma) \leqslant (\dagsize {P} \dagsize {E})^{\dagsize {P}} + (\dagsize {Q} \dagsize {E})^{\dagsize {Q}}\):

      The measure corresponds to the number of symbolic transitions possible from \(P\) and \(Q\) for a given symbolic trace, hence the bound.
      Notice that the part \(\dagsize {E}^{\dagsize {P}}\) is due to the computation of the most general unifiers modulo \(E\) in the symbolic transitions.

    \item \(\compon[3](\Gamma) \leqslant 9\dagsize{P,Q,E}^3\):

      It suffices to observe that \(\setSDF(\C^e) \leqslant \measureNC(\C^e)\) and to use the bound proved in the body of the paper.

    \item \(\compon[4](\Gamma) \leqslant \compon[2](\Gamma)\):

      Trivial.

    \item \(\compon[5](\Gamma) \leqslant \compon[2](\Gamma) \times \dagsize{E}^{\dagsize{E}} \times (18\dagsize{P,Q,E})^{27\dagsize{P,Q,E}^3}\):

      Bounding the size of \(|\setRew(\C^e)|\) can easily be done:
      the number of \(\psi \in \Solved\) possible is bounded by \(|\Solved|\), itself bounded by \(|\setSDF(\C^e)|\).
      The number of rewrite rules, position \(p\) and \(\psi_0 \in \RewF{\xi}{\ell \rightarrow r}{p}\) only depends on the rewrite systems and can be bounded by \(\dagsize{E}^{\dagsize{E}}\).
      Note that the exponential comes mainly from the number of possible positions in \(\ell\).
      We already know that the number of most general solutions is bounded by \((|\Solved(\C^e)| + 1)^{\measureNC(\C^e)}\).
      Combining with all previous results, and with the rough approximation \(9\dagsize{P,Q,E}^3+1 \leqslant 18\dagsize{P,Q,E}^3\), we obtain the above bound.

    \item \(\compon[6](\Gamma) \leqslant |E| \times \compon[2](\Gamma)\):

      To bound this number, we need to recall that we always apply the case distinction rules with the priority ordering
      \eqref{rule:satisfiable} < \eqref{rule:rewrite}.
      Thus, when we apply a rule \eqref{rule:rewrite}, there is no unsolved deduction formula in any of the extended constraint systems (otherwise we should have applied the rule \eqref{rule:satisfiable}).
      It means this measure is bounded by the number of deduction formulas produced by one instance of \eqref{rule:rewrite}.
      By definition, we know that \(|\RewF{\xi}{\ell \rightarrow r}{p}| \leqslant |\R|\) (one formula per rewrite rule).
      Thus, the rule \eqref{rule:rewrite} generates at most \(|E| \times \compon[2](\Gamma)\) deduction formulas.

    \item \(\compon[7](\Gamma) \leqslant \compon[2](\Gamma) \times 2\dagsize{E}(\dagsize{P,Q})^2 (1 + \dagsize {E})^2\):

      The application conditions stipulate that the rule can be applied either (a) on two deduction facts of \(\Solved(\C^e_i)\), or (b) on one deduction fact of \(\Solved(\C^e_i)\) in combination with a construction function symbol.

      Note that even though the rule also consider the existence of a most general solution \(\Sigma \in \mgs(\C^e_i[\Eqfst \mapsto \Eqfst \wedge \Fhyp(\FApply {\Sigma_0} {\psi} {\C^e_i})])\), the number of applications of the rule \eqref{rule:equality} will not depend on the number of possible most general solutions.
      Indeed, consider the case (a) where the rule is applied on two deduction fact \((\xi_1 \dedfact u_1),(\xi_2 \dedfact u_2) \in \Solved(\C^e_i)\).
      Thus, an equality formula with \(\xi_1\Sigma \eqf \xi_2\Sigma\) as head will be added in \(\USolved(\CApply{\Sigma}{\C^e_i})\).
      However, in the application conditions of the rule, we also require that
      \emph{for all \((\clause[S]{H}{\varphi}) \in \USolved(\C^e_i)\), \(H \neq (\xi_1 \eqf \xi_2)\)}.
      Thus, a new application of the rule \eqref{rule:equality} on \(\CApply{\Sigma}{\C^e_i}\) with the same (up to instantiation of \(\Sigma\)) deductions facts from \(\Solved(\CApply{\Sigma}{\C^e_i})\) will be prevented.

      The same situation occurs in case (b) with the condition \emph{for all \((\clause[S]{\zeta_1 \eqf \zeta_2}{\varphi}) \in \USolved(\C^e_i)\), \(\zeta_1 = \xi_1\) implies \(\rootf(\zeta_2) \neq \ffun\)}.
      We therefore conclude that the rule \eqref{rule:equality} can be applied only once per pair of deduction facts in \(\Solved\) and once per deduction fact in \(\Solved\) and function symbol in \(\sigc\).

    \item \(\compon[8](\Gamma) \leqslant \compon[2](\Gamma)\):

      Unsolved equality formulas can be generated by two rules: the case distinction rule \eqref{rule:equality} or the simplification Rule \ref{rule:vector-consequence}.
      However, once again because of the priority order \eqref{rule:satisfiable} < \eqref{rule:equality}, the two rules cannot be triggered simultaneously and the rule \eqref{rule:equality} is only triggered when there is no unsolved equality formulas.
      Note that due to the condition \emph{\(\forall i. \forall (\clause[S]{\zeta_1 \eqf \zeta_2}{\varphi}) \in \USolved_i\), \(\zeta_1 \neq \xi\) or \(\zeta_2 \neq \xi\)} in Rule \ref{rule:vector-consequence},
      two instances of the Rule \ref{rule:vector-consequence} with different recipes \(\zeta\) (e.g. if \(u_1\) can be deducible with two different recipes) cannot be applied sequentially.
      Thus, at any given moment, there is at most one unsolved equality formula per extended constraint system of \(\Gamma\), hence the bound.
  \end{enumerate}

\subsection{Bounding the increase of second order terms}

  In Section \ref{sec:exp-mgs}, Proposition \ref{prop:evol-mgs}, we gave a bound on the increase of the size of most general solutions when applying the rule \eqref{rule:satisfiable}.
  We give here the arguments to extend to the other case distinction rules.
  For that it suffices to generalise this property to a more general set of substitution \(\Sigma\):

  \begin{definition}
    Let \(\C^e\) be an extended constraint system. Let \(\Sigma\) be a second-order substitution.
    We say that \(\Sigma \in \CompatibleSubs(\C^e)\) if \(\dom(\Sigma) \subseteq \vars[2](\Df(\C^e))\) and for all \(X \in \dom(\Sigma)\),
    \(X\Sigma \in \conseq(\Solved(\C^e) \cup \Df' \cup D_\Sigma)\) where
    \(\Df' = \{ X \dedfact u \in \Df(\C^e) \mid X \not\in \dom(\Sigma) \}\) and
    \(D_\Sigma = \{ X \dedfact x \mid x \text{ fresh and } X \in \vars[2](\Sigma) \setminus \vars[2](\C^e)\}\).
  \end{definition}

  Intuitively, \(\CompatibleSubs(\C^e)\) represents the recipe substitutions \(\Sigma\) that can be applied be applied to the constraint system \(\C^e\), i.e. \(\CApply{\Sigma}{\C^e}\),
  and such that the recipes in the of \(\Sigma\) would be consequence of \(\CApply{\Sigma}{\C^e}\).
  Note that \(\mgs(\C^e) \subseteq \CompatibleSubs(\C^e)\).

  By applying Proposition \ref{prop:trans-conseq}, we can show that:
  \begin{equation}
    \text{for all }\Sigma \in \CompatibleSubs(\C^e),
    |\measureNC(\CApply{\Sigma}{\C^e})| \leqslant |\measureNC(\C^e)|
    \label{term:compsubs}
  \end{equation}
  Note that in a set of symbolic processes two extended constraint systems \(\C^e_1, \C^e_2\) always have the same \emph{recipe structure} (Invariant \(\PredStruct\)),
  i.e. \(|\Phi(\C^e_1)| = |\Phi(\C^e_2)|\), \(\vars[2](\C^e_1) = \vars[2](\C^e_2)\) and
  \(\{ \xi \mid (\xi \dedfact u) \in \Solved(\C^e_1)\} = \{ \xi \mid (\xi \dedfact u) \in \Solved(\C^e_2)\}\).
  Thus, we deduce that \(\CompatibleSubs(\C^e_1) = \CompatibleSubs(\C^e_2)\).
  Therefore, we can conclude that for any simplification and case distinction rules, \(|\measureNC(\C^e)|\) never increase for all extended constraint systems in a set of extended symbolic processes.