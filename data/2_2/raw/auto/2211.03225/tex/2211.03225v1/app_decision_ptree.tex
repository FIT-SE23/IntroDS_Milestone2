\section{Decision procedures using partition trees} \label{app:decision-proc-from-ptree}

  We detail in this appendix the technical proofs of correctness of the decision procedures for equivalences of Section~\ref{sec:ptree-eq}, assuming a partition tree \(T\) priorly constructed.

  \subsection{Trace equivalence} \label{app:decision-proc-trace-from-ptree}

    We prove in this section the following theorem:

    \thmTraceEquivPtree*

    The proof of this theorem relies on two technical lemmas extending the properties of the partition tree edges to its branches, i.e., from \(\tstep{}\) to \(\Tstep{}\).
    For example we can generalise as follows the fact that the nodes of the tree are labelled by maximal configurations, i.e., Definition~\ref{def:partition-tree}, Item~\ref{it:PT-parent-concrete-derivation}:

    \begin{lemma}[restate=PTParentConcreteDerivation,name={}] \label{lem:PT-parent-concrete-derivation}
      Assume that \((\P_1,\C_1),n \Tstep{\tr} (\P_1',\C_1'),n'\) and \((\P_2,\C_2) \Sstep {\tr} (\P_2',\C_2')\) with \((\P_2,\C_2) \in \Gamma(n)\).
      We also consider, for all \(i \in \{1,2\}\), a solution \((\Sigma',\sigma_i') \in \Sol[\pi(n')](\C_i')\) such that
      \(\Phi(\C_1') \sigma_1' \StatEq \Phi(\C_2') \sigma_2'\).
      Then we have \((\P_2,\C_2),n \Tstep {\tr} (\P_2',\C_2'),n'\).
    \end{lemma}

    \begin{proof}
      We proceed by induction on the length \(\tr\).
      The case \(\tr = \epsilon\) follows from the saturation of nodes under \(\tau\)-transition (Definition \ref{def:partition-tree}, Item \ref{it:PT-silent}).
      Otherwise we let, with \(\tr = \alpha \cdot \tilde {\tr}\),
      \begin{align*}
        (\P_1,\C_1),n & \Tstep{\alpha} (\tilde{\P_1},\tilde{\C_1}), \tilde{n} \Tstep{\tilde {\tr}} (\P_1',\C_1'),n' &
        (\P_2,\C_2) & \Sstep{\alpha} (\tilde{\P_2},\tilde{\C_2}) \Sstep{\tilde {\tr}} (\P_2',\C_2')
      \end{align*}
      We also consider the restrictions \(\Sigma = \Sigma'_{|\vars[2](n)}\) and \(\tilde{\Sigma} = \Sigma'_{|\vars[2](\tilde{n})}\).
      In particular \(\Sigma \subseteq \tilde{\Sigma}\) and there exist \(\sigma_2,\tilde{\sigma}_1,\tilde{\sigma}_2\) such that
      \begin{align*}
        (\Sigma,\sigma_2) & \in \Sol(\C_2) &
        (\tilde{\Sigma},\tilde{\sigma}_1) & \in \Sol(\tilde{\C}_1) &
        (\tilde{\Sigma},\tilde{\sigma}_2) & \in \Sol(\tilde{\C}_2)
      \end{align*}
      The hypothesis that \(\Phi(\C_1') \sigma_1' \StatEq \Phi(\C_2') \sigma_2'\) also implies that \(\Phi(\tilde{\C}_1) \tilde{\sigma}_1 \StatEq \Phi(\tilde{\C}_2) \tilde{\sigma}_2\).
      Besides since predicates are refined along branches (Definition \ref{def:partition-tree}, Item \ref{it:PT-monotonic}) and are defined on the variables of their configurations (Definition \ref{def:configuration}, Item \ref{it:configuration-pred-dom}),
      we know that \(\Sigma\) and \(\tilde{\Sigma}\) verify \(\pi(n)\) and \(\pi(\tilde{n})\), respectively.

      All in all we can use the maximality of the node \(\tilde{n}\) (Definition \ref{def:partition-tree}, Item \ref{it:PT-parent-concrete-derivation} applied to the edge \(n \xrightarrow{\alpha} \tilde{n}\)),
      which gives that \((\tilde{\P}_2,\tilde{\C}_2) \in \Gamma(\tilde{n})\).
      Hence \((\P_2,\C_2),n \Tstep {\alpha} (\tilde{\P}_2,\tilde{\C}_2),\tilde{n}\) by definition and
      the conclusion then follows from the induction hypothesis applied to the remaining of the traces.
    \end{proof}

    Combined with the soundness and the completeness of the symbolic semantics, this permits to prove one direction of Theorem \ref{thm:trace-equiv-ptree}:

    \begin{proof}[Proof of Theorem \ref{thm:trace-equiv-ptree},
      \ref{it:trace-equiv-ptree-incl}\(\Rightarrow\)\ref{it:trace-equiv-ptree-trace}.]
      Let us consider a trace \(P_1 \Tstep {\tr} (\P_1,\C_1), n\) and exhibit a trace \(P_2 \Tstep {\tr} (\P_2,\C_2), n\).
      We decompose the proof into the following steps:
      \begin{enumerate}
        \item By \emph{soundness} of the symbolic semantics we obtain a trace \(P_1 \Cstep{\tr \Sigma} (\P_1 \sigma_1, \Phi(\C_1) \sigma_1 \norm)\) for an arbitrary solution \((\Sigma,\sigma_1) \in \Sol(\C_1)\).
        \item By \emph{hypothesis \ref{it:trace-equiv-ptree-incl}} there exists a concrete trace \(P_2 \Cstep{\tr \Sigma} (\P,\Phi)\) such that \(\Phi \StatEq \Phi(\C_1) \sigma \norm\).
        \item By \emph{completeness} of the symbolic semantics we obtain a symbolic trace \(P_2 \Sstep {\tr'} (\P_2,\C_2)\) and \((\Sigma',\sigma_2) \in \Sol(\C_2)\) such that \(\tr \Sigma = \tr' \Sigma'\), \(\P_2 \sigma_2 = \P\) and \(\Phi(\C_2) \sigma_2 \norm = \Phi\).
        Due to the form of symbolic actions, we know that there exists a second-order-variable renaming \(\rho\) such that \(\tr = \tr' \rho\);
        in particular \(P_2 \Sstep {\tr} (\P_2,\C_2 \rho)\) and \((\Sigma,\sigma_2) \in \Sol(\C_2 \rho)\).
        \item By \emph{Lemma \ref{lem:PT-parent-concrete-derivation}} we therefore obtain that \(P_2 \Tstep {\tr} (\P_2,\C_2 \rho), n\), which gives the expected conclusion. \qedhere
      \end{enumerate}
    \end{proof}

    The second property of the partition tree we extend is the fact that symbolic transitions are reflected in the tree, i.e., Definition~\ref{def:partition-tree}, Item~\ref{it:PT-child-concrete-derivation}:

    \begin{lemma}[restate=PTChildConcreteDerivation,name={}] \label{lem:PT-child-concrete-derivation}
      Let \(n\) be a node of a partition tree \(T\) and \((\P,\C) \in \Gamma(n)\).
      If \((\P,\C) \Sstep {\tr} (\P',\C')\) and \((\Sigma,\sigma) \in \Sol[\pi(n)](\C')\) then there exist a node \(n'\) and a substitution \(\Sigma'\) such that \((\P,\C),n \Tstep {\tr} (\P',\C'),n'\)
      and \((\Sigma',\sigma) \in \Sol[\pi(n')](\C')\).
    \end{lemma}

    Note that unlike the definition of partition tree, we do not require that \(\Sigma'\) coincides with \(\Sigma\) on \(\vars[2](n)\).
    This additional requirement would not make the lemma false but is unnecessary to prove Theorem~\ref{thm:trace-equiv-ptree}.
    The lemma is proved by induction on \(\tr\) below:
    
    \begin{proof}
      We proceed by induction on the length of \(\tr\).
      If \(\tr = \epsilon\) it suffices to choose \(n = n'\) and the conclusion immediately follows.
      Otherwise let us decompose the symbolic trace into
      \begin{align*}
        (\P,\C) & \Sstep{\tilde{\tr}} (\tilde{\P},\tilde{\C}) \Sstep {\alpha} (\P',\C') &
        \tr & = \tilde{\tr} \cdot \alpha
      \end{align*}
      Note that \((\Sigma_{|\vars[2](\tilde{n})}, \sigma_{|\vars[1](\tilde{n})}) \in \Sol(\tilde{\C})\), and \(\Sigma_{|\vars[2](\tilde{n})}\) verifies \(\pi(n)\) by definition of a configuration
      (since \(\Sigma\) verifies it and has the same restriction to \(\vars[2](\Gamma(n))\) as \(\Sigma_{|\vars[2](\tilde{n})}\)).
      By induction hypothesis we therefore obtain \(\tilde{n},\tilde{\Sigma}\) such that \((\P,\C),n \Tstep {\tilde{\tr}} (\tilde{\P},\tilde{\C}),\tilde{n}\)
      and \((\tilde{\Sigma},\sigma_{|\vars[1](\tilde{n})}) \in \Sol[\pi(\tilde{n})](\tilde{\C})\).
      %
      Let us then consider the extension
      \[\tilde{\Sigma}^e = \tilde{\Sigma} \cup \Sigma_{|\vars[2](n') \smallsetminus \vars[2](\tilde{n})}\]
      %
      To conclude the proof it suffices to apply the Item \ref{it:PT-child-concrete-derivation} of Definition \ref{def:partition-tree} to the symbolic transition \((\tilde{\P},\tilde{\C}) \Sstep {\alpha} (\P',\C')\)
      and the solution \((\tilde{\Sigma}^e,\sigma)\);
      what remains to prove is therefore that we effectively have  \((\tilde{\Sigma}^e,\sigma) \in \Sol[\pi(\tilde{n})](\C')\).
      First of all we indeed have by construction \(\dom(\tilde{\Sigma}^e) = \vars[2](\C')\) and \(\dom(\sigma) = \vars[1](\C')\).
      We also know that \(\tilde{\Sigma}^e\) satisfies the predicate \(\pi(\tilde{n})\) because \(\tilde{\Sigma} = \tilde{\Sigma}^e_{|\vars[2](\tilde{n})}\) satisfies it.
      The first-order solution \(\sigma\) satisfies the constraints of \(\Eqfst(\C')\) since \((\Sigma,\sigma) \in \Sol(\C')\) by hypothesis.
      Finally we let \(\varphi \in \Df(\C')\) and prove that \((\Phi(\C'),\tilde{\Sigma}^e,\sigma) \models \varphi\):

      \caseitem{\emph{case 1:} \(\varphi = (X \dedfact x) \in \Df(\tilde{\C})\)}

        The conclusion follows from the fact that \((\tilde{\Sigma},\sigma_{|\vars[1](\tilde{n})}) \in \Sol(\tilde{\C})\).

      \caseitem{\emph{case 2:} \(\varphi = (X \dedfact x) \in \Df(\C') \smallsetminus \Df(\tilde{\C})\)}

        The conclusion follows from the fact that \((\Sigma,\sigma) \in \Sol(\C')\).

      \caseitem{\emph{case 3:} \(\varphi = \forall X.\, X \ndedfact x\)}

        We have to prove that \(x \sigma\) is not deducible from the frame \(\Phi(\C') \sigma\), which is a consequence from the fact that \((\Sigma,\sigma) \in \Sol(\C')\).
    \end{proof}

    Using again the soundness and completeness of the symbolic semantics, we can finally derive the other direction of Theorem~\ref{thm:trace-equiv-ptree}.

    \begin{proof}[Proof of Theorem~\ref{thm:trace-equiv-ptree},
      \ref{it:trace-equiv-ptree-trace}\(\Rightarrow\)\ref{it:trace-equiv-ptree-incl}.]
      Let us consider a trace \(P_1 \Cstep {\tr} (\P,\Phi)\) and exhibit a trace \(P_2 \Cstep {\tr} (\Q,\Psi)\) such that \(\Phi \StatEq \Psi\).
      We decompose the proof into the following steps:
      \begin{enumerate}
        \item By \emph{completeness} of the symbolic semantics we obtain a symbolic trace \(P_1 \Sstep {\tr_s} (\P_1,\C_1)\) and \((\Sigma,\sigma_1) \in \Sol(\C)\) such that \(\tr_s \Sigma = \tr\), \(\P_1 \sigma_1 = \P\) and \(\Phi(\C_1) \sigma_1 \norm = \Phi\).
        \item By \emph{Lemma \ref{lem:PT-child-concrete-derivation}} we then obtain a partition-tree trace \(P_1 \Tstep {\tr_s} (\P_1,\C_1),n\) and \(\Sigma'\) such that \((\Sigma',\sigma_1) \in \Sol[\pi(n)](\C_1)\).
        \item By \emph{hypothesis \ref{it:trace-equiv-ptree-trace}} there also exists a partition-tree trace \(P_2 \Tstep {\tr_s} (\P_2,\C_2),n\).
        By definition of a configuration we also know that there exists \(\sigma_2\) such that \((\Sigma',\sigma_2) \in \Sol[\pi(n)](\C_2)\) and \(\Phi(\C_1) \sigma_1 \StatEq \Phi(\C_2) \sigma_2\).
        \item By \emph{soundness} of the symbolic semantics applied to we then obtain a concrete trace \(P_2 \Cstep {\tr_s \Sigma'} (\Q,\Psi)\) with \(\Q = \P_2 \sigma_2\) and \(\Psi = \Phi(\C_2) \sigma_2 \norm \ \StatEq\ \Phi(\C_1) \sigma_1 \norm\ = \Phi\).
      \end{enumerate}
      However we may have \(\tr_s\Sigma' \neq \tr\) and, to conclude the proof, we prove that \(P_2 \Cstep {\tr} (\Q,\Psi)\) as well.
      For that it suffices to prove that \(\tr \Psi \norm = \tr_s \Sigma' \Psi\norm \), that is, although the recipes or \(\tr\) and \(\tr_s\Sigma'\) are different they produce the same first-order terms.
      Since \(\Phi\) and \(\Psi\) are statically equivalent, \(\tr = \tr_s \Sigma\) and \(\Phi = \Phi(\C_1) \sigma_1 \norm\), it suffices to prove that \(\tr_s \Sigma \Phi(\C_1) \sigma \norm = \tr_s \Sigma' \Phi(\C_1) \sigma\norm \).
      Let \(X \in \vars[2](\tr_s)\).
      A quick look at the rules of the symbolic semantics shows that there exists a deduction fact \((X \dedfact x) \in \Df(\C_1)\).
      In particular, since \((\Sigma,\sigma_1)\) and \((\Sigma',\sigma_1)\) are both solutions of \(\C_1\) we have
      \(X \Sigma \Phi(\C_1) \sigma_1 \norm = x \sigma_1 \norm = X \Sigma' \Phi(\C_1) \sigma_1 \norm\),
      hence the conclusion.
    \end{proof}

  \subsection{Simulations} \label{app:decision-proc-bisim-from-ptree}

    In this section, we now prove the main theorem at the basis of the decision procedure for simulations and its variants:

    \thmLabBisPtree*
  
    We only prove the case of labelled bisimilarity, as the proof for simulation is analogue.
    For that, we prove the following technical lemma by induction on the structure of the (symbolic) witness; 
    this lemma is a stronger version of the theorem for the purpose of managing the induction invariant.

    \begin{lemma} \label{lem:lab-bis-ptree}
      Let \(n\) be a node of a partition tree \(T\) and \(A_0,A_1 \in \Gamma(n)\).
      We let \(A_i = (\P_i,\C_i)\) and \(\Sigma,\sigma_0,\sigma_1\) such that \((\Sigma,\sigma_i) \in \Sol[\pi(n)](\C_i)\).
      If \(A_i^c = (\P_i \sigma_i,\Phi(\C_i) \sigma_i \norm)\), the following points are equivalent:
      \begin{enumerate}
        \item \label{it:lab-bis-ptree-equiv}
        \(A_0^c \not \LabBis A_1^c\)
        \item \label{it:lab-bis-ptree-witness}
        there exist a symbolic witness \(\witness_s\) for \((A_0,A_1,n)\) and a solution \(\fsol \in \Sol(\witness_s)\) such that \(\fsol(\rootf(\witness_s)) = \Sigma\)
      \end{enumerate}
    \end{lemma}

    To prove this lemma we first observe that, by definition of a configuration (Definition~\ref{def:configuration}), \(A_0^c \StatEq A_1^c\) because these two processes are obtained by instanciating two symbolic processes from a same node \(n\) with a common solution \(\Sigma\).
    We then prove the two directions separately.

    \medskip

    \begin{bigproof}[Proof of Lemma \ref{lem:lab-bis-ptree}, \ref{it:lab-bis-ptree-equiv}\(\Rightarrow\)\ref{it:lab-bis-ptree-witness}]
      We prove the result by induction on \(|\P_0,\P_1|\).
      The conclusion is immediate if \(|\P_0,\P_1| = 0\) as it yields a contradiction:
      the multisets \(\P_0\) and \(\P_1\) can only contain null processes and the fact that \(A_0^c \StatEq A_1^c\) justifies that \(A_0^c \LabBis A_1^c\).
      %
      Otherwise we let by Proposition \ref{prop:concrete-witness} a witness \(\witness\) of \((A_0^c,A_1^c)\).
      Thus, by definition, there exist \(b \in \{0,1\}\) and a transition \(A_b^c \cstep{\alpha} A_b^{\prime c} = (\Q, \Phi)\)
      such that for all traces \(A_{1-b}^c \Cstep{\bar{\alpha}} A_{1-b}^{\prime c}\) such that \(A_0^{\prime c} \StatEq A_0^{\prime c}\),
      we have \((A_0^{\prime c}, A_1^{\prime c}) \in \witness\)
      (and therefore \(A_0^{\prime c} \not \LabBis A_1^{\prime c}\) by Proposition \ref{prop:concrete-witness}).
      %
      Let us now construct a symbolic witness \(\witness_s\) of \((A_0,A_1,n)\) and a suitable solution \(\fsol\).

      \caseitem{\emph{case 1:} \(\alpha \neq \tau\)}

        By completeness of the symbolic semantics (Proposition \ref{prop:symbolic-sound-complete}) applied to the transition \(A_b^c \cstep{\alpha} A_b^{\prime c}\), we let a symbolic transition \(A_b \sstep {\alpha_s} A_b' = (\Q_s,\C)\) and a solution \((\Sigma', \sigma') \in \Sol (\C)\)
        such that \(\Sigma \subseteq \Sigma'\),
        \(\alpha = \alpha_s \Sigma'\), \(\Q = \Q_s \sigma'\) and \(\Phi = \Phi(\C) \sigma' \norm\).
        Note that by hypothesis \(\Sigma\) verifies \(\pi(n)\) and, therefore, so does its extension \(\Sigma'\) (since by definition predicates are stable by domain extension, recall Definition \ref{def:configuration}).
        Then since the symbolic transition \(A_b \sstep {\alpha_s} A_b'\) is reflected in \(T\) (in the sense of Definition \ref{def:partition-tree}, Item \ref{it:PT-child-concrete-derivation}),
        we obtain a transition \(A_b, n \tstep {\alpha_s} A_b', n'\) and \(\Sigma''\) such that \((\Sigma'',\sigma') \in \Sol [\pi(n')] (\C)\) and \(\Sigma_{|\vars[2](n)}'' = \Sigma_{|\vars[2](n)}' \quad (= \Sigma)\).

        \caseitem{\emph{case 1a}: there exist no \(A_{1-b}'\) such that \(A_{1-b},n \Tstep {\alpha_s} A_{1-b}', n'\)}

          Then we define \(\witness_s\) to be the tree whose root is labelled \((A_0,A_1,n)\) and that has a unique child labelled \((A_b',n')\).
          We then consider \(\fsol\) mapping the child to \(\Sigma''\) and the root to \(\Sigma''_{|\vars[2](n)} = \Sigma\), which is a solution of \(\witness_s\).
          % Besides \((\Sigma''_{|\vars[2](n)},\sigma_b) \in \Sol[\pi(n)](\C_b)\) because \((\Sigma'',\sigma') \in \Sol[\pi(n')](\C')\) and \(\sigma_b = \sigma'_{|\vars[1](n)}\).
          % This shows in particular that all recipes in \(\im(\Sigma''_{|\vars[2](n)})\) deduce the same first-order terms as those in \(\im(\Sigma)\) from the frame \(\Phi(\C_b) \sigma_b\);
          % but since \(\Phi(\C_0) \sigma_0 \StatEq \Phi(\C_1) \sigma_1\) we know that this will also be the case within the frame \(\Phi(\C_{1-b}) \sigma_{1-b}\), therefore \((\Sigma''_{|\vars[2](n)},\sigma_{1-b}) \in \Sol[\pi(n)](\C_{1-b})\) as well.

        \caseitem{\emph{case 1b}: otherwise}

          In this case we define \(\witness_s\) as follows.
          Its root is labelled \((A_0,A_1,n)\) and its children are all the nodes labelled \((A_0',A_1',n')\), with \(A_{1-b},n \Tstep {\alpha_s} A_{1-b}', n'\).
          For each such node, as explained in the beginning of the proof we have \(A_0^{\prime c} \not\LabBis A_1^{\prime c}\) which permits to apply the induction hypothesis with the solution \(\Sigma''\).
          This gives a symbolic witness rooted in this node and \(\fsol\) a solution mapping this node to \(\Sigma''\).
          Let us write more explicitly these witnesses \(\witness_s^1, \ldots, \witness_s^p\) and \(\fsol^1, \ldots, \fsol^p\) the corresponding solutions.
          To conclude it then suffices to choose \(\witness_s^1, \ldots, \witness_s^p\) as the children of the root of \(\witness_s\), and \(\fsol\) maps the root of \(\witness_s\) to \(\Sigma''_{|\vars[2](n)} = \Sigma\) and each node \(n\) of \(\witness_s^i\) to \(\fsol^i(n)\).

        \caseitem{\emph{case 2:} \(\alpha = \tau\)}

          Analogue to case 1 in the simpler case where \(n = n'\) and \(\Sigma = \Sigma' = \Sigma''\).
          Note also that the analogue of case 1a cannot arise.
    \end{bigproof}

    \begin{bigproof}[Proof of Lemma \ref{lem:lab-bis-ptree}, \ref{it:lab-bis-ptree-witness}\(\Rightarrow\)\ref{it:lab-bis-ptree-equiv}]
        We construct a concrete witness \(\witness\) of \((A_0^c, A_1^c)\) as follows:
        \[\witness = \left\{\begin{array}{r|l}
          \multirow{2}*{%
            \(\left((\P_0 \sigma_0,\Phi(\C_0) \sigma_0 \norm),
            (\P_1 \sigma_1,\Phi(\C_1) \sigma_1 \norm)\right)\)}
            & N \text{ node of \(\witness_s\) labelled } ((\P_0,\C_0),(\P_1,\C_1),n), \\
            & \forall i \in \{0,1\}, (\fsol(N), \sigma_i) \in \Sol(\C_i)
        \end{array}\right\}\]
        The fact that all \((B_0,B_1) \in \witness\) verify \(B_0 \StatEq B_1\) follows from Definition \ref{def:configuration}.
        Then let us consider \(((\P_0 \sigma_0,\Phi(\C_0) \sigma_0 \norm),
        (\P_1 \sigma_1,\Phi(\C_1) \sigma_1 \norm)) \in \witness\) using the notations of the construction of \(\witness\) above.
        By definition of a symbolic witness there exists \(b \in \{0,1\}\) and a transition \((\P_b,\C_b), n \tstep{\alpha} (\P_b',\C_b'), n'\) such that:

        \caseitem{\emph{case 1:} \(N = \rootf(\witness_s)\) has a unique child \(N'\) labelled \(\{(\P_b',\C_b')\},n'\)}

          Then consider the concrete transition \((\P_b \sigma_b,\Phi(\C_b) \sigma_b \norm) \cstep{\alpha \fsol(N')} (\P_b' \sigma_b',\Phi(\C_b') \sigma_b' \norm)\) obtained by soundness of the symbolic semantics (Proposition \ref{prop:symbolic-sound-complete})
          where \((\fsol(N'),\sigma_b') \in \Sol[\pi(n')](\C_b')\).
          By completeness of the symbolic semantics (which is possible to apply since \(\fsol(N) \subseteq \fsol(N')\) by definition of a solution of a symbolic witness) and maximality of the node \(n'\) (Definition \ref{def:partition-tree}, Item \ref{it:PT-parent-concrete-derivation}), there cannot exist any concrete trace of the form
          \[(\P_{1-b} \sigma_{1-b},\Phi(\C_{1-b}) \sigma_{1-b} \norm) \cstep{\alpha \fsol(N')} (\P,\Phi)\]
          such that \(\Phi \StatEq \Phi(\C_b') \sigma_b'\), hence the conclusion.

        \caseitem{\emph{case 2:} the children of \(N = \rootf(\witness_s)\) are all the nodes \(N'\) labelled \(((\P_0',\C_0'), (\P_1',\C_1'),n')\), where \((\P_{1-b},\C_{1-b}), n \Tstep{\alpha} (\P_{1-b}',\C_{1-b}'), n'\) (and there is at least one such child)}

          Let \(N'\) be an arbitrary child of \(N\), labelled \(((\P_0',\C_0'), (\P_1',\C_1'),n')\) with the above notations.
          As in the previous case we consider the concrete transition obtained by soundness of the symbolic semantics, \((\P_b \sigma_b,\Phi(\C_b) \sigma_b \norm) \cstep{\alpha \fsol(N')} (\P_b' \sigma_b',\Phi(\C_b') \sigma_b' \norm)\).
          Then let us consider a trace of the form
          \begin{align*}
            (\P_{1-b} \sigma_{1-b},\Phi(\C_{1-b}) \sigma_{1-b} \norm) \Cstep{\alpha \fsol(N')} (\P,\Phi) = A & &
            \Phi \StatEq \Phi(\C_b') \sigma_b'
          \end{align*}
          Our goal is to prove that \((A, (\P_b \sigma_b,\Phi(\C_b) \sigma_b \norm)) \in \witness\).
          Using the completeness of the symbolic semantics and the maximality of \(n'\) as in the previous case, we obtain a partition-tree trace \((\P_{1-b}, \C_{1-b}),n \Tstep{\alpha} (\P_{1-b}'',\C_{1-b}''), n'\) and \((\fsol(N'),\sigma_{1-b}'') \in \Sol[\pi(n')](\C_{1-b}'')\)
          with \(\P = \P_{1-b}'' \sigma_{1-b}''\) and \(\Phi = \Phi(\C_{1-b}'')\sigma_{1-b}'' \norm\).
          By hypothesis there therefore exists a node \(N''\) labelled \(((\P_b',\C_b'),(\P_{1-b}'',\C_{1-b}''),n')\) a child of \(N\).
          The conclusion then follows from the fact that \(\fsol(N') = \fsol(N'')\) by definition of a solution of a symbolic witness.
    \end{bigproof}
