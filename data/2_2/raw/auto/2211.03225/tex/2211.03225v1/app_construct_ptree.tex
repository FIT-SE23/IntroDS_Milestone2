\section{Correctness of the generation of partition trees} \label{app:ptree}

\subsection{Invariants of the procedure} \label{app:invariants}

In this section we present some additional properties that are verified all along the procedure by the nodes of the partition tree under construction.
Such nodes are sets of extended symbolic processes (i.e. tuples \((\P,\C,\C^e)\) with \(\P\) a process, \(\C\) a constraint system and \(\C^e\) an extended constraint system).
Understanding the technical details of these invariants is not necessary to understand the algorithm itself, however most of our subprocedure (e.g. the generation of most general solutions) are only correct in their context.

\paragraph{Invariant 1: Well-formedness}
  The first invariant is about the shape of the extended constraint systems.
  Two important properties are that
  all equations of \(\Eqfst\) and \(\Eqsnd\) are trivially satisfiable (they are essentially of the form \(x \eqs u\) where \(x\) appears nowhere else in the constraint system) and that those of \(\Eqsnd\) only use terms that can be constructed from the knowledge base (i.e. they are consequences of \(\Solved \cup \Df\)).

  \begin{definition} \label{def:well-formed}
    We define the predicate \(\PredWellFormed\) on extended constraint systems as follows;
    we have that \(\PredWellFormed((\Phi, \Df, \Eqfst, \Eqsnd, \Solved, \USolved))\) holds when
    \begin{itemize}
      \item Variables in \(\Solved\) and \(\USolved\): \(\vars[2](\Solved,\USolved) \subseteq \vars[2](\Df)\)
      \item Equation: \(\mgu(\Eqn[i]) \neq \bot\), \(\dom(\mgu(\Eqn[i])) \cap \vars[i](\Df) = \emptyset\), \(\vars[i](\im(\mgu(\Eqn[i]))) \subseteq \vars[i](\Df)\).
      \item Solution is consequence: \(\im(\mgu(\Eqsnd)) \subseteq \conseq(\Solved \cup \Df)\).
      \item Shape of \(\Solved\): For all \(\psi = (\xi \dedfact u) \in \Solved\), \(\psi \in \USolved\), \(u \notin \X[1]\), \(u \in \subterms(\Phi)\) and \(\subterms(\xi) \subseteq \conseq(\Solved \cup \Df)\).
      \item Shape of \(\USolved\): For all \(\clause[S]{H}{\varphi} \in \USolved(\C)\), \(S\) is empty and \(\varphi\) only contains syntactic equations as hypothesis, i.e. no deduction facts.
      Moreover \(\strsubterms[2](H) \subseteq \conseq(\Solved \cup \Df)\),
      and if \(H = \xi \dedfact u\) then either \(u \in \subterms(\Phi)\) or there exists a recipe \(\zeta\) such that \((\zeta,u) \in \conseq(\Solved \cup \Df)\).
    \end{itemize}
  \end{definition}

  This invariant is lifted to sets of extended constraint systems in the natural way, i.e. \(\PredWellFormed(S)\) holds \textit{iff} for all \(\C \in S\), \(\PredWellFormed(\C)\) holds.

\paragraph{Invariant 2: Formula soundness}
  The second invariant states that any substitution that satisfies the deduction facts of \(\Df\) and the equalities of \(\Eqfst\) also satisfies all formulas of \(\Solved \cup \USolved\).
  This means that the procedure only adds correct formulas to the constraint system, sometimes under some hypothese for the formulas of \(\USolved\).

  \begin{definition} \label{def:invariant_correctness_formula}
    We define the predicate \(\PredCorrectFormula\) on extended constraint systems as follows;
    we have that \(\PredCorrectFormula((\Phi, \Df, \Eqfst, \Eqsnd, \Solved, \USolved))\) holds when for all \(\psi \in \Solved \cup \USolved\), for all substitutions \(\Sigma,\sigma\),
    if
    \[(\Phi,\Sigma,\sigma) \models \Df' \wedge
      \equality{\Eqfst} \qquad \text{with}\ \Df' = \{\psi' \in \Df \mid \vars[2](\psi') \subseteq \vars[2](\psi)\}\]
    where \(\equality{\Eqfst}\) if the set of equations of \(\Eqfst\), then \((\Phi,\Sigma,\sigma) \models \psi\).
  \end{definition}

\paragraph{Invariant 3: Formula completeness}
  Given a formula \(\psi = \clause{H}{\varphi}\) in \(\USolved\), the soundness invariant above states that when some substitutions satisfy \(\varphi\) then they also satisfy the head \(H\).
  The next invariant can be seen as a kind of converse statement:
  when some substitutions satisfy the head \(H\), there exists a formula \(\psi' \in \USolved\) (not necessarily the same) that has the same head and whose hypotheses are satisfied.
  This means that the procedure always covers all cases when generating the potential hypotheses of a given head \(H\).

  \begin{definition} \label{def:invariant_complete_formula}
    In this definition, we say that \((\Phi,\Sigma,\sigma)\) \emph{weakly satisfies} a head \(H\) when:
    \begin{itemize}
      \item if \(H = \xi \dedfact u\) then \(\msg(\xi \Sigma \Phi \sigma)\), i.e. the recipe \(\xi\) leads to a valid message
      \item if \(H = \xi \eqf \zeta\) then \((\Phi,\Sigma,\sigma)\) satisfies \(H\) in the usual sense, i.e. the recipes \(\xi,\zeta\) lead to the same valid message.
    \end{itemize}
    Given a set of extended processes \(\Gamma\), the predicate \(\PredCompleteFormula(\Gamma)\) holds when for all \(\C^e_1,\C^e_2 \in \Gamma\), for all \((\clause{H}{\varphi}) \in \USolved(\C^e_1)\),
    for all \((\Sigma,\sigma) \in \Sol(\C^e_2)\), if  \((\Phi(\C^e_2),\Sigma,\sigma)\) weakly satisfies \(H\)
    then there exists \((\clause{H'}{\varphi'}) \in \USolved(\C^e_2)\) such that \(H' \receq H\) and \((\Phi(\C^e_2),\Sigma,\sigma) \models \varphi'\).
  \end{definition}

\paragraph{Invariant 4: Knowledge-base saturation}
  The next invariant states that the knowledge base \(\Solved\) is saturated, i.e. that we do not miss solutions by imposing that they are constructed from \(\Solved\) (see Definition \ref{def:ext-sol}).

  \begin{definition} \label{def:invariant_consequence}
    We define the predicate \(\PredConseq\) on extended constraint systems as follows.
    We have that \(\PredConseq(\C)\) holds when, considering the minimal \(k\) such that \(\vars[2](\Df(\C)) \subseteq \Xsndi{k}\),
    %
    for all \(\ffun/n \in \sigd\), for all \((\xi_1,u_1),\ldots, (\xi_n,u_n) \in \conseq(\Solved(\C))\) such that \(\xi_1,\ldots, \xi_n \in \recipeset_k\),
    if \(\ffun(u_1,\ldots, u_n) \norm\) is a constructor term then there is \(\xi \in \recipeset_k\) such that \((\xi,\ffun(u_1,\ldots, u_n) \norm) \in \conseq(\Solved(\C) \cup \Df(\C))\).
  \end{definition}

\paragraph{Invariant 5: Preservation of solutions}
  Finally the last invariant states that all the constraint solving performed on the additional data of extended constraint systems (\(\Eqsnd\), \(\Solved\), \(\USolved\)) preserves the solutions.
  That is, in an extended symbolic process \((\P,\C,\C^e)\), where \(\C\) is the constraint system obtained by only collecting constraints during the execution of the process \(\P\) (i.e. without additional constraint solving), all solutions of \(\C^e\) are solutions of \(\C\).

  \begin{definition} \label{def:invariant_solution}
    We define the predicate \(\PredSymb\) defined on extended symbolic processes where \(\PredSymb((\P,\C,\C^e))\) holds when
    \begin{itemize}
      \item for all \(i \in \N\), \(\vars[2](\C) \subseteq \Xsndi{i}\) iff \(\vars[2](\C^e) \subseteq \Xsndi{i}\).
      \item for all \((\Sigma,\sigma) \in \Sol(\C^e)\), \((\Sigma_{|\vars[2](\C)},\sigma_{|\vars[1](\C)}) \in \Sol(\C)\).
    \end{itemize}
  \end{definition}

  We restrict the substitutions to the variables of \(\C\) since our extended constraint system may introduce new variables (e.g. by applying most general solutions) but all these variables are uniquely defined by the instantiation of the variables of \(\C\).

\paragraph{Invariant 6: Component structure}
  Finally we state an invariant on components stating that all of their constraint systems have the same second-order structure.
  This invariant is preserved during the procedure by the fact that the constraint-solving rules that modify \(\Solved\) or \(\Eqsnd\) are always applied to the entire components.

  \begin{definition} \label{def:invariant_structure}
    We define the predicate \(\PredStruct\) defined on sets of extended symbolic processes where \(\PredSymb(\Gamma)\) holds when for all \((\P_1,\C_1,\C^e_1),(\P_2,\C_2,\C^e_2) \in \Gamma\),
    \begin{itemize}
      \item \(\dom(\Phi(\C^e_1)) = \dom(\Phi(\C^e_2))\)
      \item \(\vars[2](\C^e_1) = \vars[2](\C^e_2)\)
      \item \(\{\xi \mid (\xi \dedfact u) \in \Solved(\C^e_1)\} = \{ \xi \mid (\xi \dedfact u) \in \Solved(\C^e_2)\}\)
    \end{itemize}
  \end{definition}

\paragraph{Overall invariant}
  As we mentioned, all invariants are lifted to sets of extended symbolic processes in the natural way if needed.
  We refer as
  \[\PredAll(\Gamma) =
    \PredWellFormed(\Gamma) \wedge
    \PredCorrectFormula(\Gamma) \wedge
    \PredCompleteFormula(\Gamma) \wedge
    \PredConseq(\Gamma) \wedge
    \PredSymb(\Gamma) \wedge
    \PredStruct(\Gamma)\]
  the invariant of the whole procedure on the nodes of the partition tree (i.e. sets of extended symbolic processes).

\subsection{Preservation of the invariants} \label{app:invariants-preserve}

In this section we prove that the invariants of the procedure stated in Section \ref{app:invariants} are preserved all along the procedure.
First of all we prove the case of the case distinction rules:

\begin{lemma}
  Let \(\S\) be a set of set of extended symbolic processes such that \(\PredAll(\S)\).
  Let \(\S \rightarrow \S'\) by applying one case distinction rule and then normalising the result with the simplification rules.
  We have that \(\PredAll(\S')\).
\end{lemma}

For the proof we let \(\Gamma' \in \S'\) and write \(\Gamma \in \S\) the set from which \(\Gamma'\) is originated, i.e. \(\Gamma'\) is one of the sets obtained after applying one case distinction rule to \(\Gamma\) (either the positive or the negative branch) and then normalising with the simplification rules.

\begin{proof}[Proof of preservation of \(\PredWellFormed\)]
  Let \(\C^{e\prime} = (\Phi, \Df, \Eqfst, \Eqsnd, \Solved, \USolved)\) for some \((\P',\C',\C^{e\prime}) \in \Gamma'\).
  We consider each item of the definition of the predicate (Definition \ref{def:well-formed}) and show that \(\C^{e\prime}\) verifies them.

  \caseitem{\emph{property 1 (Variables in \(\Solved\) and \(\USolved\)):}
  \(\vars[2](\Solved,\USolved) \subseteq \vars[2](\Df)\).}

    We first observe that this property is preserved by all simplification rules:
    therefore it sufficies to prove that it is preserved by application of case distinction rules.
    For that we also observe that if \(\C^e\) is an extended constraint system verifying this property then for all second-oder substitutions \(\Sigma\), \(\CApply{\Sigma}{\C^e}\) verifies it as well (which is precisely why we consider this notation rather than a raw application \(\C^e \Sigma\)).
    This is sufficient for getting the expected result for Rule \eqref{rule:satisfiable}.
    The case of the negative branches of Rules \eqref{rule:rewrite} and \eqref{rule:equality} are trivial.
    Regarding their positive branches, we only treat the case of Rule \eqref{rule:equality} since the treatment of \eqref{rule:rewrite} can be obtained by using a similar reasoning on each formulas added by the rule (the sets \(\USolved_0\) in the notations of Rule \eqref{rule:rewrite}).
    The rule \eqref{rule:equality} does not introduce elements in \(\Solved\) and the only second-oder variables introduced in \(\USolved\) that are not trivially added to \(\Df\) are from recipes \(\xi_1,\xi_2\) such that \(\xi_1 \dedfact u_1, \xi_2 \dedfact u_2 \in \Solved(\C^e)\) for some \((\P,\C,\C^e) \in \Gamma\).
    In particular by hypothesis \(\vars[2](\xi_1,\xi_2) \subseteq \vars[2](\Df(\C^e)) \subseteq \vars[2](\Df(\C^{e\prime}))\), hence the conclusion.

  \caseitem{\emph{property 2 (first and second-order equations):}
  we have \(\mgu(\Eqn[i]) \neq \bot\), \(\dom(\mgu(\Eqn[i])) \cap \vars[i](\Df) = \emptyset\), and \(\vars[i](\im(\mgu(\Eqn[i]))) \subseteq \vars[i](\Df)\).}

    The fact that \(\mgu(\Eqn[i]) \neq \bot\) is a direct consequence of the facts that \(\C^{e\prime} \neq \bot\) and that \(\C^{e\prime}\) is already normalised by the simplification rules of Figure \ref{fig:normalisation_constraint_systems},
    in particular Rules \ref{rule:unifEqfst_norm} and the rules inherited from Figure \ref{fig:normalisation_formula}.
    The remaining properties are simple invariants of the mgu's that are straightforward to verify.

  \caseitem{\emph{property 3 (Solution is consequence):}
  \(\im(\mgu(\Eqsnd)) \subseteq \conseq(\Solved \cup \Df)\).}

    This property comes from the fact that \(\mgu(\Eqsnd)\) is only modified in the positive branches of the case distinction rules by applying a mgs \(\Sigma\) to the extended constraint systems \(\C^e \in \Gamma\).
    A quick induction on the constraint solving relation for computing mgs \(\simpl\) show that \(\im(\Sigma) \subseteq \conseq(\Solved(\C^e) \cup \Df(\C^e))\), hence the conclusion.

  \caseitem{\emph{property 4 (Shape of \(\Solved\)):}
  for all \(\psi = (\xi \dedfact u) \in \Solved\), \(\psi \in \USolved\), \(u \notin \X[1]\), \(u \in \subterms(\Phi)\) and \(\subterms(\xi) \subseteq \conseq(\Solved \cup \Df)\).}

    The only rule adding a deduction fact to \(\Solved\) is the rule \eqref{rule:vector-solve};
    in particular this gives \(\psi \in \USolved\).
    The added deduction facts are of the form \(\xi \dedfact u\) where, for some \((\P,\C,\C^e) \in \Gamma\), \(\xi \dedfact u \in \USolved(\C^e)\) and \((\zeta,u) \notin \conseq(\Solved(\C^e) \cup \Df(\C^e))\) for any recipe \(\zeta\).
    The fact that \(\xi \dedfact u \in \USolved(\C^e)\) and \(\PredWellFormed(\Gamma)\) (property 5) justify that \(\subterms(\xi) \subseteq \conseq(\Solved \cup \Df)\);
    this justifies that \(u \in \subterms(\Phi)\) as well when taking into account that \(u\) is not a consequence of \(\Df(\C^e)\) (in particular it cannot be a ground contructor term without names otherwise it would even be consequence of the empty set).
    Finally the property \(u \notin \X[1]\) also follows from \((\zeta,u)\) not being a consequence of \(\Df(\C^e)\).

  \caseitem{\emph{property 5 (Shape of \(\USolved\)):}
  For all \(\clause{H}{\varphi} \in \USolved(\C^{e\prime})\), \(\varphi\) only contains syntactic equations as hypothesis, i.e. no deduction facts. Moreover \(\strsubterms[2](H) \subseteq \conseq(\Solved \cup \Df)\),
  and if \(H = \xi \dedfact u\) then either \(u \in \subterms(\Phi)\) or there exists a recipe \(\zeta\) such that \((\zeta,u) \in \conseq(\Solved \cup \Df)\).}

    The property that the hypotheses of the formula only contain syntactic equations can be obtained by a straightforward inspection of each case-distinction and simplification rules.
    Note in particular that whenever a formula \(\psi\) with deduction-fact hypotheses is considered (in Rules \eqref{rule:rewrite} and \eqref{rule:equality}), they are applied to a substitution \(\Sigma\) so that \(\FApply{\Sigma}{\psi}{\C^e}\) only has equations as hypotheses.
    As for the second part of property 5 we write
    \begin{enumerate*}
      \item \label{it:inv-wf-usolved-1}
      the property \(\strsubterms[2](H) \subseteq \conseq(\Solved \cup \Df)\) and
      \item \label{it:inv-wf-usolved-2}
      the rest (the property about the head terms of deduction formulas).
    \end{enumerate*}
    We perform a case analysis on the rule that added the formula to \(\Gamma'\).
    \begin{itemize}
      \item rule \eqref{rule:vector-consequence}:
      the proof of \ref{it:inv-wf-usolved-2} directly follows from \(\PredWellFormed(\Gamma)\) since the rule does not add a deduction formula.
      Regarding \ref{it:inv-wf-usolved-1},
      using the notations of the rule, the head of the formula is \(\xi \eqf \zeta\) where \(\xi \dedfact u \in \USolved\) for some \(u\), and \(\zeta \in \conseq(\Solved \cup \Df)\).
      In particular the conclusion follows from \(\PredWellFormed(\Gamma)\) (property 4 for the subterms of \(\xi\) and property 5 for the subterms of \(\zeta\))
      \item rule \eqref{rule:rewrite}:
      we first prove \ref{it:inv-wf-usolved-1}.
      Using the notations of the rule, all non-root positions of the head of a formula of \(\USolved_0\) are either variables \(X\) such that \(\Df\) contains a deduction fact \(X \dedfact x\) (generated by the skeleton), a position of \(\xi_0\), or a public function symbol (since the rewriting system is constructor, the left-hand sides \(\ell\) of the rewrite rules only contain a destructor at their roots).
      In particular all strict subterms of \(H\) are consequences of \(\Solved \cup \Df\).
      Now let us prove \ref{it:inv-wf-usolved-2}.
      By definition of the rule, \(u\) is a constructor term in normal form obtained after applying one rewrite rule \(\ell \to r\) at the root of \(C[u_0]\) for some context \(C\) (not containing names but possibly containing variables \(x\) such that \(X \dedfact x \in \Df\) for some \(X\)).
      We recall that the rewriting system is constructor-destructor and subterm convergent, which leaves two cases.
      The first is that \(r\) is a ground constructor term, and then \(u = r\) is trivially a consequence of \(\Solved \cup \Df\) for the recipe \(\zeta = r\).
      Otherwise \(r\) is a subterm of \(\ell\), meaning that \(u\) is a subterm of \(C[u_0]\).
      This implies that \(u\) is either a subterm of \(u_0\), a subterm of the context \(C\), or a term of the form \(C'[u_0]\) for some subcontext \(C'\) of \(C\).
      In all cases, since \(u_0 \in \subterms(\Phi)\) by \(\PredWellFormed(\Gamma)\) (property 4), this gives the expected conclusion.
      \item rule \eqref{rule:equality}:
      the proof of \ref{it:inv-wf-usolved-2} follows from \(\PredWellFormed(\Gamma)\) since the rule does not add a deduction formula.
      Regarding \ref{it:inv-wf-usolved-1},
      this follows from \(\PredWellFormed(\Gamma)\) (property 4).
      \qedhere
    \end{itemize}
\end{proof}

\begin{proof}[Proof of preservation of \(\PredCorrectFormula\)]
  Let \(\C' = (\Phi, \Df, \Eqfst, \Eqsnd, \Solved, \USolved)\) for some \((\P',\C',\C^{e\prime}) \in \Gamma'\), \(\psi \in \Solved \cup \USolved\), and \((\Sigma,\sigma)\) such that
  \[(\Phi,\Sigma,\sigma) \models
    \underset {\vars[2](\psi') \subseteq \vars[2](\psi)} {\underset {\psi' \in \Df} \bigwedge} \hspace{-5mm} \psi' \quad \wedge
    \equality{\Eqfst}\]
  We have to prove that \((\Phi,\Sigma,\sigma) \models \psi\).
  Here we prove more precisely that the conclusion holds when applying one time any of the case-distinction or simplification rules.
  First of all we observe that this property is preserved by the application of a mgs to an extended constraint system.
  In particular this makes the conclusion immediate for all rules except the following:
  \begin{itemize}
    \item Rule \eqref{rule:rewrite} (positive branch):
    using the notations of the rule and recalling \(\PredCorrectFormula(\Gamma)\), the conclusion follows from the fact that if \(\xi_0 \Sigma \Phi \sigma \norm = u_0 \norm\), \(\ell \to r \in \R\) and \(C[u] = \ell \sigma'\) for some substitution \(\sigma'\), then \(C[\xi_0] \Sigma \Phi \sigma \norm = r \sigma' \norm\).

    \item Rule \eqref{rule:equality} (positive branch):
    using the notations of the rule and recalling \(\PredCorrectFormula(\Gamma)\), if \(\psi\) is the formula added to \(\USolved\) by this rule, we have \((\Phi,\Sigma,\sigma) \models \psi\) iff for some recipes \(\xi_1,\xi_2\), if \(\xi_1 \Sigma \Phi \sigma \norm = z\) and \(\xi_2 \Sigma \Phi \sigma \norm = z\)
    then \(\xi_1 \Sigma \Phi \sigma \norm = \xi_2 \Sigma \Phi \sigma \norm\).
    This naturally holds.

    \item Rule \eqref{rule:vector-solve}:
    since the element added to \(\Solved\) is originated from \(\USolved\), the conclusion follows from \(\PredCorrectFormula(\Gamma)\).

    \item Rule \eqref{rule:vector-consequence}:
    the reasoning is identical to the one for \eqref{rule:equality}. \qedhere
  \end{itemize}
\end{proof}

\begin{proof}[Proof of preservation of \(\PredCompleteFormula\)]
  Let \({\C_1^e}',{\C_2^e}'\) for some \((\P_1',\C_1',{\C^e_1}'),(\P_2',\C_2',{\C^e_2}') \in \Gamma'\), \(\psi = \clause{H}{\varphi} \in \USolved({\C^e_1}')\) and
  \((\Sigma,\sigma) \in \Sol({\C^e_2}')\).
  We assume that \((\Phi({\C^e_2}'),\Sigma,\sigma)\) weakly satisfies \(H\).
  We want to prove that there exists \(\psi' = (\clause{H'}{\varphi'}) \in \USolved({\C^e_2}')\) such that \(H' \receq H\) and \((\Phi({\C^e_2}'), \Sigma, \sigma) \models \varphi'\).
  %
  We prove the strenghtened property stating that the invariant is preserved after the application of each case-distinction and simplification rules.
  \begin{itemize}
    \item Rules \eqref{rule:unifEqfst_norm}, \eqref{rule:uniform}, \eqref{rule:Disequation removal 1}:
      the conclusion directly follows from \(\PredCompleteFormula(\Gamma)\).

    \item Rule \eqref{rule:Disequation removal 2}:
      by \(\PredCompleteFormula(\Gamma)\) we let \(\psi_0' = (\clause{H_0'}{\varphi_0'}) \in \USolved(\C^e_2)\) such that \(H_0' \receq H\) and \((\Phi(\C^e_2), \Sigma, \sigma) \models \varphi_0'\).
      The conclusion directly follows from this, except if \(\psi_0'\) is the formula removed by the rule, i.e. if \(\mgs(\C^e_2[\Eqfst \mapsto \Eqfst \wedge \varphi_0']) = \emptyset\).
      However this would yield a contradiction with \((\Phi(\C^e_2), \Sigma, \sigma_2) \models \varphi_0'\), hence the conclusion.

    \item Rule \eqref{rule:Removal of unsolved formula}:
      using the same notations as in the previous case, we let \(\psi_0'\) by \(\PredCompleteFormula(\Gamma)\) and the only non-trivial case is again the one where \(\psi_0'\) is removed from \(\C^e_2\) by the rule.
      This means that there exists \(\psi' \in \USolved({\C^e_2}')\) solved such that \(\psi' \receq H_0'\), hence the conclusion since \(\Fhyp(\psi') = \top\) and \(H_0' \receq H\).

    \item Rules \eqref{rule:vectorbot} and \eqref{rule:vector-split solved}:
      the conclusion follows from the \(\PredCompleteFormula(\Gamma)\) since \(\Gamma' \subseteq \Gamma\).

    \item Rule \eqref{rule:vector-solve}:
      directly follows from \(\PredCompleteFormula(\Gamma)\).

    \item Rule \eqref{rule:vector-consequence}:
      Let us write
      \[\Gamma' = \{ (\P,\C,\C^e[\USolved \mapsto \USolved \wedge \FApply{\Sigma}{\psi}{\C^e}]) \mid (\P,\C,\C^e) \in \Gamma \}\]
      with the notations and assumptions of the rule.
      If \(i \in \{1,2\}\) we write \(S_i = (\P,\C,\C^e_i)\).
      The only case that does not directly follows from \(\PredCompleteFormula(\Gamma)\) is the case where \(\psi = \FApply{\Sigma}{\psi}{\C^e_1}\).
      But since there exists a formula \(\xi \dedfact u_{S_2} \in \USolved(\C^e_2)\) by hypothesis, we know by \(\PredCorrectFormula(\Gamma)\) that \(\xi \Phi(\C^e_2) \Sigma \sigma_2 \norm = u_{S_2}\).
      In particular since \((\Phi(\C^e_2), \Sigma, \sigma_2)\) weakly satisfies \(\xi \eqf \zeta\), we can write \(u_{S_2}' = \zeta \Sigma \Phi(\C^e_2) \sigma_i \norm\) and have \(u_{S_2} = u_{S_2'}\).
      Since we also have by definition (after formula normalisation, see Figure \ref{fig:normalisation_formula})
      \[\FApply{\Sigma}{\psi}{\C^e_2} = (\clause{\xi \eqf \zeta}{u_{S_2} \eqs u_{S_2'}})\]
      we obtain the expected conclusion.

    \item Any case-distinction rule (negative branch) or Rule \eqref{rule:satisfiable}:
    follows from \(\PredCompleteFormula(\Gamma)\).

    \item Rule \eqref{rule:rewrite} (positive branch):
      using the notations of the rule, the only case that does not directly follow from the assumption \(\PredCompleteFormula(\Gamma)\) is the case where \(\psi \in \USolved_0\).
      The hypothesis \((\Phi(\C^e_2),\Sigma,\sigma)\) weakly models the head of \(\psi\) can then be rephrased as \(\msg(\xi' \Sigma \Phi(\C^e_2) \sigma)\) for some recipe \(\xi' = C[\xi_0]\) such that \(\rootf(C) \in \sigd\), by definition of \(\USolved_0\).
      Let us write \(u = \xi_0 \Sigma \Phi(\C^e_2) \sigma \norm\).
      Since the rewriting system is constructor-destructor and \(\rootf(C[u]) \in \sigd\), there exists at least one rewrite rule \(\ell' \to r' \in \R\) such that \(C[u]\) and \(\ell'\) are unifiable.
      In particular it sufficies to choose \(\psi'\) the formula of \(\USolved_0\) corresponding to picking this rule in the definition of \(\RewF{\xi}{\ell \to r}{p}\).

    \item Rule \eqref{rule:equality} (positive branch):
      easily follows from \(\PredCompleteFormula(\Gamma)\) since the rule only adds to each \((\P,\C,\C^e) \in \Gamma\) the same solved formula.
      \qedhere
  \end{itemize}
\end{proof}

\begin{proof}[Proof of preservation of \(\PredConseq\)]
  Let \(\C' = (\Phi, \Df, \Eqfst, \Eqsnd, \Solved, \USolved)\) for some \((\P',\C',\C^{e\prime}) \in \Gamma'\) and \(k\) such that \(\vars[2](\Df(\C)) \subseteq \Xsndi{k}\).
  We also let \(\ffun/n \in \sigd\) and \((\xi_1,u_1),\ldots, (\xi_n,u_n) \in \conseq(\Solved(\C))\) such that \(\xi_1,\ldots, \xi_n \in \recipeset_k\),
  and \(\ffun(u_1,\ldots, u_n) \norm\) is a constructor term.
  We have to prove that there is \(\xi \in \recipeset_k\) such that \((\xi,\ffun(u_1,\ldots, u_n) \norm) \in \conseq(\Solved(\C) \cup \Df(\C))\).
  This can be justified by \(\PredWellFormed(\Gamma)\) (in particular the item ``shape of \(\Solved\)'').
  Indeed, by definition the term \(\ffun(u_1,\ldots, u_n)\) contains a single destructor, which is the symbol \(\ffun\) at its root.
  In particular if \(\ffun(u_1,\ldots, u_n) \norm\) is indeed a constructor protocol term \(u\), this has been obtained after applying a single rewriting rule at the root of \(\ffun(u_1,\ldots, u_n)\) since the rewriting system is constructor-destructor.
  Therefore, by subterm convergence, \(u\) is either a ground protocol term (without names, by definition of a rewrite rule) or a subterm of one of the \(u_i\)s.
  In the former case the conclusion is immediate, in the latter it follows from the invariant \(\PredWellFormed\).
\end{proof}

\begin{proof}[Proof of preservation of \(\PredSymb\)]
  The preservation of this invariant is straightforward:
  the case distinction and simplification rules only modify the extended constraint systems by
  \begin{enumerate*}
    \item applying a mgs, or
    \item adding formulas to \(\Eqfst\), \(\Eqsnd\), \(\Solved\),..., or
    \item removing formulas from \(\USolved\), or
    \item removing trivially-false disequations from \(\Eqfst\).
  \end{enumerate*}
  In particular such operations restrict the set of solutions.
\end{proof}

We can then extend this property to the whole procedure by proving the preservation by symbolic rules.

\begin{lemma}
  Let \(\Gamma\) be a set of extended symbolic processes such that \(\PredAll(\{\Gamma\})\).
  We assume that no case-distinction or simplification rules can be applied to \(\Gamma\).
  We then let
  \begin{align*}
    \Gamma_\inp & = \{ (\P',\C',\C^{e\prime}) \mid (\P,\C,\C^e) \in \Gamma, (\P,\C,\C^e) \Sstep{\InP{Y}{X}} (\P',\C',\C^{e\prime})\}\\
    \Gamma_\outp & = \{ (\P',\C',\C^{e\prime}) \mid (\P,\C,\C^e) \in \Gamma, (\P,\C,\C^e) \Sstep{\OutP{Z}{\ax_{|\Phi(\C^e)|+1}}} (\P',\C',\C^{e\prime})\}
  \end{align*}
  and the set of set of symbolic processes \(\S\) obtained by normalising \(\{\Gamma_\inp,\Gamma_\outp\}\) with the simplification rules.
  Then \(\PredAll(\S)\).
\end{lemma}

\begin{proof}
  Most invariants are either easily seen to be preserved by application of any symbolic or simplification rules, or are a straightforward consequence of the fact that no simplification rules can be applied to \(\S\).
  The only substantial case is the Invariant 4 stating that the knowledge base is saturated.
  Let us then consider \((\P,\C,\C^e) \in \Gamma'\) for some \(\Gamma' \in \S\) and we let \(k\) the minimal index such that \(\vars[2](\Df(\C)) \subseteq \Xsndi{k}\).
  % We also let \(k_0\) the analogue minimal index for some arbitrary constraint system of \(\Gamma\).
  We then let \(\ffun/n\, \in \sigd\) and \((\xi_1,u_1), \ldots, (\xi_n,u_n) \in \conseq(\Solved(\C^e))\) such that \(u \norm\), with \(u = \ffun(u_1, \ldots, u_n)\) is a constructor term.
  We recall that no case-distinction rules can be applied to \(\{\Gamma\}\), in particular \eqref{rule:rewrite}.
  But since the rewriting system is constructor-destructor, \(\ffun \in \sigd\), and \(u \norm\) is a constructor term, this means that a rewrite rule is applicable at the root of \(u\).
  We distinguish two cases.

  \caseitem{\emph{case 1:} there exists a rule \(\ell \to r\) applicable at the root of \(u\), \(i \in \eint {1} {n}\) and \(p\) a position such that
  \(\getpos {\ell} {i \cdot p} \notin \X[1]\) and \((\getpos {\xi_i} {p} \dedfact v) \in \Solved(\C^e)\) for some \(v\).}

    We consider the instance of Rule \eqref{rule:rewrite} with the following parameters:
    the position \(i \cdot p\), the rule \(\ell \to r\), a recipe \(\xi \in \termset(\sig,\quanti{\X}{|\Phi(\C^e)|})\), the formula \(\psi_0\) obtained by considering the rule \(\ell \to r\) in \(\RewF{\xi}{\ell \to r}{i \cdot p}\),
    the deduction fact \(\getpos {\xi_i} {p} \dedfact v\), \(\Sigma_0 = \{\getpos{\xi}{i \cdot p} \mapsto \getpos {\xi_i} {p}\}\),
    and a mgs \(\Sigma\) such that the head of \(\FApply{\Sigma_0\Sigma}{\psi_0}{\CApply{\Sigma}{\C^e}}\) is of the form \(\zeta \dedfact u\norm\) for some recipe \(\zeta\).
    Since this instance of the rule is not applicable by hypothesis, we deduce that there already exists a solved formula \(\psi \in \USolved(\C^e)\) of the form \(\zeta' \dedfact u\norm\).
    Since no simplification rules are applicable neither, in particular \eqref{rule:vector-solve}, we deduce that there exists a recipe \(\xi'\) such that \((\xi',u\norm) \in \conseq(\Solved(\C^e) \cup \Df(\C^e))\).

  \caseitem{\emph{case 2:} otherwise}

    We let \(\ell \to r\) an arbitrary rewrite rule applicable at the root of \(u\).
    By hypothesis for all positions \(i \cdot p\) of \(\ell\) that are not variables we know that \(\getpos{\xi_i}{p}\) is not the recipe of a deduction fact from \(\Solved(\C^e)\);
    since \(\xi_i \in \conseq(\Solved(\C^e))\), \(\rootf(\getpos{\xi_i}{p})\) is therefore a constructor symbol.
    We rule out the immediate case where \(r\) is a ground term and only consider the one where \(r\) is a strict subterm of \(\ell\).
    Then, since the rewriting system is constructor-destructor, we can fix a ground constructor context \(C\) such that \(r = C[x_p,\ldots,x_q]\) for \(x_p, \ldots, x_p\) the variables numbered \(p\) to \(q\) of \(\ell\) (w.r.t. the lexicographic ordering on the positions of the term \(\ell\)).
    In particular the recipe \((C[\xi_p, \ldots, \xi_q], u \norm) \in \conseq(\Solved(\C^e))\).
\end{proof}

\subsection{Preliminary technical results}

In this section we prove some low level technical results that will be useful during the incoming proofs.

\paragraph{Preservation by application of substitutions}
\label{sec:properties_formulas}

  In this section, we show that applying substitution preserves in some cases the different notions we use in our algorithms, namely first-order and second-order equations, deduction and equality facts; and uniformity.

  %%
  %%% Lemma
  %%

  \begin{lemma}
  \label{lem:apply_subst_clause}
  Let \(\psi\) be either a deduction fact, or an equality fact or a first-order equation or a second-order equation.
  For all ground frame \(\Phi\), for all substitutions \(\Sigma,\Sigma',\sigma,\sigma'\) if \(\dom(\Sigma) \cap \dom(\Sigma') = \emptyset\) and \(\dom(\sigma) \cap \dom(\sigma') = \emptyset\)
  then \((\Phi,\Sigma\Sigma',\sigma\sigma') \models \psi\) is equivalent to \((\Phi,\Sigma\Sigma',\sigma\sigma') \models \psi\Sigma\sigma\) and is equivalent to \((\Phi,\Sigma',\sigma') \models \psi\Sigma\sigma\).
  \end{lemma}

  \begin{proof}
    The proof of this lemma is done by case analysis on \(\psi\).

    \medskip

    \noindent\emph{Case \(u \eqs v\):}
    Consider \((\Phi,\Sigma\Sigma',\sigma\sigma') \models u \eqs v\).
    This is equivalent to \(u\sigma\sigma' = v\sigma\sigma'\).
    Since \(\vars(\Sigma) \cap \X[1] = \emptyset\), we deduce that \((\Phi,\Sigma\Sigma',\sigma\sigma') \models u \eqs v\) is equivalent to \(u\Sigma\sigma\sigma' = v\Sigma\sigma\sigma'\).
    This is also equivalent to \(u\Sigma\sigma\sigma\sigma' = v\Sigma\sigma\sigma\sigma'\) and so \((\Phi,\Sigma\Sigma',\sigma\sigma') \models u\Sigma\sigma \eqs v\Sigma\sigma\).
    Note that \(u\Sigma\sigma\sigma' = v\Sigma\sigma\sigma'\) is also equivalent to \((\Phi,\Sigma',\sigma') \models u\Sigma\sigma \eqs v\Sigma\sigma\).

    \medskip

    \noindent\emph{Case \(\xi \eqs \xi'\):}
    Similar to the previous case.

    \medskip

    \noindent\emph{Case \(\xi \dedfact u\):}
    Consider \((\Phi,\Sigma\Sigma',\sigma\sigma') \models \xi \dedfact u\).
    It is equivalent to \(\xi\Sigma\Sigma'\Phi\norm = u\sigma\sigma'\) and \(\msg(\xi\Sigma\Sigma'\Phi)\).
    But \((\xi \dedfact u)\Sigma\sigma = \xi\Sigma \dedfact u\sigma\).
    Moreover, by definition of substitution (in particular the acyclic property), we deduce that \(\xi\Sigma\Sigma'\Phi = \xi\Sigma\Sigma\Sigma'\Phi\) and \(u\sigma\sigma' = u\sigma\sigma\sigma'\). Hence, \(\msg(\xi\Sigma\Sigma'\Phi)\) is equivalent to \(\msg(\xi\Sigma\Sigma\Sigma'\Phi)\);
    and \(\xi\Sigma\Sigma'\Phi\norm = u\sigma\sigma'\) is equivalent to \(\xi\Sigma\Sigma\Sigma'\Phi\norm = u\sigma\sigma\sigma'\).
    Hence \((\Phi,\Sigma\Sigma',\sigma\sigma') \models \xi \dedfact u\) is equivalent to \((\Phi,\Sigma\Sigma',\sigma\sigma') \models \xi \dedfact u\Sigma\sigma\).
    Note that \(\msg(\xi\Sigma\Sigma'\Phi)\) and \(\xi\Sigma\Sigma'\Phi\norm = u\sigma\sigma'\) are also equivalent to \((\Phi,\Sigma',\sigma') \models \xi\Sigma \dedfact u\sigma\).

    \medskip

    \noindent\emph{Case \(\xi \eqf \xi'\):}
    Similar to previous case.
  \end{proof}


  %%
  %%% Subsection
  %%

\paragraph{Properties on consequence of set of deduction facts}
\label{sec:properties_consequence}

  This section contains some results about the consequence relation when modifying a set of deduction facts.
  The first lemma is about the application of substitutions.

  %%
  %%% Lemma
  %%

  \begin{lemma}
  \label{lem:consequence_protocol_terms_substitution}
  Let \(S\) be a set of solved deduction facts.
  For all substitutions \(\sigma\) of protocol terms, for all \((\xi,t) \in \conseq(S)\), \((\xi,t\sigma) \in \conseq(S\sigma)\).
  \end{lemma}

  \begin{proof}
  We know that \((\xi,t) \in \conseq(S)\) implies \(\xi = C[\xi_1,\ldots, \xi_n]\) and \(t = C[t_1,\ldots, t_n]\) for some public context \(C\) and for all \(i\), \((\xi_i \dedfact t_i \in S\).
  Hence \((\xi_i \dedfact t_i\sigma \in S\sigma\) which allows us to conclude.
  \end{proof}

  %%
  %%% Lemma
  %%

  \begin{proposition}[transitivity of consequences] \label{prop:trans-conseq}
    Let \(S,S'\) be two sets of solved deduction facts.
    Let \(\varphi = \{X_i \dedfact u_i\}_{i=1}^n\) such that all \(X_i\) are pairwise distinct, let \((\xi,t) \in \conseq(S \cup \varphi)\) and \(\Sigma,\sigma\) be two substitutions.
    If for all \(i \in \eint{1}{n}\), \((X_i\Sigma,u_i\sigma) \in \conseq(S\Sigma\sigma \cup S')\) then \((\xi\Sigma,t\sigma) \in \conseq(S\Sigma\sigma \cup S')\).
  \end{proposition}

  \begin{proof}
    We prove this result by induction on \(|\xi|\).
    The base case (\(|\xi| = 0\)) is trivial as there are no terms of size \(0\) and we hence focus on the inductive step.
    %
    We perform a case analysis on the hypothesis \((\xi,t) \in \conseq(S \cup \varphi)\).

    \caseitem{\emph{case 1: \(\xi = t \in \sig_0\)}}
      We directly have by definition that \((\xi\Sigma,t\sigma) \in \conseq(S\Sigma\sigma \cup S')\).

    \caseitem{\emph{case 2: there are \(\xi_1,t_1,\ldots, t_m,\xi_m\) and \(\ffun \in \sigc\) such that \(\xi = \ffun(\xi_1,\ldots, \xi_m)\), \(t = \ffun(t_1,\ldots, t_m)\) and for all \(i \in \eint{1}{m}\), \((\xi_i,t_i) \in \conseq(S \cup \varphi)\)}}
      By induction hypothesis we know that for all \(j \in \eint{1}{m}\), \((\xi_j\Sigma,t_j\sigma) \in \conseq(S\Sigma\sigma \cup S')\).
      Writing \(\xi\Sigma = \ffun(\xi_1\Sigma,\ldots, \xi_m\Sigma)\) and \(t\sigma = \ffun(t_1\sigma,\ldots,t_m\sigma)\), we conclude that \((\xi\Sigma,t\sigma) \in \conseq(S\Sigma\sigma \cup S')\).

    \caseitem{\emph{case 3: \(\xi \dedfact t \in S \cup \varphi\)}}
      If \((\xi \dedfact t) \in S\) then \((\xi\Sigma \dedfact t\sigma) \in S\Sigma\sigma\) and the result directly holds.
      Otherwise \(\xi \dedfact t \in \varphi\) and hence by hypothesis \((\xi\Sigma, t\sigma) \in \conseq(S\Sigma\sigma \cup S')\).
  \end{proof}

  % \begin{lemma}[restate=lemconsequencesubstitutions,name={}] \label{lem:consequence_of_substitutions}
  %   Let \(S, S'\) be two sets of solved deduction facts.
  %   Let \(\varphi = \{X_i \dedfact u_i\}_{i=1}^n\) such that all \(X_i\) are pairwise distinct.
  %   For all \(\Sigma,\sigma\), for all \((\xi,t) \in \conseq(S \cup \varphi)\), if for all \(i \in \{1, \ldots, n\}\), \((X_i\Sigma,u_i\sigma) \in \conseq(S\Sigma\sigma \cup S')\)
  %   then \((\xi\Sigma,t\sigma) \in \conseq(S\Sigma\sigma \cup S')\).
  % \end{lemma}
  %
  %\begin{lemma*}[\textbf{\ref{prop:trans-conseq}}]
  %	Let \(S\), \(S'\) be two sets of solved deduction facts. Let \(\varphi = \{X_i \dedfact u_i\}_{i=1}^n\) such that all \(X_i\) are pairwise distinct. For all \(\Sigma,\sigma\), for all \((\xi,t) \in \conseq(S \cup \varphi)\), if for all \(i \in \{1, \ldots, n\}\), \((X_i\Sigma,u_i\sigma) \in \conseq(S\Sigma\sigma \cup S')\) then \((\xi\Sigma,t\sigma) \in \conseq(S\Sigma\sigma \cup S')\).
  %\end{lemma*}
  %
  % \begin{proof}
  %   We prove this result by induction on \(|\xi|\). The base case (\(|\xi| = 0\)) being trivial as there is no term of size \(0\), we focus on the inductive step.
  %
  %   Since \((\xi,t)\) is consequence of \(S \cup \varphi\), we know by definition that one of the following conditions hold:
  %   \begin{enumerate}
  %     \item \(\xi = t \in \sig_0\)
  %     \item there exists \(\xi_1,t_1,\ldots, t_m,\xi_m\) and \(\ffun \in \sigc\) such that \(\xi = \ffun(\xi_1,\ldots, \xi_m)\), \(t = \ffun(t_1,\ldots, t_m)\) and for all \(i \in \{1, \ldots,m\}\), \((\xi_i,t_i)\) is consequence of \(S \cup \varphi\).
  %     \item \(\xi \dedfact t \in S \cup \varphi\).
  %   \end{enumerate}
  %   In case 1, we directly have by definition that \((\xi\Sigma,t\sigma)\) is a consequence of \(S\Sigma\sigma \cup S'\).
  %   In case 2, by our inductive hypothesis on \(\xi_1,\ldots, \xi_m\), we have that for all \(j \in \{1,\ldots, m\}\), \((\xi_j\Sigma,t_j\sigma)\) is a consequence of \(S\Sigma\sigma \cup S'\).
  %   With \(\xi\Sigma = \ffun(\xi_1\Sigma,\ldots, \xi_m\Sigma)\) and \(t\sigma = \ffun(t_1\sigma,\ldots,t_m\sigma)\), we conclude that \((\xi\Sigma,t\sigma)\) is consequence of \(S\Sigma\sigma \cup S'\).
  %   In case 3, if \(\xi \dedfact t \in S\) then \(\xi\Sigma \dedfact t\sigma \in S\Sigma\sigma\) and so the result directly holds. Else \(\xi \dedfact t \in \varphi\) and so by hypothesis \(\xi\Sigma \dedfact t\sigma \in \conseq(S\Sigma\sigma \cup S')\).
  % \end{proof}

  The previous lemma showed that a consequence \((\xi,t)\) is preserved when applying some substitution \(\Sigma,\sigma\) under the right conditions.
  However, it is quite strong since we ensure that \(\xi\Sigma\) is consequence with \(t\sigma\).
  In some cases, we cannot guarantee that \(\xi\Sigma\) is consequence with \(t\sigma\) but with some other first-order term.
  This is the purpose of the next lemma.

  %%
  %%% Lemma
  %%

  \begin{lemma}
    \label{lem:consequence_subtitution_recipe}
    Let \(S\), \(S'\) be two sets of solved deduction facts.
    Let \(\varphi = \{X_i \dedfact u_i\}_{i=1}^n\) such that all \(X_i\) are pairwise distinct.
    For all \(\Sigma\), for all \(\xi \in \conseq(S \cup \varphi)\), if for all \(i \in \{1, \ldots, n\}\), \(X_i\Sigma \in \conseq(S\Sigma \cup S')\) then \(\xi\Sigma \in \conseq(S\Sigma \cup S')\).
  \end{lemma}

  \begin{proof}
    We prove this result by induction on \(|\xi|\).
    The base case (\(|\xi| = 0\)) being trivial as there is no term of size \(0\), we focus on the inductive step.

    Since \(\xi\) is consequence of \(S \cup \varphi\), we know by definition that there exists \(t\) such that one of the following conditions hold:
    \begin{enumerate}
      \item \(\xi = t \in \sig_0\)
      \item there exists \(\xi_1,t_1,\ldots, t_m,\xi_m\) and \(\ffun \in \sigc\) such that \(\xi = \ffun(\xi_1,\ldots, \xi_m)\), \(t = \ffun(t_1,\ldots, t_m)\) and for all \(i \in \{1, \ldots,m\}\), \((\xi_i,t_i)\) is consequence of \(S \cup \varphi\).
      \item there exists \(t\) such that \(\xi \dedfact t \in S \cup \varphi\).
    \end{enumerate}
    In case 1, we directly have by definition that \((\xi\Sigma,t)\) is a consequence of \(S\Sigma \cup S'\).
    In case 2, by our inductive hypothesis on \(\xi_1,\ldots, \xi_m\), we have that for all \(j \in \{1,\ldots, m\}\), \(\xi_j\Sigma\) is a consequence of \(S\Sigma \cup S'\) hence there exists \(t'_1,\ldots, t'_m\) such that for all \(j \in \{1,\ldots, m\}\), \((\xi_j\Sigma,t'_j)\) is a consequence of \(S\Sigma \cup S'\).
    With \(\xi\Sigma = \ffun(\xi_1\Sigma,\ldots, \xi_m\Sigma)\) and \(t' = \ffun(t'_1,\ldots,t'_m)\), we conclude that \((\xi\Sigma,t')\) is consequence of \(S\Sigma \cup S'\).
    In case 3, if \(\xi \dedfact t \in S\) then \(\xi\Sigma \dedfact t \in S\Sigma\) and so the result directly holds.
    Else \(\xi \dedfact t \in \varphi\) and so by hypothesis \(\xi\Sigma \in  \conseq(S\Sigma \cup S')\).
  \end{proof}

  In the next lemma, we show that when a recipe is consequence of the sets of solved deduction formulas \(\Solved\Sigma\sigma\) where \((\Sigma,\sigma)\) is a solution of the constraint system, then all subterms of that recipe are also consequence of \(\Solved\Sigma\sigma\).
  This property is in fact guaranted by the fact that \(\Solved\) contains itself recipes consequence of itself.
  This is an important property that allows us to generate solutions that satisfy the uniformity property.

  %%
  %%% Lemma
  %%

  \begin{lemma}
    \label{lem:consequence_subterms_uninstantiated}
    Let \(\C = (\Phi, \Df, \Eqfst, \Eqsnd, \Solved, \USolved)\) be an extended constraint system such that \(\PredWellFormed(\C)\).
    For all \(\xi \in \conseq(\Solved \cup \Df)\), \(\subterms(\xi) \subseteq \conseq(\Solved \cup \Df)\).
  \end{lemma}

  \begin{proof}
    Since \(\xi \in \conseq(\Solved \cup \Df)\), we know that \(\xi = C[\xi_1,\ldots, \xi_n]\) where \(C\) is a public context and \(\xi_1,\ldots, \xi_n\) are recipes of deduction facts from \(\Solved\) or \(\Df\).
    Hence since \(\xi' \in \subterms(\xi)\), we have that the position \(p\) of \(\xi'\) in \(\xi\) is either a position of \(C\) thus \(\xi' \in \in \conseq(\Solved \cup \Df)\) from the definition of consequence;
    or is a position of one of the \(\xi_i\) and thus we conclude by the predicate \(\PredWellFormed(\C)\).
  \end{proof}


  %%
  %%% Lemma
  %%

  \begin{lemma}
  \label{lem:consequence_subterms}
    Let \(\C = (\Phi, \Df, \Eqfst, \Eqsnd, \Solved, \USolved)\) be an extended constraint system such that \(\PredWellFormed(\C)\).
    For all \((\Sigma,\sigma) \in \Sol(\C)\), for all \(\xi \in \conseq(\Solved\Sigma\sigma)\), \(\subterms(\xi) \subseteq \conseq(\Solved\Sigma\sigma)\).
  \end{lemma}

  %\begin{proof}
  %	We prove this result by induction on \(|\xi|\). The base case (\(|\xi| = 0\)) being trivial as there is no term of size \(0\), we focus on the inductive step.
  %
  %Since \(\xi\) is consequence of \(\SDFrestr{(\Solved\Sigma\sigma)}{i}\), there exists \(t\) such that \((\xi,t)\) is consequence of \(\SDFrestr{\Solved\Sigma\sigma}{i}\) and so by definition one of the following conditions holds:
  %\begin{enumerate}
  %\item \(\xi = t \in \sig_0\)
  %\item there exist \(\xi_1,t_1,\ldots, \xi_n,t_n\) and \(\ffun \in \sigc\) such that \(\xi = \ffun(\xi_1,\ldots, \xi_n)\), \(t = \ffun(t_1,\ldots, t_n)\) and for all \(j \in \{1, \ldots, n\}\), \((\xi_j,t_j)\) is consequence of \(\SDFrestr{(\Solved\Sigma\sigma)}{i}\)
  %\item there exists \((\clause[S]{\zeta \dedfact u}{\varphi}) \in \Solved\) and \(\Sigma',\sigma'\) such that \(\dom(\Sigma') = \varssnd(S)\), \(\dom(\sigma') = \varsfst(S)\), \(\xi = \zeta\Sigma\Sigma'\), \(t = u\sigma\sigma'\) and for all \((X \dedfact x) \in \varphi\), \((X\Sigma',x\sigma')\) is a consequence of \(\SDFrestr{(\Solved\Sigma\sigma)}{i}\).
  %\end{enumerate}
  %In Case 1, we trivially have that for all \(\xi' \in \subterms(\xi)\), \(\xi = \xi'\) and \(\xi'\) is consequence of \(\SDFrestr{(\Solved\Sigma\sigma)}{i}\). In Case 2, a simple induction on \(\xi_1,\ldots, \xi_n\) allows us to conclude. Therefore, we focus on Case 3.
  %
  %First of all, by Property~\ref{lab-well_formed-shape_of_head}, we first can deduce that \(k > 0\) since \(k = 0\) would imply that \(\axioms{\zeta} = \emptyset\) which is a contradiction. Second, by Lemma~\ref{lem:instantiation_of_solved}, \(\SDFrestr{(\Solved\Sigma\sigma)}{i} = \SDFrestr{\Solved}{i}\Sigma\sigma\) hence \((\clause[S]{\zeta \dedfact u}{\varphi}) \in \SDFrestr{\Solved}{i}\). Third, by Property~\ref{lab-well_formed-shape_of_head} of \(\PredWellFormed(\C){k}\), we know that \(\zeta \in \Tdeux_k\). Let \(\ell \leq k\) such that \(\zeta \in \Tdeux_\ell \smallsetminus \Tdeux_{\ell-1}\). Note that \(\ell \leq i\). Moreover, \((\clause[S]{\zeta \dedfact u}{\varphi}) \in \SDFrestr{\Solved}{\ell}\). Thus following Property~\ref{lab-well_formed-subterm_ded} of \(\PredWellFormed(\C){k}\), we know that for all \(\zeta' \in \strsubterms(\zeta)\), \(\zeta'\) is consequence of \(\SDFrestrX{\Solved}{\ell} \cup \SDFrestrX{\Df}{\ell-1} \cup \varphi\). But $\ell \leq
  %i\( hence \)\SDFrestrX{\Solved}{\ell} \subseteq \SDFrestr{\Solved}{i}\(. This allows us to prove that \)\zeta'\( is consequence of \)\SDFrestr{\Solved}{i} \cup \SDFrestrX{\Df}{\ell-1} \cup \varphi\(. We already know that for all \)(X \dedfact x) \in \varphi\(, \)(X\Sigma', x\sigma')\( is a consequence of \)\SDFrestr{(\Solved\Sigma\sigma)}{i} = \SDFrestr{\Solved}{i}\Sigma\sigma$
  %
  %We now show that for all \((X \dedfact u) \in \SDFrestrX{\Df}{\ell-1}\), \((X\Sigma,u\sigma)\) is a consequence of \(\SDFrestr{\Solved}{i}\Sigma\sigma\).
  %%Since \((\Sigma,\sigma) \in \Sol(\C)\), we deduce that \((\Phi\sigma,\Sigma,\sigma) \models \Df\) and so \((\Phi\sigma,\Sigma,\sigma) \models \SDFrestrX{\Df}{\ell-1}\). It implies that for all \((X \dedfact u) \in \SDFrestr{\Df}{\ell-1}\), \(X\Sigma\Phi\sigma\norm = u\sigma\) and \(\msg(x\Sigma\Phi\sigma)\). Moreover, from \(\PredConseq(\C){k}\) and Lemma~\ref{lem:consequence_restrX_restr}, we deduce that \(X\Sigma\Phi\sigma\norm\) is consequence of \(\SDFrestr{\Solved\Sigma\sigma}{\ell-1}\). However, \((\Sigma,\sigma) \in \Sol(\C)\) also implies that \(\recipes\Sigma\) is uniform in \((\Phi\sigma,\Solved\Sigma\sigma)\).
  %By Property~\ref{lab-well_formed-vars_in_IT} of \(\PredWellFormed(\C){k}\), \(\varssnd(\recipes) = \varssnd(\Df) \cap \Xsnd_{k-1}\). Thus, \(\ell \leq k\) implies \(X\Sigma \in \subterms(\recipes\Sigma)\). Hence, by definition of uniformity and Lemma~\ref{lem:consequence_restr}, we deduce that \((X\Sigma,u\sigma)\) is consequence of \(\SDFrestr{\Solved\Sigma\sigma}{\ell-1}\) meaning that it is also consequence of \(\SDFrestr{\Solved\Sigma\sigma}{i} = \SDFrestr{\Solved}{i}\Sigma\sigma\). We can now apply Lemma~\ref{prop:trans-conseq} with \(S = \SDFrestr{\Solved}{i}\), \(S' = \emptyset\) and \(\varphi = \SDFrestrX{\Df}{\ell-1} \cup \varphi\) which allows us to conclude that \(\zeta'\Sigma\) is consequence of \(\SDFrestr{\Solved\Sigma\sigma}{i}\).
  %
  %We have shown that for all \(\zeta' \in \strsubterms(\zeta)\), \(\zeta'\Sigma\) is consequence of \(\SDFrestr{\Solved\Sigma\sigma}{i}\). But by Property~6 of Definition~\ref{def:invariant_well_formed}, we know that \(\zeta \notin \Xsnd\). Therefore either \(\zeta\) is a constant or \(\zeta \in \AX\) or \(\zeta = \ffun(\zeta_1,\ldots, \zeta_m)\) for some \(\ffun,\zeta_1,\ldots, \zeta_m\). In the first two cases, the result trivially holds since \(\subterms(\zeta) = \{\zeta\}\). In the latter case, we know that \(\zeta_1,\ldots, \zeta_m \in \strsubterms(\zeta)\), hence we can apply our inductive hypothesis on all \(\zeta_1\Sigma, \ldots,\zeta_m\Sigma\) which allows us to conclude.\qed
  %\end{proof}

  %Our three final results on the consequences of sequence of solved deduction facts link the first order terms and second order terms through the frame. In particular, we first show that when \((\xi,t)\) is a consequence then instances of \(\xi\) will deduce the term \(t\) in the frame of the constraint system.
  %
  %%
  %%% Lemma
  %%

  \begin{lemma}
    \label{lem:consequence_implies_dedfact}
    Let \(S\) be a set of ground deduction facts. Let \(\Phi\) be a ground frame.
    Assume that for all \(\psi \in S\), \(\Phi \models \psi\). For all \((\xi,t) \in \conseq(S)\), \(\Phi \models \xi \dedfact t\).
  \end{lemma}

  \begin{proof}
    We prove this result by induction on \(|\xi|\).
    The base case being trivial, we focus on the inductive step. Since \((\xi,t)\) is consequence of \(S\) then one of the following properties holds:
    \begin{enumerate}
      \item \(\xi = t \in \sig_0\)
      \item there exist \(\xi_1,t_1,\ldots, \xi_n,t_n\) and \(\ffun \in \sigc\) such that \(\xi = \ffun(\xi_1,\ldots, \xi_n)\), \(t = \ffun(t_1,\ldots, t_n)\) and for all \(i \in \{1,\ldots, n\}\), \((\xi_i,t_i)\) is consequence of \(S\)
      \item \(\xi \dedfact t \in S\).
    \end{enumerate}
    In Case 1, the result trivially holds.
    In case two, a simple induction on \((\xi_1,t_1),\ldots, (\xi_n,t_n)\) allows us to conclude.
    In case 3, we know by hypothesis that \(\Phi \models \xi \dedfact t\) hence the result holds
  \end{proof}

  %%%
  %%%% Lemma
  %%%
  %
  %\begin{lemma}
  %\label{lem:instantiation_consequence}
  %	Let \(\C = (\Phi, \Df, \Eqfst, \Eqsnd, \Solved, \USolved)\) be an extended constraint system such that \(\PredCorrectFormula(\C)\). Let \(\varphi = \{ X_i \dedfact u_i\}_{i=1}^m\) such that all \(X_i\) are pairwise distinct and \(X_i \notin \varssnd(\C)\). For all substitutions \(\Sigma,\Sigma_\varphi,\sigma,\sigma_\varphi\), for all \((\xi,t) \in \conseq(\Solved \cup \Df \cup \varphi)\), if \((\Phi\sigma,\Sigma,\sigma) \models \Df \wedge \equality{\Eqfst}\), \(\dom(\Sigma_\varphi) = \varssnd(\varphi)\), \(\dom(\sigma_\varphi) = \varsfst(\varphi) \smallsetminus \varsfst(\C)\) and \((\Phi\sigma,\Sigma_\varphi,\sigma\sigma_\varphi) \models \varphi\) then \((\Phi\sigma,\Sigma\Sigma_\varphi,\sigma\sigma_\varphi) \models \xi \dedfact t\).
  %\end{lemma}
  %
  %%\begin{proof}
  %%	We prove the result by induction on \(|\xi|\).
  %%
  %%	\medskip
  %%
  %%	\emph{Base case \(|\xi| = 0\):} This case is trivial since there is no recipe with size \(0\).
  %%
  %%	\medskip
  %%
  %%	\emph{Inductive step \(|\xi| > = 0\):} Since \((\xi,u)\) is consequence of \(\Solved \cup \Df \cup \varphi\), one of the following properties holds:
  %%	\begin{enumerate}
  %%	\item \(\xi = t \in \sig_0\)
  %%	\item there exist \(\xi_1,t_1,\ldots, \xi_n,t_n\) and \(\ffun \in \sigc\) such that \(\xi = \ffun(\xi_1,\ldots, \xi_n)\), \(t = \ffun(t_1,\ldots, t_n)\) and for all \(i \in \{1,\ldots, n\}\), \((\xi_i,t_i)\) is consequence of \(\Solved \cup \Df \cup \varphi\)
  %%	\item there exist \((\clause[S]{\zeta \dedfact u}{\varphi'}) \in \Solved \cup \Df \cup \varphi\) and \(\Sigma',\sigma'\) such that \(\dom(\Sigma') = \varssnd(S)\), \(\dom(\sigma') = \varsfst(S)\), \(\xi = \zeta\Sigma'\), \(t = u\sigma'\) and for all \((X \dedfact x) \in \varphi'\), \((X\Sigma',x\sigma')\) is a consequence of \(\Solved \cup \Df \cup \varphi\).
  %%	\end{enumerate}
  %%In Case 1, the result directly holds. In Case 2, a simple induction on \((\xi_1,t_1), \ldots, (\xi_n,t_n)\) allows us to conclude. Hence we focus on the third case.
  %%
  %%By definition of a deduction formula, we have that for all \(X \dedfact x \in \varphi'\), \(X \in \varssnd(\zeta)\). Hence \(|X\Sigma'| < |\xi|\). Therefore, we can apply our inductive hypothesis, meaning that for all \(X \dedfact x \in \varphi'\), \((\Phi\sigma,\Sigma\Sigma_\varphi,\sigma\sigma_\varphi) \models X\Sigma' \dedfact x\sigma'\). In other words, \((\Phi\sigma,\Sigma\Sigma_\varphi,\sigma\sigma_\varphi) \models \varphi'\Sigma'\sigma'\).
  %%
  %%Let us consider whether (a) \(\clause[S]{\zeta \dedfact u}{\varphi'} \in \Solved\) or (b) \(\clause[S]{\zeta \dedfact u}{\varphi'} \in \Df\) or (c) \(\clause[S]{\zeta \dedfact u}{\varphi'} \in \varphi\). In case (c), we have in fact \(\varphi' = \emptyset\) and \(S = \emptyset\). Hence, \(\zeta\Sigma' = \zeta\) and \(u\sigma' = u\). Moreover, by hypothesis, \((\Phi\sigma,\Sigma_\varphi,\sigma\sigma_\varphi) \models \varphi\) and \(\varssnd(\varphi) \cap \varssnd(\C) = \emptyset\). Hence, we deduce that \((\Phi\sigma,\Sigma\Sigma_\varphi,\sigma\sigma_\varphi) \models \varphi\) and so \((\Phi\sigma,\Sigma\Sigma_\varphi,\sigma\sigma_\varphi) \models \varphi \zeta\Sigma' \dedfact u\sigma'\). In case (a), we also have that \(\varphi' = \emptyset\) and \(S = \emptyset\). Hence \(\zeta\Sigma' = \zeta\) and \(u\sigma' = u\). Moreover, since \((\Phi\sigma,\Sigma,\sigma) \models \Df\), we directly that \((\Phi\sigma,\Sigma\Sigma_\varphi,\sigma\sigma_\varphi) \models \varphi \zeta\Sigma' \dedfact u\sigma'\).
  %In case (a), we know that \(\PredCorrectFormula(\C)\). Hence by Definition~\ref{def:invariant_correctness_formula}, \((\Phi\sigma,\Sigma,\sigma) \models \Df \wedge \equality{\Eqfst}\) implies \((\Phi\sigma,\Sigma,\sigma) \models \clause[S]{\zeta \dedfact u}{\varphi'}\). Since we already showed that \((\Phi\sigma,\Sigma\Sigma_\varphi,\sigma\sigma_\varphi) \models \varphi'\Sigma'\sigma'\) and since the domains of \(\Sigma,\Sigma_\varphi,\Sigma', \sigma,\sigma_\varphi,\sigma'\) are disjoint, we deduce that \((\Phi\sigma,\Sigma\Sigma_\varphi\Sigma',\sigma\sigma_\varphi\sigma') \models \varphi'\). Hence have that \((\Phi\sigma,\Sigma\Sigma_\varphi\Sigma',\sigma\sigma_\varphi\sigma') \models \zeta \dedfact u\) and so \((\Phi\sigma,\Sigma\Sigma_\varphi,\sigma\sigma_\varphi) \models \zeta\Sigma' \dedfact u\sigma'\). \qed
  %%\end{proof}
  %
  %%In this last lemma, we show that given \((\zeta,u)\)  consequence of the constraint system and a solved deduction formula \(\clause[S]{\xi \dedfact v}{\varphi}\) from the constraint system, all solutions of the constraint systems guarantee \(u \eqs v\) is equivalent to \(\zeta\) and \(\xi\) always deduce the same term.
  %%
  %%%%
  %%%%% Lemma
  %%%%
  %%
  %%\begin{lemma}
  %%\label{lem:consequence_deduction_formula}
  %%Let \(k \in\mathbb{N}\). Let \(\C = \ecsys{\Phi}{\Df}{\Eqfst}{\Eqsnd}{\Solved}{\USolved}{\recipes}\) be an extended constraint system such that \(\PredWellFormed(\C){k}\) and \(\PredCorrectFormula(\C)\). Let \(\psi = (\clause[S]{\xi \dedfact u}{\varphi}) \in \Solved \cup \USolved\) such that \(\psi\) is equation free, \(\names{\varphi} = \emptyset\) and \(\varsfst(\varphi) = \varsfst(S)\). Let \((\zeta,v)\) be a consequence of \(\Solved \cup \SDFrestr{\Df}{k-1} \cup \varphi\). For all \((\Sigma,\sigma) \in \Sol(\C)\), the following properties are equivalent:
  %%\begin{enumerate}
  %%	\item for all substitutions \(\Sigma',\sigma'\), if \(\dom(\Sigma') = \varssnd(S)\), \(\dom(\sigma') = \varsfst(S)\) and \((\Phi\sigma,\Sigma',\sigma') \models \varphi\) then \((\Phi\sigma,\Sigma\Sigma') \models \zeta \eqf \xi\)
  %%	\item for all substitutions \(\Sigma',\sigma'\), if \(\dom(\Sigma') = \varssnd(S)\), \(\sigma'\) is an injective substitution from \(\varsfst(S)\) to \(\sig_0 \setminus \names{\Sigma}\) and for all \((X \dedfact u) \in \varphi\), \(X\Sigma' = u\sigma'\) then \((\Phi\sigma,\Sigma\Sigma') \models \zeta \eqf \xi\) and \((\Phi\sigma,\Sigma',\sigma') \models \varphi\)
  %%	\item there exists \(\delta = \mgu{u}{v}\) such that \((\dom(\delta) \cup \varsfst(\im(\delta))) \cap S = \emptyset\) and \(\sigma \models \bigwedge_{x \in \dom(\delta)} x \eqs x\delta\).
  %%\end{enumerate}
  %%\end{lemma}
  %%
  %%\begin{proof}
  %%	Let \((\Sigma,\sigma) \in \Sol(\C)\). By definition, we know that \((\Phi\sigma,\Sigma,\sigma) \models \Df \wedge \equality{\Eqfst}\). Moreover, by definition of a deduction formula, we know that \(\varssnd(\varphi) \subseteq S\). Furthermore, by hypothesis on \(\psi\), we know that \(\varsfst(\varphi) = S\). Hence, up to renaming, we can consider that \(\varssnd(\varphi) \cap \varssnd(\C) = \emptyset\) and \(\varsfst(\varphi) \cap \varsfst(\C) = \emptyset\). By Lemma~\ref{lem:instantiation_consequence}, we know that for all substitutions \(\Sigma_\varphi,\sigma_\varphi\), if \(\dom(\Sigma_\varphi) = \varssnd(\varphi)\) and \(\dom(\sigma_\varphi)\) and \((\Phi\sigma,\Sigma_\varphi,\sigma\sigma_\varphi) \models \varphi\) then \((\Phi\sigma,\Sigma\Sigma_\varphi,\sigma\sigma_\varphi) \models \zeta \dedfact v\). We will prove that \(1 \Rightarrow 2\), \(2 \Rightarrow 3\) and finally \(3 \Rightarrow 1\).
  %%
  %%	\medskip
  %%
  %%	\emph{Case \(1 \Rightarrow 2\):} Let \(\Sigma',\sigma'\) such that \(\dom(\Sigma') = \varssnd(S)\), \(\sigma'\) is an injective substitution from \(\varsfst(S)\) to \(\sig_0 \setminus \names{\Sigma}\) and for all \((X \dedfact u) \in \varphi\), \(X\Sigma' = u\sigma'\). We know that \(\names{\varphi} = \emptyset\). Moreover, by definition of a deduction formula, for all \((X \dedfact u) \in \varphi\), \(u \in \T(\sigc,\X[1] \cup \N)\). Hence for all \((X \dedfact u) \in \varphi\), \(u \in \T(\sigc,\X[1])\) which implies that \(X\Sigma' = u\sigma' \in \T(\sigc,\sig_0)\). We directly obtain that for all \((X \dedfact u) \in \varphi\), \(\msg(X\Sigma'\Phi\sigma)\) and \(X\Sigma'\Phi\sigma\norm = X\Sigma' = u\sigma'\). Therefore \((\Phi\sigma,\Sigma',\sigma') \models \varphi\). From Property~1, we deduce that \((\Phi\sigma,\Sigma\Sigma') \models \zeta \eqf \xi\).
  %%
  %%
  %%	\medskip
  %%
  %%	\emph{Case \(2 \Rightarrow 3\):} By hypothesis on \(\psi\), we know that \(\varphi\) is of the form \(\bigwedge_{i=1}^n X_i \dedfact u_i\) where all \(X_i\) are pairwise distinct and \(\names{u_i} = \emptyset\). Thus, for all injective mapping \(\sigma_\varphi\) from \(\varsfst(\varphi)\) to \(\sig_0 \setminus \names{\Sigma}\), we can define \(\Sigma_\varphi = \{ X_i \rightarrow u_i \sigma_\varphi \}_{i=1}^n\). In such a case, \((\Phi\sigma,\Sigma_\varphi,\sigma_\varphi) \models \varphi\). From Property~\ref{lab-well_formed-public_name} of \(\PredWellFormed(\C){k}\), we know that \(\names{\C} \cap \sig_0 = \emptyset\). Hence with \(\sigma_\varphi\) being an injective substitution from \(\varsfst(\varphi)\) to \(\sig_0 \setminus \names{\Sigma}\), we deduce that \(\names{\sigma_\varphi} \cap \names{\sigma} = \emptyset\).
  %%
  %%	Considering our hypothesis 2, we deduce that \((\Phi\sigma,\Sigma\Sigma_\varphi) \models \zeta \eqf \xi\). Hence, \(\zeta\Sigma\Sigma_\varphi \Phi\sigma\norm = \xi\Sigma\Sigma_\varphi \Phi\sigma\norm\). But we already proved that \((\Phi\sigma,\Sigma\Sigma_\varphi,\sigma\sigma_\varphi) \models \zeta \dedfact v\). Moreover, \(\PredCorrectFormula(\C)\) also indicates that \((\Phi\sigma,\Sigma\Sigma_\varphi,\sigma\sigma_\varphi) \models \xi \dedfact u\). Thus, \(\zeta\Sigma\Sigma_\varphi\Phi\sigma\norm = v\sigma\sigma_\varphi\) and \(\xi\Sigma\Sigma_\varphi\Phi\sigma\norm = u\sigma\sigma_\varphi\). This allows us to deduce that \(v\sigma\sigma_\varphi = u\sigma\sigma_\varphi\) and so \(u\) and \(v\) are unifiable. Let \(\delta = \mgu{u}{v}\).  By definition of a most general unifier, we know that there exists \(\sigma'\) such that \(\sigma\sigma_\varphi = \delta\sigma'\).
  %%
  %%	We now show that \((\dom(\delta) \cup \varsfst(\im(\delta))) \cap S = \emptyset\). Since Property~\ref{lab-well_formed-public_name} of \(\PredWellFormed(\C){k}\), we know that \(\names{\C} \cap \sig_0 = \emptyset\). Hence, \(\names{u,v} \cap \sig_0 = \emptyset\). Thus, \(\names{\im(\delta)} \cap \sig_0 = \emptyset\). Assume that  \(x \in \dom(\delta) \cap S\). First of all, note that since \(\delta\) is a most general unifier between protocol terms then \(x \in \X[1]\). Thus, \(x \in \dom(\delta) \cap S\) implies \(x \in \varsfst(S)\). Since \(x\in \varsfst(S) = \dom(\sigma_\varphi)\) and \(\varsfst(\varphi) \cap \varsfst(\C) = \emptyset\) then \(x\sigma = x\) and so \(x\sigma\sigma_\varphi = x\sigma_\varphi \in \sig_0\). But \(x \in \dom(\delta)\) implies that \(x\delta \neq x\).
  %%	With \(\names{\im(\delta)} \cap \sig_0 = \emptyset\) and \(x\delta\sigma' \in \sig_0\), we deduce that \(x\delta \in \dom(\sigma')\). With \(\sigma\sigma_\varphi = \delta\sigma'\), we deduce that \(x\delta \in \dom(\sigma\sigma_\varphi)\). But we know that \(x\delta\sigma' = x\sigma_\varphi\) and \(\names{\sigma} \cap \names{\sigma_\varphi} = \emptyset\). Therefore, we obtain that \(x\delta \in \dom(\sigma_\varphi)\).
  %%	Thus, it would imply that there exist \(y \in \dom(\sigma_\varphi)\) such that \(y \neq x\) and \(x\sigma_\varphi = y\sigma_\varphi\). This contradicts the fact that \(\sigma_\varphi\) is an injective mapping. Hence \(S \cap \dom(\delta) = \emptyset\).
  %%
  %%	Assume now that \(x \in \varsfst(\im(\delta)) \cap S\). Since \(x \in \varsfst(\im(\delta))\), there exists \(y \in \dom(\delta)\) such that \(x \in \varsfst(y\delta)\). We already proved that \(y \notin S\). Hence \(y \in \dom(\sigma)\), \(y\sigma = y\sigma\sigma_\varphi\) and \(x\sigma_\varphi \in \subterms(y\sigma)\). But we know that \(\names{\sigma_\varphi} \cap \names{\sigma} = \emptyset\) which is a contradiction. Therefore \((\dom(\delta) \cup \varsfst(\im(\delta))) \cap S = \emptyset\).
  %%
  %%	Lastly, since \(\sigma\sigma_\varphi = \delta\sigma'\) with \((\dom(\delta) \cup \varsfst(\im(\delta))) \cap S = \emptyset\), we deduce that for all \(x \in \dom(\delta)\), \(x \in \dom(\sigma)\) and \(x\delta\sigma = x\delta\sigma\sigma_\varphi\). Thus, \(x\delta\sigma = x\delta\delta\sigma' = x\delta\sigma' = x\sigma\sigma_\varphi\). But \(x \notin S\) implies that \(x\sigma\sigma_\varphi = x\sigma\) and so \(x\delta\sigma = x\sigma\) which allows us to conclude that \(\sigma \models \bigwedge_{x \in \dom(\delta)} x \eqs x\delta\).
  %%
  %%	\medskip
  %%
  %%
  %%	\emph{Case \(3 \Rightarrow 1\):} Let \(\Sigma',\sigma'\) such that \(\dom(\Sigma') = \varssnd(\C)\), \(\dom(\sigma') = \varsfst(\C)\) and \((\Phi\sigma,\Sigma',\sigma') \models \varphi\). By Lemma~\ref{lem:instantiation_consequence}, we deduce that \((\Phi\sigma,\Sigma\Sigma',\sigma\sigma') \models \zeta \dedfact v\). Moreover, \(\PredCorrectFormula(\C)\) gives us that \((\Phi\sigma,\Sigma\Sigma',\sigma\sigma') \models \xi \dedfact u\). Hence we deduce that \(\msg(\zeta\Sigma\Sigma'\Phi\sigma)\), \(\msg(\xi\Sigma\Sigma'\Phi\sigma)\), \(\zeta\Sigma\Sigma'\Phi\sigma\norm = v\sigma\sigma'\) and \(\xi\Sigma\Sigma'\Phi\sigma\norm = u\sigma\sigma'\). But by hypothesis, there exists \(\delta = \mgu{u}{v}\) such that \((\dom(\delta) \cup \varsfst(\im\delta)}) \cap S = \emptyset\) and \(\sigma \models \bigwedge_{x \in \dom(\delta)} x \eqs x\delta\). Hence, \(u\delta = v\delta\) and for all \(x \in \dom(\delta)\), \(x\sigma = x\delta\sigma\).
  %This allows us to deduce that \(u\sigma\sigma' = u\delta\sigma\sigma' = v\delta\sigma\sigma' = v\sigma\sigma'\). Therefore, we can conclude that \((\Phi\sigma,\Sigma\Sigma') \models \zeta \eqf \xi\). \qed
  %%	\end{proof}

  %%
  %%% Subsection
  %%

\subsection{Correctness of most general solutions} \label{app:mgs}

In this section we prove the correctness of the constraint solving procedure for computing mgs' in Section \ref{sec:mgs-gen}.
%
We show that given an extended constraint system \(\C\), the Rules \eqref{rule:conseq}, \eqref{rule:res},
\eqref{rule:cons}, \eqref{rule:mgs-unsat}, and \eqref{rule:unifEqfst_simpl} allow to compute the most general solutions of \(\C\).
%
% Given a constraint system \(\C\), let us denote \(R(\C)\) the set \(\subterms[2](\im(\mgu(\Eqsnd(\C)))) \cup \strsubterms[2](\Solved(\C)) \cup \vars[2](\Df(\C))\)

%%
%%% Lemma
%%

\begin{lemma}
  Let \(\C\) an extended constraint system such that \(\PredWellFormed(\C)\) and \(\PredCorrectFormula(\C)\).
  If any rule is applicable on \(\C\) then for all \((\Sigma,\sigma) \in \Sol(\C)\), there exists \(\C'\), \(\Sigma'\) such that \(\C \SimpStep{} \C'\) and \((\Sigma\Sigma',\sigma) \in \Sol(\C')\).
\end{lemma}

\begin{proof}
  First, assume that there exist \(\xi,\zeta \in R(\C)^2\) and \(u\) such that \(\xi \neq \zeta\) and \((\xi,u),(\zeta,u) \in \conseq(\Solved \cup \Df)\).
  By \(\PredCorrectFormula(\C)\) and Lemmas \ref{lem:consequence_implies_dedfact} and Proposition \ref{prop:trans-conseq},
  we deduce that \(\Phi\sigma \models \xi\Sigma \dedfact u\sigma \wedge \zeta\Sigma \dedfact u\sigma\).
  As such we have \(\xi\Sigma\norm = u\sigma = \zeta\Sigma\norm\).
  However by definition of a solution it implies that \(\xi\Sigma = \zeta\Sigma\).
  Thus there exists \(\Sigma' = \mgu(\xi \eqs \zeta)\) such that \(\Sigma' \neq \emptyset\) and \(\Sigma' \neq \bot\).

  In such a case, let us show that \((\Sigma,\sigma) \in \Sol(\C')\) with \(\C \rightarrow \C'\) by Rule \eqref{rule:conseq}.
  We already know that \(\Sigma \models \Eqsnd(\C)\) and since \(\xi\Sigma = \zeta\Sigma\) with \(\Sigma' = \mgu(\xi \eqs \zeta)\), we directly have that \(\Sigma \models \Eqsnd(\C)\Sigma' \wedge \Sigma'\).
  Moreover, \(\Solved(\C)\Sigma = \Solved(\C)\Sigma'\Sigma\).
  Hence, the two bullets of the definition of solutions is trivially satisfied by that fact that \((\Sigma,\sigma) \in \Sol(\C)\).
  Therefore, we conclude that \((\Sigma,\sigma) \in \Sol(\C')\).

  \smallskip

  Let us now consider the case where our assumption do no hold. Thus since we assume that at least one rule is applicable on \(\C\), there exists \(X \dedfact u \in \Df(\C)\) where \(u \notin \X[1]\). Let us do a case analysis on \(X\Sigma\) since \(X\Sigma \in \conseq(\Solved\Sigma\sigma)\) by definition of a solution.
  \begin{itemize}
    \item either \(X\Sigma \in \sig_0\): in such a case, we have \(\C \rightarrow \C'\) by Rule \eqref{rule:conseq} and we can prove similarly as in the previous case that \((\Sigma,\sigma) \in \Sol(\C')\);
    \item or \(X\Sigma = \ffun(\xi_1,\ldots,\xi_n)\) where \(\xi_i \in \conseq(\Solved\Sigma\sigma)\) for all \(i\):
    note that we know that \(X\Sigma\Phi\sigma\norm = u\sigma\).
    Hence \(u = \ffun(u_1,\ldots, u_n)\) for some \(u_1,\ldots, u_n\).
    We deduce that for all \(i\), \(\Phi\sigma \models \xi_i \dedfact u_i\sigma\).
    Thus, by considering \(\Sigma' = \{ X_i \mapsto \xi_i\}_{i=1}^n\), we can conclude that \(\C \rightarrow \C'\) by Rule \eqref{rule:cons} and \((\Sigma\Sigma',\sigma) \in \Sol(\C')\);
    \item or \(X\Sigma \dedfact u\sigma \in \Solved\Sigma\sigma\) (since once again \(X\Sigma\Phi\sigma\norm = u\sigma\)):
    thus there exists \(\xi \dedfact v \in \Solved\) such that \(\xi\Sigma = X\Sigma\) and \(u\sigma = v\sigma\).
    Hence \(\mgu{\xi}{X}\) exists and \(\sigma \models u \eqs v\).
    Thereofore, we can conclude that \(\C \rightarrow \C'\) by Rule \eqref{rule:res} and \((\Sigma,\sigma) \in \Sol(\C')\). \qedhere
  \end{itemize}
\end{proof}

%%
%%% Lemma
%%

\begin{lemma}
  Let \(\C \neq \bot\) an extended constraint system such that \(\C\simplnorm = \C\), \(\PredWellFormed(\C)\) and \(\PredCorrectFormula(\C)\). If \(\C \not\simpStep{}\) and \(\C\) is a solved extended constraint system then \(\mgs(\C) = \{\mgu(\Eqsnd)\}\).
\end{lemma}

\begin{proof}
  We know that for all \((\Sigma,\sigma) \in \Sol(\C)\), \(\Sigma \models \Eqsnd(\C)\) thus we directly obtain the existence of \(\Sigma'\) such that \(\Sigma = \mgu(\Eqsnd)\Sigma'\).
  Consider now the second bullet point of the definition of most general solutions.
  We know that \(\C\) is solved.
  Hence consider a fresh bijective renaming  \(\Sigma_1\) from \(\vars[2](\Sigma_0) \cup \vars[2](\C) \setminus \dom(\Sigma_0)\) to \(\sig_0\).
  Let us define \(\sigma_1 = \{x \mapsto X\Sigma_1 \mid X \dedfact x \Df(\C)\}\).
  Thanks to \(\PredWellFormed(\C)\), \(\PredCorrectFormula(\C)\) and Lemma \ref{lem:consequence_subterms}, \ref{lem:consequence_subtitution_recipe},
  and \ref{lem:consequence_implies_dedfact} that
  \((\Phi\mgu(\Eqfst(\C))\sigma_1,\mgu(\Eqsnd)\Sigma_1, \mgu(\Eqfst(\C))\sigma_1) \models \Df \wedge \Eqfst \wedge \Eqsnd\).
  Moreover, by Lemma \ref{lem:consequence_subterms}, we know that the first bullet of the definition of solution is satisfied.
  Finally, the second bullet is satisfied otherwise Rule \eqref{rule:mgs-unsat} would be applicable which contradict \(\C\simplnorm = \C\).
  Therefore, \((\mgu(\Eqsnd)\Sigma_1, \mgu(\Eqfst(\C))\sigma_1) \in \Sol(\C)\).
  We conclude that \(\mgs(\C) = \{\mgu(\Eqsnd)\}\).
\end{proof}

%%
%%% Lemma
%%

\begin{lemma}
  Let \(\C\) an extended constraint system such that \(\C\simplnorm = \C\), \(\PredWellFormed(\C)\) and \(\PredCorrectFormula(\C)\).
  If \(\C \not\simpStep{}\) and \(\C\) is not solved then \(\Sol(\C) = \emptyset\).
\end{lemma}

\begin{proof}
  Since \(\C\) is not solved, we have two possibilities:
  Either (a) all deduction facts in \(\Df\) are have variables as right hand term but not pairwise distinct.
  But in such a case Rule \eqref{rule:conseq} would be applicable which contradicts \(\C \not\simpStep{}\);
  or (b) there exists \((X \dedfact u) \in \Df(\C)\) such that \(u \notin \X[1]\).
  Since Rule \eqref{rule:conseq} is not applicable, we deduce that \(u \notin \sig_0\) and for all \(\xi,\zeta \in \recipes(\C) \setminus \{ X\}\), \((\xi,u) \notin \conseq(\Solved \cup \Df)\).
  But rule Rule \eqref{rule:cons} is not applicable therefore, we deduce that \(u \in \Nall\).

  Assume now that \(\Sol(\C) \neq \emptyset\) and so \((\Sigma,\sigma) \in \Sol(\C)\).
  Thus \(X\Sigma\Phi\norm = u\).
  By definition of a solution, we know that \((X\Sigma,u) \in \conseq(\Solved(\C)\Sigma\sigma)\).
  Since \(u \in \Nall\) it implies that there exists \((\xi \dedfact v) \in \Solved(\C)\) such that \(X\Sigma = \xi\Sigma\) and \(u = v\sigma\).
  Note that by \(\PredWellFormed(\C)\), we also have that \(v \notin \X[1]\) and so \(u = v\).
  In such a case, we obtain a contradiction with the fact the Rule \eqref{rule:res} is not applicable.
\end{proof}

% By combining the previous three lemmas, we obtain the expected lemma \ref{lem:mgs}.

\subsection{Correctness of the partition tree} \label{app:ptree-proof}
% \todo[if time, detail the proofs more (in particular regarding the use of the completeness invariant)]

In this section we prove the correctness of the procedure generating the partition tree, using the invariants proved in Appendix \ref{app:invariants}.
%
Let us first start by noticing that the case distinction rules and simplification rules preserves the first order solutions of the extended constraint systems. This property is stated in the following lemma.

\begin{lemma} \label{lem:preservation_solutions}
  Let \(\S\) be a set of set of extended symbolic processes such that \(\PredAll(\S)\).
  Let \(\S \rightarrow \S'\) by applying only case distinction or simplifications rules (i.e. no symbolic transitions).
  Then:
  \begin{itemize}
    \item \emph{Soundness:}

    for all \(S \in \S\), for all \((\P,\C,\C^e) \in S\), for all \((\Sigma,\sigma) \in \Sol(\C^e)\),
    there exist \(S' \in \S'\), \((\P,\C,{\C^e}') \in S'\) and \((\Sigma',\sigma') \in \Sol({\C^e}')\) such that \(\sigma_{|\vars[1](\C)} = \sigma'_{|\vars[1](\C)}\)

    \item \emph{Completeness:}

    for all \(S' \in \S\), for all \((\P,\C,{\C^e}') \in S'\), for all \((\Sigma',\sigma') \in \Sol({\C^e}')\),
    there exist \(S \in \S\), \((\P,\C,\C^e) \in S\) and \((\Sigma,\sigma) \in \Sol(\C^e)\) such that \(\Sigma_{|\vars[2](\C)} = \Sigma'_{|\vars[2](\C)}\) and \(\sigma_{|\vars[1](\C)} = \sigma'_{|\vars[1](\C)}\)
  \end{itemize}
\end{lemma}

\begin{proof}
  We do a case analysis on the rule applied.

  \caseitem{\emph{case 1:} Normalisation rules (simplification rules of Figure \ref{fig:normalisation_constraint_systems})}

    First, let us notice the result directly hold for Rules \ref{rule:unifEqfst_norm}, \ref{rule:Disequation removal 2}, \ref{rule:Removal of unsolved formula}.
    Indeed, Rule \ref{rule:Disequation removal 1} does not modify constraints on recipe and preserves the constraints on protocol terms.
    Moreover, Rule \ref{rule:Disequation removal 2},\ref{rule:Removal of unsolved formula} affect \(\USolved\) which do not impact the solutions of the extended constraint system.
    For Rule \ref{rule:uniform}, since \(\mgs(\C^e) = \emptyset\), we have by definition of most general unifiers that \(\Sol(\C^e) = \emptyset\) (otherwise the first bullet of the definition is contradicted).
    Hence the result holds since \(\Sol(\bot) = \emptyset\). Similarly, Rule \ref{rule:Disequation removal 2} checks whether the disequations \(\forall \tilde{x}.\phi\) is trivially true meaning that the rule preserves the solutions.

  \caseitem{\emph{case 2:} Simplification rules on partitions of extended symbolic processes (Figure \ref{fig:normalisation_vector})}

    The rule \ref{rule:vectorbot} only removes an extended symbolic process with an extended constraint systems having no solution hence the result holds.
    Rule \ref{rule:vector-split solved} splits a set of \(\S\) into two sets thus preserving the extended symbolic processes, and Rule \ref{rule:vector-consequence} only adds element in \(\USolved\) which do not impact the solutions of a constraint system.
    Therefore, for all these rules, the result hold.
    For Rule \ref{rule:vector-solve} however, the result is not direct since the rule adds an element in the set \(\Solved\) which has an impact on the solutions of a constraint system.
    However, we know from the application condition of the rule that the head protocol terms of the deduction facts added in \(\Solved_i\) are not consequence of \(\Solved_i \cup \Df_i\).
    But we also know that \(\PredConseq(\S)\) and \(\PredWellFormed(\S)\) hold hence it implies that the recipe \(\xi\) (see Figure \ref{fig:normalisation_vector}) contains \(\ax_{|\Phi_i|}\) and \(\vars[2](\Df_i) \cap \Xsndi{|\Phi_i|} = \emptyset\).
    Hence, \(\xi\) cannot appear in the second order solutions \(\C^e_i\) which allows us to conclude that the solutions are preserved.

  \caseitem{\emph{case 3:} Case distinction rules}

    The case of case distinction rules is straightforward.
    Indeed, by definition all rules \eqref{rule:satisfiable}, \eqref{rule:equality} and \eqref{rule:rewrite} always refine a set of extended symbolic process \(\Gamma\) into \(\Gamma_1,\Gamma_2\)
    where \(\Gamma_1\) is obtained by applying a substitution \(\Sigma\) to \(\Gamma\), and \(\Gamma_2\) by adding the constraint \(\neg \Sigma\) to \(\Gamma\).
    In particular this refinement preserves the solutions as expected.
\end{proof}

% Besides by using the soundness and completeness invariants (Invariants 2 and 3) we can prove that the criterion used by Rule \eqref{rule:vector-split solved} to split equivalence classes in the partition tree is correct.
%
% \begin{lemma} \label{lem:split-correct}
%   Let \(\Gamma\) a set of symbolic extended symbolic processes such that \(\PredCompleteFormula(\Gamma)\).
%   Then for all \((\P_1,\C_1,\C^e_1), (\P_2,\C_2,\C^e_2) \in \Gamma\), for all \(\psi_1 \in \USolved(\C^e_1)\), if for all \(\psi_2 \in \USolved(\C^e_2)\), \(\psi_1 \not\receq \psi_2\)
%   then for all \((\Sigma,\sigma_1) \in \Sol(\C^e_1)\) and \((\Sigma,\sigma_2) \in \Sol(\C^e_2)\), \((\Phi(\C^e_2),\Sigma,\sigma_2)\) does not weakly satisfy the head of \(\psi_1\) (in the sense of the Definition \ref{def:invariant_complete_formula} in Invariant 3).
% \end{lemma}
%
% \begin{proof}
%   \todo 
% \end{proof}

% That is, when two extended symbolic processes \(\C_1,\C_2\) are separated during the procedure due to a formula \(\psi_1\) holding in \(\C_1\) but not in \(\C_2\), this means that the head of \(\psi_1\) is falsified by all solutions of \(\C_2\).
% Keeping in mind the completeness (Invariant 3) this gives as expected that a formula holds in \(\C_1\) but not in \(\C_2\), therefore they should be put in different nodes of the partition tree.
%
Now we can show that the static equivalence is preserved by application of the case distinction and simplification rules.

\begin{lemma}
  Let \(\S\) be a set of set of extended symbolic processes such that \(\PredAll(\S)\).
  Let \(\S \rightarrow \S'\) by applying only case distinction or simplifications rules (i.e. no symbolic transitions), \(\Gamma \in \S\), \((\P_1,\C_1,\C^e_1), (\P_2,\C_2,\C^e_2) \in \Gamma\),
  \((\Sigma,\sigma_1) \in \Sol(\C^e_1)\) and \((\Sigma,\sigma_2) \in \Sol(\C^e_2)\) such that \(\Phi(\C_1)\sigma_1 \StatEq \Phi(\C_2)\sigma_2\).

  % \todo[maybe rephrase to link more clearly with previous lemma. See p86-87 tech report]
\end{lemma}

\begin{proof}
  % \todo[WTH the proof and the statement are not related]
  Once again, let us consider the potential rule applied.

  \caseitem{\emph{case 1:} Simplification rules}

    The only non trivial case is Rule \ref{rule:vector-split solved} (the other ones do not refine the partition and the conclusion is therefore immediate).
    However by Lemma
    % \todo[retrieve Pvect with completeness]
    applied to \(\S\) we know that if a deduction fact occurs in constraint systems \(\C^e_1\) but no recipe equivalent formula can be found in the constraint system \(\C^e_2\), then no solution of \(\C^e_2\) can satisfy the head of the formula.
    Besides by \(\PredCorrectFormula(\S)\) we also know that all solutions of \(\C^e_1\) satisfy this deduction fact.
    Then since \((\Sigma,\sigma_1) \in \Sol(\C^e_1)\), \((\Sigma,\sigma_2) \in \Sol(\C^e_2)\) and \(\Phi(\C_1)\sigma_1 \sim \Phi(\C_2)\sigma_2\), we obtain a contradiction.
    Therefore, \(\C^e_1\) and \(\C^e_2\) are necessarily in the same set of \(\S'\).

  \caseitem{\emph{case 2:} Case distinction rules}

    Note that for case distinction rules, the proof is simple since each rule create a partition of the second-order solutions with respect to some mgs \(\Sigma_0\).
    Thus, assume w.l.o.g. that \((\Sigma',\sigma'_1) \in \Sol({\C^e_1}')\).

  \caseitem{\emph{case 2a:} negative branch of the rule}
    First consider that \(S'\) corresponds to branch in which we applied \(\neg \Sigma_0\).
    In such a case, since we already know that \(\Sigma'\) satisfies \(\neg \Sigma_0\) and no other constraint is added, we directly obtain from \((\Sigma,\sigma_2) \in \Sol(\C^e_2)\) that \((\Sigma',\sigma'_2) \in \Sol({\C^e_2}')\) (in this case, we even have \(\sigma_2 = \sigma'_2\)).

  \caseitem{\emph{case 2b:} positive branch of the rule}
    Now consider that \(S'\) corresponds to the branch in which we applied \(\Sigma_0\).
    In such a case, the application of \(\Sigma_0\) on \(\C^e_2\) regroups all the solution of \(\C^e_2\) that satisfies \(\Sigma_0\).
    Since we know that \((\Sigma,\sigma_2) \in \Sol(\C^e_2)\) and \({\Sigma'}_{|\vars[2](\C_1)} = {\Sigma}_{|\vars[2](\C_1)}\) which implies \({\Sigma'}_{|\vars[2](\C_2)} = {\Sigma}_{|\vars[2](\C_2)}\), the result holds.
\end{proof}

The previous two lemmas allow us to obtain the soundness and completeness properties of the partition tree.
% (i.e.~items \ref{lab-partition_tree-parent_concrete_derivation,lab-partition_tree-child_concrete_derivation}).
Note that the monoticity of the second-order predicate
% ~\ref{lab-partition_tree-monotonicity_pi}
is also proved by~\ref{lem:preservation_solutions} (Completeness part) since a solution of a child constraint system is also a solution of parent one.
We now need to prove that all nodes of the partition tree are valid configurations.
For that we prove properties on extended constraint systems such that no more case distinction rules are applicable.


\begin{lemma} \label{lem:sat-non-applicable}
  Let \(\S\) be a set of sets of extended symbolic processes such that \(\PredAll(\S)\) and no instance of the rule \eqref{rule:satisfiable} or normalisation rules (i.e. the simplification rules of Figure \ref{fig:normalisation_constraint_systems}) are applicable.
  For all \(S \in \S\), for all \((\P,\C,\C^e) \in S\), writing \(\C^e = (\Phi, \Df, \Eqfst, \Eqsnd, \Solved, \USolved)\) we have that
  \begin{enumerate}
    \item \(\C^e\) is solved
    \item all formulas \(\psi \in \USolved\) are solved
    \item \(\Eqfst\) does not contain disequations
  \end{enumerate}
\end{lemma}

\begin{proof}
  First of all the non-applicability of Rule \eqref{rule:satisfiable} case \ref{it:rule-sat-mgs} gives that either \(\C^e\) is solved or \(\mgs(\C^e) = \emptyset\);
  due to normalisation rules not being applicable we deduce that \(\mgs(\C^e) \neq \emptyset\) meaning that \(\C^e\) is solved.
  %
  Sinmilarly by the non applicability of case \ref{it:rule-sat-hyp} we know that for all \(\psi \in \USolved\), either \(\psi\) is solved or \(\mgs(\C^e[\Eqfst \mapsto \Eqfst \wedge \Fhyp(\psi)]) = \emptyset\).
  But since the normalisation rules are also not applicable, we know that \(\mgs(\C^e[\Eqfst \mapsto \Eqfst \wedge \Fhyp(\psi)]) \neq \emptyset\):
  therefore \(\psi\) is solved.
  Finally the non applicability of case \ref{it:rule-sat-diseq} and of the normalisation rules also gives us that \(\Eqfst\) is only composed of syntactic equations.
\end{proof}

\begin{lemma}
  Let \(\S\) be a set of sets of extended symbolic processes such that \(\PredAll(\S)\) and no instance of the rule \eqref{rule:satisfiable} or normalisation rules are applicable.
  For all \(S \in \S\), for all \((\P,\C,\C^e) \in S\), \(|\mgs(\C^e)| = 1\).
\end{lemma}

\begin{proof}
  Let us denote \(\C^e = (\Phi, \Df, \Eqfst, \Eqsnd, \Solved, \USolved)\).
  By Lemma \ref{lem:sat-non-applicable} we know that \(\C^e\) is solved, that all formulas \(\psi \in \USolved\) are solved, and that \(\Eqfst\) only contain equations.

  \caseitem{\emph{Step 1:} Construction of \((\Sigma,\sigma)\) such that \((\Phi,\Sigma,\sigma) \models \Df \wedge \Eqfst \wedge \Eqsnd\)}
    Since \(\C^e\) is solved, we deduce that all deduction facts in \(\Df = \{ X_i \dedfact x_i \}_{i=1}^n\) for some \(n\) and pairwise distinct \(x_i\)s and \(X_i\)s.
    Consider now the substitutions \(\Sigma_0 = \{ X_i \rightarrow n_i\}_{i=1}^n\) and \(\sigma_0 = \{ x_i \rightarrow n_i\}_{i=1}^n\) where the \(n_i\)s are pairwise distincts public names, i.e. \(n_i \in \sig_0\).
    Since no more normalisation rules are applicable, we know that the disequations in \(\Eqsnd\) not trivially unsatisfiable.
    Therefore by replacing the free variables of the disequations by names allow us to obtain that \(\Sigma_0\) the disequations of \(\Eqsnd\).
    By considering \(\Sigma = \mgu(\Eqsnd)\Sigma'\), we obtain that \(\Sigma \models \Eqsnd\).
    Moreover we proved that \(\Eqfst\) does not contain any disequations, we directly obtain that \(\mgu(\Eqfst)\sigma_0 \models \Eqfst\).
    Therefore, by defining \(\sigma = \mgu(\Eqfst)\sigma_0\), we obtain that \((\Phi,\Sigma,\sigma) \models \Df \wedge \Eqfst \wedge \Eqsnd\).

  \caseitem{\emph{Step 2:} Proof that \((\Sigma,\sigma)\) is a solution}
    To prove that \((\Sigma,\sigma)\) is an actual solution of \(\C^e\) it remains to prove that it verifies the additional required two conditions:
    \(\Solved\)-basis and uniformity. %(see Definition \ref{def:solution extended constraint system}).
    %
    Let us first prove the \(\Solved\)-basis, i.e. that for all \(\xi \in \subterms(\im(\Sigma)) \cup \strsubterms[2](\Solved \Sigma)\), \(\msg(\xi \Phi \sigma)\) and \((\xi,\xi\Phi\sigma) \in \conseq(\Solved \Sigma \sigma)\).
    The case \(\xi \in \strsubterms[2](\Solved \Sigma)\) directly follows from \(\PredWellFormed(\C^e)\).
    Let us therefore consider the case \(\xi \in \subterms(\im(\Sigma))\).
    Since \(\PredWellFormed(\C^e)\) holds we have that \(\im(\mgu(\Eqsnd)) \subseteq \conseq(\Solved \cup \Df)\);
    for the same reason we have that for all \(\zeta \dedfact u \in \Solved\), \(\subterms(\zeta) \subseteq \conseq(\Solved \cup \Df)\).
    Therefore by applying Lemma \ref{lem:consequence_subtitution_recipe}, and by a quick induction on the size of the recipe in \(\im(\Sigma)\), we obtain that \(\xi \in \conseq(\Solved\Sigma)\).
    Finally, by definition of consequence and since \(\PredCorrectFormula(\C^e)\) holds, we have \(\msg(\xi\Phi\sigma)\) hence the \(\Solved\)-basis.

    Let us now prove uniformity.
    We know that \(\C^e\) is solved which therefore means that for all recipes \(\xi,\zeta \in \stc(\im(\mgu(\Eqsnd)),\Solved \cup \Df)^2 \cup (\sig_0 \times \vars[2](\Df))\), \((\xi,u), (\zeta,u) \in \conseq(\Solved \cup \Df)\) implies \(\xi = \zeta\).
    Since \(\Sigma = \mgu(\Eqsnd)\Sigma_0\) we directly obtain that for all \(\xi,\zeta \in \stc(\Sigma,\Solved\Sigma)\), \((\xi,u),(\zeta,u) \in \conseq(\Solved\Sigma)\) implies \(\xi = \zeta\), which is exactly the uniformity.

  \caseitem{\emph{Step 3:} Unicity of the solution}
    This step is rather straightforward:
    considering that \(\Sigma = \mgu(\Eqsnd)\Sigma_0\) and any other solutions \((\Sigma',\sigma') \in \Sol(\C^e)\) satisfy \(\Sigma' \models \Eqsnd\), we deduce that \(\mgs(\C^e) = \{ \mgu(\Eqsnd) \}\) and so \(|\mgs(\C^e)| = 1\).
\end{proof}

Let us now show that all extended constraint systems in the set have the same solutions and that they are statically equivalent.

\begin{lemma}
  Let \(\S\) be a set of set of extended symbolic processes such that \(\PredAll(\S)\) and no instances of the rules \eqref{rule:satisfiable}, \eqref{rule:equality} or \eqref{rule:rewrite} or simplification rules are applicable.
  For all \(S \in \S\), for all \((\P_1,\C_1,\C^e_1), (\P_2,\C_2,\C^e_2) \in S\), if \((\Sigma,\sigma_1) \in \Sol(\C^e_1)\) then \((\Sigma,\sigma_2) \in \Sol(\C^e_2)\) and \(\Phi(\C^e_1)\sigma_1 \StatEq \Phi(\C^e_2)\sigma_2\).
\end{lemma}

\begin{proof}
  Since \eqref{rule:satisfiable} and normalisation rules are not applicable, we know by Lemma \ref{lem:sat-non-applicable} that all extended constraint systems \(\C^e \in S\) have a particular form, that is
  \begin{enumerate*}
    \item all deduction facts in \(\Df(\C^e)\) have pairwise distinct variables as right hand side; and
    \item \(\Eqfst(\C^e)\) only contain syntactic equations.
  \end{enumerate*}
  Moreover, we know that all extended constraint systems have the same structure.
  Therefore, if \((\Sigma,\sigma_1) \in \Sol(\C^e_1)\), we deduce that \(\Sigma \models \Eqsnd(\C^e_1)\) and for all \(\xi \in \subterms(\im(\Sigma))\),
  \(\xi \in \conseq(\Solved(\C^e_1)\Sigma)\), meaning that \(\Sigma \models \Eqsnd(\C^e_2)\) and \(\xi \in \conseq(\Solved(\C^e_2)\Sigma)\).
  Since the first order solutions are always completely defined by the second-order substitutions, we can build \(\sigma'_2\) such that for all \(X \dedfact x \in \Df(\C^e_2)\), \(X\Sigma(\Phi(\C^e_2)\sigma'_2)\norm = x\sigma'_2\).
  Moreover, since \(\PredCorrectFormula(\C^e_1)\) and \(\PredCorrectFormula(\C^e_2)\) both hold and since for all \(\xi \in \subterms(\im(\Sigma))\), \(\xi \in \conseq(\Solved(\C^e_2)\Sigma)\), we deduce that for all \(\xi \in \subterms(\im(\Sigma))\), \(\msg(\xi\Phi(\C^e_2)\sigma'_2)\).
  Note that we also need to satisfy the syntactic equations in \(\Eqfst\).
  However thanks to \(\PredWellFormed(\C^e_2)\) holding, we know that \(\dom(\mgu(\Eqfst(\C^e_2)) \cap \vars[1](\Df(\C^e_2)) = \emptyset\).
  Thus, we can build \(\sigma_2 = \mgu(\Eqfst(\C^e_2))\sigma'_2\) and obtain that \((\Sigma,\sigma_2) \models \Df(\C^e_2) \wedge \Eqfst(\C^e_2) \wedge \Eqsnd(\C^e_2)\).
  Note that by origination property of an extended constraint system, we have \(\Phi(\C^e_2)\sigma'_2 = \Phi(\C^e_2)\sigma_2\).
  Therefore, since we already prove that for all \(\xi \in \subterms(\im(\Sigma))\), \(\msg(\xi\Phi(\C^e_2)\sigma'_2)\) and \(\xi \in \conseq(\Solved(\C^e_2)\Sigma)\),
  it only remains to prove the second bullet point of Definition the definition of solutions extended constraint system to obtain that \((\Sigma,\sigma_2) \in \Sol(\C^e_2)\).

  To prove this it sufficies to prove that \(\Phi(\C^e_1)\sigma_1 \StatEq \Phi(\C^e_2)\sigma_2\):
  the conclusion will then follow since \((\Sigma,\sigma_1) \in \Sol(\C^e_1)\).
  %
  Therefore we let recipes \(\xi,\xi'\) and show that:
  \begin{enumerate}[label=(\roman*)]
    \item \label{it:ptree-config-msg}
      \(\msg(\xi\Phi(\C^e_1)\sigma_1)\) iff \(\msg(\xi\Phi(\C^e_2)\sigma_2)\)
    \item \label{it:ptree-config-stateq}
      if \(\msg(\xi\Phi(\C^e_1)\sigma_1),\ \xi'\Phi(\C^e_1)\sigma_1)\) then \(\xi\Phi(\C^e_1)\sigma_1\norm = \xi'\Phi(\C^e_1)\sigma_1\norm\) iff \(\xi\Phi(\C^e_2)\sigma_2\norm = \xi'\Phi(\C^e_2)\sigma_2\norm\).
  \end{enumerate}
  We prove this by lexicographic induction on \((N(\xi,\xi'),\max(|\xi|,|\xi'|)\) where \(N(\xi\,\xi')\) is the number of subterms \(\zeta \in \subterms(\xi,\xi')\) such that \(\zeta \notin \conseq(\Solved(\C^e_1)\Sigma)\)
  (recall that since \(\C^e_1\) and \(\C^e_2\) have the same structure, we have \(\zeta \in \conseq(\Solved(\C^e_1)\Sigma)\) iff \(\zeta \in \conseq(\Solved(\C^e_2)\Sigma)\)).

  \caseitem{\emph{case 1:} \(N(\xi,\xi') = 0\) and \(\max(|\xi|,|\xi'|) = 0\)}
    Impossible since there exist no terms of size \(0\).

  \caseitem{\emph{case 2:} \(N(\xi,\xi') > 0\)}
  \caseitem{\emph{subgoal 2a}: Proof of \ref{it:ptree-config-msg}}
    Assume \(\msg(\xi\Phi(\C^e_1)\sigma_1)\).
    Let us also assume by contradiction that \(\neg \msg(\xi\Phi(\C^e_2)\sigma_2)\).
    Since we know that \(N(\xi,\xi') > 0\), there exists \(\zeta \in \subterms(\xi,\xi')\) such that \(\zeta \not \in \conseq(\Solved(\C^e_1)\).
    Without loss of generality we can consider that \(\zeta \in \subterms(\xi)\) (otherwise we can apply our inductive hypothesis on \(\xi\) twice since \(N(\xi,\xi')\) would be equal to \(0\) and so we would obtain a contradiction).
    Moreover, let us consider \(\zeta\) such that \(|\zeta|\) is minimal.
    Therefore, by definition of consequence, we deduce that \(\zeta = \gfun(\zeta_1,\ldots, \zeta_n)\) with \(\gfun \in \sigd\) and for all \(i \in \{1,\ldots,n\}\), \(\zeta_i \in \conseq(\Solved(\C^e_1))\).
    Since \(\msg(\xi\Phi(\C^e_1)\sigma_1)\) we also deduce that \(\gfun(\zeta_1,\ldots, \zeta_n)\Phi(\C^e_1)\sigma_1\norm\) is a protocol term.
    Therefore, there exist a rewrite rule \(\gfun(\ell_1,\ldots, \ell_n) \rightarrow r\) and a substitution \(\gamma\) such that \(\ell_i\gamma = \zeta_i\Phi(\C^e_1)\sigma_1\norm\) for all \(i = 1\ldots n\).

    % \todo[double check that it is still correct with the condition on rewrite that the position p is not a variable]
    Recall that the rule \eqref{rule:rewrite} is not applicable on \(\C^e_1\) and \(\C^e_2\).
    Therefore we can show that provided \(\neg \msg(\xi\Phi(\C^e_2)\sigma_2)\) and \(\gfun(\zeta_1,\ldots, \zeta_n)\Phi(\C^e_1)\sigma_1\norm\) is a protocol term then we necessarily have that there exists \(\zeta'_1,\ldots, \zeta'_n\) and \(u\)
    such that \(\gfun(\zeta'_1,\ldots, \zeta'_n) \dedfact u_1 \in \USolved(\C^e_1)\) and \(\zeta'_i\Sigma\Phi(\C^e_1)\sigma_1\norm = \zeta_i\Phi(\C^e_1)\sigma_1\norm\).
    Moreover, since the normalisation rules are also not applicable (in particular Rule \ref{rule:vector-split solved}), we deduce that there exists \(u_2\) such that \(\gfun(\zeta'_1,\ldots, \zeta'_n) \dedfact u_2 \in \USolved(\C^e_2)\). By \(\PredWellFormed(\C^e_1)\),
    we know that for all \(i \in \{1,\ldots, n\}\), \(\zeta'_i \in \conseq(\Solved(\C^e_1) \cup \Df(\C^e_1))\) and so \(\zeta'_i\Sigma \in \conseq(\Solved(\C^e_1)\Sigma)\).
    Moreover, by hypothesis on \(\zeta_i\), we know that \(\zeta_i \in \conseq(\Solved(\C^e_1)\Sigma)\).
    Thus, by applying our inductive hypothesis, we obtain that \(\zeta_i\Phi(\C^e_2)\sigma_2\norm = \zeta'_i\Sigma\Phi(\C^e_2)\sigma_2\norm\).
    Moreover, by \(\PredCorrectFormula(\C^e_2)\), we know that \(\gfun(\zeta'_1,\ldots, \zeta'_n)\Sigma\Phi(\C^e_2)\sigma_2\norm = u_2\sigma_2\) which is a protocol term.
    We conclude that \(\gfun(\zeta_1,\ldots, \zeta_n)\Sigma\Phi(\C^e_2)\sigma_2\norm\) is a protocol term and thus \(\msg(\xi\Phi(\C^e_2)\sigma_2)\) gives us a contradiction.

  \caseitem{\emph{subgoal 2b}: Proof of \ref{it:ptree-config-stateq}}
    Assume now that \(\xi\Phi(\C^e_1)\sigma_1\norm = \xi'\Phi(\C^e_1)\sigma_1\norm\), \(\msg(\xi\Phi(\C^e_1)\sigma_1)\) and \(\msg(\xi'\Phi(\C^e_1)\sigma_1)\).
    Let us once again take the smallest \(\zeta \in \subterms(\xi,\xi')\) such that \(\zeta \not \in \conseq(\Solved(\C^e_1)\).
    We already proved above that there exist \(u_1,u_2\), \(\gfun\), \(\zeta'_1,\ldots, \zeta'_n, \zeta_1,\ldots, \zeta_n\) such that:
    \begin{itemize}
      \item \(\zeta = \gfun(\zeta_1,\ldots, \zeta_n)\)
      \item \(\gfun(\zeta'_1,\ldots, \zeta'_n) \dedfact u_1 \in \USolved(\C^e_1)\)
      \item \(\gfun(\zeta'_1,\ldots, \zeta'_n) \dedfact u_2 \in \USolved(\C^e_2)\)
      \item for all \(i \in \{1,\ldots, n\}\), \(\zeta_i\Phi(\C^e_2)\sigma_2\norm = \zeta'_i\Sigma\Phi(\C^e_2)\sigma_2\norm\) and \(\zeta_i\Phi(\C^e_1)\sigma_1\norm = \zeta'_i\Sigma\Phi(\C^e_1)\sigma_1\norm\).
    \end{itemize}
    By \(\PredConseq(\C^e_1)\), we know that there exists \(\beta\) such that \((\beta,u_1) \in \conseq(\Solved(\C^e_1) \cup \Df(\C^e_1)\).
    However the normalisation Rule \ref{rule:vector-consequence} is not applicable on the set of extended symbolic processes.
    Thus, we deduce that there exists \(\beta'\) such that \((\beta',u_1) \in \conseq(\Solved(\C^e_1) \cup \Df(\C^e_1)\) and \(\gfun(\zeta'_1,\ldots, \zeta'_n) \eqf \beta' \in \USolved(\C^e_1)\).
    Once again due to the normalisation Rule \ref{rule:vector-split solved}, we obtain that \(\gfun(\zeta'_1,\ldots, \zeta'_n) \eqf \beta' \in \USolved(\C^e_2)\).
    But \(\PredCorrectFormula(\C^e_2)\) and \(\PredCorrectFormula(\C^e_1)\) hold meaning that \((\Phi(\C^e_2)\sigma_2,\Sigma,\sigma_2) \models \gfun(\zeta'_1,\ldots, \zeta'_n) \eqf \beta'\) and \((\Phi(\C^e_1)\sigma_1,\Sigma,\sigma_1) \models \gfun(\zeta'_1,\ldots, \zeta'_n) \eqf \beta'\).

    Note that if \(p\) is the position of \(\zeta\) in \(\xi\) then we have \(N(\replacepos{\xi}{p}{\beta'\Sigma}, \xi') < N(\xi,\xi')\).
    Thus by applying our inductive hypothesis, we obtain that \(	(\Phi(\C^e_2)\sigma_2,\Sigma,\sigma_2) \models \replacepos{\xi}{p}{\beta'\Sigma} \eqf \xi'\).
    Since \((\Phi(\C^e_2)\sigma_2,\Sigma,\sigma_2) \models \gfun(\zeta'_1,\ldots, \zeta'_n) \eqf \beta'\)
    and \((\Phi(\C^e_2)\sigma_2,\Sigma,\sigma_2) \models \gfun(\zeta'_1,\ldots, \zeta'_n) \eqf \gfun(\zeta_1,\ldots, \zeta_n)\),
    we conclude that \((\Phi(\C^e_2)\sigma_2,\Sigma,\sigma_2) \models \xi \eqf \xi'\).

  \caseitem{\emph{case 3:} \(N(\xi,\xi') = 0\) and \(\max(|\xi|,\xi'|) > 0\)}
    In such a case, we know that \(\xi, \xi' \in \conseq(\Solved(\C^e_1)\Sigma)\) and \(\xi, \xi' \in \conseq(\Solved(\C^e_2)\Sigma)\).
    By definition of consequence and by \(\PredCorrectFormula(\C^e_1)\) and \(\PredCorrectFormula(\C^e_2)\), we directly obtain that \(\msg(\xi\Phi(\C^e_1)\sigma_1)\) and \(\msg(\xi\Phi(\C^e_2)\sigma_2)\) (same thing for \(\xi'\)).
    Now assume that \((\Phi(\C^e_1)\sigma_1,\Sigma,\sigma_1) \models \xi \eqf \xi'\).
    Since both \(\xi,\xi'\) are consequences of \(\Solved(\C^e_1)\Sigma\), we deduce that:
    \begin{itemize}
      \item either \(\xi = \ffun(\xi_1,\ldots, \xi_n)\) and \(\xi' = \ffun(\xi'_1,\ldots,\xi'_n)\) with \(\ffun \in \sigc\) and \((\Phi(\C^e_1)\sigma_1,\Sigma,\sigma_1) \models \xi_i \eqf \xi'_i\) for all \(i\).
      Therefore, we can apply our inductive hypothesis on the \((\xi_i,\xi'_i)\)s to conclude.
      \item or \(\xi\Sigma,\xi'\Sigma \in \Solved(\C^e_1)\Sigma\):
      Since we know that the rule \eqref{rule:equality} is not applicable, it implies that \(\xi \eqf \xi' \in \USolved(\C^e_1)\) and so \(\xi \eqf \xi' \in \USolved(\C^e_2)\) thanks to the normalisation Rule \ref{rule:vector-split solved}.
      Since \(\PredCorrectFormula(\C^e_2)\) holds, we can conclude that \((\Phi(\C^e_2)\sigma_2,\Sigma,\sigma_2) \models \xi \eqf \xi'\).
      \item or \(\xi\Sigma \in \Solved(\C^e_1)\Sigma\) and \(\xi' = \ffun(\xi'_1,\ldots,\xi'_n)\) with \(\ffun \in \sigc\);
      Once again since the rule \eqref{rule:equality} is not applicable, we deduce that there exists \(\zeta'_1,\ldots, \zeta'_n\) such that \(\xi \eqf \ffun(\zeta'_1,\ldots, \zeta'_n) \in \USolved(\C^e_1)\).
      Note from \(\PredCorrectFormula(\C^e_1)\) that in such a case, \((\Phi(\C^e_1)\sigma_1,\Sigma,\sigma_1) \models \xi \eqf \ffun(\zeta'_1,\ldots, \zeta'_n)\)
      meaning that \((\Phi(\C^e_1)\sigma_1,\Sigma,\sigma_1) \models \xi'_i \eqf \zeta'_i\) for all \(i \in \{1,\ldots, n\}\).
      Since \(|\xi'_i \Phi(\C^e_1)\sigma_1\norm| < |\xi\Phi(\C^e_1)\sigma_1\norm|\), we can apply our inductive hypothesis on all \((\xi'_i,\zeta'_i)\) meaning that \((\Phi(\C^e_2)\sigma_2,\Sigma,\sigma_2) \models \ffun(\zeta'_1,\ldots, \zeta'_n) \eqf \xi'\).
      However, by the rule \ref{rule:vector-split solved} not being applicable,
      \(\xi \eqf \ffun(\zeta_1,\ldots, \zeta'_n) \in \USolved(\C^e_1)\) implies \(\xi \eqf \ffun(\zeta'_1,\ldots, \zeta'_n) \in \USolved(\C^e_2)\) and so by \(\PredCorrectFormula(\C^e_2)\),
      we obtain that \((\Phi(\C^e_2)\sigma_2,\Sigma,\sigma_2) \models \xi \eqf \ffun(\zeta'_1,\ldots, \zeta'_n)\) which allows us to conclude that \((\Phi(\C^e_2)\sigma_2,\Sigma,\sigma_2) \models \xi \eqf \xi'\).
      \qedhere
  \end{itemize}
\end{proof}


% \paragraph{Additional property for bisimilarity}
%   \textcolor{blue}{\textbf{NOTE: This property is only useful for complexity, not for the correctness of the procedure. In particular it may be better to remove it from the definition of the partition tree and only present it as a intermediary lemma of the complexity analysis.}}
%
%   \textcolor{blue}{\textbf{NOTE: When computing the solution of the witness using unification of the mgs' of the nodes of the partition tree, variables may be renamed if they are from parallel nodes and were not introduced in a common ancestor.}}
%
%   Sketch of the proof (\todo[to be detailed]).
%   \begin{itemize}
%     \item only prove the property on subset of nodes that are closed by ancestor (i.e. \(n \in \S\) implies \(n\) is the root or the father of \(n\) is in \(S\)).
%     This is sufficient (\(S\) will always be the set of nodes involved in the symbolic witness, which has this property) and this permits to do a bottom-up induction on the tree structure.
%     \item to get around the issue of variable renaming, operate on a partition tree with renamed variables:
%     each node \(n\) is renamed with \(\rho(n)\) where \(\rho(n_0) = \id\) if \(n_0\) is the root, and \(\rho(n') = \rho(n) \rho\) if \(n \xrightarrow{\ell} n'\) and \(\rho\) is a fresh variable renaming of \(\vars(\ell)\).
%     \item Proof by induction:
%     let \(n \in S\) and \(n_1, \ldots, n_p\) all the children nodes of \(n\) belonging to \(S\).
%     The case \(p = 0\) is immediate.
%     Otherwise apply induction hypothesis to these children nodes to get \(\gamma_1, \ldots, \gamma_p\).
%     Write their mgu \(\gamma\) and assume \(\mgs(\CApply{\gamma}{n_i}) \neq \emptyset\) for all \(i\).
%     By completeness of the symbolic semantics, a solution of these children nodes is also a solution of the father.
%     In particular the fact that \(\mgs(\CApply{\gamma}{n}) = \{\mgs(n)\gamma\}\) will follow from the fact that \(\mgs(\CApply{\gamma}{n_1}) = \{\mgs(n_1)\gamma\}\).
%     It therefore only remains to prove that \(\mgs(\CApply{\gamma}{n_i}) = \{\mgs(n_i)\gamma\}\) for all \(i\).
%     The satisfaction of \(\Eqfst\), \(\Eqsnd\) and the uniformity is straightforward;
%     the only non-trivial point is that \(\mgs(n_i)\gamma\) verifies \(\Solved\)-basis.
%     But the construction of the partition tree + the renaming ensure that the only second-order variables shared by \(n_i,n_j\), \(n_i \neq n_j\), are from a common ancestor.
%     In particular the knowledge base is already satured, hence the conclusion.
%   \end{itemize}
