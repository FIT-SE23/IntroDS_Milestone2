\section{Computational Hardness}
\label{sec:hardness}


In this section, we show that the design of the optimal mechanism for maximizing the principal's value 
is computationally hard by reduction from the NP-hard integer valued subset sum problem. 

\paragraph{Integer valued subset sum.}
Given $n$ integers $z_1,\dots,z_n$
and a target $Z > z_i$ for all $i\in[n]$, 
does there exists a set $\effortset\in[n]$
such that $\sum_{i\in\effortset}z_i = Z$? 


Our proof idea is similar to the reduction from the subset sum to the knapsack problem.
The main challenge for reduction to our problem is that, 
in order to prevent the agent from randomly guessing the states of the tasks, 
there is a specific incentive constraint that determines the set of incentivizable tasks. 
The incentive compatible constraint potentially generates a much smaller value than the optimal set of tasks with total costs below the budget. 
To avoid this randomly guessing issue, 
we add additional tasks to the scoring rule design problem
such that the agent's utility from making any random guess
is sufficiently low, 
and that the optimal objective value of the principal exceeds a given value
if and only if the objective $Z$ of the subset sum problem can be achieved. 


\begin{theorem}\label{thm:np-hard}
Computing the optimal mechanism in the knapsack scoring problem is NP-hard even if the valuation function is additive.
\end{theorem}
\begin{proof}
Given an integer valued subset sum instance
with integer parameters $z_1,\dots,z_n$ and $Z$, we construct a knapsack scoring problem. 
Let $\bar{\val} = 1+\sum_{i\in[n]}z_n$
and $\bar{\cost} = 1+\max_{i\in[n]}z_i$. 
Let $k$ be the minimum integer such that $2^{kn} > Z+2kn\bar{\cost}+1$. 
It is easy to see that the value of $k$ is polynomial in the number of digits to represent $Z$ and $\max_{i\in[n]}z_i$.
Construct a knapsack scoring problem with $(2k+1)n$ tasks
such that if the agent exerts effort on any task $i$, 
he observes the state $\outcome_i$ with probability $1$. The values and costs of the tasks are defined in the following way:
\begin{itemize}
    \item for each task $i\leq n$, let value and cost be $\val_i = \cost_i = z_i$;
    \item for each task $n+1\leq i\leq (2k+1)n$, let $\val_i = \bar{\val}$
and $\cost_i = \bar{\cost}$.
\end{itemize}
The budget of the principal is $Z+2kn\bar{\cost}+1$.
Note that this instance can be easily converted to our problem with budget 1 by re-scaling the budget and the costs by the same factor. 
We claim that the subset sum problem is true if and only if the optimal objective value for the knapsack scoring problem is $Z+2kn\bar{\val}$. 

If the optimal objective value for the knapsack scoring problem is $Z+2kn\bar{\val}$, 
this implies that in the optimal solution, 
the agent is incentivized to exert effort on all tasks $n+1\leq i\leq (2k+1)n$, 
which has a total contribution of $2kn\bar{\val}$. 
Thus the agent must exert effort on a subset $\effortset\subseteq[n]$ such that $\sum_{i\in\effortset} \val_i = Z$. 
Since $\val_i = z_i$ for all $i\in[n]$, 
$\effortset$ is a solution for the integer valued subset sum problem. 

If there exists a set of integers $\subsetsum\subseteq[n]$
such that $\sum_{i\in\subsetsum} z_i = Z$, 
consider the threshold scoring rule with recommendation set $\effortset=\subsetsum\cup\{n+1,\dots,(2k+1)n\}$ and threshold $\threshold=|\effortset|$, which scores budget $Z+2kn\bar{\cost}+1$ if the agents predicts all tasks in recommendation set $\effortset$ correctly. It is easy to verify that the utility of the agent for exerting effort on all tasks $i\in\effortset$ is $1$. The utility of the principal on recommendation set $\effortset$ is $Z+2kn\bar{\val}$. We are going to show this threshold scoring rule is incentive compatible and optimal. 

To prove this threshold scoring rule is incentive compatible, we divide agent's deviation into two cases:  1) the agent exerts effort on a small subset, so that he has to randomly guess  on a large number of tasks, which reduces his utility; 2) the agent exerts effort on a large subset, which induces a high total cost.
\begin{itemize}
    \item If the agent chooses to exert effort on a subset with size at most $\abs{\subsetsum} + kn$, he has to make random guess on at least $kn$ tasks. The utility of the agent is at most 
$2^{-kn}\cdot (Z+2kn\bar{\cost}+1) < 1$, 
which is strictly smaller than his utility for exerting effort on all tasks $i\in\effortset$.
\item If the agent chooses to exert effort on a subset with size between $\abs{\subsetsum} + kn$ and $\abs{\subsetsum} + 2kn -1$, 
the cost of effort for the agent is at least 
$Z+kn\bar{\cost} \geq \frac{1}{2}(Z+2kn\bar{\cost}+1)$ since $Z \geq 1$. 
Moreover, the expected payment to the agent 
is at most $\frac{1}{2}(Z+2kn\bar{\cost}+1)$
since the agent has to make a random guess on at least one task. 
This implies that the agent's utility is negative given this deviating strategy. 
\end{itemize}
Thus the agent's optimal choice is to exert effort on all tasks $i\in\effortset$.


% Consider the scoring rule $\score$ that provides reward $Z+2kn\bar{\cost}+1$ to the agent
% if and only if the agent reports the states correctly for all states 
% $i\in\optset$.
% We show that the agent's optimal choice is to exert effort on all tasks $i\in\optset$, 
% where the utility of the principal in this case is $Z+2kn\bar{\val}$.
% It is easy to verify that the utility of the agent for exerting effort on all tasks $i\in\optset$ is $1$. 
% If the agent chooses to exert effort on a subset with size at most $\abs{\effortset} + kn$, 
% then the agent has to make random guess on at least $kn$ tasks.
% The utility of the agent is at most 
% $2^{-kn}\cdot (Z+2kn\bar{\cost}+1) < 1$, 
% which is strictly smaller than his utility for exerting effort on all tasks $i\in\optset$. 
% If the agent chooses to exert effort on a subset with size between $\abs{\effortset} + kn$ and $\abs{\effortset} + 2kn -1$, 
% the cost of effort for the agent is at least 
% $Z+kn\bar{\cost} > \frac{1}{2}(Z+2kn\bar{\cost}+1)$ since $Z \geq 1$. 
% Moreover, the expected payment to the agent 
% is at most $\frac{1}{2}(Z+2kn\bar{\cost}+1)$
% since the agent has to make a random guess on at least one task. 
% This implies that the agent's utility is negative given this deviating strategy. 
% Thus the agent's optimal choice is to exert effort on all tasks $i\in\optset$.
Finally, we show that the optimal utility of the principal cannot exceed $Z+2kn\bar{\val}$.
Note that for the principal to obtain utility at least $Z+2kn\bar{\val}$, 
the agent must be incentivized to exert effort on all tasks $i\in \{n+1,\dots,(2k+1)n\}$
since the sum of value in $[n]$ is strictly below the value of any task $i\in \{n+1,\dots,(2k+1)n\}$. 
Moreover, the total cost of the agent for exerting effort given the optimal scoring rule is strictly less than $Z+2kn\bar{\cost}+1$
since the agent can obtain strictly positive utility by exerting no effort and randomly guessing. 
Since the costs are integer valued, 
the total cost is at most $Z+2kn\bar{\cost}$.
As the total cost for exerting effort on tasks $i\in \{n+1,\dots,(2k+1)n\}$
is $2kn\bar{\cost}$, 
the cost of the agent on tasks within subset $[n]$ is at most $Z$. 
Since the value coincides with the cost in this case, 
the value of the principal from incentivizing the agent to exert effort on tasks within $[n]$ is at most $Z$. 
Therefore, the optimal utility of the principal is $Z+2kn\bar{\val}$.
\end{proof}

