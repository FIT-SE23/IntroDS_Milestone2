\paragraph{Related Work}

% other knapsack mechanism design problems.

Prior work has considered mechanism design problems based on strategic
versions of the knapsack problem.  One framing is that of
single-minded multi-unit demand agents as buyers with a seller with a
multi-unit supply constraint.  In this model, only the values of the
agents can be strategically manipulated.  \citet*{BKV-05} considered
welfare maximization with this framing and gave a general method for
converting polynomial time approximation schemes (including the one
for knapsack) into incentive compatible mechanisms (with the same
approximation guarantees).  \citet*{AH-06} considers the same framing
with the goal of revenue maximization and a natural prior-free
benchmark.

Another knapsack framing reverses the buyer and seller roles: The
agents are sellers with private costs (object sizes in knapsack) and
the buyer aims to hire a team (set of sellers) to maximize value
but has a budget constraint (capacity of the knapsack).
\citet*{sin-10} posed this question and gave prior-free approximation
mechanisms when the buyers value function is submodular (generalizing
the linear value function of the traditional knapsack problem).
\citet*{BCGL-12,BCGL-17} considered the budget-feasibility question in
the Bayesian and prior-independent models of mechanism design and give constant approximations.
\citet*{BH-16} consider the Bayesian budget feasibility problem and
showed that posted pricing mechanisms give good approximation to the
Bayesian optimal mechanism.  In comparison to the literature on budget
feasibility, this paper's model of scoring rule optimization has a
single agent (resp. multiple agents) with a multi-dimensional strategy
space (resp. single-dimensional), the costs are public
(resp. private), but effort is private (resp. public).  With private
effort, the principal optimizing a scoring rule can only validate the
agent's effort in so far as the agent's posterior information from
effort improves over her prior information.

Multi-dimensional mechanism design problems are notoriously difficult.
In the classical setting of selling multiple items to a single agent
with multi-dimensional preferences, the algorithmic mechanism design
literature has identified simple constant-approximation mechanisms in a
number of canonical settings.  \citet*{BILW-14,BILW-20} show that for
an agent with independent additive values for multiple items then the
better of bundling or selling separately is a constant approximation.
\citet*{RW-15,RW-18} extend this approximation result to agents with
subadditive valuations.  See \citet*{BILW-20} for discussion of the
extensive literature generalizing these results.  These bundling
versus selling separately results are paralleled by this paper's
result showing that the better of truncated separate scoring or
threshold scoring is a constant approximation.



% optimization of scoring rules for effort

This work builds on the general framework for optimizing scoring rules
for effort that was initiated by \citet*{HLSW-20}.  Their main result
considers binary effort and multi-dimensional state.  In contrast, the
model of this paper is for multi-dimensional effort and
multi-dimensional state, but with a 1-to-1 correspondence between the
dimension of effort and state.

% extensions

\citet*{CY-21} consider the design of scoring rules for maximizing a
binary effort in a max-min design framework.  For example,
complementing a prior-independent result from \citet*{HLSW-20}, they
show that the quadratic scoring rule is max-min optimal over a large
family of distributional settings.  \citet*{kon-22} apply the framework
of effort-maximization to multi-agent peer prediction where the
principal does not have access to the ground truth state and instead
must compare reports across several agents.

% other scoring rule optimization
Several papers look at optimizing for multiple levels of a
single-dimensional effort with the objective of accuracy of the
forecast (i.e., the posterior from effort which is reported in a
proper scoring rule).  \citet*{osb-89} considers optimization of
quadratic scoring rules with a continuous level of
effort.  \citet*{Z-11} characterizes the optimal single-dimensional
scoring rule when the states are partially
verifiable.  \citet*{NNW-21} consider optimization of scoring rules for
integral levels of effort where the effort corresponds to a number of
costly samples drawn.
%These models are incomparable to this paper and
%those described above.


Optimization of effort in scoring rules has similarities to the
problem of optimizing effort in contracts, the main difference being
that, in the classical model of contract design the distribution over
states for each action is common knowledge.  In contract for scoring
rules, on taking an action the agent receives a signal that gives the
agent private information about the distribution of states. For the
contract design problems, \citet*{castiglioni2022designing} show that
the optimal contract can be computed in time polynomial in the number
of potential actions of the agent even when the costs of actions are
private information.  For the multi-dimensional effort model, the
number of actions is exponential in the size of the dimensions,
and \citet*{dutting2022combinatorial} show that with binary states, the
optimal contract can be computed in polynomial time if the function
mapping the action choices to the state distributions satisfies the
gross substitutes property, but is NP-hard when the function is more
generally submodular.


\paragraph{Future Directions}
The approach of the paper is one of Bayesian mechanism design where
the prior distribution is known to both the principal (instructor) and
agent (student).  Within the Bayesian model there are three main
directions for future work.  First, the positive results of this paper
are restricted to simplistic distributions over posteriors.  As
discussed in \Cref{sec:general info}, generalizing the results beyond this case
necessitates better upper bounds and richer families of approximation
mechanisms.  Second, our multi-dimensional effort-to-state mapping is
one-to-one.  It is an open direction to combine results for
multi-dimensional effort with the model of \citet*{HLSW-20} for
single-dimensional effort with multi-dimensional state.  Third, for
our motivating application in the classroom, the cost of effort varies
across students. It is an open direction to combine our model for
optimizing scoring rules with the model of budget feasibility where
the cost of effort is private.

Bayesian mechanism design is the first model in which to consider
novel mechanism design problems.  To obtain practical mechanisms,
however, it is important to consider robust versions of the problem.
The two canonical frameworks are that of prior-independence and sample
complexity.  Prior-independent framework looks to identify one
mechanism that has the best approximation to the Bayesian optimal
mechanism in worst case over distributions.  The sample complexity
framework looks to bound the number of samples necessary to obtain a
$1+\epsilon$ approximation to the Bayesian optimal mechanism.
\citet*{HLSW-20}, for example, gave such results for the problem of
designing scoring rules for a single-dimensional effort.  These are
open directions for optimizing multi-dimensional effort via scoring
rules.


