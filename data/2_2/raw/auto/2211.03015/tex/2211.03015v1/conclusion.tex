\section{Conclusion} \label{conclusion}

Zero-click attacks are incredibly difficult to detect and pose a pervasive threat to the privacy and security of smartphone users around the world. 
This paper reports on our experiences and the lessons learned while attempting to design and implement a secure framework with standard, off-the-shelf components to protect and limit the impact of a potential zero-click attack on smartphones.
We enumerated several design requirements and presented our envisioned security architecture that shifted the attack surface from the user's device to a sandboxed virtual smartphone ecosystem where sensitive applications run in isolation. 
The users interacted with remote applications using screen sharing and remote access service to prevent the zero-click exploit from accessing the device. 
We demonstrated that it is indeed feasible to practically build such a secure framework using COTS components by accessing chat applications running remotely in containerized Android Emulators using WebRTC-based screen sharing. 
We tested the performance and usability of our implementation through an IRB-exempted research study.
Finally, we highlighted the missing components necessary to achieve security against zero-click attacks and improve users' experience. 
We hope this research will help others build effective remote and isolated systems against zero-click smartphone spyware. 
