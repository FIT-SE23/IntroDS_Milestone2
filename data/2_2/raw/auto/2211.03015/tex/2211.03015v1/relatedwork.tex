\subsection{Current State-Of-The-Art} \label{related-work}

The technical details regarding how a zero-click exploit stealthily compromises a smartphone without user interaction are largely unknown.  
For this reason, the existing proactive and reactive security systems, such as mobile anti-virus, intrusion detection and prevention systems that detect the presence of advanced mobile malware by analyzing parameters like requested permissions~\cite{permission}, API and system calls~\cite{apicall, droidcat}, network addresses~\cite{clickip}, resource consumption~\cite{resources}, etc., have completely failed to detect and prevent zero-click attacks.
Even the security experts have only put forward recommendations to keep the mobile OS and installed applications updated, which cannot be considered a panacea for secure communication. 

Over the years, Amnesty International Security Lab forensically investigated several Pegasus-infected smartphones. The lab publicly disclosed their forensic methodology~\cite{forensics} and released an automated tool, MVT~\cite{mvt}, that analyzes device backups to verify if the smartphone is infected with Pegasus spyware or not.  
However, MVT cannot detect and prevent zero-click attacks in real time or stop the leakage of sensitive information after the attack execution.  
  
In an attempt to guarantee protection against mercenary zero-click exploits, tech giant Apple recently rolled out an optional feature of \textit{Lockdown mode} with the release of iOS 16 in fall 2022~\cite{lockdown}. 
The Lockdown mode aims to strictly restrict certain features such as receiving texts from unknown numbers, displaying link previews in messages, wired connections with computers, etc.
The Lockdown mode was found to be overly prohibitive as the user cannot even download and view any attachment (other than images) on the phone. 
Since Lockdown mode disables custom fonts for websites to prevent execution of malicious JavaScript code, the website administrator can detect missing fonts on a device and indirectly know that the user is potentially a high-profile target. 
As websites already log the IP address of the visitors, the acquired information can be further used to fingerprint users or devices~\cite{lmfingerprinting}. 
While Apple users can benefit from this groundbreaking feature in the near future, Android users do not have any concrete defensive or preventive solution against zero-click attacks yet.

Unfortunately, most notable research works that claim to have high zero-day detection accuracy have been carried out on non-mobile OS~\cite{deep2020ids, network2020, network2021, cloud2015}, whereas the aforementioned zero-click exploits are carefully designed to abuse zero-day bugs in smart chat
applications.  
The few research works that focus on the detection of mobile-based zero-day malware, as discussed below, typically employ signature or anomaly-based detection schemes~\cite{survey2020}.

AndroSimilar~\cite{androsimilar2013} and DroidAnalytic~\cite{droid2013} specifically used advanced signature-based detection techniques to identify zero-day repackaged malware (i.e., unseen variants of the known mobile malware).
However, such approaches required exhaustive collection of malware samples and could not detect new zero-day vulnerabilities.  
Several studies also demonstrated the feasibility of detecting Android-based zero-day malware by analyzing features like API sequence call, opcode, permissions, etc., using machine learning approaches; Bayesian classification~\cite{bayesian2013, bernoulli2015}, and deep learning~\cite{deep2021, amin2020}.  
However, as zero-day malware evades anti-malware detection by hiding its malicious activity on detecting an isolated or sandboxed background, these detection approaches are extremely slow~\cite{mobile2019} and ineffective.

In contrast, existing anomaly-based detection approaches looked for deviations in smartphone activity from the normalized baselines~\cite{linear2013} to detect zero-day malware.  
RiskRanker~\cite{riskranker2012} and Andro-AutoPsy~\cite{andro2015} particularly observed if any application exhibits malicious activity, such as trying to launch a root exploit, whereas, DroidLight~\cite{droidlight2020} used one-class classification and probability distribution analysis to detect zero-day malware.  
However, where RiskRanker and DroidLight required intensive processing power, Andro-AutoPsy could not analyze malware that employed anti-malware and encryption techniques.  
In short, since the existing research works could not effectively detect zero-day malware on a mobile OS, they will also be ineffective against sophisticated zero-click exploits that use native libraries and advanced obfuscation techniques to hide their traces.

Owing to the lack of defensive solutions against zero-day attacks, we also investigated existing research studies on virtual smartphones~\cite{chen2010, svmp} that run Android OS on the cloud platform to run sensitive applications away from the user's device.  
However, such setups are now obsolete and do not support the latest versions of Android OS. 
To the best of our knowledge, currently, there is no holistic preventive or defensive solution that can be used as an immediate safeguard against zero-click attacks, particularly for Android users.