\section{Introduction}

The increasing use of smartphones for communications, such as banking and social networking, has made them an attractive target for cyber criminals.  
These malicious actors used social engineering to lure victims into clicking a malicious link or pressing a button, thereby causing the malware to execute, proliferate and compromise the victim's smartphone successfully.  
However, the interaction requirement has made it difficult for the attacker to compromise the technically-savvy targets.
Cyber criminals are now using zero-click exploits that do not require interaction from the user and abuse zero-day (i.e., unpatched) vulnerabilities in smart applications, typically instant messaging applications, to control the victim's smartphone.  
Zero-click attacks are stealthy in nature, and avoid persistence to escape detection by anti-malware utilities and forensic tools~\cite{forensics}.
This makes it extremely difficult to detect and prevent zero-click attacks.
Recently, it was discovered that a sophisticated spyware, named \textit{Pegasus}~\cite{pegasus}, had used zero-click exploits to spy on renowned journalists, human rights defenders, political dissidents, business executives, and lawyers around the world for several years.  
With a successful zero-click attack, the attacker can gather sensitive data (such as messages, contacts, photos, files, emails, and usage history), log the victim's keystrokes (which can further leak passwords, security tokens, and credit card information), determine the victim's location, and even remotely activate the device's camera and microphone at any time for real-time surveillance (Figure~\ref{fig:problem}). 
In short, zero-click attacks completely invade the victim's privacy and threaten personal security.  
As a matter of fact, the attacker only requires the victim's phone number to send the malicious zero-day exploit; hence acquiring access to the victim's contact list also endangers the smartphones of many more users. 

\begin{figure}[t]
\begin{center}
\includegraphics[width=\linewidth]{figures/problem.pdf}
\caption{Attacker can exploit zero-click zero-day bug in a chat application (e.g. WhatsApp) to gain root access to the target device and exfiltrate sensitive information.}
\label{fig:problem}
\end{center}
\end{figure}

Given that the zero-click attacks largely exploit zero-day vulnerabilities, existing measures such as mobile anti-virus or malware detection systems have completely failed to thwart zero-click attacks.  
%So far, there is no concrete safeguard against zero-click attacks, besides only recommendations by security experts to keep the mobile OS updated and remove unnecessary applications from the smartphone.  
Currently, Mobile Verification Toolkit (MVT)~\cite{mvt}, developed by Amnesty International Security Lab, is the only tool that can analyze device backup to find signs of potential zero-click compromise (provided such attack vectors are already known).
%Tech giant, Apple, is soon launching \textit{Lockdown mode} with the release of iOS 16, that aims to 
In the absence of any real-time counteractive measure against zero-click attacks, the user himself is responsible for securing the smartphone. 

A naive way of protecting one's privacy from zero-click exploits is not to use a smartphone or consider using a burner phone with a burner sim. 
However, such options limit the user's ability to even use the Internet for browsing.  
Alternatively, the user may resort to a secondary smartphone for running chat applications, but a zero-click exploit targeting a single chat application can also leverage cross-application vulnerabilities to compromise other chat applications on the smartphone.
Nevertheless, the user may rely on web or desktop clients instead of smart applications, e.g., for WhatsApp. However, this requires him to carry the laptop everywhere. Also, not all chat applications have web/ desktop interfaces; thus, even allowing just SMS messages to be delivered on the smartphone can risk the smartphone's security. 
In essence, there is an urgent need to design and develop security mechanisms that protect against zero-click attacks. 
More importantly, the security framework should primarily aim to significantly limit the loss of information in case of an exploit, as software bugs are here to stay. 

\begin{figure}[t]
\begin{center}
\includegraphics[width=\linewidth]{figures/benefit.pdf}
\caption{Our attempt at building a Zero-Click Secure Architecture: The user accesses remote isolated chat applications via screen sharing, which confines zero-click exploit to the targeted application only, and safeguards other applications on the server as well as the target device.}
\label{fig:benefit}
\end{center}
\end{figure}


In this paper, we report on our experiences and lessons learned in our attempt to design and develop a security framework against zero-click attacks.  
Our goal was to provide high-risk individuals, such as investigative journalists, who reached out to us with a secure solution that allowed them to use their chat application of choice while guaranteeing maximum privacy in case of a surveillance attack.  
We started off by studying the zero-click attack landscape and enumerated several properties we believe are necessary to limit the impact (e.g., amount of information leaked) and lifespan of the attack (e.g., duration of the surveillance). 
For example, as zero-click exploits primarily leverage vulnerabilities within and across chat applications to compromise a device, it was essential to prohibit any direct delivery of messages to the user's smartphone.  
Our key intuition was that a remote access mechanism was needed such that the user could interact with the messages without worrying about any potential exploit that could compromise the user's smartphone. 
To this extent, we attempted to build a zero-click secure framework that shifted the attack surface from the user's device to a virtual smartphone ecosystem (remote server) that runs each chat application in a sandbox-like isolated environment (Figure~\ref{fig:benefit}). 
The motivation for our design was to answer the question: Can we build a zero-click secure framework using readily available off-the-shelf components? 

We practically implemented a proof-of-concept using commercial off-the-shelf (COTS) based software components by running each application in a containerized Android emulator and using Web Real-Time Communication (WebRTC) for screen sharing and remote interaction.  
To understand the feasibility and practicality of our solution, we conducted a user study that primarily evaluated usability and performance.  
Note that the user study had received exemption by the Institutional Review Board (IRB) under the IRB Category-3 Benign Behavioral Interventions.  
Our evaluations highlighted several shortcomings and fundamental challenges that exist in securing smartphones against zero-click attacks.  

Specifically, this paper makes the following contributions:
\begin{enumerate}
\item We enumerate several design requirements to protect smartphones against zero-click attacks. 
We assume that zero-day vulnerabilities will continue to exist, and therefore, our requirements are geared towards building a scalable and usable security architecture that strictly prevents zero-click exploits from compromising the user's device, and limits the impact and lifespan of a potential attack. 
\item We evaluate the performance and usability of the secure framework we built to attempt to solve the zero-click problem, and highlight the challenges while building such a solution using COTS components.
\item We distill five concrete lessons from our experience of attempting to build a zero-click secure architecture for smartphones. 
We discuss our experiences in finding a reliable, scalable and cost-effective sandboxing solution to run the mobile OS on a remote server 24/7, and our attempts to create a system that works over the Internet with minimal network lag and components that allow users to interact with the apps seamlessly with a high degree of interaction quality.

\end{enumerate}

Our aim in this work is not to present a solution against zero-click attacks that works perfectly in practice. 
Rather, through this work, based on our attempts (some of which were not very successful) to build such a system using standard software components, we aim to identify the challenges, limitations, and key research opportunities towards realizing a zero-click protection system for users. 
We hope our experiences will be useful to other researchers working to address zero-click attacks against mobile devices.
