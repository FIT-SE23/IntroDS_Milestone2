\section{Our Envisioned Secure Framework} \label{design}
\subsection{Threat Scenario}

In our research, based on our discussions with dissident journalists, we considered a practical scenario where the victim is a high-profile target who relies on secure, popular, end-to-end encrypted messaging applications (i.e., WhatsApp and Signal) for official and private communication using their smartphone.  
We assume that the attacker knows the victim's phone number and is equipped with a powerful zero-click exploit code that leverages a zero-day vulnerability in one of the installed chat applications (e.g., WhatsApp) to compromise the smartphone without the victim's interaction. 
The attacker aims to attain unfettered access to the victim's smartphone and covertly extract information such as chats, contacts, keystrokes, and location or record real-time audio and video. 
The above threat mirrors recent revelations regarding the secret surveillance of high-profile targets using the Pegasus spyware.

One question that may come up is why the victim needs to use WhatsApp and cannot simply resort to secure email? 
This is because WhatsApp is ubiquitous, provides good security against eavesdropping by authorities, and in the case of dissident journalists, is frequently already used by their sources.

\subsection{Desirable Characteristics}
First, we identified the necessary properties that would make a system zero-click resistant (i.e., even in the case of a successful zero-click exploit, the information leaked to the adversary is significantly limited). 
We envision these characteristics as essential to provide ideal security against zero-click attacks with minimal system and attacker assumptions. 

\begin{itemize}
%[leftmargin=*]
\item \textbf{No direct delivery of messages to the smartphone:} 
Zero-click exploits typically abuse zero-day vulnerabilities in chat applications to compromise a device.
The mere delivery of a message containing the payload exploit is sufficient to compromise the victim's smartphone.  
Therefore, it is necessary to prevent any message from being delivered directly to the smartphone. 

\item \textbf{Remote access to messages:} 
Since the messages are not delivered directly to the smartphone, one idea would be to provide the user with a mechanism to interact with the messages remotely.  
It is important to note that the messages cannot be delivered to the device at any point, not even temporarily. 
This is in contrast to using a web application to access these messages. 
For example, opening an image sent over email in the web interface would still "download" the image on the device. 
Zero-click exploits can potentially evade input sanitization, so any malicious zero-click code can escape the browser's sandbox and access the device's data. Therefore messages need to be accessed completely in place outside the victim device. 

\item \textbf{Temporal and spatial application isolation:} 
We believe that the solution should provide temporal and spatial isolation between applications. 
This is because zero-click exploits targeting a specific chat application (e.g., WhatsApp) can potentially abuse cross-application vulnerabilities to infect other installed applications (e.g., Signal). 
Hence, it would be necessary to isolate individual chat applications to limit the impact of the attack to the vulnerable application only.
Moreover, the framework would need to limit the lateral damage of a successful zero-click attack and prevent the attacker to passively eavesdrop the vulnerable chat application for a long time.

\item \textbf{Scalable and usable:} 
Finally, it is likely that the user needs to add more third-party chat applications to the remote server in the future.
%an organization will have several high-risk individuals who need protection. 
Hence, the system must be scalable with respect to the increase in the number of applications.  
For complete adoption, usability plays a critical role. 
Therefore, the system would need to facilitate seamless access to messages over the Internet without requiring significant user interaction. 
\end{itemize}

\begin{figure}[t]
\begin{center}
\includegraphics[width=\linewidth]{figures/design.pdf}
\caption{Experimental Zero-Click Secure Architecture: User accesses remote isolated applications via screen sharing and remote control mechanism. Regular messages are routed through cloud SMS Gateway, and the server is periodically reset to the initial unaffected snapshot to remove any zero-click infection (if present).}
\label{fig:design}
\end{center}
\end{figure}

\subsection{An Experimental Zero-click Secure Architecture}
Based on the above properties, we attempted to design a zero-click secure architecture that we present below.  
One of the primary motivations for the design was to be able to realize it with off-the-shelf components. 
Since a single zero-click exploit is potent enough to gain complete access to the smartphone without any user interaction, it is imperative to block (or reduce) all possible attack vectors.
Unfortunately, zero-click exploits leverage unpatched vulnerabilities, making it difficult for smartphone users or even anti-malware solutions to detect or prevent such attacks.  
The core idea of our attempted secure framework (as illustrated in Figure \ref{fig:design}) was to shift the attack surface from the user's device (hereby called client) to a virtual smartphone (such as cloud or remote server) that runs all the sensitive applications.

Since chat applications, by default, parse messages even from unknown/ untrusted numbers, they are an obvious zero-click attack vector. 
Hence, all chat applications must be shifted to the remote server such that the user has unattended access to the remote applications at all times. 
Third-party chat applications such as WhatsApp and Signal can be configured on the server effortlessly, as they only require one-time authentication using a one-time password while registering the application.
However, shifting in-built messaging applications from the client device to the remote server is challenging.  
We leveraged a cloud-based SMS Gateway, which uses HTTPS messages to enable any device (even computers) to send and receive SMS over the telecommunication network. 
To recall, shifting chat applications remotely prevents the zero-click exploits from reaching the client device and activating its hardware for real-time surveillance.

It is pertinent to note that moving the chat applications to a remote server alone is insufficient. 
Zero-click exploits can evade message sanitization; hence, even if the server sanitizes each received message before forwarding it to the client device, there is an obvious risk of the zero-click exploit being forwarded to the device. 
Our secure framework allowed the user to indirectly access the chat applications on the remote server using a user-intuitive screen sharing and remote control mechanism. 
Unlike sending sanitized input to the client, screen sharing takes screen pixels and shares them with the client, thereby removing any possibility of the malware reaching the client device.  
In addition, the remote control feature allows the user to control the chat applications remotely.  
If the user wishes to send a message, the user's input (keystrokes) on the mirrored screen is forwarded to the remote server and then to the intended recipient.

Our design rationale was that screen sharing considerably reduces the window of attack opportunity and protects the client device against zero-click attacks; however, as a zero-click exploit targeting a specific application can also exploit cross-application vulnerabilities, the \emph{remaining applications on the server} are also susceptible to the zero-click attack. 
To limit the attack's impact, we run each application in isolation in a sandbox-like environment, for instance, a separate virtual machine or docker container.
So far, there is no evidence of \emph{cross-OS exploitation} for zero-click exploits, as the existing zero-click exploits are specifically designed to abuse a certain zero-day bug in a smart application installed on the mobile OS. 
Hence, for example, the zero-click exploit cannot exfiltrate from the Android instance A running vulnerable application A to the non-mobile host OS (e.g., Linux server) and into another Android instance B to compromise application B.  
With this setup, the zero-click attack, if launched, should be restricted to the targeted smart application only (i.e., application A).

Furthermore, to limit the time of successful zero-click exploitation and prevent the attacker from passively eavesdropping on the targeted chat application for a long time, we thought about periodically resetting each instance to the initial unaffected snapshot (e.g., every three days or as required by the user). 
Our key intuition here was that this would help terminate any malicious connection established by the attacker, as forensic investigations of Pegasus-infected smartphones have proved that Pegasus avoids persistence to evade anti-malware detection, and resetting the smartphone to factory settings removes the spyware~\cite{persistence}. 
Hence, this would require additional effort from the attacker to re-infect the target application and would increase the difficulty bar.

In essence, the fundamental idea of our secure framework was to subside the risks and damages caused by a zero-click infection by shifting the attack surface from the client to the remote server and allowing the user to access remote, isolated chat applications via screen mirroring only. 