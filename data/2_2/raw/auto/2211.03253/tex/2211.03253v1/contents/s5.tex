\section{Conclusion and future work} \label{sec: 5}
This work introduces a novel design for a soft robotic link (ProTac) which features both tactile and proximity perception enabled by changing the transparency of the skin (the PDLC film in particular). The presented design and fabrication process can be extended to other vision-based tactile sensors in literature to date. Furthermore, the evaluation results of system performance in both modes certify the feasibility and potential of this idea in real robotic arms.

However, there are still some aspects that need further elaboration. Firstly, the simulation model used to collect referenced tactile data for model training does not accurately reflect the behaviors of ProTac skin. The mechanical properties of the ProTac skin should be attributed to the coupling of the silicon-made soft layer and the PDLC film. Secondly, the mechanical stability of the robot arm constituted by ProTac links in high-load operation is still questionable. An in-depth structural analysis for static and dynamic manner should be addressed in future work. Moreover, it is necessary to attain an optimal strategy for switching back and forth between two sensing modes to facilitate dexterous control and perception of the robot arm in different tasks. The effectiveness of multimodal perception in the creation of a safe and intelligent human-robot interaction should be further investigated.
