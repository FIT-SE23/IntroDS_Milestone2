% As a general rule, do not put math, special symbols or citations
% in the abstract or keywords.
\begin{abstract}
Robots have been brought to work close to humans in many scenarios. For coexistence and collaboration, robots should be safe and pleasant for humans to interact with. To this end, the robots could be both physically soft with multimodal sensing/perception, so that the robots could have better awareness of the surrounding environment, as well as to respond properly to humans' action/intention. This paper introduces a novel soft robotic link, named \emph{ProTac}, that possesses multiple sensing modes: tactile and proximity sensing, based on computer vision and a functional material. These modalities come from a layered structure of a soft transparent silicon skin, a polymer dispersed liquid crystal (PDLC) film, and reflective markers. Here, the PDLC film can switch actively between the opaque and the transparent state, from which the tactile sensing and proximity sensing can be obtained by using cameras solely built inside the ProTac link. In this paper, inference algorithms for tactile proximity perception are introduced. Evaluation results of two sensing modalities demonstrated that, with a simple activation strategy, ProTac link could effectively perceive useful information from both approaching and in-contact obstacles. The proposed sensing device is expected to bring in ultimate solutions for design of robots with softness, whole-body and multimodal sensing, and safety control strategies.


\end{abstract}

% Note that keywords are not normally used for peerreview papers.
%\begin{IEEEkeywords}
%IEEE, IEEEtran, journal, \LaTeX, paper, template.
%\end{IEEEkeywords}

\IEEEpeerreviewmaketitle