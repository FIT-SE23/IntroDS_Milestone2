\section{Design and Fabrication} \label{sec: 2}

% \begin{figure}[ht]
%       \centering
%       \includegraphics[width=0.80\columnwidth]{contents/figs/Fig_design_concept.pdf}
%       \caption{The conceptual illustration of \emph{ProTac} sensor's skin transparency switching for two-mode sensing, based on the electrical activation of PDLC film. (\textbf{left}) \emph{opaque} state of the skin enables cameras not to interfere with external environments, resulting in effective vision-based tactile sensing inference. (\textbf{right}) \emph{transparent} state allows the internal cameras to see through the skin for proximity perception or short-range distance measurement.}
%       \label{fig:design_concept}
% \end{figure}
\subsection{Design Concept of Skin Transparency Switching}
We build ProTac upon the previous work on whole-arm tactile link (TacLink sensor) \cite{Lac21}, and extend its capability for additional proximity perception. To achieve such objective, we propose a novel combination of transparent silicone rubber and a thin flexible polymer dispersed liquid crystal (PDLC) film with attached markers. In which, the PDLC film allows switching among two modes: opaque (darken mode) and transparent mode by applying an external voltage. Here, the switching time is approximately $0.3\,$s. Therefore, the general working principle of the ProTac sensor link is as follows:
\begin{itemize}
    \item \textbf{Proximity mode}: When the PDLC film is in the transparent mode (see through), the two cameras inside the sensing link can see external objects that are close to the vicinity of the outer skin.
    \item \textbf{Tactile mode}: In contrast, the opaque (darken) mode of the PDLC film allows the two cameras to track the skin deformation through marker-featured images without external light interference, which enables the function of tactile or force sensing.
\end{itemize}

\begin{figure}[t]
      \centering
      \includegraphics[width=0.9\columnwidth]{contents/figs/Fig_link_design.pdf}
      \caption{Design of \emph{ProTac} sensor composed of three layers skin permitting to perform two sensing modes.}
      \label{fig:design_link}
\end{figure}

%\begin{figure*}[t]
%      \centering
%      \includegraphics[width=1.7\columnwidth]{contents/figs/Fig_fabrication_process.eps}
%      \caption{Fabrication process of a \emph{ProTac} sensor. Step 1 - Preparing parts (A - Part was fabricated by laser cutting. B - Part was fabricated by machining cutting. C - Part was fabricated by 3D printing technique). Step 2 - Reflective markers arrangement onto a PDLC film. Step 3 - Shaping the PDLC film. Step 4 - Molding assembly. Step 5 - Pouring deformable and transparent silicone. Step 6 - Releasing mold for a finished %\emph{ProTac} sensor.}
%      \label{fig:fabrication_process}
%\end{figure*}

Figure \ref{fig:design_link} depicts the structure of the proposed \emph{ProTac} sensor. Each end has a connector and camera housing that can accommodate a fisheye lens camera (ELP-USBFHD01M-L180). A series of LEDs were arranged circularly at the inner surface of the camera's housing to improve visibility in tactile mode. In proximity mode, they will be turned off.  Besides the use for fixing cameras, the camera housing and braces also help in shaping the cylindrical skin. On the other hand, the ProTac skin is made up of three layers: markers, PDLC film, and a deformable and transparent layer. 

\subsection{Fabrication of a ProTac Link}
The ProTac skin was basically fabricated through a molding process (see \cite{Lac21} for more details). The smoothness and evenness of the skin surface are one of the most important factors affecting the performance of both sensing modes. This can be achieved by using a set of molds with a smooth surface. Hence, a commercial acrylic tube was utilized for outside mold fabrication. Other mold parts were 3D printed using a 3D printer. Reflexive tape (R25 WHI, 3M Company) was used to form markers with a diameter of $3\,$mm. Then, the PDLC film was shaped by camera housings and the braces at two ends. Finally, a certain amount of silicon rubber was poured into the mold and the final product will be achieved after one day of curing at room temperature.
