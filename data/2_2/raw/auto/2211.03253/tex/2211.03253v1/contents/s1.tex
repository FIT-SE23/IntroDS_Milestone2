\section{Introduction} \label{sec: 1}

Nowadays, human-friendly robots are required to operate out of the safety fence zone, collaborating with humans in complex or physically demanding work, as in the so-called human-robot interaction (HRI) scenarios. Regarding the essential nature of these foreseen applications, the robots should be able to not only react to possible collisions, but also handle both unavoidable and intentional physical contacts in a safe and purposed manner \cite{Bicchi2008}. To potentially accomplish this, efforts should be made on the invention of novel robot components capable of passively reducing physical impacts based on compliant and lightweight structures. In addition, multimodal perception provided by embedded sensing systems permits reliable recognition of surroundings contact as well as non-contact events, which benefits the development of motion planning and control regimes possible to react purposefully in the HRI scenarios \cite{Cheng2019}. In this context, a soft sensory system endowed by both tactile and proximity sensations with a large coverage is essential.

\begin{figure}[t]
\centering 
\includegraphics[width=1.7\columnwidth]{contents/figs/Fig1.pdf}
\caption{Conceptual illustration for perceiving both tactile and proximity perception of \emph{ProTac} sensor's skin. This work is done by switching transparency of the PDLC film. \textbf{Tactile mode:} \emph{opaque} state of the skin enables cameras not to interfere with external environments, resulting in effective vision-based tactile sensing inference. \textbf{Proximity mode:} \emph{transparent} state allows the internal cameras to see through the skin for proximity perception or short-range distance measurement.}
\label{fig:paper_overview}
\end{figure}

Of sensing modalities, sense of touch offers a variety of tactile information from physical contacts, such as multiple applied forces, contact locations or complex touch patterns which are especially helpful in physical HRI (pHRI) \cite{ Li2020}. In contrast to the success of small-scaled tactile sensors, the development of tactile sensors with large coverage has faced tremendous challenges. Furthermore, most previous studies have concentrated on tactile sensing systems solely responding to physical touch and ignored touchless stimuli \cite{Navarro21}. Proximity perception has recently attracted attention for its capability of closing perception gaps induced by occlusions and blind spots in vision. One approach to combine these two sensing abilities in one sensor is attaching multiple feature-related sensing components onto a printed circuit boards encapsulated in soft material \cite{GordonCheng2011}. This system was successfully implemented in a variety of control frameworks and applications \cite{Leboutet2019, Cheng2019, Armleder2022} thanks to its conformability and scalability. However, since the sensing components are spatially distributed over the surface interspersed by electronic components, this design paradigm expressed only low/coarse spatial resolution. Not to mention the noise, since magnetic- \cite{7803315} or capacity-based \cite{5771603} tactile sensors often behaved differently with variation of material properties, which may cause difficulties in calibration and perception. 

The recent popularity of soft vision-based tactile sensors in the community is thanks to minimal wiring, high spatial resolution and low-cost \cite{s19183933}. Specifically, cameras are employed to bring in details of soft skin's deformation under physical tactile stimuli, by capturing visual features such as: 1) markers \cite{Lepora2018} and 2) reflective membrane \cite{s17122762}. Then, skin deformation will be interpreted into tactile information, including contact location, force, object texture, and so on, with help of analytical approaches \cite{Lac21} or data-based learning techniques \cite{Shotaro2021}. Sim2Real approaches \cite{Quan2022} have been recently deployed to enrich and cheapen training data sources with minimum compromise in sensing efficiency. Notably, to complement the sense of touch, a multimodal visuotactile sensor, integrated with RGB and ToF (time-of-flight) cameras, has been developed to deliver dual tactile information and proximity depth data by leveraging a selectively transmissive soft membrane \cite{Jessica2022}. Nevertheless, besides the fact that the scalability of this device to large sensing bodies is unclear, the price for the ToF camera, computation and energy cost to simultaneously process tactile and proximity images, are highly expensive.


% Previous study attempted to setup numerous sensors on the robot body, or introduced sophisticated sensor designs to satisfy the aforementioned requirements, however, such design faced complexity in integration and data processing, or merely impractical for whole-body sensing/interaction. Therefore, to both reduce complexity in fabrication and increase sensing capabilities in HRI, it necessitates a novel, yet simple design solution for a \emph{soft} sensing device that could deliver \emph{multimodal} tactile and proximity perception on a large-scale, whole-arm robot's body.

In this paper, we introduce a novel vision-based sensing robotic link with soft skin, named as \emph{ProTac}, that possesses a large sensing area, featuring both the sense of touch at any locations on the body (\emph{i.e.,}, tactile sensing), and the sense of distance (\emph{i.e.,}, proximity) of an object/obstacle to the vicinity of the skin. Such functions are achieved by actively switching the skin optical property between \emph{opaque} (not able to be seen through) for the \emph{tactile} sensing mode and \emph{transparency} (able to be seen through) for the \emph{proximity} sensing mode (see Fig. \ref{fig:paper_overview}). In addition, we propose strategies for learning perceptions of respective sensing modalities, which can trigger the robot behavior toward safety in HRI scenarios. In summary, the contributions of this paper are shown below:

\begin{enumerate}
    \item Design and fabrication of a soft vision-based dual-mode sensing link (\emph{ProTac}), possible for selective activation of either \emph{proximity} or \emph{tactile} sensing mode based on a mechanism of \emph{switching skin transparency}. Furthermore, the same internal RGB cameras are shared for both sensing modes.
    \item Proposal of methods to learn ProTac dual-mode perception, wherein the \emph{proximity} mode measures the closest distance to external obstacles in an observed region by using monocular depth-map construction, meanwhile based on a deep learning model the \emph{tactile} mode is possible to estimate local contact depth.

\end{enumerate}




% \begin{figure}[t]
%       \centering
% \includegraphics[width=\columnwidth]{contents/figures/Fig_research_positioning.pdf}
%       \caption{Illustration for the current state of the related field, and this research position.}
%       \label{fig:research_positioning}
% \end{figure}

