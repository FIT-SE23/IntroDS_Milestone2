% This must be in the first 5 lines to tell arXiv to use pdfLaTeX, which is strongly recommended.
\pdfoutput=1
% In particular, the hyperref package requires pdfLaTeX in order to break URLs across lines.

\documentclass[11pt]{article}

% Remove the "review" option to generate the final version.
\usepackage[]{emnlp2021}

% Standard package includes
\usepackage{times}
\usepackage{latexsym}

% For proper rendering and hyphenation of words containing Latin characters (including in bib files)
\usepackage[T1]{fontenc}
% For Vietnamese characters
% \usepackage[T5]{fontenc}
% See https://www.latex-project.org/help/documentation/encguide.pdf for other character sets

% This assumes your files are encoded as UTF8
\usepackage[utf8]{inputenc}

% This is not strictly necessary, and may be commented out,
% but it will improve the layout of the manuscript,
% and will typically save some space.
\usepackage{microtype}
\usepackage{hyperref}       % hyperlinks
\usepackage{url}            % simple URL typesetting
\usepackage{booktabs}       % professional-quality tables
\usepackage{amsfonts}       % blackboard math symbols
\usepackage{nicefrac}       % compact symbols for 1/2, etc.
\usepackage{algorithmicx}
\usepackage{algorithm}
\usepackage{algpseudocode}
\usepackage{amsmath} % for \boldsymbol macro
\usepackage{color}
\usepackage{graphicx}
\usepackage{subcaption}
\usepackage{multirow}
\usepackage{wrapfig}
\usepackage{latexsym}
\newcommand{\eg}{\textit{e.g.}}
\newcommand{\ie}{\textit{i.e.}}
\usepackage{times}
\usepackage{latexsym}
\usepackage{hyperref}
\usepackage{graphicx}
\usepackage{graphics}
\usepackage{multirow}
\usepackage{amsmath}
\usepackage{amsfonts}
\usepackage{amssymb}
\usepackage{tablefootnote}
\usepackage{color}
\usepackage{bbm}
\usepackage{graphicx}
\usepackage{xspace}
\usepackage{tikz}
\usepackage{pgfplots}
\usepackage[normalem]{ulem}
\usepackage{url}
\usepackage{subcaption}
\usepackage{fdsymbol}
% If the title and author information does not fit in the area allocated, uncomment the following
%
%\setlength\titlebox{<dim>}
%
% and set <dim> to something 5cm or larger.
\newcommand{\PET}{PET}
\newcommand{\La}{\mathcal{L}}
\newcommand{\ADAPET}{ADAPET}
\title{Prompt-based Text Entailment for Low-Resource Named Entity Recognition}

% Author information can be set in various styles:
% For several authors from the same institution:
% \author{Author 1 \and ... \and Author n \\
%         Address line \\ ... \\ Address line}
% if the names do not fit well on one line use
%         Author 1 \\ {\bf Author 2} \\ ... \\ {\bf Author n} \\
% For authors from different institutions:
% \author{Author 1 \\ Address line \\  ... \\ Address line
%         \And  ... \And
%         Author n \\ Address line \\ ... \\ Address line}
% To start a seperate ``row'' of authors use \AND, as in
% \author{Author 1 \\ Address line \\  ... \\ Address line
%         \AND
%         Author 2 \\ Address line \\ ... \\ Address line \And
%         Author 3 \\ Address line \\ ... \\ Address line}

% \author{First Author \\
%   Affiliation / Address line 1 \\
%   Affiliation / Address line 2 \\
%   Affiliation / Address line 3 \\
%   \texttt{email@domain} \\\And
%   Second Author \\
%   Affiliation / Address line 1 \\
%   Affiliation / Address line 2 \\
%   Affiliation / Address line 3 \\
%   \texttt{email@domain} \\}

% \author{

%     Dongfang Li \\
%     Harbin Institute of Technology (Shenzhen), Shenzhen, China \\\And
%     Baotian Hu \\\And
%     Qingcai Chen \\
%     % Hans Guesgen,
%     % Francisco Cruz,
%     % Marc Pujol-Gonzalez
%     % \\
% }
% \footnotemark[1]\thanks{\hspace{2mm}Co-corresponding authors}
\author{Dongfang Li$^1$, Baotian Hu$^1$\footnotemark[1]\thanks{\hspace{2mm}Corresponding authors}\hspace{2mm},  Qingcai Chen$^{1,2}$\footnotemark[1]\hspace{1mm}\\
$^1$Harbin Institute of Technology (Shenzhen), Shenzhen, China \\
$^2$Peng Cheng Laboratory, Shenzhen, China\\
\texttt{crazyofapple@gmail.com, \{hubaotian, qingcai.chen\}@hit.edu.cn}}

% \author{Dongfang Li$^1$, Qingcai Chen$^{1,2}$, Baotian Hu$^1$\\
% $^1$Harbin Institute of Technology (Shenzhen), Shenzhen, China \\
% $^2$Peng Cheng Laboratory, Shenzhen, China\\

% \texttt{crazyofapple@gmail.com, \{hubaotian, qingcai.chen\}@hit.edu.cn}}
% \affiliations{
%     %Afiliations

%     % \textsuperscript{\rm 1}Association for the Advancement of Artificial Intelligence\\
%     %If you have multiple authors and multiple affiliations
%     % use superscripts in text and roman font to identify them.
%     %For example,

%     \textsuperscript{\rm 1} Harbin Institute of Technology (Shenzhen), Shenzhen, China \\
%     \textsuperscript{\rm 2} Peng Cheng Laboratory, Shenzhen, China \\
%     crazyofapple@gmail.com, hubaotian@hit.edu.cn, qingcai.chen@hit.edu.cn
% }
\begin{document}
\maketitle
\begin{abstract}
%  To address learning challenges in these low-resource paradigms, researchers recently have interests in applying prompt-based learning into text classification and text generation. However, the application of prompt-based learning in the Named Entity Recognition (NER) task has not been fully explored. 
Pre-trained Language Models (PLMs) have been applied in NLP tasks and achieve promising results. Nevertheless, the fine-tuning procedure needs labeled data of the target domain, making it difficult to learn in low-resource and non-trivial labeled scenarios. To address these challenges, we propose Prompt-based Text Entailment (\texttt{PTE}) for low-resource named entity recognition, which better leverages knowledge in the PLMs. We first reformulate named entity recognition as the text entailment task. The original sentence with entity type-specific prompts is fed into PLMs to get entailment scores for each candidate. The entity type with the top score is then selected as final label. Then, we inject tagging labels into prompts and treat words as basic units instead of n-gram spans to reduce time complexity in generating candidates by n-grams enumeration.  Experimental results demonstrate that the proposed method \texttt{PTE} achieves competitive performance on the CoNLL03 dataset, and better than fine-tuned counterparts on the MIT Movie and Few-NERD dataset in low-resource settings.

\end{abstract}

%----------------------------------------------------------------------
%%% INTRODUCTION
%----------------------------------------------------------------------
% !TEX root = ../Main.tex


\Acp{BPM} have a long and rich history in optimization, going back at least to the introduction of \acl{MD} by Nemirovski \& Yudin \citep{NY83}.
In plain terms, \acp{BPM} are first-order (constrained) optimization algorithms that forego Euclidean projections in favor of a more sophisticated ``prox-mapping'' that minimizes a certain distance-like functional known as the Bregman divergence \citep{NY83,CT93,Bre67,Kiw97}.
When this Bregman divergence is the Euclidean distance squared, one recovers the standard projection-based methods;
other than that, depending on the problem's feasible region, different Bregman setups lead to a diverse collection of algorithms,
from exponentiated gradient descent on the simplex \citep{NY83,BecTeb03,ACBFS02},
to matrix multiplicative weights on the positive-semidefinite cone \cite{TRW05,KSST12},
variants of Karmarkar's affine scaling algorithm for linear programs \cite{VMF86},
etc.

One of the most appealing features of \acp{BPM} is that they achieve almost dimension-free convergence rates in problems with a convex structure and a favorable geometry \textendash\ such as the $L^{1}$ ball, the spectraplex, second-order cones, etc. \cite{Bub15,Nes09,BecTeb03}.
This is owed to a delicate interplay between the algorithms' non-Euclidean update scheme and the global geometry of the problem's domain.
However, these (almost) dimension-free guarantees also come with some strings attached:
they do not concern the sequence of iterates generated by the method, but only its time average
\revise{(or, through the same, ``regret-based'' analysis, the method's ``best iterate'')};
in this way, the best guarantee that can be achieved after $\run$ iterations is $\bigoh(1/\run)$.

In terms of oracle complexity, this is sufficient for problems that are not strongly convex\,/\,strongly monotone, but if one targets finer, geometric convergence rates,
\revise{the inherent averaging involved in regret-based guarantees is hard to compensate.}
And, on the other extreme, if the problem is not convex\,/\,monotone to begin with, iterate averaging does not provide any quantifiable benefits whatsoever, so it becomes crucial to study the actual trajectory of the method.


%----------------------------------------------------------------------
%%% CONTRIBS
%----------------------------------------------------------------------
\para{Our contributions}

Our paper seeks to quantify the last-iterate convergence rate of \aclp{BPM} as a function of the Bregman divergence defining the method and the local geometry that it induces.
To treat this question in as general a manner as possible, we focus on \ac{VI} problems of the form
\begin{equation}
\label{eq:VI}
\tag{VI}
\text{Find $\sol\in\points$ such that}
	\;\;
	\braket{\vecfield(\sol)}{\point - \sol}
	\geq 0
	\;\;
	\text{for all $\point\in\points$},
\end{equation}
where $\points$ is a closed convex subset of a finite-dimensional normed space $\pspace$, and $\vecfield \from \points \to \dspace$ is a (possibly non-monotone) single-valued operator on $\points$ with values in $\dspace$, the dual of $\pspace$.
This problem is a staple of many areas of mathematical programming, game theory and data science, as it provides a template for ``optimization beyond minimization'' \textendash\ \ie for problems where finding an optimal solution does not necessarily involve minimizing a loss function.
In particular, in addition to standard minimization problems \textendash\ which are recovered when $\vecfield = \nabla\obj$ for some smooth function $\obj$ \textendash\ the general formulation \eqref{eq:VI} includes saddle-point problems, games, complementarity problems, etc.;
for an introduction, see \cite{FP03} and references therein.

In this broad context, we examine the rate of convergence of a wide class of \aclp{BPM} to local solutions of \eqref{eq:VI} that satisfy a \acl{SOS} condition.
Specifically, the class of algorithms we consider includes as special cases
\begin{enumerate*}
[(\itshape i\hspace*{1pt}\upshape)]
\item
the original \acf{MD} algorithm of \cite{NY83};
\item
the \acf{MP} method of Nemirovski \cite{Nem04} \textendash\ which has the same update structure as the Bregman-based algorithm of \cite{AT05} and contains as a special case the \acf{EG} algorithm of \cite{Kor76};
\item
the so-called \acf{OMD} method of \cite{RS13-NIPS} \textendash\ itself a Bregman analogue of the modified Arrow-Hurwicz algorithm of \cite{Pop80};
\end{enumerate*}
etc.

Our first finding is a crisp characterization of last-iterate convergence rate of \acp{BPM} in terms of the local geometry induced by the underlying Bregman function near a given solution of \eqref{eq:VI}.
We make this dependence precise via the notion of the \emph{Legendre exponent}, a regularity measure for Bregman methods due to \cite{AIMM21}, which can roughly be described as the logarithmic ratio of the volume of a Euclidean ball to that of a Bregman ball of the same radius.
For example, Euclidean methods have a Legendre exponent of $\legexp = 0$ and they converge at a linear rate;
entropic methods have a Legendre exponent of $\legexp = 1/2$ at boundary points, and they converge at a rate of $\bigoh(\run^{-1})$;
more generally,
as we show in \cref{thm:general}, methods with a Legendre exponent $\legexp>0$ converge at a rate of $\bigoh(\run^{1-1/\legexp})$.
\PM{We need to fix this: the $1-1/\legexp$ exponent is not consistent with the $\bigoh(1/\run)$ expression.}
\WA{I don't see the issues, yes this expression is not well-defined for $\legexp = 0$ but this is normal, the two situations differ radically.}
The Euclidean regime ($\legexp = 0$) is perfectly aligned with existing results for the geometric last-iterate convergence rate of the \ac{EG} algorithm and its variants \citep{GBVV+19,Mal15,HIMM19,MOP20}.
By contrast, the Legendre regime ($\legexp > 0$) indicates a significant drop in the algorithm's last-iterate convergence speed, even though ergodic convergence rates \cite{Nes04} and results for bilinear games \cite{WLZL21} might suggest otherwise.

Subsequently, motivated by applications to game theory and linear programming, we take a closer look at the convergence rate of \acp{BPM} across the constraints that are active at a solution $\sol$ of \eqref{eq:VI} depending on the position of $\vecfield(\sol)$ relative to said constraints. 
This analysis reveals that Bregman proximal methods have a particularly fine structure:
along \emph{sharp directions} (\ie constraints along which $\vecfield(\sol)$ is strictly inward-pointing), \acp{BPM} converge
\begin{enumerate*}
[(\itshape i\hspace*{1pt}\upshape)]
\item
at a rate of $\bigoh(1/\run^{1/(2\legexp-1)})$ if $1/2 < \legexp < 1$;
\item
at a \emph{geometric rate} if $0 < \legexp \leq 1/2$ (\eg for entropic methods);
and
\item
in a \emph{finite} number of iterations if $\legexp=0$
\end{enumerate*}
(\cf \cref{thm:sharp}).
Thus, even though the estimates of \cref{thm:general} are, in general tight, the actual convergence rate of a Bregman method along different coordinates\,/\,constraints could be starkly different \textendash\ and, in fact, dramatically faster if the solution under study is itself sharp.

The closest antecedent of our work is the conference paper \cite{AIMM21} where the Legendre exponent was introduced to analyze the convergence of \ac{OMD} in \emph{stochastic} \ac{VI} problems (without considering sharp directions and/or faster identification rates).
The stochastic and deterministic settings are obviously very different, both in the challenges involved as well as the rates obtained, so there is no overlap in our analysis and results.
Other than that, we are not aware of any comparable results in the literature concerning the radically different convergence landscape of \acp{BPM} along active and inactive constraints.
\section{Method}

\subsection{Low-Resource Named Entity Recognition}
\label{sec:fewshot}
Given a sentence $\mathbf{X} = (x_{1}, x_{2}, \dots x_{N})$ which contains $N$ words, the task is to produce $\mathbf{Y} = (y_{1}, y_{2}, \dots y_{N})$ which is the sequence of entity tags. The tag $y_{i}\in\mathcal{Y}$ (e.g., B-LOC, I-PER, O) denotes the type of entity for each word $x_{i}$, where $\mathcal{Y}$ is a pre-defined set of tags.
We are given a low-resource NER dataset $\mathcal{D}_{\text{train}}$, where the labeled examples to each NER type (e.g., $<50$) are substantially less than that in the rich-resource NER dataset. Our goal is to train an accurate NER model under this low-resource setting.

Previous methods usually treat NER as a sequence labeling task in which a neural encoder such as LSTM and BERT is used for representing the input sequence, and a softmax or a CRF layer is equipped as the output layer to get the tag sequence.
Formally, as the standard fine-tuning, NER model $\mathcal{M}$ parameterized by $\theta$ is trained to minimize the cross-entropy loss over token representations $\mathbf{H} = [h_{1}, h_{2}, \dots h_{N}]$ that are generated from the neural encoder as follows:
\begin{equation}
\mathcal{L}=-\sum_{i=1}^{N} \log f_{x_{i}, y_{i}}(\mathbf{h} ; \theta),
\end{equation}
where $f$ is the model's predicted conditional probability for golden label.




\subsection{Prompt-based Text Entailment}
%\paragraph{Challenge.}
Towards the low-resource NER task, a common way is to pre-train the neural encoder and output layer parameters with the rich-resource NER dataset. Another feasible way is to focus on the matching function learned by prototype-based network~\cite{proto1} or nearest neighbor classification~\cite{proto2}. After that, a well-trained matching function can work well in the target tasks.
However, since the entity category is different, the parameter for the low-resource domain cannot be transferred directly from the source domain. Moreover, the metric-based meta-learning methods assume that training and test tasks are in the same distribution but this assumption may not always be satisfied in practice~\cite{yin2020meta}. 

In this work, we reformulate named entity recognition as the text entailment task. As the NER task is not a standard entailment problem, we convert NER examples into labeled entailment instances. The input includes the original sentence as premise and entity type-specific prompt as a hypothesis (i.e., template). The output is produced by an entailment classifier, predicting a label for each instance.
The entailment score is the probability of a specific token at the mask position of the prompt. Then, the entity type with the top entailment score is selected as the final label. For example, given a sentence ``\textit{Seoul is the capital of South Korea.}'' and a candidate ``\textit{Seoul}'', we define ``\textit{Seoul is an <entity\_type> entity. [MASK]}'' as prompt for each entity type. Suppose the entailment score of token ``\textit{yes}'' at [MASK] for <location> type is the highest 
of all entity types, we finally choose ``\textit{location}'' as the predicted label. 
For training, we sample three types of negative examples (see Appendix \ref{sec:templ}): false positive (i.e., replace the correct label with others), null label (i.e., replace the correct label with null), and non-entity replacement (i.e., replace golden entity with non-entity span). For example,  ``\textit{Seoul is not a named entity. [MASK]}'' is one prompt of ``false positive'' example (i.e., the [MASK] label is \textit{no}, and it exists entities). During training and inference, we can enumerate all possible text spans in the input sentence as named entity candidates~\cite{templatener}. 
To further reduce time complexity in generating candidates by n-grams enumeration, we inject tagging labels (e.g., I-location means the tag is inside a entity) into prompts and treat words as basic units instead of text spans during training and inference. In other words, we consider prompts ``\textit{<candidate\_entity\_word> is the part of a <entity\_type> entity. [MASK]}'' (e.g., ``\textit{Seoul is the part of a location entity. [MASK]''}). As PTE treats words as basic units for decoding, it optimizes time complexity at inference to O(L), which is in line with previous NER methods. It optimizes quadratic costs at inference to linear. We also apply the Viterbi algorithm at inference, where transitions are computed on the training set \cite{DBLP:conf/acl/HouCLZLLL20}. The computational complexity of n-grams enumeration is O($L^2$), increasing quadratically with sequence length L.  Overall, our method provides a unified entailment framework as the model shares the same inference pattern across different domains.

\subsection{Pattern Exploiting Training Framework} \label{sec:pet}
The basic framework of \texttt{PTE} is from {\ADAPET} \cite{ADAPET} which is a variant of  {\PET} \cite{schick2020exploiting, schick2020s}. Compared with {\PET}, {\ADAPET} uses more supervision by decoupling the losses for the label tokens and a label-conditioned MLM objective over the total original input~\cite{ADAPET}.  We introduce it by describing how to convert one example into a cloze-style question. The query-form in {\ADAPET} is defined by a Pattern-Verbalizer Pair (PVP). Each PVP consists of one pattern which describes how to convert the inputs into a cloze-style question with masked out tokens, and one verbalizer which describes the way to convert the classes into the output space of tokens. The PVP can be manually generated \cite{auxiliary-absa,lama} or obtained by using an automatic search algorithm \cite{schick-etal-2020-automatically, gao2020making}. 
% In this paper, we create a list of PVP for each task. 
After that, {\ADAPET} obtains logits from the model $G_m(x)$. Given the space of output tokens $\mathcal{Y}$, {\ADAPET} computes a softmax over $y \in \mathcal{Y}$, using the logits from $G_m(x)$. The final loss is shown as follows: 

{\small
\begin{align}
    q(y|x) &= \frac{\exp([\![G_m(x)]\!]_{y})}{\sum\limits_{y' \in \mathcal{Y}} \exp([\![G_m(x)]\!]_{y'})}, \\
    \La &= \texttt{Cross\_entropy} (q(y^*|x), y^*). \ 
    \label{eq:pet}
\end{align}
}%


 

\subsection{Cross Task and Domain Transfer}
\label{sec:template}
To address the challenge when few labeled examples are available, we further train the sentence encoder on the TE datasets (e.g., MNLI) and apply it to the NER task. Then, our method can perform more knowledge transfer between the rich-resource NER dataset and the low-resource NER dataset. Since there is no domain-related fully connected layer for fine-tuning, all parameters can be transferred in different domains even if the entity category does not match. Specially, we apply the text entailment method to the low-resource domain after firstly pre-training the NER model in the rich-resource domain. This process is simple but can effectively transfer label knowledge. As the output of our method is model-agnostic words (not tag index), the tag vocabulary with rich-resource and low-resource is a shared pre-trained language model vocabulary set. It allows our method to use the correlation of tags to enhance the effect of cross-domain transfer learning.




\section{Experiments}
We compare our methods with several baselines on both rich-resource settings and low-resource settings. We use the CoNLL2003 \cite{conll03} as the rich-resource dataset, and MIT Movie \cite{mit-dataset}, Few-NERD \cite{fewnerd} as the cross-domain low-resource datasets. And we conduct experiments on the CoNLL03 dataset in both full and low-resource settings. The dataset statistics and experimental settings are included in Appendix~\ref{sec:dataset} and~\ref{sec:exp_settings}. The standard precision, recall, and F1 score are used for model evaluation. 



\subsection{Rich-Resource NER Results}

We first use the whole training set of the CoNLL03 to train the model and evaluate its performance on the test set. 
Table~\ref{table:mp_conll} shows the performance of the comparison method and our model on the test set. We can find that although the potential applications of \texttt{PTE} is low-resource named entity recognition, it can also achieve competitive performance in rich-resource domain data sets. Compared with BERT fine-tuning reported in the previous work, the \texttt{PTE} model using discrete manual design reduces the F1 by 0.32, while the \texttt{PTE} model using the soft prompt method design mode~\cite{liu2021ptuning,ptuning} increases the F1 by 0.5. It shows that our method effectively recognizes named entities, and soft prompts can improve performance compared with manually designed prompts. More experimental results about TE patterns (\S\ref{sec:pet}) are in the Appendix~\ref{sec:pattern}.


\subsection{Cross Entity Type NER Results} 
\label{cross_type}

Following~\citet{templatener}, we sample the number of examples corresponding to different types of entities on the CoNLL03 data training set as new training set while keep test set unchanged. Among them, ``PER'' and ``ORG'' are rich-resource entity types, and ``LOC'' and ``MISC'' are low-resource entity types. The experimental results are shown in Table~\ref{table:conll_few}. The results show that our method achieves better results than baselines on the low-resource entity types, thus improving overall performance. On the other hand, our method is better than fine-tuning in both cases.

\begin{table}[t!]
\centering
\small
\resizebox{\linewidth}{!}{
\begin{tabular}{l|c|c|c}
     \hline
    {\bf Method} & {\bf Precision} & {\bf Recall} & {\bf F1} \\
    \hline
     % \multicolumn{4}{c}{Traditional Models} \\
    % \hline
    \citet{label-agnostic} & - & - & 89.94 \\
    \citet{yang-etal-2018-design} & - & - & 90.77 \\
    \citet{lstm-cnn-crf} & - & - & 91.21 \\
     BERT~\cite{templatener} & 91.93 & 91.54 & 91.73 \\
    \citet{DBLP:conf/emnlp/YamadaASTM20} & - & - & \bf{94.30} 
     \\
     \hline
     Template BART~\cite{templatener} & 90.51 & 93.34 & 91.90 \\
     
     \texttt{PTE} (discrete) & 91.27 & 91.56 & 91.41 \\
     \texttt{PTE} (soft) & 92.01 & 92.45 & \bf{92.23} \\
     \hline
\end{tabular}
}
\caption{Model performance on the CoNLL03 test set.\label{table:mp_conll}}
\end{table}

\begin{table}[t]
\centering
\small
\resizebox{\linewidth}{!}{

\begin{tabular}{l|c|c|c|c|c}
     \hline
   \textbf{Method} & {\bf PER} & {\bf ORG} & {\bf LOC} & {\bf MISC} & {\bf Overall} \\
    \hline
    BERT & 75.71 & 77.59 & 60.72 & 60.39 & 69.62 \\
    Template BART & 84.49 & 72.61 & 71.98 & 73.37 & 75.59 \\
    \texttt{PTE} (BERT) & 85.34 & 72.89 & 73.01 & 74.32 & \bf{76.40} \\
     \hline
\end{tabular}
}
\caption{Cross entity type results on the CoNLL03. LOC and MISC are low-resource entity types, where PER and ORG are rich-resource entity types. \label{table:conll_few}}
% \vspace{-2mm}
\end{table}
\subsection{Domain Transfer for Low-Resource NER}




\begin{table}[t!]
\centering
\small

\vspace{-0.2cm}
\resizebox{\linewidth}{!}{
\begin{tabular}{l|c|c|c|c|c|c}\hline
\multicolumn{7}{c}{\textit{MIT Movie} (12)}\\
\hline

     \textbf{Method} & {\bf K=10} & {\bf K=20} & {\bf K=50} & {\bf K=100} & {\bf K=200} & {\bf K=500} \\
\hline
    \citet{label-agnostic} &\ 3.1 &\ 4.5 &\ 4.1 &\ 5.3 &\ 5.4 &\ 8.6 \\
    \citet{example-ner} & 40.1 & 39.5 & 40.2 & 40.0 & 40.0 & 39.5 \\
    Sequence Labeling BERT & 28.3 & 45.2 & 50.0 & 52.4 & 60.7 & 76.8 \\
    \citet{DBLP:conf/emnlp/YamadaASTM20} & 35.6 & 49.2 & 61.8 & 72.4 & 78.7 & 82.8 \\
    % & 
    Template BART~\cite{templatener} & 42.4 & 54.2 & 59.6 & 65.3 & 69.6 & 80.3 \\ \hline
     \texttt{PTE} (discrete) & 46.9$\dagger$ & 59.2$\dagger$ & 66.9$\dagger$ & 74.9$\dagger$ & 79.9$\dagger$ & 83.6\\
     \texttt{PTE} (soft) & \textbf{47.8}$\dagger$ & \textbf{60.8}$\dagger$ & \textbf{68.1}$\dagger$ & \textbf{76.5}$\dagger$ & \textbf{83.6}$\dagger$ & \textbf{86.4}$\dagger$ \\
    \hline \hline
    \multicolumn{7}{c}{\textit{Few-NERD} (8)}\\\hline 

     \textbf{Method} & {\bf K=10} & {\bf K=20} & {\bf K=50} & {\bf K=100} & {\bf K=200} & {\bf K=500} \\
\hline
%     Source & Methods & 10 & 20 & 50 & 100 & 200 & 500 \\
     \citet{label-agnostic} &\ 5.2 &\ 4.1 &\ 4.7 &\ 7.8 &\ 12.3 &\ 10.1 \\
     \citet{example-ner} & 35.4 & 48.3 & 51.2 & 51.8 & 53.6 & 55.7 \\
     Sequence Labeling BERT & 50.6 & 59.3 & 61.3 & 61.4 & 62.5 & 66.4 \\ 
     \citet{DBLP:conf/emnlp/YamadaASTM20} & 51.7 & 60.1 & 62.3 & 61.0 & 62.5 & 66.8 \\ 
     \hline
     \texttt{PTE} (discrete) & 51.8 & 59.7 & 60.5 & 61.3 & 61.8 & 63.4\\
     \texttt{PTE} (soft) & \textbf{54.2} & \textbf{61.4} & \textbf{62.3} & \textbf{62.5} & \textbf{63.6} & \textbf{67.4} \\
 

\hline
\end{tabular}
}
\caption{\label{tab:fewshot}F1 comparison of two low-resource NER datasets. We set 6 sample size $K$ for different low-resource settings. $\dagger$ means a significant difference compared to Template BART ($p < .05$).}
\vspace{-0.2cm}
% \vspace{-2mm}
\end{table}


We do not use $N$-way $K$-shot setting~\cite{proto2,fewnerd} which samples $N$ categories and $K$ examples for training in each episode because a sentence in the NER task may contain multiple entities from different types. Thus, we randomly sample training data from the MIT Movie and Few-NERD datasets to simulate low-resource scenarios and use CoNLL03 as the rich-resource dataset. As such, we have only $K$ examples for each type of training. We choose $K\in\{10,20,50,100,200,500\}$ for experiments to evaluate the ability of the model on training data of different sizes. The experimental results are in Table~\ref{tab:fewshot}. The results show that when the $K$ value is relatively small, our \texttt{PTE} method can be better than the fine-tuning method, and this trend decreases with the increase of $K$. In addition, the soft mode is also better than the discrete mode in the case of a small number of samples. Overall, our method achieves the best results on both data sets in the low-resource scenario.


\begin{figure}[t!]
\centering
% \small
\includegraphics[width=1.0\linewidth]{ablation.pdf}
  \caption{F1 scores with different experimental settings and model variants.}
  \vspace{-4mm}
   \label{figablition}
\end{figure}
\subsection{Ablation Study}

We conduct ablation experiments and the results are shown in Figure~\ref{figablition}. The results show that (1) the selection of negative examples has a great impact on the performance of the model, especially the negative examples of the null label type. However, in rich-resource scenario, the gap between full setting and decreased setting is not as much as the low-resource scenario; (2) the low-resource scenario is a challenge to the model, and the results of some variants are not inconsistent where prompt-based learning may not be as good as fine-tuning; (3) label conditioning and soft mode have a consistent effect on the model. These findings highlight that it still has room left to use prompt for effectively transferring knowledge in the case of low-resource scenario.
% !TeX spellcheck = en_GB
%!TEX root = ../side-constrained.tex

\section{Conclusion}

We provided a counterexample to a claimed existence result for dynamic equilibria with side constraints. The implications of this counterexample were shown to be severe since solutions to the canonical infinite dimensional variational inequality are in some sense useless and other approaches seem to be necessary. 
We then established a general framework for defining side-constrained dynamic equilibria based on two key objects: A \setS{} $S$ containing all feasible flows (given as walk inflows) and correspondences $A_p$ providing the flow-dependent set of \addmEpsDev s. We showed that this equilibrium concept not only encompasses the known unconstrained equilibria with and without departure time choice and capacitated dynamic equilibria with convex \setS{}s but also allows for a whole range of new dynamic equilibria inspired by static side-constrained equilibria.
We provided conditions under which they can be characterized as solutions to a quasi-variational or even a variational inequality. The latter characterization then also gave rise to a first existence result for certain side-constrained dynamic equilibria with convex \setS.
Finally, we turned to equilibria wherein the side-constraints are given by time-varying edge-load constraints. To deal with the non-convexity of the \setS{}, we employed an augmented Lagrangian approach by relaxing the hard edge-load-capacities and replacing them by penalty functions. We demonstrated that these existence results apply, in particular, for the widely used Vickrey point queue model as well as the linear edge delay model.

Several important questions remain open. First of all, it would be interesting to find an existence result for BSDE similar to \Cref{thm:ExistenceFDAddSpaceExCP} for LPDE and MNSDE. The main obstacle to obtaining such a result seems to be the fact that for BSDE, the definition of \addmEpsDev s involves the network loading which, in general, is a very complex mapping and, even for well-studied flow models, is not fully understood yet. Note that, due to \Cref{prop:RelationshipsOfCDE}, such a result would also directly imply existence of \globalEL{} as well as providing an alternative proof for the existence of LPDE. Another aspect is the multiplicity of equilibria and
the issue of selecting a particular type of equilibrium having desirable properties.
It is an interesting research direction to characterize equilibrium concepts
that admit equilibrium selection via appropriate optimization or optimal control reformulations
whose optimal solutions provide such desirable properties.

\section*{Acknowledgements}
We thank the anonymous reviewers for their insightful comments and suggestions. This work is jointly supported by grants: Natural Science Foundation of China (No. 62006061,61872113,U1813215), Stable Support Program for Higher Education Institutions of Shenzhen (No. GXWD20201230155427003-20200824155011001) and Strategic Emerging Industry Development Special Funds of Shenzhen(No. JCYJ20200109113441941 and No. XMHT20190108009).
% Entries for the entire Anthology, followed by custom entries
\bibliography{anthology,custom}
\bibliographystyle{acl_natbib}
\clearpage

% \appendix
We briefly explain the algebraic background relevant for the definition of the the main character of this paper: the element $A \in \HF(\tau^{-1})$.
We follow the conventions for $A_{\infty}$-machinery from \cite{seidelbook}.

Suppose $\mathcal{A}$ is a homologically unital $A_\infty$-category. The Yoneda embedding is a functor
\[
\mathcal{Y} \colon \mathcal{A} \rightarrow mod_{\mathcal{A}}
\]
taking an object $L$ to the $\mathcal{A}$-module 
$\mathcal{Y}(L)$
defined by
\[
\mathcal{Y}(L)(K) := Mor_{\mathcal{A}}(K,L).
\]
and 
\[
\mu^d_{\mathcal{Y}(L)}(b,a_{d-1}, \dots, a_1) := \mu^d(b, a_{d-1}, \dots , a_1)
\]
for $a_i \in Mor_{\mathcal{A}}(K_{i-1},K_i)$, $i\in \{1, \dots , d-1\}$ 
and $b \in \mathcal{Y}(L)(K_{d-1}) = Mor_{\mathcal{A}}(K_{d-1},L)$.

By \cite[Section 2g]{seidelbook} the Yoneda embedding induces a unital, full and faithfull embedding
\[
\Homol(\mathcal{Y}) \colon \Homol(\mathcal{A}) \to \Homol(mod_{\mathcal{A}}).
\]
The derived cateogory $\mathcal{DA}$ of $\mathcal{A}$
can be constructed as follows: Take a triangulated completion of the image of $\mathcal{Y}$ in $mod_{\mathcal{A}}$ and take its homology category.

The following is an immediate consequence of the properties of the Yoneda embedding. 

\begin{cor}
 Each $f\in Mor_{D\mathcal{A}}(\mathcal{Y}(L_1), \mathcal{Y}(L_2))$ can be represented by 
 $\mathcal{Y}(\alpha)$
 for some $\alpha \in Mor_{\mathcal{A}}(L_1,L_2)$. 
 Moreover, $[\alpha]\in Mor_{H(\mathcal{A})}(L_1,L_2)$ is uniquely defined. 
 \end{cor}
 \begin{proof}
First, note that
\[
Mor_{D\mathcal{A}}(\mathcal{Y}(L_1), \mathcal{Y}(L_2))
\cong \Homol(Mor_{mod_{\mathcal{A}}}(\mathcal{Y}(L_1), \mathcal{Y}(L_2))).
\]
 For any object $K$, $\mathcal{Y}(\alpha)$ determines the map
 \[
 \mathcal{Y}(L_1)(K) \cong Mor(K,L_1) \xrightarrow{\mu^2(\alpha,-)}  
 Mor(K,L_2) \cong \mathcal{Y}(L_2)
 \]
 The existence and uniqueness of $\alpha$ follow immediately from $\Homol(\mathcal{Y})$ being full and faithful.
\end{proof}

\noindent
These notions are applied in this paper to the $A_{\infty}$-category $\mathcal{F}uk(M)$.
 



%%%%%%%%%%%%%%%%%%%%%%%%%%%%%%%%%%%%%%%%%%%%%%%%%%%%%%%%%%%%%%%%%%%%%%%%%%%%
%%%%%%%%%%%Homological Version%%%%%%%%%%%%%%%%%%%%%%%%%%%%%%%%%%%%%%%%%%%%%%
%%%%%%%%%%%%%%%%%%%%%%%%%%%%%%%%%%%%%%%%%%%%%%%%%%%%%%%%%%%%%%%%%%%%%%%%%%%%
\begin{comment}
\subsection{$A_\infty$-categories}
We work in a homological setting, in contrast to Seidel's book.
Moreover, we work in an ungraded setting, maybe later updated to a $\Z / 2\Z$-grading.

We briefly recall here the main definitions and fix notation.

Suppose $\mathcal{A}$ is a homologically unital $A_\infty$-category. The Yoneda embedding is a functor
\[
\mathcal{Y} \colon \mathcal{A} \rightarrow mod_{\mathcal{A}}
\]
taking an object $L$ to the $\mathcal{A}$-module 
$\mathcal{Y}(L)$
defined by
\[
\mathcal{Y}(L)(K) := Mor_{\mathcal{A}}(K,L).
\]
and 
\[
\mu^{\mathcal{Y}(L)}(a_1, \dots, a_{n-1},b) := \mu_n(a_1, \dots , a_{n-1},b)
\]
for $a_i \in Mor_{\mathcal{A}}(K_i,K_{i+1})$, $i\in \{1, \dots , n-1\}$ 
and $b \in \mathcal{Y}(L)(K_n) = Mor_{\mathcal{A}}(K_n,L)$.

By Seidel, section 2g, the Yoneda embedding induces a unital, full and faithfull embedding
\[
H(\mathcal{Y}) \colon H(\mathcal{A}) \to H(mod_{\mathcal{A}}).
\]

The derived cateogory $\mathcal{DA}$ of $\mathcal{A}$
can be constructed as follows: Take a triangulated completion of $mod_{\mathcal{A}}$ and take its homology category.

The following is an immediate consequence of the properties of the Yoneda embedding. We include it here, since it is relevant for this article.

\begin{cor}
 Each $f\in Mor_{D\mathcal{A}}(\mathcal{Y}(L_2), \mathcal{Y}(L_1))$ can be represented by 
 $\mathcal{Y}(\alpha)$
 for some $\alpha \in Mor_{\mathcal{A}}(L_2,L_1)$. 
 Moreover, $[\alpha]\in Mor_{H(\mathcal{A})}(L_2,L_1)$ is uniquely defined. 
 For any object $K$, $\mathcal{Y}(\alpha)$ determines the map
 \[
 \mathcal{Y}(L_2)(K) \cong Mor(K,L_2) \xrightarrow{\mu_2(-,\alpha)}  
 Mor(K,L_1) \cong \mathcal{Y}(L_1)
 \]
 
\end{cor}
\begin{proof}
First, note that
\[
Mor_{D\mathcal{A}}(\mathcal{Y}(L_2), \mathcal{Y}(L_1))
\cong H(Mor_{mod_{\mathcal{A}}}(\mathcal{Y}(L_2), \mathcal{Y}(L_1))).
\]


First, note that
\[
Mor_{D\mathcal{A}}(\mathcal{Y}(L_2), \mathcal{Y}(L_1))
\cong H(Mor_{mod_{\mathcal{A}}}(\mathcal{Y}(L_2), \mathcal{Y}(L_1))) \cong Mor_{H({\mathcal{A}})}(\mathcal{Y}_H(L_2), \mathcal{Y}_H(L_1)),
\]
where $\mathcal{Y}_H(L) = H(Mor_\mathcal{A}(K,L))$.
So $f$ consists of a collection of maps $f(K) \colon Mor_{H(\mathcal{A})}(K,L_2) \to Mor_{H(\mathcal{A})}(K,L_1)$
for every object $K$.

The existence and uniqueness of $\alpha$ follow immediately from $H(\mathcal{Y})$ being full and faithful.
\end{proof}

\end{comment}


\end{document}
