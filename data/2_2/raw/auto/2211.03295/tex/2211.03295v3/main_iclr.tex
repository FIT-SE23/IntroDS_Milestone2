
\documentclass{article} % For LaTeX2e
\usepackage{iclr2024_conference,times}

% Optional math commands from https://github.com/goodfeli/dlbook_notation.
\input{math_commands.tex}

% \usepackage{hyperref}  % iclr original
\usepackage{url}
\usepackage[pagebackref=true,breaklinks=true,letterpaper=true,colorlinks,citecolor=darkerblue,bookmarks=false]{hyperref}
\def\UrlBreaks{\do\A\do\B\do\C\do\D\do\E\do\F\do\G\do\H\do\I\do\J\do\K\do\L\do\M\do\N\do\O\do\P\do\Q\do\R\do\S\do\T\do\U\do\V\do\W\do\X\do\Y\do\Z\do\[\do\\\do\]\do\^\do\_\do\`\do\a\do\b\do\c\do\d\do\e\do\f\do\g\do\h\do\i\do\j\do\k\do\l\do\m\do\n\do\o\do\p\do\q\do\r\do\s\do\t\do\u\do\v\do\w\do\x\do\y\do\z\do\.\do\@\do\\\do\/\do\!\do\_\do\|\do\;\do\>\do\]\do\)\do\,\do\?\do\'\do+\do\=\do\#}


%%%%%%%%%%%%%%%%% my packages %%%%%%%%%%%%%%%%
\def\submissionarXiv{1}  % arXiv long version
\def\submissionICLR{2}  % ICLR review version
% \def\submission{\submissionICLR}  % submit
\def\submission{\submissionarXiv}  % preprint or final
\usepackage{graphicx}
% \usepackage{subfigure}
\usepackage{subfig}
\usepackage{subfloat}
\usepackage{booktabs} % for professional tables
\usepackage{listings}
\usepackage{multirow}
\usepackage{float}
% For theorems and such
% \usepackage{amsmath}
% \usepackage{amssymb}
\usepackage{mathtools}
\usepackage{amsthm}

%%%%%%%%%%%%%%%%%%%% my new commands %%%%%%%%%%%%%%%%%%%%
\usepackage{xcolor}  % colors
\usepackage{wrapfig}
\usepackage{colortbl}
\usepackage[makeroom]{cancel}
\usepackage{cases}

\definecolor{gray}{rgb}{0.5,0.5,0.5}
\definecolor{darkergreen}{RGB}{21, 152, 56}
\definecolor{darkerblue}{rgb}{0,0.08,0.45}
\definecolor{darkerred}{RGB}{220, 35, 120}
\definecolor{RoyalBlue}{RGB}{65,105,225}
\definecolor{YellowOrange}{RGB}{255,165,0}
\definecolor{gray94}{gray}{.92}
\definecolor{gray90}{gray}{.90}
\definecolor{gray85}{gray}{.85}

\usepackage{color,soul} % add background colors for text
\sethlcolor{gray90} % set the background color

\newcommand{\red}[1]{\textcolor{red}{#1}}
\newcommand{\green}[1]{\textcolor{green}{#1}}
\newcommand{\blue}[1]{\textcolor{blue}{#1}}
\newcommand{\gray}[1]{\textcolor{gray}{#1}}
\newcommand{\fn}[1]{\footnotesize{#1}}
\if\submission\submissionICLR  % ICLR rebuttal version
    \newcommand{\pl}[1]{\textcolor{darkerred}{#1}}  % rebuttal
\else
    \newcommand{\pl}[1]{\textcolor{black}{#1}}  % final version
\fi
\usepackage{pifont}% 
\usepackage{xspace}
\usepackage{placeins}
\newcommand{\cmark}{\ding{51}\xspace}%
\newcommand{\cmarkg}{\textcolor{gray}{\ding{51}}\xspace}%
\newcommand{\xmark}{\ding{55}\xspace}%
\newcommand{\xmarkg}{\textcolor{gray}{\ding{55}}\xspace}%
%%%%%%%%%%%%%%%%%%%%%%%%%%%%%%%%%%%%%%%%%%%%%%%%%%%%%%%%%%


% \title{Formatting Instructions for ICLR 2024 \\ Conference Submissions}
% \title{Learning Multi-order Representation in Convolutional Neural Network}  % iclr
% \title{Efficient Multi-order Gated Aggregation \\ Network}  % cvpr & arxiv
\title{MogaNet: Multi-order Gated Aggregation Network}  % iclr


% Authors must not appear in the submitted version. They should be hidden
% as long as the \iclrfinalcopy macro remains commented out below.
% Non-anonymous submissions will be rejected without review.

% \author{Antiquus S.~Hippocampus, Natalia Cerebro \& Amelie P. Amygdale \thanks{ Use footnote for providing further information
% about author (webpage, alternative address)---\emph{not} for acknowledging
% funding agencies.  Funding acknowledgements go at the end of the paper.} \\
% Department of Computer Science\\
% Cranberry-Lemon University\\
% Pittsburgh, PA 15213, USA \\
% \texttt{\{hippo,brain,jen\}@cs.cranberry-lemon.edu} \\
% \And
% Ji Q. Ren \& Yevgeny LeNet \\
% Department of Computational Neuroscience \\
% University of the Witwatersrand \\
% Joburg, South Africa \\
% \texttt{\{robot,net\}@wits.ac.za} \\
% \AND
% Coauthor \\
% Affiliation \\
% Address \\
% \texttt{email}
% }
\author{
    \hspace{-0.5em}
    Siyuan Li$^{1,2}$\thanks{Equal contribution.\ \ \ $^\dag$Corrsponding author.}~~ %\textsuperscript{\rm 1}
    Zedong Wang$^{1*}$~~%\textsuperscript{\rm 1}
    Zicheng Liu$^{1,2}$~~%\textsuperscript{\rm 1}
    Cheng Tan$^{1,2}$~~%\textsuperscript{\rm 1}
    Haitao Lin$^{1,2}$~~%\textsuperscript{\rm 1}
    Di Wu$^{1,2}$~\\%\textsuperscript{\rm 1}
    \textbf{Zhiyuan Chen}$^{1,2}$~~%\textsuperscript{\rm 1}
    \textbf{Jiangbin Zheng}$^{1,2}$~~%\textsuperscript{\rm 1}
    \textbf{Stan Z. Li}$^{1\dag}$\\%\thanks{Corrsponding Author.}\\
    % Affiliations
    $^{1}$AI Lab, Research Center for Industries of the Future, Westlake University, Hangzhou, China\\
    $^{2}$Zhejiang University, College of Computer Science and Technology, Hangzhou, China\\
    % \{lisiyuan,~jinweiyang,~wangzedong,~wufang,~liuzicheng,~tancheng,~stan.zq.li\}@westlake.edu.cn
}


% The \author macro works with any number of authors. There are two commands
% used to separate the names and addresses of multiple authors: \And and \AND.
%
% Using \And between authors leaves it to \LaTeX{} to determine where to break
% the lines. Using \AND forces a linebreak at that point. So, if \LaTeX{}
% puts 3 of 4 authors names on the first line, and the last on the second
% line, try using \AND instead of \And before the third author name.

\newcommand{\fix}{\marginpar{FIX}}
\newcommand{\new}{\marginpar{NEW}}

% for arXiv or final
\iclrfinalcopy % Uncomment for camera-ready version, but NOT for submission.
\begin{document}


\maketitle

%%%%%%%%% ABSTRACT
\begin{abstract}
% Modern ConvNets
Since the recent success of Vision Transformers (ViTs), explorations toward ViT-style architectures have triggered the resurgence of ConvNets.
% Novel view: interaction
In this work, we explore the representation ability of modern ConvNets from a novel view of multi-order game-theoretic interaction, which reflects inter-variable interaction effects w.r.t.~contexts of different scales based on game theory.
% MogaNet
Within the modern ConvNet framework, we tailor the two feature mixers with conceptually simple yet effective depthwise convolutions to facilitate middle-order information across spatial and channel spaces respectively.
% Experiments illustration
In this light, a new family of pure ConvNet architecture, dubbed MogaNet, is proposed, which shows excellent scalability and attains competitive results among state-of-the-art models with more efficient use of parameters on ImageNet and multifarious typical vision benchmarks, including COCO object detection, ADE20K semantic segmentation, 2D\&3D human pose estimation, and video prediction.
% Highlight results
Typically, MogaNet hits 80.0\% and 87.8\% top-1 accuracy with 5.2M and 181M parameters on ImageNet, outperforming ParC-Net-S and ConvNeXt-L while saving 59\% FLOPs and 17M parameters.
% code (arxiv & final version)
The source code is available at \url{https://github.com/Westlake-AI/MogaNet}.
\vspace{-1.0em}


\end{abstract}


%%%%%%%%% BODY TEXT
\section{Introduction}
\label{sec:introduction}
Reliable, fast, and efficient data processing is crucial given the growing volumes of data in both industry and research.
These needs are often addressed by using distributed dataflow frameworks like Spark~\cite{Zaharia2010}, and Flink~\cite{Carbone2015}.
As these frameworks' handle parallelism, distribution, and fault tolerance, they make it easier for users to create scalable data-parallel programs.
The resulting applications can use a variety of compute clusters for data processing.

However, it is still difficult to choose and configure resources in a way that closely meets user-specific goals and constraints~\cite{RajanKCK16,cloudcomputingchallenges2018}.
Numerous strategies have been put forth to assist users, and they can be grouped into two categories:
Model-based techniques~\cite{MaoAMK16,RajanKCK16,ShahAKW19,AlSayehS19,KirchoffXMR19,ChenLLWZ21silhouette,ScheinertTZWAWK21,WillTSBK21,AlSayehMJPS22} often rely on access to historical performance data, however, historical workload execution data is not always available.
Search-based techniques~\cite{AlipourfardLCVY17,HsuNFM18,bilal2020finding,klimovic2018selecta,fekry2020accelerating,MendesCRG20,LiuXL20,SongZLSFDS21} conduct costly profiling runs prior to executing the actual workload utilizing all, or a fraction, of the input data to iteratively create performance models.

Often enough though, the optimized resource configuration is only relevant for the workload at hand. 
Information about the underlying infrastructure are solely obtained implicitly, i.e., by measuring the performance of the target workload in one execution context.
As a consequence, a thorough understanding of utilized resources and their capabilities is lacking and insights gained cannot be easily transferred to other contexts, for instance, when profiling new workloads with different resource demands. 
This requires repeated profiling overhead for reoccurring or comparable workloads that could be avoided, rendering current approaches less resource-efficient than they could be.

Addressing these limitations, we present \emph{Perona}, a novel approach to explicit and robust infrastructure fingerprinting. 
It motivates the usage of common sets and configurations of benchmarking tools to assess the full capabilities of target infrastructures and to make the obtained benchmarking metrics directly comparable.
This explicit fingerprinting operation transparently reveals the characteristics of resources and allows ranking them.
Perona discards irrelevant benchmarking metrics in a data-driven manner by learning a dense, low-dimensional representation of input metric vectors. 
With these, more sophisticated resource decisions can be made for big data analytics, e.g., with regard to scheduling or resource allocations.
To be able to assess a recent benchmark execution, our approach incorporates results of prior benchmark executions, which is particularly useful for detecting resource degradation. 

\emph{Contributions}. The contributions of this paper are:

\begin{itemize}
    \item A novel approach for incorporating infrastructure fingerprinting into model-based methods for optimized resource configuration of workloads through ranking of resources and detection of degrading resource behavior.
    \item A method for context-aware representation learning of benchmark metrics, thereby not only discarding insignificant features but also taking prior benchmark runs and corresponding machine metrics into account. 
    \item An openly available implementation\footnote{\url{https://github.com/dos-group/perona-infrastructure-fingerprinting}} of Perona which we evaluated with regard to common metrics and in interplay with resource configuration methods for distributed dataflows and scientific workflows. 
\end{itemize}

\emph{Outline}. \autoref{sec:related_work} discusses the related work.
\autoref{sec:approach} describes the three main aspects of our approach in detail. 
\autoref{sec:evaluation} presents the results of our evaluation.
\autoref{sec:conclusion} concludes the paper and gives an outlook on future work.
In the below discussions, we use the Partial Markov Decision Process (POMDP) formation  since this setting fits in most real-world reinforcement learning tasks where the full state is not obtainable and the agent is only given observations from its onboard sensors \citep{Igl2018DeepVR}. A POMDP is defined as: $\mathcal{M}=(\mathcal{X}, \mathcal{A}, \mathcal{T}, r, \gamma)$, where  $\mathcal{X}$ denotes the observation space, $\mathcal{A}$ is the action space, $\mathcal{T}$ is the environment transition dynamics, $r$ is the reward function and $\gamma$ denotes the discount factor.

\subsection{Reusable agent-environment interaction models}
% \begin{wrapfigure}{r}{0.2\textwidth}
%     \centering
%     \includegraphics[width=0.2\textwidth]{figs/609px-Plato's_allegory_of_the_cave.jpg}\\
%     {\footnotesize Plato's allegory of the cave~\citep{Plato}. \normalsize}
%     %\label{fig:decentcem-a-architecture}
%     \vspace{-3mm}
% \end{wrapfigure}

\paragraph{(1) Embedded agency: small agent, big world}
% In 1969, Herbert Simon brought us the ``Simon's ant'' parable in his book ``{The Science of the Artificial}''\citep{simon1969science} (chapter 3) explaining that the complex agent behaviour we observed may not attribute to how complex the agent is, instead it may result from the interactions between the agent and the complex world, even the agent itself like an ant, is quite simple. It reminds us of a common setting in intelligent organisms: big world, small agent.

Unlike the \textit{Bayesian world model} approach \citep{ha2018world} that aims to construct a big model for the agent as a base knowledge about the world, the \textit{Embedded agency}\citep{orseau2012space} approach takes the perspective on how the small agent perceives, controls and gradually grows based on its own experience. Unlike a \textit{Bayesian world model} where states and models are objective, an \textit{embedded agency} approach is subjective\citep{agent} that agent's past experience will shape its behaviors in new tasks. This idea of intelligence closely relates to Descarte's theory of representationalism and mind-body dualism, and to the modern theory of embodied embedded cognition\citep{haugeland} in the philosophy of mind research. The \textit{embedded agency} approach is appealing in that it promises a scalable learning architecture to build a world model based on the agent's experience in a self-improvement manner, however, it is also ambitious. 
This paper investigates the core idea of \textit{embedded agency} about how an agent's past experience can shape its new task learning in a pre-training and reusing paradigm.  

\paragraph{(2) On agent-environment boundary}
Conceptually, from the \textit{embedded agency} perspective, a generally reusable model in heterogeneous environments and tasks should be the agent itself, since the agent is the only shared component across domains. Practically, using the agent's past experience to extract the agent model is difficult since the agent-environment boundary is not the same as we think in a physical system, like a robot. \citep{Jiang2019OnVF} provide a detailed analysis of this problem. \citep{sutton2018reinforcement} in chapter 3.1 explains that the agent-environment boundary is often task-specific in different abstraction levels and the boundary could change across tasks and environments, which makes exacting a general agent model from its various task data difficult. 

Instead of extracting an agent model, we propose to extract a general agent-environment interaction model that is commonly shared across tasks and environments, and thus can be viewed as an embodiment of the agent. This proposal is explained below.

\paragraph{(3) Generally reusable agent-environment interaction model}
In pre-training, suppose the agent's past experiences $\mathcal{D}$ are generated from M different environments and collected using unlimited behaviour policies induced by unlimited reward functions. This will compose a set of different POMDPs: $\{\mathcal{M}_{i}\}_{i=1}^{M}$, where $\mathcal{M}_{i}=(\mathcal{X}_{i}, \mathcal{A}, \mathcal{T}_{i}, \gamma)$ is a reward-free POMDP that drops the task reward for simplicity. $\mathcal{A}$ without a subscript $i$ means the agent's action space is unchanged across environments since the embedded agency considers the same agent across environments and tasks. Let's assign a one-hot vector encoding $Y_{i}$ as the label for each environment domain $\mathcal{M}_{i}$, then we have a set of domain labels $\mathcal{Y}=\{Y_{i}\}_{i=1}^{M}$, the collected dataset for pre-training can be denoted as $\mathcal{D}=\{D_{Y_{i}}\}_{i=1}^{M}$, where $D_{Y_{i}}$ is the dataset collection in environment domain $Y_{i}$.

We use successor features (SF)\citep{barreto2017successor} to capture the agent-environment interaction model since SFs summarize the dynamics induced by the environment when following a behaviour policy. The first step of building a generally reusable agent-environment interaction model is to learn cross-domain transferable successor features in order to extract a general agent-environment interaction model. For simplicity, we consider a random uniform behaviour policy $\pi_{0}$ for all environment domains. Recall that, SF is defined as the expected cumulative features of $\phi$ by following $\pi_{0}$ starting at a specific state. The SF of $(x^{i}, a)$ in environment domain $Y_{i}$ is defined as:
\begin{equation}
\begin{split}
    \psi^{\pi_{0},i}(x^{{i}}, a)  ={ \mathbb{E}}^{\pi_{0}}_{(x, a, x') \sim \mathcal{T}_{i}} [\sum_{t'=t}^{\infty} \gamma^{t'-t} \phi_{x_{t+1}} |X_{t}=x^{{i}}, A_{t}=a].
\end{split}
\end{equation}
, which satisfies the Bellman Equation as below:
\begin{equation}
\begin{split}
     \psi^{\pi_{0},i}(x^{{i}}, a)  & = \phi_{x^{i}} + \gamma {\mathbb{E}}^{\pi_{0}}_{(x, a, x') \sim \mathcal{T}_{i}} [\psi^{\pi_{0},i}(x^{{i}}_{t+1}, a_{t+1})|X_{t}=x^{{i}}, A_{t}=a]
\end{split}
\label{eq:SF_domain}
\end{equation}
, and it can be learned by minimizing the temporal difference (TD) error:
% Therefore, the successor features in environment domain $Y_{i}$ following a behaviour policy $\pi_{0}$ can be learned by minimizing the temporal difference (TD) error:
\begin{equation}
\begin{split}
     \delta_{sf, Y_{i}}^{2} = \norm {\phi_{x^{i}_{t}} + \gamma \psi^{\pi_{0},i}(x^{{i}}_{t+1}, a_{t+1}) - \psi^{\pi_{0},i}(x^{{i}}_{t}, a_{t})}
\end{split}
\label{eq:sf_loss}
\end{equation}
Next, let's consider a set of domains $\mathcal{Y}=\{Y_{i}\}_{i=0}^{M}$. The goal is to learn a successor feature approximation function $f(.;\theta_{sf}): \mathcal{X}_{i} \times \mathcal{A} \rightarrow 
 \boldsymbol {\psi}$ with parameters $\theta_{sf}$ that is transferable across all the domains $\mathcal{Y}$. Inspired by~\citep{feng2019self}, we add two constraints to Eq. \ref{eq:sf_loss} in order to make the learned SF cross-domain transferable, using the mutual information definition $I(.)$:
 \begin{equation}
\begin{split}
   {min} &  \hspace{0.3cm} \mathcal{L}_{sf}  =\frac{1}{M} \sum_{i=1}^{M}\mathbb{E}_{(x, a, x') \sim \mathcal{D}_{Y_{i}}}[\delta_{sf, Y_{i}}^{2}]\\
   s.t. & \hspace{0.3cm}  I(\mathbf{\psi}^{\pi_{0}}, Y) < \epsilon_{u}; \hspace{0.3cm} I(\mathbf{\psi}^{\pi_{0},i}, {x^{i}}) > \epsilon_{l}, \hspace{0.3cm} \forall i \in \{1, ..., M\}
 \end{split}
 \label{eq:total_loss}
\end{equation}
In the first constraint, $\psi^{\pi_{0}}$ is the SF for an arbitrary environment domain, and Y is the corresponding domain label. It limits the mutual information between an SF and its domain label to a threshold $\epsilon_{u}$ in order to make the learned successor feature domain-invariant. The domain index $Y_{i}$ is dropped since it generally applies to all domains. The second constraint term maintains the mutual information between an SF and its input domain-specific observations\footnote{We use the domain index $Y_{i}$ since it requires to estimating a specific domain's marginal distribution $p^{i}(x)$ that $x^{i} \sim p^{i}(x)$ during training, as detailed in Appendix. \ref{sec:app_cross_domain}.} above a threshold $\epsilon_{l}$, in order to prevent the optimization from collapsing to a trivial solution, for example, a random feature space is also domain-invariant but does not include any useful information for task learning. Technical details about optimizing Eq. \ref{eq:total_loss} are shown in Sec. 3.2.

Note that, learning a cross-domain transferable successor feature representation \textit{will not solve} the out-of-distribution (O.O.D.) problem when reusing it in downstream tasks with unseen changes, so that directly plugin the pre-trained successor features, as a common approach in (\citep{barreto2017successor,barreto2020fast}), \textbf{will not make our pre-trained model generally reusable}. We propose two techniques to tackle this problem, \textit{(1) embodied set construction} that discretizes the successor features into prototype sets (Sec. 3.2);  \textit{(2) feature projection} and \textit{projected Bellman updates} to enable learning stability-plasticity (Sec. 3.3). Combined together, we make the pre-trained agent-environment interaction model generally reusable.

% $\mathcal{Y}=\{Y_{1}, ..., Y_{i}, ...\}$ are the domain labels encoded in a one-hot embedding vector.

\subsection{Pre-training: embodied set construction}
The pre-training process is proposed as an embodied set construction method that has two steps: 
\begin{wrapfigure}[16]{R}{0.43\textwidth} %<-- Wrapfigure covers 6 lines
   \begingroup
\removelatexerror% Nullify \@latex@error
\begin{algorithm}[H]
% \setlength{\belowcaptionskip}{-10pt}
	\SetAlgoLined
	\small
	\KwIn{Offline dataset $\mathcal{D}=\{D_{Y_{i}}\}_{i=1}^{M}$, $\psi^{\pi_{0}}(x, a;\theta_{sf})$, embodied set size $N$}
	\KwResult{Embodied set structure $\Omega^{e}$}
	%\textcolor[rgb]{0.14,0.36,0.73}{\textbf{Initialization}}\\
	Initialize an empty embodied set $\Omega^{e}=\{\}$\\
	Initialize an empty successor feature vector list $\mathbf{L}_{sf}=\{\}$\\
	$\mathcal{D} \leftarrow $ Shuffle $(\mathcal{D})$\\
	\For{each $(x, a, x') \in \mathcal{D}$}{
	    \tcp{Compute cross-domain transferable successor features}
	    $\mathbf{L}_{sf} \leftarrow $ Append $\psi^{\pi_{0}}(x, a;\theta_{sf})$  \\
	}
	\tcp{Constructing embodied agent state set}
	K-means clustering ($\mathbf{L}_{sf}, N$)\\
	$\Omega^{e}=\{\mathbf{e}_{i}\}_{i=1}^{N}$ =  cluster-centers as behavior prototypes  \\
	\caption{Embodied Set Construction}
\end{algorithm}
\endgroup
  \end{wrapfigure}


% The inputs are the agent's past experience in multiple environment domains. The output is a constructed embodied set, which will be reused to support downstream task learning. The pre-training has three steps as below.
\paragraph{(1) Learn cross-domain transferable successor features}
Let a neural network parameterized with $\theta_{sf}$ to approximate the SF representation: $\psi^{\pi_{0}}(x, a;\theta_{sf})$, where $(x, a)$ comes from an arbitrary environment domain. Our aim is to use loss Eq. \ref{eq:total_loss} to find the optimal $\theta_{sf}$. By adding the Lagrangian multipliers $\lambda_{u}$ and $\lambda_{l}$, the Lagrangian dual of Eq. \ref{eq:total_loss} is:
\begin{equation}
\begin{split}
  \underset{\theta_{sf}}{min}  &  \hspace{0.3cm} \mathcal{L}_{sf} + \lambda_{u} I(\mathbf{\psi}^{\pi_{0}}, {Y}) - \lambda_{l} \sum_{i=1}^{M} I(\mathbf{\psi}^{\pi_{0},i}, {x^{i}}) \\
 \end{split}
 \label{eq:total_loss2}
\end{equation}
Directly optimizing the mutual information (MI) terms of Eq. \ref{eq:total_loss2} in high-dimensional space is challenging, we provide tractable solutions by approximating the upper bound and lower bound, as detailed in Appendix. \ref{sec:app_cross_domain}. Optimizing Eq. \ref{eq:total_loss2} will result in  transferable $\psi^{\pi_{0}}(x, a;\theta_{sf})$.


\paragraph{(2) Discretize to behavior prototypes: construct an embodied set structure}
To facilitate downstream task learning with our proposed \textit{embodied feature projection} and \textit{projected Bellman updates} (Sec. 3.3), we discretize the learned cross-domain successor features in three steps: (i) Compute all the successor features using the learned $\psi^{\pi_{0}}(x, a;\theta_{sf})$ from $\mathcal{D}$ containing all the environment domain samples. (ii) Cluster all the successor features using mini-batch K means. (iii) Collect the center of each cluster to construct a set structure $\Omega^{e}$. This is summarized in Algorithm 1. 

We call $\Omega^{e}$ as \textit{embodied set} since it is the component shared across environments and tasks that can be viewed as the embodiment of the agent. And each $\mathbf{e}_{i} \in \Omega^{e}$ is called a prototype since it is the center of a cluster that represents a prototype agent-environment interaction behaviour. 

  

% \begingroup
% \removelatexerror% Nullify \@latex@error
% \begin{algorithm}[]
% % \setlength{\belowcaptionskip}{-10pt}
% 	\SetAlgoLined
% 	\small
% 	\KwIn{Offline dataset $\mathcal{D}=\{D_{Y_{i}}\}_{i=1}^{M}$, $\psi^{\pi_{0}}(x, a;\theta_{sf})$, designed embodied set size $K$}
% 	\KwResult{Embodied set structure $\Omega^{e}$}
% 	%\textcolor[rgb]{0.14,0.36,0.73}{\textbf{Initialization}}\\
% 	Initialize an empty embodied set $\Omega^{e}=\{\}$\\
% 	Initialize an empty successor feature vector list $\mathbf{L}_{sf}=\{\}$\\
% 	$\mathcal{D} \leftarrow $ Shuffle $(\mathcal{D})$\\
% 	\For{each $(x, a, x') \in \mathcal{D}$}{
% 	    \tcp{Compute cross-domain transferable successor features}
% 	    $\mathbf{L}_{sf} \leftarrow $ Append $\psi^{\pi_{0}}(x, a;\theta_{sf})$  \\
% 	}
% 	\tcp{Constructing embodied agent state set}
% 	k-means $\leftarrow$ Mini-batch K-means clustering ($\mathbf{L}_{sf}, K$)\\
       
% 	$\Omega^{e}=\{\mathbf{e}_{i}\}_{i=1}^{k}$ =  k-means.cluster-centers as prototypes  \\
% 	\caption{Embodied Set Construction}
% \end{algorithm}
% \endgroup

\subsection{Re-use: a backbone for general downstream task learning}

Let a downstream task denoted as $\mathcal{M}_{u}=(\mathcal{X}^{u}, \mathcal{A}, \mathcal{T}^{u}, r^{u}, \gamma)$, where the observation space $\mathcal{X}^{u}$, dynamics $\mathcal{T}^{u}$, and task objectives $r^{u}$ can be unseen in pre-training. The action space $\mathcal{A}$ remains the same since we assume the same agent learning a new task based on its past experience. 

Reusing the pre-trained model under the above heterogeneous settings is difficult due to out-of-distribution concerns. Traditional methods will not work. For example, in methods reusing the pre-trained representation \citep{shah2021rrl,kingma2013auto}, changes in the environment will cause non-stationarity in the observation space that directly plugs in the learned representation will make the learning even worse than learning from scratch. For another example, in methods reusing the pre-trained successor features to fast compute the value functions \citep{barreto2020fast}, changes in task objective or the environment will break the linear reward feature assumption.

We propose to avoid directly plugging in the pre-trained model, but to use the pre-trained model---embodied set $\Omega^{e}$ as a base structure for projection-based techniques to accelerate the downstream task learning by tackling learning stability and plasticity explicitly.

\paragraph{(1) Stability: Retain previous knowledge by embodied feature projection} 
We define a feature projection operator $\Pi_{\Omega^{e}}(x, a)$, which takes the input of $(x, a)$, localizes its projection on the constructed embodied set $\Omega^{e}$, and returns the localized feature vector $\mathbf{e}$:
\begin{equation}
    \Pi_{\Omega^{e}}(x, a)=\mathbf{e}, \hspace{0.2cm} \forall \mathbf{e}\in \Omega^{e}, \text{   find the smallest } \xi(x, a, \mathbf{e})
\end{equation}
, where $\xi(x, a, \mathbf{e}) = \norm{\psi^{\pi_{0}}(x,a;\theta_{sf}) - \mathbf{e}}$, or it can be any other distance metric functions. The feature projection operator $\Pi_{\Omega^{e}}(x, a)$ retains previous knowledge by always matching the new task experience with the closest prototype $\mathbf{e}$ in the embodied set.

Note that $\Pi_{\Omega^{e}}(x, a)$ also works for changed sensor modalities. For example, assuming a task requires the agent to understand textual commands, then adding an extra-textual observation $z$ will augment the observation to $[x, z]$. The feature projection operator $\Pi_{\Omega^{e}}(x, a)$ still applies here by using the unchanged sensory modality part $x$. We will show that in Sec. 5.1 (3).

\paragraph{(2) Plasticity: Adapt to changes by projected Bellman Updates}
We use the below projected Bellman updates to accelerate learning the unknown downstream task $\mathcal{M}_{u}=(\mathcal{X}^{u}, \mathcal{A}, \mathcal{T}^{u}, r^{u}, \gamma)$:
\begin{equation}
    Q^{ \pi}(x,a) = \mathbb{E}_{(x, a, x') \sim \mathcal{T}^{u}}[r^{u} + \gamma V^{ \pi}_{proj}(\Pi_{\Omega^{e}}(x',\underset{a'}{\argmax}Q^{ \pi}(x',a'))]
    \label{eq:bellman1}
\end{equation}
\begin{equation}
   V^{ \pi}_{proj}(\Pi_{\Omega^{e}}(x, a))=\mathbb{E}_{(x, a, x') \sim \mathcal{T}^{u}}[r^{u} + \gamma \underset{a'}{max} Q^{ \pi}(x', a')
   \label{eq:bellman2}
\end{equation}
We maintain two value functions---the task $Q^{ \pi}$ and a projected version $V^{ \pi}_{proj}$, to support each other's learning in a bidirectional improvement manner. The motivation is that the projected function $V^{ \pi}_{proj}$ can learn faster than $Q^{ \pi}$  since it is defined on the pre-trained set $\Omega^{e}$ that retains past experience, but is not accurate since $\Omega^{e}$ does not adapt to the new task setting. Meanwhile, the task Q-value function $Q^{ \pi}$ should be more accurate but will take a longer time if learned from scratch. Learning that alternates Bellman updates Eq. \ref{eq:bellman1} and \ref{eq:bellman2} will play a trade-off between retaining the previous knowledge or adapting to the new tasks.

\paragraph{(3) Reuse example: DQN-embodied}
While the above two techniques \textit{embodied feature projection} and \textit{projected Bellman updates} can generally apply to any RL methods in learning a new task, we use DQN as an example to show how to use them to accelerate downstream task learning. Assume we use a neural network parameterized with $\theta_{u}$ to approximate $Q^{ \pi}$, and another neural network parameterized with $\mathbf{w}_{u}$ to approximate $V^{ \pi}_{proj}$. According to Eq. \ref{eq:bellman1} and \ref{eq:bellman2}, we compute the target value for $Q^{ \pi}$ and $V^{ \pi}_{proj}$ at training iteration i as:
\begin{equation}
\begin{split}
   y_{i} = \mathbb{E}_{(x, a, x') \sim \mathcal{T}^{u}}[r^{u}+\gamma V^{ \pi}_{proj}(\Pi_{\Omega^{e}}(x',\underset{a'}{\argmax}Q(x',a';\theta_{u,i-1}));\mathbf{w}_{u, i-1})]
 \end{split}
 \label{eq:dqn1}
\end{equation}
\begin{equation}
     y_{proj, i} = \mathbb{E}_{(x, a, x') \sim \mathcal{T}^{u}}[r^{u} + \gamma \underset{a'}{max} Q^{ \pi}(x', a';\theta_{u, i-1})]
     \label{eq:dqn2}
\end{equation}
Then, the learning objectives formulated as an LMSE loss can be written as:
\begin{equation}
\begin{split}
  \mathcal{L}_{i}(\theta_{u,i}) = \mathbb{E}_{(x, a, x') \sim \mathcal{T}^{u}}[(y_{i} - Q^{\pi}(x, a;\theta_{u, i}))^{2}] 
 \end{split}
 \label{eq:dqn3}
\end{equation}
\begin{equation}
    \mathcal{L}_{i}(\mathbf{w}_{u,i}) = \mathbb{E}_{(x, a, x') \sim \mathcal{T}^{u}}[( y_{proj, i} - V^{ \pi}_{proj}(\Pi_{\Omega^{e}}(x, a));\mathbf{w}_{u,i})^{2}]
    \label{eq:dqn4}
\end{equation}
During training, we alternate learning $Q^{ \pi}$ and $V^{ \pi}_{proj}$. A full description of the proposed DQN-embodied algorithm is summarized in Appendix \ref{sec:dqn_embodied}.

% \begin{wrapfigure}[20]{R}{0.6\textwidth} %<-- Wrapfigure covers 6 lines
%    \begingroup
% \removelatexerror% Nullify \@latex@error
% \begin{algorithm}[H]
%  \setlength{\belowcaptionskip}{-10pt}
% 	\SetAlgoLined
% 	\small
% 	\KwIn{Pre-trained embodied set $\Omega^{e}$, feature projection operator $\Pi_{\Omega^{e}}(x,a)$}
% 	%\textcolor[rgb]{0.14,0.36,0.73}{\textbf{Initialization}}\\
% 	Initialize $Q^{ \pi}(.;\theta_{u})$, $V^{ \pi}_{proj}(.;\mathbf{w}_{u})$,  and replay buffer $\mathcal{D}$\\
	
% 	\For{i=1:N}{
% 	    \tcp{Replay buffer}
% 	    \For{t=0:T}{
% 	        $\epsilon$ greedy select action $a_{t}$ based on ${max}_{a} Q^{ \pi}(x_{t},a_{t};\theta_{u,i})$\\
% 	        Execute action $a_{t}$ in environment, observe $x_{t+1}, r_{t}$\\
% 	        Append transition sample $(x_{t}, a_{t}, r_{t}, x_{t+1})$ in $\mathcal{D}$\\
% 	        Randomly sample batch transitions $\mathcal{B}=\{(x, a, r, x')\}$ from $\mathcal{D}$\\
% 	        \tcp{Learn $Q^{ \pi}$}
% 	        Set $y_{i} = \mathbb{E}_{(x, a, x') \sim \mathcal{T}^{u}}[r^{u}+\gamma V^{ \pi}_{proj}(\Pi_{\Omega^{e}}(x',\underset{a'}{max}Q(x',a';\theta_{u,i-1}));\mathbf{w}_{u, i-1})]$\\
% 	        Perform gradient descent step on $\mathcal{L}_{i}(\theta_{u,i})=(y_{i} - Q^{ \pi}(x, a;\theta_{u, i}))^{2}$\\
% 	        \tcp{Learn $V^{ \pi}_{proj}$ }
% 	        Set $ y_{proj, i} = \mathbb{E}_{(x, a, x') \sim \mathcal{T}^{u}}[r^{u} + \gamma \underset{a'}{max} Q^{ \pi}(x', a';\theta_{u, i-1})]$\\
% 	        Perform gradient descent step on $\mathcal{L}_{i}(\mathbf{w}_{u,i})=( y_{proj, i} - V^{ \pi}_{proj}(\Pi_{\Omega^{e}}(x, a));\mathbf{w}_{u,i})^{2}$
% 	    }
% 	}
% 	\caption{DQN-embodied}
% \end{algorithm}
% \endgroup
%   \end{wrapfigure}


\section{Experiments}
\label{sec:expriments}
To verify the effectiveness of our method, we conduct extensive experiments on ImageNet-1K (IN-1K)~\cite{cvpr2009imagenet} for image classification, COCO~\cite{2014MicrosoftCOCO} for object detection and instance segmentation, and ADE20K~\cite{Zhou2018ADE20k} for semantic segmentation. All experiments are implemented with PyTorch on Ubuntu workstations with NVIDIA A100 GPUs. \textbf{Bold} and \hl{gray} indicate the best performance and our models.

% % table: IN-1K Tiny (5M) & Small (25M)
% \begin{figure*}[t!]
% \vspace{-1.0em}
% \begin{minipage}{0.5\linewidth}
% \centering
%     \begin{table}[H]
    % \vspace{-0.25em}
    \setlength{\tabcolsep}{0.3mm}
    \centering
\resizebox{\linewidth}{!}{
\begin{tabular}{llccccc}
    \toprule
    Architecture                            & Date         & Type & Image   & Param. & FLOPs & Top-1     \\
                                            &              &      & Size    & (M)    & (G)   & Acc (\%)  \\ \hline
    ResNet-18                               & CVPR'2016    & C    & $224^2$ & 11.7   & 1.80  & 71.5      \\
    ShuffleNetV2~$2\times$                  & ECCV'2018    & C    & $224^2$ & 5.5    & 0.60  & 75.4      \\
    EfficientNet-B0                         & ICML'2019    & C    & $224^2$ & 5.3    & 0.39  & 77.1      \\
    % MobileNetV3~$1\times$                   & ICCV'2019    & C    & $224^2$ & 5.4    & 0.23  & 75.2      \\
    % RegNetY-400MF                          & CVPR'2020    & C    & $224^2$ & 5.3    & 0.40  & 74.1      \\
    RegNetY-800MF                           & CVPR'2020    & C    & $224^2$ & 6.3    & 0.80  & 76.3      \\
    DeiT-T$^\dag$                           & ICML'2021    & T    & $224^2$ & 5.7    & 1.08  & 74.1      \\
    % DeiT-2G                                & ICML'2021    & T    & $224^2$ & 13.2   & 1.90  & 75.1      \\
    PVT-T                                   & ICCV'2021    & T    & $224^2$ & 13.2   & 1.60  & 75.1      \\
    T2T-ViT-7                               & ICCV'2021    & T    & $224^2$ & 4.3    & 1.20  & 71.7      \\
    % T2T-ViT-12                             & ICCV'2021    & T    & $224^2$ & 6.9    & 1.80  & 76.5      \\
    ViT-C                                   & NIPS'2021    & T    & $224^2$ & 4.6    & 1.10  & 75.3      \\
    SReT-T$_{Distill}$                      & ECCV'2022    & T    & $224^2$ & 4.8    & 1.10  & 77.6      \\
    PiT-Ti                                  & ICCV'2021    & H    & $224^2$ & 4.9    & 0.70  & 74.6      \\
    LeViT-S                                 & ICCV'2021    & H    & $224^2$ & 7.8    & 0.31  & 76.6      \\
    CoaT-Lite-T                             & ICCV'2021    & H    & $224^2$ & 5.7    & 1.60  & 77.5      \\
    Swin-1G                                 & ICCV'2021    & H    & $224^2$ & 7.3    & 1.00  & 77.3      \\
    % Swin-2G                                 & ICCV'2021    & H    & $224^2$ & 12.8   & 2.00  & 79.3      \\
    % MobileViT-XS                            & ICLR'2022    & H    & $256^2$ & 2.3    & 1.73  & 74.8      \\
    MobileViT-S                             & ICLR'2022    & H    & $256^2$ & 5.6    & 4.02  & 78.4      \\
    % MobileFormer-151M                      & CVPR'2022    & H    & $224^2$ & 7.6    & 0.29  & 75.2      \\
    MobileFormer-294M                       & CVPR'2022    & H    & $224^2$ & 11.4   & 0.59  & 77.9      \\
    ConvNext-XT                             & CVPR'2022    & C    & $224^2$ & 7.4    & 0.60  & 77.5      \\
    VAN-B0                                  & CVMJ'2023   & C    & $224^2$ & 4.1    & 0.88  & 75.4      \\
    ParC-Net-S                              & ECCV'2022    & C    & $256^2$ & 5.0    & 3.48  & 78.6      \\
    \rowcolor{gray94}\bf{MogaNet-XT}        & Ours         & C    & $256^2$ & 3.0    & 1.04  & 77.2      \\
    \rowcolor{gray94}\bf{MogaNet-T}         & Ours         & C    & $224^2$ & 5.2    & 1.10  & 79.0      \\
    \rowcolor{gray94}\bf{MogaNet-T}$^\S$    & Ours         & C    & $256^2$ & 5.2    & 1.44  & \bf{80.0} \\
    \bottomrule
    \end{tabular}
    }
    \vspace{-1.0em}
    \caption{\textbf{IN-1K classification} with lightweight models. \small{$\S$} denotes the refined training scheme.
    % \small{$\dag$} and \small{$\S$} are RSB A2 and refined training schemes.
    }
    \label{tab:in1k_cls_tiny}
    % \vspace{-0.5em}
\end{table}


% \begin{table}[h]
%     \vspace{-0.5em}
%     \setlength{\tabcolsep}{0.7mm}
%     \centering
% \resizebox{\linewidth}{!}{
% \begin{tabular}{llcccc}
%     \toprule
%     Architecture                                     & Date         & Image   & Param. & FLOPs & Top-1     \\
%                                                      &              & Size    & (M)    & (G)   & Acc (\%)  \\ \hline
%     ResNet-18$^\dag$~\cite{he2016deep}               & CVPR'2016    & $224^2$ & 11.7   & 1.80  & 71.5      \\
%     ShuffleNetV2~$2\times$~\cite{eccv2018shufflenet} & ECCV'2018    & $224^2$ & 5.5    & 0.60  & 75.4      \\
%     EfficientNet-B0~\cite{icml2019efficientnet}      & ICML'2019    & $224^2$ & 5.3    & 0.39  & 77.1      \\
%     MobileNetV3~$1\times$~\cite{iccv2019mobilenetv3} & ICCV'2019    & $224^2$ & 5.4    & 0.23  & 75.2      \\
%     RegNetY-800M~\cite{cvpr2020regnet}               & CVPR'2020    & $224^2$ & 6.3    & 0.80  & 76.3      \\ \hline
%     DeiT-T~\cite{icml2021deit}                     & ICML'201     & $224^2$ & 5.7    & 1.08  & 72.2      \\
%     PVT-T~\cite{iccv2021PVT}                       & ICCV'2021    & $224^2$ & 13.2   & 1.60  & 75.1      \\
%     T2T-ViT-7~\cite{iccv2021t2t}                   & ICCV'2021    & $224^2$ & 4.3    & 1.20  & 71.7      \\
%     T2T-ViT-12~\cite{iccv2021t2t}                  & ICCV'2021    & $224^2$ & 6.9    & 1.80  & 76.5      \\
%     ViT-C~\cite{nips2021vitc}                      & NIPS'2021    & $224^2$ & 4.6    & 1.10  & 75.3      \\ \hline
%     PiT-Ti~\cite{iccv2021pit}                      & ICCV'2021    & $224^2$ & 4.9    & 0.70  & 74.6      \\
%     LeViT-S~\cite{iccv2021levit}                   & ICCV'2021    & $224^2$ & 7.8    & 0.31  & 76.6      \\
%     CoaT-Lite-T~\cite{iccv2021coat}                & ICCV'2021    & $224^2$ & 5.7    & 1.60  & 77.5      \\
%     MobileViT-XS~\cite{iclr2022mobilevit}          & ICLR'2022    & $256^2$ & 2.3    & 1.73  & 74.8      \\
%     MobileViT-S~\cite{iclr2022mobilevit}           & ICLR'2022    & $256^2$ & 5.6    & 4.02  & 78.4      \\
%     Mobile-Former-151M~\cite{cvpr2022MobileFormer} & CVPR'2022    & $224^2$ & 7.6    & 0.29  & 75.2      \\
%     Mobile-Former-294M~\cite{cvpr2022MobileFormer} & CVPR'2022    & $224^2$ & 11.4   & 0.59  & 77.9      \\ \hline
%     ConvNext-XT~\cite{cvpr2022convnext}            & CVPR'2022    & $224^2$ & 7.4    & 0.60  & 77.5      \\
%     VAN-B0~\cite{guo2022van}                       & arXiv'2022   & $224^2$ & 4.1    & 0.88  & 75.4      \\
%     ParC-Net-S~\cite{eccv2022edgeformer}           & ECCV'2022    & $256^2$ & 5.0    & 3.48  & 78.6      \\
%     \rowcolor{gray94}\bf{MogaNet-XT}               & Ours         & $224^2$ & 3.0    & 0.80  & 76.3      \\
%     \rowcolor{gray94}\bf{MogaNet-T}                & Ours         & $224^2$ & 5.2    & 1.10  & 79.0      \\
%     \rowcolor{gray94}\bf{MogaNet-T}                & Ours         & $256^2$ & 5.2    & 1.44  & \bf{79.6} \\
%     \bottomrule
%     \end{tabular}
%     }
%     \vspace{-0.5em}
%     \caption{ImageNet-1K classification performance of lightweight (around 5M Parameters) models.}
%     \label{tab:in1k_cls_tiny}
%     \vspace{-0.75em}
% \end{table}

% \end{minipage}
% \begin{minipage}{0.5\linewidth}
% \centering
%     \begin{table}[h]
    \vspace{-0.25em}
    \setlength{\tabcolsep}{0.8mm}
    \centering
\resizebox{\linewidth}{!}{
\begin{tabular}{llcccc}
    \toprule
    Architecture                                 & Date         & Image   & Param. & FLOPs & Top-1     \\
                                                 &              & Size    & (M)    & (G)   & Acc (\%)  \\ \hline
    ResNet-50$^\dag$~\cite{he2016deep}           & CVPR'2016    & $224^2$ & 26     & 4.1   & 80.4      \\
    EfficientNet-B4~\cite{icml2019efficientnet}  & ICML'2019    & $380^2$ & 19     & 4.2   & 82.9      \\
    RegNetY-4GF$^\dag$~\cite{cvpr2020regnet}     & CVPR'2020    & $224^2$ & 21     & 4.0   & 81.5      \\ \hline
    Deit-S~\cite{icml2021deit}                   & ICML'2021    & $224^2$ & 22     & 4.6   & 79.8      \\
    Swin-T~\cite{liu2021swin}                    & ICCV'2021    & $224^2$ & 28     & 4.5   & 81.3      \\
    T2T-ViT$_t$-14~\cite{iccv2021t2t}            & ICCV'2021    & $224^2$ & 22     & 6.1   & 81.7      \\
    CSWin-T~\cite{cvpr2022CSWin}                 & CVPR'2022    & $224^2$ & 23     & 4.3   & 82.8      \\
    SReT-S~\cite{eccv2022SReT}                   & ECCV'2022    & $224^2$ & 21     & 4.2   & 81.9      \\
    LITV2-S~\cite{nips2022hilo}                  & NIPS'2022    & $224^2$ & 28     & 3.7   & 82.0      \\ \hline
    CoaT-S~\cite{iccv2021coat}                   & ICCV'2021    & $224^2$ & 22     & 12.6  & 82.1      \\
    CoAtNet-0~\cite{nips2021coatnet}             & NIPS'2021    & $224^2$ & 25     & 4.2   & 82.7      \\
    ViTAE-S~\cite{nips2021vitae}                 & NIPS'2021    & $224^2$ & 24     & 5.6   & 82.0      \\
    UniFormer-S~\cite{iclr2022uniformer}         & ICLR'2022    & $224^2$ & 22     & 3.6   & 82.9      \\
    EfficientFormer-L3~\cite{nips2022EfficientFormer} & NIPS'2022    & $224^2$ & 31     & 3.9   & 82.4      \\ \hline
    ConvNeXt-T~\cite{cvpr2022convnext}           & CVPR'2022    & $224^2$ & 29     & 4.5   & 82.1      \\
    VAN-B2~\cite{guo2022van}                     & arXiv'2022   & $224^2$ & 27     & 5.0   & 82.8      \\
    SLaK-T~\cite{Liu2022SLak}                    & arXiv'2022   & $224^2$ & 30     & 5.0   & 82.5      \\
    HorNet-T$_{7\times 7}$~\cite{nips2022hornet} & NIPS'2022    & $224^2$ & 22     & 4.0   & 82.8      \\
    \rowcolor{gray94}\bf{MogaNet-S}              & Ours         & $224^2$ & 25     & 5.0   & \bf{83.4} \\
    \bottomrule
    \end{tabular}
    }
    \vspace{-0.5em}
    \caption{\textbf{ImageNet-1K classification} performance of small size (around 25M parameters) models.}
    \label{tab:in1k_cls_small}
    \vspace{-1.25em}
\end{table}

% \end{minipage}
% \vspace{-1.5em}
% \end{figure*}

\subsection{ImageNet Classification}
\label{sec:exp_in1k}
\paragraph{Settings.} 
For classification experiments on ImageNet-1K, we train MogaNet variants following the standard procedure \cite{icml2021deit, liu2021swin} for a fair comparison. Specifically, the models are trained for 300 epochs by AdamW~\cite{iclr2019AdamW} optimizer with $224^2$ or $256^2$ resolutions, a basic learning rate $lr$ = $1\times 10^{-3}$, 5 epochs warmup, and a Cosine scheduler~\cite{loshchilov2016sgdr}. See Appendix~\ref{app:in1k_settings} for implementation details.
We compare four typical architectures: (\romannumeral1) \textbf{Classical ConvNets} include ResNet, SENet, ShuffleNetV2, EfficientNet, MobileNetV3, and RegNet. (\romannumeral2) \textbf{Transformers} include DeiT, Swin, T2T-ViT, PVT, Focal, ViT-C, CSWin, SReT, and LiTV2. (\romannumeral3) \textbf{Hybrid architectures} of attention and convolution include PiT, LeViT, CoaT, BoTNet, ViTAE, Twins, CoAtNet, MobileViT, Uniformer, Mobile-Former, ParC-Net, EfficientFormer, and MaxViT. (\romannumeral4) \textbf{Modern ConvNets} include ConvNeXt, RepLKNet, FocalNet, VAN, SLak, and HorNet.
% We compare four types of popular network architectures: (\romannumeral1) \textbf{Classical CNN} includes ResNet~\cite{he2016deep}, SENet~\cite{hu2018squeeze}, ShuffleNetV2~\cite{eccv2018shufflenet}, EfficientNet~\cite{icml2019efficientnet}, MobileNetV3~\cite{iccv2019mobilenetv3}, and RegNet~\cite{cvpr2020regnet}. (\romannumeral2) \textbf{Transformer} includes DeiT~\cite{icml2021deit}, Swin~\cite{liu2021swin}, T2T-ViT~\cite{iccv2021t2t}, PVT~\cite{iccv2021PVT}, FocalNet~\cite{nips2021Focal}, ViT-C~\cite{nips2021vitc}, CSWin~\cite{cvpr2022CSWin}, SReT~\cite{eccv2022SReT}, and LiTV2~\cite{nips2022hilo}. (\romannumeral3) \textbf{Hybrid} Transformer and CNN architecture includes PiT~\cite{iccv2021pit}, LeViT~\cite{iccv2021levit}, CoaT~\cite{iccv2021coat}, BoTNet~\cite{cvpr2021botnet}, ViTAE~\cite{nips2021vitae}, Twins~\cite{nips2021Twins}, CoAtNet~\cite{nips2021coatnet}, MobileViT~\cite{iclr2022mobilevit}, Uniformer~\cite{iclr2022uniformer}, Mobile-Former~\cite{cvpr2022MobileFormer}, and ParC-Net~\cite{eccv2022edgeformer}. (\romannumeral3) \textbf{Post-ConvNet} includes ConvNeXt~\cite{cvpr2022convnext}, RepLKNet~\cite{cvpr2022replknet}, VAN~\cite{guo2022van}, SLak~\cite{Liu2022SLak}, and HorNet~\cite{nips2022hornet}.

% table: IN-1K Tiny (5M) & Small (25M)
\begin{table}[H]
    % \vspace{-0.25em}
    \setlength{\tabcolsep}{0.3mm}
    \centering
\resizebox{\linewidth}{!}{
\begin{tabular}{llccccc}
    \toprule
    Architecture                            & Date         & Type & Image   & Param. & FLOPs & Top-1     \\
                                            &              &      & Size    & (M)    & (G)   & Acc (\%)  \\ \hline
    ResNet-18                               & CVPR'2016    & C    & $224^2$ & 11.7   & 1.80  & 71.5      \\
    ShuffleNetV2~$2\times$                  & ECCV'2018    & C    & $224^2$ & 5.5    & 0.60  & 75.4      \\
    EfficientNet-B0                         & ICML'2019    & C    & $224^2$ & 5.3    & 0.39  & 77.1      \\
    % MobileNetV3~$1\times$                   & ICCV'2019    & C    & $224^2$ & 5.4    & 0.23  & 75.2      \\
    % RegNetY-400MF                          & CVPR'2020    & C    & $224^2$ & 5.3    & 0.40  & 74.1      \\
    RegNetY-800MF                           & CVPR'2020    & C    & $224^2$ & 6.3    & 0.80  & 76.3      \\
    DeiT-T$^\dag$                           & ICML'2021    & T    & $224^2$ & 5.7    & 1.08  & 74.1      \\
    % DeiT-2G                                & ICML'2021    & T    & $224^2$ & 13.2   & 1.90  & 75.1      \\
    PVT-T                                   & ICCV'2021    & T    & $224^2$ & 13.2   & 1.60  & 75.1      \\
    T2T-ViT-7                               & ICCV'2021    & T    & $224^2$ & 4.3    & 1.20  & 71.7      \\
    % T2T-ViT-12                             & ICCV'2021    & T    & $224^2$ & 6.9    & 1.80  & 76.5      \\
    ViT-C                                   & NIPS'2021    & T    & $224^2$ & 4.6    & 1.10  & 75.3      \\
    SReT-T$_{Distill}$                      & ECCV'2022    & T    & $224^2$ & 4.8    & 1.10  & 77.6      \\
    PiT-Ti                                  & ICCV'2021    & H    & $224^2$ & 4.9    & 0.70  & 74.6      \\
    LeViT-S                                 & ICCV'2021    & H    & $224^2$ & 7.8    & 0.31  & 76.6      \\
    CoaT-Lite-T                             & ICCV'2021    & H    & $224^2$ & 5.7    & 1.60  & 77.5      \\
    Swin-1G                                 & ICCV'2021    & H    & $224^2$ & 7.3    & 1.00  & 77.3      \\
    % Swin-2G                                 & ICCV'2021    & H    & $224^2$ & 12.8   & 2.00  & 79.3      \\
    % MobileViT-XS                            & ICLR'2022    & H    & $256^2$ & 2.3    & 1.73  & 74.8      \\
    MobileViT-S                             & ICLR'2022    & H    & $256^2$ & 5.6    & 4.02  & 78.4      \\
    % MobileFormer-151M                      & CVPR'2022    & H    & $224^2$ & 7.6    & 0.29  & 75.2      \\
    MobileFormer-294M                       & CVPR'2022    & H    & $224^2$ & 11.4   & 0.59  & 77.9      \\
    ConvNext-XT                             & CVPR'2022    & C    & $224^2$ & 7.4    & 0.60  & 77.5      \\
    VAN-B0                                  & CVMJ'2023   & C    & $224^2$ & 4.1    & 0.88  & 75.4      \\
    ParC-Net-S                              & ECCV'2022    & C    & $256^2$ & 5.0    & 3.48  & 78.6      \\
    \rowcolor{gray94}\bf{MogaNet-XT}        & Ours         & C    & $256^2$ & 3.0    & 1.04  & 77.2      \\
    \rowcolor{gray94}\bf{MogaNet-T}         & Ours         & C    & $224^2$ & 5.2    & 1.10  & 79.0      \\
    \rowcolor{gray94}\bf{MogaNet-T}$^\S$    & Ours         & C    & $256^2$ & 5.2    & 1.44  & \bf{80.0} \\
    \bottomrule
    \end{tabular}
    }
    \vspace{-1.0em}
    \caption{\textbf{IN-1K classification} with lightweight models. \small{$\S$} denotes the refined training scheme.
    % \small{$\dag$} and \small{$\S$} are RSB A2 and refined training schemes.
    }
    \label{tab:in1k_cls_tiny}
    % \vspace{-0.5em}
\end{table}


% \begin{table}[h]
%     \vspace{-0.5em}
%     \setlength{\tabcolsep}{0.7mm}
%     \centering
% \resizebox{\linewidth}{!}{
% \begin{tabular}{llcccc}
%     \toprule
%     Architecture                                     & Date         & Image   & Param. & FLOPs & Top-1     \\
%                                                      &              & Size    & (M)    & (G)   & Acc (\%)  \\ \hline
%     ResNet-18$^\dag$~\cite{he2016deep}               & CVPR'2016    & $224^2$ & 11.7   & 1.80  & 71.5      \\
%     ShuffleNetV2~$2\times$~\cite{eccv2018shufflenet} & ECCV'2018    & $224^2$ & 5.5    & 0.60  & 75.4      \\
%     EfficientNet-B0~\cite{icml2019efficientnet}      & ICML'2019    & $224^2$ & 5.3    & 0.39  & 77.1      \\
%     MobileNetV3~$1\times$~\cite{iccv2019mobilenetv3} & ICCV'2019    & $224^2$ & 5.4    & 0.23  & 75.2      \\
%     RegNetY-800M~\cite{cvpr2020regnet}               & CVPR'2020    & $224^2$ & 6.3    & 0.80  & 76.3      \\ \hline
%     DeiT-T~\cite{icml2021deit}                     & ICML'201     & $224^2$ & 5.7    & 1.08  & 72.2      \\
%     PVT-T~\cite{iccv2021PVT}                       & ICCV'2021    & $224^2$ & 13.2   & 1.60  & 75.1      \\
%     T2T-ViT-7~\cite{iccv2021t2t}                   & ICCV'2021    & $224^2$ & 4.3    & 1.20  & 71.7      \\
%     T2T-ViT-12~\cite{iccv2021t2t}                  & ICCV'2021    & $224^2$ & 6.9    & 1.80  & 76.5      \\
%     ViT-C~\cite{nips2021vitc}                      & NIPS'2021    & $224^2$ & 4.6    & 1.10  & 75.3      \\ \hline
%     PiT-Ti~\cite{iccv2021pit}                      & ICCV'2021    & $224^2$ & 4.9    & 0.70  & 74.6      \\
%     LeViT-S~\cite{iccv2021levit}                   & ICCV'2021    & $224^2$ & 7.8    & 0.31  & 76.6      \\
%     CoaT-Lite-T~\cite{iccv2021coat}                & ICCV'2021    & $224^2$ & 5.7    & 1.60  & 77.5      \\
%     MobileViT-XS~\cite{iclr2022mobilevit}          & ICLR'2022    & $256^2$ & 2.3    & 1.73  & 74.8      \\
%     MobileViT-S~\cite{iclr2022mobilevit}           & ICLR'2022    & $256^2$ & 5.6    & 4.02  & 78.4      \\
%     Mobile-Former-151M~\cite{cvpr2022MobileFormer} & CVPR'2022    & $224^2$ & 7.6    & 0.29  & 75.2      \\
%     Mobile-Former-294M~\cite{cvpr2022MobileFormer} & CVPR'2022    & $224^2$ & 11.4   & 0.59  & 77.9      \\ \hline
%     ConvNext-XT~\cite{cvpr2022convnext}            & CVPR'2022    & $224^2$ & 7.4    & 0.60  & 77.5      \\
%     VAN-B0~\cite{guo2022van}                       & arXiv'2022   & $224^2$ & 4.1    & 0.88  & 75.4      \\
%     ParC-Net-S~\cite{eccv2022edgeformer}           & ECCV'2022    & $256^2$ & 5.0    & 3.48  & 78.6      \\
%     \rowcolor{gray94}\bf{MogaNet-XT}               & Ours         & $224^2$ & 3.0    & 0.80  & 76.3      \\
%     \rowcolor{gray94}\bf{MogaNet-T}                & Ours         & $224^2$ & 5.2    & 1.10  & 79.0      \\
%     \rowcolor{gray94}\bf{MogaNet-T}                & Ours         & $256^2$ & 5.2    & 1.44  & \bf{79.6} \\
%     \bottomrule
%     \end{tabular}
%     }
%     \vspace{-0.5em}
%     \caption{ImageNet-1K classification performance of lightweight (around 5M Parameters) models.}
%     \label{tab:in1k_cls_tiny}
%     \vspace{-0.75em}
% \end{table}

\begin{table}[h]
    \vspace{-0.25em}
    \setlength{\tabcolsep}{0.8mm}
    \centering
\resizebox{\linewidth}{!}{
\begin{tabular}{llcccc}
    \toprule
    Architecture                                 & Date         & Image   & Param. & FLOPs & Top-1     \\
                                                 &              & Size    & (M)    & (G)   & Acc (\%)  \\ \hline
    ResNet-50$^\dag$~\cite{he2016deep}           & CVPR'2016    & $224^2$ & 26     & 4.1   & 80.4      \\
    EfficientNet-B4~\cite{icml2019efficientnet}  & ICML'2019    & $380^2$ & 19     & 4.2   & 82.9      \\
    RegNetY-4GF$^\dag$~\cite{cvpr2020regnet}     & CVPR'2020    & $224^2$ & 21     & 4.0   & 81.5      \\ \hline
    Deit-S~\cite{icml2021deit}                   & ICML'2021    & $224^2$ & 22     & 4.6   & 79.8      \\
    Swin-T~\cite{liu2021swin}                    & ICCV'2021    & $224^2$ & 28     & 4.5   & 81.3      \\
    T2T-ViT$_t$-14~\cite{iccv2021t2t}            & ICCV'2021    & $224^2$ & 22     & 6.1   & 81.7      \\
    CSWin-T~\cite{cvpr2022CSWin}                 & CVPR'2022    & $224^2$ & 23     & 4.3   & 82.8      \\
    SReT-S~\cite{eccv2022SReT}                   & ECCV'2022    & $224^2$ & 21     & 4.2   & 81.9      \\
    LITV2-S~\cite{nips2022hilo}                  & NIPS'2022    & $224^2$ & 28     & 3.7   & 82.0      \\ \hline
    CoaT-S~\cite{iccv2021coat}                   & ICCV'2021    & $224^2$ & 22     & 12.6  & 82.1      \\
    CoAtNet-0~\cite{nips2021coatnet}             & NIPS'2021    & $224^2$ & 25     & 4.2   & 82.7      \\
    ViTAE-S~\cite{nips2021vitae}                 & NIPS'2021    & $224^2$ & 24     & 5.6   & 82.0      \\
    UniFormer-S~\cite{iclr2022uniformer}         & ICLR'2022    & $224^2$ & 22     & 3.6   & 82.9      \\
    EfficientFormer-L3~\cite{nips2022EfficientFormer} & NIPS'2022    & $224^2$ & 31     & 3.9   & 82.4      \\ \hline
    ConvNeXt-T~\cite{cvpr2022convnext}           & CVPR'2022    & $224^2$ & 29     & 4.5   & 82.1      \\
    VAN-B2~\cite{guo2022van}                     & arXiv'2022   & $224^2$ & 27     & 5.0   & 82.8      \\
    SLaK-T~\cite{Liu2022SLak}                    & arXiv'2022   & $224^2$ & 30     & 5.0   & 82.5      \\
    HorNet-T$_{7\times 7}$~\cite{nips2022hornet} & NIPS'2022    & $224^2$ & 22     & 4.0   & 82.8      \\
    \rowcolor{gray94}\bf{MogaNet-S}              & Ours         & $224^2$ & 25     & 5.0   & \bf{83.4} \\
    \bottomrule
    \end{tabular}
    }
    \vspace{-0.5em}
    \caption{\textbf{ImageNet-1K classification} performance of small size (around 25M parameters) models.}
    \label{tab:in1k_cls_small}
    \vspace{-1.25em}
\end{table}


% table: IN-1K Base (40M) & Large (80M)
\begin{figure*}[t!]
\vspace{-1.5em}
\begin{minipage}{0.495\linewidth}
\centering
    \begin{table}[H]
    \vspace{-0.25em}
    \setlength{\tabcolsep}{0.9mm}
    \centering
\resizebox{\linewidth}{!}{
\begin{tabular}{llcccc}
    \toprule
    Architecture                                 & Date         & Image   & Param. & FLOPs & Top-1     \\
                                                 &              & Size    & (M)    & (G)   & Acc (\%)  \\ \hline
    ResNet-101$^\dag$~\cite{he2016deep}          & CVPR'2016    & $224^2$ & 45     & 7.9   & 81.5      \\
    EfficientNet-B6~\cite{icml2019efficientnet}  & ICML'2019    & $528^2$ & 43     & 19.0  & 84.0      \\
    RegNetY-8GF$^\dag$~\cite{cvpr2020regnet}     & CVPR'2020    & $224^2$ & 39     & 8.1   & 82.2      \\ \hline
    T2T-ViT-24~\cite{iccv2021t2t}                & ICCV'2021    & $224^2$ & 64     & 13.2  & 82.2      \\
    Swin-S~\cite{liu2021swin}                    & ICCV'2021    & $224^2$ & 50     & 8.7   & 83.0      \\
    Focal-S~\cite{nips2021Focal}                 & NIPS'2021    & $224^2$ & 51     & 9.1   & 83.6      \\
    CSWin-S~\cite{cvpr2022CSWin}                 & CVPR'2022    & $224^2$ & 35     & 6.9   & 83.6      \\
    LITV2-M~\cite{nips2022hilo}                  & NIPS'2022    & $224^2$ & 49     & 7.5   & 83.3      \\ \hline
    CoaT-M~\cite{iccv2021coat}                   & ICCV'2021    & $224^2$ & 45     & 9.8   & 83.6      \\
    Twins-SVT-B~\cite{nips2021Twins}             & NIPS'2021    & $224^2$ & 56     & 8.6   & 83.2      \\
    CoAtNet-1~\cite{nips2021coatnet}             & NIPS'2021    & $224^2$ & 42     & 8.4   & 83.3      \\
    UniFormer-B~\cite{iclr2022uniformer}         & ICLR'2022    & $224^2$ & 50     & 8.3   & 83.9      \\
    FAN-B-Hybrid~\cite{icml2022FAN}              & ICML'2022    & $224^2$ & 50     & 11.3  & 83.9      \\ \hline
    ConvNeXt-S~\cite{cvpr2022convnext}           & CVPR'2022    & $224^2$ & 50     & 8.7   & 83.1      \\
    FocalNet-S (LRF)~\cite{nips2022focalnet}     & NIPS'2022    & $224^2$ & 50     & 8.7   & 83.5      \\
    HorNet-S$_{7\times 7}$~\cite{nips2022hornet} & NIPS'2022    & $224^2$ & 50     & 8.8   & 84.0      \\
    VAN-B3~\cite{guo2022van}                     & arXiv'2022   & $224^2$ & 45     & 9.0   & 83.9      \\
    SLaK-S~\cite{Liu2022SLak}                    & ICLR'2023    & $224^2$ & 55     & 9.8   & 83.8      \\
    \rowcolor{gray94}\bf{MogaNet-B}              & Ours         & $224^2$ & 44     & 9.9   & \bf{84.3} \\
    \bottomrule
    \end{tabular}
    }
    \vspace{-0.5em}
    \caption{\textbf{ImageNet-1K classification} performance of medium size (around 45M parameters) models.}
    \label{tab:in1k_cls_base}
\end{table}

\end{minipage}
~\begin{minipage}{0.495\linewidth}
\centering
    \begin{table}[H]
    \vspace{-0.25em}
    \setlength{\tabcolsep}{0.9mm}
    \centering
\resizebox{\linewidth}{!}{
\begin{tabular}{llcccc}
    \toprule
Architecture                    & Date       & Image   & Param. & FLOPs & Top-1     \\
                                &            & Size    & (M)    & (G)   & Acc (\%)  \\ \hline
ResNet-152$^\dag$               & CVPR'2016  & $224^2$ & 60     & 11.6  & 82.0      \\
SE-ResNet-154$^\dag$            & CVPR'2018  & $224^2$ & 115    & 20.9  & 81.7      \\
RegNetY-16GF                    & CVPR'2020  & $224^2$ & 84     & 16.0  & 82.9      \\ \hline
DeiT-B                          & ICML'2021  & $224^2$ & 86     & 17.5  & 81.8      \\
Swin-B                          & ICCV'2021  & $224^2$ & 89     & 15.4  & 83.5      \\
Focal-B                         & NIPS'2021  & $224^2$ & 90     & 16.4  & 84.0      \\
CSWin-B                         & CVPR'2022  & $224^2$ & 78     & 15.0  & 84.2      \\
LITV2-B                         & NIPS'2022  & $224^2$ & 87     & 13.2  & 83.6      \\ \hline
BoTNet-T7                       & CVPR'2021  & $256^2$ & 79     & 19.3  & 84.2      \\
Twins-SVT-L                     & NIPS'2021  & $224^2$ & 99     & 15.1  & 83.7      \\
CoAtNet-2                       & NIPS'2021  & $224^2$ & 75     & 15.7  & 84.1      \\
FAN-B-Hybrid                    & ICML'2022  & $224^2$ & 77     & 16.9  & 84.3      \\ \hline
ConvNeXt-B                      & CVPR'2022  & $224^2$ & 89     & 15.4  & 83.8      \\
RepLKNet-31B                    & CVPR'2022  & $224^2$ & 79     & 15.3  & 83.5      \\
FocalNet-B (LRF)                & NIPS'2022  & $224^2$ & 89     & 15.4  & 83.9      \\
HorNet-B$_{7\times 7}$          & NIPS'2022  & $224^2$ & 87     & 15.6  & 84.3      \\
VAN-B4                          & CVMJ'2023  & $224^2$ & 60     & 12.2  & 84.2      \\
SLaK-B                          & ICLR'2023  & $224^2$ & 95     & 17.1  & 84.0      \\
\rowcolor{gray94}\bf{MogaNet-L} & Ours       & $224^2$ & 83     & 15.9  & \bf{84.7} \\
    \bottomrule
    \end{tabular}
    }
    \vspace{-0.5em}
    \caption{\textbf{ImageNet-1K classification} performance of large size (around 80M parameters) models.}
    \label{tab:in1k_cls_large}
\end{table}

\end{minipage}
\vspace{-1.25em}
\end{figure*}

\paragraph{Results.}
We compare the image classification performances of four widely adopted model sizes (around 5M, 25M, 45M, and 80M parameters).
As for lightweight models, Table~\ref{tab:in1k_cls_tiny} shows that MogaNet-XT/T significantly outperforms existing lightweight architectures. Using the default training settings, MogaNet-T achieves 79.0\% top-1 accuracy, which improves models with around 5M parameters by at least 1.1\% using $224^2$ resolutions, while outperforming the current best backbone ParC-Net-S by 1.0\% using $256^2$ resolutions. Meanwhile, MogaNet-XT also surpasses models with 3M parameters, \textit{e.g.,} +4.6\% and +1.5\% over T2T-ViT-7 and MobileViT-XS. Particularly, MogaNet-T$^{\S}$ achieves 80.0\% top-1 accuracy using $256^2$ resolutions and the refined settings, which adjusts $lr$ and replaces RandAugment~\cite{cubuk2020randaugment} with 3-Augment~\cite{eccv2022deit3} as detailed in Appendix~\ref{app:advanced_tiny}.
As for small-size models, Table~\ref{tab:in1k_cls_small} shows MogaNet-S achieves 83.4\% top-1 accuracy, which consistently outperforms Transformers, hybrid architectures, and ConvNets, \textit{e.g.,} +2.1\% and +1.2\% over Swin-T and ConvNeXt-T. 
As for 45M and 80M models, we summarize their performances in Table~\ref{tab:in1k_cls_base} and Table~\ref{tab:in1k_cls_large} and MogaNet-B/L still surpass the current state-of-the-art architectures, especially improving Swin-S/B and ConvNeXt-S/B by 1.2\%/ 1.1\% and 1.1\%/ 0.8\%. MogaNet also outperforms recently proposed modern ConvNets, \textit{e.g.,} +0.9\% over RepLKNet-31B and +0.2\%/ 0.3\% over HorNet-S/B$_{7\times 7}$.


\subsection{Dense Prediction Tasks}
\label{sec:exp_det_seg}
\paragraph{Object detection and segmentation on COCO.}
We evaluate MogaNet for object detection and segmentation tasks on the COCO dataset using Mask-RCNN~\cite{2017iccvmaskrcnn} as the detector. Following the training and evaluation settings in \cite{liu2021swin}, we fine-tune the models with AdamW optimizer for $1\times$ training schedule (12-epoch) on the COCO~\textit{train2017} and evaluate on the COCO~\textit{val2017}. We adopt MMDetection~\cite{mmdetection} as the codebase and measure the performance by the box mAP (AP$^{bb}$) and mask mAP (AP$^{mk}$). Refer to Appendix~\ref{app:coco_settings} for more details. Table~\ref{tab:coco} shows that models with MogaNet-T/S/B significantly outperform all previous backbones. Specifically, MogaNet-T gains 3.6\% AP$^{bb}$ and 4.6\% AP$^{mk}$ over ResNet-18; MogaNet-S outperforms Swin-T (Transformers) by 3.9\% AP$^{bb}$ and 2.7\% AP$^{mk}$, and surpasses UniFormer-S (hybrid) by 0.5\% AP$^{bb}$; MogaNet-B outperforms Swin-T and LITV2-M (Transformer) by 2.9\% AP$^{bb}$ and 1.2\% AP$^{mk}$ respectively.

\vspace{-1.0em}
\paragraph{Semantic segmentation on ADE20K.}
We then evaluate MogaNet for semantic segmentation tasks on the ADE20K dataset using Semantic FPN~\cite{cvpr2019semanticFPN} and UperNet~\cite{eccv2018upernet} following the evaluation schemes in \cite{liu2021swin, yu2022metaformer}. All experiments are implemented on MMSegmentation~\cite{mmseg2020} codebase, and the performance is measured by mIoU (single scale). Based on Semantic FPN, the models are fine-tuned for 80K iterations by the AdamW optimizer. In Table~\ref{tab:ade20k}, MogaNet-S consistently outperforms previous architectures, \textit{e.g.,} +6.6\% over Swin-T (Transformer), +1.5\% over Uniformer-S (hybrid). Based on UperNet, the models are fine-tuned 160K by AdamW optimizer. In Table~\ref{tab:ade20k}, the models with MogaNet-S improves backbones of Transformers (+3.1\% over Swin-T), hybrid architectures (+1.6\% over UniFormer-S), and modern ConvNets (+1.1\% over HorNet-T$_{7\times 7}$. Refer to Appendix~\ref{app:ade20k_settings} for more details.

% figure (interaction) & table (ablation)
\begin{figure}[hb]
\vspace{-1.25em}
\centering
\begin{minipage}{0.38\linewidth}
    \vspace{-1.25em}
    \centering
    \begin{table}[H]
    \vspace{-0.5em}
    \setlength{\tabcolsep}{0.7mm}
    \centering
\resizebox{\linewidth}{!}{
    \begin{tabular}{l|c}
    \toprule
Modules                       & Top-1     \\
                              & Acc (\%)  \\ \hline
ConvNeXt-T                    & 82.1      \\
Baseline                      & 82.2      \\ \hline
\rowcolor{gray94}Moga Block   & \bf{83.4} \\
$- \mathrm{FD}(\cdot)$        & 83.2      \\
$-$Multi-$\mathrm{DW}(\cdot)$ & 83.1      \\
$- \mathrm{Moga}(\cdot)$      & 82.7      \\
$- \mathrm{CA}(\cdot)$        & 82.9      \\
    \bottomrule
    \end{tabular}
    }
    \vspace{-0.5em}
    % \caption{\textbf{Ablation of the designed modules on ImageNet-1K}.
    % }
    \label{tab:ablation_small}
    \vspace{-1.0em}
\end{table}

    \vspace{3pt}
    % \vspace{-0.25em}
\end{minipage}
~\begin{minipage}{0.59\linewidth}
    \centering
    \includegraphics[width=1.0\linewidth,trim= 4 0 0 0,clip]{Figs/fig_ablation_interaction.pdf}
    \vspace{-2.25em}
\end{minipage}
    \caption{
    \textbf{Ablation of the proposed modules on ImageNet-1K.} \textbf{Left}: the table verifies each proposed module based on the baseline of MogaNet-S. \textbf{Right}: the figure plots distributions of the interaction strength $J^{(m)}$ and verifies that $\mathrm{Miga}(\cdot)$ contributes the most to learning multi-order interactions and better performance.
    }
    \label{fig:ablation_interaction}
\vspace{-1.5em}
\end{figure}

% figure: gradcam
\begin{figure}[hb]
    \vspace{-0.75em}
    \centering
    \includegraphics[width=1.0\linewidth,trim= 4 0 0 0,clip]{Figs/fig_analysis_gradcam.pdf}
    \vspace{-1.75em}
    \caption{
    \textbf{Grad-CAM activation maps of models trained on ImageNet-1K.} MogaNet-S shows similar activation maps as local attention architectures (Swin-T), which are located on the semantic targets. Unlike the results of previous ConvNets, which might activate some irrelevant parts, the activation maps of MogaNet-S are more gathered. See more visualizations in Appendix~\ref{app:gradcam}.
    }
    \label{fig:analysis_gradcam}
    \vspace{-1.25em}
\end{figure}

% table: COCO & ADE20K
\begin{figure*}[t!]
\vspace{-1.5em}
\begin{minipage}{0.555\linewidth}
\centering
    \begin{table}[t]
    \vspace{-0.25em}
    \setlength{\tabcolsep}{0.4mm}
    \centering
\resizebox{\linewidth}{!}{
\begin{tabular}{lllcccc}
    \toprule
    Architecture                                 & Data      & Method        & Param. & FLOPs & AP$^{b}$  & AP$^{m}$  \\
                                                 &           &               & (M)    & (G)   & (\%)      & (\%)      \\ \hline
    ResNet-101~\cite{he2016deep}                 & CVPR'2016 & RetinaNet     & 57     & 315   & 38.5      & -         \\
    PVT-S~\cite{iccv2021PVT}                     & ICCV'2021 & RetinaNet     & 34     & 226   & 40.4      & -         \\
    CMT-S~\cite{guo2021cmt}                      & CVPR'2022 & RetinaNet     & 45     & 231   & 44.3      & -         \\
    \rowcolor{gray94}\bf{MogaNet-S}              & Ours      & RetinaNet     & 35     & 253   & \bf{45.8} & -         \\ \hline
    RegNet-1.6G~\cite{cvpr2020regnet}            & CVPR'2020 & Mask R-CNN    & 29     & 204   & 38.9      & 35.7      \\
    PVT-T~\cite{iccv2021PVT}                     & ICCV'2021 & Mask R-CNN    & 33     & 208   & 36.7      & 35.1      \\
    \rowcolor{gray94}\bf{MogaNet-T}              & Ours      & Mask R-CNN    & 25     & 192   & \bf{42.6} & \bf{39.1} \\ \hline
    Swin-T~\cite{liu2021swin}                    & ICCV'2021 & Mask R-CNN    & 48     & 264   & 42.2      & 39.1      \\
    Uniformer-S~\cite{iclr2022uniformer}         & ICLR'2022 & Mask R-CNN    & 41     & 269   & 45.6      & 41.6      \\
    ConvNeXt-T~\cite{cvpr2022convnext}           & CVPR'2022 & Mask R-CNN    & 48     & 262   & 44.2      & 40.1      \\
    PVTV2-B2~\cite{cvmj2022PVTv2}                & CVMJ'2022 & Mask R-CNN    & 45     & 309   & 45.3      & 41.2      \\
    LITV2-S~\cite{nips2022hilo}                  & NIPS'2022 & Mask R-CNN    & 47     & 261   & 44.9      & 40.8      \\
    FocalNet-T~\cite{nips2022focalnet}           & NIPS'2022 & Mask R-CNN    & 49     & 267   & 45.9      & 41.3      \\
    \rowcolor{gray94}\bf{MogaNet-S}              & Ours      & Mask R-CNN    & 45     & 272   & \bf{46.7} & \bf{42.2} \\ \hline
    Swin-S~\cite{liu2021swin}                    & ICCV'2021 & Mask R-CNN    & 69     & 354   & 44.8      & 40.9      \\
    Focal-S~\cite{nips2021Focal}                 & NIPS'2021 & Mask R-CNN    & 71     & 401   & 47.4      & 42.8      \\
    ConvNeXt-S~\cite{cvpr2022convnext}           & CVPR'2022 & Mask R-CNN    & 70     & 348   & 45.4      & 41.8      \\
    HorNet-B$_{7\times 7}$~\cite{nips2022hornet} & NIPS'2022 & Mask R-CNN    & 68     & 322   & 47.4      & 42.3      \\
    \rowcolor{gray94}\bf{MogaNet-B}              & Ours      & Mask R-CNN    & 63     & 373   & \bf{47.9} & \bf{43.2} \\ \hline
    Swin-L$^\ddag$~\cite{liu2021swin}            & ICCV'2021 & Cascade Mask  & 253    & 1382  & 53.9      & 46.7      \\
    ConvNeXt-L$^\ddag$~\cite{cvpr2022convnext}   & CVPR'2022 & Cascade Mask  & 255    & 1354  & 54.8      & 47.6      \\
    RepLKNet-31L$^\ddag$~\cite{cvpr2022replknet} & CVPR'2022 & Cascade Mask  & 229    & 1321  & 53.9      & 46.5      \\
    HorNet-L$^\ddag$~\cite{nips2022hornet}       & NIPS'2022 & Cascade Mask  & 259    & 1399  & 56.0      & 48.6      \\
    \rowcolor{gray94}\bf{MogaNet-XL}$^\ddag$     & Ours      & Cascade Mask  & 238    & 1355  & \bf{56.2} & \bf{48.8} \\
    \bottomrule
    \end{tabular}
    }
    \vspace{-0.5em}
    \caption{\textbf{Object detection and instance segmentation} with RetinaNet ($1\times$), Mask R-CNN ($1\times$), and Cascade Mask R-CNN (multi-scale $3\times$) on COCO \textit{val2017}. $^\ddag$ indicates using ImageNet-21K pre-trained models. The FLOPs are measured at resolution $800\times 1280$.}
    \vspace{-1.0em}
    \label{tab:coco}
\end{table}

\end{minipage}
\begin{minipage}{0.45\linewidth}
\centering
    \begin{table}[t]
    \vspace{-0.25em}
    \setlength{\tabcolsep}{0.9mm}
    \centering
\resizebox{\linewidth}{!}{
\begin{tabular}{c|llcccc}
    \toprule
Method           & Architecture                                 & Date                   & Crop                      & Param.                & FLOPs                  & mIoU$^{ss}$                 \\
                 &                                              &                        & size                      & (M)                   & (G)                    & (\%)                        \\ \hline
                 & ResNet50~\cite{he2016deep}                   & CVPR'2016              & 512$^2$                   & 29                    & 183                    & 36.7                        \\
                 & PVT-S~\cite{iccv2021PVT}                     & ICCV'2021              & 512$^2$                   & 28                    & 161                    & 39.8                        \\
\small{Semantic} & Twins-S~\cite{nips2021Twins}                 & NIPS'2021              & 512$^2$                   & 28                    & 162                    & 44.3                        \\
FPN              & Swin-T~\cite{liu2021swin}                    & ICCV'2021              & 512$^2$                   & 32                    & 182                    & 41.5                        \\
(80K)            & Uniformer-S~\cite{iclr2022uniformer}         & ICLR'2022              & 512$^2$                   & 25                    & 247                    & 46.6                        \\
                 & LITV2-S~\cite{nips2022hilo}                  & NIPS'2022              & 512$^2$                   & 31                    & 179                    & 44.3                        \\
                 % & VAN-B2~\cite{guo2022van}                     & arXiv'2022             & 512$^2$                   & 30                    & 164                    & 46.7                        \\
                 & \cellcolor{gray94}\bf{MogaNet-S}             & \cellcolor{gray94}Ours & \cellcolor{gray94}512$^2$ & \cellcolor{gray94}29  & \cellcolor{gray94}189  & \cellcolor{gray94}\bf{47.7} \\ \hline
                 & DeiT-S~\cite{icml2021deit}                   & ICML'2021              & 512$^2$                   & 52                    & 1099                   & 44.0                        \\
                 & Swin-T~\cite{liu2021swin}                    & ICCV'2021              & 512$^2$                   & 60                    & 945                    & 46.1                        \\
                 & ConvNeXt-T~\cite{cvpr2022convnext}           & CVPR'2022              & 512$^2$                   & 60                    & 939                    & 46.7                        \\
                 & Twins-S~\cite{nips2021Twins}                 & NIPS'2021              & 512$^2$                   & 54                    & 901                    & 46.2                        \\
                 & UniFormer-S~\cite{iclr2022uniformer}         & ICLR'2022              & 512$^2$                   & 52                    & 1008                   & 47.6                        \\
                 & HorNet-T$_{7\times 7}$~\cite{nips2022hornet} & NIPS'2022              & 512$^2$                   & 52                    & 926                    & 48.1                        \\
                 & \cellcolor{gray94}\bf{MogaNet-S}             & \cellcolor{gray94}Ours & \cellcolor{gray94}512$^2$ & \cellcolor{gray94}55  & \cellcolor{gray94}946  & \cellcolor{gray94}\bf{49.2} \\ \cline{2-7} 
                 & Swin-S~\cite{liu2021swin}                    & ICCV'2021              & 512$^2$                   & 81                    & 1038                   & 48.1                        \\
                 & ConvNeXt-S~\cite{cvpr2022convnext}           & CVPR'2022              & 512$^2$                   & 82                    & 1027                   & 48.7                        \\
UperNet          & SLaK-S~\cite{Liu2022SLak}                    & ICLR'2023              & 512$^2$                   & 91                    & 1028                   & 49.4                        \\
(160K)           & \cellcolor{gray94}\bf{MogaNet-B}             & \cellcolor{gray94}Ours & \cellcolor{gray94}512$^2$ & \cellcolor{gray94}74  & \cellcolor{gray94}1050 & \cellcolor{gray94}\bf{50.1} \\ \cline{2-7} 
                 & Swin-B~\cite{liu2021swin}                    & ICCV'2021              & 512$^2$                   & 121                   & 1188                   & 49.7                        \\
                 & ConvNeXt-B~\cite{cvpr2022convnext}           & CVPR'2022              & 512$^2$                   & 122                   & 1170                   & 49.1                        \\
                 & RepLKNet-31B~\cite{cvpr2022replknet}         & CVPR'2022              & 512$^2$                   & 112                   & 1170                   & 49.9                        \\
                 & SLaK-B~\cite{Liu2022SLak}                    & ICLR'2023              & 512$^2$                   & 135                   & 1185                   & 50.2                        \\
                 & \cellcolor{gray94}\bf{MogaNet-L}             & \cellcolor{gray94}Ours & \cellcolor{gray94}512$^2$ & \cellcolor{gray94}113 & \cellcolor{gray94}1176 & \cellcolor{gray94}\bf{50.9} \\ \cline{2-7} 
                 & Swin-L$^\ddag$~\cite{liu2021swin}            & ICCV'2021              & 640$^2$                   & 234                   & 2468                   & 52.1                        \\
                 & ConvNeXt-L$^\ddag$~\cite{cvpr2022convnext}   & CVPR'2022              & 640$^2$                   & 245                   & 2458                   & 53.7                        \\
                 & RepLKNet-31L$^\ddag$~\cite{cvpr2022replknet} & CVPR'2022              & 640$^2$                   & 207                   & 2404                   & 52.4                        \\
                 & \cellcolor{gray94}\bf{MogaNet-XL}$^\ddag$    & \cellcolor{gray94}Ours & \cellcolor{gray94}640$^2$ & \cellcolor{gray94}214 & \cellcolor{gray94}2451 & \cellcolor{gray94}\bf{54.0} \\
    \bottomrule
    \end{tabular}
    }
    \vspace{-0.5em}
    \caption{\textbf{Semantic segmentation} with semantic FPN (80K) and UperNet (160K) on ADE20K validation set. $^\ddag$ indicates using IN-21K pre-trained models. The FLOPs are measured at $512\times 2048$ or $640\times 2560$ resolutions.}
    \vspace{-1.0em}
    \label{tab:ade20k}
\end{table}

% Semantic FPN
% ResNet-50 80k
% PVT-S 40k
% PVT.V2-S 40k
% Swin-T 80k
% Twins-S 80k
% Poolformer-M36 40k
% Uniformer-S 80k
% LIT.V2 80k
% VAN-B2 40k
% MogaNet-S 80k

\end{minipage}
\vspace{-1.5em}
\end{figure*}

\subsection{Ablation and Analysis}
\label{sec:exp_ablation}
We first ablate the spatial aggregation module, including \textbf{$\mathrm{FD}(\cdot)$} and $\mathrm{Moga}(\cdot)$, which contains the \textbf{gating branch} and the context branch with \textbf{multi-order DWConv layers}, and the \textbf{channel aggregation} module $\mathrm{CA}(\cdot)$.
% As verified in Table~\ref{tab:ablation}, the proposed modules yield +2.4\% performance gain to the baselines.
As verified in Table~\ref{tab:ablation} and Figure~\ref{fig:ablation_interaction} (left), all proposed modules yield improvements with a few costs. Appendix~\ref{app:ablation} provides more ablation studies.
Furthermore, we empirically verify the multi-order interactions in Figure~\ref{app:ablation_multiorder} (right) and visualize class activation maps (CAM) by Grad-CAM~\cite{cvpr2017grad} in comparison to existing models in Figure~\ref{fig:analysis_gradcam}.

\textbf{Multi-modalities} are unions of information with different forms, including but not limited to vision, text, audio, \etc~\cite{mm_book0, mm_survey0}. Early deep learning work led by Ngiam \etal ~\cite{mm_dl0} learned a fused representation for audio and video. The similar idea was also adopted across vision and text label~\cite{mm_dl0}, and across vision and language~\cite{mm_dl2}. A part of multimodal approaches focused on zero-shot learning, for instance, DiViSE~\cite{mm_zscls0} targeted mapping images on semantic space from which unseen category labels can be predicted. Socher \etal~\cite{mm_zscls1} trained a recognition model with similar ideas in which images were projected on the space of text corpus. \cite{mm_zscls2} shared the same design as DiViSE but was upgraded for a large and noisy dataset. Another set of works~\cite{mm_cls0, mm_cls1,  mm_cls2,  mm_cls3}, focused on increasing classification accuracy via multimodal training: in which~\cite{mm_cls0} and~\cite{mm_cls1} did a simple concatenation on multimodal embeddings; ~\cite{mm_cls2} proposed a gated unit to control the multimodal information flow in the network; ~\cite{mm_cls3} surveyed FastText~\cite{fasttext} with multiple fusion methods on text classification. Meanwhile, multimodal training was also wide-adopted in detection and segmentation~\cite{rcnn,maskrcnn, oneformer}% Started from R-CNN~\cite{rcnn}, series of works structured with shared backbone and multiple heads in order to train and predict class ids, detection boxes~\cite{rcnn, fpn} and object masks~\cite{maskrcnn, oneformer} 
in one shot. Another topic, VQA~\cite{mm_vqa0, mm_vqa1}, conducted cross-modal reasoning that transferred visual concepts into linguistic answers. Methods such as~\cite{mm_vqa2, mm_vqa3} extracted visual concepts into neural symbolics, and ~\cite{mm_vqa4, mm_vqa5} learned additional concept structures and hierarchies.

\textbf{Multimodal generative tasks} involve simultaneous representation learning and generation/synthesis~\cite{mm_survey1}, in which representation networks~\cite{ae, vae, gan, vqvae, wavenet, prnet} with contrastive loss~\cite{clip, cl, mm_cl0, mm_cl1, mm_cl2} played an essential role. Specifically, our model VD adopts VAEs~\cite{vae} and CLIP~\cite{clip} as the latent and context encoders, which are two critical modules for the network. VD also shares the common cross-modal concepts such as domain transfer~\cite{cgan, cyclegan} and joint representation learning~\cite{mm_dl1, mm_gm0, mm_gm1}.

\textbf{Diffusion models} (DM)~\cite{dm_early0, ddpm} consolidate large family of methods including VAEs~\cite{vae, vqvae, vqvae2}, Markov chains~\cite{mcm0, dm_early0, mcm1, mcm2}, and score matching models~\cite{scorem0, scorem1}, \etc. Differ from GAN-based\cite{gan, biggan, stylegan2} and flow-based models~\cite{flow0, flow1}, DM minimizes the lower-bounded likelihoods~\cite{ddpm, scorem0} in backward diffusion passes, rather than exact inverse in flow~\cite{flow0} or conduct adversarial training~\cite{gan}. Among the recent works, DDPM~\cite{ddpm} prompted $\epsilon$-prediction that established a connection between diffusion and score matching models via annealed Langevin dynamics sampling~\cite{dm_early1, scorem0}. DDPM also shows promising results on par with GANs in unconditional generation tasks. Another work, DDIM~\cite{ddim}, proposed an implicit generative model that yields deterministic samples from latent variables. Compared with DDPM, DDIM reduces the cost of sampling without losing quality. Regarding efficiency, FastDPM~\cite{dm_fast0} investigated continuous diffusion steps and generalized DDPM and DDIM with faster sampling schedules. Another work, ~\cite{dm_fast1}, replaced the original fixed sampling scheme with a learnable noise estimation that boosted both speed and quality. ~\cite{dm_fast2} introduced a hieratical structure with progressive increasing dimensions that expedite image generations for DM. Regarding quality, ~\cite{dm_beat_gan} compared GANs with DMs with exhaustive experiments and concluded that DMs outperformed GANs on many image generation tasks. Another work, VDM~\cite{dm_vdm}, introduced a family of DM models that reaches state-of-the-art performance on density estimation benchmarks. Diffwave~\cite{dm_diffwave} and WaveGrad~\cite{dm_wavegrad} show that DM also works well on audio. ~\cite{dm_improved_ddpm} improved DDPM with learnable noise scheduling and hybrid objective, achieving even better sampling quality. 
\cite{dm_morecontrol} introduced semantic diffusion guidance to allow image or language-conditioned synthesis with DDPM.

\textbf{Text-to-image generation}, nowadays a joint effort of multimodal and diffusion research, has drawn lots of attention. Among these recent works, GLIDE~\cite{glide} adopted pretrained language models and the cascaded diffusion structure for text-to-image generation. DALL-E2~\cite{dalle2}, a progressive version from DALL-E~\cite{dalle}, utilized CLIP model~\cite{clip} to generate text embedding and adopted the similar hieratical structure that made 256 text-guided images and then upscaled to 1024. Similarly, Imagen~\cite{imagen} explored multiple text encoders~\cite{bert, t5, clip} with conditional diffusion models and explores the trade-offs between content alignment and fidelity via various weight samplers. LDM~\cite{ldm} introduced a novel direction in which the model diffuses on VAE latent spaces instead of pixel spaces. Such design reduced the resource needed during inference time, and its latter version, SD, has proven to be equally effective in text-to-image generation.


% for ICLR submit
\if\submission\submissionFinal
    \vspace{-0.5em}
    \section{Conclusion}
    \label{sec:conclusion}
    \vspace{-0.25em}
    This paper introduces a new modern ConvNet architecture, named MogaNet, through the lens of multi-order game-theoretic interaction.
    Built upon the modern ConvNet framework, we present a compact Moga Block and channel aggregation module to force the network to emphasize the expressive but inherently overlooked interactions across spatial and channel perspectives.
    Extensive experiments verify the consistent superiority of MogaNet in terms of both performance and efficiency compared to popular ConvNets, ViTs, and hybrid architectures on various vision benchmarks.
% 
\else
% for arXiv
% \if\submission\submissionarXiv
    \section{Conclusion}
    \label{sec:conclusion}
    In this paper, we introduce a new modern ConvNet architecture named MogaNet through the lens of multi-order game-theoretic interaction.
    % We demonstrate that using a spatial aggregation block and a channel aggregation block results in stronger feature interactions of intermediate complexities efficiently, boosting the performance of ConvNet architecture substantially on diverse vision scenarios.
    Built upon the modern ConvNet framework, we present a compact Moga Block and channel aggregation module to \pl{force the network to emphasize the expressive but inherently overlooked interactions} across spatial and channel spaces.
    %
    Extensive experiments demonstrate the consistent superiority of MogaNet in terms of both accuracy and computational efficiency compared to popular ConvNets, ViTs, and hybrid architectures on various vision benchmarks.
    We hope our work can prompt people to perceive the importance of multi-order interaction in representation learning and to facilitate the development of efficient deep architecture design.
%
\fi


\section*{Acknowledgement}
This work was supported by the National Key R\&D Program of China (No. 2022ZD0115100), the National Natural Science Foundation of China Project (No. U21A20427), and Project (No. WU2022A009) from the Center of Synthetic Biology and Integrated Bioengineering of Westlake University.
This work was done when Zedong Wang and Zhiyuan Chen interned at Westlake University. We thank the AI Station of Westlake University for the support of GPUs. We also thank Mengzhao Chen, Zhangyang Gao, Jianzhu Guo, Fang Wu, and all anonymous reviewers for polishing the writing of the manuscript.



%%%%%%%%% REFERENCES
{
\bibliography{reference}
\bibliographystyle{iclr2024_conference}
}

%%%%%%%%% APPENDIX
\clearpage
\renewcommand\thefigure{A\arabic{figure}}
\renewcommand\thetable{A\arabic{table}}
\setcounter{table}{0}
\setcounter{figure}{0}

\newpage
\appendix

\section{Implementation Details}
\label{app:implement}
\subsection{Architecture Details}
\label{app:architecture}
The detailed architecture specifications of MogaNet are shown in Table~\ref{tab:app_architecture} and Figure~\ref{fig:app_moga_framework}, where an input image size of $224^2$ is assumed for all architectures. We rescale the groups of embedding dimensions the number of Moga Blocks for each stage corresponding to different models of varying magnitudes:
\romannumeral1) MogaNet-X-Tiny and MogaNet-Tiny with embedding dimensions of $\{32, 64, 96, 192\}$ and $\{32, 64, 128, 256\}$ has the competitive computational overload as recently proposed light-weight architectures~\cite{iclr2022mobilevit, cvpr2022MobileFormer, eccv2022edgeformer};
\romannumeral2) MogaNet-Small adopts embedding dimensions of $\{64, 128, 320, 512\}$ in comparison to small-scale architectures~\cite{liu2021swin, cvpr2022convnext};
\romannumeral3) MogaNet-Base with embedding dimensions of $\{64, 160, 320, 512\}$ in comparison to medium size architectures;
\romannumeral4) MogaNet-Large with embedding dimensions of $\{64, 160, 320, 640\}$ is designed for large-scale computer vision tasks.
The FLOPs are measured for image classification on ImageNet~\cite{cvpr2009imagenet} at resolution $224^2$, where a global average pooling (GAP) layer is applied to the output feature map of the last stage, followed by a linear classifier.

\begin{table}[h]
    \vspace{-0.5em}
    \setlength{\tabcolsep}{0.8mm}
    \centering
\resizebox{1.0\linewidth}{!}{
\begin{tabular}{ccc|ccccc}
\toprule
\multicolumn{1}{c|}{Stage}                    & \multicolumn{1}{c|}{Output}                                             & Layer         & \multicolumn{5}{c}{MogaNet}                                                                                                                          \\ \cline{4-8}
\multicolumn{1}{c|}{}                         & \multicolumn{1}{c|}{Size}                                               & Settings      & \multicolumn{1}{c|}{X-Tiny}              & \multicolumn{1}{c|}{Tiny}                & \multicolumn{1}{c|}{Small}               & \multicolumn{1}{c|}{Base}                & Large               \\ \hline
\multicolumn{1}{c|}{\multirow{4}{*}{Stage 1}} & \multicolumn{1}{c|}{\multirow{4}{*}{$\frac{H}{4}\times \frac{W}{4}$}}   & Stem          & \multicolumn{5}{c}{\begin{tabular}[c]{@{}c@{}}$\rm{Conv}_{3\times 3},~\rm{stride}~2, C/2$ \\ $\rm{Conv}_{3\times 3},~\rm{stride}~2, C$\end{tabular}} \\ \cline{3-8}
\multicolumn{1}{c|}{}                         & \multicolumn{1}{c|}{}                                                   & Embed. Dim.   & \multicolumn{1}{c|}{32}                  & \multicolumn{1}{c|}{32}                  & \multicolumn{1}{c|}{64}                  & \multicolumn{1}{c|}{64}                  & 64                  \\ \cline{3-8}
\multicolumn{1}{c|}{}                         & \multicolumn{1}{c|}{}                                                   & \# Moga Block & \multicolumn{1}{c|}{3}                   & \multicolumn{1}{c|}{3}                   & \multicolumn{1}{c|}{2}                   & \multicolumn{1}{c|}{4}                   & 4                   \\ \cline{3-8}
\multicolumn{1}{c|}{}                         & \multicolumn{1}{c|}{}                                                   & MLP Ratio     & \multicolumn{5}{c}{8}                                                                                                                                \\ \hline
\multicolumn{1}{c|}{\multirow{4}{*}{Stage 2}} & \multicolumn{1}{c|}{\multirow{4}{*}{$\frac{H}{8}\times \frac{W}{8}$}}   & Stem          & \multicolumn{5}{c}{$\rm{Conv}_{3\times 3}, \rm{stride}~2$}                                                                                           \\ \cline{3-8}
\multicolumn{1}{c|}{}                         & \multicolumn{1}{c|}{}                                                   & Embed. Dim.   & \multicolumn{1}{c|}{64}                  & \multicolumn{1}{c|}{64}                  & \multicolumn{1}{c|}{128}                 & \multicolumn{1}{c|}{160}                 & 160                 \\ \cline{3-8}
\multicolumn{1}{c|}{}                         & \multicolumn{1}{c|}{}                                                   & \# Moga Block & \multicolumn{1}{c|}{3}                   & \multicolumn{1}{c|}{3}                   & \multicolumn{1}{c|}{3}                   & \multicolumn{1}{c|}{6}                   & 6                   \\ \cline{3-8}
\multicolumn{1}{c|}{}                         & \multicolumn{1}{c|}{}                                                   & MLP Ratio     & \multicolumn{5}{c}{8}                                                                                                                                \\ \hline
\multicolumn{1}{c|}{\multirow{4}{*}{Stage 3}} & \multicolumn{1}{c|}{\multirow{4}{*}{$\frac{H}{16}\times \frac{W}{16}$}} & Stem          & \multicolumn{5}{c}{$\rm{Conv}_{3\times 3},~\rm{stride}~2$}                                                                                           \\ \cline{3-8}
\multicolumn{1}{c|}{}                         & \multicolumn{1}{c|}{}                                                   & Embed. Dim.   & \multicolumn{1}{c|}{96}                  & \multicolumn{1}{c|}{128}                 & \multicolumn{1}{c|}{320}                 & \multicolumn{1}{c|}{320}                 & 320                 \\ \cline{3-8}
\multicolumn{1}{c|}{}                         & \multicolumn{1}{c|}{}                                                   & \# Moga Block & \multicolumn{1}{c|}{10}                  & \multicolumn{1}{c|}{12}                  & \multicolumn{1}{c|}{12}                  & \multicolumn{1}{c|}{22}                  & 44                  \\ \cline{3-8}
\multicolumn{1}{c|}{}                         & \multicolumn{1}{c|}{}                                                   & MLP Ratio     & \multicolumn{5}{c}{4}                                                                                                                                \\ \hline
\multicolumn{1}{c|}{\multirow{4}{*}{Stage 4}} & \multicolumn{1}{c|}{\multirow{4}{*}{$\frac{H}{32}\times \frac{W}{32}$}} & Stem          & \multicolumn{5}{c}{$\rm{Conv}_{3\times 3},~\rm{stride}~2$}                                                                                           \\ \cline{3-8}
\multicolumn{1}{c|}{}                         & \multicolumn{1}{c|}{}                                                   & Embed. Dim.   & \multicolumn{1}{c|}{192}                 & \multicolumn{1}{c|}{256}                 & \multicolumn{1}{c|}{512}                 & \multicolumn{1}{c|}{512}                 & 640                 \\ \cline{3-8}
\multicolumn{1}{c|}{}                         & \multicolumn{1}{c|}{}                                                   & \# Moga Block & \multicolumn{1}{c|}{2}                   & \multicolumn{1}{c|}{2}                   & \multicolumn{1}{c|}{2}                   & \multicolumn{1}{c|}{3}                   & 4                   \\ \cline{3-8}
\multicolumn{1}{c|}{}                         & \multicolumn{1}{c|}{}                                                   & MLP Ratio     & \multicolumn{5}{c}{4}                                                                                                                                \\ \hline
\multicolumn{3}{c|}{Classifier}                                                                                                         & \multicolumn{5}{c}{Global Average Pooling, Linear}                                                                                                   \\ \hline
\multicolumn{3}{c|}{Parameters (M)}                                                                                                     & \multicolumn{1}{c|}{2.97}                & \multicolumn{1}{c|}{5.20}                & \multicolumn{1}{c|}{25.3}                & \multicolumn{1}{c|}{43.8}                & 82.5                \\ \hline
\multicolumn{3}{c|}{FLOPs (G)}                                                                                                          & \multicolumn{1}{c|}{0.80}                & \multicolumn{1}{c|}{1.10}                & \multicolumn{1}{c|}{4.97}                & \multicolumn{1}{c|}{9.93}                & 15.9               \\ \hline
\end{tabular}
    }
    \vspace{-0.5em}
    \caption{Architecture configurations of the variants of MogaNet.}
    \label{tab:app_architecture}
\end{table}


\subsection{Experimental Settings for ImageNet-1K}
\label{app:in1k_settings}
We perform regular ImageNet-1K training mostly following the training settings of DeiT~\cite{icml2021deit} and RSB A2~\cite{wightman2021rsb} in Table~\ref{tab:in1k_config}, which are widely adopted for Transformer and ConvNet models. For all models, the default input image resolution is $224^2$ for training from scratch. We adopt $256^2$ resolutions for lightweight experiments according to MobileViT~\cite{iclr2022mobilevit}. Taking training settings for the model with 25M or more parameters as the default, we train all MogaNet models for 300 epochs by AdamW \cite{iclr2019AdamW} optimizer using a batch size of 1024, a basic learning rate of $1\times 10^{-3}$, a weight decay of 0.05, and a Cosine learning rate scheduler \cite{loshchilov2016sgdr} with 5 epochs of linear warmup~\cite{devlin2018bert}.
As for augmentation and regularization techniques, we adopt most of the data augmentation and regularization strategies applied in DeiT training settings, including RandAugment \cite{cubuk2020randaugment}, Mixup~\cite{zhang2017mixup}, CutMix~\cite{yun2019cutmix}, random erasing~\cite{zhong2020random}, stochastic depth~\cite{eccv2016droppath}, and label smoothing \cite{cvpr2016inceptionv3}. Similar to ConvNeXt~\cite{cvpr2022convnext}, we do not apply Repeated augmentation \cite{cvpr2020repeat} and gradient clipping, which are designed for Transformers but do not enhance the performances of ConvNets, while using Exponential Moving Average (EMA)~\cite{siam1992ema} with the decay rate of 0.9999 by default. We also remove additional augmentation strategies~\cite{cvpr2019AutoAugment, eccv2022AutoMix, Li2021SAMix, 2022decouplemix} for ConvNets, \textit{e.g.,} ColorJitter~\cite{he2016deep} and AutoAugment~\cite{cvpr2019AutoAugment}.
Since lightweight architectures (3$\sim$10M parameters) tend to get under-fitted with strong augmentations and regularization, we adjust the training configurations for MogaNet-XT/T following \cite{iclr2022mobilevit, cvpr2022MobileFormer, eccv2022edgeformer}, including employing the weight decay of 0.03 and 0.04, Mixup with $\alpha$ of 0.1, and RandAugment of $7/0.5$ for MogaNet-XT/T. Since EMA is proposed to stabilize the training process of large models, we also remove it for MogaNet-XT/T as a fair comparison. An increasing degree of stochastic depth path augmentation is employed for larger models, \textit{i.e.}, 0.05, 0.1, 0.1, 0.2, 0.3 for MogaNet-XT, MogaNet-T, MogaNet-S, MogaNet-B, MogaNet-L, respectively. In evaluation, the top-1 accuracy using a single crop with a test crop ratio of 0.9 is reported as \cite{iccv2021t2t, yu2022metaformer, guo2022van}. All experiments are implemented on \texttt{OpenMixup}~\cite{li2022openmixup} codebase.

\begin{table}[h]
    \vspace{-0.5em}
    \setlength{\tabcolsep}{0.5mm}
    \centering
\resizebox{1.0\linewidth}{!}{
\begin{tabular}{l|c|c|ccccc}
    \toprule
    Configuration              & DeiT              & RSB               & \multicolumn{5}{c}{MogaNet}                             \\
                               &                   & A2                & X-Tiny          & Tiny    & Small   & Base    & Large   \\ \hline
    Input resolution           & 224$^2$           & 224$^2$           & \multicolumn{5}{c}{224$^2$}                   \\
    Epochs                     & 300               & 300               & \multicolumn{5}{c}{300}                       \\
    Batch size                 & 1024              & 2048              & \multicolumn{5}{c}{1024}                      \\
    Optimizer                  & AdamW             & LAMB              & \multicolumn{5}{c}{AdamW}                     \\
    AdamW $(\beta_1, \beta_2)$ & \small{$(0.9, 0.999)$} & -            & \multicolumn{5}{c}{$(0.9, 0.999)$}            \\
    Learning rate              & $1\times 10^{-3}$ & $5\times 10^{-3}$ & \multicolumn{5}{c}{$1\times 10^{-3}$}         \\
    Learning rate decay        & Cosine            & Cosine            & \multicolumn{5}{c}{Cosine}                    \\
    Weight decay               & 0.05              & 0.02              & 0.03            & 0.04    & 0.05    & 0.05    & 0.05    \\
    Warmup epochs              & 5                 & 5                 & \multicolumn{5}{c}{5}                                   \\
    Label smoothing $\epsilon$ & 0.1               & 0.1               & \multicolumn{5}{c}{0.1}                                 \\
    Stochastic Depth           & \cmarkg           & \cmarkg           & 0.05            & 0.1     & 0.1     & 0.2     & 0.3     \\
    Rand Augment               & 9/0.5             & 7/0.5             & 7/0.5           & 7/0.5   & 9/0.5   & 9/0.5   & 9/0.5   \\
    Repeated Aug               & \cmarkg           & \cmarkg           & \multicolumn{5}{c}{\xmarkg}                             \\
    Mixup $\alpha$             & 0.8               & 0.1               & 0.1             & 0.1     & 0.8     & 0.8     & 0.8     \\
    CutMix $\alpha$            & 1.0               & 1.0               & \multicolumn{5}{c}{1.0}                                 \\
    Erasing prob.              & 0.25              & \xmarkg           & \multicolumn{5}{c}{0.25}                                \\
    ColorJitter                & \xmarkg           & \xmarkg           & \multicolumn{5}{c}{\xmarkg}                             \\
    Gradient Clipping          & \cmarkg           & \xmarkg           & \multicolumn{5}{c}{\xmarkg}                             \\
    EMA decay                  & \cmarkg           & \xmarkg           & \xmarkg         & \xmarkg & \cmarkg & \cmarkg & \cmarkg \\
    Test crop ratio            & 0.875             & 0.95              & \multicolumn{5}{c}{0.90}                                \\
    \bottomrule
    \end{tabular}
    }
    \vspace{-0.5em}
    \caption{Hyper-parameters for ImageNet-1K training of DeiT, RSB A2, and MogaNet.}
    \label{tab:in1k_config}
\end{table}


% \begin{table}[H]
%     \setlength{\tabcolsep}{1.3mm}
%     \centering
%     \caption{Ingredients and hyper-parameters for ImageNet-1K training of DeiT, RSN A2, and MogaNet.}
%     \label{tab:in1k_config}
% \resizebox{\linewidth}{!}{
% \begin{tabular}{l|cccc}
%     \toprule
%     Configuration              & \multicolumn{4}{c}{MogaNet}                   \\
%                                & Tiny            & Small   & Base    & Large   \\ \hline
%     Input resolution           & 224$^2$/256$^2$ & 224$^2$ & 224$^2$ & 224$^2$ \\ \hline
%     Epochs                     & \multicolumn{4}{c}{300}                       \\
%     Batch size                 & \multicolumn{4}{c}{1024}                      \\
%     Optimizer                  & \multicolumn{4}{c}{AdamW}                     \\
%     AdamW $(\beta_1, \beta_2)$ & \multicolumn{4}{c}{$(0.9, 0.999)$}            \\
%     Weight decay               & 0.04            & 0.05    & 0.05    & 0.05    \\
%     Learning rate              & \multicolumn{4}{c}{$1\times 10^{-3}$}         \\
%     Learning rate schedule     & \multicolumn{4}{c}{Cosine}                    \\
%     Warmup epochs              & \multicolumn{4}{c}{5}                         \\
%     Label smoothing $\epsilon$ & \multicolumn{4}{c}{0.1}                       \\
%     Stochastic Depth           & 0.1             & 0.1     & 0.2     & 0.3     \\
%     Rand Augment               & 7/0.5           & 9/0.5   & 9/0.5   & 9/0.5   \\
%     Mixup $\alpha$             & 0.1             & 0.8     & 0.8     & 0.8     \\
%     CutMix $\alpha$            & \multicolumn{4}{c}{1.0}                       \\
%     Erasing prob.              & \multicolumn{4}{c}{0.25}                      \\
%     Repeated Augment           & \multicolumn{4}{c}{\xmarkg}                   \\
%     ColorJitter                & \multicolumn{4}{c}{\xmarkg}                   \\
%     Gradient Clipping          & \multicolumn{4}{c}{\xmarkg}                   \\
%     EMA                        & \multicolumn{4}{c}{\xmarkg}                   \\
%     Test crop ratio            & \multicolumn{4}{c}{0.9}                       \\
%     \bottomrule
%     \end{tabular}
%     }
% \end{table}


% fig: framework
\begin{figure*}[t]
    \vspace{-0.5em}
    \centering
    \includegraphics[width=0.92\textwidth]{Figs/fig_moga_framework.pdf}
    \vspace{-0.5em}
    \caption{\textbf{The overall framework of MogaNet.} Similar to~\cite{he2016deep, liu2021swin, yu2022metaformer, cvpr2022convnext}, MogaNet uses hierarchical architectures of 4 stages. The stage $i$ consists of an embedding stem and $N_{i}$ Moga Blocks, which contain spatial aggregation blocks and channel aggregation blocks.
    }
    \label{fig:app_moga_framework}
    \vspace{-1.0em}
\end{figure*}

% \section{Experimental Settings for Dense Prediction Tasks}
\subsection{Object Detection and Segmentation on COCO}
\label{app:coco_settings}
Following Swin~\cite{liu2021swin} and PoolFormer~\cite{yu2022metaformer}, we evaluate objection detection and instance segmentation tasks on COCO~\cite{2014MicrosoftCOCO} benchmark, which include 118K training images (\textit{train2017}) and 5K validation images (\textit{val2017}). We adopt Mask-RCNN~\cite{2017iccvmaskrcnn} as the standard detectors and use ImageNet-1K pre-trained weights as the initialization of the backbones. We employ AdamW~\cite{iclr2019AdamW} optimizer for training $1\times$ schedulers (12-epochs) with a basic learning rate of $1\times 10^{-4}$ and a batch size of 16. The shorter side of training images is resized to 800 pixels, and the longer side is resized to not more than 1333 pixels. We calculate the FLOPs of compared models at $800\times 1280$ resolutions. Experiments of COCO are implemented on \texttt{MMDetection}~\cite{mmdetection} codebase and run on 8 NVIDIA A100 GPUs.

\subsection{Semantic Segmentation on ADE20K}
\label{app:ade20k_settings}
We evaluate semantic segmentation on ADE20K~\cite{Zhou2018ADE20k} benchmark, which contains 20K training images and 2K validation images, covering 150 fine-grained semantic categories. 
We first adopt Semantic FPN~\cite{cvpr2019semanticFPN} following PoolFormer~\cite{yu2022metaformer} and Uniformer~\cite{iclr2022uniformer}, which train models for 80K iterations by AdamW~\cite{iclr2019AdamW} optimizer with a basic learning rate of $2\times 10^{-4}$ and a batch size of 16. Then, we utilize UperNet~\cite{eccv2018upernet} following Swin~\cite{liu2021swin}, which employs AdamW optimizer using a basic learning rate of $6\times 10^{-5}$, a weight decay of 0.01, a linear learning rate scheduler with a linear warmup of 1,500 iterations. We use ImageNet-1K pre-trained weights as the initialization of the backbones. The training images are resized to $512^2$ resolutions, and the shorter side of testing images is resized to 512 pixels. We calculate the FLOPs of models at $800\times 2048$ resolutions. The pre-trained weights on ImageNet-1K are used as the initialization of backbones. Experiments of ADE20K are implemented on \texttt{MMSegmentation}~\cite{mmseg2020} codebase and run on 8 NVIDIA A100 GPUs.



\section{Empirical Experiment Results}
\label{app:empirical}
\subsection{Multi-order Interaction}
\label{app:interaction}
In Sec.~\ref{sec:rep_bottleneck}, we interpret the learned representation of backbones by multi-order interaction~\cite{deng2021discovering}. The interaction complexity can be represented by the multi-order interaction $I^{(m)}(i,j)$, which measures the average interaction utility between variables $i,j$ on all contexts consisting of $m$ variables.
Empirically, the $m$-th order interaction $I^{(m)}(i,j)$ is defined to be the average interaction utility between image patches $i$ and $j$ on all contexts consisting of $m$ image patches. Note that $m$ denotes the order of contextual complexity of the interaction. Formally, given an input image $x$ with a set of $n$ patches $N = \{1,\dots,n\}$ (\textit{e.g.}, an image with $n$ pixels in total), the multi-order interaction $I^{(m)}(i,j)$ can be calculated as:
\begin{equation}
\begin{aligned}
    I^{(m)}(i,j) = \mathbb{E}_{S \subseteq N \setminus \{i,j\}, |S|=m}[\Delta f(i,j,S)],
\end{aligned}
    \label{eq:interaction}
\end{equation}
where $\Delta f(i,j,S) = f(S \cup \{i,j\}) - f(S \cup \{i\}) - f(S \cup \{j\}) + f(S)$. $f(S)$ indicates the score of output with patches in $N \setminus S$ kept unchanged but replaced with the baseline value~\cite{ancona2019explaining}, where the context $S\subseteq N$. Notice that the order $m$ reflects the contextual complexity of the interaction $I^{(m)}(i,j)$. For example, a low-order interaction (\textit{e.g.,} $m=0.05n$) means the relatively simple collaboration between variables $i,j$, while a high-order interaction (\textit{e.g.,} $m=0.05n$) corresponds to the complex collaboration. Then, we can measure the overall interaction complexity of deep neural networks (DNNs) by the relative interaction strength $J^{(m)}$ of the encoded $m$-th order interaction:
\begin{equation}
\begin{aligned}
    J^{(m)} = \frac{\mathbb{E}_{x \in \Omega}\mathbb{E}_{i,j}|I^{(m)}(i,j|x)|}{\mathbb{E}_{m^{'}}\mathbb{E}_{x \in \Omega}\mathbb{E}_{i,j}|I^{(m^{'})}(i,j|x)|},
\end{aligned}
    \label{eq:strength}
\end{equation}
where $\Omega$ is the set of all samples and $0\le m \ge n-2$. Note that $J^{(m)}$ is the average interaction strength over all possible patch pairs of the input samples and indicates the distribution (area under curve sums up to one) of the order of interactions of DNNs.
In Figure~\ref{fig:spatial_interaction}, we calculate the interaction strength $J^{(m)}$ with Eq.~\ref{eq:strength} for the models trained on ImageNet-1K using the official implementation{\footnote{\url{https://github.com/Nebularaid2000/bottleneck}}} provided by~\cite{deng2021discovering}. Specially, we use the image of $224\times 224$ resolution as the input and calculate $J^{(m)}$ on $14\times 14$ grids, \textit{i.e.,} $n=14\times 14$. And we set the model output as $f(x_S) = \log \frac{P(\hat y = y|x_S)}{1-P(\hat y = y|x_S)}$ given the masked sample $x_S$, where $y$ denotes the ground-truth label and $P(\hat y = y|x_S)$ denotes the probability of classifying the masked sample $x_S$ to the true category.


\subsection{Visualization of CAM}
\label{app:gradcam}
We further visualize more examples of Grad-CAM~\cite{cvpr2017grad} activation maps of MogaNet-S in comparison to Transformers, including DeiT-S~\cite{icml2021deit}, T2T-ViT-S~\cite{iccv2021t2t}, Twins-S~\cite{nips2021Twins}, and Swin~\cite{liu2021swin}, and ConvNets, including ResNet-50~\cite{he2016deep} and ConvNeXt-T~\cite{cvpr2022convnext}, on ImageNet-1K in Figure~\ref{fig:app_gradcam}. Due to the self-attention mechanism, the pure Transformers architectures (DeiT-S and T2T-ViT-S) show more refined activation maps than ConvNets, but they also activate some irrelevant parts. Combined with the design of local windows, local attention architectures (Twins-S and Swin-T) can locate the full semantic objects. Results of previous ConvNets can roughly localize the semantic target but might contain some background regions.
The activation parts of our proposed MogaNet-S are more similar to local attention architectures than previous ConvNets, which are more gathered on the semantic objects.


\section{More Ablation and Analysis Results}
\label{app:ablation}
In addition to Sec.~\ref{sec:exp_ablation}, we further conduct more ablation and analysis of our proposed MogaNet on ImageNet-1K. We adopt the same experimental settings as Sec.~\ref{tab:ablation}.

\subsection{Ablation of Activation Functions}
\label{app:ablation_gating}
We conduct the ablation of activation functions used in the proposed multi-order gated aggregation module on ImageNet-1K. Table~\ref{tab:ablation_gating} shows that using SiLU~\cite{elfwing2018sigmoid} activation for both branches achieves the best performance. Similar results were also found in Transformers, \textit{e.g.,} GLU variants with SiLU or GELU~\cite{hendrycks2016bridging} yield better performances than using Sigmoid or Tanh activation functions~\cite{Shazeer2020GLU, icml2022FLASH}. We guess that SiLU is the most suitable activation because it has both the property of Sigmoid (gating effects) and GELU (training friendly), which is defined as $x\cdot \mathrm{Sigmoid}(x)$.

\begin{table}[ht]
    % \vspace{-0.5em}
    \setlength{\tabcolsep}{1.1mm}
    \centering
\resizebox{0.60\linewidth}{!}{
\begin{tabular}{cl|ccc}
    \toprule
    Top-1  &                        & \multicolumn{3}{c}{Context branch}        \\
           & Acc (\%)               & None & GELU & \cellcolor{gray94}SiLU      \\ \hline
           & None                   & 76.3 & 76.7 & 76.7                        \\
    Gating & Sigmoid                & 76.8 & 77.0 & 76.9                        \\
    branch & GELU                   & 76.7 & 76.8 & 77.0                        \\
           & \cellcolor{gray94}SiLU & 76.9 & 77.1 & \cellcolor{gray94}\bf{77.2} \\
    \bottomrule
    \end{tabular}
    }
    \vspace{-0.5em}
    \caption{Ablation of activation functions for the gating and context branches in the $\mathrm{Moga}(\cdot)$ module.}
    \label{tab:ablation_gating}
    % \vspace{-1.0em}
\end{table}



\subsection{Ablation of Multi-order DWConv Layers}
\label{app:ablation_multiorder}
In addition to Sec.~\ref{sec:moga} and Sec.~\ref{sec:exp_ablation}, we also analyze the multi-order depth-wise convolution (DWConv) layers as the static regionality perception in the multi-order aggregation module $\mathrm{Moga}(\cdot)$ on ImageNet-1K. As shown in Table~\ref{tab:ablation_conv}, we analyze the channel configuration of three parallel dilated DWConv layers: $\mathrm{DW}_{5\times 5, d=1}$, $\mathrm{DW}_{5\times 5, d=2}$, and $\mathrm{DW}_{7\times 7, d=3}$ with the channels of $C_l$, $C_m$, $C_h$.
we first compare the performance of serial DWConv layers (\textit{e.g.,} $\mathrm{DW}_{5\times 5, d=1}$+$\mathrm{DW}_{7\times 7, d=3}$) and parallel DWConv layers. We find that the parallel design can achieve the same performance with fewer computational overloads because the DWConv kernel is equally applied to all channels. When we adopt three DWConv layers, the proposed parallel design reduces $C_l+C_h$ and $C_l+C_m$ times computations of $\mathrm{DW}_{5\times 5, d=2}$ and $\mathrm{DW}_{5\times 5, d=2}$ in comparison to the serial stack of these DWConv layers. Then, we empirically explore the optimal configuration of the three channels. We find that $C_l:$ $C_m:$ $C_h$ = 1: 3: 4 yields the best performance, which well balances the small, medium, and large DWConv kernels to learn low, middle, and high-order contextual representations. Similar conclusions are also found in relevant designs~\cite{nips2022hilo, nips2022iformer, nips2022hornet}, where global context aggregations take the majority (\textit{e.g.}, $\frac{1}{2} \sim \frac{3}{4}$ channels or context components). We also verify the parallel design with the optimal configuration based on MogaNet-S/B. Therefore, we can conclude that our proposed multi-order DWConv layers can efficiently learn multi-order contextual information for the context branch of $\mathrm{Moga}(\cdot)$.

\begin{table}[H]
    \setlength{\tabcolsep}{0.9mm}
    \centering
\resizebox{1.0\linewidth}{!}{
\begin{tabular}{l|ccc}
    \toprule
    Modules                                                                                  & Top-1     & Params. & FLOPs \\
                                                                                             & Acc (\%)  & (M)     & (G)   \\ \hline
    Baseline (+Gating branch)                                                                & 77.2      & 5.09    & 1.070 \\
    $\mathrm{DW}_{7\times 7}$                                                                & 77.4      & 5.14    & 1.094 \\
    $\mathrm{DW}_{5\times 5, d=1}+\mathrm{DW}_{7\times 7, d=3}$                              & 77.5      & 5.15    & 1.112 \\
    $\mathrm{DW}_{5\times 5, d=1}+\mathrm{DW}_{5\times 5, d=2}+\mathrm{DW}_{7\times 7, d=3}$ & 77.5      & 5.17    & 1.185 \\ \hline
    +Multi-order, $C_l: C_m: C_h=1: 0: 3$                                                    & 77.5      & 5.17    & 1.099 \\
    +Multi-order, $C_l: C_m: C_h=0: 1: 1$                                                    & 77.6      & 5.17    & 1.103 \\
    +Multi-order, $C_l: C_m: C_h=1: 6: 9$                                                    & 77.7      & 5.17    & 1.104 \\
    \rowcolor{gray94}+Multi-order, $C_l: C_m: C_h=1: 3: 4$                                   & \bf{77.8} & 5.17    & 1.102 \\
    \bottomrule
    \end{tabular}
    }
    \vspace{-0.5em}
    \caption{Ablation of multi-order DWConv layers in the proposed $\mathrm{Moga}(\cdot)$. The baseline adopts the MogaNet framework using the non-linear projection, $\mathrm{DW}_{5\times 5}$, and the SiLU gating branch as $\mathrm{SMixer}(\cdot)$ and using the vanilla MLP as $\mathrm{CMixer}(\cdot)$.}
    \label{tab:ablation_conv}
    % \vspace{-1.0em}
\end{table}



\subsection{Ablation of Normalization Layers}
\label{app:ablation_norm}
For most of ConvNets, BatchNorm~\cite{nips2015batchnorm} (BN) is considered an essential component to improve the convergence speed and prevent overfitting. However, BN might cause some instability~\cite{Wu2021PreciseBN} or harm the final performance of models~\cite{iclr2021characterizing, Brock2021NFNet}. Some recently proposed ConvNets~\cite{cvpr2022convnext, guo2022van} replace BN by LayerNorm~\cite{2016layernorm} (LN), which has been widely used in Transformers~\cite{iclr2021vit} and Metaformer architectures~\cite{yu2022metaformer}, achieving relatively good performances in various scenarios. Here, we conduct an ablation of normalization (Norm) layers in MogaNet on ImageNet-1K, as shown in Table~\ref{tab:ablation_norm}. As discussed in ConvNeXt~\cite{cvpr2022convnext}, the Norm layers used in each block (\textbf{within}) and after each stage (\textbf{after}) have different effects. Thus we study them separately. Table~\ref{tab:ablation_norm} shows that using BN in both places yields better performance than using LN (after) and BN (within), except MogaNet-T with $224^2$ resolutions, while using LN in both places performs the worst.
Consequently, we use BN as the default Norm layers in our proposed MogaNet for two reasons: (\romannumeral1) With pure convolution operators, the rule of combining convolution operations with BN within each stage is still useful for modern ConvNets. (\romannumeral2) Although using LN after each stage might help stabilize the training process of Transformers and hybrid models and might sometimes bring good performance for ConvNets, adopting BN after each stage in pure convolution models still yields better performance.
Moreover, we replace BN with precise BN~\cite{Wu2021PreciseBN} (pBN), which is an optimal alternative normalization strategy to BN. We find slight performance improvements (around 0.1\%), especially when MogaNet-S/B adopts the EMA strategy (by default), indicating that we can further improve MogaNet with advanced BN. As discussed in ConvNeXt, EMA might severely hurt the performances of models with BN. This phenomenon might be caused by the unstable and inaccurate BN statistics estimated by EMA in the vanilla BN with large models, which will deteriorate when using another EMA of model parameters. We solve this dilemma by exponentially increasing the EMA decay from 0.9 to 0.9999 during training as momentum-based contrastive learning methods~\cite{iccv2021dino, bao2021beit}, \textit{e.g.,} BYOL \cite{nips2020byol}. It can also be tackled by advanced BN variants~\cite{NIPS2017GhostBN, Wu2021PreciseBN}.

\begin{table}[ht]
    \vspace{-0.5em}
    \setlength{\tabcolsep}{1.5mm}
    \centering
\resizebox{0.575\linewidth}{!}{
\begin{tabular}{l|ccccc}
    \toprule
    Norm (after)    & Input   & LN   & LN        & BN                          & pBN                         \\
    Norm (within)   & size    & LN   & BN        & BN                          & pBN                         \\ \hline
    MogaNet-T       & 224$^2$ & 78.4 & \bf{79.1} & \cellcolor{gray94}79.0      & \bf{79.1}                   \\
    MogaNet-T       & 256$^2$ & 78.8 & 79.4      & \cellcolor{gray94}\bf{79.6} & \bf{79.6}                   \\
    MogaNet-S       & 224$^2$ & 82.5 & 83.2      & \cellcolor{gray94}\bf{83.3} & \bf{83.3}                   \\
    MogaNet-S (EMA) & 224$^2$ & 82.7 & 83.2      & 83.3                        & \cellcolor{gray94}\bf{83.4} \\
    MogaNet-B       & 224$^2$ & 83.4 & 83.9      & \cellcolor{gray94}84.1      & \bf{84.2}                   \\
    MogaNet-B (EMA) & 224$^2$ & 83.7 & 83.8      & 84.3                        & \cellcolor{gray94}\bf{84.4} \\
    \bottomrule
    \end{tabular}
    }
    \vspace{-0.5em}
    \caption{Ablation of normalization layers in MogaNet.}
    \label{tab:ablation_norm}
    \vspace{-0.5em}
\end{table}



\subsection{Refined Training Settings for Lightweight Models}
\label{app:advanced_tiny}
To explore the full power of lightweight models of our MogaNet, we refined the basic training settings for MogaNet-XT/T according to RSB A2~\cite{wightman2021rsb} and DeiT-III~\cite{eccv2022deit3}. Compared to the default setting as provided in Table~\ref{tab:in1k_config}, we only adjust the learning rate and the augmentation strategies for faster convergence while keeping other settings unchanged. As shown in Table~\ref{tab:advanced_tiny}, MogaNet-XT/T gain +0.4$\sim$0.6\% when use the large learning rate of $2\times 10^{-3}$ and 3-Augment~\cite{eccv2022deit3} without complex designs. Based on the advanced setting, MogaNet with $224^2$ input resolutions yields significant performance improvements against previous methods, \textit{e.g.,} MogaNet-T gains +3.5\% over DeiT-T~\cite{icml2021deit} and +1.2\% over Parc-Net-S~\cite{eccv2022edgeformer}.
Especially, MogaNet-T with $256^2$ resolutions achieves top-1 accuracy of 80.0\%, outperforming DeiT-S of 79.8\% reported in the original paper, while MogaNet-XT with $224^2$ resolutions outperforms DeiT-T under the refined training scheme by 1.2\% with only 3M parameters.

\begin{table}[t]
    \vspace{-1.0em}
    \setlength{\tabcolsep}{1.0mm}
    \centering
\resizebox{0.82\linewidth}{!}{
\begin{tabular}{lccccccc}
    \toprule
Architecture                                          & Input   & Learning          & Warmup & Rand    & 3-Augment            & EMA     & Top-1     \\
                                                      % & size    & rate              & epochs & Augment & \cite{eccv2022deit3} &         & Acc (\%)  \\ \hline
                                                      & size    & rate              & epochs & Augment &                      &         & Acc (\%)  \\ \hline
DeiT-T                                                & $224^2$ & $1\times 10^{-3}$ & 5      & 9/0.5   & \xmarkg              & \cmarkg & 72.2      \\
\rowcolor{gray94}DeiT-T                               & $224^2$ & $2\times 10^{-3}$ & 20     & \xmarkg & \cmarkg              & \xmarkg & 75.9      \\
ParC-Net-S                                            & $256^2$ & $1\times 10^{-3}$ & 5      & 9/0.5   & \xmarkg              & \cmarkg & 78.6      \\
\rowcolor{gray94}ParC-Net-S                           & $256^2$ & $2\times 10^{-3}$ & 20     & \xmarkg & \cmarkg              & \xmarkg & 78.8      \\ \hline
MogaNet-XT                                            & $224^2$ & $1\times 10^{-3}$ & 5      & 7/0.5   & \xmarkg              & \xmarkg & 76.5      \\
\rowcolor{gray94}MogaNet-XT                           & $224^2$ & $2\times 10^{-3}$ & 20     & \xmarkg & \cmarkg              & \xmarkg & 77.1      \\
MogaNet-XT                                            & $256^2$ & $1\times 10^{-3}$ & 5      & 7/0.5   & \xmarkg              & \xmarkg & 77.2      \\
\rowcolor{gray94}MogaNet-XT                           & $256^2$ & $2\times 10^{-3}$ & 20     & \xmarkg & \cmarkg              & \xmarkg & 77.6      \\
MogaNet-T                                             & $224^2$ & $1\times 10^{-3}$ & 5      & 7/0.5   & \xmarkg              & \xmarkg & 79.0      \\
\rowcolor{gray94}MogaNet-T                            & $224^2$ & $2\times 10^{-3}$ & 20     & \xmarkg & \cmarkg              & \xmarkg & 79.4      \\
MogaNet-T                                             & $256^2$ & $1\times 10^{-3}$ & 5      & 7/0.5   & \xmarkg              & \xmarkg & 79.6      \\
\rowcolor{gray94}MogaNet-T                            & $256^2$ & $2\times 10^{-3}$ & 20     & \xmarkg & \cmarkg              & \xmarkg & \bf{80.0} \\
    \bottomrule
    \end{tabular}
    }
    \vspace{-0.5em}
    \caption{Advanced training recipes for Lightweight models of MogaNet on ImageNet-1K.}
    \label{tab:advanced_tiny}
    \vspace{-0.5em}
\end{table}



\section{More Comparison Experiments}
\label{app:comparison}
In addition to Sec.~\ref{sec:exp_in1k}, we further provide comparison results for 100 and 300 epochs training on ImageNet-1K. As for 100-epoch training, we adopt the original RSB A3~\cite{wightman2021rsb} setting for all methods, which adopts LAMB \cite{iclr2020lamb} optimizer and a small training resolution of $160^2$. As for 300-epoch training, we report results of RSB A2 \cite{wightman2021rsb} for classical CNN or the original setting for Transformers or modern ConvNets. In Table~\ref{tab:in1k_app_rsb}, when compared with models of similar parameter size, our proposed MogaNet-XT/T/S/B achieves the best performance in both 100 and 300 epochs training. Results of 100-epoch training show that MogaNet has a faster convergence speed than previous architectures. For example, MogaNet-T outperforms EfficientNet-B0 and DeiT-T by 1.3\% and 7.6\%, MogaNet-S outperforms Swin-T and ConvNeXt-T by 4.1\% and 1.6\%, MogaNet-B outperforms Swin-S and ConvNeXt-S by 2.5\% and 1.0\%.

% figure: gradcam
\begin{figure*}[t]
    \vspace{-0.75em}
    \centering
    \includegraphics[width=0.98\linewidth]{Figs/fig_app_gradcam.pdf}
    \vspace{-0.5em}
    \caption{
    Visualization of Grad-CAM activation maps of the models trained on ImageNet-1K.}
    \label{fig:app_gradcam}
    % \vspace{-0.5em}
\end{figure*}

\begin{table*}[h]
    \setlength{\tabcolsep}{0.9mm}
    \centering
\resizebox{0.65\linewidth}{!}{
\begin{tabular}{llccccccc}
    \toprule
    Architecture                                      & Data       & Param. & \multicolumn{3}{c}{100-epoch} & \multicolumn{3}{c}{300-epoch} \\
                                                      &            & (M)    & Train   & Test   & Acc (\%)   & Train   & Test   & Acc (\%)   \\ \hline
    ResNet-18~\cite{he2016deep}                       & CVPR'2016  & 12     & 160     & 224    & 68.2       & 224     & 224    & 70.6       \\
    ResNet-34~\cite{he2016deep}                       & CVPR'2016  & 22     & 160     & 224    & 73.0       & 224     & 224    & 75.5       \\
    ResNet-50~\cite{he2016deep}                       & CVPR'2016  & 50     & 160     & 224    & 78.1       & 224     & 224    & 79.8       \\
    ResNet-101~\cite{he2016deep}                      & CVPR'2016  & 45     & 160     & 224    & 79.8       & 224     & 224    & 81.3       \\
    ResNeXt-50~\cite{xie2017aggregated}               & CVPR'2017  & 25     & 160     & 224    & 79.2       & 224     & 224    & 80.4       \\
    SE-ResNet-50~\cite{hu2018squeeze}                 & CVPR'2018  & 28     & 160     & 224    & 77.0       & 224     & 224    & 80.1       \\
    EfficientNet-B0~\cite{icml2019efficientnet}       & ICML'2019  & 5      & 160     & 224    & 73.0       & 224     & 224    & 77.1       \\
    EfficientNet-B1~\cite{icml2019efficientnet}       & ICML'2019  & 8      & 160     & 224    & 74.9       & 240     & 240    & 79.4       \\
    EfficientNet-B2~\cite{icml2019efficientnet}       & ICML'2019  & 9      & 192     & 256    & 77.5       & 260     & 260    & 80.1       \\
    EfficientNet-B3~\cite{icml2019efficientnet}       & ICML'2019  & 12     & 224     & 288    & 79.2       & 300     & 300    & 81.4       \\
    EfficientNet-B4~\cite{icml2019efficientnet}       & ICML'2019  & 19     & 320     & 380    & 81.2       & 380     & 380    & 82.4       \\
    RegNetY-800MF~\cite{cvpr2020regnet}               & CVPR'2020  & 6      & 160     & 224    & 73.8       & 224     & 224    & 76.3       \\
    RegNetY-4GF~\cite{cvpr2020regnet}                 & CVPR'2020  & 21     & 160     & 224    & 79.0       & 224     & 224    & 79.4       \\
    RegNetY-8GF~\cite{cvpr2020regnet}                 & CVPR'2020  & 39     & 160     & 224    & 81.1       & 224     & 224    & 79.9       \\
    RegNetY-16GF~\cite{cvpr2020regnet}                & CVPR'2020  & 84     & 160     & 224    & 81.7       & 224     & 224    & 80.4       \\
    EfficientNetV2-rw-S~\cite{icml2021EfficientNetV2} & ICML'2021  & 24     & 224     & 288    & 80.9       & 288     & 384    & 82.9       \\
    EfficientNetV2-rw-M~\cite{icml2021EfficientNetV2} & ICML'2021  & 53     & 256     & 384    & 82.3       & 320     & 384    & 81.9       \\ \hline
    DeiT-T~\cite{icml2021deit}                        & ICML'2021  & 6      & 160     & 224    & 66.7       & 224     & 224    & 72.2       \\
    DeiT-S~\cite{icml2021deit}                        & ICML'2021  & 22     & 160     & 224    & 73.8       & 224     & 224    & 79.8       \\
    DeiT-B~\cite{icml2021deit}                        & ICML'2021  & 87     & 160     & 224    & 76.0       & 224     & 224    & 81.8       \\
    Swin-T~\cite{liu2021swin}                         & ICCV'2021  & 28     & 160     & 224    & 77.0       & 224     & 224    & 81.3       \\
    Swin-S~\cite{liu2021swin}                         & ICCV'2021  & 50     & 160     & 224    & 79.7       & 224     & 224    & 83.0       \\ \hline
    ConvNeXt-T~\cite{cvpr2022convnext}                & CVPR'2022  & 29     & 160     & 224    & 79.5       & 224     & 224    & 82.1       \\
    ConvNeXt-S~\cite{cvpr2022convnext}                & CVPR'2022  & 50     & 160     & 224    & 81.2       & 224     & 224    & 83.1       \\
    VAN-B0~\cite{guo2022van}                          & arXiv'2022 & 4      & 160     & 224    & 72.6       & 224     & 224    & 75.8       \\
    VAN-B2~\cite{guo2022van}                          & arXiv'2022 & 27     & 160     & 224    & 81.0       & 224     & 224    & 82.8       \\
    VAN-B3~\cite{guo2022van}                          & arXiv'2022 & 45     & 160     & 224    & 81.9       & 224     & 224    & 83.9       \\
    \rowcolor{gray94}MogaNet-XT                       & Ours       & 3      & 160     & 224    & 72.8       & 224     & 224    & 76.5       \\
    \rowcolor{gray94}MogaNet-T                        & Ours       & 5      & 160     & 224    & 74.3       & 224     & 224    & 79.0       \\
    \rowcolor{gray94}MogaNet-S                        & Ours       & 25     & 160     & 224    & 81.1       & 224     & 224    & 83.4       \\
    \rowcolor{gray94}MogaNet-B                        & Ours       & 44     & 160     & 224    & 82.2       & 224     & 224    & 84.2       \\
    \bottomrule
    \end{tabular}
    }
    \vspace{-0.5em}
    \caption{ImageNet-1K classification performance of tiny to medium size models (5$\sim$50M) training 100 and 300 epochs. RSB A3~\cite{wightman2021rsb} setting is used for 100-epoch training of all methods. As for 300-epoch results, the RSB A2~\cite{wightman2021rsb} setting is used for ResNet, ResNeXt, SE-ResNet, EfficientNet, and EfficientNetV2, while other methods adopt settings in their original paper. }
    \label{tab:in1k_app_rsb}
\end{table*}



% \section{Discussions}
% \label{app:discussion}
% We discuss the relationship between our proposed MogaNet and prior models and summarize the limitations of MogaNet. We hope MogaNet can be a strong baseline applied to various vision tasks.

% \subsection{Comparison with Prior Art}
% \label{app:relationship}
% \paragraph{Motivations and main challenges.}
% The motivations for our work come from two aspects. As verified in previous work of hybrid architectures~\cite{nips2021coatnet, cvmj2022PVTv2, iclr2022uniformer, pinto2022impartial, iclr2022how}, which combine convolutions and self-attention mechanisms, only using advanced regionality preceptions or context aggregation modules is not enough to learn the optimal visual representation. According to empirical and theoretical analysis~\cite{hermann2020origins, naseer2021intriguing, iclr2022how, cvpr2022replknet, deng2021discovering}, we find that the gap between DNNs and human visions is DNNs have some unnatural bias (\textit{e.g.,} convolution operations are likely to be high-pass filters with the texture bias) and prefer to low-order or high-order interactions. Meanwhile, the efficiency of the model is also  emphasized with the boom of Transformer architectures~\cite{nips2020linformer, aaai2022LIT, iclr2022mobilevit, Lin2022SuperViT, icml2022FLASH, icml2022Flowformer}, which usually require quadratic complexity and might be time-consuming in fine-grained vision tasks.
% Therefore, we summarize the main challenge is how to leverage the advantages of regionality perceptions and context aggregations to learn more multi-order interactions efficiently.

% \paragraph{Relations to Efficient Transformers.}
% pass

% \paragraph{Relations to Modern ConvNets.}
% \cite{han2021demystifying} shows that local Transformer attention is equivalent to inhomogeneous dynamic depthwise convolution.

% \subsection{Limitations and Boarder Impacts}
% \label{app:limitation}
% pass



\end{document}
