\section{Experiments}
\label{sec:expriments}
To verify the effectiveness of our method, we conduct extensive experiments on ImageNet-1K (IN-1K)~\cite{cvpr2009imagenet} for image classification, COCO~\cite{2014MicrosoftCOCO} for object detection and instance segmentation, and ADE20K~\cite{Zhou2018ADE20k} for semantic segmentation. All experiments are implemented with PyTorch on Ubuntu workstations with NVIDIA A100 GPUs. \textbf{Bold} and \hl{gray} indicate the best performance and our models.

% % table: IN-1K Tiny (5M) & Small (25M)
% \begin{figure*}[t!]
% \vspace{-1.0em}
% \begin{minipage}{0.5\linewidth}
% \centering
%     \begin{table}[H]
    % \vspace{-0.25em}
    \setlength{\tabcolsep}{0.3mm}
    \centering
\resizebox{\linewidth}{!}{
\begin{tabular}{llccccc}
    \toprule
    Architecture                            & Date         & Type & Image   & Param. & FLOPs & Top-1     \\
                                            &              &      & Size    & (M)    & (G)   & Acc (\%)  \\ \hline
    ResNet-18                               & CVPR'2016    & C    & $224^2$ & 11.7   & 1.80  & 71.5      \\
    ShuffleNetV2~$2\times$                  & ECCV'2018    & C    & $224^2$ & 5.5    & 0.60  & 75.4      \\
    EfficientNet-B0                         & ICML'2019    & C    & $224^2$ & 5.3    & 0.39  & 77.1      \\
    % MobileNetV3~$1\times$                   & ICCV'2019    & C    & $224^2$ & 5.4    & 0.23  & 75.2      \\
    % RegNetY-400MF                          & CVPR'2020    & C    & $224^2$ & 5.3    & 0.40  & 74.1      \\
    RegNetY-800MF                           & CVPR'2020    & C    & $224^2$ & 6.3    & 0.80  & 76.3      \\
    DeiT-T$^\dag$                           & ICML'2021    & T    & $224^2$ & 5.7    & 1.08  & 74.1      \\
    % DeiT-2G                                & ICML'2021    & T    & $224^2$ & 13.2   & 1.90  & 75.1      \\
    PVT-T                                   & ICCV'2021    & T    & $224^2$ & 13.2   & 1.60  & 75.1      \\
    T2T-ViT-7                               & ICCV'2021    & T    & $224^2$ & 4.3    & 1.20  & 71.7      \\
    % T2T-ViT-12                             & ICCV'2021    & T    & $224^2$ & 6.9    & 1.80  & 76.5      \\
    ViT-C                                   & NIPS'2021    & T    & $224^2$ & 4.6    & 1.10  & 75.3      \\
    SReT-T$_{Distill}$                      & ECCV'2022    & T    & $224^2$ & 4.8    & 1.10  & 77.6      \\
    PiT-Ti                                  & ICCV'2021    & H    & $224^2$ & 4.9    & 0.70  & 74.6      \\
    LeViT-S                                 & ICCV'2021    & H    & $224^2$ & 7.8    & 0.31  & 76.6      \\
    CoaT-Lite-T                             & ICCV'2021    & H    & $224^2$ & 5.7    & 1.60  & 77.5      \\
    Swin-1G                                 & ICCV'2021    & H    & $224^2$ & 7.3    & 1.00  & 77.3      \\
    % Swin-2G                                 & ICCV'2021    & H    & $224^2$ & 12.8   & 2.00  & 79.3      \\
    % MobileViT-XS                            & ICLR'2022    & H    & $256^2$ & 2.3    & 1.73  & 74.8      \\
    MobileViT-S                             & ICLR'2022    & H    & $256^2$ & 5.6    & 4.02  & 78.4      \\
    % MobileFormer-151M                      & CVPR'2022    & H    & $224^2$ & 7.6    & 0.29  & 75.2      \\
    MobileFormer-294M                       & CVPR'2022    & H    & $224^2$ & 11.4   & 0.59  & 77.9      \\
    ConvNext-XT                             & CVPR'2022    & C    & $224^2$ & 7.4    & 0.60  & 77.5      \\
    VAN-B0                                  & CVMJ'2023   & C    & $224^2$ & 4.1    & 0.88  & 75.4      \\
    ParC-Net-S                              & ECCV'2022    & C    & $256^2$ & 5.0    & 3.48  & 78.6      \\
    \rowcolor{gray94}\bf{MogaNet-XT}        & Ours         & C    & $256^2$ & 3.0    & 1.04  & 77.2      \\
    \rowcolor{gray94}\bf{MogaNet-T}         & Ours         & C    & $224^2$ & 5.2    & 1.10  & 79.0      \\
    \rowcolor{gray94}\bf{MogaNet-T}$^\S$    & Ours         & C    & $256^2$ & 5.2    & 1.44  & \bf{80.0} \\
    \bottomrule
    \end{tabular}
    }
    \vspace{-1.0em}
    \caption{\textbf{IN-1K classification} with lightweight models. \small{$\S$} denotes the refined training scheme.
    % \small{$\dag$} and \small{$\S$} are RSB A2 and refined training schemes.
    }
    \label{tab:in1k_cls_tiny}
    % \vspace{-0.5em}
\end{table}


% \begin{table}[h]
%     \vspace{-0.5em}
%     \setlength{\tabcolsep}{0.7mm}
%     \centering
% \resizebox{\linewidth}{!}{
% \begin{tabular}{llcccc}
%     \toprule
%     Architecture                                     & Date         & Image   & Param. & FLOPs & Top-1     \\
%                                                      &              & Size    & (M)    & (G)   & Acc (\%)  \\ \hline
%     ResNet-18$^\dag$~\cite{he2016deep}               & CVPR'2016    & $224^2$ & 11.7   & 1.80  & 71.5      \\
%     ShuffleNetV2~$2\times$~\cite{eccv2018shufflenet} & ECCV'2018    & $224^2$ & 5.5    & 0.60  & 75.4      \\
%     EfficientNet-B0~\cite{icml2019efficientnet}      & ICML'2019    & $224^2$ & 5.3    & 0.39  & 77.1      \\
%     MobileNetV3~$1\times$~\cite{iccv2019mobilenetv3} & ICCV'2019    & $224^2$ & 5.4    & 0.23  & 75.2      \\
%     RegNetY-800M~\cite{cvpr2020regnet}               & CVPR'2020    & $224^2$ & 6.3    & 0.80  & 76.3      \\ \hline
%     DeiT-T~\cite{icml2021deit}                     & ICML'201     & $224^2$ & 5.7    & 1.08  & 72.2      \\
%     PVT-T~\cite{iccv2021PVT}                       & ICCV'2021    & $224^2$ & 13.2   & 1.60  & 75.1      \\
%     T2T-ViT-7~\cite{iccv2021t2t}                   & ICCV'2021    & $224^2$ & 4.3    & 1.20  & 71.7      \\
%     T2T-ViT-12~\cite{iccv2021t2t}                  & ICCV'2021    & $224^2$ & 6.9    & 1.80  & 76.5      \\
%     ViT-C~\cite{nips2021vitc}                      & NIPS'2021    & $224^2$ & 4.6    & 1.10  & 75.3      \\ \hline
%     PiT-Ti~\cite{iccv2021pit}                      & ICCV'2021    & $224^2$ & 4.9    & 0.70  & 74.6      \\
%     LeViT-S~\cite{iccv2021levit}                   & ICCV'2021    & $224^2$ & 7.8    & 0.31  & 76.6      \\
%     CoaT-Lite-T~\cite{iccv2021coat}                & ICCV'2021    & $224^2$ & 5.7    & 1.60  & 77.5      \\
%     MobileViT-XS~\cite{iclr2022mobilevit}          & ICLR'2022    & $256^2$ & 2.3    & 1.73  & 74.8      \\
%     MobileViT-S~\cite{iclr2022mobilevit}           & ICLR'2022    & $256^2$ & 5.6    & 4.02  & 78.4      \\
%     Mobile-Former-151M~\cite{cvpr2022MobileFormer} & CVPR'2022    & $224^2$ & 7.6    & 0.29  & 75.2      \\
%     Mobile-Former-294M~\cite{cvpr2022MobileFormer} & CVPR'2022    & $224^2$ & 11.4   & 0.59  & 77.9      \\ \hline
%     ConvNext-XT~\cite{cvpr2022convnext}            & CVPR'2022    & $224^2$ & 7.4    & 0.60  & 77.5      \\
%     VAN-B0~\cite{guo2022van}                       & arXiv'2022   & $224^2$ & 4.1    & 0.88  & 75.4      \\
%     ParC-Net-S~\cite{eccv2022edgeformer}           & ECCV'2022    & $256^2$ & 5.0    & 3.48  & 78.6      \\
%     \rowcolor{gray94}\bf{MogaNet-XT}               & Ours         & $224^2$ & 3.0    & 0.80  & 76.3      \\
%     \rowcolor{gray94}\bf{MogaNet-T}                & Ours         & $224^2$ & 5.2    & 1.10  & 79.0      \\
%     \rowcolor{gray94}\bf{MogaNet-T}                & Ours         & $256^2$ & 5.2    & 1.44  & \bf{79.6} \\
%     \bottomrule
%     \end{tabular}
%     }
%     \vspace{-0.5em}
%     \caption{ImageNet-1K classification performance of lightweight (around 5M Parameters) models.}
%     \label{tab:in1k_cls_tiny}
%     \vspace{-0.75em}
% \end{table}

% \end{minipage}
% \begin{minipage}{0.5\linewidth}
% \centering
%     \begin{table}[h]
    \vspace{-0.25em}
    \setlength{\tabcolsep}{0.8mm}
    \centering
\resizebox{\linewidth}{!}{
\begin{tabular}{llcccc}
    \toprule
    Architecture                                 & Date         & Image   & Param. & FLOPs & Top-1     \\
                                                 &              & Size    & (M)    & (G)   & Acc (\%)  \\ \hline
    ResNet-50$^\dag$~\cite{he2016deep}           & CVPR'2016    & $224^2$ & 26     & 4.1   & 80.4      \\
    EfficientNet-B4~\cite{icml2019efficientnet}  & ICML'2019    & $380^2$ & 19     & 4.2   & 82.9      \\
    RegNetY-4GF$^\dag$~\cite{cvpr2020regnet}     & CVPR'2020    & $224^2$ & 21     & 4.0   & 81.5      \\ \hline
    Deit-S~\cite{icml2021deit}                   & ICML'2021    & $224^2$ & 22     & 4.6   & 79.8      \\
    Swin-T~\cite{liu2021swin}                    & ICCV'2021    & $224^2$ & 28     & 4.5   & 81.3      \\
    T2T-ViT$_t$-14~\cite{iccv2021t2t}            & ICCV'2021    & $224^2$ & 22     & 6.1   & 81.7      \\
    CSWin-T~\cite{cvpr2022CSWin}                 & CVPR'2022    & $224^2$ & 23     & 4.3   & 82.8      \\
    SReT-S~\cite{eccv2022SReT}                   & ECCV'2022    & $224^2$ & 21     & 4.2   & 81.9      \\
    LITV2-S~\cite{nips2022hilo}                  & NIPS'2022    & $224^2$ & 28     & 3.7   & 82.0      \\ \hline
    CoaT-S~\cite{iccv2021coat}                   & ICCV'2021    & $224^2$ & 22     & 12.6  & 82.1      \\
    CoAtNet-0~\cite{nips2021coatnet}             & NIPS'2021    & $224^2$ & 25     & 4.2   & 82.7      \\
    ViTAE-S~\cite{nips2021vitae}                 & NIPS'2021    & $224^2$ & 24     & 5.6   & 82.0      \\
    UniFormer-S~\cite{iclr2022uniformer}         & ICLR'2022    & $224^2$ & 22     & 3.6   & 82.9      \\
    EfficientFormer-L3~\cite{nips2022EfficientFormer} & NIPS'2022    & $224^2$ & 31     & 3.9   & 82.4      \\ \hline
    ConvNeXt-T~\cite{cvpr2022convnext}           & CVPR'2022    & $224^2$ & 29     & 4.5   & 82.1      \\
    VAN-B2~\cite{guo2022van}                     & arXiv'2022   & $224^2$ & 27     & 5.0   & 82.8      \\
    SLaK-T~\cite{Liu2022SLak}                    & arXiv'2022   & $224^2$ & 30     & 5.0   & 82.5      \\
    HorNet-T$_{7\times 7}$~\cite{nips2022hornet} & NIPS'2022    & $224^2$ & 22     & 4.0   & 82.8      \\
    \rowcolor{gray94}\bf{MogaNet-S}              & Ours         & $224^2$ & 25     & 5.0   & \bf{83.4} \\
    \bottomrule
    \end{tabular}
    }
    \vspace{-0.5em}
    \caption{\textbf{ImageNet-1K classification} performance of small size (around 25M parameters) models.}
    \label{tab:in1k_cls_small}
    \vspace{-1.25em}
\end{table}

% \end{minipage}
% \vspace{-1.5em}
% \end{figure*}

\subsection{ImageNet Classification}
\label{sec:exp_in1k}
\paragraph{Settings.} 
For classification experiments on ImageNet-1K, we train MogaNet variants following the standard procedure \cite{icml2021deit, liu2021swin} for a fair comparison. Specifically, the models are trained for 300 epochs by AdamW~\cite{iclr2019AdamW} optimizer with $224^2$ or $256^2$ resolutions, a basic learning rate $lr$ = $1\times 10^{-3}$, 5 epochs warmup, and a Cosine scheduler~\cite{loshchilov2016sgdr}. See Appendix~\ref{app:in1k_settings} for implementation details.
We compare four typical architectures: (\romannumeral1) \textbf{Classical ConvNets} include ResNet, SENet, ShuffleNetV2, EfficientNet, MobileNetV3, and RegNet. (\romannumeral2) \textbf{Transformers} include DeiT, Swin, T2T-ViT, PVT, Focal, ViT-C, CSWin, SReT, and LiTV2. (\romannumeral3) \textbf{Hybrid architectures} of attention and convolution include PiT, LeViT, CoaT, BoTNet, ViTAE, Twins, CoAtNet, MobileViT, Uniformer, Mobile-Former, ParC-Net, EfficientFormer, and MaxViT. (\romannumeral4) \textbf{Modern ConvNets} include ConvNeXt, RepLKNet, FocalNet, VAN, SLak, and HorNet.
% We compare four types of popular network architectures: (\romannumeral1) \textbf{Classical CNN} includes ResNet~\cite{he2016deep}, SENet~\cite{hu2018squeeze}, ShuffleNetV2~\cite{eccv2018shufflenet}, EfficientNet~\cite{icml2019efficientnet}, MobileNetV3~\cite{iccv2019mobilenetv3}, and RegNet~\cite{cvpr2020regnet}. (\romannumeral2) \textbf{Transformer} includes DeiT~\cite{icml2021deit}, Swin~\cite{liu2021swin}, T2T-ViT~\cite{iccv2021t2t}, PVT~\cite{iccv2021PVT}, FocalNet~\cite{nips2021Focal}, ViT-C~\cite{nips2021vitc}, CSWin~\cite{cvpr2022CSWin}, SReT~\cite{eccv2022SReT}, and LiTV2~\cite{nips2022hilo}. (\romannumeral3) \textbf{Hybrid} Transformer and CNN architecture includes PiT~\cite{iccv2021pit}, LeViT~\cite{iccv2021levit}, CoaT~\cite{iccv2021coat}, BoTNet~\cite{cvpr2021botnet}, ViTAE~\cite{nips2021vitae}, Twins~\cite{nips2021Twins}, CoAtNet~\cite{nips2021coatnet}, MobileViT~\cite{iclr2022mobilevit}, Uniformer~\cite{iclr2022uniformer}, Mobile-Former~\cite{cvpr2022MobileFormer}, and ParC-Net~\cite{eccv2022edgeformer}. (\romannumeral3) \textbf{Post-ConvNet} includes ConvNeXt~\cite{cvpr2022convnext}, RepLKNet~\cite{cvpr2022replknet}, VAN~\cite{guo2022van}, SLak~\cite{Liu2022SLak}, and HorNet~\cite{nips2022hornet}.

% table: IN-1K Tiny (5M) & Small (25M)
\begin{table}[H]
    % \vspace{-0.25em}
    \setlength{\tabcolsep}{0.3mm}
    \centering
\resizebox{\linewidth}{!}{
\begin{tabular}{llccccc}
    \toprule
    Architecture                            & Date         & Type & Image   & Param. & FLOPs & Top-1     \\
                                            &              &      & Size    & (M)    & (G)   & Acc (\%)  \\ \hline
    ResNet-18                               & CVPR'2016    & C    & $224^2$ & 11.7   & 1.80  & 71.5      \\
    ShuffleNetV2~$2\times$                  & ECCV'2018    & C    & $224^2$ & 5.5    & 0.60  & 75.4      \\
    EfficientNet-B0                         & ICML'2019    & C    & $224^2$ & 5.3    & 0.39  & 77.1      \\
    % MobileNetV3~$1\times$                   & ICCV'2019    & C    & $224^2$ & 5.4    & 0.23  & 75.2      \\
    % RegNetY-400MF                          & CVPR'2020    & C    & $224^2$ & 5.3    & 0.40  & 74.1      \\
    RegNetY-800MF                           & CVPR'2020    & C    & $224^2$ & 6.3    & 0.80  & 76.3      \\
    DeiT-T$^\dag$                           & ICML'2021    & T    & $224^2$ & 5.7    & 1.08  & 74.1      \\
    % DeiT-2G                                & ICML'2021    & T    & $224^2$ & 13.2   & 1.90  & 75.1      \\
    PVT-T                                   & ICCV'2021    & T    & $224^2$ & 13.2   & 1.60  & 75.1      \\
    T2T-ViT-7                               & ICCV'2021    & T    & $224^2$ & 4.3    & 1.20  & 71.7      \\
    % T2T-ViT-12                             & ICCV'2021    & T    & $224^2$ & 6.9    & 1.80  & 76.5      \\
    ViT-C                                   & NIPS'2021    & T    & $224^2$ & 4.6    & 1.10  & 75.3      \\
    SReT-T$_{Distill}$                      & ECCV'2022    & T    & $224^2$ & 4.8    & 1.10  & 77.6      \\
    PiT-Ti                                  & ICCV'2021    & H    & $224^2$ & 4.9    & 0.70  & 74.6      \\
    LeViT-S                                 & ICCV'2021    & H    & $224^2$ & 7.8    & 0.31  & 76.6      \\
    CoaT-Lite-T                             & ICCV'2021    & H    & $224^2$ & 5.7    & 1.60  & 77.5      \\
    Swin-1G                                 & ICCV'2021    & H    & $224^2$ & 7.3    & 1.00  & 77.3      \\
    % Swin-2G                                 & ICCV'2021    & H    & $224^2$ & 12.8   & 2.00  & 79.3      \\
    % MobileViT-XS                            & ICLR'2022    & H    & $256^2$ & 2.3    & 1.73  & 74.8      \\
    MobileViT-S                             & ICLR'2022    & H    & $256^2$ & 5.6    & 4.02  & 78.4      \\
    % MobileFormer-151M                      & CVPR'2022    & H    & $224^2$ & 7.6    & 0.29  & 75.2      \\
    MobileFormer-294M                       & CVPR'2022    & H    & $224^2$ & 11.4   & 0.59  & 77.9      \\
    ConvNext-XT                             & CVPR'2022    & C    & $224^2$ & 7.4    & 0.60  & 77.5      \\
    VAN-B0                                  & CVMJ'2023   & C    & $224^2$ & 4.1    & 0.88  & 75.4      \\
    ParC-Net-S                              & ECCV'2022    & C    & $256^2$ & 5.0    & 3.48  & 78.6      \\
    \rowcolor{gray94}\bf{MogaNet-XT}        & Ours         & C    & $256^2$ & 3.0    & 1.04  & 77.2      \\
    \rowcolor{gray94}\bf{MogaNet-T}         & Ours         & C    & $224^2$ & 5.2    & 1.10  & 79.0      \\
    \rowcolor{gray94}\bf{MogaNet-T}$^\S$    & Ours         & C    & $256^2$ & 5.2    & 1.44  & \bf{80.0} \\
    \bottomrule
    \end{tabular}
    }
    \vspace{-1.0em}
    \caption{\textbf{IN-1K classification} with lightweight models. \small{$\S$} denotes the refined training scheme.
    % \small{$\dag$} and \small{$\S$} are RSB A2 and refined training schemes.
    }
    \label{tab:in1k_cls_tiny}
    % \vspace{-0.5em}
\end{table}


% \begin{table}[h]
%     \vspace{-0.5em}
%     \setlength{\tabcolsep}{0.7mm}
%     \centering
% \resizebox{\linewidth}{!}{
% \begin{tabular}{llcccc}
%     \toprule
%     Architecture                                     & Date         & Image   & Param. & FLOPs & Top-1     \\
%                                                      &              & Size    & (M)    & (G)   & Acc (\%)  \\ \hline
%     ResNet-18$^\dag$~\cite{he2016deep}               & CVPR'2016    & $224^2$ & 11.7   & 1.80  & 71.5      \\
%     ShuffleNetV2~$2\times$~\cite{eccv2018shufflenet} & ECCV'2018    & $224^2$ & 5.5    & 0.60  & 75.4      \\
%     EfficientNet-B0~\cite{icml2019efficientnet}      & ICML'2019    & $224^2$ & 5.3    & 0.39  & 77.1      \\
%     MobileNetV3~$1\times$~\cite{iccv2019mobilenetv3} & ICCV'2019    & $224^2$ & 5.4    & 0.23  & 75.2      \\
%     RegNetY-800M~\cite{cvpr2020regnet}               & CVPR'2020    & $224^2$ & 6.3    & 0.80  & 76.3      \\ \hline
%     DeiT-T~\cite{icml2021deit}                     & ICML'201     & $224^2$ & 5.7    & 1.08  & 72.2      \\
%     PVT-T~\cite{iccv2021PVT}                       & ICCV'2021    & $224^2$ & 13.2   & 1.60  & 75.1      \\
%     T2T-ViT-7~\cite{iccv2021t2t}                   & ICCV'2021    & $224^2$ & 4.3    & 1.20  & 71.7      \\
%     T2T-ViT-12~\cite{iccv2021t2t}                  & ICCV'2021    & $224^2$ & 6.9    & 1.80  & 76.5      \\
%     ViT-C~\cite{nips2021vitc}                      & NIPS'2021    & $224^2$ & 4.6    & 1.10  & 75.3      \\ \hline
%     PiT-Ti~\cite{iccv2021pit}                      & ICCV'2021    & $224^2$ & 4.9    & 0.70  & 74.6      \\
%     LeViT-S~\cite{iccv2021levit}                   & ICCV'2021    & $224^2$ & 7.8    & 0.31  & 76.6      \\
%     CoaT-Lite-T~\cite{iccv2021coat}                & ICCV'2021    & $224^2$ & 5.7    & 1.60  & 77.5      \\
%     MobileViT-XS~\cite{iclr2022mobilevit}          & ICLR'2022    & $256^2$ & 2.3    & 1.73  & 74.8      \\
%     MobileViT-S~\cite{iclr2022mobilevit}           & ICLR'2022    & $256^2$ & 5.6    & 4.02  & 78.4      \\
%     Mobile-Former-151M~\cite{cvpr2022MobileFormer} & CVPR'2022    & $224^2$ & 7.6    & 0.29  & 75.2      \\
%     Mobile-Former-294M~\cite{cvpr2022MobileFormer} & CVPR'2022    & $224^2$ & 11.4   & 0.59  & 77.9      \\ \hline
%     ConvNext-XT~\cite{cvpr2022convnext}            & CVPR'2022    & $224^2$ & 7.4    & 0.60  & 77.5      \\
%     VAN-B0~\cite{guo2022van}                       & arXiv'2022   & $224^2$ & 4.1    & 0.88  & 75.4      \\
%     ParC-Net-S~\cite{eccv2022edgeformer}           & ECCV'2022    & $256^2$ & 5.0    & 3.48  & 78.6      \\
%     \rowcolor{gray94}\bf{MogaNet-XT}               & Ours         & $224^2$ & 3.0    & 0.80  & 76.3      \\
%     \rowcolor{gray94}\bf{MogaNet-T}                & Ours         & $224^2$ & 5.2    & 1.10  & 79.0      \\
%     \rowcolor{gray94}\bf{MogaNet-T}                & Ours         & $256^2$ & 5.2    & 1.44  & \bf{79.6} \\
%     \bottomrule
%     \end{tabular}
%     }
%     \vspace{-0.5em}
%     \caption{ImageNet-1K classification performance of lightweight (around 5M Parameters) models.}
%     \label{tab:in1k_cls_tiny}
%     \vspace{-0.75em}
% \end{table}

\begin{table}[h]
    \vspace{-0.25em}
    \setlength{\tabcolsep}{0.8mm}
    \centering
\resizebox{\linewidth}{!}{
\begin{tabular}{llcccc}
    \toprule
    Architecture                                 & Date         & Image   & Param. & FLOPs & Top-1     \\
                                                 &              & Size    & (M)    & (G)   & Acc (\%)  \\ \hline
    ResNet-50$^\dag$~\cite{he2016deep}           & CVPR'2016    & $224^2$ & 26     & 4.1   & 80.4      \\
    EfficientNet-B4~\cite{icml2019efficientnet}  & ICML'2019    & $380^2$ & 19     & 4.2   & 82.9      \\
    RegNetY-4GF$^\dag$~\cite{cvpr2020regnet}     & CVPR'2020    & $224^2$ & 21     & 4.0   & 81.5      \\ \hline
    Deit-S~\cite{icml2021deit}                   & ICML'2021    & $224^2$ & 22     & 4.6   & 79.8      \\
    Swin-T~\cite{liu2021swin}                    & ICCV'2021    & $224^2$ & 28     & 4.5   & 81.3      \\
    T2T-ViT$_t$-14~\cite{iccv2021t2t}            & ICCV'2021    & $224^2$ & 22     & 6.1   & 81.7      \\
    CSWin-T~\cite{cvpr2022CSWin}                 & CVPR'2022    & $224^2$ & 23     & 4.3   & 82.8      \\
    SReT-S~\cite{eccv2022SReT}                   & ECCV'2022    & $224^2$ & 21     & 4.2   & 81.9      \\
    LITV2-S~\cite{nips2022hilo}                  & NIPS'2022    & $224^2$ & 28     & 3.7   & 82.0      \\ \hline
    CoaT-S~\cite{iccv2021coat}                   & ICCV'2021    & $224^2$ & 22     & 12.6  & 82.1      \\
    CoAtNet-0~\cite{nips2021coatnet}             & NIPS'2021    & $224^2$ & 25     & 4.2   & 82.7      \\
    ViTAE-S~\cite{nips2021vitae}                 & NIPS'2021    & $224^2$ & 24     & 5.6   & 82.0      \\
    UniFormer-S~\cite{iclr2022uniformer}         & ICLR'2022    & $224^2$ & 22     & 3.6   & 82.9      \\
    EfficientFormer-L3~\cite{nips2022EfficientFormer} & NIPS'2022    & $224^2$ & 31     & 3.9   & 82.4      \\ \hline
    ConvNeXt-T~\cite{cvpr2022convnext}           & CVPR'2022    & $224^2$ & 29     & 4.5   & 82.1      \\
    VAN-B2~\cite{guo2022van}                     & arXiv'2022   & $224^2$ & 27     & 5.0   & 82.8      \\
    SLaK-T~\cite{Liu2022SLak}                    & arXiv'2022   & $224^2$ & 30     & 5.0   & 82.5      \\
    HorNet-T$_{7\times 7}$~\cite{nips2022hornet} & NIPS'2022    & $224^2$ & 22     & 4.0   & 82.8      \\
    \rowcolor{gray94}\bf{MogaNet-S}              & Ours         & $224^2$ & 25     & 5.0   & \bf{83.4} \\
    \bottomrule
    \end{tabular}
    }
    \vspace{-0.5em}
    \caption{\textbf{ImageNet-1K classification} performance of small size (around 25M parameters) models.}
    \label{tab:in1k_cls_small}
    \vspace{-1.25em}
\end{table}


% table: IN-1K Base (40M) & Large (80M)
\begin{figure*}[t!]
\vspace{-1.5em}
\begin{minipage}{0.495\linewidth}
\centering
    \begin{table}[H]
    \vspace{-0.25em}
    \setlength{\tabcolsep}{0.9mm}
    \centering
\resizebox{\linewidth}{!}{
\begin{tabular}{llcccc}
    \toprule
    Architecture                                 & Date         & Image   & Param. & FLOPs & Top-1     \\
                                                 &              & Size    & (M)    & (G)   & Acc (\%)  \\ \hline
    ResNet-101$^\dag$~\cite{he2016deep}          & CVPR'2016    & $224^2$ & 45     & 7.9   & 81.5      \\
    EfficientNet-B6~\cite{icml2019efficientnet}  & ICML'2019    & $528^2$ & 43     & 19.0  & 84.0      \\
    RegNetY-8GF$^\dag$~\cite{cvpr2020regnet}     & CVPR'2020    & $224^2$ & 39     & 8.1   & 82.2      \\ \hline
    T2T-ViT-24~\cite{iccv2021t2t}                & ICCV'2021    & $224^2$ & 64     & 13.2  & 82.2      \\
    Swin-S~\cite{liu2021swin}                    & ICCV'2021    & $224^2$ & 50     & 8.7   & 83.0      \\
    Focal-S~\cite{nips2021Focal}                 & NIPS'2021    & $224^2$ & 51     & 9.1   & 83.6      \\
    CSWin-S~\cite{cvpr2022CSWin}                 & CVPR'2022    & $224^2$ & 35     & 6.9   & 83.6      \\
    LITV2-M~\cite{nips2022hilo}                  & NIPS'2022    & $224^2$ & 49     & 7.5   & 83.3      \\ \hline
    CoaT-M~\cite{iccv2021coat}                   & ICCV'2021    & $224^2$ & 45     & 9.8   & 83.6      \\
    Twins-SVT-B~\cite{nips2021Twins}             & NIPS'2021    & $224^2$ & 56     & 8.6   & 83.2      \\
    CoAtNet-1~\cite{nips2021coatnet}             & NIPS'2021    & $224^2$ & 42     & 8.4   & 83.3      \\
    UniFormer-B~\cite{iclr2022uniformer}         & ICLR'2022    & $224^2$ & 50     & 8.3   & 83.9      \\
    FAN-B-Hybrid~\cite{icml2022FAN}              & ICML'2022    & $224^2$ & 50     & 11.3  & 83.9      \\ \hline
    ConvNeXt-S~\cite{cvpr2022convnext}           & CVPR'2022    & $224^2$ & 50     & 8.7   & 83.1      \\
    FocalNet-S (LRF)~\cite{nips2022focalnet}     & NIPS'2022    & $224^2$ & 50     & 8.7   & 83.5      \\
    HorNet-S$_{7\times 7}$~\cite{nips2022hornet} & NIPS'2022    & $224^2$ & 50     & 8.8   & 84.0      \\
    VAN-B3~\cite{guo2022van}                     & arXiv'2022   & $224^2$ & 45     & 9.0   & 83.9      \\
    SLaK-S~\cite{Liu2022SLak}                    & ICLR'2023    & $224^2$ & 55     & 9.8   & 83.8      \\
    \rowcolor{gray94}\bf{MogaNet-B}              & Ours         & $224^2$ & 44     & 9.9   & \bf{84.3} \\
    \bottomrule
    \end{tabular}
    }
    \vspace{-0.5em}
    \caption{\textbf{ImageNet-1K classification} performance of medium size (around 45M parameters) models.}
    \label{tab:in1k_cls_base}
\end{table}

\end{minipage}
~\begin{minipage}{0.495\linewidth}
\centering
    \begin{table}[H]
    \vspace{-0.25em}
    \setlength{\tabcolsep}{0.9mm}
    \centering
\resizebox{\linewidth}{!}{
\begin{tabular}{llcccc}
    \toprule
Architecture                    & Date       & Image   & Param. & FLOPs & Top-1     \\
                                &            & Size    & (M)    & (G)   & Acc (\%)  \\ \hline
ResNet-152$^\dag$               & CVPR'2016  & $224^2$ & 60     & 11.6  & 82.0      \\
SE-ResNet-154$^\dag$            & CVPR'2018  & $224^2$ & 115    & 20.9  & 81.7      \\
RegNetY-16GF                    & CVPR'2020  & $224^2$ & 84     & 16.0  & 82.9      \\ \hline
DeiT-B                          & ICML'2021  & $224^2$ & 86     & 17.5  & 81.8      \\
Swin-B                          & ICCV'2021  & $224^2$ & 89     & 15.4  & 83.5      \\
Focal-B                         & NIPS'2021  & $224^2$ & 90     & 16.4  & 84.0      \\
CSWin-B                         & CVPR'2022  & $224^2$ & 78     & 15.0  & 84.2      \\
LITV2-B                         & NIPS'2022  & $224^2$ & 87     & 13.2  & 83.6      \\ \hline
BoTNet-T7                       & CVPR'2021  & $256^2$ & 79     & 19.3  & 84.2      \\
Twins-SVT-L                     & NIPS'2021  & $224^2$ & 99     & 15.1  & 83.7      \\
CoAtNet-2                       & NIPS'2021  & $224^2$ & 75     & 15.7  & 84.1      \\
FAN-B-Hybrid                    & ICML'2022  & $224^2$ & 77     & 16.9  & 84.3      \\ \hline
ConvNeXt-B                      & CVPR'2022  & $224^2$ & 89     & 15.4  & 83.8      \\
RepLKNet-31B                    & CVPR'2022  & $224^2$ & 79     & 15.3  & 83.5      \\
FocalNet-B (LRF)                & NIPS'2022  & $224^2$ & 89     & 15.4  & 83.9      \\
HorNet-B$_{7\times 7}$          & NIPS'2022  & $224^2$ & 87     & 15.6  & 84.3      \\
VAN-B4                          & CVMJ'2023  & $224^2$ & 60     & 12.2  & 84.2      \\
SLaK-B                          & ICLR'2023  & $224^2$ & 95     & 17.1  & 84.0      \\
\rowcolor{gray94}\bf{MogaNet-L} & Ours       & $224^2$ & 83     & 15.9  & \bf{84.7} \\
    \bottomrule
    \end{tabular}
    }
    \vspace{-0.5em}
    \caption{\textbf{ImageNet-1K classification} performance of large size (around 80M parameters) models.}
    \label{tab:in1k_cls_large}
\end{table}

\end{minipage}
\vspace{-1.25em}
\end{figure*}

\paragraph{Results.}
We compare the image classification performances of four widely adopted model sizes (around 5M, 25M, 45M, and 80M parameters).
As for lightweight models, Table~\ref{tab:in1k_cls_tiny} shows that MogaNet-XT/T significantly outperforms existing lightweight architectures. Using the default training settings, MogaNet-T achieves 79.0\% top-1 accuracy, which improves models with around 5M parameters by at least 1.1\% using $224^2$ resolutions, while outperforming the current best backbone ParC-Net-S by 1.0\% using $256^2$ resolutions. Meanwhile, MogaNet-XT also surpasses models with 3M parameters, \textit{e.g.,} +4.6\% and +1.5\% over T2T-ViT-7 and MobileViT-XS. Particularly, MogaNet-T$^{\S}$ achieves 80.0\% top-1 accuracy using $256^2$ resolutions and the refined settings, which adjusts $lr$ and replaces RandAugment~\cite{cubuk2020randaugment} with 3-Augment~\cite{eccv2022deit3} as detailed in Appendix~\ref{app:advanced_tiny}.
As for small-size models, Table~\ref{tab:in1k_cls_small} shows MogaNet-S achieves 83.4\% top-1 accuracy, which consistently outperforms Transformers, hybrid architectures, and ConvNets, \textit{e.g.,} +2.1\% and +1.2\% over Swin-T and ConvNeXt-T. 
As for 45M and 80M models, we summarize their performances in Table~\ref{tab:in1k_cls_base} and Table~\ref{tab:in1k_cls_large} and MogaNet-B/L still surpass the current state-of-the-art architectures, especially improving Swin-S/B and ConvNeXt-S/B by 1.2\%/ 1.1\% and 1.1\%/ 0.8\%. MogaNet also outperforms recently proposed modern ConvNets, \textit{e.g.,} +0.9\% over RepLKNet-31B and +0.2\%/ 0.3\% over HorNet-S/B$_{7\times 7}$.


\subsection{Dense Prediction Tasks}
\label{sec:exp_det_seg}
\paragraph{Object detection and segmentation on COCO.}
We evaluate MogaNet for object detection and segmentation tasks on the COCO dataset using Mask-RCNN~\cite{2017iccvmaskrcnn} as the detector. Following the training and evaluation settings in \cite{liu2021swin}, we fine-tune the models with AdamW optimizer for $1\times$ training schedule (12-epoch) on the COCO~\textit{train2017} and evaluate on the COCO~\textit{val2017}. We adopt MMDetection~\cite{mmdetection} as the codebase and measure the performance by the box mAP (AP$^{bb}$) and mask mAP (AP$^{mk}$). Refer to Appendix~\ref{app:coco_settings} for more details. Table~\ref{tab:coco} shows that models with MogaNet-T/S/B significantly outperform all previous backbones. Specifically, MogaNet-T gains 3.6\% AP$^{bb}$ and 4.6\% AP$^{mk}$ over ResNet-18; MogaNet-S outperforms Swin-T (Transformers) by 3.9\% AP$^{bb}$ and 2.7\% AP$^{mk}$, and surpasses UniFormer-S (hybrid) by 0.5\% AP$^{bb}$; MogaNet-B outperforms Swin-T and LITV2-M (Transformer) by 2.9\% AP$^{bb}$ and 1.2\% AP$^{mk}$ respectively.

\vspace{-1.0em}
\paragraph{Semantic segmentation on ADE20K.}
We then evaluate MogaNet for semantic segmentation tasks on the ADE20K dataset using Semantic FPN~\cite{cvpr2019semanticFPN} and UperNet~\cite{eccv2018upernet} following the evaluation schemes in \cite{liu2021swin, yu2022metaformer}. All experiments are implemented on MMSegmentation~\cite{mmseg2020} codebase, and the performance is measured by mIoU (single scale). Based on Semantic FPN, the models are fine-tuned for 80K iterations by the AdamW optimizer. In Table~\ref{tab:ade20k}, MogaNet-S consistently outperforms previous architectures, \textit{e.g.,} +6.6\% over Swin-T (Transformer), +1.5\% over Uniformer-S (hybrid). Based on UperNet, the models are fine-tuned 160K by AdamW optimizer. In Table~\ref{tab:ade20k}, the models with MogaNet-S improves backbones of Transformers (+3.1\% over Swin-T), hybrid architectures (+1.6\% over UniFormer-S), and modern ConvNets (+1.1\% over HorNet-T$_{7\times 7}$. Refer to Appendix~\ref{app:ade20k_settings} for more details.

% figure (interaction) & table (ablation)
\begin{figure}[hb]
\vspace{-1.25em}
\centering
\begin{minipage}{0.38\linewidth}
    \vspace{-1.25em}
    \centering
    \begin{table}[H]
    \vspace{-0.5em}
    \setlength{\tabcolsep}{0.7mm}
    \centering
\resizebox{\linewidth}{!}{
    \begin{tabular}{l|c}
    \toprule
Modules                       & Top-1     \\
                              & Acc (\%)  \\ \hline
ConvNeXt-T                    & 82.1      \\
Baseline                      & 82.2      \\ \hline
\rowcolor{gray94}Moga Block   & \bf{83.4} \\
$- \mathrm{FD}(\cdot)$        & 83.2      \\
$-$Multi-$\mathrm{DW}(\cdot)$ & 83.1      \\
$- \mathrm{Moga}(\cdot)$      & 82.7      \\
$- \mathrm{CA}(\cdot)$        & 82.9      \\
    \bottomrule
    \end{tabular}
    }
    \vspace{-0.5em}
    % \caption{\textbf{Ablation of the designed modules on ImageNet-1K}.
    % }
    \label{tab:ablation_small}
    \vspace{-1.0em}
\end{table}

    \vspace{3pt}
    % \vspace{-0.25em}
\end{minipage}
~\begin{minipage}{0.59\linewidth}
    \centering
    \includegraphics[width=1.0\linewidth,trim= 4 0 0 0,clip]{Figs/fig_ablation_interaction.pdf}
    \vspace{-2.25em}
\end{minipage}
    \caption{
    \textbf{Ablation of the proposed modules on ImageNet-1K.} \textbf{Left}: the table verifies each proposed module based on the baseline of MogaNet-S. \textbf{Right}: the figure plots distributions of the interaction strength $J^{(m)}$ and verifies that $\mathrm{Miga}(\cdot)$ contributes the most to learning multi-order interactions and better performance.
    }
    \label{fig:ablation_interaction}
\vspace{-1.5em}
\end{figure}

% figure: gradcam
\begin{figure}[hb]
    \vspace{-0.75em}
    \centering
    \includegraphics[width=1.0\linewidth,trim= 4 0 0 0,clip]{Figs/fig_analysis_gradcam.pdf}
    \vspace{-1.75em}
    \caption{
    \textbf{Grad-CAM activation maps of models trained on ImageNet-1K.} MogaNet-S shows similar activation maps as local attention architectures (Swin-T), which are located on the semantic targets. Unlike the results of previous ConvNets, which might activate some irrelevant parts, the activation maps of MogaNet-S are more gathered. See more visualizations in Appendix~\ref{app:gradcam}.
    }
    \label{fig:analysis_gradcam}
    \vspace{-1.25em}
\end{figure}

% table: COCO & ADE20K
\begin{figure*}[t!]
\vspace{-1.5em}
\begin{minipage}{0.555\linewidth}
\centering
    \begin{table}[t]
    \vspace{-0.25em}
    \setlength{\tabcolsep}{0.4mm}
    \centering
\resizebox{\linewidth}{!}{
\begin{tabular}{lllcccc}
    \toprule
    Architecture                                 & Data      & Method        & Param. & FLOPs & AP$^{b}$  & AP$^{m}$  \\
                                                 &           &               & (M)    & (G)   & (\%)      & (\%)      \\ \hline
    ResNet-101~\cite{he2016deep}                 & CVPR'2016 & RetinaNet     & 57     & 315   & 38.5      & -         \\
    PVT-S~\cite{iccv2021PVT}                     & ICCV'2021 & RetinaNet     & 34     & 226   & 40.4      & -         \\
    CMT-S~\cite{guo2021cmt}                      & CVPR'2022 & RetinaNet     & 45     & 231   & 44.3      & -         \\
    \rowcolor{gray94}\bf{MogaNet-S}              & Ours      & RetinaNet     & 35     & 253   & \bf{45.8} & -         \\ \hline
    RegNet-1.6G~\cite{cvpr2020regnet}            & CVPR'2020 & Mask R-CNN    & 29     & 204   & 38.9      & 35.7      \\
    PVT-T~\cite{iccv2021PVT}                     & ICCV'2021 & Mask R-CNN    & 33     & 208   & 36.7      & 35.1      \\
    \rowcolor{gray94}\bf{MogaNet-T}              & Ours      & Mask R-CNN    & 25     & 192   & \bf{42.6} & \bf{39.1} \\ \hline
    Swin-T~\cite{liu2021swin}                    & ICCV'2021 & Mask R-CNN    & 48     & 264   & 42.2      & 39.1      \\
    Uniformer-S~\cite{iclr2022uniformer}         & ICLR'2022 & Mask R-CNN    & 41     & 269   & 45.6      & 41.6      \\
    ConvNeXt-T~\cite{cvpr2022convnext}           & CVPR'2022 & Mask R-CNN    & 48     & 262   & 44.2      & 40.1      \\
    PVTV2-B2~\cite{cvmj2022PVTv2}                & CVMJ'2022 & Mask R-CNN    & 45     & 309   & 45.3      & 41.2      \\
    LITV2-S~\cite{nips2022hilo}                  & NIPS'2022 & Mask R-CNN    & 47     & 261   & 44.9      & 40.8      \\
    FocalNet-T~\cite{nips2022focalnet}           & NIPS'2022 & Mask R-CNN    & 49     & 267   & 45.9      & 41.3      \\
    \rowcolor{gray94}\bf{MogaNet-S}              & Ours      & Mask R-CNN    & 45     & 272   & \bf{46.7} & \bf{42.2} \\ \hline
    Swin-S~\cite{liu2021swin}                    & ICCV'2021 & Mask R-CNN    & 69     & 354   & 44.8      & 40.9      \\
    Focal-S~\cite{nips2021Focal}                 & NIPS'2021 & Mask R-CNN    & 71     & 401   & 47.4      & 42.8      \\
    ConvNeXt-S~\cite{cvpr2022convnext}           & CVPR'2022 & Mask R-CNN    & 70     & 348   & 45.4      & 41.8      \\
    HorNet-B$_{7\times 7}$~\cite{nips2022hornet} & NIPS'2022 & Mask R-CNN    & 68     & 322   & 47.4      & 42.3      \\
    \rowcolor{gray94}\bf{MogaNet-B}              & Ours      & Mask R-CNN    & 63     & 373   & \bf{47.9} & \bf{43.2} \\ \hline
    Swin-L$^\ddag$~\cite{liu2021swin}            & ICCV'2021 & Cascade Mask  & 253    & 1382  & 53.9      & 46.7      \\
    ConvNeXt-L$^\ddag$~\cite{cvpr2022convnext}   & CVPR'2022 & Cascade Mask  & 255    & 1354  & 54.8      & 47.6      \\
    RepLKNet-31L$^\ddag$~\cite{cvpr2022replknet} & CVPR'2022 & Cascade Mask  & 229    & 1321  & 53.9      & 46.5      \\
    HorNet-L$^\ddag$~\cite{nips2022hornet}       & NIPS'2022 & Cascade Mask  & 259    & 1399  & 56.0      & 48.6      \\
    \rowcolor{gray94}\bf{MogaNet-XL}$^\ddag$     & Ours      & Cascade Mask  & 238    & 1355  & \bf{56.2} & \bf{48.8} \\
    \bottomrule
    \end{tabular}
    }
    \vspace{-0.5em}
    \caption{\textbf{Object detection and instance segmentation} with RetinaNet ($1\times$), Mask R-CNN ($1\times$), and Cascade Mask R-CNN (multi-scale $3\times$) on COCO \textit{val2017}. $^\ddag$ indicates using ImageNet-21K pre-trained models. The FLOPs are measured at resolution $800\times 1280$.}
    \vspace{-1.0em}
    \label{tab:coco}
\end{table}

\end{minipage}
\begin{minipage}{0.45\linewidth}
\centering
    \begin{table}[t]
    \vspace{-0.25em}
    \setlength{\tabcolsep}{0.9mm}
    \centering
\resizebox{\linewidth}{!}{
\begin{tabular}{c|llcccc}
    \toprule
Method           & Architecture                                 & Date                   & Crop                      & Param.                & FLOPs                  & mIoU$^{ss}$                 \\
                 &                                              &                        & size                      & (M)                   & (G)                    & (\%)                        \\ \hline
                 & ResNet50~\cite{he2016deep}                   & CVPR'2016              & 512$^2$                   & 29                    & 183                    & 36.7                        \\
                 & PVT-S~\cite{iccv2021PVT}                     & ICCV'2021              & 512$^2$                   & 28                    & 161                    & 39.8                        \\
\small{Semantic} & Twins-S~\cite{nips2021Twins}                 & NIPS'2021              & 512$^2$                   & 28                    & 162                    & 44.3                        \\
FPN              & Swin-T~\cite{liu2021swin}                    & ICCV'2021              & 512$^2$                   & 32                    & 182                    & 41.5                        \\
(80K)            & Uniformer-S~\cite{iclr2022uniformer}         & ICLR'2022              & 512$^2$                   & 25                    & 247                    & 46.6                        \\
                 & LITV2-S~\cite{nips2022hilo}                  & NIPS'2022              & 512$^2$                   & 31                    & 179                    & 44.3                        \\
                 % & VAN-B2~\cite{guo2022van}                     & arXiv'2022             & 512$^2$                   & 30                    & 164                    & 46.7                        \\
                 & \cellcolor{gray94}\bf{MogaNet-S}             & \cellcolor{gray94}Ours & \cellcolor{gray94}512$^2$ & \cellcolor{gray94}29  & \cellcolor{gray94}189  & \cellcolor{gray94}\bf{47.7} \\ \hline
                 & DeiT-S~\cite{icml2021deit}                   & ICML'2021              & 512$^2$                   & 52                    & 1099                   & 44.0                        \\
                 & Swin-T~\cite{liu2021swin}                    & ICCV'2021              & 512$^2$                   & 60                    & 945                    & 46.1                        \\
                 & ConvNeXt-T~\cite{cvpr2022convnext}           & CVPR'2022              & 512$^2$                   & 60                    & 939                    & 46.7                        \\
                 & Twins-S~\cite{nips2021Twins}                 & NIPS'2021              & 512$^2$                   & 54                    & 901                    & 46.2                        \\
                 & UniFormer-S~\cite{iclr2022uniformer}         & ICLR'2022              & 512$^2$                   & 52                    & 1008                   & 47.6                        \\
                 & HorNet-T$_{7\times 7}$~\cite{nips2022hornet} & NIPS'2022              & 512$^2$                   & 52                    & 926                    & 48.1                        \\
                 & \cellcolor{gray94}\bf{MogaNet-S}             & \cellcolor{gray94}Ours & \cellcolor{gray94}512$^2$ & \cellcolor{gray94}55  & \cellcolor{gray94}946  & \cellcolor{gray94}\bf{49.2} \\ \cline{2-7} 
                 & Swin-S~\cite{liu2021swin}                    & ICCV'2021              & 512$^2$                   & 81                    & 1038                   & 48.1                        \\
                 & ConvNeXt-S~\cite{cvpr2022convnext}           & CVPR'2022              & 512$^2$                   & 82                    & 1027                   & 48.7                        \\
UperNet          & SLaK-S~\cite{Liu2022SLak}                    & ICLR'2023              & 512$^2$                   & 91                    & 1028                   & 49.4                        \\
(160K)           & \cellcolor{gray94}\bf{MogaNet-B}             & \cellcolor{gray94}Ours & \cellcolor{gray94}512$^2$ & \cellcolor{gray94}74  & \cellcolor{gray94}1050 & \cellcolor{gray94}\bf{50.1} \\ \cline{2-7} 
                 & Swin-B~\cite{liu2021swin}                    & ICCV'2021              & 512$^2$                   & 121                   & 1188                   & 49.7                        \\
                 & ConvNeXt-B~\cite{cvpr2022convnext}           & CVPR'2022              & 512$^2$                   & 122                   & 1170                   & 49.1                        \\
                 & RepLKNet-31B~\cite{cvpr2022replknet}         & CVPR'2022              & 512$^2$                   & 112                   & 1170                   & 49.9                        \\
                 & SLaK-B~\cite{Liu2022SLak}                    & ICLR'2023              & 512$^2$                   & 135                   & 1185                   & 50.2                        \\
                 & \cellcolor{gray94}\bf{MogaNet-L}             & \cellcolor{gray94}Ours & \cellcolor{gray94}512$^2$ & \cellcolor{gray94}113 & \cellcolor{gray94}1176 & \cellcolor{gray94}\bf{50.9} \\ \cline{2-7} 
                 & Swin-L$^\ddag$~\cite{liu2021swin}            & ICCV'2021              & 640$^2$                   & 234                   & 2468                   & 52.1                        \\
                 & ConvNeXt-L$^\ddag$~\cite{cvpr2022convnext}   & CVPR'2022              & 640$^2$                   & 245                   & 2458                   & 53.7                        \\
                 & RepLKNet-31L$^\ddag$~\cite{cvpr2022replknet} & CVPR'2022              & 640$^2$                   & 207                   & 2404                   & 52.4                        \\
                 & \cellcolor{gray94}\bf{MogaNet-XL}$^\ddag$    & \cellcolor{gray94}Ours & \cellcolor{gray94}640$^2$ & \cellcolor{gray94}214 & \cellcolor{gray94}2451 & \cellcolor{gray94}\bf{54.0} \\
    \bottomrule
    \end{tabular}
    }
    \vspace{-0.5em}
    \caption{\textbf{Semantic segmentation} with semantic FPN (80K) and UperNet (160K) on ADE20K validation set. $^\ddag$ indicates using IN-21K pre-trained models. The FLOPs are measured at $512\times 2048$ or $640\times 2560$ resolutions.}
    \vspace{-1.0em}
    \label{tab:ade20k}
\end{table}

% Semantic FPN
% ResNet-50 80k
% PVT-S 40k
% PVT.V2-S 40k
% Swin-T 80k
% Twins-S 80k
% Poolformer-M36 40k
% Uniformer-S 80k
% LIT.V2 80k
% VAN-B2 40k
% MogaNet-S 80k

\end{minipage}
\vspace{-1.5em}
\end{figure*}

\subsection{Ablation and Analysis}
\label{sec:exp_ablation}
We first ablate the spatial aggregation module, including \textbf{$\mathrm{FD}(\cdot)$} and $\mathrm{Moga}(\cdot)$, which contains the \textbf{gating branch} and the context branch with \textbf{multi-order DWConv layers}, and the \textbf{channel aggregation} module $\mathrm{CA}(\cdot)$.
% As verified in Table~\ref{tab:ablation}, the proposed modules yield +2.4\% performance gain to the baselines.
As verified in Table~\ref{tab:ablation} and Figure~\ref{fig:ablation_interaction} (left), all proposed modules yield improvements with a few costs. Appendix~\ref{app:ablation} provides more ablation studies.
Furthermore, we empirically verify the multi-order interactions in Figure~\ref{app:ablation_multiorder} (right) and visualize class activation maps (CAM) by Grad-CAM~\cite{cvpr2017grad} in comparison to existing models in Figure~\ref{fig:analysis_gradcam}.
