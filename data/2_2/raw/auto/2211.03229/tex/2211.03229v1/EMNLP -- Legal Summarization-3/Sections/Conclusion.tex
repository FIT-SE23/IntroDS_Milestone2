\section{Conclusion}
We presented an unsupervised graph-based model
for the summarization of long legal case decisions. Our
proposed approach incorporated diverse views of the document structure of legal cases and utilized a reweighting scheme to better select argumentative sentences. Our exploration of document structure demonstrates how using different types of document structure impacts summarization performance. Moreover, a document structure inspired reweighting scheme yields  performance gain on the CanLII case dataset.

\section*{Ethical Considerations}
The utilization of the generated summary results of legal documents remains important. Current extractive methods avoid the problem of generating hallucinated information \cite{kryscinski-etal-2020-evaluating,maynez-etal-2020-faithfulness}, which has been observed in abstractive methods that use large-scale pre-trained language models. The extracted sentences, however, may not capture the important contents of the legal documents.
Meanwhile, CanLII has taken measures to limit the disclosure of defendants' identities (such as blocking search indexing). Thus, using the dataset may need to be taken good care of and avoid impacting those efforts. 

\section*{Acknowledgement}
We thank the Pitt AI Fairness and Law group,
the Pitt PETAL group, and the
anonymous reviewers for their valuable feedback.
This research is supported by
the National Science Foundation under Grant IIS-2040490 and a gift from Amazon.

\section*{Limitations}
%We acknowledge several limitations of this work.
The dataset we used has a relatively small scale (1K) 
test set. Meanwhile, the automatic evaluation metrics may fall short compared to human evaluations, %in identifying the synonyms of legal terminologies, 
thus unfaithfully representing the final quality of generated summaries. Although lightweight, there is still a large performance gap between our unsupervised method and both the extractive oracles as well as abstractive models (Appendix \ref{sec:appendix_abstractive}), especially given the small-scale training data. There are more graph-based methods to aggregate information from the built graphs and we would like to explore and include more graph-based methods but selected the most relevant one in this work.
Moreover, our proposed reweighting paradigm heavily relied on observations about the structure of legal cases. Many other legal document types, such as bills and statutes, have inherently distinct structures. Our results also show the importance of finding the correct  structure and  weights, which can vary depending on the corpus. This will require more advanced methods to find the correct structure and weights for a dataset.

