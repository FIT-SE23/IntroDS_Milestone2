% This must be in the first 5 lines to tell arXiv to use pdfLaTeX, which is strongly recommended.
\pdfoutput=1
% In particular, the hyperref package requires pdfLaTeX in order to break URLs across lines.

\documentclass[11pt]{article}

% Remove the "review" option to generate the final version.
\usepackage[]{EMNLP2022}

% Standard package includes
\usepackage{times}
\usepackage{algorithm}
\usepackage{algpseudocode}
\usepackage{xcolor}
\usepackage{latexsym}
\usepackage{graphicx}
\usepackage{booktabs}
\usepackage{multirow}
\usepackage{mwe}
% \usepackage{graphicx}
\usepackage{subcaption}
% For proper rendering and hyphenation of words containing Latin characters (including in bib files)
\usepackage[T1]{fontenc}
% For Vietnamese characters
% \usepackage[T5]{fontenc}
% See https://www.latex-project.org/help/documentation/encguide.pdf for other character sets
\usepackage{hyperref}
% This assumes your files are encoded as UTF8
\usepackage[utf8]{inputenc}

% This is not strictly necessary, and may be commented out,
% but it will improve the layout of the manuscript,
% and will typically save some space.
\usepackage{microtype}
\definecolor{conclusiongreen}{RGB}{1, 113, 0} 
\definecolor{reasonblue}{RGB}{0, 118, 186}

% If the title and author information does not fit in the area allocated, uncomment the following
%
%\setlength\titlebox{<dim>}
%
% and set <dim> to something 5cm or larger.

\title{Computing and Exploiting Document Structure to Improve Unsupervised Extractive Summarization of Legal Case Decisions}

%\title{Document Structure-Aware  Unsupervised Extractive Summarization of Legal Documents}

% Author information can be set in various styles:
% For several authors from the same institution:
% \author{Author 1 \and ... \and Author n \\
%         Address line \\ ... \\ Address line}
% if the names do not fit well on one line use
%         Author 1 \\ {\bf Author 2} \\ ... \\ {\bf Author n} \\
% For authors from different institutions:
% \author{Author 1 \\ Address line \\  ... \\ Address line
%         \And  ... \And
%         Author n \\ Address line \\ ... \\ Address line}
% To start a seperate ``row'' of authors use \AND, as in
% \author{Author 1 \\ Address line \\  ... \\ Address line
%         \AND
%         Author 2 \\ Address line \\ ... \\ Address line \And
%         Author 3 \\ Address line \\ ... \\ Address line}

\author{Yang Zhong\\
  University of Pittsburgh \\
 Pittsburgh, PA, USA \\
  % Affiliation / Address line 3 \\
  \texttt{yaz118@pitt.edu} \\\And
 Diane Litman \\
  University of Pittsburgh \\
 Pittsburgh, PA, USA \\
  \texttt{dlitman@pitt.edu} \\}

\begin{document}
\maketitle
\begin{abstract}
%Automatic summarization of legal documents is an important and practical challenge. 
Though many %domain-independent and domain-specific summarization 
algorithms can be used to automatically summarize legal case decisions,  most 
fail to incorporate domain knowledge about how important sentences in a legal decision relate to a representation of its document structure. For example, analysis of a  legal case summarization dataset demonstrates that sentences serving different  types of argumentative roles in the decision appear in different sections %or paragraphs 
of the document. 
In this work, we propose an unsupervised graph-based ranking model that uses a reweighting algorithm to exploit properties of the document structure of legal case decisions.  We also explore the impact of using different methods to compute the  document structure. Results on the Canadian Legal Case Law %(CANLII)
dataset  show that our proposed method outperforms several strong %summarization 
baselines. % in terms of ROUGE scores and BERTScore. 

%and American Bill dataset (BillSum) 

%Automatic summarization of legal documents is an important
%and practical challenge. Though many domain-independent and domain-specific summarization algorithms can be used for this purpose,  most of these algorithms fail to incorporate the domain knowledge of what should be the important information presented in a legal case document. Our analysis on a legal case summarziation dataset demonstrates that legal document writers tend to organize the \textit{argumentative} information into different sections or paragraphs.  In this work, we propose an unsupervised graph-based ranking model that exploits the document structure of the legal texts. Results on the Canadian Legal Case Law (CANLII) dataset  and American Bill dataset (BillSum) show that our proposed method outperforms several strong baselines in terms of ROUGE scores and BERTScore. 
\end{abstract}
%----------------------------------------------------------------------
%%% INTRODUCTION
%----------------------------------------------------------------------
% !TEX root = ../Main.tex


\Acp{BPM} have a long and rich history in optimization, going back at least to the introduction of \acl{MD} by Nemirovski \& Yudin \citep{NY83}.
In plain terms, \acp{BPM} are first-order (constrained) optimization algorithms that forego Euclidean projections in favor of a more sophisticated ``prox-mapping'' that minimizes a certain distance-like functional known as the Bregman divergence \citep{NY83,CT93,Bre67,Kiw97}.
When this Bregman divergence is the Euclidean distance squared, one recovers the standard projection-based methods;
other than that, depending on the problem's feasible region, different Bregman setups lead to a diverse collection of algorithms,
from exponentiated gradient descent on the simplex \citep{NY83,BecTeb03,ACBFS02},
to matrix multiplicative weights on the positive-semidefinite cone \cite{TRW05,KSST12},
variants of Karmarkar's affine scaling algorithm for linear programs \cite{VMF86},
etc.

One of the most appealing features of \acp{BPM} is that they achieve almost dimension-free convergence rates in problems with a convex structure and a favorable geometry \textendash\ such as the $L^{1}$ ball, the spectraplex, second-order cones, etc. \cite{Bub15,Nes09,BecTeb03}.
This is owed to a delicate interplay between the algorithms' non-Euclidean update scheme and the global geometry of the problem's domain.
However, these (almost) dimension-free guarantees also come with some strings attached:
they do not concern the sequence of iterates generated by the method, but only its time average
\revise{(or, through the same, ``regret-based'' analysis, the method's ``best iterate'')};
in this way, the best guarantee that can be achieved after $\run$ iterations is $\bigoh(1/\run)$.

In terms of oracle complexity, this is sufficient for problems that are not strongly convex\,/\,strongly monotone, but if one targets finer, geometric convergence rates,
\revise{the inherent averaging involved in regret-based guarantees is hard to compensate.}
And, on the other extreme, if the problem is not convex\,/\,monotone to begin with, iterate averaging does not provide any quantifiable benefits whatsoever, so it becomes crucial to study the actual trajectory of the method.


%----------------------------------------------------------------------
%%% CONTRIBS
%----------------------------------------------------------------------
\para{Our contributions}

Our paper seeks to quantify the last-iterate convergence rate of \aclp{BPM} as a function of the Bregman divergence defining the method and the local geometry that it induces.
To treat this question in as general a manner as possible, we focus on \ac{VI} problems of the form
\begin{equation}
\label{eq:VI}
\tag{VI}
\text{Find $\sol\in\points$ such that}
	\;\;
	\braket{\vecfield(\sol)}{\point - \sol}
	\geq 0
	\;\;
	\text{for all $\point\in\points$},
\end{equation}
where $\points$ is a closed convex subset of a finite-dimensional normed space $\pspace$, and $\vecfield \from \points \to \dspace$ is a (possibly non-monotone) single-valued operator on $\points$ with values in $\dspace$, the dual of $\pspace$.
This problem is a staple of many areas of mathematical programming, game theory and data science, as it provides a template for ``optimization beyond minimization'' \textendash\ \ie for problems where finding an optimal solution does not necessarily involve minimizing a loss function.
In particular, in addition to standard minimization problems \textendash\ which are recovered when $\vecfield = \nabla\obj$ for some smooth function $\obj$ \textendash\ the general formulation \eqref{eq:VI} includes saddle-point problems, games, complementarity problems, etc.;
for an introduction, see \cite{FP03} and references therein.

In this broad context, we examine the rate of convergence of a wide class of \aclp{BPM} to local solutions of \eqref{eq:VI} that satisfy a \acl{SOS} condition.
Specifically, the class of algorithms we consider includes as special cases
\begin{enumerate*}
[(\itshape i\hspace*{1pt}\upshape)]
\item
the original \acf{MD} algorithm of \cite{NY83};
\item
the \acf{MP} method of Nemirovski \cite{Nem04} \textendash\ which has the same update structure as the Bregman-based algorithm of \cite{AT05} and contains as a special case the \acf{EG} algorithm of \cite{Kor76};
\item
the so-called \acf{OMD} method of \cite{RS13-NIPS} \textendash\ itself a Bregman analogue of the modified Arrow-Hurwicz algorithm of \cite{Pop80};
\end{enumerate*}
etc.

Our first finding is a crisp characterization of last-iterate convergence rate of \acp{BPM} in terms of the local geometry induced by the underlying Bregman function near a given solution of \eqref{eq:VI}.
We make this dependence precise via the notion of the \emph{Legendre exponent}, a regularity measure for Bregman methods due to \cite{AIMM21}, which can roughly be described as the logarithmic ratio of the volume of a Euclidean ball to that of a Bregman ball of the same radius.
For example, Euclidean methods have a Legendre exponent of $\legexp = 0$ and they converge at a linear rate;
entropic methods have a Legendre exponent of $\legexp = 1/2$ at boundary points, and they converge at a rate of $\bigoh(\run^{-1})$;
more generally,
as we show in \cref{thm:general}, methods with a Legendre exponent $\legexp>0$ converge at a rate of $\bigoh(\run^{1-1/\legexp})$.
\PM{We need to fix this: the $1-1/\legexp$ exponent is not consistent with the $\bigoh(1/\run)$ expression.}
\WA{I don't see the issues, yes this expression is not well-defined for $\legexp = 0$ but this is normal, the two situations differ radically.}
The Euclidean regime ($\legexp = 0$) is perfectly aligned with existing results for the geometric last-iterate convergence rate of the \ac{EG} algorithm and its variants \citep{GBVV+19,Mal15,HIMM19,MOP20}.
By contrast, the Legendre regime ($\legexp > 0$) indicates a significant drop in the algorithm's last-iterate convergence speed, even though ergodic convergence rates \cite{Nes04} and results for bilinear games \cite{WLZL21} might suggest otherwise.

Subsequently, motivated by applications to game theory and linear programming, we take a closer look at the convergence rate of \acp{BPM} across the constraints that are active at a solution $\sol$ of \eqref{eq:VI} depending on the position of $\vecfield(\sol)$ relative to said constraints. 
This analysis reveals that Bregman proximal methods have a particularly fine structure:
along \emph{sharp directions} (\ie constraints along which $\vecfield(\sol)$ is strictly inward-pointing), \acp{BPM} converge
\begin{enumerate*}
[(\itshape i\hspace*{1pt}\upshape)]
\item
at a rate of $\bigoh(1/\run^{1/(2\legexp-1)})$ if $1/2 < \legexp < 1$;
\item
at a \emph{geometric rate} if $0 < \legexp \leq 1/2$ (\eg for entropic methods);
and
\item
in a \emph{finite} number of iterations if $\legexp=0$
\end{enumerate*}
(\cf \cref{thm:sharp}).
Thus, even though the estimates of \cref{thm:general} are, in general tight, the actual convergence rate of a Bregman method along different coordinates\,/\,constraints could be starkly different \textendash\ and, in fact, dramatically faster if the solution under study is itself sharp.

The closest antecedent of our work is the conference paper \cite{AIMM21} where the Legendre exponent was introduced to analyze the convergence of \ac{OMD} in \emph{stochastic} \ac{VI} problems (without considering sharp directions and/or faster identification rates).
The stochastic and deterministic settings are obviously very different, both in the challenges involved as well as the rates obtained, so there is no overlap in our analysis and results.
Other than that, we are not aware of any comparable results in the literature concerning the radically different convergence landscape of \acp{BPM} along active and inactive constraints.
\section{Related Work}

The use of machine learning in computer aided design (CAD) has gained significant attention, and only a few works have been proposed in recent literature that develop machine learning approaches. SketchGraphs dataset~\cite{seff2020sketchgraphs} is a collection of sketches extracted from parametric CAD models which begin as two-dimensional (2D) sketches consisting of geometric primitives (e.g., line segments, arcs) and explicit constraints between them (e.g., coincidence, perpendicularity) that form the basis for three-dimensional (3D) construction operations. This dataset has been used for generative model of CAD sketches~\cite{willis2021engineering}, and other applications of learning in physical design~\cite{seff2021vitruvion, para2021sketchgen}.
Building a design grammar has been the subject of other works such as in \citet{zhao2020robogrammar}. There are also many other works that have focused on specific parts such as robot arms \cite{xu2021end} and airfoil design \cite{chen2022learning}. Two key differences between our approach to design and these previous works is: (1) Our focus on a pathway towards deployment by using physically realizable components; (2) The use of transformer models on our design embeddings to directly evaluate performance.


% While these works have seen a lot of success, they are not focused on the pathway towards actual deployment, whereas the focus in our work is to eventually build physically realizable designs by building a procedural generator and transformer model that operate over real components.


% Another example of a CAD dataset that is focused on physical structure is SimJEB \cite{whalen2021simjeb}, which is a dataset of crowdsourced mechanical brackets and accompanying structural simulations. DeepCAD~\cite{wu2021deepcad} is a dataset of 3D shapes corresponding to objects such as flanges, pipes and screws, represented as a sequence of operations used in a CAD framework to generate these shapes. Another dataset for 3D engineering shapes is the ABC dataset~\cite{koch2019abc}, 
% which comprises geometric models, each defined by parametric surfaces and
% associated with accurate ground truth information on the decomposition into patches. Their representation allows resampling the surface data at arbitrary resolutions into a point cloud or mesh. These datasets are excellent resources for their target application domains such as extrapolating 2D sketch to CAD designs, and generating mechanical parts.


% In another line of related work, surrogate-based optimization is widely explored in design optimization, where the goal is to learn a surrogate function to replace often expensive black-box simulators e.g., computational fluid dynamics simulators~\citep{koziel2011surrogate, han2012surrogate, viquerat2021direct}. The surrogate function aims to capture the physical properties of the design environment and reliably evaluate design samples. These approaches tend to be more scalable compared to the black-box optimization approaches~\citep{greenhill2020bayesian, belakaria2020uncertainty, deshwal2021bayesian} by avoiding the expensive black-box evaluation during optimization. Further, if the surrogate function is differentiable e.g., a neural network, the gradients are also available to the optimizer to perform an end-to-end optimization \cite{grabocka2019learning, liu2020unified, sun2021amortized}.
% Our proposed method can leverage these advances in better surrogate modeling for more efficient exploration.

% In contrast to existing methods, the design for physical systems needs to find a diverse set of designs that trade off different objectives and allow further downstream adaptation to new design objectives.  
\section{Case Decision Summarization Dataset}\label{sec:dataset}
Recent work has introduced a number of legal document summarization or salient information identification tasks with associated datasets, e.g., for bill %or case 
summarization %\cite{kornilova-eidelman-2019-billsum, malik2021semantic} 
\cite{kornilova-eidelman-2019-billsum}
and for case sentence argumentive classification \cite{xu-2021-position-case} and rhetorical role prediction
\cite{malik2021semantic}.
%However, unlike the news and scientific articles, only limited amount of annotated data point are available and the size is of orders of magnitude smaller. 
\begin{table}[t!]
% \small
% \scriptsize 
    \centering
    % \setlength\tabcolsep{2.1pt}
    \renewcommand{\arraystretch}{1}%Tighter
    \begin{tabular}{l|c}
    
    %\begin{tabular}{l|lllc|lllc|p{0.9cm}ll}
   \toprule
 %  &  CanLII  \\
 %\midrule
    Case length (avg. \# words) & 3,971\\
    Summary length  (avg. \# words) & 266\\
    Training set  (\# case/summary pairs) & 27,241 \\
    
    Testing set  (\# case/summary pairs) & 1,049\\
    
%      Name & Size & Doc. len & Summ. len. \\
%     \midrule
%     BillSum & 22,218 & 1,592 &  197 \\
%   CanLII & 28,290 & 3,971   &  266 \\
%   \midrule
%     XSum & 226k &  431 & 23 \\
%     CNN/DM &  311k & 766 & 53 \\
%     \midrule
%     PubMed  & 133k  &  3,016 & 203 \\
%     arXiv &  215k & 4,938 & 220 \\
    \bottomrule
    \end{tabular}
    \caption{Dataset statistics of CanLII.}
    \label{tab:data_stats}
\end{table}
\begin{figure}[t!]
\centering
 \includegraphics[width=.9\linewidth]{Figs/IRC_distribution.png}

  \caption{Fraction of sentences annotated as argumentative (using the IRC scheme) in the case documents versus in the summaries of the CanLII test set. Though only a small fraction of sentences in the original document are annotated as IRCs, IRCs are a large fraction of the human-written summaries.}
  \label{fig: disbribution}
\end{figure} 
Similarly to \citet{xu-2021-position-case}, we use the {\bf CanLII} (Canadian Legal Information Institute) dataset of legal case decisions and summaries\footnote{The data was obtained through an agreement with the Canadian Legal Information Institute (CanLII) (\url{https://www.canlii.org/en/}).}. 
%Persons interested in accessing the data should contact CanLII directly.}.   %Kevin suggested this wording
%\footnote{The data can be obtained through a license agreement with the Canadian Legal Information Institute (\url{https://www.canlii.org/en/}).}%\footnote{We provide examples from this dataset in Appendix \ref{appendix:output}.} 
  Full corpus statistics are provided in Table \ref{tab:data_stats}, while an example case/summary pair from the test set is provided in Figure \ref{fig:IRC_example} in Appendix \ref{appendix:1}.
%ref{appendix:output}.
%\\
    % (1) \textbf{BillSum}\footnote{\url{https://github.com/FiscalNote/BillSum}} \cite{kornilova-eidelman-2019-billsum}, which contains 22,218 US Congressional Bills with human-written references and has been split into 18,949 train bills and 3,269 test bills. \\
%\textbf{CanLII}  
%of the non-profit Canadian Legal Information Institute\footnote{\url{https://www.canlii.org/en/}}. 

\citet{xu-2021-position-case} only used a small portion of this dataset for their work in argumentative classification. Conjecturing that explicitly identifying the decision's argumentative components would be crucial in case summarization,  they annotated 1,049  case and human-written summary pairs curated from the full dataset.
%CanLII Connects website\footnote{\url{https://canliiconnects.org/en}}. %\citet{xu-2021-position-case}
In particular, they recruited legal experts to annotate the document on the sentence level, adopting an \textbf{IRC scheme} (see Figure \ref{fig:IRC_example} in Appendix \ref{appendix:1}) which classifies individual sentences into one of  four categories: \textbf{Issue} (legal question  addressed in the case), \textbf{Conclusion} (court’s decisions for the corresponding issue), \textbf{Reason} (text snippets explaining why the court made such conclusion) and \textbf{Non\_IRC} (none of the above). The distributions of the IRC labels in  the cases and summaries are shown in Figure \ref{fig: disbribution} and illustrate that argumentative sentences do indeed play an important role in human summaries. We utilized the unannotated 27,241  pairs to train a  supervised model baseline and the 1049 annotated pairs as our test set.  While none of our summarization methods  use the IRC annotations, they are used during testing as the basis of a domain-specific evaluation metric.

%For comparison, the statistics of the commonly used mews and scientific domain summaries are also included \cite{nallapati-etal-2016-abstractive, narayan-etal-2018-dont, cohan-etal-2018-discourse}.  



% \begin{table}[t!]
% % \small
% % \scriptsize 
%     \centering
%     % \setlength\tabcolsep{2.1pt}
%     \renewcommand{\arraystretch}{1}%Tighter
%     \begin{tabular}{c|c| cc}
    
%     %\begin{tabular}{l|lllc|lllc|p{0.9cm}ll}
%     \toprule
%      Name & Size & Doc. len & Summ. len. \\
%     \midrule
%     BillSum & 22,218 & 1,592 &  197 \\
%   CanLII & 28,290 & 3,971   &  266 \\
% %   \midrule
% %     XSum & 226k &  431 & 23 \\
% %     CNN/DM &  311k & 766 & 53 \\
% %     \midrule
% %     PubMed  & 133k  &  3,016 & 203 \\
% %     arXiv &  215k & 4,938 & 220 \\
%     \bottomrule
%     \end{tabular}
%     \caption{Dataset statistics on legal documents (BillSum and CanLII), news articles (CNN/Daily Mail and XSum
% ) and long scientific documents (PubMed and arXiv).}
%     \label{tab:data_stats}
% \end{table}

\section{Method and Models}
We propose a reweighting model that employs a %simple 
graph-based ranking algorithm to exploit the %document 
structures encoded in long legal case decisions. %.  
%Our method is inspired by the two-level hierarchy design from long scientific article summarization \cite{dong-etal-2021-discourse} and early summarization work tailored for legal document structures \cite{farzindar-lapalme-2004-legal}.

\subsection{Discourse-Aware Backbone Model}\label{sec:main_paper_hiporank}
The HipoRank (Hierarchical and Positional Ranking) model recently developed by \citet{dong-etal-2021-discourse} constructs a directed graph for document representation using document section and sentence hierarchies.
%\footnote{We include a more detailed introduction of the algorithm in Appendix \ref{sec:hiporank}.}. 
HipoRank computes the centrality score of each sentence as 
\begin{equation}
    c(s_i^{I}) = \mu_1 c_{inter} (s_i^{I}) + c_{intra}(s_i^{I})
\end{equation} where $s_i^{I}$ refers to the $i$-th sentence in $I$-th section. $\mu_1$ is a tunable hyper-paramter, $c_{inter}(s_i^{I})$ computes the sentence's similarity to other section representations and  $c_{intra}(s_i^{I})$ computes the average similarity of the current sentence with all others in the same section. 
%, we refer to the original article. 
HipoRank then selects the top-K ranked sentences as the  summary. More details of the algorithm are provided in Appendix \ref{sec:hiporank}.
Directly applying HipoRank to our data yielded multiple challenges (e.g.,  redundant neighboring sentences %based on computed scores
(recall Figure \ref{fig:output_figure}) as well as too many sentences from the ends of the article were selected).

\subsection{Multiple Views of Document Structure}\label{sec:views}

Before creating a HipoRank %hierarchical 
document graph, 
the document must be split into sections and sentences. %Although HipoRank is agnostic to the %section/sentence 
%splitting methods, 
The scientific %document 
datasets previously used with HipoRank %in the HipoRank study 
were already split \cite{dong-etal-2021-discourse}.   We investigate {\it the summarization impact of using different approaches to automatically compute linear sections of the document structure}.
%in the same section.}.
Figure~\ref{fig:three_view} shows different structures for the same  case.
%\footnote{We reprocessed the original HTML files to obtain the original document structures, which may lead to minor sentence formatting problems comparing to the other two structures that used non HTML data preprocessed by \citet{xu-2021-position-case}.}. % can be found in 
%Appendix \ref{apendix:three_views}.

\textbf{Original Document Structure}
This approach extracts the structure by processing the %raw 
{\it HTML files}. We use a heuristic to mark the section names with an italic and bold format as the %section
boundaries and segment the documents into multiple continuous sections. It is worth noting that 297 of the 1049 test case documents do not come with explicit section splits, thus we treat them as whole text spans\footnote{The Original Structure method processed HTML source files and split sentences using a legal %domain-specific 
sentence splitter (\url{https://github.com/jsavelka/luima_sbd}). The Topic and Thematic views used non-HTML data preprocessed by \citet{xu-2021-position-case}, but used the same sentence splitter.}.
%\footnote{We use the sentence splitter from \url{https://github.com/jsavelka/luima_sbd}, as is used in the original CanLII dataset \cite{xu-2021-position-case}.}.

%\paragraph
\textbf{Topic Segment View} Meanwhile, we also explore using a traditional, {\it domain-independent linear text segmentation} algorithm. We use  C99 \cite{choi-2000-advances} but with advanced sentence representation from SBERT \cite{reimers-gurevych-2019-sentence} to group neighboring sentences into topic blocks. %We represent the sentences with
%recent advanced sentence representations using SBERT \cite{reimers-gurevych-2019-sentence}.

\textbf{Thematic Stage View}
Early studies found that legal documents tend to have well-defined, 
{\it domain-dependent thematic structures} \cite{farzindar-lapalme-2004-legal} or rhetorical roles \cite{saravanan-etal-2008-automatic}. Following work that extracts stage views in conversations (introductions → problem exploration →
problem solving → wrap up) \cite{chen-yang-2020-multi}, we extract thematic stages
through a Hidden Markov Model (HMM). A fixed order of stages is imposed and only consecutive transitions are allowed between neighboring stages.  We again represent the sentences %with recent advanced sentence representations 
using SBERT \cite{reimers-gurevych-2019-sentence} and set the number of stages as 5, including the starting Decision Data, Introduction, Context, Judicial Analysis, and Conclusion, as introduced by \citet{farzindar-lapalme-2004-legal}. 
 
 \begin{figure*}[t!]
\centering
 \includegraphics[width=0.85\linewidth]{Figs/three_views.001.jpeg}
  \caption{Different document structure views of a legal case decision (ID: c\_2003skpc17) from our CANLLI test set. Original case sentences are annotated with \textcolor{red}{Issue}, \textcolor{reasonblue}{Reason}, and \textcolor{conclusiongreen}{Conclusion} labels. On the left side, the \textcolor{green}{green}, \textcolor{yellow}{yellow} and \textcolor{cyan}{blue} boxes refer to thematic stage, topic segmentation and the original document structure, respectively. The boxes mean the corresponding sentences on the right hand side are grouped into the same segments. For instance, for the first blue box, the original article is split by the italicized and bolded section name of ``The Fact''. }
  \label{fig:three_view}
\end{figure*}

\textbf{CanLII Header Removing}\label{sec:header_removing} Preliminary analysis demonstrated that the raw CanLII documents fail to distinguish the less important headers at the beginning  (i.e., the descriptive text before the main  content, for example,  the content above the grey splitting line and BASIS OF CLAIM in Figure \ref{fig:output_figure}). Generated summaries also tend to cover  a large portion of such information. We thus propose a legal case decision preprocessing procedure following certain heuristics\footnote{See Appendix \ref{sec:appendix_heuristics} for details.} to remove those headers, and demonstrate the improved summarization quality (for all views of document structure) in Section \ref{sec:result}. 

%We process the original articles into different sections using these three different document structures (one example in Figure \ref{fig:three_view}), and experiment with the state-of-the-art unsupervised extractive baseline HipoRank \cite{dong-etal-2021-discourse} model.


 
% We illustrate the effects of the three methods of document segmentation in Figure \ref{fig:three_view}. It is worth noting that, comparing to the original document structure split by section names, the topic segmentation will produce more fine-grained subparts. But different strategies to segment the document can have diverse effects. For instance, for the highlighted first blue segments, the \textit{issue} sentence is grouped together with all previous headers. Meanwhile, as shown in the last highlighted yellow span, the \textit{reason} sentence is connected with its neighboring sentences, thus can aid for better contextual representation. 

% TODO: Start with hiporank with different inputs.
% TODO: Section 2 reweighting (ours)
%  Yet, there exists drastic difference between the scientific domain and legal domain.  

\subsection{Reweighting the Centrality Score}\label{sec:dynamic_selection}


%Secondly, the original algorithm computed the sentence centrality score only once and greedily select the top-K ranked sentences as the summary. Our analysis \ref{sec:analysis} demonstrate that such approach fail to select sentences in the middle of context which are less similar to others, but carrying the \textit{reasoning} argument role of the legal case. % To tackle the problem of selecting headers (the meta data of the case, as shown in Figure  and the difficulty of reasoning sentences,
% we thus introduce a novel iterative reweighting framework that dynamically update the weights of the candidate nodes based on already selected sentences ( \S \ref{sec:dynamic_selection}). 

% \subsection{Legal Document Summarization}\label{sec:difference}

% \subsection{Hierarchical Directed Graph for legal document}
% Most existing work (TextRank, LexRank, PacSum) build the graph by only considering the sentence similarity and relative order of sentences in the document, which fails to capture the rich document structure of texts. \citet{dong-etal-2021-discourse} introduces a two-level directed hierarchical model that explicitly models the section structures in the scientific articles.

% Different from the clear organizations of sections in scientific articles, the legal case documents sometimes do not come with well-structured format, we instead utilize the multi-views extracted from \S \ref{sec:views}. Similar to prior work, HipoRank formulated the  summarization generation
% as a node selection problem, where those
% nodes (i.e., sentences) that are semantically similar
% to others are picked in the final summary. They proposed a section-level hierarchy, based on the assumption that sentences belonging to different sections are less similar and instead used a section embedding to model the sentence to other section relationship, thus reducing the cost of computation. Different from the strict restrictions on the cross-section sentence edges, we allow for randomized sentence-to-sentence edges, aiming at avoiding the negative effects of under-representation of section. We additionally introduce the weighted parameters to model the distance between either sentence-section or sentence-sentence edges, as inspired by \cite{liu-sigir-2021-distance}.


%Another drawback of 
The HipoRank document graph %built by HipoRank %is that it 
will not change once built, and the important sentences are greedily selected based on the aggregated centrality scores. %Such a method 
This may introduce redundancies in selecting similar sentences and ignore the contents in the middle of the source case decisions that are more important once the argumentative sentences at the beginning and end are taken into account. We propose a novel reweighting approach that can tackle this problem. A prior attempt \cite{tao2021unsupervised}  on multi-round selection looked at the local similarity between %those 
selected sentences. They iteratively recompute the sentence to sentence similarities between the selected summary sentences and recompute the final sentence centrality scores after each sentence selection. Instead, we are focusing on modeling  the relationship between the selected sentence and the other candidate sentences.  Their method is also not directly applicable to longer text due to the $n^2$ time complexity of computation given large numbers of sentences (on average 205 sentences for CanLII dataset). 
\begin{algorithm}
\caption{Reweighting Algorithm}\label{alg:cap}
\begin{algorithmic}
\Require computed centrality score $c(s_i^{I})$ for all sentence s, $c_{intra}(d)$ for different section d 's embedding, and a threshold \textit{g} for phase transition and maximum summary length $max_{len}$.

\State $Summ \gets []$
\State \textbf{PHASE 1} 
\While{$len(Summ) \leq g * max_{len}$}

    \State $Summ.append(topK(\{c(s_i^{I})\})$
\EndWhile \\
\State{\textbf{PHASE 2}} %\\
\While{$len(Summ) \leq max_{len}$}
\State $c(s_i^{I}) \gets c(s_i^{I}) - sim(c_{intra}(I), c_{intra}(J)) * \mu_2$ 
\Comment{J is the section index of last selected sentence, $\mu_2$ is a hyperparameter}
\State $Summ.append(top-1({c(s_i^{I})}))$
\EndWhile

\textbf{Return} Summ
\end{algorithmic}
\label{algorithm_1}
\end{algorithm}

Our approach can be divided into two phases, as shown in Algorithm \ref{algorithm_1}. In the first phase, we assume that important argumentative sentences at the two ends of the document can be easily detected (as shown in Figures \ref{fig:output_figure} and \ref{fig:three_view}, legal case documents generally start by introducing the issues and end with  conclusions).  A quantitative analysis of the top-5 selected sentences in CanLII in fact provides an 80\% coverage of issue or conclusion sentences.   We thus set up a threshold to pick the first k sentences based on the original document graphs. Afterward, we gradually downweight the sentence's centrality score using the location of the latest selected sentence, that is, we set a penalty score for sentences that are placed as a neighbor of the current sentence selected for the summary. Our rationale %behind such a procedure 
is that %we observe 
reasoning sentences are more likely to be located in different sections in the middle that are not shared with issues and conclusions. %As shown in Figure \ref{fig:three_view}, in legal cases the issues and conclusions are generally placed on the two ends of the article while the reasoning sentences are distributed in the middle. 
% \begin{table*}[h!]
% \small
% % \scriptsize 
%     \centering
%     % \setlength\tabcolsep{2.1pt}
%     \renewcommand{\arraystretch}{1}%Tighter
%     \begin{tabular}{c|l|ccc|ccc}
%     %\begin{tabular}{l|lllc|lllc|p{0.9cm}ll}
%     \toprule
%       ID & Model  &  R-1 & R-2 & R-L  & BS \\
%          \midrule
%          \multicolumn{6}{c}{{Oracles}} \\
%          \midrule
%           1 & IRC & 57.92 & 35.72 & 55.05 &88.15 \\
%         2 &  EXT & 61.41 & 40.44 & 59.08 & 88.19 \\
          
%     \midrule 
%     \multicolumn{6}{c}{{Supervised Extractive }} \\
%     \midrule
%     3 & BERT\textsubscript{EXT} - selection & 40.78 & 17.63 & 37.75 & 84.05 \\
%     4 & SummRunnar & & & & \\
%      5 & SentCLR & & & & \\
%     \midrule
%     \multicolumn{6}{c}{{Supervised Abstractive}} \\
%     \midrule
%     6 & BART & 47.93 & 22.34 & 44.74 &  86.16 \\
%     7 & LED & 49.56 & 23.78 & 46.48  & 86.75\\
%     \midrule 
%         \multicolumn{6}{c}{{Unsupervised Extractive}} \\
%         \midrule 
%       8 & LSA & 38.15 & 18.22 & 35.76 & 84.63  \\
%           9& LexRank & 38.63 & 17.70 & 35.87 & 84.40 \\
%             10 & TextRank   & 36.70 & 16.19 & 34.00 & 83.53&  \\
%             11 & PACSUM & 39.24 & 14.31 & 36.80 & 81.64\\
%           \midrule 
%           \multicolumn{6}{c}{{HipoRank with different structures}} \\
%           \midrule
%                   12 & HipoRank (Original) & 43.42 & 18.04&  40.38 &  83.90 \\
%           13 & C99-topic & 42.03 & 16.89 & 39.32 &  83.59\\
%           14  & HMM-stage & 42.11 & 17.03 & 39.48 &  83.85 \\
%           \midrule 
%           15 & Original w/o header & 43.59 & 18.41 & 40.74 & 84.41 \\
%           16 & C99-topic w/o header & 44.32 & 18.57 & 41.50 & \textbf{84.56} \\
%           17 & HMM-stage w/o header & 43.76 & 18.52 & 41.07 & 84.53 \\
%           \midrule 
%         %   18 & Original Dynamic & 43.99 & 18.88 & 41.24 & 83.94\\
%         %   19 & C99-topic Dynamic & 43.76 & 18.25 & 41.04 & 83.81 \\
%         %     20 & HMM-stage Dynamic & 43.43 & 17.89 & 40.69& 83.93 \\
%         % \midrule 
    
%           18 & Original Dynamic + w/o H & 44.36 & 18.98& 41.63 &  84.43\\
%           19 & C99-topic Dynamic + w/o H & \textbf{45.33} & \textbf{19.77} & \textbf{42.54} & 84.50\\
%             20 & HMM-stage Dynamic + w/o H & 44.54 & 18.76 & 41.88 & 84.31 \\
%             \hline 
%             21 & C99-topic Dynamic + w/o H + sparse sent2sec & 44.84 & 19.28 & 42.07 & -- \\
%             \hline 
%             22 & Original Dynamic + w/o H + sent2sent reward & 44.18 & 18.86 & 41.37 & -- \\
%         \midrule
%     \bottomrule
%      \end{tabular}
%     \caption{Different Structure's Performance.}
%     \label{tab:unsupervised_results}
% \end{table*}

% \begin{table*}[h!]
% % \small
%     \centering
%     \renewcommand{\arraystretch}{1}%Tighter
%     \begin{tabular}{c|l|cc|cc}
%     %\begin{tabular}{l|lllc|lllc|p{0.9cm}ll}
%     \toprule
%         & & \multicolumn{2}{c}{\textbf{CanLII}} & \multicolumn{2}{c}{\textbf{BillSum}} \\
%       ID & Model  &  R-1/R-2/R-L  & BS & R-1/R-2/R-L  & BS\\
%          \midrule
%          \multicolumn{6}{c}{{Oracles}} \\
%          \midrule
%           1 & IRC & 58.04/36.02/55.28 & 87.94 & N/A & N/A \\
%         2 &  EXT (ROUGE-L, F1) & 59.38/38.77/56.94 &87.85  & 56.04/37.50/53.10 & 87.30\\
        
   
          
%     \midrule 
%     \multicolumn{6}{c}{Baselines} \\
%     \midrule
%     \multicolumn{6}{l}{\textit{supervised}} \\
%     % 3 & BERT\textsubscript{EXT} - selection & 40.78 & 17.63 & 37.75 & 84.05 \\
%     3 & BERT\_EXT & \textit{43.44}/\textit{17.84}/\textit{40.36} & \textit{84.47} & 39.57/15.89/35.08 &  83.54 \\
%     %  5 & SentCLR & & & & \\
%     % \midrule
%     % \multicolumn{6}{c}{{Supervised Abstractive}} \\
%     % \midrule
   
%     % 6 & BART & 50.50/25.58/46.82 &  87.25& \\
%     % 7 & LED & 53.72/28.75/ 50.17 & 87.55&\\
%     % \midrule 
%         \multicolumn{6}{l}{\textit{unsupervised}} \\
%         % \midrule 
%       4 & LSA & 37.22/17.82/34.87 & \underline{\textit{84.48}} & 40.44/18.07/36.31 &  83.70 \\
%           5 & LexRank & 37.90/\underline{\textit{18.17}}/35.62 &  84.32 &  41.26/\textbf{\underline{\textit{21.19}}}/37.43 & 84.02\\
%             6 & TextRank  & 36.70/16.19/34.00 & 83.51 & 36.38/16.93/30.84 & \textbf{\textit{\underline{84.49}}}\\
%             7 & PACSUM & 40.01/15.68/37.37 & 83.52  & 40.71/18.23/37.00 & 83.08  \\
%             8 & HipoRank (Original Structure) &  \underline{41.61}/17.13/\underline{\textit{38.73}}  & 83.55  & \underline{\textit{42.41}}/19.48/\underline{\textit{38.65}} & 83.24 \\
%           \midrule 
%           \multicolumn{6}{c}{{Document Structure with HipoRank backbone}} \\
%           \midrule
%                 %   12* & Original 
%           9 & C99-topic & 41.33/16.48/38.45 &  83.53 & 41.82/18.62/37.94& 83.57 \\
%           10  & HMM-stage & 40.71/15.64/37.93 & 83.57 & \textbf{42.97}/{20.37}/\textbf{39.24} & {83.90} \\
%         %   \midrule 
%           11 & Original Structure w/o header & 42.56/17.96/39.62 & 83.62 & N/A & N/A \\
%         %   & Original w/o header + truncation & 42.85/17.46/39.88 \\ 
%           12 & C99-topic  w/o header & 43.25/18.02/40.25 & \textbf{84.48} & N/A & N/A   \\
        
%           13 & HMM-stage  w/o header & 42.64/17.38/39.76 & 83.57 & N/A & N/A \\
          
%           \midrule 
%               \multicolumn{6}{c}{{Document Structure with HipoRank backbone + reweighting Algorithm}} \\
%               \midrule
%           14 & Original Structure* & 43.18/{18.35}/40.26 &  84.20  & 42.30/19.73/38.69 & 83.11 \\ 
%           15 & C99-topic* & \textbf{43.89}/\textbf{18.67}/\textbf{41.00} &  {84.34} & 40.78/17.95/37.02 & 83.15\\
%             16 & HMM-stage* & {43.28}/17.80/40.40 &  84.22 & 42.09/19.83/38.48 & 83.56\\ 
%             % 17 & Ours + sentence\_relation & 43.26/18.33/40.38 & \\
            
%         \midrule
%     \bottomrule
%     \end{tabular}
%     \caption{The automatic evaluation results on CanLII and BillSum test sets, \textbf{bold} represents the best non-oracle extractive score, \textit{italic} for the best baselines and \underline{underline} for best unsupervised baseline. Model name with stars, we use the without header version of CanLII dataet as input and keep the original BillSum data.} % need to make sure the notation is obvious. 
%     \label{tab:unsupervised_results}
% \end{table*}


% \section{Unsupervised Summarization}




\section{Experimental Setup}
%We discuss the evaluation metrics and other models  used in our experiments. 
For supervised models, we split the training data in an 80:20 ratio for training and validation. For unsupervised models, we tune the hyperparameters on the validation set. Model training details can be found in Appendix \ref{appendix:detail}. 

% \subsection{Datasets}
% Our experiments are conducted on CanLII \cite{} and
% BillSum \cite{kornilova-eidelman-2019-billsum}, two large-scale datasets
% of long and structured legal documents with human-written summaries. The average source article
% length is xxxx and 1382 tokens, making them ideal
% candidates to test our method.

%\subsection{Oracle and Baseline Models}



\begin{table*}[t]
% \small
    \centering
    \renewcommand{\arraystretch}{1}%Tighter
    \begin{tabular}{c|l|cccc}
    %\begin{tabular}{l|lllc|lllc|p{0.9cm}ll}
    \toprule
  %      & & \multicolumn{2}{c}{\textbf{CanLII}} \\
      ID & Model  &  R-1 & R-2 & R-L   & BS\\
         \midrule
         \multicolumn{6}{c}{{Oracles}} \\
         \midrule
          1 & IRC & 58.04 & 36.02 &55.28 & 87.94  \\
        2 &  EXT (ROUGE-L, F1) & 59.38 & 38.77 & 56.94 & 87.85 \\
        
   
          
    \midrule 
    \multicolumn{6}{c}{Extractive baselines (no document structure)} \\
    \midrule
    \multicolumn{6}{l}{\textit{supervised}} \\
    % 3 & BERT\textsubscript{EXT} - selection & 40.78 & 17.63 & 37.75 & 84.05 \\
    3 & BERT\_EXT & \textit{43.44} & {17.84} & \textit{40.36} & {84.47}  \\
    %  5 & SentCLR & & & & \\
    % \midrule
    % \multicolumn{6}{c}{{Supervised Abstractive}} \\
    % \midrule
   
    % 6 & BART & 50.50/25.58/46.82 &  87.25& \\
    % 7 & LED & 53.72/28.75/ 50.17 & 87.55&\\
    % \midrule 
        \multicolumn{6}{l}{\textit{unsupervised}} \\
        % \midrule 
       4 & LSA & 37.22 & 17.82 & 34.87 & {\bf \underline{\textit{84.48}}} \\
           5 & LexRank & 37.90 &\underline{\textit{18.17}} & 35.62 &  84.32 \\
            6 & TextRank  & 36.70 &16.19 &34.00 & 83.51 \\
            7 & PACSUM & 40.01 & 15.68 &37.37 & 83.52  \\
            %8 & HipoRank (Original Structure) &  \underline{41.61} & 17.13 & \underline{\textit{38.73}}  & 83.55  \\
           \midrule 
           \multicolumn{6}{c}{{HipoRank backbone (with different computed document structures) 
           %with HipoRank backbone
           }} \\
          \midrule
                %   12* & Original 
          8 & Original Structure &  {41.61} & 17.13 & {38.73}  & 83.55  \\
          9 & C99-topic & 41.33 & 16.48 & 38.45 &  83.53  \\
          10  & HMM-stage & 40.71 & 15.64 & 37.93 & 83.57 \\
        %   \midrule 
          11 & Original Structure w/o header & 42.58 & 18.01 & 39.63 & 83.62 \\
        %   & Original w/o header + truncation & 42.85/17.46/39.88 \\ 
          12 & C99-topic  w/o header & \underline{43.25} & 18.02 & \underline{40.25} & \underline{\textit{\textbf{84.48}}} \\
        
          13 & HMM-stage  w/o header & 42.64 & 17.38 & 39.76 & 83.57\\
          
          \midrule 
              \multicolumn{6}{c}{{Ours (HipoRank backbone + Reweighting Algorithm)}} \\
              %Document Structure with HipoRank backbone + reweighting Algorithm}} \\
              \midrule
          14 & Original Structure w/o header & 43.14 & {18.46} & 40.23 &  84.20  \\ 
          15 & C99-topic w/o header & \textbf{43.90} & \textbf{18.67} & \textbf{41.00} &  {84.34} \\
            16 & HMM-stage w/o header & {43.28} & 17.80 &40.40 &  84.22 \\ 
            % 17 & Ours + sentence\_relation & 43.26/18.33/40.38 & \\
            
        \midrule
    \bottomrule
    \end{tabular}
    \caption{The automatic evaluation results on the CanLII test set. \textbf{Bold} represents the best non-oracle score, \textit{italic} the best baseline/backbone score, and \underline{underline} the best unsupervised baseline/backbone score.} %The model names with "w/o header" use the without header version of CanLII dataet as input to .} % need to make sure the notation is obvious. 
    \label{tab:unsupervised_results}
\end{table*}

\textbf{Upper Bound Oracles}\label{sec:oracle}  %Taking advantage of the annotations in our test set and 
Based on  Figure \ref{fig: disbribution}, we create a domain-dependent \textbf{IRC\_Oracle} model where test sentences  manually annotated with the IRC  labels are concatenated to form the summary.   Following \citet{nallapati2017summarunner}, we also report results for \textbf{EXT\_Oracle}, a domain-independent
summarizer %\cite{nallapati2017summarunner}, 
which greedily selects sentences from the original document based on the ROUGE-L scores compared to the abstractive human summary. %This model selects sentences whose concatenation maximizes the evaluation metric given the gold summary.

\textbf{Extractive Baselines} %We compare our proposed discourse views and reweighting algorithm with various baselines. 
For unsupervised models, we compare with LSA  \cite{Steinberger2004UsingLS},
LexRank \cite{erkan2004lexrank}, TextRank \cite{DBLP:journals/corr/BarriosLAW16}, and PACSUM \cite{zheng-lapata-2019-sentence}. We also include HipoRank \cite{dong-etal-2021-discourse} with document views. %),  as it achieved high %the state of the art 
%performance on long scientific papers.  %summarization
For supervised methods, we compare with BERT\_EXT \cite{liu-lapata-2019-text}. % uses a  document-level encoder based on BERT. % which is able to express the semantics of a document and obtain sentence representations. 
Although not our focus,  abstractive baselines are in Appendix \ref{sec:appendix_abstractive}. 

%\subsection
\textbf{Automic Evaluation Metrics}
%For automatic evaluation, 
We report ROUGE-1 (R-1), ROUGE-2 (R-2), and ROUGE-L (R-L) F1 scores, as well as % We also report
BERTScore (BS) \cite{bert-score}. %to measure the semantic meanings of the generated summary. 
%Finally, we design an automatic metric for the CanLII dataset. 
%Recall from Section  \ref{sec:dataset} that our test set comes with sentence-level annotations on both source and target texts following the ``IRC'' scheme. %, that is, a sentence can be manually classified as either Issue, Reason, Conclusion or Non-IRC types. %(more detailed description in \S \ref{sec:dataset}). 
We also propose metrics to measure the recall value of the annotated IRC types in the test set, % in source-side, as well as the recall values in the summary, 
which exploits the structure of case documents. More details are in Section \ref{sec:salient_eval}.  

 
     
\section{Results}\label{sec:result}
In this section, we aim to deal with three research questions: 
\textbf{RQ1}. How well do the extractive baselines including the HipoRank backbone deal with legal documents?  
\textbf{RQ2}. How well do the different views of document structures perform with the  HipoRank backbone?  \textbf{RQ3}. Can the reweighting algorithm help select important argumentative sentences and improve summary quality?

\subsection{Automatic Summarization Evaluation}
Table \ref{tab:unsupervised_results}
% We 
compares our methods with prior extractive models. See Appendix \ref{appendix:output} for example summaries. % can be found %in Appendix \ref{appendix:output}. 
% The first block in Table \ref{tab:unsupervised_results} includes the IRC 
%and the extractive oracles. The next
%block  presents the results of one supervised
%and several unsupervised extractive
%baseline models. The block after that presents the discourse-aware HipoRank baseline with our different views of document structure.  The final block presents our reweighting  model's results (using the "w/o header" version of the CANLLI documents as they performed best in the prior block.) 


%Overall, our approach on incorporating the different document views and selection methods obtained the best results on the datasets.

\textbf{RQ1.} Table \ref{tab:unsupervised_results} shows that there is still a gap between oracle models (rows 1 and 2) and current extractive baselines. There are around 20 points differences on R-1, R-2, and R-L. Among the baselines, the supervised model works best  only for R-1 and R-L. Unsupervised methods obtain  the highest BS (row 4) and R-2  (row 5), possibly due to the higher coverage of n-grams benefitting from longer extracted summaries (row 3 model generated summaries have an average length of 250; row 4-6 models generate on average 400-word summaries;  row 7 and 8 models have a limit of 220 words). %Comparing to the supervised model, unsupervised methods (row 4-8) have lower scores, which is reasonable given that the model is not exposed to any training data.
Without  reweighting, the HipoRank backbone % - even with the best performing document structure - 
never outperforms the best extractive baseline.  However, if only unsupervised baselines are considered,  HipoRank in row 12 does show the best performance for 3 of the 4 evaluation metrics.
%We first compare the best-performed baseline model, HipoRank's performance given documents with different structure representation. 

\textbf{RQ2.} To examine the effects of the  document views in  Section \ref{sec:views}, we split the document into different types of linear segments and then used  HipoRank  to generate summaries.  Recall that HipoRank is the only model to exploit  document structure, and as noted for RQ1, with the right structure could obtain the  best unsupervised R-1, R-L, and BS baseline scores.  
When naively constructing different document structures from the CanLII dataset without header removal, using NLP algorithms (rows 9 - 10) versus just using the HTML formatting (row 8) generally degraded results. However, when we  experimented with a modified version of the input documents (rows 11-13) where the headers were filtered through heuristics before computing the document structure, %and keyword matching and sections are reorganized 
%(see Section \ref{sec:header_removing}), %the C99 model (row 12) outperformed just using the HTML (row 11).  In addition, 
the  scores in rows 11-13 were higher (or in one case the same) than the comparable scores in rows 8-10.  
Also, without headers, the C99  topic segmentation algorithm (row 12) now outperforms the use of HTML (row 11) (obtaining an average improvement of 0.5 points across ROUGE and 0.8 for BS), suggesting
 that better  structures can  improve summarization. 
% As shown in row 11-12 in third block of Table \ref{tab:unsupervised_results}, the original document structure outperforms the other two views on the CanLII dataset and HMM-Stage obtains best performance on BillSum. 
% Through visualization of the IRC and selected sentence positions in the CanLII document (Fig \ref{fig:initial_study}), 
% we find that the stage-based segmentation perform better on selecting sentences appeared in the middle of the text (green ovals in the bottom right sub figure) while the topic-focused one succeeded in more accurately capturing the ending "conclusion" sentences (two green ovals in top-left sub figure). 
% \begin{figure}
% \centering
%  \includegraphics[width=\linewidth]{Figs/1654088145579.jpg}
%   \caption{Initial study on selected sentence positions from different views.}
%   \label{fig:initial_study}
% \end{figure}
 As shown in Table \ref{tab:segmentation_stats} (and earlier in Figure \ref{fig:three_view}), C99  creates many small sections (average number of sentences per section is 3.39 with standard deviation  of 0.67).
 We hypothesize that this encourages the selection of sentences from more fine-grained segments. In contrast, the other two methods create lengthy sections (average of more than 50 sentences) with a large standard deviation (135.40 for original structure without headers). 
%Also, the C99 model (row 12) now outperformed just using the HTML (row 11)
% With the updated CanLII dataset, we find that changing to different document structure, such as the topic segmentation, we 
In sum, with improvements in automatic metrics, we find that document structures play an important role in summarizing cases.
\begin{table}[t]
\small
    \centering
    \begin{tabular}{c|c|c}
    \toprule
        Model & avg. \# secs & avg. \# sents per sec  \\
        \midrule 
        \multicolumn{3}{c}{with header} \\
        \midrule
        Original Structure & 4.83 (6.44) &  83.82 (118.78) \\
        C99-topic & 63.47 (70.34) & 3.38 (0.71) \\
        HMM-stage & 4.00(0.83) &  54.32 (64.80) \\
        \midrule
         \multicolumn{3}{c}{without header} \\
         \midrule
        Original Structure & 3.67 (5.51) &  102.99 (135.40) \\
        C99-topic & 59.74 (69.91) & 3.39 (0.67) \\
        HMM-stage & 3.16 (1.08) &  70.19 (119.39)\\
        \bottomrule
    \end{tabular}
    \caption{Statistics about the average number of sections (avg. \# secs) and average number of sentences per section (avg. \# sents per sec) across the documents with different computed document structures (standard deviation in parenthesis).}
    \label{tab:segmentation_stats}
\end{table}

\textbf{RQ3}. The final block of Table \ref{tab:unsupervised_results} presents the reweighting  results (using the "w/o header" version of the CanLLI documents as they performed best in the prior block).  %We first implement the reweighting algorithm. 
By downweighting  sentences that appear under the same section as  previously selected ones, we observe an F1 improvement of 0.65, 0.65, and 0.75 on R-1, R-2, and R-L, respectively, on the previously best-performing topic segmented document (row 12 versus 15).
%Employing HipoRank with C99-topic segmentation and adding our reweighting algorithm 
Row 15 in fact has the best non-oracle results for all ROUGE scores.  This observation regarding the value of reweighting also holds for the original structure (row 11 vs. 14) and the HMM-stage segments (row 13 vs. 16). %\footnote{For ROUGE precision and recall, see Appendix \ref{appendix:viewFullResults}.}
%Overall, our approach gives an improvement of 2.2 on R-1, 1.5 on R-2 and 2.3 on R-L on the CanLII dataset (row 8 and row 15).   %Interestingly, the  proposed selection algorithm bring negative results on the BillSum dataset, an analysis on the position distribution of the extractive oracle sentences (see Figure \ref{fig:billsum_position_sorted} in Appendix) in the original article demonstrated that the salient information in the original article are following a uniform distribution, thus the proposed algorithm may discourage for better selection of candidates. 

% \begin{figure*}
% % \begin{adjustwidth}{-8em}{0em}
%   \begin{subfigure}[b]{0.7\textwidth}
%     \includegraphics[width=\textwidth]{Figs/Oracle_IRC_no_sort.png}
%     \caption{IRC Oracles}
%     \label{fig:}
%   \end{subfigure}
%   %
%   \begin{subfigure}[b]{0.7\textwidth}
%     \includegraphics[width=\textwidth]{Figs/positions_test_pacsum_no_order.png}
%     \caption{PacSum}
%     \label{fig:}
%   \end{subfigure}

%   \begin{subfigure}[b]{0.7\textwidth}
%     \includegraphics[width=\textwidth]{Figs/positions_test_hiporank_no_order.png}
%     \caption{HipoRank}
%     \label{fig:}
%   \end{subfigure}
%   %
%   \begin{subfigure}[b]{0.7\textwidth}
%     \includegraphics[width=\textwidth]{Figs/positions_test_theme1_1049_dynamic_no_order.png}
%     \caption{Ours}
%     \label{fig:}
%   \end{subfigure}
% \end{figure*}

\begin{figure*}[h]
        \centering
        \begin{subfigure}[b]{0.45\textwidth}
            \centering
            \includegraphics[width=\textwidth]{Figs/Oracle_IRC_no_sort.png}
            \caption[]%
            {{\small IRC Oracles}}    
            \label{IRCoracle}
        \end{subfigure}
        \hfill
        \begin{subfigure}[b]{0.45\textwidth}   
            \centering 
            \includegraphics[width=\textwidth]{Figs/positions_test_hiporank_no_order.png}
            \caption[]%
            {{\small HipoRank with headers}}    
            \label{fig:mean and std of net34}
        \end{subfigure}
        \vskip\baselineskip
        
        \begin{subfigure}[b]{0.45\textwidth}
            \centering
            \includegraphics[width=\textwidth]{Figs/positions_test_1049_noheader.png}
            \caption[]%
            {{\small HipoRank without headers}}    
            \label{IRCoracle}
        \end{subfigure}
        \hfill
        \begin{subfigure}[b]{0.45\textwidth}   
            \centering 
            \includegraphics[width=\textwidth]{Figs/positions_test_1049_noheader_dynamic.png}
            \caption[]%
            {{\small HipoRank without headers and with reweighting}}    
            \label{fig:mean and std of net34}
        \end{subfigure}
        \vskip\baselineskip
        %%%%%%%
        % \hfill
        % \begin{subfigure}[b]{0.485\textwidth}   
        %     \centering 
        %     \includegraphics[width=\textwidth]{Figs/positions_test_1049_noheader.png}
        %     \caption[]%
        %     {{\small HipoRank without headers}}    
        %     \label{fig:mean and std of net44}
        % \end{subfigure}
        % % \vskip\baselineskip
        % \hfill
        % \begin{subfigure}[b]{0.485\textwidth}   
        %     \centering             \includegraphics[width=\textwidth]{Figs/positions_test_theme1_1049_dynamic_no_order.png}
        %     \caption[]%
        %     {{\small Our with Reweighting}}    
        %     \label{fig:mean and std of net44}
        % \end{subfigure}
        
        \caption[]
        {\small Sentence positions in source cases for extractive summaries generated by different models using the original document structure on the
 test set. For (b) (c) (d), documents on the x-axis are
sorted in the same order. For IRC Oracles, \textcolor{red}{Issue}, \textcolor{reasonblue}{Reasoning} and \textcolor{conclusiongreen}{Conclusion} sentences are colored accordingly. } 
        \label{fig:comparison}
    \end{figure*}
    
    Finally, to better understand the behavior of different enhancements to the HipoRank backbone model, Figure \ref{fig:comparison} visualizes the positions of IRC sentences in the original article that are selected by a particular summarization method.  Plot (a) shows that the human-annotated IRC sentences in the summary tend to span across the source documents, with issues appearing in the beginning and conclusions in the end. Plot (b) shows that although HipoRank using the original document structure can successfully pick middle section sentences, the darkest band at the starting positions shows that the model still heavily relies on the inductive bias to pick the beginning sentences. Plot (c)
shows that removing the headers reduces the  starting sentence bias.  Finally, plot (d) shows that 
%We find that 
reweighting  reduces the number of sentences appearing on both ends. Further analyses on the complete automatic evaluation results\footnote{See Appendix \ref{appendix:reweight_effects} for ROUGE precision and recall.} suggest that the improvements come from higher recall values.
%(as shown in Fig \ref{fig:comparison}). 


\subsection{Argumentative Sentence Coverage}\label{sec:salient_eval}
Taking advantage of the sentence-level IRC annotations, we propose recall metrics to better measure the summary quality from a legal argumentation perspective ({\bf RQ3}).  We compute the recall of ``IRC'' sentences extracted from the original case as source IRC coverage (src. IRC). % of the summarizer. 
We similarly compute the coverage of IRC sentences in the human-written summaries as target IRC coverage (tgt. IRC) and all sentences as target sentence coverage (trg. cov.). To do so we apply the oracle summarizer (Section \ref{sec:oracle}) to map the generated extractive summaries to the human-written abstractive summaries. % and report the corresponding recall values. 

%\footnote{This oracle summarizer obtains a recall of 84.0 on the IRC mappings among the case and human written summaries manually annotated on ten article pairs.} 

We report these values for the IRC oracle, %the best 
an unsupervised (LexRank) and supervised (BERT\_EXT) baseline, the  discourse-aware HipoRank with the original structure, and our best reweighting model using C99-topic segmentation. Table \ref{tab: salient_coverage} shows that our model obtains the highest target IRC recall and coverage, suggesting that the  summaries are more similar to the references with respect to %conveying 
the decision's argumentation. Another unsupervised model, LexRank, obtains the highest source IRC, but its off-the-shelf package requires a fixed sentence ratio selected from the source. %original article. 
This produced longer summaries than other approaches and thus captured more IRCs in the source. 

\begin{table}[t!]
\small
    \centering
    \renewcommand{\arraystretch}{1}%Tighter
    \begin{tabular}{c|c|c|c}
    %\begin{tabular}{l|lllc|lllc|p{0.9cm}ll}
    \toprule
  \textbf{Model} & \textbf{src. IRC} & \textbf{tgt. IRC} & \textbf{trg. cov.} \\
  \midrule 
  \multicolumn{3}{l}{\textit{Oracle}}\\
  \midrule
  IRC & 1 (0.00) & 0.918 (0.18)  &  0.820 (0.25) \\
%   EXT &   62.7 & 71.7 & 66.3  \\
  \midrule 
     \multicolumn{3}{l}{\textit{Baselines}}\\
  \midrule
  BERT\_EXT & 0.804 (0.27) & 0.846 (0.23) &  0.833 (0.23)\\
  LexRank & \textbf{0.912} (0.19) & 0.811 (0.26) &  0.800 (0.27)\\
%   PACSUM & & \\
  HipoRank & 0.800 (0.25) & 0.851 (0.24) & 0.844 (0.22) \\
   
  \midrule 
  {\it Ours} & 0.823 (0.26) & \textbf{0.866} (0.20) & \textbf{0.850} (0.21) \\
  \bottomrule

    \end{tabular}
    \caption{Average recall of IRC sentences matched in the original case (src. IRC), gold summary (tgt. IRC), as well as target sentences coverage (trg. cov.) for each document (standard deviation in parenthesis).}
    \label{tab: salient_coverage}
\end{table}


\subsection{Human Evaluation Discussion}
% TODO: add the newest paper citation.

% Another limitation is that current references are fixed, which could not address the prior work's expert feedback that there is a need for different kinds of summaries in the legal field, depending on the expert's intention of use.  

As a first step towards human evaluation, we tried to extend the HipoRank  setup in \citet{dong-etal-2021-discourse} and designed a human evaluation protocol as follows. We asked human judges\footnote{All judges should be native English speakers who are at least pursuing a JD degree in law school and have experience in understanding case law.} to read the human-written reference summary and presented  extracted sentences from different summarization systems. The judges were asked to evaluate a system-extracted sentence according to two criteria: (1) \textit{Content Coverage} -  whether the presented sentence contained content from the human summary, and (2) \textit{Importance} - whether the presented sentence was important for a goal-oriented reader even if it was not in the human summary\footnote{Here we assumed the goal-oriented reader as the lawyers or law students seeking information from the case.}. The sentence selection was anonymized and randomly shuffled. We used the same sampling strategy in \citet{dong-etal-2021-discourse} to pick ten reference summaries where the system outputs were neutral (i.e., had similar R-2 scores compared to the human reference). 
%The evaluators also provide valuable feedback on keeping track of the metadata about annotators' legal backgrounds, area specialization, and experience with lawsuits. 
However, initial annotation on a small set by a legal expert demonstrated that the selected sentences may not reflect the model's capability. Most sampled system outputs had low ROUGE-2 F1 scores compared to the reference (normally below 10\% while the average model performance is 17\%), and the human evaluator reported that some of the selected sentences were not meaningful. We thus propose that a more careful sampling technique will be required for legal annotation tasks such as ours. 

To further guide our future work, we also reviewed how prior legal domain research has performed human evaluations when automatically summarizing legal documents  \cite{polsley-etal-2016-casesummarizer, zhong-2019-iterativemasking, salaun-2022-jurix}. Due to the burden of reading lengthy original documents, as in our human evaluation, most prior work evaluated summary quality using reference summaries rather than  source documents. In addition,  legal evaluations have typically been small-scale %constrained by the scale 
(5-20  summaries) due to the need to have evaluators with particular types of expertise (e.g., law graduate students or law professors), which was a similar constraint in our exploratory human evaluation.  Researchers have also designed new types of legally-relevant evaluation questions that  evaluate the summary for task-specific properties that go beyond more typical properties such as grammar, readability, and style. In our case, we would like legal experts to assess IRC coverage in the future.

% In caseSummarizer, \citet{polsley-etal-2016-casesummarizer} recruited domain-experts (without further details) and designed a tiny scale of (5 summaries) human evaluations with six questions borrowed from meeting summarization \cite{5071230}. \citet{zhong-2019-iterativemasking} recruited a law professor to examine different system outputs of 20 cases with a single question ``whether the summary adequately identifies issues and resolutions''. More recently, in \cite{}, SALAUN three experts (nlp specialist and law graduate students) annotated 16 test instances with an intrinsic evaluation framework of 7 questions to evaluate the summary from the form properties (grammar, readability, and style) and summary adequacy properties. Despite the small scale, most prior attempts avoided including the original long cases, given the time expense and difficulty in capturing all details, and instead focused on comparing the summaries to the references. Such evaluations may not be able to address the sentence-level properties of the extracted summaries.

% \textcolor{blue}{For this task, to evaluate the selected sentences' quality, following the work of \citet{dong-etal-2021-discourse}, we design a human-evaluation protocol as follows:  to read the human-written reference summary and present those extracted sentences from different summarization systems. The judges are asked to evaluate the system-extracted sentence according to two criteria: (1) \textit{Content Coverage}  whether the presented sentence contains content from the human summary and (2) \textit{Importance} whether the presented sentence is important for a goal-oriented reader even if it is not in human summary\footnote{Here we assume the goal-oriented reader as the lawyers or law students seeking information from the Case.}. The sentence selection is anonymized and randomly shuffled, so the human judges will not know which system it comes from. Given that the total amounts of sentences are too large (each system summary can have 10-20 sentences), it is infeasible to annotate all sentences of the 1,049 test cases. We instead try to sample a small portion of the data. As did in \cite{dong-etal-2021-discourse}, we selected ten reference summaries coupled with a total of 281 system-extracted sentences, where 141 are from our best model (row 15 in Table \ref{tab:unsupervised_results}) and 141 from the hiporank baseline with original structure (row 11 of Table \ref{tab:unsupervised_results}). Criteria are applied for the selection, where both system outputs have similar ROUGE scores compared to the human reference but differ dramatically from each other for neutral comparison. However, initial annotation on a small set from the 281 tasks demonstrates that the selected sentences may not represent the whole dataset. Most sampled system outputs have low ROUGE-2 scores compared to the reference (normally below 10\%), and human evaluators also report that some of the selected sentences are not good enough. We thus propose that a more careful sampling technique is required for the annotation tasks of the legal domain. The evaluators also provide valuable feedbacks on keeping track of the metadata about annotators' legal background, the specialization of area, and the experience with lawsuits.  }



% TODO: (1) Ablation Study on different views' selected results and the way to aggregate. (2). Examine the strategies to weight different section-sentence edges. (3) show different variants of the multi-round revision and demonstrate the benefits. 


% !TeX spellcheck = en_GB
%!TEX root = ../side-constrained.tex

\section{Conclusion}

We provided a counterexample to a claimed existence result for dynamic equilibria with side constraints. The implications of this counterexample were shown to be severe since solutions to the canonical infinite dimensional variational inequality are in some sense useless and other approaches seem to be necessary. 
We then established a general framework for defining side-constrained dynamic equilibria based on two key objects: A \setS{} $S$ containing all feasible flows (given as walk inflows) and correspondences $A_p$ providing the flow-dependent set of \addmEpsDev s. We showed that this equilibrium concept not only encompasses the known unconstrained equilibria with and without departure time choice and capacitated dynamic equilibria with convex \setS{}s but also allows for a whole range of new dynamic equilibria inspired by static side-constrained equilibria.
We provided conditions under which they can be characterized as solutions to a quasi-variational or even a variational inequality. The latter characterization then also gave rise to a first existence result for certain side-constrained dynamic equilibria with convex \setS.
Finally, we turned to equilibria wherein the side-constraints are given by time-varying edge-load constraints. To deal with the non-convexity of the \setS{}, we employed an augmented Lagrangian approach by relaxing the hard edge-load-capacities and replacing them by penalty functions. We demonstrated that these existence results apply, in particular, for the widely used Vickrey point queue model as well as the linear edge delay model.

Several important questions remain open. First of all, it would be interesting to find an existence result for BSDE similar to \Cref{thm:ExistenceFDAddSpaceExCP} for LPDE and MNSDE. The main obstacle to obtaining such a result seems to be the fact that for BSDE, the definition of \addmEpsDev s involves the network loading which, in general, is a very complex mapping and, even for well-studied flow models, is not fully understood yet. Note that, due to \Cref{prop:RelationshipsOfCDE}, such a result would also directly imply existence of \globalEL{} as well as providing an alternative proof for the existence of LPDE. Another aspect is the multiplicity of equilibria and
the issue of selecting a particular type of equilibrium having desirable properties.
It is an interesting research direction to characterize equilibrium concepts
that admit equilibrium selection via appropriate optimization or optimal control reformulations
whose optimal solutions provide such desirable properties.
% \section{Controllable Summary Generation}
We frame the controllable summary generation as a natural language generation task. We compare our proposed approach with multiple baselines for both extractive and abstractive summarization in \S \ref{sec:result} using multiple automatic metrics. We find that the structure aware summary generation outperforms vanilla models. We additionally propose a salient rewarding metrics that can more precisely evaluate the coverage and adherence of generation result while giving the user provided prompt.

\subsection{Structure Guided Summary Generation}
From the corpus released by (cite Huihui's newest paper),  annotators are asked to annotate individual sentences in the summary following the IRC schema (described in \S dataset). Thus, the annotated data comes with a label sequence which represents the structure of the summary. One example can be found in Figure \ref{}. For the abstractive summarziation models, we add the annotated text sequence before the original texts, thus letting the model to implicitly map the structure sequence to the generation summary sentences.

% \subsection{Long Input Split}
% Most neural extrative summarizers are limited by the length of inputs, which is usually 512 tokens cite{BERT} and fall far below the average length of 5k tokens in the legal documents. It is impractical to concatenate all sentences in a document and encode them with a large pretrained model. Following \cite{mao2021dyle}, we group consecutive sentences into \textit{chunks}. We set the maximum length at 512 and allow for 20\% overlapping with the previous chunk. This way we could include the intermediate sentences in multiple chunks, wishing to capture local dependencies within the same section or paragraph. 

\subsection{Extractive Summarization}
\paragraph{Extractive oracles.} \textit{Extractive oracles} denotes a set of selected text snippets whose concatenation maximizes the evaluation metric given the gold summary. We first utilize the \textbf{IRC\_Oracle}, which sentences manually labelled with the IRC types are concatenated together to form a extractive summary. Following \cite{nallapati2017summarunner}, we additionally report \textbf{EXT\_ORACLE}, a baseline for an
oracle summarizer \cite{nallapati2017summarunner}, which greedily select sentences from the original article based on the ROUGE-L scores comparing to the abstractive summary.

\paragraph{Extractive Baselines}
We compare the proposed framework with various baselines. For unsupervised extractive summarization models, we compare with LSA  \cite{Steinberger2004UsingLS},
LexRank \cite{erkan2004lexrank}, TextRank \cite{DBLP:journals/corr/BarriosLAW16}, PACSUM \cite{zheng-lapata-2019-sentence}. We further include HipoRank \cite{dong-etal-2021-discourse}, which achieved the state of the art performance on scientific paper summarization, which also has long text inputs.
For supervised methods, BERTEXT \cite{liu-lapata-2019-text} uses a novel document-level encoder based on BERT which is able to express the semantics of a document and obtain representations for
its sentences. 

\subsection{Abstractive Summarization}
For abstracive summarization, we experiment with BART \cite{lewis-etal-2020-bart} and the Longformer-Encoder-Decoder (LED) \cite{Beltagy2020Longformer}. The latter model can process longer input document up-to 16k tokens. 

\subsection{Implementation Details}
For PACUSUM, we used the released bert checkpoint\footnote{\url{https://github.com/mswellhao/PacSum}}.For BERTEXT without fine-tuning, we use the released checkpoint\footnote{https://github.com/nlpyang/PreSumm} trained on CNN$/$DailyMail Dataset and inference the extracted sentence from each chunk. One limitation of training the BERTEXT model is that the BERT model can only process input tokens up to 512. Following \cite{mao2021dyle}, we group consecutive sentences of a long document into \textit{chunks}. We set the maximum length at 512 and allow for 20\% overlapping with the previous chunk. This way we could include the intermediate sentences in multiple chunks, wishing to capture local dependencies within the same section or paragraph. We additionally employ the similar greedy algorithm in EXT\_ORACLE to construct the extractive labels for model training. Due to the limitation of computing resources, we set the maximum input length as 1,024 and 6,144 for BART-large and LED model respectively. For neural based models, we finetune models for at most 5 epochs with a learning rate of 2e-5, batch size of 1 with gradient accumulation steps at 4. We save the checkpoints at every 1,000 steps, utilizing early stopping based on ROUGE-2 F1 scores and report the average of 3 best checkpoints on validation set. 

\subsection{Evaluation Metrics}
For automatic evaluation, we report ROUGE-1, ROUGE-2, and ROUGE-LSum. We additionally report BERTScore \cite{bert-score} to measure the semantic meanings of the generated summary. 
\section{Results and Analysis}\label{sec:result}
We discuss the evaluation results and effects in this section and aim to answer multiple research questions.\\
\textbf{RQ1: How well can the extractive methods perform while comparing against the human-written summaries}
Table \ref{tab:all_result} shows the evaluation results on the test set. In row 1 and 2, IRC\_ORACLE can be viewed as an upper-bound for an extractive model that could accurately capture all ``Issue Reason Conclusion'' sentences in the original article. The high rouge scores imply that there exist a substantial amount of content overlapping when legal experts draft the abstractive summary, and thus extractive methods could produce high-quality candidates. \\
\textbf{RQ2: How do the unsupervised and supervised baselines work on the legal texts}
row 3-8 demonstrate the performances of frequently used unsupervised baselines. We observe that the performances are not as good as the supervised abstractive methods (row 12 and 14). Meanwhile, the traditional approaches such as LexRank and TextRank suffer from the redundant contents selection (700 tokens v.s. the reference's 240 tokens in selected sentences).\\
\textbf{RQ3: would the incorporation of longer input bring benefits for supervised extractive summarziation?}
Since there are overlapping between different chunks of the test file, we propose a simple method to post-processing all extracted sentences from the same article but within different chunks and name it as BERT\textsubscript{EXT} - selection. (We group all sentences together, de-duplicate and sort the sentences based on their original order in the article). This brings over 20 ROUGE-1 improvements, 8.7 ROUGE-2 improvements, and 18.77 ROUGE-LSum improvements comparing to the baseline model which is only exposed to the first 512 tokens. This suggests that the truncation of texts would result in the loss of information. \\
\textbf{RQ4: Whether the proposed structure-based generation lead to better performance?} We mainly compare the model's performance on the abstractive summarization in row 12-15. We observe stable improvements across all three ROUGE scores and minor improvements on BertScore. This suggests that the summaries generated by our approach is closer to the oracle references, thus being more readable and understandable. \\
Meanwhile, to 

\textbf{RQ5: Would more training data help?} The original CANLLI dataset comes with around 28k case and summary pairs. To validate this hypothesis, we retrain two baseline models of BART and LED on the non-overlapped 27k pairs that are not annotated with IRC, and report the model's performance in row 16 and 17. Not surprisingly, around 5 points improvements in all ROUGE scores are obtained, suggesting that the larger training data could benefit the model's capbilty to better summarize the salient information in long legal case documents.
\begin{table*}[h!]
\small
% \scriptsize 
    \centering
    % \setlength\tabcolsep{2.1pt}
    \renewcommand{\arraystretch}{1.1}%Tighter
    \begin{tabular}{c|l|ccc|ccc|c}
    %\begin{tabular}{l|lllc|lllc|p{0.9cm}ll}
    \toprule
    ID &  & \multicolumn{3}{c}{\textbf{Quality}} & \multicolumn{3}{|c}{\textbf{BertScore}} &\multicolumn{1}{|c}{\textbf{Length Metrics}} \\
    
      &   &   R-1 & R-2 & R-L  & F1 &  P  &  R & Abs. Len \\
         \midrule
         & Reference &  -- &	 -- &	 --  & -- & -- & -- & 240.31  \\
         
         \midrule
         \multicolumn{9}{c}{\textbf{Extractive Methods}} \\
         \midrule
         1 & IRC\_ORACLE & 57.92 & 35.72 & 55.05 &88.15& 87.78& 88.57 & 243.09 \\
        2 &  EXT ORACLE (R-L, F1) & 61.41 & 40.44 & 59.08 & 88.19& 87.51& 88.97 & 209.37 \\
         \midrule 
         
           \multicolumn{9}{l}{\textit{unsupervised}} \\
           3 & LSA \cite{Steinberger2004UsingLS} & 38.15 & 18.22 & 35.76 & 84.63& 83.48& 85.84 & 787.98  \\
           4 & LexRank \cite{erkan2004lexrank} & 38.63 & 17.70 & 35.87 & 84.40& 83.47& 85.39 & 705.56 \\
            5 & TextRank \cite{DBLP:journals/corr/BarriosLAW16}  & 36.70 & 16.19 & 34.00 & 83.53& 82.20& 84.95 & 722.46  \\
            6 & PACSUM \cite{zheng-lapata-2019-sentence} & 39.24 & 14.31 & 36.80 & 81.64& 79.80& 83.60 & 188.44\\
           7 & PACSUM - LEGAL\_BERT & 37.83 & 13.50 & 35.42  &   81.54& 79.67& 83.53& 188.21 \\
           8 & HipoRank \cite{dong-etal-2021-discourse} & 42.85 & 17.75&  40.08 & 83.53& 82.10& 85.05 & 228.09 \\
           \midrule
        %   \multicolumn{9}{l}{\textit{supervised}} \\
        %   BERT\textsubscript{EXT} - selection & -- & -- & -- \\
           
        %  \midrule
             \multicolumn{9}{l}{\textit{w/o fine-tuning}} \\
        9 &  BERT\textsubscript{EXT} - first 512 token & 18.32 & 7.31& 16.76 &81.77& 82.88& 80.75 & 43.41  \\
      
         10 & BERT\textsubscript{EXT} - selection & 39.14 & 16.38 & 36.40 & 83.84& 82.71& 85.04 & 581.45 \\
          
         \midrule
          \multicolumn{9}{l}{\textit{w/ fine-tuning}} \\

          11 & BERT\textsubscript{EXT} - selection (ckpt 30000) & 40.78 & 17.63 & 37.75 & 84.05& 83.00& 85.19 & 588.60\\
          \midrule
         \multicolumn{9}{c}{\textbf{Abstractive Methods}} \\
         \midrule
       12 & LED   & 49.56 & 23.78 & 46.48  & 86.75& 87.00& 86.56 & 199.47  \\
      
       13 & LED + prompt &  50.69	& 24.33	& 47.53 & 86.87& 86.84& 86.95 & 223.58  \\
       14 &  BART  & 47.93 & 22.34 & 44.74 &  86.16& 85.59& 86.81 & 284.08 \\
        15& BART  + prompt & 51.44 & 23.29 & 47.99  &86.51& 86.12& 86.95 & 250.40 \\
       
       \midrule 
       \multicolumn{9}{l}{\textit{w/ larger training data}}\\
       16 &  BART + larger train & 50.47 & 25.15 & 46.60 & 87.27& 87.77& 86.82 & 172.86\\
       17 &  LED + larger train & 54.41 &  29.57 & 51.17 & 87.86& 87.82& 87.95 & 234.16 \\
       

        % \midrule
        % \multicolumn{9}{l}{\textit{Different Styles}} \\
        %  Style-1 &  48.33 &	22.88 &	45.05 & 0.9613  &  9.16  &  0.74 & 210.60 & 0.1183 \\
        %  Style-2 &  48.28 &	22.61 &	44.99 & 0.9623  &  9.16  &  0.74 & 214.57 & 0.1192 \\
        %  Style-3 &  41.89 &	18.83 &	39.16 & 0.9621  &  8.29  &  0.73 & 142.75 & 0.0808 \\
        % Style-4 &  48.61 &	\textbf{23.24} &	{45.61} & 0.9627  &  9.56  &  0.74 & 223.58 & 0.1212 \\
           
        % \midrule
        
        \bottomrule
        
    \end{tabular}
    \caption{Model Performance for the original test set with 106 instances from the 1049 annotated dataset.}
    \label{tab:all_result}
\end{table*}

% OLD TABLE with abstractedness and density. 
% \begin{table*}[t!]
% \small
% % \scriptsize 
%     \centering
%     % \setlength\tabcolsep{2.1pt}
%     \renewcommand{\arraystretch}{1.1}%Tighter
%     \begin{tabular}{l|ccc|ccc|cc}
%     %\begin{tabular}{l|lllc|lllc|p{0.9cm}ll}
%     \toprule
%     & \multicolumn{3}{c}{\textbf{Quality}} & \multicolumn{3}{|c}{\textbf{Abstractiveness}} &\multicolumn{2}{|c}{\textbf{Length Metrics}} \\
    
%          &   R-1& R-2 & R-L  & Coverage  &  Density  &  2G-overlap & Abs. Len & Comp. Ratio \\
%          \midrule
%          Reference &  -- &	 -- &	 --  & 0.9338  &  7.50  &  0.61 & 240.31 & 0.1242 \\
         
%          \midrule
%          \multicolumn{9}{l}{\textbf{Extractive Methods}} \\
%          IRC\_ORACLE & 57.92 & 35.72 & 55.05 & 0.6559 & 2.18 & 0.18 & 243.09 & 0.1511\\
%          EXT ORACLE & 61.41 & 40.44 & 59.08 \\
%          \midrule 
%              \multicolumn{9}{l}{\textit{w/o fine-tuning}} \\
%          BERT\textsubscript{EXT} - first 512 token & 18.32 & 7.31& 16.76 & 0.6184 & 2.05 & 0.20 & 43.41 & 0.0260 \\
      
%           BERT\textsubscript{EXT} - selection & 39.14 & 16.38 & 36.40 & 0.6340 & 2.19 & 0.16 & 581.45 & 0.3674\\
%           \midrule
%           \multicolumn{9}{l}{\textit{unsupervised methods}} \\
%             TEXTRANK \cite{DBLP:journals/corr/BarriosLAW16}  & 32.82 & 16.37 & 30.81 \\
%             PACSUM \cite{zheng-lapata-2019-sentence} & 38.10 & 14.14 & 35.82 \\
%             PACSUM - LegalBert & 37.60 & 13.58 & 35.22 \\
%           \midrule
%           \multicolumn{9}{l}{\textit{supervised}} \\
%           BERT\textsubscript{EXT} - selection & -- & -- & -- \\
           
%          \midrule
%          \midrule
%          \multicolumn{9}{l}{\textbf{Abstractive Methods}} \\
%       LED  -- no prompt & 49.56 & 23.78 & 46.48  & 0.9655  &  10.61  &  0.77 & 199.47 & 0.1097 \\
%         LED &  50.69	& 24.33	& 47.53 & 0.9627  &  9.56  &  0.74 & 223.58 & 0.1212 \\
%         BART -- no prompt & 47.93 & 22.34 & 44.74 &  0.9508 & 8.83 & 0.68 & 284.08 & 0.1609\\
%         BART  & 51.44 & 23.29 & 47.99  & 0.9507 &  8.49  &  0.68 & 250.40 & 0.1402 \\
        
%         \midrule
%         \multicolumn{9}{l}{\textit{Different Styles}} \\
%          Style-1 &  48.33 &	22.88 &	45.05 & 0.9613  &  9.16  &  0.74 & 210.60 & 0.1183 \\
%          Style-2 &  48.28 &	22.61 &	44.99 & 0.9623  &  9.16  &  0.74 & 214.57 & 0.1192 \\
%          Style-3 &  41.89 &	18.83 &	39.16 & 0.9621  &  8.29  &  0.73 & 142.75 & 0.0808 \\
%         Style-4 &  48.61 &	\textbf{23.24} &	{45.61} & 0.9627  &  9.56  &  0.74 & 223.58 & 0.1212 \\
           
%         % \midrule
        
%         \bottomrule
        
%     \end{tabular}
%     \caption{Model Performance}
%     \label{tab:all_result}
% \end{table*}


% \begin{table*}[t!]
% \small
% % \scriptsize 
%     \centering
%     % \setlength\tabcolsep{2.1pt}
%     \renewcommand{\arraystretch}{1.1}%Tighter
%     \begin{tabular}{l|ccc|ccc|cc}
%     %\begin{tabular}{l|lllc|lllc|p{0.9cm}ll}
%     \toprule
%     & \multicolumn{3}{c}{\textbf{Quality}} & \multicolumn{3}{|c}{\textbf{BertScore}} &\multicolumn{2}{|c}{\textbf{Length Metrics}} \\
    
%          &   R-1 & R-2 & R-L  & P  &  R  &  F1 & Abs. Len & Comp. Ratio \\
%          \midrule
%          Reference &  -- &	 -- &	 --  & 0.9338  &  7.50  &  0.61 & 240.31 & 0.1242 \\
         
%          \midrule
%          \multicolumn{9}{c}{\textbf{Extractive Methods}} \\
%          \midrule
%          IRC\_ORACLE &  -- & -- & -- & \\
%          EXT ORACLE (R-L, F1) &  -- & -- & -- & \\
%          \midrule 
         
%           \multicolumn{9}{l}{\textit{unsupervised methods}} \\
%           LSA \cite{Steinberger2004UsingLS} & -- & -- & -- \\
%           LexRank \cite{erkan2004lexrank}  & -- & -- & -- \\
%             TextRank \cite{DBLP:journals/corr/BarriosLAW16}  & -- & -- & -- \\
%             PACSUM \cite{zheng-lapata-2019-sentence}  & -- & -- & -- \\
            
%             HipoRank \cite{dong-etal-2021-discourse} &  & -- & -- & --\\
%           \midrule
%         %   \multicolumn{9}{l}{\textit{supervised}} \\
%         %   BERT\textsubscript{EXT} - selection & -- & -- & -- \\
           
%         %  \midrule
%              \multicolumn{9}{l}{\textit{w/o fine-tuning}} \\
%           BERT\textsubscript{EXT} - selection & 39.14 & 16.38 & 36.40 & 0.6340 & 2.19 & 0.16 & 581.45 & 0.3674\\
          
%          \midrule
%          \multicolumn{9}{c}{\textbf{Abstractive Methods}} \\
%          \midrule
%       LED  -- no prompt &  53.72 & 28.75 & 50.17 & 87.55& 87.38& 87.77   \\
%         BART -- no prompt & 50.50 & 25.58 & 46.82 & 87.25& 87.71& 86.84 \\
        
%         \bottomrule
        
%     \end{tabular}
%     \caption{Model Performance trained on 27241 pairs and evaluated on all 1049 annotated sentences.}
%     \label{tab:all_result}
% \end{table*}



% \section*{Acknowledgements}

% This document has been adapted
% by Steven Bethard, Ryan Cotterell and Rui Yan
% from the instructions for earlier ACL and NAACL proceedings, including those for 
% ACL 2019 by Douwe Kiela and Ivan Vuli\'{c},
% NAACL 2019 by Stephanie Lukin and Alla Roskovskaya, 
% ACL 2018 by Shay Cohen, Kevin Gimpel, and Wei Lu, 
% NAACL 2018 by Margaret Mitchell and Stephanie Lukin,
% Bib\TeX{} suggestions for (NA)ACL 2017/2018 from Jason Eisner,
% ACL 2017 by Dan Gildea and Min-Yen Kan, 
% NAACL 2017 by Margaret Mitchell, 
% ACL 2012 by Maggie Li and Michael White, 
% ACL 2010 by Jing-Shin Chang and Philipp Koehn, 
% ACL 2008 by Johanna D. Moore, Simone Teufel, James Allan, and Sadaoki Furui, 
% ACL 2005 by Hwee Tou Ng and Kemal Oflazer, 
% ACL 2002 by Eugene Charniak and Dekang Lin, 
% and earlier ACL and EACL formats written by several people, including
% John Chen, Henry S. Thompson and Donald Walker.
% Additional elements were taken from the formatting instructions of the \emph{International Joint Conference on Artificial Intelligence} and the \emph{Conference on Computer Vision and Pattern Recognition}.

% % Entries for the entire Anthology, followed by custom entries
\bibliography{custom}
\bibliographystyle{acl_natbib}

\appendix

% \section{Example Appendix}
% \label{sec:appendix}
\newpage
\clearpage
\section{The HipoRank Algorithm}\label{sec:hiporank}
In this section, we provide a detailed recap  of the HipoRank algorithm \cite{dong-etal-2021-discourse}. Our approach mainly modifies the obtained document graphs by building a \textit{section-section} graph and changes the final summary selection algorithms.  
\subsection{Hierarchical Document Graph Creation}
The document is first split into its sections, then into sentences. Two levels of connections are allowed in the built hierarchical graph: intra-sectional connections and inter-sectional connections. Following the original paper, we display a toy example of these two types of connections in Figure \ref{fig:hiporank_example}.
 \begin{figure}[h]
\centering
 \includegraphics[width=\linewidth]{Figs/hiporank_demonstration.jpg}
  \caption{ (Reproduced from \cite{dong-etal-2021-discourse}) An example of a hierarchical document graph 
constructed by HipoRank approach on a toy document, which
contains two sections \{T1, T2\}, each containing three
sentences for a total of six sentences \{s1, . . . , s6\}. In the graph, 
each double-headed arrow represents two edges with
opposite directions. The solid and dashed arrows indicate intra-section and inter-section connections respectively.}
  \label{fig:hiporank_example}
\end{figure}

\paragraph{Intra-sectional connections} are designed to measure a sentence's importance score inside its section. The authors built a fully-connected subgraph over all sentences in the same section, allowing for \textit{sentence-sentence} edges, which are measured by a weighted version of the similarities of sentence embeddings.  
\paragraph{Inter-sectional connections} ``aim to model the
global importance of a sentence with respect to
other topics/sections in the document'', according to \citet{dong-etal-2021-discourse}. To reduce the expensive computation of all sentence-sentence connections spanning across a document, HipoRank's authors introduce section nodes on top of sentence nodes, and only allow for \textit{sentence-section} edges to model the global information. 
\subsection{Asymmetric Edge Weighting by
Boundary Functions}
In order to compute the weight of an edge, HipoRank measures the similarity of sentence-sentence pairs by computing the cosine similarity of encoded sentence embeddings. Similarly, for sentence-section pairs, it averages the sentences' representations in the same section, uses it as the section vector, and further computes the cosine similarity. Taking two discourse hypotheses of long scientific documents into account ((1) important sentences are near the
boundaries (start or end) of a text \cite{5392648} and (2) sections near the text boundaries (start or end) are more important \cite{teufel-1997-sentence}), the authors of HipoRank capture this asymmetry by making their hierarchical graph directed and inject asymmetric edge
weighting over intra-section and inter-section connections. We refer to the original paper for more detailed setups and algorithm details.
\subsection{Importance Computation and Summary Generation}
We talk about the importance computation approach and summary generation details in \S \ref{sec:main_paper_hiporank}. 
%in a document, the centrality score of each sentence as 
% \begin{equation}
%     c(s_i^{I}) = \mu_1 c_{inter} (s_i^{I}) + c_{intra}(s_i^{I})
% \end{equation} where $s_i^{I}$ refers to the $i$-th sentence in $I$-th section. $\mu_1$ is a tunable hyper-paramter, $c_{inter}(s_i^{I})$ computes the sentence's similarity to other section representations and  $c_{intra}(s_i^{I})$ computes the average similarity of the current sentence with all others in the same section. the algorithm greedily selects the sentence based on the ranked scores, adding it into the summary set until a pre-defined  word-limit L is passed. 


\section{Training Details and Hyperparameters}\label{appendix:detail}
All of our experiments are conducted on Quadro RTX 5000 GPUs, each of which has 16 GB RAM. For the extractive oracle baseline, we use the python package of rouge\footnote{\url{ https://pypi.org/project/rouge/}} to compute the ROUGE-L scores for sentence scoring. \paragraph{Document Segmentation}
We provide details of segmentation methods mentioned in \S \ref{sec:views} below. For sentence encoding, we use the sentence\_transformer library\footnote{\url{https://www.sbert.net/}}, and the checkpoint of ``bert-base-nli-stsb-mean-tokens'' for sentence representations. For the HMM stage segmentation, we train a GaussianHMM model with hmmlearn\footnote{\url{https://hmmlearn.readthedocs.io/en/latest/}}, setting the number of the components at 5 and train the model for 50 iterations on the validation set. For C99 algorithm, we use an implementation\footnote{\url{https://github.com/GT-SALT/Multi-View-Seq2Seq/blob/master/data/C99.py}} shared from \citet{chen-yang-2020-multi} in their original paper. We set the window size of 4 and std\_coefficient as 1. All data processing scripts are publicly available in a combined package in \url{https://github.com/cs329yangzhong/DocumentStructureLegalSum}. 

\paragraph{Supervised Model} We build our BERT\_EXT, the extractive model, on top of the official code base of the work of \citet{liu-lapata-2019-text}\footnote{\url{https://github.com/nlpyang/PreSumm}}. Since many original documents' lengths  go beyond the 512 token limits, we break the full document into different chunks and train the model to extract the top-3 sentences. For hyperparameters, we use 4 GPUs, set the learning rate of 2e-3, and save the best checkpoints at every 5,000 steps. We set the batch size as 3,000, the maximum training step at 100,000, and warm-up steps at 10,000.

\paragraph{Unsupervised Models} 
We use off-the-shelf packages for most traditional models. We use LSA\footnote{\url{https://github.com/luisfredgs/LSA-Text-Summarization}}, TextRank\footnote{\url{https://github.com/summanlp/textrank}}, and LexRank\footnote{\url{https://github.com/crabcamp/lexrank}} accordingly.

For PACSUM model, we follow the re-implementation\footnote{\url{https://github.com/mirandrom/HipoRank}} of \cite{dong-etal-2021-discourse} and keep the hyperparameters fixed with the original setup. BERT-based sentence embeddings are extracted using the fine-tuned BERT model released from the original paper \cite{zheng-lapata-2019-sentence}. We also experimented with LEGAL-BERT \cite{chalkidis-etal-2020-legal} in the early stages of our research but found it degraded performance on the baselines. 

For HipoRank, we use the publicly available code base\footnote{\url{https://github.com/mirandrom/HipoRank}}.  We experimented with various hyperparameter settings on the validation set but we find that
the original hyperparamters used in the original paper for PubMed dataset seem to be the most
stable and produce the best results. ($\lambda_1 = 0.0$, $\lambda_2 = 1.0$, $\alpha = 1.0 $, with $\mu_1 = 0.5$.)

We build our reweighting model on top of the HipoRank dataset. We search the threshold g (for phase transition between phases 1 and 2) between [0.3, 0.5, and 0.7], finding that 0.5 is the best for the CanLII dataset. 


% \section{BillSum Sent Location}
% Figure \ref{fig:billsum_position} demonstrates that the summaries in the billsum dataset has a messy distribution. We compute the extractive labels using the oracle summarizer based on the human-written abstractive summary.
% \begin{figure}
% \centering
%  \includegraphics[width=\linewidth]{Figs/positions_test_billsum.png}

%   \caption{BillSum selected sentence positions with extractive oracle, Documents on the x-axis are ordered by increasing article length from shortest to longest.}
%   \label{fig:billsum_position_sorted}
% \end{figure}

% \begin{figure}
% \centering
%  \includegraphics[width=\linewidth]{Figs/positions_test_1049_extractive.png}

%   \caption{CanLII selected sentence positions with extractive oracle, Documents on the x-axis are ordered by increasing article length from shortest to longest.}
%   \label{fig:canlii_position_sorted}
% \end{figure}

%\section{Quantitative Views Analysis}
\section{The Effects of Reweighting Algorithm}\label{appendix:reweight_effects}
We  study the effects of our reweighting algorithm by comparing different models' performances on the input documents with original structures. As shown in Table \ref{tab:original_structure_p_r_f1}, with a minor sacrifice of precision, the recall values are greatly improved with the reweighting algorithm, thus resulting in the final improvements of F1 scores.

\begin{table*}[]
    \centering
    \begin{tabular}{c|ccc|ccc|ccc}
    \toprule
   Document Structures & \multicolumn{3}{c}{ROUGE-1} & \multicolumn{3}{c}{ROUGE-2} & \multicolumn{3}{c}{ROUGE-L} \\
    \midrule 
       & P & R & F1 & P & R & F1 &P & R & F1  \\
    %   \midrule .
       \midrule
       w/o header  &  45.24 & 47.39 & 42.58 & 19.23 & 20.12 & 18.01 & 42.25 & 43.95 & 39.63 \\
       \midrule 
       w/o header + Reweighting & 44.13 & 49.80 & 43.14 & 18.97 & 21.35 & 18.46 & 41.29 & 46.26 & 40.23 \\
      \bottomrule 
    \end{tabular}
    \caption{The Precision (P), Recall (R) and F1 of ROUGE-1/2/L scores for the inputs with original document structures, with and without reweighting algorithm. We find that the reweighting algorithm improves the recall, suggesting that more argumentative sentences in the references are covered. }
    \label{tab:original_structure_p_r_f1}
\end{table*}

% \subsection{Different Views Effects on Our Approach}\label{appendix:viewFullResults}
% We look at the Precision, Recall and F1 values of the best reweighting models trained with different document structures. As shown in Table \ref{tab:three_view_p_r_f1}, the C99-topic segmented structure leads to higher Recall values of all three metrics, which suggests that models trained on topic segmented documents produced higher coverage of n-grams that have appeared in reference summaries. It also brings improvements on Precision values, suggesting a more accurate selection of sentences regarding the overlapped n-grams between candidates and references. 
% \begin{table*}[]
%     \centering
%     \begin{tabular}{c|ccc|ccc|ccc}
%     \toprule
%   Document Structures & \multicolumn{3}{c}{ROUGE-1} & \multicolumn{3}{c}{ROUGE-2} & \multicolumn{3}{c}{ROUGE-L} \\
%     \midrule 
%       & P & R & F1 & P & R & F1 &P & R & F1  \\
%     %   \midrule .
%       \midrule
%       Original Structure & 44.13 & 49.80 & 43.14 & 18.97 & 21.35 & 18.46 & 41.29 & 46.26 & 40.23 \\
%       \midrule
%       C99-topic  & 44.81 & 50.71 & 43.90 & 19.06 & 21.63 & 18.67 & 41.97 & 47.18 & 41.00\\
%       \midrule 
%       HMM-stage & 44.33 & 49.88 & 43.28 & 18.28 & 20.06 & 17.80 & 41.52 & 46.37 & 40.40\\
%       \bottomrule 
%     \end{tabular}
%     \caption{The Precision (P), Recall (R) and F1 of ROUGE-1/2/L scores for the best reweighting model with different document segmentation methods. }
%     \label{tab:three_view_p_r_f1}
% \end{table*}

\section{Examples}\label{appendix:1}

\subsection{Summary Generation Results}\label{appendix:output}
We show the reference, best baseline, and our model's output on the C99-topic view of the without header version of documents in Table \ref{tab:canlii_example}.
\begin{table*}
\small
\begin{tabular}{l|l}
\toprule
    Model &  Summary \\
    \midrule
    \multirow{10}{*}{Reference} & FIAT: The defendants, Sims, Garbriel and Dumurs, bring separate motions, pursuant to \\
    & Queen's Bench Rule 41(a), for severance of the claims against them or for an order \\
    & staying the claims against them until the plaintiffs' claim against the primary defendant,  \\
    & Walbaum have been heard and decided. || HELD: 1) || The Court will look at all of the \\
    & circumstances in deciding whether to grant an application for severance. || In this case \\
    & the plaintiffs should not be precluded from adducing evidence related to Walbaum's \\
    & dealings with each of the applicants or required to segregate the evidence into two, \\
    & three or four separate trials. || Given the likelihood that the applicants will be required \\
    & to attend portions of the trial in respect of the Walbaum Group in any event, severance \\
    & would not necessarily result in a significant saving of time and expense. || 2) The plain- \\
    &-tiffs acknowledge that only relatively small portion of trial time (perhaps less than 1 \\
    & day) will pertain to the claims against any one of the Sims, Gabriel or Dumurs. || It wou-  \\
    & -ld be unfair to require all of the applicants to participate in all of the trial when very little \\
    & of it will be relevant to them. || Specific dates and times should be set aside for the \\
    & plaintiff to call evidence with respect to its claims against each applicant group. || The \\
    & applicants should be relieved from attending the trial at any other time. \\
    \midrule
    \multirow{13}{*}{HipoRank} & QUEEN’S BENCH FOR SASKATCHEWAN ||
2007 SKQB 296 || Judicial Centre: Regina \\
&|| DUN-RITE PLUMBING \& HEATING LTD. || (d) Robert Dumur || 593340 Saskatchewan \\
& Ltd., carrying on || business as Dumur Industries (herein “the Dumurs”) || [2] The Sims, Gab-\\
& -riel and the Dumurs bring separate motions, pursuant to Queen’s Bench Rule 41(a) for \\
& severance of the claims against them or for orders staying the claims until the plaintiffs’ claims \\ 
& against the Walbaum Group have
been heard and decided. || ANALYSIS || [12] T applications \\
& are brought pursuant to Queen’s Bench Rule 41 which states: || 41 (1) Where the joinder of \\
& multiple claims or parties in the same action may unduly complicate or delay the trial, or cause \\
& undue prejudice to a party, the court may: || (a) order separate trials; || [21] It will also be left to \\
& the trial judge (or the pre-trial management judge) to designate specific days on which defence \\
& evidence may be adduced during the trial and argument presented with respect to each claim. || \\
& All defendants other than those comprising the Walbaum Group shall be relieved from attending \\
& the trial on any date not designated by counsel for the plaintiffs or designated by the trial judge \\
& for adducing defence evidence and presenting argument. || [22] Costs will be in the cause.
|| D.P. \\
& Ball \\
    \midrule
    \multirow{10}{*}{Ours} & FIAT BALL J. || August 14, 2007 || [1] The plaintiff brings this action against nine  defendants \\
    & (the claim against the defendant Albert Fazakas has been discontinued) who can be separated \\
    & into four groups: 
    || All-Rite Plumbing Heating Ltd. || [18] Although choeunsel for the plaintiffs \\
    & asserts that the evidence against all of the defendants can be adduced in no more than two and \\
    & one-half days, given the number and complexity of the claims against the Walbaum Group this \\
    & estimate seems very unrealistic. || [19] The plaintiffs acknowledge that only relatively small \\
    & portion of trial time (perhaps less than one day) will pertain to the claims against any one of the \\
    & Sims, Gabriel or the Dumurs. || It would be unfair to require all of the applicants to participate \\
    & in all of the trial when very little of it will be relevant to them. || The applicants should be \\
    & relieved from attending the trial at any other time. || The plaintiffs shall not call evidence in  \\
    & respect of those claims on any other date without leave of the court. || All defendants other than \\
    & those comprising the Walbaum Group shall be relieved from attending the trial on any date not \\
    & designated by counsel for the plaintiffs or designated by the trial judge for adducing defence \\
    & evidence and presenting argument. \\
\bottomrule
\end{tabular}
\caption{Generated summaries for a CanLII case decision (ID: 2\_2007skqb296), we use special symbol ``||'' to mark the sentence boundaries.}
\label{tab:canlii_example}
\end{table*}

% \begin{table*}
% \begin{tabular}{l|l}
% \toprule
% Model & Summary \\
% \midrule
%      \multirow{5}{*}{Reference} &  The amendments made by this section shall apply to taxable years beginning after \\ 
%     &  the date of the enactment of this Act.')","National Science Education Tax Incentive  for  \\
%     &  Businesses Act of 2007 - Amends the Internal Revenue Code to allow a general business   \\
%     & tax credit for contributions of property or services to elementary and secondary schools   \\
%     & and for teacher training to promote instruction in science, technology, engineering,  \\
%     & or mathematics .\\
%     \midrule
%     \multirow{10}{*}{HipoRank}  & <SECTION-HEADER> SHORT TITLE.
% This Act may be cited as the ""National Sc-  \\
% & 
% -ience Education Tax Incentive for Businesses Act of 2007"". CREDITS FOR CERTAIN \\
% & CONTRIBUTIONS BENEFITING  SCIENCE, TECHNOLOGY, ENGINEERING, \\
% & AND MATHEMATICS EDUCATION AT THE ELEMENTARY AND SECONDARY \\
% & SCHOOL LEVEL.
% In General.
% Subpart D of part IV of subchapter A of chapter 1 \\
% & of the Internal Revenue Code of 1986 is amended by adding at the end the following \\
% & new section: ""Section 45O.
% CONTRIBUTIONS BENEFITING SCIENCE, \\
% & TECHNOLOGY, ENGINEERING, AND MATHEMATICS EDUCATION AT THE \\
% & ELEMENTARY AND SECONDARY SCHOOL LEVEL.
% In General.
% For purposes of \\
% & section 38, the elementary and secondary science, technology, engineering, \\
% & and mathematics (STEM) contributions credit determined under this section for the \\
% & taxable year is an amount equal to 100 percent of the qualified STEM contributions of \\
% & the taxpayer for such taxable year.
% Qualified STEM Contributions.
% Conforming \\
% & Amendments. The table of sections for subpart D of part IV of subchapter A of chapter 1\\
% & of such Code is amended by adding at the end the following new item: \\
% & ""Section 45O.
% Contributions benefiting science, technology, engineering, and \\
% & mathematics education at the elementary and secondary school level.\\
% \midrule 
% \multirow{5}{*}{HipoRank}
% & Section 38(b) of such Code is amended by striking ""plus"" at the end of paragraph (30),\\
% \multirow{5}{*}{Themantic} & by striking the period at the end of paragraph (31), and inserting "", plus"", and by adding \\
% & at the end the following new paragraph: the elementary and secondary science, technology, \\
% & engineering, and mathematics (STEM) contributions credit determined under section 45O."".\\

% & Conforming Amendments.
% The term `STEM school contributions' means STEM property \\
% & contributions, and STEM service contributions.
% For purposes of this section In general.\\
% & The table of sections for subpart D of part IV of subchapter A of chapter 1 of such Code is \\
% & amended by adding at the end the following new item: ""Section 45O.
% For purposes of \\
% & section 38, the elementary and secondary science, technology, engineering, and mathematics \\
% & (STEM) contributions credit determined under this section for the taxable year is an amount  \\
% & equal to 100 percent of the qualified STEM contributions of the taxpayer for such taxable year. \\
% & Qualified STEM Contributions.
% For purposes of this section, the term `qualified STEM \\
% & contributions' means STEM school contributions, STEM teacher externship expenses, and\\
% &  STEM teacher training expenses.
% STEM School Contributions.
% Contributions benefiting \\
% & science, technology, engineering, and mathematics education at the elementary and \\
% & secondary school level."".
% Effective Date.
% The amendments made by this section shall apply to \\
% & taxable years beginning after the date of the enactment of this Act.\\

%      \bottomrule
% \end{tabular}
% \caption{Generation results of BillSum dataset.}
% \label{tab:billsum_example}
% \end{table*}


%\subsection{Document Example of different views}\label{apendix:three_views}

 \subsection{IRC Annotation}\label{appendix:irc}
We show the IRC annotation of both a case and its human summary in Figure \ref{fig:IRC_example}.

\subsection{Document Cleaning Heuristics}\label{sec:appendix_heuristics}
The heuristics for filtering the headers from cases are provided below for replication purposes; we also provide the code\footnote{\url{https://github.com/cs329yangzhong/DocumentStructureLegalSum}} to process the CanLII data (although it requires that the data must first be obtained through an agreement with the Canadian Legal Information Institute). 

\begin{enumerate}
    \item Cut the document until the sentence begins with ``Introduction''. 
    \item Cut the document until the sentence starts with an ordered number such as (1), [1].
    \item Remove rows until the judge's name or case date appeared.
\end{enumerate}

\subsection{Comparing to Abstractive Summarization}\label{sec:appendix_abstractive}
For supervised abstractive baselines, we experiment with BART \cite{lewis-etal-2020-bart} and  Longformer-Encoder-Decoder (LED) \cite{Beltagy2020Longformer}. The latter model can process longer input documents up to 16k tokens. The results  in Table \ref{tab:unsupervised_results_abstractive} show that there still exists a gap between the extractive and abstractive models.

\begin{figure*}
\centering
 \includegraphics[width=\linewidth]{Figs/example.jpeg}

  \caption{An example of the annotated \textcolor{red}{Issue}, \textcolor{cyan}{Reason}, and \textcolor{teal}{Conclusion} sentences in CanLII dataset's case and summary pair (ID: 1991canlii4370). A portion of the beginning sentences in the case are not as important as the main document, including the meta-data of the case such as the participants' names, time, locations, etc. Thus, we treated them as headers and filtered them out using a heuristic introduced in Appendix \ref{sec:appendix_heuristics}. }
  \label{fig:IRC_example}
\end{figure*}

\begin{table*}[h!]
\small
% \scriptsize 
    \centering
    % \setlength\tabcolsep{2.1pt}
    \renewcommand{\arraystretch}{1}%Tighter
    \begin{tabular}{c|l|cc}
    %\begin{tabular}{l|lllc|lllc|p{0.9cm}ll}
    \toprule
        & & \multicolumn{2}{c}{\textbf{CanLII}}\\
      ID & Model  &  R1/R2/RL  & BS\\
         \midrule
         \multicolumn{3}{c}{{Oracles}} \\
         \midrule
          1 & IRC & 58.04/36.02/55.28 & 87.94 \\
        2 &  EXT & 59.38/38.77/56.94 & 87.85 \\
        
   
          
    \midrule 
    \multicolumn{3}{c}{{Supervised Extractive }} \\
    \midrule
    % 3 & BERT\textsubscript{EXT} - selection & 40.78 & 17.63 & 37.75 & 84.05 \\
    3 & BERT\_extractor & {43.44}/{17.84}/{40.36} & {84.47} \\
    \midrule
    \multicolumn{3}{c}{{Supervised Abtractive }} \\
    \midrule
    %  5 & SentCLR & & & & \\
    % \midrule
    % \multicolumn{6}{c}{{Supervised Abstractive}} \\
    % \midrule
   
    4 & BART & 50.50/25.58/46.82 &  87.25\\
    5 & LED & 53.72/28.75/ 50.17 & 87.55\\
    % \midrule
    % \multicolumn{6}{c}{{Unsupervised Extractive }} \\
    % \midrule
    %   4 & LSA & 37.22/17.82/34.87 & {84.48} & 40.44/18.07/36.31 &  83.70 \\
    %       5 & LexRank & 37.90/{18.17}/35.62 &  84.32 &  41.26/\textit{21.19}/37.43 & 84.02\\
    %         6 & TextRank  & 36.70/16.19/34.00 & 83.51 & 36.38/16.93/30.84 & \textit{84.49}\\
    %         7 & PACSUM & 40.01/15.68/37.37 & 83.52  & 40.71/18.23/37.00 & 83.08  \\
    %         8 & HipoRank &  {41.61}/17.13/{38.73}  & 83.55  & \textit{42.41}/19.48/\textit{38.65} & 83.24 \\
    
    %     \midrule
    \bottomrule
    \end{tabular}
    \caption{The automatic evaluation results on the CanLII  test set with supervised abstractive models.}
    \label{tab:unsupervised_results_abstractive}
\end{table*}




% \section{Examples of Dataset}
% \subsection{CanLII Dataset}
% We include one example of the CanLII dataset with both case and summary. The IRC sentences are highlighted with the corresponding colors.
% \begin{table*}
% \begin{tabular}{c|c}
%      J. 2003 SKQB 487 D.I.V. A.D. 2003 No. 172 J.C. R. IN THE QUEEN’S BENCH (FAMILY LAW DIVISION) JUDICIAL CENTRE OF REGINA BETWEEN: CHRISTINE JAMES and RYAN MATTHEW ROSS (ROLLHEISER) RESPONDENT Jeffrey P. Reimer for the petitioner Donald G. Findlay for the respondent JUDGMENT DAWSON J. November 19, 2003 [1] At issue on this application is the interim custody of Keanna Marie James (Ross), child of the petitioner Christine James and the respondent Ryan Ross (Rollheiser) and, child support. INTERIM CUSTODY BACKGROUND FACTS [2] By way of background, the parties began to cohabit in May of 1998. In May of 1999, Keanna was born. Although there is conflicting affidavit evidence with regard to the date the parties ceased to cohabit, it is somewhere in the period between October of 1999 and April of 2000. Since the time the parties ceased to cohabit, Keanna has resided with the petitioner. The respondent has had access. [3] The respondent states in his affidavit dated October 3, 2003, that the parties had an arrangement in May and June of 2000, where the parties would “switch Keanna back and forth every second day,” and that after that the respondent had access to Keanna every weekend until November 2002 when, the respondent alleges, the petitioner ended access. The respondent says that in February 2003, the respondent was “allowed to have Keanna every second weekend, except for period of two weeks in March.” [4] The petitioner denies in her reply affidavit dated October 8, 2002, that the parties had any arrangement switching Keanna every second day or every weekend. She states that the parties had an arrangement from the outset whereby the respondent would see Keanna every second weekend and that is what happened. The petitioner asserts that the every second weekend access was interrupted over the years “by couple of violent episodes instigated by Ryan.” Counsel for the petitioner argued in chambers that the respondent made no attempts at gaining custody until he was served with the application of the petitioner. [5] have also reviewed the various affidavits of some of the family members and relatives of the parties. Although they are of no assistance in assisting me in determining the specific time periods wherein the respondent purported to have access to the child, they are nevertheless all consistent in establishing one thing; that the respondent did, in fact, have plenty of access to the child at various times from 1999 to 2003. [6] The petitioner herself, in her affidavit dated September 2, 2003, admits that since the parties ceased to cohabit, the respondent has “exercised access with their daughter on fairly regular basis.” As well, in her subsequent affidavit dated October 8, 2003, the petitioner confirms various access arrangements that the parties had, save for those she explicitly disputes. [7] The surplus of the various affidavits of the parties to the applications consist of accusations and constant strife about past access arrangements. The affidavits of the parties and their family members and relatives satisfy me that both parties are loving parents to their child. will also say that the mutual love and acceptance demonstrated by the families and relatives of each party, along with the overall concern for the best interests of the child, is self-evident and commendable. INTERIM CUSTODY LAW AND ANALYSIS [8] The purpose of an interim custody order is to give person the right and duty to care for child until permanent order or another interim order is made. On an interim application, the focus differs from that of permanent custody hearing. Rather than focussing on the long-term best interests of the child, court focusses on the short-term needs of the child. As a result, interim orders usually reinforce the status quo in an attempt to reflect on the best interests of the child. [9] Sections and of The Children’s Law Act, 1997, S.S. 1997, c. C-8.2 prescribe the relevant considerations in determining interim custody: 6(1) Notwithstanding sections to 5, on the application of parent or other person having, in the opinion of the court, sufficient interest, the court may, by order: (a) grant custody of or access to child to one or more persons; (b) determine any aspect of the incidents of the right to custody or access; and (c) make any additional order that the court considers necessary and proper in the circumstances. (2) Where the court grants custody of child to parent pursuant to subsection (1), the court, where in its opinion it would be in the best interests of the child to do so, may by order, authorize the parent to appoint person: (a) to have custody of the child on the parent’s death; (b) to be the guardian of the property of the child on the parent’s death; or (c) to have both of the duties mentioned in clauses (a) and (b). (3) On an application and prior to making an order pursuant to subsection (1), the court may make or vary an interim order on any terms and conditions it considers appropriate. ... In making, varying or rescinding an order for custody of child, the court shall: (a) have regard only for the best interests of the child and for that purpose shall take into account: (i) the quality of the relationship that the child has with the person who is seeking custody and any other person who may have close connection with the child; (ii) the personality, character and emotional needs of the child; (iii) the physical, psychological, social and economic needs of the child; (iv) the capacity of the person who is seeking custody to act as legal custodian of the child; (v) the home environment proposed to be provided for the child; (vi) the plans that the person who is seeking custody has for the future of the child; and (vii) the wishes of the child, to the extent the court considers appropriate, having regard to the age and maturity of the child; (b) not take into consideration the past conduct of any person unless the conduct is relevant to the ability of that person to act as parent of child; and (c) make no presumption and draw no inference as between parents that one parent should be preferred over the other on the basis of the person’s status as father or mother. [10] In considering an interim application the following questions might be posed, as noted by James G. McLeod in his book, Child Custody, Law Practice, 3d ed. (Toronto: Thomson Canada Limited, 1992, looseleaf, updated 2003): (a) Where and with whom is the child residing at the time of the interim hearing? (b) Where and with whom had the child been residing in the immediate past? (c) If there has been change in the child’s residence or primary caregiver in the recent past, why was the change made and who participated in the decision? (d) Are the child’s current living arrangements suitable for the child, taking into consideration the short-term needs of the child and the temporary nature of the order? (e) Is the current caregiver suitable role model and able to meet the child’s needs? (f) Are there any problems with the child’s current parenting or living arrangements? (g) Is there any reason to change the child’s current parenting or living arrangements? [11] Here I must take into account the current status quo, namely, that the child has and does reside with the petitioner, with access and visitation arrangements to the respondent. Keanna is doing well and has loving, stable relationship with her mother. The petitioner has provided suitable care and has met Keanna’s needs. Notwithstanding the instances of suspended access, the arrangements for access appear to have worked relatively well and appear to be in Keanna’s best interests. There is plenty of affidavit evidence to support this: (a) The petitioner admits that “in general, Ryan loves Keanna” and has exercised access to her on fairly regular basis; (b) Both parties have been willing to make arrangements in the past; (c) Efforts have been made by the respondent at contact with the child. (d) Efforts have been made by the respondent to make the child feel comfortable, happy while at his home; (e) Efforts have been made by the family members and relatives of both families to provide living environment for the child; and (f) Evidence that the child expresses content and joy with the respondent and his family (including new sister). [12] The respondent wishes to change the status quo on an interim basis to have Keanna reside with him every second week. The respondent has stated in an affidavit that he maintains odd work hours and suggests that if he had equal time with his daughter, his present wife would care for Keanna rather than having Keanna go to daycare as she presently does. This would be significant change in the status quo and one that is not yet established to be in Keanna’s best interests. [13] Based on the evidence before me, I make an order for interim custody to the petitioner with access to the respondent. The order is consistent with previous decisions of this Court. See for example Green v. Anderson, [1996] S.J. No. 752 (QL) (Q.B.), Kwok v. MacDonald, [1997] S.J. No. 478 (QL) (Q.B.), Hudson v. Hudson, 2000 SKQB 192 (CanLII); (2000), 192 Sask. R. 274 (Q.B.), Yurchak v. Yurchak, 2000 SKQB 213 (CanLII); (2000), 192 Sask. R. 312 (Q.B.), Prettyshield-Nicholls v. Maloughney, 2002 SKQB 299 (CanLII); [2002] S.J. No. 440 (QL) (Q.B.), and G.J.S. v. L.J.M., 2002 SKQB 417 (CanLII); [2002] S.J. No. 610 (QL) (Q.B.). [14] Both parents were ordered to complete the Parenting After Separation course. Both parties must file certificate of completion once the course is completed. [15] The parties raised in their material the issue of each other’s lack of cooperation and communication. While am not in position, at this stage, to decide whether or not the accusations of each party are substantiated, can say, with relative certainty, that there is an unhealthy degree of constant bickering. Keanna is likely affected to varying degrees by the discord between the parties. would go on to comment that it is evident that as long as the parties work on establishing better communication, the child will benefit from contact with both parents, with minimal disruption. [16] Accordingly, there will be the following order: (a) The petitioner shall have interim custody of Keanna Ross (James); (b) The respondent shall have reasonable access including every second weekend, reasonable telephone access and reasonable access during holiday periods. INTERIM CHILD SUPPORT [17] The respondent, on October 15, 2003, was ordered to file year to date income information. That has now been filed. find the respondent’s income for the purposes of interim child support to be $23,175.00. The respondent shall pay interim child support in the amount of $193.00 per month commencing on December 1, 2003 and a like amount on the first day of each month thereafter until further order. [18] find the petitioner’s income for the purposes of interim child support to be $4,921.96 (the average of her 2000 and 2001 income). [19] find the child care costs to be $125.00 per month. The respondent shall pay his proportionate share of childcare costs (being 82.5\%) in the amount of $103.00 per month commencing with $206.00 payable forthwith for October and November, 2003 and like amount on the first day of each month thereafter until further order. [20] The petitioner shall have costs of her application." & "At issue was the interim custody of the child and child support. HELD: Interim custody was given to the petitioner with access to the respondent. The respondent was ordered to pay interim child support in the amount of \$193 per month. On an interim custody application, the Court must take into account the status quo. \\

     
% \end{tabular}
% \caption{CanLII dataset}
% \end{table*}

\end{document}