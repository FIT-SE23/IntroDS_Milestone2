%%%%%%%% ICML 2023 EXAMPLE LATEX SUBMISSION FILE %%%%%%%%%%%%%%%%%

\documentclass[nohyperref]{article}

% Recommended, but optional, packages for figures and better typesetting:
\usepackage{microtype}
\usepackage{graphicx}
% \usepackage{subfigure}
\usepackage{booktabs} % for professional tables

% hyperref makes hyperlinks in the resulting PDF.
% If your build breaks (sometimes temporarily if a hyperlink spans a page)
% please comment out the following usepackage line and replace
% \usepackage{icml2023} with \usepackage[nohyperref]{icml2023} above.
\usepackage{hyperref}


% Attempt to make hyperref and algorithmic work together better:
\newcommand{\theHalgorithm}{\arabic{algorithm}}

% Use the following line for the initial blind version submitted for review:
% \usepackage{icml2023}

% If accepted, instead use the following line for the camera-ready submission:
\usepackage[accepted]{icml2023}

% For theorems and such
\usepackage{amsmath}
\usepackage{amssymb}
\usepackage{mathtools}
\usepackage{amsthm}
% \usepackage{stmaryrd}

\usepackage{hyperref}       % hyperlinks
\usepackage{url}            % simple URL typesetting
\usepackage{booktabs}       % professional-quality tables
\usepackage{amsfonts}       % blackboard math symbols
\usepackage{nicefrac}       % compact symbols for 1/2, etc.
\usepackage{microtype}      % microtypography
\usepackage{xcolor}         % colors
\usepackage{graphicx}
\usepackage{times}
\usepackage{latexsym}
\usepackage{multirow}
\usepackage{balance}
\usepackage{bbm}
\usepackage{paralist}
\usepackage{makecell}
\usepackage[labelformat=simple]{subcaption}
\renewcommand\thesubfigure{(\alph{subfigure})}
% \usepackage{caption}
\usepackage{xspace}
\usepackage{multicol}
\usepackage{transparent}
\usepackage{array}
\usepackage{bm}
\usepackage{tabularx}
\usepackage{setspace}
% \usepackage{floatrow, makecell}
% \usepackage{wrapfig,lipsum,booktabs}
\usepackage{pifont}
% \usepackage{algorithm2e}
\usepackage{thmtools, thm-restate}
% \usepackage[makeroom]{cancel}

% if you use cleveref..
\usepackage[capitalize,noabbrev]{cleveref}

\usepackage{caption}
\usepackage{wrapfig,lipsum,booktabs}
\usepackage{algorithm}
\usepackage[algo2e]{algorithm2e}
% \usepackage{algpseudocode}

\DeclareMathOperator*{\argmax}{argmax}
\DeclareMathOperator*{\argmin}{argmin}
\newcommand{\note}[1]{\textcolor{red}{[#1]}}
\newcommand{\eg}{\textit{e.g.}}
\newcommand{\ie}{\emph{i.e.}}
\newcommand{\method}{FewGen\xspace}
\newcommand{\bs}[1]{\boldsymbol{#1}}
\newcommand{\acc}{Acc. ($\uparrow$)}
\newcommand{\ppl}{PPL ($\downarrow$)}


\icmltitlerunning{Tuning Language Models as Training Data Generators for Augmentation-Enhanced Few-Shot Learning}

\begin{document}
% The \author macro works with any number of authors. There are two commands
% used to separate the names and addresses of multiple authors: \And and \AND.
%
% Using \And between authors leaves it to LaTeX to determine where to break the
% lines. Using \AND forces a line break at that point. So, if LaTeX puts 3 of 4
% authors names on the first line, and the last on the second line, try using
% \AND instead of \And before the third author name.


% \author{
% Yu Meng, Martin Michalski, Jiaxin Huang, Yu Zhang, Tarek Abdelzaher, Jiawei Han\\
% \,\,University of Illinois at Urbana-Champaign \\
% \texttt{\{yumeng5,martinm6,jiaxinh3,yuz9,zaher,hanj\}@illinois.edu}
% }



\twocolumn[
\icmltitle{Tuning Language Models as Training Data Generators for Augmentation-Enhanced Few-Shot Learning}

\begin{icmlauthorlist}
\icmlauthor{Yu Meng}{uiuc}
\icmlauthor{Martin Michalski}{uiuc}
\icmlauthor{Jiaxin Huang}{uiuc}
\icmlauthor{Yu Zhang}{uiuc}
\icmlauthor{Tarek Abdelzaher}{uiuc}
\icmlauthor{Jiawei Han}{uiuc}
\end{icmlauthorlist}

\icmlaffiliation{uiuc}
{University of Illinois Urbana-Champaign}

% \icmlaffiliation{yyy}{Department of XXX, University of YYY, Location, Country}
% \icmlaffiliation{comp}{Company Name, Location, Country}
% \icmlaffiliation{sch}{School of ZZZ, Institute of WWW, Location, Country}

\icmlcorrespondingauthor{Yu Meng}{yumeng5@illinois.edu}

\icmlkeywords{Few-Shot Learning, Natural Language Understanding}

\vskip 0.3in
]

\printAffiliationsAndNotice{} 

\begin{abstract}
\input{Sections/0-abs}
\end{abstract}


% \input{figs/overview.tex}
%\vspace{-0.4cm}
\section{DeepFlow Overview}\label{sec:overview}

Figure~\ref{fig:overview} shows an overview of the \name framework. \name takes the following set of \textbf{inputs}: 
%
(1) \underline{System} design hierarchy (e.g., the number of accelerator nodes per device, the number of devices in the system, the network topology connecting nodes within a device and across the devices), 
(2) \underline{Architecture template} of each accelerator node which provides a high-level definition of its components and how those components fit together. The purpose of the template is to provide a blueprint for the accelerator without committing to any specific hardware parameters.
%A component definition (e.g., minimal compute units (MCU\footnote{Examples of what we regard as MCU includes SMU in older GPUs, Tensor Cores in newer GPUs or systolic array in TPUs}), memory hierarchy, network), specification of each component (e.g., flop rate for each MCU, MCU dimensions, number of MCUs sharing a set of register files, dataflow execution model, and characteristics and scope of different levels of memory hierarchy), 
(3) \underline{Technology} parameters for each hardware component (e.g. energy per flop), 
(4) \underline{Design budgets} for each hardware component (area, power, perimeter),  
(5) \underline{Machine learning model} specification in the form of a high-level compute graph, parameters of each compute node (kernel type, tensor dimensions), and
(6) \underline{Parallelism strategy} (data, model, kernel, and/or pipeline parallelism dimensions) which distributes the compute graph across the entire system. 
(7) \underline{Device mapping} strategy which defines mapping of parallel shards onto hardware nodes.
Given these inputs, \name predicts the end-to-end performance of one iteration (i.e., single batch) of the model and finds an optimal hardware-software-technology design point as \textbf{output}. 

DeepFlow is composed of two major components.
\underline{CrossFlow} which operates in a stand-alone mode and can predict performance for any input configuration; and a search and optimization engine (\underline{SOE}) which enables design space search. 
%To do so, \name breaks the problem into multiple phases.
%Each phase or building block of \name is described in details next.
\vspace{-0.1cm}
\subsection{CrossFlow Building Blocks}

\paragraph*{\em Micro-Architecture Generator Engine (AGE)}

AGE takes the following set of \textbf{inputs}:
(1) Design constraints (i.e the power, area and perimeter budget and breakdown across micro-architectural components such as cache, network, compute cores). 
This breakdown can be provided manually by users or automatically by the Search and Optimization Engine (SOE, Section~\ref{subsec:soe}).
%We also provide technology specifications such as 
%and their physical characteristics such as area/power per core under nominal operating conditions, SRAM/register characteristics. 
(2) Technology parameters such as energy per flop, energy per data bit transfer for each level of memory and network hierarchy, threshold and maximum gate voltage, integration substrate parameters such as bump/interconnect pitch. We provide a wide range of standard and future technology libraries as baseline. (3) Architecture template which is a blueprint of the underlying accelerator chip without committing to any specific hardware parameters. Given these input, AGE performs a frequency-voltage-area scaling optimization to generate the following \textbf{output} parameters such that design budgets for all component are met: 
(1) Compute throughput.
(2) Capacity for different levels of memory hierarchy.
(3) Bandwidth to each level of memory hierarchy.
(4) Inter-node as well as intra-node network bandwidth. 
These parameters are then utilized by the performance prediction engine (PPE) to estimate the execution time of each kernel.
%As mentioned previously, 
%The output of this stage is the input to performance engine to estimate the execution time of each kernel. Next, we describe the search and optimization engine (SOE) which feeds input values to AGE, if we want to use the model for architecture search.
%\vspace{-0.2cm}
\paragraph*{\em Compute Graph Transformation and Device Placement Engine (DPE)}
The parallelization strategy and device mapping are critical in deciding the overall execution time. Here, we first transform the model graph to a `super-graph' to reflect the parallelization strategy provided by the users manually, or SOE engine (Section~\ref{subsec:soe}) automatically. For example, to apply data parallelism, the model graph is replicated and appropriate edges are added to model the gradient exchange. After generating the transformed graph, DPE assigns the vertices of the transformed graph to the system nodes following a heuristic approach to minimize the communication overhead. %
%The details are presented in section~\ref{}.

%\vspace{-0.2cm}
\paragraph*{\em Performance Prediction Engine (PPE)}
%With the device mapping for all the vertices of the compute (super-)graph known, the next step is to calculate the overall execution time for a forward pass and/or a backward pass. 
We use hierarchical roofline modeling to predict the performance of each compute node. To calculate the overall end-to-end execution time, while respecting scheduling constraints (e.g. one kernel at a time per GPU, or prioritizing one kernel launch over another) we use event-driven simulation.%
%We explain the details of the PPE in section~\ref{}.
\subsection{Search and Optimization Engine (SOE)}\label{subsec:soe}
Co-optimizing micro-architectural parameters and the parallelization strategy that minimizes the overall end-to-end execution time requires navigating a large space of design parameters. 
Search and optimization engine (SOE) enables the automatic design space search and finds an 
%that meets the total power and area constraints, and simultaneously explores software parallelization strategies to find the 
optimal design point which meets the design constraints and minimizes the overall execution time.
%Because the hardware configuration space is very large, the search algorithm we designed 
SOE takes inspiration from ML-assisted search algorithms, in particular gradient decent search with momentum and builds on top of the CrossFlow modeling engine.
%The software parallelization design space is much smaller compared to the hardware design space and therefore we employ an exhaustive grid search. 

%Gradient search is an iterative process. In each step, SOE takes the predicted time from previous iteration as input to re-adjust the following parameter settings: (1) power, area and perimeter breakdown across different architectural components. (2) a parallelization strategy. These parameters will be fed back to CrossFlow to estimate the overall execution time. This process continues until convergence or user-specified number of steps. 
%The details of SOE's search algorithm are elaborated in Section~\ref{}. 
\vspace{-0.2cm}
\subsection{Parallelism Strategy Space}
\label{subsec:par_strategy}
There are a myriad of ways to parallelize a model across a large multi-node system. Exploring the parallelism space and finding the optimal strategy is critical to overall performance and system utilization. DeepFlow explores kernel, data and layer parallelism. It uniquely identifies each parallelism strategy by following notations: $\texttt{RC-\{KP1\}-\{KP2\}-d\{DP\}-p\{LP\}}$ or $\texttt{CR-\{KP1\}-d\{DP\}-p\{LP\}}$ depending on the choice of kernel parallelism.
RC (Row-Column) and CR (Column-Row) refer to different forms of kernel parallelism, i.e. distributed GEMM through inner-product or outer-product implementation.
%\begin{equation*}
%    \texttt{RC: R{KP1\}\_C\{KP2\}\_d\{DP\}\_p\{LP\}}
%\end{equation*}
%Where \texttt{RC} or \texttt{CR} refers to the type of kernel parallelism strategy, i.e. Row-Column or Column-Row,
%\texttt{N} refers to the number of parallel nodes,
\texttt{KP1} and \texttt{KP2} are the parameters of distributed GEMM. 
For Row-Column (\texttt{RC}) or inner-product, \texttt{KP1} and \texttt{KP2} would refer to the number of ways we shard the first matrix across rows and the second matrix across columns.
For Column-Row (\texttt{CR}) or outer-product, we would only need one parameter to specify the parallelization strategy; \texttt{KP1} will refer to the number of ways we cut the first matrix across columns and the second matrix across rows.
\texttt{DP} represents the number of model replicas and data shards assigned to each to exploit data parallelism.
\texttt{LP} is the number of ways we cut layers into stages to exploit pipeline parallelism.

\begin{comment}
\subsection{Modes of Operation}
\name has two modes of operation, standalone performance estimation mode and a architecture search mode.
\paragraph{Standalone Performance (SP) Estimation Mode}
Often ML practitioners or hardware designers want to estimate the performance of a model on a particular system configuration. For example, what is the cost optimal number of accelerators that one should deploy for distributed training? Or what is the estimated performance gain from choosing an accelerator with costlier HBM2E vs HBM2? To enable one to quickly answer such questions and to estimate performance under certain known system configurations, the tool can be run in the SP mode. 

In this mode, the description of the architecture of a scale-out system consisting of multiple accelerators, the architecture of the accelerator hardware themselves and the description of the neural network is taken as input, and fed into CrossFlow, which calculates the execution time of each training step. 

%In this mode, the description of the architecture of a scale-out system consisting of multiple accelerators, the architecture of the accelerator hardware themselves and the description of the neural network is taken as input. The tool calculates the execution time of each training step. 

%In this mode, the user 
%has the flexibility to use either just the \perfE or use \perfE alongside the AGE. While using just the \perfE  alone, the user needs to provide the architectural parameters of the tiles and the system. On the other hand, while using AGE  alongside \perfE, the user 
%needs to define the technology parameters and the hardware constraints i.e., the overall area and power breakdown among the different architectural components of the system. T

%In this mode, the tool generates the micro-architectural parameters of the accelerator chip using the AGE. It then runs the compute graph transformation and the device placement engine, and uses the \perfE to predict the execution time. 

\subsubsection{Architecture Search (AS) Mode}

The insatiable demand to run large models in the shortest possible time demands that we find the optimal hardware and software design points to train these models. From the hardware perspective, it is about finding the right micro-architecture as well as the overall system architecture of the distributed system. 
From the software perspective, it is about finding the right parallelization strategy. 
Often these decisions depend on each other, and so finding the optimal design points across the stack means 
navigating a large design space.

As one can imagine, the design space of the inputs to the tool is large and iterating over the entire design space is a tedious task. To efficiently search over the input space to find the optimal hardware constraints and parallelization strategy, the tool can be run in the AS mode. 
In this mode, the SOE module is used. The user will not need to provide the exact hardware parameters and the parallelization strategy. Only the architecture template and the initial compute graph will need to be provided as input to the tool. The tool then performs a search over the design space to find the optimal parameter settings that results in minimum training time. 
%We used gradient descent algorithm (details in Section~\ref{}) for this search.

%\subsection{Inputs and Outputs}

%\paragraph{SP-Mode}
%In this mode, the hardware 

%\paragraph{AS-Mode}

\end{comment}

%\vspace{-0.4cm}
\section{DeepFlow Overview}\label{sec:overview}

Figure~\ref{fig:overview} shows an overview of the \name framework. \name takes the following set of \textbf{inputs}: 
%
(1) \underline{System} design hierarchy (e.g., the number of accelerator nodes per device, the number of devices in the system, the network topology connecting nodes within a device and across the devices), 
(2) \underline{Architecture template} of each accelerator node which provides a high-level definition of its components and how those components fit together. The purpose of the template is to provide a blueprint for the accelerator without committing to any specific hardware parameters.
%A component definition (e.g., minimal compute units (MCU\footnote{Examples of what we regard as MCU includes SMU in older GPUs, Tensor Cores in newer GPUs or systolic array in TPUs}), memory hierarchy, network), specification of each component (e.g., flop rate for each MCU, MCU dimensions, number of MCUs sharing a set of register files, dataflow execution model, and characteristics and scope of different levels of memory hierarchy), 
(3) \underline{Technology} parameters for each hardware component (e.g. energy per flop), 
(4) \underline{Design budgets} for each hardware component (area, power, perimeter),  
(5) \underline{Machine learning model} specification in the form of a high-level compute graph, parameters of each compute node (kernel type, tensor dimensions), and
(6) \underline{Parallelism strategy} (data, model, kernel, and/or pipeline parallelism dimensions) which distributes the compute graph across the entire system. 
(7) \underline{Device mapping} strategy which defines mapping of parallel shards onto hardware nodes.
Given these inputs, \name predicts the end-to-end performance of one iteration (i.e., single batch) of the model and finds an optimal hardware-software-technology design point as \textbf{output}. 

DeepFlow is composed of two major components.
\underline{CrossFlow} which operates in a stand-alone mode and can predict performance for any input configuration; and a search and optimization engine (\underline{SOE}) which enables design space search. 
%To do so, \name breaks the problem into multiple phases.
%Each phase or building block of \name is described in details next.
\vspace{-0.1cm}
\subsection{CrossFlow Building Blocks}

\paragraph*{\em Micro-Architecture Generator Engine (AGE)}

AGE takes the following set of \textbf{inputs}:
(1) Design constraints (i.e the power, area and perimeter budget and breakdown across micro-architectural components such as cache, network, compute cores). 
This breakdown can be provided manually by users or automatically by the Search and Optimization Engine (SOE, Section~\ref{subsec:soe}).
%We also provide technology specifications such as 
%and their physical characteristics such as area/power per core under nominal operating conditions, SRAM/register characteristics. 
(2) Technology parameters such as energy per flop, energy per data bit transfer for each level of memory and network hierarchy, threshold and maximum gate voltage, integration substrate parameters such as bump/interconnect pitch. We provide a wide range of standard and future technology libraries as baseline. (3) Architecture template which is a blueprint of the underlying accelerator chip without committing to any specific hardware parameters. Given these input, AGE performs a frequency-voltage-area scaling optimization to generate the following \textbf{output} parameters such that design budgets for all component are met: 
(1) Compute throughput.
(2) Capacity for different levels of memory hierarchy.
(3) Bandwidth to each level of memory hierarchy.
(4) Inter-node as well as intra-node network bandwidth. 
These parameters are then utilized by the performance prediction engine (PPE) to estimate the execution time of each kernel.
%As mentioned previously, 
%The output of this stage is the input to performance engine to estimate the execution time of each kernel. Next, we describe the search and optimization engine (SOE) which feeds input values to AGE, if we want to use the model for architecture search.
%\vspace{-0.2cm}
\paragraph*{\em Compute Graph Transformation and Device Placement Engine (DPE)}
The parallelization strategy and device mapping are critical in deciding the overall execution time. Here, we first transform the model graph to a `super-graph' to reflect the parallelization strategy provided by the users manually, or SOE engine (Section~\ref{subsec:soe}) automatically. For example, to apply data parallelism, the model graph is replicated and appropriate edges are added to model the gradient exchange. After generating the transformed graph, DPE assigns the vertices of the transformed graph to the system nodes following a heuristic approach to minimize the communication overhead. %
%The details are presented in section~\ref{}.

%\vspace{-0.2cm}
\paragraph*{\em Performance Prediction Engine (PPE)}
%With the device mapping for all the vertices of the compute (super-)graph known, the next step is to calculate the overall execution time for a forward pass and/or a backward pass. 
We use hierarchical roofline modeling to predict the performance of each compute node. To calculate the overall end-to-end execution time, while respecting scheduling constraints (e.g. one kernel at a time per GPU, or prioritizing one kernel launch over another) we use event-driven simulation.%
%We explain the details of the PPE in section~\ref{}.
\subsection{Search and Optimization Engine (SOE)}\label{subsec:soe}
Co-optimizing micro-architectural parameters and the parallelization strategy that minimizes the overall end-to-end execution time requires navigating a large space of design parameters. 
Search and optimization engine (SOE) enables the automatic design space search and finds an 
%that meets the total power and area constraints, and simultaneously explores software parallelization strategies to find the 
optimal design point which meets the design constraints and minimizes the overall execution time.
%Because the hardware configuration space is very large, the search algorithm we designed 
SOE takes inspiration from ML-assisted search algorithms, in particular gradient decent search with momentum and builds on top of the CrossFlow modeling engine.
%The software parallelization design space is much smaller compared to the hardware design space and therefore we employ an exhaustive grid search. 

%Gradient search is an iterative process. In each step, SOE takes the predicted time from previous iteration as input to re-adjust the following parameter settings: (1) power, area and perimeter breakdown across different architectural components. (2) a parallelization strategy. These parameters will be fed back to CrossFlow to estimate the overall execution time. This process continues until convergence or user-specified number of steps. 
%The details of SOE's search algorithm are elaborated in Section~\ref{}. 
\vspace{-0.2cm}
\subsection{Parallelism Strategy Space}
\label{subsec:par_strategy}
There are a myriad of ways to parallelize a model across a large multi-node system. Exploring the parallelism space and finding the optimal strategy is critical to overall performance and system utilization. DeepFlow explores kernel, data and layer parallelism. It uniquely identifies each parallelism strategy by following notations: $\texttt{RC-\{KP1\}-\{KP2\}-d\{DP\}-p\{LP\}}$ or $\texttt{CR-\{KP1\}-d\{DP\}-p\{LP\}}$ depending on the choice of kernel parallelism.
RC (Row-Column) and CR (Column-Row) refer to different forms of kernel parallelism, i.e. distributed GEMM through inner-product or outer-product implementation.
%\begin{equation*}
%    \texttt{RC: R{KP1\}\_C\{KP2\}\_d\{DP\}\_p\{LP\}}
%\end{equation*}
%Where \texttt{RC} or \texttt{CR} refers to the type of kernel parallelism strategy, i.e. Row-Column or Column-Row,
%\texttt{N} refers to the number of parallel nodes,
\texttt{KP1} and \texttt{KP2} are the parameters of distributed GEMM. 
For Row-Column (\texttt{RC}) or inner-product, \texttt{KP1} and \texttt{KP2} would refer to the number of ways we shard the first matrix across rows and the second matrix across columns.
For Column-Row (\texttt{CR}) or outer-product, we would only need one parameter to specify the parallelization strategy; \texttt{KP1} will refer to the number of ways we cut the first matrix across columns and the second matrix across rows.
\texttt{DP} represents the number of model replicas and data shards assigned to each to exploit data parallelism.
\texttt{LP} is the number of ways we cut layers into stages to exploit pipeline parallelism.

\begin{comment}
\subsection{Modes of Operation}
\name has two modes of operation, standalone performance estimation mode and a architecture search mode.
\paragraph{Standalone Performance (SP) Estimation Mode}
Often ML practitioners or hardware designers want to estimate the performance of a model on a particular system configuration. For example, what is the cost optimal number of accelerators that one should deploy for distributed training? Or what is the estimated performance gain from choosing an accelerator with costlier HBM2E vs HBM2? To enable one to quickly answer such questions and to estimate performance under certain known system configurations, the tool can be run in the SP mode. 

In this mode, the description of the architecture of a scale-out system consisting of multiple accelerators, the architecture of the accelerator hardware themselves and the description of the neural network is taken as input, and fed into CrossFlow, which calculates the execution time of each training step. 

%In this mode, the description of the architecture of a scale-out system consisting of multiple accelerators, the architecture of the accelerator hardware themselves and the description of the neural network is taken as input. The tool calculates the execution time of each training step. 

%In this mode, the user 
%has the flexibility to use either just the \perfE or use \perfE alongside the AGE. While using just the \perfE  alone, the user needs to provide the architectural parameters of the tiles and the system. On the other hand, while using AGE  alongside \perfE, the user 
%needs to define the technology parameters and the hardware constraints i.e., the overall area and power breakdown among the different architectural components of the system. T

%In this mode, the tool generates the micro-architectural parameters of the accelerator chip using the AGE. It then runs the compute graph transformation and the device placement engine, and uses the \perfE to predict the execution time. 

\subsubsection{Architecture Search (AS) Mode}

The insatiable demand to run large models in the shortest possible time demands that we find the optimal hardware and software design points to train these models. From the hardware perspective, it is about finding the right micro-architecture as well as the overall system architecture of the distributed system. 
From the software perspective, it is about finding the right parallelization strategy. 
Often these decisions depend on each other, and so finding the optimal design points across the stack means 
navigating a large design space.

As one can imagine, the design space of the inputs to the tool is large and iterating over the entire design space is a tedious task. To efficiently search over the input space to find the optimal hardware constraints and parallelization strategy, the tool can be run in the AS mode. 
In this mode, the SOE module is used. The user will not need to provide the exact hardware parameters and the parallelization strategy. Only the architecture template and the initial compute graph will need to be provided as input to the tool. The tool then performs a search over the design space to find the optimal parameter settings that results in minimum training time. 
%We used gradient descent algorithm (details in Section~\ref{}) for this search.

%\subsection{Inputs and Outputs}

%\paragraph{SP-Mode}
%In this mode, the hardware 

%\paragraph{AS-Mode}

\end{comment}


\section{Introduction}
Recent research has demonstrated the appealing few-shot learning potential of pretrained language models (PLMs)~\citep{Brown2020LanguageMA,clark2020electra,devlin2019bert,he2020deberta,liu2019roberta,meng2021coco,Meng2022PretrainingTE} on natural language understanding (NLU) tasks~\citep{Wang2019SuperGLUEAS,wang2018glue}:
Instead of relying on abundant task-specific annotations, PLMs can effectively leverage a small set of training samples to quickly learn a new task.
Such training data efficiency is usually achieved by formulating downstream tasks as prompts~\citep{Brown2020LanguageMA,gao2021making,Scao2021HowMD,Schick2021ExploitingCF,Schick2021ItsNJ} which allow the PLM to adapt its language modeling ability acquired through pretraining to new downstream tasks.

The success of prompt-based methods has stimulated numerous explorations along the line of effective few-shot learning with PLMs: 
The training samples converted to natural language prompts can be used to directly fine-tune PLMs~\citep{gao2021making,Schick2021ExploitingCF} or as in-context demonstrations to facilitate better inference~\citep{Brown2020LanguageMA,Liu2022WhatMG}.
More recent approaches aim to automate the design of prompts by gradient-based searching~\citep{Shin2020ElicitingKF} or parameterizing prompts as continuous learnable embeddings~\citep{Lester2021ThePO,Liu2021GPTUT,zhang2022differentiable,Zhong2021FactualPI}.
Other studies investigate and address specific issues in prompt-based few-shot learning~\citep{Liu2022FewShotPF,Tam2021ImprovingAS,Zhao2021CalibrateBU}.
While remarkable, the model performance still has a nontrivial gap from fully supervised models trained on massive labeled data.
Indeed, training deep models is inherently data demanding---model generalization usually benefits from more training samples~\citep{baum1988size}.

In this work, we study few-shot learning with PLMs from a different perspective:
Instead of proposing new methods for fine-tuning on few-shot samples, we focus on the generation of quality training data based on few-shot samples and using these synthesized training samples to fine-tune the classification models.
Motivated by the strong text generation power of autoregressive PLMs~\citep{Brown2020LanguageMA,Keskar2019CTRLAC,raffel2019t5}, 
previous data augmentation methods enlarge the training set by synthesizing new samples based on the few-shot samples. 
They either fine-tune the generator on the training set with the standard maximum likelihood objective~\citep{AnabyTavor2020DoNH,Kumar2020DataAU} or use the training samples as demonstrations~\citep{Yoo2021GPT3MixLL}.
However, these methods do not explicitly model the distinction across different labels and may struggle to generate accurate training samples pertaining to the desired labels for challenging NLU tasks.
% we aim to use them as generators for creating novel training data after tuning on few-shot samples.

In this paper, we study how to use few-shot samples to effectively tune PLMs to generate high quality label-discriminative training samples.
% We first analyze the 
% To ensure the samples created by the generator PLM are label-discriminative, we propose a weighted maximum likelihood objective, where the weight for each token is automatically learned from a discriminative meta-learning objective. 
% The synthesized samples by the generator can be combined with the few-shot samples for training any classification PLM with any fine-tuning method.
Our contributions are as follows: (1) We analyze the issues of using standard maximum likelihood for tuning the generator and propose a meta-weighted maximum likelihood objective for generator tuning by automatically learning token weights that emphasize label discriminativeness.
(2) We propose a simple and effective training procedure for fine-tuning classification PLMs on generated data by mitigating label noise.
(3) Under the same few-shot learning setting, our method \method outperforms existing methods by $3+$ average points on seven classification tasks of the GLUE benchmark~\citep{wang2018glue}. Ablation studies demonstrate the effectiveness of our proposed meta-weighted training objective and classifier fine-tuning method.


\vspace{-0.1cm}
\section{Related Work}
\label{sec:related_work}
\subsection{Knowledge Distillation}
Knowledge distillation, which aims to transfer knowledge from a cumbersome teacher model to a lightweight student model, has become one of the most popular model training techniques in both model compression and model performance boosting~\cite{distill_hinton,model_compression}. 
Knowledge distillation is originally proposed to train the student model to mimic the output probability distribution from teachers for image classification tasks~\cite{distill_hinton,deepmutuallearning}.
In addition to model outputs, intermediate features~\cite{fitnets,kd_crd}, attentions~\cite{attentiondistillation}, as well as relations~\cite{relational_kd,relational_kd2,fsp_kd,kd_cc}, all have been utilized for knowledge distillation.
%Then, abundant attempts have been made to distill teacher knowledge from the features~\cite{fitnets,kd_crd}, attention~\cite{attentiondistillation}, relation~
%\cite{relational_kd,relational_kd2,fsp_kd,kd_cc} perspective, 
Nowadays, knowledge distillation has successfully swept a wide range of applications, such as object detection~\cite{detectiondistillation,kd_detection4,kd_detection3}, semantic segmentation~\cite{structured_kd,he2019knowledge}, image generation~\cite{wkd}, self-supervised learning~\cite{kd_self1,kd_self2}, vision model pretraining~\cite{simclr,simclr2,moco}, language models~\cite{kd_bert1,kd_bert2}, data augmentation~\cite{labelrefine}, model robustness~\cite{kd_defense,auxiliarytraining} and so on.

\vspace{-0.4cm}
\paragraph{KD for 3D Detection}
Following the success of knowledge distillation on 2D tasks, a few works have been proposed to apply knowledge distillation to 3D object detection tasks.
For example, in the point cloud detection domain,
Zhang~\emph{et al.\@}~\cite{zhang2022pointdistiller} distills local information extracted by dynamic graph convolutions,
Cho~\emph{et al.\@}~\cite{cho2022itkd} distills the knowledge compressed by an AutoEncoder,
Yang~\emph{et~al.\@}~\cite{yang2022towards} propose to perform knowledge distillation only on the positions with high teacher classification responses,
%PointDistiller to compress point cloud-based detectors by distilling the local information extracted by dynamic graph convolution~\cite{zhang2022pointdistiller}. 
%Cho~\emph{et al.\@} proposed to distill the knowledge compressed by an AutoEncoder for point clouds-based 3D detection~\cite{cho2022itkd}.
%Yang~\emph{et~al.\@} proposed to perform knowledge distillation only on the positions with high teacher classification response~\cite{yang2022towards}.
Hou~\emph{et~al.\@}~\cite{hou2022point} propose to compress 3D segmentation by distilling the knowledge in both points and voxels,
Huang~\emph{et al.\@}~\cite{huang2022label} proposed label-guided auxiliary training which generates pseudo teacher features with label information.
In the vision based 3D detection domain,
Chong~\emph{et al.\@}~\cite{chong2022monodistill} improves the efficiency of monocular 3D detection with relation-based knowledge distillation.
%Besides, another popular research trend is to employ teachers and students in different modalities. For instance, 
Guo~\emph{et~al.\@} propose to learn stereo-based students from a LiDAR-based teacher~\cite{guo2021liga}. Sautier~\emph{et~al.\@} propose image-to-LiDAR self-supervised distillation which leverages the information of an image-based detector to improve the performance of 3D detector~\cite{sautier2022image}. 
Despite much success in these prior works, applying knowledge distillation to the modern multi-camera BEV detection paradigm has been rarely explored. We hope our attempts can inspire future research in this domain.
%Despite their success, unfortunately, specific knowledge distillation methods for BEV-based multi-view 3D detection has not been studied.
%Unfortunately, specific knowledge distillation methods for BEV-based multi-view 3D detection are still empty.


\vspace{-0.1cm}
\subsection{Camera-based 3D Detection}
Detecting 3D objects from images is one of the most challenging problems in 3D computer vision. 
With the help of modern deep learning techniques, much progress has been made for monocular 3D object detection~\cite{3d_object_1,3d_object_2,3d_object_3,fcos3d}.
However, monocular 3D object detection often suffer from truncation and occlusion problems, which are quite challenging without the help of additional sensors.
Detecting 3D objects using multi-cameras in the Birds'-Eye-View (BEV) thus have become quite popular recently.
% Mousavian~\emph{et al.\@} propose to detect 3D objects from a single image by first predicting the 2D bounding box and then estimating its 3D attributes~\cite{3d_object_2}.
% Xu~\emph{et al.\@} introduce a multi-level fusion scheme that uses a stand-alone module for disparity estimation~\cite{3d_object_3}.
% Wang~\emph{et al.\@} extend FCOS to monocular 3D object detection by transforming the 7-DoF 3D targets to the image domain~\cite{fcos3d}.
%Another rising topic in camera-based 3D detection is to predict 3D objects in Birds'-Eye-View (BEV) from multi-view images.
Huang~\emph{et~al.\@} propose BEVDet which encodes the feature of single views and then projects them to BEV space~\cite{huang2022bevdet4d,huang2021bevdet}.
Wang~\emph{et al.} propose Detr3D which manipulates predictions directly in 3D space by using 3D object queries to index the features of multi-view images~\cite{detr3d}.
PETR and PETRv2 are proposed to produce 3D position-aware features by encoding the position information of 3D coordinates into image features~\cite{liu2022petr,liu2022petrv2}.
%Then, PETRv2 is further introduced to leverage the temporal information by extending 3D positional embedding to temporal modeling~\cite{}.
Li~\emph{et al.} propose BEVFormer, which employs spatial cross-attention and temporal self-attention to merge the information of features in different spatial and temporal positions~\cite{bevformer}.
Then, BEVDepth is proposed to perform explicit depth supervision with encoded intrinsic and extrinsic parameters~\cite{bevdepth}.
Observing the fact that the optimal temporal difference between views varies significantly for different pixels and depths,
Park~\emph{et~al.} propose SOLOFusion which employs both short-term and long-term temporal stereo for depth estimation~\cite{park2022time}.
Li~\emph{et al.} propose BEVStereo which dynamically selects the scale of matching candidates to reduce the computation overhead~\cite{li2022bevstereo}. 
Following such trend, we focus on designing effective knowledge distillation strategies to push the envelope of such multi-camera BEV object detection paradigm.


\section{Reading in VR}
Reading documents in a virtual reality typically involves reading a 2D document placed in a window in a 3D virtual world. This is different from users’ experience of reading in 2D displays, as document windows can be positioned at various depths and orientations. This gap might cause discomfort and inconvenience in terms of VR reading, resulting in users being reluctant to attempt this activity on a VR platform. Since prior work has found that users did not prefer head-fixed presentation when reading a paragraph of texts in VR~\cite{rzayev2021reading} our goal was to study and develop tools that allow the reader to move freely in VR, select and attach/detach documents to a reading frame easily and enhance readability. We envisioned a scenario where readers might encounter multiple documents, both long and short, in VR and that users would need to select and deselect these documents to read them.  Whereas previous works have mostly focused on short paragraphs (approximately 100 words) with a uniform font size~\cite{rzayev2018reading} we instead chose to observe users' reading behavior with longer documents with different font sizes for structure.  Based on observations from our formative study, we identified user pain points for this task using current interaction techniques.



\subsection{Formative Study}
The goal of our formative study was to identify user pain points for reading in VR from selection to reading completion with currently available tools. We observed eight users interacting with six documents of varying lengths using available VR interaction techniques. 

\subsubsection{Selected Manipulation Method}
Users wore a state-of-the-art VR Headset and were provided with a set of interaction tools that are commonly used across VR platforms for object manipulation and 2D canvas interaction (as surveyed in Section 2). Manipulation was done using the HTC Vive controller. We first asked users to select document windows and make translation movements at distance using a "laser-pointer" raycasting method~\cite{steamvr2019, vrtoolkit2019}. When a VR controller button is pressed by the user, a laser, or a ray cast, is projected in the direction to which the user is pointing, and the first colliding object in the virtual world is selected. Users can also make translation movements by moving the controller while pressing the dedicated button. For 6DOF manipulation, we use a method where users can ``grab’' and move or rotate 2D windows, which is also a common object manipulation method~\cite{wang20116d}.  Here, when a direct collision is detected between the VR controller and a 2D document window while the assigned button is pressed, the object follows the motion of the controller. An example is depicted in Figure.~\ref{fig:formative}.
Based on these interactions, users can manipulate the 2D document windows that are at various orientations.



%\begin{figure}[t]


\begin{figure*}[t]
	\centering
	\includegraphics[width=1.0\linewidth]{pictures/vrdoc3.png}
	\caption{Interactions provided with \textit{VRDoc}. With \textit{Gaze Select-and-Snap}, (a) the document window is highlighted in green when focused and snaps in front of the user's head position when selected. \textit{Gaze MagGlass} creates a floating canvas that creates a magnifying effect shown in (b). \textit{Gaze Scroll} provides (d) gaze-activated buttons that scroll up or down by a sentence when gazed at.}
	
	\label{fig:vrdoc}
	\vspace{-1em}
\end{figure*}


\subsubsection{Participants and Procedure}
We recruited eight participants (three females, five males). Three had previous experience in VR but not in reading texts in VR. Participants wore an HTC Vive Pro Eye~\cite{vive} for the VR HMD, which has  $1440  \times 1600$ pixels per eye ($2880 \times 1600$ pixels combined), a $90$ Hz refresh rate, and a $110^{\circ}$ for the field of view. It also provides eye tracking capabilities.

Participants start from a fixed position with six documents placed in front of them as seen in Fig.~\ref{fig:formative}. The documents were placed such that they formed a semicircle around the user’s starting position. Four of the documents were short passages with about $100$ words, while the other two were long passages with about $500$ words in length. Following Dingler et al ~\cite{dingler2018vr}, the VR text presentation guideline by having a black background color and white text color for all documents. All six documents had the same canvas size, requiring the long documents to have scroll-able windows. The participants were able to scroll by selecting a document and moving their fingers vertically on the VR controller trackpad.

Each participant was given $20$ minutes to freely read all six documents without any specific order.  We followed the experience with a semi-constructed interview. The questions focused on the general experience of reading in VR, satisfaction with current interaction techniques, and desired user features.


\subsubsection{Observations and Feedback}
Through observing users’ behavior and collecting feedback through an interview, we were able to identify the following user pain points.

\paragraph{1. Positioning:} When selecting a document, it took participants multiple attempts  to re-orient the document window into the desired position. Unlike general object manipulation, participants tended to have a preferred distance and orientation (upright) for reading. Six participants noted that positioning the document to this preferred location and orientation took more time than expected. P1 commented, ``{\em It took me a while to figure out what position worked for me the best to fully view the document.}’’ All eight participants mentioned that switching their attention between multiple documents made them more aware of this inconvenience. P3 commented, ``{\em During the trial there was a moment when the document windows start to overlap as I select and position them. The pile definitely made it difficult to identify and select the documents. I wish there could be an easy way where I can quickly pick up a document that I want to read.}’’. Five participants brought up the need to automate the positioning procedure as they already knew how they wanted the document of interest to be oriented: up right in front of their head position. 

\paragraph{2. Readability:} All eight participants reported that document readability was poor due to its resolution and distortion. P2 and P5 commonly mentioned that when reading on a 2D display, they were even able to read texts sideways, but with the VR headset the text appeared blurry and this was not possible. Five participants noted that unless the document was perfectly upright, the slant created a distortion that decreased readability. We observed that participants positioned the documents with smaller font sizes closer, effectively magnifying the font, to enhance readability.   

\paragraph{3. Arm Fatigue:} During the $20$ minute trial, participants tended to hold the VR controller up constantly to interact with the documents. This behavior was consistently observed even when users were using the controller's trackpad for scrolling since the rest of the interactions, such as selecting and moving the document, required the user to hold up the physical device. Four of the participants reported arm fatigue which is a known issue ("Gorilla Arm") for gesture-based interactions~\cite{jang2017modeling} which are prevalent in VR. This is exacerbated by the additional weight of the controller when compared to watch-based gesture interactions. Such issues of fatigue would likely increase in longer VR experiences. 
\vspace{0.2em}
Based on this formative study, we developed tools specifically to address positioning, readability, and arm fatigue.


\subsection{Our Approach: VRDoc}
To address user pain points we developed three new tools: Gaze Select-and-Snap, Gaze MagGlass, and Gaze Scroll, which we collectively refer to as VRDoc, tools for better document reading in VR. This section describes our design process from user needs to potential solutions. 

Focused on the pain points of positioning, readability, and arm fatigue, our design thinking progressed as follows: 
\begin{enumerate} 
\itemsep-0.1em
\item Users do not seem to require or desire as much object manipulation freedom when reading documents.  Given the user tendency for a specific positioning with respect to documents, we should automate and simplify moving the document to a near-optimal position once selected.  Automatically positioning the document in an upright non-skewed position will enhance readability.  
\item Users manipulated documents to effectively magnify text, but this often led to documents being positioned too close for the reader to easily contextualize their place in the document.  A better solution would be to selectively magnify the current text the user is reading.
\item VR controller use and arm gestures should be minimized. While some VR controller use may still be required, in a virtual office, for longer reading tasks the user should not need to use their arms at all, enabling them to set the controller down.  We believe this will reduce fatigue and improve the overall experience.
\end{enumerate}
We believed that the novel eye tracking capability of the Vive headset could be leveraged to develop solutions for some of these issues.  Eye-tracking is becoming increasingly available in commodity HMDs~\cite{vive} and since reading naturally evokes specific eye movements, document interactions with gaze are natural and intuitive. 
%Therefore, we designed \textit{\textbf{VRDoc}}, a set of gaze-based interaction methods that would enhance users' reading experience. 
In our approach,  we use the SDK provided by HTC Vive\footnote{https://developer.vive.com/documents/718/VIVE\_Pro\_Eye\_user\_guide.pdf} for eye tracking calibration and data with the Unity game engine. The headset provides eye tracking with an accuracy of  $0.5^{\circ}$ to $1.1^{\circ}$ at 120 Hz.


%\begin{figure*}[h]
%	\centering
%	\includegraphics[width=1\linewidth]{pictures/VRDoc.png}
%	\caption{Interactions provided with \textit{VRDoc}. With \textit{Gaze Select-and-Snap}, (a) the document window is highlighted in green when selected and (b) the document snaps in front of the user's head position. (c) \textit{Gaze MagGlass} creates a floating canvas that creates a magnifying effect. \textit{d} provides gaze-activated buttons that scrolls up or down by a sentence when gazed at.}
%	\label{fig:VRDoc}
%\end{figure*}

\subsubsection{Positioning: Gaze Select-and-Snap}
The infinite freedom of object manipulation in 3D was actually a negative factor in document positioning.  Documents were only readable in the upright position near the users' direct line of sight.  We developed \textit{Gaze Select-and-Snap} to automate the action of selecting and positioning through the user's gaze. Prior work has established that users value the ability to select 3D objects with gaze and bring these closer to the user's hand~\cite{kennedy1993simulator}, but this is the first method designed for document objects (a virtual object that is tagged as ``document'') that both rotates the 3D object into a specific position and snaps it into a fixed effective 2D perspective specifically for reading. 

To engage Gaze Select-and-Snap, the user first directs their gaze toward the 3D document object, the gaze focus is detected and the document object is highlighted with a green stroke to visualize its selection for the user.  With a single click of the trigger button, the 3D document object is brought forward towards the head position and snapped into an effective 2D position in front of the user. The window is snapped parallel to the user's head position, ensuring the window stays upright. An example of this interaction is shown in Figure.~\ref{fig:vrdoc} (a).

When multiple documents overlap, Gaze Select-and-Snap first highlights the top document, then if the top document is not selected, Gaze Select-and-Snap sequentially brings hidden documents to the forefront until the desired document is identified and selected. 


\subsubsection{Readability: Gaze MagGlass}
To improve text readability once the document was in position, we incorporated a magnifying glass effect that is activated by users' eye movements: ~\textit{Gaze MagGlass}.  For low-vision computer users, video-based eye trackers have been used effectively to increase the on-screen magnification in traditional computing settings~\cite{wittich2018effectiveness, maus2020gaze}, however, to the best of our knowledge, this is the first implementation of interactive selective text magnification in VR. To enhance usability, \textit{Gaze MagGlass} is only activated when (1) the user gazes at a document, (2) the document is within a certain distance ($< 0.5 m$), and (3) when the user gazes at the document window for more than $1.5$ seconds. These robust heuristics were designed to ensure that the document in view is the specific document that the user wants to read. 

When the activation conditions are met, a second virtual camera is created on the collision point of the user’s gaze and the document object. The virtual camera is perpendicular to the document while following the user’s gaze.  The captured scene is rendered at a texture of a 2d plane that is rendered in front of the main camera. The field of view of the virtual camera and the distance from the document object are heuristically determined so that it magnifies the document by 150\% with a size that covers approximately $4$ to $5$ words of three consecutive sentences.

Directly applying raw gaze position data to the virtual camera causes great jittering as eye tracker data are inherently noisy and include tracking errors. This worsens the user experience as the jittering is visualized in a magnified way. To address this issue, we generally follow the saccade detection and smoothing algorithm from~\cite{kumar2007guide} so that the position of \textit{Gaze MagGlass} is calculated as a weighted mean of the set of points within a fixation window.

\textit{Gaze MagGlass} is automatically initiated when the activation conditions are met but can be manually turned on or off by the user if necessary. Note that activation conditions and the degree of magnification were heuristically determined for the study. An example of the activation is depicted in Figure.~\ref{fig:vrdoc} (b).

%To improve readability in an interaction perspective, we incorporated a magnifying glass effect that is activated by users' eye movements: ~\textit{Gaze MagGlass}. The idea of a gaze-guided magnifying lens has been actively researched for computer users~\cite{maus2020gaze}. \textit{Gaze MagGlass} is only activated when (1) the user gazes at a document, (2) the document is within a certain distance ($< 0.5 m$), and (3) when the user gazes at the document window for more than $1.5$ seconds. This is used to ensure that the document in view is the specific document that the user wants to read. 

%When the activation condition is met, a rectangular canvas that magnifies the document by 150\% with a size that covers approximately $5$ to $6$ words of three consecutive sentences appears on the document window. The floating canvas follows the user's eye movement. \textit{Gaze MagGlass} is automatically initiated when the activation conditions are met but can be manually turned on or off by the user if necessary. Note that activation conditions and the degree of magnification were heuristically determined for the study. An example of the activation is depicted in Figure.~\ref{fig:VRDoc} (c).

%With \textit{Gaze MagGlass}, users are relieved of the burden of continuously adjusting the document window when reading texts with different font sizes. The readability of the text is ensured because it enhances areas on which the eyes focus. Also, there is an advantage of being able to keep track of your position within the document. We evaluate it using studies and describe more details in Section 4.


\subsubsection{Gaze Scroll}
Since longer documents are rarely considered in VR reading studies, Gaze Scroll is the first tool designed specifically to help alleviate fatigue when reading longer documents.  The objective of gaze scroll is to avoid the necessity of users having to re-engage with the controller after they have begun reading.  At this point, the document should be snapped into the 2D reading position and the user should be able to put the controller down.  To facilitate document navigation without a controller, two buttons were placed within a document each on the top and the bottom of the window, as seen in Figure~\ref{fig:vrdoc} (c). When the user's gaze reaches the button for $0.5$ seconds, the document scrolls up or down by a full sentence.  Fixating the gaze on the button can increase the number of sentences. For example, if a user stares at the lower button for 2 seconds, the document scrolls down by four sentences.  This activation condition creates a controlled advancement and minimizes focal changes, which can be frequent when scrolling is rapid or uncontrolled. 

\vspace{0.5em}
\textit{VRDoc} tools are designed to facilitate a better reading experience in VR, from automating the selection and positioning of document windows to magnifying text for readability to allowing gaze-based navigation of longer documents.




\section{Experimental Setup}\label{sec:setup}

\paragraph{Downstream Tasks and Metrics.} 
We conduct evaluation on all tasks of the GLUE benchmark~\citep{wang2018glue} (more details in Appendix~\ref{app:glue}) except STS-B which is a regression task.
We follow the same data split and evaluation protocol as \cite{gao2021making}: 
Both $\mathcal{D}_{\text{train}}$ and $\mathcal{D}_{\text{dev}}$ contain $16$ samples per label and are sampled from the original training set with $5$ different random seeds.
The original development sets are used for testing.
For all reported results, we include the average and standard deviation over the $5$ different $\mathcal{D}_{\text{train}}$/$\mathcal{D}_{\text{dev}}$ splits.
F1 score is used as the metric for QQP and MRPC, Matthews correlation for CoLA, and accuracy for the remaining tasks.

\paragraph{Models, Training Settings and Hyperparameters.} 
\method is a training data generation method and can be used with any fine-tuning method on any classification model.
We use moderate-sized PLMs to ensure our results are reproducible on typical research hardware:
CTRL ($1.6$B parameters)~\citep{Keskar2019CTRLAC} as the generator $G_{\theta}$ and RoBERTa$_{\text{Large}}$ ($356$M parameters)~\citep{liu2019roberta} as the classifier $C_{\phi}$. 
% We also show the results using similar-sized PLMs (GPT-2~\citep{radford2019language}/RoBERTa~\citep{liu2019roberta}) as the generator/classifier in Section~\ref{sec:plms}.
We use prefix-tuning for training $G_{\theta}$ and prompt-based fine-tuning for training $C_{\phi}$.
For simplicity, we use the most basic manual prompt version of LM-BFF~\citep{gao2021making}.
The only exception is CoLA for which we use the standard fine-tuning since the input data might be out of the distribution of $C_{\phi}$~\citep{gao2021making}.
The hyperparameter tuning is performed on $\mathcal{D}_{\text{dev}}$. More details are in Appendix~\ref{app:impl_details}.

\paragraph{Compared Methods.} 
No-augmentation baselines include zero-shot prompting, standard fine-tuning, in-context learning, and the following strong few-shot learning methods: Four versions of LM-BFF~\citep{gao2021making}, P-Tuning~\citep{Liu2021GPTUT} and DART~\citep{zhang2022differentiable}.
We also compare \method with data augmentation methods for few-shot learning: Using back translation systems to generate paraphrases (UDA-style~\citep{Xie2020UnsupervisedDA} augmentation), GPT3Mix~\citep{Yoo2021GPT3MixLL} and standard fine-tuning of generator on the few-shot samples with prompts. All augmentation methods use LM-BFF (Man.) for fine-tuning the RoBERTa$_{\text{Large}}$ classifier.
More details about data augmentation baselines can be found in Appendix~\ref{app:aug_baselines}.
% We also include the results of fine-tuning using the entire training set. 

% \paragraph{Implementation Details.}
% For single-sequence tasks (SST-2 and CoLA), we tune $G_{\theta}$ on the entire few-shot sample and then use it to generate training data from scratch; for sequence-pair tasks (MNLI, QQP, QNLI, RTE and MRPC), we tune $G_{\theta}$ on the second sequence of the few-shot sample given the first sequence and then use it to generate training data by sampling the first sequence from the pretraining corpus (\eg, Wikipedia).
% The experiments are conducted on NVIDIA GeForce RTX 3090 and A6000 GPUs.
% More details can be found in Appendix~\ref{app:impl_details}.


\section{Evaluation}
\newcommand{\fullres}[2]{$#1_{#2}$}

\begin{table*}[!t]
\caption{
Results on seven classification tasks of the GLUE benchmark. We report average
and standard deviation (as subscripts) performance over $5$ different $\mathcal{D}_{\text{train}}$/$\mathcal{D}_{\text{dev}}$ splits defined in \cite{gao2021making}. $^\dagger$: Results from \cite{gao2021making}. $^\ddagger$: Results from \cite{zhang2022differentiable}.
Methods that use additional models apart from the final classification model are marked.
}
\vspace{-.5em}
\centering
\small 
\resizebox{\textwidth}{!}{
\begin{tabular}{l*{8}{c}}
\toprule
\multirow{2}{*}{\textbf{Method}} & \textbf{MNLI-(m/mm)} & \textbf{QQP} & \textbf{QNLI} & \textbf{SST-2} & \textbf{CoLA} & \textbf{RTE} & \textbf{MRPC} & \textbf{AVG} \\ 
% \cmidrule(lr){2-3}
& (Acc.) & (F1) & (Acc.) & (Acc.) & (Matt.) & (Acc.) & (F1) &  \\
\midrule
\multicolumn{9}{l}{\textit{Methods without Augmentation}: Few-shot samples are directly used for classifier tuning or as demonstrations for inference} \\ 
\midrule
Prompting$^\dagger$ & $50.8$/$51.7$ & $49.7$ & $50.8$ & $83.6$ & $2.0$ & $51.3$ & $61.9$ & $50.1$ \\
Fine-Tuning$^\dagger$ & \fullres{45.8}{6.4}/\fullres{47.8}{6.8} & \fullres{60.7}{4.3} & \fullres{60.2}{6.5} & \fullres{81.4}{3.8} & \fullres{33.9}{14.3} & \fullres{54.4}{3.9} & \fullres{76.6}{2.5} & $59.1$ \\
In-Context$^\dagger$ &
\fullres{52.0}{0.7}/\fullres{53.4}{0.6} & \fullres{36.1}{5.2} & \fullres{53.8}{0.4} & \fullres{84.8}{1.3} & \fullres{-1.5}{2.4} & \fullres{60.4}{1.4} & \fullres{45.7}{6.0} & $47.4$ \\
LM-BFF (Man.)$^\dagger$ & \fullres{68.3}{2.3}/\fullres{70.5}{1.9} & \fullres{65.5}{5.3} & \fullres{64.5}{4.2} & \fullres{92.7}{0.9} & \fullres{9.3}{7.3} & \fullres{69.1}{3.6} & \fullres{74.5}{5.3} & $63.6$ \\
\, + demonstration$^\dagger$ & \fullres{70.7}{1.3}/\fullres{72.0}{1.2} & \fullres{69.8}{1.8} & \fullres{69.2}{1.9} & \fullres{92.6}{0.5} & \fullres{18.7}{8.8} & \fullres{68.7}{2.3} & \fullres{77.8}{2.0} & $66.9$ \\
LM-BFF (Auto)$^\dagger$ (w. $2.9$B T5) & \fullres{68.3}{2.5}/\fullres{70.1}{2.6} & \fullres{67.0}{3.0} & \fullres{68.3}{7.4} & \fullres{92.3}{1.0} & \fullres{14.0}{14.1} & \fullres{\textbf{73.9}}{2.2} & \fullres{76.2}{2.3} & $65.8$ \\
\, + demonstration$^\dagger$ (w. $2.9$B T5) & \fullres{70.0}{3.6}/\fullres{72.0}{3.1} & \fullres{67.7}{5.8} & \fullres{68.5}{5.4} & \fullres{93.0}{0.6} & \fullres{21.8}{15.9} & \fullres{71.1}{5.3} & \fullres{78.1}{3.4} & $67.3$ \\
P-Tuning$^\ddagger$ &
\fullres{61.5}{2.1}/\fullres{-}{} & \fullres{65.6}{3.0} & \fullres{64.3}{2.8} & \fullres{92.2}{0.4} & \fullres{-}{} & \fullres{-}{} & \fullres{74.5}{7.6} & $-$ \\
DART$^\ddagger$ & \fullres{67.5}{2.6}/\fullres{-}{} & \fullres{67.8}{3.2} & \fullres{66.7}{3.7} & \fullres{93.5}{0.5} & \fullres{-}{} & \fullres{-}{} & \fullres{78.3}{4.5} & $-$ \\
\midrule
\multicolumn{9}{l}{\textit{Methods with Augmentation}: Few-shot samples are used for creating synthesized samples and for classifier tuning} \\ 
\midrule
MixText & {\fullres{65.1}{2.6}/\fullres{66.2}{2.8}} & {\fullres{60.6}{3.9}} & {\fullres{68.4}{5.1}} & {\fullres{89.1}{2.3}} & {\fullres{12.8}{9.2}} & {\fullres{66.5}{4.1}} & {\fullres{64.6}{7.6}} & {$61.1$} \\
Back Translation (w. trained Marian) & \fullres{66.9}{4.6}/\fullres{68.3}{3.8} & \fullres{59.8}{4.6} & \fullres{67.8}{4.9} & \fullres{91.1}{1.9} & \fullres{7.5}{3.7} & \fullres{62.4}{5.3} & \fullres{68.0}{11.2} & $60.6$ \\
GPT3Mix (w. $175$B GPT3) & \fullres{61.5}{3.2}/\fullres{62.6}{2.2} & \fullres{70.4}{1.9} & \fullres{69.2}{0.3} & \fullres{\textbf{93.6}}{0.6} & \fullres{\textbf{48.9}}{1.9} & \fullres{70.4}{10.0} & \fullres{69.9}{12.4} & $69.2$ \\
% \midrule
% COCO-LM results
% \method (w. $1.6$B CTRL) & \fullres{\textbf{74.6}}{1.1}/\fullres{\textbf{75.3}}{1.2} & \fullres{68.8}{3.0} &  \fullres{\textbf{69.7}}{4.8} & \fullres{93.1}{0.5} & \fullres{45.1}{8.8} & \fullres{65.6}{2.6} & \fullres{\textbf{79.9}}{1.7} & $\textbf{71.0}$ \\
% RoBERTa results
Generator Fine-Tuning (w. $1.6$B CTRL) & \fullres{68.9}{5.1}/\fullres{70.8}{5.3} & \fullres{60.4}{8.7} &  \fullres{70.9}{4.1} & \fullres{91.2}{1.2} & \fullres{18.8}{10.0} & \fullres{66.1}{4.4} & \fullres{60.8}{15.4} & 62.6 \\
\method (w. $1.6$B CTRL) & \fullres{\textbf{75.7}}{1.6}/\fullres{\textbf{77.1}}{1.0} & \fullres{\textbf{71.5}}{1.7} &  \fullres{\textbf{76.3}}{4.4} & \fullres{93.1}{0.8} & \fullres{40.0}{7.5} & \fullres{71.2}{2.4} & \fullres{\textbf{81.1}}{2.5} & $\textbf{72.8}$ \\
\midrule
Fully Supervised Fine-Tuning$^\dagger$ & \textit{89.8}/\textit{89.5} & \textit{81.7} & \textit{93.3} & \textit{95.0} & \textit{62.6} & \textit{80.9} & \textit{91.4} & $\textit{84.9}$ \\
\bottomrule
\end{tabular}
}
\label{tab:main_res}
\vspace{-.5em}
\end{table*}

\subsection{Main Results}
We present the results of \method and baselines in Table~\ref{tab:main_res}.
\method achieves overall better performance across the GLUE tasks, on average $5+$ points higher than the previous best few-shot method without augmentation, and $3+$ points better than GPT3Mix\footnote{The CoLA results reported in the original GPT3Mix paper use accuracy as the metric instead of Matthews correlation; our reimplemented GPT3Mix achieves ${79.4}_{0.6}$ on CoLA if measured by accuracy.}~\citep{Yoo2021GPT3MixLL} which uses a $100$ times larger  generator model ($175$B) than \method.
The promising results confirm the effectiveness of our proposed \method method in generating quality training data and leveraging them in combination with the few-shot training set for fine-tuning the classification model.
% The improved model performance is in accordance with our intuition that training the classification model with more data benefits generalization. 
\paragraph{Comparison with Back Translation.} Using back translation to paraphrase the few-shot samples does not improve the results, even with prompt-based fine-tuning to train the classifier -- this is probably because it does not produce samples that are sufficiently different from the few-shot training set.
The success of UDA~\citep{Xie2020UnsupervisedDA} is grounded in the augmentations from abundant unlabeled data that improve the classifier generalization.
However, under the strict few-shot learning setup, there is no access to additional task-specific unlabeled data~\citep{gao2021making}, making it challenging for paraphrase-based methods to create sufficiently diverse training samples only based on the small few-shot set. The new training samples produced by our \method method are not limited to the paraphrases of the few-shot samples, as the generator is trained via prefix-tuning to preserve the PLM's pretraining knowledge, based on which novel training samples can be synthesized.
\paragraph{Comparison with GPT3Mix.} The gigantic size of GPT3 makes it challenging for tuning on few-shot samples. Therefore, GPT3Mix~\citep{Yoo2021GPT3MixLL} uses few-shot samples as demonstrations for creating the augmentations. Such an approach suffers from two limitations: (1) Without any parameter update to the PLM, its learning ability is not fully leveraged to adapt to the few-shot training set. (2) The PLM can only use a small subset of the few-shot samples at a time for creating each augmentation, as the number of demonstrations received by the model is bounded by its maximum input sequence length. This makes the quality of the created augmentations more sensitive to the  randomly drawn training samples. Our \method method, on the other hand, can use the entire few-shot set for tuning the PLM and achieves overall even better classification results with a much smaller PLM ($<1\%$ the size of the GPT3 model) which can be deployed much more easily in practice.

\subsection{Ablation Studies}
\begin{table*}[t]
\setlength{\tabcolsep}{1.5mm}
\centering
\small
\resizebox{\textwidth}{!}{%
    \begin{tabular}{l|ccccc|ccc|cc|c|c}
        \toprule[1pt]
        & \multicolumn{5}{c|}{\textbf{Natural Language Inference}} & \multicolumn{3}{|c|}{\textbf{Sentence Completion}} & \multicolumn{2}{c|}{\textbf{Coreference}} & \multicolumn{1}{c|}{\textbf{WSD}} & \multirow{2}{*}{Avg.} \\
    & RTE & CB & ANLI1 & ANLI2 & ANLI3 & COPA & Hella. & Story. & WSC & Wino. & WiC &  \\
    \midrule[1pt]
    UD (Minimal)     & \textbf{83.75}
        & \textbf{80.36}
        & 36.80
        & \textbf{34.20}
        & \textbf{42.17}
        & \textbf{90.00}
        & \textbf{56.07}
        & \textbf{96.37}
        & \textbf{68.27}
        & \textbf{62.90}
        & \textbf{54.55}	
        & \textbf{64.13} \\
    UD (Instructive)    & 72.24 
        & 64.52 
        & \textbf{36.98} 
        & 33.40 
        & 39.73 
        & 85.31 
        & 45.15 
        & 96.01 
        & 65.38 
        & 53.94 
        & 50.94 
        & 58.51\\
    \midrule
    T0 (Minimal) & 61.56  & \textbf{57.81}  & 30.57  & 30.27  & 33.38  & 67.19  & \textbf{33.81}  & 66.56  & 60.94  & 52.81  & \textbf{51.72}  & 49.69  \\
    T0 (Instructive) & \textbf{75.05}	& 55.48	& \textbf{32.87}	& \textbf{32.29}	& \textbf{33.67}	& \textbf{84.59}	& 28.24	& \textbf{93.97}	& \textbf{62.98}	& \textbf{54.59}	& 51.16	& \textbf{54.99} \\

    \bottomrule[1pt]
    \end{tabular}}
    \caption{Zero-shot performance for UD and T0 respectively with instructive and minimal prompts. Instructive prompts are lengthy descriptions of tasks \citep{T0-paper}, while minimal prompts use a simple concatenation of input data.}
\label{tab:promptablatiion}
\end{table*}

We further analyze the effectiveness of each important component in \method. 
Specifically, we compare \method with the following ablations:
(1) Using the standard $\mathcal{L}_{\text{gen}}$ in Eq.~\eqref{eq:gen} instead of our proposed  $\mathcal{L}_{\text{w-gen}}$ in Eq.~\eqref{eq:weigh_gen} for generator tuning (w. $\mathcal{L}_{\text{gen}}$);
(2) using the directly combined $\mathcal{L}_{\text{gen}}$ and $\mathcal{L}_{\text{disc}}$ for generator tuning (w. $\mathcal{L}_{\text{gen}}+\mathcal{L}_{\text{disc}}$);
(3) without applying label smoothing in Eq.~\eqref{eq:finetune} ($-$ label smooth);
(4) without applying temporal ensembling in Eq.~\eqref{eq:finetune} ($-$ temporal ensemble).
% (4) directly fine-tuning the classification model on $\mathcal{D}_{\text{gen}}$ instead of first on $\mathcal{D}_{\text{train}}$ and then on $\mathcal{D}_{\text{gen}}$ (- fine-tune on $\mathcal{D}_{\text{train}}$).
% and (4) using RoBERTa$_{\text{Large}}$~\citep{liu2019roberta} instead of COCO-LM$_{\text{Large}}$ as the classification model (w. RoBERTa).
As shown in Table~\ref{tab:ablation}, 
(1) \& (2) using the standard maximum likelihood loss or the combination of generation and discrimination losses to tune the generator both yield lower-quality training data and lead to degraded classification performance; 
(3) \& (4) not applying regularization techniques for fine-tuning the classifier is more prone to label noise in the generated samples.
% (4) not leveraging the in-domain data $\mathcal{D}_{\text{train}}$ for training the classifier results in worse model initialization for training on $\mathcal{D}_{\text{gen}}$, and the eventual classifier will not perform very well.
% (4) the model performance is not very sensitive to the choice of classifier PLMs. 
% 
\begin{wrapfigure}[16]{wr}{0.33\textwidth}
% \vspace{-0.3cm}
% \subfigcapmargin=10pt
\centering
\includegraphics[width=0.33\textwidth]{Figs/num_data_plot.pdf}
\caption{(On MNLI) The model performance with different amount of generated samples. Dots and error bars denote the average performance and the standard deviation over $5$ different train/development set splits.}
\label{fig:num_data}
\vspace{-0.5em}
\end{wrapfigure}

To study the impact of the amount of generated training samples on the model performance, we plot the MNLI-m accuracy (mean and standard deviation) with different sizes of $\mathcal{D}_{\text{gen}}$ in Fig.~\ref{fig:num_data}. 
Both the average model performance and stability improve with more generated samples.


% \begin{figure}\TopFloatBoxes

% \begin{floatrow}
% \ffigbox[\FBwidth]{
% \label{fig:disc_loss}
% % \vspace{-0.2em}
% \includegraphics[width=0.3\textwidth]{Figs/num_data.pdf}
% \caption{Number of generated training data.}
% }
% \ffigbox[\FBwidth]{
% \begin{subfigure}[t]{0.2\textwidth}
% \centering
% \includegraphics[width=\textwidth]{Figs/disc_loss.pdf}
% \caption{MNLI-m}
% \end{subfigure}%
% ~
% \begin{subfigure}[t]{0.2\textwidth}
% \centering
% % 	\label{fig:efficency_mm}
% \includegraphics[width=\textwidth]{Figs/eval_loss.pdf}
% \caption{MNLI-mm}
% \end{subfigure}
% \caption{loss.\label{fig:efficiency} 
% }
% }
% % \hspace{0.1cm}
% \end{floatrow}
% % \vspace{-1em}
% \end{figure}

\begin{figure}[t]
\begin{minipage}{0.33\textwidth}
% \vspace{-0.2em}
\centering
\includegraphics[width=0.95\textwidth]{Figs/num_data_plot.pdf}
% \vspace{+0.5em}
\caption{MNLI-m accuracy with different amounts of generated training data.}\label{fig:num_data}
\end{minipage}\hfill
\begin{minipage}{0.625\textwidth}
\begin{subfigure}[t]{0.46\textwidth}
\centering
\includegraphics[width=\textwidth]{Figs/disc_loss.pdf}
% \vspace{-0.5em}
\end{subfigure}%
~
\begin{subfigure}[t]{0.48\textwidth}
\centering
% 	\label{fig:efficency_mm}
\includegraphics[width=\textwidth]{Figs/eval_loss.pdf}
% \vspace{-0.5em}
\end{subfigure}
\vspace{-0.1em}
\caption{With different loss functions used for generator tuning, (Left) $\mathcal{L}_{\text{disc}}$ and (Right) standard language modeling loss on the development set. Best viewed in color. \label{fig:loss_funcs} 
}
\end{minipage}
% \hspace{0.1cm}
\vspace{-1.5em}
\end{figure}


\subsection{Analyses of Loss Functions for Generator Tuning}
As shown in Table~\ref{tab:ablation}, the choice of generator loss has a significant impact on the synthesized data quality and thus the final model performance.
We conduct further analyses to compare the training processes of the generator under the following three loss functions and the resulting generated samples:
(1) $\mathcal{L}_{\text{gen}}$ which is the standard language modeling loss;
(2) $\mathcal{L}_{\text{gen}}+\mathcal{L}_{\text{disc}}$ which directly adds the discriminative loss to generator training;
and (3) $\mathcal{L}_{\text{w-gen}}$ which is our meta-weighted objective.
Fig.~\ref{fig:loss_funcs} shows the discriminative loss $\mathcal{L}_{\text{disc}}$ and the standard language modeling loss on the held-out development set throughout training.
Although using $\mathcal{L}_{\text{gen}}+\mathcal{L}_{\text{disc}}$ helps reduce the discriminative loss, it comes at the cost of hindering language modeling---the generator loss on the development set is high.
Using our meta-weighted objective $\mathcal{L}_{\text{w-gen}}$ for tuning the generator not only encourages discriminativeness but also mitigates overfitting, yielding the lowest validation set loss.
This is probably because the model receives contrastive information from other labels which facilitates more accurate modeling of the texts with the target label.
We also showcase concrete generation results for the three labels of MNLI by models trained with the three different loss functions in Table~\ref{tab:case_studies}.
The model trained with $\mathcal{L}_{\text{gen}}$ produces fluent and coherent sentences, but the generated sentences do not accurately pertain to the desired label (\ie, the ``entailment'' and ``contradiction'' generation results are in fact neutral with respect to the given sentence), lacking label discriminativeness.
When $\mathcal{L}_{\text{gen}}+\mathcal{L}_{\text{disc}}$ is used, the generated samples of different labels are more distinctive, but also become less natural and coherent due to the model's language modeling ability being hampered.
The generator tuned with $\mathcal{L}_{\text{w-gen}}$ produces both coherent and label-discriminative samples which can serve as quality training data.
We also visualize the token weights $\bs{w}$ automatically learned and used in $\mathcal{L}_{\text{w-gen}}$ in Appendix~\ref{app:vis}.

\newcommand{\wronglabel}[1]{\textcolor{red}{\textit{(#1)}}}
\newcommand{\correctlabel}[1]{\textcolor{blue}{\textit{(#1)}}}
\begin{table*}[!t]
% \renewcommand\arraystretch{1.2}
\caption{
(For MNLI) Examples of generated second sequence (hypothesis) by generators tuned with three different objectives conditioned on a given first sequence (premise) ``\textit{In 2009, hair samples from 1,137 Taiji residents were tested for mercury by the National Institute for Minamata Disease (NIMD)}''. The true label of the generated sequence is marked at the end of the sequence (if the generated sequence correctly pertains to the target label, it is marked in \textcolor{blue}{blue}; otherwise, it is in \textcolor{red}{red}).
}
\vspace{-0.5em}
\centering
\small 
\resizebox{\textwidth}{!}{
\begin{tabular}{lll}
\toprule
\textbf{Objective} & \textbf{Label} & \textbf{Generated Second Sequence} \\
\midrule
\multirow{4}{*}{$\mathcal{L}_{\text{gen}}$} & \makecell[l]{entailment\\  } & \makecell[l]{The NIMD found that hair samples from 1,137 Taiji residents had mercury in their hair. \wronglabel{neutral}} \\
& \makecell[l]{neutral\\ } & \makecell[l]{The NIMD found that there was no evidence of a link between exposure to high levels \\ of mercury and thyroid cancer. \correctlabel{neutral}} \\
& \makecell[l]{contradiction\\ } & \makecell[l]{There was no evidence of mercury in hair samples from Taiji. \wronglabel{neutral}}\\
\midrule
\multirow{3}{*}{$\mathcal{L}_{\text{gen}}+\mathcal{L}_{\text{disc}}$} & {entailment} & The number of hairs in a sample is equal to the number of people who lived in Taiji. \wronglabel{neutral} \\
& {neutral} & The results showed that there was no significant difference in levels of mercury. \correctlabel{neutral} \\
& contradiction & Hair samples from 1,137 Taiji residents were not tested. \correctlabel{contradiction} \\
\midrule
\multirow{4}{*}{$\mathcal{L}_{\text{w-gen}}$} & {entailment} & \makecell[l]{The NIMD tested hair samples from 1,137 residents of Taiji. \correctlabel{entailment}} \\
& {neutral} & \makecell[l]{There was no significant difference in levels between people who lived near a nickel mine \\ and those living far away. \correctlabel{neutral}} \\
& contradiction & The NIMD did not test any of the hair samples. \correctlabel{contradiction} \\
\bottomrule
\end{tabular}
}
\vspace{-.5em}
\label{tab:case_studies}
\end{table*}


% \begin{figure*}[t]
% \subfigcapmargin=10pt
\centering
\includegraphics[width=.9\textwidth]{Figs/vis.pdf}
\vspace{-.5em}
\caption{Visualization of learned token weights on two samples from MNLI's few-shot training set.
The generator is trained given the first sentence to generate the second.
% Darker colors reflect higher token weights.
The tokens associated with higher weights are more label indicative.
}
\label{fig:vis}
% \vspace{-0.3cm}
\vspace{-1em}
\end{figure*}
% \subsection{Visualization of Token Weight Learning}
% To gain intuitive understanding of what tokens are assigned more weight during generator tuning, we visualize the learned weights in Fig.~\ref{fig:vis}.
% The tokens with higher weights (\eg, ``weak'' in the first example and ``hates'' in the second example) are learned to be important tokens that decide the relation of the second sentence to the first sentence (\ie, the label of the training sample).
% With such tokens emphasized during training, the generator is encouraged to capture label-discriminative information that facilitates the generation of unambiguous training samples.



\section{Discussions and Conclusions}\label{sec:concl}
\paragraph{Ethical Considerations.}
% While PLMs have demonstrated remarkable text generation and understanding capability, they can come with potential risks or harms~\citep{Bender2020ClimbingTN,Bender2021OnTD,Brown2020LanguageMA} such as generating misinformation~\citep{Pagnoni2021UnderstandingFI} or amplifying harmful biases~\citep{Prabhumoye2018StyleTT}. The focus of our work is on utilizing existing PLMs to generate training data for NLU tasks instead of developing new PLMs or generation methods.
% Therefore, our method can be used in company with any bias reduction and correction techniques~\citep{Gehman2020RealToxicityPromptsEN,Ma2020PowerTransformerUC} to mitigate the risks of PLMs.
Despite the impressive text generation and representation power of PLMs, they can also come with the risk~\citep{Bender2021OnTD,Bender2020ClimbingTN,Brown2020LanguageMA} of generating disinformation~\citep{Pagnoni2021UnderstandingFI} or exacerbating biases~\citep{Prabhumoye2018StyleTT}. Instead of improving upon PLM architectures or generation techniques, our work focuses on using existing PLMs to create training data for NLU tasks. Therefore, our method can be combined with any bias reduction and correction strategies~\citep{Gehman2020RealToxicityPromptsEN,Ma2020PowerTransformerUC} in practice to reduce the adverse effects of PLMs.

\paragraph{Limitations.}

Compared to few-shot learning methods that directly train classification models on the small training set, \method requires tuning a generator PLM and using it to synthesize novel training samples, resulting in higher computation costs and longer running time.
Still, we believe that our method may bring more good than harm---when the small training data size becomes the performance bottleneck for NLU tasks, a simple yet costly solution is to obtain more human annotations.
Our method may replace or reduce the human efforts in such training data creation processes.

\paragraph{Conclusions.}
In this work, we propose \method, 
which leverages few-shot training samples to tune a generator PLM for synthesizing novel training data.
The generated data can be then used in combination with few-shot samples to fine-tune a classification model for better generalization.
To emphasize label-discriminative information during generator tuning, we propose a weighted maximum likelihood objective where the token weights are automatically learned via a discriminative meta objective.
Since the generated samples may contain label noise, we propose a simple training procedure that first trains classifiers on the few-shot training set and then on the generated set by applying temporal ensembling for noise-robustness.
Across seven classification tasks from the GLUE benchmark, \method significantly outperforms existing approaches under the same few-shot learning setting.
The effectiveness of each important component in \method is validated via ablation studies.
Future work directions may include: Using larger PLMs as the generator and the classifier, jointly training both models with each other's high-confident predictions, and developing systematic metrics for evaluating the quality of generated training samples.



\section*{Acknowledgments}
Research was supported in part by US DARPA KAIROS Program No.\ FA8750-19-2-1004 and INCAS Program No.\ HR001121C0165, National Science Foundation IIS-19-56151, IIS-17-41317, and IIS 17-04532, and the Molecule Maker Lab Institute: An AI Research Institutes program supported by NSF under Award No.\ 2019897, and the Institute for Geospatial Understanding through an Integrative Discovery Environment (I-GUIDE) by NSF under Award No.\ 2118329. Any opinions, findings, and conclusions or recommendations expressed herein are those of the authors and do not necessarily represent the views, either expressed or implied, of DARPA or the U.S. Government.
Yu Meng was supported by the Google PhD Fellowship.
We thank anonymous reviewers for valuable and insightful feedback.

\balance
\bibliography{ref}
\bibliographystyle{icml2023}



%%%%%%%%%%%%%%%%%%%%%%%%%%%%%%%%%%%%%%%%%%%%%%%%%%%%%%%%%%%%

\newpage
\onecolumn
\appendix

\section{Details of Weighting Network Implementation}\label{app:weight_net}
Since the token weights $\bs{w}$ used in Eq.~\eqref{eq:meta_disc} need to characterize the discriminativeness of each token, we use the value of discriminative objective at each token  $\mathcal{L}_{\text{disc}}^j$ as the input to the weighting network, and we use softmax to normalize the weights:
% $$
% w_j(\bs{\omega}) = g_{\bs{\omega}}(\mathcal{L}_{\text{disc}}^j).
% % \quad
% % \mathcal{L}_{\text{disc}}^j = -\log p_{\bs{\theta}_{p}}(y_l|\bs{x}_{\le j}).
% $$
% We further normalize 
$$
w_j(\bs{\omega}) =  \frac{\exp\left(g_{\bs{\omega}}(\mathcal{L}_{\text{disc}}^j)\right)}{\sum_{{j'}=1}^n \exp\left(g_{\bs{\omega}}(\mathcal{L}_{\text{disc}}^{j'})\right)}.
$$

Following \cite{Shu2019MetaWeightNetLA}, we instantiate $g_{\bs{\omega}}$ to be a feedforward network (FFN) with only one $100$-dimension hidden layer by default.
% We explore an alternative instantiation that adds one self-attention layer on top of the generator PLM's output hidden states. The meta weights are finally obtained by projecting the outputs of the self-attention layer using another linear layer.
% We evaluate the resulting generator quality via the same two metrics as in Table~\ref{tab:gen_eval}.
% Table~\ref{tab:weight_net} shows that using more complicated architectures (\eg, adding another self-attention layer) does not result in a better generator compared to using a simple FFN for meta weight learning. 
% This is probably because the generator PLM's output representations are sufficiently contextualized and contain the information necessary for learning the token weights, thus a simple FFN as the weighting network will be enough.
% Using more complicated networks, on the other hand, will introduce more randomly initialized new parameters which may not be learned well using the limited amount of few-shot training data.

\section{Implementation Details}
\label{app:impl_details}
\begin{table}[!tbh]
\caption{
Prompts used for initializing the prefix vectors and control codes (required by CTRL~\citep{Keskar2019CTRLAC}) used in generator training.
The control codes are selected to approximiate the domain of the task.
For single-sequence tasks, $\bs{x}$ denotes the training sample; for sequence-pair tasks, $\bs{x}_1$ and $\bs{x}_2$ denote the first and second sequence in the training sample, respectively. 
% The ``not entailment'' label of MRPC and ``not equivalent'' label of RTE use two prompts split by ``//'' (essentially combining the prompts used for ``neutral'' and ``contradiction'' labels of the MNLI task).
}
% \vspace{1em}
\centering
\small 
\resizebox{\columnwidth}{!}{
\begin{tabular}{lllll}
\toprule
\textbf{Task} & \textbf{Task Type} & \textbf{Control Code} & \textbf{Label} & \textbf{Initialization Prompt} \\
\midrule
\multirow{2}{*}{\textbf{SST-2}} & \multirow{2}{*}{single-sequence} & \multirow{2}{*}{Reviews} & positive & Rating: 5.0 positive movie review: $\bs{x}$ \\
& & & negative & Rating: 1.0 negative movie review: $\bs{x}$ \\
\midrule
\multirow{2}{*}{\textbf{CoLA}} & \multirow{2}{*}{single-sequence} & \multirow{2}{*}{Links} & grammatical & Linguistically correct sentence: $\bs{x}$ \\
& & & not grammatical & Linguistically incorrect sentence: $\bs{x}$ \\
\midrule
\multirow{3}{*}{\textbf{MNLI}} & \multirow{3}{*}{sequence-pair} & \multirow{3}{*}{Wikipedia} & entailment & Sentence 1 implies Sentence 2. Sentence 1: $\bs{x}_1$ Sentence 2: $\bs{x}_2$ \\
& & & neutral & Sentence 2 supplements Sentence 1. Sentence 1: $\bs{x}_1$ Sentence 2: $\bs{x}_2$ \\
& & & contradiction & Sentence 2 contradicts Sentence 1. Sentence 1: $\bs{x}_1$ Sentence 2: $\bs{x}_2$ \\
\midrule
\multirow{2}{*}{\textbf{QNLI}} & \multirow{2}{*}{sequence-pair} & \multirow{2}{*}{Links} & entailment & Paragraph is relevant to Question. Question: $\bs{x}_1$ Paragraph: $\bs{x}_2$ \\
& & & not entailment & Paragraph is irrelevant to Question. Question: $\bs{x}_1$ Paragraph: $\bs{x}_2$ \\
\midrule
\multirow{2}{*}{\textbf{RTE}} & \multirow{2}{*}{sequence-pair} & \multirow{2}{*}{Wikipedia} & entailment & Sentence 1 implies Sentence 2. Sentence 1: $\bs{x}_1$ Sentence 2: $\bs{x}_2$ \\
& & & not entailment & Sentence 2 supplements Sentence 1. Sentence 1: $\bs{x}_1$ Sentence 2: $\bs{x}_2$ \\
\midrule
\multirow{2}{*}{\textbf{MRPC}} & \multirow{2}{*}{sequence-pair} & \multirow{2}{*}{Wikipedia} & equivalent & Sentence 1 is equivalent to Sentence 2. Sentence 1: $\bs{x}_1$ Sentence 2: $\bs{x}_2$ \\
& & & not equivalent & Sentence 1 is different from Sentence 2. Sentence 1: $\bs{x}_1$ Sentence 2: $\bs{x}_2$ \\
\midrule
\multirow{2}{*}{\textbf{QQP}} & \multirow{2}{*}{sequence-pair} & \multirow{2}{*}{Links} & equivalent & Question 1 is equivalent to Question 2. Question 1: $\bs{x}_1$ Question 2: $\bs{x}_2$ \\
& & & not equivalent & Question 1 is different from Question 2. Question 1: $\bs{x}_1$ Question 2: $\bs{x}_2$ \\
\bottomrule
\end{tabular}
}
\label{tab:full_prompts}
\vspace{-1em}
\end{table}


\paragraph{Details of Initialization Prompts Used for Generator Tuning on Different Tasks.}

For generator tuning, we find it beneficial to initialize the prefix vectors with task-descriptive prompts, similar to the observations in \cite{Li2021PrefixTuningOC}.
The prefix lengths (\ie, number of trained prefix token positions) are equal to the number of tokens in the prompts.
We present details about the prompts used for initializing the prefix vectors for different tasks in Table~\ref{tab:full_prompts}.
For sequence-pair tasks, an additional infix prompt is used between the two sequences,
and we also tune the embeddings of the infix (\ie, prompt-tuning~\citep{Lester2021ThePO}) for generator training.



\paragraph{Details of Generator Tuning.}
% 
\begin{wrapfigure}[16]{wr}{0.33\textwidth}
% \vspace{-0.3cm}
% \subfigcapmargin=10pt
\centering
\includegraphics[width=0.33\textwidth]{Figs/num_data_plot.pdf}
\caption{(On MNLI) The model performance with different amount of generated samples. Dots and error bars denote the average performance and the standard deviation over $5$ different train/development set splits.}
\label{fig:num_data}
\vspace{-0.5em}
\end{wrapfigure}

% In Algorithm~\ref{alg:meta}, we use SGD with $2e-2$ as the learning rate for the first gradient update (\ie, optimizing $\hat{\theta}_{p}^{(t)}\left(\omega^{(t)}\right)$); we use SGD with $1e-2$ as the learning rate for the second gradient update (\ie, optimizing $\omega^{(t+1)}$); we use Adam~\citep{Kingma2015AdamAM} with $5e-3$ as the learning rate for the third gradient update (\ie, optimizing $\theta_{p}^{(t+1)}$).
The meta-weighted generator tuning procedure (Algorithm~\ref{alg:meta}) involves three forward and backward passes, and thus its time complexity is approximately $3$ times of standard generator training without meta learning.
However, since the few-shot training sets have a small amount of training data, the extra time cost is usually affordable.
In practice, our generator tuning with meta weight learning takes $10$ minutes to train on each task (the standard generator training time without meta-learning is $3.5$ minutes).
We use a fixed set of hyperparamters for all tasks without task-specific hyperparamter tuning: In Algorithm~\ref{alg:meta}, we set the batch size to be $2$, the learning rate for optimizing $\hat{\bs{\theta}}_{p}$ to be $2e-2$, the learning rate for optimizing $\bs{\omega}$ to be $1e-2$, the learning rate for optimizing $\bs{\theta}_{p}$ to be $5e-3$, and training epoch to be $20$.
We also experiment with larger batch sizes (\eg, $16$/$32$) and/or training for more epochs, but they result in worse language modeling quality than the default hyperparamters.

\paragraph{Details of Generating Training Data.}


Following~\cite{Meng2022GeneratingTD}, for sequence-pair tasks (MNLI, QQP, QNLI, RTE and MRPC), we randomly sample the first sequence from the pretraining corpus (\eg, Wikipedia) and use greedy sampling for generating the second sequence.
For single-sequence tasks (SST-2 and CoLA), we use top-$k$ sampling with temperature to generate training data from scratch where $k=10$.
For all tasks, we generate $5,000$ samples per label.
% To study the impact of the amount of generated training samples on the model performance, we plot the MNLI-m accuracy (mean and standard deviation) with different sizes of $\mathcal{D}_{\text{gen}}$ in Fig.~\ref{fig:num_data}. 
% Both the average model performance and stability improve with more generated samples.

For SST-2, we use one of the following tokens to start generation: ``a'', ``one'', ``the'', ``this'', ``that'', ``i'', ``you'', ``it'', ``what''. 
For CoLA, we use a random stop word to start generation.

\begin{wraptable}[19]{r}{0.35\textwidth}
% \begin{table}[h]
\caption{
Hyperparameters for generating training data for different tasks.
$\tau$: Temperature during sampling ($\tau = 0$ means greedy sampling); $\alpha$: Repetition penalty.
}
% \vspace{1em}
\centering
\small 
\begin{tabular}{ll*{2}{c}}
\toprule
\textbf{Task} & \textbf{Label} & $\tau$ & $\alpha$ \\
\midrule
\multirow{2}{*}{\textbf{SST-2}} & positive & \multirow{2}{*}{0.5} & 1.1  \\
& negative &  & 1.1 \\
\midrule
\multirow{2}{*}{\textbf{CoLA}} & grammatical & 0.3 & 1.1 \\
& not grammatical & 10 & 1.1 \\
\midrule
\multirow{3}{*}{\textbf{MNLI}} & entailment & \multirow{3}{*}{0} & 1.1 \\
& neutral &  & 1.5 \\
& contradiction &  & 1.1  \\
\midrule
\multirow{2}{*}{\textbf{QNLI}} & entailment & \multirow{2}{*}{0} & 1.0 \\
& not entailment &  & 1.5 \\
\midrule
\multirow{2}{*}{\textbf{RTE}} & entailment & \multirow{2}{*}{0} & 1.0 \\
& not entailment & & 1.5 \\
\midrule
\multirow{2}{*}{\textbf{MRPC}} & equivalent & \multirow{2}{*}{0} & 1.0 \\
& not equivalent & & 1.5 \\
\midrule
\multirow{2}{*}{\textbf{QQP}} & equivalent & \multirow{2}{*}{0} & 1.0 \\
& not equivalent & & 1.5 \\
\bottomrule
\end{tabular}
\label{tab:gen_hyperpara}
% \end{table}
% \vspace{-1em}
\end{wraptable}

We apply repetition penalty~\citep{Keskar2019CTRLAC} to the logits of tokens that have already appeared in the sequence.
Overall, the token probability distribution is post-processed as follows before conducting sampling:
\begin{align*}
\label{eq:penalty}
p_{\theta}(x_i|\bs{x}_{<i}) &= \frac{\exp(\bs{e}_i^\top \bs{h}_i/\omega)}{\sum_{j=1}^{|V|}\exp(\bs{e}_j^\top \bs{h}_i/\omega)}, \\  \omega &= \begin{cases}
\tau \alpha & x_i \in \bs{x}_{<i} \\
\tau & \text{else}
\end{cases},    
\end{align*}
where $\tau$ is the temperature hyperparameter, and $\alpha$ is the repetition penalty hyperparameter.
For labels that favor token repetitions between the first and the second sequences (\eg, paraphrase or entailment), we set $\alpha$ to be a smaller value (\eg, $1.0$), and vice versa.

The hyperparameter values for training data generation on all tasks can be found in Table~\ref{tab:gen_hyperpara}.


\paragraph{Hyperparameters for Fine-Tuning Classifier PLMs.}
% \begin{table}[t]
\caption{
Hyperparameters used for fine-tuning on different tasks (they are kept same for all tasks).
$lr$: Learning rate; $bs$: Batch size; $N|\mathcal{Y}|$: Total number of selected generated data (\ie, training set size); $B$: Ensemble prediction update interval; $T$: Number of training steps; $\epsilon$: Label smoothing parameter; $\gamma$: Temporal ensembling momentum parameter; $\delta$: Threshold for filtering out noisy data; $\lambda_{\text{max}}$: Maximum weight (after ramp-up) of temporal ensembling regularization.
}
\vspace{1em}
\centering
\small 
% \resizebox{\columnwidth}{!}{
\begin{tabular}{*{9}{c}}
\toprule
$lr$ & $bs$ & $N|\mathcal{Y}|$ & $B$ & $T$ & $\epsilon$ & $\gamma$ & $\delta$ & $\lambda_{\text{max}}$ \\
\midrule
1e-5 & 16 & 6,000 & 100 & 1,125 & 0.15 & 0.8 & 0.8 & 10 \\
\bottomrule
\end{tabular}
% }
\label{tab:finetune_hyperpara}
\end{table}
% Table~\ref{tab:finetune_hyperpara} lists the hyperparameters used in the fine-tuning stage.
For fine-tuning on the few-shot training samples $\mathcal{D}_{\text{train}}$, we search among the following hyperparameter ranges based on development set ($\mathcal{D}_{\text{dev}}$) model performance and pick the best performing model for futher fine-tuning on synthesized data:
Learning rate in $[1e-5, 2e-5]$ and batch size in $[4, 8]$.
The number of training steps is fixed to be $1000$. We also find it beneficial to apply label smoothing (smoothing weight set to $0.15$) for fine-tuning on the few-shot training set.

For fine-tuning on the synthesized training samples $\mathcal{D}_{\text{gen}}$,
we use the following hyperparameters:
$5e-6$ as the learning rate; $16$ as the batch size; label smoothing weight $\epsilon = 0.15$ ; temporal ensemble momentum $\gamma = 0.9$; temporal ensemble loss weight $\lambda = 20$; training steps $T = 6,000$.

\paragraph{Details of Temporal Ensembling for Fine-Tuning Classifier PLMs on Synthetic Data.}

We update ensembled predictions $\bar{\bs{z}}$ as follows where $\bs{p}_{\phi}$ is the current model prediction, $\gamma$ is the momentum parameter, $\hat{\bs{z}}$ is the accumulated model prediction before bias correction, $\bar{\bs{z}}$ is the accumulated model prediction after bias correction, and $t$ is the number of updates $\bar{\bs{z}}$ has received:
\begin{equation*}
\label{eq:udpate_ens}
\hat{\bs{z}} \gets \gamma\hat{\bs{z}} + (1-\gamma)\bs{p}_{\phi}, \, \bar{\bs{z}} \gets \hat{\bs{z}}/(1-\gamma^t).
\end{equation*}
The accumulated model prediction $\hat{\bs{z}}$ has a zero initialization;  the division $(1-\gamma^t)$ is for bias correction~\citep{Laine2017TemporalEF}.
After each update of $\hat{\bs{z}}$, it will be compared to a threshold value $\delta$; each synthesized sample $(\tilde{\bs{x}}, \tilde{y})$ will be included in training only if $\bar{z}_{\tilde{y}} > \delta$.

We update the ensembled predictions $\bar{\bs{z}}$ on all samples in $\mathcal{D}_{\text{gen}}$ every $200$ steps, and set the threshold value for sample filtering $\delta = 0.8$.


\paragraph{Computation Environment.}
The experiments are conducted on NVIDIA A100 GPUs.


\section{Derivation of Meta Weight Gradient Update}
\label{app:gradient}


We first write out the gradient update of $\hat{\bs{\theta}}_{p}^{(t)}\left(\bs{\omega}^{(t)}\right)$ and $\bs{\omega}^{(t+1)}$ according to Algorithm~\ref{alg:meta} as follows:

{\small
\begin{equation}
\label{eq:theta_update}
\hat{\bs{\theta}}_{p}^{(t)}\left(\bs{\omega}^{(t)}\right) 
= \bs{\theta}_{p}^{(t)} - \alpha \left . \frac{\partial\mathcal{L}_{\text{w-gen}} \left(\bs{\theta}_{p};\bs{\omega}^{(t)}\right) }{\partial \bs{\theta}_{p}} \right \vert_{\bs{\theta}_{p} = \bs{\theta}_{p}^{(t)}}
= \bs{\theta}_{p}^{(t)} - \alpha \left . \sum_{j=1}^n w_j \left(\bs{\omega}^{(t)} \right) \frac{\partial \mathcal{L}^j_{\text{gen}} (\bs{\theta}_{p}) }{\partial \bs{\theta}_{p}} \right \vert_{\bs{\theta}_{p} = \bs{\theta}_{p}^{(t)}}
\end{equation}

\begin{equation}
\label{eq:omega_update}
\bs{\omega}^{(t+1)} = 
\bs{\omega}^{(t)} - \beta \left . \frac{\partial \mathcal{L}_{\text{disc}}\left(\hat{\bs{\theta}}_{p}^{(t)}\left(\bs{\omega}\right)\right) }{\partial \bs{\omega}} \right \vert_{\bs{\omega} = \bs{\omega}^{(t)}} .
\end{equation}
}where $\alpha$ and $\beta$ are step sizes.

The gradient in Equation~\eqref{eq:omega_update} is calculated as:
{\small
\begin{align*}
& \quad \left . \frac{\partial \mathcal{L}_{\text{disc}} \left(\hat{\bs{\theta}}^{(t)}_{p}\left(\bs{\omega} \right)\right)}{\partial \bs{\omega}} \right \vert_{\bs{\omega} = \bs{\omega}^{(t)}} \\
&= \left . \frac{\partial \mathcal{L}_{\text{disc}} \left(\hat{\bs{\theta}}_{p}\right)}{\partial \hat{\bs{\theta}}_{p}} \right \vert_{\hat{\bs{\theta}}_{p} = \hat{\bs{\theta}}_{p}^{(t)} } \left . \frac{\partial \hat{\bs{\theta}}_{p}\left(\bs{\omega}\right)}{\partial \bs{\omega}} \right \vert_{\bs{\omega} = \bs{\omega}^{(t)}} \\
&= \left . \frac{\partial \mathcal{L}_{\text{disc}} \left(\hat{\bs{\theta}}_{p}\right)}{\partial \hat{\bs{\theta}}_{p}} \right \vert_{\hat{\bs{\theta}}_{p} = \hat{\bs{\theta}}_{p}^{(t)} } \left( -\alpha \sum_{j=1}^n  \left . \frac{\partial \mathcal{L}^j_{\text{gen}} (\bs{\theta}_{p}) }{\partial \bs{\theta}_{p}} \right \vert_{\bs{\theta}_{p} = \bs{\theta}_{p}^{(t)}} ^\top  \left .\frac{\partial w_j \left(\bs{\omega} \right)}{\partial \bs{\omega}}\right \vert_{\bs{\omega} = \bs{\omega}^{(t)}} \right) \tag*{Plugging in Eq.~\eqref{eq:theta_update}} \\
&= -\alpha \sum_{j=1}^n \left( \underbrace{\left . \frac{\partial \mathcal{L}_{\text{disc}} \left(\hat{\bs{\theta}}_{p}\right)}{\partial \hat{\bs{\theta}}_{p}} \right \vert_{\hat{\bs{\theta}}_{p} = \hat{\bs{\theta}}_{p}^{(t)} } \left . \frac{\partial \mathcal{L}^j_{\text{gen}} (\bs{\theta}_{p}) }{\partial \bs{\theta}_{p}} \right \vert_{\bs{\theta}_{p} = \bs{\theta}_{p}^{(t)}} ^\top}_{\triangleq d_j} \left .\frac{\partial w_j \left(\bs{\omega} \right)}{\partial \bs{\omega}}\right \vert_{\bs{\omega} = \bs{\omega}^{(t)}} \right) \\
\end{align*}
}
Therefore, 
{\small
$$
-\left . \frac{\partial \mathcal{L}_{\text{disc}} \left(\hat{\bs{\theta}}^{(t)}_{p}\left(\bs{\omega} \right)\right)}{\partial \bs{\omega}} \right \vert_{\bs{\omega} = \bs{\omega}^{(t)}} \propto \sum_{j=1}^n d_j \left .\frac{\partial w_j \left(\bs{\omega} \right)}{\partial \bs{\omega}}\right \vert_{\bs{\omega} = \bs{\omega}^{(t)}}, \,\,\,\, d_j = \left . \frac{\partial \mathcal{L}_{\text{disc}} \left(\hat{\bs{\theta}}_{p}\right)}{\partial \hat{\bs{\theta}}_{p}} \right \vert_{\hat{\bs{\theta}}_{p} = \hat{\bs{\theta}}_{p}^{(t)} } \left . \frac{\partial \mathcal{L}^j_{\text{gen}} (\bs{\theta}_{p}) }{\partial \bs{\theta}_{p}} \right \vert_{\bs{\theta}_{p} = \bs{\theta}_{p}^{(t)}} ^\top.
$$
}

\section{GLUE Tasks}
\label{app:glue}
We provide the details of the seven classification tasks included in the GLUE benchmark.

\textbf{MNLI:} Multi-genre Natural Language Inference~\citep{MNLI} requires predicting whether a given premise sentence entails, contradicts or neutral with respect to a given hypothesis sentence. 

\textbf{QQP:} Quora Question Pairs~\citep{QQP} requires judging whether a pair of questions asked are semantically equivalent.

\textbf{QNLI:} Question Natural Language Inference requires predicting whether a given sentence contains the answer to a given question sentence.

\textbf{SST-2:} Stanford Sentiment Treebank~\citep{SST-2} requires determining if a movie review has positive or negative sentiment. 

\textbf{CoLA:} Corpus of Linguistic Acceptability~\citep{COLA} requires determining whether a given sentence is linguistically acceptable or not. 

\textbf{RTE:} Recognizing Textual Entailment~\citep{RTE-5,RTE-1,RTE-3,RTE-2} requires predicting whether a given premise sentence entails a given hypothesis sentence or not.

\textbf{MRPC:} Microsoft Research Paraphrase Corpus~\citep{MRPC} requires predicting whether two sentences are semantically equivalent or not.





\section{Data Augmentation Baseline Details}
\label{app:aug_baselines}

\paragraph{Details About MixText~\citep{Chen2020MixTextLI}.}
We use the TMix version of MixText to perform data interpolation on the few-shot labeled dataset (since there is no access to unlabeled task-specific data under the strict few-shot learning setting~\cite{gao2021making}). 
% Note that under the strict few-shot learning setting~\cite{gao2021making}, there is no access to unlabeled task-specific data, so .
We adapt the label mix-up operation to fit prompt-based fine-tuning by interpolating the label words instead of categorical labels; we observe that this results in better few-shot performance than the original TMix, probably analogous to why prompt-based fine-tuning outperforms standard fine-tuning for few-shot learning.
We train the classifier with supervised loss combined with consistency loss over the interpolated samples as in the original paper.
We follow the default hyperparameters in MixText.


\paragraph{Details About Back Translation.}
We use two trained Marian~\citep{mariannmt} models to perform data augmentation via back translation. 
We translate our labeled examples from English to French, and then back to English. As in UDA~\citep{Xie2020UnsupervisedDA}, we 
employ random sampling with a tunable temperature to generate a diverse set of derivative examples. We generate $32$ 
examples from each few-shot training example and let the synthesized samples share the same label with the original few-shot training sample. 
After combining with the original examples,
we fine-tune the classifier and observe performance.


\paragraph{Details About GPT3Mix~\citep{Yoo2021GPT3MixLL}.}

\begin{table}[tbh]
% \renewcommand\arraystretch{1.2}
\caption{
Prompts used for GPT3Mix augmentation. For sequence-pair tasks, $\bs{x}_1$ and $\bs{x}_2$ denote the first and second input sequence, respectively. For single-sequence tasks, $\bs{x}$ denotes the input sequence. $y$ denotes the label name. Only one example is shown in the template for clarity; in practice, we concatenate $k=4$ samples according to the optimal setting in GPT3Mix~\citep{Yoo2021GPT3MixLL}.
}
% \vspace{1em}
\centering
\small 
\resizebox{\columnwidth}{!}{
\begin{tabular}{llll}
\toprule
\textbf{Task} & \textbf{Template} & \textbf{Label name}\\
\midrule
\multirow{3}{*}{SST-2} & Each item in the following list contains a movie review and the
respective sentiment. & positive: positive \\
& The sentiment is one of `positive' or `negative'. & negative: negative \\
& Movie review: $\bs{x}$ (Sentiment: $y$) $\dots$  \\
\midrule
\multirow{3}{*}{CoLA} & Each item in the following list contains a text and the respective grammar. & grammatical: correct \\
& The grammar is one of `correct' or `incorrect'. & not grammatical: incorrect \\
& Text: $\bs{x}$ (Grammar: $y$) $\dots$  \\
\midrule
\multirow{3}{*}{MNLI} & Each item in the following list contains a premise, a hypothesis and their logical relation. & entailment: entailment \\
& The logical relation is one of `entailment', `neutral' or `contradiction'. & neutral: neutral \\
& Premise: $\bs{x}_1$ Hypothesis: $\bs{x}_2$ (Logical relation: $y$) $\dots$ & contradiction: contradiction \\
\midrule
\multirow{3}{*}{QNLI} & Each item in the following list contains a question, an answer and their logical relation. & entailment: entailment \\
& The logical relation is one of `entailment' or `neutral'. & not entailment: neutral \\
& Question: $\bs{x}_1$ Answer: $\bs{x}_2$ (Logical relation: $y$) $\dots$ \\
\midrule
\multirow{3}{*}{RTE} & Each item in the following list contains a premise, a hypothesis and their logical relation. & entailment: entailment \\
& The logical relation is one of `entailment' or `neutral'. & not entailment: neutral \\
& Premise: $\bs{x}_1$ Hypothesis: $\bs{x}_2$ (Logical relation: $y$) $\dots$  \\
\midrule
\multirow{3}{*}{MRPC} & Each item in the following list contains two sentences and their semantic relation. & equivalent: equivalent \\
& The semantic relation is one of `equivalent' or `different'. & not equivalent: different \\
& Sentence 1: $\bs{x}_1$ Sentence 2: $\bs{x}_2$ (Semantic relation: $y$) $\dots$  \\
\midrule
\multirow{3}{*}{QQP} & Each item in the following list contains two questions and their semantic relation. & equivalent: equivalent \\
& The semantic relation is one of `equivalent' or `different'. & not equivalent: different \\
& Question 1: $\bs{x}_1$ Question 2: $\bs{x}_2$ (Semantic relation: $y$) $\dots$ \\
\bottomrule
\end{tabular}
}
% \vspace{-1em}
\label{tab:gpt3mix_prompt}
\end{table}

We use the $175$B GPT3 model for generating the augmentations. 
For creating each augmentation, we randomly sample $k=4$ (the optimal setting according to GPT3Mix) examples from the few-shot training set as demonstrations.
The prompts follow the suggested format proposed in the original paper~\citep{Yoo2021GPT3MixLL} and are shown in Table~\ref{tab:gpt3mix_prompt}.
We create $5,000$ augmented samples per label to make the resulting training set size equal to that of \method. After obtaining the augmented examples and their pseudo labels (the probability predictions over all labels by GPT3), we use them along with the real few-shot samples for fine-tuning the classifier, following the setting in GPT3Mix~\citep{Yoo2021GPT3MixLL}.

\paragraph{Details About Standard Generator Fine-Tuning.}
We fine-tune the same $1.6$B CTRL~\citep{Keskar2019CTRLAC} model as used in \method with the standard maximum likelihood objective. Different from previous studies~\citep{AnabyTavor2020DoNH,Kumar2020DataAU} that prepend categorical labels to the training samples, we enhance the generator fine-tuning with label-descriptive prompts (shown in Table~\ref{tab:full_prompts}) used in \method.
We create $5,000$ augmented samples per label to make the resulting training set size equal to that of \method.

% % \newcommand{\acc}{Acc. (\uparrow)}
% \newcommand{\ppl}{PPL (\downarrow)}

\begin{wraptable}[11]{wr}{0.5\textwidth}
\small
\centering
\caption{Study of weighting network instantiation. The default architecture is a feedforward network (FFN) with one hidden layer. We also explore adding a self-attention layer on top of the generator PLM's output hidden states (Self-attention). We use the same two metrics with Table~\ref{tab:gen_eval} for evaluation.}
\resizebox{0.5\textwidth}{!}{
\begin{tabular}{l*{4}{c}}
\toprule 
\multirow{2}{*}{\textbf{Architecture}} & \multicolumn{2}{c}{\textbf{MNLI}} & \multicolumn{2}{c}{\textbf{SST-2}} \\
& \acc & \ppl & \acc & \ppl \\
\midrule
FFN & $\textbf{72.3}$ & $\textbf{11.9}$ & $\textbf{93.2}$ & $\textbf{43.5}$ \\
Self-attention & $70.3$ & ${12.9}$ & ${92.3}$ & ${44.2}$ \\
\bottomrule
\end{tabular}
}
\label{tab:weight_net}
\end{wraptable}





% \section{Qualitative Analyses of Generator Training Objectives}\label{app:eval_gen}


% which can serve as quality training data.




\section{Concrete Generation Results}\label{app:gen_result}

We present some concrete generation results (from $\mathcal{D}_{\text{gen}}$) for all tasks in Tables~\ref{tab:case_sst2}, \ref{tab:case_cola}, \ref{tab:case_qqp}, \ref{tab:case_mnli}, \ref{tab:case_qnli}, \ref{tab:case_rte}, and \ref{tab:case_mrpc}.
To compare $\mathcal{D}_{\text{gen}}$ with $\mathcal{D}_{\text{train}}$, we also show the few-shot training samples ($\mathcal{D}_{\text{train}}$) of SST-2 in Table~\ref{tab:sst2_fewshot},

Comparing Tables~\ref{tab:case_sst2} with \ref{tab:sst2_fewshot}, it can be seen that the synthetic samples are accurate and quite different from the given training samples to serve as effective augmentations. 
For sequence-pair tasks, because we randomly sample the first sequence from the pretraining corpus and let the generator create the second sequence given certain labels, the resulting generated samples will be certainly different from the given training samples.

\begin{table}[!h]
\renewcommand\arraystretch{1.2}
\caption{
Examples of concrete generations for SST-2.
}
% \vspace{1em}
\centering
\small 
\resizebox{\textwidth}{!}{
\begin{tabular}{lll}
\toprule
\textbf{Example} & \textbf{Label} & \textbf{Generated Text} \\
\midrule
\multirow{2}{*}{$\#1$} & \multirow{1}{*}{positive} 
& i was a bit skeptical when i first saw the  trailer for this movie, but after watching it, it was quite enjoyable.\\
& \multirow{1}{*}{negative} 
& i was expecting a fun and entertaining movie with some nice special effects, but instead it was just plain boring. \\
\midrule
\multirow{4}{*}{$\#2$} & \multirow{2}{*}{positive} 
& that is, a film that has the potential to move an audience to tears, or at least to a state of mind where \\
& & you feel something.\\
& \multirow{2}{*}{negative} 
& a film that has the potential to be one of the most moving and powerful experiences ever made \\
& & is instead marred by an unfocused script, bad acting and a lack of emotional depth.\\
\midrule
\multirow{2}{*}{$\#3$} & \multirow{1}{*}{positive} 
& this is a good movie -- it's not just for people who like to see things blow up or have explosions or shoot guns.\\
& \multirow{1}{*}{negative} 
& i am a big fan of the first two movies but this one is just plain bad. \\
\bottomrule
\end{tabular}
}
\vspace{-.5em}
\label{tab:case_sst2}
\end{table}






\begin{table}[!h]
\renewcommand\arraystretch{1.2}
\caption{
Examples of concrete generations for CoLA.
}
% \vspace{1em}
\centering
\small 
\resizebox{\textwidth}{!}{
\begin{tabular}{lll}
\toprule
\textbf{Example} & \textbf{Label} & \textbf{Generated Text} \\
\midrule
\multirow{3}{*}{$\#1$} & \multirow{1}{*}{grammatical} 
& Between the two of us, I think you have a good chance to make it to the finals.\\
& \multirow{2}{*}{not grammatical} 
& A few weeks later we saw them on television. We thought there were a hundred more.\\
& & It wasn't that they came out one at time. \\
\midrule
\multirow{2}{*}{$\#2$} & \multirow{1}{*}{grammatical} 
& Doing the dishes is a task that I have to do because I am not allowed to go out and play with my friends.\\
& \multirow{1}{*}{not grammatical} 
& Only those whose lives depend on it, but only when you are at its mercy do it harm.\\
\midrule
\multirow{2}{*}{$\#3$} & \multirow{1}{*}{grammatical} 
& To be able to speak the language of the other person is a prerequisite for communication.\\
& \multirow{1}{*}{not grammatical} 
& The only good way to avoid getting into trouble for being too friendly would to let me go on a trip.\\
\bottomrule
\end{tabular}
}
\vspace{-.5em}
\label{tab:case_cola}
\end{table}


\begin{table}[thb]
\renewcommand\arraystretch{1.2}
\caption{
Examples of concrete generations for QQP. The first question (\textit{italicized}) is randomly sampled from the pretraining corpus; the second question (\underline{underlined}) is generated by $G_{\bs{\theta}_{p}}$.
}
\vspace{-.5em}
\centering
\small 
% \resizebox{\textwidth}{!}{
\begin{tabular}{lll}
\toprule
\textbf{Example} & \textbf{Label} & \textbf{Generated Text} \\
\midrule
\multirow{4}{*}{$\#1$} & \multirow{2}{*}{equivalent} 
& Question 1: \textit{How long does it take for a project to be completed?}\\ 
& & Question 2: \underline{How long does it take to complete a project?}\\
& \multirow{2}{*}{not equivalent} 
& Question 1: \textit{How long does it take for a project to be completed?}\\ 
& & Question 2: \underline{What is the total cost of completing this project?}\\
\midrule
\multirow{7}{*}{$\#2$} & \multirow{4}{*}{equivalent} 
& Question 1: \textit{What mascots or characters would you like to see included as}\\
& & \quad \quad \quad \quad \quad  \textit{Super Mario Maker DLC in the future?}\\ 
& & Question 2: \underline{What would you like to see in Super Mario Maker DLC that}\\
& & \quad \quad \quad \quad \quad  \underline{you did not see in the game?}\\
& \multirow{3}{*}{not equivalent} 
& Question 1: \textit{What mascots or characters would you like to see included as}\\
& & \quad \quad \quad \quad \quad  \textit{Super Mario Maker DLC in the future?}\\ 
& & Question 2: \underline{How do I get a copy of this game?}\\
\bottomrule
\end{tabular}
% }
\vspace{-.5em}
\label{tab:case_qqp}
\end{table}




\begin{table}[thb]
\renewcommand\arraystretch{1.2}
\caption{
Examples of concrete generations for MNLI. The first sentence (\textit{italicized}) is randomly sampled from the pretraining corpus; the second sentence (\underline{underlined}) is generated by $G_{\bs{\theta}_{p}}$.
}
\vspace{-.5em}
\centering
\small 
% \resizebox{\textwidth}{!}{
\begin{tabular}{lll}
\toprule
\textbf{Example} & \textbf{Label} & \textbf{Generated Text} \\
\midrule
\multirow{6}{*}{$\#1$} & \multirow{2}{*}{entailment} 
& Sentence 1: \textit{Air is provided for the combustion by an electric blower.}\\
& & Sentence 2: \underline{The blower provides air to a combustion chamber.}\\
& \multirow{2}{*}{neutral} 
& Sentence 1: \textit{Air is provided for the combustion by an electric blower.}\\
& & Sentence 2: \underline{Electric blowers are available in most gas stations.} \\
& \multirow{2}{*}{contradiction} 
& Sentence 1: \textit{Air is provided for the combustion by an electric blower.}\\
& & Sentence 2: \underline{The blower does not work.} \\
\midrule
\multirow{10}{*}{$\#2$} & \multirow{3}{*}{entailment} 
& Sentence 1: \textit{Since its base is almost at sea level, it is only the 15th highest light}\\
& & \textit{\quad \quad \quad \quad \quad in the United States, the first 14 being built on higher ground.}\\
& & Sentence 2: \underline{It is the 15th highest light in the United States.}\\
& \multirow{4}{*}{neutral} 
& Sentence 1: \textit{Since its base is almost at sea level, it is only the 15th highest light}\\
& & \textit{\quad \quad \quad \quad \quad in the United States, the first 14 being built on higher ground.}\\
& & Sentence 2: \underline{The lighthouse was originally constructed to be a beacon for ships}\\
& & \quad \quad \quad \quad \quad \underline{passing by and as such has been used since before World War II.} \\
& \multirow{3}{*}{contradiction} 
& Sentence 1: \textit{Since its base is almost at sea level, it is only the 15th highest light}\\
& & \textit{\quad \quad \quad \quad \quad in the United States, the first 14 being built on higher ground.}\\
& & Sentence 2: \underline{It is located on a mountain top.} \\
\bottomrule
\end{tabular}
% }
\vspace{-.5em}
\label{tab:case_mnli}
\end{table}



\begin{table}[thb]
\renewcommand\arraystretch{1.2}
\caption{
Examples of concrete generations for QNLI. The question (\textit{italicized}) is randomly sampled from the pretraining corpus; the answer (\underline{underlined}) is generated by $G_{\bs{\theta}_{p}}$.
}
\vspace{-.5em}
\centering
\small 
% \resizebox{\textwidth}{!}{
\begin{tabular}{lll}
\toprule
\textbf{Example} & \textbf{Label} & \textbf{Generated Text} \\
\midrule
\multirow{5}{*}{$\#1$} & \multirow{2}{*}{entailment} 
& Question: \textit{What makes you want to step up to the next level?}\\ 
& & Answer: \underline{I want to be the best player I can be.}\\
& \multirow{3}{*}{not entailment} 
& Question: \textit{What makes you want to step up to the next level?}\\ 
& & Answer: \underline{The new program will be called "Project 10" and it is expected that a total}\\
& & \quad \quad \quad \quad \underline{of \$450 million in federal funding would go toward it.}\\
\midrule
\multirow{6}{*}{$\#2$} & \multirow{3}{*}{entailment} 
& Question: \textit{How do all those shops know what you would like to buy?}\\ 
& & Answer: \underline{The stores are able to track your preferences and provide you with a list}\\
& & \quad \quad \quad \quad \underline{of products that are best for you.}\\
& \multirow{3}{*}{not entailment}
& Question: \textit{How do all those shops know what you would like to buy?}\\ 
& & Answer: \underline{The stores are not required by law or regulation in the United States and Canada}\\
& & \quad \quad \quad \quad \underline{but they have been known for years as a source of illegal sales on eBay.}\\
\bottomrule
\end{tabular}
% }
\vspace{-.5em}
\label{tab:case_qnli}
\end{table}







\begin{table}[thb]
\renewcommand\arraystretch{1.2}
\caption{
Examples of concrete generations for RTE. The first sentence (\textit{italicized}) is randomly sampled from the pretraining corpus; the second sentence (\underline{underlined}) is generated by $G_{\bs{\theta}_{p}}$.
}
% \vspace{1em}
\centering
\small 
\resizebox{\textwidth}{!}{
\begin{tabular}{lll}
\toprule
\textbf{Example} & \textbf{Label} & \textbf{Generated Text} \\
\midrule
\multirow{8}{*}{$\#1$} & \multirow{4}{*}{entailment} 
& Sentence 1: \textit{In 1970, the Science Council of Canada recommended that the government}\\ 
& & \quad \quad \quad \quad \quad \textit{of Canada immediately invest in industrial development of the aviation industry,}\\
& & \quad \quad \quad \quad \quad \textit{including construction of aircraft, navigation aids, and regulation of air traffic.} \\
& & Sentence 2: \underline{The government of Canada has invested in the aviation industry.}\\
& \multirow{4}{*}{not entailment} 
& Sentence 1: \textit{In 1970, the Science Council of Canada recommended that the government}\\ 
& & \quad \quad \quad \quad \quad \textit{of Canada immediately invest in industrial development of the aviation industry,}\\
& & \quad \quad \quad \quad \quad \textit{including construction of aircraft, navigation aids, and regulation of air traffic.} \\
& & Sentence 2: \underline{The Aviation Industry was established by a Royal Decree on June 1, 1970.} \\
\midrule
\multirow{6}{*}{$\#2$} & \multirow{3}{*}{entailment} 
& Sentence 1: \textit{All of the Centre's staff are fluently bilingual in both English and Chinese}\\
& & \quad \quad \quad \quad \quad \textit{and are familiar with the traditions of the Chinese culture.}\\
& & Sentence 2: \underline{The Centre is a bilingual institution.}\\
& \multirow{3}{*}{not entailment}
& Sentence 1: \textit{All of the Centre's staff are fluently bilingual in both English and Chinese}\\
& & \quad \quad \quad \quad \quad \textit{and are familiar with the traditions of the Chinese culture.}\\
& & Sentence 2: \underline{The Centre is a cultural centre for learning about China.} \\
\bottomrule
\end{tabular}
}
\vspace{-.5em}
\label{tab:case_rte}
\end{table}




\begin{table}[thb]
\renewcommand\arraystretch{1.2}
\caption{
Examples of concrete generations for MRPC. The first sentence (\textit{italicized}) is randomly sampled from the pretraining corpus; the second sentence (\underline{underlined}) is generated by $G_{\bs{\theta}_{p}}$.
}
% \vspace{1em}
\centering
\small 
\resizebox{\textwidth}{!}{
\begin{tabular}{lll}
\toprule
\textbf{Example} & \textbf{Label} & \textbf{Generated Text} \\
\midrule
\multirow{12}{*}{$\#1$} & \multirow{6}{*}{entailment} 
& Sentence 1: \textit{It embodies the aesthetic ideals of the Queen Anne style through its design}\\
& & \quad \quad \quad \quad \quad \textit{and detailing and shows the type of house that attracted prosperous Napans}\\
& & \quad \quad \quad \quad \quad \textit{at the end of the Victorian era.}\\ 
& & Sentence 2: \underline{It is a fine example of the Queen Anne style of architecture through its design}\\
& & \quad \quad \quad \quad \quad \underline{and detailing and shows the type of house that attracted prosperous Napans}\\
& & \quad \quad \quad \quad \quad \underline{at the end of the Victorian era.}\\
& \multirow{6}{*}{not entailment} 
& Sentence 1: \textit{It embodies the aesthetic ideals of the Queen Anne style through its design}\\
& & \quad \quad \quad \quad \quad \textit{and detailing and shows the type of house that attracted prosperous Napans}\\
& & \quad \quad \quad \quad \quad \textit{at the end of the Victorian era.}\\ 
& & Sentence 2: \underline{The building is a fine example in this style, with an elegant facade reminiscent}\\
& & \quad \quad \quad \quad \quad \underline{to those found on many grand mansions built by wealthy merchants during}\\
& & \quad \quad \quad \quad \quad \underline{America's Gilded Age.} \\
\midrule
\multirow{9}{*}{$\#2$} & \multirow{5}{*}{entailment} 
& Sentence 1: \textit{Crosbie ran unsuccessfully for the leadership of the Liberal Party of Newfoundland}\\
& & \quad \quad \quad \quad \quad \textit{and Labrador in 1969, losing to Smallwood, and was also a candidate in the}\\ 
& & \quad \quad \quad \quad \quad \textit{Progressive Conservative Party of Canada's 1983 leadership election, placing third.}\\
& & Sentence 2: \underline{Crosbie was a candidate in the Progressive Conservative Party of Canada's 1983}\\
& & \quad \quad \quad \quad \quad \underline{leadership election, placing third.}\\
& \multirow{4}{*}{not entailment} 
& Sentence 1: \textit{Crosbie ran unsuccessfully for the leadership of the Liberal Party of Newfoundland}\\
& & \quad \quad \quad \quad \quad \textit{and Labrador in 1969, losing to Smallwood, and was also a candidate in the}\\ 
& & \quad \quad \quad \quad \quad \textit{Progressive Conservative Party of Canada's 1983 leadership election, placing third.}\\
& & Sentence 2: \underline{He lost his bid as leader after he failed twice at running against John Diefenbaker.}\\
\bottomrule
\end{tabular}
}
\vspace{-.5em}
\label{tab:case_mrpc}
\end{table}







\begin{table}[hb]
\renewcommand\arraystretch{1.2}
% \vspace{-1em}
\caption{
16-shot training samples of SST-2.
}
\vspace{-1em}
\centering
\small 
\resizebox{\textwidth}{!}{
\begin{tabular}{lll}
\toprule
\textbf{Label} & \textbf{Example} & \textbf{Review Text} \\
\midrule
\multirow{25}{*}{positive} & \multirow{2}{*}{$\#1$} 
& (ramsay) visually transforms the dreary expanse of dead-end distaste the characters inhabit into a poem of art ,\\
& & music and metaphor .\\
& \multirow{1}{*}{$\#2$} 
& the film jolts the laughs from the audience -- as if by cattle prod . \\
& \multirow{2}{*}{$\#3$} 
& the film presents visceral and dangerously honest revelations about the men and machines behind the curtains\\
& & of our planet . \\
& \multirow{1}{*}{$\#4$} 
& a film that will enthrall the whole family . \\
& \multirow{2}{*}{$\#5$} 
& serious movie-goers embarking upon this journey will find that the road to perdition leads to a satisfying\\
& & destination . \\
& \multirow{1}{*}{$\#6$} 
& sweet and memorable film . \\
& \multirow{2}{*}{$\#7$} 
& shyamalan takes a potentially trite and overused concept (aliens come to earth) and infuses it into a \\
& & rustic , realistic , and altogether creepy tale of hidden invasion . \\
& \multirow{2}{*}{$\#8$} 
& a crisp psychological drama (and) a fascinating little thriller that would have been perfect for an old \\
& & `` twilight zone '' episode . \\
& \multirow{2}{*}{$\#9$} 
& my big fat greek wedding is not only the best date movie of the year , it 's also a -- dare i say it twice\\
& & -- delightfully charming -- and totally american , i might add -- slice of comedic bliss .\\
& \multirow{2}{*}{$\#10$} 
& a comedy-drama of nearly epic proportions rooted in a sincere performance by the title character undergoing\\
& & midlife crisis . \\
& \multirow{1}{*}{$\#11$} 
& diggs and lathan are among the chief reasons brown sugar is such a sweet and sexy film . \\
& \multirow{1}{*}{$\#12$} 
& you 're not merely watching history , you 're engulfed by it . \\
& \multirow{1}{*}{$\#13$} 
& the concept is a hoot . \\
& \multirow{2}{*}{$\#14$} 
& the filmmakers ' eye for detail and the high standards of performance convey a strong sense of the \\
& & girls ' environment . \\
& \multirow{1}{*}{$\#15$} 
& a haunting tale of murder and mayhem . \\
& \multirow{2}{*}{$\#16$} 
& neil burger here succeeded in ... making the mystery of four decades back the springboard for a more\\
& & immediate mystery in the present . \\
\midrule
\multirow{26}{*}{negative} & \multirow{1}{*}{$\#1$} 
& nothing happens , and it happens to flat characters .\\
& \multirow{1}{*}{$\#2$} 
& as lively an account as seinfeld is deadpan . \\
& \multirow{2}{*}{$\#3$} 
& so we got ten little indians meets friday the 13th by way of clean and sober , filmed on the set of carpenter 's\\
& & the thing and loaded with actors you 're most likely to find on the next inevitable incarnation of the love boat . \\
& \multirow{3}{*}{$\#4$} 
& the plot is nothing but boilerplate cliches from start to finish , and the script assumes that not only would\\
& & subtlety be lost on the target audience , but that it 's also too stupid to realize that they 've already seen this\\
& & exact same movie a hundred times \\
& \multirow{2}{*}{$\#5$} 
& ultimately , sarah 's dedication to finding her husband seems more psychotic than romantic , and nothing in\\
& & the movie makes a convincing case that one woman 's broken heart outweighs all the loss we witness . \\
& \multirow{2}{*}{$\#6$} 
& the big finish is a bit like getting all excited about a chocolate eclair and then biting into it and finding\\
& & the filling missing . \\
& \multirow{3}{*}{$\#7$} 
& this picture is mostly a lump of run-of-the-mill profanity sprinkled with a few remarks so geared toward\\
& & engendering audience sympathy that you might think he was running for office -- or trying to win over a\\
& & probation officer . \\
& \multirow{2}{*}{$\#8$} 
& just because a walk to remember is shrewd enough to activate girlish tear ducts does n't mean it 's good enough\\
& & for our girls . \\
& \multirow{1}{*}{$\#9$} 
& often lingers just as long on the irrelevant as on the engaging , which gradually turns what time is it there ?\\
& \multirow{1}{*}{$\#10$} 
& this movie , a certain scene in particular , brought me uncomfortably close to losing my lunch .\\
& \multirow{1}{*}{$\#11$} 
& but it would be better to wait for the video . \\
& \multirow{2}{*}{$\#12$} 
& a rude black comedy about the catalytic effect a holy fool has upon those around him in the cutthroat world\\
& & of children 's television .\\
& \multirow{1}{*}{$\#13$} 
& just a collection of this and that -- whatever fills time -- with no unified whole .\\
& \multirow{2}{*}{$\#14$} 
& although god is great addresses interesting matters of identity and heritage , it 's hard to shake the feeling\\
& & that it was intended to be a different kind of film . \\
& \multirow{1}{*}{$\#15$} 
& the chocolate factory without charlie . \\
& \multirow{1}{*}{$\#16$} 
& in that setting , their struggle is simply too ludicrous and borderline insulting . \\
\bottomrule
\end{tabular}
}
\vspace{-.5em}
\label{tab:sst2_fewshot}
\end{table}

% \section{Appendix}


% Optionally include extra information (complete proofs, additional experiments and plots) in the appendix.
% This section will often be part of the supplemental material.


\end{document}