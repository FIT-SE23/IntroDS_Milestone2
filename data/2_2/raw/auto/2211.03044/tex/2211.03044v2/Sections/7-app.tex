
\section{Details of Weighting Network Implementation}\label{app:weight_net}
Since the token weights $\bs{w}$ used in Eq.~\eqref{eq:meta_disc} need to characterize the discriminativeness of each token, we use the value of discriminative objective at each token  $\mathcal{L}_{\text{disc}}^j$ as the input to the weighting network, and we use softmax to normalize the weights:
% $$
% w_j(\bs{\omega}) = g_{\bs{\omega}}(\mathcal{L}_{\text{disc}}^j).
% % \quad
% % \mathcal{L}_{\text{disc}}^j = -\log p_{\bs{\theta}_{p}}(y_l|\bs{x}_{\le j}).
% $$
% We further normalize 
$$
w_j(\bs{\omega}) =  \frac{\exp\left(g_{\bs{\omega}}(\mathcal{L}_{\text{disc}}^j)\right)}{\sum_{{j'}=1}^n \exp\left(g_{\bs{\omega}}(\mathcal{L}_{\text{disc}}^{j'})\right)}.
$$

Following \cite{Shu2019MetaWeightNetLA}, we instantiate $g_{\bs{\omega}}$ to be a feedforward network (FFN) with only one $100$-dimension hidden layer by default.
% We explore an alternative instantiation that adds one self-attention layer on top of the generator PLM's output hidden states. The meta weights are finally obtained by projecting the outputs of the self-attention layer using another linear layer.
% We evaluate the resulting generator quality via the same two metrics as in Table~\ref{tab:gen_eval}.
% Table~\ref{tab:weight_net} shows that using more complicated architectures (\eg, adding another self-attention layer) does not result in a better generator compared to using a simple FFN for meta weight learning. 
% This is probably because the generator PLM's output representations are sufficiently contextualized and contain the information necessary for learning the token weights, thus a simple FFN as the weighting network will be enough.
% Using more complicated networks, on the other hand, will introduce more randomly initialized new parameters which may not be learned well using the limited amount of few-shot training data.

\section{Implementation Details}
\label{app:impl_details}
\begin{table}[!tbh]
\caption{
Prompts used for initializing the prefix vectors and control codes (required by CTRL~\citep{Keskar2019CTRLAC}) used in generator training.
The control codes are selected to approximiate the domain of the task.
For single-sequence tasks, $\bs{x}$ denotes the training sample; for sequence-pair tasks, $\bs{x}_1$ and $\bs{x}_2$ denote the first and second sequence in the training sample, respectively. 
% The ``not entailment'' label of MRPC and ``not equivalent'' label of RTE use two prompts split by ``//'' (essentially combining the prompts used for ``neutral'' and ``contradiction'' labels of the MNLI task).
}
% \vspace{1em}
\centering
\small 
\resizebox{\columnwidth}{!}{
\begin{tabular}{lllll}
\toprule
\textbf{Task} & \textbf{Task Type} & \textbf{Control Code} & \textbf{Label} & \textbf{Initialization Prompt} \\
\midrule
\multirow{2}{*}{\textbf{SST-2}} & \multirow{2}{*}{single-sequence} & \multirow{2}{*}{Reviews} & positive & Rating: 5.0 positive movie review: $\bs{x}$ \\
& & & negative & Rating: 1.0 negative movie review: $\bs{x}$ \\
\midrule
\multirow{2}{*}{\textbf{CoLA}} & \multirow{2}{*}{single-sequence} & \multirow{2}{*}{Links} & grammatical & Linguistically correct sentence: $\bs{x}$ \\
& & & not grammatical & Linguistically incorrect sentence: $\bs{x}$ \\
\midrule
\multirow{3}{*}{\textbf{MNLI}} & \multirow{3}{*}{sequence-pair} & \multirow{3}{*}{Wikipedia} & entailment & Sentence 1 implies Sentence 2. Sentence 1: $\bs{x}_1$ Sentence 2: $\bs{x}_2$ \\
& & & neutral & Sentence 2 supplements Sentence 1. Sentence 1: $\bs{x}_1$ Sentence 2: $\bs{x}_2$ \\
& & & contradiction & Sentence 2 contradicts Sentence 1. Sentence 1: $\bs{x}_1$ Sentence 2: $\bs{x}_2$ \\
\midrule
\multirow{2}{*}{\textbf{QNLI}} & \multirow{2}{*}{sequence-pair} & \multirow{2}{*}{Links} & entailment & Paragraph is relevant to Question. Question: $\bs{x}_1$ Paragraph: $\bs{x}_2$ \\
& & & not entailment & Paragraph is irrelevant to Question. Question: $\bs{x}_1$ Paragraph: $\bs{x}_2$ \\
\midrule
\multirow{2}{*}{\textbf{RTE}} & \multirow{2}{*}{sequence-pair} & \multirow{2}{*}{Wikipedia} & entailment & Sentence 1 implies Sentence 2. Sentence 1: $\bs{x}_1$ Sentence 2: $\bs{x}_2$ \\
& & & not entailment & Sentence 2 supplements Sentence 1. Sentence 1: $\bs{x}_1$ Sentence 2: $\bs{x}_2$ \\
\midrule
\multirow{2}{*}{\textbf{MRPC}} & \multirow{2}{*}{sequence-pair} & \multirow{2}{*}{Wikipedia} & equivalent & Sentence 1 is equivalent to Sentence 2. Sentence 1: $\bs{x}_1$ Sentence 2: $\bs{x}_2$ \\
& & & not equivalent & Sentence 1 is different from Sentence 2. Sentence 1: $\bs{x}_1$ Sentence 2: $\bs{x}_2$ \\
\midrule
\multirow{2}{*}{\textbf{QQP}} & \multirow{2}{*}{sequence-pair} & \multirow{2}{*}{Links} & equivalent & Question 1 is equivalent to Question 2. Question 1: $\bs{x}_1$ Question 2: $\bs{x}_2$ \\
& & & not equivalent & Question 1 is different from Question 2. Question 1: $\bs{x}_1$ Question 2: $\bs{x}_2$ \\
\bottomrule
\end{tabular}
}
\label{tab:full_prompts}
\vspace{-1em}
\end{table}


\paragraph{Details of Initialization Prompts Used for Generator Tuning on Different Tasks.}

For generator tuning, we find it beneficial to initialize the prefix vectors with task-descriptive prompts, similar to the observations in \cite{Li2021PrefixTuningOC}.
The prefix lengths (\ie, number of trained prefix token positions) are equal to the number of tokens in the prompts.
We present details about the prompts used for initializing the prefix vectors for different tasks in Table~\ref{tab:full_prompts}.
For sequence-pair tasks, an additional infix prompt is used between the two sequences,
and we also tune the embeddings of the infix (\ie, prompt-tuning~\citep{Lester2021ThePO}) for generator training.



\paragraph{Details of Generator Tuning.}
% 
\begin{wrapfigure}[16]{wr}{0.33\textwidth}
% \vspace{-0.3cm}
% \subfigcapmargin=10pt
\centering
\includegraphics[width=0.33\textwidth]{Figs/num_data_plot.pdf}
\caption{(On MNLI) The model performance with different amount of generated samples. Dots and error bars denote the average performance and the standard deviation over $5$ different train/development set splits.}
\label{fig:num_data}
\vspace{-0.5em}
\end{wrapfigure}

% In Algorithm~\ref{alg:meta}, we use SGD with $2e-2$ as the learning rate for the first gradient update (\ie, optimizing $\hat{\theta}_{p}^{(t)}\left(\omega^{(t)}\right)$); we use SGD with $1e-2$ as the learning rate for the second gradient update (\ie, optimizing $\omega^{(t+1)}$); we use Adam~\citep{Kingma2015AdamAM} with $5e-3$ as the learning rate for the third gradient update (\ie, optimizing $\theta_{p}^{(t+1)}$).
The meta-weighted generator tuning procedure (Algorithm~\ref{alg:meta}) involves three forward and backward passes, and thus its time complexity is approximately $3$ times of standard generator training without meta learning.
However, since the few-shot training sets have a small amount of training data, the extra time cost is usually affordable.
In practice, our generator tuning with meta weight learning takes $10$ minutes to train on each task (the standard generator training time without meta-learning is $3.5$ minutes).
We use a fixed set of hyperparamters for all tasks without task-specific hyperparamter tuning: In Algorithm~\ref{alg:meta}, we set the batch size to be $2$, the learning rate for optimizing $\hat{\bs{\theta}}_{p}$ to be $2e-2$, the learning rate for optimizing $\bs{\omega}$ to be $1e-2$, the learning rate for optimizing $\bs{\theta}_{p}$ to be $5e-3$, and training epoch to be $20$.
We also experiment with larger batch sizes (\eg, $16$/$32$) and/or training for more epochs, but they result in worse language modeling quality than the default hyperparamters.

\paragraph{Details of Generating Training Data.}


Following~\cite{Meng2022GeneratingTD}, for sequence-pair tasks (MNLI, QQP, QNLI, RTE and MRPC), we randomly sample the first sequence from the pretraining corpus (\eg, Wikipedia) and use greedy sampling for generating the second sequence.
For single-sequence tasks (SST-2 and CoLA), we use top-$k$ sampling with temperature to generate training data from scratch where $k=10$.
For all tasks, we generate $5,000$ samples per label.
% To study the impact of the amount of generated training samples on the model performance, we plot the MNLI-m accuracy (mean and standard deviation) with different sizes of $\mathcal{D}_{\text{gen}}$ in Fig.~\ref{fig:num_data}. 
% Both the average model performance and stability improve with more generated samples.

For SST-2, we use one of the following tokens to start generation: ``a'', ``one'', ``the'', ``this'', ``that'', ``i'', ``you'', ``it'', ``what''. 
For CoLA, we use a random stop word to start generation.

\begin{wraptable}[19]{r}{0.35\textwidth}
% \begin{table}[h]
\caption{
Hyperparameters for generating training data for different tasks.
$\tau$: Temperature during sampling ($\tau = 0$ means greedy sampling); $\alpha$: Repetition penalty.
}
% \vspace{1em}
\centering
\small 
\begin{tabular}{ll*{2}{c}}
\toprule
\textbf{Task} & \textbf{Label} & $\tau$ & $\alpha$ \\
\midrule
\multirow{2}{*}{\textbf{SST-2}} & positive & \multirow{2}{*}{0.5} & 1.1  \\
& negative &  & 1.1 \\
\midrule
\multirow{2}{*}{\textbf{CoLA}} & grammatical & 0.3 & 1.1 \\
& not grammatical & 10 & 1.1 \\
\midrule
\multirow{3}{*}{\textbf{MNLI}} & entailment & \multirow{3}{*}{0} & 1.1 \\
& neutral &  & 1.5 \\
& contradiction &  & 1.1  \\
\midrule
\multirow{2}{*}{\textbf{QNLI}} & entailment & \multirow{2}{*}{0} & 1.0 \\
& not entailment &  & 1.5 \\
\midrule
\multirow{2}{*}{\textbf{RTE}} & entailment & \multirow{2}{*}{0} & 1.0 \\
& not entailment & & 1.5 \\
\midrule
\multirow{2}{*}{\textbf{MRPC}} & equivalent & \multirow{2}{*}{0} & 1.0 \\
& not equivalent & & 1.5 \\
\midrule
\multirow{2}{*}{\textbf{QQP}} & equivalent & \multirow{2}{*}{0} & 1.0 \\
& not equivalent & & 1.5 \\
\bottomrule
\end{tabular}
\label{tab:gen_hyperpara}
% \end{table}
% \vspace{-1em}
\end{wraptable}

We apply repetition penalty~\citep{Keskar2019CTRLAC} to the logits of tokens that have already appeared in the sequence.
Overall, the token probability distribution is post-processed as follows before conducting sampling:
\begin{align*}
\label{eq:penalty}
p_{\theta}(x_i|\bs{x}_{<i}) &= \frac{\exp(\bs{e}_i^\top \bs{h}_i/\omega)}{\sum_{j=1}^{|V|}\exp(\bs{e}_j^\top \bs{h}_i/\omega)}, \\  \omega &= \begin{cases}
\tau \alpha & x_i \in \bs{x}_{<i} \\
\tau & \text{else}
\end{cases},    
\end{align*}
where $\tau$ is the temperature hyperparameter, and $\alpha$ is the repetition penalty hyperparameter.
For labels that favor token repetitions between the first and the second sequences (\eg, paraphrase or entailment), we set $\alpha$ to be a smaller value (\eg, $1.0$), and vice versa.

The hyperparameter values for training data generation on all tasks can be found in Table~\ref{tab:gen_hyperpara}.


\paragraph{Hyperparameters for Fine-Tuning Classifier PLMs.}
% \begin{table}[t]
\caption{
Hyperparameters used for fine-tuning on different tasks (they are kept same for all tasks).
$lr$: Learning rate; $bs$: Batch size; $N|\mathcal{Y}|$: Total number of selected generated data (\ie, training set size); $B$: Ensemble prediction update interval; $T$: Number of training steps; $\epsilon$: Label smoothing parameter; $\gamma$: Temporal ensembling momentum parameter; $\delta$: Threshold for filtering out noisy data; $\lambda_{\text{max}}$: Maximum weight (after ramp-up) of temporal ensembling regularization.
}
\vspace{1em}
\centering
\small 
% \resizebox{\columnwidth}{!}{
\begin{tabular}{*{9}{c}}
\toprule
$lr$ & $bs$ & $N|\mathcal{Y}|$ & $B$ & $T$ & $\epsilon$ & $\gamma$ & $\delta$ & $\lambda_{\text{max}}$ \\
\midrule
1e-5 & 16 & 6,000 & 100 & 1,125 & 0.15 & 0.8 & 0.8 & 10 \\
\bottomrule
\end{tabular}
% }
\label{tab:finetune_hyperpara}
\end{table}
% Table~\ref{tab:finetune_hyperpara} lists the hyperparameters used in the fine-tuning stage.
For fine-tuning on the few-shot training samples $\mathcal{D}_{\text{train}}$, we search among the following hyperparameter ranges based on development set ($\mathcal{D}_{\text{dev}}$) model performance and pick the best performing model for futher fine-tuning on synthesized data:
Learning rate in $[1e-5, 2e-5]$ and batch size in $[4, 8]$.
The number of training steps is fixed to be $1000$. We also find it beneficial to apply label smoothing (smoothing weight set to $0.15$) for fine-tuning on the few-shot training set.

For fine-tuning on the synthesized training samples $\mathcal{D}_{\text{gen}}$,
we use the following hyperparameters:
$5e-6$ as the learning rate; $16$ as the batch size; label smoothing weight $\epsilon = 0.15$ ; temporal ensemble momentum $\gamma = 0.9$; temporal ensemble loss weight $\lambda = 20$; training steps $T = 6,000$.

\paragraph{Details of Temporal Ensembling for Fine-Tuning Classifier PLMs on Synthetic Data.}

We update ensembled predictions $\bar{\bs{z}}$ as follows where $\bs{p}_{\phi}$ is the current model prediction, $\gamma$ is the momentum parameter, $\hat{\bs{z}}$ is the accumulated model prediction before bias correction, $\bar{\bs{z}}$ is the accumulated model prediction after bias correction, and $t$ is the number of updates $\bar{\bs{z}}$ has received:
\begin{equation*}
\label{eq:udpate_ens}
\hat{\bs{z}} \gets \gamma\hat{\bs{z}} + (1-\gamma)\bs{p}_{\phi}, \, \bar{\bs{z}} \gets \hat{\bs{z}}/(1-\gamma^t).
\end{equation*}
The accumulated model prediction $\hat{\bs{z}}$ has a zero initialization;  the division $(1-\gamma^t)$ is for bias correction~\citep{Laine2017TemporalEF}.
After each update of $\hat{\bs{z}}$, it will be compared to a threshold value $\delta$; each synthesized sample $(\tilde{\bs{x}}, \tilde{y})$ will be included in training only if $\bar{z}_{\tilde{y}} > \delta$.

We update the ensembled predictions $\bar{\bs{z}}$ on all samples in $\mathcal{D}_{\text{gen}}$ every $200$ steps, and set the threshold value for sample filtering $\delta = 0.8$.


\paragraph{Computation Environment.}
The experiments are conducted on NVIDIA A100 GPUs.


\section{Derivation of Meta Weight Gradient Update}
\label{app:gradient}


We first write out the gradient update of $\hat{\bs{\theta}}_{p}^{(t)}\left(\bs{\omega}^{(t)}\right)$ and $\bs{\omega}^{(t+1)}$ according to Algorithm~\ref{alg:meta} as follows:

{\small
\begin{equation}
\label{eq:theta_update}
\hat{\bs{\theta}}_{p}^{(t)}\left(\bs{\omega}^{(t)}\right) 
= \bs{\theta}_{p}^{(t)} - \alpha \left . \frac{\partial\mathcal{L}_{\text{w-gen}} \left(\bs{\theta}_{p};\bs{\omega}^{(t)}\right) }{\partial \bs{\theta}_{p}} \right \vert_{\bs{\theta}_{p} = \bs{\theta}_{p}^{(t)}}
= \bs{\theta}_{p}^{(t)} - \alpha \left . \sum_{j=1}^n w_j \left(\bs{\omega}^{(t)} \right) \frac{\partial \mathcal{L}^j_{\text{gen}} (\bs{\theta}_{p}) }{\partial \bs{\theta}_{p}} \right \vert_{\bs{\theta}_{p} = \bs{\theta}_{p}^{(t)}}
\end{equation}

\begin{equation}
\label{eq:omega_update}
\bs{\omega}^{(t+1)} = 
\bs{\omega}^{(t)} - \beta \left . \frac{\partial \mathcal{L}_{\text{disc}}\left(\hat{\bs{\theta}}_{p}^{(t)}\left(\bs{\omega}\right)\right) }{\partial \bs{\omega}} \right \vert_{\bs{\omega} = \bs{\omega}^{(t)}} .
\end{equation}
}where $\alpha$ and $\beta$ are step sizes.

The gradient in Equation~\eqref{eq:omega_update} is calculated as:
{\small
\begin{align*}
& \quad \left . \frac{\partial \mathcal{L}_{\text{disc}} \left(\hat{\bs{\theta}}^{(t)}_{p}\left(\bs{\omega} \right)\right)}{\partial \bs{\omega}} \right \vert_{\bs{\omega} = \bs{\omega}^{(t)}} \\
&= \left . \frac{\partial \mathcal{L}_{\text{disc}} \left(\hat{\bs{\theta}}_{p}\right)}{\partial \hat{\bs{\theta}}_{p}} \right \vert_{\hat{\bs{\theta}}_{p} = \hat{\bs{\theta}}_{p}^{(t)} } \left . \frac{\partial \hat{\bs{\theta}}_{p}\left(\bs{\omega}\right)}{\partial \bs{\omega}} \right \vert_{\bs{\omega} = \bs{\omega}^{(t)}} \\
&= \left . \frac{\partial \mathcal{L}_{\text{disc}} \left(\hat{\bs{\theta}}_{p}\right)}{\partial \hat{\bs{\theta}}_{p}} \right \vert_{\hat{\bs{\theta}}_{p} = \hat{\bs{\theta}}_{p}^{(t)} } \left( -\alpha \sum_{j=1}^n  \left . \frac{\partial \mathcal{L}^j_{\text{gen}} (\bs{\theta}_{p}) }{\partial \bs{\theta}_{p}} \right \vert_{\bs{\theta}_{p} = \bs{\theta}_{p}^{(t)}} ^\top  \left .\frac{\partial w_j \left(\bs{\omega} \right)}{\partial \bs{\omega}}\right \vert_{\bs{\omega} = \bs{\omega}^{(t)}} \right) \tag*{Plugging in Eq.~\eqref{eq:theta_update}} \\
&= -\alpha \sum_{j=1}^n \left( \underbrace{\left . \frac{\partial \mathcal{L}_{\text{disc}} \left(\hat{\bs{\theta}}_{p}\right)}{\partial \hat{\bs{\theta}}_{p}} \right \vert_{\hat{\bs{\theta}}_{p} = \hat{\bs{\theta}}_{p}^{(t)} } \left . \frac{\partial \mathcal{L}^j_{\text{gen}} (\bs{\theta}_{p}) }{\partial \bs{\theta}_{p}} \right \vert_{\bs{\theta}_{p} = \bs{\theta}_{p}^{(t)}} ^\top}_{\triangleq d_j} \left .\frac{\partial w_j \left(\bs{\omega} \right)}{\partial \bs{\omega}}\right \vert_{\bs{\omega} = \bs{\omega}^{(t)}} \right) \\
\end{align*}
}
Therefore, 
{\small
$$
-\left . \frac{\partial \mathcal{L}_{\text{disc}} \left(\hat{\bs{\theta}}^{(t)}_{p}\left(\bs{\omega} \right)\right)}{\partial \bs{\omega}} \right \vert_{\bs{\omega} = \bs{\omega}^{(t)}} \propto \sum_{j=1}^n d_j \left .\frac{\partial w_j \left(\bs{\omega} \right)}{\partial \bs{\omega}}\right \vert_{\bs{\omega} = \bs{\omega}^{(t)}}, \,\,\,\, d_j = \left . \frac{\partial \mathcal{L}_{\text{disc}} \left(\hat{\bs{\theta}}_{p}\right)}{\partial \hat{\bs{\theta}}_{p}} \right \vert_{\hat{\bs{\theta}}_{p} = \hat{\bs{\theta}}_{p}^{(t)} } \left . \frac{\partial \mathcal{L}^j_{\text{gen}} (\bs{\theta}_{p}) }{\partial \bs{\theta}_{p}} \right \vert_{\bs{\theta}_{p} = \bs{\theta}_{p}^{(t)}} ^\top.
$$
}

\section{GLUE Tasks}
\label{app:glue}
We provide the details of the seven classification tasks included in the GLUE benchmark.

\textbf{MNLI:} Multi-genre Natural Language Inference~\citep{MNLI} requires predicting whether a given premise sentence entails, contradicts or neutral with respect to a given hypothesis sentence. 

\textbf{QQP:} Quora Question Pairs~\citep{QQP} requires judging whether a pair of questions asked are semantically equivalent.

\textbf{QNLI:} Question Natural Language Inference requires predicting whether a given sentence contains the answer to a given question sentence.

\textbf{SST-2:} Stanford Sentiment Treebank~\citep{SST-2} requires determining if a movie review has positive or negative sentiment. 

\textbf{CoLA:} Corpus of Linguistic Acceptability~\citep{COLA} requires determining whether a given sentence is linguistically acceptable or not. 

\textbf{RTE:} Recognizing Textual Entailment~\citep{RTE-5,RTE-1,RTE-3,RTE-2} requires predicting whether a given premise sentence entails a given hypothesis sentence or not.

\textbf{MRPC:} Microsoft Research Paraphrase Corpus~\citep{MRPC} requires predicting whether two sentences are semantically equivalent or not.





\section{Data Augmentation Baseline Details}
\label{app:aug_baselines}

\paragraph{Details About MixText~\citep{Chen2020MixTextLI}.}
We use the TMix version of MixText to perform data interpolation on the few-shot labeled dataset (since there is no access to unlabeled task-specific data under the strict few-shot learning setting~\cite{gao2021making}). 
% Note that under the strict few-shot learning setting~\cite{gao2021making}, there is no access to unlabeled task-specific data, so .
We adapt the label mix-up operation to fit prompt-based fine-tuning by interpolating the label words instead of categorical labels; we observe that this results in better few-shot performance than the original TMix, probably analogous to why prompt-based fine-tuning outperforms standard fine-tuning for few-shot learning.
We train the classifier with supervised loss combined with consistency loss over the interpolated samples as in the original paper.
We follow the default hyperparameters in MixText.


\paragraph{Details About Back Translation.}
We use two trained Marian~\citep{mariannmt} models to perform data augmentation via back translation. 
We translate our labeled examples from English to French, and then back to English. As in UDA~\citep{Xie2020UnsupervisedDA}, we 
employ random sampling with a tunable temperature to generate a diverse set of derivative examples. We generate $32$ 
examples from each few-shot training example and let the synthesized samples share the same label with the original few-shot training sample. 
After combining with the original examples,
we fine-tune the classifier and observe performance.


\paragraph{Details About GPT3Mix~\citep{Yoo2021GPT3MixLL}.}

\begin{table}[tbh]
% \renewcommand\arraystretch{1.2}
\caption{
Prompts used for GPT3Mix augmentation. For sequence-pair tasks, $\bs{x}_1$ and $\bs{x}_2$ denote the first and second input sequence, respectively. For single-sequence tasks, $\bs{x}$ denotes the input sequence. $y$ denotes the label name. Only one example is shown in the template for clarity; in practice, we concatenate $k=4$ samples according to the optimal setting in GPT3Mix~\citep{Yoo2021GPT3MixLL}.
}
% \vspace{1em}
\centering
\small 
\resizebox{\columnwidth}{!}{
\begin{tabular}{llll}
\toprule
\textbf{Task} & \textbf{Template} & \textbf{Label name}\\
\midrule
\multirow{3}{*}{SST-2} & Each item in the following list contains a movie review and the
respective sentiment. & positive: positive \\
& The sentiment is one of `positive' or `negative'. & negative: negative \\
& Movie review: $\bs{x}$ (Sentiment: $y$) $\dots$  \\
\midrule
\multirow{3}{*}{CoLA} & Each item in the following list contains a text and the respective grammar. & grammatical: correct \\
& The grammar is one of `correct' or `incorrect'. & not grammatical: incorrect \\
& Text: $\bs{x}$ (Grammar: $y$) $\dots$  \\
\midrule
\multirow{3}{*}{MNLI} & Each item in the following list contains a premise, a hypothesis and their logical relation. & entailment: entailment \\
& The logical relation is one of `entailment', `neutral' or `contradiction'. & neutral: neutral \\
& Premise: $\bs{x}_1$ Hypothesis: $\bs{x}_2$ (Logical relation: $y$) $\dots$ & contradiction: contradiction \\
\midrule
\multirow{3}{*}{QNLI} & Each item in the following list contains a question, an answer and their logical relation. & entailment: entailment \\
& The logical relation is one of `entailment' or `neutral'. & not entailment: neutral \\
& Question: $\bs{x}_1$ Answer: $\bs{x}_2$ (Logical relation: $y$) $\dots$ \\
\midrule
\multirow{3}{*}{RTE} & Each item in the following list contains a premise, a hypothesis and their logical relation. & entailment: entailment \\
& The logical relation is one of `entailment' or `neutral'. & not entailment: neutral \\
& Premise: $\bs{x}_1$ Hypothesis: $\bs{x}_2$ (Logical relation: $y$) $\dots$  \\
\midrule
\multirow{3}{*}{MRPC} & Each item in the following list contains two sentences and their semantic relation. & equivalent: equivalent \\
& The semantic relation is one of `equivalent' or `different'. & not equivalent: different \\
& Sentence 1: $\bs{x}_1$ Sentence 2: $\bs{x}_2$ (Semantic relation: $y$) $\dots$  \\
\midrule
\multirow{3}{*}{QQP} & Each item in the following list contains two questions and their semantic relation. & equivalent: equivalent \\
& The semantic relation is one of `equivalent' or `different'. & not equivalent: different \\
& Question 1: $\bs{x}_1$ Question 2: $\bs{x}_2$ (Semantic relation: $y$) $\dots$ \\
\bottomrule
\end{tabular}
}
% \vspace{-1em}
\label{tab:gpt3mix_prompt}
\end{table}

We use the $175$B GPT3 model for generating the augmentations. 
For creating each augmentation, we randomly sample $k=4$ (the optimal setting according to GPT3Mix) examples from the few-shot training set as demonstrations.
The prompts follow the suggested format proposed in the original paper~\citep{Yoo2021GPT3MixLL} and are shown in Table~\ref{tab:gpt3mix_prompt}.
We create $5,000$ augmented samples per label to make the resulting training set size equal to that of \method. After obtaining the augmented examples and their pseudo labels (the probability predictions over all labels by GPT3), we use them along with the real few-shot samples for fine-tuning the classifier, following the setting in GPT3Mix~\citep{Yoo2021GPT3MixLL}.

\paragraph{Details About Standard Generator Fine-Tuning.}
We fine-tune the same $1.6$B CTRL~\citep{Keskar2019CTRLAC} model as used in \method with the standard maximum likelihood objective. Different from previous studies~\citep{AnabyTavor2020DoNH,Kumar2020DataAU} that prepend categorical labels to the training samples, we enhance the generator fine-tuning with label-descriptive prompts (shown in Table~\ref{tab:full_prompts}) used in \method.
We create $5,000$ augmented samples per label to make the resulting training set size equal to that of \method.

% % \newcommand{\acc}{Acc. (\uparrow)}
% \newcommand{\ppl}{PPL (\downarrow)}

\begin{wraptable}[11]{wr}{0.5\textwidth}
\small
\centering
\caption{Study of weighting network instantiation. The default architecture is a feedforward network (FFN) with one hidden layer. We also explore adding a self-attention layer on top of the generator PLM's output hidden states (Self-attention). We use the same two metrics with Table~\ref{tab:gen_eval} for evaluation.}
\resizebox{0.5\textwidth}{!}{
\begin{tabular}{l*{4}{c}}
\toprule 
\multirow{2}{*}{\textbf{Architecture}} & \multicolumn{2}{c}{\textbf{MNLI}} & \multicolumn{2}{c}{\textbf{SST-2}} \\
& \acc & \ppl & \acc & \ppl \\
\midrule
FFN & $\textbf{72.3}$ & $\textbf{11.9}$ & $\textbf{93.2}$ & $\textbf{43.5}$ \\
Self-attention & $70.3$ & ${12.9}$ & ${92.3}$ & ${44.2}$ \\
\bottomrule
\end{tabular}
}
\label{tab:weight_net}
\end{wraptable}





% \section{Qualitative Analyses of Generator Training Objectives}\label{app:eval_gen}


% which can serve as quality training data.




\section{Concrete Generation Results}\label{app:gen_result}

We present some concrete generation results (from $\mathcal{D}_{\text{gen}}$) for all tasks in Tables~\ref{tab:case_sst2}, \ref{tab:case_cola}, \ref{tab:case_qqp}, \ref{tab:case_mnli}, \ref{tab:case_qnli}, \ref{tab:case_rte}, and \ref{tab:case_mrpc}.
To compare $\mathcal{D}_{\text{gen}}$ with $\mathcal{D}_{\text{train}}$, we also show the few-shot training samples ($\mathcal{D}_{\text{train}}$) of SST-2 in Table~\ref{tab:sst2_fewshot},

Comparing Tables~\ref{tab:case_sst2} with \ref{tab:sst2_fewshot}, it can be seen that the synthetic samples are accurate and quite different from the given training samples to serve as effective augmentations. 
For sequence-pair tasks, because we randomly sample the first sequence from the pretraining corpus and let the generator create the second sequence given certain labels, the resulting generated samples will be certainly different from the given training samples.

\begin{table}[!h]
\renewcommand\arraystretch{1.2}
\caption{
Examples of concrete generations for SST-2.
}
% \vspace{1em}
\centering
\small 
\resizebox{\textwidth}{!}{
\begin{tabular}{lll}
\toprule
\textbf{Example} & \textbf{Label} & \textbf{Generated Text} \\
\midrule
\multirow{2}{*}{$\#1$} & \multirow{1}{*}{positive} 
& i was a bit skeptical when i first saw the  trailer for this movie, but after watching it, it was quite enjoyable.\\
& \multirow{1}{*}{negative} 
& i was expecting a fun and entertaining movie with some nice special effects, but instead it was just plain boring. \\
\midrule
\multirow{4}{*}{$\#2$} & \multirow{2}{*}{positive} 
& that is, a film that has the potential to move an audience to tears, or at least to a state of mind where \\
& & you feel something.\\
& \multirow{2}{*}{negative} 
& a film that has the potential to be one of the most moving and powerful experiences ever made \\
& & is instead marred by an unfocused script, bad acting and a lack of emotional depth.\\
\midrule
\multirow{2}{*}{$\#3$} & \multirow{1}{*}{positive} 
& this is a good movie -- it's not just for people who like to see things blow up or have explosions or shoot guns.\\
& \multirow{1}{*}{negative} 
& i am a big fan of the first two movies but this one is just plain bad. \\
\bottomrule
\end{tabular}
}
\vspace{-.5em}
\label{tab:case_sst2}
\end{table}






\begin{table}[!h]
\renewcommand\arraystretch{1.2}
\caption{
Examples of concrete generations for CoLA.
}
% \vspace{1em}
\centering
\small 
\resizebox{\textwidth}{!}{
\begin{tabular}{lll}
\toprule
\textbf{Example} & \textbf{Label} & \textbf{Generated Text} \\
\midrule
\multirow{3}{*}{$\#1$} & \multirow{1}{*}{grammatical} 
& Between the two of us, I think you have a good chance to make it to the finals.\\
& \multirow{2}{*}{not grammatical} 
& A few weeks later we saw them on television. We thought there were a hundred more.\\
& & It wasn't that they came out one at time. \\
\midrule
\multirow{2}{*}{$\#2$} & \multirow{1}{*}{grammatical} 
& Doing the dishes is a task that I have to do because I am not allowed to go out and play with my friends.\\
& \multirow{1}{*}{not grammatical} 
& Only those whose lives depend on it, but only when you are at its mercy do it harm.\\
\midrule
\multirow{2}{*}{$\#3$} & \multirow{1}{*}{grammatical} 
& To be able to speak the language of the other person is a prerequisite for communication.\\
& \multirow{1}{*}{not grammatical} 
& The only good way to avoid getting into trouble for being too friendly would to let me go on a trip.\\
\bottomrule
\end{tabular}
}
\vspace{-.5em}
\label{tab:case_cola}
\end{table}


\begin{table}[thb]
\renewcommand\arraystretch{1.2}
\caption{
Examples of concrete generations for QQP. The first question (\textit{italicized}) is randomly sampled from the pretraining corpus; the second question (\underline{underlined}) is generated by $G_{\bs{\theta}_{p}}$.
}
\vspace{-.5em}
\centering
\small 
% \resizebox{\textwidth}{!}{
\begin{tabular}{lll}
\toprule
\textbf{Example} & \textbf{Label} & \textbf{Generated Text} \\
\midrule
\multirow{4}{*}{$\#1$} & \multirow{2}{*}{equivalent} 
& Question 1: \textit{How long does it take for a project to be completed?}\\ 
& & Question 2: \underline{How long does it take to complete a project?}\\
& \multirow{2}{*}{not equivalent} 
& Question 1: \textit{How long does it take for a project to be completed?}\\ 
& & Question 2: \underline{What is the total cost of completing this project?}\\
\midrule
\multirow{7}{*}{$\#2$} & \multirow{4}{*}{equivalent} 
& Question 1: \textit{What mascots or characters would you like to see included as}\\
& & \quad \quad \quad \quad \quad  \textit{Super Mario Maker DLC in the future?}\\ 
& & Question 2: \underline{What would you like to see in Super Mario Maker DLC that}\\
& & \quad \quad \quad \quad \quad  \underline{you did not see in the game?}\\
& \multirow{3}{*}{not equivalent} 
& Question 1: \textit{What mascots or characters would you like to see included as}\\
& & \quad \quad \quad \quad \quad  \textit{Super Mario Maker DLC in the future?}\\ 
& & Question 2: \underline{How do I get a copy of this game?}\\
\bottomrule
\end{tabular}
% }
\vspace{-.5em}
\label{tab:case_qqp}
\end{table}




\begin{table}[thb]
\renewcommand\arraystretch{1.2}
\caption{
Examples of concrete generations for MNLI. The first sentence (\textit{italicized}) is randomly sampled from the pretraining corpus; the second sentence (\underline{underlined}) is generated by $G_{\bs{\theta}_{p}}$.
}
\vspace{-.5em}
\centering
\small 
% \resizebox{\textwidth}{!}{
\begin{tabular}{lll}
\toprule
\textbf{Example} & \textbf{Label} & \textbf{Generated Text} \\
\midrule
\multirow{6}{*}{$\#1$} & \multirow{2}{*}{entailment} 
& Sentence 1: \textit{Air is provided for the combustion by an electric blower.}\\
& & Sentence 2: \underline{The blower provides air to a combustion chamber.}\\
& \multirow{2}{*}{neutral} 
& Sentence 1: \textit{Air is provided for the combustion by an electric blower.}\\
& & Sentence 2: \underline{Electric blowers are available in most gas stations.} \\
& \multirow{2}{*}{contradiction} 
& Sentence 1: \textit{Air is provided for the combustion by an electric blower.}\\
& & Sentence 2: \underline{The blower does not work.} \\
\midrule
\multirow{10}{*}{$\#2$} & \multirow{3}{*}{entailment} 
& Sentence 1: \textit{Since its base is almost at sea level, it is only the 15th highest light}\\
& & \textit{\quad \quad \quad \quad \quad in the United States, the first 14 being built on higher ground.}\\
& & Sentence 2: \underline{It is the 15th highest light in the United States.}\\
& \multirow{4}{*}{neutral} 
& Sentence 1: \textit{Since its base is almost at sea level, it is only the 15th highest light}\\
& & \textit{\quad \quad \quad \quad \quad in the United States, the first 14 being built on higher ground.}\\
& & Sentence 2: \underline{The lighthouse was originally constructed to be a beacon for ships}\\
& & \quad \quad \quad \quad \quad \underline{passing by and as such has been used since before World War II.} \\
& \multirow{3}{*}{contradiction} 
& Sentence 1: \textit{Since its base is almost at sea level, it is only the 15th highest light}\\
& & \textit{\quad \quad \quad \quad \quad in the United States, the first 14 being built on higher ground.}\\
& & Sentence 2: \underline{It is located on a mountain top.} \\
\bottomrule
\end{tabular}
% }
\vspace{-.5em}
\label{tab:case_mnli}
\end{table}



\begin{table}[thb]
\renewcommand\arraystretch{1.2}
\caption{
Examples of concrete generations for QNLI. The question (\textit{italicized}) is randomly sampled from the pretraining corpus; the answer (\underline{underlined}) is generated by $G_{\bs{\theta}_{p}}$.
}
\vspace{-.5em}
\centering
\small 
% \resizebox{\textwidth}{!}{
\begin{tabular}{lll}
\toprule
\textbf{Example} & \textbf{Label} & \textbf{Generated Text} \\
\midrule
\multirow{5}{*}{$\#1$} & \multirow{2}{*}{entailment} 
& Question: \textit{What makes you want to step up to the next level?}\\ 
& & Answer: \underline{I want to be the best player I can be.}\\
& \multirow{3}{*}{not entailment} 
& Question: \textit{What makes you want to step up to the next level?}\\ 
& & Answer: \underline{The new program will be called "Project 10" and it is expected that a total}\\
& & \quad \quad \quad \quad \underline{of \$450 million in federal funding would go toward it.}\\
\midrule
\multirow{6}{*}{$\#2$} & \multirow{3}{*}{entailment} 
& Question: \textit{How do all those shops know what you would like to buy?}\\ 
& & Answer: \underline{The stores are able to track your preferences and provide you with a list}\\
& & \quad \quad \quad \quad \underline{of products that are best for you.}\\
& \multirow{3}{*}{not entailment}
& Question: \textit{How do all those shops know what you would like to buy?}\\ 
& & Answer: \underline{The stores are not required by law or regulation in the United States and Canada}\\
& & \quad \quad \quad \quad \underline{but they have been known for years as a source of illegal sales on eBay.}\\
\bottomrule
\end{tabular}
% }
\vspace{-.5em}
\label{tab:case_qnli}
\end{table}







\begin{table}[thb]
\renewcommand\arraystretch{1.2}
\caption{
Examples of concrete generations for RTE. The first sentence (\textit{italicized}) is randomly sampled from the pretraining corpus; the second sentence (\underline{underlined}) is generated by $G_{\bs{\theta}_{p}}$.
}
% \vspace{1em}
\centering
\small 
\resizebox{\textwidth}{!}{
\begin{tabular}{lll}
\toprule
\textbf{Example} & \textbf{Label} & \textbf{Generated Text} \\
\midrule
\multirow{8}{*}{$\#1$} & \multirow{4}{*}{entailment} 
& Sentence 1: \textit{In 1970, the Science Council of Canada recommended that the government}\\ 
& & \quad \quad \quad \quad \quad \textit{of Canada immediately invest in industrial development of the aviation industry,}\\
& & \quad \quad \quad \quad \quad \textit{including construction of aircraft, navigation aids, and regulation of air traffic.} \\
& & Sentence 2: \underline{The government of Canada has invested in the aviation industry.}\\
& \multirow{4}{*}{not entailment} 
& Sentence 1: \textit{In 1970, the Science Council of Canada recommended that the government}\\ 
& & \quad \quad \quad \quad \quad \textit{of Canada immediately invest in industrial development of the aviation industry,}\\
& & \quad \quad \quad \quad \quad \textit{including construction of aircraft, navigation aids, and regulation of air traffic.} \\
& & Sentence 2: \underline{The Aviation Industry was established by a Royal Decree on June 1, 1970.} \\
\midrule
\multirow{6}{*}{$\#2$} & \multirow{3}{*}{entailment} 
& Sentence 1: \textit{All of the Centre's staff are fluently bilingual in both English and Chinese}\\
& & \quad \quad \quad \quad \quad \textit{and are familiar with the traditions of the Chinese culture.}\\
& & Sentence 2: \underline{The Centre is a bilingual institution.}\\
& \multirow{3}{*}{not entailment}
& Sentence 1: \textit{All of the Centre's staff are fluently bilingual in both English and Chinese}\\
& & \quad \quad \quad \quad \quad \textit{and are familiar with the traditions of the Chinese culture.}\\
& & Sentence 2: \underline{The Centre is a cultural centre for learning about China.} \\
\bottomrule
\end{tabular}
}
\vspace{-.5em}
\label{tab:case_rte}
\end{table}




\begin{table}[thb]
\renewcommand\arraystretch{1.2}
\caption{
Examples of concrete generations for MRPC. The first sentence (\textit{italicized}) is randomly sampled from the pretraining corpus; the second sentence (\underline{underlined}) is generated by $G_{\bs{\theta}_{p}}$.
}
% \vspace{1em}
\centering
\small 
\resizebox{\textwidth}{!}{
\begin{tabular}{lll}
\toprule
\textbf{Example} & \textbf{Label} & \textbf{Generated Text} \\
\midrule
\multirow{12}{*}{$\#1$} & \multirow{6}{*}{entailment} 
& Sentence 1: \textit{It embodies the aesthetic ideals of the Queen Anne style through its design}\\
& & \quad \quad \quad \quad \quad \textit{and detailing and shows the type of house that attracted prosperous Napans}\\
& & \quad \quad \quad \quad \quad \textit{at the end of the Victorian era.}\\ 
& & Sentence 2: \underline{It is a fine example of the Queen Anne style of architecture through its design}\\
& & \quad \quad \quad \quad \quad \underline{and detailing and shows the type of house that attracted prosperous Napans}\\
& & \quad \quad \quad \quad \quad \underline{at the end of the Victorian era.}\\
& \multirow{6}{*}{not entailment} 
& Sentence 1: \textit{It embodies the aesthetic ideals of the Queen Anne style through its design}\\
& & \quad \quad \quad \quad \quad \textit{and detailing and shows the type of house that attracted prosperous Napans}\\
& & \quad \quad \quad \quad \quad \textit{at the end of the Victorian era.}\\ 
& & Sentence 2: \underline{The building is a fine example in this style, with an elegant facade reminiscent}\\
& & \quad \quad \quad \quad \quad \underline{to those found on many grand mansions built by wealthy merchants during}\\
& & \quad \quad \quad \quad \quad \underline{America's Gilded Age.} \\
\midrule
\multirow{9}{*}{$\#2$} & \multirow{5}{*}{entailment} 
& Sentence 1: \textit{Crosbie ran unsuccessfully for the leadership of the Liberal Party of Newfoundland}\\
& & \quad \quad \quad \quad \quad \textit{and Labrador in 1969, losing to Smallwood, and was also a candidate in the}\\ 
& & \quad \quad \quad \quad \quad \textit{Progressive Conservative Party of Canada's 1983 leadership election, placing third.}\\
& & Sentence 2: \underline{Crosbie was a candidate in the Progressive Conservative Party of Canada's 1983}\\
& & \quad \quad \quad \quad \quad \underline{leadership election, placing third.}\\
& \multirow{4}{*}{not entailment} 
& Sentence 1: \textit{Crosbie ran unsuccessfully for the leadership of the Liberal Party of Newfoundland}\\
& & \quad \quad \quad \quad \quad \textit{and Labrador in 1969, losing to Smallwood, and was also a candidate in the}\\ 
& & \quad \quad \quad \quad \quad \textit{Progressive Conservative Party of Canada's 1983 leadership election, placing third.}\\
& & Sentence 2: \underline{He lost his bid as leader after he failed twice at running against John Diefenbaker.}\\
\bottomrule
\end{tabular}
}
\vspace{-.5em}
\label{tab:case_mrpc}
\end{table}







\begin{table}[hb]
\renewcommand\arraystretch{1.2}
% \vspace{-1em}
\caption{
16-shot training samples of SST-2.
}
\vspace{-1em}
\centering
\small 
\resizebox{\textwidth}{!}{
\begin{tabular}{lll}
\toprule
\textbf{Label} & \textbf{Example} & \textbf{Review Text} \\
\midrule
\multirow{25}{*}{positive} & \multirow{2}{*}{$\#1$} 
& (ramsay) visually transforms the dreary expanse of dead-end distaste the characters inhabit into a poem of art ,\\
& & music and metaphor .\\
& \multirow{1}{*}{$\#2$} 
& the film jolts the laughs from the audience -- as if by cattle prod . \\
& \multirow{2}{*}{$\#3$} 
& the film presents visceral and dangerously honest revelations about the men and machines behind the curtains\\
& & of our planet . \\
& \multirow{1}{*}{$\#4$} 
& a film that will enthrall the whole family . \\
& \multirow{2}{*}{$\#5$} 
& serious movie-goers embarking upon this journey will find that the road to perdition leads to a satisfying\\
& & destination . \\
& \multirow{1}{*}{$\#6$} 
& sweet and memorable film . \\
& \multirow{2}{*}{$\#7$} 
& shyamalan takes a potentially trite and overused concept (aliens come to earth) and infuses it into a \\
& & rustic , realistic , and altogether creepy tale of hidden invasion . \\
& \multirow{2}{*}{$\#8$} 
& a crisp psychological drama (and) a fascinating little thriller that would have been perfect for an old \\
& & `` twilight zone '' episode . \\
& \multirow{2}{*}{$\#9$} 
& my big fat greek wedding is not only the best date movie of the year , it 's also a -- dare i say it twice\\
& & -- delightfully charming -- and totally american , i might add -- slice of comedic bliss .\\
& \multirow{2}{*}{$\#10$} 
& a comedy-drama of nearly epic proportions rooted in a sincere performance by the title character undergoing\\
& & midlife crisis . \\
& \multirow{1}{*}{$\#11$} 
& diggs and lathan are among the chief reasons brown sugar is such a sweet and sexy film . \\
& \multirow{1}{*}{$\#12$} 
& you 're not merely watching history , you 're engulfed by it . \\
& \multirow{1}{*}{$\#13$} 
& the concept is a hoot . \\
& \multirow{2}{*}{$\#14$} 
& the filmmakers ' eye for detail and the high standards of performance convey a strong sense of the \\
& & girls ' environment . \\
& \multirow{1}{*}{$\#15$} 
& a haunting tale of murder and mayhem . \\
& \multirow{2}{*}{$\#16$} 
& neil burger here succeeded in ... making the mystery of four decades back the springboard for a more\\
& & immediate mystery in the present . \\
\midrule
\multirow{26}{*}{negative} & \multirow{1}{*}{$\#1$} 
& nothing happens , and it happens to flat characters .\\
& \multirow{1}{*}{$\#2$} 
& as lively an account as seinfeld is deadpan . \\
& \multirow{2}{*}{$\#3$} 
& so we got ten little indians meets friday the 13th by way of clean and sober , filmed on the set of carpenter 's\\
& & the thing and loaded with actors you 're most likely to find on the next inevitable incarnation of the love boat . \\
& \multirow{3}{*}{$\#4$} 
& the plot is nothing but boilerplate cliches from start to finish , and the script assumes that not only would\\
& & subtlety be lost on the target audience , but that it 's also too stupid to realize that they 've already seen this\\
& & exact same movie a hundred times \\
& \multirow{2}{*}{$\#5$} 
& ultimately , sarah 's dedication to finding her husband seems more psychotic than romantic , and nothing in\\
& & the movie makes a convincing case that one woman 's broken heart outweighs all the loss we witness . \\
& \multirow{2}{*}{$\#6$} 
& the big finish is a bit like getting all excited about a chocolate eclair and then biting into it and finding\\
& & the filling missing . \\
& \multirow{3}{*}{$\#7$} 
& this picture is mostly a lump of run-of-the-mill profanity sprinkled with a few remarks so geared toward\\
& & engendering audience sympathy that you might think he was running for office -- or trying to win over a\\
& & probation officer . \\
& \multirow{2}{*}{$\#8$} 
& just because a walk to remember is shrewd enough to activate girlish tear ducts does n't mean it 's good enough\\
& & for our girls . \\
& \multirow{1}{*}{$\#9$} 
& often lingers just as long on the irrelevant as on the engaging , which gradually turns what time is it there ?\\
& \multirow{1}{*}{$\#10$} 
& this movie , a certain scene in particular , brought me uncomfortably close to losing my lunch .\\
& \multirow{1}{*}{$\#11$} 
& but it would be better to wait for the video . \\
& \multirow{2}{*}{$\#12$} 
& a rude black comedy about the catalytic effect a holy fool has upon those around him in the cutthroat world\\
& & of children 's television .\\
& \multirow{1}{*}{$\#13$} 
& just a collection of this and that -- whatever fills time -- with no unified whole .\\
& \multirow{2}{*}{$\#14$} 
& although god is great addresses interesting matters of identity and heritage , it 's hard to shake the feeling\\
& & that it was intended to be a different kind of film . \\
& \multirow{1}{*}{$\#15$} 
& the chocolate factory without charlie . \\
& \multirow{1}{*}{$\#16$} 
& in that setting , their struggle is simply too ludicrous and borderline insulting . \\
\bottomrule
\end{tabular}
}
\vspace{-.5em}
\label{tab:sst2_fewshot}
\end{table}