
\section{Method}

\subsection{Preliminaries}
\input{figs/overview.tex}
%\vspace{-0.4cm}
\section{DeepFlow Overview}\label{sec:overview}

Figure~\ref{fig:overview} shows an overview of the \name framework. \name takes the following set of \textbf{inputs}: 
%
(1) \underline{System} design hierarchy (e.g., the number of accelerator nodes per device, the number of devices in the system, the network topology connecting nodes within a device and across the devices), 
(2) \underline{Architecture template} of each accelerator node which provides a high-level definition of its components and how those components fit together. The purpose of the template is to provide a blueprint for the accelerator without committing to any specific hardware parameters.
%A component definition (e.g., minimal compute units (MCU\footnote{Examples of what we regard as MCU includes SMU in older GPUs, Tensor Cores in newer GPUs or systolic array in TPUs}), memory hierarchy, network), specification of each component (e.g., flop rate for each MCU, MCU dimensions, number of MCUs sharing a set of register files, dataflow execution model, and characteristics and scope of different levels of memory hierarchy), 
(3) \underline{Technology} parameters for each hardware component (e.g. energy per flop), 
(4) \underline{Design budgets} for each hardware component (area, power, perimeter),  
(5) \underline{Machine learning model} specification in the form of a high-level compute graph, parameters of each compute node (kernel type, tensor dimensions), and
(6) \underline{Parallelism strategy} (data, model, kernel, and/or pipeline parallelism dimensions) which distributes the compute graph across the entire system. 
(7) \underline{Device mapping} strategy which defines mapping of parallel shards onto hardware nodes.
Given these inputs, \name predicts the end-to-end performance of one iteration (i.e., single batch) of the model and finds an optimal hardware-software-technology design point as \textbf{output}. 

DeepFlow is composed of two major components.
\underline{CrossFlow} which operates in a stand-alone mode and can predict performance for any input configuration; and a search and optimization engine (\underline{SOE}) which enables design space search. 
%To do so, \name breaks the problem into multiple phases.
%Each phase or building block of \name is described in details next.
\vspace{-0.1cm}
\subsection{CrossFlow Building Blocks}

\paragraph*{\em Micro-Architecture Generator Engine (AGE)}

AGE takes the following set of \textbf{inputs}:
(1) Design constraints (i.e the power, area and perimeter budget and breakdown across micro-architectural components such as cache, network, compute cores). 
This breakdown can be provided manually by users or automatically by the Search and Optimization Engine (SOE, Section~\ref{subsec:soe}).
%We also provide technology specifications such as 
%and their physical characteristics such as area/power per core under nominal operating conditions, SRAM/register characteristics. 
(2) Technology parameters such as energy per flop, energy per data bit transfer for each level of memory and network hierarchy, threshold and maximum gate voltage, integration substrate parameters such as bump/interconnect pitch. We provide a wide range of standard and future technology libraries as baseline. (3) Architecture template which is a blueprint of the underlying accelerator chip without committing to any specific hardware parameters. Given these input, AGE performs a frequency-voltage-area scaling optimization to generate the following \textbf{output} parameters such that design budgets for all component are met: 
(1) Compute throughput.
(2) Capacity for different levels of memory hierarchy.
(3) Bandwidth to each level of memory hierarchy.
(4) Inter-node as well as intra-node network bandwidth. 
These parameters are then utilized by the performance prediction engine (PPE) to estimate the execution time of each kernel.
%As mentioned previously, 
%The output of this stage is the input to performance engine to estimate the execution time of each kernel. Next, we describe the search and optimization engine (SOE) which feeds input values to AGE, if we want to use the model for architecture search.
%\vspace{-0.2cm}
\paragraph*{\em Compute Graph Transformation and Device Placement Engine (DPE)}
The parallelization strategy and device mapping are critical in deciding the overall execution time. Here, we first transform the model graph to a `super-graph' to reflect the parallelization strategy provided by the users manually, or SOE engine (Section~\ref{subsec:soe}) automatically. For example, to apply data parallelism, the model graph is replicated and appropriate edges are added to model the gradient exchange. After generating the transformed graph, DPE assigns the vertices of the transformed graph to the system nodes following a heuristic approach to minimize the communication overhead. %
%The details are presented in section~\ref{}.

%\vspace{-0.2cm}
\paragraph*{\em Performance Prediction Engine (PPE)}
%With the device mapping for all the vertices of the compute (super-)graph known, the next step is to calculate the overall execution time for a forward pass and/or a backward pass. 
We use hierarchical roofline modeling to predict the performance of each compute node. To calculate the overall end-to-end execution time, while respecting scheduling constraints (e.g. one kernel at a time per GPU, or prioritizing one kernel launch over another) we use event-driven simulation.%
%We explain the details of the PPE in section~\ref{}.
\subsection{Search and Optimization Engine (SOE)}\label{subsec:soe}
Co-optimizing micro-architectural parameters and the parallelization strategy that minimizes the overall end-to-end execution time requires navigating a large space of design parameters. 
Search and optimization engine (SOE) enables the automatic design space search and finds an 
%that meets the total power and area constraints, and simultaneously explores software parallelization strategies to find the 
optimal design point which meets the design constraints and minimizes the overall execution time.
%Because the hardware configuration space is very large, the search algorithm we designed 
SOE takes inspiration from ML-assisted search algorithms, in particular gradient decent search with momentum and builds on top of the CrossFlow modeling engine.
%The software parallelization design space is much smaller compared to the hardware design space and therefore we employ an exhaustive grid search. 

%Gradient search is an iterative process. In each step, SOE takes the predicted time from previous iteration as input to re-adjust the following parameter settings: (1) power, area and perimeter breakdown across different architectural components. (2) a parallelization strategy. These parameters will be fed back to CrossFlow to estimate the overall execution time. This process continues until convergence or user-specified number of steps. 
%The details of SOE's search algorithm are elaborated in Section~\ref{}. 
\vspace{-0.2cm}
\subsection{Parallelism Strategy Space}
\label{subsec:par_strategy}
There are a myriad of ways to parallelize a model across a large multi-node system. Exploring the parallelism space and finding the optimal strategy is critical to overall performance and system utilization. DeepFlow explores kernel, data and layer parallelism. It uniquely identifies each parallelism strategy by following notations: $\texttt{RC-\{KP1\}-\{KP2\}-d\{DP\}-p\{LP\}}$ or $\texttt{CR-\{KP1\}-d\{DP\}-p\{LP\}}$ depending on the choice of kernel parallelism.
RC (Row-Column) and CR (Column-Row) refer to different forms of kernel parallelism, i.e. distributed GEMM through inner-product or outer-product implementation.
%\begin{equation*}
%    \texttt{RC: R{KP1\}\_C\{KP2\}\_d\{DP\}\_p\{LP\}}
%\end{equation*}
%Where \texttt{RC} or \texttt{CR} refers to the type of kernel parallelism strategy, i.e. Row-Column or Column-Row,
%\texttt{N} refers to the number of parallel nodes,
\texttt{KP1} and \texttt{KP2} are the parameters of distributed GEMM. 
For Row-Column (\texttt{RC}) or inner-product, \texttt{KP1} and \texttt{KP2} would refer to the number of ways we shard the first matrix across rows and the second matrix across columns.
For Column-Row (\texttt{CR}) or outer-product, we would only need one parameter to specify the parallelization strategy; \texttt{KP1} will refer to the number of ways we cut the first matrix across columns and the second matrix across rows.
\texttt{DP} represents the number of model replicas and data shards assigned to each to exploit data parallelism.
\texttt{LP} is the number of ways we cut layers into stages to exploit pipeline parallelism.

\begin{comment}
\subsection{Modes of Operation}
\name has two modes of operation, standalone performance estimation mode and a architecture search mode.
\paragraph{Standalone Performance (SP) Estimation Mode}
Often ML practitioners or hardware designers want to estimate the performance of a model on a particular system configuration. For example, what is the cost optimal number of accelerators that one should deploy for distributed training? Or what is the estimated performance gain from choosing an accelerator with costlier HBM2E vs HBM2? To enable one to quickly answer such questions and to estimate performance under certain known system configurations, the tool can be run in the SP mode. 

In this mode, the description of the architecture of a scale-out system consisting of multiple accelerators, the architecture of the accelerator hardware themselves and the description of the neural network is taken as input, and fed into CrossFlow, which calculates the execution time of each training step. 

%In this mode, the description of the architecture of a scale-out system consisting of multiple accelerators, the architecture of the accelerator hardware themselves and the description of the neural network is taken as input. The tool calculates the execution time of each training step. 

%In this mode, the user 
%has the flexibility to use either just the \perfE or use \perfE alongside the AGE. While using just the \perfE  alone, the user needs to provide the architectural parameters of the tiles and the system. On the other hand, while using AGE  alongside \perfE, the user 
%needs to define the technology parameters and the hardware constraints i.e., the overall area and power breakdown among the different architectural components of the system. T

%In this mode, the tool generates the micro-architectural parameters of the accelerator chip using the AGE. It then runs the compute graph transformation and the device placement engine, and uses the \perfE to predict the execution time. 

\subsubsection{Architecture Search (AS) Mode}

The insatiable demand to run large models in the shortest possible time demands that we find the optimal hardware and software design points to train these models. From the hardware perspective, it is about finding the right micro-architecture as well as the overall system architecture of the distributed system. 
From the software perspective, it is about finding the right parallelization strategy. 
Often these decisions depend on each other, and so finding the optimal design points across the stack means 
navigating a large design space.

As one can imagine, the design space of the inputs to the tool is large and iterating over the entire design space is a tedious task. To efficiently search over the input space to find the optimal hardware constraints and parallelization strategy, the tool can be run in the AS mode. 
In this mode, the SOE module is used. The user will not need to provide the exact hardware parameters and the parallelization strategy. Only the architecture template and the initial compute graph will need to be provided as input to the tool. The tool then performs a search over the design space to find the optimal parameter settings that results in minimum training time. 
%We used gradient descent algorithm (details in Section~\ref{}) for this search.

%\subsection{Inputs and Outputs}

%\paragraph{SP-Mode}
%In this mode, the hardware 

%\paragraph{AS-Mode}

\end{comment}

%\vspace{-0.4cm}
\section{DeepFlow Overview}\label{sec:overview}

Figure~\ref{fig:overview} shows an overview of the \name framework. \name takes the following set of \textbf{inputs}: 
%
(1) \underline{System} design hierarchy (e.g., the number of accelerator nodes per device, the number of devices in the system, the network topology connecting nodes within a device and across the devices), 
(2) \underline{Architecture template} of each accelerator node which provides a high-level definition of its components and how those components fit together. The purpose of the template is to provide a blueprint for the accelerator without committing to any specific hardware parameters.
%A component definition (e.g., minimal compute units (MCU\footnote{Examples of what we regard as MCU includes SMU in older GPUs, Tensor Cores in newer GPUs or systolic array in TPUs}), memory hierarchy, network), specification of each component (e.g., flop rate for each MCU, MCU dimensions, number of MCUs sharing a set of register files, dataflow execution model, and characteristics and scope of different levels of memory hierarchy), 
(3) \underline{Technology} parameters for each hardware component (e.g. energy per flop), 
(4) \underline{Design budgets} for each hardware component (area, power, perimeter),  
(5) \underline{Machine learning model} specification in the form of a high-level compute graph, parameters of each compute node (kernel type, tensor dimensions), and
(6) \underline{Parallelism strategy} (data, model, kernel, and/or pipeline parallelism dimensions) which distributes the compute graph across the entire system. 
(7) \underline{Device mapping} strategy which defines mapping of parallel shards onto hardware nodes.
Given these inputs, \name predicts the end-to-end performance of one iteration (i.e., single batch) of the model and finds an optimal hardware-software-technology design point as \textbf{output}. 

DeepFlow is composed of two major components.
\underline{CrossFlow} which operates in a stand-alone mode and can predict performance for any input configuration; and a search and optimization engine (\underline{SOE}) which enables design space search. 
%To do so, \name breaks the problem into multiple phases.
%Each phase or building block of \name is described in details next.
\vspace{-0.1cm}
\subsection{CrossFlow Building Blocks}

\paragraph*{\em Micro-Architecture Generator Engine (AGE)}

AGE takes the following set of \textbf{inputs}:
(1) Design constraints (i.e the power, area and perimeter budget and breakdown across micro-architectural components such as cache, network, compute cores). 
This breakdown can be provided manually by users or automatically by the Search and Optimization Engine (SOE, Section~\ref{subsec:soe}).
%We also provide technology specifications such as 
%and their physical characteristics such as area/power per core under nominal operating conditions, SRAM/register characteristics. 
(2) Technology parameters such as energy per flop, energy per data bit transfer for each level of memory and network hierarchy, threshold and maximum gate voltage, integration substrate parameters such as bump/interconnect pitch. We provide a wide range of standard and future technology libraries as baseline. (3) Architecture template which is a blueprint of the underlying accelerator chip without committing to any specific hardware parameters. Given these input, AGE performs a frequency-voltage-area scaling optimization to generate the following \textbf{output} parameters such that design budgets for all component are met: 
(1) Compute throughput.
(2) Capacity for different levels of memory hierarchy.
(3) Bandwidth to each level of memory hierarchy.
(4) Inter-node as well as intra-node network bandwidth. 
These parameters are then utilized by the performance prediction engine (PPE) to estimate the execution time of each kernel.
%As mentioned previously, 
%The output of this stage is the input to performance engine to estimate the execution time of each kernel. Next, we describe the search and optimization engine (SOE) which feeds input values to AGE, if we want to use the model for architecture search.
%\vspace{-0.2cm}
\paragraph*{\em Compute Graph Transformation and Device Placement Engine (DPE)}
The parallelization strategy and device mapping are critical in deciding the overall execution time. Here, we first transform the model graph to a `super-graph' to reflect the parallelization strategy provided by the users manually, or SOE engine (Section~\ref{subsec:soe}) automatically. For example, to apply data parallelism, the model graph is replicated and appropriate edges are added to model the gradient exchange. After generating the transformed graph, DPE assigns the vertices of the transformed graph to the system nodes following a heuristic approach to minimize the communication overhead. %
%The details are presented in section~\ref{}.

%\vspace{-0.2cm}
\paragraph*{\em Performance Prediction Engine (PPE)}
%With the device mapping for all the vertices of the compute (super-)graph known, the next step is to calculate the overall execution time for a forward pass and/or a backward pass. 
We use hierarchical roofline modeling to predict the performance of each compute node. To calculate the overall end-to-end execution time, while respecting scheduling constraints (e.g. one kernel at a time per GPU, or prioritizing one kernel launch over another) we use event-driven simulation.%
%We explain the details of the PPE in section~\ref{}.
\subsection{Search and Optimization Engine (SOE)}\label{subsec:soe}
Co-optimizing micro-architectural parameters and the parallelization strategy that minimizes the overall end-to-end execution time requires navigating a large space of design parameters. 
Search and optimization engine (SOE) enables the automatic design space search and finds an 
%that meets the total power and area constraints, and simultaneously explores software parallelization strategies to find the 
optimal design point which meets the design constraints and minimizes the overall execution time.
%Because the hardware configuration space is very large, the search algorithm we designed 
SOE takes inspiration from ML-assisted search algorithms, in particular gradient decent search with momentum and builds on top of the CrossFlow modeling engine.
%The software parallelization design space is much smaller compared to the hardware design space and therefore we employ an exhaustive grid search. 

%Gradient search is an iterative process. In each step, SOE takes the predicted time from previous iteration as input to re-adjust the following parameter settings: (1) power, area and perimeter breakdown across different architectural components. (2) a parallelization strategy. These parameters will be fed back to CrossFlow to estimate the overall execution time. This process continues until convergence or user-specified number of steps. 
%The details of SOE's search algorithm are elaborated in Section~\ref{}. 
\vspace{-0.2cm}
\subsection{Parallelism Strategy Space}
\label{subsec:par_strategy}
There are a myriad of ways to parallelize a model across a large multi-node system. Exploring the parallelism space and finding the optimal strategy is critical to overall performance and system utilization. DeepFlow explores kernel, data and layer parallelism. It uniquely identifies each parallelism strategy by following notations: $\texttt{RC-\{KP1\}-\{KP2\}-d\{DP\}-p\{LP\}}$ or $\texttt{CR-\{KP1\}-d\{DP\}-p\{LP\}}$ depending on the choice of kernel parallelism.
RC (Row-Column) and CR (Column-Row) refer to different forms of kernel parallelism, i.e. distributed GEMM through inner-product or outer-product implementation.
%\begin{equation*}
%    \texttt{RC: R{KP1\}\_C\{KP2\}\_d\{DP\}\_p\{LP\}}
%\end{equation*}
%Where \texttt{RC} or \texttt{CR} refers to the type of kernel parallelism strategy, i.e. Row-Column or Column-Row,
%\texttt{N} refers to the number of parallel nodes,
\texttt{KP1} and \texttt{KP2} are the parameters of distributed GEMM. 
For Row-Column (\texttt{RC}) or inner-product, \texttt{KP1} and \texttt{KP2} would refer to the number of ways we shard the first matrix across rows and the second matrix across columns.
For Column-Row (\texttt{CR}) or outer-product, we would only need one parameter to specify the parallelization strategy; \texttt{KP1} will refer to the number of ways we cut the first matrix across columns and the second matrix across rows.
\texttt{DP} represents the number of model replicas and data shards assigned to each to exploit data parallelism.
\texttt{LP} is the number of ways we cut layers into stages to exploit pipeline parallelism.

\begin{comment}
\subsection{Modes of Operation}
\name has two modes of operation, standalone performance estimation mode and a architecture search mode.
\paragraph{Standalone Performance (SP) Estimation Mode}
Often ML practitioners or hardware designers want to estimate the performance of a model on a particular system configuration. For example, what is the cost optimal number of accelerators that one should deploy for distributed training? Or what is the estimated performance gain from choosing an accelerator with costlier HBM2E vs HBM2? To enable one to quickly answer such questions and to estimate performance under certain known system configurations, the tool can be run in the SP mode. 

In this mode, the description of the architecture of a scale-out system consisting of multiple accelerators, the architecture of the accelerator hardware themselves and the description of the neural network is taken as input, and fed into CrossFlow, which calculates the execution time of each training step. 

%In this mode, the description of the architecture of a scale-out system consisting of multiple accelerators, the architecture of the accelerator hardware themselves and the description of the neural network is taken as input. The tool calculates the execution time of each training step. 

%In this mode, the user 
%has the flexibility to use either just the \perfE or use \perfE alongside the AGE. While using just the \perfE  alone, the user needs to provide the architectural parameters of the tiles and the system. On the other hand, while using AGE  alongside \perfE, the user 
%needs to define the technology parameters and the hardware constraints i.e., the overall area and power breakdown among the different architectural components of the system. T

%In this mode, the tool generates the micro-architectural parameters of the accelerator chip using the AGE. It then runs the compute graph transformation and the device placement engine, and uses the \perfE to predict the execution time. 

\subsubsection{Architecture Search (AS) Mode}

The insatiable demand to run large models in the shortest possible time demands that we find the optimal hardware and software design points to train these models. From the hardware perspective, it is about finding the right micro-architecture as well as the overall system architecture of the distributed system. 
From the software perspective, it is about finding the right parallelization strategy. 
Often these decisions depend on each other, and so finding the optimal design points across the stack means 
navigating a large design space.

As one can imagine, the design space of the inputs to the tool is large and iterating over the entire design space is a tedious task. To efficiently search over the input space to find the optimal hardware constraints and parallelization strategy, the tool can be run in the AS mode. 
In this mode, the SOE module is used. The user will not need to provide the exact hardware parameters and the parallelization strategy. Only the architecture template and the initial compute graph will need to be provided as input to the tool. The tool then performs a search over the design space to find the optimal parameter settings that results in minimum training time. 
%We used gradient descent algorithm (details in Section~\ref{}) for this search.

%\subsection{Inputs and Outputs}

%\paragraph{SP-Mode}
%In this mode, the hardware 

%\paragraph{AS-Mode}

\end{comment}

%\vspace{-0.4cm}
\section{DeepFlow Overview}\label{sec:overview}

Figure~\ref{fig:overview} shows an overview of the \name framework. \name takes the following set of \textbf{inputs}: 
%
(1) \underline{System} design hierarchy (e.g., the number of accelerator nodes per device, the number of devices in the system, the network topology connecting nodes within a device and across the devices), 
(2) \underline{Architecture template} of each accelerator node which provides a high-level definition of its components and how those components fit together. The purpose of the template is to provide a blueprint for the accelerator without committing to any specific hardware parameters.
%A component definition (e.g., minimal compute units (MCU\footnote{Examples of what we regard as MCU includes SMU in older GPUs, Tensor Cores in newer GPUs or systolic array in TPUs}), memory hierarchy, network), specification of each component (e.g., flop rate for each MCU, MCU dimensions, number of MCUs sharing a set of register files, dataflow execution model, and characteristics and scope of different levels of memory hierarchy), 
(3) \underline{Technology} parameters for each hardware component (e.g. energy per flop), 
(4) \underline{Design budgets} for each hardware component (area, power, perimeter),  
(5) \underline{Machine learning model} specification in the form of a high-level compute graph, parameters of each compute node (kernel type, tensor dimensions), and
(6) \underline{Parallelism strategy} (data, model, kernel, and/or pipeline parallelism dimensions) which distributes the compute graph across the entire system. 
(7) \underline{Device mapping} strategy which defines mapping of parallel shards onto hardware nodes.
Given these inputs, \name predicts the end-to-end performance of one iteration (i.e., single batch) of the model and finds an optimal hardware-software-technology design point as \textbf{output}. 

DeepFlow is composed of two major components.
\underline{CrossFlow} which operates in a stand-alone mode and can predict performance for any input configuration; and a search and optimization engine (\underline{SOE}) which enables design space search. 
%To do so, \name breaks the problem into multiple phases.
%Each phase or building block of \name is described in details next.
\vspace{-0.1cm}
\subsection{CrossFlow Building Blocks}

\paragraph*{\em Micro-Architecture Generator Engine (AGE)}

AGE takes the following set of \textbf{inputs}:
(1) Design constraints (i.e the power, area and perimeter budget and breakdown across micro-architectural components such as cache, network, compute cores). 
This breakdown can be provided manually by users or automatically by the Search and Optimization Engine (SOE, Section~\ref{subsec:soe}).
%We also provide technology specifications such as 
%and their physical characteristics such as area/power per core under nominal operating conditions, SRAM/register characteristics. 
(2) Technology parameters such as energy per flop, energy per data bit transfer for each level of memory and network hierarchy, threshold and maximum gate voltage, integration substrate parameters such as bump/interconnect pitch. We provide a wide range of standard and future technology libraries as baseline. (3) Architecture template which is a blueprint of the underlying accelerator chip without committing to any specific hardware parameters. Given these input, AGE performs a frequency-voltage-area scaling optimization to generate the following \textbf{output} parameters such that design budgets for all component are met: 
(1) Compute throughput.
(2) Capacity for different levels of memory hierarchy.
(3) Bandwidth to each level of memory hierarchy.
(4) Inter-node as well as intra-node network bandwidth. 
These parameters are then utilized by the performance prediction engine (PPE) to estimate the execution time of each kernel.
%As mentioned previously, 
%The output of this stage is the input to performance engine to estimate the execution time of each kernel. Next, we describe the search and optimization engine (SOE) which feeds input values to AGE, if we want to use the model for architecture search.
%\vspace{-0.2cm}
\paragraph*{\em Compute Graph Transformation and Device Placement Engine (DPE)}
The parallelization strategy and device mapping are critical in deciding the overall execution time. Here, we first transform the model graph to a `super-graph' to reflect the parallelization strategy provided by the users manually, or SOE engine (Section~\ref{subsec:soe}) automatically. For example, to apply data parallelism, the model graph is replicated and appropriate edges are added to model the gradient exchange. After generating the transformed graph, DPE assigns the vertices of the transformed graph to the system nodes following a heuristic approach to minimize the communication overhead. %
%The details are presented in section~\ref{}.

%\vspace{-0.2cm}
\paragraph*{\em Performance Prediction Engine (PPE)}
%With the device mapping for all the vertices of the compute (super-)graph known, the next step is to calculate the overall execution time for a forward pass and/or a backward pass. 
We use hierarchical roofline modeling to predict the performance of each compute node. To calculate the overall end-to-end execution time, while respecting scheduling constraints (e.g. one kernel at a time per GPU, or prioritizing one kernel launch over another) we use event-driven simulation.%
%We explain the details of the PPE in section~\ref{}.
\subsection{Search and Optimization Engine (SOE)}\label{subsec:soe}
Co-optimizing micro-architectural parameters and the parallelization strategy that minimizes the overall end-to-end execution time requires navigating a large space of design parameters. 
Search and optimization engine (SOE) enables the automatic design space search and finds an 
%that meets the total power and area constraints, and simultaneously explores software parallelization strategies to find the 
optimal design point which meets the design constraints and minimizes the overall execution time.
%Because the hardware configuration space is very large, the search algorithm we designed 
SOE takes inspiration from ML-assisted search algorithms, in particular gradient decent search with momentum and builds on top of the CrossFlow modeling engine.
%The software parallelization design space is much smaller compared to the hardware design space and therefore we employ an exhaustive grid search. 

%Gradient search is an iterative process. In each step, SOE takes the predicted time from previous iteration as input to re-adjust the following parameter settings: (1) power, area and perimeter breakdown across different architectural components. (2) a parallelization strategy. These parameters will be fed back to CrossFlow to estimate the overall execution time. This process continues until convergence or user-specified number of steps. 
%The details of SOE's search algorithm are elaborated in Section~\ref{}. 
\vspace{-0.2cm}
\subsection{Parallelism Strategy Space}
\label{subsec:par_strategy}
There are a myriad of ways to parallelize a model across a large multi-node system. Exploring the parallelism space and finding the optimal strategy is critical to overall performance and system utilization. DeepFlow explores kernel, data and layer parallelism. It uniquely identifies each parallelism strategy by following notations: $\texttt{RC-\{KP1\}-\{KP2\}-d\{DP\}-p\{LP\}}$ or $\texttt{CR-\{KP1\}-d\{DP\}-p\{LP\}}$ depending on the choice of kernel parallelism.
RC (Row-Column) and CR (Column-Row) refer to different forms of kernel parallelism, i.e. distributed GEMM through inner-product or outer-product implementation.
%\begin{equation*}
%    \texttt{RC: R{KP1\}\_C\{KP2\}\_d\{DP\}\_p\{LP\}}
%\end{equation*}
%Where \texttt{RC} or \texttt{CR} refers to the type of kernel parallelism strategy, i.e. Row-Column or Column-Row,
%\texttt{N} refers to the number of parallel nodes,
\texttt{KP1} and \texttt{KP2} are the parameters of distributed GEMM. 
For Row-Column (\texttt{RC}) or inner-product, \texttt{KP1} and \texttt{KP2} would refer to the number of ways we shard the first matrix across rows and the second matrix across columns.
For Column-Row (\texttt{CR}) or outer-product, we would only need one parameter to specify the parallelization strategy; \texttt{KP1} will refer to the number of ways we cut the first matrix across columns and the second matrix across rows.
\texttt{DP} represents the number of model replicas and data shards assigned to each to exploit data parallelism.
\texttt{LP} is the number of ways we cut layers into stages to exploit pipeline parallelism.

\begin{comment}
\subsection{Modes of Operation}
\name has two modes of operation, standalone performance estimation mode and a architecture search mode.
\paragraph{Standalone Performance (SP) Estimation Mode}
Often ML practitioners or hardware designers want to estimate the performance of a model on a particular system configuration. For example, what is the cost optimal number of accelerators that one should deploy for distributed training? Or what is the estimated performance gain from choosing an accelerator with costlier HBM2E vs HBM2? To enable one to quickly answer such questions and to estimate performance under certain known system configurations, the tool can be run in the SP mode. 

In this mode, the description of the architecture of a scale-out system consisting of multiple accelerators, the architecture of the accelerator hardware themselves and the description of the neural network is taken as input, and fed into CrossFlow, which calculates the execution time of each training step. 

%In this mode, the description of the architecture of a scale-out system consisting of multiple accelerators, the architecture of the accelerator hardware themselves and the description of the neural network is taken as input. The tool calculates the execution time of each training step. 

%In this mode, the user 
%has the flexibility to use either just the \perfE or use \perfE alongside the AGE. While using just the \perfE  alone, the user needs to provide the architectural parameters of the tiles and the system. On the other hand, while using AGE  alongside \perfE, the user 
%needs to define the technology parameters and the hardware constraints i.e., the overall area and power breakdown among the different architectural components of the system. T

%In this mode, the tool generates the micro-architectural parameters of the accelerator chip using the AGE. It then runs the compute graph transformation and the device placement engine, and uses the \perfE to predict the execution time. 

\subsubsection{Architecture Search (AS) Mode}

The insatiable demand to run large models in the shortest possible time demands that we find the optimal hardware and software design points to train these models. From the hardware perspective, it is about finding the right micro-architecture as well as the overall system architecture of the distributed system. 
From the software perspective, it is about finding the right parallelization strategy. 
Often these decisions depend on each other, and so finding the optimal design points across the stack means 
navigating a large design space.

As one can imagine, the design space of the inputs to the tool is large and iterating over the entire design space is a tedious task. To efficiently search over the input space to find the optimal hardware constraints and parallelization strategy, the tool can be run in the AS mode. 
In this mode, the SOE module is used. The user will not need to provide the exact hardware parameters and the parallelization strategy. Only the architecture template and the initial compute graph will need to be provided as input to the tool. The tool then performs a search over the design space to find the optimal parameter settings that results in minimum training time. 
%We used gradient descent algorithm (details in Section~\ref{}) for this search.

%\subsection{Inputs and Outputs}

%\paragraph{SP-Mode}
%In this mode, the hardware 

%\paragraph{AS-Mode}

\end{comment}

\paragraph{Overview.} 
We consider the strict few-shot learning setting~\citep{Perez2021TrueFL}: 
The training set $\mathcal{D}_{\text{train}} = \{(\bs{x}, y)_i\}$ consists of $K$ training samples per label where $\bs{x} = [x_1, x_2, \dots, x_n]$ is a text sequence with $n$ tokens.
The development set $\mathcal{D}_{\text{dev}}$ is of the same size as $\mathcal{D}_{\text{train}}$.
There is no access to additional task-specific unlabeled data.
The number of training samples $K$ is assumed to be very small (\eg, $K=16$), making it challenging to train a classification model $C_{\phi}$ that generalizes well to unseen data.
To mitigate such a training data scarcity issue, we propose to first train an autoregressive PLM on $\mathcal{D}_{\text{train}}$, and then use it as a generator $G_{\theta}$ to synthesize a large amount of novel samples $\mathcal{D}_{\text{gen}} = \{(\tilde{\bs{x}}, \tilde{y})_i\}$ that augment the original training set.
Finally, a classification PLM $C_{\phi}$ is fine-tuned on both $\mathcal{D}_{\text{train}}$ and $\mathcal{D}_{\text{gen}}$ to perform the task.
An overview of our \method method is shown in Fig.~\ref{fig:overview}.

\paragraph{Text Generation with Autoregressive PLMs.}
In standard fine-tuning for text generation, an autoregressive PLM $G_{\theta}$ is trained via the maximum likelihood generation loss of each token in a sequence $\bs{x}$ conditioned on previous tokens:
\begin{equation*}
% \label{eq:gen}
\min_{\theta} -\frac{1}{n}\sum_{j=1}^n \log p_{\theta}(x_j|\bs{x}_{<j}),\quad
p_{\theta}(x_j|\bs{x}_{<j}) = \frac{\exp(\bs{e}_j^\top \bs{h}_j)}{\sum_{j'=1}^{|V|}\exp(\bs{e}_{j'}^\top \bs{h}_j)}.
\end{equation*}
where the token generation probability $p_{\theta}(\cdot)$ is usually parameterized using token embeddings $\bs{e}$ and hidden states $\bs{h}$ of a Transformer~\citep{vaswani2017attention} model.
After training, $G_{\theta}$ can be used to generate novel texts by iteratively sampling tokens from its generation probability distribution.

\paragraph{Prefix-Tuning.}
Unlike fine-tuning which updates all model parameters $\theta$ of a PLM, prefix-tuning~\citep{Li2021PrefixTuningOC} freezes all pretrained Transformer parameters and only optimizes prefix vectors $\theta_p$ that are prepended to each Transformer layer.
We use prefix-tuning for training $G_{\theta_p}$ on $\mathcal{D}_{\text{train}}$ because (1) it offers better effectiveness than fine-tuning for small datasets~\citep{Li2021PrefixTuningOC} and (2) the generation models for different labels can share the same backbone Transformer parameters with only the prefix vectors being different, significantly reducing the memory requirement for multi-class classification tasks.
% After pretraining, $G_{\theta}$ can be directly used to generate new texts by recursively sampling tokens from its output probability distribution. Typically, a temperature hyperparameter $\tau > 0$ is introduced during sampling~\citep{Hinton2015DistillingTK} to adjust the sharpness of the probability distribution:
% % \yuz{any reference of using $\tau$ in generation?}
% \begin{equation}
% \label{eq:temp_prob}
% p_{\theta}(x_i|\bs{x}_{<i}) = \frac{\exp(\bs{e}_i^\top \bs{h}_i/\tau)}{\sum_{j=1}^{|V|}\exp(\bs{e}_j^\top \bs{h}_i/\tau)},
% \end{equation}
% where $\tau \to 0$ approximates greedily picking the most probable next token; $\tau \to \infty$ induces a uniform distribution.
% Additionally, sampled tokens can be confined to the top-$k$ most probable ones to avoid low-quality tokens.
% In this work, we find such top-$k$ sampling with temperature is sufficient to produce coherent and meaningful texts as training data for NLU tasks. 
% Exploring more sophisticated sampling strategies~\citep{Holtzman2020TheCC} is left for future work.

% \input{Tables/prompts}
\subsection{Label-Discriminative Text Generator Tuning with Meta Weights}

\paragraph{Motivation.} To model the conditional text generation probability $p(\bs{x}|y_l)$ on different labels, a straightforward way is to parameterize a generation model $G_{\theta_{p_l}}$ for each label $y_l$ via a set of prefix vectors $\theta_{p_l}$ so that $p(\bs{x}|y_l)=p_{\theta_{p_l}}(\bs{x})$, and then tune $\theta_{p_l}$ on the training samples $\bs{x}$ with label $y_l$:
\begin{equation}
\label{eq:gen}
\min_{\theta_{p_l}}\mathcal{L}_{\text{gen}},\quad \mathcal{L}_{\text{gen}}(\theta_{p_l}) = -\frac{1}{n}\sum_{j=1}^n \log p_{\theta_{p_l}}(x_j|\bs{x}_{<j}).
\end{equation}

\begin{wrapfigure}[14]{wr}{0.32\textwidth}
% \vspace{-0.3cm}
% \subfigcapmargin=10pt
\centering
\includegraphics[width=0.32\textwidth]{Figs/disc_no_weight.pdf}
\caption{(On MNLI) Training the generator only via $\mathcal{L}_{\text{gen}}$ does not automatically decrease $\mathcal{L}_{\text{disc}}$.}
\label{fig:motivate}
% \vspace{-0.3cm}
\end{wrapfigure}

However, such an approach only optimizes the  \emph{generative} likelihood $p(\bs{x}|y_l)$ without accounting for \emph{label discriminativeness} $p(y_l|\bs{x})$ which is essential for generating unambiguous training samples to benefit the final classification task.
Indeed, we find that optimizing $p(\bs{x}|y_l)$ separately for each label does not necessarily make the generators aware of the distinction over different labels. As shown in Fig.~\ref{fig:motivate}, $\mathcal{L}_{\text{disc}}$ (defined in Eq.~\eqref{eq:disc}) can even increase during training---It is possible that the dominating patterns in the training samples are label-indiscriminate (\eg, a movie review dataset may frequently mention ``the movie''), making the generators of different labels eventually converge to similar distributions, especially when there are limited training samples per label.

To promote the generation of label-discriminative texts, we hope to generate each token $x_j$ so that the probability of the so far generated text sequence $\bs{x}_{\le j}$ is maximized towards label $y_l$ via $\mathcal{L}_{\text{disc}}$:
\begin{equation}
\begin{split}
\label{eq:disc}
\mathcal{L}_{\text{disc}} &= -\frac{1}{n}\sum_{j=1}^n \log p_{\theta_p}(y_l|\bs{x}_{\le j})\\
% ,\,\,
p_{\theta_p}(y_l|\bs{x}_{\le j}) &= \frac{p(x_j|y_l,\bs{x}_{<j})p(y_l)}{\sum_{l'=1}^{L} p(x_j|y_{l'},\bs{x}_{<j})p(y_{l'})} 
= \frac{p_{\theta_{p_l}}(x_j|\bs{x}_{<j})}{\sum_{l'=1}^{L} p_{\theta_{p_{l'}}}(x_j|\bs{x}_{<j})}
\end{split}
\end{equation}
where $\theta_p = \{\theta_{p_l}\}|_{l=1}^L$, and uniform label prior (\ie, $p(y_l) = 1/L$) is assumed; for non-uniform prior, the result will be up to some scaling.

Although one can directly combine $\mathcal{L}_{\text{disc}}$ with $\mathcal{L}_{\text{gen}}$ to train $G_{\theta_p}$ to enforce distinction across different labels, doing so will result in two undesirable consequences: (1) A hyperparameter needs to be introduced to balance the weights of the two losses, whose optimal value is likely to vary by task; and (2) the generation-irrelevant loss $\mathcal{L}_{\text{disc}}$ will unavoidably interfere the language modeling process, making the resulting model prone to generating less fluent and coherent texts.

\paragraph{Weighted Maximum Likelihood Generator Tuning.}
To preserve the generative learning of $G_{\theta_{p}}$ while emphasizing label-discriminative tokens, we assume each token is associated with a weight in the maximum likelihood loss.
Intuitively, when our goal is to generate distinctive texts across different labels as training samples, not all tokens should contribute equally to generator training.
For example, for sentiment classification tasks, one would expect ``good/bad'' to be more label-discriminative than ``the movie'', and the former should be paid more attention to during training. 
It is thus natural to revise $\mathcal{L}_{\text{gen}}$ from Eq.~\eqref{eq:gen} to $\mathcal{L}_{\text{w-gen}}$ as in Eq.~\eqref{eq:weigh_gen} by assuming a weight $w_j$ is given for each token.
\begin{equation}
\label{eq:weigh_gen}
\min_{\theta_{p_l}}\mathcal{L}_{\text{w-gen}},\quad \mathcal{L}_{\text{w-gen}}(\theta_{p_l};\bs{w}) = -\sum_{j=1}^n w_j \log p_{\theta_{p_l}}(x_j|\bs{x}_{<j}), \quad  \sum_{j=1}^n w_j = 1.
\end{equation}
Note that in $\mathcal{L}_{\text{w-gen}}$, $\bs{w}$ is assumed to be the \emph{hyperparameter} under which $\theta_{p_l}$ is optimized.
While it is possible to manually design weighting rules for setting $\bs{w}$ to promote label-discriminative learning, they will likely necessitate task-specific knowledge and nontrivial tuning.
To facilitate the automatic learning of these weights $\bs{w}$, we propose to parameterize them as learnable hyperparameters using the idea of meta-learning.
\SetKwInput{KwInput}{Input}
\SetKwInput{KwParameter}{Parameter}
\SetKwInput{KwOutput}{Output}
\SetCommentSty{mycommfont}
\begin{algorithm}[!t]
\DontPrintSemicolon
\SetNoFillComment
\KwInput{$\mathcal{D}_{\text{train}}$: Few-shot training set.}
\KwParameter{$T$: Number of training steps.}
\KwOutput{$\theta_{p}$: Prefix parameters for all labels.}
Initialize $\theta_{p}^{(0)}$ (with task-descriptive prompts) and $\omega^{(0)}$\;

\For{$t \in [0, 1, \dots, T-1]$}    
{ 
    $\mathcal{B} \gets \text{Sample a minibatch from }\mathcal{D}_{\text{train}}$\;
    
    $\hat{\theta}_{p}^{(t)}\left(\omega^{(t)}\right) \gets$ Take one gradient step to descend $\mathcal{L}_{\text{w-gen}}\left(\theta_{p}^{(t)};\omega^{(t)}\right)$ on $\mathcal{B}$\;
    
    $\omega^{(t+1)} \gets$ Take one gradient step to descend $\mathcal{L}_{\text{disc}}\left(\hat{\theta}_{p}^{(t)}\left(\omega^{(t)}\right)\right)$ on $\mathcal{B}$\
    
	$\theta_{p}^{(t+1)} \gets$ Take one gradient step to descend $\mathcal{L}_{\text{w-gen}}\left(\theta_{p}^{(t)};\omega^{(t+1)}\right)$ on $\mathcal{B}$\;
}
\Return $\theta_{p} = \theta_{p}^{(T)}$\;
\caption{Meta-Weighted Generator Tuning.}
\label{alg:meta}
\end{algorithm}
\paragraph{Meta Objective.}
The general idea of meta-learning is to formulate a meta objective to enable automatic learning of hyperparameters. 
When $\bs{w}$ is a learnable variable, the optimal $\theta_{p_l}^*(\bs{w})={\arg\min}_{\theta_{p_l}}\mathcal{L}_{\text{w-gen}}(\theta_{p_l};\bs{w})$ will be a function of $\bs{w}$.
Since our goal is to encourage label discriminativeness, we require the optimal $\theta_{p_l}^*(\bs{w})$ obtained under $\bs{w}$ to minimize $\mathcal{L}_{\text{disc}}$ as the meta objective:
\begin{equation*}
\label{eq:meta_disc}
\min_{\bs{w}}\mathcal{L}_{\text{disc}},\quad \mathcal{L}_{\text{disc}}(\theta_{p}^*(\bs{w})) = -\frac{1}{n}\sum_{j=1}^n \log p_{\theta_{p}^*(\bs{w})}(y_l|\bs{x}_{\le j}).
\end{equation*}

\paragraph{Training Setup and Algorithm.}
The weight $\bs{w}$ needs to characterize the discriminativeness of each token and thus we parameterize it via a weighting network $\omega$ taking the value of $\mathcal{L}_{\text{disc}}$ as input:
$$
w_j(\omega) = \frac{\exp\left(\omega(\mathcal{L}_{\text{disc}}^j)\right)}{\sum_{{j'}=1}^n \exp\left(\omega(\mathcal{L}_{\text{disc}}^{j'})\right)},\quad
\mathcal{L}_{\text{disc}}^j = -\log p_{\theta_{p}}(y_l|\bs{x}_{\le j}).
$$
Following \cite{Shu2019MetaWeightNetLA}, we instantiate $\omega$ to be a feedforward network with only one $100$-dimension hidden layer.
Instead of solving the optimal $\omega^*$ and $\theta_{p}^*$ via nested optimization loops, we use an online optimization strategy~\citep{Shu2019MetaWeightNetLA} for training efficiency.
It also guarantees convergence to the critical points of both $\mathcal{L}_{\text{w-gen}}$ and $\mathcal{L}_{\text{disc}}$ under mild conditions.
Similar to the observations in \cite{Li2021PrefixTuningOC}, we find it beneficial to initialize the prefix vectors $\theta_{p}$ using task-descriptive tokens. 
The initialization prompts can be found in Appendix~\ref{app:impl_details}.
The overall training procedure is shown in Algorithm~\ref{alg:meta}.


\subsection{Classifier Fine-Tuning}

With the trained generator $G_{\theta_{p}}$, we can synthesize novel training samples $\mathcal{D}_{\text{gen}}$ that augment $\mathcal{D}_{\text{train}}$ for fine-tuning a PLM $C_{\phi}$ for classification.
The major challenge to effectively leverage $\mathcal{D}_{\text{gen}}$ is that the label noise (\ie, some generated samples may not accurately pertain to the corresponding label) may deteriorate model performance if standard supervised learning is directly used.
We propose a simple noise-robust training procedure to improve the generalization and stability of training: First fine-tune $C_{\phi}$ on $\mathcal{D}_{\text{train}}$ with standard supervised training, and then continue fine-tuning it on $\mathcal{D}_{\text{gen}}$ by applying \emph{label smoothing}~\citep{Szegedy2016RethinkingTI} and \emph{temporal ensembling}~\citep{Laine2017TemporalEF} as regularization.
% For temporal ensembling, we record the model predictions $p_{\phi}(\bs{x})$ of $C_{\phi}$ on each training sample $(\bs{x}, y)$ at different training steps, and use the accumulated moving-average predictions $\bar{\bs{z}}$ to regularize the latest model training. 
% We update ensembled predictions $\bar{\bs{z}}$ once every $B$ batches:
% \begin{equation}
% \label{eq:udpate_ens}
% \hat{\bs{z}} \gets \gamma\hat{\bs{z}} + (1-\gamma)\bs{p}_{\phi}, \, \bar{\bs{z}} \gets \hat{\bs{z}}/(1-\gamma^t),
% \end{equation}
% where $\hat{\bs{z}}$ has a zero initialization; $\gamma$ is the momentum parameter; $t$ is the number of updates $\bar{\bs{z}}$ has received; the division $(1-\gamma^t)$ is for bias correction~\citep{Laine2017TemporalEF}. 
Specifically, given a training sample $(\tilde{\bs{x}}, \tilde{y}) \in \mathcal{D}_{\text{gen}}$, we minimize the following classification loss:
\begin{equation}
\label{eq:finetune}
\min_{\phi}\mathcal{L}_{\text{class}},\,\, \mathcal{L}_{\text{class}}(\phi) = -\sum_{l=1}^{L} q_l \log(p_{\phi}(\tilde{\bs{x}})_l) - \lambda \sum_{l=1}^L \bar{z}_l \log \frac{p_{\phi}(\tilde{\bs{x}})_l}{\bar{z}_l},
\end{equation}
where $q_l = \mathbbm{1}(l = \tilde{y})(1-\epsilon) + \epsilon/L$ and $\epsilon$ is the label smoothing weight; $p_{\phi}(\tilde{\bs{x}})$ is the model prediction on $\tilde{\bs{x}}$; $\lambda$ is a regularization weight for temporal ensembling; and $\bar{\bs{z}}$ is the accumulated moving-average model predictions. 
We also use the ensembled prediction $\bar{\bs{z}}$ to filter out noisy synthesized samples: We only include those samples for training where $\bar{\bs{z}}$ strongly agrees with the label $\tilde{y}$ (\ie, $\bar{z}_{\tilde{y}} > \delta$ where $\delta>0$ is a threshold parameter).
In Eq.~\eqref{eq:finetune}, the first classification term is the cross-entropy loss with smoothed labels;
the second regularization term corresponds to temporal ensembling, which requires the current model prediction to be close to its past accumulated predictions. 
This not only neutralizes the fluctuation in model predictions for better training stability when label noise is present~\citep{Nguyen2020SELFLT} but also helps prevent catastrophic forgetting~\citep{kirkpatrick2017overcoming} of the information learned previously from the few-shot training set $\mathcal{D}_{\text{train}}$.
Please refer to Appendix~\ref{app:impl_details} for details about the temporal ensembling implementation.
The overall procedure of classifier fine-tuning is summarized in Algorithm~\ref{alg:classification}.

\SetKwInput{KwInput}{Input}
\SetKwInput{KwParameter}{Parameter}
\SetKwInput{KwOutput}{Output}
\newcommand\mycommfont[1]{\footnotesize\ttfamily\textcolor{blue}{#1}}
\SetCommentSty{mycommfont}
\begin{algorithm}[!t]
\DontPrintSemicolon
\SetNoFillComment
\KwInput{$\mathcal{D}_{\text{train}}$: Few-shot training set; $\mathcal{D}_{\text{gen}}$: Synthesized training set.}
\KwParameter{$T$: Number of training steps.}
\KwOutput{$\phi$: Trained classification model parameters.}
$\phi^{(0)} \gets$ Train on $\mathcal{D}_{\text{train}}$ with standard supervised learning

$\bar{\bs{z}} \gets \bs{0}$\; \tcp*[l]{Ensembled prediction initialization}

\For{$t \in [0, 1, \dots, T-1]$}    
{ 
    $\mathcal{B} \gets \text{Sample a minibatch from }\mathcal{D}_{\text{gen}}$\;
    
    $\phi^{(t+1)} \gets$ Take one gradient step to descend $\mathcal{L}_{\text{class}}$ in Eq.~\eqref{eq:finetune}
    on $\mathcal{B}$\;
    
    $\bar{\bs{z}} \gets$ Accumulate the current model prediction\;
    
    Update $\mathcal{D}_{\text{gen}}$ to exclude noisy samples based on $\bar{\bs{z}}$\;
}
\Return $\phi = \phi^{(T)}$\;
\caption{Classification model fine-tuning on $\mathcal{D}_{\text{train}}$ and $\mathcal{D}_{\text{gen}}$.}
\label{alg:classification}
\end{algorithm}
% \vspace{-1em}