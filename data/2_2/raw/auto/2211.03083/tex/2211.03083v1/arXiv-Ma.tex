%\documentclass[reprint,twocolumn,aps,floatfix,prb,showpacs]{revtex4-1}

%\documentclass[reprint,aps,prb,twocolumn,showpacs]{revtex4}

\documentclass[reprint,twocolumn,aps,prb,showpacs]{revtex4-1}

%\documentclass[preprint,floatfix,showpacs]{revtex4-1}

%\documentclass[prprint,showpacs]{revtex4-1}

% Useful packages
\usepackage{amsmath}
\usepackage{amssymb}
\usepackage{graphicx}
\usepackage{dcolumn}
\usepackage{bm}
\usepackage[colorlinks=true, allcolors=blue]{hyperref}
\usepackage{epstopdf}
\usepackage[utf8]{inputenc}
%\bibliographystyle{apsrev4-1}
\usepackage{amssymb}

\begin{document}

\title{T-linear resistivity in the strange-metal phase of cuprate superconductors due to
umklapp scattering from a spin excitation}

\author{Xingyu Ma}
\thanks{These authors contributed equally to this work}
\author{Minghuan Zeng}
\thanks{These authors contributed equally to this work}
\author{Zhangkai Cao}
\author{Shiping Feng}
\thanks{Corresponding author. E-mail address: spfeng@bnu.edu.cn}

\affiliation{Department of Physics, Beijing Normal University, Beijing 100875, China}

%\date{}

\begin{abstract}
The strange-metal phase of cuprate superconductors exhibits a linear in temperature resistivity,
however, the origin of this remarkable anomaly is still not well understood. Here the linear
temperature dependence of the electrical resistivity in the strange-metal phase of cuprate
superconductors is investigated from the underdoped to overdoped regimes. The momentum
dependence of the transport scattering rate arising from the umklapp scattering between
electrons by the exchange of the spin excitation is derived and employed to calculate the
electrical resistivity by making use of the Boltzmann equation. It is shown that the antinodal
umklapp scattering leads to the linear in temperature resistivity in the low-temperature with
the temperature linear coefficient that decreases with the increase of the doping concentration,
however, the nodal umklapp scattering induces a deviation from the linear in temperature
resistivity in the far lower temperature, and then the quadratic in temperature resistivity in
the far lower temperature is generated by both the antinodal and nodal umklapp scattering. The
theory also shows that the same spin excitation that acts like a bosonic glue to hold the
electron pairs together also mediates scattering of electrons in the strange-metal phase of
cuprtae superconductors responsible for the linear in temperature resistivity and the
associated electronic structure.
\end{abstract}

\pacs{74.25.Fy, 74.25.Nf, 74.20.Mn, 74.72.-h}

\maketitle

\section{Introduction}\label{Introduction}

The parent compound of cuprate superconductors is identified as a Mott insulator
\cite{Fujita12}, in which the absence of the electronic conduction is due to the strong
electron correlation. Superconductivity then emerges when charge carriers are doped into this
Mott insulator \cite{Bednorz86}. At the temperature above the superconducting (SC) transition
temperature $T_{\rm c}$, the electron is in a normal-state. Although this same strong electron
correlation that leads to the Mott insulating state persists into the doped regime, the
normal-state retains a metallic character \cite{Bednorz86}. However, this normal-state is not
normal at all, since this normal-state exhibits a number of the anomalous properties
\cite{Vishik18,Campuzano04,Damascelli03,Fink07,Keimer15,Hussey08,Timusk99,Kastner98} in the
sense that they do not fit in with the standard Landau-Fermi liquid theory
\cite{Schrieffer64,Abrikosov88,Mahan81}. This is why the phase of cuprate superconductors
above $T_{\rm c}$ is so-called as {\it the strange-metal phase}.

Among the fascinating phenomena in the strange-metal phase, the most distinct features are
(i) the low-energy electronic structure associated with the single-particle scattering rate
\cite{Campuzano04,Damascelli03,Fink07}, including the formation of the disconnected Fermi arcs
due to the collapse of the closed electron Fermi surface (EFS)
\cite{Norman98,Shi08,Sassa11,Comin14,Horio16,Loret18}, the peak-dip-hump (PDH) structure in
the electron excitation spectrum \cite{Dessau91,Norman97,Campuzano99,Wei08,DMou17}, and the
dispersion kink \cite{Kaminski01,Zhou03,Anzai10,He13,Yang19}, and (ii) the low-temperature
electrical transport associated with the transport scattering rate
\cite{Keimer15,Hussey08,Timusk99,Kastner98}. In particular, in the early experimental
measurements \cite{Allen89,Gurvitch87,Takagi92}, it was observed that the variation of the
electrical resistivity near the optimal doping is linear with temperature \cite{Takagi92},
extending to low temperatures of a few kelvin and extrapolating to zero resistivity at zero
temperature. This remarkable behaviour of the low-temperature linear temperature dependence of
the electrical resistivity is in a striking contrast to the behaviour in conventional metals
\cite{Schrieffer64,Abrikosov88,Mahan81}, where the low-temperature electrical resistivity
follows one of several simple power laws, and if the electron-electron scattering dominates,
then the electrical resistivity decreases quadratically as the temperature decreases to zero.
In the latter, this linear in temperature (T-linear) resistivity was detected experimentally
in a wide doping range from the underdoped to overdoped regimes
\cite{Martin90,Mandrus92,Ando01,Daou09,Cooper09,Legros19,Yuan22}. Moreover, the suppression
of superconductivity with a magnetic field reveals that the T-linear resistivity persists down
essentially to the zero temperature limit \cite{Ayres21}. Recently, the systematic experimental
results in the heavily overdoped regime yielded the low-temperature T-linear
resistivity all the way up to the edge of the SC dome \cite{Legros19,Yuan22,Ayres21,Grisso21}.
After intensive investigations over more than three decades, it has now become clear that the
long-standing T-linear resistivity
\cite{Allen89,Gurvitch87,Takagi92,Martin90,Mandrus92,Ando01,Daou09,Cooper09,Legros19,Yuan22,Ayres21,Grisso21}
and exotic electronic structure
\cite{Norman98,Shi08,Sassa11,Comin14,Horio16,Loret18,Dessau91,Norman97,Campuzano99,Wei08,DMou17,Kaminski01,Zhou03,Anzai10,He13,Yang19}
are the generic features in the strange-metal phase of cuprate superconductors.
In this case, a key question posed by these experimental observations is raised: is there a
common bosonic excitation that is responsible for pairing the electrons also dominantly
scatters electrons in the strange-metal phase responsible for the T-linear resistivity and
the associated electronic structure?

Although the low-temperature T-linear resistivity in the strange-metal phase of cuprate
superconductors is well established by now
\cite{Allen89,Gurvitch87,Takagi92,Martin90,Mandrus92,Ando01,Daou09,Cooper09,Legros19,Yuan22,Ayres21,Grisso21},
its origin remains the subject of the active research and debate. Theoretically, several
scenarios have been proposed for the origin of the T-linear resistivity
\cite{Varma89,Varma16,Varma20,Damle97,Sachdev11,Zaanen04,Luca07,Haldane18,Zaanen19,Hussey03,Rice17,Lee21}.
In particular, in the marginal Fermi-liquid phenomenology \cite{Varma89,Varma16,Varma20}, a
single T-linear scattering rate is introduced responsible for the T-linear resistivity.
Moreover, it has been postulated that the T-linear behaviour can be attributed to the
strongly interacting critical state anchored at a quantum critical point where a phase
transition is tuned to zero temperature \cite{Varma20,Damle97,Sachdev11}. With the close
relation to the physics of the quantum critical point, the T-linear resistivity has been
interpreted in terms of the Planckian dissipation \cite{Zaanen04,Luca07,Haldane18,Zaanen19}
in which the relaxation-time achieves a putative universal minimum value, irrespective of
the underlying mechanisms. On the other hand, it has been argued that the elastic umklapp
scattering processes that directly transfer momentum between the electron sea and the
underlying square lattice lead to the resistivity linear in temperature in the strange-metal
phase \cite{Hussey03,Rice17}. More specifically, it has been shown recently that the electrical
resistance arises from the electron umklapp scattering by the exchange of a critical bosonic
mode \cite{Lee21}, where the resistivity is characterized by a highly anisotropic scattering
rate. This highly anisotropic scattering rate is T-linear near the umklapp points and becomes
quadratic in temperature as one moves away from the umklapp point, which therefore leads to a
T-linear resistivity in the low-temperature regime and quadratic in temperature resistivity in
the far lower temperature regime \cite{Lee21}. These studies \cite{Hussey03,Rice17,Lee21} and
many others \cite{Honerkamp01,Hartnoll12,Tabis21} therefore indicate that the umklapp
scattering dominates the momentum-relaxation mechanism of the electrical transport. However,
up to now, the origin of the T-linear resistivity and of its connection to the unconventional
electronic structure have not been discussed starting from a microscopic theory, and no
explicit calculations of the doping dependence of the T-linear resistivity has been made so
far. Superconductivity with the highest $T_{\rm c}$ emerges directly as an instability of the
strange-metal phase, and it thus has long been recognized that the understanding of the
essential physics of the strange-metal phase is crucial for the understanding of the mystery
of the unconventional superconductivity.

In the recent works within the framework of the kinetic-energy-driven superconductivity, we
have studied the low-energy electronic structure of cuprate superconductors both in the SC
phase \cite{Liu21,Cao21,Zeng22} at the temperature below $T_{\rm c}$ and strange-metal phase
\cite{Feng16} at the temperature above $T_{\rm c}$, where the electron normal self-energy in
the particle-hole channel and electron anomalous self-energy in the particle-particle channel
are generated by the coupling of the electrons with the spin excitations. In particular, the
electrons are renormalized by the electron normal self-energy, and then all the exotic features
of the low-energy electronic structure arise from this renormalization of the electrons
\cite{Liu21,Cao21,Zeng22,Feng16}. In this paper, we start from this low-energy electronic
structure in the strange-metal phase of cuprate superconductors \cite{Feng16} to study the
nature of the doping dependence of the T-linear resistivity, where the angular dependence of the
transport scattering rate arising from the umklapp scattering between electrons by the exchange
of the same spin excitations is derived and employed to calculate the electrical resistivity in
terms of the Boltzmann equation. Our results show that the antinodal umklapp scattering, in
which the electron at around the antinodal region of EFS is scattered by its umklapp partner at
around the antinodal region of the neighboring EFS, leads to the T-linear resistivity in the
low-temperature with the T-linear coefficient that decreases with the increase of the doping
concentration. On the other hand, the nodal umklapp scattering, in which the electron at around
the nodal region of EFS is scattered by its umklapp partner at around the nodal region of the
neighboring EFS, induces a deviation from the T-linear resistivity in the far lower temperature,
and then the quadratic in temperature resistivity in the far lower temperature is generated by
both the antinodal and nodal umklapp scattering. Our results therefore indicate that the same
spin excitation that is responsible for pairing the electrons also mediates the scattering of
electrons in the strange-metal phase responsible for the T-linear resistivity and the
associated electronic structure.

The rest of this paper is organized as follows. In Section \ref{Formalism}, we begin by a short
summary of the unconventional features of the low-energy electronic structure due to the
interaction between electrons by the exchange of a strongly dispersive spin excitation, and
then within the framework of the Boltzmann transport theory, we formulate the essential aspects
of the electron-electron umklapp scattering between a circular EFS and its umklapp partner by
the exchange of the same spin excitation for deriving the electrical resistivity. In Section
\ref{electron-resistivity}, the Boltzmann equation is employed to study the doping dependence
of the low-temperature electrical resistivity, where we show that the angular dependence of the
transport scattering rate presents a similar behavior of the single-particle scattering rate,
and is largest at around the antinodes and smallest at around the tips of the Fermi arcs.
Although this angular dependence of the transport scattering rate is highly anisotropic in
momentum-space, it displays a T-linear behaviour in the low-temperature for all directions.
Finally, we give a summary in Section \ref{summary}. In the Appendix
\ref{electron-electron-collision}, we presents the details of the derivation of the
electron-electron collision term in the Boltzmann equation.

%\newpage

\section{Theory}\label{Formalism}

\subsection{$t$-$J$ model in the fermion-spin representation} \label{model-constraint}

The crystal structure of cuprate superconductors is characterized by the square-lattice
copper-oxide layers \cite{Fujita12,Bednorz86,Vishik18,Campuzano04,Damascelli03,Fink07}, which
are sometimes considered to contain all the essential physics
\cite{Anderson87,Yu92,Lee06,Edegger07,Spalek22}. Immediately after the discovery of
superconductivity in cuprate superconductors, it was suggested that the fundamental properties
of the doped copper-oxide layer are properly accounted by the square-lattice $t$-$J$ model
\cite{Anderson87},
\begin{eqnarray}\label{tJ-model}
H=-\sum_{ll'\sigma}t_{ll'}C^{\dagger}_{l\sigma}C_{l'\sigma}
+\mu\sum_{l\sigma}C^{\dagger}_{l\sigma}
C_{l\sigma}+J\sum_{\langle ll'\rangle}{\bf S}_{l}\cdot {\bf S}_{l'},~~~~
\end{eqnarray}
acting on the restricted Hilbert-space with no double electron occupancy
$\sum_{\sigma}C^{\dagger}_{l\sigma}C_{l\sigma}\le1$, where the operator $C^{\dagger}_{l\sigma}$
($C_{l\sigma}$) denotes the creation (annihilation) operator of an electron on site $l$ with
spin $\sigma$, ${\bf S}_{l}$ is the spin operator of the electron with its components
$S_{l}^{x}$, $S_{l}^{y}$, and $S_{l}^{z}$, and $\mu$ is the chemical potential. In this paper,
the hopping of the constrained electrons $t_{ll'}$ is restricted to the nearest-neighbor (NN)
sites with the hoping amplitude $t_{ll'}=t$ and next NN sites with the hoping amplitude
$t_{ll'}=-t'$. The summation $\langle ll'\rangle$ indicates a sum over the NN pairs, while the
summation $ll'$ is taken over all the NN and next NN pairs. Throughout this paper, we choose the
parameters as $t/J=2.5$ and $t'/t=0.3$ as in the previous discussions \cite{Feng16}. The
magnitude of $J$ and the lattice constant of the square lattice are the energy and length units,
respectively. However, when necessary to compare with the experimental data, we set $J=1000$ K.

The essence of the strongly correlated physics is reflected in the on-site local constraint of
no double occupancy \cite{Yu92,Lee06,Edegger07,Spalek22,Zhang93}. To avoid the double electron
occupancy, we employ the fermion-spin transformation for the parametrization of the constrained
electron operators $C_{l\uparrow}$ and $C_{l\downarrow}$ as \cite{Feng0494,Feng15},
\begin{eqnarray}\label{CSSFS}
C_{l\uparrow}=h^{\dagger}_{l\uparrow}S^{-}_{l},~~~~
C_{l\downarrow}=h^{\dagger}_{l\downarrow}S^{+}_{l},
\end{eqnarray}
respectively, where the spin operator $S_{l}$ keeps track of the spin degree of freedom of the
constrained electron, while the spinful fermion operator $h_{l\sigma}=e^{-i\Phi_{l\sigma}}h_{l}$
keeps track of the charge degree of freedom of the constrained electron together with some
effects of spin configuration rearrangements due to the presence of the doped hole itself (charge
carrier), and then the on-site local constraint of no double occupancy is satisfied in actual
analyses. In this fermion-spin representation (\ref{CSSFS}), the original $t$-$J$ model
(\ref{tJ-model}) can be rewritten explicitly as,
\begin{eqnarray}\label{cssham}
H&=&\sum_{ll'\sigma}t_{ll'}(h^{\dagger}_{l'\uparrow}h_{l\uparrow}S^{+}_{l}S^{-}_{l'}+
h^{\dagger}_{l'\downarrow}h_{l\downarrow}S^{-}_{l}S^{+}_{l'})-\mu_{\rm h}\sum_{l\sigma}
h^{\dagger}_{l\sigma}h_{l\sigma}\nonumber\\
&+&J_{\rm eff}\sum_{\langle ll'\rangle}{\bf S}_{l}\cdot {\bf S}_{l'},
\end{eqnarray}
where $S^{-}_{l}=S^{\rm x}_{l}-iS^{\rm y}_{l}$ and $S^{+}_{l}=S^{\rm x}_{l}+iS^{\rm y}_{l}$
are the spin-lowering and spin-raising operators for the spin $S=1/2$, respectively,
$J_{{\rm eff}}=(1-\delta)^{2}J$,
$\delta=\langle h^{\dagger}_{l\sigma}h_{l\sigma}\rangle=\langle h^{\dagger}_{l}h_{l}\rangle$
is the doping concentration, and $\mu_{\rm h}$ is the charge-carrier chemical potential. As a
natural consequence, the kinetic-energy term in the $t$-$J$ model (\ref{tJ-model}) has been
transferred as the coupling between charge and spin degrees of freedom of the constrained
electron, which reflects a basic fact that even the kinetic energy term in the $t$-$J$ model
(\ref{tJ-model}) has the strong Coulombic contribution due to the restriction of no double
occupancy at an any given site, and therefore governs the unconventional features of cuprate
superconductors.

\subsection{Coupling of electrons to the strongly dispersive spin excitation}
\label{Effective-propagator}

\begin{figure}[h!]
\centering
\includegraphics[scale=0.75]{self-energy-diagram.pdf}
\caption{The skeletal diagram for the electron normal self-energy for scattering electrons from
the strongly dispersive spin excitation. The solid-line represents the electron propagator $G$,
and the dashed-line depicts the spin propagator $D^{(0)}$, while $\square$ describes the bare
vertex function $\Lambda$. \label{self-energy-diagram}}
\end{figure}

Starting from the $t$-$J$ model (\ref{cssham}) in the fermion-spin representation, we
\cite{Feng15,Feng0306,Feng12,Feng15a} have developed the kinetic-energy-driven
superconductivity, where at low temperatures, the charge carriers are bound into the
charge-carrier pairs with the d-wave symmetry by the attractive interaction in the
particle-particle channel that arises directly from the strong coupling between charge and
spin degrees of freedom of the constrained electron in the kinetic energy of the $t$-$J$
model (\ref{cssham}) by the exchange of the strongly dispersive spin excitation, then the
d-wave electron pairs originated from the d-wave charge-carrier pairs are due to the
charge-spin recombination \cite{Feng15a}, and the condensation of these d-wave electron pairs
reveals the SC-state with the d-wave symmetry. In particular, the doping dependence of
$T_{\rm c}$ exhibits a dome-like shape with the underdoped and overdoped regimes on each side
of the optimal doping, where $T_{\rm c}$ reaches its maximum. Moreover, this same coupling
mediated by the spin excitation that induces the SC-state in the particle-particle channel
also leads to the renormalization of the electrons in the particle-hole channel \cite{Feng16}.
Within the framework of this kinetic-energy-driven superconductivity, the exotic features of
the low-energy electronic structure in the both SC phase \cite{Liu21,Cao21,Zeng22} and
strange-metal phase \cite{Feng16} of cuprate superconductors have been investigated, and the
obtained results are well consistent with the corresponding experimental observations.
Following these previous discussions \cite{Feng16}, the single-particle propagator in the
strange-metal phase of cuprate superconductors can be obtained in the condition of the SC gap
parameter $\bar{\Delta}=0$,
\begin{eqnarray}\label{EGF}
G({\bf k},\omega)={1\over \omega-\varepsilon_{\bf k}-\Sigma_{\rm ph}({\bf k},\omega)},
\end{eqnarray}
where $\varepsilon_{\bf k}=-4t\gamma_{\bf k}+4t'\gamma_{\bf k}'+\mu$ is the electron energy 
dispersion in the tight-binding approximation, with 
$\gamma_{\bf k}=({\rm cos}k_{x}+{\rm cos} k_{y})/2$ and
$\gamma_{\bf k}'={\rm cos}k_{x}{\rm cos}k_{y}$, while the electron normal self-energy
$\Sigma_{\rm ph}({\bf k},\omega)$ sketched in Fig. \ref{self-energy-diagram} is expressed as
\cite{Feng16},
\begin{widetext}
\begin{eqnarray}\label{ESE}
\Sigma_{\rm ph}({\bf k},i\omega_{n})&=&{t^{2}\over N^{2}}\sum_{{\bf p,q}}
\Lambda^{2}_{{\bf p}+{\bf q}+{\bf k}}{1\over \beta}\sum_{ip_{m}}
G({\bf p}+{\bf k},ip_{m}+i\omega_{n})\Pi({\bf p},{\bf q},ip_{m})
={t^{2}\over N}\sum_{{\bf p}}{1\over \beta}\sum_{ip_{m}}G({\bf p}+{\bf k},ip_{m}+i\omega_{n})
P^{(0)}({\bf k},{\bf p},ip_{m})\nonumber\\
&=&2\int^{\infty}_{-\infty}{{\rm d}\omega'\over 2\pi}
\int^{\infty}_{-\infty}{{\rm d}\omega''\over 2\pi}{n_{\rm B}(\omega'')+n_{\rm{F}}(\omega')
\over\omega''-\omega'+i\omega_{n}}{t^{2}\over N}\sum_{\bf p}A({\bf p}+{\bf k},\omega'){\rm Im}
P^{(0)}({\bf k},{\bf p},\omega''),~~~~
\end{eqnarray}
\end{widetext}
where $N$ is the number of lattice sites,
$\Lambda_{{\bf k}}=4\gamma_{\bf k}-4(t'/t)\gamma_{\bf k}'$ is the bare vertex function,
$\omega_{n}$ and $p_{m}$ are the fermionic and bosonic Matsubara frequencies, respectively,
$n_{\rm B}(\omega)$ and $n_{\rm F}(\omega)$ are the boson and fermion distribution functions,
respectively, ${\rm Im}P^{(0)}({\bf k},{\bf p},\omega)$ is the imaginary part of
$P^{(0)}({\bf k},{\bf p},\omega)$, while $P^{(0)}({\bf k},{\bf p},\omega)$ is so-called as
the effective spin propagator, which describes the nature of the spin excitation, and can be
expressed as,
\begin{eqnarray}\label{ESP-1}
P^{(0)}({\bf k},{\bf p},\omega)={1\over N}\sum_{\bf q}\Lambda^{2}_{{\bf p}+{\bf q}+{\bf k}}
\Pi({\bf p},{\bf q},\omega),
\end{eqnarray}
with the spin bubble $\Pi({\bf p},{\bf q},\omega)$, which is a convolution of two spin
propagators, and has been obtained as \cite{Feng16},
\begin{eqnarray}\label{spin-bubble-1}
\Pi({\bf p},{\bf q},ip_{m})={1\over\beta}\sum_{iq_{m}}D^{(0)}({\bf q},iq_{m})
D^{(0)}({\bf q}+{\bf p},iq_{m}+ip_{m}),~~~~
\end{eqnarray}
where $q_{m}$ is the bosonic Matsubara frequency, while the spin propagator
$D^{(0)}({\bf k},\omega)$ in the mean-field (MF) level has been evaluated as,
\begin{eqnarray}\label{SGF-1}
D^{(0)}({\bf k},\omega)={B_{\bf k}\over\omega^{2}-\omega^{2}_{\bf k}},
\end{eqnarray}
with the MF spin excitation energy dispersion $\omega_{\bf k}$ and the weight function of
the spin excitation spectrum $B_{\bf k}$ that are strongly dispersive, and have been given
in Ref. \onlinecite{Feng15}. Substituting the above MF spin propagator (\ref{SGF-1}) into
Eq. (\ref{spin-bubble-1}), the spin bubble $\Pi({\bf p},{\bf q},\omega)$ can be derived as,
\begin{eqnarray}\label{spin-bubble}
\Pi({\bf p},{\bf q},\omega)=-{\bar{W}^{(1)}_{{\bf p}{\bf q}}\over\omega^{2}
-[\omega^{(1)}_{{\bf p}{\bf q}}]^{2}}+{\bar{W}^{(2)}_{{\bf p}{\bf q}}\over\omega^{2}
-[\omega^{(2)}_{{\bf p}{\bf q}}]^{2}},~~~~
\end{eqnarray}
and then the effective spin propagator $P^{(0)}({\bf k},{\bf p},\omega)$ in
Eq. (\ref{ESP-1}) is obtained directly from the above spin bubble (\ref{spin-bubble}), where
$\omega^{(1)}_{{\bf p}{\bf q}}=\omega_{{\bf q}+{\bf p}}
+\omega_{\bf q}$ and $\omega^{(2)}_{{\bf p}{\bf q}}=\omega_{{\bf q}+{\bf p}}-\omega_{\bf q}$,
and the weight functions of the effective spin excitation spectrum,
\begin{eqnarray}
\bar{W}^{(1)}_{{\bf p}{\bf q}}&=&{B_{\bf q}B_{{\bf q}+{\bf p}}\over 2\omega_{\bf q}
\omega_{{\bf q}+{\bf p}}}\omega^{(1)}_{{\bf p}{\bf q}}[n_{\rm B}(\omega_{{\bf q}+{\bf p}})
+n_{\rm B}(\omega_{\bf q})+1], ~~~~~\\
\bar{W}^{(2)}_{{\bf p}{\bf q}}&=&{B_{\bf q}B_{{\bf q}+{\bf p}}\over 2\omega_{\bf q}
\omega_{{\bf q}+{\bf p}}}\omega^{(2)}_{{\bf p}{\bf q}}[n_{\rm B}(\omega_{{\bf q}+{\bf p}})
-n_{\rm B}(\omega_{\bf q})].
\end{eqnarray}

With the above effective spin propagator (\ref{ESP-1}), the electron normal self-energy
$\Sigma_{\rm ph}({\bf k},\omega)$ in Eq. (\ref{ESE}) has been derived \cite{Feng16}, and can
be expressed explicitly as,
\begin{eqnarray}\label{ESE-1}
\Sigma_{\rm ph}({\bf k},\omega)&=&{t^{2}\over N^2}\sum_{{\bf pq}\alpha=1,2}(-1)^{\alpha+1}
\Omega_{\bf kpq}\nonumber\\
&\times& \left( {F^{(\alpha)}_{1{\bf kpq}}\over\omega+\omega^{(\alpha)}_{\bf pq}
-\bar\varepsilon_{{\bf p}+{\bf k}}}+{F^{(\alpha)}_{2{\bf kpq}}\over\omega
-\omega^{(\alpha)}_{\bf pq}-\bar\varepsilon_{{\bf p}+{\bf k}}} \right),~~~~~
\end{eqnarray}
with the renormalized electron energy dispersion
$\bar\varepsilon_{\bf k}=Z_{\rm F}\varepsilon_{\bf k}$,
$\Omega_{\bf kpq}=Z_{\rm F}\Lambda^{2}_{{\bf p}+{\bf q}+{\bf k}}B_{\bf q}B_{{\bf q}+{\bf p}}
/(4\omega_{\bf q}\omega_{{\bf q}+{\bf p}})$, and the functions,
\begin{subequations}
\begin{eqnarray}
F^{(\alpha)}_{1{\bf kpq}}&=&n_{\rm F}(\bar\varepsilon_{{\bf p}+{\bf k}})
\{1+n_{\rm B}(\omega_{{\bf q}+{\bf p}})+n_{\rm B}[(-1)^{\alpha+1}\omega_{\bf q}]\}\nonumber\\
&+&n_{\rm B}(\omega_{{\bf q}+{\bf p}})n_{\bf B}[(-1)^{\alpha+1}\omega_{\bf q}],\\
F^{(\alpha)}_{2{\bf kpq}}&=&[1-n_{\rm F}(\bar\varepsilon_{{\bf p}+{\bf k}})]
\{1+n_{\rm B}(\omega_{{\bf q}+{\bf p}})+n_{\rm B}[(-1)^{\alpha+1}\omega_{\bf q}]\}\nonumber\\
&+&n_{\rm B}(\omega_{{\bf q}+{\bf p}})n_{\rm B}[(-1)^{\alpha+1}\omega_{\bf q}],
\end{eqnarray}
\end{subequations}
while the single-particle coherent weight $Z_{\rm F}$ that has been given explicitly in
Ref. \onlinecite{Feng16}. In particular, the sharp peaks visible for low-temperature in
$\Sigma_{\rm ph}({\bf k},\omega)$ and $P^{(0)}({\bf k},{\bf p},\omega)$ are actually a
$\delta$-function, broadened by a small damping used in the numerical calculation for a finite
lattice. The calculation in this paper for $\Sigma_{\rm ph}({\bf k},\omega)$ and
$P^{(0)}({\bf k},{\bf p},\omega)$ is performed numerically on a $160\times 160$ lattice in
momentum space, with the infinitesimal $i0_{+}\rightarrow i\Gamma$ replaced by a small damping
$\Gamma=0.1J$.

\subsection{Electron Fermi surface}\label{Octet-model}

The single-particle spectrum function $A({\bf k},\omega)$ now can be obtained directly from
the above single-particle propagator ({\ref{EGF}}) as,
\begin{eqnarray}\label{ESF}
A({\bf k},\omega)&=&-2{\rm Im}G({\bf k},\omega)={2\Gamma_{\bf k}(\omega)\over
[\omega-\bar{E}_{\bf k}(\omega)]^{2}+\Gamma^{2}_{\bf k}(\omega)},~~~~~~
\end{eqnarray}
with the corresponding single-particle scattering rate $\Gamma_{\bf k}(\omega)$ and
renormalized band structure $\bar{E}_{\bf k}(\omega)$,
\begin{subequations}
\begin{eqnarray}
\Gamma_{\bf k}(\omega)&=& |{\rm Im}\Sigma_{\rm ph}({\bf k},\omega)|, ~~~~~
\label{SPSR}\\
\bar{E}_{\bf k}(\omega)&=& \varepsilon_{\bf k}+{\rm Re}\Sigma_{\rm ph}({\bf k},\omega),~~~~~
\label{RBS}
\end{eqnarray}
\end{subequations}
where ${\rm Re}\Sigma_{\rm ph}({\bf k},\omega)$ and ${\rm Im}\Sigma_{\rm ph}({\bf k},\omega)$
are the real and imaginary parts of the electron normal self-energy
$\Sigma_{\rm ph}({\bf k},\omega)$, respectively.

\begin{figure}[h!]
\centering
\includegraphics[scale=0.77]{EFS-map.pdf}
\caption{(Color online) (a) The map of the electron Fermi surface and (b) the surface plot of
the electron spectral function for zero energy $\omega=0$ at $\delta=0.15$ with $T=0.002J$,
where the zone center has been shifted by [$\pi,\pi$], and AN, TFA, and ND denote the antinode,
tip of the Fermi arc, and node, respectively. \label{EFS-map}}
\end{figure}

The shape of EFS has deep consequences for the low-energy electronic properties
\cite{Vishik18,Campuzano04,Damascelli03,Fink07,Keimer15}, and has been also central to
addressing electrical transport \cite{Hussey08,Timusk99,Kastner98}. In the previous studies
\cite{Feng16}, the topology of EFS in the strange-metal phase of cuprate superconductors has
been discussed in terms of the intensity map of the single-particle spectral function
(\ref{ESF}) at zero energy $\omega=0$, where it has been shown that the highly anisotropic
momentum dependence of the single-particle scattering rate $\Gamma_{\bf k}(\omega)$ induces a
strong redistribution of the spectral weights on EFS. For a convenience in the following
discussions of the electrical transport, we plot (a) the EFS map and (b) the surface plot of
the single-particle spectral function $A({\bf k},\omega)$ for zero energy $\omega=0$ at doping
$\delta=0.15$ with temperature $T=0.002J$ in Fig. \ref{EFS-map}, where the Brillouin zone (BZ)
center has been shifted by [$\pi,\pi$], and AN, TFA, and ND indicate the antinode, tip of the
Fermi arc, and node, respectively. The most noteworthy in Fig. \ref{EFS-map} are the following:
(i) the spectral weight at around the antinodal region is suppressed strongly, reflecting a
basic fact that EFS at around the antinodal region can not be observed experimentally
\cite{Norman98,Shi08,Sassa11,Comin14,Horio16,Loret18}; (ii) the spectral weight at around the
nodal region is suppressed modestly, leading to the formation of the disconnected Fermi arcs
\cite{Norman98,Shi08,Sassa11,Comin14,Horio16,Loret18}; (iii) however, almost all the spectral
weight on the Fermi arcs is assembled at around the tips of the Fermi arcs
\cite{Norman98,Shi08,Sassa11,Comin14,Horio16,Loret18}. In other words, the electrons at around
the tips of the Fermi arcs have the largest density of states, and then the low-energy
electronic properties are largely governed by these electrons at around the tips of the Fermi
arcs. In particular, it has been observed experimentally that these characteristic features
shown in Fig. \ref{EFS-map} in the zero energy case can persist into the case for a finite
binding-energy \cite{Chatterjee06,He14}. More importantly, the suppression of the spectral
weight at around the antinodal and nodal regions can affect the electrical transport in two
ways \cite{Timusk99}: through the reduction of the number of current-carrying states, and
secondly, through the reduction in the density of electron excitations at around the antinodal
and nodal regions.

\begin{figure}[h!]
\centering
\includegraphics[scale=0.9]{single-particle-scattering.pdf}
\caption{The single-particle scattering rate $\Gamma(\theta)$ as a function of Fermi angle
$\theta$ at $\delta=0.15$ with $T=0.05J$ for $\omega=0$.
\label{single-particle-scattering}}
\end{figure}

In our previous discussions \cite{Feng16}, it has been shown that the origin of the spectral
redistribution to form the Fermi arcs can be attributed to the highly anisotropic momentum
dependence of the single-particle scattering rate $\Gamma_{\bf k}(\omega)$ in Eq. (\ref{SPSR}).
The EFS contour in momentum space is determined directly by the poles of the single-particle
propagator (\ref{EGF}) at zero energy: $\bar{E}_{{\bf k}_{\rm F}}(0)
=\varepsilon_{{\bf k}_{\rm F}}+{\rm Re}\Sigma_{\rm ph}({\bf k}_{\rm F},0)=0$, and then the
spectral weight of the single-particle spectral function $A({\bf k}_{\rm F},0)$ in
Eq. (\ref{ESF}) at EFS is dominantly governed by the inverse of the single-particle scattering
rate $1/\Gamma(\theta)$, where $\Gamma(\theta)$ is defined as
$\Gamma(\theta)=\Gamma_{{\rm k}_{\rm F}(\theta)}(0)$, and $\theta$ is the Fermi angle. To see
this highly anisotropic $\Gamma(\theta)$ in momentum space more clearly, we plot the angular
dependence of $\Gamma(\theta)$ along EFS from the antinode to the node at $\delta=0.15$ with
$T=0.05J$ in Fig. \ref{single-particle-scattering}, where the actual minimum of $\Gamma(\theta)$
does not appear at around the nodal region, but resides exactly at around the tip of the Fermi
arc. However, the maximal $\Gamma(\theta)$ appears at around the antinodal region, and then
$\Gamma(\theta)$ decreases when the Fermi angle is moved away from the antinode. In particular,
$\Gamma(\theta)$ at around the nodal region is smaller than that around the antinodal region.
This angular dependence of $\Gamma(\theta)$ therefore leads to the spectral redistribution to
form the Fermi arcs with almost all the spectral weight inhabited at around the tips of the
Fermi arcs.

With the help of the above single-particle spectrum function (\ref{ESF}), we have also studied
the electronic structure in the strange-metal phase of cuprate superconductors \cite{Feng16},
where the main results can be summarized as: (i) charge order is driven by the EFS instability
shown in Fig. \ref{EFS-map}, with a characteristic wave vector corresponding to the tips of the
Fermi arcs \cite{Comin14,Comin16}. In particular, the tips of the Fermi arcs connected by the
scattering wave vectors ${\bf q}_{i}$ construct an {\it octet} scattering model, which has been
identified as a fundamental scattering model in the interpretation of the experimental data
observed from Fourier transform scanning tunneling spectroscopy \cite{Yin21,Pan01,Fujita19},
and has been also used to explain the regions of the highest joint density of states detected
from angle-resolved photoemission spectroscopy (ARPES) \cite{Chatterjee06,He14}; (ii) the
single-particle scattering rate $\Gamma_{\bf k}(\omega)$ exhibits a well-pronounced peak
structure at around the antinodal and nodal regions, which is directly responsible for the
remarkable PDH structure in the energy distribution curve
\cite{Dessau91,Norman97,Campuzano99,Wei08,DMou17}; (iii) the dispersion kink
\cite{Kaminski01,Zhou03,Anzai10,He13,Yang19} is always accompanied by the corresponding
inflection point in ${\rm Re}\Sigma_{\rm ph}({\bf k},\omega)$, while the spectral weight at
around the dispersion kink is reduced strongly by the corresponding peak in
$\Gamma_{\bf k}(\omega)$; (iv) the electron dispersion at around the antinodal region has an
anomalously small changes of energy as a function of momentum, leading to appearance of the
unusual flat band \cite{Dessau93}. In particular, this flat band is slightly below the Fermi
energy. All these obtained results are well consistent with the corresponding ARPES
experimental observations. These results \cite{Feng16} also show that the same spin excitation
that is responsible for pairing the electrons also dominantly scatters electrons in the
strange-metal phase at the temperature above $T_{\rm c}$ responsible for the unconventional
electronic structure.

\subsection{Bolznman equation}\label{Boltzmann-theory}

Although the magnitude of the single-particle scattering rate $\Gamma(\theta)$ shown in
Fig. \ref{single-particle-scattering} at a given Fermi angle is different from that of the
corresponding transport scattering rate \cite{Varma20}, both the scattering rates may have a
similar behaviour of the angular dependence. In this sense, the result of the angular
dependence of $\Gamma(\theta)$ shown in Fig. \ref{single-particle-scattering} also indicates
that the important electron scattering responsible for the resistivity is mainly concentrated
at the antinodes and nodes. For the discussions of the anomalous transport properties, it needs
to determine how the momentum distribution relaxes in the vicinity of these antinodes and nodes
\cite{Lee21}, which can be done by solving the Boltzmann equation with the input of the
scattering processes. In the Boltzmann transport theory \cite{Abrikosov88,Mahan81}, the
essential behaviour of the electrons is depicted by the distribution function
$f({\bf r},{\bf k},t)$. In this paper, we focus on the dc conductivity in the homogeneous
system only, where the position and time dependence in the distribution function are absent,
and then the distribution function satisfies the following Boltzmann equation
\cite{Abrikosov88,Mahan81},
\begin{eqnarray}\label{Boltzmann-equation-1}
{\partial {\bf k}\over\partial t}\nabla_{\bf k}f({\bf k})
=\left ({df\over dt}\right )_{\rm collisions}
\end{eqnarray}
where the right-hand side is the time rate of change due to the electron-electron collision,
while the factor ${\partial {\bf k}/\partial t}$ is equivalent to an acceleration which is
equal to the forces on the electrons as,
\begin{eqnarray}\label{momentum-equation}
{\partial {\bf k}\over\partial t}=-e{\bf E},
\end{eqnarray}
with the charge $e$, where for a convenience in the following discussions, the magnetic
field has been dropped, i.e., ${\bf H}=0$, while only an electric field ${\bf E}$ is applied
to the system. In this case, we substitute Eq. (\ref{momentum-equation}) into
Eq. (\ref{Boltzmann-equation-1}), and rewrite the Boltzmann equation
(\ref{Boltzmann-equation-1}) as,
\begin{eqnarray}\label{Boltzmann-equation-2}
e{\bf E}\cdot\nabla_{\bf k}f({\bf k})+\left ({df\over dt}\right )_{\rm collisions}=0.
\end{eqnarray}
Following the discussions in Ref. \onlinecite{Prange64}, we now introduce the linear
perturbation from the equilibrium in terms of the distribution function as,
\begin{figure}[h!]
\centering
\includegraphics[scale=0.85]{scatter-process.pdf}
\caption{(Color online) Illustration of the umklapp scattering process in which an electron on
a circular electron Fermi surface (left) is scattered by its partner on the umklapp electron
Fermi surface (right), where the intensity map of the electron Fermi surface is the same as
shown in Fig. \ref{EFS-map}a, while the perfect circle (red) is the circle with the radius
${\rm k}^{\rm TFA}_{\rm F}$ with ${\bf k}^{\rm TFA}_{\rm F}$ that is the Fermi wave vector of
the tips of the Fermi arcs. An electron on the electron Fermi surface (left) parametrized by
the Fermi angle $\theta$ is scattered to a point parametrized by the Fermi angle $\theta'$ on
the umklapp electron Fermi surface (right) by the spin excitation carrying momentum
${\rm p}(\theta,\theta')$. This physical picture is repeated for the other three umklapp
electron Fermi surfaces that are closest to the original electron Fermi surface.
\label{scatter-process}}
\end{figure}
\begin{eqnarray}\label{distribution-function}
f({\bf k})&=&n_{\rm F}({\bar{\varepsilon}_{\bf k}})-{d n_{\rm F}({\bar{\varepsilon}_{\bf k}})
\over d{\bar{\varepsilon}_{\bf k}}}\tilde{\Phi}({\bf k}),
\end{eqnarray}
where $\tilde{\Phi}({\bf k})$ has been interpreted as a local shift of the chemical potential
at a given patch of EFS \cite{Lee21,Prange64}, and satisfies an antisymmetric relation
$\tilde{\Phi}(-{\bf k})=-\tilde{\Phi}({\bf k})$. With the help of the above distribution function
(\ref{distribution-function}), the Boltzmann equation (\ref{Boltzmann-equation-2}) can be
linearized with the result that can be expressed explicitly as,
\begin{eqnarray}\label{Boltzmann-equation-3}
e{\bf v}_{\bf k}\cdot{\bf E}{\partial n_{\rm F}({\bar{\varepsilon}_{\bf k}})\over
\partial {\bar{\varepsilon}_{\bf k}}}=-\left ({df\over dt}\right )_{\rm collisions}
=I_{\rm e-e},
\end{eqnarray}
where ${\bf v}_{\bf k}=\nabla_{\bf k}{\bar{\varepsilon}_{\bf k}}$ is the electron velocity and
$I_{\rm e-e}$ is the electron-electron collision term.

\subsection{Electron umklapp scattering}\label{Umklapp-scattering}

For the evaluation of the electron-electron collision in the Boltzmann equation
(\ref{Boltzmann-equation-3}), the mechanism of the momentum relaxation needs to be introduced
\cite{Abrikosov88,Mahan81}. After intensive investigations over more than three decades,
although the mechanism of the momentum relaxation for the T-linear resistivity still remains
controversial, the electron umklapp scattering is believed to be at the heart of the striking
behaviour of the electrical transport in the strange-metal phase of cuprate superconductors
\cite{Hussey03,Rice17,Lee21,Honerkamp01,Hartnoll12,Tabis21}. In this paper, we adopt the
electron umklapp scattering as the mechanism of the momentum relaxation, and then study the
electrical transport in the strange-metal phase of cuprate superconductors. For a convenience
in the following discussions, the schematic picture for the electron umklapp scattering process
\cite{Lee21} is illustrated in Fig. \ref{scatter-process}, where an electron on a circular EFS
(left) is scattered by its partner on the umklapp EFS (right).
\begin{figure}[h!]
\centering
\includegraphics[scale=0.75]{Feynman-diagram.pdf}
\caption{The skeletal diagram of the umklapp scattering process for scattering electrons
from the spin excitation. The solid-line represents the electron propagator $G$, and the
dashed-line depicts the spin propagator $D^{(0)}$, while $\square$ describes the bare
vertex function $\Lambda$. \label{Feynman-diagram}}
\end{figure}
In Fig. \ref{scatter-process},
the intensity map of EFS is the same as shown in Fig. \ref{EFS-map}a, while the perfect circle
(red) is the circle with the radius ${\rm k}^{\rm TFA}_{\rm F}$, where
${\bf k}^{\rm TFA}_{\rm F}$ is the Fermi wave vector of the tips of the Fermi arcs. This
perfect circle EFS (red) connects all the eight tips of the Fermi arcs, and therefore shows
that almost all the spectral weight of the electron excitation spectrum is accommodated on this
circle EFS. In the present case, the umklapp scattering of electrons is mediated by the same
spin excitation as in the case of the electron scattering for the renormalization of the
electrons in subsection \ref{Effective-propagator}. To understand the umklapp scattering
mechanism in the present case more clearly, the skeletal diagram of this umklapp scattering
mechanism in energy and momentum space is drawn in Fig. \ref{Feynman-diagram}.

In the recently pioneering work \cite{Lee21}, a model of electrons scattered being dominant
momentum relaxation-time mechanism has been studied, where the umklapp scattering process is
described as an electron-electron scattering via the exchange the critical boson propagator,
rather than the scattering between electrons via the emission and absorption of bosons.
Following these discussions \cite{Lee21}, the electron-electron collision $I_{\rm e-e}$
in Eq. (\ref{Boltzmann-equation-3}) originated from the electron umklapp scattering shown in
Fig. \ref{Feynman-diagram} can be evaluated as,
\begin{widetext}
\begin{eqnarray}\label{electron-collision-1}
I_{\rm e-e}&=&{1\over N^{2}}\sum_{{\bf k}',{\bf p}} {2\over T}
|P({\bf k},{\bf p},{\bf k}',\bar{\varepsilon}_{\bf k}
-\bar{\varepsilon}_{{\bf k}+{\bf p}+{\bf G}})|^{2}
\{\tilde{\Phi}({\bf k})+\tilde{\Phi}({\bf k'})
-\tilde{\Phi}({\bf k}+{\bf p}+{\bf G})-\tilde{\Phi}({\bf k}'-{\bf p})\}\nonumber\\
&\times&n_{\rm F}(\bar{\varepsilon}_{\bf k})n_{\rm F}(\bar{\varepsilon}_{{\bf k}'})
[1-n_{\rm F}(\bar{\varepsilon}_{{\bf k}+{\bf p}+{\bf G}})]
[1-n_{\rm F}(\bar{\varepsilon}_{{\bf k}'-{\bf p}})]
\delta(\bar{\varepsilon}_{\bf k}+\bar{\varepsilon}_{\bf k'}
-\bar{\varepsilon}_{{\bf k}+{\bf p}+{\bf G}}-\bar{\varepsilon}_{{\bf k}'-{\bf p}}),
\end{eqnarray}
\end{widetext}
which therefore leads to the appearance of electrical resistance \cite{Abrikosov88,Mahan81},
where ${\bf G}$ indicates a set of reciprocal lattice vectors, and the umklapp scattering
process in Eq. (\ref{electron-collision-1}) is described as an electron-electron scattering via
the exchange of the effective spin propagator $P({\bf k},{\bf p},{\bf k}',\omega)$, rather than
the scattering between electrons via the emission and absorption of the spin excitations, while
the effective spin propagator $P({\bf k},{\bf p},{\bf k}',\omega)$ is obtained directly from
Fig. \ref{Feynman-diagram} as,
\begin{eqnarray}\label{ESP}
P({\bf k},{\bf p},{\bf k}',\omega)&=&{1\over N}\sum_{\bf q}\Lambda_{{\bf p}+{\bf q}+{\bf k}}
\Lambda_{{\bf q}+{\bf k}'}\Pi({\bf p},{\bf q},\omega).~~~
\end{eqnarray}
The reason of the electron-electron scattering via the exchange of the effective spin propagator
in the present case is the same as in the case discussed in Ref. \onlinecite{Lee21}. For the
normal scattering (${\bf G}=0$), the conservation of the total momentum in
Eq. (\ref{electron-collision-1}) is satisfied straightforwardly \cite{Lee21}, this is because
the distribution in the case of the normal scattering will rapidly equilibrate to a fermion
distribution function with a shifted overall momentum
$\tilde{\Phi}({\bf k})\propto {\bf k}\cdot{\bf E}$, which leads to that its contribution to the
integral of the electron-electron collision in Eq. (\ref{electron-collision-1}) is exactly zero.
However, if we consider the scattering between electrons via the emission and absorption of the
spin excitations, we would have to keep track of the extra shifted boson distribution function
as well, which introduces more complications \cite{Lee21}. Moreover, it has been shown that the
vanishing of the normal scattering in the electron-electron collision
(\ref{electron-collision-1}) is more general \cite{Lee21}. This is following a basic fact that
in order to stay on EFS and conserve the total momentum and energy, the momentum of the normal 
scattering partner ${\bf k}'$ must equal to either ${\bf k}+{\bf p}$ or $-{\bf k}$. In the
former case, the last two terms in $\{...\}$ in Eq. (\ref{electron-collision-1}) cancel the
first two terms. However, in the latter case, since the antisymmetric relation
$\tilde{\Phi}(-{\bf k})=-\tilde{\Phi}({\bf k})$, the first two terms in $\{...\}$ in
Eq. (\ref{electron-collision-1}) cancel, while the same cancellation is valid for the last two
terms corresponding to the outgoing pair ${\bf k}+{\bf p}$ and ${\bf k}'-{\bf p}$. These
results therefore indicate that the contribution from the normal scattering to the integral of
the electron-electron collision (\ref{electron-collision-1}) is negligible \cite{Lee21}.

In the usual case \cite{Abrikosov88,Mahan81}, the derivation of the Boltzmann equation
starting from the nonequilibrium electron propagator involves integrating over energy $\omega$.
However, the electrons at the bottom of the band (then the deep inside EFS) can not be
thermally excited, and as a matter of the principle, all the low-temperature conduction
processes in the strange-metals should involve only states at around EFS \cite{Haldane18}. In
particular, in the early discussions \cite{Prange64}, it has been realized to pick a patch of
EFS specified by $k(\theta)$ with the range $\theta\in [0,2\pi]$, which defines a contour along
EFS parametrized by the direction $\theta$ of the Fermi momentum vector and integrate the
perpendicular momentum and hence over
$\bar{\varepsilon}_{\bf k}$ instead. This is a formula expressed entirely in terms of the EFS
property. Furthermore, this method has been employed to study the low-temperature T-linear
resistivity due to the umklapp scattering from a critical mode \cite{Lee21}. In the present
case of the umklapp scattering of electrons mediated by the spin excitation, an electron on
EFS parametrized by the Fermi angle $\theta$ is scattered to a point parametrized by the Fermi
angle $\theta'$ on the umklapp EFS by the spin excitation carrying momentum
${\rm p}(\theta,\theta')$ as shown in Fig. \ref{scatter-process}. In this case, the usual
distribution function $f({\bf k})$ can be replaced as $f[k(\theta)]$. However, in the usual
formulation, the vector ${\bf k}$ is decomposed into $k(\theta)$ and the momentum in the
perpendicular direction \cite{Lee21,Prange64}, which is then represented by
$\bar{\varepsilon}_{\bf k}$. From this EFS parametrization, the standard Boltzmann equation
(\ref{Boltzmann-equation-3}) now can be expressed simply, where the component of the momentum
${\bf k}$ perpendicular to EFS is replaced by $\bar{\varepsilon}_{\bf k}/{\bf v}_{\bf k}$ and
$\bar{\varepsilon}_{\bf k}$ in turn is replaced by $\omega$. After a straightforward
calculation [see Appendix \ref{electron-electron-collision}], the above electron-electron
collision $I_{\rm e-e}$ in Eq. (\ref{electron-collision-1}) can be derived explicitly,
and then the Boltzmann equation (\ref{Boltzmann-equation-3}) can be obtained as,
\begin{eqnarray}\label{electron-collision}
e{\bf v}_{\rm F}(\theta)\cdot {\bf E}=-2\int {d\theta'\over {2\pi}}\zeta(\theta')
F(\theta,\theta')[\Phi(\theta)-\Phi(\theta')],~~~~~
\end{eqnarray}
where $\Phi(\theta)$ is defined as $\tilde{\Phi}[{\rm k}(\theta)]$ and satisfies an
antisymmetric relation \cite{Lee21} $\Phi(\theta)=-\Phi(\theta+\pi)$, ${\bf v}_{\rm F}(\theta)$
is the Fermi velocity at the Fermi angle $\theta$, and
$\zeta(\theta')={\rm k}^{2}_{\rm F}/[4\pi^{2}{\rm v}^{3}_{\rm F}]$ is the density of states
factor at angle $\theta'$ with the Fermi wave vector ${\rm k}_{\rm F}$ and Fermi velocity
${\rm v}_{\rm F}$, while the coefficient of $\Phi(\theta)$ in the first term of the
right-hand side of Eq. (\ref{electron-collision}),
\begin{eqnarray}\label{scattering-rate}
\gamma(\theta)=2\int {d \theta' \over {2\pi}} \zeta(\theta')F(\theta,\theta'),
\end{eqnarray}
is defined as the scattering out rate \cite{Lee21} from the state of ${\rm k}(\theta)$, which
also is so-called as {\it the angular dependence of the transport scattering rate}, while the
kernel function $F(\theta,\theta')$ depends on the Fermi angles $\theta$ and $\theta'$ in
terms of the magnitude of the momentum transfer ${\rm p}(\theta,\theta')$, i.e.,
$F(\theta,\theta')$ connects the points $\theta$ and $\theta'$ on the umklapp EFS as shown in
Fig. \ref{scatter-process}, and is given by,
\begin{eqnarray}\label{kernel-function}
F(\theta,\theta')&=&{1\over T}\int {d\omega\over 2\pi}{\omega^{2}\over {\rm p}(\theta,\theta')}
{|\bar{P}[{\rm k}(\theta),{\rm p}(\theta,\theta'),\omega]|}^{2}\nonumber\\
&\times& n_{\rm B}(\omega)[1+n_{\rm B}(\omega)],~~~~~~
\end{eqnarray}
where the reduced effective spin propagator
$\bar{P}[{\rm k}(\theta),{\rm p}(\theta,\theta'),\omega]$ has been given in Appendix
\ref{electron-electron-collision}. This kernel function $F(\theta,\theta')$ can be also called
as the probability weight or the strength of the umklapp scattering in which the electron at
the $\theta$ point of EFS is scattered by its umklapp partner at the $\theta'$ point of the
neighboring EFS.

%\section{Quantitative characteristics of electrical resistivity}\label{electron-resistivity}

\section{Low-temperature Linear in temperature resistivity}\label{electron-resistivity}

The dc conductivity then is evaluated in a standard way by the momentum (then the Fermi angle
$\theta$) integral of the umklapp scattering process on EFS, where the current density is given
by \cite{Abrikosov88,Mahan81},
\begin{eqnarray}\label{current-density-1}
{\bf J}=-en_{0}{1\over N}\sum_{\bf k}{\bf v}_{\bf k}f({\bf k}),
\end{eqnarray}
with the relaxation of the momentum that is generated by the action of the electric field on
the mobile electrons \cite{Haldane18} at around EFS with the density $n_{0}$.
Substituting the distribution function $f({\bf k})$ in Eq. (\ref{distribution-function}) into
the above current density equation (\ref{current-density-1}) and performing the radial
integration, the current density now can be obtained as,
\begin{eqnarray}\label{current-density}
{\bf J} &=& en_{0}{1\over N}\sum_{\bf k}{\bf v}_{\bf k}
{dn_{\rm F}({\bar{\varepsilon}_{\bf k}})\over d\bar{\varepsilon}_{\bf k}}\tilde{\Phi}({\bf k})
\nonumber\\
&=&-en_{0}{{\rm k}_{\rm F}\over {\rm v}_{\rm F}}\int {d\theta\over (2\pi)^{2}}
{\bf v}_{\rm F}(\theta)\Phi(\theta).~~~~~
\end{eqnarray}

For the calculation of the dc conductivity, we need to obtain the local shift of the chemical
potential $\Phi(\theta)$. The spectral weight of
${\rm Im}P({\bf k}_{\rm F},{\bf p}-{\bf k}_{\rm F},{\bf k'}_{\rm F},\omega)$ in
Eq. (\ref{electron-collision-1}) achieves its maximal value at around the antinodal region
[see Fig. \ref{effective-spin-propagator} in Appendix \ref{electron-electron-collision}],
where the scattering probability for two electrons is largest. In other words, the main
contribution to the kernel function $F(\theta,\theta')$ [then the electrical resistivity] comes
from such umklapp scattering process in which the electron at around the antinodal region of
the circular EFS (left) shown in Fig. \ref{scatter-process} is scattered by its partner at
around the antinodal region of the umklapp circular EFS (right), where the Fermi angle
$\theta'$ is almost identical with the Fermi angle $\pi-\theta$, and then according to the
antisymmetric relation satisfied by $\Phi(\theta)$, the following relation,
\begin{eqnarray}\label{Phi-theta-1}
\Phi(\theta')=\Phi(\pi-\theta)=-\Phi(\theta),
\end{eqnarray}
is valid. In this relaxation-time approximation, the local shift of the chemical potential
$\Phi(\theta)$ can be derived straightforwardly from Eqs. (\ref{electron-collision})
and (\ref{scattering-rate}) as \cite{Lee21},
\begin{eqnarray}\label{Phi-theta}
\Phi(\theta)=-{e{\rm v}_{\rm F}{\rm cos}(\theta)E_{\hat{x}}\over 2\gamma(\theta)},
\end{eqnarray}
where the electric field ${\bf E}$ has been chosen along the $\hat{x}$-axis. Substituting the
above result of $\Phi(\theta)$ into Eq. (\ref{current-density}), the dc conductivity
therefore is obtained explicitly as,
\begin{eqnarray}\label{dc-conductivity}
\sigma_{\rm dc}(T)={1\over 2}e^{2}n_{0}{\rm k}_{\rm F}{\rm v}_{\rm F}
\int {d\theta\over (2\pi)^{2}}{\rm cos}^{2}(\theta)
{1\over\gamma(\theta)},
\end{eqnarray}
and then the electrical resistivity is obtained directly from the above dc conductivity as,
\begin{eqnarray}\label{dc-resistivity}
\rho(T)={1\over \sigma_{\rm dc}(T)}.
\end{eqnarray}

\begin{figure}[h!]
\centering
\includegraphics[scale=1.0]{resistivity-temperature.pdf}
\caption{(Color online) The electrical resistivity as a function of temperature at
$\delta=0.09$ (black-line), $\delta=0.12$ (red-line), $\delta=0.15$ (green-line), $\delta=0.18$
(blue-line), and $\delta=0.21$ (yellow-line). \label{resistivity-temperature}}
\end{figure}

Now we are ready to discuss the striking features of the electrical transport in the
strange-metal phase of cuprate superconductors. We have made a series of calculations for the
electrical resistivity $\rho(T)$ in Eq. (\ref{dc-resistivity}) at different doping levels, and
the results of the electrical resistivity $\rho(T)$ as a function of temperature at the doping
concentrations $\delta=0.09$ (black-line), $\delta=0.12$ (red-line), $\delta=0.15$ (green-line),
$\delta=0.18$ (blue-line), and $\delta=0.21$ (yellow-line) are plotted in
Fig. \ref{resistivity-temperature}. Apparently, the experimental results of the doping
dependence of the low-temperature electrical resistivity
\cite{Allen89,Gurvitch87,Takagi92,Martin90,Mandrus92,Ando01,Daou09,Cooper09,Legros19,Yuan22,Ayres21,Grisso21}
are qualitatively reproduced, where the highly unconventional features can be summarized as:
(i) the electrical resistivity $\rho(T)$ as a function of temperature is a perfect straight line
down to the temperature $T\sim 0.015J=15$K; (ii) the low-temperature T-linear resistivity extends
over a range of doping from the underdoped to overdoped regimes, where the T-linear coefficient
(then the strength of the T-linear resistivity) decreases with the increase of the doping
concentration; (iii) however, the electrical resistivity deviates from the pure T-linearity at
the far lower temperature range $T< 0.015J=15$K, while our numerical fit indicates that in this
far lower temperature range $T< 0.015J=15$K, the electrical resistivity decreases quadratically
as the temperature decreases. The results in Fig. \ref{resistivity-temperature} therefore also
indicate that the same spin excitation that acts like a bosonic glue to hold the electron
pairs together responsible for superconductivity \cite{Feng15,Feng0306,Feng12,Feng15a} also
dominates the electron scattering responsible for the low-temperature T-linear resistivity in
the strange-metal phase and the associated electronic structure.

Finally, it should be emphasized that the local shift of the chemical potential $\Phi(\theta)$
can be also evaluated directly by the numerical solution of the Boltzmann equation
(\ref{electron-collision}) together with an additional electron-impurity collision without
making the relaxation-time approximation \cite{Lee21}, where the Fermi angle $\theta'$ variable
in Eq. (\ref{electron-collision}) can be discretized, and then the integral-differential
equation (\ref{electron-collision}) is converted into the matrix equation. The accurate result
of $\Phi(\theta)$ is obtained in terms of the numerical calculation of the inverse of this
matrix.
\begin{figure}[h!]
\centering
\includegraphics[scale=0.90]{out-rate-theta.pdf}
\caption{The transport scattering rate $\gamma(\theta)$ as a function of Fermi angle $\theta$
at $\delta=0.15$ with $T=0.05J$. \label{out-rate-theta}}
\end{figure}
In this case, we have also performed a numerical calculation $\Phi(\theta)$ [then
$\rho(T)$], and the results show that although the resistivity saturates to a constant
$\rho_{0}(T)$ induced by the impurity, the qualitative behaviour of the electrical resistivity
is the same as that obtained in the above relaxation-time approximation except for the subtle
difference of the slopes, which is also qualitatively consistent with the results obtained
from electrons umklapp scattered by a critical bosonic mode \cite{Lee21}.

An explanation of the above obtained low-temperature T-linear resistivity in the strange-metal
phase of cuprate superconductors can be found from the Fermi angle and temperature dependence
of the transport scattering rate $\gamma(\theta,T)$ in Eq. (\ref{scattering-rate}) obtained
directly from the electron umklapp scattering process. To reveal this angular and temperature
dependence of $\gamma(\theta,T)$ more clearly, we first plot $\gamma(\theta)$ as a function
of Fermi angle $\theta$ at $\delta=0.15$ with $T=0.05J$ in Fig. \ref{out-rate-theta}.
In a comparison with the corresponding angular dependence of the single-particle scattering
rate $\Gamma(\theta)$ in Fig. \ref{single-particle-scattering}, it thus shows that although the
magnitude of $\gamma(\theta)$ at an any given Fermi angle is less than that of $\Gamma(\theta)$
at the corresponding Fermi angle, the global behaviour of the angular dependence of
$\gamma(\theta)$ is similar to that of $\Gamma(\theta)$, where $\gamma(\theta)$ is largest at
around the antinodal region, and smallest at around the tips of the Fermi arcs, which is also
consistent with the strong momentum dependence of the effective spin propagator
$P({\bf k},{\bf p}-{\bf k},{\bf k}',\omega)$ shown in Fig. \ref{effective-spin-propagator} in
Appendix \ref{electron-electron-collision}. In other words, both the transport scattering rate
$\gamma(\theta)$ and single-particle scattering rate $\Gamma(\theta)$ as a function of Fermi
angle presents the similar behavior of the effective spin propagator
$P({\bf k},{\bf p}-{\bf k},{\bf k}',\omega)$. The result in Fig. \ref{out-rate-theta} also shows
that the electrons at around the tips of the Fermi arcs are mainly responsible for the conductivity,
while the transport scattering rates at both the antinodal and nodal regions mainly determine the
magnitude of the electrical resistivity and of its behaviour of the temperature dependence.

\begin{figure}[h!]
\centering
\includegraphics[scale=0.9]{out-rate-temperature.pdf}
\caption{The transport scattering rate $\gamma(T)$ as a function of temperature for
$\delta=0.15$ at the antinode. \label{out-rate-temperature}}
\end{figure}

On the other hand, for an any given Fermi angle $\theta$, $\gamma(\theta,T)$ varies strongly
with temperature. To see this temperature dependence of $\gamma(\theta,T)$ more clearly, we
plot $\gamma(T)$ as a function of temperature for $\delta=0.15$ at the antinode in
Fig. \ref{out-rate-temperature}. It is surprising that $\gamma(T)$ is entirely T-linear in the
low-temperature range $T> 0.015J=15$K, where it decreases linearly with temperature as the
temperature decreases to $T\sim 0.015J=15$K, while this transport scattering rate $\gamma(T)$
is instead quadratic in temperature (T-quadratic) in the far lower temperature range
$T< 0.015J=15K$. Moreover, although $\gamma(\theta,T)$ is highly anisotropic in momentum-space,
this low-temperature T-linear $\gamma(\theta,T)$ occurs at an any given Fermi
angle $\theta$ (then for all directions), in agreement with the experimental observations
\cite{Grisso21}. In a comparison with the corresponding results of the temperature dependence
of the electrical resistivity shown in Fig. \ref{resistivity-temperature}, we find that the
low-temperature T-linear behaviour of $\gamma(T)$ together with the low-temperature range and
the T-quadratic behaviour of $\gamma(T)$ together with the far lower temperature range are
respectively the same as the corresponding behaviours and ranges in the electrical resistivity
$\rho(T)$, which therefore indicates that the remarkable T-linear resistivity in the
low-temperature and the T-quadratic resistivity in the far lower temperature are generated by
the T-linear transport scattering rate in the low-temperature and T-quadratic transport
scattering rate in the far lower temperature, respectively.

The expression form of the transport scattering rate $\gamma(\theta,T)$ in
Eq. (\ref{scattering-rate}) also indicates that the exotic behaviour of the temperature
dependence of $\gamma(\theta,T)$ is mainly determined by the striking features of the
temperature dependence of the kernel function (then the probability weight or the strength of
the electron umklapp scattering) $F(\theta,\theta')$ in Eq. (\ref{kernel-function}). With the
help of $\gamma(\theta,T)$ in Eq. (\ref{scattering-rate}) and
$F(\theta,\theta')$ in Eq. (\ref{kernel-function}), we now turn to show that (i) the antinodal
umklapp scattering, in which the electron at around the antinodal region of EFS is scattered
by its umklapp partner at around the antinodal region of the neighboring EFS, leads to the
T-linear behaviour of $\gamma(\theta,T)$ (then the T-linear resistivity); and (ii) the nodal
umklapp scattering, in which the electron at around the nodal region of EFS is scattered by
its umklapp partner at around the nodal region of the neighboring EFS, tends to induces a
deviation from the T-linear behaviour.
\begin{figure}
\centering
\includegraphics[scale=0.90]{kernel.pdf}
\caption{(Color online) The surface plot of the kernel function $F(\theta,\theta')$ at
$\delta=0.15$ with $T=0.05J$, where AN, TFA, and ND denote the antinode, tip of the Fermi arc,
and node, respectively. \label{kernel}}
\end{figure}
To see this physical picture more clearly, the surface
plot of $F(\theta,\theta')$ at $\delta=0.15$ with $T=0.05J$ is plotted in Fig. \ref{kernel},
where the probability weight of the electron umklapp scattering has been separated into three
characteristic regions: (i) the antinodal region, where a particularly large fraction of the
probability weight is located, leading to that $\gamma(\theta)$ is largest at around the
antinodal region; (ii) the nodal region, where a small amount of the probability weight is
inhabited, leading to that the magnitude of $\gamma(\theta)$ at around the nodal region is much
smaller than that at around the antinodal region; (iii) the region at around the tips of the
Fermi arcs, where the strength of the umklapp scattering is anomalously small, leading to the
appearance of the weakest scattering at around the tips of the Fermi arcs. The above result in
Fig. \ref{kernel} indicates that the electron umklapp scattering is concentrated at around the
antinodes and nodes, and therefore is well consistent with the result of the angular
dependence of $\gamma(\theta,T)$ shown in Fig. \ref{out-rate-theta}.
However, the strengths of the antinodal and nodal umklapp
scattering are strong temperature dependent, which induces a competition between the antinodal
umklapp scattering and nodal umklapp scattering. This competition is closely related to the
crossover from the T-linear behaviour of $\gamma(T)$ in the low-temperature into the T-quadratic
behaviour in the far lower temperature, and can be well understood in terms of the ratio of the
strength of the nodal umklapp scattering to the strength of the antinodal umklapp scattering,
\begin{eqnarray*}
R_{\rm F}(T)={F(\theta_{\rm ND},\theta'_{\rm ND},T)\over F(\theta_{\rm AN},\theta'_{\rm AN},T)}.
\end{eqnarray*}
\begin{figure}[h!]
\centering
\includegraphics[scale=0.9]{R-ratio.pdf}
\caption{The ratio of the strength of the nodal umklapp scattering to the strength of the
antinodal umklapp scattering as a function of temperature for $\delta=0.15$. \label{R-ratio}}
\end{figure}
However, as we have mentioned in subsection \ref{Effective-propagator}, the calculation in this
paper is performed numerically on a $160\times 160$ lattice in momentum space, with the
infinitesimal $i0_{+}\rightarrow i\Gamma$ replaced by a small damping $\Gamma=0.1J$, which
leads to that the weight of the $\delta$-function type peak in $F(\theta,\theta')$ at the
antinode (node) spreads on the extremely small area $\{\theta_{\rm AN}\}$
$[\{\theta_{\rm ND}\}]$ around the antinode (node) as shown in Fig. \ref{kernel}. In particular,
the summation of these spread weights around this extremely small area $\{\theta_{\rm AN}\}$
$[\{\theta_{\rm ND}\}]$ is less affected by the calculation for a finite lattice. In this case,
a more appropriate ratio can be obtained as,
\begin{eqnarray}\label{order-parameter}
\bar{R}_{\rm F}(T)={\bar{F}_{\rm ND}(T)\over \bar{F}_{\rm AN}(T)},
\end{eqnarray}
for the reduction of the size effect in the finite-lattice calculation, where
$\bar{F}_{\rm AN}(T)$ and $\bar{F}_{\rm ND}(T)$ are given by,
\begin{eqnarray*}
\bar{F}_{\rm AN}(T)&=&{1\over 2\pi}\sum_{\substack{\theta_{\rm AN}\in\{\theta_{\rm AN}\}\\
\theta'_{\rm AN}\in\{\theta'_{\rm AN}\}}}F(\theta,\theta',T),\\
\bar{F}_{\rm ND}(T)&=&{1\over 2\pi}\sum_{\substack{\theta_{\rm ND}\in\{\theta_{\rm ND}\}\\
\theta'_{\rm ND}\in\{\theta'_{\rm ND}\}}}F(\theta,\theta',T),
\end{eqnarray*}
with the summation $\theta_{\rm AN}\in\{\theta_{\rm AN}\}$
$[\theta'_{\rm AN}\in\{\theta'_{\rm AN}\}]$ that is restricted to the extremely small area
$\{\theta_{\rm AN}\}$ $[\{\theta_{\rm ND}\}]$ around the antinode (node). In this case, we have
carried out a series of calculation for the ratio $\bar{R}_{\rm F}(T)$ at different doping
levels, and the result of $\bar{R}_{\rm F}(T)$ as a function of temperature at $\delta=0.15$ is
plotted in Fig. \ref{R-ratio}, where $\bar{R}_{\rm F}(T)$ decreases monotonically with the
increase of temperature. In particular, in the lower ratio range ($\bar{R}_{\rm F}(T)< 0.395$),
which is corresponding to the low-temperature range ($T>0.015J=15$K), the strength of the nodal
umklapp scattering is quite weak, while the strength of the antinodal umklapp scattering is
particularly strong. This particularly strong antinodal umklapp scattering therefore leads to
the low-temperature T-linear behaviour of $\gamma(T)$. In other words, the low-temperature
T-linear behaviour of $\gamma(T)$ (then the T-linear resistivity) is mainly governed by the
antinodal umklapp scattering. However, although both the strengths of the nodal and antinodal
umklapp scattering decrease with the decrease of temperature, the decrease of the strength of
the nodal umklapp scattering is slower than that of the antinodal umklapp scattering. In
particular, in the higher ratio range ($\bar{R}_{\rm F}(T)> 0.395$), which is corresponding to
the far lower temperature range ($T< 0.015J=15$K), although the antinodal umklapp scattering
still tends to induce the T-linear behaviour, the strength of the nodal umklapp scattering
becomes enough strong to generate a deviation from the T-linear behaviour, and then the
T-quadratic behaviour of $\gamma(T)$ (then the T-quadratic resistivity) in the far lower
temperature results from both the antinodal and nodal umklapp scattering.

\section{Summary}\label{summary}

In the framework of the kinetic-energy-driven superconductivity, the electrons scattering with
the spin excitation forms the strange-metal liquid at the temperature above $T_{\rm c}$, where
the energy distribution curve exhibits a complicated line-shape at around the nodal and
antinodal regions, the electron dispersion has an anomalous dispersion kink along EFS, and more
specifically, the highly anisotropic single-particle scattering rate induces an redistribution
of the spectral weight on EFS to form the Fermi arcs with almost all the spectral weight
assembled at around the tips of the Fermi arcs. We start from this low-energy electronic
structure in the strange-metal phase of cuprate superconductors to study the nature of the
low-temperature T-linear resistivity. In particular, we derive explicitly the Boltzmann
equation, and then employ it to discuss the doping dependence of the electrical resistivity,
where the angular dependence of the transport scattering rate originates from the umklapp
scattering between electrons by the exchange of the same spin excitations. Our results show
that although the magnitude of the angular dependence of the transport scattering rate at an
any given Fermi angle is smaller than the corresponding value of the single-particle scattering
rate, the transport scattering rate presents the similar behavior of the single-particle
scattering rate, where the transport scattering rate is largest at around the antinodal region
and smallest at around the tips of the Fermi arcs, indicating that the electrical resistivity
is mainly dominated by the antinodal umklapp scattering and nodal umklapp scattering. Moreover,
the antinodal umklapp scattering tends to induce the T-linear behaviour of the transport
scattering rate, while the nodal umklapp scattering tends to generate a deviation from the
T-linear behaviour, therefore there is a competition between the antinodal umklapp scattering
and nodal umklapp scattering. However, in the low-temperature range, the nodal umklapp
scattering with a quite weak scattering strength is overwhelmed by the antinodal umklapp
scattering with a particularly strong scattering strength, and then this particularly strong
antinodal umklapp scattering leads to the T-linear behaviour of the transport scattering rate.
On the other hand, in the far lower temperature range, the strength of the nodal umklapp
scattering becomes strong enough to induce a deviation from the T-linear behaviour, and then
both the nodal and antinodal umklapp scattering leads to the T-quadratic behaviour of the
transport scattering rate. This T-linear behaviour of the transport scattering rate in the
low-temperature and the T-quadratic behaviour in the far lower temperature in turn generate
the T-linear resistivity in the low-temperature and the T-quadratic resistivity in the far
lower temperature. Our theory also shows that the same spin excitation that acts like a
bosonic glue to hold the electron pairs together responsible for the exceptionally high
$T_{\rm c}$ also mediates the electron scattering in the strange-metal phase responsible for
the T-linear resistivity and the associated electronic structure.

The transport scattering mechanism developed in this paper for the understanding of the
T-linear resistivity in the strange-metal phase of cuprate superconductors can be also employed
to study the exotic transport in other families of strange metals \cite{Bruin13,Grigera01}. In
particular, based on this transport scattering theory, we have also discussed the striking
T-linear resistivity in the normal-state of the electron-doped cuprate superconductors
\cite{Ma22}, where we show the common mechanism linking the transport in the normal-state of
both the hole- and electron-doped cuprate superconductors. These and the related works will
be presented elsewhere.

\section*{Acknowledgements}

This work is supported by the National Key Research and Development Program of China under
Grant No. 2021YFA1401803, and the National Natural Science Foundation of China under Grant
Nos. 11974051, 12274036, and 11734002.

%\newpage

\begin{appendix}

\section{Derivation of electron-electron collision}\label{electron-electron-collision}

The aim of this Appendix is to derive the electron-electron collision $I_{\rm e-e}$ in
Eq. (\ref{electron-collision}) of the main text. The electron-electron collision in
Eq. (\ref{electron-collision-1}) can be also rewritten as,
\begin{widetext}
\begin{eqnarray}\label{electron-collision-2}
I_{\rm e-e}&=&{1\over N^{2}}\sum_{{\bf k}',{\bf p}} {2\over T}|P({\bf k},{\bf p}
-{\bf k},{\bf k}',\bar{\varepsilon}_{\bf k}-\bar{\varepsilon}_{{\bf p}+{\bf G}})|^{2}
[\tilde{\Phi}({\bf k})+\tilde{\Phi}({\bf k'})-\tilde{\Phi}({\bf p}+{\bf G})
-\tilde{\Phi}({\bf k}'+{\bf k}-{\bf p})] \nonumber\\
&\times& n_{\rm F}(\bar{\varepsilon}_{\bf k})n_{\rm F}(\bar{\varepsilon}_{{\bf k}'})
[1-n_{\rm F}(\bar{\varepsilon}_{{\bf p}+{\bf G}})]
[1-n_{\rm F}(\bar{\varepsilon}_{{\bf k}'+{\bf k}-{\bf p}})]
\delta(\bar{\varepsilon}_{\bf k}+\bar{\varepsilon}_{\bf k'}
-\bar{\varepsilon}_{{\bf p}+{\bf G}}-\bar{\varepsilon}_{{\bf k}'+{\bf k}-{\bf p}}),~~~~
\end{eqnarray}
with the effective spin propagator $P({\bf k},{\bf p}-{\bf k},{\bf k}',\omega)$ that can be
expressed explicitly as,
\begin{eqnarray}\label{ESP}
P({\bf k},{\bf p}-{\bf k},{\bf k}',\omega)={1\over N}\sum_{\bf q}\Lambda_{{\bf p}+{\bf q}}
\Lambda_{{\bf q}+{\bf k}'}\Pi({\bf p}-{\bf k},{\bf q},\omega)
=\int^{\infty}_{-\infty}{d\omega'\over\pi}{{\rm Im}P({\bf k},{\bf p}-{\bf k},{\bf k}',\omega')
\over\omega'-\omega}, ~~~~~~
\end{eqnarray}
\end{widetext}
where the imaginary part of the effective spin propagator
${\rm Im}P({\bf k},{\bf p}-{\bf k},{\bf k}',\omega)$ is directly related to the effective
spin spectral function, and is also defined as the scattering probability for two electrons.
\begin{figure}[h!]
\centering
\includegraphics[scale=1.0]{effective-spin-propagator.pdf}
\caption{(Color online) The intensity map of the imaginary part of the effective spin propagator
${\rm Im}P({\bf k},{\bf p}-{\bf k},{\bf k}',\omega)$ along ${\bf k}={\bf k}'={\bf k}_{\rm F}$ in
a $[p_{x},p_{y}]$ plane at $\delta=0.15$ for energy $\omega=-0.05J$ with $T=0.002J$.
\label{effective-spin-propagator}}
\end{figure}
However, in our previous discussions \cite{Mou19}, we have shown that for given momentums
${\bf k}$ and ${\bf k}'$, ${\rm Im}P({\bf k},{\bf p}-{\bf k},{\bf k}',\omega)$ exhibits a
remarkable evolution with momentum ${\bf p}$ and $\omega$ except for $\omega=0$, where
${\rm Im}P({\bf k},{\bf p}-{\bf k},{\bf k}',0)=0$.
To see this unusual momentum ${\bf p}$
dependence of ${\rm Im}P({\bf k},{\bf p}-{\bf k},{\bf k}',\omega)$ more clearly, we plot
the intensity map of ${\rm Im}P({\bf k},{\bf p}-{\bf k},{\bf k}',\omega)$ along EFS
${\bf k}={\bf k}'={\bf k}_{\rm F}$ in a $[p_{x},p_{y}]$ plane for energy $\omega=-0.05J$
with $T=0.002J$ in Fig. \ref{effective-spin-propagator}, where the spectral weight of
${\rm Im}P({\bf k}_{\rm F},{\bf p}-{\bf k}_{\rm F},{\bf k'}_{\rm F},\omega)$ along EFS
${\bf k}={\bf k}'={\bf k}_{\rm F}$ converges on the corresponding EFS
${\bf p}={\bf k}_{\bf F}$, i.e.,
${\rm Im}P({\bf k}_{\rm F},{\bf p}-{\bf k}_{\rm F},{\bf k'}_{\rm F},\omega)\neq 0$ for
${\bf p}={\bf k}_{\bf F} $, and otherwise
${\rm Im}P({\bf k}_{\rm F},{\bf p}-{\bf k}_{\rm F},{\bf k'}_{\rm F},\omega)=0$. In
particular, the spectral weight of
${\rm Im}P({\bf k}_{\rm F},{\bf p}-{\bf k}_{\rm F},{\bf k'}_{\rm F},\omega)$
exhibits the largest value at around the antinodal region, however, the most striking feature
is that the actual minimum of the spectral weight of
${\rm Im}P({\bf k}_{\rm F},{\bf p}-{\bf k}_{\rm F},{\bf k'}_{\rm F},\omega)$ does not appear
at around the node, but locates exactly at the tips of the Fermi arcs. This special angular
dependence of ${\rm Im}P({\bf k}_{\rm F},{\bf p}-{\bf k}_{\rm F},{\bf k'}_{\rm F},\omega)$
therefore induces an EFS reconstruction to form the Fermi arcs as shown in Fig. \ref{EFS-map}
with almost all the spectral weight of the electron excitation spectrum that resides at
around the tips of the Fermi arcs.

The result shown in Fig. \ref{effective-spin-propagator} therefore indicates that the main
contribution in $P({\bf k},{\bf p}-{\bf k},{\bf k'},\omega)$ comes from the part of the
momentum ${\bf p}={\bf k}$. In this case, the term
$\Phi({\bf k'})-\Phi({\bf k'}+{\bf k}-{\bf p})\sim 0$ in the right-hand side of
Eq. (\ref{electron-collision-2}), and then
$\delta(\bar{\varepsilon}_{\bf k}+\bar{\varepsilon}_{\bf k'}-\bar{\varepsilon}_{{\bf p}
+{\bf G}}-\bar{\varepsilon}_{{\bf k}'+{\bf k}-{\bf p}})$ in the right-hand side of
Eq. (\ref{electron-collision-2}) can be replaced by the integral identity as \cite{Lee21},
\begin{widetext}
\begin{eqnarray}\label{identity}
\delta(\bar{\varepsilon}_{\bf k}&+&\bar{\varepsilon}_{\bf k'}
-\bar{\varepsilon}_{{\bf p}+{\bf G}}-\bar{\varepsilon}_{{\bf k}'+{\bf k}-{\bf p}})
=\int^{\infty}_{-\infty}d\omega\delta(\bar{\varepsilon}_{\bf k}
-\bar{\varepsilon}_{{\bf p}+{\bf G}}-\omega)\delta(\omega+\bar{\varepsilon}_{\bf k'}
-\bar{\varepsilon}_{{\bf k}'+{\bf k}-{\bf p}}).~~~~~
\end{eqnarray}
\end{widetext}
On the other hand, the umklapp scattering process occurs mainly at around EFS, i.e.,
${\bf k'}\approx {\bf k'}_{\rm F}$, therefore the momentum ${\bf k'}$ in the effective spin
propagator $P({\bf k},{\bf p}-{\bf k},{\bf k}',\omega)$ can be approximately replaced by the
reduced effective spin propagator $\bar{P}({\bf k},{\bf p}-{\bf k},\omega)$,
\begin{eqnarray}\label{ESP-5}
\bar{P}({\bf k},{\bf p}-{\bf k},\omega)={1\over W_{\rm sp}}
P({\bf k},{\bf p}-{\bf k},{\bf k'}_{\rm F},\omega), ~~~~~~
\end{eqnarray}
where following the common practice, the scattering probability for two electrons has been
normalized with the normalization factor $W^{2}_{\rm sp}=(1/N^{2})\sum_{{\bf k},{\bf p}}
\int |{\rm Im}\bar{P}({\bf k},{\bf p}-{\bf k},\omega)|^{2}d\omega$. Substituting above results
in Eqs. (\ref{identity}) and (\ref{ESP-5}) into Eq. (\ref{electron-collision-2}), the
electron-electron collision in Eq. (\ref{electron-collision-2}) can be expressed
explicitly as \cite{Lee21},
\begin{widetext}
\begin{eqnarray}\label{electron-collision-3}
I_{\rm e-e}&=&{1\over N^{2}}\sum_{{\bf k}',{\bf p}} {2\over T}|\bar{P}({\bf k},{\bf p}
-{\bf k},\bar{\varepsilon}_{\bf k}-\bar{\varepsilon}_{{\bf p}+{\bf G}})|^{2}
[\tilde{\Phi}({\bf k})-\tilde{\Phi}({\bf p}+{\bf G})]n_{\rm F}(\bar{\varepsilon}_{\bf k})
n_{\rm F}(\bar{\varepsilon}_{{\bf k}'})[1-n_{\rm F}(\bar{\varepsilon}_{{\bf p}+{\bf G}})]
[1-n_{\rm F}(\bar{\varepsilon}_{{\bf k}'+{\bf k}-{\bf p}})]\nonumber\\
&\times& \int^{\infty}_{-\infty}d\omega\delta(\bar{\varepsilon}_{\bf k}
-\bar{\varepsilon}_{{\bf p}+{\bf G}}-\omega)
\delta(\omega+\bar{\varepsilon}_{\bf k'}-\bar{\varepsilon}_{{\bf k}'+{\bf k}-{\bf p}})
\nonumber\\
&=&{1\over N^{2}}\sum_{{\bf k}',{\bf p}} {2\over T}|\bar{P}({\bf k},{\bf p},
\bar{\varepsilon}_{\bf k}-\bar{\varepsilon}_{{\bf p}+{\bf k}+{\bf G}})|^{2}
[\tilde{\Phi}({\bf k})-\tilde{\Phi}({\bf p}+{\bf k}+{\bf G})]
n_{\rm F}(\bar{\varepsilon}_{\bf k})
n_{\rm F}(\bar{\varepsilon}_{{\bf k}'})
[1-n_{\rm F}(\bar{\varepsilon}_{{\bf p}+{\bf k}+{\bf G}})]
[1-n_{\rm F}(\bar{\varepsilon}_{{\bf k}'-{\bf p}})]\nonumber\\
&\times& \int^{\infty}_{-\infty}d\omega\delta(\bar{\varepsilon}_{\bf k}
-\bar{\varepsilon}_{{\bf p}+{\bf k}+{\bf G}}-\omega)
\delta(\omega+\bar{\varepsilon}_{\bf k'}-\bar{\varepsilon}_{{\bf k}'-{\bf p}}).
\end{eqnarray}
\end{widetext}
Now we replace the momentum ${\bf k}'$ integration by an integration along EFS and one
perpendicular to it, i.e.,
$(1/N)\sum_{{\bf k}'}=\int {\rm k}'d{\rm k}'d\theta_{{\rm k}'}/(2\pi)^{2}$, where the
$\theta_{{\rm k}'}$ specifies a patch of EFS in the direction $\theta_{{\rm k}'}$ as shown in
Fig. \ref{scatter-process}, and then the radial integration $\int d{\rm k}'$ is replaced by an
integral over $\int d{\rm k}'=\int d\bar{\varepsilon}_{{\bf k}'}/{\rm v}_{\rm F}$. In this
case, the above electron-electron collision in Eq. (\ref{electron-collision-3}) can be
simplified as,
\begin{widetext}
\begin{eqnarray}\label{electron-collision-4}
I_{\rm e-e}&=&{1\over 2\pi}{2{\rm k}_{\rm F}\over T{\rm v}^{2}_{\rm F}}{1\over N}\sum_{{\bf p}}
{1\over |{\bf p}|}\int {d\omega\over 2\pi}|\bar{P}({\bf k},{\bf p},\bar{\varepsilon}_{\bf k}
-\bar{\varepsilon}_{{\bf p}+{\bf k}+{\bf G}})|^{2}[\tilde{\Phi}({\bf k})
-\tilde{\Phi}({\bf p}+{\bf k}+{\bf G})]n_{\rm F}(\bar{\varepsilon}_{\bf k})
[1-n_{\rm F}(\bar{\varepsilon}_{{\bf p}+{\bf k}+{\bf G}})]\nonumber\\
&\times& \omega[1+n_{\rm B}(\omega)]
\delta(\bar{\varepsilon}_{\bf k}-\bar{\varepsilon}_{{\bf p}+{\bf k}+{\bf G}}-\omega).
\end{eqnarray}
\end{widetext}
For the obtain of the above equation (\ref{electron-collision-4}), the following identity,
\begin{eqnarray}\label{identity-1}
\int_{-\infty}^{+\infty}d\varepsilon n_{\rm F}(\varepsilon-\omega)[1-n_{\rm F}(\varepsilon)]
=\omega[1+n_{\rm B}(\omega)],~~~~
\end{eqnarray}
has been used, where the appearance of the boson distribution function $n_{\rm B}(\omega)$ in
the right-hand side signals that we are describing a particle-hole effective
spin excitation which has the boson statistics \cite{Lee21}.

Now we turn to evaluate the momentum ${\bf p}$ integration, which is quite similar to the
evaluation of the momentum ${\bf k}'$ integration in Eqs. (\ref{electron-collision-3}) and
(\ref{electron-collision-4}). After a straightforward calculation for the momentum ${\bf p}$
integration in Eq. (\ref{electron-collision-4}), the electron-electron collision term can be
obtained explicitly as
\begin{widetext}
\begin{eqnarray}\label{electron-collision-5}
I_{\rm e-e}&=&{1\over (2\pi)^{2}}{2{\rm k}^{2}_{\rm F}\over T{\rm v}^{3}_{\rm F}}
\int {d\theta'\over 2\pi}\int {d\omega\over 2\pi}{1\over {\rm p}(\theta,\theta')}
|\bar{P}[k(\theta),p(\theta,\theta'),\omega]|^{2}[\Phi(\theta)-\Phi(\theta')]
n_{\rm F}(\bar{\varepsilon}_{k(\theta)})
[1-n_{\rm F}(\bar{\varepsilon}_{k(\theta)}-\omega)]\omega[1+n_{\rm B}(\omega)],~~~~~
\end{eqnarray}
%\end{widetext}
where $\tilde{\Phi}({\bf k})$ and $\tilde{\Phi}({\bf p}+{\bf k}+{\bf G})$ in the right-hand
of side have been replaced by $\Phi(\theta)$ and $\Phi(\theta')$, respectively, and at
low-energy regime, the Boltzmann equation in Eq. (\ref{Boltzmann-equation-3}) can
be expressed as,
%\begin{widetext}
\begin{eqnarray}\label{Boltzmann-equation-4}
e{\bf v}_{\rm F}\cdot{\bf E}{\partial n_{\rm F}(\bar{\varepsilon}_{k(\theta)})\over
\partial\bar{\varepsilon}_{k(\theta)}}&=&{1\over (2\pi)^{2}}{2{\rm k}^{2}_{\rm F}\over
T{\rm v}^{3}_{\rm F}}\int {d\theta'\over 2\pi}\int {d\omega\over 2\pi}
{1\over {\rm p}(\theta,\theta')}|\bar{P}[k(\theta),p(\theta,\theta'),\omega]|^{2}
[\Phi(\theta)-\Phi(\theta')]\nonumber\\
&\times& n_{\rm F}(\bar{\varepsilon}_{k(\theta)})
[1-n_{\rm F}(\bar{\varepsilon}_{k(\theta)}-\omega)]\omega[1+n_{\rm B}(\omega)].~~~~~
\end{eqnarray}
%\end{widetext}
Integrating both the left-hand and right-hand sides over the energy
$\bar{\varepsilon}_{k(\theta)}$, the Boltzmann equation in Eq. (\ref{Boltzmann-equation-4})
can be obtained explicitly as,
%\begin{widetext}
\begin{eqnarray}\label{Boltzmann-equation-5}
e{\bf v}_{\rm F}\cdot{\bf E}&=&-{1\over (2\pi)^{2}}{2{\rm k}^{2}_{\rm F}\over
T{\rm v}^{3}_{\rm F}}\int {d\theta'\over 2\pi}\int {d\omega\over 2\pi}
{1\over {\rm p}(\theta,\theta')}|\bar{P}[k(\theta),p(\theta,\theta'),\omega]|^{2}
[\Phi(\theta)-\Phi(\theta')]\omega^{2}n_{\rm B}(\omega)[1+n_{\rm B}(\omega)]\nonumber\\
&=&-2\int {d\theta'\over {2\pi}}\zeta(\theta')F(\theta,\theta')[\Phi(\theta)-\Phi(\theta')],
\end{eqnarray}
%\end{widetext}
which is the same as quoted in Eq. (\ref{electron-collision}) of the main text.

\end{widetext}

\end{appendix}

%\newpage

\begin{thebibliography}{00}

\bibitem{Fujita12} See, e.g., the review, M. Fujita, H. Hiraka, M. Matsuda, M. Matsuura, J. M.
Tranquada, S. Wakimoto, G. Xu, and K. Yamada, J. Phys. Soc. Jpan. {\bf 81}, 011007 (2012).

\bibitem{Bednorz86} J. G. Bednorz and K. A. M\"uller, Z. Phys. B {\bf 64}, 189 (1986).

\bibitem{Vishik18} See, e.g., the review, I. M. Vishik, Rep. Prog. Phys. {\bf 81}, 062501
(2018).

\bibitem{Campuzano04} See, e.g., the review, J. C. Campuzano, M. R. Norman, M. Randeira, in
{\it Physics of Superconductors}, vol. II, edited by K. H. Bennemann and J. B. Ketterson
(Springer, Berlin Heidelberg New York, 2004), p. 167.

\bibitem{Damascelli03} See, e.g., the review, A. Damascelli, Z. Hussain, and Z.-X. Shen, Rev.
Mod. Phys. {\bf 75}, 473 (2003).

\bibitem{Fink07} See, e.g., the review, J. Fink, S. Borisenko, A. Kordyuk, A. Koitzsch, J.
Geck, V. Zabalotnyy, M. Knupfer, B. Buechner, and H. Berger, in {\it Lecture Notes in Physics},
vol. 715, edited by S. H\"ufner (Springer-Verlag Berlin Heidelberg, 2007), p. 295.

\bibitem{Keimer15} See, e.g., the review, B. Keimer, S. A. Kivelson, M. R. Norman, S. Uchida,
and J. Zaanen, Nature {\bf 518}, 179 (2015).

\bibitem{Hussey08} See, e.g., the review, N. E. Hussey, J. Phys.: Condens. Matter {\bf 20},
123201 (2008).

\bibitem{Timusk99} See, e.g., the review, T. Timusk and B. Statt, Rep. Prog. Phys.
{\bf 62}, 61 (1999).

\bibitem{Kastner98} See, e.g., the review, M. A. Kastner, R. J. Birgeneau, G. Shirane, and Y.
Endoh, Rev. Mod. Phys. {\bf 70}, 897 (1998).

\bibitem{Schrieffer64} See, e.g., J. R. Schrieffer, {\it Theory of Superconductivity}, Benjamin,
New York, 1964.

\bibitem{Abrikosov88} See, e.g., A. A. Abrikosov, {\it Fundamentals of the Theory of Metals},
Elsevier Science Publishers B. V., 1988.

\bibitem{Mahan81} See, e.g., G. D. Mahan, {\it Many-Particle Physics}, (Plenum Press, New York,
1981).

\bibitem{Norman98} M. R. Norman, H. Ding, M. Randeria, J. C. Campuzano, T. Yokoya, T. Takeuchi,
T. Takahashi, T. Mochiku, K. Kadowaki, P. Guptasarma, and D. G. Hinks, Nature {\bf 392}, 157
(1998).

\bibitem{Shi08} M. Shi, J. Chang, S. Pailh\'es, M. R. Norman, J. C. Campuzano, M. M\'ansson,
T. Claesson, O. Tjernberg, A. Bendounan, L. Patthey, N. Momono, M. Oda, M. Ido, C. Mudry,
and J. Mesot, Phys. Rev. Lett. {\bf 101}, 047002 (2008).

\bibitem{Sassa11} Y. Sassa, M. Radovi\'c, M. M\'ansson, E. Razzoli, X. Y. Cui, S. Pailh\'es,
S. Guerrero, M. Shi, P. R. Willmott, F. Miletto Granozio, J. Mesot, M. R. Norman, and L.
Patthey, Phys. Rev. B {\bf 83}, 140511(R) (2011).

\bibitem{Comin14} R. Comin, A. Frano, M. M. Yee, Y. Yoshida, H. Eisaki, E. Schierle, E.
Weschke, R. Sutarto, F. He, A. Soumyanarayanan, Yang He, M. L. Tacon, I. S. Elfimov,
Jennifer E. Hoffman, G. A. Sawatzky, B. Keimer, and A. Damascelli, Science {\bf 343}, 390
(2014).

\bibitem{Horio16} M. Horio, T. Adachi, Y. Mori, A. Takahashi, T. Yoshida, H. Suzuki, L. C.
C. Ambolode II, K. Okazaki, K. Ono, H. Kumigashira, H. Anzai, M. Arita, H. Namatame, M.
Taniguchi, D. Ootsuki, K. Sawada, M. Takahashi, T. Mizokawa, Y. Koike, and A. Fujimori, Nat.
Commun. {\bf 7}, 10567 (2016).

\bibitem{Loret18} B. Loret, Y. Gallais, M. Cazayous, R. D. Zhong, J. Schneeloch, G. D. Gu,
A. Fedorov, T. K. Kim, S. V. Borisenko, and A. Sacuto, Phys. Rev. B {\bf 97}, 174521 (2018).

\bibitem{Dessau91} D. S. Dessau, B. O. Wells, Z.-X. Shen, W. E. Spicer, A. J. Arko, R. S. List,
D. B. Mitzi, and A. Kapitulnik, Phys. Rev. Lett. {\bf 66}, 2160 (1991).

\bibitem{Norman97} M. R. Norman, H. Ding, J. C. Campuzano, T. Takeuchi, M. Randeria, T. Yokoya,
T. Takahashi, T. Mochiku, and K. Kadowaki, Phys. Rev. Lett. {\bf 79}, 3506 (1997).

\bibitem{Campuzano99} J. C. Campuzano, H. Ding, M. R. Norman, H. M. Fretwell, M. Randeria, A. Kaminski,
J. Mesot, T. Takeuchi, T. Sato, T. Yokoya, T. Takahashi, T. Mochiku, K. Kadowaki, P. Guptasarma,
D. G. Hinks, Z. Konstantinovic, Z. Z. Li, and H. Raffy, Phys. Rev. Lett. {\bf 83}, 3709 (1999).

\bibitem{Wei08} J. Wei, Y. Zhang, H. W. Ou, B. P. Xie, D. W. Shen, J. F. Zhao, L. X. Yang, M. Arita,
K. Shimada, H. Namatame, M. Taniguchi, Y. Yoshida, H. Eisaki, and D. L. Feng, Phys. Rev. Lett.
{\bf 101}, 097005 (2008).

\bibitem{DMou17} Daixiang Mou, Adam Kaminski, and Genda Gu, Phys. Rev. B {\bf 95}, 174501 (2017).

\bibitem{Kaminski01} A. Kaminski, M. Randeria, J. C. Campuzano, M. R. Norman, H. Fretwell, J.
Mesot, T. Sato, T. Takahashi, and K. Kadowaki, Phys. Rev. Lett. {\bf 86}, 1070 (2001).

\bibitem{Zhou03} X. J. Zhou, T. Yoshida, A. Lanzara, P. V. Bogdanov, S. A. Kellar, K. M. Shen,
W. L. Yang, F. Ronning, T. Sasagawa, T. Kakeshita, T. Noda, H. Eisaki, S. Uchida, C. T. Lin,
F. Zhou, J. W. Xiong, W. X. Ti, Z. X. Zhao, A. Fujimori, Z. Hussain, and Z.-X. Shen, Nature
{\bf 423}, 398 (2003).

\bibitem{Anzai10} H. Anzai, A. Ino, T. Kamo, T. Fujita, M. Arita, H. Namatame, M. Taniguchi,
A. Fujimori, Z.-X. Shen, M. Ishikado, and S. Uchida, Phys. Rev. Lett. {\bf 105}, 227002
(2010).

\bibitem{He13} J. He, W. Zhang, J. M. Bok, D. Mou, L. Zhao, Y. Peng, S. He, G. Liu, X. Dong,
J. Zhang, J. S. Wen, Z. J. Xu, G. D. Gu, X..Wang, Q. Peng, Z. Wang, S. Zhang, F. Yang, C.
Chen, Z. Xu, H.-Y. Choi, C. M. Varma, and X. J. Zhou, Phys. Rev. Lett. {\bf 111}, 107005
(2013).

\bibitem{Yang19} S.-L. Yang, J. A. Sobota, Y. He, D. Leuenberger, H. Soifer, H. Eisaki,
P. S. Kirchmann, and Z.-X. Shen, Phys. Rev. Lett. {\bf 122}, 176403 (2019).

\bibitem {Allen89} See, e.g., the review, P. B. Allen, Z. Fisk, and A. Migliori, in
{\it Physical Properties of High Temperature Superconductors} I, edited by D. M. Ginsberg
(World Scientific, Singapore, 1989), p. 213.

\bibitem{Gurvitch87} M. Gurvitch and A. T. Fiory, Phys. Rev. Lett. {\bf 59}, 1337 (1987).

\bibitem{Takagi92} H. Takagi, B. Batlogg, H. L. Kao, J. Kwo, R. J. Cava, J. J. Krajewski, and
W. F. Peck, Jr., Phys. Rev. Lett. {\bf 69}, 2975 (1992).

\bibitem{Martin90} S. Martin, A. T. Fiory, R. M. Fleming, L. F. Schneemeyer, and J. V.
Waszczak, Phys. Rev. B {\bf 41}, 846(R) (1990).

\bibitem{Mandrus92} D. Mandrus, L. Forro, C. Kendziora, and L. Mihaly, Phys. Rev. B {\bf 45},
12640(R) (1992).

\bibitem{Ando01} Y. Ando, A. N. Lavrov, S. Komiya, K. Segawa, and X. F. Sun, Phys. Rev. Lett.
{\bf 87}, 017001 (2001).

\bibitem{Daou09} R. Daou, N. Doiron-Leyraud, D. LeBoeuf, S. Y. Li, F. Lalibert\'e, O.
Cyr-Choini\'ere, Y. J. Jo, L. Balicas, J.-Q. Yan, J.-S. Zhou, J. B. Goodenough, and L.
Taillefer, Nat. Phys. {\bf 5}, 31 (2009).

\bibitem{Cooper09} R. A. Cooper, Y. Wang, B. Vignolle, O. J. Lipscombe, S. M. Hayden, Y.
Tanabe, T. Adachi, Y. Koike, M. Nohara, H. Takagi, C. Proust, N. E. Hussey, Science {\bf 323},
603 (2009).

\bibitem{Legros19} A. Legros, S. Benhabib, W. Tabis, F. Lalibert\'e, M. Dion, M. Lizaire, B.
Vignolle, D. Vignolles, H. Raffy, Z. Z. Li, P. Auban-Senzier, N. Doiron-Leyraud, P. Fournier,
D. Colson, L. Taillefer, and C. Proust, Nat. Phys. {\bf 15}, 142 (2019).

\bibitem{Yuan22} J. Yuan, Q. Chen, K. Jiang, Z. Feng, Z. Lin, H. Yu, G. He, J. Zhang, X.
Jiang, X. Zhang, Y. Shi, Y. Zhang, M. Qin, Z. Cheng, N. Tamura, Y.-F. Yang, T. Xiang, J.
Hu, I. Takeuchi, K. Jin, and Z. Zhao, Nature {\bf 602}, 431 (2022).

\bibitem{Ayres21} J. Ayres, M. Berben, M. \'Culo, Y.-T. Hsu, E. van Heumen, Y. Huang, J.
Zaanen, T. Kondo, T. Takeuchi, J. R. Cooper, C. Putzke, S. Friedemann, A. Carrington, and
N. E. Hussey, Nature {\bf 595}, 661 (2021).

\bibitem{Grisso21} G. Grissonnanche, Y. Fang, A. Legros, S. Verret, F. Lalibert\'e, C.
Collignon, J. Zhou, D. Graf, P. A. Goddard, L. Taillefer, and B. J. Ramshaw, Nature
{\bf 595}, 667 (2021).

\bibitem{Varma89} C. M. Varma, P. B. Littlewood, S. Schmitt-Rink, E. Abrahams, and A. E.
Ruckenstein, Phys. Rev. Lett. {\bf 63}, 1996 (1989).

\bibitem{Varma16} See, e.g., the review, C. M Varma, Rep. Prog. Phys. {\bf 79} 082501 (2016)

\bibitem{Varma20} See, e.g., the review, C. M. Varma, Rev. Mod. Phys. {\bf 92}, 031001 (2020).

\bibitem{Damle97} K. Damle and S. Sachdev, Phys. Rev. B {\bf 56}, 8714 (1997).

\bibitem{Sachdev11} S. Sachdev, {\it Quantum Phase Transitions}, (Cambridge University Press,
1999).

\bibitem{Zaanen04} J. Zaanen, Nature {\bf 430}, 512 (2004).

\bibitem{Luca07} L. Dell'Anna and W. Metzner, Phys. Rev. Lett. {\bf 98}, 136402 (2007).

\bibitem{Haldane18} F. D. M. Haldane, arXiv:1811.12120.

\bibitem{Zaanen19} See, e.g., the review, J. Zaanen, SciPost Phys. {\bf 6}, 061 (2019).

\bibitem{Hussey03} N. E. Hussey, Eur. Phys. J. B {\bf 31}, 495 (2003).

\bibitem{Rice17} T. M. Rice, N. J. Robinson, and A. M. Tsvelik, Phys. Rev. B {\bf 96},
220502(R) (2017).

\bibitem{Lee21} P. A. Lee, Phys. Rev. B {\bf 104}, 035140 (2021).

\bibitem{Honerkamp01} C. Honerkamp, M. Salmhofer, N. Furukawa, and T. M. Rice, Phys. Rev. B
{\bf 63}, 035109 (2001).

\bibitem{Hartnoll12} S. A. Hartnoll and D. M. Hofman, Phys. Rev. Lett. {\bf 108}, 241601
(2012).

\bibitem{Tabis21} W. Tabi\'{s}, P. Pop\v{c}evi\'{c}, B. Klebel-Knobloch, I. Bialo, C. M. N.
Kumar, B. Vignolle, M. Greven, N. Bari\v{s}i\'{c}, arXiv:2106.07457.

\bibitem{Liu21} Y. Liu, Y. Lan, and S. Feng, Phys. Rev. B {\bf 103}, 024525 (2021); S. Tan,
Y. Liu, Y. Mou, and S. Feng, Phys. Rev. B {\bf 103}, 014503 (2021).

\bibitem{Cao21} Z. Cao, X. Ma, Y. Liu, H. Guo, and S. Feng, Phys. Rev. B {\bf 104}, 224503
(2021); Z. Cao, Y. Liu, H. Guo, and S. Feng, Phil. Mag. {\bf 102}, 918 (2022).

\bibitem{Zeng22} M. Zeng, X. Li, Y. Wang, and S. Feng, Phys. Rev. B {\bf 106}, 054512 (2022).

\bibitem {Feng16} S. Feng, D. Gao, and H. Zhao, Phil. Mag. {\bf 96}, 1245 (2016); H. Zhao,
D. Gao, and S. Feng, Physica C {\bf 534}, 1 (2017); X. Ma, Z. Cao, and S. Feng, unpublished.

\bibitem{Anderson87} P. W. Anderson, Science {\bf 235}, 1196 (1987).

\bibitem {Yu92} See, e.g., the review, L. Yu, in {\it Recent Progress in Many-Body Theories},
edited by T. L. Ainsworth, C. E. Campbell, B. E. Clements, and E. Krotscheck (Plenum, New York,
1992), Vol. {\bf 3}, p. 157.

\bibitem {Lee06} See, e.g., the review, P. A. Lee, N. Nagaosa, and X.-G. Wen, Rev. Mod. Phys.
{\bf 78}, 17 (2006).

\bibitem{Edegger07} See, e.g., the review, B. Edegger, V. N. Muthukumar, and C. Gros, Adv.
Phys. {\bf 56}, 927 (2007).

\bibitem {Spalek22} J. Spalek, M. Fidrysiak, M. Zegrodnik, A. Biborski, Phys. Rep. {\bf 959},
1 (2022).

\bibitem {Zhang93} L. Zhang, J. K. Jain, and V. J. Emery, Phys. Rev. B {\bf 47}, 3368 (1993).

\bibitem {Feng0494} S. Feng, J. Qin, and T. Ma, J. Phys.: Condens. Matter {\bf 16}, 343
(2004); S. Feng, Z. B. Su, and L. Yu, Phys. Rev. B {\bf 49}, 2368 (1994).

\bibitem{Feng15} See, e.g., the review, S. Feng, Y. Lan, H. Zhao, L. Kuang, L. Qin, and
X. Ma, Int. J. Mod. Phys. B {\bf 29}, 1530009 (2015).


\bibitem{Feng0306} S. Feng, Phys. Rev. B {\bf 68}, 184501 (2003); S. Feng, T. Ma, and H.
Guo, Physica C {\bf 436}, 14 (2006).

\bibitem{Feng12} S. Feng, H. Zhao, and Z. Huang, Phys. Rev. B. {\bf 85}, 054509 (2012);
Phys. Rev. B {\bf 85}, 099902(E) (2012).

\bibitem{Feng15a} S. Feng, L. Kuang, and H. Zhao, Physica C {\bf 517}, 5 (2015).

\bibitem{Chatterjee06} U. Chatterjee, M. Shi, A. Kaminski, A. Kanigel, H. M. Fretwell, K.
Terashima, T. Takahashi, S. Rosenkranz, Z. Z. Li, H. Raffy, A. Santander-Syro, K. Kadowaki,
M. R. Norman, M. Randeria, and J. C. Campuzano, Phys. Rev. Lett. {\bf 96}, 107006 (2006).

\bibitem{He14} Y. He, Y. Yin, M. Zech, A. Soumyanarayanan, M. M. Yee, T. Williams, M. C.
Boyer, K. Chatterjee, W. D. Wise, I. Zeljkovic, T. Kondo, T. Takeuchi, H. Ikuta, P. Mistark,
R. S. Markiewicz, A. Bansil, S. Sachdev, E. W. Hudson, and J. E. Hoffman, Science {\bf 344},
608 (2014).

\bibitem{Comin16} See, e.g., the review, Riccardo Comin and Andrea Damascelli, Annu. Rev.
Condens. Matter Phys. {\bf 7}, 369 (2016).

\bibitem{Yin21} See, e.g., the review, J.-X. Yin, S. H. Pan, M. Z. Hasan, Nat. Rev. Phys.
{\bf 3}, 249 (2021).

\bibitem{Pan01} S. H. Pan, J. P. \'ONeal, R. L. Badzey, C. Chamon, H. Ding, J. R. Engelbrecht,
Z. Wang, H. Eisaki, S. Uchida, A. K. Gupta, K.-W. Ng, E. W. Hudson, K. M. Lang, and J. C. Davis,
Nature {\bf 413}, 282 (2001).

\bibitem{Fujita19} S. Mukhopadhyay, R. Sharma, C. K. Kim, S. D. Edkins, M. H. Hamidian, H. Eisaki,
S. Uchida, E.-A. Kim, M. J. Lawler, A. P. Mackenzie, J. C. S. Davis, and K. Fujita, Proc. Natl.
Acad. Sci. {\bf 116}, 13249 (2019).

\bibitem{Dessau93} D. S. Dessau, Z.-X. Shen, D. M. King, D. S. Marshall, L. W. Lombardo, P. H.
Dickinson, A. G. Loeser, J. DiCarlo, C.-H Park, A. Kapitulnik, and W. E. Spicer, Phys. Rev.
Lett. {\bf 71}, 2781 (1993).

\bibitem{Prange64} R. E. Prange and L. P. Kadanoff, Phys. Rev. {\bf 134}, A566 (1964).

\bibitem{Bruin13} J. A. N. Bruin, H. Sakair, R. S. Perrya, A. P. Mackenzie, Science {\bf 339},
804 (2013).

\bibitem{Grigera01} S. A. Grigera, R. S. Perry, A. J. Schofield, M. Chiao, S. R. Julian, G. G.
Lonzarich, S. I. Ikeda, Y. Maeno, A. J. Millis, and A. P. Mackenzie, Science {\bf 294}, 329
(2001).

\bibitem{Ma22} X. Ma, M. Zeng, and S. Feng, unpubliahed.

\bibitem{Mou19} Y. Mou, Y. Liu, S. Tan, and S. Feng, Phil. Mag. {\bf 99}, 2718 (2019).

\end{thebibliography}

\end{document}
