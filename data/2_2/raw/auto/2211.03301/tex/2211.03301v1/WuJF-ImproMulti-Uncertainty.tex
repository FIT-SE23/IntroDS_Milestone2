\documentclass[preprint]{revtex4}

\usepackage{amsmath}
\usepackage{mathrsfs}
\usepackage{comment}
\usepackage{amssymb}
\usepackage{graphicx}
\usepackage[colorlinks,
            linkcolor=blue,
            anchorcolor=blue,
            citecolor=blue]{hyperref}
\newtheorem{theorem}{Theorem}
\newtheorem{corollary}{Corollary}
\newtheorem{proposition}{Proposition}
\newtheorem{example}{Example}
\def\ra{\rangle}
\def\la{\langle}
\allowdisplaybreaks[4]


\begin{document}
\title{Parameterized Multi-observable Sum Uncertainty Relations}
\author{Jing-Feng Wu$^{1}$}
\author{Qing-Hua Zhang$^{1,}$\footnotemark[1]}
\author{Shao-Ming Fei$^{1,2,}$\footnotemark[1]}

\affiliation{$^1$School of Mathematical Sciences, Capital Normal University, 100048
Beijing, China\\
$^2$Max-Planck-Institute for Mathematics in the Sciences, 04103 Leipzig, Germany}

\renewcommand{\thefootnote}{\fnsymbol{footnote}}

\footnotetext[1]{Corresponding authors. \\
\href{mailto:2190501022@cnu.edu.cn}{2190501022@cnu.edu.cn(Q. H. Zhang)}.\\
\href{mailto:feishm@cnu.edu.cn}{feishm@cnu.edu.cn(S. M. Fei)}.}
\bigskip

\bigskip

\begin{abstract}
The uncertainty principle is one of the fundamental features of quantum mechanics and plays an essential role in quantum information theory. We study uncertainty relations based on variance for arbitrary finite $N$ quantum observables. We establish a series of parameterized uncertainty relations in terms of the parameterized norm inequalities, which improve the exiting variance-based uncertainty relations. The lower bounds of our uncertainty inequalities are non-zero unless the measured state is the common eigenvector of all the observables. Detailed examples are provided to illustrate the tightness of our uncertainty relations.
\end{abstract}

\maketitle

\section{Introduction}\label{Sec1}
The uncertainty principle is one of the cornerstones of quantum mechanics, which reveals the intrinsic differences between classical and quantum world. The uncertainty principle was firstly introduced by Heisenberg in 1927 \cite{Heisenberg1927}. It shows that if one measures the momentum of a particle with certainty, one can not measure its position with certainty at the same time. Since then a lot of research works have been dedicated to interpret the uncertainty relations via different forms, such as in terms of quantum variance \cite{Nielsen2002Quantum,PhysRev.34.163,schrodinger1930sitzungsberichte,PhysRevLett.113.260401,Kennard1927Zur,Schrodinger1930Zum,Robertson1929The,mondal2017tighter,PhysRevResearch.4.013076,PhysRevResearch.4.013075}, entropy \cite{Maassen1988PhysRevLett.60.1103,Wu2009PhysRevA.79.022104,Coles2017RevModPhys.89.015002}, noise and disturbance \cite{buschPhysRevLett.111.160405}, successive measurement \cite{DeutschPhysRevLett.50.631,DistlerPhysRevA.87.062112}, majorization technique \cite{Pucha_a_2013,friedland2013universal}, skew information \cite{luoPhysRevLett.91.180403,Zhang_2021note,zhang2021tighter,ma2022product} etc. These uncertainty relations play an important role in a wide range of applications such as entanglement detection \cite{PhysRevLett.92.117903, zhang2020sufficient,zhang2021multipartite}, quantum metrology \cite{PhysRevLett.96.010401}, quantum steering \cite{SchneelochPhysRevA.87.062103}, quantum gravity \cite{hall2005exact} and quantum cryptography \cite{Fuchs1996Quantum}.

Robertson \cite{PhysRev.34.163} generalized the uncertainty relation for position and momentum to any two observables $A$ and $B$,
\begin{equation}\label{Eq1}
  \Delta A  \Delta B \geq \frac{1}{2} |\la \psi | [A, B] | \psi \ra |,
\end{equation}
where $\Delta$ stands for the standard deviation of the observable with respect to a fixed state $|\psi\ra$ and $[A, B]$ represents the commutator of the observables $A$ and $B$. Eq.~(\ref{Eq1}) was later improved by Schr\"{o}dinger \cite{schrodinger1930sitzungsberichte},
\begin{equation}\label{schrodinger}
\Delta A\Delta B \geq \frac{1}{2}\sqrt{|\langle\{A, B\}\rangle-\langle A\rangle\langle B\rangle |^{2}+|\langle[A, B]\rangle |^{2}}.
\end{equation}
Here, the lower bounds of the inequalities (\ref{Eq1}) and (\ref{schrodinger}) may vanish even if the observables $A$ and $B$ are not commutative. For instance, when the measured state $|\psi\ra$ is an eigenvector of either $A$ or $B$, the right hands of the inequality (\ref{Eq1}) and (\ref{schrodinger}) are trivially zero. To overcome the flaw, uncertainty relations with respect to the sum of variances have been presented by Maccone and Pati \cite{PhysRevLett.113.260401},
\begin{align}
\label{Pati1}  \Delta^2  A + \Delta^2 B &\geq \pm \la \psi | [A, B] | \psi \ra + | \la \psi | A \pm i B | \psi^{\perp} \ra |^2,\\
\label{Pati2}                  \Delta^2  &A + \Delta^2 B \geq \frac{1}{2} \Delta^2(A + B),
\end{align}
where the signs $\pm$ on the right-hand side of (\ref{Pati1}) are taken so that the $\pm \la \psi | [A, B] | \psi \ra$ is positive, $|\psi^{\perp}\ra$ satisfies $\la \psi| \psi^{\perp} \ra = 0$. The lower bound in (\ref{Pati1}) is nonzero for most choice of $|\psi^{\perp}\ra$ if $A$ and $B$ are not commutative.

Besides the variance-based uncertainty relations with respect to pairs of non-commutative observables, the uncertainty relations related to three non-commutative observables such as the three components of spins and angular momentums \cite{kechrimparis2014heisenberg,dammeier2015uncertainty,ma2017experimental} have been also investigated. The variance-based uncertainty relations for general multiple observables have been further studied either in summation form \cite{chen2016variance, chen2015sum, chen2019tight} or in product form \cite{qin2016multi,xiao2016mutually}. For instance, Song $et\ al.$ derived a tighter variance-based uncertainty relation in \cite{song2017stronger},
\begin{equation}\label{Song_Eq}
\sum_{i = 1}^N \Delta_{\rho}^2(A_i) \geq \frac{1}{N} \Delta_{\rho}^2  (\sum_{i = 1}^N A_i  )
+ \frac{2}{N^2(N - 1)}  \Big [ \sum_{1 \leq i <j \leq N} \Delta_{\rho}(A_i - A_j)  \Big ]^2.
\end{equation}
Recently, based on the inequalities of vector norm, Zhang $et\ al.$ \cite{zhang2022note} proposed an improved variance-based sum uncertainty
relation for $N$ arbitrary incompatible observables,
\begin{equation}\label{Zhang_Eq} %=====label zhang et al.'s conclusion=====%
\begin{aligned}
\sum_{i = 1}^N \Delta_{\rho}^2(A_i) \geq \max_{x \in \{0, 1\}}
&\frac{1}{2 N - 2}  \bigg\{ \frac{2}{N(N - 1)}  \Big [ \sum_{1 \leq i <j \leq N} \Delta_{\rho}(A_i + (- 1)^x A_j)  \Big ]^2 \\
&+ \sum_{1 \leq i < j \leq N} \Delta_{\rho}^2(A_i + (- 1)^{x + 1} A_j )  \bigg\}.
\end{aligned}
\end{equation}

This paper is aimed to improve these uncertainty relations for $N$ arbitrary observables. Motivated by the skew information-based uncertainty relations proposed in \cite{zhang2022note} and \cite{li2022metric}, we combine the parameterized parallelogram law of vector norm with Cauchy-Schwarz inequality to improve the lower bounds of uncertainty relations for $N$ observables.

%In Sec.~\ref{Sec2}, we introduce our main result, two tighter uncertainty relation, which based on variance for finite quantum observables. %In addition, three detail examples are given to show our Theorem provides tighter bounds than others. Concluding remarks are given in %Sec.~\ref{Sec3}.

\section{uncertainty relations via variance}\label{Sec2}
Denote $H_d$ the Hilbert space with $d$ dimension. The quantum variance of any quantum state $\rho\in H_d$ related to an observable $M$ is defined by
\begin{equation}
\begin{aligned}
\Delta^2_{\rho} (M)&={\rm Tr} (\rho M^2)-[{\rm Tr} (\rho M)]^2\\
&=\la \sqrt{\rho}| I_d\otimes (\delta M)^2 |\sqrt{\rho}\ra\\
&=\| I_d\otimes \delta M |\sqrt{\rho}\ra\|^2,
\end{aligned}
\end{equation}
where $I_d$ is the identity operator in $H_d$, $\delta M=M-{\rm Tr}(\rho M)$, $|\sqrt{\rho}\ra$ denotes the vectorization of $\sqrt{\rho}$ and
$\|\cdot\|$ denotes the $2$-norm of a vector. By using the parallelogram law of vector $2$-norm,
\begin{equation}
(2N-2) \sum_{i=1}^{N} \| a_i\|^2 = \sum_{1\leq i<j \leq N} \| a_i+a_j \|^2 + \sum_{1\leq i<j \leq N} \| a_i-a_j \|^2
\end{equation}
for a set of vectors $a_i$, $i=1,...,N$, we have the following uncertainty relation.

%====================Theorem1====================%
\begin{theorem} \label{Thm1}
For $N$ arbitrary observables $A_1, A_2, \dots, A_N$, the following  variance-based sum uncertainty relation holds for any quantum state $\rho$,
\begin{equation}\label{Thm1_Eq1}
\begin{aligned}
\sum_{i=1}^N \Delta_{\rho}^2 (A_i) \geq \rm{LB1}=\mathop{ \max_{ { x \in \{ 0, 1\} } \atop { y \in \{ 0, 1\} } } }
&\frac{1}{ (1 + \alpha^2) (N - 1) }  \bigg\{ \frac{2} { N (N -1) } \Big [ \sum_{ 1 \leq i < j \leq N } \Delta_{\rho} ( \alpha^{1 - x} A_i  + (- 1)^y \alpha^x A_j) \Big ]^2   \\
&  + \sum_{ 1 \leq i < j \leq N } \Delta_{\rho}^2 ( \alpha^ x A_i + (- 1)^{1 - y} \alpha^{1 - x} A_j) \bigg\},
\end{aligned}
\end{equation}
where $\alpha$ is any non-negative real number.
\end{theorem}

{\sf [Proof]} For all $x \in \{0, 1\}$ and $y\in \{0, 1\}$, the following parameterized parallelogram equality holds for $2$-norm of any vectors $a_i$,
\begin{equation}
\begin{aligned}
\sum_{ i = 1}^N \| a_i \|^2 = &\frac{1}{ ( 1 + \alpha^2 ) ( N - 1 ) }  \Big [ \sum_{ 1 \leq i < j \leq N } \| \alpha^{1 - x} a_i + (- 1)^y \alpha^x a_j \|^2  \\
&  + \sum_{ 1 \leq i < j \leq N } \| \alpha^x a_i + (- 1)^{1 - y} \alpha^{1 - x} a_j \|^2  \Big ].
\end{aligned}
\end{equation}
Using the Cauchy-Schwarz inequality,
\begin{equation}\label{cauchy}
\sum_{ 1 \leq i < j \leq N } \| \alpha^{1 - x} a_i + (- 1)^y \alpha^x a_j \|^2 \geq \frac{2}{ N ( N - 1 ) }  \Big [ \sum_{ 1 \leq i < j \leq N } \| \alpha^{1 - x} a_i + (- 1)^y \alpha^x a_j \|  \Big ]^2,
\end{equation}
we obtain
\begin{equation}
\begin{aligned}
\sum_{ i = 1 }^N \|a_i\|^2 \geq
&\frac{1}{ (1 + \alpha^2) (N - 1) }  \bigg\{ \frac{2} { N (N -1) }  \Big [ \sum_{ 1 \leq i < j \leq N } \| \alpha^{1 - x} a_i  + (- 1)^y \alpha^x a_j \|  \Big ]^2   \\
&  + \sum_{ 1 \leq i < j \leq N } \| \alpha^ x a_i + (- 1)^{1 - y} \alpha^{1 - x} a_j \|^2  \bigg\}.
\end{aligned}
\end{equation}
Set $\| a_i \|=\| I_d\otimes \delta A_i |\sqrt{\rho}\ra\|=\Delta_{\rho} A_i$ and $\| \alpha^{1 - x} a_i  + (- 1)^y \alpha^x a_j \|=\Delta_{\rho} ( \alpha^{1 - x} A_i  + (- 1)^y \alpha^x A_j) $. We have
\begin{equation}
\begin{aligned}
\sum_{ i = 1 }^N \Delta_{\rho}^2 (A_i) \geq
&\frac{1}{ (1 + \alpha^2) (N - 1) }  \bigg\{ \frac{2} { N (N -1) }  \Big [ \sum_{ 1 \leq i < j \leq N } \Delta_{\rho} ( \alpha^{1 - x} A_i  + (- 1)^y \alpha^x A_j)  \Big ]^2   \\
&  + \sum_{ 1 \leq i < j \leq N } \Delta_{\rho}^2 ( \alpha^ x A_i + (- 1)^{1 - y} \alpha^{1 - x} A_j)  \bigg\}.
\end{aligned}
\end{equation}
Namely, $\sum_{ i = 1 }^N \Delta_{\rho}^2 (A_i) \geq \rm{LB1}$. $\Box$

Theorem \ref{Thm1} provides a series of uncertainty relations depending on the values of the parameter $\alpha$. The uncertainty relations (\ref{Zhang_Eq}) is a special case of Theorem \ref{Thm1} when $\alpha=1$. Note that the lower bound of Theorem \ref{Thm1} is non-zero unless the measured state $|\psi\rangle$ is the common eigenvector of all $A_i$. That is to say, no matter whether the observables are commutable or not, the lower bound of Theorem \ref{Thm1} does not vanish if $|\psi\rangle$ is not the common eigenvector of all observables.

In fact, the lower bound in Theorem 1 should be understood under the permutation of the observables. Let $\pi\in S(N)$ be an arbitrary $N$-element permutation. Define
\begin{equation}
\begin{aligned}
{\rm LB1}_\pi=\mathop{ \max_{ { x \in \{ 0, 1\} } \atop { y \in \{ 0, 1\} } } }
&\frac{1}{ (1 + \alpha^2) (N - 1) }  \bigg\{ \frac{2} { N (N -1) } \Big [ \sum_{ 1 \leq i < j \leq N } \Delta_{\rho} ( \alpha^{1 - x} A_{\pi(i)}  + (- 1)^y \alpha^x A_{\pi(j)} ) \Big ]^2   \\
&  + \sum_{ 1 \leq i < j \leq N } \Delta_{\rho}^2 ( \alpha^ x A_{\pi(i)}  + (- 1)^{1 - y} \alpha^{1 - x} A_{\pi(j)} ) \bigg\}.
\end{aligned}
\end{equation}
The following variance-based uncertainty relation under the element permutation of all observables holds,
\begin{equation}
\sum_{i=1}^N \Delta_{\rho}^2 (A_i) \geq \max_{\pi\in S(N)}{\rm LB1}_\pi.
\end{equation}

From the following equalities,
\begin{equation}
\sum_{ 1 \leq i < j \leq N } \| a_i + a_j \|^2 =   \| \sum_{i = 1}^N a_i  \|^2 + (N - 2) \sum_{i = 1}^N \| a_i \|^2
\end{equation}
and
\begin{equation}
\sum_{ 1 \leq i < j \leq N } \| a_i - a_j \|^2 =  N \sum_{i = 1}^N \| a_i \|^2 -  \| \sum_{i = 1}^N a_i  \|^2,
\end{equation}
one has \cite{li2022metric},
\begin{equation}\label{li2022}
\begin{aligned}
 \left[\alpha N+(N-2)\beta\right] \sum_{i = 1}^N \|a_i\|^2=
&\beta \sum_{ 1 \leq i < j \leq N } \| a_i + a_j \|^2+ \alpha \sum_{ 1 \leq i < j \leq N } \| a_i - a_j \|^2   \\
&+ ( \alpha - \beta )  \| \sum_{i=1}^N a_i  \|^2 ,
\end{aligned}
\end{equation}
where both $\alpha,\beta$ are arbitrary real numbers.

%====================Theorem2====================%
\begin{theorem}\label{Thm2}
Let $A_1, A_2, \dots, A_N$ be $N$ arbitrary observables. For any quantum state $\rho$, we have the following uncertainty relation satisfied by quantum variances,
\begin{equation}
\sum_{i=1}^N \Delta_{\rho}^2(A_i) \geq \rm{LB2}=\max \{ \rm{X}, \rm{Y}, \rm{Z} \},
\end{equation}
where
\begin{equation}\label{Thm2_Eq1}
\begin{aligned}
\rm{X} =
&\frac{1}{ \alpha N + ( N - 2 ) \beta }  \bigg\{ \frac{ 2\beta }{ N ( N - 1 ) }  \Big [ \sum_{ 1 \leq i < j \leq N } \Delta_{\rho} ( A_i + A_j )  \Big ]^2  \\
&  + \alpha \sum_{ 1 \leq i < j \leq N } \Delta_{\rho}^2 ( A_i - A_j ) + ( \alpha - \beta ) \Delta_{\rho}^2 ( \sum_{ i =1}^{ N } A_i )  \bigg\}
\end{aligned}
\end{equation}
and
\begin{equation}\label{Thm2_Eq2}
\begin{aligned}
\rm{Y} =
&\frac{1}{ \alpha N + ( N - 2 ) \beta }  \bigg\{ \frac{ 2\alpha }{ N ( N - 1 ) } \Big [ \sum_{ 1 \leq i < j \leq N } \Delta_{\rho} ( A_i - A_j ) \Big ]^2  \\
&  + \beta \sum_{ 1 \leq i < j \leq N } \Delta_{\rho}^2 ( A_i + A_j ) + ( \alpha - \beta ) \Delta_{\rho}^2 ( \sum_{ i =1}^{ N } A_i )  \bigg\}
\end{aligned}
\end{equation}
for $ \alpha,\beta  > 0$,
\begin{equation}\label{Thm2_Eq3}
\begin{aligned}
\rm{Z} =
&\frac{1}{ \alpha N + ( N - 2 ) \beta }  \bigg\{ \beta \sum_{ 1 \leq i < j \leq N } \Delta_{\rho}^2 ( A_i + A_j )   \\
&  + \alpha \sum_{ 1 \leq i < j \leq N } \Delta_{\rho}^2 ( A_i - A_j )
 + \frac{ \alpha - \beta }{ ( N - 1 )^2 } \Big [\sum_{ 1 \leq i < j \leq N } \Delta_{\rho} ( A_i + A_j ) \Big ]^2  \bigg\}
\end{aligned}
\end{equation}
for $\beta > \alpha > 0$.
\end{theorem}

{\sf [Proof]} For all $ \alpha,\beta  >0$, by using (\ref{li2022}) and the Cauchy-Schwarz inequality (\ref{cauchy}), we get
\begin{equation}\label{Thm2_Pf_Eq1}
\begin{aligned}
\sum_{i=1}^N \|a_i\|^2 \geq
&\frac{1}{ \alpha N + (N - 2)\beta}  \Big [ \frac{2\beta}{N(N - 1)} ( \sum_{1 \leq i < j \leq N} \|a_i + a_j \| )^2  \\
& + \alpha\sum_{1 \leq i < j \leq N} \|a_i - a_j\|^2 + (\alpha - \beta)  \| \sum_{i = 1}^N a_i \|^2  \Big ]
\end{aligned}
\end{equation}
and
\begin{equation}\label{Thm2_Pf_Eq2}
\begin{aligned}
\sum_{i=1}^N \|a_i\|^2 \geq
&\frac{1}{ \alpha N + (N - 2)\beta}  \Big [ \beta \sum_{1 \leq i < j \leq N} \|a_i + a_j \|^2  \\
& + \frac{2\alpha}{N(N - 1)} ( \sum_{1 \leq i < j \leq N} \|a_i - a_j\| )^2 + (\alpha - \beta)  \| \sum_{i = 1}^N a_i \|^2  \Big ].
\end{aligned}
\end{equation}
When $\beta > \alpha > 0$, due to $ \|\sum\limits_{i = 1}^N a_i \|^2 \leq \frac{1}{(N - 1)^2} ( \sum\limits_{1 \leq i < j \leq N} \|a_i + a_j \| )^2$, we obtain
\begin{equation}\label{Thm2_Pf_Eq3}
\begin{aligned}
\sum_{i=1}^N \|a_i\|^2 \geq
&\frac{1}{\alpha N + (N - 2) \beta}  \Big [ \beta \sum_{1 \leq i < j \leq N} \|a_i + a_j\|^2 + \alpha \sum_{1 \leq i < j \leq N} \|a_i - a_j\|^2  \\
& + \frac{ (\alpha - \beta)}{(N - 1)^2} ( \sum_{1 \leq i < j \leq N} \|a_i + a_j\| )^2  \Big ].
\end{aligned}
\end{equation}
Substituting $\|a_i\| = \Delta_{\rho}(A_i)$ and $\|a_i \pm a_j\| = \Delta_{\rho}(A_i \pm A_j)$ into the inequalities (\ref{Thm2_Pf_Eq1})-(\ref{Thm2_Pf_Eq3}), we complete the proof. $\Box$

In Theorem 2 we note that that for given $N$ the larger $\alpha$ and the smaller $\beta$ mean larger $X$ and $Z$ given by (\ref{Thm2_Eq1}) and (\ref{Thm2_Eq3}), respectively. Nevertheless, the larger $\beta$ and the smaller $\alpha$ correspond to larger $Y$ given by (\ref{Thm2_Eq2}).
If one takes $\alpha=\beta$ in Theorem \ref{Thm2}, the lower bound of Theorem \ref{Thm2} is coincident to that of (\ref{Zhang_Eq}). If one respectively takes $\beta<\alpha$ for {\rm X} and $\beta>\alpha$ for {\rm Y}, the lower bound of Theorem \ref{Thm2} is tighter than that of (\ref{Zhang_Eq}). In \cite{li2022metric}, Li $et\ al.$ proved that (\ref{Thm2_Pf_Eq1}) and (\ref{Thm2_Pf_Eq2}) are strictly tighter than those of norm inequalities related to (\ref{Song_Eq}) and (\ref{Zhang_Eq}) for appropriate $\alpha$ and $\beta$.

In particular, when one takes $\alpha = 2$ and $\beta = 1$ for (\ref{Thm2_Eq1}), $\alpha = 1$ and $\beta = 2$ for (\ref{Thm2_Eq2}) and (\ref{Thm2_Eq3}), then $\rm{X}$, $\rm{Y}$ and $\rm{Z}$ respectively reduce to
\begin{equation}
\begin{aligned}
\rm{X} = &\frac{1}{ 3 N - 2 }  \bigg\{ \frac{ 2 }{ N ( N - 1 ) } \Big [ \sum_{ 1 \leq i < j \leq N } \Delta_{\rho} ( A_i + A_j ) \Big ]^2   \\
&   + 2 \sum_{ 1 \leq i < j \leq N } \Delta_{\rho}^2 ( A_i - A_j ) + \Delta_{\rho}^2 ( \sum_{i = 1}^N A_i )  \bigg\},
\end{aligned}
\end{equation}
\begin{equation}
\begin{aligned}
\rm{Y} =& \frac{1}{ 3 N - 4 }  \bigg\{ \frac{ 2 }{ N ( N - 1 ) } \Big [ \sum_{ 1 \leq i < j \leq N } \Delta_{\rho} ( A_i - A_j ) \Big ]^2    \\
&  + 2 \sum_{ 1 \leq i < j \leq N } \Delta_{\rho}^2 ( A_i + A_j ) - \Delta_{\rho}^2 ( \sum_{i = 1}^N A_i )  \bigg\},
\end{aligned}
\end{equation}
\begin{equation}
\begin{aligned}
\rm{Z} = &\frac{1}{ 3 N - 4 }  \bigg\{ 2 \sum_{ 1 \leq i < j \leq N } \Delta_{\rho}^2 ( A_i + A_j ) + \sum_{ 1 \leq i < j \leq N } \Delta_{\rho}^2 ( A_i - A_j )    \\
&  - \frac{ 1 }{ ( N - 1 )^2 } \Big [\sum_{ 1 \leq i < j \leq N } \Delta_{\rho} ( A_i + A_j ) \Big ]^2  \bigg\}.
\end{aligned}
\end{equation}
For convenience, we consider the above special scenario of Theorem \ref{Thm2} in the following concrete examples. We compare the lower bounds $\rm{LB1}$ and $\rm{LB2}$ respectively given in Theorem \ref{Thm1} and \ref{Thm2} with the ones given in (\ref{Song_Eq}) and (\ref{Zhang_Eq}).

%====================Example1====================%
{\emph{Example 1}} Consider the qubit mixed state given by Bloch vector $\vec{r} = (\frac{ \sqrt{3} }{2} \cos\theta, \frac{ \sqrt{3} }{2} \sin\theta, 0)$,
\begin{equation}
\rho = \frac{1}{2}(I_2 + \vec{r} \cdot \vec{\sigma}),
\end{equation}
where the components of the vector $\vec{\sigma}=(\sigma_x,\sigma_y,\sigma_z)$ are the standard Pauli matrices, $I_2$ is the $2 \times 2$ identity matrix. We choose the Pauli matrices $\sigma_x-\sigma_z$, $\sigma_y+\sigma_z$ and $\sigma_z$ as the observables $A_1$, $A_2$ and $A_3$, respectively. Set $\alpha=\frac{1}{2}$ in Theorem \ref{Thm1}, the results are shown in Fig.~\ref{Ex1_Fig}. Obviously, the lower bounds \rm{LB1} and \rm{LB2} are strictly tighter than the bounds of (\ref{Song_Eq}) and (\ref{Zhang_Eq}) in this case.
\begin{figure}[t]
  \centering
  \includegraphics[width=15cm]{wzfEx1_Fig.pdf}
  \caption{Black (dashed) line is the ${\rm sum} = \Delta_{\rho}^2(A_1) + \Delta_{\rho}^2(A_2) + \Delta_{\rho}^2(A_3)$. Red (solid) curve represents $\rm{LB1}$. Green (dashed) curve represents $\rm{LB2}$. Pink (dot-dashed) and blue (dotted) curves represent the right-hand sides (RHS) of (\ref{Song_Eq}) and (\ref{Zhang_Eq}), respectively.}
  \label{Ex1_Fig}
\end{figure}

%====================Example2====================%
{\emph{Example 2} Consider the following class of quantum states given by convex combination of the maximally entangled state and the maximally mixed state,
\begin{equation}\label{isotropic}
\rho_{\theta} = \frac{ 1 - \theta }{ d^2 - 1 } (I_{d^2} - |\Psi^+\ra\la \Psi^+|) + \theta |\Psi^+\ra\la \Psi^+|,
\end{equation}
with $0 \leq \theta \leq 1$ and $|\Psi^+\ra = \frac{1}{ \sqrt{d} } \sum_{i = 1}^d | i i \ra $. Consider the two-qubit case ($d = 2$) and take $\sigma_3 \otimes \sigma_1+\sigma_3 \otimes \sigma_2$, $\sigma_3 \otimes \sigma_2$ and $\sigma_3 \otimes \sigma_3-\sigma_3 \otimes \sigma_2$ as the observables $A_1$, $A_2$ and $A_3$, respectively. Set $\alpha=\frac{1}{2}$ in Theorem \ref{Thm1}, the comparison among our bounds, Song $et\ al.$ and Zhang $et\ al.$'s lower bounds is depicted in Fig.~\ref{Ex2_Fig}.
\begin{figure}[t]
\centering
\includegraphics[width=15cm]{wzfEx2_Fig.pdf}
\caption{Comparison among our bounds, Song $et\ al.$ and Zhang $et\ al.$'s bounds for isotropic state (\ref{isotropic}). Black (dashed) line is the ${\rm sum} = \Delta_{\rho}^2(A_1) + \Delta_{\rho}^2(A_2) + \Delta_{\rho}^2(A_3)$.  Pink (dot-dashed) and blue (dotted) curves represent the right-hand sides (RHS) of (\ref{Song_Eq}) and (\ref{Zhang_Eq}), respectively. Our bounds \rm{LB1} and \rm{LB2} are shown by the red (solid) and green (dashed) curves, which are larger than ones shown by the blue and pink curves.}
\label{Ex2_Fig}
\end{figure}

%====================Example3====================%
{\emph{Example 3} Consider the following pure state of spin-1 system,
\begin{equation}
|\psi\ra = \sin\theta \cos\phi |1\ra + \sin\theta \sin\phi |0\ra + \cos\theta|-1\ra,
\end{equation}
where $\theta \in [0, \pi]$ and $\phi \in [0,2\pi]$. We respectively take $L_x-L_y$, $L_y$ and $L_z+L_y$ as the observables $A_1, A_2$ and $A_3$, where $L_x, L_y$ and $L_z$ are the angular momentum operators ($\hbar = 1$):
\begin{equation}
L_x = \frac{1}{ \sqrt{2} }
\begin{pmatrix}
  0 & 1 & 0 \\
  1 & 0 & 1 \\
  0 & 1 & 0
\end{pmatrix},~~~
L_y = \frac{1}{ \sqrt{2} }
\begin{pmatrix}
  0 & -i & 0 \\
  i & 0 & -i \\
  0 & i & 0
\end{pmatrix},~~~
L_z =
\begin{pmatrix}
  1 & 0 & 0 \\
  0 & 0 & 0 \\
  0 & 0 & -1
\end{pmatrix}.
\end{equation}
Set $\alpha=\frac{1}{2}$ in Theorem \ref{Thm1} and $\phi = \frac{\pi}{2}$. We show in Fig.~\ref{Ex3_Fig} the comparison among our lower bounds of Theorem \ref{Thm1} and Theorem \ref{Thm2}, and those of (\ref{Song_Eq}) and (\ref{Zhang_Eq}). It is easily seen that our bounds are tighter than others in this scenario.
\begin{figure}[t]
\centering
\includegraphics[width=15cm]{wzfEx3_Fig.pdf}
\caption{Black (dashed) line is the ${\rm sum} = \Delta_{\rho}^2(A_1) + \Delta_{\rho}^2(A_2) + \Delta_{\rho}^2(A_3)$. Red (solid) curve represents $\rm{LB1}$ and green (dashed) curve represents $\rm{LB2}$. Pink (dot-dashed) and blue (dotted) curves represent the right-hand sides (RHS) of (\ref{Song_Eq}) and (\ref{Zhang_Eq}), respectively.}
\label{Ex3_Fig}
\end{figure}

\section{Conclusion}\label{Sec3}
We have studied tighter variance-based sum uncertainty relations for $N$ arbitrary observables. By employing the parameterized norm identities and Cauchy-Schwarz inequalities we have derived more general and tighter sum uncertainty relations. Furthermore, we have showed that the bounds of our uncertainty relations are tighter than the existing variance-based uncertainty ones. These results and the simple approaches used in this article may highlight further investigations on related uncertainty relations.

\bigskip
\noindent{\bf Acknowledgments}\, This work is supported by NSFC (Grant Nos. 12075159, 12171044), Beijing Natural Science Foundation (Z190005) and the Academician Innovation Platform of Hainan Province.


\noindent{\bf Data availability}\, Data sharing not applicable to this article as no data sets were generated or analyzed during the current study.

\bibliography{Refs}
\end{document}
