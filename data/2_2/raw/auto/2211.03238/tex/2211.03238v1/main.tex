% This is file JFM2esam.tex
% first release v1.0, 20th October 1996
%       release v1.01, 29th October 1996
%       release v1.1, 25th June 1997
%       release v2.0, 27th July 2004
%       release v3.0, 16th July 2014
%   (based on JFMsampl.tex v1.3 for LaTeX2.09)
% Copyright (C) 1996, 1997, 2014 Cambridge University Press

\documentclass{jfm}
\usepackage{graphicx}
\usepackage{epstopdf, epsfig}

% I added this
\usepackage{xcolor}
\usepackage{amsmath}
\usepackage{adjustbox}
\usepackage{multirow}
\usepackage{tabularx, makecell, booktabs}
\usepackage{comment}

\newtheorem{lemma}{Lemma}
\newtheorem{corollary}{Corollary}

\shorttitle{Wave breaking with multilayer model}
\shortauthor{Jiarong Wu, Stéphane Popinet, and Luc Deike}

\title{Breaking wave field statistics with a multilayer model}

\author{Jiarong Wu\aff{1},
  Stéphane Popinet\aff{2}
 \and Luc Deike\aff{1,3}\corresp{\email{ldeike@princeton.edu}}}

\affiliation{\aff{1}Mechanical and Aerospace Engineering, Princeton University, Princeton, NJ, 08540
\aff{2}Institut Jean Le Rond d’Alembert, CNRS UMR 7190, Sorbonne Université, Paris 75005, France
\aff{3}High Meadows Environmental Institute, Princeton University, Princeton, NJ, 08540}
\begin{document}

\maketitle

\begin{abstract}
    The statistics of breaking wave fields is characterised within a novel multi-layer framework, which generalises the single-layer Saint-Venant system into a multi-layer and non-hydrostatic formulation of the Navier-Stokes equations. We simulate an ensemble of phase-resolved surface wave fields in physical space, where strong non-linearities including wave breaking are modelled, without surface overturning. We extract the kinematics of wave breaking by identifying breaking fronts and their speed, for freely evolving wave fields initialised with typical wind wave spectra. The $\Lambda(c)$ distribution, defined as the length of breaking fronts (per unit area) moving with speed $c$ to $c+dc$ following Phillips 1985, is reported for a broad range of conditions. We recover the $\Lambda(c) \propto c^{-6}$ scaling without any explicit wind forcing for steep enough wave fields. A scaling of $\Lambda(c)$ based solely on the mean square slope and peak wave phase speed is shown to describe the modelled breaking distributions well. The modelled breaking distributions are found to be in good agreement with field measurements and the proposed scaling is consistent with previous empirical formulations. The present work paves the way for simulations of the turbulent upper ocean directly coupled with realistic breaking waves dynamics, including Langmuir turbulence, and other sub-mesoscale processes.
\end{abstract}

%\begin{keywords}
%\end{keywords}

\section{Introduction}
% \textcolor{blue}{Luc: do not start by wave turbulence in the intro. wave turbulence is only a tool/theory to describe the problem. start by framing the physical problem "wave breaking statistics" and its fundamental importance on upper ocean dynamics and air sea interaction. Then introduce Phillips framework as a way to describe this complex multi scale physics. Then describe the scatter in the data and the absence of numerical methods to probe wave breaking statistics and its relationships to wave dynamics and statistics. This is what we provide here, a path to finally study the breaking wave statistics within a coherent numerical framework, with as little hypothesis as possible.}.
% The dynamics and statistics of wave breaking in a particular ocean location depend mainly on the local wave state and not on the wind velocity, while current parameterizations for air–sea interactions in oceanic and atmospheric models remain based exclusively on wind speed. The processes associated with breaking waves at the ocean surface span multiple fields of oceanic and atmospheric sciences, as breaking waves regulate the ocean–atmosphere interaction from local to global scales. Breaking limits the height of ocean waves, transfers momentum from waves to currents, and modulates upper-ocean turbulence.
Wave breaking occurs at the ocean surface at moderate to high wind speed, with significant impacts on the transfer of momentum, energy, and mass between the ocean and the atmosphere \citep{MELVILLE1996,DEIKE2022}. When waves break, the water surface overturns, which generates sea spray and largely enhances the gas exchange. Visually it manifests as white-capping, widely observable at sea above a certain wind speed. Breaking acts as an energy sink for the waves: it limits the wave height by transferring the excessive wave energy into underwater turbulence and currents, therefore influencing the upper-ocean dynamics as well \citep{MCWILLIAMS2016,ROMERO2017}. 

Describing breaking waves analytically and numerically has been challenging due to its nonlinear nature and the fact that the interface becomes multi-valued. Considering a single breaker, scaling analysis have been successfully proposed for energy dissipation, validated by laboratory experiments \citep{DRAZEN2008,PERLIN2013}; and thanks to advances in numerical methods and increasing computational power, high fidelity simulations on single 3D breakers have emerged \citep{WANG2016,DEIKE2016,MOSTERT2022,GAO2021}. 

%DIGIORGIO2022 add this citation at the next round.

%another important aspect of the complexity is that wave breaking happens at many different scales with different strengths in the ocean. To capture this multi-scale feature requires a certain kind of statistical description, much like the statistical description of the wave field itself. On the other hand, wave breaking is localized in space and intermittent in time; it is often hard to determine a local wavelength associated with the breaking wave, all of which render it difficult to incorporate the effect of breaking into the spectral wave models.

\citet{PHILLIPS1985} introduced the $\Lambda(\boldsymbol{c_b})$ distribution to describe the statistics of breaking waves, where $\Lambda(\boldsymbol{c_b})d\boldsymbol{c_b}$ is the expected length per unit sea surface area of breaking fronts propagating with speeds in the range of $(\boldsymbol{c_b}, \boldsymbol{c_b}+d\boldsymbol{c_b})$. The independent variable breaking front propagating speed $\boldsymbol{c_b}$ is chosen in place of wavenumber $\boldsymbol{k}$ because it is a more observable quantity. The link to the wave spectrum is made through the core assumption that $c_{b}$ is proportional to the wave phase speed $c$, which in turn relates to $k$ by the linear dispersion relation $c=\sqrt{gk}$. The omni-directional $\Lambda(c)$ distribution is predicted to have a $c^{-6}$ shape. The moments of the distribution have a physical interpretation, with the second moment related to the whitecap coverage, the third to mass exchange, the fourth to momentum flux and the fifth to energy dissipation by breaking \citep{PHILLIPS1985,KLEISS2010,ROMERO2019,DEIKE2018,DEIKE2022}.

Several observational studies have been conducted, which provide measurements of the $\Lambda(c)$ distribution, and its moments \citep{GEMMRICH2008,KLEISS2010,SUTHERLAND2013,BANNER2014,SCHWENDEMAN2015}, made possible by technical advancement including ship-borne and air-borne visible and infrared imagery. Scaling relations have been proposed to describe the breaking statistics for a wide range of conditions, but are facing the usual challenges in scatter of field data \citep{SUTHERLAND2013,DEIKE2018}, combined with ongoing discussions about the interpretation of \citet{PHILLIPS1985} original framework \citep{BANNER2014}. 

%and the proper use of the independent variable $\boldsymbol{c}$ in post-processing of image data, which can result in a difference in results. 

Beyond the single breaker description, numerical methods have so far been unable to describe the breaking statistics emerging from an ensemble of propagating surface waves. We propose a numerical framework, leveraging a novel multi-layer formulation of the Navier-Stokes equations and its numerical implementation \citep{POPINET2020}, which is able to capture the multi-scale nonlinear wave field, together with the intermittent incidences of breaking. The wave field is initialised using characteristic wind wave spectra based on field observations. We report the kinematics of the breaking statistics, $\Lambda(c)$, and its scaling with the mean-square slope and discuss how to link our results to field measurements. 

%There is no wind forcing but the dissipation of the waves is slow enough so that the waves and breaking statistics are quasi-steady.

\section{Numerical method} \label{sec:multilayer} 
\subsection{The multi-layer framework}
We introduce the modelling framework (sketched in figure \ref{fig:layers_illustration}) proposed by \citet{POPINET2020}, based on a vertically-Lagrangian discretisation of the Navier-Stokes equations. It extends the shallow-water single-layer Saint-Venant model to include multiple layers. 
We solve a weak form of the equation (vertically-integrated conservation laws) in a generalised vertical coordinate.  
Given $N_L$ layers in total, for layer number $l$ the mass and the momentum conservation equations are \citep{POPINET2020}:
\begin{eqnarray}
    \frac{\partial h_l}{\partial t} + \nabla_H \cdot (h\boldsymbol{u})_l &=& 0\label{eqn:numerical1} \\ 
    \frac{\partial (h\boldsymbol{u})_l}{\partial t} + \nabla_H \cdot (h\boldsymbol{u}\boldsymbol{u})_l &=& -g h_l \nabla_H \eta - \nabla_H(hp_{nh})_l + [p_{nh}\nabla_H z ]_l\label{eqn:numerical2} \\ 
    \frac{\partial (hw)_l}{\partial t} + \nabla_H \cdot(hw \boldsymbol{u})_l &=& -[p_{nh}]_l\label{eqn:numerical3} \\ 
    \nabla_H \cdot (h\boldsymbol{u})_l + [w-\boldsymbol{u}\cdot \nabla_H z ]_l &=& 0 \label{eqn:numerical4}
\end{eqnarray}
with $l$ the index of the layer, $h$ its thickness, $\boldsymbol{u}$, $w$ the horizontal and vertical components of the velocity, $p_{nh}$ the non-hydrostatic pressure (divided by the density). The surface elevation $\eta = z_b + \sum_{l=0}^{N_L} h_l$, and the $[\;]_l$ operator denotes the vertical difference, i.e. $[f]_l = f_{l+1/2} - f_{l-1/2}$. 
There are four unknowns $h_l$, $\boldsymbol{u_l}$, $w_l$ and $p_{nhl}$ for each layer. Equation \ref{eqn:numerical1} represents conservation of volume in each layer for layer thicknesses $h_l$ following material surfaces (i.e. the discretization is vertically Lagrangian). Equation \ref{eqn:numerical2} and \ref{eqn:numerical3} are the horizontal and vertical momentum equations. Equation \ref{eqn:numerical4} is the mass conservation equation. The time integration includes an `advection' step and a `remapping' step. In the `advection' step, equation \ref{eqn:numerical1} to \ref{eqn:numerical4} are advanced in time. In the `remapping' step, the layers are remapped, if necessary, onto a prescribed coordinate to prevent any severe distortion of the layer interface.

\begin{comment}
Equation \ref{eqn:numerical2} is the horizontal momentum equation, where the pressure (divided by density) is split into a %\textcolor{blue}{barotropic (point of discussion: we should call it barotropic instead of hydostatic. And why is $h_l$ not included in the gradient operator?)}, 
hydrostatic part $p_h$ and a non-hydrostatic part $p_{nh}$:
\begin{equation}
    p(\boldsymbol{x},z,t)/\rho = p_{h} (\boldsymbol{x},t) + p_{nh}(\boldsymbol{x},z,t)
\end{equation}
where $\boldsymbol{x}=(x,y)$ is the horizontal coordinates. The barotropic part $p_h$ is not a function of $z$ and is given by $p_h = g\eta$
%\textcolor{blue}{(If it is hydrostatic it should be $p_h = g(\eta-z)$?).} 
The last term is the momentum flux due to the non-hydrostatic pressure projected onto the horizontal gradient of the vertical position of each layer. Equation \ref{eqn:numerical3} is the vertical momentum equation. Equation \ref{eqn:numerical4} is the mass conservation equation.
\end{comment}

Note that this set of equations does not make any assumption on the slope of the layers, which explains the $\nabla_H z$ `metric' terms appearing in the horizontal momentum equation (\ref{eqn:numerical2}) and incompressibility condition (\ref{eqn:numerical4}). This is particularly important in the context of steep breaking waves. One can further demonstrate that this set of semi-discrete equations is a consistent discretisation of the incompressible Euler equations with a free-surface and bottom boundary \citep{POPINET2020}.
Note that, in the hydrostatic and small-slope limit, generalised vertical coordinates are widely used in ocean models \citep{GRIFFIES2020}, due to the anisotropic nature of geophysical flows. The choice of the target remapped discretisation is flexible and reflects physical considerations. Here, the remapping step uses a geometric progression of the layer thicknesses which ensures higher vertical resolution of the boundary layer under the free-surface.

%In the `remapping' step, the layers are remapped onto a prescribed coordinate to prevent any severe distortion of the layer interface, which is known to be a problem for Lagrangian numerical schemes. We use a geometric progression as the remapped discretisation, with the top layer being $1/3$ of the uniform layer thickness $(\eta-z_b)/N_L$. 

%We use the quadratic piece-wise remapping of the Piecewise Polynomical Reconstruction (PPR) library by Engwirda and Kelley (2016, https://github.com/dengwirda/PPR).
%The choice of the target remapped discretization is flexible and reflects physical considerations of specific problems.

The numerical schemes (spatial and temporal discretisations, field collocation, grid remapping, etc.) are described in detail in \citet{POPINET2020}, and ensure accurate dispersion relations and momentum conservation. 

\begin{figure}
    \centering
    \includegraphics[width=0.5\linewidth]{figures/illustration.pdf}
    \caption{The layers in the multilayer model, and the fields of each layer. All the fields are functions of horizontal position $\boldsymbol{x}=(x,y)$ and time $t$. Due to the geometric progression choice, there is a fixed depth ratio between two adjacent layers.}
    \label{fig:layers_illustration}
\end{figure}

\subsection{Numerical model for breaking}
%In the shallow water equation, wave breaking is a shock wave like behaviour. During this process the mass and momentum are conserved but the energy is dissipated due to the turbulence in the front region.
% A local Froude number $Fr = U/c$ can be defined, where $c$ is the underlying wave speed and $U$ is the fluid velocity. 
% The rate of energy dissipation across a hydraulic jump is a function of the hydraulic jump inflow Froude number and the height of the jump; for breaking wave, a similar scaling can be found \cite{DRAZEN2008}, and the dissipation (per unit volumn) scales with $(gh)^{1/2}/h$.  The vertical velocity scales with $(gh)^{1/2}$.} \textcolor{red}{Not sure about this part yet.} 
The dissipation due to breaking is modelled with a simple, \emph{ad-hoc} approximation which can be related to the dissipation due to hydraulic jumps in the Saint-Venant system, known to be a surprisingly good first-order model for (shallow-water) breaking waves \citep{BROCCHINI2008}. The horizontal slope for any layer $\partial z/\partial x$ in equation \ref{eqn:numerical2} and \ref{eqn:numerical4} is limited by a maximum value $s_\text{max}$:
\begin{equation}
  \partial z/ \partial x = \left\{
    \begin{array}{ll}
      \partial z/\partial x, & |\partial z/\partial x| \leq s_\text{max}  \\[2pt]
      \text{sign}(\partial z/\partial x)s_\text{max}, & |\partial z/\partial x| > s_\text{max}.
  \end{array} \right.
\end{equation}
The maximum slope $s_{max}$ is set to be 0.577. The same applies to $\partial z/\partial y$. The slope limiter acts to stabilise the solver, and dissipates some amount of energy. %However, there is no modelling for the underwater turbulence, and the exact value of the dissipation has not been extensively studied and compared to experiments. %We know that it is on the same order of dissipation due to turbulence, but smaller. 
% We focus on the kinematic aspect of breaking, and proceed with this ad hoc breaking model. 
We have tested that altering the value of $s_\text{max}$ between 0.4 and 0.6 does not change the numerical results significantly. Note that given enough horizontal resolution and vertical layers, and added viscous diffusion terms, the multilayer model converges to the full Navier-Stokes equations, with underwater turbulence, and the dissipation rate obtained from breaking is close to that obtained with direct numerical simulations. 
% Here we are not resolving the underwater turbulence, and therefore the vertical viscous dissipation does not play any significant role.
% The dissipating mechanism is similar to that of a shock wave (following the Rankine-Hugoniot Relation?).

In the rest of the paper, we analyse the \emph{occurrence} of breaking fronts as geometric features of the surface height $\eta$, and investigate the relation between the wave statistics (wave spectrum) and breaking statistics (distribution of length of breaking crest).

%while keeping in mind that dissipation as a \emph{consequence} of breaking, is not yet accurately represented and will be investigated later. We demonstrate that valuable information about the breaking statistics and its relation to the wave statistics can be obtained with the current breaking model.

\subsection{Numerical simulations of actively breaking wave fields} 
%\subsection{Initialization process}
We initialise the wave field with an azimuth-integrated wavenumber spectrum of the following shape (inspired by field measurements such as \citet{ROMERO2010} and \citet{LENAIN2017}; see the discussion in \citet{DEIKE2022}),
\begin{equation}\label{eqn:spectrum_init}
\phi(k) = Pg^{-1/2}k^{-2.5}\exp[-1.25(k_p/k)^2].
\end{equation}
The value of $P$ controls how energetic the wave field is, and is of dimension of a velocity while $k_p$ is the peak wavenumber of the wave spectrum. Variations of the spectra parameters $k_p$ and $P$ can be summarised into a single non-dimensional global effective slope $k_p H_s$, where $H_s = 4\langle \eta^2 \rangle ^ {1/2}$ is the significant wave height. The global slope $k_p H_s$ is varied from 0.1 to 0.32 (almost no breaking waves to strongly breaking field).
The ratio $k_pL_0$, with $L_0$ the domain size, is kept constant at a sufficient large value ($k_pL_0=10\pi$) to avoid confinement effects, and we have verified that the results are independent from this ratio. The total water depth is chosen to be $2\pi/k_p$ to ensure a deep water condition.
The directional spectrum is $F(k,\theta) = (\phi(k)/k){\text{cos}^{N}(\theta)}/{\int_{-\pi/2}^{\pi/2} \cos^N(\theta) d\theta}$, with $\theta \in [-\pi/2, \pi/2]$.
%$F(k,\theta) = Pg^{-1/2}k^{-3.5}e^{-1.25(k_p/k)^2}{\text{cos}^{N}(\theta)}/{\int \cos^N(\theta) d\theta}$
The directional spreading is controlled by $N$, with $N=5$ for most cases, and we have tested $N=2$ (more spreading) and $N=10$ (less spreading). 

The initial wave field is a superposition of linear waves: $\eta = \sum_{i,j} a_{ij}\text{cos}(\psi_{ij})$, with the amplitude $a_{ij} = [2F(k_{xi},k_{yj})dk_xdk_y]^{1/2}$, and the initial random phase $\psi_{ij} = k_x x + k_y y + \psi_{\text{rand}}$. %The directional spectrum in the Cartesian coordinate $F(k_{xi},k_{yj})$ is interpolated from that in the polar coordinate $F(k,\theta)$. 
The corresponding orbital velocity is initialised similarly according to the linear wave relation. We use a uniformly spaced  initial grid of 32 $\times$ 33 array of ($k_{xi}$,$k_{yj}$). The wavenumbers are truncated, and chosen at discrete values of $k_{x} = ik_p/5$ for $i \in [1,32]$, and $k_{y} = jk_p/5$ for $j \in [-16,16]$, respectively. The horizontal resolution is $N_x = N_y= 1024$, and layer number $N_L = 15$, with a geometric progression common ratio 1.29. We have verified that the results presented here are numerically converged in terms of layer number (by running cases with 30 vertical layers); as well as horizontal resolution (see \S\ref{sec:Lambda_c}). 

%We also verified that the results are independent of the box size, i.e. the ratio of the longest wavelength to the box, $k_p L_0$ does not change the results discussed here. We value the spectrum intensity (controlled by $P$, which leads to a non-dimensional global slope $k_p H_s$ as non-dimensional parameter, where $H_s$ is the significant wave height ($H_s = 4\langle \eta^2 \rangle ^ {1/2} = 4(\int_0^\infty \phi(k) dk)^{1/2}$). The non-dimensional global slope $k_p H_s$ is varied from 0.1 to 0.32.

%The parameter values are summarized in table \ref{tab:parameters}, and the two groups of stable spectra are shown in figure \ref{fig:lambdac_scaled}(a) with orange and green curves. 

\begin{figure}
    \centering
%    \vspace{0.2cm}
    \includegraphics[width=0.75\linewidth]{figures/figure3_new.png}
    \label{fig:spectrum_evo_c} 
    \caption{(a) Snapshots of the wave field development for the case of effective slope $k_pH_s=0.233$. Breaking statistics are collected between $\omega_p t=124$ and $\omega_p t=149$ (indicated by the red box). (b) The wave energy spectrum on the frequency-wavenumber plane. The dotted white line is the linear dispersion relation of surface gravity waves $k=\omega^2/g$. (c) Time evolution of the omni-directional wave spectrum $\phi(k)$, corresponding to the snapshots in (a). (d) Energy evolution of the wave field. Purple: potential energy $E_p$; blue: kinetic energy $E_k$; pink: total energy. (e) 3D rendering of the breaking wave field with the colour indicating the surface layer flow velocity. Inset shows the curvature of the breaking fronts as the detection criterion. (f) A more focused view taken from the dotted white square in (a); The arrows are showing the velocity magnitude and direction of each length element of the breaking fronts.}
    \label{fig:spectrum_evo}
\end{figure}

%\subsection{The quasi-steady wave spectrum}
Figure \ref{fig:spectrum_evo}(a) shows the time evolution of the wave field and the corresponding wave spectrum. The wave field is visually realistic. The space-time wave elevation spectrum is shown in figure \ref{fig:spectrum_evo}(b) and the energy is localised around the curve given by the gravity wave linear dispersion relation $\omega=\sqrt{gk}$, together with an extra branch corresponding to bound waves.
% (\textcolor{blue}{Compare spectrum statistics to \cite{ROMERO2010}, like skewness and kurtosis?}), 
%and the solver captures the linear dispersion relation well, as shown by figure \ref{fig:spectrum_evo}(b). 
Since we start with a truncated spectrum, the initial wave field is smooth while the small scale features develop over time. There is an energy transfer into the higher wavenumbers which eventually leads to a stable spectrum shape. It is not a Kolmogorov--Zakharov type wave energy cascade as described by wave turbulence theory \citep{ZAKHAROV1992}, since the weak non-linearity assumption does not hold and  the small features are mostly generated by the breaking events.
% which is tested by selecting wave spectra that are nonlinear enough but not yet breaking, and we observe a much slower and insufficient energy transfer into smaller scales. 
A quasi-steady spectrum is obtained typically for $\omega_p t>100$ as seen on figure  \ref{fig:spectrum_evo}(c) with the wave statistics independent from the initial conditions, and we measure the breaking statistics between $\omega_p t=124$ and $\omega_p t=149$. Since there is no forcing mechanism, the total wave energy is slowly decaying (as shown in figure \ref{fig:spectrum_evo}(d)). The dissipation is primarily due to the slope-limiter, and is of the order of magnitude of known dissipation due to breaking \citep{DRAZEN2008}. We have also verified that the spatially and temporally-averaged statistics is a good representation of the ensemble average.

% However, the spectra are near steady state during the time window when breaking statistics are collected.}
% $\omega_p t=78$ and $\omega_p t=95$ for the other kp

%However, since the decaying mechanism due to the numerical slope limiter is finite but small, the spectrum is relatively stable within the time window when we assess the breaking statistics.

\subsection{Procedure of breaking front detection and velocity measurement}

%Here we outline the procedures to detect the breaking fronts and find their velocity from the simulation output. We briefly discuss the various methods used in field observations, and how they (include ours) follow or deviate from the original idea of Phillips.
%\subsubsection{Breaking wave fronts detection and velocity mapping}
The wave field evolves and breaking occurs intermittently in space and time. We detect the breaking fronts and their velocity, and construct the length of breaking crest distribution. The breaking fronts are defined geometrically as sharp enough ridges of the surface, as illustrated in figure \ref{fig:spectrum_evo}(e).  Given a surface elevation $\eta(x,y)$ at one time instance, we find its Gaussian curvature $\kappa_1$ and $\kappa_2$, and determine the location of the breaking fronts by the threshold $\kappa_2 < -3k_p$ (`ridges' of the $\eta$ surface), which works well across the different scales.
After the breaking regions (areas) are detected, we extract the breaking fronts (lines), shown in figure \ref{fig:spectrum_evo}(f). Then we use the surface layer Eulerian velocity ($\boldsymbol{u_{l-1}}$ in figure \ref{fig:layers_illustration}) as an estimate of the Lagrangian velocity of the breaking fronts $\boldsymbol{c_b}$. The velocity is mapped on each discretised cell on the lines, which represents an element of length $L_0/N_x$. Figure \ref{fig:spectrum_evo}(f) shows the mapped velocity magnitude and direction with arrows. The directionality of $\boldsymbol{c_b}$ is not discussed in this work, i.e. we only consider the magnitude $c_b=|\boldsymbol{c_b}|$. We have tested an alternative velocity mapping method by computing the correlation function between two consecutive images of the detected crests (similar to particle tracking velocimetry), and found no significant difference in the velocity magnitude detected or the resulting $\Lambda(c_b)$ distribution. %Therefore, the current more straightforward method is preferred.

% we compute its Hessian matrix $ H(x,y) = \begin{bmatrix}
% \eta_{xx} & \eta_{yx} \\
% \eta_{xy} & \eta_{yy} 
% \end{bmatrix}$, where $\eta_{xx}$, $\eta_{xy}$, $\eta_{yy}$ are the spatial second order derivatives of $\eta$. The principal curvatures $\kappa_1$, $\kappa_2$ ($|\kappa_1|<|\kappa_2|$) are defined as the eigenvalues of this $2 \times 2$ matrix %(since the matrix is symmetric, the eigenvalues are always real). 

% \textcolor{blue}{Does the high curvature area overlap with the slope-limiter effective area?}
% \textcolor{red}{Mark McAllister's work on how directionality greatly effects the breaking criteria.}

We follow \citet{PHILLIPS1985,KLEISS2010,SUTHERLAND2013} and assume that $c=c_b$; and we use a correspondence between the breaking front velocity and the underlying wavenumber through the dispersion relation $c=\sqrt{g/k}$. %The resulting breaking statistics $\Lambda(c)$ is plotted in figure \ref{fig:spectra+lambdac}. 
We note that observations have shown that $c_{b} = \alpha c$ where $\alpha$ is between 0.7 to 0.95 \citep{RAPP1990,BANNER2014,ROMERO2019}, at least for large breakers. In the processing we filter out the smaller scale breakers by imposing a filter $\eta(x,y) > 2.5 \langle \eta \rangle^{1/2}$. It means that only the large breakers with surface elevation above 2.5 rms value are included. As a result, no further corrections for the underlying long wave orbital velocity is needed.

%\textcolor{blue}{Phillips himself cautions that it should apply to $c>s c_p$, where $s=k_p\langle \eta^2 \rangle ^{1/2}$ is the significant slope (so this is a quantity already like $k_pH_s$).}

% The values are too small anyway, almost corresponding to the unresolved part of the spectrum, and they don't contribute that much to the total energy dissipation.

\section{Statistics of wave breaking} \label{sec:Lambda_c} 
\subsection{Wave statistics}

We study the relation of the breaking statistics with the wave spectrum. Figure \ref{fig:spectra+lambdac}(a) shows the non-dimensional wave spectra for the various conditions, with variations in spectrum maxima larger than one order of magnitude, and described by power laws ranging from $\phi(k)\propto k^{-2.5}$ to $\phi(k)\propto k^{-3}$. Although the energy close to the peak frequency varies, a fixed level of saturation seems reached for the steeper cases with overlapping spectra in the $k^{-3}$ range.

%\textcolor{blue}{To show the relevance of the parameters we also plotted some wind wave spectra from observations \citep{ROMERO2010,LENAIN2017}, and they are roughly in the same $k$ and $F(k)$ range with the spectra we use.}

Together with the global slope $k_pH_s$, wave statistics can be characterised by the root mean square slope $\sigma$ \citep{MUNK2009}, which is more sensitive to high frequencies. The low-pass filtered steepness parameter $\mu(k)$ is defined as the cumulative root mean square slope: $\mu^2(k) = \int_0^k k'^2\phi(k') dk'$, and $\sigma$ is the asymptotic value of $\mu$ with a cutoff at the highest wavenumber we can numerically resolve $k_{max}$: $\sigma^2 = \mu^2(k\to k_{max})$. As we see in figure \ref{fig:spectra+lambdac}(a), the value of $\mu(k)$ plateaus due to the drop-off of the spectrum. In weak nonlinear theories (such as wave-turbulence theory), $\mu < 0.1$ is used to justify the asymptotic expansions, at least for the range of $k$ considered \citep{ZAKHAROV1992}. All the breaking cases in our simulation have $\mu$ closer or higher than 0.1 underlying the strong non-linearities of the breaking wave field. The correlation between the two global slope parameters $k_pH_s$ (zeroth moment of the spectrum) and $\sigma$ (second moment of the spectrum) is shown by figure \ref{fig:spectra+lambdac}(b), which we caution is specific to the spectrum shape.

% Both slopes $s_1 = \sigma$ and $s_2 = k_pH_s$ are global quantities of the spectrum and non-dimensional, and they correspond to the second and the zeroth moment of $\phi(k)$ respectively. Discuss that it might be different with swell and other shape of spectrum.

% and we see that it goes above 0.1 around $k\approx 5k_p$ for the steep cases. 
% Most of the cases we studied are steep enough to be considered strongly nonlinear, except for two, Case1\_1 and Case1\_2 in table \ref{tab:parameters}. The former has a cumulative $\mu$ significant smaller than 0.1, and there is no breaking; the latter has $\mu$ very close to 0.1, there is some extent of breaking but the statistics are different from the steeper cases (further discussed in \S\ref{sec:Lambda_c}).
%\textcolor{blue}{The integral might diverge. If $F(k)\propto k^{-3}$ it increases logarithmically with $k$, but in reality there is probably an upper bound from capillarity.} 
%The values of $s_1$ and $s_2$ are summarized in table \ref{tab:parameters}. 
%In general, $s_1$ increases slower than $s_2$ with increasing initial $P$, since the large wave number part of the spectrum actually are closer despite different $P$ (see figure \ref{fig:lambdac_scaled}(a)).

\subsection{$\Lambda(c)$ distribution}
% Inset: the wave energy specta in dimensional form. Gray lines are a selection of wind wave spectra from observations: solid lines are fetch-limited measurements of the developing the low-pass filtered steepness parameter $\mu(k)$. dashed line is the high resolution measurements of \textcolor{blue}{fully developed?} winwave field from the GOTEX campaign \citep{ROMERO2010}. The dashed lines indicate d sea from the SoCal2013 campaign \citep{LENAIN2017}.

Figure \ref{fig:spectra+lambdac}(c) shows the breaking distribution $\Lambda(c)$ for increasing $k_p H_s$ (and $\sigma$) values and the various directionalities. 
% For the two $k_p$ groups, there are two clusters of $\Lambda(c)$ curves: the scale of the breaking front velocity is set primarily by the peak phase speed $c_p$. 
There is a clear peak indicating the most probable smaller breakers, increasing from $c= 0.2c_p$ to $0.3c_p$ when the slope increases. There is no breaking for the smallest $\sigma=0.065$ case ($k_p H_s= 0.117$) (not shown in figure \ref{fig:spectra+lambdac}(c)) an increase in slope to $\sigma=0.085$ ($k_p H_s=0.150$) and $\sigma=0.101$ ($k_p H_s= 0.169$) starts to generate breakers. The extent of breaking speeds is further increased for the steeper cases with $\sigma>0.101$, with a clear $\Lambda(c)\propto c^{-6}$ scaling up to around $0.9c_p$. It indicates that there exists a critical value of $\sigma$, below which the breaking wave field is not saturated, with the threshold expected to depend on the spectrum shape.

The shaded area in figure \ref{fig:spectra+lambdac}(c) spans the range of the breaker velocity between the peak $\Lambda_{max}$ and  $\Lambda(c)=0.01\Lambda_{max}$ (for the case of $\sigma=0.153$), where a $\Lambda(c)\propto c^{-6}$ scaling can be clearly observed. The same range in the $k$-space is shaded in figure \ref{fig:spectra+lambdac}(a) as well. The upper limit of $\Lambda(c)=0.01\Lambda_{max}$ corresponds to a lower limit of $k\approx4k_p$. Above that velocity, breakers near $k_p$ are very rare. We note that removing the filter of $\eta(x,y) > 2.5 \langle \eta \rangle^{1/2}$ only changes the part of the $\Lambda(c)$ distribution left of the peak, but does not affect the part with $c$ larger than the peak. Similarly, further increasing the horizontal resolution would extend the move up and toward even smaller $c$, but the presented $\Lambda(c)\propto c^{-6}$ is unchanged.

% The inset of figure \ref{fig:spectra+lambdac}(c) shows the same lines but in log-log scale, which exhibits a $\Lambda(c) \propto c^{-6}$ scaling.
$\Lambda(c)$ distributions from spectra of different directionality $N$ are also shown with different lines (dashed lines indicate more spreading ($N=2$) and dotted lines less spreading ($N=10$)). For steep enough cases ($\sigma > 0.101 $), there is little difference in the $\Lambda(c)$ distribution between cases with different $N$, while for intermediate steepness ($\sigma = 0.85$ and $0.101$), there is a notable sensitivity to $N$. For the $N=10$ cases with more concentrated wave energy, there is overall more breaking events.

\begin{figure} 
\centering
\includegraphics[width=0.75\linewidth]{figures/figure4_new-eps-converted-to.pdf}
\caption{\label{fig:spectra+lambdac} (a) The wave energy spectra in non-dimensional form; the vertical gray line is  $k_pL_0 = 10\pi$. Darker colour indicates larger global slope $k_pH_s$ (see (b) for the values). (b) The correlation of root mean square slope $\sigma$ and the global slope $k_pH_s$ in the simulated cases. (c) The non-dimensional breaking distribution $\Lambda(c)$ normalised by $c_p$ and $g$. Solid lines: directional spreading parameter $N=5$; dashed lines: $N=2$; dotted lines: $N=10$. (d) Proposed scaling for the $\Lambda(c)$ distribution using $\sigma$ and $c_p$.}
% The gray lines show the range of the fitted scaling prefactor. Notice that the value of the prefactor is close to that found in \cite{DEIKE2018}, although the scaling is slightly different. 
\end{figure}

%In a word, we focus on the statistics of the large scale breakers, whose breaking fronts speed reflects the underlying wave phase speed. In addition, we plot the $\Lambda(c)$ distribution without the additional surface height filtering, which would include more smaller and slower breaking fronts. However, the distribution to the right of the peak is not affected, and we focus the discussion on these large scale breakers. 

Non-dimensional scalings of the $\Lambda(c)$ distribution have been proposed, with \citet{PHILLIPS1985} using only the wind speed, while papers based on field data used a combination of wind speed, wave spectrum peak speed and significant wave height \citep{SUTHERLAND2013,DEIKE2018}. Since we have no wind forcing in the simulations, the breaking statistics is expected to scale only with the non-dimensional slope $\sigma$ and spectrum peak speed $c_p$. By rescaling $c$ using $\hat{c} = c/(\sigma c_p)$ and $\Lambda(c)$ using $\hat{\Lambda}(c) = \Lambda(c)c_p^3g^{-1}\sigma^{-2}$, we obtain a normalised distribution:
\begin{equation}
    \Lambda(c)c_p^3g^{-1}\sigma^{-2} \propto \left(c/(\sigma c_p)\right)^{-6} \label{eqn:lambdac_scaling}
\end{equation}
shown in figure \ref{fig:spectra+lambdac}(d), which collapses not only the $\hat{c}^{-6}$ power law region but also the peak location well (for the steep enough cases). 
%With the definition of the rms slope $\sigma$, we rescale $c$ using $\hat{c} = c/(\sigma c_p)$, and $\Lambda(c)$ using $\hat{\Lambda}(c) = \Lambda(c)c_p^3g^{-1}\sigma^{-2}$. Figure \ref{fig:spectra+lambdac}(d) shows that the scaling collapses the distribution curves well (for the steep enough cases). 
%We note that for the y axis, the scaling is mainly set by $c_p^3$, with a weak dependence on the $\sigma^{-2}$ part. 
Alternatively, since the effective slope $k_pH_s$ and $\sigma$ are correlated, a scaling using $(k_pH_s)^{1/2}$ could be proposed. However, we have found that $\sigma$ works better than $k_pH_s$ as a scaling parameter in this case.

% and therefore there is some freedom/uncertainty in choosing the power of $\sigma$ \textcolor{red}{between X and Y}. 
\subsection{Comparison to the \citet{PHILLIPS1985} theory and observations}
% \textcolor{blue}{We have shown that our proposed wave based scaling and the previous wind speed based scaling are consistent, because the underlying fetch-limited relation for wind wave development. A natural question to ask is then which scaling is more general and makes more physical sense.}
% We can see that the wind based scaling
% \begin{equation}\label{eqn:Lambda(c)_norm_wind}
%     \hat{\Lambda}(c) = \Lambda(c)c_p^3g^{-1}(c_p/u_*)^{1/2} \text{, and } \hat{c} = c(gH_s)^{-1/2}
% \end{equation}
% is similar to the one by (\ref{eqn:Lambda(c)_norm}) if we choose $s=k_pH_s$.

\citet{PHILLIPS1985} predicted a purely wind-based scaling $\Lambda(c)\propto u_*^3 g c^{-6}$ through an energy balance argument. The wave action balance equation $d[g\phi(k)/\omega]/dt = S_{nl}(k)+S_{in}(k)+S_{diss}(k)$ involves the following source terms: divergence of the nonlinear energy flux $S_{nl}$, wind input $S_{in}$, and dissipation due to breaking $S_{diss}$, written as \citep{PHILLIPS1985}
\begin{equation}
    S_{nl} \propto gk^{-3}B^3(k), \; S_{in} \propto gk^{-3}(\frac{u_*}{c})^2 B(k), \; \text{and} \; S_{diss} \propto gk^{-3} f(B(k))
\end{equation}
% \begin{equation}
%     S_{nl} \propto gk^{-4}B^3(\boldsymbol{k}), \; S_{in} \propto gk^{-4}(\frac{u_*}{c})^2 B(\boldsymbol{k}), \; \text{and} \; S_{diss} \propto gk^{-4} f(B(\boldsymbol{k}))
% \end{equation}
% $B(\boldsymbol{k})=k^4\phi(\boldsymbol{k})$
with the saturation $B(k)=k^3\phi(k)$, and $f(B(k))$ a functional dependence solely on $B(k)$ (assuming that breaking and consequent dissipation `\textit{are the result of local excesses, however these excesses are produced}'). The balance between $S_{nl}$ and $S_{diss}$ leads to $f(B)\propto B^3$, and therefore $S_{diss} \propto gk^{-3}B^{-3}$.
The breaking front distribution $\Lambda(c)$ is then obtained by writing the equality between dissipation in the $k$-space and the $c$-space: $\epsilon(k)dk = \epsilon(c)dc$. The LHS is $\epsilon(k)dk = (S_{diss}\omega) dk$; the RHS can be related to the fifth moment of $\Lambda(c)$ through a scaling argument $\epsilon(c)dc = bg^{-1}c^5\Lambda(c)dc$, where $b$ is a non-dimensional breaking parameter \citep{DUNCAN1981,PHILLIPS1985}. Substituting a spectral shape of $\phi(k)\propto k^{-5/2}$ into the $S_{diss}$ would then lead to $\Lambda(c)\propto c^{-6}$. Considering the equilibrium range, $S_{nl} \propto S_{in}$ \citep{PHILLIPS1985}, gives $\phi(k) \propto u_*g^{-1/2}k^{-2.5}$, which leads $\Lambda(c) \propto u_*^3gc^{-6}$.

Several field campaigns have observed the $c^{-6}$ power-law, despite also finding that the purely wind-based prefactor $u_*^3$ does not describe the data well. Empirical modifications have been proposed \citep{SUTHERLAND2013,DEIKE2018} using $\sqrt{gH_s}$ and $c_p$ in addition to $u_*$, in the form of $\Lambda(c)c_p^3 g^{-1} (c_p/u_*)^{1/2}\propto (c/\sqrt{g H_s})^{-6}$, which significantly improved the collapse between data sets. 

%shape of $\Lambda(c) \propto u_*^3gc^{-6}$, a few additional arguments were made. First one recognizes that the dissipation on the wavenumber plane and the phase speed plane are equal: $\epsilon(k)dk = \epsilon(c)dc$. The LHS is $\epsilon(k)dk = S_{diss}\omega dk$, while the RHS can be related to the fifth moment of $\Lambda(c)$ through a scaling argument $\epsilon(c)dc = bg^{-1}c^5\Lambda(c)dc$, where $b$ is a non-dimensional breaking parameter. 

% To summarize, for all the wave fields that are steep enough we find that the following normalization of the $\Lambda(c)$ distribution where
% % \begin{equation}\label{eqn:Lambda(c)_norm}
% %     \hat{\Lambda}(c) = \Lambda(c)P^{-1}(\sqrt{gH_s})^4g^{-3/2}, \;\; \hat{c} = c(\sqrt{gH_s})^{-1}
% % \end{equation}
% \begin{equation}\label{eqn:Lambda(c)_norm}
%     \hat{\Lambda}(c) = \Lambda(c)c_p^3g^{-1}s^m \text{, and } \hat{c} = c(s^nc_p)^{-1}
% \end{equation}
% collapses the data very well. The slope parameter can be either the root mean square slope $\sigma$ or the significant wave height defined slope $k_pH_s$, at least for a typical wind-wave-like spectrum. The range of $c$ where the breakers are relatively well defined (to the left of the peak) follows the scaling $\hat{\Lambda}(c) = \hat{K} \hat{c}^{-6}$. The value of the constant $\hat{K}$ is around 800 if we pick $s=s_1$, $n=1$ and $m=2$.

%In terms of scaling for $\Lambda(c)$, previous studies often invoke another variable, the wind velocity, whereas we have discussed the scaling of $\Lambda(c)$ with peak wave phase speed $c_p$ and a global wave slope parameter $s$, both wave field properties. 

To better interpret the empirical scaling found in field observations, we perform re-analysis of the numerical data together with the field data. 
% First, we compare the multi-layer breaking distribution and those reported in the field measurements without any scaling. Figure \ref{fig:data}(a) shows the dimensional $\Lambda(c)$, demonstrating that the modelled distribution fall within the scatter of the field measurements. 
We examine the slope-based scaling in the present work and the mixed-wind-slope-based scaling found in field data. Since there is no explicit wind forcing in our simulations, the information of wind speed and fetch/duration are encoded in the spectrum. We use the empirical (but very robust) fetch-limited relationships \citep{TOBA1972}, that link the non-dimensional wave energy $gH_s u_*^{-2}$ and the non-dimensional frequency $\omega_p g^{-1}u_*$ (wave age $u_*/c_p$) by
% Here, we first make an heuristic argument on how the two scalings are related implicitly through fetch-limited relations; then we discuss if a wave based scaling or a wind based scaling should be preferred.
% This has to do with the empirical relation between the non-dimensional wave frequency $\omega_p$ and the global wave energy $\langle{\eta}\rangle$ (or equivalently $H_s$). We argue that since there is still a great deal of uncertainties in these empirically determined power laws, introducing the wind speed parameter (which is simply not present in our numerical framework), is not beneficial. However, for the purpose of comparison with previous studies, we discuss below on how we can relate the two.
\begin{equation}
gH_s/u_*^2 = C (u_*/c_p)^{-3/2} \label{eqn:fetch_limited}
\end{equation}
where $C$ is an order 1 constant. Using (\ref{eqn:fetch_limited}) it is straightforward to show that the scaling (\ref{eqn:lambdac_scaling}) with $\sigma \propto \sqrt{k_pH_s}$ is equivalent to the scaling from \citet{SUTHERLAND2013}, since $k_pH_s \propto (c_p/u_*)^{1/2}$. Figure \ref{fig:data}(a) shows the breaking distributions scaled following \citet{SUTHERLAND2013} (the wind speed $u_*$ is inferred using (\ref{eqn:fetch_limited}) in our modelled cases), and good agreement is observed, indicating that the slope-based scaling is fully compatible with the mixed-wind-slope-based scalings in the field. Note that the observational data are obtained from complex sea states, and do not necessarily have the same spectrum shape as the current numerical data. 

%Using these infered wind speed data with the observation using the same scaling proposed by \citet{SUTHERLAND2013}, there is a good agreement. \textcolor{blue}{In fact, it is straightforward to show that the scaling based on $s_2=k_pH_s$ and the one \citet{SUTHERLAND2013} are the same since $k_pH_s \propto (c_p/u_*)^{1/2}$.} 

%\textcolor{red}{The relation of $\sigma$ to wind speed is less certain, especially with the existence of capillary waves in reality. However, the low-passed filtered part is strongly correlated with $k_pH_s$, and therefore all the above mentioned scalings are consistent.}

%\textcolor{blue}{This is then similar to the scaling proposed in \citet{SUTHERLAND201,DEIKE2018}.
%With our data set, we find that the scaling with rms slope $\sigma$ works slightly better than the effective slope $k_pH_s$, which result in $\Lambda(c) = A c_p^3 g \sigma ^4c^{-6}$, where $A\approx800$ is a constant.}

\begin{figure}
\centering
\includegraphics[height=0.27\linewidth]{figures/figure4a_level10.pdf}
\includegraphics[height=0.27\linewidth]{figures/figure4b_level10.pdf}
% \includegraphics[width=0.32\linewidth]{figures/figure4c.pdf}
\caption{\label{fig:data} Comparison with observational data. (a) Rescaled $\Lambda(c)$ distribution following \cite{SUTHERLAND2013} with simplifications \citep{DEIKE2022}. (b) Whitecap coverage $W$ as a function of 10-meter wind speed $U_{10}$. 
}
\end{figure}

%The reason that the purely wind-based scaling $\Lambda(c) \propto u_*^3gc^6$ fails is likely because the local balance between $S_{diss}$ and $S_{in}$ is too strict. 

The wave field at a certain time and space is the result of the wind forcing history, and breaking is caused by excessive energy in the wave spectrum. For a mature and steep enough wave field, breaking (particularly those at large scale) is primarily dictated by the wave spectrum itself, while for younger and less steep wave fields, breaking can be more closely coupled to wind forcing. It explains why the slope-based scaling or a mixed-wind-slope-based scaling can better fit data from various sea states. 

%An additional note is that the original derivation of the $\Lambda(c) \propto c^{-6}$ scaling is based on an energy dissipation argument, whereas in the numerical simulation, breaking is discussed in a purely kinematic sense. 
%It remains to be investigated whether there is a simpler kinematic argument that would give rise to the same $\Lambda(c)$ distribution.

% In another word, the spectrum shape of $\phi(k) = P g^{-1/2}k^{-7/2}$ or the fetch-limited relation (equation \ref{eqn:fetch_limited}) is more robust then the local balance between the wind input and the dissipation term (local in both the time-space and spectral sense).
% There are still some uncertainties in this empirical power laws in terms of the constant $C$ and the exact exponent. but they are also surprisingly robust.  
% Essentially, we have two degrees of freedom in the spectrum: $H_s$, $c_p$ or $H_s$, $u_*$. The former combination is purely wave based, while the latter involves wind information.
% \textcolor{red}{We argue that since there is still a great deal of uncertainties in these empirically determined power laws, introducing the wind speed parameter (which is simply not present in our numerical framework), is not beneficial.} 
% Especially for a more mature wind sea, its relation to the local wind speed may not be as crucial. 

% \textcolor{blue}{Both $k_pH_s$ and $MSS$ are globally defined slopes (in contrast to local saturation $B$). They only work as overall scaling parameters for the $\Lambda(c)$ distribution because the spectrum has already reached some kind of equilibrium through breaking. If for any reason there is a deviation from the equilibrium shape around some wavenumber, we shall see more breakers around that number too.}

Finally, we can infer classic breaking metrics such as the whitecap coverage from our simulations and compare with more field data sets. The whitecap coverage $W$ quantifies the fraction of the wave surface covered by white foam, and can be estimated through the second moment of $\Lambda(c)$ as $W=2\pi\gamma g^{-1}\int c^2\Lambda(c) \:d{c}$, where $\gamma$ is a dimensionless constant representing the ratio of breaking time to wave period (here $\gamma = 0.56$ following \citet{ROMERO2019}). Figure \ref{fig:data}(b) shows $W$ as a function of the 10-meter wind speed $U_{10}$ (estimated from $u_*$ for our data using the COARE parameterisation \citep{EDSON2013}). The modelled whitecap coverage falls within the scatter of recent data sets \citep{CALLAGHAN2008,KLEISS2010,SCHWENDEMAN2015,BRUMER2017}.

% If we relate the spectrum shape in equation \ref{eqn:spectrum_init} to the Toba's spectrum for the equilibrium range (\textcolor{blue}{is it $1/2\beta$}?)
% \begin{equation}\label{eqn:spectrum_toba}
% \phi_{T}(k) = \beta u_* g^{-1/2} k^{-2.5} 
% \end{equation}
% where $\beta = 0.016(c_p/u_*)^{0.53}$, we can see that $P=\beta u_* g^{-1/2}$ (\textcolor{red}{or $P\exp(-1.25) = \beta u_* g^{-1/2} $}), and the scaling of equation \label{eqn:Lambda(c)_norm} is equivalent to 
% \begin{equation}
%     \hat{\Lambda}(c) = \Lambda(c)\beta (\sqrt{gH_s})^3 g^{-1} (\sqrt{gH_s}/u_*)
% \end{equation}
% which is very close to what was proposed in \cite{DEIKE2018}, except for the $\beta$ prefactor and the power of $\sqrt{gH_s}/u_*$ (1 instead 5/3).

%Different definitions of $c$, $\Lambda(c)$, and processing methods exist in breaking statistics literature: some works define one incipient breaking speed for one breaking front \citep{GEMMRICH2008} throughout the whole breaking event, while others define a breaking speed for each small length element on a breaking front for each time snapshot of the breaking event \citep{KLEISS2010,SUTHERLAND2013}. Our method, as described above, is closer in principle to the latter.

%Previous studies \citep{KLEISS2010, BANNER2014} have mentioned that the $\Lambda(c)$ distribution is sensitive to the processing method and the definitions of breaking and breaking speed employed, especially on the small $c$ side. There are a few causes: 
%the white patch of foams used to detect the breaking fronts can persist even after the breaking; the small breakers are often results of short waves riding on long waves, and therefore corrections are needed to remove the long wave orbital velocity; the breakers tend to slow down during the whole breaking events if multiple snapshots are taken. \textcolor{blue}{A few of these factors are not a problem in our numerical experiment, but there are still cautions to be taken with the effect of longer waves on shorter wave breakers.}


\section{Conclusion}

We demonstrate that a novel multilayer model \citep{POPINET2020} can be used to study the breaking statistics associated with an ensemble of phase-resolved surface waves simulated in the physical space. We analyse the breaking front distribution introduced by \cite{PHILLIPS1985}, and find good agreement with field observations. The breaking distribution follows $\Lambda(c) \propto c^{-6}$ even in the absence of wind input, and can be scaled by the mean square slope, indicating that the universal breaking kinematics is primarily governed by the wave field itself, while the wind controls the development of the wave spectrum. The proposed scaling in terms of the mean square slope is fully compatible with empirical relationships used to describe field data. 

%With wind wave spectra that have underlying self-similarities across wavenumber $k$, we find that a global slope parameter ($\sigma$ or $k_pH_s$) can be used to rescale the $\Lambda(c)$ distribution. 

%For more complex wave spectra (e.g. bi-modal with a mix of swells and locally generated wind waves), it remains to be studied which slope parameter is preferred, and whether a parameterization based on the local saturation $B(k)$ instead of global quantities is the only viable way (see e.g. the empirical function proposed in \citet{ROMERO2019}). 
% Identifying the scale range of breaking also has an implication for wave turbulence theories, where breaking is viewed as a constant energy sink only at the upperlimit end of the equilibrium range. By identifying the wavenumber range associated with the breakers, we re-emphasize the scope where wave turbulence theories could potentially apply. For energetic wave field, the large scale breakers are very close to the wave peak, although small scale more frequent breakers are more set by the waves.

% The practical advantage is that global quantities are often easier to measure than the wave spectrum. In term of choosing the steepness parameter, the second moment based slope $\sigma$ should be preferred over the zeroth moment based slope $k_pH_s$, especially in the scenario of a complicated bi-modal wave spectrum (e.g. a mix of swell and locally generated wind waves), since it puts less weight to the small wave number part of the spectrum.

Our approach provides an unprecedented numerical framework to study breaking statistics for complex wave spectra, which could help to understand the breaking distribution in complex seas (in the presence of swell or currents) and complement existing modelling approaches such as \citet{ROMERO2019}. In addition to the physical discussion of breaking statistics, we demonstrate the capability of the multi-layer approach to solve highly nonlinear geophysical flows with strong vertical-horizontal anisotropy.

\bibliographystyle{jfm}
% Note the spaces between the initials
\bibliography{ref}

\end{document}

\begin{table}
\centering
\begin{adjustbox}{width=1\textwidth}
\begin{tabular}{lllllllll} 
\toprule
$k_pL_0$                  & $L_0$(m)             & $k_p$(m$^{-1}$)               & P     & N      & $s_1=\sigma$    & $s_2=k_pH_s$    & $N_x(N_y)$             & $N_L$                \\ 
\hline
\multirow{10}{*}{10$\pi$} & \multirow{7}{*}{200} & \multirow{7}{*}{$2\pi/40$}  & 0.016 & 5      & 0.065           & 0.117           & \multirow{5}{*}{$2^9$} & \multirow{5}{*}{15}  \\
                          &                      &                             & 0.025 & 2,5,10 & 0.085$\pm$0.003 & 0.150$\pm$0.001 &                        &                      \\
                          &                      &                             & 0.031 & 2,5,10 & 0.101$\pm$0.003 & 0.169$\pm$0.001 &                        &                      \\
                          &                      &                             & 0.050 & 2,5,10 & 0.126$\pm$0.001 & 0.211$\pm$0.001 &                        &                      \\
                          &                      &                             & 0.063 & 2,5,10 & 0.141$\pm$0.008 & 0.233$\pm$0.002 &                        &                      \\ 
\cmidrule(){8-9}
                          &                      &                             & 0.094 & 2,5,10 & 0.149$\pm$0.006 & 0.274$\pm$0.003 & $2^9$,$2^{10}$         & 15,30                \\ 
\cmidrule(){8-9}
                          &                      &                             & 0.125 & 5      & 0.157           & 0.303           & $2^9$                  & 15                   \\ 
\cmidrule(){2-9}
                          & \multirow{3}{*}{500} & \multirow{3}{*}{$2\pi/100$} & 0.094 & 5      & 0.140           & 0.232           & \multirow{3}{*}{$2^9$} & \multirow{3}{*}{15}  \\
                          &                      &                             & 0.157 & 5      & 0.160           & 0.283           &                        &                      \\
                          &                      &                             & 0.219 & 5      & 0.168           & 0.320           &                        &                      \\
\bottomrule
\end{tabular}
\end{adjustbox}
\caption{\label{tab:parameters}The parameters describing the properties of the random wave field when the statistics are taken. \textcolor{red}{We set $g=9.8\:\text{ms}^{-2}$, therefore all the quantities scales 1:1 to SI units, from a numerical perspective. however, the $200$m and $500$m domain sizes are equivalent since $k_pL_0$ is a constant.} \textcolor{blue}{Effective values of the two steepness parameters $\sigma$ and $k_pH_s$ are taken from the averaging window $t=100$ to $120$, and vary slightly due to different directional spreading parameter $N$.} }
\end{table}