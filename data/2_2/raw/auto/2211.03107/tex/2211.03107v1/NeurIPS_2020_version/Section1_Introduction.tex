\section{Introduction}

Finance is a particularly challenging playground for deep reinforcement learning (DRL) \cite{sutton2018reinforcement,hambly2021recent}, including investigating market fragility \cite{raberto2001agent}, developing profitable strategies \cite{xiong2018practical, yang2020deep,zhang2020deep}, and assessing portfolio risk \cite{lussange2021modelling, bao2019multiagent}. However, establishing near-real market environments and benchmarks on financial reinforcement learning are challenging due to three major factors, namely, low signal-to-noise ratio (SNR) of financial data, survivorship bias of historical data, and model overfitting in the backtesting stage. Such a \textit{simulation-to-reality gap} \cite{DulacArnold2020AnEI,dulac2019challenges} degrades the performance of DRL strategies in real markets. Therefore, high-quality market environments and DRL benchmarks are crucial for the research and industrialization of data-driven financial reinforcement learning.


Existing works have applied various DRL algorithms in financial applications \cite{lussange2021modelling, liu2021finrl, karpe2020multi, pricope2021deep}. Many of them have shown better trading performance in terms of cumulative return and Sharpe ratio. Several recent works \cite{lussange2021modelling, amrouni2021abides, karpe2020multi} showed the great potential of DRL-based market simulators that are not publicly available yet. Therefore, these works are difficult to reproduce. The FinRL library \cite{liu2020finrl,liu2021finrl} provided an open-source framework for financial reinforcement learning. However, it focused on guaranteeing reproducibility of backtesting performance while several market environments were provided.  A workshop version of FinRL-Meta \cite{finrl_meta_2021} provided data processors to access and clean unstructured market data, but it did not provide benchmarks back then.

The \textsf{DataOps} paradigm \cite{DataOps, atwal2019practical, ereth2018dataops} refers to a set of practices, processes, and technologies that combines automated data engineering and agile development \cite{ereth2018dataops}. It helps reduce the cycle time of data engineering and improve data quality. To deal with financial big data (usually unstructured), we follow the \textsf{DataOps} paradigm and implement an automatic pipeline in Fig. \ref{fig:conventional vs neofinrl}(left): task planning, data processing, training-testing-trading, and monitoring agents' performance. Through this pipeline, we continuously produce DRL benchmarks on dynamic market datasets.

In this paper, we present an openly accessible FinRL-Meta library that has been actively maintained by the AI4Finance community. We aim to create an infrastructure to enable near real-time paper trading and facilitate the real-world adoption of financial reinforcement learning. This is relevant to the broader RL research community since it provides a rare case of a task that can be tested against real-world performance without major investment, while robotics requires simulation or expensive equipment and games are available in simulations.

Fig. \ref{fig:conventional vs neofinrl}(right) shows an overview of data-driven financial reinforcement learning.  First, following a DataOps paradigm \cite{DataOps, atwal2019practical, ereth2018dataops}, we provide hundreds of market environments through an automatic pipeline that collects dynamic datasets from real-world markets and processes them into standard gym-style market environments.
Second, we reproduce popular papers as benchmarks, including high-frequency stock trading, cryptocurrency trading and stock portfolio allocation, serving as stepping stones for users to design new strategies. With the help of the data engineering pipeline, we hold our benchmarks
on cloud platforms so that users can visualize their own results and assess the relative performance via community-wise competitions. Third, FinRL-Meta provides tens of Jupyter/Python demos as educational materials, organized in a curriculum, and a documentation website to serve the rapidly growing community. 

The remainder of this paper is organized as follows. Section 2 reviews existing works. Section 3 describes challenges and presents an overview of our FinRL-Meta framework. Section 4 describes how we process data into market environments. In Section 5, we benchmark popular DRL papers. Finally, we conclude this paper in Section 6.

\begin{figure}[t]
\centering
\includegraphics[scale = 0.082]{figs/finrl_meta_dataops.png}
\hspace{5mm}
\includegraphics[scale = 0.17]{figs/neo.png}
\caption{DataOps paradigm (left) and data-driven financial reinforcement learning (right).}
\label{fig:conventional vs neofinrl}
\vspace{-4mm}
\end{figure}
