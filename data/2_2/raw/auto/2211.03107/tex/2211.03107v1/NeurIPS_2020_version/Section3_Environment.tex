
\section{Financial Big Data and DataOps for Dynamic Datasets} \label{datasets}

\begin{figure}
\centering
\includegraphics[scale=0.45]{figs/FinRL-Meta-Data_layer_v2_Cropped.pdf}
\caption{Data layer of FinRL-Meta.}
\label{fig:datalayer}
\vspace{-2mm}
\end{figure}

Financial big data is usually unstructured in shape and form. We process four types of data \cite{de2018advances} into market environments, including fundamental data (e.g., earning reports), market data (e.g., OHLCV data), analytics (e.g., news sentiment), and alternative data (e.g., social media data, ESG data).


\subsection{Data Layer for Unstructured Financial Big Data}

\textbf{Automated pipeline for data-driven financial reinforcement learning}:  We follow the DataOps paradigm \cite{ereth2018dataops, atwal2019practical, DataOps} in the data layer. As shown in Fig. \ref{fig:datalayer}, we establish a standard pipeline for financial data engineering, which processes data from different sources into a unified market environment following the \textit{de facto} standard of OpenAI gym \cite{brockman2016openai}. We automate this pipeline with a data processor that implements the following functionalities:
\begin{itemize} [leftmargin=*]
\item \textbf{Data accessing}:
Users can connect data APIs of different market platforms via our common interfaces. Users can access data agilely by specifying the start date, end date, stock list, time interval, and other parameters. FinRL-Meta has supported more than $30$ data sources, covering stocks, cryptocurrencies, ETFs, forex, etc. 
\item \textbf{Data cleaning}:
Raw data retrieved from different data sources are usually of various formats and with erroneous or missing data to different extents. It makes data cleaning highly time-consuming. With a data processor, we automate the data cleaning process. In addition, we use stock ticker name and data frequency as unique identifiers to merge all types of data into one unified data table.
\item \textbf{Feature engineering}: In feature engineering, FinRL-Meta aggregates effective features which can help improve model predictive performance. We provide various types of features, including but not limited to fundamental, market, analytics, and alternative features. Users can quickly add features using open-source libraries or add user-defined features. Users can add new features in two ways: 1) Write a user-defined feature extraction function directly. The returned features are added to a feature array. 2) Store the features in a file, and put it in a default folder. Then, an agent can read these features from the file.
\end{itemize}

\textbf{Automated feature engineering}: 
FinRL-Meta currently supports four types of features:
\begin{itemize} [leftmargin=*]
\item \textbf{Fundamental features}:
Fundamental features are processed based on the earnings data in SEC filings queried from WRDS. The data frequency is low, typical quarterly, e.g., four data points in a year. To avoid information leakage, we use a two-month lag beyond the standard quarter end date, e.g., Apple released its earnings report on 2022/07/28 for the third quarter (2022/06/25) of year 2022. Thus for the quarter between 04/01 and 06/30, our trade date is adjusted to 09/01 (same method for other three quarters). We also provide functions in our data processor for calculating financial ratios based on earnings data such as earnings per share (EPS), return on asset (ROA), price to earnings (P/E) ratio, net profit margin, quick ratio, etc.

\item \textbf{Market features}:
 Open-high-low-close price and volume data are the typical market data we can directly get from querying the data API. They have various data frequencies, such as daily prices from YahooFinance, TAQ (Millisecond Trade and Quote) from WRDS. In addition, we automate the calculation of technical indicators based on OHLCV data by connecting the Stockstats\footnote{Github repo: \url{https://github.com/jealous/stockstats}} or TA-lib library\footnote{Github repo: \url{https://github.com/mrjbq7/ta-lib}} in our data processor, such as Moving Average Convergence Divergence (MACD), Average Directional Index (ADX), Commodity Channel Index (CCI), etc.

\item \textbf{Analytics features}:
We provide news sentiment for analytics features. First, we get the news headline and content from WRDS \cite{xinyi_2019}. Next, we use NLTK.Vader\footnote{Github repo: \url{https://github.com/nltk/nltk}} to calculate sentiment based on the sentiment compound score of a span of text by normalizing the emotion intensity (positive, negative, neutral) of each word. For the time alignment with market data, we use the exact enter time, i.e., when the news enters the database and becomes available, to match the trade time. For example, if the trade time is every ten minutes, we collect the previous ten minutes' news based on the enter time; if no news is detected, then we fill the sentiment with 0.
\item \textbf{Alternative features}:
Alternative features are useful, but hard-to-obtain from different data sources \cite{de2018advances}, such as ESG data, social media data, Google trend searches, etc. ESG (Environmental, social, governance) data are widely used to measure the sustainability and societal impacts of an investment. The ESG data we provide is from the Microsoft Academic Graph database, which is an open resource database with records of scholar publications. We have functions in our data processor to extract AI publication and patent data, such as paper citations, publication counts, patent counts, etc. We believe these features reflect companies' research and development capacity for AI technologies \cite{fang2019practical,chen2020quantifying}. It is a good reflection of ESG research commitment.
\end{itemize}

\subsection{Environment Layer for Creating Dynamic Market Environments}
\label{sect:env_layer}

FinRL-Meta follows the OpenAI gym-style \cite{brockman2016openai} to create market environments using the cleaned data from the data layer. It provides hundreds of environments with a common interface. Users can build their environments using FinRL-Meta's interfaces, share their results and compare a strategy's trading performance. Following the gym-style \cite{brockman2016openai}, each environment has three functions as follows:
\begin{itemize}[leftmargin=*]
    \item \texttt{reset()} function resets the environment back to the initial state $s_0$
    \item \texttt{step()} function takes an action $a_t$ from the agent and updates state from $s_t$ to $s_{t+1}$.
    \item \texttt{reward()} function computes the reward value transforming from $s_t$ to $s_{t+1}$ by action $a_t$.
\end{itemize}
Detailed descriptions can be found in \cite{yang2020deep}\cite{gort2022deep}.


We plan to add more environments for users' convenience. For example, we are actively  building market simulators using Limit-order-book data \footnote{Github repo: \url{https://github.com/AI4Finance-Foundation/Market_Simulator}}, where we simulate the market from the playback of historical limit-order-book-level data and an order matching mechanism. We foresee the flexibility and potential of using a Hidden Markov Model (HMM) \cite{mamon2007hidden}  or a generative adversarial net (GAN) \cite{goodfellow2014generative} to generate market scenarios \cite{coletta2021towards}.

\textbf{Incorporating trading constraints to model market frictions}:
To better simulate real-world markets, we incorporate common market frictions (e.g., transaction costs and investor risk aversion) and portfolio restrictions (e.g., non-negative balance). 
\begin{itemize}[leftmargin=*]
\item \textbf{Flexible account settings}: Users can choose whether to allow buying on margin or short-selling.
\item \textbf{Transaction cost}: We incorporate the transaction cost to reflect market friction, e.g., $0.1\%$ of each buy or sell trade.
\item \textbf{Risk-control for market crash}: In FinRL \cite{liu2020finrl,liu2021finrl}, a turbulence index \cite{kritzman2010skulls} is used to control risk during market crash situations. However, calculating the turbulence index is time-consuming. It may take minutes, which is not suitable for paper trading and live trading. We replace the financial turbulence index  with the volatility index (VIX) \cite{whaley2009understanding} that can be accessed immediately.
\end{itemize}

\textbf{Multiprocessing training via vector environment}:
We utilize GPUs for multiprocessing training, namely, the vector environment technique of Isaac Gym \cite{makoviychuk2021isaac}, which significantly accelerates the training process.  In each CUDA core, a trading agent interacts with a market environment to produce transitions in the form of $\{$state, action, reward, next state$\}$. Then, all the transitions are stored in a replay buffer and later used to update a learner. By adopting this technique, we successfully achieve the multiprocessing simulation of hundreds of market environments to improve the performance of DRL trading agents on large datasets.

\subsection{Advantages}
Our DataOps pipeline is automatic, which gives us the following three advantages.

\textbf{Curriculum for newcomers}: We provide an educational curriculum, as shown in Fig. \ref{fig:tutorials},
for community newcomers with different levels of proficiency and learning goals. Users can grow programming skills by gradually changing the data/environment layer following instructions on our website.

\textbf{Benchmarks on cloud}: We provide demos on a cloud platform, Weights \& Biases \footnote{Website: \url{https://wandb.ai/site}}, to demonstrate the training process. We define the hyperparameter sweep, training function, and initialize an agent to train and tune hyperparameters. On the cloud platform Weights \& Biases, users are able to visualize their results and assess  the relative performance via community-wise competitions.

\textbf{Curriculum learning for agents}: Based on FinRL-Meta (a universe of market environments, say $\geq 100$), one is able to construct an environment by sampling data samples from multiple market datasets, similar to XLand \cite{team2021open}. In this way, one can apply the curriculum learning method \cite{team2021open} to train a generally capable agent for several financial tasks.
