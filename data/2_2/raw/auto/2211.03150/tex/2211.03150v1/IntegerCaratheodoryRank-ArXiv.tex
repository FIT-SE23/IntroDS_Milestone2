\documentclass[fullpage]{article}

\usepackage{amsmath,amsthm,amsfonts,amssymb,amscd,epsf,epsfig,psfrag,enumerate,bm,tikz,bbold}
\usepackage[linesnumbered]{algorithm2e}

\newtheorem{theorem}{Theorem}
\newtheorem{lemma}[theorem]{Lemma}
\newtheorem{problem}[theorem]{Problem}
\newtheorem{proposition}[theorem]{Proposition}
\newtheorem{definition}[theorem]{Definition}
\newtheorem{remark}[theorem]{Remark}
\newtheorem{corollary}[theorem]{Corollary}
\newtheorem{conjecture}[theorem]{Conjecture}
\newtheorem{assumption}[theorem]{Assumption}
\newtheorem{observation}[theorem]{Observation}
\newtheorem{notation}[theorem]{Notation}
\newtheorem{example}[theorem]{Example}
\newtheorem{claim}[theorem]{Claim}

\newcommand{\ve}{\boldsymbol}
\newcommand{\Cr}{\operatorname{CR}}
\newcommand{\aCr}{\operatorname{CR^{\rm a}}}

\DeclareMathOperator{\vol}{vol}
\DeclareMathOperator{\adj}{adj}
\DeclareMathOperator{\rank}{rank}
\DeclareMathOperator{\diag}{diag}
\DeclareMathOperator{\lcm}{lcm}
\DeclareMathOperator{\lin}{lin}
\DeclareMathOperator{\inte}{int}
\DeclareMathOperator{\cone}{cone}
\DeclareMathOperator{\conv}{conv}
\DeclareMathOperator{\im}{im}
\DeclareMathOperator{\pos}{pos}
\DeclareMathOperator{\supp}{supp}
\DeclareMathOperator{\intt}{int}
\DeclareMathOperator{\sign}{sign}
\DeclareMathOperator{\diam}{diam}
\DeclareMathOperator{\aff}{aff}
\DeclareMathOperator{\parr}{par}

%
%\newcommand{\ba}{\mathbf{a}}
%\newcommand{\bb}{\mathbf{b}}
%\newcommand{\bc}{\mathbf{c}}
%\newcommand{\bd}{\mathbf{d}}
%\newcommand{\bg}{\mathbf{g}}
%\newcommand{\bh}{\mathbf{h}}
%\newcommand{\br}{\mathbf{r}}
%\newcommand{\bu}{\mathbf{u}}
%\newcommand{\bv}{\mathbf{v}}
%\newcommand{\bw}{\mathbf{w}}
%\newcommand{\bx}{\mathbf{x}}
%\newcommand{\by}{\mathbf{y}}
%\newcommand{\bz}{\mathbf{z}}
\newcommand{\R}{\mathbb{R}}
\newcommand{\Z}{\mathbb{Z}}
\newcommand{\N}{\mathbb{N}}
\newcommand{\Q}{\mathbb{Q}}
\newcommand{\T}{\mathbb{T}}
\newcommand{\bA}{\bm{A}}
\newcommand{\bx}{\bm{x}}
\newcommand{\bC}{\bm{C}}
\newcommand{\bh}{\bm{h}}
\newcommand{\bB}{\bm{B}}
\newcommand{\ba}{\bm{a}}
\newcommand{\br}{\bm{r}}
\newcommand{\be}{\bm{e}}
\newcommand{\bS}{\bm{S}}
\newcommand{\by}{\bm{y}}
\newcommand{\bz}{\bm{z}}
\newcommand{\bc}{\bm{c}}
\newcommand{\bP}{\bm{P}}
\newcommand{\bQ}{\bm{Q}}
\newcommand{\bb}{\bm{b}}
\newcommand{\bd}{\bm{d}}
\newcommand{\bv}{\bm{v}}
\newcommand{\bw}{\bm{w}}
\newcommand{\bU}{\bm{U}}
\newcommand{\bV}{\bm{V}}
\newcommand{\bW}{\bm{W}}
\newcommand{\bH}{\bm{H}}
\newcommand{\bT}{\bm{T}}
\newcommand{\bR}{\bm{R}}
\newcommand{\blambda}{\bm{\lambda}}
\newcommand{\bmu}{\bm{\mu}}
\newcommand{\btau}{\bm{\tau}}

\newcommand{\BIGOP}[1]{\mathop{\mathchoice%
{\raise-0.22em\hbox{\huge $#1$}}%
{\raise-0.05em\hbox{\Large $#1$}}{\hbox{\large $#1$}}{#1}}}
\newcommand{\bigtimes}{\BIGOP{\times}}
% nur fuer Bigboxplus andere Korrekturen
\newcommand{\BIGboxplus}{\mathop{\mathchoice%
{\raise-0.35em\hbox{\huge $\boxplus$}}%
{\raise-0.15em\hbox{\Large $\boxplus$}}{\hbox{\large $\boxplus$}}{\boxplus}}}

%\title{New Bounds for the Integer Carath\'{e}odory Rank}
%\author{Iskander Aliev
%\and Martin Henk
%\and Mark Hogan
%\and Stefan Kuhlmann
%\and Timm Oertel
%}

\usepackage{authblk}
\title{New Bounds for the Integer Carath\'{e}odory Rank}
\author[1]{Iskander Aliev}
\author[2]{Martin Henk}
\author[1]{Mark Hogan}
\author[2]{Stefan Kuhlmann}
\author[3]{Timm Oertel}
\affil[1]{Cardiff University, United Kingdom}
\affil[2]{Technische Universit\"{a}t Berlin, Germany}
\affil[3]{Friedrich-Alexander-Universität Erlangen-Nünrberg, Germany}
\date{}                     %% if you don't need date to appear
\setcounter{Maxaffil}{0}
\renewcommand\Affilfont{\itshape\small}


\begin{document}
	\maketitle
	
\noindent \textbf{Abstract.} 
Given a pointed rational $n$-dimensional cone $C$, we obtain new parametric  and asymptotic upper bounds for the integer Carath\'{e}odory rank $\Cr(C)$, defined as the smallest integer $k$ such that any integer vector in $C$ can be expressed as a non-negative integer combination of at most $k$ elements from the Hilbert basis of $C$.  
Firstly, we significantly  improve  previously known  bounds
on the integer Carath\'{e}odory rank in an asymptotic setting, where we only consider ``most'' integer vectors in~$C$.
Secondly, we show that the equality $\Cr(C)=n$, referred to as the integer Carath\'{e}odory property, holds in arbitrary dimension for cones that possess  polyhedral representations with bimodular matrices.
%$n\times n$ subdeterminants bounded by two. 
Furthermore, we obtain a new upper bound on $\Cr(C)$ for simplicial cones represented by $k$-modular matrices with $k\le n$. Similar results are established for  cones represented by  generating sets.  



	\section{Introduction}

Given a pointed rational polyhedral cone $C\subset \R^n$, a classical result by Cara\-th\'eodory states that each point of $C$ is a non-negative combination of at most $n$ vectors which lie on extreme rays of $C$.
%
As an integral analogue one can consider the inclusion--minimal generating set $S\subset C\cap \Z^n$, called the \emph{Hilbert basis} of $S$, such that any vector in $C\cap\Z^n$ can be expressed as a non-negative integer combination of vectors in $S$.
%
Note that this set is finite and unique, provided that the cone $C$ is rational and pointed.  
%
In this setting, an integral analogue to Carath\'{e}odory's theorem aims to answer the following question:
%
{\em What is the minimum number $k$ of Hilbert basis elements needed to express any given integer vector in the cone $C$?}
%
We will refer to $k$ as the \emph{integer Carath\'eodory rank} of $C$, and denote it with $\Cr(C)$. 
In the theory of integer programming, Hilbert bases are strongly related to {\em TDI-systems}, see, e.g., \cite[Chapter 22.3]{schrijvertheorylinint86}, and {\em test sets} \cite{Aardal_Weismantel_Wolsey}.  In particular, the bounds for the integer Carath\'eodory rank provide bounds for the size of the {\em Graver bases} \cite{Graver1975} associated with integer programs in the standard form.
%
The integer Carath\'eodory rank has also been studied in connection with \emph{matroids}~\cite{pinamatroidcararank2003,gijswijtipdicp2012} and general \emph{integer cones}~\cite{alievAverkovLoeraOertel21,alievdeloesparelindio2017,eisenbrandshmonincaratheodorybounds06}.

Cook, Fonlupt, and Schrijver showed in~\cite{CookFS1986} that the integer Carath\'eodory rank can be bounded solely in terms of the dimension $n$.
%
The currently best upper bound 
\begin{equation}\label{Sebo}
\Cr(C) \le 2n-2
\end{equation}
 was obtained by Seb\H{o} in \cite{sebohilbertbasisdreidim90}.
%
It was also conjectured in \cite{sebohilbertbasisdreidim90} that for any $n$-dimensional  cone the \emph{integer Carath\'eodory property} (ICP) holds, that is $\Cr(C) = n$.
%
This conjecture was disproved by Bruns et al. in \cite{brunsgubehenkcounterexampleintcara99},
%
where it was shown that for every $n$ there exists an $n$-dimensional cone $C$ with $\Cr(C) \ge \lfloor\tfrac{7}{6} n\rfloor$.
%



%
Known results on the integer Carath\'eodory rank lead to two interesting and long-standing open questions:
%

\noindent
{\em -- What cones have the integer Carath\'eodory property?} Seb\H{o} \cite{sebohilbertbasisdreidim90} proved that the (ICP) holds for all cones with dimension at most three. In Section \ref{half-space}, we show that the (ICP) holds in arbitrary dimension for cones that possess a polyhedral representation with small subdeterminants. A similar result is established in Section \ref{Generator} for the cones represented by a generating set.
%

\noindent
{\em -- What is the best possible upper bound for the integer Carath\'eodory rank in terms of $n$?}
%
Note that reducing \eqref{Sebo} to the bound of the form $\Cr(C)\le cn$ with $c<2$ is already a difficult problem. 
In particular, it would disprove a conjecture of Gubeladze \cite{gubeladzesurveynormal2023} on the Carath\'eodory rank of the cones associated with certain polytopes.
%
Bruns and Gubeladze \cite{brunsgubeladzenormalpoly1999} introduced the {\em asymptotic integer Carath\'eodory rank} $\aCr(C)$, that is the smallest $k$ such that ``almost all'' vectors in $C \cap \Z^n$ can be represented by $k$ Hilbert basis elements (we give a rigorous definition of   $\aCr(C)$ in Section \ref{asymptotic}). Clearly, $\aCr(C)\le \Cr(C)$ and it was shown in \cite{brunsgubeladzenormalpoly1999} that $\aCr(C) \le 2n-3$.
In Section \ref{asymptotic} we reduce the latter bound to $\aCr(C) \le \lfloor \tfrac{3}{2}n\rfloor$. We also show that for every $n$ there exists an $n$-dimensional cone $C$ with $\aCr(C) \ge \lfloor\tfrac{7}{6} n\rfloor$. 
%
In addition, in line with our results on the (ICP), we improve \eqref{Sebo} for simplicial cones that have a representation with small subdeterminants (Sections \ref{half-space}--\ref{Generator}).


%See also~\cite{schmidtschinzelproblem69,lowfiniteintegercaraset1976,.,.,.}.



%From the optimization perspective, the integer Carath\'eodory problem can be reformulated as follows.
%%
%Consider an integer optimization problem 
%\[\max\{\bc^T \bx : \bA\bx=\bb,\; \bx\ge0,\;\bx\in\Z^n\},\]
%with $\bA\in\Z^{m\times n}$, $\bb\in\Z^m$ and $\bc\in\Z^n$.
%%
%Is it possible to bound the support of feasible or optimal points, provided they exist?
%%
%If the columns of $\bA$ form a Hilbert basis, then for any feasible $\bb$ there exists a feasible solution with support at most on $\Cr$; see Subsection \ref{ss_notation_def} for the definition of support.
%%
%However, in general the answer highly depends on the constraints, see~\cite{???,???}.
%%

\section{Statement of results}\label{Results}

%
%We study the integer Carath\'eodory rank in terms of the maximal subdeterminants of their representation.
%
%We distinguish between inner and outer representation. 
%
We will separately consider two representations of the cones. Recall that any polyhedral cone $C$ can either be expressed as the intersection of half-spaces, the \textit{polyhedral representation}, or as the non-negative span of generators, the \textit{generator representation}, that is 
\[
C=\left\{\bx\in\R^n : \bA\bx\ge \bm{0}\right\}=\left\{\bx=\sum_{i=1}^t\lambda_i\br^i:\lambda_i\ge0 \text{ for all } i\in[t]\right\},
\]
with appropriate matrix $\bA$ and generators $\br^1,\ldots,\br^t$. 

%Lastly, we investigate the asymptotic behavior of the integer Carath\'eodory rank.
%
%In this setting, we provide a significantly stronger upper bound.
%{\color{red}	My suggestion is to move the following paragraphs into their respective subsections.}
%	
%	Our approach is to investigate the integer Carath\'{e}odory rank and the integer Cara\-theodory property in dependence of the input. We split our work according to the two common representations of polyhedral cones. On one hand there is the \textit{half-space representation} of $C$. Here, our input is a full column rank matrix $\bA\in\Z^{m\times n}$. We define
%	\begin{align*}
%		C(\bA) := \lbrace\bx\in \R^n : \bA\bx\geq\bm{0}\rbrace	
%	\end{align*}
%	to denote the cone $C$. Note that $C(\bA)$ is pointed as $\bA$ has full column rank. On the other hand there is the \textit{generator representation}. Here, our input is given by integer vectors $\br^1,\ldots,\br^t\in \Z^n$. We refer to these integer vectors as \textit{generators}. The cone $C$ can be expressed as $\pos \lbrace \br^1,\ldots,\br^t\rbrace$, where $\pos X$ denotes the set of all finite non-negative combinations of elements in $X\subseteq\Z^n$.
%	
%	Given a cone $C(\bA)$, we study the integer Carath\'{e}odory rank of $C(\bA)$ with respect to the parameter
%	\begin{align*}
%		\Delta(\bA) := \max \left\lbrace \left|\det \bB\right| : \bB \text{ is an } n\times n \text{ submatrix of }\bA\right\rbrace.
%	\end{align*}
%	This parameter plays a key role in various recent work coming from Integer Programming. There, the challenge is to understand the influence of this parameter on the complexity of algorithms in Integer Programming and its underlying geometry.
%	
%	We define analogously a parameter for $\pos \lbrace\br^1,\ldots\br^t\rbrace$. First, let us assume that $\br^1,\ldots,\br^t$ span $\R^n$. Then, we define
%	\begin{align*}
%		\Delta(\br^1,\ldots,\br^t) := \max \left\lbrace \left| \det (\br^{i_1},\ldots,\br^{i_n})\right| : i_1,\ldots,i_n\in \lbrack t\rbrack\right\rbrace,
%	\end{align*}
%	i.e., the largest volume of a parallelepiped spanned by $n$ linearly independent generators. If the generators do not span $\R^n$, then $\pos\lbrace \br^1,\ldots,\br^t\rbrace$ is a lower-dimensional cone of dimension $k$ for some $k<n$. In this case, we generalize our definition to  $\Delta(\br^1,\ldots,\br^t)$ being the normalized maximal volume of a parallelepiped formed by a linear independent subset of $\lbrace \br^1,\ldots,\br^t\rbrace$ of size $k$, where normalization means that we divide by the lattice determinant of the integer lattice $\lin \lbrace \br^1,\ldots,\br^k\rbrace\cap\Z^n$, where $\lin \lbrace\br^1,\ldots,\br^k\rbrace$ denotes the linear hull of $\br^1,\ldots,\br^k$.
%	
%	Throughout the paper, we extensively study the case when $\br^1,\ldots,\br^t$ are linearly independent and $t = n$, i.e., the cone is full-dimensional and \textit{simplicial}. Note that in this case the Hilbert basis is contained in the parallelepiped spanned by the primitive generators; see Subsection \ref{ss_notation_def} for details. We remark that studying the (ICP) for simplicial cones is already a difficult challenge. In particular, it is not known whether simplicial cones have the (ICP). If this would be the case, then for every, not necessarily simplicial, cone $C$ there exists a set $S$ whose elements integrally generate every integer vector in $C$ with at most $\dim C$ elements of $S$ and, in particular, $S$ is contained in the zonotope, $Z = \left\lbrace \bx \in\R^n : \sum_{i = 1}^t\lambda_i\br^i\text{ with }\lambda_i\in \lbrack 0,1)\right\rbrace{\color{red}\cup\{\br^1,\ldots,\br^t\}}$, spanned by the primitive generators. This is due to the fact that each cone admits a triangulation into simplicial cones by just using the primitive generators; see, e.g., \cite[Chapter 2.2]{deloerarambausantostriangulations2010}.
%	

	\subsection{Polyhedral representation}\label{half-space}
	For a full column rank matrix $\bA\in\Z^{m\times n}$, we consider a rational pointed cone
	\begin{align*}
		C(\bA) := \lbrace\bx\in \R^n : \bA\bx\geq\bm{0}\rbrace\,
	\end{align*}
%Note that $C(\bA)$ is pointed as $\bA$ has full column rank. 
and study the integer Carath\'{e}odory rank of $C(\bA)$ with respect to the parameter
	\begin{align*}
		\Delta(\bA) := \max \left\lbrace \left|\det \bB\right| : \bB \text{ is an } n\times n \text{ submatrix of }\bA\right\rbrace.
	\end{align*}
	We refer to $\bA$ as \textit{$\Delta$-modular} if $\Delta(\bA)=\Delta$.
	%This parameter plays a key role in various recent work in integer optimization. 
	Significant effort has been recently made to understand the influence of this parameter on the complexity of the integer programming algorithms and their underlying geometry; see  \cite{artmannweiszen17,celayakuhlpaarweis22,eisenweissteinitz18} for some developments in this direction. 

%\cite{artmannweiszen17,bonisummaeisenbranddiameterpoly14,celayakuhlpaarweis22,eisenweissteinitz18,Fiorini2022IntegerPW}
	
	Our first result shows that the (ICP) holds when the $n\times n$ minors of $\bA$ are bounded by two.
	\begin{theorem}
		\label{thm_outer_unimod_bimod}
		Let $\bA\in\Z^{m\times n}$ with $\Delta(\bA)\in\lbrace 1,2\rbrace$. Then, $\Cr(C(\bA))=n$.
	\end{theorem}
	We note that the counterexample in \cite{brunsgubehenkcounterexampleintcara99} has a half-space representation with $\Delta(\bA) = 144$. Thus, somewhere between two and 144 the (ICP) fails. It is open whether cones with $\Delta(\bA) = 3$ have the (ICP).
	
	We can say more when $C(\bA)$ is \textit{simplicial}, i.e., $\bA\in \Z^{n\times n}$ is invertible. Note that $\left|\det\bA\right| = \Delta(\bA)$. The following result  essentially says that there exists a reduction to some lower-dimensional problem if the dimension of the cone is larger than the determinant of the constraint matrix.
	\begin{lemma}
		\label{lemma_outer_simplicial_reduction}
		Let $\bA\in\Z^{n\times n}$ be such that $1\leq \left|\det \bA\right|\leq n$. For every $\bz \in \intt C(\bA)\cap \Z^n$, there exist a Hilbert basis element $\bh\in C(\bA)\cap \Z^n$ and $\lambda\in \N$ such that $\bz-\lambda\bh$ lies on a facet of $C(\bA)$, that is
		\begin{align*}
			\ba_i^\top(\bz-\lambda\bh) = 0,
		\end{align*}
		where $\ba_i^\top$ is a row of $\bA$. Furthermore, if the (ICP) holds for all simplicial cones with dimension at most $\left|\det\bA\right| - 1$, then $C(\bA)$ has the (ICP).
	\end{lemma}
	
	Due to a result by Seb\H{o} we know that every cone with dimension at most three has the (ICP); see \cite[Theorem 2.2]{sebohilbertbasisdreidim90}. Observe that the zero-dimensional cone, $C(\bA) = \lbrace \bm{0}\rbrace$, trivially has the (ICP). Therefore, we can immediately apply the last statement from Lemma \ref{lemma_outer_simplicial_reduction} and get the corollary below.
	
	\begin{corollary}
		\label{cor_outer_simplicial_small_det}
		Let $\bA\in\Z^{n\times n}$ with $1\leq\left|\det \bA\right|\leq 4$. Then, $\Cr(C(\bA))=n$.
	\end{corollary}
	
	Another application of Lemma \ref{lemma_outer_simplicial_reduction} is the following novel bound on the integer Carath\'{e}o\-dory rank of simplicial cones depending on the dimension $n$ and $\left|\det \bA\right|$.
	
	\begin{theorem}
		\label{thm_outer_bound_cara_rank}
		Let $\bA\in\Z^{n\times n}$ with $5\leq \left|\det\bA\right|\leq n$. Then, \[\Cr(C(\bA))\le n + \left|\det \bA\right| - 3.\]
	\end{theorem}
%	We can interpret this result as an asymptotic version of the (ICP). For fixed $\left| \det \bA\right|$, this bound divided by the dimension asymptotically attains one. Therefore, the integer Carath\'{e}odory rank $n$ holds asymptotically for fixed $\left|\det \bA\right|$.
	
	We prove all the results regarding the half-space representation in Section \ref{section_half_space_proofs}.

	\subsection{Generator representation}\label{Generator} 
	Here, our input is given by $\br^1,\ldots,\br^t\in \Z^n$. We refer to these integer vectors as \textit{generators}. Then, the cone $C$ can be expressed as $\pos \lbrace \br^1,\ldots,\br^t\rbrace$, where $\pos X$ denotes the set of all finite non-negative combinations of elements in $X\subseteq\Z^n$.
	
	In this paper, we restrict to the case when $\br^1,\ldots,\br^t$ are linearly independent and $t = n$, i.e., the cone $\pos \lbrace \br^1,\ldots,\br^n\rbrace$ is full-dimensional and \textit{simplicial}. Note that in this case the Hilbert basis is contained in the parallelepiped spanned by the primitive generators, where $\bz=(z_1, \ldots, z_n)^\top\in \Z^n$ is primitive if $\gcd({\ve z}):=\gcd(z_1, \ldots, z_n) = 1$. We remark that studying the integer Carath\'{e}odory rank for simplicial cones is already a difficult challenge. In particular, it is not known whether simplicial cones have the (ICP). If this would be the case, then for every, not necessarily simplicial, cone $C$ there exists a set $S$ whose elements integrally generate every integer vector in $C$ with at most $\dim C$ elements of $S$ and, in particular, $S$ is contained in the set 
	\[
	Z = \left\lbrace \bx \in\R^n : \sum_{i = 1}^t\lambda_i\br^i\text{ with }\lambda_i\in \lbrack 0,1)\right\rbrace\cup\{\br^1,\ldots,\br^t\}\,.
	\] 
	This is due to the fact that each cone admits a triangulation into simplicial cones by just using the primitive generators; see, e.g., \cite[Chapter 2.2]{deloerarambausantostriangulations2010}. 
	
	It is also open whether the set $S = Z\cap \Z^n$ admits the (ICP). Interestingly, already a small rounding argument implies that the integer Carath\'{e}odory rank with respect to $Z\cap\Z^n$ is $n + 1$. Thus, it would suffice to find an argument which drops this rank by one to prove the (ICP) for this choice of $S$.
	
	Let $\bR = (\br^1,\ldots,\br^n)$ be the matrix with columns $\br^1,\ldots,\br^n$. Similar to the half-space representation, we analyze the integer Carath\'{e}odory rank with respect to the input size $\left|\det \bR\right|$. Thereby, our results resemble the statements in the half-space representation. 
	For the sake of brevity, we write $\pos\bR$ instead of $\pos\lbrace \br^1,\ldots,\br^n\rbrace$.
	
%%	Our results are based on induction. For the sake of clarity of the inductive arguments, we work in a more general setting. Let $\Lambda\subseteq \R^n$ be a lattice. Given linear independent lattice vectors $\br^1,\ldots,\br^k\in\Lambda$, one can define the Hilbert basis of $\pos \lbrace \br^1,\ldots,\br^k\rbrace$ with respect to $\Lambda$ analogously to the case $\Lambda = \Z^n$; compare with Subsection \ref{ss_notation_def}. %General Hilbert basis setting
%%	Moreover, we say that $\pos \lbrace \br^1,\ldots,\br^k\rbrace$ has the (ICP) with respect to $\Lambda$ if every element in $\pos \lbrace \br^1,\ldots,\br^k\rbrace\cap\Lambda$ is a non-negative integral combination of at most $\dim \pos \lbrace \br^1,\ldots,\br^k\rbrace\cap\Lambda$ many Hilbert basis elements, where the Hilbert basis is taken with respect to $\Lambda$. Note that this is not 
%%
%%	Furthermore, we denote by $\Lambda^*$ the dual lattice of $\Lambda$ and by $(\br^1)^*,\ldots,(\br^k)^*$ the dual basis of $\br^1,\ldots,\br^k$ in $\lin\lbrace \br^1,\ldots,\br^k\rbrace$; see Subsection \ref{ss_notation_def} for the definitions of the dual lattice and basis. Let $(\br^i)^\perp$ be the orthogonal complement of the linear space spanned by $\br^i$ and $\pi_i :\R^n \to (\br^i)^\perp$ be the orthogonal projection for $i\in\lbrack k\rbrack$.
%	Let $\br^\perp$ denote the orthogonal complement of the linear space spanned by the non-zero vector $\br\in \lin \bR$ and $\pi : \lin \bR \to \br^\perp$ the orthogonal projection onto $\br^\perp$. In the statement below, we refer to the orthogonal projection having the (ICP) if $\pi \left(\pos \bR\right)$ has the (ICP) with respect to the lattice spanned by $\pi (\lin \bR \cap \Z^n)$. Indeed, this is equivalent to our setting with the integer lattice $\Z^n$ due to the following transformation. If $\bv^1,\ldots,\bv^k$ is a basis of $\pi(\lin \bR \cap \Z^n)$, we can supplement this basis with $\bw^1,\ldots,\bw^{n-k}$ such that $\bB = (\bv^1,\ldots,\bv^k,\bw^1,\ldots,\bw^{n-k})$ is invertible. We return to the integer lattice $\Z^n$ by transforming everything with $\bB^{-1}$. This does not alter the integer Carath\'{e}odory rank and the (ICP) of $\bB^{-1}\cdot \pi(\pos \bR)$ with respect to $\Z^n$.
%	
%	Additionally, we denote by $(\br^1)^*,\ldots,(\br^k)^*$ the dual basis of $\br^1,\ldots,\br^k$ in $\lin\bR$ and by $\Lambda^*$ the dual lattice of a lattice $\Lambda$; see Subsection \ref{ss_notation_def} for the definitions.
%	\begin{lemma}
%		\label{lemma_inner_bounded_complexity}
%		Let $\bR = (\br^1,\ldots,\br^k)$ for linear independent $\br^1,\ldots,\br^k\in\Z^n$ and $L = \lin\bR$.
%		\begin{enumerate}
%			\item If $(\br^i)^* \in (L\cap\Z^n)^*$ for some $i\in\lbrack k\rbrack$ and $\pos\lbrace\br^1,\ldots,\br^{i-1},\br^{i+1},\ldots,\br^k\rbrace$ has the (ICP), then $\pos\bR$ has the (ICP).
%			\item If $(\br^i)^*- (\br^j)^*\in (L\cap\Z^n)^*$ for some $i,j\in\lbrack k\rbrack$ with $i\neq j$ and the orthogonal projections $\pi_i: L \to (\br^i)^\perp$ and $\pi_j: L\to (\br^j)^\perp$ of $\pos\bR$ have the (ICP) with respect to $\pi_i(L \cap \Z^n)$ and $\pi_j(L \cap \Z^n)$, then $\pos\bR$ has the (ICP).
%		\end{enumerate}
%		Furthermore, if the (ICP) holds for all simplicial cones with dimension at most $\Delta(\br^1,\ldots,\br^k) - 1$, then $\pos\bR$ has the (ICP).
%	\end{lemma}
	
%	Similar to Corollary \ref{cor_outer_simplicial_small_det}, we obtain the statement below since every cone with dimension at most three has the (ICP). Thereby, we note that $\Delta(\br^1,\ldots,\br^n) = \left|\det\bR\right|$ if $\br^1,\ldots,\br^n$ are linearly independent.
	
	\begin{proposition}
		\label{cor_inner_simplicial_small_det}
		Let $\bR\in\Z^{n\times n}$ with $1\leq\left|\det \bR\right|\leq 4$. Then, $\Cr(\pos\bR)=n$.
	\end{proposition}
	
	We also get a bound for the integer Carath\'{e}odory rank in dependence of $\left|\det\bR\right|$.
	
	\begin{theorem}
		\label{theorem_inner_bound_cara_rank}
		Let $\bR\in\Z^{n\times n}$ be such that $5\leq \left| \det \bR \right|\leq n$. Then, \[\Cr(\pos\bR)=n + \left| \det \bR\right| - 3.\]
%		the integer Carath\'{e}odory rank of the simplicial cone $\pos\bR$ is bounded above by $n + \left| \det \bR\right| - 3$.
	\end{theorem}
	
	We briefly discuss a possible generalization to non-simplicial cones. Given a cone $\pos \lbrace \br^1,\ldots,\br^t\rbrace$, one can consider its integer Carath\'{e}odory rank with respect to
	\begin{align*}
		\Delta(\br^1,\ldots,\br^t) := \max \left\lbrace \left| \det (\br^{i_1},\ldots,\br^{i_n})\right| : i_1,\ldots,i_n\in \lbrack t\rbrack\right\rbrace.
	\end{align*}
	Note that $\Delta(\br^1,\ldots,\br^t) = \left|\det\bR\right|$ if $t = n$. Although general, not necessarily simplicial, cones with $\Delta(\br^1,\ldots,\br^t) = 1$ have the (ICP), due to the fact that every simplicial subcone is unimodular, already the counterexample in \cite{brunsgubehenkcounterexampleintcara99} admits a representation with $\Delta(\br^1,\ldots,\br^t) = 3$. It is open whether cones with $\Delta(\br^1,\ldots,\br^t) = 2$ have the (ICP).
	
	For this extended abstract, we omit the proofs for the statements regarding the generator representation.
%	We prove all the results regarding the generator representation in Section \ref{section_ray_proofs}.
%	\newline
%	\newline
%	\noindent
%	Based on our analysis, we are able to present special cones with the (ICP) such as dual cones of simplicial Gorenstein cones, where $\left|\det\bR\right|$ has at most four divisors. All the results and proofs in this direction can be found in Section \ref{section_special_cones}.


\subsection{Asymptotic integer Carath\'eodory Rank}
\label{asymptotic}
 Bruns and Gubeladze \cite{brunsgubeladzenormalpoly1999} introduced the {\em asymptotic integer Carath\'eodory rank}  of $C$, denoted by $\aCr(C)$, which is the smallest integer $k$ such that there exists a set $D \subseteq C \cap \Z^n$ such that:
 
\begin{enumerate}
 \item[(i)] one has
\[
\lim_{\delta \to \infty} \frac{\# D\cap[-\delta,\delta]^n}{\# C \cap \Z^n\cap[-\delta,\delta]^n} = 1 \text{ and }
\]
 \item[(ii)] any point in $D$ can be expressed as a non-negative integer combination of at most $k$ Hilbert basis elements.
 \end{enumerate}
%
 
 
% \begin{minipage}[c]{0.8\textwidth}
%\color{blue} \begin{enumerate}
% \item[i)]
% $
%\lim_{\delta \to \infty} \frac{\# D\cap[-\delta,\delta]^n}{\# C \cap \Z^n\cap[-\delta,\delta]^n} = 1 \text{ and }
%$
% \item[ii)] any point in $D$ can be expressed as a non-negative integer combination of at most $k$ Hilbert basis elements.
% \end{enumerate}
% \end{minipage}\\[0.3cm]
%%
That is ``almost all'' vectors in $C \cap \Z^n$ can be represented by $k$ Hilbert basis elements.\footnote{Bruns and Gubeladze's definition of the asymptotic integer Carath\'eodory rank differs slightly, but one can easily show that they are equivalent.}
%
%The asymptotic integer Carath\'eodory number $aiC(m)$ is the largest asymptotic integer Carath\'eodory rank over all pointed, $m$-dimensional, rational cones.
%

Clearly,  $n\le\aCr(C)\le\Cr(C)$. Bruns and Gubeladze  \cite{brunsgubeladzenormalpoly1999} showed that 
\begin{equation}\label{BG_asympt}
\aCr(C)\le 2n-3 
\end{equation}
and that there exists a cone $C$ such that $\aCr(C)>n$.
%

%
The latter bound can be tightened using results of Bruns and Gubeladze \cite{brunsgubeladzenormalpoly1999} as follows.
\begin{proposition}\label{asymLowBound}
For every $n\in \Z_{>0}$ there exists an $n$-dimensional cone $C_n$ such that
\begin{equation}\label{asympt_lower_bound}
\aCr(C_n)\ge\lfloor\tfrac{7}{6}n\rfloor.
\end{equation}
\end{proposition}

%({\color{blue}We should verify this independently. Is there an elegant way to do this?})
The lower bound \eqref{asympt_lower_bound} coincides with the best known lower bound on the maximal integer Carath\'eodory rank due to Bruns et al. \cite{brunsgubehenkcounterexampleintcara99}. Further, the estimate \eqref{BG_asympt} on $\aCr(C)$ is smaller than Seb\"o's bound \eqref{Sebo} (that remains the best known bound on $\Cr(C)$ for over three decades) only by one. In this vein, it is natural to look for a bound on  $\aCr(C)$ that provides a stronger improvement compared to the bound \eqref{Sebo}.

In order to quantify the desirable improvement, note that any upper estimate of the form $cn$ with $c<2$ for $\Cr(C)$ would disprove a conjecture of Gubeladze \cite{gubeladzesurveynormal2023} on the Carath\'eodory rank of  cones associated with normal polytopes.  
%
Our next result shows that the asymptotic integer Carath\'eodory rank
admits an upper bound of the latter form with the leading coefficient $c=3/2$.

\begin{theorem}\label{asymUpBound}
Let $C$ be a pointed, rational, $n$-dimensional cone.
%
It holds that
\[
\aCr(C)\le\lfloor\tfrac{3}{2}n\rfloor.
\]
\end{theorem}


	
	\subsection{Notation and definitions}\label{ss_notation_def}
	We introduce additional notation and definitions which we use throughout the paper. We abbreviate $\lbrack m\rbrack :=\lbrace 1,\ldots,m\rbrace$. Given $\bA\in\Z^{m\times n}$, $I\subseteq\lbrack m \rbrack$, and $J\subseteq \lbrack n\rbrack$, then $\bA_{I,J}$ denotes the submatrix of $\bA$ with rows indexed by $I$ and columns indexed by $J$. 
%	%{\color{red}In the case $I=\lbrace i\rbrace$, we use $\ba_i$ instead of $\bA_{\lbrace i\rbrace,\cdot}$. DO WE NEED THIS}
%	
In the same manner, given a vector $\bx\in\R^n$ and a set $I\subseteq\lbrack n\rbrack$, we denote by $\bx_I\in\R^{|I|}$ the vector with coordinates indexed by $I$. The \textit{support of $\bx$} is defined as 
	$
		\supp(\bx) = \lbrace i\in\lbrack n\rbrack : \bx_i \neq 0\rbrace.
	$
	
%	A cone $C$ is pointed if $\bm{0}$ is the vertex of $C$. The \textit{Hilbert basis} of a pointed rational polyhedral cone $C$ with respect to $\Z^n$ is the minimal set of integral vectors in $C$ such that every element in $C\cap \Z^n$ is a non-negative integral combination of the elements in the set. We refer to integer vectors in the Hilbert basis as \textit{Hilbert basis elements}. 
%%	It is known that the Hilbert basis is finite and unique for each pointed cone $C$; see \cite{vandercorputhilberbasisunique1931}. 
%	An important property of the Hilbert basis is that its elements can not be written as the sum of two non-zero integral vectors in $C\cap\Z^n$, i.e., if $\bh\in C\cap\Z^n$ is a Hilbert basis element, then for all $\by^1,\by^2\in C\cap \Z^n$ such that $\bh = \by^1 + \by^2$ we either have $\by^1 = \bm{0}$ or $\by^2 = \bm{0}$.
	
%	Let $\br^1,\ldots,\br^t\in \Z^n$ be primitive, $C = \pos\lbrace \br^1,\ldots,\br^t\rbrace$ be a cone and $\by\in C\cap \Z^n$ with $\by = \sum_{i=1}^t\lambda_i\br^i$ and $\lambda_i\geq 0$ for all $i\in\lbrack t\rbrack$. If $\lambda_j > 1$ for some $j\in\lbrack t\rbrack$, then we can decompose $\by$ into two integral vectors such that
%	\begin{align*}
%		\by = \left((\lambda_j - 1)\br^j + \sum_{i = 1, i\neq j}^t \lambda_i\br^i \right) + \br^j.
%	\end{align*}
%	Both vectors are integral, contained in $C$, and not zero. Therefore, $\by$ is not a Hilbert basis element. So the Hilbert basis elements are contained in the zonotope spanned by the primitive generators, 
%	\begin{align*}
%		Z = \left\lbrace \bx \in\R^n : \sum_{i = 1}^t\lambda_i\br^i\text{ with }\lambda_i\in \lbrack 0,1)\right\rbrace\cup\lbrace\br^1,\ldots,\br^t\rbrace.
%	\end{align*}
	
	
%	The spindle of some cone $C(\bA)$ with respect to $\by\in C(\bA)$ is defined by
%	\begin{align*}
%		S(\by) := \lbrace \bx\in\R^n : \bm{0}\leq \bA\bx\leq \bA\by\rbrace.
%	\end{align*}
	By $\intt X$ we denote the interior of a set $X$ and $\aff X$ is the affine hull of $X$.
	A \textit{lattice} $\Lambda$ is a discrete subgroup of $\R^n$. A {\em basis} of $\Lambda$ consists of $k$ linearly independent vectors $\bb^1,\ldots,\bb^k\in \Lambda$ such that $\Lambda = (\bb^1,\ldots,\bb^k)\Z^k$. If $k = n$, the lattice is full-dimensional. The \emph{determinant} of a full-dimensional lattice $\Lambda$ is given by $\det\Lambda = \left|\det\bB\right|$ for $\bB = (\bb^1,\ldots,\bb^n)$. For a  \textit{sublattice} $\tilde{\Lambda}\subseteq\Lambda$, the quotient group $\Lambda / \tilde{\Lambda}$ is finite and the cardinality of $\Lambda / \tilde{\Lambda}$ equals $\frac{\det \tilde{\Lambda}}{\det\Lambda}$. For a general introduction to the theory of lattices see for example~\cite{GruLek87}.
	
%	The vectors $(\bb^1)^*,\ldots,(\bb^k)^*\in \lin\Lambda$ form the \textit{dual basis} of $\bb^1,\ldots,\bb^k$ if 
%	\begin{align*}
%		(\bb^i)^T(\bb^j)^* = \begin{cases}
%			1, \text{ for } i = j,\\
%			0, \text{ otherwise }
%		\end{cases}.
%	\end{align*}
%	The \textit{dual lattice} is defined by
%	\begin{align*}
%		\Lambda^* := \lbrace \bx\in \lin \Lambda : \by^T\bx\in \Z \text{ for all }\by\in\Lambda\rbrace.
%	\end{align*}
%	Note that $(\bb^1)^*,\ldots,(\bb^k)^*\in \Lambda^*$ is a basis of $\Lambda^*$ if $\Lambda = (\bb^1,\ldots,\bb^k)\Z^k$ for $i\in\lbrack k\rbrack$. Moreover, we have $\det \Lambda^* = (\det \Lambda)^{-1}$. In the case $k = n$, the dual basis of $\bB= (\bb^1,\ldots,\bb^n)$ is given by the columns of $\bB^{-T}$. 
	
	We denote by $GL(n,\Z)$ the group of all $n\times n$ unimodular matrices, i.e., $\bA \in\Z^{n \times n}$ and $|\det \bA| = 1$. The standard unit vectors of $\R^n$ are denoted by $\be^1,\ldots,\be^n$.
	
%	\subsection{Related work}
%	
%	{\color{red} Can we add this to the top, or remove it?}
%	Despite the existence of a counterexample to the question whether every cone has the (ICP), there remain challenging open problems in the area. One of them is to identify classes of cones which have the (ICP). In \cite{gijswijtipdicp2012}, the authors show that cones which arise from polyhedra with additional properties, e.g., polyhedra defined by a totally unimodular constraint matrix, or (poly)matroid base polytopes have the (ICP). Further, the integer Carath\'{e}odory rank of the integer cone corresponding to the incidence vector of bases of a matroid $M$ admits an upper bound of $n + r(M) - 1$, where $r(M)$ denotes the rank of $M$; see \cite{pinamatroidcararank2003}.
%	
%	Seb\H{o} introduced stronger versions of the (ICP) \cite{sebohilbertbasisdreidim90}. He asked whether the Hilbert basis gives a unimodular covering of the cone or, even stronger, a unimodular triangulation. Both questions turned out to be false since they would imply (ICP). Nevertheless, this lead to the search of quantitative upper bounds on the size of a set which admits a unimodular cover or triangulation; see \cite[Chapter 3B and 3C]{brunsgubeladzepolyringsktheory2009} or \cite{brunsthadenunimodtriangbound2017, thadenunimodtriangbound2021} for some more recent work.
%%	
%%	Another line of research is the investigation of asymptotic bounds related to the integer Carath\'{e}odory rank. Bruns and Gubeladze proved the asymptotic upper bound of $2n - 3$ on the integer Carath\'{e}odory rank \cite{brunsgubeladzenormalpoly1999}. This improves the current best deterministic upper bound by one. 
%	
%	In a recent survey \cite{gubeladzesurveynormal2023}, Gubeladze conjectured that for cones arising from certain polytopes the ratio of the worst case integer Carath\'{e}odory rank in fixed dimension divided by the dimension tends to two as the dimension grows.
%	
%	Instead of analyzing the integer Carath\'{e}odory rank with respect to the Hilbert basis of a cone, it is also of interest to study an integer Carath\'{e}odory rank for arbitrary generating sets with respect to fixed elements; see \cite{eisenbrandshmonincarathéodorybounds06}. Moreover, the authors provide a link to problems in Integer Programming such as the cutting stock problem.
	
	
	



	
	\section{Parametric bounds} %Proving half-space representation statements}
	\label{section_half_space_proofs}
	In this section, we prove Carath\'{e}odory-type results for cones given by a constraint matrix $\bA\in\Z^{m\times n}$ with full column rank.
	
	Throughout this section, we work with the polytope
	\begin{align*}
		P_{\bm{1}}(\bA) := \lbrace \bx\in\R^n : \bm{0}\leq \bA\bx\leq\bm{1}\rbrace.
	\end{align*}
	Note that $P_{\bm{1}}(\bA)$ is full-dimensional if and only if $C(\bA)$ is full-dimensional. Most of the proofs in this section are based on the following observation. 
	\begin{lemma}\label{lemma_outer_walk_to_face}
		Let $\bA\in\Z^{m\times n}$ be a full column rank matrix and $P_{\bm{1}}(\bA)\cap\Z^n\backslash\lbrace\bm{0}\rbrace\neq \emptyset$. Given $\bz\in \intt C(\bA)\cap \Z^n$, there exists a Hilbert basis element $\bh \in C(\bA)\cap \Z^n$ and $\lambda\in\N$ such that $\bz - \lambda\bh\in C(\bA)$ and $\ba_i^\top(\bz - \lambda\bh) = 0$ for some $i\in\lbrack m\rbrack$.
	\end{lemma}
	\begin{proof}
		Let $\bh\in P_{\bm{1}}(\bA)\cap\Z^n\backslash\lbrace\bm{0}\rbrace$ be chosen such that $\left|\supp(\bA\bh)\right|$ is minimal among all vectors in $P_{\bm{1}}(\bA)\cap\Z^n\backslash\lbrace\bm{0}\rbrace$. We observe that
		\begin{align*}
			\lambda := \min_{i\in\supp(\bA\bh)} \ba_i^\top\bz
		\end{align*}
		already yields the claim for $\bh$ as $\bA\bh\in\lbrace 0,1\rbrace^m$. So it suffices to argue that $\bh$ is a Hilbert basis element. 
		
		
		Let $\by^1,\by^2\in C(\bA)\cap\Z^n$ be such that $\bh = \by^1 + \by^2$. It is sufficient to show that one of the vectors $\by^1$, $\by^2$ is zero. Since $\bA\bh\in\lbrace 0,1\rbrace^m$, we have $\bA\by^i\in\lbrace 0,1\rbrace^m$ as well for $i=1,2$. However, this implies that $\supp(\bA\by^i)\subseteq\supp(\bA\bh)$ for $i=1,2$. The minimality of $\left|\supp(\bA\bh)\right|$ gives us that either $\by^1= \bm{0}$ or $\by^2=\bm{0}$.
	\end{proof}
	Lemma \ref{lemma_outer_walk_to_face} enables us to argue inductively over the dimension $n$. To make this precise, we have to establish a characterization for lower-dimensional faces and their minors. In order to formulate this characterization, we define
	\begin{align*}
		\gcd(\bA) := \gcd(\det \bA_{I,\cdot} : I\subseteq\lbrack m\rbrack \text{ with } |I|= n)
	\end{align*}
	for $\bA\in\Z^{m\times n}$ with $m\geq n$. 
	
	\begin{lemma}
		\label{lemma_outer_fulldim_polyhedra}
		Let $P(\bA,\bb) := \lbrace \bx\in\R^n : \bA\bx\leq\bb\rbrace$ be a polyhedron with $\bA\in\Z^{m\times n}$ having full column rank and $\bb\in\Z^m$. Further, let $F_I:= P(\bA,\bb)\cap \left(\bv + \ker \bA_{I,\cdot}\right)$ be a  $(n-k)$-dimensional face of $P(\bA,\bb)$ with $\aff F_I\cap\Z^n \neq \emptyset$, where $I\subseteq\lbrack m\rbrack$ with $\left| I \right| = k$ and $\bA_{I,\cdot}\bv = \bb_I$ hold. Then, there exists a unimodular transformation $\bU\in GL(n,\Z)$ and orthogonal projection $\pi:\R^n\to\R^{n-k}$ such that 
		\begin{enumerate}
			\item $\pi\left(\bU \cdot F_I\right)$ is a $(n-k)$-dimensional polyhedron defined by an integral matrix which is at most $\left\lfloor\frac{\Delta(\bA)}{\gcd (\bA_{I,\cdot})}\right\rfloor$-modular,
			\item there exists a one-to-one mapping between $F_I\cap \Z^n$ and $\pi\left(\bU \cdot F_I\right)\cap\Z^{n-k}$, and
			\item  $\pi\left(\bU \cdot F_I\right)$ can be defined by an integral right-hand side.
		\end{enumerate}
	\end{lemma}
	\begin{proof}
		We assume without loss of generality that $I = \lbrace 1,\ldots,k\rbrace$.
		
		We begin by presenting the transformation and afterwards we analyze its properties. There exists a unimodular transformation $\bU\in GL(n,\Z)$, such that
		\begin{align*}
			\bA_{I,\cdot}\bU^{-1} = (
			\bH,\bm{0})
		\end{align*}
		for some invertible matrix $\bH\in\Z^{k\times k}$, e.g., by transforming $\bA_{I,\cdot}$ into Hermite normal form. Moreover, we have
		\begin{align*}
			\bA\bU^{-1} = \begin{pmatrix}
				\bH & \bm{0} \\
				\star & \tilde{\bA}
			\end{pmatrix}
		\end{align*}
		for some $\tilde{\bA}\in \Z^{(m-k)\times (n-k)}$ with full column rank.
	 	
	 	Let $\pi : \R^n\to\R^{n-k}$ denote the orthogonal projection onto the last $n-k$ coordinates and let $\tilde{\bz}\in\R^k$ be the unique solution of $\bH\bx = \bb_I$. Then,
	 	\begin{align}\label{proof_lemma_outer_minor_projection}
	 		\pi\left(\bU \cdot F_I\right) = \left\lbrace \bx\in\R^{n-k} : \tilde{\bA}\bx\leq\bb_{\lbrack m \rbrack\backslash I} - \bA_{\lbrack m \rbrack\backslash I,\lbrack k \rbrack}\tilde{\bz}\right\rbrace
	 	\end{align}
	 	which is a $(n-k)$-dimensional polyhedron defined by the integral constraint matrix $\tilde{\bA}$ with not necessarily integral right-hand side. We show that $\tilde{\bA}$ is at most $\left\lfloor\frac{\Delta(\bA)}{\gcd (\bA_{I,\cdot})}\right\rfloor$-modular. For that purpose, let $\bB$ be a $(n-k)\times (n-k)$ submatrix of $\tilde{\bA}$. We can extend the matrix to 
	 	\begin{align*}
	 		\begin{pmatrix}
	 			\bH & \bm{0} \\
	 			\star & \bB
	 		\end{pmatrix}
	 	\end{align*}
 		which is an $n\times n$ submatrix of $\bA\bU^{-1}$ with determinant $\left|\det \bB\right|\left|\det\bH\right|\leq \Delta(\bA)$. We have $\left|\det \bH\right| = \gcd(\bA_{I,\cdot})$ which follows, e.g., from the Smith normal form; see for instance \cite[Chapter 4.4]{schrijvertheorylinint86} for a treatment of Smith normal forms. This and the integrality of $\tilde{\bA}$ imply that $\tilde{\bA}$ is at most $\left\lfloor\frac{\Delta(\bA)}{\gcd( \bA_{I,\cdot})}\right\rfloor$-modular.
 		
 		For the second statement, we observe that applying a unimodular transformation to $F_I$ preserves, one-to-one, integer vectors. So $\aff F_I\cap\Z^n\neq\emptyset$ implies $\aff (\bU \cdot F_I)\cap \Z^n\neq\emptyset$. Recall that $\tilde{\bz}\in\R^k$ denotes the unique solution of $\bH\bx = \bb_I$. Let $\bz\in\R^n$ be the vector $\tilde{\bz}$ with $n-k$ zeros appended. We have
 		\begin{equation*}
		\begin{aligned}
 			\aff (\bU \cdot F_I) &= \bz + \ker \bA_{I,\cdot}\bU^{-1} = \bz + \left\lbrace \bx\in \R^n : \bx_{\lbrack k \rbrack} 
			= \bm{0} \right\rbrace \\ &= \left\lbrace \bx\in \R^n : \bx_{\lbrack k \rbrack} = \tilde{\bz}\right\rbrace.
 		\end{aligned}
		\end{equation*}
 		Thus, $\aff (\bU \cdot F_I)\cap \Z^n\neq\emptyset$ implies $\tilde{\bz}\in\Z^k$. This already means that $\by\in \bU \cdot F_I\cap\Z^n$ if and only if $\pi(\by)\in \pi(\bU \cdot F_I)\cap \Z^{n-k}$. 
 		
 		Last, the right-hand side of $\pi(\bU \cdot F_I)$ is given by $\bb_{\lbrack m \rbrack\backslash I} - \bA_{\lbrack m \rbrack\backslash I,\lbrack k \rbrack}\tilde{\bz}\in\Z^{m-k}$; compare with (\ref{proof_lemma_outer_minor_projection}).
	\end{proof}
	
	As a useful consequence of this result, we can restrict ourselves to full-dimensional cones. The reason for this is that we apply the transformation from Lemma \ref{lemma_outer_fulldim_polyhedra} and obtain a full-dimensional cone in lower dimension. Moreover, the linear hull of every cone contains an integer vector, e.g., the origin. Therefore, the integrality properties carry over to the full-dimensional cone in lower dimension.

	We split the proof of Theorem \ref{thm_outer_unimod_bimod} into the cases $\Delta(\bA) = 1$ and $\Delta(\bA)=2$.
	\begin{proof}[Proof of Theorem \ref{thm_outer_unimod_bimod} for $\Delta(\bA) = 1$]
		We argue inductively over $n$. For $n = 1$, the cone is a ray and, thus, the statement holds. Let $n > 1$ and $\bz\in C(\bA)\cap\Z^n$. If $\bz$ lies on the boundary of $C(\bA)$, we restrict to the face which contains $\bz$ in the relative interior and apply Lemma \ref{lemma_outer_fulldim_polyhedra} with respect to that face. This results in a lower-dimensional cone with $\Delta(\bA) = 1$ and the claim follows by induction. Hence, we assume $\bz\in\intt C(\bA)$, which implies that $C(\bA)$ is full-dimensional.
		
		It suffices to find a non-zero integer vector in $P_{\bm{1}}(\bA)$, pass to the lower-dimensional face via Lemma \ref{lemma_outer_walk_to_face}, and then apply Lemma \ref{lemma_outer_fulldim_polyhedra}. Since $P_{\bm{1}}(\bA)$ is full-dimensional and can be defined by a unimodular matrix with integral right-hand side (c.f. Lemma \ref{lemma_outer_fulldim_polyhedra} part three), every vertex is integral. There are at least two vertices as $n\geq 2$. Hence, one vertex does not equal $\bm{0}$ and, thus, $P_{\bm{1}}(\bA)$ contains a non-zero integral vector.
	\end{proof}
	
	In the following, we say that $\bA$ is \textit{bimodular} if $\Delta(\bA) = 2$. We apply a result by Chirkov and Veselov which states that every full-dimensional polyhedron $P$ defined by a bimodular constraint matrix with integral right-hand side contains an integer vector; see \cite[Theorem 1]{veselovchirkovbimodular09}. 
	
	\begin{proof}[Proof of Theorem \ref{thm_outer_unimod_bimod} for $\Delta(\bA) = 2$]
		We argue again inductively over $n$. Similar to the $\Delta(\bA) = 1$ case, we assume that $n > 1$ and $\bz\in \intt C(\bA)\cap\Z^n$. Furthermore, we assume that every row of $\bA$ defines a facet of $C(\bA)$. If this is not the case, we remove rows of $\bA$ which do not correspond to a facet of $C(\bA)$. This only decreases $\Delta(\bA)$.
		
		As before, it suffices to find a non-zero integer vector in $P_{\bm{1}}(\bA)$, pass to the lower-dimensional face via Lemma \ref{lemma_outer_walk_to_face}, and then apply Lemma \ref{lemma_outer_fulldim_polyhedra}. Let $F_{\ba} :=\lbrace \bx\in P_{\bm{1}}(\bA) : \ba^\top\bx = 1\rbrace$ define a facet of $P_{\bm{1}}(\bA)$ for some row $\ba$ of $\bA$. Our goal is to apply Lemma \ref{lemma_outer_fulldim_polyhedra}. Here, we have to be careful since $\aff F_{\ba}$ might not contain integer vectors. This only occurs when $\gcd( \ba) = 2$. Thus, we distinguish between the cases $\gcd(\ba) = 2$ and $\gcd(\ba) = 1$. 
		
		Let $\gcd(\ba) = 2$. Every row of $\bA$ is facet defining for $C(\bA)$ as we assumed in the beginning of the proof. Therefore, there exists a facet of $P_{\bm{1}}(\bA)$ defined by $\ba^\top\bx = 0$. By Lemma \ref{lemma_outer_fulldim_polyhedra} the facet defined by $\ba$ corresponds to a polytope with unimodular constraint matrix and integral right-hand side. As the linear hull of the facet contains integral vectors, every non-zero vertex is integral. There are at least two vertices since $n\geq 2$. So there exists a non-zero integer vector in the facet.
		
		Let $\gcd(\ba) = 1$. This implies that the affine hull of the facet contains integer vectors. Applying Lemma \ref{lemma_outer_fulldim_polyhedra}, we receive a $(n-1)$-dimensional full-dimensional polytope which is bimodular. Hence, it contains an integer vector by \cite[Theorem 1]{veselovchirkovbimodular09}. Since the right-hand side equals one, this vector can not be $\bm{0}$.
	\end{proof}
	
	\begin{proof}[Proof of Lemma \ref{lemma_outer_simplicial_reduction}]
		First, we claim that the parallelepiped $P_{\bm{1}}(\bA) = \lbrace \bx\in\R^n : \bm{0}\leq \bA\bx\leq \bm{1}\rbrace$ contains a non-zero integer vector. This already implies the first part of our statement by Lemma \ref{lemma_outer_walk_to_face}.
		
		Let $\bA^{-1} = (\bw^1,\ldots,\bw^n)$ and $\Lambda := \bA^{-1}\Z^n$. We have $\bw^i\in\Lambda$ for each $i\in\lbrack n\rbrack$ by definition. Note that $\bA\bw^i = \be^i$ and hence $\bw^1,\bw^1+\bw^2,\ldots,\bw^1+\ldots+\bw^n\in P_{\bm{1}}(\bA)\cap \Lambda$. We analyze these sums with respect to the cosets of $\Lambda / \Z^n$. Note that $\left| \Lambda / \Z^n\right| = \left|\det\bA\right|$ since $\det\Lambda = \left|\det \bA\right|^{-1}$. As $n\geq \left|\det \bA\right|$, either one of the sums is integral or two sums, say $\bw^1+\ldots+\bw^p$ and $\bw^1+\ldots+\bw^q$ for $p < q$, are contained in the same coset of $\Lambda / \Z^n$ by the pigeonhole principle. This implies that $\bw^{p+1}+\ldots+\bw^{q}\in P_{\bm{1}}(\bA)\cap\Z^n$. So the  claim follows.
		
		For the second part of the statement, we assume that $C(\bA)$ is full-dimensional and that $\bz\in \intt C(\bA)\cap \Z^n$ by Lemma \ref{lemma_outer_fulldim_polyhedra}. Further, we assume that $n\geq \left|\det\bA\right|$ otherwise there is nothing show. Hence, we can apply the first claim and reduce the dimension by at least one. Applying Lemma \ref{lemma_outer_fulldim_polyhedra} we obtain a lower-dimensional cone which is at most $\left|\det \bA\right|$-modular. We repeat this procedure until the dimension is at most $\left|\det \bA\right| - 1$. For each step, we use exactly one Hilbert basis element. There are at most $n - (\left|\det \bA\right| - 1)$ steps. As a final step, we use the assumption that every cone with $n\leq \left|\det \bA\right| - 1$ has the (ICP). Thus, we obtain an integral combination of at most $n$ Hilbert basis elements.
	\end{proof}
	
	We remark that our approach of searching for a Hilbert basis element $\bh$ which we can integrally subtract from a given integer vector in order to reach a lower-dimensional face already fails when $\Delta(\bA) = 3$, $n = 2$, and $C(\bA)$ is simplicial, e.g., take 
	\begin{align*}
		\bA = \begin{pmatrix}
			1 & 0 \\ 2 & 3
		\end{pmatrix}.
	\end{align*}
	The Hilbert basis elements of $C(\bA)$ are the vectors $\be^1,\be^2$, $(2,-1)^\top$, and $(3,-2)^\top$. One can check that $P_{\bm{1}}(\bA)\cap\Z^n = \lbrace\bm{0}\rbrace$ and no Hilbert basis element has the desired property for the vector $(7, -3)^\top\in C(\bA)\cap\Z^n$.
	
	\begin{proof}[Proof of Theorem \ref{thm_outer_bound_cara_rank}]
		Since $n\geq \left|\det \bA\right|$, we can apply Lemma \ref{lemma_outer_simplicial_reduction} and reduce the dimension by at least one. Applying Lemma \ref{lemma_outer_fulldim_polyhedra} we obtain a lower-dimensional cone which is at most $\left|\det \bA\right|$-modular. We repeat this procedure until the dimension is at most $\left|\det \bA\right| - 1$. As $\left|\det \bA\right| - 1\geq 2$, we can apply Seb\H{o}'s bound, compare with \cite[Theorem 2.1]{sebohilbertbasisdreidim90}, and get at most $2\left(\left|\det\bA\right| - 1\right) - 2$ Hilbert basis elements in an integral combination. Together with our previous steps, which give us at most $n - (\left|\det\bA\right| - 1)$ Hilbert basis elements in an integral combination, we obtain
		\begin{align*}
			2\left(\left|\det\bA\right| - 1\right) - 2 + \left(n - \left(\left|\det \bA\right| - 1\right)\right) = n + \left|\det \bA\right| - 3.
		\end{align*}
	\end{proof}
	

	
	\section{Asymptotic bounds}
	
	
	\begin{proof}[Proof of Proposition~\ref{asymLowBound}]
Theorem 6.1 in \cite{brunsgubeladzenormalpoly1999} states that if $\aCr(C)=\dim C$, then $\Cr(C)=\dim C$.
%
The counterexample to the integer Carath\'eodory conjecture, shown in \cite{brunsgubehenkcounterexampleintcara99}, provides a 6-dimensional cone $C_6$ with 
$\Cr(C_6)=7$. Hence, by Theorem 6.1, we get the lower bound $\aCr(C_6)>6$.  Furthermore, the inequality $\aCr(C_6)\le\Cr(C_6)$ implies $\aCr(C_6)=7$.

%Suppose $n\geq 6$, otherwise the claim follows from $\aCr(C) \ge n$. 
%
Lemma 4.4 in \cite{brunsgubeladzenormalpoly1999} shows that $\aCr(C\times C')=\aCr(C)+\aCr(C')$.
%
Following the construction in \cite{brunsgubehenkcounterexampleintcara99} and setting
$
C_n=\left(\BIGOP{\times}_{i=1}^{\lfloor n/6 \rfloor}C_6\right)\times C',
$
where $C'$ is any pointed, full-dimensional, rational cone in $\R^{n\bmod 6}$, we obtain a cone that satisfies~\eqref{asympt_lower_bound}.
%Next, Lemma 4.4 in \cite{brunsgubeladzenormalpoly1999} shows that $\aCr(C\times C')=\aCr(C)\cdot\aCr(C')$.
%Hence, using the construction from the proof of Theorem 1.1 in  \cite{brunsgubehenkcounterexampleintcara99}, we obtain a cone $C_n$ that satisfies 
%
%\eqref{asympt_lower_bound}.
%
\end{proof}


%\begin{proof}[Proof of Theorem~\ref{asymUpBound}]
%Let $\{\bh^1,\ldots,\bh^t\}$ denote the Hilbert basis of $C$ and let $\bH\in\Z^{n\times t}$ be the matrix with columns $\bh^1,\ldots,\bh^t$.
%%
%%First we construct a set $D$ as follows.
%%
%Let further
%\begin{equation*}
%\Delta=\max\left\{|\det(\bH_{\cdot,I})|\;:\; I\subseteq[ t],\; |I|=n\right\}.
%\end{equation*}
%%\begin{equation*}
%%\Delta=\max\left\{|\det([\bh_{i_1},\ldots,\bh_{i_n}])|\;:\; i_j\in[ t]\right\}.
%%\end{equation*}
%%
%We define the set
%\[
%D=C\cap\Z^n \setminus\bigcup_{\tau\in{[ t]\choose n-1}}\left\{\sum_{i=1}^t\lambda_i\bh^i \;:\;\begin{array}{l} 0\le\lambda_i \,\quad\quad\text{ for all }i\in\tau,\\0\le\lambda_i<\Delta\text{ for all }i\in[t]\setminus\tau \end{array}\right\}.
%\]
%%
%Since $\#\tau=n-1$, it follows that 
%\[
%\lim_{\delta \to \infty} \frac{\# D \cap[-\delta,\delta]^n}{\# C \cap \Z^n\cap[-\delta,\delta]^n} = 1.
%\]
%%
%
%%
%Given  $\ve b \in \Z^n $, we will use the notation $Q(\bH, {\ve b})$
%%Without loss of generality, we will assume that $A$ has rank $m$.
%%
%%\be\label{matrix_partition}
%%\rank(A)=m\,.
%%\ee
%for the polyhedron in standard form
%%
%\begin{equation*}
%Q(\bH, {\ve b})=\{{\ve x}\in \R^t_{\ge 0}: \bH{\ve x}={\ve b}\}.
%\end{equation*}
%%
%
%Let $\bb\in D$.
%%
%It is sufficient to prove that we can express $\bb$ as a non-negative integer combination of at most $3n/2$ Hilbert basis elements.
%%
%Let us consider the following linear optimization problem:
%%
%\begin{equation}\label{Norm_maximisation}
%\max\left \{\|\bx\|_1: 
%\bx\in Q(\bH, \bb) \right\}\,,
%\end{equation}
%%
%where $\|\cdot \|_1$ stands for the $\ell_1$-norm.
%%
%Since $C$ is pointed and $\bb\in C$, it is clear that the optimization problem is feasible and bounded.
%%
%Let ${\ve \lambda}$ be a optimal vertex solution for \eqref{Norm_maximisation}. Hence, ${\ve \lambda}$ has at most $n$ non-zero entries and, renumbering the coordinates, we may assume that $\lambda_{n+1},\ldots,\lambda_t=0$.
%%
%Furthermore,  the condition $\bb\in D$ implies that $\lambda_1,\ldots,\lambda_n\ge\Delta$.
%%
%
%
%Note that $\bH_{\cdot, [n]}$ is an optimal basis for \eqref{Norm_maximisation}.
%%
%A classical result from the theory of linear programming (see for example~\cite[Section 5.1]{bertsimas-LPbook}) implies that any vector $\ve \alpha = (\alpha_1,\ldots,\alpha_n,0,\ldots,0)^\top\in\R^t_{\geq 0}$ with $\bH \ve \alpha\in \Z^n$ is an optimal solution to the linear optimization problem $\max_{\bx\in Q(\bH, \bH \ve \alpha)}\|\bx\|_1$.
%
%Let $\bar{\ve \lambda}=(\lambda_1,\ldots,\lambda_n)^\top$ and let $\{\bar{\ve\lambda}\}=(\{\lambda_1\},\ldots,\{\lambda_n\})^\top=(\lambda_1-\lfloor \lambda_1 \rfloor,\ldots,\lambda_n-\lfloor \lambda_n \rfloor)^\top$ be the vector of its fractional components.
%%
%Consider the vector  ${\ve r}=\bH_{\cdot,[n]}\{\bar{\ve \blambda}\}\in C\cap\Z^n$. We can write ${\ve r}= \sum_{i=1}^ 
%t\beta_i\bh^i$ with $\beta_i\in\Z_{\ge 0}$.
%%
%Hence,
%%
%\[
%\bb=\sum_{i=1}^n \lfloor\lambda_i\rfloor \bh^i +\sum_{i=1}^t\beta_i\bh^i.
%\]
%%
%
%Observe that ${\ve \beta}=(\beta_1, \ldots, \beta_t)^\top\in Q(\bH, \br)$ by construction and that the linear optimization problem 
%\[
%\max\{\|\bx\|_1:\bx\in Q(\bH, \br)\}
%\]
%%
%has an optimal vertex solution $(\{\lambda_1\}, \ldots, \{\lambda_n\}, 0, \ldots, 0)^\top$. 
%%
%Hence,
%$\sum_{i=1}^t\beta_i\le\sum_{i=1}^n\{\lambda_i\}$.
%%
%
%
%If $\sum_{i=1}^n\{\lambda_i\}\le n/2$, then at most $\lfloor n/2\rfloor$ of the numbers $\beta_i$ can be non-zero and the result follows.
%%
%To settle the case $\sum_{i=1}^n\{\lambda_i\} > n/2$, we consider the vector ${\ve \gamma}=(1-\{\lambda_1\},\ldots,1-\{\lambda_n\})^\top$.
%%
%We have ${\ve s}=\bH_{\cdot,[n]}{\ve \gamma}\in C\cap\Z^n$ and, consequently,  ${\ve s}=\sum_{i=1}^t\delta_i\bh_i$ with  $\delta_i\in\Z_{\ge 0}$. 
%Observe that ${\ve \delta}=(\delta_1, \ldots, \delta_t)^\top\in Q(\bH, {\ve s})$ and that the linear optimization problem 
%%
%\[
%\max\{\|\bx\|_1: \bx\in Q(H, {\ve s})\}
%\]
%%
%has an optimal vertex solution $(\gamma_1, \ldots, \gamma_n, 0, \ldots, 0)^\top$. 
%%
%Hence, 
%\[
%\sum_{i=1}^t\delta_i\le\sum_{i=1}^n\gamma_i< n/2\,
%\]
%%
%and, consequently, ${\ve s}$ can be expressed as non-negative integer combination of strictly less than $n/2$ Hilbert basis elements.
%%
%
%%
%Let $q=|\det(\bH_{\cdot,[n]})|\le\Delta$.
%%
%Then $q{\ve \gamma}$ is integral by Cramer's rule.
%%
%Recall that we have $\bb \in D$ and, in particular, $\lambda_i\ge\Delta$ for all $i\in[n]$.
%Thus, we obtain that
%\begin{equation*}
%\hat\lambda_i=\lambda_i-(q-1)\gamma_i= (\lambda_i + \gamma_i)-q \gamma_i
%\end{equation*}
%%
%are non-negative integers for all $i\in[n]$. 
%%
%We can express $\bb$ as 
%%
%\begin{equation*}
%\begin{aligned}
%{\ve b} &= \sum_{i=1}^n \lambda_i \bh^i = \sum_{i=1}^n (\hat\lambda_i+(q-1)\gamma_i)\bh^i\\
%&=\sum_{i=1}^n \hat\lambda_i \bh^i  + (q-1){\ve s}\,.
%\end{aligned}
%\end{equation*}
%%
%This completes the proof since ${\ve s}$ is the non-negative integral combination of strictly less than $n/2$ Hilbert basis elements.
%%
%\qed
%%
%\end{proof}
\begin{proof}[Proof of Theorem~\ref{asymUpBound}]
Let $\{\bh^1,\ldots,\bh^t\}$ denote the Hilbert basis of $C$ and let $\bH\in\Z^{n\times t}$ be the matrix with columns $\bh^1,\ldots,\bh^t$.
%
%First we construct a set $D$ as follows.
%
Let further
\begin{equation*}
\Delta=\max\left\{|\det(\bH_{\cdot,I})|\;:\; I\subseteq[ t],\; |I|=n\right\}.
\end{equation*}
%\begin{equation*}
%\Delta=\max\left\{|\det([\bh_{i_1},\ldots,\bh_{i_n}])|\;:\; i_j\in[ t]\right\}.
%\end{equation*}
%
We define the set
\[
D=C\cap\Z^n \setminus\bigcup_{\tau\in{[ t]\choose n-1}}\left\{\sum_{i=1}^t\lambda_i\bh^i \;:\;\begin{array}{l} 0\le\lambda_i \,\quad\quad\text{ for all }i\in\tau,\\0\le\lambda_i<\Delta\text{ for all }i\in[t]\setminus\tau \end{array}\right\}.
\]
%
Since $\#\tau=n-1$, it follows that 
\[
\lim_{\delta \to \infty} \frac{\# D \cap[-\delta,\delta]^n}{\# C \cap \Z^n\cap[-\delta,\delta]^n} = 1.
\]
%

%
Given  $\ve b \in \Z^n $, we will use the notation $Q(\bH, {\ve b})$
%Without loss of generality, we will assume that $A$ has rank $m$.
%
%\be\label{matrix_partition}
%\rank(A)=m\,.
%\ee
for the polyhedron in standard form
%
\begin{equation*}
Q(\bH, {\ve b})=\{{\ve x}\in \R^t_{\ge 0}: \bH{\ve x}={\ve b}\}.
\end{equation*}
%

Let $\bb\in D$.
%
It is sufficient to prove that we can express $\bb$ as a non-negative integer combination of at most $3n/2$ Hilbert basis elements.
%
Let us consider the following linear optimization problem:
%
\begin{equation}\label{Norm_maximisation}
\max\left \{\|\bx\|_1: 
\bx\in Q(\bH, \bb) \right\}\,,
\end{equation}
%
where $\|\cdot \|_1$ stands for the $\ell_1$-norm.
%
Since $C$ is pointed and $\bb\in C$, it is clear that the optimization problem is feasible and bounded.
%
Let ${\ve \lambda}$ be a optimal vertex solution for \eqref{Norm_maximisation}. Hence, ${\ve \lambda}$ has at most $n$ non-zero entries and, renumbering the coordinates, we may assume that $\lambda_{n+1}=\ldots=\lambda_t=0$.
%
Furthermore,  the condition $\bb\in D$ implies that $\lambda_1,\ldots,\lambda_n\ge\Delta$.
%


Note that $\bH_{\cdot, [n]}$ is an optimal basis for \eqref{Norm_maximisation}.
%
A classical result from the theory of linear programming (see for example~\cite[Section 5.1]{bertsimas-LPbook}) implies that any vector $\ve \alpha = (\alpha_1,\ldots,\alpha_n,0,\ldots,0)^\top\in\R^t_{\geq 0}$ with $\bH \ve \alpha\in \Z^n$ is an optimal solution to the linear optimization problem $\max_{\bx\in Q(\bH, \bH \ve \alpha)}\|\bx\|_1$.

Let $\ve\mu=(\lambda_1-\lfloor \lambda_1 \rfloor,\ldots,\lambda_n-\lfloor \lambda_n \rfloor)^\top$.
% be the vector of its fractional components.
%
Consider the vector  ${\ve r}=\bH_{\cdot,[n]}\ve\mu\in C\cap\Z^n$. We can write ${\ve r}= \sum_{i=1}^ 
t\beta_i\bh^i$ with $\beta_i\in\Z_{\ge 0}$.
%
Hence,
%
\[
\bb=\sum_{i=1}^n \lfloor\lambda_i\rfloor \bh^i +\sum_{i=1}^t\beta_i\bh^i.
\]
%

Observe that ${\ve \beta}=(\beta_1, \ldots, \beta_t)^\top\in Q(\bH, \br)$ by construction and that the linear optimization problem 
\[
\max\{\|\bx\|_1:\bx\in Q(\bH, \br)\}
\]
%
has an optimal vertex solution $(\mu_1, \ldots, \mu_n, 0, \ldots, 0)^\top$. 
%
Hence,
$\sum_{i=1}^t\beta_i\le\sum_{i=1}^n\mu_i$.
%


If $\sum_{i=1}^n\mu_i\le n/2$, then at most $\lfloor n/2\rfloor$ of the numbers $\beta_i$ can be non-zero and the result follows.
%
To settle the case $\sum_{i=1}^n\mu_i > n/2$, we consider the vector ${\ve \gamma}= (\lceil\lambda_1\rceil-\lambda_1,\ldots,\lceil\lambda_n\rceil-\lambda_n)^\top$.
%
We have ${\ve s}=\bH_{\cdot,[n]}{\ve \gamma}\in C\cap\Z^n$ and, consequently,  ${\ve s}=\sum_{i=1}^t\delta_i\bh_i$ with  $\delta_i\in\Z_{\ge 0}$. 
Observe that ${\ve \delta}=(\delta_1, \ldots, \delta_t)^\top\in Q(\bH, {\ve s})$ and that the linear optimization problem 
%
\[
\max\{\|\bx\|_1: \bx\in Q(H, {\ve s})\}
\]
%
has an optimal vertex solution $(\gamma_1, \ldots, \gamma_n, 0, \ldots, 0)^\top$. 
%
Hence, 
\[
\sum_{i=1}^t\delta_i\le\sum_{i=1}^n\gamma_i< n/2\,
\]
%
and, consequently, ${\ve s}$ can be expressed as non-negative integer combination of strictly less than $n/2$ Hilbert basis elements.
%

%
Let $q=|\det(\bH_{\cdot,[n]})|\le\Delta$.
%
Then $q{\ve \gamma}$ is integral by Cramer's rule.
%
Recall that we have $\bb \in D$ and, in particular, $\lambda_i\ge\Delta$ for all $i\in[n]$.
Thus, we obtain that
\begin{equation*}
\eta_i=\lambda_i-(q-1)\gamma_i= (\lambda_i + \gamma_i)-q \gamma_i
\end{equation*}
%
are non-negative integers for all $i\in[n]$. 
%
We can express $\bb$ as 
%
\begin{equation*}
\begin{aligned}
{\ve b} &= \sum_{i=1}^n \lambda_i \bh^i = \sum_{i=1}^n (\eta_i+(q-1)\gamma_i)\bh^i\\
&=\sum_{i=1}^n \eta_i \bh^i  + (q-1){\ve s}\,.
\end{aligned}
\end{equation*}
%
This completes the proof since ${\ve s}$ is the non-negative integral combination of strictly less than $n/2$ Hilbert basis elements.
%
\end{proof}
	

	
	\bibliographystyle{plain}
	
	\bibliography{references} 

\end{document}
	
