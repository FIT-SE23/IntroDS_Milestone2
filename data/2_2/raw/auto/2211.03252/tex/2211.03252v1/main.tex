% This must be in the first 5 lines to tell arXiv to use pdfLaTeX, which is strongly recommended.
\pdfoutput=1
% In particular, the hyperref package requires pdfLaTeX in order to break URLs across lines.

\documentclass[11pt]{article}

% Remove the "review" option to generate the final version.
% \usepackage[review]{EACL2023}
\usepackage[]{EACL2023}

% Standard package includes
\usepackage{times}
\usepackage{latexsym}

% For proper rendering and hyphenation of words containing Latin characters (including in bib files)
\usepackage[T1]{fontenc}
% For Vietnamese characters
% \usepackage[T5]{fontenc}
% See https://www.latex-project.org/help/documentation/encguide.pdf for other character sets

% This assumes your files are encoded as UTF8
\usepackage[utf8]{inputenc}

% This is not strictly necessary, and may be commented out.
% However, it will improve the layout of the manuscript,
% and will typically save some space.
\usepackage{microtype}

% This is also not strictly necessary, and may be commented out.
% However, it will improve the aesthetics of text in
% the typewriter font.
\usepackage{inconsolata}


% If the title and author information does not fit in the area allocated, uncomment the following
%
%\setlength\titlebox{<dim>}
%
% and set <dim> to something 5cm or larger.

\title{Zero-Shot Classification by Logical Reasoning on Natural Language Explanations}

% Author information can be set in various styles:
% For several authors from the same institution:
% \author{Author 1 \and ... \and Author n \\
%         Address line \\ ... \\ Address line}
% if the names do not fit well on one line use
%         Author 1 \\ {\bf Author 2} \\ ... \\ {\bf Author n} \\
% For authors from different institutions:
% \author{Author 1 \\ Address line \\  ... \\ Address line
%         \And  ... \And
%         Author n \\ Address line \\ ... \\ Address line}
% To start a seperate ``row'' of authors use \AND, as in
% \author{Author 1 \\ Address line \\  ... \\ Address line
%         \AND
%         Author 2 \\ Address line \\ ... \\ Address line \And
%         Author 3 \\ Address line \\ ... \\ Address line}

\author{
Chi Han $^1$, Hengzhi Pei $^1$, Xinya Du $^2$, Heng Ji $^1$ \\
$^1$ University of Illinois at Urbana-Champaign \\
$^2$ The University of Texas at Dallas \\
\texttt{\{chihan3,hpei4,hengji\}@illinois.edu,~xinya.du@utdallas.edu}
}

% \author{Author 1 \and ... \and Author n \\
%         Address line \\ ... \\ Address line}

% First Author \\
%   Affiliation / Address line 1 \\
%   Affiliation / Address line 2 \\
%   Affiliation / Address line 3 \\
%   \texttt{email@domain} \\\And
%   Second Author \\
%   Affiliation / Address line 1 \\
%   Affiliation / Address line 2 \\
%   Affiliation / Address line 3 \\
%   \texttt{email@domain} \\
% }

 
\newcommand{\model}{CLORE}
\newcommand{\Model}{CLORE}


\usepackage{blindtext}
\usepackage{graphicx}
\usepackage{booktabs}
\usepackage{algorithm}
\usepackage{algorithmic}
\usepackage{color}
\usepackage{xcolor}
\usepackage{amsmath}
\usepackage{amsfonts}
\usepackage{makecell}
\usepackage[normalem]{ulem}
\usepackage{multirow}

\definecolor{OrangeRed}{rgb}{1.0, 0.27, 0.0}
\usepackage{xparse}
\NewDocumentCommand{\heng}{ mO{} }{}
% \NewDocumentCommand{\heng}{ mO{} }{\textcolor{OrangeRed}{\textsuperscript{\textit{Heng}}\textsf{\textbf{\small[#1]}}}}
\NewDocumentCommand{\hanchi}{ mO{} }{}
% \NewDocumentCommand{\hanchi}{ mO{} }{\textcolor{green}{\textsuperscript{\textit{Chi}}\textsf{\textbf{\small[#1]}}}}
\NewDocumentCommand{\hengzhi}{ mO{} }{}
% \NewDocumentCommand{\hengzhi}{ mO{} }{\textcolor{blue}{\textsuperscript{\textit{Hengzhi}}\textsf{\textbf{\small[#1]}}}}


\begin{document}
\maketitle
\begin{abstract}
    \begin{abstract}
% Modern ConvNets
Since the recent success of Vision Transformers (ViTs), explorations toward ViT-style architectures have triggered the resurgence of ConvNets.
% Novel view: interaction
In this work, we explore the representation ability of modern ConvNets from a novel view of multi-order game-theoretic interaction, which reflects inter-variable interaction effects w.r.t.~contexts of different scales based on game theory.
% MogaNet
Within the modern ConvNet framework, we tailor the two feature mixers with conceptually simple yet effective depthwise convolutions to facilitate middle-order information across spatial and channel spaces respectively.
% Experiments illustration
In this light, a new family of pure ConvNet architecture, dubbed MogaNet, is proposed, which shows excellent scalability and attains competitive results among state-of-the-art models with more efficient use of parameters on ImageNet and multifarious typical vision benchmarks, including COCO object detection, ADE20K semantic segmentation, 2D\&3D human pose estimation, and video prediction.
% Highlight results
Typically, MogaNet hits 80.0\% and 87.8\% top-1 accuracy with 5.2M and 181M parameters on ImageNet, outperforming ParC-Net-S and ConvNeXt-L while saving 59\% FLOPs and 17M parameters.
% code (arxiv & final version)
The source code is available at \url{https://github.com/Westlake-AI/MogaNet}.
\vspace{-1.0em}


\end{abstract}

    \footnote{Code and data are available at \url{https://github.com/Glaciohound/CLORE}}
\end{abstract}

\section{Introduction}
\section{Introduction}
\label{sec:introduction}
Reliable, fast, and efficient data processing is crucial given the growing volumes of data in both industry and research.
These needs are often addressed by using distributed dataflow frameworks like Spark~\cite{Zaharia2010}, and Flink~\cite{Carbone2015}.
As these frameworks' handle parallelism, distribution, and fault tolerance, they make it easier for users to create scalable data-parallel programs.
The resulting applications can use a variety of compute clusters for data processing.

However, it is still difficult to choose and configure resources in a way that closely meets user-specific goals and constraints~\cite{RajanKCK16,cloudcomputingchallenges2018}.
Numerous strategies have been put forth to assist users, and they can be grouped into two categories:
Model-based techniques~\cite{MaoAMK16,RajanKCK16,ShahAKW19,AlSayehS19,KirchoffXMR19,ChenLLWZ21silhouette,ScheinertTZWAWK21,WillTSBK21,AlSayehMJPS22} often rely on access to historical performance data, however, historical workload execution data is not always available.
Search-based techniques~\cite{AlipourfardLCVY17,HsuNFM18,bilal2020finding,klimovic2018selecta,fekry2020accelerating,MendesCRG20,LiuXL20,SongZLSFDS21} conduct costly profiling runs prior to executing the actual workload utilizing all, or a fraction, of the input data to iteratively create performance models.

Often enough though, the optimized resource configuration is only relevant for the workload at hand. 
Information about the underlying infrastructure are solely obtained implicitly, i.e., by measuring the performance of the target workload in one execution context.
As a consequence, a thorough understanding of utilized resources and their capabilities is lacking and insights gained cannot be easily transferred to other contexts, for instance, when profiling new workloads with different resource demands. 
This requires repeated profiling overhead for reoccurring or comparable workloads that could be avoided, rendering current approaches less resource-efficient than they could be.

Addressing these limitations, we present \emph{Perona}, a novel approach to explicit and robust infrastructure fingerprinting. 
It motivates the usage of common sets and configurations of benchmarking tools to assess the full capabilities of target infrastructures and to make the obtained benchmarking metrics directly comparable.
This explicit fingerprinting operation transparently reveals the characteristics of resources and allows ranking them.
Perona discards irrelevant benchmarking metrics in a data-driven manner by learning a dense, low-dimensional representation of input metric vectors. 
With these, more sophisticated resource decisions can be made for big data analytics, e.g., with regard to scheduling or resource allocations.
To be able to assess a recent benchmark execution, our approach incorporates results of prior benchmark executions, which is particularly useful for detecting resource degradation. 

\emph{Contributions}. The contributions of this paper are:

\begin{itemize}
    \item A novel approach for incorporating infrastructure fingerprinting into model-based methods for optimized resource configuration of workloads through ranking of resources and detection of degrading resource behavior.
    \item A method for context-aware representation learning of benchmark metrics, thereby not only discarding insignificant features but also taking prior benchmark runs and corresponding machine metrics into account. 
    \item An openly available implementation\footnote{\url{https://github.com/dos-group/perona-infrastructure-fingerprinting}} of Perona which we evaluated with regard to common metrics and in interplay with resource configuration methods for distributed dataflows and scientific workflows. 
\end{itemize}

\emph{Outline}. \autoref{sec:related_work} discusses the related work.
\autoref{sec:approach} describes the three main aspects of our approach in detail. 
\autoref{sec:evaluation} presents the results of our evaluation.
\autoref{sec:conclusion} concludes the paper and gives an outlook on future work.


\section{Related Work}

\label{sec:related}

% \heng{the '~' means a blank space, so if you use '~' you get two white blanks}
\paragraph{Classification with Auxiliary Information}

This work studies the problem of classification through explanations, which is related to classification with auxiliary information. For example, in the natural language processing field,~\citet{JMLR:v11:mann10a, ganchev2010posterior} incorporate side information (such as class distribution and linguistic structures) as a regularization for semi-supervised learning. Some other efforts convert crowd-sourced explanations into pseudo-data generators for data augmentation when training data is limited~\citep{wang2020learning, hancock2018training, DBLP:conf/iclr/WangQZ0YN0R20}. However, these explanations are limited to describing linguistic patterns (e.g., ``this is class X because word A directly precedes B''), and are only used for generating pseudo labels. A probably more related topic is using explanations for generating a vector of features for classification~\cite{srivastava2017joint, srivastava2018zero}. However, they either learn a black-box final classifier on features or rely on observed attributes of data, so their ability of generalization is limited. 

% \heng{\sout{Check Manling's CVPR22 paper and its related work, there are many related papers about using NL explanations for cross-media tasks}}
The computer vision area widely uses class-level auxiliary information such as textual metadata, class taxonomy and expert-annotated feature vectors~\citep{yang2022comprehensive, akata2015evaluation, xian2016latent, lampert2009learning, akata2015label, samplawski2020zero}. However, the use of label names and class explanations is mainly limited to a simple text encoder~\citep{akata2015evaluation, xian2016latent, liu2021goal, norouzi2014zero}. This processing treats every text as one simple vector in similarity space or probability space, whereas our method aims to reason on the explanation and exploit its compositional nature.
% Recently, there has been impressive progress on vision-language pre-trained models (VLPMs)~\citep{li2022clip, radford2021learning, li2019visualbert, kim2021vilt}. These methods are trained on large-scale high-quality vision-text pairs with contrastive learning~\citep{radford2021learning, kim2021vilt, li2019visualbert} or mask prediction objective~\citep{kim2021vilt, li2019visualbert}. 
% However, these model mostly focus on representation learning than understanding the compositionality in language.
% As we will show through experiments, VLPMs fits data better at the cost of zero-shot generalization performance.
% As we will show through experiments, this approach performs less well when the inputs go beyond simple pattern matching and require more reasoning. On the contrary, 









\paragraph{Few-shot and Zero-shot Learning with Language Guidance}
% \heng{\sout{There are a lot of work on zero-shot for NLP tasks, you are missing a lot, such as Lifu's early work: https://blender.cs.illinois.edu/paper/zeroshot2018.pdf or you can focus on cross-media zero-shot tasks only}}
This work deals with the problem of learning with limited data with the help of natural language information, which is closely related to few-shot and zero-shot learning with language guidance in NLP domain~\citep{hancock2018training, DBLP:conf/iclr/WangQZ0YN0R20, srivastava2017joint, srivastava2018zero, yu2022building, huang2018zero}. Besides the discussions in the previous subsection, recent pre-trained language models (LMs)~\citep{kenton2019bert, liu2019roberta, tam-etal-2021-improving, gao-etal-2021-making, yu2022building} have made huge progress in few-shot and zero-shot learning. To adapt LMs to downstream tasks, common practices are to formulate them as cloze questions~\citep{tam-etal-2021-improving, schick2021s, menon2022clues, li2022piled} or use text prompts~\citep{mishra2022cross, ye2021crossfit, sanh2022multitask, aghajanyan-etal-2021-muppet}. These approaches hypothetically utilize the language models' implicit reasoning ability~\citep{menon2022clues}. However, in this work we demonstrate with empirical evidence that adopting an explicit logical reasoning approach can provide better interpretability and robustness to linguistic biases.

In computer vision, recently there has been impressive progress on vision-language pre-trained models (VLPMs)~\citep{li2022clip, radford2021learning, li2019visualbert, kim2021vilt}. These methods are trained on large-scale high-quality vision-text pairs with contrastive learning~\citep{radford2021learning, kim2021vilt, li2019visualbert} or mask prediction objective~\citep{kim2021vilt, li2019visualbert}. 
However, these model mostly focus on representation learning than understanding the compositionality in language.
As we will show through experiments, VLPMs fits data better at the cost of zero-shot generalization performance.

There are also efforts in building benchmarks for cross-task generalization with natural language explanations or instructions~\citep{mishra2022cross, menon2022clues}. We use the CLUES benchmark~\citep{menon2022clues} in our experiment for structured data classification, but leave~\citet{mishra2022cross} for future work as its instructions are focused on generally describing the task instead of defining categories/labels.





\paragraph{Neuro-Symbolic Reasoning for Question Answering} is also closely related to our approach. Recent work~\citep{mao2019neuro, yi2018neural, han2019visual} has demonstrated its efficacy in question answering, concept learning and image retrieval. Different from our work, previous efforts mainly focus on question answering tasks, which contains abundant supervision for parsing natural language questions. In classification tasks, however, the number of available explanations is much more limited (100$\sim$1000), which poses a higher challenge on the generalization of reasoning ability.


% \subsection{Language Classification from Explanations}

% On language classification tasks,~\cite{zhou2020nero} uses neural rules \heng{what do you mean by 'neural rules'?} on relation extraction, but the rules are explicitly given in symbolic form, and do not contain multi-step reasoning. A more similar direction is to better utilize natural language explanations from annotators. \cite{srivastava-etal-2017-joint} first proposes a joint concept learning and semantic parsing method for improving classification performance, but the programs are feature generators the classification is not explainable nor generalizable to new rules. \heng{I cannot parse this "but the programs are feature generators the classification is not explainable nor generalizable to new rules.". re-write it} \cite{wang2020learning, hancock2018training, ke2021knowledge} transforms natural language explanations into logical forms for data augmentation in the classification task, and~\cite{murty2020expbert} explores using explanations to engineer sentence representations that are beneficial for downstream tasks. Different from the present paper, these attempts focus on data augmentation rather than direct classification. Besides text classification,~\cite{menon2022clues} proposes a benchmark for classification from natural language explanations, but the task is on structured inputs (like tabular data) instead of unstructured texts. In the same setting,~\citep{srivastava2018zero} converts natural language rules to classification constraints for modelling language quantifiers.

% \subsection{Neural Reasoning}

% Our work is closely related to neuro-symbolic methods in the age of deep learning, which aims to utilize symbolic structures like programs. Symbolic structures can better convey compositional meanings so they are beneficial to modeling natural language texts which are inherently compositional. \heng{any citation to support this claim in the previous sentence?}
% Successful applications are often applied to question answering tasks such as visual reasoning \cite{hu2017learning, by2018closing, DBLP:conf/iclr/MaoGKTW19, yi2018neural}, knowledge base queries 
% \heng{'knowledge base queries' is not a task, write the task name here}
% \cite{andreas-etal-2016-learning, kapanipathi-etal-2021-leveraging} and language-driven navigation \cite{mao2021grammar}. Different from previous attempts, in this paper we adopt a different setting, text classification, which often lacks explicit questions so the model is not informed of how to reason on each input.
% \xd{\cite{berant-etal-2013-semantic} use question-answer pairs as training supervision and obtain the executable programs, but it requires mapping textual input to logical forms as preprocessing. While our method is end-to-end....}

%%% Our work is also related to predictive methods based on social elements for social tasks. \cite{emelin-etal-2021-moral} explores action and consequence classification based on norms on Social Chemistry 101 dataset \cite{forbes-etal-2020-social}. In the area of legal artificial intelligence, symbol-based methods, which aim to incorporate  interpretable symbols \cite{ashley2017artificial, surden2019artificial}, are more intriguing to legal professionals than embedding-based methods, which aim to learn text features directly \cite{zhong-etal-2020-nlp}. Some efforts are made into information extraction and legal element extraction in the legal domain \cite{truyens2014legal, vacek2019litigation, yan2017event}. But current symbol-based methods are not as effective as embedding-based methods and incorporating extracted information into prediction still needs more efforts \cite{zhong-etal-2020-nlp}.  Our neuro-symbolic method bridges this gap, by both using distributed representation of neural networks and programs.

\begin{table*}[t!]
\centering
\small
\linespread{1}

\NewDocumentCommand{\blueback}{ mO{} }{\colorbox{blue!15}{#1}}
\NewDocumentCommand{\greenback}{ mO{} }{\colorbox{green!30}{#1}}
\NewDocumentCommand{\bluecolor}{ mO{} }{\textbf{\textcolor{blue!60}{#1}}}

% (a)

\setlength{\tabcolsep}{1mm}{
\begin{tabular}{p{20mm}|p{53mm}|p{80mm}}

% \setlength{\arrayrulewidth}{.3em}

\toprule
\textbf{Class} & \textbf{Natural Language Explanation} & \textbf{Parsed Logical Structure} \\
\midrule

(car-evaluation) vgood
&
\textit{Cars \bluecolor{with higher safety} \greenback{and capacity} are highly acceptable for resale.}
&
($c_{certainty}$=0.74) $~\text{vgood}(X)=\text{\bluecolor{with higher safety}}(X)\land \text{\greenback{and capacity}}(X)$
% $ ~\text{function}_1$ = \textit{Exist}(\texttt{\bluecolor{with higher safety}}; X);
% $\text{function}_2$ = \textit{Exist}(\texttt{\greenback{with higher safety and capacity}}; X);
% $\text{function}_3$ = And($\text{function}_1$, $\text{function}_2$; X);
% return $\text{function}_3$(X)
\\

\bottomrule

\end{tabular}
}

\vspace{0.3cm}

% (b)
\begin{tabular}{p{85mm}|p{70mm}}
\toprule
\textbf{Input} & \textbf{Execution Evidence}\\
\midrule
% \texttt{SGPT | 33 [SEP] SGOT | 71 [SEP] total bilirubin | 4\bluecolor{.9 [SEP] age | 65 [SEP] direct bilirubin | 2}.7 [SEP]}
\setlength{\tabcolsep}{0.5mm}{
\ \quad \begin{tabular}[t]{ccccc}
\toprule
% \greenback{\bluecolor{safety}} & \greenback{\bluecolor{person capacity}} & \greenback{\bluecolor{buying cost}} & maintenance cost & luggage boot size \\
% \midrule
% \greenback{\bluecolor{high}} & \greenback{\bluecolor{4}} & \greenback{\bluecolor{med}} & low & med
\greenback{\bluecolor{safety}} & \greenback{\bluecolor{person capacity}} & \greenback{\bluecolor{buying cost}} & maintenance cost & $\cdots$\\
\midrule
\greenback{\bluecolor{high}} & \greenback{\bluecolor{4}} & \greenback{\bluecolor{med}} & low & $\cdots$
\\
\bottomrule
\\
\end{tabular}
}
&
% function$_1$(X) = \textit{Exist}(\texttt{group above 40}; X);
% function$_2$(X) = \textit{Exist}(\texttt{ensures liver}; X);
% function$_3$(X) = And(function$_1$, function$_2$; X);
% return function$_3$(X)
% &
% function$_1$(X)=0.58 attends to: \texttt{\bluecolor{safety|high ... buying cost|med}};
% function$_2$(X)=0.65 attends to: \texttt{\greenback{safety|high ... buying cost|med}};
% function$_3$(X)=0.58; return 0.58 (\textbf{positive}) \\
Concept ``with higher safety'' matches 
\textit{\bluecolor{safety|high ... buying cost|med}} with score 0.58 ;
Concept ``and capacity'' matches 
\textit{\greenback{safety|high ... buying cost|med}} with score 0.65 ;
\texttt{AND} operator score 0.58; return 0.58\\
\bottomrule
\end{tabular}

\vspace{3mm}

\begin{tabular}[t]{c|c|c}
\toprule
Ground-truth label: \textbf{positive} & \model{} prediction: \textbf{positive} & ExEnt prediction: \textbf{negative} \\
\bottomrule
\end{tabular}

\caption{An visualization example of parsed logical structure on an explanation and \model{}'s reasoning evidence on an input.
}
\vspace{-5mm}
\label{tab:interpretability}
\end{table*}


\section{Logical Parsing and Reasoning}
% \hanchi{needs rewriting to make it more elegant and easier to understand. The section\ref{sec:intro} gives a brief and general description. It will be something like this: we propose to first parse the logical structure of the explanation, which is an AND/OR tree over more elementary textual concepts like ``large bird'' and ``black feathers''. Then we match these elementary concepts with the input. Finally we walk along the logical structure, and gather the concept matching scores accordingly ($\max$ if AND and $\min$ if OR) to get the final classification score.}
\begin{figure*}[ht]
    \centering
    \includegraphics[width=0.8\textwidth]{graphics/figure1.pdf}
    \caption{Overview of Perona. Designed as extension to common approaches for resource configuration optimization, target infrastructures are thoroughly benchmarked, relevant information are extracted, and anomalous benchmark executions due to resource degradation or failures are detected and reported.}
    \label{fig:overview}
\end{figure*}

\section{Approach}
\label{sec:approach}
This section presents the main ideas of our approach \emph{Perona} and how it can be used to explicitly fingerprint target infrastructures and extend existing resource configuration optimization solutions.
An overview is depicted in \autoref{fig:overview}, and the main variables used are summarized in~\autoref{tab:variabledefinitions}.

\begin{table}[hb]
\centering
\caption{Overview of main Variables}
    \begin{tabular}[t]{rl}
        \toprule
        \multicolumn{2}{c}{\emph{Problem Formalization}}\\
        \toprule
        $c_j$ & resource configuration $j$; $c_j \in C$\\
        $w_i$ & workload $i$; $w_i \in W$\\
        $y_{ij}$ & vector of performance measures for $w_i$ run with $c_j$\\
        \midrule
        \multicolumn{2}{c}{\emph{Approach}}\\
        \midrule
        $p_r$ & configuration template for resource aspect $r$; $p_r \in P$\\
        $m_k$ & machine type $k$; $m_k \in M$\\
        $b_r(t)$ & benchmark execution at time $t$ with $p_r$; $b_r(t) \in B(t)$\\
        $\vec{x}(t)$ & vector of all performance metrics in $B(t)$; $\vec{x}(t)\in \mathbb{R}^F$\\
        $\vec{x'}(t)$ & compact version of $\vec{x}(t)$ after preprocessing; $\vec{x'}(t)\in \mathbb{R}^{F'}$\\
        $\vec{c}(t)$ & dense learned encoding of $\vec{x'}(t)$; $\vec{c}(t)\in \mathbb{R}^{K}$\\
        $\vec{\underline{c}}(t)$ & neighborhood-inferred encoding; $\vec{\underline{c}}(t)\in \mathbb{R}^{K}$\\
        \bottomrule
    \end{tabular}
\label{tab:variabledefinitions}
\end{table}

\subsection{Overview}

Existing solutions to resource configuration optimization of big data analytics workloads commonly explore the resource configuration search space only indirectly, i.e., by means of running a target workload with different resource configurations and observing its performance.
While eventually an expedient approach, it does not provide an explicit and general understanding of resource configurations, which could be externalized and transferred for use in other contexts.
Hence, there lies potential in tackling this limitation.

We design a novel approach called Perona for explicit infrastructure fingerprinting, i.e., generalized benchmarking of resource configurations.
As illustrated in~\autoref{fig:overview}, it is at its core composed of three steps and associated components, which will be thoroughly described in the subsequent sections:
\begin{enumerate}
    \item A comprehensive inspection of target infrastructures through standardized sets and configurations of benchmarking tools, followed by an automated preprocessing of recorded benchmark metrics (\autoref{sec:approach_step1}).
    \item An internally employed graph-based modeling approach for representation learning of benchmark executions and context-aware outlier detection (\autoref{sec:approach_step2}).
    \item An opportunistic strategy for comparing and ranking learned representations of diverse resource configurations, which can be employed to automatically improve resource efficiency (\autoref{sec:approach_step3}).
\end{enumerate}
In the following, these ordered steps are explained in detail.

\begin{figure}[b]
    \centering
    \includegraphics[width=\columnwidth, keepaspectratio]{graphics/perona_figures-approach_step1.pdf}
    \caption{Standardized sets and configurations of benchmarking tools are used to holistically assess the characteristics of target machines and to ensure comparability across machines and benchmark executions.}
    \label{fig:approach_step1}
\end{figure}

\subsection{Standardized Resource Benchmarking}
\label{sec:approach_step1}

In order to sufficiently fingerprint a target infrastructure and carve out its individual characteristics, it is imperative to not only benchmark all relevant resources using dedicated tools, but also employ reasonable and fixed configurations for these benchmarking tools, so that individual runs are comparable across machine types and even infrastructures.
This idea is further depicted in~\autoref{fig:approach_step1}.
Let $r=1, \ldots, n$ index the up to $n$ different aspects of a resource configuration (e.g. memory, CPU, network) and $P=\{p_r\}_{r=1}^n$ be the associated set of configuration templates for benchmarking tools.
We then write $b^{(m_k)}_r(t)$ to denote the execution of a benchmarking tool at time $t$ with configuration template $p_r$ for resource aspect $r$ on machine $m_k \in M$, where the latter is part of the respective target infrastructure (e.g. the set of machines to profile, or a ready-to-use cluster of different machines).
The goal is then to initially fingerprint each so far unseen machine to obtain the complete set $B^{(m_k)}(t) = \{b^{(m_k)}_r(t)\}_{r=1}^n$, and latter on reschedule certain benchmarks if required. 
In this case, the marker $t$ is only approximate, since not all benchmark executions are necessarily conducted strictly in parallel.
With this, each machine is benchmarked the same way.
In the following and for the sake of simplicity, we will however refrain from using subscripts and superscripts if not strictly necessary.
This also makes sense given that our method is for the most part concerned with node-wise benchmark executions and benchmark-wise representation learning.

Each executed element of a set $B^(t)$ returns a variety of numerical performance metrics. 
We refer to the feature vector of all performance metrics associated with $B^(t)$ as $\vec{x}^(t)\in \mathbb{R}^F$, where $F$ denotes the total number of performance metrics and hence the feature dimension.
As of now, it is not yet clear which features are exactly relevant to sufficiently describe the benchmarked resource. 
Since a manual investigation can quickly become unmanageable, we are interested in automating this process.
Therefore, we apply a series of preprocessing steps to ease downstream modeling:
\begin{enumerate}
    \item \emph{Unification}: It can not be guaranteed that performance metrics are always issued in the same units. 
    Consequently, we ensure that all recordings of each individual performance metrics are among themselves comparable through unification of their associated units.
    \item \emph{Selection}: Some performance metrics might have less of a predictive value than others, which is why we only retain metrics with a standard deviation greater equal a configurable threshold.
    In addition, we require each metric to have at least two distinct historical values, otherwise its predictive character is questionable.
    \item \emph{Orientation}: Naturally, certain metrics (e.g. latency) are meant to be minimized whereas others are not (e.g. throughput).
    For our downstream modeling, we strive to equalize the various orientations as best as possible. 
    A performance metric shall be maximized if its maximum value is closer to its median than its minimum value, otherwise minimization is desired.
    Occasionally injecting synthetic stress into running benchmarks further helps in identifying the orientation of a metric.
\end{enumerate}
These steps help to reduce the dimensionality of feature vectors and to skip irrelevant individual features. 
Lastly, we enrich each feature vector by a one-hot encoding of the respective \emph{benchmark type} (tool + configuration). 
As a result, we obtain a compact feature vector $\vec{x'}(t)\in \mathbb{R}^{F'}$ with $F' \ll F$.

\textbf{Training Notes.} Since the aforementioned steps are designed to be stateful, the metrics associated with any $b_r(t) \in B(t)$ will be processed the same way and a feature vector of fixed size will be created.
In case of non-present features, e.g., an executed benchmark $b_r(t)$ evidently lacks the metrics generated by $b_s(t)$ with $r\neq s$, the missing values are filled with the so far observed average value of the metric of interest, which is a common machine learning practice. 

\subsection{End-to-End Contextual Representation Learning}
\label{sec:approach_step2}

Correctly interpreting the results of a benchmark execution requires the consideration of relevant performance metrics and their contextualization with respect to prior benchmark executions. 
Hence, we propose a graph-based model that relies on dense representations of feature vectors learned by an encoder model and a decoder model.
It is illustrated in~\autoref{fig:approach_step2}.

Let $\vec{x'} \in \mathbb{R}^{F'}$ be a feature vector produced by our procedure detailed in~\autoref{sec:approach_step1} at an arbitrary point in time, a decoder network function $dec:\mathbb{R}^K\rightarrow \mathbb{R}^{F'}$ will try to reconstruct $\vec{x'}$ from the code $\vec{c}\in\mathbb{R}^K$ calculated by the encoder network function $enc:\mathbb{R}^{F'} \rightarrow \mathbb{R}^K$, such that $\min\Arrowvert \vec{x'} - dec(\vec{c}) \Arrowvert_p$ and $K \ll F'$ (in this context, $p$ denotes a desired $p$-norm).
This interaction enables the learning of meaningful and dense representations which can be used in downstream prediction tasks.
More precisely, the autoencoder learns the relevance of features, hence implementing an additional feature selection and realizing a dimensionality reduction.

\begin{figure}[b]
    \centering
    \includegraphics[width=\columnwidth, keepaspectratio]{graphics/perona_figures-approach_step2.pdf}
    \caption{An autoencoder is employed for dimensionality reduction and feature extraction, such that key information are preserved. 
    The relationships of subsequent benchmark executions are exploited to detect anomalous behavior.}
    \label{fig:approach_step2}
\end{figure}

At this point, we manage to grasp the relevant information from the original performance metric vector, yet we still lack a notion of normal and anomalous execution behavior since we focus on a single vector only. 
In a next step, we attempt to detect irregularities through consideration of previous benchmark executions.
Let $G=(V,E)$ be a directed and attributed graph that consists of a set of vertices $V=\{v_1, \ldots, v_n\}$ and a set of edges $E\subseteq \{(v_i,v_j)| v_i,v_j \in V\}$. 
An edge $e_{ij} \Leftrightarrow (v_i,v_j)\in E$ describes a directed connection between vertex $v_i$ and $v_j$. 
Thus, the node $v_j$ is then called a neighbor of node $v_i$, formally written as $j\in \altmathcal{N}(i)$.
Each node $v_i$ has an associated node feature vector $\vec{v_i}$, and each edge $e_{ij}$ can have an edge attribute vector $\vec{e_{ij}}$ as well. 
In the context of this work, $G$ is formed by establishing forward edge connections between chronologically sorted executions of the same benchmark type acting as nodes, with associated node feature vectors outputted by the encoder model $enc$. 
Furthermore, we use low-level metrics from the underlying compute instance obtained during a benchmark execution as well as various encodings of time intervals between each pair of benchmark executions to establish edge attributes. 
Note that graphs are composed per benchmark type and compute instance, in order to connect the relevant executions with each other.
The graph model $agg$ is then trained to predict the feature vector of a particular node via an aggregation of its respective neighborhood, i.e., through consideration of neighboring feature vectors and existing attributed edges:
\begin{equation*}
    \Vec{v_i}^{(k)} = \gamma^{(k)} \Big( \Vec{v_i}^{(k-1)}, \lambda_{j\in \altmathcal{N}(i)} \phi^{(k)} \Big( \Vec{v_i}^{(k-1)}, \Vec{v_j}^{(k-1)}, \vec{e_{ji}} \Big) \Big),
\end{equation*}
where $\lambda$ denotes a differentiable and permutation invariant function, $k$ denotes the number of hops, and both $\gamma$ and $\phi$ denote differentiable functions, e.g. feed-forward neural networks, which optionally alter the vector dimensionality.
When several such steps are performed, structural information is effectively used and passed through the graph.  
We temporarily denote the final aggregated version of a node feature vector $\vec{v_i}$ as $\vec{\underline{v_i}}$ -- the correctness of a benchmark execution can then be determined using a vector comparison.
We calculate the probability of a benchmark execution being anomalous as
\begin{equation*}
    \mathds{P}(\vec{v_i}) = \sigma (f_1(\vec{v_i} - \vec{\underline{v_i}})),
\end{equation*}
where $\sigma$ is the logistic function and $f_1$ is a non-linear transformation function with learnable parameters.

\textbf{Training Notes.} Due to the potentially significant imbalance of normal and anomalous benchmark executions, in our method implementation, we employ for the specific task of outlier detection a class-balanced focal loss~\cite{CuiJLSB19} (CBFL).
For the autoencoder, we minimize the reconstruction error, which is measured as the mean squared error (MSE) between input vectors and their reconstructed counterparts. 

\subsection{Aspect-Based Resource Ranking}
\label{sec:approach_step3}

\begin{figure}[b]
    \centering
    \includegraphics[width=\columnwidth, keepaspectratio]{graphics/perona_figures-approach_step3.pdf}
    \caption{Our trainable function approximators are instructed to maximize the distance between benchmark clusters of learned encodings.
    Within each cluster, the encodings are positioned and ranked in terms of their vector norm.}
    \label{fig:approach_step3}
\end{figure}

In the previous step, we designed a way for learning latent benchmark execution encodings and detecting potential outliers. 
Yet, we so far have limited control over the general quality of learned encodings, which makes their direct usage questionable. 
In order to address this, we conceive several side tasks so that our model is instructed to learn more meaningful representations. 
More precisely, we demand the following:
\begin{itemize}
    \item \emph{Clustering}: Learned representations of one benchmark type shall be in close proximity in terms of cosine similarity, whereas the cosine distance of representations from unequal benchmark types shall be substantially increased.
    This leads to a class-based clustering of representations and is exemplarily depicted in~\autoref{fig:approach_step3}.
    \item \emph{Classification}: Closely related is the requirement that learned representations shall be sufficiently informative and hence usable for predicting the associated benchmark type after a simple linear transformation already.
    \item \emph{Ranking}: In order to embed our method into other approaches, it is imperative that learned representations are among each other comparable and rankable.
    We enforce this by deducing a pairwise ranking groundtruth based on a desired $p$-norm of preprocessed performance vectors, and demanding our learned representations to obey to the same norm-based ranking. 
    This is depicted within the clusters in~\autoref{fig:approach_step3} and consequently introduces a weak notion of order among representations.
\end{itemize}
The described additional learning tasks not only allow for more controlled representation learning, but also act as extra regularization component, since our model needs to optimize for multiple tasks simultaneously.
For a practical application of learned representations, e.g., in the context of scheduling problems where a suitable resource needs to be found for a target processing workload, one straightforward approach would then be to calculate a vector norm to obtain a score reflecting the quality of the resource in question.
This can be done for each resource aspect, allowing for a fine-granular assessment of resources and their capabilities.

\textbf{Training Notes.} For the clustering task, we employ a triplet margin loss~\cite{SchroffKP15} (TML) and combine it with a miner to evaluate especially challenging triplets.
The classification task is monitored using a cross entropy loss (CEL) operating on the output of the aforementioned linear transformation for obtaining class probabilities.
Finally, for the ranking of learned representations, we utilize a margin ranking loss (MRL) which evaluates the pairwise ranking of representations based on a previously deduced ranking groundtruth.
For pairs of normal representations, a ranking with minimal margins is sufficient, however, for unlike representations, we enforce a margin such that the anomalous representation is at least ranked smaller than the lowest normal representation observed so far.

\begin{table}[t!]
\centering
\small
\linespread{1}

%\setlength{\tabcolsep}{1mm}{
\resizebox{\linewidth}{!}{
% \begin{tabular}{l|cccc}

% \toprule

% Top-1 acc/\% & \textbf{\model{}} & \textbf{ExEnt} & \textbf{ExEnt-BERT} & \textbf{ExEnt-GPT2} \\
% \midrule
% CLUES-Real & \textbf{57.4} & 54.8 & 46.4 & 43.8 \\
% \midrule
% \hspace{3mm}+pre-training & \textbf{55.2} & 52.7 & 50.5 & 52.4 \\

% Top-1 acc/\% & \textbf{\model{}} & \textbf{\model{}-plain}& \textbf{ExEnt} & \textbf{RoBERTa-sim} \\
% \midrule
% CLUES-Real & \textbf{57.4} & 45.8 & 54.8 & 45.1 \\
% \midrule
% \hspace{3mm}+pre-training & \textbf{55.2} & 49.8 & 52.7 & 46.3 \\


\begin{tabular}{c|ccc}
Top-1 acc/\% 
& CLUES-Real & + pre-training \\
\midrule

% \textbf{RoBERTa-sim} & 54.8 & 52.7 \\
% \textbf{ExEnt} & 45.1 & 46.3 \\

\textbf{ExEnt}  & 54.8 & 52.7 \\
\textbf{RoBERTa-sim} & 45.1 & 46.3 \\

\midrule 

\textbf{\model{}-plain} & 45.8 & 49.8 \\
\textbf{\model{}} & \textbf{57.4} & \textbf{55.2} \\

\bottomrule

\end{tabular}
}

\caption{Cross-task generalization results on CLUES dataset~\citep{menon2022clues}. The first row of results are acquired by only fine-tuning on CLUES-Real, and the second row shows results with additional pre-training on CLUES-Synthetic.
% \chihan{better baselines than GPT2 and BERT to be added, like similarity, ignoring logical structure, entailment?}
}
% \vspace{-4mm}
\label{tab:clues_main_results}
\end{table}
\section{Classification by Logical Reasoning on Natural Language Explanations}
\label{sec:clues}
In this section we conduct in-depth analysis of our proposed approach towards zero-shot classification with explanations. We start with a latest benchmark, CLUES~\citep{menon2022clues}, which evaluates the performance of classifier learning with natural language explanations. CLUES benchmark focus on the modality of structured data, where input data is a table of features describing an item. This data format is flexible enough for computers on a wide range of applications, and also benefits quantitative analysis in the rest part of this section. In the next section (\ref{sec:other_modalities}), we also extend to more datasets and modalities.
\begin{figure}
\centering
\includegraphics[width=\columnwidth]{figures/effect_of_complexity_fig.pdf}

\caption{The classification accuracy on zero-shot tasks in CLUES plotted against the proportion of compositional explanations. (There are multiple tasks with only simple explanations, so there are multiple points at $x=0$ position.)
% Our model's performance generally remains stable across different types of tasks, but ExEnt performs worse when the task contains more compositional definitions.
}

\vspace{-4mm}
\label{fig:effect_of_complexity}
\end{figure}

\subsection{CLUES benchmark}
CLUES is designed as a cross-task generalization benchmark on structured data classification. It consists of 36 real-world and 144 synthetic multi-class classification tasks, respectively. \heng{what is the data modality for this data set initially?}\hanchi{This dataset is on structured data upon construction. Do you see some problem in adopting this setting?}
The model is given a set of tasks for learning, and then evaluated on a set of unseen tasks.
The inputs in each task constitute a structured table.
Each column represents an attribute type, and each row is one input datum. In each task, for each class, CLUES provides a set of natural language explanations.
% For example, in the ``Mushroom'' task, the columns  are ``odor'', ``stalk-surface-above-ring'', ``gill-color'' and ``ring-type''. The explanations are sentences including \textit{Foul smelling are Poisonous. 7 of 7 rows,with no deviation.}
% We follow the data pre-processing in~\cite{menon2022clues} and convert each input into a text sequence. The text sequence is in the form of ``\texttt{odor | pungent [SEP] ... [SEP] ring-type | pendant}'', where ``\texttt{odor}'' is the attribute type name, and ``\texttt{pungent}'' is the attribute value for this input, and so on.
% Then we encode the sentence with BERT~\citep{kenton2019bert} as inputs $X$.

We follow the data processing in~\citet{menon2022clues} and convert each input into a text sequence. The text sequence is in the form of ``\texttt{odor | pungent [SEP] ... [SEP] ring-type | pendant}'', where ``\texttt{odor}'' is the attribute type name, and ``\texttt{pungent}'' is the attribute value for this input, so on and so forth. Then we encode the sentence with RoBERTa~\citep{kenton2019bert} and use the word embeddings as input features $X$. For baselines, we use ExEnt and its variants which is an text entailment model introduced in the CLUES paper. ExEnt uses pre-trained RoBERTa as backbone. It works by encoding concatenated explanations and inputs, and then computing an entailment score. More implementation details can be found in Appendix~\ref{appsec:configuration}.
\begin{figure}
\centering
\includegraphics[width=\columnwidth]{figures/argument_position-fig.pdf}

\caption{The position of detected concepts relative to the expert-annotated keyword spans. Y-axis is the proportion of explanations. Each interval category on x-axis denotes a relative position to the keyword span in the explanation.
% For example, the first category {\small $[-\infty, left-10]$} stands for the cases where the word with top attention is left to the start-position with a distance greater than 10.
% \xd{\sout{why not merge $[-\infty, left-10]$ and $[left-10,left-5]$ considering you don't discuss $[-\infty, left-10]$ separately}}
}

\vspace{-5mm}
\label{fig:argument_position}
\end{figure}
\subsection{Zero-Shot Classification Results}
\heng{\sout{really need to put in some qualitative analysis with examples. If you have them in appendix move some of them here since not everyone reads appendix}}
\heng{\sout{I still don't see which results show that your method achieves better interpretability}}

% \heng{\sout{Table 2 is out of margin, fix it}}
% \heng{\sout{What is ExEnt?}}
% \hanchi{\sout{Oh yeah. I just added this part above.}}

Zero-shot classification results are listed in Table~\ref{tab:clues_main_results}. \Model{} outperforms the baseline methods on main evaluation metrics.
To understand the effect of backbound model, we need to note that ExEnt also uses RoBERTa as the backbone model, so the \model{} and baselines do not exhibit a significant difference in basic representation abilities. The inferior performance of RoBERTa-sim compared to ExEnt highlights the complexity of the task, indicating that it demands more advanced reasoning skills than mere sentence similarity.
Furthermore, as an ablation study, \model{} outperforms \model{}-plain, which serves as initial evidence on the importance of logical structure in reasoning.
% Appendix~\ref{appsec:ablation} provides a more detailed ablation study on the effect of logical reasoning complexity.
% Note that due the the gap between CLUES-Real and CLUES-Synthetic, after pre-trained on CLUES-Synthetic some models observe a performance drop. This accords with the observation in the CLUES original paper~\citep{menon2022clues}.
% \hengzhi{\sout{consider to use subsection instead of paragraph}}

% \begin{figure}
\centering
\includegraphics[width=1.0\columnwidth]{figures/quantifiers_fig.pdf}

\caption{Comparison between the learned certainty coefficients $c_{certainty}$ in \model{} and expert annotations in ~\citet{srivastava2018zero}..
% Upper right show the linear regression equation and coefficient of determination ($R^2$).
% \xd{\sout{figure words too small, i cropped the white space around. in the fig, denote what y and $R^2$ is.}}
}

\vspace{-4mm}
\label{fig:quantifiers}
\end{figure}
\begin{figure}
\centering
\includegraphics[width=\columnwidth]{figures/robustness-fig.pdf}

\caption{The effect of linguistic biases on classifiers. \textit{Punctuated}, \textit{Hinted} and \textit{Verbose} are three types of biasing strategies. 
% The two horizontal lines denote the original performance of \model{} and ExEnt without linguistic biases. Error bars denote standard deviation.
}

\vspace{-5mm}
\label{fig:robustness}
\end{figure}

\subsection{Interpretability}

% One advantage logical reasoning is that it provides better interpretable intermediate results~\citep{mao2019neuro, yi2018neural}. 
\Model{} is interpretable in two senses: 1) it parses logical structures to explain how the explanations are interpreted, and 2) the logical reasoning   evidence serves as decision making rationales.
To demonstrate the interpretability of \model{}, in Table \ref{tab:interpretability} we present an example of the parsed logical structure and reasoning process. We compare \model{}'s behavior with ExEnt baseline. More interpretability examples and error analysis are listed in Appendix~\ref{appsec:interpretability}.

The upper part of Figure \ref{tab:interpretability} shows that \model{} selects ``\textit{with higher safety}'' and ``\textit{and capacity}'' as concepts candidates, and uses an \texttt{AND} operator over the concepts.
The middle part of the figure visualizes the logical reasoning evidence.
In the example, both two concepts matches with the text span \texttt{safety|high[SEP]person capacity|4[SEP]buying cost|med}, and yield scores of 0.65 and 0.58, respectively. Then the \textit{And} operation outputs classification score of 0.58 ($>$ 0.5, indicating a positive prediction). This example is correctly classified by our model, but mis-classified by the ExEnt baseline. This examplifies the efficacy of \model{} on compositional reasoning on explanations.
To quantitatively evaluate the learned concepts, we manually annotate keyword spans for 100 out of 344 explanations. These spans describe the key attributes for making the explanation.
% Each keyword span is represented as a pair (start position, end position), which are start and end token positions numbers in the explanation sentences\footnote{by using the \texttt{BERT-base-uncased} tokenizer}.
Then we plot the histogram of the relative position between top-attention tokens and annotated keyword spans in Figure~\ref{fig:argument_position}. When there are multiple concepts detected, we select the one closest to the keyword span. From the figure we can see that the majority of top-attention tokens (52\%) fall within the range of annotated keyword spans. 
The ratio increases to 81\% within distance of 5 tokens from the keyword span, and 95\% within distance of 10 tokens.

% 
\subsection{Linguistic Quantifier Understanding} Linguistic quantifiers is a topic to understand the degree of certainty in natural language~\citep{srivastava2018zero, yildirim2013linguistic}. For example, humans are more certain when saying something \textit{usually} happens, but less certain when using words like \textit{sometimes}.
We observe that the certainty coefficient $c_{certainty}$ that \model{} learns can naturally serve the purpose the of modelling quantifiers.
\hengzhi{\sout{do we have a explanation for certainty coefficient $c_{certainty}$ before?}}\hanchi{\sout{On I forgot. I added this one in Section~\ref{sec:approach}.}}
We first detect the existence of linguistic quantifiers like \textit{often} and \textit{usually} by simply word matching. % listed in Table~\ref{tab:quantifier_extraction}.
Then we take the average of $c_{certainty}$ on the matched explanations. We plot these values against expert-annotated ``quantifier probabilities'' in~\citep{srivastava2018zero}
% , which indicates how ``probable'' the sentence is supposed to be true with a certain quantifier.
in Figure~\ref{fig:quantifiers}. Results show that $c_{certainty}$ correlates positively with ``quantifier probabilities'' with Pearson correlation coefficient value of 0.271. In cases where they disagree, our quantifier coefficients also make some sense, such as assigning \textit{often} a relatively higher value but giving \textit{likely} a lower value.
% \heng{\sout{this paragraph is very hard to follow, and unclear what the point is}}
\hengzhi{and do we need a comparison for this part using the baseline?}
\hanchi{I can only think of a most naive way to do this. Let me try it in the next days}

\subsection{Robustness to linguistic bias}

Linguistic biases are prevalent in natural language, which can subtly change the emotions and stances of the text~\citep{field2018framing, ziems2021protect}. Pre-trained language models have also been found to be affected by subtle linguistic perturbations~\citep{kojima2022large} and hints~\citep{patel2021stated}.
% Given that most modern natural language processing models are built on top of these pre-trained language models, they are likely to be susceptible to linguistic biases. 

In this section we investigate how different models are affected by these linguistic biases in inputs. To this end, we experiment on 3 categories of linguistic biases. \textit{Punctuated}: inspired by discussions about linguistic hints in~\cite{patel2021stated}, we append punctuation such as ``?'' and ``...'' to the input in order to change its underlying tone. \textit{Hinted}: we change the joining character from ``\texttt{|}'' to phrases with doubting hints such as ``is claimed to be''. \textit{Verbose}: Transformer-based models are found to attend on a local window of words~\citep{child2019generating}, so we append a long verbose sentence ($\approx$ 30 words) to the input sentence to perturb the attention mechanism.

Results are presented in Figure~\ref{fig:robustness}. Compared with the original scores without linguistic biases (the horizontal lines), \model{}'s performance is not significantly affected. But ExEnt appears to be susceptible to these biases with a large drop in performance. This result demonstrates that ExEnt also inherits the sensitivity to these linguistic biases from its PLM backbone. By contrast, \model{} is encouraged to explicitly parse explanations into its logical structure and conduct compositional logical reasoning. This provides better inductive bias for classification, and regulates the model from leveraging subtle linguistic patterns .

% \section{Results and Analysis}
% \label{sec:results}
% \input{EACL_body/5_0_classification_results}
% \input{EACL_body/5_1_interpretability}
% \input{EACL_body/5_2_robustness}

\begin{table}[t!]
\resizebox{\columnwidth}{!}{
\centering
% \small
\linespread{1}

\setlength{\tabcolsep}{1mm}{
\begin{tabular}{c|c|ccc}

\toprule
\textbf{CUB-Explanations} & \textbf{Model} & $ACC_U$ & $ACC_S$ & $ACC_\textbf{H}$ \\
\midrule

\multirow{2}{30mm}{w/o VLPMs}
& TF-VAEGAN$_{\mathit{expl}}$ & 4.7 & 39.1 & 8.3 \\
% & Similarity & 4.7 & \textbf{61.2} & 8.8 \\
& \Model{} (ours) & \textbf{6.6} & \textbf{51.1} & \textbf{11.7} \\

\midrule

\multirow{3}{30mm}{w/ VLPMs}
% & CLIP$_{original}$ & 41.5 & 45.9 & 43.6 \\
& CLIP$_{linear}$ & 34.3 & 41.2 & 37.4 \\
& CLIP$_{finetuned}$ & 29.9 & \textbf{66.9} & 41.3 \\
& \Model{}$_{CLIP}$ (ours) & \textbf{39.1} & 65.8 & \textbf{49.1} \\

% \textbf{Model} & \textbf{\Model{}} & \textbf{Similarity} & ZSL\textunderscore TF-VAEGAN & \textbf{\Model{}} & CLIP \\

\bottomrule

\end{tabular}
}
}

\caption{Generalized zero-shot classification results (in \%) on the new CUB-Explanations dataset.
% $ACC_U$ denotes accuracy on unseen categories, $ACC_S$ denotes accuracy on seen categories, and $ACC_\textbf{H}$ is the harmonic average of the them. 
% The lower and upper parts show models with and without parameters from the Vision Language Pretrained Models (VLPMs).
% TF-VAEGAN$_{def}$ is our re-implementation to adapt to
% \heng{\sout{what do you mean by 'out adaptation'?}}
% CUB-Definitions.
% \Model{}$_{CLIP}$ uses visual and text encoders from CLIP as model backbone.
}
\vspace{-5mm}
\label{tab:cub_main_results}
\end{table}
\section{Extending to Textual and Visual Inputs}
\begin{table*}[t!]
\centering
\small
\linespread{1}

% \setlength{\tabcolsep}{1mm}{
% \resizebox{\linewidth}{!}{
\begin{tabular}{c|ccc|ccc}

\toprule

\multirow{2}{*}{\textbf{ECtHR-Explanations}}
& \multicolumn{3}{c}{\textbf{micro-F1}} & \multicolumn{3}{c}{\textbf{macro-F1}} \\
& A-Mean & H-Mean & G-Mean & A-Mean & H-Mean & G-Mean \\
\midrule

Legal-BERT & 0.577 & 0.464 & 0.520 & 0.416 & 0.375 & 0.396 \\
Hierarchical Legal-BERT & 0.779 & 0.756 & 0.768 & 0.449 & 0.447 & 0.448 \\
\model{}$_{Legal-BERT}$ & \textbf{0.823} & \textbf{0.800} & \textbf{0.812} & \textbf{0.460} & \textbf{0.457} & \textbf{0.458} \\
% Hierarchical Legal-BERT & 0.474 & 0.470 & 0.472 \\
% \model{} & \textbf{0.490} & \textbf{0.489} & \textbf{0.490} \\

\bottomrule

\end{tabular}
% }

\caption{Zero-shot classification results on ECtHR-Explanations dataset~\citep{chalkidis2021paragraph}. The A-Mean, H-Mean and G-Mean are the arithmetic, harmonic and geometric averages of ROC-AUC scores}
\vspace{-6mm}
\label{tab:ecthr_main_results}
\end{table*}
\label{sec:other_modalities}


% Natural language explanations are prevalent in various applications: we use language explanations to define abstract terminologies and describe real-world objects. Taking this observation, in this section we evaluate whether \model{} can be extended to other modalities.

Natural language explanations are prevalent in other applications as well. Taking this observation, in this section we evaluate whether \model{} can be extended to visual domain.
\subsection{Datasets}

\paragraph{CUB-Explanations} We build a CUB-Explanations dataset based on image classification dataset CUB-200-2011 dataset~\citep{wah2011caltech}. 
CUB-200-2011 originally includes $\sim$ 12k images involving 200 categories of birds, 150 categories are used for training and other 50 categories are left for zero-shot image classification.
In this work, we focus on the setting of zero-shot classification using natural language explanations.
Natural language explanations are more efficient to collect than the crowd-sourced attribute annotations. They are also similar to human learning process, and would be more challenging for models to utilize.
To this end, we collect natural language explanations of each bird category from Wikipedia. These explanations come from the short description part\footnote{As defined in \url{ https://en.wikipedia.org/wiki/Wikipedia:Short_description}} and the \textit{Description}, \textit{Morphology} or \textit{Identification} sections in the Wikipedia pages.
We mainly focus on the sentences that describe visual attributes that can be recognized in images (e.g. body parts, visual patterns and colors).
Finally we get 1$\sim$8 explanation sentences for each category with a total of 991 explanations.
%CUB-200-2011 originally includes $\sim$ 12k images involving 200 categories of birds, and evaluates models' performance on image classification. In zero-shot classification setting, data of 150 classes are provided for training and other 50 classes are left for evaluation.
% \heng{\sout{so you actually enriched this dataset? add it into contributions}}

% To get input features $X$ for \model{}, we use a pretrained visual encoder (we experiment with both ResNet-101~\citep{he2016deep} and CLIP~\citep{radford2021learning}) to obtain image patch representation vectors. These vectors are then flattened as a sequence and used as visual input $X$.
For evaluation, we adopt the three metrics commonly used for generalized zero-shot learning: $ACC_U$ denotes accuracy on unseen categories, $ACC_S$ denotes accuracy on seen categories, and their harmonic average $ACC_\textbf{H}=\frac{2ACC_UACC_S}{ACC_U + ACC_S}$.

\paragraph{ECtHR-Explanations} In legal domain the demand for transparent and robust classification is higher ~\citep{branting2021scalable}. This is also a representative domain where humans learn concepts by reading language definitions.
So we build a variant dataset ECtHR-Explanations based on ECtHR dataset ~\citep{chalkidis2021paragraph}, which is a recent dataset containing allegations of states violating the European Convention of Human Rights (ECHR). 
The dataset contains 11k cases of allegations in total. Each case consists of multiple natural language paragraphs describing the facts in the case, and is mapped to a list of allegedly violated ECHR articles. We use the main text under each ECHR article \heng{\sout{unclear what is language contents, reword it}} \footnote{\url{https://www.echr.coe.int/documents/convention_eng.pdf}} as the explanation sentences, thus the name ECtHR-Explanations.

For evaluation, we adopt a ``leave-one-out'' evaluation method: at each time, one category is left out  for zero-shot evaluation while the rest categories are used for training. Finally we use the averages of F1 scores as evaluation metric.

% Each input is a long document containing multiple paragraphs, so we follow~\cite{chalkidis-etal-2022-lexglue} to use paragraph embeddings as the input features $X$.
%because ROC-AUC score is more stable and does not result in lots of zero scores as F1 metric do.


\subsection{Experiment Setting and Baselines}
% In all experiments, we process the inputs into feature vector sequences $X=(x_1, x_2, \cdots, x_k)$ to provide a uniform representation across modalities.
% On CLUES dataset, this is done by following the data processing in~\citet{menon2022clues} and convert each input into a text sequence. The text sequence is in the form of ``\texttt{odor | pungent [SEP] ... [SEP] ring-type | pendant}'', where ``\texttt{odor}'' is the attribute type name, and ``\texttt{pungent}'' is the attribute value for this input, so on and so forth. Then we encode the sentence with BERT~\citep{kenton2019bert} and use the word embeddings as input features $X$. To set the maximum length of programs, we manually inspect explanations in CLUES dataset, and observe that a maximum length of 5 covers most cases.
On CUB-Explanations dataset, we use a pretrained visual encoder to obtain image patch representation vectors. These vectors are then flattened as a sequence and used as visual input $X$. We use ResNet~\citep{he2016deep} as visual backbone for \model{}.
For baselines, we make comparisons in two groups. The first group of models does not use parameters from pre-trained vision-language models (VLPMs). We adapt TF-VAEGAN~\citep{narayan2020latent}, a state-of-the-art model on the CUB-200 zero-shot classification task, to use RoBERTa-encoded explanations as auxiliary information. This results in the baseline TF-VAEGAN$_{expl}$. The second group of models are those using pre-trained VLPMs. The main baseline we compare with is CLIP~\citep{radford2021learning}, which is a well-performed pretrained VLPM. We build two of its variants: CLIP$_{linear}$, which only fine-tunes the final linear layer and CLIP$_{finetuned}$, which fine-tunes all parameters on the task. For fairer compasion, in this group wealso  replace the visual encoder with CLIP encoder in our model and get \model{}$_{CLIP}$.


On the ECtHR dataset, for baselines, we use Legal-BERT and hierarchical Legal-BERT to encode inputs, which has reported SotA performance on legal-domain datasets \citep{chalkidis-etal-2022-lexglue}. To utilize language explanations, we use another Legal-BERT~\citep{chalkidis2020legal} to encode the explanations, and compute the dot-product between the input embeddings and and explanation embeddings. The maximal dot-product across explanation sentences is used as classification score for each class.
For our model, we follow~\citet{chalkidis-etal-2022-lexglue} and adopt a hierarchical model structure. First the pre-trained language encoder encodes each paragraph as one vector. Then these vectors are concatenated into a vector sequence and used as $X$. To adapt our model to this setting, we also replace the text encoder with Legal-BERT and get \model$_{Legal-BERT}$.
\subsection{Classification Results}

% \paragraph{Results on CUB-Explanations}
Results are listed in Table~\ref{tab:cub_main_results} and Table~\ref{tab:ecthr_main_results}. On CUB-Explanations \model{} achieves the highest $ACC_U$ and $ACC_\textbf{H}$ both with and without pre-trained vision-language parameters.
Note that fine-tuning all parameters of CLIP makes it fit marginally better on seen classes, but sacrifices its generalization ability. Fine-tuning only the final linear layer (CLIP$_{linear}$) provides slightly better generalizability on unseen categories, but it is still lower than our approach.
On ECtHR-Explanations dataset, our model outperforms the SotA baseline Hierarchical Legal-BERT on all metrics, with gains around 1$\sim$5 percentage points.

\section{Conclusions and Future Work}

\label{sec:conclusions}

In this work, we propose a multi-modal zero-shot classification framework by logical parsing and reasoning on natural language explanations. Our method consistently outperforms baselines across modalities.
% on CUB-200 (image classification), CLUES (structured data classification) and Lex-GLUE and Natural Instructions (text classification).
We also demonstrate that, besides being interpretable, \model{} also benefits more from tasks that require more compositional reasoning, and is more robust against linguistic biases.

There are several future directions to be explored. The most intriguing one is how to utilize pre-trained generative language models for explicit logical reasoning, as pre-trained language models have been shown capable of planning~\citep{DBLP:conf/icml/HuangAPM22}. Another direction is to incorporate semantic reasoning ability in our approach, such as reasoning on entity relations or event roles.



% \section*{Limitations}
% The proposed approach focuses more on logical reasoning on explanations for zero-shot classification. The semantic structures in explanations, such as inter-entity relations and event argument relations, are less touched (although the pre-trained language encoders such as BERT provides semantic matching ability to some extent). Within the range of logical reasoning, our focus are more on first-order logic, while leaving the discussion about higher-order logic for future work.
% \section*{Broader Impact}
% This work is related to and partially inspired by the real-world task of legal text classification. As legal matters can affect the life of real people, and we are yet to fully understand the behaviors of deep-learning-based models, relying more on human expert opinions is still a more prudent choice. While the proposed approach can be utilized for automating the process of legal text, care must be taken before using or referring to the result produced by any machine in legal domain.


\bibliography{anthology,custom}
\bibliographystyle{acl_natbib}

\appendix
\clearpage
\newpage

% \chapter{Appendix}

\section{Appendix}

\subsection{Configuration and Experiment Setting}
\label{appsec:configuration}
We train \model{} for 30 epochs in all experiments. In the image classification task on CUB-Explanations, we adopt a two-phase training paradigm: in the first phase we fix both visual encoders and Explanation encoders in $E_\Phi$, and in the second phase we finetune all parameters in \model{}.

Across experiments in this work we use the AdamW~\citep{loshchilov2017decoupled} optimizer widely adopted for optimizing NLP tasks. For hyper-parameters in most experiments we follow the common practice of learning rate$=3e-5$, $\beta_1=0.9, \beta_2=0.999$, $\epsilon=1e-8$ and weight decay$=0.01$. An exception is the first phase in image classification where, as we fix the input encoder, the learnable parameters become much less. Therefore we use the default learning rate$=1e-3$ in AdamW. For randomness control, we use random seed of 1 across all experiments.

In Figure~\ref{fig:effect_of_complexity}, there are multiple data points at $x$-value of 0. Therefore, the data variance on data at $x=0$ is intrinsic in data, and is unsolvable theoretical for \textit{any} function fitting the data series. This causes the problem when calculating $R^2$ value, as $R^2$ measures the extent to which the data variance are ``explained'' by the fitting function. So $R^2$ can be upper bounded by:
$
    R^2 \leq 1 - \frac{Var_{intrinsic}}{Var_{total}}
$. To deal with this problem when measuring $R^2$ metric, we removed the intrinsic variance in data point set $D$ by replacing data points $(0, y_i)\sim D$ with $(0, \frac{1}{n}\sum_{(0,y_i)\sim D}y_i)$ in both series in Figure~\ref{fig:effect_of_complexity} before calculating $R^2$ value.

\subsection{Logical Structure Templates}
\label{appsec:templates}
As the number of valid logical structure templates grows exponentially with maximal concept numbers $T$, we limit $T$ to a small value, typically 3. We list the logical structure templates in Table~\ref{tab:all_program_templates}.
\begin{table}[ht]
\centering
\small
\linespread{1}

\begin{tabular}{l}

% \setlength{\arrayrulewidth}{.3em}

\toprule

label$(X)$ = concept$_1(X)$ \\
\midrule
label$(X)$ = concept$_1(X) \land$ concept$_2(X)$ \\
\midrule
label$(X)$ = concept$_1(X) \lor$ concept$_2(X)$ \\
\midrule
label$(X)$ = concept$_1(X) \land$ concept$_2(X) \land$ concept$_3(X)$ \\
\midrule
label$(X)$ = concept$_1(X) \lor$ concept$_2(X) \lor$ concept$_3(X)$ \\
\midrule
label$(X)$ = (concept$_1(X) \land$ concept$_2(X)) \lor$ concept$_3(X)$ \\
\midrule
label$(X)$ = (concept$_1(X) \lor$ concept$_2(X)) \land$ concept$_3(X)$ \\

% $\text{function}_1$ = \textit{Exist}($a_1$; X); \\
% return $\text{function}_1$(X) \\

% \midrule

% $\text{function}_1$ = \textit{Exist}($a_1$; X); \\
% $\text{function}_2$ = \textit{Exist}($a_2$; X); \\
% $\text{function}_3$ = And($\text{function}_1$, $\text{function}_2$; X); \\
% return $\text{function}_3$(X) \\

% \midrule


% $\text{function}_1$ = \textit{Exist}($a_1$; X); \\
% $\text{function}_2$ = \textit{Exist}($a_2$; X); \\
% $\text{function}_3$ = Or($\text{function}_1$, $\text{function}_2$; X); \\
% return $\text{function}_3$(X) \\

% \midrule

% $\text{function}_1$ = \textit{Exist}($a_1$; X); \\
% $\text{function}_2$ = \textit{Exist}($a_2$; X); \\
% $\text{function}_3$ = And($\text{function}_1$, $\text{function}_2$; X); \\
% $\text{function}_4$ = \textit{Exist}($a_4$; X); \\
% $\text{function}_5$ = And($\text{function}_3$, $\text{function}_4$; X); \\
% return $\text{function}_5$(X) \\

% \midrule
% $\text{function}_1$ = \textit{Exist}($a_1$; X); \\
% $\text{function}_2$ = \textit{Exist}($a_2$; X); \\
% $\text{function}_3$ = And($\text{function}_1$, $\text{function}_2$; X); \\
% $\text{function}_4$ = \textit{Exist}($a_4$; X); \\
% $\text{function}_5$ = Or($\text{function}_3$, $\text{function}_4$; X); \\
% return $\text{function}_5$(X) \\

% \midrule
% $\text{function}_1$ = \textit{Exist}($a_1$; X); \\
% $\text{function}_2$ = \textit{Exist}($a_2$; X); \\
% $\text{function}_3$ = Or($\text{function}_1$, $\text{function}_2$; X); \\
% $\text{function}_4$ = \textit{Exist}($a_4$; X); \\
% $\text{function}_5$ = And($\text{function}_3$, $\text{function}_4$; X); \\
% return $\text{function}_5$(X) \\

% \midrule
% $\text{function}_1$ = \textit{Exist}($a_1$; X); \\
% $\text{function}_2$ = \textit{Exist}($a_2$; X); \\
% $\text{function}_3$ = Or($\text{function}_1$, $\text{function}_2$; X); \\
% $\text{function}_4$ = \textit{Exist}($a_4$; X); \\
% $\text{function}_5$ = Or($\text{function}_3$, $\text{function}_4$; X); \\
% return $\text{function}_5$(X) \\

\bottomrule

\end{tabular}

\caption{The list of logical structure templates at maximum concept number $T=3$.}
\label{tab:all_program_templates}
\end{table}

\subsection{Effect of Logical Reasoning Complexity}
\label{appsec:ablation}
We also explores the effect of logical reasoning on model performance. Figure ~\ref{fig:program_length} plots the performance regarding the maximum number of concepts $T$. Generally speaking, when $T$ is larger, \model{} can model more complex logical reasoning process. When $T=1$, the model reduces to a simple similarity-based model without logical reasoning. The figure shows that when $T$ is 2$\sim$3, the model generally achieves the highest performance, which also aligns with our intuition in the section~\ref{sec:approach}. We hypothesize that a maximum logical structure length up to 4 provides insufficient regularization, and \model{} is more likely to overfit the data.

\begin{figure}[t]
\centering
\includegraphics[width=\columnwidth]{figures/program_length_fig.pdf}

\caption{The effect of maximum number of attributes $T$ on the classification performance. When $T=1$ the model reduces to a simple similarity-based model.
% The dashed line shows the scores on development set, and solid line shows the test set performance.\xd{the second sentence can be removed since the fig notation provides info.}
}
\label{fig:program_length}
\end{figure}


\subsection{Resources}

We use one Tesla V100 GPU with 16GB memory to carry out all the experiments. The training time is 1 hour for tabular data classification on CLUES, 2 hours for image classification on CUB-Explanations, and 4.5 hours for text classification on ECtHR-Explanations dataset.

\subsection{Interpretability Results}
\label{appsec:interpretability}

We add more interpretability examples in Table~\ref{tab:program_examples} and ~\ref{tab:execution_evidence}.
\begin{table*}[t!]
\centering
\small
\linespread{1}

\setlength{\tabcolsep}{1mm}{

\NewDocumentCommand{\blueback}{ mO{} }{\colorbox{blue!15}{#1}}
\NewDocumentCommand{\greenback}{ mO{} }{\colorbox{green!30}{#1}}


\begin{tabular}{p{17mm}|p{60mm}|p{78mm}}


\toprule
\textbf{Task} & \textbf{Natural Language Explanation} & \textbf{Interpreted Logical Structure} \\
\midrule


car-evaluation
&
\textit{Cars \blueback{with higher safety} \greenback{and capacity} are highly acceptable for resale.}
&
% ($c_{certainty}$=0.74) 
$~\text{Label}(X)=\text{\blueback{with\_higher\_safety}}(X)\land \text{\greenback{and\_capacity}}(X)$
\\

% \hline

% mushroom
% &
% \textit{\blueback{Foul smelling} are \greenback{Poisonous}. 7 of 7 rows,with no deviation.}
% &
% Label$(X)$ = $\blueback{Foul\_smelling}(X) \land \greenback{Poisonous}(X)$
% \\


\hline

indian-liver-patient
&
\textit{Age \blueback{group above 40} \greenback{ensures liver} patient}
&
Label$(X)$ = $\blueback{group\_above\_40}(X) \land \greenback{ensures\_liver}(X)$
\\

\hline

soccer-league-type
&
\textit{If the \blueback{league is W}-PSL then its type is women's soccer}
&
Label$(X)$ = $\blueback{league\_is\_W}(X)$
\\

\hline

award-nomination-result
&
\textit{If the name of \blueback{association has 'American'} in it then the result was mostly won.}
&
Label$(X)$ = $\blueback{association\_has\_'American'}(X)$
\\

% \hline
% dry-bean
% &
% \textit{Below the 215.00 equivalent diameter leads to the \blueback{Dermos}-an class.}
% &
% Label$(X)$ = $\texttt{\blueback{Dermos}}(X)$
% \\

\bottomrule

\end{tabular}
}

% \vspace{-2mm}
\caption{Examples of interpreted logical structures learned by \model{}. We randomly select 5 tasks from CLUES dataset, and use the alphabetically first explanation for interpretation. In each logical structure, the words corresponding to the detected attributes are colored in the explanation.
}
\label{tab:program_examples}
\end{table*}
\begin{table*}[t!]
\centering
\small
\linespread{1}

\setlength{\tabcolsep}{1mm}{

\NewDocumentCommand{\bluecolor}{ mO{} }{\textbf{\textcolor{blue!60}{#1}}}

\begin{tabular}{m{70mm}|p{30mm}|p{60mm}}

\toprule
\textbf{Input} & \textbf{Logical Structure} & \textbf{Execution Evidence}\\
\midrule


\setlength{\tabcolsep}{0.5mm}{
\ \quad \begin{tabular}[t]{ccccc}
\toprule
\bluecolor{safety} & \bluecolor{person capacity} & \bluecolor{buying cost} & maintenance cost & $\cdots$\\
\midrule
\bluecolor{high} & \bluecolor{4} & \bluecolor{med} & low & $\cdots$
\\
\bottomrule
\\
\end{tabular}
}
&
&
Concept ``with higher safety'' matches 
\textit{\bluecolor{safety|high ... buying cost|med}} with score 0.58 ;
Concept ``and capacity'' matches 
\textit{\bluecolor{safety|high ... buying cost|med}} with score 0.65 ;
\texttt{AND} operator score 0.58; return 0.58
\\

\hline



\begin{tabular}[t]{ccccc}
\toprule
SGPT & SGOT & total bilirubin & \bluecolor{age} & \bluecolor{direct bilirubin} \\
\midrule
33 & 71 & 4\bluecolor{.9} & \bluecolor{65} & \bluecolor{2}.7 \\
\bottomrule
\end{tabular}
&
Label$(X)$ = $\texttt{group above 40}(X)\land \texttt{ensures liver}(X)$
&
\texttt{group above 40} matches \texttt{.9 [SEP] age | 65 [SEP] direct bilirubin | 2} with score 0.56;
\texttt{ensures liver} matches \texttt{.9 [SEP] age | 65 [SEP] direct bilirubin | 2} with score 0.57;
\texttt{AND} operator score 0.56
\\

\hline


\begin{tabular}[t]{ccccc}
\toprule
Club & League & \bluecolor{Venue} & City & ... \\
\midrule
Tulsa Spirit & W\bluecolor{PSL} & \bluecolor{Union} 8th & Broken Arrow & ... \\
\bottomrule
\\
\end{tabular}
& 
Label$(X)$ = $\texttt{league is W}(X)$
&
\texttt{league is W} matches: \texttt{PSL [SEP] Venue | Union} with score 0.72
\\

\hline


\begin{tabular}[t]{m{25mm}m{25mm}c}
\toprule
Association & \bluecolor{Category} & Nominee \\
\midrule
American \bluecolor{Comedy award} & \bluecolor{Funniest Actor in a} Motion Picture & Meg Ryan \\
\bottomrule
\\
\end{tabular}
&
Label$(X)$ = \texttt{association has 'American'}$(X)$
&
\texttt{association has 'American'} matches: \texttt{Comedy award [SEP] Category | Funniest Actor in a} with score 0.69
\\


\bottomrule

\end{tabular}
}

\caption{Examples of logical reasoning evidence. The evidence words are colored in the left column for better visualization.}
\label{tab:execution_evidence}
\end{table*}
From the table we can see that this interpretation is readable with some interesting phenomena.
Some parsed logical structures also uses the label names (e.g., \textit{with higher safety} and \textit{with higher safety and capacity}) as concepts. This reflects that the Explanations may not provide complete guidance for classification, so \model{} also partly relies on semantic information from label names.
In the third example, only part of the complete word \textit{WPSL} is used as an concept. We attribute this to that in contextual word embeddings, a token also often includes information from neighbouring words. The contextual embedding of the single token ``\textit{W}'' may contain information of the whole word \textit{WPSL}. The similar phenomenon happens in the 5th example, where \model{} takes \textit{Dermos} as concept from the the word \textit{Dermosan}.

In logical reasoning evidence examples in Table~\ref{tab:execution_evidence}, the detected concepts generally contain the relevant information to $a_l$. For instance, in the first row the concept \texttt{group above 40} selects the text span containing the key information $age | 65$. An interesting phenomenon is when the model selects the label name instead of more specific attribute information as concept. In the first row, the second concept is \texttt{ensures liver} which is directly related to the label \texttt{indian-liver-patient}. In this example, the concept (\texttt{ensures liver}) selects the same text span as the first concept, which is probably due to the model learning the implicit semantic correspondence between this text span and the label.

\begin{figure}
\centering
\includegraphics[width=1.0\columnwidth]{figures/quantifiers_fig.pdf}

\caption{Comparison between the learned certainty coefficients $c_{certainty}$ in \model{} and expert annotations in ~\citet{srivastava2018zero}..
% Upper right show the linear regression equation and coefficient of determination ($R^2$).
% \xd{\sout{figure words too small, i cropped the white space around. in the fig, denote what y and $R^2$ is.}}
}

\vspace{-4mm}
\label{fig:quantifiers}
\end{figure}

\subsection{Linguistic Quantifier Understanding} Linguistic quantifiers is a topic to understand the degree of certainty in natural language~\citep{srivastava2018zero, yildirim2013linguistic}. For example, humans are more certain when saying something \textit{usually} happens, but less certain when using words like \textit{sometimes}.
We observe that the certainty coefficient $c_{certainty}$ that \model{} learns can naturally serve the purpose the of modelling quantifiers.
\hengzhi{\sout{do we have a explanation for certainty coefficient $c_{certainty}$ before?}}\hanchi{\sout{On I forgot. I added this one in Section~\ref{sec:approach}.}}
We first detect the existence of linguistic quantifiers like \textit{often} and \textit{usually} by simply word matching. % listed in Table~\ref{tab:quantifier_extraction}.
Then we take the average of $c_{certainty}$ on the matched explanations. We plot these values against expert-annotated ``quantifier probabilities'' in~\citep{srivastava2018zero}
% , which indicates how ``probable'' the sentence is supposed to be true with a certain quantifier.
in Figure~\ref{fig:quantifiers}. Results show that $c_{certainty}$ correlates positively with ``quantifier probabilities'' with Pearson correlation coefficient value of 0.271. In cases where they disagree, our quantifier coefficients also make some sense, such as assigning \textit{often} a relatively higher value but giving \textit{likely} a lower value.
% \heng{\sout{this paragraph is very hard to follow, and unclear what the point is}}
\hengzhi{and do we need a comparison for this part using the baseline?}
\hanchi{I can only think of a most naive way to do this. Let me try it in the next days}

\end{document}