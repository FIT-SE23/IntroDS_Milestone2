\clearpage
\newpage

% \chapter{Appendix}

\section{Appendix}

\subsection{Configuration and Experiment Setting}
\label{appsec:configuration}
We train \model{} for 30 epochs in all experiments. In the image classification task on CUB-Explanations, we adopt a two-phase training paradigm: in the first phase we fix both visual encoders and Explanation encoders in $E_\Phi$, and in the second phase we finetune all parameters in \model{}.

Across experiments in this work we use the AdamW~\citep{loshchilov2017decoupled} optimizer widely adopted for optimizing NLP tasks. For hyper-parameters in most experiments we follow the common practice of learning rate$=3e-5$, $\beta_1=0.9, \beta_2=0.999$, $\epsilon=1e-8$ and weight decay$=0.01$. An exception is the first phase in image classification where, as we fix the input encoder, the learnable parameters become much less. Therefore we use the default learning rate$=1e-3$ in AdamW. For randomness control, we use random seed of 1 across all experiments.

In Figure~\ref{fig:effect_of_complexity}, there are multiple data points at $x$-value of 0. Therefore, the data variance on data at $x=0$ is intrinsic in data, and is unsolvable theoretical for \textit{any} function fitting the data series. This causes the problem when calculating $R^2$ value, as $R^2$ measures the extent to which the data variance are ``explained'' by the fitting function. So $R^2$ can be upper bounded by:
$
    R^2 \leq 1 - \frac{Var_{intrinsic}}{Var_{total}}
$. To deal with this problem when measuring $R^2$ metric, we removed the intrinsic variance in data point set $D$ by replacing data points $(0, y_i)\sim D$ with $(0, \frac{1}{n}\sum_{(0,y_i)\sim D}y_i)$ in both series in Figure~\ref{fig:effect_of_complexity} before calculating $R^2$ value.

\subsection{Logical Structure Templates}
\label{appsec:templates}
As the number of valid logical structure templates grows exponentially with maximal concept numbers $T$, we limit $T$ to a small value, typically 3. We list the logical structure templates in Table~\ref{tab:all_program_templates}.
\begin{table}[ht]
\centering
\small
\linespread{1}

\begin{tabular}{l}

% \setlength{\arrayrulewidth}{.3em}

\toprule

label$(X)$ = concept$_1(X)$ \\
\midrule
label$(X)$ = concept$_1(X) \land$ concept$_2(X)$ \\
\midrule
label$(X)$ = concept$_1(X) \lor$ concept$_2(X)$ \\
\midrule
label$(X)$ = concept$_1(X) \land$ concept$_2(X) \land$ concept$_3(X)$ \\
\midrule
label$(X)$ = concept$_1(X) \lor$ concept$_2(X) \lor$ concept$_3(X)$ \\
\midrule
label$(X)$ = (concept$_1(X) \land$ concept$_2(X)) \lor$ concept$_3(X)$ \\
\midrule
label$(X)$ = (concept$_1(X) \lor$ concept$_2(X)) \land$ concept$_3(X)$ \\

% $\text{function}_1$ = \textit{Exist}($a_1$; X); \\
% return $\text{function}_1$(X) \\

% \midrule

% $\text{function}_1$ = \textit{Exist}($a_1$; X); \\
% $\text{function}_2$ = \textit{Exist}($a_2$; X); \\
% $\text{function}_3$ = And($\text{function}_1$, $\text{function}_2$; X); \\
% return $\text{function}_3$(X) \\

% \midrule


% $\text{function}_1$ = \textit{Exist}($a_1$; X); \\
% $\text{function}_2$ = \textit{Exist}($a_2$; X); \\
% $\text{function}_3$ = Or($\text{function}_1$, $\text{function}_2$; X); \\
% return $\text{function}_3$(X) \\

% \midrule

% $\text{function}_1$ = \textit{Exist}($a_1$; X); \\
% $\text{function}_2$ = \textit{Exist}($a_2$; X); \\
% $\text{function}_3$ = And($\text{function}_1$, $\text{function}_2$; X); \\
% $\text{function}_4$ = \textit{Exist}($a_4$; X); \\
% $\text{function}_5$ = And($\text{function}_3$, $\text{function}_4$; X); \\
% return $\text{function}_5$(X) \\

% \midrule
% $\text{function}_1$ = \textit{Exist}($a_1$; X); \\
% $\text{function}_2$ = \textit{Exist}($a_2$; X); \\
% $\text{function}_3$ = And($\text{function}_1$, $\text{function}_2$; X); \\
% $\text{function}_4$ = \textit{Exist}($a_4$; X); \\
% $\text{function}_5$ = Or($\text{function}_3$, $\text{function}_4$; X); \\
% return $\text{function}_5$(X) \\

% \midrule
% $\text{function}_1$ = \textit{Exist}($a_1$; X); \\
% $\text{function}_2$ = \textit{Exist}($a_2$; X); \\
% $\text{function}_3$ = Or($\text{function}_1$, $\text{function}_2$; X); \\
% $\text{function}_4$ = \textit{Exist}($a_4$; X); \\
% $\text{function}_5$ = And($\text{function}_3$, $\text{function}_4$; X); \\
% return $\text{function}_5$(X) \\

% \midrule
% $\text{function}_1$ = \textit{Exist}($a_1$; X); \\
% $\text{function}_2$ = \textit{Exist}($a_2$; X); \\
% $\text{function}_3$ = Or($\text{function}_1$, $\text{function}_2$; X); \\
% $\text{function}_4$ = \textit{Exist}($a_4$; X); \\
% $\text{function}_5$ = Or($\text{function}_3$, $\text{function}_4$; X); \\
% return $\text{function}_5$(X) \\

\bottomrule

\end{tabular}

\caption{The list of logical structure templates at maximum concept number $T=3$.}
\label{tab:all_program_templates}
\end{table}

\subsection{Effect of Logical Reasoning Complexity}
\label{appsec:ablation}
We also explores the effect of logical reasoning on model performance. Figure ~\ref{fig:program_length} plots the performance regarding the maximum number of concepts $T$. Generally speaking, when $T$ is larger, \model{} can model more complex logical reasoning process. When $T=1$, the model reduces to a simple similarity-based model without logical reasoning. The figure shows that when $T$ is 2$\sim$3, the model generally achieves the highest performance, which also aligns with our intuition in the section~\ref{sec:approach}. We hypothesize that a maximum logical structure length up to 4 provides insufficient regularization, and \model{} is more likely to overfit the data.

\begin{figure}[t]
\centering
\includegraphics[width=\columnwidth]{figures/program_length_fig.pdf}

\caption{The effect of maximum number of attributes $T$ on the classification performance. When $T=1$ the model reduces to a simple similarity-based model.
% The dashed line shows the scores on development set, and solid line shows the test set performance.\xd{the second sentence can be removed since the fig notation provides info.}
}
\label{fig:program_length}
\end{figure}


\subsection{Resources}

We use one Tesla V100 GPU with 16GB memory to carry out all the experiments. The training time is 1 hour for tabular data classification on CLUES, 2 hours for image classification on CUB-Explanations, and 4.5 hours for text classification on ECtHR-Explanations dataset.

\subsection{Interpretability Results}
\label{appsec:interpretability}

We add more interpretability examples in Table~\ref{tab:program_examples} and ~\ref{tab:execution_evidence}.
\begin{table*}[t!]
\centering
\small
\linespread{1}

\setlength{\tabcolsep}{1mm}{

\NewDocumentCommand{\blueback}{ mO{} }{\colorbox{blue!15}{#1}}
\NewDocumentCommand{\greenback}{ mO{} }{\colorbox{green!30}{#1}}


\begin{tabular}{p{17mm}|p{60mm}|p{78mm}}


\toprule
\textbf{Task} & \textbf{Natural Language Explanation} & \textbf{Interpreted Logical Structure} \\
\midrule


car-evaluation
&
\textit{Cars \blueback{with higher safety} \greenback{and capacity} are highly acceptable for resale.}
&
% ($c_{certainty}$=0.74) 
$~\text{Label}(X)=\text{\blueback{with\_higher\_safety}}(X)\land \text{\greenback{and\_capacity}}(X)$
\\

% \hline

% mushroom
% &
% \textit{\blueback{Foul smelling} are \greenback{Poisonous}. 7 of 7 rows,with no deviation.}
% &
% Label$(X)$ = $\blueback{Foul\_smelling}(X) \land \greenback{Poisonous}(X)$
% \\


\hline

indian-liver-patient
&
\textit{Age \blueback{group above 40} \greenback{ensures liver} patient}
&
Label$(X)$ = $\blueback{group\_above\_40}(X) \land \greenback{ensures\_liver}(X)$
\\

\hline

soccer-league-type
&
\textit{If the \blueback{league is W}-PSL then its type is women's soccer}
&
Label$(X)$ = $\blueback{league\_is\_W}(X)$
\\

\hline

award-nomination-result
&
\textit{If the name of \blueback{association has 'American'} in it then the result was mostly won.}
&
Label$(X)$ = $\blueback{association\_has\_'American'}(X)$
\\

% \hline
% dry-bean
% &
% \textit{Below the 215.00 equivalent diameter leads to the \blueback{Dermos}-an class.}
% &
% Label$(X)$ = $\texttt{\blueback{Dermos}}(X)$
% \\

\bottomrule

\end{tabular}
}

% \vspace{-2mm}
\caption{Examples of interpreted logical structures learned by \model{}. We randomly select 5 tasks from CLUES dataset, and use the alphabetically first explanation for interpretation. In each logical structure, the words corresponding to the detected attributes are colored in the explanation.
}
\label{tab:program_examples}
\end{table*}
\begin{table*}[t!]
\centering
\small
\linespread{1}

\setlength{\tabcolsep}{1mm}{

\NewDocumentCommand{\bluecolor}{ mO{} }{\textbf{\textcolor{blue!60}{#1}}}

\begin{tabular}{m{70mm}|p{30mm}|p{60mm}}

\toprule
\textbf{Input} & \textbf{Logical Structure} & \textbf{Execution Evidence}\\
\midrule


\setlength{\tabcolsep}{0.5mm}{
\ \quad \begin{tabular}[t]{ccccc}
\toprule
\bluecolor{safety} & \bluecolor{person capacity} & \bluecolor{buying cost} & maintenance cost & $\cdots$\\
\midrule
\bluecolor{high} & \bluecolor{4} & \bluecolor{med} & low & $\cdots$
\\
\bottomrule
\\
\end{tabular}
}
&
&
Concept ``with higher safety'' matches 
\textit{\bluecolor{safety|high ... buying cost|med}} with score 0.58 ;
Concept ``and capacity'' matches 
\textit{\bluecolor{safety|high ... buying cost|med}} with score 0.65 ;
\texttt{AND} operator score 0.58; return 0.58
\\

\hline



\begin{tabular}[t]{ccccc}
\toprule
SGPT & SGOT & total bilirubin & \bluecolor{age} & \bluecolor{direct bilirubin} \\
\midrule
33 & 71 & 4\bluecolor{.9} & \bluecolor{65} & \bluecolor{2}.7 \\
\bottomrule
\end{tabular}
&
Label$(X)$ = $\texttt{group above 40}(X)\land \texttt{ensures liver}(X)$
&
\texttt{group above 40} matches \texttt{.9 [SEP] age | 65 [SEP] direct bilirubin | 2} with score 0.56;
\texttt{ensures liver} matches \texttt{.9 [SEP] age | 65 [SEP] direct bilirubin | 2} with score 0.57;
\texttt{AND} operator score 0.56
\\

\hline


\begin{tabular}[t]{ccccc}
\toprule
Club & League & \bluecolor{Venue} & City & ... \\
\midrule
Tulsa Spirit & W\bluecolor{PSL} & \bluecolor{Union} 8th & Broken Arrow & ... \\
\bottomrule
\\
\end{tabular}
& 
Label$(X)$ = $\texttt{league is W}(X)$
&
\texttt{league is W} matches: \texttt{PSL [SEP] Venue | Union} with score 0.72
\\

\hline


\begin{tabular}[t]{m{25mm}m{25mm}c}
\toprule
Association & \bluecolor{Category} & Nominee \\
\midrule
American \bluecolor{Comedy award} & \bluecolor{Funniest Actor in a} Motion Picture & Meg Ryan \\
\bottomrule
\\
\end{tabular}
&
Label$(X)$ = \texttt{association has 'American'}$(X)$
&
\texttt{association has 'American'} matches: \texttt{Comedy award [SEP] Category | Funniest Actor in a} with score 0.69
\\


\bottomrule

\end{tabular}
}

\caption{Examples of logical reasoning evidence. The evidence words are colored in the left column for better visualization.}
\label{tab:execution_evidence}
\end{table*}
From the table we can see that this interpretation is readable with some interesting phenomena.
Some parsed logical structures also uses the label names (e.g., \textit{with higher safety} and \textit{with higher safety and capacity}) as concepts. This reflects that the Explanations may not provide complete guidance for classification, so \model{} also partly relies on semantic information from label names.
In the third example, only part of the complete word \textit{WPSL} is used as an concept. We attribute this to that in contextual word embeddings, a token also often includes information from neighbouring words. The contextual embedding of the single token ``\textit{W}'' may contain information of the whole word \textit{WPSL}. The similar phenomenon happens in the 5th example, where \model{} takes \textit{Dermos} as concept from the the word \textit{Dermosan}.

In logical reasoning evidence examples in Table~\ref{tab:execution_evidence}, the detected concepts generally contain the relevant information to $a_l$. For instance, in the first row the concept \texttt{group above 40} selects the text span containing the key information $age | 65$. An interesting phenomenon is when the model selects the label name instead of more specific attribute information as concept. In the first row, the second concept is \texttt{ensures liver} which is directly related to the label \texttt{indian-liver-patient}. In this example, the concept (\texttt{ensures liver}) selects the same text span as the first concept, which is probably due to the model learning the implicit semantic correspondence between this text span and the label.

\begin{figure}
\centering
\includegraphics[width=1.0\columnwidth]{figures/quantifiers_fig.pdf}

\caption{Comparison between the learned certainty coefficients $c_{certainty}$ in \model{} and expert annotations in ~\citet{srivastava2018zero}..
% Upper right show the linear regression equation and coefficient of determination ($R^2$).
% \xd{\sout{figure words too small, i cropped the white space around. in the fig, denote what y and $R^2$ is.}}
}

\vspace{-4mm}
\label{fig:quantifiers}
\end{figure}

\subsection{Linguistic Quantifier Understanding} Linguistic quantifiers is a topic to understand the degree of certainty in natural language~\citep{srivastava2018zero, yildirim2013linguistic}. For example, humans are more certain when saying something \textit{usually} happens, but less certain when using words like \textit{sometimes}.
We observe that the certainty coefficient $c_{certainty}$ that \model{} learns can naturally serve the purpose the of modelling quantifiers.
\hengzhi{\sout{do we have a explanation for certainty coefficient $c_{certainty}$ before?}}\hanchi{\sout{On I forgot. I added this one in Section~\ref{sec:approach}.}}
We first detect the existence of linguistic quantifiers like \textit{often} and \textit{usually} by simply word matching. % listed in Table~\ref{tab:quantifier_extraction}.
Then we take the average of $c_{certainty}$ on the matched explanations. We plot these values against expert-annotated ``quantifier probabilities'' in~\citep{srivastava2018zero}
% , which indicates how ``probable'' the sentence is supposed to be true with a certain quantifier.
in Figure~\ref{fig:quantifiers}. Results show that $c_{certainty}$ correlates positively with ``quantifier probabilities'' with Pearson correlation coefficient value of 0.271. In cases where they disagree, our quantifier coefficients also make some sense, such as assigning \textit{often} a relatively higher value but giving \textit{likely} a lower value.
% \heng{\sout{this paragraph is very hard to follow, and unclear what the point is}}
\hengzhi{and do we need a comparison for this part using the baseline?}
\hanchi{I can only think of a most naive way to do this. Let me try it in the next days}