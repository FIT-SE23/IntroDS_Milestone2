
\subsection{Linguistic Quantifier Understanding} Linguistic quantifiers is a topic to understand the degree of certainty in natural language~\citep{srivastava2018zero, yildirim2013linguistic}. For example, humans are more certain when saying something \textit{usually} happens, but less certain when using words like \textit{sometimes}.
We observe that the certainty coefficient $c_{certainty}$ that \model{} learns can naturally serve the purpose the of modelling quantifiers.
\hengzhi{\sout{do we have a explanation for certainty coefficient $c_{certainty}$ before?}}\hanchi{\sout{On I forgot. I added this one in Section~\ref{sec:approach}.}}
We first detect the existence of linguistic quantifiers like \textit{often} and \textit{usually} by simply word matching. % listed in Table~\ref{tab:quantifier_extraction}.
Then we take the average of $c_{certainty}$ on the matched explanations. We plot these values against expert-annotated ``quantifier probabilities'' in~\citep{srivastava2018zero}
% , which indicates how ``probable'' the sentence is supposed to be true with a certain quantifier.
in Figure~\ref{fig:quantifiers}. Results show that $c_{certainty}$ correlates positively with ``quantifier probabilities'' with Pearson correlation coefficient value of 0.271. In cases where they disagree, our quantifier coefficients also make some sense, such as assigning \textit{often} a relatively higher value but giving \textit{likely} a lower value.
% \heng{\sout{this paragraph is very hard to follow, and unclear what the point is}}
\hengzhi{and do we need a comparison for this part using the baseline?}
\hanchi{I can only think of a most naive way to do this. Let me try it in the next days}