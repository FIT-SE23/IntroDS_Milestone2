\section{\emph{Adaptable} Add-on Components}

\label{sec:adaptability}
% Unanticipated Behaviors and Personalization
Deploying research systems over long time periods requires significant adaptability in many ways. Because the user is central, in this section we concentrate on adaptability in regards to the user interface.
%Because of unanticipated human behaviors the system must be adaptable. In this section, we evaluate the adaptability of the XYZ-W System to individuals' unexpected behaviors or needs. Our takeaway message from this Section is that the technology to be deployed must be adaptable so that it will be easily adjusted when the users complain about the technology.

\subsection{EMA Wording}
\label{sec:ema_wording}
EMA apps are common interfaces to humans. They are used to send messages to the participants and receive responses. In our system messages are the interventions or recommendations based on the output of the emotion, conflict detection, and reinforcement learning recommendation modules. 

During the pre-deployment time, we collaborated with a team of psychologists and specialists from the nursing field and determined four main categories of interventions/recommendations: the breathing exercise, which encourages the participants to take deep breaths to calm down, the timeout exercise, which encourages the participants to take a timeout/break from engaging with their loved one (the dementia patient), the mindfulness exercise, which encourages them to practice mindfulness, and enjoyable activities, which encourage them to partake activities that they enjoy. Each of these categories contains multiple subcategories. Our implementation  allows the exact wordings of the messages to be easily changeable in case during the deployment time the participants preferred different wording. 

%\subsubsection{Post-deployment Stage Assessment}
Indeed, during the deployment time, we received feedback from the some participants that certain messages seemed insensitive and harsh, and the insensitivity and harshness of the messages actually discouraged them from implementing the recommended interventions. To fix this problem for these users, we went back to the EMA app and easily changed the wording based on the participants' complaints. After we made the changes to the wording for all participants, they no longer felt the wording as insensitive and harsh, and were encouraged by the messages to implement the interventions. 
%For smart health teams that are developing smart technologies all over the world that use an app similar to the EMA app to suggest recommendations, our advice is that they must make the app easily adjustable to respond to the users' complaints on the traits of such an app.

%\begin{table}[]
%\centering
%\scalebox{0.85}{
%\begin{tabular}{ccccccc}
%\toprule
%          & Dyad 1  & Dyad 2  & Dyad 3  & Dyad 4 & Dyad 5 & Dyad 6\\
%\hline
%Concern addressed       & \checkmark  & \checkmark   & \checkmark    & \checkmark & \checkmark  & \checkmark   \\
%\bottomrule
%\\
%\end{tabular}
%}
%\caption{The evaluation results for if the caregivers of the dyads like the wordings after we change the wordings. All six of the dyads like the change, indicating that the mechanism to make the change happen - the adaptability of the EMA - is very important. The check mark indicates that a specific participant likes the change.}
%\label{tab:eval_EMA_wordings}
%\end{table}

\subsection{Personalized Recommendations}
We have previously stated in Section \ref{sec:ema_wording} that there are four categories of interventions or recommendations. In this Section, we describe how we make the personalized recommendations adaptable during pre-deployment time and verify if our strategy (that makes the recommendations adaptable) succeeds during post-deployment time.

%\subsubsection{Pre-deployment Stage Preparation}
We have also stated that during pre-deployment time we made the EMA app to be easily adjustable in case we need to change the interventions or recommendations. The EMA app is easily adjustable because it reads text questions (such as the list of enjoyable activities for the participants) from a database and to change that list, we just need to go to the database instead of recompiling the app every time we make changes to the list of text surveys and interventions. The adaptability of the EMA app and, therefore, the XYZ-W system came in handy when we discovered that participants want more specific recommendations, especially under the category of enjoyable activities. In other words, instead of a generic ``now it's time to do some enjoyable activities,'' they wanted the enjoyable activities to be more specific and personalized such as ``now it's time to play with the family cat.'' To accommodate this, we asked each participant to provide us a list of their personalized enjoyable activities. Because the EMA app was made to be easily adjustable, the integration of the personalized enjoyable activities for each participant was quick and easy. 

%\subsubsection{Post-deployment Stage Assessment}
The participants of all six dyads reported that they liked the personalization (we asked them through surveys during our interviews with them). For smart health apps that make recommendations, we feel that it is imperative to be able to change them to accommodate the personal needs of the participants.

%\begin{table}[]
%\centering
%\scalebox{0.85}{
%\begin{tabular}{ccccccc}
%\toprule
%          & Dyad 1  & Dyad 2  & Dyad 3  & Dyad 4 & Dyad 5 & Dyad 6\\
%\hline
%Concern addressed       & \checkmark  & \checkmark   & \checkmark    & \checkmark & \checkmark  & \checkmark   \\
%VGG         & 44.6\%    & 100\%     & 48.3 \%   \\
%\bottomrule
%\\
%\end{tabular}
%}
%\caption{The evaluation results for if the caregivers of the dyads like the personalization. All six of the dyads like the personalization, indicating that the mechanism to make the personalization happen - the adaptability of the EMA - is very important. The check mark indicates that a specific participant likes the personalization.}
%\label{tab:eval_personalization}
%\end{table}

\subsection{Positive Feedback}
Before we deployed our XYZ-W System, we decided to only send out at the end of the day post-recommendation surveys that asked the participants if they had implemented the recommendations. Then, we received requests from the participants that they would like some positive feedback after they implemented the recommendations. The positive feedback should acknowledge the effort they put in to adhering to the recommendations and remind them of the importance of implementing the recommendations (to improve their mental health and lessen their care-giving burden). Again, because the EMA app was designed at pre-deployment time to be easily changed, integrating the feature of positive feedback into the existing system was quick and easy. The participants of all six dyads reported to have liked the positive feedback.  

%\begin{table}[]
%\centering
%\scalebox{0.85}{
%\begin{tabular}{ccccccc}
%\toprule
%          & Dyad 1  & Dyad 2  & Dyad 3  & Dyad 4 & Dyad 5 & Dyad 6\\
%\hline
%Response       & \checkmark  & \checkmark   & \checkmark    & \checkmark & \checkmark  & \checkmark   \\
%VGG         & 44.6\%    & 100\%     & 48.3 \%   \\
%\bottomrule
%\\
%\end{tabular}
%}
%\caption{The evaluation results for if the caregivers of the dyads like the positive feedback. All six of the dyads like the positive feedback, indicating that the mechanism to make the positive feedback happen - the adaptability of the EMA - is very important. The check mark indicates that a specific participant likes the positive feedback.}
%\label{tab:eval_positive_feedback}
%\end{table}

\subsection{Summary}

In this Section we briefly summarize our findings in Section \ref{sec:adaptability}. The main takeaway message in Section \ref{sec:adaptability} is that researchers need to anticipate the need of the participants and make sure that their technology (in our case, the add-on components to the XYZ System), is adaptable to those needs. For example, we initially set the positive feedback times to be in the morning and in the evening, but participants report that they want the positive feedback more often. Since we have anticipated that they might have the need (to want to see the positive feedback more often), we have designed our add-on components in such a way that allows the change to be easily made. %Our takeaway message from this Section is that the technology to be deployed must be adaptable so that it will be easily adjusted when the users complain about the technology.