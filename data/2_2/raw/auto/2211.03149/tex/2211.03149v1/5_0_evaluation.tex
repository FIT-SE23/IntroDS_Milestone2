\section{Evaluations}
\label{sec:evaluation}
Many speech processing works have collected or augmented datasets with real world sounds. Good solutions are usually then developed. But, in many cases the resultant datasets have limited in-the-wild sounds so where they actually work is limited, and many times the solutions are not validated in-the-wild, but only on the datasets. 

Our pre-deployment strategies are to significantly increase the collected and augmented datasets to increase comprehensiveness of the situations which are modeled, and specialize the datasets to a specific speech module (if needed).  In pre-deployment, we also choose a set of “best” solutions from the literature and using the comprehensive datasets, determine which ones work. If one or more off-the-shelf solution work, the best one is chosen. Below we show that often many off-the-shelf solutions do not work, but for many common speech tasks there are good solutions available, but it is necessary to verify that. We don’t want to reinvent the wheel. Then, in post deployment we validate the resulting performance on real data, a step often not performed.

Speech processing is typically performed in a pipeline depending on the overall purpose. In our system we need voice activity detection, speaker identification,  emotion detection, and conflict detection. There are commonalities and unique aspects to each of these stages so the pre-deployments strategies must account for them. We now consider each of these 4 stages in more detail. Specifically, we briefly reiterate what the components are, why they are important, and demonstrate what we did at pre-deployment time to maximize the deployment time success based on post-deployment data (how we evaluated them at pre-deployment time). In all cases the acoustic solutions had to address the real world complexities such as reverberation, deamplification, and the many types of noise found in real homes over long deployments. The relative success of the acoustic solutions (at the post-deployment time) confirms our hypothesis that it is possible to perform comprehensive and realistic pre-deployment testing to increase post deployment success.


\subsection{Voice Activity Detection}
The VAD model filters silence and other sounds that are not produced by the human vocal tract. Since the acoustic system is constantly listening to the environment, we do not want to activate the emotion and conflict classifiers when an input sound window contains no human speech. As a result, the VAD model is an important and necessary component in the XYZ System. In this Section, we describe testing on the VAD model for the question: Is it possible to perform comprehensive and realistic pre-deployment testing to improve post-deployment success? Note that the acoustic realisms that the VAD model faces are deamplification, reverberation, and (non-speech) background noise, so in our pre-deployment stage assessment we seek to find a VAD solution that is robust against these three types of acoustical realisms.

\subsubsection{Pre-deployment Stage Assessment}

During pre-deployment time, we first looked into several state-of-the-art VAD algorithms. In particular, we studied the performance of a set of SOTA VAD algorithms on the Aurora-2 database \cite{hirsch2000aurora}. The performance of the algorithms on the Aurora-2 database is a good indicator of how they might perform in the real world because Aurora-2's speech samples are mixed with noise collected from realistic settings such as streets, airports, and cars. Table \ref{tab:eval_pre_vad} is a list of existing SOTA VAD algorithms' accuracy scores on the testing set of Aurora-2. Unfortunately, as we can see, the highest-performing one is rVAD, which only achieves an accuracy score of 66.23\%, which is far from being usable in the real world. In other words, there exists solutions in the literature that do not work on datasets with deamplification, reverberation, and background noise. It is good to discover that these solutions are not likely to work at post-deployment time, because this helps us filter out existing solutions so that we won't use those solutions.

\begin{table}[]
\centering
\begin{tabular}{cc}
\toprule
Existing SOTA VAD         & Accuracy  \\
\hline
VQVAD       & 45.66\%          \\
Sohn et al. & 31.21\%          \\
Kaldi Energy VAD    & 11.72\%       \\
DSR AFE     & 40.07\%           \\
rVAD        & 66.23\%           \\
\bottomrule
\\
\end{tabular}
\caption{Evaluation of existing VAD algorithms on the Aurora-2 database. This Table is reported by Tan et al. \cite{tan2020rvad}.}
\label{tab:eval_pre_vad}
\end{table}

However, there was another algorithm, the Google Speech Recognition (GSR) algorithm, that had not been evaluated on a dataset that contained the three environmental distortions: reverberation, deamplification, and background noise. As a result, we next evaluated the Google Speech Recognition solution. we aimed to evaluate it in a comprehensive way to demonstrate that it would be robust against environmental distortions such as reverberation, deamplification, and background noise. Again, this is to set up the necessary condition to prove our hypothesis that a solution, during the pre-deployment stage, must be able to deal with the unique challenges given the real, designated environment in which it will be deployed. 
%In the case of the VAD algorithm, the unique challenges are the reverberation, deamplification, and (in-home) background noise effects.

To do so, we collected a dataset that contains diverse environmental distortions: first, we collected the clean samples - samples that are not environmentally distorted, by having an individual talk next to the microphone for 5 minutes. The 5-minute clip was then sliced into 60 5-second segments, each of which is individually labeled as positive if it contained a human voice, or negative if it did not contain human voices. Note that the individual took long pauses intentionally to make sure that there were negative samples. Second, we collected the audio clips that were deamplified and contained background noise. To do so, we took copies of the clean samples. For each of the copies, we randomly deamplify them by m decibels (0 < m < 12) as per the practice of a previous work on emotion detection \cite{gao2021emotion}. Then, we randomly picked household sounds from the household ambience dataset \cite{mesaros2016tut}. Table \ref{tab:homenoise} lists the events that occur in the dataset. Note that each of the ambience sounds is greater than 5 seconds, so we randomly picked a segment from it that was 5-seconds long, and overlaid it with a deamplified clip. We repeated this process for all 60 deamplified clips. Third, we created the data for reverberated speech. To do so, we took another set of copies of the clean samples, and overlaid each of them with reverberation that was described by the combination of the three parameters: the wet/dry ratio $r$, diffusion $d$, and decay factor $f$. Finally, we created samples that are deamplified, noise-contaminated, and reverberated. To do so, we took a set of copies of the 60 deamplified and noise-contaminated samples, and overlaid them with the same reverberation effect as the samples that only contained reverberation effect and nothing more. In the end, we have 60 clean samples, 60 deamplified and noise-contaminated samples, 60 reverberated samples, and 60 samples that had all three environmental distortions. As a result, we claim that we created a dataset that was comprehensive enough to account for all three kinds of environmental distortions.

\begin{table}
    \centering
    
    \begin{tabular}{cc}
    \toprule
    Event & Instances\\
    \midrule
    (object) rustling & 60 \\
    (object) snapping  & 57 \\
    cupboard & 40 \\
    cutlery & 76\\
    dishes & 151\\
    drawers & 51\\
    glass jingling & 36 \\
    object impact & 250 \\
    people walking & 54 \\
    washing dishes & 84 \\
    water tap running & 47 \\
    \bottomrule
\end{tabular}
\caption{Events that are present in the background noise collected from real homes from the dataset \cite{mesaros2016tut}. All of them are covered in the process of contaminating audio samples with background noise. Note that this list do not include sounds from the tv, which are very important to make sure the robustness of the emotion detection model and conflict detection model.}
\label{tab:homenoise}
\end{table}

We evaluated GSR on the dataset that we just created. \textbf{GSR achieved an accuracy score of 95.83\%}, correctly classifying 230 out of the 240 samples each of which accounted for the environmental distortions to a certain degree. The high performance led us to decide to deploy GSR as our VAD model since, during the pre-deployment stage assessment, it is shown to be robust against the challenges that it is about to encounter in the real, designated environment: reverberation, background noise, and deamplification. 

%In the following paragraphs, we seek to confirm if the hypothesis that, during a pre-deployment stage, an about-to-be-deployed algorithm being able to overcome the challenges perceived to be present in the real, designated environment is going to perform well in that real, designated environment.

%\begin{table}[]
%\centering
%\begin{tabular}{cc}
%\toprule
%          &  Accuracy\\
%\hline
%GSR & 95.83\%   \\
%\bottomrule
%\\
%\end{tabular}
%\caption{The evaluation results on the GSR pre-deployment on the dataset that we created, which accounted %for the three environmental distortions: room reverberation, deamplification, and indoor background noise.}
%\label{tab:eval_pre_vad}
%\end{table}

\subsubsection{Post-deployment Stage Assessment}

Using post-deployment data on six completed dyads, we validate how well the chosen solution worked in practice. Table \ref{tab:eval_post_vad} shows the evaluation results of the VAD model on the dyads. We randomly select samples generated by each dyad during their deployment, and have human labelers label them if they are of human speech or not. We obtained 100 samples for all the dyads. The high performance of the VAD model indicates that this part of our system is highly effective at filtering out non-human speech samples such as background music (without lyrics) and footsteps. It is noted that the VAD does not filter out TV sounds if there is human speech in the sounds, such the voices of actors or news anchors. These unwanted sounds are filtered by the next model, SID.

The VAD model achieves an accuracy score of 94.0\% to 100\% on the six dyads. The high performance on post-deployment data validates our choice of the Google Speech Recognition in the pre-deployment phase. This implies that this VAD algorithm was originally made very robust to real world complexities. The high performance also indicates that, in order for the deployment to be successful, smart health groups using audio should perform pre-deployment tests with comprehensive real-world distortions. In addition, the high performance suggests that our hypothesis holds true - recall that our hypothesis is that, during the pre-deployment stage assessment, an about-to-be-deployed algorithm must be proven to overcome the challenges that are perceived to be present in the real, designated environment in order for it to perform well in said environment. The high performance on post-deployment data also indicates that it is sometimes possible to perform comprehensive and realistic pre-deployment testing to improve VAD post-deployment success.

\begin{table}[]
\centering
\scalebox{0.9}{
\begin{tabular}{ccccccc}
\toprule
          & Dyad 1  & Dyad 2  & Dyad 3 & Dyad 4 & Dyad 5 & Dyad 6\\
\hline
GSR & 100\%  & 100\%   & 94.0\%    & 95.0\%  & 100\%   & 98.0\% \\
\bottomrule
\\
\end{tabular}}
\caption{The evaluation results for the voice activity detection model on the dyads. The high accuracy scores achieved from the dyads indicate that the VAD algorithm (Google Speech Recognition) is highly effective at differentiating non-speech from human speech audio samples.}
\label{tab:eval_post_vad}
\end{table}

\subsection{Speaker Identification (SID)}

The SID model determines the identity of a speaker. The SID is a crucial part of the Acoustic System because we only want the voices of the caregiver and patient to be sent to the emotion and conflict detection models. However, in real deployments, voices from the TV and visitors must be filtered out. In this Section, we test SID model to answer this question: is it possible to perform comprehensive and realistic pre-deployment testing to improve post-deployment success? Note that the acoustical realisms that the SID faces, in addition to (non-speech) background noise, reverberation, and deamplification, also include sounds from the tv such as the dialogues from tv characters, for the presence of another person's voice in an audio sample could confuse the speaker identification model.

\subsubsection{Pre-deployment Stage Assessment}
During pre-deployment time, we investigated a state-of-the-art SID algorithm, the Google Speaker Identification API. However, the API asks us to input the maximum number of speakers there can be in a clip. This is impractical because a dyad can have the TV on and there could be many people's voices from the TV, or there may be multiple visitors. It is good to discover that this solution is not likely to work at post-deployment time, because this helps us filter out existing solution(s).

Now we describe how we test to make sure the about-to-be-deployed SID algorithm developed by Microsoft \cite{chen2022wavlm} is robust to environmental distortions such as reverberation, background noise, and deamplification. Again, this is to verify our hypothesis that for an algorithm to be successful in the real, designated environment, it must be able to overcome the challenges present in the real, designated environment during the pre-deployment stage. In our case, the challenges are reverberation, deamplification, non-speech background noise and TV sounds. Specifically, we have two persons, P1 and P2, each of whom spoke next to the microphone for 2.5 minutes. Then, for each of their voice files, we sliced it into 28 audio samples. Because these 56 (28$\times$2) samples were collected when the speakers were right next to the microphone, they were considered clean speech, free of the three types of environmental distortions. We needed to craft environmentally distorted samples out of the 56 clean samples to ensure that the testing samples accounted for both clean and environmentally distorted samples. To do so, we copy each of the 56 clips and deamplify them by randomly choosing a real number between 0 and 12 decibels. Then, we randomly chose a noise clip from Table \ref{tab:homenoise} as well as TV sounds we recorded using a microphone, out of which we randomly chose a consecutive 5-second segment to be overlaid with one of the copies. This guaranteed samples that were deamplified and contaminated with noise, and the last step was to reverberate it. Again, the reverberation effect is described by the three parameters: the wet/dry ration $r$, diffusion $d$, and decay factor $f$, as per the practice of a previous work \cite{salekin2017distant}. When we reverberated a (noise-contaminated and deamplified) copy, the values of $r$, $d$, and $f$ are randomly chosen. In total, we had 112 samples, 56 of which belonged to P1 and the other 56 belonged to P2. We fed the 112 samples to our SID model. \textbf{The SID model achieves an f1 score of 85.7\% on P1 and 92.8\% on P2}. The high performance of the SID model led us to believe that it was reasonably robust to reverberation, deamplification, background noise, and TV sound.


%\begin{table}[]
%\centering
%\begin{tabular}{cc}
%\toprule
%          &  f1 score\\
%\hline
%P1 & 85.7\%   \\
%P2 & 92.8\%   \\
%Avg & 89.2\%    \\
%\bottomrule
%\\
%\end{tabular}
%\caption{The evaluation results on the SID algorithm that we created at pre-deployment on the dataset that we created, which accounted for the three environmental distortions: room reverberation, deamplification, and indoor background noise.}
%\label{tab:eval_pre_sid}
%\end{table}

\subsubsection{Post-deployment Stage Assessment}

To validate post-deployment success, from all audio samples that our speaker identification algorithm identifies to contain the voice of the caregiver or the patient, or both, we randomly chose 28 from the first dyad, 28 from the second dyad, and 28 from the third dyad, 100 from the fourth dyad, 80 from the fifth dyad, and 100 from the sixth dyad. The results are reported in Table \ref{tab:eval_sid}. In the following sentences we describe how we obtain the f1 scores in Table \ref{tab:eval_sid}. For a sample, if it only contains the voice of the caregiver, then it is labeled as belonging to the caregiver; it if only contains the voice of the patient, then it is labeled as belonging to the patient. If it contains voices from both the caregiver and patient, then it is labeled as belonging to both. Otherwise, it labeled as belonging to neither. With this labeling scheme, we obtain the positives and negatives of the caregiver's voice and the positives and negatives of the patient's voice. The SID model can label a sample as belonging to the caregiver, belonging to the patient, or neither. As a result, we obtain the results in Table \ref{tab:eval_sid} in which we report the f1 scores to measure the performance of our SID model for both the caregiver and patient of each dyad. The SID model achieves an f1 score in the range of 93.1\% to 97.4\% for the caregivers and 91.6\% to 98.3\% on the patients in the six dyads. The high performance in Table \ref{tab:eval_sid} indicates that our SID algorithm is effective at picking out the voices by the caregiver and the patient in each home in their real home environment. Given that the SID algorithm was specifically assessed to see if it could overcome the challenges (reverberation, deamplification, and background noise) present in the real, designated environment (homes), we have shown that for an algorithm to be successful in the real, designated environment, it must be able to overcome the challenges present in the real, designated environment during the pre-deployment stage. The high performance of the SID during the post-deployment time suggests that our way to perform comprehensive and realistic pre-deployment is effective at improving post-deployment SID success. Note that we only validate the SID solution on the voices of the caregiver and patient of each dyad, because at post-deployment time, the SID solution filtered out voice samples that belonged to neither. As a result, we only have samples that are labelled by the SID solution as either the caregiver or the patient. For samples that made through the SID solution, we have the performance reported in Table \ref{tab:eval_sid}.

In Table \ref{tab:eval_sid_other} we report the f1 score of a model \cite{lecun1995convolutional} that we did not use because at pre-deployment time it achieves bad performance (an f1 score of 79.3\% on P1 and an f1 score of 71.2\% on P2). As we can see, this model also achieves bad performance on the post-deployment data. This indicates that at pre-deployment time, the model that performs badly also performs badly at post-deployment time.

\begin{table}[]
\centering
\scalebox{0.9}{
\begin{tabular}{ccccccc}
\toprule
          & Dyad 1  & Dyad 2  & Dyad 3 & Dyad 4 & Dyad 5 & Dyad 6\\
\hline
Caregiver & 94.5\%  & 97.4\%    &  95.7\%   & 93.1\%    & 92.0\%    & 89.2\% \\
Patient &   94.6\%  & 95.9\%    &  96.0\%   & 94.6\%    & 91.6\%    & 98.3\% \\
\bottomrule
\\
\end{tabular}}
\caption{The evaluation results for the speaker identification model on the dyads. The results are the f1 scores.}
\label{tab:eval_sid}
\end{table}

\begin{table}[]
\centering
\scalebox{0.9}{
\begin{tabular}{ccccccc}
\toprule
          & Dyad 1  & Dyad 2  & Dyad 3 & Dyad 4 & Dyad 5 & Dyad 6\\
\hline
Caregiver &   57.1\% & 71.4\%    & 83.6\%   & 52.6\%    & 44.3\%    & -\\
Patient &   74.8\%  & 75.9\%    &  79.5\%   & 77.7\%    & 74.2\%    & 97.0\%\\
\bottomrule
\\
\end{tabular}}
\caption{The post-deployment evaluation results for a speaker identification model that \textbf{we did not use} because it achieved bad performance pre-deployment time. As we can observe, its performance on all dyads is bad at post-deployment time.}
\label{tab:eval_sid_other}
\end{table}
\subsection{The Emotion Detection Models}

The emotion detection model detects the emotion of the speaker in a given audio clip. During the pre-deployment study, we looked into the state-of-the-art solutions for emotion detection. Table \ref{tab:sota_emotion_detection} shows the performance of several existing state-of-the-art solutions using various datasets of emotional utterances. In this Section, we aim to test the emotion detection model: is it possible to perform comprehensive and realistic pre-deployment testing to improve post-deployment success?

\begin{table}
  \begin{tabular}{ccccc}
    \toprule
    Work & Dataset(s)  & Accuracy\\
    \midrule

    Beard et al. \cite{beard2018multi} & SAVEE, CREMA-D &  41.2\% \\
    VGG \cite{simonyan2014very} & EMO-DB & 43.0\% \\
    Huang et al. \cite{huang2018stochastic} & RAVDESS, SAVEE &  60.8\% \\
    Ghaleb et al. \cite{ghaleb2019multimodal} & RAVDESS & 67.7\%  \\
    Ghaleb et al. \cite{ghaleb2019multimodal} & CREMA-D & 74.0\% \\
    SpeechBrain \cite{ravanelli2021speechbrain} & IEMOCAP & 78.7\%\\ 
    
  \bottomrule
\end{tabular}
\caption{State-of-the-art emotion detection algorithms on different emotional speech datasets. This Table is adapted from a table in the work \cite{gao2021emotion}. As we can see, the best performing state-of-the-art algorithm's accuracy is capped at 80\%. Since SpeechBrain achieves the highest performance, we select it as our emotion detection model.}
\label{tab:sota_emotion_detection}
\end{table}

\subsubsection{Pre-deployment Stage Assessment}
At first glance, Table \ref{tab:sota_emotion_detection} suggests that 4 solutions are not viable, but that SpeechBrain \cite{ravanelli2021speechbrain} is the best candidate among all the state-of-the-art algorithms, given its high performance on the dataset IEMOCAP \cite{busso2008iemocap}. But is it capable of overcoming the challenges (reverberation, deamplification, and background noise) that are present in the real, designated environment (a dyad's home)? To answer that question, we need to look into the dataset IEMOCAP \cite{busso2008iemocap}, on which it is evaluated. If the dataset has accounted for the challenges, i.e. during the data collection and processing process, the audio samples are touched by the effects of the three challenges, then we conclude that the evaluation result yielded by SpeechBrain indicates that it had overcome the three challenges perceived to be present in the real, designated environment that is a dyad's home. The dataset IEMOCAP \cite{busso2008iemocap} indicates that the audio clips are collected when there are furniture items in the room, instead of an empty acoustic studio, which suggests that the audio clips in IEMOCAP \cite{busso2008iemocap} are touched by the effect of reverberation. The audio clips in IEMOCAP \cite{busso2008iemocap} are \emph{not} collected where a speaker is right next to the microphone. This suggests that the audio clips in IEMOCAP \cite{busso2008iemocap} are touched by the effect of deamplification. Last but not least, IEMOCAP is not collected in a studio environment and there was no indication that the indoor background noise events such as footsteps were deliberately removed. Therefore, this suggests that the audio clips in IEMOCAP are touched by the effect of background noise. Consequently, we conclude that IEMOCAP's audio samples on which SpeechBrain is evaluated on account for the challenges that are perceived to be present in the real, designated environment in which the emotion detection algorithm (SpeechBrain) is to be deployed. Here, we set the stage to prove (once again) the hypothesis that for an algorithm to work well in the real, designated environment in which it is envisioned to be deployed, during pre-deployment stage, it must show that it is capable of handling the challenges that are perceived to arise in the real, designated environment. In the meantime, we have observed that algorithms such as Huang et al. \cite{huang2018stochastic} are not likely to work sufficiently at post-deployment time. It is good to discover that these solutions are not likely to work at post-deployment time, because this helps us filter out such existing solutions. 

Note that none of the state-of-the-art approaches in Table \ref{tab:sota_emotion_detection} include TV sounds as one of the acoustical realisms that they need to address. In the future, we plan to develop an emotion detection algorithm that takes TV sounds into consideration.



\subsubsection{Post-deployment Stage Assessment}

In the following paragraphs we describe how we evaluate our emotion detection detection model post-deployment. Out of the audio clips we collected from each dyad, we first select all samples that are classified by the emotion detection model and conflict detection model as anger speech. Then, we randomly select the same number of audio clips from all the samples by that dyad that are not classified as anger speech. Each of the audio clips is \textbf{manually labeled} based on the emotion in the clip by the labelers. Table \ref{tab:eval_emotion} describes our emotion detection model's performance on the labeled samples: For the 1st dyad, there are 233 samples. For the 2nd dyads, there are 392 samples. For the 3rd dyads, there are 281 samples. For the 4th and 6th dyads, there are 100 samples each. For the 5th dyads, there are 80 samples.

During the post-deployment stage assessment, researchers should let third-party labellers label the data to obtain ground truth, instead of letting the participants do the labelling, because the participants are not necessarily good at discerning their own emotion (and if they are in a verbal conflict) if they are not trained. The XYZ system detects if a person is in a verbal conflict or is angry. Initially, we sought to validate our system's performance by survey questions using EMA, similar to many other studies. However, we quickly found out that their responses did not always agree with the decision of the system. Who was wrong: the caregiver or our machine learning solutions? There is existing literature \cite{goerlich2018multifaceted} stating that people not trained to recognize their emotions are often bad at recognizing their own emotions. To investigate, we employed 5 labellers who are approved by the IRB to listen to and label the saved clips of the participants' voices. Their labelling suggests that in some cases, the labellers annotation did not agree with the participants' self reported emotional states. This data supports the claim that people are often bad at recognizing their emotions. We were able to verify the claim only because we planed, during pre-deployment time, to save all the raw data during the full deployment time. We found that post deployment labeling is better than EMA surveys and also supports determination of ground truth which, in turn, provides a better accuracy assessment of the acoustic classifiers. Determining ground truth from deployment time data is very important and often not done in many studies. 

The emotion detection model achieves an f1 score of 85.3\% to 97.4\% on the six dyads. According to Table \ref{tab:eval_emotion}, the emotion detection algorithm (SpeechBrain)'s performances in all six homes are satisfactory, highly efficient at identifying the emotions in each clip in each of the six homes. The success of the emotion detection algorithm demonstrated by Table \ref{tab:eval_emotion} proves our hypothesis: for an algorithm to be able to work satisfactorily post-deployment in a real, designated environment, it must demonstrate that it is able to overcome the challenges perceived to arise in that environment during pre-deployment time. The high performance of the emotion detection model at post-deployment time suggests that our way to perform comprehensive and realistic pre-deployment testing is effective at improving post-deployment success. Table \ref{tab:eval_emotion} also includes the performance of VGG \cite{simonyan2014very} which yields bad results (an accuracy score of 43.0\%) at pre-deployment time. As we can see, the model that achieves bad performance at pre-deployment time also achieves bad performance (an average of an f1 score of 50.1\%).

\begin{table}[]
\centering
\scalebox{0.85}{
\begin{tabular}{ccccccc}
\toprule
          & Dyad 1  & Dyad 2  & Dyad 3  & Dyad 4 & Dyad 5 & Dyad 6\\
\hline
SpeechBrain       & 88.8\%  & 87.2\%   & 91.8\%    & 92.9\% & 85.3\%  & 97.4\%   \\
VGG         & 44.6\%    & 65.4\%     & 48.3 \%  & 45.3\% & 53.3\% & 61.5\%  \\
\bottomrule
\\
\end{tabular}
}
\caption{The post-deployment evaluation results for the emotion detection model (SpeechBrain) as well as the VGG model \cite{simonyan2014very}, \textbf{which we did not use} because at pre-deployment time it achieves bad performance with an accuracy score of 43.0\%. As we can see, it also achieves bad performance on the post-deployment data. The measurement in this Table is f1 score.}
\label{tab:eval_emotion}
\end{table}



\subsection{The Conflict Detection Model}

For conflict detection, currently, there is no available conflict detection algorithm that is acoustics-based. Therefore, we developed our own conflict detection algorithm. In this Section, we aim to test on the conflict detection model: is it possible to perform comprehensive and realistic pre-deployment testing to improve post-deployment success?

\subsubsection{Pre-deployment Stage Assessment}
%The conflict detection algorithm is developed by us. We develop the conflict detection algorithm because there is no available off-the-shelf algorithms that use voice to detect verbal conflict. 
Here we briefly describe how the new algorithm we developed is trained and why the training process makes it specifically account for the (three) challenges that arise in the real, designated environment in which the algorithm is going to be deployed. The training and testing samples are from 19 couples and each sample is labeled conflict if the content of the sample indicates that the couple are in a verbal conflict. It is labeled non-conflict if the couple are not in a verbal conflict. Since the samples are already collected from home-environments, de-amplification and reverberation are accounted for, but the samples are free of background noise. As a result, we mix each of the samples with background noise by randomly selecting a segment from a randomly chosen indoor background noise sample in Table \ref{tab:homenoise} and overlaying that segment with each sample. Out of the samples, there are 3,072 in the training set and 1,009 in the testing set. As a result, both the training and the testing set accounts for a variety range of indoor environmental distortions. Since the training samples are touched by deamplification, reverberation, and background noise, our conflict detection algorithm trained on them is designed to be able to handle the three challenges (deamplification, reverberation, and background noise).

The conflict detection model's performance on the testing set achieves an f1 score of 93.1\%. The high performance suggests that the conflict detection is robust against environmental distortions such as reverberation, background noise, and deamplification. This help us set the stage to prove our hypothesis that, for an algorithm to work sufficiently in the real, designated environment in which challenges are perceived to be present, the algorithm must show that, during pre-deployment stage, it is able to handle the challenges. Our conflict detection algorithm has indicated that during pre-deployment stage, it is able to handle the three challenges: reverberation, deamplification, and background noise. Note that we do not include TV sounds as one of the acoustic realisms that the conflict detection model needs to address. In the future, we plan to develop a conflict detection algorithm that takes TV sounds into consideration.

%\begin{table}[]
%\centering
%\begin{tabular}{cc}
%\toprule
%          & F1 \\
%\hline
%Conflict Detection & 93.1\%   \\
%\bottomrule
%\\
%\end{tabular}
%\caption{The conflict detection model's performance on the testing set we created that contains 1,009 %samples that contained background noise, reverberation effect, and deamplification effect.}
%\label{tab:eval_pre_conflict}
%\end{table}

\begin{table}[]
\centering
\begin{tabular}{ccccccc}
\toprule
          & Dyad 1  & Dyad 2  & Dyad 3 & Dyad 4 & Dyad 5 & Dyad 6\\
\hline
Ours       & 63.4\%    & 65.9\%    & 70.7\%  & 86.2\% & 82.7\% & 90.1\%\\
\bottomrule
\\
\end{tabular}
\caption{The evaluation results for the conflict detection model. The measurement is f1 score.}
\label{tab:eval_conflict}
\end{table}

\subsubsection{Post-deployment Stage Assessment}
In the post-deployment time, we seek to prove our hypothesis that, for an algorithm to work well post-deployment time in the real, designated environment in which it is going to be deployed, during pre-deployment stage it must show that it is capable of overcoming the challenges that are present in the real, designated environment. Our conflict detection algorithm has showed that it is capable of overcoming the challenges (it achieves an f1 score of 93.1\% pre-deployment time). However, is it going to work well in the post-deployment time?

Table \ref{tab:eval_conflict} shows our conflict detection algorithm's performance during the post-deployment time at the six homes. Now we explain how we obtain the f1 score results in Table \ref{tab:eval_conflict}. If a clip is labeled by the labelers such that it contains verbal conflict and the classifier also thinks this clip contains verbal conflict, then it is a hit. If the clip is labeled by the labelers as not containing verbal conflict and the classifier also thinks that it does not contain verbal conflict, then it is a hit. All other cases are misses (for example, the labelers think that a sample contain verbal conflict but the classifier fails to classify it as so). By looping through all samples produced by a dyad, we produce an f1 score on that dyad. From Table \ref{tab:eval_conflict}, we observe that the sixth dyad achieves the best performance with an f1 score of 90.1\% while the first dyad achieves the lowest performance with an f1 score of 63.4\%. For each of the dyads, we observe a drop in performance compared to 93.1\% obtained when the same model is evaluated on the dataset containing speech samples from the 19 couples. This indicates that despite our effort in mitigating environmental distortions, the effects of the environmental distortions such as room reverberation, background noise, and the deamplification effect are not fully mitigated. But the relatively satisfactory performance of the conflict detection model on dyads 4, 5 and 6 indicates that our way to perform comprehensive and realistic pre-deployment testing to improve post-deployment success is effective for exapected conditions.

We also investigate why the performance of the conflict detection model is lower in dyads 1-3 (f1 score of 63.4\% to 70.7\%). Upon communicating with the dyads, we learned that dyad 1 moved the system (which included the microphone) to the hallway which is very far away from the usual places that the participants were speaking. Dyad 2 had a construction team rennovating their home, so there was a lot of construction noise to confuse the conflict detection model. When we developed the conflict detection model, we did not take construction noises into consideration. In dyad 3, the caregiver's voice was always very low, almost inaudible, and our conflict detection model was not designed to handle such low-to-inaudible voice samples.

\subsection{Summary}

In this Section we briefly summarize our findings in Section \ref{sec:evaluation}. 

The first finding is that, to ensure post-deployment success of an algorithm, during pre-deployment time it must be rigorously tested on samples that are touched by the challenges that are perceived to be present in the real, designated environment during post-deployment time. This finding is confirmed by the pre-deployment stage assessment results and post-deployment stage assessment results of the VAD model, the SID model, the emotion detection model, and the conflict detection model. 
%In other words, during the pre-deployment stage, each of the models is tested on samples that are touched by the effects of reverberation, background noise, and deamplification (the challenges perceived to be present in the real, designated environment). 
%During the post-deployment stage, each of the models performs satisfactorily in the real, designated environment which are the homes of the dyads. The high performances of our VAD, SID, emotion detection, and conflict detection models suggest that it is possible to perform comprehensive and realistic pre-deployment testing to improve post-deployment success.

The second finding is that in the acoustic processing pipeline we should always go for off-the-shelf solutions first before developing your own algorithm. Many acoustic functions have been under study for many years and excellent solutions exist. Yet, rigorous testing is still required since not all of the available solutions will work for the environment where the system will be deployed. In our case, the VAD, SID, and emotion detection models are off-the-shelf. Since there are no state-of-the-art conflict detection models that \emph{only use (the prosody of the) voice} to detect verbal conflict, this has to be developed and will not have the luxury of having solutions refined many many researchers over many years. This likely is why the performance is lower than the well developed acoustic functions. 

%Our third finding, following the second finding, is that researchers should make sure the thoroughly test the off-the-shelf solutions. We tested off-the-shelf solutions for VAD (in Table \ref{tab:eval_pre_vad}), SID (in Table \ref{tab:eval_pre_sid}), and emotion detection (in Table \ref{tab:sota_emotion_detection}) during the pre-deployment stage, using testing samples that are touched by deamplification, background noise, and reverberation (three challenges perceived to be present in the real, designated environment).

%Our fourth finding is that during post-deployment stage assessment, researchers should let third-party labellers to label the data to obtain ground truth, instead of letting the participants (the caregivers of the dyads) to do the labelling, because the participants are not necessarily good at discerning their emotion (and if they are in a verbal conflict) if they are not trained. The XYZ system detects if a person is in a verbal conflict or is angry. Initially, we sought to validate our system's performance by survey questions using EMA, similar to many other studies. However, we quickly found out that their responses did not always agree with the decision of the system. Who was wrong: the caregiver or our machine learning solutions? There is existing literature \cite{goerlich2018multifaceted} stating that people not trained to recognize their emotions are notoriously bad at recognizing their own emotions. To investigate, we employed 5 labellers who are approved by the IRB to listen to and label the saved clips of the participants' voices. Their labelling suggests that in some cases, the labellers annotation did not agree with the participants' self reported emotional states. This data supports the claim that people are often bad at recognizing their emotions. We were able to verify the claim only because we planed, during pre-deployment time, to save all the raw data during the full deployment time. We found that post deployment labeling is better than EMA surveys and also supports determination of ground truth which, in turn, provides a better accuracy assessment of the acoustic classifiers. Determining ground truth on deployment time data is very important and often not done in many studies. 