% ****** Start of file apssamp.tex ******
%
%   This file is part of the APS files in the REVTeX 4.2 distribution.
%   Version 4.2a of REVTeX, December 2014
%
%   Copyright (c) 2014 The American Physical Society.
%
%   See the REVTeX 4 README file for restrictions and more information.
%
% TeX'ing this file requires that you have AMS-LaTeX 2.0 installed
% as well as the rest of the prerequisites for REVTeX 4.2
%
% See the REVTeX 4 README file
% It also requires running BibTeX. The commands are as follows:
%
%  1)  latex apssamp.tex
%  2)  bibtex apssamp
%  3)  latex apssamp.tex
%  4)  latex apssamp.tex
%
\documentclass[%
 reprint,
%superscriptaddress,
%groupedaddress,
%unsortedaddress,
%runinaddress,
%frontmatterverbose, 
%preprint,
%preprintnumbers,
%nofootinbib,
%nobibnotes,
%bibnotes,
 amsmath,amssymb,
 aps,
pre,
%prb,
%rmp,
%prstab,
%prstper,
floatfix,
]{revtex4-2}

\usepackage{graphicx}% Include figure files
\usepackage{dcolumn}% Align table columns on decimal point
\usepackage{bm}% bold math
\usepackage{hyperref}% add hypertext capabilities
%\usepackage[mathlines]{lineno}% Enable numbering of text and display math
%\linenumbers\relax % Commence numbering lines

%\usepackage[showframe,%Uncomment any one of the following lines to test 
%%scale=0.7, marginratio={1:1, 2:3}, ignoreall,% default settings
%%text={7in,10in},centering,
%%margin=1.5in,
%%total={6.5in,8.75in}, top=1.2in, left=0.9in, includefoot,
%%height=10in,a5paper,hmargin={3cm,0.8in},
%]{geometry}

\begin{document}

%\preprint{APS/123-QED}

\title{ Three-particle model for the nonlinear response of frictional granular materials }% Force line breaks with \\

\author{Michio Otsuki}
\email[]{otsuki@me.es.osaka-u.ac.jp}
\affiliation{
  Graduate School of Engineering Science, Osaka University, Toyonaka, Osaka 560-8531, Japan}

\author{Hisao Hayakawa}
\affiliation{Yukawa Institute for Theoretical Physics, Kyoto University, Kitashirakawaoiwake-cho, Sakyo-ku, Kyoto 606-8502, Japan}

\date{\today}% It is always \today, today,
             %  but any date may be explicitly specified
\begin{abstract}
We propose a simple model comprising three particles to study the nonlinear mechanical response of jammed frictional granular materials under oscillatory shear.
Owing to the introduction of the simple model, we obtain an exact analytical expression of the complex shear modulus for a system including many monodispersed disks, which satisfies a scaling law in the vicinity of the jamming point.
These expressions perfectly reproduce the shear modulus of the many-body system with low strain amplitudes and friction coefficients.
Even for disordered many-body systems, the model reproduces results by introducing a single fitting parameter.
\end{abstract}


                              %display desired
\maketitle


{\it Introduction---}
The rheological property of densely dispersed grains, e.g., granular materials, colloidal suspensions, and emulsions, plays an important role in physics and engineering. 
This rheological property mainly depends on the packing fraction $\phi$ of the grains.
The materials behave like fluids for $\phi<\phi_{\rm J}$ with jamming fraction $\phi_{\rm J}$ and exhibit a solid-like elastic response above $\phi_{\rm J} $\cite{Hecke,Behringer}.
In the linear response regime (i.e., for small strains), the shear modulus is characterized by the density of states \cite{Maloney2004,Maloney2006,Ishima22} and satisfies scaling laws \cite{OHern02,Wyart,Tighe11,Otsuki17}.
However, the linear response region becomes narrower as $\phi$ approaches $\phi_{\rm J}$ \cite{Coulais,Otsuki14}, and the nonlinear response becomes relevant \cite{Nagamanasa,Knowlton,Kawasaki16,Leishangthem,Clark,Boschan19,Boschan,Nakayama,Kawasaki20}.
Under quasistatic oscillatory shear, the storage modulus $G'$ decreases as the strain amplitude $\gamma_0$ increases \cite{Otsuki14,Bohy,Ishima,Otsuki21,Otsuki22}. 
The loss modulus $G''$ remains nonzero in the same region under the quasistatic shear \cite{Otsuki21,Otsuki22}.



The theoretical analysis of densely dispersed grains is challenging as a typical many-body problem in non-equilibrium systems.
To date, a few theoretical approaches have been proposed for systems related to frictionless particles.
The scaling laws for the linear elastic response were derived in terms of the vibrational density of states \cite{Wyart,Tighe11}.
The Fourier analysis of particle trajectories helps generate semi-analytical expressions for $G'$ and $G''$ \cite{Otsuki22}.
Unfortunately, these theories can not apply to frictional particles because of the history-dependent contact force \cite{Otsuki17,Otsuki21}. 


It is helpful to analyze a simple model with small degrees of freedom to understand the behavior of many-body systems, including densely dispersed grains.
This approach has been used in statistical mechanics.
The mean-field approximation of the Ising model is a typical example in which the system contains only one Ising spin under the influence of a self-consistently determined mean field \cite{Goldenfeld}.
For atomic liquids, a cell model, in which a single atom exists in a cage, was used to derive the equation of state \cite{Lennard37,Lennard38}. 
The coherent potential approximation for disordered solids has been used to understand electronic band structures \cite{Yonezawa}.
The effective medium theory reveals the elastic response of random spring networks \cite{Feng}.
In addition, a simple model consisting of two particles was proposed to reproduce the liquid-solid phase transition \cite{Awazu}.
The advantage of such few-body models is that we can obtain exact solutions. 
The qualitative behavior of the corresponding many-body systems can be determined based on the solutions of the few-body models.
Thus, we adopt this approach to determine the nonlinear responses of the frictional dispersed grains.



This study proposes a model consisting of three identical particles to describe the mechanical response of jammed frictional granular materials under oscillatory shear.
We demonstrate that the model reproduces the storage and loss moduli of many-particle systems (MPSs) without any fitting parameter if there is no disorder in the particle configuration.
This model can be analytically solved for low-strain amplitudes and friction coefficients near the jamming point.
%The solution is consistent with the result of the MPS consisting of identical particles initially placed on the triangular lattice.
We derive a scaling law for the complex shear modulus, which semi-quantitatively agrees with the numerical simulations of the MPS even if disorder exists with the introduction of a fitting parameter.



\begin{figure}[htbp]
\includegraphics[width=0.8\linewidth]{confMPS.eps}
  \caption{
    (a) Schematic of the ordered MPS.
    (b) Schematic of the disordered MPS.
}
\label{confMPS}
\end{figure}


\begin{figure}[htbp]
\includegraphics[width=0.6\linewidth]{confTPM.eps}
  \caption{
    Schematic of the three-particle model.
}
\label{Fig1}
\end{figure}



{\it Three-Particle Model--- }
We consider two-dimensional granular materials consisting of many frictional particles under oscillatory shear (Fig. \ref{confMPS}).
Moreover, we introduce a system of three identical particles to simply describe the MPS (Fig. \ref{Fig1}).
The MPS can contain polydisperse grains, while we assume that the three-particle model (TPM) is a monodisperse system.
In the TPM, the position $\boldsymbol r_i(t) = (x_i(t),y_i(t))$ of particle $i$ with diameter $d$ at time $t$ is given by
\begin{eqnarray}
\label{TE1}
  \boldsymbol r_1(t) & = & \left ( \frac{\sqrt{3} \gamma(\theta(t)) \ell}{4}, \frac{\sqrt{3} \ell} {4} \right), \\
\label{TE2}
  \boldsymbol r_{2}(t) & = & \left ( - \frac{\sqrt{3} \gamma(\theta(t)) \ell }{4} - \frac{ \ell}{ 2}, - \frac{\sqrt{3} \ell}{4} \right), \\
\label{TE3}
  \boldsymbol r_{3}(t) & = & \left ( - \frac{\sqrt{3} \gamma(\theta(t)) \ell }{4} + \frac{ \ell}{ 2}, - \frac{\sqrt{3} \ell}{4} \right),
\end{eqnarray}
where $\ell=d(1-\epsilon)$ defines the initial distance between particles with compressive strain $\epsilon \ll 1$ and $d$ denotes the diameter of the particle.
The compressive strain $\epsilon$ in the TPM corresponds to $\phi-\phi_{\rm J}$ in the MPS \cite{Supple}.
We apply shear strain as
\begin{equation}
  \gamma(\theta) = \gamma_0 \sin  \theta
  \label{g}
\end{equation}
with strain amplitude $\gamma_0$, phase $\theta = \omega t$, and angular frequency $\omega$.
Note that we need at least three particles to realize a stable interlocking state.



We adopt the interaction force $\boldsymbol f_{ij}$ between particles $i$ and $j$ given by
\begin{equation}
  \label{F}
  \boldsymbol f_{ij} = \left ( f_{ij}^{\rm (n)} \boldsymbol n_{ij}    + f_{ij}^{\rm (t)}\boldsymbol t_{ij}  \right )H(r_{ij}-d),
\end{equation}
where $f_{ij}^{\rm (n)}$ and $f_{ij}^{\rm (t) }$ denote the normal and tangential forces between the particles $i$ and $j$ \cite{DEM}.
The distance between the particles $i$ and $j$ is $r_{ij} = |\boldsymbol r_{ij}|$ with $\boldsymbol r_{ij} := \boldsymbol r_i - \boldsymbol r_j = (x_{ij},y_{ij})$.
Here, $H(x)$ is Heviside's step function satisfying $H(x)=1$ for $x>0$ and $H(x) = 0$ otherwise.
The normal and tangential unit vectors are denoted by $\boldsymbol n_{ij} := \boldsymbol r_{ij}/r_{ij} = (n_{ij,x}, n_{ij,y})$ and $\boldsymbol t_{ij} := (-n_{ij,y},n_{ij,x})$, respectively.
For simplicity, we do not consider the torque balance and, thus, the rotation of the grains.
The effect of the rotation is discussed in Ref. \cite{Supple}.

The normal force is assumed to be
\begin{equation}
f_{ij}^{\rm (n)} = - k_{\rm n} u^{\rm (n)}_{ij} 
\label{Fn}
\end{equation}
with the normal elastic constant $k_{\rm n}$ and normal relative displacement $u^{\rm (n)}_{ij} := r_{ij} - d$.
Moreover, the tangential force is assumed to be
\begin{equation}
f_{ij}^{\rm (t)} = {\rm min} \left ( |\tilde f_{ij}^{\rm (t)}|, \mu f_{ij}^{\rm (n)} \right ) {\rm sgn} (\tilde f_{ij}^{\rm (t)}),
  \label{Ft}
\end{equation}
where  $\tilde f_{ij}^{\rm (t)} = - k_{\rm t} u_{ij}^{\rm (t)}$; $k_{\rm t}$ denotes the tangential elastic constant, and $\mu$ denotes the friction coefficient.
Here, ${\rm min}(a,b)$ selects the smaller value between $a$ and $b$, ${\rm sgn}(x) = 1$ for $x \ge 0$, and ${\rm sgn}(x) = -1$ for $x < 0$. 
The tangential displacement $u_{ij}^{\rm (t)}$ satisfies $\frac{d}{dt} u_{ij}^{\rm (t)} = v_{ij}^{\rm (t)}$ for $|\tilde f_{ij}^{\rm (t)}| <\mu f_{ij}^{\rm (n,el)}$ with the tangential velocity $v_{ij}^{\rm (t)} = (\frac{d}{dt} {\boldsymbol  r}_i - \frac{d}{dt}{ \boldsymbol r}_j)\cdot \boldsymbol t_{ij}$, whereas $u_{ij}^{\rm (t)}$ remains unchanged for $|\tilde f_{ij}^{\rm (t)}| \ge \mu f_{ij}^{\rm (n,el)}$.
We refer to the contact with $|\tilde f_{ij}^{\rm (t)}| < \mu f_{ij}^{\rm (n)}$ as the stick contact and the contact with $|\tilde f_{ij}^{\rm (t)}| \ge \mu f_{ij}^{\rm (n)}$ as the slip contact.
The tangential displacement, $u_{ij}^{\rm (t)}$, is initially set to zero.



The (symmetric contact) shear stress is given by
\begin{equation}
  \sigma  = \sigma^{\rm (n)} +\sigma^{\rm (t)} 
  \label{s}
\end{equation}
with the normal component of $\sigma$
\begin{equation}
  \sigma^{\rm (n)} = - \frac{1}{A} \sum _i \sum_{j>i} \frac{x_{ij}y_{ij}}{r_{ij}} f^{\rm (n)}_{ij}
  \label{Sn}
\end{equation}
and tangential component of $\sigma$
\begin{equation}
  \sigma^{\rm (t)} = - \frac{1}{2A} \sum _i \sum_{j>i} \frac{x_{ij}^2-y_{ij}^2}{r_{ij}} f^{\rm (t)}_{ij}.
  \label{St}
\end{equation}
Here, $A$ corresponds to the area of the system, and we choose $A = \sqrt{3} \ell^2 /2$ \cite{Supple}.
The pressure is given by
\begin{equation}
  P =  \frac{1}{2A} \sum _i \sum_{j>i} (x_{ij} f_{ij,x}+y_{ij} f_{ij,y}).
  \label{P}
\end{equation}
As we are interested in quasistatic processes, we do not consider the kinetic parts of $\sigma$ and $P$. 
After several cycles of oscillatory shear, $\sigma(\theta)$ becomes periodic.
The storage and loss moduli are given by \cite{Doi}
\begin{eqnarray}
  G' & = & \frac{1}{\pi} \int_0^{2\pi} d \theta \ \sigma(\theta) \sin \theta / \gamma_0, \label{Gp}\\
  G'' & = & \frac{1}{\pi} \int_0^{2\pi} d \theta \ \sigma(\theta) \cos \theta / \gamma_0 \label{Gpp}.
\end{eqnarray}







{\it Theoretical analysis--} 
Assuming $\gamma_0 \ll \epsilon \ll 1$, we analytically obtain $G'$ and $G''$ for the TPM.
The derivation of the analytical results can be found in \cite{Supple}.


First, the normal component of the shear stress is given by \cite{Supple}
\begin{eqnarray}
  \label{sn}
  \sigma^{\rm (n)}(\theta) & = & \frac{\sqrt{3}k_{\rm n} \gamma(\theta)}{4}.
\end{eqnarray}
The tangential component of the shear stress is given by
\begin{eqnarray}
  \label{st1}
  \sigma^{\rm (t)}(\theta) & = & \frac{\sqrt{3}k_{\rm t} \gamma(\theta)}{4}
\end{eqnarray}
for $\gamma_0 < \gamma_c(\mu)$ with a critical amplitude
\begin{equation}
  \gamma_c(\mu) = \frac{4 \mu  k_{\rm n} \epsilon}{ 3 k_{\rm t}},
  \label{gc}
\end{equation}
which characterizes the transition from stick-to-slip states in the contact between the particles.
For $\gamma_0 \ge \gamma_c(\mu)$, the tangential component of the shear stress is given by
\begin{align}
  \label{st2}
  &\sigma^{\rm (t)}(\theta)  =  
   \left\{
\begin{array}{ll}
  \frac{\mu k_{\rm n} \epsilon}{\sqrt{3}} , & 0 \le \theta < \frac{\pi}{2} \\
  \frac{\mu k_{\rm n} \epsilon}{\sqrt{3}}  + \frac{\sqrt{3}{\rm t} (\gamma(\theta) - \gamma_0)}{4}, & \frac{\pi}{2} \le \theta < \frac{\pi}{2} + \Theta \\
  -\frac{\mu k_{\rm n} \epsilon}{\sqrt{3}} , & \frac{\pi}{2} + \Theta \le \theta <  \frac{3\pi}{2}\\
  -\frac{\mu k_{\rm n} \epsilon}{\sqrt{3}} + \frac{\sqrt{3}{\rm t} (\gamma(\theta) + \gamma_0)}{4}, & \frac{3\pi}{2} \le \theta <  \frac{3\pi}{2} + \Theta\\
  \frac{\mu k_{\rm n} \epsilon}{\sqrt{3}} , & \frac{3\pi}{2} + \Theta \le \theta <  2\pi,
\end{array}
\right.
\end{align}
where $\Theta = \cos^{-1} \left ( 1 - 2 \gamma_c(\mu) / \gamma_0 \right )$ \cite{Supple}.
Regions with $\frac{\pi}{2} \le \theta < \frac{\pi}{2} + \Theta$ and $\frac{3\pi}{2} \le \theta <  \frac{3\pi}{2} + \Theta$ correspond to the stick state, and the other regions correspond to the slip state.
Owing to this transition in the contact, the stress--strain curve given by Eqs. \eqref{sn}--\eqref{st2} exhibits a hysteresis loop, which is related to the dependence of the shear modulus on $\gamma_0$, as explained in Ref. \cite{Supple}.






Substituting Eqs. \eqref{s} and \eqref{sn}-\eqref{st2} into Eq. \eqref{Gp}, we obtain the storage modulus as
\begin{align}
  \label{GpT}
  & G'  =  
   \left\{
\begin{array}{ll}
  \frac{\sqrt{3}\left( k_{\rm n} + k_{\rm t}\right )}{ 4}, & \gamma_0 \le \gamma_c(\mu) \\
  \frac{\sqrt{3}}{ 4}\left\{ k_{\rm n} + \frac{k_{\rm t}}{\pi} \left ( \Theta - \sin \Theta \cos \Theta \right ) \right \}, & \gamma_0 > \gamma_c(\mu)
\end{array}
\right. 
\end{align}
As $\gamma_0$ increases beyond $\gamma_c(\mu)$, $G'$ decreases from a higher value to a lower value.
The corresponding behavior has been observed in the MPS in previous studies \cite{Otsuki17,Otsuki21}.



Substituting Eqs. \eqref{s} and \eqref{sn}-\eqref{st2} into Eq. \eqref{Gpp}, the loss modulus is given by
\begin{align}
  \label{GppT}
  & G''  =  
   \left\{
\begin{array}{ll}
  0, & \gamma_0 \le \gamma_c(\mu) \\
  \frac{\sqrt{3}k_{\rm t}}{ 4  \pi}\left ( 1 -  \cos^2  \Theta \right ), & \gamma_0 > \gamma_c(\mu) .
\end{array}
\right.
\end{align}
The loss modulus $G''$ is zero for $\gamma_0 < \gamma_c(\mu)$, whereas $G''$ sharply increases with $\gamma_0$ when $\gamma_0$ exceeds $\gamma_c(\mu)$ and decreases to $0$ after a peak.
The behavior of $G''$ for the TPM qualitatively reproduces that of the MPS in previous studies \cite{Otsuki21}.






The pressure $P_0$ with $\gamma=0$ is also obtained as \cite{Supple}
\begin{equation}
  P_0 = \sqrt{3}k_{\rm n} \epsilon.
  \label{PT}
\end{equation}
Equations \eqref{gc}, \eqref{GpT}, \eqref{GppT}, and \eqref{PT} satisfy scaling laws for a given $\epsilon$ as
\begin{align}
  \label{Gp:scale}
  & G'(\mu,\gamma_0) = G'_{\rm M}(\mu) \mathcal{G}'\left( \frac{k_{\rm t}\gamma_0}{  \mu  P_0(\mu)} \right ), \\
  \label{Gpp:scale}
  & G''(\mu,\gamma_0) = G''_{\rm M}(\mu) \mathcal{G}''\left( \frac{k_{\rm t} \gamma_0}{ \mu  P_0(\mu)} \right ),
\end{align}
where $\mathcal{G}'(x)$ and $\mathcal{G}''(x)$ denote scaling functions.
The maximum values of $G'$ and $G''$ are denoted as $G'_{\rm M}$ and $G''_{\rm M}$, respectively.
In the TPM, they are obtained as
\begin{align}
& G'_{\rm M} = \sqrt{3}\left( k_{\rm n} + k_{\rm t}\right )/4, \ \ 
 G''_{\rm M} = \sqrt{3}k_{\rm t}/(4 \pi), \\
  \label{GpS}
  & \mathcal{G}'(x)  =
   \left\{
\begin{array}{ll}
  1, & x \le x_c, \\
  \left ( 1 + \frac{k_{\rm t}}{k_{\rm n}}\frac{T(x) - S(x)}{\pi}  \right )/
  \left (1+\frac{k_{\rm t}}{k_{\rm n}} \right ), & x > x_c,
\end{array}
\right. \\
  \label{GppS}
  & \mathcal{G}''(x)  = 
   \left\{
\begin{array}{ll}
  0, & x \le x_c, \\
  1 -  \cos^2  T(x) , & x >x_c
\end{array}
\right.
\end{align}
with $T(x) = \cos^{-1}(1-2 x_c / x)$, $S(x) = \sin (2T(x))/2$, and $x_c = 4 /(3 \sqrt{3})$.






{\it Comparison with the MPS---} 
We demonstrate the relevance of the TPM analysis based on the simulation of a two-dimensional MPS consisting of $N$ frictional particles.
First, we consider a system corresponding to the TPM, where all the particles are identical and initially placed on the triangular lattice with a unit length $\ell$ (Fig. \ref{confMPS}(a)).
Next, we consider a bidisperse system where the number of particles with diameter $d$ is equal to that of particles with diameter $d/1.4$, and the particles are randomly placed with packing fraction $\phi$ (Fig. \ref{confMPS}(b)).
In both systems, the shear strain given by Eq. \eqref{g} is applied for $N_{\rm c}$ cycles using the SLLOD equation under the Lees--Edwards boundary condition \cite{Evans}.
In addition to the interactions defined in Eqs. \eqref{Fn} and \eqref{Ft}, we adopt the dissipative force characterized by normal and tangential viscous constants $\eta_{\rm n}$ and $\eta_{\rm t}$.
The systems are detailed in Ref. \cite{Supple}.
We measure $G'$, $G''$, and $P_0$ in the last cycle using Eq. \eqref{P}-\eqref{Gpp}.
The mass densities of the particles are identical.
For the ordered MPS shown in Fig. \ref{confMPS}(a), we use $N=64$, $k_{\rm t}/k_{\rm n}=1.0$, and $\epsilon = 0.001$, whereas $N=1000$, $k_{\rm t}/k_{\rm n}=0.2$, and $\phi=0.87$ are used for the disordered MPS shown in Fig. \ref{confMPS}(b).
In both systems, the other parameters are identical: $N_{\rm c}=20$, $\eta_{\rm t} = \eta_{\rm n} = \sqrt{mk_{\rm n}}$, and $\omega = 0.0001 \sqrt{m/k_{\rm n}}$ with a mass $m$ of larger particles.



\begin{figure}[htbp]
\includegraphics[width=0.7\linewidth]{Gp_3P.eps}
  \caption{
    Storage modulus $G'$ against $\gamma_0$ with $k_{\rm t} / k_{\rm n} = 1.0$ and $\epsilon = 0.001$ for various values of $\mu$.
    The points represent the results of the ordered MPS.
    The thin solid lines represent the analytical result given by Eq. \eqref{GpT}. 
    The vertical dashed lines represent the critical amplitude $\gamma_c(\mu)$ given by Eq. \eqref{gc} for $\mu = 10^{-4}, 10^{-3}, 10^{-2}, 10^{-1}$, and $1$ from left to right.
}
  \label{Gp_3P}
\end{figure}


As shown in Fig. \ref{Gp_3P}, we plot $G'$ for the ordered MPS against $\gamma_0$ with $k_{\rm t}/k_{\rm n} = 1.0$ and $\epsilon = 0.001$ for various values of $\mu$ as points.
Moreover, we plot the analytical results obtained using Eq. \eqref{GpT} as thin solid lines.
Surprisingly, the results of the TPM agree with those of the MPS for $\gamma_0 < 0.003$ without any fitting parameters.
As $\gamma_0$ increases beyond $\gamma_c(\mu)$ shown by the vertical dashed lines, $G'$ for $\mu>0$ decreases and converges to a constant, which is equal to $G'$ for $\mu=0$.
For larger $\gamma_0$, $G'$ for the MPS decreases again, whereas the theoretical $G'$ based on the TPM is constant.
This discrepancy results from the violation of condition $\gamma_0 \ll \epsilon$ for the analytical calculation.
If we numerically obtain $G'$ and $G''$ without the assumption $\gamma_0 \ll \epsilon \ll 1$, the TPM perfectly reproduces the MPS, including the second decrease, as mentioned in Ref. \cite{Supple}.


As shown in Fig. \ref{Gpp_3P}, we plot $G''$ for the MPS on the triangular lattice against $\gamma_0$ with $k_{\rm t}/k_{\rm n} = 1.0$ and $\epsilon = 0.001$ for various values of $\mu$ as points.
Moreover, we plot the analytical results obtained using Eq. \eqref{GppT} as thin solid lines.
The analytical result of the TPM agrees perfectly with that of the MPS for $\gamma_0 < 0.003$ without any fitting parameters.
As $\gamma_0$ increases beyond $\gamma_c(\mu)$ shown by the vertical dashed lines, $G''$ for $\mu>0$ increases from $0$ and decreases after reaching a peak.
The peak position of $G''$ against $\gamma_0$ increases with $\mu$.
Thus, we fail to capture the behavior of $G''$ for $\mu=1$.


\begin{figure}[htbp]
\includegraphics[width=0.7\linewidth]{Gpp_3P.eps}
  \caption{
    Loss modulus $G''$ against $\gamma_0$ with $k_{\rm t} / k_{\rm n} = 1.0$ and $\epsilon = 0.001$ for various values of $\mu$.
    The points represent the results of the ordered MPS.
    The thin solid lines represent the analytical results obtained using Eq. \eqref{GppT}. 
    The vertical dashed lines represent the critical amplitude $\gamma_c(\mu)$ given by Eq. \eqref{gc} for $\mu = 10^{-4}, 10^{-3}, 10^{-2}, 10^{-1}$, and $1$ from left to right.
}
  \label{Gpp_3P}
\end{figure}



Consider the disordered MPS shown in Fig. \ref{confMPS}(b).
The behaviors of $G'$ and $G''$ in this system are similar to those of the TPM \cite{Supple}.
Therefore, it is expected that the scaling laws in Eqs.  \eqref{Gp:scale} and \eqref{Gpp:scale} for a given $\epsilon$ in the TPM can be used even in this system with corresponding $\phi$. 
This expectation is verified by Fig. \ref{YPT10_scale}, in which we plot the scaled moduli $G'/G'_{\rm M}$ and $G''/G''_{\rm M}$ against the scaled strain $k_{\rm t} \gamma_0/ (\mu P_0)$ for various values of $\mu$ in the disordered MPS. Moreover, we plot the analytical results for the TPM obtained using Eqs. \eqref{GpS} and \eqref{GppS} as solid lines, which qualitatively agree with the MPS results.
Here, we chose $k_{\rm t}/k_{\rm n}=1.5$ for the TPM to fit the second plateau to that of the MPS.
At present, we do not know the relationship between $\phi$ and the fitting parameter.
%Although the scaling plot for the TPM exhibits a clear transition at $\gamma_c(\mu)$, the decrease of $G'/G'_{\rm m}$ for the MPS is a crossover due to disorder effect \cite{Supple}.


\begin{figure}[htbp]
\includegraphics[width=1.0\linewidth]{YPT10_scale.eps}
  \caption{
(a) Scaled storage modulus $G'/G'_{\rm M}$ against the scaled strain $k_{\rm t} \gamma_0/ (\mu P_0)$ with $\phi=0.87$ and $k_{\rm t}/k_{\rm n}=0.2$ for various values of $\mu$ in the disordered MPS.
The solid line represents the analytical result of the TPM given by Eq. \eqref{GpS} with $k_{\rm t}/k_{\rm n}=1.5$.
(b) Scaled loss modulus $G''/G''_{\rm M}$ against the scaled strain $k_{\rm t} \gamma_0/ (\mu P_0)$ with $\phi=0.87$ and $k_{\rm t}/k_{\rm n}=0.2$ for various values of $\mu$ in the disordered MPS.
The solid line represents the analytical result of the TPM given by Eq. \eqref{GppS} with $k_{\rm t}/k_{\rm n}=1.5$.
}
  \label{YPT10_scale}
\end{figure}





{\it Conclusions-- }
We proposed a TPM for the mechanical response of jammed frictional granular materials under oscillatory shear.
We analytically obtained $G'$ and $G''$, which led to the derivation of scaling laws given by Eqs. \eqref{Gp:scale} and \eqref{Gpp:scale}.
The results of the TPM agreed with those of the ordered MPS, whereas they are qualitatively similar to those of the MPS for disordered systems.
These results indicate that disorder is not essential for the mechanical properties of jammed materials.




Although the values of the plateaus in $G'$ for disordered MPS depended on $\phi - \phi_{\rm J}$ \cite{OHern02,Otsuki17}, the corresponding values of the TPM were independent of $\phi - \phi_{\rm J}$, as expressed in Eq. \eqref{GpT}.
This is because we ignored the disorder effects, resulting in the $\phi$-dependence of $G'$ \cite{Wyart}. 
To include the disorder effect, we regarded $k_{\rm t}/k_{\rm n}$ as a fitting parameter.
In previous studies on models with small degrees of freedom, e.g., the coherent potential approximation \cite{Goldenfeld,Yonezawa, Feng}, the corresponding fitting parameters were self-consistently determined.
We will discuss the self-consistent determination of the parameter for the TPM in future studies.



\begin{acknowledgments}
The authors thank K. Saitoh, D. Ishima, and S. Takada for fruitful discussions.
This study was supported by JSPS KAKENHI under Grant Nos. JP19K03670 and JP21H01006.

\end{acknowledgments}


\bibliography{export}% Produces the bibliography via BibTeX.


\clearpage

\setcounter{equation}{0}
\setcounter{figure}{0}

\renewcommand{\theequation}{S\arabic{equation}}
\renewcommand{\thefigure}{S\arabic{figure}}


\begin{center}
\textbf{\Large Supplemental Material }
\end{center}


This supplemental material provides details that are not included in the main text. 
In Sec. \ref{Analysis}, we derive the analytical expressions for $\sigma$ and $P_0$ in the three-particle model (TPM).
In Sec. \ref{Stress}, we demonstrate the stress--strain curve obtained from the analytical results, which explain the dependence of the shear modulus on $\gamma_0$.
The shear modulus obtained numerically from the TPM is presented in Sec. \ref{Direct}.
In Sec. \ref{PTL}, we show the results of the many-particle system (MPS) when the particles are initially placed on a triangular lattice.
The effect of particle rotation is described in Sec. \ref{Rotation}.
The details of the MPS with disorder are presented in Sec. \ref{MPS}.



\section{Analytical calculation of shear stress and pressure}
\label{Analysis}

In this section, we briefly explain the derivation of the normal and tangential components of shear stress and pressure for a small value of $\gamma_0$.
From Eqs. \eqref{TE1}--\eqref{TE3}, the relative distance $\boldsymbol{r}_{ij}(t)$ is given by
\begin{align}
  \label{r12}
  & \boldsymbol{r}_{12}(t) = \left ( \frac{\sqrt{3} \gamma(\theta(t)) + 1 }{2}\ell, \frac{\sqrt{3} \ell}{2} \right ), \\
  \label{r13}
  & \boldsymbol{r}_{13}(t) = \left ( \frac{\sqrt{3} \gamma(\theta(t)) -1 }{2}\ell, \frac{\sqrt{3} \ell}{2} \right ), \\
  \label{r23}
  &   \boldsymbol{r}_{23}(t) = \left ( -\ell, 0 \right ).
\end{align}
Substituting these equations into $u^{\rm (n)}_{ij} = r_{ij} - d$, the normal displacements are given by 
\begin{align}
  & u^{\rm (n)}_{12}(t) = - \epsilon d + \frac{\sqrt{3}}{4} \ell \gamma(\theta(t)) + O(\gamma_0^2), \\
      & u^{\rm (n)}_{13}(t) = - \epsilon d - \frac{\sqrt{3}}{4} \ell \gamma(\theta(t)) + O(\gamma_0^2), \\
  & u^{\rm (n)}_{23}(t) = - \epsilon d.
\end{align}
Substituting these equations into Eq. \eqref{Fn}, we obtain
the normal force as
\begin{align}
  \label{Fn0}
  & f_{12}^{\rm (n)} = k_{\rm n}  \left ( \epsilon d - \frac{\sqrt{3}}{4} \gamma(\theta) \ell \right ), \\
  \label{Fn1}
  & f_{13}^{\rm (n)} = k_{\rm n}  \left ( \epsilon d - \frac{\sqrt{3}}{4} \gamma(\theta) \ell \right ), \\
  \label{Fn2}
  & f_{23}^{\rm (n)} = k_{\rm n}  \epsilon d
\end{align}
up to $O(\gamma_0)$.



By differentiating Eqs. \eqref{r12}--\eqref{r23} with time $t$, we obtain the relative velocity as
\begin{align}
  \label{v01}
  & \boldsymbol{v}_{12}(t) = \left ( \frac{\sqrt{3} \dot \gamma(\theta(t))\ell }{2}, 0 \right ), \\
  \label{v13}
  & \boldsymbol{v}_{13}(t) = \left ( \frac{\sqrt{3} \dot \gamma(\theta(t))\ell }{2}, 0 \right ), \\
  \label{v11}
  &   \boldsymbol{v}_{23}(t) = \left ( 0, 0 \right )
\end{align}
with the strain rate $\dot \gamma(\theta(t)) = \frac{d}{dt} \gamma(\theta(t))$.
The tangential unit vector is given by
\begin{align}
  \label{t01}
  & \boldsymbol{t}_{12}(t) = \left (-\frac{\sqrt{3} \ell}{2}, \frac{\sqrt{3} \gamma(\theta(t)) + 1 }{2}\ell \right )/ \left |\boldsymbol{r}_{12} \right |, \\
  \label{t13}
  & \boldsymbol{t}_{13}(t) = \left (-\frac{\sqrt{3} \ell}{2}, \frac{\sqrt{3} \gamma(\theta(t)) - 1 }{2}\ell \right )/ \left |\boldsymbol{r}_{13} \right |, \\
  \label{t11}
  &   \boldsymbol{t}_{23}(t) = \left ( 0, -1 \right ).
\end{align}
By considering the inner product of $\boldsymbol{v}_{ij}$ and $\boldsymbol{t}_{ij}$, the tangential velocity is given by
\begin{align}
  & v^{\rm (t)}_{12}(t) = - \frac{3}{4} \ell \dot \gamma(\theta(t)) + O(\gamma_0^2), \\
  & v^{\rm (t)}_{13}(t) = - \frac{3}{4} \ell \dot \gamma(\theta(t)) + O(\gamma_0^2), \\
  & v^{\rm (t)}_{23}(t) = 0.
\end{align}



If the transition from the stick state to the slip state does not occur under oscillatory shear, the tangential displacement is obtained by integrating $v^{\rm (t)}_{ij}(t)$ as
\begin{align}
  & u^{\rm (t)}_{12}(t) = u^{\rm (t)}_{13}(t) = - \frac{3}{4} \ell \gamma(\theta(t)) + O(\gamma_0^2), \\
  & u^{\rm (t)}_{23}(t) = 0.
\end{align}
Substituting these equations into $f_{ij}^{\rm (t)} = - k_{\rm t} u_{ij}^{\rm (t)}$ yields
\begin{align}
  \label{Ft0}
  & f_{12}^{\rm (t)} = f_{13}^{\rm (t)} =   3k_{\rm t} \gamma(\theta (t)) \ell/4, \\
  \label{Ft1}
  & f_{23}^{\rm (t)} =0
\end{align}
up to $O(\gamma_0)$.
The condition that the transition does not occur is satisfied when
$f_{12}^{\rm (t)} < \mu f_{12}^{\rm (n)}$ for $\gamma = \gamma_0$.
Using Eqs. \eqref{Fn0} and \eqref{Ft0} with the assumption $\gamma_0 \ll \epsilon$, the condition is replaced by $\gamma_0 < \gamma_c$ with $\gamma_c$ given by Eq. \eqref{gc}.


For $\gamma_0 > \gamma_c$, there exist regions where $u_{ij}^{\rm (t)}$ is unchanged in the slip state as
\begin{align}
  \label{ut0}
  &u_{12}^{\rm (t)} =
   \left\{
\begin{array}{ll}
  -\frac{\mu k_{\rm n} \epsilon d}{k_{\rm t}}, & 0 \le \theta(\theta) < \frac{\pi}{2} \\
-\frac{\mu k_{\rm n} \epsilon d}{k_{\rm t}}
- \frac{3d (\gamma(\theta) - \gamma_0)}{4}, & \frac{\pi}{2} \le \theta < \frac{\pi}{2} + \Theta \\
   \frac{\mu k_{\rm n} \epsilon d}{k_{\rm t}}, & \frac{\pi}{2} + \Theta \le \theta <  \frac{3\pi}{2}\\
    \frac{\mu k_{\rm n} \epsilon d}{k_{\rm t}}
- \frac{3d (\gamma(\theta) + \gamma_0)}{4} , & \frac{3\pi}{2} \le \theta <  \frac{3\pi}{2} + \Theta\\
   -\frac{\mu k_{\rm n} \epsilon d}{k_{\rm t}}, & \frac{3\pi}{2} + \Theta \le \theta <  2\pi,
\end{array}
\right. \\
   & u_{13}^{\rm (t)} = u_{12}^{\rm (t)}, \\
  \label{ut1}
  &u_{23}^{\rm (t)} = 0,
\end{align}
where $\Theta$ satisfies
\begin{align}
-\frac{\mu k_{\rm n} \epsilon d}{k_{\rm t}}
- 3d\frac{ \gamma\left (\frac{\pi}{2} + \Theta \right) - \gamma_0}{4}
=
 \frac{\mu k_{\rm n} \epsilon d}{k_{\rm t}}.
\end{align}
This equation provides
$\Theta = \cos^{-1} \left ( 1 - 2 \gamma_c / \gamma_0 \right )$.
Substituting these equations into $f_{ij}^{\rm (t)} = - k_{\rm t} u_{ij}^{\rm (t)}$ yields
\begin{align}
  \label{Ftt0}
  &f_{12}^{\rm (t)} =
   \left\{
\begin{array}{ll}
  -\mu k_{\rm n} \epsilon d, & 0 \le \theta(\theta) < \frac{\pi}{2} \\
-\mu k_{\rm n} \epsilon d
- \frac{3k_{\rm t} d (\gamma(\theta) - \gamma_0)}{4}, & \frac{\pi}{2} \le \theta < \frac{\pi}{2} + \Theta \\
   \mu k_{\rm n} \epsilon d, & \frac{\pi}{2} + \Theta \le \theta <  \frac{3\pi}{2}\\
    \mu k_{\rm n} \epsilon d
- \frac{3k_{\rm t} d (\gamma(\theta) + \gamma_0)}{4}, & \frac{3\pi}{2}  \le \theta <  \frac{3\pi}{2} + \Theta\\
   -\mu k_{\rm n} \epsilon d, & \frac{3\pi}{2} + \Theta \le \theta <  2\pi,
\end{array}
\right. \\
  \label{Ftt0d}
 & f_{13}^{\rm (t)}= f_{12}^{\rm (t)}, \\
  \label{Ftt1}
  &f_{23}^{\rm (t)} = 0.
\end{align}



The normal component of $\sigma$ in Eq. \eqref{Sn} is given by
\begin{equation}
    \sigma^{\rm (n)} = \sigma^{\rm (n)}_{12} + \sigma^{\rm (n)}_{13}
    \label{sn123}
\end{equation}
with
\begin{align}
  \label{sn12}
    & \sigma^{\rm (n)}_{12} = - \frac{1}{A} \frac{x_{12} y_{12}}{r_{12}} f_{12}^{\rm (n)} \\
  \label{sn13}
    & \sigma^{\rm (n)}_{13} = - \frac{1}{A} \frac{x_{13} y_{13}}{r_{13}} f_{12}^{\rm (n)}.
\end{align}
Substituting Eqs. \eqref{r12} and \eqref{r13} with Eqs. \eqref{Fn0} and \eqref{Fn1} into Eqs. \eqref{sn12} and \eqref{sn13} and using Eq. \eqref{sn123}, we obtain $\sigma^{\rm (n)}$ as Eq. \eqref{sn}.


The tangential component of $\sigma$ in Eq. \eqref{St} is given by
\begin{equation}
    \sigma^{\rm (t)} = \sigma^{\rm (t)}_{12} + \sigma^{\rm (t)}_{13}
    \label{st123}
\end{equation}
with
\begin{align}
  \label{st12}
    & \sigma^{\rm (t)}_{(12)} = - \frac{1}{2 A} \frac{x_{12}^2 -  y_{12}^2}{r_{12}} f_{12}^{\rm (t)} \\
  \label{st13}
    & \sigma^{\rm (t)}_{(13)} = - \frac{1}{2 A} \frac{x_{13}^2 - y_{13}^2}{r_{13}} f_{12}^{\rm (t)}.
\end{align}
Substituting Eqs. \eqref{r12} and \eqref{r13} with Eq. \eqref{Ft0} into Eqs. \eqref{st123}, \eqref{st12}, and \eqref{st13}, we obtain $\sigma^{\rm (t)}$ as Eq. \eqref{st1} for $\gamma_0 < \gamma_c$.
Using Eqs. \eqref{Ftt0} and \eqref{Ftt0d} instead of Eq. \eqref{Ft0},
we obtain $\sigma^{\rm (t)}$ as Eq. \eqref{st2} for $\gamma_0 \ge \gamma_c$.


The pressure, i.e., $P$, in Eq. \eqref{P} is defined as
\begin{equation}
  P = P_{12} + P_{13} + P_{23}
  \label{P123}
\end{equation}
with
\begin{align}
  \label{Pij}
     P_{ij} =  \frac{1}{2 A} r_{ij} f_{ij}^{\rm (n)}.
\end{align}
Substituting Eqs. \eqref{r12}-\eqref{r23} with Eqs. \eqref{Fn0}-\eqref{Fn2} into Eqs. \eqref{P123} and \eqref{Pij} with $\gamma = 0$, we obtain $P_0$ as Eq. \eqref{PT}.




\section{Stress--strain curve}
\label{Stress}

In this section, we present the stress--strain curve obtained from the analytical results, which explains the dependence of the shear modulus on $\gamma_0$. 
As shown in Fig. \ref{st:Fig}(a), we plot $\sigma(\theta)$ against $\gamma(\theta)$ using Eqs. \eqref{g} and \eqref{s} with Eqs. \eqref{sn}-\eqref{st2} for $k_{\rm t}/k_{\rm n}=1.0$ and $\mu = 0.01$.
For $\gamma_0 = 0.00001$, which is lower than $\gamma_c(\mu)$, $\sigma$ is proportional to $\gamma$ with gradient $\sqrt{3}(k_{\rm n}+k_{\rm t})/4$.
For $\gamma_0 = 0.00005$ and $0.0003$, which are larger than $\gamma_c(\mu)$, $\sigma$ exhibits hysteresis loops, including the regions with gradients $\sqrt{3}(k_{\rm n}+k_{\rm t})/4$ and $\sqrt{3}k_{\rm n}/4$.
The region with the lower gradient $\sqrt{3}k_{\rm n}/4$ increases as $\gamma_0$ increases.
This behavior was observed in the MPS \cite{Otsuki17}.


\begin{figure}[htbp]
\includegraphics[width=1.0\linewidth]{st.eps}
  \caption{
    (a) Shear stress $\sigma$ against $\gamma$ 
    using Eqs. \eqref{g}, \eqref{s}, and \eqref{sn}-\eqref{st2} 
    for $k_{\rm t}/k_{\rm n}=1.0$, $\epsilon = 0.001$, and $\mu = 0.01$. 
    (b)  Scaled shear stress $\sigma/\gamma_0$ against $\gamma/\gamma_0$ 
    using Eqs. \eqref{g}, \eqref{s}, and \eqref{sn}-\eqref{st2} 
    for $k_{\rm t}/k_{\rm n}=1.0$, $\epsilon = 0.001$, and $\mu = 0.01$. 
    The thickest black line represents $\gamma_0 = 0.00001$.
    The second thickest red line represents $\gamma_0 = 0.00005$.
    The thin blue line represents $\gamma_0 = 0.0003$.
}
\label{st:Fig}
\end{figure}


Figure \ref{st:Fig}(b) shows the scaled shear stress $\sigma/\gamma_0$ against the scaled strain $\gamma/\gamma_0$ using Eqs. \eqref{g} and \eqref{s} with Eqs. \eqref{sn}-\eqref{st2} for $k_{\rm t}/k_{\rm n}=1.0$ and $\mu = 0.01$.
As $\gamma_0$ increases, the maximum value $\tilde \sigma_{\rm max} =( \sigma / \gamma_0)|_{\gamma/\gamma_0 = 1}$ decreases from a higher value $\sqrt{3}(k_{\rm n}+k_{\rm t})/4$ to a lower value $\sqrt{3}k_{\rm n}/4$.



The shear stress $\sigma(\theta)$ is expanded using the Fourier series as
\begin{align}
  \sigma(\theta)  =  \gamma_0 \sum_{n=1}^{\infty}  G'_{n} \sin (n \theta)  + \gamma_0 \sum_{n=1}^{\infty} G''_{n} \cos(n \theta),
\end{align}
where $G'_{n}$ and $G''_n$ with $n>1$ denote the higher harmonics, $G' = G'_1$, and $G'' = G''_1$.
By neglecting $G'_{n}$ and $G''_n$ for $n>1$,
\begin{align}
  G' \simeq \frac{\sigma\left (\theta=\pi/2 \right )}{\gamma_0} = \left . \frac{\sigma}{\gamma_0} \right |_{\gamma/\gamma_0 = 1} = \tilde \sigma_{\rm max},
\end{align}
which is the maximum value of the scaled stress--strain curve illustrated in Fig. \ref{st:Fig}(b).
This expression and the scaled stress--strain curve in Fig. \ref{st:Fig}(b) explain the decrease of $G'$ defined by Eq. \eqref{GpT}.




The area $S$ of the curve for $\sigma(\theta) / \gamma_0$ against $\gamma(\theta) / \gamma_0$ is given by
\begin{align}
  \label{S}
  S =  \int_{0}^{2 \pi} d \theta \frac{1}{\gamma_0} \frac{d \gamma(\theta)}{d \theta} \frac{\sigma (\theta)}{\gamma_0}.
\end{align}
Substituting Eq. \eqref{g} into Eq. \eqref{S} with Eq. \eqref{Gp}, we obtain	
\begin{align}
  S =  \int_{0}^{2 \pi} d \theta  \sigma (\theta) \cos \theta /\gamma_0 = \pi G'',
\end{align}
which results in $G'' = S / \pi$.
As $\gamma_0$ increases, the area $S$ of the scaled stress--strain curve in Fig. \ref{st:Fig}(b) increases first and decreases later, which explains the $\gamma_0$-dependence of $G''$ provided by Eq. \eqref{GppT}.




\section{Shear modulus numerically obtained from the TPM}
\label{Direct}

In this section, we show the behaviors of $G'$ and $G''$ in the TPM without the assumption used to obtain the analytical solution.
Here, we numerically obtain $G'$ and $G''$ using Eqs. \eqref{Gp} and \eqref{Gpp} based on the left Riemann sum, where the integration of $\Psi(\theta)$, i.e.,
\begin{align}
    \int_0^{2 \pi} d \theta \ \Psi (\theta),
\end{align}
is approximated as
\begin{align}
    \int_0^{2 \pi} d \theta \ \Psi (\theta)
   \simeq \sum_{n=1}^M \Psi(\theta_n) \Delta \theta 
\end{align}
with $\Delta \theta = 2 \pi / M$ and $\theta_n =  (n-1)\Delta \theta$.
We use $\epsilon = 0.001$ and $ \Delta \theta = 5.0 \times 10^{-5}$ in our simulation.


As shown in Fig. \ref{Gp_3PD}, we plot the storage modulus $G'$ numerically obtained from the TPM against $\gamma_0$ with $k_{\rm t} / k_{\rm n} = 1.0$ for various values of $\mu$ as points.
Moreover, we plot the analytical results derived from Eq. \eqref{GpT} as thin solid lines.
The numerical results agree with the analytical results for $\gamma_0 < 0.003$ and reproduce the second plateau of the MPS shown in Fig. \ref{Gp_3P}.



\begin{figure}[htbp]
\includegraphics[width=0.7\linewidth]{Gp_3PD.eps}
  \caption{
    Storage modulus $G'$ against $\gamma_0$ with $k_{\rm t} / k_{\rm n} = 1.0$ and $\epsilon = 0.001$ for various values of $\mu$.
    The points represent the numerical results of the TPM, while the thin solid lines represent the analytical result given by Eq. \eqref{GpT}. 
    The vertical dashed lines represent the critical amplitude $\gamma_c(\mu)$ given by Eq. \eqref{gc} for $\mu = 10^{-4}, 10^{-3}, 10^{-2}, 10^{-1}, 10^{0}$ from left to right.
}
  \label{Gp_3PD}
\end{figure}


Figure \ref{Gpp_3PD} shows the loss modulus $G''$ numerically obtained from the TPM against $\gamma_0$ with $k_{\rm t} / k_{\rm n} = 1.0$ for various values of $\mu$ as points.
We also plot the analytical results given by Eq. \eqref{GppT} as thin solid lines.
The numerical results agree with the analytical results for $\gamma_0 < 0.003$.


\begin{figure}[htbp]
\includegraphics[width=0.7\linewidth]{Gpp_3PD.eps}
  \caption{
    Loss modulus $G''$ against $\gamma_0$ with $k_{\rm t} / k_{\rm n} = 1.0$ and $\epsilon = 0.001$ for various values of $\mu$.
    The points represent the numerical results of the TPM, while the thin solid lines represent the analytical results obtained from Eq. \eqref{GppT}. 
    The vertical dashed lines represent the critical amplitude $\gamma_c(\mu)$ obtained from Eq. \eqref{gc} for $\mu = 10^{-4}, 10^{-3}, 10^{-2}, 10^{-1}, 10^{0}$ from left to right.
}
  \label{Gpp_3PD}
\end{figure}



\section{Ordered MPS}
\label{PTL}


This section explains the details of the ordered MPS consisting of monodispersed particles initially placed on a triangular lattice.
We consider a two-dimensional assembly of $N$ frictional particles in a periodic box with sizes along the $x$ and $y$ directions $L_x$ and $L_y$, respectively.
Here, we initially place $N=2 N_x N_y$ particles of diameter $d$ with integers $N_x$ and $N_y$ at $\boldsymbol r_i$ as
\begin{equation}
  \boldsymbol r_i = \left (n_x \ell - L_x/2, \sqrt{3} n_y \ell -L_y/2 \right )
\end{equation}
for $0 \le i < N_x N_y$ with integers $n_x$, $n_y$, and $i = n_x + N_x n_y$.
For $N_x N_y \le i < 2 N_x N_y$, $\boldsymbol r_i$ is defined as
\begin{equation}
  \boldsymbol r_i = \left ((n_x + 1/2)\ell - L_x/2, \sqrt{3} (n_y+ 1/2) \ell -L_y/2 \right )
\end{equation}
with $i = n_x + N_x n_y + N_x N_y$.
The initial configuration is illustrated in Fig. \ref{confCR}.
We chose $L_x = N_x \ell$ and $L_y = \sqrt{3} N_y \ell$ with $\ell=d(1-\epsilon)$.

\begin{figure}[htbp]
\includegraphics[width=0.8\linewidth]{confCR.eps}
  \caption{
    Initial configuration of mono-dispersed particles on a triangular lattice.
The red rectangle, including interactions represented by the blue lines, corresponds to the TPM.
}
  \label{confCR}
\end{figure}



The position $\boldsymbol r_i$ and peculiar momentum ${\boldsymbol p}_i$ of particle $i$ with mass $m_i$ and diameter $d_i$ are driven by the SLLOD equation under the Lees-Edwards boundary condition as \cite{Evans}
\begin{eqnarray}
\label{ri}
  \frac{d}{dt} {\boldsymbol r}_i & = & \dot \gamma(t) y_i \boldsymbol e_x + \frac{\boldsymbol p_i}{m_i}, \\
\label{pi}
   \frac{d}{dt} {\boldsymbol p}_i & = & - \dot \gamma(t) p_{i,y} \boldsymbol e_x + \boldsymbol f_i
\end{eqnarray}
where $\dot \gamma(t) = \gamma_0 \omega \cos \omega t$ and $\bm{e}_x = (1,0)$ is the unit vector along the $x$ direction.
The interaction force $\boldsymbol f_i$ is defined as 
\begin{equation}
  \label{FMPS}
  \boldsymbol f_i = \sum_{j \neq i} \left ( f_{ij}^{\rm (n)} \boldsymbol n_{ij} + f_{ij}^{\rm (t)}\boldsymbol t_{ij}  \right ) H(d_{ij} - r_{ij})
\end{equation}
with $d_{ij} = (d_i + d_j)/2$, $\boldsymbol n_{ij} = \boldsymbol r_{ij} / r_{ij}$, $\boldsymbol t_{ij} = (-n_{ij,y},n_{ij,x})$, and $\boldsymbol r_{ij} = \boldsymbol r_i -  \boldsymbol r_j = (x_{ij},y_{ij})$.
The normal force is given by
\begin{equation}
 f_{ij}^{\rm (n)} = -\left ( k_{\rm n} u^{\rm (n)}_{ij}
 + \eta_{n} v^{\rm (n)}_{ij}\right )
  \label{FnMPS}
\end{equation}
with a normal viscous constant $\eta_{\rm n}$ and
\begin{eqnarray}
  v^{\rm (n)}_{ij} = \left ( \boldsymbol v_i - \boldsymbol v_j \right ) \cdot \boldsymbol n_{ij},
\end{eqnarray}
where the velocity of particle $i$ is given by $\boldsymbol v_i = \frac{d}{dt} \boldsymbol r_i$.
The following model is adopted for the tangential force:
\begin{equation}
  f_{ij}^{\rm (t)} = {\rm min} \left ( |\tilde f_{ij}^{\rm (t)}|, \mu f_{ij}^{\rm (n,el)} \right ) {\rm sgn} (\tilde f_{ij}^{\rm (t)}),
  \label{FtMPS}
\end{equation}
where $f_{ij}^{\rm (n,el)} = - k_{\rm n} u^{\rm (n)}_{ij}$ denotes the elastic part of the normal force.
Here, $\tilde f_{ij}^{\rm (t)}$ is given by
\begin{equation}
  \tilde f_{ij}^{\rm (t)} = - \left ( k_{\rm t} u_{ij}^{\rm (t)} + \eta_{\rm t} v_{ij}^{\rm (t)} \right )
\end{equation}
with a tangential viscous constant $\eta_{\rm t}$.
The tangential velocity $v_{ij}^{\rm (t)}$ is given by 
\begin{equation}
  v_{ij}^{\rm (t)} = (\boldsymbol v_i - \boldsymbol v_j)\cdot \boldsymbol t_{ij}.
  \label{VT}
\end{equation}
The tangential displacement $u_{ij}^{\rm (t)}$ satisfies $\frac{d}{dt} u_{ij}^{\rm (t)} = v_{ij}^{\rm (t)}$ for $|\tilde f_{ij}^{\rm (t)}|< \mu f_{ij}^{\rm (n,el)}$, whereas $u_{ij}^{\rm (t)}$ remains unchanged for $|\tilde f_{ij}^{\rm (t)}| \ge \mu f_{ij}^{\rm (n,el)}$.
The tangential displacement $u_{ij}^{\rm (t)}$ is set to zero if $i$ and $j$ are detached.



If all the particles are separated, the packing fraction $\phi$ is defined as
\begin{align}
\label{phi}
    \phi = \frac{\sum_i \pi d_i^2 }{4L_x L_y}.
\end{align}
Even if contact exists between the particles, we use Eq. \eqref{phi} by assuming that the contact length $d_{ij} - r_{ij}$ is sufficiently lower than $d_{ij}$.
Using Eq. \eqref{phi}, $\phi$ is defined as
\begin{align}
    \phi = \frac{\pi}{2 \sqrt{3}(1-\epsilon)^2 }.
\end{align}
The jamming point of this system is 
\begin{align}
    \phi_{\rm J} = \frac{\pi}{2 \sqrt{3}}
\end{align}
with $\epsilon = 0$.
The distance from the jamming point is proportional to $\epsilon$.
\begin{align}
    \phi - \phi_{\rm J} \simeq \frac{\pi}{ \sqrt{3}} \epsilon
\end{align}
for $\epsilon \ll 1$.


The shear stress $\sigma$ is defined by Eq. \eqref{s} in the main article with
the normal component 
\begin{equation}
  \sigma^{\rm (n)} = - \frac{1}{ L_x L_y} \sum _i \sum_{j>i} \frac{x_{ij} y_{ij}}{r_{ij}} f^{\rm (n)}_{ij}
  \label{Sn:MPS}
\end{equation}
and tangential component
\begin{equation}
  \sigma^{\rm (t)} = - \frac{1}{ 2 L_x L_y} \sum _i \sum_{j>i} \frac{x_{ij}^2 -y_{ij}^2}{r_{ij}}f^{\rm (t)}_{ij}.
  \label{St:MPS}
\end{equation}
The pressure is defined as
\begin{equation}
  P = \frac{1}{2 L_x L_y} \sum _i \sum_{j>i} (x_{ij} f_{ij,x}+y_{ij} f_{ij,y}).
  \label{P:MPS}
\end{equation}



We use $N_x = 8$, $N_y = 4$, $N_{\rm c}=20$, $k_{\rm t} = k_{\rm n}$, and $\eta_{\rm n} =\eta_{\rm t} = k_{\rm n} \sqrt{m/k_{\rm n}}$, where $m$ denotes the mass of a grain of diameter $d$.
This model corresponds to a restitution coefficient $e = 0.043$.
We adopt the leapfrog algorithm considering a time step of $\Delta t = 0.05 t_0$.
We chose $ \omega = 1.0 \times 10^{-4} \sqrt{k_{\rm n}/m}$ as the quasistatic shear deformation because $G'$ and $G''$ are almost independent of $ \omega$ for $\omega \le 1.0 \times 10^{-3} \sqrt{k_{\rm n}/m}$.









As shown in Figs. \ref{Gp_3P} and \ref{Gpp_3P}, the behaviors of $G'$ and $G''$ of the TPM agree with that of the MPS.
We explain the theoretical background of the TPM.
The initial configuration is shown in Fig. \ref{confCR}; it contains the unit cell, which is represented by the red rectangle with length $\ell$ and height $\sqrt{3} \ell /2$. 
It contains interactions between the three particles, represented by blue lines. 
Here, we assume that the particles move affinely as
\begin{align}
   \boldsymbol r_{i}(t) = \boldsymbol r_{i}(0) + \gamma(\theta(t)) y_i(0) \boldsymbol e_x.
\end{align}
In this case, the corresponding relative distances between the particles in any unit cell are identical.

In particular, in a unit cell containing particles $i=i_1, i_2$, and $i_3$ with $i_1 = N_x N_y$, $i_2 = 0$, and $i_3 = 1$, the positions of the particles are given by
\begin{align}
\label{TEE1}
  & \boldsymbol r_{i_1}(t)  =  \left ( \gamma(\theta(t))\left ( \frac{\sqrt{3}\ell - L_y}{2} \right ) + \frac{\ell - L_x}{2}, \frac{\sqrt{3}\ell - L_y}{2}\right), \\
  & \boldsymbol r_{i_2}(t)  =  \left ( - \gamma(\theta(t)) \frac{L_y}{2} - \frac{L_x}{2},  - \frac{L_y}{2} \right), \\
\label{TEE3}
  & \boldsymbol r_{i_3}(t)  =  \left ( - \gamma(\theta(t)) \frac{L_y}{2} + \ell - \frac{L_x}{2},  -\frac{L_y}{2} \right).
\end{align}
The relative distances between these particles are identical to those of the TPM, given by Eqs. \eqref{TE1}-\eqref{TE3}, which indicates that the TPM provides the interaction forces among the three particles.
This system includes $2 N_x N_y$ unit cells with identical interaction forces. 
Hence, the normal and tangential components of $\sigma$ are given by 
\begin{align}
  & & \sigma^{\rm (n)}  =  - \frac{ 2 N_x N_y}{ L_x L_y} \sum _{i = i_1, i_2, i_3} \left \{  \sum_{\substack{j=i_1, i_2, i_2 \\ (j>i)} } \frac{x_{ij}y_{ij}}{r_{ij}} f^{\rm (n)}_{ij} \right \}, \\
& & \sigma^{\rm (t)}  =  - \frac{ N_x N_y}{  L_x L_y} \sum _{i = i_1, i_2, i_3} \left \{ \sum_{\substack{j=i_1, i_2, i_2 \\ (j>i)} } \frac{x_{ij}^2-y_{ij}^2}{r_{ij}} f^{\rm (t)}_{ij}\right \}.
\end{align}
The pressure is also given by:
\begin{equation}
  P = \frac{N_x N_y}{ L_x L_y}  
  \sum _{i = i_1, i_2, i_3} \left \{  \sum_{\substack{j=i_1, i_2, i_2 \\ (j>i)} }
   (x_{ij} f_{ij,x}+y_{ij} f_{ij,y}) \right \}.
\end{equation}
Using the relation $L_x L_y / (2 N_x N_y) = \sqrt{3} \ell ^2 / 2$ corresponding to $A= \sqrt{3} \ell ^2 / 2$, $\sigma^{\rm (n)}$, $\sigma^{\rm (t)}$, and $P$ coincide with Eqs. \eqref{s}-\eqref{P}.
Hence, if the assumptions of the affine motion, i.e., Eqs. \eqref{TEE1}--\eqref{TEE3}, are satisfied, $G'$ and $G''$ in the ordered MPS coincide with those in the TPM.


\section{Effect of particle rotation}
\label{Rotation}


In this section, we illustrate the effect of particle rotation, which was not described in Sec. \ref{PTL}.
In the model with rotation, the tangential velocity $v_{ij}^{\rm (t)}$ is given by 
\begin{equation}
  v_{ij}^{\rm (t)} = (\boldsymbol v_i - \boldsymbol v_j)\cdot \boldsymbol t_{ij} - (d_i \omega_i + d_j \omega_j)/2
\end{equation}
instead of Eq. \eqref{VT},
where $\omega_i$ denotes the angular velocity of particle $i$.
The time evolution of $\omega_i$ is given by
\begin{equation}
   I_i \frac{d}{dt} \omega_i = T_i 
   \label{omegai}
\end{equation}
with the moment of inertia $I_i = m_i d_i^2 /8$ and torque $T_i =  - \sum_j \frac{d_i}{2} \boldsymbol F_{ij}^{\rm (t)} \cdot \boldsymbol t_{ij}$.



As shown in Fig. \ref{Gp_CR}, we plot $G'$ in the MPS with and without rotation for $\mu = 0.01$ and $\epsilon = 0.001$.
The values of other parameters are the same as those in Sec. \ref{PTL}.
The effect of particle rotation is negligible, except for the region near $\gamma_c$.


\begin{figure}[htbp]
\includegraphics[width=0.7\linewidth]{Gp_CR.eps}
  \caption{
    Storage modulus $G'$ against $\gamma_0$ for the MPS on the triangular lattice with $\mu = 0.01$ and $\epsilon = 0.001$.
    The squares and circles represent the results of the particles with and without rotation, respectively.
}
  \label{Gp_CR}
\end{figure}



\section{Disordered MPS}
\label{MPS}



In this section, we present the details of the disordered MPS.
This model is an extension of the monodisperse model used in Sec. \ref{PTL}, including the dispersion of the particles and disordered initial configuration.


The system is bidisperse and includes an equal number of particles with diameters $d$ and $d/1.4$.
To simulate the disordered MPS, we randomly place the particles in a rectangular box with an initial packing fraction of $\phi_{\rm I} = 0.75$.
The system is slowly compressed until the packing fraction reaches $\phi$ \cite{Otsuki21}.
In each compression step, the packing fraction is increased by $\Delta \phi = 1.0 \times 10^{-4}$ with an affine transformation. 
Thereafter, the particles are relaxed to a mechanical equilibrium state with the kinetic temperature $T_{\rm K} = \sum_i p_i^2 /(mN) < T_{\rm th}$. 
Here, we chose $T_{\rm th} = 1.0 \times 10^{-8}k_{\rm n}d^2$.
After compression, the oscillatory shear strain given by Eq. \eqref{g} in the main text is applied for $N_{\rm c}$ cycles.
In the last cycle, we measure $G'$ and $G''$ using Eqs. \eqref{Gp} and \eqref{Gpp} with Eqs. \eqref{s}--\eqref{St}.
The pressure, $P_0$, is obtained using Eq. \eqref{P} after the last cycle.
We use $\phi=0.87$, $N=1000$, $N_{\rm c}=20$, $L_y/L_x = 1$, $k_{\rm t} = 0.2k_{\rm n}$, and $\eta_{\rm n} =\eta_{\rm t} = k_{\rm n} \sqrt{m/k_{\rm n}}$.



As shown in Fig. \ref{st_YPT10:Fig}(a), we plot the shear stress $\sigma$ against $\gamma$ in the disordered MPS with $\mu = 0.0001$. 
For $\gamma_0 = 0.00001$, $\sigma$ is almost proportional to $\gamma_0$.
As $\gamma_0$ increases, the stress--strain curve exhibits a loop in which the gradient of the curve is higher near $\gamma = \pm \gamma_0$ and lower for $\gamma \simeq 0$.
This loop is similar to that of the TPM (Fig. \ref{st:Fig}(a)).
Figure \ref{st_YPT10:Fig}(b) shows the scaled shear stress $\sigma/\gamma_0$ against the scaled strain $\gamma/\gamma_0$ in the MPS with $\mu = 0.0001$. 
The maximum value $\tilde \sigma_{\rm max}$ decreases as $\gamma_0$ increases.
The area $S$ of the curve is the largest for $\gamma_0 = 0.00003$.
This behavior is similar to that of the TPM (Fig. \ref{st:Fig}(b)).





\begin{figure}[htbp]
\includegraphics[width=1.0\linewidth]{st_YPT10.eps}
  \caption{
    (a) Shear stress $\sigma$ against $\gamma$ in the disordered MPS with $\mu = 0.0001$ and $\phi=0.870$. 
    (b)  Scaled shear stress $\sigma/\gamma_0$ against $\gamma/\gamma_0$ in the disordered MPS with $\mu = 0.01$ and $\phi=0.870$. 
    The thickest black line represents the result of $\gamma_0 = 0.00001$.
    The second thickest red line represents that of $\gamma_0 = 0.00003$.
    The thin blue line represents that of $\gamma_0 = 0.0001$.
}
\label{st_YPT10:Fig}
\end{figure}

\begin{figure}[htbp]
\includegraphics[width=0.7\linewidth]{Gp_YPT10.eps}
  \caption{
    Storage modulus $G'$ in the disordered MPS against $\gamma_0$ with $\phi=0.870$ for various values of $\mu$.
}
  \label{Gp_YPT10}
\end{figure}






Figure \ref{Gp_YPT10} shows the storage modulus $G'$ against $\gamma_0$ in the disordered MPS for various values of $\mu$.
The storage modulus $G'$ is almost independent of $\gamma_0$ for a small $\gamma_0$ and decreases as $\gamma_0$ increases.
The endpoint of the first plateau increases with $\mu$ except for $\mu=0$.  
A second plateau of $G'$ exists for $\mu=10^{-5}$.
These behaviors are analogous to those of the TPM (Fig. \ref{Gp_YPT10}).
Note that $G'$ for $\mu=0.1$ in the limit $\gamma_0 \to 0$ is different from that for $\mu \le 0.01$, which results from the $\mu$ dependence of the jamming point $\phi_{\rm J}$ \cite{Otsuki17}.

\begin{figure}[htbp]
\includegraphics[width=0.7\linewidth]{Gpp_YPT10.eps}
  \caption{
    Loss modulus $G''$ in the disordered MPS against $\gamma_0$ with $\phi=0.870$ for various values of $\mu$.
}
  \label{Gpp_YPT10}
\end{figure}






Figure \ref{Gpp_YPT10} shows the loss modulus $G''$ in the disordered MPS against $\gamma_0$ for various values of $\mu$.
For sufficiently small $\gamma_0$, $G''$ is zero.
$G''$ becomes finite as $\gamma_0$ increases. 
$G''$ starts to increase for smaller $\gamma_0$ as $\mu$ decreases.
These behaviors are similar to those of the TPM illustrated in Fig. \ref{Gpp_3P}.




\end{document}
%
% ****** End of file apssamp.tex ******
