\begin{figure*}[t]
    \centering
    \includegraphics[width=0.97\textwidth]{images/setup.png}
    \vspace{-6pt}
    \caption{\textit{Language guidance.} We extend the default DML pipeline for Visual Similarity Learning by embedding either \textbf{(A)} expert class names or \textbf{(B)} top-$k$ ImageNet pseudolabels, which require no additional expert supervision, with a pretrained language model. This provides language similarity matrices $S^\text{lang}$ which are used to guide the structuring of our finegrained visual similarity space generated by $f\circ \phi_\text{ImageNet}$ through distillation-based matching ($\mathcal{L}_\text{match}$ \& $\mathcal{L}_\text{pseudomatch}$) between $S^\text{lang}$ and image similarities $S^\text{img, X}$.
    % \oriol{The figure is really nice. Only if you have a good idea, but the only thing that "bothers" me is (C). A and B are the main methods. C is just a few examples to explain B. Could we group / name things differently so that distinction is clearer without reading caption / text?} \zeynep{I think C could be integrated inside B} \zeynep{how about you remove the car, move the baltimore oriole before the imageNet box (blue box), shift the imagenet blue box and everything to the right of it to the right. the image then is fed to the imagenet, then you can mark the top 3 labels on the blue chart next to the imageNet box, then the rest of the figure in B is the same. remove C.}
    }
    \label{fig:setup}
    \vspace{-5pt}
\end{figure*}