\documentclass[oneside,11pt,reqno]{amsart}
\usepackage[utf8]{inputenc}
\usepackage{stmaryrd}
\usepackage{bbm}
\usepackage[dvipsnames]{xcolor}
%\usepackage{xypic}
\usepackage{mathrsfs}
\usepackage{enumerate}
\usepackage{centernot}
\usepackage{latexsym,amsxtra}
\usepackage{dsfont}
\usepackage{slashed}
\usepackage[all]{xy}
\usepackage{amscd,graphics}
\usepackage{amsmath,amsfonts,amsthm,amssymb}
\usepackage{latexsym,amsmath}
\usepackage{graphicx,psfrag}
\usepackage{mathabx}
%\usepackage{pictexwd,dcpic}
%\usepackage{hyperref}
\usepackage{fancyhdr}


%%%%%%%%%%%%%%%%%%Hokuto added
\usepackage{cleveref}
\usepackage{comment} 
\usepackage{braket}
\usepackage{caption}
\captionsetup[table]{justification=centering}
\captionsetup[figure]{justification=centering}
\usepackage{enumitem}
%%%%%%%%%%%%%%%%%%
%margin control
\usepackage{kantlipsum}
\setlength{\textwidth}{\paperwidth}
\addtolength{\textwidth}{-2.5in}
\calclayout
%%%%%%%%%%%%%%%%%%

%\newtheorem{thm}{Theorem}[section]
%\newtheorem*{nota}{Notation}
%\newtheorem{fact}[thm]{Fact}
%\newtheorem*{claim}{Claim}
%\newtheorem{conj}{Conjecture}
%\newtheorem{assum}[thm]{Assumption}
%\newtheorem{notation}[thm]{Notation}
%\newtheorem{lem}[thm]{Lemma}
%\newtheorem{cor}[thm]{Corollary}
%\newtheorem{pro}[thm]{Proposition}
%\newtheorem{cons}[thm]{Construction}
%\theoremstyle{definition}
%\newtheorem{defi}[thm]{Definition}
%\newtheorem{ex}[thm]{Example}
%\theoremstyle{remark}
%\newtheorem{rmk}{Remark}

%\newtheorem{ques}[thm]{Question}

%%%%%%%%%%%%%(cleverref)%%%%%%%%%%%%%%%
\theoremstyle{plain}
\newtheorem{thm}{Theorem}[section]
\crefname{thm}{Theorem}{Theorems}
\Crefname{thm}{Theorem}{Theorems}
\newtheorem{pro}[thm]{Proposition}
\crefname{pro}{Proposition}{Propositions}
\Crefname{pro}{Proposition}{Propositions}
\newtheorem{lem}[thm]{Lemma}
\crefname{lem}{Lemma}{Lemmas}
\Crefname{lem}{Lemma}{Lemmas}
\newtheorem{cor}[thm]{Corollary}
\crefname{cor}{Corollary}{Corollaries}
\Crefname{cor}{Corollary}{Corollaries}
\newtheorem{conj}{Conjecture}
\crefname{conj}{Conjecture}{Conjectures}
\Crefname{conj}{Conjecture}{Conjectures}
\newtheorem{cons}[thm]{Construction}
\crefname{cons}{Construction}{Constructions}
\Crefname{cons}{Construction}{Constructions}
\newtheorem{claim}[thm]{Claim}
\crefname{claim}{Claim}{Claims}
\Crefname{claim}{Claim}{Claims}
\newtheorem{property}[thm]{Property}
\crefname{property}{Property}{Properties}
\Crefname{property}{Property}{Properties}
\newtheorem{problem}[thm]{Problem}
\crefname{problem}{Problem}{Problems}
\Crefname{problem}{Problem}{Problems}

\theoremstyle{definition}
\newtheorem{defi}[thm]{Definition}
\crefname{defi}{Definition}{Definitions}
\Crefname{defi}{Definition}{Definitions}
\newtheorem{nota}[thm]{Notation}
\crefname{nota}{Notation}{Notations}
\Crefname{nota}{Notation}{Notations}
\newtheorem {convention}[thm]{Convention}
\crefname{convention}{Convention}{Conventions}
\Crefname{convention}{Convention}{Conventions}
\newtheorem{cond}[thm]{Condition}
\crefname{cond}{Condition}{Conditions}
\Crefname{cond}{Condition}{Conditions}
\newtheorem{assum}[thm]{Assumption}
\crefname{assum}{Assumption}{Assumptions}
\Crefname{assum}{Assumption}{Assumptions}

\theoremstyle{remark}
\newtheorem{rmk}[thm]{Remark}
\crefname{rmk}{Remark}{Remarks}
\Crefname{rmk}{Remark}{Remarks}
\newtheorem{ex}[thm]{Example}
\crefname{ex}{Example}{Examples}
\Crefname{ex}{Example}{Examples}

\newtheorem{ques}[thm]{Question}
\crefname{ques}{Question}{Questions}
\Crefname{ques}{Question}{Questions}

\crefname{section}{Section}{Sections}
\Crefname{section}{Section}{Sections}
\crefname{subsection}{Subsection}{Subsections}
\Crefname{subsection}{Subsection}{Subsections}
\crefname{figure}{Figure}{Figures}
\Crefname{figure}{Figure}{Figures}
%%%%%%%%%%%%%%%%%%%%%%%%%%


\newcommand{\Diff}{\textnormal{Diff}}
\newcommand{\BDiff}{\textnormal{BDiff}}
\newcommand{\TDiff}{\textnormal{TDiff}}
\newcommand{\BTDiff}{\textnormal{BTDiff}}
\newcommand{\MCG}{\textnormal{MCG}^{+}}
\newcommand{\Homeo}{\textnormal{Homeo}}
\newcommand{\BHomeo}{\textnormal{BHomeo}}

\newcommand{\spinc}{\textnormal{spin}^{c}}
\newcommand{\ab}{\textnormal{ab}}

%%%%%%%%%%%%%%%%%%%%%%%%%%Hokuto added
\newcommand{\SW}{\mathrm{SW}}
\newcommand{\SWbb}{\mathbb{SW}}
\newcommand{\SWcal}{\mathcal{SW}}
\newcommand{\SWbbtot}{\mathbb{SW}_{\mathrm{tot}}}
\newcommand{\SWcaltot}{\mathcal{SW}_{\mathrm{tot}}}
\newcommand{\SWbbhalftot}{\mathbb{SW}_{\mathrm{half\mathchar`- tot}}}
\newcommand{\SWcalhalftot}{\mathcal{SW}_{\mathrm{half\mathchar`- tot}}}
\newcommand{\EDiff}{\textnormal{EDiff}}
\newcommand{\id}{\textnormal{id}}
\newcommand{\Z}{\mathbb{Z}}
\newcommand{\N}{\mathbb{N}}
\newcommand{\R}{\mathbb{R}}
\newcommand{\C}{\mathbb{C}}
\newcommand{\Q}{\mathbb{Q}}
\newcommand{\F}{\mathbb{F}}
\newcommand{\CP}{\mathbb{CP}}
\newcommand{\fraks}{\mathfrak{s}}
\newcommand{\frakt}{\mathfrak{t}}
\newcommand{\frakM}{\mathfrak{M}}
\newcommand{\scrA}{\mathscr{A}}
\newcommand{\scrB}{\mathscr{B}}
\newcommand{\scrC}{\mathscr{C}}
\newcommand{\scrD}{\mathscr{D}}
\newcommand{\scrX}{\mathscr{X}}
\newcommand{\scrE}{\mathscr{E}}
\newcommand{\scrH}{\mathscr{H}}
\newcommand{\scrN}{\mathscr{N}}
\newcommand{\scrG}{\mathscr{G}}
\newcommand{\scrR}{\mathscr{R}}
\newcommand{\scrM}{\mathscr{M}}
\newcommand{\calM}{\mathcal{M}}
\newcommand{\calU}{\mathcal{U}}
\newcommand{\calL}{\mathcal{L}}
\newcommand{\calR}{\mathcal{R}}
\newcommand{\cpt}{\mathrm{cpt}}
\newcommand{\ind}{\mathrm{ind}}
\newcommand{\Conj}{\mathrm{Conj}}
\DeclareMathOperator{\Aut}{Aut}
\DeclareMathOperator{\Met}{Met}
\DeclareMathOperator{\Map}{Map}
\DeclareMathOperator{\Int}{Int}
\DeclareMathOperator{\sign}{sign}
\newcommand{\Spinc}{\mathrm{Spin}^{c}}
\newcommand{\Spincns}{\mathrm{Spin}^{c}}
\newcommand{\circPi}{\mathring{\Pi}}
\newcommand{\del}{\partial}
\newcommand{\tilM}{\tilde{\mathscr{M}}}

%%%%%%%%%%%%%%%%%%%%%%%%%%



%\newtheorem{exer}{Exercise}[section]

%%\usepackage[palatino,gill,courier]{altfont}
%\usepackage[all]{xy} inv int im ker coker mor ob grad



\title[Homological instability for smooth 4-manifolds]{Homological instability for moduli spaces of smooth 4-manifolds}
%Other options of Titles?
%``Homological instability in dimension 4"
%``Homological instability for 4-manifolds"



\author{Hokuto Konno}
\address{Graduate School of Mathematical Sciences, the University of Tokyo, 3-8-1 Komaba, Meguro, Tokyo 153-8914, Japan \\and\\
RIKEN iTHEMS, Wako, Saitama 351-0198, Japan}
\email{konno@ms.u-tokyo.ac.jp}

\author{Jianfeng Lin}
\address{Yau Mathematical Sciences Center\\ Tsinghua University\\ Beijing\\ China.}
\email{linjian5477@mail.tsinghua.edu.cn}

%\date{\today}

\begin{document}

\maketitle

\begin{abstract}
We prove that homological stability fails for the moduli space of any simply-connected closed smooth 4-manifold in any degree of homology, unlike what happens in all dimensions $\neq 4$. We detect also the homological discrepancy between various moduli spaces, such as topological and smooth moduli spaces of 4-manifolds, and moduli spaces of 4-manifolds with positive scalar curvature metrics. To prove these results, we use the Seiberg--Witten equations to construct a new characteristic class of families of 4-manifolds, which is unstable and detects the difference between the smooth and topological categories in dimension 4.
\end{abstract}

%\tableofcontents

\section{Introduction}
\label{section Introduction}


\begin{comment}
We first recall the statement of homological stability in higher dimensions. Let $W$ be a simply-connected, smooth manifold of dimension $2n$. Suppose $\partial W\neq \emptyset$. 
Then there is a well-defined map
$$
s_{W}:\Diff_{\partial}(W)\rightarrow \Diff_{\partial}(W\# (S^{n}\times S^{n}))
$$
defined as follows: Let $f: W\rightarrow W$ be a diffeomorphism that fixes a collar neighborhood $\nu(\partial W)$ of $\partial W$ pointwise. Then we define $$s_{W}(f):=f\# \operatorname{Id}_{S^{n}\times S^{n}},$$ where the connected sum is taken in  $\nu(\partial W)$. For any $k\geq 0$, $s_{W}$ induces the map
$$
s_{W,k}:H_{k}(\BDiff_{\partial}(W);\mathbb{Z})\rightarrow H_{k}(\BDiff_{\partial}(W\#(S^{n}\times S^{n});\mathbb{Z}).
$$
\begin{defi}
The genus of $W$ as 
\[
g(W):=\max\Set{k | \exists \text{ smooth embedding } \bigsqcup_{k} (S^{n}\times S^{n}\setminus \mathring{D}^{2n})\hookrightarrow W}.
\]
\end{defi}
\begin{thm}\cite[Theorem 1.2]{Galatius2018}\label{thm: homological stable}
Suppose $2n\geq 6$ and $g(W)\geq 2k+3$. Then $s_{W,k}$ is an isomorphism.
\end{thm}

There is a variation of this theorem for local coefficients. In particular, we let $\TDiff(W)$ be the group of diffeomorphisms that acts trivially on $H_{*}(W;\mathbb{Z})$ (i.e. the Torelli group). Then there is a similarly defined map 
$$
\tilde{s}_{W,k}: H_{k}(\BTDiff_{\partial}(W);\mathbb{Z})\rightarrow H_{k}(\BTDiff_{\partial}(W\#(S^{n}\times S^{n});\mathbb{Z}).
$$
\begin{thm}\cite[Corollary I]{Krannich2019}\label{thm: Torelli stable}
Suppose $2n\geq 6$ and $g(W)\geq 2k+6$. Then $\tilde{s}_{W,k}$ is an isomorphism.
\end{thm}

The main goal of this project is to show that Theorem \ref{thm: homological stable} and Theorem \ref{thm: Torelli stable}  fails badly in dimension $4$. Actually, we will prove various results in the opposite direction of these theorems, stating that various moduli spaces of 4-manifolds do not have homological stability. 

From now on, we assume $2n=4$. 
\begin{defi} We say $W$ is \textbf{homological unstable} if for any $k\geq 1$, the map $s_{W\#n(S^{2}\times S^{2}),k}$ is not an isomorphism for infinitely many $n$. We say $W$ is \textbf{Torelli unstable} if for any $k\geq 1$, the map $\tilde{s}_{W\#n(S^{2}\times S^{2}),k}$ is not an isomorphism for infinitely many $n$.
\end{defi}

The main conjecture we want to solve is the following two conjectures:
\begin{conj}\label{conj: 4-manifolds are all homological unstable} Any simply-connected, smooth $W$ with $\partial W=S^{3}$ is homological unstable.
\end{conj}


\begin{conj}\label{conj: 4-manifolds are all torelli unstable} Any simply-connected, smooth $W$ with $\partial W=S^{3}$ is Torelli unstable.
\end{conj}

%Conjecture \ref{conj: 4-manifolds are all homological unstable} seems significantly harder than Conjecture \ref{conj: 4-manifolds are all torelli unstable} (because we have to use total Seiberg--Witten invariant instead of Seiberg--Witten invariant for individual spinc structure). There is a variation which should be much easier.
%\begin{conj}\label{conj: homological instability}
%For any $k\geq 0$, there exists a sequence of simply-connected spin $4$-manifolds $\{W_{m}\}_{m\in \mathbb{N}}$ with $\partial W_{m}=S^{3}$ such that $\lim g(W_{m})=\infty$ and the map
%$
%s_{W_{m},k}$ is not an isomorphism for any $m$.
%\end{conj}



%There is another related conjecture. We reacll the following theorem by Kreck:
%\begin{thm}\cite{Kreck1999} Let $W,W'$ be two closed, smooth manifolds of dimension $2n>4$. Suppose $W\#m(S^{n}\times S^{n})\cong W\#m(S^{n}\times S^{n})$ for some $m$. Then the following conclusion holds:
%\begin{itemize}
%    \item If $n$ is odd and $M$ is simply-connected, then $M\cong M'$.
 %       \item If $n$ is even and $M\cong M_{0}\#(S^{2}\times S^{2})$ and $M$ is simply-connected, then $M\cong M'$.
  %      \item If $|\pi_{1}(M)|<\infty$ and $M\cong M_{0}\#2(S^{2}\times S^{2}) $, then $M\cong M'$.
%\end{itemize}
%\end{thm}
%In dimension $4$, Kreck and Hambleton showed similar result in the topological category. The corresponding result in the smooth category is not ture because there exists exotic smooth structures on $n(S^{2}\times S^{2})$ for all $n$ large enough. However, the following conjecture remains unsolved:

%\begin{conj}\label{conj: no cancellation in dimension 4}
%There exists a sequence of simply-connected, spin manifolds $X_{m}$ with $\sigma(X_{m})=0$ and a sequence $a_{m}\rightarrow \infty$ such that $$X_{m}\#a_{m} (S^{2}\times S^{2})\not\cong\# (a_{m}+\frac{b_{2}(X_{m})}{2})(S^{2}\times S^{2}).$$ 
%\end{conj}
%In particular, Conjecture \ref{conj: no cancellation in dimension 4} implies that for any $k>0$, there exists an exotic smooth structure that survives $k$-stabilizations. 


In this paper, we will prove  Conjecture \ref{conj: 4-manifolds are all homological unstable}. Namely, all simply-connected 4-manifolds $W$ with $\partial W=S^3$ are homological unstable. (Maybe also Conjecture 2?)
\end{comment}

\subsection{Background and setup}

Given a smooth manifold $M$, the classifying space $\BDiff(M)$ (also called the moduli space of $M$) classifies smooth fiber bundles with fiber $M$. Cohomology classes of $\BDiff(M)$ one-to-one correspond to characteristic classes for such bundles. For this reason, understanding the (co)homology group of $\BDiff(M)$ has been an important topic in manifold theory. A celebrated result of Harer~\cite{Harer85} proved homological stability for mapping class groups of oriented surfaces. This result can be interpreted in terms of
moduli spaces of Riemann surfaces. More recently, Galatius and Randal-Williams \cite{Galatius2018} established analogous homological stability for moduli spaces of manifolds of even dimension $\geq 6$, which is one of the highlights of the study of manifolds in the last decade. Similar homological stability results have also been established for moduli spaces of manifolds of odd dimension in \cite{Krannich2019}  for $\dim \geq 5$ and in \cite{Lam15} for $\dim=3$, and also for  mapping class groups of 3-manifolds in  \cite{Hatcher10}. As a natural question, one may ask what happens in the remaining dimension 4. 
%Does analogous stability result hold? Or there are some new instability phenomena that reflect the specialty of dimension 4?
%\marginpar{Which phrase is best here? (1) answer this question (2) address this question (3) provide our answer to this question.} 
The main purpose of this paper is to answer this question by establishing a phenomenon in the opposite direction of homological stability, which we call {\it homological instability}, in broad generality. The second purpose is to establish comparison results between various moduli spaces of 4-manifolds and 4-manifolds with metrics.
%: we shall prove that, unlike in dimensions $\neq 4$, homological stability fails for the moduli space of any simply-connected closed smooth 4-manifolds in any degree of homology.

To state our results, let us describe the standard setup to discuss homological stability for moduli spaces of manifolds.
Let $W$ be a compact smooth $2n$-dimensional manifold with non-empty boundary.
Let $\Diff_\del(W)$ denote the group of diffeomorphisms that are the identity near $\del W$.
We equip $\Diff_\del(W)$ with the $C^\infty$-topology.
One may define the {\it stabilization map}
\begin{align*}
s : \Diff_\del(W) \to \Diff_\del(W\#S^n\times S^n)
\end{align*}
as follows.
Form the connected sum of $S^n \times S^n$ with $W$ in the following way:
let $K = ([0,1] \times \del W)\#S^n\times S^n$ be the inner connected sum, and set
$W \# S^n\times S^n = W \cup_{\del W=\{0\} \times \del W} K$.
Then the map $s$ is defined by extending diffeomorphisms lying in $\Diff_\del(W)$ by the identity of $S^n \times S^n$.
This induces the map, called the stabilization map, between homologies of moduli spaces
\[
s_\ast : H_k(\BDiff_\del(W);\Z) \to H_k(\BDiff_\del(W\#S^n\times S^n);\Z)
\]
for every $k \geq 0$. 
If $W$ is obvious from the context, for $N\geq 0$, we use 
\[
s_{N,\ast} : H_k(\BDiff_\del(W\#N S^n \times S^n);\Z) \to H_k(\BDiff_\del(W\#(N+1) S^n \times S^n);\Z)
\]
to denote the stabilization map for $W\#N S^n \times S^n$.

The aforementioned results of Harer~\cite{Harer85} for $2n=2$ and of Galatius and Randal-Williams~\cite{Galatius2018} for $2n \geq 6$ state that, for any simply-connected $W$ and any $k$, the stabilization map $s_{N,\ast}$ an
is an isomorphism for any $N>0$ large enough relative to $k$. 
As a consequence, the direct system \[\{(H_k(\BDiff_{\del}(W\#NS^n\times S^n);\Z), s_{N,\ast})\}_{N=0}^\infty\]
stabilizes after finitely many terms. This stability phenomenon is fundamental in the study of the homology group of moduli spaces. It provides a \emph{stable range} in which one can identify $H_{k}(\BDiff_{\del}(W\# N S^n\times S^n))$ with the stable homology group \[\lim_{N \to +\infty}H_k(\BDiff_{\del}(W\# N S^n\times S^n)),\] which can be computed purely in terms of homotopy theory \cite{MW07,Galatius17}. (Note that this method of computing stable homology group is valid in all even dimensions including 4.)

A closely related object of interest is the diffeomorphism group with the discrete topology, denoted by $\Diff_{\partial}(W)^{\delta}$. The Eilenberg--MacLane space $\BDiff_\del(W)^{\delta}$ classifies smooth fiber bundles with flat structures. 
Homology of $\BDiff_\del(W)^{\delta}$ is the \emph{group homology} of $\Diff_\del(W)$ and it reflects the purely algebraic (rather than topological) structure of $\Diff_{\partial}(W)$.


One can similarly define the stabilization map 
\[
s^{\delta}_{N,\ast}: H_k(\BDiff_\del(W\#N S^n \times S^n)^{\delta};\Z) \to H_k(\BDiff_\del(W\#(N+1) S^n \times S^n)^{\delta};\Z).
\]
When $W$ is a punctured $g( S^{n}\times S^{n})$ with $n\neq 2$, Nariman~\cite{Nariman17a,Nariman17} proved that $s^{\delta}_{N,\ast}$ is an isomorphism for any $N$ large enough relative to $k$.

\subsection{Homological instability}
Now we present the first main result of this paper.
Let us focus on when $2n=4$ and $W$ is a punctured 4-manifold.
For a closed smooth 4-manifold $X$, choose an embedded 4-disk $D^4$ in $X$ and set $\mathring{X} = X \setminus \Int(D^4)$.
We begin with a concise version of the theorem, which implies that, unlike dimensions $\neq 4$, homological stability fails for the diffeomorphism group of any simply-connected closed 4-manifolds, with either the $C^\infty$-topology or the discrete topology, in any positive degree of homology:
\begin{thm}[\bf{Homological instability: concise version}]
\label{thm: simpler main thm}
Let $X$ be a simply-connected closed smooth 4-manifold and let $k>0$.
Then there are infinitely many positive integers $N$ such that the stabilization maps
\[
s_{N,\ast} : H_k(\BDiff_\del(\mathring{X}\#N S^2 \times S^2);\Z) \to H_k(\BDiff_\del(\mathring{X}\#(N+1) S^2 \times S^2);\Z)
\]
are not isomorphic.
An analogous result holds also for $s_{N,\ast}^\delta$.
\end{thm}
In fact, we shall prove a much more precise result. To state it, we let $\Homeo_\del(\mathring{X})$ denote the group of diffeomorphisms restricting to the identity near $\del \mathring{X}$ and let $i : \Diff_\del(\mathring{X}) \hookrightarrow \Homeo_\del(\mathring{X})$ be the inclusion maps.
We call an element of 
\[
\ker(i_\ast : H_\ast(\BDiff_\del(\mathring{X});\Z) \to H_\ast(\BHomeo_\del(\mathring{X});\Z))
\]
a {\it topologically trivial} homology class.
%We call a non-zero element of
%\[
%\ker(s_\ast : H_\ast(\BDiff_\del(\mathring{X});\Z) \to H_\ast(\BDiff_\del(\mathring{X}\#S^2\times S^2);\Z))
%\]
%or 
%\[
%\ker(s^{\delta}_\ast : H_\ast(\BDiff_\del(\mathring{X})^{\delta};\Z) \to H_\ast(\BDiff_\del(\mathring{X}\#S^2\times S^2)^{\delta};\Z))
%\]
%an {\it unstable} homology class. 
%An important topic in 4-dimensional topology is explore the difference between the smooth category and the topology category. For this purpose, 
The following is a more precise version of the main result of this paper. It establishes a sequence of 2-torsion, topologically trivial unstable homology classes.
Also, the ``non-isomorphic" result for infinitely many stabilization maps in \cref{thm: simpler main thm} will be refined to non-injectivity and non-surjectivity results for infinitely many stabilization maps:
\begin{thm}[\bf{Homological instability: precise version}]
\label{thm: main cal}
Let $X$ be a simply-connected closed smooth 4-manifold and let $k>0$.
Then there is a sequence of positive integers $N_1<N_2<\cdots \to +\infty$ and a sequence of nonzero, 2-torsion elements \[\alpha_{i}\in H_k(\BDiff_{\del}(\mathring{X}\#N_iS^2 \times S^2);\Z)\] that satisfy the following properties.
\begin{description}%[label=(\roman*)]
\item[(a) {[Topologically trivial]}] $\alpha_{i}$ is topologically trivial. 
\item[(b) {[Non-injective]}] $\alpha_{i}$ belongs to the kernel of the stabilization map $s_{N_{i},*}$. In particular, the map $s_{N_{i},*}$ is not injective.
\item[(c) {[Non-surjective]}] $\alpha_{i}$ does not belong to the image of the composition $$s_{N_{i}-k-1,*}\circ\cdots\circ s_{N_{i}-1,*}.$$ In particular, $s_{N_{i}-l,*}$ is not surjective for some $l\in \{1,\cdots,k+1\}$.
\item[(d) {[Discrete]}] $\alpha_{i}$ equals the image of a 2-torsion element 
$$
\alpha^{\delta}_{i}\in  H_k(\BDiff_{\del}(\mathring{X}\#N_i S^2 \times S^2)^{\delta};\Z)
$$
under the forgetful map
\[
H_k(\BDiff_{\del}(\mathring{X}\#N_iS^2 \times S^2)^{\delta};\Z)
\rightarrow H_k(\BDiff_{\del}(\mathring{X}\#N_iS^2 \times S^2);\Z).
\]
Furthermore, $\alpha^{\delta}_{i}$ belongs to the kernel of $s^{\delta}_{N_{i},*}$ and does not belong to the image of the composition $s^{\delta}_{N_{i}-k-1,*}\circ\cdots\circ s^{\delta}_{N_{i}-1,*}$.
\end{description}
\end{thm}




%In other words, for any simply-connected closed 4-manifold $X$ and any degree $k\geq 1$, the direct system $$
%H_k(\BDiff_{\del}(\mathring{X});\Z)
%\xrightarrow{s_\ast}   H_k(\BDiff_{\del}(\mathring{X}\# S^2\times S^2);\Z)
%\xrightarrow{s_\ast} \cdots
%$$
%never stabilizes after finitely many terms, and the failure of homological stability can be realized by a sequence of topologically trivial unstable homology classes.

%\begin{rmk}
%\textcolor{blue}{
%[Remove this as planned?]
%For topological moduli spaces, $\BHomeo_\del(W)$, a result of Kupers~\cite{kupers2015} establishes homological stability analogous to the smooth case \cite{Galatius2018} in dimension $\geq 6$.
%Note that in dimension 2, homological stability in the topological category follows from the smooth case \cite{Harer85}. (See, for example, \cite[Remark~1.1]{kupers2015}.) Interestingly, an upcoming work of Kupers--Powell [] proves the homological stability in dimension 4 and in the topological category. Hence the dimension 4 in the smooth category is the only case where homological stability fails.
%}
%\end{rmk}

%``stable homology group"
%\[
%\lim_{N \to +\infty}H_k(\BDiff_{\del}(\mathring{X}\#NS^2\times S^2);\Z)
%\]
%is {\it not} well-defined 
%\marginpar{Jianfeng: Actually, one can always define the stable homology group of the moduli space as the direct limit $\lim_{N \to +\infty}H_k(\BDiff_{\del}(\mathring{X}\#NS^2\times S^2);\Z)$. And Galatius-Randal-Williams were able to compute this group. What we prove here is that unlike other dimensions, this direct system never stabilize in dimension 4.}


\subsection{Smooth and topological moduli spaces}

\cref{thm: main cal}~(a) reflects the homological discrepancy of the topological and the smooth moduli spaces, which is of independent interest.
In fact, an analogous statement can be deduced for a closed 4-manifold $X$ instead of the punctured $\mathring{X}$, as summarized below.
Given an orientation of $X$, let $\Diff^+(X)$ and $\Homeo^+(X)$ denote the groups of orientation-preserving diffeomorphisms and homeomorphisms. As before, we use $i : \Diff^+(X) \hookrightarrow \Homeo^+(X)$ to denote the natural inclusion map and use $i_{*}$ and $i^{*}$ to denote the induced map on homology and cohomology respectively. We also use $i^{\delta}_{*}$ and $i_{\delta}^{*}$ to denote the corresponding maps for the diffeomorphism/homeomorphism groups with the discrete topologies.

\begin{thm}[\bf{BDiff vs. BHomeo for closed 4-manifolds}]
\label{thm: Diff Homeo sequence general}
Let $X$ be a simply-connected closed oriented smooth 4-manifold and let $k>0$.
Then there exists a sequence of positive integers $N_1<N_2<\cdots \to +\infty$ such that, for all $i$, none of the maps
\[
i_\ast : H_k(\BDiff^+(X\#N_i S^2 \times S^2);\Z)
\to H_k(\BHomeo^+(X\#N_iS^2 \times S^2);\Z)
\]
are injective and none of the maps 
\[
i^\ast : H^k(\BHomeo^+(X\#N_i S^2 \times S^2);\Z/2)
\to H^k(\BDiff^+(X\#N_iS^2 \times S^2);\Z/2)
\]
are surjective. 
An analogous result holds also for $i^{\delta}_{*}$ and $i^{*}_{\delta}$.
\end{thm}


\cref{thm: Diff Homeo sequence general} gives the first result that in any degree $k$ there is a 4-manifold $X$ such that the map $i_\ast : H_k(\BDiff^+(X))
\to H_k(\BHomeo^+(X))$ is not isomorphic, and a similar remark applies also to the statement for $i_\ast^{\delta}$. 
The cohomological part of the statement implies that in any positive degree $k$, there exists a non-topological characteristic class which is defined for \emph{all} orientable smooth fiber bundle whose fiber $X$ satisfies $b^{+}(X)>k$. It appears that characteristic classes with such properties were unknown before.
(See \cref{rem: relation with Rub and K,rmk: MMM class}.)
This also gives an explicit construction of a new unstable characteristic class for smooth 4-manifolds bundles (See \cref{subsection Intro SW char}).



\begin{rmk}
The results until here (\cref{thm: main cal,thm: Diff Homeo sequence general}) are closely related to an upcoming work by Auckly and Ruberman~\cite{AucklyRuberman}, which generalizes Ruberman's work \cite{Rub98,Rub99}.
A significant difference between \cite{AucklyRuberman} and this paper is which group of diffeomorphisms is considered.
While we consider $\Diff^+(X)$ in this paper, Auckly and Ruberman treat certain classes of proper subgroups $G(X)$ of $\Diff^+(X)$, including (subgroups of) the Torelli group 
\[
\TDiff(X) := \Set{f \in \Diff(X) | f_\ast=\id \text{ \ on\ } H_\ast(X;\Z)}.
\]
%They prove theorems analogous to \cref{thm: main cal,thm: Diff Homeo sequence general} for $G(X)$ and for some 4-manifolds $X$.
%The subgroup $G(X)$ can be taken, for example, to be the identity component $G(X)=\Diff_0(X)$ and the Torelli group $G(X)=\TDiff(X) := \Set{f \in \Diff(X) | f_\ast=\id \text{ \ on\ } H_\ast(X;\Z)}$.
%and the ``mod 2 Torelli group"  $G(X) = \Set{f \in \Diff(X) | f_\ast=\id \text{ \ on\ } H_\ast(X;\Z/2)}$.
Also, for $G(X) \subset \TDiff(X)$, they detect torsion free elements in the kernels of the stabilization maps $H_*(BG(X);\Z) \to H_*(BG(X\#S^2\times S^2);\Z)$ for some $X$, while we detected 2-torsions for the whole diffeomorphism group $\Diff^+(X)$.
\end{rmk}

It is worth noting that the choice of group of diffeomorphisms can make essential differences in the homology groups of moduli spaces.
For instance, there is an example of a 4-manifold $X$ for which the kernel of the natural map
\[
H_\ast(\BTDiff(X);\Z) \to H_\ast(\BDiff^+(X);\Z)
\]
contains a subgroup isomorphic to $\Z^\infty=\bigoplus_{\Z}\Z$ (\cref{pro: Torelli vs the whole strong}).


\subsection{Abelianizations of diffeomorphism groups}

Next we focus on homological instability in degree $k=1$, which is closely related to the abelianized mapping class group. Given $X$, we consider the smooth mapping class group $\MCG(X)=\pi_0(\Diff^+(X))$ and the topological mapping class group $\MCG_{\operatorname{Top}}(X)=\pi_0(\Homeo^+(X))$. We use $G_{\ab}$ to denote the abelianization of a group $G$. Then by definition, we have  
\[\begin{split}
H_{1}(\BDiff^+(X);\mathbb{Z})&=\MCG(X)_{\ab},\\
H_{1}(\BHomeo^+(X);\mathbb{Z})&=\MCG_{\operatorname{Top}}(X)_{\ab}.
\end{split} \]
By a series of deep results of Edwards--Kirby~\cite{Edwards71} and Thurston~\cite{Thurston74}, the unit components of $\Diff(X)$ and $\Homeo(X)$ are both simple. Therefore,  we also have 
\[\begin{split}
H_{1}(\BDiff^+(X);\mathbb{Z})&\cong \Diff^+(X)_{\ab},\\
H_{1}(\BHomeo^+(X);\mathbb{Z})&\cong \Homeo^{+}(X)_{\ab}.
\end{split}\]
Also note that when $X$ is simply-connected, the work of Freedman \cite{Fre82} and work of Quinn \cite{Q86} and Perron~\cite{P86} give the isomorphism
\begin{align}
\label{FreeQuinnPerron}
\MCG_{\operatorname{Top}}(X)\cong \Aut(Q_{X}).
\end{align}
Here $\Aut(Q_{X})$ denotes the group of automorphism on $H^{2}(X;\mathbb{Z})$ that preserves the intersection form $Q_{X}$.  Combining this isomorphism with a result of Wall \cite[Theorem 2]{Wall64D}, we see that the forgetful map
\begin{equation}\label{eq: MCG surjective}
\MCG(X\#N S^2 \times S^2)\rightarrow \MCG_{\operatorname{Top}}(X\#NS^2 \times S^2).    
\end{equation}
is always surjective when $N\geq 2$.  

Regarding the stabilization map for a closed manifold $X$, there is no well-defined stabilization map from $\BDiff^{+}(X)$ to $\BDiff^{+}(X\#S^2\times S^2)$ as there is no a fixed ball to form connected sums. But as proved in \cite[Theorem 5.3]{DKPR}, there is indeed a well-defined stabilization map \[\MCG(X)\rightarrow \MCG(X\#S^2\times S^2).\] 


With these results in mind, we can restate the $k=1$ case of \cref{thm: Diff Homeo sequence general,thm: main cal} as follows. 
\begin{thm}[\bf{Abelianizations of Diff and MCG}]
\label{thm: abelianisation noninjective}
Let $X$ be a simply-connected closed oriented smooth 4-manifold and let $k>0$.
Then there exists a sequence of positive integers $N_1<N_2<\cdots \to +\infty$ and a sequence of nonzero, 2-torsion elements $$\beta_{i}\in \Diff^+(X\#N_i S^2 \times S^2)_{\ab}\cong \MCG(X\#N_i S^2 \times S^2)_{\ab}$$
such that the following properties hold for all $i$. 
\begin{enumerate}[label=(\roman*)]
    \item $\beta_{i}$ can be represented by an exotic diffeomorphism (i.e. a diffeomorphism which is topologically isotopic to the identity but not smoothly so). In particular, $\beta_i$ belongs to the kernel of the map \[
\Diff^+(X\#N_i S^2 \times S^2)_{\ab}\rightarrow \Homeo^+(X\#N_iS^2 \times S^2)_{\ab}.
\]
\item $\beta_{i}$ belongs to the kernel of the map 
\[
\MCG(X\#N_i S^2 \times S^2)_{\ab}\rightarrow \MCG(X\#(N_i+1) S^2 \times S^2)_{\ab}.
\]
\item $\beta_{i}$ does not belong to the image of the map 
\[
\MCG(X\#(N_i-2) S^2 \times S^2)_{\ab}\rightarrow \MCG(X\#N_i S^2 \times S^2)_{\ab}.
\]
\end{enumerate}
\end{thm}
\cref{thm: abelianisation noninjective} also implies that for all simply-connected 4-manifolds, the degree-$1$ homological stability of smooth mapping class group fails. 


%We note that is a well-defined stabilization map 
%\[
%\MCG(X)\rightarrow \MCG(X\#S^{2}\times S^2).
%\]
%without puncturing $X$. It is defined by first isotoping a diffeomorphism so that it fix a small ball $D^{4}$ and then taking connected sum in its interior. 


\begin{ex} In dimension $\neq 4$, there are various tools to compute the mapping class groups and their abelianizations. In particular, let $W^{2n}_{g}=g(S^n\times S^n)$. Then $W^{2}_{g}$ is just a closed surface of genus $g$ and it is proved by Mumford \cite{Mumford67} and Powell \cite{Powell78} that
\[
\MCG(W^{2}_{g})_{\ab}\cong \begin{cases}
\mathbb{Z}/12 \quad &g=1,\\
\mathbb{Z}/10 \quad &g=2,\\
0 \quad &\text{otherwise}.
\end{cases}
\]
 For $n\geq 3$, Kreck  described the group $\MCG(W^{2n}_{g})$ up to extensional problems \cite{Kreck79}. Galatius and Randal-Williams computed $\MCG(W^{2n}_{g})_{\ab}$ for $n\geq 3$ and $g\geq 5$ \cite{Galatius16}. It is also proved in \cite{Galatius16} that
$
\operatorname{Aut}(Q_{W^{4}_{g}})_{\ab}\cong \mathbb{Z}/2\oplus \mathbb{Z}/2
$
whenever $g\geq 2$. Theorem \ref{thm: abelianisation noninjective} applied to $X=S^{4}$ shows that $\MCG(W^{4}_{g})$ is strictly larger than $\mathbb{Z}/2\oplus \mathbb{Z}/2$ for infinitely many $g$. For such $g$, the kernel of the surjective map 
$$
f_{\ab}:\MCG(W^{4}_{g})_{\ab}\rightarrow \mathbb{Z}/2\oplus \mathbb{Z}/2
$$ contains a 2-torsion element that is represented by an exotic diffeomorphism on $W^{4}_{g}$ and is not stabilized from a mapping class on $W^{4}_{g-2}$.
\end{ex}

\begin{rmk}\label{rmk: twisted stabilization} All of our results also hold for twisted stabilizations (i.e. taking connected sum with $\mathbb{CP}^{2}\#\overline{\mathbb{CP}^2}$ instead of $S^2\times S^2$). Actually, suppose one replaces $S^2\times S^2$ with  $\mathbb{CP}^{2}\#\overline{\mathbb{CP}^2}$ in the statements of \cref{thm: simpler main thm}, \cref{thm: main cal} and \cref{thm: abelianisation noninjective}. The conclusions of these results still hold. See \cref{rmk: twisted stablization details} for a detailed explanation.
\end{rmk}



\subsection{Moduli spaces of manifolds with metrics}
\label{subsection:Moduli spaces of manifolds with metrics}

Until here we considered moduli spaces of manifolds.
Now we discuss moduli spaces of pairs of manifolds and metrics, in particular with positive scalar curvature.
Given an oriented smooth manifold $X$, let $\scrR(X)$ and $\scrR^+(X)$ denote the space of Riemannian metrics and the space of metrics with positive scalar curvature, respectively.
The group $\Diff^+(X)$ acts on $\scrR(X)$ by pull-back, and this action preserves  $\scrR^+(X)$.
%As the actions of $\Diff^+(X)$ on $\scrR(X)$ and $\scrR^+(X)$ are not free in general, it is convenient to consider homotopy quotients rather than honest quotients:
Define $\scrM(X)$ and $\scrM^+(X)$ by 
\[
\scrM(X) = \EDiff^+(X) \times_{\Diff^+(X)} \scrR(X),\quad
\scrM^+(X) = \EDiff^+(X) \times_{\Diff^+(X)} \scrR^+(X).
\]

As we shall see in \cref{subsection Vanishing under positive scalar curvature},
$\scrM(X)$ classifies oriented fiber bundles with fiber $X$ equipped with fiberwise metrics, and similarly $\scrM^+(X)$ classifies fiber bundles with fiberwise positive scalar curvature metrics.
Let
$\iota : \scrM^+(X) \hookrightarrow \scrM(X)$
denote the injection induced from the inclusion $\scrR^+(X) \hookrightarrow \scrR(X)$.
The moduli space $\scrM(X)$ gives a model of $\BDiff^+(X)$, and $\iota$ is the same as the projection map $\scrM^+(X) \to \BDiff^+(X)$.
We shall prove the following comparison result:

\begin{thm}[\bf{Moduli spaces of manifolds with metrics vs. with psc metrics}]
\label{thm: psc main intro}
Let $X$ be a simply-connected closed oriented smooth 4-manifold and let $k>0$.
Then there exists a sequence of positive integers $N_1<N_2<\cdots \to +\infty$ such that the induced maps
\begin{align*}
\iota_\ast &: H_k(\scrM^+(X\# N_iS^2\times S^2);\Z) \to H_k(\scrM(X\# N_iS^2\times S^2);\Z)
\end{align*}
are not surjective for all $i$.
\end{thm}

We shall also prove analogous results for pointed 4-manifolds and punctured 4-manifolds in \cref{thm: psc main}, which are described in terms of observer moduli spaces $\scrM_{x_0}^+(X)$ and relative versions $\scrM_{\del}^+(\mathring{X})$.
See \cref{subsection Vanishing under positive scalar curvature} for the definition of these moduli spaces.

\begin{rmk}
If $X$ is spin and $\sign(X)\neq0$, then $X\# NS^2\times S^2$ do not admit positive scalar curvature metrics for all $N\geq0$. In this case, the claim of \cref{thm: psc main} is equivalent to the nonvanishing result $H_k(\scrM(X\# N_iS^2\times S^2);\Z)\neq 0$. 
On the other hand, if $X$ is either non-spin or $\sign(X)=0$, it follows from Wall's theorem \cite{Wall64} that $X\#N S^2 \times S^2$ admit positive scalar curvature metrics for all $N \gg 0$.
\end{rmk}

\cref{thm: psc main intro} gives the first result that in any degree $k$ there is a 4-manifold $X$ such that the map $\iota_\ast : H_k(\scrM^+(X))
\to H_k(\scrM(X))$ is not isomorphic.
Compared with higher dimensions, it is difficult to study the space of/moduli space of positive scalar curvature metrics on a 4-manifold because of the failure of surgery techniques.
Ruberman~\cite{Rub01} proved that $\scrR^+(X)$ are not connected for some 4-manifolds $X$.
In fact, the above non-surjectivity result (\cref{thm: psc main intro}) for $k=1$ implies that $\scrR^+(X\#N_i S^2 \times S^2)$ is disconnected (\cref{lem psc disconnected}).
Thus \cref{thm: psc main intro} can be regarded as a higher degree generalization of  Ruberman's result.
It is also worth noting that a recent work by Botvinnik and Watanabe~\cite{BW22} combined with Watanabe's work on the Kontsevich characteristic class~\cite{Wa18} implies a result for $D^4$ in a complementary direction to the above, detection of the image of (a relative version of) $\iota_\ast$.



\subsection{Tool: characteristic classes from Seiberg--Witten theory}\label{subsection Intro SW char} 
To derive the results explained until the last subsection,
we shall construct and calculate a new characteristic class, which is based on Seiberg--Witten theory. 
The setup is as follows.
For $k>0$, let $X$ be a closed oriented smooth 4-manifold with $b^{+}(X)>k+1$.
We shall construct characteristic classes
\begin{align*}
\SWbbtot^k(X) &\in H^k(\BDiff^+(X); \Z_{\EDiff^+(X)}),\\
\SWbbhalftot^k(X) &\in H^k(\BDiff^+(X); \Z/2),
\end{align*}
where $\Z_{\EDiff^+(X)}$ is a certain local coefficient system with fiber $\Z$, determined by the monodromy action on what is called the homology orientation.
We call $\SWbbtot^k(X)$ and $\SWbbhalftot^k(X)$ the {\it $k$-th total Seiberg--Witten characteristic class} and the {\it $k$-th half-total Seiberg--Witten characteristic class} respectively.

For some computational reason explained later, $\SWbbhalftot^k(X)$ will be used rather than $\SWbbtot^k(X)$ in our applications.
Here are notable properties of $\SWbbhalftot^k(X)$.
(Similar results hold also for $\SWbbtot^k(X)$, but we omit them here.)
\begin{itemize}
\item $\SWbbhalftot^k(X)$ is a cohomology class of the whole moduli space $\BDiff^{+}(X)$, rather than $\textnormal{B}G$ for a proper subgroup $G\subset\Diff^{+}(X)$. So this characteristic class can be defined for all oriented smooth bundles with fiber $X$, with no constraint on the monodromy.
This is a major difference from known gauge-theoretic characteristic classes \cite{K21}.
\item $\SWbbhalftot^k(X)$ is {\it unstable} under stabilizations by $S^2\times S^2$ (\cref{cor: vanishing}).
The non-surjectivity result in \cref{thm: main cal} is a consequence of this property.
\item Just as the Seiberg--Witten invariant detects exotic structures of 4-manifolds,  $\SWbbhalftot^k(X)$ detects subtle differences between smooth families of 4-manifolds.
This yields the non-injectivity result in \cref{thm: main cal} and the comparison results on the smooth and topological categories (\cref{thm: main cal} (a), \cref{thm: Diff Homeo sequence general}).
\item $\SWbbhalftot^k(X)$ can be non-trivial on $\BDiff^+(X)^\delta$. 
Precisely, the pull-back of $\SWbbhalftot^k(X)$ to $H^k(\BDiff^+(X)^\delta;\Z/2)$ under the map induced from the identity  $\Diff^+(X)^\delta \to \Diff^+(X)$ is non-zero for some $X$.
This makes it useful in the study of algebraic structure of the discrete group $\Diff^{+}(X)^{\delta}$ and yields homological instability for $\BDiff^+(X)^{\delta}$ (\cref{thm: main cal} (d)). 
\item $\SWbbhalftot^k(X)$ vanishes for families that admit fiberwise positive scalar curvature metrics (\cref{thm: vanishing by psc,cor: vanishing for fiberwise psc}). 
This yields a comparison result on moduli spaces of Riemannian metrics and of positive scalar curvature metrics (\cref{thm: psc main intro,thm: psc main}).
%\item $\SWbbhalftot^{k}(X)$ is non-trivial for some positive $k$ and some minimal algebraic surfaces. For example, as we will see in \cref{ex: K3}, the class 
%\[\SWbbhalftot^{2}(K3)\in H^{2}(\BDiff^{+}(K3);\mathbb{Z}/2)\]
%is nonzero. \textcolor{red}{This makes $\SWbbhalftot^{k}(X)$ potentially useful in the study of moduli space of algebraic surfaces.}
\end{itemize}


%One of notable properties of $\SWbbtot^k(X)$ and $\SWbbhalftot^k(X)$ is that they are {\it unstable} under stabilizations by $S^2\times S^2$ (\cref{cor: vanishing}).
%The non-surjectivity result in \cref{thm: main cal} is a consequence of this property.
%Also, as the Seiberg--Witten invariant detects exotic structures of 4-manifolds, it turns out that $\SWbbhalftot^k(X)$ detects subtle differences between smooth families of 4-manifolds.
%This yields the non-injectivity result in \cref{thm: main cal} and the comparison results on the smooth and topological categories (\cref{thm: main cal} (i), \cref{thm: Diff Homeo sequence general}).

From a technical point of view, the upshot of the characteristic classes $\SWbbtot^k(X)$ and $\SWbbhalftot^k(X)$ is that they are free from a choice of $\spinc$ structure, despite the dependence of the Seiberg--Witten equations on a $\spinc$ structure. 
The construction is modeled on a characteristic class defined by the first author~\cite{K21} based on the Seiberg--Witten equations, but the characteristic class in \cite{K21} does depend on the choice of a $\spinc$ structure.
To get rid of such constraint, we use an idea due to Ruberman~\cite{Rub01} to define a numerical Seiberg--Witten-type invariant of a diffeomorphism that does not necessarily preserve a given $\spinc$ structure.
See \cref{rem: relation with Rub and K} for a detailed comparison of this work with \cite{K21} and \cite{Rub01}.

Here is a rough idea of the definition of $\SWbbtot^k(X)$.
Let $d(\fraks)$ denote the formal dimension of the Seiberg--Witten moduli space for $\fraks$, and let $\Spinc(X,k)$ denote the set of $\spinc$ structures on $X$ with $d(\fraks)=-k$.
The diffeomorphism group $\Diff^+(X)$ naturally acts on $\Spinc(X,k)$.
%One may see that this action lifts to an action on the collection of the Seiberg--Witten equations $\SW_{\fraks}$ indexed by elements of $\Spinc(X,k)$.
Once we fix a pair $\sigma$ of a fiberwise metric and perturbation for the universal bundle $\EDiff^+(X) \to \BDiff^+(X)$, we have a family over $\BDiff^+(X)$ of ``collection (indexed by elements of $\Spinc(X,k)$) of the moduli spaces" of solutions to the Seiberg--Witten equations.
%over the classifying space:
%\[
%\EDiff^+(X) \times_{\Diff^+(X)}\left(\bigsqcup_{\fraks \in \Spinc(X,k)} \SW_{\fraks}\right) \to \BDiff^+(X).
%\]
%This yields a parameterized moduli space over $\BDiff^+(X)$ for the collection of the Seiberg--Witten equations.
We define a $k$-cochain $\SWcaltot^k(X,\sigma)$ on $\BDiff^+(X)$ by counting this family of collections of moduli spaces over each $k$-cell of $\BDiff^+(X)$.
This cochain shall be shown to be a cocycle and the cohomology class $\SWbbtot^k(X) = [\SWcaltot^k(X,\sigma)]$ will turn out to be independent of the choice of $\sigma$.

The other characteristic class $\SWbbhalftot^k(X)$ is a variant of $\SWbbtot^k(X)$,  corresponding to the quotient of all Seiberg--Witten moduli spaces divided by the charge conjugation on $\spinc$ structures.
For our purpose to prove the homological instability, $\SWbbhalftot^k(X)$ shall be effectively used because the conjugation symmetry annihilates $\SWbbtot^k(X)$ in a lot of situations. See \cref{subsection Motivation} for more detailed motivation for introducing $\SWbbhalftot^k(X)$.
 


%There are several computational reasons to consider $\SWbbhalftot^k(X)$, summarized as follows: We do not have control of the action of $\Diff^+(X)$ on $\Spinc(X,k)$ and on homology orientation, which pushes us to consider a $\Z/2$-coefficient invariant defined by the ``total" construction along Ruberman's idea \cite{Rub01}, summing up counts of moduli spaces over $\Spinc(X,k)$.
%However, the charge conjugation symmetry on $\Spinc(X,k)$ often kills mod 2 invariants, which leads us to consider the ``half" of $\Spinc(X,k)$ to get non-trivial examples.

\subsection{Outline}
The remaining sections of this paper are as follows.
In \cref{sec: The Seiberg--Witten characteristic classes}, we construct characteristic classes $\SWbbtot^\bullet(X), \SWbbhalftot^\bullet(X)$, and study some of the basic properties.
The definition is given in \cref{subsec cochain SWtot}, and subsections until there are devoted to giving technical preliminaries.
We prove the vanishing of these characteristic classes under stabilizations and positive scalar curvatures in \cref{subsectionVanishing under stabilizations,subsection Vanishing under positive scalar curvature}, respectively.
In \cref{sec Calculation} we calculate the characteristic class $\SWbbhalftot$.
A key computational result is \cref{thm key computation source}.
The main results of this paper, \cref{thm: main cal,thm: Diff Homeo sequence general,thm: abelianisation noninjective,thm: psc main intro}, are proven in \cref{subsectionProof of the main instability theorem}.
Main ingredients of the proofs of them are the key computation (\cref{thm key computation source}) and a result on geography involving the usual (i.e. unparameterized) Seiberg--Witten invariant, which is given as \cref{thm: 4-mfds that dissolves}.
\cref{construction of 4-manifolds} is devoted to proving \cref{thm: 4-mfds that dissolves}. 

\subsection{Acknowledgement} The authors would like to thank Dave Auckly, David Baraglia, Yi Gu, Sander Kupers, Ciprian Manolescu, Mark Powell, Oscar Randal-Williams, Danny Ruberman, and Tadayuki Watanabe for many enlightening discussions and comments on an earlier version of the paper.
H.~Konno was partially supported by JSPS KAKENHI Grant Numbers 19K23412 and 21K13785, and Overseas Research Fellowships.


\section{The Seiberg--Witten characteristic classes}
\label{sec: The Seiberg--Witten characteristic classes} 


\subsection{Motivation}
\label{subsection Motivation}
In this \lcnamecref{sec: The Seiberg--Witten characteristic classes}, given $k\geq0$ and a 4-manifold $X$ with $b^+(X)>k+1$, we define characteristic classes $\SWbbtot^k(X), \SWbbhalftot^k(X)$
mentioned in \cref{subsection Intro SW char}.
As the construction of $\SWbbhalftot^k(X)$ is rather complicated than that of $\SWbbtot^k(X)$,
here we clarify why we need the characteristic class $\SWbbhalftot^k(X)$ to motivate readers.
First we describe why we adopt (twisted) $\Z$-coefficient for  $\SWbbtot$. 
If one tries to define a characteristic class using the Seiberg--Witten equations in full generality, it would be natural to define a refinement $\widetilde{\mathbb{SW}}_{\mathrm{tot}}^k(X) \in H^k(\BDiff^+(X;\tilde{\Z}_E^\infty)$ of $\SWbbtot^k(X)$,
where $\tilde{\Z}_E^\infty$ is a local system with fiber $\bigoplus_{\fraks \in \Spinc(X,k)}\Z$ determined by the monodromy action for $\EDiff^+(X)$ on $\Spinc(X,k)$ and on the homology orientation.
The class $\SWbbtot$ can be recovered from $\widetilde{\mathbb{SW}}_{\mathrm{tot}}$ via a homomorphism 
$\bigoplus_{\fraks \in \Spinc(X,k)}\Z  \to \Z$ defined by summing up entries.
However, the action of $\Diff^+(X)$
on $\Spinc(X,k)$ may be quite complicated, and we do not have control of the local system $\tilde{\Z}_E^\infty$.
Thus we eventually pass to $\SWbbtot$ for a computational reason.

Moreover, even after passing to $\SWbbtot$, one still has to work with the local system $\Z_E=\Z_{\EDiff^+(X)}$ because the action of $\Diff^+(X)$ on the homology orientation is nontrivial in general. Since our main interest is the homology of moduli spaces for untwisted coefficient, we may hope to study $\SWbbtot$ after the mod 2 reduction.

However, $\SWbbtot$ corresponds to summing up the counts of the moduli spaces over $\Spinc(X,k)$, so almost every time the charge conjugation symmetry of the Seiberg--Witten equations kills the $\Z/2$-coefficient $\SWbbtot$.
Now we arrive at the reason why we need to take the ``half" of $\Spinc(X,k)$, namely $\Spinc(X,k)/\Conj$, to get interesting results for a broad class of 4-manifolds.

Nevertheless, in this paper we shall also describe the construction of  $\SWbbtot$ not only $\SWbbhalftot$, because the construction of $\SWbbtot$ gives a good guide in the construction of the more complicated object $\SWbbhalftot$.





\subsection{Virtual neighborhoods for families}
\label{subsection: Virtual neighborhoods for families}

To define the characteristic classes $\SWbbtot^\bullet(X)$, $\SWbbhalftot^\bullet(X)$ for a 4-manifold $X$,
we need to treat the issue of transversality in counting parameterized moduli spaces over $\BDiff^+(X)$, which is not a smooth manifold, just a CW complex; at least there is no canonical choice of structure of (infinite-dimensional) smooth manifold.
This may cause a complication, in particular, in proving that cochains we construct to define $\SWbbtot^\bullet(X), \SWbbhalftot^\bullet(X)$ are cocycles (corresponding to \cref{SWtot cocycle}).
To bypass this transversality issue, we shall adopt virtual neighborhood technique along Ruan~\cite{Ru98}.
This technique is easily generalized to a parameterized setup \cite[Subsection~5.2]{K21}.
For the reader's convenience,
in this \lcnamecref{subsection: Virtual neighborhoods for families}, we summarize necessary facts on virtual neighborhoods for families along \cite[Subsection~5.2]{K21}.

\begin{rmk}
Another potential way to handle the transversality issue could be taking a particular model of $\BDiff^+(X)$ that is an infinite-dimensional manifold such as the space of embeddings of $X$ into $\R^\infty$ divided by $\Diff^+(X)$ as in,  e.g., \cite{Galatius2018}, but the authors never checked details along this line.
\end{rmk}

\begin{convention}
In this paper, unless otherwise stated, a {\it section} of a continuous fiber bundle over a topological space means a continuous section.
\end{convention}

We first fix our notation.
Recall that the Seiberg--Witten equations give rise to a Fredholm section of a certain Hilbert bundle over a certain Hilbert manifold.
We shall consider a parameterized version of this setup, and abstract it as follows (see \cite[Subsection~5.2]{K21} for the precise setup).
The objects we are going to introduce are summarized in the diagram \eqref{diagram for parameterized Fred section}:
\begin{equation}
\label{diagram for parameterized Fred section}
\vcenter{
\xymatrix{
    \scrE \ar[dd]\ar[rd]&{}\\
    {}& \scrX \ar[ld]^{\pi} \ar@/_1pc/[lu]_{s}\\
    B. &
}
}
\end{equation}
Let us explain the notations in \eqref{diagram for parameterized Fred section}.
First, $B$ is a normal space, which is used as the base space or parameter space in this \lcnamecref{subsection: Virtual neighborhoods for families}.
Next, $\scrX = \bigcup_{b \in B}\scrX_b$ is a continuous (locally trivial) family of Hilbert manifolds over $B$ with the projection $\pi : \scrX \to B$, and $\scrE$ be a parameterized Hilbert bundle over $B$.
%Precisely, $\pi : \scrX \to B$ is a continuous fiber bundle whose fiber $\scrX_b = \pi^{-1}(b)$ over each point $b \in B$ is a (paracompact Hausdorff) Hilbert manifold whose model Hilbert space is separable.
Namely, $\scrE = \bigcup_{b \in B}\scrE_b$ is a continuous (locally trivial) family of vector bundles $\scrE_b \to \scrX_b$ with Hilbert space fiber $\scrH_b$.
Finally, $s : \scrX \to \scrE$ be a parameterized Fredholm section.
Namely, $s$ is a continuous map commuting with the projections onto $B$ such that, at each $b \in B$, the restriction $s_b : \scrX_b \to \scrE_b$ of $s$ is a smooth Fredholm section.

Set $\calM_b = s_b^{-1}(0)$ for each $b \in B$, called the {\it unparameterized moduli space} at $b$.
Let $\calM$ denote the union of the unparameterized moduli space:
\[
\calM
= \bigcup_{b \in B}\calM_b.
\]
We call $\calM$ the {\it parameterized moduli space}.

\begin{assum}
Throughout this \lcnamecref{subsection: Virtual neighborhoods for families}, suppose that $\calM$ is compact.
\end{assum}

\begin{comment}
At each zero point $x \in \calM$, 
one can consider the differential 
\[
d(s_{\pi(x)})_x : T_x \scrX_{\pi(x)} \to \scrH_{\pi(x)}.
\]
Here, to get a map to $\scrH_{\pi(x)}$,
we used the fact that there is a canonical projection from $T_{s(x)}\scrE_{\pi(x)}$ to $\scrH_{\pi(x)}$ at a zero point of $s$.
Henceforth, we make the following assumptions:

\begin{assum}
\label{assum: compact parameterized moduli}
We assume the following:
\begin{itemize}
    \item The parameterized moduli space $\calM$ is compact.
    \item The Fredholm index of $d(s_{\pi(x)})_x$ is independent of $x \in \calM$.
    \item The differential $ds_b$ continuously depends on $b$.
    Namely, $\bigsqcup_{b\in B}T\scrX_b$ and $\bigsqcup_{b\in B}T\scrE_b$ form continuous fiber bundles over $B$, and the section of $\mathrm{Hom}(\bigsqcup_{b\in B}T\scrX_b, s^\ast \bigsqcup_{b\in B}T\scrE_b) \to B$ induced by $ds$ is continuous.    
\end{itemize}
\end{assum}

Note that, if we considered a continuous family of Seiberg--Witten equations parameterized by a compact base space, \cref{assum: compact parameterized moduli} is satisfied.
\end{comment}

Under this assumption, one may construct a (typically non-locally trivial) family of smooth manifolds
\[
\calU = \bigcup_{b \in B}\calU_b
\]
over $B$ such that $\calM$ is embedded in the interior of $\calU$ for a natural topology on $\calU$.
We call $\calU$ a {\it families virtual neighborhood} for $\calM$ (or for $s$) \cite[Definition~5.10]{K21}.

We shall briefly sketch the construction of a families virtual neighborhood below.
The first step is to find a sufficiently large natural number $N$ and a fiberwise smooth map
\[
\varphi : \scrX \times \R^N \to \scrE
\]
that satisfies that $\scrX \times \{0\} \subset \varphi^{-1}(0)$ and the following property: set $\tilde{s} = s+\varphi$.
Then, for each $x \in \calM$, the differential
\[
d\left(\tilde{s}_{\pi(x)}\right)_{(x,0)} : T_x\scrX_{\pi(x)} \oplus \R^N \to \scrH_{\pi(x)}
\]
is surjective.


Such $N$ and $\varphi$ can be found by using the compactness of $\calM$: at each point $x \in \calM$, one can easily find a natural number $N_x$ and a linear map $f_x : \R^{N_x} \to \scrH_{\pi(x)}$ such that $ds_{\pi(x)}+f_x : T\scrX_{\pi(x)} \oplus \R^{N_x}\to \scrH_{\pi(x)}$ is surjective.
Since the surjectivity is an open condition, we may take an open neighborhood $U_x$ of $x$ in $\scrX$ so that, for all $y \in U_x \cap \calM$, $ds_{\pi(y)}+f_x$ are surjective.
By the compactness of $\calM$, we may cover $\calM$ with finitely many $U_x$.
Let $N$ be the sum of $N_x$ for such $x$, and then we may obtain $\varphi$ above by summing the pull-back of $f_x$ to $\scrX \times \R^N$ with multiplying cut-off functions on $U_x$.

Again since the surjectivity is an open condition, we may find an open neighborhood $\scrN$ of $\calM \times \{0\}$ in $\scrX \times \R^N$ such that the fiber-direction differential of $\tilde{s}_{(\pi(x),0)}$ is surjective at any point $x \in \tilde{s}^{-1}(0) \cap \scrN$.
We set
\[
\calU = \calU(s, N, \varphi, \scrN) := \tilde{s}^{-1}(0) \cap \scrN,
\]
which is the desired virtual neighborhood.
Denote by $\calU_b$ the fiber of $\calU$ over $b \in B$ under the obvious projection $\calU \to B$.
By the implicit function theorem, for each $b$, $\calU_b$ is a smooth manifold of dimension $\ind(ds_b) + N$.
Clearly, $\calU$ is a neighborhood of $\calM \times \{0\}$ in $\scrX \times \R^N$, and for each $b \in B$, $\calU_b$ is a neighborhood of $\calM_b$ in $\scrX_b \times \R^N$.

Note that $\calU$ is not uniquely determined by $s$: there are various choices of $N, \varphi$, and $\scrN$.
Note also that $\calU$ is not necessarily a locally trivial family over $B$.
For example, $\calU$ is supported only over $\pi(\calM) \subset B$: if we have $\calM_b=\emptyset$ for $b \in B$, then (a small neighborhood of $\calM_b$ in) $\calU_b$ is also empty. 

Whereas $\calU$ is not determined by $s$, we may extract a well-defined cohomology class on $B$ through virtual neighborhoods.
To describe this, first we consider the {\it integration along fiber} for the family $\calU \to B$.
Suppose that the index of $d(s_{\pi(x)})_x$ 
is independent of $x \in \calM$ and denote by $\ind{s}$ the index.
Let $\det{ds} \to \scrX$ denote the parameterized determinant line bundle of the parameterized linearized Fredholm sections $ds$.
Let $\calL$ be the local system  on $\scrX$ with fiber $\Z$ given as the orientation local system of $\det{ds}$.
Pick a point $b \in B$, and suppose that the monodromy action of $\pi_1(\scrX_b)$ on a fiber of $\det{ds_b}$  is trivial.
In this case, $\calL$ is isomorphic to the pull-back of a local system over $B$ along $\scrX \to B$, which we denote also by $\calL$.
%Also, let us use the same notation $\calL$ also for the pull-back of $\calL$ to $\scrX \times \R^N$ along the projection $\scrX \times \R^N \to \scrX$.
Then, while $\calU \to B$ is not necessarily locally trivial, we may define the integration along fibers for the family $\calU \to B$ (see \cite[explanations below Equation (21)]{K21}):
\[
\pi_! : H^\ast_\cpt(\calU; \Z) \to H^{\ast - (\ind{s}+N)}_\cpt(B; \calL),
\]
where $\pi_!$ is a map between compactly supported cohomology groups with $\calL$-coefficient.
Note that $\ind{s}+N$ coincides with the dimension of $\calU_b$ for any $b \in B$.

Next, we consider a certain relative Euler class.
Let $h_\calU : \calU \to \R^N$ denote the `height' function, defined as the restriction of the projection $\scrX \times \R^N \to \R^N$ to $\calU \subset \scrX \times \R^N$.
The parameterized moduli space $\calM$ is the level set for $h_{\calU}$ of height zero: $\calM \times \{0\} = h_\calU^{-1}(0)$.
The map $h_\calU: \calU \to \R^N$ gives rise to a section of the trivial bundle $\calU \times \R^N \to \calU$, denoted by the same notation, $h_\calU : \calU \to \calU \times \R^N$.
Consider the relative Euler class for $h_\calU$, namely
\[
e_\calU := h_\calU^\ast \tau(\calU \times \R^N \to \calU) \in H^N_\cpt(\calU; \Z),
\]
where $\tau$ denotes the Thom class.
The reason why $e_\calU$ is obtained as a compactly supported cohomology is that $\calM$ was supposed to be compact.

Now, for a given cohomology class $\alpha \in H^\ast(\scrX; \Z)$, we define
\[
\frakM(s, \alpha) := \pi_!(e_\calU \cup \alpha) \in H^{\deg{\alpha} - \ind{s}}_\cpt(B; \calL),
\]
where the pull-back of $\alpha$ to $\scrX \times \R^N$ is also denoted by the same symbol.
This cohomology class $\frakM(s, \alpha)$ depends only on $s$ and $\alpha$, and independent of the choice of $\calU$ \cite[Lemma~5.15]{K21}.
Also, if $\calM=\emptyset$, we have $\frakM(s, \alpha)=0$.

%If $B$ is a compact oriented smooth manifold, $\calL$ is a trivial local system, and $s$ is a generic section so that $\calM$ is a smooth manifold, then $\frakM(s, \alpha)$ corresponds to the cohomology class integrated over $\calM$, so symbolically:
%\[
%\int_\calM \alpha = \frakM(s, \alpha).
%\]
To our purpose in this paper, we shall take $\alpha$ to be just the constant function $1 \in H^0(\scrX;\Z)$, and mainly consider the situation that $\ind{s}<0$.
In this case, if $B$ is a smooth oriented closed manifold of dimension $-\ind{s}$ and $\calL$ is a trivial local system, and further $s$ is generic so that $\calM$ is a (0-dimensional) smooth manifold,
the evaluation of $\frakM(s, 1)$ on the fundamental class $[B]$ corresponds to the count of $\calM$ with signs:
\[
\#\calM = \left< \frakM(s, 1), [B] \right>.
\]

As described in \cite{K21}, the above construction immediately generalizes to a relative setup.
Suppose that, over a subset $B' \subset B$, the section $s$ is nowhere vanishing.
Then, given $\alpha \in H^\ast(\scrX;\Z)$, one may obtain a cohomology class
\[
\frakM(s, \alpha; B') \in H^{\deg{\alpha} - \ind{s}}_\cpt(B, B'; \calL)
\]
through a virtual neighborhood $\calU$, which is, however, independent of the choice of $\calU$ as well.

We shall use the following naturality result in the construction of our characteristic class,
which was given in \cite[Corollary~5.19]{K21} (there is an obvious typo in \cite[Corollary~5.19]{K21}, corrected below):

\begin{lem}
\label{lem: first naturality}
Let $\scrX \to B, \scrE \to \scrX \to B, s : \scrX \to \scrE$ be as above, described at the beginning of this \lcnamecref{subsection: Virtual neighborhoods for families}.
Let $A$ be a normal space and $f : A \to B$ be a continuous map.
Suppose that, for the pull-back section $f^\ast s : f^\ast \scrX \to f^\ast \scrE$,
the parameterized zero set $(f^\ast s)^{-1}(0)$ is also compact as well as $s^{-1}(0)$.
Then, for every cohomology class $\alpha \in H^\ast(\scrX;\Z)$, we have
\[
f^\ast \frakM(s, \alpha)
= \frakM(f^\ast s, \bar{f}^\ast\alpha)
\]
in $H^{\deg{\alpha}-\ind{s}}_\cpt(A;f^\ast\calL)$, where $\bar{f} : f^\ast\scrX \to \scrX$ is the natural map that covers $f$. 
Similarly, for subsets $A' \subset A$, $B'\subset B$ with $f(A') \subset B'$, if $s$ is nowhere vanishing over $B'$, we have
\[
f^\ast \frakM(s, \alpha; B')
= \frakM(f^\ast s, \bar{f}^\ast\alpha; A')
\]
in $H^{\deg{\alpha}-\ind{s}}_\cpt(A, A';f^\ast\calL)$.
\end{lem}

\subsection{Family of collections of Hilbert bundles}
\label{subsectionFamily of collections of Hilbert bundles}

For the purpose of this paper, it is important to note that all constructions in \cref{subsection: Virtual neighborhoods for families} can be generalized to a Fredholm section of a `family of collections' of Hilbert bundles, defined as follows.
To avoid confusion, throughout we shall use the word `family' for the case that the parameter space is not discrete, and the word `collection' for the case that the parameter space is discrete.

Let $\Lambda$ be a set equipped with the discrete topology. 
(In our application to Seiberg--Witten theory, we shall finally take $\Lambda$ to be the set of $\spinc$ structures on a 4-manifold with a fixed formal dimension.)
Suppose that we are given a collection of (unparameterized) Hilbert bundles over Hilbert manifolds:
\[
(\scrE_\lambda^0 \to \scrX_\lambda^0)_{\lambda \in \Lambda}.
\]
(The superscript ``$0$" indicates that the object is unparameterized.)
An {\it automorphism} of the collection $(\scrE_\lambda^0 \to \scrX_\lambda^0)_{\lambda}$ consists of a bijection $\alpha : \Lambda \to \Lambda$ and a collection of isomorphisms of Hilbert bundles 
\[
\xymatrix{
    \scrE_\lambda^0 \ar[r]\ar[d] & \scrE_{\alpha(\lambda)}^0\ar[d]\\
    \scrX_\lambda^0 \ar[r] & \scrX_{\alpha(\lambda)}^0.
}
\]
Let $\Aut\left((\scrE_\lambda^0)_\lambda\right)$ denote the set of all automorphisms of $(\scrE_\lambda^0 \to \scrX_\lambda^0)_{\lambda}$, which forms a group under composition.
We call an $\Aut\left((\scrE_\lambda^0)_\lambda\right)$-bundle a {\it family of collections of Hilbert bundles} modeled on $(\scrE_\lambda^0 \to \scrX_\lambda^0)_\lambda$.

Let $\scrE \to \scrX \to B$ be a family over a normal space $B$ of collections of Hilbert bundles modeled on $(\scrE_\lambda^0)_\lambda$.
For each $b \in B$, let $\scrX_b$ denote the fiber of $\scrX \to B$ over $b$, and similarly define $\scrE_b$.
Then, on each $b \in B$, we have a pair of uncanonical identifications
\begin{align}
\label{eq: indentifications}
\scrX_b \cong \bigsqcup_{\lambda \in \Lambda}\scrX_\lambda^0,\quad
\scrE_b \cong \bigsqcup_{\lambda \in \Lambda}\scrE_\lambda^0,
\end{align}
and the set of choices of identification is given by $\Aut\left((\scrE_\lambda^0)_\lambda\right)$.
Note that we have a natural map $\scrE \to \scrX$ induced from the collection of the projections $\scrE_\lambda^0 \to \scrX_\lambda^0$.
Thus we have the notions of a (parameterized) section $s : \scrX \to \scrE$, which is a family of sections $s_b : \scrX_b \to \scrE_b$.
Similarly, we have the notion that $s$ is (fiberwise) Fredholm.
This is equivalent to say that, for each $b \in B$,
once we fix an identification \eqref{eq: indentifications}, 
the induced sections $(s_\lambda^0)_b : \scrX_\lambda^0 \to \scrE_\lambda^0$ are Fredholm for all $\lambda$.


\begin{comment}
We shall consider a locally trivial family whose fiber is modeled on the collection $(s_\lambda^0 : \scrX_\lambda^0 \to \scrE_\lambda^0)_{\lambda \in \Lambda}$ as follows.
For $\lambda, \lambda' \in \Lambda$, an {\it isomorphism}
\[
\Phi_{\lambda \lambda'} :
(s_\lambda^0 : \scrX_\lambda^0 \to \scrE_\lambda^0)
\to (s_{\lambda'}^0 : \scrX_{\lambda'}^0 \to \scrE_{\lambda'}^0)
\]
from $s_\lambda$ to $s_{\lambda'}$ is an isomorphism of Hilbert bundles 
\[
\xymatrix{
    \scrE_\lambda^0 \ar[r]^{\Phi}\ar[d] & \scrE_{\lambda'}^0\ar[d]\\
    \scrX_\lambda^0 \ar[r]^{\phi} & \scrX_{\lambda'}^0
}
\]
that commutes with $s_\lambda, s_{\lambda'}$, namely $\Phi \circ s_{\lambda} = s_{\lambda'} \circ \phi$.
An {\it automorphism} of the collection $(s_\lambda^0)_{\lambda}$ consists of a bijection $\alpha : \Lambda \to \Lambda$ and a collection of isomorphisms 
\[
\left(\Phi_{\lambda \alpha(\lambda)} :
(s_\lambda^0 : \scrX_\lambda^0 \to \scrE_\lambda^0)
\to (s_{\alpha(\lambda)}^0 : \scrX_{\alpha(\lambda)}^0 \to \scrE_{\alpha(\lambda)}^0)\right)_{\lambda \in \Lambda}.
\]
Let $\Aut\left((s_\lambda^0)_\lambda\right)$ denote the set of all automorphisms of the collection $(s_\lambda^0)_\lambda$, which forms a group under composition.
We call an $\Aut\left((s_\lambda^0)_\lambda\right)$-bundle a {\it family of collection of Fredholm sections} modeled on $(s_\lambda^0)_\lambda$.

Let $s : \scrX \to \scrE$ be a family of collection of Fredholm sections modeled on $(s_\lambda^0)_\lambda$ over a base space $B$.
For each $b \in B$, let $\scrX_b$ denote the fiber of $\scrX \to B$ over $b$, and similarly define $\scrE_b$ and $s_b$.
Then, on each $b \in B$, we have uncanonical identifications
\[
\psi : \scrX_b \cong \bigsqcup_{\lambda \in \Lambda}\scrX_\lambda^0,\quad
\Psi : \scrE_b \cong \bigsqcup_{\lambda \in \Lambda}\scrE_\lambda^0
\]
such that $\Psi, \psi$ commute with $s_b$ and $(s_\lambda^0)_\lambda$.

For each $\lambda \in \Lambda$, suppose that we are given a parametrized Fredholm section $s_\lambda : \scrX_\lambda \to \scrH_\lambda$ of a parametrized Hilbert bundle $\scrH_\lambda$ over a parametrized Hilbert manifold $\scrX_\lambda$.
Here $\scrX_\lambda, \scrH_\lambda, s_\lambda$ are parameterized over a normal space $B$, which is common to all $\lambda$, and they satisfy the assumptions on $\scrX, \scrH, s$ described at the beginning of this \lcnamecref{subsection: Virtual neighborhoods for families}.
\end{comment}

Let $s :\scrX \to \scrE$ be a parameterized Fredholm section.
For each $b \in B$, set $\calM_{b} = s_b^{-1}(0)$.
Again, if we fix an identification \eqref{eq: indentifications} and use the induced sections, there is a homeomorphism
\[
\calM_{b} 
\cong \bigsqcup_{\lambda \in \Lambda}(s_\lambda^0)_b^{-1}(0).
\]
Define
\[
\calM := 
\bigcup_{b \in B}\calM_{b} \subset \scrX.
\]

Suppose that $\calM$ is compact, and the Fredholm indices of the linearizations of $s_\lambda$ to the fiber-direction are independent of $\lambda$.
Then we may repeat the constructions of a virtual neighborhood $\calU$ for $s$ and the cohomology class $\frakM(s, \alpha)$ and \cref{lem: first naturality} explained in \cref{subsection: Virtual neighborhoods for families} word for word.



\subsection{Collection of Seiberg--Witten equations}
\label{subsection Collection of Seiberg--Witten equations}

We shall apply the abstract machinery explained until \cref{subsectionFamily of collections of Hilbert bundles} to the Seiberg--Witten equations.
First we fix our notation on Seiberg--Witten equations.
Let $X$ be a closed oriented smooth 4-manifold.
As in \cite[Section~2]{K21}, we define a $\spinc$ structure on $X$ using the double covering of $\mathrm{SL}(4,\R)$ instead of $\mathrm{Spin}(4)$ to avoid using a Riemannian metric on $X$.
Once  we take a metric on $X$, then a $\spinc$ structure in our sense gives rise to a $\spinc$ structure in the usual sense, and the isomorphism classes of $\spinc$ structures in our sense bijectively correspond to those of $\spinc$ structures in the usual sense.
Let $\Spinc(X)$ denote the set of isomorphism classes of $\spinc$ structures on $X$.

Given a Riemannian metric $g$ and a $\spinc$ structure $s$ on $X$, let $L$ denote the determinant line bundle of $s$ and let $S^+(X,s,g)$ and $S^-(X,s,g)$ denote the positive and negative spinor bundles respectively.
Let $\Lambda^+_g(X)$ denote the self-dual part of $\Lambda^2(X)$ for the metric $g$.
Fix $l > 1$, and let $\scrA(X,s,g)_{L^2_l}$ and $L^2_l(S^\pm(X,s,g))$ denote the space of $U(1)$-connections of $L$ and sections of $S^+(X,s,g)$ completed by the $L^2_l$-norm.
Set
\begin{align*}
 &\scrC(X,s,g) := \scrA(X,s,g)_{L^2_l} \times L^2_l(S^+(X,s,g)),\\
 &\scrC^\ast(X,s,g) := \scrA(X,s,g)_{L^2_l} \times (L^2_l(S^+(X,s,g)) \setminus \{0\}),\\
 &\scrD(X,s,g) := L^2_{l-1}(i\Lambda^+_g(X)) \times L^2_{l-1}(S^-(X,s,g)).
\end{align*}
Let $\scrR(X)$ denote the space of Riemannian metrics on $X$ and define
\[
\Pi_\ast(X) = \bigcup_{g \in \scrR(X)} L^2_{l-1}(i\Lambda^+_g(X)),
\]
which is a Hilbert bundle over $\scrR(X)$.
(We shall consider a subspace of $\Pi_\ast(X)$ with a boundedness condition, which will be denoted by $\Pi(X)$ dropping $\ast$.)
Let $\pi : \Pi_\ast(X) \to \scrR(X)$ denote the natural projection.
Set
\[
\circPi_\ast(X) = \bigcup_{g \in \scrR(X)} L^2_{l-1}(i\Lambda^+_g(X)) \setminus \mathrm{Im}(d^+_g : L^2_{l}(i\Lambda^1(X)) \to L^2_{l-1}(i\Lambda^+_g(X))).
\]
This space $\circPi_\ast(X)$ is a subspace of $\Pi_\ast(X)$ of codimension-$b^+(X)$.
Since $\Pi_\ast(X)$ is contractible, $\circPi_\ast(X)$ is $(b^+(X)-2)$-connected.

For each $\mu \in \Pi_\ast(X)$,
let 
\[
\tilde{s}_\mu : \scrC(X,s,\pi(\mu)) \to \scrD(X,s,\pi(\mu))
\]
be the map corresponding to the $\mu$-perturbed Seiberg--Witten equations:
\[
\tilde{s}_\mu(A,\Phi)
= (F^{+_{\pi(\mu)}}_A - \mu - \sigma(\Phi, \Phi), D_A\Phi),
\]
where $\sigma(\Phi, \Phi)$ denotes the quadratic term for spinors in the Seiberg--Witten equations and $D_A\Phi$ is the $\spinc$ Dirac operator.
Set $\scrG = L^2_{l+1}(X, U(1))$ and 
\begin{align*}
&\scrB(X,s,g) = \scrC(X,s,g)/\scrG,\\
&\scrB^\ast(X,s,g) = \scrC^\ast(X,s,g)/\scrG,\\
&\scrE(X,s,g) = (\scrC^\ast(X,s,g) \times \scrD(X,s,g))/\scrG,
\end{align*}
then $\scrE(X,s,g) \to \scrB^\ast(X,s,g)$ is a Hilbert bundle with fiber $\scrD(X,s,g)$.
Since $\tilde{s}_\mu$ is $\scrG$-equivariant, it gives rise to a section
\[
s_\mu : \scrB^\ast(X,s,\pi(\mu)) \to \scrE(X,s,\pi(\mu)),
\]
which is a Fredholm section of index
\[
d(s) = \frac{1}{4}(c_1(s)^2 -2\chi(X) -3\sign(X)).
\]

Let $\Aut(X,s)$ denote the automorphism group of a $\spinc$ 4-manifold $(X,s)$ and $\fraks$ denote the isomorphism class of $\fraks$.
Then have an exact sequence
\[
1 \to \Map(X,U(1)) \to \Aut(X,s) \to \Diff^+(X).
\]
The image of the last map is given as $\Diff(X, \fraks)$, the group of diffeomorphisms that preserve the isomorphism class $\fraks$.
Let $g$ be a metric on $X$ and let $s'$ be a $\spinc$ structure $s'$ on $X$ that is isomorphic to $s$.
Once we choose an isomorphism from $s$ to $s'$, we have an isomorphism
\begin{align}
\label{eq: iso scrC}
\scrC(X,s,g) \to \scrC(X,s',g)
\end{align}
of Hilbert manifolds, and it gives rise to an isomorphism
\begin{align}
\label{eq: iso scrB}
\scrB^\ast(X,s,g) \to \scrB^\ast(X,s',g)
\end{align}
of Hilbert manifolds.
While the isomorphism \eqref{eq: iso scrC} depends on the choice of the isomorphism from $s$ to $s'$, the isomorphism \eqref{eq: iso scrB} between quotients by $\scrG$ is canonically determined by $s$ and $s'$, since the choice of the isomorphism from $s$ to $s'$ is in $\scrG$.
The same remark also applies to $\scrE(X,s,g)$.
Hence we write $\scrB^\ast(X,\fraks,g)$ and $\scrE(X,\fraks,g)$ for $\scrB^\ast(X,s,g)$ and $\scrE(X, s,g)$ respectively.

Now we consider the collection of the above objects indexed by $\fraks$.
Set
\begin{align*}
&\scrB^\ast(X) = \bigsqcup_{\fraks \in \Spinc(X)}
\bigcup_{g \in \scrR(X)}
\scrB^\ast(X,\fraks,g),\\
&\scrE(X) = \bigsqcup_{\fraks \in \Spinc(X)}
\bigcup_{g \in \scrR(X)}
\scrE(X,\fraks,g).
\end{align*}
Here $\bigcup_{g \in \scrR(X)}
\scrB^\ast(X,\fraks,g)$ is regarded as a locally trivial family of Hilbert manifolds over $\scrR(X)$, and $\scrB^\ast(X)$ is a disjoint union of such families of Hilbert manifolds.
We note that $\Diff^+(X)$ acts on $\scrB^\ast(X)$ and on $\scrE(X)$ in a natural way.
To see this, let $f \in \Diff^+(X)$, $\fraks \in \Spinc(X)$ and $g \in \scrR(X)$.
If we take a representative $s$ of $\fraks$ and an isomorphism $\tilde{f} : f^\ast s \to s$ that makes the following diagram commutes
\[
\xymatrix{
    f^\ast s \ar[r]^{\tilde{f}}\ar[d] & s \ar[d]\\
    X \ar[r]^f & X,
}
\]
we obtain an isomorphism
\begin{align}
\label{eq: pullback}
f^\ast : \scrB^\ast(X,\fraks,g) \to \scrB^\ast(X,f^\ast\fraks,f^\ast g)
\end{align}
of Hilbert manifolds.
However, by the same reason that the isomorphism \eqref{eq: iso scrB} is canonically determined by $s$ and $s'$, the isomorphism \eqref{eq: pullback} is independent of the choice of $s$ and $\tilde{f}$.
Thus we obtain a canonical isomorphism \eqref{eq: pullback} determined only by $f$, and hence $f$ yields a canonical isomorphism of $\scrB^\ast(X)$, a disjoint union of families of Hilbert manifolds.
Thus $\Diff^+(X)$ acts on $\scrB^\ast(X)$, and similarly also on $\scrE(X)$ so that the projection $\scrE(X) \to \scrB^\ast(X)$ is $\Diff^+(X)$-equivariant.

Also, we note that the above action of $\Diff^+(X)$ is compatible with the Fredholm section corresponding to the Seiberg--Witten equations.
Notice that we have a natural map $\scrE(X) \to \scrB^\ast(X)$, and the sections $s_\mu : \scrB^\ast(X,s,\pi(\mu)) \to \scrE(X,s,\pi(\mu))$ yield a section
\[
s : \scrB^\ast(X) \to \scrE(X)
\]
along this natural map.
Since $s_\mu$ was induced from a $\scrG$-equivariant map, again by the same reason that the canonicality of \eqref{eq: pullback}, $s$ is $\Diff^+(X)$-equivariant under the above actions on $\scrB^\ast(X), \scrE(X)$.


For a $\spinc$ structure $\fraks$ on $X$, let $d(\fraks)$ denote the formal dimension of the Seiberg--Witten moduli space for $\fraks$:
\begin{align*}
d(\fraks)
= \frac{1}{4}(c_1(\fraks)^2 -2\chi(X) -3\sign(X)).
\end{align*}
Given an integer $k$, let $\Spinc(X,k)$ denote the set of $\spinc$ structures $\fraks$ on $X$ with $d(\fraks)=-k$.
We define subsets $\scrB^\ast(X, k), \scrE(X, k)$ of $\scrB^\ast(X), \scrE(X)$ to be
\begin{align*}
&\scrB^\ast(X,k) = \bigsqcup_{\fraks \in \Spinc(X,k)}
\bigcup_{g \in \scrR(X)}
\scrB^\ast(X,\fraks,g),\\
&\scrE(X,k) = \bigsqcup_{\fraks \in \Spinc(X,k)}
\bigcup_{g \in \scrR(X)}
\scrE(X,\fraks,g).
\end{align*}
Then $\Diff^+(X)$ acts also on $\scrB^\ast(X,k), \scrE(X,k)$ as well, and the section $s^k : \scrB^\ast(X,k) \to \scrE(X,k)$ obtained as the restriction of the above $s$ is $\Diff^+(X)$-equivariant.

From now on, we shall consider a parameterized setup. Let $X \to E \to B$ be a fiber bundle with fiber $X$ with structure group $\Diff^+(X)$.

\begin{nota}
\label{notation data for bundle}
Let $A(X,a)$ be an object determined by $X$ and a datum $a$, and suppose that $A(X,a)$ is acted by $\Diff^+(X)$.
We denote by the associated bundle with $E$ with fiber $A(X,a)$ by 
\[
A(X,a) \to A(E,a) \to B.
\]
\end{nota}

For example, the all objects above, which are acted by $\Diff^+(X)$, give rise to fiber bundles associated with $E$:
\begin{align*}
\scrB^\ast(X,k) \to \scrB^\ast(E,k) \to B,&\\
\scrE(X,k) \to \scrE(E,k) \to B,&\\
\Pi_\ast(X) \to \Pi_\ast(E) \to B,&\\
\circPi_\ast(X) \to \circPi_\ast(E) \to B,&\\
\scrR(X) \to \scrR(E) \to B.&
\end{align*}

\begin{rmk}
The notation $\scrR(E)$ does not mean ``the space of metrics on the total space $E$". 
(Note that the total space of $E$ does not have a natural manifold structure if $B$ is not a manifold.)
Instead, a section of $\scrR(E) \to B$ corresponds to a fiberwise metric on $E$.
Similar remarks apply to many families over $B$ that shall appear later.
\end{rmk}

We denote the fiber of each bundle over a point $b \in B$ by, for example, $\scrB^\ast(E_b,k)$.
Also, since $s^k : \scrB^\ast(X,k) \to \scrE(X,k)$ is $\Diff^+(X)$-equivariant, this induces a section
\[
s^k : \scrB^\ast(E,k) \to \scrE(E,k)
\]
that commutes with the projections $\scrB^\ast(E,k) \to B$, $\scrE(E,k) \to B$.

To get some compactness later, we consider bounded perturbations.
Note that the fiber of $\Pi_\ast(E) \to B$ over a point $b \in B$ is of the form
\[
(\Pi_\ast(E))_b 
= \bigcup_{g \in \scrR(E_b)} L^2_{l-1}(i\Lambda^+_g(E_b)).
\]
Let $\mathring{D}(i\Lambda^+_g(E_b))$ denote the open ball centered at the origin of radius 1 with respect to the $L^2_{l-1}$-norm.
Then we obtain a subset $\Pi(E)$ of $\Pi_\ast(E)$ defined by
\[
\Pi(E) = \bigcup_{b \in B}
\bigcup_{g \in \scrR(E_b)} \mathring{D}(i\Lambda^+_g(E_b)).
\]
We also set 
\[
\circPi(E) = \circPi_\ast(E) \cap \Pi(E).
\]

%Let $g_E$ be a section of $\scrR(E) \to B$, i.e. a fiberwise metric of $E$.

Now we shall choose a fiberwise metric and self-dual 2-forms on $E$.
Recall that we denoted by $\pi$ the projection $\Pi_\ast(X) \to \scrR(X)$.
This gives rise to a map (denoted by the same symbol) $\pi : \Pi_\ast(E) \to \scrR(E)$.
Restricting this, we have a map $\pi : \circPi(E) \to \scrR(E)$.
Taking a section $\sigma : B \to \circPi(E)$ corresponds to taking a fiberwise metric and a fiberwise self-dual 2-form (with fiberwise boundedness) on $E$ that continuously vary over $B$.
Let us consider the projections $\scrB^\ast(X,k) \to \scrR(X)$ and $\scrE(X,k) \to \scrR(X)$.
For example, the fiber of the projection $\scrB^\ast(X,k) \to \scrR(X)$ over a metric $g \in \scrR(X)$ is the collections (over $\Spinc(X,k)$) of all configurations for the common metric $g$.
These projections give rise to natural maps
\[
p_{\scrB} : \scrB^\ast(E,k) \to \scrR(E),\quad
p_{\scrE} : \scrE(E,k) \to \scrR(E).
\]
We set
\[
\scrB^\ast(E,k,\sigma)
:= p_{\scrB}^{-1}(\pi\circ\sigma(B)),\quad
\scrE^\ast(E,k,\sigma)
:= p_{\scrE}^{-1}(\pi\circ\sigma(B)).
\]

The above section $s^k : \scrB^\ast(E,k) \to \scrE(E,k)$ corresponds to a union of families over $B$ of Seiberg--Witten equations over all fiberwise metrics and fiberwise self-dual 2-forms.
Restricting this to the image of 
$\sigma$, we obtain a section
\[
s^k_{\sigma} : \scrB^\ast(E,k,\sigma) \to \scrE(E,k,\sigma).
\]

The resulting section $s^k_{\sigma}$ is a family of collections of Hilbert bundles in the sense of \cref{subsectionFamily of collections of Hilbert bundles}.

\begin{defi}
Set
\begin{align}
\label{eq: SWtot moduli}
\calM_{\sigma} := \bigcup_{b \in B}(s^k_{\sigma})_{b}^{-1}(0).
\end{align}
We call $\calM_{\sigma}$ the {\it family of total collections of moduli spaces} associated with $\sigma$.
\end{defi}

\begin{lem}
\label{lem: finiteness1}
If $B$ is compact, $\calM_{\sigma}$ is also compact.
\end{lem}

\begin{proof}
Recalling that we use a common fiberwise metric $\pi \circ \sigma$ for all $\spinc$ structures to define $\calM_\sigma$, and $\circPi(E)$ consists of bounded perturbations, the claim follows from the standard compactness of the Seiberg--Witten moduli space: the compactness of the moduli space for a fixed $\spinc$ structure and the finiteness of the numbers of $\spinc$ structures for which the Seiberg--Witten moduli spaces are non-empty.
\end{proof}

Thus we may apply the virtual neighborhood technique of \cref{subsection: Virtual neighborhoods for families} to $s_{\sigma}$, as explained in \cref{subsectionFamily of collections of Hilbert bundles}.
In particular, we obtain a cohomology class
\begin{align}
\label{eq: coh compact base tot}
\frakM(s^k_{\sigma},1) \in H^{k}(B; \Z_E).
\end{align}
Here $\Z_E$ is a local system over $B$ with fiber $\Z$ that is determined by the monodromy action for $E$ on the homology orientation, the orientation of $H^1(X;\R) \oplus H^+(X)$, where $H^+(X)$ denotes a maximal-dimensional positive-definite subspace of $H^2(X;\R)$ with respect to the intersection form.
Similarly, if the section $s^k_{\sigma}$ is nowhere vanishing over a subset $B' \subset B$, we have a relative cohomology class
\begin{align}
\label{eq: coh compact base tot rel}
\frakM(s^k_{\sigma}, 1; B') \in H^{k}(B, B'; \Z_E).
\end{align}

\begin{rmk}
In \cref{subsec cochain SWtot}, given an oriented closed 4-manifold $X$, we shall construct a cohomology class $\SWbbtot^\bullet(X) \in H^\bullet(\BDiff^+(X))$ using cohomology class \eqref{eq: coh compact base tot} (precisely, we use \eqref{eq: coh compact base tot rel} in a certain relative setup).
The reason why we cannot define $\SWbbtot^\bullet(X)$ to be just \eqref{eq: coh compact base tot} is that we need the compactness of the parameterized moduli space to get \eqref{eq: coh compact base tot}, while we cannot hope this for a family over $B=\BDiff^+(X)$.
To resolve this compactness issue, we shall combine the cohomology class \eqref{eq: coh compact base tot} (or \eqref{eq: coh compact base tot rel}) with an inductive construction using a CW structure of $\BDiff^+(X)$, which shall be introduced in \cref{subsection Inductive sections}.
\end{rmk}




\subsection{Charge conjugation}
\label{subsection Complex conjugation}

In this \lcnamecref{subsection Complex conjugation}, we shall consider the charge conjugation on the Seiberg--Witten equations in our context, which is a preliminary to the half-total characteristic class $\SWbbhalftot^\bullet(X)$.
Henceforth, when there is no risk of confusion, we call an isomorphism class $\fraks \in \Spinc(X)$ just a $\spinc$ structure.

Let $X$ be a closed oriented smooth 4-manifold.
Recall that, for each $\fraks \in \Spinc(X)$, there is a unique $\spinc$ structure $\bar{\fraks}$ on $X$ called the charge conjugation of $\fraks$, which satisfies $c_1(\bar{\fraks})=-c_1(\fraks)$.
The group $\Z/2$ acts on $\Spinc(X)$ by the charge conjugation.
Let $\Spinc(X)/\Conj$ denote the quotient.

Let $\fraks \in \Spinc(X)$, and
take a metric $g$ on $X$ and a representative $s$ of the isomorphism class $\fraks$.
Let $\bar{s}$ be the charge conjugation of $s$.
The charge conjugation induces isomorphisms between Hilbert (affine) spaces
\[
c : \scrA(X,s,g)_{L^2_l} \to \scrA(X,s,g)_{L^2_l},\quad
c : L^2_l(S^\pm(X,s,g)) \to L^2_l(S^\pm(X,\bar{s},g)).
\]
Let $(-1) : \Omega^+_g(X) \to \Omega^+_g(X)$ be the $(-1)$-muptiplication.
Then the Seiberg--Witten equations are compatible with these maps in the sense that the following diagram commutes for each $\mu \in L^2_{l-1}(i\Lambda^+_g(X))$:
\[
\xymatrix{
    \scrC(X,s,g) \ar[r]^{\tilde{s}_\mu} \ar[d]_{c \times c} & \scrD(X,s,g) \ar[d]^{(-1) \times c} \\
\scrC(X,\bar{s},g) \ar[r]^{\tilde{s}_{-\mu}} & \scrD(X,\bar{s},g). 
}
\]
This implies that the diagram
\[
\xymatrix{
    \scrB^\ast(X,s,g) \ar[r]^{s_\mu} \ar[d] & \scrE(X,s,g) \ar[d] \\
\scrB^\ast(X,\bar{s},g) \ar[r]^{s_{-\mu}} & \scrE(X,\bar{s},g)
}
\]
commutes, where the vertical maps are the isomorphisms induced from the charge conjugation.

Now we consider collections over $\spinc$ structures.
Fix $k \in \Z$ henceforth.
To avoid unnecessary complications, let us make the following assumption for a while:

\begin{assum}
\label{assum: no spin structure}
Assume that $\Spinc(X,k)$ does not contain a $\spinc$ structure that comes from a spin structure.
\end{assum}

Under this assumption, $\Z/2$ acts freely on $\Spinc(X)$.
In \cref{rmk: if spin exists} we describe how to remove this assumption.

%We shall use the following notation: the subscript `ns' indicates an object which is a union over $\Spincns(X)$ or $\Spincns(X,k)$, such as
%\begin{align*}
%&\scrB^\ast(X) = \bigsqcup_{\fraks \in \Spincns(X)}
%\bigcup_{g \in \scrR(X)}
%\scrB^\ast(X,\fraks,g),\\
%&\scrB^\ast(X,k) = \bigsqcup_{\fraks \in \Spincns(X,k)}
%\bigcup_{g \in \scrR(X)}
%\scrB^\ast(X,\fraks,g).
%\end{align*}
Then the charge conjugation (together with the $(-1)$-multiplication) makes $\scrE(X) \to \scrB^\ast(X)$ a $\Z/2$-equivariant bundle, and the above observation implies that
$s : \scrB^\ast(X) \to \scrE(X)$ is a $\Z/2$-equivariant section.
Similarly, for $k>0$, $s^k : \scrB^\ast(X,k) \to \scrE(X,k)$ is also a $\Z/2$-equivariant section.

Set 
\begin{align*}
\scrB^\ast(X,k)' 
&:= (\scrB^\ast(X,k))/(\Z/2),\\
\scrE(X,k)' 
&:= (\scrE(X,k))/(\Z/2),\\
\circPi(X,k)' &:= (\Spincns(X,k) \times \circPi(X))/(\Z/2).
\end{align*}
The diffeomorphism group $\Diff^+(X)$ acts on these three spaces via the pull-back, since the pull-back commutes with both of the charge conjugation and the $(-1)$-multiplication.
Note that all of these three spaces are equipped with the natural forgetful map onto $\Spincns(X,k)/\Conj$.
Also, $\Diff^+(X)$ acts on $\Spincns(X,k)/\Conj$ via the pull-back.
Thus we obtain fiber bundles associated with $E$ whose fibers are these objects, denoted by
\begin{align*}
\scrB^\ast(X,k)' \to
\scrB^\ast(E,k)' \to B,&\\
\scrE(X,k)' \to
\scrE(E,k)' \to B,&\\
\circPi(X,k)' \to \circPi(E,k)' \to B,&\\
\Spinc(X,k)/\Conj \to \Spinc(E,k)/\Conj \to B.&
\end{align*}

\begin{comment}
We shall use the following notation:
let $X \to E \to B$ be a $\Diff^+(X)$-bundle over a CW complex $B$, and $Z$ be a space acted by $\Diff^+(X)$.
Then we define
\[
Z_E := E \times_{\Diff^+(X)} Z.
\]
Note that there is a natural map $Z_E \to B$ and $Z_E$ is a $\Diff^+(X)$-bundle with fiber $Z$ along this map.
\end{comment}

Now, the forgetful map $\scrB^\ast(X,k)' \to \Spincns(X,k)/\Conj$
gives rise to a surjection
\begin{align*}
\scrB^\ast(E,k)'
\to \Spincns(E,k)/\Conj,
\end{align*}
and similarly, we have surjections
\begin{align}
\scrE(E,k)'
\to \Spincns(E,k)/\Conj,\nonumber\\
\label{eq: map with connected fiber}
\circPi(E,k)'
\to \Spincns(E,k)/\Conj.
\end{align}
Evidently, all of these maps commute with the projections onto $B$.

Since the natural map $\scrE(X,k) \to \scrB^\ast(X,k)$ is $\Diff^+(X)$-equivariant and $\Z/2$-equivariant, this induces a map
\[
\scrE(E,k)' \to \scrB^\ast(E,k)'
\]
that commutes both with the projections onto $B$ and projections onto $\Spincns(E,k)/\Conj$.
Moreover, since $s^k$ is  $\Diff^+(X)$-equivariant too, and also $\Z/2$-equivariant as seen above, it induces a section
\[
s^k_{\mathrm{half}} : \scrB^\ast(E,k)' \to \scrE(E,k)'
\]
that commutes both with the projections onto $B$ and projections onto $\Spincns(E,k)/\Conj$.

%Note that the fiber of the map \eqref{eq: map with connected fiber} is given by $\circPi(X)$, and hence is $(b^+(X)-2)$-connected.
%This ensures that we may take a section of the map \eqref{eq: map with connected fiber} over
%\[
%\left.
%(\Spincns(E,k)/\Conj)
%\right|_{B^{(b^+(X)-1)}},
%\]
%where we regard $\Spincns(E,k)/\Conj$ as a covering space of $B$ with fiber $\Spincns(X,k)/\Conj$ and $B^{(b^+(X)-1)}$ denotes the $(b^+(X)-1)$-skeleton of $B$.
%(In this paper, we allow a non-connected space as the total space of a covering space.)

Just like $\pi, p_{\scrB}, p_{\scrE}$, we have natural maps
\[
\pi': \circPi(E,k)' \to \scrR(E), \quad
p_{\scrB}' : \scrB^\ast(E,k)' \to \scrR(E),\quad
p_{\scrE}' : \scrE(E,k)' \to \scrR(E).
\]
Let $\widetilde{g}: B\rightarrow \scrR(E)$ be a section and let 
\[
\sigma' : \Spincns(E,k)/\Conj
\to \circPi(E,k)'
\]
be a section that fits into the following diagram.
\begin{align}
\label{eq: diagram half-tot section}
\begin{split}
\xymatrix{
\Spincns(E,k)/\Conj\ar[r]^-{\sigma'}\ar[d]& \circPi(E,k)'\ar[d] \\
B\ar[r]^{\widetilde{g}} & \scrR(E)}
\end{split}
\end{align}
%Then $\sigma'$ automatically commutes with the projections onto $B$.
We set
\[
\scrB^\ast(E,k,\sigma')'
:= (p_{\scrB}')^{-1}(\widetilde{g}),\quad
\scrE^\ast(E,k,\sigma)
:= (p_{\scrE}')^{-1}(\widetilde{g}).
\]
%\[
%\scrB^\ast(E,k,\sigma')'
%:= (p_{\scrB}')^{-1}(\pi' \circ\sigma'(B)),\quad
%\scrE^\ast(E,k,\sigma)
%:= (p_{\scrE}')^{-1}(\pi' \circ\sigma'(B)).
%\]
Then the restriction of $s^k_{\mathrm{half}}$ gives rise to a section
\[
(s^k_{\mathrm{half}})_{\sigma'} : \scrB^\ast(E,k,\sigma')' \to \scrE(E,k,\sigma')'.
\]

The section $(s^k_{\mathrm{half}})_{\sigma'}$ constructed now corresponds to the collection of the families Seiberg--Witten equations for the `half' of $\Spinc(X,k)$, namely $\Spinc(X,k)/\Conj$.


\begin{defi}
Set 
\[
\calM_{\sigma', \mathrm{half}} := \bigcup_{b \in B}((s^k_{\mathrm{half}})_{\sigma'})_{b}^{-1}(0).
\]
\end{defi}

Just as in \cref{lem: finiteness1}, we have:

\begin{lem}
If $B$ is compact, $\calM_{\sigma', \mathrm{half}}$ is also compact.
\end{lem}

We call $\calM_{\sigma', \mathrm{half}}$ the {\it family of half collections of moduli spaces} associated with $\sigma'$.
Thus we obtain a cohomology class
\begin{align}
\label{eq: half coh 1}
\frakM((s^k_{\mathrm{half}})_{\sigma'},1) \in H^{k}(B; \Z/2)
\end{align}
through the virtual neighborhood technique,
as well as $\frakM(s^k_{\sigma},1)$.
If the section $(s^k_{\mathrm{half}})_{\sigma'}$ is nowhere vanishing over a subset $B' \subset B$, we have also a relative cohomology class
\begin{align}
\label{eq: half coh 2}
\frakM((s^k_{\mathrm{half}})_{\sigma'}, 1; B') \in H^{k}(B, B'; \Z/2).
\end{align}
Note that we exclude $\widetilde{g}$ in our notation because it is determined by $\sigma'$.


\begin{rmk}
\label{rem: why mod 2}
In addition to a computational motivation,
one reason why here we consider $\Z/2$-coefficient cohomology rather than $\Z_H$-coefficient is as follows.
As explained,
for a $\spinc$ structure $s$, the charge conjugation induces an isomorphism
\begin{align}
\label{eq: isom base conj}
\scrB^\ast(X,s,g)
\to \scrB^\ast(X,\bar{s},g)
\end{align}
of Hilbert manifolds.
Corresponding to the determinant line of the linearization of the Seiberg--Witten equations, we have trivial real line bundles $L_{s} \to \scrB^\ast(X,s,g)$ and $L_{\bar{s}} \to \scrB^\ast(X,\bar{s},g)$.
A choice of the homology orientation determines trivializations of $L_{s}$ and $L_{\bar{s}}$, which correspond to the orientations of the moduli spaces for $s$ and $\bar{s}$ respectively.
Although the isomorphism \eqref{eq: isom base conj} naturally lifts to an isomorphism between $L_{s}$ and $L_{\bar{s}}$, the trivialization of $L_s$ and that of $L_{\bar{s}}$ does not necessarily correspond via this lifted isomorphism.
(This causes that the Seiberg--Witten invariant for $s$ equals that for $\bar{s}$ {\it only up to sign} in general.)
Therefore the collection over $\Spincns(X,k)$ of the trivializations of $L_s$'s does not descend to the collection over $\Spincns(X,k)/\Conj$ in general.
As a consequence of this, the pull-back of $\Z_E$ to $\Spincns(E,k)/\Conj$ along a natural map $\Spincns(E,k)/\Conj \to B$ does not necessarily give an appropriate local system over $\Spincns(E,k)/\Conj$ that coherently takes into account the collection of the orientations of the moduli spaces.
\end{rmk}

Finally, we discuss how to remove \cref{assum: no spin structure}:

\begin{rmk}
\label{rmk: if spin exists}
Let us suppose that there are spin structures on $X$ with formal dimension $-k$.
In this case, let us decompose $\Spinc(X,k)$ into two parts:
\[
\Spinc(X,k) = \Spinc_0(X,k) \sqcup \Spinc_1(X,k).
\]
Here $\Spinc_0(X,k)$ denotes the subset of $\Spinc(X,k)$ consisting of $\spinc$ structures coming from spin structures, and $\Spinc_1(X,k)$ is the complement of $\Spinc_0(X,k)$ in $\Spinc(X,k)$.
Note that the $\Z/2$-action on $\Spinc(X,k)$ preserves this decomposition, and $\Z/2$ acts trivially on $\Spinc_0(X,k)$.
Thus we can carry out the construction of \cref{subsection Collection of Seiberg--Witten equations} for collections over $\Spinc_0(X,k)$, and the construction of this \lcnamecref{subsection Complex conjugation} for collections over $\Spinc_1(X,k)$.
Thus we may define the moduli space $\calM_{\sigma', \mathrm{half}}$ and the cohomology classes \eqref{eq: half coh 1} and \eqref{eq: half coh 2} as well.
\end{rmk}


\subsection{Inductive families perturbations}
\label{subsection Inductive sections}

In the constructions of the Seiberg--Witten characteristic classes,
we shall use a specific way to construct families perturbations.
We summarize it in this \lcnamecref{subsection Inductive sections} with basic examples.

Throughout this \lcnamecref{subsection Inductive sections}, we fix $k>0$ and let $X$ be a closed oriented smooth 4-manifold with $b^+(X) \geq k+2$.
Let $B$ be a CW complex, and let $X \to E \to B$ be a fiber bundle with structure group $\Diff^+(X)$.
For an $i$-cell $e \subset B$, let $\varphi_{e} : D_{e}^{i} \to \bar{e} \subset B$ denote the characteristic map of $e$, where $D^{i}_{e}$ is the standard $i$-dimensional disk indexed by $e$.

First, as in \cite[Subsection~6.1]{K21},
we inductively construct a section
\[
\sigma^{(i)} : B^{(i)} \to \circPi(E)|_{B^{(i)}}
\]
for $i \leq k+1$ as follows.
Choosing a generic point in $\circPi(E_{b})$ for each $b \in B^{(0)}$, we have $\sigma^{(0)} : B^{(0)} \to \circPi(E)|_{B^{(0)}}$.
Assume that we have constructed $\sigma^{(i-1)} : B^{(i-1)} \to \circPi(E)|_{B^{(i-1)}}$ for $i \leq k+1$ such that $\calM_{\sigma^{(i-1)}} = \emptyset$, where $\calM_{\sigma^{(i-1)}}$ is the family of total collections of moduli spaces associated with $\sigma^{(i-1)}$, introduced in \eqref{eq: SWtot moduli}.
Let $e$ be an $i$-cell of $B$.
Since the pull-back bundle $\varphi_{e}^{\ast} \circPi(E) \to D_{e}^{i}$ under the characteristic map is trivial, there is a trivialization 
\[
\psi_{e} : \varphi_{e}^{\ast} \circPi(E) \to D_{e}^{k} \times \circPi(X).
\]
Let $p : \del D_{e}^{i} \times \circPi(X) \to \circPi(X)$ denote the projection.
Since $b^{+}(X) \geq k+2$, we can extend the continuous map 
\[
p \circ \psi_{e} \circ (\varphi_{e}|_{\del D_{e}^{i}})^{\ast}\sigma^{(i-1)}
: \del D_{e}^{i} \to (\varphi_{e}|_{\del D_{e}^{i}})^{\ast} \circPi(E) \to \del D_{e}^{i} \times \circPi(X) \to \circPi(X)
\]
to a map from $D_{e}^{i}$ into $\circPi(X)$, rather than $\Pi_\ast(X)$, which corresponds to choosing a family of self-dual 2-forms avoiding the wall.
This extended map gives rise to a section of $\circPi(E)|_{\bar{e}} \to \bar{e}$.
We may choose the extension generically so that $\calM_{\sigma^{(i)}|_{e}} = \emptyset$ for $i<k$ (see \cref{rmk: generic ext} below).
Thus we obtain a section $\sigma^{(i)} : B^{(i)} \to \Pi(E)|_{B^{(i)}}$ for $i \leq k+1$, which satisfies that $\calM_{\sigma^{(i)}} = \emptyset$ if $i<k$.
We call such $\sigma^{(\bullet)}$ an {\it inductive families perturbation} (called an inductive section in \cite{K21}).

\begin{rmk}
\label{rmk: generic ext}
Let us explain why one can extend a {\it continuous} map
\[
\Phi := p \circ \psi_{e} \circ (\varphi_{e}|_{\del D_{e}^{i}})^{\ast}\sigma^{(i-1)}
: \del D_{e}^{i} \to \circPi(X)
\]
to a generic map 
\[
D_{e}^{i} \to \circPi(X)
\]
so that the parameterized moduli space over $D_{e}^{i}$ is empty for $i < k$.
First, recall that the parameterized moduli space over $\del D_{e}^{i}$ was supposed to be empty.
Take a smooth map $\Phi' : \del D_{e}^{i} \to \circPi(X)$ that is close enough to $\Phi$.
Since $\Phi'$ is smooth, one can find a smooth and generic extension of $\Phi'$ to a map $\tilde{\Phi}' : D_{e}^{i} \to \circPi(X)$ as usual.

On the other hand, the emptiness of the moduli space is an open condition,
we may suppose that there is a continuous homotopy $\Phi_t$ between $\Phi$ and $\Phi'$ such that the moduli space is empty for all $\Phi_t$.
Pick a collar neighborhood $N(\del D_{e}^{i})$ of $\del D_{e}^{i}$ in $D_{e}^{i}$ and define a continuous extension of $\Phi$ to a map $\hat{\Phi} : N(\del D_{e}^{i}) \to \circPi(X)$ using $\Phi_t$ by assigning $t$ to the radial coordinate.
Gluing $\hat{\Phi}$ with (rescaled) $\tilde{\Phi}'$, we may obtain an extension of $\Phi$ to $D_{e}^i$ for which the parameterized moduli space over $D_{e}^{i}$ is empty for $i < k$.
\end{rmk}

The above way to construct an inductive section shall be used to define the total characteristic class $\SWbbtot^\bullet$.
We also describe a similar construction to define the half-total characteristic class $\SWbbhalftot^\bullet$.
In this case, we need to construct a section (denoted by the same symbol as the above case)
\[
\sigma'^{(i)} : \left(\Spincns(E,k)/\Conj\right)^{(i)}
\to \circPi(E,k)'
\]
for $i \leq k+1$ inductively. %\marginpar{Jianfeng: We first inductively construct $\widetilde{g}$. Then construct $\sigma'$ inductively.}
First, note that $\Spincns(E,k)/\Conj$ is a covering space of $B$ with fiber $\Spincns(X,k)/\Conj$ with projection the natural map $\Spincns(E,k)/\Conj \to B$.
We equip $\Spincns(E,k)/\Conj$ with the CW complex structure induced from that of $B$: the lifted CW complex structure along the map $\Spincns(E,k)/\Conj \to B$.
For an $i$-cell $e \subset B$ with the characteristic map $\varphi_e : D^i_e \to B$, let
\[
(e_{[\fraks]})_{[\fraks] \in \Spincns(X,k)/\Conj}
\]
be the corresponding lifted $i$-cells of $\Spincns(E,k)/\Conj$ with the characteristic maps
\[
\hat{\varphi}_{e_{[\fraks]}} : D^i_e \to \Spincns(E,k)/\Conj,
\]
which are lifts of $\varphi_e$.
Note that the fiber of the natural map $\circPi(E,k)' \to \Spincns(E,k)/\Conj$ is given by $\circPi(X)$.
Also, a trivialization $\psi_{e} : \varphi_{e}^{\ast} \circPi(E) \to D_{e}^{k} \times \circPi(X)$ induces trivializations
\[
\hat{\psi}_{e_{[\fraks]}} : \hat{\varphi}_{e_{[\fraks]}}^{\ast} \circPi(E,k)' \to D_{e}^{k} \times \circPi(X).
\]
Thus we may repeat the above construction of an inductive families perturbation using $(e_{[\fraks]}, \hat{\varphi}_{e_{[\fraks]}}, \hat{\psi}_{e_{[\fraks]}})$ in place of $(e, \varphi_e, \psi_e)$, and obtain a section $\sigma'^{(i)} : \Spincns(E,k)/\Conj
\to \circPi(E,k)'$ with the property that 
$\calM_{\sigma'^{(i)}, \mathrm{half}} = \emptyset$ if $i<k$.
We call this $\sigma'^{(i)}$ an {\it inductive half-total families perturbation}.

We give a few examples of (inductive) half-total families perturbations.

\begin{ex}
\label{ex: trivial monodromy}
Let $X$ and $k$ be as above, and let $X \to E \to B$ be a $\Diff^+(X)$-bundle with trivial monodromy on $\Spincns(X,k)$.
In this case, the covering space $\Spincns(E,k)/\Conj \to B$ is just the product with an obvious projection: 
$B \times \Spincns(X,k)/\Conj \to B$.
Therefore a section $\sigma' : \Spincns(E,k)/\Conj \to \circPi(E,k)'$ corresponds to a collection of sections
\[
\left(\sigma_{[\fraks]} : B \to \circPi(E)\right)_{[\fraks] \in \Spincns(X)/\Conj}.
\]
There is no constraint on this collection of sections.
For example, we may take all $\sigma_{[\fraks]}$ to be a common section $\sigma : B \to \circPi(E)$.
\end{ex}


\begin{ex}
\label{ex: perturbation single mapping torus 1}
Let $X$ and $k$ be as above, and let $f \in \Diff^+(X)$.
Let $X \to X_f \to S^1$ be the mapping torus of $f$.
Then a continuous section 
\[
\sigma' : \Spincns(X_f,k)/\Conj \to \circPi(X_f,k)'
\]
corresponds to a family of continuous sections
\[
\left(\sigma_{t} : (\Spincns(X,k)/\Conj) \to \circPi(X,k)' \right)_{t \in [0,1]}
\]
that satisfies the condition $f^\ast \circ \sigma_{0}=\sigma_1 \circ f^\ast$.

Let us describe $\sigma'$ by data before passing to the quotients by $\Z/2$.
Let $\Z/2$ act on $[0,1]$ trivially.
Then a family of sections $\sigma_{t} : (\Spincns(X,k)/\Conj) \to \circPi(X,k)'$ corresponds to a $\Z/2$-equivariant continuous map
\[
\tilde{\sigma} = (\tilde{\sigma}^1, \tilde{\sigma}^2) : [0,1] \times \Spincns(X,k) \to \Spincns(X,k) \times \circPi(X)
\]
such that $\tilde{\sigma}^1(t, \fraks) = \fraks$ or $\tilde{\sigma}^1(t,\fraks) = \bar{\fraks}$ for each $t \in [0,1]$ and $\fraks \in \Spincns(X,k)$.
Note that $\tilde{\sigma}^1(t,\fraks)$ is independent of $t$ because of the connectivity of $[0,1]$, so can be denoted by $\tilde{\sigma}^1(\fraks)$.
Note that, we fix a section 
\[
\tau : \Spincns(X,k)/\Conj \to \Spincns(X,k)
\]
of the surjection $\Spincns(X,k) \to \Spincns(X,k)/\Conj$, we may obtain such 
$\tilde{\sigma}^1$ by $\tilde{\sigma}^1(\fraks)=\tau([\fraks])$.
Then the condition $f^\ast \circ \sigma_{0}(\fraks) = \sigma_1 \circ f^\ast(\fraks)$ is equivalent to that 
\[
[(f^\ast\tau([\fraks]), f^\ast\tilde{\sigma}^2(0,\fraks))]
= [(\tau([f^\ast \fraks]), \tilde{\sigma}^2(1,f^\ast\fraks))]
\]
holds in the quotient $(\Spincns(X,k) \times \circPi(X))/(\Z/2) = \circPi(X,k)'$.


\begin{comment}
Then giving a $\Z/2$-equivariant map $\tilde{\sigma}$ is equivalent to giving a map
\[
\tilde{\sigma} = (\tilde{\sigma}^1, \tilde{\sigma}^2) : [0,1] \times \tau(\Spincns(X,k)/\Conj) \to \Spincns(X,k) \times \circPi(X).
\]
Then it follows that
\[
f^\ast\circ\tilde{\sigma}(0,\fraks)
= (f^\ast\tilde{\sigma}^1(\fraks), f^\ast\tilde{\sigma}^2(0,\fraks)),\ 
\tilde{\sigma}(1, f^\ast(1,\fraks))
= (\tilde{\sigma}^1(f^\ast \fraks), \tilde{\sigma}^2(1,f^\ast\fraks)).
\]
\end{comment}

Refreshing the notation,
we may rephrase the above description as follows: 
Fix a section $\tau : \Spincns(X,k)/\Conj \to \Spincns(X,k)$.
A collection of continuous maps
\begin{align*}
\left(\sigma_{[\fraks]} : [0,1] \to \circPi(X)
\right)_{[\fraks] \in \Spincns(X,k)/\Conj}
\end{align*} gives rise to a half-total families perturbation \[\sigma' : \Spincns(X_f,k)/\Conj \to \circPi(X_f,k)'\] for the mapping torus $X_f$
if it satisfies that, for $[\fraks] \in \Spincns(X,k)/\Conj$,
\[
f^\ast\sigma_{[\fraks]}(0) =
\left\{
\begin{array}{ll}
    \sigma_{f^\ast[\fraks]}(1) \quad &\text{if} \quad f^\ast\tau([\fraks]) \in \tau(\Spincns(X,k)/\Conj),\\
    -\sigma_{f^\ast[\fraks]}(1) \quad &\text{if} \quad f^\ast\tau([\fraks]) \notin \tau(\Spincns(X,k)/\Conj).
\end{array}
\right.
\]
Here $ -\sigma_{f^\ast[\fraks]}(1)$ means keeping the metric unchanged and multiplying the perturbation by $-1$.
Not as in the above \cref{ex: trivial monodromy}, we may not take all $\sigma_\fraks$ to be a common map $[0,1] \to \circPi(X)$ in general.
\end{ex}

\begin{ex}
\label{ex: perturbation multi map torus}
Let $X$ and $k>0$ be as above, and let $f_1, \ldots, f_k \in \Diff^+(X')$ and suppose that they mutually commute.
Let $X \to E \to T^k$ denote the multiple mapping torus of $f_1, \ldots, f_k$ with fiber $X$.
The above \cref{ex: perturbation single mapping torus 1} can be easily generalized as follows.
For $i_1, \ldots, i_l \in \{1, \ldots, k\}$ with $i_1<\cdots<i_l$, let $f_{i_1, \ldots, i_l}^\ast$ denote the composition of pull-backs $f_{i_1}^\ast, \ldots,  f_{i_l}^\ast$.
Fix a section $\tau : \Spincns(X,k)/\Conj \to \Spincns(X,k)$.
Let
\begin{align*}
\left(\sigma_{[\fraks]} : [0,1]^k \to \circPi(X)
\right)_{[\fraks] \in \Spincns(X,k)/\Conj}
\end{align*}
be a collection of continuous maps.
Suppose that, for $[\fraks] \in \Spincns(X,k)/\Conj$ and distinct $i_1, \ldots, i_l \in \{1, \ldots, k\}$ with $i_1<\cdots<i_l$ and $\mathbf{t} = (t_1, \ldots, t_n) \in [0,1]^k$ with $t_{i_1}=\dots=t_{i_l}=0$, we have
\begin{equation}
\label{eq: peruturb general mapping tori}
f_{i_1, \ldots, i_l}^\ast\sigma_{[\fraks]}(\mathbf{t}) =
\left\{
\begin{array}{ll}
    \sigma_{f_{i_1, \ldots, i_l}^\ast[\fraks]}(\overline{\mathbf{t}}^{i_1, \ldots, i_l}) \quad &\text{if} \quad f_{i_1, \ldots, i_l}^\ast\tau([\fraks]) \in \tau(\Spincns(X,k)/\Conj),\\
    -\sigma_{f_{i_1, \ldots, i_l}^\ast[\fraks]}(\overline{\mathbf{t}}^{i_1, \ldots, i_l}) \quad &\text{if} \quad f_{i_1, \ldots, i_l}^\ast\tau([\fraks]) \notin \tau(\Spincns(X,k)/\Conj),
\end{array}
\right.
\end{equation}
where $\overline{\mathbf{t}}^{i_1, \ldots, i_l} = (t_1',\dots, t_k')$ with $t_j'=t_j$ if $j \notin \{i_1, \ldots, i_l\}$ and $t_j'=1$ if $j \in \{i_1, \ldots, i_l\}$.
Then this collection $(\sigma_{[\fraks]})_{[\fraks]}$ gives rise to a half-total families perturbation $\sigma' : \Spincns(E,k)/\Conj \to \circPi(E,k)'$ for the multiple mapping torus $E$.

If we equip $T^k$ with the standard cell structure, we may construct such a collection $(\sigma_{[\fraks]})_{[\fraks]}$ inductively from $0$-cells, and it gives rise to an inductive half-total families perturbation $\sigma'$.
\end{ex}







\subsection{The characteristic classes $\SWbbtot$ and $\SWbbhalftot$}
\label{subsec cochain SWtot}

In this \lcnamecref{subsec cochain SWtot}, we shall define the characteristic classes $\SWbbtot$ and $\SWbbhalftot$.

To help our intuition,
first we define the 0-th degree part $\SWbbtot^0(X)$ and $\SWbbhalftot^0(X)$, which are numerical invariants.

\begin{defi}
Let $X$ be a closed oriented smooth 4-manifold with $b^+(X) \geq 2$.
Fix a homology orientation of $X$.
Let $\tau : \Spincns(X,k)/\Conj \to \Spincns(X,k)$ be a section.
We define
\begin{align*}
\SWbbtot^0(X) &= \sum_{\Spinc(X,0)}\SW(X,\fraks) \in \Z,\\
\SWbbhalftot^0(X) &= \sum_{\fraks \in \tau(\Spincns(X,0)/\Conj)}\SW(X,\fraks) \in \Z/2.    
\end{align*}
Here $\SW(X,\fraks)$ denotes the standard Seiberg--Witten invariant or its mod 2 reduction.
\end{defi}

Note that the $\Z/2$-valued invariant $\SWbbhalftot^0(X)$ is independent of $\tau$ because of
the formula $\SW(X,\fraks) = \pm\SW(X,\bar{\fraks})$ in $\Z$.

\begin{ex}
\label{ex: minimal algebraic surfaces of general type}
A typical example of a 4-manifold $X$ with $\SWbbhalftot^0(X) \neq 0$ is a minimal algebraic surface of general type, since it has a unique Seiberg--Witten basic class up to conjugation.
One may produce more examples by the fiber sum operation, which shall be discussed in \cref{construction of 4-manifolds}.
(See, for example, \cref{pro: fiber sum has nontrivial SW-tot}.)
\end{ex}

Now we define cohomological generalizations of the above numerical invariants, which are the characteristic classes $\SWbbtot^\bullet$ and $\SWbbhalftot^\bullet$ with $\bullet>0$.
Henceforth, throughout this \lcnamecref{subsec cochain SWtot}, let $k>0$ and let $X$ be a closed oriented smooth 4-manifold with $b^+(X) \geq k+2$ and with a fixed homology orientation.
Let $B$ be a CW complex, and let $X \to E \to B$ be a fiber bundle with structure group $\Diff^+(X)$.
We shall define cohomology classes $\SWbbtot^k(E) \in H^k(B, \Z_E)$ and $\SWbbhalftot^k(E) \in H^k(B;\Z/2)$, where $\Z_E$ is the local system with fiber $\Z$ defined in \cref{subsection Collection of Seiberg--Witten equations}, which is determined by the monodromy action for $E$ on the homology orientation.

Let
\[
C_\ast(B;\Z_E),\quad C^\ast(B;\Z_E),\quad
C_\ast(B),\quad
C^\ast(B)
\]
denote the cellular chain complex, cochain complex with local coefficients $\Z_E$ and the cellular chain complex, cochain complex with coefficient $\Z/2$, respectively.
The boundary operator and coboundary operator are denoted by $\del$ and $\delta$ for any (co)chain complex.
Here we adopt the following classical model due to Steenrod~\cite{Steenrod43} as the definition of (co)chain complex with local coefficients:
For each cell $e$, we choose a reference point $x(e)$ in $e$.
Let $\Z_E(e)$ denote the fiber of $\Z_E$ over $x(e)$. 
The $i$-th chain group $C_i(B;\Z_E)$ is defined be the set of formal finite sums $\Sigma_{e}a_{e}e$, where $e$ runs over $i$-cells of $B$ and $a_{e} \in \Z_E(e)$.
The cochain group $C^i(B;\Z_E)$ is defined to be the set of functions that send each $i$-cell $e$ to an element of $\Z_E(e)$.
The (co)boundary operator is defined by twisting the ordinal definition by the automorphism of $\Z$ corresponding to the paths connecting the respective reference points.

The first step of the construction of $\SWbbtot^k(E)$ is to define a $k$-cochain of $B$, which will be shown to be a cocycle and whose cohomology class is $\SWbbtot^k(E)$.
Let $e$ be a $k$-cell of $B$, and let $\varphi_e : D^k_e \to B^{(k)}$ be the characterisric map of $e$.
Take an inductive families perturbation
\[
\sigma = \sigma^{(k)} : B^{(k)} \to \circPi(E)|_{B^{(k)}}
\]
over the $k$-skeleton of $B$,
explained in \cref{subsection Inductive sections}.
Consider the pull-back section
\[
\varphi_e^\ast \sigma : D^k_e \to \varphi_e^\ast\circPi(E).
\]
Note that $\varphi_e^\ast\circPi(E) \subset \circPi(\varphi_e^\ast E)$, and hence $\varphi_e^\ast \sigma$ gives a section of $\circPi(\varphi_e^\ast E) \to D^k_e$.
Applying the construction in \cref{subsection Collection of Seiberg--Witten equations} to  this pull-back section $\varphi_e^\ast \sigma$, we obtain a collection of Fredholm sections
\[
s^k_{\varphi_e^\ast \sigma} : 
\scrB^\ast(\varphi_e^\ast E,k,\varphi_e^\ast\sigma) \to \scrE(\varphi_e^\ast E,k,\varphi_e^\ast\sigma)
\]
and the parameterized moduli space $\calM_{\varphi_e^\ast \sigma}$, parameterized over $D^k_e$.
Since any inductive families perturbation was constructed to satisfy $\calM_{\sigma^{(i)}} = \emptyset$ for $i<k$, we have 
\[
\calM_{\varphi_e^\ast \sigma|_{\del D^k_e}} = \emptyset.
\]
Thus we obtain a relative cohomology class 
\[
\frakM(s^k_{\varphi_e^\ast \sigma}, 1; \del D^k_e) \in H^{k}(D^k_e, \del D^k_e; \varphi_e^\ast \Z_E),
\]
as explained in \cref{subsection Collection of Seiberg--Witten equations}.

Note that there is a canonical isomorphism
\[
H^{k}(D^k_e, \del D^k_e; \varphi_e^\ast \Z_E) \cong \Z_E(e)
\]
via $\varphi_e$, without ambiguity of $\Aut(\Z) \cong \Z/2$.
This isomorphism gives us a pairing 
\[
\left<-, - \right> : H^{k}(D^k_e, \del D^k_e; \varphi_e^\ast \Z_E)
\otimes
H_{k}(D^k_e, \del D^k_e; \Z)
\to \Z_E(e).
\]

Instead of taking an inductive families perturbation $\sigma$,
if we take an inductive half-total families perturbation
\[
\sigma'=\sigma'^{(k)} : \left(\Spincns(E,k)/\Conj\right)^{(k)}
\to \circPi(E,k)',
\]
the `half-total version' of the above argument works:
For each $k$-cell $e$,
the pull-back under $\varphi_e$ gives rise to a section
\[
\varphi_e^\ast \sigma' : \varphi_e^\ast (\Spincns(E,k)/\Conj) \to \varphi_e^\ast (\circPi(E)').
\]
Here we identified $\varphi_e^\ast(\Spincns(E,k)/\Conj)$ with $\Spincns(\varphi_e^\ast E,k)/\Conj$.
Applying the construction of \cref{subsection Complex conjugation} to this pull-back section,
we may obtain a relative cohomology class
\[
\frakM((s_{\mathrm{half}}^k)_{\varphi_e^\ast \sigma'}, 1; \del D^k_e) 
\in H^{k}(D^k_e, \del D^k_e; \Z/2)
\]
in a similar manner.


\begin{defi}
\label{defi: cochain}
For an inductive families perturbation $\sigma$ and an inductive half-total families perturbation $\sigma'$, we define cochains
\[
\SWcaltot^k(E, \sigma) \in C^k(B;\Z_E),\  \SWcalhalftot^k(E, \sigma') \in C^k(B)
\]
by
\begin{align*}
\SWcaltot^k(E, \sigma)(e)
&=\left<\frakM(s^k_{\varphi_e^\ast \sigma}, 1; \del D^k_e), [(D^k_e, \del D^k_e)]\right> \in \Z_E(e),\\
\SWcalhalftot^k(E, \sigma')(e)
&=\left<\frakM((s_{\mathrm{half}}^k)_{\varphi_e^\ast \sigma'}, 1; \del D^k_e), [(D^k_e, \del D^k_e)]\right> \in \Z/2
\end{align*}
for each $k$-cell $e$, respectively.
\end{defi}



In subsequent \cref{subsection Well-definedness and naturality}, we shall prove:

\begin{pro}
\label{SWtot cocycle}
The cochains $\SWcaltot^k(E, \sigma), \SWcalhalftot^k(E, \sigma')$ constructed above are cocycles.
\end{pro}

\begin{pro}
\label{SWtot indep}
The cohomology classes $[\SWcaltot^k(E, \sigma)] \in H^k(B;\Z_E)$, $[\SWcalhalftot^k(E, \sigma')] \in H^k(B;\Z/2)$ are independent of the choice of $\sigma$ and $\sigma'$ respectively.
\end{pro}

Thus we arrive at the following definition:

\begin{defi}
Define 
\[
\SWbbtot^k(E) \in H^k(B;\Z_E),\  \SWbbhalftot^k(E) \in H^k(B;\Z/2)
\]
by
\begin{align*}
\SWbbtot^k(E) &= [\SWcaltot^k(E, \sigma)],\\
\SWbbhalftot^k(E) &= [\SWcalhalftot^k(E, \sigma')],
\end{align*}
where $\sigma : B^{(k)} \to \circPi(E)|_{B^{(k)}}$ is an inductive families perturbation, and $\sigma' : \left(\Spincns(E,k)/\Conj\right)^{(k)}
\to \circPi(E,k)'$ is an inductive half-total families perturbation, respectively.

We define also 
\[
\SWbbtot^k(X) \in H^k(\BDiff^+(X);\Z_{\EDiff^+(X)}),\  \SWbbhalftot^k(X) 
\in H^k(\BDiff^+(X);\Z/2)
\]
by
\[
\SWbbtot^k(X) := \SWbbtot^k(\EDiff^+(X)),\ 
\SWbbhalftot^k(X) := \SWbbhalftot^k(\EDiff^+(X)).
\]
\end{defi}

As expected, we have naturality for the assignments $E \mapsto \SWbbtot^k(E)$ and $E \mapsto \SWbbhalftot^k(E)$ (\cref{SWtot functoriality}). 



\begin{rmk}
\label{rem: relation with Rub and K}
There are two preceding constructions deeply related to $\SWbbtot^k(X)$.
Given a $\spinc$ structure  $\fraks$ with $d(\fraks)=-k$ is fixed, let $\Diff^+(X,\fraks)$ denote the group of diffeomorphisms that preserve orientation and (the isomorphism class of) $\fraks$.
A characteristic class $\SWbb(X,\fraks) \in H^k(\BDiff^+(X, \fraks))$ was constructed by the first author \cite{K21}, which is defined by counting moduli spaces for $\fraks$ over $k$-cells.
The class $\SWbbtot^k(X)$ is, instead, defined by summing up the counts over all $\fraks$.
A significant difference between this paper and \cite{K21} is that $\SWbbtot^k(X)$ and $\SWbbhalftot^k(X)$ can be defined over the whole $\BDiff^+(X)$, in other words, defined for all oriented fiber bundles with fiber $X$, while the characteristic class in \cite{K21} can be defined only fiber bundles whose monodromy preserve $\fraks$.

The idea of the ``total" construction is inspired by Ruberman's construction of his {\it total} Seiberg--Witten invariant of diffeomorphisms $\mathrm{SW}_{\rm{tot}}(f,\fraks)$ \cite{Rub01}.
The class $\SWbbtot^k(X)$ is, roughly, a cohomological refinement of Ruberman's total Seiberg--Witten invariant $\mathrm{SW}_{\rm{tot}}(f,\fraks)$.
More strictly speaking, although $\mathrm{SW}_{\rm{tot}}(f,\fraks)$ was defined by summing up the moduli spaces over the orbit $\{(f^n)^\ast \fraks\}_{n \in \Z}$, to define an invariant $\SWbbtot^k(X)$ independent of $\fraks$, we shall use the sum over all $\spinc$ structures $\fraks$ with $d(\fraks)=-k$. This is why we need to consider also $\SWbbhalftot^k(X)$, not as in \cite{Rub01}, since in most cases the charge conjugation symmetry kills all $\Z/2$-valued invariant, and we shall work over $\Z/2$ eventually since the monodromy may not preserve a homology orientation.
\end{rmk}


%\begin{ex}\label{ex: K3}
%Given a smooth bundle $X\rightarrow E\rightarrow B$ over a $k$-dimensional closed smooth manifold $B$, suppose that $\pi_{1}(B)$ acts trivially on $\Spinc(X,k)$ and preserves the homology orientation (e.g. when $B$ is simply-connected). Then  $\SWbbtot^k(E)$ and $\SWbbhalftot^k(E)$ can be recovered from the usual families Seiberg--Witten invariants $\SW(E,\mathfrak{s})$ (see \cref{subsection Multiple mapping torus pre}) for $\fraks \in \Spinc(X,k)$ by the following formula:
%\[
%\langle \SWbbtot^k(E), [B]\rangle =\sum_{\mathfrak{s}\in \Spinc(X,k)} \SW(E,\mathfrak{s})\in \mathbb{Z}
%\]
%and 
%\[
%\langle \SWbbhalftot^k(E), [B]\rangle =\sum_{[\mathfrak{s}]\in \Spinc(X,k)/\Conj} \SW(E,\mathfrak{s})\in \mathbb{Z}/2.
%\]
%In \cite{B21}, Baraglia computed the families Seiberg--Witten invariants for some smooth fiber bundles $K3\rightarrow E\rightarrow S^2$. By \cite[Theorem 1.4]{B21}, we see that 
%$\langle \SWbbhalftot^2(E), [S^2]\rangle\neq 0. $ Therefore,  $\SWbbhalftot^2(K3)\in H^{2}(\BDiff^{+}(K3);\mathbb{Z}/2)$ is a nontrivial characteristic class. 
%\end{ex}


\subsection{Well-definedness}
\label{subsection Well-definedness and naturality}

In this \lcnamecref{subsection Well-definedness and naturality}, we prove \cref{SWtot cocycle} (cocycle) and  \cref{SWtot indep} (well-definedness).
Some parts of the proofs are simple adaptions of the argument of \cite[Subsection~6.2]{K21}, and in such cases we just indicate corresponding propositions of \cite{K21}.

First, we shall note naturality results.
As in \cref{subsec cochain SWtot}, let $k>0$ and let $X$ be a closed oriented smooth 4-manifold with $b^+(X) \geq k+2$. 
Let $B$ be a CW complex, and let $X \to E \to B$ be a fiber bundle with structure group $\Diff^+(X)$.
The following is a straightforward consequence of 
\cref{lem: first naturality}:

\begin{lem}
\label{lem: frakM calSW}
In the setup of \cref{defi: cochain}, suppose that $B$ is compact.
Then we have 
\begin{align*}
\frakM(s^k_{\sigma}, 1; B^{(k-1)})
&= \SWcaltot^k(E,\sigma),\\
\frakM((s^k_{\mathrm{half}})_{\sigma'}, 1; B^{(k-1)})
&= \SWcalhalftot^k(E,\sigma')
\end{align*}
in $H^k(B^{(k)}, B^{(k-1)};\Z_E) = C^k(B;\Z_E)$ and in $H^k(B^{(k)}, B^{(k-1)};\Z/2) = C^k(B)$, respectively.
\end{lem}

\begin{proof}
Using \cref{lem: first naturality}, we can adapt the proof of \cite[Lemma~6.13]{K21}:
we may repeat the proof using $\SWcaltot^k(E,\sigma)$ and $\SWcalhalftot^k(E,\sigma')$, in place of $\mathcal{A}(E,\sigma)$ in the proof of \cite[Lemma~6.13]{K21}.
\end{proof}

This \lcnamecref{lem: frakM calSW} implies the naturality at the level of cochains:

\begin{pro}
\label{prop: cochain natural}
Let $B'$ be a CW complex and $f : B' \to B$ be a cellular map.
Then we have 
\begin{align*}
f^\ast\SWcaltot^k(E,\sigma)
&= \SWcaltot^k(f^\ast E,f^\ast \sigma),\\
f^\ast\SWcalhalftot^k(E,\sigma)
&= \SWcalhalftot^k(f^\ast E,f^\ast \sigma)
\end{align*}
in $C^k(B;\Z_{f^\ast E})$ and in $C^k(B)$, respectively.
\end{pro}

\begin{proof}
Using \cref{lem: first naturality} and \cref{lem: frakM calSW},
we can just adapt the proof of \cite[Proposition~6.14]{K21}.
\end{proof}

As a direct consequence of \cref{prop: cochain natural} and the cellular approximation theorem, we have:

\begin{pro}
\label{SWtot functoriality}
Let $B'$ be a CW complex and $f : B' \to B$ be a continuous map.
Then we have 
\[
f^\ast\SWbbtot^k(E) = \SWbbtot^k(f^\ast E),\quad
f^\ast\SWbbhalftot^k(E) = \SWbbhalftot^k(f^\ast E)
\]
in $H^k(B'; f^\ast \Z_E) (\cong H^k(B'; \Z_{f^\ast E}))$ and in $H^k(B';\Z/2)$, respectively.
\end{pro}

Note that,
applying \cref{SWtot functoriality} to the identity map, we have that the cohomology class $\SWbbtot^k(E)$ is independent of the choice of CW structure of $B$.
It also follows from a formal argument that $\SWbbtot^k(E)$ can be defined for any oriented smooth fiber bundle with fiber $X$ over an arbitrary topological space, not only over a CW complex (see \cite[Lemma  6.8]{K21}).


\begin{proof}[Proof of \cref{SWtot cocycle}]
This corresponds to \cite[Proof of Proposition~6.2]{K21}.
As the statement for $\SWcaltot$ is proven by a rather direct adaption of \cite[Proof of Proposition~6.2]{K21}, here we give a proof of the statement for $\SWcalhalftot$ for the reader's convenience.
As noted in \cref{subsection Inductive sections}, we can extend the construction of the inductive half-total families perturbation to the $(k+1)$-skeleton, thanks to the condition that $b^+(X) \geq k+2$.
Let 
\[
\sigma' : \left(\Spincns(E,k)/\Conj\right)^{(k+1)}
\to \circPi(E,k)'
\]
be an extension of $\sigma'$ to the $(k+1)$-skeleton.

Let $e$ be a $(k+1)$-cell of $B$  with the characteristic map $\varphi_e : D^{k+1}_e \to B$.
Fix a homeomorphism between $D^{k+1}_e$ and $\Delta^{k+1}$, the standard $(k+1)$-simplex.
Let $D^{k+1}_e$ be equipped with a CW structure via this homeomorphism and the standard CW structure on $\Delta^{k+1}$.
By cellular approximation, we can suppose that $\varphi_e$ is cellular.

We rewrite $\delta \SWcalhalftot(E, \sigma')(e)$ as follows.
First, it follows from a direct computation with \cref{prop: cochain natural} that
\begin{align}
\label{eq: proof cocycle 1}
\delta \SWcalhalftot(E, \sigma')(e)
= \SWcalhalftot(\varphi_e^\ast E, \varphi_e^\ast\sigma')(\del \Delta^{k+1}).
\end{align}
Next, it follows from \cref{lem: frakM calSW} that 
\begin{align}
\label{eq: proof cocycle 2}
\frakM((s^k_{\mathrm{half}})_{\varphi_e^\ast\sigma'|_{(\Delta^{k+1})^{(k)}}}, 1; (\Delta^{k+1})^{(k-1)})
= \SWcalhalftot(\varphi_e^\ast E, \varphi_e^\ast \sigma'),
\end{align}
where $\varphi_e^\ast\sigma'|_{(\Delta^{k+1})^{(k)}}$ denotes the restricted section
\[
\varphi_e^\ast\sigma'|_{(\Delta^{k+1})^{(k)}}:
\left(\Spincns(\varphi_e^\ast E,k)/\Conj\right)^{(k+1)}|_{(\Delta^{k+1})^{(k)}}
\to \circPi(\varphi_e^\ast E,k)'|_{(\Delta^{k+1})^{(k)}}.
\]
As a last piece of the proof, because of the vanishing of the moduli space over the $(k-1)$-skeleton,
we may obtain a cohomology class 
\[
\frakM((s^k_{\mathrm{half}})_{\varphi_e^\ast\sigma'}, 1; (\Delta^{k+1})^{(k-1)}) \in H^{k}(\Delta^{k+1}, (\Delta^{k+1})^{(k-1)}; \Z/2).
\]
Let
\[
i : ((\Delta^{k+1})^{(k)}, (\Delta^{k+1})^{(k-1)})
\hookrightarrow (\Delta^{k+1}, (\Delta^{k+1})^{(k-1)})
\]
denote the inclusion.
Then we have 
\begin{align}
\label{eq: proof cocycle 3}
i^\ast \frakM((s^k_{\mathrm{half}})_{\varphi_e^\ast\sigma'}, 1; (\Delta^{k+1})^{(k-1)})
= \frakM((s^k_{\mathrm{half}})_{\varphi_e^\ast\sigma'|_{(\Delta^{k+1})^{(k)}}}, 1; (\Delta^{k+1})^{(k-1)}).
\end{align}

Now, a straightforward calculation using \eqref{eq: proof cocycle 1}, \eqref{eq: proof cocycle 2}, \eqref{eq: proof cocycle 3} with $i_\ast \del \Delta^{k+1}=0$ verifies that
$\delta \SWcalhalftot(E, \sigma')(e)
= 0$, which proves that $\SWcalhalftot(E, \sigma')$ is a cocycle.
\end{proof}


\begin{proof}[Proof of \cref{SWtot indep}]
This is an adaption of \cite[Proof of Theorem~6.4]{K21}.
Again we just give a proof of the statement for $\SWcalhalftot$.
For $j=0,1$, let
\[
\sigma_j' = \sigma_j'^{(k)}: \left(\Spincns(E,k)/\Conj\right)^{(k)}
\to \circPi(E,k)'
\]
be an inductive half-total families perturbations.
Set 
\[
\SWcal_j = \SWcalhalftot^k(E,\sigma_j').
\]
Let $p : B \times [0,1] \to B$ be the projection.
For each cell $e$ of $B$ equipped with a characteristic map $\varphi_e : D^k_e \to B$, we fix a trivialization
\[
\tilde{\psi}_e : (\varphi_e \times \id)^\ast p^\ast \circPi(E) \to D^{\dim e} \times [0,1] \times \circPi(X).
\]
As in \cref{subsection Inductive sections}, we regard $\Spincns(E,k)/\Conj$ as a covering space over $B$ with fiber $\Spincns(X,k)/\Conj$.
Let $\hat{p} : (\Spincns(E,k)/\Conj) \times [0,1] \to \Spincns(E,k)/\Conj$ be the projection.
Recall that $\varphi_e$ lifts to characteristic maps $(\hat{\varphi}_{e_{[\fraks]}})_{[\fraks] \in \Spincns(E,k)/\Conj}$ of cells $(e_{[\fraks]})_{[\fraks] \in \Spincns(E,k)/\Conj}$ of $\Spincns(E,k)/\Conj$ which are lifts of $e$.
Also, $\tilde{\psi}_e$ induces trivializations
\[
\hat{\tilde{\psi}}_{e_{[\fraks]}} : (\hat{\varphi}_{e_{[\fraks]}} \times \id)^\ast \hat{p}^\ast \circPi(E,k)' \to D^{\dim e} \times [0,1] \times \circPi(X).
\]

We inductively construct a section
\[
\tilde{\sigma}' = \tilde{\sigma}'^{(k)}: (\Spincns(E,k)/\Conj)^{(k)} \times [0,1] \to \hat{p}^\ast \circPi(E,k)'
\]
as follows.
%First, we take  a section $\tilde{\sigma}'^{(0)} : (\Spincns(E,k)/\Conj)^{(0)} \times [0,1] \to \hat{p}^\ast \circPi(E,k)'$ such that:
%\begin{itemize}
%\item We have
%\[
%\tilde{\sigma}'^{(0)}|_{(\Spincns(E,k)/\Conj)^{(0)} \times {\{j\}}}
%= \sigma_j'^{(0)}
%\]
%for $j=0,1$.
%\item The composition
%\begin{align*}
%&D^0_{e_{[\fraks]}} \times (0,1) \xrightarrow{(\hat{\varphi}_{e_{[\fraks]}} \times \id)^\ast \tilde{\sigma}'^{(0)}} (\hat{\varphi}_{e_{[\fraks]}} \times \id)^\ast \hat{p}^\ast \circPi(E,k)'\\ &\xrightarrow{\hat{\tilde{\psi}}_{e_{[\fraks]}}} D^{\dim e} \times [0,1] \times \circPi(X) \xrightarrow{p_2} \circPi(X)
%\end{align*}
%is generic for every $[\fraks] \in \Spincns(X,k)/\Conj$ so that $\calM_{\tilde{\sigma}'^{(0)}, \mathrm{half}} = \emptyset$, where the last map $p_2$ is the projection.
%\end{itemize}
Assume that we have constructed a section
$\tilde{\sigma}'^{(i-1)}$ for $i \leq k$ such that 
\[
\tilde{\sigma}'^{(i)}|_{(\Spincns(E,k)/\Conj)^{(i)} \times {\{j\}}}
= \sigma_j'^{(i)}
\]
for $j=0,1$ and that $\calM_{\tilde{\sigma}'^{(i-1)}, \mathrm{half}} = \emptyset$.
(This can be easily done for $i=0$.)
For a $i$-cell $e$ of $B$, we take a section 
\[
\tilde{\sigma}'^{(i)} : (\Spincns(E,k)/\Conj)|_{\bar{e}} \times [0,1] \to \hat{p}^\ast \circPi(E,k)'
\]
so that:
\begin{itemize}
\item We have
\[
\tilde{\sigma}'^{(i)}|_{(\Spincns(E,k)/\Conj)|_{\bar{e}} \times {\{j\}}}
= \sigma_j'^{(i)}|_{(\Spincns(E,k)/\Conj)|_{\bar{e}}}
\]
for $j=0,1$.
\item We have
\[
\tilde{\sigma}'^{(i)}|_{(\Spincns(E,k)/\Conj)|_{\bar{e} \setminus \{e\}} \times [0,1]}
= \sigma_j'^{(i-1)}|_{(\Spincns(E,k)/\Conj)|_{\bar{e} \setminus \{e\}} \times [0,1]}.
\]
\item The composition
\begin{align*}
&D^i_{e_{[\fraks]}} \times (0,1) \xrightarrow{(\hat{\varphi}_{e_{[\fraks]}} \times \id)^\ast \tilde{\sigma}'^{(i)}} (\hat{\varphi}_{e_{[\fraks]}} \times \id)^\ast \hat{p}^\ast \circPi(E,k)'\\ &\xrightarrow{\hat{\tilde{\psi}}_{e_{[\fraks]}}} D^{i} \times [0,1] \times \circPi(X) \xrightarrow{p_2} \circPi(X)
\end{align*}
is generic for every $[\fraks] \in \Spincns(X,k)/\Conj$, where $p_2$ is the second projection.
\end{itemize}
We may take the last composition map to be a map to $\circPi(X)$ rather than to $\Pi_\ast(X)$ since we have $b^+(X) \geq k+2$.
Thus we obtain a section
\[
\tilde{\sigma}'^{(i)} : (\Spincns(E,k)/\Conj)^{(i)} \times [0,1] \to \hat{p}^\ast \circPi(E,k)'.
\]

Equip $[0,1]$ with the standard CW structure and let $I$ denote the unique $1$-cell of $[0,1]$.
We define a cochain $\widetilde{\SWcal} \in C^{k-1}(B)$ by 
\[
\widetilde{\SWcal} = \SWcalhalftot^k(p^\ast E, \tilde{\sigma}')(e \times I)
\]
for a $(k-1)$-cell $e$.

We claim that $\delta \widetilde{\SWcal} = \SWcal_1 + \SWcal_0$, which completes the proof.
First, we note the isomorphisms
\[
C^\ast(B \times [0,1], B \times \{0,1\})
\cong C^\ast(B) \otimes C^1([0,1])
\cong C^\ast(B).
\]
Here the first isomorphism is the K{\"u}nneth isomorphism and the second one is the map $C^\ast(B) \to C^\ast(B) \otimes C^1([0,1])$ given by $u \mapsto u \otimes I$.
Let $e$ be a $k$-cell of $B$.
Using that $\SWcalhalftot^k$ is a cocycle (\cref{SWtot cocycle}) and the naturality (\cref{prop: cochain natural}), we have
\begin{align*}
0=&\delta \SWcalhalftot^k(p^\ast E, \tilde{\sigma}')(e \otimes I)\\
=&\SWcalhalftot^k(p^\ast E, \tilde{\sigma}')(\del (e \otimes I))\\
=& \widetilde{\SWcal}(\del e)
+ \SWcalhalftot^k(p^\ast E, \tilde{\sigma}')(e \otimes 1) + \SWcalhalftot^k(p^\ast E, \tilde{\sigma}')(e \otimes 1)\\
=& \delta\widetilde{\SWcal}(e)
+ \SWcal_1(e) + \SWcal_0(e),
\end{align*}
which shows the claim.
\end{proof}


\subsection{Vanishing under stabilizations}
\label{subsectionVanishing under stabilizations}

In this \lcnamecref{subsectionVanishing under stabilizations}, 
we give vanishing theorems for $\SWbbtot$ and $\SWbbhalftot$ under stabilizations.
Let $M, N$ closed oriented smooth 4-manifolds.
Choose smoothly embedded 4-disks in $M, N$, and let $\mathring{M}, \mathring{N}$ denote the punctured manifolds obtained from $M, N$.
We denote by $\Diff_\del(\mathring{M})$ the group of diffeomorphisms that are the identities on collar neighborhoods of $\del\mathring{M}$. 
We consider the stabilization map  
\[
s_{M,M\# N} : \Diff_\del(\mathring{M}) \to \Diff^+(M\# N)
\]
defined by extending elements in $\Diff_\del(\mathring{M})$ with the identity of $N$.
%Also, by gluing $\EDiff_\del(\mathring{M})$ and $\EDiff_\del(\mathring{N})$


\begin{thm}
\label{thm: vanishing}
Let $k\geq0$ and let $M,N$ be closed oriented smooth 4-manifolds with  $b^+(M) \geq k+1$ and $b^{+}(N)\geq 1$.
Then we have
\[
s_{M,M\# N}^\ast\SWbbtot^k(M\#N)=0.
\]
\end{thm}

\begin{proof}
Set $X=M\#N$, $B=\BDiff_\del(M)$ and $E(\mathring{M}) = \EDiff_\del(\mathring{M})$.
By gluing $E(\mathring{M})$ with the trivial bundle $B \times \mathring{N}$ along $B \times S^3$, we obtain a bundle with fiber $X$, which we denote by $E(X) \to B$.
Let denote by $X_b$ the fiber of $E(X)$ over $b \in B$.
We use similar notations also for other bundles.

By the construction of $\SWbbtot$, the claim of the \lcnamecref{thm: vanishing} follows once we construct an inductive families perturbation $\sigma : B^{(k)} \to \circPi(E(X))|_{B^{(k)}}$ with the following property:
for every $b \in B^{(k)}$ and
every $\spinc$ structure $\fraks_{{X}_b} \in \Spinc(X_b)$ with $d(\fraks_{{X}_b}) \leq -k$, the Seiberg--Witten equations for $(X_b,\fraks_{X_b})$ perturbed by $\sigma(b)$ do not have either irreducible or reducible solution.


To construct such $\sigma$, we shall consider a fiberwise neck stretching argument.
Let us describe a setup to discuss it.
Let $E(\mathring{M}^\ast) \to B$ denote the bundle with fiber $\mathring{M}^\ast = \mathring{M} \cup ([0,\infty)\times S^3)$ obtained by gluing $E(\mathring{M})$ with $B \times ([0,\infty)\times S^3)$.
We fix the standard positive scalar metric on the cylinder $[0,\infty)\times S^3$.
By requiring the restriction to the cylinder is given by this standard metric and the zero perturbation 2-form, one has straightforward variants, such as $\circPi(\mathring{M}^\ast)$, of spaces of perturbations introduced in \cref{subsection Collection of Seiberg--Witten equations} for the cylindrical-end family $\mathring{M}^\ast$.
Whenever we consider the Seiberg--Witten equations on a cylindrical-end 4-manifold, say $\mathring{M}^\ast$, we impose the standard boundary condition: asymptotic to the unique reducible solution on $S^3$.

Similarly, we may also consider the space of perturbations, say $\circPi(\mathring{M})$, for a punctured 4-manifold by imposing that the restriction to neighborhoods of $\del \mathring{M}$ is given by the standard positive scalar curvature metric on $[0,\epsilon) \times S^3$ for some $\epsilon>0$.
Whenever we consider the Seiberg--Witten equations on a punctured 4-manifold, we impose the Atiyah--Patodi--Singer boundary condition.

By the assumption that $b^+(M)\geq k+1$, as in \cref{subsection Inductive sections}, we can construct an inductive families perturbation $\sigma_{\mathring{M}} : B^{(k)} \to \circPi(\mathring{M})$, with an underlying families metric, %$g_{E(\mathring{M})}$,
that satisfies the following property:
for every $b \in B^{(k)}$ and
every $\spinc$ structure $\fraks_{{M}_b} \in \Spinc(M_b)$ with $d(\fraks_{{M}_b}) < -k$, the Seiberg--Witten equations for $(\mathring{M}_b,\fraks_{M_b})$ perturbed by $\sigma_{\mathring{M}}(b)$ have neither irreducible nor reducible solution.

On the other hand, by $b^+(N)\geq 1$, we can find a perturbation  $\sigma_{\mathring{N}}^0 \in \circPi(\mathring{N})$, with an underlying metric,
that satisfies the following property:
the Seiberg--Witten equations on $\mathring{N}$ perturbed by $\sigma_{\mathring{N}}^0$ do not have reducible solution for every $\spinc$ structure $\fraks_{N} \in \Spinc(N)$, and have neither reducible nor irreducible solution for every $\spinc$ structure $\fraks_{N} \in \Spinc(N)$ with $d(\fraks_N)<0$.
Define $\sigma_{\mathring{N}} : B^{(k)} \to \circPi(B \times \mathring{N})$ to be the constant families metric/perturbation for the trivial family $B \times \mathring{N}$ given by $\sigma_{\mathring{N}}^0$. 

We equip $\mathring{M}_b$ for $b \in B$ and $\mathring{N}$ with the underlying metrics of $\sigma_{\mathring{M}}(b)$ and $\sigma_{\mathring{N}}^0$.
For $L>0$, define a Riemannian manifold $M_b\#_{L}N$ by inserting a cylinder of length $2L$ between $\mathring{M}_b$ and $\mathring{N}$,
\[
M_b\#_{L}N = 
\mathring{M}_b \cup ([-L,L] \times S^3) \cup \mathring{N},
\]
where the cylinder $[-L,L] \times S^3$ is equipped with the standard positive scalar curvature metric.
By zero extension, we obtain from $\sigma_{\mathring{M}}(b), \sigma_{\mathring{N}}^0$ a new perturbation $\sigma_{\mathring{M}}(b)\#_L\sigma_{\mathring{N}}^0 \in \circPi(E(X)_b)$ whose underlying metric is given by the metric on $M_b\#_L N$ described now.

Now we define a function $f : B^{(k)} \to [0,\infty) \cup \{\infty\}$ by
\[
f(b) = \sup\Set{L \in [0,\infty) | \text{Property }(*)_{b,L}},
\]
where Property~$(*)_{b,L}$ is that $M_b\#_L N$ admits a solution to the Seiberg--Witten equations with respect to $\sigma_{\mathring{M}}(b) \#_L \sigma_{\mathring{N}}^0$ for some $\spinc$ structure $\fraks$ on $M_b\#N$ with $d(\fraks) \leq -k$.
(Here we adopt the convention that $\sup\emptyset=0$.)

Now we claim that $f(b)<\infty$ for any $b \in B^{(k)}$, so $f$ is regarded to be a function $f : B^{(k)} \to [0,\infty)$, and $f$ is upper semi-continuous.
Once we verify this, it is straightforward to inductively construct a continuous function $\mathcal{L} : B^{(k)} \to [0,\infty)$ with $\mathcal{L} > f$ over $B^{(k)}$.
Then the perturbations/metrics $\sigma_{\mathring{M}}(b)\#_{\mathcal{L}(b)}\sigma_{\mathring{N}}^0$ on  $M_b\#_{\mathcal{L}(b)}N$ yield the desired families perturbation $\sigma : B^{(k)} \to \circPi(E(X))$, which completes the proof of the \lcnamecref{thm: vanishing}.

To verify that $f$ is finite-valued and is upper semicontinuous, it is enough to get a contradiction from the following assumption: there are sequences $\{b_i\}_{i=1}^\infty \subset B^{(k)}$ and $\{L_i\}_{i=1}^\infty \subset \R$ such that:
\begin{itemize}
\item $\{b_i\}$ converges to some point $b_{\infty} \in B^{(k)}$, and $L_i \to +\infty$ as $i \to \infty$,
\item for each $i$, there exists a solution $\gamma_i$ to the Seiberg--Witten equations for 
\[
(M_{b_i}\#_{L_i}N, \sigma_{\mathring{M}}(b_i)\#_{L_i}\sigma_{\mathring{N}}^0)
\]
and for some $\spinc$ structure $\fraks_i$ on $M_{b_i}\#N$ with $d(\fraks_i) \leq -k$.
\end{itemize}

To get a contradiction, first note that we have $\#\Set{\fraks_i|i \in \mathbb{N}}<\infty$, as in the proof of the finiteness of the basic classes.
Thus, after passing to a subsequence, we may suppose that all $\fraks_i$ coincide with a common $\spinc$ structure, denoted by $\fraks$.
Then the solutions $\{\gamma_i\}$ converge to a pair of solutions $\gamma_\infty = (\gamma_\infty^M, \gamma_\infty^N)$ on $(\mathring{M}_{b_\infty}^\ast,  \mathring{N}^\ast)$ for the pair of perturbations $(\sigma_{\mathring{M}}(b_\infty), \sigma_{\mathring{N}}^0)$, extended to $\mathring{M}^\ast$ and $\mathring{N}^\ast$ by zero, and for the pair of $\spinc$ structures $(\fraks|_{M}, \fraks|_{N})$.
The existence of the solution $\gamma_{\infty}^N$ and the choice of $\sigma_{\mathring{N}}^0$ implies that $d(\fraks|_N) \geq 0$.
Now $d(\fraks|_{M})<-k$ follows from this combined with $d(\fraks) \leq -k$ and a general formula $d(\fraks|_{M})+d(\fraks|_{N})+1=d(\fraks)$.
However, then the existence of the solution $\gamma_\infty^M$ contradicts the choice of $\sigma_{\mathring{M}}$.
Thus we completed to prove the existence of the desired families perturbation $\sigma$ and this completes the proof of \lcnamecref{thm: vanishing}.
\end{proof}

For $\SWbbhalftot$, 
a statement analogous to \cref{thm: vanishing} does not hold in general.
The upshot is that, for $\SWbbhalftot$, one may not take a constant families perturbation for $N$ as in the proof of \cref{thm: vanishing}.

\begin{ex}
To get a counterexample to a  statement for $\SWbbhalftot$ analogous to \cref{thm: vanishing}, let us take $N$ to be $S^2\times S^2$ and $M$ to be a complete intersection of general type defined by real coefficient polynomials so that $M$ admits an involution $f : M \to M$ given by the complex conjugation.
Then we have
\begin{align}
\label{eq: counter ex}
s_{M,M\#S^2\times S^2}^\ast(\SWbbhalftot^1(M\#N)) \neq 0.
\end{align}
As (\ref{eq: counter ex}) is not needed in the proofs of our results, we only sketch the argument and summerize the upshot.
By isotopy, we may assume that $f$ fixes a 4-disk in $M$.
Let $M_f \to S^1$ denote the mapping torus with fiber $M$ defined by $f$.
Let $BM_f : S^1 \to \BDiff_\del(M)$ be the classifying map of $M_f$ and define $\alpha := (BM_f)_\ast([S^1]) \in H_1(\BDiff^+(M);\Z/2)$.
The non-vanishing \eqref{eq: counter ex} follows from the fact that 
\begin{align}
\label{eq: counter ex pairing}
\left<\SWbbhalftot^1(M\#N),(s_{M,M\#S^2\times S^2})_\ast(\alpha)\right> \neq 0,
\end{align}
which is derived by an argument similar \cref{thm key computation source} which we shall prove.
The point is, since $f$ sends the canonical class of $M$ to its conjugate, one needs to take a families perturbation for $N$ that causes the wall-crossing in the calculation of the pairing \eqref{eq: counter ex pairing}.
\end{ex}

However, we still have the following vanishing for $\SWbbhalftot$:

\begin{thm}
\label{thm: vanishing2}
Let $k\geq0$ and let $M,N$ be closed oriented smooth 4-manifolds with  $b^+(M),b^{+}(N)\geq k+1$.
Then we have
\[
s_{M,M\# N}^\ast\SWbbhalftot^k(M\#N)=0.
\]
\end{thm}

\begin{proof}
The proof is quite similar to that of \cref{thm: vanishing}, so we describe only the upshot.
We use notations in the proof of \cref{thm: vanishing}.
Also, set $E(N) = B \times N$ and use notations $E(\mathring{N})$, $E(\mathring{N}^\ast)$ in a similar manner to $M$.

First, note that the $\Z/2$-actions induced by the charge conjugation make
a natural bijection 
\begin{align}
\label{eq: identify}
\Spinc(X,k) \cong \bigtimes_{k=p+q+1} (\Spinc(M,p) \times \Spinc(N,q))
\end{align} 
$\Z/2$-equivariant, where the  $\Z/2$-action on the right-hand side is the diagonal action.
%Let $\Spinc(M,N,k)/\Conj$ denote the quotient of the right-hand side by the conjugation.
%Using the action of $\Diff^+(M)$ on $\Spinc(M,p)$, associated to the fiber bundle $E(M) \to B$, we obtain a covering space with fiber $\Spinc(M,p)$, denoted by $\Spinc(E(M),p) \to B$.
%Taking the fiberwise product of $\Spinc(E(M),p) \to B$ with the trivial covering $B \times \Spinc(N,q) \to B$, we obtain a covering space with fiber $\Spinc(M,p) \times \Spinc(N,q)$, denoted by $\Spinc(E(M), E(N), p,q) \to B$.
%Since the conjugation commutes with the pull-backs under diffeomorphisms, we obtain a covering with fiber $(\Spinc(M,p) \times \Spinc(N,q))/\Conj$, denoted by $\Spinc(E(M), E(N),p,q)/\Conj \to B$.
With this in mind,
we may modify the definition of (spaces of)  half-total families perturbations in \cref{subsection Complex conjugation,subsection Inductive sections} to a pair of (punctured) 4-manifolds.
%in place of $\Spinc(E)/\Conj$, we use $\Spinc(E(M), E(N))/\Conj$, and use $\mathring{M}, \mathring{N}$ or $\mathring{M}^\ast, \mathring{N}^\ast$.
We define
\[
\circPi(\mathring{M}, \mathring{N},k)' = \left(\bigtimes_{k=p+q+1}\left((\Spinc(M,p) \times \circPi(\mathring{M})) \times (\Spinc(N,q) \times \circPi(\mathring{N}))\right)\right)/(\Z/2),
\]
where $\Z/2$ acts as the diagonal action.
Then, using the $\Z/2$-equivariant bijection \eqref{eq: identify}, $E(M)$ and $E(N)$ induce a fiber bundle
\[
\circPi(E(\mathring{M}), E(\mathring{N}),k)' \to \Spinc(E(X),k)/\Conj
\]
with fiber $\circPi(\mathring{M}, \mathring{N},k)'$.
A half-total families perturbation is formulated as a section 
\[
\sigma'_{M,N} : \Spinc(E(X),k)/\Conj \to \circPi(E(\mathring{M}), E(\mathring{N}),k)'
\]
that covers, as in the diagram \eqref{eq: diagram half-tot section}, a pair of families metrics $(\tilde{g}_M, \tilde{g}_N)$ for $E(\mathring{M})$, $E(\mathring{N})$ with cylindrical metrics near boundary.
We require that the self-dual 2-forms that appear in a half-total families perturbation is zero near the boundaries of $\mathring{M}_b, \mathring{N}_b$.
Then we may glue $\sigma_{M,N}'$ near $\del \mathring{M}_b, \del \mathring{N}_b$ to get a half-total families perturbation for $E(X)$ in the sense of \cref{subsection Inductive sections}. 
We denote by 
\[
\sigma_{\mathring{M}}'\#_L\sigma_{\mathring{N}}' : \Spinc(E(X),k)/\Conj \to \circPi(E(X),k)'
\]
the glued half-total families perturbation obtained by inserting the cylinder $[-L,L] \times S^3$.
Note also that $\sigma_{M,N}'$ induces half-total families perturbations
\begin{align*}
&\sigma_{\mathring{M}}' : \Spinc(E(M),p)/\Conj \to \circPi(E(\mathring{M}),p)',\\
&\sigma_{\mathring{N}}' : \Spinc(E(N),q)/\Conj \to \circPi(E(\mathring{N}),q)'
\end{align*}
for $E(\mathring{M})$ and $E(\mathring{N})$ in the sense of \cref{subsection Inductive sections}, only with the modification that we allow punctured 4-manifolds not only closed 4-manifolds.


By $b^+(M), b^+(N) \geq k+1$,
one can take an inductive half-total families perturbation
$\sigma'_{M,N}$ so that the parameterized moduli spaces
$\calM_{\sigma'_{\mathring{M}}, \mathrm{half},k}$ and $\calM_{\sigma'_{\mathring{N}}, \mathrm{half},k}$ over $B^{(k)}$ for $\sigma'_{\mathring{M}}$ and $\sigma'_{\mathring{N}}$ are empty.
Here $\calM_{\sigma'_{\mathring{M}}, \mathrm{half},k}$ and $\calM_{\sigma'_{\mathring{N}}, \mathrm{half},k}$ are defined by considering collections over $\bigsqcup_{l > k}\Spinc(M,l)/\Conj$ and $\bigsqcup_{l > k}\Spinc(N,l)/\Conj$, respectively.

Let $\pi : \Spinc(E(X),k)/\Conj \to B$ be the projection.
Now we define a function $f' : B^{(k)} \to [0,\infty) \cup \{\infty\}$ by
\[
f'(b) = \sup\Set{L \in [0,\infty) | \text{Property }(*)'_{b,L}},
\]
where Property~$(*)'_{b,L}$ is that $M_b\#_L N_b$ admits a solution to the Seiberg--Witten equations with respect to some perturbation that lies in $(\sigma_{\mathring{M}}' \#_L \sigma_{\mathring{N}}')(\pi^{-1}(b))$ and for some $\spinc$ structure $\fraks$ on $M_b\#N_b$ with $d(\fraks) \leq -k$.

We claim that $f'<+\infty$ over $B^{(k)}$ and $f'$ is upper semicontinuous.
Then, as in the proof of \cref{thm: vanishing}, there is a continuous function $\mathcal{L}' : B^{(k)} \to [0,\infty)$ with $\mathcal{L}'>f'$ over $B^{(k)}$, and $\sigma_{\mathring{M}}'\#_{\mathcal{L}'(b)} \sigma_{\mathring{N}}' : \Spinc(E(X),k)/\Conj \to \circPi(E(X),k)'$ is an inductive half-total families perturbation for which all moduli spaces relevant to $s_{M,M\# N}^\ast\SWbbhalftot^k(M\#N)$ are empty, which completes the proof.

To prove the claim that $f'<+\infty$ over $B^{(k)}$ and $f'$ is upper semicontinuous, assume that this is not the case.
Then, as in the proof of \cref{thm: vanishing}, there are a point $b_\infty \in B^{(k)}$, a lift $\tilde{b}_{\infty} \in \pi^{-1}(b_{\infty})$ of $b_{\infty}$, a $\spinc$ structure $\fraks \in \Spinc(X)$ with $d(\fraks) \leq -k$, and a pair of solutions $(\gamma_\infty^M, \gamma_\infty^N)$ for $((\mathring{M}_{b_{\infty}}, \sigma'_{\mathring{M}}(\tilde{b}_{\infty}),\fraks|_{M}), (\mathring{N}_{b_{\infty}}, \sigma'_{\mathring{N}}(\tilde{b}_{\infty}),\fraks|_{N}))$.
This contradicts the formula $-k \geq d(\fraks) = d(\fraks|_M)+d(\fraks|_N)+1$ and that $\calM_{\sigma'_{\mathring{M}}, \mathrm{half},k}$ and $\calM_{\sigma'_{\mathring{N}}, \mathrm{half},k}$ are empty.
\end{proof}


Now, for an oriented closed smooth 4-manifold $X$, let 
\[
s_X : \Diff_\del(\mathring{X})
\to \Diff^+(X\# S^2\times S^2)
\]
denote the stabilization map for $X$ by $S^2 \times S^2$.
\cref{thm: vanishing,thm: vanishing2} immediately imply the following, which states that $\SWbbtot$ and $\SWbbhalftot$ are {\it unstable} characteristic classes with respect to stabilizations by $S^2\times S^2$:

\begin{cor}
\label{cor: vanishing}
Let $k\geq0$ and let $X$ be a closed oriented smooth 4-manifolds. 
Then the following results hold.
\begin{enumerate}
\item 
If $b^+(X) \geq k+1$, then $s_{X}^\ast\SWbbtot^k(X\#S^2\times S^2)=0$.
\item If $b^+(X) \geq k+1$, then 
\[
s_{X}^\ast \circ \dots \circ s_{X\#kS^2\times S^2}^\ast \SWbbhalftot^k(X\#(k+1)S^2\times S^2)=0.
\]
\end{enumerate}
\end{cor}

Note that, in the above proofs of \cref{thm: vanishing2}, the triviality of the bundle $E(N)=B\times N$ is not used, and a similar (actually rather simpler) argument applies also to $\SWbbtot$.
We record this below, while we do not use it in the rest of this paper.
Let $M, N$ be closed oriented smooth 4-manifolds, and $B$ be a CW complex.
Let $E(M) \to B$ and $E(N) \to B$ be oriented smooth fiber bundles with fiber $M$ and $N$ with structure group $\Diff_\del(\mathring{M})$ and $\Diff_\del(\mathring{N})$, respectively.
Then we can form an oriented fiber bundle $E(M)\#_fE(N) \to B$ with fiber $M\#N$ obtained by taking the fiberwise connected sum of $E(M)$ and $E(N)$ along the trivial families of 4-disks in $E(M), E(N)$ corresponding to the punctures of $M, N$. 

\begin{thm}
In the above setup, let $k>0$ and suppose that $b^+(M), b^+(N)\geq k+1$. Then we have
\[
\SWbbtot^k(E(M)\#_fE(N))=0,\quad
\SWbbhalftot^k(E(M)\#_fE(N))=0.
\]
\end{thm}


\subsection{Positive scalar curvature}
\label{subsection Vanishing under positive scalar curvature}

The vanishing of the Seiberg--Witten invariant for a positive scalar curvature metric extends to invariants for families, and this generalization is useful to study the topology of spaces of positive scalar curvature metrics  \cite{Rub01,K19}.
In this \lcnamecref{subsection Vanishing under positive scalar curvature}, we formulate a corresponding vanishing theorem in our setup.

Given an oriented smooth manifold $X$,  recall that we defined notations
\[
\scrR(X),\ \scrR^+(X),\ \scrM(X),\  \scrM^+(X)
\]
in \cref{subsection:Moduli spaces of manifolds with metrics}.
%let $\scrR(X)$ and $\scrR^+(X)$ denote the space of Riemannian metrics and space of metrics with positive scalar curvatures, respectively.
%The diffeomorphism group $\Diff^+(X)$ acts on $\scrR(X)$ by pull-back, and this action preserves  $\scrR^+(X)$.
%The quotients $\scrM(X) = \scrR(X)/\Diff^+(X)$ and $\scrM^+(X) = \scrR^+(X)/\Diff^+(X)$ are often called the moduli spaces of Riemannian metrics and that of positive scalar curvature metrics.
%As the actions of $\Diff^+(X)$ on $\scrR(X)$ and $\scrR^+(X)$ are not free in general, it is more convenient to consider homotopy quotients:
%Define $\scrM(X)$ and $\scrM^+(X)$ by 
%\[
%\scrM(X) = \EDiff^+(X) \times_{\Diff^+(X)} \scrR(X),\quad
%\scrM^+(X) = \EDiff^+(X) \times_{\Diff^+(X)} \scrR^+(X).
%\]
%We have natural forgetful maps $\scrM(X) \to \scrM(X)$, $\scrM^+(X) \to \scrM^+(X)$ by ignoring the $\EDiff^+(X)$-factor.
%Also, the inclusion $\scrR^+(X) \hookrightarrow \scrR(X)$ gives rise to an injection
%\[
%\iota : \scrM^+(X) \hookrightarrow \scrM(X).
%\]
We start by noting that the spaces $\scrM(X)$ and $\scrM^+(X)$ classify oriented fiber bundles with fiber $X$ with fiberwise metrics, and $\scrM^+(X)$ classifies oriented fiber bundles with fiber $X$ with fiberwise positive scalar curvature metrics, respectively.
Indeed, let $B$ be a CW complex and let $X \to E \to B$ be an oriented smooth fiber bundle with fiber $X$ over $B$.
Let $\scrR(X) \to \scrR(E) \to B$ denote the fiber bundle associated with $E$ with fiber $\scrR(X)$.
Suppose that $E$ is equipped with a fiberwise metric $g_E$.
This corresponds to a section of $\scrR(E) \to B$, which gives a lift of the classifying map $B \to \BDiff^+(X)$ of $E$ to $\EDiff^+(X) \times_{\Diff^+(X)} \scrR(X)$.
Similarly, a fiberwise positive scalar curvature metric $g_E$ on $E$ corresponds to a lift of the classifying map $B \to \BDiff^+(X)$ of $E$ to $\EDiff^+(X) \times_{\Diff^+(X)} \scrR^+(X)$:
\begin{align}
\label{eq: moduli of psc}
\begin{split}
\xymatrix{
     & \EDiff^+(X)\times_{\Diff^+(X)} \scrR^+(X) = \scrM^+(X) \ar[d] \\
    B \ar[ru]^-{g_E} \ar[r]_-{E} & \BDiff^+(X).
   }   
\end{split}  
\end{align}
Thus $\scrM(X)$ classifies pairs $(E,g_E)$ of fiber bundles $E$ and fiberwise metrics $g_E$ on $E$ up to deformation equivalence, and $\scrM^+(X)$ classifies pairs $(E,g_E)$  of fiber bundles and fiberwise positive scalar curvature metrics up to deformation equivalence. 

There are also other kinds of natural moduli spaces.
Fix a point $x_0 \in X$, and let $\Diff_{x_0}(X)$ denote the group of diffeomorphisms that fix $x_0$ and act on $T_{x_0}X$ trivially.
If we suppose that $X$ is a closed manifold, then $\Diff_{x_0}(X)$ acts freely on $\scrR(X)$ (see, e.g. \cite[Lemma~1.2]{BHSW10}).
The quotients 
\[
\scrM_{x_0}(X) = \scrR(X)/\Diff_{x_0}(X),\quad
\scrM_{x_0}^+(X) = \scrR^+(X)/\Diff_{x_0}(X)
\]
are called the {\it observer moduli spaces} of Riemannian metrics and metrics of positive scalar curvature, respectively.
These moduli spaces were introduced in \cite{AB02}, and have been extensively studied by several authors for higher-dimensional manifolds.
See such as \cite{BHSW10,HSS14}.

Relative versions are also natural moduli spaces.
Choose an embedded 4-disk $D^4$ in $X$, and set $\mathring{X} = X \setminus \mathrm{Int}(D^4)$.
Let $\scrR_{\del}(\mathring{X})$ denote the space of Riemannian metrics on $X$ for which the restrictions to collar neighborhoods of $\del\mathring{X}$ coincide with the standard cylindrical metric of positive scalar curvature on $[0,1] \times S^3$.
Set $\scrR_{\del}^+(\mathring{X}) = \scrR_{\del}(\mathring{X}) \cap \scrR^+(\mathring{X})$.
It is easy to see that $\Diff_{\del}(\mathring{X})$ acts freely on $\scrR_{\del}^+(\mathring{X})$.
Define
\[
\scrM_{\del}(\mathring{X}) = \scrR_\del(\mathring{X})/\Diff_{\del}(\mathring{X}),\quad
\scrM_{\del}^+(\mathring{X}) = \scrR^+_{\del}(\mathring{X})/\Diff_{\del}(\mathring{X}).
\]

For $\scrM_{\del}(\mathring{X})$, we may consider stabilizations.
Fix a positive scalar curvature metric on $([0,1] \times S^3)\#S^2\times S^2$ whose restriction to a collar neighborhood of the boundary is a pair of the standard cylindrical metrics on $[0,1] \times S^3$.
Then we may define a stabilization map $\scrR(\mathring{X}) \to \scrR(\mathring{X}\#S^2\times S^2)$ by gluing the fixed metric on $([0,1] \times S^3)\#S^2\times S^2$.
This gives rise to a stabilization map between moduli spaces,
$\scrM_{\del}(\mathring{X}) \to \scrM_{\del}(\mathring{X}\#S^2\times S^2)$.

\begin{defi}
We say that a homology class $\alpha \in H_*(\scrM_{\del}(\mathring{X});\Z)$ is {\it unstable} if $\alpha$ lies in the kernel of the map
\[
H_*(\scrM_{\del}(\mathring{X});\Z)
\to H_*(\scrM_{\del}(\mathring{X}\#S^2\times S^2);\Z)
\]
induced by the stabilization map.
\end{defi}

Let us compare the various moduli spaces above.
%For a group $G$ and a space $Z$ acted by $G$, set $Z \sslash G = \mathrm{E}G\times_{G}Z$.
Let
\[
\iota : \scrM^+(X) \hookrightarrow \scrM(X),\quad
\iota_{x_0} : \scrM_{x_0}^+(X) \hookrightarrow \scrM_{x_0}(X),\quad
\iota_{\del} : \scrM_{\del}^+(\mathring{X}) \hookrightarrow \scrM_{\del}(\mathring{X})
\]
be injections induced from the inclusion $\scrR^+(X) \hookrightarrow \scrR(X)$.
Since the actions of $\Diff_{x_0}(X)$ and $\Diff_{\del}(\mathring{X})$ on spaces of metrics are free, the honest quotients by these groups are homotopy equivalent to corresponding homotopy quotients.
Also, let $D^4$ be equipped with the restriction of the standard round metric of $S^4$.
Gluing relative metrics with this round metric on $D^4$,
we have natural extension maps $\scrR_{\del}(\mathring{X}) \hookrightarrow \scrR(X)$ and $\scrR_{\del}^+(\mathring{X}) \hookrightarrow \scrR^+(X)$.
With homotopy inverses from honest quotients to homotopy quotients, the injections $\Diff_{\del}(\mathring{X}) \hookrightarrow \Diff_{x_0}(X) \hookrightarrow \Diff^+(X)$ and extensions $\scrR_{\del}(\mathring{X}) \hookrightarrow \scrR(X)$,  $\scrR_{\del}^+(\mathring{X}) \hookrightarrow \scrR^+(X)$ give rise to a homotopy commutative diagram
\begin{align}
\label{eq: diagram three psc moduli comparison}
\begin{split}
\xymatrix{
\scrM_\del(\mathring{X})
\ar[r]& 
\scrM_{x_0}(X)
\ar[r] &
\scrM(X)\\
\scrM_\del^+(\mathring{X})
\ar[r] \ar[u]^{\iota_{\del}}& 
\scrM_{x_0}^+(X)
\ar[r] \ar[u]^{\iota_{x_0}}&
\scrM^+(X). \ar[u]^{\iota}
}
\end{split}
\end{align}

Now we discuss the vanishing of the Seiberg--Witten characteristic classes.
Let $k>0$ and suppose that $X$ is an oriented closed smooth 4-manifold with $b^+(X) \geq k+2$ so that the Seiberg--Witten characteristic classes are well-defined.
Since $\scrR(X)$ is contractible, $\EDiff^+(X) \times \scrR(X)$ is also contractible, and $\EDiff^+(X) \times \scrR(X)$ is acted by $\Diff^+(X)$ freely.
Thus $\scrM(X)$ gives a model of the classifying space of $\Diff^+(X)$, and hence we may regard the Seiberg--Witten characteristic classes as cohomology classes on $\scrM(X)$:
\[
\SWbbtot^k(X) \in H^k(\scrM(X);\Z_{\EDiff^+(X)}),\quad
\SWbbhalftot^k(X) \in H^k(\scrM(X);\Z/2).
\]

\begin{thm}
\label{thm: vanishing by psc}
We have
\[
\iota^\ast\SWbbtot^k(X)=0 \text{\quad and\quad }
\iota^\ast\SWbbhalftot^k(X)=0
\]
in $H^k(\scrM^+(X);\iota^\ast\Z_{\EDiff^+(X)})$ and in $H^k(\scrM^+(X);\Z/2)$ respectively.
\end{thm}

\begin{proof}
Fix a model of the universal principal $\Diff^+(X)$-bundle $\EDiff^+(X) \to \BDiff^+(X)$.
Then the bundle $\EDiff^+(X) \times \scrR(X) \to \scrM(X)$  gives another model of the universal pricipal $\Diff^+(X)$-bundle.
Let $E \to \scrM^+(X)$ denote the pull-back of 
this bundle $\EDiff^+(X) \times \scrR(X) \to \scrM(X)$ under $\iota$.
The claim of the \lcnamecref{thm: vanishing by psc} is equivalent to that $\SWbbtot^k(E)=0$ and $\SWbbhalftot^k(E)=0$.

We first prove $\SWbbtot^k(E)=0$.
Let $E_b$ denote the fiber of $E$ over a point $b \in \scrM^+(X)$.
Recall the notation
$\Pi_\ast(X) = \bigcup_{g \in \scrR(X)} L^2_{l-1}(i\Lambda^+_g(X))$ and
regard $\scrR(X)$ as a subset of $\Pi_\ast(X)$.
From the construction of $E$, we have $E = \EDiff^+(X) \times \scrR^+(X)$.
It follows from this that a $\Diff^+(X)$-equivariant map $E \to \Pi_\ast(X)$ gives rise to a section of the fiber bundle
\[
\pi : \Pi_\ast(E) \to \scrM^+(X)
\]
with fiber $\Pi_\ast(X)$, where $\Pi_\ast(E)$ is an associated fiber bundle with fiber $\Pi_\ast(X)$ defined in  \cref{subsection Collection of Seiberg--Witten equations}.
Similarly, we have an associated fiber bundle $\scrR^+(E) \to \scrM^+(X)$ with fiber $\scrR^+(X)$.
Note that we have a fiberwise inclusion $\scrR^+(E) \hookrightarrow \Pi_\ast(E)$ induced from the inclusion $\scrR^+(X) \hookrightarrow \Pi_\ast(X)$.

Let $\phi$ be a $\Diff^+(X)$-equivariant map $E \to \Pi_\ast(X)$ defined as the composition of the projection $\EDiff^+(X) \times \scrR^+(X) \to \scrR^+(X)$ and the inclusion $\scrR^+(X) \hookrightarrow \Pi_\ast(X)$.
Let 
\[
\sigma: \scrM^+(X) \to \Pi_\ast(E)
\]
denote the section induced from $\phi$.
Then $\sigma$ factors through $\scrR^+(E)$, and hence, for each $b \in \scrM^+(X)$, the Seiberg--Witten equations for $(E_b, \sigma(b))$ have no irreducible solutions for all $\spinc$ structures because of the standard vanishing for positive scalar curvature metrics.

Now we slightly perturb $\sigma$ to avoid reducibles.
First, the non-existence of irreducible solutions is an open condition, hence there is an open neighborhood $\mathcal{N} \subset \Pi_\ast(E)$ of the image of $\sigma$ for which, for every $n \in \mathcal{N}$, the Seiberg--Witten equations for $n$ have no irreducible solutions for all $\spinc$ structures.
We may suppose that $\mathcal{N}$ lies in $\Pi(E)$, the space of perturbations with uniformly bounded norms.
Since $\mathcal{N} \cap \circPi(E_b)$ is a codimension-$b^+(X)$ subspace of $\mathcal{N}_b = \Set{n \in \mathcal{N} | \pi(n)=b}$,
$\mathcal{N} \cap \circPi(E_b)$ is $(b^+(X)-2)$-connected, as well as $\circPi(X)$.
Thus, as in \cref{subsection Inductive sections},
we may construct an inductive families perturbations $\tilde{\sigma} : \scrM^+(X)^{(k)} \to \circPi(E)$ whose image lies in $\mathcal{N}$.
This implies that $\SWbbtot^k(E)=0$.

The proof of that $\SWbbhalftot^k(E)=0$ is similar to the above.
Let us take the subset $\mathcal{N} \subset \Pi_\ast(E)$ as above.
Define $\Pi_\ast(X,k)' = (\Spinc(X,k) \times \Pi_\ast(X))/(\Z/2)$.
Recall that we let $\Z/2$ act on $\Pi_\ast(E)$ as the fiberwise $(-1)$-multiplication, where we regard $\Pi_\ast(E)$ as a fiber bundle with fiber $L^2_{l-1}(i\Lambda^+_g(X))$ for a metric $g$.
By taking $\mathcal{N}$ smaller, we may assume that $\mathcal{N}$ is preserved under the $\Z/2$-action.
This $\mathcal{N}$ induces an open neighborhood $\mathcal{N}' \subset \Pi_\ast(E,k)'$ of $(\Spinc(X,k) \times \scrR^+(X))/(\Z/2)$ for which, for every $n \in \mathcal{N}'$, the Seiberg--Witten equations for $n$ have no irreducible solutions.
Then we can construct an inductive half-total families perturbation
$\sigma' : \left(\Spincns(E,k)/\Conj\right)^{(i)}
\to \circPi(E,k)'$
whose image lies in $\mathcal{N}'$.
This proves that $\SWbbhalftot^k(E)=0$.
\end{proof}

\begin{cor}
\label{cor: vanishing for fiberwise psc}
Let $X \to E \to B$ be an oriented smooth fiber bundle with fiber $X$ that admits a fiberwise positive scalar curvature metric.
Then we have 
\[
\SWbbtot^k(E)=0,\quad
\SWbbhalftot^k(E)=0.
\]
\end{cor}

\begin{proof}
This is a direct consequence of \cref{thm: vanishing by psc} combined with the commutative diagram \eqref{eq: moduli of psc}.
\end{proof}

%The above vanishing (\cref{thm: vanishing by psc}) implies also vanishing results for other moduli spaces.
%Fix a point $x_0 \in X$, and let $\Diff_{x_0}(X)$ denote the group of diffeomorphisms that fix $x_0$ and act on $T_{x_0}X$ trivially.
%Note that $\Diff_{x_0}(X)$ is a subgroup of $\Diff^+(X)$.
%It is easy to see that $\Diff_{x_0}(X)$ acts freely on $\scrR(X)$ (see, e.g. \cite[Lemma~1.2]{BHSW10}).
%The quotients 
%\[
%\scrM_{x_0}(X) = \scrR(X)/\Diff_{x_0}(X),\quad
%\scrM_{x_0}^+(X) = \scrR^+(X)/\Diff_{x_0}(X)
%\]
%are called the {\it observer moduli spaces} of Riemannian metrics and metrics of positive scalar curvature, respectively.
%These moduli spaces were introduced in \cite{AB02}, and have been extensively studied by several authors for higher-dimensional manifolds.
%See such as \cite{BHSW10,HSS14}.
%Note that the inclusion $\scrR^+(X) \hookrightarrow \scrR(X)$ gives rise to an injection $\iota_{x_0} : \scrM_{x_0}^+(X) \hookrightarrow \scrM_{x_0}(X)$.
%Other kinds of moduli spaces are relative versions.
%Choose an embedded 4-disk $D^4$ in $X$, and set $\mathring{X} = X \setminus \mathrm{Int}(D^4)$.
%Let $\scrR_{\del}(\mathring{X})$ denote the space of Riemannian metrics on $X$ for which the restrictions to collar neighborhoods of $\del\mathring{X}$ coincide with the standard cylindrical metric of positive scalar curvature on $[0,1] \times S^3$.
%Set $\scrR_{\del}^+(\mathring{X}) = \scrR_{\del}(\mathring{X}) \cap \scrR^+(\mathring{X})$.
%It is easy to see that $\Diff_{\del}(\mathring{X})$ acts freely on $\scrR_{\del}^+(\mathring{X})$.
%Define
%\[
%\scrM_{\del}(\mathring{X}) = \scrR_\del(\mathring{X})/\Diff_{\del}(\mathring{X}),\quad
%\scrM_{\del}^+(\mathring{X}) = \scrR^+_{\del}(\mathring{X})/\Diff_{\del}(\mathring{X}).
%\]
%The inclusion $\scrR^+(X) \hookrightarrow \scrR(X)$ gives rise to an injection $\iota_{\del} : \scrM_{\del}^+(\mathring{X}) \hookrightarrow \scrM_{\del}(\mathring{X})$.
%It is worth noting that we can define natural stabilization maps for the relative versions.
%Let $K = ([0,1] \times S^3)\#S^2\times S^2$ denote the inner connected sum.
%The manifold $K$ admits a positive scalar curvature whose restriction to a collar neighborhood of the boundary is a pair of the standard cylindrical metrics on $[0,1] \times S^3$.
%Fixing such a metric on $K$, we may define a stabilization map
%\[
%\scrR(\mathring{X}) \to \scrR(\mathring{X}\#S^2\times S^2).
%\]
%This gives rise to a stabilization map between moduli spaces as well:
%\[
%\scrM_{\del}(\mathring{X}) \to \scrM_{\del}(\mathring{X}\#S^2\times S^2).
%\]
%\begin{defi}
%We say that a homology class $\alpha \in H_*(\scrM_{\del}(\mathring{X});\Z)$ is {\it unstable} if $\alpha$ lies in the kernel of the map
%\[
%H_*(\scrM_{\del}(\mathring{X});\Z)
%\to H_*(\scrM_{\del}(\mathring{X}\#S^2\times S^2);\Z)
%\]
%induced by the stabilization map.
%\end{defi}


\begin{cor}
\label{cor: psc non-surj}
Suppose that there is a homology class $\alpha \in H_k(\BDiff^+(X);\Z)$ such that either $\langle\SWbbtot^k(X),\alpha\rangle \neq0$ or $\langle\SWbbhalftot^k(X),\alpha\rangle\neq0$ holds.
Then we have the following:
\begin{enumerate}[label=(\roman*)]
\item The map
\[
\iota_\ast : H_k(\scrM^+(X);\Z) \to H_k(\scrM(X);\Z)
\]
is not surjective.
\item If $\alpha$ lies in the image of the map $H_k(\BDiff_{x_0}(X);\Z) \to H_k(\BDiff^+(X);\Z)$, then the map
\[
(\iota_{x_0})_\ast :H_k(\scrM^+_{x_0}(X);\Z) \to H_k(\scrM_{x_0}(X);\Z)
\]
is not surjective.
\item If $\alpha$ lies in the image of the map $H_k(\BDiff_{\del}(\mathring{X});\Z) \to H_k(\BDiff^+(X);\Z)$, then the map
\[
(\iota_{\del})_\ast:
H_k(\scrM^+_{\del}(\mathring{X});\Z) \to H_k(\scrM_{\del}(\mathring{X});\Z)
\]
is not surjective.
\end{enumerate}
\end{cor}

\begin{proof}
By abuse of notation, let $E : B \to \BDiff^+(X)$ denote the classifying map of $E$.
Then $E_\ast(\alpha) \in H_k(\BDiff^+(X);\Z) \cong H_k(\scrM(X);\Z)$ is non-zero, and it follow from \cref{thm: vanishing by psc} that $E_\ast(\alpha)$ lies in  the cokerel of $\iota_\ast$.
This proves the claim (i).
The remaining claims (ii), (iii) follow from the homotopy commutative diagram \eqref{eq: diagram three psc moduli comparison} with homotopy equivalences $\scrM_{x_0}(X) \simeq \BDiff_{x_0}(X)$ and $\scrM_{\del}(\mathring{X}) \simeq \BDiff_{\del}(\mathring{X})$.
\end{proof}


We remark that the non-surjectivity result (\cref{cor: psc non-surj}) for $k=1$ implies the disccoectivity of $\scrR^+(X)$:

\begin{lem}
\label{lem psc disconnected}
Let $X$ be a smooth manifold.
If $\iota_\ast : H_1(\scrM^+(X);\Z) \to H_1(\scrM(X);\Z)$ is not surjective, then $\scrR^+(X)$ is not connected.
\end{lem}

\begin{proof}
Let $\gamma \in H_1(\scrM(X);\Z)$ be a non-zero element in the cokernel of $\iota_\ast$.
Since the Hurewicz map $h : \pi_1(\scrM(X)) \to H_1(\scrM(X);\Z)$ is surjective, we may take an element $\hat{\gamma} \in h^{-1}(\gamma)$.
Since the Hurewicz maps are compatible with $\iota_\ast$, we have that $\hat{\gamma} \notin \mathrm{Im}(\pi_1(\scrM^+(X)) \to \pi_1(\scrM(X)))$.

Now consider the following commutative diagram induced from the fibrations $\Diff^+(X) \to \EDiff^+(X) \times \scrR(X) \to \scrM(X)$ and $\Diff^+(X) \to \EDiff^+(X) \times \scrR^+(X) \to \scrM^+(X)$:
\begin{align*}
\begin{split}
\xymatrix{
0 \ar[r]&
\pi_1(\scrM(X))
\ar[r]^-{\del}_-{\cong}& 
\pi_0(\Diff^+(X))
\ar[r] &
0\\
\pi_1(\scrR^+(X)) \ar[r]\ar[u]&
\pi_1(\scrM^+(X))
\ar[r] \ar[u]& 
\pi_0(\Diff^+(X))
\ar[r]^{i_\ast} \ar@{=}[u]&
\pi_0(\scrR^+(X)). \ar[u]
}
\end{split}
\end{align*}
Here the two rows are exact.
By $\hat{\gamma} \notin \mathrm{Im}(\pi_1(\scrM^+(X)) \to \pi_1(\scrM(X)))$, it is easy to check that $i_\ast \circ \del(\hat{\gamma}) \in \pi_0(\scrR^+(X))$ is non-trivial.
\end{proof}


We shall prove the following \lcnamecref{thm: psc main} in \cref{subsectionProof of the main instability theorem} that generalizes \cref{thm: psc main intro} explained in \cref{section Introduction}: 

\begin{thm}
\label{thm: psc main}
Let $X$ be a simply-connected closed oriented smooth 4-manifold and let $k>0$.
Then there exists a sequence of positive integers $N_1<N_2<\cdots \to +\infty$ such that for all $i$, none of the maps
\begin{align*}
\iota_\ast &: H_k(\scrM^+(X\# N_iS^2\times S^2);\Z) \to H_k(\scrM(X\# N_iS^2\times S^2);\Z),\\
(\iota_{x_0})_\ast &: H_k(\scrM_{x_0}^+(X\# N_iS^2\times S^2);\Z) \to H_k(\scrM_{x_0}(X\# N_iS^2\times S^2);\Z),\\
(\iota_{\del})_\ast &: H_k(\scrM_{\del}^+(\mathring{X}\# N_iS^2\times S^2);\Z) \to H_k(\scrM_{\del}(\mathring{X}\# N_iS^2\times S^2);\Z)
\end{align*}
are surjective.
Furthermore, the cokernels of the maps $(\iota_{\del})_\ast$ contain non-zero elements 
\[
\alpha_{i} \in H_k(\scrM_{\del}(\mathring{X}\# N_iS^2\times S^2);\Z)
\]
that are unstable.
\end{thm}



\section{Calculation}
\label{sec Calculation}


\subsection{Multiple mapping torus}
\label{subsection Multiple mapping torus pre}

This \lcnamecref{subsection Multiple mapping torus pre} is devoted to giving preliminary results on multiple mapping tori to carry out calculations of the Seiberg--Witten characteristic class in subsequent discussions. 

We start by setting up some notations. Let $W$ be an oriented 4-manifold with (possibly empty) boundary. We use $\Diff_{\del}(W)$ to denote the group of orientation-preserving diffeomorphisms on $W$ that equals identity near the boundary. (So $\Diff_{\del}(W)$ is just $\Diff^{+}(W)$ when $W$ is closed.) We use $\Diff_{\del}(W)^{\delta}$ to denote the same group equipped with the discrete topology. Consider a smooth, oriented bundle $W\rightarrow E\rightarrow B$. We also use $E$ to denote the corresponding map $E: B\rightarrow \BDiff_{\partial}(W)$. This map is well-defined up to homotopy. Similarly, suppose $E$ carries a flat structure, we also use $E$ to denote the corresponding map from $B$ to the moduli space $\BDiff_{\partial}(W)^{\delta}$.

Let $f_1, \ldots, f_k$ be mutually commuting orientation-preserving diffeomorphisms of $W$ supported in $\Int(W)$.
Let 
\[
W \to W_{f_1, \ldots, f_k} \to T^k
\]
denote the multiple mapping torus of $f_1, \ldots, f_k$ with fiber $W$. The bundle $W_{f_1, \ldots, f_k}$ carries a canonical flat structure since it is the quotient of the product bundle $W\times \mathbb{R}^{k}$ by a free $\mathbb{Z}^{k}$-action. Hence we get an induced map 
\[W_{f_1, \ldots, f_k}: T^{k}\rightarrow \BDiff_{\partial}(W)^{\delta}.\]
%Because discussing the families Seiberg--Witten invariant of multiple mapping torus.
We give the following technical lemma, which will be useful later.
\begin{lem}\label{lem: 2-torsion} Let $W'\subset \Int(W)$ be a  compact  embedded 4-dimensional submanifold with boundary and let $U$ be an open neighborhood of $W\setminus W'$ in $W$.  
Suppose that, for any $i\geq 2$, the restriction $f_{i}|_{W'}$ is the identity. Moreover, suppose that there is a smooth isotopy $F^{t}_{1}$ from $f_{1}$ to the identity that fixes $U$ pointwise. Then we have 
\[
W_{f_1,\cdots,f_k,*}[T^{k}]=0\in H_{k}(\BDiff_{\del}(W)^{\delta};\mathbb{Z})
\]
\end{lem}
\begin{proof}
By our assumption, the restriction $f_{1}|_{W'}$ belong to the unit component of $\Diff_{\del}(W')$, denoted by $\Diff^{0}_{\del}(W')$. By Thurston's result \cite{Thurston74}, the group $\Diff^{0}_{\del}(W')$ is simple. So $f_{1}|_{W'}$ equals a product of commutators of some elements in $\Diff^0_{\del}(W')$. By extended these elements with the identity map on $W\setminus W'$, we obtain diffeomorphisms $g_{1},\cdots,g_{n}, h_{1},\cdots, h_{n}$ which satisfy
\[
f_{1}=\prod_{j=1}^{n}g_{j}h_{j}g^{-1}_{j}h^{-1}_{j}.
\]
Furthermore, for any $1\leq j\leq n$ and $2\leq i\leq k$, both $g_{j}$ and $h_{j}$  commute with $f_{i}$ because their supports do not intersect. Let $\Sigma$ be a compact oriented surface of genus $n$ and a single boundary component. Consider a homomorphism 
\[
\rho: \pi_{1}(\Sigma\times T^{k-1})\rightarrow \Diff_{\partial}(W)
\]
that sends the generators to $g_{1},\cdots,g_{n},h_{1},\cdots,h_{n},f_{2},\cdots, f_{k}$. Then $\rho$ induces a flat bundle \[W\rightarrow E\rightarrow \Sigma\times T^{k-1}\] whose restriction to $\partial (\Sigma\times T^{k-1})$ is isomorphic to $W_{f_1,\cdots,f_k}$. As a result, the map
\[
W_{f_1,\cdots,f_k}: T^{k}\rightarrow \BDiff_{\del}(W)^\delta
\]
can be extended to $\Sigma\times T^{k-1}$. This directly implies $W_{f_1,\cdots,f_k,*}[T^{k}]=0$. 
\end{proof}


Now we start calculating the families Seiberg--Witten invariants of multiple mapping tori. For the rest of this \lcnamecref{subsection Multiple mapping torus pre},
we fix $k>0$ and set $W=X$ to be a closed oriented smooth 4-manifold with $b^+(X) \geq k+2$. We first note that one may interpret the evaluation
\[
\langle\SWbbhalftot^k(X_{f_1, \ldots, f_k}), [T^k] \rangle \in \Z/2
\]
in the following way. We fix a section \[\tau : \Spincns(X,k)/\Conj \to \Spincns(X,k).\]
As explained in \cref{ex: perturbation multi map torus}, we can get a families perturbation for $X_{f_1, \dots, f_k}$ from a collection of maps
\begin{align*}
\left(\sigma_{[\fraks]} : [0,1]^k \to \circPi(X)
\right)_{[\fraks] \in \Spincns(X,k)/\Conj}
\end{align*}
with the patching condition \eqref{eq: peruturb general mapping tori}.
For each $\fraks' \in \Spinc(X)$ and each $\sigma_{[\fraks]}$, we may consider the parameterized moduli space
\[
\calM(X, \fraks', \sigma_{[\fraks]})
= \bigcup_{\mathbf{t} \in [0,1]^k} \calM(X, \fraks', \sigma_{[\fraks]}(\mathbf{t})),
\]
where $\calM(X, \fraks', \sigma_{[\fraks]}(\mathbf{t}))$ is just the usual Seiberg--Witten moduli space for the perturbation 
$\sigma_{[\fraks]}(\mathbf{t})$.
If we take generic $\sigma_{[\fraks]}$ so that the parameterized moduli spaces for $\sigma_{[\fraks]}$ over $\del [0,1]^k$ are empty for all $\spinc$ structures, this gives rise to an inductive families perturbation $\sigma : \Spincns(E,k)/\Conj \to \circPi(E,k)'$ in  the sense of \cref{subsection Inductive sections}.
Also we take $\sigma_{[\fraks]}$ so that the parameterized moduli spaces for $\sigma_{[\fraks]}$ is smooth over $\Int([0,1]^k)$ for all $\spinc$ structures.
Then we have
\begin{align}
\label{eq: eval sum}
\langle\SWbbhalftot^k(X_{f_1, \ldots, f_k}), [T^k] \rangle  = \sum_{\fraks \in \tau(\Spincns(X,k)/\Conj)}\#\calM(X, \fraks, \sigma_{[\fraks]})
\end{align}
in $\Z/2$, where $\#$ denotes the mod 2 count of the moduli space.

We shall calculate some part of the sum of the right-hand side of \eqref{eq: eval sum}.
It is convenient to introduce the numerical families Seiberg--Witten invariant here.
Given a family $X \to E \to B$ with structure group $\Diff^+(X)$ over a closed manifold $B$ of dimension $k$ and $\fraks \in \Spinc(X,k)$, if the monodromy action of $E$ fixes $\fraks$,
one may define the numerical families Seiberg--Witten invariant \cite{BK20,K21}:
\[
\SW(E, \fraks) \in \Z/2.
\]
This is similar to the families Seiberg--Witten invariant defined by Li--Liu~\cite{LiLiu01} for families of $\spinc$ 4-manifolds, but one may define $\SW(E, \fraks)$ for a family of 4-manifolds (rather than $\spinc$ 4-manifolds) as far as the monodromy action fixes the isomorphism class $\fraks$.
If we have a generic families perturbation as explained in \cref{subsection Inductive sections}, we can calculate $\SW(E, \fraks)$ by counting the parameterized moduli space for the families perturbation.
Using the characteristic class $\SWbb(E,\fraks) \in H^k(B;\Z/2)$ introduced in \cite{K21}, $\SW(E, \fraks)$ can be written as $\SW(E, \fraks)=\left<\SWbb(E,\fraks), [B]\right>$.
The following \lcnamecref{lem: vanish deg family}  immediately follows from the naturality of the characteristic class:

\begin{lem}
\label{lem: vanish deg family}
Let $B'$ be a closed manifold of dimension $<k$, and $\phi : B \to B'$ be a continuous map.
Let $X \to E' \to B'$ be a $\Diff^+(X)$-bundle whose monodromy preserves $\fraks$.
Then we have
\begin{align*}
\SW(\phi^\ast E', \fraks)=0.
\end{align*}
\end{lem}

Let $k_0 \in \{1, \dots, k\}$, and let $p_{k_0} : T^k \to T^k$ be a surjective map obtained as the product of $k_0$-th copies of a degree 2 map $S^1 \to S^1$ and $(k-k_0)$-th copies of the identity of $S^1$.
If $(f_i^2)^\ast$ preserves a $\spinc$ structure $\fraks$ for $i \leq k_0$ and $f_i$ preserves $\fraks$ for $i \geq k_0+1$, the monodromy of
the bundle $p_{k_0}^\ast X_{f_1, \dots, f_k}$ preserves $\fraks$, and thus we can define $\SW(p_{k_0}^\ast X_{f_1, \dots, f_k}, \fraks)$.

For distinct $i_1, \ldots, i_l=0 \in \{1, \ldots, k\}$ with $i_1<\cdots<i_l$ and $a_{i_1}, \ldots, a_{i_l} \in \Z$, let $(f_{i_1, \ldots, i_l}^{a_{i_1}, \ldots, a_{i_l}})^\ast$ denote the composition of pull-backs $(f_{i_1}^{a_{i_1}})^\ast, \ldots,  (f_{i_l}^{a_{i_l}})^\ast$.

\begin{lem}
\label{lem: cancel lemma}
In the above setup, let $\fraks \in \tau(\Spincns(X,k)/\Conj)$ and
let $k_0 \in \{1,\dots,k\}$, and suppose that $f_i^\ast\fraks\neq\fraks,
(f_i^2)^\ast\fraks=\fraks$ if $i \leq k_0$, and $f_i^\ast\fraks=\fraks$ for $i\geq k_0+1$.
Suppose that $(f_{1, \dots, k}^{a_{1}, \ldots, a_{k}})^\ast\fraks \in \tau(\Spincns(X,k)/\Conj)$ for all $a_i \in \{0,1\}$.
Then we have
\begin{align}
\label{eq: lem pair zero}
\begin{split}
&\sum_{\{(f_{1, \dots, k}^{a_{1}, \ldots, a_{k}})^\ast\fraks|a_1, \ldots, a_k \in \{0,1\}\}}
\#\calM(X, (f_{1, \dots, k}^{a_{1}, \ldots, a_{k}})^\ast\fraks, \sigma_{(f_{1, \dots, k}^{a_{1}, \ldots, a_{k}})^\ast[\fraks]}) \\
=& \SW(p_{k_0}^\ast X_{f_1, \dots, f_k}, \fraks)
\end{split}
\end{align}
in $\Z/2$.

Moreover, if there are $i \in \{1, \ldots, k_0\}$ and a smooth isotopy $F_i^t$ between $f_i^2$ and the identity such that $f_1^2, \ldots, F_i^t, \ldots, f_{k_0}^2, f_{k_0+1}, \dots, f_k$ mutually commute for all $t$, then $\SW(p_{k_0}^\ast X_{f_1, \dots, f_k}, \fraks)=0$ in $\Z/2$ and hence the left-hand side of \eqref{eq: lem pair zero} is also zero.
\begin{comment}
$\sigma' : [0,1]^{k} \to \circPi(X)$ is a families perturbation satisfying the following condition:
for distinct $i_1, \ldots, i_l \in \{1, \ldots, k\}$ with $i_1<\cdots<i_l$ and $\mathbf{t} = (t_1, \ldots, t_n) \in [0,1]^k$ with $t_{i_1}=\dots=t_{i_l}=0$, we have
\begin{equation}
\label{eq: peruturb general mapping tori deloop}
(f_{i_1, \ldots, i_l}^{o(i_1), \dots, o(i_l)})^\ast\sigma_{[\fraks]}(\mathbf{t}) =
    \sigma_{[\fraks]}(\overline{\mathbf{t}}^{i_1, \ldots, i_l}),
\end{equation}
where $\overline{\mathbf{t}}^{i_1, \ldots, i_l} = (t_1',\dots, t_k')$ with $t_j'=t_j$ if $j \notin \{i_1, \ldots, i_l\}$ and $t_j'=1$ if $j \in \{i_1, \ldots, i_l\}$.
\end{comment}
\end{lem}

\begin{proof}
We shall use the notation of \cref{ex: perturbation multi map torus}.
Set $o(f_i)=2$ for $i \leq k_0$ and $o(f_i)=1$ for $i \geq k_0+1$.
We prove \eqref{eq: lem pair zero} by showing that the both hand sides of 
\eqref{eq: lem pair zero} are equal to the count of a parameterized moduli space $\#\calM(X,\fraks,\sigma')$ for some families perturbation $\sigma' : [0,1]^{k} \to \circPi(X)$.
%A standard cobordism argument shows that $\#\calM(X,\fraks,\sigma')$ depends only on $\sigma'|_{\del[0,1]^k}$ rather than the whole $\sigma'$, and by abuse of notation, we use the same symbol $\sigma'$ also for any families perturbation whose restriction to $\del [0,1]^k$ is the same as that of $\sigma'$.

First, by the patching condition \eqref{eq: peruturb general mapping tori} and our assumption \[(f_{1, \dots, k}^{a_{1}, \ldots, a_{k}})^\ast\fraks \in \tau(\Spincns(X,k)/\Conj),\] it follows that for each $i \leq k_0$, $1 \leq i_1<\dots<i_l \leq k$ and $\mathbf{t} = (t_1, \dots, t_k) \in [0,1]^k$ with $t_i=0$,
\[
f_{i}^\ast(\sigma_{f_{i_1}^\ast\dots f_{i_l}^\ast[\fraks]}(\mathbf{t}))
= \sigma_{f_i^\ast f_{i_1}^\ast\dots f_{i_l}^\ast[\fraks]}(\overline{\mathbf{t}}^i).
\]
Thus the families perturbations $\sigma_{f_{i_1}^\ast\dots f_{i_l}^\ast[\fraks]}$ and $\sigma_{f_i^\ast f_{i_1}^\ast\dots f_{i_l}^\ast[\fraks]}$ can be glued via $f_i^\ast$ along $I_{i,0}^{k-1}$ and $I_{i,1}^{k-1}$, where for $a=0,1$,
\begin{align}
\label{eq: k-1 face}
I_{i,a}^{k-1} = \Set{(t_1,\dots,t_k) \in [0,1]^k| t_i=a}.
\end{align}
\cref{fig: pathching perturbations} illustrates this gluing for $k=k_0=2$.
\begin{figure}
\includegraphics[clip,width = 6.3cm]{patched_perturbation.pdf}
\caption{Gluing families perturbations for $k=k_0=2$. Families perturbations parameterized by $\bigsqcup_4 [0,1]^2$ are glued and form a families perturbation parameterized by $[0,2]^2$.}
\label{fig: pathching perturbations}
\end{figure}

Therefore the collection
\begin{align}
\label{eq: disjoint per}
\bigsqcup_{\{(f_{1, \dots, k}^{a_{1}, \ldots, a_{k}})^\ast\fraks|a_1, \ldots, a_k \in \{0,1\}\}}
(\sigma_{f_{i_1}^\ast\dots f_{i_l}^\ast[\fraks]} : [0,1]^k \to \circPi(X))
\end{align}
of $2^{k_0}$-tuple of maps is glued via $f_i's$ with $i \leq k_0$ and forms a continuous map $\sigma'' : [0,2]^{k_0} \times [0,1]^{k-k_0} \to \circPi(X)$, and let $\sigma' : [0,1]^{k} \to \circPi(X)$ be the map obtained from $\sigma''$ by rescaling $[0,2]$ by $1/2$.
Thus the left-hand side of \eqref{eq: lem pair zero}, which is the count for the disjoint union \eqref{eq: disjoint per}, coincides with $\#\calM(X,\fraks, \sigma')$.

Next, one may see that, using \eqref{eq: peruturb general mapping tori} and $(f_i^2)^\ast\fraks=\fraks$, the above families perturbation $\sigma'$ satisfies the following condition:
for distinct $i_1, \ldots, i_l \in \{1, \ldots, k\}$ with $i_1<\cdots<i_l$ and $\mathbf{t} = (t_1, \ldots, t_n) \in [0,1]^k$ with $t_{i_1}=\dots=t_{i_l}=0$, we have
\begin{equation}
\label{eq: peruturb general mapping tori deloop}
(f_{i_1, \ldots, i_l}^{o(i_1), \dots, o(i_l)})^\ast\sigma'(\mathbf{t}) =
    \sigma'(\overline{\mathbf{t}}^{i_1, \ldots, i_l}),
\end{equation}
where $\overline{\mathbf{t}}^{i_1, \ldots, i_l} = (t_1',\dots, t_k')$ with $t_j'=t_j$ if $j \notin \{i_1, \ldots, i_l\}$ and $t_j'=1$ if $j \in \{i_1, \ldots, i_l\}$.
Now it is easy to see that $\#\calM(X,\fraks,\sigma')$ coincides with the right-hand side of \eqref{eq: lem pair zero}, which completes the proof of the claim.
This is because, just as in \cref{ex: perturbation multi map torus}, the families perturbation $\sigma'$ with the patching condition \eqref{eq: peruturb general mapping tori deloop} gives rise to a families perturbation for the multiple mapping torus $p_{k_0}^\ast X_{f_1, \dots, f_k} = X_{f_1^2, \ldots, f_{k_0}^2, f_{k_0+1}, \dots, f_k} \to T^k$.

To see the `moreover' part, note that
\[
p_{k_0}^\ast X_{f_1, \dots, f_k} \cong X_{f_1^2, \ldots, f_{k_0}^2, f_{k_0+1}, \dots, f_k}
\cong
X_{f_1^2, \dots, \id, \dots, f_{k_0}^2, f_{k_0+1}, \dots, f_k}
\]
follows from the assumption, where $\id$ is in the $i$-th entry.
Let $\phi : T^k \to T^{k-1}$ be the projection onto the $\{1,\dots,k\} \setminus \{i\}$-factors of $S^1$.
Then the above isomorphism implies that
\[
p_{k_0}^\ast X_{f_1, \dots, f_k} \cong
\phi^\ast X_{f_1^2, \dots, \widecheck{\id}, \dots, f_{k_0}^2, f_{k_0+1}, \dots, f_k},
\]
where $\widecheck{\id}$ indicates that this entry is removed.
It follows from this combined with \cref{lem: vanish deg family} that $\SW(p_{k_0}^\ast X_{f_1, \dots, f_k}, \fraks)=0$.
\end{proof}

\begin{cor}
\label{cor: cancel}
Let $f_1, \ldots, f_k \in \Diff^+(X)$ be mutually commuting diffeomorphisms with the following conditions:
\begin{itemize}
    \item For every $i=1,\dots, k$, $f_i$ preserves $\tau(\Spincns(X,k)/\Conj)$ and the action on $\tau(\Spincns(X,k)/\Conj)$ satisfies the conditions that $f_i^\ast \neq \id$ and  $(f_i^2)^\ast=\id$.
    \item There exist isotopies $(F_{i}^t)_{t\in [0,1]}$ from $f_{i}^2$ to the identity for all $i \in \{1,\dots,k\}$ such that, for any $i\neq j$ and any $t\in [0,1]$, the maps $F_{i}^t$ and $f_{j}$ commute with each other.
    \end{itemize}
Then we have    
\begin{align}
\label{eq: eval sum2}
\langle\SWbbhalftot^k(X_{f_1, \ldots, f_k}), [T^k] \rangle  =
\sum_{\substack{\fraks \in \tau(\Spincns(X,k)/\Conj)\\ f_i^\ast \fraks=\fraks\ (\forall i \in \{1, \dots, k\})}}\SW(X_{f_1, \ldots, f_k}, \fraks).
\end{align}
\end{cor}

\begin{proof}
The diffeomorphisms $f_1, \dots, f_k$ give rise to a $(\Z/2)^k$-action on the set $\tau(\Spincns(X,k)/\Conj)$.
Given $\fraks \in \tau(\Spincns(X,k)/\Conj)$, if there is $i \in \{1, \ldots, k\}$ such that $f_i^\ast\fraks\neq \fraks$, we may apply \cref{lem: cancel lemma} to $\fraks$ (by renaming indices of $f_i$) and obtain that the contribution of the $(\Z/2)^k$-orbit of $\fraks$ in the sum of the right-hand side of \eqref{eq: eval sum} is zero.

As in \cref{ex: trivial monodromy,ex: perturbation single mapping torus 1}, if a $\spinc$ structure $\fraks \in \tau(\Spincns(X,k)/\Conj)$ is preserved by the monodromy of $X_{f_1, \dots, f_k}$, the families perturbation $\sigma_{[\fraks]}$ can be taken so that $\#\calM(X,\fraks,\sigma_{[\fraks]})=\SW(X_{f_1, \dots, f_k},\fraks)$.
Thus the claim of the \lcnamecref{cor: cancel} follows from \eqref{eq: eval sum}.
\end{proof}


Another important input in the main calculation is a gluing formula for the families Seiberg--Witten invariant.
To state this, we need to introduce a few notations.
In general, for an oriented closed 4-manifold $X$ and for an oriented fiber bundle $X \to E \to B$ with fiber $X$, a vector bundle $\R^{b^+(x)} \to H^+(E) \to B$ is associated (e.g. \cite{LiLiu01, BK20}).
Roughly, $H^+(E)$ is defined by the action of $\Diff^+(X)$ on the Grassmannian of maximal-dimensional positive-definite subspaces of $H^2(X;\R)$ together with the transition functions of $E$.
While $H^+(E)$ is determined by $E$ only up to isomorphism, once we take a fiberwise metric on $E$, the set of fiberwise self-dual harmonic 2-forms gives a model of $H^+(E)$.

Now let $X$ be an oriented closed smooth 4-manifold with $b^+(X) \geq 2$ and $b_1(X)=0$, and let $B$ be a closed smooth manifold $B$ of dimension $k>0$.
Let $X \to E_1 \to B$ denote the trivial bundle: $E_1= B \times X$.
Let $kS^2 \times S^2 \to E_2 \to B$ be a $\Diff^+(kS^2 \times S^2)$-bundle over $B$ and suppose that $E_2$ admits a section.
Then we may form a fiberwise connected sum $X\#k(S^2\times S^2) \to E_1\#_f E_2 \to B$
 by choosing a section of $E_1$.
Let $\fraks_S$ denote the spin structure on $kS^2\times S^2$.
For a $\spinc$ structure $\frakt$ with $d(\frakt)=0$ on $X$, the connected sum $\frakt\#\fraks_S$ is invariant under the monodromy of $E_1\#_f E_2$ and we have $d(\frakt\#\fraks_S)=-k$. Thus the numerical families invariant $\SW(E_1\#_f E_2, \frakt\#\fraks_S) \in \Z/2$ makes sense.

\begin{pro}
\label{prop: gluing BK}
In the above setup, suppose that the Stiefel--Whitney class $w_{k}(H^+(E_2)) \in H^k(B;\Z/2)$ is non-zero.
Then, for a spin$^c$ structure $\frakt$ on $X$ with $d(\frakt)=0$, we have
\[
\SW(E_1\#_f E_2, \frakt\#\fraks_S)
= \SW(X,\frakt)
\]
in $\Z/2$.
In particular, we have
\[
\sum_{\frakt \in \tau_0(\Spincns(X,0)/\Conj)}\SW(E_1\#_f E_2, \frakt\#\fraks_S)
= \SWbbhalftot^0(X)
\]
in $\Z/2$.
\end{pro}

\begin{proof}
This is a direct application of
a families gluing formula proven in \cite[Theorem~1.1]{BK20}.
\end{proof}


\subsection{Key calculation}
\label{subsection Multiple mapping torus}

In this \lcnamecref{subsection Multiple mapping torus}, we calculate the Seiberg--Witten characteristic class for a special type of  multiple mapping tori (\cref{thm key computation source}).
This is the key computation to prove our instability theorem.

We again start with some notations. Let $W$ be a smooth 4-manifold with a single boundary component. We use $W_{S}$ to denote the once stabilized manifold
\[
W \cup_{\del W} \left(([0,1] \times \del W)\#S^2\times S^2\right).
\]
Here $([0,1] \times \del W)\#S^2\times S^2$ denotes the inner connected sum.
For $k\geq 1$, we use  $W_{kS}$ to denote the $k$-th stabilized manifold: we use  $W_{kS}$ instead of $W\#kS^2\times S^2$ when we want to specify the way of doing connected sums. Note that we have natural stabilization maps
\begin{equation}\label{eq: stabilization map}
s_{W}: \BDiff_{\del}(W) \rightarrow \BDiff_{\del}(W_{S}) \end{equation}
and
\begin{equation}\label{eq: stabilization map  discrete}
s_{W}^{\delta}: \BDiff_{\del}(W_{kS})^{\delta} \rightarrow \BDiff_{\del}(W_{(k+1)S})^{\delta}
\end{equation}
defined by extending a diffeomorphism with the identity map. We write $s^{\delta}$ and $s$ when $W$ is clear from the context.

We can also define the stabilization of a closed manifold $X$. We pick a smooth embedded $D^{4}$ in a local chart and consider the punctured manifold $\mathring{X}:=X\setminus
\Int(D^4)$. Then we define \[
X_{kS}:=\mathring{X}_{kS}\cup_{S^3}D^4.
\]
Note that the stabilization maps (\ref{eq: stabilization map}) and (\ref{eq: stabilization map  discrete} are not well-defined for closed $X$. 
Next, we discuss a general construction of multiple mapping with fiber $W_{kS}$. Let $S=S^2 \times S^2 \setminus \Int(D^4)$ and let $f \in \Diff_{\del}(S)$ be a relative diffeomorphism that acts as the $(-1)$-multiplication on $H_2(S;\mathbb{Z})$.
Such $f$ can be obtained from the diffeomorphism defined as the product of two copies of the reflection on $S^2$ about the equator, after deforming near $D^4$ by isotopy. Note that $f^2$ is the boundary Dehn twist on $S^2 \times S^2 \setminus \Int(D^4)$. Using the circle action on $S^{2}\times S^2$, one find a smooth isotopy $\{F^{t}\}_{t\in [0,1]}$ from $f^2$ to the identity.

Note that there are $k$ disjoint copies of embedded $S$ in $W_{kS}$, which we denote by $S_{1},\cdots, S_{k}$. By extending the diffeomorphism $f$ on $S_{i}$ with the identity map on $W\setminus S_{i}$, we obtain a diffeomorphisms \[f_{1},\cdots, f_{k}:W_{kS}\rightarrow W_{kS}\]
which mutually commute with each other.
Furthermore, by extending the isotopy $\{F^{t}\}_{t\in [0,1]}$ on $S_{i}$ with the identity map, we obtain a smooth isotopy 
\begin{equation}\label{eq: isotopy Ft}
\{F^{t}_{i}:W_{kS}\rightarrow W_{kS}\}_{t\in[0,1]}    
\end{equation}
 from $f^2_{i}$ to the identity map, which is supported in $S_{i}$.


Now we take the multiple mapping torus 
\begin{equation}\label{eq: multiple mapping torus}
W_{kS}\rightarrow (W_{kS})_{f_{1},f_2,\cdots,f_{k}}\rightarrow T^{k}
\end{equation}
and denote it by $E^{k}(W)$. The fiber bundle $E^{k}(W)$ carries a natural flat structure and hence induces a map 
\[
E^{k}(W): T^{k}\rightarrow \BDiff_{\partial}(W_{kS})^{\delta}.
\]

\begin{lem}
\label{lem: flat 2-torsion}
For any $W$ with zero or one boundary component, one has 
\[2\cdot E^{k}(W)_{*}[T^{k}]=0\in H_{k}(\BDiff_{\del}(W_{kS})^{\delta};\mathbb{Z}).\]
\end{lem}
\begin{proof}
Consider the multiple mapping torus 
\[
W_{kS}\rightarrow (W_{kS})_{f^2_{1},f_2,\cdots,f_{k}}\rightarrow T^{k}.
\]
It is the pull-back of $E^{k}(W)$ under a degree-$2$ map $T^{k}\rightarrow T^{k}$. Hence we have
\[2\cdot E^{k}(W)_{*}[T^{k}]=((W_{kS})_{f^2_{1},f_2,\cdots,f_{k}})_{*}[T^{k}].\]
We can apply Lemma \ref{lem: 2-torsion} to finish the proof.
\end{proof}


\begin{lem}\label{lem: Ek stabilization}
For any $W$ with a single boundary component, the following diagram commutes up to homotopy \begin{align}\label{diagram: stabilization}
\begin{split}
\xymatrix{
     & \BDiff_{\del}(W_{(k+1)S})^{\delta} \\
    T^k \ar[ru]^-{E^{k}(W_{S})} \ar[r]_-{E^{k}(W)} & \BDiff_{\del}(W_{kS})^{\delta}. \ar[u]_{s_{W_{kS}}^{\delta}}
   }
\end{split}   
\end{align}
Furthermore, let $W'$ be  another smooth manifold with boundary. Then $E^{k}(W)$ and $E^{k}(W')$ are isomorphic as smooth bundles with flat structures if $W$ and $W'$ are diffeomorphic. And $E^{k}(W)$ and $E^{k}(W')$ are isomorphic as topological bundles with flat structures if $W$ and $W'$ are homeomorphic.
\end{lem}
\begin{proof}
This is a straightforward checking using definition.
\end{proof}
Now we set $W$ to be a closed manifold $X$ and discuss the families Seiberg--Witten invariant. 
\begin{lem}\label{lem: SW-halftot for mapping tori}
For any oriented closed smooth 4-manifold $X$ with $b^{+}(X)\geq k+2$, we have
\[ 
\langle\SWbbhalftot^{k}(E^{k}(X)),[T^{k}]\rangle=\SWbbhalftot^{0}(X).\]
\end{lem}
\begin{proof} 
First, we fix a section 
\[
\tau_0 : \Spincns(X)/\Conj \to \Spincns(X).
\]
Then, for given $[\fraks] \in \Spincns(X_{kS})/\Conj$, we set $\tau([\fraks])$ to be the $\spinc$ structure $\fraks'$ with $[\fraks']=[\fraks]$ that satisfies $\fraks'|_{X} = \tau_0([\fraks'|_{X}])$.
This defines a section \[\tau_1 : \Spincns(X_{kS})/\Conj \to \Spincns(X_{kS}),\] and
restricting this, we obtain a section \[\tau : \Spincns(X_{kS},k)/\Conj \to \Spincns(X_{kS},k).\]

Note that the isotopy $\{F^{t}_{i}\}$ satisfies the conditions of \cref{cor: cancel}. 
So we can reduce the calculation of $\SWbbhalftot^k(E')$ to monodromy invariant $\spinc$ structures;
By \cref{cor: cancel}, we have  
\begin{align}
\label{eq: eval sum4}
\langle\SWbbhalftot^k(E^k(X)), [T^k] \rangle  =
\sum_{\substack{\fraks \in \tau(\Spincns(X_{kS},k)/\Conj)\\ f_i^\ast \fraks=\fraks\ (\forall i \in \{1, \dots, k\})}}\SW(E^{k}(X), \fraks).
\end{align}
On the other hand, the action of $f_{i}$ on the second homology group equals $-1$ on $(S^{2}\times S^{2})_{i}$ and equals $1$ on the other summands. From this, we see that a $\spinc$ structure on $X_{kS}$ is preserved by all $f_{i}$ if and only if it can be written as the connected sum $\mathfrak{t}\#\mathfrak{s}_{S}$ between a $\spinc$ structure $\mathfrak{t}$ on $X$ and the spin structure $\mathfrak{s}_{S}$ on $k(S^2\times S^2)$. So we can apply \cref{prop: gluing BK} and get 
\[
\begin{split}
\sum_{\substack{\fraks \in \tau(\Spincns(X_{kS},k)/\Conj)\\ f_i^\ast \fraks=\fraks\ (\forall i \in \{1, \dots, k\})}}\SW(E^{k}(X), \fraks)&=\sum_{\frakt \in \tau(\Spincns(X,0)/\Conj)}\SW(E^{k}(X), \frakt\#\fraks_{S})\\
&=\sum_{\frakt \in \tau(\Spincns(X,0)/\Conj)}\SW(X, \frakt)\\
&=\SWbbhalftot^0(X)
\end{split}
\]
This finishes the proof.\end{proof}



Next, let us consider a natural map 
\begin{equation}\label{eq: extension map}
\rho : \BDiff_{\del}(\mathring{X}) \to \BDiff^+(X)
\end{equation}
induced from a map extending a relative diffeomorphism by the identity of $D^4$.
We call $\rho$ the extension map, and drop $X$ from our notation.
%We denote by
%\[
%s_{X,k} : H_k(\BDiff_{\del}(\mathring{X});\Z)
%\to H_k(\BDiff_{\del}(\mathring{X}_{S});\Z)
%\]
%the map on $k$-th homologies induced by 
%the natural stabilization map $s$.
To state the main result of this \lcnamecref{subsection Multiple mapping torus}, we need one more definition. 


\begin{defi}
Let $X$ be a simply-connected, closed, smooth, oriented 4-manifold. We say that $X$ dissolve if it is diffeomorphic to $a\mathbb{CP}^{2}\#b\overline{\mathbb{CP}}^{2}$ or $c(S^2\times S^2)\#d K3$ for some $a,b,c\in \mathbb{N}$ and $d\in \mathbb{Z}$. We say $X$ dissolves after $n$ stabilizations if $X\#n(S^{2}\times S^2)$ dissolves. 
\end{defi}

Now we are ready to state the main result of this \lcnamecref{subsection Multiple mapping torus}.


\begin{thm}
\label{thm key computation source}
Let $k > 0$ and let $X$ be a simply-connected closed oriented indefinite smooth 4-manifold with $b^+(X) \geq k+2$.
Assume that $\SWbbhalftot^0(X)=1$ in $\Z/2$, $X$ dissolves after one stabilization, and $X$ is not homotopic to $K3$.
Then there exist an element \[\alpha^{\delta}_{k}(X)\in H_{k}(\BDiff_{\del}(\mathring{X}_{kS})^{\delta};\mathbb{Z}) \]
That satisfies the following properties:
\begin{enumerate}[label=(\roman*)]
\item $\alpha^{\delta}_{k}(X)$ is $2$-torsion,
\item $\alpha^{\delta}_{k}(X)$ belongs to the kernel of the stabilization map 
\[
s^{\delta}_{*}:  H_{k}(\BDiff_{\del}(\mathring{X}_{kS})^{\delta};\mathbb{Z})\rightarrow  H_{k}(\BDiff_{\del}(\mathring{X}_{(k+1)S})^{\delta};\mathbb{Z}).\]
\item Let $\alpha_{k}(X)$ be the image of $\alpha^{\delta}_{k}(X)$ under the forgetful map 
\[
H_{k}(\BDiff_{\del}(\mathring{X}_{kS})^{\delta};\mathbb{Z})\rightarrow H_{k}(\BDiff_{\del}(\mathring{X}_{kS});\mathbb{Z}).
\] Then 
\[0 \neq \langle \SWbbhalftot^{k}(X_{kS}),\rho_{*}(\alpha_{k}(X))\rangle\in \mathbb{Z}/2\]
\item $\alpha_{k}(X) $ and $\alpha_k^{\delta}(X)$ are both  topologically trivial. Namely, we have 
\[
\begin{split}
\alpha^{\delta}_{k}(X) \in \ker(H_\ast(\BDiff_\del(\mathring{X}_{kS})^{\delta};\Z) &\to H_\ast(\BHomeo_\del(\mathring{X}_{kS})^{\delta};\Z)),\\
\alpha_{k}(X) \in \ker(H_\ast(\BDiff_\del(\mathring{X}_{kS});\Z) &\to H_\ast(\BHomeo_\del(\mathring{X}_{kS});\Z)).
\end{split}
\]
\end{enumerate}
\end{thm}

\begin{rmk}
\label{rmk: MMM class}
\cref{thm key computation source} implies that $\SWbbhalftot$ detects topologically trivial non-zero homology class of $\BDiff^+(X)$ in $\Z$ or $\Z/2$-coefficient.
On the other hand, the (generalized) Mumford--Morita--Miller classes, which are basic characteristic classes for smooth fiber bundles, are generalized also for topological fiber bundles over $\mathbb{F} = \Q$ or $\Z/2$ \cite{ERW14}.
Thus the Mumford--Morita--Miller classes do not detect kernels of $H_\ast(\BDiff^+(X);\mathbb{F}) \to H_\ast(\BHomeo^+(X);\mathbb{F})$.
\end{rmk}

\begin{proof}[Proof of \cref{thm key computation source}]By the assumption on the dissolving and that $X$ is not homotopic to $K3$, there is a 4-manifold $X'$ such that the following conditions hold:
\begin{enumerate}
   \item $X\# S^2\times S^2$ is diffeomorphic to   $X'\# S^2\times S^2$.
    \item $X'$ is homeomorphic to $X$.
    \item The Seiberg--Witten invariant vanishes for any $\spinc$ structure on $X'$. In particular, this implies $\SWbbhalftot^{0}(X')=0$.
\end{enumerate}

Consider the bundles $E^{k}(\mathring{X})$ and $E^{k}(\mathring{X}')$. Since $X'_{kS}$ is diffeomorphic to $X_{kS}$, we have two maps
\[
E^{k}(\mathring{X}), E^{k}(\mathring{X}'): T^k\rightarrow \BDiff_{\del}(\mathring{X}_{kS})^{\delta}.
\]
Then we set 
\[\alpha^{\delta}_{k}(X):=E^{k}(\mathring{X})_{*}[T^{k}]-E^{k}(\mathring{X}')_{*}[T^{k}].\]
Now we verify the four properties as follows:
\begin{itemize}
    \item (i) directly follows from \cref{lem: flat 2-torsion}. 
    \item  By 
\cref{lem: Ek stabilization}, we have \[
s^{\delta}_{*}E^{k}(\mathring{X})_{*}[T^{k}]=E^{k}(\mathring{X}_{S})_{*}[T^k],\quad s^{\delta}_{*}E^{k}(\mathring{X}')_{*}[T^{k}]=E^{k}(\mathring{X}'_{S})_{*}[T^k].
\]
Since $\mathring{X}_{S}$ is diffeomorphic to $\mathring{X}'_{S}$, (ii) is proved.
\item (iii) is equivalent to 
\begin{equation*}
\langle \SWbbhalftot^{k}(E^{k}(X)),[T^{k}]\rangle \neq \langle \SWbbhalftot^{k}(E^{k}(X')),[T^{k}]\rangle.
\end{equation*}
By \cref{lem: SW-halftot for mapping tori}, this is equivalent to 
\[
\SWbbhalftot^{0}(X)\neq \SWbbhalftot^{0}(X'),
\]
which is exactly our assumption.
\item 
Since $X$ is homeomorphic to $X'$, $E^{k}(\mathring{X})$ and $E^{k}(\mathring{X}')$ are isomorpic as topological bundles with flat structure. Therefore, the maps $E^{k}(\mathring{X})$ and $ E^{k}(\mathring{X}')$ are homotopic after being composed with the forgetful map
\[
\BDiff_\del(\mathring{X}_{kS})^{\delta}\rightarrow \BHomeo_\del(\mathring{X}_{kS})^{\delta}.
\]
This proves that $\alpha^{\delta}_{k}(X)$ is topologically trivial, which implies that $\alpha_{k}(X)$ is trivial as well.
\end{itemize}
\end{proof}



\begin{cor}
\label{thm: Diff Homeo closed 4-manifold}
Let $k > 0$ and let $X$ be a simply-connected closed oriented indefinite smooth 4-manifold with $b^+(X) \geq k+2$.
Assume that $\SWbbhalftot^0(X)=1$ in $\Z/2$, $X$ dissolves after one stabilization, and $X$ is not homotopic to $K3$. Consider the forgetful map $i:\Diff^+(X_{kS})\rightarrow \Homeo^{+}(X_{kS})$.
Then the induced map 
\[
i_{*}:H_k(\BDiff^+(X_{kS});\Z)
\to H_k(\BHomeo^+(X_{kS});\Z)
\]
is not injective and the induced map 
\[
i^{*}:H^k(\BHomeo^+(X_{kS});\Z/2)\rightarrow H^k(\BDiff^+(X_{kS});\Z/2)
\]
is not surjective. Analogous results hold for the maps $i^{\delta}_{*}$ and $i_{\delta}^{*}$ induced by the inclusion $i^\delta:\Diff^+(X_{kS})^{\delta}\hookrightarrow \Homeo^{+}(X_{kS})^{\delta}$.
\end{cor}
\begin{proof}
Set $\alpha=\rho_{*}(\alpha_{k}(X))\in H_k(\BDiff^+(X_{kS});\Z)$.
Then $\alpha\neq 0$ by \cref{thm key computation source} (ii). By \cref{thm key computation source} (iv),  $\alpha$ maps to zero in  $H_k(\BHomeo^+(X_{kS});\Z)$. This implies that $\SWbbhalftot^k(X_{kS})$ is not in the image of $i^{*}$. By using $\alpha^{\delta}_{k}(X)$ instead of $\alpha_{k}(X)$, one can prove analogous results for $i^{\delta}_{*}$ and $i_{\delta}^{*}$.
\end{proof}
\cref{thm: Diff Homeo closed 4-manifold} combined with a certain result on geography implies
\cref{thm: Diff Homeo sequence general}.
We shall carry this out in \cref{subsectionProof of the main instability theorem}.

\begin{ex}
\label{ex: Diff Homeo example}
There are many examples of 4-manifolds $X$ that satisfy the assumption of \cref{thm: Diff Homeo closed 4-manifold}.
As explained in \cref{ex: minimal algebraic surfaces of general type}, we know that a minimal algebraic surface $X$ of general type satisfies that $\SWbbhalftot^0(X)\neq0$. And many symplectic manifolds with the same property were constructed in \cite{Fintushel2002}. By using these manifolds as building blocks and doing fiber sum, we will construct a large number of smooth 4-manifolds $X$ with $\SWbbhalftot^0(X)\neq 0$ that dissolve after one stabilization. (See Theorem \ref{thm: 4-mfds that dissolves}.)
Thus, for such $X$, we have that \[
H_k(\BDiff^+(X_{kS});\Z)
\to H_k(\BHomeo^+(X_{kS});\Z)
\]
is not injective for any $k$.
\end{ex}


\subsection{Proof of the main results}
\label{subsectionProof of the main instability theorem}

Now we have prepared all necessary results on families Seiberg--Witten theory to prove the main results described in \cref{section Introduction}.
%main instability theorem (\cref{thm: main cal}), the comparison result on $\Diff$ vs. $\Homeo$ for closed 4-manifolds (\cref{thm: Diff Homeo sequence general}), and the result on abelianizations of diffeomorphism groups and mapping class groups (\cref{thm: abelianisation noninjective}).
The only remaining piece of the proofs of those is the following result on geography involving the usual (i.e. unparameterized) Seiberg--Witten invariant:
\begin{thm}\label{thm: 4-mfds that dissolves} For any integer $l\in \mathbb{Z}$, there exists a sequence of simply connected $4$-manifolds $\{X^{l}_{i}\}^{\infty}_{i=1}$ that satisfies the following properties:
\begin{enumerate}
\item $\sign(X^l_i)=l$,
\item $b_2(X^l_i) \to +\infty$ as $i \to +\infty$,
\item $X^{l}_{i}$ is of Seiberg--Witten simple type,
\item $
\SWbbhalftot^{0}(X^{l}_{i})=1\in \mathbb{Z}/2,
$ and
\item $X^l_i$ dissolves after one stabilization.
\item Suppose $l$ is divisible by $16$. Then we can pick $X^l_i$ to be spin or nonspin. 
\end{enumerate}
\end{thm}

To avoid a digression, we postpone the proof of \cref{thm: 4-mfds that dissolves} to \cref{construction of 4-manifolds}.
Let us give the proofs of the main results in \cref{section Introduction} assuming  \cref{thm: 4-mfds that dissolves}.

\begin{proof}[Proof of \cref{thm: main cal}]
We set $l=\sign(X)$ and denote by $X_i$ the 4-manifold $X^l_i$ in \cref{thm: 4-mfds that dissolves}. Here, we choose $X_{i}$ to be of the same type (i.e. spin/non-spin) as $X$.
By Wall's theorem \cite{Wall64}, there exists a positive integer $n(X)$ such that $X_{n(X)S}$ dissolves. There exists an integer $i_0$ such that for any $i>i_0$, the manifold $X_{i}$ is indefinite, not homotopy equivalent to $K3$ and satisfies 
\[
b_{2}(X_{i})+2k\geq b_{2}(X)+2n(X).
\]
For any $i>0$, we set
\[
N_{i}:=\frac{b_2(X_{i_0+i})-b_{2}(X)}{2}+k.
\]
Then we have $N_i \to +\infty$. Furthermore, $(X_{i_0+i})_{kS}$ is diffeomorphic to $X_{N_{i}S}$ since they have the same signature, Euler number, type and both dissolve. We apply \cref{thm key computation source} to the manifold $X_{i_0+i}$ and obtain homology classes 
\[
\alpha^{\delta}_{k}(X_{i_0+i})\in H_{k}(\BDiff_{\del}(\mathring{X}_{N_{i}S})^{\delta};\mathbb{Z})\text{ and }\alpha_{k}(X_{i_0+i})\in H_{k}(\BDiff_{\del}(\mathring{X}_{N_{i}S});\mathbb{Z}).
\]
We set $\alpha^{\delta}_{i}=\alpha^{\delta}_{k}(X_{i_0+i})$ and $\alpha_{i}=\alpha_{k}(X_{i_0+i})$. It remains to verify that they satisfy the desired property. By  \cref{thm key computation source}(iv), $\alpha^{\delta}_{i}$ and $\alpha_{i}$ are both topologically trivial. 
By \cref{thm key computation source}(ii), $\alpha^{\delta}_{i}\in \ker s^{\delta}_{N_{i},*}$. This implies that $\alpha_{i}\in \ker s_{N_{i},*}$. 
By \cref{thm key computation source}(iii), we have 
\begin{align}
\label{eq: non-vanishing pairing proof of main}
 \langle \SWbbhalftot^{k}(X_{N_iS}),\rho_{*}(\alpha_{i})\rangle\neq 0.
\end{align}
This combined with \cref{cor: vanishing} show that \[\alpha_{i}\notin \operatorname{Image}(s_{N_{i}-k-1,*}\circ\cdots\circ s_{N_{i}-1,*}).\]
 This further implies that 
 \[\alpha^{\delta}_{i}\notin \operatorname{Image}(s^{\delta}_{N_{i}-k-1,*}\circ\cdots\circ s^{\delta}_{N_{i}-1,*}),
 \]
which completes the proof.
\end{proof}

%The comparison result (\cref{thm: Diff Homeo}) on groups of diffeomorphisms and homeomorphisms described in the introduction is a direct consequence of the following:



\begin{proof}[Proof of \cref{thm: Diff Homeo sequence general}] We let $N_{i}$ be chosen as in the proof of \cref{thm: main cal}.
Then we apply \cref{thm: Diff Homeo closed 4-manifold} to the manifold $X_{i_0+i}$ to conclude that the maps $i_{*}, i^{\delta}_{*}$ are both non-injective and the maps $i^{*}, i^{*}_{\delta}$ are both non-surjective. The proof is finished by the fact that $(X_{i_0+i})_{kS}$ is diffeomorphic to $X_{N_{i}S}$.
\end{proof}


\begin{rmk}
\label{rmk: all ex dissolve}
In \cref{thm: Diff Homeo sequence general}, all $X\#N_i S^2 \times S^2$ are are connected sums of $\CP^2, S^2\times S^2$, $K3$ and their orientation reversals.
This is because we choose $n_i$ in the proof of \cref{thm: Diff Homeo sequence general} so that $X\#n_iS^2\times S^2$ dissolve, and so do $X\#N_iS^2\times S^2$.
\end{rmk}


\begin{proof}[Proof of \cref{thm: abelianisation noninjective}] We set $N_i$ to be the same as in \cref{thm: main cal} and let $\beta_{i}=\rho_{*}(\alpha_i)$. It suffices to show that $\beta_i$ can be represented by an exotic diffeomorphism, since all other assertions directly follow from the corresponding assertions in  \cref{thm: main cal} and \cref{thm: Diff Homeo sequence general}. 

By the construction of $\alpha_i$ in the proof of \cref{thm: main cal} and \cref{thm key computation source}, we see that $\beta_{i}$, when treated as a homology class in $H_{1}(\BDiff^{+}(X_{N_i}S);\mathbb{Z})$, can be expressed as the difference 
\[
(X_{N_{i}S})_{f_1,*}[T^1]-(X_{N_{i}S})_{f'_1,*}[T^1]\in H_{1}(\BDiff^{+}(X_{N_i}S);\mathbb{Z}).
\]
Here $(X_{N_{i}S})_{f_1}$ and $(X_{N_{i}S})_{f'_1}$ are mapping tori for diffeomorphisms  $f_1,f_1'\in \Diff^{+}(X_{N_{i}S})$. Furthermore, by \eqref{FreeQuinnPerron}, there is a homeomorphism $g:X_{N_{i}S}\rightarrow  X_{N_{i}S}$ such that $f_1$ is topologically isotopic to $g\circ f'_1\circ g^{-1}$. 
We may assume $g$ is actually a diffeomorphism since the map (\ref{eq: MCG surjective}) is surjective for $N=N_i$. Then \[f\circ g\circ f'^{-1}\circ g^{-1}: X_{N_iS}\rightarrow X_{N_iS}\]
is an exotic diffeomorphism that represents $\beta_i$. 
\end{proof}


\begin{proof}[Proof of \cref{thm: psc main}]
Let $\alpha_i \in H_k(\BDiff_{\del}(\mathring{X}\# N_i S^2\times S^2);\Z)$ be the homology class that was constructed in the proof of \cref{thm: main cal}.
Recall that $\alpha_i$ is unstable and the pairing with $\SWbbhalftot^k$ is non-zero, \eqref{eq: non-vanishing pairing proof of main}.
Thus the claim of the \lcnamecref{thm: psc main} follows from \cref{cor: psc non-surj}.
\end{proof}


\begin{rmk}\label{rmk: twisted stablization details} As we mentioned in \cref{rmk: twisted stabilization}, the whole argument can be adapted to twisted stabilizations  (i.e., taking connected sum with $\mathbb{CP}^{2}\#\overline{\mathbb{CP}^2}$ instead of $S^2\times S^2$). We summarize the adaptions needed here. For $k\geq 0$, we use $X_{k\tilde{S}}$ to denote $X\#k(\mathbb{CP}^{2}\#\overline{\mathbb{CP}^2})$. 
For simply-connected nonspin $X$, there is a diffeomorphism between $X_{kS}$ and $X_{k\tilde{S}}$ for any $k$
\cite[Corollary~1]{Wall64D}.
%(see \cite{Baykur13}).
For such $X$, the multiple mapping torus $E^{k}(X)$ is a smooth bundle with fiber $X_{k\tilde{S}}$.  (See \eqref{eq: multiple mapping torus}.)  Note that we do not adopt the construction of $E^{k}(X)$ so the monodromies are still diffeomorphisms supported on copies of on embedded $(S^2\times S^2)\setminus \Int(D^4)$, not embedded $(\mathbb{CP}^{2}\#\overline{\mathbb{CP}^2})\setminus \Int(D^4)$. With this in mind, one can repeat the proof of \cref{thm key computation source} for non-spin $X$ and show that \cref{thm key computation source} remains true if one replaces $X_{kS}$ with $X_{k\widetilde{S}}$ and instead consider the twisted stabilization map
\[
\tilde{s}^{\delta}_{*}:  H_{k}(\BDiff_{\del}(\mathring{X}_{kS})^{\delta};\mathbb{Z})\rightarrow  H_{k}(\BDiff_{\del}(\mathring{X}_{(k+1)S})^{\delta};\mathbb{Z}).\]
Furthermore, by applying \cref{thm: vanishing2} to $N=(k+1)(\mathbb{CP}^{2}\#\overline{\mathbb{CP}^2}))$, one obtain a variation of \cref{cor: vanishing} for twisted stabilizations. With these results proved, we can repeat the proofs of \cref{thm: main cal,thm: Diff Homeo sequence general,thm: abelianisation noninjective} word by word for twisted stabilizations on a non-spin $X$. For a spin $X$, we simply replace it with $X_{\tilde{S}}$ and reduce it to the non-spin case.
\end{rmk}

%\subsection{ $\Z/2$-coefficient versions}

%Here is a remark about coefficents.
%The instability theorem (\cref{thm: main cal}) and a consequence of its proof, a comparison result on $\Diff$ vs. $\Homeo$ (\cref{thm: Diff Homeo sequence general}), were proven by seeing the non-triviality of a homology class of the classifying space.
%Since the non-triviality was checked by the pairing with a $\Z/2$-coefficient cohomology class, the unstable homology class we detected is still non-trivial in  $\Z/2$-coefficient homology.
%Hence a similar instability and comparison results holds also in $\Z/2$-coefficient homology.
%We record these results:

%\begin{thm}
%\label{theo: instablity Z mod 2}
%Let $X$ be a simply-connected closed oriented smooth 4-manifold and let $k>0$.
%Then there exists an increasing sequence of positive integers $N_1<N_2<\cdots \to +\infty$ such that
%the stabilization maps
%\begin{align*}
%H_k(\BDiff_{\del}(\mathring{X}\#N_iS^2 \times S^2);\Z/2) \to H_k(\BDiff_{\del}(\mathring{X}\#(N_i+1)S^2 \times S^2);\Z/2)
%\end{align*}
%are not injective for all $i$.
%Furthermore, for all $i$, there are non-zero unstbale and topologically trivial homology classes in $H_k(\BDiff_{\del}(\mathring{X}\#N_iS^2 \times S^2);\Z/2)$.

%Also, the natural maps
%\[
%H_k(\BDiff^+(X\#N_i S^2 \times S^2);\Z/2)
%\to H_k(\BHomeo^+(X\#N_iS^2 \times S^2);\Z/2)
%\]
%are not injective for all $i$.
%\end{thm}

%However, we do not know whether analogous results hold for homology with another coefficient, such as rational homology.


\subsection{Comparison with Torelli groups}

\cref{thm: main cal} also leads us to an interesting comparison between different types of diffeomorphism groups.
Let $\TDiff(X)$ denote a subgroup of $\Diff(X)^+$ defined to be the group of diffeomorphisms that act trivially on homology $H_\ast(X;\Z)$.
This subgroup $\TDiff(X)$ is often called the {\it Torelli group}.

\begin{pro}
\label{pro: Torelli vs the whole}
Let $X$ be a simply-connected closed oriented indefinite smooth 4-manifold with $b^+(X) \geq 3$.
Assume that $\SW(X,\frakt)\neq 0$ for some $\frakt \in \Spinc(X,0)$, $X$ dissolves after one stabilization, and $X$ is not homotopic to $K3$.
Then both of the following natural maps are not injective:
\begin{align}
H_1(\BTDiff(X\#S^2\times S^2);\Z) &\to H_1(\BDiff^+(X\#S^2\times S^2);\Z),\label{eq: ker H1s}\\
\pi_1(\BDiff^+(X\#S^2\times S^2)) &\to H_1(\BDiff^+(X\#S^2\times S^2);\Z).\label{eq: ker pi1H1}
\end{align}
Furthermore, the kernels of these maps contain subgroups isomorphic to $\Z$.
\end{pro}

\begin{proof}
In general, for a closed oriented smooth 4-manifold $Z$ with a fixed $\spinc$ structure $\fraks$ and a homology orientation $\mathcal{O}$,
let $\Diff(Z,\fraks,\mathcal{O})$ be the group of diffeomorphisms that preserve orientation of $Z$, $\fraks$ and $\mathcal{O}$.
A characterisric class $\SWbb(Z,\fraks,\mathcal{O}) \in H^1(\BDiff^+(Z,\fraks,\mathcal{O});\Z)$ was defined in \cite{K21}.
We may pull-back $\SWbb(Z,\fraks,\mathcal{O})$ under the natural map $\BTDiff(Z) \to \BDiff(Z,\fraks,\mathcal{O})$, and obtain a cohomology class denoted by $\SWbb^T(Z,\fraks,\mathcal{O}) \in H^1(\BTDiff(Z);\Z)$.

Now let $X$ be as in the statement of \lcnamecref{pro: Torelli vs the whole}, and set $Z=X\#S^2\times S^2$.
Let $\fraks \in \Spinc(Z,1)$ be the connected sum of $\frakt$ with the spin structure on $S^2 \times S^2$.
Fix a homology orientation $\mathcal{O}$ on $Z$.

Let us consider the bundle $E$ given in the proof of \cref{thm key computation source}.
First, by the construction of $E$, the structure group of $E$ reduces to $\TDiff(Z)$.
Let $E^T : S^1 \to \BTDiff(Z)$ be the classifying map of the family $E$ with structure group $\TDiff(Z)$.
From the gluing formula, \cref{prop: gluing BK}, it follows that the (numerical) families Seiberg--Witten invariant $\SW(E,\fraks)$ is non-zero.
This implies that 
\[
\langle\SWbb^T(Z,\fraks,\mathcal{O}), E^T_\ast([S^1])\rangle \neq 0.
\]
This implies that $E^T_\ast([S^1])$ generates a subgroup of $H_1(\BTDiff(Z);\Z)$ isomorphic to $\Z$.

As we can see from the proof of  \cref{thm: main cal}, $\alpha=E_\ast([S^1]) \in H_k(\BDiff_{\del}(Z);\Z)$ is a 2-torsion, and thus we have that $2E^T_\ast([S^1])$ generates a subgroup of the kernel of \eqref{eq: ker H1s} isomorphic to $\Z$.

The claim on the map \eqref{eq: ker pi1H1} immediately follows from the above argument combined with the following commutative diagram, where $h$ denote the Hurewicz maps:
\begin{align}
\label{eq: Hurewicz diagram}
\begin{split}
\xymatrix{
    \pi_1(\BTDiff(Z))\ar[r]^-{h} \ar[d]^-{\cong} & H_1(\BTDiff(Z);\Z) \ar[r] & H_1(\BDiff^+(Z);\Z) \\
    \pi_0(\TDiff(Z))\ar[r]_-{\subset} & \pi_0(\Diff^+(Z)) \ar[r]^{\cong} &
    \pi_1(\BDiff^+(Z))\ar[u]_-{h}.
   }
\end{split}
\end{align}
\end{proof}

Adapting Ruberman's argument \cite{Rub99} in the Seiberg--Witten setup, we can strengthen
\cref{pro: Torelli vs the whole} for some class of 4-manifolds $X$.
Let $\mathcal{B}(X)$ denote the set of Seiberg--Witten basic classes of $X$ of formal dimension zero.

\begin{pro}
\label{pro: Torelli vs the whole strong}
Let $X$ be a simply-connected closed oriented indefinite smooth 4-manifold with $b^+(X) \geq 3$.
Assume that there are infinitely many smooth oriented 4-manifolds $\{X_i\}_{i=1}^\infty$ such that $X_i \#S^2\times S^2$ is diffeomorphic to $X\#S^2\times S^2$ for every $i$, and $\#\mathcal{B}(X_i) \to +\infty$.
Then the kernels of both of the following natural maps
\begin{align}
H_1(\BTDiff(X\#S^2\times S^2);\Z) &\to H_1(\BDiff^+(X\#S^2\times S^2);\Z),\label{eq: ker H1s'}\\
\pi_1(\BDiff^+(X\#S^2\times S^2)) &\to H_1(\BDiff^+(X\#S^2\times S^2);\Z).\label{eq: ker pi1H1'}
\end{align}
contain subgroups isomorphic to $\Z^\infty=\oplus_{\Z}\Z$.
\end{pro}

\begin{proof}
Set $Z=X\#S^2\times S^2$.
By assumption, there is a sequence $\{\frakt_i\}_i \subset \Spinc(X,0)$ such that $\frakt_i \in \mathcal{B}(X_i) \setminus \mathcal{B}(X_{i-1})$ for all $i$.
Let $\fraks_i \in \Spinc(Z,1)$ be the connected sum of $\frakt_i$ with the spin structure on $S^2 \times S^2$.
By repeating the construction of the bundle $E \to S^1$ in \cref{thm key computation source} using $(X,X_i)$ in place of $(X,X')$, we obtain a diffeomorphism $f_i \in \TDiff(Z)$ such that
\begin{itemize}
\item $f_i$ is decomposed into $f_i=g_i \circ g_i'$, where the squares of $g_i, g_i'$ are smoothly isotopic to the identity.
\item $\SW(E_i,\fraks_i) \neq 0$, where $E_i \to S^1$ is the mapping torus of $f_i$ with fiber $Z$.
\item $\SW(E_i,\fraks_j) = 0$ for all $j>i$.
\end{itemize}
As we can see from the proof of  \cref{thm: main cal}, we have that $(E_i)_\ast([S^1]) \in H_1(\BDiff^+(Z);\Z)$ are 2-torsions.
On the other hand, let $E_i^T : S^1 \to \TDiff(X)$ be the classifying map of $E_i$ with structure group $\TDiff(Z)$.
Then, by evaluating $\{(E_i^T)_\ast([S^1])\}_i$ by 
the homomorphism
\[
\bigoplus_{\fraks \in \Spinc(Z,1)} \left<\SWbb(Z,\fraks,\mathcal{O}),-\right>
: H_1(\BTDiff(Z);\Z)
\to \bigoplus_{\fraks \in \Spinc(Z,1)}\Z \cong \Z^\infty,
\]
we have that $\{(E_i^T)_\ast([S^1])\}_i$ are linearly independent in $H_1(\BTDiff(Z);\Z)$.
Thus $\{2(E_i^T)_\ast([S^1])\}_i$ generates a subgroup in the kernel of \eqref{eq: ker H1s'} isomorphic to $\Z^\infty$.
The claim on the map \eqref{eq: ker pi1H1'} follows from this and the commutative diagram \eqref{eq: Hurewicz diagram} again.
\end{proof}

It is easy to find an example of $X$ and $X_i$ that satisfy the assumptions of \cref{pro: Torelli vs the whole strong} by the knot surgery formula due to Fintushel--Stern \cite{FS98}.

%\begin{ex}
%It is easy to find an example of $X$ and $X_i$ that satisfy the assumptions of \cref{pro: Torelli vs the whole strong}.
%For example, suppose that a 4-manifold $X$ has non-vanishing Seiberg--Witten invariant for some $\spinc$ structure and $X$ contains a torus $T$ with self-intersection zero such that and $[T] \in H_2(X;\Z)$ is not torsion, and $\pi_1(X\setminus T)=1$.
%Let $X_i$ by the knot surgery of $X$ along $T$ by a sequence of knots $K_i$ such that the degrees of Alexander polynomials of $K_i$ go to infinity as $i \to \infty$.
%Then, by the knot surgery formula due to Fintushel--Stern \cite{FS98}, we have $\#\mathcal{B}(X_i) \to +\infty$.
%\end{ex}


%For a knot $K$ in $S^3$, let $X_{K}$ denote the knot surgery of $X$ along $T$ using $K$. 
%Let $K_i$ be a sequence of knots such that the degrees of Alexander polynomials $\Delta_{K_i}(t)$ go to infinity as $i \to \infty$.
%Then it follows from  that $\#\mathcal{B}(X_{K_i}) \to +\infty$.






\section{Manifolds that dissolve after one stabilization}\label{construction of 4-manifolds}

In this section, we prove \cref{thm: 4-mfds that dissolves}. This will finish our argument. 
Our proof is inspired by the paper \cite{Gompf91,Hanke03,Akhmedov15}, where the authors constructed many examples of symplectic 4-manifolds that dissolve after one stabilization. The idea is to start with some symplectic 4-manifolds with a unique Seiberg--Witten basic class up to sign (e.g., minimal algebraic surfaces of general type) and do fiber sum along embedded surfaces with self-intersection $0$ and genus $\geq 2$. (We avoid doing fiber sum along tori or spheres because they usually annihilate the half-total Seiberg--Witten invariants.) 

We start with some notations. Let $M_{1}, \cdots ,M_{n}$ be smooth 4-manifolds. Let $F$ be an oriented genus $g$ closed surface and let  $F_{i}\subset M_{i}$ be a smoothly embedded surface $F_{i}$ of the same genus and self-intersection $0$.  We let  $D^{2}_{1},\cdots ,D^{2}_{n}$ be disjoint disks in $S^{2}$.
For each $i$, we pick an open tubular neighborhood $\nu(F_{i})$ and an orientation reversing diffeomorphism 
$$
\varphi_{i}: D^{2}_{i}\times F\rightarrow \overline{\nu(F_{i})}
$$
that covers a diffeomorphism $F\rightarrow F_{i}$. Then the fiber sum is defined as 
$$
M_{1}\#_{F}\cdots \#_{F}M_{n}:=((S^{2} \setminus \bigsqcup_{i} D^{2}_{i})\times F)\bigcup_{\varphi_{i}|_{\partial D^{2}_{i}\times F}} (\bigsqcup_{i} (M_{i}\setminus \nu(F_{i}))). 
$$
Of course, the result depends on the choice of the gluing function $\varphi_{i}$. Using $\varphi_{i}$, we also get a boundary parametrization $\partial(M_{i}\setminus \nu(F_{i})) \cong S^{1}\times F$. One has 
\[
\begin{split}
\sign(M_{1}\#_{F}\cdots \#_{F}M_{n})&=\sum^{n}_{i=1}\sign(M_{i}),\\    
\chi(M_{1}\#_{F}\cdots \#_{F}M_{n})&=\sum^{n}_{i=1}\chi(M_{i})+(n-1)(4g-4).
\end{split}
\]
Furthermore, by suitably choosing $\varphi_{i}$, one can always make $M$ not spin. And suppose $M_{i}\setminus \nu(F_{i})$ is spin for all $i$. Then one can also choose  $\varphi_{i}$ to make $M_{1}\#_{F}\cdots \#_{F}M_{n}$ spin. Furthermore, suppose for each $i$, $M_{i}$ is symplectic and $F_{i}$ is a symplectic surface. By a theorem of Gompf \cite{Gompf95}, for any choice of gluing function, the manifold $M_{1}\#_{F}\cdots \#_{F}M_{n}$ always carries a symplectic structure.


\begin{pro}\label{pro: Seiberg--Witten product}
Let $M=M_{1}\#_{F}M_{2}$ be the fiber sum of two smooth 4-manifolds along a surface $F$ of genus $g\geq 2$. We assume that $b^{+}(M_{i})>1, b_{1}(M_{i})=0$ and $M_{i}$ is of Seiberg--Witten simple-type. Let $\mathfrak{s}_{i}$ be a $\spinc$ structure on $M_{i}\setminus \nu(F_i)$. Let $\mathfrak{s}^{k}$ be the unique $\spinc$ structure on $F\times S^1$ that is pulled back from $F$ and satisfies $\langle c_{1}(\mathfrak{s}^{k}), [F]\rangle =2k$. We consider the Seiberg--Witten invariants
$$
\SW(M,\mathfrak{s}_{1},\mathfrak{s}_{2}):= \sum_{\substack{\{\mathfrak{s}\text{ s.t. } \mathfrak{s}|_{M_{i}\setminus \nu(F_{i})}=\mathfrak{s}_{i},\\ d(\mathfrak{s})=0\}}} \SW(M,\mathfrak{s})
$$
and 
$$
\SW(M_{i},\mathfrak{s}_{i}):= \sum_{\substack{\{\mathfrak{s}\text{ s.t. } \mathfrak{s}|_{M_{i}\setminus \nu(F_{i})}=\mathfrak{s}_{i},\\ d(\mathfrak{s})=0\}}} \SW(M_{i},\mathfrak{s}).
$$ Then we have 
$$
\SW(M,\mathfrak{s}_{1},\mathfrak{s}_{2})= \begin{cases}\pm \SW(M_1,\mathfrak{s}_{1})\cdot \SW(M_2,\mathfrak{s}_{2})&\text{ if }\mathfrak{s}_{1}|_{F\times S^{1}}=\mathfrak{s}_{2}|_{F\times S^{1}}=\mathfrak{s}^{\pm (g-1)},\\
0 &\text{ otherwise.}
\end{cases}
$$
\end{pro}
\begin{proof}
Set $Y=F\times S^1$. We focus on the case $\mathfrak{s}_{1}|_{Y}=\mathfrak{s}_{2}|_{Y}=\mathfrak{s}^{0}$ because the other cases are proved by Mu\~{n}oz--Wang \cite[Theorem 1.2]{MunozWang} \footnote{In the first version of \cite{MunozWang}, the authors also included a proof for the remaining case, by assuming an inequality on the rank of the monopole Floer homology of $F\times S^1$ for the torsion $\spinc$ structure. By the computation of Jabuka--Mark \cite[Theorem 9.4]{Jabuka08} on Heegaard-Floer homology, this inequality is now known to hold.} and Morgan--Szab\'o \cite[Corollary 9.10]{Morgan96}). The argument here is adapted from \cite{MunozWang} and \cite{Jabuka08} (which proved an analogous result for Ozsv\'ath--Szab\'o's mix invariants). 
We start by considering the graded abelian group
\[
X(g,g-1):=\oplus_{i=0}^{g-1}\Lambda^{i}H^{1}(F;\mathbb{Z})\otimes\frac{U^{i-g+1}\cdot \mathbb{Z}[U]}{U\cdot \mathbb{Z}[U]}.
\]
Here the degree of $U$ equals $-2$. We use $X(g,g-1)_{k}$ to denote the degree-$k$ component of $X(g,g-1)$.
Let $\eta$ be the 1-cycle in $Y$ given by the $*\times S^{1}$. Then one can use $\eta$ to define a local system $\Gamma_{\eta}$ over the configuration space for $Y$. The fiber of $\Gamma_{\eta}$ equals the field of formal Laurent series with rational coefficients
$$
\mathcal{L}:=\mathbb{Q}[[t^{-1},t].
$$
Under this local system, one has a canonical isomorphism between the two versions of monopole Floer homology groups
$$
\widecheck{HM}_{*}(Y,\mathfrak{s}^0;\Gamma_{\eta})\cong \widehat{HM}_{*}(Y,\mathfrak{s}^0;\Gamma_{\eta})
$$
and we just denote it by $\mathcal{H}$.
The Heegaard Floer homology of $Y$ is computed by Jabuka--Mark \cite[Theorem 9.4]{Jabuka08}.\footnote{In \cite{Jabuka08}, the authors did the completion in the other direction and used the field $\mathbb{Q}[t^{-1},t]]$ instead. We follow the convention of \cite{KM07} and do negative completion.} We can translate their result to monopole homology using the work of Kutluhan--Lee--Taubes \cite{Kutluhan20}. (See \cite[Theorem 3.1]{Lee19} for a version for twisted coefficients) or alternatively, the work of Colin--Honda--Gighini \cite{Colin2011} and Taubes \cite{Taubes10}. This gives us an isomorphism of vector spaces  over $\mathcal{L}$ with relative $\mathbb{Z}$-grading
\begin{equation}\label{eq: HM for surface times a circle}
\mathcal{H}\cong X(g,g-1)\otimes \mathcal{L}.
\end{equation}
We use $\mathcal{H}_{k}$ to denote the component corresponding to  $X(g,g-1)_{k}\otimes \mathcal{L}$. Then $\mathcal{H}_{0}\cong \mathcal{H}_{2g-2}\cong \mathcal{L}$.
The orientation-reversing diffeomorphism on $Y$, given by the reflection in the $S^{1}$-component, induces a nondegenerate  pairing 
\[
\langle\cdot,\cdot\rangle: \mathcal{H}\otimes_{\mathcal{L}}\mathcal{H}\rightarrow \mathcal{L}.
\]
Elements in $\mathcal{H}_{k}$ and $\mathcal{H}_{l}$ can have nonzero pairing only if $k+l=2g-2$. 
The vector space $\mathcal{H}$ is also a graded module over the ring 
$$
\mathbb{A}(F)=\Lambda^{*}H_{1}(F;\mathbb{Z})\otimes \mathbb{Z}[U].
$$
The action of $U$ is just the multiplication. For any $\gamma\in H_{1}(F;\mathbb{Z})$ and any $\alpha\in \Lambda^{i}H^{1}(F;\mathbb{Z})$, the action is given by 
\[
\gamma\cdot (\alpha\otimes U^{l})=\iota_{\gamma}\alpha\otimes U^{l}+ (\operatorname{PD}(\gamma)\wedge \alpha)\otimes U^{l+1}+\text{terms with negative $t$-degree.}
\]
Here $\iota$ denotes the contraction. In particular, we see that for any element in $\alpha\in \mathcal{H}\setminus \mathcal{H}_{0}$, there exists an homogeneous $\tau\in \mathbb{A}(F)$ with positive degree such that $\tau\cdot \alpha$ is a nonzero element in $\mathcal{H}_0$.
To prove the claim, we consider the manifold $N=F\times D^{2}$ with boundary $Y$ and the relative 2-cycle $\nu_{N}=*\times D^2$ bounded $\eta$. Let $\mathfrak{s}_{N}$ be the unique $\spinc$ that extends $\mathfrak{s}^0$. Then the relative Seiberg--Witten invariant of $N$ gives a map that preserves the structure of $\mathbb{Z}$-graded $\mathbb{A}(F)$-modules
\[
\phi^{SW}_{N,\nu_{N}}(\mathfrak{s}_{N},-):\mathbb{A}(F)\rightarrow \mathcal{H}.
\]
We claim that
\begin{equation}\label{eq: relative invariant for D2 cross F}
\phi^{SW}_{N,\nu_{N}}(\mathfrak{s}_{N},1)\in \mathcal{H}_{2g-2}\setminus \{0\}.    
\end{equation}
To see this, we consider the manifold $\widetilde{N}=N\cup_{Y}N=F\times S^2$. We let $\mathfrak{s}_{\widetilde{N}}$ be unique torsion $\spinc$ structure on $\widetilde{N}$ and let $\nu_{\widetilde{N}}$ be the 2-cycle $*\times S^2$. We use $\operatorname{PD}[F]\in H^{2}(\widetilde{N};\mathbb{Z})$ to fix a component of the positive cone 
$$
\{\alpha\in H^{2}(\widetilde{N};\mathbb{R})\mid \alpha\cdot \alpha\geq 0, \alpha\neq 0\}.
$$
This allows us to define the Seiberg--Witten invariant for the negative chamber 
\[
\SW_{-}(\widetilde{N},\mathfrak{s},-):\mathbb{A}(F)\rightarrow \mathbb{Z}
\]
for any $\spinc$ structure $\mathfrak{s}$ on $\widetilde{N}$. By applying the gluing formula for Seiberg--Witten invariants \cite[Propposition 27.5.1]{KM07}, one obtains the equality
\[
\sum_{k\in \mathbb{Z}}\SW_{-}(\widetilde{N},\mathfrak{s}_{\widetilde{N}}+n\operatorname{PD}[F],U^{g-1})t^{2n}=\langle \phi^{SW}_{N,\nu_{N}}(\mathfrak{s}_{N},1),U^{g-1}\cdot \phi^{SW}_{N,\nu_{N}}(\mathfrak{s}_{N},1)\rangle.
\]
Using the positive scalar curvature metric on $\widetilde{N}$, we obtain that 
\[
\SW_{-}(\widetilde{N},\mathfrak{s}_{\widetilde{N}}+n\operatorname{PD}[F],U^{g-1})=\SW_{+}(\widetilde{N},\mathfrak{s}_{\widetilde{N}}-n\operatorname{PD}[F],U^{g-1})=0
\]
for any $n\geq 0$. Using the wall-crossing formula (see \cite[Theorem 1.2]{Li95} and \cite[Theorem 16]{Okonek96}), we further obtain that
\[
\SW_{-}(\widetilde{N},\mathfrak{s}_{\widetilde{N}}+n\operatorname{PD}[F],U^{g-1})=\begin{cases}
0 \quad&\text{if }n\geq 0,\\
(-n)^{g} \quad&\text{if }n\leq -1.
\end{cases}
\]
In particular, this implies $U^{g-1}\cdot \phi^{SW}_{N,\nu_{N}}(\mathfrak{s}_{N},1)\neq 0$. Hence the homogeneous element $\phi^{SW}_{N,\nu_{N}}(\mathfrak{s}_{N},1)$ must be a nonzero element in $\mathcal{H}_{2g-2}$. This proves the claim (\ref{eq: relative invariant for D2 cross F}).
For $i=1,2$, we let $\nu_{i}$ be a relative 2-cycle in $M_{i}\setminus \nu(F)$ bounded by $\eta$ and consider the relative Seiberg--Witten invariant 
\[
\phi^{SW}_{M_{i}\setminus \nu(F)}(\mathfrak{s}_{i},-): \mathbb{A}(F)\rightarrow \mathcal{H}.
\]
Then the gluing theorem applied to $M_{i}$ gives the relation
\[
\sum_{\{\mathfrak{s}\mid \mathfrak{s}|_{M_{i}\setminus\nu(F)}\}}\SW(M_{i},\mathfrak{s},\alpha)t^{\langle c_{1}(\mathfrak{s}),\nu_{i}\cup \nu_{N}\rangle}=\langle \alpha\cdot \phi^{SW}_{M_{i}\setminus \nu(F)}(\mathfrak{s}_{i},1), \phi^{SW}_{N,\nu_{N}}(\mathfrak{s}_{N},1))\rangle
\]
for any $\alpha\in \mathbb{A}(F)$. We actually have $\phi^{SW}_{M_{i}\setminus \nu(F)}(\mathfrak{s}_{i},1)\in \mathcal{H}_0$. Otherwise we can find some $\alpha$ with positive degree such that \[\alpha\cdot \phi^{SW}_{M_{i}\setminus \nu(F)}(\mathfrak{s}_{i},1)\in \mathcal{H}_0.\]
Then by (\ref{eq: relative invariant for D2 cross F}), we have 
\[
\langle \alpha\cdot \phi^{SW}_{M_{i}\setminus \nu(F)}(\mathfrak{s}_{i},1), \phi^{SW}_{N,\nu_{N}}(\mathfrak{s}_{N},1))\rangle\neq 0.
\]
That implies $\SW(M_{i},\mathfrak{s},\alpha)\neq 0$ for some $\mathfrak{s}$, which contradicts our assumption that $M_{i}$ is of Seiberg--Witten simple type. 
Since any two elements in $\mathcal{H}_0$ pairs trivially, we get
\[
\langle \phi^{SW}_{M_{1}\setminus \nu(F)}(\mathfrak{s}_{1},1),\phi^{SW}_{M_{2}\setminus \nu(F)}(\mathfrak{s}_{2},1)\rangle=0.
\]
As the last step, we apply the gluing formula to the manifold $M$ and obtain that 
\[
\begin{split}
\sum_{\substack{\{\mathfrak{s}\text{ s.t. } \mathfrak{s}|_{M_{i}\setminus \nu(F_{i})}=\mathfrak{s}_{i},\\ d(\mathfrak{s})=0\}}}\SW(M,\mathfrak{s})\cdot t^{\langle c_{1}(\mathfrak{s}),\nu_1\cup \nu_2\rangle}&=\langle \phi^{SW}_{M_{1}\setminus \nu(F)}(\mathfrak{s}_{1},1),\phi^{SW}_{M_{2}\setminus \nu(F)}(\mathfrak{s}_{2},1)\rangle\\
&=0\in \mathcal{L}.
\end{split}
\]
Comparing the coefficients of both sides, we finish the proof.
\end{proof}
%And the corresponding result for Ozsv\'ath-Szab\'o mixed invariant is proved in \cite[Corollary 1.2]{Jabuka08}. Using \cite[Theorem 9.4]{Jabuka08} regarding the Heegaard-Floer homology of a surface times a circle for torsion $\spinc$ structure, one can adapt the proof to our situation. Details to be filled.}




%\begin{defi}
%We say a smooth 4-manifold $M$ is of SW general type if $b^{+}(M)>1$ and $M$ has a unique Seiberg--Witten basic class up to conjugacy.
%\end{defi}

\begin{pro}\label{pro: fiber sum has nontrivial SW-tot} For $2\leq n$ and $1\leq i\leq n$, let $M_{i}$ be a symplectic manifold with $b^{+}(M_{i})>1$. Let $F_{i}\subset M_{i}$ be an embedded symplectic surface of genus $g\geq 2$ and self-intersection $0$. Take the fiber sum $M=M_{1}\#_{F}\cdots  \#_{F}M_{n}$. Suppose $M_{i}$ has a unique Seiberg--Witten basic class up to sign and suppose $H_{1}(M_{i}\setminus \nu(F_{i});\mathbb{R})=0$. 
 Then we have  
 \begin{equation}\label{eq: SW-halftot of fiber sum}
 \SWbbhalftot^{0}(M)=1.    
 \end{equation}
\end{pro}
\begin{proof}
 By the work of Taubes, the canonical $\spinc$ structure $\mathfrak{s}_{J,i}$ on $M_{i}$ and its conjugate both have Seiberg--Witten invariant $\pm 1$.
 By our assumptions, all the other $\spinc$ structures on $M_{i}$ have trivial Seiberg--Witten invariants.
 Note that  $\mathfrak{s}_{J,i}$ is not torsion since the  adjunction formula $\langle c_{1}(\mathfrak{s}_{J,i}),[F_{i}]\rangle =2g-2$
holds. 
 
 
 Let $\mathfrak{s}_{i}$ be a $\spinc$ structure on $M_{i}\setminus \nu(F_{i})$. Consider the  quantity
$$
\SW(M,\mathfrak{s}_{1},\cdots,\mathfrak{s}_{n}):= \sum_{\substack{\{\mathfrak{s}\text{ s.t. } \mathfrak{s}|_{M_{i}\setminus \nu(F_{i})}=\mathfrak{s}_{i},\\ d(\mathfrak{s})=0\}}} \SW(M,\mathfrak{s})
$$
Note that the fiber sum of $M_{i}$ carries a symplectic structure so is of Seiberg--Witten simple-type. Via a repeated application of Proposition \ref{pro: Seiberg--Witten product}, we see that 
\begin{equation}\label{eq: SW fiber sum}
\SW(M,\mathfrak{s}_{1},\cdots,\mathfrak{s}_{n})=\begin{cases}
\pm 1 &\text{ if }\mathfrak{s}_{i}=\mathfrak{s}_{J,i}|_{M_i\setminus\nu(F_{i})} \text{ for all }i\\
\pm 1 &\text{ if }\mathfrak{s}_{i}|_{M_{i}}=\overline{\mathfrak{s}_{J,i}}|_{M_i\setminus\nu(F_{i})} \text{ for all }i\\
0 &\text{ otherwise}
\end{cases}
\end{equation}
To deduce (\ref{eq: SW-halftot of fiber sum}) from (\ref{eq: SW fiber sum}), it suffices to show that any $\spinc$ structure $\mathfrak{s}$ that satisfies 
$$
\mathfrak{s}|_{M\setminus \nu(i)}=\overline{ \mathfrak{s}|_{M\setminus \nu(i)}} \text{ for any }i
$$
does not contribute to $\SWbbhalftot^{0}(M)$. Actually, for such $\mathfrak{s}$, we have 
$$
c_{1}(\mathfrak{s})\in \ker (H^{2}(M;\mathbb{R})\rightarrow \oplus^{n}_{i=1} H^{2}(M_{i}\setminus \nu(F_{i});\mathbb{R}))
$$
This implies that $\operatorname{PD}(c_{1}(\mathfrak{s}))$ can be represented by a cycle in $(S^{2} \setminus \bigsqcup_{i} D^{2}_{i})\times F$. Therefore, $c^{2}_{1}(\mathfrak{s})=0$ and we have 
$$
d(\mathfrak{s})=-\frac{2\chi(M)+3\sign(M)}{4}=-\sum^{n}_{i=1}\frac{2\chi(M_{i})+3\sign(M_{i})}{4}-(n-1)(2g-2).
$$
By Taubes's result \cite{Taubes00}, we have $2\chi(M_{i})+3\sign(M_{i})\geq 0$. So $d(\mathfrak{s})< 0$ and it does not contribute to $\SWbbhalftot^{0}(M)$.
\end{proof}

As the next step, we will construct three symplectic manifolds $M_{1}, M_{2}, M_{3}$, which serve as the building blocks in our construction of the manifold $X^{l}_{i}$.

The manifold $M_{1}$ is constructed by Persson-Peters-Xiao \cite{Persson}. We sketch its construction as follows: Let $Y_{0}=\mathbb{CP}^{1}\times \mathbb{CP}^{1}$ and let $\pi: Y_0\rightarrow \mathbb{CP}^{1}$ be the projection to the second factor.  Consider a triple sequence of double covers 
$$
Y_3\xrightarrow{p_{3}}Y_2 \xrightarrow{p_2}Y_1\xrightarrow{p_1} Y_0
$$
with branched loci being $B'_3\subset Y_2, p^{-1}_{1}(B_2)\subset Y_1$ and $B_1\subset Y_0$ respectively. Here $B_1, B_2$ are singular curves in $Y_0$ and $B'_3$ is a singular curve in $Y_2$ that is linearly equivalent to $p^{-1}_{2}p^{-1}_{1}(B_3)$ for some $B_3\in Y_0$. The algebraic surface $Y_{3}$ has certain singularities coming from the singular points of the branched loci. And $M_{1}$ is defined to be the resolution of $Y_3$. By suitable choosing $B_1, B_2, B_3$ and $B'_3$, the manifold $M_1$ can be made into a  simply-connected algebraic surface of general type with $\sign(M_1)>0$. 
Suppose the bi-degree of $B_{i}$ equals $(a_{i},b_{i})$. Then one can ensure that $M_{1}$ is spin by choosing $B_{i}$ so that both $a_{1}+a_{2}+a_{3}$ and $b_{1}+b_{2}+b_{3}$ are divisible by $4$. Furthermore, the map $p: M_1\rightarrow \mathbb{CP}^{1}$ defined by the composition
$$
M_1\rightarrow Y_3\xrightarrow{p_{3}}Y_2 \xrightarrow{p_2}Y_1\xrightarrow{p_1} Y_0\xrightarrow{\pi} \mathbb{CP}^{1}
$$
makes $M_1$ into a Lefschetz fibration. We let $F_{1}$ be a generic fiber. Then $F_{1}$ is a $\mathbb{Z}/2\oplus \mathbb{Z}/2\oplus \mathbb{Z}/2$ branched cover of $\mathbb{CP}^{2}$. %There are $b_{1}+b_{2}+b_{3}$ branched points and each of them has four preimages. 
Using the Riemann--Hurewicz formula, we see that 
\[
1<g(F_1)=2(b_{1}+b_{2}+b_{3})-7\equiv 1\mod 8.
\]
In \cite{Persson}, the authors proved that $M_1$ is simply-connected. The same argument implies the simply-connectedness of $M_{1}\setminus F_{1}$. Actually, using \cite[Corollary F]{Persson}, the authors found a disk $D^2\subset \mathbb{CP}^{1}$ such that $p^{-1}(D^2)$ is simply-connected. Since the fibration $p: M_1\rightarrow D^2$ has no multiple fibers, by the same argument as \cite[Lemma 3.20]{Persson81}, this implies 
$
\pi_{1}(M_1\setminus F_{1})=1.
$


The manifold $M_2$ is constructed in \cite[Proof of Corollary D]{Persson} in a similar flavor. For any positive integers $a,b$, we let $Y_{a,b}$ be the double cover of $Y_0$ branched over  
$$(\mathbb{CP}^{1}\times \{2a \text{ generic points}\})\cup (\{2b \text{ generic points}\}\times \mathbb{CP}^{1}).$$
We let $M'_2$ be the fiber product 
$$
\xymatrix{
M'_2 \ar[d] \ar[r] & Y_{a_{4},b_{4}}\ar[d]\\
Y_{a_{5},b_{5}} \ar[r] & Y_0.}
$$
Then $M'_{2}$ has discrete singular points coming from double points of the branched loci. We let $M_2$ be the minimal resolution of these singularities. Topologically, that means we remove tubular neighborhoods of these singular points (which are cones over $\mathbb{RP}^{3}$) and glue copies of the disk bundle for the $TS^{2}$.  It is proved in \cite{Persson} that $M_2$ is a simply connected surface of general type, with signature 
$$
\sign(M_2)=-8(a_4b_4+a_5b_5)<0.
$$
Furthermore, $M_2$ is spin if $a_4+a_5,b_4+b_5$ are even. 

As before, we consider the composition 
$$
M_2\rightarrow Y_0\xrightarrow{\pi} \mathbb{CP}^{1}
$$
and let $F_2$ be a generic fiber. Then $F_2$ is a $\mathbb{Z}/2\oplus \mathbb{Z}/2$ cover of $\mathbb{CP}^{1}$ branched at  $2(b_4+b_5)$ points, each with two preimages. From this, we compute that $g(F_2)=2(b_4+b_5)-3$. We set 
$$
b_{4}=\frac{g(F_1)+7}{4},\quad  b_{5}=\frac{g(F_1)-1}{4}.
$$
Then $g(F_2)=g(F_1)$. Note that 
\[
\frac{\sign(M_{2})}{16}=-a_{4}-\frac{b_{5}(a_{4}+a_{5})}{2}.
\]
By picking suitable $a_4,a_5$, we may assume that  $\frac{\sign(M_1)}{16}, \frac{\sign(M_2)}{16}$ are coprime to each other. Since $M_2$ is a minimal algebraic surface of general type, it has a unique Seiberg--Witten basic class up to sign.
One can use a similar argument as $M_1$ to show that $\pi_{1}(M_{2}\setminus F_{2})=1$.


The manifold $M_3$ is constructed by Fintushel-Park-Stern \cite{Fintushel2002}. Unlike $M_1, M_2$, the manifold $M_{3}$ is not complex. In \cite{Fintushel2002}, the authors constructed many symplectic manifolds with one Seiberg--Witten basic class up to sign. In particular, for any positive integer $p$, they constructed a simply-connected symplectic $4$-manifold $X_{p}$ by taking the fiber sum of two rational surfaces $R(2p-3)$ and $S(p)$ along a symplectic surface of genus $2p-5$. And one has 
$$\sign(X_{p})=25-14p.$$ We let $M_3=X_{p}$ with $p=\frac{g(F_{2})+5}{2}$ and let $F_3$ be the surface on which the fiber sum is conducted. It is proved in \cite{Fintushel2002} that $\pi_{1}(M_3\setminus F_3)=1$. Furthermore, one can find an embedded surface in $S_{p}\setminus F_{3}\subset X_{p}\setminus F_{3}$ that has an odd self-intersection number. 

So far, we have constructed the building blocks $(M_{i},F_{i})$ and established some of their important properties. We summarize these properties in the following lemma.


\begin{lem}\label{lem: building block}
There exist symplectic manifolds $M_1, M_{2}, M_{3}$ and embedded symplectic surfaces $F_{i}\subset M_{i}$ of self-intersection $0$ such that the following conditions are satisfied:
\begin{enumerate}
\item $g(F_{1})=g(F_2)=g(F_3)\geq 2$.
\item $M_{i}$ has a unique Seiberg--Witten basic class up to conjugacy.
    \item $M_{1}, M_{2}$ are spin. $M_3$ is not spin.
    \item $\pi_{1}(M_i\setminus F_i)=1$ for $i=1,2,3$.
    \item $\sign(M_1)>0$ while $\sign(M_{2}), \sign(M_3)<0$.
    \item $\frac{\sign(M_{1})}{16}$ and $\frac{\sign(M_{2})}{16}$ are coprime to each other.
    \item $\sign(M_3)$ is odd.
    \item $M_{3}\setminus F_{3}$ contains an embedded surface with odd self-intersection number.
\end{enumerate}
\end{lem}


The last ingredient we need is the following proposition

\begin{pro}[Mandelbaum \cite{Mandelbaum79}
]\label{pro: fiber sum stabilization}(see also \cite[Corollary 5]{Gompf91}) Let $M, M'$ be smooth, closed 4-manifolds. Let $F\subset M$, $F'\subset M'$ be closed, connected, orientable surfaces with self-intersection $0$ and of the same genus $g$. 
Suppose $M$ is spin and suppose $\pi_{1}(M')=\pi_{1}(M
\setminus F)=1$. Then we have 
$$
(M\#_{F} M')\#(S^{2}\times S^{2})
\cong M\# M'\# 2g(S^2\times S^2).
$$
if $M\#_{F} M'$ is spin and 
$$
(M\#_{F} M')\#(S^{2}\times S^{2})
\cong M\# M'\# 2g(\mathbb{CP}^{2}\# \overline{\mathbb{CP}}^{2}).
$$
if $M\#_{F} M'$ is not spin.
\end{pro}
\begin{proof}[Proof of Theorem \ref{thm: 4-mfds that dissolves}] For $a,b,c\geq 0$, we use $M(a,b,c)$ to denote the fiber sum of $a$ copies of $M_1$, $b$ copies of $M_2$ and $c$ copies of $M_{2}$ along the embedded surfaces $F_{1}, F_{2}, F_{3}$. Then $M(a,b,c)$ is always simply-connected. By Gompf's result \cite{Gompf95}, $M(a,b,c)$ is symplectic for any $a,b,c$ and can be chosen to be spin when $c=0$. When $c\neq 0$, $M(a,b,c)$ is not spin by by Lemma \ref{lem: building block} (8).

By Wall's result \cite{Wall64}, there exists $n\gg 0$ such that $M_{1}, M_{2}, M_{3}$ all dissolves after $n$ stabilizations.

Now we construct nonspin $X_{l}^{i}$ for any $l\in \mathbb{Z}$. By Lemma \ref{lem: building block} (5), (6), (7), there exists integers $a,b$ and a positive integer $c$ such that $$
a\cdot\sign(M_{1})+b\cdot\sign(M_2)+c\cdot\sign(M_{3})=l.
$$
Then we set  
$$
X^{l}_{i}= M(a-(n+i)\sign(M_2), b+(n+i)\sign(M_{1}), c).
$$
By Proposition \ref{pro: fiber sum has nontrivial SW-tot}, $\SWbbhalftot(X^{l}_{i})=1$. By Proposition \ref{pro: fiber sum stabilization}, $X^{l}_{i}$ dissolves after one stabilization. (Again note that for nonspin, simply-connected manifolds, connected summing with $S^2\times S^2$ is the same as connected summing with $\mathbb{CP}^{2}\# \overline{\mathbb{CP}}^{2}$.)

Next, we construct spin $X^{l}_{i}$ assuming  
 $16|l$. In this case, we can find $a,b$ such that 
 $$
a\cdot\sign(M_{1})+b\cdot\sign(M_2)=l.
$$
We set 
$$
X^{l}_{i}= M(a-(n+i)\sign(M_2), b+(n+i)\sign(M_{1}), 0).
$$
Again by Proposition \ref{pro: fiber sum stabilization}, $X^{l}_{i}$ dissolves after one stabilization.
\end{proof}













%\begin{thebibliography}{}
%\bibitem{Galatius2018} Galatius, Soren, and Oscar Randal-Williams. "Homological stability for moduli spaces of high dimensional manifolds. I." Journal of the American Mathematical Society 31.1 (2018): 215-264.
%\bibitem{Krannich2019} Krannich, Manuel. "Homological stability of topological moduli spaces." Geometry \& Topology 23.5 (2019): 2397-2474.
%\bibitem{Kreck1999}Kreck, Matthias. "Surgery and duality." Annals of Mathematics 149.3 (1999): 707-754.
%\end{thebibliography}


\bibliographystyle{plain}
\bibliography{mainref}


\end{document}
