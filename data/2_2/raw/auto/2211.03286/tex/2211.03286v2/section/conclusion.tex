\section{Conclusions and Future Work}\label{sec:conclusion}

\subsection{Conclusions}

Many optimization model-based multi-agent task allocation frameworks implement an agent capability and task requirement model, where the capability and requirement parameters are assumed given.
This paper addresses the estimation problem of the capability and requirement parameters in such models and, therefore, facilitates the application of these model-based frameworks to practical problems.

The parameter estimation problem is formulated as a linear program that tries to find the agent capabilities and task requirements that optimally fit the team configuration and task performance pairs.
For situations where the team configuration space is huge and an exhaustive performance evaluation is impractical, a randomly chosen subset provides sufficient information for the estimation.
A comprehensive computational evaluation shows that the learning model can fit the data with low errors (\(\leq\) 2\% for most cases) and short training times (a few seconds with randomly chosen data).

A Gazebo-based simulation platform is developed, and the learning framework is validated in a practical multi-agent problem.
Using the task completion time obtained from the simulation as the performance metric, the agent and task setups in \tableref{}s \ref{tab:real_agent_capability}-\ref{tab:real_task_requirement} can be successfully reflected by the learned values.
Finally, the learned capabilities and task requirements are embedded as constraints in a task allocation optimization, and the system is validated in a practical multi-agent exploration and manipulation example (Sec. \ref{sec:result_task_allocation_simulation}).

\subsection{Limitations and Future Work}

First, the capability model in this paper assumes all agent capabilities are cumulative which results in a linear representation. In practice, there can be non-cumulative capabilities. For example, the speed that a team can drive is not the sum of the agent speeds. Additional modeling can be performed to learn and apply non-cumulative capabilities for task allocation.
Second, the task performance of a team configuration can vary due to the disturbances and uncertainties in the system.
Future work will consider developing learning models that encode and estimate such uncertainties in the agents' task capabilities.
In addition, experimental verification will be conducted in the real world for the framework developed in this paper.
