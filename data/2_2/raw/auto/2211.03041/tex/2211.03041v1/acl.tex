% This must be in the first 5 lines to tell arXiv to use pdfLaTeX, which is strongly recommended.
\pdfoutput=1
% In particular, the hyperref package requires pdfLaTeX in order to break URLs across lines.

\documentclass[11pt]{article}

% Remove the "review" option to generate the final version.
\usepackage[]{EMNLP2022}

% Standard package includes
\usepackage{times}
\usepackage{mathtools}
\usepackage{latexsym}
\usepackage{times}
\usepackage{url}
\usepackage{latexsym}
\usepackage[utf8]{inputenc}
\usepackage{amsmath}
\usepackage{amssymb}
\usepackage{times}
\usepackage{helvet}
\usepackage{courier}
\usepackage{url}
\usepackage{graphicx}
\usepackage{amsfonts}
\usepackage{xspace}
%TODO
\usepackage{verbatim}
\usepackage{CJKutf8}
\usepackage{multirow}
\usepackage{tabularx}


\usepackage{array}
\usepackage{adjustbox}
\usepackage{booktabs}
\usepackage{amsthm}
\usepackage{subcaption}
\usepackage{algorithm}
\usepackage{algpseudocode}
% \usepackage{algorithmicx}
% \usepackage{algorithm}
\newcommand\BibTeX{B\textsc{ib}\TeX}
\usepackage{color}
\usepackage{colortbl}
\usepackage{xcolor}
\usepackage{relsize}
\usepackage{cleveref}
\usepackage{bbm}
\usepackage{xcolor}
\usepackage{tabularx}
\usepackage{soul}
\usepackage{algorithmicx}
\usepackage{algorithm}
\usepackage{xspace}
\newcolumntype{b}{X}
\newcolumntype{s}{>{\hsize=.5\hsize}X}
\newcolumntype{P}[1]{>{\centering\arraybackslash}p{#1}}
\newcolumntype{M}[1]{>{\centering\arraybackslash}m{#1}}
\definecolor{RED}{RGB}{255,0,0}
\definecolor{GREEN}{RGB}{0,172,78}
\newcommand{\thetav}{{\boldsymbol \theta}}
\newcommand{\hlc}[2][yellow]{{%
		\colorlet{foo}{#1}%
		\sethlcolor{foo}\hl{#2}}%
}
% \definecolor{Red}{red}{0.8}
% For proper rendering and hyphenation of words containing Latin characters (including in bib files)
\usepackage[T1]{fontenc}
% For Vietnamese characters
% \usepackage[T5]{fontenc}
% See https://www.latex-project.org/help/documentation/encguide.pdf for other character sets

% This assumes your files are encoded as UTF8
\usepackage[utf8]{inputenc}
\newcommand{\approach}{\textsc{REGEX}}
% This is not strictly necessary, and may be commented out,
% but it will improve the layout of the manuscript,
% and will typically save some space.
\usepackage{microtype}
\newcommand\integratedgrads{\textit{IG}}
\newcommand{\synteq}{::=}
\theoremstyle{definition}
\newtheorem{definition}{Definition}[section]
% If the title and author information does not fit in the area allocated, uncomment the following
%
%\setlength\titlebox{<dim>}
%
% and set <dim> to something 5cm or larger.

% \title{Calibration Meets Explanation:\\ How Can We Get Better Model Confidence?}
\title{Calibration Meets Explanation: \\
A Simple and Effective Approach for Model Confidence Estimates}
\author{Dongfang Li$^1$, Baotian Hu$^1$\footnotemark[1]\thanks{\hspace{2mm}Corresponding authors}\hspace{2mm},  Qingcai Chen$^{1,2}$\footnotemark[1]\hspace{1mm}\\
$^1$Harbin Institute of Technology (Shenzhen), Shenzhen, China \\
$^2$Peng Cheng Laboratory, Shenzhen, China\\
\texttt{crazyofapple@gmail.com, \{hubaotian, qingcai.chen\}@hit.edu.cn}}
\begin{document}
\maketitle
\begin{abstract}

Calibration strengthens the trustworthiness of black-box models by producing better accurate confidence estimates on given examples. However, little is known about if model explanations can help confidence calibration. Intuitively, humans look at important features attributions and decide whether the model is trustworthy. Similarly, the explanations can tell us when the model may or may not know. Inspired by this, we propose a method named \textbf{CME} that leverages model explanations to make the model less confident with non-inductive attributions. The idea is that when the model is not highly confident, it is difficult to identify strong indications of any class, and the tokens accordingly do not have high attribution scores for any class and vice versa. We conduct extensive experiments on six datasets with two popular pre-trained language models in the in-domain and out-of-domain settings. The results show that CME improves calibration performance in all settings. The expected calibration errors are further reduced when combined with temperature scaling. Our findings highlight that model explanations can help calibrate posterior estimates.


\end{abstract}

\section{Introduction}


Accurate estimates of posterior probabilities are crucial for neural networks in various Natural Language Processing (NLP) tasks~\cite{icml17,DBLP:conf/nips/Lakshminarayanan17}. For example, it would be helpful for humans if the models deployed in practice abstain or interact when they cannot make a decision with high confidence~\cite{DBLP:journals/jamia/JiangOKO12}. While Pre-trained Language Models (PLMs) have improved the performance of many NLP tasks~\cite{bert,roberta}, how to better avoid miscalibration is still an open research problem ~\cite{calibration_emnlp20,dan_roth_emnlp21}. 
\begin{table}[t!]
    \centering
    \begin{tabular}{l|p{0.65\columnwidth}}
    \hline

    %  Example 1: & It is \hlc[cyan!10]{a} \hlc[red!40]{warm} \hlc[red!60]{funny} \hlc[red!40]{engaging} \hlc[cyan!20]{film} . \\ \hline
     Positive & a fast \hlc[green!10]{funny} \hlc[green!40]{highly} \hlc[green!80]{enjoyable} movie.\\ \hline
    %  like a south of the border melrose place
     
     Negative & It's about \hlc[red!5]{following} your \hlc[green!10]{dreams} \hlc[red!10]{no} matter \hlc[red!5]{what} your \hlc[green!5]{parents} think.\\
    \hline
  \end{tabular}
    \caption{Two motivating examples with highlight explanations~\cite{SST}. The saturation of the colors signifies the magnitude. The confidence of the model should be easily recognized by looking at token attributions.}
    % \vspace{-4mm}
    \label{tab:example-m}
\end{table}
In this paper, we investigate if and how model explanations can help calibrate the model. 

Explanation methods have attracted considerable research interest in recent years for revealing the internal reasoning processes behind models~\cite{IG,Uncertainty_Aware_Attention,deeplift}. Token attribution scores generated by explanation methods represent the contribution to the prediction~\cite{diagnostic}. Intuitively, one can draw some insight for analyzing and debugging neural models from these scores if they are correctly attributed, as shown in Table~\ref{tab:example-m}. For example, when the model identifies a highly indicative pattern, the tokens involved would have high attribution scores for the predicted label and low attribution scores for other labels. Similarly, if the model has difficulty recognizing the inductive information of any class (i.e., the attribution scores are not high for any label), the model should not be highly confident. As such, the computed explanation of an instance could indicate the confidence of the model in its prediction to some extent.
 
Inspired by this, we propose a simple and effective method named \textbf{CME} that can be applied at training time and improve the performance of the confidence estimates. The estimated confidence measures how confident the model is for a specific example. Ideally, reasonable confidence estimates should have higher confidence for correctly classified examples resulting in higher attributions than incorrect ones. Hence, given an example pair during training with an inverse classification relationship, we regularize the classifier by comparing the wrong example's attribution magnitude and the correct example's attribution magnitude.

Our work is related to recent works on incorporating explanations into learning. Different from previous studies that leverage explanations to help users predict model decisions~\cite{DBLP:journals/corr/abs-2102-02201} or improve the accuracy~\cite{DBLP:conf/icml/RiegerSMY20}, we focus on answering the following question: \textit{are these explanations of black-box models useful for calibration?} If so, how should we exploit the predictive power of these explanations? Considering the model may be uninterpretable due to the nature of neural networks and limitations of explanation method~\cite{Fragile,DBLP:conf/nips/YehHSIR19}, a calibrated model by CME at least can output the unbiased confidence. Moreover, we exploit intrinsic explanation during training, which does not require designing heuristics~\cite{xiye1} and additional data augmentation~\cite{mixup21acl}.
% Are these explanations useful for calibrating the model?

We conduct extensive experiments using BERT~\cite{bert} and RoBERTa~\cite{roberta} to show the efficacy of our approach on three natural language understanding tasks (i.e., natural language inference, paraphrase detection, and commonsense reasoning) under In-Domain (ID) and Out-of-Domain (OD) settings. CME achieves the lowest expected calibration error without accuracy drops compared with strong SOTA methods, e.g.,~\citet{mixup21acl}. When combined with Temperature Scaling (TS)~\cite{icml17}, the expected calibration errors are further reduced as better calibrated posterior estimates under these two settings.




\section{Method}
\subsection{Problem Formulation}
A well-calibrated model is expected to output prediction confidence (e.g., the highest probability after softmax activation) comparable to or aligned with its task accuracy (i.e., empirical likelihood). For example, given 100 examples with the prediction confidence of 0.8 (or 80\%), we expect that 80 examples will be correctly classified. Following~\citet{icml17}, we estimate the calibration error by empirical approximations. Specifically, we partition all examples into $K$ bins of equal size according to their prediction confidences. Formally, for any $p\in[\ell_k,u_k)$, we define the empirical calibration error as:
\begin{equation}
\hat{\mathcal{E}}_k=\frac{1}{|\mathcal{B}_k|}\Big|\sum_{i\in\mathcal{B}_k}\big[\mathbbm{1}(\hat{y}_i=y_i)-\hat{p}_i\big]\Big|,
\end{equation}
where $y_i$, $\hat{y}_i$ and $\hat{p}_i$ are the true label, predicted label and confidence for $i$-th example, and $\mathcal{B}_k$ denotes the bin with prediction confidences bounded between $\ell_k$ and $u_k$.
To evaluate the calibration error of classifiers, we further adopt a weighted average of the calibration errors of all bins as the Expected Calibration Error (ECE)~\citep{DBLP:conf/aaai/NaeiniCH15}:
\begin{align}
    \textrm{ECE} =\sum_{k=1}^K\frac{|\mathcal{B}_k|}{n} \hat{\mathcal{E}}_{k},
    \label{eq:ece}
\end{align}
where $n$ is the example number and lower is better.
Note that the calibration goal is to minimize the calibration error without significantly sacrificing prediction accuracy. 


\subsection{Our Approach}

Generally, text classification models are optimized by Maximum Likelihood Estimation (MLE), which minimizes the cross-entropy loss between the predicted and actual probability over $k$ different classes.
To minimize the calibration error, we add a regularization term to the original cross-entropy loss as a multi-task setup.

Our intuition is that if the error of the model on example $i$ is more significant than its error on example $j$ (i.e., example $i$ is considered more difficult for the classifier), then the magnitude of attributions on example $i$ should not be greater than the magnitude of attributions on example $j$. Moreover, we penalize the magnitude of attributions with the model confidence~\cite{DBLP:conf/acl/XinTYL20}, as the high error examples also should not have high confidence. Compared to the prior post-calibration methods (e.g., temperature scaling learns a single parameter with a validation set to rescale all the logits), our method is more flexible and sufficient to calibrate the model during training.

% The magnitude of attributions is gathered by  $\ell_{2}$ normalization.
Formally, given a training set $\mathcal{D} =$ $\{(\boldsymbol{x}_{1},y_{1})$$,\cdots,$$(\boldsymbol{x}_{n},y_{n})\}$ where $\boldsymbol{x}_{i}$ is the embeddings of input tokens and $y_{i}$ is the one-hot vector corresponding to its true label, an attribution of the golden label for input $\boldsymbol{x}_i$ is a vector $\boldsymbol{
a}_i = (a_{i1},\cdots,a_{il})$, and $a_{ij}$ is defined as the attribution of $x_{ij}$ ($l$ is the length). Here, attention scores are taken as the self-attention weights induced from the start index to all other indices in the penultimate layer of the model; this excludes weights associated with any special tokens added. Then, the token attribution $a_{ij}$ is the normalized attention score~\cite{FRESH} scaled by the corresponding gradients $\nabla \alpha_{ij}= \frac{\partial \hat{y}}{\partial \alpha_{ij}}$~\cite{serrano-smith-2019-attention}. At last, our training minimizes the following loss: 
\vspace{-2mm}
\begin{equation}
\label{loss_function}
    \mathcal{L}_{CME} = \mathcal{L}_{classify} + \lambda \mathcal{L}_{calib},
\end{equation} where $\lambda$ is a weighted hyperparameter. The $L_{calib}$ is calculated as follows:
\vspace{-2mm}

\begin{align}
    \mathcal{L}_{calib} &= \sum_{1\leq i,j\leq n}\Psi_{i,j} \mathbbm{1}[e_i > e_j], \label{eqn:atten1}\\
    % \Psi_{i,j}=max\left\{ 0, t(\boldsymbol{x}_i) - t(\boldsymbol{x}_j)\right\}^{2}, \label{eqn:atten2} \\
     \Psi_{i,j} &= \max[0, t(\boldsymbol{x}_i) - t(\boldsymbol{x}_j)]^{2}, \label{eqn:atten2} \\
    t(\boldsymbol{x}_i) &=  \lVert{ a_{ij}}\rVert_2 * c_i, \label{eqn:atten3}
\end{align}
%  L_{2}\left ( a_{ij} \right )
where $e_i$ and $e_j$ are the error of example $i$ and example $j$, the confidence $c_i$ is estimated by the max probability of output~\cite{DBLP:conf/iclr/HendrycksG17}, with the L2 aggregation. The products could be further scaled by $\sqrt{l}$. 
In practice, strictly computing $L_{calib}$ for all example pairs is computationally prohibitive. Alternatively, we only consider examples from the mini-batch (similar lengths) of the current epoch. In other words, we consider all pairs where $e_i$ = 1 and $e_j$ = 0 where $e$ is calculated by using zero-one error function. The comparisons of example pairs can also be calculated from more history after every epoch or by splitting examples into groups, and we leave it to future work. 

\begin{algorithm}[t!]
 \small
\caption{{\small{Explanation-based Calibrated Training}}}\label{euclid}
 \textbf{Inputs} : Train set $\mathcal{D}$, Number of epochs $T$, Learning rate $\eta$, Optimizer $G$.
\\
\textbf{Output}: Calibrated Text Model $M$
\begin{algorithmic}[1]
%\Require
%\Require{}
%\Require{$\mathcal{Q}$: }
% \State {Let $\mathcal{Q}$ : Empirical Probability Matrix $\in \mathbb{R}^{B \times K}$}
% \State {Random initialization of $\Theta$}
\State Random Initialize $\thetav$.
\For{epoch $= 1 \ldots T$}
    \State{Split $\mathcal{D}$ into random mini-batches \{$b$\}.}
    \For{a batch $b$ from $\mathcal{D}$}{}
        \State{Backward model $M$ for $\nabla_{\thetav} \mathcal{L}_{classify}(\thetav,\mathcal{Y})$.}  
        %\State{Update $\Theta$ by SGD using Loss}
        \State{Calculate the attribution by scaled attention.}
        \State{Computes absolute value of attributions.}
        \State{Normalized it by applying \textrm{Softmax} function.}
        % \If{current step $\in \mathcal{S}$}
        %     \State{$\hat{p}$ = softmax($\Theta,\mathcal{D}$)}
        %     \State{$\mathcal{Q} \leftarrow CalEmpProb(\hat{p},B)$}
        % \EndIf
        \State{Calculate $\mathcal{L}_{CME}$ by Eqn.~\ref{loss_function},~\ref{eqn:atten1},~\ref{eqn:atten2},~\ref{eqn:atten3}.}
        \State{Optimize the model parameters $\thetav$ by G:}
        \State{\hspace*{\algorithmicindent}$\thetav \leftarrow  \thetav - \eta \nabla_{\thetav}\mathcal{L}_{CME}(\thetav,\mathcal{Y})$.}  
    \EndFor
    %\For{$x,y \in \mathcal{D}$}
    %\State{
    %\EndFor
\EndFor
\end{algorithmic}
\label{alg:alg}

\end{algorithm}

Full training details are shown in Algorithm~\ref{alg:alg}. To compute the gradient w.r.t the learnable weight independently, we retain the computation graph in the first back-propagation of classification loss. The model explanations are dynamically produced during training and then used to update the model parameters, which can be easily applied to most off-the-shelf neural networks. \footnote{Code is available here: \url{https://github.com/crazyofapple/CME-EMNLP2022/}}




\label{sec:ex}

\begin{itemize}
    \item ML-SSP: multi-lingual self-supervised pretraining
    \item LA-ML-SSP: language-adapted ML-SSP
    \item Opt. LAL: optimum language-adapted learning with LEAP
\end{itemize}

\begin{table*}[h]
   \begin{minipage}{\textwidth}
	\centering
	\begin{tabular}{|c||c|c|c|c|c|c|c|c|c|c|}
	    \hline
	           & de & it & ru & fr & fr-CA & es & es-MX & pt & pt-BR & AVG. \\ \hline\hline
        Baseline  &  &  & &  & & & & & & \\ \hline
        ML-SSP    &  &  & &  & & & & & & \\ \hline
        LA-ML-SSP &  &  & &  & & & & & & \\ \hline
        Opt. LAL  &  &  & &  & & & & & & \\ \hline
	\end{tabular}
	\caption{WER with the single multi-lingual network on the in-domain language data.}
    \label{tab:ml-indomain}
    \end{minipage}
    %\vspace{-0.3cm}
    \hfill
    \begin{minipage}{\textwidth}
    \centering
	\begin{tabular}{|c||c|c|c|c|c|c|c|c|c|c|}
	    \hline
	           & de & it & ru & fr & fr-CA & es & es-MX & pt & pt-BR & AVG. \\ \hline\hline
        Baseline  &  &  & &  & & & & & & \\ \hline
        ML-SSP    &  &  & &  & & & & & & \\ \hline
        LA-ML-SSP &  &  & &  & & & & & & \\ \hline
        Opt. LAL &  &  & &  & & & & & & \\ \hline
	\end{tabular}
	\caption{WER with the fine-tuned language-dependent network on the in-domain language data.}
    \label{tab:ft-indomain}
    \end{minipage}
\end{table*}\hfill

Table~\ref{tab:ml-indomain} shows the word error rate (WER) with each single multi-lingual network on the in-domain language data. 
\newpage

\begin{table}[b]
	\centering
	\begin{tabular}{|c||c|c|c|}
	    \hline
	              & \multicolumn{3}{|c|}{ro} \\ \hline\hline
                  & 50 hrs. & 100 hrs. & 690 hrs.   \\ \hline
        Baseline  &  & &  \\ \hline
        ML-SSP &  & &  \\ \hline
        Opt. LAL &  & &  \\ \hline
	\end{tabular}
	\caption{WER on the new language data.}
    \label{tab:outdomain}
    %\vspace{-0.3cm}
\end{table}

\section{Conclusion}
In this paper, we extend the idea of SynGEC \cite{zhang2022syngec} and propose the CSynGEC approach to enhance GEC models by exploiting tailored constituent-based syntax. Experimental results show that incorporating constituent-based syntax produced by a GEC-oriented constituency parser can effectively help GEC models. 
Furthermore, we attempt to combine dependency-based and constituent-based syntax from both intra-model and inter-model aspects, and find that simultaneously using two kinds of syntax leads to more obvious improvement.




% Entries for the entire Anthology, followed by custom entries
\bibliography{anthology,custom}
\bibliographystyle{acl_natbib}
\clearpage
\appendix

\section{Supplemental Tables}

%\section{Hyperparameters of Other Bandit Algorithms}
%\label{sec:bandit_hyperparams}
%Table~\ref{tab:hyperparams} lists the hyperparameters for bandit algorithms other than dBE.

\newcommand\topmidheader[2]{\multicolumn{#1}{c}{\textbf{#2}}\\%
                \addlinespace[1ex]}

\newcommand{\midheader}[2]{%
        \midrule\topmidheader{#1}{#2}}

\newcommand{\specialcell}[3][c]{% 
        \begin{tabular}[#1]{@{}#2@{}}#3\end{tabular}}%

\aptLtoX[graphic=no,type=env]{\begin{table}[htb]
  \centering
  \caption{Hyperparameters of bandit algorithms}
  \label{tab:hyperparams}
  \begin{tabular}{llc}
    \toprule
    Sign & Description & Value \\
    \multicolumn{3}{c}{\textbf{UCB1}}\\
    $c$ & Parameter to control the confidence level used in $\sqrt{c \cdot {\log{t}}/{N_t(arm)}}$ & 0.5  \\
    \multicolumn{3}{c}{\textbf{Thompson Sampling}}\\
    $p(\theta)$ & Prior Distribution & $\mathcal{B}(1, 1)$ \\
    \multicolumn{3}{c}{\textbf{discounted Thompson Sampling}}\\
    $\gamma$ & Discount factor & $1-10^{-8}$ \\
    \multicolumn{3}{c}{\textbf{discounted Thompson Samplingadaptive shrinking Thompson Sampling}}\\
    $M$ & Parameter to control memory usage in a data structure ADWIN2 \cite{ADWIN} & 10 \\
    $\delta$ & Parameter to control the confidence level in a data structure ADWIN2 & $1-10^{-7}$ \\
    \multicolumn{3}{c}{\textbf{EXP-IX}}\\
    $\eta_t$ & Parameter used for weights of arms & $\sqrt{\frac{2 \cdot \log{K}}{K \cdot t}}$ \\
    \addlinespace[1ex]
    $\gamma_t$ & Parameter used for loss estimates & $\frac{\eta_t}{2}$ \\
    \multicolumn{3}{c}{\textbf{EXP3++}}\\
    $\alpha$ & Constant used in calculating $\xi_t(a)$ & $3$ \\
    $\beta$ & Constant used in calculating $\xi_t(a)$ & $256$ \\
    \bottomrule
  \end{tabular}
\end{table}}{\begin{table}[htb]
  \centering
  \caption{Hyperparameters of bandit algorithms}
  \label{tab:hyperparams}
  \begin{tabular}{llc}
    \toprule
    Sign & Description & Value \\
    \midheader{3}{UCB1}
    $c$ & \specialcell{l}{Parameter to control the confidence \\ level used in $\sqrt{c \cdot {\log{t}}/{N_t(arm)}}$} & 0.5  \\
    \midheader{3}{Thompson Sampling}
    $p(\theta)$ & Prior Distribution & $\mathcal{B}(1, 1)$ \\
    \midheader{3}{discounted Thompson Sampling}
    $\gamma$ & Discount factor & $1-10^{-8}$ \\
    \midheader{3}{adaptive shrinking Thompson Sampling}
    $M$ & \specialcell{l}{Parameter to control memory usage \\ in a data structure ADWIN2 \cite{ADWIN}} & 10 \\
    $\delta$ & \specialcell{l}{ Parameter to control the confidence \\ level in a data structure ADWIN2} & $1-10^{-7}$ \\
    \midheader{3}{EXP-IX}
    $\eta_t$ & Parameter used for weights of arms & $\sqrt{\frac{2 \cdot \log{K}}{K \cdot t}}$ \\
    \addlinespace[1ex]
    $\gamma_t$ & Parameter used for loss estimates & $\frac{\eta_t}{2}$ \\
    \midheader{3}{EXP3++}
    $\alpha$ & Constant used in calculating $\xi_t(a)$ & $3$ \\
    $\beta$ & Constant used in calculating $\xi_t(a)$ & $256$ \\
    \bottomrule
  \end{tabular}
\end{table}}

\begin{table}[htb]
  \centering
  \caption{Commit IDs of the PUTs used in our vulnerability discovery and AFL++ used as the baseline.}
  \begin{tabular}{lc}
    \toprule
    Program & Commit \\
    \midrule

    AFL++ & 32a0d6ac315 (ver ++3.14c) \\
    Bloaty &  60209eb \\
    HarfBuzz & 77eeec5 \\
    libarchive & 86c9361 \\
       libxml2 & dea91c9 \\
    MuPDF & ef3d68d \\
   PHP & fdf0455f \\
    Poppler & 6d72d82 \\
    PROJ & 76dfefe \\
    QPDF &  3794f8e \\
    libtpm2 & bc3bb26 \\
    Wireshark  & 1fc621e \\
    Xpdf & N/A (ver 4.03) \\

    \bottomrule
  \end{tabular}
\label{tab:commit-ids}
\end{table}


\begin{table}[htb]
  \centering
  \caption{Initial and theoretical maximum values of code coverage of the PUTs in FuzzBench. 
           Initial values were investigated only in the PUTs used.}
  \begin{tabular}{lcc}
    \toprule
    PUT & Initial & Maximum \\
    \midrule

bloaty\_fuzz\_target & N/A & 83114 \\
curl\_curl\_fuzzer\_http & N/A & 78362 \\
freetype2-2017 & 1517 & 26262 \\
harfbuzz-1.3.2 & N/A & 12212 \\
jsoncpp\_jsoncpp\_fuzzer & N/A & 2114 \\
lcms-2017-03-21 & 149 & 7036 \\
libjpeg-turbo-07-2017 & N/A & 9384 \\
libpcap\_fuzz\_both & 2 & 7294 \\
libpng-1.2.56 & 138 & 3736 \\
libxml2-v2.9.2 & 258 & 67994 \\
libxslt\_xpath & N/A & 51456 \\
mbedtls\_fuzz\_dtlsclient & N/A & 12888 \\
openssl\_x509 & 6026 & 54116 \\
openthread-2019-12-23 & N/A & 19846 \\
php\_php-fuzz-parser & N/A & 215210 \\
proj4-2017-08-14 & 46 & 6534 \\
re2-2014-12-09 & 1 & 3982 \\
sqlite3\_ossfuzz & 4767 & 28766 \\
systemd\_fuzz-link-parser & N/A & 1798 \\
vorbis-2017-12-11 & 410 & 4082 \\
woff2-2016-05-06 & N/A & 5708 \\
zlib\_zlib\_uncompress\_fuzzer & N/A & 910 \\

    \bottomrule
  \end{tabular}
\label{tab:fuzzbench_max_cov}
\end{table}

\begin{table}[htb]
\centering
\caption{List of unique bugs found in the 7-day trial (manually triaged).}
\begin{minipage}{\columnwidth}

\centering
\begin{tabular}{lll}
\toprule

ID & PUT & Bug Type \\
\midrule
Bug-A & bloaty & NULL Pointer Deref \\
Bug-B & harfbuzz & Out-of-bounds Read \\
Bug-C & mupdf & Assertion Fail \\
Bug-D & mupdf & NULL pointer deref \\
Bug-E & xpdf & Stack Overflow \\
Bug-F & xpdf & NULL Pointer Deref \\
Bug-G \footnote{CVE-2022-24106 is issued.} & xpdf & Use of Uninitialized Value \\
Bug-H \footnote{CVE-2022-24107 is issued.} & xpdf & Integer Overflow \\
Bug-I & php & Use-After-Free \\
Bug-J & php & Use-After-Free \\
Bug-K & php & NULL Pointer Deref \\
Bug-L & php & Use-After-Free \\ 
Bug-M & php & NULL Pointer Deref \\
Bug-N & php & Assertion Fail \\
Bug-O & php & Use-After-Free \\
Bug-P & php & Use-After-Free \\
Bug-Q \footnote{CVE-2022-23308 is issued.} & libxml2 & Use-After-Free \\
\bottomrule
\end{tabular}

\label{tab:7d-bug}
\end{minipage}
\end{table}

\begin{table*}[htb]
  \centering
  \caption{List of the PUTs used in Section~\ref{sec:banditcomparison}. If the source code of a PUT was maintained in Git, the latest version at the time of the experiment in the master (or main) branch was used for the build. The `+' sign in a version indicates that the used source code is not the official release version of the source code.}
  \renewcommand\tabularxcolumn[1]{m{#1}}
  \renewcommand{\arraystretch}{1.2}
  \begin{tabularx}{\textwidth}{lXllXc}
    \toprule
    Project & Version & Commit ID & PUT & Format of Initial Seeds & Initial Edge Coverage \\
    \midrule
    Bloaty & v1.1+ & 60209eb & fuzz\_target & Executable (e.g., ELF, PE, Mach-O) & 4773\\
    libmpeg2 & N/A & 5432dc1 & mpeg2\_dec\_fuzzer & MPEG2 & 2428 \\
    PHP & 8.0+ & fdf0455f & php-fuzz-execute & PHP source code & 25241 \\
    HarfBuzz & 3.1.0 & 77eeec5 & hb-shape-fuzzer & Font (e.g., TrueType, OpenType) & 15298 \\
    Xpdf & 4.03 & N/A & fuzz\_pdfload & PDF & 4755 \\
    libtpm2 & N/A & bc3bb26 & tpm2\_execute\_command\_fuzzer & TPM command & 3884\\
    libyaml & v0.2.5+ & f8f760f & libyaml\_dumper\_fuzzer & YAML & 1310 \\
    libzip & 1.8.0+ & bff2eb9 & zip\_read\_fuzzer & ZIP & 805 \\
    libgit2 & v1.3.0+ & 50b4d53 & download\_refs\_fuzzer & Git packet & 3911 \\
    file & 5.41+ & fcbb5d8 & magic\_fuzzer & any (e.g., Zstd compressed file) & 1171 \\
%    MuPDF & 1.19.0+ & ef3d68d & pdf\_fuzzer & PDF & 16936 \\
%    libxml2 & 2.9.12+ & dea91c9 & xml & XML & 7027 \\
    \bottomrule
  \end{tabularx}
\label{tab:put_details}
\end{table*}

%\section{Full Results of Some Experiments}
%\label{sec:full_result}

%Table~\ref{tab:alg_cmp_all}, Figure \ref{fig:vis_bandits} and Figure \ref{fig:full_ablation_time_vs_cov} show the omitted results.

\begin{table*}[htb]
\centering
\caption{Median edge coverage obtained by AFL++ and 8 versions of \OurMethodName-AFL++ in 10 PUTs after 24 h. }

\begin{tabular}{lccccccccc}
\toprule

PUT & AFL++ & UCB1 & KLUCB & TS & dTS & dBE & ADS-TS & EXP3-IX & EXP3++ \\
\midrule

bloaty & \textit{1845.5} & 2198.5 & 2246.0 & 2232.5 & 2191.0 & 2292.0 & \textbf{2340.0} & 2181.5 & 2231.5 \\
harfbuzz & \textit{13497.5} & 14031.5 & 14247.5 & 14360.5 & \textbf{14374.0} & 14067.5 & 14149.0 & 13883.0 & 13891.0 \\
xpdf & \textit{3384.0} & 3494.0 & 3812.5 & \textbf{4618.5} & 4166.5 & 3791.5 & 3902.0 & 3860.0 & 3615.0 \\
libzip & \textit{267.5} & 272.0 & 274.0 & 268.0 & 268.5 & 271.5 & \textbf{276.0} & 271.5 & 268.0 \\
libgit2 & 898.0 & 888.5 & 890.5 & 906.5 & \textbf{916.0} & 884.0 & 914.0 & 899.5 & \textit{881.0} \\
php & \textit{9841.5} & 11861.0 & 13551.5 & \textbf{14324.0} & 14187.5 & 12657.5 & 13408.0 & 11423.5 & 11828.5 \\
libmpeg2 & \textit{1873.5} & 1900.5 & 1905.0 & 1905.5 & \textbf{1906.5} & 1903.0 & \textbf{1906.5} & 1897.0 & 1902.0 \\
tpm2 & \textit{281.5} & 299.5 & 313.0 & 317.0 & \textbf{317.5} & 305.0 & 311.0 & 298.5 & 291.0 \\
libyaml & 2811.5 & 2841.0 & \textbf{2841.5} & \textit{2800.5} & 2837.0 & 2827.5 & 2831.5 & 2828.0 & 2834.5 \\
file & 830.5 & 829.5 & 828.0 & 827.0 & 827.5 & 833.5 & \textbf{840.5} & 826.5 & \textit{826.0} \\

\bottomrule

\end{tabular}

\label{tab:alg_cmp_all}
\end{table*}

\begin{table*}[htb]
\centering
\caption{P-value of Mann-Whitney's U test (Holm-Bonferroni corrected) and Vargha-Delaney's $\hat{A}_{12}$ between AFL++ and the fuzzer in the column for the evaluation conducted in Section~\ref{subsec:eval-vs-existing}. If the p-value is bold, the difference is significant in the test ($p < 0.01$). The characters `L', `M', `S' and `N' in parentheses indicate that the effect size is large, medium, small, and none, respectively, according to \cite{A12}. The `+' sign means the fuzzer in the column is superior to AFL++ when compared by rank sum as well as $\hat{A}_{12}$, and the `-' sign means the opposite.}
\begin{tabular}{lllllllllllll}
 \toprule

  & \multicolumn{2}{c}{MOpt} & \multicolumn{2}{c}{CMFuzz} & \multicolumn{2}{c}{Karamcheti} & \multicolumn{2}{c}{\HavocMAB{}} & \multicolumn{2}{c}{SLOPT} \\
  \cmidrule(r){2-3}\cmidrule(r){4-5}\cmidrule(r){6-7} \cmidrule(r){8-9} \cmidrule(r){10-11}
  PUT & $p$ & $\hat{A}_{12}$ & $p$ & $\hat{A}_{12}$ & $p$ & $\hat{A}_{12}$ & $p$ & $\hat{A}_{12}$ & $p$ & $\hat{A}_{12}$ \\
\midrule

openssl\_x509 & \textbf{ < 0.001 } & 0.82 (+L) & \textbf{ 0.023 } & 0.71 (+L) & \textbf{ < 0.001 } & 0.92 (+L) & \textbf{ < 0.001 } & 0.82 (+L) & \textbf{ < 0.001 } & 0.91 (+L) \\
re2-2014-12-09 & \textbf{ < 0.001 } & 0.18 (-L) & > 0.1 & 0.37 (-S) & > 0.1 & 0.38 (-S) & > 0.1 & 0.47 (-N) & > 0.1 & 0.52 (+N) \\
proj4-2017-08-14 & \textbf{ < 0.001 } & 0.08 (-L) & \textbf{ < 0.001 } & 0.86 (+L) & \textbf{ < 0.001 } & 0.99 (+L) & > 0.1 & 0.54 (+N) & \textbf{ < 0.001 } & 0.92 (+L) \\
sqlite3\_ossfuzz & > 0.1 & 0.55 (+N) & \textbf{ < 0.001 } & 0.85 (+L) & \textbf{ < 0.001 } & 0.93 (+L) & 0.1 & 0.68 (+M) & \textbf{ < 0.001 } & 1.00 (+L) \\
libxml2-v2.9.2 & \textbf{ < 0.001 } & 0.08 (-L) & \textbf{ < 0.001 } & 0.93 (+L) & \textbf{ < 0.001 } & 0.98 (+L) & \textbf{ < 0.001 } & 0.97 (+L) & \textbf{ < 0.001 } & 0.84 (+L) \\
freetype2-2017 & \textbf{ < 0.001 } & 0.08 (-L) & 0.094 & 0.33 (-M) & > 0.1 & 0.54 (+N) & > 0.1 & 0.52 (+N) & \textbf{ < 0.001 } & 0.79 (+L) \\
libpcap\_fuzz\_both & > 0.1 & 0.57 (+S) & \textbf{ < 0.001 } & 0.79 (+L) & \textbf{ < 0.001 } & 0.80 (+L) & \textbf{ < 0.001 } & 0.87 (+L) & \textbf{ < 0.001 } & 0.81 (+L) \\
libpng-1.2.56 & > 0.1 & 0.42 (-S) & > 0.1 & 0.36 (-M) & > 0.1 & 0.49 (-N) & > 0.1 & 0.56 (+S) & 0.049 & 0.68 (+M) \\
lcms-2017-03-21 & > 0.1 & 0.45 (-N) & \textbf{ 0.037 } & 0.70 (+M) & \textbf{ < 0.001 } & 0.85 (+L) & > 0.1 & 0.37 (-S) & \textbf{ < 0.001 } & 0.88 (+L) \\
vorbis-2017-12-11 & > 0.1 & 0.39 (-S) & > 0.1 & 0.56 (+S) & \textbf{ < 0.001 } & 0.20 (-L) & > 0.1 & 0.62 (+S) & 0.092 & 0.65 (+M) \\

\bottomrule
\end{tabular}
\label{tab:statistics}
\end{table*}

\clearpage

\section{Algorithm Overview}

\begin{algorithm}[H]

\centering
\caption{Pseudocode of \OurMethodName{}}
\label{alg:slopt}

\begin{algorithmic}[0]

\Require{\mbox{}\\
    $initial\_seeds$ -- a set of initial test cases \\
    $program$ -- a PUT to be fuzzed
}

\Ensure{\mbox{}\\
    $queue$ -- a set of valuable test cases \\
    $crashes$ -- a set of test cases that trigger crashes
}

%\begin{adjustwidth}{-9pt}{}
%\setstretch{0.85}
\vspace{5pt}

\Function{RandomMutation}{$seed, instance_{mut}, instances_{bat}$}
\State $input$ $\gets$ \Call{CopyBytesFromSeed}{$seed$}
\State $mutation$ $\gets$ \Call{SelectArm}{$instance_{mut}$}
\State $idx$ $\gets$ \Call{GetGroupIndex}{$len(input)$}
\State $batch\_size$ $\gets$ \Call{SelectArm}{$instances_{bat}[idx][mutation]$}
\For{$i$ $\gets$ $1$ \textbf{to} $batch\_size$}
    \State $pos$ $\gets$ \Call{SelectPosition}{$input$}
    \State $input$ $\gets$ \Call{ApplyOperator}{$mutation, input, pos$}
\EndFor
\State \textbf{return} $input, mutation, batch\_size$
\EndFunction

%\end{adjustwidth}

%\vspace{-6pt}

%\begin{adjustwidth}{-9pt}{}
%\setstretch{0.85}

\vspace{5pt}

\Function{MutationFuzzing}{$initial\_seeds, program$}

\State $crashes$ $\gets$ $\varnothing$
\State $queue$ $\gets$ \Call{ConstructQueue}{$initial\_seeds$}
\State $instance_{mut}$ $\gets$ \Call{CreateBanditArms}{$number\_of\_mutations$}
\For{$i$ $\gets$ $1$ \textbf{to} $5$}
 \For{$j$ $\gets$ $1$ \textbf{to} $number\_of\_mutations$}
  \State $instances_{bat}[i][j]$ $\gets$ \Call{CreateBanditInstance}{$7$}
 \EndFor
\EndFor

\State

\While{ $\neg$ \Call{UserWantsStop}{\null}}
 \State $seed$ $\gets$ \Call{SelectSeed}{$queue$}
 \State $energy$ $\gets$ \Call{DecideEnergy}{$seed$}
 \For{$i$ $\gets$ $1$ \textbf{to} $energy$}
  \State $input, mutation, batch\_size$ 
  \State $\gets$ \Call{RandomMutation}{$seed, instance_{mut}, instances_{bat}$}
  \State $result$ $\gets$ \Call{ExecutePUT}{$program, input$}
  \State $b$ $\gets$ \Call{WasInputValuable}{$result$}
  \State \Call{RewardArm}{$mutation, b$}
  \State \Call{RewardArm}{$batch\_size, b$}
  \State \Call{SaveInputIfValuable}{$queue, input, result$}
  \State \Call{SaveInputIfCrash}{$crashes, input, result$}
 \EndFor
\EndWhile
\EndFunction

%\end{adjustwidth}

\end{algorithmic}
\end{algorithm}




\end{document}