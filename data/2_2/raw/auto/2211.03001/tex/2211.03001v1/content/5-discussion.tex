\section{Discussion and Analysis}
In this section, we analyze and discuss our findings on the quantitative results and post-experiment interviews by revisiting our research questions in Section 4.

\subsection{Gaze Select-and-Snap on Positioning}
Gaze Select-and-Snap showed the most promising and excellent results among the three VRDoc tools. We were able to observe readers significantly reducing time in selecting and reading document-of-interest out of multiple documents.
Three participants commented that they enjoyed using Gaze Select-and-Snap not only for bringing documents up front to read but to move around multiple documents quickly. P4 mentioned, ``I found it easier to scatter documents around with my gaze rather than holding up the controller and swaying my arm constantly. I could just look somewhere else to move away documents''. P5 and P11 commonly mentioned how with a larger VR space, they would find Gaze Select-and-Snap even more useful to find and orient document windows.

Although we cannot make direct comparisons between tasks since the conditions differ, when using baseline, the results in workload show that overall users’ perceived workload was higher with multiple short passages (Task 1) than with longer passages (Task 3). Implying that reading multiple documents was more demanding than reading a single long document as it involved frequently re-orienting document windows. This further strengthens the usability of Gaze Select-and-Snap, and how much positioning is important in reading documents in VR settings.

\subsection{Gaze MagGlass on Readability}
The quantitative results of the `readability' prove that Gaze MagGlass successfully enhances users' perceived readability when reading in VR.
Gaze MagGlass was the only feature that had an on/off option, since, even in a real-world scenario, we do not require magnification glasses at all times. For Task 2 and Task 4, we were able to observe that Gaze MagGlass was activated for more than half of the time. This means that this is a necessary feature for reading in VR that will likely be used. When participants were only provided with the baseline, they often moved their position to better read the passage or reached out to grab the document and manually move the document towards them.

Six participants commented that they anticipate this feature to be utilized when reading documents with a more complex structure such as paragraphs with varying fonts or multi-column documents. Five of the participants mentioned how when used with Gaze Scroll in Task 4, Gaze MagGlass helped in keeping track of where they are in the document.

\subsection{Gaze Scroll on Reducing Feelings of Fatigue}
Gaze Scroll showed promising results in the measured participants' subjective perception, yet there were some mixed reviews.
Four participants commented that Gaze Scroll was convenient in the initial reading phase since the buttons were naturally placed at the end of the reader's gaze. However, during the reading comprehension test phase, when the reader had to come back to the passage to look for information, they would prefer to use the controller since it was easier to quickly navigate to the top or bottom of the page. This implies that, for actual implementation, it would be a meaningful approach to integrate the existing trackpad-scrolling with Gaze Scroll to enhance users' reading experience.

In terms of fatigue, the SSQ results did not reveal a significant difference in fatigue when compared with the baseline. This may be because in the experiment setting, readers were not required to go through documents for hours as much as we do in our daily lives. However, Gaze Scroll is proved to be effective in reducing perceived workload as it showed significantly smaller TLX scores compared to the baseline.


\subsection{Usability, Efficiency, and Preference of VRDoc}
Subjective measures proved that individually, readers had a positive experience with each of the tools. When combined, we are able to see that all three subjective measures; effectiveness, readability, and preference were higher than the individual scores of each tool. This implies that with the three tools combined, the tools form a synergy effect.

As shown in the results, with VRDoc, users stated that they perceived improved readability for both long passages and short passages. We observe a similar conclusion with respect to how efficient the methods were in assisting users' reading experience. All these led to VRDoc having a significantly higher preference. Eight of the participants collectively mentioned how VRDoc presented a new possibility in reading in VR, which is what they have not imagined before. P13 commented, ``\textit{I used to think reading in virtual reality would be unbearable, but this opened my eyes that with the right tool, it is quite enjoyable.}’’. Similarly, P11 mentioned ``\textit{I especially liked how I felt in control of the reading space. With appropriate annotation tools, I can see reading in VR could be more useful in certain cases.}’’ Five participants pointed out that reader-specific interactions are essential in the virtual environment, and the lack of them prevents them from even attempting to read long texts in a VR setup.
Hence, for real-world applications, developers should consider providing readers in VR with a set of tools that tackle pain points to maximize users' performance.






% \subsection{TCT \& Reading Comprehension}
% Our user study results indicate that subject generally read the document faster when using Gaze Select-and-Snap or VRDoc (Task1 and Task 4) compared to using basic 3d object manipulation methods. The solving time for VRDoc was slightly shorter, although no significant difference was found. While the errors made in the reading comprehension tests with VRDoc were slightly lower than \textit{baseline} only, there were no significant differences. We can conclude that VRDoc significantly improves time efficiency in the overall reading experience while also having potential to improve users' comprehension of the document content.

% \subsection{Usability and Workload}
% With respect to usability and workload, VRDoc resulted in improved results for both R1 and R2.
% As for usability, six subjects commented ``{\em I needed some time to get familiar with the way the method works but once I got used to it, it turned out to be very useful when carrying out a trial.}'' Four of the users mentioned ``{\em I especially liked the feature that brought the document window right towards me. It definitely saved me a lot of time.}''
% The results of usability are compatible with those of work load.

% Although we cannot make direct comparisons between R1 and R2 since the conditions and the available interactions differ, the results in workload show that overall users’ perceived workload was higher with R1 than R2. Implying that reading multiple documents was more demanding than reading a single long document as it involved frequently re-orienting document windows.


% \subsection{Subjective Measures}
% Users perceived with VRDoc that the documents had better readability for both R1 and R2. We observe a similar conclusion with respect to Efficiency.
% All these led to the difference in preference: user preferred having VRDoc for both R1 and R2. To summarize, with VRDoc, subjects were able to efficiently read documents with enhanced readability. Overall, they prefer using gaze-based interactions.

% \subsection{Using VRDoc}
% All participants used the interaction features of VRDoc at least once for both reading tasks. 

% \textit{Gaze Select-and-Snap} was actively used in both R1 and R2. In R1, we  observed that participants tend to use this feature to initially position the document in front of them, and further adjust the orientation with the `grab’ method if needed. On the other hand, as R2 involved a single document, the feature was not activated as often. Nevertheless, it was observed that 13 out of 17 participants used \textit{Gaze Select-and-Snap} while not using the laser pointer manipulation at all in R2. 

% With \textit{Gaze MagGlass}, participants especially preferred to use this feature when reading different fonts. Participants tend to set the initial position where they are most comfortable to view the bigger font, and use the magnifying feature for the smaller font. When they were only provided with the \textit{baseline}, participants often moved their position to better read the passage or reached out to grab the document and manually move the document forward.

% For \textit{Gaze Scroll}, there were mixed reviews. Six of the participants mentioned that with this feature, they were able to read the document in a natural flow as they did not have to intentionally focus on the scrolling event. However, four participants particularly mentioned that scrolling with the VR controller trackpad was sufficient, and they did not like how they had to constantly look at the button when they wanted to go to the bottom or top of the document right away. 

% \subsection{Design Insights}
% Throughout our study, we observed that participants generally preferred to have the document window at a certain distance rather than to have it up close. This naturally comes from users' experience in a real-world scenario, where the size of the texts they encounter in textbooks or 2D monitors is fairly small than large. Thus, it is important to determine the minimum distance that users are comfortable with in correlation with the font size that they would use. Users who require further adjustment or enhancement can make use of the `grab' method or \textit{Gaze MagGlass}.

% From our post-experiment interview, some of the participants mentioned that they would prefer if VRDoc was automatically augmented when the system detects that the user is in a working environment.

