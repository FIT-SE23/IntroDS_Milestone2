\section{Conclusion, Limitations and Future work}
We present VRDoc, a set of three gaze-based interactions that improve users’ reading experience in VR. Through a formative study, we identified  major issues users face when reading documents with conventional object manipulation methods in a VR setting. 
%Participants’ feedback indicated the need of interaction methods that address the difficulty in selecting positioning 2D document windows, poor readability, and arm fatigue due to constant mid-air interaction with the VR controller.
We utilize eye-tracking as a solution to streamline basic interactions for users’ convenience while minimizing the need to continuously hold up a VR controller.  VRDoc consists of three gaze-based interactions that each addresses the aforementioned problems: \textit{Gaze Select-and-Snap} for selecting and positioning document windows, \textit{Gaze MagGlass} for improving readability through magnification, and \textit{Gaze Scroll} for scrolling long documents with gaze. We evaluated our method through a series of reading tasks 
%: reading multiple documents and reading a long-structured document. The 
and overall results indicated that the combined VRDoc tools significantly improve participants’ performance speed, usability with less workload when compared to using conventional methods. 
%Participants also found VRDoc to be more readable and effective, and preferred over the conventional method.

Our study offers insights on how gaze-based interactions could assist VR reading interfaces.  Our current approach assumes that good eye tracking capabilities are available in the VR system. Since participants’ performances are highly governed by the experiment settings, further investigation is needed to evaluate VRDoc in general applications corresponding to virtual offices or remote collaboration.
The detailed setting used for VRDoc were all empirically determined for our study. Further research should be conducted to investigate the optimal distance to place document windows for Gaze Select-and-Snap, the degree of magnification and size of canvas for Gaze MagGlass, and the placement of buttons and activation conditions for Gaze Scroll.

The experiment setting used in our approach assumed minimal movement  from the participants. It is anticipated that there will be a different set of interactions that users would need assistance with when VR locomotion is considered.  Our current study follows the guidelines on how to present texts in VR~\cite{dingler2018vr}. However, these guidelines might not always be applicable when we read documents with complex layouts or structures, such as academic papers with two columns or documents with multimedia resources. Similar to recent efforts on identifying the optimal format for reading in mobile phones~\cite{adobe2021liquidmode}, it would be worth investigating a detailed format of how the structures should be converted for optimal reading  experiences in VR. 

Going beyond simply consuming documents, it is notable that VR offers a possibility of a \emph{collaborative} workspace. For our future work, we aim to investigate collaborative interactions when multiple users read, share and discuss documents. The findings can be adapted to various multi-user applications such as virtual classrooms and virtual conferences.  As new display technologies and headsets are developed to deal with the challenging issues related to pixel density, pixel resolution, field-of-view, refresh rate and distortion, it would be useful to evaluate their impact on VR reading experiences. 

