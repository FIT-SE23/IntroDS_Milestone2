\documentclass[]{article}
\usepackage{fullpage}
\usepackage{graphicx}
\usepackage[colorlinks=true,citecolor=black,linkcolor=black,urlcolor=blue]{hyperref}
\usepackage{tikz,color,float,epsf,caption,subcaption}
\usepackage{multirow}
\usepackage[round]{natbib}   % omit 'round' option if you prefer square brackets
\bibliographystyle{plainnat}



%opening
% \title{Identification of Bottlenecks in an Airline Network using Network~Science to Improve Operations}
\title{A network science approach to identify disruptive elements of an airline}
\author{Vinod Kumar Chauhan$^{1,2}\thanks{vinod.kumar@eng.ox.ac.uk (This work was done at University of Cambridge)}$, Anna Ledwoch$^{1,3}$, Alexandra Brintrup$^1$, \\ Manuel Herrera$^1$, Vaggelis Giannikas$^3$, Goran Stojkovic$^4$, Duncan Mcfarlane$^1$\\ $^1$Institute for Manufacturing, University of Cambridge\\ $^2$ Department of Engineering Science, University of Oxford\\
$^3$ School of Management, University of Bath\\ 
$^4$The Boeing Company}

\begin{document}
	
	\maketitle
	
	\begin{abstract}
    	Nowadays, flight delays are quite notorious and propagate from an originating flight to connecting flights, which lead to big disruptions in the overall schedule. These disruptions cause huge economic losses, affect the reputation of airlines, lead to a wastage of time and money of passengers, and have a direct environmental impact. This paper presents a novel network science approach for modelling and analysis of an airline’s flight schedules and its historical operational data. The final aim is to find out the most disruptive airports, flights, flight-connections and connection-type in an airline network. In this regard, disruptive elements refer to influential or critical entities of an airline network. These are the elements which either can cause (as per airline schedules) or has caused (as per the historical data) the biggest disturbances in the network. An airline, then can improve their operations by avoiding disruptive elements. This can be achieved through introduction of an extra slack time between connecting flights and by creation of alternate arrangements for aircraft and crew members for the disruptive flights and flight-connections. The proposed network science approach for disruptive elements' analysis is validated with a case-study of an operating airline. Interestingly, the analysis shows that (potential) disruptive elements in the schedule of the airline are also (actual) disruptive elements in the historical data and should be attended first to improve operations.
    % 	Besides, another remarkable result is that passenger connections are the main bottleneck connection for propagation of delays from one flight to another.
		
		\textbf{Keywords:} air transport, flight delays, airline disruptions, delay propagation, network science.
	\end{abstract}
	
	\section{Introduction}
	\label{sec_intro}
	Nowadays, flight delays occur quite frequently and it has been observed that in 2017, 20\% of flights arrived late by at least 15 minutes or more in Europe (\cite{Walker2017}). These delays can be classified, mainly into three categories: airline issues, airport issues and weather issues. The airline issues arise due to problems at the airline's end, e.g., delay in passenger boarding and disembarkation, aircraft repairs and sudden unavailability of crew members etc. The airport issues arise due to problems at the airport authority's end, e.g., unavailability of slots for flights to take off or land, longer time in security checks and airport closures etc. The third category is the severe weather conditions, which can't be controlled like the other two, e.g., storms and snow falls can disturb the airport operations (\cite{brueckner2022airline}).
	
	Airline operations represent a complex distributed transportation system, which have several interacting and inter-dependent entities, like passengers, crews, airlines and airports etc. So, when one thing goes wrong, it has the potential to affect the whole system and because of that when one flight is delayed, delays can propagate to other flights, which is called delay propagation, secondary delay or reactionary delay. In extreme cases, delays can affect hundreds and thousands of flights, e.g., \cite{Ledwoch2022} discussed a case where 2565 flights were affected and delays were propagated for more than 12 days due to the severe weather conditions. Delay propagation is a major issue for airlines, and for example, accounted for approximately 30\% to 60\% of delays in the European airports in 2017 (\cite{Walker2017}) and around 34\% in the US in 2007 (\cite{AhmadBeygi2008}).
	
	`Flight disruption' is the technical term used to refer to flight delays, which is defined as a `situation where a scheduled flight is cancelled, or delayed for two hours or more, within 48 hours of the original scheduled departure time'\footnote{Airports Council International (ACI), Sixth Worldwide Air Transport Conference (ATCONF), Montreal, 18 to 22 March 2013.}. But nowadays, this definition does not fit well to represent the practical situations. This is because even smaller delays, say few minutes, can propagate and affect many flights. Moreover, airlines consider a flight delayed only if it is delayed by at least 15 minutes which is an industry standard to study the on-time performance of flights. But even this standard is not very practical because as discussed earlier, a small delay, say 10 minutes, has the tendency to propagate so to understand the root causes and impact of flight delays, it is important to consider even small delays (\cite{Ledwoch2022}).
	
	Flight disruptions have a huge impact on its stakeholders which can be mainly classified into three categories: airlines and governments, passengers and environment. Disruptions cause huge economic losses to the airlines and to the economy of the governments. It also affects reputation of the airlines and retainability of their passengers. According to a report in Amadaus~(2016), disruptions cost around $\$$60~billion to the airlines and their clients annually around the world (\cite{Gershkoff2016}). Flight delays also cost time and money to the passengers and can make them to miss important events. Flight delays also cause environmental pollution by extra emission of harmful gases, like carbon-dioxide ($CO_2$), due to the combustion of fuel. In a time, when we are facing the problem of climate change and global warming, it's a big concern. Thus, keeping in view the huge impact of disruptions on airlines, environment and passengers, there is a need to understand the delays and their propagation so that airlines can develop robust schedules to reduce the delays and their propagations.
	
	Nowadays, networks are present everywhere and problems in technology, biology and social sciences etc. can be formulated as networks (\cite{Newman2018}). All this has led to the emergence of a new field called `network science' (\cite{Brandes2013}). The concepts and techniques of network science can be applied to problems in different domains to better understand them and explain certain phenomenon, e.g., prevention of infectious diseases, understanding the root causes and impact of flight delay propagation and study of extinction of species etc. (\cite{Newman2018}).
	
	Airline networks are also studied extensively using network science techniques and data are modelled as single airline network (\cite{Reggiani2010,Ledwoch2022}), multiple airlines network (\cite{Baruah2019,Yang2019}), airport network (\cite{Lordan2019,Wandelt2019,Jiang2017,cumelles2021cascading}) and flight delay network (\cite{Ledwoch2022}) etc. Multilayer networks are the recent developments to study the air transport network which help to compare different airlines and draw new insights (\cite{Jiang2019,Mikko2014,Costa2018,Lordan2017}).
	
	In this paper, we have proposed a network science approach to find the most disruptive elements' in an airline flight network. The disruptive elements are those elements of the airline network which either have the potential to cause the big disruption (according to schedule of airline) or have actually caused the biggest disruptions (according to the historical operational data). So, the disruptive elements are the major causes (potential as well as actual) for delay propagations, and can be used to improve the operations of the airline. The proposed approach, first models the airline schedule or historical operational data as a network and then network science techniques are applied to identify the disruptive elements. For example, suppose Fig.~\ref{fig_idea} models an airline's schedules/delays (refer to Sec.~\ref{sec_network_science} for different ways of modelling the airline data) where nodes refer to flights and edges refer to flight-connections. In this network, nodes 1 and 5, and edge (1,5) are one of the most disruptive elements in the network as they are one of the most influential elements in the network. Analysis of the airline schedules gives potential disruptive elements, while analysis of historical data, more precisely delay propagation gives actual disruptive elements. For example, Fig.~\ref{fig_idea} presents a delay network where each node is a delayed flight and each edge is delay propagation. From the figure, it is clear that delay to flight 5 lead to delay propagation to flights 9, 10, 11, 12 and 13. Hence, there is a need to identify disruptive elements of an airline network to improve operations because avoiding delays to disruptive elements' can avoid big disruptions (\cite{brueckner2021airline}.
	
	\begin{figure}
	    \centering
	    \includegraphics[width=0.6\textwidth]{idea.jpg}
	    \caption{A representation of an arbitrary airline network: nodes 1 and 5, and edge (1,5) are one of the most disruptive elements in the network and by avoiding delay to these disruptive elements could avoid big disruptions.}
	    \label{fig_idea}
	\end{figure}
	
	The contributions of the paper are summarised below:
	\begin{itemize}
		\item We have presented novel modelling of an airline's data (refer to Subsec.~\ref{subsec_types} for details) into airport--flight, flight--flight-connection and multilayer flight--flight-connection networks, each with two variations as connection network and delay network, modelling schedules and historical operational data, respectively. This modelling approach enables us to pinpoint the major causes of flight delay propagation, in terms of airports, flights, flight-connections and connection-type.
		
		\item We have proposed a novel application of network science techniques, like centrality and percolation etc., to find the disruptive elements' of an airline's flight network. This work helps to identify influential/central airports, flights, flight-connections and connection-type from an airline flight network, which can cause or had caused major disruptions. Modelling of schedules give us potential disruptive elements whereas modelling of historical operational data provide us the actual disruptive elements. Thus, the proposed approach helps to find issues with the airline schedule and operations to identify disruptive elements which need the first attention to improve the airline operations.
		
		\item The proposed approach is validated with a case study of an airline and we observe an interesting point that the potential disruptive elements of the airline have been identified as the actual disruptive elements. This indicates an issue with the airline schedule itself for causing disruptions and need immediate attention to improve operations.
		
		\item This analysis provides a tool to the airline to identify airports, flights and flight-connections which cause big disruptions. It also helps to analyse the schedules and identify the issues with the schedule by looking for the overlap between potential disruptive elements (in connection network) and actual disruptive elements (in delay network). The findings of this analysis gives an airline insights to make their schedule robust to the disruptions by paying attention to the disruptive elements and making extra arrangements, like increasing the slack time for disruptive flights and arranging alternate crew members for disruptive crew connections etc.
	\end{itemize}
	
	Rest of the paper is organised as: Section~\ref{sec_literature} presents literature review, Section~\ref{sec_network_science} describes the network science approach used and modelling of data into networks. Case study of an airline is discussed in Section~\ref{sec_analysis} and concluding remarks are presented in Section~\ref{sec_conclusion}.
	
	\section{Literature Review}
	\label{sec_literature}
	Nowadays, the network science has been widely used to analyse and provide new insights to real life problems across different disciplines from social science, engineering, technology and biology to airline problems (\cite{Newman2018}). In airline problems, the data have been modelled as different types of networks and the analysis has been targeted to draw different types of insights. The airline problems can be categorised according to the type of data and corresponding network modelling as: single airline network (\cite{Ledwoch2022,Reggiani2010}), multiple airline network (\cite{Yang2019,Jin2019,Du2018}), airport network (\cite{Song2017,Baruah2019,Lordan2019,Wang2019}) and multilayer airline network problems (\cite{Mikko2014,Lordan2017,Costa2018}).
	
	In airport networks, airports are taken as nodes and flights between airports are used as links/edges. For example, \cite{Lordan2019} discussed the global airport network and divided the network into seven regions, compared network properties of different regions and robustness by finding the core and critical cities. \cite{Baruah2019} compared airport network of India for three major airlines Indigo, Jet Airways and Air India, using network parameters and performed robustness analysis of airlines by studying the change of parameters on the removal of key nodes from the network. \cite{Wang2019} analysed the resilience of an airport network by considering structural and dynamical aspects and by combining network science and operational dynamics. \cite{Olariaga2018} studied the configuration of the traffic in the airport network as a result of the liberalization of the air transport industry of Columbia. \cite{Jiang2017,Hossain2017,Jia2014,Couto2015,Baruah2019,cumelles2021cascading} analysed the structural properties of the airport networks of China, Australia, US, Brazil and India, respectively. \cite{Wandelt2019} presented a review and comparative analysis of evolution of domestic airport networks for India, US, Europe, China, Russia, Brazil, Australia and Canada.
	
	Single and multiple airline problems focus on the analysis of a single airline data or comparative analysis of multiple airlines. For example, \cite{Reggiani2010} analysed Lufthansa airline's network for understanding the changing patterns in the network configurations. \cite{Ledwoch2022} discussed a case study of anonymous airline to find the root-causes and impact of delay propagations through the airline's flight network, and frequent delay patterns. \cite{Yang2019} discussed the structural properties of Chinese airline network and \cite{Jin2019} used network motif to analyse the structural characteristics of China's passenger airline networks. \cite{Du2018} discussed the structure and dynamics of delay propagation using delay causality networks based on Granger causality test. \cite{Alderighi2007} assessed the point-to-point and hub-and-spoke airline configurations based on spatial and temporal dimensions, and concluded that temporal analysis is helpful to provide clear distinction between full-service carriers and low-cost carriers, while spatial analysis helped to find peculiarities within groups. \cite{Lordan2015} analysed the robustness of the three airline alliances (i.e., SkyTeam, oneworld and Star alliance) through giant-component size vs number of isolated airports and using a multi-scale vulnerability measure and concluded that Star alliance is the most resilient. \cite{Lordan2014} proposed three levels of analysis, airlines, airline alliances and global route network, to analyse the robustness of air transport networks.
	
	Multilayer network considers the different types of relationships/connectivity between the nodes and adds extra value to the analysis of networks (\cite{Mikko2014}). For example, \cite{Lordan2017} analysed the multilayered structure of European airlines using core, bridge and periphery as layers of the network and concluded that the network is more robust to the isolation of the core nodes than isolation of a combination of bridge and core nodes. \cite{Hong2016} studied the structural properties of the Chinese multilayered air transportation network using different airlines as layers, and major and low cost airlines. The authors found that rich-club effect and small-world property are mainly caused by major airlines. \cite{Costa2018} analysed the multilayered and time varying structure of the Brazilian air transport network using different airlines as layers, and unveiled the strategies used by airlines to deal with the disruptions in the flight network and assessed the impact of the economic crisis on airlines. \cite{Jiang2017b} studied the transition point of air transport network using airlines as layers and found that Chinese air transport still has some scope to improve the operations while European airline network is already operating at its nearly optimum level. \cite{Wang2017} analysed the multilayered air transport network using airport network, airlines network, air route network and air traffic management network as layers and concluded that, for flight delays, airports with similar geographical locations exhibit similar dynamics and delays or failure propagation decays slowly.
	
	Centrality is an important concept in network science which helps to identify influential nodes and edges in the network (\cite{Newman2018}). \cite{Clark2018} discussed the robustness and recovery strategies of US airport network using network science techniques including different centrality measures. \cite{Sathanur2019} also studied US airport network from the perspective of delay propagation and identified critical/influential airports based on influence maximization algorithm and diffusion simulator. \cite{Jin2019} studied the robustness of network and identified critical nodes' group in the network using eight different attack strategies, like random failure, target attack and based on different centrality measures etc. \cite{Du2017} studied the Chinese air route network and identified the vital edges in the network using a memetic algorithm, and showed that topologically important edges are not necessarily vital edges. \cite{Li2018} again analysed the Chinese transport network as a multilayered network and identified influential nodes based on evidence theory. \cite{Cong2016} also modelled the Chinese air transport network using airports as nodes and the correlations between air traffic flow of airports as edges, and identified the critical airports based on spatial and temporal correlations among airports.
	
	There has been extensive research on flight delay predictions for a long time and variety of approaches have been used to study and predict the delays using network science, statistical modelling and machine learning etc. For example, \cite{Wang2003} developed analytical model to study delay propagation in US airports which separates fixed and variable parts of delays and delay propagation. \cite{Abdelghany2004} modelled the airlines data into directed acyclic graph and used shortest path algorithm to predict flight delay propagations for irregular operations. \cite{Jani2005} developed an analytical model for hub-and-spoke airline network to estimate the economic impact of large disruptions. \cite{Xu2005} developed a Bayesian Network to investigate and visualise delay propagation among airports. \cite{Liu2008} also used Bayesian Networks to model the arrival delays at a hub airport and study the delay propagation within and from the airport. \cite{AhmadBeygi2008} proposed propagation trees to study the potential of delay propagation from the airline schedules and their actual operations. \cite{Wong2012} used Cox proportional hazards model to study delay propagation in Taiwanese domestic airline. Recently, a lot of machine learning based models are used to predict the future flight delays, e.g., deep learning (\cite{Gui2020}), regression (\cite{Gopalakrishnan2017}), classification and regression trees (\cite{Gopalakrishnan2017}), Support Vector Machine (\cite{Wu2019}), Gradient Boosting Classifier (\cite{Thiagarajan2017}) and random forests (\cite{Gopalakrishnan2017}) etc.
	
	In this paper, we have modelled the data from a single airline into different connection networks, delay networks and multilayer networks. Further, we have utilized network science techniques, like centrality and percolation etc., to identify disruptive airports, flight-connections and flights, having either potential to affect several other flights (according to the airline schedule) or who actually lead to highest delay propagations to other flights (according to the historical operational data). This work is a novel application of network science techniques and provides useful tool for airlines to improve their flight operations by paying attention to the most disruptive elements and build robust schedules. This paper addresses the problem of delay propagation but it is different from above literature in terms of modelling of data into networks, analysis and benefits of the analysis. To the best of our knowledge, this is the first work to focus on the major causes of the delay propagation. Similar to \cite{AhmadBeygi2008}, it analyses schedules and operational data for delay propagation but the authors used propagation trees to study the potential of delay propagation and this work uses network science approach to analyse schedules and operational data to identify major causes of delay propagation.
	
	
	\section{Network Science Approach}
	\label{sec_network_science}
	This section presents the modelling of an airline's schedules and historical operational data, and different network science properties and techniques to study the airline's network.
	
	\subsection{Types of Networks}
	\label{subsec_types}
	Network modelling determines the insights which could be drawn using the network science approach. So, we have modelled the problem, mainly, by three types of networks: connection network, delay network and multilayer flight connection network, which are discussed below and their summary is presented in Table~\ref{tab:networks}.
	
	\subsubsection{Connection Network}
	\label{subsub_con_net}
	Connection Network (CN) models schedules of an airline along with information about possible shared connections, in terms of passengers, crew and tail (i.e. aircraft) between connecting flights. It takes into consideration the scheduled arrival, departure, origin and destination airport but it does not take into consideration actual operations of the data, like actual arrival and departure of flights. CNs cover the schedule for a given period of time. We have three types of CNs: Airport-Flight Connection Network (AFCN), Flight-Connection Connection Network (FCCN) and Multilayer Connection Network (MLCN), as discussed below (MLCN is discussed later in \ref{subsubsec_multi_net}). An example of Flight-Connection connection network is represented in Sub-figure~\ref{subfig_con_net}. CNs help to analyse the airline schedule and find the potential disruptive elements as the analysis is on schedules. They help to identify busiest elements of the network which can be potential cause for big disruptions. We can cross-check the disruptive elements from CN with corresponding delay network (DN; discussed below), and if there is no overlap between the disruptive elements in CN (potential disruptive elements) and in DN (actual disruptive elements) then that is an indication that airline schedule has no major issue, else overlap is an indication that something might be wrong with the schedules itself and airline should fix the overlapped disruptive elements first.
	
	\textbf{AFCN:} In AFCN, airports are the nodes and flights are considered as edges with direction from origin airport to destination airport. Two airports have an edge between them if there is at least one flight between them and if there are more than one flights between the airports then that is represented as weight of the edge. The analysis of AFCN gives us potential disruptive airports and potential disruptive airport pairs, as discussed in \ref{subsec_connection}, which have highest traffic between them.
	
	\textbf{FCCN:} Each flight is taken as a node, where flight is represented using flight number, scheduled departure time, origin airport and destination airport, e.g., suppose flight number 100 is scheduled to depart at 10:10 from airport A1 to airport A2 then 100.10:10.A1.A2 is considered as a unique flight and as a node. If two flights share passengers, crew or tail then this is called a flight-connection and is taken as edge of the network with direction from first flight to second flight. The frequency of flight-connection between flights is taken as the weight of the edges, e.g., suppose above flight number 100 flies everyday then in one week its weight would be seven. The analysis of FCCN gives us potential disruptive flights and flight-connections, as discussed in \ref{subsec_FCN}, which are those flights which have potential to cause disruption to several others.
	
	\begin{figure*}[htb!]
		\centering
		\begin{subfigure}[t]{0.33\textwidth}			
			\centering
			\includegraphics[width=\textwidth]{CN.png}
			\caption{Flight-Connection connection network}
			\label{subfig_con_net}
		\end{subfigure}%
		~ 
		\begin{subfigure}[t]{0.3\textwidth}			
			\centering
			\includegraphics[width=\textwidth]{DN.png}
			\caption{Flight-Connection Delay network}
			\label{subfig_del_net}
		\end{subfigure}
		~ 
		\begin{subfigure}[t]{0.3\textwidth}			
			\centering
			\includegraphics[width=\textwidth]{multinet}
			\caption{An example of multilayer network with 5 nodes and three layers, viz., A1, A2 and A3.}
			\label{subfig_mln}
		\end{subfigure}		
		\caption{Network Types: solid rings are the delayed flights and others are on time flights; T, C and P refers to tail, crew and passenger connections, respectively, and number refers to the frequency of flights.}
		\label{fig_networs}
	\end{figure*}
	
	\subsubsection{Delay Network}
	\label{subsubsec_del_net}
	Delay Network (DN) models historical operational data of an airline, mainly, focusing on the delays and their propagation through the network. DNs can be seen as a subset of CNs considering only the delayed flights and flight connections which lead to delay propagation, e.g., sub-figure \ref{subfig_del_net} represents Flight-Connection delay network. DNs help to analyse the disruptive elements from the historical data as it take into consideration actual departure, arrival times and delays propagated when airline schedule was executed. It helps to identify those flights and flight-connections which had lead to the biggest disruptions in the network. DN can be used to review airline operations and take decisions to improve the operations by paying attention to the disruptive elements in the network. We have three types of DNs: Airport-Flight Delay Network (AFDN), Flight-Connection Delay Network (FCDN) and Multilayer Delay Network (MLDN). MLDN is discussed later in \ref{subsubsec_multi_net}.
	
	\textbf{AFDN:} It is similar to AFCN except that AFDN considers only delayed flights for forming edges and delayed flight frequencies as weight of the edge. AFDN helps to identify the disruptive airports and airport pairs, which actually caused the biggest disruptions, as discussed in \ref{subsec_connection}.
	
	\textbf{FCDN:} It is similar to FCCN except that it considers only delayed flights as nodes and  flight connections with delay propagation only. The characterisation of the delay propagation is followed as used in \cite{Ledwoch2022} and propagated delay is equal to arrival delay of incoming first flight minus slack time, where slack time is time difference between departure of second flight (connecting flight) and arrival of first flight minus minimum ground time to prepare for the second flight. The delay propagation frequency of flight connection is used as weight of the edge. FCDN helps to identify disruptive flights and disruptive flight-connections which actually caused the biggest disruptions in the network, as discussed in \ref{subsec_FCN}.
	
	\subsubsection{Multilayer Flight Connection Network}
	\label{subsubsec_multi_net}
	Multilayer Networks (ML) are the networks where interaction between elements of the network may change over time, can have different types of connections or include other types of complications so they can be divided into sub-systems depending upon these interactions (for details refer to \cite{Mikko2014}). These networks add extra value to the network analysis by considering the types of connectivity in the network. We use three types of connections, like crew connection, tail connection and passenger connection between flights, as different layers. ML can have two types as Multilayer Connection Network (MLCN) and Multilayer Delay Networks (MLDN). MLCN and MLDN, are similar to FCCN and FCDN, respectively, where each layer is a FCCN/FCDN network with connections of one type. An example of ML is shown in \ref{subfig_mln} which has five nodes, common to all the layers, and three layers with their own edges depending upon the type of the edges.
	
	\begin{table}[htb]
		\caption{Summary of different network models}
		\label{tab:networks}
		\begin{tabular}{ |p{1.1cm}|p{7cm}|p{7cm}|}
			\hline
			\multicolumn{1}{|c|}{\textbf{Network}} & \multicolumn{1}{c|}{\textbf{Description}} & \multicolumn{1}{c|}{\textbf{Insights}} \\ \hline
			\textbf{AFCN} & Node: airports, Edge: two airports have an edge between them if there is at least one flight between them, Edge weight: number of flights between the airports. & AFCN helps to analyse the airline schedules. The analysis of AFCN gives us busiest airports and airport pairs, which have highest traffic between them. \\ \hline
			\textbf{AFDN} & Similar to AFCN but here we consider only delayed flights. & AFDN helps to analyse the airline historical operational data. It helps to identify the airports and airport pairs which caused the biggest disruptions. \\ \hline
			\textbf{FCCN} & Node: flights, Edge: shared flight connection, Weight: frequency of connection & FCCN helps to analyse the airline schedules. It gives us flights and flight connections which have potential to cause disruption to several others. \\ \hline
			\textbf{FCDN} & It is similar to FCCN except that it considers only delayed flights as nodes and flight connections with delay propagation only. & FCDN helps to analyse the airline historical operational data. It helps to identify flights and flight connections which caused the biggest disruptions in the network. \\ \hline
			\textbf{MLCN} & Multilayer Network adds extra value to the network analysis by considering the types of connectivity in the network. MLCN are similar to FCCN except that each layer considers connections of one type. & MLCN helps to analyse the airline schedules. It helps to identify the connection type which have potential to cause the biggest disruptions in the network. \\ \hline
			\textbf{MLDN} & MLDN are similar to FCDN except that each layer considers connections of one type. & MLDN helps to analyse the airline historical operational data. It helps to identify the connection type which caused the biggest disruptions in the network. \\ \hline
		\end{tabular}
	\end{table}
	
	\subsection{Network science techniques and insights}
	\label{subsec_properties}
	In this subsection, we have discussed different network properties and techniques used to analyse the airline network, modelled in the previous subsection, along with the insights provided by them about the problem, the summary of which can be found in the Table~\ref{tab_properties}.
	
	\textbf{Network:} Networks are used to represent data with non-linear relationships and can be defined as a set of nodes $V$ and set of links $E$, where nodes represent some objects and links capture some kind of relationship between those objects, e.g., in Airport-Flight network, airports are the nodes and two nodes have a link between them if flights operate between them. Networks can be, mainly, of two types directed and undirected. When there is some sense of direction associated with links in network then network is called directed otherwise called undirected, e.g., flights take off from origin airport to destination airport so Airport-Flight network is a directed network.
	
	\textbf{Degree Distribution:} In a network, degree of a node is number of other nodes to which it is connected in undirected network. For directed network, we have in degree and out degree, where in degree of a node is the number of nodes with incoming connections, and out degree of a node is the number of out going connections from the node to other nodes. Degree distribution represents the probability distribution of the degree over the whole network and it is useful to give structural insights about the network.
	
	\textbf{Centrality:} As the name suggests, it helps to identify the central elements of the network depending on their impact on other elements. Thus, this concept helps to identify the flights, airports and connections which have large impact on other flights, in terms of delay propagation. Here, we study out degree centrality and betweenness centrality. The first helps to identify the important nodes based on their out degree, i.e., flights which lead delay propagation to others. And second helps to identify important nodes and edges based on flow of information through them, i.e., identify flights and connections through which delays propagate. The betweenness centrality of a node $v$ is given below:
	\begin{equation}
		c_B \left( v \right) = \sum_{s,t\in V} \frac{\sigma\left(s,t|v\right)}{\sigma\left(s,t\right)},
	\end{equation}
	where $V$ is the set of vertices, $\sigma\left(s,t \right)$ is number of shortest paths between all pairs of nodes $\left(s,t\right)$ and $\sigma\left(s,t|v\right)$ is number of shortest paths between all pairs $\left(s,t\right)$ through node $v$. And the edge betweenness centrality of an edge $e$ is given below:
	\begin{equation}
		c_B \left( e \right) = \sum_{s,t\in V} \frac{\sigma\left(s,t|e\right)}{\sigma\left(s,t\right)},
	\end{equation}
	where $\sigma\left(s,t|e\right)$ is number of shortest paths between all pairs of nodes $\left(s,t\right)$ through edge $e$. The out degree centrality of node $v$ is given below as:
	\begin{equation}
		c_D \left( v \right) =  \frac{o(v)}{n},
	\end{equation}
	%	\[c_D \left( v \right) =  \frac{o}{n},\]
	where $o(v)$ is the out degree of the node $v$ and $n$ is number of nodes. Generally, degree centrality is normalised by dividing the value by maximum possible value, i.e., $n-1$.
	
	\textbf{Percolation:} It is a study of removal of nodes/edges of a network and their effect on the connectivity of the network. In airline scenario, when applied to a delay network for node percolation, it tells us that to which flights delay should be avoided to avoid big disruptions. It can be seen as dynamic betweenness where we calculate betweenness of the network and remove the node/edge with highest value, then again betweenness is calculated for the network and then node/edge with highest value is removed, and so on.
	
	\textbf{Diameter:} Diameter of a connected network is defined as the largest shortest distance between any two nodes. Diameter of a disconnected network is infinite because some nodes can't be reached. This concept tells us about the minimum number of flights required for the propagation of the delays from any part of network to any other part. Smaller the value of the diameter, better is the air connectivity but greater are the chances of disruptions because it would require less flights for the propagation of delays to the whole network.
	
	\textbf{Small-world effect:} When one can reach from any node of the network to any other node in the network in few steps, then network is said to exhibit small-world effect. In an airline network, if small-world effect occurs in the network that means network is susceptible to big disruptions because delays can propagate to the entire network with few flight delays. This effect can be measured using the diameter of the network so if the network has small diameter that means the network exhibits small-world effect.
	
	%	\textbf{Assortativity:} This property of a graph measures the tendency of similar nodes to connect together. This gives us the idea weather different disruptions tend to effect each other. Moreover it can be used to compare different layers of multilayer network. %to find the bottleneck layer.
	
	%	\textbf{Clustering Coefficient:}  This is used to measure the degree to which nodes in a network tend to cluster. This can also be used to compare different layers of multilayer network.
	%	
	%	\textbf{Size of component:} It helps to find the number of flights involved in the largest disruption and can be used to compare different layers of multilayer network.
	
	\textbf{Density:} It can be defined as the ratio of number of links in the network to the total number of possible links in the network. It can be good metrics to compare the different layers of the multilayer flight connection network. The density of a network is given as:
	\begin{equation}
		d = \frac{m}{n(n-1)},
	\end{equation}
	%	\[d = \frac{m}{n(n-1)},\]
	where \(m\) is the number of edges in the network. The density of the network is 0 if it has no edges and 1 if it is a complete network.
	
	\begin{table}[htb!]
		\caption{Network science techniques/properties and insights drawn from them}
		\label{tab_properties}
		\begin{tabular}{ |p{3cm}|p{7cm}|p{6cm}|}
			\hline
			\multicolumn{1}{|c|}{\textbf{Property}} & \multicolumn{1}{c|}{\textbf{Definition}} & \multicolumn{1}{c|}{\textbf{Insights}} \\ \hline
			\textbf{Network} & Network is a set of nodes and set of links where nodes represent some objects and links capture some kind of relationship between those objects. & It is used to represent and analyse data with non-linear relationships.  \\ \hline
			\textbf{Degree distribution} & Degree distribution represents the probability distribution of the degree over the whole network. & It provides us the structural insights about the network.  \\ \hline
			\textbf{Centrality} & It helps to identify the central elements of the network depending on their impact on other elements. &  This concept helps to identify the most influential flights, airports and connections which have large impact on other flights, in terms of delay propagation.\\ \hline
			\textbf{Out degree Centrality} & It helps to identify the important nodes based on their out degree. &  This concept helps to identify the flights and airports which have large impact on other flights.\\ \hline
			\textbf{Node Betweenness Centrality} &  It helps to identify important nodes based on flow of information through them. &  This concept helps to identify influential flights and airports through which highest delays either propagated or has potential for that.\\ \hline
			\textbf{Edge Betweenness Centrality} & It  helps to identify important edges based on flow of information through them. &  This concept helps to identify flight connections and airport-pairs through which highest delays either propagated or has potential for that.\\ \hline
			\textbf{Percolation} &It is a study of removal of nodes/edges of a network and their effect on the connectivity of the network. & In airline scenario, when applied to a delay network for node percolation, it tells us that to which flights delay should be avoided to avoid big disruptions.   \\ \hline
			\textbf{Diameter} & The diameter of a connected network is defined as the largest shortest distance between any two nodes and the diameter of a disconnected network is infinite.  & This concept tell us about the minimum number of flights required for the propagation of the delays from any part of network to any other part.  \\ \hline
			\textbf{Small-world effect} & When one can reach from any node of the network to any other node in the network in few steps, then network is said to exhibit small-world effect.  & In an airline network, if small-world effect occurs in the network that means network is susceptible to big disruptions because delays can propagate to the entire network with few flight delays.  \\ \hline
			%			\textbf{Assortativity} &This property of a graph measures the tendency of similar nodes to connect together.   & This gives us the idea weather different disruptions tend to effect each other. Moreover it can be used to compare different layers of multilayer network to find the bottleneck layer. \\ \hline
			%			\textbf{Clustering coefficient} & This is used to measure the degree to which nodes in a network tend to cluster. & This can also be used to compare different layers of multilayer network.  \\ \hline
			%			\textbf{Size of component} & It helps to find the number of flights involved in the largest disruption. &It can be used to compared different layers of multilayer network. \\ \hline
			\textbf{Density} & It is defined as the ratio of number of links in the network to the total number of possible links in the network. &  This is a good metrics to compare the different layers of the multilayer flight connection network. \\ \hline
			%		\textbf{Transitivity} &  &  \\ \hline
		\end{tabular}
	\end{table}
	
	\section{Case Study}
	\label{sec_analysis}
	In this section, we discuss about the data used in the analysis and results of analysis with different types of networks to which data has been modelled, namely, Airport-Flight, Flight-Connection and multilayer flight connection networks. Each of the network has two variations, connection and delay networks, as discussed in Sec.~\ref{sec_network_science}, and we have used out degrees to discuss results because in degrees are similar to out degrees for airline networks and out degrees can be used to track the propagation of disruptions/delays from one airport/flight to other airports/flights. Analysis of connection networks tell us about the potential disruptive elements and that of delay networks tell about the actual disruptive elements of the network. There arises many cases from the results of these two analysis which should be interpreted carefully, as discussed below: 
	\begin{enumerate}
		\item[(a)] If there is no overlap between potential disruptive elements and actual disruptive elements then that means the schedule of the airline has no major issue and in order to improve the operations of the airline, we should consider the actual disruptive elements in order and should think of ways to avoid delays to them. 
		\item[(b)] If there are some overlaps between disruptive elements from connection and delay networks then that means there could be some issues with the schedule of the airline itself, and in order to improve the operations of the airline, we have to first focus on the overlaps between the two in the order from delay networks. 
		\item[(c)] If there are overlaps not only in potential and actual disruptive elements of one criterion figure (say disruptive elements using out degree centrality) but across different criteria figures, such as disruptive elements using percolation and disruptive elements using betweenness centrality, then those disruptive elements are considered to have the highest priority and should be attended first to improve the airline operations.
		\item[(d)] Another case with second highest priority is the case when disruptive elements are common in the delay networks of different criteria but not in connection networks.
	\end{enumerate}
	\subsection{Data}
	\label{subsec_data}
	We have taken the case study of an airline with hub-and-spoke topology (which is way of connecting nodes in a network with most of the nodes are connected to one central node and other nodes have few connections among them) -- as confirmed by the results -- whose name is not revealed on request of the airline. Data received from the airline for a period of six months which includes flight details, such as aircraft used, origin and destination airports, scheduled and actual departure and arrival times, minimum ground times, few anonymised details about passenger, like number of passengers of different travelling classes, and crew connection details, like crew ids, shared between connecting flights, as given in the Table~\ref{table_data}. Flight delays are calculated as difference between scheduled and actual times of the flight and delays are said to be propagated from one flight to other if first flight was delayed and led to the delay of second flight and propagated delay is equal to arrival delay of incoming first flight minus slack time, where slack time is time difference between departure of second flight (connecting flight) and arrival of first flight minus minimum ground time to prepare for the second flight (for mathematical derivation, please refer to \cite{Ledwoch2022}. The data is anonymised and all the flight numbers and airport names are assigned names as 1, 2, 3,... and A1, A2, A3,..., respectively.
	
	\begin{table}[htb]
		\caption{Airline Data Description}
		\label{table_data}
		\begin{tabular}{ll} \hline
			\textbf{Name} & \textbf{Description} \\ \hline
			Flight Number & This is number assigned to each flight which might not be unique \\
			Origin Airport & This is the airport from where flight takes off \\
			Departure Airport & This is the airport where flight lands \\
			STA & This is scheduled time of arrival of flight \\
			ATA & This is actual time of arrival of flight \\
			STD & This is scheduled time of departure of flight \\
			ATD & This is actual time of departure of flight \\
			MGT & This is minimum amount of time a flight need to get ready for next flight \\
			Passengers & Number of passengers travelling in different classes for the connecting flight \\
			Crew ID & This is unique (anonymised) id associated with each crew member \\
			Crew flights & Flights served by a particular crew member\\ \hline
		\end{tabular}
	\end{table}
% 	The airline had total 95,581 flights with 1249 unique flights using 251 aircrafts covering 143 airports over a period of six months. Out of which 42,557 (without industry standard, i.e, a flight is delayed if it is delayed by even a single minute), i.e., 44.5\% were delayed. Flight Connection Connection Network has a total of 93,491 tail connections, 57,413 crew connections and 933,027 passenger connections, whereas, Flight Connection Delay Network, has a total of 11,346 tail connections, 3,034 crew connections and 22,373 passenger connections.
	
	%%	, with 1240 (without industry standard), i.e., 99.3\% unique flights faced delay and delay propagation. The flights covered 143 airports.\\
	
	
	
	%	connection net flights connections
	%	total tail connections added: 93491
	%	total crew connections added: 57413
	%	total pax connections added: 933027
	%	most frequent  100.20:20.A31.A1 : 181
	%	Total Connections: 1083931.0
	%	Total Unique Flights: 1249
	%	Total tails/aircraft: 251
	%	Total airport: 143
	
	%	delay_net_flights_connections
	%	total delated tails connections added: 11346
	%	total delayed crew connections added: 3034
	%	total delayed pax connections added: 22373
	%	most delayed flight: 156.06:15.A1.A44 : 166
	%	self.unique_del_flight_count 1200
	%	Total Connections with Delay Propagation: 36737.0
	%	Total unique tails/aircraft delayed: 249
	
	%%	multi_layer_connection_net_flights_connections
	%	Total Unique Flights: 1249
	%	Total Connections in Tail: 93491.0
	%	Total Connections in Crew: 57413.0
	%	Total Connections in Pax: 933027.0
	%
	%%	multi_layer_delay_net_flights_connections
	%	Total Connections with Delay Prop: 36737.0
	%	Total Connections in Tail with Delay Prop: 11330.0
	%	Total Connections in Crew with Delay Prop: 3034.0
	%	Total Connections in Pax with Delay Prop: 22373.0
	
	
	
	\subsection{Airport-Flight Network}
	\label{subsec_connection}
	The diameter of AFCN network is four that means it will take minimum of four flights to propagate delay to the whole network of the airline, i.e., AFCN exhibits small-world effect which means one can reach from node to any other node of the network with few edges (\cite{Newman2018}). The results for analysis of Airport-Flight network (AFN) are represented in Figs.~\ref{subfig_AF_CN}--\ref{subfig_BF_B_AF}. Interestingly, disruptive elements in connection network are exactly same as disruptive elements in delay network of AFN, unlike Flight-Connection networks (shown in next subsection), so for AFN we have shown the results using single figure for each category. This is mainly because AFN represents a high level view of an airline where it studies the disruptive airports and disruptive airport pairs over a period of six months, and does not go to flight level.
	
	\begin{figure}[htb]
		\centering
		\includegraphics[width=0.5\linewidth]{AF_net}
		\caption{Airport-Flight Network out-degree distributions}
		\label{subfig_AF_CN}
	\end{figure}
	Fig.~\ref{subfig_AF_CN} represents the out degree distribution for the network. This is same for the connection network and for the delay network because if some airport has out going flights to four other airports then even the delay network will have connections to those airports because there is great probability that at least one flight to those four airports will get late in six months. As it is clear from the figure, the degree distribution is quite different than real life networks which follow power law (\cite{Cong2016}). This is because the network has one airport with large number of connections to other airports and most of the airports are connected to few other airports, which means the network topology is hub-and-spoke and there is one hub airport through which other airports are connected.

	\begin{figure*}[htb]
		\centering
		\begin{subfigure}[t]{0.5\textwidth}			
			\centering
			\includegraphics[width=\linewidth]{BA_D_AFN.pdf}
			%			\caption{Airport-Flight connection network}
			%			\label{subfig_BA_D_AFN}
		\end{subfigure}%
		%		~ 
		%		\begin{subfigure}[t]{0.5\textwidth}			
			%			\centering
			%			\includegraphics[width=\linewidth]{BA_D_AFDN}
			%			\caption{Airport-Flight delay network}
			%			\label{subfig_BA_D_AFDN}
			%		\end{subfigure}
		\caption{Disruptive airports using out degree centrality in AFN}
		\label{fig_BA_D_AFN}
	\end{figure*}
	
	\begin{figure*}[htb]
		\centering
		\begin{subfigure}[t]{0.5\textwidth}			
			\centering
			\includegraphics[width=\linewidth]{BA_B_AFN.pdf}
			%			\caption{Airport-Flight connection network}
			%			\label{subfig_BA_B_AFN}
		\end{subfigure}%
		%		~ 
		%		\begin{subfigure}[t]{0.5\textwidth}
			%			\centering
			%			\includegraphics[width=\linewidth]{BA_B_AFDN}
			%			\caption{Airport-Flight delay network}
			%			\label{subfig_BA_B_AFDN}
			%		\end{subfigure}
		\caption{Disruptive airports using betweenness centrality in AFN}
		\label{subfig_BA_B_AF}
	\end{figure*}
	
	Fig.~\ref{fig_BA_D_AFN} represents disruptive airports using out-degree centrality. From the figure, it is clear that A1 is the central airport, i.e., many connecting flights take off from A1, and there is huge difference, in terms of air traffic, with other airports so A1 is the hub airport and the network follows a hub-and-spoke topology, as already indicated by the degree distribution. Fig.~\ref{subfig_BA_B_AF} represents disruptive airports according to betweenness centrality and presents top airports with high traffic flow and delay propagation through them. Once again, A1 airport has highest traffic and delays because most of flights pass through it. Fig.~\ref{subfig_BF_B_AF} represents disruptive airport pairs with highest traffic flow and delay propagation through them. Most of the airports contain A1 as one of the airport because A1 is the hub airport. Moreover, top three disruptive airport pairs have flights from A1 to A106, A76 and A98, but the difference is not much between different pairs.
	\begin{figure*}[htb!]
		\centering
		\begin{subfigure}[t]{0.5\textwidth}
			\centering
			\includegraphics[width=\linewidth]{BF_B_AFN.pdf}
			%			\caption{Airport-Flight connection network}
			%			\label{subfig_BF_B_AFN}
		\end{subfigure}%
		%		~ 
		%		\begin{subfigure}[t]{0.5\textwidth}
			%			\centering
			%			\includegraphics[width=\linewidth]{BF_B_AFDN}
			%			\caption{Airport-Flight delay network}
			%			\label{subfig_BF_B_AFDN}
			%		\end{subfigure}
		\caption{Disruptive airport pairs using betweenness centrality in AFN}
		\label{subfig_BF_B_AF}
	\end{figure*}
	
	Clearly, the hub airport A1 has the highest traffic and also causes highest number of delay propagations so need the most attention to improve the operations of the airline. Moreover, there are overlaps among results represented by figures from \ref{fig_BA_D_AFN}--\ref{subfig_BF_B_AF}, e.g., A98 and A55 are common to all of them, so these airports need second highest attention to improve the operations of the airline.
	
	
	\subsection{Flight-Connection Network}
	\label{subsec_FCN}
	The results for Flight-Connection network (FCN) are represented in Figs.~\ref{subfig_FC}--\ref{subfig_BF_P_FC}. Fig.~\ref{subfig_FC} represents the degree distribution and like Airport-Flight network, the distribution is quite different from real life problems with large number of peaks at tail for both connection and delay networks. The degree distribution for the connection network shows that most of flights can influence only one-two flights and the degree distribution for the delay network shows that most of the delays were propagated to one or two flights, i.e, the schedules of the airline are quite robust. We can see some peaks in the tail of the degree distribution, for the connection network which shows that some flights can influence a huge number of flights because they have connections to several other flights, and for the delay network it shows that in some cases delays were actually propagated to large number of flights.
	\begin{figure}[htb!]
		\centering
		\includegraphics[width=0.5\linewidth]{FC_net}
		\caption{FCN out-degree distributions}
		\label{subfig_FC}
	\end{figure}
	\begin{figure*}[htb!]
		\centering
		\begin{subfigure}[t]{0.5\textwidth}
			\centering
			\includegraphics[width=\linewidth]{BF_D_FCN.pdf}
			\caption{Connection network}
			\label{subfig_BF_D_FCN}
		\end{subfigure}%
		~ 
		\begin{subfigure}[t]{0.5\textwidth}
			\centering
			\includegraphics[width=\linewidth]{BF_D_FCDN.pdf}
			\caption{Delay network}
			\label{subfig_BF_D_FCDN}
		\end{subfigure}
		\caption{Disruptive flights using out degree centrality in FCN}
		\label{subfig_BF_D_FC}
	\end{figure*}
	
	Fig.~\ref{subfig_BF_D_FC} represents top disruptive flights using out degree centrality, where Sub-figure~\ref{subfig_BF_D_FCN} shows the flights which can influence a large number of flights because as per the schedule they have many following connections, and Sub-figure~\ref{subfig_BF_D_FCDN} shows the flights which actually lead to delay propagation to a large number of flights. There are some overlaps between Sub-figure~\ref{subfig_BF_D_FCN} and Sub-figure~\ref{subfig_BF_D_FCDN}, e.g., flight number 451 and 457, which means these flights were expected to cause delay propagation as per the schedule and they actually lead to delay propagations.
	
	\begin{figure*}[htb!]
		\centering
		\begin{subfigure}[t]{0.5\textwidth}
			\centering
			\includegraphics[width=\linewidth]{BF_B_FCN.pdf}
			\caption{Connection network}
			\label{subfig_BF_B_FCN}
		\end{subfigure}%
		~ 
		\begin{subfigure}[t]{0.5\textwidth}
			\centering
			\includegraphics[width=\linewidth]{BF_B_FCDN.pdf}
			\caption{Delay network}
			\label{subfig_BF_B_FCDN}
		\end{subfigure}
		\caption{Disruptive flights using the node-betweenness centrality in FCN}
		\label{subfig_BF_B_FC}
	\end{figure*}
	
	Fig.~\ref{subfig_BF_B_FC} represents disruptive flights as per the node-betweenness centrality, where Sub-figure~\ref{subfig_BF_B_FCN} shows flights with high connectivity through them and making them potential disruptive elements because if anything goes wrong with them, they can influence several other flights, and Sub-figure~\ref{subfig_BF_B_FCDN} shows top flights through which delays were propagated to large number of flights. Flight numbers like 179, 169 and 14 etc., are common in Sub-figure~\ref{subfig_BF_B_FCN} and Sub-figure~\ref{subfig_BF_B_FCDN}, which means that as per the airline schedule they were potential disruptive elements and also found actual disruptive elements when the schedule was executed so these flights need special attention to improve operations of the airline.
	\begin{figure*}[htb!]
		\centering
		\begin{subfigure}[t]{0.5\textwidth}
			\centering
			\includegraphics[width=\linewidth]{BC_B_FCN.pdf}
			\caption{Connection network}
			\label{subfig_BC_B_FCN}
		\end{subfigure}%
		~ 
		\begin{subfigure}[t]{0.5\textwidth}
			\centering
			\includegraphics[width=\linewidth]{BC_B_FCDN.pdf}
			\caption{Delay network}
			\label{subfig_BC_B_FCDN}
		\end{subfigure}
		\caption{Disruptive connections using edge-betweenness centrality in FCN}
		\label{subfig_BC_B_FC}
	\end{figure*}
	
	Fig.~\ref{subfig_BC_B_FC} represents the disruptive flight connections using the edge-betweenness centrality, where Sub-figure~\ref{subfig_BC_B_FCN} represents flight pairs, i.e., connections through which there is highest traffic flow and are the potential disruptive elements because delay propagation on these connections can cause delay propagation to several other flights, and Sub-figure~\ref{subfig_BC_B_FCDN} represents connections which were delayed and lead to the highest delay propagation to other flights. Connections (438,439), (446,447) and (260,261) etc. are common to potential and actual disruptive elements so need attention to improve the airline operations. Labels on the axis are shortened by keeping only the flight number to keep the figures clean.
	\begin{figure*}[htb!]
		\centering
		\begin{subfigure}[t]{0.5\textwidth}
			\centering
			\includegraphics[width=\linewidth]{BF_P_FCN.pdf}
			\caption{Connection network}
			\label{subfig_BF_P_FCN}
		\end{subfigure}%
		~ 
		\begin{subfigure}[t]{0.5\textwidth}
			\centering
			\includegraphics[width=\linewidth]{BF_P_FCDN.pdf}
			\caption{Delay network}
			\label{subfig_BF_P_FCDN}
		\end{subfigure}
		\caption{Disruptive flights using percolation in FCN}
		\label{subfig_BF_P_FC}
	\end{figure*}
	
	Percolation is an interesting phenomenon in network science and it can be very useful in delay networks to avoid big disruptions (see Table~\ref{tab_properties}). We have applied node percolation to delay network and the result are represented in Fig.~\ref{subfig_BF_P_FC}. It tells us that which flights should be avoid delay so as to avoid the delay propagation to become a big disruption. It can be seen as a dynamic betweenness centrality, i.e., removing the most influential flight using betweenness centrality scores and then again calculate the centrality scores and remove the flight with highest score and so on. This gives better results than betweenness centrality because when a flight is removed from the network, the connectivity of the remaining flights changes so second highest flight in first calculation of centrality may not be same after calculating the new centralities. It is to be noted that the order of flights represents the flights as per their importance with the most influential flight on the left hand side but their centralities are not in order. This is because centralities are calculated on networks of different sizes due to the removal of flights. Flight numbers 14, 169 and 100 etc. are potential and actual disruptive elements and need attention.
	
	From all these results for FCN, it is clear that there are some overlaps in disruptive elements of the connection network and delay network, which means as per the schedule some flights were expected to cause delays due to their tight schedules, and those flights actually lead to delay propagation to other flights. So, the airlines should give attention to overlapped disruptive elements to improve their operations. In addition to that, some flights are common to different results, e.g., flight number 14 is an actual disruptive element from Figs.~\ref{subfig_BF_D_FC} to \ref{subfig_BF_P_FC}, such flights need the most attention to improve the operations. Moreover, we notice that not all potential disruptive flights are actually disruptive elements, e.g., flight number 390 in Fig.~\ref{subfig_BF_D_FCN}. This could be due to multiple reasons, for example flights might have sufficient slack time between the connections so the delays from the incoming delayed flights are absorbed by slack time or there could be alternate crew and aircraft arrangements to tackle the late connections. It is difficult to tell what exactly happened with the flight at the time of operations from the available data because airlines use different recovery strategies. But from the data, we observe that flight number 390, had large number of potential connections, most of which were passenger connections and had sufficient slack times, and of these only few lead to actual delay propagations so flight 390 is not an actual disruptive element.
	
	
	\subsection{Multilayer Network}
	\label{subsec_MLN}
	The results for the multilayer flight connection network are represented in Fig.~\ref{subfig_MLNet} and Table~\ref{tab_mln_results}. Fig.~\ref{subfig_MLNet} represents the degree distribution for different layers of the multilayer connection and delay networks using box plots. From sub-figure~\ref{subfig_MLN_CN} and \ref{subfig_MLN_DN}, it is clear that there is greater variability in the mean of passenger connections as well as larger outliers for both connection and delay networks, which means passenger connections are not only potential causes for big disruptions but also the actual cause for the biggest disruptions in the airline network. Tail connections are second major disruptive elements and crew connections do not lead to big disruptions. So, passenger connections is the disruptive connection type (potential as well as actual) in the airline network.
	
	%	\begin{figure*}[htb!]
		%		\centering
		%		\begin{subfigure}[t]{0.5\textwidth}
			%			\centering
			%			\includegraphics[width=\linewidth]{MLN_CN_distri}
			%			\caption{Connection network}
			%			\label{subfig_MLN_CN_distri}
			%		\end{subfigure}%
		%		~ 
		%		\begin{subfigure}[t]{0.5\textwidth}
			%			\centering
			%			\includegraphics[width=\linewidth]{MLN_DN_distri}
			%			\caption{Delay network}
			%			\label{subfig_MLN_DN_distri}
			%		\end{subfigure}
		%		\caption{Multilayer Network out-degree distributions}
		%		\label{subfig_MLN}
		%	\end{figure*}
	
	\begin{figure*}[htb]
		\centering
		\begin{subfigure}[t]{0.5\textwidth}
			\centering
			\includegraphics[width=\linewidth]{MLN_CN}
			\caption{Connection network}
			\label{subfig_MLN_CN}
		\end{subfigure}%
		~ 
		\begin{subfigure}[t]{0.5\textwidth}
			\centering
			\includegraphics[width=\linewidth]{MLN_DN}
			\caption{Delay network}
			\label{subfig_MLN_DN}
		\end{subfigure}
		\caption{multilayer flight connection network out-degree distributions}
		\label{subfig_MLNet}
	\end{figure*}
	
	\begin{table}[htb]
		\centering
		\caption{Comparative study of different connection types of multilayer flight connection network}
		\label{tab_mln_results}
		{\small \begin{tabular}{|l|l|l|l|l|l|}
				\hline
				\multicolumn{2}{|c|}{\textbf{Property}}                                           & \multicolumn{1}{c|}{\textbf{all}} & \multicolumn{1}{c|}{\textbf{tail}} & \multicolumn{1}{c|}{\textbf{crew}} & \multicolumn{1}{c|}{\textbf{pax}} \\ \hline
				%				\multirow{2}{*}{Transitivity}                                                 & CN & 0.000268              & 0.000520                                  & 0                                  & 0               \\ \cline{2-6} 
				%				& DN & 9.584e-06             & 0               & 0              & 0             \\ \hline
				%				\multirow{2}{*}{\begin{tabular}[c]{@{}l@{}}Average\\ clustering\end{tabular}} & CN & 0.007546             & 0.006591                              & 0                                  & 0               \\ \cline{2-6} 
				%				& DN & 0.001328              & 0.001398               & 0              & 0             \\ \hline
				\multirow{2}{*}{Density}                                                      & CN & 0.063292               & 0.013347               & 0.007182               & 0.053296              \\ \cline{2-6} 
				& DN & 0.005596               & 0.000986                & 0.000182                & 0.004820                \\ \hline
				%				\multirow{2}{*}{Assortativity}                                                & CN & 0.044700               & 0.233557               & 0.132432                & -0.024617                \\ \cline{2-6} 
				%				& DN & 0.259977               & 0.330013                & -0.075751                & 0.037767              \\ \hline
		\end{tabular}}
	\end{table}
	
	Table~\ref{tab_mln_results} compares the network properties for different connection types of the network. From the table, it is clear that the density of passenger connections in connection network is more that the crew and tail connections, which means airline has more passenger connections than the crew and tail connections. Similarly, density of passenger connection is more than crew and tail in the delay network, which means that most of the delays are propagated through the passenger connections, making passenger connection the most disruptive connection type for the airline, as already pointed by the degree distribution, and thus airline need to pay attention to passenger connections to avoid delay propagations.
	
	\section{Conclusion and discussions}
	\label{sec_conclusion}
	We proposed to model schedules and historical operational data of an airline as airport-flight network, flight-connection network and multilayer flight connection networks, each with two variants as connection and delay networks. We applied network science concepts and techniques like centrality, percolation and multilayer networks to find the disruptive airports, flights, flight-connections and conection-type. Connection networks model the schedules of an airline as networks and provide potential disruptive elements, whereas delay networks model the historical operational data and provide actual disruptive elements.
	
	The proposed idea is validated with a case study of an airline and observed an interesting point that there is some overlap between potential and actual disruptive elements, which is an indication that there are some issues with the airline schedules itself for causing delay propagations. The airline network has small-world effect with diameter four, which means delays can propagate from any part of the network to any other part with minimum of just four flight delays.
	
	This analysis helps an airline in two ways, first it helps to evaluate the schedules and improve it by looking for overlapped disruptive elements in connection and delay networks, which are an indication of issues with the schedule. Secondly, it helps an airline to look at the historical operational data and to identify the culprits for big disruptions by looking for the disruptive elements and paying attention to them in the future to avoid the same disruptions again. Airline can deal with the disruptive elements in different ways, e.g., it can introduce extra slack time for the disruptive passenger connections, arrange extra crew for the disruptive crew connections, or arrange extra aircraft for the disruptive tail connections etc. The hub airport also need special attention as it is the busiest and one of the most disruptive element. Similarly, for the disruptive flights, there should be sufficient slack time or extra arrangements of crew or tail, depending on the delay codes of the disruptive element. Thus, the analysis provides a very good tool to improve the operations of the airline by properly handling the disruptive elements.
	
	The present disruptive element analysis presents a static view of an airline operations over a period of time, which is helpful for the airline to look back, analyse its performance and take decisions to improve operations. But the airline schedule can be further improved and can be made more robust to disruptions by closely analysing the operations of the airline over hour to hour basis. Thus, possible extension of the work is dynamic analysis of the disruptive elements in the airline networks, to further improve the operations and will be studied in the future extension of the work.
	
	\section*{Acknowledgement}
	This work is part of a BOEING project `Airline Performance and Disruption Management Across Extended Networks (APEMEN)' funded with research grant number 46599.
	
	%% BibTeX users please use one of
	%	\bibliographystyle{plain}      % basic style, author-year citations
	%	\bibliographystyle{spmpsci}      % mathematics and physical sciences
	%	\bibliographystyle{spphys}       % APS-like style for physics
%	\bibliography{APEMEN}   % name your BibTeX data base
	\begin{thebibliography}{48}
		\providecommand{\natexlab}[1]{#1}
		\providecommand{\url}[1]{\texttt{#1}}
		\expandafter\ifx\csname urlstyle\endcsname\relax
		\providecommand{\doi}[1]{doi: #1}\else
		\providecommand{\doi}{doi: \begingroup \urlstyle{rm}\Url}\fi
		
		\bibitem[Abdelghany et~al.(2004)Abdelghany, Shah, Raina, and
		Abdelghany]{Abdelghany2004}
		Khaled~F. Abdelghany, Sharmila~S. Shah, Sidhartha Raina, and Ahmed~F.
		Abdelghany.
		\newblock A model for projecting flight delays during irregular operation
		conditions.
		\newblock \emph{Journal of Air Transport Management}, 10\penalty0 (6):\penalty0
		385 -- 394, 2004.
		\newblock ISSN 0969-6997.
		
		\bibitem[AhmadBeygi et~al.(2008)AhmadBeygi, Cohn, Guan, and
		Belobaba]{AhmadBeygi2008}
		Shervin AhmadBeygi, Amy Cohn, Yihan Guan, and Peter Belobaba.
		\newblock Analysis of the potential for delay propagation in passenger airline
		networks.
		\newblock \emph{Journal of Air Transport Management}, 14\penalty0 (5):\penalty0
		221 -- 236, 2008.
		\newblock ISSN 0969-6997.
		
		\bibitem[Alderighi et~al.(2007)Alderighi, Cento, Nijkamp, and
		Rietveld]{Alderighi2007}
		Marco Alderighi, Alessandro Cento, Peter Nijkamp, and Piet Rietveld.
		\newblock Assessment of new hub‐and‐spoke and point‐to‐point airline
		network configurations.
		\newblock \emph{Transport Reviews}, 27\penalty0 (5):\penalty0 529--549, 2007.
		
		\bibitem[Baruah and Bharali(2019)]{Baruah2019}
		Dimpee Baruah and Ankur Bharali.
		\newblock Comparison of the airline networks of india with ani based on network
		parameters.
		\newblock In Bhabesh Deka, Pradipta Maji, Sushmita Mitra, Dhruba~Kumar
		Bhattacharyya, Prabin~Kumar Bora, and Sankar~Kumar Pal, editors,
		\emph{Pattern Recognition and Machine Intelligence}, pages 74--81, Cham,
		2019. Springer International Publishing.
		\newblock ISBN 978-3-030-34869-4.
		
		\bibitem[Brandes et~al.(2013)Brandes, Robins, McCranie, and
		Wasserman]{Brandes2013}
		Ulrik Brandes, Garry Robins, Ann McCranie, and Stanley Wasserman.
		\newblock What is network science?
		\newblock \emph{Network Science}, 1\penalty0 (1):\penalty0 1–15, 2013.
		\newblock \doi{10.1017/nws.2013.2}.
		
		\bibitem[Brueckner et~al.(2021)Brueckner, Czerny, and
		Gaggero]{brueckner2021airline}
		Jan~K Brueckner, Achim~I Czerny, and Alberto~A Gaggero.
		\newblock Airline mitigation of propagated delays via schedule buffers: Theory
		and empirics.
		\newblock \emph{Transportation Research Part E: Logistics and Transportation
			Review}, 150:\penalty0 102333, 2021.
		
		\bibitem[Brueckner et~al.(2022)Brueckner, Czerny, and
		Gaggero]{brueckner2022airline}
		Jan~K Brueckner, Achim~I Czerny, and Alberto~A Gaggero.
		\newblock Airline delay propagation: A simple method for measuring its extent
		and determinants.
		\newblock \emph{Transportation Research Part B: Methodological}, 162:\penalty0
		55--71, 2022.
		
		\bibitem[{Clark} et~al.(2018){Clark}, {Bhatia}, {Kodra}, and
		{Ganguly}]{Clark2018}
		K.~L. {Clark}, U.~{Bhatia}, E.~A. {Kodra}, and A.~R. {Ganguly}.
		\newblock Resilience of the u.s. national airspace system airport network.
		\newblock \emph{IEEE Transactions on Intelligent Transportation Systems},
		19\penalty0 (12):\penalty0 3785--3794, Dec 2018.
		\newblock ISSN 1558-0016.
		\newblock \doi{10.1109/TITS.2017.2784391}.
		
		\bibitem[Cong et~al.(2016)Cong, Hu, Dong, Wang, and Feng]{Cong2016}
		Wei Cong, Minghua Hu, Bin Dong, Yanjun Wang, and Cheng Feng.
		\newblock Empirical analysis of airport network and critical airports.
		\newblock \emph{Chinese Journal of Aeronautics}, 29\penalty0 (2):\penalty0 512
		-- 519, 2016.
		\newblock ISSN 1000-9361.
		
		\bibitem[Costa et~al.(2018)Costa, Bechara, Wehmuth, and Ziviani]{Costa2018}
		Bernardo Costa, Jo{\~{a}}o~Victor Bechara, Klaus Wehmuth, and Artur Ziviani.
		\newblock A multilayer and time-varying structural analysis of the brazilian
		air transportation network.
		\newblock \emph{CoRR}, abs/1709.03360, 2018.
		
		\bibitem[Couto et~al.(2015)Couto, Silva, Ruiz, and Benevenuto]{Couto2015}
		Guilherme~S Couto, Ana Paula Couto~DA Silva, Linnyer~B Ruiz, and Fabr{\'\i}cio
		Benevenuto.
		\newblock Structural properties of the brazilian air transportation network.
		\newblock \emph{Anais da Academia Brasileira de Ci{\^e}ncias}, 87\penalty0
		(3):\penalty0 1653--1674, 2015.
		
		\bibitem[Cumelles et~al.(2021)Cumelles, Lordan, and
		Sallan]{cumelles2021cascading}
		Joel Cumelles, Oriol Lordan, and Jose~M Sallan.
		\newblock Cascading failures in airport networks.
		\newblock \emph{Journal of Air Transport Management}, 92:\penalty0 102026,
		2021.
		
		\bibitem[Du et~al.(2018)Du, Zhang, Zhang, Cao, and Zhang]{Du2018}
		Wen-Bo Du, Ming-Yuan Zhang, Yu~Zhang, Xian-Bin Cao, and Jun Zhang.
		\newblock Delay causality network in air transport systems.
		\newblock \emph{Transportation Research Part E: Logistics and Transportation
			Review}, 118:\penalty0 466 -- 476, 2018.
		\newblock ISSN 1366-5545.
		
		\bibitem[Du et~al.(2017)Du, Liang, Yan, Lordan, and Cao]{Du2017}
		Wenbo Du, Boyuan Liang, Gang Yan, Oriol Lordan, and Xianbin Cao.
		\newblock Identifying vital edges in chinese air route network via memetic
		algorithm.
		\newblock \emph{Chinese Journal of Aeronautics}, 30\penalty0 (1):\penalty0 330
		-- 336, 2017.
		\newblock ISSN 1000-9361.
		
		\bibitem[Gershkoff(2016)]{Gershkoff2016}
		Ira Gershkoff.
		\newblock Shaping the future of airline disruption management (irops).
		\newblock \emph{Tech. rept.}, 2016.
		
		\bibitem[Giannikas et~al.(2022)Giannikas, Ledwoch, Stojković, Costas,
		Brintrup, Al-Ali, Chauhan, and McFarlane]{Ledwoch2022}
		Vaggelis Giannikas, Anna Ledwoch, Goran Stojković, Pablo Costas, Alexandra
		Brintrup, Ahmed Ali~Saeed Al-Ali, Vinod~Kumar Chauhan, and Duncan McFarlane.
		\newblock A data-driven method to assess the causes and impact of delay
		propagation in air transportation systems.
		\newblock \emph{Transportation Research Part C: Emerging Technologies},
		143:\penalty0 103862, 2022.
		\newblock ISSN 0968-090X.
		\newblock \doi{https://doi.org/10.1016/j.trc.2022.103862}.
		
		\bibitem[Gopalakrishnan and Balakrishnan(2017)]{Gopalakrishnan2017}
		Karthik Gopalakrishnan and Hamsa Balakrishnan.
		\newblock A comparative analysis of models for predicting delays in air traffic
		networks.
		\newblock ATM Seminar, 2017.
		
		\bibitem[{Gui} et~al.(2020){Gui}, {Liu}, {Sun}, {Yang}, {Zhou}, and
		{Zhao}]{Gui2020}
		G.~{Gui}, F.~{Liu}, J.~{Sun}, J.~{Yang}, Z.~{Zhou}, and D.~{Zhao}.
		\newblock Flight delay prediction based on aviation big data and machine
		learning.
		\newblock \emph{IEEE Transactions on Vehicular Technology}, 69\penalty0
		(1):\penalty0 140--150, 2020.
		
		\bibitem[Hong et~al.(2016)Hong, Zhang, Cao, and Du]{Hong2016}
		Chen Hong, Jun Zhang, Xian-Bin Cao, and Wen-Bo Du.
		\newblock Structural properties of the chinese air transportation multilayer
		network.
		\newblock \emph{Chaos, Solitons \& Fractals}, 86:\penalty0 28 -- 34, 2016.
		\newblock ISSN 0960-0779.
		
		\bibitem[Hossain and Alam(2017)]{Hossain2017}
		Md.~Murad Hossain and Sameer Alam.
		\newblock A complex network approach towards modeling and analysis of the
		australian airport network.
		\newblock \emph{Journal of Air Transport Management}, 60:\penalty0 1 -- 9,
		2017.
		\newblock ISSN 0969-6997.
		
		\bibitem[Jani{\'c}(2005)]{Jani2005}
		Milan Jani{\'c}.
		\newblock Modeling the large scale disruptions of an airline network.
		\newblock volume 131, pages 249--260, 2005.
		
		\bibitem[Jia et~al.(2014)Jia, Qin, and Shan]{Jia2014}
		Tao Jia, Kun Qin, and Jie Shan.
		\newblock An exploratory analysis on the evolution of the us airport network.
		\newblock \emph{Physica A: Statistical Mechanics and its Applications},
		413:\penalty0 266 -- 279, 2014.
		\newblock ISSN 0378-4371.
		
		\bibitem[Jiang et~al.(2017{\natexlab{a}})Jiang, Han, Zhang, and Li]{Jiang2017b}
		J.~Jiang, J.~H. Han, R.~Zhang, and W.~Li.
		\newblock The transition point of the chinese multilayer air transportation
		networks.
		\newblock \emph{International Journal of Modern Physics B}, 31\penalty0
		(26):\penalty0 1750186, 2017{\natexlab{a}}.
		
		\bibitem[Jiang et~al.(2017{\natexlab{b}})Jiang, Yao, Wang, Feng, and
		Kong]{Jiang2017}
		Yonglei Jiang, Baozhen Yao, Lu~Wang, Tao Feng, and Lu~Kong.
		\newblock Evolution trends of the network structure of spring airlines in
		china: A temporal and spatial analysis.
		\newblock \emph{Journal of Air Transport Management}, 60:\penalty0 18 -- 30,
		2017{\natexlab{b}}.
		\newblock ISSN 0969-6997.
		
		\bibitem[Jiang et~al.(2019)Jiang, Zeng, Liu, and Ma]{Jiang2019}
		Zhong-Yuan Jiang, Yong Zeng, Zhi-Hong Liu, and Jian-Feng Ma.
		\newblock Identifying critical nodes’ group in complex networks.
		\newblock \emph{Physica A: Statistical Mechanics and its Applications},
		514:\penalty0 121 -- 132, 2019.
		\newblock ISSN 0378-4371.
		
		\bibitem[Jin et~al.(2019)Jin, Wei, Xiu, Song, and Yang]{Jin2019}
		Ying Jin, Ye~Wei, Chunliang Xiu, Wei Song, and Kaixian Yang.
		\newblock Study on structural characteristics of china’s passenger airline
		network based on network motifs analysis.
		\newblock \emph{Sustainability}, 11\penalty0 (9):\penalty0 2484, 2019.
		
		\bibitem[Kivelä et~al.(2014)Kivelä, Arenas, Barthelemy, Gleeson, Moreno, and
		Porter]{Mikko2014}
		Mikko Kivelä, Alex Arenas, Marc Barthelemy, James~P. Gleeson, Yamir Moreno,
		and Mason~A. Porter.
		\newblock {Multilayer networks}.
		\newblock \emph{Journal of Complex Networks}, 2\penalty0 (3):\penalty0
		203--271, 07 2014.
		\newblock ISSN 2051-1310.
		
		\bibitem[Li et~al.(2018)Li, Zhang, and Deng]{Li2018}
		Meizhu Li, Qi~Zhang, and Yong Deng.
		\newblock Evidential identification of influential nodes in network of
		networks.
		\newblock \emph{Chaos, Solitons \& Fractals}, 117:\penalty0 283 -- 296, 2018.
		\newblock ISSN 0960-0779.
		
		\bibitem[Liu et~al.(2008)Liu, Cao, and Ma]{Liu2008}
		Yu-Jie Liu, Wei-Dong Cao, and Song Ma.
		\newblock Estimation of arrival flight delay and delay propagation in a busy
		hub-airport.
		\newblock In \emph{2008 Fourth International Conference on Natural
			Computation}, volume~4, pages 500--505. IEEE, 2008.
		
		\bibitem[Lordan and Sallan(2017)]{Lordan2017}
		Oriol Lordan and Jose~M. Sallan.
		\newblock Analyzing the multilevel structure of the european airport network.
		\newblock \emph{Chinese Journal of Aeronautics}, 30\penalty0 (2):\penalty0 554
		-- 560, 2017.
		\newblock ISSN 1000-9361.
		
		\bibitem[Lordan and Sallan(2019)]{Lordan2019}
		Oriol Lordan and Jose~M. Sallan.
		\newblock Core and critical cities of global region airport networks.
		\newblock \emph{Physica A: Statistical Mechanics and its Applications},
		513:\penalty0 724 -- 733, 2019.
		\newblock ISSN 0378-4371.
		
		\bibitem[Lordan et~al.(2014)Lordan, Sallan, and Simo]{Lordan2014}
		Oriol Lordan, Jose~M. Sallan, and Pep Simo.
		\newblock Study of the topology and robustness of airline route networks from
		the complex network approach: a survey and research agenda.
		\newblock \emph{Journal of Transport Geography}, 37:\penalty0 112 -- 120, 2014.
		\newblock ISSN 0966-6923.
		
		\bibitem[Lordan et~al.(2015)Lordan, Sallan, Simo, and
		Gonzalez-Prieto]{Lordan2015}
		Oriol Lordan, Jose~M. Sallan, Pep Simo, and David Gonzalez-Prieto.
		\newblock Robustness of airline alliance route networks.
		\newblock \emph{Communications in Nonlinear Science and Numerical Simulation},
		22\penalty0 (1):\penalty0 587 -- 595, 2015.
		\newblock ISSN 1007-5704.
		
		\bibitem[Newman(2018)]{Newman2018}
		Mark Newman.
		\newblock \emph{Networks}.
		\newblock Oxford university press, 2018.
		
		\bibitem[Olariaga and Zea(2018)]{Olariaga2018}
		Oscar~Díaz Olariaga and José~F. Zea.
		\newblock Influence of the liberalization of the air transport industry on
		configuration of the traffic in the airport network.
		\newblock \emph{Transportation Research Procedia}, 33:\penalty0 43 -- 50, 2018.
		\newblock ISSN 2352-1465.
		\newblock XIII Conference on Transport Engineering, CIT2018.
		
		\bibitem[Reggiani et~al.(2010)Reggiani, Nijkamp, and Cento]{Reggiani2010}
		Aura Reggiani, Peter Nijkamp, and Alessandro Cento.
		\newblock Connectivity and concentration in airline networks: a complexity
		analysis of lufthansa's network.
		\newblock \emph{European Journal of Information Systems}, 19\penalty0
		(4):\penalty0 449--461, 2010.
		
		\bibitem[Sathanur et~al.(2019)Sathanur, Sripimonwan, Halappanavar, Chatterjee,
		Ganguly, and Clark]{Sathanur2019}
		Arun~V Sathanur, Brandon Sripimonwan, Mahantesh Halappanavar, Samrat
		Chatterjee, Auroop Ganguly, and Kevin Clark.
		\newblock Identification of critical airports from the perspective of delay and
		disruption propagation in air travel networks.
		\newblock \emph{2019 IEEE International Symposium on Technologies for Homeland
			Security}, 2019.
		
		\bibitem[Song and Yeo(2017)]{Song2017}
		Min~Geun Song and Gi~Tae Yeo.
		\newblock Analysis of the air transport network characteristics of major
		airports.
		\newblock \emph{The Asian Journal of Shipping and Logistics}, 33\penalty0
		(3):\penalty0 117 -- 125, 2017.
		\newblock ISSN 2092-5212.
		
		\bibitem[{Thiagarajan} et~al.(2017){Thiagarajan}, {Srinivasan}, {Sharma},
		{Sreekanthan}, and {Vijayaraghavan}]{Thiagarajan2017}
		B.~{Thiagarajan}, L.~{Srinivasan}, A.~V. {Sharma}, D.~{Sreekanthan}, and
		V.~{Vijayaraghavan}.
		\newblock A machine learning approach for prediction of on-time performance of
		flights.
		\newblock In \emph{2017 IEEE/AIAA 36th Digital Avionics Systems Conference
			(DASC)}, pages 1--6, 2017.
		
		\bibitem[Walker(2017)]{Walker2017}
		C~Walker.
		\newblock Coda digest.
		\newblock \emph{Tech. rept.}, 2017.
		
		\bibitem[Wandelt et~al.(2019)Wandelt, Sun, and Zhang]{Wandelt2019}
		Sebastian Wandelt, Xiaoqian Sun, and Jun Zhang.
		\newblock Evolution of domestic airport networks: a review and comparative
		analysis.
		\newblock \emph{Transportmetrica B: Transport Dynamics}, 7\penalty0
		(1):\penalty0 1--17, 2019.
		
		\bibitem[{Wang} et~al.(2003){Wang}, {Schaefer}, and {Wojcik}]{Wang2003}
		P.~T.~R. {Wang}, L.~A. {Schaefer}, and L.~A. {Wojcik}.
		\newblock Flight connections and their impacts on delay propagation.
		\newblock In \emph{Digital Avionics Systems Conference, 2003. DASC '03. The
			22nd}, volume~1, pages 5.B.4--5.1--9 vol.1, 2003.
		
		\bibitem[Wang et~al.(2017)Wang, Xu, Hu, and Zhan]{Wang2017}
		Yanjun Wang, Xinhua Xu, Minghua Hu, and Jianming Zhan.
		\newblock The structure and dynamics of the multilayer air transport system.
		\newblock 2017.
		
		\bibitem[Wang et~al.(2019)Wang, ZHAN, XU, LI, CHEN, and HANSEN]{Wang2019}
		Yanjun Wang, Jianming ZHAN, Xinhua XU, Lishuai LI, Ping CHEN, and Mark HANSEN.
		\newblock Measuring the resilience of an airport network.
		\newblock \emph{Chinese Journal of Aeronautics}, 32\penalty0 (12):\penalty0
		2694 -- 2705, 2019.
		\newblock ISSN 1000-9361.
		
		\bibitem[Wong and Tsai(2012)]{Wong2012}
		Jinn-Tsai Wong and Shy-Chang Tsai.
		\newblock A survival model for flight delay propagation.
		\newblock \emph{Journal of Air Transport Management}, 23:\penalty0 5 -- 11,
		2012.
		\newblock ISSN 0969-6997.
		
		\bibitem[{Wu} et~al.(2019){Wu}, {Cai}, {Yan}, and {Li}]{Wu2019}
		W.~{Wu}, K.~{Cai}, Y.~{Yan}, and Y.~{Li}.
		\newblock An improved svm model for flight delay prediction.
		\newblock In \emph{2019 IEEE/AIAA 38th Digital Avionics Systems Conference
			(DASC)}, pages 1--6, 2019.
		
		\bibitem[Xu et~al.(2005)Xu, Donohue, Laskey, and Chen]{Xu2005}
		Ning Xu, George Donohue, Kathryn~Blackmond Laskey, and Chun-Hung Chen.
		\newblock Estimation of delay propagation in the national aviation system using
		bayesian networks.
		\newblock In \emph{6th USA/Europe Air Traffic Management Research and
			Development Seminar}, 2005.
		
		\bibitem[Yang et~al.(2019)Yang, Xu, and Wu]{Yang2019}
		Yong Yang, Kaijun Xu, and Jiayi Wu.
		\newblock Empirical structural analysis on chinese airline network.
		\newblock In Xinguo Zhang, editor, \emph{The Proceedings of the 2018
			Asia-Pacific International Symposium on Aerospace Technology (APISAT 2018)},
		pages 2757--2764, Singapore, 2019. Springer Singapore.
		\newblock ISBN 978-981-13-3305-7.
		
	\end{thebibliography}
	
\end{document}