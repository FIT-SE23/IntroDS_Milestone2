\documentclass{article}

\def\neurips{0}
\def\supplemental{0} % to be used with neurips=0
\def\tpdp{0} % to be used with neurips = 0 


\ifnum\neurips=1
    % ready for submission
    %\usepackage{neurips_2021}
    
    % to compile a preprint version, e.g., for submission to arXiv, add add the
    % [preprint] option:
         \usepackage[final]{neurips_2021}
    
    % to compile a camera-ready version, add the [final] option, e.g.:
    %     \usepackage[final]{neurips_2021}
    
    % to avoid loading the natbib package, add option nonatbib:
    %    \usepackage[nonatbib]{neurips_2021}
\else 
\fi

\definecolor{myblue}  {RGB}{3,122,235}
\definecolor{mypurple}{RGB}{176,095,183}
\definecolor{myorange}{RGB}{252,128,8}
\definecolor{mygreen} {RGB}{0,143,0}
\definecolor{myred}   {RGB}{231,091,093}
\definecolor{mymaroon}   {RGB}{175,012,035}
\definecolor{mygray}  {RGB}{234,234,241}
\definecolor{nugray}  {RGB}{220,220,227}

\definecolor{mydarkgray}  {RGB}{80,80,80}

\newcommand{\TODO}[1]{\textbf{{\color{todocolor}TODO: #1}}}

\newcommand{\figref}[1]{\Cref{fig:#1}}
\newcommand{\secref}[1]{\Cref{sec:#1}}
\newcommand{\subsecref}[1]{\S\,\ref{sec:#1}}
\newcommand{\eqnref}[1]{Eq.~\ref{eqn:#1}}
\newcommand{\eqnsref}[2]{Eqs.~\ref{eqn:#1} and~\ref{eqn:#2}}
\newcommand{\eqnrref}[2]{Eqs.~\ref{eqn:#1}--\ref{eqn:#2}}
\newcommand{\defref}[1]{Definition~\ref{def:#1}}
\newcommand{\tabref}[1]{\Cref{tab:#1}}
\newcommand{\HIDE}[1]{}

\newcommand{\pluseq}{\mathrel{+}=}
\newcommand{\asteq}{\mathrel{*}=}

\newcommand\defeq{\mathrel{\overset{\makebox[0pt]{\mbox{\normalfont\scriptsize\sffamily def}}}{=}}}

\newcommand\code[1]{\lstinline[mathescape=true,basicstyle=\ttfamily\normalsize]|#1|}
\newcommand\codesmall[1]{\lstinline[mathescape=true,basicstyle=\ttfamily\footnotesize]|#1|}
\newcommand{\spmv}{\texttt{SpMV}\xspace}

\algnewcommand{\LineComment}[1]{\State // #1}

\newcommand{\olivia}[1]{{\color{purple} {\bf Olivia:} #1}}

% Commands for BNF notation.
\newcommand{\bnfdef}{\mathrel{::=}}
\newcommand{\bnfalt}{\mathrel{\mid}}
\newcommand{\mT}{\mathcal{T}}


% Commands for coiteration algorithm 
\newcommand{\iterFn}{lowerIter}

\colorlet{emititerationcolor}{myblue}
\colorlet{emitmappingcolor} {mypurple}
\colorlet{emitassemblycolor} {mygreen}
\colorlet{emitcomputecolor}  {mygreen}
\newcommand\emititerationcolor[1]{\textcolor{emititerationcolor}{#1}}
\newcommand\emitmappingcolor[1]{\textcolor{emitmappingcolor}{#1}}
\newcommand\emitcomputecolor[1]{\textcolor{emitcomputecolor}{#1}}
\newcommand\emitassemblycolor[1]{\textcolor{emitassemblycolor}{#1}}

\colorlet{emititerationrefcolor}{emititerationcolor}
\colorlet{emitmappingrefcolor}  {emitmappingcolor}
\colorlet{emitcomputerefcolor}  {emitcomputecolor}
\colorlet{emitassemblyrefcolor} {emitassemblycolor}

\newcommand\emititerationref[1]{\hypersetup{linkcolor=emititerationrefcolor}\textcolor{emititerationrefcolor}{\secref{#1}}\hypersetup{linkcolor=black}}
\newcommand\emitmappingref[1]{\hypersetup{linkcolor=emitmappingrefcolor}\textcolor{emitmappingrefcolor}{\secref{#1}}\hypersetup{linkcolor=black}}
\newcommand\emitcomputeref[1]{\hypersetup{linkcolor=emitcomputerefcolor}\textcolor{emitcomputerefcolor}{\secref{#1}}\hypersetup{linkcolor=black}}
\newcommand\emitassemblyref[1]{\hypersetup{linkcolor=emitassemblyrefcolor}\textcolor{emitassemblyrefcolor}{\secref{#1}}\hypersetup{linkcolor=black}}

\newcommand\EmitPseudo[2]{
  \expandafter\newcommand\csname #1\endcsname{%
    \textbf{emit} #2
  }
}

\EmitPseudo{initIters}{initialize iterators}
\EmitPseudo{initIterMeta}{initialize iterator metadata}
\EmitPseudo{loopHeader}{loop header}
\EmitPseudo{loopFooter}{loop footer}
\EmitPseudo{accessIters}{access iterators}
\EmitPseudo{resolveCoord}{resolve the coordinate of $i$}
\EmitPseudo{locateLocators}{locate from locators }
\EmitPseudo{condHeader}{conditional header}
\EmitPseudo{condFooter}{conditional footer}
\EmitPseudo{advanceIters}{advance iterators}

\EmitPseudo{mapTo}{map candidate coordinates to the original space}
\EmitPseudo{mapFrom}{map resolved coordinate to each derived space}

\newcommand{\ldotspack}{.\hskip-.5ex.\hskip-.5ex.}
%\newcommand{\ldotspack}{...}
\EmitPseudo{denseIter}{\texttt{Foreach or Reduce(\ldotspack{}=> i\ldotspack{})}}
\EmitPseudo{interIter}{\texttt{Foreach(Scan(\ldotspack{}or\ldotspack{}=> i\ldotspack{})}}
\EmitPseudo{unionIter}{\texttt{Foreach(Scan(\ldotspack{}and\ldotspack{}=> i\ldotspack{})}}
\EmitPseudo{sparseBVIter}{\texttt{Foreach(\ldotspack{}=> pos\ldotspack{})}}
\EmitPseudo{sparseIter}{\texttt{Foreach(Scan(\ldotspack{}=> i\ldotspack{})}}

\EmitPseudo{genBVtwo}{$\mathcal{B}_2=$ \Call{genBitvector}{$\mT_2$}}
\EmitPseudo{genBVone}{$\mathcal{B}_1=$ \Call{genBitvector}{$\mT_1$}}
\EmitPseudo{genBVResult}{scanner for result positions}

\EmitPseudo{computeCode}{compute code}
\EmitPseudo{assemblyCode}{assembly code}
\EmitPseudo{segmentInsert}{position insert code}

\algnewcommand\algorithmicswitch{\textbf{switch}}
\algnewcommand\algorithmiccase{\textbf{case}}
\algnewcommand\algorithmicdefault{\textbf{default}}

\algdef{SE}[SWITCH]{Switch}{EndSwitch}[1]{\algorithmicswitch\ #1\ \algorithmicdo}{\algorithmicend\ \algorithmicswitch}%
\algdef{SE}[CASE]{Case}{EndCase}[1]{\algorithmiccase\ #1}{\algorithmicend\ \algorithmiccase}%
\algdef{SE}[DEFAULT]{Default}{EndDefault}[1]{\algorithmicdefault\ }{\algorithmicend\ \algorithmicdefault}%
\algtext*{EndSwitch}%
\algtext*{EndCase}%
\algtext*{EndDefault}%

\algtext*{EndWhile}
\algtext*{EndIf}

\newcommand{\name}{Stardust\xspace}

\newcommand*\circled[1]{\tikz[baseline=(char.base)]{
            \node[shape=circle,draw,inner sep=2pt] (char) {#1};}}
            
\sisetup{detect-all}


\ifnum\neurips=1
    \usepackage[utf8]{inputenc} % allow utf-8 input
    \usepackage[T1]{fontenc}    % use 8-bit T1 fonts
    \usepackage{hyperref}       % hyperlinks
    \usepackage{url}            % simple URL typesetting
    \usepackage{booktabs}       % professional-quality tables
    \usepackage{amsfonts}       % blackboard math symbols
    \usepackage{nicefrac}       % compact symbols for 1/2, etc.
    \usepackage{microtype}      % microtypography
    \usepackage{xcolor}         % colors
\else 
\fi

\ifnum\neurips=0
\title{Differentially Private Sampling from Distributions \footnote{A preliminary version of this work appeared at NeurIPS 2021 \cite{RaskhodnikovaSSS21}} \\}
\fi

\ifnum\neurips=1
\title{Differentially Private Sampling from Distributions \\}
\fi

\ifnum\neurips=1
    \author{%
    Sofya Raskhodnikova \\
    %\thanks{Use footnote for providing further information
    %about author (webpage, alternative address)---\emph{not} for acknowledging
    %funding agencies.} \\
    Department of Computer Science\\
    Boston University\\
%    Boston, MA 02215 \\
    \texttt{sofya@bu.edu} \\
    % examples of more authors
    \And
    Satchit Sivakumar \\
    %\thanks{Use footnote for providing further information
    %about author (webpage, alternative address)---\emph{not} for acknowledging
    %funding agencies.} \\
    Department of Computer Science\\
    Boston University\\
%    Boston, MA 02215 \\
    \texttt{satchit@bu.edu} \\
    \And
    Adam Smith \\
    %\thanks{Use footnote for providing further information
    %about author (webpage, alternative address)---\emph{not} for acknowledging
    %funding agencies.} \\
    Department of Computer Science\\
    Boston University\\
%    Boston, MA 02215 \\
    \texttt{ads22@bu.edu} \\
    \And
    Marika Swanberg \\
    %\thanks{Use footnote for providing further information
    %about author (webpage, alternative address)---\emph{not} for acknowledging
    %funding agencies.} \\
    Department of Computer Science\\
    Boston University\\
%    Boston, MA 02215 \\
    \texttt{marikas@bu.edu} \\
    }
\else 
    \author{Sofya Raskhodnikova\thanks{Department of Computer Science, Boston University. \texttt{\{sofya,satchit,ads22,marikas\}@bu.edu}} \thanks{SR was supported in part by NSF award CCF-1909612.}  \and Satchit Sivakumar\footnotemark[2] \thanks{SS was supported in part by NSF award CNS-2046425, as well as Cooperative Agreement CB20ADR0160001 with the Census Bureau. The views expressed in this paper are those of the author and not those of the U.S. Census Bureau or any other sponsor.} \and Adam Smith\footnotemark[2] \thanks{AS and MS were supported in part by NSF awards CCF-1763786 and CNS-2120667, as well as faculty research awards from Google and Apple.} \and Marika Swanberg\footnotemark[2]~\footnotemark[5]}
\fi

\date{}

\begin{document}
\maketitle 

\begin{abstract}
    We initiate an investigation of  private sampling from distributions. Given a dataset with $n$ independent observations from an unknown distribution $P$, a sampling algorithm must output a single observation from a distribution that is close in total variation distance to $P$ while satisfying differential privacy. 
    Sampling abstracts the goal of generating small amounts of realistic-looking data.
    We provide tight upper and lower bounds for the dataset size needed for this task for three natural families of distributions: arbitrary distributions on $\{1,\ldots ,k\}$, arbitrary product distributions on $\{0,1\}^d$, and product distributions on $\{0,1\}^d$ with bias in each coordinate bounded away from 0 and 1. We demonstrate that, in some parameter regimes, private sampling requires asymptotically fewer observations %samples 
    than learning a description of $P$ nonprivately; in other regimes, however, private sampling proves to be as difficult as private learning. Notably, for some classes of distributions, the overhead in the number of observations needed for private learning compared to non-private learning is completely captured by the number of observations needed for private sampling.
\end{abstract}
%\newpage

 

\ifnum\neurips=0
\ifnum\tpdp=0
\setcounter{tocdepth}{2}
\setlength{\columnsep}{1cm}
\begin{multicols*}{2}
    {\small \tableofcontents}
\end{multicols*}
\newpage 
\fi
\fi


\ifnum\tpdp=0
\section{Introduction}\label{sec:intro}
\fi

\ifnum\tpdp=0
Statistical machine learning models trained on sensitive data are now widely deployed in domains such as education, finance, criminal justice, medicine, and public health. The personal data used to train such models are more detailed than ever, and 
there is a growing awareness of the privacy risks that come with their use. Differential privacy is a standard for confidentiality that is now well studied and increasingly deployed. 

Differentially private algorithms ensure %, roughly, %Too many ``roughly''s 
that whatever is learned about an individual from the algorithm's output would be roughly the same whether or not that individual's record was actually part of the input dataset. This requirement entails a strong guarantee, but limits the design of algorithms significantly. As a result, there 
\else
There
\fi 
is a substantial line of work on developing good differentially private methodology for learning and statistical estimation tasks, and on understanding how the sample size required for specific tasks increases relative to unconstrained algorithms.
%(\cite{KUblog20} provide a short introduction and survey). 
A typical task investigated in this area is \emph{distribution learning}: informally, given records drawn i.i.d.\  from an unknown distribution $\distr$, the algorithm aims to produce a description of a distribution $Q$ that is close to~$\distr$.


However, often the task at hand
requires much less than full-fledged learning. We may simply need to generate a small amount of data that has the same distributional properties as the population, or perhaps simply ``looks plausible'' for the population. For example, one might need realistic data for debugging a software program or for getting a quick idea of the range of values in a dataset.

In this work, we study a basic problem that captures these seemingly less stringent aims. 
Informally, the goal is to design a sampling algorithm that, given a dataset with $n$ observations drawn i.i.d.\ from a distribution $\distr$, generates a single observation from a distribution $Q$ that is close to~$\distr$. 

%Informally: given a dataset on $n$ observations drawn i.i.d.\ from a distribution $\distr$, an accurate sampling algorithm generates a single observation from a distribution $Q$ that is close to $\distr$. 

To formulate the problem precisely, consider a randomized algorithm $\sampler:\universe^n\to \universe$ that takes as input a dataset of $n$ records from some universe $\universe$ and outputs a single element of $\universe$. 
Given a distribution~$\distr$ on $\universe$, let $\Datarv=(\datarv_1,\dots,\datarv_n)$ be a random dataset with entries drawn i.i.d.\ from $\distr$. Let $\sampler(\Datarv)$ denote the random variable corresponding to the output of the algorithm $\sampler$ on input $\Datarv$. 
This random variable depends on both the selection of entries of $\Datarv$ from $\distr$ and the coins of $\sampler$. 
Let $\distroutput{\sampler, \distr}$ denote the distribution of $\sampler(\Datarv)$, so that
$$\distroutput{\sampler, \distr}(z) = \Pr_{\substack{\Datarv~\sim_{i.i.d.} \distr \\ \text{coins of \sampler}}}(\sampler(\Datarv)=z) = \sum_{\Datafixed \in \universe^n} \Bparen {\Pr(\Datarv = \Datafixed) \cdot \Pr_{\text{coins of \sampler}}(\sampler(\Datafixed)=z) }\, .$$
The dataset size $n$ is a parameter of $\sampler$ and thus defined implicitly.%
\ifnum\neurips=0
\footnote{We sometimes consider datasets whose size is randomly distributed. In such cases, the random variable $\sampler(\Datarv)$ (and its distribution $\distroutput{\sampler, \distr}$) incorporates randomness from the choice of the input size in addition to the selection of the data and the coins of $\sampler$.}%\asnote{Maybe the point about randomly sized inputs can be made when we define Poisson samplers.} 
\fi 
~We measure the closeness between the input and output distributions in total variation distance, denoted $d_{TV}$.

\begin{definition}[Accuracy of sampling \cite{Axelrod0SV20}]
\label{def:acc1} A sampler $\sampler$ is {\em $\alpha$-accurate on a distribution} $\distr$ %on $\universe$ 
if 
\begin{equation*}
    d_{TV}(\distroutput{\sampler, \distr}, \distr) \leq \alpha.
\end{equation*}
A sampler is {\em $\alpha$-accurate on a class $\class$ of distributions} if it is $\alpha$-accurate on every $\distr$ in $\class$.
\end{definition}

% \begin{definition}[Accuracy of sampling~\cite{Axelrod0SV20}]
% \label{def:acc-intro} A sampler $\sampler$ is $\alpha$-accurate on a class $\class$ if for all distributions $\distr \in \class,$ 
% \begin{equation*}
%     d_{TV}(\distroutput{\sampler, \distr}, \distr) \leq \alpha.
% \end{equation*}
% \end{definition}

The class $\class$ in the accuracy definition above effectively encodes what the sampler is allowed to assume about $\distr$. For example, $\class$ might include all distributions on $k$ elements or all product distributions on $\bit{d}$ (that is, distributions on $d$-bit strings for which the individual binary entries are independent). 

Our aim in formulating Definition~\ref{def:acc1} was to capture the weakest reasonable 
task that abstracts the goal of generating data with the same distributional properties as the input data.
Without any privacy constraints, this task is trivial to achieve: an algorithm that simply outputs its first input record will sample from exactly the distribution $\distr$, so the interesting problem  is to generate a sample of size $m>n$ when given only $n$ observations (as in the work of Axelrod
%, Garg, Sharan, and Valiant 
et al.~\cite{Axelrod0SV20}, which inspired our investigation). However, a differentially private algorithm cannot, in general, simply output one of its records in the clear. Even producing a single correctly distributed sample is a non-trivial task. 

To understand the task and compare our results to existing work, it is helpful to contrast our definition of sampling with the more stringent goal of learning a
%the
distribution. A learning algorithm gets input in the same format as a sampling algorithm. We state a definition of distribution learning formulated 
%We state a formulation 
with a single parameter $\alpha$ that captures both the distance between distributions and the failure probability.

\begin{definition}[Distribution learning]
\label{def:learn-intro} An algorithm $\cB$
{\em learns 
a distribution class $\class$} to within error $\alpha$ if, given a dataset consisting of independent observations from a distribution $P\in\class$, algorithm $\cB$ outputs a description of a distribution  that satisfies
%its input distribution to within error $\alpha$ on a distribution class $\class$ if (a) for all inputs, $\cB$'s output is a description of a distribution  over $\universe$ and (b)~for all distributions $\distr \in \class,$ 
\begin{equation*} 
    \Pr_{\substack{\Datarv~\sim_{i.i.d.} \distr \\ \text{coins of \cB}}}\bparen{d_{TV}(\cB(\Datarv), \distr) \leq \alpha} \geq 1-\alpha.
\end{equation*}
\end{definition}

An algorithm $\cB$ that learns class $\class$ to within error $\alpha$ immediately yields a $2\alpha$-accurate sampler for class $\class$: 
%one may construct a sampler $\sampler$ that 
the sampler first runs $\cB$ to get a distribution $\hat P$ and then generates a single sample from~$\hat P$. 
Thus, it is instructive to compare bounds on the dataset size required for sampling to known results on the sample complexity of distribution learning.
Lower bounds for sampling imply lower bounds for all more stringent tasks, including learning, whereas separations between the complexity of sampling and that of learning suggest settings where weakening the learning objective might be productive.


% We consider sampling algorithms that get independent observations from a distribution $\distr$ on $\universe$ and output a single record in $\universe$. For a dataset $\Datafixed \in \universe^*$, we use $\datafixed_i$ to denote the $i^{th}$ record of $\Datafixed$. 
% Given a distribution $\distr$ on $\universe$ and size $n$, a random dataset $\Datarv$ is a random variable whose entries $\datarv_i$ are drawn i.i.d.\ from $\distr$. We sometimes consider datasets whose size is randomly distributed.
%
%The dataset size, or distribution on the dataset size, is a parameter of the algorithm $\sampler$. We let $\sampler(\Datarv)$ denote the random variable corresponding to the output of the algorithm $\sampler$ on input $\Datarv$. This random variable depends on the choice of dataset size (when it is random), the selection of entries of $\Datarv$ from $\distr$, and the coins of $\sampler$. Let $\distroutput{\sampler, \distr}$ denote the distribution of $\sampler(\Datarv)$.









% Along the way, we consider an even simpler task which often captures the difficulty of sampling. Namely, we abstract the task of generating ``plausible-looking" data as that of generating a single point that could arise with nonzero probability. 

% \begin{description}
% \item [Finding a point in the support:] Given records $X_1,...,X_n$ drawn i.i.d.\ from distribution $P$, output a value from the support of $P$ (i.e., a value that arises with nonzero probability under $P$). 
% \end{description}

% The difficulty of both of these tasks depends on the structure of the data—that is, the universe from which it is drawn—as well as what the algorithm knows ahead of time about the distribution $P$. 


\paragraph{Differential privacy} Differential privacy is a constraint on the algorithm $\sampler$ that processes a dataset. It does not rely on any distributional property of the data. We generally use uppercase letters (e.g., $\Datarv =(\datarv_1,...,\datarv_n)$) when viewing the data as a random variable, and lowercase symbols (e.g., $\Datafixed = (\datafixed_1,...,\datafixed_n)$) when treating the data as fixed. 
Two datasets $\Datafixed, \Datafixed'\in \universe^n$ are {\em neighbors} if they differ in at most one entry. If each entry  corresponds to the data of one person, then neighboring datasets differ by replacing one person's data with an alternate record. Informally, differential privacy requires that an algorithm's output distributions are similar on all pairs of neighboring datasets.

\begin{definition}[Differential Privacy~\cite{DworkMNS06j,DworkKMMN06}]\label{def:DP} A randomized algorithm $\sampler: \universe^n \rightarrow \mathcal{Y}$ is {\em $(\eps, \delta)$-differentially private} (in short, $(\eps,\delta)$-DP) if for every pair of neighboring datasets $\Datafixed, \Datafixed'\in \universe^n$ and for all subsets $Y\subseteq \mathcal{Y}$,
\begin{equation*}
    \Pr[\sampler(\Datafixed) \in Y] \leq e^\eps \cdot \Pr[\sampler(\Datafixed') \in Y] + \delta.
\end{equation*}
\end{definition}

\ifnum\tpdp=0 
For $\eps \leq 1$ and $\delta < 1/n$, this guarantee implies, roughly, that one can learn the same things about any given individual from the output of the algorithm as one could had their data not been used in the computation \cite{KasiviswanathanS14}. 
When $\delta=0$, the guarantee is referred to as pure differential privacy.
\ifnum\neurips=0
We generally consider the approximate case where $\delta>0$. 
In the analysis of some of the algorithms,
we also use a variant of the definition, \emph{zCDP}~\cite{bun2016concentrated} (see Definition~\ref{def:CDP}).
\fi 
\fi 


% \paragraph{Understanding the challenges}
%  To get a sense of why sampling behaves differently from learning, consider the simple setting of Bernoulli data.\asnote{I'm not sure this section sets up our results discussion in the right way. Remove or revisit.}
%  Suppose the algorithm gets $X_1,...,X_n \sim_{\text{i.i.d.}} \Ber(p)$ where the bias $p$ is unknown. Absent privacy constraints, we could simply output the first bit of the input and get an exactly correct sample. Equivalently, we could compute the sample proportion $\hat p=\frac {1}n \sum_i X_i$, and then generate a coin flip from $\Ber(\hat p)$. A natural differentially private analogue would compute a noisy proportion $\tilde p = \clip{\hat p + \Lap(\frac{1}{\eps n})}$, where $\clip{\ \cdot \ }$ denotes rounding an arbitrary real number to the nearest value in $[0,1]$, and then generate a value from $\Ber(\tilde p)$. 

% The algorithm is always differentially private, since releasing the noisy proportion $\tilde p$ is $(\eps,0)$-DP. What can we say about its accuracy? On one hand, because the noise it adds to the proportion has expected magnitude $\frac{1}{n\eps}$ and sampling from the proportion has zero error, the algorithm has error  $\alpha =O( \frac 1{n\eps})$ (alternatively, sample complexity $n = O(\frac1{\alpha\eps})$). On the other hand, this error analysis is loose when $p$ is bounded away from 0 or 1. For instance, for $p\in[\frac13, \frac 23]$, the same algorithm has exponentially small error $\alpha = \exp(-\Omega(\eps n))$ (since the noise it adds is unbiased, and the sample proportion is typically far enough from 0 or 1 for the rounding to have essentially no effect). Thus, the ``hardest'' parameters $p$ for this algorithm are those near 0 and 1. In fact, we show that in this range the algorithm's $\Theta(\frac1 {\eps \alpha})$ sample complexity cannot be improved by any differentially private algorithm (Theorem~\ref{thm:k-ary-lb}).

% How does this compare to the sample complexity of learning? The algorithm which simply releases the noisy proportion $\tilde p$ has optimal error among DP algorithms, requiring a sample complexity of  $n = \Theta(\frac{1}{\alpha^2} + \frac{1}{\alpha\eps})$ for the distribution $\Ber(\tilde p)$ to be within distance $\alpha$ of $\Ber(p)$ in the worst case over $p$. We can view this bound as the sum of two terms: the sample complexity of nonprivate learning ($n=\Theta\bparen{\frac 1 {\alpha^2}}$) plus a term to account for the privacy constraint ($\Theta\bparen{\frac 1 {\alpha \eps}}$). One interpretation of our lower bound is that \emph{the extra privacy term in the complexity of learning can be explained by the complexity of sampling, a seemingly much simpler task}. 

% A more complex picture arises when we take a closer look at the hard instances that give rise to particular lower bounds. The worst case parameter range for both DP and nonprivate learning is when $p$ is bounded away from 0 and 1; the lower bound is proven by showing that the distributions with $p_1 = \frac 1 2 - \alpha$ and $p_2=\frac 12 +\alpha$ are hard to tell apart for any algorithm that is given a sample of size  $n= o(\frac{1}{\alpha^2} + \frac{1}{\alpha\eps})$. Sampling is relatively easy in that range, requiring only $n=O\bparen{\frac 1 \eps + \log(\frac 1 \alpha)}$ (Theorem~\ref{thm:bernoulli-bb}). In other words, \textit{the regimes in which sampling is hard are those in which learning is easiest, and vice versa}.

% Thus, studying the complexity of generating even a single sample helps us understand the true source of difficulty for certain tasks and sheds light on when we might be able to engage in nontrivial statistical tasks even with very little data.

\ifnum\neurips=1
\subsection{Our results}
\else 
    \ifnum\tpdp=1
    \paragraph{Our Results}
    \else 
    \subsection{Our Results}
    \fi
\fi 
\label{sec:results}

We initiate a systematic investigation of the sample complexity of differentially private sampling. We provide upper and lower bounds for the  sample size needed for this task for three natural families of distributions: arbitrary distributions on $[k]=\{1,\ldots ,k\}$, arbitrary product distributions on $\{0,1\}^d$, and product distributions on $\{0,1\}^d$ with bounded bias. We demonstrate that, in some parameter regimes, private sampling requires asymptotically fewer samples than learning a description of $P$ nonprivately; in other regimes, however, sampling proves to be as difficult as private learning. 

% We formulate the problem of differential private sampling and provide the first systematic investigation of its complexity. We study two families of distributions: distributions on $k$ elements (with Bernoulli distributions as a special case) and products of Bernoulli distributions.

\begin{table}[t] 
    \begin{center} \small
     \caption{Sample complexity of sampling and estimation tasks. Our negative results hold for $(\eps,\delta)$-differential privacy when $\delta<1/n$. In this table, $\eps \leq 1$ and $\delta = 1/n^c$ for constant $c>1$.}
    \label{tab:results}
\begin{tabular}{ |c||c||c|c| } 
      \hline
 & Distributions on $[k]$ 
 & Product distributions 
 & Product distributions 
 \\
 & 
 & on $\bit{d}$
 & with $p_j \in [\frac 1 3 , \frac 2 3]$ 
 \\ 
 \hline
 {{\color{white} \rule{1pt}{11pt}} Nonprivate learning } & 
 $\Theta\paren{\dfrac{k}{\alpha^2}}$
 & \multicolumn{2}{|c|}{$\Theta\paren{ \dfrac{d}{\alpha^2}}$}
 \\ 
 \hline
 {{\color{white} \rule{1pt}{11pt}} $(\eps, \delta)$-DP sampling} 
 & $\Theta\paren{\dfrac{k}{\alpha \eps}}$ 
 & $\tilde \Theta\paren{\dfrac{d }{\alpha \eps}}$
 & $\tilde O \paren{\dfrac{\sqrt{d}}{\eps} + \log\dfrac{d}{\alpha}}$,
 \quad $\Omega\paren{\dfrac{\sqrt{d}}\eps}$ 
 %$O \paren{\dfrac{\sqrt{d\log(1/\delta)}}{\eps} + \log(\dfrac{d}{\alpha})}$
 
 \\
 (this work) & Theorems~\ref{thm:kupperb}, \ref{thm:intro-k-ary-lb} & Theorems~\ref{thm:bernoulli-product-alg-intro}, \ref{thm:bernoulli-product-lb-intro}  & Theorems~\ref{thm:intro-bernoulli-product-bb}, \ref{thm:intro-product-bb-lb}
 \\
 %\cline{3-4}
 %& &$\Omega(\frac{d}{\alpha \eps})$ \quad Theorem~\ref{thm:bernoulli-product-lb-intro} &{$ \Omega\paren{\sqrt{d}/\eps}$ \quad Theorem \ref{thm:intro-product-bb-lb}} \\
 \hline
 {{\color{white} \rule{1pt}{11pt}} $(\eps, \delta)$-DP learning} & $\Theta\paren{\dfrac{k}{\alpha^2} + \dfrac{k}{\alpha \eps}}$ \quad \cite{diakonikolas2015differentially} 
 & \multicolumn{2}{|c|}{$\tilde \Theta\paren{\dfrac{d }{\alpha \eps} + \dfrac{d}{\alpha^2}}$ \cite{KLSU19,BunKSW21J}}
 \\ 
 \hline
\end{tabular}
\end{center}
\end{table}


Our results are summarized in Table~\ref{tab:results}.
\ifnum\neurips=1
Proofs of all results are included in the supplementary material.
\fi
\ifnum\tpdp=1
Proofs of all results are included in the full version, which is forthcoming on arxiv.
\fi 
For simplicity, the table and our informal discussions focus  on $(\eps,\delta)$-differential privacy (Definition~\ref{def:DP})  in the setting\footnote{This setting precludes trivial solutions (which are  possible when $\delta =\Omega(1/n)$), but allows us to treat factors of $\log(1/\delta)$ as logarithmic in $n$ and absorb them in $\tilde O$ expressions.} where $\delta = 1/n^c$ for some constant $c>1$.




Let $\class_k$ be the class of $k$-ary distributions (that is, distributions on $[k]$). We show that  $n=\Theta\bparen{\frac{k}{\alpha \eps}}$ observations are necessary and sufficient for differentially private sampling from $\class_k$ in the worst case.

\begin{theorem}\label{thm:kupperb}
For all $k\geq 2,\eps>0$, and  $\alpha\in (0,1)$, there exists an $(\eps,0)$-DP sampler 
%of discrete distributions over the universe $[k]$ 
that is  $\alpha$-accurate on the distribution class $\class_k$ for datasets of size $n=O(\frac{k}{\alpha \eps})$.
\end{theorem}

\begin{theorem}\label{thm:intro-k-ary-lb}
For all  $k\geq 2$, $n \in \N$, $\alpha \in (0,\frac 1 {50}]$, $\eps  \in (0,1]$, and $\delta \in (0, \frac{1}{5000n}]$, 
%every
if there is an 
$(\eps,\delta)$-DP sampler  that is $\alpha$-accurate on the distribution class $\class_k$ 
on datasets of size $n$, then 
%requires dataset size
$n = \Omega\paren{ \frac{k}{\alpha\eps}}$.
% For all sufficiently small $\alpha>0$, $k \geq 2$, $\eps \in (0, 1]$, and $\delta \in \big[0, \frac 1{5000 n}\big]$, every $(\eps,\delta)$-differentially private sampler that is $\alpha$-accurate on
% the class $\class_k$
% needs datasets of size $n=\Omega(\frac{k}{\alpha\eps})$.
\end{theorem}


The second major class we consider consists of products of $d$ Bernoulli distributions, for $d\in \N$. We denote this class by $\cB^{\otimes d}$.  Each distribution in $\cB^{\otimes d}$ is described by a vector $(p_1,...,p_d) \in [0,1]^d$ of $d$ probabilities, called \emph{biases}; a single observation in $\bit{d}$ is generated by flipping $d$ independent coins with respective probabilities $p_1,...,p_d$ of heads. % We give an $(\eps,\delta)$-differentially private algorithm that is $\alpha$-accurate for an input dataset of size $n=\tilde O(\frac{d}{\alpha \eps})$. We give a lower bound of $n=\Omega(\sqrt{d}/\log(d))$ on the input size needed for sampling from such distributions. 
We show that  $n=\tilde\Theta\bparen{\frac{d}{\alpha \eps}}$ observations are necessary and sufficient for differentially private sampling from $\cB^{\otimes d}$ in the worst case.


\begin{theorem}\label{thm:bernoulli-product-alg-intro}
For all $d,n\in\N$ and $\eps,\delta,\alpha  \in (0,1)$, 
%$d\in\N,\eps \in (0,1]$ and $\alpha,\delta  \in (0,1)$, 
there exists an $(\eps, \delta)$-DP sampler %for the class of products of $d$ Bernoulli distributions 
that is $\alpha$-accurate on the distribution class $\cB^{\otimes d}$ for datasets of size $n=\tilde{O}\big(\frac{d}{\alpha\eps}\big)$, assuming $\log(1/\delta) = poly(\log n)$.
\end{theorem}

\begin{theorem}\label{thm:bernoulli-product-lb-intro}
For all sufficiently small $\alpha>0$, and
for all $d,n\in \N$, 
$\eps \in (0, 1]$, and $\delta \in \big[0, \frac 1{5000 n}\big]$,  
%\sr{every}
if there is an 
$(\eps,\delta)$-DP sampler that is $\alpha$-accurate on the distribution class $\cB^{\otimes d}$ on datasets of size $n$, then 
%requires dataset size
$n=\Omega(\frac{d}{\alpha \eps})$.
\end{theorem}

Finally, we give better samplers and matching lower bounds 
for  Bernoulli distributions and,  more generally, products of Bernoulli distributions, with bias bounded away from 0 and 1. For simplicity, we consider distributions with bias $p_j\in [\frac 1 3, \frac 2 3]$ in each coordinate $j\in[d]$.
For this class, we show that differentially private sampling can be performed with datasets of size roughly $\sqrt{d}/\eps$, significantly smaller than in the general case. Curiously, the accuracy parameter $\alpha$ has almost no effect on the sample complexity.
For Bernoulli distributions with bounded bias, we achieve this with pure differential privacy, that is, with $\delta =0$.  For products of Bernoulli distributions, we need $\delta>0$.


\begin{theorem}%[Upper bound, bounded biases]
\label{thm:intro-bernoulli-product-bb}
For all $d\in\N$ and $\eps,\delta,\alpha  \in (0,1)$,
%For all $\eps \in(0,1]$ and $\alpha,\delta\in (0,1)$, 
there exists an $(\eps,\delta)$-DP sampler that is  $\alpha$-accurate on the class %$\cBB^{\otimes d}$ 
of products of $d$ Bernoulli distributions with biases in $\big[\frac 13,\frac 23 \big]$  for datasets of size $n=O\Big(\frac {\sqrt{d\log(1/\delta)}}{\eps}   + \log \frac d {\alpha}
\Big)$.
%
When $d=1$, the sampler has $\delta=0$ and $n=O(\frac 1{\eps}+\log \frac 1 {\alpha})$. 
\end{theorem}


\begin{theorem}%[Lower bound, bounded biases]
\label{thm:intro-product-bb-lb} 
For all sufficiently small $\alpha>0$, 
and for all $d,n\in \N$, 
$\eps \in (0,1]$,  and $\delta \in [0, \frac 1{100n}]$, 
%every
if there exists an 
$(\eps,\delta)$-DP sampler that is $\alpha$-accurate on
the class %$\cBB^{\otimes d}$ 
of products of $d$ Bernoulli distributions with biases in $\big[\frac 13,\frac 23 \big]$ 
on datasets of size $n$, then  
%requires dataset size
$n={\Omega}(\sqrt{d}/\eps)$.
%
% For all  sufficiently small $\alpha>0$, 
% for all  $d\in \N$, $\eps \in (0,1]$,  and $\delta \leq \frac 1{100n}$, every $(\eps,\delta)$-differentially private sampler that is $\alpha$-accurate on
% the class of products of $d$ Bernoulli distributions with bias in $\big[\frac 13,\frac 23 \big]$ needs datasets of size ${\Omega}(\sqrt{d}/\eps)$.
\end{theorem}



\paragraph{Implications}

Our results show that the sample complexity of private sampling can differ substantially from that of private learning (for which known bounds are stated in Table~\ref{tab:results}). In some settings, sampling is much easier than learning: for example, for products of Bernoulli distributions with bounded biases, private sampling has a lower dependence on the dimension (specifically, $\sqrt{d}$ instead of $d$) and essentially no dependence on $\alpha$. Even for arbitrary biases or arbitrary $k$-ary distributions, private sampling is easier when $\alpha\ll\eps$. In other settings, however, private sampling can be as hard as private learning: e.g., for $\eps \leq \alpha$, the worst-case complexity of sampling and learning $k$-ary distributions and product distributions is the same. 

A more subtle point is that, in settings where private sampling is as hard as private learning, sampling accounts for the entire cost of privacy in learning. Specifically,  the optimal sample complexity of differentially private learning for arbitrary $k$-ary distributions is $n=\Theta\bparen{\frac{k}{\alpha^2} + \frac{k}{\alpha \eps}}$ (e.g., see \cite[Theorem 13]{AcharyaSZ21}). 
This bound is the sum of two terms: the sample complexity of nonprivate learning, $\Theta\bparen{\frac k {\alpha^2}}$, plus a term to account for the privacy constraint, $\Theta\bparen{\frac k {\alpha \eps}}$. One interpretation of our result that private sampling requires $n=\Theta\bparen{\frac k {\alpha \eps}}$ observations is that \emph{the extra privacy term in the complexity of learning can be explained by the complexity of privately generating a single sample with approximately correct distribution}. 
\ifnum\neurips=0
Analogously, for product distributions with arbitrary biases, the sample complexity of private learning is the sum of the sample complexity of nonprivate learning, $\Theta\paren{\frac{d}{\alpha^2}}$, and the extra privacy term, $\tilde \Theta\paren{\frac{d }{\alpha \eps}}$. %\asnote{I removed a tilde over the complexity of nonprivate learning. I also replaced parentheses with commas.}
Our results demonstrate that the latter term is the number of observations needed to produce one private sample, giving a new interpretation of the overhead in the number of samples stemming from the privacy constraint.
\fi
%Such a phenomenon may also occur for product distributions with arbitrary biases,  though our lower bounds are not currently strong enough to prove that (to establish that, they would have to match the $n=\tilde O\bparen{\frac d {\alpha \eps}}$ upper bound).

Another implication of our results is that, \emph{in some settings, the distributions that are hardest for learning---nonprivate or private---are the easiest for sampling, and vice versa}. Consider the simple case of Bernoulli distributions (i.e., product distributions with $d=1$). The ``hard'' instances for nonprivate learning to within error $\alpha$ are distributions with bias $p = \frac{1\pm\alpha}{2}$, but private sampling is easiest in that parameter regime. In contrast, the ``hard'' instances in our $\Omega(\frac{1}{\alpha\eps})$ lower bound for Bernoulli distributions have bias  $10\alpha$, that is, close to 0 as opposed to close to 1/2.
%shows that sampling requires $n=\Omega(\frac{1}{\alpha\eps})$ for distributions with bias $O(\alpha)$---very close to 0. 
A simple variance argument shows that nonprivate learning is easy in that parameter regime, requiring only $O(\frac1 \alpha)$ observations. Similarly, for product distributions, we show that the complexity of private sampling is only $\tilde \Theta(\sqrt d)$ when biases are bounded away from 0 and 1. For the same class, however, the complexity of private and nonprivate learning is $\Theta(d)$.


Our final point is
%it is interesting to observe 
that our lower bounds for $k$-ary distributions and general product distributions only require that the sampler generate a value \textit{in the support} of the distribution with high probability. They thus apply to a weaker problem, similar in spirit to the interior point problem that forms the basis of lower bounds for private learning of thresholds \cite{BunNSV15}. %\ssnote{An example here is useful, but interior point may not be the best fit since interior point is equivalent to private learning of thresholds. A better comparison may be private prediction vs private learning}.

Taken together, our results show that  studying the complexity of generating a single sample helps us understand the true source of difficulty for certain tasks and sheds light on when we might be able to engage in nontrivial statistical tasks with very little data.


% \as{How does this compare to the sample complexity of learning? For arbitrary $k$-ary distributions, the optimal sample complexity of DP learning is $n=\Theta\Bparen{\frac{k}{\alpha^2} + \frac{k}{\alpha \eps}}$ (e.g., see \cite[Theorem 13]{AcharyaSZ21}). 
% We can view this bound as the sum of two terms: the sample complexity of nonprivate learning ($n=\Theta\bparen{\frac 1 {\alpha^2}}$) plus a term to account for the privacy constraint ($\Theta\bparen{\frac 1 {\alpha \eps}}$). One interpretation of our lower bound is that \emph{the extra privacy term in the complexity of learning can be explained by the complexity of sampling, a seemingly much simpler task}. 
% }
 
\ifnum\tpdp=0

\paragraph{Open questions}
%Our results leave 
Our work raises many questions about the complexity of private sampling. First, our upper bounds achieve only the minimal goal of generating a single observation. In most settings, one would presumably want to generate several such samples. One can do so by repeating our algorithms several times on disjoint subsamples, but  in general this is not the best way to proceed. 
%
Second, we study only three classes of distributions. It is likely that the picture of what is possible for many classes is more complex and nuanced. It would be interesting, for example, to study private sampling for 
Gaussian distributions, since they demonstrate intriguing data/accuracy tradeoffs for nonprivate sampling~\cite{Axelrod0SV20}.
%In the nonprivate setting, Gaussian distributions demonstrate interesting data/accuracy tradeoffs for sampling~\cite{Axelrod0SV20}.
% Third, even for the settings we study, there remain gaps between our upper and lower bounds. Notably, for product distributions 
% \as{with arbitrary biases, is is not clear if the linear dependence on $d$ is necessary. \ssnote{No longer applicable} It is also not clear if $\delta=0$ is necessary---the upper bound for private learning can be realized with $\eps=0$ \cite{BunKSW19}.} \ssnote{Should the last line read $\delta = 0$ and not $\eps = 0$?}

% \paragraph{Societal impact} We study the feasibility of basic inference under privacy constraints. Our work is motivated by societal concerns, but focused on fundamental theoretical limits. We do not anticipate direct practical impact. \ssnote{Do we still need this paragraph?}

\ifnum\neurips=1
\subsection{An overview of our proofs and techniques}\label{intro:overview}
\else 
\subsection{An Overview of Our Proofs and Techniques}\label{intro:overview}
\fi 
\label{sec:techniques}

For both algorithms and lower bounds, our results require the development of new techniques. On the algorithmic side, we take advantage of the fact that sampling algorithms need only be correct on average over samples drawn from a given distribution. One useful observation that underlies  our positive results is that sampling based on an \emph{unbiased} estimate $\hat P$ of a probability distribution $P$ (in the sense that 
$\E[{\hat{P}(u)]=P(u)}$ for all elements $u$ in the universe $\universe$, where the expectation is taken over the randomness in the dataset and the coins of the algorithm)
%$\E{}\bparen{\hat P} = P$) 
has zero error, even though the learning error, e.g., $d_{TV}(\hat P,P)$ might be  large. For product distributions with bounded biases, we  also exploit the randomness of the sampling process itself to gain privacy without explicitly adding noise.
%additional distortion.

For negative results, we cannot generally use existing lower bounds for learning or estimation, because of a fundamental  obstacle. 
The basic framework used in proving most lower bounds on sample complexity of learning problems is based on \textit{identifiability}: to show that a large sample is required to learn class $\class$, one first finds a set of distributions $P_1,....,P_t$ in the class $\class$ that are far apart from each other and then shows that the output of any sufficiently accurate learning algorithm allows an outside observer to determine exactly which distribution in the  collection generated the data. The final step is to show that algorithms in a given family (say, differentially private algorithms with a certain sample size) cannot reliably identify the correct distribution.
%
This general approach is embodied in recipes such as Fano's, Assouad's, and Le Cam's methods from classical statistics (see, e.g., \cite{AcharyaSZ21} for a summary of these methods and their differentially private analogues).
For many sampling problems, the identifiability approach breaks down: a single observation is almost never enough to identify the originating distribution. 

One of our approaches to proving lower bounds is to leverage ways in which the algorithm's output directly signals a failure to sample correctly. For instance, our lower bound for $k$-ary distribution relies on the fact that an $\alpha$-accurate sampler must produce a value in the support of the true distribution $P$ with high probability. Another approach is to reduce from other distribution (sampling or estimation) problems. For example, our lower bound for product distributions with bounded biases is obtained via a reduction from an estimation problem, by observing that a small number of samples from a nearby distribution suffices for a very weak estimate of the underlying distribution's attribute biases.

\medskip

We break down our discussion of techniques according to the specific distribution classes we consider. 

\paragraph{Distributions on ${[k]}$} 
For the class of distributions on $[k]$, Theorem~\ref{thm:kupperb} shows that $\alpha$-accurate $(\eps,0)$-differently private sampling can be performed with a dataset of size  $O(\frac k {\alpha\eps})$. 
Our private sampler  computes, for each $j\in[k]$, the proportion $\hat P_j$ of its dataset that is equal to $j$, adds Laplace noise to each count, uses $L_1$ projection to obtain a valid vector of probabilities $\tilde P =(\tilde P_1,\dots,\tilde P_k)$, and finally outputs an element of $[k]$ sampled according to $\tilde P$.

Theorem~\ref{thm:intro-k-ary-lb} provides a matching lower bound on $n$ that holds for all $(\eps,\delta)$-differentially private algorithms with $\delta=o(1/n).$
We prove our lower bound separately for Bernoulli distributions and for discrete distributions with support size $k\geq 3$, using different analyses.
For Bernoulli distributions, we first exploit the group privacy property of  differentially private algorithms and the fact that the sampler must be accurate for the Bernoulli distribution $\Ber(0)$ to show that, on input with $t$ ones, a differentially private sampler outputs 1 with probability at most
$2\alpha e^{\eps t}.$ Then we consider $P=\Ber(10\alpha)$. We use $\alpha$-accuracy in conjunction with group privacy to give a lower and an upper bound on the probability of the output being~1 when the input is drawn i.i.d.\ from $P$. This allows us to relate the parameters involved in order to obtain the desired lower bound on~$n.$

The lower bound for distributions on $[k]$ with $k\geq 3$ is more involved. We start by identifying general properties of samplers that allow us to restrict our attention to relatively simple algorithms. First, we observe that every sampler can be converted to a {\em Poisson algorithm}, that is, an algorithm that, instead of receiving an input of a fixed size, gets an input with the number of records that follows a Poisson distribution.
This observation allows us to use a standard technique, Poissonization, that  
makes the empirical frequencies of different elements independent.
%helps break up dependencies between  random variables used in the analysis by  modifying a probability experiment to replace a fixed quantity with a variable quantity that follows a Poisson distribution. 
Next, we observe that privacy for samplers can be easily amplified, so that we can assume w.l.o.g.\ that $\eps$ is small. Finally, we observe that every sampler for 
\ifnum\neurips=1
the class of $k$-ary distributions
\else
a {\em label-invariant} class
\fi
can be converted to a {\em frequency-count-based algorithm}. 
\ifnum\neurips=0
A class $\class$ of distributions is {\em label-invariant} if, for every distribution $P\in\class$, every distribution obtained from $P$ by permuting the names of the elements in the support is also in $\class.$ 
\fi
A sampler is {\em frequency-count-based} if the probability it outputs a specific element depends only on the number of occurrences of this element in its input and the frequency counts\footnote{The vector of frequency counts is called a {\em fingerprint} or a {\em histogram} in previous work.
\ifnum\neurips=0
We avoid the first term to ensure there is no confusion with fingerprinting codes that are used in other lower bounds for differential privacy. In the literature on differential privacy, a histogram refers to the vector of frequencies of each element in the dataset, as opposed to the number of elements with the specified frequencies. To avoid confusion, we use the term ``frequency counts''.
\fi
} of the input (that is, the number of elements that occur zero times, once, twice, thrice, and so on).  Frequency-count-based algorithms have been studied for a long time in the context of understanding properties of distributions (see, e.g.,~\cite{batu2001testing, batu2000testing, raskhodnikova2009strong}).

Equipped with the three observations, we restrict our attention to Poisson, frequency-count-based algorithms, with small $\eps$ in the privacy guarantee. In contrast to our lower bound for Bernoulli samplers, we show that when the support size is at least 3 and the dataset size, $n$, is too small, the sampler is likely to output an element outside of the support of the input distribution $\distr.$ Here, we exploit group privacy, which implies that the probability that a sampler outputs a specific element which appears $j$ times in its input differs by at most a factor of $e^{\eps j}$ from the probability that it outputs a specific element that does not appear in the input. Then we consider a distribution $\distr$ that has most of its mass (specifically, $1-O(\alpha)$) on a special element, and the remaining mass spread uniformly among half of the remaining domain elements. That is, $\distr$ is a mixture of a unit distribution on the special element and a uniform distribution on half of the remaining elements. We show that, when the dataset size is too small, the sampler is nearly equally likely to output any non-special element. But it has to output non-special elements with probability $\Omega(\alpha)$ to be $\alpha$-accurate. This means that when the database size is too small, the sampler outputs a non-special element outside the support of $\distr$ with probability $\Omega(\alpha)$.
The details of the proof are quite technical and appear in 
\ifnum\neurips=1 
    Section~\ref{sec:kary-short}.
\else
    Section~\ref{sec:k-ary-lb-final}.
\fi 

%We show that the probability that a sampler \sampler outputs an element not in the support of $P$ is bounded from below by the expectation of a certain random variable $R$ that is proportional to the probability that \sampler outputs one of the non-special elements. Since \sampler is accurate, the expectation of that probability must be close to $60\alpha$. Using several tools from probability theory, we show that when the dataset size is too small, the lower bound on the expectation of $R$ is greater than $\alpha$, showing that the sampler cannot be $\alpha$-accurate.

\paragraph{Product distributions}
Our private sampler for product distributions over $\{0,1\}^d$, used to prove Theorem~\ref{thm:bernoulli-product-alg-intro},
builds on the recursive private preconditioning procedure designed by Kamath et al.~\cite{KLSU19} for learning this distribution class. In our case, the sampler gets a dataset of size which is asymptotically smaller than necessary for learning this distribution class in some important parameter regimes. Let $(\biasesfixed_1, \dots, \biasesfixed_d)$ be the attribute biases for the product distribution $\distr $ from which the data is drawn. For simplicity,  assume w.l.o.g.\ that all the marginal biases $\biasesfixed_j$ are less than 1/2. The main idea in~\cite{KLSU19} is that smaller biases have lower sensitivity in the following sense: if we know that a set of attributes has biases $p_j$ that are all at most some bound $u$, then, since the data is generated from a product distribution, the number of ones in those attributes should at most $2ud$ with high probability. We can enforce this bound on the number of ones in those coordinates by truncating each row appropriately,  and thus learn the biases of those coordinates to higher accuracy than we knew before. 
% if we know (without looking at the data) that a bias of an attribute is at at most $p$ then the empirical estimate of that attribute will be, say, at most $1.5p$ with high probability, so we can truncate the empirical estimate to be at most $1.5p$ without losing too much in accuracy. But the truncated version has sensitivity at most $1.5p/n$ instead of $1/n$, so it can be released privately with less noise. 
Building  on that idea, we partition the input into smaller datatsets and run our algorithm in rounds, each using fresh data, a different truncation threshold, and noise of different magnitude.

Our algorithm consists of two phases. In the bucketing phase, we use half of the dataset and the technique of \cite{KLSU19} to get rough \textit{multiplicative} estimates of all biases $p_j$ except the very small ones (where $p_j<1/d$). This allows us to partition the coordinates into $\log(d)$ buckets, where biases within each bucket are similar. We show this crude estimation only requires $n=\tilde O(d/\alpha \eps)$. 
%
% We bucket the $d$ attributes into intervals according to their biases $p_j$ \ssnote{I think we want to say `our best estimates of their biases' because we don't know the true biases}. Roughly speaking, there are $\log d$ buckets, and bucket $I_r$ corresponds to the interval $(\frac 1 {2^{r+1}}, \frac 1 {2^r}]$, with the last bucket corresponding to the interval $[0,\frac 1 d)$.
% %The top bucket corresponds to an interval of constant length and then lengths decrease by a constant factor until they reach length $\Theta(1/d)$. 
% The algorithm proceeds in two phases, each consisting of $\log d$ rounds. Each round uses the same privacy budget and fresh data.  In the bucketing phase of the algorithm, we estimate the buckets for all attributes. In round $r$, we assume that we already identified all attributes from buckets $I_1$ to $I_{r-1}.$ We truncate the remaining attributes using a threshold appropriate for smaller buckets and add noise corresponding to that threshold. All attributes that have sufficiently high noisy truncated mean are deemed to be in bucket $I_r$ and are removed from subsequent rounds.
%
In the sampling phase, we use the buckets to generate our output sample. For each bucket, we can get a fresh estimate of the biases using the other half of the dataset and, again, the technique from  \cite{KLSU19} to scale the noise proportionally to the upper bound on the biases in that bucket. These estimates are essentially unbiased. Flipping $d$ coins independently according to the estimated biases produces an observation with essentially the correct distribution. 

The proof of our lower bound for general product distributions proceeds via reduction from sampling of $k$-ary distributions for $k=d$. {In contrast,} %We found this surprising since 
lower bounds for private \textit{learning} of product distributions rely on fingerprinting codes (building on the framework of Bun et al. \cite{BunUV14j}). Although fingerprinting codes are indeed useful when reasoning about samplers for product distributions with bounded biases (discussed below), our approach relies, instead, on the fact that samplers must distinguish coordinates with bias 0 from coordinates with small bias. Specifically, given a distribution $\distr$ on $[2k]$ that is uniform on a subset $S$ of $[2k]$ of size $k$, we define a product distribution $\corr$ on $\bit{2k}$ with biases $p_j = \distr(j) = \frac{1}{k}$ for $j\in S$ and $p_j=0$ otherwise. We use Poissonization and coupling between Poisson and binomial distributions
%
%Specifically, given a distribution $\distr$ on $[k]$ that is uniform on a subset $S \subset [k]$, we define a product distribution $\corr$ on $\bit{k}$ with biases $p_j = \distr(j) = \frac{1}{|S|}$ for $j\in S$ and $p_j=0$ otherwise. A draw from $\corr$ will have a single nonzero entry in expectation. If there are any nonzero entries, selecting one of them uniformly at random will produce an index distributed according to $\distr$. We use this structure 
to design a reduction that, given observations from $\distr$, first creates an appropriately distributed sample of almost the same size drawn according to $\corr$, then runs a hypothetical sampler for product distributions to get a vector in $\bit{2k}$ (drawn roughly according to $\corr$), and finally converts that vector back to a single element of $[2k]$ distributed roughly as $\distr$. The details are subtle since most draws from $\corr$ will not have exactly one nonzero element---this complicates conversion in both directions.  \ifnum\neurips=0 See Section~\ref{sec:prod-lb}. \fi



\paragraph{Distributions with bounded bias}


Interestingly, our sampler for product distributions with bounded bias does not directly add noise to data. It performs the following step independently for each attribute: compute the empirical mean of the sample in that coordinate, obtain the clipped mean by rounding the empirical mean to the interval $[1/4,3/4]$, and sample a bit according to the clipped mean. 
The key idea in the analysis of accuracy is that, conditioned on no rounding (that is, the empirical mean already being in the required interval), the new bit is sampled from the correct distribution, and rounding occurs with small probability.
We argue that this sampler is $(4/n,0)$-differentially private for the case when $d=1$.
For larger $d$, the sampler is a composition of $d$ differentially private algorithms, and the main bound follows from compositions theorems (and conversions between different variants of differential privacy).
\ifnum\neurips=0
\asnote{Wishlist: Explain somewhere that these algorithms show you can do better than just finding a DP approximation in TV to the empirical distribution?}
\fi

The lower bound for this class proceeds by a reduction from the following weak estimation problem: Given samples from a product distribution with biases $p_1,p_2,...,p_d$, output estimates $\tilde p_1,...,\tilde p_d$ such that each $\tilde p_i$ is within additive error $\frac1{20}$ of $p_i$, that is, $|\tilde p_i - p_i|\leq \frac 1 {20}$, with probability at least $1-\frac 1 {20}$ (where $\frac1{20}$ is just a sufficiently small constant). This problem is known to require datasets of size $\tilde\Omega(\sqrt{d})$ for differentially private algorithms~\cite{BunUV14j}. However, nonprivate algorithms require only $O(1)$ data records to solve this same problem! We can thus reduce from estimation to sampling with very little overhead: Given a private sampler, we run it a constant number of times on disjoint datasets to obtain enough samples to (nonprivately) estimate each of the $p_i$'s. The nonprivate estimation is a postprocessing of an output of a differentially private algorithm, so the overall algorithm is differentially private. Some care is required in the reduction, since we must ensure that the nonprivate estimation algorithm is robust (i.e., works even when the samples are only close in TV distance to the correct distribution) and that the  lower bound of \cite{BunUV14j} applies even when the biases $p_i$ lie in $[\frac 1 3, \frac 2 3]$.

\ifnum\neurips=0
The approach used to prove the lower bound is quite general: informally, we can reduce from a differentially private estimation problem for a class of distributions $\class$ to a combination of differentially private sampling and a non-private estimation problem for $\class$ as follows: Given a dataset sampled independently from a distribution $\distr$ in a class $\class$, first split the dataset into many parts, and run a private sampler on each part to get a set of samples that look like they were sampled independently from $\distr$. Then use these samples to solve the estimation problem non-privately. This simple approach does not always yield tight bounds---for example, for general products of $d$ Bernoulli distributions and discrete $k$-ary distributions it gives lower bounds of $\tilde{\Omega}(\frac{\sqrt{d}}{\eps})$ and $\Omega(1/\eps)$ respectively on the number of samples required for $\alpha$-accurate sampling. However, it does give a near-tight lower bound for bounded-bias product distributions. %Unfortunately, this approach often gives loose lower bounds- naively applying it to general products of $d$ Bernoulli distributions or discrete $k$-ary distributions would give sub-optimal lower bounds of $\tilde{\Omega}(\frac{\sqrt{d}}{\eps})$ and $\Omega(1/\eps)$ on the number of samples required for $\alpha$-accurate sampling for these classes.

% \as{The approach used to prove the lower bound is quite general: informally, for any $\alpha$ and $\eps$ and class of distributions $\class$, and for any given measure of estimation accuracy, we get the following relation between sampling complexities: 
% $$\text{(DP sampling)} \geq \frac{\text{(DP estimation)}}{\text{(robust nonprivate estimation)}} \, . $$ 
%We omit an exact statement of the general relationship.}
\fi

\fi 




\ifnum\tpdp =0 
\subsection{Related work} 
\else
\paragraph{Related Work}
\fi

To the best of our knowledge, the private sampling problem we formulate has not been studied previously. There is work on nonprivate sample-size amplification, in which an initial dataset of $n$ records is used to generate an output sample of size $n'>n$ from a nearby distribution \cite{Axelrod0SV20, AxelrodGHS22}. Our formulation corresponds to the  case where $n'=1 \ll n$. 
%
There is also work on sampling (also called \textit{simulation}) in a distributed setting~\cite{AcharyaCT20,AcharyaCT20a}
%,AcharyaCFST21 does not have results on sampling, according to Clément.
where each of $n$ participants receives a single observation and is limited to $\ell\geq 1$ bits of communication. We are not aware of a direct technical connection with our work, even though the distributed setting is closely tied to that of \textit{local} differential privacy.

In the literature on differential privacy, there is work on private algorithms for sampling parameters from Bayesian posterior distributions  \cite{DimitrakakisNMR14,DimitrakakisNZM17,ZhangRD16,WangFS15}, often driven by the intuition that the randomness inherent in sampling provides some level of privacy ``for free''. Our algorithms for product distributions with bounded biases leverage a similar idea. We are not aware of any analysis of the Bayesian approach which provides guarantees along the lines of our formulation (Def.~\ref{def:acc1}). 

The most substantially related work in the privacy literature is on upper and lower bounds for learning and estimation tasks, dating back to Warner~\cite{Warner65} (see \cite{DworkS09,KamathUprimerpaper20} for partial surveys). The investigation of differentially private learning was initiated in \cite{KasiviswanathanLNRS11}. For learning $k$-ary distributions, the directly most relevant works are those of \cite{diakonikolas2015differentially,AcharyaSZ21}, though lower bounds for estimating histograms were known as folklore since the mid-2000s. For product distributions, tight upper and lower bounds for  $(\eps,\delta)$-differentially private estimation in TV distance appear in \cite{KLSU19}; the upper bound was later shown to be achievable by an $(\eps,0)$-differentially private algorithm \cite{BunKSW21J}. Our upper bound for general product distributions adapts the technique of~\cite{KLSU19}; it remains open whether our bound is achievable with $\delta=0$. 
%\as{Our lower bound for product distributions with bounded biases} uses the fingerprinting codes approach pioneered by~\cite{BunUV14j}.

Less directly related is the line of work on the generation of synthetic datasets that match the data distribution in a set of specified characteristics, such as linear queries~(e.g., \cite{blum2013learning}; see \cite{Vadhan17} for a tutorial). The goal in those works is to generate enough synthetic data to allow an analyst given only the output to estimate the specified characteristics---a much more difficult task than sampling, and one for which a different set of lower bound techniques apply~\cite{Vadhan17}. 
%
% \sr{Another related series of papers~\cite{AcharyaCT20,AcharyaCT20a,AcharyaCFST21} studies inference in a distributed model, where each sample from a distribution is held by a different party, and the communication channel between each party and the central server is subject to communication or local differential privacy constraints. The goal of the server is to perform a specified learning or property testing task.}\srnote{I added Clement's papers: one that was mentioned by the reviewer and two more. The second one mentioned by the reviewer appears to be unpublished and is possibly superceded by the papers I cited. But I am not sure.}



\ifnum\neurips=1



%%%%%%%%%%%%%%%%%%%%%%%%%%%%%%%%%%%%%%%%%%%%%%%%%%%%
%%%%%%%%%%%%%%%%%%%%%%%%%%%%%%%%%%%%%%%%%%%%%%%%%%%%
%
% This is short version of the k-ary lower bound proof that appeared in the NeurIPS 2021 submission. Should be obsolete now. 
%
%%%%%%%%%%%%%%%%%%%%%%%%%%%%%%%%%%%%%%%%%%%%%%%%%%%%
%%%%%%%%%%%%%%%%%%%%%%%%%%%%%%%%%%%%%%%%%%%%%%%%%%%%



\section{The lower bound for $k$-ary distributions}\label{sec:k-ary-lb-frequency-count-based}\label{sec:kary-short}

In this section, we prove the lower bound for sampling from discrete distributions with the universe of size at least 3.  As discussed in Section~\ref{intro:overview}, we prove that $n = \Omega(k/\alpha \eps)$. The crux of the proof is the case where the sampler is frequency-count-based, Poisson, and is $(\eps, \delta)$-DP for small $\eps$. The transformation from general samplers to this restricted type of sampler is presented in the supplementary materials. This transformation together with the following Lemma~\ref{lem:main-k-ary-lb} completes the proof of Theorem~\ref{thm:intro-k-ary-lb} for discrete distributions with the universe of size at least 3.

\begin{lemma}\label{lem:main-k-ary-lb} Fix $k, %\geq 1, % 1, not 3
n \in{\mathbb N}, \alpha \in (0,0.02], \eps \in(0,1/\ln (1/\alpha)],$ and $\delta \in [0, 0.1\alpha\eps/k]$. Let $\class_{2k+1}$ denote the class of discrete distributions over the universe $[2k+1]$. If sampler \sampler is $(\eps, \delta)$-DP, frequency-count-based, and $\alpha$-accurate on class $\class_{2k+1}$ with dataset size distributed as $\Po(n)$, then $n > \frac 1 {60}\cdot \frac{k}{\alpha \eps}$.
\end{lemma}

\begin{proof}
We consider the following distribution $\distr$ over the universe $\universe=[2k+1].$ Fix $\alpha^*=60 \alpha$ and a set $S^*\subset[2k]$ of size $k.$ Distribution $\distr$ has mass $\alpha^*/k$ on each element in $S^*$ and mass $1-\alpha^*$ on the {\em special} element $2k+1.$ 

Consider a sampler \sampler satisfying the conditions of Lemma~\ref{lem:main-k-ary-lb}. Let $\distroutput{\sampler, \distr}$ denote the output distribution of $\sampler$ when the dataset size $N \sim \Po(n)$ and the dataset $\Datarv \sim \distr^{\otimes N}$.
Observe that
\begin{eqnarray}\label{eq:main-dist-lb-delta}
d_{TV}(\distroutput{\sampler,\distr}, \distr)
  \geq \Pr_{\substack{N\sim\Po(n) \\ \Datarv\sim  \distr^{\otimes N}}}[\sampler(\Datarv) \notin Supp(\distr)].  
\end{eqnarray}
We show that if $n\leq \frac k{60\alpha \eps}$  and $\eps$ and $\delta$ are in the specified range, the right-hand side of (\ref{eq:main-dist-lb-delta}) is large. We start by deriving a lower bound on $\Pr[\sampler(\Datafixed)\notin Supp(\distr)]$ for a fixed dataset $\Datafixed$ of a fixed size $N$. Let $p_{j,F(\Datafixed)}$ be the probability that \sampler outputs a specific element in $[2k]$ that occurs $j$ times in $\Datafixed$, where $F(\Datafixed)$ is the frequency-count of \Datafixed; these probabilities are well-defined because \sampler is frequency-count-based. Let $F^*_0(\Datafixed)$ denote the number of elements in $[2k]$ that occur $0$ times in $\Datafixed$ (excluding the special element $2k+1$ from this count).
 By definition, $F^*_0(\Datafixed)\leq 2k.$ Consequently,
\begin{equation}\label{eq:notinsupport-delta}
\Pr[\sampler(\Datafixed) \notin Supp(\distr)] = k\cdot p_{0,F(\Datafixed)} \geq \frac{1}{2}\cdot  F^*_0(\Datafixed)\cdot p_{0,F(\Datafixed)}.
\end{equation}

The next claim uses the fact that %sampler 
\sampler is $(\eps,\delta)$-DP to show that the probability $p_{j,F(\Datafixed)}$ cannot be much larger than the probability that \sampler outputs a specific element in $\universe$ that does not appear in $\Datafixed$.

\begin{claim}\label{clm:epsdelfing}
For all $(\eps, \delta)$-DP samplers, frequency counts $f \in \mathbb{Z}^*$, and elements $j \in \universe$,
\begin{equation}\label{eq:grouppriv}
 p_{j,f} 
 \leq e^{\eps j} \Big( p_{0,f} + \frac{\delta}{\eps} \Big).
\end{equation}
\end{claim}

\begin{proof}
Consider a frequency count $f$ and a dataset $\Datafixed$ with $F(\Datafixed) = f$. Note that (\ref{eq:grouppriv}) is true trivially for all $j$ such that $F_j(\Datafixed)=0$ because, in that case, $p_{j,F(\Datafixed)}$ is set to $0$.

Fix any $j \in \universe$ with $F_j(\Datafixed) > 0$. Let $a$ be any element in $\universe$ that occurs $j$ times in the dataset $\Datafixed$. Let $b$ be any element in $\universe$ that is not in the support of the distribution $\distr$. Let $\Datafixed|_{a\rightarrow b}$ denote the dataset obtained by replacing every instance of $a$ in the dataset $\Datafixed$ with element $b$. By group privacy~\cite{DworkMNS06j},
\begin{equation}\label{eq:group_privacy-delta}
\Pr[\sampler(\Datafixed) = a] \leq e^{j \eps} \Pr[\sampler( \Datafixed|_{a \rightarrow b}) = a ] + \delta \cdot \frac{e^{\eps j} -1}{e^{\eps} - 1}.
\end{equation}
Note that the dataset $\Datafixed|_{a\rightarrow b} $ does not contain element $a$, since we've replaced every instance of it with $b$. Importantly, $F(\Datafixed|_{a \rightarrow b}) = F(\Datafixed)$ because $b$ is outside of the support of the distribution $\distr$ and hence does not occur in $\Datafixed$. Since $\sampler$ is frequency-count-based and $F(\Datafixed) = F(\Datafixed|_{a \rightarrow b})$, we get that $p_{0,F(\Datafixed)} = p_{0,F(\Datafixed|_{a \rightarrow b})}$. 
%
Substituting this into (\ref{eq:group_privacy-delta}) and using the fact that $e^{\eps}-1\geq \eps$ for all $\eps$, we get 
%when $F_j(\Datafixed) > 0$,
\begin{equation*}%\label{eq:relabel_assumption}
p_{j,F(\Datafixed)} \leq e^{j \eps}\cdot p_{0,F(\Datafixed)} + \delta \cdot \frac{e^{\eps j} - 1}{e^\eps -1}
\leq e^{\eps j} \Big(p_{0,f} + \frac{\delta}{\eps} \Big).
\end{equation*}
This completes the proof of Claim~\ref{clm:epsdelfing}.
\end{proof}

For a dataset $\Datafixed$ and $i\in[2k+1]$, let $N_i(\Datafixed)$ denote the number of occurrences of element $i$ in $\Datafixed$.
Next, we give a lower bound on $\Pr[\sampler(\Datafixed) \notin Supp(\distr)]$ in terms of the counts $N_i(\Datafixed)$.

\begin{claim}\label{clm:nonsupport-lb-fixed-delta}
Let $N\in\mathbb{N}$ and $\Datafixed \in [2k+1]^N$ be a fixed dataset. Set %$Y = \sum_{j=1}^N F_j(\Datafixed)e^{\eps j}$.
$Y=\sum_{i \in S^*}  \left[ e^{N_i(\Datafixed) \eps} \right]$.
Then
$$\Pr[\sampler(\Datafixed) \notin Supp(\distr)] \geq\frac{1}{2}
\cdot\frac{\Pr[\sampler(\Datafixed) \in [2k]]}{1+Y/k} -\frac{k\delta}\eps.$$
\end{claim}

\begin{proof}
In the following derivation, we use the fact that that an element $j\in[2k]$ that appears $j$ times in $\Datafixed$ is returned by \sampler with probability $p_{j,F(\Datafixed)}$, then split the elements into those that do not appear in $\Datafixed$ and those that do, next use the fact that all elements from $[2k]$ that appear in $\Datafixed$ must be in $S^*$, then apply Claim~\ref{clm:epsdelfing}, and finally substitute $Y$ for $\sum_{i \in S^*}  \left[ e^{N_i(\Datafixed) \eps} \right]$:
%$\sum_{j=1}^N F_j(\Datafixed)e^{\eps j}$:
\begin{align*}
\Pr[\sampler(\Datafixed) \in[2k]] 
&=\sum_{i\in[2k]}  p_{N_i(\Datafixed),F(\Datafixed)} 
=F^*_0(\Datafixed)\cdot p_{0,F(\Datafixed)}+\sum_{i\in[2k]\cap\Datafixed}  p_{N_i(\Datafixed),F(\Datafixed)} \\
& \leq F^*_0(\Datafixed)\cdot p_{0,F(\Datafixed)}+\sum_{i\in S^*}  p_{N_i(\Datafixed),F(\Datafixed)} \\
& \leq F^*_0(\Datafixed)\cdot p_{0,F(\Datafixed)}+\sum_{i\in S^*}  p_{0,F(\Datafixed)}\cdot \Big(e^{\eps N_i(\Datafixed)} + \frac{\delta}{\eps} \Big) \leq \Big(F^*_0(\Datafixed)+Y\Big)\Big(p_{0,F(\Datafixed)}+  \frac{\delta}{\eps}\Big).
\end{align*} 
We rearrange the terms to get \ 
$\displaystyle
p_{0,F(\Datafixed)}
\geq \frac{\Pr[\sampler(\Datafixed) \in[2k]]}{F^*_0(\Datafixed)+Y}-\frac \delta \eps.
$

Substituting this bound on $p_{0,F(\Datafixed)}$ into (\ref{eq:notinsupport-delta}), we obtain that $\Pr[\sampler(\Datafixed)\notin Support(P)]$ is at least
\begin{align*}
    \frac{1}{2} \cdot \frac {F^*_0(\Datafixed)\Pr[\sampler(\Datafixed) \in [2k]]}{F^*_0(\Datafixed) + Y} -\frac 12 \cdot\frac{F^*_0(\Datafixed)\cdot\delta}{\eps} 
  % &= \frac{1}{2} \cdot \frac {\Pr[\sampler(\Datafixed) \in [2k]]}{1 + Y/F^*_0(\Datafixed)} -\frac 12 \cdot\frac{F^*_0(\Datafixed)\cdot\delta}{\eps}\\
   \geq\frac{1}{2}
\cdot\frac{\Pr[\sampler(\Datafixed) \in [2k]]}{1+Y/k} -\frac{k\delta}\eps,
\end{align*}
where in the inequality, we used that $k\leq F^*_0(\Datafixed) \leq 2k$.
This holds since the support of $\distr$ excludes $k$ elements from $[2k]$ and since $F^*_0(\Datafixed)$ counts only elements from $[2k]$ that do not appear in~$\Datafixed.$
\end{proof}


Finally, we give a lower bound on the right-hand side of (\ref{eq:main-dist-lb-delta}). Assume for the sake of contradiction that $n\leq \frac{k}{\alpha^* \eps}$. 
%
By Claim~\ref{clm:nonsupport-lb-fixed-delta},
%
\begin{align}
 \Pr_{N, \Datarv}[\sampler(\Datarv) \notin Supp(\distr)]
 &\geq \E_{N, \Datarv}\Big[\frac{1}{2}
\cdot \frac{\Pr[\sampler(\Datarv) \in [2k]]}{1+Y/k} -\frac{k\delta}\eps\Big]\nonumber\\
& = \frac{1}{2} \cdot \E_{N, \Datarv}\Big[\frac{\Pr[\sampler(\Datarv) \in [2k]]}{1+Y/k}\Big] -  \frac{k\delta}{\eps}. \label{eq:mainlbcond}
\end{align}
Next, we analyze the expectation in (\ref{eq:mainlbcond}). Let $E$ be the event that $\frac{Y}{k} \leq e^3$. By the law of total expectation, 
\begin{align}\label{eq:mainlbcondfirst}
     \E_{N, \Datarv}\Big[\frac{\Pr[\sampler(\Datarv) \in [2k]]}{1+Y/k}\Big]
     & \geq \E_{N, \Datarv}\Big[\frac{\Pr[\sampler(\Datarv) \in [2k]]}{1+Y/k} \big | E\Big] \Pr(E).
\end{align}
In Claims~\ref{claim:eventE} and~\ref{claim:exp-of-regular-output}, we argue that both $\Pr(E)$ and $\E_{N, \Datarv}\left[\frac{\Pr[\sampler(\Datarv) \in [2k]]}{1+Y/k} \big | E\right]$ are large.
\begin{claim}\label{claim:eventE}
Suppose $n\leq \frac k{60\alpha \eps}$. Let $E$ be the event that $\frac{Y}{k} \leq e^3$. Then
    $\Pr(E) \geq 1 - \alpha$.
    \end{claim}
\begin{proof}
Recall that $Y$ was defined as $\sum_{i \in S^*}  \left[ e^{N_i(\Datafixed) \eps} \right]$ for a fixed dataset $\Datafixed.$ Now consider the case when dataset $\Datarv$ is a random variable. 
If $N\sim\Po(n)$ and $\Datarv \sim \distr^{\otimes N}$ then $N_i(\Datarv) \sim \Po(\frac{\alpha^* n}{k})$ for all $i \in S^*$ and, additionally, the random variables $N_i(\Datarv)$ are mutually independent. When $\Datarv$ is clear from the context, we write $N_i$ instead of $N_i(\Datarv)$. Now we calculate the moments of $\frac{Y}{k}$.  For all $\lambda > 0$, 
\begin{align}
\E_{\substack{N\sim\Po(n) \\ \Datarv \sim  \distr^{\otimes N}}} \Big[\Big(\frac{Y}{k}\Big)^{\lambda} \Big] 
 = \E_{\substack{N\sim\Po(n) \\ \Datarv \sim  \distr^{\otimes N}}} \Big[\Big(\frac{1}{k} \sum_{i \in S^*} e^{N_i(\Datarv) \eps} \Big)^{\lambda} \Big] 
 = \E_{N_1, \cdots, N_k \sim\Po(\frac{\alpha^* n}{k})} \Big[ \Big(\frac{1}{k} \sum_{i \in S^*} e^{N_i \eps} \Big)^{\lambda} \Big]. \label{eq:mompoiss}
\end{align}

%%%%%%%%%%%%%%%%%%%%%%%%
%% Moments of the average claim was moved to Appendix.
%%%%%%%%%%%%%%%%%%%%%%%%%%%%%%%%%%%%%%%

%Applying Claim~\ref{claim:momavg} to (\ref{eq:mompoiss}), we get that 
%\begin{align}\label{eq:mompoissfinal}
%\E_{\substack{N\sim\Po(n) \\ \Datafixed \sim  \distr^{\otimes N}}} \left[\left(\frac{Y}{k}\right)^{\lambda} \right] \nonumber  = %\E_{N_1, \cdots, N_k \sim\Po(\frac{\alpha^* n}{k})} \left[ \left(\frac{1}{k} \sum_{i \in S^*} e^{N_i \eps} \right)^{\lambda} \right] 
%    \leq \E_{N_1 \sim\Po(\frac{\alpha^* n}{k})} \left[ \left(e^{N_1 \eps} \right)^{\lambda} \right].
%\end{align}
Finally, we bound the probability of event $E$. Set $c=e^3$ and $\lambda = \ln \frac{1}{\alpha}$. By definition of $E$, 
\begin{align}
    \Pr(\overline{E}) 
    & = \Pr\Big(\frac{Y}{k} \geq c\Big)\nonumber 
     = \Pr\Big(\Big(\frac{Y}{k}\Big)^\lambda \geq c^{\lambda}\Big) 
    \leq \frac 1{c^{\lambda}}\cdot {\E_{\substack{N\sim\Po(n) \\ \Datarv \sim  \distr^{\otimes N}}}
    \Big[ \Big(\frac{Y}{k}\Big)^\lambda \Big]} \nonumber \\
    & \leq  \frac 1{c^{\lambda}}\cdot {\E_{N_1, \cdots, N_k \sim\Po(\frac{\alpha^* n}{k})} \Big[ \Big(\frac{1}{k} \sum_{i \in S^*} e^{N_i \eps} \Big)^{\lambda} \Big]} 
     \leq  \frac 1{c^{\lambda}}\cdot{\E_{N_1 \sim\Po(\frac{\alpha^* n}{k})} \Big[ \Big( e^{N_1 \eps} \Big) ^{\lambda} \Big]} \label{eq:moments2} \\
    & = c^{-\lambda}\cdot {e^{\frac{\alpha^* n}{k}(e^{\lambda \eps} - 1)}}
    \leq e^{-3\lambda}\cdot {e^{\frac{(e^{\lambda \eps} - 1 )}{\eps}}}
    \leq e^{-3\lambda}\cdot e^{2\lambda}= e^{-\lambda}
    =e^{-\ln (1/\alpha)}=\alpha, \label{eq:moments3}
\end{align}
where we use $\lambda > 0$ in the second equality, then apply Markov's inequality. To get the inequalities in (\ref{eq:moments2}), we apply (\ref{eq:mompoiss}) and then use the fact that the moments of the average of random variables is less than the moment of a single random variable (the proof of this fact is in the supplementary material). To get (\ref{eq:moments3}), we use the moment generating function of a Poisson random variable, and then we substitute $c=e^3$ and use the assumption that $n\leq \frac k{60\alpha \eps} =\frac k{\alpha^*\eps}$. The second inequality in (\ref{eq:moments3}) holds because $\lambda = \ln \frac{1}{\alpha}$ and $\eps \in (0,1/\ln \frac{1}{\alpha}]$, so $\lambda \eps \leq 1$ and hence $e^{\lambda \eps} \leq 1 + 2\lambda \eps$.
The final expression is obtained by substituting the value of $\lambda.$
We get that $\Pr(E) \geq 1-\alpha$, completing the proof of Claim~\ref{claim:eventE}.
\end{proof}
\begin{claim}\label{claim:exp-of-regular-output}
$\displaystyle\E_{\substack{N\sim\Po(n) \\ \Datarv\sim  \distr^{\otimes N}}}\Big[\frac{\Pr[\sampler(\Datarv) \in [2k]]}{1+Y/k} \big| E \Big] \geq 2.3\alpha.$
\end{claim}
\begin{proof}
When event $E$ occurs, $1+\frac{Y}{k} \leq 1+e^3<22$. Then  
\begin{align}\label{eq:conditioning}
    \E_{N, \Datarv}\Big[\frac{\Pr[\sampler(\Datarv) \in [2k]]}{1+Y/k} \big | E \Big] > \E_{N, \Datarv}\Big[\frac{\Pr[\sampler(\Datarv) \in [2k]]}{22} \big | E\Big]  = \frac 1{22}\cdot\E_{N, \Datarv}\Big[\Pr[\sampler(\Datarv) \in [2k]] \mid E \Big].
\end{align}
By the product rule,
$$\Pr[\sampler(\Datarv) \in [2k]] \mid E ]
=\frac{\Pr[\sampler(\Datarv) \in [2k]] \wedge E]}{\Pr[E]}
\geq \Pr[\sampler(\Datarv) \in [2k]] \wedge E]
\geq  {\Pr[\sampler(\Datarv) \in [2k]] - \Pr[\overline{E}]}.$$
Substituting this into (\ref{eq:conditioning}) and recalling that $\alpha^*=60\alpha$, we get
\begin{align*}
    \E_{N, \Datarv}\Big[\frac{\Pr[\sampler(\Datarv) \in [2k]]}{1+Y/k} \big | E \Big]
    &\geq  \frac 1{22}\cdot\E_{N, \Datarv}\Big[\Pr[\sampler(\Datarv) \in [2k]] - \Pr[\overline{E}] \Big]
%    = \Pr[\sampler(\Datarv) \in [2k]] \mid E ] \cdot \Big( \frac{1}{30} \Big) 
 %    = \frac{\Pr[\sampler(\Datarv) \in [2k]] \wedge E]}{\Pr[E]} \cdot \Big( \frac{1}{30} \Big) \\
 %    &\geq \frac{\Pr[\sampler(\Datarv) \in [2k]] - \Pr[\overline{E}]}{\Pr[E]} \cdot \Big( \frac{1}{30} \Big) 
     \geq \frac 1{22}\cdot \Big( \alpha^* - \alpha - \alpha \Big) 
     \geq 2.3\alpha,
\end{align*}
since sampler $\sampler$ is $\alpha$-accurate on $\distr$, and  $\distr$ has mass $\alpha^*$ on $[2k]$, and by Claim~\ref{claim:eventE}. 
\end{proof}
%Applying Claims~\ref{claim:eventE} and~\ref{claim:exp-of-regular-output} to (\ref{eq:mainlbcondfirst}), we get that
%\begin{align}
%     \frac{1}{2} \cdot \E_{\substack{N\sim\Po(n) \\ \Datarv\sim  \distr^{\otimes N}}}\Big[\frac{\Pr[\sampler(\Datarv) \in [2k]]}{1+Y/k}\Big] 
%      \geq  \frac{1}{2} \cdot \frac{\alpha^* - 2\alpha}{30}\cdot \Big(1 - \alpha\Big) 
%      \geq \frac{\alpha^* - \frac{5\alpha}{4}}{120}. \label{eq:alphainexp}
%\end{align}
%Substituting  (\ref{eq:alphainexp}) into (\ref{eq:mainlbcond}), we get that
%\begin{align}
% \Pr_{\substack{N\sim\Po(n) \\ \Datarv\sim  \distr^{\otimes N}}}[\sampler(\Datarv) \notin Supp(\distr)]
%& > \frac{\alpha^* - \frac{5\alpha}{4}}{120} - k\cdot \frac{\delta}{\eps}
%\end{align}

Combining (\ref{eq:main-dist-lb-delta}), (\ref{eq:mainlbcond}), and (\ref{eq:mainlbcondfirst}), applying Claims~\ref{claim:eventE} and~\ref{claim:exp-of-regular-output}, and recalling that $\delta\leq 0.1\cdot\alpha\eps/k$, we get
\begin{align*}
    d_{TV}&(\distr,\distroutput{\sampler, \distr})
  \geq \Pr_{N, \Datarv}[\sampler(\Datarv) \notin Supp(\distr)]
  \geq \frac{1}{2} \cdot \E_{N, \Datarv}\Big[\frac{\Pr[\sampler(\Datarv) \in [2k]]}{1+Y/k}\Big]  - \frac{k\delta}{\eps} \\
  &\geq \frac 12\cdot   \E_{N, \Datarv}\Big[\frac{\Pr[\sampler(\Datarv) \in [2k]]}{1+Y/k} \big | E\Big] \Pr(E) - 0.1\alpha 
  \geq  \frac{1}{2} \cdot 2.3\alpha\cdot \Big(1 - \alpha\Big) - 0.1\alpha 
  > \alpha,
%     \geq\frac{1}{1 + \frac 1k\E_{N\sim\Po(n),\Datarv\sim  \distr^{\otimes N}} \left[\sum_{i\in \Datarv} e^{N_i(\Datarv) \cdot \eps} \right]}  
 %   & > \frac{\alpha^* - \frac{5\alpha}{4}}{120}  - \frac{k\delta}{\eps} \\
%    & \geq \frac{\alpha^* - \frac{5\alpha}{4}}{120}  - \frac{\alpha}{120} \quad \text{(Since $\delta \leq \frac{\alpha\eps}{120k}$)}
%    \\
%    & > \alpha \quad \text{($\alpha^*$ set to $50 \alpha$)}
\end{align*}
where the last inequality holds since $\alpha\leq 0.02$. This contradicts $\alpha$-accuracy of $\sampler$ on datasets of size $\Po(n)$, where $n\leq \frac{k}{\alpha^* \eps}$, and completes the proof of Lemma~\ref{lem:main-k-ary-lb}.
\end{proof}
\color{black}


\begin{ack} We are grateful for helpful conversations with Clément Canonne, Thomas Steinke, and Jonathan Ullman. 
Sofya Raskhodnikova was partially supported by NSF award CCF-1909612.
Satchit Sivakumar was supported in part by NSF award CNS-2046425, as well as Cooperative Agreement CB20ADR0160001 with the Census Bureau. 
Adam Smith and Marika Swanberg were supported in part by NSF award CCF-1763786 as well as a Sloan Foundation research award.
The views expressed in this paper are those of the authors and not those of the U.S. Census Bureau or any other sponsor.
\end{ack}

\bibliographystyle{plain}
\bibliography{refs}{}


\fi


\ifnum\supplemental=1 
\newpage
\pagenumbering{roman}
\appendix

\begin{center}
    \Huge Supplementary Materials 
    
    \rule{3in}{1pt}
    
    \Large Differentially Private Sampling from Distributions
\end{center}

These supplementary materials are organized as follows. Section~\ref{sec:defs} collects standard definitions and mathematical tools.  Next, in Section~\ref{sec:properties}, we describe general transformations of samplers that we use in our lower bounds. In Section~\ref{sec:kary}, we prove upper and lower bounds for the task of private sampling from $k$-ary distributions, corresponding to Theorems~\ref{thm:kupperb} and~\ref{thm:intro-k-ary-lb} in the introduction. In Section~\ref{sec:prod}, we prove upper bounds for private sampling from product distributions over $\{0,1\}^d$, corresponding to Theorem~\ref{thm:bernoulli-product-alg-intro}. We defer the proof of Theorem~\ref{thm:bernoulli-product-lb-intro} to the full version of the paper. In Section~\ref{sec:bounded-bias}, we present our upper and lower bounds for private sampling from product distributions with bounded attribute biases, corresponding to Theorems~\ref{thm:intro-bernoulli-product-bb} and~\ref{thm:intro-product-bb-lb} in the introduction. Finally, in Section~\ref{sec:inequalities}, we discuss some standard results that we use in our proofs and, in Section~\ref{sec:resow}, we state some results from other papers that we use in our proofs. 


\fi


\ifnum\tpdp=0
\ifnum\neurips=0
\section{Definitions}
\label{sec:defs}

% \subsection{Definitions of sampling accuracy}

% \asnote{TO do: rework this section} Below we present three definitions of sampling accuracy. The first is adapted from Axelrod et al.~\cite{axelrod2019sample}. Definition~\ref{def:acc2} is the expectation version of the ``with high probabiliy'' Definition~\ref{def:acc3}, and they are essentially equivalent, up to changes in the parameter $\alpha$. Furthermore, Definitions~\ref{def:acc2} and~\ref{def:acc3} are equivalent to non-private learning (that is, if one can generate one accurate sample with high probability, then one can continue generating samples until there are enough to learn $p$). In contrast, it is unclear whether the latter definitions are  equivalent to \emph{private} learning. 

% Let $\universe$ denote the universe of possible values for each record in a dataset.
% We consider sampling algorithms that get independent observations from a distribution $\distr$ on $\universe$ and output a single record in $\universe$. 
% %
% %Specifically, each algorithm works on a specific distribution class $\class$ which are defined over a data universe $\universe$.
% %
% For a dataset $\Datafixed \in \universe^*$, we use $\datafixed_i$ to denote the $i^{th}$ record of $\Datafixed$. Given a distribution $\distr$ on $\universe$ and size $n$, a random dataset $\Datarv$ is a random variable whose entries $\datarv_i$ are drawn i.i.d.\ from $\distr$. We sometimes consider datasets whose size is randomly distributed.
% %
% The dataset size, or distribution on the dataset size, is a parameter of the algorithm $\sampler$. We let $\sampler(\Datarv)$ denote the random variable corresponding to the output of the algorithm $\sampler$ on input $\Datarv$. This random variable depends on the choice of dataset size (when it is random), the selection of entries of $\Datarv$ from $\distr$, and the coins of $\sampler$. Let $\distroutput{\sampler, \distr}$ denote the distribution of $\sampler(\Datarv)$.

% \msnote{Need to add some context about the classes we consider here and some context on definitions of accuracy that follow.}

% \begin{definition}[Accuracy, Axelrod et al.~\cite{Axelrod0SV20}]
% \label{def:acc1} A sampler $\sampler$ is $\alpha$-accurate on a distribution class $\class$ if for all distributions $\distr \in \class,$ 
% \begin{equation*}
%     d_{TV}(\distroutput{\sampler, \distr}, \distr) \leq \alpha.
% \end{equation*}
% \end{definition}

% \begin{definition}[Accuracy on average] \label{def:acc2}A sampler $\sampler$ is $\alpha$-accurate-on-average on a distribution class $\class$ if for all distributions $\distr\in \class,$ 
% \begin{equation*}
% \E_{\Datafixed \sim \distr^{\otimes n}}[d_{TV}(\sampler(\Datafixed), \distr)] \leq \alpha.
% \end{equation*}
% \end{definition}

% \begin{definition}[Accuracy with high probability] \label{def:acc3}A sampler $\sampler$ is $(\alpha, \beta)$-accurate-whp on a distribution class $\class$ if for all $\distr \in \class$, with probability (over $X^n \sim \distr^{\otimes n})$ at least $1-\beta$,
% \begin{equation*}
% d_{TV}(\sampler(\Datafixed), \distr) \leq \alpha.
% \end{equation*}
% \end{definition}

%Add relationships between definitions



\subsection{Differential Privacy}



A dataset $\Datafixed = (\datafixed_1, \ldots, \datafixed_n) \in \universe^n$ is a vector of elements from universe \universe. Two datasets are {\em neighbors} if they differ in at most one coordinate. Informally, differential privacy requires that an algorithm's output distributions are similar on all pairs of neighboring datasets. We use two different variants of differential privacy. The first one (and the main one used in this paper) is the standard definition of differential privacy.

\begin{definition}[Differential Privacy~\cite{DworkMNS06j,DworkKMMN06}]\label{def:differentially private} A randomized algorithm $\sampler: \universe^n \rightarrow \mathcal{Y}$ is {\em $(\eps, \delta)$-differentially private} if for every pair of neighboring datasets $\Datafixed, \Datafixed'\in \universe^n$ and for all subsets $Y\subseteq \mathcal{Y}$,
 \begin{equation*}
    \Pr[\sampler(\Datafixed) \in Y] \leq e^\eps \cdot \Pr[\sampler(\Datafixed') \in Y] + \delta.
 \end{equation*}
 \end{definition}
 In addition to standard differential privacy (Definition~\ref{def:DP}), we use a variant called {\em zero-mean concentrated differential privacy} \cite{bun2016concentrated} that is defined in terms of R\'enyi divergence.
\begin{definition}[R\'enyi divergence] Consider two probability distributions $P$ and $Q$ over a discrete domain~$S$. Given a positive $\alpha\neq 1$, R\'enyi divergence of order $\alpha$ of distributions $P$ and $Q$ is 
\begin{equation*}
    D_\alpha (P || Q) = \frac{1}{1-\alpha} \log\left(\sum_{\datafixed \in S} P(\datafixed)^\alpha Q(\datafixed)^{1-\alpha} \right).
\end{equation*}

\end{definition}

\begin{definition}[Zero-Mean Concentrated Differential Privacy (zCDP)~\cite{bun2016concentrated}] \label{def:CDP}
A randomized algorithm $\sampler : \universe^n \rightarrow \mathcal{Y}$ is $\rho$-zCDP if for every pair of neighboring datasets $\Datafixed, \Datafixed' \in \universe^n$,
\begin{equation*}
    \forall \alpha \in (1, \infty) \quad D_\alpha\left(\sampler(\Datafixed) ||\sampler(\Datafixed')\right) \leq \rho \alpha,
\end{equation*}
where $D_\alpha(\sampler(\Datafixed) ||\sampler(\Datafixed'))$ is the $\alpha$-R\'enyi divergence between $\sampler(\Datafixed)$ and $\sampler(\Datafixed')$.
\end{definition}

\begin{lemma}[Relationships Between $(\eps, \delta)$-Differential Privacy and $\rho$-CDP~\cite{bun2016concentrated}]\label{prelim:relate_dp_cdp} For every $\eps \geq 0$,
\begin{enumerate}
    \item If \sampler is $(\eps, 0)$-differentially private, then \sampler is $\frac{\eps^2}{2}$-zCDP.
    \item If \sampler is $\frac{\eps^2}{2}$-zCDP, then \sampler is $\left(\frac{\eps^2}{2} + \eps\sqrt{2 \log(1/\delta)}, \delta\right)$-differentially private for every $\delta > 0$.
\end{enumerate}
\end{lemma}

Both definitions of differential privacy are closed under post-processing.
\begin{lemma}[Post-Processing~\cite{DworkMNS06j,bun2016concentrated}]\label{prelim:postprocess} If $\sampler: \universe^n \rightarrow \mathcal{Y}$ is $(\eps, \delta)$-differentially private, and $\mathcal{B} : \mathcal{Y} \rightarrow \mathcal{Z}$ is any randomized function, then the algorithm $\mathcal{B} \circ \sampler$ is $(\eps, \delta)$-differentially private. Similarly, if $\sampler$ is $\rho$-zCDP then the algorithm $\mathcal{B} \circ \sampler$ is $\rho$-zCDP.
\end{lemma}


Importantly, both notions of differential privacy are closed under adaptive composition. For a fixed dataset \Datafixed, {\em adaptive composition} states that the results of a sequence of computations satisfies differential privacy even when the chosen computation $\sampler_t(\cdot)$ at time $t$ depends on the outcomes of previous computations $\sampler_1(\Datafixed), \ldots, \sampler_{t-1}(\Datafixed)$. Under adaptive composition, the privacy parameters add up.

\begin{definition}[Composition of $(\eps, \delta)$-differential privacy and $\rho$-zCDP~\cite{DworkMNS06j,bun2016concentrated}]\label{prelim:composition} Suppose $\sampler$ is an adaptive composition of differentially private algorithms $\sampler_1, \ldots, \sampler_T$.
\begin{enumerate}
    \item If for each $t \in [T]$, algorithm $\sampler_t$ is $(\eps_t, \delta_t)$-differentially private, then \sampler is $\left(\sum_t \eps_t, \sum_t \delta_t\right)$-differentially private.
    \item If for each $t \in [T]$, algorithm $\sampler_t$ is $\rho_t$-zCDP, then \sampler is $\left(\sum_t \rho_t\right)$-zCDP.
\end{enumerate}
\end{definition}



Standard $(\eps, \delta)$-differential privacy protects the privacy of groups of individuals.

\begin{lemma}[Group Privacy]\label{prelim:group_privacy}%\srnote{Is it the right reference for group privacy?} 
Each $(\eps, \delta)$-differentially private algorithm \sampler is $\left(k\eps, \delta\frac{e^{k\eps} -1}{e^\eps-1}\right)$-differentially private for groups of size $k$. That is, for all datasets $\Datafixed, \Datafixed'$ such that $\|\Datafixed - \Datafixed' \|_0 \leq k$ and all subsets $Y \subseteq \mathcal{Y}$,
\begin{equation*}
    \Pr[\sampler(\Datafixed) \in Y] \leq e^{k\eps} \cdot \Pr[\sampler(\Datafixed') \in Y] + \delta \cdot \frac{e^{k\eps} -1}{e^\eps-1}.
\end{equation*}
\end{lemma}

\paragraph{Laplace Mechanism} 

 Our algorithms use the standard Laplace Mechanism to ensure differential privacy. 

\begin{definition}[Laplace Distribution] The Laplace distribution with parameter $b$ and mean $0$, denoted by $\Lap(b)$, is defined for all $x \in \mathbb{R}$ and has probability density
\begin{equation*}
    h(\ell) = \frac{1}{2b}e^{-\frac{|\ell|}{b}}.
\end{equation*}
\end{definition}

\begin{definition}[$\ell_1$-Sensitivity] Let $f: \universe^n \rightarrow \mathbb{R}^d$ be a function. Its $\ell_1$-sensitivity is
\begin{equation*}
    \Delta_f = \max_{\substack{\Datafixed, \Datafixed' \in \universe^n \\ \Datafixed, \Datafixed' \text{neighbors}}} \|f(\Datafixed) - f(\Datafixed')\|_1.
\end{equation*}
\end{definition}

\begin{lemma}[Laplace Mechanism]\label{prelim:laplace_dp} Let $f : \universe^n \rightarrow \mathbb{R}^d$ be a function with $\ell_1$-sensitivity $\Delta_f$. Then the Laplace mechanism is algorithm
\begin{equation*}
    \sampler_f(\Datafixed) = f(\Datafixed) + (Z_1, \ldots, Z_d),
\end{equation*}
where $Z_i \sim \Lap\left(\frac{\Delta_f}{\eps}\right)$. Algorithm $\sampler_f$ is $(\eps, 0)$-differentially private.
\end{lemma}


\paragraph{Gaussian Mechanism} Our algorithms also use the common Gaussian Mechanism to ensure differential privacy.
\begin{definition}[Gaussian Distribution] The Gaussian distribution with parameter $\sigma$ and mean 0, denoted $\Gauss(\sigma)$, is defined for all $\ell \in \mathbb{R}$ and has probability density
\begin{equation*}
    h(\ell) = \frac{1}{\sigma \sqrt{2\pi}} e^{-\frac{\ell^2}{2\sigma^2}}.
\end{equation*}
\end{definition}

\begin{definition}[$\ell_2$-Sensitivity] Let  $f: \universe^n \rightarrow \mathbb{R}^d$ be a function. Its $\ell_2$-sensitivity is
\begin{equation*}
    \Delta_f = \max_{\substack{\Datafixed, \Datafixed' \in \universe \\ \Datafixed, \Datafixed' \text{neighbors}}} \|f(\Datafixed) - f(\Datafixed')\|_2.
\end{equation*}
\end{definition}

\begin{lemma}[Gaussian Mechanism]\label{prelim:gauss_cdp} Let $f : \universe^n \rightarrow \mathbb{R}^d$ be a function with $\ell_2$-sensitivity $\Delta_f$. Then the Gaussian mechanism is algorithm
\begin{equation*}
    \sampler_f(\Datafixed) = f(\Datafixed) + (Z_1, \ldots, Z_d),
\end{equation*}
where $Z_i \sim \Gauss\left(\left(\frac{\Delta_f}{\sqrt{2\rho}}\right)^2 \cdot \mathbb{I}\right)$. Algorithm $\sampler_f$ is $\rho$-zCDP.
\end{lemma}

\begin{lemma}[Exponential Mechanism \cite{McTalwar}]\label{lem:expmech}
Let $L$ be a set of outputs and $g: L \times \mathcal{X}^n \to \mathbb{R}$ be a function that measures the quality of each output on a dataset. Assume that for every $m \in L$, the function $g(m,.)$ has $\ell_1$-sensitivity at most $\Delta$. Then, for all $\eps>0$, $n \in \mathbb{N}$ and for all datasets $\Datafixed \in \mathcal{X}^n$, there exists an $(\eps, 0)$-DP mechanism that, on input $\Datafixed$, outputs an element $m\in L$ such that, for all $a>0$, we have
\begin{equation*}
    \Pr\left[\max_{i \in [L]} g(i,\Datafixed) -  g(m,\Datafixed) \geq 2\Delta \frac{\ln |L| + a}{\eps}\right] \leq e^{-a}. 
\end{equation*}
\end{lemma} 

\subsection{Distributions}
Additionally, we use the Bernoulli, binomial, multinomial, and Poisson distributions as well as total variation distance.

\begin{definition}[Bernoulli Distribution] \label{prelim:bern_def}
The Bernoulli distribution with bias $\biasesfixed \in [0,1]$, denoted $\Ber(\biasesfixed)$, is defined for $\ell \in \{0,1\}$. It has probability mass
\begin{equation*}
    h(\ell) =  
    \begin{cases} 
      \biasesfixed & \text{ if } \ell = 1;\\
      1-\biasesfixed & \text{ if } \ell = 0.
   \end{cases}
\end{equation*}

\end{definition}

\begin{definition}[Binomial Distribution] The binomial distribution with parameters $n$ and $\biasesfixed$, denoted $\Bin(n,
\biasesfixed)$, is defined for all nonnegative integers $\ell$ such that  $\ell \leq n$. It has probability mass
\begin{equation*}
    h(\ell) = \binom{n}{\ell} p^{\ell} (1-p)^{n-\ell}.
\end{equation*}
\end{definition}

\begin{definition}[Multinomial Distribution] The multinomial distribution with parameters $n$ and  $\Biasesfixed \in \Delta^k$, denoted $\Mult(n, \Biasesfixed),$ is defined for all nonnegative integer vectors $\mathbf{\ell} = (\ell_1, \ldots, \ell_k)$ such that $\sum_{i\in[k]} \ell_i \leq n$. It has probability mass
\begin{equation*}
    h(\ell) = 
    \begin{cases}
     \frac{n!}{\ell_1! \cdots \ell_k!} \cdot \biasesfixed_1^{\ell_1} \cdot \ldots \cdot \biasesfixed_k^{\ell_k} & \text{if } \sum_{i\in[k]} \ell_i = n;\\
     0, & \text{otherwise. }
    \end{cases}
\end{equation*}

\end{definition}

\begin{definition}[Poisson Distribution] The Poisson distribution with parameter $\lambda$, denoted $\Po(\lambda)$, is defined for all nonnegative integers $\ell$ with probability mass
\begin{equation*}
    h(\ell) = \frac{\lambda^{\ell} e^{-\ell}}{\ell!}.
\end{equation*}
\end{definition}
We use the following relationship between Poisson and Multinomial distributions.
%\begin{lemma}
%Let $X_1,\dots,X_{\ell}$ be independent Poisson random variables with means $\lambda_1,\dots,\lambda_{\ell}$. Then the random variables $X_i \mid \{\sum_i X_i = k \}$ are jointly distributed as $\Mult(k,\Biasesfixed)$ where $\biasesfixed_i = \frac{\lambda_i}{\sum_{j=1}^{\ell} \lambda_j}$.
%\end{lemma}
\begin{lemma}[Poissonization]\label{lem:multtopois}
Fix $h,\lambda > 0$. Let $B \sim \Po(\lambda)$. Let $B_1,\dots,B_{\ell}$ be random variables such that the random variables $B_j \mid \{B = h \}$ are jointly distributed as $\Mult(h,\Biasesfixed)$ where $\Biasesfixed = (\biasesfixed_1,\dots,\biasesfixed_{\ell})$. Then the random variables $B_j$ are mutually independent and distributed as $\Po(\lambda \biasesfixed_j)$.
\end{lemma}

\begin{definition}[Total Variation Distance] \label{def:TV} Let $P$ and $Q$ be discrete probability distributions over some domain $S$. Then
\begin{equation*}
    d_{TV}(P, Q) := \frac{1}{2}\|P - Q\|_1 = \sup_{E\subseteq S} |\Pr_{P}(E) - \Pr_{Q}(E)|.
\end{equation*}
\end{definition}

We use the fact that the total variation distance between two product distributions is subadditive. 
\begin{lemma}[Subadditivity of TV Distance for Product Distributions]\label{lem:subaddTV} Let $P$ and $Q$ be product distributions over some domain $S$. Let $P^1, \dots, P^d$ be the marginal distributions of $P$ and $Q^1, \dots, Q^d$ be the marginal distributions over $Q$. Then
\begin{equation*}
    d_{TV}(P, Q) \leq \sum_{i=1}^d d_{TV}(P^i, Q^i).
\end{equation*}
\end{lemma}
We also use the following lemma regarding total variation distance between a distribution $\distr$ and the distribution obtained by conditioning $\distr$ on a high probability event $E$.
\begin{lemma}[Claim 4, \cite{RaskhodnikovaS06}]\label{lem:TVcond}
Fix $\delta \in (0,1)$. Let $D$ be a distribution and let $E$ be an event that happens with probability $1-\beta$ under the distribution $D$. Let $D|_{E}$ be the distribution of $D$ conditional on event $E$. Then 
$$d_{TV}(D|_{E},D) \leq \frac{\beta}{1-\beta}.$$
\end{lemma}
Finally, we will need the following lemma.
\begin{lemma}[Information Processing Inequality]\label{lem:postTV} Let $A$ and $B$ be random variables over some domain $S$. Let $f$ be a randomized function mapping from $S$ to any codomain $T$. Then
\begin{equation*}
    d_{TV}(f(A), f(B)) \leq d_{TV}(A,B).
\end{equation*}
\end{lemma}

\begin{definition}[KL Divergence] Let $P$ and $Q$ be discrete probability distributions over some domain $S$. Then
\begin{equation*}
    d_{KL}(P, Q) := \frac{1}{2}\sum_{x \in S} P(x)\log\left( \frac{P(x)}{Q(x)}\right)
\end{equation*}
\end{definition}

\begin{claim}\label{clm:bern_acc_eq}
For a Bernoulli distribution $\Ber(p)$, we can simplify the definition of $\alpha$-accuracy (Definition~\ref{def:acc1}) of a sampler \sampler with inputs of size $n$ to require that
\begin{equation*}
d_{TV}(\distroutput{\sampler, \Ber(p)}, \Ber(p)) =
    \Big\lvert \Pr_{\Datarv\sim (\Ber(p))^{\otimes n}}[\sampler(\Datarv) = 1] - p \Big\rvert 
=   \Big|\E_{\Datarv\sim (\Ber(p))^{\otimes n}}
[\indicator_{\sampler(\Datarv) = 1}]-p\Big| 
    \leq \alpha.
\end{equation*}
\end{claim}

%In some of our proofs, we require the definition of stochastic domination.

%\begin{definition}[Stochastic Domination]
%Consider two random variables $A$ and $B$ defined over the real line. We say that $A$ stochastically dominates $B$ if for all $r \in \mathbb{R}$, $\Pr(A > r) \geq \Pr(B > r)$. 
%\end{definition}

%\begin{claim}\label{claim:stochdombin}
%Let $A$ be a random variable distributed as $Bin(n,p)$. Let $B$ be a random variable distributed as $Bin(n,q)$ where $p \leq q$. Then $B$ stochastically dominates $A$. 
%\end{claim}




\fi


%%%%%%%%%%%%%%%%%%%%%%%%%%%%%%%
%This is the end of the NeurIPS submission
%%%%%%%%%%%%%%%%%%%%%%%%%%%%%%%


\ifnum\neurips=0 


\ifnum\neurips=0
\section{Properties of Samplers}
\else 
\section{Properties of samplers}
\fi
\label{sec:properties}

In this section, we describe three general transformations of samplers that allow us to assume without loss of generality that samplers can take a certain specific form. All three transformations are used in our lower bound proofs.

\ifnum\neurips=1
\subsection{General samplers to Poisson samplers}\label{sec:gentopoiss}
\else 
\subsection{General Samplers to Poisson Samplers}\label{sec:gentopoiss}
\fi

In the first transformation, we show that any private sampling task can be performed by an algorithm that gets a dataset with size distributed as a Poisson random variable instead of getting a dataset of a fixed size. This enables the use of the technique called {\em Poissonization} to break dependencies between quantities that arise in trying to reason about samplers. Recall that $\Po(\lambda)$ denotes a Poisson distribution with mean $\lambda.$
\begin{lemma}\label{lem:poisson} 
If there exists an $(\eps, \delta)$-differentially private sampler $\sampler$ that is $\alpha$-accurate on distribution class~$\class$ for datasets of size $n$, then there exists an $(\eps, \delta)$-differentially private sampler  $\posampler$ that is $(\alpha + e^{-n/6})$-accurate on class $\class$ for datasets of size  distributed as $\Po(2n)$.
\end{lemma}

\begin{proof}
Algorithm~\ref{alg:poisson} is the desired sampler $\posampler$. It is $(\eps, \delta)$-differentially private since $\sampler$ is $(\eps, \delta)$-differentially private. Let $\Datarv$ represent the random variable corresponding to the dataset fed to \sampler.
\begin{algorithm}
    \caption{Sampler \posampler with dataset size $N \sim \Po(2n)$}
    \label{alg:poisson}
    \hspace*{\algorithmicindent} \textbf{Input:} dataset $\Datafixed = (\datafixed_1, \ldots, \datafixed_{N})$,  universe $\universe$, oracle access to $(\eps, \delta)$-DP sampler $\sampler$, parameter~$n$\\
    \hspace*{\algorithmicindent} \textbf{Output:} $i\in \universe$
    \begin{algorithmic}[1] % The number tells where the line numbering should start
            \State Fix an element $\el \in \universe$. 
            \If {$N < n$}  $i=\el$
            \Else  $\text{ } i \gets \sampler(\Datafixed)$\\
            \Return $i$
            \EndIf
    \end{algorithmic}
\end{algorithm}

We use the following tail bound for Poisson random variables \cite{clementpoiss}.
\begin{claim}[\cite{clementpoiss}]\label{lem:poiss_tail}
If $Y\sim \Po(\lambda)$, then $\Pr(Y \leq \lambda - y) \leq e^{-\frac{y^2}{2(\lambda + y)}}$ and $\Pr(Y \geq \lambda + y) \leq e^{-\frac{y^2}{2(\lambda + y)}}$ for all $y>0$.
\end{claim}
Let event $E$ correspond to $N < n$. Let $\overline{E}$ represent the complement of $E$. Then 
\begin{equation}\label{eq:event_tail}
\Pr(E) \leq e^{-\frac{n}{6}}
\end{equation}
by an application of Claim~\ref{lem:poiss_tail}. 
Let $\distr \in \class$ and $\Datarv \sim \distr^{\otimes N}$. Then
\begin{align*}
 d_{TV}(\distroutput{\posampler, \distr}, \distr) 
& =\frac 1 2 \sum_{i \in \universe}|\Pr_{N, \posampler, \Datarv}(\posampler(\Datarv)=i) - \distr(i)| \\
& =\frac 1 2   \sum_{i\in \universe}\left|\Pr_{N, \posampler, \Datarv}(\posampler(\Datarv)=i \land \overline{E}) + \Pr_{N, \posampler, \Datarv}(\posampler(\Datarv)=i \land  E) - \distr(i) \left(\Pr_{N}(\overline{E})+\Pr_{N}(E)\right) \right| \\
& \leq \frac 12 \sum_{i \in \universe}\left(\left|\Pr_{N, \posampler, \Datarv}(\posampler(\Datarv)=i \land  \overline{E}) - \distr(i)\Pr_{N}(\overline{E})\right| +  \Pr_{N, \posampler, \Datarv}(\posampler(\Datarv)=i \land E)+ \Pr_{N}(E)\distr(i)\right)  \\
%& \leq \frac 12 \sum_{o \in \universe}\Big|\Pr_{N, \posampler, \Datarv}(\posampler(\Datarv)=o \mid \overline{E})\Pr(\overline{E}) - \distr(o)\Big| + \frac 12 \Pr_{N'}(E) \\
& = \frac 12 \sum_{i \in \universe}\left|\Pr_{N, \posampler, \Datarv}(\posampler(\Datarv)=i \mid \overline{E})\Pr(\overline{E}) - \distr(i)\Pr_{N}(\overline{E})\right| + 
\frac 12 \cdot(\Pr_{N}(E)+\Pr_{N}(E))\\
%\Pr_{N', \sampler, \Datarv}(\sampler(\Datarv)=o, E) \\
& = \frac 12 \sum_{i \in \universe}\Pr_{N}(\overline{E})\cdot\left|\Pr_{\sampler, \Datarv, N \mid \overline{E}}(\sampler(\Datarv)=i \mid \overline{E}) - \distr(i)\right| + \Pr_{N}(E)\\
%\Pr_{N', \sampler, \Datarv}(\sampler(\Datarv)=o, E) | \\
%& \leq  2 \Pr_{N'}(E) + \sum_{o \in \universe}|\Pr_{\sampler', \Datarv, N' \mid \overline{E}}(\sampler'(\Datarv)=o) - \distr(o)|  \\
%& \leq 2 e^{-n/6} + \sum_{o \in \universe}\Big|\E_{N' \mid \overline{E}} \Big [\Pr_{\sampler', \Datarv}(\sampler'(\Datarv)=o) - \distr(o) \Big]\Big|\\
%&  \leq  2 e^{-n/6} + \E_{N' \mid \overline{E}} \Big [ \sum_{o \in \universe} \Big| \Pr_{\sampler', \Datarv}(\sampler'(\Datarv)=o) - \distr(o) \Big| \Big]\\
& \leq \frac 12 \sum_{i \in \universe}\left|\Pr_{\sampler, \Datarv}(\sampler(\Datarv)=i \mid \overline{E}) - \distr(i)\right| + \Pr_{N}(E)\\
%& \leq \frac 12 \sum_{o \in \universe}\Big|\Pr_{\posampler, \Datarv}(\sampler(\Datarv)=o \mid \overline{E}) - \distr(o)\Big| +   {e^{-n/6}} \\
& \leq   \alpha +{e^{-n/6}},
\end{align*} 
where the first equality is by the definition of total variation distance, the second equality is because $\Pr(\overline{E}) + \Pr(E) = 1$ and because $\Pr(a) = \Pr(a,E) + \Pr(a,\overline{E})$ for every event $a$, the first inequality is because of the triangle inequality, the third equality is by the product rule
%because $\Pr(a,b) = \Pr(a \mid b)\Pr(b)$ for any events $a$ and $b$ 
and by marginalizing over the outputs, the fourth equality is because when $N > n$, $\posampler$ sets the output to be $\sampler(\Datarv)$, the second inequality is by the fact that $\Pr_{N}(\overline{E}) \leq 1$, and the final inequality is by (\ref{eq:event_tail}) and the fact that the sampler $\sampler$ is $\alpha$-accurate when it gets any fixed number of samples larger than $n$. 
Hence, $\posampler$ is $(\alpha + e^{-n/6})$-accurate on $\class$. 
\end{proof}

\begin{comment}
\begin{lemma}\label{lem:poisson2} 
If there exists an $(\eps, \delta)$-differentially private sampler $\posampler$ that is $\alpha$-accurate on distribution class~$\class$ for datasets of size $Po(n)$, then there exists an $(\eps, \delta)$-differentially private sampler  $\sampler$ that is $(\alpha + e^{-n/6})$-accurate on class $\class$ for datasets of size $2n$.
\end{lemma}

\begin{proof}
Algorithm~\ref{alg:poisson} is the desired sampler $\sampler$. It is $(\eps, \delta)$-differentially private since $\sampler$ is $(\eps, \delta)$-differentially private. Let $\Datarv$ represent the random variable corresponding to the dataset fed to \sampler.
\begin{algorithm}
    \caption{Sampler \sampler with dataset size $2n$}
    \label{alg:poisson2}
    \hspace*{\algorithmicindent} \textbf{Input:} dataset $\Datafixed = (\datafixed_1, \ldots, \datafixed_{n})$,  universe $\universe$, oracle access to $(\eps, \delta)$-DP sampler $\sampler$, parameter~$n$\\
    \hspace*{\algorithmicindent} \textbf{Output:} $i\in \universe$
    \begin{algorithmic}[1] % The number tells where the line numbering should start
            \State Fix an element $\el \in \universe$. 
            \State Sample $N \sim \Po(n)$.
            \If {$N > 2n$}  $i=\el$.
            \Else  $\text{ } i \gets \posampler(\Datafixed)$
            \Return $i$.
            \EndIf
    \end{algorithmic}
\end{algorithm}

We use the following tail bound for Poisson random variables \cite{clementpoiss}.
\begin{claim}[\cite{clementpoiss}]\label{lem:poiss_tail}
If $Y\sim \Po(\lambda)$, then $\Pr(Y \geq \lambda + y) \leq e^{-\frac{y^2}{2(\lambda + y)}}$ for all $y>0$.
\end{claim}
Let event $E$ correspond to $N > 2n$. Let $\overline{E}$ represent the complement of $E$. Then 
\begin{equation}\label{eq:event_tail}
\Pr(E) \leq e^{-\frac{n}{4}}
\end{equation}
by an application of Claim~\ref{lem:poiss_tail}. 
Let $\distr \in \class$ and $\Datarv \sim \distr^{\otimes N}$. Then
\begin{align*}
 d_{TV}(\distroutput{\sampler, \distr}, \distr) 
& =\frac 1 2 \sum_{i \in \universe}|\Pr_{N, \sampler}(\sampler(\Datafixed)=i) - \distr(i)| \\
& =\frac 1 2   \sum_{i\in \universe}\left|\Pr_{N, \sampler}(\sampler(\Datafied)=i \land \overline{E}) + \Pr_{N, \sampler, \Datarv}(\sampler(\Datarv)=i \land  E) - \distr(i) \left(\Pr_{N}(\overline{E})+\Pr_{N}(E)\right) \right| \\
& \leq \frac 12 \sum_{i \in \universe}\left(\left|\Pr_{N, \sampler, \Datarv}(\sampler(\Datarv)=i \land  \overline{E}) - \distr(i)\Pr_{N}(\overline{E})\right| +  \Pr_{N, \sampler, \Datarv}(\sampler(\Datarv)=i \land E)+ \Pr_{N}(E)\distr(i)\right)  \\
%& \leq \frac 12 \sum_{o \in \universe}\Big|\Pr_{N, \posampler, \Datarv}(\posampler(\Datarv)=o \mid \overline{E})\Pr(\overline{E}) - \distr(o)\Big| + \frac 12 \Pr_{N'}(E) \\
& = \frac 12 \sum_{i \in \universe}\left|\Pr_{N, \sampler, \Datarv}(\sampler(\Datarv)=i \mid \overline{E})\Pr(\overline{E}) - \distr(i)\Pr_{N}(\overline{E})\right| + 
\frac 12 \cdot(\Pr_{N}(E)+\Pr_{N}(E))\\
%\Pr_{N', \sampler, \Datarv}(\sampler(\Datarv)=o, E) \\
& = \frac 12 \sum_{i \in \universe}\Pr_{N}(\overline{E})\cdot\left|\Pr_{\sampler, \Datarv, N \mid \overline{E}}(\sampler(\Datarv)=i \mid \overline{E}) - \distr(i)\right| + \Pr_{N}(E)\\
%\Pr_{N', \sampler, \Datarv}(\sampler(\Datarv)=o, E) | \\
%& \leq  2 \Pr_{N'}(E) + \sum_{o \in \universe}|\Pr_{\sampler', \Datarv, N' \mid \overline{E}}(\sampler'(\Datarv)=o) - \distr(o)|  \\
%& \leq 2 e^{-n/6} + \sum_{o \in \universe}\Big|\E_{N' \mid \overline{E}} \Big [\Pr_{\sampler', \Datarv}(\sampler'(\Datarv)=o) - \distr(o) \Big]\Big|\\
%&  \leq  2 e^{-n/6} + \E_{N' \mid \overline{E}} \Big [ \sum_{o \in \universe} \Big| \Pr_{\sampler', \Datarv}(\sampler'(\Datarv)=o) - \distr(o) \Big| \Big]\\
& \leq \frac 12 \sum_{i \in \universe}\left|\Pr_{\sampler, \Datarv}(\sampler(\Datarv)=i \mid \overline{E}) - \distr(i)\right| + \Pr_{N}(E)\\
%& \leq \frac 12 \sum_{o \in \universe}\Big|\Pr_{\posampler, \Datarv}(\sampler(\Datarv)=o \mid \overline{E}) - \distr(o)\Big| +   {e^{-n/6}} \\
& \leq   \alpha +{e^{-n/6}},
\end{align*} 
where the first equality is by the definition of total variation distance, the second equality is because $\Pr(\overline{E}) + \Pr(E) = 1$ and because $\Pr(a) = \Pr(a,E) + \Pr(a,\overline{E})$ for every event $a$, the first inequality is because of the triangle inequality, the third equality is by the product rule
%because $\Pr(a,b) = \Pr(a \mid b)\Pr(b)$ for any events $a$ and $b$ 
and by marginalizing over the outputs, the fourth inequality is because when $N > n$, $\posampler$ sets the output to be $\sampler(\datafixed)$, the second inequality is by the fact that $\Pr_{N}(\overline{E}) \leq 1$, and the final inequality is by (\ref{eq:event_tail}) and the fact that the sampler $\sampler$ is $\alpha$-accurate when it gets any fixed number of samples larger than $n$. 
Hence, $\posampler$ is $(\alpha + e^{-n/6})$-accurate on $\class$. 
\end{proof}
\end{comment}

\ifnum\neurips=1
\subsection{Privacy amplification for samplers}\label{sec:privacy-amplification}
\else 
\subsection{Privacy Amplification for Samplers}\label{sec:privacy-amplification}
\fi
Our second general transformation shows how to amplify privacy (that is, decrease privacy parameters) of a sampler by subsampling its input. The transformation does not affect the accuracy. The following lemma quantifies how the privacy parameters and the dataset size are affected by privacy amplification. This result is needed in the proof of the lower bound for $k$-ary distributions, because the main technical lemma in that proof (Lemma~\ref{lem:main-k-ary-lb}) only applies to samplers with small $\eps.$ It is well known that subsampling amplifies differential privacy (see, e.g., \cite{NissimRS07,li2012sampling}).
\begin{lemma}\label{lem:amplify}
Fix $\eps\in(0,1], \delta \in (0,1),$ and $\beta \in (0, 1)$. If there exists an $(\eps,\delta)$-differentially private sampler~$\sampler$ that is $\alpha$-accurate on distribution class \class  for datasets of size distributed as $\Po(n)$ then there exists an $(\eps\beta, \delta \frac{\beta}2)$-differentially private sampler $\sampler_{\beta/2}$ that is $\alpha$-accurate on class \class for datasets of size distributed as $\Po\Big(n\cdot \frac{2}{\beta}\Big)$.
\end{lemma}
 
\begin{proof}
We construct $\sampler_{\beta/2}$ from sampler $\sampler$ as follows: Given a dataset $\Datafixed$, sampler $\sampler_{\beta/2}$ subsamples each record in $\Datafixed$ independently with probability $\beta/2$ to get a new dataset $\Datafixed^*$ and then returns $\sampler(\Datafixed^*)$.

First, we argue that if sampler $\sampler$ is $(\eps,\delta)$-differentially private, then sampler $\sampler_{\beta/2}$  is $(\eps\beta, \delta\frac{\beta}{2})$-differentially private. 
% Sofya: I don't think we need a paragraph break here. O.w., we get a 1-sentence paragraph.
This follows from \cite[Theorem 1]{li2012sampling} which we state as Theorem~\ref{thm:LQS12} in the appendix. By Theorem~\ref{thm:LQS12}, algorithm 
$\sampler_{\beta/2}$ is $(\eps',\delta\cdot\frac \beta 2)$-differentially private with

\begin{align*}
    \eps'=\ln\Big(1 + \frac{\beta}{2}\cdot(e^{\eps} - 1 )\Big) 
    \leq\ln\Big(1 + \frac{\beta}{2}\cdot(2\eps )\Big)
    =\ln(1+\beta\eps)
    \leq \ln(e^{\beta\eps}) = \eps\beta,
\end{align*}
where the inequalities hold because $e^{\eps} -1\leq 2\eps$ for all $\eps \leq 1$ and  $1+\eps\beta \leq e^{\eps\beta}$ for all $\eps\beta$. 

Next, we argue that if sampler $\sampler$ is  $\alpha$-accurate on class \class for datasets of size $\Po(n)$, then sampler $\sampler_{\beta/2}$ is $\alpha$-accurate on class \class for datasets of size $\Po(n\cdot \frac2{\beta})$. 
% Sofya: removed paragraph break.
Suppose $\sampler_{\beta/2}$ is given a sample $\Datarv$ of size $\Po\big(n\cdot \frac{2}{\beta}\big)$ drawn i.i.d.\ from some distribution $P$.
Then the size of $\Datarv^*$, obtained by subsampling each entry of $\Datarv$ with probability $\beta/2$, has distribution $\Po(n)$, and entries of $\Datarv^*$ are i.i.d.\ from $P$. Since the output distributions of $\sampler_{\beta/2}(\Datarv)$ and $\sampler(\Datarv^*)$ are the same, the accuracy guarantee is the same for both algorithms.
\end{proof}

\ifnum\neurips=1
\subsection{General samplers to frequency-count-based samplers}\label{sec:frequency-counts}
\else 
\subsection{General Samplers to Frequency-Count-Based Samplers}\label{sec:frequency-counts}
\fi




Our final transformation shows that algorithms that sample from distribution classes with certain symmetries can be assumed without loss of generality to use only frequency counts of their input dataset in their decisions. Before stating this result (Lemma~\ref{lem:frequency-counts}), we define {\em frequency counts}, {\em frequency-count-based algorithms}, and the type of symmetries relevant for the transformation.

\begin{definition}[Frequency Counts]
Given a dataset $\Datafixed$ and an integer $j\geq 0$, let $F_j(\Datafixed)$ denote the number of elements that occur $j$ times in $\Datafixed$.
The vector $F(\Datafixed)$ of {\em frequency counts} of a dataset $\Datafixed$ of size $n$ is $(F_{0}(\Datafixed), \dots ,F_{n}(
\Datafixed) ).$
\end{definition}

\begin{definition}[Frequency-count-based algorithms]\label{def-freqcountbased}
A sampler is {\em frequency-count-based} if, for every element $i$ in the universe, the probability that the algorithm outputs $i$ when given a dataset $\Datafixed$ only depends on $j$, the number of occurrences of $i$ in $\Datafixed$, and on $F(\Datafixed)$. If $\Datafixed$ contains an element $i \in \universe$ that occurs $j$ times in $\Datafixed$, then let $p_{j,F(\Datafixed)}$ denote the probability that the sampler outputs $i$; otherwise, let $p_{j,F(\datafixed)} = 0$.
\end{definition}
%In this section, we show that every sampler for the class of $k$-ary distributions can be transformed into a frequency-count-based sampler with the same guarantees. In fact, this transformation works for a more general distribution class, defined next.
Next, we define the type of distribution classes for which our transformation works.
\begin{definition}\label{def:label-invariant}
A class \class of distributions over a universe $\universe$ is \emph{label-invariant} if for all distributions $\distr \in \class$ and permutations $\pi:\universe \to \universe$, we have $\pi(\distr) \in \class$, where $\pi(\distr)$ is the distribution obtained by applying permutation $\pi$ to the support of $\distr$ (that is, $\Pr_{\pi(\distr)}(\el) = \Pr_{\distr}(\pi^{-1}(\el))$ for all $\el \in \universe$).
\end{definition}

Examples of label-invariant classes include the class of all Bernoulli distibutions and, more generally, the class of all $k$-ary distributions, for any $k.$

\begin{lemma}\label{lem:frequency-counts} Fix  a label-invariant distribution class $\class$. If there exists an  $(\eps, \delta)$-differentially private sampler~$\sampler$ that is  $\alpha$-accurate on \class with a particular distribution on the dataset size, then there exists  an $(\eps, \delta)$-differentially private frequency-count-based sampler $\fpsampler$ that is $\alpha$-accurate on \class with the same distribution on the dataset size.
\end{lemma}

\begin{proof}
Consider an $\alpha$-accurate sampler $\sampler$ for the class $\class$. %that is not frequency-count-based. (Sofya: commented out, since this restriction is not needed.)
Construct the sampler $\fpsampler$ given in Algorithm~\ref{alg:frequency-counts}.

\begin{algorithm}
    \caption{Sampler \fpsampler}
    \label{alg:frequency-counts}
    \hspace*{\algorithmicindent} \textbf{Input:} dataset $\Datafixed$, universe $\universe$\\
    \hspace*{\algorithmicindent} \textbf{Output:} $i\in \universe$
    \begin{algorithmic}[1] % The number tells where the line numbering should start
            \State Choose a permutation $\pi : \universe \rightarrow \universe$ uniformly at random. \label{step:randperm}
%            \State $j \gets \sampler(\pi(\Datafixed))$
            \State \Return $\pi^{-1}(\sampler(\pi(\Datafixed)))$
    \end{algorithmic}
\end{algorithm}
First, we show that sampler \fpsampler is $\alpha$-accurate for \class. For all $\distr \in \class$ and $\Datafixed \sim \distr$, denote by $\distroutput{\fpsampler(\Datafixed)}$ the distribution of outputs of $\fpsampler(\Datafixed)$. Define $\distroutput{\sampler(\Datafixed)}$ similarly. Then, for a fixed permutation $\pi,$ 
\begin{align}
    d_{TV} (\distroutput{\fpsampler(\Datafixed)}, \distr) 
    & =  d_{TV}\left(\pi(\distroutput{\fpsampler(\Datafixed)}), \pi(\distr)\right)\nonumber\\
    & = d_{TV}\left(\distroutput{\sampler(\pi(\Datafixed))}, \pi(\distr)\right)
%    & =  d_{TV}\left(\pi^{-1}(\distroutput{\fpsampler(\pi(\Datafixed))}), \distr\right)\nonumber\\
%    & = d_{TV}\left(\distroutput{\fpsampler(\pi(\Datafixed))}, \pi(\distr)\right)
     \leq \alpha, \label{calc:alpha_acc}
\end{align}
where the equalities hold by the definition of $\pi$ and $\fpsampler$, and the inequality holds because $\pi(\Datafixed)\sim \pi(\distr)$ and since $\class$ is label-invariant and $\sampler$ is $\alpha$-accurate for $\class$.
%
 For a fixed $\pi'$, let $\distroutput{\fpsampler(\Datafixed)|\pi'}$ represents the output distribution of \fpsampler conditioned on $\pi'$ being chosen in Step~\ref{step:randperm} of Algorithm~\ref{alg:frequency-counts}.
 
 For a uniformly chosen $\pi$, 
\begin{align*}
    d_{TV}(\distroutput{\fpsampler(\Datafixed)}, \distr) 
    & = d_{TV} \left(\E_\pi[\distroutput{\fpsampler(\Datafixed)|\pi}], \distr \right)\\
    & \leq \E_\pi [d_{TV}(\distroutput{\fpsampler(\Datafixed)|\pi}, \distr)] \quad\quad \text{By the triangle inequality} \\
    & \leq \max_{\pi} \{ d_{TV}(\distroutput{\fpsampler(\Datafixed)|\pi}{}, \distr)\}\\
    & = \max_{\pi} \{ d_{TV}(\distroutput{\sampler(\pi(\Datafixed))}, \pi(\distr)) \}
    \leq \alpha. \quad\quad \text{By (\ref{calc:alpha_acc})}
\end{align*}
Thus, algorithm $\fpsampler$ is $\alpha$-accurate for \class. 

Next, we show that $\fpsampler$ is frequency-count-based by proving that for all permutations $\pi^*$ on the universe and for all $i$ in the universe, $\Pr[\fpsampler(\pi^*(\Datafixed)) = \pi^*(i)] =
\Pr[\fpsampler(\Datafixed) = i] $. Let $\pi_0 = \pi \circ \pi^*$. We can characterise the output distribution of \fpsampler for a fixed $\Datafixed$ as follows
\begin{align*}
    \Pr[\fpsampler(\pi^*(\Datafixed)) = \pi^*(i)]
    & =  \frac{1}{|\universe|!} \sum_{\pi_0 \in [\universe!]} \Pr[\sampler(\pi_0\circ \pi^*(\Datafixed)) = \pi_0\circ \pi^*(i)]\\
    & =  \frac{1}{|\universe|!} \sum_{\pi_0 \circ \pi^*\in [\universe!]} \Pr[\sampler(\pi_0\circ \pi^*(\Datafixed)) = \pi_0\circ \pi^*(i)]\\
     & =  \frac{1}{|\universe|!} \sum_{\pi\in [\universe!]} \Pr[\sampler(\pi(\Datafixed)) = \pi(i)] \\
    & = \frac{1}{|\universe|!} \sum_{\pi\in [\universe!]} \Pr[\pi^{-1}(\sampler(\pi(\Datafixed))) = i]\\
    & = \Pr[\fpsampler(\Datafixed) = i]
\end{align*}
The third equality holds since the permutations $\pi, \pi^*$ are bijections. Thus \fpsampler is frequency-count-based.

Furthermore, the sizes of the input datasets to \fpsampler and \sampler are identical, so if \sampler takes a sample with size distributed according to some distribution $\distr$, then so does the frequency-count-based sampler \fpsampler.

Finally, \fpsampler inherits the privacy of \sampler since it simply permutes its input dataset before passing it to the $(\eps, \delta)$-differentially private sampler \sampler. Its output is a postprocessing of the output it receives from \sampler.
\end{proof}


\ifnum\neurips=1
\section{k-ary discrete distributions}
\else 
\section{\texorpdfstring{$k$}{k}-ary Discrete Distributions}
\fi 
We consider privately sampling from the class of discrete distribution over $[k] := \{1, 2, \ldots, k\}$. We call this class $\class_k$. We prove in this section that the sample complexity of this task is $\Theta(k/\alpha \eps)$, corresponding to Theorems~\ref{thm:kupperb} and~\ref{thm:intro-k-ary-lb}. The proof of Theorem~\ref{thm:intro-k-ary-lb} is split into two cases: Theorem~\ref{thm:bernoulli-lb} deals with the case where $k=2$ and Theorem~\ref{thm:k-ary-lb} deals with the case where $k \geq 3$. We combine these theorems appropriately at the end of Section~\ref{sec:k-ary-lb-final}.
 
\ifnum\neurips=1
\subsection{Optimal private sampler for \texorpdfstring{$k$}{k}-ary distributions}\label{sec:k-ary-ub}
\else 
\subsection{Optimal Private Sampler for \texorpdfstring{$k$}{k}-ary Distributions}\label{sec:k-ary-ub}
\fi

In this section, we prove Theorem~\ref{thm:kupperb}.

\begin{proof}[Proof of Theorem~\ref{thm:kupperb}]
Algorithm~\ref{alg:kary} is the desired $(\eps,0)$-differentially private sampler for $\class_k$. The algorithm computes the empirical distribution, adds Laplace noise to each count in $[k]$, and then projects the result onto $\class_k$ in order to sample from the resulting distribution. The $L_1$ projection onto $\class_k$ is defined as $L_1Proj(\distr) = \argmin_{\distr' \in \class_k} \|\distr - \distr' \|_1$.
\begin{algorithm}
    \caption{Sampler $\karysampler$ for $\class_k$}
    \label{alg:kary}
    \hspace*{\algorithmicindent} \textbf{Input:} dataset $\Datafixed \in [k]^n, \text{ parameter } \eps>0$\\
    \hspace*{\algorithmicindent} \textbf{Output:} $i\in [k]$
    \begin{algorithmic}[1] % The number tells where the line numbering should start
            \For{$j=1$ \text{to} $k$}
            \State $\hat{\distr_j} \gets \frac{1}{n}\sum_{i=1}^n \indicator_{[\datafixed_i=j]}$ \Comment{Compute the empirical distribution}
            \State $Z_j\sim \Lap(2/\eps n)$ \Comment{Sample Laplace noise}
            \State $\hat{\distr}^{noisy}_j \gets \hat{\distr}_j + Z_j$ \Comment{Compute noisy empirical estimate}
            \EndFor
            \State $\tilde{\distr} \gets L_1Proj(\hat{\distr}^{noisy})$ \Comment{Do $L_1$ projection of private empirical estimate to $\class_k$}
            \State $i\sim \tilde{\distr}$ \Comment{Sample from resulting distribution}
            \State \Return $i$
    \end{algorithmic}
\end{algorithm}


First, we argue that Algorithm~\ref{alg:kary} is $\alpha$-accurate. Let $\distr$ be the input distribution represented by a vector of length $k$. As defined in Algorithm~\ref{alg:kary}, let $\hat{\distr}$ be the empirical distribution, $\hat{\distr}^{noisy}$ be the empirical distribution with added Laplace noise, and $\tilde{\distr}$ be the distribution obtained after applying $L_1$ projection (all represented by vectors of length $k$). Then $\E_{\Datarv}[\hat{\distr}] = \distr$, since $\hat{\distr}$ is the empirical distribution of a dataset sampled from~$P$. Let $\distroutput{\karysampler, \distr}$ be the distribution of the sampler's output for dataset $\Datarv \sim \distr^n$. Then $\distroutput{\karysampler, \distr} = \E_{\Datarv, \sampler} [\tilde{\distr}],$ since the output of \karysampler is sampled from $\tilde{\distr}$. We get    
\begin{align}
   d_{TV}(\distroutput{\karysampler, \distr},\distr)
   & = \frac{1}{2}\Big\|\distr-\distroutput{\karysampler, \distr}\Big\|_1 
   = \frac{1}{2}\Big\|\distr - \E_{\Datarv, \karysampler} [\tilde{\distr}]\Big\|_1 
    = \frac{1}{2}\Big\| \E_{\Datarv, \karysampler}[ \hat{\distr} - \tilde{\distr}] \Big\|_1 \nonumber \\
   & \leq \frac{1}{2} \cdot \E_{\Datarv, \karysampler}\left[ \|\hat{\distr} - \tilde{\distr}\|_1 \right], \label{eq:upperbound_jensens}
\end{align}
where we applied Jensen's inequality in the last step of the derivation. Additionally,
\begin{align*}
\|\hat{\distr} - \tilde{\distr}\|_1  
& = \|\tilde{\distr} - \hat{\distr}^{noisy}+ \hat{\distr}^{noisy} - \hat{\distr}\|_1  \\
& \leq \|\hat{\distr} - \hat{\distr}^{noisy}\|_1 + \|\tilde{\distr} - \hat{\distr}^{noisy}\|_1 \quad & \text{By the triangle inequality} \\
& \leq 2\|\hat{\distr} - \hat{\distr}^{noisy}\|_1. \quad & \text{Definition of $L_1$ projection} 
\end{align*}
Substituting this into (\ref{eq:upperbound_jensens}), we get that
\begin{equation*}
   d_{TV}(\distr,\distroutput{\karysampler, \distr}) \leq \frac{1}{2} \cdot\E_{\Datarv, \karysampler}\Big[ 2\|\hat{\distr} - \hat{\distr}^{noisy}\|_1\Big] 
   = \E_{\karysampler}\Big[\sum_{j\in[k]} |Z_j|\Big] 
   =  \frac{2k}{n\eps}, 
\end{equation*}
since the expectation of the absolute value of a random variable distributed according to the Laplace distribution $\Lap(\frac{2}{n \eps})$ is $\frac{2}{n \eps}$. 
%
We conclude that with $n\geq\frac{2k}{\alpha \eps}$, Algorithm~\ref{alg:kary} is $\alpha$-accurate. 

Next, we show that Algorithm~\ref{alg:kary} is $(\eps, 0)$-differentially private. The sensitivity of a function $f: \mathcal{X}^n \to \mathbb{R}^d$ is defined as $\max_{\Datafixed,\Datafixed' \in \mathcal{X}^n, \|\Datafixed - \Datafixed'\|_0 = 1} \|f(\Datafixed) - f(\Datafixed')\|_1$. 
Recall that $\hat{P}$ is the empirical distribution of dataset $\Datafixed$ (represented by a vector of length $k$).
Changing one element of $\Datafixed$ can change only two components of $\hat{P}$ by $\frac{1}{n}$ each. Hence, the sensitivity of the empirical distribution is $\frac{2}{n}$. Algorithm~\ref{alg:kary} adds Laplace noise scaled to the sensitivity of the empirical distribution to each component of the empirical distribution and then post-processes the output. This is an instantiation of the Laplace Mechanism (proved in \cite{DworkMNS06j} to be $(\eps, 0)$-differentially private) followed by post-processing. Algorithm~\ref{alg:kary} is $(\eps, 0)$-differentially private since differential privacy is preserved under post-processing.
\end{proof}

%%%%%%%%%%%%%%%%%%%%%%%%%%%%%%%%%%%%%%%%%%%%%%%
%
%%%%%%%%%%%%%%%%%%%%%%%%%%%%%%%%%%%%%%%%%%%%%%%

\ifnum\neurips=1
\subsection{The lower bound for the class of Bernoulli distributions}
\else 
\subsection{The Lower Bound for the Class of Bernoulli Distributions}
\fi

We consider the class $\cB$ of Bernoulli distributions with an unknown bias $p.$ For all $p\in[0,1]$, distribution $\Ber(p)\in \cB$ outputs 1 with probability $p$ and 0 with probability $1-p$. Algorithm~\ref{alg:kary} for the special case of $k=2$ shows that $O(\frac 1 {\alpha\eps})$ samples are sufficient for $(\eps,0)$-differentially private $\alpha$-accurate sampling from $\cB$. In this section, we show that this bound is tight, even for $(\eps,\delta)$-differentially private samplers.



\begin{theorem}\label{thm:bernoulli-lb}
If $\eps \in (0,1],\alpha\in(0,1)$, and $\delta \leq \alpha\eps$, then every  $(\eps, \delta)$-differentially private sampler that is $\alpha$-accurate on the class $\cB$ of Bernoulli distributions requires $\Omega(\frac 1{\alpha\eps})$ samples.
\end{theorem}
\begin{proof}
The following lemma captures how differential privacy affects a sampler for Bernoulli distributions.
\begin{lemma} \label{lem:bern_sampler} 
Suppose $\delta \leq \alpha\eps$. If sampler \sampler is  $(\eps, \delta)$-differentially private and $\alpha$-accurate on the class $\cB$ of Bernoulli distributions then, for all $t\in[n],$
\begin{equation*}
    \Pr[\sampler(1^t0^{n-t}) = 1] \leq 2\alpha e^{\eps t}.
\end{equation*}
\end{lemma}

\begin{proof}
Fix $n$ and $t\in[n].$
Since $\sampler$ is $\alpha$-accurate on $\Ber(0),$ we have $\Pr[\sampler(0^n)=1]\leq\alpha.$
We start with the dataset $1^t0^{n-t}$ and replace 1s with 0s one character at a time until we reach $0^n.$
Since $\sampler$ is $(\eps, \delta)$-differentially private, its output distribution does not change dramatically with every replacement. Specifically,
\begin{align*}
    \Pr[\sampler(1^t 0^{n-t}) = 1] &\leq e^\eps \cdot \Pr[\sampler(1^{t-1} 0^{n-t+1}) = 1] + \delta\\
    &\leq  e^\eps(e^\eps \cdot \Pr[\sampler(1^{t-2} 0^{n-t+2}) = 1] + \delta)+\delta \leq\dots\\
     & \leq e^{\eps t}\cdot \Pr[\sampler(0^n)=1] + \delta\cdot \sum_{i = 0}^{t-1} e^{\eps t} 
     = e^{\eps t}\cdot \Pr[\sampler(0^n)=1] +\delta \cdot \frac{e^{\eps t} -1}{e^{\eps} - 1}\\
    & \leq  e^{\eps t}\cdot\alpha + \delta \cdot \frac{e^{\eps t} -1}{e^{\eps} - 1}
    \leq e^{\eps t}\Big(\alpha +\frac \delta \eps\Big)
    \leq 2\alpha e^{\eps t},
\end{align*}
where the last two inequalities hold because $e^\eps-1\leq\eps$ for all $\eps$ and since $\delta\leq\alpha\eps.$
\end{proof}


Consider a sampler \sampler, as described in Theorem~\ref{thm:bernoulli-lb}. By Lemma~\ref{lem:frequency-counts}, since the class $\cB$ is label-invariant, we may assume w.l.o.g.\ that \sampler is frequency-count-based. In particular, the output distribution of \sampler is the same on datasets with the same number of 0s and 1s.

Consider a Bernoulli distribution with $p= 10\alpha$. Let $T$ be a random variable that denotes the number of 1s in $n$ independent draws from $\Ber(p)$.
Then $T$ has binomial distribution $Bin(n,10\alpha).$
By Claim~\ref{clm:bern_acc_eq}  and  $\alpha$-accuracy of \sampler for $\Ber(p)$, we get%\srnote{Should the expectation be just of A(X), not of A(X)=1. After that the expectation is over coins of A and T, not just T?}
\begin{align}
    9\alpha \leq \E_{\Datarv\sim (\Ber(p))^{\otimes n}, \sampler}[\sampler(X)]
    &=\E_{T\sim Bin(n,10\alpha), \sampler}[\sampler(1^{T} 0^{n-T})]\nonumber\\
    &\leq \E_{T\sim Bin(n,10\alpha)}[2\alpha e^{\eps T}]
    =2\alpha (10\alpha(e^\eps-1)+1)^n \label{eq:ber2}\\
    &\leq 2\alpha (20\alpha\eps+1)^n
    \leq 2\alpha e^{20\alpha\eps n},\label{eq:ber3}
\end{align}
where, to get (\ref{eq:ber2}), we used Lemma~\ref{lem:bern_sampler} and then the moment generating function of the binomial distribution; in  (\ref{eq:ber3}), we used that $e^\eps-1\leq 2\eps$ for all $\eps\in (0,1]$ and, finally, that $x+1\leq e^x$ for all $x$ (applied with $x=20\alpha\eps$).
We obtained that $9\alpha \leq 2\alpha \cdot e^{200\alpha \eps n}$, so $n\geq \frac {20}{\ln 4.5} \frac 1 {\alpha\eps}$ samples are required. This completes the proof of Theorem~\ref{thm:bernoulli-lb}.
\end{proof}


\ifnum\neurips=1
\subsection{The lower bound for the class of $k$-ary distributions}\label{sec:kary}
\else 
\subsection{The Lower Bound for the Class of $k$-ary Distributions}\label{sec:kary}
\fi

In this section, we prove Theorem~\ref{thm:intro-k-ary-lb} by providing a lower bound for the universe size at least 3 (Theorem~\ref{thm:k-ary-lb}) and combining it with the previously proved lower bound for the binary case (Theorem~\ref{thm:bernoulli-lb}). The crux of the proof of Theorem~\ref{thm:k-ary-lb} is presented in Section~\ref{sec:k-ary-lb-frequency-count-based}, where we state and prove the lower bound for the special case of Poisson, frequency-count-based samplers with sufficiently small~$\eps$ (that is, a strong privacy guarantee). In Section~\ref{sec:k-ary-lb-final}, we complete the proof of the theorem by generalizing the lower bound from Section~\ref{sec:k-ary-lb-frequency-count-based}
with the help of the transformation lemmas (Lemmas~\ref{lem:poisson},~\ref{lem:amplify}, and~\ref{lem:frequency-counts}) that allow us to convert general samplers to Poisson, frequency-count-based algorithms with small privacy parameter~$\eps.$

\ifnum\neurips=1
\subsubsection{The lower bound for Poisson, frequency-count-based samplers with small $\eps$}\label{sec:k-ary-lb-frequency-count-based}
\else 
\subsubsection{The Lower Bound for Poisson, Frequency-Count-Based Samplers with Small $\eps$}\label{sec:k-ary-lb-frequency-count-based}
\fi

%\as{TODO: Make corollary 4.8 into lemma 4.3 and remove current 4.3. Consistently use \carb for the set of distributions with mass $1-\alpha$ on one element $j \in [2k+1]$, and the remaining mass uniform over a size-$k$ subset of $\{1,...,2k\}$.}

We start by defining the class of distributions used in our lower bound.
\begin{definition}\label{def:ksubclass}
Let \carb denote the the set of distributions with mass $1-60\alpha$ on one {\em special} element $s \in [2k+1]$, and the remaining mass uniform over a size-$k$ subset of $[2k+1] \setminus \{s\}$.
\end{definition} 
Observe that the class $\carb$ is label-invariant (Definition~\ref{def:label-invariant}). This allows us to focus on a simple class of sampling algorithms, called frequency-count-based algorithms (Definition~\ref{def-freqcountbased}), to prove our lower bound.

\begin{lemma}\label{lem:main-k-ary-lb} 
 Fix $k,n \in{\mathbb N}, \alpha \in (0,0.02], \eps \in (0,1/\ln (1/\alpha)],$ and $\delta \in [0, 0.1\alpha\eps/k]$. Let \carb denote the subclass of discrete distributions over the universe $[2k+1]$ specified in the previous paragraph. Let $\alpha^*=60 \alpha$.
 If sampler \sampler is $(\eps, \delta)$-differentially private, frequency-count-based, and $\alpha$-accurate on class \carb with dataset size distributed as $\Po(n)$, then $n > \frac 1 {60}\cdot \frac{k}{\alpha \eps}$. 
\end{lemma}

\begin{proof}
We consider the following distribution $\distr\in \carb.$ Let $\alpha^*=60 \alpha$. Fix a set $S^*\subset[2k]$ of size $k.$ Distribution $\distr$ has mass $\alpha^*/k$ on each element in $S^*$ and mass $1-\alpha^*$ on the {\em special} element $2k+1.$ 
%
Consider a sampler \sampler satisfying the conditions of Lemma~\ref{lem:main-k-ary-lb}. Let $\distroutput{\sampler, \distr}$ denote the output distribution of $\sampler$ when the dataset size $N \sim \Po(n)$ and the dataset $\Datarv \sim \distr^{\otimes N}$.
Observe that
\begin{eqnarray}\label{eq:main-dist-lb-delta}
d_{TV}(\distroutput{\sampler,\distr}, \distr)
  \geq \Pr_{\substack{N\sim\Po(n) \\ \Datarv\sim  \distr^{\otimes N}}}[\sampler(\Datarv) \notin Supp(\distr)].  
\end{eqnarray}
We will show that when $n\leq \frac k{60\alpha \eps}$  and $\eps$ and $\delta$ are in the specified range, the right-hand side of (\ref{eq:main-dist-lb-delta}) is large. 

We start by deriving a lower bound on $\Pr[\sampler(\Datafixed)\notin Supp(\distr)]$ for a fixed dataset $\Datafixed$ of a fixed size $N$. Since \sampler is frequency-count-based, the probability that it outputs a specific element in $[2k]$ that occurs $0$ times in $\Datafixed$ is $p_{0,F(\Datafixed)}$. Let $F^*_0(\Datafixed)$ denote the number of elements in $[2k]$ that occur $0$ times in $\Datafixed$ (note that the special element $2k+1$ is excluded from this count).
 By definition, $F^*_0(\Datafixed)\leq 2k.$ Consequently,
\begin{equation}\label{eq:notinsupport-delta}
\Pr[\sampler(\Datafixed) \notin Supp(\distr)] = k\cdot p_{0,F(\Datafixed)} \geq \frac{1}{2}\cdot  F^*_0(\Datafixed)\cdot p_{0,F(\Datafixed)}.
\end{equation}

The next claim uses the fact that sampler \sampler is $(\eps,\delta)$-differentially private to show that the probability $p_{j,F(\Datafixed)}$ (that \sampler outputs some specific element in the universe $\universe$  that appears $j$ times in the dataset $\Datafixed$) cannot be much larger than the probability that \sampler outputs a specific element in $\universe$ that does not appear in $\Datafixed$.

\begin{claim}\label{clm:epsdelfing}
For every $(\eps, \delta)$-differentially private sampler and every frequency count $f \in \mathbb{Z}^*$ and index $j \in \universe$,
\begin{equation}\label{eq:grouppriv}
 p_{j,f} 
 \leq e^{\eps j} \left( p_{0,f} + \frac{\delta}{\eps} \right).
\end{equation}

\end{claim}

\begin{proof}
Consider a frequency count $f$ and a dataset $\Datafixed$ with $F(\Datafixed) = f$. Note that (\ref{eq:grouppriv}) is true trivially for all $j$ such that $F_j(\Datafixed)=0$ because, in that case, $p_{j,F(\Datafixed)}$ is set to $0$.

Fix any $j \in \universe$ such that $F_j(\Datafixed) > 0$. Let $a$ be any element in $\universe$ that occurs $j$ times in the dataset $\Datafixed$. Let $b$ be any element in $\universe$ that is not in the support of the distribution $\distr$. Let $\Datafixed|_{a\rightarrow b}$ denote the dataset obtained by replacing every instance of $a$ in the dataset $\Datafixed$ with element $b$. By group privacy \cite{DworkMNS06j},
\begin{equation}\label{eq:group_privacy-delta}
\Pr[\sampler(\Datafixed) = a] \leq e^{\eps j} \Pr[\sampler( \Datafixed|_{a \rightarrow b}) = a ] + \delta \cdot \frac{e^{\eps j} -1}{e^{\eps} - 1}.
\end{equation}
Note that the dataset $\Datafixed|_{a\rightarrow b} $ does not contain element $a$, since we've replaced every instance of it with $b$. Importantly, $F(\Datafixed|_{a \rightarrow b}) = F(\Datafixed)$ because $b$ is outside of the support of the distribution $\distr$ and hence does not occur in $\Datafixed$. Since $\sampler$ is frequency-count-based and $F(\Datafixed) = F(\Datafixed|_{a \rightarrow b})$, we get that $p_{0,F(\Datafixed)} = p_{0,F(\Datafixed|_{a \rightarrow b})}$. 
%
Substituting this into (\ref{eq:group_privacy-delta}) and using the fact that $e^{\eps}-1\geq \eps$ for all $\eps$, we get that
%when $F_j(\Datafixed) > 0$,
\begin{equation*}%\label{eq:relabel_assumption}
p_{j,F(\Datafixed)} \leq e^{\eps j}\cdot p_{0,F(\Datafixed)} + \delta \cdot \frac{e^{\eps j} - 1}{e^\eps -1}
\leq e^{\eps j} \left( p_{0,f} + \frac{\delta}{\eps} \right).
\end{equation*}
This completes the proof of Claim~\ref{clm:epsdelfing}.
\end{proof}

For a dataset $\Datafixed$ and $i\in[2k+1]$, let $N_i(\Datafixed)$ denote the number of occurrences of element $i$ in $\Datafixed$.
Next, we give a lower bound on $\Pr[\sampler(\Datafixed) \notin Supp(\distr)]$ in terms of the counts $N_i(\Datafixed)$.

\begin{claim}\label{clm:nonsupport-lb-fixed-delta}
Let $N\in\mathbb{N}$ and $\Datafixed \in [2k+1]^N$ be a fixed dataset. Set %$Y = \sum_{j=1}^N F_j(\Datafixed)e^{\eps j}$.
$Y=\sum_{i \in S^*}  \left[ e^{N_i(\Datafixed) \eps} \right]$.
Then
$$\Pr[\sampler(\Datafixed) \notin Supp(\distr)] \geq\frac{1}{2}
\cdot\frac{\Pr[\sampler(\Datafixed) \in [2k]]}{1+Y/k} -\frac{k\delta}\eps.$$
\end{claim}

\begin{proof}
In the following derivation, we use the fact that that an element $j\in[2k]$ that appears $j$ times in $\Datafixed$ is returned by \sampler with probability $p_{j,F(\Datafixed)}$, then split the elements into those that do not appear in $\Datafixed$ and those that do, next use the fact that all elements from $[2k]$ that appear in $\Datafixed$ must be in $S^*$, then apply Claim~\ref{clm:epsdelfing}, and finally substitute $Y$ for $\sum_{i \in S^*}  \left[ e^{N_i(\Datafixed) \eps} \right]$:
%$\sum_{j=1}^N F_j(\Datafixed)e^{\eps j}$:
\begin{align*}
\Pr[\sampler(\Datafixed) \in[2k]] 
&=\sum_{i\in[2k]}  p_{N_i(\Datafixed),F(\Datafixed)} 
=F^*_0(\Datafixed)\cdot p_{0,F(\Datafixed)}+\sum_{i\in[2k]\cap\Datafixed}  p_{N_i(\Datafixed),F(\Datafixed)} \\
&\leq F^*_0(\Datafixed)\cdot p_{0,F(\Datafixed)}+\sum_{i\in S^*}  p_{N_i(\Datafixed),F(\Datafixed)}\\
&\leq F^*_0(\Datafixed)\cdot p_{0,F(\Datafixed)}+\sum_{i\in S^*}  p_{0,F(\Datafixed)}\cdot \left(e^{\eps N_i(\Datafixed)} + \frac{\delta}{\eps} \right)\\
%\leq \sum_{j=0}^N F_j(\Datafixed) \cdot e^{j \eps}\left(p_{0,F(\Datafixed)} +  \frac{\delta}{\eps}\right)
&\leq \Big(F^*_0(\Datafixed)+Y\Big)\Big(p_{0,F(\Datafixed)}+  \frac{\delta}{\eps}\Big).
\end{align*} 
We rearrange the terms to get
$$
p_{0,F(\Datafixed)}
\geq \frac{\Pr[\sampler(\Datafixed) \in[2k]]}{F^*_0(\Datafixed)+Y}-\frac \delta \eps.
$$
Substituting this bound on $p_{0,F(\Datafixed)}$ into (\ref{eq:notinsupport-delta}), we obtain
\begin{align*}
   \Pr[\sampler(\Datafixed) \notin Supp(\distr)] 
%   &\geq \frac{1}{2} \cdot F_0(\Datafixed) \cdot p_{0,F(\Datafixed)} \\
   &\geq \frac{1}{2} \cdot \frac {F^*_0(\Datafixed)\Pr[\sampler(\Datafixed) \in [2k]]}{F^*_0(\Datafixed) + Y} -\frac 12 \cdot\frac{F^*_0(\Datafixed)\cdot\delta}{\eps} \\
   &= \frac{1}{2} \cdot \frac {\Pr[\sampler(\Datafixed) \in [2k]]}{1 + Y/F^*_0(\Datafixed)} -\frac 12 \cdot\frac{F^*_0(\Datafixed)\cdot\delta}{\eps}\\
   &\geq\frac{1}{2}
\cdot\frac{\Pr[\sampler(\Datafixed) \in [2k]]}{1+Y/k} -\frac{k\delta}\eps,
\end{align*}
where in the last inequality, we used that $k\leq F^*_0(\Datafixed) \leq 2k$.
This holds since the support of $\distr$ excludes $k$ elements from $[2k]$ and since $F^*_0(\Datafixed)$ counts only elements from $[2k]$ that do not appear in $\Datafixed.$
\end{proof}


Finally, we give a lower bound on the right-hand side of (\ref{eq:main-dist-lb-delta}). Assume for the sake of contradiction that $n\leq \frac{k}{\alpha^* \eps}$. 
%
By Claim~\ref{clm:nonsupport-lb-fixed-delta},
%
\begin{align}
 \Pr_{\substack{N\sim\Po(n) \\ \Datarv\sim  \distr^{\otimes N}}}[\sampler(\Datarv) \notin Supp(\distr)]
 &\geq \E_{\substack{N\sim\Po(n) \\ \Datarv\sim  \distr^{\otimes N}}}\left[\frac{1}{2}
\cdot \frac{\Pr[\sampler(\Datarv) \in [2k]]}{1+Y/k} -\frac{k\delta}\eps\right]\nonumber\\
& = \frac{1}{2} \cdot \E_{\substack{N\sim\Po(n) \\ \Datarv\sim  \distr^{\otimes N}}}\left[\frac{\Pr[\sampler(\Datarv) \in [2k]]}{1+Y/k}\right] -  \frac{k\delta}{\eps}. \label{eq:mainlbcond}
\end{align}
Next, we analyze the expectation in (\ref{eq:mainlbcond}). Let $E$ be the event that $\frac{Y}{k} \leq e^3$. By the law of total expectation, 
\begin{align}\label{eq:mainlbcondfirst}
     \E_{\substack{N\sim\Po(n) \\ \Datarv\sim  \distr^{\otimes N}}}\left[\frac{\Pr[\sampler(\Datarv) \in [2k]]}{1+Y/k}\right]
     & \geq \E_{\substack{N\sim\Po(n) \\ \Datarv\sim  \distr^{\otimes N}}}\left[\frac{\Pr[\sampler(\Datarv) \in [2k]]}{1+Y/k} \big | E\right] \Pr(E).
\end{align}
In Claims~\ref{claim:eventE} and~\ref{claim:exp-of-regular-output}, we argue that both $\Pr(E)$ and $\E_{\substack{N\sim\Po(n) \\ \Datarv\sim  \distr^{\otimes N}}}\left[\frac{\Pr[\sampler(\Datarv) \in [2k]]}{1+Y/k} \big | E\right]$ are sufficiently large.
\begin{claim}\label{claim:eventE}
Suppose $n\leq \frac k{60\alpha \eps}$. Let $E$ be the event that $\frac{Y}{k} \leq e^3$. Then
    $\Pr(E) \geq 1 - \alpha$.
    \end{claim}
\begin{proof}
Recall that $Y$ was defined as $\sum_{i \in S^*}  \left[ e^{N_i(\Datafixed) \eps} \right]$ for a fixed dataset $\Datafixed.$ Now we consider the case when dataset $\Datarv$ is a random variable. 
If $N\sim\Po(n)$ and $\Datarv \sim \distr^{\otimes N}$ then $N_i(\Datarv) \sim \Po(\frac{\alpha^* n}{k})$ for all $i \in S^*$ and, additionally, the random variables $N_i(\Datarv)$ are mutually independent. When $\Datarv$ is clear from the context, we write $N_i$ instead of $N_i(\Datarv)$. Now we calculate the moments of $\frac{Y}{k}$.  For all $\lambda > 0$, 
\begin{align}
\E_{\substack{N\sim\Po(n) \\ \Datarv \sim  \distr^{\otimes N}}} \left[\left(\frac{Y}{k}\right)^{\lambda} \right] 
 = \E_{\substack{N\sim\Po(n) \\ \Datarv \sim  \distr^{\otimes N}}} \left[\left(\frac{1}{k} \sum_{i \in S^*} e^{N_i(\Datarv) \eps} \right)^{\lambda} \right] 
 = \E_{N_1, \dots, N_k \sim\Po(\frac{\alpha^* n}{k})} \left[ \left(\frac{1}{k} \sum_{i \in S^*} e^{N_i \eps} \right)^{\lambda} \right]. \label{eq:mompoiss}
\end{align}

%%%%%%%%%%%%%%%%%%%%%%%%
%% Moments of the average claim was moved to Appendix.
%%%%%%%%%%%%%%%%%%%%%%%%%%%%%%%%%%%%%%%

%Applying Claim~\ref{claim:momavg} to (\ref{eq:mompoiss}), we get that 
%\begin{align}\label{eq:mompoissfinal}
%\E_{\substack{N\sim\Po(n) \\ \Datafixed \sim  \distr^{\otimes N}}} \left[\left(\frac{Y}{k}\right)^{\lambda} \right] \nonumber  = %\E_{N_1, \dots, N_k \sim\Po(\frac{\alpha^* n}{k})} \left[ \left(\frac{1}{k} \sum_{i \in S^*} e^{N_i \eps} \right)^{\lambda} \right] 
%    \leq \E_{N_1 \sim\Po(\frac{\alpha^* n}{k})} \left[ \left(e^{N_1 \eps} \right)^{\lambda} \right].
%\end{align}
Finally, we bound the probability of event $E$. Set $c=e^3$ and $\lambda = \ln \frac{1}{\alpha}$. By definition of $E$, 
\begin{align}
    \Pr(\overline{E}) 
    & = \Pr\left(\frac{Y}{k} \geq c\right)\nonumber 
     = \Pr\left(\left(\frac{Y}{k}\right)^\lambda \geq c^{\lambda}\right) 
    \leq \frac 1{c^{\lambda}}\cdot {\E_{\substack{N\sim\Po(n) \\ \Datarv \sim  \distr^{\otimes N}}}
    \left[ \left(\frac{Y}{k}\right)^\lambda \right]} \nonumber \\
    & \leq  \frac 1{c^{\lambda}}\cdot {\E_{N_1, \dots, N_k \sim\Po(\frac{\alpha^* n}{k})} \left[ \left(\frac{1}{k} \sum_{i \in S^*} e^{N_i \eps} \right)^{\lambda} \right]} 
     \leq  \frac 1{c^{\lambda}}\cdot{\E_{N_1 \sim\Po(\frac{\alpha^* n}{k})} \left[ \left( e^{N_1 \eps} \right) ^{\lambda} \right]} \label{eq:moments2} \\
    & = c^{-\lambda}\cdot {e^{\frac{\alpha^* n}{k}\left(e^{\lambda \eps} - 1 \right)}}
    \leq e^{-3\lambda}\cdot {e^{\frac{\left(e^{\lambda \eps} - 1 \right)}{\eps}}}
    \leq e^{-3\lambda}\cdot e^{2\lambda}= e^{-\lambda}
    =e^{-\ln (1/\alpha)}=\alpha, \label{eq:moments3}
\end{align}
where we use $\lambda > 0$ in the second equality, then apply Markov's inequality. To get the inequalities in (\ref{eq:moments2}), we apply (\ref{eq:mompoiss}) and then Claim~\ref{claim:momavg} on the moments of the average of random variables. To get (\ref{eq:moments3}), we use the moment generating function of a Poisson random variable, and then we substitute $c=e^3$ and use the assumption that $n\leq \frac k{60\alpha \eps} =\frac k{\alpha^*\eps}$. The second inequality in (\ref{eq:moments3}) holds because $\lambda = \ln \frac{1}{\alpha}$ and $\eps \in (0,1/\ln \frac{1}{\alpha}]$, so $\lambda \eps \leq 1$ and hence $e^{\lambda \eps} \leq 1 + 2\lambda \eps$.
The final expression is obtained by substituting the value of $\lambda.$
We get that $\Pr(E) \geq 1-\alpha$, completing the proof of Claim~\ref{claim:eventE}.
\end{proof}
\begin{claim}\label{claim:exp-of-regular-output}
$\displaystyle\E_{\substack{N\sim\Po(n) \\ \Datarv\sim  \distr^{\otimes N}}}\left[\frac{\Pr[\sampler(\Datarv) \in [2k]]}{1+Y/k} \big| E \right] \geq 2.3\alpha.$
\end{claim}
\begin{proof}
When event $E$ occurs, $1+\frac{Y}{k} \leq 1+e^3<22$. Then  
\begin{align}\label{eq:conditioning}
    \E_{\substack{N\sim\Po(n) \\ \Datarv\sim  \distr^{\otimes N}}}\left[\frac{\Pr[\sampler(\Datarv) \in [2k]]}{1+Y/k} \big | E \right]
    & > \E_{\substack{N\sim\Po(n) \\ \Datarv\sim  \distr^{\otimes N}}}\left[\frac{\Pr[\sampler(\Datarv) \in [2k]]}{22} \big | E\right] 
     = \frac 1{22}\cdot\E_{\substack{N\sim\Po(n) \\ \Datarv\sim  \distr^{\otimes N}}}\left[\Pr[\sampler(\Datarv) \in [2k]] \mid E \right].
\end{align}
By the product rule,
$$\Pr[\sampler(\Datarv) \in [2k]] \mid E ]
=\frac{\Pr[\sampler(\Datarv) \in [2k]] \wedge E]}{\Pr[E]}
\geq \Pr[\sampler(\Datarv) \in [2k]] \wedge E]
\geq  {\Pr[\sampler(\Datarv) \in [2k]] - \Pr[\overline{E}]}.$$
Substituting this into (\ref{eq:conditioning}) and recalling that $\alpha^*=60\alpha$, we get
\begin{align*}
    \E_{\substack{N\sim\Po(n) \\ \Datarv\sim  \distr^{\otimes N}}}\left[\frac{\Pr[\sampler(\Datarv) \in [2k]]}{1+Y/k} \big | E \right]
    &\geq  \frac 1{22}\cdot\E_{\substack{N\sim\Po(n) \\ \Datarv\sim  \distr^{\otimes N}}}\left[\Pr[\sampler(\Datarv) \in [2k]] - \Pr[\overline{E}] \right]
%    = \Pr[\sampler(\Datarv) \in [2k]] \mid E ] \cdot \left( \frac{1}{30} \right) 
 %    = \frac{\Pr[\sampler(\Datarv) \in [2k]] \wedge E]}{\Pr[E]} \cdot \left( \frac{1}{30} \right) \\
 %    &\geq \frac{\Pr[\sampler(\Datarv) \in [2k]] - \Pr[\overline{E}]}{\Pr[E]} \cdot \left( \frac{1}{30} \right) 
     \geq \frac 1{22}\cdot \left( \alpha^* - \alpha - \alpha \right) 
     \geq 2.3\alpha,
\end{align*}
since sampler $\sampler$ is $\alpha$-accurate on $\distr$, and  $\distr$ has mass $\alpha^*$ on $[2k]$, and by Claim~\ref{claim:eventE}. 
\end{proof}
%Applying Claims~\ref{claim:eventE} and~\ref{claim:exp-of-regular-output} to (\ref{eq:mainlbcondfirst}), we get that
%\begin{align}
%     \frac{1}{2} \cdot \E_{\substack{N\sim\Po(n) \\ \Datarv\sim  \distr^{\otimes N}}}\left[\frac{\Pr[\sampler(\Datarv) \in [2k]]}{1+Y/k}\right] 
%      \geq  \frac{1}{2} \cdot \frac{\alpha^* - 2\alpha}{30}\cdot \left(1 - \alpha\right) 
%      \geq \frac{\alpha^* - \frac{5\alpha}{4}}{120}. \label{eq:alphainexp}
%\end{align}
%Substituting  (\ref{eq:alphainexp}) into (\ref{eq:mainlbcond}), we get that
%\begin{align}
% \Pr_{\substack{N\sim\Po(n) \\ \Datarv\sim  \distr^{\otimes N}}}[\sampler(\Datarv) \notin Supp(\distr)]
%& > \frac{\alpha^* - \frac{5\alpha}{4}}{120} - k\cdot \frac{\delta}{\eps}
%\end{align}

Combining (\ref{eq:main-dist-lb-delta}), (\ref{eq:mainlbcond}), and (\ref{eq:mainlbcondfirst}), applying Claims~\ref{claim:eventE} and~\ref{claim:exp-of-regular-output}, and recalling that $\delta\leq 0.1\cdot\alpha\eps/k$, we get
\begin{align*}
    d_{TV}(\distr,\distroutput{\sampler, \distr})
  &\geq \Pr_{\substack{N\sim\Po(n) \\ \Datarv\sim  \distr^{\otimes N}}}[\sampler(\Datarv) \notin Supp(\distr)]
  \geq \frac{1}{2} \cdot \E_{\substack{N\sim\Po(n) \\ \Datarv\sim  \distr^{\otimes N}}}\left[\frac{\Pr[\sampler(\Datarv) \in [2k]]}{1+Y/k}\right]  - \frac{k\delta}{\eps} \\
  &\geq \frac 12\cdot   \E_{\substack{N\sim\Po(n) \\ \Datarv\sim  \distr^{\otimes N}}}\left[\frac{\Pr[\sampler(\Datarv) \in [2k]]}{1+Y/k} \big | E\right] \Pr(E) - 0.1\alpha 
  \geq  \frac{1}{2} \cdot 2.3\alpha\cdot \left(1 - \alpha\right) - 0.1\alpha 
  > \alpha,
%     \geq\frac{1}{1 + \frac 1k\E_{N\sim\Po(n),\Datarv\sim  \distr^{\otimes N}} \left[\sum_{i\in \Datarv} e^{N_i(\Datarv) \cdot \eps} \right]}  
 %   & > \frac{\alpha^* - \frac{5\alpha}{4}}{120}  - \frac{k\delta}{\eps} \\
%    & \geq \frac{\alpha^* - \frac{5\alpha}{4}}{120}  - \frac{\alpha}{120} \quad \text{(Since $\delta \leq \frac{\alpha\eps}{120k}$)}
%    \\
%    & > \alpha \quad \text{($\alpha^*$ set to $50 \alpha$)}
\end{align*}
where the last inequality holds since $\alpha\leq 0.02$. This contradicts $\alpha$-accuracy of $\sampler$ on datasets of size $\Po(n)$, where $n\leq \frac{k}{\alpha^* \eps}$, and completes the proof of Lemma~\ref{lem:main-k-ary-lb}.
\end{proof}

Next, we prove our lower bound by removing the assumptions that $\eps$ is small and that the samplers are frequency-count-based. This uses the properties of samplers that we proved in Section~\ref{sec:properties}.

\begin{lemma}\label{lem:k-ary-lb-pois}
For all sufficiently small $\alpha>0$, $k,n \in \mathbb{N}$, $\eps \in (0, 1]$, %$\alpha\in(0,1-e^{-n/6}]$,
and $\delta \in \big[0, \frac{1}{5000n} \big]$, if there exists an $(\eps,\delta)$-differentially private sampler that is $\alpha$-accurate on the class $\carb$ with dataset size distributed as $\Po(n)$, then $n=\Omega(\frac{k}{ \alpha\eps})$.
\end{lemma}

\begin{proof}
First, we will prove the lemma assuming the range of $\delta$ is $\delta \in \big[0, 0.1 \cdot \frac{\alpha \eps}{k} \big]$. Then we will extend to the claimed range of $\delta$. We prove the lemma for $\delta \in \big[0, 0.1 \cdot \frac{\alpha \eps}{k} \big]$ by applying Lemmas~\ref{lem:amplify} and~\ref{lem:frequency-counts} to generalize the lower bound in Lemma~\ref{lem:main-k-ary-lb} to work for all differentially private samplers and all privacy parameters $\eps\in(0,1]$.

Suppose there exists an $(\eps, \delta)$-differentially private sampler $\sampler$ that is $\alpha$-accurate on the class $\carb$ with dataset size distributed as $\Po(n)$, for some $n \in \mathbb{N}, \eps \in (0,1], \delta \in \big[0, 0.1\cdot\frac{\alpha\eps}k \big]$, and $\alpha \in (0,0.01]$. 

%We can assume without loss of generality that $n \geq 6\ln(1/\alpha)$, since, if this is not the case, \sampler can ignore extra samples. 
%By Lemma~\ref{lem:poisson}, there exists an $(\eps, \delta)$-differentially private sampler $\posampler$ that is $(\alpha + e^{-n/6})$-accurate on $\class_{2k+1}$ when its dataset size is distributed as $\Po(2n)$. Since $n \geq 6\ln(1/\alpha)$, this gives $e^{-n/6} \leq \alpha$, so sampler \posampler is $2\alpha$-accurate. 
By Lemma~\ref{lem:amplify}, we can amplify the privacy to construct an $(\eps',\delta')$-differentially private sampler~$\sampler'$  that is $\alpha$-accurate for datasets with size distributed as $\Po(4n\ln(1/\alpha'))$,
$\eps'=\frac{\eps}{\ln(1/\alpha)},$ and $\delta'=\frac{\delta}{2\ln(1/\alpha)}$. Then $\eps' \leq \frac{1}{\ln(1/\alpha)}, \delta'\leq 0.01 \frac{\alpha\eps'}k,$ and $\alpha\leq 0.02,$ as required to apply Lemma~\ref{lem:main-k-ary-lb} with privacy parameters $\eps',\delta'$ and  accuracy parameter $\alpha$. By Lemma~\ref{lem:frequency-counts}, we can assume the sampler is frequency-count-based with no changes in the privacy and accuracy parameters. Now, applying Lemma~\ref{lem:main-k-ary-lb} gives 
\begin{align*}
    4n\ln(1/\alpha) \geq \frac{1}{60} \cdot\frac{k}{2\alpha \cdot \frac{\eps}{\ln(1/\alpha)}}\;.
\end{align*}
Therefore, $n \geq \frac{k}{480\alpha\eps}=\Omega(\frac k {\alpha\eps})$.

Next, we extend this argument to all $\delta \in [0, \frac{1}{5000n}]$. Observe that when $n < \frac{k}{480 \alpha \eps} < \frac{1}{5000\delta}$, a direct application of our theorems proves the lower bound. When $n < \frac{1}{5000\delta} < \frac{k}{480 \alpha \eps}$, assume by way of contradiction that there exists an $(\eps, \delta)$-DP sampler that is $\alpha$-accurate on the class of $k$-ary distributions for input datasets with size distributed as $\Po(n)$. 
 
One can find a triple of values $(k',\alpha',\eps')$ such that $2 \leq k'\leq k$, $0.01\geq\alpha'\geq \alpha$ and $1 \geq \eps' \geq \eps$ such that $n\leq  \frac{k'}{480 \alpha'\eps'} <\frac 1 {5000 \delta}$. In particular, $\delta < 0.1\cdot \frac{\alpha'\eps'}{k'}$.  
%
Next, note that a sampler that achieves accuracy $\alpha < 0.01$ is also a sampler that achieves accuracy $\alpha'$ for all $0.01 \geq \alpha' > \alpha$. Additionally, since the class of discrete distributions over $[k']$ is a subclass of the class of discrete distributions over $[k]$, an $(\eps,\delta)$-differentially private sampler that is $\alpha$-accurate on the class of discrete distributions over $[k]$ when given an input dataset with size distributed as $\Po(n)$ is $(\eps',\delta)$-differentially private and $\alpha'$-accurate on the class of discrete distributions over $[k']$ with sample size distributed as $\Po(n)$.
%
We can then apply the lower bound obtained for the range $\delta \in \big[0,0.1 \cdot \frac{\alpha' \eps'}{k'}\big]$ to show that no such sampler exists. This proves the theorem for all $\delta \in [0, \frac{1}{5000n}]$.
\end{proof}


\ifnum\neurips=1
\subsubsection{Final lower bound for $k$-ary distributions}\label{sec:k-ary-lb-final}
\else 
\subsubsection{Final Lower Bound for $k$-ary Distributions}\label{sec:k-ary-lb-final}
\fi

In this section, we complete the proof of Theorem~\ref{thm:k-ary-lb}. 

\begin{theorem}\label{thm:k-ary-lb}
For all sufficiently small $\alpha>0$, $k,n \in \mathbb{N}$, $\eps \in (0, 1]$, , %$\alpha\in(0,1-e^{-n/6}]$,
and $\delta \in \big[0, \frac{1}{5000n}\big]$, if there exists an $(\eps,\delta)$-differentially private sampler that is $\alpha$-accurate on the class $\class_{2k+1}$ of discrete distributions over universe $[2k+1]$ on datasets of size $n$, then $n=\Omega(\frac{k}{ \alpha\eps})$.
\end{theorem}

\begin{proof}
First, observe that a sampler that is $\alpha$-accurate on all distributions in $\class_{2k+1}$, is in particular $\alpha$-accurate on all distributions in the subclass $\carb$. Hence, a lower bound on the number of samples needed to achieve $\alpha$-accuracy for the class $\carb$ also applies to $\class_{2k+1}$. Hence, we can work with \carb for the rest of the proof. We apply Lemma~\ref{lem:poisson} to generalize the lower bound in Lemma~\ref{lem:k-ary-lb-pois} to work for samplers with fixed input dataset sizes. Suppose there exists an $(\eps, \delta)$-differentially private sampler $\sampler$ that is $\alpha$-accurate on the class $\carb$ for datasets of size $n$, for some $n \in \mathbb{N}, \eps \in (0,1], \delta \in \big[0, \frac{1}{5000n} \big]$, and $\alpha \in (0,0.01]$. Then, by Lemma~\ref{lem:poisson}, there exists an $(\eps, \delta)$-differentially private sampler $\posampler$ that is $(\alpha + e^{-n/6})$-accurate on $\class^*_{2k+1}$ when its dataset size is distributed as $\Po(2n)$. We can assume without loss of generality that $n \geq 6\ln(1/\alpha)$, since, if this is not the case, \sampler can ignore extra samples. This gives $e^{-n/6} \leq \alpha$, so sampler \posampler is $2\alpha$-accurate. By Lemma~\ref{lem:k-ary-lb-pois}, we have that $n = \Omega(\frac{k}{\alpha \eps})$, completing the proof. 
\end{proof}


%\begin{proof}[\ms{OLD PROOF}]
%We apply Lemmas~\ref{lem:poisson},~\ref{lem:amplify}, and~\ref{lem:frequency-counts} to generalize the lower bound in Lemma~\ref{lem:main-k-ary-lb} to work for all differentially private samplers and all privacy parameters $\eps\in(0,1]$.

%Suppose there exists an $(\eps, \delta)$-differentially private sampler $\sampler$ that is $\alpha$-accurate on the class $\class_{2k+1}$ for datasets of size $n$, for some $n \in \mathbb{N}, \eps \in (0,1], \delta \in \big[0, 0.1\cdot\frac{\alpha\eps}k \big]$, and $\alpha \in (0,0.01]$. We can assume without loss of generality that $n \geq 6\ln(1/\alpha)$, since, if this is not the case, \sampler can ignore extra samples. By Lemma~\ref{lem:poisson}, there exists an $(\eps, \delta)$-differentially private sampler $\posampler$ that is $(\alpha + e^{-n/6})$-accurate on $\class_{2k+1}$ when its dataset size is distributed as $\Po(2n)$. Since $n \geq 6\ln(1/\alpha)$, this gives $e^{-n/6} \leq \alpha$, so sampler \posampler is $2\alpha$-accurate. By Lemma~\ref{lem:amplify}, we can amplify the privacy to construct an $(\eps',\delta')$-differentially private sampler~$\sampler'$  that is $\alpha'$-accurate for datasets with size distributed as $\Po(4n\ln(1/\alpha'))$, where $\alpha'=2\alpha$,
%$\eps'=\frac{\eps}{\ln(1/\alpha')},$ and $\delta'=\frac{\delta}{2\ln(1/\alpha')}$. Then $\eps'
%\frac{\eps}{\ln(1/\alpha)} 
%\leq \frac{1}{\ln(1/\alpha')}, \delta'\leq 0.01 \frac{\alpha'\eps'}k,$ and $\alpha'\leq 0.02,$ as required to apply Lemma~\ref{lem:main-k-ary-lb} with privacy parameters $\eps',\delta'$ and  accuracy parameter $\alpha'$. By Lemma~\ref{lem:frequency-counts}, we can assume the sampler is frequency-count-based with no changes in the privacy and accuracy parameters. Now, applying Lemma~\ref{lem:main-k-ary-lb} gives 
%\begin{align*}
%    4n\ln(1/\alpha') \geq \frac{1}{60} \cdot\frac{k}{2\alpha \cdot \frac{\eps}{\ln(1/\alpha')}}\;.
%\end{align*}
%Therefore, $n \geq \frac{k}{480\alpha\eps}=\Omega(\frac k {\alpha\eps})$, as claimed in the theorem statement.
%\end{proof}
 Now we can combine Theorem~\ref{thm:bernoulli-lb} (for $k=2$) and Theorem~\ref{thm:k-ary-lb} (for $k \geq 3$) to get the lower bound in Theorem~\ref{thm:intro-k-ary-lb} for all $k \geq 2$. %Note that directly applying these theorems would give a lower bound for $\delta \in [0, 0.1\frac{\alpha \eps}{k}]$. However, this can be extended to any $\delta \in [0, \frac{1}{5000n}]$.%
 %\ifnum\supplemental=1
%\fi
% To see this, observe that when $n < \frac{k}{480 \alpha \eps} < \frac{1}{5000\delta}$, a direct application of our theorems proves the lower bound. When $n < \frac{1}{5000\delta} < \frac{k}{480 \alpha \eps}$, assume by way of contradiction that there exists an $(\eps, \delta)$-DP sampler that is $\alpha$-accurate on the class of $k$-ary distributions for input datasets of size $n$. 
 
%One can find a triple of values $(k',\alpha',\eps')$ such that $2 \leq k'\leq k$, $0.01\geq\alpha'\geq \alpha$ and $1 \geq \eps' \geq \eps$ such that $n\leq  \frac{k'}{480 \alpha'\eps'} <\frac 1 {5000 \delta}$. In particular, $\delta < 0.1\cdot \frac{\alpha'\eps'}{k'}$.  
%
%Next, note that a sampler that achieves accuracy $\alpha < 0.01$ is also a sampler that achieves accuracy $\alpha'$ for all $0.01 \geq \alpha' > \alpha$. Additionally, since the class of discrete distributions over $[k']$ is a subclass of the class of discrete distributions over $[k]$, an $(\eps,\delta)$-differentially private sampler that is $\alpha$-accurate on the class of discrete distributions over $[k]$ when given an input dataset of size $n$ is $(\eps',\delta)$-differentially private and $\alpha'$-accurate on the class of discrete distributions over $[k']$ for the same dataset size $n$.
%
%We can then apply the lower bounds in either Theorem~\ref{thm:k-ary-lb} or Theorem~\ref{thm:bernoulli-lb} to show that no such sampler exists. This proves the theorem for all $\delta \in [0, \frac{1}{5000n}]$.
\ifnum\neurips=1
\section{Product distributions over $\{0,1\}^d$}\label{sec:prod}
\else 
\section{Product Distributions Over $\{0,1\}^d$}\label{sec:prod}
\fi 
In this section, we consider the problem of privately sampling from the class $\cB^{\otimes d}$ of product distributions over $\{0,1\}^d$. 
We present and analyze a $\rho$-zCDP sampler for $\cB^{\otimes d}$ (Theorem~\ref{thm:bernoulli-product-alg}) and then %, in Section~\ref{sec:prod-ub-main}, 
apply a standard conversion from $\rho$-zCDP to $(\eps, \delta)$-differential privacy (Theorem~\ref{prelim:relate_dp_cdp}) to prove Theorem~\ref{thm:bernoulli-product-alg-intro}. Then, in Section~\ref{sec:prod-lb}, we prove the matching lower bound stated in Theorem~\ref{thm:bernoulli-product-lb-intro}.
%
%
%In Section~\ref{sec:product-upper}, we prove Theorem~\ref{thm:bernoulli-product-alg-intro}. 
%In Section~\ref{sec:prod-lb}, we prove Theorem~\ref{thm:bernoulli-product-lb-intro} that gives the lower bound on the sample complexity of $(\eps, \delta)$-differentially private sampling from the class $\cB^{\otimes d}$.

% \ifnum\neurips=1
% \subsection{Private sampler for product distributions}\label{sec:product-upper}
% \else 
% \subsection{Private Sampler for Product Distributions}\label{sec:product-upper}
% \fi
\subsection{Upper Bound for Products of Bernoulli Distributions}
\begin{theorem}[Upper bound for product distributions]\label{thm:bernoulli-product-alg}
For all $\rho\in(0,1]$, $\alpha\in (0,1)$, and $d$ greater than some sufficiently large constant, there exists a $\rho$-zCDP sampler for the class $\cB^{\otimes d}$ of product Bernoulli distributions that is  $\alpha$-accurate on datasets of size $n=O\Big(\frac {d}{\alpha\sqrt{\rho}} \cdot \Big[\log^{9/4}d + \log^{5/4}\frac{1}{\alpha \sqrt{\rho}}\Big]\Big)$.
\end{theorem}
This theorem implies Theorem~\ref{thm:bernoulli-product-alg-intro}; we prove the implication at the end of this section. %Section~\ref{sec:product-upper}.

%\paragraph{Truncated means and recursive preconditioning \cite{KLSU19}.}
Our main technical tool is the recursive preconditioning technique of \cite{KLSU19}.
Let $\Biasesfixed = (\biasesfixed_1, \dots, \biasesfixed_d)$ be the unknown attribute biases for the product distribution $\distr \in \cB^{\otimes d}$ from which the data is drawn. 
%
For some intuition, consider the following natural differentially private algorithm for sampling from a product distribution: First,  privately estimate each of the attribute biases $\biasesfixed_j$ by adding noise to the sample mean; then sample each attribute independently from a Bernoulli with this estimated bias. This approach does not work directly because the $\ell_2$-sensitivity of the vector of sample means is $\sqrt{d}/n$. To accurately estimate tiny biases, we require a large sample size $n$. For instance, in the case where all the attribute biases are roughly $1/d$, the naive algorithm described above would require $n=\Omega(d^{3/2})$ records to be $\alpha$-accurate for a small constant $\alpha$.


%
%In order to privately estimate the attribute biases, a natural first attempt would be to simply take the mean of the input data, which would give estimates of all of the attribute biases, and then add independent Gaussian noise with appropriate standard deviation to each coordinate in order to preserve privacy. However, as observed in \cite[Section 1.2.2]{KLSU19}, the $\ell_2$-sensitivity of the mean operation over $m$ data entries with $d$ coordinates is $\sqrt{d}/m$. As discussed in \cite{KLSU19}, in the case where all the attribute biases are small, i.e., roughly $1/d$, the naive algorithm described above would require $\Omega(d^{3/2})$ samples to obtain a good estimate to the product distribution in total variation distance. 
%
To get around this (in the context of distribution learning),  Kamath et al. \cite{KLSU19} observe that 
when the biases are small and the input is drawn from a product distribution, the number of $1$s in each record is constant---say, at most  $10$---with high probability. Viewing the records as vectors in $\mathbb{R}^d$, we can therefore truncate every record so that its $\ell_2$-norm is at most $\sqrt{10}$ (that is, we leave short vectors alone and shrink longer records) and then average the truncated data entries to obtain a \textit{truncated mean}. We call $\sqrt{10}$ the \emph{truncation ceiling}. Truncation reduces sensitivity, which allows one add less noise---and thus give better attribute bias estimates---while preserving privacy. When the biases are at most $1/d$, the sample complexity for constant accuracy $\alpha$ is reduced to $O(d/\eps)$. (A similar idea works for biases very close to 1. For simplicity, we assume that all attribute biases $\biasesfixed_i$ are between $0$ and $1/2$. See Footnote~\ref{foot:flip}.)

%Additionally, with high probability, the number of $1$s in every record is less than $10$ and hence there is no truncation of any record. 
%Thus, truncating appropriately gives better attribute bias estimates in this case.

The challenge with this approach is that we don't know biases ahead of time; when coordinates have large bias, setting the truncation ceiling too low leads to high error. Kamath et al. address this by estimating the attribute biases in rounds: in round $j$, attributes with biases close to $2^{-j}$ are estimated reasonably accurately, while smaller biases are passed to the next round where truncation can be applied more aggressively. This process is called \textit{recursive preconditioning}, and it is an important part of our algorithm.




%However, if all the attribute biases $\biasesfixed_j$ were $1/2$, then using the truncated mean with truncation ceiling $10$ would result in bad attribute bias estimates (since with high probability, every record has at least $d/4$ ones).  
%Hence, the major challenge is to calibrate the level of truncation to the specific attribute biases. The problem is that in advance, the attribute biases are unknown. This is handled in \cite{KLSU19} by estimating the attribute biases in rounds, using an idea they call recursive private preconditioning. We use this idea in our sampling algorithm.



% \srnote{But we don't know in advance if it is greater or smaller than 1/2. Do we estimate both and somehow decide between the two estimates?} 
% \ssnote{Good point; I guess we could first use some data to check if each attribute bias is greater than 1/2 or smaller than 1/2 and proceed from there (if we are wrong on any bias, we can argue that the bias must be very close to 1/2, and for such biases it doesn't really matter which estimate we use). I'm not sure what the most succinct way of saying that is.} 


%Next, we describe the thought process behind our algorithm. The algorithm consists of two phases. 

%\paragraph{Algorithm Outline.} 
Our algorithm proceeds in two phases. 

\begin{itemize}
    \item \textbf{Bucketing Phase:} This phase implements recursive private preconditoning from \cite{KLSU19} to estimate the attribute biases $\biasesfixed_i$. The main difference is that, for coordinates with large bias, we require less accurate estimates than \cite{KLSU19} and can thus use fewer samples. 
    
    In a bit more detail: The interval $[0,\frac 12]$ is divided into $\lceil \log_2 d\rceil +1$ overlapping sub-intervals that we call \textit{buckets}. The $r^{th}$ bucket corresponds to the interval $[\frac 1 4 \cdot 2^{-r},\   2^{-r}]$. The exception is the smallest bucket, which corresponds to $[0,\frac 1 d]$. 
    
    We proceed in rounds, one per bucket. The bucketing phase uses half of the overall dataset and, for simplicity, those records are split evenly among rounds. Each round thus uses  $m\approx \frac{n}{2\log d}$ records. At round $r$, some coordinates are classified as having attributes in bucket $r$, while others are passed to the next round. With high probability, we maintain the invariants that (a) \textit{only} coordinates with bias at most $2^{-r}$ are passed to round $r$, and (b) \textit{all} coordinates with bias at most $2^{-r-2}$ are passed to round $r+1$ (except for the last round, in which no records are passed on). As a result, coordinates classified in round $r$ have biases in the bucket $[2^{-r-2},2^{-r}]$; records left in the last round have bias at most $1/d$.
    
    %\asnote{Perhaps the next two paragraphs have too much detail? I guess I don't see the harm in leaving them in.}
    For example, the first round corresponds to bucket $[\frac 1 8, \frac 12]$. All coordinates are passed to that round (they have bias at most $\frac 1 2$ by assumption). 
    Using its batch of $m$ records, this round of the algorithm computes the empirical means for all coordinates, adds Gaussian noise about $\frac{\sqrt{d}}{m}$ to each, and releases the list of noisy means. We select $n$ large enough for these noisy estimates to each be within $\frac 1 {16}$ of the true attribute bias with high probability (over the sampling of both the data and the noise).  Attributes with noisy estimates below $3/16$ are passed to round 2, while the rest are assigned to bucket 1. One can check that the invariants are maintained: attributes with bias below 1/8 are passed to round 2; those with bias at least 1/4 are assigned to bucket 1; and those in between may go either way.%
    \footnote{\label{foot:flip} One can also handle biases larger than $1/2$ at this phase. Specifically, the first round of noisy measurements allows us to divide the coordinates into three disjoint sets, each containing only coordinates with biases in $[0,1/4]$, $[1/8,7/8]$, and $[3/4,0]$, respectively. We can work with the coordinates in the first two sets as they are. For coordinates in the third set, we can flip all entries (from 0 to 1 and vice-versa), treat them as if their biases were in $[0,1/4]$, and flip the corresponding output bits.}
    
    At round $r$, we proceed similarly except that we can restrict records to those attributes that were passed to this round and we can truncate records so their $\ell_2$ norm is at most $T_r\approx \sqrt{d} / 2^r$. When the data are from a product distribution and prior rounds were correct, this truncation has essentially no effect on the records but allows us to add less noise. We get noisy means that are within $\pm \frac 1 8 \cdot 2^{-r}$ of the true biases.
    %, and can thus correctly  distinguish records with bias above $\frac 1 2 \cdot 2^{-r}$ from those with bias at most $\frac 1 4 \cdot 2^{-r}$. 
    The invariants are maintained if we pass biases with noisy means below $\frac 3 8 \cdot 2^{-r}$ to the next round, and assign the rest to bucket $r$.

    %The first bucket is the interval $[1/8,1/2]$. Consider all attribute biases $\biasesfixed_j \in [1/4,1/2]$, and restrict the data for round $1$ to this set of attributes. With high probability, the fraction of ones in every record (in the restricted dataset) is at least $7/32$. Consider the truncated mean of these data entries with the truncation ceiling set to $\frac{\sqrt{d}}{m}$. We can add Gaussian noise with standard deviation roughly $\sqrt{d}/m$ to the truncated mean to preserve privacy. The resulting noisy empirical estimates for the attribute biases in the interval $[1/4,1/2]$ will all be larger than $3/16$ with high probability. Next, consider attribute biases $\biasesfixed_j < 1/8$. Analogously, the noisy empirical estimates for these attribute biases in the first round will be smaller than $3/16$ with high probability. Hence, setting a threshold of $3/16$ for the first round and filtering to the next round only those attribute biases $\biasesfixed_j$ whose noisy empirical estimates are smaller than the threshold, we can be confident that all attribute biases $\biasesfixed_j < 1/8$ will filter to the second round, and all attribute biases $\biasesfixed_j > 1/4$ will not.% 
    %\footnote{\label{foot:flip} One can also handle biases larger than $1/2$ at this phase. Specifically, the first round of noisy measurements allows us to divide the coordinates into three disjoint sets, each containing only coordinates with biases in $[0,1/4]$, $[1/8,7/8]$, and $[3/4,0]$, respectively. We can work with the coordinates in the first two sets as they are. For coordinates in the third set, we can flip all entries (from 0 to 1 and vice-versa), treat them as if their biases were in $[0,1/4]$, and flip the corresponding output bits.}
    %Finally, consider the attribute biases $\biasesfixed_j \in [1/8,1/4]$. Some of them will filter through to the second round and some of them will not. Setting a threshold of $3/32$ for the second round, and setting a truncation ceiling of about $\sqrt{d/2}/m$ (with standard deviation of noise scaled appropriately to preserve privacy), we can be sure that these attribute biases $\biasesfixed_j$ will not filter from the second to the third round, since the second bucket is the interval $[1/16,1/4]$. If attribute biases $\biasesfixed_j$ don't filter from round $r$ to round $r+1$, we will say that they are bucketed in round $r$. Hence, it is clear that with high probability, all the biases $\biasesfixed_j \in [1/8,1/4]$ will either be bucketed into the first bucket or the second bucket; this guarantees us a good multiplicative approximation of such attribute biases. Observe that having the buckets overlap in such a way that the middle of the $1^{st}$ bucket coincides with the right end-point of the $2^{nd}$ bucket is a crucial reason why we get this good approximation.
    
    %Recursive private preconditioning involves using the same idea over $R$ rounds, reducing the threshold by a factor of two and the truncation ceiling by a factor of $\sqrt{2}$ in each round. As before, this gives us a constant factor multiplicative approximation for every attribute bias $\biasesfixed_i$, except the smallest ones, for which we get a good additive approximation.

    \item \textbf{Sampling Phase:} 
    In the second phase, 
    we use fresh data for the sampling phase to construct new, \textit{unbiased} noisy empirical estimates of the attribute biases. In round $r$ of this phase, we restrict records to the attributes assigned to bucket $r$. We can  truncate the records to have norm $T_r$ (because the biases are at most $2^{-r}$) and add noise as before. This gives us a list of noisy means, which we clip to $[0,1]$ by rounding up negative values and rounding down values above 1. We sample one bit for each attribute independently, according to these clipped noisy means.
    
    For attributes in all buckets except the last, we get noisy means that lie in $[0,1]$ with high probability (because the biases are at least $\frac 1 4 \cdot 2^{-r}$). Since the estimates are unbiased and no clipping occurs, we sample from the correct distribution. For the attributes in the last bucket, we may get negative noisy means. However, the noise is small in these attributes, and we can bound the overall effect on the distribution. Interestingly, almost all the error of our algorithm comes from these low-bias attributes.
    
    %After the bucketing phase, we have good approximations of the attribute biases. 
    %We use the buckets to calibrate the truncation, in order to reduce the sensitivity and add less noise while maintaining privacy. Let $S_r$ be the set of attribute biases $\biasesfixed_j$ that fall in the $r^{th}$ bucket $[2^{-r-2},2^{-r}]$. Then, with high probability, for every record, the number of ones in that record restricted to attributes $S_r$ is at most $\frac{|S_r|}{2^{r-1}}$. Hence, we set the truncation ceiling to roughly $\frac{|S_r|}{2^{r-1}}$, and add Gaussian noise with standard deviation scaled accordingly. Finally, we clip these estimates to $[0,1]$, and sample each attribute independently from a Bernoulli with bias equal to the clipped noisy empirical bias.
    
        %One idea could be to sample each attribute independently with bias equal to the middle of the bucket\srnote{This seems a bit contrived. Why not use the estimates themselves?} it was thrown into in the bucketing phase. However, this would not work. For instance, consider the case where all the attribute biases are $3/4d$. In this case, if the bucketing was successful, all the biases would be in the final bucket, and we would sample independently for each attribute with bias roughly $1/2d$. This would result in a sampler that is at most $1/4$-accurate for the class of product Bernoulli distributions. This tells us that the approximations from the bucketing phase are not good enough to use directly. 

\end{itemize}

%\paragraph{Our Algorithm} 
We present our sampler $\sampler_{prod}$ for $\cB^{\otimes d}$ in Algorithm~\ref{alg:prod}. 
%The description of the algorithm uses the following notation. 
Let $\Datafixed = (\datafixed_1, \dots, \datafixed_n)$ be a dataset with $n$ records. The truncated mean operation, used in the algorithm, 
%in order to reduce the sensitivity of the computation so that less noise is needed for preserving privacy. The operation 
is defined as follows:
\begin{align*}
    trunc_B(\datafixed_i)
    &=
   \begin{cases}
    \datafixed_i        & \text{if } \|\datafixed_i\|_2 \leq B; \\
    \frac{B}{\|\datafixed_i\|_2}\; \datafixed_i   & \text{otherwise;}
   \end{cases}\\
    tmean_{B}(\Datafixed) &= \frac{1}{n}\sum_{i=1}^n trunc_B(\datafixed_i).
\end{align*}
Recall that we assume that all of the attribute biases $\biasesfixed_j \in [0,1/2]$. 


%\renewcommand{\algorithmiccomment}[1]{\hfill\eqparbox{COMMENT}{$\triangleright$ #1}}

\begin{algorithm}[ht]
    \caption{Sampler $\sampler_{prod}$ for $\cB^{\otimes d}$}
    \label{alg:prod}
    \hspace*{\algorithmicindent} \textbf{Input:} dataset $\Datafixed \in \{0,1\}^{d\times n}$, privacy parameter  $\rho \in (0,1]$, failure parameter $\beta > 0$. \\
    \hspace*{\algorithmicindent} \textbf{Output:} $b \in \{0,1\}^d$ 
    \begin{algorithmic}[1] % The number tells where the line numbering should start
            \State Set $R \gets \log_2 \frac d{40}, m \gets \frac{n}{2R+1}$.  \Comment{For analysis,  assume 
            $m = \frac{1200d}{\alpha \sqrt{2 \rho}} \log^{5/4}\frac{d R}{\alpha \beta \sqrt{2\rho}}$}
            %$m \geq \frac{1200 \log^{5/4} \left( \frac{dR}{\alpha \beta (2\rho)^{1/2}}\right) d}{\alpha (2 \rho)^{1/2}}$.}
            \State Split $\Datafixed$ into $2R+1$ datasets $\Datafixed^1,\dots,\Datafixed^{2R+1}$ 
            %with $m$ records each.
            of size $m$ 
            %\State \sstext{Define restrictions of vector outside algo. }For a set $J \subseteq [d]$, let $\Datarv^r[J]$ denote the dataset $\Datarv^r$ restricted to  attributes in $J$. Similarly, let $q[J]$ represent the vector $q$ restricted to attributes in $J$.
            % \State Split $\Datafixed$ into two parts $\Datarv$ and $\Datarvy$ of equal size. Split $\Datarv$ into $R$ samples of size $m$, denoted $\Datarv^r$ for $r \in [R]$ and split $\Datarvy$ into $R+1$ samples of size $m$. Set $q[j] \gets 0$ for all $j \in [d]$. For set $J \subseteq [d]$, let $\Datarv^r[J]$ denote the dataset $\Datarv^r$ restricted to those attributes $j$ that belong to $J$. Similarly, let $q[J]$ represent the vector $q$ restricted to attributes $j \in J$.
            \medskip
            
            %$\triangleright$ 
            \textbf{Bucketing Phase}
            \State Set $S_1 \gets [d]$, $u_1 \gets \frac{1}{2}, \tau_1 \gets \frac{3}{16}$
             %\Comment{\textbf{Bucketing Phase}}
            \For{$r=1$ to $R$}%$u_r|S_r| \geq 1$}
            \State $S_{r+1} \gets \emptyset$, $T_r \gets \sqrt{6u_r |S_r| \log\frac{mR}{\beta}}$.
            \State Set $\tilde{\biasesfixed}[S_r] \gets \text{tmean}_{T_r}(\Datafixed^r[S_r]) + \Gauss(0, \frac{T_r^2}{2 \rho m^2} \mathbb{I})$ \label{Step:gauss1} \Comment{Form noisy bias estimates}
            \For{$j \in S_r$} 
            \If{$\tilde{\biasesfixed}[j] < \tau_r$} \Comment{Compare noisy bias estimate to threshold} 
            \State $S_{r+1} \gets S_{r+1} \cup \{j\}$ \Comment{Send $j$ to next round}
            \EndIf
            \EndFor
            \State $S_{R+r} \gets S_r \setminus S_{r+1}, T_{R+r} \gets T_r$
            \State $\tau_{r+1} \gets \frac{\tau_r}{2}, u_{r+1} \gets \frac{u_r}{2}$
            \EndFor
            \medskip
            
            \textbf{Sampling Phase}
            \State $T_{2R+1} \gets  \sqrt{200 \log\frac{m}{\beta}}, S_{2R+1} \gets S_{R+1}, S_{R+1} \gets S_1 \setminus S_2$
            %\Comment{\textbf{Sampling Phase}}
            \For{$r=R+1$ to $2R+1$} 
            \State $\tilde{\biasesfixed}[S_r] \gets \text{tmean}_{T_{r}}({\Datafixed}^r[S_r]) + \Gauss(0, \frac{T_{r}^2}{2 \rho m^2}\mathbb{I} )$ \label{Step:gauss2} \Comment{Estimate biases using fresh data and noise}
            \For{$j \in S_r$,}
            \State Set $q[j] \gets [\tilde{\biasesfixed}[j]]_0^1$ \label{step:qdef} \Comment{Clip to lie in the interval [0,1]}
            \State Sample $b_j \sim Ber(q[j])$ \Comment{Sample from estimated marginal distribution}
            \EndFor
            \EndFor
            \State \Return $(b_1, \dots, b_d)$
    \end{algorithmic}
\end{algorithm}
%
First, we argue that this algorithm is private.
\begin{lemma}\label{lem:prod-privacy}
$\sampler_{prod}$ is $\rho$-zCDP.
\end{lemma}
\begin{proof}
Each input record $\datafixed_i$ is used only in one round in one phase. Assume without loss of generality that this round is in the bucketing phase.  The $\ell_2$-sensitivity of the truncated mean $tmean_{T_r}(\Datarv^r[S_r])$ is $T_r/m$. By the privacy of the Gaussian mechanism (Lemma~\ref{prelim:gauss_cdp}), the step that produces this estimate is $\rho$-zCDP. The remaining steps simply post-process this estimate. Hence, by Lemma~\ref{prelim:postprocess}, Algorithm~\ref{alg:prod} is $\rho$-zCDP.  
\end{proof}

\subsubsection{Overview of Accuracy Analysis}
We analyze the two phases of Algorithm~\ref{alg:prod} separately. Our analysis of the bucketing phase mirrors that of~\cite{KLSU19}. (Their results are not directly applicable to our setting because our algorithm use fewer samples. We therefore give new lemma statements and proofs.)

In Section \ref{sec:techlemmas}, we prove technical lemmas that are used multiple times in the analysis of both phases. In Section~\ref{sec:bucketing}, we show that with high probability, the bucketing phase is successful---that is, we classify all of the attribute biases into the right buckets. This is encapsulated by Lemma~\ref{thm:bucketsuccess}. This corresponds to obtaining good multiplicative approximations of all attribute biases except the smallest ones, for which we obtain good additive approximations.

Next, in Section~\ref{sec:sampling}, we prove a key lemma regarding the success of the sampling phase. 

%\asnote{Wishlist item: combine this summary with the overview in the previous section. }
The intuition behind the analysis of this phase is as follows. Algorithm $\sampler_{prod}$ samples its output from a product distribution. Since the input distribution is in $\cB^{\otimes d}$, each attribute of an input record is sampled from a Bernoulli distribution. By the subadditivity of total variation distance for product distributions, the overall accuracy of $\sampler_{prod}$ can be bounded by showing that $\sampler_{prod}$ samples each attribute independently from a Bernoulli distribution with bias close to the true attribute bias $\biasesfixed_j$. 

The main idea is that the empirical attribute bias has expectation equal to the true attribute bias. Additionally, to preserve privacy, we add zero-mean Gaussian noise. Hence, a noisy empirical estimate of the true attribute bias has mean equal to that attribute bias. If we knew for sure that the noisy empirical estimate for an attribute bias in the sampling phase was always between $0$ and $1$, then the sampler would sample this attribute from exactly the right Bernoulli distribution. 
    
Alas, the noisy empirical estimate of an attribute bias could be less than $0$ or larger than $1$, and we would have to clip it to the interval $[0,1]$ before sampling. This clipping introduces error since we no longer necessarily sample from the right distribution in expectation. We get around this by proving that for attribute biases $\biasesfixed_j$ larger than $\frac{\alpha}{d}$, clipping happens with low probability, and hence the loss in accuracy caused by clipping is small in expectation. However, for attribute biases $\biasesfixed_j$ that are smaller than $\frac{\alpha}{d}$, clipping could occur with high probability. For such attribute biases, we argue that the absolute difference between the clipped noisy empirical mean estimates and the true attribute biases is small enough (at most $\frac{\alpha}{d}$) with high probability. This argument is described in Lemma~\ref{lem:mainacc}.

Finally, we prove the main upper bound theorem in Section~\ref{sec:prod-ub-main} by putting everything together.

\subsubsection{Analysis of Good Events}\label{sec:techlemmas}
In the accuracy analysis, we assume that 
$m \geq    \frac{1200 \ d}{\alpha\sqrt{2 \rho}} \cdot \log^{5/4}\paren{\frac{d R}{\alpha \beta \sqrt{2\rho}}}$ and $d$ is sufficiently large (that is, greater than some positive constant).
%
In this section, we define three good events $G_1,G_2,$ and $G_3$ that, respectively, represent that empirical means are close to the attribute biases in all $2R+1$ datasets into which Algorithm~\ref{alg:prod} subdivides its input, that truncation does not occur in any round (assuming successful bucketing for that round), and that the added Gaussian noise is sufficiently small. We show that each of these events fails to occur only with small probability.


First, we prove that the empirical means are close to the attribute biases with high probability. Define the empirical mean $\hat{\biasesfixed}_r[j] := \frac{1}{m} \sum_{i=1}^m \datafixed_i^{r}[j]$.




\begin{lemma}\label{lem:empiricalest}
Let $G_1$ be the event that for all rounds $r \in [2R+1]$, the following conditions hold:
%Fix a round $r \in [2R+1]$. Then each of the below events occurs with probability at least $1-\frac{\beta}{R+1}$.
\begin{enumerate}
    \item For all $j \in [d]$, if $\frac{\alpha}{d} \leq \biasesfixed_j \leq \frac{1}{2}$ then $|\hat{\biasesfixed}_r[j] - \biasesfixed_j| \leq \frac{\biasesfixed_j}{16}.$
    \item For all $j \in [d]$, if $\biasesfixed_j < \frac{\alpha}{d}$ then  $|\hat{\biasesfixed}_r[j] - \biasesfixed_j| \leq \frac{\alpha}{4d}$.
\end{enumerate}
Then $\Pr\Big[\overline{G_1}\Big]\leq 2\beta,$ where the probability is over the randomness of the input data and the coins of $\sampler_{prod}$.
\end{lemma}

\begin{proof}
Fix $r\in [2R+1]$ and $j \in [d]$. Note that $\E[\hat{\biasesfixed}_r[j]] = \biasesfixed_j$ for all $r\in [2R+1]$. 

We prove Item 1 of the lemma by a case analysis on $\biasesfixed_j$. First, when  $\biasesfixed_j \geq \frac{\alpha}{4d}$, we use the multiplicative Chernoff bound from Claim~\ref{claim:cher_bounds} for $\gamma \in (0,1)$:
\begin{align*}
  \Pr\Big[\hat{\biasesfixed}_{r}[j] > \biasesfixed_j\Big(1 + \frac{\alpha}{4d\biasesfixed_j}\Big)\Big] 
  \leq \exp\Big({-\frac{\alpha^2 \biasesfixed_j m}{48 d^2(\biasesfixed_j)^2}}\Big) 
  = \exp\Big({-\frac{\alpha^2m}{48 d^2\biasesfixed_j}} \Big)
  \leq \exp\Big({-\frac{\alpha}{12 d}\frac{1000 d}{\alpha}} \log\frac{dR}\beta \Big)
  \leq \frac{\beta}{4d(R+1)},
\end{align*}
where in the third inequality we used that $\biasesfixed_j \leq \frac{\alpha}{d}$ and substituted in a lower bound for~$m$. 

Secondly, when $\biasesfixed_j < \frac{\alpha}{4d}$, we use the multiplicative Chernoff bound for all $\gamma > 0$ from Claim~\ref{claim:cher_bounds}:
\begin{align*}
  \Pr\Big[\hat{\biasesfixed}_r[j] > \biasesfixed_j\Big(1 + \frac{\alpha}{4d\biasesfixed_j}\Big)\Big] 
  &\leq \exp\Big(- \frac{\alpha^2\biasesfixed_j m}{16d^2\biasesfixed_j^2(2 + \frac{\alpha}{4d\biasesfixed_j})}\Big)
  \leq \exp\Big(- \frac{\alpha^2 m}{12d^2\biasesfixed_j(\frac{\alpha}{d})}\Big)\\
  &= \exp\Big(- \frac{\alpha m}{12d\biasesfixed_j}\Big)
  \leq  \exp\Big({-\frac{\alpha}{12 d} \frac{1000 d\log(dR/\beta)}{\alpha}} \Big)
  \leq \frac{\beta}{4d(R+1)},
\end{align*}
where in the first inequality we used that since  $\frac{\alpha}{4d\biasesfixed_j} > 1$,  $\frac{\alpha}{4d\biasesfixed_j} + 2 \leq 3\frac{\alpha}{4d\biasesfixed_j}$, and in the third inequality we substituted a lower bound for the value of $m$ and upper bounded $\biasesfixed_j$ by $\alpha$.  

Similar inequalities hold for the lower tails of $\hat{\biasesfixed}_r[j]$. Taking a union bound over all $j \in [d]$ such that $\biasesfixed_j \leq \frac{\alpha}{d}$ completes the the proof of Item~1 in Lemma~\ref{lem:empiricalest}.

Next, assume that $\frac{\alpha}{d} \leq \biasesfixed_j \leq \frac{1}{2}$. By the Chernoff bound from Claim~\ref{claim:cher_bounds} for $\gamma \in (0,1)$, 
\begin{align*}
    \Pr\left[\hat{\biasesfixed}_r[j] - \biasesfixed_j \geq \frac{\biasesfixed_j}{16}\right] 
    = \Pr\Big[\hat{\biasesfixed}_r[j] 
    \geq \biasesfixed_j \Big(1 + \frac{1}{16}\Big)\Big] 
    \leq \exp\left(-\frac{\biasesfixed_j m}{3\cdot 16}\right)
    \leq \frac{\beta}{2d(R+1)},
\end{align*}
where the final inequality holds since $\biasesfixed_j m \geq \frac{\alpha}{d}1000\frac{d}{\alpha}\log\frac{dR}{\beta}$. A similar bound holds for the lower tail of $\hat{\biasesfixed}_r[j]$. Taking a union bound over all $j \in [d]$, and all $r \in [2R+1]$ gives the result.
\end{proof}

Next, we argue that truncation is unlikely in any round (given successful bucketing). Recall that $u_r=1/2^r$ for all $r\in[R]$ (see Algorithm~\ref{alg:prod}). For all $r \in [R]$, let $u_{R+r}=u_r$. Let $u_{2R+1}=20/d$. A version of the following lemma is stated and proved in~\cite{KLSU19} (for us, the smallest upper bound of a bucket is $u_{2R+1}=20/d$ instead of $1/d,$ but the truncation ceiling $T_{2R+1}$ is also larger than in~\cite{KLSU19} to balance this out.)

\begin{lemma}[\cite{KLSU19}, Claims 5.10 and 5.18]\label{lem:trunc}
Let $G_2$ be the following event that, for every round $r\in [2R+1]$, the following holds: 
if $\biasesfixed_j \leq u_r$ for all $j \in S_r$, then 
%Fix a round $r \in [2R+1]$. Assume that $\biasesfixed_j \leq u_r$ for all $j \in S_r$. Let $h$ be the probability that 
for every $i \in [m]$, $$\|\datafixed_i^r [S_r] \|_2 \leq  T_r,$$ 
that is, no rows are truncated in the calculation of $\text{tmean}_{T_r}(\Datafixed^r[S_r])$ in Steps~\ref{Step:gauss1} or~\ref{Step:gauss2} of Algorithm~\ref{alg:prod}. 
%Then, the following statements are true.
%\begin{enumerate}
%    \item If $r \in [2R]$ then $h \geq 1 - \frac{\beta}{R}$.
%     \item If $r = 2R+1$ then $h \geq 1 - \beta$.
%\end{enumerate}
Then $\Pr\Big[\overline{G_2}\Big]\leq 3\beta,$ where the probability is over the randomness of the input data and the coins of $\sampler_{prod}$.
\end{lemma}

Finally, we prove that the amount of noise added in any round is unlikely to be large.
%
For all $r \in [2R+1]$, let $Z_r$ be a $d$-dimensional random vector representing the noise added in round $r$ as in Steps~\ref{Step:gauss1} and~\ref{Step:gauss2} of Algorithm~\ref{alg:prod}. For attributes $j \in [d]$ to which no noise is added in round $r$, the coordinate $Z_r[j] = 0$. The remaining $Z_r[j]$ are drawn from independent zero-mean Gaussians with standard deviation specified in Steps~\ref{Step:gauss1} and~\ref{Step:gauss2} of Algorithm~\ref{alg:prod}. 

\begin{lemma}\label{lem:Gaussnoise}
%Let $m \geq \frac{1200d \log^{5/4} \left( \frac{d R}{\alpha \beta (2\rho)^{1/2}} \right)}{\alpha (2 \rho)^{1/2}}$. 
Let $G_3$ be the event that for all rounds $r \in [2R+1]$, for all $j \in S_r$,
\begin{equation*}
    |Z_r[j]| \leq \frac{\alpha u_r}{100}.
\end{equation*}
Then $\Pr\Big[\overline{G_3}\Big]\leq \beta,$ where the probability is over the randomness of the input data and the coins of $\sampler_{prod}$. 
\end{lemma}
\begin{proof}
For rounds $r \in [2R]$, the standard deviation of univariate Gaussian noise $Z_r[j]$ added in round $r$ is $\sigma_r = \sqrt{\frac{3u_r |S_r|}{\rho m^2} \log \frac{mR}{\beta}}$. Set $t = \sqrt{2 \ln \frac{6dR}{\beta}}$. By Lemma~\ref{lem:gaussconc} on the concentration of a zero-mean Gaussian random variable along with a union bound,
$$ \Pr(\max_{j \in S_r} |Z_r[j]| \geq t \sigma_r) \leq \sum_{j \in S_r} \Pr(|Z_r[j]| \geq t \sigma_r) \leq  \sum_{j \in S_r} 2e^{-t^2/2} \leq \frac{\beta}{2R+1}.$$
Since $m\geq\frac{600 d}{\alpha \sqrt{\rho}} \log^{5/4}(\frac{dR}{\alpha \beta \sqrt{\rho}})$ and , and because $u_r \geq \frac{40}{d}$, for all $r \in [2R]$,
\begin{align*}
    t \sigma_r & =
    \sqrt{\frac{6 u_r |S_r| \log \frac{mR}{\beta} \ln \frac{6dR}{\beta} }{\rho m^2}} 
     \leq \alpha \sqrt{\frac{6 u_r |S_r| \log \frac{dR}{\alpha \beta \sqrt{\rho}} \ln \frac{6dR}{\beta} }{36000 d^2 \log^{10/4}(\frac{dR}{\alpha \beta \sqrt{\rho}})}} 
    \leq \alpha \sqrt{\frac{6 u_r d}{3600 d^2 \log^{1/4}(\frac{dR}{\beta})}} \leq \frac{\alpha u_r}{100}  ,
\end{align*}
where the first inequality is because $\log\frac{mR}{\beta} / m$ is a decreasing function for $m$ and hence we can upper bound the expression by using a lower bound of $m$. We also use the fact that the term $\log \frac{mR}{\beta} \leq 10 \log \frac{dR}{\alpha \beta \sqrt{\rho}})$. The second inequality follows by cancelling out some log terms and using the fact that $|S_r| \leq d$, and the last inequality follows because $\frac{1}{d} \leq \frac{u_r}{40}$, $\beta \leq 1$, and because $d$ is sufficiently large.
For $r = 2R+1$, the standard deviation $\sigma_r = \sqrt{\frac{100}{\rho m^2}\log\frac{m}{\beta}}$, so with the same value of $t$, we get the same result. Taking a union bound over all $r \in [2R+1]$ gives the result. 
\end{proof}
The following corollary summarizes our analysis of good events and follows from Lemmas~\ref{lem:empiricalest}%
%,~\ref{lem:trunc}, and~
--\ref{lem:Gaussnoise} by a union bound.
\begin{corollary}\label{cor:good-event-prod}
Let $G$ be the event $G_1\cap G_2\cap G_3.$ Then $\Pr[\overline{G}] \leq 6 \beta$, where the probability is over the randomness of the input data and the coins of $\sampler_{prod}$.
\end{corollary}
\subsubsection{Success of the Bucketing Phase}\label{sec:bucketing}
In this section, we argue that if the good event $G$ occurs, then the bucketing phase succeeds.

\begin{lemma}\label{thm:bucketsuccess} 
Let \bucket be the event that that the bucketing procedure is successful, namely, for all rounds $r \in [R]\cup\{2R+1\}$ and for all coordinates $j \in [d]$, the following statements hold:
\begin{enumerate}
    \item If $r \in [R]$ and $\biasesfixed_j \in S_{R+r}$, then $u_r / 4 \leq \biasesfixed_j \leq u_{r}$.
    \item If $p_j \in S_{2R+1}$ then
    $\biasesfixed_j \leq u_{2R+1}$.
\end{enumerate}
If the good event $G$ defined in Corollary~\ref{cor:good-event-prod} occurs then $\bucket$ occurs.
%If $\Datafixed^1, \dots, \Datafixed^{2R+1}$ each contain $m$ records, where $m \geq \frac{1200d}{\alpha \sqrt{2 \rho}} \log^{5/4} \left( \frac{d R}{\alpha \beta \sqrt{2\rho}} \right)$, then \bucket occurs with probability at least $1 - 4\beta$ over the randomness of the input data and the coins of $\sampler_{prod}$.
\end{lemma}
\begin{proof}
Assume that $B$ occurs. We prove this lemma by induction on $r$. Recall that $S_{R+r}=S_{r} \setminus S_{r+1}$ for all $r \in [R]$. To prove Item 1, we show that, for all rounds $r \in [R]$, if $j \in S_r$ then $\biasesfixed_j \leq u_r$, and if $j \not \in S_r$ then $\biasesfixed_j \geq u_r / 2$.
For the first round (the base case of the induction), since $u_1 = 1/2$, and since by assumption $\biasesfixed_j \leq 1/2$ for all $j \in [d]$, we have that $\biasesfixed_j \leq u_1$. Additionally, since $S_1 = [d]$, it vacuously holds that $p_j \geq u_{1}/2$ for all $j \not \in S_{1}$. Next, fix any $r \in [R-1]$. The inductive hypothesis is that for round $r$, if $j \in S_{r}$ then $\biasesfixed_j \leq u_{r}$ and if $j \not \in S_{r}$ then $\biasesfixed_j \geq u_r / 2 = u_{r+1}$.

We prove that this statement holds for round $r+1$. 
For all $j \in S_r$, let $\Tilde{\biasesfixed}_r[j]$ be the noisy empirical estimate obtained for coordinate $j$ in Step~\ref{Step:gauss1} of Algorithm~\ref{alg:prod} (in round $r$).
%
By Item 1 of the definition of event~$G_1$, for all $j \in S_r$ with $p_j > \frac{\alpha}{d}$,
$$|\hat{\biasesfixed}_r[j] - \biasesfixed_j| \leq \frac{\biasesfixed_j}{16} \leq \frac{u_r}{16},$$
where the second inequality is by the induction hypothesis.
%
Similarly, by Item 2 of the definition of event $G_1$, for all $j \in S_r$ with $p_j \leq \frac{\alpha}{d}$,
$$|\hat{p}_r[j] - p_j| \leq \frac{\alpha}{4d} \leq  \frac{u_r}{160},$$
where the second inequality holds since $u_r \geq \frac{40}{d}$ for all $r \in [R]$ and $\alpha \leq 1$. 

By the inductive hypothesis, $\biasesfixed_j \leq u_r$ for all $j \in S_r$. Hence, by the definition of event $G_2$, no truncation occurs in round $r$. Also, $|Z_r[j]| \leq \frac{u_r}{100}$, by the definition of event $G_3$, since $\alpha \leq 1$. Hence, for all $j \in S_r$,
\begin{align*}
    |\hat{\biasesfixed}_r[j] - \Tilde{\biasesfixed}_r[j]| \leq \frac{u_r}{100}.
\end{align*}

By the triangle inequality, we get that for all $j \in S_r$,
\begin{equation}\label{eq:triangle-bucketing}
|\Tilde{\biasesfixed}_r[j] - \biasesfixed_j| \leq  |\hat{\biasesfixed}_r[j] - \Tilde{\biasesfixed}_r[j]| + |\hat{\biasesfixed}_r[j] - \biasesfixed_j| \leq + \frac{u_r}{100} + \frac{u_r}{16}  \leq \frac{u_r}{8}.
\end{equation}
%
Fix any $j \in S_r$. Recall that $\tau_r = \frac{3u_{r+1}}{4}$. If $\Tilde{\biasesfixed}_r[j] \leq \tau_r$ then, by (\ref{eq:triangle-bucketing}), $\biasesfixed_j \leq \tau_r + \frac{u_r}{8} = \frac{3u_{r+1}}{4} + \frac{u_{r+1}}{4} = u_{r+1}$. 
%
Similarly, if $\Tilde{\biasesfixed}_r[j] \geq \tau_r$, then $\biasesfixed_j \geq \frac{3u_{r+1}}{4} - \frac{u_{r+1}}{4} = \frac{u_{r+1}}{2}$.

This completes the inductive step and proves that at the beginning of round $r+1$, we have that $p_j \leq u_{r+1}$ for all  $j \in S_{r+1}$, and $p_j \geq \frac{u_{r+1}}{2}$ for all $j \not \in S_{r+1}$. Item 2 follows from an extension of the same argument.
\end{proof}

\subsubsection{Success of the Sampling Phase}\label{sec:sampling}
\begin{lemma}[Success of sampling phase]\label{lem:mainacc}
For all $j \in [d]$, for $q[j]$ defined as in Step~\ref{step:qdef} of algorithm $\sampler_{prod}$, when $\sampler_{prod}$ is run with failure probability parameter $\beta \in(0, \frac{1}{12}]$ and target accuracy $\alpha \in(0, 1]$, 
\begin{enumerate}
    \item if  $\frac{\alpha}{d} < \biasesfixed_j \leq \frac 1 2$, then
    $\displaystyle| \mathbb{E}[ q[j] - \biasesfixed_j ] | \leq 12\beta;$
    \item if  $\biasesfixed_j \leq \frac{\alpha}{d}$, then
    $\displaystyle | \mathbb{E}[ q[j] - \biasesfixed_j ] | \leq \frac{\alpha}{2d} + 6\beta;$
\end{enumerate}
where the expectations are taken over the randomness of the data and the noise.
\end{lemma}
\begin{proof}
We start by proving Item 1. 

Fix any $j \in [d]$ with $\frac{\alpha}{d} \leq \biasesfixed_j \leq \frac{1}{2}$. First, we argue that if event $G$ occurs, then no noisy empirical means are clipped in the sampling phase. By construction, $(S_{R+1},\dots,S_{2R+1})$ is a partition of $[d]$. For all $j\in[d]$, let $r(j)$ denote the round $r\in\{R+1,\dots,2R+1\}$ such that $j\in S_r.$
%
%
%\begin{lemma}\label{claim:no_clip}
% Let $\clipped$ be the event that some $j \in [d]$ with $\frac{\alpha}{d} < \biasesfixed_j \leq \frac 1 2$ gets clipped in its last round $r^j$. Then,
%$$\Pr[\clipped] \leq 7\beta,$$
%where the probability is over the noise and dataset.
%\end{lemma}
%\begin{proof}
%Recall the definition of event \bucket, successful bucketing. By the law of total probability and  Lemma~\ref{thm:bucketsuccess},
%\begin{align}
%    \Pr[\clipped ] = \Pr[\clipped \mid \bucket ] \Pr(\bucket) + \Pr[\clipped \mid \overline{\bucket} ] \Pr(\overline{\bucket}) \leq \Pr[\clipped \mid \bucket ] + 4\beta. \label{eq:intermed}
%\end{align}
Now suppose $G$ occurred. 
By the triangle inequality and since $G$ implies $G_1,G_2,G_3,$ and $\bucket,$
\begin{align}\label{eq:clip}
    |\biasesfixed_j- \tilde{\biasesfixed}_{r(j)}[j]| \leq |\biasesfixed_j- \hat{\biasesfixed}_{r(j)}[j]| + |\hat{\biasesfixed}_{r(j)}[j]- \tilde{\biasesfixed}_{r(j)}[j]| \leq \frac{\biasesfixed_j}{16} + |Z_{r(j)}[j]| \leq \frac{\biasesfixed_j}{16} + \frac{\alpha u_{r(j)}}{100} \leq \frac{\biasesfixed_j}{3},
\end{align}
where the second inequality is by the definition of event $G_1$, the fact that $G$ implies $\bucket$, and the definition of event $G_2$ and the third inequality is by the definition of event $G_3$. The final inequality uses the fact that event $B$ occurs; if $\biasesfixed_j > \frac{5}{d}$ then $u_{r(j)} \leq 4\biasesfixed_j$, and otherwise $u_{r(j)} = \frac{20}{d}$ and hence $\frac{\alpha u_{r(j)}}{100} = \frac{20 \alpha }{100 d} \leq \frac{\biasesfixed_j}{5}$.

If $G$ occurs, by (\ref{eq:clip}) and since $0 < \biasesfixed_j \leq \frac{1}{2}$, we have
\begin{align*}
    0 < \frac{2\biasesfixed_j}{3} \leq  \tilde{\biasesfixed}_{r(j)}[j]
    \leq \frac{4\biasesfixed_j}{3} < 1,
\end{align*}
and thus $\tilde{\biasesfixed}_{r(j)}[j]$ does not get clipped.


Next, by the law of total expectation,
\begin{align*}
   \mathbb{E}[q[j] -  \biasesfixed_j]
   & =  \mathbb{E}[q[j] -  \biasesfixed_j \mid G] \cdot \Pr[G]
    +  \mathbb{E}[q[j] -  \biasesfixed_j \mid \overline{G}]\cdot \Pr[\overline{G}] \\
   & \leq \mathbb{E}[q[j]  \mid G] - \biasesfixed_j  + 6\beta  \leq \frac{\mathbb{E}[\hat{\biasesfixed}_{r(j)}[j] + Z_{r(j)}[j]] }{\Pr[G]}- \biasesfixed_j  + 6\beta \nonumber\\
   & \leq \frac{\biasesfixed_j }{1-6\beta}- \biasesfixed_j   + 6\beta \leq \frac{1}{2}\left(\frac{1}{1-6\beta}- 1\right)  + 6\beta \leq 12 \beta, 
\end{align*}
where the first inequality holds by Corollary~\ref{cor:good-event-prod} and the fact that $\mathbb{E}[q[j] -  \biasesfixed_j] \leq 1$. The second inequality uses the fact that when $G$ occurs, there is no clipping and truncation, and $\E[A \mid E] \leq \E[A]/\Pr[E]$ for all random variables $A$ and events $E$. The third inequality is by the fact that $\mathbb{E}[\hat{\biasesfixed}_{r(j)}[j]] = \biasesfixed_j$ and  $\E[Z_{r(j)}[j]] = 0$, and by Corollary~\ref{cor:good-event-prod}. The last inequality holds because $\beta \leq \frac {1}{12}$ and $\biasesfixed_j \leq \frac{1}{2}$ by assumption. 

Analogously, $\mathbb{E}[\biasesfixed_j - q[j]] \leq 12 \beta$, which completes the proof of Item 1.

Next, we prove Item 2. Recall the event $G$ defined in Corollary~\ref{cor:good-event-prod} and the event $B$ defined in Lemma~\ref{thm:bucketsuccess}. 
Fix a coordinate $j \in [d]$ with $\biasesfixed_j \leq \frac{\alpha}{d}$.
By Lemma~\ref{thm:bucketsuccess}, the law of total expectation, and the fact that $|\E[ q[j] - \biasesfixed_j \mid \overline{\bucket}]| \leq 1$, we get
\begin{align}
 |\E[ q[j] - \biasesfixed_j ]|
  \leq |\E[ q[j] - \biasesfixed_j \mid G]|\cdot\Pr[G] + |\E[ q[j] - \biasesfixed_j \mid \overline{B}]|\cdot\Pr[\overline{G}] 
 \leq |\E[ q[j] - \biasesfixed_j \mid G]| + 6\beta. \label{eq:interim3}
\end{align}
Now, we show that $|\E[ q[j] - \biasesfixed_j \mid G| \leq  \frac{\alpha}{2d}$. Conditioned on event $G$, using Lemma~\ref{thm:bucketsuccess}, event $B$ occurs, and the output bits for all coordinates $j$ with $\biasesfixed_j \leq \frac{\alpha}{d}$ are sampled in round $2R+1$.  Conditioned on event $G$, using the fact that event $B$ occurs, and using the definition of event $G_2$ on truncation of empirical estimates, 
$$|[\tilde{\biasesfixed}_{2R+1}[j]]_0^1 - \biasesfixed_j | \leq |\tilde{\biasesfixed}_{2R+1}[j] - \biasesfixed_j| = |Z_{2R+1}[j]| \leq \frac{\alpha u_{2R+1}}{100} \leq \frac{\alpha}{2d},$$
where the second to last inequality is by the definition of event $G_3$, and the last inequality is since $u_{2R+1} = \frac{20}{d}$.
%
Thus, $|\E[ q[j] - \biasesfixed_j \mid G]|\leq \frac{\alpha}{2d}$. Combining this with (\ref{eq:interim3}) proves Item 2 of Lemma~\ref{lem:mainacc}. 
\end{proof}

\subsubsection{Proof of Main Theorem}\label{sec:prod-ub-main}
Finally, we use Lemma~\ref{lem:mainacc} to prove the theorem.
%\begin{theorem}[Detailed upper bound for product distributions]\label{thm:prodacc}
%Suppose $d$ is greater than some sufficiently large constant $c$. 
%Then sampler $\sampler_{prod}$ is $\alpha$-accurate on $\cB^{\otimes d}$ for input datasets of size at least $n$.
%\end{theorem}
\begin{proof}[Proof of Theorem~\ref{thm:bernoulli-product-alg}.]
Fix $\rho \in(0, 1], \alpha \in (0,1), \beta = \frac{\alpha}{12d}$, and $R = \log_2 (d/40)$. Fix the sample size 
$$n = \frac{1200(2R+1)d}{\alpha \sqrt{2 \rho}}  \log^{5/4}  \frac{dR}{\alpha \beta \sqrt{2\rho}} = \tilde{O} \Big(\frac{d}{\alpha\sqrt{\rho}} \Big)$$ for this setting of $\beta$ and $R$. 

First, by Lemma~\ref{lem:prod-privacy}, we have that $\sampler_{prod}$ is $\rho$-zCDP.

Next, we reason about accuracy. Let $\distroutput{\sampler_{prod}, \distr^{\otimes d}}$ be the distribution of the output of the sampler $\sampler_{prod}$ with randomness coming from the data and coins of the algorithm. Observe that  $\distroutput{\sampler_{prod}, \distr^{\otimes d}}$ is a product distribution and that the marginal bias of each coordinate $j \in [d]$ is $\mathbb{E}[q[j]]$. Let the marginal distributions of  $\distroutput{\sampler_{prod}, \distr^{\otimes d}}$  be $Q_1, \dots, Q_d$. By the subadditivity of total variation distance between two product distributions (Lemma~\ref{lem:subaddTV}),
\begin{align*}
    d_{TV}(\distroutput{\sampler_{prod}, \distr}, \distr^{\otimes d}) 
    & \leq \sum_{i=1}^d d_{TV}(Q_i, \distr) 
     = \sum_{i=1}^d |\mathbb{E}[ q[j]-  \biasesfixed_j]| \\
    & = \sum_{i: \biasesfixed_i > \frac{\alpha}{d}}  |\mathbb{E}[ q[j] -  \biasesfixed_j]| + \sum_{i: \biasesfixed_i \leq \frac{\alpha}{d}}  |\mathbb{E}[ q[j] -  \biasesfixed_j]| \\
    & \leq  \sum_{i: \biasesfixed_i > \frac{\alpha}{d}} 12\beta  + \sum_{i: \biasesfixed_i \leq \frac{\alpha}{d}} \left(\frac{\alpha}{2d} + 6\beta\right)
    \leq \alpha,
\end{align*}
where we got the first equality by substituting the expression for the total variation distance between two Bernoulli distributions, the second inequality is by Lemma~\ref{lem:mainacc} (since $\beta = \frac{\alpha}{12d} \leq \frac{1}{12}$, this lemma is applicable), and the final inequality holds because $\beta = \frac{\alpha}{12d}$.
\end{proof}

Finally, we complete this section by proving Theorem~\ref{thm:bernoulli-product-alg-intro} from the introduction. 
\begin{proof}[Proof of Theorem~\ref{thm:bernoulli-product-alg-intro}]
Set $\rho = \frac{\eps^2}{16\log(1/\delta)}$. By Lemma~\ref{prelim:relate_dp_cdp}, for all $\delta\in(0,1/2]$, algorithm $\sampler_{prod}$ is $(\eps, \delta)$-differentially private. Substituting this value of $\rho$ into Theorem~\ref{thm:bernoulli-product-alg}, we get that the sampler $\sampler_{prod}$ is $\alpha$-accurate for input datasets of size
$$n = O\left( \frac{d}{\alpha \eps} \cdot \sqrt{\log\frac{1}{\delta}}  \Bigg( \log^{9/4} d + \log^{5/4} \frac{\log\frac{1}{\delta}}{\alpha \eps} \Bigg) \right).$$ For $\log(1/\delta) = polylog(n)$, we get that $\sampler_{prod}$ is $\alpha$-accurate for input datasets of size 
$n = \tilde{O}(\frac{d}{\alpha \eps})$. This proves the theorem, since $\delta \leq 1/2$.
\end{proof}

%\newpage

%%%%%%%%%%%%%%%%%%%%%
%%%%%%%%%%%%%%%%%%%%%% PRODUCT LOWER BOUND AFTER THIS IN PROGRESS
%%%%%%%%%%%%%%%%%%%%%%%%%

\newpage 
\subsection{Lower bound for Products of Bernoulli Distributions}\label{sec:prod-lb}
In this section, we prove the following theorem.
\begin{theorem}\label{thm:product-bern-lb}
For all $d,n\in\N$, $\eps \in (0,\frac{1}{2}]$, $\delta \in [0,\frac 1 {5000n}]$, and sufficiently small $\alpha>0$, every $(\eps,\delta)$-differentially private sampler that is $\alpha$-accurate on the class of products of $d$ Bernoulli distributions needs datasets of size $n=\Omega(\frac{d}{\alpha \eps})$.
\end{theorem}

The theorem is proved via a reduction from the problem of private sampling from discrete $k$-ary distributions.
Recall that in Section~\ref{sec:kary}, we proved a lower bound for this problem
%sampling from discrete $k$-ary distributions 
by considering the class $\carb$ (see Definition~\ref{def:ksubclass}) and giving a lower bound of $\Omega(k/\alpha \eps)$ on the sample complexity of $\eps$-differentially private $\alpha$-accurate sampling from this class. For each distribution in $\carb$, we consider the corresponding product distribution, defined next.

\begin{definition}\label{def:corrprod}
For every distribution $\distr \in \carb$ with a special element $s$,  its {\em corresponding product distribution} is the distribution $\corr \in \cB^{\otimes 2k}$, where the bias of each coordinate $j\in[2k] $ of $\corr$ is $P(j)$ for $j<s$ and $P(j+1)$ for $j\in [s,2k]$ (note that the special element is eliminated). 
\end{definition}


%For any distribution $\distr$ in $\carb$, we can define a related product distribution (which we call the corresponding probability distribution to $\distr$, see Definition~\ref{def:corrprod}) with $2k$ attributes, with each attribute corresponding to a non-special element of $\distr$, with the attribute biases equal to the corresponding probability masses in $\distr$. 
The initial idea that inspired our final reduction is to create a private sampler for $\carb$ from a private sampler for product distributions as follows. Replace each non-special entry of the input dataset sampled from $\distr \in \carb$ with its one-hot encoding (write element $j \in [2k]$ as a binary vector of size $2k$, with a $1$ in position $j$ and $0$'s in all other positions) to create a dataset that looks like it was drawn from the corresponding product distribution $\corr$. Then apply the private product distribution sampler to this modified dataset. Finally, map the resulting sample in $\{0,1\}^{2k}$ back to $[2k+1]$ by uniformly sampling the index of a random nonzero element and returning the special element if the sample is the all-zeros vector. 

This initial idea does not work, because one-hot encoding produces vector entries in $\{0,1\}^{2k}$ that always have a single $1$, whereas samples from a product distribution could have multiple $1$'s, or no $1$'s. However, the reduction presented here has a similar structure. The starting point of our reduction is Lemma~\ref{lem:k-ary-lb-pois}, which applies to Poisson samplers. It states that for every $(\eps,\delta)$-differentially private sampler that is $\alpha$-accurate on the class $\carb$ with dataset size distributed as $\Po(n)$, we have $n=\Omega(\frac{k}{ \alpha\eps})$.
The Poisson sampler $\sampler_{red}$ for $\carb$ in our reduction is described in Algorithm~\ref{alg:redsampler}. It first privately identifies the ``special element'' (using a simple instantiation of the exponential mechanism described in Algorithm~\ref{alg:special_element}) and then proceeds in three steps: {\em dataset transformation, sampling,} and {\em universe transformation.}
{\em Dataset transformation} transforms the input dataset drawn from a distribution $\distr$ in $\carb$ to a dataset distributed as independent samples
%that looks like it was independently sampled 
from the corresponding product distribution $\corr$ in $\cB^{\otimes 2k}$. The {\em sampling step} runs a private sampler $\sampler_{prod}$ for class $\cB^ {\otimes 2k}$ with privacy parameters $\eps$ and $\delta$, accuracy parameter $\alpha$, and the dataset from the previous step to obtain one sample $\prodoutput$ in $\{0,1\}^{2k}$. 
The {\em universe transformation} transforms the output $\prodoutput$ from the previous step into a single sample in $[2k+1]$. This sample is the final output of Algorithm~\ref{alg:redsampler}. We will show that if the original input table was drawn i.i.d.\ from a distribution $\carb$, then the final output is a sample from a nearby distribution.

%\begin{inparadesc}%can also use "asparadesc" or "compactdesc" to get line breaks.
%\item [Dataset transformation:] First, the reduction transforms the input dataset drawn from a distribution $\distr$ in $\carb$ to a dataset that looks like it was independently sampled from the corresponding product distribution $\corr$ in $\cB^{\otimes 2k}$ using Algorithm~\ref{alg:datatrans}. 
%\item [Sampling:] Next, it runs a private sampler $\sampler_{prod}$ for class $\cB^ {\otimes 2k}$ on the dataset from the previous step to obtain one sample $y$ in $\{0,1\}^{2k}$. 
%\item[Universe transformation:] Finally, the reduction transforms the output $y$ from the previous step into a single sample in $[2k+1]$ using Algorithm~\ref{alg:univtrans}. This sample is the final output of Algorithm~\ref{alg:redsampler}. We will show that if the original input table was drawn i.i.d.\ from a distribution $\carb$, the final output is a sample from a nearby distribution.
%\end{inparadesc}
\begin{algorithm}
        \caption{Sampler $\sampler_{red}$ for $\carb$ with dataset size $N \sim \Po(n)$}
    \label{alg:redsampler}
    \hspace*{\algorithmicindent} \textbf{Input:} Dataset $\Datafixed = (\datafixed_1,\dots,\datafixed_N)\in [2k+1]^N$, privacy parameters $\eps, \delta>0$, parameters $k,n \in \N$, accuracy parameter $\alpha \in (0,1)$, black box access to private sampler $\sampler_{prod}$ for $\cB^ {\otimes 2k}$ \\
    \hspace*{\algorithmicindent} \textbf{Output:} $y' \in [2k+1]$
    \begin{algorithmic}[1] % The number tells where the line numbering should start
           %\State Let $C$ be a sufficiently large constant. Draw $L \sim \Po(\frac{2C \log k}{\alpha \eps})$ \sstext{from the coupling witnessing the stochastic domination of $\Po(n)$ over $\Po(\frac{2C \log k}{\alpha \eps})$, conditioned on $\Po(n)=N$. If no such coupling exists, fail.} 
           \State Let $C$ be a sufficiently large constant. Draw $L \sim \Bin(N, \frac{2C \log k}{\alpha \eps n})$. Set $R = N-L$. \label{step:datasplitval}
           \State If $L < C\frac{\log k}{\alpha \eps}$, output $2k+1$ and break. \label{step:largespec}
           
           %set $\hat{s }\gets 2k+1$ and skip Steps~\ref{step:partition} and~\ref{step:specialel}. 
           \State Partition $\Datafixed$ into datasets $\Datafixed^L$ and $\Datafixed^R$, where $\Datafixed^L$ has $L$ records.
           \label{step:partition}
           \State Run Algorithm~\ref{alg:special_element} on $\Datafixed^L$, value $k$, and privacy parameter $\frac{\eps}{2}$ to obtain a candidate special element $\hat{s}$. \label{step:specialel}
           \State Run Algorithm~\ref{alg:datatrans} to obtain a dataset $\Datafixedy \gets \reduction^{\rightarrow}(\Datafixed^R, k, n, 60\alpha, \hat{s})$. \label{step:datasettrans}
           %\State Initialize array $M$
           \State $\prodoutput \gets \sampler_{prod}(\frac{\eps}{4},\frac{\delta}{2}, \frac{\alpha}{25}, \Datafixedy)$ \label{step:privateprodsamp}
           
           %Run the private product sampler $A_{prod}$ with privacy parameters $(\eps/20,\delta)$, number of attributes $d = 2k$ and accuracy parameter $\frac{\alpha}{10}$ independently on each of the 10 datasets $\datafixedy_i$ output by the dataset transformation algorithm. \label{step:privateprodsamp}
           \State Run Algorithm~\ref{alg:univtrans} to obtain $z \gets \reduction^{\leftarrow}(\prodoutput, \hat{s})$. %the universe transformation algorithm $\reduction^{\leftarrow}$ (Algorithm~\ref{alg:univtrans}) on the 10 samples produced by the independent runs of the product sampler, with value $\alpha^* = 60\alpha$ and special element $S$, and output the corresponding $y' \in \{2k+1\}$. \label{step:univtrans}
           \State Output $z$.
    \end{algorithmic}
\end{algorithm}

\begin{algorithm}
        \caption{Special element picker $SE$}
    \label{alg:special_element}
    \hspace*{\algorithmicindent} \textbf{Input:} Dataset $\Datafixed = (\datafixed_1,\dots,\datafixed_n)$, privacy parameter $\eps$, value $k$ \\
    \hspace*{\algorithmicindent} \textbf{Output:} $\hat{s} \in [2k+1]$
    \begin{algorithmic}[1] % The number tells where the line numbering should start
           \State For $j \in [2k+1]$, define the utility function $u(\Datafixed, j) = \frac{1}{n}\sum_{i=1}^n \indicator[\datafixed_i = j]$.
           \State Run the exponential mechanism \cite{McTalwar} with privacy parameter $\eps$ and utility function $u(\cdot, \cdot)$ on dataset $\Datafixed$ to choose an index $\hat{s} \in [2k+1]$. (Specifically, each index $j$ is selected with probability $\propto e^{\frac{\eps\cdot u(\Datafixed,j)}{2}}$).
           \State Output $\hat{s}$.
    \end{algorithmic}
\end{algorithm}

We now explain the dataset and universe transformations. 



%First, we describe the reduction. Fix any $n>0$. The first step in the reduction transforms a dataset of size $n$ drawn independently from any distribution in $\carb$ to another dataset that looks like it was independently sampled from a distribution in $\cB^{\otimes 2k}$ with dataset size $n'=\frac{n}{10}$.  This procedure is described in Algorithm~\ref{alg:datatrans}. 




\paragraph{Dataset Transformation (Algorithm~\ref{alg:datatrans})} 
%
The goal in this step is to produce a dataset that is distributed as an i.i.d.\ sample from a product distribution. Such distributions are characterized by two properties: 
\begin{enumerate}
    \item The marginal distributions over the columns are independent.
    \item The bits of each column are  mutually independent Bernoulli random variables.
\end{enumerate}
%The distribution of a dataset sampled from a product distribution can be completely specified by the following properties- the marginal distributions over the columns are mutually independent, the number of ones in each column is distributed as a binomial random variable with appropriate success probability, and all permutations of the ones in each column are equiprobable.
%
The dataset transformation uses this characterization to convert a Poisson number of samples from a $k$-ary distribution to a dataset distributed as independent samples
%that looks like it was sampled 
from the corresponding product distribution. %Our idea is inspired by a result of Klenke and Mattner \cite[Theorem 1.f)]{klenke2010stochastic}, who show that $\Po(\lambda)$ stochastically dominates $\Bin(\frac{n}{2}, p)$ for $p < 0.5$ and $\lambda = \frac{n}{2} \ln \frac{1}{1-p}$ with a simple and elegant coupling.  
Algorithm~\ref{alg:datatrans} is inspired by an elegant coupling by Klenke and Mattner \cite[Theorem 1.f]{klenke2010stochastic} that shows that a Poisson random variable stochastically dominates a binomial random variable with sufficiently small probability parameter.  Algorithm~\ref{alg:datatrans} relies on the fact that if the size of the input dataset follows a Poisson distribution, then  the number of appearances $\hist[i]$ of each element $i$ are mutually independent Poisson random variables.  

The transformed dataset is built column by column. We first compute the histogram of the input dataset. Then, for each element $i \in [2k+1] \setminus \{s\}$, we sample random variables $A_1,\dots,A_{n/2}$ from the multinomial distribution with $\hist[i]$ trials and probability vector $(2/n,\dots,2/n)$. We truncate each of these random variables to have value at most $1$. Finally, we independently set each random variable to $0$ with a carefully chosen probability. Column $i$ of the transformed dataset is then set to $(A_1,\dots,A_{n/2})$. (There is no column $s$.)

We argue that after thresholding, $A_1,\dots,A_{n/2}$  are mutually independent Bernoulli random variables. Independently setting each of them to $0$ with some probability adjusts their biases. Additionally, as discussed earlier, the element counts in the input dataset are independent, and hence the marginal distributions of the columns are independent as well. This guarantees that the rows in the dataset look like independently sampled entries from the corresponding product distribution. We formalize these ideas in Lemma~\ref{lem:datatrans}.




\begin{algorithm}
    \caption{Dataset transformation algorithm $\reduction^{\rightarrow}$}
    \label{alg:datatrans}
    \hspace*{\algorithmicindent} \textbf{Input:} Dataset $\Datafixed = (\datafixed_1, \ldots, \datafixed_{N})\in [2k+1]^N$; parameters $k,n \in\N, \alpha^*\in(0,1)$, special element $s$ \\
    \hspace*{\algorithmicindent} \textbf{Output:} Dataset $\Datafixedy = (\datafixedy_1, \ldots, \datafixedy_{n/2})\in (\{0,1\}^{2k})^{n/2}$
    \begin{algorithmic}[1] % The number tells where the line numbering should start
            \State Initialize $\Datafixedy$ as an empty matrix \Comment{We will build the $\frac{n}{2} \times 2k$ matrix column by column}
            \State $\hist \gets histogram(\Datafixed)$ \label{step:hist}
            %\State Compute a coupling $(A,B)$ between random variables distributed as $\Po(\frac{\alpha^*n}{k})$ and $\Bin(n', \alpha/k)$, where $B \leq A$ (see Remark~\ref{rem:poissbincoup} along with the proof of Lemma~\ref{lem:datatrans} for a description of such a coupling) %If such a coupling doesn't exist, \textbf{fail}. \ms{We may want to make this more specific to make sure the correct coupling is used} 
            \For{$i \in [2k+1] \setminus \{s\}$}
                \State Sample $(A_1,\dots,A_{n/2}) \sim Mult(\hist[i], (2/n,\dots,2/n))$. \label{step:mult}
                \For{$j \in [n/2]$}
                \State Set $A_j\gets \min (A_j, 1)$ \label{step:threshold}
                \State Set $A_j\gets 0$ with probability $1-\frac p{1-e^{-2p}}$, where $p=\frac{\alpha^*}k.$ \Comment{Reduce the probability that $A_j$ is $1$}
                %\State Flip $a_j$ with probability $\frac{1-\frac{\alpha^*}{k}-e^{-2\alpha^*/k}}{1-e^{-2\alpha^*/k}}$. 
                \label{step:flip}
                \EndFor
                %\State Set $R_i = \sum_{i=1}^{n/2} min(a_i,1)$. \label{step:condsamp}
                \State Set column $i$ of $\Datafixedy$ to $(A_1,\dots,A_{n/2})$.
                %(Using the method in
                %\State Sample $(A,R_i) \sim (A,B)|_{A=\hist[i]}$ (Using the method in Remark~\ref{rem:sampcoup})\label{step:condsamp}
                %\State $v_i \gets (1, \ldots, 1, 0 ,\ldots, 0)$, where there are $R_i$ ones followed by $n/2 - R_i +1$ zeros
                %\State Choose a permutation $\pi: \universe \rightarrow \universe$ uniformly at random
                %\State Permute $v_i \gets \pi(v_i)$
                %\State Append column $v_i$ to $\Datafixedy$
            \EndFor
            \State Output $\Datafixedy$
    \end{algorithmic}
\end{algorithm}

\paragraph{Universe Transformation (Algorithm~\ref{alg:univtrans})} Our universe transformation procedure takes an element in the universe $\{0,1\}^{2k}$ and converts it to an element in the universe $[2k+1]$. Once we apply our product distribution sampler on the dataset obtained in Algorithm~\ref{alg:datatrans}, we use this procedure to transform the resulting sample so that the final output looks like it was sampled from a distribution in $\carb$. %The procedure is given in Algorithm~\ref{alg:univtrans}. 
%
For any distribution $\distr$ in $\carb$, we call the $k$ non-special elements with nonzero probability mass the \textbf{participating elements}. When the input coordinates are all $0$s, we output the special element. Otherwise, we output a random non-special element corresponding to one of the coordinates with value 1. In Lemma~\ref{lem:easyunivtrans}, we prove that when this procedure is run with a sample from the right product distribution, it outputs a sample from a distribution~$Q$ such that $d_{TV}(\distr,Q)\leq (\alpha^*)^2$.
%
%When we see a $1$ in any of the input coordinates, we randomly output one of the corresponding non-special elements, since there is no reason to discriminate between them (since all $k$ participating elements have the same probability mass in $\distr$).  On the other hand, when the input coordinates are all $0$s, we output the special element. %If multiple indices have a $1$, then we could choose to output one of the corresponding non-special elements randomly, since there is no reason to discriminate between them (since all $k$ participating elements have the same probability mass in $\distr$). 
%It turns out that when run with a sample from the right product distribution, this procedure outputs a sample from a distribution $Q$ such that $d_{TV}(\distr,Q)\leq (\alpha^*)^2$. This is sufficient for our purposes, and 
%is formalized in Lemma~\ref{lem:easyunivtrans}. 
%%In fact, in the appendix, we argue that a more sophisticated adaptation of the above idea can achieve an even better bound of $d_{TV}(\distr,Q) = 0$.

In Lemma~\ref{lem:specialel}, we show that the special element is chosen with high probability. Finally, we combine Lemmas~\ref{lem:datatrans}--\ref{lem:specialel} with the lower bound in Lemma~\ref{lem:k-ary-lb-pois} to prove Theorem~\ref{thm:product-bern-lb}.
\begin{algorithm}
        \caption{Universe transformation algorithm $\reduction^{\leftarrow}$}
    \label{alg:univtrans}
    \hspace*{\algorithmicindent} \textbf{Input:} A sample $\prodoutput \in \{0,1\}^{2k}$, special element $s$ \\
    \hspace*{\algorithmicindent} \textbf{Output:} $z \in [2k+1]$
    \begin{algorithmic}[1] % The number tells where the line numbering should start
           \If{$\prodoutput = (0,\dots,0)$}
           \State Set $z \gets s$
           \Else 
           \State Choose $z$ uniformly at random from $\{j\in[2k+1]: [j<s \wedge \prodoutput_j=1]\vee [j>s \wedge \prodoutput_{j-1}=1]\}$ \hspace{5cm}
\Comment{Shift by 1 to skip the special element} 
           \EndIf 
           \State Output $z$
    \end{algorithmic}
\end{algorithm}

%\begin{remark}[Poisson-binomial coupling \cite{klenke2010stochastic}]\label{rem:poissbincoup}
%Fix $p \in (0,1), n>0$, and let $\lambda = n\ln\frac{1}{1-p}$. Then, $\Po(\lambda)$ stochastically dominates $Bin(n,p)$. The coupling witnessing this is as follows. Let $A_1,\dots,A_n$ be random variables sampled independently from  $\Po(\lambda/n)$. Consider the joint distribution $(A,B)$, where $A= \sum_{i=1}^n A_i$ and $B=\sum_{i=1}^n \min(A_i,1)$. Then the distribution of $A$ is $\Po(\lambda)$.  In the sum $\sum_{i=1}^n \min(A_i,1)$, each term is a Bernoulli random variable with bias $p$; they are mutually independent, and hence $B$ is distributed as $Bin(n,p)$. Additionally, $\sum_{i=1}^n A_i\geq \sum_{i=1}^n \min(A_i,1)$. Hence, $(A, B)$ gives the required coupling.
%\end{remark}

%In our data transformation algorithm, we sample from the joint distribution described in Remark~\ref{rem:poissbincoup} conditioned on the Poisson random variable taking a particular value. We describe a method to do this in Remark~\ref{rem:sampcoup}, which will facilitate our privacy analysis.

%\begin{remark}[Sampling from the coupling]\label{rem:sampcoup}
%This remark describes a method to sample from the joint distribution described in Remark~\ref{rem:poissbincoup}, as well as a way to sample from the joint distribution conditioned on the Poisson random variable taking a particular value. First, sample $A \sim \Po(\lambda)$. Next, sample $(a_1,\dots,a_n)$ from $\Mult(A,(1/n,\dots,1/n))$. Finally, set \ssnote{Sofya: Directly use the RHS} $a_i =\min(a_i,1)$ and output $(A,\sum_{i=1}^n a_i)$. By the fact that the joint distribution of $n$ Poisson Random Variables with equal mean conditional on their sum being A is $\Mult(A,(1/n,\dots,1/n))$, this gives a sample from the coupling. Additionally, to sample from the joint distribution conditional on $\Po(\lambda)=\ell$, sample $(a_1,\dots,a_n)$ from $\Mult(\ell,(1/n,\dots,1/n))$, set $a_i =\min(a_i,1)$ and output $\sum_{i=1}^n a_i$. 
%\end{remark} \msnote{I wonder if there's a better ``container'' to put the coupling information in. The ``remark'' environment is mainly for non-essential side information rather than info that is on the critical path of a proof, I think. Also, could we combine the description of the coupling with how to sample from it?}
%\ssnote{Even $n$ vs odd $n$}

%\begin{lemma}\label{lem:stochdomposbin}
%Fix a sufficiently small $\alpha > 0$, set $\alpha^* = 60 \alpha$, and fix a distribution $\distr \in \cA$. Let $\eps \in (0,1], \delta > 0$, fix $n\geq 2$. Let dataset $\Datarv \sim \distr^{\otimes N}$ where $N\sim \Po(n)$. Then, for all participating elements $i \in [2k+1] \setminus \{S\}$, the random variables $R_i$ defined in line~\ref{step:condsamp} of Algorithm~\ref{alg:datatrans} when the algorithm is run on inputs $\Datarv, k, \alpha^*$ and $n$ are distributed as $Bin(n/2,\alpha^*/k)$, and additionally have the property that $R_i \leq \hist[i]$. 
%\end{lemma}
%\begin{proof}
%Since $a_1,\dots,a_{n/2}$ are sampled from $Mult(\hist[i],(2/n,\dots,2/n))$, we have that $\sum_{i=1}^{n/2} a_i = \hist[i]$. Hence, $R_i \leq \hist[i]$. This proves the second part of the lemma. We now prove the first part. First, observe that for any participating element $i$, the number of occurrences of $i$ is distributed as $\Po(\frac{\alpha^* n}{k}$.



%Next, we argue that Steps~\ref{step:mult} and ~\ref{step:condsamp} of Algorithm~\ref{alg:datatrans} give a sample from this coupling conditional on the Poisson random variable being $\hist[i]$. This is by the fact that the joint distribution of $n/2$ Poisson Random Variables with equal mean conditional on their sum being $\hist(i)$ is $\Mult(\hist(i),(2/n,\dots,2/n))$. Hence, since $\hist(i)$ is distributed as a Poisson random variable, by sampling $a_1,\dots,a_{n/2}$ in this manner, we are guarantee that they are distributed as independent Poisson random variables. Then, in step~\ref{step:condsamp}, we proceed exactly as we do in the regular coupling. 
%\sstext{Add in the algorithm the coupling from $\hist[i]$ to $\lambda$.}
%Hence, we have that $R_i$ is distributed as $Bin(n',\frac{\alpha^*}{k})$.
%\end{proof}


\begin{lemma}\label{lem:datatrans}
Fix a sufficiently small $\alpha > 0$, set $\alpha^* = 60 \alpha$, fix a distribution $\distr \in \carb$, and let $\corr$ be the corresponding product distribution. Let $\eps \in (0,1], \delta \geq 0$, and $n\geq 2$. Let $\reduction^{\rightarrow}$ be Algorithm~\ref{alg:datatrans}. Let dataset $\Datarv \sim \distr^{\otimes N}$ where $N\sim \Po(n)$. Then the random variable $\reduction^{\rightarrow}(\Datarv,k,n,\alpha^*,s)$ 
%is distributed identically to 
has distribution $\corr^{\otimes (n/2)}$. %, where $\corr$ is the product distribution corresponding to $\distr$.
\end{lemma}
\begin{proof}
Let $\Datarv \sim \distr^{\otimes N}$ be the input dataset of size $N \sim \Po(n)$. Let $p=\frac{\alpha^*}{k}$. %The number of times an element $i \in [2k+1]$ occurs in \Datarv is distributed as $\Po(\frac{\alpha^* n}{k})$ if $i$ is a
%participating element of $\distr$ and $\Po((1-\alpha^*)n)$ if $i$ is the special element. 

The output $\Datarvy$ of Algorithm~\ref{alg:datatrans} is a dataset of size $n/2$. We want to show that it is independently sampled from the corresponding product distribution $\corr$. This is equivalent to the following two conditions: 
\begin{enumerate}
    \item For all columns  $i \in[2k]$, the bits of column $i$ in $\Datarvy$ are  mutually independent Bernoullis with bias $p$ if $i$ is a participating element of $\distr$, and bias $0$ if $i$ is non-participating.
    \item The columns in $\Datarvy$ are  mutually independent.
\end{enumerate}

We start by proving the first condition. 
Fix a participating element $i$. Since the input dataset has size distributed as $\Po(n)$, by Lemma~\ref{lem:multtopois}, the count of element $i$ (defined as $\hist[i]$ in Step~\ref{step:hist} of Algorithm~\ref{alg:datatrans}) is distributed as $\Po(pn)$. In Step~\ref{step:mult} of Algorithm~\ref{alg:datatrans}, we draw $A_1,\dots,A_{n/2}$ from $\Mult(\hist[i],(2/n,\dots,2/n))$. By Lemma~\ref{lem:multtopois}, they are mutually independent Poisson random variables with mean $2p$. Thus, the random variables $A_j$ after thresholding at $1$ in Step~\ref{step:threshold} of Algorithm~\ref{alg:datatrans} are mutually independent Bernoulli random variables that are $0$ with probability $e^{-2p}$. After setting each random variable $A_j$ independently to $0$ with some probability in Step~\ref{step:flip} (note that the probability $1-p-e^{-2p} \geq 1-p- (1-2p + 4p^2) \geq  0$ and so this step is well defined), we get that the random variables $A_j$ are mutually independent Bernoullis that are $0$ with probability $$e^{-2p} + (1-e^{-2p}) \frac{1-p-e^{-2p}}{1-e^{-2p}} = 1-p.$$
On the other hand, for non-participating elements $i$, the $A_j$s are sampled from a multinomial distribution with $0$ trials and hence will be identically $0$. They are not changed in the remaining steps of Algorithm~\ref{alg:datatrans}. Hence, they can be thought of as mutually independent Bernoullis with bias $0$.

Next, we prove the second condition. Lemma~\ref{lem:multtopois} applied to the element counts $\hist[i]$ implies that they are mutually independent. Column $i$ in $\Datarvy$ only depends on the element count $\hist[i]$, and hence, the columns of $\Datarvy$ are also mutually independent. 

Thus, the output dataset $\Datarvy$ is correctly distributed.
%For any natural number $n>0$, Klenke and Mattner \cite[Theorem 1.f)]{klenke2010stochastic} (See remark~\ref{rem:poissbincoup}) show that $\Po(\lambda)$ stochastically dominates $\Bin(n, p)$ for $p < 0.5$ and $\lambda = n \ln \frac{1}{1-p})$. The coupling witnessing this is as follows. Let $A_1,\dots,A_n$ be random variables sampled independently from  $\Po(\lambda/n)$. Consider the joint distribution $(A,B)$, where $A= \sum_{i=1}^n A_i$ and $B=\sum_{i=1}^n \min(A_i,1)$. Then the distribution of $A$ is $\Po(\lambda)$.  In the sum $\sum_{i=1}^n \min(A_i,1)$, each term is a Bernoulli random variable with bias $p$; they are mutually independent, and hence $B$ is distributed as $Bin(n,p)$. Additionally, $\sum_{i=1}^n A_i\geq \sum_{i=1}^n \min(A_i,1)$. Hence, $(A, B)$ gives the required coupling. Additionally, consider the following technique for sampling from the coupling conditional on the Poisson random variable being $\hist[i]$. Sample $(a_1,\dots,a_{n/2})$ from $\Mult(\hist[i],(2/n,\dots,2/n))$. Finally, output $(A,\sum_{i=1}^{n/2} min(1,a_i))$. By the fact that the joint distribution of $n$ Poisson Random Variables with equal mean conditional on their sum being A is $\Mult(A,(1/n,\dots,1/n))$, this gives a sample from the coupling. Additionally, to sample from the joint distribution conditional on $\Po(\lambda)=\ell$, sample $(a_1,\dots,a_n)$ from $\Mult(\ell,(1/n,\dots,1/n))$, set $a_i =\min(a_i,1)$ and output $\sum_{i=1}^n a_i$. 
%
%For $p <0.5$, we have $\lambda < 2np$. Thus, $\Po(2np)$ stochastically dominates $\Bin(n,p)$ for $p < 0.5$. Thus, based on the description above, $\Po(\frac{\alpha^* n}{k})$ stochastically dominates $\Bin(n/2, \frac{\alpha^*}k)$ for $\frac{\alpha^*}{k} < 0.5$. Additionally, the number of ones $R_i$, and thus the number of ones in each column, $R_i$, is distributed correctly if $i$ is a participating element of $\distr$. For participating elements $i$, $R_i$ is strictly less than the number of times element $i$ occurs in $\Datarv$. If $i$ has probability mass $0$, by stochastic domination, $R_i=0$, as required. 
\end{proof}
%\begin{definition}
%For any fixed product distribution $\corr$, define the \textbf{corresponding compounded product distribution} $\corr'$ to be the distribution of a vector of 10 independent samples $\Datafixedy_1, \ldots, \Datafixedy_{10} \sim \corr^{\otimes 10}$ from $\corr$. \msnote{Isn't $\corr'$ just denoted $\corr^{\otimes 10}$? Or is there a reason we need this separate definition?}
%\textbf{corresponding zero-shifted product distribution} $\corr'$: sample 10 values $\datafixedy_1, \ldots, \datafixedy_{10} \sim \corr^{\otimes 10}$ from $\corr$ independently. If $\datafixedy_i = 0$ for all $i\in[10]$, then output 0; otherwise, return $\datafixedy_i$ for a uniformly random $i \in [10]$.
%\end{definition}
%\ssnote{Fix distr and let S be special element phrasing- use in other places too.}

\begin{lemma}\label{lem:easyunivtrans}
Let $\alpha \in (0, 1/60]$ and $\alpha^* = 60\alpha$. Fix any distribution $\distr \in \carb$. Let $s$ be its special element and $B$ be a sample from the corresponding product distribution $\corr$. Let $\reduction^{\leftarrow}$ be Algorithm~\ref{alg:univtrans}. Then $d_{TV}(\reduction^{\leftarrow}(B,s),\distr) \leq (\alpha^*)^2$. 
\end{lemma}
\begin{proof}
Let $q$ be the probability that $B \neq (0,\dots,0)$. Since $B$ is sampled from the corresponding product distribution $\corr$, $1-q=(1-\frac{\alpha^*}{k})^k$, which is between $1-\alpha^*$ and $ 1-\alpha^* + (\alpha^*)^2$ for all $\alpha^* \in (0,1)$ and $k\geq 1$. The lower bound of $1-\alpha^*$ follows from Bernoulli's inequality (Lemma~\ref{lem:bernoulli}) with $a = -\alpha^*/k$ and $r=k$. The upper bound holds because $(1-\frac{\alpha^*}{k})^k \leq e^{-\alpha^*} \leq 1-\alpha^* + (\alpha^*)^2$, where the last inequality is obtained by truncating the Taylor expansion of $e^{-\alpha^*}$.  


Algorithm $\reduction^{\leftarrow}(B,s)$ returns the special element $s$ with probability $1-q$. Thus, the probability that $\reduction^{\leftarrow}(B,s)$ returns the special element $s$ is within $(\alpha^*)^2$ of the probability mass of $s$ in $\distr$, which is $1-\alpha^*$.

Conditioned on not returning the special element $s$, the reduction $\reduction^{\leftarrow}(B,s)$ will output an element uniformly from $supp(\distr) \setminus \{s\}$ by symmetry. (This is because all coordinates in $[2k+1] \setminus \{s\}$ have bias either 0 or $\frac{\alpha^*}{k}$ under $\corr$.) Specifically, the reduction places mass $q/k$ on all elements in the support of $P$ other than $s$.  Therefore,
\begin{eqnarray*}
d_{TV}(\reduction^{\leftarrow}(B,s),\distr) 
&=& \frac{1}{2} 
\sum_{i \in [2k+1]} |\Pr(\reduction^{\leftarrow}(B,s) = i) - \distr(i)| \\
&=&\frac 1 2  \left[\left| (1-q) - (1-\alpha^*) \right| + k \cdot \left| \frac q  k - \frac{\alpha^*} k \right|\right] \\
&=& |(1-q) - (1-\alpha^*)|\leq (\alpha^*)^2 \, . \hspace*{2cm} \hfill  \qedhere
\end{eqnarray*}
\end{proof} 


Next, we prove that Algorithm~\ref{alg:special_element} returns the special element with high probability.

\begin{lemma}\label{lem:specialel}
Let $\eps \in (0,1]$. Let $\alpha \in (0,1]$ be sufficiently small, and $k \in \mathbb{N}$ be sufficiently large. Let $\alpha^* = 60\alpha$. Fix a distribution $\distr \in \carb$, and let $s$ be its special element. Let $\Datarv$ be a dataset drawn from $\distr^{\otimes n}$, where $n=\frac{C\log k}{\alpha \eps}$ for a sufficiently large constant $C$. Then $\Pr[SE(\Datarv,\eps,k) = s]\geq 1-\frac{\alpha}{2}$, where $SE$ is described in Algorithm~\ref{alg:special_element}.
\end{lemma}

\begin{proof}
The probability $\distr(s)$ of the special element $s$ under the distribution $\distr$ is $1-\alpha^*$. Let $W$ be the number of times $s$ occurs in $\Datarv$. Since $W$ is a sum of $n$ Bernoulli random variables with bias $(1-\alpha^*)$, the variance of $\frac{W}{n}$ is $\frac{\alpha^*(1-\alpha^*)}{n}\leq \frac{\alpha^*}n$. Let $E$ be the event that $\left|\frac{W}{n} - \distr(s)\right| > \sqrt{\alpha}$. By Chebyshev's inequality,
$$\Pr[E] \leq\frac{\text{Var}\big[\frac W n\big]} \alpha \leq\frac{\alpha^*}{n\alpha} \leq \frac{\alpha}{4}$$
for $n=\frac{C\log k}{\alpha \eps}$. Note that for sufficiently small $\alpha$, event $\overline{E}$ implies that $\frac{W}{n} > 0.6$, which implies that in $\Datarv$, the number of occurrences of the special element is at least $0.2n$ more than the number of occurrences of any other element. 

The sensitivity of the score function used by the exponential mechanism in Algorithm~\ref{alg:special_element} is $\frac{1}{n}$. Hence, by Lemma~\ref{lem:expmech} on the accuracy of the exponential mechanism, for any fixed dataset, with probability greater than or equal to $1-e^{-t}$, the exponential mechanism outputs an element that occurs $an$ times, where $a> \max_{j \in [2k+1]} \left(\frac{1}{n}\sum_{i=1}^n \indicator[\datafixed_i = j]\right) - \frac{2}{n \eps}(\log(2k+1) + t)$. Using $n=\frac{C\log k}{\alpha \eps}$, and setting $t = \log \frac{4}{\alpha}$, we get that $\frac{2}{n \eps}(\log(2k+1) + t) \leq \alpha(2 + \log \frac{4}{\alpha}) < \frac{1}{5}$ and $e^{-t} = \frac{\alpha}{4}$ for sufficiently large $k$ and sufficiently small $\alpha$. Hence, conditioned on event $\overline{E}$, since the number of occurrences of the special element is at least $0.2n$ more than the number of occurrences of any other element, and the guarantee of the exponential mechanism gives that with high probability, the element output occurs $an$ times where $an > \max_{j \in [2k+1]} \left(\frac{1}{n}\sum_{i=1}^n \indicator[\datafixed_i = S]\right) - 0.2n$, we get that the special element is output with probability at least $1 - \frac{\alpha}{4}$. Using the fact that $\Pr(\overline{E}) > 1 - \frac{\alpha}{4}$, we get that with probability at least $1 - \frac{\alpha}{2}$, $SE(\Datarv,\eps,k)=s$.
\end{proof}

\begin{lemma} \label{lem:privred}
The sampler defined in Algorithm~\ref{alg:redsampler} is $(\eps, \delta)$-differentially private. 
\end{lemma}

\begin{proof}
Fix $k,n\in\mathbb{N}$ and  $\eps, \delta > 0$. Fix two neighboring datasets $\Datafixed, \Datafixed' \in [2k+1]^n$. Let $i^* \in [n]$ be the index on which they differ. Let $\datafixed_{i^*}=u$ and $\datafixed'_{i^*}=\ell$. By the privacy of the exponential mechanism, the subroutine call to the Special element picker (Algorithm~\ref{alg:special_element}) in Step~\ref{step:specialel} of Algorithm~\ref{alg:redsampler} is $\eps/2$-differentially private. The output datasets obtained in the data transformation step (Step~\ref{step:datasettrans} of Algorithm~\ref{alg:redsampler}) when run on $\Datafixed$ and $\Datafixed'$ can differ only in the $u^{th}$ and $\ell^{th}$ columns. Without loss of generality, consider the $u^{th}$ column, and let $\hist(u)$ and $\hist(u)+1$ be the number of times $u$ occurs in $\Datafixed$ and $\Datafixed'$. Then, in Step~\ref{step:mult} of Algorithm~\ref{alg:datatrans} run on datasets $\Datafixed$ and $\Datafixed'$, the random variables $(A_1,\dots,A_{n/2})$ are sampled from $\Mult(\hist(u),(1/n,\dots,1/n))$ and $\Mult(\hist(u)+1,(1/n,\dots,1/n)) = \Mult(\hist(u),(1/n,\dots,1/n)) + \Mult(1,(1/n,\dots,1/n))$, respectively, before being post-processed in the same way in Step~\ref{step:flip} of Algorithm~\ref{alg:datatrans}. Thus, for fixed coins of the algorithm, the $u^{th}$ columns of the datasets produced in Step~\ref{step:datasettrans} of Algorithm~\ref{alg:redsampler} when run on $\Datafixed$ and $\Datafixed'$ respectively, differ by at most $1$ entry. Similarly, the $\ell^{th}$ columns obtained in these two runs also differ in at most a single entry. Hence, for fixed coins of the algorithm, the datasets $\Datafixedy$ obtained in these two runs differ in at most $2$ elements. %\ssnote{Is the above discussion precise enough? It isn't technically completely specified since the output datasets are random variables so `neighbouring' is technically ill-defined. What it actually means is that you can use the law of total probability to condition on the first $h(j)$ trials of the multinomial and then reason about each term (using the idea that the fixed columns are neighbouring). But this is how we wrote it in other sections so I figured it was fine.} 
By group privacy and the law of total probability, the subroutine call to sampler $\sampler_{prod}$ in Step~\ref{step:privateprodsamp} of Algorithm~\ref{alg:redsampler} is then $(\eps/2, \delta)$-differentially private. The last step is just post-processing. Thus, using basic composition, the sampler described in Algorithm~\ref{alg:redsampler} is $(\eps, \delta)$-differentially private. 
\end{proof}

Armed with the above lemmas, we prove the main theorem. The high-level structure of the proof is as follows. We reduce the problem of privately sampling from the class $\carb$ given a dataset with size distributed as $\Po(n)$ to privately sampling from the class $\cB^{\otimes 2k}$ given a dataset of size $\frac{n}{2}$ (the reduction is formally described in Algorithm~\ref{alg:redsampler}). Let $\distr$ be the unknown distribution from $\carb$. %We use Lemma~\ref{lem:datatrans} to argue that the dataset transformation step applied to a dataset consisting of a Poisson number of independent samples from $\distr$ produces a dataset of roughly the same size drawn from the corresponding product distribution $\corr$. 
We assume that we have a private product distribution sampler that is accurate given $n/2$ samples. We then use Lemmas~\ref{lem:datatrans} and~\ref{lem:easyunivtrans} to argue that conditioned on the special element being chosen correctly in Algorithm~\ref{alg:redsampler}, the output distribution of Algorithm~\ref{alg:redsampler} is close in total variation distance to $\distr$. We then use Lemma~\ref{lem:specialel} to argue that the special element is chosen in Algorithm~\ref{alg:redsampler} with high probability. This implies that Algorithm~\ref{alg:redsampler} is an accurate sampler for $\carb$. By Lemma~\ref{lem:privred}, it is $(\eps, \delta)$-differentially private. Hence, we can invoke the lower bound for privately sampling from $\carb$ to obtain a lower bound on the number of samples for accurately sampling from $\cB^{\otimes 2k}$.
\begin{proof}[Proof of Theorem~\ref{thm:product-bern-lb}] 

 For any distribution $\distr$ and randomized function $f$, we will use $f(\distr)$ to represent the distribution of the random variable obtained by applying $f$ to a random variable distributed according to $\distr$. Also, without loss of generality, we can assume that $d$ is greater than some fixed constant. This is because for $d$ smaller than that constant, the lower bound for privately sampling from Bernoulli distributions in Theorem~\ref{thm:bernoulli-lb} directly gives the lower bound.
 
 Fix $n > \frac{2C\log k}{\alpha \eps}$. Set $k=\frac{d}{2}$. Fix any distribution $\distr \in \carb$, and let its special element be $s$. Let $C$ be the constant in the statement of Lemma~\ref{lem:specialel}. Let $\Datarv$ be a dataset of size $\Po(n)$ with entries drawn independently from $\distr$. 
 Fix an arbitrary $(\eps/2, \delta)$-DP sampler $\sampler_{prod}$ for $\cB^{\otimes d}$ that is $\frac{\alpha}{25}$-accurate when given $\frac{1}{2}[n - \frac{2C \log k}{\alpha \eps}]$ samples. Let $\corr$ be the corresponding product distribution of $\distr$. Run $\sampler_{red}$ with the following inputs: dataset $\Datarv$, privacy parameters $\eps,\delta$, the values $k,n$, accuracy parameter $\alpha$, and black box access to $\sampler_{prod}$. Let event $E_{s}$ represent  successfully choosing the special element $s$ in Algorithm~\ref{alg:redsampler} (this event always occurs when $L$ satisfies the condition in Step~\ref{step:largespec} and the special element is chosen correctly in  Step~\ref{step:specialel}). Let the output distribution of Algorithm~\ref{alg:redsampler} conditional on $E_{s}$ be $Q_{E_s}$. 

By Poissonization (Lemma~\ref{lem:multtopois}), $L$ and $R$ sampled in Step~\ref{step:datasplitval} of Algorithm~\ref{alg:redsampler} are independent Poisson random variables with means $\frac{2C \log k}{\alpha \eps}$ and $n - \frac{2C \log k}{\alpha \eps}$, respectively.

For the next part of the proof, condition on event $E_s$ occurring. We show that conditioned on this event, the output distribution of Algorithm~\ref{alg:redsampler} is close in total variation distance to distribution $\distr$.

Observe that the conditioning on $E_s$ does not change the distribution of $R$, since only the first partition consisting of $L$ samples is used to determine the special element (and $L$ is independent of $R$). By Lemma~\ref{lem:datatrans}, Step~\ref{step:datasettrans} then produces a dataset of size $\frac{1}{2}[n-\frac{2C \log k}{\alpha \eps}]$ that is distributed as independent samples from $\corr$. Next, let $Q_{prod}$ be the output distribution of Step~\ref{step:privateprodsamp}. By the accuracy of the private product sampler, $d_{TV}(Q_{prod}, \corr) \leq \alpha/25$. By the information processing inequality for total variation distance (Lemma~\ref{lem:postTV}), $$d_{TV}(\reduction^{\leftarrow}(Q_{prod}, s), \reduction^{\leftarrow}(\corr,s)) \leq \alpha/25.$$
By Lemma~\ref{lem:easyunivtrans}, $d_{TV}(\reduction^{\leftarrow}(\corr,s), \distr) \leq (\alpha^*)^2$, where $\alpha^* = 60\alpha$. Additionally,  $\reduction^{\leftarrow}(Q_{prod},s) = Q_{E_s}$ is the output distribution of Algorithm~\ref{alg:redsampler} conditional on event $E_s$. Hence, for sufficiently small $\alpha$, by the triangle inequality, 
\begin{align} \label{eq:totalvarcond}
d_{TV}(Q_{E_s},\distr) \leq \alpha/25 + (\alpha^*)^2 \leq \alpha/13.
\end{align}
Next, we show that the probability of choosing the special element incorrectly (event $\overline{E_s}$) in Algorithm~\ref{alg:redsampler} is small. By a union bound, this probability is at most the sum of the probability that the random variable $L$ was not large enough in Step~\ref{step:largespec}, and the probability that Step~\ref{step:specialel} chooses the wrong special element. By a tail bound on Poisson random variables (Claim~\ref{lem:poiss_tail} \cite{clementpoiss}), the probability that $L<\frac{C\log k}{\alpha \eps}$ is at most $e^{-\frac{C \log k}{6\alpha \eps}} \leq e^{-\frac{C \log(4/\alpha)}{6}} \leq  \frac{\alpha}{4}$ for sufficiently small $\alpha$ (upper bounding $\eps$ by $1$, lower bounding $\log k$ by $1$, and lower bounding $\frac{1}{\alpha}$ by $\log(4/\alpha)$). By Lemma~\ref{lem:specialel}, the probability of choosing the wrong special element in Step~\ref{step:specialel} is at most $\frac{\alpha}{2}$. Hence, the probability of $\overline{E_s}$ is at most $\alpha/4 + \alpha/2 = 3\alpha/4$.

Let $\distroutput{\sampler_{red}, \distr}$ be the output distribution of Algorithm~\ref{alg:redsampler}. Next, we show that $\distroutput{\sampler_{red}, \distr}$
is close to distribution~$\distr$. 

We first invoke a standard lemma to argue that the output distribution of Algorithm~\ref{alg:redsampler} (i.e. distribution $\distroutput{\sampler_{red}, \distr}$)
is close to the output distribution conditioned on event $E_s$ (i.e. distribution $Q_{E_s}$).  
Accounting for choosing the special element incorrectly, we can use Lemma~\ref{lem:TVcond} with $\beta = 3\alpha/4$, event $E_s$ and distribution $D$ being the output distribution $\distroutput{\sampler_{red}, \distr}$ to argue that 
\begin{align}\label{eq:condtvspecial}
    d_{TV}(\distroutput{\sampler_{red}, \distr}, Q_{E_s}) \leq \frac{3\alpha/4}{1-3\alpha/4} \leq \frac{3\alpha/4}{1-3/16} \leq  \frac{12\alpha}{13},
\end{align}
where we have used $\alpha \leq 0.25$. Finally, by the triangle inequality, using (\ref{eq:totalvarcond}) and (\ref{eq:condtvspecial}), we get that $$d_{TV}(\distroutput{\sampler_{red}, \distr}, \distr) \leq d_{TV}(\distroutput{\sampler_{red}, \distr}, Q_{E_s}) + d_{TV}(Q_{E_s}, \distr) \leq \frac{12\alpha}{13} + \frac{\alpha}{13} = \alpha.$$
%\begin{align*}
% d_{TV}(\distroutput{\sampler_{red}, \distr}, \distr) 
%& =\frac 1 2 \sum_{i \in \universe}|\Pr[\sampler_{red}(\Datarv)=i] - \distr[i]| \\
%& =\frac 1 2   \sum_{i\in \universe} \left|\Pr[\sampler_{red}(\Datarv)=i \land \overline{E}] + \Pr[\sampler_{red}(\Datarv)=i \land  E] - \distr[i] \left(\Pr[\overline{E}]+\Pr[E]\right) \right| \\
%& \leq \frac 12 \sum_{i \in \universe}\left(\left|\Pr[\sampler_{red}(\Datarv)=i \land  \overline{E}] - \distr[i]\cdot\Pr[\overline{E}]\right| +  \Pr[\sampler_{red}(\Datarv)=i \land E]+ \Pr[E]\distr[i]\right)  \\
%& = \frac 12 \cdot(\Pr[E]+\Pr[E])+ \frac 12 \sum_{i \in \universe}\left|\Pr[\sampler_{red}(\Datarv)=i \mid \overline{E}]\cdot\Pr[\overline{E}] - \distr[i]\cdot\Pr[\overline{E}]\right|  \\
%\Pr_{N', \sampler, \Datarv}(\sampler(\Datarv)=o, E) \\
%& = \Pr[E] + \frac 12 \sum_{i \in \universe}\Pr[\overline{E}]\cdot\left|\Pr[\sampler_{red}(\Datarv)=i \mid \overline{E}] - \distr[i]\right| \\
%&  \leq  \Pr[E] + \frac 12 \sum_{i \in \universe}\left|\Pr[\sampler_{red}(\Datarv)=i \mid \overline{E}] - \distr[i]\right| \\
%& \leq \frac 12 \sum_{o \in \universe}\Big|\Pr_{\posampler, \Datarv}(\sampler(\Datarv)=o \mid \overline{E}) - \distr(o)\Big| +   {e^{-n/6}} \\
%& \leq  \frac{3\alpha}{4} + \frac{\alpha}{4} = \alpha.
%\end{align*} 
%where the first inequality is by the triangle inequality, the third equality is by marginalizing over $i$, and the final inequality is using Lemma~\ref{lem:specialel} and (\ref{eq:totalvarcond}).
Hence, Algorithm~\ref{alg:redsampler} is $\alpha$-accurate for $\carb$. By Lemma~\ref{lem:privred}, Algorithm~\ref{alg:redsampler} is $(\eps, \delta)$-differentially private. Putting both together, Algorithm~\ref{alg:redsampler} is an $(\eps, \delta)$-differentially private, $\alpha$-accurate sampler for $\carb$ with sample size distributed as $\Po(n)$. Since $d$ (and hence $k$) is sufficiently large, the lower bound for Poisson samplers for $\carb$ (Lemma~\ref{lem:k-ary-lb-pois}) then implies that $n \geq \frac{c k}{\alpha \eps}$ for some constant $c$. For some other constant $c'$, this implies that $\frac{1}{2}[n-\frac{2C \log k}{\alpha \eps}] > \frac{c' 2k}{\alpha \eps} = \frac{c'd}{\alpha \eps}$. Hence, for sufficiently large $d$, and $\eps \leq \frac{1}{2}$, every $(\eps, \delta)$-DP sampler that is $\frac{\alpha}{25}$-accurate on the class $\cB^{\otimes d}$ needs a dataset of size $\Omega(\frac{d}{\alpha \eps})$. %\sstext{Is a line here necessary saying that the assumption $n > \frac{2C\log k}{\alpha \eps}$ is not a problem because if there is no alg for larger $n$, there is no alg for smaller $n$?} 
%This proves the theorem for small $\eps$.
%
%To prove the theorem for $\eps \in (1/\log \frac{4}{\alpha},1]$, we  use privacy amplification by subsampling (without replacement). Fix $\eps$ in this range. Assume there exists an $(\eps,\delta)$-DP sampler $\sampler$ that is $\alpha$-accurate on \carb for datasets of size $n < \frac{c' d}{2\alpha \eps}$. Fix $\eps' = \frac{1}{2\log \frac{4}{\alpha}}$. Then, for $m > n$, by using subsampling without replacement (see Theorem~\ref{thm:BalleGG2018}) with $\frac{n}{m} = \frac{1}{2\eps \log \frac{4}{\alpha}},$ we can argue that there exists a $(2\eps',\delta\frac{n}{m})$-DP sampler $\sampler$ that is $\alpha$-accurate on $\carb$ for datasets of size $ m < \frac{c' d}{2 \alpha \eps'}$. However, this contradicts the lower bound for small $\eps$ and hence proves the theorem for large $\eps$.
\end{proof}
%%%%%%%%%%%%
%%%%%%%%%%%%BEFORE THIS THE PRODUCT LOWER BOUND
%%%%%%%%%%%%

\ifnum\neurips=1
\section{Products of Bernoulli distributions with bounded bias}\label{sec:bounded-bias}
\else
\section{Products of Bernoulli Distributions with Bounded Bias}\label{sec:bounded-bias}
\fi



\subsection{Sampling Algorithms for Products of Bernoullis with Bounded Bias}

In this section, we consider Bernoulli distributions and, more generally, products of Bernoulli distributions with bounded bias. We show that, when the bias is bounded, differentially private sampling can be performed with datasets of significantly smaller size than in the general case. For Bernoulli distributions with bounded bias, we achieve this (in Theorem~\ref{thm:bernoulli-bb}) with pure differential privacy, that is, with $\delta=0$. For products of Bernoulli distributions, we give a zCDP algorithm (see Theorem~\ref{thm:bernoulli-product-bb}).
Theorems~\ref{thm:bernoulli-bb} and~\ref{thm:bernoulli-product-bb}, in conjunction with Lemma~\ref{prelim:relate_dp_cdp} relating $\rho$-zCDP and differential privacy, directly yield Theorem~\ref{thm:intro-bernoulli-product-bb}. In Section~\ref{sec:bb-lb}, we prove our lower bound for products of Bernoulli distributions with bounded bias, encapsulated in Theorem~\ref{thm:intro-product-bb-lb}. 
%%%%%%%%%%%%%%%%%%%%%%%%%%%%%%%%%%%%%%%%%%%
\ifnum\neurips=1
\subsection{Private sampler for Bernoulli distributions with bounded bias}\label{sec:bernoulli-bb}
\else 
\subsubsection{Private Sampler for Bernoulli Distributions with Bounded Bias}\label{sec:bernoulli-bb}
\fi


%%%%%%%% Old proof with $\times\ln \frac 1 \alpha$ in sample complexity
%In this section, we consider the class $\cBB$ of Bernoulli distributions (see Definition~\ref{prelim:bern_def}) with an unknown bias $p\in\big[\frac 13,\frac 23\big].$ 
%Even though class $\cBB$ is the hardest to learn privately among the classes of Bernoulli distributions, we show in the next theorem that private sampling from this class is easy.
%\begin{theorem}\label{thm:bernoulli-bb}
%For all $\eps>0$ and $\alpha\in (0,1)$, there exists an $(\eps,0)$-differentially private sampler for the class $\cBB$ of Bernoulli distributions with bias in $\big[\frac 13,\frac 23 \big]$ that is  $\alpha$-accurate with the input sample of size $n=O(\frac 1{\eps}\cdot\ln \frac 1 {\alpha})$.
%\end{theorem}
%\begin{proof}
%Recall that $\clip{\ \cdot \ }$ denotes rounding an arbitrary real number to the nearest value in $[0,1]$.
%We use the following variant of Algorithm~\ref{alg:kary} for the special case of $k=2$: compute the sample proportion $\hat p=\frac {1}n \sum_{i\in[n]} \datafixed_i$, obtain a noisy proportion $\tilde p = \clip{\hat p + Z}$, where $Z\sim\Lap(\frac{1}{\eps n})$, and output $b\sim\Ber(\tilde p)$. 
%
%We show that this algorithm is $\alpha$-accurate on the class $\cBB$ with $n\geq \max\{18,\frac {6}{\eps}\}\cdot \ln \frac {12}{\alpha}$ samples. Let $E$ be the ``good'' event that $|\hat p - p|\leq\frac 1 6$ and $|Z|\leq\frac 16$. If $E$ occurs then $\tilde p = \hat p+ Z$, that is, no rounding is needed. Since $p\in[1/3,2/3]$, we get
%\begin{align}\label{eq:prob-of-E-bar}
%\Pr[\overline{E}] %=\Pr[\hat p+ Z \notin [0,1]]
%= \Pr\Big[|\hat p - p|\leq\frac 1 6 \bigwedge |Z|\leq\frac 16\Big]
%\leq 2 e^{-n/18} + 2^{-\eps n/6}
%\leq \frac\alpha 3,
%\end{align}
%where we applied the Hoeffding inequality for $\hat p$ (specifically, that $\Pr[|\hat p -\E[\hat p]|\geq t]\leq 2 e^{-2nt^2}$) and a tail bound for the Laplace distribution for $Z$ (specifically, that $\Pr[|\Lap(b)|\geq t\cdot b]\leq e^{-t}$) and, at the end, used our lower bound on $n$.
%
%Let $Q$ be the distribution of the output bit $b$ for a dataset selected i.i.d.\ from $\Ber(p).$ Then, by Claim~\ref{clm:bern_acc_eq} and the description of the algorithm,
%$$
%d_{TV}(Q, \Ber(p)) =|\E(b)-p|
%=|\E(\tilde p)-\E(\hat p+Z)|.
%$$
%Next, we observe that $\E[\tilde p | E] = \E[\hat p+ Z | E]$ and use (\ref{eq:prob-of-E-bar}) to bound $d_{TV}(Q, \Ber(p)).$ Specifically,
%\begin{align*}
%\E[\hat p+Z] &=\E[\hat p+Z | E]\cdot\Pr[E]+\E[\hat p+Z | \overline{E}]\cdot\Pr[\overline{E}]
%\leq \E[\hat p+Z | E]\cdot 1+(\E[\hat p | \overline{E}]+\E[Z | \overline{E}]) \cdot\Pr[\overline{E}]\\
%&\leq \E[\hat p+Z | E] +\frac \alpha 3 
%= \E[\tilde p | E] +\frac \alpha 3
%\leq \frac{\E[\tilde p]}{\Pr[E]}+\frac \alpha 3
%\leq \frac{\E[\tilde p]}{1-\alpha/3}+\frac \alpha 3\\
%&\leq \E[\tilde p] + \big(\frac 1 {1-\alpha/3} -1\big) +\frac \alpha 3
%\leq \E[\tilde p] + \frac \alpha 2 +\frac \alpha 3
%\leq \E[\tilde p] +\alpha,
%\end{align*}
%where in the first inequality we used the fact that $\Pr[E]\leq 1$ and linearity of conditional expectation; to obtain the second inequality, in addition to (\ref{eq:prob-of-E-bar}), we used that $\hat p\leq 1$ and that the noise $Z$ is symmetric, so that $\E[\hat p | \overline{E}]\leq 1$ and $\E[Z|\overline{E}]=0$.
%
%Similarly, $\E[\tilde p]\leq \E[\hat p +Z]+\alpha.$
%We get that $d_{TV}(Q, \Ber(p))=|\E(\tilde p)-\E(\hat p+Z)|\leq \alpha,$ completing the accuracy analysis.
%
%
%The specified algorithm is $(\eps,0)$-differentially private, since the sensitivity of $\hat p = 1/n$, and thus the algorithm uses Laplace mechanism with postprocessing.  This completes the proof of Theorem~\ref{thm:bernoulli-bb}.
%\end{proof}



%%%%%%%%%%%%%%%%%%%%%%
%%%% New proof with additive $\ln \frac 1 \alpha$ in sample complexity
First, we consider the class $\cBB$ of Bernoulli distributions (see Definition~\ref{prelim:bern_def}) with an unknown bias $p\in\big[\frac 13,\frac 23\big].$ 
Even though class $\cBB$ is the hardest to learn privately among the classes of Bernoulli distributions, we show in the next theorem that private sampling from this class is easy.
\begin{theorem}\label{thm:bernoulli-bb}
For all $\eps>0$ and $\alpha\in (0,1)$, there exists an $(\eps,0)$-differentially private sampler for the class $\cBB$ of Bernoulli distributions with bias in $\big[\frac 13,\frac 23 \big]$ that is  $\alpha$-accurate for datasets of size  
$n=O(\frac 1{\eps}+\ln \frac 1 {\alpha})$.
\end{theorem}
\begin{proof}
We use $\clipab{\ \cdot \ }$ to denote rounding an arbitrary real number to the nearest value in $[a,b]$.
Consider the following sampler $\sampler_{clip}$: on input $\Datafixed\in \{0,1\}^n,$
compute the  sample proportion $\hat p=\frac {1}n \sum_{i\in[n]} \datafixed_i$, obtain a clipped proportion $\tilde p = \clipquarter{\hat p}$, and output $b\sim\Ber(\tilde p)$. 

\begin{claim}\label{claim:bernoulli-bb-accuracy}
Sampler $\sampler_{clip}$ is $\alpha$-accurate on the class $\cBB$ with dataset of size $n\geq 72\ln\frac {6}{\alpha}$. 
\end{claim}
\begin{proof}
Let the ``good'' event $E$ be that no rounding occurs when sample proportion is clipped, that is, $\tilde p=\hat p$.
Since $p\in[1/3,2/3]$,
\begin{align}\label{eq:prob-of-E-bar}
\Pr[\overline{E}]=\Pr\Big[\hat p\notin [1/4,3/4]\Big]
\leq\Pr\Big[|\hat{p}-p|\geq\frac 1{12}\Big]\leq 2e^{-n/72}\leq\frac \alpha{3},
\end{align}
where we applied the Hoeffding bound (specifically, that $\Pr[|\hat p -\E[\hat p]|\geq t]\leq 2 e^{-2nt^2}$) and our lower bound on $n.$ 

Let $Q$ be the distribution of the output bit $b$ for a dataset selected i.i.d.\ from $\Ber(p).$ Then, by Claim~\ref{clm:bern_acc_eq}  and the description of $\sampler_{clip}$,
$$
d_{TV}(Q, \Ber(p)) =|\E(b)-p|
=|\E(\tilde p)-\E(\hat p)|.
$$
Next, we observe that $\E[\tilde p | E] = \E[\hat p | E]$ and use (\ref{eq:prob-of-E-bar}) to bound $d_{TV}(Q, \Ber(p)).$ Specifically,
\begin{align*}
\E[\hat p] &=\E[\hat p | E]\cdot\Pr[E]+\E[\hat p | \overline{E}]\cdot\Pr[\overline{E}]
\leq \E[\hat p | E]\cdot 1+1\cdot\Pr[\overline{E}]\\
&\leq \E[\hat p | E] +\frac \alpha 3 
= \E[\tilde p | E] +\frac \alpha 3
\leq \frac{\E[\tilde p]}{\Pr[E]}+\frac \alpha 3
\leq \frac{\E[\tilde p]}{1-\alpha/3}+\frac \alpha 3\\
&\leq \E[\tilde p] + \Big(\frac 1 {1-\alpha/3} -1\Big) +\frac \alpha 3
\leq \E[\tilde p] + \frac \alpha 2 +\frac \alpha 3
\leq \E[\tilde p] +\alpha.
\end{align*}
Similarly, $\E[\tilde p]\leq \E[\hat p]+\alpha.$
We get that $d_{TV}(Q, \Ber(p))=|\E(\tilde p)-\E(\hat p)|\leq \alpha,$ completing the accuracy analysis.
\end{proof}

\begin{claim}\label{claim:bernoulli-bb-privacy}
Sampler $\sampler_{clip}$ is $(4/n,0)$-differentially private. 
\end{claim}
\begin{proof}
By definition of the sampler, $\Pr[b=1]=\tilde p.$ Consider two datasets $\Datafixed$ and $\Datafixed'$ that differ in one record. The sample proportions
 $\hat p=\frac {1}n \sum_{i\in[n]} \datafixed_{i}$ and $\hat p'=\frac {1}n \sum_{i\in[n]} \datafixed'_{i}$ differ by at most $1/n$. Let $\tilde p$ and $\tilde p'$ be the corresponding clipped proportions, which also differ by at most $1/n$. Then, since $\tilde p\geq 1/4,$
 $$\tilde p'\leq \tilde p+\frac 1 n\leq \tilde p + \tilde p \frac 4 n
 =\tilde p\Big(1+\frac 4 n\Big)
 \leq \tilde p\cdot  e^{4/n},
 $$
where we used the fact that $1+t\leq e^t$ for all $t.$
Similarly, since $\tilde p\leq 3/4,$ the probabilities of returning $b=0$ for inputs $\Datafixed$ and $\Datafixed'$ also differ by at most a factor of $e^{4/n}$. Thus, $\sampler_{clip}$ is $4/n$-differentially private.
\end{proof}
Now we set $n\geq\max\big\{72\ln\frac 6\alpha,\frac 4 \eps\big\}$ and use Claims~\ref{claim:bernoulli-bb-accuracy} and~\ref{claim:bernoulli-bb-privacy} to get both accuracy and privacy guarantees. Observe that when $n\geq 4/\eps$, we get that $4/n\leq\eps$, that is, $\sampler_{clip}$ is $(\eps,0)$-differentially private.
This completes the proof of Theorem~\ref{thm:bernoulli-bb}.
\end{proof}




%%%%%%%%%%%%%%%%%%%%%%%%%%%%%%%%%%%%%

\ifnum\neurips=1
\subsubsection{Private sampler for products of Bernoulli distributions with bounded bias}\label{sec:bernoulli-prod-bb}
\else 
\subsubsection{Private Sampler for Products of Bernoulli Distributions with Bounded Bias}\label{sec:bernoulli-prod-bb}
\fi

In this section, we consider product distributions, where each marginal is a Bernouli distribution with a bias between 1/3 and 2/3. For this class, significantly fewer samples are needed for private sampling than for the general class of products of Bernoulli distributions.

\begin{theorem}\label{thm:bernoulli-product-bb}
For all all $\rho >0$ and $\alpha\in (0,1)$, there exists a $\rho$-zCDP sampler for the class $\cBB^{\otimes d}$ of products of Bernoulli distributions with bias in $\big[\frac 13,\frac 23 \big]$ that is  $\alpha$-accurate for datasets of size $n=O\Big(\frac {\sqrt{d}}{\sqrt{\rho}} + \log \frac d {\alpha}\Big)$.
\end{theorem}



%In particular, the algorithm is $(\eps,\delta)$-DP when $n\geq c\frac{\sqrt{d \log(1/\delta)}}{\eps}$.

\begin{proof}
The input to a sampler for $\cBB^{\otimes d}$ is an $n\times d$ matrix $\Datafixed\in \{0,1\}^{n\times d},$ 
where row $i$ contains the $i$-th record and column $j$ contains all input bits for $j$-th attribute. Recall sampler $\sampler_{clip}$ from the proof of Theorem~\ref{thm:bernoulli-bb}.
For each $j\in[d]$, our sampler runs sampler $\sampler_{clip}$ on column $j$ of $\Datafixed$ and records its output bit $b_j$; it returns the vector $b=(b_1,\dots,b_d)$.
%with the row $\Datafixed_{i}$ denoting the $i$-th record and the bit $\Datafixed_{ij}$ denoting the $j$-th coordinate of the $i$-th record.
%As before, we use $\clipab{\ \cdot \ }$ to denote rounding an arbitrary real number to the nearest value in $[a,b]$.
%Consider the following sampler $\sampler_{clip}$:
%for each coordinate $j\in[d]$,
%compute the sample proportion $\hat p_j=\frac {1}n \sum_{i\in[n]} \Datafixed_{ij}$, obtain a clipped proportion $\tilde p_j = \clipquarter{\hat p_j}$, and sample $b_j\sim\Ber(\tilde p_j)$; return the vector $b=(b_1,\dots,b_n)$ of sampled bits.

%%Algorithm: For each $j$: compute $\tilde p_j = [\hat p_j]_{1/4}^{3/4}$. Flip a coin w.p. $\tilde p_j$. 

%%Accuracy: Probability of any clipping is at most $d\exp(-\Omega(n))$. Absent clipping, get exactly distribution. So need $n \geq \ln(d/\alpha)$.

We show that this sampler is $\alpha$-accurate on the class $\cBB^{\otimes d}$ with $n\geq 72 \ln \frac{6d}{\alpha}$ samples. 
Let $P=P_1\otimes\dots\otimes P_d$ be the input product distribution, where $P_j= \Ber(p_j)$ for all coordinates $j\in[d]$.
Let $Q=Q_1\otimes\dots\otimes Q_d$ be the distribution of the output vector $b$ for a dataset selected i.i.d.\ from $P.$  (Since the coordinates of $b$ are mutually independent, $Q$ is indeed a product distribution.) By Claim~\ref{claim:bernoulli-bb-accuracy} applied with $\alpha/d$ as accuracy parameter, $d_{TV} (Q_j,P_j)\leq \frac\alpha d$. By subadditivity of the statistical distance between two product distributions, 
$$d_{TV}(Q,P)\leq\sum_{j\in[d]} d_{TV} (Q_j,P_j)
\leq d\cdot\frac\alpha d =\alpha,$$
%
%
%Fix $j\in[d]$. To upper bound $d_{TV} (Q_j,P_j),$ define the ``good'' event $E_j$ that no rounding occurs when the $j$-th sample proportion is clipped, that is, $\tilde p_j=\hat p_j$.
%Since $p_j\in[1/3,2/3]$, 
%$$
%\Pr[\overline{E}_j]=\Pr\Big[\hat p_j\notin [1/4,3/4]\Big]
%\leq\Pr\Big[|\hat{p}_j-p_j|\geq\frac 1{12}\Big]\leq 2e^{-n/72}\leq\frac \alpha{3d},
%$$
%where we applied the Hoeffding bound and our lower bound on $n.$ 
%Then, by Claim~\ref{clm:bern_acc_eq}  and the description of $\sampler_{clip}$,
%$$
%d_{TV}(Q_j, \Ber(p_j)) =|\E(b_j)-p_j|
%=|\E(\tilde p_j)-\E(\hat p_j)|.
%$$
%We proceed as in the proof of Theorem~\ref{thm:bernoulli-bb} and get that 
%$$\E[\hat p_j]\leq \E[\hat p_j |E_j]+\frac{\alpha}{3d}
%=\E[\tilde p_j |E_j]+\frac{\alpha}{3d}
%\leq \E[\tilde p_j] + \big(\frac 1 {1-\alpha/(3d)} -1\big) +\frac \alpha {3d}
%\leq \E[\tilde p_j] +\frac \alpha d
%$$
%and, similarly, $ \E[\tilde p]\leq \E[\hat p_j]+ \alpha/ d.$ Thus, $d_{TV}(Q_j, \Ber(p_j))
%=|\E(\tilde p_j)-\E(\hat p_j)|\leq \alpha/d$ and, consequently, 
%$d_{TV}(Q, P)\leq \sum_{j\in[d]} d_{TV} (Q_j,P_j)\leq d\cdot\frac\alpha d =\alpha,$ completing the proof that $\sampler_{clip}$ is $\alpha$-accurate.
%
completing the proof that our sampler is $\alpha$-accurate.


%When the dataset is drawn i.i.d.\ from $P$, the $d$-dimensional vector $\hat p$ is also distributed according to $P.$ By construction of the output $b$, it has the same distribution as the vector $\tilde p$ of clipped proportions. 
%The key observation is that, when $E$ occurs, the vectors $\hat p$ and $\tilde p$ are the same. Consequently,
%$P_{|E}=Q_{|E}.$ 
%To bound $d_{TV}(P,Q),$ we apply the following claim from ~\cite[Claim4]{RasS06}.
%
%\begin{claim}[Claim 4 of \cite{RasS06}]\label{claim:RS06}
%Let $E$ be  an  event  that  happens  with  probability  at  least $1-\beta$ under  the  distribution $\cal D$ and let
%$\cal D '$ denote the conditional distribution ${\cal D}_{|E}$.  Then $d_{TV}({\cal D}, {\cal D}')\leq  \frac \beta {1-\beta}$.
%\end{claim}

%By Claim~\ref{claim:RS06}, our bound on the probability of $\bar E,$ and the triangle inequality, we get
%$$
%d_{TV}(P,Q)\leq d_{TV}(P,P_{|E}) + d_{TV}(P_{|E},Q_{|E}) +d_{TV}(Q_{|E},Q). 
%$$

Finally, we show that our sampler %$\sampler_{clip}$ 
is $\rho$-zCDP when $n\geq \sqrt{8d/\rho}$. The sampler is a composition of $d$ algorithms, each returning one bit. 
%We will prove that 
By Claim~\ref{claim:bernoulli-bb-privacy},
these algorithms are $(4/n,0)$-differentially private and, consequently, also $\frac 8{n^2}$-zCDP. A composition of $d$ such algorithms is then $\frac {8d}{n^2}$-zCDP.
%
%Fix $j\in [d].$ By definition of the sampler, $\Pr[b_j=1]=\tilde p_j.$ Consider two datasets $\Datafixed$ and $\Datafixed'$ that differ in one row. The sample proportions
% $\hat p_j=\frac {1}n \sum_{i\in[n]} \Datafixed_{ij}$ and $\hat p'_j=\frac {1}n \sum_{i\in[n]} \Datafixed'_{ij}$ differ by at most $1/n$. Let $\tilde p_j$ and $\tilde p'_j$ be the corresponding clipped proportions, which also differ by at most $1/n$. Then, since $\tilde p_j\geq 1/4,$
% $$\tilde p'_j\leq \tilde p_j+\frac 1 n\leq \tilde p_j + \tilde p_j \frac 4 n
% =\tilde p_j\Big(1+\frac 4 n\Big)
% = \tilde p_j\cdot  e^{4/n},
% $$
%where we used the fact that $1+t\leq e^t$ for all $t.$
%Similarly, since $\tilde p_j\leq 3/4,$ we get that the probabilities of returning $b_j=0$ for inputs $\Datafixed$ and $\Datafixed'$ also differ by at most a factor of $e^{4/n}$. Thus, the algorithm that outputs~$b_j$ is $4/n$-differentially private. Consequently, it is also $\frac 8{n^2}$-zCDP. Sampler $\sampler_{clip}$, which is a composition of $d$ such algorithms, is then $\frac {8d}{n^2}$-zCDP.
That is, when $n\geq \sqrt{8d/\rho}$, it is $\rho$-zCDP, as required.
%
%Privacy: Composition of $d$ mechanisms; each $(\eps,0)$-DP for $\eps = \ln(1 + 4/n) = O(1/n)$. So each mechanism is $O(\eps^2) = O(\frac{1}{n^2)}$-zCDP. % 
%Thus $\rho$-zCDP for $\rho =O(d\eps^2)= O(d/n^2)$. 
\end{proof}

\subsection{Lower Bound for Products of Bernoullis with Bounded Bias}
\label{sec:bb-lb}

We now prove a lower bound that matches the guarantees of the algorithm of the previous section. 

\begin{theorem}[Theorem~\ref{thm:intro-product-bb-lb}, restated]\label{thm:bb-lb}
For all  sufficiently small $\alpha>0$, and
for all  $d,n \in \N$, $\eps \in (0,1]$,  and $\delta \leq \frac 1{100n}$, if there exists an $(\eps,\delta)$-differentially private sampler that is $\alpha$-accurate on
the class of products of $d$ Bernoulli distributions with biases in $\big[\frac 13,\frac 23 \big]$ on datasets of size $n$, then $n={\Omega}(\sqrt{d}/\eps)$. \asnote{Need to resolve dependency on $\delta$.}
\end{theorem}
To prove the theorem, we reduce the problem of accurately estimating the marginal biases of a product distribution over $\{0,1\}^d$ to the problem of sampling from the product distribution. This involves dividing the dataset into a constant number of disjoint parts and passing each part separately to a sampler for product distributions to obtain a constant number of independent samples. Then, by averaging the samples obtained, we get an estimate of the marginal biases. We also observe that a marginal estimator for the class $\cBB^{\otimes d}$ can be converted into a marginal estimator for the class $\cB^{\otimes d}$ with only a constant factor loss in accuracy. To do this, we flip every bit of every sample with probability $1/3$, which gives us a dataset that looks like it is drawn from a product distribution in $\cBB^{\otimes d}$. We then use the marginal estimator for $\cBB^{\otimes d}$ and transform the estimated biases back to the original range by multiplying by $3$ and subtracting $1$. Finally, applying the lower bound of Bun et al.~\cite{BunUV14j} for the sample complexity of marginal estimation for the class $\cB^{\otimes d}$, we obtain a lower bound on the sample complexity for the problem of accurately sampling from product distributions with bounded biases.

\begin{definition}[Marginal Estimator] For $\alpha', \beta', \gamma \in [0,1]$, and a class \class of distributions on $\bit{d}$, an algorithm \me is an $(\alpha', \beta', \gamma, \class)$-marginal estimator with sample size $n$ if, given $\Datarv \sim \distr^{\otimes n}$ where $\distr \in \class$,  with probability at least $1-\gamma$, algorithm \me returns $\tilde{\biasesfixed}_1, \ldots, \tilde{\biasesfixed}_d$ such that
\begin{equation*}
    |\{j \in [d] : |\biasesfixed_j - \tilde{\biasesfixed}_j| > \alpha' \}| < \beta' d \, .
\end{equation*}
\end{definition}

\begin{algorithm}
    \caption{Marginal Estimator $\me_c$ for $\cBB^{\otimes d}$}
    \label{alg:marginal_est}
    \hspace*{\algorithmicindent} \textbf{Input:} dataset $\Datafixed \in \bit{c n\times d}$, constant $c$, query access to sampler \sampler\\
    \hspace*{\algorithmicindent} \textbf{Output:} marginal estimates $\tilde{\Biasesfixed} = (\tilde{\biasesfixed}_1, \ldots, \tilde{\biasesfixed}_d)$
    \begin{algorithmic}[1] % The number tells where the line numbering should start
            \State Partition dataset \Datafixed into $c$ equal parts: $\Datafixed^{(1)}, \ldots, \Datafixed^{(c)}$
            \For{$i = 1$ to $c$}:
            \State $Y_i \gets \sampler(\Datafixed^{(i)})$ \Comment{Get $c$ independent samples from \sampler}
            \label{step:sample} 
            \EndFor
            \State $\tilde{\Biasesfixed} \gets \frac{1}{c} \sum_{i=1}^c Y_i$ \Comment{Compute marginal estimates} \label{step:computeemp}
            \State \Return $\tilde{\Biasesfixed}$
    \end{algorithmic}
\end{algorithm}


\begin{lemma}[Reduction from Marginal Estimation to Sampling]\label{lem:sam_to_me}
For all $\alpha, \beta_0, \gamma_0 \in (0,1)$, there exists $c\in \N$ such that for all $\eps,\delta>0$: 
if \sampler is an $(\eps, \delta)$-DP sampler that is $\alpha$-accurate on class $\cBB^{\otimes d}$ with sample size $n$, then $\me_c$ (Algorithm~\ref{alg:marginal_est}) is an $(\eps, \delta)$-DP, $(2 \alpha , \beta_0,\gamma_0, \cBB^{\otimes d})$-marginal estimator with sample size $cn$. 
%
% Let $c \in \mathbb{N}$, $\alpha \in [0,1]$, $\eps, \delta > 0$. If \sampler is an $(\eps, \delta)$-DP sampler that is $\alpha$-accurate on class $\cBB^{\otimes d}$ with sample size $n$, then \me (in Algorithm~\ref{alg:marginal_est}) is an $(\eps, \delta)$-DP, $( \alpha + \frac{2}{\sqrt{c}}, \frac{1}{75}, \frac{1}{3}, \cBB^{\otimes d})$-marginal estimator with sample size $cn$. 
\end{lemma}
\begin{proof}
Fix a distribution $\distr \in \cBB^{\otimes d}$. Let $\biasesfixed$ represent the vector of biases corresponding to $\distr$, and $Y_i$ for $i \in [c]$  and $\tilde{\biasesfixed}$ be as defined in Steps~\ref{step:sample} and~\ref{step:computeemp} of Algorithm~\ref{alg:marginal_est}. Consider any index $j \in [d]$. 
The expectations $\E[Y_i[j]]$ are the same for  all $i \in [c]$. Define $q[j]=\E[Y_1[j]]$.
Since $\sampler$ is $\alpha$-accurate with sample size $n$, we have $|q[j] - \biasesfixed_j| \leq \alpha$. 
Let $D$ be a positive constant to be set later. By Hoeffding's inequality (Claim~\ref{claim:hoeff}), 
$\Pr(|q[j] - \tilde{\biasesfixed}_j| \geq \frac{D}{\sqrt{c}}) \leq 2e^{-2D^2}$. By the triangle inequality, with probability at least $1-2e^{-2D^2}$, 
\begin{align}\label{eq:prob-of-bad-indices}
|\tilde{\biasesfixed_j} - \biasesfixed_j| \leq |\tilde{\biasesfixed_j} - q[j]| + |q[j] - \biasesfixed_j| \leq \alpha + \tfrac{D}{\sqrt{c}}.
\end{align}
Since (\ref{eq:prob-of-bad-indices}) holds for all $j \in [d]$,  the expected number of $j \in [d]$ such that $|q[j] - \biasesfixed_j| > \alpha + \frac{D}{\sqrt{c}}$ is at most $2d e^{-2D^2}$. By Markov's inequality, 
$$\Pr(|\{j \in [d] : |\biasesfixed_j - \tilde{\biasesfixed}_j| > \alpha + \tfrac{D}{\sqrt{c}} \}| \geq \tfrac{2d}{\gamma_0}  e^{-2D^2})) \leq \gamma_0.$$ 
Setting $D^2 = \frac 1 2 \ln(\frac{2}{\beta_0\gamma_0})$ ensures $\tfrac{2d}{\gamma_0}e^{-2D^2} \leq \beta_0 d$. Setting $c =\ceil{\frac{D^2}{\alpha^2}}$ further ensures that $ \alpha + \tfrac{D}{\sqrt{c}} \leq 2\alpha$. We thus get the desired accuracy guarantee on $\me_c$ when $c = \ceil{\frac{1}{2\alpha^2} \ln(\frac{2}{\beta_0\gamma_0})}$.

Finally, changing one entry in the dataset $\Datafixed$ changes a single entry in only one of the parts $\Datafixed^{(i)}$, and only this part is fed to the $i^{th}$ call to $\sampler$. Since $\sampler$ is $(\eps, \delta)$-DP, so is $\me_c$. This proves the lemma. 
\end{proof}

Next we show how to transform a marginal estimator 
for a product of bounded Bernoulli distributions into a marginal estimator for a product of arbitrary Bernoulli distributions.
%\sr{for $\cBB^{\otimes d}$ into a margin estimator for $\cB^{\otimes d}$.}
Let \bsc denote a \emph{binary symmetric channel} with bias $1/3$. That is, on input $\Datafixed \in \bit{n\times d}$, each bit gets flipped independently with probability $1/3$. In particular, $\bsc(\Datafixed) = \Datafixed \oplus \Datarvz$, where $\Datarvz \sim \Ber(1/3)^{\otimes n \times d}$.

\begin{algorithm}
    \caption{Marginal Estimator $\bernest$ for $\cB^{\otimes d}$}
    \label{alg:marginal_est_unbounded}
    \hspace*{\algorithmicindent} \textbf{Input:} dataset $\Datafixed \in \bit{n\times d}$, query access to marginal estimator $\me$ for $\cBB^{\otimes d}$\\
    \hspace*{\algorithmicindent} \textbf{Output:} marginal estimates $\tilde{\Biasesfixed} = (\tilde{\biasesfixed}_1, \ldots, \tilde{\biasesfixed}_d)$
    \begin{algorithmic}[1] % The number tells where the line numbering should start
            \State $\Datafixed^* \gets \bsc(\Datafixed)$  \Comment{Change initial distribution}
            \State $\Biasesfixed^* \gets \me(\Datafixed^*)$  \Comment{Get empirical estimates}
            \State $\tilde{\Biasesfixed} \gets (3\cdot\biasesfixed^*_1 - 1, \ldots, 3\cdot\biasesfixed^*_d - 1)$ \Comment{Rescale empirical estimates}
            \State \Return $\tilde{\Biasesfixed}$
    \end{algorithmic}
\end{algorithm}

\begin{lemma}[Reduction from General  to  Bounded Biases]\label{lem:me_bounded_to_unbounded}
If \me is an $(\alpha', \beta', \gamma, \cBB^{\otimes d})$-marginal estimator with sample size $n$, then $\bernest$ (in Algorithm~\ref{alg:marginal_est_unbounded}) is a $(3\alpha', \beta', \gamma, \cB^{\otimes d})$-marginal estimator with sample size $n$. If $\me$ is $(\eps, \delta)$-differentially private, then so is $\bernest$.
%is also $(\eps, \delta)$-differentially private.
\end{lemma}

\begin{proof}
We begin with the accuracy proof. Fix an $(\alpha', \beta', \gamma, \cBB^{\otimes d})$-marginal estimator \me with sample size~$n$. Fix a distribution $\distr \in \cB^{\otimes d}$ with  biases $\Biasesfixed = (\biasesfixed_1, \ldots, \biasesfixed_d)$. Let $\Datarv \sim \distr^{\otimes n}$. Denote the output distribution of $\bsc(\Datarv)$ by $\corr$ and its biases by $\Biasesfixed' = (\biasesfixed'_1, \ldots, \biasesfixed'_d)$. Let $\bsc(\Datarv)^j_i = \Datarv^j_i \oplus \Datarvz$ be the output for the $j$th attribute on the $i$th data record, where $\datarvz^j_i \sim \Ber(1/3)$. Then, for all $j\in[d]$ and all $i \in [n]$, 
\begin{align*}
    \biasesfixed'_j 
    & = \Pr[\bsc(\Datarv)^j_i = 1] 
     = \Pr[\datarv^j_i \oplus \datarvz^j_i = 1 ] 
     = \Pr[\datarv^j_i = 1 \land \datarvz^j_i = 0] + \Pr[\datarv^j_i = 0 \land \datarvz^j_i = 1] \\
    & = \biasesfixed_j \cdot \frac{2}{3} + (1-\biasesfixed_j) \cdot \frac{1}{3} = \frac{\biasesfixed_j}{3} + \frac{1}{3}.
\end{align*}
Thus, $\bsc(\Datarv)\in \cBB^{\otimes d}$, as desired. Since $\me$ is a $(\alpha', \beta', \gamma, \cBB^{\otimes d})$-marginal estimator with sample size~$n$, estimator \me returns $(\biasesfixed^*_1, \ldots, \biasesfixed^*_d)$ such that with probability at least $1-\gamma$,
\begin{equation}\label{eq:marg_est}
    |\{j \in [d] : |\biasesfixed'_j - \biasesfixed^*_j| > \alpha' \}| < \beta' d.
\end{equation}
Substituting $\biasesfixed'_j=\biasesfixed_j/3+1/3$ in the left-hand side of in~\eqref{eq:marg_est} and  then using $\tilde{\biasesfixed}_j=3\biasesfixed^*_j+1$, we get
\begin{align*}
    |\{j \in [d] : |\frac{\biasesfixed_j}{3} + \frac{1}{3} -  \biasesfixed^*_j| > \alpha' \}|
    & = |\{j \in [d] : |\biasesfixed_j - (3\biasesfixed^*_j - 1)| > 3\alpha' \}| \\
    & = |\{j \in [d] : |\biasesfixed_j - \tilde{\biasesfixed}_j| \leq 3\alpha' \}|.
\end{align*}
Thus, \bernest is a $(3\alpha', \beta', \gamma, \cB^{\otimes d})$-marginal estimator with sample size $n$. 

Finally, we show that  \bernest is differentially private.
Suppose \me is $(\eps, \delta)$-differentially private. Fix neighboring datasets \Datafixed and $\Datafixed'$ that differ on record $i$. Then $\bsc(\Datafixed)$ and $\bsc(\Datafixed')$ still only differ on record $i$. The output of \bernest is a post-processing of \me, and thus \bernest is $(\eps, \delta)$-differentially private.
\end{proof}

We prove our main result by combining Lemmas~\ref{lem:sam_to_me}--\ref{lem:me_bounded_to_unbounded} with a lower bound on marginal estimation that is obtained using the fingerprinting codes technique of Bun, Ullman and Vadhan~\cite{BunUV14j}. We use a corollary of the version of the result from~\cite{BunUV14j} presented by Kamath and Ullman~\cite{KamathUprimerpaper20}.

\begin{theorem}[Consequence of \cite{KamathUprimerpaper20}, Theorem 3.3% was \cite{BunUV14j}, Theorem 3.10
]\label{thm:buv_LB}
Suppose there exists a function $n = n(d)$, such that for every $d \in \N$, there is a $(\alpha_0, \beta_0, \gamma_0, \cB^{\otimes d})$-marginal estimator $\me: \{0,1\}^{n \times d} \to \R^d$ that is  $(\eps,\frac{1}{100n})$-DP, where $\alpha_0, \beta_0, \gamma_0 \in (0,1)$ are sufficiently small absolute constants. Then $n  = {\Omega}(\sqrt{d}/\eps)$.
\end{theorem}
%Finally, we combine Lemmas~\ref{lem:sam_to_me} and \ref{lem:me_bounded_to_unbounded} to prove Theorem~\ref{thm:bb_lb}.
\begin{proof}[Proof of Theorem~\ref{thm:bb-lb}]
Let $\eps,\delta \in (0,1]$ with $\delta<\frac{1}{100n}$. 
Let $\alpha_0,\beta_0,\gamma_0$ be the constants from Theorem~\ref{thm:buv_LB}, and set $\alpha = \frac{\alpha_0}{6}$. 
Let $\sampler$ be an $\alpha$-accurate, $(\eps,\delta)$-DP sampler for the class $\cBB^{\otimes d}$ for datasets of some size $n$. By Lemma~\ref{lem:sam_to_me}, there exists an $(\eps, \delta)$-differentially private,  $(\frac{\alpha_0}{3},\beta_0, \gamma_0,\cBB^{\otimes d})$-marginal estimator $\me_c$ for datasets of size $cn$ for an absolute constant $c= c(\alpha,\beta_0,\gamma_0)$. 
%
By Lemma~\ref{lem:me_bounded_to_unbounded}, \bernest defined in Algorithm~\ref{alg:marginal_est_unbounded} is a $(\eps, \delta)$-differentially private, $(\alpha_0, \beta_0,\gamma_0,\cB^{\otimes d})$-marginal estimator for datasets of size $cn$. 
%
By Theorem~\ref{thm:buv_LB}, $n = {\Omega}(\sqrt{d}/\eps)$, as desired.
%
%
% Set $\alpha = 1/100$.
% Suppose there exists a function $n = n(d)$ such that for every $d\in \N$, there exists an algorithm $\sampler: \bit{n\times d} \to \R^d$ that is an $(\eps, o(\frac{1}{n})$-differentially private $\frac{1}{100}$-accurate sampler for $\cBB^{\otimes d}$ with $n$ samples. By Lemma~\ref{lem:sam_to_me}, there exists an $(\eps, o(\frac{1}{n})$-differentially private,  $(\frac{1}{50}, \frac{1}{75},\frac{1}{3},\cBB^{\otimes d})$-marginal estimator \me for datasets of size $40000n$.~\footnote{We made no effort to optimize the leading constant.}\srnote{Indeed, I think it can be set to 400. It is distracting to have large constants and more complicated arithmetic than necessary. Since here we need $\alpha'=1/3$, we can set this parameter to 1/9 for the bounded case. So, we can set $2/\sqrt{c}$ to 1/10.}
% \srnote{State what $c$ we use when we apply Lemma C.3. } By Lemma~\ref{lem:me_bounded_to_unbounded}, \bernest defined in Algorithm~\ref{alg:marginal_est_unbounded} is a $(\eps, o(\frac{1}{n})$-differentially private, $(\frac{3}{50}, \frac{1}{75},\frac{1}{3},\cB^{\otimes d})$-marginal estimator for datasets of size $40000n$. Thus, by Theorem~\ref{thm:buv_LB}, $n = \tilde{\Omega}(\sqrt{d})$, as desired.
\end{proof}
%\input{prod-ub-old}








%%%%%%%%%%%%%%%%%%%%%%%%%%%%%%%%%%%%%%%%%%%%%%%
%
%%%%%%%%%%%%%%%%%%%%%%%%%%%%%%%%%%%%%%%%%%%%%%%

%%%%%%%%%%%%%%%%%%%%%%%%%%%%%%%%%%%%%%%%%%%%%%%
%
%%%%%%%%%%%%%%%%%%%%%%%%%%%%%%%%%%%%%%%%%%%%%%%

\addcontentsline{toc}{section}{References}
\bibliographystyle{plain}
\bibliography{refs}{}

%%%%%%%%%%%%%%%%%%%%%%%%%%%%%%%%%%%%%%%%%%%%%%%
%
%%%%%%%%%%%%%%%%%%%%%%%%%%%%%%%%%%%%%%%%%%%%%%%



\ifnum\supplemental=0
\appendix
\newpage
\section*{Appendix}
\fi 

\section{Inequalities Used in Technical Sections} \label{sec:inequalities}
We argue that the moments of the average of several identically distributed random variables are no larger than the corresponding moments of the individual random variables.
\begin{claim}\label{claim:momavg}
If random variables $A_1, \dots, A_k$ are identically distributed, then, for all $\lambda > 0$,
$$\E\left[ \left(\frac{1}{k}\sum_{i=1}^k A_i \right)^{\lambda} \right] \leq \E\left[ A_1^{\lambda} \right].$$
\end{claim}
\begin{proof}
By Jensen's inequality, $\left(\frac{1}{k}\sum_{i=1}^k A_i \right)^{\lambda} \leq \frac{1}{k}\sum_{i=1}^k A_i^{\lambda}$ for any fixed values of $A_1, \dots, A_k$. We take expectation on both sides, then use the linearity of expectation and that  $A_1, \dots, A_k$ are identically distributed:
\begin{align*}
    \E\left[\left(\frac{1}{k}\sum_{i=1}^k A_i \right)^{\lambda}\right] 
    \leq \E\left[\frac{1}{k}\sum_{i=1}^k A_i^{\lambda}\right] 
     = \frac{1}{k}\sum_{i=1}^k \E\left[ A_i^{\lambda}\right] = \E\left[ A_1^{\lambda} \right].
\end{align*}
\end{proof}
%\begin{claim}[\cite{KLSU19}, Lemma 5.6]\label{claim:Chernoff1}
%If $A_1, \dots, A_m$ are drawn independently from $\Ber(p)$, then, for all $\gamma > 0$,
%$$ \Pr\left( \frac{1}{m} \sum_{i=1}^m A_i \geq p + \gamma \right) \leq e^{-d_{KL}(p || p + \gamma)m } $$ and 
%$$ \Pr\left( \frac{1}{m} \sum_{i=1}^m A_i \leq p - \gamma \right) \leq e^{-d_{KL}(p || p - \gamma)m }.$$ 
%\end{claim}
We also use Bernoulli's inequality in our lower bound for product distributions.
\begin{lemma}[Bernoulli's inequality]\label{lem:bernoulli}
For all real $a \geq -1$ and nonnegative integers $r$, $(1+a)^r \geq 1+ar$. 
\end{lemma}

\subsection{Concentration Inequalities}
\begin{claim}[Chernoff Bounds]\label{claim:cher_bounds}
Let $A$ be the average of $m$ independent 0-1 random variables with $\mu=\E[A]$. For $\gamma \in (0,1)$,
\begin{align*}
  \Pr[A \geq \mu(1+\gamma)] \leq e^{-\frac{\gamma^2 \mu m}{3}};\\ 
  \Pr[A \leq \mu(1-\gamma)] \leq e^{-\frac{\gamma^2 \mu m}{2}}.
\end{align*}
%
For $\gamma \geq 0$, 
\begin{align*}
\Pr[A \geq \mu(1+\gamma)] \leq e^{-\frac{\gamma^2 \mu m}{2+ \gamma}};\\
\Pr[A \leq \mu(1-\gamma)] \leq e^{-\frac{\gamma^2 \mu m}{2 + \gamma}}.
\end{align*}
\end{claim}

\begin{claim}[\cite{KLSU19}, Lemma 2.8, Gaussian Concentration]\label{lem:gaussconc}
If $A$ is drawn from $\mathcal{N}(0,\sigma^2)$, then, for all $t > 0$,
$$ \Pr\left( |A| > t \sigma \right) \leq 2e^{-t^2 / 2}.$$
\end{claim}

\begin{claim}[Hoeffding's Inequality]\label{claim:hoeff}
Let $A$ be the average of $m$ independent random variables in the interval $[0,1]$ with $\mu=\E[A]$. For $h \geq 0$,
\begin{align*}
  \Pr[A - \mu \geq h] \leq e^{-2mh^2}. \\
  \Pr[\mu - A \geq h] \leq e^{-2mh^2}.
\end{align*}
\end{claim}


\section{Lemmas on Privacy Amplification by Subsampling}\label{app:cited-results} \label{sec:resow}
\begin{definition}[\cite{li2012sampling}, Definition 3] An algorithm \sampler is $(\beta, \eps, \delta)$-DPS if and only if $\beta > \delta$ and the algorithm $\sampler^\beta$ is $(\eps, \delta)$-DP where $\sampler^\beta$ denotes the algorithm to first sample with probability $\beta$ (include each tuple in the input dataset with probability $\beta$), and then apply \sampler to the sampled dataset.
\end{definition}

\begin{theorem}[\cite{li2012sampling}, Theorem 1]\label{thm:LQS12}
Any $(\beta_1, \eps_1, \delta_1)$-DPS algorithm is also $(\beta_2, \eps_2, \delta_2)$-DPS for any $\beta_2 < \beta_1$ where $\eps_2 = \ln\left(1 + \left(\frac{\beta_2}{\beta_1} (e^{\eps_1} - 1) \right)\right)$, and $\delta_2 = \frac{\beta_2}{\beta_1}\delta_1$.
\end{theorem}

%\begin{theorem}[\cite{BalleGG2018}, Theorem 9]\label{thm:BalleGG2018}
%Let $m,n \in \mathbb{N}$, with $m>n$. If algorithm $A$ is $(\eps, \delta)$-DP for datasets $\Datafixed$ of size $m$, and algorithm $B$ samples a dataset $\Datafixed'$ of size $n$, uniformly from all subsets of size $n$ in $\Datafixed$, and outputs $A(\Datafixed')$, then $B$ is $(\eps', \delta')$-DP, where $\eps' = \ln\left(1 + \left(\frac{n}{m} (e^{\eps} - 1) \right)\right)$, and $\delta' = \frac{m}{n}\delta$
%\end{theorem}

%\section{Better Universe Transformation}
%\sstext{Add exposition.}
%\begin{algorithm}
%        \caption{Universe transformation algorithm $\reduction^{\leftarrow}$}
%    \label{alg:univtranssoph}
%    \hspace*{\algorithmicindent} \textbf{Input:} Ten samples $y_1, y_2, \ldots, y_{10} \in \{0,1\}^{2k}$, value $\alpha^*$, special element $S$ \\
%    \hspace*{\algorithmicindent} \textbf{Output:} $y' \in [2k+1]$
%    \begin{algorithmic}[1] % The number tells where the line numbering should start
%           \If{$y_i = (0,\dots,0)$ for all $i\in[10]$}
%           \State Set $y' \gets S$
%           \Else 
 %          \State Let $J$ be the set of indices in $y_1, \ldots, y_{10}$ that are equal to $1$ 
%           \State Set $q \gets (1-\frac{\alpha^*}{k})^{10k}$
%           \State Sample $c \sim \Ber(\frac{1-\alpha^* - q}{1-q})$
%           \If{$c=1$}
%           \State Set $y' \gets S$
%           \Else 
 %          \State Choose $y'$ uniformly at random from $J$
 %          \If{ $y' \geq S$}
 %          \State Set $y' \gets y'+1$ \Comment{Since we removed column $S$ from the input samples}
 %          \EndIf 
 %          \EndIf
 %          \EndIf
 %          \State Output $y'$
 %   \end{algorithmic}
%\end{algorithm}
%\begin{lemma}\label{lem:univtrans}
%Fix sufficiently small $\alpha >0$, let $\alpha^* = 60\alpha$, and fix a distribution $\distr \in \ca$ with special element $j$, and let $\corr$ be the corresponding product distribution and $\corr'$ be $\corr^{\otimes 10}$. If $\datarvy \sim \corr'$, then the random variable $\reduction^\leftarrow(\datarvy, \alpha^*,j)$ is distributed exactly as \distr. 
%\end{lemma}

%\begin{proof}
%First, note that we output the special element if the ten input samples were all 0's. For samples drawn from $\corr'$, this occurs with probability $(1-\frac{\alpha^*}{k})^{10k}\leq e^{-10\alpha^*} \leq 1-\alpha^*$ for sufficiently small $\alpha$.\srnote{Where is the bottleneck for choosing ten datasets? Can we use a smaller number, e.g., 2 or 3?} \ssnote{Yes, I just checked, and any of those should work- they will just change the upper limit on $\alpha$- smaller the number, smaller the constant upper limit. As always, I picked 10 because I am a coward and wanted to be sure if would work :)} In Algorithm~\ref{alg:univtrans} we define $q = (1-\frac{\alpha^*}{k})^{10k}$. Secondly, if at least one of the samples has a nonzero entry, we also output the special element  when the coin $c$ is equal to 1. In total, the probability of returning the special element is then
%$$
% q + (1-q) \cdot \frac{1-\alpha^* - q}{1-q} = 1-\alpha^*,
%$$
%as desired.

%With the remaining probability we return a uniformly random participating element of $\distr$, so each of them is returned with probability $\frac{\alpha^*}k$, as desired.
%\end{proof}
%\fi

\fi

\begin{comment}
%%%%%%%%%%%%%%%%%%%%%%%%%%%%%%%%%%%%%%%%%%%%%%%%%%%%%%%%%%%%
\ifnum\neurips=1   
%%%%%%%%%%%%%%%%%%%%%%%%%%%%%%%%%%%%%%%%%%%%%%%%%%%%%%%%%%%%
\section*{Checklist}

\begin{enumerate}

\item For all authors...
\begin{enumerate}
  \item Do the main claims made in the abstract and introduction accurately reflect the paper's contributions and scope?
    \answerYes{}
  \item Did you describe the limitations of your work?
    \answerYes{See Section~\ref{sec:results}}
  \item Did you discuss any potential negative societal impacts of your work?
    \answerYes{See Section~\ref{sec:results}}
  \item Have you read the ethics review guidelines and ensured that your paper conforms to them?
    \answerYes{}
\end{enumerate}

\item If you are including theoretical results...
\begin{enumerate}
  \item Did you state the full set of assumptions of all theoretical results?
    \answerYes{}
	\item Did you include complete proofs of all theoretical results?
    \answerYes{We include one proof in the main body, and all other proofs are in supplementary material.}
\end{enumerate}

\item If you ran experiments...
\begin{enumerate}
  \item Did you include the code, data, and instructions needed to reproduce the main experimental results (either in the supplemental material or as a URL)?
    \answerNA{}
  \item Did you specify all the training details (e.g., data splits, hyperparameters, how they were chosen)?
    \answerNA{}
	\item Did you report error bars (e.g., with respect to the random seed after running experiments multiple times)?
    \answerNA{}
	\item Did you include the total amount of compute and the type of resources used (e.g., type of GPUs, internal cluster, or cloud provider)?
    \answerNA{}
\end{enumerate}

\item If you are using existing assets (e.g., code, data, models) or curating/releasing new assets...
\begin{enumerate}
  \item If your work uses existing assets, did you cite the creators?
    \answerNA{}
  \item Did you mention the license of the assets?
    \answerNA{}
  \item Did you include any new assets either in the supplemental material or as a URL?
    \answerNA{}
  \item Did you discuss whether and how consent was obtained from people whose data you're using/curating?
    \answerNA{}
  \item Did you discuss whether the data you are using/curating contains personally identifiable information or offensive content?
    \answerNA{}
\end{enumerate}

\item If you used crowdsourcing or conducted research with human subjects...
\begin{enumerate}
  \item Did you include the full text of instructions given to participants and screenshots, if applicable?
    \answerNA{}
  \item Did you describe any potential participant risks, with links to Institutional Review Board (IRB) approvals, if applicable?
    \answerNA{}
  \item Did you include the estimated hourly wage paid to participants and the total amount spent on participant compensation?
    \answerNA{}
\end{enumerate}

\end{enumerate}
\fi 
\end{comment}
\end{document}
