Since the COVID-19 outbreak, the use of digital learning or education platforms has significantly increased. Teachers now digitally distribute homework and
%pose learning queries
provide exercise questions. In both cases,
%to prevent their students from cheating,
teachers need to continuously develop novel and individual questions. This process can be very time-consuming and should be facilitated and accelerated both through exchange with other teachers and by using Artificial Intelligence (AI) capabilities. To address this need, we propose a multilingual Wikimedia framework that allows for collaborative worldwide teacher knowledge engineering and subsequent AI-aided question generation, test, and correction.
As a proof of concept, we present >>PhysWikiQuiz<<, a physics question generation and test engine. Our system (hosted by Wikimedia at \url{https://physwikiquiz.wmflabs.org}) retrieves physics knowledge from the open community-curated database Wikidata. It can generate questions in different variations and verify answer values and units using a Computer Algebra System (CAS). We evaluate the performance on a public benchmark dataset at each stage of the system workflow. For an average formula with three variables, the system can generate and correct up to 300 questions for individual students based on a single formula concept name as input by the teacher.
%TODO: maybe include some (more) results
%\keywords{First keyword  \and Second keyword \and Another keyword.}
%keywords:
%Artificial Intelligence
%Education Systems
%Question Generation
%Information Systems
%Information Retrieval
%Wikidata