\centerline{{\bf \Large Acknowledgments}} \vspace{5mm} \noindent

Firstly, I would like to express my most sincere gratitude to my supervisor, Professor Pascale Fung, for her knowledgeable guidance and supervision throughout my Ph.D. The valuable research opportunities she suggested, such as the 'Chatbot Millionaire Challenge' and 'Kaggle COVID-19 Open Research Dataset Challenge', have triggered my research interest and led to my research outcomes on question answering. Furthermore, her confidence in me and continuous support for my research work have inspired me to push myself to achieve huge impacts and made my Ph.D. journey less lonely. Furthermore, her excellent research taste, which I have been learning from, allowed me to grow constantly as a researcher. Also, as my role model, her passionate and positive attitude in life and work has affected my mindset and helped me conquer many difficulties. 

Next, I would like to express my appreciation to Professor Bertram Shi, Professor Qifeng Chen, Professor Lei Chen, and Professor Heng Ji for taking their time to be on my thesis examining committee. I am also thankful to Professor Bertram Shi and Professor Qifeng Chen for being on my thesis proposal committee and providing me with insightful feedback. I also want to thank Tania Leigh Wilmshurst, who proofread my papers many times and gave me insightful and useful advice on academic writing.

In the last four years, I have had an exciting journey in the CAiRE lab with amazing lab colleagues. Therefore, I want to thank Yan Xu, Tiezheng Yu, Peng Xu, Elham Barezi and Farhad Bin Siddique, Ziwei Ji, Wenliang Dai, and Zihan Liu for the past collaborations. My thanks go to Yan Xu for the collaborations on my early stage question answering projects, including the 'Chatbot Millionaire Challenge' and 'Kaggle COVID-19 Open Research Dataset Challenge'; it is a great pleasure to work with her; to Tiezheng Yu for the multiple great collaborations on summarization research and on the World Health Organization(WHO) projects; to Peng Xu for the insightful discussions on question generation projects; to Elham Barezi and Farhad Bin Siddique for the support on 'Kaggle COVID-19' projects; to Zihan Liu for discussions on the prompting based context generation work; and to Wenliang Dai and Ziwei Ji for their help on the multi-hop question generation paper. Thanks also go to Andrea Madotto, Zhaojiang Lin, and Genta Winata, who provided useful suggestions at the beginning of my Ph.D., and to Samuel Cahyawijaya, Yejin Bang, Etsuko Ishii, Nayoen Lee, and many others for the invaluable research experiences. 

I would further like to express my thanks to Dr. Xin Jiang, Xiaoguang Li, Dr. Lifeng Shang, and Professor Qun Liu, from Huawei Noah's Ark AI Lab, for the valuable research discussions and insightful feedback on the long-form question-answering projects during my internship there; their passion at research and their research attitude also affected me. I would like to thank Dr. Mostofa Patwary, Dr. Shrimai Prabhumoye, Dr. Peng Xu, Dr. Ryan Prenger, Dr. Mohammad Shoeybi, and Prof. Anima Anandkumar, and Bryan Catanzaro for their time, support and significant research interactions during my internship at Nvidia ADLR group. The insightful discussions on weekly meetings on my project with the whole group, and personal meetings with Mohammad every week, have helped improve my research habits and enhanced my research skills. 

Finally, I would like to express my gratitude to my friends in Shenzhen and Hong Kong, who have motivated me during the challenging times. The WeChat chatting during the COVID-19 pandemic and the relaxed Friday night dinner moment has given me unforgettable memories. I want to give my deepest appreciation to my family for supporting me all the way during my Ph.D., especially to my 3-years old niece 'little candy' who has brought so much happiness to me with her pure and innocent smile. Last but not least, I would like to thank my boyfriend, Dr. Jindi Zhang, for the quality time together and for his support all the time.