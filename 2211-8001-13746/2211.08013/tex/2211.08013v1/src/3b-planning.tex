\if\draft \subsection{Data structure used for planning} \fi
To estimate the volume of the surface, the robot needs to collect information of the whole surface, moving on a path suitable for this purpose. Two main approaches exist in path planning, that achieve this goal. Coverage path planning \cite{galceran2013survey} is a method to design paths that visit all points of interest while avoiding obstacles. Typical cost functions minimize the length of the path, whereas with informative path planning \cite{stache2021adaptive} the objective is to maximize the amount of information in a feasible path. An overview of path planning algorithms can be found in \cite{lavalle2006planning}.
%The data structure used for volume estimation is too computationally demanding for the purposes of planning. Instead we use a simplified data structure where we set
%\begin{equation}
%	k'(s) = \begin{cases} 1& \text{if }s=0\\ 0 &\text{otherwise}\end{cases}\\,
%\end{equation}
%which forces $\Sigma^{f}_\info$ to be diagonal.
%\stef{This is not super clear. What is k and what is s? I assume it's related to the grid and the case if the grid point is measured or not so far. Maybe, we need to define the grid first (also maybe draw it), and then it should be easy to explain equations related to it.}
%In this formulation, the height $\hat h$ of each inducing point is described by an univariate normal distribution, with scalar mean $\mu$ and standard deviation $\sigma$, where we preserve the expected value and variance for the inducing points from the more complex model, but lose the predictive capability for points not coinciding with the grid
%from the solution of the optimization problem 
%\begin{argmini}[2]
%	{\info}{ ||M^{f}_{\hat\info_{\iteration}} - M^{f}_{\info} ||_2}{}{\info^* = }
%	\addConstraint{\Sigma^{f}_\info}{\quad \mathrm{diagonal}}{}
%\end{argmini}
%\begin{equation}
%	\info^* = [M^{f_*}_{\hat\info_{\iteration}}, 
%\end{equation}
%\begin{equation}
%	\hat h({x_1}_{ij}, {x_2}_{ij}) \doteq \hat h_{ij} \sim \mathcal{N}(\mu_{ij}, \sigma_{ij}) = \mathcal{N}({(M^{f}_{\info})}_{ij}, {(\Sigma^{f}_\info)}_{ij}).
%	\label{eq:h-diagonal}
%\end{equation}

For the purposes of planning we
\begin{enumerate*}
	\item assume a constant nominal surface height $h_0$, which allows us to compute $d_l$;
	\item decouple the motion of the drone and the rotating LiDAR scan, since the latter is at least one order of magnitude faster;
	\item neglect the possible effect of shadows, as defined in Figure \ref{fig:DroneLidarHit};
	\item we find a discrete set of waypoints instead of a continuous time trajectory.
\end{enumerate*}
%The LiDAR scan is described by a line segment defined by the tuple $(\Omega, \phi, r)$, where $\Omega$ are the $xy$ components of the drone position, $\phi$ is yaw, and $r=\sqrt{d_{max}^2 - (z - h_0)}$ the length of the scan line, as illustrated in Figure \ref{fig:scanline}.
%\td{Scan line template}
%We can compute the expected information gain from the template scan line.
%\subsection{Planning Algorithms}
%\subsubsection{Greedy}
We use a greedy algorithm for planning, where the next point to be picked is the one with that minimizes the volume estimate uncertainty
\begin{argmaxi!}[2]
	{r_{k+1}}{ \| \sigma^\Volume_{k+1}(r_{k+1})\|_2}{}{r^*_{k+1} = }
	\addConstraint{\label{eq:motion} \| r_{k+1} - r_k \|_2 \leq ~}{R}{}
	\addConstraint{ r \in}{\{\PlanningSpace_i\}_{i=1}^4}{},
\end{argmaxi!}
where $\sigma^\Volume$ is computed with \eqref{eq:V-sigma},
$r$ is the reference,
$R$ is the radius of a ball where the next step can lie.
The simulations of the next section consider only the $x$ and $y$ components of $r$, and fix $z$ and the yaw.
The greedy algorithm is suboptimal but fast to evaluate, and allows us to test the surface reconstruction method. Future work will focus on implementing more advanced planning algorithms.
%In general the optimal reference trajectory lies in $\Reals^4$. \td{}
