We demonstrate our approach for volume estimation on a topographic map of the Alps mountain range, rich on features, which we scale down by a factor of 1000x.
%Figure \ref{fig:alps} shows a sample of a volume used in simulation.
\begin{figure}
\begin{center}
	\includegraphics[width=.49\columnwidth]{img-results/Altitude-ground-truthP} %xy	
	%\importsvg[\columnwidth]{img-results}{Altitude-ground-truth}    % The printed column width is 8.4 cm.
		\includegraphics[width=.48\columnwidth]{img-results/quality-of-fixP} %xy
		\caption{\label{fig:ground-truth} \label{quality-of-fix}
		The left plot shows the ground truth surface height used in simulation.
		The right plot shows the quality of fix $q_{pos}$ as a function of the $xy$ coordinates, for $z=7\,\mathrm{m}$ and zero yaw.
		Overlayed are the trajectories of the \tractor in solid blue and the greedy algorithm in dashed red.
		The yellow region on the top of the plot indicates constraints in the position coordinates.}  
	\end{center}
\end{figure}
%
\begin{figure}
\begin{center}
	\includegraphics[trim=35pt 3pt 17pt 3pt, clip, width=.49\columnwidth]{img-results/Tractor-reconstruction-sigma020P} %xy	
	\includegraphics[trim=35pt 3pt 17pt 3pt, clip, width=.49\columnwidth]{img-results/Tractor-reconstruction-sigma050P} %xy	
	%\importsvg[.49\columnwidth]{img-results}{Tractor-reconstruction-sigma020}    % The printed column width is 8.4 cm.
	%\importsvg[.49\columnwidth]{img-results}{Tractor-reconstruction-sigma050}    % The printed column width is 8.4 cm.
	\caption{Uncertainty map of the surface reconstruction with the \tractor after $20$ steps (left plot) and $50$ steps (right plot). \label{fig:tractor-reconstruction-sigma}
	}\end{center}
\begin{center}
	\includegraphics[width=.49\columnwidth]{img-results/Tractor-reconstruction020P} %xy	
	\includegraphics[ width=.49\columnwidth]{img-results/Tractor-reconstruction050P} %xy	
	%\importsvg[.49\columnwidth]{img-results}{Tractor-reconstruction020}    % The printed column width is 8.4 cm.
	%\importsvg[.49\columnwidth]{img-results}{Tractor-reconstruction050}    % The printed column width is 8.4 cm.
	\caption{Surface reconstruction with the \tractor after $20$ steps (left plot) and $50$ steps (right plot).\label{fig:tractor-reconstruction}
	}\end{center}
\end{figure}
%
%
\begin{figure}
\begin{center}
	\includegraphics[trim=35pt 3pt 17pt 3pt, clip, width=.49\columnwidth]{img-results/Greedy-reconstruction-sigma020P} %xy	
	\includegraphics[trim=35pt 3pt 17pt 3pt, clip, width=.49\columnwidth]{img-results/Greedy-reconstruction-sigma050P} %xy	
%	\importsvg[.49\columnwidth]{img-results}{Greedy-reconstruction-sigma020}    % The printed column width is 8.4 cm.
%	\importsvg[.49\columnwidth]{img-results}{Greedy-reconstruction-sigma050}    % The printed column width is 8.4 cm.	
	\caption{Uncertainty map of the surface reconstruction with the greedy planner after $20$ steps (left plot) and $50$ steps. \label{fig:greedy-reconstruction-sigma}
	}\end{center}
\begin{center}
	\includegraphics[ width=.49\columnwidth]{img-results/Greedy-reconstruction020P} %xy	
	\includegraphics[width=.49\columnwidth]{img-results/Greedy-reconstruction050P} %xy
%	\importsvg[.49\columnwidth]{img-results}{Greedy-reconstruction020}    % The printed column width is 8.4 cm.
%	\importsvg[.49\columnwidth]{img-results}{Greedy-reconstruction050}    % The printed column width is 8.4 cm.
	\caption{Surface reconstruction with the greedy planner after $20$ steps (left plot) and $50$ steps (right plot). \label{fig:greedy-reconstruction}
	}\end{center}
\end{figure}
%
We show simulation results for two trajectories, where the first one is a fixed, manually created \tractor, and the second is the trajectory resulting from applying the path planning algorithm from Section \ref{sec:planning}, in the variables $r_x$ and $r_y$, while keeping $r_z$ and $r_{\mathrm{yaw}}$ fixed.
The disposition of visually identifiable features in the example used in this simulations, shown in Figure \ref{fig:alps}, creates an uneven uncertainty of position map, as shown in the right plot of Figure \ref{quality-of-fix}, where darker colors represent lower uncertainty. 
%\ar{We have to comment on the greedy planner which "surrounds" the region with high uncertainty of figure 5 (right panel)}
%\ar{We note, that an uncertainty in the region of interest given in figure \ref{fig:ground-truth} is introduced by removing some of the markers used for localization. Figure \ref{fig:quality of fix} provides a visualization of the resulting uneven quality of position fix (the dark triangle at the bottom results from missing markers).}


%In the first case we use a fixed \tractor.
Figure \ref{fig:tractor-reconstruction-sigma} shows the path of the drone and the altitude uncertainty map, and
Figure \ref{fig:tractor-reconstruction} the surface reconstruction map for the fixed \tractor. Traversing the whole region results in reduced uncertainty of surface reconstruction. It is in general not trivial to manually design paths that avoid constraints or regions with insuficient localization quality.
%
As for the greedy algorithm and its resulting trajectory, we show in
Figure \ref{fig:greedy-reconstruction-sigma} the path of the drone and the altitude uncertainty map, and in
Figure \ref{fig:greedy-reconstruction} the surface reconstruction map.
The simulation results show that a feasible reference trajectory is found that visits most of the region of interest through regions with a high quality of position fix and drives the uncertainty of the volume down to $\sigma^\Volume/\mu^\Volume=2.26 \%$ and a relative error of $2.53\%$ for the greedy algorithm, comparable with $\sigma^\Volume/\mu^\Volume=2.42 \%$ with a relative error of $2.30\%$ for the \tractor.
We speculate that the two methods perform similarly in this example due to the greedy nature of the planning algorithm used, which optimizes only for the next step ahead, thus finding only an approximate solution for the optimization problem defined in \eqref{main-optimization-problem}. Also note that in this preliminary result the yaw and altitude are kept fixed, so the planner has less degrees of freedom to explore. These factors taken together make it difficult to improve on the benchmark \tractor.
%\ar{This deserves a comment. Can we speculate that it is because there is more information in the \tractor case, coming from the region with large uncertainty, which is surrounded in the case of the greedy planner? So the greedy planner results in slightly less accurate estimation on average, but with less uncertainty, and it is overall resulting in safer path for the drone?}
Figure \ref{fig:volume-vs-iteration} shows the evolution of the volume estimate as well as the uncertainty as a function of the number of samples collected. The two paths have the same length and number of samples, and the evolution of the volume and its uncertainty is similar for both paths.
%
\begin{figure}
\begin{center}
	\includegraphics[width=.49\columnwidth]{img-results/Tractor-volume-estimate051P} %xy	
	\includegraphics[width=.49\columnwidth]{img-results/Greedy-volume-estimate051P} %xy	
	\caption{\label{fig:volume-vs-iteration} Evolution of the volume estimate and its uncertainty along the trajectory. The left plot shows the result for the \tractor and the right plot for the greedy algorithm.
	}\end{center}
\end{figure}
%
Furthermore, the final reconstructions from Figures \ref{fig:tractor-reconstruction} and \ref{fig:greedy-reconstruction} approximate the ground truth shown in Figure \ref{fig:ground-truth} adequately.
In summary the simple greedy planner finds a path that is by all metrics similar to the manually designed \tractor, allowing the automation of the task of designing paths for experimental volume estimation campaigns.
%Comparing the resulting final reconstructions from Figures \ref{fig:tractor-reconstruction} and \ref{fig:greedy-reconstruction} with the ground truth \ref{fig:ground-truth} visually, we see that the fixed pattern results in a less accurate reconstruction. The final uncertainty of volume estimation is also more than the one using the greedy path planning, amounting to XY vs XZ for the greedy planner.
%Figure \ref{fig:volume-vs-iteration} shows that the greedy algorithm achieves a more accurate estimate of the volume faster, at iteration X as compared to iteration Y for the fixed pattern.
