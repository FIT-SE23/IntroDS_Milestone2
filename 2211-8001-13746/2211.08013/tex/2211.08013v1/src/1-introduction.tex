Autonomous robotic platforms are increasingly used for data collection, for example in structural inspection \cite{almadhoun2016survey}, agriculture surveillance, search and rescue, and industrial environments.
%In industrial settings, mobile sensing platforms have been used to automate the manual reading of sensors displays, obtain thermal and acoustic data in remote operations, using wheel-, legged-based, or flying robots.
In industrial warehouses it is common to store raw materials as stockpiles, and determining the current amount of material in stock is of paramount importance for logistics. However, to the best of our knowledge, the task of estimating the volume with an automated robotic platform has not been addressed. On the contrary, current business practice is to take differential measurements of the volume added or removed, which is prone to drift over time, and on periodic inspections from experts, which are costly and inaccurate.
%Furthermore, for certain tasks, a robot can do better than a human. Quadcopters for example can explore the space in 3D, manouvering in cluttered spaces and are able to achieve observational poses which would otherwise be impossible, while transporting a usefull payload of sensors. % Needs to be rewritten
A quadcopter equipped with an adequate suite of sensors could be used for this purpose, since it can fly above the pile of bulk material taking advantage of its maneuverability to take measurements from poses otherwise unreachable while avoiding obstacles in cluttered environments.
Estimating the volume using autonomous quadcoptor in indoor environment imposes several requirements, e.g.:
\begin{enumerate*}
	\item indoor localization, where the location uncertainty depends on the drone state,
	\item accurate measurements in environments with uneven light and dust,
	\item an efficient surface reconstruction method, able to cope with large amounts of data, and
	\item path planning, due to the limited time budget to fly the drone and the presence of obstacles.
\end{enumerate*}

%\stef{The start is good, but we need to transition to the task that we are tackling. It's not quite obvious that measuring stockpile volumes is necessary at all (I don't think there are other papers tackling this task). You can say: it's shown that drones can be used in different industrial inspection tasks such as remote sensing etc. [cite, cite, cite]. However, one interesting task was not addressed so far. In industry raw materials are quite often stored as stockpiles and currently there are no fast and reliable approaches to measure the volume of these stockpiles (Also, we might want to add few examples of stockpiled materials). We believe that this task is interesting because: 1) it requires indoor localization; 2) we need to precisely measure volumes 3) the measurements need to be done fast and effectively (i.e. we need path planning). I would also emphasise that 1) + 2) make the task interesting since there is inherent inaccuracy coming from drone position measurements.}
\if\draft \subsection{Literature review} \fi
\if\draft \subsubsection{Indoor localization} \fi
Localizing a mobile platform in a GPS-deprived environment with a known map can be achieved using information obtained from dead-reckoning, infrared, radio or sound-based distance measurements, visual information using a motion capture system, or from an onboard camera. In this work localization is inferred from an onboard camera, detecting a set of fixed, previously mapped features, given the lower overall system cost, flexibility and taken into account the accuracy requirements.

\if\draft \subsubsection{LiDAR measurement} \fi
Surface reconstruction can be performed from images, with photogrammetry methods such as structure from motion (SfM)~(\cite{newcombe2011kinectfusion}). This methods are however problematic for objects with homogeneous surfaces or improper lighting. An alternative way is via active laser scanners (LiDAR)
which project a laser beam and measure the time of flight of the reflected light. This sensors are precise and are not affected by the effect of scale uncertainty present in vision based measurements. Furthermore 2D LiDAR systems are lighter than their 3D counterparts, allowing the use of more maneuverable quadrotors.
There are commercial examples of drone-based solutions that use this method in outdoor environments~(\cite{dronedeploy}).
%
\if\draft \subsubsection{Surface model} \fi
\if\draft \subsubsection{Planning} \fi
The reconstruction quality depends to a large degree on the availability and quality of measurements. Classic approaches for quality-driven and automated 3D scanning use volumetric (\cite{khalfaoui2013efficient}) and Poisson mesh-based metric.
%Leveraging the capabilities of a flying 3D scanner, algorithms have been proposed to optimize the viewpoints of a drone to attain a high-quality reconstruction of its environment (\cite{hepp2018plan3d}).
Algorithms also been proposed for multi-view stereo reconstruction (\cite{hepp2018plan3d}), defining heuristics to decide on the utility of the next measurements and optimize set viewpoints based on initial scans. 
%These algorithms usually define heuristics to decide on the utility of the next measurements and optimize set viewpoints based on initial scans. Using 3D-LiDAR is also imposing constraints on the weight and maneuverability of the quadcopters.
Alternatively, 2D laser scanner has been used for 3D mapping in \cite{elasticlidar, loam, loamsimilar}, often mounting the 2D scanner on a rotating motor to emulate 3D LiDAR properties.%, which requires additional equipment, thus imposing  operational constraints. \td{Trim paragraph}

In this paper, we consider the problem of estimating the volume of material within a given domain using a drone mounted 2D LiDAR unit operating in an indoor environment, leveraging information about the surface.
%\stef{ I would here say why you use lidar and why this system is a good solution (compared for example to SfM reconstruction). You also want to give some insights about the system and about your method as well. For example: We chose LiDAR for surface reconstruction because the measurements give absolute distances, which eliminates scale uncertainty present in vision based measurements. Furthermore, our approach does not require heavy 3D lidars, and hence we can use lightweight quadrotors which are more suitable for the task since they can fly in narrow warehouse spaces. Although it is quite challenging to make volume estimation which such system, we demonstrate that our approach can precisely estimate the volume using simple 2D lidar sensor. To achieve this we leverage mathematical model of the surface and uncertainty estimation of the positioning system. }
This setting has challenges unique to GPS denied environments, notably the uncertainty in the localization depends on the position of the drone, which must follow trajectories that keep enough features in view to maintain localization accuracy.	
In \cite{hollinger2012uncertainty} the authors consider visual inspection of ship hulls using underwater vehicles. In \cite{zhu2021online} the authors consider inspection of 3D object, and \cite{popovic2020informative} deals with a similar problem setting, but localization uncertainty is not taken into account. This work differs since it deals with indoor environments and carefully considers the uncertainty of the measurements.
 Our contributions are threefold: 
\begin{enumerate*}
\item we derive measurement models and uncertainty estimates for the camera-based localization scheme and the LIDAR system used to measure the surface;
\item we propose a scalable methodology for estimating the volume of material based on LIDAR measurements and qualifying the uncertainty of our estimate; 
\item we propose a preliminary informative path planning method that greedily minimizes the uncertainty in the volume estimate.
\end{enumerate*}

