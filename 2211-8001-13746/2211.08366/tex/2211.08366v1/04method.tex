\section{User-Specific Biclustering-based Collaborative Filtering}
\label{sec:usbcf}

Grounded on the previously surveyed contributions and further empirical evidence, this section introduces a new approach, User-Specific Biclustering-based Collaborative Filtering (USBCF). 

The proposed USBCF approach is divided into four major steps. First, biclusters are mined from rating data using biclustering searches that guarantee the exhaustive discovery of flexible structures of statistically significant, and well-defined rating patterns (section 4.1). 
Second, and inspired by the Singh and Mehotra contributions \citep{BCF-impactbiclusteringcf-Singh-2018}, the found biclusters are used to create larger subspaces personalized for each user (section 4.2). These subspaces are denser and show stronger preference correlation than the original rating data. Third, we introduce a novel matching mechanism between an active user and the enlarged subspaces that considers both the coverage and correlation between the user and the biclusters (section 4.3). By integrating item coverage and correlation profile, USBCF guarantees that the selected subspaces per user are relevant, showing reduced propensity to contain divergent preferences. Fourth, and once the personalized subspace is created for a particular user, a well-established CF algorithm is used to originate a unique predictive model for each system user (section \ref{coclussec}). Finally, secondary rating estimates from checkboard coclustering structures are further collected to account for missing ratings that may fall outside each personalized subspace.

The methodology can be easily parallelized since the entire process in independent for each user. \autoref{fig:usbcfapproach} presents and overview of the USBCF methodology applied over an illustrative sample of user-item ratings and the corresponding user-specific CF model.

\begin{figure*}[h]
    \centering
    \includegraphics[width=\linewidth]{usbcfapproach.pdf}
    \caption{USBCF approach applied on a small U-I matrix.}
    \label{fig:usbcfapproach}
\end{figure*}


\subsection{Bicluster discovery in user-item rating data}

The first step of the approach consists of mining biclusters within user-item rating data, such that three major properties of interest are observed: i) nearly-exhaustive biclustering searches are employed to guarantee that all potentially relevant biclusters are identified, thus maximizing the coverage of the original rating dataset and, as a result, supporting the prediction of missing ratings; ii) the targeted type of biclusters (Fig.~2) should guarantee that meaningful and well-delineate preference patterns are retrieved, allowing coherent preference variations to take place on different items of a subspace; and iii) the retrieved biclusters should pass statistically significance tests, thus minimizing the susceptibility to use false positive preference patterns (retrieved preference patterns that occur by chance) in the prediction process. %i.e. preference patterns i.e. they do not occur by chance,  minimizing the  an adequa to discover biclusters. The goal is to obtain a biclustering solution that highlights the maximum amount of correlations in the dataset.

Existing biclustering-based CF approaches generally fail to satisfy some of the properties due to the underlying biclustering algorithmic choices. In this work, we select QUBIC2\footnote{\url{https://github.com/OSU-BMBL/QUBIC2}}  \citep{BC-QUBIC2-2019} as the biclustering algorithm. QUBIC2 is an efficient nearly-exhaustive deterministic biclustering algorithm that identifies statistically significant scaling-patterned biclusters with constant-values on the rows. 

When applied over an U-I Matrix, QUBIC2 identifies groups of users who have the same preference for the bicluster items. The tolerated noise per bicluster can be further parameterized in QUBIC2. Following empirical evidence, we restrict the biclustering search to discover non-noisy maximal biclusters as noise and preference divergences can be accommodated in the subsequent subspace enlarging step (section 4.2). The minimum size of the biclusters is controlled by a parameter \textit{minCols}. Different values of \textit{minCols} may lead to different biclustering solutions. Instead of setting the \textit{minCols} parameter to a fixed value, in this work, we created an enriched biclustering solution that aggregates the solutions from running QUBIC2 with different \textit{minCols} setups. A detailed description of the algorithm can be found in QUBIC's original works \citep{BC-QUIBIC-2009,BC-QUBIC2-2019}.

\subsection{Biclustering neighborhood formation}

The goal of this second step is to create, for each user, a new data space with meaningful and personalized information. To do so, we take advantage of the biclusters found by the biclustering algorithm in the previous module. After the biclusters are generated, we can expect that some of those biclusters represent a user's preferences better than others. From this perspective, we try to find a \enquote{neighborhood} of biclusters. In other words, a subset of overall found biclusters such that each bicluster in this subset, $B\in\mathcal{B}$, has a preference pattern, $\varphi_B$, correlated with the observed preferences for the active user. Once this neighborhood of biclusters is identified, we followed the principles proposed by \cite{BCF-impactbiclusteringcf-Singh-2018} to create a personalized user-specific dataset. 

The procedure to obtain the personalized dataset consists in merging all the users and items of the bicluster neighborhood, corresponding to a full outer join of the subspaces in the neighborhood. %and obtaining the submatrix of the original U-I that corresponds to those rows and columns. 
For example, considering $B_3=(U_3=\{u_4,u_5\},I_3=\{i_3,i_5\})$ and $B_6=(U_6=\{u_6,u_8\},I_6=\{i_4,i_6\})$ in \autoref{fig:usbcfapproach}, the union set of users and items of this neighborhood is $(U_N = \{u_1, u_2, u_6, u_7, u_8\}, I_N = \{i_4, i_5, i_6\}$. %and the union set of the items is $ $. 
\autoref{fig:usbcfmergebiclusters} illustrates the process.

\begin{figure}[H]
    \centering
    \includegraphics[width=0.6\linewidth]{usbcfmergebiclusters.pdf}
    \caption{Example of aggregating two biclusters to create a new dataset.}
    \label{fig:usbcfmergebiclusters}
\end{figure}


Finally, the active user is added to the personalized rating dataset when he is not amongst the set of users of his own space. This step is crucial to accommodate available preferences from the active user and to guarantee that the subsequent application of Collaborative Filtering approaches can generate recommendations for his omissive preferences.

\subsection{Similarity between individual and subspace preferences}

To identify neighborhoods for personalized data creation, the subset of biclusters with preference patterns correlated with the preferences of an active user need to be identified. \cite{BCF-nearestbicsconstcoherent-Symeonidis-2008} introduced the concept of similarity between a bicluster and a user. However, their similarity is solely based on the rated items' interception, thus being unable to guarantee that the preferences of the active user are correlated with the preferences of the users in the biclusters. Understandably, this is an undesirable condition in the presence of binary, ordinal or numeric rating data.  

To address this limitation, we propose a new user-bicluster similarity that, besides the rated items' interception, also considers the values of ratings. %Below, we explain the details of our new similarity.
%In order to find the nearest biclusters to each system user, our approach uses a user-bicluster similarity. 
Following empirical evidence, this similarity is calculated through the product of two distinct scores which we refer to as \textbf{items' match} ($sim_{match}$) and \textbf{pattern fit} ($sim_{fit}$), respectively.
 
The \textbf{items' match score} captures the portion of bicluster items which were rated by the user, %and is calculated as in \autoref{eq:sim_match}:
 
\begin{equation}\label{eq:sim_match}
    \mathit{sim}_{\mathit{match}}(u,B_k) = \frac{|I_u \cap I_k|}{|I_k|},
\end{equation}
\vskip 0.1cm

\noindent where $I_u$ are the items rated by the user \textit{u} and $I_k$ are the items in bicluster $B_k$. Its values range between [0,1], with 1 representing the maximum similarity score between a user and bicluster. 
Understandably, a similarity entirely based on rated items' is insufficient as it disregards the values of the ratings. For instance, in \autoref{fig:bicactiveuserdifferent}, we have an example of bicluster and an active user that rating all the items in the bicluster. Using similarity criteria solely based on \autoref{eq:sim_match}, we would yield a maximum similarity. However, the preferences between the active user and the ones from the given bicluster are divergent, thus the inclusion of such bicluster in the neighborhood would hamper the subsequent CF task. %values of those ratings follow a completely different rating pattern. 
\begin{figure}[H]
    \centering
    \includegraphics[width=0.46\linewidth]{bicactiveuserdifference.pdf}
    \caption{Example of a bicluster and an active user with distinct rating pattern.}
    \label{fig:bicactiveuserdifferent}
\end{figure}

Given this, we introduce the \textbf{pattern fit score}, which is responsible for measuring the resemblance between the user rating pattern and the bicluster preference pattern (Def.~2.4). Given a user $u$ and a bicluster $B_k=(U_k,I_k)$, 

\begin{equation}\label{eq:sim_fit}
    \mathit{sim}_{\mathit{fit}}(u,B_k) = 1 - \frac{\mathit{RMSE}(\varphi_{B_k} , \varphi_{(u,I_u\cap I_k)} )}{r_{\max}-r_{\min}},
\end{equation}
\vskip 0.1cm

%\myworries{Aqui nao tenho a certeza do $u$, está coerente? representa os ratings do u?}

\noindent where $\varphi_{B_k}$ is the bicluster pattern (Def.~\ref{patterndef}), $\varphi_{(u,I_u\cap I_k)}$ is the user preferences along the items $I_k$ in the bicluster, $r_{max}$ ($r_{min}$) correspond to the maximum (minimum) observed rating, and RMSE corresponds to the root mean squared error. %\varphi_{ (I_u \cap I_b), J_{b}} , \varphi_{(I_u \cap I_b), u} )}{r_{\max}-r_{\min}}$ is 
In sum, the pattern fit scores corresponds to the normalized root-mean-squared differences between the preferences defined by the subspace pattern and the ratings of the user.

The similarity score defined in \autoref{eq:sim_fit} considers a rating error applied to biclusters with a constant coherent pattern. However, this similarity could be easily adapted to different coherence types, such as order-preserving, by adopting a Spearman's rank correlation coefficient or a Kendall rank correlation coefficient, instead of the root mean squared error, to take into account order preservation in the patterns.

The final similarity between a user and a biclusters is the simple product between the scores previously defined,

\vskip -0.1cm
\begin{equation}
    \mathit{sim}(u,B_k) = \mathit{sim}_{\mathit{match}} \times \mathit{sim}_{\mathit{fit}},
\end{equation}
\vskip 0.1cm

\noindent as revealed by empirical evidence from the experimental comparison of weighted sums, $\sum_i \alpha_i x_i$, and weighted products, $\prod_i x_i^{\alpha_i}$.%$  reveals the items's match score...xxxxx}

Both \textbf{items' match score} and \textbf{pattern fit score} values range between [0,1], so its product will also be in the same range. The product operation further establishes the relevance for both components to yield good levels for a bicluster to be seen as relevant for an active user. In other words, a poor value in one of the components will result in a low similarity score. %adequa, as  desirable condition 

A \textit{minSim} threshold is finally considered to filter the relevant biclusters for an active user. Only the biclusters with similarity with the user above the threshold are selected to produce its personalized dataset. 


%\subsubsection{Aggregating the Neighborhood}
%After selecting the most similar biclusters to each user, we aggregate the biclusters in each neighborhood in order to create a personalized dataset for each user. 



\subsection{Learning User-Specific Recommendation Models}
\label{coclussec}

Once the personalized datasets are created, a Collaborative Filtering algorithm is subsequently trained to create a unique recommendation model for each individual. Although the USBCF approach can be parameterized with any Collaborative Filtering approach, Item-based CF approaches are a particularly suitable option due to the inherent properties of the personalized data spaces. Despite their inherent simplicity, Item-based CF approaches are one of the form CF approaches and still recognized for producing state-of-the-art rating predictions. USBCF generates smaller and denser U-I matrices which, when allied with these approaches, allow the models to perform the recommendation tasks more effectively.

It is possible that the personalized data spaces do not cover the entire recommendation space. For instance, in \autoref{fig:usbcfapproach}, the personalized dataset does not contain items $i_1$ and $i_2$, originating a user-model with limited coverage capability, i.e. missing preferences for the items not included in the personalized data space cannot be estimated using the personalized. These situations occur when there are no shared local preferences containing those items, which poses a particularly difficult predictive setting irrespective of the selected CF approach. 

To address this challenge, it is possible to find subspaces with lower homogeneity, specifically an exhaustive partitioning of the original space into a set of subspaces that cover the entire itemset. Following the principles introduced by \cite{BiclustCF-scalablecf-george-2005}, USBCF relies on the previously proposed coclustering approach to estimate the typical small fraction of items falling into this condition, resorting to its default behavior for items included in the personalized data spaces.  

%In the case of adopting a biclustering algorithm which finds biclusters with missing values, the missings may be estimated 

%discovering a biclusters with missing values, these 

%\myworries{--- caso a técnica de biclustering permita encontrar bicluster com omissos, estes omissos devem ser estimados com base nos subespaços que os contêm}
