\section{Conclusions and Future Work}
\label{conclusions}

In this paper, we propose a user-specific CF approach that uses biclustering as a sub-step to improve traditional CF methods. %\myworries{recuperar contribuicoes do metodo }%
Principles for superior CF were introduced, including superior matching of active user preferences against preference patterns to account for preference diversity; formation of user-personalized data spaces to account for preference locality and sparsity; and secondary rating estimates from coclustering structures to yield high predictive coverage.

Experimental results on the popular MovieLens-100k and Movielens-1M benchmark datasets %with the USBCF approach. In particular, we studied the proposed approach, comparing it 
against state-of-the-art CF methods %biclustering-based CF. Experimental results 
show that USBCF yields competitive rating prediction errors, significantly improving classic memory-based CF approaches by alleviating the sparsity of the datasets. Moreover, the proposed approach outperforms the state-of-the-art biclustering-based approach (BBCF) in prediction quality while successfully overcoming BBCF's major limitation, the predictive coverage capability. 

This work displays the potentialities of the biclustering technique for recommendation purposes. Accordingly, we highlight potentially relevant research avenues for future work:

\begin{itemize}
    

    %\item  explore different QUBIC2's parameterization setups. The biclustering algorithm plays a significant role in the USBCF approach. In \autoref{sec:usbcf} we proposed the usage of QUBIC2 with a standard parameterization. However, we could, for instance, allow a controlled percentage of noisy elements to be included in the biclusters. By investigating how its parameterization affects the CF approach's quality, it may be possible to improve the USBCF methodology. 
    
    \item explore the impact that different coherence assumptions, degrees of noise tolerance, and biclustering searches yield on the USBCF approach. %Given the inhere \autoref{sec:usbcf} we used QUBIC2 as our biclustering algorithm for the USBCF approach. 
    As USBCF can be parameterized with any biclustering algorithm with minor modifications to the overall methodology, we aim at assessing the role of order-preserving preference patterns in CF recurring to BicPAMS \citep{BC-BicPams-Rui-2017}, quality relaxations (percentage of noisy and missing elements) \citep{henriques2014bicpam,BC-QUBIC2-2019}, and iterative search-and-masking procedures \citep{BC-BicPams-Rui-2017} to guarantee a better coverage of the original rating data space; %explore the same methodology on order-preserving biclusters can be an interesting experiment. 

    \item incorporate available background knowledge associated with the profile of users, as well as metadata related to the composition and characteristics of items, to guide the USBCF approach using knowledge-enriched biclustering searches \citep{BC-BIC2PAM-Rui-2016}; 

    \item extend USBCF to online settings. USBCF is as-is readily applicable to new users. Still, to handle preference drift along time, the underlying biclustering solutions used to create the personalized data spaces must be updated with newly arriving user-item ratings. To prevent the need to continuously retrain USBCF, we aim at developing an incremental version of USBCF based on updatable biclustering searches where new items and users can be added/remove to the subspaces in order to explore the inherently temporal context of user-item ratings. %thus improving . %adding new users and items to biclusters would improve the applicability of the approach to more real-world applications.
\end{itemize}

