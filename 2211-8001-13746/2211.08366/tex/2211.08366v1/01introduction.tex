\section{Introduction}
\label{introduction}



%In the current era, we are confronted with massive options and choice
Decision making plays a vital role in everyone's lives. %particularly given the increasing range of %Every single day we are confronted with 
%options and choices. for which our decisions determine the outcome of our lives. Effective decision-making can be difficult, especially in domains where the pool of options is massive. 
As an attempt to cope with massive range of options, there has been large academic and industry interest in automatically recommending items to individuals since last century. Spotify, Amazon, Netflix, and Facebook are some popular platforms that actively use recommender systems \citep{CF-Survey-Ekstrand-2011}. From e-commerce to online advertisement, these systems are unavoidable in our daily online journeys to suggest items in a personalized way. Collaborative Filtering (CF) approaches, firstly proposed by \cite{CF-UserBasedCF-Goldberg-1992}, are currently seen as the widest implemented and most mature of the technologies to build recommender systems. Given a set of observed item ratings, CF aims at estimating unknown preferences based on the assumption that users with similar preferences in the past will yield similar preferences in the future. Despite the role of Collaborative Filtering, significant challenges limit its effectiveness, including the diversity and locality of user preferences, the structural sparsity of user-item ratings, the subjectivity of rating scales, and the increasingly large user and item bases \citep{CF-Survey-Ekstrand-2011,Survey-CF-Su-2009}. %high item dimensionalit.
%\myworries{temos que no segundo parágrafo introduzir desafios e logo no terceiro clustering e biclustering}

%To address some of these challenges, dimensionality reduction approaches, such as , are commonly employed by model-based approaches 
To address the diversity of user profiles, reduce the dimensionality and minimize rating sparsity, 
matrix factorization and clustering approaches have been combined within CF approaches for two decades \citep{CF-Survey-Ekstrand-2011}. %In particular, %Cluster analysis or simply clustering is the process of partitioning a set of data objects into groups in accordance with object similarity. 
%Each group is a cluster, such that objects in a cluster are similar to one another yet dissimilar to objects in other clusters. %Clustering is fundamental in many applications, such as business intelligence, image pattern recognition, Web search, biology, and security \citep{Clust-Book-DataMiningConceptandTechnique-jiawei-2011}.
However, traditional clustering techniques are typically applied to either group users or items separately. %Moreover, when computing the clusters of objects, these algorithms use the entire dimension (all the columns or all the rows). 
In real-world CF scenarios, the preferences of a subset of users is frequently only significantly correlated on a subset of the overall items, and vice versa \citep{BC-Survey-Kelvin-2013}. \textit{Biclustering} performs clustering in two dimensions simultaneously, being able to find these local preference patterns that correspond to data subspaces (biclusters) \citep{BC-Survey-Sara-2004}. Although biclustering has been originally proposed in biomedical domains \citep{BC-geneexpression-Church-2000, BC-geneexpression-Sara-2010, BC-geneexpression-Gupta-2010}, %Nonetheless, although the massive impact biclustering is having in biological applications, 
it increasingly shows promising results in the recommendation domain \citep{BiclustCF-scalablecf-george-2005,BCF-impactbiclusteringcf-Singh-2018, BCF-improvetopnrec-Feng-2020}. 
In this context, \cite{BiclustCF-scalablecf-george-2005} introduced co-clustering as a tool to scale Collaborative Filtering. Despite efficient, their approach can incur in predictive errors associated with the difficulty in guaranteeing a strong homogeneity in the found sub-spaces (co-clusters). More recently, \cite{BCF-impactbiclusteringcf-Singh-2018} proposed a novel biclustering-based CF approach that outperformed state-of-the-art rating prediction approaches, however, yielding a small coverage of unknown rating estimates. 

To address these challenges, this work proposes %a novel biclustering-based collaborative filtering approach that uses biclustering to improve the traditional memory-based CF methods. The approach, named 
a novel CF approach, referred as \textit{User-specific Bicluster-based Collaborative Filtering} (USBCF), that identifies subspaces of shared preferences to create user-specific small and denser user-item matrices, which are then used to guide the training of traditional CF models. USBCF yields three major contributions of interest:
\begin{itemize}
    \item[--] state of the art biclustering searches for the discovery of high-coverage solutions of rating patterns with %strong homogeneity and 
    statistical significance guarantees; 
    \item[--] superior matching of user preferences against the found rating subspaces by taking into account both the matching extent and fitness of preferences;
    \item[--] higher predictive coverage without hampering predictive accuracy by placing primary rating estimates from flexible biclustering structures, followed by secondary rating estimates from checkboard coclustering structures.
\end{itemize}

%uses biclustering to find groups of users with similar preferences under a particular group of items (biclusters). Then, it uses these biclusters that . 
The proposed approach is assessed against baseline CF methods and %as well as the 
peer biclustering-based CF approaches using reference real-world rating data. %Collaborative Filtering} (BBCF) \citep{BCF-impactbiclusteringcf-Singh-2018}. 
The gathered results show that the proposed approach successfully surpasses the limited coverage of state-of-the-art biclustering-based approaches %BBCF %limitation of %the previously proposed state-of-the-art biclustering-based Collaborative Filtering (BBCF) since BBCF can only output predictions for a small subset of the system users and item (small coverage). 
and improves the rating prediction accuracy of traditional memory-based approaches.

The manuscript is organized as follows. Section 2 provides essential background on CF and Biclustering. Section 3 surveys relevant work on biclustering-based CF. Section 4 proposes USBCF, introducing the principles to address current shortcomings. Section 5 assesses USBCF against state-of-the-art peer approaches in real data, discussing its behavior. Finally, concluding remarks and future directions are drawn.



