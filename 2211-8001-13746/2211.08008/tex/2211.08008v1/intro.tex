\section{Introduction}\label{sec:intro}

Many safety-critical applications,
such as autonomous robots~\cite{zhu2021can},
self-driving~\cite{eykholt2018robust},
search engines~\cite{tolias2019targeted}, \etc{}
are becoming increasingly powerful and reliant
on deep neural networks (DNNs).
Despite the monumental success of DNNs
on these applications,
they are highly susceptible
to adversarial examples ---
an attacker can add tiny delibrate perturbations
to the input data,
misleading the model
into giving incorrect results~\cite{szegedy14,goodfellow15}.
Such adversarial attacks
could pose a significant threat
to the safety and reliability
of deep learning applications.

To mitigate this threat,
many defense strategies~\cite{madry18,zhang19trades,carmon19}
based on adversarial training~\cite{madry18}
have been proposed
to improve model robustness.
Adversarial training, however,
gains robustness at the expense of model accuracy
on clean natural images~\cite{tsipras2018robustness}.
Ensemble defenses~\cite{
    pang2019adp,kariyappa2019gal,yang2020dverge,yang2021trs}
have thus emerged
to combine multiple predictions
from independent sub-models.
The intuition is
that an ensemble of models
can often lead to higher accuracy,
while learning to stop
adversarial example transfer among sub-models
may improve robustness against adversarial attacks.
This approach could potentially
offer a promising research direction
to improve model robustness
while preserving high accuracy on natural inputs.

\input{figs/motivation}
\vspace{-0.1cm}
\section{Motivation}
High-level algorithmic design decisions such as batch size, parallelism strategy and degrees of parallelism stress the underlying hardware components in different ways. %To maximize the underlying hardware utilization, different hardware designs are required. 
One important metric that guides a balanced compute-memory design is computation intensity. Computation intensity is a workload property defined as the ratio of the number of computation flops to number of accesses to main memory.
%This ratio dictates the optimal ratio of computation throughput to memory bandwidth in the underlying hardware accelerator.

Figure~\ref{fig:compint} (left) shows the computation intensity distribution across different number of GPUs. We performed this analysis for a GEMM problem of size $(64K, 64K, 64K)$ distributed across many GPUs. Depending on the parallelism strategy and number of available GPUs, each GPU gets a non-regular matrix shards for compute.
Each boxplot shows the spread of computation intensity for different number of GPUs.
For each level of parallelism, we see a large spread of compute intensities, particularly for lower parallelism degrees. This is the result of different parallelization strategies as well as different tiling strategies.
It is clear from this figure that computation intensity is much smaller at higher degrees of parallelism, implying the need for a different design point. 

%Besides the parallelism degree, the choice of parallelization strategy has a direct impact on the utilization of the underlying hardware.
There are a myriad of ways to parallelize a model across a large multi-node system.
Figure~\ref{fig:compint}(right) shows the distribution of computation intensity across different parallelization strategies for a fixed level of parallelism (64K GPUs). 
On the X-axis, we show various parallelization strategies across 64K GPUs. 
RC or CR refers to Row-Column or Column-Row distributed GEMM (a.k.a kernel parallelism, more details in Section~\ref{subsec:par_strategy}).
%It is clear from the figure that computation intensity is different across different parallelization strategies,  implying 
As shown, optimal design point is different for different parallelization strategies.
%%High-level parallelism not only controls the design of the hardware nodes but also the network connecting them together. 
%One metric that guides the optimal design point at network level is the ratio of the number of bytes transferred to far memory over network to near memory. 
%We refer to this metric as communication intensity. Figure~\ref{fig:comint}(bottom) shows the distribution of communication intensity across different degrees of parallelism.
%
%Parallelization not only influences the design of the accelerator node but also the network infrastructure that connects them together. 
%This includes network topology and network bandwidth.
%For a given network topology, the metric that guides the bandwidth decision is the amount of data bytes transferred from one node to the next. 
%The data bytes transferred over the network from one node to the next vary by levels of parallelism and the type of parallelism strategy (RC vs CR).
%
%Therefore, the optimal hardware architecture strongly depends on high-level algorithmic and software decisions. Hence, an exploration framework is needed that would allow one to obtain and analyze the various possible optimal system and algorithm design combinations. Furthermore, 
%
%These results indicate dependency between high-level algorithmic design decisions and low-level hardware design, hence the need for a cross-stack co-design. 

Large training workloads are rapidly becoming the applications driving massive investments in semiconductor technology development all the way down to fabrication equipment, making such a cross-layer pathfinding framework immensely valuable to ML engineers, system architects and technology developers alike. 

\vspace{-0.1cm}
\section{Motivation}
High-level algorithmic design decisions such as batch size, parallelism strategy and degrees of parallelism stress the underlying hardware components in different ways. %To maximize the underlying hardware utilization, different hardware designs are required. 
One important metric that guides a balanced compute-memory design is computation intensity. Computation intensity is a workload property defined as the ratio of the number of computation flops to number of accesses to main memory.
%This ratio dictates the optimal ratio of computation throughput to memory bandwidth in the underlying hardware accelerator.

Figure~\ref{fig:compint} (left) shows the computation intensity distribution across different number of GPUs. We performed this analysis for a GEMM problem of size $(64K, 64K, 64K)$ distributed across many GPUs. Depending on the parallelism strategy and number of available GPUs, each GPU gets a non-regular matrix shards for compute.
Each boxplot shows the spread of computation intensity for different number of GPUs.
For each level of parallelism, we see a large spread of compute intensities, particularly for lower parallelism degrees. This is the result of different parallelization strategies as well as different tiling strategies.
It is clear from this figure that computation intensity is much smaller at higher degrees of parallelism, implying the need for a different design point. 

%Besides the parallelism degree, the choice of parallelization strategy has a direct impact on the utilization of the underlying hardware.
There are a myriad of ways to parallelize a model across a large multi-node system.
Figure~\ref{fig:compint}(right) shows the distribution of computation intensity across different parallelization strategies for a fixed level of parallelism (64K GPUs). 
On the X-axis, we show various parallelization strategies across 64K GPUs. 
RC or CR refers to Row-Column or Column-Row distributed GEMM (a.k.a kernel parallelism, more details in Section~\ref{subsec:par_strategy}).
%It is clear from the figure that computation intensity is different across different parallelization strategies,  implying 
As shown, optimal design point is different for different parallelization strategies.
%%High-level parallelism not only controls the design of the hardware nodes but also the network connecting them together. 
%One metric that guides the optimal design point at network level is the ratio of the number of bytes transferred to far memory over network to near memory. 
%We refer to this metric as communication intensity. Figure~\ref{fig:comint}(bottom) shows the distribution of communication intensity across different degrees of parallelism.
%
%Parallelization not only influences the design of the accelerator node but also the network infrastructure that connects them together. 
%This includes network topology and network bandwidth.
%For a given network topology, the metric that guides the bandwidth decision is the amount of data bytes transferred from one node to the next. 
%The data bytes transferred over the network from one node to the next vary by levels of parallelism and the type of parallelism strategy (RC vs CR).
%
%Therefore, the optimal hardware architecture strongly depends on high-level algorithmic and software decisions. Hence, an exploration framework is needed that would allow one to obtain and analyze the various possible optimal system and algorithm design combinations. Furthermore, 
%
%These results indicate dependency between high-level algorithmic design decisions and low-level hardware design, hence the need for a cross-stack co-design. 

Large training workloads are rapidly becoming the applications driving massive investments in semiconductor technology development all the way down to fabrication equipment, making such a cross-layer pathfinding framework immensely valuable to ML engineers, system architects and technology developers alike. 

\vspace{-0.1cm}
\section{Motivation}
High-level algorithmic design decisions such as batch size, parallelism strategy and degrees of parallelism stress the underlying hardware components in different ways. %To maximize the underlying hardware utilization, different hardware designs are required. 
One important metric that guides a balanced compute-memory design is computation intensity. Computation intensity is a workload property defined as the ratio of the number of computation flops to number of accesses to main memory.
%This ratio dictates the optimal ratio of computation throughput to memory bandwidth in the underlying hardware accelerator.

Figure~\ref{fig:compint} (left) shows the computation intensity distribution across different number of GPUs. We performed this analysis for a GEMM problem of size $(64K, 64K, 64K)$ distributed across many GPUs. Depending on the parallelism strategy and number of available GPUs, each GPU gets a non-regular matrix shards for compute.
Each boxplot shows the spread of computation intensity for different number of GPUs.
For each level of parallelism, we see a large spread of compute intensities, particularly for lower parallelism degrees. This is the result of different parallelization strategies as well as different tiling strategies.
It is clear from this figure that computation intensity is much smaller at higher degrees of parallelism, implying the need for a different design point. 

%Besides the parallelism degree, the choice of parallelization strategy has a direct impact on the utilization of the underlying hardware.
There are a myriad of ways to parallelize a model across a large multi-node system.
Figure~\ref{fig:compint}(right) shows the distribution of computation intensity across different parallelization strategies for a fixed level of parallelism (64K GPUs). 
On the X-axis, we show various parallelization strategies across 64K GPUs. 
RC or CR refers to Row-Column or Column-Row distributed GEMM (a.k.a kernel parallelism, more details in Section~\ref{subsec:par_strategy}).
%It is clear from the figure that computation intensity is different across different parallelization strategies,  implying 
As shown, optimal design point is different for different parallelization strategies.
%%High-level parallelism not only controls the design of the hardware nodes but also the network connecting them together. 
%One metric that guides the optimal design point at network level is the ratio of the number of bytes transferred to far memory over network to near memory. 
%We refer to this metric as communication intensity. Figure~\ref{fig:comint}(bottom) shows the distribution of communication intensity across different degrees of parallelism.
%
%Parallelization not only influences the design of the accelerator node but also the network infrastructure that connects them together. 
%This includes network topology and network bandwidth.
%For a given network topology, the metric that guides the bandwidth decision is the amount of data bytes transferred from one node to the next. 
%The data bytes transferred over the network from one node to the next vary by levels of parallelism and the type of parallelism strategy (RC vs CR).
%
%Therefore, the optimal hardware architecture strongly depends on high-level algorithmic and software decisions. Hence, an exploration framework is needed that would allow one to obtain and analyze the various possible optimal system and algorithm design combinations. Furthermore, 
%
%These results indicate dependency between high-level algorithmic design decisions and low-level hardware design, hence the need for a cross-stack co-design. 

Large training workloads are rapidly becoming the applications driving massive investments in semiconductor technology development all the way down to fabrication equipment, making such a cross-layer pathfinding framework immensely valuable to ML engineers, system architects and technology developers alike. 

Yet surprisingly,
under the white-box threat model,
existing state-of-the-art (SOTA)
adversarial attacks
with strong performance
on conventional DNN models
performed poorly on ensemble models,
sizeably overestimating their robustness
(\Cref{fig:scatter/gal,fig:scatter/dverge}).
This also suggests, to some extent,
that ensemble defenses
may rely on two potential design flaws below
that cause obfuscated gradients~\cite{athalye2018obfuscated},
\ie{}, they are either deliberately non-differentiable,
or give no useful gradients,
thus inducing overestimated robustness:

(a) \emph{%
    Gradient obfuscation
    via ensemble-forming strategy.}
They typically form ensembles
by averaging probability vectors (softmax)
of sub-models,
and softmax operations
can easily cause gradient obfuscation.
% Their robustness results were self-reported
% by attacking this output as a loss function,
% which has a relatively flat loss surface,
% and can impede gradient-based attacks.
While the model's actual robustness
is pertinent to the strategy
used to form an ensemble,
this indicates
that gradient-based attacks
have to \emph{also} leverage this effectively.

(b) \emph{Gradient diversification.}
Motivated by the reasoning
that a majority of sub-models
may need to be fooled
for successful attacks,
they learn to reduce adversarial transferability
among sub-models,
often via gradient diversification.
This intuitively
causes sub-models to counteract each other,
averaging to a small or inaccurate overall gradient.
% even though individual sub-models are weak defenders.
% However,
% as shown in this paper,
% existing attacks
% are not effective in evaluating ensembles
% created in this manner,
% even though they are not robust.
Attacking only the ensemble loss
would fool most sub-models,
but the ensemble may remain still correct;
conversely, it is actually possible to fool an ensemble,
despite the majority of its sub-models
giving correct predictions (\Cref{fig:num_models/logits}).
% Creating an effective attack
% could thus require reweighing the importance
% of individual sub-model.

From the above observations,
it is perceivable that
the practical evaluation of ensemble robustness
cannot be solely done
by treating such models holistically.
To this end,
this paper introduces \attack,
\underline{mo}del-\underline{r}eweighing \underline{a}ttack,
to adaptively adjust the importance
of sub-models in attack iterations.
Sub-models are reweighed
according to their respective ``ease of attack'',
which is in turn evaluated
by the gradient of the difference
of ensemble classification outputs
\wrt{} the ones of individual sub-models.
Pushing the limits of the current SOTA
in ensemble robustness evaluation,
it draws inspiration
from recent effective attack tactics,
\eg, momentum~\cite{dong18momentum,croce20aa},
step size schedule~\cite{croce20aa,ye2022aaa},
% random restarts~\cite{tramer2020adaptive}
loss normalization~\cite{lafeat},
and multiple targets~\cite{croce20aa,tramer2020adaptive}.
We summarize our contributions:
\begin{itemize}
    \item This paper presents the first extensive study
    on the robustness of ensemble defenses
    under multiple ensemble-forming strategies.

    \item By reweighing the importance weights of sub-models
    to steer adversarial example synthesis,
    we show that
    gradient-based attacks on ensemble defenses
    can often be orders of magnitude faster,
    while enjoying a higher success rate.

    % To evade these obstacles,
    % we introduce how individual sub-model gradients
    % a new surrogate loss \( \attackloss \),
    % is introduced
    % which bypass the gradient obfuscation
    % posed by the original ensemble loss.

    \item
    Empirical results
    on a wide variety
    of different ensemble defenses
    show that \attack{}
    outperforms competing attacks
    in both performance and convergence rate.
    Finally, this paper
    provides extensive ablation of its components
    and sensitivity analyses of hyperparameters.
\end{itemize}

To our best knowledge,
\attack{}
is currently the strongest attack
against a wide range of ensemble defenses.
% As existing ensemble defenses
% may rely on design decisions
% that stymie even the strongest competing attacks
% we tested,
% attacks on ensemble robustness
% should not treat the model under attack
% as a holistic loss function.
We make \attack{} open source
with reproducible results and pre-trained models;
moreover,
we maintain a leaderboard of ensemble defenses
under various attack strategies.
%\footnote{%
%    \url{https://github.com/lafeat/mora}.}.
% namely averaging by probability vectors (softmax),
% majority vote (voting)
% and averaging by logits (logits).
