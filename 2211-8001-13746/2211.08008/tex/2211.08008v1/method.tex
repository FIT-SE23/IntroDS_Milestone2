\section{%
    The Model-Reweighing Attack (\attack)
}\label{sec:method}

\subsection{%
    Problem Formulation \& High-Level Overview
}\label{sec:method:overview}

As discussed in \Cref{sec:intro},
existing ensemble defenses
may obfuscate gradients
with the ensemble-forming mode
and gradient diversification,
such that the final loss of the ensemble model
can no longer provide effective signals
for gradient descent.
It is therefore
desirable to find an alternative \( \loss \)
to the original SCE loss \( \sceloss \)
on the ensemble,
such that for a given number of iterations \( I \),
the original \( \sceloss \) loss can be maximized:
\begin{equation}
    \textstyle
    \begin{aligned}
    &{\max}_{\loss} \sceloss\parens{\xadv_I, y}
    \quad \text{where} \quad
    \xadv_{0} = \project\parens{\x + \m},\, \\
    &\quad\xadv_{i + 1} = \mathrm{PGD}\parens{
        \loss\parens{
            \f{E}\parens{\xadv_i}, \f{[1:M]}\parens{\xadv_i},
        y}
    },
    \end{aligned}
\end{equation}
and \( \mathrm{PGD}\parens{\cdot} \)
denotes a PGD step
along the gradient of loss function \(
    \loss\parens{
        \f\E\parens{\xadv_i},
        \f{[1:M]}\parens{\xadv_i},
    y}
\),
which not only takes the ensemble predictions
\( f_E\parens{\xadv_i} \) as input,
but can further utilize sub-model predictions
\( \f{[1:M]} \)
to guide the PGD iterations.
The challenge at hand
is, therefore,
to find a suitable \( \loss \)
which can generate attacks
on ensemble defenses efficiently and effectively.

% \subsection{High-Level Overview}

\input{figs/overview.tex}
%\vspace{-0.4cm}
\section{DeepFlow Overview}\label{sec:overview}

Figure~\ref{fig:overview} shows an overview of the \name framework. \name takes the following set of \textbf{inputs}: 
%
(1) \underline{System} design hierarchy (e.g., the number of accelerator nodes per device, the number of devices in the system, the network topology connecting nodes within a device and across the devices), 
(2) \underline{Architecture template} of each accelerator node which provides a high-level definition of its components and how those components fit together. The purpose of the template is to provide a blueprint for the accelerator without committing to any specific hardware parameters.
%A component definition (e.g., minimal compute units (MCU\footnote{Examples of what we regard as MCU includes SMU in older GPUs, Tensor Cores in newer GPUs or systolic array in TPUs}), memory hierarchy, network), specification of each component (e.g., flop rate for each MCU, MCU dimensions, number of MCUs sharing a set of register files, dataflow execution model, and characteristics and scope of different levels of memory hierarchy), 
(3) \underline{Technology} parameters for each hardware component (e.g. energy per flop), 
(4) \underline{Design budgets} for each hardware component (area, power, perimeter),  
(5) \underline{Machine learning model} specification in the form of a high-level compute graph, parameters of each compute node (kernel type, tensor dimensions), and
(6) \underline{Parallelism strategy} (data, model, kernel, and/or pipeline parallelism dimensions) which distributes the compute graph across the entire system. 
(7) \underline{Device mapping} strategy which defines mapping of parallel shards onto hardware nodes.
Given these inputs, \name predicts the end-to-end performance of one iteration (i.e., single batch) of the model and finds an optimal hardware-software-technology design point as \textbf{output}. 

DeepFlow is composed of two major components.
\underline{CrossFlow} which operates in a stand-alone mode and can predict performance for any input configuration; and a search and optimization engine (\underline{SOE}) which enables design space search. 
%To do so, \name breaks the problem into multiple phases.
%Each phase or building block of \name is described in details next.
\vspace{-0.1cm}
\subsection{CrossFlow Building Blocks}

\paragraph*{\em Micro-Architecture Generator Engine (AGE)}

AGE takes the following set of \textbf{inputs}:
(1) Design constraints (i.e the power, area and perimeter budget and breakdown across micro-architectural components such as cache, network, compute cores). 
This breakdown can be provided manually by users or automatically by the Search and Optimization Engine (SOE, Section~\ref{subsec:soe}).
%We also provide technology specifications such as 
%and their physical characteristics such as area/power per core under nominal operating conditions, SRAM/register characteristics. 
(2) Technology parameters such as energy per flop, energy per data bit transfer for each level of memory and network hierarchy, threshold and maximum gate voltage, integration substrate parameters such as bump/interconnect pitch. We provide a wide range of standard and future technology libraries as baseline. (3) Architecture template which is a blueprint of the underlying accelerator chip without committing to any specific hardware parameters. Given these input, AGE performs a frequency-voltage-area scaling optimization to generate the following \textbf{output} parameters such that design budgets for all component are met: 
(1) Compute throughput.
(2) Capacity for different levels of memory hierarchy.
(3) Bandwidth to each level of memory hierarchy.
(4) Inter-node as well as intra-node network bandwidth. 
These parameters are then utilized by the performance prediction engine (PPE) to estimate the execution time of each kernel.
%As mentioned previously, 
%The output of this stage is the input to performance engine to estimate the execution time of each kernel. Next, we describe the search and optimization engine (SOE) which feeds input values to AGE, if we want to use the model for architecture search.
%\vspace{-0.2cm}
\paragraph*{\em Compute Graph Transformation and Device Placement Engine (DPE)}
The parallelization strategy and device mapping are critical in deciding the overall execution time. Here, we first transform the model graph to a `super-graph' to reflect the parallelization strategy provided by the users manually, or SOE engine (Section~\ref{subsec:soe}) automatically. For example, to apply data parallelism, the model graph is replicated and appropriate edges are added to model the gradient exchange. After generating the transformed graph, DPE assigns the vertices of the transformed graph to the system nodes following a heuristic approach to minimize the communication overhead. %
%The details are presented in section~\ref{}.

%\vspace{-0.2cm}
\paragraph*{\em Performance Prediction Engine (PPE)}
%With the device mapping for all the vertices of the compute (super-)graph known, the next step is to calculate the overall execution time for a forward pass and/or a backward pass. 
We use hierarchical roofline modeling to predict the performance of each compute node. To calculate the overall end-to-end execution time, while respecting scheduling constraints (e.g. one kernel at a time per GPU, or prioritizing one kernel launch over another) we use event-driven simulation.%
%We explain the details of the PPE in section~\ref{}.
\subsection{Search and Optimization Engine (SOE)}\label{subsec:soe}
Co-optimizing micro-architectural parameters and the parallelization strategy that minimizes the overall end-to-end execution time requires navigating a large space of design parameters. 
Search and optimization engine (SOE) enables the automatic design space search and finds an 
%that meets the total power and area constraints, and simultaneously explores software parallelization strategies to find the 
optimal design point which meets the design constraints and minimizes the overall execution time.
%Because the hardware configuration space is very large, the search algorithm we designed 
SOE takes inspiration from ML-assisted search algorithms, in particular gradient decent search with momentum and builds on top of the CrossFlow modeling engine.
%The software parallelization design space is much smaller compared to the hardware design space and therefore we employ an exhaustive grid search. 

%Gradient search is an iterative process. In each step, SOE takes the predicted time from previous iteration as input to re-adjust the following parameter settings: (1) power, area and perimeter breakdown across different architectural components. (2) a parallelization strategy. These parameters will be fed back to CrossFlow to estimate the overall execution time. This process continues until convergence or user-specified number of steps. 
%The details of SOE's search algorithm are elaborated in Section~\ref{}. 
\vspace{-0.2cm}
\subsection{Parallelism Strategy Space}
\label{subsec:par_strategy}
There are a myriad of ways to parallelize a model across a large multi-node system. Exploring the parallelism space and finding the optimal strategy is critical to overall performance and system utilization. DeepFlow explores kernel, data and layer parallelism. It uniquely identifies each parallelism strategy by following notations: $\texttt{RC-\{KP1\}-\{KP2\}-d\{DP\}-p\{LP\}}$ or $\texttt{CR-\{KP1\}-d\{DP\}-p\{LP\}}$ depending on the choice of kernel parallelism.
RC (Row-Column) and CR (Column-Row) refer to different forms of kernel parallelism, i.e. distributed GEMM through inner-product or outer-product implementation.
%\begin{equation*}
%    \texttt{RC: R{KP1\}\_C\{KP2\}\_d\{DP\}\_p\{LP\}}
%\end{equation*}
%Where \texttt{RC} or \texttt{CR} refers to the type of kernel parallelism strategy, i.e. Row-Column or Column-Row,
%\texttt{N} refers to the number of parallel nodes,
\texttt{KP1} and \texttt{KP2} are the parameters of distributed GEMM. 
For Row-Column (\texttt{RC}) or inner-product, \texttt{KP1} and \texttt{KP2} would refer to the number of ways we shard the first matrix across rows and the second matrix across columns.
For Column-Row (\texttt{CR}) or outer-product, we would only need one parameter to specify the parallelization strategy; \texttt{KP1} will refer to the number of ways we cut the first matrix across columns and the second matrix across rows.
\texttt{DP} represents the number of model replicas and data shards assigned to each to exploit data parallelism.
\texttt{LP} is the number of ways we cut layers into stages to exploit pipeline parallelism.

\begin{comment}
\subsection{Modes of Operation}
\name has two modes of operation, standalone performance estimation mode and a architecture search mode.
\paragraph{Standalone Performance (SP) Estimation Mode}
Often ML practitioners or hardware designers want to estimate the performance of a model on a particular system configuration. For example, what is the cost optimal number of accelerators that one should deploy for distributed training? Or what is the estimated performance gain from choosing an accelerator with costlier HBM2E vs HBM2? To enable one to quickly answer such questions and to estimate performance under certain known system configurations, the tool can be run in the SP mode. 

In this mode, the description of the architecture of a scale-out system consisting of multiple accelerators, the architecture of the accelerator hardware themselves and the description of the neural network is taken as input, and fed into CrossFlow, which calculates the execution time of each training step. 

%In this mode, the description of the architecture of a scale-out system consisting of multiple accelerators, the architecture of the accelerator hardware themselves and the description of the neural network is taken as input. The tool calculates the execution time of each training step. 

%In this mode, the user 
%has the flexibility to use either just the \perfE or use \perfE alongside the AGE. While using just the \perfE  alone, the user needs to provide the architectural parameters of the tiles and the system. On the other hand, while using AGE  alongside \perfE, the user 
%needs to define the technology parameters and the hardware constraints i.e., the overall area and power breakdown among the different architectural components of the system. T

%In this mode, the tool generates the micro-architectural parameters of the accelerator chip using the AGE. It then runs the compute graph transformation and the device placement engine, and uses the \perfE to predict the execution time. 

\subsubsection{Architecture Search (AS) Mode}

The insatiable demand to run large models in the shortest possible time demands that we find the optimal hardware and software design points to train these models. From the hardware perspective, it is about finding the right micro-architecture as well as the overall system architecture of the distributed system. 
From the software perspective, it is about finding the right parallelization strategy. 
Often these decisions depend on each other, and so finding the optimal design points across the stack means 
navigating a large design space.

As one can imagine, the design space of the inputs to the tool is large and iterating over the entire design space is a tedious task. To efficiently search over the input space to find the optimal hardware constraints and parallelization strategy, the tool can be run in the AS mode. 
In this mode, the SOE module is used. The user will not need to provide the exact hardware parameters and the parallelization strategy. Only the architecture template and the initial compute graph will need to be provided as input to the tool. The tool then performs a search over the design space to find the optimal parameter settings that results in minimum training time. 
%We used gradient descent algorithm (details in Section~\ref{}) for this search.

%\subsection{Inputs and Outputs}

%\paragraph{SP-Mode}
%In this mode, the hardware 

%\paragraph{AS-Mode}

\end{comment}

%\vspace{-0.4cm}
\section{DeepFlow Overview}\label{sec:overview}

Figure~\ref{fig:overview} shows an overview of the \name framework. \name takes the following set of \textbf{inputs}: 
%
(1) \underline{System} design hierarchy (e.g., the number of accelerator nodes per device, the number of devices in the system, the network topology connecting nodes within a device and across the devices), 
(2) \underline{Architecture template} of each accelerator node which provides a high-level definition of its components and how those components fit together. The purpose of the template is to provide a blueprint for the accelerator without committing to any specific hardware parameters.
%A component definition (e.g., minimal compute units (MCU\footnote{Examples of what we regard as MCU includes SMU in older GPUs, Tensor Cores in newer GPUs or systolic array in TPUs}), memory hierarchy, network), specification of each component (e.g., flop rate for each MCU, MCU dimensions, number of MCUs sharing a set of register files, dataflow execution model, and characteristics and scope of different levels of memory hierarchy), 
(3) \underline{Technology} parameters for each hardware component (e.g. energy per flop), 
(4) \underline{Design budgets} for each hardware component (area, power, perimeter),  
(5) \underline{Machine learning model} specification in the form of a high-level compute graph, parameters of each compute node (kernel type, tensor dimensions), and
(6) \underline{Parallelism strategy} (data, model, kernel, and/or pipeline parallelism dimensions) which distributes the compute graph across the entire system. 
(7) \underline{Device mapping} strategy which defines mapping of parallel shards onto hardware nodes.
Given these inputs, \name predicts the end-to-end performance of one iteration (i.e., single batch) of the model and finds an optimal hardware-software-technology design point as \textbf{output}. 

DeepFlow is composed of two major components.
\underline{CrossFlow} which operates in a stand-alone mode and can predict performance for any input configuration; and a search and optimization engine (\underline{SOE}) which enables design space search. 
%To do so, \name breaks the problem into multiple phases.
%Each phase or building block of \name is described in details next.
\vspace{-0.1cm}
\subsection{CrossFlow Building Blocks}

\paragraph*{\em Micro-Architecture Generator Engine (AGE)}

AGE takes the following set of \textbf{inputs}:
(1) Design constraints (i.e the power, area and perimeter budget and breakdown across micro-architectural components such as cache, network, compute cores). 
This breakdown can be provided manually by users or automatically by the Search and Optimization Engine (SOE, Section~\ref{subsec:soe}).
%We also provide technology specifications such as 
%and their physical characteristics such as area/power per core under nominal operating conditions, SRAM/register characteristics. 
(2) Technology parameters such as energy per flop, energy per data bit transfer for each level of memory and network hierarchy, threshold and maximum gate voltage, integration substrate parameters such as bump/interconnect pitch. We provide a wide range of standard and future technology libraries as baseline. (3) Architecture template which is a blueprint of the underlying accelerator chip without committing to any specific hardware parameters. Given these input, AGE performs a frequency-voltage-area scaling optimization to generate the following \textbf{output} parameters such that design budgets for all component are met: 
(1) Compute throughput.
(2) Capacity for different levels of memory hierarchy.
(3) Bandwidth to each level of memory hierarchy.
(4) Inter-node as well as intra-node network bandwidth. 
These parameters are then utilized by the performance prediction engine (PPE) to estimate the execution time of each kernel.
%As mentioned previously, 
%The output of this stage is the input to performance engine to estimate the execution time of each kernel. Next, we describe the search and optimization engine (SOE) which feeds input values to AGE, if we want to use the model for architecture search.
%\vspace{-0.2cm}
\paragraph*{\em Compute Graph Transformation and Device Placement Engine (DPE)}
The parallelization strategy and device mapping are critical in deciding the overall execution time. Here, we first transform the model graph to a `super-graph' to reflect the parallelization strategy provided by the users manually, or SOE engine (Section~\ref{subsec:soe}) automatically. For example, to apply data parallelism, the model graph is replicated and appropriate edges are added to model the gradient exchange. After generating the transformed graph, DPE assigns the vertices of the transformed graph to the system nodes following a heuristic approach to minimize the communication overhead. %
%The details are presented in section~\ref{}.

%\vspace{-0.2cm}
\paragraph*{\em Performance Prediction Engine (PPE)}
%With the device mapping for all the vertices of the compute (super-)graph known, the next step is to calculate the overall execution time for a forward pass and/or a backward pass. 
We use hierarchical roofline modeling to predict the performance of each compute node. To calculate the overall end-to-end execution time, while respecting scheduling constraints (e.g. one kernel at a time per GPU, or prioritizing one kernel launch over another) we use event-driven simulation.%
%We explain the details of the PPE in section~\ref{}.
\subsection{Search and Optimization Engine (SOE)}\label{subsec:soe}
Co-optimizing micro-architectural parameters and the parallelization strategy that minimizes the overall end-to-end execution time requires navigating a large space of design parameters. 
Search and optimization engine (SOE) enables the automatic design space search and finds an 
%that meets the total power and area constraints, and simultaneously explores software parallelization strategies to find the 
optimal design point which meets the design constraints and minimizes the overall execution time.
%Because the hardware configuration space is very large, the search algorithm we designed 
SOE takes inspiration from ML-assisted search algorithms, in particular gradient decent search with momentum and builds on top of the CrossFlow modeling engine.
%The software parallelization design space is much smaller compared to the hardware design space and therefore we employ an exhaustive grid search. 

%Gradient search is an iterative process. In each step, SOE takes the predicted time from previous iteration as input to re-adjust the following parameter settings: (1) power, area and perimeter breakdown across different architectural components. (2) a parallelization strategy. These parameters will be fed back to CrossFlow to estimate the overall execution time. This process continues until convergence or user-specified number of steps. 
%The details of SOE's search algorithm are elaborated in Section~\ref{}. 
\vspace{-0.2cm}
\subsection{Parallelism Strategy Space}
\label{subsec:par_strategy}
There are a myriad of ways to parallelize a model across a large multi-node system. Exploring the parallelism space and finding the optimal strategy is critical to overall performance and system utilization. DeepFlow explores kernel, data and layer parallelism. It uniquely identifies each parallelism strategy by following notations: $\texttt{RC-\{KP1\}-\{KP2\}-d\{DP\}-p\{LP\}}$ or $\texttt{CR-\{KP1\}-d\{DP\}-p\{LP\}}$ depending on the choice of kernel parallelism.
RC (Row-Column) and CR (Column-Row) refer to different forms of kernel parallelism, i.e. distributed GEMM through inner-product or outer-product implementation.
%\begin{equation*}
%    \texttt{RC: R{KP1\}\_C\{KP2\}\_d\{DP\}\_p\{LP\}}
%\end{equation*}
%Where \texttt{RC} or \texttt{CR} refers to the type of kernel parallelism strategy, i.e. Row-Column or Column-Row,
%\texttt{N} refers to the number of parallel nodes,
\texttt{KP1} and \texttt{KP2} are the parameters of distributed GEMM. 
For Row-Column (\texttt{RC}) or inner-product, \texttt{KP1} and \texttt{KP2} would refer to the number of ways we shard the first matrix across rows and the second matrix across columns.
For Column-Row (\texttt{CR}) or outer-product, we would only need one parameter to specify the parallelization strategy; \texttt{KP1} will refer to the number of ways we cut the first matrix across columns and the second matrix across rows.
\texttt{DP} represents the number of model replicas and data shards assigned to each to exploit data parallelism.
\texttt{LP} is the number of ways we cut layers into stages to exploit pipeline parallelism.

\begin{comment}
\subsection{Modes of Operation}
\name has two modes of operation, standalone performance estimation mode and a architecture search mode.
\paragraph{Standalone Performance (SP) Estimation Mode}
Often ML practitioners or hardware designers want to estimate the performance of a model on a particular system configuration. For example, what is the cost optimal number of accelerators that one should deploy for distributed training? Or what is the estimated performance gain from choosing an accelerator with costlier HBM2E vs HBM2? To enable one to quickly answer such questions and to estimate performance under certain known system configurations, the tool can be run in the SP mode. 

In this mode, the description of the architecture of a scale-out system consisting of multiple accelerators, the architecture of the accelerator hardware themselves and the description of the neural network is taken as input, and fed into CrossFlow, which calculates the execution time of each training step. 

%In this mode, the description of the architecture of a scale-out system consisting of multiple accelerators, the architecture of the accelerator hardware themselves and the description of the neural network is taken as input. The tool calculates the execution time of each training step. 

%In this mode, the user 
%has the flexibility to use either just the \perfE or use \perfE alongside the AGE. While using just the \perfE  alone, the user needs to provide the architectural parameters of the tiles and the system. On the other hand, while using AGE  alongside \perfE, the user 
%needs to define the technology parameters and the hardware constraints i.e., the overall area and power breakdown among the different architectural components of the system. T

%In this mode, the tool generates the micro-architectural parameters of the accelerator chip using the AGE. It then runs the compute graph transformation and the device placement engine, and uses the \perfE to predict the execution time. 

\subsubsection{Architecture Search (AS) Mode}

The insatiable demand to run large models in the shortest possible time demands that we find the optimal hardware and software design points to train these models. From the hardware perspective, it is about finding the right micro-architecture as well as the overall system architecture of the distributed system. 
From the software perspective, it is about finding the right parallelization strategy. 
Often these decisions depend on each other, and so finding the optimal design points across the stack means 
navigating a large design space.

As one can imagine, the design space of the inputs to the tool is large and iterating over the entire design space is a tedious task. To efficiently search over the input space to find the optimal hardware constraints and parallelization strategy, the tool can be run in the AS mode. 
In this mode, the SOE module is used. The user will not need to provide the exact hardware parameters and the parallelization strategy. Only the architecture template and the initial compute graph will need to be provided as input to the tool. The tool then performs a search over the design space to find the optimal parameter settings that results in minimum training time. 
%We used gradient descent algorithm (details in Section~\ref{}) for this search.

%\subsection{Inputs and Outputs}

%\paragraph{SP-Mode}
%In this mode, the hardware 

%\paragraph{AS-Mode}

\end{comment}

%\vspace{-0.4cm}
\section{DeepFlow Overview}\label{sec:overview}

Figure~\ref{fig:overview} shows an overview of the \name framework. \name takes the following set of \textbf{inputs}: 
%
(1) \underline{System} design hierarchy (e.g., the number of accelerator nodes per device, the number of devices in the system, the network topology connecting nodes within a device and across the devices), 
(2) \underline{Architecture template} of each accelerator node which provides a high-level definition of its components and how those components fit together. The purpose of the template is to provide a blueprint for the accelerator without committing to any specific hardware parameters.
%A component definition (e.g., minimal compute units (MCU\footnote{Examples of what we regard as MCU includes SMU in older GPUs, Tensor Cores in newer GPUs or systolic array in TPUs}), memory hierarchy, network), specification of each component (e.g., flop rate for each MCU, MCU dimensions, number of MCUs sharing a set of register files, dataflow execution model, and characteristics and scope of different levels of memory hierarchy), 
(3) \underline{Technology} parameters for each hardware component (e.g. energy per flop), 
(4) \underline{Design budgets} for each hardware component (area, power, perimeter),  
(5) \underline{Machine learning model} specification in the form of a high-level compute graph, parameters of each compute node (kernel type, tensor dimensions), and
(6) \underline{Parallelism strategy} (data, model, kernel, and/or pipeline parallelism dimensions) which distributes the compute graph across the entire system. 
(7) \underline{Device mapping} strategy which defines mapping of parallel shards onto hardware nodes.
Given these inputs, \name predicts the end-to-end performance of one iteration (i.e., single batch) of the model and finds an optimal hardware-software-technology design point as \textbf{output}. 

DeepFlow is composed of two major components.
\underline{CrossFlow} which operates in a stand-alone mode and can predict performance for any input configuration; and a search and optimization engine (\underline{SOE}) which enables design space search. 
%To do so, \name breaks the problem into multiple phases.
%Each phase or building block of \name is described in details next.
\vspace{-0.1cm}
\subsection{CrossFlow Building Blocks}

\paragraph*{\em Micro-Architecture Generator Engine (AGE)}

AGE takes the following set of \textbf{inputs}:
(1) Design constraints (i.e the power, area and perimeter budget and breakdown across micro-architectural components such as cache, network, compute cores). 
This breakdown can be provided manually by users or automatically by the Search and Optimization Engine (SOE, Section~\ref{subsec:soe}).
%We also provide technology specifications such as 
%and their physical characteristics such as area/power per core under nominal operating conditions, SRAM/register characteristics. 
(2) Technology parameters such as energy per flop, energy per data bit transfer for each level of memory and network hierarchy, threshold and maximum gate voltage, integration substrate parameters such as bump/interconnect pitch. We provide a wide range of standard and future technology libraries as baseline. (3) Architecture template which is a blueprint of the underlying accelerator chip without committing to any specific hardware parameters. Given these input, AGE performs a frequency-voltage-area scaling optimization to generate the following \textbf{output} parameters such that design budgets for all component are met: 
(1) Compute throughput.
(2) Capacity for different levels of memory hierarchy.
(3) Bandwidth to each level of memory hierarchy.
(4) Inter-node as well as intra-node network bandwidth. 
These parameters are then utilized by the performance prediction engine (PPE) to estimate the execution time of each kernel.
%As mentioned previously, 
%The output of this stage is the input to performance engine to estimate the execution time of each kernel. Next, we describe the search and optimization engine (SOE) which feeds input values to AGE, if we want to use the model for architecture search.
%\vspace{-0.2cm}
\paragraph*{\em Compute Graph Transformation and Device Placement Engine (DPE)}
The parallelization strategy and device mapping are critical in deciding the overall execution time. Here, we first transform the model graph to a `super-graph' to reflect the parallelization strategy provided by the users manually, or SOE engine (Section~\ref{subsec:soe}) automatically. For example, to apply data parallelism, the model graph is replicated and appropriate edges are added to model the gradient exchange. After generating the transformed graph, DPE assigns the vertices of the transformed graph to the system nodes following a heuristic approach to minimize the communication overhead. %
%The details are presented in section~\ref{}.

%\vspace{-0.2cm}
\paragraph*{\em Performance Prediction Engine (PPE)}
%With the device mapping for all the vertices of the compute (super-)graph known, the next step is to calculate the overall execution time for a forward pass and/or a backward pass. 
We use hierarchical roofline modeling to predict the performance of each compute node. To calculate the overall end-to-end execution time, while respecting scheduling constraints (e.g. one kernel at a time per GPU, or prioritizing one kernel launch over another) we use event-driven simulation.%
%We explain the details of the PPE in section~\ref{}.
\subsection{Search and Optimization Engine (SOE)}\label{subsec:soe}
Co-optimizing micro-architectural parameters and the parallelization strategy that minimizes the overall end-to-end execution time requires navigating a large space of design parameters. 
Search and optimization engine (SOE) enables the automatic design space search and finds an 
%that meets the total power and area constraints, and simultaneously explores software parallelization strategies to find the 
optimal design point which meets the design constraints and minimizes the overall execution time.
%Because the hardware configuration space is very large, the search algorithm we designed 
SOE takes inspiration from ML-assisted search algorithms, in particular gradient decent search with momentum and builds on top of the CrossFlow modeling engine.
%The software parallelization design space is much smaller compared to the hardware design space and therefore we employ an exhaustive grid search. 

%Gradient search is an iterative process. In each step, SOE takes the predicted time from previous iteration as input to re-adjust the following parameter settings: (1) power, area and perimeter breakdown across different architectural components. (2) a parallelization strategy. These parameters will be fed back to CrossFlow to estimate the overall execution time. This process continues until convergence or user-specified number of steps. 
%The details of SOE's search algorithm are elaborated in Section~\ref{}. 
\vspace{-0.2cm}
\subsection{Parallelism Strategy Space}
\label{subsec:par_strategy}
There are a myriad of ways to parallelize a model across a large multi-node system. Exploring the parallelism space and finding the optimal strategy is critical to overall performance and system utilization. DeepFlow explores kernel, data and layer parallelism. It uniquely identifies each parallelism strategy by following notations: $\texttt{RC-\{KP1\}-\{KP2\}-d\{DP\}-p\{LP\}}$ or $\texttt{CR-\{KP1\}-d\{DP\}-p\{LP\}}$ depending on the choice of kernel parallelism.
RC (Row-Column) and CR (Column-Row) refer to different forms of kernel parallelism, i.e. distributed GEMM through inner-product or outer-product implementation.
%\begin{equation*}
%    \texttt{RC: R{KP1\}\_C\{KP2\}\_d\{DP\}\_p\{LP\}}
%\end{equation*}
%Where \texttt{RC} or \texttt{CR} refers to the type of kernel parallelism strategy, i.e. Row-Column or Column-Row,
%\texttt{N} refers to the number of parallel nodes,
\texttt{KP1} and \texttt{KP2} are the parameters of distributed GEMM. 
For Row-Column (\texttt{RC}) or inner-product, \texttt{KP1} and \texttt{KP2} would refer to the number of ways we shard the first matrix across rows and the second matrix across columns.
For Column-Row (\texttt{CR}) or outer-product, we would only need one parameter to specify the parallelization strategy; \texttt{KP1} will refer to the number of ways we cut the first matrix across columns and the second matrix across rows.
\texttt{DP} represents the number of model replicas and data shards assigned to each to exploit data parallelism.
\texttt{LP} is the number of ways we cut layers into stages to exploit pipeline parallelism.

\begin{comment}
\subsection{Modes of Operation}
\name has two modes of operation, standalone performance estimation mode and a architecture search mode.
\paragraph{Standalone Performance (SP) Estimation Mode}
Often ML practitioners or hardware designers want to estimate the performance of a model on a particular system configuration. For example, what is the cost optimal number of accelerators that one should deploy for distributed training? Or what is the estimated performance gain from choosing an accelerator with costlier HBM2E vs HBM2? To enable one to quickly answer such questions and to estimate performance under certain known system configurations, the tool can be run in the SP mode. 

In this mode, the description of the architecture of a scale-out system consisting of multiple accelerators, the architecture of the accelerator hardware themselves and the description of the neural network is taken as input, and fed into CrossFlow, which calculates the execution time of each training step. 

%In this mode, the description of the architecture of a scale-out system consisting of multiple accelerators, the architecture of the accelerator hardware themselves and the description of the neural network is taken as input. The tool calculates the execution time of each training step. 

%In this mode, the user 
%has the flexibility to use either just the \perfE or use \perfE alongside the AGE. While using just the \perfE  alone, the user needs to provide the architectural parameters of the tiles and the system. On the other hand, while using AGE  alongside \perfE, the user 
%needs to define the technology parameters and the hardware constraints i.e., the overall area and power breakdown among the different architectural components of the system. T

%In this mode, the tool generates the micro-architectural parameters of the accelerator chip using the AGE. It then runs the compute graph transformation and the device placement engine, and uses the \perfE to predict the execution time. 

\subsubsection{Architecture Search (AS) Mode}

The insatiable demand to run large models in the shortest possible time demands that we find the optimal hardware and software design points to train these models. From the hardware perspective, it is about finding the right micro-architecture as well as the overall system architecture of the distributed system. 
From the software perspective, it is about finding the right parallelization strategy. 
Often these decisions depend on each other, and so finding the optimal design points across the stack means 
navigating a large design space.

As one can imagine, the design space of the inputs to the tool is large and iterating over the entire design space is a tedious task. To efficiently search over the input space to find the optimal hardware constraints and parallelization strategy, the tool can be run in the AS mode. 
In this mode, the SOE module is used. The user will not need to provide the exact hardware parameters and the parallelization strategy. Only the architecture template and the initial compute graph will need to be provided as input to the tool. The tool then performs a search over the design space to find the optimal parameter settings that results in minimum training time. 
%We used gradient descent algorithm (details in Section~\ref{}) for this search.

%\subsection{Inputs and Outputs}

%\paragraph{SP-Mode}
%In this mode, the hardware 

%\paragraph{AS-Mode}

\end{comment}


\attack{} aims to provide
a potential optimization route
towards the above problem formulation.
Namely,
in addition to the original output
of the ensemble \( \f\E\parens{\xadv} \),
we leverage the sub-model predictions
\( \f{[1:M]}\parens{\xadv} \)
to facilitate the optimization.
By way of illustration,
\Cref{fig:method:overview}
shows a high-level overview
of the model-reweighing attack,
where we compliment the ensemble loss,
with a newly added sub-model reweighing loss
\( \attackloss \),
as an auxiliary attack vector
alongside the original objective.
Not only can the new loss
bypass the ensemble-forming strategy
to work around its obfuscated gradients,
but it further exploits information
present in the individual sub-model
and ensemble predictions
to steer the direction of adversarial example synthesis.

\subsection{%
    Adaptive Sub-model Importance
}\label{sec:method:weight}

Before we begin,
assume that
\( \logit{m} \triangleq f_m \parens{\x} \)
represents the \( \ordinal{m} \)
sub-model output,
and let \( \logit{m}_t \)
denote the corresponding logit of label \( t \).
We define the difference of logits (DL)~\cite{carlini17}
\( \dl{m} \triangleq \logit{m}_y - \logit{m}_\hy \),
which is the difference
between the predictions
of the ground truth \( \logit{m}_y \)
and the maximum of the remaining classes
\(
    \logit{m}_\hy
    \triangleq
    \max_{i \in \classset / y} \logit{m}_i
\),
where \( \classset / y \)
is the set of all class labels except \( y \).
Similarly,
we let \( \logit{\E} \triangleq f_\E\parens{\x} \),
and \( \logit{\E}_t \) and \( \dl{\E} \)
be the respective variants of the ensemble prediction.
It is notable that
a successful attack happens
when \( \dl\E < 0 \),
and similarly \( \dl{m} < 0 \)
means the \( \ordinal{m} \) sub-model
is producing incorrect classification.

Ensemble defenses
% such as GAL~\cite{kariyappa2019gal},
% ADP~\cite{pang2019adp} and TRS~\cite{yang2021trs}
tend to diversify sub-model gradients,
for instance, ADP~\cite{pang2019adp}
minimizes the cosine-similarity \(
    \angles{
        \nabla \ell\parens{\logit{a}},
        \nabla \ell\parens{\logit{b}}
    }
\) among each loss function gradient pairs
of sub-models \(
    \ell\parens{\logit{a}}
\) and \(
    \ell\parens{\logit{b}}
\).
Their intuition
is that it may lower transferability
among these sub-models,
such that attacks
with the overall gradient of the ensemble,
\ie{}, \(
    \nabla \ell\parens{\logit\E}
    = \frac1M \sum_{m \in [1:M]}
        \nabla \ell\parens{\logit{m}}
\),
are becoming less effective
in misleading all sub-models simultaneously,
as individual gradients in \(
    \nabla \ell\parens{\logit{m}}
\) are encouraged to be orthogonal to each other.
To this end,
we propose to reweigh the importance of sub-models,
by instead considering the modified gradient:
\begin{equation}
    \textstyle
    \widehat{\nabla \ell\parens{\logit\E}}
    = \frac1M \sum_{m \in [1:M]}
        { \imp{m}\parens{\logit{m}} }
        \nabla \ell\parens{\logit{m}},
\end{equation}
where
\( \imp{m} \)
assigns weights to important sub-models
to contribute more heavily
to the attack gradient.

While the adversarial examples of the ensemble
could present a challenge to discover,
individual sub-models
are weak defenders
which can be easily defeated.
Based on this property,
we propose to weigh sub-models importance
with the rate of change in \( \dl{\E} \)
\wrt{} that of \( \dl{m} \),
\ie{}, sub-models would be given higher weights
if attacking it would bring a significant change
to the ensemble's prediction.
Following this idea,
for all ensemble-forming strategies (softmax, voting, logits),
we rewrite \( \dl{\E} \)
as a function of \( \dl{m} \),
where the term below
can become a function of \( \dl{m} \):
\begin{equation}
    \textstyle
    \begin{aligned}
    \dl{\E} = \ensop\parens{\logit{m}}_\y -
    \ensop\parens{\logit{m}}_\hy
    &= \ensop\parens{\logit{m} - \logit{m}_\hy}_\y -
      \ensop\parens{\logit{m} - \logit{m}_\hy}_\hy \\
    &= \ensop\parens{\dl{m}, \cdots}_\y -
      \ensop\parens{\dl{m}, \cdots}_\hy
    \triangleq h_m\parens{\dl{m}}.
    \end{aligned}
\end{equation}
The weights are thus defined as follows:
\begin{equation}
    \textstyle
    \imp{m}\parens{\logit{m}}
    = \frac{\partial \dl{\E}\parens{\dl{m}}}{\partial \dl{m}}
    = \frac\partial{\partial\dl{m}}\parens*{
        {\frac1M} {\ssum}_{m\in[1:M]}
            \frac{\partial h_m\parens{\dl{m}}}{\partial \dl{m}}
    }
    = \frac1M \frac{\partial h_m\parens{\dl{m}}}{\partial \dl{m}}.
\end{equation}
While it is possible
to compute the weights using gradient back-propagation,
we can simply derive the following closed-form solution
of the weights
for each of the three ensemble-forming strategies.
For \( \wta \),
we use the softened version of \( \wta \)
as defined in~\eqref{eq:softwta}
and can derive the weights as follows:
\begin{equation}
    \newcommand{\sv}{\mathbf{s}}
    \imp{m}\parens{\logit{m}} =
    \indicator\bracks{\dl{m} > 0} \cdot
    \detach\parens*{
        {\textstyle\frac1{\tau M}}
        \sv_\hy \parens*{1 + \sv_\y - \sv_\hy}
    },
    \quad \text{where}\,\,
    \sv = \softmax\parens*{\nicefrac{\logit{m}}\tau}.
    \label{eq:weight}
\end{equation}
Here \( \indicator\bracks{\dl{m} > 0} \)
is the indicator function
that equals 1 if \( \dl{m} > 0 \),
or 0 otherwise,
effectively stopping the attack
on the \( \ordinal{m} \) sub-model upon success,
and the \( \detach \) operator
admits no backward propagation to its input.
In the case of using sums
of sub-model softmax outputs
to form an ensemble decision,
\ie{}, \( \ensop = \softmax \),
it is a special case of \( \softwta_\tau \)
where the temperature coefficient
can be fixed at \( \tau = 1 \).
% but we further searched \( \tau \)
% for optimality.
Finally,
when \( \ensop = \id \),
\ie{}, forming ensembles by summing logits,
\( \imp{m}\parens{\logit{m}} \)
simply reduces to \( \indicator\bracks{\dl{m} > 0} \)
for the \( \ordinal{m} \) sub-model.
%--------------------------------------------------------------------------------------
%--------------------------------------------------------------------------------------
\subsection{The algorithm}
%--------------------------------------------------------------------------------------

 {We} consider that $X$ follows the mixture model defined by \eqref{def:RMM} and consider $X_{1} , \ldots ,X_{n}$ i.i.d copies of $X$.  We now consider the "empirical fixpoint function", i.e we will consider, denoting $\tau = \left( \tau_{1} , \ldots , \tau_{k} \right)$, and $\tau_{k} = \left( \tau_{1,k} , \ldots , \tau_{n,k} \right)$,
%\SR{
%\begin{align*}
%& \hat{g}_{2,k} \left( \tau_{k} ,  m_{k} \right)= \frac{\sum_{i=1}^{n} \tau_{i,k} \frac{X_{i}}{\left\| X_{i} - m_{k} \right\|} }{\sum_{i=1}^{n} \tau_{i,k} \frac{1}{\left\| X_{i} - m_{k} \right\|}}\\
%& \hat{g}_{3,k} \left( \tau_{k} ,  m_{k} , V_{k} \right)= \frac{\sum_{i=1}^{n}\tau_{i,,k}\frac{\left( X_{i} - m_{k} \right) \left( X_{i} - m_{k} \right)^{T}}{\left\| \left( X_{i} - m_{k} \right) \left( X_{i} - m_{k} \right)^{T} - V_{k} \right\|_{F}}}{\sum_{i=1}^{n} \tau_{i,k} \frac{1}{\left\| \left( X_{i} - m_{k} \right) \left( X_{i} - m_{k} \right)^{T} - V_{k} \right\|_{F}}} .
%\end{align*}
%}
{
\begin{align*}
\hat{g}_{2,k} \left( \tau_{k} ,  m_{k} \right)
& = \left({\sum_{i=1}^{n} \frac{\tau_{i,k} X_{i}}{\left\| X_{i} - m_{k} \right\|} } \right) \left/ \left({\sum_{i=1}^{n} \frac{\tau_{i,k}}{\left\| X_{i} - m_{k} \right\|}} \right) \right. \\
\hat{g}_{3,k} \left( \tau_{k} ,  m_{k} , V_{k} \right)
& = \left( {\sum_{i=1}^{n}\frac{ \tau_{i,,k} \left( X_{i} - m_{k} \right) \left( X_{i} - m_{k} \right)^{T}}{\left\| \left( X_{i} - m_{k} \right) \left( X_{i} - m_{k} \right)^{T} - V_{k} \right\|_{F}}} \right) \left/ \left({\sum_{i=1}^{n} \frac{\tau_{i,k}}{\left\| \left( X_{i} - m_{k} \right) \left( X_{i} - m_{k} \right)^{T} - V_{k} \right\|_{F}}} \right) \right..
\end{align*}
}
This leads to the following algorithm:

\begin{algorithm}[Fix Point algorithm for Robust Mixture Model]
  Starting from $\phi^0 = (\pi^0, m^0, V^0)$, repeat until convergence:
  \begin{enumerate}
  \item Compute for each $1 \leq i \leq n$ and $1 \leq k \leq K$
  $$
  \tau_k^{h+1}(X_i) = \frac{\pi_{k}^{h}\phi_{m_{k}^{h}, \hat{\Psi}_{u} \left( V_{k}^{h} \right)} \left( X_{i} \right)}{\sum_{\ell = 1}^{K} \pi_{\ell}^{h}\phi_{m_{\ell}^{h}, \hat{\Psi}_{u} \left( V_{\ell}^{h} \right)} \left( X_{i} \right)} ,
  $$
  where $\hat{\Psi}_{U}$ is the application which enables, given $V_{k}$, to "rebuild" $\Sigma_{k}$ with the help of one of the method proposed in Section \ref{sec:variance}; 
  \item Based on the fix point relations (see Proposition \eqref{prop:fixpoint}), update, for each $1 \leq k \leq K$,
  $$
  \pi^{h+1}_k = \frac1n \sum_{i=1}^n \tau_k^{h+1}(X_i), \qquad
  m^{h+1}_k = \text{FixPoint}\left( \widehat{g}_{2k}(\tau_{k}^{h},.) \right), \qquad
  V^{h+1}_k =\text{FixPoint}\left( \widehat{g}_{3k}(\tau_{k}^{h},m_{k}^{h},.)\right).
  $$
  where $\text{FixPoint}\left(f(.) \right)$ denotes the fix point of the functional $f$.
  \end{enumerate}
\end{algorithm}
Note that estimating the fix points leads to estimate the weighted median and MCM considering weights $\tau_{k}^{h}$. More intuitively, this algorithm consists in updating $\tau_{i,k}$ replacing the empirical mean and variance of each class by their robust estimates based on the median and the MCM of each class, before updating $\pi$ (as usually).


%--------------------------------------------------------------------------------------
%--------------------------------------------------------------------------------------
\subsection{Choosing the number of clusters}
%--------------------------------------------------------------------------------------

To determine the number of clusters $K$, we resort to two standard penalized-likelihood criteria, namely BIC (\cite{Sch78}) and ICL (\cite{BCG00,MaP00}).
More specifically, denoting by $D_K$ the number of independent parameters involved in a mixture with $K$ clusters and by $\widehat{\mathcal{L}}_K(X)$ the log-likelihood of the dataset $X$ evaluated with the parameter estimates resulting from the proposed estimation procedure:
$$
\widehat{\mathcal{L}}_K(X) = \sum_{i=1}^n \log\left(\sum_{k=1}^K \widehat{\pi}_k \phi_{\widehat{\mu}_k, \widehat{\Sigma}_k}(X_i)\right), 
$$
we used
\begin{equation} \label{eq:modelSel}
  BIC(K) = \widehat{\mathcal{L}}_K(X) - \log(n) D_K/2, \qquad 
  ICL(K) = BIC(K) + \sum_{i=1}^n \sum_{k=1}^K \widehat{\tau}_{i, k} \log \widehat{\tau}_{i, k}.
\end{equation}
We remind that the additional penalty term in the ICL criterion corresponds to the entropy of the conditional distribution of the latent variables $\{Z_i\}_{1 \leq i \leq n}$, conditional on the observed ones $\{X_i\}_{1 \leq i \leq n}$.  {This additional penalty is supposed to favor clusterings with lower classification uncertainty.}

%--------------------------------------------------------------------------------------
\subsection{Initialization of the algorithm}
%--------------------------------------------------------------------------------------
 {
We considered two ways of initializing the algorithm:
\begin{enumerate}
\item[•] Use the robust hierarchical clustering proposed by \cite{gagolewski2016genie}, to get $\tau^{1}$, and run our algorithm from there ;
\item[•] Randomly choose $K$ centers from the data and take $\Sigma_{k} = I_{d}$ and $\pi_{k} = \frac{1}{K}$ for all $k$. 
\end{enumerate}
Remark that the later way can tried several times, so to keep initialization leading to the best final log-likelihood.
We may also use the two ways and keep the best result in term of log-likelihood. 
}
 
% %--------------------------------------------------------------------------------------
% \subsubsection{Initialization of the algorithm}
% %--------------------------------------------------------------------------------------
% Two way for initializing the algorithm are considered: 
% \begin{itemize}
% \item[•] One can initialize the algorithm considering the clustering given by the robust hierarchical clustering proposed by \cite{gagolewski2016genie}, which enables to have $\tau^{1}$, and one can run the end of the algorithm.
% \item[•] One can chose randomly $K$ centers from the data and take $\Sigma_{k} = I_{d}$ and $\pi_{k} = \frac{1}{K}$ for all $k$. Remark that this can be done for several random choice, and one can take the initialization leading to the best final log-likelihood.
% \end{itemize} 
% Remark that one chose these two kind of initialization and take the best choice (in term of maximizing the log-likelihood).

% \SR{
% %--------------------------------------------------------------------------------------
% \subsubsection{Modification of the estimates of $\tau$ and $\pi$}
% %--------------------------------------------------------------------------------------
% The following procedure has been chosen to calculate $\tau_{i,.}^{h+1}$:
% \begin{align*}
% \tau_{i,k}^{h+1/3} & = \max \left\lbrace  \phi_{m_{k}^{h},\hat{\Sigma}_{U}\left( V_{k}^{h}  \right)}\left( X_{i} \right),  e^{-100} \right\rbrace & \forall k \\
% \tau_{i,k}^{h+2/3} & = \frac{\pi_{k}^{h}\tau_{i,k}^{h+1/3}}{\sum_{\ell = 1}^{K}\pi_{\ell}^{h}\tau_{i,\ell}^{h+1/3}} \\
% \tau_{i,k}^{h+1} & = \frac{\tau_{i,k}^{h+2/3} + \epsilon_{\Pi}}{\sum_{\ell = 1}^{K}\left(  \tau_{i,\ell}^{h+2/3} + \epsilon_{\Pi} \right)}
% \end{align*}
% }{[à mettre en annexe ?]}


% \SR{
% %--------------------------------------------------------------------------------------
% \subsubsection{Modification of the robust estimates of the variances}
% %--------------------------------------------------------------------------------------
% Remark that gradient and Robbins-Monro methods can lead to negative estimates of the eigenvalues of the variance (due to estimation error). In order to ensure that the variance of each cluster is positive, and considering $\left( \hat{\lambda}_{1,k} , \ldots ,\hat{\lambda}_{d,k} \right)$ the eigenvalues obtained with the help of MCM combined with one of the Monte Carlo methods, we replace this vector by $\left( \max \left\lbrace \hat{\lambda}_{1,k} ,  \epsilon_{\text{v}} \right\rbrace , \ldots , \max \left\lbrace \hat{\lambda}_{d,k} , \epsilon_{\text{v}} \right\rbrace \right)$ with $\epsilon_{\text{v}}$ chosen arbitrarilly small.
% }{[à mettre en annexe ?]}

% \SR{
% %--------------------------------------------------------------------------------------
% \subsubsection{Chosing the number of clusters}
% %--------------------------------------------------------------------------------------
% The number of clusters is chosen minimizing the ICL criterion. Remark that .... parler des outliers...
% }{[section dédiée dans les simuls]}


\subsection{The \attack{} Loss}\label{sec:method:loss}

For reference,
defenses mechanisms we examine
in this paper
aim to find \( \xadv \)
which maximizes the SCE loss
\(
    \sceloss\parens*{
        % \ssum_{m=1}^M \ensop\parens{}
        \logit{\E}, \y
    },
\)
to evaluate the ensemble robustness.
The \attack{} loss improves this further
by proposing two additional modifications
to the untargeted loss function used to attack ensembles:
\begin{equation}
    \attackloss
        \parens{\logit{1:M}, \logit{\E}, y}
    \triangleq
    \sceloss\parens*{
        \beta \scenorm\parens*{
            \ssum_{m \in [1:M]}
            \imp{m}\parens{\logit{m}} \cdot \logit{m}
        }
        + (1 - \beta) \scenorm\parens*{\logit\E}, y
    }.
    \label{eq:attackloss}
\end{equation}
First,
it additionally introduces a sum
of the \( \imp{m} \)-weighted variant
of sub-model logits,
in order to expose sub-model logits
with adaptive reweighing described in \Cref{sec:method:weight}.
Second,
\( \beta \)
interpolates the importance
of the newly added auxiliary logits
and the original ensemble logits.  % TODO \beta = ???
Finally,
inspired by the effective surrogate loss
in~\cite{lafeat},
it further normalizes the logits
by their respective DL using:
\begin{equation}
    \scenorm\parens{\z} \triangleq
        \indicator\bracks*{\z_\y - \z_\hy > 0}
        \cdot
        {\z} / {\detach\parens{\z_\y - \z_\hy}}.
    \label{eq:scenorm}
\end{equation}

Finally, the targeted variant
of the \attack{} loss simply replaces \( y \) with \( t \)
where \( t \) is the intended target.

\subsection{Improving the State-of-the-art}

While the new \( \attackloss \) loss
is highly effective against
ensemble defenses we test in this paper,
we strive for further advances
in \attack's ability
to generate faster and better adversarial examples.
Inspired by recent publications,
we borrow ideas
from related adversarial attack tactics,
which includes
adopting a cosine step-size schedule~\cite{ye2022aaa},
momentum~\cite{dong18momentum,croce20aa},
random restarts~\cite{tramer2020adaptive}
and multiple target attacks~\cite{croce20aa,tramer2020adaptive}.
We provide the overall algorithm
in \Cref{alg:overview},
which computes
an adversarial image \( \xadv_I \)
as return,
by taking as input
the sub-models \( \f{[1:M]} \),
natural image \( \x \),
ground truth label \( \y \),
\( \beta \) to interpolate between
the auxiliary logits and the original,
\( \tau \) controls the temperature,
momentum \( \mu = 0.75 \)
following~\cite{lafeat,croce20aa},
\( \epsilon \) perturbation bound,
and finally the maximum number of iterations \( I \).
