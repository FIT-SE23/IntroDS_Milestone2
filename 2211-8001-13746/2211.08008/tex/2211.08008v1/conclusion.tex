\section{Conclusions}

This paper identifies
severe robustness overestimation
in many ensemble defense techniques,
and further investigates
problem the robustness evaluation
under three ensemble-forming strategies.
To efficiently and accurately
evaluate the robustness of ensembles,
we introduce \attack{},
a new attack technique
which reweighs sub-model importance
adaptively by their respective ``ease-of-attack''
during attack iterations.
\attack{}
enjoys a much improved success rate
and convergence rate
compared with other SOTA attacks.
Moreover,
we found several surprising observations
related to ensemble defenses,
for instance,
(1) misleading a minority of sub-models
is sufficient to fool the ensemble,
(2) summing by logits
is the simplest yet most robust way
to form ensembles,
(3) with adversarial training,
ensemble defenses may actually
harm robustness, \etc{}
We hope the above observations
may help to guide future avenue
on ensemble defenses,
and provide a strong attack baseline
for potential approaches.
Finally, \attack{} is open source
with reproducible results and pre-trained models;
and we continually update
a leaderboard of ensemble defenses
under various attack strategies.

\section*{Acknowledgements}

This work is supported in part
by National Key R\&D Program of China
(\numero{2019YFB2102100}),
Key-Area Research and Development Program
of Guangdong Province (\numero{2020B010164003}),
Science and Technology Development Fund
of Macao S.A.R (FDCT)
under \numero{0015/2019/AKP},
and Shenzhen Science and Technology Innovation Commission
(\numero{JCYJ20190812160003719}).
This work was carried out
in part at SICC
which is supported by SKL-IOTSC,
University of Macau.
