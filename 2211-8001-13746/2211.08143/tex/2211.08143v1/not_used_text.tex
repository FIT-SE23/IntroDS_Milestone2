\subsection{The 5W2H Method}
\label{sec:background-5w2h}

The 5W2H method (5-Wh and 2-How questions) is an introductory method of clarifying a
problem, an issue, an error, or nonconformity, and also to facilitate the implementation of effective corrective and preventive actions. The aim of its application is determining the root cause of a given failure. The 5W2H method was originally developed by Sakichi Toyoda and initially used by the Toyota Motor Corporation to develop manufacturing methodologies \cite{ohno1982toyota}. This method is part of industry training in problem solving. The five Wh-questions help to clarify the cause of a problem in search of corrective and preventive actions. The tool commonly applied in the automotive industry, but has been extended outside of Toyota, being applied in product and quality management \cite{pacaiova2015analysis} and also in software engineering to design, develop and evaluate the user-centered gamification \cite{klock20165w2h}. The 5W2H method consists on applying seven questions that represent the dimensions of a problem to be analyzed: \textit{Who?}, \textit{What?}, \textit{Why?}, \textit{Where?}, \textit{When?}, \textit{How?} and \textit{How much?}.

\subsection{Stage 1 - Experiment with Software Engineers}
\label{sec:method-experiment}

We ran an experiment with software engineers and students - who can be future OSS newcommers' contributors - to understand what information they would use to decide about a task to contribute.

Here we present our experiment planning (Section~\ref{sec:planning}), data collection (Section~\ref{sec:collection}), and analysis procedures (Section~\ref{sec:analysis}), as depicted in %Figure~\ref{fig:method}.

\begin{figure*}[htb]
\centering
\includegraphics[width=0.8\textwidth]{Figures/method-stage1.png}
\caption{Research Method - Stage 1 - Overview}
\label{fig:method-stage1}
\end{figure*}

\subsubsection{Experiment Planning}
\label{sec:method-experiment-planning}

In a previous study, we had the goal to evaluate the relevancy of the API-domain labels generated by a classifier %\ref{}.
We presented two versions of the JabRef issues page. The treatment group received a page with the original labels plus additional API-domain labels (generated by our classifier), and the control group a page with only the original labels. Besides the page with a list of issues, we prepared an issue information page mapping the fields (title, labels, status, body, comments, linked issue, code snipped, and participants) for the participants report where they found relevant information support their decision about choosing that issue. 

We then asked participants to rank three issues they would chose to contribute and to fill a follow-up survey about which information from the page helped them to decide about which issue to contribute and what information they would like to see. 

The experiment data and pages are available in a replication package. From the 120 participants who started the experiment, 74 completed the experiment's steps and answered the survey. We used 22 JabRef issues to mock the project page to present to the groups, pretending those issues were open. The labels presented in the Treatment group's we obtained by our classifier from a previous work \cite{}. The mocked page used by the control group presented only the original labels from the JabRef project. All the experiment pages are content of the replication package.

\subsubsection{Participants}
\label{sec:method-experiment-participants} 
We contacted software engineers managers from the industry and and professors from the academia explaining our project and asked them to send invitation to their teams and students. The invites covered a software engineer Department of a large and global company and one medium-size IT startup hosted in Brazil from the industry side. From the Students side the recruitment covered undergraduates and graduates from computer science classes, data science classes from one university in the US and two others in Brazil. The recruitment process considered the companies and classes a source of possible new contributors to the JabRef project. From the industry side we contacted one CEO and one Senior executive manager. The students side we reach them contacting two professors and one course coordinator. % We present the demographics of the participants in Table \ref{tab:demographics}. 

A Amazon Gift card was offered to who finished the survey as a compensation and to increase the number of participants.

%o novice and experienced coders, splitting our sample in half---below and above the average ``years as professional developer'' (4). We also segmented the participants between industry practitioners and students. 

The participants identification was comprised by fours digits: The first digit was "P". The second a sequential number. The third was a character from the participant's origin "U" for the University and "I" for the Industry. Finally the forth digit represent the group: "T" for Treatment and "C" for Control.

Participants received a link to run the experiment and they were split into two groups, Control and Treatment, randomically.  61.7\% of the participants ended up the survey being 41 from the Treatment group and 33 from the Control group. 

\subsubsection{Follow-up Survey}
\label{sec:method-experimento-survey} 

Next we presented to participants a follow-up survey to understand their perceptions about what information (regions) they considered important to decide about an issue to contribute. The survey was composed by the open questions: ``Why is the information you selected relevant?" and ``What kind of label would you like to see in the issues?"

The survey collected demographics data covering experience level, experience as an OSS contributor, and expertise level in JabRef's main technologies.

\subsubsection{Data Analysis}
\label{sec:method-experiment-analysis}

We qualitatively analyzed the answers by inductively applying open coding in groups, where we identified the reason of considering the provided information as relevant and what information the participant would like to be provided. We built post-formed codes as the analysis progressed and associated them with respective parts of the transcribed text, so as to code the information relevance according to the participants' perspectives.

Researchers met weekly to discuss the coding. We discussed the codes and categorization until reaching consensus about the meaning of and relationships among the codes. The outcome was a set of higher-level categories as cataloged in our codebook~\footnote{XXX}.

%The previous study \ref{} used the remaining questions' data.

In the following, we present the analysis of (1) what information is relevant for contributors to choose a task and (2) which labels they would like the issue to have (RQ1), and (3) how do maintainers infer the skills necessary for a contributor to work on a task and (4) which information they use to create the labels (RQ2).

\subsection{RQ1. What information is relevant for contributors when choosing a task?}
\label{sec:results-rq1}

Our analysis of the question (see Section~\ref{sec:method-experiment-analysis}) revealed 21 categories of information reported as relevant by contributors when they decide about a task to contribute. We organized these categories using the questions of the 5W2H method (5W - Who, What, Why, When, Where, 2H - How, How much)~\cite{klock20165w2h} (see Section~\ref{sec:background-5w2h}), as can be seen in Figure \ref{fig:results-rq1}. The 21 categories covered all seven questions of the 5W2H method, which included:
\begin{itemize}
    \item \textit{Who?} Who will solve the issue?
    \item \textit{What?} What is the issue?
    \item \textit{Why?} Why is that an issue?
    \item \textit{Where?} Where is the issue?
    \item \textit{When?} When the issue was or will be solved?
    \item \textit{How?} How to solve the issue?
    \item \textit{How much?} How big is the issue?
\end{itemize}

\begin{figure*}[htb]
\centering
\includegraphics[width=1\textwidth]{Figures/result_rq1.png}
\caption{The information reported by contributors from the present study as relevant to choose a task. We mapped our participants' definitions (shown as colored squares) to 5W2H method~\cite{klock20165w2h}, which organize information for decision-making in seven questions.}
\label{fig:results-rq1}
\end{figure*}


%Table \ref{tab:codes} presents the number of participants (interviews and survey) whose responses fit in each region. 

In the following, we present our findings organized by 5W2H questions.

\subsubsection{Who will solve the issue} has the goal of arousing the newcomer's interest in the issue. A newcomer can \textsc{get attracted} when \textit{``filtering labels to search issues that [he/she] would like to contribute the most"} (P10I0) and reading the title (P3N1) to see if it \textit{``includes something that is not too wordy and if it uses words [he/she] could easily understand"} (P3N0). When opening the issue, participants also reported the body and the comments were relevant to \textit{``gain interest on the issue"} (P14B0), as a \textit{``detailed body and helpful comments from experienced people in the project is extremely helpful to make the newcomer \textsc{feeling safe} to try the issue"} (P14I0). The confidence can increase when the contributor matches his \textsc{experience level} with the indication of difficulty to solve the issue (P11I1, P14B0), which could be shown in a label as ``easy, medium, hard" (P14B0), and also as "good first issue" (P14I0) or "good challenging issue" (P3C0). Besides the experience, contributors can try to match the \textsc{required skills} to work on the issue with their skills, and judge \textit{``if they have the skill to help"} (P18B0) or \textit{``whether or not [are] capable of finding a solution"} (P4B1). The required skills mentioned by participants included the programming language of the code (P3N0,P7B1,P13I0), the architecture layer - front-end, back-end, interface (P7B1, P13B0,P4B0), APIs (P7B0, P12B0), database (P13I10), frameworks and libraries (P9B1).
\subsubsection{Why is that an issue} can also help to raise interest in new contributors, as when they know the \textsc{goal to solve} and \textit{``what is the purpose of the issue"} (P8B0), and the
\textsc{benefits of solving} or 
\textit{``why solving it will help users"} (P9B0). Besides, the \textsc{expected behavior} of the software can help to clarify why the issue is an issue, being \textit{``a critical information to decide what about is happening in the system and what is expected"} (P19I0). Indeed, one participant reported: \textit{``I would only contribute something that I know how it works"} (P6B1).

\subsubsection{When the issue was or will be solved} was reported as the \textsc{deadline to solve} the issue, \textit{``indicating that the issue is critical, important"} (P9B1) or \textit{``urgent"} (P11I0). Participants suggested the priority to appear in a label (P17I1) and be defined according to the impact that the issue is causing to business or users (P10P1). Another idea of time is the \textit{``status to check the issue's state"} (P13I0) that can be \textsc{open, closed, or ongoing}, so contributors can use a filter in the issues' page, as they "don't look at closed issues much, so the open flag grabs [their] attention" (P5N1). Suppose any contributor is currently working on the solution. In that case, they should have their names assigned to the issue and include a comment with the  \textsc{description of an ongoing solution} which should "demonstrate the issue's status" (P11B0).

%When the issue needs or should to be solved according to the impact that having this issue to the business or client. Possible priorities suggested by participants were major/minor, high, medium, low, critical, urgent.

\subsubsection{What is the issue} should bring the \textsc{issues' description}, including both a summarized \textit{``idea of what the issue is about"} (P5B0) and a comprehensive explanation \textit{``to help understand what is the problem"} (P9B0) about. When an issue provides both levels of details, it \textit{``tells about the problem, first in a general term and later giving me details about it"} (P3B0). The issue's \textsc{type} in labels complement to \textit{``demonstrate [...] how [the issue] is classified"} (P11B0). The participants suggested the issue to have \textit{``labels that inform precisely which type of issue is"} (P21I1): bug (P7B0), a new feature (P12B0), performance (P12B0), enhancement (P12B0), security. One participant (P5N1) emphasized that "all issues should have a type so [they] can see if [their] skill set is useful" (P5N1).

\subsubsection{Where is the issue} is a piece of information that can help contributors on both checking if they have the skills to work on that issue and as a \textsc{start point} or \textit{``where to start looking at in the code/library to investigate the problem"} (P14B0). The \textsc{local in code} can be in the fields code snipped, in body and labels, indicating where the issue is happening and \textsc{connected areas}. This information would \textit{``give some hints about what areas have a connection with the problem occurring"} (P14B0), and \textit{``code snippet to provide context for wherein the program this issue was happening"} (P3N1).

\subsubsection{How to solve the issue} complements what and where questions by providing more information to help a new contributor deciding about choosing the issue to solve or not. The  \textit{``\textsc{previous attempts} to solve"} (P6B0) an issue \textit{``contains valuable information about what has already been done and properly documented"} (P12B0). Contributors who are deciding if they will work on an issue can read \textit{``\textsc{[solving] challenges}"} (P11B0) avoid and evaluate if they can follow different directions to achieve the solution. When working on the issue, \textsc{steps to reproduce} \textit{``helps to understand how to reproduce"} (P9B0) the issue on a controlled environment and \textit{``linked issues and comments can help  understand the \textsc{scenario}"} (P13I0), while  \textsc{steps to debug} help to \textit{``decipher what the problems really is"} (P7B0).

\subsubsection{How big is the issue} is an information that can provide visibility on the \textsc{required effort} for \textit{``[a contributor] to work on alone until solve it"} (P3C0). If the issue does not have this information, the newcomer tries to \textit{``grasp what's the idea of the issue, to better measure how long it would take to solve it"} (P22I0).

***The 5W2H is a method designed for problem solving and using information to answer the seven questions can help contributors not only to decide about choosing an issue, but also to solve it.

\subsubsection{Newcomers' strategies results} %Schulze version

Looking to the figure \ref{fig:activitiesflow}%\ref{fig:activitiesFreqContrib}, \ref{fig:activitiesMaintainers} and \ref{fig:activitiesNewContrib}, 
we can observe the distribution of the preferences from the three groups: %all respondents,  all contributors, 
new contributors, frequent contributors and maintainers. %All importance levels sum to 100\%.
 %All respondents represents all answers received in the survey while 
%``Contributors'' represents the merged dataset containing the answers from new contributors and frequent contributors. 

While the first ranked activity for new contributors is ``Setup the environment'', for frequent contributors is ``Understand what needs to be changed'' and a tie between ``Understand the issue'' and ``What needs to be changed'' for the maintainers group.%, %and 35\% for all contributors, 
%the maintainers consider this relative importance only 23.68\%. 
The ``Attempt to feel confident'' was last ranked by all groups and ``Understand the issue'' is important for all groups being the second option for newcomers and regular contributors. ``Setup the environment'' was low ranked by the regular contributors and maintainers. ``Understand what need'' to be changed is the first option for regular contributors and maintainers while is only the third for newcomers. 

%The ``understand the context'' newcomers' strategy was last ranked by the maintainers with only 5.26\% of preference and considered important up to 20\% from the new contributors. The ``understand the issue'' and ``understand what needs to be changed'' newcomers' strategies had a small variation in relative importance. In the former, frequent contributors ranked it from 19.05\% and maintainers group 26.32\%. The latter was ranked from 15\% from new contributors to 23.68\% from the maintainers. Both had a range of 7-8\%.

%Maintainers considered ``understand the issue'' followed closer by a tie between ``Fell confident to contribute'' and ``understand what need to be done''. New contributors and frequent contributors agreed ``fell confident to contribute'' had the highest relative importance while ``understand the issue'' is the second option for both groups tied with ``understand what need to be done'' in the case of the frequent contributors group.

%Looking to the figure~\ref{fig:activitiesflow} we can wee the newcomers' strategies relative importance flowing from the roles demographic subgroups.

\textbf{Mismatches in maintainers' strategies} %schulze results

%The strategies adopted by the projects had a slightly less variation in responses. In fact, as we have seven strategies it might lead to a better distribution range among them. 

``Have good documentation'' was considered more important for new contributors and frequent or regular contributors. It received the second rank for maintainers. Similarly, ``Improve the process'' is important for all the groups being the first for maintainers and regular contributors and second voted for new contributors. Additionally for regular contributors the communication is tied in the first place. ``Support the newcomers onboarding'' is second most voted for maintainers and regular contributors and in opposition one of the last for newcomers. ``Label the issues'' is more important for maintainers than for newcomers and regular contributors. ``Have good communication'' is pretty important for regular contributors but is the lat voted for maintainers and one of the lat ranked for regular contributors.

%In summary, newcomers and frequent contributors shared concerns about documentation and the contribution process.
%(19.05\%) while maintainers gave it only 10\% of relative importance. In general the ``improve the project quality'' strategy was considered less important for all contributors but maintainers, who gave to it not the last position, (see %table \ref{tab:conjointStrategies} and 
%figure \ref{fig:strategiesflow}).

%s \ref{fig:strategiesNewContrib}, \ref{fig:strategiesFreqContrib} and \ref{fig:strategiesMaintainers}).

%For new contributors the main strategies are ``have good documentation'' and ``have good communication''. Frequent contributors and maintainers agreed with the ``improve the process'' relative importance ranked the first for both groups. The same groups agreed in second place is ``label the issues'' where we had a tie between ``label the issues'' and ``support newcomers'' in the frequent contributors' view. 

\textbf{Maintainers and contributors fight same battles with different perspectives} %Schulze results

Maintainers, frequent contributors ans new contributors have different goals.
The maintainers are concerned to keep the project running smoothly attending to their customers and managing the increase of tasks. A study for the linux Kernel OSS project \citep{zhou2017scalability} shows the number of files and commits grows particularly in the some modules while the flow of joiners is stable or even drop. Also the maintainers effort would increase with author churn \citep{zhou2017scalability}. This is particularly observed in many OSS project. The fear of facing an unmanageable project and failure suggests the maintainers prioritize the effort to improve the process and support the newcomers onboarding while keep the documentation updated.  %the team to improve productivity and support the results found in the conjoint analysis.%: improving the project quality, labeling the issues, improving the process, and organizing the issues.

On the other hand, new contributors may be looking for the benefits of the contribution. Thus, easy access to technical tips through documentation or communication (in third place) have a special role when looking for a project to start, since it allows for a fast understanding of the project. Indirectly the learning process may leverage the career (extrinsic motivations).%The idea is mentioned in the a recent study about the shifts of motivation when newcomers are driven by learning and career (networking needs communication) \cite{gerosa2021shifting}. In addition,
A recent study about the shifts of motivation \cite{gerosa2021shifting} confirms new contributors join and senior developers keep contributing for diverse reasons. The first group needs to learn and aim to improve their career, therefore are thirsty for documentation and networking. While the experienced (regular contributors) probably want a smooth process to keep exercising the altruism or ideology (intrinsic motivations), therefore, it suggests a balanced project with many good strategies will assist them. They believe the onboarding process is really important. Perhaps the altruism might direct them to help newcomers.

\subsection{Maintainers and contributors fight differently in battles} %Schulze results

Maintainers have a deeper knowledge of their projects and the ideology they implement. Therefore, the main important task is to read and understand what is there in the issue content. As high-ranked officers, they know the battlefield. On the other hand, rookies, look for confidence and how is the environment to engage in a project. The work of \cite{den2008allocation} shows evidence that open-source projects allocation is influenced by code characteristics and complexity. Indeed, possibly an easy way to realize the skills and the complexity level of a task is by looking into the documentation, figuring out how to setup the environment or asking questions to the core developers.

Newcomers also needs to start with specific king of problems involving less complex, contained and low workload \cite{tan2020first}. The ``Good First Issue'' labeling also aims to elevate the newcomers' morale \cite{tan2020first}. Therefore, they are important resources for the OSS project but needs to be employed in easy missions to have odds to accomplish them. In addition \cite{steinmacher2021being} confirms the difficulties mentors face to assist the newcomers. Lack of information about skills, complexity, create a social fear etc. The presence of well written documentation, a communication channel that welcomes newcomers and the success creating the developer environment might direct the newcomers to the tasks and easy the first contribution.

As contributors get mature and become frequent contributors, they navigate in the project's issues and find the necessary resources. It suggests more concern about a direct approach to understand the issue and how to solve it.%  the confidence is more relevant to the decision.

%RQs spoils for schulze results

We found the set of maintainers' and newcomers' strategies revealed in the interviews have a gap from the three roles (newcomer's, frequent contributors, and maintainers) point of view regarding the relative importance (rank) defined in this study. Newcomers are very concerned about being able to setup the environment. Frequent contributors and  maintainers believe ``Understand what needs to be changed'' should be the main newcomers' strategy. Regarding the communities or maintainers' strategies, newcomers look for some guidance through documentation and communication with the team members while maintainers prioritize the contribution process and the newcomers onboarding process. On the other hand, frequent contributors are more concerned with the newcomers' support than newcomers. Maintainers think the labels have more relative importance than the newcomers' participants. Communication is not ranked as important strategy the same for regular and new contributors.

\subsubsection{Data Analysis}%schulze results
\label{sec:survey_analysis}

We used the Schulze method to rank the survey responses and their association with groups from demographic data ~\cite{wohlin2015towards, schulze2003new}. 

\paragraph{Schulze Method}

The Schulze method is a election technique to compute a single or a ordered list of preferences n the concept of proportional. The method prioritizes votes for candidates who wins the pairwises comparisons for each candidate \cite{schulze2011new, schulze2003new}. The intuition is the best ranked strategies have higher relative importance for the survey participants.  

In our case, we are evaluating which strategy is most worth investing in an OSS project. In fact, we want to discover the degree of importance of each one in order to invest resources compatible with the importance of each strategy to be adopted by OSS projects. Similarly, we will do the same for newcomers' strategies, that is, evaluating which activity should receive more attention and how important they are to be distributed so that the new contributor knows where to invest their energy and personal effort. The outcomes of the Schulze method are a rank of newcomers' strategies and maintainers ' strategies ordered by their relative importance. It would suggest where the priority effort must be applied for the maintainers' strategies and for the new contributor's strategies. 

\paragraph{Schulze Setup}

The Schulze configuration takes into account the specification of the factor set and its respective levels that define the usefulness of the idea (maintainers' and newcomer's strategies). In our case, the factors coincide with both strategies identified in the previous stage of the interview.

We created the ballot list counting how many times each maintainers' and newcomers' strategies were chosen in each level the survey and aggregating the number of times each ranking order was chosen. We used the R package ``votesys''\cite{votesys}. 

After identifying the strategies, we grouped them into a set of higher-level categories as cataloged in our codebook of maintainers' strategies to help new contributors find a suitable issue~\footnote{https://bit.ly/strCode} and a codebook of newcomers' strategies to choose a task from a project’s issue tracker~\footnote{https://bit.ly/actCode}.

\paragraph{Comparison with recent studies}

This section presents the results comparing the findings with recent literature. Some strategies were identified in use by OSS projects and we are reporting here. We are surprised that after many papers about strategies, we still found actionable strategies.

The newcomer and community strategies identified are similar to some found in recent research, as presented in Figure \ref{fig:previousStrategies}:

Labeling strategies have been focus on recent literature and employing different directions (see Figure \ref{fig:previousStrategies} *6). Components are used by the JabRef project. \footnote{https://github.com/JabRef/jabref}. 

We believe there is still room for some new approaches. For example, label for knowledge area were proposed by P13: ``...Now, how they are labeled, component maps to modules in the in the project. ... And area in the know(ledge) if there is some mapping of component an area? An area is more of a general area of knowledge or concepts more conceptual''. Expected outcome labelling is another case (P9) and the context label (P16) reported in section \label{sec:results_community_strategies}.

For good documentation and communications strategies we have many references in the literature as shown in (see Figure \ref{fig:previousStrategies} *1 *2). Despite recent studies did not mention clearly the needs of tutorials we have evidence of the use of them \footnote{https://www.elastic.co/cloud/as-a-service} and code base \footnote{https://devdocs.jabref.org/} and relevant links \footnote{https://devdocs.jabref.org/contributing} they are used in by OSS projects as well. 

Organize the issues is a recurrent technique and covered by recent studies \ref{fig:previousStrategies} *2 *3 *7). Here we found some new opportunities including the link issues with users impacted (not developers working on it), that would bring value since we might know the business unit impacted or interested in the solution of the issue. GitHub provide a issue split functionality \footnote{https://twitter.com/github/status/1433811581922578462?lang=en} and issue deduplication is addressed in a tool proposed by \citep{zhang2020ilinker}.
 

Support the onboarding of newcomers -- The aforementioned studies also have proposals related to this strategy \ref{fig:previousStrategies} *1 *2 *3 *5) as well the Improve the process strategy has proposals to improve the code reviews process and provide clear guidance on the governance process \ref{fig:previousStrategies} *4). GitHub issue guidelines are available using this strategy, as an example: \footnote{https://github.com/necolas/issue-guidelines/blob/master/CONTRIBUTING.md#bugs}.

Improve the project quality is a concern explored in some studies \ref{fig:previousStrategies} *5). Despite is not clearly described in the recent literature we found evidences of this strategy in projects in the OSS communities (test \footnote{https://devdocs.jabref.org/getting-into-the-code/code-howtos#test-cases}) code quality \footnote{https://devdocs.jabref.org/advanced-reading/code-quality.html} and the industry since many products are available (code analysis \footnote{https://www.sonarqube.org/}). 

Multi-teaming research may be helpful to address the team management encompassing  strategies to integrate newcomers, manage the quality and preparing the project to use a contribution process suitable to dynamic teams with high churn volume, difficulties of communications and diverse levels of members commitment \cite{gupta2018productivity}.

The Setup environment newcomers' strategy were covered (see Figure \ref{fig:previousStrategies} *2). The importance of this strategy for newcomer probably is due to the increasing complexity brought by the recent applications and the plurality of the configuration. Since OSS projects are not contained in a single company the configuration management is hard to pursuit creating additional challenges for this strategy. Future work can address specific strategies to handle this complexity. 

While the strategies proposed in the literature barely tackle which strategies newcomers should use to communicate with the community, some communities' strategies may help to increase the confidence of newcomers, such as acting with kindness and putting effort to help newcomers feel part of the team and not afraid of the community(see Figure \ref{fig:previousStrategies} *2).


    
 
    









%To break the difficulty of choosing/finding theirs tasks, \citet{steinmacher2015understanding} mention possible related root causes we can map to the strategies we found. The recent studies do not have a perfect semantic for all the newcomers' strategies. Thus, we can mention: lack of confidence (attempt to feel confident to contribute), newcomers do not know how to reduce the scope (understand the context), problems with the information provided in the issues (understand the issue). However, some maintainers' and newcomers' strategies are in different abstraction levels and can subsume or be subsumed by the content in the studied literature. 
