\section{Introduction}

Selecting an appropriate task to contribute, is one of the most challenging, but crucial steps in open source projects ~\cite{wang2011bug,steinmacher2015understanding,steinmacher2015systematic,steinmacher2015social, 10.1145/2675133.2675215,stanik2018simple}. 
This is a known problem and projects employ different various strategies to attract newcomers and assist them in finding starter tasks. Existing works have a compendium of such strategies~\cite{steinmacher2013newcomers, huang2021characterizing, santos2021can}. For example, research has proposed labeling issues that signal newcomer friendliness (e.g., starter task, newcomer task) as an important strategy to aid newcomers identify tasks they can undertake~\cite{ADD ME}. Others have proposed mechanisms that aid newcomers understand the issue to be solved~\cite{ADD ME}. A majority of these strategies are based on research on where newcomers struggle and contributors' recommendations of overcoming these barriers.    
A missing piece in this picture is if the newcomers use these strategies or think them as important for them. A discrepancy between these two perspectives---newcomers and existing contributors---can create a gulf of expectations. Such a gulf, in turn means that the strategies put forth by the projects are less likely to succeed and newcomers continue to struggle.

%newcomers often use their own strategies to contribute to OSS projects~\cite{elazhary2019not}.
%it is still unclear whether there is a divergence between maintainers' and contributors' perspectives~\cite{elazhary2019not}, and current strategies may be ineffective. 

%Adding labels to the issues (a.k.a tasks, bug reports) help newcomers when they are choosing their tasks~\cite{steinmacher2018let, santos2021can}. 
%However, community managers find that labeling issues is challenging and time-consuming~\cite{9057411} because projects require skills in different languages, frameworks, databases, and Application Programming Interfaces (APIs). 
%However, the labels (and other fields as title and body) should bring relevant information to support the newcomer's decision on choosing their task.

Our goal in this paper is to investigate the difference in perspectives of newcomers and existing contributors in what strategies: (i) newcomers use to choose a task and (ii) communities need to employ to support newcomers.
Exploring the different perspectives can help OSS communities devise tailored strategies that match newcomers' needs.


%Given this context, in this paper, we investigate what information a novice contributor would like to have to support this decision, and how would maintainers infer the skills necessary to work on a task.
%Even at the design level, where it can be argued that information is aggregated into layers of higher abstractions, 

%Exploring all the project details is a time-consuming task, and the landscape is better understood with human-aid and well-distributed signs~\cite{dagenais2010moving}. 

%Previous qualitative work compared the guidelines and GitHub process and proposed strategies to break the barriers found by the newcomers~\cite{steinmacher2015systematic}, however they do not explain why some difficulties still prevail, check if the newcomers feel the proposed strategies helpful or investigate the way newcomers look for the skills before get involved in a issue solution. Maintainers might be blind being managing their repositories unaware the newcomers needs are different thus a gap may exist. Without knowing the gap, the newcomers keep struggling and possibly failing to engage.

%In this paper, we catalog and identify how maintainers, regular contributors, and newcomers perceive the importance of newcomers' strategies to infer required skills (to work on a task) and maintainers' strategies to facilitate task selection. 
We aim to answer the following research questions:

%\textbf{RQ1.} What strategies do newcomers employ to identify issue skills from the perspective of maintainers?
%\textbf{RQ2.} How important are the strategies from the point of view of regular contributors and newcomers?

\textbf{RQ1.} What strategies newcomers use in choosing a task in OSS?

\textbf{RQ2.} How do newcomers and existing contributors differ in their opinions of which strategies are important for newcomers?

%RQs spoils for schulze results

We conducted a qualitative study to identify strategies that maintainers' think are important to assist newcomers. We focused on maintainers as they  typically have knowledge and ownership of the project and strategies they propose have a higher likelihood of being implemented. 
%
We then did a follow on survey to understand to what extent newcomers and frequent contributors thought about these strategies.

We found instances where newcomers and other contributors agreed, but also situations where newcomers, frequent contributors, and maintainers differed in what they thought newcomers needed to select a task and start to contribute to an OSS project. In our survey, frequent contributors and maintainers believed that ``Understand the issue'' would be the most important newcomer strategy, but newcomers had a different perspective (``set up environment'' was seen as most important by newcomers). Regarding community strategies, newcomers and frequent contributors agreed on the importance of good documentation and project quality, but differed widely about improving newcomer onboarding and contribution process. %On the other hand, frequent contributors are more concerned with newcomers' support than newcomers. Maintainers think communication has more relative importance than does the newcomers' participants. Labeling is not considered a priority strategy for regular or newcomers.

%Spoil for conjoint
%We found the set of maintainers' and newcomers' strategies revealed in the interviews have a gap from the three roles (newcomers, frequent contributors, and maintainers) point of view regarding the relative importance (rank) defined in this study. Newcomers and frequent contributors are very concerned about being confident to contribute. Maintainers believe ``Understand the issue'' should be the main newcomers' strategy. Regarding the communities' or maintainers' strategies, newcomers look for some guidance through documentation and communication with the team members while maintainers prioritize the contribution process. On the other hand, frequent contributors are more concerned with the newcomers' support than maintainers and think the labels have more relative importance than the newcomers' participants.

%We found that maintainers and contributors have different perspectives on how newcomers choose tasks and how to support them. 

%RQ1. What information is relevant for contributors when choosing a task?
%To answer RQ1, we conducted a study with 74 participants from both academia and industry. After asking participants to select and rank real issues they would like to contribute to, we provided a follow-up survey to determine what information was relevant to make the decision.

%RQ1a. Which labels contributors would like an issue to have?

%RQ1. How do a contributor choose a task to match their own skills?
%RQ3. How do maintainers and contributors agree about the relative importance of the maintainer's strategies and newcomer's strategies?

