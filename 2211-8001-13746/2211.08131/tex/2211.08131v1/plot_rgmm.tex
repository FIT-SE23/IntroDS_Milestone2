In this Section, we explain how to use the R package \texttt{RGMM} to illustrate the results. First, let us consider the following R code:
\begin{verbatim}
> mu <- matrix( c(rep(0,10),rep(2,10),rep(-2,10)), byrow=T, nrow=3)
> ech <- Gen_MM(nk = rep(200,3), delta=0.1,mu=mu)
> X<- ech$X
> Result <- RobMM(X)
\end{verbatim}
The function \texttt{Gen$\_$MM} enables to generate a sample of mixture model (Gaussian, Student or Laplace), whose centers are the raws of the matrix \texttt{mu}. The number  of data by cluster is given by \texttt{nk} while \texttt{delta} gives the proportion of contaminated data. The function \texttt{RobMM} gives the results obtained with the help of our method.  One can see the vignette for more details and to see the different options. 

We now focus on the function \texttt{RMMplot} which enables to illustrate the results. More precisely, we now comment the different available graphics.

\begin{verbatim}
> RMMplot(Result,graph=c('Two_Dim'))
\end{verbatim}

\begin{figure}[H]
\centering
    \includegraphics[scale=0.5]{figures/plot_two_dim.pdf}    
\end{figure}

The option \texttt{'Two$\_$Dim'} enables to represent the 2 first principal components of the data using robust principal component analysis components (RPCA) (see \cite{CG2015}).


\begin{verbatim}
> RMMplot(Result,graph=c('Two_Dim_Uncertainty'))
\end{verbatim}


\begin{figure}[H]
\centering
    \includegraphics[scale=0.5]{figures/plot_two_dim_uncertainty.pdf}    
\end{figure}

The option \texttt{'Two$\_$Dim$\_$Uncertainty'} enables also to represent the 2 first principal components, but the size of the points is proportional to the uncertainty of the classification of the data.


\begin{verbatim}
> RMMplot(Result,graph=c('ICL'))
> RMMplot(Result,graph=c('BIC'))
\end{verbatim}

\begin{figure}[H]
\centering
    \includegraphics[scale=0.5]{figures/plot_icl.pdf}    
    \includegraphics[scale=0.5]{figures/plot_bic.pdf}    
\end{figure}

Options \texttt{'ICL'} and \texttt{'BIC'} enables to visualize the evolution of the criterion with respect to the number of clusters $K$.


\begin{verbatim}
RMMplot(Result,graph=c('Profiles'))
\end{verbatim}

\begin{figure}[H]
\centering
    \includegraphics[scale=0.5]{figures/plot_profiles.pdf}    
\end{figure}

Option \texttt{'Profiles'} allows to visualize data points in dimensions higher than $3$. More precisely, we represent data as curves that we call "profiles", gathered it by cluster, and represented the centers of the groups in red.

\begin{verbatim}
RMMplot(Result,graph=c('Uncertainty'))
\end{verbatim}

\begin{figure}[H]
\centering
    \includegraphics[scale=0.5]{figures/plot_uncertainty.pdf}    
\end{figure}

Option \texttt{'Uncertainty'} enables to visualize the uncertainty of classification of the data.
