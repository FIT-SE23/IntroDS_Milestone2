\paragraph{Problem.}
Grouping observations into homogeneous groups (or "clusters") is one of the most typical tasks in statistical data analysis. Among the many methods that have been proposed over the years, model-based clustering is one of the most popular (\cite{MaP00}). Model-based clustering relies on the assumption that the observed data come from a mixture model, meaning that the observations can be divided into a finite (but often unknown) number of clusters, and that each cluster is characterized by a specific distribution, often called the {\em emission} distribution.

One reason for the popularity of model-based clustering is that the emission distributions of the clusters are usually chosen with a parametric class (e.g. a multivariate Gaussian), which makes the interpretation of the results particularly easy. Another reason for this popularity is that the maximum likelihood estimates of the parameters can be obtained via the well-known EM algorithm (\cite{DLR77}), accompanied by statistical guarantees. 

Nevertheless, one of the weaknesses of model-based clustering methods is their sensitivity to misspecification of emission distributions or to the presence of (possibly numerous) outliers. In both cases, this results in a high proportion of misclassified observations  or a poor estimate of the number of clusters \cite{GGM10}.

% \begin{itemize}
%  \item Clustering = one of the most typical task in data analysis
%  \item Model-based = statistically grounded + interpretability
%  \item Poor robustness to mis-specification of the emission distribution
% \end{itemize}

\paragraph{Robust approaches.} 
A series of robust approaches have been proposed to overcome these limitations. These approaches can be classified into three main categories. 
% Most of the time: parametric alternatives
A first track sticks to the parametric framework, but uses emission distributions with heavier tails, {such as multivariate student for Gaussian mixtures} (see, e.g., \cite{PeM00,WaL15,SPI15,RoS19}). Alternatively, a component associated with (possibly improper) parametric distribution can be added, in order to capture outliers (\cite{BaR93,CoH16,CoH17,FaP20}). The outlier distribution may typically be uniform emission over a large domain.
% Trimming methods
A second approach is to prune the observations, so that the outliers do not weigh too heavily on the estimates \cite{GGM08}. A final approach is to use a dedicated weighted contrast (instead of negative log-likelihood: \cite{GYZ19,GMY21}). The latter approach has some similarities with {the method we propose}. 


\paragraph{Our contribution.}
This paper focuses on the robustness of model-based clustering methods to the presence of outliers, meaning that we make no assumptions about how outliers deviate from prescribed emission distributions. To this end, we adopt a fully parametric model-based clustering framework, but modify the EM algorithm (more specifically, the M-step) to ensure robustness. Our   method is valid for any symmetric emission distribution and resorts to the estimation of the median vector and the median covariation matrix  {in place} of the mean vector and the covariance matrix. The estimation of these quantities benefits from a series of recent contributions \cite{VZ00,HC,CG2015}. It was especially proven (see \cite{KrausPanaretos2012}) that for symmetric distributions, the MCM and the usual covariance have the same eigenvectors. Nevertheless, although the recursive estimation of the MCM has been studied in \cite{CG2015}, no method for building the covariance from the MCM has been proposed. In this paper, we first propose methods to get  robust estimates of the covariance  before applying it to robust model-based clustering. 

\paragraph{Outline.}
 The following section gives a comparative introduction to recent algorithms for estimating median vectors and median covariation matrices. In Section \ref{sec:RMM}, we show how these estimates can be used for robust inference of mixture models. The resulting algorithm is given in Section \ref{sec:algo}. A comprehensive simulation study is presented in Section \ref{sec:simul}: different estimators of the median covariation matrix are first compared in Section \ref{sec:simVariance}, then the classification accuracy of the proposed EM-type algorithm is evaluated in Section \ref{sec:simMixture}. All the proposed methods are available in the R package \texttt{RGMM} accessible on CRAN\footnote{\url{cran.r-project.org/package=RGMM}}.
 
