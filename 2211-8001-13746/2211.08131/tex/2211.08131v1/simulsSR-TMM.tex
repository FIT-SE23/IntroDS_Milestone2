We then simulated $B=100$ datasets according to a Student mixture model with the same set of parameter configurations (see Section \ref{sec:simDesign}). Again, we only present here the results with $n_k = 500$ observations in each group ($n = 1500$).  For the same reason as in the Gaussian case, we only present the results obtained with the $BIC$ criterion. 

Figure \ref{fig:simTMM1500} is organized in the same way as Figure \ref{fig:simGMM1500}. In terms of classification, we observe a dramatic drop of the accuracy obtained with maximum likelihood inference (TMM: as performed by the {\tt teigen} R package), as compared to its robust counterpart (RTMM). We also observe that, depending on the simulation scenario, the classification accuracy of the robust approach decreases more or less rapidely, the better results being obtained under the scenarios where outliers can each be associated with a clusters (($c$) and ($e$)).

\begin{figure}[ht]
  \centering
%   \begin{center}
    \begin{tabular}{c|m{.2\textwidth}m{.2\textwidth}|m{.2\textwidth}m{.2\textwidth}}
      & 
      \multicolumn{2}{c|}{Classification} & 
      \multicolumn{2}{c}{Parameter estimation} \\ 
      &
      \multicolumn{1}{c}{ARI} & \multicolumn{1}{c|}{$\widehat{K}$} & 
      \multicolumn{1}{c}{$MSE(\mu)$} & \multicolumn{1}{c}{$MSE(\Sigma)$} \\ 
      \hline
%       & & & \\
      ($a$) &
      \includegraphics[width=.2\textwidth, trim=10 10 10 50, clip=]{figures/simRTMM-p5-K3-n1500-s1-ARI} &
      \includegraphics[width=.2\textwidth, trim=10 10 10 50, clip=]{figures/simRTMM-p5-K3-n1500-s1-K} &
      \includegraphics[width=.2\textwidth, trim=10 10 10 50, clip=]{figures/simRTMM-p5-K3-n1500-s1-MSEmu} &
      \includegraphics[width=.2\textwidth, trim=10 10 10 50, clip=]{figures/simRTMM-p5-K3-n1500-s1-MSEsigma} \\
      ($b$) &
      \includegraphics[width=.2\textwidth, trim=10 10 10 50, clip=]{figures/simRTMM-p5-K3-n1500-s2-ARI} &
      \includegraphics[width=.2\textwidth, trim=10 10 10 50, clip=]{figures/simRTMM-p5-K3-n1500-s2-K} &
      \includegraphics[width=.2\textwidth, trim=10 10 10 50, clip=]{figures/simRTMM-p5-K3-n1500-s2-MSEmu} &
      \includegraphics[width=.2\textwidth, trim=10 10 10 50, clip=]{figures/simRTMM-p5-K3-n1500-s2-MSEsigma} \\
      ($c$) &
      \includegraphics[width=.2\textwidth, trim=10 10 10 50, clip=]{figures/simRTMM-p5-K3-n1500-s3-ARI} &
      \includegraphics[width=.2\textwidth, trim=10 10 10 50, clip=]{figures/simRTMM-p5-K3-n1500-s3-K} &
      \includegraphics[width=.2\textwidth, trim=10 10 10 50, clip=]{figures/simRTMM-p5-K3-n1500-s3-MSEmu} &
      \includegraphics[width=.2\textwidth, trim=10 10 10 50, clip=]{figures/simRTMM-p5-K3-n1500-s3-MSEsigma} \\
      ($d$) &
      \includegraphics[width=.2\textwidth, trim=10 10 10 50, clip=]{figures/simRTMM-p5-K3-n1500-s4-ARI} &
      \includegraphics[width=.2\textwidth, trim=10 10 10 50, clip=]{figures/simRTMM-p5-K3-n1500-s4-K} &
      \includegraphics[width=.2\textwidth, trim=10 10 10 50, clip=]{figures/simRTMM-p5-K3-n1500-s4-MSEmu} &
      \includegraphics[width=.2\textwidth, trim=10 10 10 50, clip=]{figures/simRTMM-p5-K3-n1500-s4-MSEsigma} \\
      ($e$) &
      \includegraphics[width=.2\textwidth, trim=10 10 10 50, clip=]{figures/simRTMM-p5-K3-n1500-s5-ARI} &
      \includegraphics[width=.2\textwidth, trim=10 10 10 50, clip=]{figures/simRTMM-p5-K3-n1500-s5-K} &
      \includegraphics[width=.2\textwidth, trim=10 10 10 50, clip=]{figures/simRTMM-p5-K3-n1500-s5-MSEmu} &
      \includegraphics[width=.2\textwidth, trim=10 10 10 50, clip=]{figures/simRTMM-p5-K3-n1500-s5-MSEsigma} 
    \end{tabular}
    \caption{Student mixture model: classification accuracy ($ARI$), estimated number of clusters $\widehat{K}$, estimation error fu the mean ($MSE(\mu)$) and for the variance ($MSE(\Sigma)$) for scenarios ($a$) to ($e$), with $n_k = 500$ observation in each of the $K^*$ clusters ($n = 1500$). Same legend as Figure \ref{fig:simGMM1500}. \label{fig:simTMM1500}}
%   \end{center}
\end{figure}

The last two columns of Figure \ref{fig:simTMM1500} also shows better performances of the robust approach RTMM as compared to maximum likelihood TMM in terms of precision accuracy. Observe that several curves associated with TMM display an erratic behavior due to convergence issues of the EM algorithm (see Figure \ref{fig:simTMM1500fail} in Appendix \ref{app:simResults}).

Similarly to de Gaussian case, Figure \ref{fig:simTMM300}, given in Appendix \ref{app:simResults}, is the same as Figure \ref{fig:simTMM1500} for $n_k = 100$ observations per cluster ($n = 300$): again similar conclusions can be drawn from it.


\FloatBarrier
