%-------------------------------------------------------------------------------
%-------------------------------------------------------------------------------
\subsection{Variance and median estimation} \label{sec:simVariance}
%-------------------------------------------------------------------------------

The first part our simulations focuses on the robust estimation of the first two moments in one single cluster (no mixture). 

%-------------------------------------------------------------------------------
\subsubsection{Gaussian case}
%-------------------------------------------------------------------------------

%-------------------------------------------------------------------------------
\paragraph{No outlier.}
In this section,  we first consider the estimation of the variance and median in absence of outliers.  {To this aim, we consider} $X \sim \mathcal{N}(0,\Sigma)$, with $\Sigma = \Sigma_0$, as given in Equation \eqref{eq:Sigma}, Appendix \ref{app:simDesign}. We first focus on the accuracy of each method to estimate the variance. To do so, we consider $n= 10^5$ i.i.d copies of $X$ and estimate the MCM with the help of the Weiszfeld's algorithm. 

In Figure \ref{fig:estim_robust_cov}, we show the evolution of the quadratic mean error of the estimates with respect to the sample size. More precisely, we compared the estimates obtained with fix point algorithm, with $10$, $20$ and $50$ iterations, with the iterative gradient algorithm with  $10$, $20$ and $50$ iterations  and the averaged Robbins-Monro estimates (Robbins-Monro). \\
We also compared the behavior of the methods but with  {fixed} computation {budget}. More precisely, if a sample of size $N$ has been generated for the Monte-Carlo method for an iterative method with $T=50$, a sample of size $5 N$ is generated for an iterative method with $T=10$ iterations, and a sample of size $50N$ will be generated for the Robbins-Monro method.  {The results are} based on $\SR{}{B=}50$  {replicates}. 

We observe that all methods achieve convergence and have similar behaviors when they use samples with same sizes. Nevertheless, for  {fixed} computation  {budget}, the method based on the Robbins-Monro algorithm seams (without surprise) to lead to better results.

\begin{figure}[H]
\centering
    \includegraphics[scale=0.5]{figures/plot_variance_gaussian.pdf}    
    \caption{Evolution of the quadratic mean error of the different methods with respect to the sample size (on the left) and to computation time (on the right).}
    \label{fig:estim_robust_cov}
\end{figure}

%-------------------------------------------------------------------------------
\paragraph{With outliers.}
We then introduced an increasing fraction $\delta$ of outliers according to scenarios ($a$), ($b$) and ($d$) described in Section \ref{sec:simDesign}. 
We considered samples with size $n = 5000$, and estimated the MCM with the help of the Weiszfeld algorithm (indicated by (W)) or with the ASGD (indicated by (R)). We estimated the  eigenvalues of the variance with the three proposed methods and with a sample size of $N = 2000$ for the Monte-Carlo method before building the variance. For iterative methods, we used $T=50$ iterations. 

 {
All robust methods provide accurate estimates of the variance, even in presence of a large fraction of outliers.}
In addition, one can see that even if Robbins-Monro method  {is slightly less precise than} the other robust alternatives, but performs well any way. 
 {Yet, as the} Robbins-Monro procedure is less expansive in term of {computation} time, and {because it turns out to be more accurate than the other methods with a same computational budget} (see Appendix \ref{app::variance}), this procedure will be preferred for robust mixture models.

\begin{table}[H]
\centering
\begin{tabular}[b]{cc|rrrrrrr}
& \rotatebox[origin=r]{360}{$\delta$ ($\%$)}  & \rotatebox{270}{FixPoint (R)}    & \rotatebox{270}{FixPoint (W)}    & \rotatebox{270}{Gradient (R)}    & \rotatebox{270}{Gradient (W)}    & \rotatebox{270}{Robbins (R)}    & \rotatebox{270}{Robbins (W)}    & \rotatebox[origin=l]{270}{Variance}       \\  
   \hline
\multirow{8}{*}{\rotatebox{90}{ {($a$): $U$}}}& 0  & 0.32 & 0.24 & 0.34 & 0.31 & 0.45 & 0.36 & \textbf{0.11} \\ 
&   2 & 0.39 & \textbf{0.34} & 0.36 & \textbf{0.34} & 0.40 & 0.36 & 39.75 \\ 
&   3 & \textbf{0.36} & 0.39 & 0.39 & \textbf{0.36} & 0.43 & 0.38 & 78.20 \\ 
&   5 & 0.63 & \textbf{0.51} & 0.59 & 0.57 & 0.57 & 0.59 & 212.60 \\ 
&   9 & 1.35 & 1.36 & 1.29 & 1.21 & 1.28 & \textbf{1.06} & 682.80 \\ 
&   16 & 4.01 & 3.88 & 3.91 & 3.89 & 3.41 & \textbf{3.36}   & $2.10^{3}$ \\ 
&  28 & 16.65 & 17.56 & 16.21 & 16.13 & 13.78 & \textbf{13.51} & $7.10^{3}$ \\ 
 & 50 & 154.52 & 165.05 & 133.19 & 142.32 &\textbf{ 109.12} & 116.59 & $2.10^{4}$ \\ 
\hline
\multirow{8}{*}{\rotatebox{90}{ {($b$): $T_1$}}} &   
0 & 0.31 & 0.29 & 0.32 & 0.34 & 0.38 & 0.40 & \textbf{0.10} \\ 
&   2 & 0.33 & 0.31 & \textbf{0.30} & 0.31 & 0.44 & 0.37 & $2.10^{8}$ \\ 
&   3 & 0.36 & \textbf{0.28} & 0.29 & 0.35 & 0.40 & 0.36 & $2.10^{7}$ \\ 
&   5 &\textbf{ 0.35} & 0.36 & 0.41 & 0.40 & 0.43 & 0.54 & $10^{9}$ \\ 
&   9 & 0.49 & \textbf{0.46} & 0.48 & 0.47 & 0.67 & 0.65 & $7.10^{9}$ \\ 
&   16 & 0.86 & 0.77 & 0.80 & \textbf{0.76 }& 0.98 & 0.93 & $8.10^{13}$ \\ 
&   28 & 1.74 & 1.76 & \textbf{1.64} & 1.78 & 2.01 & 1.92 & $5.10^{11}$ \\ 
&   50 & 5.49 & \textbf{5.28} & 5.38 & 5.52 & 5.59 & 5.84 & $2.10^{13}$ \\ 
   \hline
\multirow{8}{*}{\rotatebox{90}{ {($e$): $T_2$}}} &    0 & 0.29 & 0.28 & 0.37 & 0.29 & 0.46 & 0.33 & \textbf{0.12} \\ 
&   2 & 0.33 & 0.33 & \textbf{0.31} & 0.34 & 0.41 & 0.48 & 1.06 \\ 
&   3 & \textbf{0.35} & 0.40 & 0.42 & 0.38 & 0.63 & 0.41 & 0.59 \\ 
&   5 & 0.52 & 0.60 & \textbf{0.48} & 0.49 & 0.66 & 0.76 & 7.03 \\ 
&   9 & 0.86 & 1.02 & \textbf{0.79} & 0.98 & 1.10 & 1.20 & 6.10 \\ 
&   16 & \textbf{1.99} & 2.07 & 2.08 & 2.21 & 2.50 & 2.54 & 330.59  \\ 
&   28 & 5.80 & 5.59 & \textbf{5.50} & 5.88 & 5.92 & 6.20 & $9.10^{6 }$ \\ 
&   50 & \textbf{14.84} & 15.12 & 14.99 & 15.16 & 15.38 & 15.31 & $2.10^{4}$\\ 
   \hline
\end{tabular}
\caption{
%\SR{Gaussian case: Quadratic mean error}{Mean quadratic error} of the variance estimates for the different methods and different contamination fractions $\delta$.
 {Multivariate Gaussian case: Mean quadratic error} of the estimates of the variance for the different methods and for different contamination scenarios and fractions $\delta$.}
\end{table}


%-------------------------------------------------------------------------------
\subsubsection{Student case}
%-------------------------------------------------------------------------------

{We used a similar scheme for the Student distribution.}

%-------------------------------------------------------------------------------
\paragraph{No outlier.}

We considered a Student distribution with null mean vector, with variance $\Sigma_0$ and 3 degrees of freedom. 

We first focus on the accuracy of each method to estimate the variance and follow the same simulation plan as for the Gaussian case. Observe that in this case, the weighted averaged Robbins-Monro method is slightly less accurate for fixed sample sizes, but is slightly better for fixed computational budget. In Table \ref{tab::rob::student}, we first remark that the usual estimate of the variance clearly underperform, even for uncontaminated data. In addition, although gradient method with $50$ iterations is undoubtedly better, the Robbins-Monro alternative is a serious competitor.  Then, coupled with what has been observed in the Gaussian case, the method based on the Robbins-Monro algorithm seems the best option for estimating the variances of the clusters of robust mixture models.


\begin{figure}[H]
\centering
    \includegraphics[scale=0.5]{figures/plot_variance_student.pdf}     
    \caption{Evolution of the quadratic mean error of the different methods with respect to the sample size (on the left) and to computation time (to the right).}
    \label{fig:Testim_robust_cov}
\end{figure}

%-------------------------------------------------------------------------------
\paragraph{With outliers.}

We then introduced an increasing fraction $\delta$ of outliers according to same three scenarios ($a$), ($b$) and ($d$) from Section \ref{sec:simDesign}. 

{The conclusion are the same as in the Gausian case.}

\begin{table}[H]
\centering
\begin{tabular}{cc|rrrrrrr}
& \rotatebox[origin=r]{360}{$\delta$ ($\%$)}  & \rotatebox{270}{FixPoint (R)}    & \rotatebox{270}{FixPoint (W)}    & \rotatebox{270}{Gradient (R)}    & \rotatebox{270}{Gradient (W)}    & \rotatebox{270}{Robbins (R)}    & \rotatebox{270}{Robbins (W)}    & \rotatebox[origin=l]{270}{Variance}       \\  
   \hline
\multirow{8}{*}{\rotatebox{90}{{($a$): $U$}}} &   0 & 0.29 & 0.25 & \textbf{0.20} & \textbf{0.20} & 0.50 & 0.46 & 19.78 \\ 
&   2 & 0.37 & 0.36 & 0.27 & \textbf{0.22} & 0.46 & 0.44 & 43.43\\ 
&   3 & 0.41 & 0.37 & 0.34 & \textbf{0.27} & 0.68 & 0.55 & 103.51\\ 
&   5 & 0.79 & 0.63 & 0.62 & \textbf{0.53} & 1.06 & 0.84 & 207.81 \\ 
&   9 & 2.01 & 1.82 & 1.91 &\textbf{ 1.63} & 2.34 & 1.90 & 733.99 \\ 
&   16 & 6.73 & 5.83 & 6.11 & \textbf{5.61} & 6.87 & 6.89 & $2.10^{3}$ \\ 
&   28 & 29.82 & 27.17 & 26.84 & \textbf{25.25} & 30.72 & 28.27 & $7.10^{3}$  \\ 
&   50 & 393.82 & 374.55 & 273.07 & \textbf{260.37} & 336.28 & 324.98 & $2.10^{4}$ \\ 
\hline
\multirow{8}{*}{\rotatebox{90}{{($b$): $T_1$}}} &   0 & 0.27 & 0.26 & \textbf{0.17} & 0.16 & 0.38 & 0.48 & 16.14 \\ 
&   2 & 0.37 & 0.31 & 0.21 & \textbf{0.17} & 0.52 & 0.46 & $10^{8}$\\ 
&   3 & 0.35 & 0.27 & 0.23 & \textbf{0.20} & 0.52 & 0.45 & $10^{10}$\\ 
&   5 & 0.44 & 0.39 & 0.31 & \textbf{0.27} & 0.62 & 0.69 & $3.10^{9}$ \\ 
&   9 & 0.83 & 0.75 & 0.67 & \textbf{0.59} & 1.22 & 0.93 & $10^{10}$ \\ 
&   16 & 2.18 & 1.97 & 1.90 & \textbf{1.77} & 2.74 & 1.98 & $2.10^{10}$ \\ 
&   28 & 6.54 & 6.17 & 6.08 & \textbf{5.64} & 7.00 & 6.05 & $5.10^{12}$  \\ 
&   50 & 32.39 & 30.08 & 29.16 & \textbf{27.99} & 31.48 & 29.79 & $2.10^{18}$ \\ 
\hline
\multirow{8}{*}{\rotatebox{90}{{($e$): $T_2$}}} &   0 & 0.30 & 0.26 & 0.19 & \textbf{0.18} & 0.40 & 0.34 & 12.77 \\ 
&   2 & 0.37 & 0.30 & \textbf{0.21} & \textbf{0.21} & 0.51 & 0.45 & 3.81  \\ 
&   3 & 0.31 & 0.32 &\textbf{ 0.21} & \textbf{0.21} & 0.42 & 0.40 & 9.72 \\ 
 &  5 & 0.33 & 0.29 & \textbf{0.22} & \textbf{0.22} & 0.50 & 0.43 & 38.64  \\ 
 &  9 & 0.44 & 0.39 & 0.34 & \textbf{0.30} & 0.61 & 0.59 & 14.00 \\ 
 &  16 & 0.84 & 0.80 & \textbf{0.69} & \textbf{0.69} & 0.98 & 0.95 & 778.37 \\ 
 &  28 & 2.08 & 1.96 & 1.95 & \textbf{1.91} & 2.27 & 2.14 & $3.10^{3}$ \\ 
 &  50 & 6.57 & 6.35 & 6.56 & \textbf{6.42} & 7.23 & 6.45 & 401.01 \\ 
   \hline
\end{tabular}
\caption{\label{tab::rob::student} {Multivariate Student case: Mean quadratic error} of the estimates of the variance for the different methods and for different contamination scenarios and fractions $\delta$.}
\end{table}


