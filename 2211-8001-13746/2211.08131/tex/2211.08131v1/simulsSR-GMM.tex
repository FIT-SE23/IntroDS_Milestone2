We simulated $B=100$ datasets according to a Gaussian mixture model with each of the parameter configurations described in Section \ref{sec:simDesign}. We only present here the results for a total sample size of $n = 1500$) (that is $n_k = 500$ observations in each group).  We did not observe substantial differences between the results obtained when selecting the number of clusters $K$ with $BIC$ and $ICL$. As a consequence, we only present the results obtained with $BIC$. 

The first two columns of Figure \ref{fig:simGMM1500} compare the results of maximum-likelihood (GMM) inference with the proposed approach (RGMM) in terms of classification. When fixing the number of clusters to its true value $K^* = 3$, we observe a dramatic drop of the classification accuracy of GMM estimation, even for a very moderate fraction of outliers ($\delta=2\%$), as compared to RGMM, in all scenarios. We observe that estimating the number of clusters with $BIC$ improves the classification performances of GMM, at the price of an increase of the number of clusters. On the contrary, the RGMM approach keeps selecting the right number of clusters, even with a medium fraction of outliers $(\delta \sim 10-20\%)$. As a consequence, model selection does not  {significantly} improve the classification accuracy of RGMM. Lastly, we observe that the difference between GMM and RGMM is even more obvious when outliers can each be associated with one cluster, that is under scenarios ($c$) and ($e$), as opposed to scenarios ($b$) and ($d$), respectively. 

\begin{figure}[ht]
  \centering
%   \begin{center}
    \begin{tabular}{c|m{.2\textwidth}m{.2\textwidth}|m{.2\textwidth}m{.2\textwidth}}
      & 
      \multicolumn{2}{c|}{Classification} & 
      \multicolumn{2}{c}{Parameter estimation} \\ 
      &
      \multicolumn{1}{c}{ARI} & \multicolumn{1}{c|}{$\widehat{K}$} & 
      \multicolumn{1}{c}{$MSE(\mu)$} & \multicolumn{1}{c}{$MSE(\Sigma)$} \\ 
      \hline
%       & & & \\
      ($a$) &
      \includegraphics[width=.2\textwidth, trim=10 10 10 50, clip=]{figures/simRGMM-p5-K3-n1500-s1-ARI} &
      \includegraphics[width=.2\textwidth, trim=10 10 10 50, clip=]{figures/simRGMM-p5-K3-n1500-s1-K} &
      \includegraphics[width=.2\textwidth, trim=10 10 10 50, clip=]{figures/simRGMM-p5-K3-n1500-s1-MSEmu} &
      \includegraphics[width=.2\textwidth, trim=10 10 10 50, clip=]{figures/simRGMM-p5-K3-n1500-s1-MSEsigma} \\
      ($b$) &
      \includegraphics[width=.2\textwidth, trim=10 10 10 50, clip=]{figures/simRGMM-p5-K3-n1500-s2-ARI} &
      \includegraphics[width=.2\textwidth, trim=10 10 10 50, clip=]{figures/simRGMM-p5-K3-n1500-s2-K} &
      \includegraphics[width=.2\textwidth, trim=10 10 10 50, clip=]{figures/simRGMM-p5-K3-n1500-s2-MSEmu} &
      \includegraphics[width=.2\textwidth, trim=10 10 10 50, clip=]{figures/simRGMM-p5-K3-n1500-s2-MSEsigma} \\
      ($c$) &
      \includegraphics[width=.2\textwidth, trim=10 10 10 50, clip=]{figures/simRGMM-p5-K3-n1500-s3-ARI} &
      \includegraphics[width=.2\textwidth, trim=10 10 10 50, clip=]{figures/simRGMM-p5-K3-n1500-s3-K} &
      \includegraphics[width=.2\textwidth, trim=10 10 10 50, clip=]{figures/simRGMM-p5-K3-n1500-s3-MSEmu} &
      \includegraphics[width=.2\textwidth, trim=10 10 10 50, clip=]{figures/simRGMM-p5-K3-n1500-s3-MSEsigma} \\
      ($d$) &
      \includegraphics[width=.2\textwidth, trim=10 10 10 50, clip=]{figures/simRGMM-p5-K3-n1500-s4-ARI} &
      \includegraphics[width=.2\textwidth, trim=10 10 10 50, clip=]{figures/simRGMM-p5-K3-n1500-s4-K} &
      \includegraphics[width=.2\textwidth, trim=10 10 10 50, clip=]{figures/simRGMM-p5-K3-n1500-s4-MSEmu} &
      \includegraphics[width=.2\textwidth, trim=10 10 10 50, clip=]{figures/simRGMM-p5-K3-n1500-s4-MSEsigma} \\
      ($e$) &
      \includegraphics[width=.2\textwidth, trim=10 10 10 50, clip=]{figures/simRGMM-p5-K3-n1500-s5-ARI} &
      \includegraphics[width=.2\textwidth, trim=10 10 10 50, clip=]{figures/simRGMM-p5-K3-n1500-s5-K} &
      \includegraphics[width=.2\textwidth, trim=10 10 10 50, clip=]{figures/simRGMM-p5-K3-n1500-s5-MSEmu} &
      \includegraphics[width=.2\textwidth, trim=10 10 10 50, clip=]{figures/simRGMM-p5-K3-n1500-s5-MSEsigma} 
    \end{tabular}
    \caption{Gaussian mixture model: classification accuracy ($ARI$), estimated number of clusters $\widehat{K}$, estimation error fu the mean ($MSE(\mu)$) and for the variance ($MSE(\Sigma)$) for scenarios ($a$) to ($e$), with $n_k = 500$ observation in each of the $K^*$ clusters ($n = 1500$). Black: maximum likelihood (GMM); red: robust estimation (RGMM). Solid line ($\bullet$): with true number of clusters $K^*$; dotted line ($\square$): with number of clusters estimated with $BIC$. \label{fig:simGMM1500}}
%   \end{center}
\end{figure}

The last two columns of Figure \ref{fig:simGMM1500} compare the respective accuracies of GMM and RGMM in terms of parameter estimation. The precision achieved by RGMM is several order of magnitude better than this of GMM, and, except under scenario ($a$), this accuracy remains the same for large contamination fractions (up to $\delta = 50\%$). Again, model selection does not improve the estimation precision of the robust approach. 

% Figure \ref{fig:simGMM150}, given in Appendix \ref{app:simResults}, is the same as Figure \ref{fig:simGMM1500}, but was obtained with $n_k = 200$ observations in each groups (that is $n = 150$). The same conclusions, although less contrasted, can be drawn from it.

Figure \ref{fig:simGMM300}, given in Appendix \ref{app:simResults}, is the same as Figure \ref{fig:simGMM1500}, but was obtained with $n_k = 100$ observations in each cluster (that is $n = 300$). The same conclusions, although less contrasted, can be drawn from it.

% \SR{}{
% \begin{itemize}
% %  \item A la rédaction, il me semble que les figures des ARI avec seulement les non-outlier ne servent à rien. Tu peux les voir en dé-commentant {tabs/appendixTab1500C0} dans l'appendice.
%  \item Je te laisse choisir si on présente $n=150$ ou $=600$ en appendice (en plus de $n=1500$) dans le corps du texte. Idem, il suffit de dé-commenter les lignes correspondantes dans l'appendice.
% \end{itemize}
% }

\FloatBarrier
