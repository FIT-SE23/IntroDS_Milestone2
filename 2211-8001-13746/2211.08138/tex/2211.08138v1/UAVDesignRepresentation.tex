\section{Representation of UAV Designs for AI Consumption}

While the tree representation outlined in the previous section provides an initial level of abstraction, a further level is required before being able to pass a full UAV design through a machine learning model. In designing an appropriate embedding we have two key objectives:
\begin{itemize}
    \item We want to work with sequences to leverage the performance of sequence models.
    \item The sequences must encode metadata of design components such as motor, propeller, and battery properties.
\end{itemize}
Following these two objectives, the first step in achieving our desired embedding is to convert the tree representation into a sequence. The tree to sequence conversion is a preorder traversal that corresponds to a flattening of the tree. In the process of flattening the tree, the parameters of each component are also included. Their order is predetermined via the design language such that we know exactly which arguments are to be expected for each type of component. For example in the conversion of the tree in Figure \ref{fig:tree}, we get the sequence as shown in Figure \ref{fig:seq}. Within this sequence, we see an example of how the arguments of each component can be flattened. First, we use key-value pairs to provide context for each value. For example, the float `\texttt{0.0}' has the key `\texttt{angle}' for context. Second, given the design language we know the order in which to expect certain key-value pairs. For example, the key-value pair `\texttt{\{`node\_type': `PropArm'\}}' means we expect to see the arm length parameter, followed by the motor, propeller, and ESC (electronic speed control) respectively. This predetermined ordering of the hierarchy means that parsing between trees and sequences is simple. Finally, our symbolic representation of design also enables exploitation of symmetry to compress design description. 
For example, if we want all four propeller arms to contain the same subsystem, then we only need to specify it once (as a single child in the design tree) and use a symmetry tag over the hub, for example, `ConnectedHub4\_Sym' denotes a hub with four connections - each having the same subsystem. 
The design assumes that the same subsystem will connect to all the arms.
\begin{figure}[h!]
\centering
\framebox[0.97\columnwidth]{%
\begin{minipage}[t]{0.8\columnwidth}
  % \texttt{\textbf{design\_seq.json:}}
  \scriptsize
  \texttt{
  \shortstack[l]{[\\ \{`node\_type': `ConnectedHub4\_Sym'\},\\
    \{`node\_type': `PropArm'\},\\
    \{`armLength': 210.88760375976562\},\\
     \{`motorType': `t\_motor\_MN2212KV780'\},\\
     \{`propType': `apc\_propellers\_12x5'\},\\
     \{`escType': `t\_motor\_T\_80A'\},\\
     \{`offset': -3.2862548828125\},\\
     \{`offset': 4.2498626708984375\},\\
     \{`angle': 0.0\},\\
     \{`x1\_offset': 4.219192504882812\},\\
     \{`z1\_offset': 3.637290954589844\},\\
     \{`batteryType': `TurnigyGraphene1400mAh3S75C'\}\\ ]}}
\end{minipage}
}
\caption{The sequence representation of a symmetric quadcopter, such as the one in Figure \ref{fig:tree}. We build this sequence by following a preorder traversal of the design tree and ensure that the parameters of each component are included in accordance to the grammar.}\label{fig:seq}
\end{figure}

Our second objective is to encode metadata such as motor, propeller, and battery properties. This means that that a machine learning model is expected to have meaning associated with an input such as \texttt{\{`motorType': `t\_motor\_MN2212KV780'\}}. As a result, when we convert each key-value pair into a token embedding, we encode any associated metadata into the token as well. Overall each token of the sequence contains a one-hot encoding for the 18 classes of keys (e.g. \texttt{motorType}, \texttt{propType}, etc.) and a one-hot encoding for the 671 classes of values (e.g. \texttt{0.0}, \texttt{t\_motor\_T\_80A}, etc.), as well as 51 possible attributes of electrical/mechanical components, and a final dimension for float values. Figure \ref{fig:emb} provides an overview of this conversion.
\begin{figure}[h]
\centering
\includegraphics[width=\columnwidth]{./images/embedding.pdf}
\caption{Shows how each token in the design sequence is embedded into a vector for consumption via the machine learning model. The key and value embeddings follow a one-hot encoding, and the electrical/mechanical metadata is included according to the component being tokenized.}\label{fig:emb}
\end{figure}
