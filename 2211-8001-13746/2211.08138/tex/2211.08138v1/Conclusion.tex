% 2. Emerging Applications of AI

% Emerging applications papers ‘bridge the gap’ between basic AI research and case studies of deployed AI applications, by discussing efforts to apply AI tools, techniques, or methods to real-world problems in novel ways. Emerging applications focus on aspects of AI applications that are not yet sufficiently deployed to be submitted as case studies in the Deployed track.

% This track is distinguished from reports of purely scientific AI research appropriate for AAAI in that the objective of the efforts reported at IAAI should be the potential application of AI technologies, including engineering considerations. A key requirement for papers is to discuss the path forward for achieving deployment of the technology.

% Papers will be judged primarily by the following criteria: significance (of the problem, and the tool or methodology); relevance of AI technology to the problem; innovation; path to deployment; content; evaluation; technical quality; and clarity. Authors are advised to bear these questions in mind while writing their papers. Authors are also advised to consider the novelty of applying AI to the particular problem domain.

% Papers in this track may have up to 6 pages in the prescribed AAAI style, plus at most one more page which may only contain references.
\section{Future Deployment}

All the electrical and mechanical components used in our UAV design pipeline are readily available UAV parts that exist on the market. Therefore, while we are yet to manufacture and deploy physical models of these designs in the real world, we believe that this is the obvious next step in the path towards full deployment. We see two challenges associated with this next step of achieving full deployment. The first challenge is taking a design proposed by our AI-assisted pipeline and ensuring that it is physically possible to build. It is likely that some designs deemed valid according to the high-fidelity simulator may in fact not be feasible designs. This potential inconsistency between the flight simulator and the real-world behavior is often an inevitable part of the prototyping process. 
% Sources of inconsistencies could be attributed to some assumptions that are required to be made due to the complexity of modeling certain real-world behaviors such as the computational fluid dynamics. 
Overcoming this challenge will require human input on our most promising designs to ensure that the location of components, such as electrical, are all able to be connected in a way that is physically realizable. We believe that the fact that we are using real parts and account for interferences between components means that we have mitigated against this challenge to a large degree, however we expect human input to be important at this stage. Finally, the second challenge is the need for a formal way of incorporating more human input. This challenge, which is also linked to the first challenge, will be key going forward for almost all AI-assisted CAD tools across multiple domains. In its current form, our pipeline represents an important and novel first step towards using AI to reduce the computational cost of UAV design. As a result, our AI-assisted pipeline can guide the designer into regions of the design space that they may never have thought of before. We are actively looking into incorporating new objective functions for our transformer model that better suit designer preferences, as well as using new models to better incorporate feedback. 

\section{Conclusion}

The application of AI for the computational design of cyberphysical systems is an emerging domain where the gap between basic AI research and real deployment is becoming narrower. In this paper, we have demonstrated that the use of machine learning approaches can significantly speed up the design process for UAVs. Specifically, in using transformer encoder models trained on UAV design sequences, we have shown the utility of sequence models for design. The novelty of this work is both in our definition of the UAV design language and our deployment of a transformer model. Our UAV design language can represent a huge diversity of designs in a way that can then be tokenized for use in sequence models. Our approach reduced the computational cost of evaluating $100{,}000$ UAV designs from $277.7$ days of compute to $60.6$ days by early identification and removal of poorly performing UAV designs. Finally, our AI-assisted design pipeline led to a large diversity of high-performing UAV designs, of which many do not follow the standard choices that an expert would select (e.g. see Figure \ref{fig:UAVExamples}). As a result, our approach to UAV design opens up the possibility of using AI-tools for developing new designs that could be more efficient and better suited to future task-specific applications. 


% AI-human interaction / Data sets for other AIs?