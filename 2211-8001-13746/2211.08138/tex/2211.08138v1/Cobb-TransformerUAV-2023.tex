%File: anonymous-submission-latex-2023.tex
\documentclass[letterpaper]{article} % DO NOT CHANGE THIS
% \usepackage[submission]{aaai23}  % DO NOT CHANGE THIS
\usepackage{aaai23}
\usepackage{times}  % DO NOT CHANGE THIS
\usepackage{helvet}  % DO NOT CHANGE THIS
\usepackage{courier}  % DO NOT CHANGE THIS
\usepackage[hyphens]{url}  % DO NOT CHANGE THIS
\usepackage{graphicx} % DO NOT CHANGE THIS
\urlstyle{rm} % DO NOT CHANGE THIS
\def\UrlFont{\rm}  % DO NOT CHANGE THIS
\usepackage{natbib}  % DO NOT CHANGE THIS AND DO NOT ADD ANY OPTIONS TO IT
\usepackage{caption} % DO NOT CHANGE THIS AND DO NOT ADD ANY OPTIONS TO IT
\frenchspacing  % DO NOT CHANGE THIS
\setlength{\pdfpagewidth}{8.5in} % DO NOT CHANGE THIS
\setlength{\pdfpageheight}{11in} % DO NOT CHANGE THIS
%
% These are recommended to typeset algorithms but not required. See the subsubsection on algorithms. Remove them if you don't have algorithms in your paper.
\usepackage{algorithm}
\usepackage{algorithmic}

%
% These are are recommended to typeset listings but not required. See the subsubsection on listing. Remove this block if you don't have listings in your paper.
\usepackage{newfloat}
\usepackage{listings}
\DeclareCaptionStyle{ruled}{labelfont=normalfont,labelsep=colon,strut=off} % DO NOT CHANGE THIS
\lstset{%
	basicstyle={\footnotesize\ttfamily},% footnotesize acceptable for monospace
	numbers=left,numberstyle=\footnotesize,xleftmargin=2em,% show line numbers, remove this entire line if you don't want the numbers.
	aboveskip=0pt,belowskip=0pt,%
	showstringspaces=false,tabsize=2,breaklines=true}
\floatstyle{ruled}
\newfloat{listing}{tb}{lst}{}
\floatname{listing}{Listing}
%
% Keep the \pdfinfo as shown here. There's no need
% for you to add the /Title and /Author tags.
\pdfinfo{
/TemplateVersion (2023.1)
}

%ADAM:
\graphicspath{{./images/}}
\usepackage{xcolor}
\usepackage{subcaption}


% DISALLOWED PACKAGES
% \usepackage{authblk} -- This package is specifically forbidden
% \usepackage{balance} -- This package is specifically forbidden
% \usepackage{color (if used in text)
% \usepackage{CJK} -- This package is specifically forbidden
% \usepackage{float} -- This package is specifically forbidden
% \usepackage{flushend} -- This package is specifically forbidden
% \usepackage{fontenc} -- This package is specifically forbidden
% \usepackage{fullpage} -- This package is specifically forbidden
% \usepackage{geometry} -- This package is specifically forbidden
% \usepackage{grffile} -- This package is specifically forbidden
% \usepackage{hyperref} -- This package is specifically forbidden
% \usepackage{navigator} -- This package is specifically forbidden
% (or any other package that embeds links such as navigator or hyperref)
% \indentfirst} -- This package is specifically forbidden
% \layout} -- This package is specifically forbidden
% \multicol} -- This package is specifically forbidden
% \nameref} -- This package is specifically forbidden
% \usepackage{savetrees} -- This package is specifically forbidden
% \usepackage{setspace} -- This package is specifically forbidden
% \usepackage{stfloats} -- This package is specifically forbidden
% \usepackage{tabu} -- This package is specifically forbidden
% \usepackage{titlesec} -- This package is specifically forbidden
% \usepackage{tocbibind} -- This package is specifically forbidden
% \usepackage{ulem} -- This package is specifically forbidden
% \usepackage{wrapfig} -- This package is specifically forbidden
% DISALLOWED COMMANDS
% \nocopyright -- Your paper will not be published if you use this command
% \addtolength -- This command may not be used
% \balance -- This command may not be used
% \baselinestretch -- Your paper will not be published if you use this command
% \clearpage -- No page breaks of any kind may be used for the final version of your paper
% \columnsep -- This command may not be used
% \newpage -- No page breaks of any kind may be used for the final version of your paper
% \pagebreak -- No page breaks of any kind may be used for the final version of your paperr
% \pagestyle -- This command may not be used
% \tiny -- This is not an acceptable font size.
% \vspace{- -- No negative value may be used in proximity of a caption, figure, table, section, subsection, subsubsection, or reference
% \vskip{- -- No negative value may be used to alter spacing above or below a caption, figure, table, section, subsection, subsubsection, or reference

\setcounter{secnumdepth}{0} %May be changed to 1 or 2 if section numbers are desired.

% The file aaai23.sty is the style file for AAAI Press
% proceedings, working notes, and technical reports.
%

% Title

% Your title must be in mixed case, not sentence case.
% That means all verbs (including short verbs like be, is, using,and go),
% nouns, adverbs, adjectives should be capitalized, including both words in hyphenated terms, while
% articles, conjunctions, and prepositions are lower case unless they
% directly follow a colon or long dash
% \title{AI for CAD: Designing Drones with Transformers}
\title{Design of Unmanned Air Vehicles Using Transformer Surrogate Models}
\author{
    %Authors
    % All authors must be in the same font size and format.
    % Written by AAAI Press Staff\textsuperscript{\rm 1}\thanks{With help from the AAAI Publications Committee.}\\
    % AAAI Style Contributions by Pater Patel Schneider,
    % Sunil Issar,\\
    % J. Scott Penberthy,
    % George Ferguson,
    % Hans Guesgen,
    % Francisco Cruz\equalcontrib,
    % Marc Pujol-Gonzalez\equalcontrib
    Adam D. Cobb, Anirban Roy, Daniel Elenius, Susmit Jha
}
\affiliations{
    %Afiliations
    % \textsuperscript{\rm 1}Association for the Advancement of Artificial Intelligence\\
    % If you have multiple authors and multiple affiliations
    % use superscripts in text and roman font to identify them.
    % For example,

    % Sunil Issar, \textsuperscript{\rm 2}
    % J. Scott Penberthy, \textsuperscript{\rm 3}
    % George Ferguson,\textsuperscript{\rm 4}
    % Hans Guesgen, \textsuperscript{\rm 5}.
    % Note that the comma should be placed BEFORE the superscript for optimum readability

    % 1900 Embarcadero Road, Suite 101\\
    % Palo Alto, California 94303-3310 USA\\
    % % email address must be in roman text type, not monospace or sans serif
    % publications23@aaai.org
%
% See more examples next
Computer Science Laboratory, SRI International\\
\{adam.cobb, anirban.roy, daniel.elenius, susmit.jha\}@sri.com
}

%Example, Single Author, ->> remove \iffalse,\fi and place them surrounding AAAI title to use it
\iffalse
\title{My Publication Title --- Single Author}
\author {
    Author Name
}
\affiliations{
    Affiliation\\
    Affiliation Line 2\\
    name@example.com
}
\fi

\iffalse
%Example, Multiple Authors, ->> remove \iffalse,\fi and place them surrounding AAAI title to use it
\title{My Publication Title --- Multiple Authors}
\author {
    % Authors
    First Author Name,\textsuperscript{\rm 1}
    Second Author Name, \textsuperscript{\rm 2}
    Third Author Name \textsuperscript{\rm 1}
}
\affiliations {
    % Affiliations
    \textsuperscript{\rm 1} Affiliation 1\\
    \textsuperscript{\rm 2} Affiliation 2\\
    firstAuthor@affiliation1.com, secondAuthor@affilation2.com, thirdAuthor@affiliation1.com
}
\fi


% REMOVE THIS: bibentry
% This is only needed to show inline citations in the guidelines document. You should not need it and can safely delete it.
\usepackage{bibentry}
% END REMOVE bibentry

\begin{document}

\maketitle

% 2. Emerging Applications of AI

% Emerging applications papers ‘bridge the gap’ between basic AI research and case studies of deployed AI applications, by discussing efforts to apply AI tools, techniques, or methods to real-world problems in novel ways. Emerging applications focus on aspects of AI applications that are not yet sufficiently deployed to be submitted as case studies in the Deployed track.

% This track is distinguished from reports of purely scientific AI research appropriate for AAAI in that the objective of the efforts reported at IAAI should be the potential application of AI technologies, including engineering considerations. A key requirement for papers is to discuss the path forward for achieving deployment of the technology.

% Papers will be judged primarily by the following criteria: significance (of the problem, and the tool or methodology); relevance of AI technology to the problem; innovation; path to deployment; content; evaluation; technical quality; and clarity. Authors are advised to bear these questions in mind while writing their papers. Authors are also advised to consider the novelty of applying AI to the particular problem domain.

% Papers in this track may have up to 6 pages in the prescribed AAAI style, plus at most one more page which may only contain references.



\begin{abstract}
Computer-aided design (CAD) is a promising new area for the application of artificial intelligence (AI) and machine learning (ML). The current practice of design of cyber-physical systems uses the {\it digital twin} methodology, wherein the actual physical design is preceded by building detailed models that can be evaluated by physics simulation models. These physics models are often slow and the manual design process often relies on exploring near-by variations of existing designs. AI holds the promise of breaking these design silos and increasing the diversity and performance of designs by accelerating the exploration of the design space. In this paper, we focus on the design of electrical unmanned aerial vehicles (UAVs). The high-density batteries and purely electrical propulsion systems have disrupted the space of UAV design, making this domain an ideal target for AI-based design. In this paper, we develop an AI Designer that 
% addresses this challenge and 
synthesizes novel UAV designs. Our approach uses a deep transformer model with a novel domain-specific encoding such that we can evaluate the performance of new proposed designs without running expensive flight dynamics models and CAD tools. We demonstrate that our approach significantly reduces the overall compute requirements for the design process and accelerates the design space exploration. Finally, we identify future research directions to achieve full-scale deployment of AI-assisted CAD for UAVs.
\end{abstract}

%----------------------------------------------------------------------
%%% INTRODUCTION
%----------------------------------------------------------------------
% !TEX root = ../Main.tex


\Acp{BPM} have a long and rich history in optimization, going back at least to the introduction of \acl{MD} by Nemirovski \& Yudin \citep{NY83}.
In plain terms, \acp{BPM} are first-order (constrained) optimization algorithms that forego Euclidean projections in favor of a more sophisticated ``prox-mapping'' that minimizes a certain distance-like functional known as the Bregman divergence \citep{NY83,CT93,Bre67,Kiw97}.
When this Bregman divergence is the Euclidean distance squared, one recovers the standard projection-based methods;
other than that, depending on the problem's feasible region, different Bregman setups lead to a diverse collection of algorithms,
from exponentiated gradient descent on the simplex \citep{NY83,BecTeb03,ACBFS02},
to matrix multiplicative weights on the positive-semidefinite cone \cite{TRW05,KSST12},
variants of Karmarkar's affine scaling algorithm for linear programs \cite{VMF86},
etc.

One of the most appealing features of \acp{BPM} is that they achieve almost dimension-free convergence rates in problems with a convex structure and a favorable geometry \textendash\ such as the $L^{1}$ ball, the spectraplex, second-order cones, etc. \cite{Bub15,Nes09,BecTeb03}.
This is owed to a delicate interplay between the algorithms' non-Euclidean update scheme and the global geometry of the problem's domain.
However, these (almost) dimension-free guarantees also come with some strings attached:
they do not concern the sequence of iterates generated by the method, but only its time average
\revise{(or, through the same, ``regret-based'' analysis, the method's ``best iterate'')};
in this way, the best guarantee that can be achieved after $\run$ iterations is $\bigoh(1/\run)$.

In terms of oracle complexity, this is sufficient for problems that are not strongly convex\,/\,strongly monotone, but if one targets finer, geometric convergence rates,
\revise{the inherent averaging involved in regret-based guarantees is hard to compensate.}
And, on the other extreme, if the problem is not convex\,/\,monotone to begin with, iterate averaging does not provide any quantifiable benefits whatsoever, so it becomes crucial to study the actual trajectory of the method.


%----------------------------------------------------------------------
%%% CONTRIBS
%----------------------------------------------------------------------
\para{Our contributions}

Our paper seeks to quantify the last-iterate convergence rate of \aclp{BPM} as a function of the Bregman divergence defining the method and the local geometry that it induces.
To treat this question in as general a manner as possible, we focus on \ac{VI} problems of the form
\begin{equation}
\label{eq:VI}
\tag{VI}
\text{Find $\sol\in\points$ such that}
	\;\;
	\braket{\vecfield(\sol)}{\point - \sol}
	\geq 0
	\;\;
	\text{for all $\point\in\points$},
\end{equation}
where $\points$ is a closed convex subset of a finite-dimensional normed space $\pspace$, and $\vecfield \from \points \to \dspace$ is a (possibly non-monotone) single-valued operator on $\points$ with values in $\dspace$, the dual of $\pspace$.
This problem is a staple of many areas of mathematical programming, game theory and data science, as it provides a template for ``optimization beyond minimization'' \textendash\ \ie for problems where finding an optimal solution does not necessarily involve minimizing a loss function.
In particular, in addition to standard minimization problems \textendash\ which are recovered when $\vecfield = \nabla\obj$ for some smooth function $\obj$ \textendash\ the general formulation \eqref{eq:VI} includes saddle-point problems, games, complementarity problems, etc.;
for an introduction, see \cite{FP03} and references therein.

In this broad context, we examine the rate of convergence of a wide class of \aclp{BPM} to local solutions of \eqref{eq:VI} that satisfy a \acl{SOS} condition.
Specifically, the class of algorithms we consider includes as special cases
\begin{enumerate*}
[(\itshape i\hspace*{1pt}\upshape)]
\item
the original \acf{MD} algorithm of \cite{NY83};
\item
the \acf{MP} method of Nemirovski \cite{Nem04} \textendash\ which has the same update structure as the Bregman-based algorithm of \cite{AT05} and contains as a special case the \acf{EG} algorithm of \cite{Kor76};
\item
the so-called \acf{OMD} method of \cite{RS13-NIPS} \textendash\ itself a Bregman analogue of the modified Arrow-Hurwicz algorithm of \cite{Pop80};
\end{enumerate*}
etc.

Our first finding is a crisp characterization of last-iterate convergence rate of \acp{BPM} in terms of the local geometry induced by the underlying Bregman function near a given solution of \eqref{eq:VI}.
We make this dependence precise via the notion of the \emph{Legendre exponent}, a regularity measure for Bregman methods due to \cite{AIMM21}, which can roughly be described as the logarithmic ratio of the volume of a Euclidean ball to that of a Bregman ball of the same radius.
For example, Euclidean methods have a Legendre exponent of $\legexp = 0$ and they converge at a linear rate;
entropic methods have a Legendre exponent of $\legexp = 1/2$ at boundary points, and they converge at a rate of $\bigoh(\run^{-1})$;
more generally,
as we show in \cref{thm:general}, methods with a Legendre exponent $\legexp>0$ converge at a rate of $\bigoh(\run^{1-1/\legexp})$.
\PM{We need to fix this: the $1-1/\legexp$ exponent is not consistent with the $\bigoh(1/\run)$ expression.}
\WA{I don't see the issues, yes this expression is not well-defined for $\legexp = 0$ but this is normal, the two situations differ radically.}
The Euclidean regime ($\legexp = 0$) is perfectly aligned with existing results for the geometric last-iterate convergence rate of the \ac{EG} algorithm and its variants \citep{GBVV+19,Mal15,HIMM19,MOP20}.
By contrast, the Legendre regime ($\legexp > 0$) indicates a significant drop in the algorithm's last-iterate convergence speed, even though ergodic convergence rates \cite{Nes04} and results for bilinear games \cite{WLZL21} might suggest otherwise.

Subsequently, motivated by applications to game theory and linear programming, we take a closer look at the convergence rate of \acp{BPM} across the constraints that are active at a solution $\sol$ of \eqref{eq:VI} depending on the position of $\vecfield(\sol)$ relative to said constraints. 
This analysis reveals that Bregman proximal methods have a particularly fine structure:
along \emph{sharp directions} (\ie constraints along which $\vecfield(\sol)$ is strictly inward-pointing), \acp{BPM} converge
\begin{enumerate*}
[(\itshape i\hspace*{1pt}\upshape)]
\item
at a rate of $\bigoh(1/\run^{1/(2\legexp-1)})$ if $1/2 < \legexp < 1$;
\item
at a \emph{geometric rate} if $0 < \legexp \leq 1/2$ (\eg for entropic methods);
and
\item
in a \emph{finite} number of iterations if $\legexp=0$
\end{enumerate*}
(\cf \cref{thm:sharp}).
Thus, even though the estimates of \cref{thm:general} are, in general tight, the actual convergence rate of a Bregman method along different coordinates\,/\,constraints could be starkly different \textendash\ and, in fact, dramatically faster if the solution under study is itself sharp.

The closest antecedent of our work is the conference paper \cite{AIMM21} where the Legendre exponent was introduced to analyze the convergence of \ac{OMD} in \emph{stochastic} \ac{VI} problems (without considering sharp directions and/or faster identification rates).
The stochastic and deterministic settings are obviously very different, both in the challenges involved as well as the rates obtained, so there is no overlap in our analysis and results.
Other than that, we are not aware of any comparable results in the literature concerning the radically different convergence landscape of \acp{BPM} along active and inactive constraints.

\section{Towards Practical UAV design}

UAVs are an exciting technology that will pave the way for many unforeseen useful future use-cases. Even with today's technology, we already see a significant variation in UAV designs from food delivery/logistics \cite{gu2020vehicle}, to the detection of sharks along highly populated beaches \cite{kiszka2016using}, to air taxis. As a result, one single optimized design will not be enough for all applications. Furthermore, for each application a designer will be required to go through the same design process, but with a different set of objectives. Unfortunately, the design process of UAVs is complex and many of the design choices are highly coupled resulting in non-linear relationships. One example of such a design choice is in the selection of the motor, propeller, battery combination. All three components are intrinsically linked and therefore it is not necessarily clear in which order to select these components (see our previous work \cite{cobb2021physical}). Such coupled relationships make UAV design especially challenging, even for domain experts.

CAD plays an important role in the design of cyberphysical systems, and for UAVs this is no different. A large part of the design process is not only making important design (topological/parametric) decisions, but is also simulating designs. Once an agent, generally a human, uses their experience to propose a design, the next stage is to simulate the performance using flight dynamics models \cite{Bapty2022design, walker2022flight}. Depending on the fidelity of such models and the accuracy desired from the designer, such models can take anything from a few minutes to a few days to evaluate. Finally, once a designer is happy with their design, they must then go through the manufacturing process, prototype, do real world safety checks etc. The focus of our work is on the CAD component of the design process. Automating the evaluation of each proposed design by avoiding the use of high-fidelity models is of high value. In fact machine learning models even allow for parallelization where one can pass a batch of designs into a machine learning model in order to evaluate their performance in a matter of seconds. Replacing such simulators with machine learning models is often referred to as surrogate modeling. 

One reason why the path forward to automating the design of UAVs is now achievable is due to the recent success of  deep learning models for structured data. Unlike for fully-connected neural networks and convolutional neural networks, models such as transformers \cite{vaswani2017attention} and graph neural networks \cite{wu2020comprehensive} can ingest highly structured data. This data format is more suitable to designing cyberphysical systems such as UAVs. In our work, we use a design language where UAV designs naturally follow a tree structure. This is a structure that is also prevalent for other cyberphysical systems such as in robotics \cite{zhao2020robogrammar}. As an example, Figure \ref{fig:tree} shows how a quadcopter (4-propeller UAV) can be represented as a tree, with the full design being shown in the top left corner. While this is a simple example, we are able to represent a huge diversity of designs with this tree representation as demonstrated in Figure \ref{fig:uavs}. 
\begin{figure}[h]
\centering
\includegraphics[width=\columnwidth]{./images/UAVtreesym.pdf}
\caption{A pictorial example of our UAV tree representation of a symmetric quadcopter. Since the design is symmetric only one propeller arm needs to be defined, which is then repeated four times when compiled to the full model as shown in the top left.}\label{fig:tree}
\end{figure}

\begin{figure}[h]
\centering
\includegraphics[width=\columnwidth]{./images/UAVs.pdf}
\caption{A sample of UAV designs that we are able to build with our grammar to demonstrate the representational power of our design language.}\label{fig:uavs}
\end{figure}

\section{Representation of UAV Designs for AI Consumption}

While the tree representation outlined in the previous section provides an initial level of abstraction, a further level is required before being able to pass a full UAV design through a machine learning model. In designing an appropriate embedding we have two key objectives:
\begin{itemize}
    \item We want to work with sequences to leverage the performance of sequence models.
    \item The sequences must encode metadata of design components such as motor, propeller, and battery properties.
\end{itemize}
Following these two objectives, the first step in achieving our desired embedding is to convert the tree representation into a sequence. The tree to sequence conversion is a preorder traversal that corresponds to a flattening of the tree. In the process of flattening the tree, the parameters of each component are also included. Their order is predetermined via the design language such that we know exactly which arguments are to be expected for each type of component. For example in the conversion of the tree in Figure \ref{fig:tree}, we get the sequence as shown in Figure \ref{fig:seq}. Within this sequence, we see an example of how the arguments of each component can be flattened. First, we use key-value pairs to provide context for each value. For example, the float `\texttt{0.0}' has the key `\texttt{angle}' for context. Second, given the design language we know the order in which to expect certain key-value pairs. For example, the key-value pair `\texttt{\{`node\_type': `PropArm'\}}' means we expect to see the arm length parameter, followed by the motor, propeller, and ESC (electronic speed control) respectively. This predetermined ordering of the hierarchy means that parsing between trees and sequences is simple. Finally, our symbolic representation of design also enables exploitation of symmetry to compress design description. 
For example, if we want all four propeller arms to contain the same subsystem, then we only need to specify it once (as a single child in the design tree) and use a symmetry tag over the hub, for example, `ConnectedHub4\_Sym' denotes a hub with four connections - each having the same subsystem. 
The design assumes that the same subsystem will connect to all the arms.
\begin{figure}[h!]
\centering
\framebox[0.97\columnwidth]{%
\begin{minipage}[t]{0.8\columnwidth}
  % \texttt{\textbf{design\_seq.json:}}
  \scriptsize
  \texttt{
  \shortstack[l]{[\\ \{`node\_type': `ConnectedHub4\_Sym'\},\\
    \{`node\_type': `PropArm'\},\\
    \{`armLength': 210.88760375976562\},\\
     \{`motorType': `t\_motor\_MN2212KV780'\},\\
     \{`propType': `apc\_propellers\_12x5'\},\\
     \{`escType': `t\_motor\_T\_80A'\},\\
     \{`offset': -3.2862548828125\},\\
     \{`offset': 4.2498626708984375\},\\
     \{`angle': 0.0\},\\
     \{`x1\_offset': 4.219192504882812\},\\
     \{`z1\_offset': 3.637290954589844\},\\
     \{`batteryType': `TurnigyGraphene1400mAh3S75C'\}\\ ]}}
\end{minipage}
}
\caption{The sequence representation of a symmetric quadcopter, such as the one in Figure \ref{fig:tree}. We build this sequence by following a preorder traversal of the design tree and ensure that the parameters of each component are included in accordance to the grammar.}\label{fig:seq}
\end{figure}

Our second objective is to encode metadata such as motor, propeller, and battery properties. This means that that a machine learning model is expected to have meaning associated with an input such as \texttt{\{`motorType': `t\_motor\_MN2212KV780'\}}. As a result, when we convert each key-value pair into a token embedding, we encode any associated metadata into the token as well. Overall each token of the sequence contains a one-hot encoding for the 18 classes of keys (e.g. \texttt{motorType}, \texttt{propType}, etc.) and a one-hot encoding for the 671 classes of values (e.g. \texttt{0.0}, \texttt{t\_motor\_T\_80A}, etc.), as well as 51 possible attributes of electrical/mechanical components, and a final dimension for float values. Figure \ref{fig:emb} provides an overview of this conversion.
\begin{figure}[h]
\centering
\includegraphics[width=\columnwidth]{./images/embedding.pdf}
\caption{Shows how each token in the design sequence is embedded into a vector for consumption via the machine learning model. The key and value embeddings follow a one-hot encoding, and the electrical/mechanical metadata is included according to the component being tokenized.}\label{fig:emb}
\end{figure}


\section{UAV Procedural Generator}

In the previous two sections we defined a tree representation and a corresponding procedure to convert from the tree to a sequence. An important and necessary advantage of defining a logical UAV grammar, is that we can build a procedural generator that generates UAV design trees with minimal compute. In fact, we can sample from our generator, by sampling its stochastic parameters to draw hundreds of thousands of designs in the order of minutes. Therefore, design generation becomes trivial, but design evaluation remains a challenge due to the computational cost of running the UAV simulation software.

We note that defining a language for each cyberphysical system requires significant domain expertise. A procedural generator requires a suitable design representation, e.g. a tree, and a broad enough vocabulary to capture a large diversity of designs. In our development of our procedural generator (or declarative probabilistic program), we started by initially generating variants of the quadcopter (see Figure \ref{fig:tree}) and iterated towards more complex designs that included wings and a multitude of different component combinations. In summary, our procedural generator can generate a variety of designs that are topologically valid. By \textit{topologically valid}, we mean that they can be evaluated by the available scientific models. However, they may not be structurally sound or able to fly. In the next section we demonstrate how machine learning approaches can aid in producing working UAV designs.

\section{Transformers as Surrogate Models for UAV Design}

As alluded to earlier, we are focused on speeding up the computational stage of the UAV design process. Therefore, rather than relying on expensive flight simulation models, our aim is to reduce the cost of computation time to the order of seconds per design. Reducing this computational cost will facilitate faster design exploration and will therefore open up the UAV design domain to further machine learning exploratory approaches.  

\subsection{Flight Dynamics Model}

The performance of each UAV is assessed using a pipeline comprising CAD tools such as Creo~\cite{creo} and a custom flight dynamics model (FDM)~\cite{walker2022flight,Bapty2022design}. Each UAV is assessed on controllability (existence of trim states) at different speeds. In particular, the FDM evaluates whether the UAV can fly at a specific velocity by adjusting the controls and orientation of a vehicle until the state of the vehicle reaches the desired value. These adjustments are achieved numerically, using the MINPACK package and the nonlinear simplex algorithm (see \citet{walker2022flight} for more details). Before evaluating the design in the FDM, each design must also be compiled and evaluated in a solid modeling computer-aided design tool, such as creo. This tool provides information to the FDM such as the overall mass, the moments of inertia, and potential interferences between parts. The output of the FDM provides a range of UAV statistics, such as trim states and electrical performance (power, current, voltage etc.).

\subsection{Objective}

As design is an iterative process, and we have the capability of producing hundreds of thousands of topologically valid designs, one might want to initially filter through these designs (quickly) to find a subset that meet designer specifications. A first step in this process is to find UAV designs that are able to fly or hover. As a result, we define the first objective as one where we aim to predict whether a design can hover, as this is often a vital characteristic of UAVs.\footnote{We note that some UAVs, such as many fixed wing UAVs, are launched (or catapulted) and therefore are not required to hover. For this class of UAVs, we would choose a different objective for filtering.} We therefore use a binary cross-entropy loss as the objective, whereby we label a design as $y = 1$ for hover times greater than $0$ and $y = 0$ for hover times of $0$.


\subsection{Transformer Model}

We will now introduce our transformer model that will operate over the sequence of token embeddings as highlighted in Figure \ref{fig:emb}. One key innovation of modern deep learning approaches is in the ability of some architectures to ingest structured data. In this paper so far, we have shown how we were able to define a flexible UAV grammar that allows us to represent a large diversity of designs as sequences. Transformers \cite{vaswani2017attention} are well-known to be extremely effective at operating on sequence data. They have seen huge success as part of large language models \cite{devlin2018bert, brown2020language} and we will now show how our UAV design embedding can bring success to the cyberphysical domain. 

In order to classify whether a design can hover, we build a transformer encoder that consists of a linear encoding layer followed by a positional encoding layer. The output of the positional encoding is then passed through a transformer encoding layer (\texttt{nn.TransformerEncoder} in PyTorch) with 8 layers and 2 heads. The final token from the transformer encoder is then passed through a linear layer that goes from the embedding dimension of 200 to a single output dimension. During training we use the PyTorch in-built SGD optimizer with a learning rate of 0.01 and set the number of epochs to 2500. 
% As a baseline comparison for predicting over sequence data, we also built an LSTM model \cite{hochreiter1997long} that consists of a linear input layer followed by an LSTM layer (\texttt{nn.LSTM} in PyTorch) with an input size of 512, a hidden size of 512.

\subsection{Initial Data and Results}

For the initial training of our model, we require a labeled data set. We therefore run the full CAD pipeline for $6{,}352$ UAV designs, where each design is sampled randomly from the procedural generator. Within this data set, only $794$ UAVs have a hover time greater than $0$~s. While computationally expensive, this initial data set is large enough to train our transformer model. We set aside $80~\%$ of the data for training and $20~\%$ for test. The resulting performance of our transformer model is an accuracy of $93.6~\%$. However, at this accuracy the recall for hovering UAVs is $0.63$, which means that we would miss around $40 \%$ of designs with this desired characteristic in the downstream tasks. Therefore, with careful calibration by changing the classification threshold from $0.5$ to $0.15$, we can achieve a recall of $0.88$ at the expense of a precision of $0.53$. When we describe the pipeline in the next section, a threshold of $0.15$ will result in the collection of a balanced data set of diverse hovering UAVs. Figure \ref{fig:PRCurve} provides further context for our threshold choice. The plot shows the precision-recall curve for the transformer encoder, where we have highlighted our chosen precesion-recall threshold with a red star. Depending on the recall, precision, and available compute, one could select a threshold accordingly by analyzing this graph.

\begin{figure}[h!]%{0.7\columnwidth}
\centering
    \includegraphics[width=\columnwidth]{./images/PR-IAAI.pdf}
    \caption{Precision-recall curve for the transformer encoder. The graph highlights the relative performance compared to a random classifier and indicates our chosen precision-recall threshold with the red star. } 
    %(hover time and flight distance is shown along with each design).
    \label{fig:PRCurve}
\end{figure}

\subsection{Design Pipeline and Experimental Results}

We can now describe the entire design pipeline and our bootstrapping method for jointly improving overall UAV design and our surrogate models. Figure \ref{fig:pipeline} outlines the three staged process. In stage 1, highlighted in blue, we collect the initial labeled data. This requires sampling trees from our procedural generator and then flattening these trees into the appropriate format for the scientific models. We then run the scientific models (creo and the FDM) to evaluate the performance to build the initial data set. In stage 2, highlighted in orange, we train our transformer model as described in the previous section. Finally in stage 3, highlighted in yellow, we use our transformer model to filter through designs as sampled from the procedural generator in order to build a higher quality data set.

\begin{figure}[h]
\centering
\includegraphics[width=\columnwidth]{./images/Pipeline.pdf}
\caption{Graphic to provide an overview of the UAV design pipeline. The initial stage is to build a data set by sampling from the stochastic procedural generator (the declarative probabilistic program) in order to build a data set via running these designs through the scientific flight dynamics models. The next stage is to train the transformer on this data. In the last stage, we use the trained transformer to perform rejection sampling over the procedural generator to help find higher-quality designs.}\label{fig:pipeline}
\end{figure}

To demonstrate our pipeline, we sample $100{,}000$ new designs from the procedural generator and then use the transformer encoder with a threshold of $0.15$ to filter out designs that are predicted to be unable to hover. The result is $21{,}800$ UAV designs. This result is consistent with the performance over the validation data which reported an expected precision of around $50~\%$. Given that randomly sampling from the generator provides roughly $10~\%$ of designs that can hover and our model has a precision of $0.53$ and a recall of 0.88, we would expect to be left with around $20{,}000$ UAV designs of which $10{,}000$ should be able to hover. This result is significant as each evaluation through the scientific models takes 4-10 minutes depending on the complexity of the design. We can estimate a lower bound on the compute time by assuming 4 minutes per FDM evaluation, then $100{,}000$ UAVs would take $277.7$ days of compute, compared to $60.6$ days. 

To determine the success of our filtering approach, we evaluated a subset of $6{,}621$ UAV designs (out of the proposed $21{,}800$) using the scientific models. It is at this point where we see the generalization of our approach. Of these $6{,}621$ designs, $3{,}308$ met the desired objective of being able to hover. This new data set therefore consists of $50 \%$ of UAVs that have the ability to hover compared to the $12.5 \%$ from the previous iteration (stage 1 in Figure \ref{fig:pipeline}). This result is consistent with the performance over the validation set, where we also saw a comparable precision of $0.53$.
One potential concern is that the transformer encoder could identify UAV designs that are less diverse. However we show with both quantitative and qualitative results that a lack of diversity is not an issue. For quantitative results we refer to Figure \ref{fig:bars}. Figure \ref{fig:bars} presents the distribution of propellers (\ref{fig:props}) and wings (\ref{fig:wings}) for all hovering UAV designs that were filtered out via stage 3. We can see that the designs retrieved using our transformer model have a large range of values for both number of propellers and number of wings. Notably, we see from Figure \ref{fig:props} that a large proportion of the UAVs that hover have an even number of propellers. This observation meets what we see in practice with many UAVs taking on the form of quadcopters and hexcopters. However, this design process also provided some novel designs such as a 13 propeller UAV (or \textit{``tridecacopter''}) with a hover time of $204$~s. We see a similar pattern in Figure \ref{fig:wings} where there are more hovering UAVs with an even number of wings than odd. We further note that the vast majority of hovering UAV designs do not have wings. We postulate that the objective requirement of just being able to hover does not favor the inclusion of wings compared to other objectives such as maximum distance. For qualitative results, we select a range of filtered designs and present them in Figure \ref{fig:UAVExamples}. Importantly the UAVs in this figure highlight the diversity in designs that were retrieved via our AI-assisted UAV design process. These designs all meet the criteria of being able to hover, as well as having some other favorable performance metrics that were not previously specified such as reasonable flight distances. While some of these designs conceivably could have been designed by domain experts, such as the hexcopters, many designs would not have been likely expert choices. Therefore, our AI-assisted design process could lead to many more interesting new designs that may not have been considered before.

\begin{figure}
     \centering
     \begin{subfigure}[b]{0.9\columnwidth}
         \centering
         \includegraphics[width=\textwidth]{./images/prop_bar_fly.pdf}
         \caption{\# Propellers in UAV designs that can hover.}
         \label{fig:props}
     \end{subfigure}
     \hfill
     \begin{subfigure}[b]{0.9\columnwidth}
         \centering
         \includegraphics[width=\textwidth]{./images/wing_bar_fly.pdf}
         \caption{\# Wings in UAV designs that can hover.}
         \label{fig:wings}
     \end{subfigure}
        \caption{Quantitative results of the filtering process. After using the transformer encoder, we are left with a diverse range of UAV designs that can hover. We see designs ranging from two propellers to thirteen (a) and designs with number of wings ranging from none to 4 (b).}
        \label{fig:bars}
\end{figure}

\begin{figure*}[h!]%{0.7\columnwidth}
\centering
    \includegraphics[width=0.75\textwidth]{./images/ColoredFlyingExamples.pdf}
    \caption{A subset of UAV designs that were found via the transformer-assisted design pipeline. The high proportion of UAV designs that meet the objective of being able to hover are also diverse. Many designs are novel and do not follow the standard choices that an expert UAV designer would necessarily select. } 
    %(hover time and flight distance is shown along with each design).
    \label{fig:UAVExamples}
\end{figure*}

We now ask how does this new data set help with the design process and what does it mean for the future of UAV design and the design of cyberphysical systems in general. The designs in Figure \ref{fig:UAVExamples} would not have been realistically possible to compute without the use of AI, namely the transformer encoder. The ability of the transformer to accurately predict whether a design would be able to hover without the need of running the scientific models is a big step in the process of automating the UAV design process. The transformer reduces the $1$ in $10$ chance of evaluating a hovering design to $1$ in $2$, and saves significant compute time. As a result of this saving, we enable the possibility of evaluating many more interesting novel designs, such as those shown in the Figure. For future UAV design our process as described in this paper has the potential to facilitate even more sophisticated machine learning approaches that are able to leverage the higher quality data sets that our approach produces. More generally, our case study demonstrates how a cyberphysical system such as a UAV can be represented as a tree structure using a domain-specific design language. In highlighting this demonstration, we believe other cyberphysical systems could be represented in flexible grammars that could then utilize sequence-based transformer approaches as shown here.


\section{Related Work}

The use of machine learning in computer aided design (CAD) has gained significant attention, and only a few works have been proposed in recent literature that develop machine learning approaches. SketchGraphs dataset~\cite{seff2020sketchgraphs} is a collection of sketches extracted from parametric CAD models which begin as two-dimensional (2D) sketches consisting of geometric primitives (e.g., line segments, arcs) and explicit constraints between them (e.g., coincidence, perpendicularity) that form the basis for three-dimensional (3D) construction operations. This dataset has been used for generative model of CAD sketches~\cite{willis2021engineering}, and other applications of learning in physical design~\cite{seff2021vitruvion, para2021sketchgen}.
Building a design grammar has been the subject of other works such as in \citet{zhao2020robogrammar}. There are also many other works that have focused on specific parts such as robot arms \cite{xu2021end} and airfoil design \cite{chen2022learning}. Two key differences between our approach to design and these previous works is: (1) Our focus on a pathway towards deployment by using physically realizable components; (2) The use of transformer models on our design embeddings to directly evaluate performance.


% While these works have seen a lot of success, they are not focused on the pathway towards actual deployment, whereas the focus in our work is to eventually build physically realizable designs by building a procedural generator and transformer model that operate over real components.


% Another example of a CAD dataset that is focused on physical structure is SimJEB \cite{whalen2021simjeb}, which is a dataset of crowdsourced mechanical brackets and accompanying structural simulations. DeepCAD~\cite{wu2021deepcad} is a dataset of 3D shapes corresponding to objects such as flanges, pipes and screws, represented as a sequence of operations used in a CAD framework to generate these shapes. Another dataset for 3D engineering shapes is the ABC dataset~\cite{koch2019abc}, 
% which comprises geometric models, each defined by parametric surfaces and
% associated with accurate ground truth information on the decomposition into patches. Their representation allows resampling the surface data at arbitrary resolutions into a point cloud or mesh. These datasets are excellent resources for their target application domains such as extrapolating 2D sketch to CAD designs, and generating mechanical parts.


% In another line of related work, surrogate-based optimization is widely explored in design optimization, where the goal is to learn a surrogate function to replace often expensive black-box simulators e.g., computational fluid dynamics simulators~\citep{koziel2011surrogate, han2012surrogate, viquerat2021direct}. The surrogate function aims to capture the physical properties of the design environment and reliably evaluate design samples. These approaches tend to be more scalable compared to the black-box optimization approaches~\citep{greenhill2020bayesian, belakaria2020uncertainty, deshwal2021bayesian} by avoiding the expensive black-box evaluation during optimization. Further, if the surrogate function is differentiable e.g., a neural network, the gradients are also available to the optimizer to perform an end-to-end optimization \cite{grabocka2019learning, liu2020unified, sun2021amortized}.
% Our proposed method can leverage these advances in better surrogate modeling for more efficient exploration.

% In contrast to existing methods, the design for physical systems needs to find a diverse set of designs that trade off different objectives and allow further downstream adaptation to new design objectives.  

% !TeX spellcheck = en_GB
%!TEX root = ../side-constrained.tex

\section{Conclusion}

We provided a counterexample to a claimed existence result for dynamic equilibria with side constraints. The implications of this counterexample were shown to be severe since solutions to the canonical infinite dimensional variational inequality are in some sense useless and other approaches seem to be necessary. 
We then established a general framework for defining side-constrained dynamic equilibria based on two key objects: A \setS{} $S$ containing all feasible flows (given as walk inflows) and correspondences $A_p$ providing the flow-dependent set of \addmEpsDev s. We showed that this equilibrium concept not only encompasses the known unconstrained equilibria with and without departure time choice and capacitated dynamic equilibria with convex \setS{}s but also allows for a whole range of new dynamic equilibria inspired by static side-constrained equilibria.
We provided conditions under which they can be characterized as solutions to a quasi-variational or even a variational inequality. The latter characterization then also gave rise to a first existence result for certain side-constrained dynamic equilibria with convex \setS.
Finally, we turned to equilibria wherein the side-constraints are given by time-varying edge-load constraints. To deal with the non-convexity of the \setS{}, we employed an augmented Lagrangian approach by relaxing the hard edge-load-capacities and replacing them by penalty functions. We demonstrated that these existence results apply, in particular, for the widely used Vickrey point queue model as well as the linear edge delay model.

Several important questions remain open. First of all, it would be interesting to find an existence result for BSDE similar to \Cref{thm:ExistenceFDAddSpaceExCP} for LPDE and MNSDE. The main obstacle to obtaining such a result seems to be the fact that for BSDE, the definition of \addmEpsDev s involves the network loading which, in general, is a very complex mapping and, even for well-studied flow models, is not fully understood yet. Note that, due to \Cref{prop:RelationshipsOfCDE}, such a result would also directly imply existence of \globalEL{} as well as providing an alternative proof for the existence of LPDE. Another aspect is the multiplicity of equilibria and
the issue of selecting a particular type of equilibrium having desirable properties.
It is an interesting research direction to characterize equilibrium concepts
that admit equilibrium selection via appropriate optimization or optimal control reformulations
whose optimal solutions provide such desirable properties.


\section*{Acknowledgments}
%This material is based upon work supported by the United States Air Force and DARPA under Contract No. FA8750-20-C-0002.  Any opinions, findings and conclusions or recommendations expressed in this material are those of the author(s) and do not necessarily reflect the views of the United States Air Force and DARPA.
This project was supported by DARPA under the Symbiotic
Design for Cyber-Physical Systems (SDCPS) with contract 
FA8750-20-C-0002. 
The views, opinions and/or findings expressed
are those of the author and should not be interpreted as
representing the official views or policies of the Department
of Defense or the U.S. Government.


% Use \bibliography{yourbibfile} instead or the References section will not appear in your paper
\bibliography{aaai23}


\end{document}
