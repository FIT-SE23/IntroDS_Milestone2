\section{Towards Practical UAV design}

UAVs are an exciting technology that will pave the way for many unforeseen useful future use-cases. Even with today's technology, we already see a significant variation in UAV designs from food delivery/logistics \cite{gu2020vehicle}, to the detection of sharks along highly populated beaches \cite{kiszka2016using}, to air taxis. As a result, one single optimized design will not be enough for all applications. Furthermore, for each application a designer will be required to go through the same design process, but with a different set of objectives. Unfortunately, the design process of UAVs is complex and many of the design choices are highly coupled resulting in non-linear relationships. One example of such a design choice is in the selection of the motor, propeller, battery combination. All three components are intrinsically linked and therefore it is not necessarily clear in which order to select these components (see our previous work \cite{cobb2021physical}). Such coupled relationships make UAV design especially challenging, even for domain experts.

CAD plays an important role in the design of cyberphysical systems, and for UAVs this is no different. A large part of the design process is not only making important design (topological/parametric) decisions, but is also simulating designs. Once an agent, generally a human, uses their experience to propose a design, the next stage is to simulate the performance using flight dynamics models \cite{Bapty2022design, walker2022flight}. Depending on the fidelity of such models and the accuracy desired from the designer, such models can take anything from a few minutes to a few days to evaluate. Finally, once a designer is happy with their design, they must then go through the manufacturing process, prototype, do real world safety checks etc. The focus of our work is on the CAD component of the design process. Automating the evaluation of each proposed design by avoiding the use of high-fidelity models is of high value. In fact machine learning models even allow for parallelization where one can pass a batch of designs into a machine learning model in order to evaluate their performance in a matter of seconds. Replacing such simulators with machine learning models is often referred to as surrogate modeling. 

One reason why the path forward to automating the design of UAVs is now achievable is due to the recent success of  deep learning models for structured data. Unlike for fully-connected neural networks and convolutional neural networks, models such as transformers \cite{vaswani2017attention} and graph neural networks \cite{wu2020comprehensive} can ingest highly structured data. This data format is more suitable to designing cyberphysical systems such as UAVs. In our work, we use a design language where UAV designs naturally follow a tree structure. This is a structure that is also prevalent for other cyberphysical systems such as in robotics \cite{zhao2020robogrammar}. As an example, Figure \ref{fig:tree} shows how a quadcopter (4-propeller UAV) can be represented as a tree, with the full design being shown in the top left corner. While this is a simple example, we are able to represent a huge diversity of designs with this tree representation as demonstrated in Figure \ref{fig:uavs}. 
\begin{figure}[h]
\centering
\includegraphics[width=\columnwidth]{./images/UAVtreesym.pdf}
\caption{A pictorial example of our UAV tree representation of a symmetric quadcopter. Since the design is symmetric only one propeller arm needs to be defined, which is then repeated four times when compiled to the full model as shown in the top left.}\label{fig:tree}
\end{figure}

\begin{figure}[h]
\centering
\includegraphics[width=\columnwidth]{./images/UAVs.pdf}
\caption{A sample of UAV designs that we are able to build with our grammar to demonstrate the representational power of our design language.}\label{fig:uavs}
\end{figure}