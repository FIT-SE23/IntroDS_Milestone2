\section{UAV Procedural Generator}

In the previous two sections we defined a tree representation and a corresponding procedure to convert from the tree to a sequence. An important and necessary advantage of defining a logical UAV grammar, is that we can build a procedural generator that generates UAV design trees with minimal compute. In fact, we can sample from our generator, by sampling its stochastic parameters to draw hundreds of thousands of designs in the order of minutes. Therefore, design generation becomes trivial, but design evaluation remains a challenge due to the computational cost of running the UAV simulation software.

We note that defining a language for each cyberphysical system requires significant domain expertise. A procedural generator requires a suitable design representation, e.g. a tree, and a broad enough vocabulary to capture a large diversity of designs. In our development of our procedural generator (or declarative probabilistic program), we started by initially generating variants of the quadcopter (see Figure \ref{fig:tree}) and iterated towards more complex designs that included wings and a multitude of different component combinations. In summary, our procedural generator can generate a variety of designs that are topologically valid. By \textit{topologically valid}, we mean that they can be evaluated by the available scientific models. However, they may not be structurally sound or able to fly. In the next section we demonstrate how machine learning approaches can aid in producing working UAV designs.