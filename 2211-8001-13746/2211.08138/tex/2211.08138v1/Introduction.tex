% 2. Emerging Applications of AI

% Emerging applications papers ‘bridge the gap’ between basic AI research and case studies of deployed AI applications, by discussing efforts to apply AI tools, techniques, or methods to real-world problems in novel ways. Emerging applications focus on aspects of AI applications that are not yet sufficiently deployed to be submitted as case studies in the Deployed track.

% This track is distinguished from reports of purely scientific AI research appropriate for AAAI in that the objective of the efforts reported at IAAI should be the potential application of AI technologies, including engineering considerations. A key requirement for papers is to discuss the path forward for achieving deployment of the technology.

% Papers will be judged primarily by the following criteria: significance (of the problem, and the tool or methodology); relevance of AI technology to the problem; innovation; path to deployment; content; evaluation; technical quality; and clarity. Authors are advised to bear these questions in mind while writing their papers. Authors are also advised to consider the novelty of applying AI to the particular problem domain.

% Papers in this track may have up to 6 pages in the prescribed AAAI style, plus at most one more page which may only contain references.

\section{Introduction}
% should we reposition the intro more on design .... because we aren't really using generative modeling .. 
Artificial intelligence (AI) is transforming multiple application domains, especially those pertaining to areas of creativity and design. Examples of this success can be seen in both art and music ~\cite{briot2020deep,cetinic2022understanding,ramesh2021zero,razavi2019generating}, with particularly impressive results in text-to-image generation \cite{ramesh2022hierarchical,saharia2022photorealistic}, showing the potential of human-AI interaction for creative design. However, in comparison to domains such as text-to-image generation, in scientific design contexts (such as unmanned air vehicles and cyber-physical systems, in general) the requirements of the final design are much more structured and subject to the satisfaction of correctness requirements and optimization of performance objectives. These domains include program synthesis, drug discovery, architecture, and the design of mechanical components~\cite{parisotto2016neuro,korshunova2021openchem,buonamici2020generative,kolata2021decline,granados2021machine}. Unlike the generation of images and text, in these structured-sensitive domains, even generating valid artifacts is nontrivial.

Our paper focuses on the application of machine learning for the design of UAVs. We propose a tool that both speeds up the design process and enables the discovery of new design regimes. One key challenge is in the representation of designs. How can we ensure that the entire UAV design can be captured in a representation that is easily read by any potential machine learning algorithm? A further key challenge is then selecting an appropriate machine learning model. In our work, we show that representing a UAV design as a sequence of tokens built from a preorder traversal of a design tree is sufficient for task-specific design objectives, such as estimating the potential flight distance and hover time of each design. In more detail, we show that representing topological design information as a sequences is flexible enough to include additional design information, such as the motor and propeller parameters. We then show how our novel design embedding is effective when combined with a transformer model for predicting design performance. As a result, we demonstrate that using transformers as surrogate models for design, enables faster performance evaluations and therefore more opportunity to explore a larger variation of diverse designs. 