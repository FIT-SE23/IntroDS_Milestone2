%% if you are submitting an initial manuscript then you should have submission as an option here
%% if you are submitting a revised manuscript then you should have revision as an option here
%% otherwise options taken by the article class will be accepted
\documentclass[a4paper,draft,reqno]{amsart}
\usepackage[a4paper,margin=25mm]{geometry}
\usepackage[utf8]{inputenc}
\usepackage{mathtools}
\usepackage{amsfonts}
\usepackage{amsthm}
\usepackage{amsmath}
\usepackage{amssymb}
\usepackage{tikz}
\usepackage{hyperref}
\usepackage{authblk}

%% but DO NOT pass any options (or change anything else anywhere) which alters page size / layout / font size etc

%% note that the class file already loads {amsmath, amsthm, amssymb}
%\addbibresource
\usepackage{mathtools}

\newtheorem{Problem}{Problem}
\newtheorem{Question}{Question}
\newtheorem{Remark}{Remark}
\newtheorem{Theorem}{Theorem}
\newtheorem{Proposition}{Proposition}
\newtheorem{Lemma}{Lemma}
\newtheorem{Conjecture}{Conjecture}
\newtheorem{Corollary}{Corollary}
\newtheorem{Algorithm}{Algorithm}
\newtheorem{Definition}{Definition}
\newtheorem{Example}{Example}
\newtheorem{Notation}{Notation}

\def\clap#1{\hbox to0pt{\hss#1\hss}}
\def\eatspace#1{#1}
\def\step#1#2{\par\kern1pt\dimen44=#2em\advance\dimen44 1.67em\hangindent\dimen44\hangafter=1\noindent\rlap{\small#1}\kern\dimen44\relax\eatspace}
\let\set\mathbb
\def\<#1>{\langle#1\rangle}
\def\K{\set K}


\makeatletter
\def\testb#1{\testb@i#1,,\@nil}%
\def\testb@i#1,#2,#3\@nil{%
  \draw[->] (O) --++(#1);
  \ifx\relax#2\relax\else\testb@i#2,#3\@nil\fi}
\makeatother
\newcommand{\makediag}[1]{
    \coordinate (O) at (0,0); \coordinate (N) at (0,0.8);
    \coordinate (NE) at (0.8,0.8); \coordinate (E) at (0.8,0);
    \coordinate (SE) at (0.8,-0.8); \coordinate (S) at (0,-0.8);
    \coordinate (SW) at (-0.8,-0.8);\coordinate (W) at (-0.8,0);
    \coordinate (NW) at (-0.8,0.8); \coordinate (B1) at (1.2,1.2);
    \coordinate (B2) at (-1.2,-1.2);
    \testb{#1}
}
\newcommand{\diagr}[1]{
  \begin{tikzpicture}[baseline=-3pt,scale=0.3]\makediag{#1}\end{tikzpicture}
}

\usepackage[backend=bibtex,firstinits=true]{biblatex}

\usepackage{lipsum}

%% define your title in the usual way
\title[]{The Orbit-Sum Method\\ for Higher Order Equations}

%% define your authors in the usual way
%% use \addressmark{1}, \addressmark{2} etc for the institutions, and use \thanks{} for contact details
%\author[]{Manfred Buchacher\thanks{\href{mailto:manfred.buchacher@ricam.oeaw.ac.at}{manfred.buchacher@ricam.oeaw.ac.at} }\addressmark{1} \and Manuel Kauers\thanks{\href{mailto:manuel.kauers@jku.at}{manuel.kauers@jku.at}\\
%The authors acknowledge the support by the Austrian FWF grant P31571-N32.
%}\addressmark{2}}
%
%%% then use \addressmark to match authors to institutions here
%\address{\addressmark{1} RICAM, Austrian Academy of Sciences\\ \addressmark{2} Institute for Algebra, Johannes Kepler University Linz}

%% put the date of submission here

%% leave this blank until submitting a revised version
%\revised{}

%% put your English abstract here, or comment this out if you don't have one yet
%% please don't use custom commands in your abstract / resume, as these will be displayed online
%% likewise for citations -- please don't use \cite, and instead write out your citation as something like (author year)

%% put your French abstract here, or comment this out if you don't have one
%\resume{\lipsum[2]}

%% put your keywords here, or comment this out if you don't have them yet
\keywords{lattice walks, generating functions, functional equations, orbit-sum method}

%% you can include your bibliography however you want, but using an external .bib file is STRONGLY RECOMMENDED and will make the editor's life much easier
%% regardless of how you do it, please use numerical citations; i.e., [xx, yy] in the text

%% this sample uses biblatex, which (among other things) takes care of URLs in a more flexible way than bibtex
%% but you can use bibtex if you want
\usepackage[backend=bibtex]{biblatex}
\usepackage{multicol}
\addbibresource{orbitSumMethod.bib}
%% note the \printbibliography command at the end of the file which goes with these biblatex commands

\begin{document}

\maketitle

\begin{multicols}{2}
\begin{center}
\hspace{2cm}
\begin{tabular}{@{}c@{}}
    Manfred Buchacher\\
Austrian Academy of Sciences\\
    \normalsize manfredi.buchacher@gmail.com
  \end{tabular}%
  \end{center}
  \columnbreak
\begin{center}
\hspace{-2cm}
\begin{tabular}{@{}c@{}}
    Manuel Kauers\\
Johannes Kepler University Linz\\
    \normalsize manuel.kauers@jku.at
  \end{tabular}
  \end{center}
\end{multicols}
%% note that you DO NOT have to put your abstract here -- it is generated by \maketitle and the \abstract and \resume commands above
\begin{abstract}
The orbit-sum method is an algebraic version of the reflection-principle
that was introduced by Bousquet-M\'{e}lou and Mishna to solve functional equations that arise in the enumeration of lattice walks with small steps restricted to $\mathbb{N}^2$. 
Its extension to walks with large steps was started by Bostan, Bousquet-M\'{e}lou and Melczer. We continue it here, making use of the primitive element theorem, Gr\"{o}bner bases and the shape lemma, and the Newton-Puiseux algorithm. 
\end{abstract}


\section{Introduction}


%Why not like this:
%enumerating combinatorial objects by
%solving functional equations their generating functions satisfy
%discrete differential equations is class of functional equations that arise in enumeration of LW and maps
%Definitions: discrete derivative and discrete differential equations,
%what does it mean to solve them?
%what are the methods to solve them? 
%the kernel method: elementary power series algebra involving only operations such as... references
%the orbit-sum method for solving linear PDDE, reference, example 
%analysis of the example,
%goal: algorithmization of the method to linear PDDE of higher order
%further references


%Many generating functions can be described as solutions of certain functional equations.
%One important type of such functional equations is the class of discrete differential equations.
%In its most simple version, such an equation has the form $F(x;t)=P(x)+t Q(x,F(x;t),F(0;t))$,
%where $F$ is an unknown series and $P$ and $Q$ are known polynomials.
%It is well understood how to solve such equations via the so-called kernel method~\cite{mbm&jehanne,issac}.
%
%Partial discrete differential equations are more difficult. These are equations
%like the equation above, but with several $x$ rather than just one. An example is the equation
%\[
% F(x,y;t) = 1 + t (x+y) F(x,y;t) + \frac tx\bigl(F(x,y;t) - F(0,y;t)\bigr) + \frac ty \bigl(F(x,y;t) - F(x,0;t)\bigr).
%\]
%Such equations systematically arise in the context of counting lattice walks with small steps restricted
%to the quarter plane, an area that has received a lot of attention in recent years~\cite{..,..,..,..,..,..}.
%The orbit sum method is a variant of the kernel method that has been developed for solving such equations~\cite{..}.
%Although it does not succeed in all cases, it was possible to clarify the nature of many generating
%functions using this method. The method has also been studied from a computational perspective~\cite{ct-paper,rika}.
%
%Even more difficult are partial discrete differential equations of higher order. These are equations
%of the form
%\[
%  F = P(x,y) + t Q(x,y,t,F,\Delta_x F,\dots,\Delta_x^{k} F, \Delta_y F,\dots,\Delta_y^{l} F)
%\]
%where for a variable $z$ the discrete differential operator $\Delta_z$ is defined as the operator that
%maps a series $F$ to the series $\frac1z(F - F|_{z=0})$.
%Partial discrete differential equations of higher order arise in the context of counting restricted
%lattice walks with step sets that contain longer steps.
%Such lattice walk models were first studied in~\cite{..}, and a number of cases have been solved there
%by an adapted version of the orbit sum method.\\

Many generating functions can be described as solutions of certain functional equations.
One important type of such functional equations is the class of discrete differential equations (DDE's). 
They arise in the context of the enumeration of lattice walks restricted to cones, a systematic study of them was initiated in~\cite{marni,smallSteps}. 

Discrete differential equations are equations of the form
\begin{equation}\label{eq:DDE}
 F = P(x,y) + t Q(x,y,t,\Delta_x^k \Delta_y^{l} F: k,l\in\mathbb{N})
\end{equation}
where $F\in\mathbb{Q}[x,y][[t]]$ is unknown, and $P\in\mathbb{Q}[x,y]$ and $Q\in\mathbb{Q}[x,y,t,v_{kl}: k,l\in\mathbb{N}]$ are given polynomials. The operator $\Delta_x$ is the discrete derivative with respect to $x$, which acts on $\mathbb{Q}[x,y][[t]]$ by
\begin{equation*}
F(x,y;t) \mapsto \frac{F(x,y;t)-F(0,y;t)}{x}.
\end{equation*}
The operator $\Delta_y$, the discrete derivative with respect to $y$, is defined analogously.
A DDE is a partial discrete differential equation (PDDE) if it involves discrete derivatives of $F$ with respect to both $x$ and $y$, and an ordinary discrete differential equations (ODDE) otherwise. The degree of the equation is the total degree of $P$ with respect to the $v_{kl}$'s. If its degree is at most $1$, it is a linear DDE, otherwise it is non-linear. If~$k+l$ is maximal among the discrete derivatives $\Delta_x^k\Delta_y^l F$ appearing in equation~(\ref{eq:DDE}), then $k+l$ is the order of the equation.    
%\begin{Definition}
%The discrete derivative with respect to $x$ is an operator $\Delta_x$ that acts on $\mathbb{Q}[x,y][[t]]$ by 
%\begin{equation*}
%F(x,y;t) \mapsto \frac{F(x,y;t)-F(0,y;t)}{x}.
%\end{equation*}
%Analogously, we define the discrete derivative $\Delta_y$ with respect to $y$. For $k,l\in\mathbb{N}$, we write $\Delta_x^k$ and $\Delta_y^l$ for the operators that result from applying $\Delta_x$ and $\Delta_y$, respectively, $k$ and $l$ times. 
%\end{Definition}

%\begin{Definition} 
%Let $P\in\mathbb{Q}[x,y]$ and $Q\in\mathbb{Q}[x,y,t,v_{kl}: k,l\in\mathbb{N}]$. An equation of the form 
%\begin{equation}\label{eq:DDE}
% F = P(x,y) + t Q(x,y,t,\Delta_x^k \Delta_y^{l} F: k,l\in\mathbb{N})
%\end{equation}
%involving only finitely many of the $\Delta_x^k \Delta_y^l$'s is a discrete differential equation (DDE) for $F\in\mathbb{Q}[x,y][[t]]$. It is a partial discrete differential equation (PDDE) if it involves discrete derivatives of $F$ with respect to both $x$ and $y$, and an ordinary discrete differential equations (ODDE) otherwise. The degree of the equation is the total degree of $P$ with respect to the $v_{kl}$'s. If $k+l$ is maximal among the discrete derivatives $\Delta_x^k\Delta_y^l F$ appearing in equation~(\ref{eq:DDE}), then $k+l$ is the order of the equation.     
%\end{Definition}
A DDE has a unique solution $F\in\mathbb{Q}[x,y][[t]]$ as can be seen from the recurrence relation for $[t^n] F$ that results from extracting the coefficient of $t^n$ from equation~\eqref{eq:DDE}.
% implies that 
%\begin{equation*}
%[t^n] F =
%\begin{cases*}
%      & \hspace{-7pt}$P(x,y)$, \quad if $n = 0$,\\
%      &\hspace{-7pt}$ [t^{n-1}]Q(x,y,t,\Delta_x^k \Delta_y^l F : k,l\in\mathbb{N})$, \quad if $n>0$.
%    \end{cases*}
%\end{equation*} 
%and $[t^{n-1}]Q(x,y,t,\Delta_x^k \Delta_y^l F : k,l\in\mathbb{N})$ involves only coefficients $[t^k]F$ of order at most $n-1$.
To solve a DDE means to decide whether its solution is algebraic, D-finite or D-algebraic, and in case it is, to determine a polynomial or differential equation satisfied by it.

There is a family of methods for solving DDE's~\cite{bousquet2006polynomial, bostan2022algorithms, buchacher2018inhomogeneous, smallSteps,bostan2017hypergeometric, large, bousquet2016elementary,mishna2009two,melczer2014singularity, raschel2020counting} that involve only operations such as
\begin{equation*}
+, \ \cdot,\ \circ \quad \text{and} \quad [x^>] \text{ and } [y^>],
\end{equation*}
that is, the addition, multiplication and composition of series, and the operation of discarding all terms of a series which involve non-positive powers in $x$ and $y$, respectively. The orbit-sum method~\cite{smallSteps,large} is one of them, used to solve linear DDE's. It proceeds in three steps. In the first step, a set of substitutions, the so-called orbit, is determined
that can be applied to the given functional equation.
In the second step, a linear combination of the various transformed versions of the functional equation
is formed to the end of eliminating all the evaluations of $F(x,0)$ and $F(0,y)$.
The resulting equation then only contains the unknown series $F$ and various series obtained from it
by substituting the elements of the orbit.
In the third step of the method, by means of coefficient extraction, an expression for the unknown
series $F$ is obtained.

\begin{Example}\label{example:simple}
We solve the equation
\begin{equation}\label{eq:DDE2}
 F = 1 + t (x+y) F + t \Delta_x F + t \Delta_y F
\end{equation}
for $F\in\mathbb{Q}[x,y][[t]]$. It is equivalent to
\begin{equation}\label{eq:simple}
xy(1 - t S) F(x,y) = xy - tx F(x,0) - ty F(0,y),
\end{equation}
where $S := x+y+\bar{x}+\bar{y}$, and $\bar{x} := 1/x$ and $\bar{y}:=1/y$, and $F(x,y)\equiv F(x,y;t)$. We exploit the symmetry of $S$ and the fact that the unknowns on the right of the equation either do not depend on $x$ or on $y$. By iteratively performing the substitutions $x\mapsto \bar{x}$ and $y\mapsto \bar{y}$ that leave $S$ invariant we can derive three additional equations,
 \begin{align*}
  \bar{x} y (1-t S) F(\bar{x},y) &= \bar{x}y - t \bar{x} F(\bar{x};0) - ty F(0,y),\\
  \bar{x} \bar{y} (1-t S) F(\bar{x},\bar{y}) &= \bar{x}\bar{y} - t \bar{x} F(\bar{x};0) - t\bar{y} F(0,\bar{y}),\\
  x \bar{y}(1-t S) F(x,\bar{y}) &= x\bar{y} - t x F(x;0) - t\bar{y} F(0,\bar{y}),
 \end{align*}
which, together with equation~(\ref{eq:simple}), can be linearly combined to
 \begin{equation*}
  xy F(x,y) - \bar{x}y F(\bar{x},y) + \bar{x}\bar{y} F(\bar{x},\bar{y}) - x\bar{y} F(x,\bar{y}) =  \frac{xy -\bar{x}y + \bar{x}\bar{y} - x\bar{y}}{1-t S}.
 \end{equation*}
Since $xyF(x,y)$ involves only positive powers of $x$ and $y$, and because all the other terms on the left-hand side of this equation in $\mathbb{Q}[x,y,\bar{x},\bar{y}][[t]]$ involve a negative power of $x$ or a negative power in $y$, we find that
 \begin{equation}\label{eq:pos}
  xy F(x,y) = [x^>y^>] \frac{xy -\bar{x}y + \bar{x}\bar{y} - x\bar{y}}{1-t S}.
 \end{equation}
Consequently, $F$ is D-finite, as rational functions are D-finite and the class of D-finite functions is closed under applying $[x^>y^>]$~\cite{lipshitz}.
\end{Example}

For linear partial discrete differential equations of higher order an algorithm for determining the orbit was presented in~\cite[Sec. 3]{large}.
The substitutions determined by this algorithm are algebraic functions given by their minimal polynomials. 
Algebraic functions cause difficulties in the second and third step of the orbit sum method.
In the present paper, we discuss these difficulties.
In order to carry out the second step algorithmically (Sect.~\ref{sec:orbit}), we need to construct an
algebraic function field that contains all the algebraic functions appearing in the orbit. This
step can be done on the level of ``formal'' algebraic extensions.
The third step however crucially depends on series interpretations of the algebraic functions,
so in order to carry it out algorithmically (Sect.~\ref{sec:posPart}), we will need to embed the algebraic
function field into suitably chosen fields of series. The question is then whether for the equation
at hand there exists an embedding that allows the orbit sum method to conclude. To answer this
question, we offer a sufficient and algorithmic condition.

%ordinary discrete differential equations algorithmically well understood order 1 (issac paper).
%orbit-sum method was invented for partial discrete differential equations order 1 with  (small steps paper; ),
%also reasonably well understood (creative telescoping paper, rika's thesis).
%higher order equations appeared in the long-steps paper, not so well understood.
%problem: requires algebraic extensions, which have to be ``pulled back'' to the context of series in order to be able to
%extract ``positive parts''.

%orbit sum method has three parts: (1) determine orbit (2) cancel sections [if possible] (3) extract positive part [if possible]
%(1) was addressed by the others.
%in the present paper we discuss how step (2) can be implemented in a computer algebra setting [sect ...], and we propose
%an algorithmically checkable sufficient condition for (3) [sect ...].

%We begin with introducing discrete differential equations, explain how they arise in enumerative combinatorics and  illustrate the orbit-sum method by solving a partial discrete differential equation of order~$1$.


\section{Orbits, Orbit Equations, and the Orbit-Sum}\label{sec:orbit}

The substitutions we used to solve equation~\eqref{eq:DDE2} had the following property: for every substitution $(x',y')$, there were other substitutions $(x'',y'')$ and $(x''',y''')$ such that $x' = x''$ and $S(x',y') = S(x'',y'')$, and $y' = y'''$ and $S(x',y') = S(x''',y''')$. They allowed us to modify equation~\eqref{eq:DDE2} without altering~$S(x,y)$, and without altering one of the unknown evaluations of $F(x,0)$ and $F(0,y)$. These observations are captured by the definition of the orbit of a polynomial~\cite[Definition 1]{large}. 
\begin{Definition}
Given $p\in\mathbb{Q}[x,y,\bar{x},\bar{y}]$, let $\sim$ be the relation on~$\overline{\mathbb{Q}(x,y)}^2$ defined by  
\begin{equation*}
(u_1,u_2)\sim (v_1,v_2)\quad  :\Longleftrightarrow \quad u_1 = v_1 \text{ or } u_2 = v_2, \text{ and } p(u_1,u_2) = p(v_1,v_2),
\end{equation*}
and let $\approx$ be the equivalence relation resulting from taking its transitive closure. The orbit of $p$ is the set of elements of $\overline{\mathbb{Q}(x,y)}^2$ which are equivalent to $(x,y)$.
\end{Definition}

The elements of an orbit can be represented by their minimal polynomials. A (semi-) algorithm that determines them was presented in~\cite[Section 3.2]{large}. It takes as input a Laurent polynomial and outputs, if the orbit is finite, the minimal polynomials of its elements.
%\\Here, we just state how it is specified.
%\begin{Algorithm}\label{alg:4}
%  Input: A Laurent polynomial $p\in\mathbb{Q}[x,y,\bar{x},\bar{y}]$.\\
%  Output: The set of (pairs of) minimal polynomials of elements of the orbit of $p$. 
%  \step 10 Set $p = p(x,y) - p(X,Y)$, and $\mathrm{done} = \emptyset$, and $\mathrm{todo} = \{(-X+x,-Y+y)\}$.
%  \step 20 While $\mathrm{todo}\neq \emptyset$, do:
%  \step 31 Remove an element $(P_{old},Q_{old})$ from $\mathrm{todo}$ and add it to $\mathrm{done}$.
%  \step 41 Compute the set $P_{new}$ of irreducible factors of the resultant $\mathrm{res}_Y(Q_{old},p)$ of $Q_{old}$ and $p$ with respect to $Y$, which are not free of and not equal to $Y$.
%  \step 51 Compute the set $Q_{new}$ of irreducible factors of the resultant $\mathrm{res}_X(P_{old},p)$ of $P_{old}$ and $p$ with respect to $X$ which are not free of and not equal to $X$.
%  \step 61 Enlarge $\mathrm{todo}$ by the pairs consisting of $P_{old}$ and the elements of $Q_{new}$, and of elements of $P_{new}$ and of $Q_{old}$ unless they are elements of $\mathrm{done}$.
%  \step 70 Return $\mathrm{done}$.
%\end{Algorithm}
%The components of the elements of the orbit lie in the splitting field of the set of their minimal polynomials over $\mathbb{Q}(x,y)$. 
If the orbit is finite, then the splitting field of the minimal polynomials (of the components) of its elements is a finite field extension of $\mathbb{Q}(x,y)$. Using a constructive version of the primitive element theorem, we can do computations in this field.
\begin{Theorem} (Primitive Element Theorem)
Let $K$ be a field of characteristic $0$, and let $L/K$ be a finite field extension. Then there is an $\alpha \in L$ such that $L=K(\alpha)$. If $m(X)\in K[X]$ is the minimal polynomial of $\alpha$, then
\begin{equation*}
L \cong K[X]/\langle m(X) \rangle.
\end{equation*}
\end{Theorem}


%$L$ is generated by finitely many elements that are algebraic over $K$, and so $L/K$ is a finite extension. Since $K$ has characteristic zero, $L/K$ is separable, and the primitive element theorem implies that there is an element~$\alpha$ of $L$ such that $L = K(\alpha)$. 
%Let $m(X)$ denote the minimal polynomial of $\alpha$ over $K$, and let $d$ be its degree.
%The evaluation at $\alpha$ induces an isomorphism between~$K[X]/\langle m(X)\rangle$ and $K(\alpha)$ and allows to identify elements of $K(\alpha)$ with polynomials of $K[X]$ whose degree is smaller than $d$, their so-called canonical representatives. 
Given the minimal polynomial $m(X)$ of a primitive element $\alpha$ of the splitting field of a set of polynomials $m_1(X),\dots,m_n(X)$ over $\mathbb{Q}(x,y)$, computations just amount to polynomial arithmetic in $\mathbb{Q}(x,y)[X]/\langle m(X) \rangle$, that is, adding and multiplying polynomials over $\mathbb{Q}(x,y)$, performing division with remainder and computing modular inverses using the extended Euclidean algorithm. It remains to clarify how the minimal polynomial of a primitive element can be found, and how elements of the splitting field can be expressed in terms of the primitive element. Gr{\"o}bner bases and the shape lemma~\cite[Theorem~3.7.25]{kreuzer2000} provide an answer.

\begin{Definition}
Let $I\subseteq K[x_1,\dots,x_n]$ be a zero-dimensional ideal. It is said to be in normal $x_i$-position, $i\in\{1,\dots,n\}$, if any two zeros $(a_1,\dots,a_n)$ and~$(b_1,\dots,b_n)$ of $I$ in $\overline{K}^n$ satisfy~$a_i\neq b_i$.
\end{Definition}

\begin{Theorem}\label{theorem:prime} (Shape Lemma)
Let $K$ be a field of characteristic $0$, and let $I\subseteq K[x_1,\dots,x_n]$ be a $0$-dimensional radical ideal in normal $x_n$-position. Then $I$ has a Gr{\"o}bner basis with respect to lex order which is of the form
\begin{equation*}
\{x_1-g_1,\dots ,x_{n-1}-g_{n-1},g_n\}
\end{equation*}
 for some $g_1,\dots, g_n \in K[x_n]$. In particular, the set $\mathrm{Z}(I)$ of zeros of $I$ is 
 \begin{equation*}
 \mathrm{Z}(I) = \{ (g_1(a),\dots,g_{n-1}(a),a)\in K^n : g_n(a) = 0 \}.
 \end{equation*}
\end{Theorem}

Assume that $m_1(X),\dots, m_n(X)\in\mathbb{Q}(x,y)[X]$ are irreducible and pairwise distinct, and let $I$ be the ideal generated by $m_{i}(X_{ij})$ and $1- Y \prod_{ij\neq kl} (X_{ij}-X_{kl})$ and $ Z - \sum_{ij} a_{ij}X_{ij}$, where the $X_{ij}$'s and $Y$ and $Z$ are variables and $a_{ij}\in\mathbb{Q}$, for $i=1,\dots,n$ and $j=1,\dots,\deg_X(m_i)$. It is no restriction to assume that all the assumptions of the shape lemma are satisfied as we can choose the $a_{ij}$'s such that $I$ is in normal $Z$-position~\cite[Definition~3.7.21]{kreuzer2000} and replace $I$ by its radical $\sqrt{I}$ without altering the set of its zeros~\cite[Corollary~3.7.16]{kreuzer2000}. The shape lemma implies that the Gr{\"o}bner basis of $I$ gives rise to a polynomial $m(X)\in\mathbb{Q}(x,y)[X]$ whose roots $\alpha$ are primitive elements of the splitting field of $\{m_1(X),\dots, m_n(X)\}$ over $\mathbb{Q}(x,y)$ and polynomials $p_{ij}(X)\in\mathbb{Q}(x,y)[X]$ such that the roots of $m_1(X),\dots, m_n(X)$ are given by the $p_{ij}(\alpha)$'s. 

Let $Q\in\mathbb{Q}[x,y]$ and $P_{kl}\in\mathbb{Q}[x,y]$ be polynomials, and let
\begin{equation}\label{eq:kernelEquation}
F = Q(x,y) + t \sum_{k,l} P_{kl}(x,y) \Delta_x^k \Delta_y^l F
\end{equation}
be a linear discrete differential equation for $F\in\mathbb{Q}[x,y][[t]]$. The kernel polynomial of the equation is the Laurent polynomial that appears as the coefficient of $F(x,y)$ when all the terms involving it are collected on the left hand side of the equation. The orbit of the equation is the orbit of its kernel polynomial. If it is finite, we can now assume that there is some $\alpha\in\overline{\mathbb{Q}(x,y)}$ such that its elements are of the form $(p_1(\alpha),p_2(\alpha))$ and given in terms of $p_1(X),p_2(X)\in\mathbb{Q}(x,y)[X]$ and the minimal polynomial~$m(X)\in\mathbb{Q}(x,y)[X]$ of $\alpha$. The orbit equations result from replacing $(x,y)$ in equation~\eqref{eq:kernelEquation} by the elements of the orbit, and an orbit-sum is any $\mathbb{Q}(x,y)[\alpha]$-linear combination of the orbit equations that does not involve any of the sections $F(\cdot,0)$ and $F(0,\cdot)$. Computing a basis of the vector space of such equations amounts to making an ansatz with undetermined coefficients for the linear combination, setting the coefficients of the sections equal to zero, and solving a system of linear equations over the field $\mathbb{Q}(x,y)[X]/\langle m(X) \rangle$.

%
%\begin{Algorithm}\label{alg:5}
%  Input: A set $\{m_1(X),\dots,m_n(X)\}$ of monic, irreducible and pairwise distinct polynomials over $\mathbb{Q}(x,y)$.\\
%  Output: The minimal polynomial of a generator of the splitting field of these polynomials over $\mathbb{Q}(x,y)$ and a representation of their roots as polynomials in this generator over $\mathbb{Q}(x,y)$.
%  \step 10 Let $d_i=\deg_X p_i$, and let $Z$, $t$ and $X_{ij}$ for $i\in\{1,\dots,n\}$ and $j\in\{1,\dots,d_i\}$ be variables.
%  \step 20 Define $q_1 = 1 - t\prod_{i=1}^n \prod_{1\leq j_1 < j_2\leq d_i} (X_{ij_1}-X_{ij_2})$.
%  \step 30 Define a polynomial $q_2 = Z - \sum_{i=1}^n \sum_{j=1}^{d_i} a_{ij} X_{ij}$ with random integer coefficients $a_{ij}$.
%  \step 40 Compute a Gr{\"o}bner basis of the ideal 
%  \begin{equation*}
%  I = \langle p_i(x_{ij}) \mid i\in\{1,\dots,n\}, j\in\{1,\dots, d_j\} \rangle + \langle q_1,q_2 \rangle
%  \end{equation*}
%  in the ring of polynomials in $Z, t$ and the $X_{ij}$'s over $\mathbb{Q}(x,y)$ with respect to a lexicographic order where $t$ is the largest and $Z$ the smallest variable, if it is radical, otherwise do it for its radical.
%  \step 50 Repeat steps $3$ and $4$ until the Gr{\"o}bner basis contains a non-zero polynomial $g$ in $\mathbb{Q}(x,y)[z]$ and polynomials of the form $X_{ij} - g_{ij}$ with $g_{ij}$ in $\mathbb{Q}(x,y)[z]$ for each $i$ and $j$.
%  \step 60 Return $g$ and the $g_{ij}$'s.
%\end{Algorithm}


%Given the (finite) orbit of a polynomial in terms of the minimal polynomials of its elements, Algorithm~\ref{alg:5} allows to find the minimal polynomial of a generator~$\alpha$ of their splitting field over $\mathbb{Q}(x,y)$ as well as canonical representatives of the elements of the orbit in terms of polynomials in $\alpha$. 
%We note that, if $(p_1,p_2)$ is a pair of minimal polynomials of an element of the orbit of a polynomial $p(x,y)$, and~$\alpha_1$ and $\alpha_2$ denote roots of $p_1$ and $p_2$, respectively, then $p(\alpha_1,\alpha_2)$ need not equal $p(x,y)$, i.e. $(\alpha_1,\alpha_2)$ need not be an element of the orbit of $p(x,y)$. So this has to be checked additionally. We also note that the size of the polynomials in the output of Algorithm~\ref{alg:5} is sensitive to the choice of $q_2$ defined in step $3$ of the algorithm. The problem of finding primitive elements that are nice in this respect is discussed in~\cite{van2008}.

\section{Positive-Part-Extraction}\label{sec:posPart}

In the previous section we recalled how the minimal polynomials of the algebraic substitutions required by the orbit-sum method are determined and explained how Gr\"{o}bner bases and the shape lemma allow to reduce computations in their splitting field to polynomial arithmetic. As a consequence the first two steps of the orbit-sum method can be performed algorithmically, the result of the computations being a basis of the vector space of section-free orbit equations whose elements are of the form 
\begin{equation}\label{eq:orbitEquation}
F(x,y) + \sum_{(p_1,p_2,p_3)} p_3(\alpha) F(p_1(\alpha),p_2(\alpha)) = p(\alpha),
\end{equation}
where $F\in\mathbb{Q}[x,y][[t]]$ is unknown, $\alpha$ is an element of $\overline{\mathbb{Q}(x,y)}$, given by its minimal polynomial over $\mathbb{Q}[x,y]$, and~$p_1(\alpha), p_2(\alpha)$ and $p_3(\alpha)$ are polynomials in $\alpha$ over $\mathbb{Q}(x,y)$, and $p(\alpha)$ is a polynomial in $\alpha$ over $\mathbb{Q}(x,y,t)$. The purpose of this section is to give a meaning to 
\begin{equation}\label{eq:posPart}
[x^\geq y^\geq] p_3(\alpha) F(p_1(\alpha),p_2(\alpha)),
\end{equation}
and to present a sufficient and effective condition for equation~\eqref{eq:orbitEquation} to imply that 
\begin{equation*}
F(x,y) = [x^\geq y^\geq] p(\alpha).
\end{equation*}
This requires to interpret elements of $\overline{\mathbb{Q}(x,y)}$ as series in $x$ and $y$. The positive part is then the series which results from discarding all terms which involve a non-positive power of $x$ or $y$, respectively. We did not stress this point in Example~\ref{example:simple} because the right hand side of equation~\eqref{eq:pos} can unambiguously be understood as an element of~$\mathbb{Q}[x,y,\bar{x},\bar{y}][[t]]$ whose positive part with respect to $x$ and $y$ is well-defined. In general, however, more care is necessary.

\begin{Example}
It is ambiguous to speak of the positive part of the series solution $Y$ of 
\begin{equation*} 
(1-x)Y-1=0.
\end{equation*}
The solution of the equation depends on the field of Laurent series over which it is solved. While in $\mathbb{Q}((x))$ it is $Y = \sum_{k=0}^\infty x^k$, in $\mathbb{Q}((\bar{x}))$ it is $Y = -\sum_{k=1}^\infty \bar{x}^k$, and depending on which of them we choose, we have $[x^>] Y = \sum_{k=1}^\infty x^k$ or $[x^>] Y = 0$.
%When the equation is solved over $\mathbb{Q}(x)$, its solution is $Y=1/(1-x)$. In order to define the positive part of such a rational function, we need to associate a series to it. There are two options according to the choice which of the two terms in the denominator of $1/(1-x)$ is considered as the leading term.
%We will see below that the choice of the term ordering induces an embedding of $\mathbb{Q}(x)$ into a field of Laurent series. For the moment we just note that the choice of $1$ as the leading term of $1-x$ corresponds to considering $\mathbb{Q}(x)$ as a subfield of $\mathbb{Q}((x))$, while the choice of $-x$ corresponds to considering it as a subfield of $\mathbb{Q}((\bar{x}))$. In the first case~$Y = 1/(1-x)$ expands to $Y=\sum_{k=1}^{\infty} x^k$, in the second case to $Y=\sum_{k = 1}^{\infty} \bar{x}^k$.
\end{Example}

%Let $\{m_1,\dots,m_n\}$ be a set of polynomials over $\mathbb{Q}(x,y)$, let $m(X)$ be the minimal polynomial of a primitive element $\alpha$ of their splitting field, and let $p_1(\alpha),\dots,p_k(\alpha)$ be representations of their roots as polynomials in $\alpha$ over $\mathbb{Q}(x,y)$ as output by Algorithm~\ref{alg:5}. 
To give a meaning to expression~\eqref{eq:posPart} we embed the splitting field $\mathbb{Q}(x,y)[X]/\langle m(X) \rangle$ into a field of Puiseux series and derive information about the support of the series that correspond to $\alpha$ and $p_1(\alpha)$, $p_2(\alpha)$ and $p_3(\alpha)$. Our reasoning is based on~\cite{Manuel}, an exposition of a theory of Laurent series in several variables, and on~\cite{MacDonald,buchacher2022effective}, a discussion of a (generalized) Newton-Puiseux algorithm. For details, in particular for proofs, we refer to these references. 

\begin{Definition}
A subset $C$ of $\mathbb{R}^n$ is called a cone if $\lambda C = C$ for every $\lambda\in\mathbb{R}_{\geq 0}$. It is called a polyhedral cone if there are $v_1,\dots,v_k \in \mathbb{R}^n$ such that $C = \mathbb{R}_{\geq 0} v_1 + \dots + \mathbb{R}_{\geq 0} v_k$, and rational if $v_1,\dots,v_k$ can be chosen to be elements of $\mathbb{Q}^n$. A cone $C$ is called convex if $\lambda v + (1-\lambda) w\in C$ for all $v,w\in C$ and all $\lambda\in[0,1]$, and strictly convex if, in addition, $C\cap (-C) = \{0\}$. The dual $C^*$ of $C$ is $C^*=\{ u\in\mathbb{R}^n \mid \langle u, C\rangle \leq 0 \}$.
\end{Definition}


Let $\preceq$ be an additive total order on $\mathbb{Q}^2$ and denote by $\mathbb{C}_{\preceq}((x,y))$ the set of series
\begin{equation*}
\phi = \sum_{(i,j) \in \mathbb{Q}^2} a_{ij} x^iy^j
\end{equation*}
such that 
\begin{equation*}
\mathrm{supp}(\phi) \subseteq \left(v + C\right) \cap \frac{1}{k}\mathbb{Z}^2
\end{equation*}
for some $v\in\mathbb{R}^2$, some strictly convex rational cone $C\subseteq \mathbb{R}^2$ which has a maximal element with respect to $\preceq$, and some positive integer $k\in\mathbb{Z}$. The proof of~\cite[Theorem 15]{Manuel} shows that $\mathbb{C}_{\preceq}((x,y))$ is a field, and by~\cite{MacDonald} it is algebraically closed.

Any~$w\in\mathbb{R}^2$ whose components are linearly independent over $\mathbb{Q}$ defines an additive total order $\preceq$ on~$\mathbb{Q}^2$ by
\begin{equation*}
\alpha \preceq \beta \quad :\Longleftrightarrow \quad \langle \alpha, w\rangle \leq \langle \beta, w\rangle.
\end{equation*}
In case it exists, the maximal element of a rational polyhedral set $P\subseteq \mathbb{R}^2$ with respect to any such total order is a vertex of $P$. The next theorem~\cite[Theorem 4]{robbiano1985} implies that this is also true for any other additive total order. We will therefore restrict ourselves to total orders induced by elements of $\mathbb{R}^2$ whose components are linearly independent over $\mathbb{Q}$.

\begin{Definition}
Let $w\in\mathbb{R}^n$. The rational dimension of $w$, denoted by $\mathrm{d}(w)$, is the dimension of the $\mathbb{Q}$-vector space generated by the components of $w$.
\end{Definition}
\begin{Theorem}
For any additive total order $\preceq$ on $\mathbb{Q}^n$, there exist non-zero pairwise orthogonal vectors $u_1,\dots,u_s\in\mathbb{R}^n$ such that $\mathrm{d}(u_1) + \dots + \mathrm{d}(u_s) = n$ and
\begin{equation*}
\iota: (\mathbb{Q}^n, \preceq) \rightarrow (\mathbb{R}^n, \preceq_{lex}) \quad \text{defined by} \quad \iota(v) = (v\cdot u_1,\dots, v\cdot u_s)
\end{equation*}
is an injective order homomorphism.
\end{Theorem}

Having chosen a total order $\preceq$ on $\mathbb{Q}^2$ we can identify the field of rational functions~$\mathbb{C}(x,y)$ with a subfield of $\mathbb{C}_{\preceq}((x,y))$: the series in $\mathbb{C}_{\preceq}((x,y))$ associated with a rational function $p/q\in\mathbb{C}(x,y)$ is 
\begin{equation*}
\frac{p}{\mathrm{lt}_{\preceq}(q)} \sum_{k \geq 0} \left( 1- \frac{q}{\mathrm{lt}_{\preceq}(q)}\right)^k.
\end{equation*}
Note that this series depends on the total order $\preceq$ only to the extent of what the leading term $\mathrm{lt}_{\preceq}(q)$ of $q$ with respect to it is. Viewing $\mathbb{C}(x,y)$ as a subfield of~$\mathbb{C}_{\preceq}((x,y))$, any series root $\phi\in\mathbb{C}_{\preceq}((x,y))$ of $m(X)$ induces an embedding 
\begin{align*}
p(X) + \langle m(X) \rangle \quad \mapsto \quad  p(\phi)
\end{align*}
of $\mathbb{Q}(x,y)[X]/ \langle m(X) \rangle$ into $\mathbb{C}_{\preceq}((x,y))$. This embedding allows us to study equation~\eqref{eq:orbitEquation} in the form
\begin{equation}\label{eq:orbitEquation1}
F(x,y) + \sum_{(p_1,p_2,p_3)} p_3(\phi) F(p_1(\phi),p_2(\phi)) = p(\phi),
\end{equation}
which involves only series to which $[x^\geq y^\geq]$ can be applied. The question how such series roots, and hence such embeddings, can be constructed is answered by the Newton-Puiseux algorithm. We only state the specification of the algorithm here, for a detailed discussion we refer to~\cite{MacDonald,buchacher2022effective}. 

\begin{Algorithm}[Newton-Puiseux Algorithm]\label{alg:NPA}
Input: A square-free and non-constant polynomial $p\in\mathbb{Q}[\mathbf{x},y]$, an element $w\in\mathbb{R}^n$ inducing a total order on $\mathbb{Q}^n$, and an integer $k$.\\
  Output: A list of $\deg_y(p)$ many pairs $(c_1\mathbf{x}^{\alpha_1}+\dots+c_N \mathbf{x}^{\alpha_N},C)$ with $c_1\mathbf{x}^{\alpha_1},\dots,c_N\mathbf{x}^{\alpha_N}$ being the first $N$ terms of a series solution $\phi\in\mathbb{C}_{\preceq}((x,y))$ of $p(\mathbf{x},\phi) = 0$, ordered with respect to $w$, and $C$ being a strictly convex rational cone such that $\mathrm{supp}(\phi)\subseteq \{\alpha_1,\dots,\alpha_{N-1}\} \cup \left(\alpha_N + C\right)$, where $N\geq k$ is minimal such that the series solutions can be distinguished by their first $N$ terms.
%  \step 10 Compute the roots $c$ of $p_e(t)=\sum_{I} a_I t^{I_{n+1}-\mathrm{m}(e)_{n+1}}$, where $a_I = [(\mathbf{x},y)^I]p$ and the sum runs over all $I$ in~$e\cap \mathrm{supp}(p)$, set $L$ equal to the list of pairs~$(\phi,e)$ with $\phi = c x^{-\mathrm{S}(e)}$ and $N$ equal to~$1$.
%  \step 20 While $|L|\neq \mathrm{M}(e)_{n+1}-\mathrm{m}(e)_{n+1}$ or $N< k$, do:
%  \step 31 Set $\tilde{L} = \{\}$ and $N = N+1$.
%  \step 41 For each $(\phi,e)\in L$ with $\phi$ not having $k$ terms or $p_e(t)$ not having only simple roots, do:
%  \step 52 If $\phi$ satisfies $p(\mathbf{x},\phi)=0$, append $(\phi,e)$ to $\tilde{L}$, otherwise compute the Newton polytope of $p(\mathbf{x},\phi+y)$ and determine its unique edge path $e_1,\dots,e_l$ such that $\mathrm{m}(e_1)_{n+1}$ equals zero, and $\mathrm{M}(e_l)$, but not $\mathrm{m}(e_l)$, lies on the line through $e$, and $w\in \bigcap C^*(e_i)$.
%  \step 62 For each edge $e$ of the edge path, do:
%  \step 73 Compute the roots $c$ of $p_e(t) = \sum_{I} a_I t^{I_{n+1}-\mathrm{m}(e)_{n+1}}$, where $a_I = [(\mathbf{x},y)^I]p(\mathbf{x},\phi+y)$ and the sum runs over all elements $I$ in $e\cap \mathrm{supp}(p(\mathbf{x},\phi+y))$, and append to $\tilde{L}$ all pairs $(\phi + c\mathbf{x}^{-\mathrm{S}(e)},e)$.
%  \step 81 Set $L = \tilde{L}$.
%  \step 90 Replace each pair $(\phi + c\mathbf{x}^{-\mathrm{S}(e)},e)$ of $L$ by $(\phi + c\mathbf{x}^{-S(e)}, C)$, where $C$ is the barrier cone of $e$ with respect to $p(\mathbf{x},\phi+y)$, and return $L$. 
\end{Algorithm}

%It is conjectured~\cite[Conjecture 2]{?} that the series solutions output by the Newton-Puiseux algorithm only depend on the edge that is used in the input but not on the total order.
The Newton-Puiseux algorithm is not only useful for constructing series solutions of polynomial equations but also for encoding these series by a finite amount of data, performing effective arithmetic on the level of these encodings, and deriving information about the convex hull of their supports~\cite[Sec. 5]{buchacher2022effective}. Important for us is that for $p_i(\phi)$ in equation~\eqref{eq:orbitEquation1} we can compute the (finitely many) vertices of the convex hull of its support, and that for each of these vertices $v$ we can determine a cone $C$ such that $\mathrm{supp}(p_i(\phi))\subseteq v + C$. Though these cones are strictly convex and rational, it is an open problem~\cite[Problem 1]{buchacher2022effective} how to find the vertex cones, i.e. the cones that are minimal. Another open problem is whether for any polynomial $p\in\mathbb{C}[\mathbf{x},y]$ there are only finitely many series solutions $\phi$ of~$p(\mathbf{x},\phi) = 0$ and to which extent they depend on the total order given to the Newton-Puiseux algorithm.

It remains to clarify how to derive an estimate for the support of $F(p_1(\phi),p_2(\phi))$. The following theorem~\cite[Theorem 17]{Manuel} gives a sufficient condition for the composition of Puiseux series to be well-defined, and in case it is, it provides a cone that contains its support. 

\begin{Theorem}\label{theorem:comp}
Let $C\subseteq\mathbb{R}^n$ be a strictly convex cone and $F(x_1,\dots,x_n)$ a series such that $\mathrm{supp}(F)\subseteq C$, let~$\preceq$ be an additive order on $\mathbb{Z}^m$ and $g_1,\dots,g_n\in\mathbb{C}_{\preceq}((y_1,\dots,y_m))\setminus \{0\}$. Furthermore, let $M\in\mathbb{Z}^{m\times n}$ be the matrix whose $i$-th column consists of the leading exponent of~$g_i(y_1,\dots,y_m)$ with respect to $\preceq$, and let $C'$ be a cone that contains the image of $C$ under~$M$ and $\mathrm{supp}(g_i / \mathrm{lt}(g_i))$ for $i=1,\dots,n$. If $C \cap \mathrm{ker}(M) = \{0\}$ and if $C'$ is strictly convex, then~$F(g_1,\dots,g_n)$ is well-defined and $\mathrm{supp}(F(g_1,\dots,g_n))\subseteq C'$.
\end{Theorem}

We are interested in applying Theorem~\ref{theorem:comp} when $F\in\mathbb{C}[x,y][[t]]$ satisfies $[t^0] F = 1$, and $g_1,g_2\in\mathbb{C}_{\preceq}((x,y))$ and $g_3 = t$. The next lemma states that in this case the assumptions of the theorem are always fulfilled.

\begin{Lemma}
Let $C$ be a strictly convex cone in $\mathbb{R}^3$ such that $C\cap \left(\mathbb{R}^2\times \{0\}\right) = \{0\}$, and let $F\in\mathbb{C}[x,y][[t]]$ be such that $\mathrm{supp}(F)\subseteq C$. Let $\preceq$ be an additive total order on $\mathbb{Q}^3$ and $g_1,g_2\in\mathbb{C}_{\preceq}((x,y))$, and let $M$ be the matrix whose columns are the leading exponents of $g_1,g_2$ and $t$. Then $C\cap \mathrm{ker}(M) = \{0\}$, and the cone generated by $MC$ and $\mathrm{supp}(g_i/\mathrm{lt}(g_i))$ for~$i\in\{1,2\}$ is strictly convex.
\end{Lemma}
\begin{proof}
The series $g_1$ and $g_2$ do not depend on $t$, therefore $\mathrm{ker}(M) \subseteq \mathbb{R}^2\times \{0\}$, and so~$C\cap \mathrm{ker}(M) = \{0\}$, by assumption on $C$. Since $g_1$ and $g_2$ are elements of $\mathbb{C}_{\preceq}((x,y))$, the cone generated by the support of $g_1/\mathrm{lt}(g_1)$ and $g_2/\mathrm{lt}(g_2)$ is strictly convex, and because $g_1$ and $g_2$ are independent of $t$, it is contained in $\mathbb{R}^2\times \{0\}$. The shape of $M$ implies $MC \cap\left( \mathbb{R}^2\times \{0\}\right) = M\left(C \cap\left( \mathbb{R}^2\times \{0\}\right) \right) =  \{0\}$. To finish the proof of the lemma, it is therefore sufficient to show that $MC$ is strictly convex. Assume that there is a~$v\neq 0$ such that $v\in MC$ and $-v\in MC$. Then there are $u_1,u_2\in C$ such that~$Mu_1=v$ and $Mu_2=-v$. But then $M(u_1+u_2) = 0$, i.e. $u_1+u_2\in\mathrm{ker}(M)$. Together with~$u_1+u_2\in C$ and $C\cap\mathrm{ker}(M) = \{0\}$ this implies that $u_1 + u_2 = 0$. Since~$C$ is strictly convex, $u_1=0=u_2$, and therefore $v=0$. So $MC$ is strictly convex as well.
\end{proof}

To summarize, we can construct series roots $\phi$ of $m(X)$ to embed $\mathbb{C}(x,y)[X]/\langle m(X) \rangle$ into fields $\mathbb{C}_{\preceq}((x,y))$ of Puiseux series, for each such embedding we can determine the vertices of the convex hull of the support of $p_i(\phi)$, and for each of these vertices $v$ we can compute a strictly convex rational cone $C_v$ such that $\mathrm{supp}(p_i(\phi))\subseteq v + C_v$, and finally, we are able to find a strictly convex cone $C$ such that $\mathrm{supp}(F(p_1(\phi),p_2(\phi);t)) \subseteq C$. The support of $p_3(\phi) F(p_1(\phi),p_2(\phi);t)$ is then contained in $v + C_v + C$. If $\left( \mathbb{Q}_{\geq 0}^2 \times \mathbb{Q}\right) \cap \left( v + C_v+C\right) = \emptyset$, then $[x^\geq y^\geq] p_3(\phi) F(p_1(\phi),p_2(\phi);t) = 0$.

\begin{Example}
We solve the system of discrete differential equations 
\begin{align}\label{eq:system}
\begin{split}
F_0 &= 1 + t F_1 + t \Delta_x \Delta_y F_1\\
F_1 &= t(1+x+y)F_0 + ty \Delta_x F_0
\end{split}
\end{align}
for $F_0,F_1\in\mathbb{Q}[x,y][[t]]$ and show that their solution is D-finite. We begin with eliminating~$F_1(x,y;t)$ from the first of these equations and continue working with
\begin{equation}\label{eq:kernelEqu}
(1-t^2S_0 S_1) F_0 = 1 - t\bar{x}\bar{y} (F_1(x,0) + F_1(0,y) - F_1(0,0)) - t^2(\bar{x}\bar{y}+1)\bar{x}y F_0(0,y),
\end{equation}
where 
\begin{equation*}
S_0 := \bar{x}y+y+x+1 \quad \text{and} \quad S_1 := \bar{x}\bar{y} + 1.
\end{equation*}
The Laurent polynomial $S_0S_1$ has a finite orbit. Its elements are $(x,y),(x,\bar{y})$, $(p_1(\alpha),y)$ and $(p_{-1}(\alpha),y)$, and $(p_1(\alpha),\bar{y})$ and $(p_{-1}(\alpha),\bar{y})$, where
\begin{equation*}
p_i(X) = \frac{x + y + x y + x y^2 + i X}{2 x^2 y} \quad \text{and} \quad \alpha = \sqrt{4 x^3 y^2 + (x + y + x y + x y^2)^2}.
\end{equation*}
We consider their components as elements of the extension of $\mathbb{C}(x,y)$ by a root $\alpha$ of
\begin{equation*}
m(X) = X^2 - 4 x^3 y^2 - (x + y + x y + x y^2)^2.
\end{equation*}
Plugging the elements of the orbit into equation~\eqref{eq:kernelEqu},
forming a linear combination of the resulting equations with undetermined coefficients, and equating the  coefficients of the sections of~$F_0$ and $F_1$ to zero results in a linear system over $\mathbb{C}(x,y)[\alpha]$. The vector space of solutions is $1$-dimensional, and so is the vector space of section-free orbit equations. The latter is generated by the equation
\begin{equation*}
F_0(x,y) - \bar{y}^2 F_0(x,\bar{y}) - \sum_{i,j = \pm1} c_{ij}(\alpha) F_0(p_i(\alpha),y^j) = \frac{(-1 + y^2) (2 y - x^3 y + x (1 + y + y^2))}{x^3 y^3(1 - t^2 S_0 S_1)}.
\end{equation*}
The coefficients $c_{ij}(\alpha)$ in $\mathbb{C}(x,y)[\alpha]$ are
\begin{align*}
c_{ij}(\alpha) = i \frac{x + 2 y + x y + x y^2}{2 x^3 y} + j \frac{\alpha (2 y^2 + 2 x^3 y^2 + 3 x y (1 + y + y^2) + x^2 (1 + y + y^2)^2)}{2 x^3 y (y^2 + 4 x^3 y^2 + 2 x y (1 + y + y^2) + x^2 (1 + y + y^2)^2)}.
\end{align*}
Let $\preceq$ be the total order on $\mathbb{Q}^2$ defined by $w = (\sqrt{2},1/2)$ and let $\phi$ be the series solution of $m(X) = 0$ in $\mathbb{C}_{\preceq}((x,y))$ whose first term is $2x^{3/2}y$.
% and whose support is contained in $(3/2,1)+ \langle (-1,2),(-1,-2) \rangle$. 
We identify $p_i(\alpha)$ and~$c_{ij}(\alpha)$ with $p_i(\phi)$ and $c_{ij}(\phi)$ in $\mathbb{C}_{\preceq}((x,y))$ and show that the only term on the left hand side of the equation that remains when applying $[x^{\geq}y^{\geq}]$ is $F_0(x,y)$. Consequently, $F_0$ is the non-negative part of a rational function, and therefore D-finite, and so is $F_1$ by the second of the equations in~\eqref{eq:system}. Obviously,~$[x^{\geq}y^{\geq}] F_0(x,y) = F_0(x,y)$ and $[x^{\geq}y^{\geq}] \bar{y}^2F_0(x,\bar{y}) = 0$. Using the Newton-Puiseux algorithm, one can show that 
\begin{equation*}
\mathrm{supp}(p_i(\phi)) \subseteq (-1/2,0,0) + \langle (-1,2,0),(-1,-2,0) \rangle
\end{equation*}
and 
\begin{equation*}
\mathrm{supp}(c_{ij}(\phi)) \subseteq (-3/2,-1+j,0) + \langle (-1,2,0),(-1,-2,0) \rangle.
\end{equation*}
Theorem~\ref{theorem:comp} then implies that
\begin{equation*}
\mathrm{supp}(F_0(p_i(\phi),y^j))\subseteq \langle (0,0,1),(0,j,1),(-1,2,0),(-1,-2,0) \rangle.
\end{equation*}
Therefore, 
\begin{equation*}
\mathrm{supp}(c_{ij}(\phi)F_0(p_i(\phi),y^j)) \subseteq (-3/2,-1+j,0) + \langle (0,0,1),(0,j,1),(-1,2,0),(-1,-2,0) \rangle,
\end{equation*}
and so 
\begin{equation*}
[x^{\geq}y^{\geq}] c_{ij}(\phi) F_0(p_i(\phi),y^j) = 0.
\end{equation*}
\end{Example}

We saw before that for every rational function $p/q\in\mathbb{C}((x,y))$ there are only finitely many ways to consider it as a series, although there are infinitely many different fields $\mathbb{C}_{\preceq}((x,y))$ of Puiseux series such a series is an element of. As a consequence we will not work with a single total order but families of them, and we will describe them by convex cones. A convex cone $C\subseteq\mathbb{R}^2$ will encode the family of total orders induced by elements of the dual $C^*$ whose components are independent over $\mathbb{Q}$. For instance, the cone $C:=\langle (1,0), (0,1)\rangle\subseteq \mathbb{R}^2$ represents the family of total orders $\preceq$ on $\mathbb{Q}^2$ with respect to which $1/(1-x-y)\in\mathbb{C}_{\preceq}((x,y))$ is given by $\sum_{k,l\geq 0}\binom{k}{k-l}x^ky^{k-l}$. Equivalently, $C$ is the smallest convex cone such that for any total order $\preceq$ defined by an element of $C^*$ the support of $\sum_{k,l\geq 0}\binom{k}{k-l}x^ky^{k-l}$ has a maximal element with respect to it. We say that $C$ is the order cone of the series.

We give a sufficient and effective condition for the application of $[x^\geq y^\geq]$ to the orbit-sum~\eqref{eq:orbitEquation} to result in an expression of $F$ as the non-negative part of an algebraic function. 
%Let $m(X)$ be the minimal polynomial of a primitive element $\alpha$, and let~$L$ be the list of tuples $(p_1(X),p_2(X),p_3(X))$ of polynomials that appear in equation~(\ref{eq:orbitEquation}). Compute (an estimate of) the order cone $C_1$ of a series root $\phi$ of $m(X)$, and determine a 

\begin{Algorithm}\label{alg:ppe}
Input: An irreducible polynomial $m(X)$ over $\mathbb{Q}(x,y)$, a list $L_0$ whose elements are tuples $(p_1(X),p_2(X),p_3(X))$ of polynomials over $\mathbb{Q}(x,y)$, and a cone $C_0\subseteq \mathbb{R}^3$ that contains the support of a series $F\in\mathbb{Q}[x,y][[t]]$ such that $C_0\cap\left(\mathbb{R}^2\times\{0\}\right) = \{0\}$.\\
Output: True or Failed, with the output being True only if there is a series root $\phi$ of $m(X)$ such that $[x^{\geq} y^{\geq}]p_3(\phi) F(p_1(\phi),p_2(\phi)) = 0$ for all $(p_1(X),p_2(X),p_3(X))$ in $L$.
 \step 10 For each series root $\phi$ of $m(X)$, do: 
 \step 21 Compute an estimate $C_1$ of the order cone of $\phi$.
 \step 31 Determine the maximal list $L_1$ of minimal cones $C$ such that for every polynomial $p(X)$ which appears as a component of an element of $L_0$ its series expansion in $\mathbb{C}_{\preceq}(x,y)[X]$ does only depend on the cone $C$ but not on the specific total order induced by an element of~$C^*$.
 \step 41 For each $C\in L_1$ such that $C + C_1$ is strictly convex, do:
 \step 52 Choose any total order $\preceq$ on $\mathbb{Q}^2$ induced by some element of $(C+C_1)^*$, and determine for each $p(X)$ which appears as a component of an element of $L_0$ a list $L_p$ of pairs $(v_p,C_{v_p})$ such that $v_p$ is a vertex of the convex hull of the support of $p(\phi)$ in $\mathbb{C}_{\preceq}((x,y))$ and $C_{v_p}$ is an estimate of the corresponding vertex cone.
 \step 62 If for each $(p_1(X),p_2(X),p_3(X))$ in $L_0$ there are $(v_{p_i},C_{v_{p_i}})$ in $L_{p_i}$ such that for the cone $C'$ computed from $C_0$ and $(v_{p_1},C_{v_{p_1}})$ and $(v_{p_2},C_{v_{p_2}})$ using Theorem~\ref{theorem:comp} we have
 \begin{equation*}
 \left( \mathbb{Q}_{\geq 0}^2\times \mathbb{Q}\right) \cap \left(v_{p_3} + C_{v_{p_3}} + C' \right)= \emptyset,
 \end{equation*}
 then return True.
  \step 70 Return Failed. 
\end{Algorithm}

%\begin{Remark}
%The series solutions $\phi$ over which Algorithm~\ref{alg:ppe} loops in step~$1$ are supposed to be computed using the Newton-Puiseux algorithm, and so are the cones in step~$2$. The cones in step~$3$ can be determined simply by investigating the possible geometric series expansions of the rational functions that appear as coefficients of the polynomials in $L$. The strictly convex cones $C_1+C_2$ over which the loop in step~$4$ iterates are constructed such that any element $w$ of the dual of~$(C_1+C_2)^*$ that induces a total order $\preceq$ on $\mathbb{Q}^3$ induces a series expansion of $p(\phi)$ in~$\mathbb{C}_{\preceq}((x,y,t))$, for every polynomial $p(X)$ which appears as a component of an element of $L$, that does not depend on the specific choice of $w$ in $(C_1+C_2)^*$. Therefore, also the pairs~$(\alpha_p,C_p)$ determined in step~$5$, again using the Newton-Puiseux algorithm, do not depend on the specific choice of $w$ in $(C_1+C_2)^*$. 
%\end{Remark}

%\begin{Remark}
%In general, the vector space of orbit equations of a model need not be of dimension $1$. If its dimension is $2$ or higher it is not clear which orbit equation should be chosen to perform the extraction of its positive part. This degree of freedom for choosing the input of Algorithm~\ref{alg:ppe} can be reduced by determining the list of $F(p_1(\alpha),p_2(\alpha))$ for which $[x^{\geq}y^{\geq}] p_3(\alpha) F(p_1(\alpha),p_2(\alpha))$ is different from $0$, regardless of the field $\mathbb{C}_{\preceq}((x,y,t))$ of Laurent series $p_1(\alpha)$ and $p_2(\alpha)$ are considered to be elements of, and regardless of the choice of $p_3(\alpha)$ in $\mathbb{C}_{\preceq}((x,y,t))\setminus\{0\}$. One can then compute the vector space of orbit equations which does not involve any of these terms, and in case it has dimension $1$, apply Algorithm~\ref{alg:ppe}.
%\end{Remark}

\section{Conclusion}
We have extended the applicability of the orbit-sum method for linear DDE's of higher order. However, there remain many equations where the method fails, and there are basically two reasons for that. First, there are equations which do not admit a solution by the orbit-sum method, simply because the shape of the equation does not allow the method to conclude.
%Though there are many equations that can linear discrete differential equations that can be solved by the orbit-sum method, there are many that can not, and the reasons for that are diverse. 
%that the orbit-sum method allows to solve 
%In many cases we cannot apply $[x^> y^>]$ to a section-free orbit equation to find an expression for its generating function. 
In some cases, for instance, the orbit is not finite and in others the orbit is finite but there is no section-free orbit equation, and again for others the section-free orbit equations only have a zero orbit-sum. 
Second, there are equations for which the orbit-sum method as presented here fails because we have not addressed some of the problems that can arise. For instance, if there is essentially more than one section-free orbit equation it is not clear which of them should be chosen to extract the non-negative part. 
% dimension of the vector space of section-free orbit equations is greater than one 
% and again for others the orbit-sum is different from zero but the dimension of the vector space formed by these equations is greater than $1$ and it is not clear which equation should be chosen to extract the non-negative part. 
It is natural to ask whether it can happen that Algorithm~\ref{alg:ppe} returns Failed although the only term on the left-hand side of the orbit equation that remains when applying~$[x^{\geq}y^{\geq}]$ is $F$. In~\cite[Proposition~24]{large} it was shown that it can happen that there are two terms $p_3(\alpha)F(p_1(\alpha),p_2(\alpha))$ whose expansions involve terms with non-negative powers in $x$ and $y$, although their sum does not. The answer to the question whether this is the only reason is certainly no, if the estimates of the support of $F$ and of the order cones and vertex cones are too big. 



%\bibliographystyle{plain}
%\bibliography{orbitSumMethod}


%% if you use biblatex then this generates the bibliography
%% if you use some other method then remove this and do it your own way
\printbibliography



\end{document}
