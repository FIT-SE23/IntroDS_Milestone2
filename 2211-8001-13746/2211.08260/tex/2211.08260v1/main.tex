\documentclass[12pt]{amsart} 

%\usepackage[notref,notcite]{showkeys} 
%\renewcommand*\showkeyslabelformat[1]{%
%	\fbox{\normalfont\tiny\ttfamily#1}}
%\usepackage[landscape]{geometry}
\usepackage{amssymb}
\usepackage{amsmath}
\usepackage{amsthm}

%\usepackage{fancyhdr}

\usepackage{enumerate}
\usepackage[backend=biber, sorting=nyt, defernumbers=true, style=alphabetic]{biblatex}
\addbibresource{bibliography.bib}

\usepackage{tikz-cd}
\usepackage[english]{babel}
\usepackage{csquotes}
% For dealing with references we use the comment environment
\usepackage{verbatim}
\newenvironment{reference}{\comment}{\endcomment}
%\newenvironment{reference}{}{}
\newenvironment{slogan}{\comment}{\endcomment}
\newenvironment{history}{\comment}{\endcomment}

% For commutative diagrams we use Xy-pic
\usepackage[all]{xy}

% We use 2cell for 2-commutative diagrams.
%\xyoption{2cell}
%\UseAllTwocells

% We use multicol for the list of chapters between chapters
\usepackage{multicol}

% This is generall recommended for better output
\usepackage{lmodern}
\usepackage[T1]{fontenc}

% For cross-file-references
\usepackage{xr-hyper}
\usepackage{hyperref}
% Theorem environments.
%

\newtheorem{theorem}{Theorem}[section]
\newtheorem{proposition}[theorem]{Proposition}
\newtheorem{lemma}[theorem]{Lemma}


\theoremstyle{definition}
\newtheorem{definition}[theorem]{Definition}
\newtheorem{example}[theorem]{Example}
\newtheorem{exercise}[theorem]{Exercise}
\newtheorem{situation}[theorem]{Situation}
\newtheorem{cor}[theorem]{Corollary}


\newtheorem{remark}[theorem]{Remark}
\newtheorem{remarks}[theorem]{Remarks}

\numberwithin{equation}{subsection}

% Macros
%


\def\lim{\mathop{\mathrm{lim}}\nolimits}
\def\colim{\mathop{\mathrm{colim}}\nolimits}
\def\Spec{\mathop{\mathrm{Spec}}}
\def\Hom{\mathop{\mathrm{Hom}}\nolimits}
\def\Ext{\mathop{\mathrm{Ext}}\nolimits}
\def\SheafHom{\mathop{\mathcal{H}\!\mathit{om}}\nolimits}
\def\SheafExt{\mathop{\mathcal{E}\!\mathit{xt}}\nolimits}
\def\Sch{\mathit{Sch}}
\def\Mor{\mathop{Mor}\nolimits}
\def\Ob{\mathop{\mathrm{Ob}}\nolimits}
\def\Sh{\mathop{\mathit{Sh}}\nolimits}
\def\NL{\mathop{N\!L}\nolimits}
\def\CH{\mathop{\mathrm{CH}}\nolimits}
\def\proetale{{pro\text{-}\acute{e}tale}}
\def\etale{{\acute{e}tale}}
\def\QCoh{\mathit{QCoh}}
\def\Ker{\mathop{\mathrm{Ker}}}
\def\Pic{\mathop{mathrm{Pic}}}
\def\Im{\mathop{\mathrm{Im}}}
\def\Coker{\mathop{\mathrm{Coker}}}
\def\Coim{\mathop{\mathrm{Coim}}}


%
\def\QCohstack{\mathcal{QC}\!\mathit{oh}}
\def\Cohstack{\mathcal{C}\!\mathit{oh}}
\def\Spacesstack{\mathcal{S}\!\mathit{paces}}
\def\Quotfunctor{\mathrm{Quot}}
\def\Hilbfunctor{\mathrm{Hilb}}
\def\Curvesstack{\mathcal{C}\!\mathit{urves}}
\def\Polarizedstack{\mathcal{P}\!\mathit{olarized}}
\def\Complexesstack{\mathcal{C}\!\mathit{omplexes}}
% \Pic is the operator that assigns to X its picard group, usage \Pic(X)
% \Picardstack_{X/B} denotes the Picard stack of X over B
% \Picardfunctor_{X/B} denotes the Picard functor of X over B
\def\Pic{\mathop{\mathrm{Pic}}\nolimits}
\def\Picardstack{\mathcal{P}\!\mathit{ic}}
\def\Picardfunctor{\mathrm{Pic}}
\def\Deformationcategory{\mathcal{D}\!\mathit{ef}}
\def\etale{\'etale{}}
\def\deg{\mathop{\mathrm{deg}}\nolimits}
\def\max{\mathop{\mathrm{max}}\nolimits}
\def\min{\mathop{\mathrm{min}}\nolimits}
\def\hom{\mathop{\mathrm{hom}}\nolimits}
\def\det{\mathop{\mathrm{det}}\nolimits}
\def\supp{\mathop{\mathrm{supp}}\nolimits}
\def\dim{\mathop{\mathrm{dim}}\nolimits}
\def\rk{\mathop{\mathrm{rk}}\nolimits}
\def\kernel{\mathop{\mathrm{ker}}\nolimits}
\def\image{\mathop{\mathrm{im}}\nolimits}
\def\pt{\mathop{\mathrm{pt}}\nolimits}
\def\length{\mathop{{l}}\nolimits}
\def\l{\mathop{{l}}\nolimits}
\def\ch{\mathop{\mathrm{ch}}\nolimits}
\def\ext{\mathop{\mathrm{ext}}\nolimits}
\def\colim{\mathop{\mathrm{colim}}\nolimits}
\def\lim{\mathop{\mathrm{lim}}\nolimits}
\def\et{\mathop{\text{\'et}}\nolimits}
\newtheorem{thmx}{Theorem}
\usepackage{todonotes}
%\usepackage{refcheck}
\makeatletter 
\def\l@subsection{\@tocline{2}{0pt}{1pc}{5pc}{}} \def\l@subsection{\@tocline{2}{0pt}{2pc}{6pc}{}} 
\makeatother
\title[A functorial approach to stability]{A functorial approach to the stability of vector bundles}

\author{Dario Weissmann}

\address[D.~Weissmann]{Fakult\"{a}t f\"{u}r Mathematik \\
Universit\"{a}t Duisburg-Essen \\
Universit\"{a}ts-strasse 2 \\
45141 Essen \\
Germany}
\email{\href{mailto: dario.weissmann@uni-due.de}{dario.weissmann@uni-due.de}}

\date{\today}

\begin{document}

\begin{abstract}
    On a normal projective variety the locus of $\mu$-stable bundles that
    remain $\mu$-stable on {\it all} Galois covers prime to the characteristic 
    is open in the moduli space of Gieseker semi-stable sheaves.
    On a smooth projective curve of genus at least $2$ this locus
    is big in the moduli space of stable bundles, i.e., its complement has codimension at least $2$.
\end{abstract}

\maketitle

\tableofcontents

\section{Introduction}

Consider the stack of vector bundles of rank $r\geq 2$ 
on a smooth projective curve $C$ over an algebraically closed field $k$ of
characteristic $p\geq 0$.
Semi-stability is a property of vector bundles
which is tailored to obtain a moduli space.
Via the Harder-Narasimhan-filtration (HN-filtration for short) it also reveals additional 
structure of the category of vector bundles
and immediately implies that semi-stability is functorial under pullback by finite separable morphisms.
Even more structure is revealed via the Jordan-Hölder-filtration (JH-filtration for short). However,
in contrast to the HN-filtration the JH-filtration is not unique and thus functoriality fails for stability.

Recently, those morphisms that preserve the stability of vector bundles have been identified:
for curves these are exactly the genuinely ramified morphisms, see \cite{bp}, Theorem 5.3.
In higher dimension the genuinely ramified morphisms also preserve stability, see \cite{bdp}, Theorem 1.2.

The main goal of this paper is to address a way to measure the failure of stability to be functorial under 
all finite separable pullbacks. 
To be precise call a vector bundle on $C$ {\it prime to $p$ stable} if it remains stable
after pullback by {\it all} finite Galois morphisms $D\to C$ which have degree prime to $p$.
The locus of prime to $p$ stable bundles is open - a direct consequence 
of the following theorem.

\begin{thmx}[Theorem \ref{theorem-very-large-cover} for curves]
Let $r\geq 1$.
There exists a prime to $p$ Galois cover $\pi:C_{r-good}\to C$
such that a vector bundle $V$ of rank $r$ is prime to $p$ stable
iff $\pi^{\ast}V$ is stable.
\end{thmx}

An analogous statement holds for normal projective varieties, see Theorem \ref{theorem-very-large-cover}.
Having identified this locus as open one should also address non-emptiness:

\begin{thmx}[Theorem \ref{theorem-non-empty}]
Let $r\geq 2$.
If $C$ has genus $g_C\geq 2$, then the prime to $p$ stable locus
$M^{p'-s,r,d}_C$ is big in the moduli space of stable bundles
$M^{s,r,d}_C$. More precisely, we have
    \[
        \dim(M^{s,r,d}_C\setminus M^{p'-s,r,d}_C)\leq rr_0(g_C-1)+1,
	 \]
where $r_0$ denotes the largest proper divisor of $r$.
If $p$ is not the smallest proper divisor of $r$, then equality holds.
\end{thmx}

This should be compared to \cite{dm}, Theorem 1.1 and Theorem 1.2, asserting
that the \'etale trivializable bundles with trivial determinant
are dense in the moduli space of stable bundles of rank $r$
and trivial determinant $M^{s,r}_{\mathcal{O}_C}$
if the characteristic $p>0$.
The above theorem shows that this does not have an analogue in characteristic $0$, see Corollary \ref{cor-not-dense}.
In rank $2$ and characteristic $0$ 
such a non-density result has been independently obtained by A. Ghiasabadi and S. Reppen,
see \cite{gr}, Corollary 4.16.
In positive characteristic denseness also no longer holds if one only considers bundles which are trivialized on
some prime to $p$ cover.

The paper is structured as follows.
In §2 we give the definitions of the functorial stability notions and study them for genus $g_C\leq 1$.

In §3 we construct a prime to $p$ cover that checks whether a vector bundle is prime to $p$ stable.
This immediately implies that the prime to $p$ stable locus is open.

In §4 we investigate certain strata which arise as the complement of the prime to $p$ stable locus and estimate their dimension.

In the appendix we include some probably well-known results
which the author could not find in the literature.

\subsection*{Notation}
	We work over an algebraically closed field $k$ of characteristic $p\geq 0$.
	A variety is a separated integral scheme of finite type over $k$.
	A curve is a variety of dimension $1$.
	
	Let $X$ be a variety.
	The function field of $X$ is denoted by $\kappa(X)$.
    We call an open subset $U$ of $X$ {\it big}
    if $X\setminus U$ has codimension at least 2.
	
	If $X$ is a projective variety we implicitly choose an ample bundle
	$\mathcal{O}_X(1)$ on $X$. If we consider a finite morphism $\pi:Y\to X$ we
	set $\mathcal{O}_Y(1)=\pi^{\ast}\mathcal{O}_X(1)$.
	By (semi)stability we mean $\mu$-(semi)stability of torsion free sheaves with respect to $\mathcal{O}_X(1)$.
	
	We denote the moduli space of (semi)stable bundles of rank $r$ and degree $d$ on
	a curve $C$ by $M^{s,r,d}_C$ (resp. $M^{ss,r,d}_C$).
	On a projective variety $X$ the stable sheaves with Hilbert polynomial $P$ 
	form an open $M^{s,P}_X$ in the moduli space of Gieseker semistable sheaves $M^{G-ss,P}_X$.
		
	Given a morphism $\pi:Y\to X$ of varieties and a sheaf $F$ on $X$ we denote the pullback
	$\pi^{\ast}F$ also by $F_{\mid Y}$.
	
	By a Galois morphism $Y\to X$ of varieties we mean a finite separable morphism
	such that the extension of function fields $\kappa(Y)/\kappa(X)$ is Galois.
	A (Galois) cover is a finite \'etale (Galois) morphism $Y\to X$ of varieties.
	
\subsection*{Acknowledgements}
The author would like to thank his PhD advisor Georg Hein for his support and many inspiring and fruitful discussions.
He would also like to thank his second advisor Jochen Heinloth for some helpful discussions, in particular Lemma \ref{lemma-jochen}.
Thanks also go to Stefan Reppen for pointing out the literature \cite{dm} and a discussion on \'etale trivializable bundles.

\section{First observations}
In this section we introduce the functorial notions of stability and give a complete analysis for genus $\leq 1$.

\begin{definition}
    A finite group $G$ is called {\it prime to} $p$ if
    $p\nmid \#(G)$.
    A finite separable morphism (resp. cover) $\pi:Y\to X$ of varieties is
    {\it prime to} $p$ if the Galois hull of $\kappa(Y)/\kappa(X)$ (resp. of $Y/X$) has Galois group prime to $p$.
\end{definition}

\begin{definition}
	Let $X$ be a projective variety. 
	A sheaf $V$ on $X$ is called 
	{\it finite-stable} (resp. {\it separable-stable}, resp. {\it \'etale-stable}, resp. {\it prime-to-}$p${\it -stable}) 
	if for every finite (resp. finite separable, resp. finite \'etale, resp. finite \'etale prime to $p$) 
	morphism $\varphi:Y\to X$ of varieties the pullback
	$\varphi^{\ast}V$ is stable with respect to
	$\varphi^{\ast}\mathcal{O}_X(1)$.
\end{definition}

\begin{remark}
    \label{remark-finite-stable}
We have the implications 
\[
    \text{finite-stable} \Rightarrow \text{separable-stable} \Rightarrow \text{\'etale-stable},
\]
\[
    \text{\'etale-stable} 
    \Rightarrow \text{prime to }p\text{-stable}, \text{ and}
\]
\[
    \text{finite-stable}\Rightarrow \text{strongly-stable}.
\]
\end{remark}

\begin{example}
        Every line bundle is finite-stable. 
        If $p>0$, then a semi-stable vector bundle
        of rank $r=p^n$, $n\geq 1$, and degree coprime to $p$
        is prime to $p$ stable.
\end{example}
A finite separable morphism has two parts: an \'etale part and a genuinely ramified part. We recall the definition:
\begin{definition}
	\label{definition-genuinely-ramified}
	Let $f:Y\to X$ be a morphism of varieties.
	We say that $f$ is {\it genuinely ramified} if it is finite separable and
	every factorization $Y\to Y'\to X$ of $f$ such that $Y'\to X$ is an \'etale
	morphism of varieties satisfies that $Y'\to X$ is an isomorphism.
\end{definition}

\begin{theorem}[\cite{bp} Theorem 5.3, \cite{bdp} Theorem 1.2]
\label{theorem-bdp}
    A finite separable morphism $D\to C$ of smooth projective curves is genuinely ramified iff
    it pulls back stable vector bundles to stable vector bundles.
    
    A genuinely ramified morphism $Y\to X$ of normal projective varieties
    pulls back stable vector bundles to stable vector bundles.
\end{theorem}


As every finite separable morphism $f:Y\to X$ is the composition of a genuinely
ramified morphism $Y\to Y'$ and an \'etale morphism $Y'\to X$
we obtain the following corollary.
\begin{cor}
	\label{cor-pro-separable-pro-etale}
	On a normal projective variety the notions of \'etale-stability and
	separable-stability agree for vector bundles.
\end{cor}

Being able to only focus on covers instead of all separable morphisms
yields two technical advantages: descent theory is simpler for Galois covers
and there are - up to isomorphism - only finitely many covers of fixed degree.
This is a direct consequence of the \'etale fundamental group of a normal
projective variety being topologically finitely generated, see 
\cite{popp}, Satz 13.1.

The notion of \'etale-stability on a curve $C$ is only interesting if 
$g_C\geq 2$.

\begin{lemma}
	\label{lemma-small-genus}
	Let $C$ is a smooth projective curve of genus $g_C\leq 1$.
	Then the following hold:
	\begin{enumerate}[(i)]
	    \item If $g_C=0$, then the only stable bundles are line bundles.
	    \item If $g_C=1$, then a stable vector bundle of rank $r$ and degree $d$
	            is prime to $p$ stable iff $(r,d)=(1)$ and $r$ is a power of $p$.
	    \item If $g_C=1$ and $C$ is an ordinary elliptic curve, then the only
	        \'etale stable bundles are line bundles.
	    \item If $g_C=1$ and $C$ is supersingular, then the notions of prime to $p$ stable and
	    \'etale stable agree.
	\end{enumerate} 
\end{lemma}
\begin{proof}
	If $g_C=0$, then (i) is clear by \cite{sp},
	\href{https://stacks.math.columbia.edu/tag/0C6U}{Tag 0C6U}, and
	\cite{hl}, Theorem 1.3.1.
	
	In the following we use that semi-stability is preserved under pullback 
	by a finite separable morphism and the behaviour of the degree under pullback,
	see Lemma \ref{lemma-stability-pullback}.
	
	If $g_C=1$, we use \cite{atiyah}, Theorem 5 and Theorem 7, which are both
	valid in arbitrary characteristic. These theorems
	immediately imply that there are no stable bundles of rank $r>1$ and integral slope
	over an elliptic curve. 
	In fact more can be said: a semi-stable vector bundle of rank $r$ and degree $d$
    is stable iff $(r,d)=(1)$, a direct consequence of \cite{oda}, Corollary 2.5.
    
	Consider a stable bundle $V$ of rank $r > 1$ and degree $d$ such that $(r,d)=(1)$.
	On a cover of degree non-coprime to $r$ 
	the pullback of $V$ can not be stable by the previous discussion.
	This proves the claim (iii) for ordinary elliptic curves as 
	they have covers of any square degree. Indeed, for $d$
	not divisible by $p$ multiplication by $d$ is of degree $d^2$. 
	For $d=p$ the dual of the Frobenius $F^{\lor}:E\to E^{(p)}$ is \'etale of
	degree $p$.
	
	If $r$ is a power of $p$ and $(r,d)=(1)$,
	then on all prime to $p$ covers we still have coprime rank and degree.
 	This proves (ii).
 	
	If $C$ is supersingular, then every cover is prime to $p$ and we obtain (iv). 
\end{proof}

\section{Proof of Theorem 1}
The idea to prove Theorem 1 is simple: There are two types of failure for a stable 
bundle to remain stable after pullback. 
Both of these failures can be detected on single cover. We make this more precise on a smooth projective curve $C$.

The key observation is that a stable bundle $V$ of rank $r$ on $C$
decomposes on a Galois cover $D\to C$ as $V_{\mid D}=\bigoplus_{i=1}^n M_i^{\oplus e}$ for some pairwise non-isomorphic stable bundles $M_i$ on
$D$ such that the Galois group acts transitively on the isomorphism classes of the $M_i$, see Lemma \ref{lemma-pullback-galois}.
This is somewhat similar to the decomposition of a prime ideal in a Galois extension of number fields.

If $n\geq 2$, this decomposition behaviour can already be detected 
on a Galois cover $C_{r-large}$, a cover dominating all
covers of degree dividing $\rk(V)=r$, see Lemma \ref{lemma-large-cover}.

If $V$ remains stable on $C_{r-large}$, then for any Galois cover 
$D\to C$ the decomposition is $V_{\mid D}=M^{\oplus e}$.
Pretending that $M$ descends to a stable bundle $W$ on $C$ 
(this is not clear at all but we provide a technical workaround, see Lemma \ref{lemma-determinant-descend}) 
we can compare the
descent data associated to $W^{\oplus e}$ and $V$ to obtain 
a $\mathrm{Gl}_e$-representation $\rho$ of the Galois group $\mathrm{Gal}(D/C)=G$.
The descent data agree on the kernel of $\rho$ and 
we are reduced to $G$ being a finite subgroup of $\mathrm{Gl}_e$.
If $G$ is prime to $p$, then Jordan's Theorem 
- which also has a positive characteristic version -
has a particularly nice form:
\begin{theorem}[\cite{jor} p.114 for characteristic $0$, \cite{lp} Theorem 0.4 for positive characteristic]
    \label{theorem-jordan}
    Let $r\geq 1$. There exists a constant $J(r)$ such that for every finite prime to $p$ subgroup
    $G\subset \mathrm{Gl}_r$ there exists a normal abelian subgroup $N\subseteq G$ of index $\leq J(r)$.
\end{theorem}
Thus, there exists a normal abelian subgroup $N\subseteq G$ of index
$\leq J(e)$, where $J(e)$ denotes the constant from Jordan's Theorem. 
As a finite abelian subgroup is simultaneously triagonalizable 
the decomposition $V_{\mid D}=M^{\oplus e}$ can already be detected on $D/N$. 
We obtain a prime to $p$ Galois cover $C_{r-good}$
which detects the stability of $V_{\mid D}$.

We split the construction of $C_{r-good}$ into two parts. First we show the key lemma and construct $C_{r-large}$.
This construction can also be carried out over any normal projective variety.

Then we continue with the workaround for descending $M$ and finally construct $C_{r-good}$.
The same type of cover works over a normal projective variety $X$. 
However, the workaround for descent only works for curves.
Thus, one has to complete the descent setup on the level of 
$X$ and then restrict the setup to a large curve.
\subsection{A large cover}
For the convenience of the reader we collect the basic properties 
of vector bundles under a separable pullback.
\begin{lemma}
	\label{lemma-stability-pullback}
	Let $\pi:Y\to X$ be a finite separable morphism of normal projective
	varieties of degree $d$.
	Let $F$ be a torsion free sheaf on $X$ and $G$ be a torsion free sheaf on
	$Y$. Then the following hold:
	\begin{enumerate}[(i)]
		\item $\mu(G)=d(\mu(\pi_{\ast}G)-\mu(\pi_{\ast}\mathcal{O}_Y))$.
		\item $\mu(\pi^{\ast}F)=d\mu(F)$.
		\item $F$ is semistable iff $\pi^{\ast}F$ is semistable.  
		\item If $F$ is poly-stable and $\kappa(Y)/\kappa(X)$ is Galois, then
			$\pi^{\ast}F$ is poly-stable. If $\pi$ is prime to $p$,
			then $\pi^{\ast}F$ is poly-stable iff $F$ is poly-stable.
		\item If $\pi^{\ast}F$ is stable, then so is $F$.
	\end{enumerate}
\end{lemma}

\begin{proof}
	(i) - (iv) are proven in \cite{hl}, Lemma 3.2.1, Lemma 3.2.2, and Lemma 3.2.3. 
	Note that the proofs are independent of the characteristic except for \cite{hl}, Lemma 3.2.3.
	Here the additional prime to $p$ assumption saves the splitting of the trace.
	
	These results use descent for Galois morphisms which is a bit trickier than for \'etale ones.
	We spell this out
	in Lemma \ref{lemma-invariant-subsheaf}.

	(v): A proper subsheaf (in codimension $1$) of $F$ of slope $\geq \mu(F)$ pulls back to a proper subsheaf (in codimension $1$) 
	of $\pi^{\ast}F$ of slope $\geq \mu(\pi^{\ast}F)$ by (ii). The claim follows.
\end{proof}

\begin{definition}
    Let $Y\to X$ be a Galois morphism of varieties with Galois group $G$.
    A vector bundle $V$ on $Y$ is said to be {\it $G$-invariant} if
    for every $\sigma\in G$ we have $V\cong \sigma^{\ast}V$.
    
    A subsheaf $W\subseteq V$ of a $G$-invariant bundle $V$ is
    called {\it $G$-invariant} if the isomorphisms
    $V\cong \sigma^{\ast}V$ induce isomorphisms $W\cong \sigma^{\ast}W$.
\end{definition}

\begin{definition}
    Let $Y\to X$ be a Galois morphism of varieties with Galois group $G$. 
    A vector bundle $V$ on $Y$ is said to admit a {\it $G$-linearization}
    if for all $\sigma \in G$ there exists an isomorphism $\psi_{\sigma}:V\to \sigma^{\ast}V$ 
    such that $\tau^{\ast}\psi_{\sigma}\circ \psi_{\tau}=\psi_{\sigma\tau}$ for all $\sigma,\tau\in G$.
\end{definition}

\begin{remark}
    By definition a $G$-invariant subsheaf $W\subseteq V$ of a vector bundle that admits a $G$-linearization
    admits a $G$-linearization as well.
\end{remark}

\begin{lemma}
\label{lemma-invariant-subsheaf}
    Let $Y\to X$ be a Galois morphism of normal varieties with Galois group $G$.
    Then there exists an action of $G$ on $Y/X$.
    
    Let $V$ be a torsion-free sheaf on $X$. Then a $G$-invariant saturated subsheaf $W$ of $V_{\mid Y}$ in codimension $1$
    descends to a saturated subsheaf of $V$ in codimension $1$.
    
    If $Y\to X$ is in addition to the other assumptions flat, then 
    $W$ descends to a saturated subsheaf of $V$.
\end{lemma}

\begin{proof}
    The action is obtained by thinking of $Y$ as the normal closure of $X$ in $\kappa(Y)$.
    
    Let $\eta_Y$ be the generic point of $Y$ and $\eta_X$ the generic point of $X$.
    A finite separable morphism of normal varieties is flat at all codimension $1$ points.
    Thus, we can replace $X$ by the flat locus and $Y$ by the pre-image of the flat locus.
    
    Consider a $G$-invariant saturated subsheaf $W\subseteq V_{\mid Y}$.
    Restricting the inclusion to $\eta_Y$ we obtain a $G$-invariant subspace 
    $W_{\mid \eta_Y}\subseteq (V_{\eta_X})_{\mid \eta_Y}$.
    
    The field extension $\kappa(Y)/\kappa(X)$ is a $G$-torsor and we can apply descent theory.
     We obtain $W'_{\eta_X}\subseteq V_{\eta_X}$ such that $W'_{\eta_X}\otimes_{\kappa(X)}\kappa(Y)=W_{\eta_Y}$ as subspaces of $(V_{\eta_X})_{\mid \eta_Y}$.
     
     By \cite{langton}, Proposition 1 (which also holds for varieties not just projective varieties),
     there is a unique saturated subsheaf $W'\subseteq V$ inducing the inclusion
     $W'_{\eta_X}\subseteq V_{\eta_X}$. Pulling back along the flat morphism $Y\to X$ 
     we obtain a saturated subsheaf $W'_{\mid Y}\subseteq V_{\mid Y}$
     which agrees with the inclusion $W_{\eta_Y}\subseteq (V_{\eta_X})_{\eta_Y}$ on the generic point.
     By another application of \cite{langton}, Proposition 1, we conclude $W'_{\mid Y}=W$.
\end{proof}

    We provide examples which show that neither "saturated" nor "subsheaf of a sheaf which descends"
    can be removed in Lemma \ref{lemma-invariant-subsheaf}.
\begin{example}
    Let $E$ be an elliptic curve and $\pi:E\to \mathbb{P}^1$ be a $2:1$ Galois morphism ramified at $4$ points.
    Denote the non-trivial element of the Galois group $G=\mathbb{Z}/2$ by $\sigma$.
    
    Consider a line bundle $L$ of degree $1$ on $E$.
    Then $L\oplus \sigma^{\ast}L$ admits a $G$-linearization, but does not descend to $\mathbb{P}^1$.
    Indeed, if there was a vector bundle $V$ on $\mathbb{P}^1$ such that $V_{\mid E}=L\oplus \sigma^{\ast}L$,
    then $V$ is semi-stable of slope $\frac{1}{2}$ by Lemma \ref{lemma-stability-pullback}.
    The classification of vector bundles on $\mathbb{P}^1$ does not allow for such a bundle, see e.g. \cite{hl}, Theorem 1.3.1.
    
    Consider a point $e\in E$ at which $\pi$ is ramified.
    Let $I$ be the effective Cartier divisor which cuts out $e\in E$.
    Then $I$ is a $G$-invariant subsheaf of $\mathcal{O}_E$ but does not descend to a subsheaf $I'$ of $\mathcal{O}_C$.
    Indeed, by Lemma \ref{lemma-stability-pullback} such a subsheaf $I'$ would be a line bundle of slope $\frac{1}{2}$ which
    is impossible.
\end{example}

The starting point for the (non-)functoriality of stability is the following lemma. A stable bundle can only decompose in a very special way after
a Galois pullback.
\begin{lemma}
	\label{lemma-pullback-galois}
	Let $\pi:Y\to X$ be a finite Galois morphism of 
	normal projective varieties with Galois group $G$. 
	Let $V$ be a stable vector bundle on $X$ of rank $r$.
	Then $V_{\mid Y}=(\bigoplus_{i=1}^n W_i)^{\oplus e}$ for some
	pairwise non-isomorphic stable vector bundles $W_i$ on $Y$ and $G$ acts
	transitively on the set of isomorphism classes $\{W_i \mid i=1,\dots, n\}$.
	In particular, the $W_i$ have the same rank. 
\end{lemma}

\begin{proof}
	By Lemma \ref{lemma-stability-pullback} the bundle $V_{\mid Y}$ is
	poly-stable. Applying \cite{hl}, Corollary 1.6.11, we find that $V_{\mid Y}=\bigoplus_{i=1}^n W_i^{\oplus e_i}$ 
	for pairwise non-isomorphic stable vector bundles $W_i$ on $Y$.  
	Let $\iota:W\to V_{\mid Y}$ denote the inclusion of one of the $W_i$.
	The image of $\bigoplus_{\sigma\in G}\sigma^{\ast}W
	\xrightarrow{\oplus\sigma^{\ast}\iota}V_{\mid Y}$ is a $G$-invariant subbundle and
	descends to a subbundle $E$ of $V$ by Lemma \ref{lemma-invariant-subsheaf}. 
	As $E$ has the same slope as $V$, the stability of $V$ implies $E=V$.
	We obtain that $\bigoplus_{\sigma\in G}\sigma^{\ast}W\to V_{\mid Y}$ is
	surjective. Using the stability of the $W_i$ we find that the group $G$ acts transitively on
	the isomorphism classes of the $W_i$. Clearly, $\rk(\sigma^{\ast}W)=\rk(W)$ for
	all $\sigma\in G$.
	
	Let $e=e_{i_0}$ be the smallest index among the $e_i$ and $W=W_{i_0}$.
	For each $W_i$ there is a $\sigma_i\in G$ such that $\sigma_i^{\ast}W=W_i$.
	The inclusion $W_i^{\oplus e_i}\to V_{\mid Y}$ induces an inclusion
	$W^{\oplus e_i}\to V_{\mid Y}$ after pullback by $\sigma_i^{-1}$.
	We obtain $e_i\leq e$. By definition of $e$ we have equality.
\end{proof}

\begin{lemma}
	\label{lemma-decomposition-small-degree} 
	Let $\pi:Y\to X$ be a finite Galois cover of normal projective
	varieties with Galois group $G$.
	Further, let $V$ be a stable vector bundle of rank $r$ on $X$ such that 
	the decomposition $V_{\mid Y}=\bigoplus_{i=1}^n W_i^{\oplus e}$ 
	of Lemma \ref{lemma-pullback-galois} satisfies $n\geq 2$.  
	Then there is a factorization of $Y\to X$ into $Y\to
	Y'\xrightarrow{\pi'}X$ 
	such that $\text{deg}(\pi')=n$ and $V_{\mid Y'}$
	is not stable.
	
	More precisely, $V_{\mid Y'}=V'\oplus W'$, where $W'$ is of rank $r/n$
	and $V'_{\mid Y}$ is a direct sum of
	conjugates of $W'_{\mid Y}$ under $G$.  
\end{lemma}

\begin{proof} 
	By assumption there are at least
	two different $W_i$.
	Consider the stabilizer $H$ of $W:=W_i^{\oplus e}$ for some $i$
	and fix an inclusion $\iota:W\to V_{\mid Y}$.
	The image $E$ of $\bigoplus_{\sigma\in H}\sigma^{\ast}W
	\xrightarrow{\oplus \sigma^{\ast}\iota} V_{\mid Y}$ is
	an $H$-invariant subsheaf. Using the stability of the $W_j$ we find that $E$ 
	is isomorphic to $W$. Therefore, the direct summand 
	$W$ of $V_{\mid Y}$
	descends to a direct summand $W'$ of $V_{\mid Y'}$, where $Y'=Y/H$ and
	$Y\to Y'\xrightarrow{\pi'} X$ are the induced
	morphisms.  Note that $\pi'$ has degree $\#(G/H)=n$.

	Let $V_{\mid Y'}=W'\oplus V'$.
	As $G$ acts transitively on the isomorphism classes of the $W_i$ 
	we have that $V'_{\mid Y}$ is a direct sum of
	$\sigma^{\ast}W'_{\mid Y}$ for some $\sigma\in G$.  
\end{proof}

\begin{lemma}
	\label{lemma-large-cover}
	Let $X$ be a normal projective variety and $r\geq 2$.  
	Then we have the following:
	\begin{enumerate}[(i)]
		\item There exists a Galois cover
			$X_{r-large}\to X$ satisfying the following property: 
			If $V$ is a vector bundle of rank $r$ on $X$ of such that
			$V_{\mid X_{r-large}}$ is stable, then 
			for all Galois covers $Y\to X$
			we have $V_{\mid Y}=W^{\oplus e}$ for some stable vector bundle $W$ on $Y$
			and $e\geq 1$.  
		\item There is a prime to $p$ Galois cover
			$X'_{r-large}\to X$
			satisfying the following property:
			If $V$ is a vector bundle of rank $r$ on $X$ such that
			$V_{\mid X'_{r-large}}$ is stable,
			then for all prime to $p$ Galois covers $Y\to X$
			we have $V_{\mid Y}=W^{\oplus e}$ for some stable vector bundle $W$ on $Y$
			and $e\geq 1$.
	\end{enumerate}
\end{lemma}

\begin{proof}
	(i): Decomposing into different stable vector bundles descends to some cover of
	degree $n$, $n\mid r$, see Lemma \ref{lemma-decomposition-small-degree}.
	There are only finitely many such
	covers up to isomorphism. In particular, there is a Galois cover
	$X_{r-large}$ dominating all
	covers of degree dividing $r$. 
	This is the desired cover.

	(ii): Define $X'_{r-large}$ as a prime to $p$ Galois cover
	dominating all prime to $p$ covers
	of degree dividing $r$. This is the desired cover.
\end{proof}

\subsection{A good cover}

The following lemma is the technical workaround for descent of $G$-invariant stable bundles.
If one is only interested in the case of curves, then there is an honest descent lemma one could use instead, see Lemma \ref{lemma-jochen}.

\begin{lemma}
	\label{lemma-determinant-descend}
	Let $D\to C$ be a finite Galois morphism of smooth projective curves with
	Galois group $G$.
	Let $V$ be a simple $G$-invariant vector bundle of rank $r$ on $D$ and
	let $r'$ be the prime
	to $p$ part of $r$. 
	Further, assume that $\det(V)$ admits a $G$-linearization.
	
	Then there exists a lift of the $G$-linearization of $\det(V)$ 
	to a system of isomorphisms $\psi_{\sigma}:V\to \sigma^{\ast}V$.
	Furthermore, there exists  a finite cyclic Galois morphism $\varphi:D'\to D$
	such that
	\begin{enumerate}[(i)]
	    \item $\deg(\varphi)\mid r'$, 
	    \item $D'\to D\to C$ is a Galois morphism,
	    \item $\mathrm{Gal}(D'/D)\subseteq \mathrm{Gal}(D'/C)$ is central, and
	    \item there exists a $1$-cocycle $\alpha: \mathrm{Gal}(D'/C)\to \mu_{k}$ such that 
	         \[
	            \varphi^{\ast}\big(\psi_{pr(\sigma')}\big)\cdot \alpha(\sigma')^{-1}:
	            V_{\mid D'} \to \sigma'^{\ast}V_{\mid D'}
	        \]
	          defines a $\mathrm{Gal}(D'/C)$-linearization of $V_{\mid D'}$, 
	          where $pr$ denotes the natural morphism $\mathrm{Gal}(D'/C)\to G$.
	\end{enumerate}
\end{lemma}

\begin{proof}
    In this proof cohomology denotes group cohomology.
    
	For two simple isomorphic bundles $V$ and $W$ we have a surjective morphism
	$\Hom(V,W)\xrightarrow{\det}\Hom(\det(V),\det(W))$.
	Applying this to $V$ and $\sigma^{\ast}V$ the $G$-linearization
	of $\det(V)$ lifts to isomorphisms $\psi_{\sigma}:V\to \sigma^{\ast}V$
	such that $\tau^{\ast}\psi_{\sigma}\circ \psi_{\tau} \circ
	\psi_{\sigma\tau}^{-1}=\lambda_{\sigma,\tau}\in \mu_r$.
	Indeed,	after identifying $\Hom(V,V)$
	with $k$ the determinant corresponds to the $r$-th power map.
	
	A computation, see \cite{dk}, Proposition 2.8,
	shows that the family $\lambda_{\sigma,\tau}$ defines a $2$-cocycle.
	Let $p^n=r/r'$ and $\lambda'_{\sigma,\tau}=\lambda_{\sigma,\tau}^{p^n}$.
	The $2$-cocycle condition for $\lambda_{\sigma,\tau}$ implies the
	$2$-cocycle condition for $\lambda'_{\sigma,\tau}$.
	We obtain an element
	$\lambda'=(\lambda'_{\sigma,\tau})\in H^2(G,\mu_{r'}).$

	Let $\mathrm{Gal}$ be the absolute Galois group of $\kappa(C)$.
	As $C$ is a curve over an algebraically closed field, $\kappa(C)$ is a $C_1$ field
	by Tsen's Theorem, see \cite{nsw}, Corollary 6.5.5.
	In particular, $H^2(\mathrm{Gal},(\kappa(C)^{sep})^*)$ vanishes, see \cite{nsw}, Proposition 6.5.8.
	By Hilbert 90 we also have the vanishing of $H^1(\mathrm{Gal},(\kappa(C)^{sep})^*)$, see \cite{nsw},
	Theorem 6.2.1. 
	Applying these two vanishing results to the long exact
	cohomology sequence
	of the short exact sequence
	\[
	    0 \to \mu_{r'}\to (\kappa(C)^{sep})^*
	    \xrightarrow{x\mapsto x^{r'}}(\kappa(C)^{sep})^* \to 0
	\]
	we obtain $H^2(\mathrm{Gal}, \mu_{r'})=0$.

	By  \cite{nsw} Theorem 1.2.4, $\lambda'\in H^2(G,\mu_{r'})$ corresponds to an
	extension 
	\[
	    0\to \mu_{r'}\to G'\to G \to 0
	\]
	inducing the action of $G$ on $\mu_{r'}$. As the action of $G$ on
	$\mu_{r'}$ is trivial, we find that $\mu_{r'}$ is central in $G'$.
	Write $G$ as a quotient of $\mathrm{Gal}$. 
	Since $H^2(\mathrm{Gal},\mu_{r'})=0$, we obtain that the central extension 
	\[
	    0\to \mu_{r'}\to \mathrm{Gal}\times_G G'\to \mathrm{Gal} \to 0,
	\]
	is trivial, i.e., $\mathrm{Gal}\times_G G'=\mathrm{Gal}\times \mu_{r'}$.
	In particular, there is a surjection $\mu_{r'}\times \mathrm{Gal} \to G'$.
	Let $H$ denote the image of $0\times \mathrm{Gal}$ under this morphism.
	By construction $H\to G'\to G$ is surjective
	and $H\times_G G'=H\times\mu_{r'}$.
	The kernel $K$ of $H\twoheadrightarrow G$ is a subgroup of $\mu_{r'}$.
	In particular, $K\subseteq H$ is central and cyclic.
	Denote by $\kappa(D')$ the field extension of $\kappa(C)$ corresponding
	to $\mathrm{Gal}\twoheadrightarrow H$ and by $D'$ the associated curve.
	We obtain Galois morphisms
	$D'\xrightarrow{\varphi} D\to C$ such that 
	$\mathrm{Gal}(D'/D)\subseteq \mathrm{Gal}(D'/C)$ is central
	and cyclic. Furthermore, the obstruction $\lambda'\in H^2(G,\mu_{r'})$
	becomes trivial in $H^2(H,\mu_{r'})$. 
	
	The triviality of the $2$-cocycle $\varphi^{\ast}\lambda'\in
	H^2(H,\mu_{r'})$
	means that there is a 1-cocycle $\alpha':H\to \mu_{r'}$ such that 
	$\partial(\alpha')(\sigma,\tau)=\lambda'_{pr(\sigma),pr(\tau)}$,
	where $pr:H\to G$ denotes the surjection constructed above.  
	
	Recall that in positive characteristic $p$-th roots are unique.
	Thus, there is a 1-cocycle $\alpha:H\to \mu_{r},\sigma\mapsto
	\alpha'(\sigma)^{1/p^n}$ such that $\partial(\alpha)(\sigma,\tau)=\lambda_{pr(\sigma),pr(\tau)}$.
	By construction the isomorphisms $\varphi^{\ast}\psi_{pr(\sigma)}\cdot
	\alpha(\sigma)^{-1}, \sigma\in H,$ are compatible. 
	Indeed, 
	\begin{align*}
		\tau^{\ast}\varphi^{\ast}\psi_{pr(\sigma)}\cdot \alpha(\sigma)^{-1}\circ
	\varphi^{\ast}\psi_{pr(\tau)}\cdot \alpha(\tau)^{-1}\circ
		\big(\varphi^{\ast}\psi_{pr(\sigma\tau)}\cdot \alpha(\sigma\tau)^{-1}\big)^{-1} & =\\
		\lambda_{pr(\sigma),pr(\tau)}\cdot(\partial(\alpha)(\sigma,\tau))^{-1} &  =1.
	\end{align*} 
\end{proof}

\begin{remark}
	A shorter (but less precise) argument is the following:
	Recall that $H^2(\mathrm{Gal},\mu_{r'})=\colim H^2(G',\mu_{r'})$, see \cite{nsw},
	Proposition 1.2.5, where
	the colimit is taken over all finite Galois extensions of $\kappa(C)$ and
	$G'$ denotes the Galois group. We obtain $\mathrm{Gal}\twoheadrightarrow
	G'\twoheadrightarrow G$ such that
	the obstruction $\lambda$ vanishes on the associated curve. However,
	this does not give us a way to control the kernel which is crucial.
\end{remark}

The advantage of the following descent lemma is that the determinant-descent can be setup
on any normal projective variety.
\begin{lemma}
    \label{lemma-invariant-inclusion}
    Let $\pi:D\to C$ be a prime to $p$ Galois cover with Galois group $G$.
    Let $V$ be a vector bundle on $C$ such that $V_{\mid D}\cong M^{\oplus e}$ for some
    simple $G$-invariant vector bundle $M$ satisfying that $\det(M)$ descends to $C$.
    Let $J(e)$ denote the constant from Jordan's theorem, 
    see Theorem \ref{theorem-jordan}.
    
    Then there exists a normal subgroup $N\subseteq G$ of index $\leq J(e)$ 
    and $M'\subseteq V_{\mid C'}$ such that $M'_{\mid D}=M$, where
    $D\rightarrow C':=D/N \rightarrow C$ are the natural morphisms.
\end{lemma}
\begin{proof}
    Denote the rank of $M$ by $m$.
    Let $\varphi^M_{\sigma}:M\to \sigma^{\ast}M, \sigma\in G$, be a system of isomorphisms lifting the
    descent datum of $\det(M)$.
    By Lemma \ref{lemma-determinant-descend} there exists a Galois morphism $\varphi:D'\to D$ with prime to $p$ 
    cyclic Galois group $H$ such that $D'\to D\to C$ is a Galois morphism with Galois group $G'$
    and there exists $\alpha:G'\to \mu_m$ such that $\varphi^{\ast}\varphi^M_{\sigma}\alpha(\sigma')^{-1}$
    is a compatible system of isomorphisms, where $\sigma$ denotes the image of $\sigma'$ in $G$.
    Furthermore, $H\subseteq G'$ is cyclic.
    
    Our goal is find a normal subgroup $N'\subseteq G'$ of index $\leq J(e)$ containing $H$ 
    and an $N'$-invariant subbundle $M_{\mid D'}\subseteq V_{\mid D'}$.
    By Lemma \ref{lemma-invariant-subsheaf} the inclusion $M_{\mid D'}\subseteq V_{\mid D'}$ 
    descends to $C'$, where $C'$ denotes the normal closure of $C$ in 
    the fixed field
    $\kappa(D')^{N'}$. Then the lemma follows as $C'=C/N$, where $N$ is the image of $N'$ in $G$.
    
    Let $\varphi^V_{\sigma}:V_{\mid D}\to \sigma^{\ast}V_{\mid D}$ be the descent datum associated to $V$.
    Choose an isomorphism $\psi:V_{\mid D}\to M^{\oplus e}$ which exists by assumption.
    Define a map 
	\[
	    \rho:G'\to \mathrm{Gl}_{e}, \sigma'\mapsto
	\text{diag}(\alpha(\sigma'))
	((\varphi_{\sigma}^{M})^{-1})^{\oplus e}\circ
	\sigma^{\ast}(\psi)\circ\varphi^{V}_{\sigma}\circ\psi^{-1},
	\]
	where $\sigma=pr(\sigma')$, i.e., $\rho$ measures the failure
	of the following diagram
	\[
	    \begin{tikzcd}
	        M^{\oplus e} \ar[rr, "\psi^{-1}"] & &
	        V_{\mid D} \ar[d, "\varphi^{V}_{\sigma}"]\\
	        \sigma^{\ast}M^{\oplus e} \ar[u, "((\varphi^{M}_{\sigma})^{-1})^{\oplus e}"] 
	        &\ & \sigma'^{\ast}V_{\mid D} \ar[ll, " \sigma^{\ast}(\psi)"]
	    \end{tikzcd}
	\]
	to commute twisted by $\mathrm{diag}(\alpha(\sigma'))$.
	Another way to put this is that $\rho$ compares the $G'$-linearizations
	$(\varphi'^{\ast}(\varphi^{M}_{\sigma})^{-1})^{\oplus e}\mathrm{diag}(\alpha(\sigma'))$ and $\varphi'^{\ast}(\varphi^{V}_{\sigma})$
	on $D'$.
	
	We claim that $\rho$ defines a group morphism.
	Indeed, for $\sigma',\tau'\in G'$ mapping to $\sigma$ (resp. $\tau$)
	in $G$ we have
	\begin{align*}
		\rho(\tau')\rho(\sigma') & = \ \\
	\text{diag}(\alpha(\tau'))
		((\varphi_{\tau}^{M})^{-1})^{\oplus e}
	\tau^{\ast}(\psi)\varphi^{V}_{\tau}\psi^{-1}
	\text{diag}(\alpha(\sigma'))
		((\varphi_{\sigma}^{M})^{-1})^{\oplus e}
		\sigma^{\ast}(\psi)\varphi^{V}_{\sigma}\psi^{-1} & = \ \\
		\text{diag}(\alpha(\tau')\alpha(\sigma'))
		((\varphi_{\tau}^{M})^{-1})^{\oplus e}
	\tau^{\ast}(\psi)\varphi^{V}_{\tau}\psi^{-1} 
		((\varphi_{\sigma}^{M})^{-1})^{\oplus e}
		\sigma^{\ast}(\psi)\varphi^{V}_{\sigma}\psi^{-1} & =\ \\
		\text{diag}(\alpha(\tau')\alpha(\sigma'))
		((\varphi_{\sigma}^{M})^{-1})^{\oplus e}
		\sigma^{\ast}{\bigg (}((\varphi_{\tau}^{M})^{-1})^{\oplus e}
		\tau^{\ast}(\psi)\varphi^{V}_{\tau}\psi^{-1}{\bigg )} 
		\sigma^{\ast}(\psi)\varphi^{V}_{\sigma}\psi^{-1} & = \ \\
		\text{diag}(\alpha(\tau')\alpha(\sigma'))
		((\varphi_{\sigma}^{M})^{-1})^{\oplus e}
		\sigma^{\ast}((\varphi_{\tau}^{M})^{-1})^{\oplus e}
		\sigma^{\ast}\tau^{\ast}(\psi)\sigma^{\ast}(\varphi^{V}_{\tau})
		\varphi^{V}_{\sigma}\psi^{-1} & = \ \\
			\text{diag}(\alpha(\tau'\sigma'))
		((\varphi_{\tau\sigma}^{M})^{-1})^{\oplus e}
		(\tau\sigma)^{\ast}(\psi)\varphi^{V}_{\tau\sigma}
		\psi^{-1}  & = \ \\
		\rho(\tau'\sigma'), & \ 
	\end{align*}
	where only the third and fifth equality require an explanation. 
	We use that $((\varphi_{\sigma}^{M})^{-1})^{\oplus e}$ commutes with matrices 
	and that matrices with entries in $k$ do not change under pullback to obtain the third equality.
	To obtain the fifth equation we note that by construction of $\alpha$, see Lemma \ref{lemma-determinant-descend},
    $\varphi^{\ast}(\varphi^{M}_{\sigma})\alpha(\sigma')^{-1}$ defines a $G'$-linearization.
	
	Replacing $D'$ by $D'/\ker(\rho)$ we can assume that $G'$ is a subgroup of $\mathrm{Gl}_{e}$.
	By Jordan's theorem, see Theorem \ref{theorem-jordan}, there is a normal abelian subgroup $N'\subseteq G'$
	such that $G'/N'$ has cardinality at most $J(e)$. As $H$ is central in $G'$ the subgroup $N'+H$ is normal, abelian, and 
	contains $H$. As a finite
	abelian subgroup of $\mathrm{Gl}_{e}$ is simultaneously triagonalizable,
	we find the desired	
	$(N'+H)$-invariant inclusion $M_{\mid D'}\subseteq V_{\mid D'}$. 
\end{proof}
    
\begin{lemma}
\label{lemma-root-preparation}
	Let $X$ be a normal projective variety of dimension $n$. Let $d$ be an integer prime to $p$.
	Further, let $L$ be a line bundle on $X$.
	Then there exists a finite cyclic Galois morphism $\varphi:X'\to X$ such that
	$\deg(\varphi)\mid d$ and
	$L_{\mid X'}$ admits a $d$-th root on a big open.
\end{lemma}

\begin{proof}
	Let $\mathcal{O}_X(1)$ be an ample line bundle. Clearly, it suffices to find a morphism $X'\to X$ 
	as in the statement such that $L_{\mid X'}\otimes \mathcal{O}_{X'}(1)^{\otimes nd}$ has a $d$-th root for some $n$.
	Thus, we can assume that $L$ admits a non-zero global section, i.e.,
	$L=\mathcal{O}_X(D)$ for some effective Cartier divisor $D$. Observe
	that it suffices to proof the Lemma for $\mathcal{O}_X(-D)$ instead of $L$.
	
	Choose an affine open $U$ containing the generic point of $D$ in $X$ such
	that $D_{\mid U}=V(f)$ for some non-zero divisor $f\in \mathcal{O}_U$.
	Consider the field extension $K/\kappa(X)$ generated by a $d$-th root of $f$.
	As $p\nmid d$ the extension $K/\kappa(X)$ is cyclic of order $d'\mid d$.
	Let $X'$ denote the normalization of $X$ in $K$. Note that
	there is a canonical morphism $\varphi:X'\to X$ which is finite by \cite{sp}, \href{https://stacks.math.columbia.edu/tag/0AVK}{Tag 0AVK}.
	It is separable by construction. Also observe that $X'$ is a variety with generic point $K$.
	
	Consider $U':=\varphi^{-1}(U)\cup \varphi^{-1}(X\setminus D)$. By construction $U'$ is big.
	We show that the effective Cartier divisor 
	$D_{\mid X'}$ admits a $d$-th root on $U'$.
	Denote the $d$-th root of $f$ on $\varphi^{-1}(U)$ by $t$.
	Then $t$ defines an effective Cartier divisor $D'$ on $U'$. 
	We have $D'^{\otimes d}=D_{\mid U'}$ as $t^d=f$ on $\varphi^{-1}(U)$
	and both are trivial on $\varphi^{-1}(X\setminus D)$.
\end{proof}

\begin{definition}
    A morphism $\pi:Y\to X$ of varieties is called {\it quasi-\'etale}
    if there is some big open $U\subseteq X$ such that $\pi^{-1}(U)\to U$ is \'etale.
    If $\pi^{-1}(U)\to U$ is a Galois cover with Galois group $G$, then we also say that $\pi:Y\to X$
    is a quasi-\'etale Galois cover with Galois group $G$.
\end{definition}

The following lemma sets up the determinant descent of 
Lemma \ref{lemma-invariant-inclusion} on a normal projective variety.
\begin{lemma}
	\label{lemma-descend-preparation-higher-dim}
	Let $X$ be a normal projective variety.
	Let $\pi:Y\to X$ be a prime to $p$ Galois cover with Galois group $G$.
	Further, let $V$ be a stable vector bundle of rank $r$ on $X$
	such that $V$ is stable on $X'_{r-large}$.
 	Then there exists a commutative diagram
	of normal projective varieties 
	\[
	\begin{tikzcd}
		Y' \ar[r,"\pi'"] \ar[d,"\varphi'"] & X' \ar[d,"\varphi"]\\
		Y \ar[r,"\pi"] & X
	\end{tikzcd} 
	\ \ \ \text{such that}
	\]
	\begin{enumerate}[(i)]
		\item we have $V_{\mid Y'}=M'^{\oplus e'}$ such that $M'$ is stable and
	          $\det(M')_{|V'}$ descends to $U'$, where $U'$ is a big open of $X'$ and $V'=\pi'^{-1}(U')$,
		\item $Y'\to X'$ and $X'\to X$ are prime to $p$ Galois morphisms,
		\item $\varphi$ is cyclic such that $\deg(\varphi)\mid r$, and
	    \item[(iv)] $\pi'$ is a quasi-\'etale Galois cover.
	\end{enumerate}
\end{lemma} 

\begin{proof}
	Consider the decomposition $V_{\mid Y}=M^{\oplus e}$ of Lemma
	\ref{lemma-pullback-galois}. Clearly, $\det(M)^{\otimes e}$ and
	$\bigotimes_{\sigma \in G}\sigma^{\ast}\det(M)=\det(M)^{\otimes
	\#(G)}$ descend to $X$ . Therefore, $\text{det}(M)^{\otimes d}$ descends to $X$ as well,
	where $d=\mathrm{gcd}(e,\#(G))$. Thus, there exists a line bundle $L$
	on $X$ such that $L_{\mid Y}=\text{det}(M)^{\otimes d}$.  
	Note that $p\nmid d$ since $G$ is prime to $p$. 
	
	We can apply Lemma \ref{lemma-root-preparation} to find $\varphi:X'\to X$
    such that $L_{\mid X'}$ has a $d$-th root $L'$ on a big open $U'$ of $X'$.
    Consider a connected component $Y''$ of the normalization of the reduced fibre product $(Y\times_X X')_{red}$.
    Denote the natural morphism by $\psi:Y''\to X'$.
    Then 
    \[
        M'':=\det(M)_{\mid \psi^{-1}(U')}
        \otimes L'^{-1}_{\mid \psi^{-1}(U')}
    \]
    is a line bundle of order dividing $d$ and the spectral cover $V'\to \psi^{-1}(U')$ associated to $M''$ trivializes 
    $M''$.
    
    Let $Y'$ denote the normalization of $Y''$ in $K$, where $K$ is the Galois hull of $\kappa(V')/\kappa(X')$.
    Applying Lemma \ref{lemma-galois-tower-away-from-char} for $q=p$ 
    yields that the Galois morphism $Y'\to X$ is prime to $p$.
    Then the commutative diagram
    \[
	\begin{tikzcd}
		Y' \ar[r,"\pi'"] \ar[d,"\varphi'"] & X' \ar[d,"\varphi"]\\
		Y \ar[r,"\pi"] & X
	\end{tikzcd}
	\]
    satisfies the conditions (ii), (iii), and (iv) of the Lemma.
    
    If $M_{\mid Y'}$ is stable, then $V_{\mid Y'}=M_{\mid Y'}^{\oplus e}$ and we obtain (i) by construction on some big open.
    
    If $M_{\mid Y'}$ is not stable, then
    we repeat the above construction
    replacing $Y$ by the Galois closure of the \'etale part of $Y'/X$.
    Then we have $V_{\mid Y}=M'^{\oplus e'}$ for $e'> e$ and $M'$ stable.
    As the integer $e'$ is at most $r$,
    this process stops after finitely many iterations. We obtain 
    that $\det(M')_{\mid V'}$ descends to some big open $U'$ of $X'$, where $V'=\pi'^{-1}(U')$.
\end{proof}


\begin{theorem}
	\label{theorem-very-large-cover}
	Let $X$ be a normal projective variety of dimension at least $1$.
	Let $r\geq 1$.
	Then there exists
	a prime to $p$ Galois cover $X_{r-good}\to X$
	such that a vector bundle $V$ of rank $r$ on $X$ is
    prime to $p$ stable iff $V_{\mid X_{r-good}}$ is stable.
	
	In particular,
	prime to $p$ stability is an open property
	in the moduli space of Gieseker semi-stable sheaves on $X$.
\end{theorem}

\begin{proof}
    Let $X_{r-good}$ be a prime to $p$ Galois cover dominating $X'_{r-large}$
    from Lemma \ref{lemma-large-cover} and all prime to $p$ covers of degree $\leq J(r)r$, where $J(r)$
	is the bound from Jordan's Theorem, see Theorem \ref{theorem-jordan}.

    The "only if" part is trivial.
    For the "if" part consider a prime to $p$ Galois cover
	$\pi:Y\to X$ and let $V_{\mid Y}\cong M^{\oplus e}$ be the decomposition
	of Lemma \ref{lemma-pullback-galois}.
	Applying Lemma \ref{lemma-descend-preparation-higher-dim}
	we obtain a commutative diagram
	\[
	\begin{tikzcd}
		Y'\ar[r,"\pi'"] \ar[d,"\varphi'"] & X'\ar[d,"\varphi"]\\
		Y \ar[r,"\pi"] & X
	\end{tikzcd}
	\]
	satisfying the properties (i) - (iv) of Lemma
	\ref{lemma-descend-preparation-higher-dim}. 
	In particular, we have an isomorphism $V_{\mid Y'}\cong M'^{\oplus e'}$ for some stable bundle $M'$
	such that $\det(M')_{\mid V'}$ descends to $U'$, where $U'$ is a big 
	open of $X'$ and $V'=\pi'^{-1}(U')$.

    Let $\mu'$ be the slope of $V_{\mid X'}$ and 
    observe that $V_{\mid X'}$ is stable
    as the degree of the \'etale part of $\varphi$ is at most $r$.
	Iterating the restriction theorem in arbitrary characteristic for normal projective varieties, see \cite{yeh}, Theorem  7.17,
	or \cite{la22}, Theorem 0.1, for positive characteristic, and \cite{hl}, Theorem 7.2.8, for characteristic $0$, 
	and the functorial version of Bertini's Theorem, Theorem \ref{functorial-bertini},
	the general $C'$ in $c_1(\mathcal{O}_{X'}(-N_1))\dots c_1(\mathcal{O}_{X'}(-N_{n-1}))$ for $N_i\gg 0$
	satisfies that for all intermediate quasi-\'etale Galois covers
	$Y'\to Y'' \to X'$ with Galois group $G''$ and $Y''$ normal
	the following holds:
	\begin{enumerate}[(i)]
		\item $C'$ and $D'':=Y''\times_{X'} C'$
			are smooth curves.
        \item The natural morphism 
        $D''\to C'$ is an \'etale Galois cover 
        with Galois group $G''$.
		\item A vector bundle $W'$ on $Y''$
			of slope $\mu''=\mu'\cdot \#(G'')$, $\rk(W')= r$, and discriminant $\Delta(V_{\mid Y''})$ 
			is stable iff $W'_{\mid D''}$ is stable.
		\item A vector bundle $N'$ on $Y'$ of slope $\mu(M')$, $\rk(N')\leq r$,	and discriminant $\Delta(M')$ is stable iff $N'_{\mid D'}$ is stable,	where $D'=C'\times_{X'}Y'$.
	\end{enumerate} 
	
	Restricting the isomorphism $V_{\mid Y'}\cong M'^{\oplus e'}$ to such a $D'$ we obtain an isomorphism
    $(V'_{\mid C'})_{\mid D'}\cong (M'_{\mid D'})^{\oplus e'}$.
	Note that $M'_{\mid D'}$ is stable and for general $C'$ its determinant $\det(M'_{\mid D'})$ descends to $C'$. Hence, we are in a position to apply
	Lemma \ref{lemma-invariant-inclusion}.
	Thus, there is an intermediate cover $D' \to D''\to C'$ of degree $\leq J(e')$
	such that there is a subbundle $M''\subseteq V_{\mid D''}$ pulling back to $M'$ on $D'$.
	
	The intermediate cover $D' \to D''\to C'$ can be lifted to a quasi-\'etale factorization of $Y'\to Y''\to X'$.
	Indeed, let $K$ be the kernel of the natural morphism $\mathrm{Gal}(D'/D'')\to \mathrm{Gal}(D'/C')$.
    By the property (ii) above we have $\mathrm{Gal}(Y'/X')=\mathrm{Gal}(D'/C')$.
	Then we define $Y''$ to be the normalization of $X$ in the field extension $\kappa(Y')^{K}/\kappa(X)$.
	Note that $Y''\to X$ is prime to $p$ of degree at most $rJ(e')\leq rJ(r)$.
	Consider the factorization $Y'' \to Y'''\to X'$ into its \'etale and genuinely ramified part.
	We find that $V_{\mid Y'''}$ is stable by assumption. By Theorem \ref{theorem-bdp} the bundle $V_{\mid Y''}$ is stable as well.
	By the property (iii) above, we obtain the stability of $V_{\mid D''}$.
	In particular, $M''=V_{\mid D''}$ and thus $V_{\mid D'}=M'$, i.e., $e'=1$.
	Clearly, $e\leq e'$ and we conclude that $V_{\mid Y}$ is stable.
	
	
	\begin{comment}
	Thus, there is
	\begin{enumerate}[(i)]
	    \item a lifting of the descent datum of
	$\det(M'_{\mid D'})$ to a system of
	isomorphisms $\varphi_{\sigma'}:M'_{\mid D'}\to \sigma'^{\ast}M'_{\mid D'}$,
	    \item a prime to $p$ Galois cover $\varphi'':D''\to D'$ with Galois group $H$ such that 
	        \begin{enumerate}[(a)]
	            \item $D''\to C$ is Galois with Galois group $G''$,
	            \item $H$ is cyclic and central in $G''$,
	        \end{enumerate}
	    \item there is a $1$-cycle $\alpha:G''\to \mu_{r/e'}$
	such that the isomorphisms $\varphi''^{\ast}\varphi_{pr(\sigma'')}$ define a descent datum
	of $M'_{\mid D''}$ after multiplying by $\alpha(\sigma'')^{-1}$.
	\end{enumerate}
	Let $\varphi^{V'}_{\sigma'}$
	denote the descent datum of $V'_{\mid D'}$ to $C'$.
	Define a map 
	\[
	    \rho:G''\to \mathrm{Gl}_{e'}, \sigma''\mapsto
	\text{diag}(\alpha(\sigma''))
	((\varphi_{\sigma'}^{M'})^{-1})^{\oplus e}\circ
	\sigma'^{\ast}(\psi)\circ\varphi^{V'}_{\sigma'}\circ\psi^{-1},
	\]
	where $\sigma'=pr(\sigma'')$, i.e., $\rho$ measures the failure
	of the following diagram
	\[
	    \begin{tikzcd}
	        M'^{\oplus e'}_{\mid D'} \ar[rr, "\psi^{-1}"] & &
	        V'_{\mid D'} \ar[d, "\varphi^{V'}_{\sigma'}"]\\
	        \sigma'^{\ast}M'^{\oplus e'}_{\mid D'} \ar[u, "((\varphi^{M'}_{\sigma'})^{-1})^{\oplus e'}"] 
	        &\ & \sigma''^{\ast}V'_{\mid D'}  \ar[ll, " \sigma'^{\ast}(\psi)"]
	    \end{tikzcd}
	\]
	to commute twisted by $\mathrm{diag}(\alpha(\sigma''))$.
	Another way to put this is that $\rho$ compares the descent data 
	$(\varphi''^{\ast}(\varphi^{M'}_{\sigma'})^{-1})^{\oplus e'}\mathrm{diag}(\alpha(\sigma''))$ and $\varphi''^{\ast}(\varphi^{V'}_{\sigma'})$ on $D''$.
	
	We claim that $\rho$ defines a group morphism.
	Indeed, consider $\sigma'',\tau''\in G''$ mapping to $\sigma'$ (resp. $\tau'$)
	in $G'$. 
	Then we have
	\begin{align*}
		\rho(\tau'')\rho(\sigma'') & = \ \\
	\text{diag}(\alpha(\tau''))
		((\varphi_{\tau'}^{M'})^{-1})^{\oplus e'}
	\tau'^{\ast}(\psi)\varphi^{V'}_{\tau'}\psi^{-1}
	\text{diag}(\alpha(\sigma''))
		((\varphi_{\sigma'}^{M'})^{-1})^{\oplus e'}
		\sigma'^{\ast}(\psi)\varphi^{V'}_{\sigma'}\psi^{-1} & = \ \\
		\text{diag}(\alpha(\tau'')\alpha(\sigma''))
		((\varphi_{\tau'}^{M'})^{-1})^{\oplus e'}
	\tau'^{\ast}(\psi)\varphi^{V'}_{\tau'}\psi^{-1} 
		((\varphi_{\sigma'}^{M'})^{-1})^{\oplus e'}
		\sigma'^{\ast}(\psi')\varphi^{V'}_{\sigma'}\psi^{-1} & =\ \\
		\text{diag}(\alpha(\tau'')\alpha(\sigma''))
		((\varphi_{\sigma'}^{M'})^{-1})^{\oplus e'}
		\sigma'^{\ast}{\bigg (}((\varphi_{\tau'}^{M'})^{-1})^{\oplus e'}
		\tau'^{\ast}(\psi)\varphi^{V'}_{\tau'}\psi^{-1}{\bigg )} 
		\sigma'^{\ast}(\psi)\varphi^{V'}_{\sigma'}\psi^{-1} & = \ \\
		\text{diag}(\alpha(\tau'')\alpha(\sigma''))
		((\varphi_{\sigma'}^{M'})^{-1})^{\oplus e'}
		\sigma'^{\ast}((\varphi_{\tau'}^{M'})^{-1})^{\oplus e'}
		\sigma'^{\ast}\tau'^{\ast}(\psi)\sigma'^{\ast}(\varphi^{V'}_{\tau'})
		\varphi^{V'}_{\sigma'}\psi^{-1} & = \ \\
			\text{diag}(\alpha(\tau''\sigma''))
		((\varphi_{\tau'\sigma'}^{M'})^{-1})^{\oplus e'}
		(\tau'\sigma')^{\ast}(\psi)\varphi^{V'}_{\tau'\sigma'}
		\psi^{-1}  & = \ \\
		\rho(\tau''\sigma''), & \ 
	\end{align*}
	where only the third and fifth equality require an explanation. 
	We use that $((\varphi_{\sigma'}^{M'})^{-1})^{\oplus e'}$ commutes with matrices 
	and that matrices with entries in $k$ do not change under pullback to obtain the third equality.
	To obtain the fifth equation we note that by construction of $\alpha$, see Lemma \ref{lemma-determinant-descend},
	we have that $\varphi'^{\ast}(\varphi^{M'}_{\sigma'})\alpha(\sigma'')^{-1}$ defines a descent datum.
	
	Replacing $D''$ by $D''/\ker(\rho)$ we can assume $G''$ is a subgroup of $\mathrm{Gl}_{e'}$.
	By Jordan's theorem, see Theorem \ref{theorem-jordan}, there is a normal abelian subgroup $N\subseteq G''$
	such that $G''/N$ has cardinality at most $J(e')$. As $H$ is central in $G''$ the subgroup $N+H$ is normal and abelian.
    Consider the intermediate cover $D''/N+H$ of $D'\to C'$. This is the restriction of an intermediate quasi-\'etale cover
	of $Y'\to Y''\to X'$. 
	As the \'etale part of $Y''\to X$ has degree at most $J(r)r$ and $V'$ is assumed to be stable on $X_{r-large}$
	we obtain that $V'$ is stable on $Y''$. Then $V'$ is also stable on
	$D''/N+H$. Indeed, $D'$ was chosen to large enough to check for 
	stability using the effective restriction
	theorem. Then any intermediate curve is also large enough to check for stability.
	
	We are reduced to the abelian case. A finite
	abelian subgroup of $\mathrm{Gl}_{e'}$ is simultaneously triagonalizable. Thus, there is a
	$G''$-equivariant morphism $\varphi^{\ast}M'_{\mid D''}\to \pi'^{\ast}V'_{\mid D''}$. 
	As $V'_{\mid C'}$ is stable we obtain 
	$\varphi^{\ast}M'_{\mid C'}=\pi'^{\ast}V'_{\mid D'}$, i.e., $e'=1$. 
	As $e\leq e'$ we conclude that $e=1$ as well, i.e., $\pi^{\ast}V=M$ is stable.
	\end{comment}
\end{proof}

\section{Proof of Theorem 2}
To obtain the non-emptiness of the locus of prime to $p$ stable bundles $M^{p'-s,r,d}_C$
we find estimates for the dimension of the complement 
\[
    Z:=M^{s,r,d}_C \setminus M^{p'-s,r,d}_C.
\]
This complement decomposes into two strata $Z=Z_1 \sqcup Z_2$, where
\[
    Z_1:=\{ V\in M^{s,r,d}_C \mid V_{\mid C_{r-good}}=M^{\oplus e}, M \text{ stable on } C_{r-good}, e\geq 2\} \text{ and}
\]
\[
    Z_2:=\{ V\in M^{s,r,d}_C \mid V_{\mid C_{r-good}}=\bigoplus_{i=1}^n M_i^{\oplus e}, M_i \text{ stable on } C_{r-good}, n\geq 2\}
\]
are obtained via applying Lemma \ref{lemma-pullback-galois} to $C_{r-good}\to C$.

To this end we first reprove a theorem due Faltings which gives us the flexibility 
to compute the dimension after a finite separable pullback

Finding an estimate for $\dim(Z_2)$ is fairly simple: the transitive action of the Galois group
means that the information of the decomposition $V_{| C_{r-good}}=\bigoplus_{i=1}^n M_i^{\oplus e}$
can be recovered from a semistable vector bundle $W'$ on an intermediate cover $D'\to C$ of degree $n$.

To find an estimate for $\dim(Z_1)$ one has to find a way to descent the bundle $M$
which admits a system of isomorphisms but which may fail to be a linearization.
However, up to twist by a line bundle such a descent lemma holds as $M$ is simple, see Lemma \ref{lemma-jochen}.

\subsection{Pullback is finite}

\begin{lemma}
	\label{lemma-etale-pushforward}
	Let $\pi:D\to C$ be a finite \'etale morphism of smooth projective curves.
	Then we have the following:
	\begin{enumerate}[(i)]
		\item The pushforward of a semi-stable bundle on $D$ to $C$ is again
			semi-stable.
		\item Let $V$ be a semistable vector bundle on $C$.
			Then $\pi_{\ast}\mathcal{O}_D\otimes V$ is semistable
			of slope $\mu=\mu(V)$.
	\end{enumerate}
\end{lemma}

\begin{proof}
	(i) Let $W$ be a semi-stable bundle of slope $\mu$ and rank $r$ on $D$.
	The pushforward $\pi_{\ast}W$ has slope $\mu/\text{deg}(\pi)$ and rank
	$\text{deg}(\pi)r$.
	If $\pi_{\ast}W$ was not semistable, consider the maximal destabilizing
	subbundle $V$ of $\pi_{\ast}W$. By adjunction $\pi^{\ast}V\to W$ is a
	non-zero morphism of semistable bundles. As
	\[
	\mu(\pi^{\ast}V)
	=\text{deg}(\pi)\mu(V)>\text{deg}(\pi)\mu(\pi_{\ast}W)
	=\mu(W)
	\]
	this is a contradiction.
	
	(ii) Let $V$ be a semistable bundle on $C$.
	As $\pi$ is \'etale the bundle $\pi_{\ast}\mathcal{O}_D$ is
	semistable of degree $0$. We obtain
	$\mu(V)=\mu(\pi_{\ast}(\mathcal{O}_D)\otimes V)$.
	By the projection formula $\pi_{\ast}\mathcal{O}_D\otimes
	V=\pi_{\ast}\pi^{\ast}V$ which is semistable by (i) and
	Lemma \ref{lemma-stability-pullback} (iii).
\end{proof}

    As semistable bundles stay semistable under a finite separable pullback $\pi:D\to C$ 
    we obtain a morphism $\pi^{\ast}:M^{ss,r,d}_C \to M^{ss,r,\deg(\pi)d}_D$.
	The finiteness of $\pi^{\ast}$ can be proven using the degree of the theta divisor.
	This can be found in
	\cite{heingen}, Theorem 4.2, which originally is due
	Faltings, see \cite{faltings}, Theorem I.4.  

	Here we give a shorter proof only using \cite{bp}, Theorem 4.4, and basic properties of finite \'etale
	morphisms.
\begin{theorem}	\label{theorem-finite-fibres}
	Let $\pi:D\to C$ be a finite separable morphism of smooth projective curves.  
	Let $r\geq 1$ and $d\in \mathbf{Z}$.
	Then the induced morphism 
	\[
	    \pi^{\ast}:M^{ss,r,d}_C\to M^{ss,r,\text{deg}(\pi)d}_D 
	\]
	is finite. If $e$ denotes the degree of the \'etale part of $\pi$,
	then the fibre of $\pi^{\ast}$ at a stable bundle $W$ on $D$ has cardinality at most $e$.
\end{theorem}

\begin{proof}

	As each finite separable morphism factors as a finite \'etale and a
	genuinely ramified morphism
    it suffices to show the
	theorem for these two types of morphisms separately.
    Furthermore, as $\pi^{\ast}$ is a morphism of projective varieties
	it suffices to show that it is quasi-finite.
	
	Let $\pi$ be genuinely ramified. Consider poly-stable
	bundles $V$ and $W$ on $C$ such that $\pi^{\ast}V\cong\pi^{\ast}W$.
	Pullback along a genuinely ramified morphism preserves stability
	see \cite {bp}, Theorem 4.4. 
	Furthermore, by Lemma 4.3, \cite{bp}, if two stable bundles on $C$ are
	isomorphic after pullback by $\pi$, they are already isomorphic on $C$.
	Therefore, $V=W$.
	This shows the theorem for genuinely ramified
	$\pi$. In fact, we have shown that $\pi^{\ast}$ is injective on
	points.

	It remains to consider the case where $\pi$ is a finite \'etale morphism.
	Let $V$ be a poly-stable bundle on $C$. 
	Consider the poly-stable bundle $\pi^{\ast}V=\bigoplus W_i$, 
	where the $W_i$ are stable on $D$, see Lemma \ref{lemma-stability-pullback}.
	By Lemma \ref{lemma-etale-pushforward} all bundles $\pi_{\ast}W_i$ are semi-stable
	of slope $\mu(V)$. The projection formula implies that
	$\pi_{\ast}(\mathcal{O}_D)\otimes V=\pi_{\ast}\pi^{\ast}V$.
	Thus, $V\subseteq \pi_{\ast}\pi^{\ast}V$ appears in the JH-filtration of $\bigoplus \pi_{\ast}W_i$.
	As the graded object associated to the JH-filtration
	is unique, there are only finitely many choices for $V$ if we fix $\bigoplus W_i$.
	
	If $V_{\mid D}=W$ is stable on $D$, then comparing the ranks of $V$ and $\pi_{\ast}W$ 
	we find that there can be at most $\deg(\pi)$ many different such $V$.
\end{proof}

\subsection{Strata and dimension}
\begin{lemma}
	\label{lemma-jochen} 
	Let $\pi:D\to C$ be a finite Galois cover of smooth projective curves
	with Galois group $G$.
	Let $M$ be a simple bundle of rank $r$ on $D$ which is $G$-invariant.
	Then there exists a line bundle $L$ on $D$ such that
	$M\otimes L$ descends to $C$.
\end{lemma}

\begin{proof}
	In the following cohomology denotes \'etale cohomology.
	Note that for a smooth algebraic group $G$ a $G$-torsor over $C$
	corresponds to an element of $\check{H}^1(C,G)$ as a smooth morphism admits \'etale locally a section.
	The same holds for $D$.
	
	We have $H^2(C,\mathbb{G}_m)=0$, see
	\cite{sp}, \href{https://stacks.math.columbia.edu/tag/03RM}{Tag 03RM},
	and the same holds for $D$.
	By \cite{milne}, Corollary 2.10, p.101, we also obtain
	\[
		\check H^2(C,\mathbb{G}_m)=0=\check H^2(D,\mathbb{G}_m).
	\]

	Consider the short exact sequence
	of \'etale sheaves on $C_{\et}$
	\[
		0\to \mathbb{G}_m \to \mathrm{Gl}_r\to \mathrm{PGl}_r\to 0.	
	\]
	Applying the functors $\Gamma(D,-)$ and $\Gamma(C,-)$
	we obtain a commutative diagram of exact sequences of pointed sets
	\begin{equation}
	\label{diagram-check}
    \begin{tikzcd}
		\check H^1(D,\mathbb{G}_m) \ar[r] & \check H^1(D,\mathrm{Gl}_r) \ar[r] &
		\check H^1(D,\mathrm{PGl}_r) \ar[r]&0\\
		\check H^1(C,\mathbb{G}_m) \ar[u]\ar[r] &\check H^1(C,\mathrm{Gl}_r)\ar[u] \ar[r] &
		\check H^1(C,\mathrm{PGl}_r)\ar[u] \ar[r] & 0 \ar[u, equal].
	\end{tikzcd}
	\end{equation}
	As $\mathbb{G}_m$ lies in the center of $\mathrm{Gl}_r$ 
	exactness at $\check H^1(\mathrm{Gl}_r)$ is stronger than usual: 
	If two $\mathrm{Gl}_r$-torsors map to the same $\mathrm{PGl}_r$-torsor they differ
	by a twist of a line bundle, see Lemma
	\ref{lemma-truncated-les}.  

	The bundle $M$ is an element in $\check H^1(D,\mathrm{Gl}_r)$.
	By definition of $G$-invariance we have
	isomorphisms $\varphi_{\sigma}:M\to\sigma^{\ast}M$ for all $\sigma\in G$.
	The obstruction for descent
	$\lambda_{\sigma,\tau}:=\tau^{\ast}\varphi_{\sigma}\circ
	\varphi_{\tau}\circ\varphi_{\sigma\tau}^{-1}$ is an isomorphism of $M$.
	By assumption $M$ is simple and $\lambda_{\sigma,\tau}$ lies in $k^{\ast}$,
	i.e., considered as a $\mathrm{PGl}_r$-torsor $M$ descends to $C$, see \cite{fga}, Theorem 1.4.46.  
	By the surjectivity $\check{H}^1(C,\mathrm{Gl_r})\to \check{H}^1(C,\mathrm{PGl}_r)$
	we find a vector bundle $N$ on $C$ such that
	$N_{\mid D}=M$ as $\mathrm{PGl}_r$-torsors.
	By the exactness of (\ref{diagram-check}) the vector bundles $N_{\mid D}$ and $M$ agree up to tensoring
	with a line bundle $L$ on $D$.
\end{proof}

\begin{lemma}
	\label{lemma-high-codimension}
	Let $\pi:D\to C$ be a Galois cover of a smooth projective curve $C$ of
	genus $g_C \geq 2$.
	Let $r\geq 2$ and $d\in \mathbf{Z}$.
	Denote by $Z$ the closed subset of
	$M^{s,r,d}_C$ given by stable bundles that do not remain stable
	after pullback to $D$.
	Then $Z=Z_1\sqcup Z_2$, where 
	\[
	    Z_1:=\{ V \in M^{s,r,d}_C \mid V_{\mid D}=M^{\oplus e}, M\in M^{s,\frac{r}{e},\frac{\deg(\pi)d}{e}}_D, e\geq 2 \} \text{  and }
	\]
	\[
	    Z_2:=\{ V \in M^{s,r,d}_C \mid V_{\mid D}=\bigoplus_{i=1}^n M_i^{\oplus e}, M_i\in M^{s,\frac{r}{en},\frac{\deg(\pi)d}{en}}_D, n\geq 2 \}
	\]
	are the strata induced by Lemma \ref{lemma-pullback-galois}.
	Furthermore, 
	\[
	    \dim(Z_1)\leq r_0^2(g_C-1)+1 \text{ and } \dim(Z_2)\leq r_0r(g_C-1)+1,
	\]
	where $r_0$ is the largest proper divisor of $r$, i.e., $r_0\mid r$ and $r_0\neq r$.
	
	If $\pi$ is a prime to $p$ cover and $r$ is a power of $p$, then $Z_2$ is empty.
\end{lemma}

\begin{proof}
	Clearly, $Z=Z_1\sqcup Z_2$ by Lemma \ref{lemma-pullback-galois}.
	
	We begin by showing the estimate for $Z_1$.
	Consider $V\in Z_1$ and $M$ stable on $D$ such that $V_{\mid D}=M^{\oplus e}$ for some $e\geq 2$. As the Galois group acts trivially on the isomorphism class of $M$ we can apply Lemma \ref{lemma-jochen}.
	Thus, there is a line
	bundle $L$ on $D$ such that $M\otimes L=N_{\mid D}$ for some vector bundle
	$N$ on $C$.  After twisting $N$ by a line bundle on $C$, we can assume that
	$0\leq \text{deg}N<r$. Note that $M\otimes L$ is stable and so is $N$ by
	Lemma \ref{lemma-stability-pullback}. 
	Fixing the degree of $N$ also fixes the degree of $L$ as
	$\text{deg}M+\frac{r}{e}\text{deg}L=\text{deg}(\pi)\text{deg}(N)$.

	We have $\det(M)^{\otimes e}=\det(V)_{\mid D}$ which implies
	that $L^{\otimes r}$ descends to $C$. 
	As multiplication by $r$ on $\Pic_{D/k}$ is a finite morphism,
	we obtain that the dimension of all possible line bundles $L$ (with fixed degree)
	is at most $g_C$. Write $P(f)$ for the moduli space of line bundles on $D$
	of degree $(\text{deg}(\pi)f - \text{deg}(M))\cdot \frac{e}{r}$ such that
	their $r$-th power descends to $C$, where $f$ is an integer.
	
	Let $0\leq f < r$ and fix a line bundle $L'$ of degree $f$ on $C$.
	Consider the morphism 
	\[
	M^{s,\frac{r}{e}}_{L'}\times_k P(f)\to
	M^{ss,r,\text{deg}(\pi)d}_D, (N,L)\mapsto N_{\mid D}^{\oplus e} \otimes L
	\]
	and denote the image by $Z_{f,e}$.
	Observe that $Z_{f,e}$ is closed and so is the finite union $Z'=\bigcup_{f=0}^{r-1}\bigcup_{e\mid r,e\neq 1}Z_{f,e}$.  

	The above discussion shows that $\pi^{\ast}(Z_1)\subseteq Z'$. 
	By Theorem \ref{theorem-finite-fibres}, we have that $\pi^{\ast}$ is a finite morphism and obtain 
	$\text{dim}(Z_1)=\text{dim}(\pi^{\ast}(Z_1))$.
	Computing the dimension we find 
	\[
	\text{dim}(Z')=\text{max}_{e\mid r, e\neq 1}((r/e)^2-1)(g_C-1)
	+g_C \leq r_0^2(g_C-1)+1,
	\]
	where $r_0$ is the largest proper divisor of $r$.  
	This concludes the estimate of $\text{dim}(Z_1)$.
		
	To obtain a bound for $\text{dim}(Z_2)$ consider $V\in Z_2$.
	By Lemma \ref{lemma-decomposition-small-degree}
	there is an intermediate cover
	$D\to D'\to C$ of degree $n$ such that $V_{\mid D'}=V'\oplus W'$,
	where $V'$ is semistable of rank $r/n$ and $W'_{\mid D}$ is a direct sum of conjugates of $V'_{\mid D}$.
    
    Let $\Sigma$ be a subset of $G$ of cardinality $n$.
	Consider the morphism 
	\[
	    M^{ss,\frac{r}{n},d}_{D'}\to
	    M^{ss,r,\text{deg}(\pi)d}_D,
	    V'\mapsto \bigoplus_{\sigma\in \Sigma}\sigma^{\ast}V'_{\mid D}
	\]
	and denote the image by $Z_{D',\Sigma}$.
	Observe that $Z_{D',\Sigma}$ is closed as the image of a finite morphism
	and by construction $V_{\mid D}\in Z_{D',\Sigma}$ for some $\Sigma$
	and $D'$.
	
	Observe that $\pi^{\ast}Z_2$ is contained in the union of all such
	$Z_{D',\Sigma}$, where $D\to D'\to C$ is an intermediate cover
	and $\Sigma$ is a subset of $G$ of cardinality $n$.
	Up to isomorphism there are only finitely many intermediate covers
	$D\to D'\to C$ and clearly
	there are only finitely many $\Sigma$.
	Thus, we can estimate the dimension
	\[
	\dim(Z_2)=\dim(\pi^{\ast}Z_2)\leq
	\text{max}_{D',\Sigma}\dim(Z_{D',\Sigma}),
	\]
	where $D'$ and $\Sigma$ are as above.
	By Theorem \ref{theorem-finite-fibres} we have that
	\[
	\text{dim}(Z_{D',\Sigma})\leq \frac{r^2}{n^2}((g_D')-1)+1.
	\]
	Applying Riemann-Hurwitz we obtain
	\[
	    \dim(Z_{D',\Sigma})\leq n\frac{r^2}{n^2}(g_C-1)+1=r\frac{r}{n}(g_C-1)+1
	\]
    and conclude
	\[
	    \text{dim}(Z_2)\leq \text{max}_{n\mid r, n\neq 1}r\frac{r}{n}(g_C-1)+1=rr_0(g_C-1)+1.
	\]
	
	If $G$ is prime to $p$ and $r$ is a power of $p$, then a decomposition of the form 
	$V_{\mid D}=\bigoplus_{i=1}^n M_i^{\oplus e}, n\geq 2$,
	can not happen. Indeed, $n\rk(M_i)e=r$ and we find that $n$ is a power of $p$ as well. 
	By Lemma \ref{lemma-decomposition-small-degree} there is an intermediate cover of $D\to C$ of degree $n$.
	However, $G$ being prime to $p$ only allows for such an intermediate cover if $n=1$.
\end{proof}
	
	As a direct consequence of the dimension estimate we obtain
	the existence of stable bundles that remain stable on a fixed cover.
\begin{lemma}
	\label{lemma-etale-non-empty}
	Let $\pi:D\to C$ be cover of a smooth projective curve $C$
	of genus $g_C\geq 2$. Let $r\geq 2$ and $d$ be integers.
	Let $Z$ be the closed
	subset $Z$ of $M^{s,r,d}_{C}$ of stable bundles that are not stable 
	after pullback to $D$.
	Then $\mathrm{codim}_{M^{s,r,d}_C}(Z)\geq 2$.
	
	In particular, there are stable bundles of rank $r$ and degree $d$
	on $C$ that remain stable after pull back to $D$.
\end{lemma}

\begin{proof}
    Observe that we can replace $D\to C$ by its Galois closure.
	By Lemma \ref{lemma-high-codimension} we have 
	\[
	    \text{dim}(Z)\leq r_0 r(g_C-1)+1,
	\]
	where $r_0$ is the largest proper divisor of $r$.
    As
	\[
	    r^2(g_C-1)+1=\text{dim}(M^{s,r,d}_C),
    \]
	$g_C\geq 2$, and $r\geq 2$, we conclude
	\[
	    \mathrm{codim}(Z)\geq r(r - r_0)(g_C-1) \geq 2.
	\]
\end{proof}

The estimate obtained in Lemma \ref{lemma-high-codimension}
is sharp if the rank is prime to $p$. To show this we need a way to construct
stable bundles with prescribed decomposition behaviour after pullback.
This can be done for cyclic covers. We start with a descent lemma for such covers.
\begin{lemma} 
	\label{lemma-cyclic-descend}
	Let $Y\to X$ be a cyclic \'etale cover of varieties with Galois group $G$.
	Let $V$ be a simple sheaf on $Y$.
	Then $V$ descends to $X$ iff $V$ is $G$-invariant.
\end{lemma}

\begin{proof}
	The "only if" implication is trivial.
	For the "if" implication let $\sigma$ be a generator of $G$ of order $n$.
	Fix an isomorphism $\varphi_{\sigma}:V\to \sigma^{\ast}V$.
	For $2\leq l < n$ define $\varphi_{\sigma^l}:V\to (\sigma^l)^{\ast}V$
	inductively as the composition $\sigma^{\ast}\varphi_{\sigma^{l-1}}\circ
	\varphi_{\sigma}$. Further define $\varphi_{e}=\text{id}_V$, where $e$ denotes
	the identity of $G$.
	
	Consider $\sigma^{\ast}\varphi_{\sigma^{n-1}}\circ \varphi_{\sigma}$.
	This is an automorphism of $V$. As $V$ is simple it corresponds to 
	a scalar $\lambda \in k^{\ast}$.
	As $k$ is algebraically closed we can find an $n$-th
	root $\lambda^{1/n}$ of $\lambda$.
	The automorphisms $\psi_{\sigma^{l}}:=\lambda^{-l/n}\varphi_{\sigma^{l}}$
	define a $G$-linearization of $V$.
	Indeed, for $1\leq l,l'$ such that $l'+l<n$ we have
	\[
	    (\sigma^{l})^{\ast}\psi_{\sigma^{l'}}\circ \psi_{\sigma^{l}}=
	\lambda^{-l-l'/n}\cdot(\sigma^{l+l'-1})^{\ast}\varphi_{\sigma}\circ \dots \circ
	\sigma^{\ast}\varphi_{\sigma}\circ \varphi_{\sigma}=	
	\psi_{\sigma^{l+l'}}
	\]
	by definition. It remains to check this property for $l+l'=n$.
	We have
	\[
	(\sigma^{l})^{\ast}\psi_{\sigma^{l'}}\circ \psi_{\sigma^{l}}=
	\lambda^{-1}\cdot(\sigma^{n-1})^{\ast}\varphi_{\sigma}\circ \dots \circ
	\sigma^{\ast}\varphi_{\sigma}\circ \varphi_{\sigma}=\lambda^{-1}\lambda=1
	\]
	by definition of $\lambda$.  
\end{proof}
	
\begin{lemma}
	\label{lemma-Z_2-strata}
	Let $C$ be a smooth projective curve of genus $g_C\geq 2$.
	Let $r\geq 2$ such that $p$ is not the smallest proper divisor of $r$
	if $\text{char}(k)=p>0$.
	Then there is a prime to $p$ cyclic cover $\pi:D\to C$ such that 
	$\dim(Z_2)=rr_0(g_C-1)+1$, 
	where $r_0$ denotes the largest proper divisor of $r$
	and $Z_2\subset M^{s,r,d}_C$ is defined as in Lemma \ref{lemma-high-codimension}.
\end{lemma}

\begin{proof}
    Let $\pi:D\to C$ be a cyclic cover of degree $r/r_0$, i.e., with Galois group $\mu_{r/r_0}$. 
    As $p$ is not the smallest divisor 
    of $r$ the cover $\pi$ is prime to $p$. Note that $r/r_0$ is prime. Thus, there are no intermediate covers.
    Consider $U:=M^{s,r_0,\deg(\pi)d}_D \cap (M^{ss,r_0,\deg(\pi)d}_D \setminus \pi^{\ast} M^{s,r_0,d}_C)$.
    As $\pi^{\ast}$ is finite the set $U$ is open and non-empty. 
    Thus, $U$ has dimension 
    \[
    \dim(U)=r_0^2(g_D-1)+1=r r_0(g_C-1)+1
    \]
    by Riemann-Hurwitz. 
    
    Consider a closed point $M\in U$. Then the orbit $O$ of $M$ under the action of $\mu_{r/r_0}$ is contained in $M^{s,r_0,\deg(\pi)d}_D$.
    By Lemma \ref{lemma-cyclic-descend} the orbit $O$ has cardinality $r/r_0$ as otherwise $M$ would descend to $C$.
    Clearly, no conjugate of $M$ can descend to $C$ as well, i.e., $O\subset U$.
    
    Consider the bundle $W:=\bigoplus_{\sigma \in \mu_{r/r_0}}\sigma^{\ast}M$.
    Then $W$ has rank $r$, admits a $\mu_{r/r_0}$-linearization, and no poly-stable summand of $W$ admits a
    $\mu_{r/r_0}$-linearization. Thus, there exists $V\in M^{s,r,d}_C$ such that $V_{\mid D}=W$.
    By construction $V$ lies in $Z_2$. In particular, $\pi^{\ast}Z_2$ 
    contains the image of
    \[
        U\to M^{s,r,\deg(\pi)d}_D, M\mapsto \bigoplus_{\sigma \in \mu_{r/r_0}}\sigma^{\ast}M,
    \]
    which has dimension
    $\dim(U)$. We obtain that 
    \[
        \dim(Z_2)\geq r r_0(g_C-1)+1.
    \]
    As we already have the other inequality from Lemma
    \ref{lemma-high-codimension} we conclude.
\end{proof}

\begin{remark}
    Let $\pi:D\to C$ be a Galois cover.
    One can take a closer look at $Z_2$ and further decompose it into the strata
    \[
        Z_2(n,e):=\{ V\in M^{s,r,d}_C | V_{\mid D} = \bigoplus_{i=1}^n W_i^{\oplus e} \},
    \]
    where $V_{\mid D}=\bigoplus_{i=1}^n W_i^{\oplus e}$ is the decomposition of Lemma
    \ref{lemma-pullback-galois}. One can compute the dimension of $Z_2(n,1)$ in an analogue manner
    if $D\to C$ is prime to $p$ and cyclic of degree $n$. The only change being that one has to remove
    all bundles of rank $r/n$ arising from an intermediate cover $D\to D'\to C$, $D'\neq D$.
\end{remark}

\begin{theorem}
	\label{theorem-non-empty}
	Let $C$ be a smooth projective curve of genus $g_C\geq 2$.
	Let $r\geq 2$ and $d\in\mathbf{Z}$. Then the following hold:
	\begin{enumerate}[(i)]
		\item If $k$ is uncountable, then there are separable-stable
			bundles of rank $r$ and degree $d$. 
		\item The prime to $p$ stable bundles of rank $r$
			 and degree $d$ form a big open subset $M^{p'-s,r,d}_C\subseteq M^{s,r,d}_C$.
		\item We have
			 \[
			    \dim(M^{s,r,d}_C\setminus M^{p'-s,r,d}_C)\leq rr_0(g_C-1)+1,
			 \]
			 where $r_0$ denotes the largest proper divisor of $r$.
	        If $p$ is not the smallest proper divisor of $r$, then equality holds.		 
	\end{enumerate}
\end{theorem}

\begin{proof}
	As already mentioned separable-stable and \'etale-stable coincide
	for curves, see Corollary \ref{cor-pro-separable-pro-etale}.

	(i): By Lemma \ref{lemma-etale-non-empty} the stable bundles that remain
	stable after pull back to an \'etale cover form a non-empty open subset
	of $M^{s,r,d}_{C}$.  Note that there are only countably many \'etale covers
	of $C$ up to isomorphism.
    Intersecting countably many non-empty open subsets of a
	quasi-projective variety over an uncountable algebraically closed
	field is non-empty by Lemma \ref{lemma-countable-intersection}.

	(ii) is a direct consequence of Theorem \ref{theorem-very-large-cover}
	and Lemma \ref{lemma-etale-non-empty}.
	
	(iii) is a direct consequence of Lemma \ref{lemma-Z_2-strata} and Lemma \ref{lemma-high-codimension}.
\end{proof}

Extending a prime to $p$ stable vector bundle from a large curve to a surrounding smooth projective variety using Mathur's extension theorem, \cite{mathur} Theorem 1, we obtain the existence of prime to $p$ stable vector bundles in higher dimensions. 
However, we can not control the numerical data, i.e., which components of the stack of bundles 
admit prime to $p$ stable bundles.
\begin{cor}
    Let $X$ be a smooth projective variety of dimension $n$.
    For $r\geq n$ there are prime to $p$ stable vector bundles on $X$.  
\end{cor}

\begin{cor}
	Let $C$ be a smooth projective curve of genus $g_C \geq 2$.
	Let $r\geq 2$ and $d\in\mathbf{Z}$. Then 
	$\Pic(M^{p'-s,r,d}_{C})=\Pic(M^{s,r,d}_{C})$.
\end{cor}
\begin{proof}
	By Theorem \ref{theorem-non-empty} we have that $M^{p'-s,r,d}_C\subseteq M^{s,r,d}_C$ is a big open.
	As $M^{s,r,d}_C$ is smooth they have the same Picard group.
\end{proof}
As the general bundle is prime to $p$ stable, we obtain:
\begin{cor}
\label{cor-not-dense}
    Let $C$ be a smooth projective curve of genus $g_C\geq 2$.
    Let $r\geq 2$. Then the stable bundles of rank $r$
    that are trivialized on a prime to $p$ cover
    are not dense in $M^{s,r,0}_C$.
\end{cor}

\section{Appendix}

\begin{lemma}
\label{lemma-truncated-les}
	Let $\mathcal{C}$ be a site and $1 \to A \to B \xrightarrow{\varphi} C \to 1$ 
	be a short exact sequence of sheaves of (possibly non-commutative)
	groups.  Then for $U\in \mathcal{C}$ we obtain a truncated long exact sequence 
	\[ 
	\begin{tikzcd} 
		1 \ar[r] & \check H^0(U,A) \ar[r] & \check H^0(U,B) \ar[r] & \check H^0(U,C)
		\ar[out=0, in=180, looseness=2, overlay]{dll}\\ & \check H^1(U,A)
		\ar[r] & \check H^1(U,B) \ar[r] &
	\check H^1(U,C),
	\end{tikzcd} 
	\]
	where the $0$-th cohomology groups are groups
	while the first cohomology groups are only pointed sets. 
	
	If $A$ is central in $B$, then $\check
	H^1(B)(U)$ has an $\check H^1(A)(U)$-action and $b,b'\in \check H^1(B)(U)$
	map to the same element in $\check H^1(C)(U)$ iff there is an $a\in \check
	H^1(A)(U)$ such that $ab=b'$.
\end{lemma}

\begin{proof}
    This is proven in \cite{milne}, p.122, chapter III, Proposition 4.5,
    except for the "$A$ central in $B$" part.  
    Assume in the following that $A$ is central in $B$.
	
	Let $\mathcal{U}=(U_i\to U)$ be a cover of $U$ in $\mathcal{C}$.
	For $a\in \check H^1(A)(\mathcal{U})$ and $b\in \check H^1(B)(\mathcal{U})$
	define $a\cdot b:=(a_{ij}b_{ij})$, where $(a_{ij})_{ij}$ (resp. $(b_{ij})_{ij}$)
	is a representative of $a$ resp. $b$.
	To see that this is well-defined note that for $(a_i)\in \prod_{i} A(U_i)$
	and $(b_i)\in \prod_{i} B(U_i)$
	we have
	\[
	    a_i a a_j^{-1} \cdot b_i b b_j^{-1} = a_ia_{ij}a_j^{-1}b_ib_{ij}b_j^{-1}=
	    a_ib_i a_{ij}b_{ij}(a_jb_j)^{-1}
	\]
	using that $A$ lies central in $B$.
	Thus, we have defined an action of $\check H^1(A)(\mathcal{U})$ on $\check
	H^1(B)(\mathcal{U})$ which clearly is compatible with restriction
	along a refinement. Taking colimits
	we obtain an action of $\check H^1(A)(U)$ on $\check H^1(B)(U)$.
	
	Let $b,b'\in\check H^1(B)(U)$ map to the same element $c\in \check H^1(C)(U)$.
	Then there is a cover $\mathcal{U}=(U_i\to U)$ such that 
	$b,b'$ lift to elements $b,b'\in \check H^1(B)(\mathcal{U})$ which map to
	$c\in\check H^1(C)(\mathcal{U})$.
	Choose representatives $b=(b_{ij}), b'=(b'_{ij}),$ and $c=(c_{ij})$. 
	As $b$ and $b'$ both map to $c$ there exists
	$(c_i)\in \prod_{i} C(U_i)$ such
	that $c_i\varphi(b_{ij})c_j^{-1}=\varphi(b'_{ij})$.
	After refining the cover $\mathcal{U}$ we can assume that there is
	$(b_i)\in \prod B(U_i)$ mapping to $(c_i)$.
	We obtain
	$\varphi(b_ib_{ij}b_j^{-1} b_{ij}'^{-1})=1\in C(U_{ij})$, i.e.,
	$a_{ij}:=b_i b_{ij}b_j^{-1} b_{ij}'^{-1}\in A(U_{ij})$.
	
	As $A$ is central, this defines an element $a=(a_{ij})\in\check H^1(A)(\mathcal{U})$
	and thus in $H^1(A)(U)$.
	By construction $a\cdot b'=b\in \check H^1(B)(U)$.  
	Indeed,
	\[
	    a_{ij}b'_{ij}=b_i b_{ij}b_j^{-1}b_{ij}'^{-1} b'_{ij} = b_i
	    b_{ij}b_j^{-1} \sim b_{ij}.
	\]
\end{proof}

\begin{lemma}
	\label{lemma-countable-intersection}
	Let $X$ be an algebraic scheme over an algebraically closed field $k$ of
	cardinality $\kappa$. Assume that
	$X$ has positive dimension.  Then for a family $(Z_l)_{l\in\lambda}$
	of closed subsets of $X$ of
	codimension $\geq 1$
	we have $X\neq \bigcup_{l\in\lambda}Z_l$
	if $\lambda<\kappa$.
\end{lemma}

\begin{proof} 
	An algebraic scheme has only finitely many irreducible components
	and we reduce to the case where $X$ is a variety.
	The lemma is clear for finite $\lambda$ and we further assume that
	$\lambda$ is infinite.

	Any closed subset of an algebraic scheme only has finitely many irreducible
	components and it suffices to show the lemma for a family of irreducible
	closed subsets of codimension $\geq 1$.  

	Let us first proof the result for $\mathbf{A}^n_k$ and $n\geq 1$ by induction.  
	For $n=1$ we want to show that a union of $\lambda<\kappa$ many closed points
	is not $\mathbf{A}^1_k$. This is true as closed points are in bijection to
	elements of $k$ and $\#k=\kappa$.
	
	Induction step by contradiction: Assume the lemma for $\mathbf{A}^n$ 
	and some $n\geq 1$.
	Further, assume that $\bigcup_{l\in\lambda}Z_l=\mathbf{A}^{n+1}$,
	where the $Z_l$ are closed irreducible subsets in $\mathbf{A}^{n+1}$ of codimension $\geq 1$.  
	Consider for a closed point $x\in
	\mathbf{A}^{1}_k$ the morphism 
	\[
	    \varphi_x:\mathbf{A}^{n}_k\to
	    \mathbf{A}^{n+1}_k, (x_1,\dots,x_{n})\mapsto (x_1,\dots,x_{n},x).
	\]
	Then $(\varphi_{x}^{-1}(Z_l))f_{l\in\lambda}$ is a family of closed subsets of
	cardinality $\lambda<\kappa$ such that their union is $\mathbf{A}^{n}$.
	By the induction hypothesis this implies that
	$\varphi_x^{-1}(Z_{l_x})=\mathbf{A}^n$
	for some $l_x\in \lambda$. The irreducibility of $Z_{l_x}$ and
	$\text{codim}(Z_{l_x})\geq 1$ imply $Z_{l_x}=\mathbf{A}^n_k\times \{x\}$.
	Mapping $x\in\mathbf{A}^1_k$ to $l_x$ we obtain a map
	$\varphi:\kappa\to\lambda$. As $\lambda<\kappa$ the map $\varphi$
	can not be injective. A contradiction to $pr_{n+1}(Z_{l_x})=\{ x\} $, where
	$pr_{n+1}$ denotes the $n+1$-th projection.

	Let $U=\Spec(R)$ be a non-empty affine open of $X$ of
	dimension $d\geq 1$.
	By the Noether normalization lemma there is a finite morphism
	$\varphi:U\to \mathbf{A}^d_k$. Note that $\varphi$ is surjective. Indeed,
	the image of $\varphi$ has dimension $d$, is closed, and $\mathbf{A}^d_k$
	is irreducible.
	If $X$ is the union of $\lambda<\kappa$ many closed
	subsets, then the same is true for $U$. Finite morphisms preserve closed
	subsets and we also obtain $\mathbf{A}^d_k$ as such a union. 
	This contradicts the already known statement for $\mathbf{A}^d_k$.
\end{proof}

\begin{theorem}[Functorial Bertini]
	\label{functorial-bertini}
	Let $f:Y\to X$ be a finite quasi-\'etale morphism of projective varieties of dimension $\geq 2$.
	Further assume that $X$ is normal.
	Then the general complete intersection curve $C$ in $c_1(\mathcal{O}_X(-N_1))\cdots c_1(\mathcal{O}_X(-N_{n-1})), N_i\gg 0,$
	is a smooth projective curve such that $Y\times_X C$ is a smooth projective curve as well.
	
	If $Y\to X$ is Galois with Galois group $G$, then so is
	$Y\times_X C\to C$.
\end{theorem}

\begin{proof}
	For $N\gg 0$ the linear system $|\mathcal{O}_X(N)|$ defines
	a closed immersion $X\to \mathbf{P}^n_k$ for some $n$.
	Apply \cite{jou}, Corollaire 6.11 (3), to $Y\to X\to \mathbf{P}^n_k$ 
	to obtain that the general hyperplane $H$ in $|\mathcal{O}_X(N)|$ is irreducible after pullback to $X$ and $Y$.
	By \cite{sei}, Theorem 7, the general hyperplane $H$ satisfies that $H\cap X$ is normal.
	Furthermore, the general hyperplane also intersects the non-\'etale locus of $f$ transversally,
	i.e., $Y\times_{\mathbb{P}^n_k} H\to X\times_{\mathbb{P}^n_k}H$ is quasi-\'etale.
	
	Thus, by induction we find a complete intersection curve $C$ in $\mathbb{P}^n_k$ such that
	$C_{\mid Y}\to C_{\mid X}$ is an \'etale morphism of irreducible projective varieties of dimension $1$
	and $C_{\mid X}$ is smooth. Then clearly $C_{\mid Y}$ is smooth as well.
	
	The second part of the statement is clear by construction as $C$ lies in the \'etale locus of $f$.
\end{proof}

\begin{lemma}
	\label{lemma-galois-tower-away-from-char}
	Fix a prime $q$.
	Let $M/L/K$ be a tower of field extensions. Assume that $M/L$ and $L/K$ are Galois
	and $q\nmid [M:K]$.
	Then the Galois closure $F$ of $M/K$ satisfies $q\nmid [F:K]$.
\end{lemma}

\begin{proof}
	As $M/K$ is separable there is an $\alpha\in M$ such that $M=K(\alpha)$,
	see \href{https://stacks.math.columbia.edu/tag/030N}{Tag 030N}.
	Clearly, $M=L(\alpha)$. Let $f$ be the minimal polynomial of $\alpha$ over $K$
	and $g$ the minimal polynomial of $\alpha$ over $L$.
	We have $g\mid f$ as polynomials over $L$. 
    For $\sigma \in \mathrm{Gal}(L/K)$ we obtain $\sigma^{\ast}g\mid
	\sigma^{\ast}f=f$. 
	
	We claim that $\prod_{\sigma\in \mathrm{Gal}(L/K)}\sigma^{\ast}g=f$ 
	is the prime factorization of $f$ in $L[x]$.
	Indeed, if $g=\sigma^{\ast}g$ for some $\sigma\in G$, then
	the coefficients of $g$ lie in $L^{H}$, where $H$ is the subgroup of $G$ 
	generated by $\sigma$. In particular, $[M:L]=[M:L^H]$ and we obtain $L=L^H$, i.e., $\sigma$ must be trivial.
	Thus, the prime factorization of $f$ is of desired form up to a unit $u\in L$.
	However, all polynomials in question are monic and we obtain $u=1$.
	
	Define $M_{\sigma}:=L[x]/\sigma^{\ast}g$. The isomorphism $\sigma:L\to L$
	induces an isomorphism $M=L[x]/g\to M_{\sigma}$. As $M/L$ is Galois so
	is $M_{\sigma}/L$. The Galois closure $F$ of $M/K$ is the composite
	$\prod_{\sigma\in \mathrm{Gal}(L/K)} M_{\sigma}$.
	Inductively, we obtain $q\nmid [M_{\sigma_1}\dots M_{\sigma_k}:K]$
	as we have a subgroup
	\[
	    \mathrm{Gal}\big(M_{\sigma_1}\dots M_{\sigma_k}/K\big)\subseteq 
	    \mathrm{Gal}\big(M_{\sigma_2}\dots M_{\sigma_k}/K\big)\times \mathrm{Gal}\big(M_{\sigma_1}/K\big).
	\]
	%as the Galois group of a composite is a subgroup of the product of the Galois groups.
	%bosch, 4.1, proposition 12, Algebra From the Viewpoint of Galois Theory
\end{proof}


\printbibliography

\end{document}
