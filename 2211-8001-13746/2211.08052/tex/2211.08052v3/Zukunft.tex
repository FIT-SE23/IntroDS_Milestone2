\section{Future stability}\label{sec:fut}

\noindent The goal of this section is to show the following theorem:

\begin{theorem}[Future stability of Milne spacetime]\label{thm:fut-stab-simple} Let the rescaled initial data $(\fg,\bm{k},\nabla\phi,\phi^\prime)$ on $M$ be sufficiently close to $(\gamma,\frac13\gamma,0,0)$ in $H^5\times H^4\times H^4\times H^4$ on some initial hypersurface $\Sigma_{\tau=\tau_0}$ (see Definition \ref{def:fut-rescaled} and Assumption \ref{ass:fut-init}). Then, its maximal globally hyperbolic development $(\M,\g,\phi)$ within the Einstein scalar-field system in CMCSH gauge is foliated by the CMC Cauchy hypersurfaces $(\Sigma_{\tau})_{\tau\in[\tau_0,0)}$, is future (causally) complete and exhibits the following asymptotic behaviour:
\[(\fg,\bm{k},\phi^\prime,\nabla\phi)(\tau)\longrightarrow(\gamma,\frac13\gamma,0,0)\text{ as }\tau\uparrow 0\]
\end{theorem}

\noindent Since the control of geometric perturbations uses the same arguments as in \cite{AndFaj20}, the focus in this section will lie on dealing with the the scalar field. The key idea is controlling decay of the scalar field using an indefinite corrective term on top of the canonical energy (see Definition \ref{def:fut-stab}).

\subsection{Preliminaries}\label{subsec:fut-prelim}

\subsubsection{Notation, gauge and spatial reference geometry}

Within this section, we will decompose the Lorentzian metric as follows:
\begin{subequations}
\begin{equation}\label{eq:fut-metric}
\g=-n^2dt^2+g_{ab}(dx^a+X^a)(dx^b+X^bdt)
\end{equation}
We impose CMCSH gauge (see \cite{AM03}) via
\begin{equation}\label{eq:CMCSH}
t=\tau,\,g^{ij}(\Gamma^{a}_{ij}-\Gamhat^{a}_{ij})=0\,,
\end{equation}
\end{subequations}
where $\Gamhat$ refers to the Christoffel symbols with regards to the spatial reference metric $\gamma$.\\
We extend the notation from the Big Bang stability analysis regarding foliations, derivatives, indices and schematic term notation to this setting (see Section \ref{subsec:notation}). In particular, $\Sigma_{T}$ and $\Sigma_{\tau}$ will refer to spatial hypersurfaces along which the logarithmic time $T$ (see \eqref{eq:fut-time}) and the mean curvature $\tau$ are constant (see \eqref{eq:fut-time} on why these are interchangeable), and we will write for example $\Sigma_{T=0}$ when inserting a specific value to avoid potential ambiguity. We use similar notation for scalar functions and tensors that depend on $T$ or, respectively, $\tau$.\\

For the extent of the future stability analysis, we have to introduce an additional condition for the spatial geometry beyond Definition \ref{def:spatial-mf}:

\begin{definition}[Spectral condition for the Laplacian of the spatial reference manifold]\label{def:spatial-mf-spectral}
Let $\mu_0(\gamma)$ to be the smallest positive eigenvalue of the Laplace operator $-\Lap_\gamma=\changefinal{-(\gamma^{-1})^{ab}\nabhat_a\nabhat_b}$ acting on scalar functions, where $(M,\gamma)$ is as in Definition \ref{def:spatial-mf}. $(M,\gamma)$ additionally is assumed to satisfy
\[\mu_0(\gamma)>\frac19\,.\]
\end{definition}

\begin{remark}[\change{Manifolds that satisfy Definition \ref{def:spatial-mf-spectral}}]\label{rem:weeks-and-friends}

\changefinal{The available literature on spectra of $-\Lap_\gamma$ usually focuses on hyperbolic manifolds with sectional curvature $\kappa=-1$. Thus, one needs to check that $\mu_0$ is strictly greater than $1$ to verify the analogue of Definition \ref{def:spatial-mf-spectral} after rescaling.\\%the homothety $\gamma\leadsto \frac19\gamma$.

\noindent Numerical works, e.g., \cite{Cornish99, Ino01}, provide evidence for over 250 compact hyperbolic $3$-manifolds to satisfy this spectral bound, many of which are closed. In particular, both \cite{Cornish99}\footnote{These results have to be interpreted cautiously since the numerical method cannot detect eigenvalues below $1$.} and \cite{Ino01} consider the smallest closed orientable hyperbolic $3$-manifold, the Weeks space m003(-3,1), and compute that it falls under Definition \ref{def:spatial-mf-spectral} with $\mu_0\approx 27,8$ in \cite[Table IV]{Cornish99} and $26\lessapprox\mu_0\lessapprox 27,8$ in \cite[Table 2]{Ino01}. Moreover, as demonstrated in \cite[Figure 6]{Ino01}, many manifolds with small enough diameter $d$ satisfy this condition. In fact, the analytical bound
%\[\mu_0\geq\max\left\{\frac{\pi^2}{d^2}-2,\frac{8}{d^2}\exp\left(-\frac{d^2}{4}\right),\frac{8}{d^2}\left(1+\frac{2d}3\right)\exp\left(-d\right)\right\}\,\]
\[\mu_0\geq\max\left\{\frac{\pi^2}{2d^2}-\frac12,\sqrt{\frac{\pi^4}{d^4}+\frac14}-\frac32,\frac{\pi^2}{d^2}e^{-d}\right\}\]
(see \cite[Theorem 1.1-1.2]{ChZ95} with $L=2$) implies that $\mu_0>10$ holds for Weeks space, which has diameter $d\approx 0,843$ (see \cite[Table V]{Cornish99}). Furthermore, \cite{Ino01} finds no closed hyperbolic manifolds that violate this bound. More recently, the Selberg trace formula has been used in \cite{LinLip22,LinLip24,BoMaPa25} to compute candidates for eigenvalues of $-\Lap_\gamma$ and related operators, based on an optimization approach originating in \cite{BooSt07}. In particular, the calculations visualized in \cite[Figure 3]{BoMaPa25} demonstrate that one must have $\mu_0\geq 27,6$ on the Weeks manifold.\\

We also note that it is conjectured that one at least has $\mu_0\geq 1$ for any arithmetic hyperbolic 3-manifold (see \cite[Conjecture 2.3]{Ber03}). In fact, this is tied to the Ramanujan conjecture for automorphic forms. Finally, one can construct compact manifolds with boundary and with constant sectional curvature $-1$ where $\mu_0$ becomes arbitrarily small, see \cite[Corollary 4.4]{Cal94}.}
\end{remark}



\subsubsection{Rescaled variables and Einstein equations}

We will use the standard rescaling of the solution variables by $\tau$:

\begin{definition}[Rescaled variables for future stability]\label{def:fut-rescaled}
\begin{subequations}
\begin{gather}
\fg_{ij}=\tau^2g_{ij}\,,(\fg^{-1})^{ij}=\tau^{-2}g^{ij}\,,\ \fk_{ij}=\tau \hat{k}_{ij}\label{eq:fut-resc-metric}\\
\fn=\tau^2 n\,,\ \fN=\frac{\fn}3-1\,,\ \fX^a=\tau X^a\label{eq:fut-resc-gauge}
\end{gather}
Furthermore, we introduce the logarithmic time 
\begin{equation}\label{eq:fut-time}
T=-\log\left(\frac{\tau}{\tau_0}\right)\,\Leftrightarrow\,\tau=\tau_0e^{-T}
\end{equation}
which satisfies $\del_T=-\tau\del_\tau$. Toward the future, $\tau$ increases from $\tau_0$ to $0$, and thus $T$ increases from $0$ to $\infty$. We additionally introduce:
\begin{gather}
\fdel=\del_T+\Lie_{\fX}=-\tau(\del_\tau-\Lie_X)\label{eq:fut-del0}\\
\phi^\prime=\fn^{-1}\fdel\phi=n^{-1}(-\tau)^{-1}(\del_\tau-\Lie_X)\phi\label{eq:fut-delphi}
\end{gather}
Moreover, for any scalar function $\zeta$, we denote by $\overline{\zeta}$ the mean integral with respect to $(\Sigma_{T},\fg_T)$.
\end{subequations}
\end{definition}
\noindent For symmetric $(0,2)$-tensors $h$, we define the \change{perturbed Lichnerowicz Laplacian}
\begin{equation}\label{eq:LG}
\LG h_{ab}=-\frac1{\mu_{\fg}}\nabhat_k\left((\fg^{-1})^{kl}\mu_{\fg}\nabhat_lh_{ab}\right)-2\Riem[\gamma]_{akbl}(\fg^{-1})^{kk^\prime}(\fg^{-1})^{ll^\prime}h_{k^\prime l^\prime}\,.
\end{equation}
\change{This operator satisfies
\begin{equation}\label{eq:fut-Ric-ell}
\left(\Ric[\fg]-\Ric[\gamma]\right)_{ij}=\frac12\LG(\fg-\gamma)_{ij}+J_{ij},\quad \|J\|_{H^{l-1}}\lesssim \|\fg-\gamma\|_{H^l}\,,
\end{equation}
see \cite[Pf. of Theorem 3.1]{AM03B}. }Under our conditions for the reference geometry, \cite{Kroen15} implies that the smallest positive eigenvalue of $\mathcal{L}_{\gamma,\gamma}$, denoted by $\lambda_0$, satisfies $\lambda_0\geq\frac19$, and that $\mathcal{L}_{\gamma,\gamma}$ has trivial kernel. The spectral condition in Definition \ref{def:spatial-mf-spectral} is not necessary for this to hold true.\\

\change{We now collect the $(3+1)$-decomposition of the Einstein scalar-field equations in CMCSH gauge with the help of \cite[(2.13)-(2.18)]{AndFaj20}:

%\begin{lemma}\label{lem:fut-matter-resc}[Rescaled matter components] The rescaled matter quantities as in \cite[(2.22)]{AndFaj20} take the following form in the Einstein scalar-field system:
%\begin{subequations}
%\begin{align}
%\rho=&\,8\pi(-\tau)^{-3}n^2T^{00}=2\pi\tau^{-1}\left[\lvert\phi^\prime\rvert^2+\lvert\nabla\phi\rvert_{\fg}^2\right]\label{eq:fut-rho}\\
%\underline{\eta}=&\,4\pi(-\tau)^{-5}T=6\pi\tau^{-3}\lvert\phi^\prime\rvert^2-2\pi\tau^{-3}\lvert\nabla\phi\rvert_{\fg}^2\label{eq:fut-eta}\\
%\jmath^a=&\,8\pi(-\tau)^{-5}n{T^{0a}}=8\pi\tau\cdot\phi^\prime\cdot (\fg^{-1})^{ac}\nabla_c\phi\label{eq:fut-j}\\
%S_{ab}=&\,8\pi(-\tau)^{-1}\left[T_{ab}-\frac12g_{ab}{T^{\mu}}_{\mu}\right]=8\pi(-\tau)^{-1}\nabla_a\phi\nabla_b\phi\label{eq:fut-S}
%\end{align}
%\end{subequations}
%\end{lemma}
%
%With this notation and rewriting the wave equation, we can carry over the CMCSH system from \cite[(2.13)-(2.18)]{AndFaj20}:

\begin{lemma}[Rescaled CMCSH equations] The rescaled CMCSH Einstein scalar-field equations take the following form:
\begin{subequations}
The constraint equations
\begin{equation}\label{eq:fut-constr}
R[\fg]-\lvert\fk\rvert_{\fg}^2-\frac23=%4\tau\rho
8\pi\left[\lvert\phi^\prime\rvert^2+\lvert\nabla\phi\rvert_{\fg}^2\right],\ \div_{\fg}\fk_b=%2\tau^2\fg_{ab}\jmath^b
8\pi\tau^{3}\phi^\prime\nabla_{b}\phi,
\end{equation}
the elliptic lapse and shift equations
\begin{align*}
\left(\fLap-\frac13\right)\fn=&\,\fn\left(\lvert\fk\rvert_{\fg}^2+4\pi\left[\lvert\phi^\prime\rvert^2+\lvert\nabla\phi\rvert_{\fg}^2\right]\right)-1,\label{eq:fut-lapse-eq}\numberthis\\
%\left(\fLap-\frac13\right)\fn=&\,\fn\left(\lvert\fk\rvert_{\fg}^2+\tau\eta\right)-1,\label{eq:fut-lapse-eq}\numberthis\\
\fLap \fX^a+(\fg^{-1})^{ab}\Ric[\fg]_{bm}\fX^m=&\,2(\fg^{-1})^{am}(\fg^{-1})^{bn}\nabla_b\fn\cdot\fk_{mn}-(\fg^{-1})^{ab}\nabla_b\fN+8\pi\fn\tau^3\phi^\prime\nabla_b\phi\numberthis\label{eq:fut-shift-eq}\\%+2\fn\tau^2\jmath^a
&\,-2(\fg^{-1})^{bk}((\fg^{-1})^{cl}\fn\cdot\fk_{bc}-\nabla_b \fX^l)(\Gamma^a_{kl}-\Gamhat^a_{kl})\,,
\end{align*}
the geometric evolution equations
\begin{align*}
\fdel\fg_{ab}=&\,2\fn\fk_{ab}+2\fN\fg_{ab}\,\numberthis\label{eq:fut-eq-g}\,,\\
\fdel(\fg^{-1})^{ab}=&\,-2\fn(\fg^{-1})^{ac}(\fg^{-1})^{bd}\fk_{cd}-2\fN(\fg^{-1})^{ab}\numberthis\label{eq:fut-eq-g-1}\,,\\
\fdel\fk_{ab}=&\,-2\fk_{ab}-\fn\left(\Ric[\fg]_{ab}+\frac29\fg_{ab}\right)+\nabla_a\nabla_b\fn\numberthis\label{eq:fut-eq-Sigma}\\
&\,+2\fn\cdot(\fg^{-1})^{mn}\fk_{am}\fk_{bn}-\frac13\fN\fg_{ab}-\fN\Sigma_{ab}-8\pi \fn\nabla_a\phi\nabla_b\phi
%+N\tau S_{ab}
\end{align*}
and the wave equation
\begin{equation}\label{eq:fut-wave}
\fdel\phi^\prime=\langle \nabla\fn,\nabla\phi\rangle_{\fg}+\fn\fLap\phi+(1-\fn)\phi^\prime\,.
\end{equation}
\end{subequations}
\end{lemma}}

\subsubsection{Energies and data assumptions}

The proof will rely on the following corrected energy quantities:

\begin{definition}[Energies for future stability]\label{def:fut-stab}
\begin{subequations}
\begin{align*}
\fE^{(l)}=&\,(-1)^l\int_{M}\left[\phi^\prime\fLap^l\phi^\prime-\phi\fLap^{l+1}\phi\right]\vol{\fg}, \qquad
\fC^{(l)}=(-1)^l\int_{M}(\phi-\phim)\fLap^l\phi^\prime\,\vol{\fg}\numberthis\\
E_{SF}=&\,\sum_{m=0}^4 \left(\fE^{(m)}+\frac23\fC^{(m)}\right)\numberthis\\
\fEg=&\,\sum_{m=1}^5\biggr(\frac92\int_{M}\langle\fg-\gamma,\LG^m(\fg-\gamma)\rangle_{\fg}\vol{\fg}+\frac12\int_{M}\langle6\fk,\LG^{m-1}(6\fk)\rangle_{\fg}\vol{\fg}\numberthis\\
&\,\phantom{\sum_{m=1}^l}+c_E\int_{M}\langle 6\fk,\LG^{m-1}(\fg-\gamma)\rangle_{\fg}\vol{\fg}\biggr)
\end{align*}
\end{subequations}
The constant $c_E$ is given by
\begin{equation}\label{eq:fut-corr-const}
c_E=\begin{cases}
1 & \lambda_0>\frac19 \\
9(\lambda_0-\epsilonnew^\prime) & \lambda_0=\frac19\,,
\end{cases}
\end{equation}
where $\epsilonnew^\prime>0$ is chosen to be small enough within the argument.
\end{definition}

The Sobolev norms $H_{\fg}^l$ and $C_{\fg}^l$ are defined analogously to Definitions \ref{def:sob-norms} and \ref{def:sup-norms}, with similar conventions on suppressing time dependence in notation whereever possible. Since norms with respect to $\fg$ and $\gamma$ are equivalent under the bootstrap assumption (and consequently throughout the entire argument), we will simply denote the norms by $H^l$ and $C^l$ throughout unless the specific metric is crucial.

\begin{assumption}[Initial data assumption]\label{ass:fut-init} The initial data on the spatial hypersurface $\Sigma_{T=0}$ is assumed to be small in the following sense:
\begin{align*}
\numberthis\label{eq:fut-init}\change{\|\fg-\gamma\|_{C^3}+\|\fk\|_{C^2}+\|\fN\|_{C^4}+\|\fX\|_{C^4}+\|\phi^\prime\|_{C^2}+\|\nabla\phi\|_{C^2}&\\
\change{+\|\fg-\gamma\|_{H^5}+\|\fk\|_{H^4}+\|\fN\|_{H^6}+\|\fX\|_{H^6}+\|\phi^\prime\|_{H^4}+\|\nabla\phi\|_{H^4}}}&\,\leq\epsilonnew^2
\end{align*}
\end{assumption}


\begin{remark}[Local well-posedness toward the future]\label{rem:fut-lwp}
Under the above initial data assumption, local well-posedness is satisfied by analogizing the arguments for local well-posedness in the vacuum setting (see \change{\cite[Theorem 3.1]{AM03B}}) with the matter coupling added. Since this only consists of adding another wave equation to the hyperbolic system, the argument is structurally unchanged given appropriate smallness assumptions on $\phi^\prime$ and $\nabla\phi$ (where $\phi$ itself does not enter into the Einstein system). \change{As before, we can without loss of generality assume that the initial is sufficiently regular to ensure that }$E_{geom},\,\E^{(l)}_{SF}$ and $\fC^{(l)}$ initially are continuously differentiable (in time) for any $l\leq 4$. \delete{Beside obtaining this existence result, we will only need smallness of initial data of up to one order less to prove our stability result.}
\end{remark}

\begin{assumption}[Bootstrap assumption]\label{ass:fut-bootstrap} On the bootstrap interval $T\in[0,T_{Boot})$, \change{we assume one has}
%\begin{subequations}
\begin{align*}
\change{\label{eq:fut-bootstrap}\numberthis\|\fg-\gamma\|_{C^3}+\|\fk\|_{C^2}+\|\fN\|_{C^4}+\|\fX\|_{C^4}+\|\phi^\prime\|_{\change{C^2}}+\|\nabla\phi\|_{C^2}}&\\
\change{+\|\fg-\gamma\|_{H^5}+\|\fk\|_{H^4}+\|\fN\|_{H^6}+\|\fX\|_{H^6}+\|\phi^\prime\|_{H^4}+\|\nabla\phi\|_{H^4}}&\change{\,\leq\epsilonnew e^{-\frac{T}2}\,.}
\end{align*}
%as well as
%\begin{equation*}
%\label{eq:fut-bootstrap-phi-mean}\|\phi\|_{C^0(\Sigma_T)}\leq \|\phi\|_{C^0(\Sigma_{T=0})}+1\,.
%\end{equation*}
%\end{subequations}
\end{assumption}

\noindent We only choose not to use \enquote{$\lesssim$}-notation in the above assumptions for notational convenience in some technical computations. As before, $\epsilonnew$ can be chosen to have been sufficiently small for the following estimates to hold and for the decay estimates we derive from the bootstrap assumptions to be strict improvements. Moreover, note that \eqref{eq:fut-bootstrap} is satisfied since all of the norms are continuous in time (see Remark \ref{rem:fut-lwp})\delete{, and \eqref{eq:fut-bootstrap-phi-mean} is satisfied local-in-time since the spatial hypersurfaces are compact and $\phi$ is continuous.}\\

Before moving on to the energy estimates, we quickly collect the following immediate consequence of the bootstrap assumptions:

\begin{lemma}[Sobolev estimate for the curvature]\label{lem:fut-Ric-est}
The following estimate holds for any $l\in\N_0$:
\begin{subequations}
\begin{equation}\label{eq:fut-Ric-est}
\left\|\Ric[\fg]+\frac29\fg\right\|_{H^l}\lesssim \|\fg-\gamma\|_{H^{l+2}}+\|\fg-\gamma\|_{H^{l+1}}^2
\end{equation}
Under the bootstrap assumptions, this implies
\begin{equation}\label{eq:fut-Ric-bs}
\left\|\Ric[\fg]+\frac29\fg\right\|_{C^1}+\left\|\Ric[\fg]+\frac29\fg\right\|_{H^3}\lesssim \epsilonnew e^{-\frac{T}2}
\end{equation}
\end{subequations}
\end{lemma}

\begin{proof}
By \change{\eqref{eq:fut-Ric-ell}}, one has
\begin{align*}
\left\|\Ric[\fg]+\frac29\fg\right\|_{H^l}\leq&\,\frac12\|\LG(\fg-\gamma)\|_{H^l}+K\|\fg-\gamma\|_{H^{l+1}}^2\,
\end{align*}
for some suitably large $K>0$ along with the fact that $\LG$ is elliptic. This implies the first inequality, while the latter follows from directly from the bootstrap assumption \eqref{eq:fut-bootstrap} and by applying the standard Sobolev embedding.
\end{proof}

\subsection{Elliptic estimates}\label{subsec:fut-ell-est} We briefly collect the elliptic estimates for lapse and shift:

\begin{lemma}[Elliptic estimates for lapse and shift]\label{lem:fut-ell-est}
Let $l\in\{3,4,5,6\}$. Then, one has $\fn\in(0,3)$ (thus $\fN\in(-1,0)$) and the following estimates hold:
\begin{subequations}
\begin{align*}
\numberthis\label{eq:ell-est-lapse}\|\fN\|_{H^l}\lesssim&\change{\,\epsilonnew e^{-\frac{T}2}\|\fk\|_{H^{l-2}}+\epsilonnew^2e^{-T}\|\fg-\gamma\|_{H^{l-2}}+\epsilonnew e^{-\frac{T}2}\left[\|\phi^\prime\|_{H^{l-2}}+\|\nabla\phi\|_{H^{l-2}}\right]}\\
\numberthis\label{eq:ell-est-shift}\|\fX\|_{H^l}\lesssim&\,\change{\epsilonnew e^{-\frac{T}2}\|\fk\|_{H^{l-2}}+\epsilonnew e^{-\frac{T}2}\|\fg-\gamma\|_{H^{l-1}}+\epsilonnew e^{-\frac{T}2}\left[\|\phi^\prime\|_{H^{l-2}}+\|\nabla\phi\|_{H^{l-2}}\right]}
\end{align*} 
\end{subequations}
\end{lemma}
\begin{proof}
The pointwise bounds on $\fn$ follow via \eqref{eq:fut-lapse-eq} and the maximum principle as in Lemma \ref{lem:lapse-maxmin}. \change{For the remaining estimates, applying elliptic regularity theory to \eqref{eq:fut-lapse-eq} and \eqref{eq:fut-shift-eq} implies:}
\begin{align*}
\|\fN\|_{H^l}\lesssim&\,\change{\|\fk\|_{C^{\lfloor\frac{l-2}2\rfloor}}\|\fk\|_{H^{l-2}}+\|\nabla\phi\|_{C^2}^2\|\fg-\gamma\|_{H^{l-2}}}\\
&\,\change{+\left[\|\nabla\phi\|_{C^2}\left(1+\|\fg-\gamma\|_{C^2}\right)+\|\phi^\prime\|_{C^2}\right]\left[\|\phi^\prime\|_{H^{l-2}}+\|\nabla\phi\|_{H^{l-2}}\right]}\\
\|\fX\|_{H^l}\lesssim&\,\change{\|\fk\|_{C^{\lfloor\frac{l-2}2\rfloor}}\|\fk\|_{H^{l-2}}+\|\fg-\gamma\|_{H^{l-1}}^2+\|\nabla\phi\|_{C^1}\|\fg-\gamma\|_{H^{l-3}}}\\
&\,\change{+\left[\|\nabla\phi\|_{C^2}^2\left(1+\|\fg-\gamma\|_{C^2}\right)+\|\phi^\prime\|_{C^2}\right]\left[1+\|\fN\|_{C^2}\right]\left[\|\phi^\prime\|_{H^{l-2}}+\|\nabla\phi\|_{H^{l-2}}\right]}
\end{align*}
\change{The statement then follows by inserting \eqref{eq:fut-bootstrap}.}
%Applying \eqref{eq:fut-eta}, the lapse equation \eqref{eq:fut-lapse-eq} reads
%\[\left(\fLap-\frac13\right)\fn=\fn\left(\lvert\fk\rvert_{\fg}^2+8\pi\lvert\phi^\prime\rvert^2\right)-1\,.\]
%The pointwise bounds for $\fn$ now follow as in Lemma \ref{lem:lapse-maxmin} from the maximum principle.\\
%From \cite[Proposition 17]{AndFaj20}, we take the following estimates:
%\begin{align*}
%\|\fN\|_{H^l}\lesssim&\,\|\fk\|_{H^{l-2}}^2+\lvert\tau\rvert\|\rho\|_{H^{l-2}}+\tau^3\|\underline{\eta}\|_{H^{l-2}}\\
%\|\fX\|_{H^l}\lesssim&\,\|\fk\|_{H^{l-2}}^2+\|\fg-\gamma\|_{H^{l-1}}^2++\lvert\tau\rvert\|\rho\|_{H^{l-2}}+\tau^3\|\underline{\eta}\|_{H^{l-2}}+\|\fn\jmath\|_{H^{l-2}}
%\end{align*}
%The lapse estimate then follows directly by inserting the expressions from Lemma \ref{lem:fut-matter-resc}, and the shift estimate from inserting these as well as \eqref{eq:ell-est-lapse} and $\fn\in(0,3)$.
\end{proof}

%\begin{corollary}[Improved bounds for lapse and shift]
%\begin{equation*}
%\|\fN\|_{C^4}+\|\fX\|_{C^4}+\|\fN\|_{H^6}+\|\fX\|_{H^6}\lesssim \epsilonnew^2
%\end{equation*}
%\end{corollary}
%\begin{proof}
%\todo{Insert bootstrap assumptions}
%\end{proof}



\subsection{Scalar field energy estimates}\label{subsec:fut-ESF}

\subsubsection{Near-coercivity of $E_{SF}$}

We will be able to prove a decay estimate via a Gronwall argument only for the corrected energy $E_{SF}$ . Hence, we first need to verify that this energy controls the solution norms, for which we first show that it controls the \enquote{canonical} scalar field energies:

\begin{lemma}[Positivity of corrected scalar field energies]\label{lem:fut-ESF-coercivity}
Let
\[Q=\frac{\sqrt{1+9q}-1}{\sqrt{1+9q}}\ \text{with}\ q=\frac12\left(\mu_0(\gamma)-\frac19\right)\,.\]
Then, for any $l\in\{0,1,2,3,4\}$ and $\epsilonnew>0$ small enough, one has
\begin{equation}\label{eq:fut-ESF-coercivity}
Q\fE^{(l)}\leq \fE^{(l)}+\frac23\fC^{(l)},\quad \text{hence}\quad Q\sum_{m=0}^4\fE^{(l)}\leq E_{SF}
\end{equation}
\end{lemma}
\begin{proof}
We denote the smallest positive eigenvalue of \changefinal{$-\fLap$ }acting on scalar functions on $\Sigma_T$ by $\mu_0(\fg_T)$. By the bootstrap assumption \eqref{eq:fut-bootstrap} and since $\mu_0$ depends continuously on the metric, we obtain the following for small enough $\epsilonnew>0$:
\begin{equation*}
\mu_0(\fg_T)\geq \mu_0(\gamma)-\frac12\left(\mu_0(\gamma)-\frac19\right)\geq \frac19+q
\end{equation*}
By the Poincaré inequality applied on $(\Sigma_T,\fg_T)$ (see \cite[p.1037]{CBM01}), the above spectral bound implies the following for any $\zeta\in H^1(\Sigma_T)$:
\begin{equation}\label{eq:adapted-poincare}
\|\zeta-\overline{\zeta}\|_{L^2_{\fg}(\Sigma_T)}^2\leq \mu_0(\fg_T)^{-1}\|\nabla\zeta\|_{L^2_{\fg}(\Sigma_T)}^2\leq \left(\frac19+q\right)^{-1}\|\nabla\zeta\|_{L^2_{\fg}(\Sigma_T)}^2
\end{equation}
For $l=0$, this means
\begin{align*}
\fE^{(0)}+\frac23\fC^{(0)}\geq&\,\|\phi^\prime\|_{L^2_{\fg}}^2+\|\nabla\phi\|_{L^2_{\fg}}^2-\frac23\|\phi-\phim\|_{L^2_{\fg}}\|\phi^\prime\|_{L^2_{\fg}}\\
\geq&\,\|\phi^\prime\|_{L^2_{\fg}}^2+\|\nabla\phi\|_{L^2_{\fg}}^2-2\left(1+9q\right)^{-\frac12}\|\nabla\phi\|_{L^2_{\fg}}\|\phi^\prime\|_{L^2_{\fg}}\\
\geq&\, \frac{\sqrt{1+9q}-1}{\sqrt{1+9q}}\fE^{(0)}\,.
\end{align*}
For $l=1$, notice that we can rewrite $\fC^{(1)}$ as 
\[\fC^{(1)}=\int_{M}\langle\nabla\phi,\nabla\phi^\prime\rangle_{\fg}\,\vol{\fg}=\int_{M}\left\langle\nabla\phi,\nabla\left(\phi^\prime-\overline{\phi^\prime}\right)\right\rangle_{\fg}\,\vol{\fg}=-\int_{M}\left(\phi^\prime-\overline{\phi^\prime}\right)\Lap_{\fg}\phi\,\vol{\fg}\,.\]
Hence, applying \eqref{eq:adapted-poincare} to $\zeta=\phi^\prime$ yields
\[\fE^{(1)}+\frac23\fC^{(1)}\geq \fE^{(1)}-2\left(1+9q\right)^{-\frac12}\|\nabla\phi^\prime\|_{L^2_{\fg}}\|\Lap_{\fg}\phi\|_{L^2_{\fg}}\geq \frac{\sqrt{1+9q}-1}{\sqrt{1+9q}}\fE^{(1)}\,.\]
For $l=2,3,4$, notice $\overline{\fLap\phi}=\overline{\fLap\phi^\prime}=\overline{\fLap^2\phi}=0$ holds due to the divergence theorem, hence the argument proceeds as in $l=0,1$.
%\begin{align*}
%\fE^{(2)}+\frac23\fC^{(2)}\geq&\,\|\fLap\phi^\prime\|_{L^2_{\fg}}^2+\|\nabla\fLap\phi\|_{L^2_{\fg}}^2-\frac23\|\fLap\phi\|_{L^2_{\fg}}\|\fLap\phi^\prime\|_{L^2_{\fg}}\\
%\geq&\,\|\fLap\phi^\prime\|_{L^2_{\fg}}^2+\|\nabla\fLap\phi\|_{L^2_{\fg}}^2-2\left(1+9q\right)^{-\frac12}\|\nabla\fLap\phi\|_{L^2_{\fg}}\|\fLap\phi^\prime\|_{L^2_{\fg}}\\
%\geq&\,\frac{\sqrt{1+9q}-1}{\sqrt{1+9q}}\fE^{(2)}
%\end{align*}
%For $l=3$, we have $\overline{\fLap\phi^\prime}=0$ and hence \eqref{eq:fut-ESF-coercivity} follows as for $l=1$ by applying \eqref{eq:adapted-poincare} to $\zeta=\fLap\phi^\prime$, and for $l=4$, it follows identically to $l=2$ with $\overline{\fLap^2\phi}=0$. 
\end{proof}

\begin{lemma}[Near-coercivity of corrected scalar field energy]\label{lem:fut-Sob-est}
For any differentiable function $\zeta$ and $k\in\{1,2\}$, one has the following under the bootstrap assumptions:
\begin{align*}
\int_M\lvert\nabla^2\zeta\rvert_{\fg}^2\,\vol{\fg}\lesssim&\,\int_M\lvert\fLap\zeta\rvert_{\fg}^2+\lvert\nabla\zeta\rvert_{\fg}^2\,\vol{\fg}\\
\|\zeta\|_{\dot{H}^{2k}}^2\lesssim&\,\|\fLap^{k}\zeta\|_{L^2}^2+\left(\|\zeta\|_{\dot{H}^{2k-1}}^2+\|\zeta\|_{\dot{H}^{2k-2}}^2\right)+\|\nabla\zeta\|_{C^{1}}^2\left\|\Ric[\fg]+\frac29\fg\right\|_{H^{2k-2}}^2\\
\|\nabla\zeta\|_{\dot{H}^{2k}}^2\lesssim&\,\|\nabla\fLap^{k}\zeta\|_{L^2}^2+\left(\|\nabla\zeta\|_{\dot{H}^{2k-1}}^2+\|\nabla\zeta\|_{\dot{H}^{2k-2}}^2\right)+\|\nabla\zeta\|_{C^{2}}^2\left\|\Ric[\fg]+\frac29\fg\right\|_{H^{2k-2}}^2
\end{align*}
Consequently, the following estimate holds:
\begin{align*}
\numberthis\label{eq:fut-coerc}\|\phi^\prime\|_{H^4}^2+\|\nabla\phi\|_{H^4}^2%\lesssim&\, \sum_{l=0}^4\fE^{(l)}+\left(\|\phi^\prime\|_{C^2}^2+\|\nabla\phi\|_{C^2}^2\right)\cdot\left\|\Ric[\fg]+\frac29\fg\right\|_{H^2}^2\\
\lesssim&\,E_{SF}^{(4)}+\left(\|\phi^\prime\|_{C^2}^2+\|\nabla\phi\|_{C^2}^2\right)\left\|\Ric[\fg]+\frac29\fg\right\|_{H^2}^2
\end{align*}
\end{lemma}
\begin{proof}
The inequalities for $\zeta$ follows from the same arguments as Lemma \ref{lem:Sobolev-norm-equivalence-improved}, except that we have $\|\Ric[\fg]\|_{C^1_{\fg}}\lesssim 1+\epsilonnew\lesssim 1$ by Lemma \ref{lem:fut-Ric-est}.  The final estimate then follows by applying these estimates to $\zeta=\phi^\prime$ and $\zeta=\phi$ and applying Lemma \ref{lem:fut-ESF-coercivity}.
\end{proof}


\subsubsection{Preparations for energy estimates}

Before proving the energy estimate, we need to establish two technical lemmas: First, we collect a formula to differentiate integrals, and then some estimates needed to deal with the mean value of $\phi$ in the base level correction term.
\begin{lemma}[Differentiation of integrals, future stability version] For any diffentiable function $\zeta$, one has
\begin{equation}\label{eq:fut-delt-int}
\del_T\int_M\zeta\vol{\fg}=\int_M\left(\fdel\zeta+3\fN\zeta\right)\,\vol{\fg}\,.
\end{equation}
\end{lemma}
\begin{proof}
\change{As in the proof of \eqref{eq:delt-int}, we obtain
\begin{align*}
\del_T\int_M\zeta\vol{\fg}=&\,\int_M\del_T\zeta+\frac{\del_T\mu_{\fg}}{\mu_{\fg}}\zeta\,\vol{\fg}\\
=&\,\int_M\del_T\zeta+3\fN\zeta-\frac12(\fg^{-1})^{ab}\Lie_{\fX}\fg_{ab}\zeta\,\vol{\fg}\\
=&\,\int_M\del_T\zeta+3\fN\zeta-\div_{\fg}\fX\cdot\zeta\,\vol{\fg}
\end{align*}
The statement now follows by applying Stokes' theorem to the final term and rearranging.}
\end{proof}

\begin{lemma}[Decay estimate for the integrated time derivative]\label{lem:fut-SF-technicality} For any $T>0$, we have
\begin{equation}\label{eq:fut-SF-technicality-1}
\int_{\Sigma_T}\phi^\prime\,\vol{\fg}=\left(\int_{\Sigma_{T=0}}\phi^\prime\,\vol{\fg}\right)\cdot e^{-2T}\,.
\end{equation}
Consequently, the bootstrap assumptions imply
\begin{equation}\label{eq:fut-SF-technicality-2}
\left\lvert\int_{\Sigma_T}\fdel\phim\cdot\phi^\prime\,\vol{\fg}\right\rvert\lesssim\,\epsilonnew^3e^{-\change{\frac52}T}
\end{equation}
for $\epsilonnew>0$ small enough.
\end{lemma}
\begin{proof}
Using that the integral of $\div_{\fg}(\fn\nabla\phi)$ vanishes, we compute:
\changediss{
\begin{align*}
\del_T\left(\int_M\phi^\prime\,\vol{\fg}\right)=&\,\int_{M}\left(\fdel\phi^\prime+3\fN\phi^\prime\right)\,\vol{\fg}=\int_M\left[(1-\fn)\phi^\prime+(\fn-3)\phi^\prime\right]\,\vol{\fg}\\
=&\,-2\left(\int_M\phi^\prime\,\vol{\fg}\right)
\end{align*}}
%\[\del_T\left(\int_M\phi^\prime\,\vol{\fg}\right)=\int_{M}\left(\fdel\phi^\prime+3\fN\phi^\prime\right)\,\vol{\fg}%=\int_M\left[(1-\fn)\phi^\prime+(\fn-3)\phi^\prime\right]\,\vol{\fg}=-2\left(\int_M\phi^\prime\,\vol{\fg}\right)\]
Hence, \eqref{eq:fut-SF-technicality-1} precisely describes the solution to this ODE ($f^\prime=-2f$) with prescribed initial value at $T=0$, and the initial data assumption \eqref{eq:fut-init} implies
\[\left\lvert\int_M\phi^\prime\,\vol{\fg}\right\rvert\leq \|\phi^\prime\|_{C^0(\Sigma_{T=0})}\vol{\fg}(\Sigma_{T=0})e^{-2T}\lesssim \epsilonnew^2e^{-2T}\,.\]
Furthermore, one has by \eqref{eq:fut-delt-int} \change{that }%and the bootstrap assumption that
\begin{equation}\label{eq:fut-volume-evol}
\change{\del_T\vol{\fg}\left(\Sigma_{T}\right)=\int_{\Sigma_T}3\fN\vol{\fg}}%\deletemath{\in\change{\left[-3\epsilonnew e^{-\frac{T}2}\cdot \vol{\fg}\left(\Sigma_{T}\right),0\right)}}\,.
\end{equation}
\change{Consequently, one has
\[\fdel\phim=\left[-\frac{\del_T\vol{\fg}\left(\Sigma_{T}\right)}{\vol{\fg}\left(\Sigma_{T}\right)}\cdot\phim+\frac1{\vol{\fg}\left(\Sigma_{T}\right)}\int_M\left(\fdel\phi+3\fN\phi\right)\,\vol{\fg}\right]=\int_M\left(\fn\phi^\prime+3\fN(\phi-\phim)\right)\,\vol{\fg}\,.\]
By applying $\lvert\fn\rvert<3$, the adapted Poincare inequality \eqref{eq:adapted-poincare} and the bootstrap assumptions \eqref{eq:fut-bootstrap}, this implies
\[\left\lvert\fdel\phim\right\rvert\lesssim\|\phi^\prime\|_{L^2}+\|\nabla\phi\|_{L^2}\|\fN\|_{L^2_{\fg}}\lesssim\epsilonnew e^{-\frac{T}2}\,.\]
The bound \eqref{eq:fut-SF-technicality-2} now follows by combining this with \eqref{eq:fut-SF-technicality-1}.}
%with the upper bound by the pointwise estimate in Lemma \ref{lem:fut-ell-est}. Hence, $\vol{\fg}(\Sigma_T)$ is decreasing and one has
%\[\left\lvert\frac{\del_T\vol{\fg}\left(\Sigma_{T}\right)}{\vol{\fg}\left(\Sigma_{T}\right)}\right\rvert\lesssim \change{\epsilonnew}\,.\]
%Since, by the initial data assumption, the volume forms of $\gamma$ and $\fg_{T=0}$ differ by a small error, as in Remark \ref{eq:rem-vol-form}, \eqref{eq:fut-volume-evol} also implies
%\[\vol{\fg}\left(\Sigma_{T}\right)\gtrsim \change{1}\,.\]
%Altogether and using $\fdel\phi=\fn\phi^\prime=(3+3\fN)\phi^\prime$, we obtain:
%\begin{align*}
%\left\lvert\int_M\fdel\phim\cdot\phi^\prime\,\vol{\fg}\right\rvert=&\,\left\lvert\left[-\frac{\del_T\vol{\fg}\left(\Sigma_{T}\right)}{\vol{\fg}\left(\Sigma_{T}\right)}\cdot\phim+\frac1{\vol{\fg}\left(\Sigma_{T}\right)}\int_M\left(\fdel\phi+3\fN\phi\right)\,\vol{\fg}\right]\cdot\int_M\phi^\prime\,\vol{\fg}\right\rvert\\
%\lesssim&\,\left[\epsilonnew\phim+\change{\left\lvert\int_M \phi^\prime\,\vol{\fg}\right\rvert}+\frac1{\vol{\fg}(\Sigma_T)}\left\lvert\int_M\fN(\phi^\prime+\phi)\,\vol{\fg}\right\rvert\right]\cdot\left\lvert\int_M\phi^\prime\,\vol{\fg}\right\rvert\\
%\lesssim&\,\left[\epsilonnew\left\lvert\phim\right\rvert+\change{\epsilonnew^2e^{-2T}}+\|\fN\|_{L^\infty}\left(\left\lvert\overline{\phi^\prime}\right\rvert+\left\lvert\phim\right\rvert\right)\right]\cdot \epsilonnew^2e^{-2T}
%\end{align*}
%The bootstrap assumptions \eqref{eq:fut-bootstrap}-\eqref{eq:fut-bootstrap-phi-mean} imply
%\[\left\lvert\phim\right\rvert+\left\lvert\overline{\phi^\prime}\right\rvert\leq \|\phi\|_{C^0(\Sigma_T)} + \|\phi^\prime\|_{C^0(\Sigma_T)}\leq (1+\|\phi\|_{C^0(\Sigma_{T=0})})+ \epsilonnew e^{-\frac{T}2}\lesssim 1\,,\]
%and this yields the statement \change{along }with \eqref{eq:fut-bootstrap}.
\end{proof}

\subsubsection{Energy estimates}

Now, we can collect the following estimates for the corrected scalar field energies:

\begin{lemma}[Base level estimate for the corrected scalar field energy]\label{lem:fut-en-est-ESF0} Under the bootstrap assumptions, the following estimate holds \change{for some $K>0$}:
\begin{equation}
\del_T E^{(0)}_{SF}\leq-2E^{(0)}_{SF}+K\delta e^{-\frac{T}2}\sqrt{E^{(0)}_{SF}}\left(\sqrt{E^{(0)}_{SF}}+\|\bm{\Sigma}\|_{L^2}+\|\fg-\gamma\|_{L^2}\right)+K\delta^3e^{-\frac{5T}2}
\end{equation}
\end{lemma}
\begin{proof}
We compute, using $[\fdel,\nabla]\phi=0$, $\fdel\phi=\fn\phi^\prime$ and the rescaled wave equation \eqref{eq:fut-wave}:
\begin{align*}
\del_T\fE^{(0)}=&\,\int_M \left[2\fdel\phi^\prime\cdot\phi^\prime+2\left\langle\nabla\phi,\nabla\fdel\phi\right\rangle_{\fg}+\left(\fdel\fg^{-1}\right)^{ab}\nabla_a\phi\nabla_b\phi+3\fN\left(\lvert\phi^\prime\rvert^2+\lvert\nabla\phi\rvert_{\fg}^2\right)\right]\,\vol{\fg}\\
=&\,\int_M \biggr[2\left(\langle\nabla\fn,\nabla\phi\rangle_{\fg}+\fn\fLap\phi+(1-\fn)\phi^\prime\right)\phi^\prime-2(\fn\phi^\prime)\cdot\fLap\phi\\
&\,\,\phantom{\int_M}-2\fn\langle\fk,\nabla\phi\nabla\phi\rangle_{\fg}+3\fN\lvert\phi^\prime\rvert^2+\fN\lvert\nabla\phi\rvert_{\fg}^2\biggr]\,\vol{\fg}
\end{align*}
With $2(1-\fn)=-4-6\fN$, integration by parts and using the bootstrap assumption \eqref{eq:fut-bootstrap} on $C$-norms, we get for some constant $K>0$ that we update from line to line:
\begin{align*}
\del_T\fE^{(0)}\leq&\,\int_M -4\lvert\phi^\prime\rvert_{\fg}^2\,\vol{\fg}+K\left[\|\nabla\phi\|_{C^0}\|\fN\|_{H^1}\sqrt{\fE^{(0)}}+\left(\|\fk\|_{C^0}+\|\fN\|_{C^0}\right)\fE^{(0)}\right]\\
\leq&\int_M-4\lvert\phi^\prime\rvert^2\,\vol{\fg}+K\epsilonnew e^{-\frac{T}2}\left(\change{\sqrt{\fE^{(0)}}}{\|\fN\|_{H^1}}+\fE^{(0)}\right)
\end{align*}
Similarly and using the same evolution equations, we obtain:
\begin{align*}
\del_T\fC^{(0)}=&\,\int_M \left[\fdel\phi\cdot\phi^\prime-\fdel\phim\cdot\phi^\prime+(\phi-\phim)\fdel\phi^\prime+3\fN(\phi-\phim)\phi^\prime\right]\,\vol{\fg}\\
=&\,\int_M\left[3\lvert\phi^\prime\rvert^2+3\fN\lvert\phi^\prime\rvert^2+\left(\phi-\phim\right)\cdot \div_{\fg}\left(\fn\nabla\phi\right)-2\left(\phi-\phim\right)\phi^\prime-\fdel\phim\cdot\phi^\prime\right]\,\vol{\fg}\\
\leq&\,-2\fC^{(0)}+\int_M3\left[\lvert\phi^\prime\rvert^2-\lvert\nabla\phi\rvert_{\fg}^2\right]\,\vol{\fg}+3\|\fN\|_{C^0}\fE^{(0)}-\int_M\left(\fdel\phim\cdot\phi^\prime\right)\,\vol{\fg}
\end{align*}
Applying Lemma \ref{lem:fut-SF-technicality} to the last term, we get:
\begin{align*}
\del_T\fC^{(0)}\leq&-2\fC^{(0)}+K\epsilonnew e^{-\frac{T}2}\fE^{(0)}+\change{K\epsilonnew^3e^{-\frac52T}}
\end{align*}
Combining these two estimates, inserting \eqref{eq:ell-est-lapse} and \eqref{eq:fut-ESF-coercivity} as well as updating $K$ yields:
\begin{align*}
\del_TE_{SF}^{(0)}=&\,\del_T\fE^{(0)}+\frac23\del_T\fC^{(0)}\\
=&\,\int_M \left[\left(-4+\frac23\cdot 3\right)\lvert\phi^\prime\rvert^2-\frac23\cdot 3\lvert\nabla\phi\rvert_{\fg}^2\right]\,\vol{\fg}-2\cdot\frac23\change{\fC^{(0)}}\\
&\,+\change{K\delta e^{-\frac{T}2}\change{\left(\sqrt{\fE^{(0)}}{\|\hat{\bm{n}}\|_{H^1}}+\fE^{(0)}\right)}+K\delta^3 e^{-\frac52T}}\\
\leq&\,-2E_{SF}^{(0)}+K\delta \change{e^{-\frac{T}2}}\sqrt{E^{(0)}_{SF}}\left(\|\Sigma\|_{L^2}+\|\fg-\gamma\|_{L^2}+\sqrt{E^{(0)}_{SF}}\right)+K\delta^3 e^{-\frac52T}
\end{align*}
\end{proof}

\begin{lemma}[Higher order estimates for the corrected scalar field energy]\label{lem:fut-en-est-ESF}
For any $l\in\{1,\dots,4\}$, the following estimate holds:
\begin{align*}
\del_T\left(\fE^{(l)}+\frac23\fC^{(l)}\right)\leq&\,-2\left(\fE^{(l)}+\frac23\fC^{(l)}\right)+K\epsilonnew e^{-\frac{T}2}\left(\sum_{m=0}^l\sqrt{\fE^{(m)}}\right)\cdot\\
&\,\qquad\cdot\change{\left(\|\phi^\prime\|_{H^{l}}+\|\nabla\phi\|_{H^{l}}+\|\Sigma\|_{H^{l}}+\|\fg-\gamma\|_{H^l}%+\epsilonnew e^{-\frac{T}2}\left\|\Ric[\fg]+\frac29\fg\right\|_{H^{l-1}}
\right)}
\end{align*}
\end{lemma}
\begin{proof}
Starting with $l=2k,\,k\in\{1,2\}$, one calculates:
\begin{subequations}
\begin{align}
\del_T\fE^{(2k)}=&\,\int_M\biggr[2\fLap^k\fdel\phi^\prime\cdot\fLap^k\phi^\prime+2\langle\nabla\fLap^{k}\phi,\nabla\fLap^k\fdel\phi\rangle_{\fg}\label{eq:fESF-1}\\
&\,\phantom{\int_M}+(\fdel {\fg}^{-1})^{ab}\cdot\nabla_a\fLap^k\phi\cdot\nabla_b\fLap^k\phi+3\fN\left(\lvert\Lap^k\phi^\prime\rvert_{\fg}^2+\lvert\nabla\Lap^k\phi\rvert_{\fg}^2\right)\label{eq:fESF-2}\\
&\,\phantom{\int_M}+2[\fdel,\fLap^k]\phi^\prime\cdot\fLap^k\phi^\prime+2\left\langle[\fdel,\nabla\fLap^k]\phi,\nabla\fLap^k\phi\right\rangle_{\fg}\biggr]\,\vol{\fg}\label{eq:fESF-3}
\end{align}
We insert the rescaled wave equation \eqref{eq:fut-wave} and $\fdel\phi=\fn\phi^\prime$ into the right hand side of \eqref{eq:fESF-1} and obtain for some constant $K>0$:
\begin{align*}
\eqref{eq:fESF-1}\leq &\,\int_M \biggr[-4\lvert\fLap^k\phi^\prime\rvert^2 -6\fN\lvert\fLap^k\phi^\prime\rvert^2+\fn\fLap^{k+1}\phi\cdot\fLap^k\phi^\prime\biggr]\,\vol{\fg}\\
&\,+K\|\Lap^k\phi^\prime\|_{L^2}\left(\|\fN\|_{H^{2k+1}}\|\nabla\phi\|_{C^0}+\|\nabla\phi\|_{H^{2k}}\|\fN\|_{C^{2k}}\right)\\
&\,+\int_M\left[-\fn\fLap^{k}\phi^\prime\cdot\fLap^{k+1}\nabla\phi-3\langle\nabla\fN,\nabla\fLap^k\phi\rangle_ {\fg}\cdot\fLap^k\phi^\prime\right]\,\vol{\fg}\\
&\,+K\|\nabla\fLap^k\phi\|_{L^2}\left(\changefinal{\|\fN\|_{H^{2k+1}}\|\phi^\prime\|_{C^0}+\|\fN\|_{C^{2k}}\|\phi^\prime\|_{H^{2k}}}\right)\\
\leq&\,\int_M -4\lvert\fLap^k\phi^\prime\rvert^2\,\vol{\fg}\\
&\,+K\sqrt{\fE^{(2k)}}\cdot\Bigr[\left(\|\nabla\phi\|_{C^0}+\|\phi^\prime\|_{C^0}\right)\cdot\|\fN\|_{H^{2k+1}}+\left(\|\nabla\phi\|_{H^{2k}}+\|\phi^\prime\|_{H^{2k}}\right)\cdot\|\fN\|_{C^{2k}}\Bigr]
\end{align*}
\end{subequations}
For \eqref{eq:fESF-2}, we use \eqref{eq:fut-eq-g-1} and the bootstrap assumption \eqref{eq:fut-bootstrap} to bound it by $K\epsilonnew e^{-\frac{T}2}\fE^{(2k)}$. Regarding \eqref{eq:fESF-3}, the commutator formulas \eqref{eq:[fdel,Lapk]}-\eqref{eq:[fdel,nablaLapk]} imply%\footnote{Note that, for $k=1$, the Sobolev norms in $\Ric[\fg]+\frac29\fg$ are entirely redundant, while for $k=2$, higher order terms in the curvature carry at least one derivative, and $\nabla\Ric[\fg]=\nabla\left(\Ric[\fg]+\frac29\fg\right)$ holds. Hence, we can provide these estimates in norms with regards to $\fg$, and then norm equivalence between $H_\gamma$ and $H_{\fg}$ allows us to drop that specification of metric.}
\begin{align*}
\|[\fdel,\fLap^k]\phi^\prime\|_{L^2}\lesssim&\,\change{\|\fn\|_{C^{2k-1}}\left(\|\phi^\prime\|_{C^1}\|\fk\|_{\dot{H}^{2k-1}}+\|\fk\|_{C^{2k-2}}\|\phi^\prime\|_{H^{2k}}\right)}\changefinal{+\|\fN\|_{C^{2k-1}}\|\phi^\prime\|_{H^{2k}}}\\
%&\,+\|\fX\|_{C^{2k-1}}\left(\left\|\Ric[\fg]+\frac29\fg\right\|_{H^{2k-2}}\|\phi^\prime\|_{C^{2k-2}}+\|\Ric[\fg]\|_{C^1}\|\phi^\prime\|_{H^{2k-1}}\right)\,,\\
\|[\fdel,\nabla\fLap^k]\phi\|_{L^2}\lesssim&\,\change{\|\fn\|_{C^{2k}}\left(\|\nabla\phi\|_{C^1}\|\fk\|_{{H}^{2k}}+\|\fk\|_{C^{2k-2}}\|\nabla\phi\|_{H^{2k}}\right)}\changefinal{+\|\fN\|_{C^{2k}}\|\nabla\phi\|_{H^{2k}}}
%&\,+\|\fX\|_{C^{2k}}\left(\|\nabla\phi\|_{C^1}\left\|\Ric[\fg]+\frac29\fg\right\|_{H^{2k-1}}+\|\Ric[\fg]\|_{C^{1}}\|\phi^\prime\|_{H^{2k}}\right)\,.
\end{align*}
Summarizing, inserting the $C$-norm bounds from the bootstrap assumption \eqref{eq:fut-bootstrap} and updating $K$, this implies
\begin{align*}
\del_T\fE^{(2k)}\leq&\,\int_M-4\lvert\fLap^k\phi^\prime\rvert^2\,\vol{\fg}+K\epsilonnew e^{-\frac{T}2}\fE^{(2k)}\\
&\,+K\epsilonnew e^{-\frac{T}2}\sqrt{\fE^{(2k)}}\left(\|\phi^\prime\|_{H^{2k}}+\|\nabla\phi\|_{H^{2k}}\right)\\
&\,+K\epsilonnew e^{-\frac{T}2}\sqrt{\fE^{(2k)}}\left(\change{\|\fN\|_{H^{2k+1}}+\|\Sigma\|_{H^{2k}}}%+\epsilonnew e^{-\frac{T}2}\left\|\Ric[\fg]+\frac29\fg\right\|_{H^{2k-1}}
\right)
\end{align*}
Moving on to the corrective term, we compute:
\begin{subequations}
\begin{align*}
\del_T\fC^{(2k)}%=&\,\del_T\int_M\fLap^k\phi\cdot\fLap^k\phi^\prime\,\vol{\fg}\\
=&\,\int_M\biggr[\fLap^k\fdel\phi\cdot\fLap^k\phi^\prime+\fLap^k\phi\cdot\fLap^k\fdel\phi^\prime\numberthis\label{eq:fESFcorr-1}\\
&\,\phantom{\int_M}+3\fN\cdot\fLap^k\phi\cdot\fLap^k\phi^\prime+[\fdel,\fLap^k]\phi\cdot\fLap\phi^\prime+\fLap^k\phi\cdot[\fdel,\fLap^k]\phi^\prime\biggr]\,\vol{\fg}\numberthis\label{eq:fESFcorr-2}
\end{align*}
\end{subequations}
Inserting the evolution equations into the right hand side of \eqref{eq:fESFcorr-1}, we can bound that line by
\begin{align*}
\leq&\,\int_M\left[3\lvert\fLap^k\phi^\prime\rvert^2+3\fN\lvert\fLap^k\phi^\prime\rvert^2\right]\vol{\fg}+K\|\fN\|_{C^{2k}}\|\phi^\prime\|_{H^{2k-1}}\|\Lap^k\phi^\prime\|_{L^2}\\
&+\int_M\left[ -2\fLap^k\phi\cdot\fLap^k\phi^\prime+3\fN\fLap^k\phi\cdot\fLap^k\phi^\prime +3\fLap^k\phi\cdot\fLap^{k+1}\phi+3\fN\fLap^k\phi\cdot\fLap^{k+1}\phi\right]\,\vol{\fg}\\
&\,+K\left[\|\fN\|_{C^{2k}}\left(\|\nabla\phi\|_{H^{2k}}+\|\phi^\prime\|_{H^{2k-1}}\right)+\left(\|\nabla\phi\|_{C^0}+\|\phi^\prime\|_{C^0}\right)\|\fN\|_{H^{2k+1}}\right]\|\fLap^k\phi\|_{L^2}
\end{align*}
Note that, after integrating by parts, the last two terms in the second line can be bounded by
\[\int_M-3\lvert\nabla\fLap\phi\rvert_{\fg}^2\,\vol{\fg}+\|\fN\|_{C^1}(\|\nabla\Lap^k\phi\|_{L^2}+\|\Lap^k\phi\|_{L^2})\|\nabla\Lap^k\phi\|_{L^2}\,.\]
For the terms in \eqref{eq:fESFcorr-2}, notice that the first term can be bounded by $ \epsilonnew e^{-\frac{T}2}\|\nabla\phi\|_{H^{2k-1}}\sqrt{\fE^{(2k)}}$, 
%\[\|\fN\|_{C^0}\|\Lap^k\phi\|_{L^2}\|\Lap^k\phi^\prime\|_{L^2}\lesssim \epsilonnew e^{-\frac{T}2}\|\nabla\phi\|_{H^{2k-1}}\sqrt{\fE^{(2k)}}\,,\]
while the commutator terms can be estimated as before, with
\begin{align*}
\|[\fdel,\fLap^k]\phi\|_{L^2}\lesssim&\,\change{\|\nabla\phi\|_{C^{0}}\|\fn\|_{C^0}\|\fk\|_{\dot{H}^{2k-1}}}%+\|\fX\|_{C^0}\left\|\Ric[\fg]+\frac29\fg\right\|_{H^{2k-2}}\right)\\
%&\,+\left(
\change{+\|\fn\|_{C^{2k}}\|\fk\|_{C^{2k-2}}}%+\|\fX\|_{C^{2k}}\|\Ric[\fg]\|_{C^1}\right)
\change{\|\nabla\phi\|_{H^{2k-1}}}
%\|[\fdel,\fLap^k]\phi^\prime\|_{L^2}\lesssim&\,\|\phi^\prime\|_{C^1}\left(\|\fn\|_{C^0}\|\fk\|_{\dot{H}^{2k-1}}+\|\fX\|_{C^0}\left\|\Ric[\fg]+\frac29\fg\right\|_{H^{2k-2}}\right)\\
%&\,+\left(\|\fn\|_{C^{2k}}\|\fk\|_{C^{2k-2}}+\|\fX\|_{C^{2k}}\|\Ric[\fg]\|_{C^1}\right)\|\phi^\prime\|_{H^{2k}}
\end{align*}

\noindent Combining all of the above, we get
\begin{align*}
\del_T\fC^{(2k)}\leq&\,-2\fC^{(2k)}+\int_M \left[3\lvert \fLap^k\phi^\prime\rvert-3\lvert\nabla\fLap^k\phi\rvert_{\fg}\right]\,\vol{\fg}%+K\epsilonnew e^{-\frac{T}2} \fE^{(k)}
\\
&\,+K\epsilonnew e^{-\frac{T}2}\change{\left[\|\phi^\prime\|_{H^{2k}}+\|\nabla\phi\|_{H^{2k}}+\|\fN\|_{H^{2k+1}}+\|\fk\|_{H^{2k}}%+\epsilonnew e^{-\frac{T}2}\left\|\Ric[\fg]+\frac29\fg\right\|_{H^{2k-2}}
\right]}\\
&\,\phantom{+K}\cdot\left(\sqrt{\fE^{(2k)}}+\sqrt{\fE^{(2k-1)}}\right)
\end{align*}

\noindent Finally, combining both differential estimates %and then inserting the bootstrap assumption \eqref{eq:fut-bootstrap} for all occuring $C$-norms 
yields the statement for $l=2k$.
%\begin{align*}
%\del_T\left(\fE^{(2k)}+\frac23\fC^{(2k)}\right)\leq&\,\int_M\left[\left(-4+\frac23\cdot 3\right)\lvert\fLap^l\phi^\prime\rvert^2-3\cdot\frac23\lvert\nabla\fLap^k\phi\rvert_{\fg}^2\right]\,\vol{\fg}-2\cdot\frac23\fC^{(2k)}\\
%&\,+K\epsilonnew e^{-\frac{T}2}\fE^{(2k)}+K\epsilonnew\left(\sum_{m=0}^{2k}\sqrt{\fE^{(m)}}\right)\cdot\\
%&\,\phantom{+K}\cdot\left(\|\phi^\prime\|_{H^{2k}}+\|\nabla\phi\|_{H^{2k}}+\|\Sigma\|_{H^{2k}}+\epsilonnew\left\|\Ric[\fg]+\frac29\right\|_{H^{2k-1}}\right)\\
%%\leq&\, -2\left(\fE^{(2l)}+\frac23\fC^{(2l)}\right)+K\epsilonnew^3
%\end{align*}
For $l=2k-1,\,k\in\{1,2\}$, the argument is completely analogous and hence omitted.
%so we only sketch how the correction term is handled:
%\begin{align*}
%\del_T\fC^{(2k-1)}=&\,\int_M \biggr[\left\langle\nabla\Lap^{k-1}\fdel\phi,\nabla\Lap^{k-1}\phi^\prime\right\rangle_{\fg}+\left\langle\nabla\Lap^{k-1}\phi,\nabla\Lap^{k-1}\fdel\phi^\prime\right\rangle_{\fg}\numberthis\label{eq:fESF-corr-4}\\
%&\phantom{\int_M}+\left(\fdel{\fg}^{-1}\right)^{ab}\cdot\nabla_a\Lap^{k-1}\phi\cdot\nabla_b\Lap^{k-1}\phi^\prime+3\fN\cdot\left\langle\nabla\Lap^{k-1}\phi,\nabla\Lap^{k-1}\phi^\prime\right\rangle_{\fg}\\
%&\,\phantom{\int_M}+\left\langle[\fdel,\nabla\Lap^{k-1}]\phi,\nabla\Lap^{k-1}\phi^\prime\right\rangle_{\fg}+\left\langle[\fdel,\nabla\Lap^{k-1}]\phi^\prime,\nabla\Lap^{k-1}\phi\right\rangle_{\fg}\biggr]\,\vol{\fg}
%\end{align*}
%The final line results at most in error terms analogous to those in the even order case, while the second line can be estimated by
%\[\lesssim\epsilonnew e^{-\frac{T}2}\|\nabla\Lap^{k-1}\phi\|_{L^2}\|\nabla\Lap^{k-1}\phi^\prime\|_{L^2}\lesssim \epsilonnew e^{-\frac{T}2}\|\nabla\phi\|_{H^{2k-2}}\sqrt{\fE^{(2k-1)}}\,,\]
%leaving only the terms on the right hand side \eqref{eq:fESF-corr-4}, which can be treated as follows:
%\begin{align*}
%\eqref{eq:fESF-corr-4}\leq&\,\int_M \left[3\lvert\nabla\Lap^{k-1}\phi^\prime\rvert_{\fg}^2+3\fN\lvert\nabla\Lap^{k-1}\phi^\prime\rvert_{\fg}^2\right]\,\vol{\fg}+K\|\fN\|_{C^{2k-1}}\|\phi^\prime\|_{H^{2k-2}}\|\nabla\Lap\phi^\prime\|_{L^2}\\
%&\,+\int_M\left[-2\langle\nabla\fLap^{k}\phi,\nabla\fLap^{k}\phi^\prime\rangle_{\fg}+3\fN\langle\nabla\fLap^{k}\phi,\nabla\fLap^{k}\phi^\prime\rangle_{\fg}-3\lvert\fLap^{k}\phi\rvert^2-3\fN\lvert\fLap^{k}\phi\rvert^2\right]\,\vol{\fg}\\
%&\,+K\epsilonnew e^{-\frac{T}2}\left[\|\fN\|_{C^{2k-1}}\left(\|\nabla\phi\|_{H^{2k-1}}+\|\phi^\prime\|_{H^{2k-2}}\right)+\left(\|\nabla\phi\|_{C^0}+\|\phi^\prime\|_{C^0}\right)\|\fN\|_{H^{2k}}\right]
%\end{align*}
%Otherwise, the proof proceeds as in the even order case.
\end{proof}

\subsection{Geometric variables}\label{subsec:fut-geom-var} We can take the following results from prior literature, where \change{we additionally apply the elliptic estimates in Lemma \ref{lem:fut-ell-est}}:

%We can obtain the necessary geometric energy estimate by inserting Lemma \ref{lem:fut-matter-resc} into more general results:

\begin{lemma}[Coercivity of geometric energies, {\cite[Lemma 7.4]{AM11}}]\label{lem:fut-en-geom-coercivity}
For sufficiently small $\epsilonnew>0$, the following estimate holds:
\begin{equation}\label{eq:fut-en-geom-coerc}
\|\fg-\gamma\|_{H^5}^2+\|\Sigma\|_{H^4}^2\lesssim \fEg
\end{equation}
\end{lemma}

\begin{lemma}[Geometric energy estimate, {\cite[Lemma 20]{AndFaj20}}]\label{lem:fut-geom-est}
Let $\epsilonnew>0$ be chosen appropriately small, and let
\begin{equation}\label{eq:fut-alpha}
\alpha=\begin{cases}
1 & \lambda_0>\frac19\\
1-3\sqrt{\epsilonnew^\prime} & \lambda_0=\frac19\,,
\end{cases}
\end{equation}
where $\epsilonnew^\prime>0$ is the same as in \eqref{eq:fut-corr-const}, in particular suitably small. Then, there exists some constant $K>0$ such that the following estimate holds:
\begin{align*}
\numberthis\label{eq:fut-geom-est}\del_T\fEg\leq&\,-2\alpha\fEg+K\fEg^\frac32+\change{K\epsilonnew e^{-\frac{T}2}}\sqrt{E_{geom}}\left[\|\phi^\prime\|_{H^4}+\|\nabla\phi\|_{H^4}\right]\\
%&\,+K\sqrt{\fEg}\left[1+\|\fN\|_{C^2}\right]\left[\|\phi^\prime\|_{C^2}+\|\nabla\phi\|_{C^2}(1+\|\fg-\gamma\|_{C^2})\right]\left[\|\phi^\prime\|_{H^4}+\|\nabla\phi\|_{H^4}\right]
\end{align*}
\end{lemma}
%\begin{proof}
%By \cite[Lemma 20]{AndFaj20}, one has:
%\begin{align*}
%\del_T\fEg\leq &\,-2\alpha\fEg + 6\sqrt{\fEg}\lvert\tau\rvert\|\fn S\|_{H^4}+K(\fEg)^{\frac32}\\
%&\,+K\sqrt{\fEg}\left(\lvert\tau\rvert\|\rho\|_{H^4}+\lvert\tau\rvert^3\|\underline{\eta}\|_{H^4}+\lvert\tau\rvert^2\|\fn\jmath\|_{H^3}\right)
%\end{align*}
%The statement now follows by inserting the expressions for the rescaled matter quantities in Lemma \ref{lem:fut-matter-resc} to \cite[Lemma 20]{AndFaj20}.
%\end{proof}


\subsection{Closing the bootstrap}\label{subsec:fut-bs-imp}

Now, we can collect our estimates to improve the bootstrap assumptions:

\begin{prop}[Improved bounds for future stability]\label{prop:fut-bs-imp} Let the bootstrap assumption (see Assumption \ref{ass:fut-bootstrap}) be satisfied for $T\in[0,T_{Boot})$ and assume the initial data assumption holds at $T=0$ (see Assumption \ref{ass:fut-init}). For $\epsilonnew>0$ sufficiently small and $\alpha$ as in \eqref{eq:fut-alpha} with $\epsilonnew^\prime>0$ sufficiently small, the following estimates hold: 
\begin{subequations}
\begin{align}
\|\phi^\prime\|_{C^2}+\|\nabla\phi\|_{C^2}+\|\phi^\prime\|_{H^4}+\|\nabla\phi\|_{H^4}\lesssim&\,\epsilonnew^\frac32 \change{e^{-\alpha T}}\label{eq:fut-sf-imp}\\
\|\fg-\gamma\|_{C^3}+\|\Sigma\|_{C^2}+\|\fg-\gamma\|_{H^5}+\|\Sigma\|_{H^4}\lesssim&\,\epsilonnew^\frac32e^{-\alpha T}\label{eq:fut-geom-imp}\\
\|\fN\|_{C^4}+\|\fX\|_{C^4}+\|\fN\|_{H^6}+\|\fX\|_{H^6}\lesssim&\,\epsilonnew^3e^{-2\alpha T}\label{eq:fut-ell-imp}
\end{align}
\delete{Further[...]}% there exists a constant $K>0$ such that the following estimate holds:
%\begin{equation}\label{eq:fut-phim-imp}
%\|\phi\|_{L^\infty}\lesssim\|\phi(T=0,\cdot)\|_{L^\infty(\Sigma_{T=0}}+K\epsilonnew^\frac32
%\end{equation}
\end{subequations}
\end{prop}
\begin{proof}\change{In the following, the positive constant $K$ may be updated from line to line.\\
Combining the estimate from Lemma \ref{lem:fut-en-est-ESF0} as well as those from Lemma \ref{lem:fut-en-est-ESF} at each level with Lemma \ref{lem:fut-geom-est} and applying the (near)-coercivity estimates \eqref{eq:fut-coerc} and \eqref{eq:fut-en-geom-coerc} to the right hand sides, we obtain:
\begin{align*}
\del_T\left(E^{(4)}_{SF}+E_{geom}\right)\leq&\, -2E_{SF}+K\epsilonnew e^{-\frac{T}2} \sqrt{E_{SF}}\left(\sqrt{E_{SF}+\epsilonnew^2e^{-T}\left\|\Ric[\fg]+\frac29\fg\right\|_{H^2}^2}+\sqrt{E_{geom}}\right)\\
&\,+K\delta^3 e^{-\frac52T}\\
&\,-2\alpha E_{geom}+KE^\frac32_{geom}+K\epsilonnew e^{-\frac{T}2}\sqrt{E_{geom}}\sqrt{E_{SF}+\epsilonnew^2e^{-T}\left\|\Ric[\fg]+\frac29\fg\right\|_{H^2}^2}
\end{align*}
Applying \eqref{eq:fut-Ric-est} to the curvature norms, as well as \eqref{eq:fut-en-geom-coerc} to the resulting norms on $\fg-\gamma$ and \eqref{eq:fut-bootstrap} (which implies $\sqrt{E_{geom}}\lesssim\epsilonnew e^{-\frac{T}2}$), this becomes
\[\del_T\left(E_{SF}+E_{geom}\right)\leq -2\alpha (E_{SF}^{(4)}+E_{geom})+K\epsilonnew e^{-\frac{T}2}\left(E_{SF}+E_{geom}\right)+K\epsilonnew^3e^{-\frac52T}\,.\]
and consequently, since $\alpha\leq1$,
\[\del_T\left[e^{2\alpha T}\left(E_{SF}+E_{geom}\right)\right]\lesssim \epsilonnew e^{-\frac{T}2}\cdot e^{2\alpha T}\left(E_{SF}+E_{geom}\right)+\epsilonnew^3 e^{-\frac{T}2}\]
The Gronwall lemma, along with the initial data assumption \eqref{eq:fut-init}, now implies
\begin{equation}\label{eq:fut-en-imp}
E_{SF}+E_{geom}\lesssim \epsilonnew^3e^{-2\alpha T}\,.
\end{equation}
}
Lemma \ref{lem:fut-en-geom-coercivity} and the standard Sobolev embedding then imply \eqref{eq:fut-geom-imp}. In particular, this means
\begin{equation}\label{eq:fut-curv-imp}
\left\|\Ric[\fg]+\frac29\fg\right\|_{H^2}\change{\lesssim} \epsilonnew^\frac32e^{-\alpha T}
\end{equation}
due to Lemma \ref{lem:fut-Ric-est}, and for $\epsilonnew^\prime>0$ small enough, inserting \eqref{eq:fut-en-imp} and \eqref{eq:fut-curv-imp} into \eqref{eq:fut-coerc} shows \eqref{eq:fut-sf-imp}. Moreover, \eqref{eq:fut-ell-imp} follows directly from \changefinal{the proof of }Lemma \ref{lem:fut-ell-est} and the already obtained improvements. \delete{Finally [..]}
% notice that $\fn\in(0,3)$ (see Lemma \ref{lem:fut-ell-est}) as well as the attained decay estimates for $\phi^\prime, \nabla\phi$ and $\fX$ yield the following:
%\[\lvert\del_T\phi\rvert\lesssim \lvert\fn\phi^\prime+\fX^c\nabla_c\phi\rvert\lesssim \epsilonnew^\frac32e^{-T}+\epsilonnew^\frac92e^{-(1+2\alpha)T}\]
%After integrating on $[0,T]$ and taking the supremum over $\Sigma_T$, we get \eqref{eq:fut-phim-imp}.
\end{proof}


\begin{proof}[Proof of Theorem \ref{thm:fut-stab-simple}]

The problem is locally well-posed as outlined in Remark \ref{rem:fut-lwp}. There then is some maximal interval $[0,T_{Boot})$ for the logarithmic time $T$ -- or, equivalently, some maximal time interval $[\tau_0,\tau_{Boot})$ -- on which the solution exists and the bootstrap assumptions (see Assumption \ref{ass:fut-bootstrap}) are satisfied. By the analogous argument to the proof of Theorem \ref{thm:main}, the decay estimates in Proposition \ref{prop:fut-bs-imp} are strictly stronger than the bootstrap assumptions for small enough $\epsilonnew,\epsilonnew^\prime>0$. This implies $T_{Boot}=\infty$ (resp. $\tau_{Boot}=0$) since we could else extend the solution strictly beyond $T_{Boot}$ while also satisfying the bootstrap assumptions. \delete{To be a bit more precise [...]}%, $(g,k,\nabla\phi,\fdel\phi)$ would actually remain in $H^6\times H^5\times H^5\times H^5$ beyond $T_{Boot}$ if it were finite since any breakdown in regularity would have to be reflected in lower regularity blow-up (see, for comparison, the vacuum analogues in \cite[Theorem 3.1]{AM03B} or the blow-up statements in Section \ref{subsec:lwp}). Hence, not only do we find a solution on $[T_0,\infty)\times M$, but all energies used above remain continuously differentiable which means the decay estimates hold in the entire future of $\Sigma_{\tau_0}$. 
This proves the convergence statement in Theorem \ref{thm:fut-stab-simple}.

Finally, the decay estimates imply that $\lvert\nabla n\rvert_{g}$, respectively $\lvert k\rvert_{g}$, are bounded by $\tau^{\alpha-1}$, respectively $\tau^{\alpha+1}$, up to constant on $[\tau_0,\tau)$. Since $\alpha$ is at worst slightly smaller than $1$, both functions are integrable on $[\tau_0,0)$ for suitably small $\epsilonnew^\prime>0$ . By \cite{CB02}, this means the spacetime is future complete.
\end{proof}

\section{Global stability}\label{sec:full-stab}

To prove Theorem \ref{thm:main-full}, what still needs to be shown is that initial data as in Theorem \ref{thm:main-past} develops from $\Sigma_{t_0}$ to some hypersurface $\Sigma_{t_1}\equiv \Sigma_{\tau(t_1)}$ in its future such that the data in $\Sigma_{t_1}$ is near-Milne in the sense of Assumption \ref{ass:fut-init} and in CMCSH gauge. From there, near-Milne stability yields the behaviour in the future of $\Sigma_{\tau(t_1)}$, and hence future stability of near-FLRW spacetimes as in Theorem \ref{thm:main-full}. %That combining the future and past developments of the initial data at $\Sigma_{t_0}$, despite moving between gauges, constitutes a global solution $(\M,\g,\phi)$ follows from uniqueness of the maximal globally hyperbolic development within the Einstein scalar-field system.

 %Since we need to develop the initial data used for Big Bang stability forward in time, we revert back to using the physical time coordinate $t$ with CMC gauge $\tau=-3\frac{\dot{a}}a$.\\
\begin{proof}[Proof of Theorem \ref{thm:main-full}] Within this proof, $t$ will denote the \enquote{physical} time coordinate used throughout the Big Bang stability analysis, while $\tau$ denotes the mean curvature time used within CMCSH gauge.

Consider initial data $(g,k,\nabla\phi,\del_0\phi)$ induced on the CMC hypersurface $\Sigma_{t_0}$ within $\M$ such that the rescaled variables are close to FLRW reference data in the sense of Theorem \ref{thm:main}. Moreover, let $(\mathring{g},\mathring{k},\mathring{\pi},\mathring{\psi})$ be the geometric initial data on $M$ that induce it via the embedding $\iota:M\hookrightarrow \M$.% This data is close to $(a(t_0)^2\gamma,-{\dot{a}(t_0)}a(t_0)\gamma,0,Ca(t_0)^{-3})$ in $H_\gamma^{18}\times H_\gamma^{18}\times H_\gamma^{18}\times H_\gamma^{19}$. \\

Notice that
\[P:H^{20}_\gamma(M)\rightarrow H^{18}_\gamma(M),\,Y^i\mapsto \Lap_\gamma Y^i+(\gamma^{-1})^{il}{\Ric[\gamma]}_{lj}Y^j=\Lap_{\gamma} Y^i-\frac29 Y^i\]
is an isomorphism since $\Lap_\gamma$ has no positive eigenvalues. Hence, using \cite[Theorem 2.5, Remark 2.6]{FajKr20}, there is a metric $\mathring{g}^\prime$ isometric to $\mathring{g}$ that remains \changefinal{close in $H^{18}_\gamma(M)$ }to $\gamma$ and satisfies
\[((\mathring{g}^\prime)^{-1})^{ij}\left(\Gamma[\mathring{g}^\prime]^k_{ij}-\Gamhat[\gamma]^{k}_{ij}\right)=0.\]
Let $\theta\in\text{Diff}(M)$ be the diffeomorphism such that $\theta^\ast \mathring{g}=\mathring{g}^\prime$, then the proof of \cite[Theorem 2.5]{FajKr20} implies that $\theta$ can be chosen close to the identity map within $H^{18}(\text{Diff}(M))$, and consequently that $\theta^\ast\mathring{k}=\mathring{k}^\prime,\, \theta^\ast\mathring{\pi}=\mathring{\pi}^\prime$ and $\theta^\ast\mathring{\psi}=\mathring{\psi}^\prime$ remain close to $-\dot{a}(t_0)a(t_0)\gamma,\,0$ and $Ca(t_0)^{-3}$ in $H^{18}_\gamma(M)$. By the same argument as in Remark \ref{rem:CMC-hypersurface}, we can now evolve this data locally and obtain a new initial hypersurface $\Sigma^\prime$ close to $\Sigma_{t_0}$ that is in CMCSH gauge and that $(g,k,\nabla\phi,\del_0\phi)$ is close to the reference data in the sense of Assumption \ref{ass:init}, exchanging the initial time $t_0$ by some close time $t_0^\prime$.\\

Since $\tau$ is strictly increasing, $t\equiv t(\tau)$ exists and we can interchangeably view $a$ as a function in $t$ or $\tau$ with some abuse of notation. The Friedman equation \eqref{eq:Friedman} implies $\del_ta\geq\frac19$ and thus $a(t)\geq\frac19t$ on $(0,\infty)$, as well as 
\[-\tau=3\frac{\dot{a}}a= \frac1a+\langle\text{lower order terms}\rangle\ \text{as}\ t\to \infty\ (\text{resp. }\tau\to 0)\,.\]

We choose $t_1>\max\{1,t_0^\prime\}$ large enough (resp. $\tau(t_1)\equiv\tau_0$ small enough) that the following estimates hold for some small $\chi\in(0,\frac12)$ that depends only on $\epsilonnew$:
\begin{align}
Ca(t_1)^{-3}\tau(t_1)^{-1}\leq&\, \chi \label{eq:connect-time-1}
\\
 -\tau(t_1)\cdot{a(t_1)}\in&\,[1-\chi,1+\chi] \label{eq:connect-time-3}
\end{align}

As the solution \change{is Cauchy stable, i.e.,~ it and }its maximal time of existence depend continuously upon the initial data,\footnote{For the argument for Einstein vacuum in CMCSH gauge, see \cite[Theorem 3.1]{AM03}. As with local existence, the argument in the Einstein scalar-field system is largely identical since the only difference amounts to coupling the hyperbolic parts of the system with a further hyperbolic one.} one can choose $\epsilon>0$ in the analogue of Assumption \ref{ass:init} small enough to ensure the following: The solution exists until $t_1>t_0^\prime$ and $(a^{-2}g,a\hat{k},\nabla\phi,a^{3}\fdel\phi)$ remain $K\epsilon$-close to $(\gamma,0,0,C)$ in $H_\gamma^6\times H_\gamma^5\times H_\gamma^5\times H_\gamma^5$ for some suitable $K>0$ along the slab $\cup_{s\in [t_0^\prime,t_1]}\Sigma_s$. What now remains to be shown is that this implies Assumption \ref{ass:fut-init} in the sense that, if $\epsilon$ is small enough, $\epsilonnew$ can be made as small as necessary for Theorem \ref{thm:fut-stab-simple} to apply.\\

Note that the scalings in Definition \ref{def:fut-rescaled} can be rewritten as
\begin{gather*}
\fg-\gamma=(\tau \cdot a)^2\cdot (a^{-2} g-\gamma)+(\tau^2\cdot a^2-1)\gamma,\quad \fk=\frac{\tau}a (a\hat{k})\,,\\
\phi^\prime=C\left(-\tau^{-1}\cdot a^{-3}\right)+\left(-\tau^{-1}\cdot a^{-3}\right)\cdot (a^3n^{-1}(\del_\tau-\Lie_X)\phi-C)\,.
\end{gather*}
Since \eqref{eq:connect-time-3} implies $\tau\cdot a$ is close to $-1$ at $t_1$, $\|(\tau\cdot a)^2(a^{-2}g-\gamma)\|_{H^6}$ can be bounded by $\frac{\epsilonnew^3}2$ for small enough $\epsilon$. Choosing $\chi<\frac{\epsilonnew^3}2$ then implies $\|\fg-\gamma\|_{H^6(\Sigma_{\tau_0})}<\epsilonnew^3$. That $\|\fk\|_{H^5}$ can be made smaller than $\epsilonnew^3$ for small enough $\epsilon>0$ follows since $\frac\tau{a}$ behaves like $\frac1{a^2}$ up to a constant by \eqref{eq:connect-time-3}.\\
For the normal derivative of the wave, notice that $\lvert C\left(-\tau^{-1}\cdot a^{-3}\right)\rvert $ is bounded by $\chi$ due to \eqref{eq:connect-time-1}, and that $-\tau^{-1}a^{-3}$ is equivalent to $a^{-2}$ by \eqref{eq:connect-time-3}. Hence, we can similarly ensure that $\phi^\prime$ is bounded in $H^5$ by $\epsilonnew^3$. Since $\nabla\phi$ is not changed in either rescaling, and bounds on lapse and shift (up to constant) follow from the elliptic estimates in Lemma \ref{lem:fut-ell-est}, it follows each individual norm in Assumption \ref{ass:fut-init} can be bounded by $\epsilonnew^3$ up to constants that depend only on $\gamma$, and hence the initial data assumption itself can be satisfied for suitably small $\epsilonnew>0$.\\
This proves that we can develop from initial data for the Big Bang stability proof to near-Milne initial data within a CMCSH foliation, and thus we obtain Theorem \ref{thm:main-full} from Theorems \ref{thm:main-past} and \ref{thm:fut-stab-simple}.
%Finally, the past development of $\Sigma_{t_0}$, the local development between $\Sigma_{t_0}$ and $\Sigma^\prime$, the development from $\Sigma^\prime$ and $\Sigma_{\tau_0}$ and the latter's future development all combine to a single maximal globally hyperbolic development of the initial data at $\Sigma_{t_0}$ within the Einstein scalar-field equations.
\end{proof}

%Further, we can choose $\epsilon>0$ in the initial data assumption for Big Bang stability (see Assumption \ref{ass:init}) such $(G,\Sigma,\nabla\phi,\Psi)$ induces a solution to the Einstein scalar-field system on the entirety of $[t_0,t_1]$ and such that
%\begin{align*}
%B_{K\epsilon}^{5,4,4,4}(t_0)&\,\longrightarrow \left[H_\gamma^5\times H_\gamma^4\times H_\gamma^4\times H_\gamma^4\right](\Sigma_{t_1})
%(G,\Sigma,\nabla\phi,\Psi)(t_0,\cdot)&\,\mapsto (G,\Sigma,\nabla\phi,\Psi)(t_1,\cdot)
%\end{align*}
%is continuous, where $B_{\epsilon^\prime}^{5,4,4,4}(t)$ denotes the $\epsilon^\prime$-ball around FLRW data $(\gamma,0,0,Ca(t)^{-3})$ at $\Sigma_{t}$ in $\left[H_\gamma^5\times H_\gamma^4\times H_\gamma^4\times H_\gamma^4\right](\Sigma_{t_0})$ \todo{[cite AM or examine RS closely]}. As a consequence, for any $\epsilon_f>0$, we can find some $\epsilon_p<K\epsilon$ such that $(G,\Sigma,\nabla\phi,\psi)(t_1,\cdot)\in B_{\epsilon_f}^{5,4,4,4}(t_1)$ holds if the initial data at $\Sigma_{t_0}$ is $\epsilon_p$-close to FLRW initial data.\\
%
%What now remains to be shown is that $(G,\Sigma,\nabla\phi,\Psi)(t_1,\cdot)\in B_{\epsilon_f}^{5,4,4,4}(t_1)$ that, after switching from CMC gauge with zero shift to CMCSH gauge, implies Assumption \ref{ass:fut-init} is $\epsilon_p$ is chosen small enough that $\epsilon_f$ is sufficiently small compared to $\epsilonnew^2$. Since lapse and shift can be sufficiently controlled by the other quantities (see Remark \todo{[ref]}), one only needs to prove that $(\fg,\fk,\nabla\phi,\phi^\prime)(\tau(t_1),\cdot)$ can be arbitrarily close to $(\gamma,0,0,0)$ at $t_1$ if $\epsilon_f$ is small enough. 


%This can be argued along similar lines as in the vacuum case in \todo{AM}, given that the coupling of the elliptic-hyperbolic system in the vacuum setting with the wave equation keeps the structure of the geometric evolution equations in tact and only adds an additional system of the same type. The elliptic system is in $(n,X)$ is of the type
%\[A\begin{pmatrix} n\\ X\end{pmatrix}=\begin{pmatrix}
%B & 0\\
%E & P
%\end{pmatrix}\begin{pmatrix} n\\ X\end{pmatrix}=\begin{pmatrix} 1 \\ 0\,,\]
%and what needs to be shown is that $A: H^2\longrightarrow L^2$ is invertible given $(\mathring{g},\mathring{k},\mathring{\pi},\mathring{\psi})$, to which is suffices to show that $B,P:H^2\longrightarrow L^2$ are isomorphisms. For the Einstein scalar-field system, these operators take the following form
%\begin{align*}
%Bf=&\,\left(-\Lap_{\mathring{g}}+\lvert \mathring{k}\rvert_\mathring{g}^2+8\pi\lvert\mathring{\psi}\rvert^2\right)f\\
%PY^i=&\,\Lap_{\mathring{g}}Y^i+{R[\mathring{g}]^i}_fY^f-\Lie_{Y}V^{i}-2\nabla{\mathring{g}}^mY^n\left(\Gamma[\mathring{g}]^{i}_{mn}-\Gamhat^i_{mn}\right)\,
%\end{align*}
%where $V[\mathring{g}]^i=(\mathring{g}^{-1})^{mn}\Gamma[\mathring{g}]^{i}_{mn}-\Gamhat^i_{mn})$. $B$ is immediately seen to be invertible since $\Lap_{\mathring{g}}$ has nonpositive spectrum and since $\mathring{\psi}$ is close to $C$, ensuring that $\lvert \mathring{k}\rvert_\mathring{g}^2+8\pi\lvert\mathring{\psi}\rvert^2$ is strictly positive.\\
%To show that $P$ is an isomorphism, first note that the $\lvert V\rvert_{\gamma}$ must be small given that $\mathring{g}$ is close to $\gamma$ in $C^1_\gamma$, that $V[\gamma]=0$ trivially holds and that $V$ is continuous in $\mathring{g}$. With this in hand, we argue as in \todo{Lemma 5.2, AM} to rewrite $P$ as
%\[PY^\alpha=\mathcal{L}_{\mathring{g},\gamma}Y^\alpha\]
%
%

%\section{Old version}
%
%Assume we have a local stability statement of the type that, when initial data near enough to the FLRW reference, we can develop to the future by some time increment that does not depend on the initial data size while at most multiplying the data size by a constant.
%
%Then, what needs to be ensured is that, when developing data from near the Big Bang to the far future, $\phi^\prime$ becomes small (Near the Big Bang, $\tau\approx a^{-3}$, hence $\phi^\prime\approx a^3\del_t\phi\approx C>0$).\footnote{We need to connect from the early universe forwards, since the future stability includes near-Milne spacetimes which we cannot connect backward to a spacetime that exhibits a Big Bang singularity}. That the smallness of $G,\Sigma$ and $\nabla\phi$ ensures the smallness of $\fg,\fk$ and $\nabla\phi$ for large enough times follows simply from the fact that $\tau$ asymptotically behaves like $-\frac1a$ for large times. 
%
%To this end, assume that local stability (in CMC gauge with zero shift) was able to ensure that $G-\gamma$, $\nabla\phi$ and $N$ remain small at sufficiently high regularity. Then one has, for some small $\chi>0$,
%\[\del_t(a^3\del_t\nabhat^{I}\phi)=a\nabhat\left[\langle\nabla N,\nabla\phi\rangle_G+(N+1)\Lap\phi\right]\simeq a\cdot \chi\]
%and hence
%\[\del_t\nabhat^I\phi\approx (C\pm \epsilonnew^2)a^{-3}\pm \chi \cdot a^{-3}\int_{t_0}^ta(s)\,ds\,.\]
%Since $a$ is strictly increasing and bounded from below by $\frac{t}3$, this implies
%\[\lvert\del_t\nabhat^I\phi\rvert\lesssim t^{-3}+\chi t^{-2}(t-t_0)\leq t^{-3}+\epsilonnew t^{-1}\]
%For large $t$, this implies
%\[\lvert\tau^{-1}\del_t\nabhat^I\phi\rvert\lesssim t^{-2}+\chi \,,\]
%which is $\lesssim\epsilonnew$ for large enough $t$, and hence $\lvert\nabhat^I\phi^\prime\rvert\lesssim\chi$.
%
%
