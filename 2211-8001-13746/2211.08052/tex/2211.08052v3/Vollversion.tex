\documentclass[11pt,leqno,textwidth=10cm]{amsart}
\usepackage{amssymb}
\allowdisplaybreaks
\setcounter{tocdepth}{1}

\usepackage{mathrsfs}
\usepackage{amsmath, amssymb, mathtools, nicefrac}
\usepackage{csquotes}
\usepackage{abstract}
\usepackage[hidelinks]{hyperref}
\usepackage{marginnote}
\usepackage{xcolor}
\usepackage{graphicx}%To scale align blocks in appendix


\usepackage{etoolbox}
\makeatletter
\patchcmd{\@maketitle}{\newpage}{}{}{} 
\makeatother
\numberwithin{equation}{section}

%
\setlength{\hoffset}{-.75in}
\setlength{\textwidth}{6.5in}
\setlength{\voffset}{-.5in}
\setlength{\textheight}{9.0in}
\setlength{\parindent}{2em}

%%%%%%%%%%%%%%%%%%%%%%%%%%%%%%%%%%%%%%%%%%%%%%%%%%%%%%%%%%%%%%%%%%%%%%%

%%%%%%%%%%%%%%%%%%%%%%%%%%%%%%%%%%%%%%%%%%%%%%%%%%%%%

\theoremstyle{definition}
\newtheorem{definition}{Definition}[section]
\newtheorem{example}[definition]{Example}
\newtheorem{remark}[definition]{Remark}
\theoremstyle{plain}
\newtheorem{theorem}[definition]{Theorem}
\newtheorem{lemma}[definition]{Lemma}
\newtheorem{corollary}[definition]{Corollary}
\newtheorem{prop}[definition]{Proposition}
\newtheorem{assumption}[definition]{Assumption}
\newtheorem{conjecture}[definition]{Conjecture}

%Liams Header

\newcommand{\1}{\mathbbm{1}}
\newcommand{\A}{\mathcal{A}}
\newcommand{\C}{\mathbb{C}}
\newcommand{\D}{\mathbb{D}}
\newcommand{\E}{\mathcal{E}}
\newcommand{\f}{\overline{f}}
\newcommand{\g}{\overline{g}}
\newcommand{\G}{\mathbb{G}}
\renewcommand{\H}{\mathcal{H}}
\newcommand{\I}{\mathbb{I}}
\newcommand{\J}{\mathcal{J}}
\newcommand{\K}{\mathbb{K}}
\renewcommand{\L}{\mathcal{L}}
\newcommand{\M}{\overline{M}}
\newcommand{\Mat}{\text{Mat}}
\newcommand{\N}{\mathbb{N}}
\newcommand{\Nn}{\mathcal{N}}
\newcommand{\nabbar}{\overline{\nabla}}
\renewcommand{\O}[1]{\mathcal{O}\left(#1\right)}
\renewcommand{\P}{\mathcal{P}}
\newcommand{\Q}{\mathcal{Q}}
\newcommand{\R}{\mathbb{R}}
\newcommand{\Ran}{\text{Ran}}
\newcommand{\Ric}{\text{\normalfont{Ric}}}
\newcommand{\Riem}{\text{\normalfont{Riem}}}
\newcommand{\supp}{\text{supp}}
\renewcommand{\S}{\mathbb{S}}
\newcommand{\T}{\mathfrak{T}}
\newcommand{\V}{\mathcal{V}}
\newcommand{\vol}[1]{{\text{\normalfont{vol}}}_{#1}}
\newcommand{\X}{\mathcal{X}}
\newcommand{\Z}{\mathbb{Z}}
\renewcommand{\epsilon}{\varepsilon}
%\renewcommand{\phi}{\varphi}
\newcommand{\phitilde}{\tilde{\phi}}
\newcommand{\phibar}{{\phi}_{FLRW}}
\newcommand{\del}{\partial}
\newcommand{\Lap}{\Delta}
\renewcommand{\div}{\text{\normalfont{div}}}
\newcommand{\curl}{\text{\normalfont{curl}}}

\newcommand{\LG}{\L_{\fg,\gamma}}
\newcommand{\Lapg}{\hat{\Lap}_{g,\gamma}}
\newcommand{\LapG}{\hat{\Lap}_{G,\gamma}}
\newcommand{\Gamhat}{\hat{\Gamma}}
\newcommand{\nabhat}{\hat{\nabla}}
\newcommand{\Ltilde}{\tilde{\L}}

\newcommand{\curve}{\bm{\alpha}}

\newcommand{\numberthis}{\addtocounter{equation}{1}\tag{\theequation}}
\renewcommand{\thesection}{\arabic{section}}
\renewcommand{\theequation}{\arabic{section}.\arabic{equation}}

\newcommand{\equp}[1]{\mathrel{\mathop{=}^{\mathrm{#1}}}}
\newcommand{\lequp}[1]{\mathrel{\mathop{\leq}^{\mathrm{#1}}}}
\newcommand{\rightarrowup}[1]{\mathrel{\mathop{\longrightarrow}^{{#1}}}}

\newcommand\geup[1]{\mathrel{\overset{\makebox[0pt]{\mbox{\normalfont\tiny #1}}}{>}}}
\newcommand\leup[1]{\mathrel{\overset{\makebox[0pt]{\mbox{\normalfont\tiny #1}}}{<}}}
\newcommand\gequp[1]{\mathrel{\overset{\makebox[0pt]{\mbox{\normalfont\tiny #1}}}{\geq}}}

\newcommand\Rightarrowup[1]{\mathrel{\overset{\makebox[0pt]{\mbox{\tiny #1}}}{\Rightarrow}}}
\newcommand\Leftrightarrowup[1]{\mathrel{\overset{\makebox[0pt]{\mbox{\tiny #1}}}{\Leftrightarrow}}}
\newcommand\longmapstoup[1]{\mathrel{\overset{\makebox[0pt]{\mbox{\tiny #1}}}{\longmapsto}}}

\usepackage{bm}
\newcommand{\RE}{\bm{E}}
\newcommand{\RB}{\bm{B}}
\newcommand{\epsilonLC}{\bm{\epsilon}}



%Future header

\newcommand{\epsilonnew}{\delta}
\newcommand{\fg}{\bm{g}}
\newcommand{\fk}{\bm{\Sigma}}
\newcommand{\fn}{\bm{n}}
\newcommand{\fN}{\hat{\fn}}
\newcommand{\fX}{\bm{X}}
\newcommand{\fLap}{\Lap_{\fg}}
\newcommand{\fdel}{\widetilde{\del_0}}

\newcommand{\Lie}{\mathcal{L}}
\newcommand{\phim}{\overline{\phi}}

\newcommand{\fE}{\mathbb{E}_{SF}}
\newcommand{\fC}{\mathcal{C}_{SF}}
\newcommand{\fEg}{E_{\text{geom}}}

%ToDos
\newcommand{\todo}[1]{\textcolor{red}{#1}}
\newcommand{\change}[1]{#1}
\newcommand{\changefinal}[1]{#1}
\newcommand{\changediss}[1]{#1}
\usepackage{ulem}
\newcommand{\delete}[1]{}
\usepackage{cancel}
\newcommand{\deletemath}[1]{}

\title{Cosmic Censorship near FLRW spacetimes with negative spatial curvature}
\author[D.~Fajman, L.~Urban]{David Fajman, Liam Urban}
\address{
\begin{tabular}[h]{l@{\extracolsep{8em}}l} 
David Fajman  & Liam Urban \\
Faculty of Physics & Faculty of Mathematics\\ 
University of Vienna & University of Vienna \\
Boltzmanngasse 5 & Oskar-Morgenstern-Platz 1 \\
1090 Vienna, Austria & 1090 Vienna, Austria\\
david.fajman@ univie.ac.at & liam.urban@ univie.ac.at 
\end{tabular}
}

\keywords{Einstein scalar-field system, stability, blow-up profile, cosmic censorship, Big Bang singularity}


\begin{document}
\maketitle


\begin{abstract}
We consider \change{general initial data }for the Einstein scalar-field system on a closed \delete{orientable }$3$-manifold \change{$(M,\gamma)$ }which is close to data for a Friedman-Lema{\^\i}tre-Robertson-Walker solution with \change{homogeneous }scalar field matter and a negative Einstein metric $\gamma$ as spatial geometry. We prove that \change{the maximal globally hyperbolic development of such initial data in the Einstein scalar-field system is past incomplete in the contracting direction and exhibits stable collapse into a Big Bang curvature singularity}. Under an additional condition on the first positive eigenvalue of \change{$-\Delta_\gamma$ }satisfied, for example, \change{by closed hyperbolic 3-manifolds of small diameter}, we prove that the data evolves to a future complete spacetime in the expanding direction which asymptotes to a vacuum Friedman solution with \change{$(M,\gamma)$ }as the expansion normalized spatial geometry. In particular, the Strong Cosmic Censorship conjecture holds for this class of solutions in the $C^{2}$-sense.
\end{abstract}




\setcounter{tocdepth}{2}
%\tableofcontents
%\newpage


\section{Introduction}

Generative modeling has been the dominant approach for large-scale pretraining and zero-shot generalization~\cite{gpt3-paper,artetxe2021efficient,rae2021scaling}. 
Combined with prompts~\cite{gpt3-paper}, most of the natural language processing (NLP) tasks can be formulated into the fill-in-the-blank format and perform generative language modeling.
Based on the unified generative formulation, pretrained models such as GPT-3~\cite{gpt3-paper}, BERT~\cite{devlin2018bert,PET-paper}, T5~\cite{T5-paper}, can perform zero-shot inference on new tasks. 


More recent work~\cite{T0-paper} proposed to further pretrain a generative T5~\cite{T5-paper} with multitask prompted datasets and has substantially enhanced the performance of zero-shot generalization. 
In contrast, methods based on discriminative modeling~\cite{devlin2018bert} have not been able to achieve state-of-the-art performance on zero-shot learning. The adoption of discriminative approaches for zero-shot learning has been limited in the literature.


% Although there are a few works using discriminative modeling to perform zero-shot or few-shot learning, such as CLS finetuning using BERT or prompting using ELECTRA
% For example, BERT was CLS finetuned to perform zero-shot/few-shot learning, however, the zero-shot/few-shot performance are lagged far behind.

% \zy{Add a note: although BERT can be CLS finetuned (which is discriminative), but it is not the SOTA approach for zero-shot and few-shot learning.}

\begin{figure}%[htbp]
     \centering
     \includegraphics[width=1.05\linewidth]{figure/final_sota.png}
     \vspace{-15pt}
     \caption{Average zero-shot performance over 11 zero-shot tasks for our Universal Discriminator and T0~\cite{T0-paper}. Our universal discriminator significantly outperforms T0 across three different scales.}
     \label{fig:sota}
     \vspace{-15pt}
 \end{figure} 


In this work, we challenge the convention of zero-shot learning and propose to study and improve discriminative approaches. This is motivated by the fact that many NLP tasks can be framed as selecting from a few options; e.g., telling whether sentence A entails sentence B, or predicting which answer is correct for a given question. We call these tasks \textit{discriminative tasks}. As we will discuss in later sections, a significant portion of NLP tasks is in fact discriminative tasks. We hypothesize that discriminative approaches perform better for discriminative tasks.
% Despite the recent progress, it remains unknown how discriminative approaches perform in zero-shot generalization. Motivated by the fact that discriminative modeling learns to distinguish among options and goes better with discriminative tasks (e.g., telling whether sentence A entails sentence B, or telling which option correctly answer the question), we hypothesize that discriminative modeling would be better at zero-shot generalization, especially on discriminative tasks.

To verify the hypothesis, we propose the \textbf{universal discriminator (UD)}, which substantially improves zero-shot generalization over the previous generative state-of-the-art (SOTA)~\cite{T0-paper}, as Figure~\ref{fig:sota} shows.
The main idea is to train a single discriminator to predict whether a text sample comes from the true data distribution of natural language, similar to GANs \cite{goodfellow2014generative}. Given a set of training tasks with labeled data, we construct a dataset with positive and negative examples, where positive ones are in-distribution natural language samples and negative ones are out-of-distribution. There are two major types of discriminative tasks. The first type is tasks with multiple options, such as multi-choice question answering and news classification. We fill the options into the sentences and the ones with correct options are considered positive samples. The second type is tasks with yes/no options, which can be formulated as a binary discrimination problem itself. For example, natural language inference aims to predict whether a premise entails a hypothesis. In this case, we use a prompt to concatenate the premise $A$ and the hypothesis $B$ into a sentence ``Premise: $A$. Hypothesis: $B$.'' If entailment holds, this sample is treated as positive in-distribution samples and otherwise negative out-of-distribution ones.



% We define the true data distribution using multiple training tasks with labeled data. Specifically, since discriminative tasks can be formulated as selecting from a few options, samples with correct options form an empirical data distribution, while samples with incorrect options are considered out of distribution. In other words, our discriminator is trained to predict ``true'' for samples with correct options and ``false'' for incorrect ones. We use simple concatenation to minimize prompting efforts. For example, given an example (premise, hypothesis), a natural language inference task predicts whether the premise entails the hypothesis. We concatenate the premise and hypothesis, and assign the label ``true'' for entailment and ``false'' for non-entailment.


% First off, since many of the NLP tasks can be formulated as selecting from several options, we first reformulate the task data into natural text samples by concatenating different fields \zy{what are fields? undefined here. try using another word.}.
% For example, given an example of \zy{the} natural language inference task (\textit{Premise}, \textit{Hypothesis}, \textit{Label}), the natural text is reformulated as ``\textit{\{Premise\} || \{Hypothesis\}}'' labeled with \textit{\{Label\}}. \footnote{Here we use ``||'' to represents direct concatenation.} 
% Another example of topic classification task (\textit{Text}, \textit{Label}) where the \textit{Label} indicates the first option of \{Sports, Fashion, Politics\}, the corresponding natural texts are formulated as ``\textit{Text} || Sports'' labeled with 1, ``\textit{Text} || Fashion'' and ``\textit{Text} || Politics'' both labeled with 0.
% Secondly, we pretrain a pretrained model with reformulated multitask datasets to distinguish whether the text sample comes from the true data distribution. ~\footnote{An assumption is that negative-labeled text samples are artificially constructed thus do not come from the true data distribution, and vice versa.}

For the performance of zero-shot generalization, our approach achieves new state-of-the-art on the T0 benchmark, outperforming T0 by 16.0\%, 7.8\%, and 11.5\% respectively on different scales. 
UD also achieves state-of-the-art performance on a wide range of supervised NLP tasks, using only 1/4 parameters of previous methods.
Compared with the previous generative prompt-based methods, our universal discriminator requires minimal prompting, which is simple, robust, and applicable in real-world scenarios.

% By further scaling the number of tasks, our approach also sets the new state-of-the-art on \textbf{\color{red}[xxx]} tasks with less than 10\% of model parameters \zy{need to give a range} under the setting of standard finetuning.
% In the setting of finetuning, our approach also outperforms the generative baselines consistently across a wide range of tasks.


In addition, we also generalize UD to a larger scope of tasks, such that UD can perform discriminative and generative tasks at the same time. Specifically, we extend UD to the encoder-decoder architecture for training on generative tasks, and restrict the model's prediction on "yes"/"no" tokens for jointly training discriminative tasks. Results prove that generalized UD maintains UD's advantages on discriminative tasks and achieves comparable results on generative tasks (See \S~\ref{sec:generalizedud}). 
% We leave expanding UD to a broader range of generative tasks and achieve greater performance on generative tasks as our future work


% \xhk{I admit the limitation on generative tasks here as our future work.}

%\xhk{Although UD is designed for improving zero-shot performance for discriminative tasks, we can also combine this idea to train a generalized UD model which simultaneously solves both discriminative tasks and generative tasks, maintaining UD's advantage on discriminative tasks and get comparable results on generative tasks (See \S~\ref{sec:generalizedud}).}

% The universal discriminator provides a new perspective for zero-shot generalization---Compared with generating the true verbalizer that indicates task label with extensive prompt engineering, distinguishing between options with minimal prompting efforts is simple, robust, and high-performing, thus is more applicable in real-world scenarios. \zy{rewirte the above sentence, just focus on one point---minimal prompting}


\noindent \textbf{Acknowledgements.} \changefinal{This research was funded in whole or in part by the Austrian Science Fund (FWF) 10.55776/Y963 and 10.55776/P34313. For open access purposes, the author has applied a CC BY public copyright license to any author-accepted manuscript version arising from this submission. }\change{Liam Urban is a recipient of a DOC Fellowship of the Austrian Academy of Sciences at the Faculty of Mathematics at the University of Vienna. }Liam Urban also thanks the German Academic Scholarship Foundation (Studienstiftung des deutschen Volkes) for their scholarship. The authors thank \changefinal{Ian Agol, Klaus Kröncke, Michael Lipnowski, Dalimil Mazac and Roman Prosanov }for their help in seeking out \changefinal{numerical and analytic }evidence for the spectral condition used in Section \ref{sec:fut}, and Michael Eichmair \change{and the anonymous referees }for \changefinal{their detailed, constructive and warm }feedback on a previous version of this manuscript. \changefinal{The authors would like to thank the Erwin Schrödinger International Institute for Mathematics and Physics in Vienna for hosting the authors during the Thematic Programs \enquote{Mathematical Perspectives of Gravitation beyond the Vacuum Regime}, \enquote{Spectral Theory and Mathematical Relativity} and \enquote{Nonlinear Waves and General Relativity} during which research for this work was done and parts of this paper were written.}

%\tableofcontents

\section{Big Bang stability: Preliminaries}\label{sec:prelim}

\subsection{Notation}\label{subsec:notation}

\subsubsection{Foliations}\label{subsubsec:notation-foliation}

\change{On a spacetime manifold $(\M,\g)$, we assume the existence of a spacelike Cauchy hypersurface $\Sigma_{t_0}$ that is diffeomorphic to $M$. As \changefinal{we argue }in Remark \ref{rem:CMC-hypersurface}, we can assume without loss of generality that it has constant mean curvature. We will ultimately show that there exists a time function $t$ such that the past of $\Sigma_{t_0}=t^{-1}(t_0)$ can be foliated by $\Sigma_{s}=t^{-1}(s)$ for $s\in (0,t_0)$, and that where the solution exists, this is at least possible up to some $T\in(0,t_0)$. These constant time surfaces are then also spacelike Cauchy hypersurfaces diffeomorphic to $M$ and CMC. We will use this notation throughout with little comment and often simply view $\Sigma_s$ as $\{s\}\times M$.}

\subsubsection{Metrics}\label{subsubsec:notation-metric}

The spacetime metric $\g$ on $\M$ takes the general form
\[\g=-n^2dt^2+g_{ab}dx^adx^b\]
where $n\equiv n(t,x)$ is the lapse function and $g\vert_{\Sigma_t}\equiv g\vert_{\Sigma_t}(t,x)$ is a Riemannian metric on $\Sigma_t$. We will often simply denote the spatial metric by $g$. Furthermore, we denote the rescaled spatial metric \changefinal{by $G_{ij}=a^{-2}g_{ij}$ (see Definition \ref{def:rescaled}) }and the tensor-field induced by the matrix inverse of $(G_{ij})$ by $G^{-1}$. Similarly, $\det g$ and $\det G$ are also meant as the determinants in the matrix sense. Finally, we define $\vol{g}$ and $\mu_g$ as the volume form and volume element with regard to $g$, and the same for $\gamma$ and $G$.

\subsubsection{Indices and coordinates}\label{subsubsec:notation-indices}

Greek indices $\alpha,\beta,\dots,\mu,\nu,\dots$ run from $0$ to $3$, lowercase latin indices $a,b,\dots$, $i,j,\dots$ from $1$ to $3$. The spatial indices on some coordinate neighbourhood $V\subseteq\M$ are always with regard to the local frame induced by coordinates $(x^1,x^2,x^3)$ on $M$, applied to each $V\cap\Sigma_t$ by the standard embedding where this intersection is non-empty. The index $0$ always denotes components relative to $\del_0=n^{-1}\del_t$, where $\del_t$ is the derivative associate to the time function $t$. The Levi-Civita connections associated to $\g$, respectively $g$ and $G$, are denoted by $\nabbar$, respectively $\nabla$.\footnote{Note that $g$ and $G$ have the same Levi-Civita-connection since, on every hypersurface $\Sigma_t$, they are related by a scalar multiple. }Additionally, for the hyperbolic spatial reference metric $\gamma$ on $M$ (see  Definition \ref{def:spatial-mf}), we write the Levi-Civita connection as $\nabhat$.\\

Whenever we raise or lower Greek (resp. Latin) indices without additional notation, it is with regard to $\g$ (resp. $g$). When we raise indices of a tensor $\mathfrak{T}$ with regard to the rescaled spatial metric $G$, we flag this by writing $\mathfrak{T}^\sharp$. We never raise or lower with respect to $\gamma$. Refer to Section \ref{subsubsec:notation-derivatives} as to how we distinguish taking multiple covariant derivatives from index raising.

\subsubsection{$\Sigma_t$-tangent tensors}\label{subsubsec:notation-tangent}

For any $\Sigma_t$-tangent tensor ${\xi^{\alpha_1\dots\alpha_r}}_{\beta_1\dots\beta_s}$, we write ${{\xi(t)}^{a_1\dots a_r}}_{b_1\dots b_r}$ for the $\g$-orthogonal projection of $\xi$ onto the hypersurface $\Sigma_t$. When clear from context, we will drop the time dependency in notation.


\subsubsection{Sign conventions}\label{subsubsec:notation-sign}

Within this paper, the second fundamental form with regard to $\Sigma_t$ is defined \change{as the $(0,2)$-tensor $k$ given by
\[k(X,Y)=-\g(\nabbar_X\del_0,Y)\,,\]
where $X$ and $Y$ are $\Sigma_t$-tangent vectors}. The Riemann curvature tensor of $\g$ is taken to be
\change{\[\nabbar_\alpha\nabbar_\beta Z_\gamma-\nabbar_\beta \nabbar_{\alpha}Z_\gamma={\Riem[\g]_{\alpha\beta\gamma}}^\delta Z_\delta\]}
for the covariant vector field $(Z_\mu)$, and the analogous convention holds for all other Riemann curvature tensors that appear.

\subsubsection{Constants}\label{subsubsec:notation-constants}

For two nonnegative scalar functions $\zeta_1,\zeta_2$, we write $\zeta_1\lesssim\zeta_2$ \change{if and only if } there exists a constant $K>0$ such that $\zeta_1\leq K\zeta_2$. This implicit constant may depend on information from the FLRW reference solution at the starting point of the evolution (in particular on $\gamma$ and $a(t_0)$, see \change{Definition \ref{def:spatial-mf}}) and combinatorial quantities. We extend this notation to a real function $\zeta_1^\prime$ by
\[\zeta_1^\prime\lesssim\zeta_2 :\Leftrightarrow \max\left(\zeta_1^\prime,0\right)\lesssim\zeta_2\,.\]
Additionally, we write $\zeta_1\simeq\zeta_2$ \change{if and only if } $\zeta_1\lesssim\zeta_2\lesssim\zeta_1$ is satisfied.\\

%Furthermore, $c>0$ is a positive constant that may be updated in each line. We do not allow updates that depend on initial data size $\epsilon$ (see Assumption \ref{ass:init}) and the bootstrap parameter $\sigma$ (see Assumption \ref{ass:bootstrap}), but do allow varying the constant dependent on data of the FLRW reference metric and $t_0$ as well as combinatorial factors. As long as $\epsilon$ and $\sigma$ are chosen suitably, this is not only possible but also allows to balance any changes within this constant $c$ as we will argue in the proof of Theorem \ref{thm:main}.

\subsubsection{Tensor contractions}\label{subsubsec:notation-contract}

We denote by $\epsilonLC_{\alpha\beta\gamma\delta}$ the Levi Civita tensor with regard to $\g$ and define the Levi-Civita tensor on spatial hypersurfaces $\Sigma_t$ by $\epsilonLC[g]_{ijk}=\epsilonLC_{0ijk}$. Notice that this corresponds to the Levi-Civita tensor associated to $g$. Further, $\epsilonLC[G]_{ijk}=a^{-3}\epsilonLC[g]_{ijk}$ is the Levi-Civita tensor with respect to the rescaled metric $G$ (see \eqref{eq:rescalingGK}).

For $\Sigma_t$-tangent $(0,2)$-tensors $A,\tilde{A}$ and vector field $v$, we define the following objects \change{as in \cite[Section A.2]{AM03}}:
\begin{align*}
A\cdot\tilde{A}&=\changefinal{A_{ab}\tilde{A}^{ab}}=\langle A,\tilde{A}\rangle_g\\
(A\odot_g\tilde{A})_{ij}&=\change{A_{ik}{\tilde{A}^k}_{\ j}}\\
(A\wedge \tilde{A})_i&={\epsilonLC_i}^{jp}{A_j}^q\tilde{A}_{qp}\\
(v\wedge A)_{ab}&={\epsilonLC_a}^{cd}v_cA_{db}+{\epsilonLC_b}^{cd}v_cA_{ad}\\
(A\times \tilde{A})_{ij}&={\epsilonLC_{i}}^{ab}{\epsilonLC_j}^{pq}A_{ap}\tilde{A}_{bq}+\frac13(A\cdot\tilde{A})g_{ij}-\frac13(tr_gA\cdot tr_g\tilde{A})g_{ij}\\
(\curl A)_{ij}=(\curl_g A)_{ij}&=\frac12\left[{\epsilonLC_i}^{cd}\nabla_dA_{cj}+{\epsilonLC_j}^{cd}\nabla_dA_{ci}\right]\\
(\div_g A)_i&=\nabla^bA_{ib}
\end{align*}
The operations $\odot_G$, $\langle\cdot,\cdot\rangle_G$ and $\div_G$ are defined analogously, with all indices raised and lowered by $G$ instead of $g$. Finally, for two $(0,1)$-tensors $\pi,\tilde{\pi}$, we denote their symmetrized product by 
\[(\pi\otimes\tilde{\pi})_{ij}=\frac12\left(\pi_i\tilde{\pi}_j+\pi_j\tilde{\pi}_i\right)\,.\]
\change{For pointwise estimates of these quantities, refer to Lemma \ref{lem:BelRobinsonLemmas}.}

\subsubsection{Schematic term notation}\label{subsubsec:schematic-notation}
We will denote as $\mathfrak{T}_1\ast\dots\ast\mathfrak{T}_l$, \change{where $\mathfrak{T}_i$ }are $\Sigma_t$-tangent tensors, any multiple of $(\mathfrak{T}_i)$, with regard to the rescaled adapted spatial metric $G$ or as standard multiplication if no summation over indices occurs between factors. Constant prefactors and contractions with regard to $G$ are also \change{suppressed }in this notation. 

When working with terms where such notation is used, we will estimate these inner products by $\lesssim \prod_{i=1}^l\lvert\mathfrak{T}_i\rvert_G$, making any constant in front irrelevant, and further we can view any contraction with regard to $G$ as a product of the non-contracted tensor $\mathfrak{T}$ with $G$ or $G^{-1}$, and estimate that up to constant by $\lvert G\rvert_G\lvert\mathfrak{T}\rvert_G$, where the first factor is simply $\sqrt{3}$.\\
For similar products with respect to $\gamma$, we denote them by $\ast_{\gamma}$.

\subsubsection{On multiple derivatives of variables}\label{subsubsec:notation-derivatives}

For a scalar function $\zeta$, an $(r,s)$-tensor field $\mathfrak{T}$ and \textit{capitalized} integers $I,J,\change{\ldots}\in\N_0$, we denote by $\nabla^I\zeta$ and $\nabla^I\mathfrak{T}$ the tensors $\nabla_{l_1}\dots\nabla_{l_I}\zeta$ and\linebreak
$\nabla_{l_1}\dots\nabla_{l_I}{\mathfrak{T}^{i_1\dots i_r}}_{j_1\dots j_s}$. We extend this notation to other covariant derivatives analogously. To avoid potential ambiguity with an index raised by \change{$g$}, we will apply the following convention:
\begin{itemize}
\item If a covariant derivative carries an uppercase letter, a formula with more than one symbol or a positive integer in its superscript, this refers to taking a derivative of that order.
\item If a covariant derivative carries a lowercase letter or $0$ in its superscript, this refers to an index.
\end{itemize}
Further, we will only apply this notation where the precise distribution of indices is not important (e.g.\,in schematic notation, see Section \ref{subsubsec:schematic-notation}).
%We will further only use this notation within the schematic notation in as mentioned above, or in cases where how indices are contracted is obvious (e.g.\,$\lvert\nabla^l\zeta\rvert_G$), or for $\nabla^2\zeta$ and similar symmetric expressions which is symmetric in the indices of $\nabla$. Thus, we don't lose any relevant information by not tracking if/how which indices in the covariant derivatives are contracted with others. 

\subsection{FLRW spacetimes and the Friedman equations}\label{subsec:FLRW}

Herein, we collect the properties of the reference FLRW solution to the Einstein scalar-field system in CMC-transported coordinates. Our main focus will lie on the behaviour of the scale factor as determined by the Friedman equations. Before moving on to that, we collect the information on the spatial geometry we will need:

\begin{definition}[Hyperbolic reference geometry]\label{def:spatial-mf}
$(M,\gamma)$ is a three-dimensional, connected, closed, orientable Riemannian manifold with constant sectional curvature $-\frac19$, and hence Ricci tensor $\Ric[\gamma]=-\frac29\gamma_{ij}$ and scalar curvature $R[\gamma]=-\frac23$. 
\end{definition}
\change{\begin{remark}[Orientability is not a restriction]
We assume that $M$ is orientable for the sake of simplicity. If $M$ should be non-orientable, we may pass the initial data to the oriented double cover and solve the problem there. Since the result is equivariant with respect to the double covering map, this then solves the original problem.
\end{remark}}

With this in hand, we can express our classical \change{family of solutions }to the Einstein scalar-field system as follows:

\begin{lemma}[FLRW solutions and Friedman equations]\label{lem:FLRW}
Consider FLRW spacetimes $(\M,{\g}_{FLRW})$ with $\M=(0,\infty)\times M$, where $(M,\gamma)$ is as in Definition \ref{def:spatial-mf} and where
\begin{equation}\label{eq:FLRW-metric}
{\g}_{FLRW}=-dt^2+a(t)^2\gamma
\end{equation}
holds for some $a\in C^\infty((0,\infty))$, with the conventions $a(0)=0$ and $\dot{a}(T)>0$ for some arbitrary $T>0$. Further, choose a (smooth) scalar function $\phibar$ such that
\begin{equation}\label{eq:FLRW-wave}
\del_t\phibar=C\cdot a(t)^{-3},\quad \nabla\phibar=0, \quad \square_{\g_{FLRW}}\phibar=0\,.
\end{equation}
Such a pair $(\g_{FLRW},\phibar)$ solves the Einstein scalar-field system \eqref{eq:ESF1}-\eqref{eq:EMT} on $\M$ if and only if $a$ satisfies \change{the Friedman equation
\begin{equation}\label{eq:Friedman}
\dot{a}=\sqrt{\frac19+\frac{4\pi}3C^2a^{-4}}\,.
\end{equation}
In particular, one has
\begin{equation}\label{eq:Friedman2}
\ddot{a}=-\frac{8\pi}3C^2a^{-5}\,.
\end{equation}}
%Additionally, \eqref{eq:Friedman2} follows from \eqref{eq:Friedman}.
\end{lemma}
\begin{proof}
The first statement follows from explicitly computing $\Ric[\g]$ as in \cite[p.345]{ONeill83}. That \eqref{eq:Friedman} implies \eqref{eq:Friedman2} follows simply by computing the derivative of $\dot{a}^2$.
\end{proof}

In the subsequent analysis, the following properties of $a$ that follow from \eqref{eq:Friedman} will be crucial for our analysis:

\begin{lemma}[Scale factor analysis]\label{lem:scale-factor}Let $a$ solve \eqref{eq:Friedman} with $a(0)=0$. Then $a$ is analytic on $(0,\infty)$ and extends to a continuous function on $[0,\infty)$ with $a(t)\simeq t^{\frac13}$ being satisfied near $t=0$. Further, for any $p>0$, there exist constants $c>0$ and $K_p>0$, where $c$ is independent of $p$ and $K_p$ depends analytically on $p$, such that, for any $t\in(t,t_0]$, one has
\begin{equation}\label{eq:a-exp-est}
\exp\left(p\int_t^{t_0}a(s)^{-3}\,ds\right)\leq K_pa(t)^{-cp}
\end{equation}
and
\begin{equation}\label{eq:a-integrals}
\int_t^{t_0}a(s)^{-3-p}\,ds\changefinal{\lesssim}\frac1p a(t)^{-p},\quad \int_t^{t_0}a(s)^{-3+p}\,ds\lesssim\frac{1}p\,.
\end{equation}
Moreover, for any $t\in(0,t_0]$ and any $q>0$, \change{there exist constants $c>0$ and $K>0$ which both are independent of $q$ such that one has
\begin{equation}\label{eq:log-est}
\int_t^{t_0}a(s)^{-3}\,ds\leq \frac{K}qa(t)^{-cq}\,.}
\end{equation}
Finally, \eqref{eq:Friedman} also implies
\begin{equation}\label{eq:diff-ineq-Friedman}
\sqrt{\frac{4\pi}3}C a^{-2}\leq \dot{a}
\end{equation}
\end{lemma}
\begin{remark}\label{rem:scale-factor}
We will use the estimates in Lemma \ref{lem:scale-factor} where $p$ is a positive power of $\epsilon$ up to algebraic constants. Then, we can simply replace $K_p$ in \eqref{eq:a-exp-est} by a uniform constant. %\\
%
%Further, the only point in Lemma \ref{lem:scale-factor} that does not extend to flat or (replacing $(0,\infty)$ with a bounded open interval) to spherical spatial geometry is \eqref{eq:diff-ineq-Friedman} for $\kappa=1$. Here, we only have
%\begin{equation}\label{eq:diff-ineq-Friedman-weak}
%\dot{a}\leq \sqrt{\frac{4\pi}3}Ca^{-2}+K
%\end{equation}
%for some $K>0$ near the Big Bang at $t=0$. Upon close inspection, it actually turns out this is also sufficient for our analysis by taking a little more care with error terms. We will point out how to adapt the argument at these crucial points, but won't perform the entire argument explicitly.
\end{remark}
\begin{proof}
For the first statement, we refer to \cite[Lemma 2.1]{Urban22} with $\gamma=2$. We also collect from there\footnote{\label{fn:t0-small}In \cite[Lemma 2.1]{Urban22}, one at first only has $\int_t^{t_0}a(s)^{-3}\,ds\lesssim \log(t_0)-\log(t)$ for $t_0$ small enough that we can estimate $a(t)$ by $t^\frac13$ up to constant. However, assuming this inequality were to hold up to $\overline{t}>0$ and one had $t_0>\overline{t}$, the contribution $\int_{\overline{t}}^{t_0}a(s)^{-3}\,ds$ only adds a constant that we can absorb into our notation. Similarly, $a(t)\simeq t^\frac13$ can be assumed to hold on $(0,t_0]$ for any fixed $t_0>0$, and we can ignore this technicality in proving the integral formulas in Lemma \ref{lem:scale-factor}.} that, for $t<t_0$, 
\[\changefinal{\int_t^{t_0}a(s)^{-3}\,ds\lesssim 1+\left\lvert\log\left(\frac{t}{t_0}\right)\right\rvert}\]
is satisfied. Hence, there exists some $c^\prime>0$ such that
\[\changefinal{\exp\left(p\int_t^{t_0}a(s)^{-3}\,ds\right)\leq K_p\exp(c^\prime\cdot p\log(t_0))\cdot\exp(-c^\prime\cdot p)\leq K_p^\prime t^{-c^\prime p}}\,.\]
Then \eqref{eq:a-exp-est} follows by applying $a(t)\simeq t^\frac13$. Noting that $a^{-3}\simeq \nicefrac{\dot{a}}a$ holds, one further has
\begin{equation}
\int_t^{t_0}a(s)^{-3-p}\,ds\lesssim\int_{a(t)}^{a(t_0)}y^{-1-p}dy=\frac1p\left(a(t)^{-p}-a(t_0)^{-p}\right)\leq\frac1p a(t)^{-p}\,,
\end{equation}
and the other inequality in \eqref{eq:a-integrals} follows analogously. Finally, \eqref{eq:log-est} follows directly from \eqref{eq:a-integrals} when assuming without loss of generality that $a\lvert_{(t,t_0)}$ only takes values in $(0,1)$.
%For \eqref{eq:log-est}, we again assume $t<1$ without loss of generality and collect
%\[\int_t^{t_0}a(s)^{-3}\,ds\lesssim \lvert\log(t)\rvert=\log(t^{-1})\,.\]
%Consider $f(y)=\log(y)y^{-q}$ on $y\in[1,\infty)$. Note that $f$ only vanishes at $y=1$ and that $\lim_{y\to\infty}f(y)=\lim_{y\to\infty}(qy^q)^{-1}=0$, so it attains a global maximum where $f^\prime(y)=(1-q\log(y))y^{-q-1}=0$ holds, namely $f(e^{\frac1q})=(qe)^{-1}$. Inserting this above, we get
%\[\int_t^{t_0}a(s)^{-3}\,ds\lesssim \frac1{qe} t^{-q}\lesssim \frac1qa(t)^{-3q}\,.\]

\end{proof}


\subsection{Solutions to the Einstein scalar-field equations in CMC gauge}\label{subsec:eq}

From here on out, we impose the CMC condition\change{\footnote{\change{Recall that $k$ is negative in our sign convention, see Section \ref{subsubsec:notation-sign}.}}}
\begin{equation}\label{eq:CMC}
{k^{l}}_l(t,\cdot)=\tau(t)=-3\frac{\dot{a}(t)}{a(t)}\,.
\end{equation}
We use \eqref{eq:Friedman} and \eqref{eq:Friedman2} to collect the following formulas for the mean curvature:
\begin{align}
\label{eq:Hubble} \del_t\tau%=(-3)\frac{a\cdot\ddot{a}-\dot{a}^2}{a^2}\\
%&=3\left(\frac{8\pi}3C^2a^{-6}+\frac19 a^{-2}+\frac{4\pi}3C^2a^{-6}\right)\\
&=12\pi C^2a^{-6}+\frac13a^{-2}\\
\label{eq:Hubble2} \tau^2&=9\frac{\dot{a}^2}{a^2}=12\pi C^2a^{-6}+a^{-2}\,.
\end{align}
We consequently define the trace-free component $\hat{k}$ of $k$ as
\begin{equation}
\hat{k}_{ij}=k_{ij}-\frac{\tau}3g_{ij}
\end{equation}
and recall that the future directed unit normal to our foliation is written as
\begin{equation}
\del_0=n^{-1}\del_t\,.
\end{equation}

With this, we can express the Einstein scalar-field equations in our gauge as follows:

\begin{prop}[The Einstein scalar-field system in CMC gauge]\label{prop:eq}
A pair $(\g,\phi)$ solves the Einstein scalar-field equations \eqref{eq:ESF1}-\eqref{eq:ESF2} on $I\times M$ in CMC gauge \eqref{eq:CMC} for some interval $I\subseteq (0,t_0]$, where the scale factor satisfies \eqref{eq:Friedman}, if and only if the following equations are satisfied on $I\times M$:\\
The \textbf{metric evolution equations}
\begin{subequations}
\begin{align*}
\del_t g_{ij}=&\,-2nk_{ij}=-2n\hat{k}_{ij}+2n\frac{\dot{a}}ag_{ij}\,, \numberthis\label{eq:EEqg}\\
\numberthis\label{eq:EEqk}\del_t \hat{k}_{ij}%&\,\del_tk_{ij}-\frac{\del_t\tau}3g_{ij}-\frac{\tau}3\del_tg_{ij}\\
%=&\,-\nabla_i\nabla_jn+n\left[\Ric[g]_{ij}+\tau k_{ij}-2k_{il}k^l_j-8\pi\nabla_i\phi\nabla_j\phi\right]\\
%&\,-(4\pi C^2a^{-6}+a^{-2})g_{ij}+\frac{2\tau}3n\hat{k}_{ij}+\frac{2\tau^2}9ng_{ij}\\
%=&\,-\nabla_i\nabla_jn+n\left[\Ric[g]_{ij}+\frac{\tau}3\hat{k}_{ij}+\frac{\tau^2}3g_{ij}-2\hat{k}_{il}\hat{k}^l_j-8\pi\nabla_i\phi\nabla_j\phi\right]\\
%&\quad-(4\pi C^2a^{-6}+a^{-2})g_{ij}\\
=&\,-\nabla_i\nabla_jn+n\left[\Ric[g]_{ij}-\frac{\dot{a}}a\hat{k}_{ij}-2\hat{k}_{il}\hat{k}^l_j-8\pi\nabla_i\phi\nabla_j\phi\right]\\
&\quad+4\pi C^2a^{-6}(n-1)g_{ij}+\frac19\left(3n-1\right)a^{-2}g_{ij}\,,%\\
%%\numberthis\label{eq:EEqksharp1}\del_tk^a_b=&\,-\nabla^a\nabla_bn+n\left[\Ric[g]^a_{b}+\tau{k}^a_b-8\pi\nabla^a\phi\nabla_b\phi\right]\,,\\
%\numberthis\label{eq:EEqksharp2}{\del_t\hat{k}^i}_j=&\,-\nabla^i\nabla_jn+n\left[{\Ric[g]^i}_{j}+{\tau\hat{k}^i}_j-8\pi\nabla^i\phi\nabla_j\phi\right]+\I^i_j\underbrace{\left[\frac{\tau^2}3-\frac{\del_t\tau}3\right]}_{=\frac29a^{-2}}+(n-1)\frac{\tau^2}3\I^i_j\,,
\end{align*}
\end{subequations}
the \textbf{Hamiltonian} and \textbf{momentum constraint equations}
\begin{subequations}
\begin{align}
R[g]+\frac23\tau^2-\langle\hat{k},\hat{k}\rangle_g=&8\pi\left[\lvert \del_0\phi\rvert^2+\lvert\nabla\phi\rvert_g^2\right] \label{eq:Hamilton}\,,\\
\nabla^l\hat{k}_{lj}=&-8\pi\nabla_j\phi\cdot \del_0\phi\,, \label{eq:Momentum}
\end{align}
\end{subequations}
the \textbf{lapse equation}
\begin{subequations}
\begin{align*}
\numberthis\label{eq:EEqLapse}\Lap_g n%=&-12\pi C^2a^{-6}-\frac13a^{-2}+n\left[\hat{k}_{ij}\hat{k}^{ij}+\frac{\tau^2}3+8\pi\lvert n^{-1}\del_t\phi\rvert^2\right] \\
=&-12\pi C^2a^{-6}-\frac13a^{-2}+n\left[\frac13a^{-2}+4\pi C^2a^{-6}+\langle\hat{k},\hat{k}\rangle_g+8\pi\lvert \del_0\phi\rvert^2\right]\,,
\end{align*}
or equivalently by \eqref{eq:Hamilton}
\begin{equation}\label{eq:EEqLapse2}
\Lap_g n=-12\pi C^2a^{-6}-\frac13a^{-2}+n\left[R[g]-8\pi\lvert\nabla\phi\rvert_g^2+12\pi C^2a^{-6}+a^{-2}\right]\,,
\end{equation}
\end{subequations}
and the \textbf{wave equation}
\begin{equation}\label{eq:wave}
\square_{\g}\phi=-\del_0^2\phi+n^{-1}g^{ij}\nabla_in\nabla_j\phi+\Lap_g\phi+\tau \del_0\phi=0\,.
\end{equation}
%Finally, we also collect the \textbf{Ricci evolution equation}
%\begin{align}
%\numberthis\label{eq:EEqR}\del_t\Ric[g]_{ab}%&=\Lap_g(nk_{ab})+\nabla_a\nabla_b(n\tau)-\nabla^d\nabla_a(nk_{db})-\nabla^d\nabla_b(nk_{da})\\
%&=\Lap_g(n\hat{k}_{ab})-\nabla^d\nabla_a(n\hat{k}_{db})-\nabla^d\nabla_b(n\hat{k}_{da})+\frac\tau3(g_{ab}\Lap_gn+\nabla_a\nabla_bn)\,.
%\end{align}
\end{prop}
\begin{proof}
These are standard equations that follow from \cite[Chapter 2.3]{Rendall08} and applying \eqref{eq:CMC}-\eqref{eq:Hubble2}.
\end{proof}



\subsection{Bel-Robinson variables}\label{subsec:BR}
In this subsection, we briefly (re-)establish Bel-Robinson variables and how they behave within the Einstein scalar-field system.\\

\noindent\change{ Recall that the Weyl tensor $W\equiv W[\g]$ is the trace-free component of the spacetime curvature and, in the Einstein scalar-field system, takes the form
\begin{align*}
W_{\alpha\beta\gamma\delta}=&\,\Riem[\g]_{\alpha\beta\gamma\delta}-P[\g]_{\alpha\beta\gamma\delta}\,,\\
P[\g]_{\alpha\beta\gamma\delta}=&\,\frac12\left(\g_{\alpha\gamma}\Ric[\g]_{\beta\delta}-\g_{\beta\gamma}\Ric[\g]_{\alpha\delta}-\g_{\alpha\delta}\Ric[\g]_{\gamma\beta}+\g_{\beta\delta}\Ric[\g]_{\alpha\gamma}\right)-\frac16 R[\g]\left(\g_{\alpha\gamma}\g_{\beta\delta}-\g_{\alpha\delta}\g_{\beta\gamma}\right)\\
=&\,4\pi\left(\g_{\alpha\gamma}\nabbar_{\beta}\phi\nabbar_\delta\phi-\g_{\beta\gamma}\nabbar_\alpha\phi\nabbar_\delta\phi-\g_{\alpha\delta}\nabbar_\beta\phi\nabbar_{\gamma}\phi+\g_{\beta\delta}\nabbar_{\alpha}\phi\nabbar_{\gamma}\phi\right)\\
&\,-\frac{4\pi}3 \left(\nabbar^\rho\phi\nabbar_\rho\phi\right)\left(\g_{\alpha\gamma}\g_{\beta\delta}-\g_{\alpha\delta}\g_{\beta\gamma}\right)\,.
\end{align*}}
We define the dual $W^\ast$ of the Weyl tensor as
\[W^\ast_{\alpha\beta\gamma\delta}=\frac12\epsilonLC_{\alpha\beta\mu\nu}{W^{\mu\nu}}_{\gamma\delta}\,.\]
The electric and magnetic components of the Weyl tensor\change{, referred to as the Bel-Robinson variables from here on, }are now defined as
\[E(W)_{\alpha\beta}=W_{\alpha\mu\beta\nu}\del_0^{\mu}\del_0^{\nu}=W_{\alpha 0\beta 0},\ B(W)_{\alpha\beta}=W^\ast_{\alpha\mu\beta\nu}\del_0^\mu \del_0^\nu=W^{\ast}_{\alpha 0\beta 0}\,.\]
We note that, conversely, the Weyl tensor can be fully reconstructed from $E$ and $B$ since the following identities hold:
\begin{equation}\label{eq:Weyl-reconstruct}
W_{a0c0}=E_{ac},\quad W_{abc0}=-{\epsilonLC_{ab}}^mB_{mc},\quad W_{abcd}=-\epsilonLC_{abi}\epsilonLC_{cdj}E^{ij}
\end{equation}

By the symmetries of the Weyl tensor as a whole, $E$ and $B$ are symmetric and one has $E_{0\beta}=0=B_{0\beta}$. Hence, $E$ and $B$ are symmetric, tracefree $\Sigma_t$-tangent $(0,2)$-tensors which we shall simply denote as $E_{ij}$ and $B_{ij}$.\\
Further, we define
\begin{equation}
J_{\beta\gamma\delta}:=\nabbar^\alpha W_{\alpha\beta\gamma\delta},\quad J^\ast_{\beta\gamma\delta}:=\nabbar^\alpha W^\ast_{\alpha\beta\gamma\delta}\,.
\end{equation}
By applying the Bianchi identity for $\Riem[\g]$, we gain the explicit expression
\begin{equation}
J_{\beta\gamma\delta}=\frac12\left(\nabbar_\gamma \Ric[\g]_{\delta\beta}-\nabbar_\delta \Ric[\g]_{\gamma\beta}\right)-\frac1{12}\left(\g_{\beta\delta}\nabbar_\gamma R[\g]-\g_{\beta\gamma}\nabbar_\delta R[\g]\right)\,.
\end{equation}
%Recall that, if $\g$ satisfies the Einstein scalar field equation \eqref{eq:ESF1}, we have
%\begin{equation}
%\Ric[\g]_{\mu\nu}=8\pi(T_{\mu\nu}-\frac12 T g_{\mu\nu})=8\pi\nabbar_\mu\phi\nabbar_\nu\phi,\quad R=8\pi\nabbar^\alpha\phi\nabbar_\alpha\phi\,.
%\end{equation}
Using \eqref{eq:ESF1}, we collect:
\begin{align*}
\change{\numberthis\label{eq:J}J_{i0j}%=&\,4\pi\Bigr[\nabbar_0(\nabbar_i\phi\nabbar_j\phi)-\nabbar_j(\nabbar_i\phi\nabbar_0\phi)\Bigr]
%-\frac{2\pi}3\Bigr[\nabbar_0(\nabbar^\alpha\phi\nabbar_\alpha\phi)\Bigr]g_{ij}\\
=}&\change{\,4\pi\Bigr[\nabla_i(\del_0\phi)\nabla_j\phi+{k^l}_i\nabla_l\phi\nabla_j\phi-\del_0\phi\nabla_i\nabla_j\phi\\}
&\change{\,\phantom{4\pi}-k_{ij}(\del_0\phi)^2-n^{-1}\nabla_in\cdot\nabla_j\phi\cdot\del_0\phi\Bigr]-\frac{2\pi}3\Bigr[\del_0(\nabbar^\alpha\phi\nabbar_\alpha\phi)\Bigr]g_{ij}\\}
\numberthis\label{eq:Jast}J^\ast_{i0j}%=&\,\frac12J_i^{\ \mu\nu}\epsilonLC_{\mu\nu 0j}=\,\frac12\epsilonLC_{\mu\nu 0j}\nabla^{[\mu}{R[\g]^{\nu]}}_i+\frac{1}{12}\epsilonLC_{i\nu0j}\nabbar^\nu R[\g]\\
%=&\,4\pi\epsilonLC_{lmj}\nabbar^{[l}\nabbar_i\phi\nabbar^{m]}\phi
%+\frac{2\pi}3\epsilonLC_{imj}\nabla^m\Bigr[\nabbar^\alpha\phi\nabbar_\alpha\phi\Bigr]\\
=&\,4\pi\epsilonLC_{lmj}\left(\nabla^l\nabla_i\phi+{k^l}_i\del_0\phi\right)\nabla^m\phi+\frac{2\pi}3\epsilonLC_{imj}\nabla^m\left(\nabbar^\alpha\phi\nabbar_\alpha\phi\right)
\end{align*}


%Beside these evolution equations, decomposing the Weyl tensor also yields the following new constraint equations for the Weyl tensor:
\change{\noindent Note that expressions containing $\nabbar^\alpha\phi\nabbar_\alpha\phi$ can be ignored throughout our analysis since they are either pure trace or antisymmetric and thus will cancel in inner products with $E$, $B$, $\hat{k}$ and their rescaled analogues.\\}
\noindent The Bel-Robinson variables then behave as follows:

\begin{lemma}[Constraint and evolution equations for Bel-Robinson variables]\label{lem:EEqBR} If $(\g,\phi)$ is a classical solution to the Einstein scalar-field system \eqref{eq:ESF1}-\eqref{eq:EMT} in CMC gauge (see \eqref{eq:CMC}), $E$ and $B$ satisfy the following constraint equations:
\begin{subequations}
\begin{align}
E&\,=\Ric[g]+\frac29\tau^2g+\frac\tau3\hat{k}-\hat{k}\odot_g\hat{k}-4\pi(\nabla\phi\otimes\nabla\phi)-\left(\frac{8\pi}3\left\lvert \del_0\phi\right\rvert^2+\frac{4\pi}3\lvert\nabla\phi\rvert_g^2\right)g\label{eq:constr-E}\\
\label{eq:constr-B} B&\,=-\curl\hat{k}
\end{align}
\end{subequations}
Further, they satisfy the following evolution equations:
\begin{subequations}
\begin{align}
\del_tE_{ij}&=n\curl{B}_{ij}-(\nabla n\wedge B)_{ij}-\frac52n\left(E\times k\right)_{ij}-\frac23n(E\cdot k)g_{ij}-\frac{\tau}2n\cdot E_{ij}-\change{\frac{n}2\left(J_{i0j}+J_{j0i}\right)} \label{eq:EEqE}\\
\del_tB_{ij}&=-n\curl{E}_{ij}+(\nabla n\wedge E)_{ij}-\frac52n\left(B\times k\right)_{ij}-\frac23n(B\cdot k)g_{ij}-\frac{\tau}2n\cdot B_{ij}-\change{\frac{n}2\left(J_{i0j}^\ast+J_{j0i}^\ast\right)}\label{eq:EEqB}
\end{align}
\end{subequations}
%In particular, $E$ and $B$ are uniquely determined on any hypersurface $\Sigma_t$ by a solution $(g,\hat{k},n,\nabla\phi,\del_t\phi)$ to the equations in Proposition \ref{prop:eq} evaluated on $\Sigma_t$, and $E$ and $B$ satisfy the evolution equations \eqref{eq:EEqE}-\eqref{eq:EEqB} wherever said solution exists.
\end{lemma}
\begin{proof}
For \eqref{eq:EEqE}-\eqref{eq:EEqB}, we refer to \cite[(3.11a)-(3.11b)]{AM03}.\change{\footnote{\change{Note that there is a minor error in the statement in \cite{AM03}, where the authors forget to symmetrize the $J$-tensors when applying the symmetrization to (3.14) of that work. This error seems to also occur in \cite[(7.2.2jk)]{ChrKl93}.}} } Equations \eqref{eq:constr-E}-\eqref{eq:constr-B} follow as in \cite[(3.63a)-(3.63b)]{Wang19} from contracting the Gauss-Codazzi constraints.
%, we start out with the Gauss equation (see \cite[Chapter 2.3, (2.23)]{Rendall08}):
%\begin{equation}\label{eq:Gauss-constr}
%\begin{aligned}
%\Ric[\g]_{ab}=&\,\Ric[g]_{ab}+\tau k_{ab}-k_{ac}{k^c}_b-\Ric[\g]_{\rho a\sigma b}{\del_0}^\rho {\del_0}^\sigma\\
%=&\,\Ric[g]_{ab}+\frac29\tau^2g_{ab}+\frac{\tau}3\hat{k}_{ab}-\hat{k}_{ab}\hat{k}^c_b-\Ric[\g]_{\rho a\sigma b}\del_0^\rho\del_0^\sigma
%\end{aligned}
%\end{equation}
%Regarding the final term, one has
%\begin{equation*}
%\Ric[\g]_{\rho a\sigma b}\del_0^\rho \del_0^\sigma=E_{ab}+P[\g]_{\rho a\sigma b}\del_0^\rho \del_0^\sigma
%\end{equation*}
%with
%\begin{align*}
%&\,P[\g]_{\rho a\sigma b}\del_0^\rho \del_0^\sigma\\
%=&\,\frac12\left(\g_{\rho\sigma}\Ric[\g]_{ab}-\g_{\rho b}\Ric[\g]_{\sigma a}+\g_{ab}\Ric[\g]_{\rho\sigma}-\g_{a\sigma}\Ric[\g]_{\rho b}\right)\del_0^\rho \del_0^\sigma-\frac16 R[\g](\g_{\rho\sigma}\g_{ab}-\g_{\rho b}\g_{\sigma a})\del_0^\rho\del_0^\sigma\\
%=&\,-\frac12\Ric[\g]_{ab}+\frac12\Ric[\g]_{00}g_{ab}+\frac16R[\g]\g_{ab}\\
%=&\,-\frac12\cdot 8\pi\nabla_a\phi\nabla_b\phi+4\pi\left\lvert \del_0\phi\right\rvert^2g_{ab}+\frac16\cdot 8\pi\Bigr(\nabbar^\alpha\phi\nabbar_\alpha\phi\Bigr)\g_{ab}\\
%=&\,-4\pi\nabla_a\phi\nabla_b\phi+\frac{8\pi}3\left\lvert \del_0\phi\right\rvert^2g_{ab}+\frac{4\pi}3\lvert\nabla\phi\rvert_g^2g_{ab}\,.
%\end{align*}
%Altogether, \eqref{eq:Gauss-constr} now reads
%\begin{align*}
%E_{ab}=&\,-\Ric[\g]_{ab}+\Ric[g]_{ab}+\frac29\tau^2g_{ab}+\frac\tau3\hat{k}_{ab}-\hat{k}_{ac}\hat{k}^c_b+4\pi\nabla_a\phi\nabla_b\phi\\
%&\,-\left(\frac{8\pi}3\left\lvert \del_0\phi\right\rvert^2+\frac{4\pi}3\lvert\nabla\phi\rvert_g^2\right)g_{ab}\,
%\end{align*}
%which gives \eqref{eq:constr-E} after inserting \eqref{eq:ESF1}.\\
%
%\noindent On the other hand, the Codazzi equation reads
%\[\Riem[\g]_{\sigma abc}\del_0^\sigma=-\nabla_c\hat{k}_{ab}+\nabla_b\hat{k}_{ac}\,.\]
%Thus,
%\begin{align*}
%B_{ij}&=W^\ast_{i\mu j\nu}\del_0^\mu \del_0^\nu=\frac12\epsilonLC_{i\mu\sigma\tau}{W^{\sigma\tau}}_{j\nu}\del_0^\mu \del_0^\nu\\
%&=-\frac12\epsilonLC_{0i\sigma\tau}\g^{\sigma\tilde{\sigma}}\g^{\tau\tilde{\tau}}W_{\tilde{\sigma}\tilde{\tau}j\nu}\del_0^\nu=\frac12\epsilonLC_{irs}\g^{ra}\g^{sb}W_{\nu jab}\del_0^\nu\\
%&=\frac12\epsilonLC_{irs} g^{ra}g^{sb}\left[\nabla_b\hat{k}_{ja}-\nabla_a\hat{k}_{jb}\right]-\underbrace{\frac12\epsilonLC_{irs}\g^{ra}\g^{sb}P_{\nu jab}\del_0^\nu}_{(\ast)}
%\end{align*}
%with
%\begin{align*}
%(\ast)=&\,-\frac12\epsilonLC_{i\mu\sigma\tau}{P^{\sigma\tau}}_{j\nu}\del_0^\mu \del_0^\nu\\
%=&\,-\frac12\epsilonLC_{i0\sigma\tau}\left[\frac12\left(\I^\sigma_j\Ric[\g]^\tau_0-\I^\sigma_0\Ric[\g]^\tau_j+\I^\tau_j\Ric[\g]^\sigma_0-\I^\tau_0\Ric[\g]^\sigma_j\right)-\frac{R[\g]}6\left(\I^\sigma_j\I^\tau_0-\I^\sigma_0\I^\tau_j\right)\right]\\
%=&\,0\,,
%\end{align*}
%and \eqref{eq:constr-B} follows.
\end{proof}

%\begin{remark}[Bel-Robinson variables in FLRW spacetimes]\label{rem:FLRW-BR}
%The electric and magnetic components of the Weyl tensor of the FLRW reference solution to the Einstein scalar field equations (see Lemma \ref{lem:FLRW}) vanish: Since $\hat{k}[\g_{FLRW}]=0$ is satisfied, \eqref{eq:constr-B} immediately implies $B=0$. On the other hand, \eqref{eq:constr-E} now reads
%\begin{align*}
%E_{ij}=&\,\Ric[\gamma]_{ij}+\frac29\tau^2\cdot a^2\gamma_{ij}-\frac{8\pi}3C^2a^{-6}\cdot a^2\gamma_{ij}\\
%=&\,\left(\dot{a}^2-\frac19-\frac{4\pi}3C^2a^{-4}\right)\cdot 2(g_{FLRW})_{ij}\\
%=&\,0\,,
%\end{align*}
%with the last equality due to \eqref{eq:Friedman}.
%\end{remark}
\begin{remark}[Initial data for Bel-Robinson variables]\label{rem:init-BR}
Since the Weyl tensor vanishes over FLRW spacetimes, so do $E(W[\g_{FLRW}])$ and $B(W[\g_{FLRW}])$. Furthermore, note that given initial data $(M,\mathring{g},\mathring{k},\mathring{\pi},\mathring{\psi})$ on $\Sigma_{t_0}$ in the sense \change{discussed in Section \ref{subsubsec:initial-data}}, and defining $\hat{\mathring{k}}=\mathring{k}-\frac{\tau}3\mathring{g}$, we can use \eqref{eq:constr-E} and \eqref{eq:constr-B} to define the following $(0,2)$-tensors:
\begin{subequations}
\begin{align}
\label{eq:init-data-E}\mathring{E}=&\,\Ric[\mathring{g}]+\frac29\tau^2\mathring{g}+\frac{\tau}3\hat{\mathring{k}}-\hat{\mathring{k}}\odot_{\mathring{g}}\hat{\mathring{k}}-4\pi\left(\mathring{\pi}\otimes\mathring{\pi}\right)-\left(\frac{8\pi}3\left\lvert\mathring{\psi}\right\rvert_{\mathring{g}}^2+\frac{4\pi}3\left\lvert\mathring{\pi}\right\rvert_{\mathring{g}}^2\right)\mathring{g}\\
\label{eq:init-data-B}\mathring{B}=&\,-\curl_{\mathring{g}}\hat{\mathring{k}}
\end{align}
\end{subequations}
These are easily seen to be symmetric, and the constraints \eqref{eq:init-Hamilton} and \eqref{eq:init-momentum} on the initial data ensure that they are also tracefree. Hence, any choice of initial data for the Cauchy problem immediately also contains a unique choice of initial data for the Bel-Robinson variables that is consistent with solutions to the Einstein scalar-field equations in CMC gauge. 
\end{remark}

\subsection{Rescaled variables and equations}\label{subsec:REEq}

It will be more convenient to work the rescaled and shifted solution variables to measure their distance from the FLRW reference solution. In this subsection, we introduce the renormalized solution variables and restate the Einstein scalar-field system in terms of these variables.

\begin{definition}[Rescaled variables for Big Bang stability]\label{def:rescaled}
We will consider the rescaled variables
\begin{subequations}
\begin{gather}
\label{eq:rescalingGK}
G_{ij}=a^{-2}g_{ij},\quad (G^{-1})^{ij}=a^2g^{ij},\quad \Sigma_{ij}=a\hat{k}_{ij}\\
\label{eq:rescalingL} N=n-1\\
\label{eq:rescalingBR}
\RE_{ij}=a^4\cdot E_{ij},\quad \RB_{ij}=a^4\cdot B_{ij}\\
\label{eq:rescalingMatter}
\Psi=a^3\del_0\phi-C,
\end{gather}
\end{subequations}
\end{definition}
We note that the scaling of $\RB$ in \eqref{eq:rescalingBR} is \textbf{not} the asymptotic rescaling of $B$ -- in fact, we expect $B$ to have (approximate) leading order $a^{-2}$ as one can see in \eqref{eq:APmidB}. However, keeping this scaling parallel to that of $\RE$ makes the structurally very similar evolution equations significantly easier to deal with. We also do not rescale $N$ asymptotically\change{, unlike \cite{Rodnianski2014, Speck2018}, but note that $N$ converges to $0$ at an order slightly below $a^4$ at low orders (see \eqref{eq:BsN} and \eqref{eq:asymp-lapse}).}
\begin{prop}[The rescaled Einstein scalar-field system]\label{prop:REEq}
The Einstein scalar-field system in CMC gauge as in Proposition \ref{prop:eq} are solved by  $(g,\hat{k},n,\nabla\phi,\del_0\phi)$ if the rescaled variables\linebreak $(G,\Sigma,N,\nabla\phi,\Psi,\RE,\RB)$ as in Definition \ref{def:rescaled} solve the following set of equations:\footnote{\change{We refer to Lemma \ref{lem:BelRobinsonLemmas} for the scalings that occur when switching between tensor field operations like $\wedge$ and $\wedge_G$.}} The \textbf{rescaled metric evolution equations}
\begin{subequations}
\begin{align*}
\del_tG_{ij}=&-2(N+1)a^{-3}\Sigma_{ij}+2N\frac{\dot{a}}{a}G_{ij} \numberthis\label{eq:REEqG}\\
\del_t(G^{-1})^{ij}=&\,2(N+1)a^{-3}(\Sigma^\sharp)^{ij}-2N\frac{\dot{a}}{a}(G^{-1})^{ij}\numberthis\label{eq:REEqG-1}\\
\del_t\Sigma_{ij}=&\,-a\nabla_i\nabla_jN+(N+1)\left[a\Ric[G]_{ij}-2a^{-3}(\Sigma\odot_G\Sigma)_{ij}-8\pi a\nabla_i\phi\nabla_j\phi\right]\numberthis\label{eq:REEqSigma}\\
&+4\pi C^2a^{-3}\cdot NG_{ij}+\frac19\left(3N+2\right)aG_{ij}+N\frac{\dot{a}}{a}\Sigma_{ij}\\
\del_t{(\Sigma^\sharp)^a}_b=&\,\tau N{(\Sigma^\sharp)^a}_b-a\nabla^{\sharp a}\nabla_bN+(N+1)a\left[\left(\Ric[G]^\sharp\right)^a_b+\frac 29\I^a_b\right]\,\numberthis\label{eq:REEqSigmaSharp}\\
&-8\pi (N+1)a\nabla^{\sharp a}\phi\nabla_b\phi+N\left(4\pi C^2a^{-3}+\frac19 a\right)\I^a_b\,,
\end{align*}
\end{subequations}
the \textbf{rescaled Hamiltonian and momentum constraints}
\begin{subequations}
\begin{align}
R[G]+\frac23-a^{-4}\langle\Sigma,\Sigma\rangle_G&=8\pi\left[a^{-4}\Psi^2+2Ca^{-4}\Psi+\lvert\nabla\phi\rvert_G^2\right] \label{eq:REEqHam}\\
\nabla^{\sharp m}\Sigma_{ml}&=-8\pi\nabla_l\phi(\Psi+C) \label{eq:REEqMom}
\end{align}
with their Bel-Robinson analogues
\begin{align*}
\RE_{ij}=&\,a^4\left(\Ric[G]_{ij}+\frac29G_{ij}\right)+\frac{\tau}3a^3\Sigma_{ij}-(\Sigma\odot_G\Sigma)_{ij}-4\pi a^4\nabla_i\phi\nabla_j\phi\numberthis\label{eq:REEqConstrE}\\
&\,-\left[\frac{4\pi}3a^4\lvert\nabla\phi\rvert_G^2+\frac{8\pi}3\Psi^2+\frac{16\pi}3C\Psi\right]G_{ij}\\
\RB_{ij}=&\,-\change{a^2\curl_G\Sigma_{ij}}\numberthis\label{eq:REEqConstrB}\,,
\end{align*}
\end{subequations}
the \textbf{rescaled lapse equation}
\begin{subequations}
\begin{equation}\label{eq:REEqLapse1}
\Lap N=\left(12\pi C^2a^{-4}+\change{\frac13}\right)N+(N+1)a^{-4}\left[\langle\Sigma,\Sigma\rangle_G+8\pi\Psi^2+16\pi C\Psi\right]
\end{equation}
or equivalently
\begin{equation}\label{eq:REEqLapse2}
\Lap N=\left(12\pi C^2a^{-4}+\change{\frac13}\right)N+(N+1)\left[R[G]+\frac23-8\pi\lvert\nabla\phi\rvert_G^2\right]\,,
\end{equation}
\end{subequations}
the \textbf{rescaled evolution equations for the Bel-Robinson \changefinal{variables}}
\begin{subequations}
\begin{align*}
\numberthis\label{eq:REEqE}\del_t\RE_{ij}=&\,\left(3-N\right)\frac{\dot{a}}a\RE_{ij}\change{-a^{-1}(\nabla N\wedge_G\RB)_{ij}+(N+1)a^{-1}\curl_G\RB_{ij}}\\
&\,-(N+1)\left[\change{\frac52a^{-3}(\RE\times_G\Sigma)_{ij}}+\frac23a^{-3}\langle\RE,\Sigma\rangle_GG_{ij}\right]\\
&\,+4\pi(N+1) a^{-3}(\Psi+C)^2\Sigma_{ij}-4\pi(N+1) \dot{a}a^3\nabla_i\phi\nabla_j\phi\change{+4\pi a\nabla_{(i}N\nabla_{j)}\phi(\Psi+C)}\\
&\,-4\pi a(N+1)\left[\nabla_i\Psi\nabla_j\phi+\nabla_j\Psi\nabla_i\phi+\change{(\Sigma^\sharp)^l_{(i}\nabla_{j)}\phi}\nabla_l\phi-(\Psi+C)\nabla_i\nabla_j\phi\right]\\
&\,+(N+1)\left[\frac{2\pi}3a^6\del_0\left(a^{-6}(\Psi+C)^2+a^{-2}\lvert\nabla\phi\rvert_G^2\right)+4\pi\frac{\dot{a}}a(\Psi+C)^2\right]G_{ij}\\[0.5em]
\numberthis\label{eq:REEqB}\del_t\RB_{ij}=&\,\frac{\dot{a}}a\left(3-N\right)\RB_{ij}\change{+a^{-1}(\nabla N\wedge_G\RE)_{ij}-(N+1)a^{-1}\curl_G\RE_{ij}}\\
&\,-(N+1)\left[\change{\frac52a^{-3}(\RB\times_G\Sigma)_{ij}+\frac23a^{-3}\langle\RB,\Sigma\rangle_GG_{ij}}\deletemath{+\curl_G\RE_{ij}}\right]\,\\
&\,-4\pi(N+1)\epsilonLC[G]_{lmj}\left(a^3\nabla^{\sharp l}\nabla_{\change{j}}\phi\nabla^{\sharp m}\phi+a^{-1}{(\Sigma^\sharp)^l}_i\nabla^{\sharp m}\phi(\Psi+C)\right)%\\
%&\deletemath{\,-(N+1)\epsilonLC[G]_{imj}\left(4\pi a^2\nabla^{\sharp m}\phi(\Psi+C)+\frac{2\pi}3a^5\nabla^{\sharp m}\left(a^{-6}(\Psi+C)^2+a^{-2}\lvert\nabla\phi\rvert_G^2\right)\right)\,,}
\end{align*}
\end{subequations}
%with $J_{i0j}$ and $J^\ast_{i0j}$ as in \eqref{eq:J} and \eqref{eq:Jast}, i.e.
%\begin{align*}
%a^3J_{i0j}=&\,4\pi\left[\nabla_i\Psi\nabla_j\phi+(\Psi+C)\nabla_i\nabla_j\phi+4(N+1)^{-1}(\Psi+C)\nabla_{(i}N\nabla_{j)}\phi\right.\\
%&\,\left.-a^5\nabla_0\left(-a^{-6}|\Psi+C|^2+a^{-2}|\nabla\phi|_G^2\right)G_{ij}\right]\numberthis\label{eq:J-REEq}\\
%a^3J_{i0j}^\ast=&\, 8\pi \left[a^2\epsilonLC[G]_{klj}\nabla^{\sharp [k}\nabla_i\phi\nabla^{\sharp l]}\phi+\epsilonLC[G]_{ikj}\nabla^{\sharp k}\left(a^{-2}|\Psi+C|^2+a^2|\nabla\phi|_G^2\right)\right]\,\numberthis\label{eq:Jstar-REEq},
%\end{align*}
and the \textbf{rescaled wave equation}
\begin{subequations}
\begin{equation}\label{eq:REEqWave}
\del_t\Psi=a\langle\nabla N,\nabla\phi\rangle_G+a(N+1)\Lap\phi-3\frac{\dot{a}}{a}N(\Psi+C)
\end{equation}
along with the evolution equation
\begin{equation}\label{eq:REEqNablaPhi}
\del_t\nabla\phi=a^{-3}(N+1)\nabla\Psi+a^{-3}(\Psi+C)\nabla N\,.
\end{equation}
\end{subequations}
Finally, we collect the \textbf{rescaled Ricci evolution equation}
\begin{align*}
\numberthis\label{eq:REEqRic}\del_t\Ric[G]_{ab}%\,&a^{-3}\Lap_G((N+1)\Sigma_{ab})-a^{-3}\nabla^{\sharp d}\nabla_a((N+1)\Sigma_{db})-a^{-3}\nabla^{\sharp d}\nabla_b((N+1)\Sigma_{da})\\
%&+\frac\tau3 (G_{ab}\Lap_G(N+1)+\nabla_a\nabla_b(N+1))\\
=&\,a^{-3}(N+1)(\Lap_G\Sigma_{ab}-\nabla^{\sharp d}\nabla_a\Sigma_{db}-\nabla^{\sharp d}\nabla_b\Sigma_{da})\\
&\,+a^{-3}\nabla^{\sharp d}N(2\nabla_d\Sigma_{ab}-\nabla_a\Sigma_{db}-\nabla_b\Sigma_{da})\\
&\,-a^{-3}\left(\nabla_a N(\div_G\Sigma)_b+\nabla_b(\div_G\Sigma)_a\right)
+\Lap_GN(a^{-3}\Sigma_{ab}+\frac{\tau}3G_{ab})\\
&\,\change{-}a^{-3}\Bigr(\nabla^{\sharp d}\nabla_a N\cdot \Sigma_{db}+\nabla^{\sharp d}\nabla_b N\cdot\Sigma_{da}\Bigr)+\frac\tau3\nabla_a\nabla_bN
\end{align*}
as well as (in a coordinate neighbourhood) the Christoffel evolution equation
%\footnote{A quick remark on why the first line holds: It sufficient to show that this holds in the centre $(t,p)$ of any local coordinate system in $\M$ since the right hand side is obviously a tensor and the left hand side coincides with the Lie derivative of the tensor $(\Gamma[G]-\Gamhat)_{ij}^k$. Note that $\nabla_iT|_{(t,p)}=\del_iT|_{(t,p)}$ holds for any tensor $T$ on $\M$ in these coordinates, so we get
%\begin{equation*}
%\del_t\Gamma^k_{ij}|_{(t,p)}=\left.\left[\frac12(\del_tG^{-1})^{kl}(\del_iG_{lj}+\del_jG_{li}-\del_lG_{ij})+\frac12(G^{-1})^{kl}(\del_i\del_tG_{lj}+\del_j\del_tG_{li}-\del_l\del_tG_{ij})\right]\right\rvert_{(t,p)}\,.
%\end{equation*}
%The statement now follows since $\del_qG_{rs}|_{(t,p)}=0$ in these coordinates causes the first summand to vanish, while we can rewrite the second back into the coordinate invariant form that yields our intended right hand side.}
\begin{align*}
\del_t\Gamma_{ij}^k[G]=&\,\frac12 (G^{-1})^{kl}\left(\nabla_i(\del_tG_{jl})+\nabla_j(\del_tG_{il})-\nabla_l(\del_tG_{ij})\right) \numberthis\label{eq:REEqChr}\\
=&\,-(N+1)a^{-3}\left[\nabla_i{(\Sigma^\sharp)^k}_j+\nabla_j{(\Sigma^\sharp)^k}_i-\nabla^{\sharp k}\Sigma_{ij}\right]\\
&\,-a^{-3}\left[\nabla_iN{(\Sigma^\sharp)^k}_j+\nabla_jN{(\Sigma^\sharp)^k}_i-\nabla^{\sharp k}N\Sigma_{ij}\right]\\
&\,+\frac{\dot{a}}a\left[\nabla_i N\cdot\I^k_j+\nabla_jN\cdot\I^k_i-\nabla^{\sharp k}N\cdot G_{ij}\right]
\end{align*}
\end{prop}

\begin{proof}
For the first identity in \eqref{eq:REEqChr}, we refer to \cite[Lemma 2.27]{Chow06} and insert the evolution equation \eqref{eq:REEqG}. Otherwise, all equations simply follow by computing the effects of rescaling on the equations from Proposition \ref{prop:eq} (respectively the Ricci evolution equation as in \cite[Chapter 2.3, (2.32)]{Rendall08}) as well as the Bel-Robinson evolution equations \eqref{eq:EEqE}-\eqref{eq:EEqB} and constraint equations \eqref{eq:constr-E}-\eqref{eq:constr-B}. Notice that one already finds a solution to the system in Proposition \eqref{prop:eq} with the rescaled variables excluding \eqref{eq:REEqE}, \eqref{eq:REEqB}, \eqref{eq:REEqConstrE} and \eqref{eq:REEqConstrB}. Conversely, all of the rescaled equations are satisfied by a solution to Proposition \ref{prop:eq} at sufficiently high regularity. Hence, solving the full rescaled system is always sufficient to solve the Einstein system in Proposition \ref{prop:eq} and they are equivalent if the initial data is regular enough to ensure sufficiently high regularity of solutions.\\
\end{proof}

\subsection{Commuted equations}\label{subsec:comm-eq}

We collect Laplace-commuted versions of the equations for the rescaled variables in Proposition \ref{prop:REEq} in this subsection. For the sake of brevity, we will not state all possible commutations for every equation, but restrict ourselves to the ones we actually need within the bootstrap argument. We also refer to Appendix \ref{subsec:commutators} for expressions for commutators of spatial differential operators with each other and with $\del_t$.

The terms written down explicity in Lemma \ref{lem:laplace-commuted-eq} are ones that dominate the evolution behaviour or that are the largest higher order terms, both of which require careful treatment within the bootstrap argument. The error terms are broadly categorized into three groups:
\begin{itemize}
\item \enquote{Borderline} terms are terms that critically contribute to the fact that the energies diverge toward the Big Bang singularity. This almost always takes the form of adding energy terms at the same order as the evolved variable scaled by factors of the type $\epsilon a^{-3}$ or $\epsilon a^{-3-c\sqrt{\epsilon}}$, which causes the energies to slightly diverge since $a^{-3}$ is barely not integrable (see \eqref{eq:a-integrals}).
\item \enquote{Junk} terms are terms that are subcritical in the sense that they lead to integrable error terms, or terms that only contain lower order derivatives of the solution variables.
\item \enquote{Top} order terms (which only appear in \eqref{eq:comeq-RE} and \eqref{eq:comeq-RB}) are terms that are junk terms for low order energies, but become borderline terms at top order.
\end{itemize}
All of these error terms are tracked schematically in Section \ref{subsec:error-terms}. Since we will only need $L^2_G$-bounds on these error terms, which are given in Section \ref{subsec:L2-error-est}, we will treat them as notational \enquote{black boxes} outside of the appendix.

\begin{lemma}[Laplace-commuted rescaled equations]\label{lem:laplace-commuted-eq}
Let $L\in2\N,\,L\geq 2$. With error terms as defined in Appendix \ref{subsec:error-terms}, the system in Proposition \ref{prop:REEq} leads to the following Laplace-commuted equations:\\
The \textbf{Laplace-commuted rescaled evolution equations for the second fundamental form}
%metric
%\begin{align*}
%\del_t\Lap^\frac{L}2(G-\gamma)=&\,-[\del_t,\Lap^\frac{L}2]\gamma\\
%=&\,a^{-3}(N+1)\left(\Sigma\ast\nabla^2\Lap^{\frac{L}2-1}\gamma+\nabla\Sigma\ast\nabla^3\Lap^{\frac{L}2-2}\gamma+\nabla\gamma\ast\nabla\Lap^{\frac{L}2-1}\Sigma+\gamma\ast\Lap^{\frac{L}2}\gamma\right)\\
%&\,+a^{-3}(N+1)\gamma\ast\Lap^{\frac{L}2}\Sigma+\frac{\dot{a}}a\Lap^\frac{L}2N(G-\gamma)+\frac{\dot{a}}a\nabla\Lap^{\frac{L}2-1}N\ast\nabla\gamma\\
%&\,+a^{-3}(N+1)\Sigma\ast\gamma\ast\nabla^2\Lap^{\frac{L}2-2}\Ric[G]+\mathfrak{J}([\del_t,\Lap^\frac{L}2]\gamma)\\[2em]
%\end{align*}
\begin{equation}\label{eq:comeq-Sigma}
\del_t\Lap^\frac{L}2\Sigma=-a\nabla^2\Lap^\frac{L}2N+a(N+1)\Lap^\frac{L}2\Ric[G]+\mathfrak{S}_{L,Border}+\mathfrak{S}_{L,Junk}\,,
\end{equation}
the \textbf{Laplace-commuted rescaled momentum constraint equations}
\begin{subequations}
\begin{equation}
\div_G\Lap^\frac{L}2\Sigma=-8\pi (\Psi+C)\left[\nabla\Lap^{\frac{L}2}\phi+\Lap^{\frac{L}2-1}\Ric[G]\ast\nabla\phi\right]+\nabla\Lap^{\frac{L}2-1}\Ric[G]\ast\Sigma+\mathfrak{M}_{L,Junk}\label{eq:comeq-mom-div}
\end{equation}
and
\begin{equation}
\change{\curl_G\Lap^{\frac{L}2}\Sigma=-a^{-2}\Lap^{\frac{L}2}\RB+\epsilonLC[G]\ast\nabla\Lap^{\frac{L}2-1}\Ric[G]\ast\Sigma+\tilde{\mathfrak{M}}_{L,Junk}}\numberthis\,,\label{eq:comeq-mom-curl}
\end{equation}
the \textbf{Laplace-commuted rescaled Hamiltonian constraint equations}
\begin{align*}
\numberthis\label{eq:comeq-Ham}&\,\Lap^{\frac{L}2}R[G]+a^{-4}\sum_{I_1+I_2=L}\nabla^{I_1}\Sigma\ast\nabla^{I_2}\Sigma\\
=&\,16\pi Ca^{-4}\Lap^{\frac{L}2}\Psi+a^{-4}\sum_{I_1+I_2=L}\left[\nabla^{I_1}\Psi\ast\nabla^{I_2}\Psi+\nabla^{I_1+1}\phi\ast\nabla^{I_2+1}\phi\right]
\end{align*}
and
\begin{equation}
\Lap^{\frac{L}2}\Ric[G]=a^{-4}\Lap^\frac{L}2\RE-\frac{\tau}3a^{-1}\Lap^{\frac{L}2}\Sigma+\mathfrak{H}_{L,Border}+\mathfrak{H}_{L,Junk}\,,\label{eq:comeq-Ham-BR}
\end{equation}
\end{subequations}
\begin{subequations}
the \textbf{Laplace-commuted rescaled lapse equations}
\begin{align}
\label{eq:comeq-lapse}\Lap^{\frac{L}2+1}N=&\,\left(12\pi C^2a^{-4}+\frac13\right)\Lap^{\frac{L}2}N+16\pi Ca^{-4}\cdot\Lap^{\frac{L}2}\Psi+\mathfrak{N}_{L,Border}+\mathfrak{N}_{L,Junk}\,,\\
\label{eq:comeq-lapse-odd}\nabla\Lap^{\frac{L}2+1}N=&\,\left(12\pi C^2a^{-4}+\frac13\right)\nabla\Lap^{\frac{L}2}N+16\pi Ca^{-4}\cdot\nabla\Lap^{\frac{L}2}\Psi+\mathfrak{N}_{L+1,Border}+\mathfrak{N}_{L+1,Junk}\,,
\end{align}
\end{subequations}
the \textbf{Laplace-commuted rescaled Bel-Robinson evolution equations}
\begin{subequations}
\begin{align*}
\del_t\Lap^{\frac{L}2}\RE=&\,\frac{\dot{a}}a\left(3-N\right)\Lap^{\frac{L}2}\RE\change{+(N+1)a^{-1}\curl_G\Lap^{\frac{L}2}\RB-a^{-1}\nabla\Lap^\frac{L}2 N\wedge_G \RB}\numberthis\label{eq:comeq-RE}\\
&\,+4\pi C^2a^{-3}(N+1)\Lap^{\frac{L}2}\Sigma\change{+4\pi a(\Psi+C)\nabla\Lap^{\frac{L}2}N\otimes\nabla\phi}\\
&\,+4\pi a(\Psi+C)(N+1)\nabla^2\Lap^\frac{L}2\phi-8\pi a(N+1)\left(\nabla\phi\otimes\nabla\Lap^\frac{L}2\Psi\right)\\
&\,+\mathfrak{E}_{L,Border}+\mathfrak{E}_{L,top}+\mathfrak{E}_{L,Junk}\\
\del_t\Lap^{\frac{L}2}\RB=&\,\frac{\dot{a}}a(3-N)\Lap^{\frac{L}2}\RB\change{-(N+1)a^{-1}\curl_G\Lap^{\frac{L}2}\RE+a^{-1}\nabla\Lap^{\frac{L}2}N\wedge_G\RE} \numberthis\label{eq:comeq-RB}\\
&\,+a^3\epsilonLC[G]\ast\nabla^2\Lap^{\frac{L}2}\phi\ast\nabla\phi+\mathfrak{B}_{L,Border}+\mathfrak{B}_{L,top}+\mathfrak{B}_{L,Junk}\,,
\end{align*}
\end{subequations}
the \textbf{Laplace-commuted rescaled matter evolution equations}
\begin{subequations}
\begin{align}
\del_t\Lap^\frac{L}2\Psi=&\,a\langle\nabla\Lap^\frac{L}2N,\nabla\phi\rangle_G+a(N+1)\Lap^{\frac{L}2+1}\phi-3C\frac{\dot{a}}a\Lap^{\frac{L}2}N+\mathfrak{P}_{L,Border}+\mathfrak{P}_{L,Junk} \label{eq:comeq-Psi-even}\\
%%%%%%%%%%%%%%%%%%%%%%%%
\del_t\nabla\Lap^\frac{L}2\phi=&\,a^{-3}(N+1)\nabla\Lap^{\frac{L}2}\Psi+Ca^{-3}\nabla\Lap^\frac{L}2 N+\mathfrak{Q}_{L,Border}+\mathfrak{Q}_{L,Junk} \label{eq:comeq-nablaphi-even}
\end{align}
as well as (also allowing $L=0$ for \eqref{eq:comeq-nablaphi-odd})
\begin{align*}
\del_t\nabla_l\Lap^{\frac{L}2}\Psi=&\,a\nabla_l\nabla^{\sharp j}\Lap^{\frac{L}2}N\nabla_j\phi+a(N+1)\nabla_l\Lap^{\frac{L}2+1}\phi-3C\frac{\dot{a}}a\nabla_l\Lap^{\frac{L}2}N \numberthis\label{eq:comeq-Psi-odd}\\
&\,+\left(\mathfrak{P}_{L+1,Border}\right)_l+\left(\mathfrak{P}_{L+1,Junk}\right)_l \\
%%%%%%%%%%%%%%%%%%%%%%%%%%%
\numberthis\label{eq:comeq-nablaphi-odd}\del_t\Lap^{\frac{L}2+1}\phi=&\,a^{-3}(N+1)\Lap^{\frac{L}2+1}\Psi+Ca^{-3}\Lap^{\frac{L}2+1}N +\mathfrak{Q}_{L+1,Border}+\mathfrak{Q}_{L+1,Junk}
\end{align*}
\end{subequations}
and the \textbf{Laplace-commuted rescaled Ricci evolution equations}
\begin{subequations}
\begin{align*}
\numberthis\label{eq:comeq-Ric-even}\del_t\Lap^\frac{L}2\Ric[G]_{ij}=&\,a^{-3}\left(\Lap^{\frac{L}2+1}\Sigma_{ij}-2\nabla^{\sharp m}\nabla_{(i}\Lap^{\frac{L}2}\Sigma_{j)m}\right)\\
&\,-\frac{\dot{a}}a\left(\nabla_{i}\nabla_j\Lap^\frac{L}2N+\Lap^{\frac{L}2+1}N\cdot G_{ij}\right)+\left(\mathfrak{R}_{L,Border}\right)_{ij}+\left(\mathfrak{R}_{L,Junk}\right)_{ij}\\
%%%%%%%%%%%%%%%%%%%%
\numberthis\label{eq:comeq-Ric-odd}\del_t\nabla_k\Lap^\frac{L}2\Ric[G]_{ij}=&\,a^{-3}\left(\nabla_k\Lap^{\frac{L}2+1}\Sigma_{ij}-2\nabla_k\nabla^{\sharp m}\nabla_{(i}\Lap^{\frac{L}2}\Sigma_{j)m}\right)\\
&\,-\frac{\dot{a}}a\left(\nabla_k\nabla_i\nabla_j\Lap^\frac{L}2N+\nabla_k\Lap^{\frac{L}2+1}N\cdot G_{ij}\right)\\
&\,+\left(\mathfrak{R}_{L+1,Border}\right)_{ijk}+\left(\mathfrak{R}_{L+1,Junk}\right)_{ijk}\,.
\end{align*}
\end{subequations}
\end{lemma}
%\noindent Note that, despite being phrased with regard to some arbitrary coordinate system for notational convenience, \eqref{eq:comeq-Psi-odd} and \eqref{eq:comeq-Ric-even}-\eqref{eq:comeq-Ric-odd} establish evolution equations for the globally defined tensor fields $\nabla\Lap^{\frac{L}2}\Psi,\,\Lap^\frac{L}2\Ric[G]$ and $\nabla\Lap^\frac{L}2\Ric[G]$.
\begin{proof}
The equations \eqref{eq:comeq-Ham},\eqref{eq:comeq-Ham-BR} and \eqref{eq:comeq-lapse} are obtained by simply applying $\Lap^\frac{L}2$ on both sides of \eqref{eq:REEqHam},\eqref{eq:REEqConstrE} and \eqref{eq:REEqLapse1} respectively. For \eqref{eq:comeq-mom-div} and \eqref{eq:comeq-mom-curl}, we additionally use the commutator formulas \eqref{eq:[Lap-l,div]T} and \eqref{eq:[Lap-l,curl]}, while for the evolution equations, we apply the respective commutator of $\del_t$ and spatial derivatives as collected in Lemma \ref{lem:com-time} and commute Laplacians past $\nabla$ and $\curl$ where needed (see the commutators in Lemma \ref{lem:comm-space}). The commutators with $\del_t$ only cause additional borderline and junk terms that do not substantially influence the behaviour, while the spatial commutators often lead to high order curvature terms, for example the Ricci terms in \eqref{eq:comeq-mom-div}, that need to be more carefully tracked.
\end{proof}

\begin{remark}[Simplified junk term notation]\label{rem:notation-parallel}
For junk terms that occur in an inner product with a tracefree symmetric tensor, any terms that are \change{pure trace }will immediately cancel and thus do not need to be taken into consideration for the following estimates, even if they have to be written down in the junk terms. Hence, we will denote with a superscript \enquote{$\parallel$} (for example $\mathfrak{H}_{L,Junk}^\parallel$) on a schematic error term the expressions that arise when dropping all terms of the form $\zeta\cdot G$ for some scalar function $\zeta$ \change{that }occur in this term's definition (see, for example, \eqref{eq:comeq-Ham-junk}).%\footnote{We could also move directly to the tracefree components of these schematic terms, but this creates further notational baggage in identifying the tracefree parts of all individual terms, so we stick to this simplified notation that is sufficient for our purposes}
\end{remark}


%
%\subsubsection{Auxiliary commuted equations}\label{subsubsec:comm-eq-aux}
%
%While the Laplace-commuted equations in Lemma \ref{lem:laplace-commuted-eq} will be essential in the energy argument due to specific cancellations and identities we need to utilize, their structure is not well suited to derive strong low-order estimates in $C_G$-norms as we obtain in Section \ref{subsec:AP}. The issue here is that we would need some way to derive ellipticity estimates within $C_G$-norms, and consequently explicitly compute what time dependence may be hidden within ellipticity constants. At best, this requires perturbation theory for infinite dimensional operators, and at worst, this may not yield strong estimates if the time dependence can only be controlled by incurring $a^{-c\sigma}$ as a prefactor.
%
%We circumvent this issue by also commuting many of the equations with $\nabla^J$ directly,\footnote{Recall from Section \ref{subsubsec:notation-derivatives} that $\nabla^J$ refers to a $J$-th order covariant derivative, while $\nabla^j$ would refer to the spatial index $j$.}  which will allow us to establish direct control over $C_G^l$-norms. Similarly, these auxiliary equations are a little more convenient when we derive Sobolev norm estimates for variables that aren't part of the core energy formalism, in particular $G-\gamma$.
%
%
%\begin{lemma}[Auxiliary commuted equations]\label{lem:comm-eq-aux} Let $J\in\N$. Then, if the equations in Proposition \ref{prop:REEq} hold, the following schematic evolution equations are (formally) satisfied:\\
%For the metric variables, one has
%\begin{subequations}
%\begin{align*}
%\del_t\nabla^J(G-\gamma)=&\,a^{-3}\sum_{I_N+I_\Sigma=J}\nabla^{I_N}(N+1)\ast\nabla^{I_\Sigma}\Sigma+2\frac{\dot{a}}a\nabla^JN\cdot G+[\del_t,\nabla^J](G-\gamma)\numberthis\label{eq:EEq-aux-G}\,,\\
%\end{align*}\vspace{-3em}
%\begin{align*}
%\numberthis\label{eq:EEq-aux-Sigma}\del_t\nabla^J\Sigma=&\,\frac{\dot{a}}a\sum_{I_N+I_\Sigma=J}\nabla^{I_N}N\ast\nabla^{I_\Sigma}\Sigma-a\nabla^{J+2}N\\
%&\,+a\sum_{I_N+I_\Ric=J}\nabla^{I_N}(N+1)\ast\nabla^{I_\Ric}(\Ric[G]+\frac29G)\\
%&\,+a\sum_{I_N+I_1+I_2=J}\nabla^{I_N}(N+1)\ast\nabla^{I_1+1}\phi\ast\nabla^{I_2+1}\phi+(4\pi C^2a^{-3}+\frac19a)\nabla^JN\cdot G\\
%&\,+[\del_t,\nabla^{J}]\Sigma\,\text{ and}\\
%\numberthis\label{eq:EEq-aux-Ric}\del_t\nabla^J\Ric[G]=&\,a^{-3}\sum_{I_N+I_\Sigma=J+2}\nabla^{I_N}(N+1)\ast\nabla^{I_\Sigma}\Sigma+\frac{\dot{a}}a\nabla^{J+2}(N+1)(1+G)\,+[\del_t,\nabla^J]\Ric[G],
%\end{align*}
%\end{subequations}
%while the Bel-Robinson variables satisfy
%\begin{subequations}
%\begin{align*}
%\numberthis\label{eq:EEq-aux-E}\nabla^J\RE_{ij}=&\,a^4\nabla^J\Ric[G]-\dot{a}a^2\nabla^J\Sigma+\sum_{I_1+I_2=J}\nabla^{I_1}\Sigma\ast\nabla^{I_2}\Sigma\\
%&\,+a^4\sum_{I_1+I_2=J}\nabla^{I_1+1}\phi\ast\nabla^{I_2+1}\phi+\nabla^Jh(a,\nabla\phi,\Psi)\cdot G_{ij}\,,
%\end{align*}
%%%%%%% Old evolution equation version %%%%%%%
%%\begin{align*}
%%\numberthis\label{eq:EEq-aux-E}\del_t\nabla^l\RE=&\,-\tau\left(1-\frac13(N+1)\right)\nabla^l\RE+\frac{\dot{a}}a\sum_{I_N+I_{\RE}=l,I_N>0}\nabla^{I_N}N\ast\nabla^{I_{\RE}}\RE\\
%%&\,+a^{-1}\sum_{I_N+I_{\RB}=l+1}\nabla^{I_N}(N+1)\ast\epsilonLC[G]\ast\nabla^{I_{\RB}}\RB\\
%%&\,+a^{-3}\sum_{I_N+I_\Sigma+I_{\RE}=l}\epsilonLC[G]\ast\epsilonLC[G]\ast\nabla^{I_N}(N+1)\ast\nabla^{I_\Sigma}\Sigma\ast\nabla^{I_{\RE}}\RE+h_{l,\RE,aux}\cdot G\\
%%&\,+a^3\nabla^l\left((N+1)J_{(\cdot)0(\cdot)}\right)+[\del_t,\nabla^{l}]\RE\,,
%%\end{align*}
%with $h(a,\nabla\phi,\Psi)=-\frac{4\pi}3a^4\lvert\nabla\phi\rvert_G^2-\frac{8\pi}3\Psi^2-\frac{16\pi}3C\Psi$, and
%\begin{equation}\label{eq:EEq-aux-B}
%\nabla^J\RB=-a^2\cdot\epsilonLC[G]\ast\nabla^{J+1}\Sigma\,,
%\end{equation}
%\end{subequations}
%and the matter variables satisfy the following:
%\begin{subequations}
%\begin{align}
%\label{eq:EEq-aux-Psi}\del_t\nabla^J\Psi=&\,a\sum_{I_N+I_\phi=J+1}\nabla^{I_N}(N+1)\ast\nabla^{I_\phi+1}\phi-3\frac{\dot{a}}a\sum_{I_N+I_\Psi=J}\nabla^{I_N}N\ast\nabla^{I_\Psi}(\Psi+C)+[\del_t,\nabla^J]\Psi\,,\\
%\label{eq:EEq-aux-phi}\del_t\nabla^J\nabla\phi=&\,a^{-3}\sum_{I_N+I_\Psi=J+1}\nabla^{I_N}(N+1)\ast\nabla^{I_\Psi}(\Psi+C)+[\del_t,\nabla^J]\nabla\phi
%\end{align}
%\end{subequations}
%We also collect the following auxiliary commuted Christoffel evolution equations (on a coordinate neighbourhood):
%\begin{equation}\label{eq:auxeq-Chr}
%\del_t\nabla^J\Gamma[G]^k_{ij}=a^{-3}\sum_{I_N+I_\Sigma=J+1}\nabla^{I_N}(N+1)\ast\nabla^{I_\Sigma}\Sigma+2\frac{\dot{a}}a\nabla^{J}\nabla_{(i}N\cdot\I^k_{j)}+\nabla^{\sharp k}N\cdot G_{ij}+[\del_t,\nabla^J]\Gamma[G]^{k}_{ij}
%\end{equation}
%\end{lemma}
%\begin{proof}
%\eqref{eq:EEq-aux-E} and \eqref{eq:EEq-aux-B} simply follow by applying $\nabla^J$ on both sides of \eqref{eq:REEqConstrE} and \eqref{eq:REEqConstrB}. For the evolution equations, we additionally commute $\del_t$ and $\nabla^J$ (see Lemma \ref{lem:aux-comm} for the explicit form of these commutators).
%\end{proof}
%


\section{Big Bang stability: Norms, energies and bootstrap assumptions}\label{sec:norm-en-bs}

Herein, we state the norms and energies we use to control the solution variables. These will allow us to state our initial data and bootstrap assumptions, and we then provide which improvement we aim to achieve for the latter. Note that we do not provide the coerciveness of our energies immediately (and actually cannot, at least not in a manner useful to our analysis), but will establish Sobolev norm control in the proof of Corollary \ref{cor:H-imp}, the key ingredient being Lemma \ref{lem:Sobolev-norm-equivalence-improved}. Furthermore, we collect a local well-posedness statement from previous work in Section \ref{subsec:lwp}.

\subsection{Norms}\label{subsec:norm}

Recall that $\gamma$ is the hyperbolic spatial reference metric on $M$ introduced in Definition \ref{def:spatial-mf}, which we view as a metric on any foliation hypersurface $\Sigma_t$ (see Section \ref{subsubsec:initial-data}), and $G$ is the rescaled spatial metric arising from the evolution (see Definition \ref{def:rescaled}).

\begin{definition}[Pointwise norms and volume forms] \label{def:pw-stuff}
We denote by $\lvert\cdot\rvert_\gamma$ (resp. $\lvert\cdot\rvert_{G(t,\cdot)}$) the pointwise norm with regard to $\gamma$ (resp. $G(t,\cdot)$). For the sake of simplicity, we define $\lvert\zeta\rvert_\gamma=\lvert\zeta\rvert_{G(t,\cdot)}=\lvert\zeta(t,\cdot)\rvert$ for any scalar function $\zeta$ on $\Sigma_t$. The volume forms on $\Sigma_t$ with respect to $\gamma$ and $G(t,\cdot)$ are written as $\vol{\gamma}$ and $\vol{G(t,\cdot)}$ (or just $\vol{G}$).
\end{definition}

\begin{definition}[$L^2$-norms] Let $\mathfrak{T}$ be a $\Sigma_t$-tangent $(r,s)$-tensor field (for $r,s\geq 0$). Then, we define:
\begin{subequations}
\begin{align}
\|\mathfrak{T}\|_{L^2_\gamma(\Sigma_t)}^2=\|\mathfrak{T}(t,\cdot)\|_{L^2_\gamma(\Sigma_t)}^2&\,:=\int_M \lvert\mathfrak{T}(t,\cdot)\rvert_\gamma^2\,\vol{\gamma},\\
\|\mathfrak{T}\|_{L^2_G(\Sigma_t)}^2=\|\mathfrak{T}(t,\cdot)\|_{L^2_G(\Sigma_t)}^2&\,:=\int_M\lvert\mathfrak{T}(t,\cdot)\rvert_{G(t,\cdot)}^2\,\vol{G(t,\cdot)}\,
\end{align} 
\end{subequations}
\end{definition}

\begin{definition}[Sobolev norms]\label{def:sob-norms} Let $\mathfrak{T}$ be as above and $J\in\N_0$. We define:
\begin{subequations}
\begin{align}
\|\mathfrak{T}\|_{\dot{H}^J_\gamma(\Sigma_t)}^2=\|\mathfrak{T}(t,\cdot)\|_{\dot{H}^J_\gamma}^2=&\,\int_{\Sigma_t}\left\lvert\nabhat^J\mathfrak{T}(t,\cdot)\right\rvert_\gamma^2\,\vol{\gamma}\\
\|\mathfrak{T}\|_{\dot{H}^J_G(\Sigma_t)}^2=\|\mathfrak{T}(t,\cdot)\|_{\dot{H}^J_G}^2=&\,\int_{\Sigma_t}\left\lvert\nabla^J\mathfrak{T}(t,\cdot)\right\rvert_{G(t,\cdot)}^2\,\vol{G(t,\cdot)}\\
\|\mathfrak{T}\|_{H^J_\gamma(\Sigma_t)}^2=\|\mathfrak{T}(t,\cdot)\|_{H^J_\gamma}^2=&\,\sum_{k=0}^J\|\mathfrak{T}\|_{\dot{H}^k_\gamma(\Sigma_t)}^2\\
\|\mathfrak{T}\|_{H^J_G(\Sigma_t)}^2=\|\mathfrak{T}(t,\cdot)\|_{H^J_G}^2=&\,\sum_{k=0}^J\|\mathfrak{T}\|_{\dot{H}^k_G(\Sigma_t)}^2
\end{align}
\end{subequations}
\end{definition}

\begin{definition}[Supremum norms]\label{def:sup-norms} For $\mathfrak{T}$ as above and $J\in\N_0$, we set:
\begin{subequations}
\begin{align}
\|\mathfrak{T}\|_{\dot{C}^J_\gamma(\Sigma_t)}=\|\mathfrak{T}(t,\cdot)\|_{\dot{C}^J_\gamma}=&\,\sup_{p\in \Sigma_t}\left\lvert\nabhat^J\mathfrak{T}(t,\cdot)\right\rvert_\gamma,& \|\mathfrak{T}\|_{C^J_\gamma(\Sigma_t)}=\sum_{k=0}^J\|\mathfrak{T}\|_{\dot{C}^k_\gamma(\Sigma_t)}\\
\|\mathfrak{T}\|_{\dot{C}^J_G(\Sigma_t)}=\|\mathfrak{T}(t,\cdot)\|_{\dot{C}^J_G}=&\,\sup_{p\in \Sigma_t}\left\lvert\nabla^J\mathfrak{T}(t,\cdot)\right\rvert_{G(t,\cdot)},& \|\mathfrak{T}\|_{C^J_G(\Sigma_t)}=\sum_{k=0}^J\|\mathfrak{T}\|_{\dot{C}^k_G(\Sigma_t)}
\end{align}
\end{subequations}
\end{definition}

\begin{remark}[Time dependence is \change{suppressed }in notation]
When the choice of $t$ and $\Sigma_t$ is clear from context, we will often drop time dependences of $G$, $\lvert\cdot\rvert_G$, $\vol{G}$ and $\mathfrak{T}$, suppress the hypersurface $\Sigma_t$ in the Sobolev and supremum norms, and simply write $\int_M$ instead of $\int_{\Sigma_t}$. For example, we write
\[\|\mathfrak{T}\|_{L^2_G}^2=\int_M\lvert\mathfrak{T}\rvert_G^2\,\vol{G}\,.\]
\end{remark}

\begin{definition}[Solution norms]\label{def:sol-norm} We define the following norms to measure the size of near-FLRW solutions:
\begin{subequations}
\begin{align*}
\numberthis\label{eq:def-H}\mathcal{H}=&\,\|\Psi\|_{\change{H^{18}_G}}+\|\nabla\phi\|_{\change{H^{17}_G}}+a^2\|\nabla\phi\|_{\change{\dot{H}^{18}_G}}\\
&\,+\|\Sigma\|_{H^{18}_G}+\|\RE\|_{H^{18}_G}+\|\RB\|_{H^{18}_G}\\
&\,+\|G-\gamma\|_{H^{18}_G}+\|\Ric[G]+\frac29G\|_{H^{16}_G}+a^{-4}\|N\|_{H^{16}_G}+a^{-2}\|N\|_{\dot{H}^{17}_G}+\|N\|_{\dot{H}^{18}_G}\\
\numberthis\label{eq:def-H-top}\change{\mathcal{H}_{top}=}&\,\change{a^2\|\Psi\|_{{\dot{H}^{19}_G}}}\change{+a^4\|\nabla\phi\|_{\change{\dot{H}^{19}_G}}}\change{+a^2\|\Sigma\|_{\dot{H}^{19}_G}}\change{+a^2\|\Ric[G]+\frac29G\|_{\dot{H}^{17}_G}}\,\\
\mathcal{C}=&\,\|\Psi\|_{\change{C^{16}_G}}+\|\nabla\phi\|_{\change{C^{15}_G}}+\|\Sigma\|_{C^{16}_G}+\|\RE\|_{C^{16}_G}+\|\RB\|_{C^{16}_G} \numberthis\label{eq:def-C}\\
&\,+\|G-\gamma\|_{C^{16}_G}%+\|\Gamma-\Gamhat\|_{C^{M-5}_G}
+\|\Ric[G]+\frac29G\|_{C^{14}_G}+a^{-4}\|N\|_{C^{14}_G}+a^{-2}\|N\|_{\dot{C}^{15}_G}+\|N\|_{\dot{C}^{16}_G}\\
\mathcal{C}_\gamma=&\,\|\Psi\|_{\change{C^{16}_\gamma}}+\|\nabla\phi\|_{\change{C^{15}_\gamma}}+\|\Sigma\|_{C^{16}_\gamma}+\|\RE\|_{C^{16}_\gamma}+\|\RB\|_{C^{16}_\gamma} \numberthis\label{eq:def-C-gamma}\\
&\,+\|G-\gamma\|_{C^{16}_\gamma}%+\|\Gamma-\Gamhat\|_{C^{M-5}_G}
+\|\Ric[G]+\frac29G\|_{C^{14}_\gamma}+a^{-4}\|N\|_{C^{14}_\gamma}+a^{-2}\|N\|_{\dot{C}^{15}_\gamma}+\|N\|_{\dot{C}^{16}_\gamma}
%\mathcal{H}=&\,\|\Psi\|_{H^{M-1}_G}+\|\nabla\phi\|_{H^{M-2}_G}+a^2\|\nabla\phi\|_{\dot{H}^{M-1}_G}+\|\Sigma\|_{H^{M-2}_G}+\|\RE\|_{H^{M-2}_G}+\|\RB\|_{H^{M-2}_G} \numberthis\label{eq:def-H}\\
%&\,+\|G-\gamma\|_{H^{M-2}_G}+\|\Ric[G]+2G\|_{H^{M-4}_G}+a^{-4}\|N\|_{H^{M-4}_G}+a^{-2}\|N\|_{\dot{H}^{M-3}_G}+\|N\|_{\dot{H}^{M-2}_G}\\
%\mathcal{C}=&\,\|\Psi\|_{C^{M-3}_G}+\|\nabla\phi\|_{C^{M-4}_G}+\|\Sigma\|_{C^{M-4}_G}+\|\RE\|_{C^{M-4}_G}+\|\RB\|_{C^{M-4}_G} \numberthis\label{eq:def-C}\\
%&\,+\|G-\gamma\|_{C^{M-4}_G}%+\|\Gamma-\Gamhat\|_{C^{M-5}_G}
%+\|\Ric[G]+2G\|_{C^{M-6}_G}+a^{-4}\|N\|_{C^{M-6}_G}+a^{-2}\|N\|_{\dot{C}^{M-5}_G}+\|N\|_{\dot{C}^{M-4}_G}\\
%\mathcal{C}_\gamma=&\,\|\Psi\|_{C^{M-3}_\gamma}+\|\nabla\phi\|_{C^{M-4}_\gamma}+\|\Sigma\|_{C^{M-4}_\gamma}+\|\RE\|_{C^{M-4}_\gamma}+\|\RB\|_{C^{M-4}_\gamma} \numberthis\label{eq:def-C-gamma}\\
%&\,+\|G-\gamma\|_{C^{M-4}_\gamma}%+\|\Gamma-\Gamhat\|_{C^{M-5}_G}
%+\|\Ric[G]+2G\|_{C^{M-6}_\gamma}+a^{-4}\|N\|_{C^{M-6}_\gamma}+a^{-2}\|N\|_{\dot{C}^{M-5}_\gamma}+\|N\|_{\dot{C}^{M-4}_\gamma}
\end{align*}
\end{subequations}
\end{definition}

\begin{remark}[Choice of metric and controlling Christoffel symbols]\label{rem:relation-metric-Chr}
We could equivalently also phrase $\mathcal{H}$ in terms of $\gamma$-norms, or predominately use $\mathcal{C}_\gamma$ instead of $\mathcal{C}$, since we include the norms on $G-\gamma$ and $\Ric[G]+\frac29G=(\Ric[G]+\frac29\gamma)+\frac29(G-\gamma)$. We will demonstrate in Lemma \ref{lem:G-gamma-norm-switch} how $H_G$ and $C_\gamma$ norms can be used to control $H_\gamma$ and $C_G$ norms. We will also indicate how the initial data and bootstrap assumptions for $\mathcal{C}_\gamma$ and $\mathcal{C}$ are equivalent in Remarks \ref{rem:init-Cgamma} and \ref{rem:Bs-Cgamma}. The main reason for this is that, by successively replacing local coordinates in the expressions of $\Gamma-\Gamhat$ by $\nabhat$, one has
\begin{equation}\label{eq:Christoffel-norm-handwaving}
\|\Gamma-\Gamhat\|_{C^{l-1}_\gamma\change{(M)}}\lesssim P_l(\|G-\gamma\|_{C^l_\gamma(M)},\|G^{-1}-\gamma^{-1}\|_{C^l_\gamma(M)})\,.
\end{equation}

We choose to work predominately with norms in terms of the rescaled metric since quantities appearing in the Einstein system are naturally contracted by $G$, not $\gamma$, and we commute with differential operators associated with $G$.%, so it is significantly more convenient to work completely with the resclaed metric whereever possible.\\ 

%To understand why measuring with $G$ or $\gamma$ is essentially equivalent, we need to control how Christoffel symbols differ: Notice that, on any coordinate neighbourhood $U\subseteq\Sigma_t$, we can rewrite the standard local coordinate formulas for $\Gamma-\Gamhat$ by replacing coordinate derivatives with $\nabhat$ and $\Gamhat$ terms and estimate all purely $\gamma$-dependent terms in $C^0_\gamma$ by a suitable constant. Hence, we get
%\[\left\lvert\Gamma-\Gamhat\right\rvert_\gamma\lesssim\lvert G^{-1}\rvert_\gamma\left(\lvert\nabhat (G-\gamma)\rvert_\gamma+\lvert G-\gamma\rvert_\gamma\right)+\lvert G^{-1}-\gamma^{-1}\rvert_\gamma\,.\]
%Subsequently, we get iterating this process that
%\begin{equation}\label{eq:Christoffel-norm-handwaving}
%\|\Gamma-\Gamhat\|_{C^{l-1}_\gamma(U)}\lesssim P_l(\|G-\gamma\|_{C^l_\gamma(M)},\|G^{-1}-\gamma^{-1}\|_{C^l_\gamma(M)})
%\end{equation}
%holds for some multivariate polynomial $P_l$ and any $l\in\N_0$, and the analogous estimates replacing $C_\gamma$ with $C_G$.
\end{remark}

\begin{remark}[Redundancies in the solution norms]\label{rem:redundancy}
The solution norms $\mathcal{H}$, $\mathcal{C}$ and $\mathcal{C}_\gamma$ aren't \enquote{optimal} in the sense that controlling the norms of $\Psi,\,\nabla\phi,\Sigma$ and $G-\gamma$ is entirely sufficient to gain the claimed control (up to constant) on $N$ via the lapse equation, $\RE$ and $\RB$ via to the constraint equations and $\Ric[G]+\frac29G$ via local coordinates. We choose to include all variables in the norms and subsequent assumptions mainly for the sake of convenience.
%The solution norms $\mathcal{H}$, $\mathcal{C}$ and $\mathcal{C}_\gamma$ aren't \enquote{optimal} in the sense that many norms could be dropped entirely -- to be more precise, one could adjust these norms to only include $\Psi,\,\nabla\phi,\,\Sigma$ and $G-\gamma$: Elliptic estimates will allow the lapse norms to be controlled by curvature and scalar field quantities, the curvature norms could be controlled by the metric up to Christoffel symbol errors as above, and the constraint equations \eqref{eq:REEqConstrE}-\eqref{eq:REEqConstrB} make all Bel-Robinson norms redundant if one additionally considers $a^2\|\Sigma\|_{\dot{H}^{19}_G}$ in $\mathcal{H}$ and similarly in the other norms. The latter could also be controlled by the estimates derived in Section \ref{sec:bs-imp} and could thus be included without issue. We choose the expressions above mainly for convenience.
\end{remark}


\subsection{Energies}\label{subsec:en}

The fundamental objects used to control the solution variables are the energies that take the following form:

\begin{definition}[Energies]\label{def:energies}
Let $l\in\N_0$. We define:
\begin{subequations}
\begin{align*}
\numberthis\changefinal{\E^{(l)}(\phi,t)=}&\,(-1)^l\int_M\Psi\Lap^l\Psi-a^4\phi\Lap^{l+1}\phi\,\vol{G}\\
\label{eq:energydef-ibp}=&\,\begin{cases}
\int_M\lvert\Lap^\frac{l}2\Psi\rvert^2+a^4\lvert\nabla\Lap^\frac{l}2\phi\rvert_G^2\,\vol{G} & l\text{ even}\\
\int_M\lvert\nabla\Lap^\frac{l-1}2\Psi\rvert_G^2+a^4\lvert\Lap^\frac{l+1}2\phi\rvert_G^2\,\vol{G} & l\text{ odd}
\end{cases}\\
\numberthis\E^{(l)}(W,t)=&\,(-1)^l\int_M \langle\RE,\Lap^l \RE\rangle_G +\langle\RB,\Lap^l\RB\rangle_G\,\vol{G}\\
\numberthis\E^{(l)}(\Sigma,t)=&\,(-1)^l\int_M\langle\Sigma,\Lap^l\Sigma\rangle_G\,\vol{G}\\
%\E^{(l)}(G,\cdot)&=(-1)^l\int_M\langle G-\gamma,\Lap^l(G-\gamma)\rangle_G\,\vol{G}\\
\numberthis\E^{(l)}(\Ric,\cdot)=&\,(-1)^l\int_M\left\langle \Ric[G]+\frac29G,\Lap^l\left(\Ric[G]+\frac29G\right)\right\rangle_G\,\vol{G}\\
\numberthis\E^{(l)}(N,\cdot)=&\,(-1)^l\int_M\langle N,\Lap^lN\rangle_G\,\vol{G}
\end{align*}
\end{subequations}
Usually, we will use integration by parts to distribute derivatives within the energies as in \eqref{eq:energydef-ibp}. Further, we introduce the notation
\begin{align}
%\underline{\E}^{(l)}(\cdot,t)&=\sup_{\tau\in(t,t_0]}\E^{(l)}(\cdot,\tau)\\
\E^{(\leq l)}&=\sum_{i=0}^l\E^{(i)}
%\underline{\E}^{(\leq l)}&=\sum_{i=0}^l\underline{\E}^{(l)}
\end{align}
for any of the energies above.
\end{definition}
\noindent For any $l\in\N_0$ and any smooth functions $f_1,f_2\in C^\infty(\R_+)$, we have
\begin{equation}\label{eq:ibp-trick}
f_1\cdot f_2\cdot\E^{(2l+1)}\leq \frac{f_1^2}2\E^{(2l)}+\frac{f_2^2}2\E^{(2l+2)}\,.
\end{equation}
Performing the calculation for $\Sigma$ as an example, we have:
\begin{align*}
\E^{(2l+1)}(\Sigma,\cdot)&\,=-\int_M \langle\Lap^l\Sigma,\Lap^{l+1}\Sigma\rangle_G\,\vol{G}\leq \int_M\lvert\Lap^l\Sigma\rvert_G\lvert\Lap^{l+1}\Sigma\rvert_G\,\vol{G}\leq \sqrt{\E^{(2l)}(\Sigma,\cdot)}\sqrt{\E^{(2l+2)}(\Sigma,\cdot)}
\end{align*}
Now, \eqref{eq:ibp-trick} then follows from the Young inequality. As a consequence, we also have
\begin{equation}\label{eq:drop-odd-en}
\E^{(\leq 2l)}\lesssim \sum_{m=0}^l\E^{(2m)}\,,\,\E^{(\leq 2l+1)}\lesssim\sum_{m=0}^{l+1}\E^{(2m)}\,.
\end{equation}
This allows us to largely restrict our analysis to even order energies, outside of how we close the bootstrap argument at top order.


\subsection{Assumptions on the initial data}\label{subsec:init}

With the necessary solution norms and energies now defined, we can now state what we assume near-FLRW initial data to satisfy:

\begin{assumption}[Near-FLRW initial data]\label{ass:init}
For some small enough $\epsilon\in(0,1)$ and the solution norms $\mathcal{H}\change{, \mathcal{H}_{top}}$ and $\mathcal{C}$ as in Definition \ref{def:sol-norm}, we assume the rescaled initial data to be close to that of the FLRW solution in Lemma \ref{lem:FLRW} in the following sense:
%\begin{subequations}
\begin{equation}\label{eq:init-ass}
\mathcal{H}(t_0)+\mathcal{H}_{top}(t_0)+\mathcal{C}(t_0)\lesssim\epsilon^2
\end{equation}
\delete{Additionally [...]} %with energies as in Definition \ref{def:energies}, we make the following assumption for top order data:
%\begin{align*}
%\numberthis\label{eq:init-ass-top}\E^{(20)}(\phi,t_0)+\E^{(20)}(\Sigma,t_0)+\E^{(20)}(W,t_0)+\|\Lap^{10}\phi\|_{L^2_G}^2+\E^{(18)}(\Ric,t_0)&\\
%+a(t_0)^4\E^{(21)}(\phi,t_0)+a(t_0)^4\E^{(21)}(\Sigma,t_0)+a(t_0)^4\E^{(19)}(\Ric,t_0)&\,
%%\E^{(M)}(\phi,t_0)+\E^{(M)}(\Sigma,t_0)+\E^{(M)}(W,t_0)+\|\Lap^{\frac{M}2}\phi\|_{H^{M}_G}^2+\E^{(M-2)}(\Ric,\cdot)&\\
%+a(t_0)^2\E^{(M)}(W,t_0)+a(t_0)^4\E^{(M+1)}(\phi,t_0)+a(t_0)^4\E^{(M+1)}(\Sigma,t_0)+a(t_0)^4\E^{(M-1)}(\Ric,t_0)&\,
%\lesssim\epsilon^4
%\end{align*}
%Finally, we assume the following quantities to be finite:
%\begin{equation}\label{eq:ass-lwp}
%\|G-\gamma\|_{H^{23}_\gamma(\Sigma_{t_0})}+\|\Sigma\|_{H^{22}_\gamma(\Sigma_{t_0})}+\|\Psi\|_{H^{22}_\gamma(\Sigma_{t_0})}+\|\nabla\phi\|_{H^{22}_\gamma(\Sigma_{t_0})}<\infty
%\end{equation}
%\end{subequations}
\end{assumption}
%\eqref{eq:ass-lwp} is only needed to ensure local well-posedness at sufficiently high regularity (see Lemma \ref{lem:lwp}), hence we break from convention and phrase it in terms of $H_\gamma$-norms to relate it to previous work more easily. 
\change{The assumptions }on $\mathcal{H}\change{+\mathcal{H}_{top}}$ also imply the following:
\begin{align*}
\numberthis\label{eq:init-ass-en}
\E^{(\leq \change{18})}(\phi,t_0)+\E^{(\leq 18)}(\Sigma,t_0)+\E^{(\leq 18)}(W,t_0)+\E^{(\leq 16)}(\Ric,t_0)&\\
+\|\nabla\phi\|_{H^{18}_G}^2+\E^{(\change{18})}(N,t_0)+a(t_0)^{-4}\E^{(\change{17})}(N,t_0)+a(t_0)^{-8}\E^{(\change{\leq 16})}(N,t_0)\\
\change{+a(t_0)^4\E^{(19)}(\phi,t_0)+a(t_0)^4\E^{(19)}(\Sigma,t_0)+a(t_0)^4\E^{(17)}(\Ric,t_0)}
%\E^{(\leq M-1)}(\phi,t_0)+\E^{(\leq M-2)}(\Sigma,t_0)+\E^{(\leq M-2)}(W,t_0)+\E^{(\leq M-4)}(\Ric,t_0)&\\
%+\|\nabla\phi\|_{H^{M-3}_G}^2+a^{-4}\E^{(\leq M-4)}(N,t_0)+a^{-2}\E^{(M-3)}(N,t_0)+\E^{(M-2)}(N,t_0)
&\,\lesssim\epsilon^4
\end{align*}
%\todo{Note to self: Essentially, I could just assume $E_{top}(t_0)\lesssim\epsilon^4$, since that is precisely what I need, but the epsilon powers would seem a bit weird and random at this stage and we are far from optimal anyhow.}

\begin{remark}[Initial data size in $\mathcal{C}_\gamma(t_0)$]\label{rem:init-Cgamma}
%We briefly sketch how this implies that the initial data is close to the refence solution in $C_\gamma$ as well, the converse and the same interplay between $\mathcal{H}$ and an analogue defined with $H_\gamma$-norms would follow by similar arguments.\\
Notice that by \eqref{eq:Christoffel-norm-handwaving}, \eqref{eq:init-ass} implies \change{that}%\delete{, on any [...] }%coordinate neighbourhood $U_{t_0}\subseteq \Sigma_{t_0}$,
\begin{subequations}
\begin{equation}\label{eq:init-ass-Chr-C}
\|\Gamma-\Gamhat\|_{C^{15}_G(\change{\Sigma_{t_0}})}\lesssim\epsilon^4\,,
%\|\Gamma-\Gamhat\|_{C^{M-5}_G(U_{t_0})}\lesssim\epsilon^4\,,
\end{equation}
and arguing along similar lines and using $L^2-L^\infty$-estimates, also
\begin{equation}
\|\Gamma-\Gamhat\|_{H^{17}_G(\change{\Sigma_{t_0}})}\lesssim\epsilon^4\,.
%\|\Gamma-\Gamhat\|_{H^{M-3}_G(U_{t_0})}\lesssim\epsilon^4\,.
\end{equation}
\end{subequations}
In particular, since moving from $C_\gamma^l$ to $C_G^l$ only requires control on Christoffel symbols to order $l-1$ for general tensors and $l-2$ for scalar functions, as well as zero order control on $G-\gamma$, it follows from \eqref{eq:init-ass} that
\begin{equation}\label{eq:init-ass-Cgamma}
\mathcal{C}_\gamma(t_0)\lesssim\epsilon^2
\end{equation}
We refer to the proof of Lemma \ref{lem:G-gamma-norm-switch} for a more detailed term analysis and how a similar argument also applies to the Sobolev norms.
\end{remark}

\begin{remark}[Redundancies in the initial data assumptions]
Similar to Remark \ref{rem:redundancy}, one could also reduce the \change{initial data }assumptions in \eqref{eq:init-ass}\change{, especially at top order}. In particular, we highlight that the Bel-Robinson energy can be entirely controlled by the other terms that occur due to the additional scaled $\Sigma$-energy at order \change{$19$}, or vice versa we could drop the latter in favour of the former. This will be reflected in Lemma \ref{lem:en-est-Sigma-top}.
\end{remark}

\begin{remark}[The volume form]\label{eq:rem-vol-form}
Let $\mu_G$ and $\mu_\gamma$ denote the volume elements of $G$ and $\gamma$ respectively. Since the determinant is a smooth map on invertible matrices, the initial data assumptions on $G-\gamma$ also imply
\begin{equation}\label{eq:init-vol-el}
\|\mu_{G}-\mu_{\gamma}\|_{C^0_G(\Sigma_{t_0})}=\|\mu_{G}-\mu_{\gamma}\|_{C^0_\gamma(\Sigma_{t_0})}\lesssim \epsilon^2\,.
\end{equation}
Consequently, we have
\[\|\vol{G}-\vol{\gamma}\|_{C^0_\gamma(\Sigma_{t_0})}=\mu_{\gamma}^{-1}\|\vol{\gamma}\|_{C^0_\gamma(\Sigma_{t_0})}\|\mu_{G(t_0,\cdot)}-\mu_{\gamma}\|_{C^0_\gamma(\Sigma_{t_0})}\lesssim\epsilon^2\]
and, since $\|G^{-1}-\gamma^{-1}\|_{C^0_\gamma(\Sigma_{t_0})}\lesssim\epsilon^2$ also follows by a von Neumann series argument from the initial data assumption on $G-\gamma$,
\begin{equation}\label{eq:init-vol-form}
\|\vol{G}-\vol{\gamma}\|_{C^0_G(\Sigma_{t_0})}\lesssim \epsilon^2\,.
\end{equation}
\end{remark}


\subsection{Local well-posedness and continuation criteria}\label{subsec:lwp}

For everything that follows, we need to establish that the initial data assumptions above also ensure local well-posedness. For the core system, we translate the local well-posedness result for stiff fluids in \cite{Rodnianski2014} to the sub-case of the scalar field system. While statement and proof there are for vanishing spatial sectional curvature and what corresponds to choosing $C=\sqrt{\nicefrac23}$, the arguments for our setting are completely analogous.

\begin{lemma}[Local well-posedness and continuation criteria for the Einstein scalar-field system (Big Bang version), see {\cite[Theorem 14.1]{Rodnianski2014}}]\label{lem:lwp}
Let $N\geq 4$ be an integer and $(M,\mathring{g},\mathring{k},\mathring{\pi},\mathring{\psi})$ be geometric initial data to the Einstein scalar-field system (see \change{Section \ref{subsubsec:initial-data}}) and assume that one has
\[\|\mathring{g}-a(t_0)^2\gamma\|_{H^{N+1}_\gamma(M)}+\|\mathring{k}+\frac{\tau(t_0)}3\cdot a(t_0)^2\gamma\|_{H^N_\gamma(M)}+\|\mathring{\pi}\|_{H^{N}_{\gamma}(M)}+\|\mathring{\psi}-Ca(t_0)^{-3}\|_{H^N_\gamma(M)}<\infty\,\]
as well as, for some sufficiently small $\eta^\prime>0$,
\[\|\mathring{\psi}-Ca(t_0)^{-3}\|_{C^0_\gamma(M)}\leq\eta^\prime\,.\]
Then, the CMC-transported Einstein scalar-field system (respectively the rescaled system) is locally well-posed in the following sense: The initial data $(\mathring{g},\mathring{k},\mathring{\pi},\mathring{\psi})$ launches a unique classical solution $(g,k,n,\nabla\phi,\del_t\phi)$ to \eqref{eq:Hamilton}-\eqref{eq:Momentum}, \eqref{eq:EEqg}-\eqref{eq:EEqk}, \eqref{eq:wave} and \eqref{eq:EEqLapse} on $[t_1,t_0]\times M$ for some $t_1\in(0,t_0)$ that satisfies ${k^l}_l=-3\dot{a}a^{-1}$ and $n>0$, launches a solution to the Einstein scalar-field system and such that the variables enjoy the following regularity:
\begin{align*}
g\in&C^{N-1}_{dt^2+\gamma}([t_1,t_0]\times M)\cap C^0([t_1,t_0],H_{\gamma}^{N+1}(M))\\
k\in&C^{N-2}_{dt^2+\gamma}([t_1,t_0]\times M)\cap C^0([t_1,t_0],H_{\gamma}^{N}(M))\\
\nabla\phi\in&C^{N-2}_{dt^2+\gamma}([t_1,t_0]\times M)\cap C^0([t_1,t_0],H_{\gamma}^{N}(M))\\
\del_t\phi\in&C^{N-2}_{dt^2+\gamma}([t_1,t_0]\times M)\cap C^0([t_1,t_0],H_{\gamma}^{N}(M))\\
n\in&\,C^N_{dt^2+\gamma}([t_1,t_0]\times M)\cap C^0([t_1,t_0],H_{\gamma}^{N+2}(M))
\end{align*}
The rescaled variables $(G,\Sigma,N,\nabla\phi,\Psi)$ enjoy the analogous regularity. If $(\mathfrak{t},t_0]$ is the maximal interval on which the above statements hold, then one either has $\mathfrak{t}=0$ or one of the following blow-up criteria are satisfied:
\begin{enumerate}
\item The smallest eigenvalue of $g(t_m,\cdot)$ converges to 0 for some sequence $(t_m,x_m)\subseteq (\mathfrak{t},t_0]\times M$ with $t_m\downarrow \mathfrak{t}$.
\item $n(t_m,x_m)$ converges to $0$ for some sequence $(t_m,x_m)\subseteq (\mathfrak{t},t_0]\times M$ with $t_m\downarrow \mathfrak{t}$.
\item $\left(\lvert\del_0\phi\rvert^2+\lvert\nabla\phi\rvert_g^2\right)(t_m,x_m)$ converges to $0$ for some sequence $(t_m,x_m)\subseteq (\mathfrak{t},t_0]\times M$ with $t_m\downarrow \mathfrak{t}$.
\item $s\in(\mathfrak{t},t_0]\mapsto\|g\|_{C^2_\gamma(\Sigma_s)}+\|k\|_{C^1_\gamma(\Sigma_s)}+\|n\|_{C^2_\gamma(\Sigma_s)}+\|\del_t\phi\|_{C^1_\gamma(\Sigma_s)}+\|\nabla\phi\|_{C^1_\gamma(\Sigma_s)}$ is unbounded.
\end{enumerate}
\end{lemma}
\begin{proof}[A note on the proof]
Note that the additional initial data requirement in the stiff-fluid setting that the pressure is strictly positive is covered by the smallness assumption on $\mathring{\psi}-Ca(t_0)^2$, since the pressure corresponds to $\lvert\mathring{\psi}\rvert^2+\lvert\mathring{\pi}\rvert_{\mathring{g}}^2$ and the assumptions on $\del_t\phi$ and $\nabla\phi$ ensure that (after embedding) this quantity behaves like $C^2a(t_0)^{-6}+\O{\eta^\prime}$ at $\Sigma_{t_0}$.
\end{proof}
\begin{corollary}[Local well-posedness for the Bel-Robinson variables]\label{cor:lwp-BR}
Under the assumptions of Lemma \ref{lem:lwp}, the Bel-Robinson variables $E$ and $B$ corresponding to the Lorentzian metric $\g=-n^2dt^2+g$ satisfy the equations \eqref{eq:constr-E}-\eqref{eq:constr-B}, are the unique classical solutions to the evolution equations \eqref{eq:EEqE}-\eqref{eq:EEqB}, and satisfy
\[E,B\in C^{N-3}([t_1,t_0]\times M)\cap C([t_1,t_0], H^{N-1}_\gamma(M))\]
\end{corollary}
\begin{proof}
That $E$ and $B$ satisfy the constraint equations, solve the evolution equations and have the stated regularity on the interval of existence is a direct consequence of Lemma \ref{lem:lwp} and the computations in Section \ref{subsec:BR}. Furthermore, with initial data derived from the constraint equations as in Remark \ref{rem:init-BR}, the hyperbolic system \eqref{eq:constr-E}-\eqref{eq:constr-B} launches a unique solution satisfying the regularity above that must then be $(E,B)$.
\end{proof}
\change{For sufficiently regular initial data ($N\geq 21$), it follows that
\[\E^{(\leq 19)}(\phi,\cdot),\E^{(\leq 18)}(W,\cdot),\E^{(\leq 19)}(\Sigma,\cdot),\E^{(\leq 17)}(\Ric,\cdot),\|G-\gamma\|_{H^{18}_G}\in C^{1}([t_1,t_0])\,,\]
and similarly the square of any supremum norm occurring in $\mathcal{C}$ is continuously differentiable on $[t_1,t_0]$. Strictly speaking, we would need to assume this additional regularity on our initial data for the computations in the following sections (especially Section \ref{sec:en-est}) to hold. However, since smooth functions are dense in $H^l(M)$ for any $l\in\N_0$, any bounds on $\mathcal{H}(t)$ and $\mathcal{C}(t)$ that we prove assuming sufficient regularity at $\Sigma_{t_0}$ then immediately extend to data only satisfying the regularity implied by \eqref{eq:init-ass}. }
%Assumption \ref{ass:init} (and in particular \eqref{eq:ass-lwp}) ensures the assumptions of Lemma \ref{lem:lwp} for $N=22$ as long as one chooses $\epsilon$ to be sufficiently small compared to $C$. Hence, Lemma \ref{lem:lwp} and Corollary \ref{cor:lwp-BR} are satisfied at this order, and one has
%\[\E^{(\leq 21)}(\phi,\cdot),\E^{(\leq 20)}(W,\cdot),\E^{(\leq 21)}(\Sigma,\cdot),\|\nabla\phi\|_{H^{20}_G},\E^{(\leq 19)}(\Ric,\cdot)\in C^{1}([t_1,t_0])\]
%by analyzing the respective commuted evolution equations.}%%One could likely lower $N$ by one order given that the cancellations we will utilize in Sections \ref{sec:en-est} and \ref{sec:bs-imp} will show that the highest order terms within the evolution equations cancel when computing the time derivative of the full energies, but we refrain from establishing this beforehand for the sake of simplicity.

\noindent \change{Thus, }from here on out, we will assume \change{without loss of \changefinal{generality }that }all energies \change{and squared norms are }continuously differentiable on the domain of existence, and similarly all variables \change{are }continuously differentiable for the lower order $C_G$-norm improvements in Section \ref{subsec:AP}.
%one has:
%\[(G,\Sigma,\Psi,\nabla\phi,\RE,\RB)\in C^1\left([t_1,t_0],\left(H^{N-1}_G,H^{N-1}_G,H^{N-1}_{\gamma},H^{N-1}_G,H^{N-2}_G,H^{N-2}_G\right)\right)(M)\]
%\end{corollary}
%\begin{proof}
%Notice that the regularity in Lemma \ref{lem:lwp} and Corollary \ref{cor:lwp-BR} also extends to $H_G$-norms since $a$ is smooth and $g,k$ and $n$ are sufficiently regular. \footnote{For a more precise argument as to why we can exchange between these norms, we refer to the analysis in Lemma \ref{lem:G-gamma-norm-switch} which could easily be analogized to this setting.} Now, the statement immediately follows by confirming that the time derivatives exist and have the claimed regularity by applying Lemma \ref{lem:lwp} and Corollary \ref{cor:lwp-BR} to the respective right hand sides of the auxiliary commuted equations in Lemma \ref{lem:comm-eq-aux}.
%\end{proof}
%Assumption \ref{ass:init} (and in particular \eqref{eq:ass-lwp}) ensures the assumptions of Lemma \ref{lem:lwp} for $N=22$ as long as one chooses $\epsilon$ to be sufficiently small compared to $C$. Hence, Lemma \ref{lem:lwp} and Corollary \ref{cor:lwp-BR} are satisfied at this order and all energies that will occur are continuously differentiable\footnote{\todo{A careful argument might be able to lower $N$ by 1 since the only terms that cause issues are the high order linear terms we cancel with divergence identities within the analysis, but the analysis as written only makes sense if everything is differentiable a priori, at least in how it is written?}}, which we will tacitly use from here on out.

%To start off, we refer to %\cite[Theorem 6]{Shao11} or \cite[Section 14.3]{Rodnianski2014}\footnote{While the proof there is sketched for vanishing spatial sectional curvature, the proof for negative sectional curvature is analogous.} for a standard local well-posedness result of the system of equations in Proposition \ref{prop:eq} that also ensures sufficient regularity on some initial time interval for all of our calculations. Note that this statement also contains explicit blow-up criteria. In particular, since the Einstein equations themselves are locally well-posed, this uniquely defines the Bel-Robinson variables locally via the constraint equations \eqref{eq:constr-E}-\eqref{eq:constr-B}. Additionally, $E$ and $B$ must satisfy the symmetric hyperbolic system \eqref{eq:EEqE}-\eqref{eq:EEqB} with initial data arising from the geometric data via the Hamiltonian and momentum constraints (see Remark \ref{rem:init-BR}). Given the local solution $(g,\hat{k},\nabla\phi,\del_t\phi)$ as above, \eqref{eq:EEqE}-\eqref{eq:EEqB} is then also locally well-posed in and of itself without having to consider the constraints, and the local solution arising from this system hence must coincide with \eqref{eq:constr-E} and \eqref{eq:constr-B}.



\subsection{Bootstrap assumption}\label{subsec:bs}

To keep an overview of the entire bootstrap argument, we state all of the assumptions and comprehensively list how we intend to improve them.

\begin{assumption}[Bootstrap assumption]\label{ass:bootstrap}
Fix some $t_{Boot}\in[0,t_0)$. Further, let $c_0>0$, let $\sigma\in(\epsilon^\frac18,1]$ be suitably small such that $c_0\sigma<1$, and $K_0>0$ a suitable constant. For any $t\in(t_{Boot},t_0]$, we assume 
\begin{equation}\label{eq:BsC}
\mathcal{C}(t)\leq K_0\epsilon a(t)^{-c_0\sigma}\,.
\end{equation}
\delete{as well as the energy inequalities [...]}
%\begin{subequations}
%\begin{align}
%\E^{(\leq 19)}(\phi,t)\leq&\,K_0\epsilon^{\frac{5}2}a(t)^{-c_0\sigma} \label{eq:BsEnSF}\\
%\E^{(\leq 18)}(\Sigma,t)\leq&\,K_0\epsilon^\frac94a(t)^{-c_0\sigma} \label{eq:BsEnSigma}\\
%\E^{(\leq 18)}(W,t)\leq&\,K_0\epsilon^\frac94a(t)^{-c_0\sigma} \label{eq:BsEnW}\\
%\E^{(\leq 16)}(\Ric,t)\leq&\,K_0\epsilon^2a(t)^{-c_0\sigma} \label{eq:BsEnRic}\\
%\|\nabla\phi\|_{H^{18}_G(\Sigma_t)}^2\leq&\,K_0\epsilon^2a(t)^{-c_0\sigma} \label{eq:BsSobnablaphi}\\
%\E^{(\leq 16)}(N,t)+a(t)^4\E^{(17)}(N,t)+a(t)^8\E^{(18)}(N,t)\leq&\,K_0\epsilon^2a(t)^{8-c_0\sigma} \label{eq:BsEnN}
%%\E^{(\leq M-1)}(\phi,t)\lesssim&\,\epsilon^{\frac{21}8}a(t)^{-c_0\sigma} \label{eq:BsEnSF}\\
%%\E^{(\leq M-2)}(\Sigma,t)\lesssim&\,\epsilon^\frac94a(t)^{-c_0\sigma} \label{eq:BsEnSigma}\\
%%\E^{(\leq M-2)}(W,t)\lesssim&\,\epsilon^\frac52a(t)^{-c_0\sigma} \label{eq:BsEnW}\\
%%\E^{(\leq M-4)}(\Ric,t)\lesssim&\,\epsilon^2a(t)^{-c_0\sigma} \label{eq:BsEnRic}\\
%%\|\nabla\phi\|_{H^{M-3}_G(\Sigma_t)}^2\lesssim&\,\epsilon^2a(t)^{-c_0\sigma} \label{eq:BsSobnablaphi}\\
%%\E^{(\leq M-4)}(N,t)+a(t)^4\E^{(M-3)}(N,t)+a(t)^8\E^{(M-2)}(N,t)\lesssim&\,\epsilon^2a(t)^{8-c_0\sigma} \label{eq:BsEnN}
%\end{align}
%\end{subequations}
%%and the high order inequalities
%%\begin{align*}
%%\E^{(M)}(\phi,t)\lesssim&\,\epsilon^2a(t)^{-c\sigma}\\
%%\E^{(M)}(\Sigma,t)\lesssim&\,\epsilon^2a(t)^{-c\sigma}\\
%%\E^{(M-2)}(\Ric,t)\lesssim&\,\epsilon^2a(t)^{-c\sigma}\\
%%%\E^{(M-1)}(\Ric,t)\lesssim&\,\epsilon^\frac{11}8a(t)^{-4-c\sigma}\\
%%\|\nabla\phi\|_{H^{M-1}_G}^2\lesssim&\,\epsilon^\frac32 a(t)^{-c\sigma}
%%\end{align*}
\end{assumption}
%\todo{if the argument goes through optimally, we may actually be able to drop the high order inequalities entirely, but I put the ones in that I may use and have used in previous versions}

\begin{remark}
More explicitly, \eqref{eq:BsC} means
\begin{subequations}
\begin{align}
\|\Psi\|_{\change{C^{16}_G}}\leq&\,K_0\epsilon a^{-c_0\sigma} \label{eq:BsPsi}\\
\|\nabla\phi\|_{\change{C^{15}_G}}\leq&\,K_0\epsilon a^{-c_0\sigma} \label{eq:Bsnablaphi}\\
\|\Sigma\|_{C^{16}_G}\leq&\,K_0\epsilon a^{-c_0\sigma} \label{eq:BsSigma}\\
\|\RE\|_{C^{16}_G}\leq&\,K_0\epsilon a^{-c_0\sigma} \label{eq:BsE}\\
\|\RB\|_{C^{16}_G}\leq&\,K_0\epsilon a^{-c_0\sigma} \label{eq:BsB}\\
\|\Ric[G]+\frac29G\|_{C^{14}_G}\leq&\,K_0\epsilon a^{-c_0\sigma} \label{eq:BsRic}\\
\|G-\gamma\|_{C^{16}_G}\leq&\,K_0\epsilon a^{-c_0\sigma} \label{eq:BsG}\\
\|N\|_{C^{14}_G}+a^2\|N\|_{\dot{C}^{15}_G}+a^4\|N\|_{\dot{C}^{16}_G}\leq&\,K_0\epsilon a^{4-c_0\sigma} \label{eq:BsN}\\
%\|\Psi\|_{C^{M-3}_G}\lesssim&\,\epsilon a^{-c_0\sigma} \label{eq:BsPsi}\\
%\|\nabla\phi\|_{C^{M-4}_G}\lesssim&\,\epsilon a^{-c_0\sigma} \label{eq:Bsnablaphi}\\
%\|\Sigma\|_{C^{M-4}_G}\lesssim&\,\epsilon a^{-c_0\sigma} \label{eq:BsSigma}\\
%\|\RE\|_{C^{M-4}_G}\lesssim&\,\epsilon a^{-c_0\sigma} \label{eq:BsE}\\
%\|\RB\|_{C^{M-4}_G}\lesssim&\,\epsilon a^{-c_0\sigma} \label{eq:BsB}\\
%\|\Ric[G]+2G\|_{C^{M-6}_G}\lesssim&\,\epsilon a^{-c_0\sigma} \label{eq:BsRic}\\
%\|G-\gamma\|_{C^{M-6}_G}\lesssim&\,\epsilon a^{-c_0\sigma} \label{eq:BsG}\\
%\|N\|_{C^{M-6}_G}+a^2\|N\|_{\dot{C}^{M-5}_G}+a^4\|N\|_{\dot{C}^{M-4}_G}\lesssim&\,\epsilon a^{4-c_0\sigma} \label{eq:BsN}
\change{\|\Gamma-\Gamhat\|_{C^{15}_G}\leq}&\change{\,K_0\epsilon a^{-c_0\sigma}} \label{eq:BsChr}
%\|\Gamma-\Gamhat\|_{C^{M-5}_G(U)}\lesssim \epsilon a^{-c_0\sigma}
\end{align}
\end{subequations}
\end{remark}

\begin{remark}[Bootstrap assumptions with respect to $\gamma$]\label{rem:Bs-Cgamma}
%Again, we could equivalently define the bootstrap assumptions in $C_\gamma$- and $H_\gamma$- norms and then relate these to energies, but since the core mechanism lies within the energies, working with energies that are adapted to the evolution and $C_G$-norms is more natural. For example, the analogous estimate on $C_\gamma$ follows from \ref{eq:BsC} as follows: 
Note again that we could equivalently make the above bootstrap assumptions with respect to $H_\gamma$- and $C_\gamma$-norms: For example, the assumptions \eqref{eq:BsChr} and \eqref{eq:BsG} imply
\begin{align*}
\|\zeta\|_{C^l_\gamma}\lesssim&\,a^{-c\sigma}\|\zeta\|_{C^l_G}+\|\zeta\|_{C^{\lceil\frac{l-1}2\rceil}_\gamma}\epsilon a^{-c\sigma},\quad
\|\mathfrak{T}\|_{C^l_\gamma}\lesssim a^{-c\sigma}\|\mathfrak{T}\|_{C^l_G}+\|\mathfrak{T}\|_{C^{\lceil\frac{l}2\rceil}_\gamma}\epsilon a^{-c\sigma}
\end{align*}
for any smooth function $\zeta\in C^\infty(\Sigma_t)$, any $\Sigma_t$-tangent tensor $\mathfrak{T}$ and a constant $c>0$. This is essentially a direct consequence of \eqref{eq:Christoffel-norm-handwaving}, and we will prove an improved version of this rigorously in Lemma \ref{lem:G-gamma-norm-switch}. Applying this to each norm in $\mathcal{C}$, we get
\begin{equation}\label{eq:BsCgamma}
\mathcal{C}_\gamma\lesssim\epsilon a^{-c\sigma}
\end{equation} 
for some updated constant $c\geq c_0$.
\end{remark}


\begin{remark}[Strategy for the bootstrap improvement]\label{rem:bs-strategy}

Our goal is to improve the $C$-norm estimate to
%\begin{align*}
%\E^{(\leq 19)}(\phi,t)\leq &\,K_1\epsilon^{\frac{11}4}a(t)^{-c_1\epsilon^\frac18}\\
%\E^{(\leq 18)}(\Sigma,t)\leq &\,K_1\epsilon^\frac52a(t)^{-c_1\epsilon^\frac18}\\
%\E^{(\leq 18)}(W,t)\leq &\,K_1\epsilon^\frac{5}2a(t)^{-c_1\epsilon^\frac18}\\
%\E^{(\leq 16)}(\Ric,t)\leq &\,K_1\epsilon^\frac94a(t)^{-c_1\epsilon^\frac18}\\
%\|\nabla\phi\|_{H^{18}_G}^2\leq &\,K_1\epsilon^\frac52a^{-c_1\epsilon^\frac18}\\
%\E^{(\leq 16)}(N,t)+a^4\E^{(17)}(N,t)+a^8\E^{(18)}(N,t)\leq &\,K_1\epsilon^2a^{8-c_1\epsilon^\frac18}
%\end{align*}
%and additionally
\begin{equation*}
\mathcal{C}\leq K_1\epsilon^\frac98a^{-c_1\epsilon^\frac18}\,,
\end{equation*}
where $c_1,K_1>0$ are positive constants independent of $\sigma$ and $\epsilon$. Notice how this is actually an improvement if we choose $\sigma$ suitably and then choose $\epsilon$ sufficiently small: Any update between $K_0$ and $K_1$ can be balanced out since we gain at least the additional prefactor $\epsilon^\frac18$ in each estimate, which we can then choose to have been suitably small. Similarly, we improve the power of $a$ if we have $\epsilon^{\frac18}\cdot\sigma^{-1}<\frac{c_0}{c_1}$. If we then retroactively choose $\sigma$ large enough compared to $\epsilon$ but small overall -- for example $\sigma=\epsilon^\frac{1}{16}$ -- and then ensure that $\max\{c_0,c_1\}\epsilon^\frac1{16}<1$ as well as $c_1\epsilon^\frac1{16}<c_0$ are satisfied by choosing $\epsilon$ to have been small enough, we have strictly improved the bootstrap assumptions.
%Using this information, we will be able to show
%\[\mathcal{H}(t)\lesssim \epsilon^{\frac98}a^{-c\epsilon^\frac18}\]
%by using \todo{the exchange lemma}, and additionally
%\[\|\Sigma\|_{H^{M-2}_G}+\|N\|_{H^{M-2}_G}\lesssim \epsilon^\frac54 a^{-c\epsilon^\frac18}\]
%which will in turn allow us to deduce
%\[\mathcal{C}(t)\lesssim \epsilon^\frac98 a^{-c\epsilon^\frac18}\]
%by \todo{adapted Sobolev embedding}, where $c$ is again silently updated, closing the bootstrap.
\end{remark}

\begin{remark}[Conventions within the bootstrap argument]
Throughout the rest of the argument, we tacitly assume $t\in(t_{Boot},t_0]$ if not stated otherwise, and we assume $\epsilon$ and $\sigma$ to be sufficiently small. In the proof of Theorem \ref{thm:main}, we will choose $\sigma=\epsilon^\frac1{16}$, but this explicit choice will not be used or needed up to that point. Finally, %since we observed in Remark \ref{rem:bs-strategy} that we can deal with any updates between $K_0$ and $K_1$ as well as between $c_0$ and $c_1$ at the end of the argument, 
we allow $c\geq c_0$ be a constant that we may update from line to line, and will similarly deal with prefactors by \enquote{$\lesssim$}-notation where the constant may change in each line. These updates will always be independent of $\sigma$ and $\epsilon$, but may depend on $t_0$, and the quantities arising from the FLRW reference solution. Hence, we not only assume $c_0\sigma<1$, but $c\sigma<1$ throughout the argument.
\end{remark}

\section{Big Bang stability: A priori estimates}\label{sec:ap}

In this section, we collect strong low order $C_G$-norm estimates that follow as an immediate consequence from the bootstrap assumptions, starting with key estimates at the base level and followed by weaker, but still improved estimates at higher levels. Finally, we collect a differentiation formula for integrals with respect to $\vol{G}$ as well as a Sobolev estimate that lays the groundwork for energy coercivity. In particular, using the strong $C_G$-norm estimates, said estimate proves that moving between energies and norms at most incurs an error involving lower order energies of the controlled variable and curvature energies, scaled by $a^{-c\sqrt{\epsilon}}$.

\subsection{Strong $C^0_G$-estimates}\label{subsec:APlow}

First, we establish a pointwise bound on the lapse that actually holds irrespective of the bootstrap assumptions:

\begin{lemma}[Maximum principle for the lapse]\label{lem:lapse-maxmin} The lapse remains positive and bounded throughout the evolution:
\begin{equation}
n=N+1\in(0,3]
\end{equation}
\end{lemma}
\begin{proof}
Let $t\in\R_+$ be arbitrary and let $n_{min}$ be the minimum of $n$ over $\Sigma_t$ at $(t,x_{min})$. Then, $(\Lap_g n)(t,x_{min})> 0$ holds. If $n_{min}$ were nonpositive, \eqref{eq:EEqLapse} would lead to the following contradiction:
\[0\geq -12\pi C^2a^{-6}-\frac13a^{-2}+n_{min}\left[\frac13a^{-2}+4\pi C^2a^{-6}+\langle\hat{k},\hat{k}\rangle_g+8\pi\lvert\del_0\phi\rvert^2\right]=\Lap_gn(t,x_{min})>0\]
This shows $n>0$, and the upper bound follows analogously.
\end{proof}


The following estimate will be essential in dealing with borderline terms throughout the bootstrap argument:

\begin{lemma}[Strong $C^0_G$ estimates]\label{lem:APzero}The following estimates hold:
\begin{subequations}
\begin{align}
\|\Psi\|_{C^0_G}\lesssim&\,\epsilon\label{eq:APPsi}\\
\|\Sigma\|_{C^0_G}\lesssim&\,\epsilon\label{eq:APSigma}\\
\|\RE\|_{C^0_G}\lesssim&\,\epsilon\label{eq:APE}
\end{align}
\end{subequations}
\end{lemma}
\begin{proof}
\underline{\eqref{eq:APPsi}:} From \eqref{eq:REEqWave}, we obtain the following using Lemma \ref{lem:lapse-maxmin} for $n$, the bootstrap assumptions \eqref{eq:BsN} and \eqref{eq:Bsnablaphi} and that $\dot{a}\simeq a^{-2}$ by \eqref{eq:Friedman}:
\begin{align*}
\lvert\del_t\Psi\rvert%\leq&\,a\lvert\nabla N\rvert_G\lvert\nabla\phi\rvert_G+a\lvert N+1\rvert\lvert\Lap\phi\rvert+3\frac{\dot{a}}a\lvert N\rvert\lvert\Psi\rvert+3C\lvert N\rvert\cdot\frac{\dot{a}}a\\
\lesssim&\,\epsilon a^{5-c\sigma}+\epsilon a^{1-c\sigma}+\epsilon a^{1-c\sigma}\lvert\Psi\rvert+{\epsilon} a^{1-c\sigma}
\end{align*}
After integration, we thus obtain using the initial data assumption \eqref{eq:init-ass}:
\begin{align*}
\lvert\Psi(t)\rvert\lesssim&\,\lvert\Psi(t_0)\rvert+\int_t^{t_0}\epsilon a(s)^{1-c\sigma}\,ds+\int_t^{t_0}{\epsilon} a(s)^{1-c\sigma}\lvert\Psi(s)\rvert\,ds\\
\lesssim&\,\epsilon\left(1+\int_t^{t_0}a(s)^{1-c\sigma}\,ds\right)+\int_t^{t_0}{\epsilon}a(s)^{1-c\sigma}\lvert\Psi(s)\rvert\,ds
\end{align*}
By \eqref{eq:a-integrals}, the integral over $a^{1-c\sigma}$ is bounded since $c\sigma<1$, so the Gronwall lemma now yields \eqref{eq:APPsi}.\\

\underline{\eqref{eq:APSigma}:} Notice that
\begin{equation}\label{eq:deltSigma2}
\del_t\lvert\Sigma\rvert_G^2={(\del_t\Sigma^\sharp)^l}_m{(\Sigma^\sharp)^m}_l+{(\Sigma^\sharp)^l}_m{(\del_t\Sigma^\sharp)^m}_l=2{(\del_t\Sigma^\sharp)^l}_m{(\Sigma^\sharp)^m}_l\leq 2\lvert\del_t\Sigma^\sharp\rvert_G\lvert\Sigma\rvert_G\,.
\end{equation}
Now, we consider \eqref{eq:REEqSigmaSharp} and, using the bootstrap assumptions \eqref{eq:BsN}, \eqref{eq:BsRic} and \eqref{eq:Bsnablaphi}, get:
\begin{align*}
\lvert\del_t\Sigma^\sharp\rvert_G\lesssim&\,\tau\lvert N\rvert\lvert\Sigma^\sharp\rvert_G+\lvert\nabla^\sharp\nabla N\rvert_G a+\lvert N+1\rvert a\left\lvert\Ric[G]^\sharp+\frac29G^\sharp\right\rvert_G+\lvert N+1\rvert a\lvert\nabla^\sharp\phi\nabla\phi\rvert_G\\
&+\sqrt{3}\lvert N\rvert\cdot\left(4\pi C^2a^{-3}+\frac19a\right)\\
%\lesssim&\,{\epsilon}a^{1-c\sigma}\lvert\Sigma\rvert_G+{\epsilon}a^{5-c\sigma}+{\epsilon}a^{1-c\sigma}+\epsilon a^{1-c\sigma}+(\epsilon a^{1-c\sigma}+\epsilon a^{5-c\sigma})\\
\lesssim&\,{\epsilon} a^{1-c\sigma}\lvert\Sigma\rvert_G+{\epsilon}a^{1-c\sigma}
\end{align*}
We can now apply Lemma \ref{lem:weak-ftoc} with $f=\lvert\Sigma\rvert_G^2$, and thus have along with \eqref{eq:init-ass} and \eqref{eq:deltSigma2}:
\[\lvert\Sigma\rvert_G(t)\leq\lvert\Sigma\rvert_G(t_0)+\int_t^{t_0}\lvert\del_t\Sigma^\sharp\rvert_G(s)\,ds\lesssim\epsilon\]

\underline{\eqref{eq:APE}:} Using the constraint equation \eqref{eq:REEqConstrE} and that $\langle G,\RE\rangle_G=\text{tr}_G\RE=0$, one sees
\begin{equation*}
\lvert\RE\rvert_G^2=\left\langle a^4\left(\Ric[G]+\frac29G\right)-\dot{a}a^2\Sigma-\Sigma\odot_G\Sigma-4\pi a^4\nabla\phi\nabla\phi,\RE\right\rangle_G\,.
\end{equation*}
Then, applying the bootstrap assumptions \eqref{eq:BsRic} and \eqref{eq:Bsnablaphi} shows the Ricci and matter terms are bounded by $\epsilon a^{4-c\sigma}\lvert\RE\rvert_G$, and the a priori estimate \eqref{eq:APSigma} along with $\dot{a}a^2\simeq 1$ by \eqref{eq:Friedman} bounds the remaining terms by $\epsilon\lvert\RE\rvert_G$. The statement then follows by dividing by $\lvert\RE\rvert_G$ and taking the supremum.
\end{proof}
\change{Note that, in the proof of \eqref{eq:APSigma}, it was essential that we used
\eqref{eq:REEqSigmaSharp} instead of \eqref{eq:REEqSigma}, since using the latter would incur terms of the type $\lvert\del_tG\rvert_G\lvert\Sigma\rvert_G^2$and $a^{-3}\lvert \Sigma\rvert_G^3$ when computing the time derivative of $\lvert\Sigma\rvert_G^2$, which, at this point, behave like $\epsilon a^{-3-c\sigma}\lvert \Sigma\rvert_G^2$, and thus not yield the sharp estimate (or even an improved estimate) that we will need to control borderline terms.}
%\begin{remark}
%It is crucial that we used \eqref{eq:REEqSigmaSharp} to derive the estimate for $\Sigma$ since else, we would get
%\[\del_t\lvert\Sigma\rvert_G^2=2\langle\del_t\Sigma,\Sigma\rangle_G+2\left(\del_tG^{-1}\right)\ast G^{-1}\ast\Sigma\ast\Sigma\,.\]
%Without any prior improvements, \eqref{eq:REEqG} only implies that the second summand is bounded by $a^{-3-c\sigma}\lvert\Sigma\rvert_G^2$, which we could not utilize for a Gronwall argument since the prefactor diverges too strongly. However, as we will see in the proof of \eqref{eq:APmidG-1}, we need \eqref{eq:APSigma} to improve $\del_tG^{-1}$.
%\end{remark}

%\begin{lemma}\label{lem:APG-Cgamma}
%\begin{align}
%\|G-\gamma\|_{C^0_\gamma}\lesssim&\,\sqrt{\epsilon}a^{-c\sqrt{\epsilon}}\label{eq:APG-gamma}\\
%\|G^{-1}-\gamma^{-1}\|_{C^0_\gamma}\lesssim&\,\sqrt{\epsilon}a^{-c\sqrt{\epsilon}}\label{eq:APGinv-gamma}
%\end{align}
%\end{lemma}
%\begin{proof}
%We rewrite \eqref{eq:REEqG} as
%\[\del_t(G-\gamma)_{ij}=-2(N+1)a^{-3}\Sigma_{ij}+2N\frac{\dot{a}}a\left[(G-\gamma)_{ij}+\gamma_{ij}\right]\]
%and deduce
%\[\frac12\del_t|G-\gamma|_{\gamma}^2=-2(N+1)a^{-3}\langle\Sigma,G-\gamma\rangle_\gamma+2N\frac{\dot{a}}a\left(|G-\gamma|_\gamma^2+\langle \gamma,G-\gamma\rangle_\gamma\right)\]
%Since \eqref{eq:BsN} gives us that $|N|\frac{\dot{a}}a\lesssim {\epsilon} a^{1-c\sigma}$, we obtain using the a priori estimate for $\Sigma$ (see \eqref{eq:APSigma}) that
%\[-\del_t|G-\gamma|_\gamma\lesssim \epsilon a^{-3}+{\epsilon} a^{1-c\sigma}(1+|G-\gamma|_\gamma),\]
%or after integration
%\[|G-\gamma|_\gamma(t)\lesssim |G-\gamma|_{\gamma}(t_0)+\int_t^{t_0}\epsilon a(s)^{-3}\,ds+\int_t^{t_0}{\epsilon}a(s)^{1-c\sigma}|G-\gamma|_\gamma(s)\,ds\]
%Applying a Gronwall argument, the final term only results in a bounded pre-factor, so we now get using the initial data assumption on $G-\gamma$ (see \eqref{eq:init-ass-Cgamma}) that
%\[|G-\gamma|_\gamma(t)\lesssim \epsilon\left(1+\int_t^{t_0}a(s)^{-3}\,ds\right)\lesssim \sqrt{\epsilon}a(t)^{-c\sqrt{\epsilon}}\]
%and we obtain \eqref{eq:APG-gamma}. \eqref{eq:APGinv-gamma} follows analogously.
%\end{proof}
%
%\begin{corollary}\label{cor:C0-exchange-improved} For any $\Sigma_t$-tangent tensor $\mathfrak{T}$, we have
%\begin{subequations}
%\begin{align}
%\|\mathfrak{T}\|_{C^0_G}\lesssim&\,(1+\sqrt{\epsilon}a^{-c\sqrt{\epsilon}})\|\mathfrak{T}\|_{C^0_\gamma}\\
%\|\mathfrak{T}\|_{C^0_\gamma}\lesssim&\,(1+\sqrt{\epsilon}a^{-c\sqrt{\epsilon}})\|\mathfrak{T}\|_{C^0_G}
%\end{align}
%\end{subequations}
%\end{corollary}
%\begin{proof} For a tensor of rank $(r,s)$, we have:
%\begin{align*}
%\left\lvert\lvert\mathfrak{T}\rvert_G^2-\lvert\mathfrak{T}\rvert_\gamma^2\right\rvert=&\,(G^{-1}-\gamma^{-1})\ast G^{-1}\ast\dots G^{-1}\ast\mathfrak{T}\ast\mathfrak{T}\\
%&\,+\dots+\gamma^{-1}\ast\dots\ast\gamma^{-1}\ast (G^{-1}-\gamma^{-1})\ast\mathfrak{T}\ast\mathfrak{T}\\
%\lesssim&\,\|G-\gamma\|_{C^0_G}\left(1+\|\gamma\|_{C^0_G}+\dots+\|\gamma\|_{C^0_G}^{r+s-1}\right)\lvert\mathfrak{T}\rvert_G^2
%\end{align*}
%\end{proof}
%
%
%\begin{corollary}\label{lem:APG-CG}
%\begin{align}
%\|G-\gamma\|_{C^0_G}\lesssim&\,\sqrt{\epsilon}a^{-c\sqrt{\epsilon}}\label{eq:APG}\\
%\|G^{-1}-\gamma^{-1}\|_{C^0_G}\lesssim&\,\sqrt{\epsilon}a^{-c\sqrt{\epsilon}}\label{eq:APGinv}
%\end{align}
%\end{corollary}
%\begin{proof}
%This is an immediate consequence of and Lemma \ref{lem:APG-Cgamma} Corollary \ref{cor:C0-exchange-improved}.
%\end{proof}
%
%\begin{lemma}\label{lem:APnablaphi}
%\begin{align}
%\|\nabla\phi\|_{C^0_\gamma}\lesssim&\,\sqrt{\epsilon}a^{-c\sqrt{\epsilon}}\label{eq:APphi-gamma}\\
%\|\nabla\phi\|_{C^0_G}\lesssim&\,\sqrt{\epsilon}a^{-c\sqrt{\epsilon}}\label{eq:APphi}
%\end{align}
%\end{lemma}
%\begin{proof}
%By Corollary \eqref{cor:C0-exchange-improved}, the second inequality immediately follows from the former, so we just need to prove the first one.\\
%By \eqref{eq:REEqNablaPhi}, we have
%\[|\nabla\phi(t)|_\gamma^2\lesssim |\nabla\phi(t_0)|_\gamma^2+\left|\int_t^{t_0}a(s)^{-3}\left[(\Psi(s,\cdot)+C)\nabla N(s,\cdot)+(N+1)\nabla\Psi(s,\cdot)\right]\right|_\gamma^2\,.\]
%Taking a quick detour, note that for any smooth function $h:\M\rightarrow\R$, one has
%\begin{align*}
%\left|\int_t^{t_0}a(s)^{-3}\nabla h(s,\cdot)\,ds\right|_\gamma^2=&\,\int_t^{t_0}\int_t^{t_0}a(s)^{-3}a(\tau)^{-3}\langle h(s,\cdot),h(\tau,\cdot)\rangle_\gamma\,dsd\tau\\
%\leq&\,\int_t^{t_0}\int_t^{t_0}a(s)^{-3}|h(s,\cdot)|_\gamma\cdot a(\tau)^{-3}|h(\tau,\cdot)|_\gamma\,dsd\tau\\
%\leq&\,\left(\int_t^{t_0}a(s)^{-3}|h(s,\cdot)|_\gamma ds\right)^2\\
%\leq&\,\int_t^{t_0}a(s)^{-3}\,ds\cdot \int_t^{t_0}a(s)^{-3}|\nabla h(s,\cdot)|_\gamma^2\,ds\\
%\lesssim &\,\frac1{\sqrt{\epsilon}}a^{-c\sqrt{\epsilon}}\int_t^{t_0}a(s)^{-3}|\nabla h(s,\cdot)|_\gamma^2\,ds
%\end{align*}
%Hence, we now obtain (using smallness of the initial data)
%\[|\nabla\phi(t)|_\gamma^2\lesssim \epsilon^4+\frac1{\sqrt{\epsilon}}a^{-c\sqrt{\epsilon}}\int_t^{t_0}a^{-3}|(\Psi(s,\cdot)+C)\nabla N(s,\cdot)+(N+1)\nabla\Psi(s,\cdot)|_\gamma^2\,ds\]
%Regarding the integral, we compute using \eqref{eq:APPsi-gamma}, Lemma \ref{lem:lapse-maxmin} for $N$, the bootstrap assumption \eqref{eq:BsN} for $\nabla N$ (and switching between pointwise norms with \eqref{eq:APG}):
%\begin{align*}
%|(\Psi(s,\cdot)+C)\nabla N(s,\cdot)+(N+1)\nabla\Psi(s,\cdot)|_\gamma^2&\lesssim (N+1)^2|\nabla\Psi|_\gamma^2+(|\Psi|+C)^2|\nabla N|_G^2\cdot a^{-c\sigma}\\
%&\lesssim |\nabla\Psi|_\gamma^2+|\nabla N|_G^2\cdot a^{-c\sigma}\\
%&\lesssim \epsilon^2+\epsilon^2 a^{8-c\sigma}\lesssim \epsilon^2\\
%\end{align*}
%Altogether, this means
%\begin{align*}
%|\nabla\phi(t,\cdot)|_\gamma^2%\lesssim&\,\epsilon^4+\frac1{\sqrt{\epsilon}}a(t)^{-c\sqrt{\epsilon}}\int_t^{t_0}a^{-3}\left[|\nabla\Psi|_G^2+|\nabla N|_G^2\right]\,ds\numberthis\label{eq:nabla-phi-improve}\\
%\lesssim &\,\epsilon^4+\frac1{\sqrt{\epsilon}}a(t)^{-c\sqrt{\epsilon}}\left(\int_t^{t_0}\epsilon^2a(s)^{-3}\,ds+\int_t^{t_0}\epsilon^2 a(s)^{5-c\sigma}\,ds\right)\,\\
%\lesssim &\,\epsilon^4+\epsilon a(t)^{-c\sqrt{\epsilon}}\\
%\lesssim &\,\epsilon a(t)^{-c\sqrt{\epsilon}}
%\end{align*}
%which proves the statement.
%\end{proof}



\subsection{Strong low order $C_G$-norm estimates}\label{subsec:AP}

Now, we can prove the main supremum norm estimates in this section:

\begin{lemma}[Strong low order $C_G$-norm estimates]\label{lem:AP} The following estimates hold:
\begin{subequations}
\begin{align}
\|\Psi\|_{C^{13}_G}\lesssim&\,\epsilon a^{-c\sqrt{\epsilon}}\,\label{eq:APmidPsi}\\
\|\Sigma\|_{C^{12}_G}\lesssim&\,\epsilon a^{-c\sqrt{\epsilon}}\,\label{eq:APmidSigma}\\
\|G-\gamma\|_{C^{12}_G}\lesssim&\sqrt{\epsilon}a^{-c\sqrt{\epsilon}}\,\label{eq:APmidG}\\
\|G^{-1}-\gamma^{-1}\|_{C^{12}_G}\lesssim&\sqrt{\epsilon}a^{-c\sqrt{\epsilon}}\,\label{eq:APmidG-1}\\
\|\nabla\phi\|_{C^{12}_G}\lesssim&\,\sqrt{\epsilon}a^{-c\sqrt{\epsilon}}\label{eq:APmidphi}\\
\|\Ric[G]+\frac29G\|_{C^{10}_G}\lesssim&\,\sqrt{\epsilon}a^{-c\sqrt{\epsilon}}\label{eq:APmidRic}\\
%\|N\|\lesssim&\,\sqrt{\epsilon}a^{4-c\sqrt{\epsilon}}\label{eq:APmidN}\\
\|\RB\|_{C^{11}_G}\lesssim&\,\epsilon \change{a^{2-c\sqrt{\epsilon}}}\label{eq:APmidB}\\
\|\RE\|_{C^{12}_G}\lesssim&\,\epsilon a^{-c\sqrt{\epsilon}}\label{eq:APmidE}
\end{align}
\end{subequations}
\end{lemma}
\begin{proof}
Before going into the individual estimates, we collect the following commutator term estimates from the expressions in \eqref{eq:commutator-aux-scalar}-\eqref{eq:commutator-aux-tensor}:
\begin{align}
\|[\del_t,\nabla^J]\zeta\|_{C^0_G}\lesssim&\,a^{-3}\|N+1\|_{C^{J-1}_G}\|\Sigma\|_{C^{J-1}_G}\|\zeta\|_{C^{J-1}_G}+\frac{\dot{a}}a\|N\|_{C^{J-1}_G}\|\zeta\|_{C^{J-1}_G} \label{eq:aux-comm-est-zeta}\\
\|[\del_t,\nabla^J]\mathfrak{T}\|_{C^0_G}\lesssim&\,a^{-3}\|N+1\|_{C^{J}_G}\left(\|\nabla^J\Sigma\|_{C^0_G}\|\mathfrak{T}\|_{C^0_G}+\|\Sigma\|_{C^{J-1}_G}\|\mathfrak{T}\|_{C^{J-1}_G}\right)+\frac{\dot{a}}a\|N\|_{C^J_G}\|\mathfrak{T}\|_{C^{J-1}_G} \label{eq:aux-comm-est-T}
\end{align}
With this in hand, we will prove each estimate by iterating over the derivative order as long as the bootstrap assumptions can be applied. In each step, we use the previously obtained estimates at lower order to control the commutator term (with some additional care for $\mathfrak{T}=\Sigma$ which we need to consider first), while we can use similar arguments to those at order $0$ to control the \enquote{core} of the evolution equations. 

To start out, we apply \eqref{eq:APSigma} on $\Sigma$ and the bootstrap assumption \eqref{eq:BsN} on $N$ to the rescaled evolution equations \eqref{eq:REEqG}-\eqref{eq:REEqG-1} and deduce
\begin{equation}\label{eq:AP-deltG}
\lvert \del_tG^{\pm1}\rvert_G=\lvert \del_t(G^{\pm 1}-\gamma^{\pm1})\rvert_G\lesssim \epsilon a^{-3}+\epsilon a^{1-c\sigma}\lesssim \epsilon a^{-3}\,.
\end{equation}

\underline{\eqref{eq:APmidSigma}:} We assume
\begin{equation}\label{eq:APmidSigma-indhyp}
\|\Sigma\|_{{C}^{J-1}_G}\lesssim \epsilon a^{-c\sqrt{\epsilon}}
\end{equation}
to be satisfied for some $J\in\{1,\dots,12\}$ (For $J=1$, this is true by \eqref{eq:APSigma}). Observe the following:
\[\del_t\lvert\nabla^J\Sigma\rvert_G^2=2\langle\del_t\nabla^J\Sigma,\nabla^J\Sigma\rangle_G+\del_tG^{-1}\ast\nabla^J\Sigma\ast\nabla^J\Sigma\]
Now, we commute \change{\eqref{eq:REEqSigmaSharp} }with $\nabla^J$: As before, $\nabla^J\del_t\Sigma$ is bounded by $\epsilon a^{-c\sigma}$ for any admissible $J$. Hence and using \eqref{eq:AP-deltG},
\[\del_t\lvert\nabla^J\Sigma\rvert_G^2\lesssim \epsilon a^{-3}\lvert\nabla^J\Sigma\rvert_G^2+\left(\epsilon a^{1-c\sigma}+\|[\del_t,\nabla^J]\Sigma\|_{C^0_G}\right)\lvert\nabla^J\Sigma\rvert_G\]
is satisfied. Looking at the commutator term using \eqref{eq:aux-comm-est-T}, we have with \eqref{eq:APSigma} that
\[\|[\del_t,\nabla^J]\Sigma\|_{C^0_G}\lesssim \epsilon a^{-3}\|\Sigma\|_{\dot{C}^J_G}+a^{-3}\cdot\|\Sigma\|_{C^{J-1}_G}^2+\epsilon a^{1-c\sigma}\|\Sigma\|_{C^{J-1}_G}\,.\]
Altogether, we obtain
\[\del_t\lvert\nabla^J\Sigma\rvert_G^2\lesssim\left(\epsilon a^{-3}\|\Sigma\|_{\dot{C}^J_G}+\epsilon a^{-c\sigma}+\epsilon^2a^{-3-c\sqrt{\epsilon}}\right)\lvert\nabla^J\Sigma\rvert_G\,.\]
With Lemma \ref{lem:weak-ftoc} as well as the initial data assumption \eqref{eq:init-ass} and the integral formula \eqref{eq:a-integrals} with $p=c\sqrt{\epsilon}$, this implies
\[\lvert\nabla^J\Sigma\rvert_G(t)\lesssim\,\int_t^{t_0}\epsilon a^{-3}\|\Sigma\|_{\dot{C}^J_G(\Sigma_s)}\,ds+\epsilon \left(1+\sqrt{\epsilon}a^{-c\sqrt{\epsilon}}\right)\]
and consequently, after taking the supremum on the left and applying the Gronwall lemma,
\[\|\Sigma\|_{\dot{C}^J_G(\Sigma_s)}\lesssim \epsilon a^{-c\sqrt{\epsilon}}\,.\]
Combining this with \eqref{eq:APmidSigma-indhyp} proves the statement up to order $J$, and hence shows \eqref{eq:APmidSigma} by iterating the argument up to $J=12$.\\

\underline{\eqref{eq:APmidPsi}:} We again assume that
\begin{equation}\label{eq:APmidPsi-indhyp}
\|\Psi\|_{C^{J-1}_G}\lesssim \epsilon a^{-c\sqrt{\epsilon}}
\end{equation}
holds for $J\in\{1,2,\dots,13\}$. Observe that
\[\lvert\del_t\nabla^J\Psi\rvert_G\lesssim a\|N+1\|_{C^{J+1}_G}\|\nabla\phi\|_{C^{J+1}_G}+\frac{\dot{a}}a\|\nabla N\|_{C^{J}_G}(1+\|\Psi\|_{C^J_G})+\|[\del_t,\nabla^J]\Psi\|_{C^0_G}\]
By \eqref{eq:BsN}, \eqref{eq:Bsnablaphi} and \eqref{eq:BsPsi}, the first two summands can be bounded (up to constant) by $\epsilon a^{1-c\sigma}$. By \eqref{eq:APmidPsi-indhyp}, \eqref{eq:APmidSigma} and \eqref{eq:BsN} and using \eqref{eq:aux-comm-est-zeta}, the commutator term is bounded (up to constant) by $\epsilon^2a^{-3-c\sqrt{\epsilon}}$. Altogether, 
\[\lvert\del_t\nabla^J\Psi\rvert_G\lesssim \epsilon a^{1-c\sigma}+\epsilon^2a^{-3-c\sqrt{\epsilon}}\]
follows. Inserting this and \eqref{eq:AP-deltG} into
\begin{equation*}
\lvert\del_t\left(\lvert\nabla^J\Psi\rvert_G^2\right)\rvert\leq\lvert\del_tG^{-1}\rvert_G\lvert\nabla^J\Psi\rvert_G^2+2\lvert\del_t\nabla^J\Psi\rvert_G\cdot\lvert\nabla^J\Psi\rvert_G
\end{equation*}
implies, with Lemma \ref{lem:weak-ftoc},
\begin{align*}
\lvert\nabla^J\Psi\rvert_G(t)\leq&\,\lvert\nabla^J\Psi\rvert(t_0)+\int_t^{t_0}\left(\frac12\lvert\del_t G^{-1}\rvert\lvert\nabla^J\Psi\rvert_G+\lvert\del_t\nabla^J\Psi\rvert_G\right)(s)\,ds\\
\lesssim&\,\epsilon^2+\int_t^{t_0}\left(\epsilon a(s)^{-3}\lvert\nabla^J\Psi(s,\cdot)\rvert_G+\epsilon a(s)^{1-c\sigma}+\epsilon^2a(s)^{-3-c\sqrt{\epsilon}}\right)\,ds\,.
\end{align*}
We obtain using \eqref{eq:a-integrals}:
\begin{equation*}
\lvert\nabla^J\Psi\rvert_G(t)\lesssim \epsilon a(t)^{-c\sqrt{\epsilon}}+\int_t^{t_0}\epsilon a(s)^{-3}\lvert\nabla^J\Psi(s,\cdot)\rvert_G\,ds
\end{equation*}
The Gronwall lemma, applying \eqref{eq:a-exp-est} and taking the supremum over $\Sigma_t$ then implies $\lvert\nabla^J\Psi\rvert_{\dot{C}^J_G}\lesssim \epsilon a^{-c\sqrt{\epsilon}}$. This proves \eqref{eq:APPsi} by iterating over $J$ and adding up the individual seminorms.\\

\underline{\eqref{eq:APmidG}-\eqref{eq:APmidG-1}:} Note that \eqref{eq:AP-deltG} implies \eqref{eq:APmidG} at order $0$ since one has
\[\lvert\del_t(\lvert G-\gamma\rvert_G)^2\rvert\lesssim\lvert\del_tG^{-1}\rvert_G\lvert G-\gamma\rvert_G^2+\lvert\del_t(G-\gamma)\rvert_G\lvert G-\gamma\rvert_G\lesssim \epsilon a^{-3}(1+\lvert G-\gamma\rvert_G)\lvert G-\gamma\rvert_G\]
which we can apply the Gronwall lemma to after integrating, along with \eqref{eq:log-est} for the error term, as in the proof of \eqref{eq:APmidSigma}.\\
For higher orders, commuting \eqref{eq:REEqG} with $\nabla^J$ and inserting \eqref{eq:APmidSigma} and \eqref{eq:BsN} implies
\[\|\del_t\nabla^J(G-\gamma)\|_{C^0_G}\lesssim \epsilon a^{-3-c\sqrt{\epsilon}}+\epsilon a^{1-c\sigma}+\|[\del_t,\nabla^J](G-\gamma)\|_{C^0_G}\]
with
\[\|[\del_t,\nabla^J](G-\gamma)\|_{C^0_G}\lesssim \left(\epsilon a^{-3-c\sqrt{\epsilon}}+\epsilon a^{1-c\sigma}\right)\|G-\gamma\|_{C^{J-1}_G}\,.\]
Once again doing the same iterative argument over $J\leq 12$ and assuming the estimate to hold up to $J-1$, this altogether becomes
\[\|\del_t\nabla^J(G-\gamma)\|_{C^0_G}\lesssim \epsilon a^{-3-c\sqrt{\epsilon}},\]
implying with \eqref{eq:a-integrals}
\[\|\nabla^J(G-\gamma)\|_{C^0_G}\lesssim \epsilon^2+\epsilon\int_t^{t_0}a(s)^{-3-c\sqrt{\epsilon}}\,ds\lesssim\sqrt{\epsilon}a^{-c\sqrt{\epsilon}}\,.\]
The argument for $G^{-1}-\gamma^{-1}$ is completely analogous.\\

\underline{\eqref{eq:APmidphi}:} We only prove the statement for $C^0_G$, the full estimate extends from there by the same iterative arguments as above. Considering \eqref{eq:REEqNablaPhi}, Lemma \ref{lem:lapse-maxmin}, \eqref{eq:APPsi} and \eqref{eq:Friedman}, we have
\begin{equation*}
\lvert\del_t\nabla\phi\rvert_G\lesssim a^{-3}\left(\lvert\nabla\Psi\rvert_G+\lvert\nabla N\rvert_G\right)
\end{equation*}
and thus, with \eqref{eq:APmidPsi} and the bootstrap assumption \eqref{eq:BsN},
\begin{equation*}
\lvert\del_t\nabla\phi\rvert_G\lesssim \epsilon a^{-3-c\sqrt{\epsilon}}\,.
\end{equation*}
With \eqref{eq:AP-deltG}, this implies
\begin{equation*}
\lvert\del_t\lvert\nabla\phi\rvert_G^2\rvert\lesssim \epsilon a^{-3}\lvert\nabla\phi\rvert_G^2+\epsilon a^{-3-c\sqrt{\epsilon}}\lvert\nabla\phi\rvert_G
\end{equation*}
and the statement follows as usual by applying Lemma \ref{lem:weak-ftoc}, \eqref{eq:log-est} and the Gronwall lemma.\\

\change{\underline{\eqref{eq:APmidRic}:} This follows as in the proof of \eqref{eq:APmidG} using \eqref{eq:REEqRic} and \eqref{eq:REEqG} and their commuted analogues.}

Once again, for $C^0_G$, we have
%\[\left\lvert\del_t\left(\left\lvert\Ric[G]+\frac29G\right\rvert_G^2\right)\right\rvert\lesssim \left(\left\lvert\del_t\left(\Ric[G]+\frac29G\right)\right\rvert_G+\left\lvert\del_tG\right\rvert\left\lvert\Ric[G]+\frac29G\right\rvert_G\right)\left\lvert\Ric[G]+\frac29G\right\rvert_G\,.\]
%Additionally, by \eqref{eq:REEqRic} and \eqref{eq:REEqG}
%\begin{align*}
%\left\lvert\del_t(\Ric[G]+\frac29G)\right\rvert_G\lesssim&\,a^{-3}\|\Sigma\|_{C^2_G}+a^{-3}\|N\|_{C^1_G}\|\Sigma\|_{C^1_G}+\frac{\dot{a}}a\|N\|_{C^2_G}
%\end{align*}
%is satisfied and hence, using \eqref{eq:APmidSigma}, \eqref{eq:BsN} and \eqref{eq:AP-deltG} as well as the Friedman equation \eqref{eq:Friedman}, we have
%\[\left\lvert\del_t\left\lvert\Ric[G]+\frac29G\right\rvert_G^2\right\rvert\lesssim \left(\epsilon a^{-3}\left\lvert\Ric[G]+\frac29G\right\rvert_{G}+\epsilon a^{-3-c\sqrt{\epsilon}}\right)\cdot\left\lvert\Ric[G]+\frac29G\right\rvert\,.\]
%This gives the statement for $C^0_G$ and can be iterated up to $C^{10}_G$ as above.\\
%
\underline{\eqref{eq:APmidB}:} This is obtained immediately from commuting \eqref{eq:REEqConstrB} with $\nabla^J$ and applying \eqref{eq:APmidSigma}. Notice that the Levi-Civita tensor can be absorbed into the implicit constants since $\lvert\epsilon[G]\rvert_G=\sqrt{6}$ holds (see \eqref{eq:LCS-contr3}).\\

\underline{\eqref{eq:APmidE}:} This follows like in the proof of \eqref{eq:APE} from applying \eqref{eq:BsRic}, \eqref{eq:Bsnablaphi} and \eqref{eq:APmidSigma} to the constraint equation \eqref{eq:REEqConstrE} commuted with $\nabla^J$.

%%%%% OLD RE PROOF%%%%%
%Regarding \eqref{eq:APmidE}, we first recognize that, since $\RE_{ij}$ is tracefree, we have
%\begin{equation}\label{eq:matter-REEqE}
%\left|\langle a^3J_{(\cdot)0(\cdot)},\RE\rangle_G\right|\lesssim\left[|\nabla\Psi|_G|\nabla\phi|_G+|\Psi+C||\nabla^2\phi|_G+4|(N+1)^{-1}||\Psi+C||\nabla N|_G|\nabla\phi|_G\right]|\RE|_G
%\end{equation}
%and hence, simply from the bootstrap assumptions \eqref{eq:BsPsi}, \eqref{eq:Bsnablaphi} and \eqref{eq:BsN},
%\begin{equation*}
%\left|\langle a^3(N+1)J_{(\cdot)0(\cdot)},\RE\rangle_G\right|\lesssim \epsilon a^{-c\sigma}\,.\end{equation*}
%Further, note that
%\[-\langle-\tau(1-\frac13(N+1))\RE,\RE\rangle_G=-(3-(N+1))\frac{\dot{a}}a|\RE|_G^2\]
%and since $N+1\leq 3$ by Lemma \ref{lem:lapse-maxmin}, we can drop this term in an upper estimate. Applying these facts to \eqref{eq:REEqE}, and again using that $\RE$ is traceless, we get using \eqref{eq:APSigma} and the bootstrap assumptions for $N$ and $\RB$:
%\begin{align*}
%-\langle\del_t\RE,\RE\rangle_G\lesssim&\,\left[|\nabla N\wedge \RB|_G^2+a^{-1}|\RE\times\Sigma|_G+|\curl \RB|_G+\epsilon a^{-c\sigma}\right]|\RE|_G\\
%\lesssim&\,\left[a^{-1}|\nabla N|_G|\nabla\RB|_G+a^{-3}|\RE|_G|\Sigma|_G|+a^{-1}|\nabla\RB|_G+\epsilon a^{-c\sigma}\right]|\RE|_G\\
%\lesssim&\,\epsilon a^{-3}|\RE|_G^2+\epsilon a^{-2-c\sigma}|\RE|_G\,.
%\end{align*}
%From this, we immediately obtain, dealing with $\del_tG$ as usual with \eqref{eq:APSigma}:
%\begin{align*}
%-\del_t|\RE|_G^2\leq&\, |\del_tG|_G|\RE|_G^2-2\langle\del_t\RE,\RE\rangle_G\\
%\leq&\,\epsilon a^{-3}|\RE|_G^2+\epsilon a^{-1-c\sigma}|\RE|_G
%\end{align*}
%and the statement for $l=0$ now follows as usual after dividing by $|\RE|_G$.\\
%Now, we again assume \eqref{eq:APmidE} to be satisfied at order $l-1$ and consider \eqref{eq:EEq-aux-E}. We can deal with the first term as at order $0$, and the second term in the first line is bounded by $\epsilon a^{1-c\sigma}$ by the bootstrap assumptions for $N$ and $\RE$. Further, note for the last term that we analogously have
%\[\lvert\langle a^3\nabla^lJ_{(\cdot)0(\cdot)},\nabla^l\RE\rangle_G\rvert\lesssim \left[\|N+1\|_{C^l_G}\left(\|\Psi\|_{C^{l+1}_G}\|\nabla\phi\|_{C^l_G}+\|\nabla\phi\|_{C^{l+1}_G}\right)+\|N\|_{C^{l+1}_G}\|\nabla\phi\|_{C^l_G}\right]|\nabla^l\RE|_G\]
%Else, observe that all remaining terms in \eqref{eq:EEq-aux-E} beside the commutator term are bounded by $\epsilon a^{-1-c\sigma}+\epsilon a^{-3}|\RE|_G$, using the same arguments as before. Finally, we have using the bootstrap assumption for $N$ and $\RE$, the induction hypothesis and \eqref{eq:APmidSigma},
%\[\|[\del_t,\nabla^l]\RE\|_{C^0_G}\lesssim \epsilon^\frac32 a^{-3-c\sqrt{\epsilon}}+\epsilon^2 a^{1-c\sigma}.\]
%With all of this in hand, we can now deduce
%\[-\del_t|\nabla^l\RE|_G^2\lesssim \epsilon a^{-3}|\nabla^l\RE|_G^2+\epsilon^\frac32 a^{-3-c\sqrt{\epsilon}}|\RE|_G^2\]
%and thus prove the induction step by dividing by $|\nabla^l\RE|_G$ and using the Gronwall lemma.
\end{proof}
%\begin{remark}
%Note that we need to use that $N$ converges in the proof of \eqref{eq:APmidPsi}, which is necessary to get \eqref{eq:APmidSigma} and subsequently all other a priori estimates. As we will see in the next section, we only get this convergence behaviour at orders at which we can also control $\Ric[G]$, and thus there must be a gap between the orders we can control with a priori estimates as a direct result of the bootstrap assumption on $\mathcal{C}$ and the norms present on $\mathcal{C}$ itself. Essentially, the rest of the bootstrap argument is devoted to bridging this gap.
%\end{remark}



\subsection{Other useful a priori observations}\label{subsec:AP-misc}

Before moving on to the energy estimates, we collect a differentiation identity and lay the groundwork for energy coercivity:

\begin{lemma}[The volume form and differentiation of integrals]\label{lem:delt-int}
Let $\mu_G=\sqrt{\det G}$ denote the volume element with regard to $G$. It satisfies
\begin{equation}\label{eq:delt-muG}
\del_t\mu_G=\frac12\mu_G(G^{-1})^{ij}\del_tG_{ij}=-N\tau\mu_G\,\,,
\end{equation}
and hence one has
\begin{equation}\label{eq:APvol}
\|\mu_{G}-\mu_{\gamma}\|_{C^0_G}\lesssim\epsilon \,.
\end{equation}
on $(\Sigma_t)_{t\in(t_{Boot},t_0]}$. Further, for any differentiable function $\zeta$, one has
\begin{equation}\label{eq:delt-int}
\del_t\int_M \zeta\vol{G} = \int_M \del_t\zeta\vol{G}-\int_MN\tau \cdot \zeta\vol{G}
\end{equation}
\end{lemma}
\begin{proof}
From \eqref{eq:REEqG}, we obtain $(G^{-1})^{ij}\del_tG_{ij}=-2N\tau$,
%\begin{align*}
%(G^{-1})^{ij}\del_tG_{ij}&=-2(N+1)a^{-3}(G^{-1})^{ij}(\Sigma_{ij})-2\frac{\dot{a}}{a}\cdot3\\
%&=-2\tau(N+1)-6\frac{\dot{a}}{a}\\
%&=-2N\tau\,,
%\end{align*}
and \eqref{eq:delt-muG} follows by
\[\del_t\mu_G%=\frac1{2\sqrt{\det G}}\del_t\det G
=\frac12\sqrt{\det G}(G^{-1})^{ij}\del_tG_{ij}=-N\tau\mu_G\,.\]
Hence, we have using \eqref{eq:BsN} and the initial data estimate \eqref{eq:init-vol-el} that
\begin{align*}
\lvert\mu_G-\mu_\gamma\rvert(t,\cdot)%\lesssim&\, \epsilon^2+\int_t^{t_0}\epsilon a(s)^{1-c\sigma}\lvert\mu_G-\mu_\gamma\rvert(s,\cdot)\,ds+\mu_\gamma\int_t^{t_0}\epsilon a(s)^{1-c\sigma}\,ds\\
\lesssim&\, \epsilon+\int_t^{t_0}\epsilon a(s)^{1-c\sigma}\lvert\mu_G-\mu_\gamma\rvert(s.\cdot)\,ds
\end{align*}
holds, and thus \eqref{eq:APvol} after applying the Gronwall lemma.\\
Finally, we obtain \eqref{eq:delt-int} by writing $\vol{G}=\frac{\mu_G}{\mu_\gamma}\vol{\gamma}$ and inserting \eqref{eq:delt-muG}
%\begin{align*}
%\del_t\int_M\zeta\vol{G}&=\int_M\del_t\zeta\vol{G}+\int_M\zeta\frac{\del_t\mu_G}{\mu_\gamma}\vol{\gamma}\\
%&=\int_M\del_t\zeta-N\tau \zeta\,\vol{G}
%\end{align*}
\end{proof}

\begin{lemma}[Preliminary Sobolev norm estimates]\label{lem:Sobolev-norm-equivalence-improved} Let $\zeta$ be a scalar function and $\mathfrak{T}$ be a symmetric $\Sigma_t$-tangent $(0,2)$-tensor, and let $l\in\{1,\dots,\change{9}\}$. Then, on $(t_{Boot},t_0]$, the following estimates are satisfied: For $l>5$, one has:
\begin{subequations}
\begin{align}
\|\nabla^2\zeta\|_{L^2_G}^2\lesssim&\,\|\Lap\zeta\|_{L^2_G}^2+a^{-c\sqrt{\epsilon}}\|\nabla\zeta\|_{L^2_G}^2\label{eq:Sobolev-norm-equiv-zetalow}\\
\|\zeta\|_{H^{2l}_G}^2\lesssim&\,\|\Lap^{l} \zeta\|_{L^2_G}^2+a^{-c\sqrt{\epsilon}}\left(\sum_{m=0}^{l-1}\|\Lap^m\zeta\|_{L^2_G}^2+\change{\|\zeta\|_{C_G^{\change{2l-12}}}^2\E^{(\leq 2l-3)}(\Ric,\cdot)}\right) \label{eq:Sobolev-norm-equiv-zeta2l}\\
\sum_{m=1}^{2l+1}\|\zeta\|_{\dot{H}^{m}_G}^2\lesssim&\,\|\nabla\Lap^{l}\zeta\|_{L^2_G}^2+a^{-c\sqrt{\epsilon}}\left(\sum_{m=0}^{l-1}\|\nabla\Lap^m\zeta\|_{L^2_G}^2+\|\nabla\zeta\|_{C_G^{\change{2l-12}}}^2\E^{(\leq2l-2)}(\Ric,\cdot)\right)\label{eq:Sobolev-norm-equiv-zeta2l+1}\\
\changefinal{\sum_{m=1}^{2l}\|\nabla\zeta\|_{\dot{H}^{m}_G}^2\lesssim}&\changefinal{\,\|\nabla\Lap^l\zeta\|_{L^2_G}^2+a^{-c\sqrt{\epsilon}}\left(\sum_{m=0}^{l-1}\|\nabla\Lap^m\zeta\|_{L^2_G}^2+\|\nabla\zeta\|_{C_G^{\change{2l-11}}}^2\E^{(\leq 2l-2)}(\Ric,\cdot)\right)\label{eq:Sobolev-norm-equiv-nablazeta2l}}
\end{align}
\end{subequations}
and
\begin{subequations}
\begin{align}
\|\mathfrak{T}\|_{H^{2l}_G}^2\lesssim&\,\|\Lap^{l} \mathfrak{T}\|_{L^2_G}^2+a^{-c\sqrt{\epsilon}}\left(\sum_{m=0}^{l-1}\|\Lap^m\mathfrak{T}\|_{L^2_G}^2+\|\mathfrak{T}\|_{C_G^{\change{2l-11}}}^2\E^{(\leq 2l-2)}(\Ric,\cdot)\right)\label{eq:Sobolev-norm-equiv-T2l}\\
\sum_{m=1}^{2l+1}\|\mathfrak{T}\|_{\dot{H}^{m}_G}^2\lesssim&\,\|\nabla\Lap^{l}\mathfrak{T}\|_{L^2_G}^2+a^{-c\sqrt{\epsilon}}\left(\sum_{m=0}^{l-1}\|\nabla\Lap^m\mathfrak{T}\|_{L^2_G}^2+\|\mathfrak{T}\|_{C_G^{\change{2l-10}}}^2\E^{(\leq2l-1)}(\Ric,\cdot)\right)\label{eq:Sobolev-norm-equiv-T2l+1}
\end{align}
\end{subequations}
More precisely, the Ricci energy terms can be dropped in all of the above estimates for $l\leq 5$.
\end{lemma}
\begin{remark}\label{rem:Sobolev-norm-equivalence-improved}
We stress that Lemma \ref{lem:Sobolev-norm-equivalence-improved} is crucial for everything that follows in multiple ways:\\

Firstly, the $L^2_G$-norms containing $\zeta$ and $\mathfrak{T}$ on the right hand sides \delete{of all inequalities }above except \eqref{eq:Sobolev-norm-equiv-nablazeta2l} are in precisely the form the energies in Definition \ref{def:energies} take. Hence, this is what will actually yield near-coercivity of \change{our energies since the $C_G$-norms can be controlled by $a^{-c\sqrt{\epsilon}}$ or better using the a priori estimates from Lemma \ref{lem:AP}, as well as \eqref{eq:BsN} for the lapse.} This will be shown more \change{explicitly }as an intermediary step in improving the bootstrap assumptions for $\mathcal{C}$ (see proof of Corollary \ref{cor:H-imp}).\\

Secondly, a downside of using $\Lap$ as the main differential operator to commute with the Einstein scalar-field system is that it creates error terms that we can only bound by Sobolev norms and not directly express as energies. Thus, we need a way to translate this information back to energies to formulate energy inequalities. A lot of this is done \enquote{under the hood} in the error term estimates in subsection \ref{subsec:L2-error-est}.\\

Finally, some top order terms also do not appear in a way that their $L^2$-norm is directly the square root of an energy (see, for example, the term $a\nabla^2\Lap^{\frac{L}2}N$ in \eqref{eq:comeq-Sigma}), and some borderline terms would lead to nonintegrable divergences \delete{at that level of derivatives }if we were to incur additional divergences in estimation (see, for example, the first term in \eqref{eq:comeq-Sigma-border}). Lemma \ref{lem:Sobolev-norm-equivalence-improved} precisely provides a way to relate these terms to energies \delete{without losing precision at top order in terms of powers of $a$}. Additionally, by applying these estimates for terms of the form $\Lap^\frac{L}2\zeta$ and $\Lap^\frac{L}2\mathfrak{T}$, one can avoid high order curvature energies \change{that run the risk of breaking the energy hierarchy.}
\end{remark}
\begin{proof} Since the arguments for all of the inequalities above are very similar, we only prove \eqref{eq:Sobolev-norm-equiv-T2l} in full and then briefly \changefinal{address }the other estimates.\\
Letting $\tilde{\mathfrak{T}}_{i_1\dots i_{2l}k_1k_2}=\nabla_{i_1}\dots\nabla_{i_{2l}}\mathfrak{T}_{k_1k_2}\,$, we compute with the commutator formula \eqref{eq:[Lap,nabla]T} and strong $C_G$-norm estimate \eqref{eq:APmidRic}:
\begin{align*}
\int_M\lvert\nabla^2\tilde{\mathfrak{T}}\rvert_G^2=&\,-\int_M\langle\nabla\tilde{\mathfrak{T}},\Lap\nabla\tilde{\mathfrak{T}}\rangle_G\,\vol{G}\\
=&-\int_M\langle\nabla\tilde{\mathfrak{T}},\nabla\Lap\tilde{\mathfrak{T}}\rangle_G\,\vol{G}+\int_M\nabla\tilde{\mathfrak{T}}\ast\left[\nabla\Ric[G]\ast \tilde{\mathfrak{T}}+\Ric[G]\ast\nabla\tilde{\mathfrak{T}}\right]\,\vol{G}\\
\lesssim& \int_M\lvert\Lap\tilde{\mathfrak{T}}\rvert_G^2\,\vol{G}+\changefinal{\left(1+\left\|\Ric[G]+\frac29G\right\|_{{C}^1_G}\right)}\cdot\left[\int_M\lvert\tilde{\mathfrak{T}}\rvert_G^2\,\vol{G}+\int_M\lvert\nabla\tilde{\mathfrak{T}}\rvert_G^2\,\vol{G}\right]\\
\lesssim&\int_M \lvert\Lap\tilde{\mathfrak{T}}\rvert_G^2+a^{-c\sqrt{\epsilon}}\int_M\lvert\tilde{\mathfrak{T}}\rvert_G^2\,\vol{G}
\end{align*}
In the final step, we used integration by parts to obtain
\[a^{-c\sqrt{\epsilon}}\int_M\lvert\nabla\tilde{\mathfrak{T}}\rvert_G^2\,\vol{G}\leq \int_M\lvert\Lap\tilde{\mathfrak{T}}\rvert_G\cdot a^{-c\sqrt{\epsilon}}\lvert\tilde{\mathfrak{T}}\rvert_G\,\vol{G}\lesssim \int_M\left(\lvert\Lap\tilde{\mathfrak{T}}\rvert_G^2+a^{-2c\sqrt{\epsilon}}\lvert\tilde{\mathfrak{T}}\rvert_G^2\right)\,\vol{G}\]
and updated $c$. This already shows \eqref{eq:Sobolev-norm-equiv-T2l} for $l=1$. Assume now that \eqref{eq:Sobolev-norm-equiv-T2l} holds up to some $l\in\N,\, l\leq 9$ and \textit{any} symmetric $\Sigma_t$-tangent $(0,2)$-tensor field. By applying \change{\eqref{eq:[Lap,nabla2]T}}, we have
\change{\begin{align*}
\Lap\tilde{\mathfrak{T}}%=&\,\Lap\nabla^{2l}\mathfrak{T}\\
=&\,\nabla^{2l}\Lap\mathfrak{T}+[\Lap,\nabla^2]\nabla^{2l-2}\mathfrak{T}+\dots+\nabla^{2l-2}[\Lap,\nabla^2]\mathfrak{T}\\
=&\,\nabla^{2l}\Lap\mathfrak{T}+\sum_{I_\Ric+I_\mathfrak{T}=2l}\nabla^{I_\Ric}(\Ric[G]+\frac29G)\ast\nabla^{I_\mathfrak{T}}\mathfrak{T}+G\ast\nabla^{2l}\mathfrak{T}\numberthis\label{eq:lap-sob-equiv-crucial}
%&\,+\sum_{m=0}^{l-2}\sum_{I_1+\dots+I_{l-m}+I_\mathfrak{T}=2m+2}\nabla^{I_1}\Ric[G]\ast\dots\ast\nabla^{I_{l-m}}\Ric[G]\ast\nabla^{I_{\mathfrak{T}}}\mathfrak{T}\\
%=&\,\nabla^{2l}\Lap\mathfrak{T}+\sum_{m=0}^l\underbrace{(\Ric[G]\ast\dots\ast\Ric[G])}_{(m+1)-\text{times}}\ast\nabla^{2(l-m)}\mathfrak{T}\\
%&\,+\sum_{I_\Ric+I_\mathfrak{T}=2l,I_{\Ric}>0}\nabla^{I_\Ric}\left(\Ric[G]+\frac29G\right)\ast\nabla^{I_\mathfrak{T}}\mathfrak{T}+(NL)\,.
\end{align*}}
\change{Subsequently}, we have \change{for $l>5$ }using the strong $C_G$-norm estimate \eqref{eq:APmidRic} \change{for any Ricci term of order 10 or lower }that
\begin{align*}
\|\mathfrak{T}\|_{\dot{H}^{2(l+1)}_G}^2=\int_M \lvert\nabla^2\tilde{\mathfrak{T}}\rvert_G^2\,\vol{G}\lesssim \|\Lap\mathfrak{T}\|_{\dot{H}^{2l}_G}^2+\left(1+\sqrt{\epsilon}a^{-c\sqrt{\epsilon}}\right)\|\mathfrak{T}\|_{H^{2l}_G}^2+\|\mathfrak{T}\|_{C_G^{\change{2l-11}}}^2\|\Ric[G]+\frac29G\|_{H^{2l}_G}^2\,,
\end{align*}
\change{and get the same estimate without the final term for $l\leq 5.$ }By assumption, we can estimate $\|\Lap\mathfrak{T}\|_{H^{2l}_G}^2$, $\|\mathfrak{T}\|_{H^{2l}_G}^2$ and \changefinal{$\|\Ric[G]+\frac29G\|_{H^{2l}_G}^2$ }as in \eqref{eq:Sobolev-norm-equiv-T2l}, and get the following \change{for $l>5$} :
\begin{align*}
\int_M\lvert\nabla^{2l+2}\mathfrak{T}\rvert_G^2\,\vol{G}\lesssim&\,\left[\|\Lap^{l}\Lap\mathfrak{T}\|^2_{L^2_G}+a^{-c\sqrt{\epsilon}}\sum_{m=0}^{l-1}\|\Lap^{m+1}\mathfrak{T}\|_{L^2_G}^2+a^{-c\sqrt{\epsilon}}\|\Lap\mathfrak{T}\|_{C_G^{\change{2l-12}}}^2\E^{(\leq 2l-2)}(\Ric,\cdot)\right]\\
&\,+\left[a^{-c\sqrt{\epsilon}}\|\Lap^l\mathfrak{T}\|_{L^2_G}^2+a^{-c\sqrt{\epsilon}}\sum_{m=0}^{l-1}\|\Lap^{m}\mathfrak{T}\|^2_{L^2_G}+a^{-c\sqrt{\epsilon}}\|\mathfrak{T}\|_{C^{\change{2l-12}}_G}^2\E^{(\leq 2l-2)}(\Ric,\cdot)\right]\\
&\,+a^{-c\sqrt{\epsilon}}\|\mathfrak{T}\|_{C_G^{\change{2l-11}}}^2\left[\E^{(2l)}(\Ric,\cdot)+\left(\E^{(\leq 2l-2)}(\Ric,\cdot)+\epsilon a^{-c\sqrt{\epsilon}}\E^{(\leq 2l-2)}(\Ric,\cdot)\right)\right]\\
\lesssim&\,\|\Lap^{l+1}\mathfrak{T}\|_{L^2_G}^2+a^{-c\sqrt{\epsilon}}\left(\sum_{m=0}^l\|\Lap^m\mathfrak{T}\|_{L^2_G}^2+\|\mathfrak{T}\|_{C^{\change{2l-10}_G}}^2\E^{(\leq 2l)}(\Ric,\cdot)\right)
\end{align*}
\change{For $l=5$, we get analogous estimates dropping the Ricci energies in the first two lines, and for $l=4$, the same with all curvature terms dropped.\\}
To prove the statement for $l+1$, it now remains to be shown that $\|\mathfrak{T}\|_{\dot{H}^{2l+1}_G}^2$ can be bounded by the same right hand side (up to constant) as above. By integration by parts, one has
\[\|\nabla^{2l+1}\mathfrak{T}\|_{L^2_G}^2\lesssim \|\nabla^{2l}\mathfrak{T}\|_{L^2_G}^2+\|\Lap\nabla^{2l}\mathfrak{T}\|_{L^2_G}^2,\]
where the latter tensor is precisely $\Lap\tilde{\mathfrak{T}}$ which we just treated, and the former is covered by the induction assumption at order $2l$. So, \eqref{eq:Sobolev-norm-equiv-T2l} now follows for $l+1$, and thus by iteration up to $l=10$.\\

\noindent The proof of \eqref{eq:Sobolev-norm-equiv-T2l+1} is analogous -- we note that since we actually only needed a strong estimate on $\|\Ric[G]+2G\|_{C^9_G}$ for the previous inequality, but \eqref{eq:APmidRic} holds at $C^{10}_G$, this gives enough room to extend the argument in full despite the extra derivative order.\\
For both, note that we only need to estimate the Ricci terms in the $L^2_G$-norm if one cannot apply the a priori estimate \eqref{eq:APmidRic} to all $\nabla^{I_\Ric}\Ric[G]$ that occur in \eqref{eq:lap-sob-equiv-crucial}, and thus we could easily adjust the proof such that the Ricci energy does not occur in any of the proofs as long as $2l-1\leq 10$ is satisfied, so for $l\leq 5$.\\

The estimates \eqref{eq:Sobolev-norm-equiv-zeta2l}-\eqref{eq:Sobolev-norm-equiv-nablazeta2l} are proved identically, the only difference being that one order of curvature less enters in the commutator terms in \eqref{eq:lap-sob-equiv-crucial}, leading to one order less in curvature in total. For \eqref{eq:Sobolev-norm-equiv-zetalow}, we note that we can avoid incurring any $L^2$-norm by carefully repeating the argument we made for $\tilde{\mathfrak{T}}$ using \eqref{eq:[Lap,nabla]SF}:
\begin{align*}
\int_M\lvert\nabla^2\zeta\rvert_G^2\,\vol{G}%=&\,-\int_M\langle\nabla\zeta,\Lap\nabla\zeta\rangle_G\,\vol{G}\\
=&\,\int_M-\langle\nabla\zeta,\nabla\Lap\zeta\rangle_G\,\vol{G}+\int_M\Ric[G]\ast\nabla\zeta\ast\nabla\zeta\,\vol{G}\\
\lesssim&\,\int_M\lvert\Lap\zeta\rvert_G^2\,\vol{G}+a^{-c\sqrt{\epsilon}}\int_M\lvert\nabla\zeta\rvert_G^2\,\vol{G}
\end{align*}
\end{proof}

\section{Big Bang stability: Elliptic lapse estimates}\label{sec:lapse}

In this section, we study the elliptic structure of the equations \eqref{eq:REEqLapse1}-\eqref{eq:REEqLapse2}, which admit estimates controlling (time-scaled) lapse energies by other energy quantities. To this end, we recast these equations as follows:


\begin{definition}[Elliptic operators] For any (sufficiently regular) scalar function $\zeta$ on $\Sigma_t$, we define the differential operators
\begin{subequations}
\begin{align}
\L \zeta=&\,a^4\Lap \zeta-f\cdot \zeta,&f=\frac13a^4+12\pi C^2+\underbrace{\langle\Sigma,\Sigma\rangle_G+8\pi\Psi^2+16\pi C\Psi}_{=:F}\,, \\
\Ltilde \zeta=&\,a^4\Lap\zeta-\tilde{f}\cdot\zeta,&\quad \tilde{f}=\frac13a^4+12\pi C^2+\underbrace{a^4\left[R[G]+\frac23-8\pi\lvert\nabla\phi\rvert^2_G\right]}_{=\tilde{F}}\,.
\end{align}
\end{subequations}
%with their gradient vector field analogues
%\begin{align*}
%(\L_1\nabla\zeta)_i=&\,a^4\Lap\nabla_i\zeta-f\nabla\zeta-a^4\Ric[G]^{\sharp n}_i\nabla_n\zeta=\nabla_i(\L\zeta)+\zeta\cdot\nabla_i f\\
%(\Ltilde_1\nabla\zeta)_i=&\,a^4\Lap\nabla_i\zeta-\tilde{f}\nabla\zeta-a^4\Ric[G]^{\sharp n}_i\nabla_n\zeta=\nabla_i(\L\zeta)+\zeta\cdot\nabla_i \tilde{f}\,.
%\end{align*}
\end{definition}
\noindent Note that the lapse equations \eqref{eq:REEqLapse1}, respectively \eqref{eq:REEqLapse2}, now read
\begin{equation}\label{eq:lapse-with-op}
\L N=F,\ \text{respectively}\ \Ltilde N=\tilde{F}.\,
\end{equation}
Furthermore, observe that
\begin{subequations}
\begin{align}
\label{eq:[L,Lap]}[\L,\Lap]\zeta=&\,\Lap f\cdot\zeta+2\langle\nabla f,\nabla\zeta\rangle_G=\change{\Lap F}\cdot\zeta+2\langle\nabla F,\nabla\zeta\rangle_G\,,\\
\label{eq:[Ltilde,Lap]}[\Ltilde,\Lap]\zeta=&\,\change{\Lap \tilde{F}}\cdot\zeta+2\langle\nabla \tilde{F},\nabla\zeta\rangle_G\,.
\end{align}
\end{subequations}
%and
%\begin{equation}\label{eq:lapse-with-op-tilde}
%\L_1\nabla N=(N+1)\nabla F,\ \text{resp.}\ \Ltilde_1 N=(N+1)\nabla\tilde{F}\,.
%\end{equation}


\subsection{Elliptic lapse estimates with $\L$}\label{subsec:lapse-L}

We first study the elliptic operator $\L$, which will admit weak lapse energy estimates in terms of scalar field quantities and $\Sigma$, up to curvature errors, that can in particular be utilized at high orders without having to resort to higher derivative levels. Before moving on to the estimates themselves, we collect a couple of inequalities we can deduce from the bootstrap assumptions and strong $C_G$-norm estimates.

\begin{remark}\label{ass:lapse}
There exists a constant $K>0$ such that, for $\epsilon>0$ small enough, the following estimates hold:
\begin{itemize}
\item $F\geq -K\epsilon$, and equivalently $f\geq 12\pi C^2-K\epsilon$. This is ensured by \eqref{eq:APSigma} and \eqref{eq:APPsi}. In particular, we can assume $\epsilon$ to have been small enough such that $f-6\pi C^2$ can be bounded from below by a positive constant that is independent of $\epsilon$ (for example $3\pi C^2$).
\item $\lvert\nabla f\rvert_{G}=\lvert\nabla F\rvert_{G}\leq  K\epsilon a^{-c\sqrt{\epsilon}}$. This is given by \eqref{eq:APmidSigma} and \eqref{eq:APmidPsi}.
%\item For any covector $v$ and vector $w$, one has $\left\lvert v_i\left[(\Ric[G]^\sharp)^i_j+2\delta^i_j\right]v^{\sharp j}\right\rvert\leq K_2{\epsilon} a^{-c\sigma}\lvert v\rvert_G\lvert w\rvert_G$. This follows immediately from \eqref{eq:BsRic}.
%\item One has \[6\pi C^2-{K}_1\epsilon-{K}_2\epsilon a(t_0)^{4-c\sigma}>0\]
%and consequently, for any $t\in(0,t_0]$,
%\[{F}(t)+6\pi C^2-{K}_2\epsilon a(t)^{4-c\sigma}\geq \text{const.}>0\,.\]
%Since this is a purely algebraic requirement, we can simply assume $\epsilon$ to have been chosen to be sufficiently small to ensure this holds.
\end{itemize}
\end{remark}

\begin{lemma}[Elliptic estimates with $\L$]\label{lem:ell-lapse-1}
Consider scalar functions $\zeta,Z$ on $\Sigma_t$ that satisfy
\begin{equation}\label{eq:ell-L-eq}\L \zeta=Z\,.\end{equation} Then,
\begin{equation}\label{eq:ell-L-est}
a^4\|\Lap \zeta\|_{L^2_G}+a^2\|\nabla \zeta\|_{L^2_G}+\|\zeta\|_{L^2_G}\lesssim \|Z\|_{L^2_G}
\end{equation}
\end{lemma}

\begin{proof} The proof follows along the same lines as that of \cite[Lemma 16.5]{Speck2018}: First, we obtain the following by \change{multiplying \eqref{eq:ell-L-eq} with $-\zeta$ and integrating}:
%\[\int_M\left(a^4\lvert\nabla \zeta\rvert_G^2+f\lvert\zeta\rvert^2\right)\vol{G}=-\int_MZ\zeta\vol{G}\leq \frac1{24\pi C^2}\|Z\|^2_{L^2_G}+6\pi C^2\|\zeta\|^2_{L^2_G}\]
%Hence, after rearranging
 and using that $f-6\pi C^2$ is bounded from below by a positive constant (see the first point in Remark \ref{ass:lapse}):%, we compute
\[\int_M (a^4\lvert\nabla \zeta\rvert_G^2+\lvert\zeta\rvert^2)\vol{G}\lesssim \|Z\|^2_{L^2_G}%\,.
\]
Next, we \change{multiply }\eqref{eq:ell-L-eq} with $a^4\Lap \zeta$ and obtain%, integrate and use integration by parts:
\begin{align*}
\int_{M}\left(a^8\lvert\Lap \zeta\rvert^2+a^4\lvert\nabla \zeta\rvert_G^2f\right)\,\vol{G}%&=\int_M \left(a^4Z\Lap \zeta-a^4\langle\nabla f,\nabla \zeta\rangle_G \zeta\right)\,\vol{G}\\
&\leq\int_M\frac12\lvert Z\rvert^2+\frac{a^8}2\lvert\Lap \zeta\rvert^2+\left(\frac12\lvert\zeta\rvert^2+\frac{a^4}2\lvert\nabla \zeta\rvert_G^2\right)a^2\|\nabla f\|_{L^\infty_G}\,\vol{G}
\end{align*}
Using the second point in Remark \ref{ass:lapse} as well as the previous step, we can now conclude
\begin{align*}
\int_{M} \left(a^8\lvert\Lap \zeta\rvert^2+a^4\lvert\nabla \zeta\rvert_G^2\right)\,\vol{G}%&\lesssim \|Z\|_{L^2_G}^2+K\epsilon a^{2-c\sqrt{\epsilon}}\int_{M}\left(\lvert\zeta\rvert^2+a^4\lvert\nabla \zeta\rvert_G^2\right)\,\vol{G}\\
&\lesssim (1+\epsilon)\|Z\|_{L^2_G}^2
\end{align*}
and thus the statement after rearranging.
\end{proof}

%\begin{lemma}\label{lem:ell-lapse-2}
%Consider $\zeta\in C^\infty(M), W\in\X(M)$ such that
%\[\L_1\nabla\zeta=W\,.\]
%The following elliptic estimates hold:
%\begin{equation}
%a^4\|\Lap\nabla \zeta\|_{L^2_G}+a^2\|\nabla^2 \zeta\|_{L^2_G}+\|\nabla \zeta\|_{L^2_G}\leq \|W\|_{L^2_G}
%\end{equation}
%\end{lemma}
%\begin{proof}
%Using the third point in Remark \ref{ass:lapse}, we have
%\[\left\lvert a^4\nabla_i \zeta(\Ric[G]^\sharp)^i_j\nabla^{\sharp j}\zeta\right\rvert\leq 2a^4\lvert\nabla \zeta\rvert_G^2+{K}_2{\epsilon}a^{4-c\sigma}\lvert\nabla \zeta\rvert_G^2\,. \]
%Hence, by testing the equation by $\nabla^\sharp \zeta$, we compute:
%\begin{align*}
%&\int_{M}\left(a^4\lvert\nabla^2\zeta\rvert_G^2+({f}-{K}_2{\epsilon}a^{4-c\sigma}-2a^4)\lvert\nabla \zeta\rvert_G^2\right)\,\vol{G}\\
%\leq&\, \int_M-a^4\langle\Lap\nabla \zeta,\nabla \zeta\rangle_G+{f}\langle\nabla \zeta,\nabla \zeta\rangle_G-\nabla_i\zeta(\Ric[G]^\sharp)^i_j\nabla^{\sharp j}\zeta\,\vol{G}\\
%=&\,\int_M\langle-\L_1\nabla \zeta,\nabla \zeta\rangle_G\,\vol{G}=\int_M-\langle W,\nabla \zeta\rangle_G\,\vol{G}\\
%\leq&\, \int_M\left[\frac1{24\pi C^2}\lvert W\rvert_G^2+6\pi C^2\lvert\nabla \zeta\rvert_G^2\right]\,\vol{G}\,.
%\end{align*}
%Note that
%\[{f}-{K}_2\epsilon a^{4-c\sigma}-2a^4-6\pi C^2=6\pi C^2+{F}+a^4-{K}_2\epsilon a^{4-c\sigma}>\text{const.}>0\]
%by the last point in Remark \ref{ass:lapse}, so we get after rearranging:
%\begin{equation*}
%\int_M a^4\lvert\nabla^2\zeta\rvert_G^2+\lvert\nabla \zeta\rvert_G^2\,\vol{G}\lesssim \int_M \lvert W\rvert_G^2
%\end{equation*}
%Further, we have
%\begin{align*}
%a^8\lvert(\Ric[G]^{\sharp})^j_i\nabla_j\zeta\Lap\nabla^i\zeta\rvert\leq&\, a^8(2+K_2\sqrt{\epsilon}a^{-c\sigma})\lvert\nabla \zeta\rvert_G\lvert\Lap\nabla \zeta\rvert_G\\
%\leq&\, \frac{a^8}4\lvert\Lap\nabla \zeta\rvert_G^2+(2a^8+K_2^2\epsilon^2 a^{8-c\sigma})\lvert\nabla \zeta\rvert_G^2
%\end{align*}
%Now testing the equation instead by $a^4\Lap\nabla^{\sharp i}\zeta$ after having used the equation above in an upper estimate, we obtain as before:
%\begin{align*}
%&\int_M \left(a^8\lvert\Lap\nabla \zeta\rvert_G^2+a^4({f}-2a^4+{K}_2^2\epsilon a^{4-c\sigma})\lvert\nabla^2\zeta\rvert_G^2\right)\,\vol{G}\\
%\leq&\int_M \left(\left[a^4\Lap\nabla_i\zeta-{f}\nabla_i\zeta+a^4(\Ric[G]^\sharp)_i^j\nabla_j\zeta\right]a^4\Lap\nabla^{\sharp i}\zeta\right.\\
%&\left.-a^4\nabla^{\sharp a}{f}\nabla^{\sharp i}\zeta\nabla_a\nabla_j\zeta\right)\,\vol{G}\\
%=&\, \int_M \left(a^4W_i\Lap\nabla^{\sharp i}\zeta-a^4\nabla^{\sharp b}{f}\nabla^{\sharp i}u\nabla_b\nabla_j\zeta\right)\,\vol{G}\\
%\leq&\, \int_M \left(\lvert W\rvert_G^2+\frac{a^8}4\lvert\Lap\nabla \zeta\rvert_G^2+\left(\frac12\lvert\nabla \zeta\rvert_G^2+\frac{a^4}2\lvert\nabla^2\zeta\rvert_G^2\right)\cdot a^2\|\nabla f\|_{L^{\infty}_G}\right)\,\vol{G}
%%using by rewriting |\nabla u\nabla f|_G^2=|\nabla u|_G^2|\nabla f|_G^2
%\end{align*}
%Once again, we rearrange, use our control $\nabla f$ and the previously obtained estimate to get
%\[\int_M \left(a^8\lvert\Lap\nabla \zeta\rvert_G^2+a^4\lvert\nabla^2\zeta\rvert_G^2\right)\,\vol{G}\lesssim (1+\epsilon a^{2-c\epsilon})\|W\|_{L^2_G}^2\lesssim \|W\|_{L^2_G}^2\]
%and the statement now follows.
%\end{proof}

\begin{corollary}[Intermediary elliptic lapse estimate with $\L$]\label{cor:ell-lapse-L}
The following estimates hold for any $l\in\{0,\dots,\change{10}\}$:
\begin{align*}
&a^4\|\Lap^{l+1} N\|_{L^2_G}+a^2\|\nabla\Lap^l N\|_{L^2_G}+\|\Lap^lN\|_{L^2_G}\\
\lesssim&\,\|\Lap^lF\|_{L^2_G}+\underbrace{\epsilon a^{-c\sqrt{\epsilon}}\|F\|_{H^{2(l-1)}_G}}_{\text{not present for }l=0}+\underbrace{\epsilon^2a^{4-c\sigma}\sqrt{\E^{(\leq 2l-4)}(\Ric,\cdot)}}_{\text{not present for }l\leq 1}\\
%&a^4\|\nabla\Lap^{l+1} N\|_{L^2_G}+a^2\|\nabla^2\Lap^l N\|_{L^2_G}+\|\nabla\Lap^l N\|_{L^2_G}\\
%\lesssim&\,\|\nabla \Lap^l F\|_{L^2_G}+\underbrace{\epsilon a^{-c\sqrt{\epsilon}}\|F\|_{H^{2(l-1)+1}_G}}_{\text{not present for }l=0}+\underbrace{\epsilon^2a^{4-c\sigma}\sqrt{\E^{(\leq 2l-3)}(\Ric,\cdot)}}_{\text{not present for }l=0,1}
%\|\Lap^2N\|_{L^2_G}a^4+\|\nabla\Lap N\|_{L^2_G}a^2+\|\Lap N\|_{L^2_G}&\lesssim \epsilon a^{-c\epsilon}\|F\|_{L^2_G}+\|\Lap F\|_{L^2_G}\\
%\|N\|_{H^2_G}&\lesssim a^{-c\sqrt{\epsilon}}\|F\|_{L^2_G}+\|\Lap F\|_{L^2_G}
\end{align*}
\end{corollary}
\begin{proof}
We prove the statement by induction over $l\in\N$: For $l=0$, the estimates immediately follow from \eqref{eq:lapse-with-op} and Lemma \ref{lem:ell-lapse-1}. %, respectively \eqref{eq:lapse-with-op-tilde} and Lemma \ref{lem:ell-lapse-2}.
Assume the statement to be satisfied up to $l-1$ for some $l\in\N_0,\,l\leq 11$. We get, applying \eqref{eq:[L,Lap]} iteratively,
\begin{align*}
\L\Lap^lN%=&\,[\L,\Lap]\Lap^{l-1}N+\Lap (\L\Lap^{l-1}N)\\
%=&\,2\langle\nabla F,\nabla\Lap^{l-1} N\rangle+\Lap F\cdot\Lap^{l-1}N+\Lap[\L,\Lap]\Lap^{l-2}N+\Lap^2(\L\Lap^{l-2}N)\\
=&%\,\dots=%
\sum_{I=1}^{2l-1}\nabla^IF\ast\nabla^{2l-I}N+(N+1)\Lap^lF\,.
\end{align*}
Applying Lemma \ref{lem:ell-lapse-1} to $\zeta=\Lap^l N$ as well as Lemma \ref{lem:lapse-maxmin} yields
\[a^4\|\Lap^{l+1}N\|_{L^2_G}+a^2\|\nabla\Lap^{l}N\|_{L^2_G}+\|\Lap^l N\|_{L^2_G}\lesssim \sum_{I=1}^{2l-1}\|\nabla^IF\ast\nabla^{2l-I}N\|_{L^2_G}+\|\Lap^lF\|_{L^2_G}\]
%Looking at the first term, we can apply the strong $C_G$-norm estimates \eqref{eq:APmidPsi} and \eqref{eq:APmidSigma} for $I\leq 12$ to get
%\[\|\nabla^I F\ast\nabla^{2l-I}N\|_{L^2_G}\lesssim \epsilon a^{-c\sqrt{\epsilon}}\|\nabla^{2l-I}N\|_{L^2_G}\,;\]
%for $I\geq 13$, the bootstrap assumption \eqref{eq:BsN} yields
%\[\|\nabla^I F\ast\nabla^{2l-I}N\|_{L^2_G}\lesssim \epsilon a^{4-c{\sigma}}\|\nabla^{I}F\|_{L^2_G}\,.\]
Hence, using \eqref{eq:Sobolev-norm-equiv-zeta2l+1} (replacing $l$ with $l-1$) and  \eqref{eq:BsN}, \eqref{eq:APmidPsi} and \eqref{eq:APmidSigma} to estimate low order terms, we get
\begin{align*}
&a^4\|\Lap^{l+1}N\|_{L^2_G}+a^2\|\nabla\Lap^{l}N\|_{L^2_G}+\|\Lap^l N\|_{L^2_G}\\ 
\lesssim&\,\|\Lap^lF\|_{L^2_G}+\epsilon a^{-c\sqrt{\epsilon}}\left(\sum_{m=0}^{l-1}\|\nabla\Lap^mN\|_{L^2_G}+\epsilon a^{4-c\sigma}\sqrt{\E^{(\leq 2l-4)}(\Ric,\cdot)}\right)\\
&\,+\epsilon a^{4-c\sigma}\left(\|F\|_{H^{2(l-1)}_G}+\|\nabla\Lap^{l-1}F\|_{L^2_G}+\epsilon a^{-c\sqrt{\epsilon}}\sqrt{\E^{(\leq 2l-4)}(\Ric,\cdot)}\right)
\end{align*}
For the top order lapse term, we can redistribute the divergent prefactor as follows:
\begin{equation*}
\epsilon a^{-c\sqrt{\epsilon}}\|\nabla\Lap^{l-1}N\|_{L^2_G}\lesssim \epsilon \|\Lap^lN\|_{L^2_G}+\epsilon a^{-2c\sqrt{\epsilon}}\|\Lap^{l-1}N\|_{L^2_G}
\end{equation*}
The lower order lapse terms as well as $\|\nabla\Lap^{l-1}F\|_{L^2_G}$ can be estimated similarly, just without having to redistribute the prefactor. Updating $c>0$ and rearranging then yields the statement at order $l$ for suitably small $\epsilon>0$, and thus the entire statement after iteration.
%\begin{align*}
%&\,a^4\|\Lap^{l+1}N\|_{L^2_G}+a^2\|\nabla\Lap^{l}N\|_{L^2_G}+(1-K\epsilon)\|\Lap^l N\|_{L^2_G}\\
%\lesssim&\,\epsilon a^{-c\sqrt{\epsilon}}\sum_{m=0}^{l-1}\|\Lap^m N\|_{L^2_G}+\|\Lap^lF\|_{L^2_G}+\|F\|_{H^{2(l-1)}_G}+\epsilon^2a^{4-c\sigma}\sqrt{\E^{(\leq 2l-4)}(\Ric,\cdot)}
%\end{align*}
%The inequality now follows for $l$ by applying the assumption that it is satisfied up to $l-1$ for the first term, assuming $\epsilon>0$ to have been suitably small and adjusting constants. This then shows the statement for any $l\leq 11$.
%%The second inequality follows completely analogously.
\end{proof}

\begin{corollary}[Lapse energy estimates with $\L$]\label{cor:en-est-lapse}
For any $l\in\{0,\dots,\change{9}\}$, one has
\begin{align*}
&\,a^8\E^{(2(l+1))}(N,\cdot)+a^4\E^{(2l+1)}(N,\cdot)+\E^{(2l)}(N,\cdot)\\
\lesssim&\,\epsilon^2\E^{(2l)}(\Sigma,\cdot)+\E^{(2l)}(\phi,\cdot)+\underbrace{\epsilon^2a^{-c\sqrt{\epsilon}}\left[\E^{(\leq 2(l-1))}(\Sigma,\cdot)+\E^{(\leq 2(l-1))}(\phi,\cdot)\right]}_{\text{not present for }l=0}\\
&\,+\underbrace{\left(\epsilon^4a^{-c\sqrt{\epsilon}}+\epsilon^2a^{8-c\sigma}\right)\E^{(\leq 2l-3)}(\Ric,\cdot)}_{\text{not present for }l\leq 1}
\end{align*}
%as well as
%\begin{align*}
%&\,a^8\E^{(2(l+1)+1)}(N,\cdot)+a^4\E^{(2(l+1))}(N,\cdot)+\E^{(2l+1)}(N,\cdot)\\\lesssim&\,\epsilon^2\E^{(2l+1)}(\Sigma,\cdot)+\E^{(2l+1)}(\phi,\cdot)+\epsilon^2a^{-c\sqrt{\epsilon}}\left[\E^{(\leq 2l-1)}(\Sigma,\cdot)+\E^{(\leq 2l-1)}(\phi,\cdot)\right]\\
%&\,+\epsilon^4a^{-c\sqrt{\epsilon}}\E^{(\leq 2l-2)}(\Ric,\cdot)+\epsilon^2\left[a^{16-c\sigma}\E^{(2l)}(\Ric,\cdot)+a^{8-c\sigma}\E^{(\leq 2l-2)}(\Ric,\cdot)\right]
%\end{align*}
\end{corollary}
\begin{proof}
%We will prove this for even order $L=2l$, and then sketch the differences for $L=2l+1$.
Note that, by Corollary \ref{cor:ell-lapse-L}, all that needs to be done is to relate all Sobolev norms of $F$ that occur to the respective energies. Schematically, we have
\[\Lap^{l}F=16\pi(\Lap^l\Psi)(\Psi+C)+2\langle\Lap^l\Sigma,\Sigma\rangle_G+\sum_{I=1}^{2l-1}\left(\nabla^I\Psi\ast\nabla^{2l-I}\Psi+\nabla^I\Sigma\ast\nabla^{2l-I}\Sigma\right)\,.\]
For the first two terms, we can use \eqref{eq:APPsi} and \eqref{eq:APSigma} to bound $\lvert\Sigma\rvert_G$ and $\lvert\Psi+C\rvert$ by $\epsilon$ and $1$ up to constant, respectively. For the remaining terms, we similarly always bound the lower order in $L^\infty_G$ with \eqref{eq:APmidPsi}-\eqref{eq:APmidSigma} and bound the higher order with the energy estimates in Lemma \ref{lem:Sobolev-norm-equivalence-improved}. Further, we can use \eqref{eq:ibp-trick} to redistribute divergent prefactors onto energies of order $l-2$ and lower. %Note that we can estimate the term caused by $\|\nabla\Sigma\ast\nabla^{2l-1}\Sigma\|_{L^2_G}^2$ with \eqref{eq:APmidSigma} and \eqref{eq:Sobolev-norm-equiv-T2l+1} as
%\begin{align*}
%\|\nabla\Sigma\|_{L^\infty_G}^2\|\nabla^{2l-1}\Sigma\|_{L^2_G}^2\lesssim&\,\epsilon^2a^{-c\sqrt{\epsilon}}\left(\E^{(\leq 2l-1)}(\Sigma,\cdot)+\epsilon^2a^{-c\sqrt{\epsilon}}\E^{(\leq 2l-3)}(\Ric,\cdot)\right)\,,
%\end{align*}
%and rearranging $a^{-c\sqrt{\epsilon}}\E^{(2l-1)}(\Sigma,\cdot)$ as in \eqref{eq:ibp-trick},
%\begin{align*}
%\|\nabla\Sigma\|_{L^\infty_G}^2\|\nabla^{2l-1}\Sigma\|_{L^2_G}^2\lesssim&\,\epsilon^2\E^{(2l)}(\Sigma,\cdot)+\epsilon^2a^{-c\sqrt{\epsilon}}\E^{(\leq 2l-2)}(\Sigma,\cdot)+\epsilon^4a^{-c\sqrt{\epsilon}}\E^{(\leq 2l-3)}(\Ric,\cdot)
%\end{align*}
%and can deal the equivalent term arising from $\Psi$ similarly.
This already incurs the terms on the right hand side of the claimed estimate, and the lower order norms of $F$ only incur at equivalent or weaker error terms.
%For odd order, note that the left hand side in Lemma \ref{lem:ell-lapse-2} isn't quite a sum of roots of energies yet, unlike in the previous case. More precisely, we collect
%\begin{align*}
%\int_M|[\Lap^{l+1},\nabla]N|_G^2\,\vol{G}\lesssim \epsilon^2a^{8-c\sigma}\E^{(\leq 2l)}(\Ric,\cdot)+a^{-c\sqrt{\epsilon}}\E^{(\leq 2l-1)}(N,\cdot)
%\end{align*}
%using \todo{commutator identity}, and collect from \todo{Lemma \ref{lem:Sobolev-norm-exchange-improved} [need the inverse version]} for $\zeta=\Lap^lN$ that
%\begin{align*}
%\|\Lap^{l+1}N\|_{L^2_G}^2\lesssim \|\Lap^lN\|_{H^2_G}^2+a^{-c\sqrt{\epsilon}}\E^{(\leq 2l)}(N,\cdot)+\epsilon^2a^{8-c\sigma}\E^{(\leq 2l-1)}(\Ric,\cdot)\,.
%\end{align*}
%So, we get altogether that
%\begin{align*}
%&\,a^8\E^{(2(l+1)+1)}(N,\cdot)+a^4\E^{(2(l+1))}(N,\cdot)+\E^{(2l+1)}(N,\cdot)\\
%\lesssim&\,a^8\int_M|\Lap^{l+1}\nabla N|_G^2+|[\Lap^{l+1},\nabla]N|_G^2\,\vol{G}+a^4\|\Lap(\Lap^lN)\|_{L^2_G}^2+\|\nabla\Lap^lN\|_{L^2_G}^2\\
%\lesssim&\,a^8\|\Lap\nabla N\|_{L^2_G}^2+a^4\|\nabla^2 \Lap^lN\|_{L^2_G}^2+\|\nabla\Lap N\|_{L^2_G}^2\\
%&\,+a^{8-c\sqrt{\epsilon}}\E^{(\leq 2l)}(N,\cdot)+\epsilon^2a^{16-c\sigma}\E^{(2l)}(\Ric,\cdot)+\epsilon^2a^{12-c\sigma}\E^{(\leq 2l-1)}(\Ric,\cdot)
%\end{align*}
%Note that we can absorb the order $2l$ for the lapse in the second line into the norm in the first line by \todo{the standard integration-by-parts argument}, and similarly the Ricci energies become
%\[\epsilon^2a^{16-c\sigma}\E^{(2l)}(\Ric,\cdot)+\epsilon^2a^{8-c\sigma}\E^{(\leq 2(l-1)}(\Ric,\cdot)\,.\]
%From here, we can now work with the first line as in the even order case with analogous results, proving the statement.
\end{proof}

\subsection{Elliptic lapse estimates with $\Ltilde$}\label{subsec:lapse-Ltilde}

While the estimates in the previous subsection are useful at high orders, they are not enough to close the bootstrap assumptions for $N$. This can be achieved by deriving estimates in terms of $\Ltilde$ -- however, due to the explicit presence of Ricci terms in this version of the lapse equation, we use this to bound $N$ at \change{lower }orders. Since the arguments are largely identical to the ones above, we only sketch the proofs.

\begin{remark}\label{rem:L-Ltilde-bridge}
Note that, when replacing $f$ by $\tilde{f}$ and $F$ by $\tilde{F}$ in Remark \ref{ass:lapse}, the same statements hold for a suitable constant $K$. In fact, the bootstrap assumptions on $\Ric[G]$ and $\nabla\phi$ even imply $\|\tilde{F}\|_{C^1_G}\lesssim \epsilon a^{4-c\sigma}$, noting
\[\left\lvert R[G]+\frac23\right\rvert_G\leq\lvert G^{-1}\rvert_G\left\lvert\Ric[G]+\frac29G\right\rvert_G\lesssim \left\lvert\Ric[G]+\frac29G\right\rvert_G\]
and
\[\lvert\nabla R[G]\rvert_G=\left\lvert\nabla\left(R[G]+\frac23\right)\right\rvert_G\lesssim \left\lvert\nabla\left(\Ric[G]+\frac29G\right)\right\rvert_G\,.\]%, which more than ensures the first two points, while we just copy the third point, and the fourth can once again be ensured by choosing $\epsilon$ sufficiently small. 
\end{remark}

\begin{lemma}\label{lem:ell-lapse-Ltilde}
Any scalar functions $\zeta$ and $Z$ such that
\[\Ltilde \zeta=Z\]
holds satisfy the estimate
\begin{align*}
a^4\|\Lap\zeta\|_{L^2_G}+a^2\|\nabla \zeta\|_{L^2_G}+\|\zeta\|_{L^2_G}\lesssim&\,\|Z\|_{L^2_G}\,.%\text{resp.}\\
%a^4\|\Lap\nabla \zeta\|_{L^2_G}+a^2\|\nabla^2\zeta\|_{L^2_G}+\|\nabla \zeta\|_{L^2_G}\lesssim&\,\|Z\|_{L^2_G}
\end{align*}
\end{lemma}
\begin{proof}
The proof follows identically to Lemma \ref{lem:ell-lapse-1} %, respectively Lemma \ref{lem:ell-lapse-2},
 since all tools relating to $f$ and $F$ used in proving these statements were collected in Remark \ref{ass:lapse}, and these extend to $\Ltilde$ by Remark \ref{rem:L-Ltilde-bridge}.
\end{proof}

\begin{corollary}For $l\in\{0,\dots,\change{8}\}$
\label{cor:en-est-lapse-tilde}
\begin{align*}
a^8\E^{(2(l+1))}(N,\cdot)+a^4\E^{(2l+1)}(N,\cdot)+\E^{(2l)}(N,\cdot)\lesssim&\,a^8\E^{(\leq 2l)}(\Ric,\cdot)+{\epsilon}a^{8-c\sqrt{\epsilon}}\|\nabla\phi\|_{H^{2l}_G}^2
%\sum_{m=0}^l\|\nabla\Lap^mN\|_{L^2_G}\lesssim&\,a^4\sum_{m=0}^l\|\nabla\Lap^m(\Ric[G]+2G)\|_{L^2_G}+\sqrt{\epsilon}a^{4-c\sqrt{\epsilon}}\|\nabla\phi\|_{H^{2l+1}_G}\\
\end{align*}
\end{corollary}
\begin{proof} As in the proof of Corollary \ref{cor:ell-lapse-L}, this follows by commuting $\Ltilde$ with $\Lap^l$ iteratively and applying \eqref{eq:APmidphi} and \eqref{eq:APmidRic} to bound lower order terms within the nonlinearities.
%We again go through the same iterative procedure, where we assume the estimate to hold up to $l-1$, and noting that $l=0$ follows immediately from Lemma \ref{lem:ell-lapse-Ltilde}, \eqref{eq:APmidphi} and $\lvert R[G]+\nicefrac23\rvert\leq \sqrt{3}\lvert\Ric[G]+\nicefrac29G\rvert_G$.\\
%As in the proof of Corollary \ref{cor:ell-lapse-L}, we have using \eqref{eq:[Ltilde,Lap]}:
%\[[\Ltilde,\Lap^l]N=\sum_{I=1}^{2l-1}\nabla^I\tilde{F}\ast\nabla^IN+(N+1)\Lap^l\tilde{F}\]
%Using \eqref{eq:APmidphi} and \eqref{eq:APmidRic} (with $\lvert\nabla^I(R[G]+\nicefrac23)\rvert_G\lesssim\lvert\nabla^I(\Ric[G]+\nicefrac29G)\rvert_G$), we can bound the $L^2_G$-norm of the sum by
%\begin{equation*}
%\epsilon a^{4-c\sigma}\left(a^4\|R[G]+\nicefrac23\|_{{H}^{2l-1}_G}+\sqrt{\epsilon}a^{4-c\sqrt{\epsilon}}\|\nabla\phi\|_{{H}^{2l-1}_G}\right)+\sqrt{\epsilon}a^{4-c\sqrt{\epsilon}}\|N\|_{H^{2l-1}_G}
%\end{equation*}
%as well as
%\begin{equation*}
%\|\Lap^l\tilde{F}\|_{L^2_G}\lesssim a^4\|\Lap^l(R[G]+\nicefrac23)\|_{L^2_G}+\sqrt{\epsilon}a^{4-c\sqrt{\epsilon}}\|\nabla\phi\|_{H^{2l}_G}\,.
%\end{equation*}
%Applying Lemma \ref{lem:Sobolev-norm-equivalence-improved}, we obtain (for $l\geq 2$, with analogous estimates for $l=0,1$)
%\begin{align*}
%&\,a^4\|\Lap^{l+1}N\|_{L^2_G}+a^2\|\nabla\Lap^lN\|_{L^2_G}+\|\Lap^lN\|_{L^2_G}\\
%\lesssim&\,\sqrt{\epsilon}a^{2-c\sqrt{\epsilon}}\left(a^2\sum_{m=0}^{l-1}\|\nabla\Lap^mN\|_{L^2_G}+\epsilon a^{6-c\sigma}\sqrt{\E^{(\leq 2l-4)}(\Ric,\cdot)}\right)+\sqrt{\epsilon}a^{4-c\sqrt{\epsilon}}\|\nabla\phi\|_{H^{2l}_G}\\
%&\,+a^4\sqrt{\E^{(2l)}(\Ric,\cdot)}+\epsilon a^{8-c\sigma}\left(\sqrt{\E^{(\leq 2l-1)}(\Ric,\cdot)}+\sqrt{\epsilon} a^{-c\sqrt{\epsilon}}\sqrt{\E^{(\leq 2l-3)}(\Ric,\cdot)}\right)
%\end{align*}
%The statement follows by inserting the iteration assumption to estimate the first term on the right hand side, estimating all occurring curvature terms by $a^4\sqrt{\E^{(\leq 2l)}(\Ric,\cdot)}$, squaring the inequality and adjusting constants.
\end{proof}
%\begin{remark}
%\todo{We will use this estimate to improve the $C$-norm bootstrap assumptions on $N$ towards the end of the paper. The reason we refrain from doing this immediately is that even switching from the norms on Laplacians above to a full Sobolev norm requires $C$-norm control on the lapse, and improved control to yield improvements. Hence, we need Sobolev embeddings where the behaviour of $G-\gamma$ is already improved, which we only have at the end of the argument.}
%\end{remark}

\section{Big Bang stability: Energy and norm estimates}\label{sec:en-est}

In this section, we derive energy estimates for matter variables and the geometric quantities as well as Sobolev norm estimates for spatial derivatives of $\phi$ and for metric quantities. To derive all of the inequalities in this section beside the elliptic inequality in Lemma \ref{lem:en-est-Sigma-top} \change{and the bound on $\nabla\phi$ in Lemma \ref{lem:norm-est-nablaphi}}, we will use the same basic strategy. Hence, we give a brief overview on the form our integral inequalities are going to take and how we intend to obtain \change{improved energy bounds }from there:

\begin{remark}[Integral inequalities and the Gronwall argument]\label{rem:en-est-strat}
Let $\mathcal{F}_L$ denote an energy or a squared Sobolev(-type) norm at derivative level $L\in2\N$, for example, $\E^{(L)}(\phi,\cdot)$. To derive an integral inequality for $\mathcal{F}_L$, we will take its time derivative, apply the respective commuted evolution equations in the integrand, estimate the resulting terms and integrate that inequality. Schematically, the resulting integral inequalities for $\mathcal{F}_L$ then take the following form:
\begin{align*}
&\,\mathcal{F}_L(t)+\int_t^{t_0}\langle\text{ultimately nonnegative contributions}\rangle\,ds\\
\lesssim&\,\mathcal{F}_L(t_0)+\int_t^{t_0}\left(\epsilon^\frac18a(s)^{-3}+a(s)^{-1-c\sqrt{\epsilon}}\right)\mathcal{F}_L(s)\,ds\\
&\,+\int_t^{t_0}a(s)^{-3}\langle\text{other energies/squared Sobolev norms at same derivative level}\rangle\,ds\\
&\,+\int_t^{t_0}a(s)^{-3-c\sqrt{\epsilon}}\langle\text{energies/squared Sobolev norms at derivative levels up to }L-2\rangle\,ds
\end{align*}

For some inequalities, we will not be able to derive any \changefinal{beneficial }$\epsilon$-prefactors in the penultimate line. \change{For example, for $\E^{(L)}(\Sigma,\cdot)$, linear lapse terms in the evolution of $\Sigma$ incur a term of the form
\[\int_t^{t_0}a(s)^{-3}\cdot a(s)^4\|\Lap^\frac{L}2N\|_{\dot{H}^2_G}\cdot\sqrt{\E^{(L)}(\Sigma,s)}\,ds\]
on the right hand side, which after applying lapse energy estimates creates $\epsilon^{-\frac18}\E^{(L)}(\phi,\cdot)$ on the right}. However, combining the respective inequalities for the core energy mechanism at each derivative level with appropriate \changefinal{$\epsilon$-weights}, this will then combine to an inequality of the following form for a total energy, which we informally denote by $\mathcal{F}_{\text{total},L}$:

\change{\begin{align*}
\numberthis\label{eq:toy-energy}&\,\mathcal{F}_{\text{total},L}(t)+\int_t^{t_0}\langle\text{nonnegative quantity}\rangle\,ds\\
\lesssim&\,\mathcal{F}_{\text{total},L}(t_0)+\int_t^{t_0}\left(\epsilon^\frac18a(s)^{-3}+a(s)^{-1-c\sqrt{\epsilon}}\right)\mathcal{F}_{\text{total},L}(s)\,ds\\
&\,+\underbrace{\int_t^{t_0}\epsilon^\frac18 a(s)^{-3-c\epsilon^\frac18}\langle\text{already improved terms}\rangle\,ds}_{\text{not present for }L=0}\\
&\,+\epsilon^\frac14\mathcal{F}_{\text{total},L}(t)+\underbrace{\sqrt{\epsilon}\cdot\langle\text{small lower order terms}\rangle(t)}_{\text{not present for }L=0}
\end{align*}
In the mentioned example, \changefinal{multiplying $\E^{(L)}(\Sigma,\cdot)$ with the weight $\epsilon^\frac14$, in turn, mitigates the otherwise offending term to $\epsilon^\frac18a(s)^{-3}\E^{(L)}(\phi,s)$, which can be absorbed into the first line.\\}}

%Here, the already improved terms are mainly terms of derivative order $L-2$ or lower. When deriving additional bounds on norms outside of the core energy mechanism (e.g.\,metric quantities), this can also contain terms at the same derivative order as $\mathcal{F}_{\text{total},L}$ if their behaviour can be improved beforehand.\\
\changefinal{Furthermore, $\epsilon^\frac14\mathcal{F}_{\text{total},L}(t)$ in the penultimate line of (6-1) can be absorbed into the left-hand side after updating the implicit constant in \enquote{$\lesssim$}. Applying the Gronwall lemma (see Lemma \ref{lem:gronwall}) and the initial data assumption (which implies $\mathcal{F}_{\text{total},L}(t_0)\lesssim \epsilon^4$) then yields
\begin{align*}
\mathcal{F}_{\text{total},L}(t)\lesssim&\,\biggr(\epsilon^4+\underbrace{\int_t^{t_0}\epsilon^\frac18 a(s)^{-3-c\epsilon^\frac18}\langle\text{already improved terms}\rangle\,ds+\sqrt{\epsilon}\cdot\langle\text{lower order terms}\rangle(t)}_{\text{not present for }L=0}\biggr)\\
&\,\cdot\exp\left(K\cdot \int_t^{t_0}\epsilon^\frac18a(s)^{-3}+a(s)^{-1-c\sqrt{\epsilon}}\,ds\right)
\end{align*}
for some constant $K>0$. \eqref{eq:log-est} and \eqref{eq:a-integrals}, the exponential factor can be bounded by $a^{-c\epsilon^\frac18}$, up to constant and updating $c>0$.
%Regarding the terms at level $L+1$, we can insert the bootstrap assumptions for energies and Sobolev norms (see \eqref{eq:BsEnSF}-\eqref{eq:BsEnN}), and since $a^{-1-c\sigma}$ is integrable, this results in an error term that is at worst bounded by $\epsilon^{2+\frac14}=\epsilon^\frac94$. In fact, it will turn out to be bounded by $\epsilon^\frac{11}4$ due to the precise nature of these terms and scaling hierarchy in \eqref{eq:BsEnSF}-\eqref{eq:BsEnRic}.
 Hence}, for $L=0$, this implies $\mathcal{F}_{\text{total},0}\lesssim\change{\epsilon^4}a^{-c\epsilon^\frac18}$, and thus leads to improved bounds for base level energy quantities (see Remark \ref{rem:bs-strategy} for the precise scaling hierarchy that will achieve). 
By iterating this argument for $L>0$, the already improved terms will then be bounded (at worst) by $\change{\epsilon^4}a^{-c\epsilon^\frac18}$ , and \eqref{eq:a-integrals} shows that the first line can be bounded by $\change{\epsilon^4}a^{-c\epsilon^\frac18}$ after updating $c$. \change{This allows us }to bound $\mathcal{F}_{\text{total},L}$ by $\change{\epsilon^4}a^{-c\epsilon^\frac18}$ for any $L$ up to and including top order. \\

\change{Finally, we mention that, to control energies at order $L$, we need to consider scaled energies at order $L+1$ within $\mathcal{F}_{total,L}$ -- this arises since the scalar field occurs at first order in the evolution equations for $\RE$ and $\RB$. We avoid losing derivatives by employing the div-curl-estimate in Lemma \ref{lem:en-est-Sigma-top} at order $L+1$, which allows us to control $a^4\E^{(L+1)}(\Sigma,\cdot)$ by quantities at order $L$. This is precisely what generates the non-integral terms in the schematics above. We note that it is crucial that the scalar field occurs at no worse scaling than $a^{-1}$ in \eqref{eq:comeq-RE}-\eqref{eq:comeq-RB} -- else, moving to these time-scaled estimates at order $L+1$ would lose too many powers of $a$ and lead to exponentially divergent terms after applying the Gronwall argument.}
\end{remark}
Recall that $L^2_G$-norm estimates for error terms arising in the Laplace-commuted equations in Lemma \ref{lem:laplace-commuted-eq} are collected in Section \ref{subsec:L2-error-est}. Low order estimates (in particular estimates for $L=2$) could often be improved if needed by more carefully avoiding curvature error terms, but we refrain from doing so where it is not necessary to keep estimates as unified as possible. \delete{Moreover [...]}%% from here on out, we use $M$ to denote energy estimates at the top derivative level to easily distinguish them from lower levels while keeping the estimates easy to compare. We will later choose $M=20$.%, but will only use that it is even and between $4$ and $20$ until then.

\subsection{Integral and energy estimates for the scalar field}\label{subsec:en-SF}

\subsubsection{Scalar field energy estimates}\label{subsubsec:en-SF}

Over the following two lemmas, we prove the core energy estimates to control the matter variables, which are immediately prepared differently at base, intermediate and top order for the total energy estimates in Section \ref{sec:bs-imp}.

\begin{lemma}\label{lem:en-est-SF}
[\change{Even order scalar field energy estimates}] Let $t\in(t_{Boot},t_0]$. Then, one has
\begin{align*}
\numberthis\label{eq:en-est-SF0}&\,\E^{(0)}(\phi,t)+\int_t^{t_0}\dot{a}(s)a(s)^{3}\E^{(1)}(N,s)+\frac{\dot{a}(s)}{a(s)}\E^{(0)}(N,s)\,ds\\
\lesssim&\,\E^{(0)}(\phi,t_0)+\int_t^{t_0}\epsilon a(s)^{-3}\E^{(0)}(\phi,s)+\epsilon a^{-3}\E^{(\change{0})}(\Sigma,s)\,ds\,.
\end{align*}
Further, for any $L\in 2\N,\,2\leq L\leq \change{18}$, the following estimate is satisfied: 
\begin{align*}
\numberthis\label{eq:en-est-SF}&\,\E^{(L)}(\phi,t)+\int_t^{t_0}\dot{a}(s)a(s)^{3}\E^{(L+1)}(N,s)+\frac{\dot{a}(s)}{a(s)}\E^{(L)}(N,s)\,ds\\
\lesssim&\,\E^{(L)}(\phi,t_0)+\int_t^{t_0}\left(\change{\epsilon}a(s)^{-3}+a(s)^{-1-c\sqrt{\epsilon}}\right)\E^{(L)}(\phi,s)\,ds\\
&\,+\int_t^{t_0}\epsilon a(s)^{-3}\E^{(L)}(\Sigma,s)+\epsilon^\frac32 a(s)^{-3}\E^{(L-2)}(\Ric,s)\,ds\\
&\,+\int_t^{t_0}\sqrt{\epsilon}a(s)^{-3-c\sqrt{\epsilon}}\E^{(\leq L-2)}(\phi,s)+\epsilon a(s)^{-3-c\sqrt{\epsilon}}\E^{(\leq L-2)}(\Sigma,s)\,ds\\
&\underbrace{+\int_t^{t_0}\epsilon^\frac32a(s)^{-3-c\sqrt{\epsilon}}\E^{(\leq L-4)}(\Ric,s)\,ds}_{\text{if }L\geq4}
%&\,+\begin{cases}\displaystyle\int_t^{t_0}\left(\epsilon^\frac32a(s)^{-3-c\sqrt{\epsilon}}+\epsilon a^{-1-c\sqrt{\epsilon}}\right)\|\nabla\phi\|_{L^2_G(\Sigma_s)}^2\,ds & L=2\\
%\displaystyle\int_t^{t_0}\,\left[\left(\epsilon^\frac32a(s)^{-3}+\epsilon a(s)^{-1-c\sqrt{\epsilon}}\right)\left(\|\Lap^{\frac{L}2}\phi\|_{L^2_G(\Sigma_s)}^2+a^{-c\sqrt{\epsilon}}\|\nabla\phi\|_{H^{L-3}_G(\Sigma_s)}^2\right)\right.&\\
%\,\phantom{+\int_t^{t_0}}+\left.\epsilon^\frac32a(s)^{-3-c\sqrt{\epsilon}}\E^{(\leq L-4)}(\Ric,s)\right]\,ds& L\geq 4
%\end{cases}
%%\langle \text{SF-Err}\rangle_L(s)\,ds
\end{align*}
%%with
%%\begin{equation*}
%%\langle\text{SF-Err}\rangle_2(s)=\left(\epsilon^\frac32a(s)^{-3-c\sqrt{\epsilon}}+\epsilon a^{-1-c\sqrt{\epsilon}}\right)\|\nabla\phi\|_{L^2_G(\Sigma_s)}^2\\
%%\end{equation*}
%%and, for $L\geq 4$,
%%\begin{align*}
%%\langle\text{SF-Err}\rangle_L(s)=&\,\left(\epsilon^\frac32a(s)^{-3}+\epsilon a(s)^{-1-c\sqrt{\epsilon}}\right)\left(\|\Lap^{\frac{L}2}\phi\|_{L^2_G(\Sigma_s)}^2+a^{-c\sqrt{\epsilon}}\|\nabla\phi\|_{H^{L-3}_G(\Sigma_s)}^2\right)\\
%%&\,+\epsilon^\frac32a(s)^{-3-c\sqrt{\epsilon}}\E^{(\leq L-4)}(\Ric,s)
%%\end{align*}
\end{lemma}
\begin{remark}
This proof relies on two mechanisms: Firstly, we use the structure of the wave equation and integration by parts to cancel the highest order scalar field derivative terms. Getting this cancellation is what necessitates scaling the potential term in the scalar field energy by $a^4$. Secondly, we deal with the highest order lapse terms using the elliptic structure of the (Laplace-commuted) lapse equation -- both in an indirect way by invoking the elliptic energy estimate in Corollary \ref{cor:en-est-lapse} as well as by directly inserting \eqref{eq:comeq-lapse} to cancel some ill-behaved terms. While the framework significantly differs from the scalar field energy estimates \cite{Speck2018}, these two core mechanisms also appear there and play similarly crucial roles.
\end{remark}
\begin{proof}
Since the arguments are essentially the same, we will only write down the proof for $L\geq 2$ in full and make short comments throughout the argument which terms do not occur for $L=0$.\\
We use the evolution equations \eqref{eq:comeq-Psi-even} and \eqref{eq:comeq-nablaphi-even} and Lemma \ref{lem:delt-int} to compute, for $L\geq 2$,
\begin{subequations}
\begin{align*}
-\del_t\E^{(L)}(\phi,\cdot)=\int_M&\,-2\del_t\Lap^{\frac{L}2}\Psi\cdot\Lap^{\frac{L}2}\Psi-2a^4\langle\del_t\nabla\Lap^{\frac{L}2}\phi,\nabla\Lap^{\frac{L}2}\phi\rangle_G-a^4(\del_tG^{-1})^{ij}\nabla_i\Lap^{\frac{L}2}\phi\nabla_j\Lap^{\frac{L}2}\phi\\
&\,-3N\frac{\dot{a}}a\left[\lvert\Lap^{\frac{L}2}\Psi\rvert^2+a^4\lvert\nabla\Lap^\frac{L}2\phi\rvert_G^2\right]-4\frac{\dot{a}}a\cdot a^4\lvert\nabla\Lap^\frac{L}2\phi\rvert_G^2\,\vol{G}\\
%%%%%%%%%%%%%%%%%%%%%%%%%%%
\numberthis\label{eq:diff-eq-SF1}=\int_M&\, \left(-2a(N+1)\Lap^{\frac{L}2+1}\phi-2a\langle\nabla\Lap^{\frac{L}2}N,\nabla\phi\rangle_G+6C\frac{\dot{a}}a\Lap^{\frac{L}2}N\right)\cdot(\Lap^{\frac{L}2}\Psi)\\
\numberthis\label{eq:diff-eq-SF2}&\,-2a(N+1)\langle\nabla\Lap^{\frac{L}2}\Psi,\nabla\Lap^{\frac{L}2}\phi\rangle_G-2Ca\langle\nabla\Lap^{\frac{L}2}N,\nabla\Lap^{\frac{L}2}\phi\rangle_G\\
\numberthis\label{eq:diff-eq-SF3}&\,-2\left(\mathfrak{P}_{L,Border}+\mathfrak{P}_{L,Junk}\right)\cdot\Lap^{\frac{L}2}\Psi-2a^4\langle\mathfrak{Q}_{L,Border}+\mathfrak{Q}_{L,Junk},\nabla\Lap^{\frac{L}2}\phi\rangle_G\\
\numberthis\label{eq:diff-eq-SF4}&\,+2(N+1)a\cdot(\Sigma^\sharp)^{ij}\nabla_i\Lap^{\frac{L}2}\phi\nabla_j\Lap^{\frac{L}2}\phi-2N\frac{\dot{a}}a\cdot a^4\lvert\nabla\Lap^{\frac{L}2}\phi\rvert_G^2\\
\numberthis\label{eq:diff-eq-SF5}&\,-3N\frac{\dot{a}}a\left[\lvert\Lap^{\frac{L}2}\Psi\rvert^2+a^4\lvert\nabla\Lap^\frac{L}2\phi\rvert_G^2\right]-4\frac{\dot{a}}a\cdot a^4\lvert\nabla\Lap^\frac{L}2\phi\rvert_G^2\,\vol{G}\,.
\end{align*}
\end{subequations}
Note that, for $L=0$, the equivalent equality holds where the borderline and junk terms are replaced by $-2a^4\Psi\langle\nabla N,\nabla\phi\rangle_G$ (to verify this, insert \eqref{eq:REEqWave} and \eqref{eq:REEqNablaPhi} instead of \eqref{eq:comeq-Psi-even} and \eqref{eq:comeq-nablaphi-even}). We now go through \eqref{eq:diff-eq-SF1}-\eqref{eq:diff-eq-SF5} term by term:\\
After integrating by parts, the first term in \eqref{eq:diff-eq-SF1} reads
\begin{equation}\label{eq:SF-cancellation-trick}
\int_M 2a(N+1)\langle\nabla\Lap^{\frac{L}2}\phi,\nabla\Lap^{\frac{L}2}\Psi\rangle_G+2a\langle\nabla N,\nabla\Lap^{\frac{L}2}\phi\rangle_G\cdot\Lap^{\frac{L}2}\Psi\,\vol{G}\,.
\end{equation}
The first term \textbf{precisely} cancels the first term in \eqref{eq:diff-eq-SF2}, while we can use the bootstrap assumption \eqref{eq:BsN} to estimate the other term in \eqref{eq:SF-cancellation-trick} up to constant by
\begin{equation*}
\epsilon a^{3-c\sigma}\cdot a^2\|\nabla\Lap^{\frac{L}2}\phi\|_{L^2_G}\|\Lap^{\frac{L}2}\Psi\|_{L^2_G}\lesssim\epsilon a^{3-c\sigma}\E^{(L)}(\phi,\cdot)\,.
\end{equation*}
For the second term in \eqref{eq:diff-eq-SF1}, we use \eqref{eq:APmidphi} to estimate $\nabla\phi$ and Corollary \ref{cor:en-est-lapse} at order $L$ to deal with the lapse, getting
\begin{align*}
\left\lvert\int_M 2a\langle\nabla\Lap^{\frac{L}2}N,\nabla\phi\rangle_G\cdot \Lap^{\frac{L}2}\Psi\,\vol{G}\right\rvert\lesssim&\,\sqrt{\epsilon}a^{-1-c\sqrt{\epsilon}}\sqrt{a^4\E^{(L+1)}(N,\cdot)}\sqrt{\E^{(L)}(\phi,\cdot)}\\
\lesssim&\,\epsilon a^{-1}\cdot a^4\E^{(L+1)}(N,\cdot)+a^{-1-c\sqrt{\epsilon}}\E^{(L)}(\phi,\cdot)\\
\lesssim&\,\epsilon^3a^{-1}\E^{(L)}(\Sigma,\cdot)+a^{-1-c\sqrt{\epsilon}}\E^{(L)}(\phi,\cdot)\\
&\,+\epsilon^2a^{-1-c\sqrt{\epsilon}}\E^{(\leq L-2)}(\Sigma,\cdot)+\epsilon^2a^{-1-c\sqrt{\epsilon}}\E^{(\leq L-2)}(\phi,\cdot)\\
&\,+\underbrace{\epsilon^2a^{-1-c\sqrt{\epsilon}}\E^{(\leq L-3)}(\Ric,\cdot)}_{\text{not present for }L=2}\,.
\end{align*}
Repeating this argument for $L=0$, the last two lines do not appear.\\

To deal with the remaining term in \eqref{eq:diff-eq-SF1}, we can insert the following zero on the right hand side of the differential equality, where the equality \eqref{eq:SF-en-lapse-trick} holds due to \eqref{eq:comeq-lapse}:
\begin{align*}
0=&\,-\frac3{8\pi}\dot{a}a^3\int_M\div_G\left(\nabla\Lap^{\frac{L}2} N\cdot \Lap^{\frac{L}2}N\right)\,\vol{G}\\
=&\,-\frac3{8\pi}\dot{a}a^3\int_M \Lap^{{\frac{L}2}+1}N\cdot\Lap^\frac{L}2N+\lvert\nabla\Lap^{\frac{L}2}N\rvert_G^2\,\vol{G}\\
=&\numberthis\label{eq:SF-en-lapse-trick}\,\int_M -\frac3{8\pi}\dot{a}a^3\lvert\nabla\Lap^{\frac{L}2}N\rvert_G^2-\frac{3}{8\pi}\left(\frac13\dot{a}a^3+12\pi C^2\frac{\dot{a}}a\right)\lvert\Lap^{\frac{L}2}N\rvert^2\\
&\quad -6 C\frac{\dot{a}}a\Lap^{\frac{L}2}N\cdot \Lap^{\frac{L}2}\Psi-\frac3{8\pi}\dot{a}a^3\left[\mathfrak{N}_{L,Border}+\mathfrak{N}_{L,Junk}\right]\cdot\Lap^{\frac{L}2}N\,\vol{G}
\end{align*}
Note that the first line has a negative sign, so (after absorbing a few terms into it without changing the sign, see namely lapse quantities in \eqref{eq:en-est-SF-Friedman-ineq} and \eqref{eq:diff-ineq-SF1}), we pull it to the left hand side of the differential inequality. Further, the first term in the second line of \eqref{eq:SF-en-lapse-trick} precisely cancels the third term in \eqref{eq:diff-eq-SF1}. That leaves the borderline and junk terms in \eqref{eq:SF-en-lapse-trick}, for which we use \eqref{eq:L2-Border-N} and \eqref{eq:L2-junk-N} (along with $\dot{a}\simeq a^{-2}$ due to \eqref{eq:Friedman}) to get, for $L\geq 4$,
\begin{align*}
&\,\frac{3}{8\pi}\dot{a}a^3\left\lvert\int_M\left[\mathfrak{N}_{L,Border}+\mathfrak{N}_{L,Junk}\right]\cdot\Lap^{\frac{L}2}N\,\vol{G}\right\rvert\\
\lesssim&\,\epsilon a^{-3}\left[\E^{(L)}(\phi,\cdot)+\E^{(L)}(\Sigma,\cdot)+\E^{(L)}(N,\cdot)\right]\\
&+\,\epsilon a^{-3-c\sqrt{\epsilon}}\left[\E^{(\leq L-2)}(\phi,\cdot)+\E^{(\leq L-2)}(\Sigma,\cdot)+\E^{(\leq L-2)}(N,\cdot)\right]\\
&\,+\underbrace{\epsilon^3 a^{-3}\E^{(\leq L-2)}(\Ric,\cdot)+\epsilon^3 a^{-3-c\sqrt{\epsilon}}\E^{(\leq L-3)}(\Ric,\cdot)}_{\text{not present for }L=2}\,.
\end{align*}
Again, the same estimate holds for $L=0$ with the last two lines dropped.

From \eqref{eq:diff-eq-SF1}-\eqref{eq:diff-eq-SF2}, only the term $-2Ca\langle\nabla\Lap^{\frac{L}2}N,\nabla\Lap^\frac{L}2\phi\rangle_G$ still needs to be handled: Using the inequality \eqref{eq:diff-ineq-Friedman} arising from the Friedman equation, we can estimate this by
\begin{equation}\label{eq:en-est-SF-Friedman-ineq}
\int_M2\sqrt{\frac{3}{4\pi}}\dot{a}a^3\lvert\nabla\Lap^{\frac{L}2}N\rvert_G\lvert\nabla\Lap^{\frac{L}2}\phi\rvert_G\,\vol{G}\leq \int_M4\dot{a}a^3\lvert\nabla\Lap^\frac{L}2\phi\rvert_G^2+\frac{3}{16\pi}{\dot{a}}a^3\lvert\nabla\Lap^{\frac{L}2} N\rvert_G^2\,\vol{G}\,.
\end{equation}
Note that the first term precisely cancels the final term in \eqref{eq:diff-eq-SF5}, while the second term can be absorbed into the first term in \eqref{eq:SF-en-lapse-trick} while preserving that term's sign.\\

To bound the error terms in \eqref{eq:diff-eq-SF3}, we insert the borderline term estimates \eqref{eq:L2-Border-P-even} and \eqref{eq:L2-Border-Q-even} as well as the junk term estimates \eqref{eq:L2-junk-P-even} and \eqref{eq:L2-junk-Q-even}, where \eqref{eq:ibp-trick} is used to estimate odd order by even order energies where needed. Furthermore, observe that we can estimate the $\mathfrak{Q}_L$-terms as
\[\left(a^2\|\mathfrak{Q}_L\|_{L^2_G}\right)\cdot \sqrt{\E^{(L)}(\phi,\cdot)},\]
so all borderline and junk terms arising from it, beside the scalar field \change{energies}, are dominated by terms occurring elsewhere.\\
Finally, all terms that remain, namely \eqref{eq:diff-eq-SF4} and the first term in \eqref{eq:diff-eq-SF5}, can be bounded by $\epsilon a^{-3}\E^{(L)}(\phi,\cdot)$ due to the strong base level estimate \eqref{eq:APSigma} and \eqref{eq:BsN}. In summary, and always only keeping the worst terms for each energy and squared norm, this yields for $L\geq 4$:
\begin{subequations}
\begin{align*}
&-\del_t\E^{(L)}(\phi,\cdot)+\dot{a}a^{3}\E^{(L+1)}(N,\cdot)+\frac{\dot{a}}a\E^{(L)}(N,\cdot)\\
\numberthis\label{eq:diff-ineq-SF1}\lesssim&\,\left(\change{\epsilon}a^{-3}+a^{-1-c\sqrt{\epsilon}}\right)\E^{(L)}(\phi,\cdot)+\left(\epsilon a^{-3}+\sqrt{\epsilon}a^{-1-c\sqrt{\epsilon}}\right)\left(a^4\E^{(L+1)}(N,\cdot)+\E^{(L)}(N,\cdot)\right)\\
\numberthis\label{eq:diff-ineq-SF2}&\,+\epsilon a^{-3}\E^{(L)}(\Sigma,\cdot)+\epsilon^\frac32 a^{-3}\E^{(L-2)}(\Ric,\cdot)+%\sqrt{\epsilon}
\change{\epsilon}a^{-3-c\sqrt{\epsilon}}\E^{(\leq L-2)}(\phi,\cdot)
\\
\numberthis\label{eq:diff-ineq-SF3}&\,\change{%+\epsilon^\frac32a^{-3-c\sqrt{\epsilon}}\|\nabla\Lap^{\frac{L}2-1}\phi\|^2_{L^2_G}+\epsilon a^{-1-c\sqrt{\epsilon}}\|\nabla\phi\|_{H^{L-2}_G}^2\\
+\epsilon a^{-3-c\sqrt{\epsilon}}\E^{(\leq L-2)}(\Sigma,\cdot)+\left[\epsilon a^{-3-c\sqrt{\epsilon}}+\sqrt{\epsilon}a^{-1-c\sqrt{\epsilon}}\right]\E^{(\leq L-2)}(N,\cdot)}\\
\numberthis\label{eq:diff-ineq-SF4}&\,\underbrace{
%\epsilon^\frac32a^{-3-c\sqrt{\epsilon}}\|\nabla\phi\|^2_{H^{L-3}_G}}
\change{+\epsilon^\frac32a^{-3-c\sqrt{\epsilon}}\E^{(\leq L-4)}(\Ric,\cdot)}}_{\text{not present for }L=2}
\end{align*}
\end{subequations}
The lapse energies in \eqref{eq:diff-ineq-SF1} can now also be absorbed into those on the left hand side of the inequality by updating the implicit constant in \enquote{$\lesssim$}. We can treat the lower order lapse energies in \change{\eqref{eq:diff-ineq-SF3} }with Corollary \ref{cor:en-est-lapse} and see that the resulting terms are all dominated by terms we already have on the right hand side of the inequality above.\\

%\noindent For $L\geq 4$, we can estimate the second term in \eqref{eq:diff-ineq-SF3} (updating $c$) by
%\[\epsilon^\frac32a^{-3-c\sqrt{\epsilon}}\|\Lap^{\frac{L}2}\phi\|_{L^2_G}\|\Lap^{\frac{L}2-1}\phi\|_{L^2_G}\lesssim \epsilon^\frac32a^{-3}\|\Lap^{\frac{L}2}\phi\|_{L^2_G}^2+\epsilon^\frac32a^{-3-c\sqrt{\epsilon}}\|\nabla\phi\|_{H^{L-3}_G}^2\,.\]
%The last term in \eqref{eq:diff-ineq-SF3}, can be controlled using \eqref{eq:Sobolev-norm-equiv-zeta2l+1} by
%\[\lesssim \epsilon a^{-1-c\sqrt{\epsilon}}\|\nabla\Lap^{\frac{L}2-1}\phi\|_{L^2_G}^2+\epsilon a^{-1-c\sqrt{\epsilon}}\|\nabla\phi\|_{H^{L-3}}^2\]
%up to curvature terms dominated by those in \eqref{eq:diff-ineq-SF5}, and the first term can be dealt with as before. 
\change{Inserting }these estimates and integrating over $(t,t_0]$ then yields \eqref{eq:en-est-SF} for $L\geq 4$, and the statement for $L=2$ is obtained completely analogously.
%Consequently, for $L\geq 4$, the scalar field norms in \eqref{eq:diff-ineq-SF3} and \eqref{eq:diff-ineq-SF5} can be replaced by
%\[\left(\epsilon^\frac32a(s)^{-3}+\epsilon a(s)^{-1-c\sqrt{\epsilon}}\right)\left(\|\Lap^\frac{L}2\phi\|_{L^2_G}^2+a^{-c\sqrt{\epsilon}}\|\nabla\phi\|_{H^{L-3}_G}^2\right)\,.\]
%\eqref{eq:en-est-SF} then follows for $L\geq 4$ by integrating over $(t,t_0]$, and the statement for $L=2$ is obtained completely analogously.\\

\noindent As mentioned earlier, \eqref{eq:diff-eq-SF3} is replaced by the following term for $L=0$:
\[\int_M -2\change{a}\Psi\langle\nabla N,\nabla\phi\rangle_G\,\vol{G}\lesssim\epsilon a^{-3}\int_Ma^2\lvert\nabla N\rvert_G\cdot a^2\lvert\nabla\phi\rvert_G\,\vol{G}\lesssim \epsilon \dot{a}a^3 \E^{(1)}(N,\cdot)+\epsilon a^{-3}\E^{(0)}(\phi,\cdot)\,,\]
Here, we applied \eqref{eq:APPsi} and \eqref{eq:Friedman}. Both of these terms can be absorbed into terms that are already present, and \eqref{eq:en-est-SF0} then follows by dealing with terms in $\del_t\E^{(0)}(\phi,\cdot)$ as described and integrating. 
\end{proof}

%\begin{remark}
%\delete{The greatest technical issue with the estimate \eqref{eq:en-est-SF} comes from the scalar field Sobolev norms on the right hand side of the inequality -- on the one hand, we can't replace these by energy terms since we would have to incur $a(s)^{-4}$ to match the way such expressions occur in our energies. On the other hand, one can intuit from the expected behaviour of the scalar field wave and the a priori estimate \eqref{eq:APmidphi} that this likely more of a defect of our method than signifying how our variable behaves. In fact, we will ultimately be able to control these terms with the inequalities in Lemma \ref{lem:int-est-nablaphi}.}
%Furthermore, while we use \eqref{eq:diff-ineq-Friedman} to estimate the second term in \eqref{eq:diff-eq-SF2} by the left hand side of \eqref{eq:en-est-SF-Friedman-ineq} and thus rely explicitly on nonpositive sectional curvature, using the weaker version \eqref{eq:en-est-SF-Friedman-ineq} that would hold for $\kappa>0$ would only gain the error terms
%\[Ka^{-1}\sqrt{\E^{(L+1)}(N,\cdot)}\sqrt{\E^{(L)}(\phi,\cdot)}\lesssim \sqrt{a}\E^{(L+1)}(N,\cdot)+a^{-\frac52}\E^{(L)}(\phi,\cdot)\,.\]
%Close enough to the Big Bang, we can absorb the former term into $-\frac3{8\pi}\dot{a}a^3\E^{(L+1)}(N,\cdot)$ while keeping the contribution of this energy on the right hand side negative, and the latter term ultimately only contributes an additional bounded factor in the Gronwall argument. Thus, at least close to the Big Bang hypersurface, our way to derive the scalar field energy estimate seems to extend to all sectional curvature values.
%\end{remark}

To close the \change{argument}, we will need a scaled scalar field energy estimate at the \change{odd orders $L+1$}, which is not covered by the previous lemma and we hence establish separately:

%have to use an elliptic estimate on $\Sigma$-energies (see Lemma \ref{lem:en-est-Sigma-top}). However, this results in only being able to bound $a^4\E^{(M+1)}(\Sigma,\cdot)$ by a small and slightly divergent factor and not the energy itself, so we can also only control $a^4\E^{(M+1)}(\phi,\cdot)$ in that fashion, but will also need to gain some control on this scalar field energy. Since odd orders are not covered by the previous lemma, we need to derive this estimate separately; however, the argument is structurally mostly unchanged:

\begin{lemma}[Odd order scalar field energy estimate]\label{lem:en-est-SF-top}
For $L\in2\N$, $\change{2\leq L\leq \change{18}$}, we have:
\begin{align*}
\numberthis\label{eq:en-est-SF-top}&\,a(t)^4\E^{(L+1)}(\phi,t)+\int_t^{t_0}\left\{\dot{a}(s)a(s)^7\E^{(L+2)}(N,s)+\dot{a}(s)a(s)^3\E^{(L+1)}(N,s)\right\}\,ds\\
\lesssim&\,a(t_0)^4\E^{(L+1)}(\phi,t_0)+\int_t^{t_0}\left(\epsilon a(s)^{-3}+a(s)^{-1-c\sqrt{\epsilon}}\right)\cdot a(s)^4\E^{(L+1)}(\phi,s)\,ds\\
&\,+\int_t^{t_0}\left\{\epsilon a(s)^{-3}\cdot a(s)^4\E^{(L+1)}(\Sigma,s)+\left(\epsilon a(s)^{-3}+a(s)^{-1-c\sqrt{\epsilon}}\right)\E^{(L)}(\phi,s)+\epsilon a(s)^{-3}\E^{(L)}(\Sigma,s)\right.\\
&\,\phantom{+\int_t^{t_0}}+\epsilon a(s)^{-1-c\sqrt{\epsilon}}\cdot a(s)^4\E^{(L-1)}(\Ric,s)+\left(\epsilon^3a^{-3}+\epsilon a^{-1-c\sqrt{\epsilon}}\right)\E^{(L-2)}(\Ric,s)\\
&\,\phantom{+\int_t^{t_0}}+{\epsilon}a(s)^{-3-c\sqrt{\epsilon}}\left(\E^{(\leq L-2)}(\phi,s)+\E^{(\leq L-2)}(\Sigma,s)\right)\\
%&\,\phantom{+\int_t^{t_0}}+\epsilon^2 a(s)^{-1-c\sqrt{\epsilon}}\left(\|\nabla^2\Lap^{\frac{L}2-1}\phi\|_{L^2_G(\Sigma_s)}^2+\|\nabla\phi\|_{H^{L-2}_G(\Sigma_s)}^2\right)+\epsilon a(s)^{3-c\sqrt{\epsilon}}\cdot \|\nabla\phi\|_{H^L_G(\Sigma_s)}^2\\
&\change{\,\phantom{+\int_t^{t_0}}\underbrace{\left.+\left(\epsilon^3a(s)^{-3-c\sqrt{\epsilon}}+\epsilon^2a(s)^{-1-c\sqrt{\epsilon}}\right)\E^{(\leq L-4)}(\Ric,s)\right\}\,ds}_{\text{not present for }L=2}}
\end{align*}
\change{At order $1$, the analogous estimate holds where the last three lines of \eqref{eq:en-est-SF-top} are dropped.}
\end{lemma}

\begin{proof}
\change{These estimates follow }completely analogously to Lemma \ref{lem:en-est-SF}, with the exception that high order lapse terms can now be estimated at order $L+2$ due to the scalar field energy being scaled by $a^4$. In particular, we note that to deal with the analogous term to \eqref{eq:diff-eq-SF1}, one now inserts the following zero on the right and applies the commuted lapse equation \eqref{eq:comeq-lapse-odd}:
\begin{align*}
0=&\,-\frac{3}{8\pi}\dot{a}a^7\int_M\div_G\left(\nabla\Lap^{\frac{L}2}N\cdot\Lap^{\frac{L}2+1}N\right)\,\vol{G}\\
=&\,\int_M\left\{-\frac{3}{8\pi}\dot{a}a^7\lvert\Lap^{\frac{L}2+1}N\rvert^2-\frac{3}{8\pi}\left(\frac13\dot{a}a^3+12\pi C^2\frac{\dot{a}}a\right)\cdot a^4\lvert\nabla\Lap^{\frac{L}2}N\rvert_G^2\right.\\
&\,\,\phantom{\int_M}\left.-6C\frac{\dot{a}}a\cdot a^4\langle\nabla\Lap^{\frac{L}2}N,\nabla\Lap^{\frac{L}2}\Psi\rangle_G-\frac{3}{8\pi}\dot{a}a^7\langle\mathfrak{N}_{L+1,Border}+\mathfrak{N}_{L+1,Junk},\nabla\Lap^{\frac{L}2}N\rangle_G\right\}\,\vol{G}
\end{align*}
\change{For $L=0$, the argument is again the same as at higher orders with less complicated junk terms. We briefly highlight some specific junk terms: The term analogous to \eqref{eq:diff-eq-SF3} is now estimated as follows using \eqref{eq:Sobolev-norm-equiv-zetalow}:
\begin{align*}
a^4\cdot \int_M-2a\langle\nabla\Psi\nabla N,\nabla^2\phi\rangle_G\lesssim&\,\epsilon \int_M a^\frac12\lvert \nabla N\rvert_G\cdot a^{\frac12-c\sqrt{\epsilon}}\cdot a^4\lvert \nabla\phi\rvert_G\\
\lesssim&\,\epsilon \dot{a}a^3\E^{(1)}(N,\cdot)+\epsilon a^{1-c\sqrt{\epsilon}}\cdot a^4\E^{(\leq 1)}(\phi,\cdot) 
\end{align*}
Further, note that, by the commutator formula \eqref{eq:[del-t,Lap]zeta} and applying \eqref{eq:APmidphi}, one has
\begin{align*}
\left\lvert\int_Ma^8[\del_t,\Lap]\phi\cdot\Lap\phi\,\vol{G}\right\rvert\lesssim&\,\epsilon a^{5-c\sqrt{\epsilon}}\left(\|\nabla\Sigma\|_{L^2_G}+\|\nabla N\|_{L^2_G}\right)\|\Lap\phi\|_{L^2_G}\\
\lesssim&\,\epsilon a^{-1-c\sqrt{\epsilon}}\left(a^4\E^{(1)}(\phi,\cdot)+a^4\E^{(1)}(\Sigma,\cdot)\right)+\epsilon a^{6-c\sigma}\cdot \dot{a}a^3\E^{(1)}(N,\cdot)\,.
\end{align*}
}
\end{proof}

%\begin{remark}
%Beside necessary notational differences between the top order estimate and the estimates in Lemma \ref{lem:en-est-SF} throughout the proof, deriving this estimate relies on the same mechanisms as deriving the lower order analogues. However, the highest order lapse terms that don't have favourable sign are now of order $M+2$, but we can still only use Corollary \ref{cor:en-est-lapse} for $l=M$, which means we have to carefully track that the additional scaling of the scalar field energy by $a^4$ is sufficient to keep this term under control.
%\end{remark}
%\begin{proof}
%Once again, we start by computing the derivative and inserting \eqref{eq:comeq-Psi-odd}:
%\begin{subequations}
%\begin{align*}
%-\del_t\left(a^4\E^{(M+1)}(\phi,\cdot)\right)=&\,\int_M\left\{-2a^4\langle\del_t\nabla\Lap^{\frac{M}2}\Psi,\nabla\Lap^{\frac{M}2}\Psi\rangle_G-a^4(\del_tG^{-1})^{ij}\nabla_i\Lap^{\frac{M}2}\Psi\nabla_j\Lap^{\frac{M}2}\Psi\right.\\
%&\,\phantom{\int_M}-2a^4\cdot a^4\del_t\Lap^{\frac{M}2+1}\phi\cdot\Lap^{\frac{M}2+1}\phi-4a^4\cdot\frac{\dot{a}}a\cdot a^4\lvert\Lap^{\frac{M}2+1}\phi\rvert^2\\
%&\,\phantom{\int_M}\left.-3N\frac{\dot{a}}a\cdot a^4\left[\lvert\nabla\Lap^{\frac{M}2}\Psi\rvert_G^2+a^4\lvert\Lap^{\frac{M}2+1}\phi\rvert^2\right]\right\}\,\vol{G}\\
%&\,-4\frac{\dot{a}}a\cdot a^4\E^{(M+1)}(\phi,\cdot)\\[0.5em]
%%%%%%%%%%
%\numberthis\label{eq:SF-diff-eq-top1}=&\,\int_M\left\{a^4\langle-2a(N+1)\nabla\Lap^{\frac{M}2+1}\phi,\nabla\Lap^{\frac{M}2}\Psi\rangle_G-2a^4\cdot a\langle\nabla^2\Lap^{\frac{M}2}N,\nabla\phi\nabla\Lap^{\frac{M}2}\Psi\rangle_G\right.\\
%\numberthis\label{eq:SF-diff-eq-top2}&\,\phantom{\int_M}+6C\frac{\dot{a}}a\cdot a^4\langle\nabla\Lap^{\frac{M}2}N,\nabla\Lap^{\frac{M}2}\Psi\rangle_G\\
%\numberthis\label{eq:SF-diff-eq-top3}&\,\phantom{\int_M}-2a^8\left(a^{-3}(N+1)\Lap^{\frac{M}2+1}\Psi+Ca^{-3}\Lap^{\frac{M}2+1}N\right)\Lap^{\frac{M}2+1}\phi-4\frac{\dot{a}}a a^8\lvert\Lap^{\frac{M}2+1}\phi\rvert^2\\
%\numberthis\label{eq:SF-diff-eq-top4}&\,\phantom{\int_M}-2a^4\langle\mathfrak{P}_{M+1,Border}+\mathfrak{P}_{M+1,Junk},\nabla\Lap^\frac{M}2\Psi\rangle_G\\
%\numberthis\label{eq:SF-diff-eq-top5}&\,\phantom{\int_M}-2a^8\left(\mathfrak{Q}_{M+1,Border}+\mathfrak{Q}_{M+1,Junk}\right)\cdot\Lap^{\frac{M}2+1}\phi\\
%\numberthis\label{eq:SF-diff-eq-top6}&\,\phantom{\int_M}-a^{4}\cdot a^{-3}(N+1)\Sigma\ast\nabla\Lap^{\frac{M}2}\Psi\ast\nabla\Lap^{\frac{M}2}\Psi-2N\frac{\dot{a}}a\cdot a^4\lvert\nabla\Lap^{\frac{M}2+1}\Psi\rvert_G^2\\
%\numberthis\label{eq:SF-diff-eq-top7}&\,\phantom{\int_M}\left.-3N\frac{\dot{a}}a\cdot a^4\left[\lvert\nabla\Lap^{\frac{M}2}\Psi\rvert_G^2+a^4\lvert\Lap^{\frac{M}2+1}\phi\rvert_G^2\right]\right\}\,\vol{G}\\
%\numberthis\label{eq:SF-diff-eq-top8}&\,-4\frac{\dot{a}}a\cdot a^4\E^{(M+1)}(\phi,\cdot)
%\end{align*}
%\end{subequations}
%Since \eqref{eq:SF-diff-eq-top8} is nonnegative, we can simply drop said term. Most of the arguments now go through as in Lemma \ref{lem:en-est-SF}:\\
%
%For the first terms in \eqref{eq:SF-diff-eq-top1} and \eqref{eq:SF-diff-eq-top3}, we once again get a cancellation via integration by parts, up to an error term bounded by $\epsilon a^{3-c\sigma}\cdot a^4\E^{(M+1)}(\phi,\cdot)$.\\
%
%For the second term in \eqref{eq:SF-diff-eq-top1}, we apply the strong $C_G$-norm estimate \eqref{eq:APmidphi}, then \eqref{eq:Sobolev-norm-equiv-zetalow} to $\zeta=\Lap^{\frac{M}2}N$ and subsequently Corollary \ref{cor:en-est-lapse} at order $M$ to estimate the lapse energies, and obtain the following:
%\begin{align*}
%&\,\left\lvert\int_M 2a^4\cdot a\langle\nabla^2\Lap^{\frac{M}2}N,\nabla\phi\nabla\Lap^{\frac{M}2}\Psi\rangle_G\,\vol{G}\right\rvert\\
%\lesssim&\, \sqrt{\epsilon}a^{-1-c\sqrt{\epsilon}}\cdot a^4\|\nabla^2\Lap^{\frac{M}2}N\|_{L^2_G}\cdot\sqrt{a^4\E^{(M+1)}(\phi,\cdot)}\\
%\lesssim&\,\epsilon a^{-1-c\sqrt{\epsilon}}\cdot a^8\left(\E^{(M+2)}(N,\cdot)+ {\E^{(M+1)}(N,\cdot)}\right)+a^{-1-c\sqrt{\epsilon}}\cdot a^4\E^{(M+1)}(\phi,\cdot)\\
%\lesssim&\,\epsilon a^{-1-c\sqrt{\epsilon}}\cdot \left(a^8\E^{(M+2)}(N,\cdot)+ a^4{\E^{(M+1)}(N,\cdot)}\right)+a^{-1-c\sqrt{\epsilon}}\cdot a^4\E^{(M+1)}(\phi,\cdot)\\
%\lesssim&\,  \left[\epsilon a^{-1-c\sqrt{\epsilon}}\E^{(M)}(\phi,\cdot)+\epsilon^3a^{-1-c\sqrt{\epsilon}}\E^{(M)}(\Sigma,\cdot)+\epsilon^3a^{-1-c\sqrt{\epsilon}}\left(\E^{(\leq M-2)}(\phi,\cdot)+\E^{(\leq M-2)}(\Sigma,\cdot)\right)\right.\\
%&\,\left.+\epsilon^3a^{-1-c\sqrt{\epsilon}}\E^{(\leq M-2)}(\Ric,\cdot)\right]+a^{-1-c\sqrt{\epsilon}}\cdot a^4\E^{(M+1)}(\phi,\cdot)
%\end{align*}
%For \eqref{eq:SF-diff-eq-top2}, we insert the following zero and apply \eqref{eq:comeq-lapse-odd}:
%\begin{align*}
%0=&\,-\frac{3}{8\pi}\dot{a}a^7\int_M\div_G\left(\nabla\Lap^{\frac{M}2}N\cdot\Lap^{\frac{M}2+1}N\right)\,\vol{G}\\
%=&\,\int_M\left\{-\frac{3}{8\pi}\dot{a}a^7\lvert\Lap^{\frac{M}2+1}N\rvert^2-\frac{3}{8\pi}\left(\frac13\dot{a}a^3+12\pi C^2\frac{\dot{a}}a\right)\cdot a^4\lvert\nabla\Lap^{\frac{M}2}N\rvert_G^2\right.\\
%\,&\phantom{\int_M}\left.-6C\frac{\dot{a}}a\cdot a^4\langle\nabla\Lap^{\frac{M}2}N,\nabla\Lap^{\frac{M}2}\Psi\rangle-\frac{3}{8\pi}\dot{a}a^7\langle\mathfrak{N}_{M+1,Border}+\mathfrak{N}_{M+1,Junk},\nabla\Lap^{\frac{M}2}N\rangle_G\right\}\,\vol{G}
%\end{align*}
%Again, the critical term in \eqref{eq:SF-diff-eq-top2} is cancelled by the first term in the second line, and we get two terms of favourable sign in the first line. Applying \eqref{eq:L2-Border-N-odd} and \eqref{eq:L2-junk-N-odd} and then Corollary \ref{cor:en-est-lapse} up to order $M$, we can bound the remainder up to constant by
%\begin{align*}
%&\,a^3\left(\|\mathfrak{N}_{M+1,Border}\|_{L^2_G}+\|\mathfrak{N}_{M+1,Junk}\|_{L^2_G}\right)\cdot \sqrt{a^4\E^{(M+1)}(N,\cdot)}\\
%%\lesssim&\,\sqrt{a^4\E^{(M+1)}(N,\cdot)}\cdot a^3\left[\epsilon a^{-4}\left(\sqrt{\E^{(M+1)}(\Sigma,\cdot)}+\sqrt{\E^{(M+1)}(\phi,\cdot)}+\sqrt{\E^{(M+1)}(N,\cdot)}\right)\right.\\
%%&\,+\epsilon a^{-4-c\sqrt{\epsilon}}\left(\sqrt{\E^{(\leq M-1)}(\Sigma,\cdot)}+\sqrt{\E^{(\leq M-1)}(\phi,\cdot)}+\sqrt{\E^{(\leq M-1)}(N,\cdot)}\right)\\
%%&\,\left.+\epsilon^2a^{-4-c\sqrt{\epsilon}}\sqrt{\E^{(\leq M-2)}(\Ric,\cdot)}\right]\\
%%\lesssim&\,\epsilon a^{-3}\left(a^4\E^{(M+1)}(\Sigma,\cdot)+a^4\E^{(M+1)}(\phi,\cdot)+a^4\E^{(M+1)}(N,\cdot)\right)+a^{-1-c\sqrt{\epsilon}}\cdot a^4\E^{(M+1)}(N,\cdot)\\
%%&\,+\epsilon^2 a^{-1-c\sqrt{\epsilon}}\left({\E^{(\leq M-1)}(\Sigma,\cdot)}+{\E^{(\leq M-1)}(\phi,\cdot)}+{\E^{(\leq M-1)}(N,\cdot)}\right)\\
%%&\,+\epsilon^4a^{-1-c\sqrt{\epsilon}}\E^{(\leq M-2)}(\Ric,\cdot)\\
%\lesssim&\,\epsilon a^{-3}\left(a^4\E^{(M+1)}(\Sigma,\cdot)+a^4\E^{(M+1)}(\phi,\cdot)\right)+\left(\epsilon a^{-3}+a^{-1-c\sqrt{\epsilon}}\right)\E^{(M)}(\phi,\cdot)+\epsilon a^{-3}\E^{(M)}(\Sigma,\cdot)\\
%&\,+\epsilon^4a^{-1-c\sqrt{\epsilon}}\E^{(M-2)}(\Ric,\cdot)+\epsilon^2a^{-3-c\sqrt{\epsilon}}\left({\E^{(\leq M-2)}(\Sigma,\cdot)}+{\E^{(\leq M-2)}(\phi,\cdot)}\right)\\
%&\,+\epsilon^3a^{-3-c\sqrt{\epsilon}}\E^{(\leq M-3)}(\Ric,\cdot)
%\end{align*}
%Also, note that by the applying the standard integration-by-parts trick \eqref{eq:ibp-trick}, we can replace the curvature terms on the right hand side by
%\[\epsilon^3a^{-3}\E^{(M-2)}(\Ric,\cdot)+\epsilon^3a^{-3-c\sqrt{\epsilon}}\E^{(\leq M-4)}(\Ric,\cdot)\]
%in an upper estimate.\\
%The second term in \eqref{eq:SF-diff-eq-top3} can be treated as the second term \eqref{eq:diff-eq-SF2} by applying \eqref{eq:diff-ineq-Friedman} and absorbing the second term in \eqref{eq:SF-diff-eq-top7}.\\
%One can also proceed with the terms in \eqref{eq:SF-diff-eq-top4}-\eqref{eq:SF-diff-eq-top7} as before: We insert the borderline term estimates \eqref{eq:L2-Border-P-odd} and \eqref{eq:L2-Border-Q-odd} as well as the junk term estimates \eqref{eq:L2-junk-P-odd} and \eqref{eq:L2-junk-Q-odd}. Note that, when working in the borderline and junk term estimates, we apply Corollary \ref{cor:en-est-lapse} to lapse energies of order $M+1$ and $M+2$ at order $M$. In particular, we can control the lapse energies bounding in $\|\mathfrak{Q}_{M+1,Junk}\|_{L^2_G}$ this way: The latter appears in the differential inequality as 
%\[a^4\|\mathfrak{Q}_{M+1,Junk}\|_{L^2_G}\cdot\sqrt{a^4\E^{(M+1)}(\phi,\cdot)}\]
%and we thus have enough powers of $a$ at our disposal to apply said elliptic estimates at order $M$. Similarly, note that beside the lapse terms, this scaling means that many terms that were borderline in lower orders now carry prefactor $a^{-1-c\sqrt{\epsilon}}$ before applying the Young inequality, and thus we are able to distribute powers of $\epsilon$ more efficiently here.\\
%Finally, we estimate \eqref{eq:SF-diff-eq-top6} and what remains of \eqref{eq:SF-diff-eq-top7} by $\epsilon a^{-3}\cdot a^4\E^{(M+1)}(\phi,\cdot)$. 
%%\begin{align*}
%%&\,-\del_t\left(a^4\E^{(M+1)}(\phi,\cdot)\right)+\dot{a}a^7\E^{(M+2)}(\phi,\cdot)+\dot{a}a^3\E^{(M+1)}(\phi,\cdot)\\
%%\lesssim&\,\left(\sqrt{\epsilon}a^{-3}+a^{-1-c\sqrt{\epsilon}}\right)\left(\E^{(M+1)}(\phi,\cdot)+\E^{(M)}(\phi,\cdot)\right)+\epsilon a^{-1-c\sqrt{\epsilon}}\E^{(M)}(\Sigma,\cdot)\\
%%&\,{\epsilon}a^{-3-c\sqrt{\epsilon}}\E^{(\leq M-2)}(\phi,\cdot)+\epsilon^2 a^{-1-c\sqrt{\epsilon}}\left(\|\nabla^2\Lap^{\frac{M}2-1}\phi\|_{L^2_G}^2+\|\nabla\phi\|_{H^{M-2}_G}\right)+\epsilon a^{1-c\sqrt{\epsilon}}\|\nabla\phi\|_{H^L_G}\\
%%&\,+\epsilon^3\cdot a^4\E^{(M-1)}(\Ric,\cdot)+\left(\epsilon^3a^{-3-c\sqrt{\epsilon}}+\epsilon^2a^{-1-c\sqrt{\epsilon}}\right)\E^{(\leq M-2)}(\Ric,\cdot)\\
%%\end{align*}
%\end{proof}



\subsubsection{\change{Sobolev norm estimate for $\nabla\phi$}}\label{subsubsec:int-nabla-phi}

\change{To improve the bootstrap assumptions on $\nabla\phi$, we will need sharper bounds than those on $a^4\|\nabla\phi\|_{H^L}^2$ that will follow from bounds on $\E^{(L)}(\phi,\cdot)$:

\begin{lemma}\label{lem:norm-est-nablaphi}
Let $l\in(t_{Boot},t_0]$. Then, for $l\in\Z_+$, $l\leq 17$, the following estimate holds:
\begin{align*}
\|\nabla\phi\|_{H^l_G(\Sigma_t)}\lesssim (1+\epsilon a(t)^{-c\sqrt{\epsilon}})\|\Sigma\|_{H^{l+1}_G(\Sigma_t)}+\epsilon a(t)^{-c\sqrt{\epsilon}}\|\Psi\|_{H^{l}_G(\Sigma_t)}
\end{align*}
\end{lemma}
\begin{proof}
By \eqref{eq:APPsi}, $\Psi+C>\frac{C}2$ holds if $\epsilon$ is chosen small enough. Consequently, we can rearrange \eqref{eq:REEqMom} and apply the product rule to obtain
\[\lvert\nabla^l\nabla\phi\rvert_G=\frac{1}{8\pi}\left\lvert\nabla^l\left(\frac{\div_G\Sigma}{\Psi+C}\right)\right\rvert_G\lesssim \sum_{I_\Sigma+I_\Psi=l}\lvert\nabla^{I_\Sigma+1}\Sigma\rvert_G\lvert\nabla^{I_\Psi}\changefinal{(\Psi+C)}\rvert_G\,.\]
The statement then follows by integrating over $\Sigma_t$ and applying \eqref{eq:APPsi} and \eqref{eq:APSigma}.
\end{proof}
}
%As mentioned, we need additional auxiliary estimates to control scalar field error terms in the energy estimates and to ultimately improve the bootstrap assumptions for $\nabla\phi$ without the scaling that is carried in the scalar field energies. This is dealt with by the following estimates:

%\begin{lemma}\label{lem:int-est-nablaphi}[Integral estimates for $\nabla\phi$] 
%Let $t\in(t_{Boot},t_0]$.
%Then, for $L\in2\Z_+$, $2\leq L\leq 20$, the following estimate holds:
%\begin{align*}
%\numberthis\label{eq:int-est-Lapphi}\|\Lap^{\frac{L}2}\phi\|_{L^2_G(\Sigma_t)}^2\lesssim&\,\|\Lap^{\frac{L}2}\phi\|_{L^2_G(\Sigma_{t_0})}^2+\int_t^{t_0}\epsilon^\frac18a(s)^{-3}\|\Lap^{\frac{L}2}\phi\|_{L^2_G(\Sigma_s)}^2\,ds\\
%&\,+\int_t^{t_0}a(s)^{-3}\left(\epsilon^{-\frac18}\E^{(L)}(\phi,s)+\epsilon^{\frac{15}8}\E^{(L)}(\Sigma,s)\right)\,ds\\
%&\,+\int_t^{t_0}a(s)^{-3-c\sqrt{\epsilon}}\left(\epsilon^\frac78\E^{(\leq L-2)}(\phi,s)+\epsilon^\frac78\E^{(\leq L-2)}(\Sigma,\cdot)\right)\,ds\\
%&\,+\begin{cases}
%\displaystyle\int_t^{t_0}\epsilon^\frac78a(s)^{-3-c\sqrt{\epsilon}}\|\nabla\phi\|_{L^2_G}^2\,ds & L=2\\
%\displaystyle\int_t^{t_0}a(s)^{-3-c\sqrt{\epsilon}}\left(\epsilon^\frac78\|\nabla\phi\|_{H^{L-3}_G(\Sigma_s)}^2+\epsilon^{\frac{15}8}\E^{(\leq L-3)}(\Ric,s)\right)\,ds & L\neq 2
%\end{cases}
%%\numberthis\label{eq:int-est-nablaphi}\|\nabla\Lap^{\frac{L}2}\phi\|_{L^2_G(\Sigma_t)}^2\lesssim&\,\|\nabla\Lap^{\frac{L}2}\phi\|_{L^2_G(\Sigma_{t_0})}^2+\int_t^{t_0}\epsilon^\frac18 a(s)^{-3}\|\nabla\Lap^{\frac{L}2}\phi\|_{L^2_G(\Sigma_s)}^2\,ds\\
%%&\,+\int_t^{t_0}\left(a(s)^{-3}\left(\epsilon^{-\frac18}\E^{(L+1)}(\phi,s)+\epsilon^{-\frac18}\E^{(L+1)}(N,s)\right)\right.\\
%%&\,\phantom{\int_t^{t_0}}+a(s)^{-3-c\sqrt{\epsilon}}\left(\epsilon^\frac78\E^{(\leq L)}(\phi,s)+\epsilon^\frac{23}8\E^{(L)}(\Sigma,s)+\underbrace{\epsilon^\frac78\E^{(\leq L-1)}(\Sigma,\cdot)}_{\text{not present for }L=0}\right.\\
%%&\,\phantom{\int_t^{t_0}}\left.\left.+\underbrace{\epsilon^\frac78\|\nabla\phi\|_{H^{L-2}_G(\Sigma_s)}^2+\epsilon^{\frac{15}8}\E^{(\leq L-2)}(\Ric,s)}_{\text{not present for }L=0}\right)\right)ds
%\end{align*}
%Additionally, one has
%\begin{align*}
%\numberthis\label{eq:int-est-nablaphi0}\|\nabla\phi\|_{L^2_G(\Sigma_t)}^2\lesssim&\,\|\nabla\phi\|_{L^2_G(\Sigma_{t_0})}^2+\int_t^{t_0}\epsilon^\frac18 a(s)^{-3}\|\nabla\phi\|_{L^2_G(\Sigma_s)}^2\,ds\\
%&\,+\int_t^{t_0}a(s)^{-3}\left(\epsilon^{-\frac18}\E^{(1)}(\phi,s)+\epsilon^{-\frac18}\E^{(1)}(N,s)\right)\,ds\,.
%\end{align*}
%\end{lemma}
%%\begin{remark}
%%The benefit of these auxiliary estimates is that they allow to control purely spatial derivatives of the scalar field via integral inequalities of similar structure to other energies without any additional scaling -- the scalar field energy $\E^{(L-1)}(\phi,\cdot)$ only admits control of $a^4\|\Lap^\frac{L}2\phi\|_{L^2_G}^2$. Hence, it is now possible to control error terms of this type without having to \enquote{pay for it} by estimating them by $a^{-4}\E^{(L-1)}(\phi,\cdot)$ and incurring prefactors that diverge at significantly stronger rates. The cost of this will be that we need to balance the inverse powers of $\epsilon$ that currently occur (see the second lines of \eqref{eq:int-est-Lapphi} and \eqref{eq:int-est-nablaphi0}) by scaling these scalar field norms by $\epsilon^\frac14$ in the total energies.
%%\end{remark}
%\begin{proof}
%For \eqref{eq:int-est-Lapphi}, we use Lemma \ref{lem:delt-int} and \eqref{eq:comeq-nablaphi-odd} to compute:
%\begin{align*}
%-\del_t\|\Lap^\frac{L}2\phi\|_{L^2_G}^2=&\,\int_M -2\del_t\Lap^{\frac{L}2}\phi\cdot\Lap^{\frac{L}2}\phi-3N\frac{\dot{a}}a\lvert\Lap^\frac{L}2\phi\rvert_G^2\,\vol{G}\\
%=&\,\int_M-2a^{-3}(N+1)\Lap^{\frac{L}2}\Psi\cdot\Lap^\frac{L}2\phi-2Ca^{-3}\Lap^\frac{L}2N\cdot\Lap^{\frac{L}2}\phi\\
%&\,\phantom{\int_M}-2\left(\mathfrak{Q}_{L-1,Border}+\mathfrak{Q}_{L-1,Junk}\right)\cdot\Lap^{\frac{L}2}\phi-3N\frac{\dot{a}}a\lvert\Lap^\frac{L}2\phi\rvert_G^2\,\vol{G}
%\end{align*}
%Estimating the base level lapse terms with Lemma \ref{lem:lapse-maxmin} and \eqref{eq:BsN}, we obtain 
%\begin{align*}
%-\del_t\|\Lap^\frac{L}2\phi\|_{L^2_G}^2\lesssim&\,a^{-3}\left(\sqrt{\E^{(L)}(\phi,\cdot)}+\sqrt{\E^{(L)}(N,\cdot)}\right)\cdot\|\Lap^\frac{L}2\phi\|_{L^2_G}\\
%&\,+(\|\mathfrak{Q}_{L-1,Border}\|_{L^2_G}+\|\mathfrak{Q}_{L-1,Junk}\|_{L^2_G})\cdot\|\Lap^\frac{L}2\phi\|_{L^2_G}+\epsilon a^{1-c\sigma}\cdot\|\Lap^\frac{L}2\phi\|_{L^2_G}^2\\
%\lesssim&\,\epsilon^{\frac18}a^{-3}\|\Lap^{\frac{L}2}\phi\|_{L^2_G}^2+\epsilon^{-\frac18}a^{-3}\left(\E^{(L)}(\phi,\cdot)+\E^{(L)}(N,\cdot)\right)\\
%&\,+\left(\|\mathfrak{Q}_{L-1,Border}\|_{L^2_G}+\|\mathfrak{Q}_{L-1,Junk}\|_{L^2_G}\right)\cdot\|\Lap^\frac{L}2\phi\|_{L^2_G}
%\end{align*}
%The inequality \eqref{eq:int-est-Lapphi} now follows by inserting the $L^2$-estimates \eqref{eq:L2-border-Q-1-alt} and \eqref{eq:L2-junk-Q-1} for $L=2$, respectively \eqref{eq:L2-Border-Q-odd} and \eqref{eq:L2-junk-Q-odd} for $L\geq 4$, estimating the lapse energies with Corollary \ref{cor:en-est-lapse} and integrating. 
%
%For \eqref{eq:int-est-nablaphi0}, we similarly calculate with \eqref{eq:REEqNablaPhi} and \eqref{eq:REEqG-1} as well as \eqref{eq:BsN}, Lemma \ref{lem:lapse-maxmin} the strong base level estimate \eqref{eq:APPsi}:
%\begin{align*}
%-\del_t\int_M\lvert\nabla\phi\rvert_G^2\,\vol{G}=&\,\int_M-2\langle\del_t\nabla\phi,\nabla\phi\rangle_G-3N\frac{\dot{a}}a\lvert\nabla\phi\rvert_G^2-(\del_tG^{-1})^{ij}\nabla_i\phi\nabla_j\phi\,\vol{G}\\
%\lesssim&\,\int_Ma^{-3}\left(\lvert\nabla\Psi\rvert_G+\lvert\nabla N\rvert_G\right)\lvert\nabla\phi\rvert_G+\epsilon a^{-3}\lvert\nabla\phi\rvert_G^2\,\vol{G}\\
%\lesssim&\,\epsilon^\frac18a^{-3}\int_M\lvert\nabla\phi\rvert_G^2\,\vol{G}+\epsilon^{-\frac18}a^{-3}\left(\E^{(1)}(\phi,\cdot)+\E^{(1)}(N,\cdot)\right)
%\end{align*}
%%With \eqref{eq:ibp-trick} and applying Corollary \ref{cor:en-est-lapse}, we get 
%%\begin{align*}
%%-\del_t\int_M\lvert\nabla\phi\rvert_G^2\,\vol{G}\lesssim&\,\epsilon^\frac18a^{-3}\int_M\lvert\nabla\phi\rvert_G^2\,\vol{G}+a^{-3}\left(\epsilon^{-\frac18}\E^{(2)}(\phi,\cdot)+\epsilon^{\frac{15}8}\E^{(2)}(\Sigma,\cdot)\right)\\
%%&\,+a^{-3-c\sqrt{\epsilon}}\left(\epsilon^{-\frac18}\E^{(0)}(\phi,\cdot)+\epsilon^\frac{15}8\E^{(0)}(\Sigma,\cdot)\right)
%%\end{align*}
%The inequality now follows by integrating.\\
%%The general statement follows by the exact same series of arguments, up to dealing with error terms as in the first case and not inserting any estimates for the top order lapse energy.
%%%%%%%%%%%%%PROOF FOR NABLAPHI%%%%%%%%%%%%%%
%%\begin{align*}
%%-\del_t\int_M\lvert\nabla\Lap^{\frac{L}2}\phi\rvert_G^2\,\vol{G}=&\,-\int_M2\langle\del_t\nabla\Lap^{\frac{L}2}\phi,\nabla\Lap^{\frac{L}2}\phi\rangle_G+(\del_tG^{-1})^{ij}\nabla_i\Lap^{\frac{L}2}\phi\nabla_j\Lap^{\frac{L}2}\phi\,\vol{G}\\
%%\lesssim&\,\int_Ma^{-3}(N+1)\lvert\nabla\Lap^{\frac{L}2}\Psi\rvert_G\lvert\nabla\Lap^{\frac{L}2}\phi\rvert_G+a^{-3}\lvert\nabla\Lap^{\frac{L}2}\phi\rvert_G\lvert\nabla\Lap^{\frac{L}2}\phi\rvert_G+\left(\lvert\mathfrak{Q}_{L,Border}\rvert_G+\lvert\mathfrak{Q}_{L,Junk}\rvert_G\right)\lvert\nabla\Lap^{\frac{L}2}\phi\rvert_G\\
%%&\,+\epsilon a^{-3}\lvert\nabla\Lap^{\frac{L}2}\phi\rvert_G^2\,\vol{G}\\
%%\lesssim&\,a^{-3}\left[\sqrt{\E^{(L+1)}(\phi,\cdot)}+\sqrt{\E^{(L+1)}(N,\cdot)}\right]\sqrt{\int_M\lvert\nabla\Lap^{\frac{L}2}\phi\rvert_G^2\,\vol{G}}\\
%%&\,+a^{-3}\left(\|\mathfrak{Q}_{L,Border}\|_{L^2_G}+\|\mathfrak{Q}_{L,Junk}\|_{L^2_G}\right)\sqrt{\int_M\lvert\nabla\Lap^{\frac{L}2}\phi\rvert_G^2\,\vol{G}}\\
%%&\,+\epsilon a^{-3}\int_M\lvert\nabla\Lap^{\frac{L}2}\phi\rvert_G^2\,\vol{G}\\
%%\lesssim&\,\epsilon^{\frac18}a^{-3}\int_M\lvert\nabla\Lap^{\frac{L}2}\phi\rvert_G^2\,\vol{G}+\epsilon^{-\frac18}a^{-3}\left(\E^{(L+1)}(\phi,\cdot)+\E^{(L+1)}(N,\cdot)\right)\\
%%&\,+\epsilon^{-\frac18}a^{-3}\left(\|\mathfrak{Q}_{L,Border}\|_{L^2_G}^2+\|\mathfrak{Q}_{L,Junk}\|_{L^2_G}^2\right)
%%\end{align*}
%%Now, we can apply the \todo{adapted borderline} and \todo{junk term} estimates to the final line, rearrange and obtain the statement by the standard Gronwall argument. In particular, note that we can use the standard integration by parts trick to write
%%\[\E^{(\leq L)}(N,\cdot)=\sum_{m=0}^{\frac{L}2}\E^{(2m)}(N,\cdot)\]
%%and apply \todo{elliptic energy estimate with $\L$} to each summand.\\
%%
%%The other inequality follows mostly analogously, taking note of the fact that the borderline and junk terms have different structure and that no (additional) $\del_tG^{-1}$-term as above occurs. \todo{dealing with lapse}
%\end{proof}

%\noindent We further prepare the estimate \eqref{eq:int-est-nablaphi0} for the analysis at base level in Section \ref{subsec:bs-imp-core}: 

%\begin{corollary}[Improved estimate for $\|\nabla\phi\|_{L^2_G}^2$]\label{cor:int-est-nablaphi0-botch} For any $t\in(t_{Boot},t_0]$, the following holds:
%\begin{equation}\label{eq:int-est-nablaphi0-botch}
%\|\nabla\phi\|_{L^2_G(\Sigma_t)}^2\lesssim\left(\epsilon^\frac{15}8+\epsilon^{-\frac14}\sup_{r\in(t,t_0]}\E^{(1)}(\phi,r)\right)a^{-c\epsilon^\frac18}
%\end{equation}
%\end{corollary}
%\begin{proof}
%By inserting \eqref{eq:BsEnN} and the initial data assumption \eqref{eq:init-ass} into \eqref{eq:int-est-nablaphi0} and applying the Gronwall lemma along with \eqref{eq:a-exp-est}, we obtain
%\[\|\nabla\phi\|_{L^2_G(\Sigma_t)}^2\lesssim a^{-c\epsilon^\frac18}\left[\epsilon^4+\int_t^{t_0}\left(\epsilon^{-\frac18}a(s)^{-3}\E^{(1)}(\phi,s)+\epsilon^\frac{15}8 a(s)^{5-c\sigma}\right)\,ds\right]\,.\]
%By estimating the scalar field energy by its supremum on $(t,t_0]$, the integral can be bounded by
%\[\sup_{r\in(t,t_0]}{\E}^{(1)}(\phi,r)\cdot \epsilon^{-\frac18}\int_t^{t_0}a(s)^{-3}\,ds+\epsilon^\frac{15}8\int_t^{t_0}a(s)^{5-c\sigma}\,ds\,.\]
%The statement now follows with \eqref{eq:log-est} and \eqref{eq:a-integrals} and updating $c>0$.
%\end{proof}


\subsection{Energy estimates for the Bel-Robinson variables}\label{subsec:en-BR}

In this subsection, we collect the energy estimates for the Bel-Robinson variables:

\begin{lemma}[\change{Bel-Robinson energy estimates}]\label{lem:en-est-BR} Let $t\in(t_{Boot},t_0]$. Then one has
\begin{align*}
\numberthis\label{eq:en-est-BR0}&\,\E^{(0)}(W,t)+\int_t^{t_0}\int_M\left[8\pi C^2a(s)^{-3}(N+1)\langle\Sigma,\RE\rangle_G+6\frac{\dot{a}(s)}{a(s)}\changefinal{(N+1)}\lvert\RE\rvert_G^2\right]\,\vol{G}\,ds\\
\lesssim&\,\E^{(0)}(W,t_0)+\int_t^{t_0}\left({\epsilon}a(s)^{-3}+a(s)^{-1-c\sqrt{\epsilon}}\right)\E^{(0)}(W,s)+\change{\epsilon^{-\frac18}a(s)^{-3}\cdot a(s)^4\E^{(1)}(\phi,s)}\,ds\\
&\,\phantom{\E^{(0)}(W,t_0)}+\int_t^{t_0}a(s)^{-1-c\sqrt{\epsilon}}\E^{(0)}(\phi,s)+\epsilon a(s)^{-3}\E^{(0)}(\Sigma,s)\,ds
\end{align*}
as well as, for $L\in 2\N,\,2\leq L\leq \change{18}$,
\change{\begin{align*}
\numberthis\label{eq:en-est-BR}&\,\E^{(L)}(W,t)+\int_t^{t_0}\int_M\left[8\pi C^2a(s)^{-3}(N+1)\langle\Lap^\frac{L}2\Sigma,\Lap^\frac{L}2\RE\rangle_G+6(N+1)\frac{\dot{a}(s)}{a(s)}\lvert\Lap^\frac{L}2\RE\rvert_G^2\right]\,\vol{G}\,ds\\
&\,\lesssim\E^{(L)}(W,t_0)+\int_t^{t_0}\left(\epsilon^\frac18 a(s)^{-3}+a(s)^{-1-c\sqrt{\epsilon}}\right)\E^{(L)}(W,s)\,ds\\
&\,+\int_t^{t_0}\left\{\epsilon^{-\frac18}a(s)^{-3}\cdot a(s)^4\E^{(L+1)}(\phi,s)+\left(\epsilon^\frac18 a(s)^{-3}+a(s)^{-1}\right)\E^{(L)}(\phi,s)\right.\\
&\,\phantom{\int_t^{t_0}}+\epsilon a(s)^{-3}\E^{(L)}(\Sigma,s)+\epsilon^\frac{7}8a(s)^{-3}\cdot a(s)^4\E^{(L-1)}(\Ric,s)+\epsilon^\frac{31}8 a(s)^{-3}\E^{(\leq L-2)}(\Ric,s)\\
&\,\phantom{\int_t^{t_0}}+\left(\epsilon^\frac{15}8 a(s)^{-3-c\sqrt{\epsilon}}+a(s)^{-1-c\sqrt{\epsilon}}\right)\E^{(\leq L-2)}(\phi,s)\\
&\,\left.\phantom{\int_t^{t_0}}+\epsilon^\frac{15}8 a(s)^{-3-c\sqrt{\epsilon}}\left(\E^{(\leq L-2)}(\Sigma,s)+\E^{(\leq L-2)}(W,s)\right)+\underbrace{\epsilon^\frac{15}8 a(s)^{-3-c\sqrt{\epsilon}}\E^{(\leq L-4)}(\Ric,s)\,ds}_{\text{not present for }L=2}\right\}\,.
\end{align*}}
\end{lemma}
\begin{remark}
We preemptively note that the error terms on the left hand side, once combined with the similar terms on the left hand side in Lemma \ref{lem:en-est-Sigma} and given suitable weights, will turn out to have positive sign, even if they do not have definite sign in isolation.\\

The main idea in deriving this inequality is that we can use the algebraic identity \eqref{eq:div-to-curl} and integration by parts to exploit the Maxwell system that lies at the core of the Bel-Robinson evolution equations. As a result, we avoid having higher order energies of the Bel-Robinson variables on the right hand side of the integral energy inequalities (which would break the bootstrap argument), then only having to deal with scalar field and Ricci energies at the next derivative \change{level. }%Below top order, the bootstrap assumptions will then suffice to bound these terms.
\end{remark}
\begin{proof}
We first prove \eqref{eq:en-est-BR}, and then explain how the same ideas lead to the simpler estimate \eqref{eq:en-est-BR0}. To this end, we start out by taking the time derivative of the energy as usual:
\begin{align*}
-\del_t\E^{(L)}(W,\cdot)=&\,\int_M-3N\frac{\dot{a}}a\left[\lvert\Lap^{\frac{L}2}\RE\rvert_G^2+\lvert\Lap^{\frac{L}2}\RB\rvert_G^2\right]-2\left(\langle\del_t\Lap^{\frac{L}2}\RE,\Lap^{\frac{L}2}\RE\rangle_G+\langle\del_t\Lap^{\frac{L}2}\RB,\Lap^{\frac{L}2}\RB\rangle_G\right)\\
&\quad-2(\del_tG^{-1})^{i_1j_1}(G^{-1})^{i_2j_2}\left[\Lap^{\frac{L}2}\RE_{i_1i_2}\Lap^{\frac{L}2}\RE_{j_1j_2}+\Lap^{\frac{L}2}\RB_{i_1i_2}\Lap^{\frac{L}2}\RB_{j_1j_2}\right]\,\vol{G}
\end{align*}
$\RE$ and $\RB$ are symmetric and tracefree, thus symmetrizations become redundant, and any scalar product with a tensor that is pure trace or with an antisymmetric tensor can be dropped.\footnote{Recall the superscript \enquote{$\parallel$} notation for error terms, see Remark \ref{rem:notation-parallel}.} With this in hand, we get, inserting \eqref{eq:comeq-RE} and \eqref{eq:comeq-RB}:

\begin{subequations}
\begin{align}
-\del_t\E^{(L)}(W,\cdot)=&\,\int_M\biggr\{\frac{\dot{a}}a(-6(N+1)+9N)\left(\lvert\Lap^{\frac{L}2}\RE\rvert_G^2+\lvert\Lap^{\frac{L}2}\RB\rvert_G^2\right)\label{eq:en-eq-BR1}\\
&\,\phantom{\int_M}\change{+2(N+1)a^{-1}\left(\langle\curl_G\Lap^{\frac{L}2}\RE,\Lap^{\frac{L}2}\RB\rangle_G-\langle\curl_G\Lap^{\frac{L}2}\RB,\Lap^{\frac{L}2}\RE\rangle_G}\right)\label{eq:en-eq-BR2}\\
&\,\phantom{\int_M}+\change{2a^{-1}\left(\langle\nabla\Lap^\frac{L}2N\wedge_G\RB,\Lap^\frac{L}2\RE\rangle_G-\langle\nabla\Lap^{\frac{L}2}N\wedge_G\RE,\Lap^{\frac{L}2}\RB\rangle_G\right)}\label{eq:en-eq-BR3}\\
&\,\phantom{\int_M}-8\pi C^2a^{-3}(N+1)\langle\Lap^\frac{L}2\Sigma,\Lap^\frac{L}2\RE\rangle_G-8\pi a(\Psi+C)\langle\nabla\Lap^{\frac{L}2}N\nabla\phi,\Lap^\frac{L}2\RE\rangle_G\label{eq:en-eq-BR4}\\
&\,\phantom{\int_M}-8\pi a(\Psi+C)(N+1)\langle\nabla^2\Lap^{\frac{L}2}\phi,\Lap^{\frac{L}2}\RE\rangle_G\label{eq:en-eq-BR5}\\
&\,\phantom{\int_M}+16\pi a(N+1)\langle \nabla\phi\nabla\Lap^{\frac{L}2}\Psi,\Lap^\frac{L}2\RE\rangle_G\label{eq:en-eq-BR6}\\
&\,\phantom{\int_M}+a^3\epsilonLC[G]\ast\nabla\phi\ast\nabla^2\Lap^{\frac{L}2}\phi\ast\Lap^{\frac{L}2}\RB\label{eq:en-eq-BR7}\\
&\,\phantom{\int_M} +(N+1)a^{-3}\Sigma\ast\left(\Lap^{\frac{L}2}\RE\ast\Lap^{\frac{L}2}\RE+\Lap^{\frac{L}2}\RB\ast\Lap^{\frac{L}2}\RB\right)\label{eq:en-eq-BR8}\\
&\,\phantom{\int_M} -2\langle\mathfrak{E}_{L,Border}+\mathfrak{E}_{L,top}+\mathfrak{E}_{L,Junk}^\parallel,\Lap^{\frac{L}2}\RE\rangle_G\label{eq:en-eq-BR9}\\
&\,\phantom{\int_M} -2\langle\mathfrak{B}_{L,Border}+\mathfrak{B}_{L,top}+\mathfrak{B}_{L,Junk}^\parallel,\Lap^{\frac{L}2}\RB\rangle_G\biggr\}\,\vol{G}\label{eq:en-eq-BR10}
\end{align}
\end{subequations}
For \eqref{eq:en-eq-BR1}, we pull $6(N+1){\dot{a}}a^{-1}\lvert\Lap^{\frac{L}2}\RE\rvert_G^2$ to the left. This leaves
\[\int_M-6\frac{\dot{a}}a\lvert\Lap^\frac{L}2\RB\rvert_G^2+3N\frac{\dot{a}}a\lvert\Lap^\frac{L}2\RB\rvert_G^2+9N\frac{\dot{a}}a\lvert\Lap^\frac{L}2\RE\rvert_G^2\,\vol{G},\]
where we can estimate the last two terms up to constant by $\epsilon a^{1-c\sigma}\E^{(L)}(W,\cdot)$ by \eqref{eq:BsN} and can drop the first term since it is nonpositive.\\
Regarding \eqref{eq:en-eq-BR2}, note that \delete{by \eqref{eq:div-to-curl}, }we have
\begin{align*}
\change{a^{-1}\left(\langle\curl_G\Lap^{\frac{L}2}\RE,\Lap^{\frac{L}2}\RB\rangle_G-\right.}&\change{\,\left.\langle\curl_G\Lap^{\frac{L}2}\RB,\Lap^{\frac{L}2}\RE\rangle_G\right)}%=a^4\left(\curl\Lap^{\frac{L}2}\RE\cdot\Lap^{\frac{L}2}\RB-\curl\Lap^{\frac{L}2}\RB\cdot\Lap^{\frac{L}2}\RE\right)\\
%=&\,-a^4\div_g\left(\Lap^{\frac{L}2}\RE\wedge\Lap^{\frac{L}2}\RB\right)
\change{=-a^{-1}\div_G\left(\Lap^{\frac{L}2}\RE\wedge_G\Lap^{\frac{L}2}\RB\right).}
\end{align*}
Hence, the absolute value of \eqref{eq:en-eq-BR2}, using \eqref{eq:wedge} for the wedge product and \eqref{eq:BsN}, can be bounded by:
\begin{align*}
\left\lvert\int_M \change{2a^{-1}(N+1)\div_G\left(\Lap^{\frac{L}2}\RE\wedge_G\Lap^{\frac{L}2}\RB}\right)\,\vol{G}\right\rvert=&\left\lvert\int_M\change{2a^{-1}\langle\nabla N,\Lap^{\frac{L}2}\RE\wedge_G\Lap^{\frac{L}2}\RB\rangle_G}\,\vol{G}\right\rvert\\
\lesssim&\int_M a^{-1}\lvert\nabla N\rvert_G\lvert\Lap^{\frac{L}2}\RE\rvert_G\lvert\Lap^{\frac{L}2}\RB\rvert_G\,\vol{G}\\
\lesssim&\,\epsilon a^{3-c\sigma}\E^{(L)}(W,\cdot)
\end{align*}
For \eqref{eq:en-eq-BR3}, we use the pointwise wedge product estimate \eqref{eq:wedge2} and a priori estimates \eqref{eq:APE} and \eqref{eq:APmidB} to bound it as follows:
\begin{align*}
\lvert\eqref{eq:en-eq-BR3}\rvert\leq&\, 2a^{-1}\lvert\nabla\Lap^\frac{L}2N\rvert_G\left(\lvert\RB\rvert_G\cdot\lvert\Lap^\frac{L}2\RE\rvert_G+\lvert\RE\rvert_G\cdot\lvert\Lap^\frac{L}2\RB\rvert_G\right)\\
\lesssim&\,\epsilon a^{-3}\sqrt{a^4\E^{(L+1)}(N,\cdot)}\sqrt{\E^{(L)}(W,\cdot)}\\
\lesssim&\,\epsilon a^{-3}\left(\E^{(L)}(W,\cdot)+a^4\E^{(L+1)}(N,\cdot)\right)
\end{align*}
We pull the first term of \eqref{eq:en-eq-BR4} to the left as well, and estimate the second using the strong $C_G$-norm estimates \eqref{eq:APPsi} and \eqref{eq:APmidphi} by
\[\sqrt{\epsilon}a^{-1-c\sqrt{\epsilon}}\sqrt{a^4\E^{(L+1)}(N,\cdot)}\sqrt{\E^{(L)}(W,\cdot)}\lesssim a^{-1-c\sqrt{\epsilon}}\E^{(L)}(W,\cdot)+\epsilon a^{-1}\cdot a^4\E^{(L+1)}(N,\cdot)\,.\]
Moving on to \eqref{eq:en-eq-BR5}-\eqref{eq:en-eq-BR7}, we see [using \eqref{eq:APPsi}, \eqref{eq:APmidphi}, \eqref{eq:Sobolev-norm-equiv-zetalow} with $\zeta=\Lap^\frac{L}2\phi$ and \eqref{eq:ibp-trick}]:
\begin{align*}
\change{\lvert\eqref{eq:en-eq-BR5}\rvert\lesssim}&\change{\,\left(a^{-3}\sqrt{a^4\E^{(L+1)}(\phi,\cdot)}+a^{-1}\sqrt{\E^{(L)}(\phi,\cdot)}+a^{-1-c\sqrt{\epsilon}}\sqrt{\E^{(L-2)}(\phi,\cdot)}\right)\sqrt{\E^{(L)}(W,\cdot)}\\
\lesssim&\,\left(\epsilon^\frac18a^{-3}+a^{-1-c\sqrt{\epsilon}}\right)\E^{(L)}(W,\cdot)+\epsilon^{-\frac18}a^{-3}\cdot a^4\E^{(L+1)}(\phi,\cdot)+a^{-1}\E^{(\leq L)}(\phi,\cdot)\\[0.5em]}
\left\lvert\eqref{eq:en-eq-BR6}\right\rvert\lesssim&\,\sqrt{\epsilon}a^{1-c\sqrt{\epsilon}}\sqrt{\E^{(L+1)}(\phi,\cdot)}\sqrt{\E^{(L)}(W,\cdot)}\\
\lesssim&\,a^{1-c\sqrt{\epsilon}}\E^{(L)}(W,\cdot)+\epsilon a^{1-c\sqrt{\epsilon}}\E^{(L+1)}(\phi,\cdot)\,,\\[0.5em]
\left\lvert\eqref{eq:en-eq-BR7}\right\rvert\lesssim&\,\sqrt{\epsilon}a^{1-c\sqrt{\epsilon}}\cdot a^2\|\nabla^2\Lap^{\frac{L}2}\phi\|_{L^2_G}\cdot\sqrt{\E^{(L)}(W,\cdot)}\\
\lesssim&\,\sqrt{\epsilon}a^{1-c\sqrt{\epsilon}}\left(\sqrt{\E^{(L+1)}(\phi,\cdot)}+a^{-c\sqrt{\epsilon}}\sqrt{\E^{(L-1)}(\phi,\cdot)}\right)\cdot\sqrt{\E^{(L)}(W,\cdot)}\,\\
\lesssim&\,a^{1-c\sqrt{\epsilon}}\E^{(L)}(W,\cdot)+\epsilon a^{1-c\sqrt{\epsilon}}\left[\E^{(L+1)}(\phi,\cdot)+\E^{(L)}(\phi,\cdot)+\E^{(\leq L-2)}(\phi,\cdot)\right]
\end{align*}
We can estimate \eqref{eq:en-eq-BR8} by $\epsilon a^{-3}\E^{(L)}(W,\cdot)$ as usual, and obtain the following in summary:
\begin{align*}
&\,-\del_t\E^{(L)}(W,\cdot)+8\pi C^2a^{-3}\int_M(N+1)\langle \change{\Lap^\frac{L}2\Sigma,\Lap^\frac{L}2\RE}\rangle_G\,\vol{G}+6\frac{\dot{a}}a\int_M(N+1)\lvert\Lap^\frac{L}2\RE\rvert_G^2\,\vol{G}\\
\lesssim&\,\left(\epsilon a^{-3}+a^{-1-c\sqrt{\epsilon}}\right)\E^{(L)}(W,\cdot)+a^{-1}\E^{(L+1)}(\phi,\cdot)+a^{-1}\E^{(L)}(\phi,\cdot)\\
&\,+\epsilon a^{-3}\cdot a^{4}\E^{(L+1)}(N,\cdot)+a^{-1}\E^{(\leq L-2)}(\phi,\cdot)\\
&\,\left[\|\mathfrak{E}_{L,Border}\|_{L^2_G}+\|\mathfrak{E}_{L,top}\|_{L^2_G}+\|\mathfrak{E}_{L,Junk}^\parallel\|_{L^2_G}\right.\\
&\,\left.+\|\mathfrak{B}_{L,Border}\|_{L^2_G}+\|\mathfrak{B}_{L,top}\|_{L^2_G}+\|\mathfrak{B}_{L,Junk}^\parallel\|_{L^2_G}\right]\sqrt{\E^{(L)}(W,\cdot)}
\end{align*}
We can now apply Corollary \ref{cor:en-est-lapse} for $2l=L$ to estimate the lapse energy in the second line (leading to borderline scalar field energy and $\Sigma$-energy contributions as well as junk terms), and insert the borderline (see \eqref{eq:L2-Border-BR}), top (see \eqref{eq:L2-top-E} and \eqref{eq:L2-top-B}) and junk estimates (see \eqref{eq:L2-junk-BR-par}), dealing with the lapse energies there analogously. \change{In particular, the top order curvature terms arise as follows:
\begin{align*}
\|\mathfrak{E}_{L,top}\|_{L^2_G}\sqrt{\E^{(L)}(W,\cdot)}\lesssim&\,\sqrt{\epsilon} a^{-1-c\sqrt{\epsilon}}\sqrt{a^4\E^{(L-1)}(\Ric,\cdot)}\sqrt{\E^{(L)}(W,\cdot)}\\
\lesssim&\,\epsilon^\frac18a^{-1-c\sqrt{\epsilon}}\E^{(L)}(W,\cdot)+\epsilon^\frac{7}8a^{-1}\cdot a^4\E^{(L-1)}(\Ric,\cdot)\\
\|\mathfrak{B}_{L,top}\|\sqrt{\E^{(L)}(W,\cdot)}\lesssim&\,\epsilon a^{-3}\sqrt{a^4\E^{(L-1)}(\Ric,\cdot)}\sqrt{\E^{(L)}(W,\cdot)}\\
\lesssim&\,\epsilon^\frac18 a^{-3}\E^{(L)}(W,\cdot)+\epsilon^\frac{15}8 a^{-3}\cdot a^4\E^{(L-1)}(\Ric,\cdot)
\end{align*}
Hence, both top order curvature terms can be bounded by $\epsilon^\frac78a^{-3}\cdot a^4\E^{(L-1)}(\Ric,\cdot)$.\\
Integrating the inequality yields \eqref{eq:en-est-BR}.\\}

For \eqref{eq:en-est-BR0}, we get applying \eqref{eq:REEqE} and \eqref{eq:REEqB} and again using that $\RE$ and $\RB$ are symmetric and tracefree:
\begin{align*}
-\del_t\E^{(0)}(W,\cdot)=&\,\int_M\biggr\{\frac{\dot{a}}a(-6(N+1)+9N)\left(\lvert\RE\rvert_G^2+\lvert\RB\rvert_G^2\right)\\
&\,\phantom{\int_M}+2(N+1)\left(\langle\curl\RE,\RB\rangle_G-\langle\curl\RB,\RE\rangle_G\right)\\
&\,\phantom{\int_M}+2\left(\langle\nabla N\wedge \RB,\RE\rangle_G-\langle\nabla N\wedge\RE,\RB\rangle_G\right)\\
&\,\phantom{\int_M}+(N+1)a^{-3}\Sigma\ast\left(\RE\ast\RE+\RB\ast\RB\right)\\
&\,\phantom{\int_M}-8\pi a^{-3}(N+1)(\Psi+C)^2\langle\Sigma,\RE\rangle_G\\
&\,\phantom{\int_M}+\left[\dot{a}a^3\nabla\phi\ast\nabla\phi+a(\Psi+C)\cdot\nabla N\ast\nabla\phi\right]\ast\RE\\
&\,\phantom{\int_M}+a(N+1)\left[\nabla\phi\ast\nabla\Psi+\Sigma\ast\nabla\phi\ast\nabla\phi+(\Psi+C)\nabla^2\phi\right]\ast\RE\\
&\,\phantom{\int_M}+(N+1)\epsilonLC[G]\ast\left(a^3\nabla^2\phi\ast\nabla\phi+a^{-1}(\Psi+C)\Sigma\ast\nabla\phi\right)\ast\RB\biggr\}\,\vol{G}
\end{align*}
The first two lines are treated as in the general case. For the third line, we get $\epsilon a^{3-c\sigma}\E^{(0)}(W,\cdot)$ with \eqref{eq:wedge2} and \eqref{eq:BsN}, while the fourth term is bounded by $\epsilon a^{-3}\E^{(0)}(W,\cdot)$ with \eqref{eq:APSigma}. This leaves the surviving matter terms in the final four lines.\\
We pull $\int_M8\pi a^{-3}(N+1)C^2\langle\Sigma,\RE\rangle_G\vol{G}$ to the left as before. For the remaining terms, we can apply a priori estimates \eqref{eq:APPsi}, \eqref{eq:APmidPsi} and \eqref{eq:APmidphi}, the bootstrap assumption \eqref{eq:BsN} and Lemma \ref{lem:lapse-maxmin} for $N$, which yields the following bound up to constant remaining terms in the last three lines:
\begin{align*}
\sqrt{\E^{(0)}(W,\cdot)}\cdot&\left[a^{-1}\cdot a^2\|\nabla^2\phi\|_{L^2_G}+\sqrt{\epsilon}a^{-1-c\sqrt{\epsilon}}\sqrt{\E^{(0)}(\phi,\cdot)}\right.\\
&\left.+\left(\epsilon a^{-3}+\sqrt{\epsilon}a^{-1-c\sqrt{\epsilon}}\right)\sqrt{\E^{(0)}(\Sigma,\cdot)}\right]
\end{align*}
%\begin{align*}
%\sqrt{\E^{(0)}(W,\cdot)}&\,\left[\epsilon a^{-3}\sqrt{\E^{(0)}(\Sigma,\cdot)}\right.\\
%&+\sqrt{\epsilon}a^{-1-c\sqrt{\epsilon}}\sqrt{\E^{(0)}(\phi,\cdot)}+\epsilon a^{3-c\sigma}\sqrt{\E^{(0)}(\phi,\cdot)}\\
%&\left.+\sqrt{\epsilon}a^{-1-c\sqrt{\epsilon}}\sqrt{\E^{(0)}(\phi,\cdot)}+\epsilon a^{1-c\sqrt{\epsilon}}\sqrt{\E^{(0)}(\Sigma,\cdot)}+a^{-1}\cdot a^2\|\nabla^2\phi\|_{L^2_G}\right.\\
%&\left.+\sqrt{\epsilon}a^{1-c\sqrt{\epsilon}}\sqrt{\E^{(0)}(\phi,\cdot)}+\sqrt{\epsilon}a^{-1-c\sqrt{\epsilon}}\sqrt{\E^{(0)}(\Sigma,\cdot)}\right]
%\end{align*}
Applying \eqref{eq:Sobolev-norm-equiv-zetalow} to the scalar field norm and then \eqref{eq:ibp-trick}, this leads to \eqref{eq:en-est-BR0} along with the previous observations.
% to the last term in the third line, we can bound it by
%\[a^{-1}\sqrt{\E^{(1)}(\phi,\cdot)}+a^{-1-c\sqrt{\epsilon}}\sqrt{\E^{(0)}(\phi,\cdot)}\,.\]
%Hence, we can estimate the entire block, up to constant and updating $c$, by
%\begin{equation*}
%a^{-1-c\sqrt{\epsilon}}\E^{(0)}(W,\cdot)+a^{-1}\E^{(1)}(\phi,\cdot)+\epsilon a^{-1-c\sqrt{\epsilon}}\E^{(0)}(\phi,\cdot)+\epsilon a^{-3}\E^{(0)}(\Sigma,\cdot)
%\end{equation*}
%The statement now follows by collecting all of these observations and integrating.
\end{proof}


\subsection{Energy estimates for the second fundamental form}\label{subsec:en-Sigma}

For the energy estimates for $\Sigma$, we again first derive \change{even order }integral estimates:

\begin{lemma}[Energy estimates for the second fundamental form \change{for even orders}]\label{lem:en-est-Sigma} Let $t\in(t_{Boot},t_0]$. Then, one has:
\begin{align*}
\numberthis\label{eq:en-est-Sigma0}&\,\E^{(0)}(\Sigma,t)+2\int_t^{t_0}\int_M\left[a(s)^{-3}(N+1)\langle\RE,\Sigma\rangle_G+\frac{\dot{a}(s)}{a(s)}(N+1)\lvert\Lap^{\frac{L}2}\Sigma\rvert_G^2\right]\,\vol{G}\,ds\\
\lesssim&\,\E^{(0)}(\Sigma,t_0)+\int_t^{t_0}\epsilon^\frac18a(s)^{-3}\E^{(0)}(\Sigma,s)\,ds+\int_t^{t_0}\epsilon^{-\frac18}a(s)^{-3}\E^{(0)}(\phi,s)\,ds\\
\end{align*}
For $L\in 2\N, L\leq \change{18}$, the following holds:
\begin{align*}
\numberthis\label{eq:en-est-Sigma}&\,\E^{(L)}(\Sigma,t)+2\int_t^{t_0}\int_M\left[a(s)^{-3}(N+1)\langle\Lap^\frac{L}2\RE,\Lap^{\frac{L}2}\Sigma\rangle_G+\frac{\dot{a}(s)}{a(s)}(N+1)\lvert\Lap^{\frac{L}2}\Sigma\rvert_G^2\right]\,\vol{G}\,ds\\
&\,\lesssim\E^{(L)}(\Sigma,t_0)+\int_t^{t_0}\epsilon^\frac18a(s)^{-3}\E^{(L)}(\Sigma,s)\,ds\\
&\,+\int_t^{t_0}\Bigr\{\epsilon^{-\frac18}a(s)^{-3}\E^{(L)}(\phi,s)+\epsilon^\frac{15}8a(s)^{5-c\sigma}\E^{(L-1)}(\Ric,s)+\epsilon^2a(s)^{-3}\E^{(L-2)}(\Ric,s)\\
&\,\phantom{\int_t^{t_0}}+\epsilon^{\frac{15}8}a(s)^{-3-c\sqrt{\epsilon}}\E^{(\leq L-2)}(\Sigma,s)+\left(\epsilon^\frac{15}8a(s)^{-3-c\sqrt{\epsilon}}+\epsilon a(s)^{-1-c\sqrt{\epsilon}}\right)\E^{(\leq L-2)}(\phi,s)\\
&\,\phantom{\int_t^{t_0}}+\underbrace{\epsilon^{\frac{15}8}a(s)^{-3-c\sqrt{\epsilon}}\E^{(\leq L-4)}(\Ric,s)}_{\text{not present for }L=2}\Bigr\}\,ds
\end{align*}
\end{lemma}
\begin{remark}
The main hurdle of dealing with the second fundamental form is that a high order curvature term occurs in the evolution equation. It is to precisely this end that the Bel-Robinson variables needed to be introduced, since \eqref{eq:comeq-Ham-BR} is what facilitates controlling said term without having to use $\E^{(L)}(\Ric,\cdot)$ or similar high order metric energies. Again, the resulting leading terms will turn out to have definite sign when combined with the Bel-Robinson energy estimates above.
\end{remark}
\begin{proof}
Here, we omit the proof for the inequality at order zero since is completely analogous in structure to the one for orders 2 and higher and the only differences that arise are that lower order error terms do not occur. \\
Once again, we start out by differentiating $-\E^{(L)}(\Sigma,\cdot)$ and insert \eqref{eq:comeq-Sigma}:
\begin{align*}
-\del_t\E^{(L)}(\Sigma,\cdot)=&\,\int_M -2\left\langle\del_t\Lap^{\frac{L}2}\Sigma,\Lap^{\frac{L}2}\Sigma\right\rangle_G+(\del_tG^{-1})\ast G^{-1}\ast\Lap^{\frac{L}2}\Sigma\ast\Lap^{\frac{L}2}\Sigma-3N\frac{\dot{a}}a\lvert\Lap^{\frac{L}2}\Sigma\rvert_G^2\,\vol{G}\\
=&\,\int_M \left\{2a\langle\nabla^2\Lap^{\frac{L}2} N,\Lap^{\frac{L}2}\Sigma\rangle_G-2a(N+1)\langle\Lap^{\frac{L}2}\Ric[G],\Lap^{\frac{L}2}\Sigma\rangle_G+\right.\\
&\,\phantom{\int_M}+(\del_tG^{-1})\ast G^{-1}\ast\Lap^{\frac{L}2}\Sigma\ast\Lap^{\frac{L}2}\Sigma-3N\frac{\dot{a}}a\lvert\Lap^{\frac{L}2}\Sigma\rvert_G^2\\
&\,\phantom{\int_M}\left.-2\langle\mathfrak{S}_{L,Border},\Lap^{\frac{L}2}\Sigma\rangle_G-2\langle\mathfrak{S}_{L,Junk}^\parallel,\Lap^{\frac{L}2}\Sigma\rangle_G\right\}\,\vol{G}
\end{align*}

For the first term, one can use \eqref{eq:Sobolev-norm-equiv-zetalow}, Corollary \ref{cor:en-est-lapse} at order $L$ and \eqref{eq:ibp-trick} to bound its absolute value by the following:
\begin{align*}
\lesssim&\,a\|\Lap^{\frac{L}2}N\|_{\dot{H}^2_G}\sqrt{\E^{(L)}(\Sigma,\cdot)}\\
\lesssim&\,\left[a^{-3}\sqrt{a^8\E^{(L+2)}(N,\cdot)}+a^{1-c\sqrt{\epsilon}}\sqrt{\E^{(L)}(N,\cdot)}\right]\sqrt{\E^{(L)}(\Sigma,\cdot)}\\
\lesssim&\,\biggr[\epsilon a^{-3}\sqrt{\E^{(L)}(\Sigma,\cdot)}+a^{-3}\E^{(L)}(\phi,\cdot)+\epsilon a^{-3-c\sqrt{\epsilon}}\left(\sqrt{\E^{(\leq L-2)}(\Sigma,\cdot)}+\sqrt{\E^{(\leq L-2)}(\phi,\cdot)}\right)\\
&\,+\underbrace{\left(\epsilon^2a^{-3-c\sqrt{\epsilon}}+\epsilon a^{1-c\sigma}\right)\sqrt{\E^{(\leq L-3)}(\Ric,\cdot)}}_{\text{not present for }L=2}\biggr]\sqrt{\E^{(L)}(\Sigma,\cdot)}\\
%\lesssim&\,\Bigr\{\epsilon^\frac{15}8 a^{-3}\E^{(L)}(\Sigma,\cdot)+\epsilon^{-\frac18}a^{-3}{\E^{(L)}(\phi,\cdot)}+\epsilon^\frac{15}8 a^{-3-c\sqrt{\epsilon}}\left[\E^{(\leq L-2)}(\Sigma,\cdot)+\E^{(\leq L-2)}(\phi,\cdot)\right]\\
%&\,+\underbrace{\left(\epsilon^\frac{31}8a^{-3-c\sqrt{\epsilon}}+\epsilon^2a^{1-c\sigma}\right)\E^{(\leq L-3)}(\Ric,\cdot)}_{\text{not present for }L=2}\Bigr\}+\left(\epsilon^\frac18 a^{-3}+a^{1-c\sigma}\right)\E^{(L)}(\Sigma,\cdot)\\
\lesssim&\,\left(\epsilon^\frac18 a^{-3}+a^{1-c\sigma}\right)\E^{(L)}(\Sigma,\cdot)+\epsilon^{-\frac18}a^{-3}\E^{(L)}(\phi,\cdot)+\epsilon a^{-3-c\sqrt{\epsilon}}\left[\E^{(\leq L-2)}(\Sigma,\cdot)+\E^{(\leq L-2)}(\phi,\cdot)\right]\\
&\,+\underbrace{\left(\epsilon^{\frac{31}8}a^{-3}+\epsilon^2 a^{1-c\sigma}\right)\E^{(L-2)}(\Ric,\cdot)+\left(\epsilon^\frac{31}8a^{-3-c\sqrt{\epsilon}}+\epsilon^2a^{1-c\sigma}\right)\E^{(\leq L-4)}(\Ric,\cdot)}_{\text{not present for }L=2}
\end{align*}
Next, we replace the high order curvature term as follows, using the commuted rescaled Hamiltonian constraint equation \eqref{eq:comeq-Ham-BR} that $\Lap^\frac{L}2\Sigma$ is tracefree and symmetric:
\begin{align*}
&\,\int_M-2a(N+1)\langle\Lap^{\frac{L}2}\Ric[G],\Lap^{\frac{L}2}\Sigma\rangle_G\,\vol{G}\\
=&\,\int_M-2(N+1)a^{-3}\langle\Lap^{\frac{L}2}\RE,\Lap^{\frac{L}2}\Sigma\rangle_G-2(N+1)\frac{\dot{a}}{a}|\Lap^{\frac{L}2}\Sigma|_G^2+\langle\mathfrak{H}_{L,Border}+\mathfrak{H}_{L,Junk}^\parallel,\Lap^{\frac{L}2}\Sigma\rangle_G\,\vol{G}
\end{align*}
We pull the first two terms to left, only keeping the error terms on the right. After inserting the borderline and junk term estimates for the Hamiltonian constraint equations (\eqref{eq:L2-Border-H} and \eqref{eq:L2-junk-H-par}) and the evolution equation itself (\eqref{eq:L2-Border-S} and \eqref{eq:L2-junk-S}), as well as bounding $\lvert\del_tG^{-1}\rvert\lesssim \epsilon a^{-3}$ and inserting \eqref{eq:BsN} as usual, we obtain \eqref{eq:en-est-Sigma} by integrating.
\end{proof}

\noindent Additionally, we can exploit the structure of the momentum constraint equations to gain an elliptic estimate for $\E^{(\change{L}+1)}(\Sigma,\cdot)$. Crucially, the upper bound only depends on $\Sigma$-, scalar field and Bel-Robinson energies up to order $\change{L}$, and appropriately small and time-scaled curvature contributions up to order $\change{L}-1$. This will allow us to close the argument since we do not need to consider the Bel-Robinson energy at order $\change{L}+1$ to control $\Sigma$ at that order%to balance out error terms as on the left hand side of \eqref{eq:en-est-Sigma}
, which would require higher order scalar field and curvature energies\change{.}

\begin{lemma}[\change{Odd } order energy estimate for the second fundamental form]\label{lem:en-est-Sigma-top} For any $L\in 2\Z_+$, $2\leq L\leq \change{18}$, we have
\begin{align*}
\numberthis\label{eq:en-est-Sigma-top}a^4\E^{(\change{L}+1)}(\Sigma,\cdot)\lesssim&\,\left(a^{4-c\sqrt{\epsilon}}+\epsilon a^{2-c\sqrt{\epsilon}}\right)\E^{(\change{L})}(\Sigma,\cdot)+\E^{(\change{L})}(\phi,\cdot)+\E^{(\change{L})}(W,\cdot)+\epsilon^2a^4\E^{(\change{L}-1)}(\Ric,\cdot)\\
&\,+\change{\epsilon a^{-c\sqrt{\epsilon}}}\E^{(\leq \change{L}-2)}(\phi,\cdot)+a^{2-c\sqrt{\epsilon}}\E^{(\leq \change{L}-2)}(\Sigma,\cdot)+\epsilon a^{2-c\sqrt{\epsilon}}\E^{(\leq \change{L}-2)}(\Ric,\cdot)\,.
\end{align*}
\change{For $L=0$, one analogously has}
\begin{equation}\label{eq:en-est-Sigma-1}
\change{a^4\E^{(1)}(\Sigma,\cdot)\lesssim\,\left(a^{4-c\sqrt{\epsilon}}+\epsilon a^{2-c\sqrt{\epsilon}}\right)\E^{(0)}(\Sigma,\cdot)+\E^{(0)}(\phi,\cdot)+\E^{(0)}(W,\cdot)\,.}
\end{equation}
\end{lemma}
\begin{proof}\change{We prove the statement for \changefinal{$L\geq2$}, since the proof of \eqref{eq:en-est-Sigma-1} is entirely analogous.}\\
By \cite[(A.22)]{AM03}, since $(\Sigma_t,g)$ is a three-dimensional compact Riemannian manifold for any $t\in(t_{Boot},t_0]$, any tracefree $(0,2)$ tensor $U_{ij}$ on $(\Sigma_t,g)$ satisfies
\begin{equation}\label{eq:curl-div-elliptic}
\int_{\Sigma_t} \lvert\nabla U\rvert_g^2+3\Ric[g]\cdot U\cdot U-\frac{R[g]}2\lvert U\rvert_g^2\,\vol{g}=\int_{\Sigma_t} \lvert\curl U\rvert_g^2+\frac32\lvert\div_g U\rvert_g^2\,\vol{g}\,.
\end{equation}
In particular, for $U=\Lap^{\frac{\change{L}}2}\Sigma$ and after rescaling, this reads:
\begin{align*}
&\,\int_M \left\lvert\nabla\Lap^{\frac{\change{L}}2}\Sigma\right\rvert_G^2+3{\left(\Ric[G]^\sharp\right)^i}_j{\left(\Lap^{\frac{\change{L}}2}\Sigma^\sharp\right)^j}_l{\left(\Lap^{\frac{\change{L}}2}\Sigma^\sharp\right)^l}_i-\frac{R[G]}2\lvert\Lap^\frac{\change{L}}2\Sigma\rvert_G^2\,\vol{G}\\
=&\,\int_M\frac32\lvert\div_G\Lap^\frac{\change{L}}2\Sigma\rvert_G^2+a^2\lvert\curl\Lap^\frac{\change{L}}2\Sigma\rvert_G^2\,\vol{G}
\end{align*}
The last two terms on the left hand side can be estimated by $(1+\sqrt{\epsilon}a^{-c\sqrt{\epsilon}})\E^{(\change{L})}(\Sigma,\cdot)$ in absolute value using the strong $C_G$-norm estimate \eqref{eq:APmidRic}. Thus, inserting the Laplace-commuted rescaled momentum constraint equations \eqref{eq:comeq-mom-div} and \eqref{eq:comeq-mom-curl}, we obtain for a suitable constant $K>0$:
\begin{align*}
&\E^{(\change{L}+1)}(\Sigma,\cdot)-K\left(1+\sqrt{\epsilon}a^{-c\sqrt{\epsilon}}\right)\E^{(\change{L})}(\Sigma,\cdot)\\
\lesssim&\,\int_M\left\{\lvert\Psi+C\rvert^2\left\lvert\nabla\Lap^{\frac{\change{L}}2}\phi\right\rvert_G^2+\lvert\nabla\phi\rvert_G^2\left\lvert\Lap^{\frac{\change{L}}2-1}\Ric[G]\right\rvert_G^2+\lvert\Sigma\rvert_G^2\left\lvert\nabla\Lap^{\frac{\change{L}}2-1}\Ric[G]\right\rvert_G^2+\lvert\mathfrak{M}_{\change{L},Junk}\rvert_G^2\right.\\
&\,\phantom{\int_M}\left.+a^{-4}\lvert\Lap^\frac{\change{L}}2\RB\rvert_G^2+\lvert\Sigma\rvert_G^2\left\lvert\nabla\Lap^{\frac{\change{L}}2-1}\Ric[G]\right\rvert_G^2+\lvert\nabla\Sigma\rvert_G^2\left\lvert\nabla^2\Lap^{\frac{\change{L}}2-2}\Ric[G]\right\rvert_G^2+\left\lvert\tilde{\mathfrak{M}}_{\change{L},Junk}\right\rvert_G^2\,\right\}\vol{G}
\end{align*}
\noindent After rearranging, using the strong $C_G$-norm estimates \eqref{eq:APPsi}, \eqref{eq:APmidphi}, \eqref{eq:APSigma} and \eqref{eq:APmidSigma} and multiplying by $a^4$ on both sides, we get
\begin{align*}
a^4\E^{(\change{L}+1)}(\Sigma,\cdot)\lesssim&\,\left(1+\sqrt{\epsilon}a^{-c\sqrt{\epsilon}}\right)a^4\E^{(\change{L})}(\Sigma,\cdot)+\E^{(\change{L})}(\phi,\cdot)+\E^{(\change{L})}(W,\cdot)+\epsilon^2a^4\E^{(\change{L}-1)}(\Ric,\cdot)\\
&\,+\epsilon^2a^{4-c\sqrt{\epsilon}}\E^{(\change{L}-2)}(\Ric,\cdot)+a^4\|\mathfrak{M}_{\change{L},Junk}\|_{L^2_G}^2+a^4\|\tilde{\mathfrak{M}}_{\change{L},Junk}\|_{L^2_G}^2\,.
\end{align*}
The statement follows inserting the estimates \eqref{eq:L2-junk-M} and \eqref{eq:L2-junk-Mtilde}.
\end{proof}

\subsection{Energy estimates for the curvature}\label{subsec:en-Ric}

To control commutator errors, we will also need some additional estimates on curvature energies. Unlike the other energies, these inequalities do not rely on any delicate structure within the equations and instead just rely on pointwise estimates, the Young inequality and near-coercivity of energies in the sense of Lemma \ref{lem:Sobolev-norm-equivalence-improved}. For the sake of convenience, we phrase these estimates for $\E^{(L-2)}(\Ric,\cdot)$ since this is the order needed when improving behaviour of the total energy at order $L$.

\begin{lemma}[Curvature energy estimates \change{at even orders}]\label{lem:en-est-Ric}
Let $L\in 2\Z$, $4\leq L\leq \change{16}$ and $t\in(t_{Boot},t_0]$. Then, one has
\begin{align*}
\numberthis\label{eq:en-est-Ric}\E^{(L-2)}(\Ric,t)\lesssim&\,\E^{(L-2)}(\Ric,t_0)+\int_t^{t_0}\left(\epsilon^\frac18a(s)^{-3}+a(s)^{8-c\sigma}\right)\E^{(L-2)}(\Ric,s)\,ds\\
&\,+\int_t^{t_0}\Bigr\{\epsilon^{-\frac18}a(s)^{-3}\left(\E^{(L)}(\phi,s)+\E^{(L)}(\Sigma,s)\right)\\
&\,\phantom{\int_t^{t_0}}+\epsilon^{-\frac18}a(s)^{-3-c\sqrt{\epsilon}}\left(\E^{(\leq L-2)}(\phi,s)+\E^{(\leq L-2)}(\Sigma,s)\right)\\
&\,\phantom{+\int_t^{t_0}}+\epsilon^\frac78a(s)^{-3-c\sqrt{\epsilon}}\E^{(\leq L-4)}(\Ric,s)\Bigr\}\,ds\,.
\end{align*}
Additionally,
\begin{align*}
\numberthis\label{eq:en-est-Ric0}\E^{(0)}(\Ric,t)\lesssim&\,\E^{(0)}(\Ric,t_0)+\int_t^{t_0}\epsilon^\frac18a(s)^{-3}\E^{(0)}(\Ric,s)\,ds\\
&\,+\int_t^{t_0}\epsilon^{-\frac18}a(s)^{-3}\left(\E^{(0)}(\phi,s)+\E^{(0)}(\Sigma,s)\right)ds\,.
\end{align*}
\end{lemma}
\begin{proof}
First, we note that
\[\|\div_G^\sharp\nabla\Lap^{\frac{L}2-1}\Sigma\|_{L^2_G}\lesssim \|\nabla^2\Lap^{\frac{L}2-1}\Sigma\|_{L^2_G}\lesssim \|\Lap^{\frac{L}2}\Sigma\|_{L^2_G}+a^{-c\sqrt{\epsilon}}\sqrt{\E^{(L-2)}(\Sigma,\cdot)}\]
holds using the low order version of \eqref{eq:Sobolev-norm-equiv-T2l} with $\mathfrak{T}=\Lap^{\frac{L}2-1}\Sigma$ for $l=2$, and similarly
\[\|\nabla^2\Lap^{\frac{L}2-1}N\|_{L^2_G}\lesssim \|\Lap^{\frac{L}2}N\|_{L^2_G}+a^{-c\sqrt{\epsilon}}\sqrt{\E^{(L-2)}(N,\cdot)}\]
using \eqref{eq:Sobolev-norm-equiv-zeta2l} at order $2$. Now, using $\Lap^{\frac{L}2-1} G=0$ for $L\geq 4$, we continue as usual by applying \eqref{eq:comeq-Ric-even} to the expression below:
\begin{align*}
-\del_t\E^{(L-2)}(\Ric,\cdot)%=&\,\int_M\Bigr\{-2\langle\del_t\Lap^{\frac{L}2-1}\Ric[G],\Lap^{\frac{L}2-1}\Ric[G]\rangle_G-3N\frac{\dot{a}}a\lvert\Lap^{\frac{L}2-1}\Ric[G]\rvert_G^2\,\\
%&\,\phantom{\int_M}-2\left(\del_tG^{-1}\right)\ast(G^{-1})\ast\Lap^{\frac{L}2-1}\Ric[G]\ast\Lap^{\frac{L}2-1}\Ric[G]\Bigr\}\,\vol{G}\\
\lesssim&\,\int_M \Bigr\{a^{-3}\left(\lvert\Lap^{\frac{L}2}\Sigma\rvert_G+\lvert\nabla^2\Lap^{\frac{L}2-1}\Sigma\rvert_G\right)\lvert\Lap^{\frac{L}2-1}\Ric[G]\rvert_G\\
&\,+\frac{\dot{a}}a\left(\lvert\nabla^2\Lap^{\frac{L}2-1}N\rvert_G+\lvert\Lap^{\frac{L}2}N\rvert_G\right)\lvert\Lap^{\frac{L}2-1}\Ric[G]\rvert_G\\
&\,+\left(\lvert\mathfrak{R}_{L-2,Border}\rvert_G+\lvert\mathfrak{R}_{L-2,Junk}\rvert_G\right)\cdot\lvert\Lap^{\frac{L}2-1}\Ric[G]\rvert_G\\
&\,+a^{-3}\Sigma\ast\Lap^{\frac{L}2-1}\Ric[G]\ast\Lap^{\frac{L}2-1}\Ric[G]+N\frac{\dot{a}}a\lvert\Lap^{\frac{L}2-1}\Ric[G]\rvert_G^2\Bigr\}\,\vol{G}
\end{align*}
Due to the estimates above as well as \eqref{eq:APSigma} and \eqref{eq:BsN}, this implies
\begin{align*}
-\del_t\E^{(L-2)}(\Ric,\cdot)\lesssim&\,a^{-3}\left[\sqrt{\E^{(L)}(\Sigma,\cdot)}+\sqrt{\E^{(L)}(N,\cdot)}\right]\sqrt{\E^{(L-2)}(\Ric,\cdot)}\\
&\,+a^{-3-c\sqrt{\epsilon}}\left[\sqrt{\E^{(\leq L-2)}(\Sigma,\cdot)}+\sqrt{\E^{(\leq L-2)}(N,\cdot)}\right]\sqrt{\E^{(L-2)}(\Ric,\cdot)}\\
&\,+\left(\|\mathfrak{R}_{L-2,Border}\|_{L^2_G}+\|\mathfrak{R}_{L-2,Junk}\|_{L^2_G}\right)\sqrt{\E^{(L-2)}(\Ric,\cdot)}\\
&\,+\epsilon a^{-3}\E^{(L-2)}(\Ric,\cdot)\,.
\end{align*}
Using Corollary \ref{cor:en-est-lapse} at order $L$ and distributing terms containing $\E^{(L-3)}(\Ric,\cdot)$ with \eqref{eq:ibp-trick} as usual, we get
\begin{align*}
%-\del_t\E^{(L-2)}(\Ric,\cdot)\lesssim&\,a^{-3}\left[\sqrt{\E^{(L)}(\Sigma,\cdot)}+\sqrt{\E^{(L)}(\phi,\cdot)}\right]\sqrt{\E^{(L-2)}(\Ric,\cdot)}\\
%&\,+a^{-3-c\sqrt{\epsilon}}\left[\sqrt{\E^{(\leq L-2)}(\Sigma,\cdot)}+\sqrt{\E^{(\leq L-2)}(\phi,\cdot)}\right]\sqrt{\E^{(L-2)}(\Ric,\cdot)}\\
%&\,+\left[\epsilon^\frac18a^{-3}+a^{8-c\sigma}\right]\E^{(L-2)}(\Ric,\cdot)+\left[\epsilon^\frac{31}8a^{-3-c\sqrt{\epsilon}}+\epsilon^2a^{8-c\sigma}\right]\E^{(\leq L-4)}(\Ric,\cdot)\\
%&\,+\left(\|\mathfrak{R}_{L-2,Border}\|_{L^2_G}+\|\mathfrak{R}_{L-2,Junk}\|_{L^2_G}\right)\sqrt{\E^{(L-2)}(\Ric,\cdot)}\\
-\del_t\E^{(L-2)}(\Ric,\cdot)\lesssim&\,\left[\epsilon^{\frac18} a^{-3}+a^{8-c\sigma}\right]\E^{(L-2)}(\Ric,\cdot)+\epsilon^{-\frac18}a^{-3}\left[\E^{(L)}(\Sigma,\cdot)+\E^{(L)}(\phi,\cdot)\right]\\
&\,+\left[\epsilon^\frac{31}8a^{-3-c\sqrt{\epsilon}}+\epsilon^2a^{8-c\sigma}\right]\E^{(\leq L-4)}(\Ric,\cdot)\\
&\,+\epsilon^{-\frac18}a^{-3-c\sqrt{\epsilon}}\left[\E^{(\leq L-2)}(\Sigma,\cdot)+\E^{(\leq L-2)}(\phi,\cdot)\right]\\
&\,+\left(\|\mathfrak{R}_{L-2,Border}\|_{L^2_G}+\|\mathfrak{R}_{L-2,Junk}\|_{L^2_G}\right)\sqrt{\E^{(L-2)}(\Ric,\cdot)}\,.
\end{align*}
Equation \eqref{eq:en-est-Ric} now follows inserting the borderline and junk term estimates \eqref{eq:L2-Border-R-even} and \eqref{eq:L2-junk-R-even} and applying the lapse energy estimates from Corollary \ref{cor:en-est-lapse}.\\
Equation \eqref{eq:en-est-Ric0} follows almost identically by inserting \eqref{eq:REEqRic} instead of \eqref{eq:comeq-Ric-even} as well as \eqref{eq:REEqG} for the additional $\del_tG\ast(\Ric[G]+\nicefrac29G)$-terms. These can be estimated as
\begin{align*}
\lesssim&\,\int_Ma^{-3}\Sigma\ast\left(\Ric[G]+\frac29G\right)+\frac{\dot{a}}a N\cdot G\ast\left(\Ric[G]+\frac29G\right)\,\vol{G}\\
\lesssim&\,a^{-3}\left(\sqrt{\E^{(0)}(\Sigma,\cdot)}+\sqrt{\E^{(0)}(N,\cdot)}\right)\sqrt{\E^{(0)}(\Ric,\cdot)}\,,
\end{align*}
which can be treated as at higher orders.
\end{proof}

%Furthermore, we will need the following strongly degenerate estimate to deal with a high order commutator:
%\begin{lemma} For $L\in2\Z_+$, we have
%\begin{align*}
%\E^{(L)}(\Ric,\cdot)\lesssim&\, a^{-6}\E^{(L)}(W,\cdot)+a^{-4}\E^{(L)}(\Sigma,\cdot)+\epsilon^2a^{-8-c\sqrt{\epsilon}}\E^{(\leq L-2)}(\Sigma,\cdot)+\left(a^{-8}+\epsilon a^{-8-c\sqrt{\epsilon}}\right)\E^{(\leq L)}(\phi,\cdot)\\
%&\,+\epsilon^2a^{-8-c\sqrt{\epsilon}}{\E^{(\leq L-2)}(\Sigma,\cdot)}+\epsilon^4a^{-8-c\sqrt{\epsilon}}\E^{(\leq L-3)}(\Ric,\cdot)+\epsilon^2a^{-c\sqrt{\epsilon}}\E^{(\leq L-2)}(\Ric,\cdot)
%\end{align*}
%\end{lemma}
%\begin{proof}
%This follows directly from the \todo{rescaled Hamiltonian constraint} along with the \todo{border and junk term estimates}.
%\end{proof}
\begin{lemma}[\change{Odd }order curvature energy estimate]\label{lem:en-est-Ric-top} For $\change{L}\in2\N$, $4\leq \change{L}\leq \change{18}$ and $t\in(t_{Boot},t_0]$,
\begin{align*}
\numberthis\label{eq:en-est-Ric-top}a(t)^4\E^{(\change{L}-1)}(\Ric,t)\lesssim&\,a(t_0)^4\E^{(\change{L}-1)}(\Ric,t_0)\\
&\,+\int_t^{t_0}\left(\epsilon^\frac18a(s)^{-3}+a(s)^{-1-c\sqrt{\epsilon}}\right)\left(a(s)^4\E^{(\change{L}-1)}(\Ric,s)\right)\,ds\\
&\,+\int_t^{t_0}\epsilon^{-\frac18}a(s)^{-3}\cdot a(s)^4\E^{(\change{L}+1)}(\Sigma,s)\,ds\\
&\,+\int_t^{t_0}\Bigr\{\epsilon^{-\frac18}a(s)^{-3}\E^{(\change{L})}(\phi,s)+\left(\epsilon^\frac{15}8a(s)^{-3}+a(s)^{-1-c\sqrt{\epsilon}}\right)\E^{(\change{L})}(\Sigma,s)\\
&\,\phantom{+\int_t^{t_0}}+\left(\epsilon^\frac{15}8a(s)^{-3-c\sqrt{\epsilon}}+a(s)^{-1-c\sqrt{\epsilon}}\right)\left(\E^{(\leq \change{L}-2)}(\phi,s)+\E^{(\leq \change{L}-2)}(\Sigma,s)\right)\\
&\,\phantom{+\int_t^{t_0}}+\epsilon^\frac{15}8a(s)^{-3}\E^{(\change{L}-2)}(\Ric,\cdot)+\epsilon^\frac{15}8a^{-3-c\sqrt{\epsilon}}\E^{(\leq \change{L}-4)}(\Ric,s)\Bigr\}ds
\end{align*}
\end{lemma}
\begin{proof}
The proof is very similar to that of Lemma \ref{lem:en-est-Ric} since we did not exploit any structure within \eqref{eq:comeq-Ric-even} that does not equally occur in \eqref{eq:comeq-Ric-odd}, and thus we omit the details. As in the proof of Lemma \ref{lem:en-est-SF-top}, we note that the differences within the estimate come from how top order lapse terms are treated: The scaling of the top order energy allows one to estimate $a^4\E^{(\change{L}+1)}(N,\cdot)$ by scalar field energies and $\Sigma$-energies of up to order $\change{L}$ and curvature energies up to order $\change{L}-3$.
%\begin{align*}
%-\del_t\left(a^4\E^{(M-1)}(\Ric,\cdot)\right)\leq&\,-a^4\del_t\E^{(M-1)}(\Ric,\cdot)\\
%\lesssim&\,a^{-3}\left[\sqrt{a^4\E^{(M+1)}(\Sigma,\cdot)}+\sqrt{a^4\E^{(M+1)}(N,\cdot)}\right]\sqrt{a^4\E^{(M-1)}(\Ric,\cdot)}\\
%&\,+a^{-1-c\sqrt{\epsilon}}\left[\sqrt{\E^{(\leq M-1)}(\Sigma,\cdot)}+\sqrt{\E^{(\leq M-1)}(N,\cdot)}\right]\sqrt{a^4\E^{(M-1)}(\Ric,\cdot)}\\
%&\,+\left(a^2\|\mathfrak{R}_{M-1,Border}\|_{L^2_G}+a^2\|\mathfrak{R}_{M-1,Junk}\|_{L^2_G}\right)\sqrt{a^4\E^{(M-1)}(\Ric,\cdot)}\\
%&\,+\epsilon a^{-3}\cdot\left(a^4\E^{(M-1)}(\Ric,\cdot)\right)
%\end{align*}
%Note that we can now apply the lapse energy estimate from Corollary \ref{cor:en-est-lapse} at order $M$ to $a^4\E^{(M+1)}(N,\cdot)$ directly. Further, we estimate the lapse term in the second line by $a^{-1-c\sqrt{\epsilon}}\sqrt{\E^{(\leq M-1)}(N,\cdot)}$, where the latter term is then also covered by Corollary \ref{cor:en-est-lapse} at order $M$. Altogether, we obtain
%\begin{align*}
%-\del_t\left(a^4\E^{(M-1)}(\Ric,\cdot)\right)\lesssim&\,\left(\epsilon^\frac18a^{-3}+a^{-1-c\sqrt{\epsilon}}\right)\cdot\left(a^4\E^{(M-1)}(\Ric,\cdot)\right)+\epsilon^{-\frac18}a^{-3}\cdot\left(a^4\E^{(M+1)}(\Sigma,\cdot)\right)\\
%&\,+\epsilon^{\frac{15}8}a^{-3}\E^{(M)}(\Sigma,\cdot)+\epsilon^{-\frac18}a^{-3}\E^{(M)}(\phi,\cdot)\\
%&\,+\epsilon^{\frac{15}8}a^{-3-c\sqrt{\epsilon}}\left[\E^{(\leq M-2)}(\Sigma,\cdot)+\E^{(\leq M-2)}(\phi,\cdot)\right]+\epsilon^{\frac{15}8}a^{-3}\E^{(M-2)}(\Ric,\cdot)\\
%&\,+a^{-1-c\sqrt{\epsilon}}\E^{(\leq M-1)}(\Sigma,\cdot)+a^{-1-c\sqrt{\epsilon}}\E^{(\leq M-2)}(N,\cdot)\\
%&\,+\epsilon^\frac{15}8a^{-3-c\sqrt{\epsilon}}\E^{(\leq M-4)}(\Ric,\cdot)\\
%&\,+\left(a^2\|\mathfrak{R}_{M-1,Border}\|_{L^2_G}+a^2\|\mathfrak{R}_{M-1,Junk}\|_{L^2_G}\right)\sqrt{a^4\E^{(M-1)}(\Ric,\cdot)}\,,\\
%\end{align*}
%where we estimated $a^{-1-c\sqrt{\epsilon}}\E^{(M-1)}(\Sigma,\cdot)$ and similar odd order terms with \eqref{eq:ibp-trick}. Again, the statement follows from inserting the borderline and junk term estimates \eqref{eq:L2-Border-R-odd} and \eqref{eq:L2-junk-R-odd} and applying the lapse energy estimates from Corollary \ref{cor:en-est-lapse}. Note in particular that the scaling by $a^2$ implies all terms arising from $\mathfrak{R}_{M-1,Border}$ and $\mathfrak{R}_{M-1,Junk}$ are dominated by what is already present in the leading order terms.
\end{proof}

\subsection{Sobolev norm estimates for metric objects}\label{subsec:int-metric}

To close the bootstrap argument, we need to improve the behaviour of metric quantities in addition to the energy formalism, both to capture the intrinsic behaviour of the metric and to relate energies to supremum norms.

\begin{lemma}[Sobolev norm estimates for Christoffel symbols]\label{lem:int-est-Chr} Let $U$ be a coordinate neighbourhood on $M$, viewed as a coordinate neighbourhood on $\Sigma_t$ for $t\in(t_{Boot},t_0]$. For any $l\in\N,\,l\leq \changefinal{17}$, the following Sobolev estimate then holds:
\begin{equation}\label{eq:norm-est-Chr}
\|\Gamma-\Gamhat\|^2_{H^l_G(U)}\lesssim a^{-c\epsilon^\frac18}\left(\epsilon^4+\epsilon^{-\frac14}\sup_{s\in(t,t_0)}\left(\|N\|_{H^{l+1}_G(\Sigma_s)}^2+\|\Sigma\|_{H^{l+1}_G(\Sigma_s)}^2\right)\right)
\end{equation}
\end{lemma}
\begin{proof}
Commuting the evolution equation \eqref{eq:REEqChr} with $\nabla^J$, we get for $J\in\N$, $J\leq 17$:
\begin{align*}
-\del_t\|\Gamma-\Gamhat\|_{\dot{H}^J_G}^2=&\,\int_U\Bigr[(\del_tG^{-1})\ast G^{-1}\ast\dots\ast G^{-1}\ast G\ast\nabla^J(\Gamma-\Gamhat)\ast\nabla^J(\Gamma-\Gamhat)\\
&\,\phantom{\int_M}+(G^{-1})\ast\dots\ast(G^{-1})\ast\del_tG\ast\nabla^J(\Gamma-\Gamhat)\ast\nabla^J(\Gamma-\Gamhat)\\
&\,\phantom{\int_M}+\left(a^{-3}\sum_{I_N+I_\Sigma=J+1}\nabla^{I_N}(N+1)\ast\nabla^{I_\Sigma}\Sigma+\frac{\dot{a}}a\nabla^{J+1}N\right)\ast\nabla^J(\Gamma-\Gamhat)\\
&\,\phantom{\int_M}+2\langle[\del_t,\nabla^J](\Gamma-\Gamhat),\nabla^J(\Gamma-\Gamhat)\rangle_G-3N\frac{\dot{a}}a\left\lvert\nabla^J(\Gamma-\Gamhat)\right\rvert_G^2\Bigr]\,\vol{G}
\end{align*}
We recall that \eqref{eq:APmidG} implies
\begin{equation}\label{eq:APChr}
\|\Gamma-\Gamhat\|_{C^{11}_G}\lesssim\sqrt{\epsilon}a^{-c\sqrt{\epsilon}}
\end{equation}
by \eqref{eq:Christoffel-norm-handwaving}. It follows from inserting this in \eqref{eq:commutator-aux-tensor} along with \eqref{eq:APSigma}, \eqref{eq:APmidSigma} and \eqref{eq:BsN} that
\begin{align*}
\|[\del_t,\nabla^J](\Gamma-\Gamhat)\|_{L^2_G}\lesssim&\, \sqrt{\epsilon}a^{-3-c\sqrt{\epsilon}}\|\Sigma\|_{H^J_G}+\sqrt{\epsilon} a^{-3-c\sqrt{\epsilon}}\|N\|_{H^J_G}+\epsilon a^{-3}\|\Gamma-\Gamhat\|_{H^{J-1}_G}
\end{align*}
is satisfied. Consequently and using the same strong $C_G$-norm bounds along with Lemma \ref{lem:lapse-maxmin}, the differential inequality becomes
\begin{align*}
-\del_t\|\Gamma-\Gamhat\|_{\dot{H}^J_G}^2%\lesssim&\,\epsilon a^{-3}\|\nabla^J(\Gamma-\Gamhat)\|_{L^2_G}^2\\
%&\,+a^{-3}\left(\|N\|_{H^{J+1}_G}+\|\Sigma\|_{H^{J+1}_G}\right)\|\nabla^J(\Gamma-\Gamhat)\|_{L^2_G}\\
%&\,+\left(\sqrt{\epsilon} a^{-3-c\sqrt{\epsilon}}\|N\|_{H^{J}_G}+\epsilon a^{1-c\sigma}\|\Sigma\|_{H^{J}_G}\right)\|\nabla^J(\Gamma-\Gamhat)\|_{L^2_G}\\
%&\,+\epsilon a^{-3}\|\Gamma-\Gamhat\|_{H^{J-1}_G}\|\nabla^J(\Gamma-\Gamhat)\|_{L^2_G}\\
\lesssim&\,\epsilon^\frac18 a^{-3}\|\Gamma-\Gamhat\|_{\dot{H}^{J}_G}^2+\epsilon^{-\frac18}a^{-3}\left(\|N\|_{H^{J+1}_G}^2+\|\Sigma\|_{H^{J+1}_G}^2\right)\\
&\,+\epsilon^\frac{7}8a^{-3-c\sqrt{\epsilon}}\|\Sigma\|_{H^{J}_G}^2+\epsilon^\frac78a^{-3-c\sqrt{\epsilon}}\|N\|_{H^J_G}^2\\
&\,+\epsilon^\frac{15}8a^{-3-c\sqrt{\epsilon}}\|\Gamma-\Gamhat\|_{H^{J-1}_G}^2\,.
\end{align*}
Further, we analogously get
\begin{equation*}
-\del_t\|\Gamma-\Gamhat\|_{L^2_G}^2\lesssim\epsilon^\frac18a^{-3}\|\Gamma-\Gamhat\|_{L^2_G}^2+\epsilon^{-\frac18}a^{-3}\left(\|\Sigma\|_{H^1_G}^2+\|N\|_{H^1_G}^2\right)\,
\end{equation*}
and thus, with the Gronwall lemma and \eqref{eq:log-est}, 
\begin{align*}
\|\Gamma-\Gamhat\|_{L^2_G(\Sigma_t)}^2\lesssim&\,a^{-c\epsilon^\frac18}\left(\epsilon^4+\int_t^{t_0}\epsilon^{-\frac18}a(s)^{-3}\left(\|\Sigma\|_{H^1_G(\Sigma_s)}^2+\|N\|^2_{H^1_G(\Sigma_s)}\right)\,ds\right)\\
\lesssim&\,a^{-c\epsilon^{\frac18}}\left(\epsilon^4+\epsilon^{-\frac14}\sup_{s\in(t,t_0)}\left(\|\Sigma\|_{H^1_G(\Sigma_s)}^2+\|N\|^2_{H^1_G(\Sigma_s)}\right)\right)\,.
\end{align*}
This proves \eqref{eq:norm-est-Chr} for $l=0$, and we assume for an iterative argument that the statement has been proved for $l=J-1$. Then, we obtain (estimating the error terms in $\Sigma$ and $N$ by their supremum immediately):
\begin{align*}
-\del_t\|\Gamma-\Gamhat\|_{\dot{H}^J_G}^2\lesssim&\,\epsilon^\frac18a^{-3}\|\Gamma-\Gamhat\|_{\dot{H}^J_G}^2+\epsilon^{-\frac18}a^{-3}\left(\|N\|_{H^J_G}^2+\|\Sigma\|_{H^J_G}^2\right)\\
&\,+\epsilon^{\frac{7}8+4}a^{-3-c\epsilon^\frac18}+\epsilon^\frac{7}8a^{-3-c\epsilon^\frac18}\sup_{s\in(\cdot,t_0)}\left(\|N\|_{H^{J-1}(\Sigma_s)}^2+\|\Sigma\|_{H^{J-1}(\Sigma_s)}^2\right)\,.
\end{align*}
After integrating, applying the Gronwall lemma and dealing with the first line as before, we get
\begin{align*}
\|\Gamma-\Gamhat\|_{\dot{H}^J_G}^2\lesssim&\,\epsilon^4a^{-c\epsilon^\frac18}+\epsilon^{-\frac14}a^{-c\epsilon^\frac18}\sup_{s\in(\cdot,t_0]}\left(\|N\|_{H^J_G(\Sigma_s)}^2+\|\Sigma\|_{H^J_G(\Sigma_s)}^2\right)\\
&\,+\epsilon^{\frac{6}8+4}a^{-c\epsilon^\frac18}+\epsilon^\frac{6}8a^{-c\epsilon^\frac18}\sup_{s\in(\cdot,t_0)}\left(\|N\|_{H^{J-1}(\Sigma_s)}^2+\|\Sigma\|_{H^{J-1}(\Sigma_s)}^2\right)\,,
\end{align*}
where the second line can obviously be absorbed into the first up to constant. Combining this with the assumption yields \eqref{eq:norm-est-Chr} for $l=J$ and thus iteratively for all $l\leq 17$.
\end{proof}

\begin{lemma}[Sobolev norm estimates for the metric]\label{lem:norm-est-G} For any $t\in(t_{Boot},t_0]$ and any $l\in\N,\,l\leq 18$, we have:
\begin{equation}\label{eq:norm-est-G}
\|G-\gamma\|_{H^l_G(\Sigma_t)}^2\lesssim a^{-c\epsilon^\frac18}\left(\epsilon^4+\epsilon^{-\frac14}\sup_{s\in(t,t_0)}\left(\|N\|_{H^{l}_G(\Sigma_s)}^2+\|\Sigma\|_{H^{l}_G(\Sigma_s)}^2\right)\right)
\end{equation}
\end{lemma}
\begin{proof}
For $l=0$, we compute the following using \eqref{eq:REEqG} and \eqref{eq:REEqG-1}:
\begin{align*}
-\del_t\|G-\gamma\|_{L^2_G}^2=&\,\int_M\Bigr\{-2(\del_tG^{-1})^{i_1j_1}(G^{-1})^{i_2j_2}(G-\gamma)_{i_1i_2}(G-\gamma)_{j_1j_2}-2\langle\del_tG,G-\gamma\rangle_G\\
&\,\phantom{\int_M}-3N\frac{\dot{a}}a\lvert G-\gamma\rvert_G^2\,\Bigr\}\vol{G}\\
=&\,\int_M \Bigr\{(N+1)a^{-3} \left[\Sigma\ast(G-\gamma)+\Sigma\right]\ast(G-\gamma)+N\frac{\dot{a}}a\lvert G-\gamma\rvert_G^2\\
&\,\phantom{\int_M}-4N\frac{\dot{a}}a\langle G,G-\gamma\rangle_G\Bigr\}\vol{G}
\end{align*}
We apply \eqref{eq:APSigma} and \eqref{eq:BsN} and get
\begin{align*}
-\del_t\|G-\gamma\|_{L^2_G}^2\lesssim&\,\epsilon a^{-3}\|G-\gamma\|_{L^2_G}^2+a^{-3}\left(\|\Sigma\|_{L^2_G}+\|N\|_{L^2_G}\right)\|G-\gamma\|_{L^2_G}\\
\lesssim&\,\epsilon^\frac18 a^{-3}\|G-\gamma\|_{L^2_G}^2+\epsilon^{-\frac18}a^{-3}\left(\|\Sigma\|_{L^2_G}^2+\|N\|_{L^2_G}^2\right)\,.
\end{align*}
After integrating and applying the Gronwall lemma (as well as the initial data assumption), we obtain
\begin{align*}
\|G-\gamma\|_{L^2_G(\Sigma_t)}^2\lesssim&\, a^{-c\epsilon^\frac18}\left(\epsilon^4+\epsilon^{-\frac18}\int_t^{t_0}a(s)^{-3}\left(\|\Sigma\|_{L^2_G(\Sigma_s)}^2+\|N\|_{L^2_G(\Sigma_s)}^2\right)\,ds\right)\,.
\end{align*}
The statement for $l=0$ now follows taking the supremum over the norms under the integral and applying \eqref{eq:log-est}. This extends to higher orders via the same iteration argument as in Lemma \ref{lem:int-est-Chr}.\\

%\noindent For $l=J>0$, note that we have
%\[\del_t\left(\nabla^J(G-\gamma)\right)=-\del_t\nabla^J\gamma=-[\del_t,\nabla^J]\gamma\,.\]
%Hence, we can insert \eqref{eq:EEq-aux-G} and the commutator term estimates \eqref{eq:commutator-aux-tensor} below, apply the a priori estimates \eqref{eq:APmidSigma} and \eqref{eq:APmidG} along with the bootstrap assumption \eqref{eq:BsN} for the lapse, and get the following:
%\begin{align*}
%-\del_t\|G-\gamma\|_{\dot{H}^J_G}^2\lesssim&\,\int_M\Bigr\{ (\del_tG^{-1})\ast G^{-1}\ast\dots\ast G^{-1}\ast\nabla^J\gamma\ast\nabla^J\gamma+2\langle[\del_t,\nabla^J]\gamma,\nabla^J(G-\gamma)\rangle_G\\
%&\,\phantom{\int_M}+3N\frac{\dot{a}}a\lvert\nabla^J(G-\gamma)\rvert_G^2\Bigr\}\vol{G}\\
%\lesssim&\,\int_M \epsilon a^{-3}\lvert G-\gamma\rvert_G^2+\frac{\dot{a}}a\sum_{I_N+I_\gamma=J,\,I_N>0}\lvert\nabla^{I_N}N\rvert_G\lvert\nabla^{I_\gamma}(G-\gamma)\rvert_G\lvert\nabla^J(G-\gamma)\rvert_G\,\\
%&\,+a^{-3}\sum_{I_N+I_\Sigma+I_G=J,\,I_\gamma<J}\lvert\nabla^{I_N}(N+1)\rvert_G\lvert\nabla^{I_\Sigma}\Sigma\rvert_G\lvert\nabla^{I_\gamma}\gamma\rvert_G\lvert\nabla^J(G-\gamma)\rvert_G\vol{G}\\
%\lesssim&\,\epsilon^\frac18 a^{-3}\|G-\gamma\|_{{H}^J_G}^2+\epsilon^{-\frac18} a^{-3-c\sqrt{\epsilon}}\left(\|\Sigma\|_{H^{J}_G}^2+\|N\|_{H^{J}_G}^2\right)+\epsilon^\frac{15}8a^{-3-c\sqrt{\epsilon}}\|G-\gamma\|_{H^{J-1}_G}
%\end{align*}
%After plugging all Sobolev semi-norms up to order $l$ together, integrating and applying the Gronwall lemma, we get
%\begin{align*}
%\|G-\gamma\|_{{H}^J_G(\Sigma_t)}^2\lesssim&\,a^{-c\epsilon^\frac18}\left(\epsilon^4+\epsilon^{-\frac18}\int_t^{t_0}a(s)^{-3-c\epsilon^\frac18}\left(\|\Sigma\|_{H^{J}_G(\Sigma_s)}^2+\|N\|_{H^{J}_G(\Sigma_s)}^2\right)\,ds\right)\\
%&\,+\epsilon^\frac{15}8a^{-3-c\sqrt{\epsilon}}\|G-\gamma\|_{H^{J-1}_G}^2
%\end{align*}
%The statement now follows as in the proof of Lemma \ref{lem:int-est-Chr}.
\end{proof}

\section{Big Bang stability: Improving the bootstrap assumptions}\label{sec:bs-imp}

In this section, we combine the energy estimates obtained in the last two sections to improve the boostrap assumptions for the energies themselves, and then show how this improves the behaviour of the solution norms. For an outline of the energy improvement arguments that we perform in \change{Section \ref{subsec:bs-imp-core}}, we refer back to Remark \ref{rem:en-est-strat}.

Before carrying out the improvements themselves, we quickly collect an estimate that shows that combining Lemmas \ref{lem:en-est-BR} and \ref{lem:en-est-Sigma} yields sufficient control on the energies themselves:

\begin{lemma}\label{lem:en-error-cancellation} Let $L\in 2\N$. Then, the following estimate is satisfied:
\begin{equation}
16\pi C^2a^{-3}(N+1)\langle\Lap^\frac{L}2\RE,\Lap^{\frac{L}2}\Sigma\rangle_G+8\pi C^2\frac{\dot{a}}a(N+1)\lvert\Lap^\frac{L}2\Sigma\rvert_G^2+6\frac{\dot{a}}a(N+1)\lvert\Lap^\frac{L}2\RE\rvert_G^2\geq 0
\end{equation}
\end{lemma}
\begin{proof}
First, we recall that $N+1>0$ holds by Lemma \ref{lem:lapse-maxmin}. Additionally, we can apply \eqref{eq:diff-ineq-Friedman} and the Young inequality and get
\begin{align*}
\left\lvert 16\pi C^2a^{-3}(N+1)\langle\Lap^\frac{L}2\RE,\Lap^{\frac{L}2}\Sigma\rangle_G\right\rvert\leq&\,16\pi C^2\cdot\sqrt{\frac3{4\pi C^2}}\frac{\dot{a}}a\cdot (N+1)\lvert\Lap^{\frac{L}2}\RE\rvert_G\lvert\Lap^{\frac{L}2}\Sigma\rvert_G\\
\leq&\,4(N+1)\frac{\dot{a}}a \cdot\left(\sqrt{3}\cdot\lvert\Lap^\frac{L}2\RE\rvert_G\right)\cdot\left(\sqrt{4\pi C^2}\lvert\Lap^{\frac{L}2}\Sigma\rvert_G\right)\\
\leq%&\, 4(N+1)\frac{\dot{a}}a\cdot\left[\frac32\lvert\Lap^\frac{L}2\RE\rvert_G^2+\frac{4\pi C^2}2\lvert\Lap^\frac{L}2\Sigma\rvert_G^2\right]\\
&\,6\frac{\dot{a}}a(N+1)\lvert\Lap^\frac{L}2\RE\rvert_G^2+8\pi C^2\frac{\dot{a}}a(N+1)\lvert\Lap^\frac{L}2\Sigma\rvert_G^2\,.
\end{align*}
\end{proof}
%\begin{remark}\label{rem:en-error-cancellation}
This shows that $\E^{(L)}(W,\cdot)+4\pi C^2\E^{(L)}(\Sigma,\cdot)$ is controlled by the sum of the left hand sides of the inequalities in Lemmas \ref{lem:en-est-BR} and \ref{lem:en-est-Sigma} for $L\in 2\N, 0\leq L\leq \change{18}$. %While this specific form relies on the inequality \eqref{eq:diff-ineq-Friedman}, using \eqref{eq:diff-ineq-Friedman-weak} instead would only add energy terms of the form $a^{-1}\left(\E^{(L)}(W,\cdot)+\E^{(L)}(\Sigma,\cdot)\right)$ on the right hand side, which can easily be dealt with in the Gronwall argument. Thus, while the Lemma itself relies on the sectional curvature being nonpositive, the Gronwall argument can easily be adapted to $\kappa=1$ as well.
%\end{remark}


\subsection{\change{Improving energy bounds}}\label{subsec:bs-imp-core}


\change{\begin{prop}[Improved energy bounds]\label{prop:en-bs-imp} Under the bootstrap assumptions (see Assumption \ref{ass:bootstrap}) and the initial data assumptions in Assumption \eqref{ass:init}, the following improved estimates hold on $(t_{Boot},t_0]$:
\begin{subequations}
\begin{align}
\E^{(\leq \change{18})}(\phi,\cdot)\lesssim&\,\epsilon^4a^{-c\epsilon^\frac18}\label{eq:en-bs-imp-SF}\\
\E^{(\leq \change{18})}(\Sigma,\cdot)\lesssim&\,\epsilon^\frac{15}4a^{-c\epsilon^\frac18}\label{eq:en-bs-imp-Sigma}\\
\E^{(\leq \change{18})}(W,\cdot)\lesssim&\,\epsilon^\frac{15}4a^{-c\epsilon^\frac18}\label{eq:en-bs-imp-BR}\\
\E^{(\leq 16)}(\Ric,\cdot)\lesssim&\,\epsilon^\frac72a^{-c\epsilon^\frac18}\label{eq:en-bs-imp-Ric}\\
%\sum_{m=1}^{10}\|\Lap^m\phi\|^2_{L^2_G}+\|\nabla\phi\|_{H^{18}_G}^2\lesssim&\,\epsilon^\frac52 a^{-c\epsilon^\frac18}\label{eq:en-bs-imp-phi}\\
\E^{(\leq 16)}(N,\cdot)+a^4\E^{(17)}(N,\cdot)+a^8\E^{(18)}(N,\cdot)\lesssim&\,\epsilon^\frac72a^{8-c\epsilon^\frac18}\label{eq:en-bs-imp-N}
\end{align}
\end{subequations}
\end{prop}}
\change{
\begin{proof}
We prove this estimate by performing an induction over even energy orders. Starting at order $0$, we first observe that by Lemma \ref{lem:en-error-cancellation}, we can bound the (base level) total energy
\[\E^{(0)}_{total}:=\E^{(0)}(\phi,\cdot)+\epsilon^\frac14\left(\E^{(0)}(W,\cdot)+4\pi C^2\E^{(0)}(\Sigma,\cdot)\right)+a^4\E^{(1)}(\phi,\cdot)+\epsilon^\frac12\E^{(1)}(\Sigma,\cdot)\]
by the sum of the left hand side of \eqref{eq:en-est-SF0}, the left hand side of \eqref{eq:en-est-BR0} weighted by $\epsilon^\frac14$ and the left hand side of \eqref{eq:en-est-Sigma0} weighted by $4\pi C^2\cdot\epsilon^\frac14$, and the left hand sides of \eqref{eq:en-est-SF-top} and $\epsilon^\frac12\cdot$\eqref{eq:en-est-Sigma-top} extended to $L=0$.\footnote{We need to weight $\E^{(0)}(\Sigma,\cdot)$ in the total energy by $K\cdot\epsilon^\frac14$ for some $K>0$ to balance out the $\epsilon^{-\frac18}$-weight from the scalar field energy on the right hand side of \eqref{eq:en-est-Sigma0}. The weight on the Bel-Robinson energy is then needed to obtain the cancellation in Lemma \ref{lem:en-error-cancellation}\change{. The additional weight on $a^4\E^{(1)}(\Sigma,\cdot)$ is needed so that the div-curl-estimates only generates a term of size $\epsilon^\frac14\E^{(L)}_{total}$ that can be absorbed later in the argument.}} Combining said estimates and inserting the initial data assumption from \eqref{eq:init-ass-en}, the following holds in total:
\begin{equation}\label{eq:en-eq-total0}
\E^{(0)}_{total}(t)\lesssim\epsilon^4+\int_t^{t_0}\left(\epsilon^\frac18a(s)^{-3}+a(s)^{-1-c\sqrt{\epsilon}}\right)\E^{(0)}_{total}(s)\,ds
\end{equation}
Applying the Gronwall lemma (see Lemma \ref{lem:gronwall}) to \eqref{eq:en-eq-total0}, we get for some suitable constant $c^\prime>0$:
\begin{equation*}\label{eq:en-bs-imp-base}
\E^{(0)}_{total}(t)\lesssim \epsilon^4\exp\left(c^\prime\int_t^{t_0}\epsilon^\frac18 a(s)^{-3}+a(s)^{-1-c\sqrt{\epsilon}}\,ds\right)\lesssim \epsilon^4a^{-c^\prime\epsilon^\frac18}
\end{equation*}
%In other words, there exist some $c_1,K_1>0$ such that, for any $t\in(t_{Boot},t_0]$, one has:
%\begin{subequations}
%\begin{align}
%\label{eq:en-bs-imp-base-SF}\E^{(0)}(\phi,t)\leq&\,K_1\epsilon^\frac{11}4a(t)^{-c_1\epsilon^\frac18}\\
%\E^{(0)}(\Sigma,t)\leq&\,K_1\epsilon^\frac{5}2a(t)^{-c_1\epsilon^\frac18}\\
%\label{eq:en-bs-imp-base-BR}\E^{(0)}(W,t)\leq&\,K_1\epsilon^\frac{5}2a(t)^{-c_1\epsilon^\frac18}
%\end{align}
%\end{subequations}
Now assume that, for $L\in2\N, 2\leq L\leq 18$, we have already shown
\begin{subequations}
\begin{align}
\E^{(\leq L-2)}(\phi,\cdot)+\epsilon^\frac14\left(\E^{(\leq L-2)}(\Sigma,\cdot)+\E^{(\leq L-2)}(W,\cdot)\right)\lesssim&\,\epsilon^4 a^{-c\epsilon^\frac18}\label{eq:ind-ass-core-en}
\end{align}
on $(t_{Boot},t_0]$. Note that \eqref{eq:en-bs-imp-base} means this holds true for $L=2$ after updating $c>0$. Further, if $L\geq 4$ holds, we assume
\begin{align}
%\sum_{m=1}^{\frac{L}2-1}\|\Lap^m\phi\|^2_{L^2_G}+\|\nabla\phi\|_{H^{L-3}_G}^2\lesssim&\,\epsilon^\frac52 a^{-c\epsilon^\frac18}\label{eq:ind-ass-phi}\\
\E^{(\leq L-4)}(\Ric,\cdot)\lesssim&\,\epsilon^{\frac72}a^{-c\epsilon^\frac18}\,.\label{eq:ind-ass-Ric}
\end{align}
\end{subequations}
We will show that these assumptions hold at $L=4$ after having shown the induction step for $L=2$.
We define, for $2\leq L\leq 18$,
\begin{align*}
\E^{(L)}_{total}:=&\,\E^{(L)}(\phi,\cdot)+\epsilon^\frac14\left(\E^{(L)}(W,\cdot)+4\pi C^2\E^{(L)}(\Sigma,\cdot)\right)+a^4\E^{(L+1)}(\phi,\cdot)+\epsilon^\frac12a^4\E^{(\changefinal{L+1})}(\Sigma,\cdot)\\
&\,+\epsilon^\frac12\E^{(L-2)}(\Ric,\cdot)+\epsilon^\frac34a^4\E^{(L-1)}(\Ric,\cdot)\,.
\end{align*}
We combine the respective energy estimates with the appropriate scalings\footnote{The weights on all terms beside the curvature energies are necessary for the same reasons as at order $0$. We need to scale the curvature energy at order $L$ by $\epsilon^\frac12$ to account for $\epsilon^{-\frac18}a^{-3}\E^{(L)}(\Sigma,\cdot)$ in \eqref{eq:en-est-Ric}, and the weight on the $L+1$-order curvature energy again needs to be chosen according to that on $\E^{(L+1)}(\Sigma,\cdot)$.}, namely (in the listed order) \eqref{eq:en-est-SF}, \eqref{eq:en-est-BR}, \eqref{eq:en-est-Sigma}, \eqref{eq:en-est-SF-top}, \eqref{eq:en-est-Sigma-top}, \eqref{eq:en-est-Ric} and \eqref{eq:en-est-Ric-top}. Observe that the sum of these scaled left hand sides controls $\E_{total}^{(L)}$ by Lemma \ref{lem:en-error-cancellation}. Combining all of these estimates and inserting the initial data assumption \eqref{eq:init-ass-en}, we get the following estimate:

\begin{subequations}
\begin{align}
\E_{total}^{(L)}(t)\lesssim&\,\epsilon^4+\int_t^{t_0}\left(\epsilon^\frac18a(s)^{-3}+a(s)^{-1-c\sqrt{\epsilon}}\right)\E_{total}^{(L)}(s)\label{eq:en-est-top-total1}\,ds\\
&\,+\int_t^{t_0}\left\{\epsilon^\frac38a(s)^{-3-c\sqrt{\epsilon}}\left[\E^{(\leq L-2)}(\phi,s)+\E^{(\leq L-2)}(\Sigma,s)\right]\right.\label{eq:en-est-top-total2}\\
%&\,\phantom{+\int_t^{t_0}\bigr\{}+\epsilon^2a^{-1-c\sqrt{\epsilon}}\left(\|\nabla^2\Lap^{L-2}\phi\|_{L^2_G}^2+\|\nabla\phi\|_{H^{L-2}_G(\Sigma_s)}^2\right)+\epsilon a^{3-c\sqrt{\epsilon}}\|\nabla\phi\|_{H^{L}_G(\Sigma_s)}^2\label{eq:en-est-top-total3}\\
&\,\phantom{+\int_t^{t_0}\bigr\{}%+\langle\text{Err}\rangle_L(s) \bigr\}\,ds
+\epsilon^\frac{17}8a(s)^{-3-c\sqrt{\epsilon}}\E^{(\leq L-2)}(W,s)+\underbrace{\epsilon^\frac{11}8\E^{(\leq L-4)}(\Ric,s)}_{\text{if }L=4}\bigr\}\,ds
\label{eq:en-est-top-total4}\\
&\,+\epsilon^\frac14\left(a(t)^{4-c\sqrt{\epsilon}}+\epsilon a(t)^{2-c\sqrt{\epsilon}}\right)\cdot \epsilon^\frac12\E^{(L)}(\Sigma,t)+\epsilon^\frac12\E^{(L)}(\phi,t)+\epsilon^\frac14\cdot \epsilon^\frac14\E^{(L)}(W,t)\label{eq:en-est-top-total5}\\
&\,+\epsilon^\frac74\cdot \epsilon^\frac34a^4\E^{(L-1)}(\Ric,t)+\epsilon^\frac52 a^{-c\sqrt{\epsilon}}\E^{(\leq L-2)}(\phi,t)+\epsilon^\frac12 a^{2-c\sqrt{\epsilon}}\E^{(\leq L-2)}(\Sigma,t)\label{eq:en-est-top-total6}\\
&\,+\epsilon^\frac32 a^{2-c\sqrt{\epsilon}}\E^{(\leq L-2)}(\Ric,t)\label{eq:en-est-top-total7}
\end{align}
\end{subequations}
%with
%\begin{equation*}
%\langle\text{Err}\rangle_L=\begin{cases}
%\displaystyle\int_t^{t_0}\left(\epsilon^\frac98a(s)^{-3-c\sqrt{\epsilon}}+\epsilon a^{-1-c\sqrt{\epsilon}}\right)\|\nabla\phi\|_{L^2_G(\Sigma_s)}^2\,ds\\[0.5em]
%\displaystyle\int_t^{t_0}\epsilon^\frac98a(s)^{-3-c\sqrt{\epsilon}}\|\nabla\phi\|_{H^{L-3}_G(\Sigma_s)}^2+\epsilon^\frac{11}8\E^{(\leq L-4)}(\Ric,s)\,ds& L\geq 4
%\end{cases}
%\end{equation*}


We briefly summarize which inequalities contain the listed error term bounds as explicit terms: The first two terms in \eqref{eq:en-est-top-total2} come from \eqref{eq:en-est-Ric} and the latter from \eqref{eq:en-est-BR}%, those in \eqref{eq:en-est-top-total3} from \eqref{eq:en-est-SF-top}
, those in \eqref{eq:en-est-top-total4} from %\eqref{eq:int-est-Lapphi},
\eqref{eq:en-est-SF} and \eqref{eq:en-est-Ric}, and finally the last three lines are precisely the scaled right hand side of \eqref{eq:en-est-Sigma-top}. Regarding the curvature energies in the various individual energy estimates, any summand with $\E^{(L-3)}(\Ric,\cdot)$ can be split using \eqref{eq:ibp-trick}, the resulting summands containing $\E^{(L-2)}(\Ric,\cdot)$ can always be absorbed into the total energy term in the first line, and anything with $\E^{(\leq L-4)}(\Ric,\cdot)$ is tracked in $\langle\text{Err}\rangle_L$ for $L\geq 4$.\\
%In short, what now needs to be done is to reduce \eqref{eq:en-est-top-total2}-\eqref{eq:en-est-top-total4} to quantities we can bound with the induction assumptions \eqref{eq:ind-ass-core-en}-\eqref{eq:ind-ass-Ric} (up to absorbing high order terms into \eqref{eq:en-est-top-total1}), while \eqref{eq:en-est-top-total5}-\eqref{eq:en-est-top-total7} must be shown to carry strong enough $\epsilon$-weights that they only contribute negligible error terms.

\noindent Inserting \eqref{eq:ind-ass-core-en}-\eqref{eq:ind-ass-Ric}, \eqref{eq:en-est-top-total2}-\eqref{eq:en-est-top-total4} can be bounded up to constant by $\epsilon^\frac{33}8a^{-3-c\epsilon^\frac18}$. Here, the error term dominating all others arises from 
\[\epsilon^\frac38a(s)^{-3-c\sqrt{\epsilon}}\E^{(\leq L-2)}(\Sigma,s)\,.\]


%For the terms in \eqref{eq:en-est-top-total3}-\eqref{eq:en-est-top-total4}, we consider $L=2$ and $L\geq 4$ separately:\\
%${\bm {L=2}}$: For $L=2$, we get with Corollary \ref{cor:int-est-nablaphi0-botch} and updating $c>0$ that
%\begin{align*}
%\langle\text{Err}\rangle_2\lesssim&\,\int_t^{t_0}\epsilon^3a(s)^{-3-c\epsilon^\frac18}\,+\epsilon^\frac{23}8a(s)^{-1-c\epsilon^\frac18}ds\\
%&\,+\int_t^{t_0}\left(\epsilon^\frac78a(s)^{-3-c\epsilon^\frac18}+\epsilon^\frac34a(s)^{-1-c\epsilon^\frac18}\right)\sup_{r\in(s,t_0]}\E^{(1)}(\phi,r)\,ds\,.
%\end{align*}
%The first line is negligible compared to terms already present due to our bound on \eqref{eq:en-est-top-total2}. The first term in the second line can be estimated using integration by parts and applying \eqref{eq:en-bs-imp-base} as follows:
%\begin{align*}
%&\,\int_t^{t_0}a(s)^{-3-c\epsilon^\frac18}\cdot\sup_{r\in(s,t_0]}\epsilon^\frac78\E^{(1)}(\phi,r)\,ds\\
%\lesssim&\,\int_t^{t_0}a(s)^{-3-c\epsilon^\frac18}\left(\sup_{r\in(s,t_0]}\sqrt{\epsilon^\frac18\E^{(2)}(\phi,r)}\right)\cdot\left(\sup_{r\in(s,t_0]}\sqrt{\epsilon^\frac{13}8\E^{(0)}(\phi,r)}\right)\,ds\\
%\lesssim&\,\int_t^{t_0} \left\{\epsilon^\frac18 a(s)^{-3}\sup_{r\in(s,t_0]}\E^{(2)}_{total}(r)+\epsilon^\frac{13}8a(s)^{-3-c\epsilon^\frac18}\cdot \epsilon^\frac{11}4a(s)^{-c\epsilon^\frac18}\right\}\,ds
%\end{align*}
%The second term on the right hand side is also negligible compared to already present. Analogously, we obtain for the remaining line in the estimate of $\langle\text{Err}\rangle_2$:
%\begin{align*}
%&\,\int_t^{t_0}\epsilon^\frac34a(s)^{-1-c\epsilon^\frac18}\sup_{r\in(s,t_0]}\E^{(1)}(\phi,r)\,ds\\
%\lesssim&\,\int_t^{t_0}\left\{a(s)^{-1-c\epsilon^\frac18}\sup_{r\in(s,t_0]}\E_{total}^{(2)}(r)+\epsilon^\frac32a(s)^{-3-c\epsilon^\frac18}\cdot \epsilon^\frac{11}4a(s)^{-c\epsilon^\frac18}\right\}\,ds
%\end{align*}
%The second term is again dominated by the bound on \eqref{eq:en-est-top-total2}.\\ 
%
%\noindent${\bm{ L\geq 4}}$: First, by \eqref{eq:Sobolev-norm-equiv-zetalow}, one can replace $\|\nabla^2\Lap^9\phi\|_{L^2_G}^2$ by $\|\Lap^{10}\phi\|_{L^2_G}^2+\|\Lap^9\phi\|_{L^2_G}^2$ up to updating constants. The former can now be estimated by $a(s)^{-1-c\sqrt{\epsilon}}\E_{top}(s)$ and thus absorbed into \eqref{eq:en-est-top-total1}. For the remaining two terms, we can use \eqref{eq:Sobolev-norm-equiv-zeta2l+1} with $l=9$ and $l=10$ and integration by parts to estimate them as follows:
%\begin{align*}
%&\,\epsilon^2a^{-1-c\sqrt{\epsilon}}\|\nabla\phi\|_{H^{L-2}_G(\Sigma_s)}^2+\epsilon a^{3-c\sqrt{\epsilon}}\|\nabla\phi\|_{H^{L}_G(\Sigma_s)}^2\\
%\lesssim&\,a^{-1-c\sqrt{\epsilon}}\left(\epsilon^2\|\nabla\Lap^{\frac{L}2-1}\phi\|_{L^2_G}^2+\epsilon a^{4}\|\nabla\Lap^{\frac{L}2}\phi\|_{L^2_G}^2+\epsilon a^{-1-c\sqrt{\epsilon}}\|\nabla\phi\|_{H^{L-4}_G}^2\right)+\epsilon^2 a^{-1-c\sqrt{\epsilon}}\E^{(\leq 18)}(\Ric,\cdot)\\
%\lesssim&\,\left[\epsilon^2a^{-1-c\sqrt{\epsilon}}\|\Lap^{\frac{L}2}\phi\|_{L^2_G}^2+\epsilon a^{-1-c\sqrt{\epsilon}}\E^{(L)}(\phi,\cdot)+\epsilon^2a^{-1-c\sqrt{\epsilon}}\E^{(L-2)}(\Ric,\cdot)\right]\\
%&\,+\epsilon a^{-1-c\sqrt{\epsilon}}\|\nabla\phi\|_{H^{L-4}_G}^2+\epsilon^2a^{-1-c\sqrt{\epsilon}}\E^{(\leq L-4)}(\Ric,\cdot)\\
%\lesssim&\,a^{-1-c\sqrt{\epsilon}}\E_{top}+\epsilon a^{-1-c\sqrt{\epsilon}}\|\nabla\phi\|_{H^{L-4}_G}^2+\epsilon^2a^{-1-c\sqrt{\epsilon}}\E^{(\leq 16)}(\Ric,\cdot)
%\end{align*}
%Now, inserting \eqref{eq:ind-ass-phi} and \eqref{eq:ind-ass-Ric} to bound $\langle\text{Err}\rangle_L$, one can bound \eqref{eq:en-est-top-total3}-\eqref{eq:en-est-top-total4} by
%\[\lesssim\int_t^{t_0}a(s)^{-1-c\sqrt{\epsilon}}sup_{r\in(s,t_0]}\E_{total}^{(L)}(r)+\epsilon^{\frac{7}2}a(s)^{-3-c\epsilon^\frac18}\,ds\,.\]
%In both cases, we thus obtain that \eqref{eq:en-est-top-total1}-\eqref{eq:en-est-top-total5} are bounded by
%Hence, one has
%\[\lesssim\int_t^{t_0}\left[\epsilon^\frac18a(s)^{-3}+a(s)^{-1-c\sqrt{\epsilon}}\right]\E_{total}^{(L)}(s)+\epsilon^{\frac{35}8}a(s)^{-3-c\epsilon^\frac18}\,ds\,.\]
%Here, the second summand dominating all other error terms arises from 
%\[\epsilon^\frac38a(s)^{-3-c\sqrt{\epsilon}}\E^{(\leq 18)}(\Sigma,s)\,.\]

Regarding \eqref{eq:en-est-top-total5}-\eqref{eq:en-est-top-total7}, notice that the first four summands can be bounded from above by $\epsilon^\frac14\E_{total}^{(L)}(t)$ up to constant. For the remaining three terms, we can again insert the induction assumptions \eqref{eq:ind-ass-core-en}-\eqref{eq:ind-ass-Ric}, bounding them by $\epsilon^\frac{17}4a(t)^{-c\epsilon^\frac18}$. \\

\noindent In summary and after rearranging, \eqref{eq:en-est-top-total1}-\eqref{eq:en-est-top-total7} becomes for some constant $K>0$:
\begin{align*}
(1-K\epsilon^\frac14)\E_{total}^{(L)}(t)\lesssim&\,\epsilon^4+\int_t^{t_0}\left(\epsilon^\frac18a(s)^{-3}+a(s)^{-1-c\sqrt{\epsilon}}\right)\E_{total}^{(L)}(s)\,ds+\int_t^{t_0}\epsilon^\frac{33}8a(s)^{-3-c\epsilon^\frac18}\,ds\\
&\,+\epsilon^\frac{17}4a(t)^{-c\epsilon^\frac18}\\
\lesssim&\,\epsilon^4a(t)^{-c\epsilon^\frac18}+\int_t^{t_0}\left(\epsilon^\frac18a(s)^{-3}+a(s)^{-1-c\sqrt{\epsilon}}\right)\E_{total}^{(L)}(s)\,ds\,
\end{align*} 
The prefactor on the left hand side is positive for small enough $\epsilon>0$, and the Gronwall lemma then yields
\begin{equation}\label{eq:total-en-imp}
\E^{(L)}_{total}(t)\lesssim \epsilon^4a^{-c\epsilon^\frac18}\,.
\end{equation}
In particular, this directly implies that the induction assumptions \eqref{eq:ind-ass-core-en} and \eqref{eq:ind-ass-Ric}, using \eqref{eq:drop-odd-en} to cover the skipped odd order, hold at order $L$, completing the induction step, and clearly also that \eqref{eq:ind-ass-Ric} holds for $L-2=2$ using \eqref{eq:total-en-imp} at order $2$. This completes the induction argument, proving \eqref{eq:en-bs-imp-SF}-\eqref{eq:en-bs-imp-Ric}.
 %, which only leaves \eqref{eq:ind-ass-phi} to be extended to order $L$.\\
%Note that the improved total energy bound also implies
%\[\|\nabla\Lap^9\phi\|^2_{L^2_G}\lesssim\left(\|\Lap^9\phi\|_{L^2_G}^2+\|\Lap^{10}\phi\|_{L^2_G}^2\right)\lesssim \epsilon^\frac{5}2a^{-c\epsilon^\frac18}\,.\]
%Hence, Lemma \ref{lem:Sobolev-norm-equivalence-improved} implies that \eqref{eq:ind-ass-phi} also extends.
%First, we show that \eqref{eq:ind-ass-phi}-\eqref{eq:ind-ass-Ric} hold at $L=4$. Note that only the bound on $\|\nabla\phi\|_{H^1_G}$ does not follow directly from \eqref{eq:total-en-imp} at order $L=4$. Since, by \eqref{eq:Sobolev-norm-equiv-zetalow},
%\[\|\nabla^2\phi\|_{L^2_G}\lesssim \|\Lap\phi\|_{L^2_G}+a^{-c\sqrt\epsilon}\|\nabla\phi\|_{L^2_G}\]
%holds, we only even need to show $\|\nabla\phi\|_{L^2_G}\lesssim\epsilon^\frac52 a^{-c\epsilon^\frac18}$.\\
%Note that, since
%\[\E^{(\leq 2)}(\phi,\cdot)+\epsilon^\frac14\E^{(\leq 2)}(\Sigma,\cdot)\lesssim \E^{(0)}_{total}+\E^{(2)}_{total},\] \eqref{eq:en-bs-imp-base}, \eqref{eq:total-en-imp} and Corollary \ref{cor:en-est-lapse} imply
%\[\E^{(\leq 2)}(\phi,\cdot)+\E^{(\leq 2)}(N,\cdot)\lesssim \epsilon^\frac{11}4a^{-c\epsilon^\frac18}\,.\]
%Applying this to \eqref{eq:int-est-nablaphi0}, we obtain
%\begin{equation}\label{eq:nabla-phi-imp0}
%\|\nabla\phi\|_{L^2_G(\Sigma_t)}^2\lesssim a^{-c\epsilon^\frac18}\left(\epsilon^4+\int_t^{t_0}\epsilon^\frac{21}8a(s)^{-3-c\epsilon^\frac18}\,ds\right)\lesssim \epsilon^\frac{5}2a^{-3-c\epsilon^\frac18}\,.
%\end{equation}
%This proves \eqref{eq:en-bs-imp-phi} for $L=4$.\\
%We return now to the induction argument at order $L$ and show that \eqref{eq:total-en-imp} extends \eqref{eq:ind-ass-phi}-\eqref{eq:ind-ass-Ric} to order $L$, where again only the bound on $\|\nabla\phi\|_{H^{L-3}_G}$ still needs to be obtained. To this end, we can apply \eqref{eq:Sobolev-norm-equiv-nablazeta2l+1} and \eqref{eq:Sobolev-norm-equiv-zetalow} with $\zeta=\phi$ as well as \eqref{eq:Sobolev-norm-equiv-zeta2l} for $\zeta=\Lap^m\phi,\,m\geq 1$:
%\begin{align*}
%\|\nabla\phi\|_{H^{L-1}_G}^2%=&\,\|\nabla\phi\|_{L^2_G}^2+\sum_{m=1}^{L-1}\|\nabla\phi\|^2_{\dot{H}^{m}_G}\\
%\lesssim&\, a^{-c\sqrt{\epsilon}}\left[\|\nabla\phi\|_{L^2_G}^2+\sum_{m=0}^{\frac{L}2-1}\|\nabla^2\Lap^m\phi\|^2_{L^2_G}+\epsilon \E^{(\leq L-4)}(\Ric,\cdot)\right]\\
%\lesssim&\,a^{-c\epsilon^\frac18}\left(\|\nabla\phi\|_{L^2_G}^2+\sum_{m=1}^\frac{L}2\|\Lap^m\phi\|_{L^2_G}^2\right)+\epsilon a^{-c\epsilon^\frac18} \E^{(\leq L-4)}(\Ric,\cdot)
%\end{align*}
%Inserting \eqref{eq:total-en-imp} yields
%\begin{align*}
%\|\nabla\phi\|_{H^{L-1}_G}^2\lesssim&\,a^{-c\epsilon^\frac18}\|\nabla\phi\|_{L^2_G}^2+\epsilon^\frac52a^{-c\epsilon^\frac18}\lesssim \epsilon^\frac52 a^{-c\epsilon^\frac18}\,,
%\end{align*}
%where we take the bound on $\|\nabla\phi\|_{L^2_G}^2$ from \eqref{eq:nabla-phi-imp0}. This completes the induction step, and thus proves \eqref{eq:en-bs-imp-SF}-\eqref{eq:en-bs-imp-phi}.
Finally, applying the obtained improved estimates for $\nabla\phi$ and $\Ric[G]$ to Corollary \ref{cor:en-est-lapse-tilde}, we also get \eqref{eq:en-bs-imp-N}. 
\end{proof}}

%Due to Lemma \ref{lem:en-error-cancellation}, this is controlled by the respective weighted left hand sides of \eqref{eq:en-est-SF}, \eqref{eq:en-est-BR}, \eqref{eq:en-est-Sigma}, \eqref{eq:en-est-Ric} and \eqref{eq:int-est-Lapphi}. As a result and again applying the initial data assumption \eqref{eq:init-ass-en}, we have the following:
%\begin{align*}
%\E^{(L)}_{total}(t)\lesssim&\,\epsilon^4+\int_t^{t_0}\left(\epsilon^\frac18 a(s)^{-3}+a(s)^{-1-c\sqrt{\epsilon}}\right)\E^{(L)}_{total}(s)\,ds\\
%&\,+\int_t^{t_0}\epsilon^\frac14 a(s)^{-1}\E^{(L+1)}(\phi,s)+\epsilon^\frac54 a(s)^{-1}\E^{(L-1)}(\Ric,s)\,ds\\
%&\,+\int_t^{t_0}\left\{a(s)^{-3-c\sqrt{\epsilon}}\left(\epsilon^\frac38 \E^{(\leq L-2)}(\phi,s) +\epsilon^\frac54\E^{(\leq L-2)}(W,s)+\epsilon^\frac38 \E^{(\leq L-2)}(\Sigma,s)\right)\right.\\
%&\,\phantom{+\int_t^{t_0}}+\left.\epsilon^\frac14a(s)^{-1-c\sqrt{\epsilon}}\E^{(\leq L-2)}(\phi,s)\right\}\,ds\\
%&\,+\langle\text{Err}\rangle_L
%\end{align*}
%with
%\begin{equation*}
%\langle\text{Err}\rangle_2=\int_t^{t_0}\left(\epsilon^\frac98a(s)^{-3-c\sqrt{\epsilon}}+\epsilon a^{-1-c\sqrt{\epsilon}}\right)\|\nabla\phi\|_{L^2_G(\Sigma_s)}^2\,ds
%\end{equation*}
%and, for $L\geq 4$,
%\begin{equation*}
%\langle\text{Err}\rangle_L=\int_t^{t_0}\left\{\left(\epsilon^\frac98a(s)^{-3-c\sqrt{\epsilon}}+\epsilon a(s)^{-1-c\sqrt{\epsilon}}\right)\|\nabla\phi\|_{H^{L-3}_G(\Sigma_s)}^2+\epsilon^\frac{11}8a(s)^{-3-c\sqrt{\epsilon}}\E^{(\leq L-4)}(\Ric,s)\right\}\,ds\,.
%\end{equation*}
%The error term for $L=2$ arises from \eqref{eq:int-est-Lapphi} and \eqref{eq:en-est-SF}, and for $L\geq 4$ from \eqref{eq:en-est-SF} exclusively. Regarding the curvature energies in the various individual energy estimates, any summand with $\E^{(L-3)}(\Ric,\cdot)$ can be split using \eqref{eq:ibp-trick}, the resulting summands containing $\E^{(L-2)}(\Ric,\cdot)$ can always be absorbed in to the total energy term in the first line, and anything with $\E^{(\leq L-4)}(\Ric,\cdot)$ is tracked in $\langle\text{Err}\rangle_L$ for $L\geq 4$.\\

%By inserting the bootstrap assumptions \eqref{eq:BsEnSF} and \eqref{eq:BsEnRic} in the second line of the total energy inequality and the induction assumptions \eqref{eq:ind-ass-core-en} in the following two lines, we obtain while updating $c$:
%\begin{align*}
%\numberthis\label{eq:total-diff-ineq-induction}\E^{(L)}_{total}(t)\lesssim&\,\epsilon^4+\int_t^{t_0}\left(\epsilon^\frac18 a(s)^{-3}+a(s)^{-1-c\sqrt{\epsilon}}\right)\E^{(L)}_{total}(s)\,ds\\
%&\,+\int_t^{t_0}\left(\epsilon^\frac{11}4+\epsilon^{\frac{13}4}\right)a(s)^{-1-c\sigma}\,ds+\int_t^{t_0} \left\{\epsilon^\frac{23}8a(s)^{-3-c\epsilon^\frac18}+\epsilon^3a(s)^{-1-c\epsilon^\frac18}\right\}\,ds\\
%&\,+\langle\text{Err}\rangle_L
%%&\,+\underbrace{\int_t^{t_0}\epsilon^\frac{29}8a(s)^{-3-c\epsilon^\frac18}+\epsilon^\frac{7}2a(s)^{-1-c\epsilon^\frac18}}_{\text{not present for }L=2}
%\end{align*}


%Thus, \eqref{eq:total-diff-ineq-induction} can be rewritten as
%\begin{align*}
%\E^{(2)}_{total}(t)\lesssim&\,\epsilon^4+\int_t^{t_0}\left(\epsilon^\frac18 a(s)^{-3}+a(s)^{-1-c\sqrt{\epsilon}}\right)\sup_{r\in(s,t_0]}\E^{(2)}_{total}(r)\,ds\\
%&\,+\int_t^{t_0}\epsilon^\frac{11}4a(s)^{-1-c\sigma}\,ds+\int_t^{t_0} \epsilon^\frac{23}8a(s)^{-3-c\epsilon^\frac18}+\epsilon^3a(s)^{-1-c\epsilon^\frac18}\,ds\,,
%\end{align*}
%and since the right hand side strictly increases as $t$ decreases, we can replace the left hand side by $\sup_{r\in(t,t_0]}\E^{(L)}_{total}(r)$. Using \eqref{eq:a-exp-est} to compute the remaining integrals over $a$, applying the Gronwall lemma and updating $c$, this yields
%\begin{align*}
%\E^{(2)}_{total}(t)\leq\sup_{r\in(t,t_0]}\E^{(2)}_{total}(r)\lesssim a(t)^{-c\epsilon^\frac18}\left(\epsilon^4+\epsilon^\frac{11}4+\epsilon^\frac{11}4+\epsilon^3\right)\lesssim \epsilon^\frac{11}4a(t)^{-c\epsilon^\frac18}\,.
%\end{align*}
%


%For $L\geq 4$, we can simply insert \eqref{eq:ind-ass-phi} and \eqref{eq:ind-ass-Ric} and (updating $c>0$) obtain
%\[\langle\text{Err}\rangle_L\lesssim \int_t^{t_0}\epsilon^\frac{29}8a(s)^{-3-c\epsilon^\frac18}+\epsilon^\frac72a(s)^{-1-c\epsilon^\frac18}\,ds\,.\]
%Both of these terms are negligible compared to the terms in the second line of \eqref{eq:total-diff-ineq-induction}, and 
%\[\E_{total}^{(L)}\lesssim \epsilon^\frac{11}4a^{-c\epsilon^\frac18}\]
%follows from there by the same argument as for $L=2$. This shows that \eqref{eq:ind-ass-core-en} and \eqref{eq:ind-ass-Ric} can be extended to level $L$ (where $L-1$ is immediately covered by \eqref{eq:drop-odd-en}), and the bound on $\sum_{m=0}^{\frac{L}2-1}\|\Lap^m\phi\|_{L^2_G}^2$ in \eqref{eq:ind-ass-phi} can be extended to  $\|\Lap^\frac{L}2\phi\|_{L^2_G}$. Hence, it remains to show
%$\|\nabla\phi\|_{H^{L-1}_G}^2\lesssim \epsilon^\frac52a^{-c\epsilon^\frac18}$
%%and
%%\[\E^{(\leq L-4)}(N,\cdot)+a^4\E^{(L-3)}(N,\cdot)+a^8\E^{(L-2)}(N,\cdot)\lesssim \epsilon^\frac94a^{8-c\epsilon^\frac18}\] 
%to close the induction: To this end, we can apply \eqref{eq:Sobolev-norm-equiv-nablazeta2l+1} and \eqref{eq:Sobolev-norm-equiv-zetalow} with $\zeta=\phi$ as well as \eqref{eq:Sobolev-norm-equiv-zeta2l} for $\zeta=\Lap^m\phi,\,m\geq 1$:
%\begin{align*}
%\|\nabla\phi\|_{H^{L-1}_G}^2%=&\,\|\nabla\phi\|_{L^2_G}^2+\sum_{m=1}^{L-1}\|\nabla\phi\|^2_{\dot{H}^{m}_G}\\
%\lesssim&\, a^{-c\sqrt{\epsilon}}\left[\|\nabla\phi\|_{L^2_G}^2+\sum_{m=0}^{\frac{L}2-1}\|\nabla^2\Lap^m\phi\|^2_{L^2_G}+\epsilon \E^{(\leq L-4)}(\Ric,\cdot)\right]\\
%\lesssim&\,a^{-c\epsilon^\frac18}\left(\|\nabla\phi\|_{L^2_G}^2+\sum_{m=1}^\frac{L}2\|\Lap^m\phi\|_{L^2_G}^2\right)+\epsilon a^{-c\epsilon^\frac18} \E^{(\leq L-4)}(\Ric,\cdot)
%\end{align*}
%Now inserting the improved estimate for $\E_{total}^{(L)}$ yields
%\begin{align*}
%\|\nabla\phi\|_{H^{L-1}_G}^2\lesssim&\,a^{-c\epsilon^\frac18}\|\nabla\phi\|_{L^2_G}^2+\epsilon^\frac52a^{-c\epsilon^\frac18}\,\,
%\end{align*}
%which means it only remains to show the improvement for $\|\nabla\phi\|_{L^2_G}^2$. Note that by \eqref{eq:en-bs-imp-base} %, we have 
%%\[\E^{(\leq 2)}(\phi,\cdot)+\epsilon^\frac14\E^{(\leq 2)}(\Sigma,\cdot)\lesssim \E^{(2)}_{total}\lesssim \epsilon^\frac{11}4a^{-c\epsilon^\frac18}\,,\]
%%and hence using
%and Corollary \ref{cor:en-est-lapse}, one has
%\[\E^{(\leq 2)}(\phi,\cdot)+\E^{(\leq 2)}(N,\cdot)\lesssim \epsilon^\frac{11}4a^{-c\epsilon^\frac18}\,.\]
%Applying this to \eqref{eq:int-est-nablaphi0}, we obtain
%\[\|\nabla\phi\|_{L^2_G(\Sigma_t)}^2\lesssim a^{-c\epsilon^\frac18}\left(\epsilon^4+\int_t^{t_0}\epsilon^\frac{21}8a(s)^{-3-c\epsilon^\frac18}\,ds\right)\lesssim \epsilon^\frac{5}2a^{-3-c\epsilon^\frac18}\]
%and then \eqref{eq:en-bs-imp-phi} follows at order $L$. This closes the induction for $L\leq 18$ and proves \eqref{eq:en-bs-imp-SF}-\eqref{eq:en-bs-imp-phi}.\\

 %We have now improved all low order energy assumptions except for $\E^{(M-1)}(\phi,\cdot)$, in particular (plugging everything together with \todo{exchange lemma}) also for $\nabla\phi_{H^{M-2}_G}$.
%%\begin{remark}[Energy estimates aren't necessarily optimal]
%%We don't claim that the improved estimates above are optimal -- for example, we could likely improve the power of $\epsilon$ in the prefactor of \eqref{eq:en-bs-imp-phi}. However, \eqref{eq:en-bs-imp-Ric} and consequently \eqref{eq:en-bs-imp-N} are the best we can achieve under the estimates from the previous sections.
%%\end{remark}


%\subsection{Improving energies at top order}\label{subsec:bs-imp-top}
%
%To be able to ultimately control $\mathcal{H}$, we do not have sufficient energy control on the scalar field yet (beside not controlling the metric so far). Doing this and closing the bootstrap argument requires adding further scaled energy quantities in the total energy at top order:
%
%\begin{prop}[Improved energy behaviour, part 2]\label{prop:en-bs-imp-top}Define
%\begin{align*}
%\E_{top}=&\,\E^{(20)}(\phi,\cdot)+\epsilon^\frac14\left(\E^{(20)}(W,\cdot)+4\pi C^2\E^{(20)}(\Sigma,\cdot)\right)+a^4\E^{(21)}(\phi,\cdot)+\epsilon^\frac12a^4\E^{(21)}(\Sigma,\cdot)\\
%&\,+\epsilon^\frac14\|\Lap^{10}\phi\|_{L^2_G}^2+\epsilon^\frac12\E^{(18)}(\Ric,\cdot)+\epsilon^\frac34a^4\E^{(19)}(\Ric,\cdot)\,.
%\end{align*}
%Then, on $(t_{Boot},t_0]$, one has for small enough $\epsilon>0$:
%\begin{equation}\label{eq:en-bs-imp-top}
%\E_{top}\lesssim\epsilon^\frac{11}4a^{-c\epsilon^\frac18}
%\end{equation}
%In particular, this shows:
%\begin{subequations}
%\begin{align}
%\E^{(\leq 20)}(\phi,\cdot)\lesssim&\,\epsilon^\frac{11}4a^{-c\epsilon^\frac18} \label{eq:en-bs-imp-top-SF}\\
%\|\nabla\phi\|_{H^{18}_G}^2\lesssim&\,\epsilon^\frac{5}2a^{-c\epsilon^\frac18}\label{eq:en-bs-imp-top-phi}
%\end{align}
%\end{subequations}
%\end{prop}
%%\todo{Remark for here or later -- in Top order SF est., can deal with the $M$ order scalar field Sobolev norm by transforming to energy up to negl. curvature terms, and them term itself is negl. Similar trick for $\nabla^2\Lap^{\frac{M}2-1}$.}\\
%\begin{remark}
%To close the argument, we need the elliptic estimate in Lemma \ref{lem:en-est-Sigma-top} that only admits control on $a^4\E^{(21)}(\Sigma,\cdot)$. This also necessitates the same weight on the highest order scalar field and curvature energies. The contribution of $a^4\E^{(21)}(\Sigma,\cdot)$ in the total energy must be also be weighted sufficiently by $\epsilon$ so that the elliptic estimate itself only contributes small error terms, and $a^4\E^{(21)}(\Ric,\cdot)$ must be weighted accordingly. For the remaining energies, the weights are chosen as within $\E_{total}^{(L)}$ in the proof of Proposition \ref{prop:en-bs-imp}.
%\end{remark}
%
%
%\begin{proof}
%Again, we combine the respective estimates with the appropriate scalings, namely (for $L=M=20$ and in the listed order) \eqref{eq:en-est-SF}, \eqref{eq:en-est-BR}, \eqref{eq:en-est-Sigma}, \eqref{eq:en-est-SF-top}, \eqref{eq:en-est-Sigma-top}, \eqref{eq:int-est-Lapphi}, \eqref{eq:en-est-Ric} and \eqref{eq:en-est-Ric-top}. Observe that the sum of these scaled left hand sides control $\E_{top}$ by Lemma \ref{lem:en-error-cancellation}. Combining all of these estimates and inserting the initial data assumption \eqref{eq:init-ass-top}, we get the following estimate:
%
%\begin{subequations}
%\begin{align}
%\E_{top}(t)\lesssim&\,\epsilon^4+\int_t^{t_0}\left(\epsilon^\frac18a(s)^{-3}+a(s)^{-1-c\sqrt{\epsilon}}\right)\E_{top}(s)\label{eq:en-est-top-total1}\,ds\\
%&\,+\int_t^{t_0}\left\{\epsilon^\frac38a(s)^{-3-c\sqrt{\epsilon}}\left[\E^{(\leq 18)}(\phi,s)+\E^{(\leq 18)}(\Sigma,s)\right]+\epsilon^\frac{17}8a(s)^{-3-c\sqrt{\epsilon}}\E^{(\leq 18)}(W,s)\right.\label{eq:en-est-top-total2}\\
%&\,\phantom{+\int_t^{t_0}\bigr\{}+\epsilon^2a^{-1-c\sqrt{\epsilon}}\left(\|\nabla^2\Lap^9\phi\|_{L^2_G}^2+\|\nabla\phi\|_{H^{18}_G(\Sigma_s)}^2\right)+\epsilon a^{3-c\sqrt{\epsilon}}\|\nabla\phi\|_{H^{20}_G(\Sigma_s)}^2\label{eq:en-est-top-total3}\\
%&\,\phantom{+\int_t^{t_0}\bigr\{}\left.+\epsilon^\frac98a(s)^{-3-c\sqrt{\epsilon}}\|\nabla\phi\|_{H^{17}_G(\Sigma_s)}^2+\epsilon^\frac{11}8\E^{(\leq 16)}(\Ric,s)\right\}\,ds\label{eq:en-est-top-total4}\\
%&\,+\epsilon^\frac14\left(a(t)^{4-c\sqrt{\epsilon}}+\epsilon a(t)^{2-c\sqrt{\epsilon}}\right)\cdot \epsilon^\frac12\E^{(20)}(\Sigma,t)+\epsilon^\frac12\E^{(20)}(\phi,t)+\epsilon^\frac14\cdot \epsilon^\frac14\E^{(20)}(W,t)\label{eq:en-est-top-total5}\\
%&\,+\epsilon^\frac74\cdot \epsilon^\frac34a^4\E^{(19)}(\Ric,t)+\epsilon^\frac52 a^{2-c\sqrt{\epsilon}}\E^{(\leq 18)}(\phi,t)+\epsilon^\frac12 a^{2-c\sqrt{\epsilon}}\E^{(\leq 18)}(\Sigma,t)\label{eq:en-est-top-total6}\\
%&\,+\epsilon^\frac32 a^{2-c\sqrt{\epsilon}}\E^{(\leq 18)}(\Ric,t)\label{eq:en-est-top-total7}
%\end{align}
%\end{subequations}
%
%We briefly summarize which inequalities contain the listed error term bounds as explicit terms: The first two terms in \eqref{eq:en-est-top-total2} come from \eqref{eq:en-est-Ric} and the latter from \eqref{eq:en-est-BR}, those in \eqref{eq:en-est-top-total3} from \eqref{eq:en-est-SF-top}, those in \eqref{eq:en-est-top-total4} from \eqref{eq:int-est-Lapphi} and \eqref{eq:en-est-Ric} respectively, and finally the last three lines are precisely the scaled right hand side of \eqref{eq:en-est-Sigma-top}. In short, what now needs to be done is to reduce \eqref{eq:en-est-top-total2}-\eqref{eq:en-est-top-total4} to quantities we can bound with the improved bounds in Proposition \ref{prop:en-bs-imp} (up to absorbing high order terms into \eqref{eq:en-est-top-total1}), while \eqref{eq:en-est-top-total5}-\eqref{eq:en-est-top-total7} must be shown to carry strong enough $\epsilon$-weights that they only contribute negligible error terms.\\
%\end{proof}

\subsection{Improving solution norm control}\label{subsec:bs-imp-norm}

To close the bootstrap argument, it now remains to show that the improved energy bounds also imply improved bounds for $\mathcal{H}$ and $\mathcal{C}$. The former follows almost directly using Lemma \ref{lem:Sobolev-norm-equivalence-improved}:

\begin{corollary}[Improved Sobolev norm bounds]\label{cor:H-imp} On $(t_{Boot},t_0]$, the following estimates hold:
\begin{subequations}
\change{\begin{align}
\mathcal{H}\lesssim&\,\epsilon^\frac74 a^{-c\epsilon^\frac18}\label{eq:H-norm-imp}\\
\|\Sigma\|_{H^{18}_G}^2\lesssim&\,\epsilon^\frac{15}4a^{-c\epsilon^\frac18}\label{eq:H-Sigma-imp}\\
\|N\|_{H^{18}_G}^2\lesssim&\,\epsilon^4a^{-c\epsilon^\frac18}\label{eq:H-lapse-imp}
\end{align}}
\end{subequations}
\end{corollary}
\begin{proof}
First, we apply the improved energy estimates from \change{Proposition \ref{prop:en-bs-imp} }as well as the strong $C_G$-norm bounds from Lemma \ref{lem:AP} to the near-coercivity estimates in Lemma \ref{lem:Sobolev-norm-equivalence-improved}. With this, we directly obtain the following Sobolev norm estimates (updating $c$):
\change{\begin{align*}
\|\Psi\|_{H_G^{\change{18}}}^2\lesssim&\,\epsilon^4a^{-c\epsilon^\frac18}+\epsilon a^{-c\epsilon^\frac18}\cdot\epsilon^\frac{15}4a^{-c\epsilon^\frac18}\lesssim\epsilon^4a^{-c\epsilon^\frac18}\\
\|\Sigma\|_{H^{18}_G}^2\lesssim&\,\epsilon^\frac{15}4a^{-c\epsilon^\frac18}\\
\|\Ric[G]+\frac29G\|_{H^{16}_G}^2\lesssim&\,\epsilon^\frac72 a^{-c\epsilon^\frac18}\\
\|\RE\|_{H^{18}_G}^2+\|\RB\|_{H^{18}_G}^2\lesssim&\,\epsilon^\frac{15}4a^{-c\epsilon^\frac18}
%a^8\|N\|_{\dot{H}^{18}_G}^2+a^4\|N\|^2_{\dot{H}^{17}_G}+\|N\|^2_{\dot{H}^{16}_G}\lesssim&\,\epsilon^\frac94a^{4-c\epsilon^\frac18}\,,
\end{align*}}
%Note that, from Proposition \change{\ref{prop:en-bs-imp}}, we use \change{\eqref{eq:en-bs-imp-SF}} to obtain the bootstrap improvement for $\Psi$, and $\|\nabla\phi\|_{H^{18}_G}$ is already sufficiently controlled by \eqref{eq:en-bs-imp-phi} -- the rest follow from Proposition \ref{prop:en-bs-imp}.\\
\change{By Lemma \ref{lem:norm-est-nablaphi}, we also have
\changefinal{\begin{equation}\label{eq:nabla-phi-norm-imp}
\|\nabla\phi\|_{H^{17}_G}\lesssim \left(1+\epsilon a^{-c\sqrt{\epsilon}}\right)\|\Sigma\|_{H^{18}_G}+\epsilon\|\Psi\|_{H^{18}_G}\lesssim \epsilon^\frac{15}4a^{-c\epsilon^\frac18}\,.
\end{equation}}}
We take particular care in showing that the improved bound holds for \changefinal{$a^2\|\nabla\phi\|_{\dot{H}^{18}_G}$}: First, note that \eqref{eq:en-bs-imp-Ric} \change{implies }$\E^{(\leq 17)}(\Ric,\cdot)\lesssim\epsilon^\frac72 a^{-c\epsilon^\frac18}$. Applying this along with \eqref{eq:APmidphi} to \changefinal{\eqref{eq:Sobolev-norm-equiv-nablazeta2l}}, as well as \eqref{eq:Sobolev-norm-equiv-zetalow} in the second line and \change{\eqref{eq:en-bs-imp-SF} as well as \eqref{eq:nabla-phi-norm-imp} }in the final step, we obtain:
\changefinal{\begin{align*}
a^4\|\nabla\phi\|_{H^{18}_G}^2\lesssim&\, a^4\|\nabla\Lap^9\phi\|_{L^2_G}^2+a^{4-c\sqrt{\epsilon}}\sum_{m=0}^8\|\nabla\Lap^m\phi\|_{L^2_G}^2+\epsilon a^{4-c\sqrt{\epsilon}}\cdot \E^{(\leq 16)}(\Ric,\cdot)\\
\lesssim&\,a^{4-c\sqrt{\epsilon}}\left(\|\nabla\Lap^{9}\phi\|_{L^2_G}^2+\|\nabla\phi\|_{H^{17}_G}^2\right)+\epsilon^\frac{9}2a^{-c\epsilon^\frac18}\\
\lesssim&\,a^{-c\sqrt{\epsilon}}\cdot \E^{(\leq 18)}(\phi,\cdot)+a^{4-c\sqrt{\epsilon}}\|\nabla\phi\|_{H^{17}_G}^2+\epsilon^\frac{9}2a^{-c\epsilon^\frac18}\\
\lesssim&\,\epsilon^\frac{15}4 a^{-c\epsilon^\frac18}
\end{align*}}
%Further, we get from \todo{integral estimate} that\\
%\begin{equation*}
%\|\nabla\Lap^{9}\phi\|_{L^2_G(\Sigma_t)}^2\lesssim\epsilon^4+\int_t^{t_0}\epsilon^\frac18a(s)^{-3}\|\nabla\Lap^9\phi\|_{L^2_G(\Sigma_s)}^2\,ds+\int_t^{t_0}\epsilon^\frac{21}8a(s)^{-3-c\epsilon^\frac18}\,ds
%\end{equation*}
%and thus
%\begin{equation*}
%\|\nabla\Lap^9\phi\|_{L^2_G}^2\lesssim \epsilon^\frac{5}2a^{-3-c\epsilon^\frac18}\,.
%\end{equation*}
Further, inserting \eqref{eq:en-bs-imp-SF}, \eqref{eq:en-bs-imp-Sigma} and \eqref{eq:en-bs-imp-Ric} into Corollary \ref{cor:en-est-lapse} implies
\begin{align*}
\,a^8\|\Lap^{10} N\|^2_{L^2_G}+a^4\|\nabla\Lap^{9}N\|^2_{L^2_G}+\sum_{m=0}^{9}\|\Lap^mN\|_{L^2_G}^2%\\
%\lesssim&\,\left(\epsilon^{2+\frac52}+\epsilon^\frac{11}4+\epsilon^{4+\frac94}\right)a^{-c\epsilon^\frac18}+\epsilon^{2+\frac94}a^{8-c\sigma}\\
\lesssim&\,\epsilon^\frac{11}4a^{-c\epsilon^\frac18}
\end{align*}
and subsequently, applying Lemma \ref{lem:Sobolev-norm-equivalence-improved} as before,
\begin{equation*}
a^8\|N\|^2_{\dot{H}^{20}_G}+a^4\|N\|^2_{\dot{H}^{19}_G}+\|N\|^2_{H^{18}_G}\lesssim\epsilon^4a^{-c\epsilon^\frac18}\,.
\end{equation*}
Finally, having now shown \eqref{eq:H-Sigma-imp} and \eqref{eq:H-lapse-imp}, we can apply these to \eqref{eq:norm-est-G} to get
\begin{equation*}
\|G-\gamma\|^2_{H^{18}_G}\lesssim a^{-c\epsilon^\frac18}\left(\epsilon^4+\epsilon^{-\frac14+\frac{15}4}+\epsilon^{-\frac14+4}\right)\lesssim\epsilon^\frac{7}2a^{-c\epsilon^\frac18}\,,
\end{equation*}
%Note that we can drop the supremum since $a^{-c\epsilon^\frac18}$ takes its maximum on $[t,t_0]$ in $t$. This now finally shows 
proving \eqref{eq:H-norm-imp}.
\end{proof}

Intuitively, the bounds on $\mathcal{C}$ should now follow from $\mathcal{H}$ by the standard Sobolev embedding. However, since both of these norms are with respect to $G$, the embedding constant may be time dependent. To circumvent this issue, we need to switch between norms with \changefinal{respect }to $G$ and $\gamma$ and then apply the embedding with respect to $C_\gamma$ and $H_\gamma$. The following lemma ensures that we still obtain bootstrap improvements after performing these norm switches:

\begin{lemma}[Moving between norms]\label{lem:G-gamma-norm-switch} Let $l\in\N,\,l\leq \change{18}$, $\zeta$ be a scalar field, $\mathfrak{T}$ be an arbitrary $\Sigma_t$-tangent tensor\delete{ and let $U$ be either a coordinate neighbourhood on $\Sigma_t$ or $\Sigma_t$ itself}. Then, for some multivariate polynomial $P_l$ with $P_l(0,0)=0$, we have
\begin{subequations}
\begin{align}
\|\zeta\|_{C^l_G(U)}\lesssim&\,a^{-c\sqrt{\epsilon}}\|\zeta\|_{C^l_\gamma(\change{M})}+a^{-c\sqrt{\epsilon}}\|\zeta\|_{C_\gamma^{\max\left\{0,\lfloor\frac{l-1}2\rfloor\right\}}(\change{M})}\cdot P_l\left(\|G-\gamma\|_{C^{l-1}_\gamma(\change{M})},\|G^{-1}-\gamma^{-1}\|_{C^{l-1}_\gamma(\change{M})}\right) \label{eq:C-norm-exch-zeta}\\
\|\mathfrak{T}\|_{C^l_G(\change{M})}\lesssim&\,a^{-c\sqrt{\epsilon}}\|\mathfrak{T}\|_{C^l_\gamma(\change{M})}+a^{-c\sqrt{\epsilon}}\|\mathfrak{T}\|_{C_\gamma^{\max\left\{0,\lfloor\frac{l-1}2\rfloor\right\}}(\change{M})}\cdot P_l\left(\|G-\gamma\|_{C^l_\gamma(\change{M})},\|G^{-1}-\gamma^{-1}\|_{C^{l}_\gamma(\change{M})}\right)\label{eq:C-norm-exch-T}
\end{align}
\end{subequations}
as well as the same inequalities with the roles of $G$ and $\gamma$ reversed. For $l\leq 12$, this reduces to:
\begin{subequations}
\begin{align}
a^{c\sqrt{\epsilon}}\|\zeta\|_{C^l_\gamma(\change{M})}\lesssim\|\zeta\|_{C^l_G(\change{M})}\lesssim\,a^{-c\sqrt{\epsilon}}\|\zeta\|_{C^l_\gamma(\change{M})} \label{eq:AP-exch-zeta}\\
a^{c\sqrt{\epsilon}}\|\mathfrak{T}\|_{C^l_\gamma(\change{M})}\lesssim\|\mathfrak{T}\|_{C^l_G(\change{M})}\lesssim\,a^{-c\sqrt{\epsilon}}\|\mathfrak{T}\|_{C^l_\gamma(\change{M})}\,\label{eq:AP-exch-T}
\end{align}
\end{subequations}
Further, one has
\begin{subequations}
\begin{align}
\|\zeta\|^2_{H^l_\gamma(\change{M})}\lesssim&\,a^{-c\sqrt{\epsilon}}\|\zeta\|^2_{H^l_G(\change{M})}+a^{-c\epsilon^\frac18}\|\zeta\|_{C_G^{\lceil \frac{l-1}2\rceil}(M)}^2\left(\epsilon^4+\epsilon^{-\frac14}\sup_{s\in(\cdot,t_0)}\left(\|N\|_{H^{l-1}_G(\Sigma_s)}^2+\|\Sigma\|_{H^{l-1}_G(\Sigma_s)}^2\right)\right)\label{eq:H-norm-exch-zeta}\,,\\
\|\mathfrak{T}\|^2_{H^l_\gamma(\change{M})}\lesssim&\,a^{-c\sqrt{\epsilon}}\|\mathfrak{T}\|^2_{H^l_G(\change{M})}+a^{-c\epsilon^\frac18}\|\mathfrak{T}\|_{C_G^{\lceil \frac{l-1}2\rceil}(M)}^2\left(\epsilon^4+\epsilon^{-\frac14}\sup_{s\in(\cdot,t_0)}\left(\|N\|_{H^{l}_G(\Sigma_s)}^2+\|\Sigma\|_{H^{l}_G(\Sigma_s)}^2\right)\right) \label{eq:H-norm-exch-T}
\end{align}
\end{subequations}
%with
%\begin{equation}
%\|\zeta\|_{H^1_\gamma}\lesssim a^{-c\sqrt{\epsilon}}\|\zeta\|_{H^1_G}\label{eq:H1-norm-exch-zeta}
%\end{equation}
\end{lemma}
\begin{remark}
While we only need the tensorial inequalities for gradient vector fields and $(0,2)$-tensors when applied to norms in $\mathcal{H}$ and $\mathcal{C}$, the proof is simpler when considering tensors of arbitrary rank.
\end{remark}
\begin{proof}
We restrict ourselves to proving the tensorial statements; the scalar field analogues follow analagously except for the fact that, since $\nabla_i\zeta=\nabhat_i\zeta=\del_i\zeta$, error terms caused by Christoffel symbols always enter at one order less. Thus, it remains to show \eqref{eq:C-norm-exch-T}, \eqref{eq:AP-exch-T} and \eqref{eq:H-norm-exch-T} by iterating over derivative order. \\
Starting with the base level estimates, we have if $\mathfrak{T}$ is of rank $(r,s)$:
\begin{align*}
\lvert\mathfrak{T}\rvert_G^2-\lvert\mathfrak{T}\rvert_\gamma^2=&\,\Bigr[G_{i_1j_1}\dots G_{i_rj_r} (G^{-1})^{p_1q_1}\dots(G^{-1})^{p_sq_s}-\gamma_{i_1j_1}\dots \gamma_{i_rj_r} (\gamma^{-1})^{p_1q_1}\dots(\gamma^{-1})^{p_sq_s}\Bigr]\cdot\\
&\,\cdot{\mathfrak{T}^{i_1\dots i_r}}_{p_1\dots p_s}{\mathfrak{T}^{j_1\dots j_r}}_{q_1\dots q_s}
%\lesssim&\,\|G-\gamma\|_{C^0_G}\left(1+\|G-\gamma\gamma\|_{C^0_G}\right)\lvert\mathfrak{T}\rvert_\gamma^2
\end{align*}
We successively replace $G^{\pm 1}$ by $(G^{\pm 1}-\gamma^{\pm 1})+\gamma^{\pm 1}$, %to transform the first factor into a sum of products of $(G^{\pm 1}-\gamma^{\pm 1})$ and $\gamma^{\pm 1}$, where at least one of the former and one of the latter always occurs. Taking
take the $\lvert\cdot\rvert_\gamma$-norm of each factor and use \eqref{eq:APmidG}-\eqref{eq:APmidG-1}. This yields%as well as $\lvert\gamma^{\pm 1}\rvert_\gamma=\sqrt{3}$, this yields
\begin{equation*}
\left\lvert\lvert\mathfrak{T}\rvert_{G}^2-\lvert\mathfrak{T}\rvert_\gamma^2\right\rvert\lesssim \sqrt{\epsilon}a^{-c\sqrt{\epsilon}}\lvert\mathfrak{T}\rvert_\gamma^2\,,
\end{equation*}
implying 
%\begin{equation*}
%\lvert\mathfrak{T}\rvert_G^2\lesssim (1+\sqrt{\epsilon}a^{-c\sqrt{\epsilon}})\lvert\mathfrak{T}\rvert_\gamma^2\lesssim a^{-c\sqrt{\epsilon}}\|\mathfrak{T}\|_{C^0_\gamma}^2
%\end{equation*}
\eqref{eq:C-norm-exch-T} (and \eqref{eq:AP-exch-T}) for $l=0$ after rearranging and taking supremums suitably.\\
To show \eqref{eq:H-norm-exch-T} at base level, consider
\begin{align*}
&\,\int_{\change{M}} \lvert\mathfrak{T}\rvert_G^2\,\vol{G}-\int_{\change{M}}\lvert\mathfrak{T}\rvert_{\gamma}^2\,\vol{\gamma}\\
=&\,\int_{\change{M}} \left(\lvert\mathfrak{T}\rvert_G^2-\lvert\mathfrak{T}\rvert_\gamma^2\right)\,\vol{G}+\int_{\change{M}}\lvert\mathfrak{T}\rvert_\gamma^2\frac{\mu_G-\mu_\gamma}{\mu_\gamma}\,\vol{\gamma}\,.
\end{align*}
We can control the first summand on the right hand side % by
%\[\lesssim \sqrt{\epsilon}a^{-c\sqrt{\epsilon}}\int_M\lvert\mathfrak{T}\rvert_G^2\,\vol{G}\]
as before, while we have $\lvert\mu_G-\mu_\gamma\rvert\lesssim \epsilon$ by \eqref{eq:APvol}. Hence,
\[(1-K\epsilon)\|\mathfrak{T}\|_{L^2_\gamma}^2\lesssim (1+\sqrt{\epsilon}a^{-c\sqrt{\epsilon}})\|\mathfrak{T}\|_{L^2_G}^2\]
follows for a suitable constant $K>0$, implying the statement for small enough $\epsilon>0$.

Next, we perform the iteration for \eqref{eq:C-norm-exch-T}, assuming the statement and the analogue with $\gamma$ and $G$ reversed to hold \change{up to order $l-1$}. As above, note that
\begin{equation*}
\left\lvert\,\left\lvert\nabla^J\mathfrak{T}\right\rvert_G^2-\left\lvert\nabhat^J\mathfrak{T}\right\rvert_\gamma^2\right\rvert\lesssim\sqrt{\epsilon}a^{-c\sqrt{\epsilon}}\left\lvert\nabhat^J\mathfrak{T}\right\rvert_\gamma^2+\change{(1+\sqrt{\epsilon}a^{-c\sqrt{\epsilon}})}\left\lvert\left\lvert\nabla^J\mathfrak{T}\right\rvert_\gamma^2-\left\lvert\nabhat^J\mathfrak{T}\right\rvert_\gamma^2\right\rvert\,
\end{equation*}
where we can rewrite the second term as
\begin{equation*}
\left\lvert 2\langle\nabhat^J\mathfrak{T}-\nabla^J\mathfrak{T},\nabhat\mathfrak{T}\rangle_\gamma-\lvert\nabla^J\mathfrak{T}-\nabhat^J\mathfrak{T}\rvert_\gamma^2\right\rvert
\end{equation*}
and hence obtain (moving between pointwise norms as before)
\begin{align*}
\left\lvert\,\left\lvert\nabla^J\mathfrak{T}\right\rvert_G^2-\left\lvert\nabhat^J\mathfrak{T}\right\rvert_\gamma^2\right\rvert%\lesssim &\,(1+\sqrt{\epsilon}a^{-c\sqrt{\epsilon}})\left\lvert\nabhat^J\mathfrak{T}\right\rvert_\gamma^2+\left\lvert \nabla^J\mathfrak{T}-\nabhat^J\mathfrak{T}\right\rvert_\gamma^2\\
\lesssim&\,a^{-c\sqrt{\epsilon}}\left\lvert\nabhat^J\mathfrak{T}\right\rvert_\gamma^2+a^{-c\sqrt{\epsilon}}\left\lvert \nabla^J\mathfrak{T}-\nabhat^J\mathfrak{T}\right\rvert_\gamma^2\,.
\end{align*}
Regarding $\nabla^J\mathfrak{T}-\nabhat^J\mathfrak{T}$, we have the following schematic decomposition \delete{locally on some coordinate neighbourhood $V\subseteq\Sigma_t$}:
\begin{align*}
\numberthis\label{eq:nabla-nabhat}\nabla^J\mathfrak{T}-\nabhat^J\mathfrak{T}=&\,\sum_{I=0}^{J-1}\nabhat^{J-I-1}(\Gamma-\Gamhat)\ast_\gamma\left(\nabla^{I}\mathfrak{T}+\nabhat^{I}\mathfrak{T}\right)\\
&\,+\langle\text{at least cubic nonlinear terms}\rangle\,,
\end{align*} 
Here, $\ast_\gamma$ encodes the analogous schematic product notation with regard to $\gamma$ (see subsection \ref{subsubsec:schematic-notation}). Regarding the Christoffel symbols, notice \eqref{eq:C-norm-exch-T} with roles of $\gamma$ and $G$ reversed holding up to $l-1$ implies that, for any $m\in\{0,\dots,l-1\}$ and some multivariate polynomial $\tilde{P}_m$, we have
\[\|\Gamma-\Gamhat\|_{C_\gamma^{m}(\change{M})}\lesssim a^{-c\sqrt{\epsilon}}\tilde{P}_m(\|\Gamma-\Gamhat\|_{C_G^{m}(\change{M})},\|G-\gamma\|_{C^{\change{m}}_G(\change{M})},\|G^{-1}-\gamma^{-1}\|_{C^{\change{m}}_G(\change{M})})\,.\]
As explained in Remark \ref{rem:relation-metric-Chr}, we can bound $\|\Gamma-\Gamhat\|_{C_G^{m}(\change{M})}$ by a polynomial in $\|G-\gamma\|_{C_G^{m+1}(\change{M})}$. Hence, we can apply \eqref{eq:APmidG} to obtain 
\begin{equation}\label{eq:APmidChr}
\|\Gamma-\Gamhat\|_{C_\gamma^{l-1}(\change{M})}\lesssim \sqrt{\epsilon}a^{-c\sqrt{\epsilon}}\,.
\end{equation}
Moving back to \eqref{eq:nabla-nabhat} and just considering the first line for now, this implies 
\begin{align*}
\numberthis\label{eq:C-norm-exch-T-last-step}\left\lvert\|\mathfrak{T}\|^2_{\dot{C}^l_G(\change{M})}-\|\mathfrak{T}\|^2_{\dot{C}_\gamma^l(\change{M})}\right\rvert\lesssim&\,a^{-c\sqrt{\epsilon}}\left(\|\mathfrak{T}\|^2_{C^l_\gamma(\change{M})}+\sum_{m=0}^{l-1}\|\nabla^{m}\mathfrak{T}\|^2_{C^0_\gamma(\change{M})}\right)\\
&\,+\left(\|\mathfrak{T}\|^2_{C^{\lceil\frac{l-1}2\rceil}_\gamma(\change{M})}+\sum_{m=0}^{\lceil\frac{l-1}2\rceil}\|\nabla^{m}\mathfrak{T}\|^2_{C^0_\gamma(\change{M})}\right)\|\Gamma-\Gamhat\|^2_{C^{l-1}_\gamma(\change{M})}\,\\
&\,+\langle\text{at least cubic nonlinear terms}\rangle\,.
\end{align*}
We can rewrite $\nabla^m\mathfrak{T}$-norms in $C_\gamma$ as ones in $C_G$ up to $a^{-c\sqrt{\epsilon}}$ as before. Then, we can apply the already obtained estimates up to order $l-1$ show that the first two lines of the right hand side can be estimated by the right hand side of \eqref{eq:C-norm-exch-T}. The highly nonlinear terms can be dealt with similarly, closing the induction over admissible $l$. \delete{ on local coordinate [...]} %neighbourhoods. Since $\Sigma_t$ is compact, this also implies \eqref{eq:C-norm-exch-T} for $U=\Sigma_t$ after updating constants. 
The inequality in \eqref{eq:AP-exch-T} immediately follows by applying \eqref{eq:APmidG}-\eqref{eq:APmidG-1} and \eqref{eq:APmidChr}.

%The terms with higher order nonlinearities are essentially similar since no terms with higher order derivatives occur, neither do we get additional factors in $\mathfrak{T}$, we only incur additional powers of $(\Gamma-\Gamhat)$-norms that can be controlled by terms that are already present. Further, we can again control any Christoffel terms in $C^{l-1}_\gamma$ by a polynomial in $\|G-\gamma\|_{C^l_\gamma(M)}$. Hence, the nonlinear terms don't qualitatively change the estimates, proving \eqref{eq:C-norm-exch-T} at order $l$ on any coordinate neighbourhood after rearranging as well as the analogue with $\gamma$ and $G$ switched. Since any terms on the right hand side can be estimated by their norms on $\Sigma_t$ and $\Sigma_t$ is compact, this then also implies the estimate on $\Sigma_t$ at order $l$ and closes the proof of \eqref{eq:C-norm-exch-T}. \\

%\eqref{eq:AP-exch-T} follows by the analogous iterative argument until reaching \eqref{eq:C-norm-exch-T-last-step}. Here, the iteration assumption can be used to control all $C$-norms that occur on the right hand side of \eqref{eq:C-norm-exch-T-last-step} (after replacing $\gamma$ by $G$ where needed), and \eqref{eq:APmidChr} can even be applied to the high order Christoffel terms in the second line of \eqref{eq:C-norm-exch-T-last-step}. The right hand side then collapses to $a^{-c\sqrt{\epsilon}}\|\mathfrak{T}\|_{C^l_\gamma}^2$.\\

Now, assume \eqref{eq:H-norm-exch-T} to be proven up to order $J-1$. By analogous arguments as at order zero, we get, \change{after rearranging},
\change{\[\int_M\lvert\nabhat^J\mathfrak{T}\rvert_\gamma^2\,\vol{\gamma}\lesssim \left\lvert\int_M\left(\lvert\nabla^J\mathfrak{T}\rvert_G^2-\lvert\nabhat^J\mathfrak{T}\rvert_{\gamma}^2\right)\,\vol{G}\right\rvert+\int_M\lvert\nabla^J\mathfrak{T}\rvert_G^2\,\vol{G}\,,\]}
so we only need to concern ourselves with the first summand. Reversing roles of $G$ and $\gamma$ compared to the proof of \eqref{eq:C-norm-exch-T}, we get
\begin{equation*}
\left\lvert\,\left\lvert\nabla^J\mathfrak{T}\right\rvert_G^2-\left\lvert\nabhat^J\mathfrak{T}\right\rvert_\gamma^2\right\rvert\lesssim\sqrt{\epsilon}a^{-c\sqrt{\epsilon}}\left\lvert\nabla^J\mathfrak{T}\right\rvert_G^2+\change{a^{-c\sqrt{\epsilon}}}\left\lvert 2\langle\nabla^J\mathfrak{T}-\nabhat^J\mathfrak{T},\nabla\mathfrak{T}\rangle_G-\lvert\nabla^J\mathfrak{T}-\nabhat^J\mathfrak{T}\rvert_G^2\right\rvert
,
\end{equation*}
and have the following, applying Lemma \ref{lem:int-est-Chr} immediately to estimate $\|\Gamma-\Gamhat\|_{H^{l-1}_G}$:
\begin{align*}
&\,\left\rvert\int_V\left\{2\langle\nabla^l\mathfrak{T}-\nabhat^l\mathfrak{T},\nabla\mathfrak{T}\rangle_G-\lvert\nabla^l\mathfrak{T}-\nabhat^l\mathfrak{T}\rvert_G^2\right\}\vol{G}\right\rvert\\
%\lesssim&\,\|\Gamma-\Gamhat\|_{H^{l-1}_G(\change{M})}\left(\|\mathfrak{T}\|_{C^{\lceil\frac{l-1}2\rceil}_G(\change{M})}+\|\nabhat^{\leq l-1}\mathfrak{T}\|_{C^{0}_G(\change{M})}\right)\|\mathfrak{T}\|_{\dot{H}_G^l(\change{M})}\\
%&\,+\sqrt{\epsilon}a^{-c\sqrt{\epsilon}}\left(\|\mathfrak{T}\|_{H^{l-1}_G(\change{M})}^2+\|\nabhat^{\leq l-1}\mathfrak{T}\|_{H^{l-1}_G(\change{M})}^2\right)\right]\cdot\\
\lesssim&\,a^{-c\sqrt{\epsilon}}\left(\|\mathfrak{T}\|^2_{H^l_G(\change{M})}+\|\nabhat^{\leq l-1}\mathfrak{T}\|^2_{H^{0}_G(\change{M})}\right)\\
&\,+\left(\|\mathfrak{T}\|^2_{C^{\lceil\frac{l-1}2\rceil}_G(M)}+\|\nabhat^{\leq \lceil\frac{l-1}2\rceil}\mathfrak{T}\|^2_{C^{0}_G(M)}\right)\cdot \\
&\,\quad \cdot a^{-c\epsilon^\frac18}\left(\epsilon^4+\epsilon^{-\frac14}\sup_{s\in(\cdot,t_0)}\left(\|N\|_{H^{l}_G(\Sigma_s)}^2+\|\Sigma\|_{H^{l}_G(\Sigma_s)}^2\right)\right)
\end{align*}
By the same arguments as earlier, we have $\|\nabhat^{\leq l-1}\mathfrak{T}\|_{H^{l-1}_G(\change{M})}\lesssim a^{-c\sqrt{\epsilon}}\|\mathfrak{T}\|_{H^{l-1}_\gamma(\change{M})}$ and can then apply the induction hypothesis. \change{This proves \eqref{eq:H-norm-exch-T}.}
\end{proof}

\begin{corollary}[Improved $C$-norm bounds]\label{cor:C-impr} On $(t_{Boot},t_0]$, the following estimate is satisfied:
\begin{equation}\label{eq:C-impr}
\change{\mathcal{C}+\mathcal{C}_\gamma\lesssim\changefinal{\epsilon^\frac74}a^{-c\epsilon^\frac18}}
\end{equation}
\end{corollary}
\begin{proof}
We first apply the Sobolev norm estimates in Lemma \ref{lem:G-gamma-norm-switch} \change{to \eqref{eq:H-norm-imp}}, to then control $\mathcal{C}_\gamma$ via the standard Sobolev embedding $H^{l+2}_\gamma(M)\hookrightarrow C^l(M)$, and finally control $\mathcal{C}$ with \eqref{eq:C-norm-exch-zeta}-\eqref{eq:C-norm-exch-T}.\\
Note that by Lemma \ref{lem:AP}, we can control the $C_G$-norm up to order $10$ of every quantity occurring in $\mathcal{H}$ beside the lapse by at worst $\sqrt{\epsilon}a^{-c\sqrt{\epsilon}}$, while the bootstrap assumption already implies better behaviour for the lapse. Thus, we can apply \eqref{eq:H-norm-exch-zeta}-\eqref{eq:H-norm-exch-T} to every norm appearing in $\mathcal{H}$, and obtain by applying \eqref{eq:H-Sigma-imp} and \eqref{eq:H-lapse-imp} in the second line:
\begin{align*}
%&\,\|\Psi\|_{C^{17}_\gamma}^2+\|\nabla\phi\|^2_{C^{16}_\gamma}+\|\Sigma\|^2_{C^{16}_\gamma}+\|\RE\|^2_{C^{16}_\gamma}+\|\RB\|^2_{C^{16}_\gamma}\\
%&\,+\|G-\gamma\|^2_{C^{16}_\gamma}%+\|\Gamma-\Gamhat\|_{C^{M-5}_G}
%+\|\Ric[G]+2G\|^2_{C^{14}_\gamma}\\
\mathcal{C}_\gamma^2\lesssim&\,a^{-c\sqrt{\epsilon}}\cdot \mathcal{H}^2+\epsilon a^{-c\sqrt{\epsilon}}\cdot a^{-c\epsilon^\frac18}\left(\epsilon^4+\epsilon^{-\frac14}\sup_{s\in(t,t_0)}\left(\|N\|_{H^{18}_G(\Sigma_s)}^2+\|\Sigma\|_{H^{18}_G(\Sigma_s)}^2\right)\right)\\
\lesssim&\,\change{\epsilon^\frac72a^{-c\epsilon^\frac18}}+\epsilon a^{-c\epsilon^\frac18}\left(\epsilon^4+\change{\epsilon^\frac72}\right)\\
\lesssim&\,\change{\epsilon^\frac72}\cdot a^{-c\epsilon^\frac18}
\end{align*}
In particular, we can update $c$ such that
\[\lvert P(\|G-\gamma\|_{C^{16}_\gamma(\Sigma_t)},\|G-\gamma\|_{C^{16}_\gamma(\Sigma_t)})\rvert\lesssim\change{\epsilon^\frac72} a^{-c\epsilon^\frac18}\]
holds for any multivariate polynomial $P$ that appears when applying \eqref{eq:C-norm-exch-zeta}-\eqref{eq:C-norm-exch-T}. Again using the strong $C_G$-norm estimates from Lemma \ref{lem:AP}, this then implies $\mathcal{C}\lesssim \changefinal{\epsilon^\frac74} a^{-c\epsilon^\frac18}$.
%\[\mathcal{C}\lesssim a^{-c\sqrt{\epsilon}}\mathcal{C}_\gamma+\sqrt{\epsilon}a^{-c\sqrt{\epsilon}}\cdot \epsilon^\frac{9}8a^{-c\epsilon^\frac18}\lesssim \epsilon^\frac98a^{-c\epsilon^\frac18}\,.\]
%Now taking care of the lapse, we have 
%\[\|N\|_{C^1_G}\lesssim a^{-c\sqrt{\epsilon}}\|N\|_{C^1_\gamma}\lesssim \epsilon^\frac98 a^{4-c\epsilon^\frac18}\] at low order due to \todo{[simplified exchange]}. Applying first \todo{$C_G-C_\gamma$-exchange} and then \todo{$H_\gamma-H_G$-exchange} iteratively then also shows
%\[a^{-4}\|N\|_{C^{14}_G}+a^{-2}\|N\|_{\dot{C}^{15}_G}+\|N\|_{\dot{C}^{16}_G}\lesssim\epsilon^\frac98 a^{-c\epsilon^\frac18}\]
%and hence the statement follows.
\end{proof}

\section{Big Bang stability: The main theorem}\label{sec:main-thm}

In this section, we provide the proof of the first main result, Theorem \ref{thm:main-past}, which we state in more detail in Theorem \ref{thm:main} below. As in \cite{Rodnianski2014,Speck2018}, most of the work has already been done by establishing the necessary bounds on solution norms.

\begin{remark}[Existence of a CMC hypersurface]\label{rem:CMC-hypersurface}
As mentioned in \change{Section \ref{subsubsec:initial-data}}, it may seem that the generality of the results in Theorem \ref{thm:main} is restricted by taking the initial data on $\Sigma_{t_0}$ to be CMC. %%\change{However, note that the foliation of FLRW spacetime by constant time hypersurfaces is CMC. Furthermore, using gauges that do not put restrictions on the initial data (e.g.,~ harmonic gauge), one can show that there are local solutions on $(t_1,t_0]\times M$ to the Einstein scalar-field system in an open neighbourhood of FLRW initial data, and construct a Banach space over these solutions. Then, one considers a map from such solutions and their foliations by hypersurfaces diffeomorphic to $M$ to the resulting mean curvature functions. An implicit function theorem argument then, morally, yields the following type of statement:\\
%%For any initial data $(g,k)$ in a sufficiently small neighbourhood of $(\gamma,\gamma)$ in $C^M\times C^{M-1}$, the respective local solution $\g$ has an open neighbourhood $V$ in the solution space in which any element admits a CMC foliation.\\
However, as long as one remains close enough to a constant time hypersurface of the FLRW reference metric (which is CMC), one can locally evolve the perturbed data in harmonic gauge to a nearby hypersurface that is CMC and remains close to the FLRW reference solution. %For more detailed arguments, see \cite[Chapter 14]{Rodnianski2014} in flat and spherical spatial geometry, and \cite[Section 2.5]{FajKr20} for positive cosmological constant in vacuum, both of which can be extended to our setting.
\change{To make this a bit more precise, and also since this is a little less involved than the arguments in \cite{Rodnianski2014}, we will briefly sketch how the arguments from \cite[Section 2.5]{FajKr20} extend to our setting. \\%
First, we once again assume without loss of generality that our initial data is sufficiently regular. Note that we can locally evolve our data within harmonic gauge to get a $C^{17}$-regular family of metrics with near-FLRW initial data (for well-posedness, consider the analogue of \cite[Proposition 14.1]{Rodnianski2014}). Consider the Banach manifold $\mathcal{M}^{17}$ formed by the set of $C^{17}$ Lorentz metrics on $I\times\M$ for an open interval $I$ around $t_0$ such that the surfaces of constant time are Riemannian, endowed with the norm
\[\|\tilde{g}\|=\|\tilde{n}^2\|_{C^{17}_{dt^2+\gamma}(I\times M)}+\|\tilde{X}\|_{C^{17}_{dt^2+\gamma}(I\times M)}+\|\tilde{g}_t\|_{C^{17}_{dt^2+\gamma}(I\times M)}\,,\]
where $\tilde{g}\in\mathcal{M}^{17}$ has lapse $\tilde{n}$, shift $\tilde{X}$ and spatial metrics $(\tilde{g}_t)_{t\in I}$. Further, for any $f\in C^{17}(M,I)$, we define the embedding $\iota_f: M\hookrightarrow \M$ by $x\mapsto (f(x),x)$, and subsequently define the smooth map
\begin{gather*}
H_0:\mathcal{D}:=\{(\tilde{g},f)\in \mathcal{M}^{17}\times C^{17}(M,I)\vert \iota_f^\ast\tilde{g}\text{ is Riemannian}\}\longrightarrow C^{16}(M)\\
(\tilde{g},f)\mapsto \text{mean curvature of }(M,\tilde{g}_t)\text{ embedded along }\iota_f\,.
\end{gather*}
%Finally, we take $H=H_0-\tau(t_0)$Then, we obviously have $H(\g_{FLRW},t_0)=0$, and can compute the differential of $H$ in the second argument using \eqref{eq:Friedman} and \eqref{eq:Friedman2}:
%\begin{align*}
%dH_{(\g_{FLRW},t_0)}(0,w)=&\,\frac13\left[\Lap_{a(t_0)^2\gamma} w-\left(\Ric_{\g_{FLRW}}(\del_t,\del_t)+\left\lvert k_{FLRW}(t_0)\right\rvert_{a(t_0)^2\gamma}^2\right)w\right]\\
%=&\,\frac13a(t_0)^{-2}\Lap_\gamma w+\left(\frac{\ddot{a}(t_0)}{a(t_0)}-\left(\frac{\dot{a}(t_0)}{a(t_0)}\right)^2\right)w\\
%=&\,a(t_0)^{-2}\left(\frac13\Lap_\gamma -\left(\frac19+4\pi C^2a(t_0)^{-4}\right)\right)w\\
%\end{align*}
%Since $\Lap_\gamma$ has no positive eigenvalues for any Riemannian metric, it follows that $(\g_{FLRW},t_0)$ is a regular value of $H$.
One easily checks that $(\g_{FLRW},t_0)$ is a regular point of $H_0$. By the implicit function theorem for Banach manifolds, this means there is a (unique) smooth function $F$ that maps an open neighbourhood of $\g_{FLRW}$ in $\mathcal{M}^{17}$ to an open neighbourbood of the constant function $x\mapsto t_0$ in $C^{17}(M,I)$ such that $H_0(\cdot,F(\cdot))=\tau(t_0)$ holds in that neighbourhood.\\
Thus, we can choose a surface $\Sigma^\prime$ with mean curvature $\tau(t_0)$ near the original $\Sigma_{t_0}$. Furthermore, for small enough $\epsilon>0$, the initial data on $\Sigma^\prime$ remains close to the FLRW initial data in the sense of Assumption \ref{ass:init}, using similar arguments to control Sobolev norms. Thus, we can replace $\Sigma_{t_0}$ by $\Sigma^\prime$ without loss of generality, proving that the CMC assumption \eqref{eq:CMC} is not a true restriction.}
\end{remark}

\begin{theorem}[Stability of Big Bang formation]\label{thm:main}
Let $(M,\mathring{g},\mathring{k},\mathring{\pi},\mathring{\psi})$ be initial data to the Einstein scalar-field system as discussed in Section \ref{subsubsec:initial-data}. Further, let the data be embedded into a time-oriented 4-manifold such that it induces initial data for the rescaled solution variables (see Definition \ref{def:rescaled}) at the initial hypersurface $\Sigma_{t_0}$. We also assume this rescaled initial data is close to that of the FLRW reference solution (see \eqref{eq:FLRW-metric} and \eqref{eq:FLRW-wave}) in the sense that
\begin{equation}\label{eq:ass-init-main-thm}
\mathcal{H}(t_0)\change{+\mathcal{H}_{top}(t_0)}+\mathcal{C}(t_0)\change{\leq}\epsilon^2
\end{equation}
is satisfied (with $\mathcal{H}$ and $\mathcal{C}$ as in Definition \ref{def:sol-norm}).  \delete{along with [...] }%the top order energy assumption \eqref{eq:init-ass-top} for some small enough $\epsilon>0$. Finally, we make the additional regularity assumption \eqref{eq:ass-lwp}.
\footnote{Essentially, this translates to smallness in $H_\gamma^{\change{19}}$ and $C_\gamma^{17}$ \delete{and [...]}. %$H_\gamma^{23}$-regularity for our solution variables.
For $\epsilon=0$, the solution is the FLRW reference solution.}\\

Then, the past maximal globally hyperbolic development $((0,t_0]\times M,\g,\phi)$ of this data within the Einstein scalar-field system \eqref{eq:ESF1}-\eqref{eq:ESF2} in CMC gauge \eqref{eq:CMC} with zero shift is foliated by the CMC hypersurfaces $\Sigma_s=t^{-1}(\{s\})$, and one has
\begin{equation}\label{eq:bs-imp-main-thm}
\mathcal{H}(t)+\mathcal{C}(t)+\mathcal{C}_{\gamma}(t)\lesssim \change{\epsilon^\frac74} a(t)^{-c\epsilon^\frac18}
\end{equation}
for some $c>0$ and any $t\in(0,t_0]$. In particular, this implies the following statements:\\[1em]

\textbf{Asymptotic behaviour of solution variables:} We denote the solution metric as $\g=-n^2dt^2+g$, the second fundamental form (viewed as a $(1,1)$-tensor) with respect to $\Sigma_t$ as $k$ and the volume form with regard to $g$ on $\Sigma_t$ by $\vol{g}$. There exist a smooth function $\Psi_{Bang}\in C_\gamma^{15}(M)$, a $(1,1)$-tensor field $K_{Bang}\in C^{15}_\gamma(M)$ and a volume form $\vol{Bang}\in C^{15}_\gamma(M)$ such that the following estimates hold for any $t\in(0,t_0]$:%\footnote{We view these footprint states interchangeably as objects on $M$ or as time independent objects on $(0,t_0]\times M$.}
\begin{subequations}\label{eq:asymp}
\begin{align}
\label{eq:asymp-lapse}\|n-1\|_{C^l_\gamma(\Sigma_t)}\lesssim&\,\begin{cases} 
%\epsilon a^{4-c\sqrt{\epsilon}} & l\leq 12\\
\epsilon a(t)^{4-c\epsilon^\frac18} & l\leq 14\\
\epsilon a(t)^{2-c\epsilon^\frac18} & l=15
\end{cases}\\
%%%%%%%%%%%%%%%%%%
\label{eq:asymp-vol}\left\|a^{-3}\vol{g}-\vol{Bang}\right\|_{C^{l}_\gamma(\Sigma_t)}\lesssim&\,\begin{cases} 
%\epsilon a^{4-c\sqrt{\epsilon}} & l\leq 12\\
\epsilon a(t)^{4-c\epsilon^\frac18} & l\leq 14\\
\epsilon a(t)^{2-c\epsilon^\frac18} & l=15
\end{cases}\\
%%%%%%%%%%%%%%%%%%%
\label{eq:asymp-Psi}\left\|a^3\del_t\phi-(\Psi_{Bang}+C)\right\|_{C^{l}_\gamma(\Sigma_t)}\lesssim&\,\begin{cases} 
%\epsilon a^{4-c\sqrt{\epsilon}} & l\leq 13\\
\epsilon a(t)^{4-c\epsilon^\frac18} & l\leq 14\\
\epsilon a(t)^{2-c\epsilon^\frac18} & l=15
\end{cases}\\
%%%%%%%%%%%%%%%%%%%%
\label{eq:asymp-phi}\changefinal{\left\|\phi(t,\cdot)-\phi(t_0,\cdot)+\int_t^{t_0}a(s)^{-3}\,ds\cdot(\Psi_{Bang}+C)\right\|_{\dot{C}^l_\gamma(\Sigma_t)}}\lesssim&\,\begin{cases}
%\epsilon a^{4-c\sqrt{\epsilon}} & l\leq 13\\
\epsilon a(t)^{4-c\epsilon^\frac18} & 1\leq l\leq 14\\
\epsilon a(t)^{2-c\epsilon^\frac18} & l=15
\end{cases}\\
%%%%%%%%%%%%%%%%%%%%%%%
\label{eq:asymp-K}\left\|a^{3}k-K_{Bang}\right\|_{C^{l}_\gamma(\Sigma_t)}\lesssim&\,\begin{cases} 
%\epsilon a^{4-c\sqrt{\epsilon}} & l\leq 12\\
\epsilon a(t)^{4-c\epsilon^\frac18} & l\leq 14\\
\epsilon a(t)^{2-c\epsilon^\frac18} & l=15\,
\end{cases}
\end{align}
\end{subequations}

\noindent Further, these footprint states satisfy the equations\footnote{These are precisely the (generalized) Kasner relations, see Section \ref{subsec:FLRW-Kasner}.}
\begin{subequations}
\begin{align}
\label{eq:Bang-CMC}{(K_{Bang})^a}_a=&\,-\sqrt{12\pi}C\,,\\
8\pi(\Psi_{Bang}+C)^2+{(K_{Bang})^a}_b{(K_{Bang})^b}_a=&\,12\pi C^2
\label{eq:Bang-Hamil}
\end{align}
\end{subequations}
and remain close to the data of the reference solution in the following sense, where $\I$ denotes the Kronecker symbol:
\begin{subequations}\label{eq:footprint}
\begin{align}
\left\|\vol{\gamma}-\vol{Bang}\right\|_{C^{15}_\gamma(M)}\lesssim&\,\epsilon \label{eq:footprint-vol}\\
\left\|\Psi_{Bang}\right\|_{C^{15}_\gamma(M)}\lesssim&\,\epsilon \label{eq:footprint-Psi}\\
\left\|K_{Bang}+\sqrt{\frac{4\pi}3}C\I\right\|_{C^{15}_\gamma(M)}\lesssim&\,\epsilon\,\label{eq:footprint-K}
\end{align}
\end{subequations}
Additionally, there exists a $(0,2)$-tensor field $M_{Bang}\in C^{15}_\gamma(M)$ satisfying
\begin{equation}\label{eq:footprint-G}
\left\|M_{Bang}-\gamma\right\|_{C^{15}_\gamma(M)}\lesssim\epsilon
\end{equation}
and, with $\odot$ and $\exp$ meant in the matrix product and exponential sense respectively, one has
\begin{equation}
\label{eq:asymp-G}\left\|g\odot\exp\left[\left(\change{-2}\int_t^{t_0}a(s)^{-3}\,ds\right)\cdot K_{Bang}\right]-M_{Bang}\right\|_{C^l_\gamma(\Sigma_t)}\lesssim\begin{cases}
%\epsilon a(t)^{4-c\sqrt{\epsilon}} & l\leq 12\\
 \epsilon a(t)^{4-c\epsilon^\frac18} & l\leq 14\\
\epsilon a(t)^{2-c\epsilon^\frac18} & l=15\,.
\end{cases}
\end{equation}
Moreover, the Bel-Robinson variables $E$ and $B$ satisfy the estimates
\begin{subequations}
\begin{align}
\label{eq:asymp-E}\|E\|_{C^{16}_\gamma(\Sigma_t)}\lesssim&\,%\begin{cases}
%\epsilon a^{-2-c\sqrt{\epsilon}} & l\leq 10\\
%\epsilon a^{-2-c{\epsilon}^\frac18} & l\leq 14\\
\epsilon a^{-4-c\epsilon^\frac18}% & l\leq 16\,,
%\end{cases}
\\
\label{eq:asymp-B}\|B\|_{C^l_\gamma(\Sigma_t)}\lesssim&\,\begin{cases}
%\epsilon a^{-2-c\sqrt{\epsilon}} & l\leq 11\\
\epsilon a^{-2-c{\epsilon}^\frac18} & l\leq 15\\
\epsilon a^{-4-c\epsilon^\frac18} & l\leq 16\,.
\end{cases}
\end{align}
\end{subequations}

\textbf{Causal disconnectedness:} Let $\curve$ be a past directed causal curve on $(\change{(0,t]}\times M,\g)$ \change{ for $t\leq t_0$ }with domain $[s_1,s_{max})$ such that $\curve(s_1)\in \change{\Sigma_{t}}$ and $s_{max}$ is maximal. Then, there exists a constant $\mathcal{K}>0$ that does not depend on $\curve$ such that one has
\begin{equation}\label{eq:causal-disconn}
L[\curve]=\int_{s_1}^{s_{max}}\sqrt{(\gamma_{ab})_{\curve(s)}\dot{\curve}^a(s)\dot{\curve}^b(s)}\,ds\leq \mathcal{K} a(t)^{2-c\epsilon^\frac18}\,,
\end{equation}
where $\gamma$ is the negative Einstein spatial reference metric on $M$ (see Definition \ref{def:spatial-mf}). \change{Hence, for points $p,q\in\Sigma_t$ with $\text{dist}_\gamma(p,q)>2\mathcal{K}a(t)^{2-c\epsilon^\frac18}$, the causal pasts of $p$ and $q$ cannot intersect.}\\

\textbf{Geodesic incompleteness:} \changefinal{Let $\curve(\mathcal{A})$ be a past directed, affinely parametrized causal geodesic emanating from $\Sigma_{t_0}$, where $\mathcal{A}:(0,t_0]\rightarrow [0,\infty)$ denotes the parameter time that is normalized to $\mathcal{A}(t_0)=0$. }Then,
\begin{equation}\label{eq:geod-incomp}
\mathcal{A}(0)\leq \mathcal{K}_1\cdot \lvert \mathcal{A}^\prime(t_0)\rvert \cdot a(t_0)^{1+K_2\epsilon}\int_0^{t_0}a(s)^{-1-\mathcal{K}_2\epsilon}\,ds<\infty\,,
\end{equation}
holds for suitable constants $\mathcal{K}_1,\mathcal{K}_2>0$ that are independent of $\curve$, and thus any such geodesic crashes into the Big Bang hypersurface in finite affine parameter time.\\

\textbf{Blow-up:} \begin{subequations}
The norm $\lvert k\rvert_g$ behaves toward the Big Bang hypersurface as follows:
\begin{equation}\label{eq:blowup-k}
\left\|a^6\lvert k\rvert_g^2-{(K_{Bang})^i}_j{(K_{Bang})^j}_i\right\|_{C^0_\gamma(\Sigma_t)}\lesssim \epsilon a^{4-c{\epsilon}^\frac18}
\end{equation}
Further, with $W[\g]$ denoting the Weyl curvature and $P[\g]=\Riem[\g]-W[\g]$,
\begin{equation}\label{eq:blowup-P}
\left\|a^{12}P_{\alpha\beta\gamma\delta}P^{\alpha\beta\gamma\delta}-\frac{5}3\cdot(8\pi)^2(\Psi_{Bang}+C)^4\right\|_{C^0_\gamma(M)}\lesssim \epsilon a^{4-c{\epsilon}^\frac18}\,
\end{equation}
is satisfied, whereas there exists a scalar footprint $W_{Bang}\in C^{15}_\gamma(M)$ such that one has
\begin{equation}\label{eq:blowup-W}
\left\|a^{12}W_{\alpha\beta\gamma\delta}W^{\alpha\beta\gamma\delta}-W_{Bang}\right\|_{C^0(M)}\lesssim \epsilon a^{2-c{\epsilon}^\frac18}\,.
\end{equation}
Here, $W_{Bang}$ is a fourth order polynomial in $\hat{K}_{Bang}=K_{Bang}+\sqrt{\frac{4\pi}3}C\I$ and $\Psi_{Bang}$ and satisfies
\begin{equation}\label{eq:footprint-W}
\|W_{Bang}\|_{C^{15}_\gamma(M)}\lesssim\epsilon\,.
\end{equation}
Finally, \change{the scalar curvature $R[\g]$ and the Ricci curvature invariant $\Ric[\g]_{\alpha\beta}\Ric[\g]^{\alpha\beta}$ blow up with the asymptotics
\begin{align}
\label{eq:blowup-scalar} \|a^6R[\g]-8\pi(\Psi_{Bang}+C)^2\|_{C^0(M)}\lesssim&\,\epsilon a^{4-c\epsilon^\frac18}\,,\\
\label{eq:blowup-Ricci}\|a^{12}\Ric[\g]_{\alpha\beta}\Ric[\g]^{\alpha\beta}-(8\pi)^2(\Psi_{Bang}+C)^4\|_{C^0(M)}\lesssim&\,\epsilon a^{4-c\epsilon^\frac18}\,,
\end{align}
and }the Kretschmann scalar $\mathcal{K}=\Riem[\g]_{\alpha\beta\gamma\delta}\Riem[\g]^{\alpha\beta\gamma\delta}$ exhibits stable blow-up in the following sense:
\begin{equation}\label{eq:blowup-Kretschmann}
\change{\left\|a^{12}\mathcal{K}-\frac{5}3\cdot(8\pi)^2(\Psi_{Bang}+C)^4-W_{Bang}\right\|_{C^0(M)}\lesssim \epsilon a^{2-c\epsilon^\frac18}}
\end{equation}
\end{subequations}
\end{theorem}
\begin{remark}[The solution variables exhibit AVTD behaviour]\label{rem:AVTD} The estimates \eqref{eq:asymp-lapse}-\eqref{eq:asymp-K} and \eqref{eq:asymp-G} imply that the solution is asymptotically velocity term dominated (AVTD) in the sense that, toward the Big Bang singularity, they behave at leading order like solutions to the (formal) velocity term dominated equations. These arise by dropping any terms containing spatial derivatives in the decomposed Einstein system, i.e.\,in \eqref{eq:EEqg}, \eqref{eq:EEqk}, \eqref{eq:EEqLapse} and \eqref{eq:wave}.
\end{remark}
\begin{proof}
\change{As argued at the end of Section \ref{subsec:lwp}, we can assume without loss of generality that our initial data is sufficiently regular. Hence, the local existence statement in Lemma \ref{lem:lwp} and the initial data requirements \eqref{eq:ass-init-main-thm} ensure that there exists a local solution to the Einstein scalar-field system on $[t_1,t_0]\times M$ and that the bootstrap assumption (see Assumption \ref{ass:bootstrap}) holds }on $[t_1,t_0]\times M$ with $t_1\in(0,t_0)$ and $\sigma=\epsilon^\frac1{16}.$ 
Let $\mathfrak{t}\in(0,t_0)$ be such that $(\mathfrak{t},t_0]\times M$ is the maximal domain on which the solution variables exist and satisfy the bootstrap assumptions. For contradiction, we now assume that $\mathfrak{t}>0$ were to hold.\\
%\footnote{\todo{Note that the bootstrap assumption on the Bel-Robinson variables hold precisely due to the well-posedness of the system \eqref{eq:REEqE}-\eqref{eq:REEqB} with the other variables already given, since the constraints alone would only give us that $\E^{(\leq 16)}(W,\cdot)$ is small initially due to the Ricci tensor appearing in \eqref{eq:REEqConstrE}. Both $\RE$ and $\RB$ must satisfy the constraints and the evolution equations at lower orders since both arise as direct computations from the classical solution that exists independently of them, and hence the local solution given by the rescaled evolution equations must actually coincide with the constraint equations on the shared domain of existence.}}\\

Due to Corollary \ref{cor:C-impr}, there exist (summarizing all updates) constants $c_1,K_1>0$ such that, for any $t\in(\mathfrak{t},t_0]$,
\begin{equation}\label{eq:bs-imp-total}
\mathcal{C}(t)\leq K_1\changefinal{\epsilon^\frac74}a(t)^{-c_1\epsilon^\frac18}\,
\end{equation}
%is satisfied \delete{as well as [...]}%, due to \change{Proposition \ref{prop:en-bs-imp},}
%\begin{subequations}
%\begin{align}
%\E^{(\leq \change{18})}(\phi,t)\leq&\,K_1\epsilon^{\frac{11}4}a(t)^{-c_1\epsilon^\frac18}\label{eq:en-imp-main-SF}\\
%\E^{(\leq 18)}(\Sigma,t)\leq&\,K_1\epsilon^\frac{5}2a(t)^{-c_1\epsilon^\frac18}\label{eq:en-imp-main-Sigma}\\
%\E^{(\leq 18)}(W,t)\leq&\,K_1\epsilon^\frac{21}8a(t)^{-c_1\epsilon^\frac18}\label{eq:en-imp-main-BR}\\
%\E^{(\leq 16)}(\Ric,t)\leq&\,K_1\epsilon^\frac{9}4a(t)^{-c_1\epsilon^\frac18}\label{eq:en-imp-main-Ric} \\
%\|\nabla\phi\|_{H^{17}_G(\Sigma_t)}^2\leq&\,K_1\epsilon^\frac{9}4a(t)^{-c_1\epsilon^\frac18}\label{eq:en-imp-main-phi}\\
%\E^{(\leq 16)}(N,t)+a(t)^4\E^{(17)}(N,t)+a(t)^8\E^{(18)}(N,t)\leq&\,K_1\epsilon^\frac94a(t)^{8-c_1\epsilon^\frac18}\,\label{eq:en-imp-main-N}.
%\end{align}
%\end{subequations}
If $\epsilon$ is small enough such that $K_1\epsilon^\frac18<K_0$ and $c_1\epsilon^\frac1{8}<c_0\sigma$ hold, this is a strict improvement of the bootstrap assumption. Furthermore, argued exactly as in the proof of \cite[Theorem 15.1]{Rodnianski2014}, above improvement ensures none of the blow-up criteria of Lemma \ref{lem:lwp} are satisfied if $\mathfrak{t}>0$ were to hold, essentially as a direct consequence of \eqref{eq:bs-imp-total}. Hence, the solution could be classically extended to a CMC hypersurface $\Sigma_{\mathfrak{t}}$ diffeomorphic to $M$ while satisfying the improved estimates by continuity, and further to an interval $(\mathfrak{t}^\prime,t_0]$ for some $0<\mathfrak{t}^\prime<\mathfrak{t}$ on which the bootstrap assumptions must then be satisfied, also by continuity. This contradicts the maximality of $(\mathfrak{t},t_0]$.\\

Thus, the rescaled solution variables induce a unique solution to the Einstein scalar-field system on $(0,t_0]\times M$ such that \eqref{eq:bs-imp-total} \change{is }satisfied for any $t\in(0,t_0]$. The core estimate \eqref{eq:bs-imp-main-thm} follows since Corollaries \ref{cor:H-imp} and \ref{cor:C-impr} now hold on $(0,t_0]$.\\

From \eqref{eq:bs-imp-main-thm}, the asymptotic behaviour in \eqref{eq:asymp-lapse}-\eqref{eq:asymp-K} and \eqref{eq:asymp-G} is established as in \cite[Theorem 15.1]{Rodnianski2014}, which we briefly outline: First, we note that \eqref{eq:asymp-lapse} follows directly from \eqref{eq:bs-imp-main-thm}. For the remaining estimates, the arguments are similar, so consider for example $\del_t\phi$: By the rescaled wave equation \eqref{eq:REEqWave} and \eqref{eq:asymp-lapse}, we have that
\begin{equation*}
\left\|\del_t\Psi\right\|_{C^l_\gamma(\Sigma_t)}\lesssim\begin{cases}
\epsilon a^{1-c\epsilon^\frac18} & l\leq 14\\
\epsilon a^{-1-c\epsilon^\frac18} & l=15
\end{cases}
\end{equation*}
Hence, for an arbitrary decreasing sequence $(t_m)_{m\in\N},$ on $(0,t_0]$ that converges to zero, we have
\[\|\Psi(t_{m_1},\cdot)-\Psi(t_{m_2},\cdot)\|_{C^l_\gamma(M)}\lesssim\begin{cases}
\epsilon a(t_{m_1})^{4-c\epsilon^\frac18} & l\leq 14\\
\epsilon a(t_{m_1})^{2-c\epsilon^\frac18} & l=15
\end{cases}\]
for any $m_1,m_2\in\N,\,m_1<m_2$ by \eqref{eq:a-integrals}. This shows that $\Psi(t_{m_1},\cdot)$ is a Cauchy sequence in $C_\gamma^{15}(M)$ and hence there exists a limit function ${\Psi}_{Bang}\in C_\gamma^{15}(M)$ that satisfies
\[\|\Psi(t,\cdot)-{\Psi}_{Bang}\|_{C^l_\gamma(M)}\lesssim\begin{cases}
\epsilon a(t)^{4-c\epsilon^\frac18} & l\leq 14\\
\epsilon a(t)^{2-c\epsilon^\frac18} & l=15
\end{cases}\]
for any $t\in(0,t_0]$. Since $\Psi=a^3n^{-1}\del_t\phi-C$ holds by definition, \eqref{eq:asymp-Psi} now follows by examining the Taylor expansion of $n^{-1}-1$ at $0$ using \eqref{eq:asymp-lapse}.\\

\noindent The identity \eqref{eq:Bang-CMC} follows directly from the CMC condition \eqref{eq:CMC}, the asymptotic behaviour \eqref{eq:asymp-K} of $a^3k$ and the Friedman equation \eqref{eq:Friedman}, while \eqref{eq:Bang-Hamil} follows from the asymptotic limit of the Hamiltonian constraint \eqref{eq:Hamilton} with \eqref{eq:Friedman}, \eqref{eq:asymp-lapse}, \eqref{eq:asymp-Psi} and \eqref{eq:asymp-K} as well as \eqref{eq:bs-imp-main-thm} for lower order terms. The asymptotics in \eqref{eq:asymp-G} follows exactly as in \cite[Theorem 15.1]{Rodnianski2014}, and \eqref{eq:footprint-vol}-\eqref{eq:footprint-G} are a direct result of the initial data assumptions and applying the respective asymptotic estimates to $t=t_0$.

\noindent For the first estimate in \eqref{eq:asymp-B}, we apply the momentum constraint \eqref{eq:REEqConstrB} to get
\[\lvert\nabla^J B\rvert_G=\change{a^{-4}\lvert\nabla^J \RB\rvert_G}= \change{a^{-2}\lvert\nabla^J\curl_G\Sigma\rvert_G}\lesssim a^{-2}\lvert\nabla^{J+1}\Sigma\rvert_G\]
and consequently, with Lemma \ref{lem:G-gamma-norm-switch} as well as \eqref{eq:APmidB} and \eqref{eq:bs-imp-main-thm},
\begin{align*}
\|B\|_{C^{15}_\gamma(\Sigma_t)}\lesssim&\,a^{-c\sqrt{\epsilon}}\|B\|_{C^{15}_G(\Sigma_t)}+\epsilon a^{-2-c\sqrt{\epsilon}}\cdot P_{{15}}(\|G-\gamma\|_{C^{15}_\gamma(\Sigma_t)})\\
\lesssim&\,a^{-2-c\sqrt{\epsilon}}\|\Sigma\|_{C^{16}_G(\Sigma_t)}+\epsilon a^{-2-c\sqrt{\epsilon}}\cdot P_{15}(\|G-\gamma\|_{C^{15}_\gamma(\Sigma_t)})\\
\lesssim&\,\epsilon a^{-2-c\epsilon^\frac18}\,.
\end{align*}
The remaining estimates in \eqref{eq:asymp-E} and \eqref{eq:asymp-B} are contained in \eqref{eq:bs-imp-main-thm}. The results \eqref{eq:causal-disconn} and \eqref{eq:geod-incomp} follow as in the proofs of (15.6) and (15.7) in \cite[Theorem 15.1]{Rodnianski2014} from the asymptotic behaviour of the solution variables in \eqref{eq:asymp-lapse}-\eqref{eq:asymp-K} and \eqref{eq:asymp-G}. \change{We briefly sketch the proof of \eqref{eq:geod-incomp}: Consider a geodesic $\curve$ affinely parametrized by $\mathcal{A}$ as in the statement. The geodesic equations then lead to the following estimate for some suitable $\mathcal{K}>0$:
\[\lvert \mathcal{A}^{\prime\prime}\rvert\leq \frac{\dot{a}}a\lvert\mathcal{A}^\prime\rvert+\mathcal{K}\left[\frac{\dot{a}}a\lvert N\rvert+n^{-1}\lvert\del_tN\rvert+n^{-1}\lvert \nabla N\rvert_g+n\lvert\hat{k}\rvert_g\right]\lvert\mathcal{A}^\prime\rvert\,.\]
The leading term is hereby arises from the mean curvature condition. Arguing as with the elliptic estimates in Section \ref{sec:lapse}, one can show that $\lvert\del_tN\rvert\lesssim \epsilon a^{-1-c\epsilon^\frac18}$. Thus, along with the other pointwise bounds on $n$, $g$ and $\hat{k}$, one obtains
\[\lvert \mathcal{A}^{\prime\prime}\rvert\leq \frac{\dot{a}}a\left(1+c\epsilon\right)\lvert\mathcal{A}^\prime\rvert\]
and consequently
\[\lvert\mathcal{A}^\prime(t)\rvert\leq \lvert \mathcal{A}^\prime(t_0)\rvert a(t)^{-1-c\epsilon}\]
by the Gronwall lemma. \eqref{eq:geod-incomp} follows by integrating.\\}


Turning to the blow-up behaviour of geometric invariants, observe \eqref{eq:blowup-k} is a direct consequence of \eqref{eq:asymp-K}. Regarding \eqref{eq:blowup-W}, we first compute
%\[W_{\alpha\beta\gamma\delta}W^{\alpha\beta\gamma\delta}=W_{abcd}W^{abcd}+4W_{a0c0}W^{a0c0}-W_{abc0}W^{abc0}\]
using \eqref{eq:Weyl-reconstruct} and standard algebraic manipulations that
\[a^{12}W_{\alpha\beta\gamma\delta}W^{\alpha\beta\gamma\delta}=a^{12}\left(\change{8}\lvert E\rvert_g^2+8\lvert B\rvert_g^2\right)=\change{8}\lvert\RE\rvert_G^2+8\lvert\RB\rvert_G^2\,.\]
%The key in deducing\eqref{eq:blowup-W} is now to use that $\lvert\RE\rvert_G^2$ is better behaved than $\RE$ itself:
By the rescaled constraint equation \eqref{eq:REEqConstrE}, we have
\begin{equation*}
\RE_{ij}=-\dot{a}a^2\Sigma_{ij}+(\Sigma\odot\Sigma)_{ij}-\left[\frac{8\pi}3\Psi^2+\frac{16\pi}3C\Psi\right]G_{ij}+\O{\epsilon a^{4-c\epsilon^\frac18}}\,,
\end{equation*}
for $t\downarrow 0$. Further, by expanding \eqref{eq:Friedman} around $a=0$, we have 
$\dot{a}a^2=\sqrt{\frac{4\pi}3}C+\O{a^2}$.
%Hence, one obtains the following:
%\begin{align*}
%\lvert\RE\rvert_G^2=&\,\left\lvert-\sqrt{\frac{4\pi}3}C\Sigma+(\Sigma\odot\Sigma)-\left[\frac{8\pi}3\Psi^2+\frac{16\pi C^2}3C\Psi\right]G\right\rvert_G^2+\O{\epsilon a^{2-c\epsilon^\frac18}}\\
%=&\,\frac{4\pi}3C^2{(\Sigma^\sharp)^i}_j{(\Sigma^\sharp)^j}_i-2\sqrt{\frac{4\pi}3}C\cdot{(\Sigma^{\sharp})^i}_j{(\Sigma^{\sharp})^j}_l{(\Sigma^{\sharp})^l}_i+{(\Sigma^{\sharp})^i}_j(\Sigma^{\sharp})^j_l{(\Sigma^{\sharp})^l}_m{(\Sigma^{\sharp})^m}_i\\
%&\,-2\cdot\left[\frac{8\pi}3\Psi^2+\frac{16\pi}3C\Psi\right]{(\Sigma^\sharp)^i}_j{(\Sigma^\sharp)^j}_i+3\cdot\left[\frac{8\pi}3\Psi^2+\frac{16\pi}3C\Psi\right]+\O{\epsilon a^{2-c\epsilon^\frac18}}
%\end{align*}
Since $\Sigma^\sharp$ and $\Psi$ converge to footprint states $\hat{K}_{Bang}=K_{Bang}+\sqrt{\frac{4\pi C}3}\I$ and $\Psi_{Bang}$ in $C^{15}_\gamma(M)$ respectively, this shows that $8\lvert\RE\rvert_G^2$ converges to some $W_{\text{Bang}}\in C^{15}_\gamma(M)$ that can be expressed as a fourth-order polynomial in $\hat{K}_{\mathrm{Bang}}$ and $\Psi_{Bang}$ and satsifies
\[\left\|\lvert\RE\rvert_G^2-\change{\frac1{8}}W_{Bang}\right\|_{C^0(M)}\lesssim \epsilon a^{2-c\epsilon^\frac18}\]
as well as \eqref{eq:footprint-W}. Due to \eqref{eq:asymp-B}, the $\lvert\RB\rvert_G^2$-term in the Weyl curvature scalar is negligible in comparison, and thus \eqref{eq:blowup-W} immediately follows.\\
Furthermore, %denoting the Schouten tensor with regard to $\g$ by $\text{Sch}$, 
one has 
\begin{align*}
P_{\alpha\beta\gamma\delta}P^{\alpha\beta\gamma\delta}%=&\,8\text{Sch}_{\alpha\beta}\text{Sch}^{\alpha\beta}-4{\text{Sch}^\alpha}_{\alpha}\\
=&\,2\Ric[\g]_{\alpha\beta}\Ric[\g]^{\alpha\beta}-\frac29 R[\g]^2\deletemath{-\frac16R[\g]},\,
\end{align*}
and \eqref{eq:blowup-P} is a direct consequence of \eqref{eq:blowup-Ricci} and \eqref{eq:blowup-scalar}, which follow once more with  \eqref{eq:asymp-Psi} and \eqref{eq:asymp-lapse} as well as \eqref{eq:bs-imp-main-thm} for error terms. Finally, \eqref{eq:blowup-Kretschmann} is obtained from \eqref{eq:blowup-P}-\eqref{eq:blowup-W}.
%Inserting $\Ric[\g]_{\mu\nu}=8\pi\nabla_\mu\phi\nabla_\nu\phi$ and the asymptotic behaviour \eqref{eq:asymp-Psi} of $\del_t\phi$ (and equivalently of $n^{-1}\del_t\phi$ due to \eqref{eq:asymp-lapse}), and recognizing that all terms with $\lvert\nabla\phi\rvert_g^2$ become negligible by \eqref{eq:bs-imp-main-thm}, \eqref{eq:blowup-P} follows.
\end{proof}

\section{Future stability}\label{sec:fut}

\noindent The goal of this section is to show the following theorem:

\begin{theorem}[Future stability of Milne spacetime]\label{thm:fut-stab-simple}
Let $\epsilonnew>0$ Let the rescaled initial data $(\fg,\bm{k},\nabla\phi,\phi^\prime)$  (see Def. \ref{def:fut-rescaled}) be sufficiently close to $(\gamma,\frac13\gamma,0,0)$ in $H^5\times H^4\times H^4\times H^4$ on some initial hypersurface $\Sigma_{\tau=\tau_0}$. Then, its maximal globally hyperbolic development $(\g,\phi)$ within the Einstein scalar-field system in CMCSH gauge along the constant time foliation $(\Sigma_{\tau})_{\tau\in[\tau_0,0)}$ is future (causally) complete and exhibits the following asymptotic behaviour:
\[(\fg,\bm{k},\phi^\prime,\nabla\phi)(\tau)\longrightarrow(\gamma,\frac13\gamma,0,0)\text{ as }\tau\uparrow 0\]
\end{theorem}

Since the control of geometric perturbations uses the same arguments as in \cite{AndFaj20}, the focus in this section will lie on dealing with the the scalar field. The key idea herein is controlling decay of the scalar field using an indefinite corrective term on top of the canonical energy (see Definition \ref{def:fut-stab}).

\subsection{Preliminaries}\label{subsec:fut-prelim}

\subsubsection{Notation, gauge and spatial reference geometry}

Within this section, we will (with slight abuse of notation) decompose the Lorentzian metric as follows:
\begin{subequations}
\begin{equation}\label{eq:fut-metric}
\g=-n^2dt^2+g_{ab}(dx^a+X^a)(dx^b+X^bdt)
\end{equation}
We impose CMCSH gauge (see \cite{AM03}) via
\begin{equation}\label{eq:CMCSH}
t=\tau,\,g^{ij}(\Gamma^{a}_{ij}-\Gamhat^{a}_{ij})=0\,
\end{equation}
\end{subequations}
where $\Gamhat$ refers to the Christoffel symbols with regards to the spatial reference metric $\gamma$.\\
For the extent of the future stability analysis, we have to make additional assumptions on the spatial geometry beyond Definition \ref{def:spatial-mf}:

\begin{definition}[Spectral assumption on the Laplacian of the spatial reference manifold]\label{def:spatial-mf-spectral}
Let $\mu_0(\gamma)$ to be the smallest positive eigenvalue of the Laplace operator $-\Lap_\gamma=(\gamma^{-1})^{ab}\nabla_a\nabla_b$ acting on scalar functions, where $(M,\gamma)$ is as in Definition \ref{def:spatial-mf}. $(M,\gamma)$ additionally is assumed to satisfy
\[\mu_0(\gamma)>\frac19\,.\]
\end{definition}

This spectral bound is strongly related to the geometry of the hyperboloid: The negative Laplacian on the three-dimensional hyperboloid (with constant sectional curvature $-1$) has eigenvalues precisely on $[1,\infty)$ (see \cite{Cornish99}), and once hence often attempts to study how wave numbers $k_{\lambda}=\sqrt{\lambda^2-1}$ are distributed when analysing the spectrum of (closed orientable) hyperbolic 3-manifolds instead of the eigenvalues $\lambda$ themselves, as in \cite{Cornish99}. In this nomenclature, Definition \ref{def:spatial-mf-spectral} means that there are no positive wavenumbers. We are not aware of any closed orientable hyperbolic 3-manifolds violating this constraint, and to the extent that numerical work as \cite{Cornish99} covers small wave numbers, there is also no numerical evidence for such manifolds.

Otherwise, we extend the notation from the Big Bang stability analysis regarding foliations, derivatives, indices and schematic term notation to this setting (see Section \ref{subsec:notation}). In particular, $\Sigma_{T}$ and $\Sigma_{\tau}$ will refer to spatial hypersurfaces along which the logarithmic time $T$ and the mean curvature $\tau$ are constant (see \eqref{eq:fut-time} on why these are interchangeable), and we will write for example $\Sigma_{T=0}$ when inserting a specific value to avoid potential ambiguity. We use similar notation for scalar functions and tensors that depend on $T$ (respectively $\tau$).\\

\subsubsection{Rescaled variables and Einstein equations}

We will use the standard rescaling of the solution variables by $\tau$:

\begin{definition}[Rescaled variables for future stability]\label{def:fut-rescaled}
\begin{subequations}
\begin{gather}
\fg_{ij}=\tau^2g_{ij}\,,(\fg^{-1})^{ij}=\tau^{-2}g^{ij}\,,\ \fk_{ij}=\tau \hat{k}_{ij}\label{eq:fut-resc-metric}\\
\fn=\tau^2 n\,,\ \fN=\frac{\fn}3-1\,,\ \fX^a=\tau X^a\label{eq:fut-resc-gauge}
\end{gather}
Furthermore, we introduce the logarithmic time 
\begin{equation}\label{eq:fut-time}
T=-\log\left(\frac{\tau}{\tau_0}\right)\,\Leftrightarrow\,\tau=\tau_0e^{-T}
\end{equation}
which satisfies $\del_T=-\tau\del_\tau$. Toward the future, $\tau$ increases from $\tau_0$ toward $0$, and thus $T$ increases from $0$ to $\infty$. We additionally introduce:
\begin{gather}
\fdel=\del_T+\Lie_{\fX}=-\tau(\del_\tau-\Lie_X)\label{eq:fut-del0}\\
\phi^\prime=\fn^{-1}\fdel\phi=n^{-1}(-\tau)^{-1}(\del_\tau-\Lie_X)\phi\label{eq:fut-delphi}
\end{gather}
Moreover, for any scalar function $\zeta$, we denote by $\overline{\zeta}$ the mean integral with regards to $(\Sigma_{T},\fg_T)$.
\end{subequations}
\end{definition}

\noindent For symmetric $(0,2)$-tensors $h$, we define the operator
\begin{equation}\label{eq:LG}
\LG h_{ab}=-\frac1{\mu_{\fg}}\nabhat_k\left((\fg^{-1})^{kl}\mu_{\fg}\nabhat_lh_{ab}\right)-2\Riem[\gamma]_{akbl}(\fg^{-1})^{kk^\prime}(\fg^{-1})^{ll^\prime}h_{k^\prime l^\prime}
\end{equation}
Under our assumptions on the reference geometry, \cite{Kroen15} implies that the smallest positive eigenvalue of $\mathcal{L}_{\gamma,\gamma}$, denoted by $\lambda_0$, satisfies $\lambda_0\geq\frac19$, and that $\mathcal{L}_{\gamma,\gamma}$ has trivial kernel. \\

Before collecting the CMCSH Einstein scalar-field equations, we compute some rescaled matter quantities:

\begin{lemma}\label{lem:fut-matter-resc}[Rescaled matter components] The rescaled matter quantities as in \cite[(2.22)]{AndFaj20} take the following form in the Einstein scalar-field system:
\begin{subequations}
\begin{align}
\rho=&\,8\pi(-\tau)^{-3}n^2T^{00}=2\pi\tau^{-1}\left[\lvert\phi^\prime\rvert^2+\lvert\nabla\phi\rvert_{\fg}^2\right]\label{eq:fut-rho}\\
\underline{\eta}=&\,4\pi(-\tau)^{-5}T=6\pi\tau^{-3}\lvert\phi^\prime\rvert^2-2\pi\tau^{-3}\lvert\nabla\phi\rvert_{\fg}^2\label{eq:fut-eta}\\
\jmath^a=&\,8\pi(-\tau)^{-5}n{T^{0a}}=8\pi\tau\phi^\prime\cdot (\fg^{-1})^{ac}\nabla_c\phi\label{eq:fut-j}\\
S_{ab}=&\,8\pi(-\tau)^{-1}\left[T_{ab}-\frac12g_{ab}{T^{\mu}}_{\mu}\right]=8\pi(-\tau)^{-1}\nabla_a\phi\nabla_b\phi\label{eq:fut-S}
\end{align}
\end{subequations}
\end{lemma}

With this notation and rewriting the wave equation, we can carry over the CMCSH system from \cite[(2.13)-(2.18)]{AndFaj20}:

\begin{lemma}[Rescaled CMCSH equations] The rescaled CMCSH Einstein scalar-field equations take the following form:
\begin{subequations}
The constraint equations
\begin{equation}\label{eq:fut-constr}
R[\fg]-\lvert\fk\rvert_{\fg}^2-\frac23=4\tau\rho,\ \div_{\fg}\fk_a=2\tau^2\fg_{ab}\jmath^b,
\end{equation}
the elliptic lapse and shift equations
\begin{align*}
\left(\fLap-\frac13\right)\fn=&\,\fn\left(\lvert\fk\rvert_{\fg}^2+\tau\eta\right)-1,\label{eq:fut-lapse-eq}\numberthis\\
\fLap \fX^a+(\fg^{-1})^{ab}\Ric[\fg]_{bm}\fX^m=&\,2(\fg^{-1})^{am}(\fg^{-1})^{bn}\nabla_b\fn\cdot\fk_{mn}-(\fg^{-1})^{ab}\nabla_b\fN+2\fn\tau^2\jmath^a\numberthis\label{eq:fut-shift-eq}\\
&\,-2(\fg^{-1})^{bk}((\fg^{-1})^{cl}\fn\cdot\fk_{bc}-\nabla_b \fX^l)(\Gamma^a_{kl}-\Gamhat^a_{kl})\,,
\end{align*}
the geometric evolution equations
\begin{align*}
\fdel\fg_{ab}=&\,2\fn\fk_{ab}+2\fN\fg_{ab}\,\numberthis\label{eq:fut-eq-g}\,,\\
\fdel(\fg^{-1})^{ab}=&\,-2\fn(\fg^{-1})^{ac}(\fg^{-1})^{bd}\fk_{cd}-2\fN(\fg^{-1})^{ab}\numberthis\label{eq:fut-eq-g-1}\,,\\
\fdel\fk_{ab}=&\,-2\fk_{ab}-\fn\left(\Ric[\fg]_{ab}+\frac29\fg_{ab}\right)+\nabla_a\nabla_n\fn\numberthis\label{eq:fut-eq-Sigma}\\
&\,+2\fn\cdot(\fg^{-1})^{mn}\fk_{am}\fk_{bn}-\frac13\fN\fg_{ab}-\fN\Sigma_{ab}+N\tau S_{ab}
\end{align*}
and the wave equation
\begin{equation}\label{eq:fut-wave}
\fdel\phi^\prime=\langle \nabla\fn,\nabla\phi\rangle_{\fg}+\fn\fLap\phi+(1-\fn)\phi^\prime\,.
\end{equation}
\end{subequations}
\end{lemma}

\subsubsection{Energies and data assumptions}

The proof will rely on the following corrected energy quantities:

\begin{definition}[Energies for future stability]\label{def:fut-stab}
\begin{subequations}
\begin{align*}
\fE^{(l)}=&\,(-1)^l\int_{M}\left[\phi^\prime\fLap^l\phi^\prime-\phi\fLap^{l+1}\phi\right]\vol{\fg}\numberthis\\
\fC^{(l)}=&\,(-1)^l\int_{M}(\phi-\phim)\fLap^l\phi^\prime\,\vol{\fg}\numberthis\\
E_{SF}=&\,\sum_{m=0}^4 \left(\fE^{(m)}+\frac23\fC^{(m)}\right)\numberthis\\
\fEg=&\,\sum_{m=1}^5\biggr(\frac92\int_{M}\langle\fg-\gamma,\LG^m(\fg-\gamma)\rangle_{\fg}\vol{\fg}+\frac12\int_{M}\langle6\fk,\LG^{m-1}(6\fk)\rangle_{\fg}\vol{\fg}\numberthis\\
&\,\phantom{\sum_{m=1}^l}+c_E\int_{M}\langle 6\fk,\LG^{m-1}(\fg-\gamma)\rangle_{\fg}\vol{\fg}\biggr)
\end{align*}
\end{subequations}
The constant $c_E$ is given by
\begin{equation}\label{eq:fut-corr-const}
c_E=\begin{cases}
1 & \lambda_0>\frac19 \\
9(\lambda_0-\epsilonnew^\prime) & \lambda_0=\frac19\,,
\end{cases}
\end{equation}
where $\epsilonnew^\prime>0$ is chosen to be small enough within the argument.
\end{definition}

The Sobolev norms $H_{\fg}^l$ and $C_{\fg}^l$ are defined analogously to Definition \ref{def:sob-norms} and \ref{def:sup-norms}, with similar conventions on surpressing time dependence in notation whereever possible. Since norms with respect to $\fg$ and $\gamma$ are equivalent under the bootstrap assumption (and consequently throughout the entire argument), we will simply denote the norms by $H^l$ and $C^l$ throughout unless the specific metric is crucial.

\begin{assumption}[Initial data assumption]\label{ass:fut-init} The initial data on the spatial hypersurface $\Sigma_{T=0}$ is assumed to be small in the following sense:
\begin{align*}
\numberthis\label{eq:fut-init}\|\fg-\gamma\|_{C^4}+\|\fk\|_{C^3}+\|\fN\|_{C^3}+\|\fX\|_{C^3}+\|\phi^\prime\|_{C^3}+\|\nabla\phi\|_{C^3}&\\
+\|\fg-\gamma\|_{H^6}+\|\fk\|_{H^5}+\|\fN\|_{H^5}+\|\fX\|_{H^5}+\|\phi^\prime\|_{H^5}+\|\nabla\phi\|_{H^5}&\,\leq\epsilonnew^2
\end{align*}
\end{assumption}


\begin{remark}[Local well-posedness toward the future]\label{rem:fut-lwp}
Under the above initial data assumption, local well-posedness is satisfied by analogizing the arguments for local well-posedness in the vacuum setting (see \cite{AM03B}) with the matter coupling added. Since this only consists of adding another wave equation to the hyperbolic system, the argument is structurally unchanged given appropriate smallness assumptions on $\phi^\prime$ and $\nabla\phi$ (where $\phi$ itself does not enter into the Einstein system). Comparing to the regularity of solutions in \cite[Theorem 3.1]{AM03B}, above assumptions imply that $E_{geom},\,\E^{(l)}_{SF}$ and $\fC^{(l)}$ initially are continuously differentiable (in time) for any $l\leq 4$. Beside obtaining this existence result, we will only need smallness of initial data of up to one order less to prove our stability result.
\end{remark}

\begin{assumption}[Bootstrap assumption]\label{ass:fut-bootstrap} We assume that, on the bootstrap interval $T\in[0,T_{Boot})$,
\begin{subequations}
\begin{align*}
\label{eq:fut-bootstrap}\numberthis\|\fg-\gamma\|_{C^3}+\|\fk\|_{C^2}+\|\fN\|_{C^4}+\|\fX\|_{C^4}+\|\phi^\prime\|_{C^3}+\|\nabla\phi\|_{C^2}&\\
+\|\fg-\gamma\|_{H^5}+\|\fk\|_{H^4}+\|\fN\|_{H^6}+\|\fX\|_{H^6}+\|\phi^\prime\|_{H^4}+\|\nabla\phi\|_{H^4}&\,\leq\epsilonnew e^{-\frac{T}2}
\end{align*}
as well as
\begin{equation}
\label{eq:fut-bootstrap-phi-mean}\|\phi\|_{C^0(\Sigma_T)}\leq \|\phi\|_{C^0(\Sigma_{T=0})}+1\,.
\end{equation}
\end{subequations}
\end{assumption}

\noindent We only choose not to use \enquote{$\lesssim$}-notation in the above assumptions for notational convenience in some technical computations. As before, $\epsilonnew$ can be chosen to have been sufficiently small for the following estimates to hold and for the decay estimates we derive from the bootstrap assumptions to be strict improvements. Moreover, note that \eqref{eq:fut-bootstrap} is satisfied since all of the norms are continuous in time (see Remark \ref{rem:fut-lwp}), and \eqref{eq:fut-bootstrap-phi-mean} is satisfied local-in-time since the spatial hypersurfaces are compact and $\phi$ is continuous.\\

Before moving on to the energy estimates, we quickly collect the following immediate consequence of the bootstrap assumptions:

\begin{lemma}[Sobolev estimate for the curvature]\label{lem:fut-Ric-est}
The following estimate holds for any $l\in\N_0$:
\begin{subequations}
\begin{equation}\label{eq:fut-Ric-est}
\left\|\Ric[\fg]+\frac29\fg\right\|_{H^l}\lesssim \|\fg-\gamma\|_{H^{l+2}}+\|\fg-\gamma\|_{H^{l+1}}^2
\end{equation}
Under the bootstrap assumptions, this implies
\begin{equation}\label{eq:fut-Ric-bs}
\left\|\Ric[\fg]+\frac29\fg\right\|_{C^1}+\left\|\Ric[\fg]+\frac29\fg\right\|_{H^3}\lesssim \epsilonnew e^{-\frac{T}2}
\end{equation}
\end{subequations}
\end{lemma}

\begin{proof}
By \cite[Pf. of Theorem 3.1]{AM03B}, one has
\begin{align*}
\left\|\Ric[\fg]+\frac29\fg\right\|_{H^l}\leq&\,\frac12\|\LG(\fg-\gamma)\|_{H^l}+K\|\fg-\gamma\|_{H^{l+1}}^2\,
\end{align*}
for some suitably large $K>0$ and that $\LG$ is elliptic. This implies the first inequality, while the latter follows from directly from the bootstrap assumption \eqref{eq:fut-bootstrap} and by applying the standard Sobolev embedding.
\end{proof}

\subsection{Elliptic estimates}\label{subsec:fut-ell-est}

\begin{lemma}[Elliptic estimates for lapse and shift]\label{lem:fut-ell-est}
Let $l\in\{3,4,5,6\}$. Then, one has $\fn\in(0,3)$ (thus $\fN\in(-1,0)$) and the following estimates hold:
\begin{subequations}
\begin{align*}
\numberthis\label{eq:ell-est-lapse}\|\fN\|_{H^l}\lesssim&\,\|\fk\|_{H^{l-2}}^2+\|\nabla\phi\|_{C^2}^2\|\fg-\gamma\|_{H^{l-2}}\\
&\,+\left[\|\nabla\phi\|_{C^2}\left(1+\|\fg-\gamma\|_{C^2}\right)+\|\phi^\prime\|_{C^2}\right]\left[\|\phi^\prime\|_{H^{l-2}}+\|\nabla\phi\|_{H^{l-2}}\right]\\
\numberthis\label{eq:ell-est-shift}\|\fX\|_{H^l}\lesssim&\,\|\fk\|_{H^{l-2}}^2+\|\fg-\gamma\|_{H^{l-1}}^2+\|\nabla\phi\|_{C^1}\|\fg-\gamma\|_{H^{l-3}}\\
&\,+\left[\|\nabla\phi\|_{C^2}^2\left(1+\|\fg-\gamma\|_{C^2})+\|\phi^\prime\|_{C^2}\right)\right]\left[1+\|\fN\|_{C^2}\right]\left[\|\phi^\prime\|_{H^{l-2}}+\|\nabla\phi\|_{H^{l-2}}\right]
\end{align*} 
\end{subequations}
\end{lemma}
\begin{proof}
Applying \eqref{eq:fut-eta}, the lapse equation \eqref{eq:fut-lapse-eq} reads
\[\left(\fLap-\frac13\right)\fn=\fn\left(\lvert\fk\rvert_{\fg}^2+8\pi\lvert\phi^\prime\rvert^2\right)-1\,.\]
The pointwise bounds for $\fn$ now follow as in Lemma \ref{lem:lapse-maxmin} from the maximum principle.\\
From \cite[Proposition 17]{AndFaj20}, we take the following estimates:
\begin{align*}
\|\fN\|_{H^l}\lesssim&\,\|\fk\|_{H^{l-2}}^2+\lvert\tau\rvert\|\rho\|_{H^{l-2}}+\tau^3\|\underline{\eta}\|_{H^{l-2}}\\
\|\fX\|_{H^l}\lesssim&\,\|\fk\|_{H^{l-2}}^2+\|\fg-\gamma\|_{H^{l-1}}^2++\lvert\tau\rvert\|\rho\|_{H^{l-2}}+\tau^3\|\underline{\eta}\|_{H^{l-2}}+\|\fn\jmath\|_{H^{l-2}}
\end{align*}
The lapse estimate then follows directly by inserting the expressions from Lemma \ref{lem:fut-matter-resc}, and the shift estimate from inserting these as well as \eqref{eq:ell-est-lapse} and $\fn\in(0,3)$.
\end{proof}

%\begin{corollary}[Improved bounds for lapse and shift]
%\begin{equation*}
%\|\fN\|_{C^4}+\|\fX\|_{C^4}+\|\fN\|_{H^6}+\|\fX\|_{H^6}\lesssim \epsilonnew^2
%\end{equation*}
%\end{corollary}
%\begin{proof}
%\todo{Insert bootstrap assumptions}
%\end{proof}



\subsection{Scalar field energy estimates}\label{subsec:fut-ESF}

\subsubsection{Near-coercivity of $E_{SF}$}

We will only be able to prove a decay estimate for the corrected energy $E_{SF}$ via a Gronwall argument. Hence, we first need to verify that this energy controls the solution norms, for which we first show that it controls the \enquote{canonical} scalar field energies:

\begin{lemma}[Positivity of corrected scalar field energies]\label{lem:fut-ESF-coercivity}
Let
\[Q=\frac{\sqrt{1+9q}-1}{\sqrt{1+9q}}\ \text{with}\ q=\frac12\left(\mu_0(\gamma)-\frac19\right)\,.\]
Then, for any $l\in\{0,1,2,3,4\}$ and $\epsilonnew>0$ small enough, one has
\begin{equation}\label{eq:fut-ESF-coercivity}
Q\fE^{(l)}\leq \fE^{(l)}+\frac23\fC^{(l)}
\end{equation}
and consequently
\begin{equation}
Q\sum_{m=0}^4\fE^{(l)}\leq E_{SF}
\end{equation}
\end{lemma}
\begin{proof}
We denote the smallest positive eigenvalue of $\fLap$ acting on scalar functions on $\Sigma_T$ as $\mu_0(\fg(t))$. By the bootstrap assumption \eqref{eq:fut-bootstrap} and since $\mu_0$ depends continuously on the metric, we obtain the following for small enough $\epsilonnew>0$:
\begin{equation*}
\mu_0(\fg_T)\geq \mu_0(\gamma)-\frac12\left(\mu_0(\gamma)-\frac19\right)\geq \frac19+q
\end{equation*}
By the Poincaré inequality applied on $(\Sigma_T,\fg_T)$ (see \cite[p.1037]{CBM01}), above spectral bound implies the following for any $\zeta\in H^1(\Sigma_T)$:
\begin{equation}\label{eq:adapted-poincare}
\|\zeta-\overline{\zeta}\|_{L^2_{\fg}(\Sigma_T)}^2\leq \mu_0(\fg_T)^{-1}\|\nabla\zeta\|_{L^2_{\fg}(\Sigma_T)}^2\leq \left(\frac19+q\right)^{-1}\|\nabla\zeta\|_{L^2_{\fg}(\Sigma_T)}^2
\end{equation}
For $l=0$, this means
\begin{align*}
\fE^{(0)}+\frac23\fC^{(0)}\geq&\,\|\phi^\prime\|_{L^2_{\fg}}^2+\|\nabla\phi\|_{L^2_{\fg}}^2-\frac23\|\phi-\phim\|_{L^2_{\fg}}\|\phi^\prime\|_{L^2_{\fg}}\\
\geq&\,\|\phi^\prime\|_{L^2_{\fg}}^2+\|\nabla\phi\|_{L^2_{\fg}}^2-2\left(1+9q\right)^{-\frac12}\|\nabla\phi\|_{L^2_{\fg}}\|\phi^\prime\|_{L^2_{\fg}}\\
\geq&\, \frac{\sqrt{1+9q}-1}{\sqrt{1+9q}}\fE^{(0)}\,.
\end{align*}
For $l=1$, notice that we can rewrite $\fC^{(1)}$ as follows:
\[\fC^{(1)}=\int_{M}\langle\nabla\phi,\nabla\phi^\prime\rangle_{\fg}\,\vol{\fg}=\int_{M}\left\langle\nabla\phi,\nabla\left(\phi^\prime-\overline{\phi^\prime}\right)\right\rangle_{\fg}\,\vol{\fg}=-\int_{M}\left(\phi^\prime-\overline{\phi^\prime}\right)\Lap_{\fg}\phi\,\vol{\fg}\]
Hence, applying \eqref{eq:adapted-poincare} to $\zeta=\phi^\prime$ yields
\[\fE^{(1)}+\frac23\fC^{(1)}\geq \fE^{(1)}-2\left(1+9q\right)^{-\frac12}\|\nabla\phi^\prime\|_{L^2_{\fg}}\|\Lap_{\fg}\phi\|_{L^2_{\fg}}\geq \frac{\sqrt{1+9q}-1}{\sqrt{1+9q}}\fE^{(1)}\,.\]
For $l=2$, notice $\overline{\fLap\phi}=0$ holds due to the divergence theorem, hence we have
\begin{align*}
\fE^{(2)}+\frac23\fC^{(2)}\geq&\,\|\fLap\phi^\prime\|_{L^2_{\fg}}^2+\|\nabla\fLap\phi\|_{L^2_{\fg}}^2-\frac23\|\fLap\phi\|_{L^2_{\fg}}\|\fLap\phi^\prime\|_{L^2_{\fg}}\\
\geq&\,\|\fLap\phi^\prime\|_{L^2_{\fg}}^2+\|\nabla\fLap\phi\|_{L^2_{\fg}}^2-2\left(1+9q\right)^{-\frac12}\|\nabla\fLap\phi\|_{L^2_{\fg}}\|\fLap\phi^\prime\|_{L^2_{\fg}}\\
\geq&\,\frac{\sqrt{1+9q}-1}{\sqrt{1+9q}}\fE^{(2)}
\end{align*}
For $l=3$, we have $\overline{\fLap\phi^\prime}=0$ and hence \eqref{eq:fut-ESF-coercivity} follows as for $l=1$ by applying \eqref{eq:adapted-poincare} to $\zeta=\fLap\phi^\prime$, and for $l=4$, it follows identically to $l=2$ with $\overline{\fLap^2\phi}=0$. 
\end{proof}

\begin{lemma}[Near-coercivity of corrected scalar field energy]\label{lem:fut-Sob-est}
For any scalar function $\zeta$ and $k\in\{1,2\}$, one has the following under the bootstrap assumptions:
\begin{align*}
\int_M\lvert\nabla^2\zeta\rvert_{\fg}^2\,\vol{\fg}\lesssim&\,\int_M\lvert\fLap\zeta\rvert_{\fg}^2+\lvert\nabla\zeta\rvert_{\fg}^2\,\vol{\fg}\\
\|\zeta\|_{\dot{H}^{2k}}^2\lesssim&\,\|\fLap^{k}\zeta\|_{L^2}^2+\left(\|\zeta\|_{\dot{H}^{2k-1}}^2+\|\zeta\|_{\dot{H}^{2k-2}}^2\right)+\|\nabla\zeta\|_{C^{1}}^2\left\|\Ric[\fg]+\frac29\fg\right\|_{H^{2k-2}}^2\\
\|\nabla\zeta\|_{\dot{H}^{2k}}^2\lesssim&\,\|\nabla\fLap^{k}\zeta\|_{L^2}^2+\left(\|\nabla\zeta\|_{\dot{H}^{2k-1}}^2+\|\nabla\zeta\|_{\dot{H}^{2k-2}}^2\right)+\|\nabla\zeta\|_{C^{2}}^2\left\|\Ric[\fg]+\frac29\fg\right\|_{H^{2k-2}}^2
\end{align*}
Consequently, the following estimate holds:
\begin{align*}
\numberthis\label{eq:fut-coerc}\|\phi^\prime\|_{H^4}^2+\|\nabla\phi\|_{H^4}^2\lesssim&\, \sum_{l=0}^4\fE^{(l)}+\left(\|\phi^\prime\|_{C^2}^2+\|\nabla\phi\|_{C^2}^2\right)\cdot\left\|\Ric[\fg]+\frac29\fg\right\|_{H^2}^2\\
\lesssim&\,E_{SF}^{(4)}+\left(\|\phi^\prime\|_{C^2}^2+\|\nabla\phi\|_{C^2}^2\right)\left\|\Ric[\fg]+\frac29\fg\right\|_{H^2}^2
\end{align*}
\end{lemma}
\begin{proof}
The inequalities for $\zeta$ follows from the same arguments as Lemma \ref{lem:Sobolev-norm-equivalence-improved}, except that we have $\|\Ric[\fg]\|_{C^1_{\fg}}\lesssim 1+\epsilonnew\lesssim 1$ by Lemma \ref{lem:fut-Ric-est}.  The final estimate then follows by applying these estimates to $\zeta=\phi^\prime$ and $\zeta=\phi$.
\end{proof}


\subsubsection{Preparations for energy estimates}

Before proving the energy estimate, we need to establish two technical lemmas: First, we collect the following formula to differentiate integrals:
\begin{lemma}[Differentiation of integrals, future stability version] For any diffentiable function $\zeta$, one has
\begin{equation}\label{eq:fut-delt-int}
\del_T\int_M\zeta\vol{\fg}=\int_M\left(\fdel\zeta+3\fN\zeta\right)\,\vol{\fg}\,.
\end{equation}
\end{lemma}
\begin{proof}
As in the proof of \eqref{eq:delt-int}, we obtain
\begin{align*}
\del_T\int_M\zeta\vol{\fg}=&\,\int_M\del_T\zeta+\frac{\del_T\mu_{\fg}}{\mu_{\fg}}\zeta\,\vol{\fg}\\
=&\,\int_M\del_T\zeta+3\fN\zeta-\frac12(\fg^{-1})^{ab}\Lie_{\fX}\fg_{ab}\zeta\,\vol{\fg}\\
=&\,\int_M\del_T\zeta+3\fN\zeta-\div_{\fg}\fX\cdot\zeta\,\vol{\fg}
\end{align*}
The statement now follows by applying Stokes' theorem to the final term and rearranging.
\end{proof}

Furthermore, the following is needed to deal with the mean value of $\phi$ in the base level correction term:

\begin{lemma}[Decay estimate for the integrated time derivative]\label{lem:fut-SF-technicality} For any $T>0$, we have
\begin{equation}\label{eq:fut-SF-technicality-1}
\int_{\Sigma_T}\phi^\prime\,\vol{\fg}=\left(\int_{\Sigma_{T=0}}\phi^\prime\,\vol{\fg}\right)\cdot e^{-2T}\,.
\end{equation}
Consequently, the bootstrap assumptions imply
\begin{equation}\label{eq:fut-SF-technicality-2}
\left\lvert\int_{\Sigma_T}\fdel\phim\cdot\phi^\prime\,\vol{\fg}\right\rvert\lesssim\,\epsilonnew^3e^{-2T}
\end{equation}
for $\delta>0$ small enough.
\end{lemma}
\begin{proof}
Using that the integral of $\div_{\fg}(\fn\nabla\phi)$ vanishes, we compute:
\[\del_T\left(\int_M\phi^\prime\,\vol{\fg}\right)=\int_{M}\left(\fdel\phi^\prime+3\fN\phi^\prime\right)\,\vol{\fg}=\int_M\left[(1-\fn)\phi^\prime+(\fn-3)\phi^\prime\right]\,\vol{\fg}=-2\left(\int_M\phi^\prime\,\vol{\fg}\right)\]
Hence, \eqref{eq:fut-SF-technicality-1} precisely describes the solution to this ODE ($f^\prime=-2f$) with initial value at $T=0$ inserted, and the initial data assumption implies
\[\left\lvert\int_M\phi^\prime\,\vol{\fg}\right\rvert\leq \|\phi^\prime\|_{C^0(\Sigma_{T=0})}\vol{\fg}(\Sigma_{T=0})e^{-2T}\leq \epsilonnew^2e^{-2T}\,.\]
Furthermore, one has by \eqref{eq:fut-delt-int} and the bootstrap assumption that
\begin{equation}\label{eq:fut-volume-evol}
\del_T\vol{\fg}\left(\Sigma_{T}\right)=\int_{\Sigma_T}3\fN\vol{\fg}\in\left[-3\epsilonnew\cdot \vol{\fg}\left(\Sigma_{T}\right),0\right)\,
\end{equation}
with the upper bound by the pointwise estimate in Lemma \ref{lem:fut-ell-est}. Hence, $\vol{\fg}(\Sigma_T)$ is decreasing and one has
\[\left\lvert\frac{\del_T\vol{\fg}\left(\Sigma_{T}\right)}{\vol{\fg}\left(\Sigma_{T}\right)}\right\rvert\leq 3\epsilonnew\,.\]
Since, by the initial data assumption, the volume forms of $\gamma$ and $\fg_{T=0}$ differ by a small error (as in Remark \ref{eq:rem-vol-form}), \eqref{eq:fut-volume-evol} also implies
\[\vol{\fg}\left(\Sigma_{T}\right)\gtrsim e^{-3\epsilonnew T}\,.\]
Altogether and using $\fdel\phi=\fn\phi^\prime=(3+3\fN)\phi^\prime$, we obtain:
\begin{align*}
\left\lvert\int_M\fdel\phim\cdot\phi^\prime\,\vol{\fg}\right\rvert=&\,\left\lvert\left[-\frac{\del_T\vol{\fg}\left(\Sigma_{T}\right)}{\vol{\fg}\left(\Sigma_{T}\right)}\cdot\phim+\frac1{\vol{\fg}\left(\Sigma_{T}\right)}\int_M\left(\fdel\phi+3\fN\phi\right)\,\vol{\fg}\right]\cdot\int_M\phi^\prime\,\vol{\fg}\right\rvert\\
\lesssim&\,\left[\epsilonnew\phim+e^{3\epsilonnew T}\left\lvert\int_M \phi^\prime\,\vol{\fg}\right\rvert+\frac1{\vol{\fg}(\Sigma_T)}\left\lvert\int_M\fN(\phi^\prime+\phi)\,\vol{\fg}\right\rvert\right]\cdot\left\lvert\int_M\phi^\prime\,\vol{\fg}\right\rvert\\
\lesssim&\,\left[\epsilonnew\left\lvert\phim\right\rvert+e^{3\epsilonnew T}\cdot \epsilonnew^2e^{-2T}+\|\fN\|_{L^\infty}\left(\left\lvert\overline{\phi^\prime}\right\rvert+\left\lvert\phim\right\rvert\right)\right]\cdot \epsilonnew^2e^{-2T}
\end{align*}
The bootstrap assumptions \eqref{eq:fut-bootstrap}-\eqref{eq:fut-bootstrap-phi-mean} imply
\[\left\lvert\phim\right\rvert+\left\lvert\overline{\phi^\prime}\right\rvert\leq \|\phi\|_{C^0(\Sigma_T)} + \|\phi^\prime\|_{C^0(\Sigma_T)}\leq (1+\|\phi\|_{C^0(\Sigma_{T=0})})+ \epsilonnew e^{-\frac{T}2}\lesssim 1\,,\]
and this yields the statement for $\epsilonnew\leq\frac23$ along with \eqref{eq:fut-bootstrap}.
\end{proof}

\subsubsection{Energy estimates}

Now, we can collect the following estimates for the corrected scalar field energies:

\begin{lemma}[Base level estimate for the corrected scalar field energy]\label{lem:fut-en-est-ESF0} Under the bootstrap assumptions, the following estimate holds:
\begin{equation}
\del_TE_{SF}^{(0)}\leq -2E_{SF}^{(0)}+K\epsilonnew^3 e^{-\frac32T}
\end{equation}
\end{lemma}
\begin{proof}
We compute, using $[\fdel,\nabla]\phi=0$,$\fdel\phi=\fn\phi^\prime$ and the rescaled wave equation \eqref{eq:fut-wave}:
\begin{align*}
\del_T\fE^{(0)}=&\,\int_M \left[2\fdel\phi^\prime\cdot\phi^\prime+2\left\langle\nabla\phi,\nabla\fdel\phi\right\rangle_{\fg}+\left(\fdel\fg^{-1}\right)^{ab}\nabla_a\phi\nabla_b\phi+3\fN\left(\lvert\phi^\prime\rvert^2+\lvert\nabla\phi\rvert_{\fg}^2\right)\right]\,\vol{\fg}\\
=&\,\int_M \biggr[2\left(\langle\nabla\fn,\nabla\phi\rangle_{\fg}+\fn\fLap\phi+(1-\fn)\phi^\prime\right)\phi^\prime-2(\fn\phi^\prime)\cdot\fLap\phi\\
&\,\,\phantom{\int_M}-2\fn\langle\fk,\nabla\phi\nabla\phi\rangle_{\fg}+3\fN\lvert\phi^\prime\rvert^2+\fN\lvert\nabla\phi\rvert_{\fg}^2\biggr]\,\vol{\fg}
\end{align*}
With $2(1-\fn)=-4-6\fN$, integration by parts and using the bootstrap assumption \eqref{eq:fut-bootstrap} on $C$-norms, we get for some constant $K>0$ that we update from line to line:
\begin{align*}
\del_T\fE^{(0)}\leq&\,\int_M -4\lvert\phi^\prime\rvert_{\fg}^2\,\vol{\fg}+K\left[\|\nabla\phi\|_{C^0}\|\fN\|_{H^1}\sqrt{\fE^{(0)}}+\left(\|\fk\|_{C^0}+\|\fN\|_{C^0}\right)\fE^{(0)}\right]\\
\leq&\int_M-4\lvert\phi^\prime\rvert^2\,\vol{\fg}+K\epsilonnew e^{-\frac{T}2}\left[\fE^{(0)}+\|\fN\|_{H^1}^2\right]
\end{align*}
Similarly and using the same evolution equations,
\begin{align*}
\del_T\fC^{(0)}=&\,\int_M \left[\fdel\phi\cdot\phi^\prime-\fdel\phim\cdot\phi^\prime+(\phi-\phim)\fdel\phi^\prime+3\fN(\phi-\phim)\phi^\prime\right]\,\vol{\fg}\\
=&\,\int_M\left[3\lvert\phi^\prime\rvert^2+3\fN\lvert\phi^\prime\rvert^2+\left(\phi-\phim\right)\cdot \div_{\fg}\left(\fn\nabla\phi\right)-2\left(\phi-\phim\right)\phi^\prime-\fdel\phim\cdot\phi^\prime\right]\,\vol{\fg}\\
\leq&\,-2\fC^{(0)}+\int_M3\left[\lvert\phi^\prime\rvert^2-3\lvert\nabla\phi\rvert_{\fg}^2\right]\,\vol{\fg}+3\|\fN\|_{C^0}\fE^{(0)}-\int_M\left(\fdel\phim\cdot\phi^\prime\right)\,\vol{\fg}\,.
\end{align*}
Applying Lemma \ref{lem:fut-SF-technicality} to the last term, we get:
\begin{align*}
\del_T\fC^{(0)}\leq&-2\fC^{(0)}+K\epsilonnew e^{-\frac{T}2}\fE^{(0)}+K\epsilonnew^3e^{-2T}
\end{align*}
Combining these two estimates, inserting \eqref{eq:fut-bootstrap} (which in particular implies $\fE^{(0)}\lesssim\delta^2 e^{-T}$) yields:
\begin{align*}
\del_TE_{SF}^{(0)}=&\,\del_T\fE^{(0)}+\frac23\del_T\fC^{(0)}\\
=&\,\int_M \left[\left(-4+\frac23\cdot 3\right)\lvert\phi^\prime\rvert^2-\frac23\cdot 3\lvert\nabla\phi\rvert_{\fg}^2\right]\,\vol{\fg}-2\cdot\frac23\fC^{0}\\
&\,+K\epsilonnew e^{-\frac{T}2}\left[\fE^{(0)}+\|\fN\|_{H^1}^2\right]+K\delta^3 e^{-2T}\\
\leq&\,-2E_{SF}^{(0)}+K\epsilonnew^3 e^{-\frac32T}
\end{align*}
\end{proof}

\begin{lemma}[Higher order estimates for the corrected scalar field energy]\label{lem:fut-en-est-ESF}
For any $l\in\{1,\dots,4\}$, the following estimate holds:
\begin{align*}
\del_T\left(\fE^{(l)}+\frac23\fC^{(l)}\right)\leq&\,-2\left(\fE^{(l)}+\frac23\fC^{(l)}\right)+K\epsilonnew e^{-\frac{T}2}\left(\sum_{m=0}^l\sqrt{\fE^{(m)}}\right)\cdot\\
&\,\qquad\cdot\left(\|\phi^\prime\|_{H^{l}}+\|\nabla\phi\|_{H^{l}}+\|\Sigma\|_{H^{l}}+\epsilonnew e^{-\frac{T}2}\left\|\Ric[\fg]+\frac29\fg\right\|_{H^{l-1}}\right)
\end{align*}
\end{lemma}
\begin{proof}
Starting with $l=2k,\,k\in\{1,2\}$, one calculates:
\begin{subequations}
\begin{align}
\del_T\fE^{(2k)}=&\,\int_M\biggr[2\fLap^k\fdel\phi^\prime\cdot\fLap^k\phi^\prime+2\langle\nabla\fLap^{k}\phi,\nabla\fLap^k\fdel\phi\rangle_{\fg}\label{eq:fESF-1}\\
&\,\phantom{\int_M}+(\fdel {\fg}^{-1})^{ab}\cdot\nabla_a\fLap^k\phi\cdot\nabla_b\fLap^k\phi+3\fN\left(\lvert\Lap^k\phi^\prime\rvert_{\fg}^2+\lvert\nabla\Lap^k\phi\rvert_{\fg}^2\right)\label{eq:fESF-2}\\
&\,\phantom{\int_M}+2[\fdel,\fLap^k]\phi^\prime\cdot\fLap^k\phi^\prime+2\left\langle[\fdel,\nabla\fLap^k]\phi,\nabla\fLap^k\phi\right\rangle_{\fg}\biggr]\,\vol{\fg}\label{eq:fESF-3}
\end{align}
We insert the rescaled wave equation \eqref{eq:fut-wave} and $\fdel\phi=\fn\phi^\prime$ into the right hand side of \eqref{eq:fESF-1} and obtain for some constant $K>0$ (using $k\leq 2$):
\begin{align*}
\eqref{eq:fESF-1}\leq &\,\int_M \biggr[-4\lvert\fLap^k\phi^\prime\rvert^2 -6\fN\lvert\fLap^k\phi^\prime\rvert^2+\fn\fLap^{k+1}\phi\cdot\fLap^k\phi^\prime\biggr]\,\vol{\fg}\\
&\,+K\|\Lap^k\phi^\prime\|_{L^2}\left(\|\fN\|_{H^{2k+1}}\|\nabla\phi\|_{C^0}+\|\nabla\phi\|_{H^{2k}}\|\fN\|_{C^{2k}}\right)\\
&\,+\int_M\left[-\fn\fLap^{k}\phi^\prime\cdot\fLap^{k+1}\nabla\phi-3\langle\nabla\fN,\nabla\fLap^k\phi\rangle_ {\fg}\cdot\fLap^k\phi^\prime\right]\,\vol{\fg}\\
&\,+K\|\nabla\fLap^k\phi\|_{L^2}\left(\|\fN\|_{H^{2k+1}}\|\phi^\prime\|_{C^0}+\|\fN\|_{C^{2k}}\|\phi^\prime\|_{H^{2k}}\right)\\
\leq&\,\int_M -4\lvert\fLap^k\phi^\prime\rvert^2\,\vol{\fg}\\
&\,+K\sqrt{\fE^{(2k)}}\cdot\Bigr[\left(\|\nabla\phi\|_{C^0}+\|\phi^\prime\|_{C^0}\right)\cdot\|\nabla\phi\|_{H^{2k}}+\left(\|\fN\|_{H^{2k+1}}+\|\phi^\prime\|_{H^{2k}}\right)\cdot\|\fN\|_{C^{2k}}\Bigr]
\end{align*}
\end{subequations}
For \eqref{eq:fESF-2}, we use \eqref{eq:fut-eq-g-1} and the bootstrap assumption \eqref{eq:fut-bootstrap} to bound it by $K\epsilonnew e^{-\frac{T}2}\fE^{(2k)}$. Regarding \eqref{eq:fESF-3}, the commutator formulas \eqref{eq:[fdel,Lapk]}-\eqref{eq:[fdel,nablaLapk]} imply the following:%\footnote{Note that, for $k=1$, the Sobolev norms in $\Ric[\fg]+\frac29\fg$ are entirely redundant, while for $k=2$, higher order terms in the curvature carry at least one derivative, and $\nabla\Ric[\fg]=\nabla\left(\Ric[\fg]+\frac29\fg\right)$ holds. Hence, we can provide these estimates in norms with regards to $\fg$, and then norm equivalence between $H_\gamma$ and $H_{\fg}$ allows us to drop that specification of metric.}
\begin{align*}
\|[\fdel,\fLap^k]\phi^\prime\|_{L^2}\lesssim&\,\|\fn\|_{C^{2k-1}}\left(\|\phi^\prime\|_{C^1}\|\fk\|_{\dot{H}^{2k-1}}+\|\fk\|_{C^{2k-2}}\|\phi^\prime\|_{H^{2k}}\right)\\
&\,+\|\fX\|_{C^{2k-1}}\left(\left\|\Ric[\fg]+\frac29\fg\right\|_{H^{2k-2}}\|\phi^\prime\|_{C^{2k-2}}+\|\Ric[\fg]\|_{C^1}\|\phi^\prime\|_{H^{2k-1}}\right)\,,\\
\|[\fdel,\nabla\fLap^k]\phi\|_{L^2}\lesssim&\,\|\fn\|_{C^{2k}}\left(\|\nabla\phi\|_{C^1}\|\fk\|_{{H}^{2k}}+\|\fk\|_{C^{2k-2}}\|\nabla\phi\|_{H^{2k}}\right)\\
&\,+\|\fX\|_{C^{2k}}\left(\|\nabla\phi\|_{C^1}\left\|\Ric[\fg]+\frac29\fg\right\|_{H^{2k-1}}+\|\Ric[\fg]\|_{C^{1}}\|\phi^\prime\|_{H^{2k}}\right)
\end{align*}
Summarizing, inserting the $C$-norm bounds from the bootstrap assumption \eqref{eq:fut-bootstrap} and updating $K$, this implies
\begin{align*}
\del_T\fE^{(2k)}\leq&\,\int_M-4\lvert\fLap^k\phi^\prime\rvert^2\,\vol{\fg}+K\epsilonnew e^{-\frac{T}2}\fE^{(2k)}\\
&\,+K\epsilonnew e^{-\frac{T}2}\sqrt{\fE^{(2k)}}\left(\|\phi^\prime\|_{H^{2k}}+\|\nabla\phi\|_{H^{2k}}\right)\\
&\,+K\epsilonnew e^{-\frac{T}2}\sqrt{\fE^{(2k)}}\left(\|\fN\|_{H^{2k+1}}+\|\Sigma\|_{H^{2k}}+\epsilonnew e^{-\frac{T}2}\left\|\Ric[\fg]+\frac29\fg\right\|_{H^{2k-1}}\right)
\end{align*}
Moving on to the corrective term, we compute:
\begin{subequations}
\begin{align*}
\del_T\fC^{(2k)}=&\,\del_T\int_M\fLap^k\phi\cdot\fLap^k\phi^\prime\,\vol{\fg}\\
&\,\int_M\biggr[\fLap^k\fdel\phi\cdot\fLap^k\phi^\prime+\fLap^k\phi\cdot\fLap^k\fdel\phi^\prime\numberthis\label{eq:fESFcorr-1}\\
&\,\phantom{\int_M}+3\fN\cdot\fLap^k\phi\cdot\fLap^k\phi^\prime+[\fdel,\fLap^k]\phi\cdot\fLap\phi^\prime+\fLap^k\phi\cdot[\fdel,\fLap^k]\phi^\prime\biggr]\,\vol{\fg}\numberthis\label{eq:fESFcorr-2}
\end{align*}
\end{subequations}
Inserting the evolution equations into \eqref{eq:fESFcorr-1}, we obtain
\begin{align*}
\eqref{eq:fESFcorr-1}\leq&\,\int_M\left[3\lvert\fLap^k\phi^\prime\rvert^2+3\fN\lvert\fLap^k\phi^\prime\rvert^2\right]\vol{\fg}+K\|\fN\|_{C^{2k}}\|\phi^\prime\|_{H^{2k-1}}\|\Lap^k\phi^\prime\|_{L^2}\\
&+\int_M\left[ -2\fLap^k\phi\cdot\fLap^k\phi^\prime+3\fN\fLap^k\phi\cdot\fLap^k\phi^\prime +3\fLap^k\phi\cdot\fLap^{k+1}\phi+3\fN\fLap^k\phi\cdot\fLap^{k+1}\phi\right]\,\vol{\fg}\\
&\,+K\left[\|\fN\|_{C^{2k}}\left(\|\nabla\phi\|_{H^{2k}}+\|\phi^\prime\|_{H^{2k-1}}\right)+\left(\|\nabla\phi\|_{C^0}+\|\phi^\prime\|_{C^0}\right)\|\fN\|_{H^{2k+1}}\right]\|\fLap^k\phi\|_{L^2}
\end{align*}
Note that, after integrating by parts, the last two terms in the second line can be bounded by
\[\int_M-3\lvert\nabla\fLap\phi\rvert^2\,\vol{g}+\|\fN\|_{C^1}(\|\nabla\Lap^k\phi\|_{L^2}+\|\Lap^k\phi\|_{L^2})\|\nabla\Lap^k\phi\|_{L^2}\,.\]
For the terms in \eqref{eq:fESFcorr-2}, notice that the first term can be bounded by 
\[\|\fN\|_{C^0}\|\Lap^k\phi\|_{L^2}\|\Lap^k\phi^\prime\|_{L^2}\lesssim \epsilonnew e^{-\frac{T}2}\|\nabla\phi\|_{H^{2k-1}}\sqrt{\fE^{(2k)}}\,,\]
while the commutator terms can be estimated as before:
\begin{align*}
\|[\fdel,\fLap^k]\phi\|_{L^2}\lesssim&\,\|\nabla\phi\|_{C^{0}}\left(\|\fn\|_{C^0}\|\fk\|_{\dot{H}^{2k-1}}+\|\fX\|_{C^0}\left\|\Ric[\fg]+\frac29\fg\right\|_{H^{2k-2}}\right)\\
&\,+\left(\|\fn\|_{C^{2k}}\|\fk\|_{C^{2k-2}}+\|\fX\|_{C^{2k}}\|\Ric[\fg]\|_{C^1}\right)\|\nabla\phi\|_{H^{2k-1}}\\
\|[\fdel,\fLap^k]\phi^\prime\|_{L^2}\lesssim&\,\|\phi^\prime\|_{C^1}\left(\|\fn\|_{C^0}\|\fk\|_{\dot{H}^{2k-1}}+\|\fX\|_{C^0}\left\|\Ric[\fg]+\frac29\fg\right\|_{H^{2k-2}}\right)\\
&\,+\left(\|\fn\|_{C^{2k}}\|\fk\|_{C^{2k-2}}+\|\fX\|_{C^{2k}}\|\Ric[\fg]\|_{C^1}\right)\|\phi^\prime\|_{H^{2k}}
\end{align*}

\noindent Combining all of the above, we get
\begin{align*}
\del_T\fC^{(2k)}\lesssim&\,-2\fC^{(2k)}+\int_M \left[3\lvert \fLap^k\phi^\prime\rvert-3\lvert\nabla\fLap^k\phi\rvert_{\fg}\right]\,\vol{\fg}%+K\epsilonnew e^{-\frac{T}2} \fE^{(k)}
\\
&\,+K\epsilonnew e^{-\frac{T}2}\left[\|\phi^\prime\|_{H^{2k}}+\|\nabla\phi\|_{H^{2k}}+\|\fN\|_{H^{2k+1}}+\|\fk\|_{H^{2k}}+\epsilonnew e^{-\frac{T}2}\left\|\Ric[\fg]+\frac29\fg\right\|_{H^{2k-2}}\right]\\
&\,\phantom{+K}\cdot\left(\sqrt{\fE^{(2k)}}+\sqrt{\fE^{(2k-1)}}\right)
\end{align*}

\noindent Finally, combining both differential estimates %and then inserting the bootstrap assumption \eqref{eq:fut-bootstrap} for all occuring $C$-norms 
yields the statement for $l=2k$.
%\begin{align*}
%\del_T\left(\fE^{(2k)}+\frac23\fC^{(2k)}\right)\leq&\,\int_M\left[\left(-4+\frac23\cdot 3\right)\lvert\fLap^l\phi^\prime\rvert^2-3\cdot\frac23\lvert\nabla\fLap^k\phi\rvert_{\fg}^2\right]\,\vol{\fg}-2\cdot\frac23\fC^{(2k)}\\
%&\,+K\epsilonnew e^{-\frac{T}2}\fE^{(2k)}+K\epsilonnew\left(\sum_{m=0}^{2k}\sqrt{\fE^{(m)}}\right)\cdot\\
%&\,\phantom{+K}\cdot\left(\|\phi^\prime\|_{H^{2k}}+\|\nabla\phi\|_{H^{2k}}+\|\Sigma\|_{H^{2k}}+\epsilonnew\left\|\Ric[\fg]+\frac29\right\|_{H^{2k-1}}\right)\\
%%\leq&\, -2\left(\fE^{(2l)}+\frac23\fC^{(2l)}\right)+K\epsilonnew^3
%\end{align*}
For $l=2k-1,\,k\in\{1,2\}$, the argument is very similar, so we only sketch how the correction term is handled:
\begin{align*}
\del_T\fC^{(2k-1)}=&\,\int_M \biggr[\left\langle\nabla\Lap^{k-1}\fdel\phi,\nabla\Lap^{k-1}\phi^\prime\right\rangle_{\fg}+\left\langle\nabla\Lap^{k-1}\phi,\nabla\Lap^{k-1}\fdel\phi^\prime\right\rangle_{\fg}\numberthis\label{eq:fESF-corr-4}\\
&\phantom{\int_M}+\left(\fdel{\fg}^{-1}\right)^{ab}\cdot\nabla_a\Lap^{k-1}\phi\cdot\nabla_b\Lap^{k-1}\phi^\prime+3\fN\cdot\left\langle\nabla\Lap^{k-1}\phi,\nabla\Lap^{k-1}\phi^\prime\right\rangle_{\fg}\\
&\,\phantom{\int_M}+\left\langle[\fdel,\nabla\Lap^{k-1}]\phi,\nabla\Lap^{k-1}\phi^\prime\right\rangle_{\fg}+\left\langle[\fdel,\nabla\Lap^{k-1}]\phi^\prime,\nabla\Lap^{k-1}\phi\right\rangle_{\fg}\biggr]\,\vol{\fg}
\end{align*}
The final line results at most in error terms analogous to those in the even order case, while the second line can be estimated by
\[\lesssim\epsilonnew e^{-\frac{T}2}\|\nabla\Lap^{k-1}\phi\|_{L^2}\|\nabla\Lap^{k-1}\phi^\prime\|_{L^2}\lesssim \epsilonnew e^{-\frac{T}2}\|\nabla\phi\|_{H^{2k-2}}\sqrt{\fE^{(2k-1)}}\,,\]
leaving only the terms on the right hand side \eqref{eq:fESF-corr-4}, which can be treated as follows:
\begin{align*}
\eqref{eq:fESF-corr-4}\leq&\,\int_M \left[3\lvert\nabla\Lap^{k-1}\phi^\prime\rvert_{\fg}^2+3\fN\lvert\nabla\Lap^{k-1}\phi^\prime\rvert_{\fg}^2\right]\,\vol{\fg}+K\|\fN\|_{C^{2k-1}}\|\phi^\prime\|_{H^{2k-2}}\|\nabla\Lap\phi^\prime\|_{L^2}\\
&\,+\int_M\left[-2\langle\nabla\fLap^{k}\phi,\nabla\fLap^{k}\phi^\prime\rangle_{\fg}+3\fN\langle\nabla\fLap^{k}\phi,\nabla\fLap^{k}\phi^\prime\rangle_{\fg}-3\lvert\fLap^{k}\phi\rvert^2-3\fN\lvert\fLap^{k}\phi\rvert^2\right]\,\vol{\fg}\\
&\,+K\epsilonnew e^{-\frac{T}2}\left[\|\fN\|_{C^{2k-1}}\left(\|\nabla\phi\|_{H^{2k-1}}+\|\phi^\prime\|_{H^{2k-2}}\right)+\left(\|\nabla\phi\|_{C^0}+\|\phi^\prime\|_{C^0}\right)\|\fN\|_{H^{2k}}\right]
\end{align*}
Otherwise, the proof proceeds as in the even order case.
\end{proof}

\subsection{Geometric variables}\label{subsec:fut-geom-var}

We can obtain the necessary energy estimates by inserting Lemma \ref{lem:fut-matter-resc} into more general results:

\begin{lemma}[Coercivity of geometric energies, {\cite[Lemma 7.4]{AM11}}]\label{lem:fut-en-geom-coercivity}
For sufficiently small $\epsilonnew>0$, the following estimate holds:
\begin{equation}\label{eq:fut-en-geom-coervivity}
\|\fg-\gamma\|_{H^5}^2+\|\Sigma\|_{H^4}^2\lesssim \fEg
\end{equation}
\end{lemma}

\begin{lemma}[Geometric energy estimate]\label{lem:fut-geom-est}
Let $\epsilonnew>0$ be chosen appropriately small, and let
\begin{equation}\label{eq:fut-alpha}
\alpha=\begin{cases}
1 & \lambda_0>\frac19\\
1-3\sqrt{\epsilonnew^\prime} & \lambda_0=\frac19\,,
\end{cases}
\end{equation}
where $\epsilonnew^\prime>0$ is the same as in \eqref{eq:fut-corr-const}, in particular suitably small. Then, there exists some constant $K>0$ such that the following estimate holds:
\begin{align*}
\numberthis\label{eq:fut-geom-est}\del_T\fEg\leq&\,-2\alpha\fEg+K\fEg^\frac32\\
&\,+K\sqrt{\fEg}\left[1+\|\fN\|_{C^2}\right]\left[\|\phi^\prime\|_{C^2}+\|\nabla\phi\|_{C^2}(1+\|\fg-\gamma\|_{C^2})\right]\left[\|\phi^\prime\|_{H^4}+\|\nabla\phi\|_{H^4}\right]
\end{align*}
\end{lemma}
\begin{proof}
By \cite[Lemma 20]{AndFaj20}, one has:
\begin{align*}
\del_T\fEg\leq &\,-2\alpha\fEg + 6\sqrt{\fEg}\lvert\tau\rvert\|\fn S\|_{H^4}+K(\fEg)^{\frac32}\\
&\,+K\sqrt{\fEg}\left(\lvert\tau\rvert\|\rho\|_{H^4}+\lvert\tau\rvert^3\|\underline{\eta}\|_{H^4}+\lvert\tau\rvert^2\|\fn\jmath\|_{H^3}\right)
\end{align*}
The statement now follows by inserting the expressions for the rescaled matter quantities in Lemma \ref{lem:fut-matter-resc} to \cite[Lemma 20]{AndFaj20}.
\end{proof}


\subsection{Closing the bootstrap}\label{subsec:fut-bs-imp}

Now, we can collect our estimates to improve the bootstrap assumptions:

\begin{prop}[Improved bounds for future stability]\label{prop:fut-bs-imp} Let the bootstrap assumption (see Assumption \ref{ass:fut-bootstrap}) be satisfied for $T\in[0,T_{Boot})$ and assume the initial data holds at $T=0$ (see Assumption \ref{ass:fut-init}). For $\epsilonnew>0$ sufficiently small and $\alpha$ as in \eqref{eq:fut-alpha} with $\delta^\prime>0$ sufficiently small, the following estimates hold: 
\begin{subequations}
\begin{align}
\|\phi^\prime\|_{C^2}+\|\nabla\phi\|_{C^2}+\|\phi^\prime\|_{H^4}+\|\nabla\phi\|_{H^4}\lesssim&\,\epsilonnew^\frac32 e^{-T}\label{eq:fut-sf-imp}\\
\|\fg-\gamma\|_{C^3}+\|\Sigma\|_{C^2}+\|\fg-\gamma\|_{H^5}+\|\Sigma\|_{H^4}\lesssim&\,\epsilonnew^\frac32e^{-\alpha T}\label{eq:fut-geom-imp}\\
\|\fN\|_{C^4}+\|\fX\|_{C^4}+\|\fN\|_{H^6}+\|\fX\|_{H^6}\lesssim&\,\epsilonnew^3e^{-2\alpha T}\label{eq:fut-ell-imp}
\end{align}
Further, there exists a constant $K>0$ such that the following estimate holds:
\begin{equation}\label{eq:fut-phim-imp}
\|\phi\|_{L^\infty}\lesssim\|\phi(T=0,\cdot)\|_{L^\infty(\Sigma_{T=0}}+K\epsilonnew^\frac32
\end{equation}
\end{subequations}
\end{prop}
\begin{proof}
Combining the estimate from Lemma \ref{lem:fut-en-est-ESF0} as well as those from Lemma \ref{lem:fut-en-est-ESF} at each level and applying the bootstrap assumptions (see Assumption \ref{ass:fut-bootstrap}) to the right hand sides, we obtain:
\begin{equation*}
\del_TE^{(4)}_{SF}\leq -2E^{(4)}_{SF}+K\epsilonnew^3e^{-\frac32T}\,
\end{equation*}
After integrating over $[0,T]$ and inserting the initial data assumption \eqref{eq:fut-init}, this reads
\begin{equation*}
E^{(4)}_{SF}\leq -2\int_0^TE^{(4)}_{SF}(s)\,ds+\epsilonnew^4+\frac32K\epsilonnew^3(1-e^{-\frac32T})\,.
\end{equation*}
Since the last two summands are, updating $K>0$, bounded by $K\epsilonnew^3$, the Gronwall lemma implies
\begin{equation}\label{eq:fut-ESF-imp}
E^{(4)}_{SF}\lesssim \epsilonnew^3e^{-2T}\,.
\end{equation}
Analogously, the geometric energy estimate in Lemma \ref{lem:fut-geom-est} implies
\begin{equation*}
\fEg\lesssim \epsilonnew^3e^{-2\alpha T}\,.
\end{equation*}
Lemma \ref{lem:fut-en-geom-coercivity} and the standard Sobolev embedding then imply \eqref{eq:fut-geom-imp}. In particular, this means
\begin{equation}\label{eq:fut-curv-imp}
\left\|\Ric[\fg]+\frac29\fg\right\|_{H^2}\leq \epsilonnew^\frac32e^{-\alpha T}
\end{equation}
due to Lemma \ref{lem:fut-Ric-est}, and for $\delta^\prime>0$ small enough, inserting \eqref{eq:fut-ESF-imp} and \eqref{eq:fut-curv-imp} into \eqref{eq:fut-coerc} shows \eqref{eq:fut-sf-imp}. This in particular improves the $C$-norms we just used the bootstrap assumption for, and reinserting this improvement into \eqref{eq:fut-coerc} improves the bound to \eqref{eq:fut-ESF-imp}. \\
Moreover, \eqref{eq:fut-ell-imp} follows directly from Lemma \ref{lem:fut-ell-est} and the already obtained improvements.\\
Finally, notice that $\fn\in(0,3)$ (see Lemma \ref{lem:fut-ell-est}) as well as the attained decay estimates for $\phi^\prime, \nabla\phi$ and $\fX$ yield the following:
\[\lvert\del_T\phi\rvert\lesssim \lvert\fn\phi^\prime+\fX^c\nabla_c\phi\rvert\lesssim \epsilonnew^\frac32e^{-T}+\epsilonnew^\frac92e^{-(1+2\alpha)T}\]
After integrating on $[0,T]$ and taking the supremum over $\Sigma_T$, we get \eqref{eq:fut-phim-imp}.
\end{proof}


\begin{proof}[Proof of Theorem \ref{thm:fut-stab-simple}]

The problem is locally well-posed as outlined in Remark \ref{rem:fut-lwp}. There then is some maximal interval $[0,T_{Boot})$ for the logarithmic time $T$ (or, equivalently, some maximal time interval $[\tau_0,\tau_{Boot})$) on which the solution exists and the bootstrap assumptions (see Ass. \ref{ass:fut-bootstrap}) are satisfied. By the analogous argument to the proof of Theorem \ref{thm:main}, the decay estimates in Proposition \ref{prop:fut-bs-imp} are strictly stronger than the bootstrap assumptions for small enough $\epsilonnew,\delta^\prime>0$. This implies $T_{Boot}=\infty$ (resp. $\tau_{Boot}=0$) since we could else extend the solution strictly beyond $T_{Boot}$ while also satisfying the bootstrap assumptions.\\

\noindent To be a bit more precise, $(g,k,\nabla\phi,\fdel\phi)$ would actually remain in $H^6\times H^5\times H^5\times H^5$ beyond $T_{Boot}$ if it were finite since any breakdown in regularity would have to be reflected in lower regularity blow-up (see, for comparison, the vacuum analogues in \cite[Theorem 3.1]{AM03B} or the blow-up statements in Section \ref{subsec:lwp}). Hence, not only do we find a solution on $[T_0,\infty)\times M$, but all energies used above remain continuously differentiable which means the decay estimates hold in the entire future of $\Sigma_{\tau_0}$. This proves the convergence statement in Theorem \ref{thm:fut-stab-simple}.\\
Finally, the decay estimates imply that $\lvert\nabla n\rvert_{g}$, respectively $\lvert k\rvert_{g}$, are bounded by $\tau^{\alpha-1}$, respectively $\tau^{\alpha+1}$, up to constant on $[\tau_0,\tau)$. Since $\alpha$ is at worst slightly smaller than $1$, both functions are integrable on $[\tau_0,0)$ for suitably small $\delta^\prime>0$ . By \cite{CB02}, this means the spacetime is future complete.
\end{proof}

\newpage

\section{Global stability}\label{sec:full-stab}

To prove Theorem \ref{thm:main-full}, what still needs to be shown is that initial data as in Theorem \ref{thm:main-past} develops from $\Sigma_{t_0}$ to some hypersurface $\Sigma_{t_1}\equiv \Sigma_{\tau(t_1)}$ in its future such that the data in $\Sigma_{t_1}$ is near-Milne in the sense of Assumption \ref{ass:fut-init} and in CMCSH gauge. From there, near-Milne stability yields the behaviour in the future of $\Sigma_{\tau(t_1)}$, and hence future stability of near-FLRW spacetimes as in Theorem \ref{thm:main-full}. %That combining the future and past developments of the initial data at $\Sigma_{t_0}$, despite moving between gauges, constitutes a global solution $(\M,\g,\phi)$ follows from uniqueness of the maximal globally hyperbolic development within the Einstein scalar-field system.

 %Since we need to develop the initial data used for Big Bang stability forward in time, we revert back to using the physical time coordinate $t$ with CMC gauge $\tau=-3\frac{\dot{a}}a$.\\
\begin{proof}[Proof of Theorem \ref{thm:main-full}] Within this proof, $t$ will denote the \enquote{physical} time coordinate used throughout the Big Bang stability analysis, while $\tau$ denotes the mean curvature time used within CMCSH gauge.

Consider initial data $(g,k,\nabla\phi,\del_0\phi)$ induced on the CMC hypersurface $\Sigma_{t_0}$ within $\M$ such that the rescaled variables are close to FLRW reference data in the sense of Theorem \ref{thm:main}. Moreover, let $(\mathring{g},\mathring{k},\mathring{\pi},\mathring{\psi})$ be the geometric initial data on $M$ that induce it via the embedding $\iota:M\hookrightarrow \M$. This data is close to $(\gamma,\frac13\gamma,0,C)$ in $(H_\gamma^{18}\times H_\gamma^{18}\times H_\gamma^{18}\times H_\gamma^{19})(M)$. \\

\noindent Notice that
\[P:H^{20}_\gamma(M)\rightarrow H^{18}_\gamma(M),\,Y^i\mapsto \Lap_\gamma Y^i+(\gamma^{-1})^{il}{R[\gamma]}_{lj}Y^j=\Lap_{\gamma} Y^i-\frac29 Y^i=0\]
is an isomorphism since $\Lap_\gamma$ has no positive eigenvalues. Hence, using \cite[Theorem 2.5, Remark 2.6]{FajKr20}, there is an metric $\mathring{g}^\prime$ isometric to $\mathring{g}$ that remains $H^{18}_\gamma(M)$-close to $\gamma$ and satisfies
\[((\mathring{g}^\prime)^{-1})^{ij}\left(\Gamma[\mathring{g}^\prime]^k_{ij}-\Gamhat[\gamma]^{k}_{ij}\right)=0.\]
Let $\theta\in\text{Diff}(M)$ be the diffeomorphism such that $\theta^\ast \mathring{g}=\mathring{g}^\prime$, then the proof of \cite[Theorem 2.5]{FajKr20} implies that $\theta$ can be chosen close to the identity within $H^{18}(\text{Diff}(M))$, and consequently that $\theta^\ast\mathring{k}=\mathring{k}^\prime,\, \theta^\ast\mathring{\pi}=\mathring{\pi}^\prime$ and $\theta^\ast\mathring{\psi}=\mathring{\psi}^\prime$ remain close to $\frac13\gamma, 0$ and $C$ in $H^{18}_\gamma(M)$. By the same argument as in Remark \ref{rem:CMC-hypersurface}, we can now evolve this data locally and obtain a new initial hypersurface $\Sigma^\prime$ close to $\Sigma_{t_0}$ that is in CMCSH gauge and that $(g,k,\nabla\phi,\del_0\phi)$ is close to the reference data in the sense of Assumption \ref{ass:init}, exchanging the initial time $t_0$ by some close time $t_0^\prime$.\\

Since $\tau$ is strictly increasing, $t\equiv t(\tau)$ exists and we can interchangeably view $a$ as a function in $t$ or $\tau$ with some abuse of notation. The Friedman equation \eqref{eq:Friedman} implies $\del_ta\geq\frac19$ and thus $a(t)\geq\frac19t$ on $(0,\infty)$, as well as 
\[-\tau=3\frac{\dot{a}}a= \frac1a+\langle\text{lower order terms}\rangle\ \text{as}\ t\to \infty\ (\text{resp. }\tau\to 0)\,.\]

We choose $t_1>\max\{1,t_0^\prime\}$ large enough (resp. $\tau(t_1)\equiv\tau_0$ small enough) that the following estimates hold for some small $\chi\in(0,\frac12)$ that depends only on $\delta$:
\begin{align}
Ca(t_1)^{-3}\tau(t_1)^{-1}\leq&\, \chi \label{eq:connect-time-1}
\\
 -\tau(t_1)\cdot{a(t_1)}\in&\,[1-\chi,1+\chi] \label{eq:connect-time-3}
\end{align}

As the solution and its maximal time of existence depend continuously upon the initial data\footnote{For the argument for Einstein vacuum in CMCSH gauge, see \cite[Theorem 3.1]{AM03}. As with local existence, the argument in the Einstein scalar-field system is largely identical since the only difference amounts to coupling the hyperbolic parts of the system with a further hyperbolic one.}, one can choose $\epsilon>0$ in the analogue of Assumption \ref{ass:init} small enough to ensure the following: The solution exists until $t_1>t_0^\prime$ and $(a^{-2}g,a\hat{k},\nabla\phi,a^{3}\fdel\phi)$ remain $K\epsilon$-close to $(\gamma,0,0,C)$ in $H_\gamma^6\times H_\gamma^5\times H_\gamma^5\times H_\gamma^5$ for some suitable $K>0$ along the slab $\cup_{s\in [t_0^\prime,t_1]}\Sigma_s$. What now remains to be shown is that this implies Assumption \ref{ass:fut-init} in the sense that, if $\epsilon$ is small enough, $\delta$ can be made as small as necessary for Theorem \ref{thm:fut-stab-simple} to apply.\\

\noindent Note that the scalings in Definition \ref{def:fut-rescaled} can be rewritten as follows:
\begin{gather*}
\fg-\gamma=\tau^2g-\gamma=(\tau \cdot a)^2\cdot (a^{-2} g-\gamma)+(\tau^2\cdot a^2-1)\gamma\\
\fk=\tau \hat{k}=\frac{\tau}a (a\hat{k})\\
\phi^\prime=(-\tau)^{-1}\cdot n^{-1}(\del_\tau-\Lie_X)\phi=C\left(-\tau^{-1}\cdot a^{-3}\right)+\left(-\tau^{-1}\cdot a^{-3}\right)\cdot (a^3n^{-1}(\del_\tau-\Lie_X)\phi-C)
\end{gather*}
Since \eqref{eq:connect-time-3} implies $\tau\cdot a$ is close to $-1$ at $t_1$, $\|(\tau\cdot a)^2(a^{-2}g-\gamma)\|_{H^6}$ can be bounded by $\frac{\delta^3}2$ for small enough $\epsilon$. Choosing $\chi<\frac{\delta^3}2$ then implies $\|\fg-\gamma\|_{H^6(\Sigma_{\tau_0})}<\delta^3$. That $\|\fk\|_{H^5}$ can be made smaller than $\delta^3$ for small enough $\epsilon>0$ follows since $\frac\tau{a}$ behaves like $\frac1{a^2}$ up to a constant by \eqref{eq:connect-time-3}.\\
For the normal derivative of the wave, notice that $\lvert C\left(-\tau^{-1}\cdot a^{-3}\right)\rvert $ is bounded by $\chi$ due to \eqref{eq:connect-time-1}, and that $-\tau^{-1}a^{-3}$ is equivalent to $a^{-2}$ by \eqref{eq:connect-time-3}. Hence, we can similarly ensure that $\phi^\prime$ is bounded in $H^5$ by $\delta^3$. Since $\nabla\phi$ is not changed in either rescaling, and bounds on lapse and shift (up to constant) follow from the elliptic estimates in Lemma \ref{lem:fut-ell-est}, it follows each individual norm in Assumption \ref{ass:fut-init} can be bounded by $\delta^3$ up to constants that depend only on $\gamma$, and hence the initial data assumption itself can be satisfied for suitably small $\delta>0$.\\
This proves that we can develop from initial data for the Big Bang stability proof to near-Milne initial data within a CMCSH foliation, and thus we obtain Theorem \ref{thm:main-full} from Theorem \ref{thm:main-past} and Theorem \ref{thm:fut-stab-simple}.\\

%Finally, the past development of $\Sigma_{t_0}$, the local development between $\Sigma_{t_0}$ and $\Sigma^\prime$, the development from $\Sigma^\prime$ and $\Sigma_{\tau_0}$ and the latter's future development all combine to a single maximal globally hyperbolic development of the initial data at $\Sigma_{t_0}$ within the Einstein scalar-field equations.
\end{proof}

%Further, we can choose $\epsilon>0$ in the initial data assumption for Big Bang stability (see Assumption \ref{ass:init}) such $(G,\Sigma,\nabla\phi,\Psi)$ induces a solution to the Einstein scalar-field system on the entirety of $[t_0,t_1]$ and such that
%\begin{align*}
%B_{K\epsilon}^{5,4,4,4}(t_0)&\,\longrightarrow \left[H_\gamma^5\times H_\gamma^4\times H_\gamma^4\times H_\gamma^4\right](\Sigma_{t_1})
%(G,\Sigma,\nabla\phi,\Psi)(t_0,\cdot)&\,\mapsto (G,\Sigma,\nabla\phi,\Psi)(t_1,\cdot)
%\end{align*}
%is continuous, where $B_{\epsilon^\prime}^{5,4,4,4}(t)$ denotes the $\epsilon^\prime$-ball around FLRW data $(\gamma,0,0,Ca(t)^{-3})$ at $\Sigma_{t}$ in $\left[H_\gamma^5\times H_\gamma^4\times H_\gamma^4\times H_\gamma^4\right](\Sigma_{t_0})$ \todo{[cite AM or examine RS closely]}. As a consequence, for any $\epsilon_f>0$, we can find some $\epsilon_p<K\epsilon$ such that $(G,\Sigma,\nabla\phi,\psi)(t_1,\cdot)\in B_{\epsilon_f}^{5,4,4,4}(t_1)$ holds if the initial data at $\Sigma_{t_0}$ is $\epsilon_p$-close to FLRW initial data.\\
%
%What now remains to be shown is that $(G,\Sigma,\nabla\phi,\Psi)(t_1,\cdot)\in B_{\epsilon_f}^{5,4,4,4}(t_1)$ that, after switching from CMC gauge with zero shift to CMCSH gauge, implies Assumption \ref{ass:fut-init} is $\epsilon_p$ is chosen small enough that $\epsilon_f$ is sufficiently small compared to $\delta^2$. Since lapse and shift can be sufficiently controlled by the other quantities (see Remark \todo{[ref]}), one only needs to prove that $(\fg,\fk,\nabla\phi,\phi^\prime)(\tau(t_1),\cdot)$ can be arbitrarily close to $(\gamma,0,0,0)$ at $t_1$ if $\epsilon_f$ is small enough. 


%This can be argued along similar lines as in the vacuum case in \todo{AM}, given that the coupling of the elliptic-hyperbolic system in the vacuum setting with the wave equation keeps the structure of the geometric evolution equations in tact and only adds an additional system of the same type. The elliptic system is in $(n,X)$ is of the type
%\[A\begin{pmatrix} n\\ X\end{pmatrix}=\begin{pmatrix}
%B & 0\\
%E & P
%\end{pmatrix}\begin{pmatrix} n\\ X\end{pmatrix}=\begin{pmatrix} 1 \\ 0\,,\]
%and what needs to be shown is that $A: H^2\longrightarrow L^2$ is invertible given $(\mathring{g},\mathring{k},\mathring{\pi},\mathring{\psi})$, to which is suffices to show that $B,P:H^2\longrightarrow L^2$ are isomorphisms. For the Einstein scalar-field system, these operators take the following form
%\begin{align*}
%Bf=&\,\left(-\Lap_{\mathring{g}}+\lvert \mathring{k}\rvert_\mathring{g}^2+8\pi\lvert\mathring{\psi}\rvert^2\right)f\\
%PY^i=&\,\Lap_{\mathring{g}}Y^i+{R[\mathring{g}]^i}_fY^f-\Lie_{Y}V^{i}-2\nabla{\mathring{g}}^mY^n\left(\Gamma[\mathring{g}]^{i}_{mn}-\Gamhat^i_{mn}\right)\,
%\end{align*}
%where $V[\mathring{g}]^i=(\mathring{g}^{-1})^{mn}\Gamma[\mathring{g}]^{i}_{mn}-\Gamhat^i_{mn})$. $B$ is immediately seen to be invertible since $\Lap_{\mathring{g}}$ has nonpositive spectrum and since $\mathring{\psi}$ is close to $C$, ensuring that $\lvert \mathring{k}\rvert_\mathring{g}^2+8\pi\lvert\mathring{\psi}\rvert^2$ is strictly positive.\\
%To show that $P$ is an isomorphism, first note that the $\lvert V\rvert_{\gamma}$ must be small given that $\mathring{g}$ is close to $\gamma$ in $C^1_\gamma$, that $V[\gamma]=0$ trivially holds and that $V$ is continuous in $\mathring{g}$. With this in hand, we argue as in \todo{Lemma 5.2, AM} to rewrite $P$ as
%\[PY^\alpha=\mathcal{L}_{\mathring{g},\gamma}Y^\alpha\]
%
%

%\section{Old version}
%
%Assume we have a local stability statement of the type that, when initial data near enough to the FLRW reference, we can develop to the future by some time increment that does not depend on the initial data size while at most multiplying the data size by a constant.
%
%Then, what needs to be ensured is that, when developing data from near the Big Bang to the far future, $\phi^\prime$ becomes small (Near the Big Bang, $\tau\approx a^{-3}$, hence $\phi^\prime\approx a^3\del_t\phi\approx C>0$).\footnote{We need to connect from the early universe forwards, since the future stability includes near-Milne spacetimes which we cannot connect backward to a spacetime that exhibits a Big Bang singularity}. That the smallness of $G,\Sigma$ and $\nabla\phi$ ensures the smallness of $\fg,\fk$ and $\nabla\phi$ for large enough times follows simply from the fact that $\tau$ asymptotically behaves like $-\frac1a$ for large times. 
%
%To this end, assume that local stability (in CMC gauge with zero shift) was able to ensure that $G-\gamma$, $\nabla\phi$ and $N$ remain small at sufficiently high regularity. Then one has, for some small $\chi>0$,
%\[\del_t(a^3\del_t\nabhat^{I}\phi)=a\nabhat\left[\langle\nabla N,\nabla\phi\rangle_G+(N+1)\Lap\phi\right]\simeq a\cdot \chi\]
%and hence
%\[\del_t\nabhat^I\phi\approx (C\pm \epsilonnew^2)a^{-3}\pm \chi \cdot a^{-3}\int_{t_0}^ta(s)\,ds\,.\]
%Since $a$ is strictly increasing and bounded from below by $\frac{t}3$, this implies
%\[\lvert\del_t\nabhat^I\phi\rvert\lesssim t^{-3}+\chi t^{-2}(t-t_0)\leq t^{-3}+\delta t^{-1}\]
%For large $t$, this implies
%\[\lvert\tau^{-1}\del_t\nabhat^I\phi\rvert\lesssim t^{-2}+\chi \,,\]
%which is $\lesssim\delta$ for large enough $t$, and hence $\lvert\nabhat^I\phi^\prime\rvert\lesssim\chi$.
%
%


\section{Appendix -- Big Bang Stability}\label{sec:appendix}

\subsection{Basic formulas and estimates}\label{subsec:basic}

\subsubsection{Tools from elementary calculus}

\begin{lemma}[A Gronwall lemma]\label{lem:gronwall} Let $f,\chi,\xi:[a,b]\longrightarrow \R$ be continuous functions such that $\chi\geq 0$, $\xi$ is decreasing and, for any $s\in[a,b]$,
\[f(s)\leq \int_s^b\chi(r)f(r)\,dr+\xi(s)\]
is satisfied. Then, for any $t\in[a,b]$, we have
\[f(t)\leq \xi(t)\exp\left(\int_t^b\chi(r)\,dr\right)\,.\]
\end{lemma}
\begin{proof}
%Fix some $T\in[a,b)$. For any $s\in[T,b]$, we have $\xi(s)\leq\xi(T)$ and hence, for the constant $K=\xi(T)$, we have the integral inequality
%\begin{equation}\label{eq:gronwall-step}
%f(s)\leq K+\int_s^b\chi(r)f(r)\,dr\,,
%\end{equation}
%and it suffices to show that this implies
%\begin{equation}\label{eq:gronwall-step1}
%f(T)\leq K\exp\left(\int_T^b\chi(r)f(r)\,dr\right)=\xi(T)\exp\left(\int_T^b\chi(r)f(r)\,dr\right)
%\end{equation}
%since $T$ was arbitrary. 
This follows by standard arguments as in \cite[Corollary 2-3]{Dra03}.%\\
%We define $z(s)=\int_s^b\chi(s)f(s)\,ds$. Then, we have with \eqref{eq:gronwall-step}
%\begin{align*}
%\left[z\cdot\exp\left(-\int_{(\cdot)}^b\chi(r)\,dr\right)\right]^\prime(s)=&\,\left[-\chi(s)f(s)+\chi(s)z(s)\right]\exp\left(-\int_s^b\chi(r)\,dr\right)\\
%\geq&\,-K\chi(s)\exp\left(-\int_s^b\chi(r)\,dr\right)\,ds
%\end{align*}
%Integrating on $[T,b]$ and multiplying with $-\exp\left(-\int_T^b\chi(r)\,dr\right)\leq 0$ on both sides leads to
%\[f(T)\leq K+z(T)\leq K+\int_T^bK\chi(s)\exp\left(\int_T^s\chi(r)\,dr\right)\,ds\,.\]
%Furthermore, the fundamental theorem of calculus yields
%\[K\exp\left(\int_T^b\chi(r)\,dr\right)-K=\int_T^bK\exp\left(\int_T^s\chi(r)\,dr\right)\chi(s)\,ds\]
%and, inserting this above, obtain
%\[f(T)\leq K+\left(K\exp\left(\int_T^b\chi(r)\,dr\right)-K\right)\,.\]
%This proves \eqref{eq:gronwall-step1} and thus the lemma itself.
\end{proof}

\begin{lemma}[A weak fundamental theorem of calculus for square roots]\label{lem:weak-ftoc}
Let $f:(0,t_0]\longrightarrow\R^+_0$ be a $C^1$-function. Then, we have for any $t\in(0,t_0]$:
\begin{equation}\label{eq:weak-ftoc}
\sqrt{f(t)}\leq \sqrt{f(t_0)}+\int_t^{t_0}\frac{\lvert f^\prime(s)\rvert}{2\sqrt{f(s)}}\,ds
\end{equation}
\end{lemma}
\begin{proof} This follows from a straightforward application of the monotone convergence theorem to $g_n=\sqrt{f+\frac1n}$.
%Without loss of generality, we may assume $\frac{f^\prime}{2\sqrt{f}}\in L^1((0,t_0])$ holds (else, the statement is trivial).
%We define $g_n:=\sqrt{f+\frac1n}$. Then the chain rule implies $g_n\in C^1((T,t_0])$ with
%\[g^\prime_n=\frac{f^\prime}{2\sqrt{f+\frac1n}}.\]
%Note that $(\lvert g_n^\prime\rvert)_{n\in\N}$ is a sequence in $L^1_{loc}$ with $g_n(t)\leq g_{m}(t)$ for $n<m$ converging pointwise to
%\[h:(0,t_0]\longrightarrow \R^+_0\cup\{\infty\},\,h(t):=\frac{\lvert f^\prime(t)\rvert}{2\sqrt{f(t)}}\,.\]
%Hence, by monotone convergence, we have
%\[\int_t^{t_0}h(s)\,ds=\lim_{n\to\infty}\int_t^{t_0}\lvert g_n^\prime(s)\rvert\,ds\geq \lim_{n\to\infty}\left\lvert\int_t^{t_0}g_n^\prime(s)\,ds\right\rvert=\lvert \sqrt{f(t_0)}-\sqrt{f(t)}\rvert\,.\]
%The statement now follows with the reverse triangle inequality and rearranging.
\end{proof}

\subsubsection{Levi-Civita tensor identities}\label{subsubsec:basic-contract}
Herein, we collect some basic identities for the Levi-Civita tensor $\epsilonLC[g]$: Firstly, it satisfies the contraction identities, where $\I^a_b$ denotes the Kronecker-symbol:
\begin{subequations}
\begin{align}
\epsilonLC^{ai_1i_2}\epsilonLC_{aj_1j_2}=&\,\I^{i_1}_{j_1}\I^{i_2}_{j_2}-\I^{i_1}_{j_2}\I^{i_2}_{j_1}\label{eq:LCS-contr1}\\
\epsilonLC^{abi}\epsilonLC_{abj_2}=&\,2\I^{i}_j\label{eq:LCS-contr2}\\
\epsilonLC^{abc}\epsilonLC_{abc}=&\,6\label{eq:LCS-contr3}\\
\nabla\epsilonLC=&\,0\label{eq:LCS-cov}
\end{align}
\end{subequations}
The analogous formulas hold for $\epsilonLC[G]$ when raising indices with regard to $G$ instead of $g$.\\
\noindent For a tracefree and symmetric $\Sigma_t$-tangent $(0,2)$-tensor $\mathfrak{T}$ and a $\Sigma_t$-tangent $(0,2)$-tensor $\mathfrak{A}$, the following simplified identities hold:
\begin{subequations}
\begin{align}
(\mathfrak{T}\times \mathfrak{A})_{ij}&={\epsilonLC_i}^{ab}{\epsilonLC_j}^{pq}\mathfrak{T}_{ap}\mathfrak{A}_{bq}+\frac13(\mathfrak{T}\cdot\mathfrak{A})g_{ij} \label{eq:times-lemma-1}\\
(\mathfrak{T}\times g)_{ij}&=-\mathfrak{T}_{ij} \label{eq:times-lemma-2}\\
(\mathfrak{T}\times k)_{ij}&=-\frac{\tau}3\mathfrak{T}_{ij}+(\mathfrak{T}\times\hat{k})_{ij} \label{eq:times-lemma-3}
\end{align}
Further, note the following formulas (for $\tilde{\mathfrak{T}}$ as $\mathfrak{T}$,  $\tilde{\mathfrak{A}}$ as $\mathfrak{A}$ and any $\Sigma_t$-tangent $(0,1)$-tensor $\xi$) (see \cite[p.30]{AM03}):
\begin{align}
\div_g(\mathfrak{A}\wedge\tilde{\mathfrak{A}})&=-\curl \mathfrak{A}\cdot\tilde{\mathfrak{A}}+\mathfrak{A}\cdot\curl\tilde{\mathfrak{A}} \label{eq:div-to-curl}\\
\mathfrak{A}\cdot(\xi\wedge\tilde{\mathfrak{A}})&=-2\xi\cdot(\mathfrak{A}\wedge\tilde{\mathfrak{A}})\\
\mathfrak{T}\cdot(\mathfrak{A}\times\tilde{\mathfrak{T}})&=(\mathfrak{T}\times \mathfrak{A})\cdot\tilde{\mathfrak{T}}
\end{align}
\end{subequations}

\subsubsection{Estimates on contracted tensors}\label{subsubsec:basic-est}

\begin{lemma}\label{lem:BelRobinsonLemmas}
Let $\mathfrak{S},\mathfrak{T}$ be traceless and symmetric $\Sigma_t$-tangent $(0,2)$-tensors, $\mathfrak{M},\mathfrak{N}$ symmetric $\Sigma_t$-tangent $(0,2)$-tensors and $\xi$ a $\Sigma_t$-tangent $(0,1)$-tensor. We define $G, G^{-1}$ and $\lvert\cdot\rvert_G$ via \eqref{eq:rescalingGK}). Then:
\begin{subequations}
\change{\begin{align}
\lvert \mathfrak{M}\odot_G\mathfrak{N}\rvert_G\leq\lvert \mathfrak{M}\rvert_G\lvert \mathfrak{N}\rvert_G,&\quad \mathfrak{M}\odot_g\mathfrak{N}= a^{-2}\mathfrak{M}\odot_G\mathfrak{N}\label{eq:odot}\\
\lvert \mathfrak{S}\times_G \mathfrak{T}\rvert_G\lesssim\lvert \mathfrak{S}\rvert_G\lvert \mathfrak{T}\rvert_G,&\quad (\mathfrak{S}\times \mathfrak{T})_{ij}=a^{-3}(\mathfrak{S}\times_G \mathfrak{T})_{ij}\label{eq:times}\\
\lvert \mathfrak{S}\wedge_G \mathfrak{T}\rvert_G\leq \lvert \mathfrak{S}\rvert_G\lvert \mathfrak{T}\rvert_G,&\quad \left(\mathfrak{S}\wedge \mathfrak{T}\right)_l= a^{-3}\left(\mathfrak{S}\wedge_G \mathfrak{T}\right)\label{eq:wedge}\\
\lvert \xi\wedge_G \mathfrak{T}\rvert_G\leq\lvert \xi\rvert_G\lvert \mathfrak{T}\rvert_G,&\quad \left(\xi\wedge\mathfrak{T}\right)_{ij}= a^{-1}(\xi\wedge_G\mathfrak{T})_{ij}\label{eq:wedge2}\\
\lvert\curl_G\mathfrak{M}\rvert_G\lesssim \lvert\nabla \mathfrak{M}\rvert_G,&\quad \curl \mathfrak{M}_{ij}= a^{-1}\curl_G\mathfrak{M}_{ij}\label{eq:curl}
\end{align}}
\end{subequations}
\end{lemma}
\begin{proof} The estimates with respect to the unrescaled metric are direct consequences of the contraction identities \eqref{eq:LCS-contr1}-\eqref{eq:LCS-contr3} \change{replacing $g$ with $G$}, and \change{the scalings }follow simply by tracking the effects of the rescaling in Definition \ref{def:rescaled}. In particular, 
\begin{equation}\label{eq:epsilonLC-resc}
{\epsilonLC[g]_{i}}^{cd}=g^{cj}g^{dk}\epsilonLC[g]_{ijk}=\left(a^{-2} (G^{-1})^{cj}\right)\left(a^{-2} (G^{-1})^{dk}\right)a^3\epsilonLC[G]_{ijk}=a^{-1}{\epsilon[G]_{i}}^{\sharp cd}\
\end{equation}
determines the scaling of the Levi-Civita-Symbol.
%First, note that $\lvert \mathfrak{M}\odot_g\mathfrak{N}\rvert_g^2$ is the Frobenius norm of the matrix given by ${(\mathfrak{M}\odot \mathfrak{N})^i}_j$, which is also the matrix product of ${\mathfrak{M}^i}_k$ and ${\mathfrak{N}^k}_j$. Thus, the submultiplicativity of the Frobenius norm, along with ${\mathfrak{M}^i}_j{\mathfrak{M}^j}_i=\lvert \mathfrak{M}\rvert_g^2$, gives the statement. The statement for the rescaled metric follows analogously.\\
%
%Regarding \eqref{eq:times}, using \eqref{eq:times-lemma-1}, we get:
%\begin{align*}
%\lvert \mathfrak{S}\times \mathfrak{T}\rvert_g^2&=\epsilonLC^{iab}\epsilonLC^{jpq}\mathfrak{S}_{ap}\mathfrak{T}_{bq}{\epsilonLC_i}^{cd}{\epsilonLC_j}^{rs}\mathfrak{S}_{cr}\mathfrak{T}_{ds}+\frac23(\mathfrak{S}\cdot \mathfrak{T})\epsilonLC^{iab}{\epsilonLC_i}^{ pq}\mathfrak{S}_{ap}\mathfrak{T}_{bq}+\frac13(S\cdot T)^2\\
%&=\Bigr[\I^a_c\I^b_d-\I^a_d\I^b_c\Bigr]\Bigr[\I^p_r\I^q_s-\I^p_s\I^q_r\Bigr]\mathfrak{S}_{ap}\mathfrak{S}^{cr}\mathfrak{T}_{bq}\mathfrak{T}^{ds}+\frac23(\mathfrak{S}\cdot \mathfrak{T})\left[\I^a_p\I^b_q-\I^a_q\I^b_p\right]\mathfrak{S}_{a}^p\mathfrak{T}_b^{q}+\frac13\langle \mathfrak{S},\mathfrak{T}\rangle_g^2\\
%&=\left[(\mathfrak{S}\cdot \mathfrak{S})(\mathfrak{T}\cdot \mathfrak{T})-((\mathfrak{S}\odot_g\mathfrak{T})\cdot (\mathfrak{S}\odot_g \mathfrak{T}))-((\mathfrak{S}\odot_g \mathfrak{S})\cdot (\mathfrak{T}\odot_g \mathfrak{T}))\right]+\frac23(\mathfrak{S}\cdot \mathfrak{T})^2\\
%&\lesssim \lvert \mathfrak{S}\rvert_g^2\lvert \mathfrak{T}\rvert_g^2
%\end{align*}
%The respective inequality with regard to $G$ now follows by rescaling.\\
%Next, we compute \eqref{eq:wedge}:
%\begin{align*}
%\lvert \mathfrak{S}\wedge \mathfrak{T}\rvert_g^2&=\epsilonLC^{jkp}\epsilonLC_{jlr}\mathfrak{S}_k^q\mathfrak{T}_{qp}\mathfrak{S}^l_s\mathfrak{T}^{sr}\\
%&=\left[\I^k_l\I^p_r-\I^k_r\I^p_l\right]\mathfrak{S}_k^q\mathfrak{T}_{qp}\mathfrak{S}^l_s\mathfrak{T}^{sr}\\
%&=\langle \mathfrak{S}\odot_g\mathfrak{S},\mathfrak{T}\odot_g\mathfrak{T}\rangle_g-\lvert \mathfrak{S}\odot_g\mathfrak{T}\rvert^2_g\\
%&\leq \lvert \mathfrak{S}\rvert_g^2\lvert \mathfrak{T}\rvert_g^2\,,\\
%\lvert \mathfrak{S}\wedge \mathfrak{T}\rvert_G^2&=a^2\lvert \mathfrak{S}\wedge \mathfrak{T}\rvert_g^2\leq a^2\lvert \mathfrak{S}\rvert_g^2\lvert \mathfrak{T}\rvert_g^2\leq a^{-6}\lvert \mathfrak{S}\rvert_G^2\lvert \mathfrak{T}\rvert_G^2
%\end{align*}
%The proof for \eqref{eq:wedge2} is completely analogous. Finally,
%\begin{align*}
%\lvert \curl \mathfrak{M}\rvert_g^2&=\frac14g^{i_1j_1}g^{i_2j_2}\left[{\epsilonLC_{i_1}}^{cd}\nabla_d\mathfrak{M}_{ci_2}+{\epsilonLC_{i_2}}^{cd}\nabla_d\mathfrak{M}_{ci_1}\right]\left[{\epsilonLC_{j_1}}^{ef}\nabla_f\mathfrak{M}_{ej_2}+{\epsilon_{j_2}}^{ef}\nabla_f\mathfrak{M}_{ej_1}\right]\\
%&=\frac12\left[\epsilonLC^{jcd}\epsilonLC_{jef}\nabla_d\mathfrak{M}_{ci}\nabla^f\mathfrak{M}^{ei}+\epsilonLC^{jcd}\epsilonLC^{ief}\nabla_d\mathfrak{M}_{ci}\nabla_f\mathfrak{M}_{ej}\right]\\
%&=\frac12\left[\nabla_e\mathfrak{M}_{fi}\nabla^e\mathfrak{M}^{fi}-\nabla_f\mathfrak{M}_{ei}\nabla^e\mathfrak{M}^{fi}+(\epsilonLC^{ief}\nabla_f\mathfrak{M}_{ej})(\epsilonLC^{jcd}\nabla_d\mathfrak{M}_{ci})\right]\\
%&=\frac12\left[\lvert\nabla \mathfrak{M}\rvert_g^2-\langle N1,N2\rangle_g+\langle E1,E2\rangle_g\right]
%\end{align*}
%with
%\[(N1)_{fei}=\nabla_e\mathfrak{M}_{fi},(N2)_{fei}=(N1)_{efi}, (E1)^i_j=\epsilonLC^{ief}\nabla_f\mathfrak{M}_{ej}, (E2)^i_j=(E1)^j_i\,.\]
%One sees
%\[\lvert\langle N1,N2\rangle_g\rvert\leq \lvert N1\rvert_g\lvert N2\rvert_g=\lvert\nabla \mathfrak{M}\rvert_g^2\,,\]
%\[\lvert\langle E1,E2\rangle_g\rvert\leq\lvert\epsilonLC^{(\cdot)ef}\nabla_f\mathfrak{M}_{e(\cdot)}\rvert^2=\epsilonLC^{ief}\epsilonLC_{iab}\nabla_e\mathfrak{M}_{fj}\nabla^a\mathfrak{M}^{bj}=\lvert\nabla \mathfrak{M}\rvert_g^2-\nabla_e\mathfrak{M}_{fj}\nabla^f\mathfrak{M}^{ej}\leq 2\lvert\nabla \mathfrak{M}\rvert_g^2\]
%Thus, we altogether obtain
%\[\lvert\curl \mathfrak{M}\rvert_g^2\leq 2\lvert\nabla \mathfrak{M}\rvert_g^2,\quad \lvert\curl \mathfrak{M}\rvert_G^2=a^4\lvert\curl \mathfrak{M}\rvert_g^2\leq 2a^4\lvert\nabla \mathfrak{M}\rvert_g^2=2a^{-2}\lvert\nabla \mathfrak{M}\rvert_G^2\]
\end{proof}


\subsection{Commutators}\label{subsec:commutators} Herein, we collect a variety of commutators of spatial derivative operators with each other as well as with time derivatives. While these mostly follow by standard computations, we use the fact that our spatial hypersurfaces are three-dimensional to significantly simplify the spatial commutator formulas, and need to apply the rescaled equations from Proposition \ref{prop:REEq} for the time derivative formulas.

For higher order commutators, we denote by $\mathfrak{J}$ terms within the commutator formula that contribute junk terms at any point where this commutator formula is used. Furthermore, in the following, $\zeta$ denotes a scalar function on $\M$ and $\mathfrak{T}$ denotes a $\Sigma_t$-tangent, symmetric $(0,2)$-tensor, always with sufficient regularity for the equations to make sense. Moreover, recall the schematic $\ast$-notation as introduced in subsection \ref{subsubsec:schematic-notation}.

%\begin{lemma}[First order spatial commutators]\label{lem:comm-space-first}
%The following commutator formulas hold:
%\begin{subequations}
%\begin{align*}
%\numberthis\label{eq:[Lap,nabla]SF-full}[\Lap,\nabla_i]\zeta=&\,\Ric[G]_i^{\sharp n}\nabla_n\zeta\\
%%%%%%%%%%%%%%%%%%%%%%%%%%%
%\numberthis\label{eq:[Lap,nabla2]SF-full}[\Lap,\nabla_i\nabla_j]\zeta=&\,\Ric[G]^{\sharp c}_i\nabla_c\nabla_j\zeta-\nabla^{\sharp a}{\Riem[G]_{jai}}^c\nabla_a\zeta-2{\Riem[G]_{jai}}^c\nabla^{\sharp a}\nabla_c\zeta\\
%&\,-(\nabla_i\Ric[G]_j^{\sharp c})\nabla_c\zeta-\Ric[G]^{\sharp c}_j\nabla_i\nabla_c\zeta\\
%%%%%%%%%%%%%%%%%%%%%%%%
%\numberthis\label{eq:[Lap,nabla]T-full}[\Lap,\nabla_i]\mathfrak{T}_{kl}=&\,\Ric[G]^{\sharp n}_i\nabla_n \mathfrak{T}_{kl}+4{\Riem[G]_{mi(k}}^n\nabla^{\sharp m}\mathfrak{T}_{nl)}+2\left(\nabla_{(k}\Ric[G]_i^{\sharp n}\right)\mathfrak{T}_{nl)}\\
%&\,-2\left(\nabla_i\Ric[G]^{\sharp n}_{(k}\right)\mathfrak{T}_{nl)}\\
%%%%%%%%%%%%%%%%%%%%%%%
%\numberthis\label{eq:[Lap,nabla2]T-full}[\Lap,\nabla_i\nabla_j]\mathfrak{T}_{kl}=&\,\Ric[G]^{\sharp n}_i\nabla_n\nabla_j\mathfrak{T}_{kl}+{\Riem[G]_{mij}}^n\nabla^{\sharp m}\nabla_n\mathfrak{T}_{kl}+2{\Riem[G]_{mi(k}}^n\nabla^{\sharp m}\nabla_i\mathfrak{T}_{nl)}\\
%&\,+2\left(\nabla^{\sharp m}{\Riem[G]_{mi(k}}^n\right)\nabla_j\mathfrak{T}_{nl)}+2{\Riem[G]_{mi(k}}^n\nabla^{\sharp m}\mathfrak{T}_{nl)}\\
%&\,+2\left(\nabla^{\sharp m}{\Riem[G]_{mi(k}}^n\right)\nabla_j\mathfrak{T}_{nl)}+\left(\nabla_i\Ric[G]^{\sharp n}_j\right)\nabla_n\mathfrak{T}_{kl}+\Ric[G]^{\sharp n}_j\nabla_i\nabla_n\mathfrak{T}_{kl}\\
%&\,+4\left(\nabla_i{\Riem[G]_{mj(k}}^n\right)\nabla^{\sharp m}\mathfrak{T}_{nl}+4{\Riem[G]_{mj(k}}^n\nabla_i\nabla^{\sharp m}\mathfrak{T}_{nl}\\
%&\,+2\left(\nabla_i\nabla_{(k}\Ric[G]^{\sharp n}_j\right)\mathfrak{T}_{nl)}+2\left(\nabla_{(k}\Ric[G]^{\sharp n}_j\right)\nabla_i\mathfrak{T}_{nl)}\\
%&\,-2\left(\nabla_i\nabla_j\Ric[G]^{\sharp n}_{(k}\right)\mathfrak{T}_{nl)}-2\left(\nabla_j\Ric[G]^{\sharp n}_{(k}\right)\nabla_i\mathfrak{T}_{nl)}\\
%%%%%%%%%%%%%%%%%%%%%%%%%%%%%%%
%\numberthis\label{eq:[Lap,div]-full}([\Lap_G,\div_G]\mathfrak{T})_{l}=&\,-{\Ric[G]^{\sharp n}}_i\nabla_n{\mathfrak{T}^{\sharp i}}_l+\left(\nabla_l{\Ric[G]^{\sharp n}}_i\right){\mathfrak{T}^{\sharp i}}_n-\left(\nabla_i{\Ric[G]^{\sharp n}}_l\right){\mathfrak{T}^{\sharp i}}_n\\
%&\,+3{\Riem[G]_{mkl}}^n\nabla^{\sharp m}\mathfrak{T}_{cn}\\
%%%%%%%%%%%%%%%%%%%%%%%%%%%%5
%\numberthis\label{eq:[Lap,curl]-full}[\Lap_G,\curl_g]\mathfrak{T}_{ij}=&\,{\epsilonLC[g]_{(i}}^{cd}{\Ric[G]_d}^{\sharp n}\nabla_n\mathfrak{T}_{cj)}+2{\epsilonLC[g]_{(i}}^{cd}{\Riem[G]^n}_{mdc}\nabla^{\sharp m}\mathfrak{T}_{nj)}\\
%&\,+2{\epsilonLC[g]_{(i}}^{cd}{\Riem[G]_{mdj)}}^n\nabla^{\sharp m}\mathfrak{T}_{cn}+{\epsilonLC[g]_{(i}}^{cd}\left(\nabla_c{\Ric[G]^n}_d\right)\mathfrak{T}_{nj)}\\
%&\,+{\epsilonLC[g]_{(i}}^{cd}\nabla_{j)}{\Ric[G]^{\sharp n}}_d\mathfrak{T}_{cn}-{\epsilonLC[g]_{(i}}^{cd}\left(\nabla_d{\Ric[G]^{\sharp n}}_c\right)\mathfrak{T}_{nj)}\\
%&\,-{\epsilonLC[g]_{(i}}^{cd}\left(\nabla_d{\Ric[G]^{\sharp n}}_{j)}\right)\mathfrak{T}_{cn}
%\end{align*}
%\end{subequations}
%\end{lemma}
%
%\begin{proof}
%These all follow from straightforward calculations.
%\end{proof}
%
\begin{corollary}[Schematic first order spatial commutators]\label{lem:comm-space-first}
For $\zeta$ and $\mathfrak{T}$ as above, the following identities hold:
\begin{subequations}
\begin{align}
[\Lap,\nabla]\zeta=&\,\Ric[G]\ast\change{\nabla\zeta}\label{eq:[Lap,nabla]SF}\\
[\Lap,\nabla^2]\zeta=&\,\Ric[G]\ast\nabla^2\zeta+\nabla\Ric[G]\ast\nabla\zeta\label{eq:[Lap,nabla2]SF}\\
[\Lap,\nabla]\mathfrak{T}=&\,\Ric[G]\ast\nabla\mathfrak{T}+\nabla \Ric[G]\ast\mathfrak{T}\label{eq:[Lap,nabla]T}\\
[\Lap,\nabla^2]\mathfrak{T}=&\,\Ric[G]\ast\nabla^2\mathfrak{T}+\nabla\Ric[G]\ast\nabla\mathfrak{T}+\nabla^2\Ric[G]\ast\mathfrak{T}\label{eq:[Lap,nabla2]T}\\
[\Lap,\div_G]\mathfrak{T}=&\,\Ric[G]\ast\nabla\mathfrak{T}+\nabla\Ric[G]\ast\mathfrak{T}\label{eq:[Lap,div]}\\
\change{[\Lap,\curl_G]}=&\,\change{\epsilonLC[G]\ast\left(\Ric[G]\ast\nabla\mathfrak{T}+\nabla\Ric[G]\ast\mathfrak{T}\right)}\label{eq:[Lap,curl]}
\end{align}
\end{subequations}
\end{corollary}
\begin{proof}
Since we are working in three spatial dimensions, the following identity holds:
\begin{align*}
\Riem[G]_{ijkl}=&\,G_{ik}\Ric[G]_{jl}-G_{il}\Ric[G]_{jk}+G_{jl}\Ric[G]_{ik}-G_{jk}\Ric[G]_{il}\\
&\,-\frac12 (G^{-1})^{mn}\Ric[G]_{mn}(G_{ik}G_{jl}-G_{il}G_{jk})
\end{align*}
Hence, for any $I\in\N_0$, any $\nabla^I\Riem[G]$-term reduces to a sum of products and contractions of $\nabla^I\Ric[G]$ with various metric tensors that are all suppressed in schematic notation. With this in mind, the above statements are simply direct consequences of standard commutation cormulas and \eqref{eq:epsilonLC-resc}. 
\end{proof}

\begin{lemma}[Higher order spatial commutators]\label{lem:comm-space}
For $l\in\N,\,l\geq 2$, the following formulas hold (and extend to $l=1$ when dropping any term involving $\Lap^{l-2}$):
\begin{subequations}
\begin{align*}
\numberthis\label{eq:[Lap-l,nabla]SF}[\Lap^l,\nabla]\zeta=&\,\Lap^{l-1}\Ric[G]\ast\nabla\zeta+\nabla\Lap^{l-2}\Ric[G]\ast\nabla^2\zeta+\mathfrak{J}([\Lap^l,\nabla]\zeta)\\
\numberthis\label{eq:[Lap-l,nabla2]SF}[\Lap^l,\nabla^2]\zeta=&\,\nabla\Lap^{l-1}\Ric[G]\ast\nabla\zeta+\nabla^2\Lap^{l-2}\Ric[G]\ast\nabla^2\zeta+\mathfrak{J}([\Lap^l,\nabla^2]\zeta)\\
\numberthis\label{eq:[Lap-l,nabla]T}[\Lap^l,\nabla]\mathfrak{T}=&\,\nabla\Lap^{l-1}\Ric[G]\ast \mathfrak{T}+\nabla^2\Lap^{l-2}\Ric[G]\ast\nabla \mathfrak{T}+\mathfrak{J}([\Lap^l,\nabla]\mathfrak{T})\,,\\
\numberthis\label{eq:[Lap-l,nabla2]T}[\Lap^l,\nabla^2]\mathfrak{T}=&\,\nabla^2\Lap^{l-1}\Ric[G]\ast \mathfrak{T}+\nabla^3\Lap^{l-2}\Ric[G]\ast \nabla \mathfrak{T}+\mathfrak{J}([\Lap^l,\nabla^2]\mathfrak{T})\,,\\
\numberthis\label{eq:[Lap-l,div]T}[\Lap^l,\div_G]\mathfrak{T}=&\,\nabla\Lap^{l-1}\Ric[G]\ast \mathfrak{T}+\nabla^2\Lap^{l-2}\Ric[G]\ast \nabla \mathfrak{T}+\mathfrak{J}([\Lap^l,\div_G]\mathfrak{T})\,,\\
\numberthis\label{eq:[Lap-l,curl]}\change{[\Lap^l,\curl_G]\mathfrak{T}=&\change{\,\epsilonLC[G]\ast\left(\nabla\Lap^{l-1}\Ric[G]\ast \mathfrak{T}+\nabla^2\Lap^{l-2}\Ric[G]\ast\nabla \mathfrak{T}\right)+\mathfrak{J}([\Lap^l,\curl_G]\mathfrak{T})}}
\end{align*}
with junk terms, where $\mathcal{I}=I_1+\dots+I_{l-m}$,
{\scriptsize 
\begin{align*}
\mathfrak{J}([\Lap^l,\nabla]\zeta)=&\,\sum_{\substack{I_1+I_\zeta=2(l-1),\\\,I_\zeta\geq 2}}\nabla^{I_1}\Ric[G]\ast\nabla^{I_\zeta+1}\zeta+\sum_{m=0}^{l-2}\sum_{\mathcal{I}+I_{\zeta}=2m}\nabla^{I_1}\Ric[G]\ast\dots\ast\nabla^{I_{l-m}}\Ric[G]\ast\nabla^{I_\zeta+1}\zeta\\
\mathfrak{J}([\Lap^l,\nabla^2]\zeta)=&\,\sum_{\substack{I_1+I_\zeta=2(l-1)+1,\\I_1,I_\zeta\geq 2}}\nabla^{I_1}\Ric[G]\ast\nabla^{I_\zeta+1}\zeta+\sum_{m=0}^{l-2}\sum_{\mathcal{I}+I_\zeta=2m+1}\nabla^{I_1}\Ric[G]\ast\dots\ast\nabla^{I_{l-m}}\Ric[G]\ast\nabla^{I_\zeta+1}\zeta\\
\mathfrak{J}([\Lap^l,\nabla]\mathfrak{T})=&\,\sum_{\substack{I_1+I_\mathfrak{T}=2(l-1)+1,\\I_\mathfrak{T}\geq2}}\nabla^{I_1}\Ric[G]\ast\nabla^{I_\mathfrak{T}}\mathfrak{T}+\sum_{m=0}^{l-2}\sum_{\mathcal{I}+I_\mathfrak{T}=2m+1}\nabla^{I_1}\Ric[G]\ast\dots\ast\nabla^{I_{l-m}}\Ric[G]\ast\nabla^{I_\mathfrak{T}}\mathfrak{T}\\
\mathfrak{J}([\Lap^l,\nabla^2]\mathfrak{T})=&\,\sum_{\substack{I_1+I_\mathfrak{T}=2l,\\I_\mathfrak{T}\geq2}}\nabla^{I_1}\Ric[G]\ast\nabla^{I_\mathfrak{T}}\mathfrak{T}+\sum_{m=0}^{l-2}\sum_{\mathcal{I}+I_\mathfrak{T}=2m+2}\nabla^{I_1}\Ric[G]\ast\dots\ast\nabla^{I_{l-m}}\Ric[G]\ast\nabla^{I_\mathfrak{T}}T\\
\mathfrak{J}([\Lap^l,\div_G]\mathfrak{T})=&\,\sum_{\substack{I_1+I_\mathfrak{T}=2(l-1)+1,\\I_\mathfrak{T}\geq2}}\nabla^{I_1}\Ric[G]\ast\nabla^{I_\mathfrak{T}}\mathfrak{T}+\sum_{m=0}^{l-2}\sum_{\mathcal{I}+I_\mathfrak{T}=2m+1}\nabla^{I_1}\Ric[G]\ast\dots\ast\nabla^{I_{l-m}}\Ric[G]\ast\nabla^{I_\mathfrak{T}}\mathfrak{T}\\
\change{\mathfrak{J}([\Lap^l,\curl_G]\mathfrak{T})}=&\change{\epsilonLC[G]\ast\left[\sum_{\substack{I_1+I_\mathfrak{T}=2(l-1)+1,\\I_\mathfrak{T}\geq2}}\nabla^{I_1}\Ric[G]\ast\nabla^{I_\mathfrak{T}}\mathfrak{T}+\sum_{m=0}^{l-2}\sum_{\substack{\mathcal{I}+\\+I_\mathfrak{T}=2m+1}}\nabla^{I_1}\Ric[G]\ast\dots\ast\nabla^{I_{l-m}}\Ric[G]\ast\nabla^{I_\mathfrak{T}}\mathfrak{T}\right]}\,.
\end{align*}}
\end{subequations}
\end{lemma}
\begin{proof}
The formulas follow by applying the formulas from Lemma \ref{lem:comm-space-first} inductively.
\end{proof}



\begin{lemma}[Time derivative commutators]\label{lem:com-time-first} With respect to a solution to the Einstein scalar-field system as in Proposition \ref{prop:eq}, the following commutator formulas hold:
\begin{subequations}
\begin{align*}
\numberthis\label{eq:[del-t,nabla]zeta}[\del_t,\nabla_i]\zeta=&\,0\\
\numberthis\label{eq:[del-t,nabla-sharp]zeta}[\del_t,\nabla^{\sharp i}]\zeta=&\,2(N+1)a^{-3}\Sigma^{\sharp ij}\nabla_j\zeta-2N\frac{\dot{a}}a\nabla^{\sharp i}\zeta\\
\numberthis\label{eq:[del-t,Lap]zeta}[\del_t,\Lap]\zeta=&\,2(N+1)a^{-3}\langle\Sigma,\nabla^2\zeta\rangle_G-2N\frac{\dot{a}}a\Lap\zeta\\
&\,-2(N+1)a^{-3}\langle\div_G\Sigma,\nabla\zeta\rangle_G-2a^{-3}\langle\Sigma,\nabla N\nabla\zeta\rangle_G+\frac{\dot{a}}a\langle\nabla N,\nabla\zeta\rangle_G\\
%&\,-16\pi a^{-3}(N+1)(\Psi+C)\langle\nabla\phi,\nabla\zeta\rangle_G+2a^{-3}\langle\Sigma,\nabla N\nabla\zeta\rangle_G+\frac{\dot{a}}a\langle\nabla N,\nabla\zeta\rangle_G\\
\numberthis\label{eq:[del-t,nabla]T}
[\del_t,\nabla]\mathfrak{T}=&\,a^{-3}\left((N+1)\,\nabla\Sigma+\Sigma\ast\nabla N\right)\ast\mathfrak{T}+\frac{\dot{a}}a\,\nabla N\ast\mathfrak{T}\\
%[\del_t,\nabla_i]\mathfrak{T}_{kl}%%=&\,-2\del_t\Gamma^m_{i(k}T_{ml)}\\
%=&\,\left\{2(N+1)a^{-3}\left[\nabla_i\Sigma_{(k}^m+\nabla_{(k}\Sigma_i^{\sharp m}-\nabla^m\Sigma_{i(k}\right]\right.\\
%&\ +2a^{-3}\left.\left[\nabla_i N\Sigma^{\sharp m}_{(k}+\change{\Sigma^{\sharp m}_i\nabla_{(k}N}-\nabla^{\sharp m}N\Sigma_{i(k}\right]\right\}\change{\mathfrak{T}_{l)m}}\\
%&-\frac{\dot{a}}a\left[\nabla_iN\mathfrak{T}_{kl}+\nabla_{(k}N\change{\mathfrak{T}_{l)i}}-\nabla^{\sharp m}NG_{i(k}\change{\mathfrak{T}_{l)m}}\right]\\
\numberthis\label{eq:[del-t,Lap]T}[\del_t,\Lap]\mathfrak{T}%&\,(\del_tG^{-1})^{ij}\nabla_i\nabla_j\mathfrak{T}_{kl}-[(G^{-1})^{ij}\del_t\Gamma_{ij}^m]\nabla_m\mathfrak{T}_{kl}\\
%&-2\del_t\Gamma_{i(k}^m\nabla^{\sharp i}\mathfrak{T}_{ml)}-2\nabla^{\sharp i}\del_t\Gamma^m_{i(k}\mathfrak{T}_{ml)}-2\del_t\Gamma^m_{i(k}\nabla^{\sharp i}\mathfrak{T}_{ml)}\\
=&\,a^{-3}(N+1)\Sigma\ast\nabla^2\mathfrak{T}+\frac{\dot{a}}aN\change{\Lap\mathfrak{T}}+a^{-3}\nabla((N+1)\Sigma)\ast\nabla \mathfrak{T}+\frac{\dot{a}}a\nabla N\ast\nabla \mathfrak{T}\\
&\,+a^{-3}\nabla^2((N+1)\Sigma)\ast \mathfrak{T}-\frac{\dot{a}}a\nabla^2N\ast \mathfrak{T}
\end{align*}
\end{subequations}
\end{lemma}
\begin{proof}
Equation \eqref{eq:[del-t,nabla]zeta} is simply that coordinate derivatives commute, and \eqref{eq:[del-t,nabla-sharp]zeta} follows by applying \eqref{eq:REEqG-1} and the product rule.\\
For the \change{commutators \eqref{eq:[del-t,Lap]zeta}, }\eqref{eq:[del-t,nabla]T} and \eqref{eq:[del-t,Lap]T}, we write out the covariant derivatives in local coordinates, apply the product rule, and then the evolution equations \eqref{eq:REEqG-1} and \eqref{eq:REEqChr} for the inverse metric and Christoffel symbols. \delete{For \eqref{eq:[del-t,Lap]zeta} [...]}%, we do the same at first and get the following:
%\begin{align*}
%[\del_t,\Lap]\zeta=&\,(\del_tG^{-1})^{ij}\nabla_i\nabla_j\phi-(G^{-1})^{ij}\Gamma[G]^{k}_{ij}\nabla_k\zeta\\
%=&\,2(N+1)a^{-3}\langle\Sigma,\nabla^2\zeta\rangle_G-2N\frac{\dot{a}}a\Lap\zeta\\
%&\,-2(N+1)a^{-3}\langle\div_G\Sigma,\nabla\zeta\rangle_G-2a^{-3}\langle\Sigma,\nabla N\nabla\zeta\rangle_G+\frac{\dot{a}}a\langle\nabla N,\nabla\zeta\rangle_G
%\end{align*}
%\eqref{eq:[del-t,Lap]zeta} now follows with the rescaled momentum constraint \eqref{eq:REEqMom}.
\end{proof}

\begin{lemma}[High order time derivative commutators]\label{lem:com-time}
For $l\in\N,\,l\geq 2$, the time derivative commutators take the form
\begin{subequations}
\begin{align*}
\numberthis\label{eq:[del-t,Lap-l]zeta}[\del_t,\Lap^l]\zeta=&\,2a^{-3}(N+1)\langle\Sigma,\nabla^2\Lap^{l-1}\zeta\rangle_G\change{+a^{-3}\nabla\Sigma\ast\nabla^3\Lap^{l-2}\zeta}\\
%&\,+a^{-3}(N+1)(\Psi+C)\left[\nabla\phi\ast\nabla\Lap^{l-1}\zeta+\nabla\Lap^{l-1}\phi\ast\nabla\zeta\right]\\
%&\,+a^{-3}\langle\nabla\phi,\nabla\zeta\rangle_G\left[\Lap^{l-1}N(\Psi+C)+(N+1)\Lap^{l-1}\Psi\right]\\
&\change{\,-2(N+1)a^{-3}\langle\div_G\Lap^{l-1}\Sigma,\nabla\zeta\rangle_G+(N+1)a^{-3}\nabla^{2l-3}\Ric\ast\Sigma\ast\nabla\zeta+\mathfrak{J}([\del_t,\Lap^l]\zeta),}\\
%%%%%%%%%%%%%%%%%%%%%%%%%%%%%%%
\numberthis\label{eq:[del-t,nabla-Lap-l]zeta}[\del_t,\nabla\Lap^l]\zeta=&\,2a^{-3}(N+1)\langle\Sigma,\nabla^3\Lap^{l-1}\zeta\rangle_G+a^{-3}(N+1)\nabla\Sigma\ast\nabla^{2l}\zeta\\
%&\,+a^{-3}(N+1)(\Psi+C)\left[\nabla\phi\ast\nabla^2\Lap^{l-1}\zeta+\nabla\Lap^{l-1}\phi\ast\nabla\zeta\right]\\
%&+a^{-3}\langle\nabla\phi,\nabla\zeta\rangle_G\left[\nabla\Lap^{l-1}N(\Psi+C)+(N+1)\Lap^{l-1}\Psi\right]\\
&\,-2(N+1)a^{-3}\langle \nabla\div_G\Lap^{l-1}\Sigma,\nabla\zeta\rangle_G\\
&\,+\frac{\dot{a}}a\langle\nabla^2\Lap^{l-1}N,\nabla\zeta\rangle_G+(N+1)a^{-3}\nabla^{2l-2}\Ric[G]\ast\Sigma\ast\nabla\zeta\\
&\,+\mathfrak{J}([\del_t,\nabla\Lap^l]\zeta)\\
%%%%%%%%%%%%%%%%%%%%%%%
%%%%%%%%%%%%%%%%%%%%%%%%%%%%%%%%%
\numberthis\label{eq:[del-t,Lap-l]T}[\del_t,\Lap^l]\mathfrak{T}=&\,a^{-3}\left(\Sigma\ast\nabla^2\Lap^{l-1}\mathfrak{T}+\nabla\Sigma\ast\nabla^3\Lap^{l-2}\mathfrak{T}+\nabla \mathfrak{T}\ast\nabla\Lap^{l-1}\Sigma+\mathfrak{T}\ast\Lap^{l}\Sigma\right)\\
&\,+a^{-3}\left((N+1)\Sigma\ast \mathfrak{T}\ast\nabla^2\Lap^{l-2}\Ric[G]+\nabla((N+1)\Sigma\ast \mathfrak{T})\ast\nabla^{2l-3}\Ric[G]\right)\\
&\,+\frac{\dot{a}}a\Lap^lN\cdot \mathfrak{T}+\frac{\dot{a}}a\nabla\Lap^{l-1}N\ast\nabla \mathfrak{T}+\mathfrak{J}([\del_t,\Lap^l])\mathfrak{T}\,,\\
%&\,+a^{-3}T(\Psi+C)(N+1)\Lap^l\phi+a^{-3}(\Psi+C)T\ast\nabla\phi\ast\nabla\Lap^{l-1}N\\
%&\,+a^{-3}(N+1)\nabla\Lap^{l-1}\Psi\cdot T\ast\nabla\phi+a^{-3}\nabla\Lap^{l-1}T\ast\nabla\phi(N+1)(\Psi+C)\\
%%%%%%%%%%%%%%%%%%%%%%%%%%%%%%
\numberthis\label{eq:[del-t,nabla-Lap-l]T}[\del_t,\nabla\Lap^l]\mathfrak{T}=&\,a^{-3}\nabla\Sigma\ast\Lap^l\mathfrak{T}+a^{-3}(N+1)\Sigma\ast\nabla^3\Lap^{l-1}\mathfrak{T}+a^{-3}(N+1)\mathfrak{T}\ast\nabla\Lap^l\Sigma\\
&+\frac{\dot{a}}a\nabla\Lap^lN\ast \mathfrak{T}+\frac{\dot{a}}a\nabla^2\Lap^{l-1}N\ast\nabla \mathfrak{T}+a^{-3}(N+1)\Sigma\ast\nabla^3\Lap^{l-2}\Ric[G]\ast \mathfrak{T}\\
&\,+\mathfrak{J}([\del_t,\nabla\Lap^l]\mathfrak{T})\,,
\end{align*}
where the junk terms are, where $\mathcal{I}=\sum_{i=1}^{l-m-1}I_i$,
{\scriptsize
\change{\begin{align*}
%%%%%%%%%%%%%%
\numberthis\label{eq:[del-t,Lap]zeta-junk}\mathfrak{J}([\del_t,\Lap^l]\zeta)=&\,a^{-3}\sum_{\substack{I_N+I_\Sigma+I_\zeta=2(l-1)\\I_\zeta\leq2(l-2)}}\nabla^{I_N}(N+1)\ast\nabla^{I_\Sigma}\Sigma\ast\nabla^{I_\zeta+2}\zeta\\
%&\,+a^{-3}\sum_{\substack{I_N+I_\Psi+I_\phi+I_\zeta=2(l-1)\\\,I_N,I_\Psi,I_\phi,I_\zeta\neq2(l-1)}}\nabla^{I_N}(N+1)\ast\nabla^{I_\Psi}(\Psi+C)\ast\nabla^{I_\phi+1}\phi\ast\nabla^{I_\zeta+1}\zeta\\
&\,+\frac{\dot{a}}a\sum_{\substack{I_N+I_\zeta=2l\\ I_\zeta\geq 2}}\nabla^{I_N}N\ast\nabla^{I_\zeta}\zeta\\
&\,+a^{-3}\sum_{m=0}^{l-2}\sum_{\substack{I_N+I_\Sigma+I_\zeta+\mathcal{I}=2m}}\nabla^{I_N}(N+1)\ast\nabla^{I_\Sigma}\Sigma\ast\nabla^{I_1}\Ric[G]\ast\dots\ast\nabla^{I_{l-m-1}}\Ric[G]\ast\nabla^{I_\zeta+2}\zeta\\
&\,+a^{-3}\sum_{m=0}^{l-2}\sum_{\substack{I_N+I_\Sigma+I_\zeta+\mathcal{I}=2m\\I_1\neq 2l-4}}\nabla^{I_N}(N+1)\ast\nabla^{I_\Sigma}\Sigma\ast\nabla^{I_1+1}\Ric[G]\ast\dots\ast\nabla^{I_{l-m-1}}\Ric[G]\ast\nabla^{I_\zeta+1}\zeta\\
%&\,+a^{-3}\sum_{m=0}^{l-2}\sum_{\substack{I_N+I_\Psi+I_\phi+I_\zeta+\mathcal{I}=2m}}\nabla^{I_N}(N+1)\ast\nabla^{I_\Psi}(\Psi+C)\ast\nabla^{I_1}\Ric[G]\ast\dots\ast\nabla^{I_{l-m-1}}\Ric[G]\\
%&\,\phantom{+a^{-3}\sum_{m=0}^{l-2}\sum}\ast\nabla^{{I_\phi}+1}\phi\ast\nabla^{I_\zeta+1}\zeta\\
&\,+\frac{\dot{a}}a\sum_{m=0}^{l-1}\sum_{\substack{I_N+\mathcal{I}+I_\zeta=2m-1\\\,I_\zeta\neq2(l-1)}}\nabla^{I_N}N\ast\nabla^{I_1}\Ric[G]\ast\dots\ast\nabla^{I_{l-m-1}}\Ric[G]\ast\nabla^{I_\zeta+1}\zeta\\
%%%%%%%%%%%%%%%%%%%%%%%%%%%%%%%%%%%%%%%
\numberthis\label{eq:[del-t,nabla-Lap-l]zeta-junk}\mathfrak{J}([\del_t,\nabla\Lap^l]\zeta)=&\,\frac{\dot{a}}a\sum_{\substack{I_N+I_\zeta=2l\\\,I_\zeta\neq 0}}\nabla^{I_N}N\ast\nabla^{I_\zeta+1}\zeta+a^{-3}\sum_{\substack{I_N+I_\Sigma+I_\zeta=2(l-1)+1\\(I_\Sigma, I_\zeta)\neq(0,2(l-1)+1),(1,2(l-1))}}\nabla^{I_N}(N+1)\ast\nabla^{I_\Sigma}\Sigma\ast\nabla^{I_\zeta+1}\zeta\\
%&\,+a^{-3}\sum_{\substack{I_N+I_\Psi+I_\phi+I_\zeta=2(l-1)+1\\\,I_N,I_\Psi,I_\phi,I_\zeta\neq2(l-1)+1}}\nabla^{I_N}(N+1)\ast\nabla^{I_\Psi}(\Psi+C)\ast\nabla^{I_\phi+1}\phi\ast\nabla^{I_\zeta+1}\zeta\\}
&\,+a^{-3}\sum_{m=0}^{l-2}\sum_{\substack{I_N+I_\Sigma+I_\zeta+\mathcal{I}=2m+1}}\nabla^{I_N}(N+1)\ast\nabla^{I_\Sigma}\Sigma\ast\nabla^{I_1}\Ric[G]\ast\dots\ast\nabla^{I_{l-m-1}}\Ric[G]\ast\nabla^{I_\zeta+2}\zeta\\
&\,+a^{-3}\sum_{m=0}^{l-2}\sum_{\substack{I_N+I_\Sigma+I_\zeta+\mathcal{I}=2m+1\\I_1\neq 2l-3}}\nabla^{I_N}(N+1)\ast\nabla^{I_\Sigma}\Sigma\ast\nabla^{I_1+1}\Ric[G]\ast\dots\ast\nabla^{I_{l-m-1}}\Ric[G]\ast\nabla^{I_\zeta+1}\zeta\\
%&\,+a^{-3}\sum_{m=0}^{l-2}\sum_{\substack{I_N+I_\Psi+I_\phi+I_\zeta+\mathcal{I}=2m+1}}\nabla^{I_N}(N+1)\ast\nabla^{I_\Psi}(\Psi+C)\ast\nabla^{I_1}\Ric[G]\ast\dots\ast\nabla^{I_{l-m-1}}\Ric[G]\\
%&\,\phantom{+a^{-3}\sum_{m=0}^{l-2}\sum{fill}}\ast\nabla^{{I_\phi}+1}\phi\ast\nabla^{I_\zeta+1}\zeta\\}
&\,+\frac{\dot{a}}a\sum_{m=0}^{l-1}\sum_{\substack{I_N+\mathcal{I}+I_\zeta=2m\\\,I_\zeta\neq2(l-1)}}\nabla^{I_N}N\ast\nabla^{I_1}\Ric[G]\ast\dots\ast\nabla^{I_{l-m-1}}\Ric[G]\ast\nabla^{I_\zeta+1}\zeta\\}
%%%%%%%%%%%%%%%%%%%%%%%%%%%%%%%%%%%%%%%
\numberthis\label{eq:[del-t,Lap-l]T-junk}\mathfrak{J}([\del_t,\Lap^l])\mathfrak{T}=&\,a^{-3}\sum_{I_N+I_\Sigma+I_\mathfrak{T}=2l}\nabla^{I_N}N\ast\nabla^{I_\Sigma}\Sigma\ast\nabla^{I_\mathfrak{T}}\mathfrak{T}+a^{-3}\sum_{\substack{I_\Sigma+I_\mathfrak{T}=2l\\I_\Sigma,I_\mathfrak{T}\geq 2}}\nabla^{I_\Sigma}\Sigma\ast\nabla^{I_\mathfrak{T}}\mathfrak{T}\\
&\,+a^{-3}\sum_{m=0}^{l-1}\sum_{\substack{I_N+I_\Sigma+I_\mathfrak{T}+\mathcal{I}=2m\\ I_1<2l-3}}\nabla^{I_N}(N+1)\ast\nabla^{I_\Sigma}\Sigma\ast\nabla^{I_1}\Ric[G]\ast\dots\ast\nabla^{I_{l-m-1}}\Ric[G]\ast\nabla^{I_\mathfrak{T}}\mathfrak{T}\\
&\,+\frac{\dot{a}}a\sum_{I_N+I_\mathfrak{T}=2l,\,I_\mathfrak{T}\geq 2}\nabla^{I_N}N\ast\nabla^{I_\mathfrak{T}}\mathfrak{T}\\
%&\,+a^{-3}\sum_{\substack{I_N+I_\Psi+I_\phi+I_\mathfrak{T}=2l-1\\I_i\neq0}}\nabla^{I_N}(N+1)\ast\nabla^{I_\Psi}(\Psi+C)\ast\nabla^{I_\phi+1}\phi\ast\nabla^{I_\mathfrak{T}}\mathfrak{T}\\
%&\,+a^{-3}\sum_{m=0}^{l-2}\sum_{\substack{I_N+I_\Psi+I_\phi+I_\mathfrak{T}+\mathcal{I}=2m-1}}\nabla^{I_N}(N+1)\ast\nabla^{I_\Psi}(\Psi+C)\ast\nabla^{I_\phi+1}\phi\ast\nabla^{I_1}\Ric\ast\dots\ast\nabla^{I_{l-m-1}}\Ric\ast\nabla^{I_\mathfrak{T}}\mathfrak{T}
\numberthis\label{eq:[del-t,nabla-Lap-l]T-junk}\mathfrak{J}([\del_t,\nabla\Lap^l]\mathfrak{T})=&\,a^{-3}\Sigma\ast\nabla N\ast\Lap^l\Ric[G]+a^{-3}N\ast\nabla\Sigma\ast\Lap^l\mathfrak{T}+\frac{\dot{a}}a\nabla N\ast\Lap^l\mathfrak{T}+\nabla\mathfrak{J}([\del_t,\Lap^l]\mathfrak{T})
\end{align*}}
\end{subequations}
\noindent We can extend the formulas to $l=1$ by dropping any term which would contain negative powers of $\Lap$ or a multiindex of negative order.
\end{lemma}
\begin{proof}
This follows by iteratively applying the commutators in Lemma \ref{lem:com-time-first}.
\end{proof}


%\begin{remark}
%\todo{The commutator representations for $[\del_t,\Lap^l]$ acting on scalar functions and tensors differ significantly on first glance -- the discrepancy basically arises from us using the Momentum constraint to replace many terms in the first commutator, while simply treating $\div_G\Sigma$ as a $\nabla\Sigma$-term in the latter. The reason for this discrepancy is that, when using this constraint equation, we lose one derivative in our solution variables in exchange for introducing $\nabla\phi$ where it wasn't before, where we need to introduce a term generating the power $a^4$ if we want to deal with it directly. This is fine when working with scalar field energies, since the order will either be low enough that we can use [insert: ftoc-est] to work around this, or we are precisely at that scaled term in the scalar field energy anyhow. However, for the other energies, we don't have that luxury, while just keeping $\div_G\Sigma$ doesn't generate worse terms than occur already. We may also drop this trick for the scalar field energy later on, but for now, this is more precise there, so we'll keep it.}
%\end{remark}

While all of the above commutators will be essential for the mainline argument, the a priori estimates require the following commutators:

\begin{lemma}[Auxiliary commutators]\label{lem:aux-comm} Let $J\in\N$. Then, we have:
\begin{subequations}
\begin{align*}
\numberthis\label{eq:commutator-aux-scalar}[\del_t,\nabla^J]\zeta=&\,a^{-3}\sum_{I_N+I_\Sigma+I_\zeta=l-1,\,I_\zeta<J-1}\nabla^{I_N}(N+1)\ast\nabla^{I_\Sigma}\Sigma\ast\nabla^{I_\zeta+1}\zeta\\
&\,+\frac{\dot{a}}a\sum_{I_N+I_\zeta=J-1,I_N>0}\nabla^{I_N}N\ast\nabla^{I_\zeta+1}\zeta\\
\numberthis\label{eq:commutator-aux-tensor}[\del_t,\nabla^J]\mathfrak{T}=&\,a^{-3}\sum_{I_N+I_\Sigma+I_{\mathfrak{T}}=J,\,I_\mathfrak{T}< J}\nabla^{I_N}(N+1)\ast\nabla^{I_\Sigma}\Sigma\ast\nabla^{I_\mathfrak{T}}\mathfrak{T}\\
&\,+\frac{\dot{a}}a\sum_{I_N+I_\mathfrak{T}=J,I_N>0}\nabla^{I_N}N\ast\nabla^{I_\mathfrak{T}}\mathfrak{T}\\
\end{align*}
\end{subequations}
\end{lemma}
\begin{proof}
For $J=1$, this has already been shown in \eqref{eq:[del-t,nabla]zeta} and \eqref{eq:[del-t,nabla]T}. For higher orders, the formulas follow from a straightforward induction argument using that, in local coordinates, we schematically have
\begin{align*}
[\del_t,\nabla^J]\zeta=[\del_t,\nabla]\nabla^{J-1}\zeta+\nabla[\del_t,\nabla^{J-1}]\zeta=(\del_t\Gamma[G])\ast\nabla^{J-1}\zeta+\nabla[\del_t,\nabla^{J-1}]\zeta
\end{align*}
and analogously replacing $\zeta$ with $\mathfrak{T}$. 
\end{proof}

\subsection{Borderline and junk terms}\label{subsec:error-terms}

\begin{definition}[Error terms]\label{def:error-terms} Let $L\in2\N,\,L\geq 2$. Then, the error terms in the Laplace-commuted equations stated in Lemma \ref{lem:laplace-commuted-eq} take the following form:\\

\noindent For the constraint equations, we have
{\scriptsize \begin{subequations}
\begin{align*}
\numberthis\label{eq:comeq-mom-div-junk}\mathfrak{M}_{L,Junk}=&\,-8\pi(\Psi+C)\nabla\Lap^{\frac{L}2-2}\Ric[G]\ast\nabla^2\phi+\nabla^{L-2}\Ric[G]\ast\nabla\Sigma+\underbrace{\nabla^{L-3}\Ric[G]\ast\nabla^2\Sigma}_{\text{if }L\neq 2}\\
&\,+\sum_{I_\Psi+I_\phi=L,\,I_\Psi\neq 0}\nabla^{I_\Psi}\Psi\ast\nabla^{I_\phi+1}\phi+8\pi(\Psi+C)\mathfrak{J}([\Lap^{\frac{L}2},\nabla]\phi)-\mathfrak{J}([\Lap^{\frac{L}2},\div_G]\Sigma)\\
\numberthis\label{eq:comeq-mom-curl-junk}\change{\tilde{\mathfrak{M}}_{L,Junk}=}&\change{\,\underbrace{-\epsilonLC[G]\ast\nabla^{L-3}\Ric[G]\ast\nabla\Sigma}_{\text{if }L\neq 2}-\mathfrak{J}([\Lap^\frac{L}2,\curl_G]\Sigma)}\\
%%%%%%%%%%%%%%%%%%%%%%%%%%%%%%%
\mathfrak{H}_{L,Border}=&\,a^{-4}\left[\Sigma\ast\Lap^{\frac{L}2}\Sigma+\nabla\Sigma\ast\nabla^{L-1}\Sigma\right] \numberthis\label{eq:comeq-Ham-border}\\
%%%%%%%%%%%%%%%%%%%%%%%%%%%%%%%
\mathfrak{H}_{L,Junk}=&\,\sum_{I_1+I_2=L}\nabla^{I_1+1}\phi\ast\nabla^{I_2+1}\phi+a^{-4}\sum_{I_1+I_2=L,I_i\geq2}\nabla^{I_1}\Sigma\ast\nabla^{I_2}\Sigma \numberthis\label{eq:comeq-Ham-junk}\\
&\,+\Lap^\frac{L}2\left[\frac{4\pi}3\lvert\nabla\phi\rvert_G^2+\frac{8\pi}3a^{-4}\Psi^2+\frac{16\pi}3Ca^{-4}\Psi\right]\cdot G
%=&\,\sum_{I_1+I_2=L}\nabla^{I_1+1}\phi\ast\nabla^{I_2+1}\phi+a^{-4}\sum_{I_1+I_2=L,I_i\geq2}\nabla^{I_1}\Sigma\ast\nabla^{I_2}\Sigma\\
%&\,\left[\frac{8\pi}3\langle\nabla\phi,\nabla\Lap^{\frac{L}2}\phi\rangle_G+\frac{16\pi}3(\Psi+C)a^{-4}\Lap^{\frac{L}2}\Psi\right]\cdot G\\
%&\sum_{I_1+I_2=L,I_i>0}\nabla^{I_1}\Psi\ast\nabla^{I_2}\Psi
\end{align*}
\end{subequations}}
\noindent The lapse equation error terms are
\begin{subequations}
{\scriptsize \begin{align*}
\mathfrak{N}_{L,Border}=&\,a^{-4}(N+1)\left(\Sigma\ast\Lap^{\frac{L}2}\Sigma+\nabla\Sigma\ast\nabla^{L-1}\Sigma+\Psi\ast\Lap^{\frac{L}2}\Psi+\nabla\Psi\ast\nabla^{L-1}\Psi\right) \numberthis \label{eq:comeq-lapse-border}\\
&\,+a^{-4}\left[\lvert\Sigma\rvert_G^2+\Psi^2+\Psi\right]\ast\Lap^{\frac{L}2}N+a^{-4}\nabla\left[\lvert\Sigma\rvert_G^2+\Psi^2+\Psi\right]\ast\nabla^{L-1}N\,\\
\mathfrak{N}_{L,Junk}=&\,a^{-4}\sum_{\substack{I_N+I_1+I_2=L;\\I_N\leq L-2;\,I_N>0\text{ or }I_1\leq I_2\leq L-2}}\nabla^{I_N}(N+1)\ast\left(\nabla^{I_1}\Sigma\ast\nabla^{I_2}\Sigma+\nabla^{I_1}\Psi\ast\nabla^{I_2}\Psi\right) \numberthis\label{eq:comeq-lapse-junk}\\
&\,+a^{-4}N\ast\Lap^{\frac{L}2}\Psi+a^{-4}\sum_{I_N+I_\Psi=L;\,I_\Psi\geq 2,\,I_N\geq 1}\nabla^{I_N}N\ast\nabla^{I_\Psi}\Psi\,,
\end{align*}}
as well as
{\scriptsize \begin{align*}
\numberthis \label{eq:comeq-lapse-border-odd}\mathfrak{N}_{L+1,Border}=&\,a^{-4}(N+1)\Bigr(\Sigma\ast\nabla\Lap^{\frac{L}2}\Sigma+\nabla\Sigma\ast\nabla^{L}\Sigma+\nabla^2\Sigma\ast\nabla^{L-1}\Sigma+\Psi\ast\nabla\Lap^{\frac{L}2}\Psi\\
&\,+\nabla\Psi\ast\nabla^{L}\Psi+\nabla^2\Psi\ast\nabla^{L-1}\Psi\Bigr)+a^{-4}\left[\lvert\Sigma\rvert_G^2+\Psi^2+\Psi\right]\ast\nabla\Lap^{\frac{L}2}N\\
&\,+\nabla\Psi\ast\nabla^{L}N+\nabla^2\Psi\ast\nabla^{L-1}\Psi\,\\
\mathfrak{N}_{L+1,Junk}=&\,a^{-4}\sum_{\substack{I_N+I_1+I_2=L;\\I_N<L+1;\,I_N>0\text{ or }I_1\geq I_2>2}}\nabla^{I_N}(N+1)\ast\left(\nabla^{I_1}\Sigma\ast\nabla^{I_2}\Sigma+\nabla^{I_1}\Psi\ast\nabla^{I_2}\Psi\right) \numberthis\label{eq:comeq-lapse-junk-odd}\\
&\,+a^{-4}N\ast\nabla\Lap^{\frac{L}2}\Psi+a^{-4}\sum_{I_N+I_\Psi=L+1;\,I_N,I_\Psi>2}\nabla^{I_N}N\ast\nabla^{I_\Psi}\Psi
\end{align*}}
\end{subequations}
whereas the scalar field error terms read
\begin{subequations}
{\scriptsize \begin{align*}
\numberthis\label{eq:comeq-Psi-even-border}\mathfrak{P}_{L,Border}=&\,-3\Psi\frac{\dot{a}}a\Lap^{\frac{L}2}N+\frac{\dot{a}}a\nabla\Psi\ast\nabla^{L-1}N +2a^{-3}(N+1)\langle\Sigma,\nabla^2\Lap^{\frac{L}2-1}\Psi\rangle_G+2a^{-3}(N+1)\nabla^{L-3}\Ric\ast\Sigma\ast\nabla\Psi\\
&\,\change{-2a^{-3}(N+1)\langle\div_G\Lap^{\frac{L}2-1}\Sigma,\nabla\Psi\rangle_G+a^{-3}(N+1)\nabla\Sigma\ast\nabla^3\Lap^{\frac{L}2-2}\Psi}\\
%&\,+a^{-3}(N+1)(\Psi+C)\left[\nabla\phi\ast\nabla\Lap^{\frac{L}2-1}\Psi+\nabla\Lap^{\frac{L}2-1}\phi\ast\nabla\Psi\right]\\
%&\,+a^{-3}(\Psi+C)\nabla\Lap^{\frac{L}2-1}N\ast\nabla\phi\ast\nabla\Psi+a^{-3}(N+1)\nabla\Psi\ast\Lap^{\frac{L}2-1}\Psi\ast\nabla\phi\\
%%%%%%%%%%%%%%%%%%%%%%%%%%%%%%%
\mathfrak{P}_{L,Junk}=&\,\frac{\dot{a}}a\sum_{I_N+I_\Psi=L,\,I_\Psi\geq 2}\nabla^{I_N}N\ast\nabla^{I_\Psi}\Psi+a\sum_{I_N+I_\phi=L+1,\,I_N,I_\phi\neq 0}\nabla^{I_N}N\ast\nabla^{I_\phi+1}\phi+\mathfrak{J}([\del_t,\Lap^\frac{L}2]\Psi)\numberthis\label{eq:comeq-Psi-even-junk}\\
%%%%%%%%%%%%%%%%%%%%%%%%%%%%%%%%
\numberthis\label{eq:comeq-Q-even-border}\mathfrak{Q}_{L,Border}=&\,a^{-3}\Psi\nabla\Lap^{\frac{L}2}N+a^{-3}(N+1)\Sigma\ast\nabla^3\Lap^{\frac{L}2-1}\phi\change{+a^{-3}(N+1)\nabla^{L}\phi\ast\nabla \Sigma}
%+a^{-3}(N+1)(\Psi+C)\nabla\phi\ast\nabla^2\Lap^{\frac{L}2-1}\phi}
\\
%%%%%%%%%%%%%%%%%%%%%%%%%%%%%%%%%%%%
\mathfrak{Q}_{L,Junk}=&\,a^{-3}\sum_{I_N+I_\Psi=L+1,\,I_N,I_\Psi\neq 0}\nabla^{I_N}N\ast\nabla^{I_\Psi}\Psi+a^{-3}\nabla\Lap^{\frac{L}2-1}N\ast\nabla^2\phi\ast\Sigma\numberthis\label{eq:comeq-Q-even-junk}\\
&\,\change{+a^{-3}(N+1)\nabla^2\Lap^{\frac{L}2-1}\Sigma\ast\nabla\phi}+(N+1)a^{-3}\nabla\Lap^{\frac{L}2-1}\Ric[G]\ast\Sigma\ast\nabla\phi\\
&\,+a^{-3}\nabla^{L-2}\Ric[G]\ast\left((N+1)\ast\Sigma\ast\nabla\phi\right)+\frac{\dot{a}}a\langle\nabla^2\Lap^{\frac{L}2-1}N,\nabla\phi\rangle_G%\\
%&\,+a^{-3}\left[(\Psi+C)\nabla\Lap^{\frac{L}2-1}N+(N+1)\nabla\Lap^{\frac{L}2-1}\Psi\right]\nabla\phi\ast\nabla\phi
\change{+\mathfrak{J}([\del_t,\nabla\Lap^{\frac{L}2}]\phi)}
\end{align*}}
and
{\scriptsize \begin{align*}
\mathfrak{P}_{L+1,Border}=&\,-3\Psi\frac{\dot{a}}a\nabla\Lap^{\frac{L}2}N+\frac{\dot{a}}a\nabla\Psi\ast\nabla^2\Lap^{\frac{L}2-1}N +2a^{-3}\langle\Sigma,\nabla^3\Lap^{\frac{L}2-1}\Psi\rangle_G\numberthis\label{eq:comeq-Psi-odd-border}\\
&\,+a^{-3}(N+1)\nabla\Sigma\ast\nabla^L\Psi+2a^{-3}\nabla^{L-2}\Ric\ast\Sigma\ast\nabla\Psi\\
\change{&\,+a^{-3}(N+1)\nabla^2\Lap^{\frac{L}2-1}\Sigma\ast\nabla\Psi}\\
%&\,+a^{-3}(N+1)(\Psi+C)\left[\nabla\phi\ast\nabla^2\Lap^{\frac{L}2-1}\Psi+\nabla^2\Lap^{\frac{L}2-1}\phi\ast\nabla\Psi\right]\\
%&\,+a^{-3}(\Psi+C)\nabla\Lap^{\frac{L}2-1}N\ast\nabla\phi\ast\nabla\Psi\\
%%%%%%%%%%%%%%%%%%%%%%%%%%%%%%%%%5
\mathfrak{P}_{L+1,Junk}=&\,\frac{\dot{a}}a\sum_{I_N+I_\Psi=L+1,\,I_\Psi\geq 2}\nabla^{I_N}N\ast\nabla^{I_\Psi}\Psi+a\sum_{I_N+I_\phi=L+2,\,I_N,I_\phi\neq 0}\nabla^{I_N}N\ast\nabla^{I_\phi+1}\phi \numberthis\label{eq:comeq-Psi-odd-junk}\\
&\,%+a^{-3}(N+1)\nabla\phi\ast\nabla\Psi\ast\nabla\Lap^{\frac{L}2-1}\Psi\\
\change{+\mathfrak{J}([\del_t,\nabla\Lap^\frac{L}2]\Psi)}\\
%%%%%%%%%%%%%%%%%%%%%%%%%%%%%%%%
\mathfrak{Q}_{L+1,Border}=&\,a^{-3}\Psi\Lap^{\frac{L}2+1}N+a^{-3}\nabla\Psi\ast\nabla\Lap^{\frac{L}2}N+a^{-3}(N+1)\Sigma\ast\nabla^2\Lap^{\frac{L}2}\phi+\frac{\dot{a}}a\nabla\Lap^\frac{L}2N\ast\nabla\phi\numberthis\label{eq:comeq-Q-odd-border}\\
&\change{+a^{-3}(N+1)\nabla\Sigma\ast\nabla^2\Lap^{\frac{L}2-1}\phi}\\
%\,+a^{-3}(N+1)(\Psi+C)\nabla\phi\ast\nabla\Lap^{\frac{L}2}\phi}\\
%%%%%%%%%%%%%%%%%%%%%%%%%%%%%%
\mathfrak{Q}_{L+1,Junk}=&\,a^{-3}\sum_{I_N+I_\Psi=L+2,\,2\leq I_\Psi\leq L+1}\nabla^{I_N}N\ast\nabla^{I_\Psi}\Psi+a^{-3}(N+1)\nabla^{L-2}\Ric[G]\ast\Sigma\ast\nabla\phi\\
&\,%+a^{-3}\left[(\Psi+C)\Lap^{\frac{L}2}N+(N+1)\Lap^{\frac{L}2}\Psi\right]\nabla\phi\ast\nabla\phi
\change{+a^{-3}(N+1)\nabla^2\Lap^{\frac{L}2-1}\Sigma\ast\nabla\phi+\mathfrak{J}([\del_t,\Lap^{\frac{L}2+1}]\phi)}\numberthis\label{eq:comeq-Q-odd-junk}
\end{align*}}
as well as
{\scriptsize \begin{align*}
%%%%%%%%%%%%%%%%%%%%%%%%%%%%%%%%
\numberthis\label{eq:comeq-Q-1-border}\mathfrak{Q}_{1,Border}=&\,a^{-3}\Psi\Lap N+a^{-3}(N+1)\Sigma\ast\nabla^2\phi\deletemath{+a^{-3}(N+1)(\Psi+C)\nabla\phi\ast\nabla\phi}\\
%%%%%%%%%%%%%%%%%%%%%%%%%%%%%%
\numberthis\label{eq:comeq-Q-1-junk}\mathfrak{Q}_{1,Junk}=&\,a^{-3}\nabla\Psi\ast\nabla N%+a^{-3}\left[(\Psi+C)N+(N+1)\Psi\right]\nabla\phi\ast\nabla\phi
\change{+a^{-3}(N+1)\nabla\Sigma\ast\nabla\phi+\mathfrak{J}([\del_t,\Lap]\phi)}\,.
\end{align*}}
\end{subequations}
The commuted rescaled evolution equation for $\Sigma$ has the error terms
{\scriptsize \begin{subequations}
\begin{align*}
\numberthis\label{eq:comeq-Sigma-border}\mathfrak{S}_{L,Border}=&\,a^{-3}(N+1)\left(\Sigma\ast\nabla^2\Lap^{\frac{L}2-1}\Sigma+\nabla\Sigma\ast\nabla^3\Lap^{\frac{L}2-2}\Sigma\right)+a^{-3}\left(\Lap^{\frac{L}2}N\cdot(\Sigma\ast\Sigma)+\nabla\Lap^{\frac{L}2-1}N\ast\nabla\Sigma\ast\Sigma\right)\\
&\,+a^{-3}(N+1)\Sigma\ast\Sigma\ast\nabla^2\Lap^{\frac{L}2-2}\Ric[G]+\frac{\dot{a}}a\Lap^{\frac{L}2}N\ast\Sigma+\frac{\dot{a}}a\nabla\Lap^{\frac{L}2-1}N\ast\nabla\Sigma\\
&\,+\underbrace{a^{-3}[(N+1)\nabla\Sigma\ast\Sigma+\nabla N\ast\Sigma\ast\Sigma]\ast\nabla^{L-3}\Ric[G]}_{\text{not present for }L=2}\\
\numberthis\label{eq:comeq-Sigma-junk}\mathfrak{S}_{L,Junk}=&\,-a[\Lap^\frac{L}2,\nabla^2]N+a\sum_{I_N+I_\Ric=L,I_N\neq 0}\nabla^{I_N} N\ast\nabla^{I_\Ric}\Ric[G]+\frac{\dot{a}}a\sum_{I_N+I_\Sigma=L}\nabla^{I_N}N\ast\nabla^{I_\Sigma}\Sigma\\
&\,+a^{-3}\sum_{I_1+I_2=L,\,I_i>0}\nabla^{I_1}\Sigma\ast\nabla^{I_2}\Sigma+a^{-3}\sum_{I_N+I_{1}+I_{2}=L,\,I_N<L}\nabla^{I_N}N\ast\nabla^{I_1}\Sigma\ast\nabla^{I_2}\Sigma\\
%&\,+a\sum_{m=0}^{\frac{L}2}\sum_{I_N+I_{\phi,1}+I_{\phi,2}+\sum_{i=1}^{\nicefrac{L}2-m}I_i=2m}\nabla^{I_N}(N+1)\ast\nabla^{I_1}\Ric[G]\ast\dots\ast\nabla^{I_{\frac{L}2-m}}\Ric[G]\\
&\,+a\sum_{I_N+I_1+I_2=L}\nabla^{I_N}(N+1)\ast\nabla^{I_1+1}\phi\ast\nabla^{I_2+1}\phi+\left(4\pi C^2a^{-3}+\frac13a\right)\Lap^\frac{L}2N\cdot G+\mathfrak{J}([\del_t,\Lap^{\frac{L}2}]\Sigma)
\end{align*}
\end{subequations}}
while the commuted Ricci tensor evolution equations have error terms, where $\mathcal{I}=\sum_{i=1}^{\nicefrac{L}2-m+1}I_i$,
{\scriptsize
\begin{subequations}
\begin{align*}
\mathfrak{R}_{L,Border}=&\,a^{-3}\left[\nabla^{L+2}N\cdot\Sigma+\nabla^{L+1}N\ast\nabla\Sigma+\Sigma\ast\nabla^2\Lap^{\frac{L}2-1}\Ric[G]+\nabla\Sigma\ast\nabla^{L-1}\Ric[G]\right] \numberthis\label{eq:comeq-Ric-even-border}\\
%%%%%%%%%%%%%%%%%%%%%%%%%%%%%
\mathfrak{R}_{L+1,Border}=&\,a^{-3}\left[\nabla^{L+3}N\cdot\Sigma+\nabla^{L+2}N\ast\nabla\Sigma+\Sigma\ast\nabla^3\Lap^{\frac{L}2-1}\Ric[G]+\nabla\Sigma\ast\nabla^L\Ric[G]\right] \numberthis\label{eq:comeq-Ric-odd-border}\\
%%%%%%%%%%%%%%%%%%%%%%%%%%%%%%%
\mathfrak{R}_{L,Junk}=&\,a^{-3}\sum_{I_N+I_\Sigma=L+2,\,I_\Sigma\geq 2}\nabla^{I_N}N\ast\nabla^{I_\Sigma}\Sigma \numberthis\label{eq:comeq-Ric-even-junk}\\
&+a^{-3}\sum_{\substack{I_N+I_\Sigma+I_\Ric=L\\ (I_\Sigma,I_\Ric)\neq (0,L),(1,L-1)}}\nabla^{I_N}(N+1)\ast\nabla^{I_\Sigma}\Sigma\ast\nabla^{I_\Ric}\Ric[G]\\
&\,+a^{-3}\sum_{m=0}^{\frac{L}2-1}\sum_{{I_N+I_\Sigma+\mathcal{I}=2m}}\nabla^{I_N}(N+1)\ast\nabla^{I_\Sigma}\Sigma\ast\nabla^{I_1}\Ric[G]\ast\dots\ast\nabla^{I_{\frac{L}2-m+1}}\Ric[G]\\
&\,+\frac{\dot{a}}a\left([\Lap^\frac{L}2,\nabla^2]N+\Lap^\frac{L}2N\ast\Ric[G]+\nabla^{L-1}N\ast\nabla\Ric[G]\right)+\mathfrak{J}([\del_t,\Lap^{\frac{L}2}]\Ric[G])\\
%%%%%%%%%%%%%%%%%%%%%%%%%%%%%
\mathfrak{R}_{L+1,Junk}=&\,a^{-3}\sum_{I_N+I_\Sigma=L+3,\,I_\Sigma\geq 2}\nabla^{I_N}N\ast\nabla^{I_\Sigma}\Sigma \numberthis\label{eq:comeq-Ric-odd-junk}\\
&+a^{-3}\sum_{\substack{I_N+I_\Sigma+I_\Ric=L\\ (I_\Sigma,I_\Ric)\neq (0,L+1),(1,L)}}\nabla^{I_N}(N+1)\ast\nabla^{I_\Sigma}\Sigma\ast\nabla^{I_\Ric}\Ric[G]\\
&\,+a^{-3}\sum_{m=0}^{\frac{L}2-1}\sum_{{I_N+I_\Sigma+\mathcal{I}=2m+1}}\nabla^{I_N}(N+1)\ast\nabla^{I_\Sigma}\Sigma\ast\nabla^{I_1}\Ric[G]\ast\dots\nabla^{I_{\frac{L}2-m+1}}\Ric[G]\\
&\,+\frac{\dot{a}}a\left(\nabla[\Lap^\frac{L}2,\nabla^2]N+\nabla\Lap^{\frac{L}2}N\ast\Ric[G]+\nabla^2\Lap^{\frac{L}2-1}N\ast\nabla\Ric[G]\right)+\mathfrak{J}([\del_t,\nabla\Lap^\frac{L}2]\Ric[G])\\
\end{align*}
\end{subequations}}

Finally, the Bel-Robinson evolution error terms are
{\scriptsize
\begin{subequations}
\begin{align*}
\mathfrak{E}_{L,Border}=&\,\frac{\tau}3\left(\Lap^{\frac{L}2}N\cdot\RE+\nabla^{L-1}N\ast\nabla\RE\right)-a^{-1}\left(\Lap^\frac{L}2\RE\times\Sigma+\RE\times\Lap^{\frac{L}2}\Sigma\right)\numberthis\label{eq:comeq-RE-border}\\
&\,+a^{-3}\epsilonLC[G]\ast\epsilonLC[G]\ast\left(\nabla^{L-1}\RE\ast\nabla\Sigma+\nabla\RE\ast\nabla^{L-1}\Sigma\right)\\
&\,+a^{-3}\Lap^{\frac{L}2}N\cdot(\RE\ast\Sigma)+a^{-3}\nabla\Lap^{\frac{L}2-1}N\ast[\nabla\RE\ast\Sigma+\RE\ast\nabla\Sigma]\\
&\,+a^{-3}\left(\Sigma\ast\nabla^2\Lap^{\frac{L}2-1}\RE+\nabla\Sigma\ast\nabla^3\Lap^{\frac{L}2-2}\RE+\nabla\RE\ast\nabla\Lap^{\frac{L}2-1}\Sigma+\RE\ast\Lap^{\frac{L}2}\Sigma\right)\\
&\,+a^{-3}\left[(N+1)\Sigma\ast\RE\ast\nabla^2\Lap^{\frac{L}2-2}\Ric[G]+\underbrace{\nabla\left((N+1)\ast\Sigma\ast\RE\right)\ast\nabla^{L-3}\Ric[G]}_{\text{if }L\neq 2}\right]\\
&\,+4\pi a^{-3}(\Psi+C)^2\Lap^\frac{L}2N\cdot\Sigma\,+4\pi a^{-3}\nabla^{L-1}N\ast\left[(\Psi+C)^2\nabla\Sigma+2(\Psi+C)\ast\nabla\Psi\ast\Sigma\right]\\
&\,+4\pi a^{-3}(N+1)\left[(\Psi^2+2C\Psi)\Lap^\frac{L}2\Sigma+2(\Psi+C)\Lap^\frac{L}2\Psi\cdot\Sigma\right]\\
&\,+4\pi a^{-3}\nabla^{L-1}\Sigma\ast\left[(\Psi+C)^2\nabla N+2(N+1)(\Psi+C)\nabla\Psi\right]\\
&\,+4\pi a^{-3}(\Psi+C)\nabla^{L-1}\Psi\ast\left[(N+1)\nabla\Sigma+\nabla N\ast\Sigma\right]\\
%&\,+\frac{\dot{a}}a\left(\Lap^{\frac{L}2}N\cdot\RE+\nabla\Lap^{\frac{L}2-1}N\ast\nabla\RE\right)+a^{-3}(N+1)\Sigma\ast\nabla^2\Lap^{\frac{L}2-2}\Ric[G]\ast\RE\\
\mathfrak{E}_{L,top}=&\,a^{-1}(N+1)\epsilonLC[G]\ast \RB\ast\nabla\Lap^{\frac{L}2-1}\Ric[G]+a(N+1)(\Psi+C)\nabla\Lap^{\frac{L}2-1}\Ric[G]\ast\nabla\phi \numberthis\label{eq:comeq-RE-top}\\
%%%%%%%%%%%%%%%%%%%%%%%%%
\numberthis\label{eq:comeq-RE-junk}\mathfrak{E}_{L,Junk}=&\,\frac{\dot{a}}a\sum_{I_N+I_{\RE}=L,\,I_{N}\leq L-2}\nabla^{I_N}N\ast\nabla^{I_{\RE}}\RE\\
&\,+a^{-1}\epsilonLC[G]\ast\left[\sum_{I_N+I_{\RB}=L+1,\,I_{N},I_{\RB}\leq L}\nabla^{I_N}N\ast\nabla^{I_{\RB}}\RB+(N+1)\nabla^2\Lap^{\frac{L}2-2}\Ric[G]\ast\nabla\RB\right] \\
&\,+a^{-3}\epsilonLC[G]\ast\epsilonLC[G]\ast\sum_{\substack{I_N+I_{\RE}+I_\Sigma=L,\\ I_N\leq L-2;\,I_N>0\text{ or }I_{\RE},I_{\Sigma}\geq 2}}\nabla^{I_N}N\ast\nabla^{I_{\RE}}\RE\ast\nabla^{I_\Sigma}\Sigma\\
&\,+a\sum_{\substack{I_N+I_\Psi+I_\phi=L+1\\ I_N,I_\Psi,I_\phi\neq L+1}}\nabla^{I_N}(N+1)\ast\nabla^{I_\Psi}(\Psi+C)\ast\nabla^{I_\phi+1}\phi\\
%&\,+\sum_{I_N+I_\Psi+I_\phi=L+1,I_\phi\neq L+1}\nabla^{I_N}N\ast\nabla^{I_\Psi}\Psi\ast\nabla^{I_\phi+1}\phi\\
&\,+a^{-3}\sum_{\substack{I_N+I_\Sigma+I_1+I_2=L \\ I_N,I_\Sigma,I_i\leq L-2}}\nabla^{I_N}(N+1)\ast\nabla^{I_\Sigma}\Sigma\ast\nabla^{I_1}(\Psi+C)\ast\nabla^{I_2}(\Psi+C)\\
&\,+\dot{a}a^3\sum_{I_N+I_1+I_2=L}\nabla^{I_N}(N+1)\ast\nabla^{I_1+1}\phi\ast\nabla^{I_2+1}\phi\\
&\,+a\sum_{I_N+I_\Sigma+I_1+I_2=L}\nabla^{I_N}(N+1)\ast\nabla^{I_\Sigma}\Sigma\ast\nabla^{I_1+1}\phi\ast\nabla^{I_2+1}\phi\\
&\,+a^{-1}\epsilon[G]\ast\RB\ast[\Lap^\frac{L}2,\nabla]N+4\pi a(N+1)(\Psi+C)\left[\nabla^{L-2}\Ric[G]\ast\nabla^2\phi+\mathfrak{J}([\Lap^\frac{L}2,\nabla^2]\phi)\right]\\
&\,+a\left\{(\Psi+C)[\Lap^\frac{L}2,\nabla]N+(N+1)[\Lap^\frac{L}2,\nabla]\Psi\right\}\ast\nabla\phi+\change{(N+1)a^{-1}\mathfrak{J}([\Lap^\frac{L}2,\curl_G]\RB)}+\mathfrak{J}([\del_t,\Lap^{\frac{L}2}]\RE)\\
&\,+\Lap^\frac{L}2\left[a^{-3}(N+1)\RE\ast\Sigma+\frac{2\pi}3a^6(N+1)\left(\del_0\left(a^{-6}(\Psi+C)^2+a^{-2}\lvert\nabla\phi\rvert_G^2\right)+4\pi\frac{\dot{a}}a(\Psi+C)^2\right)\right]\cdot G\\[2em]
%%%%%%%%%%%%%%%%%%%%%%%%%%%%
\mathfrak{B}_{L,Border}=&\,\frac{\tau}3\left(\Lap^{\frac{L}2}N\cdot\RB+\nabla^{L-1}N\ast\nabla\RB\right)-a^{-1}\left(\Lap^{\frac{L}2}\RB\times\Sigma+\RB\ast\Lap^{\frac{L}2}\Sigma\right)\numberthis\label{eq:comeq-RB-border}\\
&\,+a^{-3}\epsilonLC[G]\ast\epsilonLC[G]\ast\left(\nabla^{L-1}\RB\ast\nabla\Sigma+\nabla\RB\ast\nabla^{L-1}\Sigma\right)\\
&\,+a^{-3}\Lap^{\frac{L}2}N\cdot(\RB\ast\Sigma)+a^{-3}\nabla\Lap^{\frac{L}2-1}N\cdot\left[\nabla\RB\ast\Sigma+\RB\ast\nabla\Sigma\right]\\
&\,+a^{-3}\left(\Sigma\ast\nabla^2\Lap^{\frac{L}2-1}\RB+\nabla\Sigma\ast\nabla^3\Lap^{\frac{L}2-2}\RB+\nabla\RB\ast\nabla\Lap^{\frac{L}2-1}\Sigma+\RB\ast\Lap^{\frac{L}2}\Sigma\right)\\
&\,+a^{-3}\left[(N+1)\Sigma\ast\RB\ast\nabla^{L-2}\Ric[G]+\underbrace{\nabla\left((N+1)\ast\Sigma\ast\RB\right)\ast\nabla^{L-3}\Ric[G]}_{\text{if }L\neq 2}\right]\\
%&\,+\frac{\dot{a}}a\Lap^{\frac{L}2}N\cdot\RB+\frac{\dot{a}}a\nabla\Lap^{\frac{L}2-1}N\ast\nabla\RB\\
&\,+a^{-1}(N+1)(\Psi+C)\cdot\epsilonLC[G]\ast\nabla\Lap^\frac{L}2\phi\ast\Sigma+a^{-1}\epsilonLC[G]\ast\nabla^2\nabla^L\phi\ast\nabla\left((N+1)(\Psi+C)\Sigma\right)\\
%%%%%%%%%%%%%%%%%%%%%%%
\mathfrak{B}_{L,top}=&\,a^3(N+1)\epsilonLC[G]\ast\nabla\Lap^{\frac{L}2-1}\Ric[G]\ast\nabla\phi\ast\nabla\phi+a^{-1}\epsilonLC[G]\ast\RE\ast\nabla\Lap^{\frac{L}2-1}\Ric[G] \numberthis\label{eq:comeq-RB-top}\\
%%%%%%%%%%%%%%%%%%%%%
\mathfrak{B}_{L,Junk}=&\,\frac{\dot{a}}a\sum_{I_N+I_{\RB}=L,I_{N}\leq L-2}\nabla^{I_N}N\ast\nabla^{I_{\RB}}\RB \numberthis\label{eq:comeq-RB-junk}\\
&\,+a^{-1}\epsilonLC[G]\ast\left[\sum_{I_N+I_{\RE}=L+1,I_{N},I_{\RE}\leq L}\nabla^{I_N}N\ast\nabla^{I_{\RE}}\RE+\nabla^2\Lap^{\frac{L}2-2}\Ric[G]\ast\nabla\RE\right] \\
&\,+a^{-3}\epsilonLC[G]\ast\epsilonLC[G]\ast\sum_{\substack{I_N+I_{\RE}+I_\Sigma=L,\\ I_N\leq L-2;\,I_N>0\text{ or }I_{\RB},I_{\Sigma}\geq 2}}\nabla^{I_N}(N+1)\ast\nabla^{I_{\RB}}\RB\ast\nabla^{I_\Sigma}\Sigma\\
&\,+a^3\epsilonLC[G]\ast\sum_{\substack{I_N+I_{1}+I_{2}=L,\\ \,I_N>0\text{ or }I_2< L}}\nabla^{I_N}(N+1)\ast\nabla^{I_{1}+1}\phi\ast\nabla^{I_{2}+2}\phi\\
%&\,+a^3\sum_{m=0}^{\frac{L}2-1}\sum_{\substack{I_N+I_{\phi,1}+I_{\phi,2}\\+\sum_{i=0}^{\frac{L}2-m}I_i=2m+1}}\nabla^{I_N}(N+1)\ast\nabla^{I_{\phi,1}+1}\phi\ast\nabla^{I_{\phi,2}+1}\phi\,\ast%\\
%%&\,\phantom{+a^3\sum_{m=0}^{\frac{L}2-1}\sum_{\sum_{i=0}^{\frac{L}2-m}I_i=2m+1,\,I_1\neq L-1}}\ast
%\nabla^{I_1}\Ric[G]\ast\dots\ast\nabla^{I_{\frac{L}2-m}}\Ric[G]\\
&\,+a^{-1}\epsilonLC[G]\ast\sum_{\substack{I_N+I_\Psi+I_\phi+I_\Sigma=L\\I_\phi\leq L-2}}\nabla^{I_N}(N+1)\ast\nabla^{I_\Psi}(\Psi+C)\ast\nabla^{I_\phi+1}\phi\ast\nabla^{I_\Sigma}\Sigma\\
&\,+a^{-1}\epsilon[G]\ast\RE\ast[\Lap^{\frac{L}2},\nabla]N\\
&\,+a^{-1}(N+1)(\Psi+C)\cdot\epsilonLC[G]\ast\Sigma\ast\left[\Lap^\frac{L}2,\nabla\right]\phi\\
&\,+a^3(N+1)\epsilonLC[G]\ast\nabla\phi\ast\left(\nabla^2\Lap^{\frac{L}2-2}\Ric[G]\ast\nabla^2\phi+\mathfrak{J}([\Lap^{\frac{L}2},\nabla]\phi)\right)\\
&\,\change{-a^{-1}(N+1)\mathfrak{J}([\Lap^\frac{L}2,\curl_G]\RE)}+\mathfrak{J}([\del_t,\Lap^{\frac{L}2}]\RB)+\Lap^\frac{L}2\left[a^{-3}(N+1)\RB\ast\Sigma\right]\cdot G\\
&\,+\Lap^{\frac{L}2}\left[4\pi a^2\nabla^{\sharp m}\phi(\Psi+C)+\frac{2\pi}3a^5\Lap^{\frac{L}2}\nabla^{\sharp m}\left(a^{-6}(\Psi+C)^2+a^{-2}\lvert\nabla\phi\rvert_G^2\right)\right]\epsilonLC[G]_{(\cdot)m(\cdot)}
\end{align*}
\end{subequations}}
\end{definition}

\subsection{$L^2_G$ error term estimates}\label{subsec:L2-error-est}

In this subsection, we collect how the error terms can be controlled in terms of energies as well as homogeneous Sobolev norms of $\phi$. We do not claim that these estimates are optimal -- in particular, we note that at low order (like $L=2$), many of the curvature errors that appear in the estimates below could be avoided entirely: These arise as a result of applying the general estimates in Lemma \ref{lem:Sobolev-norm-equivalence-improved} where the Ricci tensor does not naturally occur in the respective equations, and can be avoided at low orders by applying \eqref{eq:APmidRic} on all curvature terms that occur. \\
Instead of optimality, we try to keep both notation and form of the error term estimates as simple as possible and the energy estimates between base and top level as unified as possible. In particular, we track the \enquote{worst} curvature energy occurring at high orders for all estimates below, even if these terms are added in artificially for low orders.

\begin{lemma}[Estimates for borderline error terms] Let $L\in 2\Z_+,\,L\leq 20$. Then, the following estimates hold:
\begin{subequations}
\begin{align*}
\numberthis\label{eq:L2-Border-H}\|\mathfrak{H}_{L,Border}\|_{L^2_G}\lesssim&\,\epsilon a^{-4}\sqrt{\E^{(L)}(\Sigma,\cdot)}+\epsilon a^{-4-c\sqrt{\epsilon}}\sqrt{\E^{(\leq L-2)}(\Sigma,\cdot)}+\epsilon^2a^{-4-c\sqrt{\epsilon}}\sqrt{\E^{(\leq L-2)}(\Ric,\cdot)}\\
%%%%%%%%%%%%%%%%%%%%%%%%%%%%
\numberthis\label{eq:L2-Border-N}\|\mathfrak{N}_{L,Border}\|_{L^2_G}\lesssim&\,\epsilon a^{-4}\left[\sqrt{\E^{(L)}(\phi,\cdot)}+\sqrt{\E^{(L)}(\Sigma,\cdot)}\right]+\epsilon a^{-4}\sqrt{\E^{(L)}(N,\cdot)}\\
&\,+\epsilon a^{-4-c\sqrt{\epsilon}}\left[\sqrt{\E^{(\leq L-2)}(\phi,\cdot)}+\sqrt{\E^{(\leq L-2)}(\Sigma,\cdot)}+\sqrt{\E^{(\leq L-2)}(N,\cdot)}\right]\\
&\,+\underbrace{\epsilon^2 a^{-4}\sqrt{\E^{(\leq L-2)}(\Ric,\cdot)}+\epsilon a^{-4-c\sqrt{\epsilon}}\sqrt{\E^{(\leq L-4)}(\Ric,\cdot)}}_{\text{not present for }L=2}\\
%%%%%%%%%%%%%%%%%%%%%%%%%%%%%%%%%%%%%
\numberthis\label{eq:L2-Border-N-odd}\|\mathfrak{N}_{L+1,Border}\|_{L^2_G}\lesssim&\,\epsilon a^{-6}\left[\sqrt{a^4\E^{(L+1)}(\phi,\cdot)}+\sqrt{a^4\E^{(L+1)}(\Sigma,\cdot)}\right]+\epsilon a^{-6}\sqrt{a^4\E^{(L+1)}(N,\cdot)}\\
&\,+\epsilon a^{-4}\left[\sqrt{\E^{(L)}(\phi,\cdot)}+\sqrt{\E^{(L)}(\Sigma,\cdot)}+\sqrt{\E^{(L)}(N,\cdot)}\right]\\
&\,+\epsilon a^{-4-c\sqrt{\epsilon}}\left[\sqrt{\E^{(\leq L-2)}(\phi,\cdot)}+\sqrt{\E^{(\leq L-2)}(\Sigma,\cdot)}+\sqrt{\E^{(\leq L-2)}(N,\cdot)}\right]\\
&\,+\epsilon^2a^{-4-c\sqrt{\epsilon}}\sqrt{\E^{(\leq L-2)}(\Ric,\cdot)}\\
%%%%%%%%%%%%%%%%%%%%%%%%%%%%%%%%
\numberthis\label{eq:L2-Border-P-even}\|\mathfrak{P}_{L,Border}\|_{L^2_G}\lesssim&\,%\sqrt{\epsilon}
\change{\epsilon}a^{-3}\sqrt{\E^{(L)}(\phi,\cdot)}+\epsilon a^{-3}\sqrt{\E^{(L)}(N,\cdot)}+\epsilon a^{-3-c\sqrt\epsilon}\sqrt{\E^{(\leq L-2)}(N,\cdot)}\\
&\,+%\sqrt{\epsilon}
\change{\epsilon}a^{-3-c\sqrt{\epsilon}}\sqrt{\E^{(\leq L-2)}(\phi,\cdot)}
\change{+\epsilon a^{-3}\sqrt{\E^{(L)}(\Sigma,\cdot)}+\epsilon a^{-3-c\sqrt{\epsilon}}\sqrt{\E^{(L-2)}(\Sigma,\cdot)}}\\
%+\epsilon a^{-3-c\sqrt{\epsilon}}\|\nabla\Lap^{\frac{L}2-1}\phi\|_{L^2_G}\\
&\,+\underbrace{\epsilon^2 a^{-3-c\sqrt{\epsilon}}\sqrt{\E^{(\leq L-3)}(\Ric,\cdot)}}_{\text{not present for }L=2}\\
%%%%%%%%%%%%%%%%%%%%%%%%%%%%%%%%%%
\numberthis\label{eq:L2-Border-Q-even}\|\mathfrak{Q}_{L,Border}\|_{L^2_G}\lesssim&\,\epsilon a^{-3}\sqrt{\E^{(L+1)}(N,\cdot)}%+\epsilon a^{-3}\left[\sqrt{a^{-4}\E^{(L)}(\phi,\cdot)}+\sqrt{a^{-4-c\sqrt{\epsilon}}\E^{(L-2)}(\phi,\cdot)}\right]\\&\,
+%\sqrt{\epsilon}
\change{\epsilon}a^{-3}\sqrt{a^{-4}\E^{(L)}(\phi,\cdot)}\\
&\,+%\sqrt{\epsilon}
\change{\epsilon}a^{-3-c\sqrt{\epsilon}}\sqrt{a^{-4}\E^{(\leq L-2)}(\phi,\cdot)}\\
%%%%%%%%%%%%%%%%%%%%%%%%%%%%%%%%%%
\numberthis\label{eq:L2-Border-P-odd}\|\mathfrak{P}_{L+1,Border}\|_{L^2_G}\lesssim&\,%\sqrt{\epsilon}
\change{\epsilon}a^{-3}\sqrt{\E^{(L+1)}(\phi,\cdot)}+\epsilon a^{-3}\sqrt{\E^{(L+1)}(N,\cdot)}+\epsilon a^{-3-c\sqrt\epsilon}\sqrt{\E^{(\leq L-1)}(N,\cdot)}\\
&\,+\change{\epsilon}a^{-3-c\sqrt{\epsilon}}\sqrt{\E^{(\leq L-1)}(\phi,\cdot)}+\change{\epsilon a^{-3}\sqrt{\E^{(L+1)}(\Sigma,\cdot)}+\epsilon a^{-3-c\sqrt{\epsilon}}\sqrt{\E^{(L-1)}(\Sigma,\cdot)}}\\
%+\epsilon a^{-3-c\sqrt{\epsilon}}\|\nabla^2\Lap^{\frac{L}2-1}\phi\|_{L^2_G}\\
&\,+\epsilon^2 a^{-3-c\sqrt{\epsilon}}\sqrt{\E^{(\leq L-2)}(\Ric,\cdot)}\\
%%%%%%%%%%%%%%%%%%%%%%%%%%%%%%%%%
\numberthis\label{eq:L2-Border-Q-odd}\|\mathfrak{Q}_{L+1,Border}\|_{L^2_G}\lesssim&\,\epsilon a^{-3}\sqrt{\E^{(L+2)}(N,\cdot)}+\epsilon a^{-3-c\sqrt{\epsilon}}\sqrt{\E^{(L+1)}(N,\cdot)}\\
&\,+\change{\epsilon} a^{-3}\sqrt{a^{-4}\E^{(L+1)}(\phi,\cdot)}+\change{\epsilon}a^{-3-c\sqrt{\epsilon}}\sqrt{a^{-4}\E^{(\leq L-1)}(\phi,\cdot)}\\
%%%%%%%%%%%%%%%%%%%%%%%%%%%%%%%%%%%
\numberthis\label{eq:L2-Border-S}\|\mathfrak{S}_{L,Border}\|_{L^2_G}\lesssim&\,\epsilon a^{-3}\sqrt{\E^{(L)}(\Sigma,\cdot)}+\epsilon a^{-3}\sqrt{\E^{(L)}(N,\cdot)}+\epsilon^2a^{-3}\sqrt{\E^{(L-2)}(\Ric,\cdot)}\\
&\,+\epsilon a^{-3-c\sqrt{\epsilon}}\sqrt{\E^{(\leq L-2)}(\Sigma,\cdot)}+\epsilon a^{-3-c\sqrt{\epsilon}}\sqrt{\E^{(\leq L-2)}(N,\cdot)}\\
&\,+\underbrace{\epsilon^2a^{-3-c\sqrt{\epsilon}}\sqrt{\E^{(\leq L-4)}(\Ric,\cdot)}}_{\text{not present for }L=2}\\
%%%%%%%%%%%%%%%%%%%%%%%%%%%%%%%
\numberthis\label{eq:L2-Border-R-even}\|\mathfrak{R}_{L,Border}\|_{L^2_G}\lesssim&\,\epsilon a^{-3}\sqrt{\E^{(L+2)}(N,\cdot)}+\epsilon a^{-3-c\sqrt{\epsilon}}\sqrt{\E^{(\leq L)}(N,\cdot)}\\
&\,+\epsilon a^{-3}\sqrt{\E^{(L)}(\Ric,\cdot)}+\epsilon a^{-3-c\sqrt{\epsilon}}\sqrt{\E^{(\leq L-2)}(\Ric,\cdot)}\\
%%%%%%%%%%%%%%%%%%%%%%%%%%%%%%%
\numberthis\label{eq:L2-Border-R-odd}\|\mathfrak{R}_{L+1,Border}\|_{L^2_G}\lesssim&\,\epsilon a^{-3}\sqrt{\E^{(L+3)}(N,\cdot)}+\epsilon a^{-3-c\sqrt{\epsilon}}\sqrt{\E^{(\leq L+1)}(N,\cdot)}\\
&\,+\epsilon a^{-3}\sqrt{\E^{(L+1)}(\Ric,\cdot)}+\epsilon a^{-3-c\sqrt{\epsilon}}\sqrt{\E^{(\leq L-1)}(\Ric,\cdot)}
\end{align*}
\vspace{-1em}
\begin{align*}
\numberthis\label{eq:L2-Border-BR}&\,\|\mathfrak{E}_{L,Border}\|_{L^2_G}+\|\mathfrak{B}_{L,Border}\|_{L^2_G}\\
\lesssim&\,\epsilon a^{-3}\sqrt{\E^{(L)}(\phi,\cdot)}+\epsilon a^{-3-c\sqrt{\epsilon}}\sqrt{\E^{(\leq L-2)}(\phi,\cdot)}\\
&\,+\epsilon a^{-3}\left(\sqrt{\E^{(L)}(N,\cdot)}+\sqrt{\E^{(L)}(\Sigma,\cdot)}\right)+\epsilon a^{-3}\sqrt{\E^{(L)}(W,\cdot)}\\
&\,+\epsilon a^{-3-c\sqrt{\epsilon}}\left(\sqrt{\E^{(\leq L-2)}(N,\cdot)}+\sqrt{\E^{(\leq L-2)}(\Sigma,\cdot)}+\sqrt{\E^{(\leq L-2)}(W,\cdot)}\right)\\
&\,+\epsilon^2a^{-3}\sqrt{\E^{(L-2)}(\Ric,\cdot)}+\underbrace{\epsilon^2a^{-3-c\sqrt{\epsilon}}\sqrt{\E^{(\leq L-4)}(\Ric,\cdot)}}_{\text{not present for }L=2}
\end{align*}
\delete{Further, we have the alternative estimates [...]}
%\begin{align*}
%%\numberthis\label{eq:L2-border-Q-even-alt}\|\mathfrak{Q}_{L,Border}\|_{L^2_G}\lesssim&\,\epsilon a^{-3}\sqrt{\E^{(L+1)}(N,\cdot)}+\epsilon a^{-3}\left[\|\nabla \Lap^{\frac{L}2}\phi\|_{L^2_G}+a^{-c\sqrt{\epsilon}}\|\nabla\Lap^{\frac{L}2-1}\phi\|_{L^2_G}\right]\\
%%&\,+\sqrt{\epsilon}a^{-3}\left[\|\Lap^{\frac{L}2}\phi\|_{L^2_G}+a^{-c\sqrt{\epsilon}}\underbrace{\|\Lap^{\frac{L}2-1}\phi\|_{L^2_G}}_{\text{not present for }L=2}\right]\\
%\numberthis\label{eq:L2-border-Q-1-alt}\|\mathfrak{Q}_{1,Border}\|_{L^2_G}\lesssim&\,\epsilon a^{-3}\sqrt{\E^{(2)}(N,\cdot)}+\sqrt{\epsilon}a^{-3}\left[\|\Lap\phi\|_{L^2_G}+a^{-c\sqrt{\epsilon}}\|\nabla\phi\|_{L^2_G}\right]\,,\\
%\numberthis\label{eq:L2-border-Q-odd-alt}\|\mathfrak{Q}_{L+1,Border}\|_{L^2_G}\lesssim&\,\epsilon a^{-3-c\sqrt{\epsilon}}\sqrt{\E^{(L+2)}(N,\cdot)}+\epsilon a^{-3-c\sqrt{\epsilon}}\sqrt{\E^{(L)}(N,\cdot)}\\
%&\,+\sqrt{\epsilon}a^{-3}\left[\|\Lap^{\frac{L}2+1}\phi\|_{L^2_G}+a^{-c\sqrt{\epsilon}}\|\Lap^{\frac{L}2}\phi\|_{L^2_G}\right]\,.\\
%\end{align*}
\end{subequations}
\end{lemma}
\begin{proof}
All of these estimates follow from applying $L^2_G$-$L^\infty_G$-type Hölder estimates to the individual nonlinear terms. The lower order terms are either controlled by the zero order estimates in subsection \ref{subsec:APlow} or the a priori estimates in Lemma \ref{lem:AP}. Furthermore, we apply Lemma \ref{lem:Sobolev-norm-equivalence-improved}, along with again Lemma \ref{lem:AP}, to translate $L^2_G$-norms into energies up to additional curvature energy terms. For the sake of simplicity, we always estimate $\frac{\dot{a}}a$ by $a^{-3}$ up to constant (see \eqref{eq:Friedman}), and liberally apply \eqref{eq:ibp-trick} to deal with odd order energies and to distribute $a^{-c\sqrt{\epsilon}}$ factors to lower orders while updating $c>0$ wherever this is convenient.
%\todo{$\frac{\dot{a}}a\simeq a^{-3}$},\todo{Use the ibp trick for the $L-1$-orders}\\\todo{for $\nabla^2$-terms, note pseudo-equivalence doesn't incur curvature terms in the scalar field case}\\\todo{red marked term needs the f.t.o.c. step in between, prepare that or keep it for later}
\end{proof}

\begin{lemma}[Estimates for top order error terms]
\begin{subequations}
\begin{align*}
\numberthis\label{eq:L2-top-E}\|\mathfrak{E}_{L,top}\|_{L^2_G}\lesssim&\,\sqrt{\epsilon} a^{1-c\sqrt{\epsilon}}\sqrt{\E^{(L-1)}(\Ric,\cdot)}=\sqrt{\epsilon}a^{-1-c\sqrt{\epsilon}}\sqrt{a^4\E^{(L-1)}(\Ric,\cdot)}\\
\numberthis\label{eq:L2-top-B}\|\mathfrak{B}_{L,top}\|_{L^2_G}\lesssim&\,\epsilon a^{-1}\sqrt{\E^{(L-1)}(\Ric,\cdot)}=\change{\epsilon a^{-3}\sqrt{a^4\E^{(L-1)}(\Ric,\cdot)}}
\end{align*}
\end{subequations}
\end{lemma}
\begin{proof}
This follows directly using \eqref{eq:APE} and \eqref{eq:APmidB} for the Bel-Robinson terms as well as \eqref{eq:APmidphi}.
\end{proof}


\begin{lemma}[Junk terms] Recalling $\parallel$-notation from Remark \ref{rem:notation-parallel}, the following hold:
\begin{subequations}
\begin{align*}
\numberthis\label{eq:L2-junk-M}\|\mathfrak{M}_{L,Junk}\|_{L^2_G}\lesssim
&\,\epsilon a^{-2-c\sqrt{\epsilon}}\sqrt{\E^{(\leq L-1)}(\phi,\cdot)}+a^{-2-c\sqrt{\epsilon}}\sqrt{\E^{(\leq L-2)}(\phi,\cdot)}%+a^{-c\sqrt{\epsilon}}\E^{(\leq L-3)}(\Ric,\cdot)
\\
&\,+a^{-c\sqrt{\epsilon}}\sqrt{\E^{(\leq L-1)}(\Sigma,\cdot)}+\sqrt{\epsilon}a^{-c\sqrt{\epsilon}}\sqrt{\E^{(\leq L-2)}(\Ric,\cdot)}\\
%%%%%%%%%%%%%%%%%%%%%%
\numberthis\label{eq:L2-junk-Mtilde}\|\tilde{\mathfrak{M}}_{L,Junk}\|_{L^2_G}\lesssim&\,\epsilon a^{-1-c\sqrt{\epsilon}}\sqrt{\E^{(\leq L-2)}(\Ric,\cdot)}+a^{-1-c\sqrt{\epsilon}}\sqrt{\E^{(\leq L-1)}(\Sigma,\cdot)}\\[1em]
%%%%%%%%%%%%%%%%%%%%%%%%
%\|\mathfrak{H}_{L,Junk}\|_{L^2_G}\lesssim&\,\epsilon a^{-4-c\sqrt{\epsilon}}\sqrt{\E^{(\leq L-2)}(\Sigma,\cdot)}+\left(a^{-4}+\sqrt{\epsilon}a^{-4-c\sqrt{\epsilon}}\right)\sqrt{\E^{(\leq L)}(\phi,\cdot)}\\
%&\,+\epsilon a^{-c\sqrt{\epsilon}}\sqrt{\E^{(\leq L-2)}(\Ric,\cdot)}\\
%%%%%%%%%%%%%%%%%%%%%%%%%
\numberthis\label{eq:L2-junk-H-par}\|\mathfrak{H}_{L,Junk}^\parallel\|_{L^2_G}\lesssim&\,\epsilon a^{-4-c\sqrt{\epsilon}}\sqrt{\E^{(\leq L-2)}(\Sigma,\cdot)}+\sqrt{\epsilon}a^{-2-c\sqrt{\epsilon}}\sqrt{\E^{(\leq L)}(\phi,\cdot)}\\
&\,+\epsilon a^{-c\sqrt{\epsilon}}\sqrt{\E^{(\leq L-2)}(\Ric,\cdot)}\\
%%%%%%%%%%%%%%%%%%%%%%%%%%%%%%%%
\numberthis\label{eq:L2-junk-N}\|\mathfrak{N}_{L,Junk}\|_{L^2_G}\lesssim&\,\epsilon a^{-4-c\sqrt{\epsilon}}\sqrt{\E^{(\leq L-2)}(N,\cdot)}+\epsilon a^{-c\sigma}\sqrt{\E^{(L)}(\phi,\cdot)}\\
&\,+\epsilon a^{-4-c\sqrt{\epsilon}}\left[\sqrt{\E^{(\leq L-2)}(\phi,\cdot)}+\sqrt{\E^{(\leq L-2)}(\Sigma,\cdot)}\right]\\
&\,+\underbrace{\epsilon a^{-4}\sqrt{\E^{(\leq L-2)}(\Ric,\cdot)}+\epsilon a^{-4-c\sqrt{\epsilon}}\sqrt{\E^{(\leq L-4)}(\Ric,\cdot)}}_{\text{not present for }L=2}\\
%%%%%%%%%%%%%%%%%%%%%%%%
\numberthis\label{eq:L2-junk-N-odd}\|\mathfrak{N}_{L+1,Junk}\|_{L^2_G}\lesssim&\,\epsilon a^{-4-c\sqrt{\epsilon}}\sqrt{\E^{(\leq L-1)}(N,\cdot)}+\epsilon a^{-c\sigma}\left(\sqrt{\E^{(L+1)}(\phi,\cdot)}+\sqrt{\E^{(L+1)}(\Sigma,\cdot)}\right)\\
&\,+\epsilon a^{-4-c\sqrt{\epsilon}}\left[\sqrt{\E^{(\leq L-1)}(\phi,\cdot)}+\sqrt{\E^{(\leq L-1)}(\Sigma,\cdot)}\right]\\
&\,+\epsilon^2a^{-4}\sqrt{\E^{(\leq L-1)}(\Ric,\cdot)}+\underbrace{\epsilon^2a^{-4-c\sqrt{\epsilon}}\sqrt{\E^{(\leq L-3)}(\Ric,\cdot)}}_{\text{not present for }L=2}\\
%%%%%%%%%%%%%%%%%%%%%%%%
\numberthis\label{eq:L2-junk-P-even}\|\mathfrak{P}_{L,Junk}\|_{L^2_G}\lesssim&\,\epsilon a^{1-c\sigma}\sqrt{\E^{(L)}(\phi,\cdot)}+\epsilon a^{-3-c\sqrt{\epsilon}}\sqrt{\E^{(\leq L-2)}(\phi,\cdot)}\\
&\,+\epsilon a^{-3-c\sqrt{\epsilon}}\sqrt{\E^{(\leq L-2)}(\Sigma,\cdot)}+\sqrt{\epsilon}a^{1-c\sqrt{\epsilon}}\sqrt{\E^{(L)}(N,\cdot)}\\
&\,+\left[\epsilon a^{-3-c\sqrt{\epsilon}}+\sqrt{\epsilon}a^{-1-c\sqrt{\epsilon}}\right]\sqrt{\E^{(\leq L-2)}(N,\cdot)}\\
&\,+\underbrace{\epsilon^2a^{1-c\sigma}\sqrt{\E^{(\leq L-2)}(\Ric,\cdot)}+\epsilon^2 a^{-3-c\sqrt{\epsilon}}\sqrt{\E^{(\leq L-3)}(\Ric,\cdot)}%+\epsilon a^{-3-c\sqrt{\epsilon}}\|\nabla\phi\|_{H^{L-3}_G}
}_{\text{not present for }L=2}\\
%%%%%%version going term by term:
%&\,\sqrt{\epsilon}a^{1-c\sqrt{\epsilon}}\sqrt{\E^{(\leq L)}(N,\cdot)}+\epsilon a^{3-c\sigma}\left(\sqrt{\E^{(L)}(\phi,\cdot)}+\sqrt{\E^{(\leq L-2)}(\phi,\cdot)}\right)\\
%&\,+\epsilon^\frac32 a^{1-c\sigma}\sqrt{\E^{(\leq L-5)}(\Ric,\cdot)}+{\epsilon} a^{-3}\sqrt{\E^{(L)}(N,\cdot)}+\epsilon a^{-3-c\sqrt{\epsilon}}\sqrt{\E^{(\leq L-2)}(N,\cdot)}\\
%&\,+\epsilon a^{-3-c\sqrt{\epsilon}}\left(\sqrt{\E^{(\leq L-2)}(\Sigma,\cdot)}+\sqrt{\E^{(\leq L-4)}(\Ric,\cdot)}\right)+\epsilon a^{-3-c\sqrt{\epsilon}}\sqrt{\E^{(\leq L-2)}(\phi,\cdot)}\\
%&\,\todo{+\epsilon a^{-3-c\sqrt{\epsilon}}\|\nabla\phi\|_{H^{L-3}_G}}\\
\numberthis\label{eq:L2-junk-P-odd}\|\mathfrak{P}_{L+1,Junk}\|_{L^2_G}\lesssim&\,\epsilon a^{1-c\sigma}\sqrt{\E^{(L+1)}(\phi,\cdot)}+\epsilon a^{-3-c\sqrt{\epsilon}}\sqrt{\E^{(\leq L-1)}(\phi,\cdot)}\deletemath{+\epsilon a^{-3-c\sqrt{\epsilon}}\|\nabla\phi\|_{H^{L-2}_G}}\\
&\,+\epsilon a^{-3-c\sqrt{\epsilon}}\sqrt{\E^{(\leq L-1)}(\Sigma,\cdot)}+\sqrt{\epsilon}a^{1-c\sqrt{\epsilon}}\sqrt{\E^{(L+1)}(N,\cdot)}\\
&\,+\left[\epsilon a^{-3-c\sqrt{\epsilon}}+\sqrt{\epsilon}a^{-1-c\sqrt{\epsilon}}\right]\sqrt{\E^{(\leq L-1)}(N,\cdot)}\\
&\,+\epsilon^2a^{-1-c\sigma}\sqrt{a^4\E^{(L-1)}(\Ric,\cdot)}\\
&\,+\left(\epsilon^2a^{-1-c\sigma}+\epsilon^2 a^{-3-c\sqrt{\epsilon}}\right)\sqrt{\E^{(\leq L-2)}(\Ric,\cdot)}\\
%%%%%%%%%%%%%%%%%%%%%%%%
\numberthis\label{eq:L2-junk-Q-even}\|\mathfrak{Q}_{L,Junk}\|_{L^2_G}\lesssim&\,%\epsilon a^{-3-c\sqrt{\epsilon}}\sqrt{\E^{(\leq L)}(\phi,\cdot)}+\sqrt{\epsilon}a^{-3-c\sqrt{\epsilon}}\left(\|\Lap^\frac{L}2\phi\|_{L^2_G}+\|\nabla\phi\|_{H^{L-3}_G}\right)\\
\change{\epsilon a^{-1-c\sqrt{\epsilon}}\sqrt{a^{-4}\E^{(L)}(\phi,\cdot)}+\epsilon a^{-3-c\sqrt{\epsilon}}\sqrt{\E^{(\leq L-2)}(\phi,\cdot)}}\\
&\,+\change{\sqrt{\epsilon}a^{-3-c\sqrt{\epsilon}}\sqrt{\E^{(\leq L)}(\Sigma,\cdot)}}+\sqrt{\epsilon}a^{-3-c\sqrt{\epsilon}}\sqrt{\E^{(\leq L)}(N,\cdot)}\\
&\,+\underbrace{\epsilon a^{-3-c\sqrt{\epsilon}}\sqrt{\E^{(\leq L-2)}(\Ric,\cdot)}}_{\text{not present for }L=2}\\
%%%%%%%%%%%%%%%%%%%%%%%%%
\numberthis\label{eq:L2-junk-Q-1}\|\mathfrak{Q}_{1,Junk}\|\lesssim&\,\epsilon a^{-3-c\sqrt{\epsilon}}\sqrt{\E^{(1)}(N,\cdot)}+\epsilon^\frac32a^{-3-c\sqrt{\epsilon}}\|\nabla\phi\|_{L^2_G}+\change{\sqrt{\epsilon}a^{-3-c\sqrt{\epsilon}}\sqrt{\E^{(\leq 1)}(\Sigma,\cdot)}}\\
%%%%%%%%%%%%%%%%%%%%%%%%
\numberthis\label{eq:L2-junk-Q-odd}\|\mathfrak{Q}_{L+1,Junk}\|_{L^2_G}\lesssim&\,%\epsilon a^{-3-c\sqrt{\epsilon}}\sqrt{\E^{(\leq L)}(\phi,\cdot)}+\epsilon a^{-3-c\sqrt{\epsilon}}\|\nabla\phi\|_{H^{L}_G}
\change{\epsilon a^{-1-c\sqrt{\epsilon}}\sqrt{a^{-4}\E^{(L+1)}(\phi,\cdot)}+\epsilon a^{-3-c\sqrt{\epsilon}}\E^{(\leq L-1)}(\phi,\cdot)}\\
&\,\change{+\sqrt{\epsilon}a^{-3-c\sqrt{\epsilon}}\sqrt{\E^{(\leq L+1)}(\Sigma,\cdot)}}+\sqrt{\epsilon} a^{-3-c\sqrt{\epsilon}}\sqrt{\E^{(\leq L+1)}(N,\cdot)}\\
&\,+\epsilon a^{-3-c\sqrt{\epsilon}}\sqrt{\E^{(\leq L-1)}(\Ric,\cdot)}\\
\numberthis\label{eq:L2-junk-S}\|\mathfrak{S}_{L,Junk}^\parallel\|_{L^2_G}\lesssim&\,\epsilon a^{1-c\sigma}\sqrt{\E^{(L)}(\Sigma,\cdot)}+\epsilon a^{-3-c\sqrt{\epsilon}}\sqrt{\E^{(\leq L-2)}(\Sigma,\cdot)}+\sqrt{\epsilon}a^{-1-c\sqrt{\epsilon}}\sqrt{\E^{(L)}(\phi,\cdot)}\\
&\,+\left(\epsilon a^{-3}+a^{1-c\sqrt{\epsilon}}\right)\sqrt{\E^{(\leq L)}(N,\cdot)}+\epsilon a^{5-c\sigma}\sqrt{\E^{(\leq L-1)}(\Ric,\cdot)}\\
&\,+\underbrace{\epsilon a^{-3}\sqrt{\E^{(L-2)}(\Ric,\cdot)}+\epsilon a^{-3-c\sqrt{\epsilon}}\sqrt{\E^{(\leq L-4)}(\Ric,\cdot)}}_{\text{not present for }L=2}\\
\numberthis\label{eq:L2-junk-R-even}\|\mathfrak{R}_{L,Junk}\|_{L^2_G}\lesssim&\,\epsilon^2a^{1-c\sigma}\sqrt{\E^{(\leq L-1)}(\Ric,\cdot)}+\epsilon a^{-3-c\sqrt{\epsilon}}\sqrt{\E^{(\leq L-2)}(\Ric,\cdot)}\\
&\,+\epsilon a^{1-c\sigma}\sqrt{\E^{(\leq L+2)}(\Sigma,\cdot)}+a^{-3-c\sqrt{\epsilon}}\sqrt{\E^{(\leq L)}(\Sigma,\cdot)}\\
&\,+a^{-3-c\sqrt{\epsilon}}\sqrt{\E^{(\leq L)}(N,\cdot)}\\
\numberthis\label{eq:L2-junk-R-odd}\|\mathfrak{R}_{L+1,Junk}\|_{L^2_G}\lesssim&\,\epsilon^2a^{1-c\sigma}\sqrt{\E^{(\leq L)}(\Ric,\cdot)}+\epsilon a^{-3-c\sqrt{\epsilon}}\sqrt{\E^{(\leq L-1)}(\Ric,\cdot)}\\
&\,+\epsilon a^{1-c\sigma}\sqrt{\E^{(\leq L+3)}(\Sigma,\cdot)}+a^{-3-c\sqrt{\epsilon}}\sqrt{\E^{(\leq L+1)}(\Sigma,\cdot)}\\
&\,+a^{-3-c\sqrt{\epsilon}}\sqrt{\E^{(\leq L+1)}(N,\cdot)}
\end{align*}
\vspace{-1em}
\begin{align*}
\numberthis\label{eq:L2-junk-BR-par}\|\mathfrak{E}_{L,Junk}^\parallel\|_{L^2_G}+\|\mathfrak{B}_{L,Junk}^\parallel\|_{L^2_G}\lesssim&\,\epsilon a^{-1-c\sqrt{\epsilon}}\sqrt{\E^{(\leq L)}(W,\cdot)}+\epsilon a^{-3-c\sqrt{\epsilon}}\sqrt{\E^{(\leq L-2)}(W,\cdot)}\\
&\,+\epsilon a^{-1-c\sqrt{\epsilon}}\sqrt{\E^{(L)}(\phi,\cdot)}+\left(\epsilon a^{-3-c\sqrt{\epsilon}}+a^{-1-c\sqrt{\epsilon}}\right)\sqrt{\E^{(\leq L-2)}(\phi,\cdot)}\\
&\,+\sqrt{\epsilon} a^{1-c\sqrt{\epsilon}}\sqrt{\E^{(\leq L)}(N,\cdot)}+\epsilon a^{-3-c\sqrt{\epsilon}}\sqrt{\E^{(\leq L-2)}(N,\cdot)}\\
&\,+\epsilon a^{-1-c\sigma}\sqrt{\E^{(L)}(\Sigma,\cdot)}+\epsilon a^{-3-c\sqrt{\epsilon}}\sqrt{\E^{(\leq L-2)}(\Sigma,\cdot)}\\
&\,+\epsilon a^{-3}\sqrt{\E^{(L-2)}(\Ric,\cdot)}+\epsilon a^{-3-c\sqrt{\epsilon}}\underbrace{\sqrt{\E^{(\leq L-4)}(\Ric,\cdot)}}_{\text{not present for }L=2}
\end{align*}
\end{subequations}
\end{lemma}

\begin{proof}
Once again, this follows by applying the a priori estimates from subsection \ref{subsec:APlow} and Lemma \ref{lem:AP}, as well as the bootstrap assumption \eqref{eq:BsN} for the lapse, to deal with the lower order terms in the nonlinearities, and then applying Lemma \ref{lem:Sobolev-norm-equivalence-improved} as well as \eqref{eq:ibp-trick} wherever this is necessary. Further, especially in \eqref{eq:L2-junk-BR-par}, it is often more convenient to use the bootstrap assumption for $\|\nabla\phi\|_{C_G}$ instead of the a priori estimate \eqref{eq:APmidphi} to gain higher powers of $\epsilon$ in prefactors. \\
Recognizing that every low order curvature term can be estimated up to constant by $a^{-c\sqrt{\epsilon}}$ at worst (see \eqref{eq:APmidRic}), we also note that any of the highly nonlinear curvature terms in $\mathfrak{J}$-expressions turn out to be negligible after updating $c$ compared to Ricci energies arising from applying Lemma \ref{lem:Sobolev-norm-equivalence-improved} or compared to junk terms in which $\Ric[G]$ is tracked explicitly. %\todo{comment that the highly nonlinear curvature terms that arise in $\mathfrak{J}$-expressions are entirely negligible since the orders involved are increasingly low}\\
\end{proof}

\section{Appendix -- Future stability}\label{sec:appendix-fut}

Here, we collect the commutators in CMCSH gauge necessary to study the commuted scalar-field equations:

\begin{lemma}[Commutator formulas for future stabilty]\label{lem:fut-comm-formula} Let $\zeta$ be a scalar function on $\Sigma_T$. Then, the following formulas hold:
\begin{align*}
[\fdel,\nabla]\zeta=&\,0\\
[\fdel,\fLap]\zeta=&\,(\fdel(\fg^{-1})^{ab})\nabla_a\nabla_b\zeta-2(\fg^{-1})^{ab}\left(\div_{\fg}(\fn\fk)_a-2\nabla_a\fn\right)\nabla_b\zeta+\fX^k[\nabla_k,\fLap]\zeta
\end{align*}
Schematically, for $k\in\N$, this implies
\begin{subequations}
\begin{align*}
\numberthis\label{eq:[fdel,Lapk]}[\fdel,\fLap^k]\zeta=&\,\sum_{I_{\fn}+I_{\fk}+I_{\zeta}=2k-1}\nabla^{I_{\fn}}\fn\ast_{\fg}\nabla^{I_{\fk}}\fk\ast_{\fg}\nabla^{I_{\zeta}+1}\zeta\\
&\,+\sum_{I_{\fX}+I_{\Ric}+I_{\zeta}=2k-2}\nabla^{I_{\fX}}{\fX}\ast_{\fg}\nabla^{I_{\Ric}}\Ric[\fg]\ast_{\fg}\nabla^{I_\zeta+1}\zeta\\
\numberthis\label{eq:[fdel,nablaLapk]}[\fdel,\nabla\fLap^k]\zeta=&\,\sum_{I_{\fn}+I_{\fk}+I_{\zeta}=2k}\nabla^{I_{\fn}}\fn\ast_{\fg}\nabla^{I_{\fk}}\fk\ast_{\fg}\nabla^{I_{\zeta}+1}\zeta\\
&\,+\sum_{I_{\fX}+I_{\Ric}+I_{\zeta}=2k-1}\nabla^{I_{\fX}}{\fX}\ast_{\fg}\nabla^{I_{\Ric}}\Ric[\fg]\ast_{\fg}\nabla^{I_\zeta+1}\zeta
\end{align*}
\end{subequations}
\end{lemma}
\begin{proof}
This follows from straightforward computation.
\end{proof}


%%Eintrag im Inhaltsverzeichnis hinzufuegen?
%\addtocounter{page}{1}
%\addcontentsline{toc}{chapter}{Bibliography}
%\addtocounter{page}{-1}

% Stil festlegen: plain, alpha, unsrt, abbrv, ... 
\bibliographystyle{alpha}

\bibliography{bibliography}


\end{document}