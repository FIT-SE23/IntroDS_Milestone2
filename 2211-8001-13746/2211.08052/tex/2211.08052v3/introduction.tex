\section{Introduction}\label{sec:intro}

\subsection{Setting and main results}

We consider the Einstein scalar-field system 
\begin{subequations}
\begin{align}
\label{eq:ESF1}\Ric[\g]_{\mu\nu}-\frac12R[\g]\g_{\mu\nu}=&\,8\pi T_{\mu\nu}[\g,\phi]\\
\label{eq:EMT}T_{\mu\nu}=&\,\change{\nabbar_{\mu}\phi}\nabbar_{\nu}\phi-\frac12\g_{\mu\nu}\nabbar^\alpha\phi\nabbar_\alpha\phi\\
\label{eq:ESF2}\square_{\g}\phi=&\,0\,
\end{align}
\end{subequations}
with initial data $(g_0,k_0,\pi_0,\psi_0)$ on a closed\delete{ orientable} 3-manifold $M$ that admits a negative Einstein metric $\gamma$.\footnote{Here and throughout, $\pi_0$ and $\psi_0$ prescribe data for $\nabla\phi\vert_{\Sigma_{t_0}}$  and $\del_0\phi\vert_{\Sigma_{t_0}}$ respectively.} In this paper, we determine the maximal globally hyperbolic development emanating from such initial data given that it is sufficiently close to the initial data of a homogeneous solution with a non-trivial scalar field. \\
In the collapsing direction, we prove a stable Big Bang formation and curvature blow-up result, which requires the presence of a non-trivial scalar field. \change{The results complement those in \cite{Rodnianski2014,Speck2018}, which cover flat and spherical spatial geometry}. In the expanding direction, we prove a nonlinear future stability result of the corresponding vacuum background solution, which is the Milne model, under a mild condition for the first positive eigenvalue of \changefinal{$-\Lap_{\gamma}$ }(see Definition \ref{def:spatial-mf-spectral}). As discussed in more detail \change{in Remark \ref{rem:weeks-and-friends}}, numerical studies (see \cite{Cornish99, Ino01}) show that this condition holds for an analogue of Weeks space, and suggest that this may hold for all \change{closed }hyperbolic $3$-manifolds with sectional curvature $-\frac19$ .\\

Connecting the two regions, we prove the global stability \change{(i.e.,~ past and future stability) }of the spacetime
\begin{subequations}
\begin{equation}\label{eq:intro-ref1}
\left([0,\infty)\times M,-dt^2+a(t)^2\gamma\right),
\end{equation}
given \change{a negative Einstein manifold $(M,\gamma)$ obeying }the aforementioned spectral condition, with
\begin{equation}\label{eq:intro-ref2}
a(0)=0,\ \dot{a}=\sqrt{\frac19+\frac{4\pi}3C^2a^{-4}}
\end{equation}
for some given constant $C>0$, and the scalar field given by 
\begin{equation}\label{eq:intro-ref3}
\del_t\phi=Ca^{-3},\ \nabla\phi=0\,.
\end{equation}
The scale factor consequently exhibits the following asymptotic behaviour:
\begin{equation}\label{eq:intro-ref4}
a(t)\simeq t^{\frac13} \mbox{ as }t\searrow0 \mbox{ and } a(t)\simeq t \mbox{ as }t\nearrow \infty  
\end{equation}
\end{subequations}

\noindent The main result can be split into two parts:

\begin{theorem}[Big Bang stability -- rough version]\label{thm:main-past}
Let $(M,g_0,k_0,\pi_0,\psi_0)$ be initial data for the Einstein scalar-field system that is sufficiently close to $(M,a(t_0)^2\gamma,-\dot{a}(t_0)a(t_0)\gamma,0,Ca(t_0)^{-3})$, where $C>0$ and $(M,\gamma)$ is a closed \delete{orientable }Riemannian 3-manifold with $\Ric[\gamma]=-\frac29\gamma$ \change{(i.e.,~ a closed negative Einstein manifold with scalar curvature $-\frac23$)}.\\

Then, the past maximal globally hyperbolic development $((0,t_0]\times M,\g,\phi)$ of the initial data within the Einstein scalar-field system \eqref{eq:ESF1}-\eqref{eq:ESF2} admits a foliation by CMC hypersurfaces $\Sigma_s=t^{-1}(\{s\})$ \change{with zero shift}. This development remains close to the FLRW solution described in \eqref{eq:intro-ref1}-\eqref{eq:intro-ref3} in the past of the initial data slice $\Sigma_{t_0}$. In particular, the solution exhibits curvature blow-up of order $t^{-4}$ and \change{every causal geodesic becomes incomplete }as $t$ approaches $0$.
\end{theorem}
\begin{theorem}[Global stability]\label{thm:main-full}
Let $(M,g_0,k_0,\pi_0,\psi_0)$ be initial data as in Theorem \ref{thm:main-past}. In addition, we \change{suppose }that the smallest positive eigenvalue of $-\Lap_\gamma$ acting on scalar functions is strictly greater than $\frac19$.\\ Then, the initial data admits a \change{maximal globally hyperbolic development }$((0,\infty)\times M,\g,\phi)$ \change{solving }the Einstein scalar-field system that, in addition to the results of Theorem \ref{thm:main-past}, is future (causally) complete. As $t\nearrow\infty$, the solution is attracted by Milne spacetime in the sense that the expansion normalized variables $(\fg,\bm{k},\nabla\phi,\phi^\prime)$ \change{(see Definition \ref{def:fut-rescaled}) }converge toward $(\gamma,\frac13\gamma,0,0)$. 
\end{theorem}
A more detailed statement of Theorem \ref{thm:main-past} is provided in Theorem \ref{thm:main}. The additional spectral condition in Theorem \ref{thm:main-full} is discussed at the end of Subsection \ref{subsec:prev}, and the statement itself is proven in Section \ref{sec:full-stab} to be an extension of the Milne future stability result in Theorem \ref{thm:fut-stab-simple}.

%\footnote{In general, the Strong Cosmic Censorship Conjecture states that, generically, the maximal globally hyperbolic development of solutions to the Einstein equations is inextendible. For cosmological solutions, this is specified to conjecturing that the Kretschmann scalar blows up in directions of causal geodesic incompleteness, implying $C^2$-inextendibility. For further details, we refer to \cite[Chapter 17]{Ring09}.}  


%\begin{corollary}\label{cor:SCC}
%Let $(M,\gamma)$ be a closed negative Riemannian Einstein 3-manifold with $R[\gamma]=-\frac23$. Then there exists an open neighborhood of $(M,\gamma,\frac13\gamma,0,C)$ in the space of initial data to the Einstein scalar-field system for which the Strong Cosmic Censorship Conjecture holds in the $C^2$-sense.
%\end{corollary}
\change{
\subsection{Background material}

We now provide context for the previously discussed setting and the results in Theorems \ref{thm:main-past} - \ref{thm:main-full}:\\

\subsubsection{Initial data to the Einstein scalar-field equations}\label{subsubsec:initial-data}

It is well known that the Einstein equations can, via the 3+1 decomposition, be viewed as an elliptic-hyperbolic system of PDEs (see, for example, \cite{AM03B}). This reduces solving the Einstein equations to two problems: finding admissible Einstein initial data in physical space, and then solving the corresponding initial value problem. Regarding the former, initial data to the Einstein scalar-field system takes the form
\[(M,\mathring{g},\mathring{k},\mathring{\pi},\mathring{\psi}),\]
where $\mathring{g}$ and $\mathring{k}$ are symmetric $(0,2)$-tensors on $M$,  $\mathring{\pi}$ is an exact $(0,1)$-tensor (corresponding to $\nabla\phi$) and $\mathring{\psi}$ is a scalar function (corresponding to the future directed normal derivative $\del_0\phi$ of the scalar field). The initial data must satisfy the Hamiltonian and momentum constraints
\begin{subequations}
\begin{align}
\text{R}[\mathring{g}]+\left({\mathring{k}^{a}}_{\ a}\right)^2-\left({\mathring{k}^{a}}_{\ b}{\mathring{k}^b}_{\ a}\right)=&\,8\pi\left[\lvert\mathring{\psi}\rvert^2+\lvert\mathring{\pi}\rvert_{\mathring{g}}^2\right] \label{eq:init-Hamilton}\,,\\
\div_{\mathring{g}}\mathring{k}=&\,-8\pi\cdot\mathring{\pi}\cdot\mathring{\psi}\,\label{eq:init-momentum}
\end{align}
\end{subequations}
(see \eqref{eq:Hamilton} and \eqref{eq:Momentum}), where the indices of $\mathring{k}$ in the first line are raised with respect to $\mathring{g}$.\\ 
We note that, in our argument, we will additionally assume that our initial data has constant mean curvature so that our gauges can be satisfied initially -- this is enforced on the level of initial data by requiring
\changefinal{\begin{equation*}
\text{tr}_{\mathring{g}}\mathring{k}=-3\frac{\dot{a}(t_0)}{a(t_0)}\,
\end{equation*}
(see \eqref{eq:CMC}).} We will argue in Remark \ref{rem:CMC-hypersurface} why the initial data being near-FLRW allows us to assume the initial hypersurface to be CMC without loss of generality.\\

The results of \cite{FB52, CBGer69} show that there exists an embedding\footnote{We usually ignore the embedding in notation.} $\iota:M\hookrightarrow\iota(M)\subset\M$ and a \changefinal{maximal }solution $(\M,\g,\nabla\phi,\del_0\phi)$ to the Einstein scalar-field equations such that $\iota(M)=\Sigma_{t_0}$ is a Cauchy hypersurface and such that
\[\iota^\ast\g=\mathring{g},\,\iota^\ast k=\mathring{k},\,\iota^\ast\pi=\mathring{\pi},\,\text{and }\iota^\ast \del_0\phi=\mathring{\psi_0}\,.\]
We will perturb around initial \changefinal{data corresponding to data for an FLRW spacetime at time $t=t_0$, i.e.,~
\[(M\cong\Sigma_{t_0},a(t_0)^2\gamma,-\dot{a}(t_0)\,a(t_0)\,\gamma,0,C\,a(t_0)^{-3})\,.\]} Furthermore, the maximal globally hyperbolic development (MGHD) is unique (up to diffeomorphism), and thus we can assume $(\M,\g,\nabla\phi,\del_0\phi)$ to be globally hyperbolic. However, these statements provide little information on the properties of the MGHD in the future and past of the initial data slice.

%\todo{Build this remark in}
%\begin{remark}[On initial data]\label{rem:initial-data} To properly analyse the Cauchy problem for the system of equations in Proposition \ref{prop:eq}, we quickly recall the form initial data $(M,\mathring{g},\mathring{k},\mathring{\pi},\mathring{\psi})$ has to take on the initial CMC hypersurface $\Sigma_{t_0}$ (beside $M$ being as in Definition \ref{def:spatial-mf}): We let $\mathring{g}$ and $\mathring{k}$ be symmetric $(0,2)$-tensors on $\Sigma_{t_0}$,  let $\mathring{\pi}$ be a $(0,1)$-tensor on $\Sigma_{t_0}$ and $\mathring{\psi}$ be a scalar function from $\Sigma_{t_0}$ to $\R$. We then smoothly embed $\Sigma_{t_0}$, which is diffeomorphic to $M$ (see \ref{def:spatial-mf}) into a time-oriented spacetime $(\M,\g)$ along $\iota$ such that $\iota(\Sigma_{t_0})$ is a Cauchy hypersurface and such that
%\[\iota^\ast\g=\mathring{g},\,\iota^\ast k=\mathring{k},\,\iota^\ast\pi=\mathring{\pi},\,\text{and }\iota^\ast \del_0\phi=\mathring{\psi_0},\]
%where $\del_0$ is the future directed unit normal of $\iota(\Sigma_{t_0})$.\ \footnote{We usually ignore the embedding in notation.}\\
%The initial data must now satisfy the Hamiltonian and momentum constraints
%\begin{subequations}
%\begin{align}
%\text{R}[\mathring{g}]+\left({\mathring{k}^{a}}_a\right)^2-\left({\mathring{k}^{a}}_b{\mathring{k}^b}_a\right)=&\,8\pi\left[\lvert\mathring{\psi}\rvert^2+\lvert\mathring{\pi}\rvert_{\mathring{g}}^2\right] \label{eq:init-Hamilton}\\
%\div_{\mathring{g}}\mathring{k}=&\,-8\pi\cdot\mathring{\pi}\cdot\mathring{\psi}\,\label{eq:init-momentum}
%\end{align}
%\end{subequations}
%(see \eqref{eq:Hamilton} and \eqref{eq:Momentum}), where the indices of $\mathring{k}$ in the first line are raised with respect to $\mathring{g}$. Further, we impose CMC gauge by requiring that the tensor
%\begin{equation*}
%\hat{\mathring{k}}=\mathring{k}-\frac{\tau}3\mathring{g}
%\end{equation*}
%is tracefree, where $\tau=-3\frac{\dot{a}}a$ is defined by the scale factor solving the Friedman equation \eqref{eq:Friedman}. We will argue in Remark \ref{rem:CMC-hypersurface} why the initial data being near-FLRW allows us to assume the initial hypersurface to be CMC without loss of generality.
%\end{remark}
%

\subsubsection{Strong Cosmic Censorship}

In their groundbreaking papers on singularity theorems, Hawking \cite{Hawk67} and Penrose \cite{Pen65} established very general criteria for the MGHD of spacetimes to become causally geodesically incomplete. Many spacetimes of physical relevance satisfy these criteria, including the spacetimes considered in this article. While giving us more information on the MGHD than the existence and uniqueness results mentioned above, a key issue in the application of this mathematical result to General Relativity is that no statement is made on how precisely the singularity comes about: In particular, such incompleteness (within a given regularity class) could either mean that the geodesic is inextendible -- which must be caused by the blow-up of some geometric quantity -- or that there exist multiple inequivalent extensions. While the latter behaviour is exhibited even for some cosmological spacetimes (see, for example, the Taub solutions discussed in \cite{ChrIs93}), such behaviour is usually considered to be unphysical since it would imply a breakdown of determinism. The \textit{Strong Cosmic Censorship Conjecture} (SCCC) posits in its most general form that, for generic solutions to the Einstein equations, this incompleteness instead manifests as inextendibility at a given level of regularity (e.g., $C^0, C^2, C^\infty,\dots$).\\

\noindent In certain frameworks in the homogeneous cosmological setting -- i.e. for homogeneous initial data on a closed spatial hypersurface --, it was shown in fundamental works by Chrusciel-Rendall \cite{CR95} and Ringström \cite{Ring09} that the so called Kretschmann scalar $R_{\alpha\beta\gamma\delta}R^{\alpha\beta\gamma\delta}$ is unbounded where incompleteness manifests. Thus, it is the driving force behind geodesic incompleteness in these cases, forcing $C^2$-inextendibility of the MGHD. For the purposes of analyzing cosmologically relevant spacetimes, the SCCC is hence often rephrased as follows:
\begin{conjecture}[Cosmological SCCC](See e.g. \cite[Chapter 17]{Ring09})
For generic initial data, the Kretschmann scalar is unbounded where causal geodesics become incomplete.
\end{conjecture}
Theorem \ref{thm:main-past}, in short, shows that this conjecture is rigorously supported in the case of FLRW spacetimes with negative spatial curvature. More precisely, the past asymptotics of such spacetimes, determined by initial data on $\Sigma_{t_0}$ as discussed above, are generic in the following sense: There exists an open neighbourhood of said FLRW data within the set of Einstein scalar-field initial data  such that the solutions past directed causal geodesics become incomplete, and the incompleteness is driven by blow-up of Kretschmann scalar with the same asymptotics as the FLRW solution. The global result in Theorem \ref{thm:main-full} portrays the other side of Cosmic Censorship -- as with the past evolution, near-FLRW data fully determines the future of the spacetime in the sense that the MGHD is future complete, again showing that this feature of FLRW spacetimes with negative spatial sectional curvature is generic.

\subsubsection{FLRW and generalized Kasner spacetimes with scalar fields}\label{subsec:FLRW-Kasner}

On a large scale, the universe is often viewed as spatially homogeneous and isotropic, i.e., no point in space and no direction are distinguishable from any other point and direction \changefinal{(referred to as the \enquote{Cosmological Principle})}. In 1935, it was shown by Robertson and Walker that, under a few very natural additional assumptions, this restricts the class of potential spacetimes to the FLRW class
\[\left(I\times\tilde{M},\,\tilde{g}_{FLRW}=-dt^2+a(t)^2\tilde{\gamma}\right)\,,\]
where $(\tilde{M},\tilde{\gamma})$ is a manifold of constant sectional curvature $\kappa$ and where the scale factor $a$ depends smoothly on $t$. 
This holds before taking the Einstein equations into consideration -- when doing so, the matter model determines how space expands within the cosmological model via $a$. We refer to Lemma \ref{lem:FLRW} for the scalar-field solution for $\kappa=-\frac19$, but note that the scale factor behaves like $t^\frac13$ for scalar-field matter, regardless of spatial geometry, and that the Kretschmann scalar blows up at order $\O{t^{-4}}$ toward the Big Bang ($t\downarrow 0$).\\
Spatially flat FLRW spacetimes are a subclass of the closely related \textit{generalized Kasner spacetimes}, which are still spatially homogeneous but anisotropic in general. For scalar field matter, the spacetime metric is given by
\changefinal{\begin{gather*}
\g_{Kasner}=-dt^2+\sum_{i=1}^Dt^{2p_i}dx^i\otimes dx^i,\quad
\sum_{i=1}^D{p_i}=1,\,\sum_{i=1}^Dp_i^2=1-8\pi A^2,\quad\overline{\phi}_{Kasner}(t)=A\log(t)\,.
\end{gather*}}
The standard Kasner family is obtained by considering the vacuum case ($A=0$), and the spatially flat FLRW spacetime by setting \changefinal{$D=3, p_i=\frac13, A=\sqrt{\frac1{12\pi}}$}. If more than one of the Kasner exponents is non-zero, the generalized Kasner family satisfies the SCCC, also by exhibiting Kretschmann scalar blow-up of order $t^{-4}$as $t\downarrow 0$ (see \cite[(1.8)]{Rodnianski2014}).\\

Kasner spacetimes are of particular relevance to cosmology due to their relationship with the \textit{BKL conjecture}: Heuristically, this conjecture states that the dynamics of cosmological spacetimes \changefinal{near a spacelike singularity }generically exhibit chaotic and highly oscillatory behaviour, often referred to as \enquote{Mixmaster} behaviour. This behaviour is driven by velocity terms within the Einstein equations and is locally comparable to that of (vacuum) Kasner solutions. However, even if the BKL picture is to be believed in general, scalar-field (or, more generally, stiff-fluid) solutions seem to form an exception to it: They have a dampening effect on said oscillations, thus generating Big Bang stability as shown rigorously in \cite{Rodnianski2014, RodSpFou20} for Kasner spacetimes (for more details, see Section \ref{subsec:prev}). This scenario, often referred to as \textit{quiescent cosmology}, was studied in, for example, \cite{BK73,Bar78,AnRen01}. With this in mind, both the aforementioned Kasner results and the results within this article, along with the prior FLRW results \cite{Rodnianski2014,Speck2018}, confirm this quiescent effect of scalar fields in \changefinal{cosmology.}\\\

We note that one can view this as a scalar field ensuring a specific scenario in the very early universe given a class of initial data, namely matching the asymptotic behaviour of the Big Bang singularity. This fits into the recent use of nonlinear scalar fields in string cosmology, where specific choices of field are made to specific behaviours (e.g.,~inflation) in the early universe. For a recent review, we refer to \cite{StringCosm23}.
}\\

\subsection{Relation to previous work}\label{subsec:prev}

Theorem \ref{thm:main-full} is the first theorem about the full global structure of FLRW spacetimes with negatively curved spatial geometry. For such solutions, \changefinal{prior }results exclusively concern future stability, which we further discuss below. Besides \cite{Speck2018} covering the $\mathbb{S}^3$-case, it is the only open set of \change{initial data for cosmological spacetimes }(i.e.,\,\,without symmetry assumptions) with $\Lambda=0$ and in absence of accelerated expansion for which the global (future and past) dynamics are now fully understood.\footnote{\change{For a related future stability result in accelerated expansion, see \cite{Ring08} which considers scalar fields with a non-trivial potential.}}\\

Scalar field matter (and, more generally, \changefinal{matter obeying }semilinear wave equations or fluid matter) and their asymptotic behaviour on fixed cosmological backgrounds have been studied extensively, for example in \cite{AlRen10, Franzen18, Bach19, Ring19, BO24, Ring21linear, Ring21wave, Wang21}. While many of the results, in particular \cite{Ring21linear}, manage to analyze very general classes of equations and spacetime geometries\change{, }including the wave equation on the FLRW backgrounds studied in \cite{Franzen18, FU22}, the methods used are often difficult to apply to the full Einstein scalar-field system. In \cite{Urban22}, we extended the approach of \cite{Franzen18} to be able to deal with various warped product spacetimes, and in particular FLRW spacetimes \change{with negatively curved spatial geometry}, by using the spatial Laplace operator to control high order derivatives. The perturbation-adapted analogue of this strategy is at the basis of the energy method in this paper.\\

\indent We also note that, by the results of \cite{Girao19}, there are non-trivial waves on fixed FLRW backgrounds that converge toward the Big Bang singularity, even if, as demonstrated in \cite{Franzen18,Urban22}, this behaviour is non-generic. Such waves can give rise to convergent asymptotics on cosmological backgrounds as studied in \cite{Ring21linear}. Thus, it will likely be difficult to replace \eqref{eq:intro-ref3} with an arbitrary non-trivial reference wave while keeping past stability intact. However, by restricting to an open neighbourhood near the solution described in \eqref{eq:intro-ref1}-\eqref{eq:intro-ref3}, \changefinal{potential non-generic solutions of this type are excluded. For the more general conditions on initial data that lead to quiescent asymptotics, we refer to \cite{GPR23}, which will be discussed further below.}\\
%Since the past and future developments are analyzed in two independent parts, it makes sense to discuss these separately:

Theorem \ref{thm:main-past} forms the counterpart to the pioneering works by Rodnianski-Speck \cite{Rodnianski2018,Rodnianski2014} and Speck \cite{Speck2018}, which cover nonlinear Big Bang stability for \change{FLRW spacetimes with }spatial geometry $\mathbb{T}^3$ and $\S^3$ respectively.
\change{These results were extended to Kasner spacetimes in \cite{RodSp22} with $\lvert q_i\rvert<\frac16$, and to the full subcritical regime in \cite{RodSpFou20}, i.e., (generalized) Kasner spacetimes as discussed in Section \ref{subsec:FLRW-Kasner} with $\changefinal{\max_{i,j,k=1,\dots,D}(p_i+p_j-p_k)<1}$. The former necessitates considering $1+D$-dimensional Kasner spacetimes with $D\geq 38$, while the latter result also can be satisfied in $D=3$ for generalized Kasner spacetimes. Recall that this means, in contrast to our setting, that the reference spacetime can be anisotropic, even \changefinal{if the }conditions on Kasner exponents rule out extremely anisotropic regimes. As a result, the analysis therein becomes significantly more involved, especially at top order, since approximately monotonic energy identities as used in our work as well as in \cite{Rodnianski2014,Speck2018} have not been found in these anisotropic settings.\\

\changefinal{We note that the argument in \cite{RodSpFou20} relies on identifying an almost-diagonal structure for the asymptotics of (combined) connection coefficients for an adapted frame that is carried along by Fermi-Walker transport; this is precisely where subcriticality enters. Given that these no longer can vanish in a reference frame adapted to near-hyperbolic spatial geometry, it is a priori unclear whether this structure is sufficiently maintained.\\
The impressive recent preprint \cite{GPR23} by Oude Groeniger, Petersen and Ringström circumvents this issue and uses the equations considered in \cite{RodSpFou20} to establish general conditions for initial data to the Einstein (non-linear) scalar-field equations to give rise to quiescent singularities (see \cite[Theorem 12]{GPR23}). Additionally, they show that a large class of cosmological model solutions to exhibit stable Big Bang formation (see \cite[Theorem 49]{GPR23}). In particular, by only requiring that the mean curvature is sufficiently large compared to the expansion-normalised data, the rescaled connection coefficients can be made to be sufficiently small even if they are non-trivial in the reference. However, this high level of generality comes at the cost of no longer being able to ensure that the expansion-normalized solution variables themselves, in particular the generalized Kasner exponents, remain close to the reference solution, in contrast to our asymptotic results in Theorem \ref{thm:main}.\\}

Furthermore, Beyer and Oliynyk have recently shown in \cite{BeyOl21} that, over $\mathbb{T}^3$, the Big Bang formation can be localized in the sense that data given solely on a ball within the initial hypersurface must also cause stable blow-up on a (smaller) ball on the Big Bang hypersurface. While this result further indicates that blow-up behaviour of near-FLRW spacetimes might be, at least, independent of global geometric properties as it seems to be a localizable, we note that proof of localized stability crucially relies on the flatness of the conformal reference spacetime. To be more precise, the proof relies on extending the local initial data to global data for a Fuchsian system of metric and matter quantities as well as, again, connection coefficients for an adapted, Fermi-Walker transported frame. However, the derivation of the system for the former explicitly seems to use flat spatial geometry to obtain the necessary Fuchsian form. This \changefinal{form }seems to similarly be broken as soon as the connection coefficients are not perturbed around $0$, since this would lead to inhomogeneous error terms of order $t^{-1}$ for the rescaled variables which are stronger than what the method, so far, accounts for.\\

By contrast, in \cite{Rodnianski2014, Speck2018}, the reference frame itself is used in the commutator method to obtain the necessary energy identities at high orders. In all of these works, it hence is a priori unclear how one could extend these methods to the negative spatial Einstein geometry of $(M,\gamma)$. }We provide an alternative approach that, besides establishing the complementary stability result to \cite{Rodnianski2014,Speck2018}, does not rely on any information on the spatial geometry of the reference manifold in its methodology (although it is of course relevant in determining the FLRW reference solution that we are studying). Instead, we rely on differential operators adapted to the evolved spatial metric. %Hence, we will sketch throughout where we believe arguments could be adapted to also cover \cite{Rodnianski2014, Speck2018}, 
Hence, we believe that our approach may also prove useful for stability problems in spatially inhomogeneous \change{(and hence also anisotropic) }settings. In light of \cite{RodSp22, RodSpFou20} in particular, the main challenge in achieving this would either be to find approximately monotonic energy identities with our Bel-Robinson approach that have not been observed previously, or to also find ways to circumvent the lack thereof. \\

To obtain Theorem \ref{thm:main-past}, we use the Laplace-Beltrami-operator (acting, respectively, on scalar functions and tensor fields) with respect to the (rescaled) evolved metric as our commutating operator instead of a fixed reference frame. This, in turn, leads us to replacing the wave-like system for metric and second fundamental form exploited in \cite{Rodnianski2014, Speck2018} by an evolutionary system in the second fundamental form and Bel-Robinson variables. \changefinal{The latter technique dates back to the fundamental works by Christodoulou-Klainerman \cite{ChrKl90,ChrKl93}, where it was used to analyse field equations on Minkowski space and then to show global stability of Minkowski space itself. It has also been applied to the future stability of Milne spacetimes in the vacuum Einstein equations by Anderson-Moncrief in \cite{AM03} and, more recently, within the massive Einstein Klein-Gordon system by Wang in \cite{Wang19}. }As far as we are aware, this method has not yet been applied to solutions that are not near-vacuum or in the context of Big Bang singularity formation.\\

\indent Toward the Big Bang, the solutions exhibit asymptotically velocity dominated (AVTD) behaviour in the sense that they behave, to leading order, like solutions to the Einstein scalar-field equations in CMC gauge with zero shift with all terms involving spatial derivatives set to zero (the \enquote{velocity term dominated} (VTD) equations). This behaviour also matches results obtained by studying high regularity solutions (e.g.,\,\cite{AnRen01}), or related works using Fuchsian methods that prescribe a behaviour at the singularity and then develop it locally, often under additional symmetry assumptions (e.g.,\,\cite{DHRW02, CBIM04, IsMon02, FL23}). \change{In particular, this asymptotic behaviour leads to the same types of \enquote{Kasner footprint states} as in \cite{Rodnianski2018,Rodnianski2014}: As one approaches the Big Bang, the rescaled variables converge toward tensor fields on the Big Bang hypersurface that precisely solve the truncated VTD equations. Further,  the distance between the footprints of the FLRW and the perturbed solution are controlled by the initial data. For example, the rescaled Weingarten map $a^3{k^a}_{b}$ converges to ${(K_{Bang})^a}_b$ on the Big Bang hypersurface, which is close to $\frac{\sqrt{4\pi}}3C\I^a_b$, the rescaled FLRW footprint (see \eqref{eq:asymp-K} and \eqref{eq:footprint-K}).}\\

What remains to be considered to obtain Theorem \ref{thm:main-full} is future stability, which we can reduce to future stability of the vacuum solution in the Einstein scalar-field system. This solution, called the Milne spacetime, has been shown to be stable within the set of vacuum solutions -- see \cite{AM11} -- and a range of other Einstein systems -- see, for example\,,  \cite{Wang19,AndFaj20,FajWy21,FOW24,BaFaj20,BraFajKr19} and related work in lower dimensions, e.g. \cite{AMT97, Mon08, Faj17, Faj20, Mondal20}. As such, our contribution to the study of future stability of Milne spacetimes is that we deal with the massless scalar field matter via corrected energy estimates which are inspired by work of Choquet-Bruhat and Moncrief in \cite{CBM01} for vacuum Einstein equations with $U(1)$-symmetry. \change{Out of the works listed above, only \cite{Wang19, FajWy21} deal with scalar field matter at all, namely the massive case. These fields exhibit stronger decay toward the future, making the matter components easier to deal with than in our analysis.}\\
The additional spectral condition is needed to ensure coercivity of the corrected scalar field energy. Numerical work, e.g. \cite{Cornish99, Ino01}, does not suggest that this condition is violated by any \change{closed }3-manifold with constant sectional curvature $\kappa=-\frac19$, and verifies that is is satisfied, for example, by an analogue of Weeks space in which the metric is appropriately scaled to have the required sectional curvature. \changefinal{The latter is also verified by the recent result \cite{BoMaPa25} that, amongst considering more general related settings, sufficiently constrains the spectrum of the Laplacian on Weeks space. }We refer to Remark \ref{rem:weeks-and-friends} where this discussed in more detail.

\subsection{Challenges in the proof}

The contracting and expanding regimes of near-FLRW spacetime are analyzed in two separate and methodologically independent parts. Before providing an overview of both arguments, we summarize the challenges that arise:

\subsubsection{Big Bang stability}

The main difficulties in establishing Big Bang stability are three-fold:\\

Firstly, we have to expect that the solutions are asymptotically velocity term dominated (as argued in Remark \ref{rem:AVTD}, we end up proving that this is the case)\change{, and thus that rescaled variables at best exhibit the same asymptotic behaviour as their counterparts in FLRW spacetime, up to a small perturbation in the asymptotic footprint.  For example, note that, in the reference FLRW spacetime, one has \[{(k_{FLRW})^i}_j=-3\frac{\dot{a}}a\I^{i}_j\approx-\frac1t\I^{i}_j\,.\] At best, the shear ${\hat{k}^i_j}$ of the perturbed solution then behaves like $\frac{\epsilon}t$. In fact, we show that this is the case in \eqref{eq:APSigma}. This implies that the contraction rescaled metric $G_{ij}=a^{-2}g_{ij}$ can only be controlled up to $\O{t^{-c\sqrt{\epsilon}}}$ (see \eqref{eq:APmidG}), since one has $\del_tg_{ij}\approx -2g_{il}{k^l}_j$ and thus
\[\del_t G_{ij}\approx G_{il}{\hat{k}^l}_{\ j}\approx \frac{\epsilon}t\ast G\,.\]}
%When comparing with the solutions of the velocity term dominated equations, we must, in fact, expect the metric to behave towards the Big Bang like
%\[a^2{M}^\prime\odot\exp\left(-2\hat{{K}}^\prime\int_t^{t_0}a(s)^{-3}\,ds\right),\]
%where ${M}^\prime$ and $\hat{{K}}^\prime$ are reference tensors on $M$ close to $\gamma$ and $0$ respectively, the latter is tracefree, and where $\odot$ and $\exp$ are meant as matrix products and exponentials (see also \eqref{eq:asymp-G} in the main stability result). \\
However, to be able to use the structure of the evolution equations to cancel terms in our energy arguments, we have to work with adapted quantities. For example, we need to use integration by parts with respect to $(\Sigma_t,G_t)$ to cancel high order scalar field terms with help of the (rescaled) wave equation that contains $\Lap_G$, or to obtain elliptic estimates from the lapse equation via the operator $\Lap_G$ or from the adapted div-curl-system for $\Sigma$ arising from the constraint equations. \\

\indent As a result, even the rescaled solution variables \change{will diverge at order $\O{t^{-c\sqrt{\epsilon}}}$} toward the singularity, so we need to track and control their rate of divergence within the bootstrap argument. This significantly complicates dealing with nonlinear terms, where the bootstrap assumptions often cannot be inserted naively. This in turn makes coercivity of the energies more involved to establish \change{(see Lemma \ref{lem:Sobolev-norm-equivalence-improved} and Remark \ref{rem:Sobolev-norm-equivalence-improved}), since this only holds up to curvature errors that also diverge and thus need to be carefully tracked.}\\ %In particular, this is also the reason we cannot use CMCSH gauge toward the Big Bang, see Remark \ref{rem:why-not-CMCSH}.\\

\change{Secondly, and in contrast to \cite{Rodnianski2014,Speck2018}, replacing the wave structure of the geometric evolution in the Einstein equations with our less geometry dependent Bel-Robinson framework seems to lose regularity at first glance: The \changefinal{energy estimates for the }evolution system for the scalar field energy and the geometric energies can be caricatured as follows\changefinal{: 
\begin{align*}
-\frac{d}{dt}\E^{(L)}(\phi,\cdot)\lesssim&\,\frac{\epsilon^\frac18}t\left[\E^{(L)}(\phi,\cdot)+\E^{(L)}(\Sigma,\cdot)\right]+\dots\\
-\frac{d}{dt}\left[\E^{(L)}(\Sigma,\cdot)+\E^{(L)}(W,\cdot)\right]+\dots\lesssim&\,\frac {\epsilon^\frac18}t\left[\E^{(L)}(\Sigma,\cdot)+\E^{(L)}(W,\cdot)\right]+\frac{\epsilon^{-\frac18}}t\cdot a^{4}\E^{(L+1)}(\phi,\cdot)+\dots\\
\end{align*}
Herein, the superscript refers to the order of derivatives, while $\E^{(L)}(\phi,\cdot), \E^{(L)}(\Sigma,\cdot)$ and $\E^{(L)}(W,\cdot)$ refer to energies for the scalar field, the rescaled tracefree part $\Sigma$ of second fundamental form and the Bel-Robinson variables respectively. }Thus, it seems that we lose derivatives in the scalar field and are not able to close the argument. This is remedied using the div-curl-system in $\Sigma$, see \eqref{eq:comeq-mom-div} and \eqref{eq:comeq-mom-curl}, which yields a weak estimate of the form
\[a^4\E^{(L+1)}(\Sigma,\cdot)\lesssim \E^{(L)}{(\phi,\cdot)}+\E^{(L)}(W,\cdot)+\E^{(L)}(\Sigma,\cdot)+\dots\,.\]
Combining these estimates to improve the bootstrap assumptions then necessitates an intricately constructed total energy to balance these different types of estimates against one another.\\}

Finally, given \eqref{eq:intro-ref3}, the rescaled time derivative of the scalar field is not small and does not become so toward the Big Bang. This leads to various terms within the core linearized evolutionary system of both matter and geometry that, if estimated naively, could lead to exponential blow-up toward the singularity. \change{When such terms occur in the scalar field energy evolution, this can be dealt with along similar lines as in \cite{Rodnianski2014, Speck2018}, but we incur additional large terms in our geometric evolution that only cancel using the explicit form of the Friedman equations, which we highlight in Lemma \ref{lem:en-error-cancellation} and its proof.}

\subsubsection{Future and global stability}

For Milne stability, the canonical Sobolev energies for the scalar field variables\change{, i.e.,~
\[\int_M\lvert\phi^\prime\rvert_{\fg}^2+\lvert\nabla\phi\rvert_{\fg}^2\,\vol{\fg}\]
and higher order analogues, }do not obey useful energy estimates. This can be overcome by adding an indefinite correction term \change{of the type 
\[\int_M\phi^\prime(\phi-\overline{\phi})\vol{\fg}\]
}to the canonical energy\change{, see Definition \ref{def:fut-stab}. This is similar to what was done in \cite{CBM01} in a $2+1$-dimensional setting, as well as similar to the indefinite terms we introduce in our geometric energy to control the wave system in the metric variables, as in previous work on Milne stability in different matter models, including \cite{AndFaj20, FajWy21}. That this corrected energy controls Sobolev norms relies on the aforementioned spectral condition. As a result, and unlike for past stability, the specific spatial geometry is crucial in generating decay from energy estimates, even before considering the geometric evolution.}\\

Moreover, we need to transition from the near-FLRW data used to analyze the contracting regime to data in the expanding regime on a distant enough future hypersurface such that it is near-Milne and the future stability result applies. \change{This requires \changefinal{a gauge }switch from CMC gauge with zero shift to CMCSH gauge, as well as careful control of the solution variables over a finite time interval using continuous dependence on initial data. For the former, close inspection of \cite{FajKr20} gives us a diffeomorphism close to the identity that maps the initial data for the metric to new data satisfying the spatially harmonic gauge condition, thus allowing us to switch gauges without losing proximity to the reference solution. }This is discussed in detail in Section \ref{sec:full-stab}.

%\subsubsection{Singularity formation and Strong Cosmic Censorship}
%\subsubsection{Other results on Big Bang formation in the Einstein equations}
%\subsubsection{Other stability results for FLRW spacetimes}
%\todo{[symmetry assumptions, Fuchsian methods, linear wave equations (us, Ringström, Oliynyk,...)]}

%\subsection{Main result - big bang formation}\label{subsec:intro-main}

%\begin{theorem}[Main result - short version]\label{thm:main-short}
%Let $(M,\mathring{g},\mathring{k},\mathring{\phi},\mathring{\psi})$ be geometric initial data satisfying the Gauss-Codazzi-constraint equations, where $M$ is closed, orientable three-dimensional manifold that admits a metric $\gamma$ of constant negative sectional curvature. Further, when embedding it on the CMC hypersuface $\Sigma_{t_0}$ on $\M=(0,\infty)\times M$, assume it is  $\epsilon^2$-close in $H_\gamma^{N}$ and $C^{N-4}_\gamma$ to FLRW initial data $(a(t_0)^2\gamma,-\frac{\dot{a}(t_0)}{a(t_0)}\gamma,0,C\cdot a(t_0)^{-3})$ for sufficiently large $N\in\N$. \\
%
%Then, this initial data admits a solution $(\g,\phi)$ to the Einstein scalar-field system \eqref{eq:ESF1}-\eqref{eq:ESF2} foliated by CMC hypersurfaces $\Sigma_s=t^{-1}(\{s\})$ that remains close to the FLRW solution. For $\g=-n^2dt^2+g$, second fundamental form $k$ and spatial volume form $\vol{g}$, the following estimates hold:
%\begin{align*}
%\|n-1\|_{C^0_\gamma(\Sigma_t)}\lesssim&\,\epsilon a(t)^{4-c\epsilon^\frac18}\\
%\|a^{-3}\vol{g}-\vol{Bang}\|_{C^0_{\gamma}(\Sigma_t)}\lesssim&\,\epsilon a(t)^{4-c\epsilon^\frac18}\\
%\|a^3\del_t\phi-(\Psi_{Bang}+C)\|_{C^0_{\gamma}(\Sigma_t)}\lesssim&\,\epsilon a(t)^{4-c\epsilon^\frac18}\\
%\left\|\phi-\int_t^{t_0}a(s)^{-3}\,ds\cdot(\Psi_{Bang}+C)\right\|_{C^0_\gamma(\Sigma_t)}\lesssim&\,\epsilon a(t)^{4-c\epsilon^\frac18}\\
%\|a^3k-K_{Bang}\|_{C^0_\gamma(\Sigma_t)}\lesssim&\,\epsilon a(t)^{4-c\epsilon^\frac18}\\
%\left\|g_{(\cdot)l}\exp\left[(2\int_t^{t_0}a(s)^{-3}\,ds\cdot K_{Bang})\right]^l_{(\cdot)}-M_{Bang}\right\|_{C^0_\gamma(\Sigma_t)}\lesssim&\, \epsilon a(t)^{4-c\epsilon^\frac18}
%\end{align*}
%Here, $\exp$ is meant as a matrix exponential, $\vol{Bang},\,\Psi_{Bang},\,K_{Bang}$ and $M_{Bang}$ are all footprint states on $M$ that are $K\epsilon$-close in $C^0_\gamma$ to the FLRW footprints $\vol{\gamma},\,0,\,-\sqrt{\frac{4\pi}3}C\delta$ and $\gamma$ respectively, $\Psi_{Bang}$ and $K_{Bang}$ satisfy algebraic identities that ensure consistency with the CMC condition and the Gauss-Codazzi-constraints.\\
%Finally, $(\M,\g)$ becomes geodesically incomplete at the Big Bang hypersurface $\Sigma_0$, the Kretschmann scalar of $\g$ blows up toward $t=0$ like $a^{-12}$ (i.e. $t^{-4}$), and its Weyl curvature invariant like $\epsilon a^{-12}$ (i.e. $\epsilon t^{-4}$).
%\end{theorem}

\subsection{Proof outline}\label{subsec:intro-pf-outline}\changediss{\phantom{m}\\}
\subsubsection{Big Bang stability}\phantom{m}\\

\textbf{The big picture.} The key argument in our Big Bang stability proof is a hierarchized series of energy estimates that establishes the asymptotic behaviour of solution variables toward the singularity. We rely on a bootstrap argument which establishes that energies $\E^{(L)}$ (see Definition \ref{def:energies}) \change{for the scalar field, the rescaled shear, the Bel-Robinson variables, the lapse and the curvature }at worst only diverge slightly. Here, $0\leq L\leq \change{18}$ denotes the order of \change{derivatives considered}. To this end, we make a bootstrap assumption on the solution \changefinal{norm $\mathcal{C}$ }(see Definition \ref{def:sol-norm}) which controls the distance of \change{these rescaled variables, as well as the metric itself, }to their FLRW counterparts in \change{terms of supremum norms with respect to $G$}, where \change{$G=a^{-2}g$ }is the rescaled \textit{adapted} spatial metric (see Definition \ref{def:rescaled}). We refer to Assumption \ref{ass:bootstrap} and Remark \ref{rem:bs-strategy} for the detailed bootstrap assumptions and improvements\change{, as well as to Lemma \ref{lem:lwp} for the underpinning local well-posedness result. }That this bootstrap argument implies Theorem \ref{thm:main-past} follows from a straightforward adaptation of the arguments in \cite[Theorem 15.1]{Rodnianski2014}.\\

We work with evolution-adapted \change{norms }even though $G(t,x)$ degenerates toward the Big Bang singularity. Indeed, since we need to exploit the structure of the evolutionary equations, it is more convenient to have these adapted quantities controlled by the solution norms $\mathcal{H}$ and $\mathcal{C}$ directly instead of having to perform changes of metric at that point. Once the improved energy estimates are shown, a (time-scaled) coercivity notion (see Lemma \ref{lem:Sobolev-norm-equivalence-improved} and the proof of Corollary \ref{cor:H-imp}) and Sobolev embeddings with respect to the reference metric $\gamma$ then ensure that these improved estimates translate to $\mathcal{H}$ and $\mathcal{C}$. This then closes the bootstrap. To actually achieve this improved energy behaviour, we derive elliptic energy estimates or integral-type estimates that, once suitably combined and scaled, yield the desired improvements by straightforwardly applying the Gronwall lemma. \change{Additionally, note that we assume that the initial data is close to FLRW data not just in $\mathcal{H}$, which contains precisely the norms needed to control $\mathcal{C}$ by Sobolev embedding, but also scaled smallness assumptions at one order higher, contained in the top order semi-norm $\mathcal{H}_{top}$ (see Assumption \ref{ass:init}). This is needed to ensure that the top order energy is small initially, and thus to close the bootstrap.}\\

\textbf{Scale factor \changefinal{$a(t)$}.} The precise structure of the Friedman equations \eqref{eq:Friedman}-\eqref{eq:Friedman2} is crucial not only to control time integral quantities up to the Big Bang hypersurface (see Lemma \ref{lem:scale-factor}), but also to ensure that certain terms in the evolution that would otherwise cause large divergences contribute with favourable sign (see the arguments in Lemma \ref{lem:en-est-SF} as well as Lemma \ref{lem:en-error-cancellation}). It turns out that the sectional curvature entering the Friedman equations actually is not of key importance to large parts of the Big Bang stability analysis\changefinal{: The leading order behaviour }of the scale factor toward the Big Bang singularity is determined via the Friedman equation \eqref{eq:intro-ref2} by the matter term, not the sectional curvature. This indicates that our method might extend to different settings.\\

\textbf{Gauge choice, commutation method and Bel-Robinson variables.} We commute the resulting elliptic-hyperbolic Einstein system with the Laplace-Beltrami operator $\Lap_G$ with respect to the rescaled evolved spatial metric \change{$G(t,x)$ }to obtain higher order energy control. Commuting with this operator has the advantage of leaving many integration-by-parts \change{identities }intact. These are needed to provide specific cancellations, e.g., \change{to cancel $\Lap^{\frac{L}2+1}\phi$-terms arising from the wave equation when computing $\del_t\E^{(L)}(\phi,\cdot)$. We also note that the only feature of the adapted metric we use is that it is close to $\gamma$, and do not use any further information on the geometry, e.g., by choosing a specific reference frame in our commutation method. Further, we employ CMC gauge with zero shift to avoid badly behaved shift terms (see Remark \ref{rem:why-not-CMCSH}).}\\

We still, however, need to deal with the Ricci term in the evolution equation for the second fundamental form. To \change{this end}, we consider the Bel-Robinson variables $E$ and $B$ which are $\Sigma_t$-tangent symmetric tracefree $(0,2)$-tensors and contain all information of the spacetime Weyl tensor $W[\g]$ (see Subsection \ref{subsec:BR}). Suitably projecting the Gauss-Codazzi equations admits additional constraint equations in terms of $E$ and $B$ that allow us to replace the Ricci tensor at the \enquote{cost} of introducing Bel-Robinson energies into the formalism\change{, see \eqref{eq:constr-E} and the rescaled version \eqref{eq:REEqConstrE}}. Further, $E$ and $B$ satisfy a Maxwell-type system (see Lemma \ref{lem:EEqBR}) that can be exploited to obtain energy estimates and, as with the other evolution equations, is well adapted to commutation with $\Lap_G$.\\

\textbf{A priori low order $C_G$-control.}  By applying the bootstrap assumptions on $\mathcal{C}$ to the evolution equations, we can immediately deduce improved low order estimates in $C_G^{l}$ for $l\geq 10$ for the solution variables by inserting them into the respective evolution equations (see Lemma \ref{lem:AP}), as well as via the maximum principle for the lapse (see Lemma \ref{lem:lapse-maxmin}). These usually still diverge slightly, mostly due to the asymptotic behaviour of $G$. However and crucially to our argument, at order $0$, the renormalized time derivative \change{$\Psi$ }of the wave, the rescaled tracefree part \change{$\Sigma$ }of the second fundamental form and the rescaled Bel-Robinson variable $\RE$ are in fact $K\epsilon$-small in $C^0_G$ on the bootstrap interval (see Lemma \ref{lem:APzero}). \change{If these estimates did not hold, it would lead to terms that diverge at order $\O{a^{-3-c\sqrt{\epsilon}}}$ in the differential inequalities, and thus cause exponential energy blow-up of order $\O{e^{a^{-c\sqrt{\epsilon}}}}$ that we could no longer control. This behaviour is closely related to the fact that $\Psi$ and $\Sigma$ converge toward footprint states on the Big Bang hypersurface that remain $K\epsilon$-small (see \eqref{eq:asymp-Psi} and \eqref{eq:asymp-K}), and then pass this convergence on to $\lvert\RE\rvert_G$ (see \eqref{eq:asymp-E}).}\\

\textbf{Energy estimates and hierarchy.} The main part of the analysis is establishing various energy estimates.
\begin{itemize}
\item For the \underline{lapse} (see Section \ref{sec:lapse}), the relevant estimates are direct results of the elliptic lapse equations \change{\eqref{eq:REEqLapse1}-\eqref{eq:REEqLapse2}. The non-lapse terms on the right hand side \changefinal{of \ref{eq:REEqLapse1} }only diverge slightly toward the Big Bang, in contrast to the divergence at \changefinal{order }$a^{-4}$ in \eqref{eq:REEqLapse1}, and thus allows one to show that, at lower derivative order, the lapse converges to $1$. However, since the right hand side of \eqref{eq:REEqLapse2} contains the scalar curvature of $G$, this estimate loses derivatives. On the other hand, \eqref{eq:REEqLapse1} does not lose derivatives, and the elliptic nature in fact allows one to \changefinal{estimate }lapse energies of order $L+2$ by energies in $\Sigma$ and the scalar field of \changefinal{order }$L$. This makes it possible to control the higher order lapse term occurring, for example, in \eqref{eq:REEqSigma}, without losing regularity. Conversely, both of these gains in regularity are at the cost of losing powers of $a$. In short, \eqref{eq:REEqLapse2} is needed to establish the asymptotic behaviour of the lapse, and \eqref{eq:REEqLapse1} to obtain improved energy bounds as a whole.}
\item The core \underline{matter} energy estimate (see Lemma \ref{lem:en-est-SF}) relies on delicate cancellations when computing the time derivative of $\E^{(L)}(\phi,\cdot)$. While we derive this in a fashion that differs from the energy flux method used in \cite{Speck2018}, the necessary cancellations to arrive at Lemma \ref{lem:en-est-SF} are similar. 
\item The (rescaled) tracefree component of the \underline{second fundamental form} $\Sigma$ (see Lemma \ref{lem:en-est-Sigma}) and the (rescaled) \underline{Bel-Robinson variables} $\RE$ and $\RB$ (see Lemma \ref{lem:en-est-BR}) need to be treated simultaneously to deal with the leading curvature term in the evolution of the former by inserting a constraint equation in which $\RE$ occurs as the leading term (see \eqref{eq:comeq-Ham-BR}). However, the matter terms within the evolution of $\RE$ and $\RB$ contain, firstly, terms where we again need very precise estimates to show that they do not contribute large $a^{-3}$-divergences, and, secondly, matter terms that lose one order of derivative.\\
\change{This order of regularity can be regained using }the momentum constraint equation \eqref{eq:comeq-mom-div} and its Bel-Robinson counterpart \eqref{eq:comeq-mom-curl} containing $\RB$, {which leads }to a div-curl-system for $\Sigma$ \change{(see Lemma \ref{lem:en-est-Sigma-top}). This is, again, at the cost of losing powers of $a$.}
\item As a result, the \underline{core Gronwall argument} performed in Proposition \ref{prop:en-bs-imp} combines energies for the matter variables, $\Sigma$ and the Bel-Robinson variables, as well as energies for $\Ric[G]$. \change{In particular, the curvature energies are necessary to handle commutation errors within the energy estimates, and improved bounds on them need to be obtained to apply the coercivity results in Lemma \ref{lem:Sobolev-norm-equivalence-improved} -- else, none of energy improvements would extend to improved Sobolev norm bounds and the bootstrap argument would not close. }\\
As many of the a priori $C_G$-norm estimates add small additional divergences, it is necessary to perform an induction over derivative orders within this mechanism to deal with lower order error terms. Since $\Lap_G$ is elliptic, it is sufficient to perform this for even orders. \change{Along with energies at order $L\in 2\N_0$, the total energy also includes the energy controlling $\Sigma$ as well as the scalar field and curvature energies at order $L+1$, appropriately scaled to account for the degenerate elliptic estimate for $\Sigma$ from Lemma \ref{lem:en-est-Sigma-top}. This remedies the derivative loss in the Bel-Robinson energy and allows one to improve the total energy at each order until reaching $L=18$, at which point the bootstrap argument can be closed.}
\item Note that \textit{the \underline{metric} itself does not enter the core energy mechanism}. In fact, trying to replace control of the Ricci tensor by control of $G$ is likely too imprecise in dealing with high order curvature errors. Instead, control of $G-\gamma$ \change{and }$\Gamma[G]-\Gamhat[\gamma]$ is a consequence of a simple integral energy inequality and the improvements achieved for $\Sigma$ and matter variables (see Lemma \ref{lem:norm-est-G} and Corollary \ref{cor:H-imp}). Since we cannot utilize any additional structure in dealing with the metric, we have to construct our argument carefully to allow for the metric control to be weaker than what one gets for the core variables, while still being sufficiently strong to constitute an improvement and allowing to switch between $H_G$ and $H_\gamma$ (and, respectively, $C_G$ and $C_\gamma$) norms.
\end{itemize}
We also point to Remark \ref{rem:en-est-strat} for a more detailed sketch of how the integral inequalities for the core Gronwall argument are structured and how this leads to the bootstrap improvement for the energies.

\subsubsection{Future stability and connecting the regions}

We follow similar lines as in \cite{AndFaj20, FajWy21} to prove that near-FLRW spacetimes in negative spatial geometry are future stable. \change{Since $\del_t\phi$ decays like $a^{-3}\simeq t^{-3}$ in the reference spacetime, the sectional curvature becomes dominant in the Friedman equations and the scale factor approaches that of Milne spacetime as $t$ approaches $\infty$. Hence, if one moves sufficiently far \changefinal{to the future}, choosing near-FLRW data with a homogeneous scalar field is equivalent to choosing near-vacuum data. Thus, }what we prove first in Section \ref{sec:fut} is future stability of near-Milne spacetimes under the Einstein scalar-field system. Once this is established, we argue in Section \ref{sec:full-stab} how early near-FLRW initial data evolves to data that is sufficiently close to Milne for large enough times, which is essentially a consequence of the scale factor and the (physical) mean curvature approaching that of Milne, up to a multiplicative constant.\\

In terms of dealing with geometric and elliptic estimates, we can essentially carry over the results of \cite{AndFaj20}, as was also done in \cite{FajWy21}, by working in CMCSH gauge and verifying that the matter components are indeed only perturbative terms within the geometric evolution.

This leaves only the scalar field to be examined. Here, we introduce corrective terms to the energies (see Definition \ref{def:fut-stab}) which yield decay estimates for the corrected scalar field energy (see Lemmas \ref{lem:fut-en-est-ESF0} and \ref{lem:fut-en-est-ESF}). That these energies are coercive (see Lemmas \ref{lem:fut-ESF-coercivity} and \ref{lem:fut-Sob-est}) requires the aforementioned lower bound for the first positive eigenvalue of \changefinal{$-\Lap_\gamma$}.


\begin{remark}[Why not use CMCSH gauge to prove Big Bang stability?]\label{rem:why-not-CMCSH}
\change{One might consider applying this gauge to Big Bang stability as well since this is precisely the choice of gauge turning the geometric evolution into a wave-like system in $(g,k)$, which seems simpler than our chosen approach in CMC gauge with zero shift. }In particular, this would also not rely on any choice of reference frame, and keep the wave structure of the geometric evolution intact, unlike when using Bel-Robinson variables. However, the issue with this approach lies in the shift equation, which would take the following form for the rescaled shift vector $X=a^3\tilde{X}$:
\begin{align*}
\Lap_GX^l+\Ric[G]^l_mX^m=&-2(N+1)(G^{-1})^{im}(G^{-1})^{jn}\Sigma_{ij}\left(\Gamma_{mn}^l-\Gamhat_{mn}^l\right) \numberthis\label{eq:REEqShift}\\
&+2(G^{-1})^{im}\nabla_iX^n\left(\Gamma_{mn}^l-\Gamhat_{mn}^l\right)\\
&\,+\langle\text{error terms in lapse and matter}\rangle
\end{align*}
As a result, the first term has to be expected to diverge at the same rate as the metric, i.e., we expect even low order norms of $\tilde{X}$ to behave like $a^{-3-c\sqrt{\epsilon}}$ at best up to small prefactors. However, computing the time derivative of an integral over $\lvert G-\gamma\rvert_G^2$ (or derivatives thereof) becomes the integral over the $(\del_t-\Lie_{\tilde{X}})$-derivative of this quantity, and hence we get explicit terms of the form $\Lie_{\tilde{X}}\gamma$ which always exist at highest order and diverge worse than $t^{-1}$. In short, the fact that the metric cannot be expected to converge to a footprint state leads to leading order terms in the differential energy estimates to carry strongly divergent pre-factors in CMCSH gauge. This obstructs improvements in a tentative bootstrap argument.
%Notice that, even on the level of elliptic estimates, the norm of $\fX$ is quadratic in Sobolev norms of $\fk$ and $\fg-\gamma$. For future stability, this is beneficial since the bootstrap assumption implies that these quantities are small on the bootstrap interval, so this only contributes error terms when studying the geometric evolution equations. However, since we have to expect at least $G-\gamma$ to diverge, so do these geometric energies, and in particular in ways we cannot pre-improve. 
\end{remark}

\subsection{Paper outline}\label{subsec:intro-paper-outline}

\begin{itemize}
\item Sections \ref{sec:prelim}-\ref{sec:main-thm} cover the proof of Big Bang stability:
\begin{itemize}
\item In Section \ref{sec:prelim}, we introduce notation and provide the necessary information on the FLRW background solution as well as the equations relevant to the subsequent analysis.
\item Then, in Section \ref{sec:norm-en-bs}, we discuss the solution norms and energies and state the initial data and bootstrap assumptions.
\item In Section \ref{sec:ap}, improved low order $C_G$-norm estimates that follow directly from the bootstrap assumptions are established, along with additional formulas and a priori estimates.
\item Section \ref{sec:lapse} concerns the elliptic estimates for the lapse. 
\item \change{In Section \ref{sec:en-est}, }we discuss the energy and Sobolev norm estimates for all other variables, \change{all of which are integral estimates except for the aforementioned elliptic estimate for $\Sigma$, as well as a norm bound for $\nabla\phi$ that is not needed for the energy improvement.}
\item These are all combined in Section \ref{sec:bs-imp} to improve the bootstrap assumptions -- first for the energies, then for $\mathcal{H}$ and finally $\mathcal{C}$. \item In Section \ref{sec:main-thm}, we show how this bootstrap argument implies the main Big Bang stability result (see Theorem \ref{thm:main}, which is the formal version of Theorem \ref{thm:main-past}).
\end{itemize}
\item Section \ref{sec:fut} contains the proof of near-Milne future stability. 
\item \change{In Section \ref{sec:full-stab}, }we show that this is sufficient for future stability of near-FLRW spacetimes, proving Theorem \ref{thm:main-full}.
\item The appendices (Sections \ref{sec:appendix}-\ref{sec:appendix-fut}) collect various basic formulas and commutator expressions as well as error terms and how these can be estimated.
\end{itemize}