\section{Introduction}\label{sec:intro}

\subsection{Setting and main results}

We consider the Einstein scalar-field system 
\begin{subequations}
\begin{align}
\label{eq:ESF1}\Ric[\g]_{\mu\nu}-\frac12R[\g]\g_{\mu\nu}=&\,8\pi T_{\mu\nu}[\g,\phi]\\
\label{eq:EMT}T_{\mu\nu}=&\,\nabbar_{\mu}\nabbar_{\nu}\phi-\frac12\g_{\mu\nu}\nabbar^\alpha\phi\nabbar_\alpha\phi\\
\label{eq:ESF2}\square_{\g}\phi=&\,0\,
\end{align}
\end{subequations}
with initial data $(g_0,k_0,\pi_0,\psi_0)$\footnote{(Here and throughout, $\pi_0$ and $\psi_0$ prescribe data for $\nabla\phi\vert_{\Sigma_{t_0}}$  and $\del_0\phi\vert_{\Sigma_{t_0}}$ respectively.} on a closed orientable 3-manifold $M$ that admits a negative Einstein metric $\gamma$. In this paper, we determine the maximal globally hyperbolic development emanating from this initial data set given that it is sufficiently close to the initial data of a homogeneous solution with a non-trivial scalar field. In the collapsing direction, we prove a stable Big Bang formation and curvature blow-up result, which requires the presence of a non-trivial scalar field. In the expanding direction, we prove a nonlinear future stability result of the corresponding vacuum background solution, which is the Milne model, under a mild additional assumption on the spectrum of $\Lap_{\gamma}$ (see Definition \ref{def:spatial-mf-spectral}). Connecting the two regions, we prove the global stability of the spacetime
\begin{subequations}
\begin{equation}\label{eq:intro-ref1}
\left([0,\infty)\times M,-dt^2+a(t)^2\gamma\right)\,,
\end{equation}
given the aforementioned spectral condition, with
\begin{equation}\label{eq:intro-ref2}
a(0)=0,\ \dot{a}=\sqrt{\frac19+\frac{4\pi}3C^2a^{-4}}
\end{equation}
and the scalar field given by 
\begin{equation}\label{eq:intro-ref3}
\del_t\phi=Ca^{-3},\ \nabla\phi=0\,.
\end{equation}
The scale factor asymptotically behaves as follows:
\begin{equation}\label{eq:intro-ref4}
a(t)\simeq t^{\frac13} \mbox{ as }t\searrow0 \mbox{ and } a(t)\simeq t \mbox{ as }t\nearrow \infty  
\end{equation}
\end{subequations}

\noindent The main result can be split into two parts:

\begin{theorem}[Big Bang stability -- rough version]\label{thm:main-past}
Let $(M,g_0,k_0,\pi_0,\psi_0)$ be initial data to the Einstein scalar-field system that is sufficiently close to $(M,a(t_0)^2\gamma,-\dot{a}(t_0)a(t_0)\gamma,0,Ca(t_0)^{-3})$, where $C>0$ and $(M,\gamma)$ is a closed orientable negative Riemannian Einstein 3-manifold with $\Ric[\gamma]=-\frac29\gamma$.\\
Then, the past maximal globally hyperbolic development $((0,t_0]\times M,\g,\phi)$ of the initial data under the Einstein scalar-field system \eqref{eq:ESF1}-\eqref{eq:ESF2} admits a foliation by CMC hypersurfaces $\Sigma_s=t^{-1}(\{s\})$. This development remains close to the FLRW solution described in \eqref{eq:intro-ref1}-\eqref{eq:intro-ref3} toward the past of the initial data slice $\Sigma_{t_0}$. In particular, the solution exhibits curvature blow-up of order $t^{-4}$ and becomes geodesically incomplete as $t$ approaches $0$.
\end{theorem}
\begin{theorem}[Global stability]\label{thm:main-full}
Let $(M,g_0,k_0,\pi_0,\psi_0)$ be geometric initial data as above, where one additionally assumes that the smallest positive eigenvalue of $-\Lap_\gamma$ acting on scalar functions is strictly larger than $\frac19$.\\ Then, the initial data admits a global solution $((0,\infty)\times M,\g,\phi)$ to the Einstein scalar-field system that, in addition to the past behaviour of Theorem \ref{thm:main-past}, is future (causally) complete. As $t$ approaches $\infty$, the solution is attracted by the Milne spacetime in the sense that the expansion normalized variables $(\fg,\bm{k},\nabla\phi,\phi^\prime)$ converge toward $(\gamma,\frac13\gamma,0,0)$. 
\end{theorem}
A more detailed version of Theorem \ref{thm:main-past} is provided in Theorem \ref{thm:main}. The additional spectral assumption in Theorem \ref{thm:main-full} is further discussed at the end of Subsection \ref{subsec:prev}, and the statement itself is proven in Section \ref{sec:full-stab} to be an extension of the Milne stability result in Theorem \ref{thm:fut-stab-simple}.\\
In particular, these statements imply that there exists an open neighborhood of \linebreak $(M,a(t_0)^2\gamma,-\dot{a}(t_0)a(t_0)\gamma,0,Ca(t_0)^{-3})$ in the space of initial data sets to the Einstein scalar-field system for which the Strong Cosmic Censorship Conjecture holds in the $C^2$-sense:  The Kretschmann scalar is shown to exhibit stable blow-up of order $t^{-4}$ toward the Big Bang hypersurface $\Sigma_{t=0}$, which all causal geodesics emanating from the initial hypersurface reach in finite affine parameter time, while it is geodesically complete toward the future.
%\begin{corollary}\label{cor:SCC}
%Let $(M,\gamma)$ be a closed negative Riemannian Einstein 3-manifold with $R[\gamma]=-\frac23$. Then there exists an open neighborhood of $(M,\gamma,\frac13\gamma,0,C)$ in the space of initial data to the Einstein scalar-field system for which the Strong Cosmic Censorship Conjecture holds in the $C^2$-sense.
%\end{corollary}

\subsection{Relationship to previous work}\label{subsec:prev}

Theorem \ref{thm:main-full} is the first theorem about the global structure of FLRW spacetimes with negatively curved spatial geometry. For such solutions, existing results exclusively concern future stability, which we further discuss below. Beside \cite{Speck2018} covering $\mathbb{S}^3$-case, it is the only open set of cosmological spacetimes (i.e.\,without symmetry assumptions) with $\Lambda=0$ (and in absence of accelerated expansion generated by a scalar-field potential, see \cite{Ring08}) for which the global (future and past) dynamics are completely determined.\\

Scalar field matter, and more generally semilinear wave equations or fluid matter, and its asymptotic behaviour on fixed cosmological backgrounds have been studied extensively, for example in \cite{AlRen10, Franzen18, Bach19, Ring19, beyer2020relativistic, Ring21linear, Ring21wave, Wang21}. While many of the results, in particular \cite{Ring21linear}, manage to analyze very general classes of equations and spacetime geometries including the wave equation on the FLRW backgrounds studied in \cite{Franzen18, Urban22}, the methods used are often difficult to apply to the full Einstein scalar-field system. In \cite{Urban22}, we extended the approach of \cite{Franzen18} to be able to deal with various warped product spacetimes, and in particular FLRW spacetimes with spatial negative Einstein geometry, by using the spatial Laplace operator to control high order derivatives. The perturbation-adapted analogue of this will be at the basis of our energy method in the present paper.\\
We also note that, by the results of \cite{Girao19}, there are non-trivial waves on fixed FLRW backgrounds that converge toward the Big Bang singularity, even if this behaviour is non-generic as demonstrated in \cite{Franzen18,Urban22}. Such waves can give rise to convergent asymptotics on cosmological backgrounds as studied in \cite{Ring21linear}. Thus, it will likely be difficult to replace \eqref{eq:intro-ref3} with an arbitrary non-trivial reference wave while keeping past stability intact, but by restricting to an open set near the solution described in \eqref{eq:intro-ref1}-\eqref{eq:intro-ref3}, we exclude this non-generic class of solutions.\\
%Since the past and future developments are analyzed in two independent parts, it makes sense to discuss these separately:

Theorem \ref{thm:main-past} forms the counterpart to the pioneering works by Rodnianski-Speck \cite{Rodnianski2018,Rodnianski2014} and Speck \cite{Speck2018}, which cover nonlinear Big Bang stability for spatial geometry $\mathbb{T}^3$ and $\S^3$ respectively. These were extended to generalized Kasner systems in \cite{Rodnianski2018, RodSpFou20}, while \cite{BeyOl21} have recently shown that over $\mathbb{T}^3$, the Big Bang formation can be localized. \\
For all of the aforementioned results, the analysis is performed with regards to a reference frame on $\mathbb{T}^3$, respectively $\S^3$. The results in \cite{RodSpFou20, BeyOl21} rely on propagating the frame via Fermi-Walker transport, and thus need to stay close to the reference frame throughout the analysis, while in \cite{Rodnianski2014, Speck2018}, the reference frame itself is used in a commutation method. In any of these cases, albeit to somewhat varying degrees, it is a priori unclear how one could extend these methods to the negative spatial Einstein geometry of $(M,\gamma)$. \\
We provide an alternative approach that, beside establishing the complementary stability result to \cite{Rodnianski2014,Speck2018}, does not rely on any information on the spatial geometry of the reference manifold in its methodology (although it is of course relevant in determing the FLRW reference solution that we are studying), instead relying on differential operators adapted to the evolved spatial metric. %Hence, we will sketch throughout where we believe arguments could be adapted to also cover \cite{Rodnianski2014, Speck2018}, 
Hence, we believe that this approach may also prove useful for stability problems in spatially inhomogeneous settings.\\

To obtain Theorem \ref{thm:main-past}, we use the Laplace-Beltrami-operator (acting on scalar functions and tensor fields) with respect to the (rescaled) evolved metric as our commutating operator instead of a fixed reference frame. This, in turn, leads us to replacing the wave-like system for metric and second fundamental form exploited in \cite{Rodnianski2014, Speck2018} by an evolutionary system in the second fundamental form and Bel-Robinson variables. The latter technique to show stability of near-vacuum solutions dates back to \cite{AM03} and even to the classic proof of global stability of Minkowski space in \cite{ChrKl93}, which both covered the Einstein vacuum system. It has recently also been applied to the future stability of Milne spacetimes in the massive Einstein Klein-Gordon system \cite{Wang19}. As far as we are aware, this method has not been applied yet to solutions that are not near-vacuum or in the context of Big Bang singularity formation.\\
Toward the Big Bang, the solutions exhibit asymptotically velocity dominated behaviour in the sense that they behave, at leading order, like solutions to the Einstein scalar-field equations in CMC gauge with zero shift wherein all terms involving spatial derivatives are set to zero. This behaviour also matches results obtained by studying high regularity solutions (e.g. \cite{AnRen01}), or related works using Fuchsian methods that prescribe a behaviour at the singularity and then develop it locally, often under additional symmetry assumptions (e.g. \cite{CBIM04, IsMon02}). \\

What then remains to be considered to obtain Theorem \ref{thm:main-full} is future stability, which we can reduce to future stability of the vacuum solution in the Einstein scalar-field system. This solution, called the Milne spacetime, has been shown to be stable within the vacuum equations \cite{AM11} and various other matter sources \cite{Wang19,AndFaj20,FajWy21,FOW21,BaFaj20,BraFajKr19} (for related work in lower dimensions, see \cite{AMT97, M08, Faj17, Faj20, Mondal20}). As such, our contribution to the study of future stability of Milne spacetimes is dealing with the massless scalar field matter via corrected energy estimates inspired by \cite{CBM01} for vacuum Einstein equations with $U(1)$-symmetry. The additional spectral assumption is to ensure coercivity of the corrected scalar field energy -- numerical work, e.g. \cite{Cornish99}, does not seem to point toward this assumption being violated by any closed orientable 3-manifold with constant sectional curvature $\kappa=-\frac19$.

\subsection{Challenges in the proof}

The contracting and expanding regimes of spacetime are analyzed in two separate and methodologically independent parts. Before providing an overview of both arguments, we summarize which challenges that arise:

\subsubsection{Big Bang stability}

The main difficulties in establishing Big Bang stability are three-fold:\\

Firstly, we have to expect that the solutions are asymptotically velocity term dominated (as argued in Remark \ref{rem:AVTD}, we end up proving that this is the case). When comparing with the solutions of the velocity term dominated equations, one sees that the metric must at least be expected to diverge like
\[a^2{M}^\prime\odot\exp\left(-2\hat{{K}}^\prime\int_t^{t_0}a(s)^{-3}\,ds\right),\,\]
toward the Big Bang, where $\tilde{M}$ and $\hat{{K}}^\prime$ are reference tensors on $M$ close to $\gamma$ and $0$ respectively, the latter is a tracefree $(1,1)$-tensor and $\odot$ and $\exp$ are meant as matrix products and exponentials (see also \eqref{eq:asymp-G} in the main stability result). \\
Hence, after performing contraction adapted rescaling by $a^{-2}$, the rescaled metric $G$ must be expected to diverge at least slightly, since the integral over $a^{-3}$ behaves like $\log(t)$ toward the Big Bang and $\hat{K}$ has positive and negative eigenvalues if it does not happen to vanish. However, to be able to use the structure of the evolution equations to cancel terms in our energy arguments, we have to work with adapted quantities: For example, we need to use integration by parts with respect to $(\Sigma_t,G_t)$ to cancel high order scalar field terms with help of the (rescaled) wave equation that contains $\Lap_G$, or to obtain elliptic estimates from the lapse equation via the operator $\Lap_G$ or from the adapted div-curl-system for $\Sigma$ arising from the constraint equations. \\
As a result, even the rescaled solution variables cannot be assumed to be small toward the singularity, and we need to track and decrease their rate of divergence within the bootstrap argument. This significantly complicates dealing with nonlinear terms where the bootstrap assumptions often cannot be inserted naively. This in turn makes coercivity of the energies more involved to establish. In particular, this is also the reason we cannot use CMCSH gauge toward the Big Bang (see Remark \ref{rem:why-not-CMCSH}).\\

Secondly, and in contrast to \cite{Rodnianski2018,Speck2018}, replacing the wave structure of the geometric evolution in the Einstein equations with our less geometry dependent Bel-Robinson framework necessitates using the div-curl-system in $\Sigma$ (see \eqref{eq:comeq-mom-div} and \eqref{eq:comeq-mom-curl}) to close the energy argument, since the evolution equations for Bel-Robinson variables incur first order derivatives of matter. However, in regaining this derivative, the quantitative control on the energies is significantly weaker. Top order energies need to be scaled by $a^4$ to account for this, and hence one needs to derive different estimates below and at top order that account for these differences in behaviour.\\

Finally, given \eqref{eq:intro-ref3}, the rescaled time derivative of the scalar field is not small and does not become so toward the Big Bang. This leads to various terms within the core linearized evolutionary system of both matter and geometry that, if estimated naively, could lead to exponential blow-up toward the singularity. Dealing with these terms requires delicate cancellation mechanisms that use the explicit form of the Friedman equations, one of which we highlight in Lemma \ref{lem:en-error-cancellation} and its proof.

\subsubsection{Future and global stability}

For Milne stability, the canonical Sobolev energies for the scalar field variables do not obey good energy estimates. This can be overcome by adding an indefinite correction term to the canonical energy as done in \cite{CBM01} in a similar context, c.f.\,Definition \ref{def:fut-stab}. Moreover, the transition between the contracting and expanding regimes of the development of near-FLRW data requires a gauge switch and to evolve from the initial data hypersurface to a distant enough future hypersurface while still remaining near-FLRW so that this new data then becomes near-Milne. This is discussed in detail in Section \ref{sec:full-stab}

%\subsubsection{Singularity formation and Strong Cosmic Censorship}
%\subsubsection{Other results on Big Bang formation in the Einstein equations}
%\subsubsection{Other stability results for FLRW spacetimes}
%\todo{[symmetry assumptions, Fuchsian methods, linear wave equations (us, Ringström, Oliynyk,...)]}

%\subsection{Main result - big bang formation}\label{subsec:intro-main}

%\begin{theorem}[Main result - short version]\label{thm:main-short}
%Let $(M,\mathring{g},\mathring{k},\mathring{\phi},\mathring{\psi})$ be geometric initial data satisfying the Gauss-Codazzi-constraint equations, where $M$ is closed, orientable three-dimensional manifold that admits a metric $\gamma$ of constant negative sectional curvature. Further, when embedding it on the CMC hypersuface $\Sigma_{t_0}$ on $\M=(0,\infty)\times M$, assume it is  $\epsilon^2$-close in $H_\gamma^{N}$ and $C^{N-4}_\gamma$ to FLRW initial data $(a(t_0)^2\gamma,-\frac{\dot{a}(t_0)}{a(t_0)}\gamma,0,C\cdot a(t_0)^{-3})$ for sufficiently large $N\in\N$. \\
%
%Then, this initial data admits a solution $(\g,\phi)$ to the Einstein scalar-field system \eqref{eq:ESF1}-\eqref{eq:ESF2} foliated by CMC hypersurfaces $\Sigma_s=t^{-1}(\{s\})$ that remains close to the FLRW solution. For $\g=-n^2dt^2+g$, second fundamental form $k$ and spatial volume form $\vol{g}$, the following estimates hold:
%\begin{align*}
%\|n-1\|_{C^0_\gamma(\Sigma_t)}\lesssim&\,\epsilon a(t)^{4-c\epsilon^\frac18}\\
%\|a^{-3}\vol{g}-\vol{Bang}\|_{C^0_{\gamma}(\Sigma_t)}\lesssim&\,\epsilon a(t)^{4-c\epsilon^\frac18}\\
%\|a^3\del_t\phi-(\Psi_{Bang}+C)\|_{C^0_{\gamma}(\Sigma_t)}\lesssim&\,\epsilon a(t)^{4-c\epsilon^\frac18}\\
%\left\|\phi-\int_t^{t_0}a(s)^{-3}\,ds\cdot(\Psi_{Bang}+C)\right\|_{C^0_\gamma(\Sigma_t)}\lesssim&\,\epsilon a(t)^{4-c\epsilon^\frac18}\\
%\|a^3k-K_{Bang}\|_{C^0_\gamma(\Sigma_t)}\lesssim&\,\epsilon a(t)^{4-c\epsilon^\frac18}\\
%\left\|g_{(\cdot)l}\exp\left[(2\int_t^{t_0}a(s)^{-3}\,ds\cdot K_{Bang})\right]^l_{(\cdot)}-M_{Bang}\right\|_{C^0_\gamma(\Sigma_t)}\lesssim&\, \epsilon a(t)^{4-c\epsilon^\frac18}
%\end{align*}
%Here, $\exp$ is meant as a matrix exponential, $\vol{Bang},\,\Psi_{Bang},\,K_{Bang}$ and $M_{Bang}$ are all footprint states on $M$ that are $K\epsilon$-close in $C^0_\gamma$ to the FLRW footprints $\vol{\gamma},\,0,\,-\sqrt{\frac{4\pi}3}C\delta$ and $\gamma$ respectively, $\Psi_{Bang}$ and $K_{Bang}$ satisfy algebraic identities that ensure consistency with the CMC condition and the Gauss-Codazzi-constraints.\\
%Finally, $(\M,\g)$ becomes geodesically incomplete at the Big Bang hypersurface $\Sigma_0$, the Kretschmann scalar of $\g$ blows up toward $t=0$ like $a^{-12}$ (i.e. $t^{-4}$), and its Weyl curvature invariant like $\epsilon a^{-12}$ (i.e. $\epsilon t^{-4}$).
%\end{theorem}

\subsection{Proof outline}\label{subsec:intro-pf-outline}
\subsubsection{Big Bang stability}\phantom{m}\\

\textbf{The big picture.} The key argument in our Big Bang stability proof is a hierarchized series of energy estimates that establishes the asymptotic behaviour of solution variables toward the singularity. We rely on a bootstrap argument that establishes that energies $\E^{(L)}$ (see Definition \ref{def:energies}) at most only diverge slightly, where $0\leq L\leq 20$ denotes the derivative order. To this end, we make a bootstrap assumption on the solution norms $\mathcal{H}$ and $\mathcal{C}$ (see Definition \ref{def:sol-norm}), which control the distance of the rescaled variables to their FLRW counterparts in Sobolev and supremum norms with respect to $G$ respectively, where $G$ is the rescaled \textit{adapted} spatial metric (see Definition \ref{def:rescaled}). We refer to Assumption \ref{ass:bootstrap} and Remark \ref{rem:bs-strategy} for the precise bootstrap assumptions and improvements. That this bootstrap argument implies Theorem \ref{thm:main-past} follows from very straightfoward computations along the same lines as in \cite[Theorem 15.1]{Rodnianski2014}.\\

We work with evolution-adapted quantities even though $G(t,x)$ degenerates toward the Big Bang singularity, and as such it is more convenient to have these adapted quantities controlled by the solution norms $\mathcal{H}$ and $\mathcal{C}$ directly. Once the improved energy estimates are shown, a (time-scaled) coercivity notion (see Lemma \ref{lem:Sobolev-norm-equivalence-improved} and the proof of Corollary \ref{cor:H-imp}) and Sobolev embeddings with respect to the reference metric $\gamma$ then ensure that these improved estimates translate to $\mathcal{H}$ and $\mathcal{C}$, which then closes the bootstrap. To actually achieve this improved energy behaviour, we derive elliptic energy estimates or integral-type estimates that, once suitably combined and scaled, yield the desired improvements by applying the Gronwall lemma. Additionally, note that we need smallness of (scaled) initial data at up to three orders higher than present in $\mathcal{H}$ to close the argument at top order, as well as some additional Sobolev regularity assumptions on the initial data to ensure time-differentiability of all energies (see Assumption \ref{ass:init}). \\

\textbf{Scale factor.} The precise structure of the Friedman equations \eqref{eq:Friedman}-\eqref{eq:Friedman2} is crucial not just in controlling time integral quantities up to the Big Bang hypersurface (see Lemma \ref{lem:scale-factor}), but also to ensure that certain terms in the evolution that would otherwise cause large divergences contribute with favourable sign (cf. the arguments in Lemma \ref{lem:en-est-SF} as well as Lemma \ref{lem:en-error-cancellation}). However, the sectional curvature entering the Friedman equations actually does not turn out to be of key importance for large parts of the Big Bang stability analysis given it is asymptotically dominated by the matter term in the Friedman equations toward the Big Bang singularity. This further indicates that our method might extend to different settings.\\

\textbf{Gauge choice, commutation method and Bel-Robinson variables.} We commute the resulting elliptic-hyperbolic Einstein system with the Laplace-Beltrami operator $\Lap_G$ with respect to the rescaled evolved spatial metric $G$ to obtain higher order energy control. Commuting with this operator has the advantage that it leaves many integration-by-parts identites intact that are needed to provide specific cancellations e.g. for the wave equation in the scalar field energy, still allows to extract Sobolev norm control via ellipticity arguments and does not rely on the reference geometry in any way.\\

We still, however, need to deal with the Ricci term in the evolution equation for the second fundamental form. The divergent behaviour of the metric toward the Big Bang serves as an obstruction to using CMCSH gauge (see Remark \ref{rem:why-not-CMCSH}), which would deal with said term by making the system in geometric variables wave-like. Instead, we employ CMC gauge with zero shift to avoid badly behaved shift terms.

To deal with this Ricci term in said gauge, we consider the Bel-Robinson variables $E$ and $B$ which are $\Sigma_t$-tangent symmetric tracefree $(0,2)$-tensors and contain all information of the spacetime Weyl tensor $W[\g]$ (see Subsection \ref{subsec:BR}). Suitably projecting the Gauss-Codazzi equations admits additional constraint equations in terms of $E$ and $B$ that allow to replace the Ricci tensor at the \enquote{cost} of introducing Bel-Robinson energies into the formalism. Further, $E$ and $B$ satisfy a Maxwell-type system that can be exploited to obtain energy estimates and, like the other evolution equations, is well adapted to commutation with $\Lap_G$ (see Lemma \ref{lem:EEqBR}).\\

\textbf{A priori low order $C_G$-control.}  By applying the bootstrap assumptions on $\mathcal{C}$ to the evolution equations, we can immediately deduce improved low order estimates in $C_G^{l}$ for $l\geq 10$ for the solution variables by inserting them into the respective evolution equations (see Lemma \ref{lem:AP}), as well as via the maximum principle for the lapse (see Lemma \ref{lem:lapse-maxmin}). These usually still diverge slightly, mostly due to the asymptotic behaviour of $G$. However, at order $0$, the renormalized time derivative of the wave, the rescaled tracefree part of the second fundamental form and the rescaled Bel-Robinson variable $\RE$ crucially are in fact $K\epsilon$-small in $C^0_G$ on the bootstrap interval (see Lemma \ref{lem:APzero}).\\

\textbf{Energy estimates and hierarchy.} The main part of the analysis is establishing various energy estimates.
\begin{itemize}
\item For the \underline{lapse} (see Section \ref{sec:lapse}), the relevant estimates are direct results of the elliptic lapse equations which give strong control at mid level derivative orders ($L\leq 16$) that are used to improve the bootstrap assumptions, and weaker control at higher orders ($L\leq 22$) that allow to replace high order lapse energies by other terms in lower regularity.
\item The core \underline{matter} energy estimate (see Lemma \ref{lem:en-est-SF}) relies on delicate cancellations when computing the time derivative of $\E^{(L)}(\phi,\cdot)$. While we derive this differently than the energy flux method in \cite{Speck2018}, the necessary cancellations to arrive at Lemma \ref{lem:en-est-SF} are similar. 
\item The (rescaled) tracefree component of the \underline{second fundamental form} $\Sigma$ (see Lemma \ref{lem:en-est-Sigma}) and the (rescaled) \underline{Bel-Robinson variables} $\RE$ and $\RB$ (see Lemma \ref{lem:en-est-BR}) need to be treated in unison to deal with the leading curvature term in the evolution of the former by inserting a constraint equation in which $\RE$ occurs as the leading term (see \eqref{eq:comeq-Ham-BR}). However, the matter terms within the evolution of $\RE$ and $\RB$ contain, firstly, terms where we again need very precise estimates to show that they do not contribute large $a^{-3}$-divergences, and secondly, matter terms that lose one order of derivatives. However, the momentum constraint equation \eqref{eq:comeq-mom-div} and its Bel-Robinson counterpart \eqref{eq:comeq-mom-curl} containing $\RB$ lead to a div-curl-system for $\Sigma$, allowing to regain an order of regularity, at the cost that this control is weaker in terms of scaling (see Lemma \ref{lem:en-est-Sigma-top}). 
\item As a result, the \underline{core Gronwall argument} performed in Proposition \ref{prop:en-bs-imp} and Proposition \ref{prop:en-bs-imp-top} combines energies for the matter variables, $\Sigma$ and the Bel-Robinson variables, as well as energies for $\Ric[G]$ and additional scalar field quantities to control borderline error terms. As many of the a priori $C_G$-norm estimates add small additional divergences, it is necessary to perform an induction over derivative orders within this mechanism to deal with lower order error terms. Since $\Lap_G$ is elliptic, it is sufficient to perform this for even orders of $L$ (and simpler to formulate). This is done up to top order by applying the bootstrap assumptions for $\E^{(L+1)}(\phi,\cdot)$ and $\E^{(L-1)}(\Ric,\cdot)$ to improve the estimate for order $L$. The argument is closed by applying the degenerate elliptic estimate for $\Sigma$ from Lemma \ref{lem:en-est-Sigma-top} and adapting the total energy accordingly. Due to the high order elliptic lapse estimates, this turns out to be sufficient to gain a weak control of the top order scalar field energy, which remedies the derivative loss in the Bel-Robinson energy and allows the bootstrap argument for the energies to close.
\item Note that \textit{the \underline{metric} itself does not enter the core energy mechanism}. In fact, trying to replace control of the Ricci tensor by control of $G$ is likely too imprecise in dealing with high order curvature errors. Instead, control on $G-\gamma$ (and, locally, on $\Gamma[G]-\Gamhat[\gamma]$) is a consequence of a simple integral energy inequality and the improvements achieved for $\Sigma$ and matter variables (see Lemma \ref{lem:norm-est-G} and Corollary \ref{cor:H-imp}). Since we cannot utilize any additional structure in dealing with the metric, we have to construct our argument carefully to allow for the metric control to be weaker than what one gets for the core variables, while still being sufficiently strong to constitute an improvement and allowing to switch between $H_G$ and $H_\gamma$ (resp. $C_G$ and $C_\gamma$) norms.
\end{itemize}
We also point to Remark \ref{rem:en-est-strat} for a more detailed sketch of how the integral inequalities for the core Gronwall argument are structured and how this leads to the bootstrap improvement for the energies.

\subsubsection{Future stability and connecting the regions}

We proceed under similar lines as \cite{AndFaj20, FajWy21} to prove that near-FLRW spacetimes in negative spatial geometry are future stable. In fact, what we prove at first in Section \ref{sec:fut} is future stability of near-Milne spacetimes under the Einstein scalar-field system. Once this is established, we argue in Section \ref{sec:full-stab} how early near-FLRW initial data evolves to data that is sufficiently close to Milne for large enough times, which is essentially a consequence of the scale factor and the (physical) mean curvature approaching that of Milne (up to a multiplicative constant).\\

In terms of dealing with geometric and elliptic estimates, like \cite{FajWy21}, we can essentially carry over the results of \cite{AndFaj20} by working in CMCSH gauge and verifying that the matter components are indeed only perturbative terms within the geometric evolution.

This leaves only the scalar field to be examined. Here, we introduce corrective terms to the energies (see Definition \ref{def:fut-stab}) which yield decay estimates for the corrected scalar field energy (see Lemmata \ref{lem:fut-en-est-ESF0} and \ref{lem:fut-en-est-ESF}). That these energies are coercive requires the aforementioned spectral bound on the Laplace-Beltrami operator (see Lemmata \ref{lem:fut-ESF-coercivity} and \ref{lem:fut-Sob-est}).


\begin{remark}[Why not use CMCSH gauge for Big Bang stability?]\label{rem:why-not-CMCSH}
Given the comparative simplicity of the arguments for future stability, one might consider also applying this gauge to Big Bang stability. In particular, this would also not rely on any choice of reference frame, but would keep the wave structure of the geometric evolution intact, unlike when using Bel-Robinson variables. However, the issue in this approach lies in the shift equation, which would take the following form for the rescaled shift vector $X=a^3\tilde{X}$:
\begin{align*}
\Lap_GX^l+\Ric[G]^l_mX^m=&-2(N+1)(G^{-1})^{im}(G^{-1})^{jn}\Sigma_{ij}\left(\Gamma_{mn}^l-\Gamhat_{mn}^l\right) \numberthis\label{eq:REEqShift}\\
&+2(G^{-1})^{im}\nabla_iX^n\left(\Gamma_{mn}^l-\Gamhat_{mn}^l\right)\\
&\,+\langle\text{error terms in lapse and matter}\rangle
\end{align*}
As a result, the first term has to be expected to diverge at the same rate as the metric, i.e. we expect even low order norms of $\tilde{X}$ to behave like $a^{-3-c\sqrt{\epsilon}}$ at best (up to small prefactors). However, computing the time derivative of an integral over $\lvert G-\gamma\rvert_G^2$ (or derivatives thereof) becomes the integral over the $(\del_t-\Lie_{\tilde{X}})$-derivative of this quantity, and hence we get explicit terms of the form $\Lie_{\tilde{X}}\gamma$ which always exist at highest order and diverge worse than $\frac1t$. In short, the fact that the metric cannot be expected to converge to a footprint state leads to leading order terms in the differential energy estimates to carry strongly divergent pre-factors in CMCSH gauge. This obstructs any bootstrap improvements.
%Notice that, even on the level of elliptic estimates, the norm of $\fX$ is quadratic in Sobolev norms of $\fk$ and $\fg-\gamma$. For future stability, this is beneficial since the bootstrap assumption implies that these quantities are small on the bootstrap interval, so this only contributes error terms when studying the geometric evolution equations. However, since we have to expect at least $G-\gamma$ to diverge, so do these geometric energies, and in particular in ways we cannot pre-improve. 
\end{remark}

\subsection{Paper outline}\label{subsec:intro-paper-outline}

Sections \ref{sec:prelim}-\ref{sec:main-thm} cover the proof of Big Bang stability: Section \ref{sec:prelim} introduces notation and provides the necessary information on the FLRW background solution as well as the equations relevant to the subsequent analysis. Section \ref{sec:norm-en-bs} then discusses the solution norms and energies and states the initial data and bootstrap assumptions. In Section \ref{sec:ap}, improved low order $C_G$-norm estimates that follow directly from the bootstrap assumptions are established, along with additional formulas and a priori estimates. Section \ref{sec:lapse} concerns the elliptic estimates for the lapse, while Section \ref{sec:en-est} discusses the energy and Sobolev norm estimates for all other variables, all but one of which are integral estimates. These are all combined in Section \ref{sec:bs-imp} to improve the bootstrap assumptions -- first for the energies, then for $\mathcal{H}$ and finally $\mathcal{C}$. Section \ref{sec:main-thm} shows how this bootstrap argument implies the main Big Bang stability result (see Theorem \ref{thm:main}, which is the formal version of Theorem \ref{thm:main-past}). \\
Section \ref{sec:fut} contains the proof of near-Milne future stability, and we demonstrate that this is sufficient for future stability of near-FLRW spacetimes in Section \ref{sec:full-stab}, proving Theorem \ref{thm:main-full}. The appendices (Sections \ref{sec:appendix}-\ref{sec:appendix-fut}) collect various basic formulas and commutator expressions as well as error terms and how these can be estimated. \\

\noindent \textbf{Acknowledgements.} D.F. acknowledges support by the Austrian Science Fund (FWF) through the project \enquote{Relativistic Fluids in cosmology} (project number P34313).  L.U. acknowledges the support by the START-Project \enquote{Isoperimetric study of initial data for the Einstein equations} (project number Y963-N35), also by FWF, as well as support through the DOC Fellowship of the Austrian Academy of Sciences. L.U. also thanks the \enquote{Studienstiftung des Deutschen Volkes} for their scholarship.