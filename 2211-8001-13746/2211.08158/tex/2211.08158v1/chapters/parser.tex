\section{GEC-Oriented Parser}
We first build a GEC-oriented parser (GOPar) to get the tailored constituent-based syntax of ungrammatical inputs.
\label{sec:par}

\textbf{Extended syntax representation scheme for constituency parsing.} As discussed in \citet{zhang2022syngec}, one stumbling block to parsing ungrammatical sentences is the inadaptability of the syntax representation scheme. More specifically, the original scheme may not represent the non-canonical structures arising from errors well. So, following \citet{zhang2022syngec}, we extend the syntax representation scheme for constituency parsing via several straightforward rules.

The basic idea is to insert pseudo non-terminal nodes\footnote{In constituency trees, terminal nodes are the words in sentences and non-terminal nodes are hierarchical constituents, such as ``VP'', ``NP'', ``SBAR'', etc.}  into the constituency tree to identify erroneous words. We design rules to handle \textit{substituted}, \textit{redundant}, and \textit{missing} errors, as discussed below.

\begin{enumerate}
    \item[(1)] \textbf{Substituted errors (SUB)} mean that erroneous words should be substituted with other words. For such errors, we insert a pseudo ``SUB'' non-terminal node as the new head of the erroneous word, as shown in Figure \ref{fig:s-error}.
    \item[(2)] \textbf{Redundant errors (RED)} mean that erroneous words should be deleted. We put a redundant word into the phrase to which its right-side word belongs, even if its right-side word is redundant also. If the redundant word is at the end of the sentence, we choose its left-side word instead. Then, we insert a pseudo ``RED'' non-terminal node as the new head of the redundant word, as shown in Figure \ref{fig:r-error}.
    \item[(3)] \textbf{Missing errors (MISS)} mean that some words should be inserted. For each missing word, we insert a pseudo ``MISS'' non-terminal node as the head of its right-side adjacent word, as shown in Figure \ref{fig:m-error}. We choose the left-side word if the missing word is at the end of the sentence. If multiple words are missed at the same position, we still only insert one ``MISS'' node.
\end{enumerate}

\textcolor{black}{\textbf{Training GOPar for constituency parsing.}} We take the idea from \citet{zhang2022syngec} to train a GOPar for constituency parsing using parallel GEC data as a pivot. Firstly, we parse the target-side correct sentences in GEC corpora by an off-the-shelf constituency parser. Secondly, we extract all errors in the source-side incorrect sentences through ERRANT\footnote{\url{https://github.com/chrisjbryant/errant}} \cite{bryant2017automatic}. Thirdly, we project the target-side constituency trees to the source-side ones by adjusting the structures of the erroneous part via the rules mentioned above and keeping the correct part unchanged. Finally, we utilize the projected trees to train GOPar.

\textbf{Discussion.} Compared with the dependency-based syntax scheme proposed by \citet{zhang2022syngec}, our scheme based on constituent-based syntax can represent grammatical errors more flexibly. Specifically, their scheme fails to handle the case that different errors occur at the same position, as multiple labels can not be assigned to one dependency arc. Instead, we can insert multiple non-terminal nodes for marking one word. Please kindly note that our scheme is lightweight and still has much room for improvement. 
