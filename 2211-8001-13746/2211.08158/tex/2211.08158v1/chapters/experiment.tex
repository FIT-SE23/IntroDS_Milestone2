\section{Experiments}
\subsection{Experimental Setup}
Following previous work \cite{zhang2022syngec}, we use the cleaned version of the Lang8 dataset (CLang8) \cite{rothe2021recipe}, the FCE dataset \cite{yannakoudakis2011new}, the NUCLE dataset \cite{dahlmeier2013building} and the WI+LOCNESS train-set \cite{bryant2019bea} for training. We use the  BEA-19-\textit{Dev} dataset \cite{bryant2019bea} for validating. We report average (P)recision, (R)ecall, and (F$_{0.5}$) metrics on the CoNLL-14-\textit{Test} \cite{ng2014conll} and BEA-19-\textit{Test} \cite{bryant2019bea} datasets with their official evaluation tools \cite{dahlmeier2012better,bryant2017automatic} with 3 random seeds.
We present more data statistics and implementation details in Appendix \ref{sec:app:a}.




\begin{table}[tp!]
\centering
\scalebox{0.6}{
\begin{tabular}{lccc}
\toprule
                                   &\textbf{Extra}     & \textbf{CoNLL-14-\textit{test}}     & \textbf{BEA-19-\textit{test}}     \\
                                 \textbf{Model}& \textbf{Data Size}  & \textbf{P/R/$\mbox{\textbf{F}}_{0.5}$}    & \textbf{P/R/$\mbox{\textbf{F}}_{0.5}$} \\ \hline
                                  \multicolumn{4}{c}{\textbf{w/o PLM}} \\ \hline
                                  \multicolumn{4}{l}{\textbf{w/o syntax}}     \\
                                   \citet{kiyono2019empirical} &   70M         & 67.9/44.1/61.3          & 65.5/59.4/64.2          \\
                                  \citet{lichtarge-etal-2020-data}&   340M        & 69.4/43.9/62.1          & 67.6/62.5/66.5          \\
                                  \citet{stahlberg2021synthetic}&   540M     & 72.8/49.5/\textbf{66.6}          & 72.1/64.4/\textbf{70.4}        \\
                                  \textbf{Our Baseline} &   2.4M &  66.9/40.3/59.1   & 66.8/55.5/64.2         \\
                                  \hdashline 
                                  \multicolumn{4}{l}{\textbf{w/ syntax}} \\ 
                                   \citet{wan2021syntax} &   10M       & 74.4/39.5/63.2          & 74.5/48.6/67.3          \\
                                 \citet{li2022syntax} &   30M      & 66.7/38.3/58.1          & -/-/-          \\
                                  \textbf{DSynGEC} \cite{zhang2022syngec} &   2.4M  & 70.0/46.2/\textbf{63.5}          & 70.9/59.9/\textbf{68.4}          \\
                                    \textbf{CSynGEC} (this work) &   2.4M  & 69.7/46.3/63.3          & 69.4/60.3/67.4 
                                  \\\hline \hline
                                     \multicolumn{4}{c}{\textbf{w/ PLM}} \\ \hline 
                                     \citet{DBLP:conf/acl/SunGWW20}&   300M          & 71.0/52.8/66.4       & 74.7/66.4/72.9          \\
                                      \citet{rothe2021recipe}$^{*}$ &   2.4M         & -/-/\textbf{68.8}          & -/-/\textbf{75.9}          \\  \textbf{Our Baseline} &   2.4M &  73.6/48.6/66.7   & 74.0/64.9/72.0         \\
                                       \textbf{DSynGEC} \cite{zhang2022syngec} &   2.4M  & 74.7/49.0/67.6          & 75.1/65.5/72.9          \\
                                       \textbf{CSynGEC} (this work) &   2.4M  & 74.0/50.7/67.7         & 74.4/66.1/72.6          \\
                                      
                                
\bottomrule
\end{tabular}
}
\caption{\textbf{Single-model} results. ``\textbf{w/ syntax}'' means using syntactic knowledge. ``\textbf{w/ PLM}'' means using pre-trained language models. $^{*}$ denotes current SOTA.}

\label{tab:main:results}
\end{table}

\subsection{Experimental Results}
\textbf{Main results} are shown in Table \ref{tab:main:results}.
\textcolor{black}{When not using pre-trained language models (w/o PLM), CSynGEC gets 63.3 and 67.4 F$_{0.5}$ scores on CoNLL-14-\textit{Test} and BEA-19-\textit{Test}, respectively, which is quite competitive. Compared with the baseline, incorporating tailored constituent-based syntax leads to significant F$0.5$ gains (+4.2/+3.2) on both benchmarks, which clearly shows its effectiveness. We also build a stronger baseline by initializing the parameters of the Transformer backbone from BART \cite{lewis2020bart} (w/ PLM). CSynGEC still achieves +1.0/+0.6 F$_{0.5}$ improvements over BART, which indicates that the effectiveness of our method will not easily be overwhelmed by PLMs.}

To distinguish, we rename the method in \citet{zhang2022syngec} as DSynGEC (Dependency SynGEC). We observe that CSynGEC outperforms all other syntax-enhanced counterparts, except DSynGEC. 
\textcolor{black}{Specifically, we find that DSynGEC always performs better in precision, while CSynGEC always achieves higher recall. Such a discrepancy further motivates us to combine them.}

The SOTA results are kept by \citet{rothe2021recipe}, which leverages a huge PLM with up to 11B parameters\footnote{While CSynGEC only has 70M parameters.}. Since CSynGEC is model-agnostic, we will test it in more SOTA baselines in the future.

\begin{table}[tp!]
\centering
\scalebox{0.65}{
\begin{tabular}{lcc}
\toprule
                                     & \textbf{CoNLL-14-\textit{test}}     & \textbf{BEA-19-\textit{test}}     \\
                                 \textbf{Model}  & \textbf{P/R/$\mbox{\textbf{F}}_{0.5}$}    & \textbf{P/R/$\mbox{\textbf{F}}_{0.5}$} \\ \hline
                                 \textbf{Baseline}        & 66.7/40.3/59.1        & 66.8/55.5/64.2         \\ \hdashline
                                 \textbf{DSynGEC \cite{zhang2022syngec}}        &\textbf{70.0}/46.2/\textbf{63.5}         & \textbf{70.9}/59.9/\textbf{68.4}          \\ 
                                   \textbf{CSynGEC}        & 69.9\textbf{/46.3/}63.3       & 69.4\textbf{/60.3/}67.4        \\ 
                                   \hspace{0.3cm} \textbf{w/ off-the-shelf parser}        & 66.6/41.4/59.4          & 67.0/56.4/64.6          \\
                                   \hspace{0.3cm} \textbf{w/o extended scheme}        & 68.4/43.2/61.3          & 67.6/56.7/65.1          \\ \hdashline
                                   \multicolumn{3}{c}{\textit{\textbf{intra-model combination}}} \\ 
                                         \textbf{CSynGEC + DSynGEC}        & \textbf{69.6/47.7/63.8}          & \textbf{70.4/61.7/68.5}          \\
                                   \hspace{0.3cm} \textbf{w/o independent cross-attention}        & 67.6/47.5/62.3          & 68.5/61.6/67.0          \\\hdashline
                                   \multicolumn{3}{c}{\textit{\textbf{inter-model combination}}} \\ 
                                         \textbf{6 $\times$ DSynGEC}        & 77.0/39.4/64.7          & 81.2/54.7/74.1          \\
                                             \textbf{6 $\times$ CSynGEC}        & 77.7/\textbf{40.7}/65.7         & 81.8/\textbf{55.0}/74.5          \\
                                             \textbf{3 $\times$ DSynGEC + 3 $\times$ CSynGEC}        & \textbf{79.0}/40.1/\textbf{66.2}          & \textbf{83.0}/54.6/\textbf{75.2 }         \\
                                
\bottomrule
\end{tabular}
}
\caption{Results of model ablation and ensemble. }

\label{tab:abb}
\end{table}

\textbf{Benefits of GOPar.} We first study the effectiveness of using syntax derived from GOPar. We replace GOPar with an off-the-shelf constituency parser trained on the PTB treebank \cite{marcinkiewicz1994building}. As shown in Table \ref{tab:abb}, the performance of CSynGEC dramatically drops after changing the parser (-3.9/-2.8), which confirms the necessity of GOPar. We conjecture that off-the-shelf parsers tend to provide low-quality parses for ungrammatical sentences, which may introduce noise for GEC.

\textbf{Impact of extended syntax scheme.} We extend the constituent-based syntax scheme by inserting pseudo non-terminal nodes to represent grammatical errors. To study the impact of this task-oriented adaptation, we remove all inserted pseudo nodes before feeding the trees produced by GOPar into GEC models. From Table \ref{tab:abb}, we can see that CSynGEC heavily degenerates without the extended scheme (-2.0/-2.3). However, the performance is still better than using an off-the-shelf parser, as the high-quality parses for the correct part are kept.


\textcolor{black}{\textbf{Results of syntax combination.} From Table \ref{tab:abb}, we can see that the intra-model combination significantly improves recall over DSynGEC (+1.5/+1.8) and CSynGEC (+1.4/+1.4), and achieves better overall performance on both test sets. Besides, we note that this method will not work if we use a sharing cross-attention layer to attend to different kinds of syntactic information.}

\textcolor{black}{Under the inter-model combination setting, ``3 $\times$ DSynGEC + 3 $\times$ CSynGEC'' substantially improves precision over ``6 $\times$ DSynGEC'' (+2.0/+1.8) and ``6 $\times$ CSynGEC'' (+1.3/+1.2), and achieves the best F$_{0.5}$ scores. All these results demonstrate that constituent-based and dependency-based syntax has intrinsic complementary strength for helping GEC models. We also give a case in Appendix \ref{sec:case} to show how both kinds of syntax work.}



