\section{Syntax Combination}
\label{sec:3.4}
\textcolor{black}{We present two methods to combine constituent-based syntax from this work and dependency-based syntax from \citet{zhang2022syngec} to help GEC.}

\textcolor{black}{\textbf{Intra-model combination.} This method integrates two syntax formalisms within a single Transformer. We first build two separate GCNs to encode two kinds of syntactic information. Then, we leverage two multi-head cross-attention layers to attend to them in the decoder. The results are added together for subsequent calculation. The whole procedure is depicted in Figure \ref{fig:combine}.}

\textcolor{black}{\textbf{Inter-model combination.} This method ensembles independent models enhanced with different kinds of syntax. We directly adopt the ensemble technique proposed by \citet{qorib-etal-2022-frustratingly}. We train multiple models based on either one kind of syntax, and gather all edits predicted by those models. Then, we use logistic regression to predict whether each edit should be retrained or discarded, and re-apply the preserved edits to get the final results.}



