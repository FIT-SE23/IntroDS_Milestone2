% This must be in the first 5 lines to tell arXiv to use pdfLaTeX, which is strongly recommended.
\pdfoutput=1
% In particular, the hyperref package requires pdfLaTeX in order to break URLs across lines.

\documentclass[11pt]{article}

% Remove the "review" option to generate the final version.
\usepackage{acl}

% Standard package includes
\usepackage{times}
\usepackage{latexsym}
\usepackage{graphicx}
\usepackage{todonotes}
\usepackage{amsmath,amssymb}
\usepackage{url}
\usepackage{color,soul}
\usepackage{makecell} 
\usepackage{multirow}
\usepackage{tikz}
\usepackage{enumitem}
\usepackage{caption, subcaption}

\def\checkmark{\tikz\fill[scale=0.4](0,.35) -- (.25,0) -- (1,.7) -- (.25,.15) -- cycle;}
\renewcommand{\UrlFont}{\ttfamily\small}
\DeclareMathOperator{\E}{\mathbb{E}}
% This is not strictly necessary, and may be commented out,
% but it will improve the layout of the manuscript,
% and will typically save some space.

% For proper rendering and hyphenation of words containing Latin characters (including in bib files)
\usepackage[T1]{fontenc}
% For Vietnamese characters
% \usepackage[T5]{fontenc}
% See https://www.latex-project.org/help/documentation/encguide.pdf for other character sets

% This assumes your files are encoded as UTF8
\usepackage[utf8]{inputenc}

% This is not strictly necessary, and may be commented out,
% but it will improve the layout of the manuscript,
% and will typically save some space.
\usepackage{microtype}

% If the title and author information does not fit in the area allocated, uncomment the following
%
%\setlength\titlebox{<dim>}
%
% and set <dim> to something 5cm or larger.

\title{Multilingual and Multimodal Topic Modelling with Pretrained Embeddings}

% Author information can be set in various styles:
% For several authors from the same institution:
% \author{Author 1 \and ... \and Author n \\
%         Address line \\ ... \\ Address line}
% if the names do not fit well on one line use
%         Author 1 \\ {\bf Author 2} \\ ... \\ {\bf Author n} \\
% For authors from different institutions:
% \author{Author 1 \\ Address line \\  ... \\ Address line
%         \And  ... \And
%         Author n \\ Address line \\ ... \\ Address line}
% To start a seperate ``row'' of authors use \AND, as in
% \author{Author 1 \\ Address line \\  ... \\ Address line
%         \AND
%         Author 2 \\ Address line \\ ... \\ Address line \And
%         Author 3 \\ Address line \\ ... \\ Address line}

%\author{First Author \\
%  Affiliation / Address line 1 \\
%  Affiliation / Address line 2 \\
%  Affiliation / Address line 3 \\
%  \texttt{email@domain} \\\And
%  Second Author \\
%  Affiliation / Address line 1 \\
%  Affiliation / Address line 2 \\
%  Affiliation / Address line 3 \\
%  \texttt{email@domain} \\}

\author{
  Elaine Zosa \and Lidia Pivovarova \\
  Department of Computer Science\\
  University of Helsinki\\
  Helsinki, Finland\\
  \texttt{firstname.lastname@helsinki.fi} \\ 
}

\begin{document}
\maketitle
\begin{abstract}

This paper presents M3L-Contrast---a novel multimodal multilingual (M3L) neural topic model for comparable data that maps texts from multiple languages and images into a shared topic space. Our model is trained jointly on texts and images and takes advantage of pretrained document and image embeddings to abstract the complexities between different languages and modalities. As a multilingual topic model, it produces aligned language-specific topics and as multimodal model, it infers textual representations of semantic concepts in images. We demonstrate that our model is competitive with a zero-shot topic model in predicting topic distributions for comparable multilingual data and significantly outperforms a zero-shot model in predicting topic distributions for comparable texts and images. We also show that our model performs almost as well on unaligned embeddings as it does on aligned embeddings.

\end{abstract}
\section{Introduction}
\section{Introduction}

Generative modeling has been the dominant approach for large-scale pretraining and zero-shot generalization~\cite{gpt3-paper,artetxe2021efficient,rae2021scaling}. 
Combined with prompts~\cite{gpt3-paper}, most of the natural language processing (NLP) tasks can be formulated into the fill-in-the-blank format and perform generative language modeling.
Based on the unified generative formulation, pretrained models such as GPT-3~\cite{gpt3-paper}, BERT~\cite{devlin2018bert,PET-paper}, T5~\cite{T5-paper}, can perform zero-shot inference on new tasks. 


More recent work~\cite{T0-paper} proposed to further pretrain a generative T5~\cite{T5-paper} with multitask prompted datasets and has substantially enhanced the performance of zero-shot generalization. 
In contrast, methods based on discriminative modeling~\cite{devlin2018bert} have not been able to achieve state-of-the-art performance on zero-shot learning. The adoption of discriminative approaches for zero-shot learning has been limited in the literature.


% Although there are a few works using discriminative modeling to perform zero-shot or few-shot learning, such as CLS finetuning using BERT or prompting using ELECTRA
% For example, BERT was CLS finetuned to perform zero-shot/few-shot learning, however, the zero-shot/few-shot performance are lagged far behind.

% \zy{Add a note: although BERT can be CLS finetuned (which is discriminative), but it is not the SOTA approach for zero-shot and few-shot learning.}

\begin{figure}%[htbp]
     \centering
     \includegraphics[width=1.05\linewidth]{figure/final_sota.png}
     \vspace{-15pt}
     \caption{Average zero-shot performance over 11 zero-shot tasks for our Universal Discriminator and T0~\cite{T0-paper}. Our universal discriminator significantly outperforms T0 across three different scales.}
     \label{fig:sota}
     \vspace{-15pt}
 \end{figure} 


In this work, we challenge the convention of zero-shot learning and propose to study and improve discriminative approaches. This is motivated by the fact that many NLP tasks can be framed as selecting from a few options; e.g., telling whether sentence A entails sentence B, or predicting which answer is correct for a given question. We call these tasks \textit{discriminative tasks}. As we will discuss in later sections, a significant portion of NLP tasks is in fact discriminative tasks. We hypothesize that discriminative approaches perform better for discriminative tasks.
% Despite the recent progress, it remains unknown how discriminative approaches perform in zero-shot generalization. Motivated by the fact that discriminative modeling learns to distinguish among options and goes better with discriminative tasks (e.g., telling whether sentence A entails sentence B, or telling which option correctly answer the question), we hypothesize that discriminative modeling would be better at zero-shot generalization, especially on discriminative tasks.

To verify the hypothesis, we propose the \textbf{universal discriminator (UD)}, which substantially improves zero-shot generalization over the previous generative state-of-the-art (SOTA)~\cite{T0-paper}, as Figure~\ref{fig:sota} shows.
The main idea is to train a single discriminator to predict whether a text sample comes from the true data distribution of natural language, similar to GANs \cite{goodfellow2014generative}. Given a set of training tasks with labeled data, we construct a dataset with positive and negative examples, where positive ones are in-distribution natural language samples and negative ones are out-of-distribution. There are two major types of discriminative tasks. The first type is tasks with multiple options, such as multi-choice question answering and news classification. We fill the options into the sentences and the ones with correct options are considered positive samples. The second type is tasks with yes/no options, which can be formulated as a binary discrimination problem itself. For example, natural language inference aims to predict whether a premise entails a hypothesis. In this case, we use a prompt to concatenate the premise $A$ and the hypothesis $B$ into a sentence ``Premise: $A$. Hypothesis: $B$.'' If entailment holds, this sample is treated as positive in-distribution samples and otherwise negative out-of-distribution ones.



% We define the true data distribution using multiple training tasks with labeled data. Specifically, since discriminative tasks can be formulated as selecting from a few options, samples with correct options form an empirical data distribution, while samples with incorrect options are considered out of distribution. In other words, our discriminator is trained to predict ``true'' for samples with correct options and ``false'' for incorrect ones. We use simple concatenation to minimize prompting efforts. For example, given an example (premise, hypothesis), a natural language inference task predicts whether the premise entails the hypothesis. We concatenate the premise and hypothesis, and assign the label ``true'' for entailment and ``false'' for non-entailment.


% First off, since many of the NLP tasks can be formulated as selecting from several options, we first reformulate the task data into natural text samples by concatenating different fields \zy{what are fields? undefined here. try using another word.}.
% For example, given an example of \zy{the} natural language inference task (\textit{Premise}, \textit{Hypothesis}, \textit{Label}), the natural text is reformulated as ``\textit{\{Premise\} || \{Hypothesis\}}'' labeled with \textit{\{Label\}}. \footnote{Here we use ``||'' to represents direct concatenation.} 
% Another example of topic classification task (\textit{Text}, \textit{Label}) where the \textit{Label} indicates the first option of \{Sports, Fashion, Politics\}, the corresponding natural texts are formulated as ``\textit{Text} || Sports'' labeled with 1, ``\textit{Text} || Fashion'' and ``\textit{Text} || Politics'' both labeled with 0.
% Secondly, we pretrain a pretrained model with reformulated multitask datasets to distinguish whether the text sample comes from the true data distribution. ~\footnote{An assumption is that negative-labeled text samples are artificially constructed thus do not come from the true data distribution, and vice versa.}

For the performance of zero-shot generalization, our approach achieves new state-of-the-art on the T0 benchmark, outperforming T0 by 16.0\%, 7.8\%, and 11.5\% respectively on different scales. 
UD also achieves state-of-the-art performance on a wide range of supervised NLP tasks, using only 1/4 parameters of previous methods.
Compared with the previous generative prompt-based methods, our universal discriminator requires minimal prompting, which is simple, robust, and applicable in real-world scenarios.

% By further scaling the number of tasks, our approach also sets the new state-of-the-art on \textbf{\color{red}[xxx]} tasks with less than 10\% of model parameters \zy{need to give a range} under the setting of standard finetuning.
% In the setting of finetuning, our approach also outperforms the generative baselines consistently across a wide range of tasks.


In addition, we also generalize UD to a larger scope of tasks, such that UD can perform discriminative and generative tasks at the same time. Specifically, we extend UD to the encoder-decoder architecture for training on generative tasks, and restrict the model's prediction on "yes"/"no" tokens for jointly training discriminative tasks. Results prove that generalized UD maintains UD's advantages on discriminative tasks and achieves comparable results on generative tasks (See \S~\ref{sec:generalizedud}). 
% We leave expanding UD to a broader range of generative tasks and achieve greater performance on generative tasks as our future work


% \xhk{I admit the limitation on generative tasks here as our future work.}

%\xhk{Although UD is designed for improving zero-shot performance for discriminative tasks, we can also combine this idea to train a generalized UD model which simultaneously solves both discriminative tasks and generative tasks, maintaining UD's advantage on discriminative tasks and get comparable results on generative tasks (See \S~\ref{sec:generalizedud}).}

% The universal discriminator provides a new perspective for zero-shot generalization---Compared with generating the true verbalizer that indicates task label with extensive prompt engineering, distinguishing between options with minimal prompting efforts is simple, robust, and high-performing, thus is more applicable in real-world scenarios. \zy{rewirte the above sentence, just focus on one point---minimal prompting}


\section{Related Work}
\section{Related Work}

\subsection{Zero-Shot Generalization Using PLMs}
Pretrained language models (PLM) can transfer knowledge from training data to downstream tasks.
Prompting methods further narrow the gap between training data and downstream tasks. \citet{PET-paper} reformulate NLP tasks into cloze filling using prompts so that PLMs can conduct zero-shot inference by generating tokens given prompted inputs. \citet{meng2022generating} use PLMs to generate class-conditioned texts with the guidance of prompts without seeing any task-specific data.
Most recently, researchers have introduced natural language prompts to unify various kinds of tasks and propose a multi-task prompted training framework to achieve great zero-shot performance even faced with unseen downstream tasks (\citet{FLAN,T0-paper}).
However, zero-shot learning has been dominated by generative approaches.
% The success of this new paradigm motivates us to follow and iterate this line of research by proposing a universal discriminator to unify and train all tasks.


\subsection{Prompt-based and Prompt-free Methods in NLP}
Prompting is the method of reformatting NLP tasks using natural language templates to adapt to downstream tasks \cite{T5-paper,PET-paper}.
% With the help of prompts, LMs can directly predict the desired answer without training on task-specific data, making it possible to perform well under few-shot, even zero-shot setting. Also, prompts serve as a decent lubricant for multitask learning which facilitates the development of text-to-text pretrained models such as T5 (\cite{T5-paper}). 
To reduce the instability and labor costs brought by prompting, researchers have tried various approaches (\citet{ptuning-paper,he2021towards}) to learn continuous prompts. 

Recently, prompt-free methods are also being explored. \citet{mahabadi2022prompt} adopts task-specific adapters to learn task descriptions implicitly for few-shot learning with PLMs. 
It has also been indicated that using null prompts without task-specific templates can achieve decent performance compared with manually-designed prompts on various tasks (\citet{logan2021cutting}).

Our work further shows that minimal prompting performs better with our discriminative formulation in the multi-task zero-shot learning setting.

% liberate the time-cosuming manual prompt design lying in multi-task prompted pretraining (\citet{T0-paper,FLAN}) by adopting a universal task formulation and utilizing a universal discriminator to achieve better zero-shot generalization.

\subsection{Discriminative Models in NLP}

% Pretrained language models often adopt masked language modeling (MLM) as the self-supervised learning objective, e.g., BERT, RoBERTA (\citet{devlin2018bert,liu2019roberta}), where the input texts are corrupted by masked tokens and the models are trained to recover the original tokens. They performed well on a wide range of downstream tasks, yet require a large number of training samples and parameters to achieve decent generalizibility.

PLMs trained with masked language modeling (MLM) \cite{devlin2018bert,liu2019roberta} can be finetuned in a discriminative manner for downstream tasks. 
ELECTRA \cite{clark2020electra} trains a discriminator to detect whether a token has been replaced. WKLM \cite{xiong2019pretrained} employs an entity-centric approach for pretraining and predicts whether an entity has been replaced.
However, finetuning for these methods is usually based on one separate CLS head per task, which is not suitable for zero-shot generalization.

% To build a more sample-efficient pretraining framework, ELECTRA (\citet{clark2020electra}) changes the MLM objective to a replaced token detection (RTD) objective, where a generator replace several tokens in the training texts and a discriminator will predict whether each token is replaced or not. WKLM (\citet{xiong2019pretrained}) develops a entity-centric pretraining objective which detects and replaces the entity words with the entities of the same type in each document and train the model to predict whether each entity is replaced.

Recently, prompting has been combined with token-level discriminators based on ELECTRA for few-shot learning \cite{yao2022prompt,xia2022prompting}. While these are also discriminative approaches, there are a few key differences from our approach. First, these methods use ELECTRA to perform token-level discrimination, while we perform sequence-level discrimination, which is more flexible to tasks that have multi-word verbalizers or do not have verbalizers at all. Second, we unify all discriminative tasks into one single task with minimal prompting, which is convenient for zero-shot generalization and removes the need for designing task-specific prompts. Third, these methods are specific to ELECTRA-like pretraining, while our approach accepts arbitrary pretrained encoders. In our experiments, we will also make a direct comparison with these approaches to demonstrate our effectiveness.

% \xhk{Another concurrent work \cite{UniMC} shares a similar idea with our discriminative approach. In specific, given an example of a multiple-choice task, their method concatenates the tokens of input passage, question, and all the options together as their model's input text. Then they use their proposed option masked language modeling (O-MLM) and option prediction (OP) in the training phase. \zy{too many details} Although their method also entails a similar idea of the "discriminative approach", our UD method implements this idea in a completely different way: We are training UD to predict whether a text sample combined with every single option comes from the true data distribution of natural language. Our training method is more concise, more efficient in terms of the token length, and more beneficial for other potential settings \zy{not clear based on context} (see fine-tune Section~\ref{sec:ud_finetune}).} 

% \zj{Another concurrent work \cite{UniMC} shares a similar idea with our discriminative approach. In specific, given an example of a multiple-choice task, they concatenate all inputs and options together and then use option masked language modeling (O-MLM) and option prediction (OP) for training. Although their method also entails a similar idea of the ``discriminative approach'', our UD method implements this idea more concisely and efficiently. We are training UD to predict whether a text sample combined with every single option comes from the true data distribution of natural language. (See fine-tune Section~\ref{sec:ud_finetune} for details)} 

% \xhk{[haike: not sure if it is better to just stop here, or mention empirical performance comparison]}

% \xhk{It is hard to fairly compare our UD's empirical performance with UniMC because UniMC selects a set of multiple-choice training tasks to induce better zero-shot performance, whereas we restrict our UD using exactly the same training datasets in T0. Furthermore, UniMC is not tested on some tasks in the T0 benchmark and it is only implemented using small backbone models.}

% Recently, with the development and popularity of prompt-based approaches, researchers further adapt prompt-based few-shot learning to ELECTRA and achieves better performance on downstream tasks (\citet{yao2022prompt,xia2022prompting}).

% In our work, we borrow the similar idea from the discriminative pretrained language models and adapt to multi-task prompted training paradigm. The major difference is that discriminative models such as ELECTRA focuses on the pretraining stage, while our model puts emphasis on the multi-task finetuning stage and demonstrate that unifying all tasks (e.g., T0) into \xhk{discriminative tasks} classification problem can greatly boost the zero-shot task generalization while reducing the prompt-design efforts and the number of training samples.


%% to be discussed
% \subsection{Generative Adversarial Nets}



\section{Multilingual and Multimodal Model}
\subsection{Neural multilingual topic model}
\label{sec:models-multilingual}

\begin{figure}[t!]
    \centering
    \includegraphics[width=0.8\columnwidth]{figures/m3l-contrast-new2.png}
    \caption{Proposed M3L-Contrast topic model. (a) Multilingual topic model with language-specific encoders and inference networks; (b) Extension to the multimodal setting. The loss function is detailed in Equation~\ref{eq:loss-m3l-contrast}.}
    \label{fig:archi-m3l-contrast}
\end{figure}

We first propose a neural multilingual topic model for comparable multilingual data that uses pretrained document embeddings. Our multilingual model is based on ZeroshotTM~\cite{bianchi-etal-2021-cross}, a zero-shot cross-lingual topic model. However, we are not aiming for a zero-shot model. Instead, our model infers aligned \textit{language-specific topics} for each language present in the dataset. Moreover, our approach does not require the pretrained document embeddings to be aligned beforehand. This property makes it advantageous in settings where a multilingual encoder that includes our desired language might not exist such as in low-resource settings.

Figure~\ref{fig:archi-m3l-contrast}(a) shows the multilingual model architecture. The model uses independent inference networks for each language. To align language-specific topics, the model minimizes the Kullback-Leibler (KL) divergence between the topic distributions of comparable documents from different languages and, in addition, uses a contrastive loss to map similar instances close to each other in the topic space and keep non-related instances apart.  

For each tuple of aligned documents in the comparable multilingual dataset, we encode the documents from each language using their own separate encoders (whether aligned or non-aligned) and then the embeddings are passed to language-specific inference networks that infers the mean, $\mu$, and variance, $\sigma^2$, of the Gaussian distribution from which we sample latent document-topic distributions. At this point, the languages are independent of each other and have not yet shared any information.  

After sampling topic distributions for each document, we induce a shared topic space by minimizing the pairwise KL divergence between the language-specific distributions whose parameters are estimated from their own inference network. We also add a contrastive objective so that aligned examples are kept away from other examples in the topic space. 
We use InfoNCE~\cite{van2018representation} as our contrastive loss. The positive pairs are all possible combinations of document pairs from the same tuple and negative pairs are all other pairs of documents from different tuples within a batch. For instance, for a comparable dataset with two languages and batch size $N$, we would have $N$ positive pairs and $N^2-N$ negative pairs per batch. For three languages, that would be $3N$ positive pairs and $3(N^2-N)$ negative pairs, etc.
% (n choose 2) = n! / (2 (n-2)!)

Thus, the loss consists of the three components: the reconstruction loss; the KL divergence between topic distributions; and the contrastive loss. %that ensures that pairs from the same tuple have similar topic distributions while pairs from different tuples are dissimilar in the topic space. 
Formally, the loss function is written as:
\begin{align}
\mathcal{L} & = \sum_{l=0}^{L} \mathbb{E}_{q}[w^{\top} \log(\text{softmax}(\beta_{l}\theta_{d}))] - \nonumber \\
& \sum_{\substack{a,b=0\\a \neq b}}^{n} \mathbb{KL}(p(\theta_{i}^{a} | x_{i}^{a}) || q(\theta_{i}^{b} | x_{i}^{b})) - \nonumber \\ 
& s \sum_{\substack{a,b=0\\a \neq b}}^{n} \log \frac{\exp( (\theta_{i}^{a} \cdot \theta_{i}^{b}) / \tau)}{ \sum_{j=0}^{N} \sum_{c,d=0}^{n} \exp( (\theta_{i}^{c} \cdot \theta_{j}^{d}) / \tau)}    
\label{eq:loss-m3l-contrast}
\end{align} 

The first term is the sum of the bag-of-words (BoW) reconstruction losses of each language in the corpus. %$w_d$ is the BoW representation of document $d$ for language $l$, $\beta_{l}$ is topic-term matrix for language $l$ and $\theta_d$ is the sampled topic representation for document $d$. 
We refer the reader to~\cite{srivastava2017autoencoding} for further details on the reconstruction loss.

The second term is the sum of the KL divergences between the language-specific document distributions, $p()$ and $q()$, whose mean and variance are estimated from language-specific inference networks; $\theta$ refers to the sampled topic representation of a document in a tuple where $i$ is the tuple index, $a$ and $b$ are the indices of the documents inside the tuple and $n$ is the size of the tuple. Lastly, $x$ refers to a document embedding.    

The third term is the InfoNCE loss where $(\theta_{i}^{a} \cdot \theta_{i}^{b})$ are positive pairs (they belong to the same tuple) and $(\theta_{i}^{c} \cdot \theta_{j}^{d})$ are negative pairs (they are from different tuples). $N$ is the batch size, $\tau$ is the temperature and $s$ is a constant to give additional weight to the contrastive loss.

\subsection{Extension to multimodal setting}
\label{sec:models-multimodal}
We now extend the proposed multilingual topic model to the \textit{multimodal} setting. Figure~\ref{fig:archi-m3l-contrast}(b) shows the architecture of the proposed multilingual \textit{and} multimodal topic model.  

We can think of the multimodal case as a generalization of the multilingual model. The loss function in Equation~\ref{eq:loss-m3l-contrast} remains essentially the same. Since a BoW representation is not available for images, the reconstruction loss is computed only on texts and the first loss term is unchanged. In the second term of the loss function, $x$ can be a document \textit{or} image embedding and $\theta$ is the sampled topic distribution for that embedding. 

Since the document or image embeddings abstract the modality of the data, the topic distributions are now modality-agnostic. Thus, the third term is also unchanged, except for the tuple size $n$. A multimodal dataset with one language and one image view would have $N$ positive and $N^2-N$ negative pairs, the same as in the bilingual case. For two languages \textit{and} one image, we would have $3N$ positive pairs and $3(N^2-N)$ negative pairs, as in the trilingual case. 

We refer to our proposed topic model as \textbf{M3L-Contrast} for \textit{multimodal multilingual (M3L) topic model with contrastive learning}.




\section{Experimental Setup}
% \newpage

\begin{table*}[t]
\setlength{\tabcolsep}{1.5mm}
\centering

\subtable[On 11 discriminative test tasks following the T0 benchmark.]{
\resizebox{\textwidth}{!}{%
    \begin{tabular}{l|l|c|ccccc|ccc|cc|c|c}
    \toprule[1pt]
    \multirow{2}{*}{Base Model} &
    \multirow{2}{*}{Method} &
    \multirow{2}{*}{\#Params} & 
    \multicolumn{5}{|c|}{\textbf{Natural Language Inference}} & \multicolumn{3}{|c|}{\textbf{Sentence Completion}} & \multicolumn{2}{c|}{\textbf{Coreference}} & \multicolumn{1}{c|}{\textbf{WSD}} 
    & \multirow{2}{*}{Avg.}\\
    & & & RTE & CB & ANLI1 & ANLI2 & ANLI3 & COPA & Hella. & Story. & WSC & Wino. & WiC &  \\
    \midrule[1pt]
    Decoder-only & GPT-3 & 175B 
        &63.5 &46.4
        &34.6 &	35.4&	34.5&	91.0&	78.9&	83.2&	65.4&	70.2&	- & -\\
    Decoder-only & GLaM & 137B 
        & 56.3	& 39.3	& 39.7	& 35.5	& 34.1	& 90.0	& 76.7	& 81.1	& 82.1	& 71.3	& 50.6 & 59.7\\
    MoE Decoder-only & GLaM & 64B 
        & 66.8	& 33.9	& 40.9	& 38.2	& 40.9	& 90.0	& 77.1	& 82.5	& 83.5	& 73.4	& 50.5 & 61.6\\
    Decoder-only & PaLM & 540B 
        & 72.9	& 51.8	& 48.0	& 44.2	& 45.7	& 93.0	& 83.4	& 84.6	& 89.1	& 81.1	& 59.1 & 68.5\\
    Decoder-only & FLAN & 137B 
        & 78.3	& 64.1	& 47.7	& 43.9	& 47.0	& 90.6	& 56.4	& 92.2	& 80.8	& 67.3 & - & -\\
    \midrule[1pt]
    \multirow{3}*{\shortstack{ELECTRA}}
    & PE-CLS & 335M
        & 60.2	& 57.4	& 34.1	& 34.4	& 36.4	& 92.7	& 44.1	& 96.0	& 62.8	& 56.3	& 50.7	& 56.8
        \\
    & PE-PROB & 335M
        & 54.0	& 49.2	& 32.3	& 33.3	& 33.5	& 81.9	& 36.7	& 89.5	& 64.3	& 50.7	& 50.9	& 52.4 \\
    & PE-REP & 335M
        & 69.0	& 61.3	& 36.1	& 35.0	& 39.4	& 91.2	& 47.0	& 96.8	& 70.0	& 56.2	& 51.1	& 58.5
        \\
    \midrule
    \multirow{1}*{\shortstack{DeBERTaV3}}
    & \multirow{1}*{{UD (ours)}} & 304M
        & \multirow{1}*{71.1}
        & \multirow{1}*{76.8}
        & \multirow{1}*{43.8}
        & \multirow{1}*{41.3}
        & \multirow{1}*{45.7}
        & \multirow{1}*{96.0}
        & \multirow{1}*{60.7}
        & \multirow{1}*{97.4}
        & \multirow{1}*{66.4}
        & \multirow{1}*{83.6}
        & \multirow{1}*{53.3}
        & \multirow{1}*{66.9}
    \\
    \midrule[1pt]
    \multirow{2}*{\shortstack{T5-Large}}
    & \multirow{1}*{T0 $\star$} & 800M
        & 75.1	& 55.5	& 32.9	& 32.3	& 33.7	& 84.6	& 28.2	& 94.0	& 63.0	& 54.6	& 51.2	& 55.0 \\


    & {UD (ours)} & 400M
        & \textbf{83.8}
        & \textbf{80.4}
        & \textbf{36.8}
        & \textbf{34.2}
        & \textbf{42.2}
        & \textbf{90.0}
        & \textbf{56.1}
        & \textbf{96.4}
        & \textbf{68.3}
        & \textbf{62.9}
        & \textbf{54.6}	
        & \textbf{64.1} \\
    \midrule[1pt]
    \multirow{3}*{\shortstack{T5-XL}}
    & \multirow{1}*{T0 $\dagger$} & 3B
        & 64.6 
        & 45.4
        & 33.8
        & 33.1
        & 33.3
        & 72.4
        & 27.3
        & 84.0
        & 65.1
        & 51.0
        & 50.7
        & 51.0 \\

    & \multirow{1}*{T0 $\star$} & 3B
    & \textbf{79.7}	& 68.9	& \textbf{43.1}	& \textbf{38.5}	& 42.3	& \textbf{94.1}	& 31.5	& 97.5	& 68.8	& 61.3	& \textbf{54.1}	& 61.8\\

 
    & {UD (ours)} & 1.5B
        & 78.7
        & \textbf{73.2}
        & 41.2
        & 36.3
        & \textbf{45.4}
        & 94.0
        & \textbf{70.1}
        & \textbf{97.9}
        & \textbf{72.1}
        & \textbf{70.6}
        & 53.0	
        & \textbf{66.6} \\
    \midrule[1pt]
    \multirow{4}*{\shortstack{T5-XXL}}
    & \multirow{1}*{T0 $\dagger$} & 11B
        & 80.8
        & 70.1
        & 43.6
        & 38.7
        & 41.3
        & 90.0
        & 33.6
        & 92.4
        & 61.5
        & 59.9
        & 56.6
        & 60.8 \\

    & \multirow{1}*{T0 $\star$} & 11B
    & \textbf{85.8}	& 73.3	& 47.3	& 42.0	& 46.1	& 94.4	& 31.5	& 98.4	& 62.8	& 72.8	& 56.0	& 64.6 \\

    & {UD (ours)} & 5.5B
    & 80.5	& 87.5	& 49.0	& 42.9 & 	48.8	& 95.0	& 77.4	& \textbf{98.6}	& 73.1	& 82.2	& 57.1	& 72.0 \\

    & {UD+ (ours)} & 5.5B
    & 82.0	& \textbf{89.3}	& \textbf{53.4} & \textbf{48.1} & \textbf{51.0} & \textbf{96.0} & \textbf{78.9} & 96.7	& \textbf{75.0}	& \textbf{86.4}	& \textbf{58.5}	& \textbf{74.1} \\
    \bottomrule[1pt]
\end{tabular}
}
\label{tab:maintable:top}
}


\subtable[On 13 discriminative BigBench tasks following the T0 benchmark]{
\resizebox{0.7\textwidth}{!}{%
    \begin{tabular}{l|cc|cc|ccc|}
    \toprule[1pt]
    \multirow{1}{*}{Model} 
        & \multirow{1}{*}{\shortstack{T0-Large}}
        & \multirow{1}{*}{\shortstack{UD-large}}
        & \multirow{1}{*}{\shortstack{T0-XL}}
        & \multirow{1}{*}{\shortstack{UD-XL}}
        & \multirow{1}{*}{\shortstack{T0-XXL}}
        & \multirow{1}{*}{\shortstack{UD-XXL}}
        & \multirow{1}{*}{\shortstack{UD+-XXL}}\\
    \midrule[1pt]
    BigBench (Avg.) & 39.6 & \textbf{43.5} & 44.8 & \textbf{48.9} & 47.4 & 55.5 & \textbf{58.7} \\
    \bottomrule[1pt]
    \end{tabular}%
    }
\label{tab:maintable:middle}
}

\subtable[On 22 discriminative BBH tasks]{
\resizebox{\textwidth}{!}{%
    \begin{tabular}{l|ccc|ccc|cccc|}
    \toprule[1pt]
    \multirow{1}{*}{Model} 
        & \multirow{1}{*}{\shortstack{T0-Large}}
        & \multirow{1}{*}{\shortstack{Flan-T5-Large}}
        & \multirow{1}{*}{\shortstack{UD-Large}}
        & \multirow{1}{*}{\shortstack{T0-XL}}
        & \multirow{1}{*}{\shortstack{Flan-T5-XL}}
        & \multirow{1}{*}{\shortstack{UD-XL}}
        & \multirow{1}{*}{\shortstack{T0-XXL}}
        & \multirow{1}{*}{\shortstack{Flan-T5-XXL}}
        & \multirow{1}{*}{\shortstack{UD-XXL}}
        & \multirow{1}{*}{\shortstack{UD+-XXL}}\\
    \midrule[1pt]
    BBH (Avg.) & 38.9 & 39.5 & \textbf{44.2} & 40.4 & 44.6 & \textbf{47.3} & 45.0 & 49.4 & 51.3 & \textbf{56.7} \\
    \bottomrule[1pt]
    \end{tabular}%
    }
\label{tab:maintable:bottom}
}
\caption{
Zero-shot performance of our UD and baselines.
Results in the first block are reported by previous work, respectively from GPT-3~\cite{gpt3-paper}, GLaM~\cite{glam}, PaLM~\cite{palm}, and FLAN~\cite{FLAN}.
Note that we provide these reported results for reference, and do not compare directly. Some of the reported tasks are evaluated on the test split, while we follow the better baseline method T0 to report on validation splits.
Results with $\dagger$ are reported by~\citeauthor{T0-paper}, and results with $\star$ are reproduced in our framework. We reproduced the three variants of prompting ELECTRA~\cite{xia2022prompting} under our setting, denoted as ``PE-CLS'', ``PE-PROB'', ``PE-REP''.
Results for Flan-T5-Large/Xl/XXL~\citep{flant5} are reproduced by testing zero-shot performance on their released checkpoints.
In the same group, T0 and Flan-T5 has 2x model parameters compared to UD. For abbreviation, we denote UD based on T5-XX as ``UD-XX'', e.g., UD-XL refers to UD based on the T5-XL model.
}
\label{tab:maintable}
\vspace{-0.7cm}
\end{table*}
% \begin{table}[t]
\setlength{\tabcolsep}{4.5mm}
\centering
    \resizebox{0.5\textwidth}{!}{%
    \begin{tabular}{lcccc}
        \toprule[1pt]
        % \textbf{Finetuned Task} & \textbf{Task Type} & \textbf{Metric} & \textbf{Eval Set} & \textbf{SOTA Reference} & \textbf{SOTA} & \textbf{Ours}  \\
        \textbf{Finetuned Task} & \textbf{T0} & \textbf{UD}  \\
        \midrule[1pt]
        MRPC & 90.5 & 89.7 (to be improve) \\
        QQP & 85.9 (to improve) & \textbf{91.6} \\
        PAWS & 95.1 & \textbf{97.2} \\
        WikiQA  & 96.1 & \textbf{96.5}\\
        CosmosQA & 88.4 & \textbf{90.7}\\
        DREAM & 90.5 & \textbf{91.6} \\
        QuAIL & 65.6 & \textbf{80.2} \\
        QuaRel & 88.2 & \textbf{95.3}\\
        QuaRTz & 94.1 & \textbf{94.5} \\
        SciQ & 97.6& \textbf{98.1}\\
        SocialIQA &  \textbf{82.2} & 81.7 \\
        WikiHop & & 58.6\\
        Amazon & \textbf{97.6}(to improve) & 97.3 \\
        IMDB & \textbf{96.9} & 96.7\\
        Rotten & 93.4 & \textbf{93.6} \\
        Yelp & 72.28 (first two prompts) & 68.1 (hard to improve)\\
        AGNews & 95.1 & \textbf{95.3} \\
        DBPedia & & \\
        TREC & 96.7 & \textbf{97.8}\\
        \bottomrule[1pt]
    \end{tabular}
    }
    \caption{Results on finetuned tasks for UD and the baseline T0. Both methods use T5-XXL as a base model. T0 has 2x model parameters compared to UD.}

    % compared with state-of-the-art results.}
    \label{tab:finetunedtasks}
\end{table}
\begin{table*}[!htp]
\setlength{\tabcolsep}{1.5mm}
\centering
\subtable[On 11 discriminative test tasks following the T0 benchmark.]{
\resizebox{\textwidth}{!}{%
    \begin{tabular}{l|ccccc|ccc|cc|c|c}
        \toprule[1pt]
        \multirow{2}*{Method}
        & \multicolumn{5}{c|}{\textbf{Natural Language Inference}} & \multicolumn{3}{c|}{\textbf{Sentence Completion}} & \multicolumn{2}{c|}{\textbf{Coreference}} & \multicolumn{1}{c|}{\textbf{WSD}} & \multirow{2}{*}{Avg.} \\
    & RTE & CB & ANLI1 & ANLI2 & ANLI3 & COPA & Hella. & Story. & WSC & Wino. & WiC &  \\
    \midrule[1pt]
    T0-XL %
        & \textbf{79.7}	& 68.9	& 43.1	& 38.5	& 42.3	& \textbf{94.1}	& 31.5	& \textbf{97.5}	& \textbf{68.8}	& 61.3	& \textbf{54.1}	& 61.8\\
    GenUD-XL %
        & 71.5	& \textbf{80.4}	& \textbf{43.1}	& \textbf{39.5}	& \textbf{42.6}	& 94.0	& \textbf{55.8}	& 96.7	& 63.5	& \textbf{75.5}	& 52.8	& \textbf{65.0}\\
    \bottomrule[1pt]
    \end{tabular}%
}
\label{tab:genud:top}
}
\subtable[On 13 discriminative Big-Bench tasks following the T0 benchmark.]{
\resizebox{\textwidth}{!}{%
    \begin{tabular}{l|ccccccccccccc|c}
    \toprule[1pt]
    \multirow{2}{*}{Model} 
        & \multirow{2}{*}{\shortstack{code \\ desc.}}
        & \multirow{2}{*}{\shortstack{conce\\-ptual}}
        & \multirow{2}{*}{\shortstack{known\\unknowns}}
        & \multirow{2}{*}{\shortstack{logic \\ grid}}
        & \multirow{2}{*}{\shortstack{logic \\ deduction}}
        & \multirow{2}{*}{\shortstack{miscon\\-ceptions}}
        & \multirow{2}{*}{\shortstack{novel\\concepts}}
        & \multirow{2}{*}{\shortstack{strate\\-gyqa}}
        & \multirow{2}{*}{\shortstack{wino\\-why}}
        & \multirow{2}{*}{\shortstack{syllo\\-gisms}}
        & \multirow{2}{*}{\shortstack{movie\\dialog}}
        & \multirow{2}{*}{\shortstack{lang\\-uage\_id}}
        & \multirow{2}{*}{\shortstack{vita\\-minc}} 
        & \multirow{2}{*}{Avg.} \\
    &&&&&&&&&&&&&&\\
    \midrule
    T0-XL & 23.4 & 48.1 & 64.6 & \textbf{42.5} & 50.1 & \textbf{52.7} & 25.0    & 53.1 & 45.4 & 50.2 & 47.7 & \textbf{19.0} & 60.0 & 44.8 \\
    GenUD-XL & \textbf{60.0} & \textbf{64.1} & \textbf{69.6} & 38.2 & \textbf{52.8}  & 48.9 & \textbf{44.1} & \textbf{57.1} & \textbf{46.5} & \textbf{50.4} & \textbf{50.9} & 15.5 & \textbf{66.8} & \textbf{48.9} \\
    \bottomrule[1pt]
    \end{tabular}%
\label{tab:genud:mid}
}
}




\subtable[On 15 generative tasks from Big-Bench]{
\resizebox{\textwidth}{!}{%
    \begin{tabular}{l|ccccccccccccccc|c}
    \toprule[1pt]
    \multirow{3}{*}{Model}
        & \multirow{3}{*}{\shortstack{auto \\ debugging}}
        & \multirow{3}{*}{\shortstack{simple \\ arith \\ -metic}}
        & \multirow{3}{*}{\shortstack{repeat\\copy \\ logic}}
        & \multirow{3}{*}{\shortstack{sufficient \\ information}}
        & \multirow{3}{*}{\shortstack{simple \\ text \\ editing}}
        & \multirow{3}{*}{\shortstack{scientific \\ press \\ release}}
        & \multirow{3}{*}{\shortstack{code\\ names}}     
        & \multirow{3}{*}{\shortstack{emoji\\movies}}
        & \multirow{3}{*}{\shortstack{penguins\\in a \\ table}}
        & \multirow{3}{*}{\shortstack{few \\ shot\\nlg}}
        & \multirow{3}{*}{\shortstack{operators}}
        & \multirow{3}{*}{\shortstack{tense}}
        & \multirow{3}{*}{\shortstack{geometric\\shapes}}
        & \multirow{3}{*}{\shortstack{chinese \\ remainder\\ theorem}}
        & \multirow{3}{*}{\shortstack{temporal\\sequences}}
        & \multirow{3}{*}{\shortstack{Avg.}}\\
    &&&&&&&&&&&&&&&&\\[1em]
    \midrule
    T0-XL & 11.2 & 6.7 & \textbf{25.8} & 33.8 & 7.5 & \textbf{6.7} & \textbf{44.8} & \textbf{8.7} & \textbf{11.4} & 17.4 & \textbf{10.5} & 80.7 & 0.0 & 0.0 & 14.0 & \textbf{18.6}\\
    GenUD-XL & \textbf{15.5} & 6.7 & 8.2 & \textbf{34.4} & \textbf{12.6} & 6.4 & 25.1 & 0.0 & 8.1 & \textbf{20.5} & 3.7 & \textbf{80.9} & 0.0 & 0.0 & \textbf{33.5}  & 17.0\\
    \bottomrule[1pt]
    \end{tabular}%
}
}

\caption{Zero-shot performance for generalized UD and T0 on discriminative and generative tasks. 
We select the top 15 uncommon generative tasks from BigBench basing on ascending order of data size. (We assume that datasets with smaller sizes are less common, and more suitable for zero-shot tests.) The metrics are respectively accuracy for discriminative tasks and ROUGE1 for generative tasks. ``GenUD'' denotes our generalized UD method.}
\label{tab:genud}
\end{table*}







\begin{table}[htbp]
\setlength{\tabcolsep}{1.5mm}
  \centering
\resizebox{0.35\textwidth}{!}{
    \begin{tabular}{lcc}
    \toprule
    \textbf{Dataset} & \textbf{SOTA} & \textbf{UD+-XXL} \\
    \midrule
    QQP     & \textbf{90.60}  & 90.44 \\
    DREAM     & 91.80  & \textbf{94.95} \\
    QuAIL   & 87.20  & \textbf{88.13} \\
    IMDB    & 97.30  & \textbf{97.44}  \\
    AgNews   & \textbf{95.58}  & 95.56  \\
    OBQA   & 87.20  & \textbf{89.20} \\
    STSB     & 92.30  & \textbf{92.90} \\
    CSQA    & \textbf{84.90}  & 84.68  \\
    SST-2     & 97.30  & \textbf{97.48} \\
    QNLI    & 96.50  & \textbf{96.56} \\
    AbductiveNLI &  89.80  & \textbf{93.20} \\
    VitaminC   & 91.10  & \textbf{92.62} \\
    MNLI  &  \textbf{92.10}  & 92.03  \\
    MCScript &  97.30  & \textbf{98.03} \\
    MCScript 2.0 &  97.90  & \textbf{98.01} \\
    AdversarialNLI (r3) &53.50  & \textbf{67.83 } \\
    COLA   & \textbf{71.50}  & 71.42  \\
    \midrule
    Avg.   & 89.05  & \textbf{90.62} \\
    \bottomrule
    \end{tabular}%
}
  \caption{Results on fully-supervised tasks for UD, which is based on the encoder of T5-xxl. Previous sota model \citep{ul2} has 4x model parameters compared to UD. }
  \label{tab:finetune}%
\vspace{-0.7cm}
\end{table}%



\section{Experiments}

\begin{table*}[t]
\setlength{\tabcolsep}{1.5mm}
\centering
\small
\resizebox{\textwidth}{!}{%
    \begin{tabular}{l|ccccc|ccc|cc|c|c}
        \toprule[1pt]
        & \multicolumn{5}{c|}{\textbf{Natural Language Inference}} & \multicolumn{3}{|c|}{\textbf{Sentence Completion}} & \multicolumn{2}{c|}{\textbf{Coreference}} & \multicolumn{1}{c|}{\textbf{WSD}} & \multirow{2}{*}{Avg.} \\
    & RTE & CB & ANLI1 & ANLI2 & ANLI3 & COPA & Hella. & Story. & WSC & Wino. & WiC &  \\
    \midrule[1pt]
    UD (Minimal)     & \textbf{83.75}
        & \textbf{80.36}
        & 36.80
        & \textbf{34.20}
        & \textbf{42.17}
        & \textbf{90.00}
        & \textbf{56.07}
        & \textbf{96.37}
        & \textbf{68.27}
        & \textbf{62.90}
        & \textbf{54.55}	
        & \textbf{64.13} \\
    UD (Instructive)    & 72.24 
        & 64.52 
        & \textbf{36.98} 
        & 33.40 
        & 39.73 
        & 85.31 
        & 45.15 
        & 96.01 
        & 65.38 
        & 53.94 
        & 50.94 
        & 58.51\\
    \midrule
    T0 (Minimal) & 61.56  & \textbf{57.81}  & 30.57  & 30.27  & 33.38  & 67.19  & \textbf{33.81}  & 66.56  & 60.94  & 52.81  & \textbf{51.72}  & 49.69  \\
    T0 (Instructive) & \textbf{75.05}	& 55.48	& \textbf{32.87}	& \textbf{32.29}	& \textbf{33.67}	& \textbf{84.59}	& 28.24	& \textbf{93.97}	& \textbf{62.98}	& \textbf{54.59}	& 51.16	& \textbf{54.99} \\

    \bottomrule[1pt]
    \end{tabular}}
    \caption{Zero-shot performance for UD and T0 respectively with instructive and minimal prompts. Instructive prompts are lengthy descriptions of tasks \citep{T0-paper}, while minimal prompts use a simple concatenation of input data.}
\label{tab:promptablatiion}
\end{table*}

\begin{table*}[ht]
\setlength{\tabcolsep}{0.9mm}
\centering
\resizebox{\textwidth}{!}{%
    \begin{tabular}{l|l|ccccc|ccc|cc|c|c}
        \toprule[1pt]
        & \multirow{2}*{Base Model}
        & \multicolumn{5}{c|}{\textbf{Natural Language Inference}} & \multicolumn{3}{|c|}{\textbf{Sentence Completion}} & \multicolumn{2}{c|}{\textbf{Coreference}} & \multicolumn{1}{c|}{\textbf{WSD}} & \multirow{2}{*}{Avg.} \\
    & & RTE & CB & ANLI1 & ANLI2 & ANLI3 & COPA & Hella. & Story. & WSC & Wino. & WiC &  \\
    \midrule[1pt]
    \multirow{2}*{\shortstack{Encoder}}
    & DeBERTa-V3 (304M) 
        & 71.1
        & 76.8
        & 43.8
        & 41.3
        & 45.7
        & 96.0
        & 60.7
        & 97.4
        & 66.4
        & 83.6
        & 53.3
        & 66.9 \\
    & DeBERTa-V2 (1.5B) 
        & 77.6
        & 80.4
        & 43.2
        & 39.3
        & 44.8
        & 95.0
        & 67.2
        & 98.2
        & 74.0	& 82.1 & 56.0	& 68.9\\ \midrule
    \multirow{2}*{\shortstack{Enc-Dec}} & T5-Encoder (400M) 
        & 75.1	& 55.5	& 32.9	& 32.3	& 33.7	& 84.6	& 28.2	& 94.0	& 63.0	& 54.6	& 51.2	& 55.0 \\
    & T5-Encoder (1.5B)  & 79.7	& 68.9	& 43.1	& 38.5	& 42.3	& 94.1	& 31.5	& 97.5	& 68.8	& 61.3	& 54.1	& 61.8\\
    \midrule
    \multirow{1}*{\shortstack{Decoder}}
    & \multirow{1}*{GPT-XL (1.5B)}
        & \multirow{1}*{71.1}
        & \multirow{1}*{75.0}
        & \multirow{1}*{30.4}
        & \multirow{1}*{31.8}
        & \multirow{1}*{37.8}
        & \multirow{1}*{71.0}
        & \multirow{1}*{40.9}
        & \multirow{1}*{87.7}
        & \multirow{1}*{62.5}
        & \multirow{1}*{54.5}
        & \multirow{1}*{50.3}
        & \multirow{1}*{55.7}
    \\
    \bottomrule[1pt]
    \end{tabular}}
    \caption{Ablation study on different backbone models. We experiment with base models of different architectures and scales. ``Enc-Dec'' refers to models that are pretrained in an encoder-decoder manner.}
    \label{tab:ablationbasemodel}
\end{table*}
\begin{table}
\centering
\setlength{\tabcolsep}{3.0mm}
\resizebox{0.5\textwidth}{!}{%
\begin{tabular}{l|c}
    \toprule[1pt]
    Setting & Accuracy \\
    \midrule[1pt]
    True Data vs Manually-Generated Data & 80.0 \\
    True Data vs Model-Generated Data & 74.4 \\
    \bottomrule[1pt]
    \end{tabular}%
    }
    \caption{
    The accuracy of UD discriminating real data and generated data. We feed UD with a real sample $x$ from the real-world data distribution, and a sample $x'$ from manual generation or model-based generation. 
    If UD assigns higher score to $x$ than $x'$ (i.e., $D(x)>D(x')$), it is considered an accurate prediction.
    }
  \label{tab:explain}%
\end{table}%



\subsection{Experimental Setup}\label{sec:setup}

We performed extensive experiments to validate the performance of the zero-shot generalization of our UD. We follow the same zero-shot setting as T0~\citep{T0-paper} by training on multi-task datasets and evaluating a held-out set of tasks that are never seen during training. 

\paragraph{Datasets}
The original T0 training set consists of 38 tasks of 8 different types.
% ~\footnote{We did not consider T0+ and T0++, since they are partially intersected with the test sets, making some test tasks unable to be evaluated under the zero-shot setting.}
There are in total 21/38 discriminative training tasks, with which we train the UD.
% ~\footnote{The original paper~\citep{T0-paper} claims 39 training datasets but releases a training set with 38 datasets (``common\_gen''  excluded). We directly start with the released data.}. 
% It consists of a majority of discriminative tasks and a small number of generative tasks.
% We train our \method with 21 discriminative tasks within.
% , which is around 55\% of the original T0 training data.\xhk{change to 21/38=0.55}
The evaluation set covers four types of tasks, including natural language inference (RTE~\citep{2005_RTE}, CB~\citep{de2019_CB}, ANLI/R1-R3~\citep{NieWDBWK20_ANLI}), coreference resolution (WSC~\citep{WSC2012}, Winogrande~\citep{SakaguchiBBC20_winogrande}), sentence completion (COPA~\citep{COPA2011}, StoryCloze~\citep{story_cloze}, Hellaswag~\citep{ZellersHBFC19_hellaswag}), and word sense disambiguation (WiC~\citep{wic-paper}).
Following T0, we use accuracy on the validation split as the evaluation metric.
For prompt-based baselines, we report the average accuracy over multiple prompts for each test task.
Besides, we also evaluate zero-shot performance on several BigBench~\cite{bigbench} tasks, which are also adopted by T0~\cite{T0-paper}.\footnote{The original T0 reported results on 14 BigBench tasks. We separately report the results of 13 discriminative tasks and the other generative task in the following.}


% \lzy{Noted that we also evaluate the zero-shot performance on a subset of Big-Bench Benchmark~\cite{bigbench} adopted by original T0 paper~\cite{T0-paper}.\footnote{The original T0 reported results on 14 BigBench tasks. In our work, we focus on 13 discriminative tasks, leaving improving performance of the only generative tasks for future exploration. \zy{No need to say that. We also have generation results. Just say we're gonna report generation result separately.}}}


\paragraph{Baselines}
We primarily compare our method with T0~\citep{T0-paper}, which is a generative approach.
% that shares the same goal as UD (i.e., zero-shot generalization), but uses a totally different framework (i.e., generative or discriminative) as well as input format (i.e., prompt or minimal prompt).
Another baseline is prompting ELECTRA~\cite{xia2022prompting} which is a recent work on discriminative modeling.
Since it was proposed in a different setting (i.e., a  few-shot setting or direct zero-shot inference without any finetuning), we reproduced their method under our multitask zero-shot setting for comparison.

For a fair comparison, we follow T0 to use the T5-V1.1-LM-Adapted~\citep{T5-paper} as the backbone model, and we experimented with three different scales, respectively 800M, 3B, and 11B. 
For UD, it only makes use of the encoder of T5-v1.1 and additionally replaces the output layer with a classification head.
Moreover, for direct comparison with \citet{xia2022prompting}, we use DeBERTaV3-Large \citep{debertav3} as the backbone model which shares the same bidirectional architecture and has a smaller number of parameters.

In addition, we also provide reported zero-shot results of several large language models (with hundreds of billions of parameters) for reference, including GPT-3~\cite{gpt3-paper}, GLaM~\cite{glam}, PaLM~\cite{palm}, and FLAN~\cite{FLAN}.


% we also experiment with another backbone DeBERTaV3-Large~\citep{debertav3} to achieve better zero-shot performance.

\paragraph{Training}
% We implemented both baselines and our method, and perform experiments with exactly the same environments.
During training, we truncate the input sequence to 256 tokens and use a batch size of 256. For optimization, we use the Adam optimizer with a fixed learning rate of 1e-5 and a dropout rate of 0.1. Each experiment is trained with 10, 8, and 5 epochs respectively for 800M, 3B, and 11B models.
% We perform checkpoint selection by directly using the final (fixed-epoch) checkpoint for evaluation.
% \xhk{by choosing the one with the maximal average zero-shot performance per xxx steps.}

% For data processing, similar to T0, we truncate any dataset with over MAX\_DATA\_SIZE to have MAX\_DATA\_SIZE / num\_prompts. 
% Different from ~\citet{T0-paper} that uses a value of 500k for MAX\_DATA\_SIZE, we use a value of 50k, which experimentally yields better zero-shot performance for the T0 baseline.
% The training data of UD are produced by \xhk{replacing different prompted data version from T0 training data with only one minimal prompted version}, which strictly guarantees all methods share same raw task data.




% \subsection{Main Results}
\subsection{Main Results on Zero-Shot Tasks}

\paragraph{UD Zero-Shot Results}
The main results are presented in Table~\ref{tab:maintable}.
We compare methods of similar scales. 
Results in Table \ref{tab:maintable:top} show that our UD substantially outperforms the T0 baseline on average by a large margin of around 9, 5, and 7 points respectively at Large, XL, and XXL scales.
Comparing the results of UD-T5-Large, UD-DeBERTaV3, and prompting ELECTRA, both variants of UD also substantially outperform prompting ELECTRA by more than 6 points.
% In addition, UD also demonstrates superior zero-shot ability compared with models with hundreds of billions of parameters (see results in the first block of Table~\ref{tab:maintable:top}).
On BIG-Bench datasets, results in Table \ref{tab:maintable:bottom} show that our UD outperforms the T0 baseline by a margin of around 4-8 points.
Overall, these results demonstrate the advantages of UD at every scale, and a broad range of tasks compared with baselines.

Another interesting finding is that the advantages of UD significantly increase along with scaling.
When scaling from Large-scale to XL-scale (i.e., around 3.75x of the parameters), the average performance improves by around 2 points. However, when scaling from XL-scale to XXL-scale (i.e., 3.6x of the parameters), the improvements of average zero-shot performance enlarge to 8 points.
Based on the observation, we hypothesize that UD can achieve even better performance of zero-shot generalization if further scaling to an even larger models, which we leave to future work.

% Results show that our \method substantially outperforms our baseline T0 on average zero-shot performance, by a large margin of around 8, 5, and 7 points respectively at the Large (800M), XL (3B), and XXL (11B) scales.
% \xhk{Our UD also substantially outperforms ELECTRA in the Large (800M) scale.} 

To further boost the zero-shot performance, we also train a new variant of UD at 11B scale by scaling to more training tasks, including the discriminative English tasks used in \citet{1600tasks}, and the discriminative English tasks used in \citet{ul2}. The new model is denoted as UD+.
UD+ achieves the highest average accuracy among all the zero-shot evaluation tests.

% \begin{comment}
% ul2 (CommonsenseQA \cite{commonsense_qa}, ),
% csqa2.json 9264
% glue_cola.json 8551
% glue_sst2.json 67349
% glue_stsb.json 5749
% mcscript.json 19462
% mcscript2.json 28382
% openbookqa.json 19828
% qasc.json 40670
% qasc_with_ir.json 40670
% race_high.json 249780
% race_middle.json 101684
% social_i_qa.json 100230
% super_glue_boolq.json 9427
% super_glue_multirc.json 27243
% ai2_science_elementary.json 2493
% ai2_science_middle.json 2424
% onestopqa_advanced.json 1296
% physical_iqa.json 33211
% protocol_comparison_harsht.json 3698
% reclor.json 3726
% ai2_arc_ARC_Easy.json 9002
% ai2_arc_ARC_Challenge.json 4476r,  
% \end{comment}
% \zy{what data?}

% \xhk{move bigbench result in appendix A.1 here. into Table 2}
% \yn{add bigbench analysis}

% , in the mean time still guaranteeing that there is no overlap between training tasks and the held-out tasks.

% \xhk{We also extend our training datasets (please refer to appendix~\ref{sec:ud_plus_data}) and train a model UD+. 
% UD+ achieves the highest average accuracy among all the zero-shot evaluation test, in the mean time still guaranteeing that there is no overlap between training tasks and the held-out tasks.}

% \yn{do we need to add the following?}
% \yn{
% Interestingly, we also have observed some findings on zero-shot performance along with scaling.
% For baseline T0, the zero-shot performance keeps improving on most of the datasets when scaling to larger-scale models.
% Exceptions are Hellaswag and WSC, where zero-shot performance on them are basically unchanged when scaling.
% For our \method, the performance of zero-shot generalization consistently improves with the model scale increasing on all sentence completion and coreference resolution tasks, and partial NLI tasks.
% Exceptions are that RTE, CB and WSC demonstrates a degradation on zero-shot performance when scaling from large to XL scale.
% This could be explained that 
% % {\color{red} xxxxx}
% % \xhk{I guess if we add minimal prompts, RTE and CB will improve for larger model...? so maybe we can remove this observation for now?}
% }


% \paragraph{Results on Finetuned Tasks}

% To evaluate the performance on finetuned tasks, we finetuned T0/UD respectively on each training task. This is similar to multi-task finetuning \cite{T5-paper}.
% % We use this experiment to test the effectiveness of UD with abundant labels. 
% We experimented with all the T0 discriminative training tasks.
% Table~\ref{tab:finetunedtasks} shows the finetuning results on T0 and UD at the 11B scale.
% We observe that UD outperforms T0 on \textbf{\color{red} xxx/19} of the considered finetuned tasks.
% To be specific, on topic classification tasks, paraphrase identification tasks, and multiple-choice QA tasks, UD shows the largest advantages against T0. These finetuning results demonstrate that UD does not only perform well in the zero-shot setting but also improves performance when abundant labels are available.



% \yn{add a new subsection of seq2seqUD}
\paragraph{Generalized UD Zero-Shot Results}

The zero-shot results of generalized UD on 11 T0 discriminative test tasks and on 13 Big-Bench tasks are respectively reported in Table~\ref{tab:gen_ud:top} and Table~\ref{tab:gen_ud:mid}.
In addition, to test how generalized UD performs on zero-shot generative tasks, we also select 4 generative tasks from Big-Bench for evaluation. Results are presented in Table~\ref{tab:gen_ud:bottom}.


Analyses are as follows.
(1) Comparing the results of generalized UD and T0, generalized UD still holds significant improvements on discriminative tasks.
(2) Comparing generalized UD with our previous UD (in Table~\ref{tab:maintable}), we observe there is a slight decrease in average performance, proving that adding generative tasks into training could have impacted a little bit, in trade for capability for handling generative tasks.
(3) On 4 generative zero-shot tasks, both generalized UD and T0 show comparable results.
(4) On 13 discriminative BigBench tasks, we observe that UD-Large outperforms T0-Large by 6.67\%, UD-XL outperforms T0-XL by over 4\%, and Generalized UD-XL outperforms T0-XL by over 6\%, further indicating the effectiveness of our proposed framework.


%\yn{recheck the analysis along with table data!}

% Comparing Generalized UD with methods in Table~\ref{tab:maintable} (i.e., UD and T0) of similar scales, we observe the zero-shot performance on discriminiative tasks slightly decrease but generally hold still, compared to UD (ours).

% It still significantly outperforms baseline T0 to a large degree.
% From Table~\ref{tab:gen_ud}, we shall observe, on generative tasks both generalized UD and T0 show comparable results.}









\subsection{SOTA Results on Finetuned Tasks}
\label{sec:ud_finetune}

To explore how UD performs on fully-supervised tasks, we finetuned UD for a wide range of downstream tasks and reported their results in Table \ref{tab:finetune}.
% To explore whether UD can help improve the performance in fully-supervised learning, we conduct experiments by finetuning each downstream task. 
For each finetuning experiment, the maximum training epoch is set to be 10.
We search a hyper-parameter space with learning rate in \{2e-5, 1e-5, 5e-6\}, batch size in \{32, 64, 128\}.
We select the best checkpoint using a validation set with early stopping.
% We set the maximum training epoch to 10, search the hyper-parameters (learning rate in \{2e-5, 1e-5, 5e-6\}, batch size in \{32, 64, 128\}) and select the best checkpoint based on the validation set with early stopping.

% Results are in Table \ref{tab:finetune}.
From results in Table \ref{tab:finetune}, we find that UD can achieve remarkable performance on most of the downstream tasks. 
We achieve state-of-the-art performance on 12 out of the 17 tasks we evaluated. The results also show that more challenging tasks (tasks that require more knowledge) will benefit more from the multi-task training period, especially some QA tasks.








\subsection{Ablation Study}

We have also conducted ablation studies to further explore how several factors affect the performance of zero-shot generalization. 

\subsubsection{Instructive Prompts vs Minimal Prompts}

UD employs minimal prompts that use simple concatenation, while previous approaches rely on lengthy instructive prompts to provide more detailed instructions \cite{T0-paper,FLAN,gpt3-paper}. 
Statistically, we count the average number of prompt words (excluding raw input) for both minimal and instructive prompts, and statistics are respectively $0.4$ versus $>10$.
% \xhk{A statistic comparison on the average number of the prompt word count (excluding raw input) is $0.4$ for minimal prompts versus $>10$ for previous instructive prompts.}  
We compare these two types of prompts in the following experiment.
We adopt the instructive prompts from T0 and apply them on UD without changing the discriminator formulation. To construct minimal prompts for T0, we remove all the instructive words similar to UD.

% It is an interesting question whether minimal prompts also play a role in the \method, 
%considering that concatenating task data with prompts theoretically indeed reduces all tasks into the original LM tasks, hence improving task generalization.
% considering that simple concatenation of task data's keywords with a minimal prompt is enough to unify it into the UD format.

% We compare the zero-shot performance when using prompt and minimal prompt for \method and prompt and prompt-free for T0. 



% To construct instructive prompts for UD, we adopt the instructive prompts from T0 we concatenated the prompted inputs and each target choice (verbalizer).~\footnote{Here we use the same prompts as T0.} The corresponding \method label is 1 when concatenating correct target choice and 0 otherwise.

% \xhk{To construct prompt-free inputs for T0, we directly remove all the prompt words, still letting the model to predict the target verbalizer.}


Results are shown in Table~\ref{tab:promptablatiion}. We observe that minimal prompts yield better performance for UD than instructive prompts. In contrast, for T0, instructive prompts perform much better than minimal prompts. These results are consistent with our motivation that UD tends to unify the tasks better with a shared discrimination formulation. As a result, task-specific instructions are not necessary and might hurt generalization performance. Generative approaches, on the other hand, rely on instructive prompts to better distinguish different tasks.


% \xhk{for our UD method, minimal prompt version has better accuracy, because under the unified UD task format, task descriptive language in prompt is no longer needed and may even increase the sentence complexity to be understood by LM. Additionally, UD's tasks is to discriminate between correct and wrong choices where prompts are identical phrases in each choice's concatenated sentence, so prompts actually play no role in the discriminating process. However, for T0, prompted version has better accuracy than the prompt-free version (note that prompt-free is the extreme and usual case for minimal prompting) because generative model's goal is to generate the correct verbalizer from the huge vocabulary, which can be efficiently narrowed by the existence of prompts. Therefore, we can conclude that minimal prompted format works well for discriminative models and prompted format works well for generative models.}



\subsubsection{Ablation on Base Models}

We also study the effects of using different backbone pretrained models. We experiment with three backbone models of different types, respectively the encoder part of an encoder-decoder model, an encoder model, and a decoder model. Specifically, we use the T5 encoder, DeBERTa \cite{debertav3}, and GPT \cite{radford2018gpt} respectively for these three types. It is noteworthy that though similar in architecture for both T5 encoder and DeBERTa, they are pretrained with different self-supervised language modeling tasks, which in fact leads to huge differences in zero-shot generalization, as we will show in Table~\ref{tab:ablationbasemodel}.
% We study the effect of different backbone pretrained models. We experiment with three types of backbone models---using the encoder part of an encoder-decoder model, using an encoder model, and using a decoder model. We use the T5 encoder, DeBERTa \cite{debertav3}, and GPT \cite{radford2018gpt} respectively for these three types.






% We study the effect of different types of models (discriminative vs. generative), or backbone models (auto-encoding vs. auto-regressive), on zero-shot generalization with \method. In addition to T5-Encoder, we also experiment the advanced DeBERTaV3-Large~\cite{debertav3} that has achieved new SOTA on a diverse set of tasks. We also implement GPT-XL for comparision.

% Results are shown in Table~\ref{tab:promptablatiion}.

% shows the results between discriminative and generative models with fixed prompted or not version. It can be observed \xhk{no matter we use promped data or minimal prompt/prompt-free data, our discriminative models always have better zero-shot generalization performance than generative models.}



Results of different backbone models are presented in Table \ref{tab:ablationbasemodel}. 
Among all three types of backbone models, the encoder backbone models appear to be the most suitable type of backbone, where both encoder models of two scales respectively achieve the best and the second best results, outperforming all the others by more than 5 points.

Using the same number of parameters (i.e., 1.5B), both DeBERTa-V2 and T5-Encoder significantly outperform GPT-XL, which demonstrates that a bidirectional architecture works better than the unidirectional architecture for the discriminator formulation.
In addition, DeBERTa-V2 outperforms T5-Encoder by 7 points, implying that not only model architecture but also the self-supervised pretraining task determines the ability of UD discrimination. Models pretrained with masked language modeling tasks are more suitable for UD.

The impacts of the architecture and pretraining tasks of backbone models are even larger than the influence of scale, as we also observe that an encoder model with 300M parameters (i.e., DeBERTaV3) achieves much better performance than the T5 encoder and GPT-XL with 1.5B parameters.

% Results are shown in Table \ref{tab:ablationbasemodel}. Using the same number of parameters, encoder backbone models (i.e., DeBERTa) substantially outperform the T5 encoder and the GPT decoder. This indicates that pretrained encoders are more suitable for our discriminator formulation. Interestingly, an encoder model with 300M parameters (i.e., DeBERTaV3) achieves much better performance than the T5 encoder and GPT-XL with 1.5B parameters.








% Table~\ref{tab:ablationbasemodel} shows the results for different discriminative models, where DeberTa consists of solely an encoder, T5-Encoder is the encoder part of the full T5 model, GPT-XL consists of an encoder and a decoder. We can observe that the encoder structure performs better for discriminative tasks.

% \subsection{What Contribute to the Zero-Shot Generalization of \method?}

\subsection{How Well UD Generalizes to a Broader Domain?} \label{sec:generalize}

In the previous sections, we have trained UD to solve the task of discriminating whether a text sample comes from the true data distribution of natural language. So far we have constrained the problem to supervised labeled tasks. However, this discrimination problem formulation is in fact general and can be applied to a broader domain of natural language. We conduct the following experiment to see how UD generalizes.


% In order to explore the mechanism of the universal discriminator and explain how it promotes zero-shot generalization. We conduct the following extensive experiment.

To test whether a model discriminates against the true data distribution, a straightforward way of verification is to compare the probability of real data with that of some generated, fake data. This form of verification is not specific to any downstream task and can be viewed as generalizing to a broader domain. Formally, given a text sample $x$, let $D(x)$ be the output of UD, which estimates the probability that $x$ is sampled from the true data distribution, i.e., $P(\text{true} | x)$. Given a true data sample $x$ and a generated data sample $x'$, we expect a well-trained UD to predict $D(x) > D(x')$.

% First, we assume that the essence of our universal discriminator $D$ is to learn whether the data are sampled from the real text distribution or not. A straightforward way to verify this key point is to compare the likelihood of the real data label given real data x computed as $D(x)=p(y=1|x)$ with the likelihood of the real data label given generated data $x'$ computed as $D(x’)=p(y=1|x’)$. 

Specifically, we randomly select 2,600 real data samples $x$ from the validation set of the T0 training data and generate the data $x’$ in two different ways: model-based generation and manual generation.

For a model-based generation, we utilize the T0-Large model with a paraphrase prefix ``Paraphrase the sentence:'' to generate data $x'$. It is expected that the generated samples $x'$ are similar to true samples $x$ to some extent but demonstrate some flaws that are unique to generated data. For a manual generation, we manually create some conflict or contradiction in the real sample $x$. Specifically, we manually attach wrong answers to the original data and obtain $x’$ , which is similar to what we have done in constructing negative samples in our main framework. 

We then use our \method based on T5-Encoder Large to compute the probability $D(x)$ and $D(x')$ for both real and generated data. As displayed in Table~\ref{tab:explain}, we find that the \method assigns a higher score for $x$ than $x'$ $80\%$ of the time for manually-generated data. When tested with model-generated data, UD assigns a high probability for real data in $74\%$ of the cases.
This is probably because manually generated data are more paradoxical and logically incoherent and thus are easier for UD to discriminate. Overall, these results demonstrate that the discrimination ability of UD is not limited to the downstream tasks on which it was trained, but is also generalizable to a broader domain of text data. This indicates a possibility of extending UD to other scenarios such as model pretraining and generation tasks.


% For model-based generation, we utilize two models which generate high-quality and low-quality data $x$. It should be noted that we hope the generated $x'$are similar to $x$ to some extent.
% First, we leverage a T5-small model [citation] to generate similar semantics to real data x by feeding the $x$ with the prefix  ‘paraphrase:’. Obviously, the generated $x'$ are bound to be far from real data distribution. Then, we utilize the T5-small model to finetune on quora for paraphrase identification task. Then we leverage the finetuned T5-small model to do the same paraphrase generation as before and yield $x’$ with relatively high quality. 

% For heuristic-based generation, we manually create some conflict or contradiction in the real data $x$. In detail, we randomly shuffle the words given each real data sample and get inconsistent data $x’$.

% After generating the data $x’$ from different approaches, we evaluate the likelihood of real data distribution given real data $x$ and generated data $x’$, which is formulated as $D(x)=p(y|x)$ and $D(x’)=p(y|x’)$ respectively. The results are shown in Table [reference] and the generated data examples are presented in Appendix [reference].






%\section{Ablation Study}
%\begin{table*}[t]
\setlength{\tabcolsep}{1.5mm}
\centering
\small
\resizebox{\textwidth}{!}{%
    \begin{tabular}{l|ccccc|ccc|cc|c|c}
        \toprule[1pt]
        & \multicolumn{5}{c|}{\textbf{Natural Language Inference}} & \multicolumn{3}{|c|}{\textbf{Sentence Completion}} & \multicolumn{2}{c|}{\textbf{Coreference}} & \multicolumn{1}{c|}{\textbf{WSD}} & \multirow{2}{*}{Avg.} \\
    & RTE & CB & ANLI1 & ANLI2 & ANLI3 & COPA & Hella. & Story. & WSC & Wino. & WiC &  \\
    \midrule[1pt]
    UD (Minimal)     & \textbf{83.75}
        & \textbf{80.36}
        & 36.80
        & \textbf{34.20}
        & \textbf{42.17}
        & \textbf{90.00}
        & \textbf{56.07}
        & \textbf{96.37}
        & \textbf{68.27}
        & \textbf{62.90}
        & \textbf{54.55}	
        & \textbf{64.13} \\
    UD (Instructive)    & 72.24 
        & 64.52 
        & \textbf{36.98} 
        & 33.40 
        & 39.73 
        & 85.31 
        & 45.15 
        & 96.01 
        & 65.38 
        & 53.94 
        & 50.94 
        & 58.51\\
    \midrule
    T0 (Minimal) & 61.56  & \textbf{57.81}  & 30.57  & 30.27  & 33.38  & 67.19  & \textbf{33.81}  & 66.56  & 60.94  & 52.81  & \textbf{51.72}  & 49.69  \\
    T0 (Instructive) & \textbf{75.05}	& 55.48	& \textbf{32.87}	& \textbf{32.29}	& \textbf{33.67}	& \textbf{84.59}	& 28.24	& \textbf{93.97}	& \textbf{62.98}	& \textbf{54.59}	& 51.16	& \textbf{54.99} \\

    \bottomrule[1pt]
    \end{tabular}}
    \caption{Zero-shot performance for UD and T0 respectively with instructive and minimal prompts. Instructive prompts are lengthy descriptions of tasks \citep{T0-paper}, while minimal prompts use a simple concatenation of input data.}
\label{tab:promptablatiion}
\end{table*}


\section{Results and Discussion}
In this section we experimentally evaluate our method -- herein termed NIFM for Neural Integration-free Flow Maps -- for both 2D and 3D time-varying vector fields, comparing against various baselines that accelerate flow map computation in different ways. \new{A requirement that is common to all baselines is access to samples of the flow map. Unless otherwise stated (c.f. Sec.~\ref{subsec:error}), the methods against which we compare NIFM are based on flow maps generated via $4^{th}$ order Runge-Kutta integration (RK4), with step size set to half of the temporal voxel size. We also use this very integration scheme to generate ground-truth flow map samples for the purposes of evaluation.} In Table~\ref{tab:datasets} we list the datasets used for comparison purposes. Further, all reported computational timings are based on a system with 12-core CPU AMD Ryzen 9 3900X, 16GB RAM, and GPU NVIDIA GeForce RTX 2080 Ti with 12GB memory.

We consider the flow map super resolution technique proposed by Jakob et al.~\cite{jakob2020fluid}, wherein we train a convolutional neural network (CNN) model using the 2D fluid flow dataset provided by the authors. To train the CNN we generate 16x downsampled flow maps along with their corresponding high-resolution ground truth flow maps, varying start times and time span of the integration, to permit model generalization for arbitrary start time/duration.

Additionally, we compare our method with the deep learning based Lagrangian interpolation technique proposed by Han et al.~\cite{han2021exploratory}. This technique uses an encoder-decoder network and is most similar to ours in terms of the input data the model expects, and the output of the model. We train the model on flow map samples computed by, first, generating seeds sampled uniformly at random in space and time, and secondly, integrating for varying small time spans. This flow map sampling technique is intended to resemble the Lagrangian short generation scheme proposed by the authors. We made a minor modification to the network by removing the ReLU activation function used in the output layer, allowing the model to output negative values. Further, we compare our method with a SIREN~\cite{sitzmann2020implicit} that tacks time span on as an additional coordinate, along with particle space-time coordinates (c.f. Fig.~\ref{fig:illustrative_network}(a)). We train the SIREN with the same data used to train the encoder-decoder model. Note that we could use a hybrid grid-MLP model~\cite{muller2022instant,weiss2021fast} in lieu of a standard coordinate-based MLP, but for 3D unsteady flows this would require storage of a 5D grid, which is not feasible.

We also compare our method against the recent work by Li et al.~\cite{li2022efficient}, where the authors showed an improvement over prior work in efficiently interpolating Lagrangian representation to obtain new trajectories. \new{Note that the representation of flow in our datasets is Eulerian, whereas Li et al. works with particle-based data, thus, requiring a conversion from the former to the latter. For a fair comparison, we convert the Eulerian representation into a Lagrangian one by first placing $n_s$ number of seeds in the domain uniformly at random, where $n_s$ is the spatial resolution of the vector field data, and integrate these seed points via RK4. The temporal frequency with which we store particle positions is set as the temporal resolution of the field. Furthermore, the Lagrangian representation is limited to the temporal duration on which we are evaluating, to have a better distribution of particles throughout the domain}.

Last, we compare our method with the streakline vector field (SVF) work of Weinkauf et al.~\cite{weinkauf2010streak}. \new{Specifically, the SVF is first precomputed by estimating flow map derivatives, computed via RK4, and then at runtime streaklines are generated by integrating the SVF. We view this as a fair comparison to our technique in that both approaches incur a precomputation cost, and thus we aim to compare the computation and storage requirement for the representations, as well as the accuracy and computation efficiency for generating streaklines.}

\begin{table}[]
\caption{We list all datasets and their respective sizes used in experiments.}
\label{tab:datasets}
\centering
\scalebox{0.8}{
% \begin{tabular}{|c|c|}
% \hline
% Dataset       & Res (t,x,y(,z)) \\ \hline
% Double Gyre   & 500x400x200     \\ \hline
% Cylinder      & 1001x400x50     \\ \hline
% Boussinesq    & 2001x450x150    \\ \hline
% Tornado       & 50x128x128x128  \\ \hline
% Scalar Flow   & 151x100x178x100 \\ \hline
% Half-Cylinder & 151x640x240x80  \\ \hline                 
% \end{tabular}}
% \end{table}
\begin{tabular}{cc}
\hline
Dataset       & Res {[}t,x,y(,z){]} \\ \hline
Double Gyre   & 500x400x200         \\
Cylinder      & 1001x400x50         \\
Boussinesq    & 2001x450x150        \\
Fluid Simulation & 1001x512x512     \\
Tornado       & 50x128x128x128      \\
Scalar Flow   & 151x100x178x100     \\
Half-Cylinder & 151x640x240x80   
\end{tabular}}
\end{table}

\begin{figure*}[t]
\centering
\includegraphics[width=1\linewidth]{figures/quantitative.pdf}
\caption{We show the quantitative evaluation of flow map approximation methods across different datasets, and across different time spans, beginning at start times for which flow features have largely resolved.} 
\label{fig:quantitative}
\end{figure*}


\subsection{Implementation details}
We first describe the details of our network architecture, followed by details on optimization.

\textbf{Network architecture settings}
The design of $f_{\nu}$ and $f_{\tau}$ rely on parameter settings related to the multi-level feature grid, as well as the MLP. The feature grids for $f_{\nu}$ and $f_{\tau}$ are of identical design, where we use a 4-level feature grid, and each level is of a different spatial resolution. Specifically, for a given axis of resolution $w$ at level $l$, we set the resolution at the next level to be $w^{s \cdot l}$, with resolution scaling factor $s$ set to 1.65, following the guidance of M{\"u}ller et al.~\cite{muller2022instant}. Each grid stores $8$-dimensional feature vectors at its nodes, and thus the resulting concatenated feature is $32$-dimensional. We employ 2 and 1-layer MLPs for $f_{\nu}$ and $f_{\tau}$, respectively, along with activation $\sigma_{\tau}$ chosen to be a Swish activation~\cite{hayou2018selection}. Experimentally we found Swish to outperform other more standard activations for INRs, e.g. ReLU, sin, consistent with findings in AutoInt~\cite{lindell2021autoint}. We control for the size of the network by a compression ratio, expressed as the ratio of the vector field size to the network size. We adjust the spatial resolution of the feature grids to best match a provided compression ratio, but leave the MLPs unchanged as they comprise a tiny portion of the model. Last, we use a 3-layer MLP with $64$ layer width for the residual network. Unless otherwise specified, we use a compression ratio of $10$ for all 2D datsets, and customize compression ratios for 3D as appropriate.
 
\begin{table}[]
\caption{We report the preprocessing times for different methods across 2D unsteady flows, along with corresponding timings for FTLE computation, varying time span and image resolution.}
\label{tab:time}
\centering
\scalebox{0.7}{%
    \begin{tabular}{|c|c|c|c|c|c|c|c|}
    \hline
    Dataset                      & FTLE res           & $\tau$ & \begin{tabular}[c]{@{}c@{}}Inference \\time(s)\end{tabular} & \begin{tabular}[c]{@{}c@{}}Preprocessing \\time(min)\end{tabular} & CR & \begin{tabular}[c]{@{}c@{}}Storage \\(MB)\end{tabular} &method       \\ \hline
    \multirow{4}{*}{Fluid Sim} & \multirow{4}{*}{512x512}  & \multirow{4}{*}{7}   & 21.161                                                 & -              & -       &   2003 & GT \\ \cline{4-8} 
                                      &                           &                      & \textbf{0.585}                                & 48.01             & 10     &   189   & NIFM         \\ \cline{4-8} 
                                      &                           &                      & 2.010                                        & 63.33              & 1       &              & Siren   \\ \cline{4-8} 
                                      &                           &                      & 4.701                                          & 1104.60           & -       &              & FSR \\ \hline
    \multirow{4}{*}{Cylinder}         & \multirow{4}{*}{1200x150}  & \multirow{4}{*}{1}   & 1.853                                         & -                & -         &     153     & GT \\ \cline{4-8} 
                                      &                           &                      &  \textbf{0.055}                              & 33.50              & 10         &     16     & NIFM         \\ \cline{4-8}
                                      &                           &                      & 0.554                                        & 74.26              & -           &          & ED   \\ \cline{4-8} 
                                      &                           &                      & 0.324                                        & 41.76              & 1            &        & Siren   \\ \cline{4-8} 
                                      &                           &                      & 29.94                                              & 0.04         & -             &       & Spline    \\ \hline
    \multirow{4}{*}{Boussinesq}       & \multirow{4}{*}{450x1350} & \multirow{4}{*}{0.5} & 2.220                                          & -                 & -            &  1030  & GT \\ \cline{4-8} 
                                      &                           &                      &  \textbf{0.079}                               & 37.25             & 10            &      97    & NIFM         \\ \cline{4-8} 
                                      &                           &                      & 0.938                                        & 122.89              & -             &        & ED   \\ \cline{4-8} 
                                      &                           &                      & 0.621                                       & 63.28                & 1       &             & Siren   \\ \cline{4-8} 
                                      &                           &                      & 91.57                                             & 0.15            & -       &       & Spline    \\ \hline
    \multirow{4}{*}{\begin{tabular}[c]{@{}l@{}}Double Gyre\end{tabular}}      & \multirow{4}{*}{1200x600}  & \multirow{4}{*}{10}  & 34.453                      & -        & -   &  611   & GT \\ \cline{4-8} 
                                      &                           &                      &  \textbf{1.020}                               & 34.80                 & 10        &      29     & NIFM         \\ \cline{4-8}
                                      &                           &                      & 6.278                                        & 40.70                  & -         &         & ED   \\ \cline{4-8}
                                      &                           &                      & 1.689                                        & 19.84                 & 1       &    & Siren   \\ \cline{4-8} 
                                      &                           &                      & 252.63                                             & 0.29            & -        &          & Spline    \\ \hline
    \end{tabular}%
}
\end{table}
 
\textbf{Optimization details}
For both phases of optimization we use Adam~\cite{kingma2015adam}, where we take a total of $40,000$ optimization steps and decay the learning rate every $8,000$ steps. Specific to optimization phase, in fitting to the vector field we use a learning rate of $0.02$, while for flow map optimization we use a learning rate of $0.01$ -- fitting the flow map derivative to the vector field is quite stable, and benefits from larger learning rates. In optimizing for the flow map, we have the choice of leaving the instantaneous velocity portion of the network frozen, or fine-tuning its weights to compensate for the remainder of the network. Although we find that both give results of comparable accuracy, in some occasions we found that fine-tuning can mitigate small grid-based artifacts in the output when leaving these weights frozen, and hence we fine-tune this portion of the network, using a learning rate of $0.0008$.

Recall that our method supports a maximum time span $\tau_{max}$ on which to sample during optimization. Though in principle we could optimize for the full time span of a given dataset, we find that performance can suffer, especially for datasets exhibiting complex temporal dynamics. Thus, as a compromise we set a limit on $\tau_{max}$ during optimization, and at inference time, for any target $\tau > \tau_{max}$ we take multiple steps with our network until reaching the desired span $\tau$. Specifically, for all 2D datasets, expressed in terms of grid units we set $\tau_{max} = 48$ unless otherwise specified. For 3D datasets we customize $\tau_{max}$ based on grid resolution, and complexity of the flows.

\begin{figure*}[t]
    \centering
    \includegraphics[width=1\linewidth]{figures/fluid_sim-compressed.pdf}
    \caption{We compare FTLE (top row) and integration error (bottom row) for two Fluid Simulation datasets (Re 16 and Re 101.6) across different baselines. The left column corresponds to particles integrated beginning at $t_0 = 0$ for duration $\tau=7$, while the right column corresponds to particles integrated starting at $t_0 = 2$ and $\tau = 7$.} 
    \label{fig:ftle_fluid}
\end{figure*}

\begin{figure*}[t]
    \centering
    \includegraphics[width=1\linewidth]{figures/dg-compressed.pdf}
    \caption{We compare FTLE (top row) and integration error (bottom row) for different baselines for the Double Gyre dataset. Particles are integrated from $t_0=0$ for a time-span $\tau=10$.} 
    \label{fig:ftle_dg}
\end{figure*}

\subsection{2D unsteady flow}

We first conduct experimental comparisons for various 2D time-varying flow fields. Specifically, we evaluate different techniques by computing the error in flow map approximations over varying seed points (spatial position and starting time) that have been integrated for varying time spans. We express error as the averaged Euclidean distance between the ground-truth flow map output, and the approximation scheme's output, normalized by the domain's bounding-box diagonal length. In Fig.~\ref{fig:quantitative} we present quantitative results comparing our method against different baselines, and in Table~\ref{tab:time} we report inference and preprocessing times. Specifically, for the pathline interpolation approach of Li et al.~\cite{li2022efficient}, preprocessing refers to the time required to fit B-splines, while for Jakob et al.~\cite{jakob2020fluid} this refers to the time required to optimize the CNN for super resolution. For all remaining methods, preprocessing refers to the time required for optimizing to an individual flow field.

\begin{figure}[!t]
    \centering
    \includegraphics[width=1.0\linewidth]{figures/cy_h.pdf}
    \caption{We show the FTLE (top of each pair) and error maps (bottom of each pair) for the flow over cylinder dataset generated by integrating particles starting at $t_0=18$ for a time-span $\tau=1$.} 
    \label{fig:ftle_cy}
\end{figure}

\begin{figure}[!t]
    \centering
    \includegraphics[width=1.0\linewidth]{figures/bo.pdf}
    \caption{We show the FTLE (top of each pair) and error maps (bottom of each pair) for the Boussinesq dataset generated by integrating particles starting at $t_0=11.3$ for a time-span $\tau=0.5$.} 
    \label{fig:ftle_bo}
\end{figure}


In comparing the fluid simulation flows of varying Reynolds numbers, we find that our method sees consistent improvement in accuracy over SIREN and super resolution, while achieving faster inference times. We note that the super resolution approach requires optimizing a CNN over a collection of flow maps just once, and thus can generalize to low-resolution flow maps at inference time, albeit restricted to flows resembling those observed during training. Our method is limited to just a single dataset at a time, but nevertheless, our training times scale well in terms of standard INRs (e.g. SIREN), while exhibiting faster inference and more accurate flow map approximations. Qualititative results for the fluid simulation flows are shown in Fig.~\ref{fig:ftle_fluid} in the form of the FTLE -- \new{computed using the method of Haller~\cite{haller2001lagrangian}} -- and color-encoded flow map errors. For high Reynolds number flows, we see that the super resolution method can fail to adapt to the rate at which particles separate, as indicated by the color shift, while also blurring out detailed ridges in the FTLE.  Our method, however, excels in capturing FTLE ridges, while remaining efficient to compute, since the super resolution method still requires computing a low-resolution flow map as input to a (otherwise highly efficient) CNN. Recall that our method employs a compression ratio of $10$ for all 2D experiments, which limits the grid resolution, and thus might limit the details we can reproduce in the flow map. However, from these results, we see that the coarser feature grid resolution does not limit the spatial resolution of the FTLE.


\begin{figure*}[t]
\centering
\includegraphics[width=1\linewidth]{figures/streaklines_new.pdf}
\caption{We compare our method's ability to compute streaklines against the streakline vector field technique~\cite{weinkauf2010streak}, which only necessitates integrating a derived vector field. Qualitatively and quantitatively we find that our method produces comparable results, where we show varying step sizes used for evaluating the flow map.}
\label{fig:streaklines}
\end{figure*}

In comparing our method to other baselines (c.f. Fig.~\ref{fig:quantitative}) for Double Gyre, Cylinder, and Boussinesq, we find that our method obtains higher accuracy in relation to other techniques. Prior INR methods such as the encoder-decoder architecture of Han et al.~\cite{han2021exploratory}, or a pure coordinate-based approach~\cite{sitzmann2020implicit} poorly generalize. We find that for small step sizes, the performance of these methods in fact steeply declines, as numerical error accumulates with the more steps taken. We attribute this to the basic limitations of the network architectures employed, failing to address the properties (identity mapping, instantaneous velocity) we target in our network design. The inability to generalize in these methods is further demonstrated qualitatively for Figs.~\ref{fig:ftle_dg} - \ref{fig:ftle_bo}. \new{Pathline interpolation~\cite{li2022efficient} is notable in its small precomputation cost. Nevertheless, the method is less accurate in preserving the flow map, while incurring a high computation cost at runtime.}

We additionally evaluate our technique both quantitatively and qualitatively for the computation of streaklines. In Fig.~\ref{fig:streaklines} we show streaklines for the Cylinder dataset. We compare our method with SVF~\cite{weinkauf2010streak}. We can see that both the techniques are able to capture the vortices of the dataset faithfully, and are visually indistinguishable from the ground truth streaklines. Quantitatively both the techniques consistently incur low streakline error staying within the margin of $10^{-3}$ magnitude (relative to the bounding box diagonal). Interestingly, we find that both methods have comparable inference time as well, as reported in Table~\ref{tab:streaklines}, despite the fact the streakline vector field evaluates its field fewer times than our neural flow map, since we must take multiple steps for sufficiently long time spans. However, an advantage of our method lies in data parallelism; we can evaluate the flow map over varying space/time/duration in a single batch, whereas integrating the streakline vector field is, by necessity, a sequential process. We further note that SVF precomputation is quite expensive, both in terms of speed and storage space. In Table~\ref{tab:streaklines} we can see that the computation of the entire 4D SVF has very large storage requirements (160GB), whereas our method is in proportion to the size of the vector field (77MB). We note that while our technique can be easily scaled to 3D datasets, SVF preprocessing for 3D unsteady flows is infeasible in practice, necessitating a 5D grid for storage.


\begin{table}[!t]
\caption{We report storage requirements, preprocessing time and inference time for computing streaklines on the Cylinder dataset, comparing our method against the streakline vector field technique~\cite{weinkauf2010streak}.}
\label{tab:streaklines}
\centering
\scalebox{0.9}{
\begin{tabular}{|c|c|c|c|}
\hline
Method  & \begin{tabular}[c]{@{}c@{}}Preprocessing\\ Time\\ (min)\end{tabular} & \begin{tabular}[c]{@{}c@{}}Inference\\ Time\\ (sec)\end{tabular} & \begin{tabular}[c]{@{}c@{}}Storage\\ \end{tabular} \\ \hline
Ground Truth & NA  & 21.391 & 160.20 MB \\  \hline
SVF & 130.407 & 1.204 & 160.36 GB \\ \hline
NIFM (16 grid steps) & \multirow{2}{*}{40.060} & 0.952 & \multirow{2}{*}{77.20 MB}\\ \cline{1-1} \cline{3-3}
NIFM (24 grid steps) & &0.671 & \\ \hline
\end{tabular}}
\end{table}

\begin{table}[!t]
\caption{We report the processing times as well the FTLE computation times for different method across different 3D unsteady flow datasets.}
\label{tab:3d_ftle_times}
\centering
\scalebox{0.8}{
\begin{tabular}{|c|c|c|c|c|c|c|}
\hline
Dataset                                        & FTLE res                     & $\tau$                                     & \multicolumn{1}{l|}{\begin{tabular}[c]{@{}l@{}}Inference\\ times (s)\end{tabular}} & \multicolumn{1}{l|}{\begin{tabular}[c]{@{}l@{}}Processing\\ times(m)\end{tabular}} & CR & \multicolumn{1}{l|}{Method} \\ \hline
\multicolumn{1}{|c|}{\multirow{4}{*}{Tornado}} & \multirow{4}{*}{128x128x128} & \multirow{4}{*}{50}                     & 27.16                                                                              & -                                                                                 & -  & GT                          \\ \cline{4-7} 
\multicolumn{1}{|c|}{}                         &                              &                                         & \textbf{3.60}                                                                               & 35.55                                                                       & 10        & NIFM                        \\ \cline{4-7} 
\multicolumn{1}{|c|}{}                         &                              &                                         & 14.14                                                                              & 93.21                                                                           & 10   & SIREN                       \\ \cline{4-7} 
\multicolumn{1}{|c|}{}                         &                              &                                         & 286.29                                                                             & 0.87
& - & Spline                      \\ \hline
\multirow{4}{*}{Scalar Flow}                   & \multirow{4}{*}{100x178x100} & \multirow{4}{*}{2.5}                    & 81.72                                                                              & -                                                                              & -     & GT                          \\ \cline{4-7} 
                                               &                              &                                         & \textbf{2.55}                                                                               & 41.66                                                                          & 10     & NIFM                        \\ \cline{4-7} 
                                               &                              &                                         & 21.48                                                                              & 95.57                                                                      & 10         & SIREN                       \\ \cline{4-7} 
                                               &                              &                                         & 291.39                                                                             & 0.81                                                                        & -        & Spline                      \\ \hline
\multirow{3}{*}{Half-Cylinder}                 & \multirow{3}{*}{640x240x80}  & \multicolumn{1}{c|}{\multirow{3}{*}{2}} & 137.41                                                                             & -                                                                                & -   & GT                          \\ \cline{4-7} 
                                               &                              & \multicolumn{1}{c|}{}                   & \textbf{3.82}                                                                               & 45.56                                                                       & 40        & NIFM                        \\ \cline{4-7} 
                                               &                              & \multicolumn{1}{c|}{}                   & 53.52                                                                              & 103.13                                                                        & 40      & SIREN                       \\ \hline
\end{tabular}}
\end{table}

\begin{figure*}[t]
    \centering
    \includegraphics[width=0.9\linewidth]{figures/3d_results.pdf}
    \caption{We compare, both qualitatively (volume rendering of FTLE field) and quantitatively (flow map evaluation), our method with standard coordinate-based networks~\cite{sitzmann2020implicit} as well as pathline interpolation techniques~\cite{li2022efficient} for modeling the flow map in 3D unsteady flows. We find our method is quantitatively an improvement over other methods, and qualitatively our method contains fewer visual artifacts.}
    \label{fig:ftle_scalar}
\end{figure*}

\begin{figure*}[t]
    \centering
    \includegraphics[width=0.9\linewidth]{figures/half_cylinder_v2.pdf}
    \caption{In this figure, we compare our method both quantitatively and qualitatively against SIREN for the Half-Cylinder dataset. We find that our method is able to scale reasonably well to this large dataset, whereas, the SIREN fails to learn meaningful flow maps as can be seen from the FTLE.}
    \label{fig:ftle_half_cylinder}
\end{figure*}

\subsection{3D unsteady flow}

We next evaluate our method on a set of 3D unsteady flows, comparing our method with a SIREN-based flow map~\cite{sitzmann2020implicit} as well as the B-spline pathline interpolation technique~\cite{li2022efficient}. We first compare to the Tornado and Scalar Flow datasets, where we set the $\tau_{max}$ to $8$ and $24$, respectively, to match the temporal complexity in the flows. Fig.~\ref{fig:ftle_scalar} shows qualitative results, via volume-rendering of the FTLE, as well as quantitative results. Our method is an improvement, if not comparable, to prior methods, but we obtain significant gains in inference time, as reported in Table~\ref{tab:3d_ftle_times}. We further compare to the Half Cylinder dataset, a large-scale unsteady flow dataset that cannot be readily stored in memory. We found the pathline interpolation method~\cite{li2022efficient} failed to fit to the data, and thus we limit our comparison to SIREN, please see Fig.~\ref{fig:ftle_half_cylinder}. In this experiment we set $\tau_{max} = 8$ and the compression ratio to $40$ to compensate for the larger data size. We find our method captures turbulent features in the wake of the half-cylinder object ($Re=320$), whereas SIREN faces difficulties in accurately modeling the data. Notably, for this dataset we find our training scheme scales well (c.f. Table~\ref{tab:3d_ftle_times}) relative to the 2D unsteady flow datasets, whereas SIREN's increase in model size leads to slower training times.

\vspace{-.9em}

\new{
\subsection{Error analysis: numerical integration}
\label{subsec:error}
Our method can be viewed as a novel technique for integrating a vector field, and thus, it is worth asking: how does our method compare to conventional numerical integration schemes? To help answer this question, we compare NIFM to existing numerical schemes, namely Euler and RK4, evaluated under varying step sizes. For the purpose of evaluation we use the Sine Ridge dataset provided by Kuhn et al.~\cite{kuhn2012benchmark} - as this is a steady flow we adapt our method accordingly. The dataset has an analytically-defined flow map that allows us to compute the flow map error across different schemes. In Fig.~\ref{fig:analytical}, we show the FTLE (first row) and the flow map error (second row) for Euler, RK4, and NIFM. The FTLE is computed for a duration $\tau=1.2$ with step size set to 30, where a single step amounts to 0.01 in the physical domain. We can see that NIFM best captures the FTLE, while maintaining low error in the flow map, in contrast with Euler and RK4. This provides evidence that our method is not merely a fixed linear (e.g. Euler), or higher-order (e.g. RK4) integration scheme, but rather adapts to the features of the data. We further show quantitative results for duration 0.6 and 1.2, again varying the step size. We can see that while NIFM has a consistent performance across all step sizes, the flow map error increases significantly for both RK4 and Euler with increasing step size.
}
\begin{figure}[!t]
    \centering
    \includegraphics[width=1.0\linewidth]{figures/analytical_example.pdf}
    \caption{We compare NIFM to Euler and RK4 integration schemes, showing FTLE (top), flow map error (middle), and quantitative evaluation (bottom). We find NIFM performs consistently well across step sizes as compared to Euler and RK4 which can become numerically unstable.} 
    \label{fig:analytical}
\end{figure}

\subsection{Ablation: compression and supervision}

Last, we run model ablations to study the effects of various design choices. Due to space limitations we limit ablation to compression, as well as the role of supervision in learning flow maps. Further experiments regarding the architecture choices (number of levels in the multiresolution grid) and optimization scheme (number of steps to take, c.f. Eq.~\ref{eq:steps}) are detailed in the appendix.

In Fig.~\ref{fig:grid_artifact} we show the results of our model, for the FTLE of the Boussinesq, optimized under varying compression ratios. In this experiment we specifically wish to study how compression might impart visual artifacts in derived quantities of the flow map approximation, as a higher level of compression results in coarser feature grids. Indeed, we find that lower levels of compression lead to fewer grid-like artifacts in the resulting FTLE when taking a smaller steps, e.g. in this setting, a step size of 48 grid units in time amounts to an evaluation of the model just 3 times per position. We further report inference times for the smallest and largest level of compressions, and as expected, a larger number of steps requires longer inference times (e.g. more feedforward passes with the network). Interestingly, we find the inference time is fairly consistent across these compression ratios, indicating that the increased resolution of the grid has a negligible impact on this matter. As detailed in the appendix, we also find that the flow map accuracy takes just a small hit in performance across compression ratios, indicating that flow map accuracy might not be predictive of visual artifacts in derived quantities. Nevertheless, as shown in the figure, training times come at a cost with smaller compression ratios. We thus see natural trade-offs in the (1) flow map quality, (2) inference time (hinging on step size), and (3) training time.

\begin{figure}[!t]
    \centering
    \includegraphics[width=0.9\linewidth]{figures/grid-artifact.pdf}
    \caption{We qualitatively compare our model under varying compression ratios, showing the effect of compression on the step size taken by our model to produce the FTLE for the Boussinesq flow.} 
    \label{fig:grid_artifact}
\end{figure}

\begin{figure}[!t]
    \centering
    \includegraphics[width=1\linewidth]{figures/flowmap_vs_self.pdf}
    \caption{For the Boussinesq flow we compare our self-consistency criterion with that of directly supervising on flow samples, finding that our method produces comparable, if not improved, flow map approximations, without ever accessing the ground-truth flow map.} 
    \label{fig:self_consistency_exp}
\end{figure}

Our choice to learn flow maps via a self-supervisory signal is in contrast with how numerous visualization techniques interpolate~\cite{chandler2014interpolation,li2022efficient}, or build models~\cite{han2021exploratory} given samples of the flow map, e.g. typically as densely-sampled pathlines. Therefore we ask: is our self-consistency criterion an inferior objective to directly supervising on flow map samples? To this end, we have gathered a large collection of flow map samples, and modified our objective (Eq.~\ref{eq:composition-detail}) to accept the ground-truth flow map, and its corresponding derivative at the output position. We optimize for Boussinesq, using 20M and 50M flow map samples, and compare with our proposed objective, please see Fig.~\ref{fig:self_consistency_exp} for the results. We find that our method is able to learn comparable, if not better, flow map approximations, without ever observing flow map samples. In particular, at 50M samples we find that flow map supervision starts to become competitive with our method. Although supervising an on even larger number of samples might be more beneficial, clearly the data requirement starts to become prohibitively expensive, both for integrating the flow field, as well as storage requirements. In contrast, our method avoids these issues by requiring the vector field as the only supervision.



\section{Ablation Study}
\begin{table*}[t]
\setlength{\tabcolsep}{1.5mm}
\centering
\small
\resizebox{\textwidth}{!}{%
    \begin{tabular}{l|ccccc|ccc|cc|c|c}
        \toprule[1pt]
        & \multicolumn{5}{c|}{\textbf{Natural Language Inference}} & \multicolumn{3}{|c|}{\textbf{Sentence Completion}} & \multicolumn{2}{c|}{\textbf{Coreference}} & \multicolumn{1}{c|}{\textbf{WSD}} & \multirow{2}{*}{Avg.} \\
    & RTE & CB & ANLI1 & ANLI2 & ANLI3 & COPA & Hella. & Story. & WSC & Wino. & WiC &  \\
    \midrule[1pt]
    UD (Minimal)     & \textbf{83.75}
        & \textbf{80.36}
        & 36.80
        & \textbf{34.20}
        & \textbf{42.17}
        & \textbf{90.00}
        & \textbf{56.07}
        & \textbf{96.37}
        & \textbf{68.27}
        & \textbf{62.90}
        & \textbf{54.55}	
        & \textbf{64.13} \\
    UD (Instructive)    & 72.24 
        & 64.52 
        & \textbf{36.98} 
        & 33.40 
        & 39.73 
        & 85.31 
        & 45.15 
        & 96.01 
        & 65.38 
        & 53.94 
        & 50.94 
        & 58.51\\
    \midrule
    T0 (Minimal) & 61.56  & \textbf{57.81}  & 30.57  & 30.27  & 33.38  & 67.19  & \textbf{33.81}  & 66.56  & 60.94  & 52.81  & \textbf{51.72}  & 49.69  \\
    T0 (Instructive) & \textbf{75.05}	& 55.48	& \textbf{32.87}	& \textbf{32.29}	& \textbf{33.67}	& \textbf{84.59}	& 28.24	& \textbf{93.97}	& \textbf{62.98}	& \textbf{54.59}	& 51.16	& \textbf{54.99} \\

    \bottomrule[1pt]
    \end{tabular}}
    \caption{Zero-shot performance for UD and T0 respectively with instructive and minimal prompts. Instructive prompts are lengthy descriptions of tasks \citep{T0-paper}, while minimal prompts use a simple concatenation of input data.}
\label{tab:promptablatiion}
\end{table*}


\section{Conclusion}
\section{Conclusion}
In this paper, we extend the idea of SynGEC \cite{zhang2022syngec} and propose the CSynGEC approach to enhance GEC models by exploiting tailored constituent-based syntax. Experimental results show that incorporating constituent-based syntax produced by a GEC-oriented constituency parser can effectively help GEC models. 
Furthermore, we attempt to combine dependency-based and constituent-based syntax from both intra-model and inter-model aspects, and find that simultaneously using two kinds of syntax leads to more obvious improvement.




\bibliographystyle{acl_natbib}
\bibliography{anthology,acl2021,custom}

\newpage
\section*{Appendix}
\appendix

\section{Supplemental Tables}

%\section{Hyperparameters of Other Bandit Algorithms}
%\label{sec:bandit_hyperparams}
%Table~\ref{tab:hyperparams} lists the hyperparameters for bandit algorithms other than dBE.

\newcommand\topmidheader[2]{\multicolumn{#1}{c}{\textbf{#2}}\\%
                \addlinespace[1ex]}

\newcommand{\midheader}[2]{%
        \midrule\topmidheader{#1}{#2}}

\newcommand{\specialcell}[3][c]{% 
        \begin{tabular}[#1]{@{}#2@{}}#3\end{tabular}}%

\aptLtoX[graphic=no,type=env]{\begin{table}[htb]
  \centering
  \caption{Hyperparameters of bandit algorithms}
  \label{tab:hyperparams}
  \begin{tabular}{llc}
    \toprule
    Sign & Description & Value \\
    \multicolumn{3}{c}{\textbf{UCB1}}\\
    $c$ & Parameter to control the confidence level used in $\sqrt{c \cdot {\log{t}}/{N_t(arm)}}$ & 0.5  \\
    \multicolumn{3}{c}{\textbf{Thompson Sampling}}\\
    $p(\theta)$ & Prior Distribution & $\mathcal{B}(1, 1)$ \\
    \multicolumn{3}{c}{\textbf{discounted Thompson Sampling}}\\
    $\gamma$ & Discount factor & $1-10^{-8}$ \\
    \multicolumn{3}{c}{\textbf{discounted Thompson Samplingadaptive shrinking Thompson Sampling}}\\
    $M$ & Parameter to control memory usage in a data structure ADWIN2 \cite{ADWIN} & 10 \\
    $\delta$ & Parameter to control the confidence level in a data structure ADWIN2 & $1-10^{-7}$ \\
    \multicolumn{3}{c}{\textbf{EXP-IX}}\\
    $\eta_t$ & Parameter used for weights of arms & $\sqrt{\frac{2 \cdot \log{K}}{K \cdot t}}$ \\
    \addlinespace[1ex]
    $\gamma_t$ & Parameter used for loss estimates & $\frac{\eta_t}{2}$ \\
    \multicolumn{3}{c}{\textbf{EXP3++}}\\
    $\alpha$ & Constant used in calculating $\xi_t(a)$ & $3$ \\
    $\beta$ & Constant used in calculating $\xi_t(a)$ & $256$ \\
    \bottomrule
  \end{tabular}
\end{table}}{\begin{table}[htb]
  \centering
  \caption{Hyperparameters of bandit algorithms}
  \label{tab:hyperparams}
  \begin{tabular}{llc}
    \toprule
    Sign & Description & Value \\
    \midheader{3}{UCB1}
    $c$ & \specialcell{l}{Parameter to control the confidence \\ level used in $\sqrt{c \cdot {\log{t}}/{N_t(arm)}}$} & 0.5  \\
    \midheader{3}{Thompson Sampling}
    $p(\theta)$ & Prior Distribution & $\mathcal{B}(1, 1)$ \\
    \midheader{3}{discounted Thompson Sampling}
    $\gamma$ & Discount factor & $1-10^{-8}$ \\
    \midheader{3}{adaptive shrinking Thompson Sampling}
    $M$ & \specialcell{l}{Parameter to control memory usage \\ in a data structure ADWIN2 \cite{ADWIN}} & 10 \\
    $\delta$ & \specialcell{l}{ Parameter to control the confidence \\ level in a data structure ADWIN2} & $1-10^{-7}$ \\
    \midheader{3}{EXP-IX}
    $\eta_t$ & Parameter used for weights of arms & $\sqrt{\frac{2 \cdot \log{K}}{K \cdot t}}$ \\
    \addlinespace[1ex]
    $\gamma_t$ & Parameter used for loss estimates & $\frac{\eta_t}{2}$ \\
    \midheader{3}{EXP3++}
    $\alpha$ & Constant used in calculating $\xi_t(a)$ & $3$ \\
    $\beta$ & Constant used in calculating $\xi_t(a)$ & $256$ \\
    \bottomrule
  \end{tabular}
\end{table}}

\begin{table}[htb]
  \centering
  \caption{Commit IDs of the PUTs used in our vulnerability discovery and AFL++ used as the baseline.}
  \begin{tabular}{lc}
    \toprule
    Program & Commit \\
    \midrule

    AFL++ & 32a0d6ac315 (ver ++3.14c) \\
    Bloaty &  60209eb \\
    HarfBuzz & 77eeec5 \\
    libarchive & 86c9361 \\
       libxml2 & dea91c9 \\
    MuPDF & ef3d68d \\
   PHP & fdf0455f \\
    Poppler & 6d72d82 \\
    PROJ & 76dfefe \\
    QPDF &  3794f8e \\
    libtpm2 & bc3bb26 \\
    Wireshark  & 1fc621e \\
    Xpdf & N/A (ver 4.03) \\

    \bottomrule
  \end{tabular}
\label{tab:commit-ids}
\end{table}


\begin{table}[htb]
  \centering
  \caption{Initial and theoretical maximum values of code coverage of the PUTs in FuzzBench. 
           Initial values were investigated only in the PUTs used.}
  \begin{tabular}{lcc}
    \toprule
    PUT & Initial & Maximum \\
    \midrule

bloaty\_fuzz\_target & N/A & 83114 \\
curl\_curl\_fuzzer\_http & N/A & 78362 \\
freetype2-2017 & 1517 & 26262 \\
harfbuzz-1.3.2 & N/A & 12212 \\
jsoncpp\_jsoncpp\_fuzzer & N/A & 2114 \\
lcms-2017-03-21 & 149 & 7036 \\
libjpeg-turbo-07-2017 & N/A & 9384 \\
libpcap\_fuzz\_both & 2 & 7294 \\
libpng-1.2.56 & 138 & 3736 \\
libxml2-v2.9.2 & 258 & 67994 \\
libxslt\_xpath & N/A & 51456 \\
mbedtls\_fuzz\_dtlsclient & N/A & 12888 \\
openssl\_x509 & 6026 & 54116 \\
openthread-2019-12-23 & N/A & 19846 \\
php\_php-fuzz-parser & N/A & 215210 \\
proj4-2017-08-14 & 46 & 6534 \\
re2-2014-12-09 & 1 & 3982 \\
sqlite3\_ossfuzz & 4767 & 28766 \\
systemd\_fuzz-link-parser & N/A & 1798 \\
vorbis-2017-12-11 & 410 & 4082 \\
woff2-2016-05-06 & N/A & 5708 \\
zlib\_zlib\_uncompress\_fuzzer & N/A & 910 \\

    \bottomrule
  \end{tabular}
\label{tab:fuzzbench_max_cov}
\end{table}

\begin{table}[htb]
\centering
\caption{List of unique bugs found in the 7-day trial (manually triaged).}
\begin{minipage}{\columnwidth}

\centering
\begin{tabular}{lll}
\toprule

ID & PUT & Bug Type \\
\midrule
Bug-A & bloaty & NULL Pointer Deref \\
Bug-B & harfbuzz & Out-of-bounds Read \\
Bug-C & mupdf & Assertion Fail \\
Bug-D & mupdf & NULL pointer deref \\
Bug-E & xpdf & Stack Overflow \\
Bug-F & xpdf & NULL Pointer Deref \\
Bug-G \footnote{CVE-2022-24106 is issued.} & xpdf & Use of Uninitialized Value \\
Bug-H \footnote{CVE-2022-24107 is issued.} & xpdf & Integer Overflow \\
Bug-I & php & Use-After-Free \\
Bug-J & php & Use-After-Free \\
Bug-K & php & NULL Pointer Deref \\
Bug-L & php & Use-After-Free \\ 
Bug-M & php & NULL Pointer Deref \\
Bug-N & php & Assertion Fail \\
Bug-O & php & Use-After-Free \\
Bug-P & php & Use-After-Free \\
Bug-Q \footnote{CVE-2022-23308 is issued.} & libxml2 & Use-After-Free \\
\bottomrule
\end{tabular}

\label{tab:7d-bug}
\end{minipage}
\end{table}

\begin{table*}[htb]
  \centering
  \caption{List of the PUTs used in Section~\ref{sec:banditcomparison}. If the source code of a PUT was maintained in Git, the latest version at the time of the experiment in the master (or main) branch was used for the build. The `+' sign in a version indicates that the used source code is not the official release version of the source code.}
  \renewcommand\tabularxcolumn[1]{m{#1}}
  \renewcommand{\arraystretch}{1.2}
  \begin{tabularx}{\textwidth}{lXllXc}
    \toprule
    Project & Version & Commit ID & PUT & Format of Initial Seeds & Initial Edge Coverage \\
    \midrule
    Bloaty & v1.1+ & 60209eb & fuzz\_target & Executable (e.g., ELF, PE, Mach-O) & 4773\\
    libmpeg2 & N/A & 5432dc1 & mpeg2\_dec\_fuzzer & MPEG2 & 2428 \\
    PHP & 8.0+ & fdf0455f & php-fuzz-execute & PHP source code & 25241 \\
    HarfBuzz & 3.1.0 & 77eeec5 & hb-shape-fuzzer & Font (e.g., TrueType, OpenType) & 15298 \\
    Xpdf & 4.03 & N/A & fuzz\_pdfload & PDF & 4755 \\
    libtpm2 & N/A & bc3bb26 & tpm2\_execute\_command\_fuzzer & TPM command & 3884\\
    libyaml & v0.2.5+ & f8f760f & libyaml\_dumper\_fuzzer & YAML & 1310 \\
    libzip & 1.8.0+ & bff2eb9 & zip\_read\_fuzzer & ZIP & 805 \\
    libgit2 & v1.3.0+ & 50b4d53 & download\_refs\_fuzzer & Git packet & 3911 \\
    file & 5.41+ & fcbb5d8 & magic\_fuzzer & any (e.g., Zstd compressed file) & 1171 \\
%    MuPDF & 1.19.0+ & ef3d68d & pdf\_fuzzer & PDF & 16936 \\
%    libxml2 & 2.9.12+ & dea91c9 & xml & XML & 7027 \\
    \bottomrule
  \end{tabularx}
\label{tab:put_details}
\end{table*}

%\section{Full Results of Some Experiments}
%\label{sec:full_result}

%Table~\ref{tab:alg_cmp_all}, Figure \ref{fig:vis_bandits} and Figure \ref{fig:full_ablation_time_vs_cov} show the omitted results.

\begin{table*}[htb]
\centering
\caption{Median edge coverage obtained by AFL++ and 8 versions of \OurMethodName-AFL++ in 10 PUTs after 24 h. }

\begin{tabular}{lccccccccc}
\toprule

PUT & AFL++ & UCB1 & KLUCB & TS & dTS & dBE & ADS-TS & EXP3-IX & EXP3++ \\
\midrule

bloaty & \textit{1845.5} & 2198.5 & 2246.0 & 2232.5 & 2191.0 & 2292.0 & \textbf{2340.0} & 2181.5 & 2231.5 \\
harfbuzz & \textit{13497.5} & 14031.5 & 14247.5 & 14360.5 & \textbf{14374.0} & 14067.5 & 14149.0 & 13883.0 & 13891.0 \\
xpdf & \textit{3384.0} & 3494.0 & 3812.5 & \textbf{4618.5} & 4166.5 & 3791.5 & 3902.0 & 3860.0 & 3615.0 \\
libzip & \textit{267.5} & 272.0 & 274.0 & 268.0 & 268.5 & 271.5 & \textbf{276.0} & 271.5 & 268.0 \\
libgit2 & 898.0 & 888.5 & 890.5 & 906.5 & \textbf{916.0} & 884.0 & 914.0 & 899.5 & \textit{881.0} \\
php & \textit{9841.5} & 11861.0 & 13551.5 & \textbf{14324.0} & 14187.5 & 12657.5 & 13408.0 & 11423.5 & 11828.5 \\
libmpeg2 & \textit{1873.5} & 1900.5 & 1905.0 & 1905.5 & \textbf{1906.5} & 1903.0 & \textbf{1906.5} & 1897.0 & 1902.0 \\
tpm2 & \textit{281.5} & 299.5 & 313.0 & 317.0 & \textbf{317.5} & 305.0 & 311.0 & 298.5 & 291.0 \\
libyaml & 2811.5 & 2841.0 & \textbf{2841.5} & \textit{2800.5} & 2837.0 & 2827.5 & 2831.5 & 2828.0 & 2834.5 \\
file & 830.5 & 829.5 & 828.0 & 827.0 & 827.5 & 833.5 & \textbf{840.5} & 826.5 & \textit{826.0} \\

\bottomrule

\end{tabular}

\label{tab:alg_cmp_all}
\end{table*}

\begin{table*}[htb]
\centering
\caption{P-value of Mann-Whitney's U test (Holm-Bonferroni corrected) and Vargha-Delaney's $\hat{A}_{12}$ between AFL++ and the fuzzer in the column for the evaluation conducted in Section~\ref{subsec:eval-vs-existing}. If the p-value is bold, the difference is significant in the test ($p < 0.01$). The characters `L', `M', `S' and `N' in parentheses indicate that the effect size is large, medium, small, and none, respectively, according to \cite{A12}. The `+' sign means the fuzzer in the column is superior to AFL++ when compared by rank sum as well as $\hat{A}_{12}$, and the `-' sign means the opposite.}
\begin{tabular}{lllllllllllll}
 \toprule

  & \multicolumn{2}{c}{MOpt} & \multicolumn{2}{c}{CMFuzz} & \multicolumn{2}{c}{Karamcheti} & \multicolumn{2}{c}{\HavocMAB{}} & \multicolumn{2}{c}{SLOPT} \\
  \cmidrule(r){2-3}\cmidrule(r){4-5}\cmidrule(r){6-7} \cmidrule(r){8-9} \cmidrule(r){10-11}
  PUT & $p$ & $\hat{A}_{12}$ & $p$ & $\hat{A}_{12}$ & $p$ & $\hat{A}_{12}$ & $p$ & $\hat{A}_{12}$ & $p$ & $\hat{A}_{12}$ \\
\midrule

openssl\_x509 & \textbf{ < 0.001 } & 0.82 (+L) & \textbf{ 0.023 } & 0.71 (+L) & \textbf{ < 0.001 } & 0.92 (+L) & \textbf{ < 0.001 } & 0.82 (+L) & \textbf{ < 0.001 } & 0.91 (+L) \\
re2-2014-12-09 & \textbf{ < 0.001 } & 0.18 (-L) & > 0.1 & 0.37 (-S) & > 0.1 & 0.38 (-S) & > 0.1 & 0.47 (-N) & > 0.1 & 0.52 (+N) \\
proj4-2017-08-14 & \textbf{ < 0.001 } & 0.08 (-L) & \textbf{ < 0.001 } & 0.86 (+L) & \textbf{ < 0.001 } & 0.99 (+L) & > 0.1 & 0.54 (+N) & \textbf{ < 0.001 } & 0.92 (+L) \\
sqlite3\_ossfuzz & > 0.1 & 0.55 (+N) & \textbf{ < 0.001 } & 0.85 (+L) & \textbf{ < 0.001 } & 0.93 (+L) & 0.1 & 0.68 (+M) & \textbf{ < 0.001 } & 1.00 (+L) \\
libxml2-v2.9.2 & \textbf{ < 0.001 } & 0.08 (-L) & \textbf{ < 0.001 } & 0.93 (+L) & \textbf{ < 0.001 } & 0.98 (+L) & \textbf{ < 0.001 } & 0.97 (+L) & \textbf{ < 0.001 } & 0.84 (+L) \\
freetype2-2017 & \textbf{ < 0.001 } & 0.08 (-L) & 0.094 & 0.33 (-M) & > 0.1 & 0.54 (+N) & > 0.1 & 0.52 (+N) & \textbf{ < 0.001 } & 0.79 (+L) \\
libpcap\_fuzz\_both & > 0.1 & 0.57 (+S) & \textbf{ < 0.001 } & 0.79 (+L) & \textbf{ < 0.001 } & 0.80 (+L) & \textbf{ < 0.001 } & 0.87 (+L) & \textbf{ < 0.001 } & 0.81 (+L) \\
libpng-1.2.56 & > 0.1 & 0.42 (-S) & > 0.1 & 0.36 (-M) & > 0.1 & 0.49 (-N) & > 0.1 & 0.56 (+S) & 0.049 & 0.68 (+M) \\
lcms-2017-03-21 & > 0.1 & 0.45 (-N) & \textbf{ 0.037 } & 0.70 (+M) & \textbf{ < 0.001 } & 0.85 (+L) & > 0.1 & 0.37 (-S) & \textbf{ < 0.001 } & 0.88 (+L) \\
vorbis-2017-12-11 & > 0.1 & 0.39 (-S) & > 0.1 & 0.56 (+S) & \textbf{ < 0.001 } & 0.20 (-L) & > 0.1 & 0.62 (+S) & 0.092 & 0.65 (+M) \\

\bottomrule
\end{tabular}
\label{tab:statistics}
\end{table*}

\clearpage

\section{Algorithm Overview}

\begin{algorithm}[H]

\centering
\caption{Pseudocode of \OurMethodName{}}
\label{alg:slopt}

\begin{algorithmic}[0]

\Require{\mbox{}\\
    $initial\_seeds$ -- a set of initial test cases \\
    $program$ -- a PUT to be fuzzed
}

\Ensure{\mbox{}\\
    $queue$ -- a set of valuable test cases \\
    $crashes$ -- a set of test cases that trigger crashes
}

%\begin{adjustwidth}{-9pt}{}
%\setstretch{0.85}
\vspace{5pt}

\Function{RandomMutation}{$seed, instance_{mut}, instances_{bat}$}
\State $input$ $\gets$ \Call{CopyBytesFromSeed}{$seed$}
\State $mutation$ $\gets$ \Call{SelectArm}{$instance_{mut}$}
\State $idx$ $\gets$ \Call{GetGroupIndex}{$len(input)$}
\State $batch\_size$ $\gets$ \Call{SelectArm}{$instances_{bat}[idx][mutation]$}
\For{$i$ $\gets$ $1$ \textbf{to} $batch\_size$}
    \State $pos$ $\gets$ \Call{SelectPosition}{$input$}
    \State $input$ $\gets$ \Call{ApplyOperator}{$mutation, input, pos$}
\EndFor
\State \textbf{return} $input, mutation, batch\_size$
\EndFunction

%\end{adjustwidth}

%\vspace{-6pt}

%\begin{adjustwidth}{-9pt}{}
%\setstretch{0.85}

\vspace{5pt}

\Function{MutationFuzzing}{$initial\_seeds, program$}

\State $crashes$ $\gets$ $\varnothing$
\State $queue$ $\gets$ \Call{ConstructQueue}{$initial\_seeds$}
\State $instance_{mut}$ $\gets$ \Call{CreateBanditArms}{$number\_of\_mutations$}
\For{$i$ $\gets$ $1$ \textbf{to} $5$}
 \For{$j$ $\gets$ $1$ \textbf{to} $number\_of\_mutations$}
  \State $instances_{bat}[i][j]$ $\gets$ \Call{CreateBanditInstance}{$7$}
 \EndFor
\EndFor

\State

\While{ $\neg$ \Call{UserWantsStop}{\null}}
 \State $seed$ $\gets$ \Call{SelectSeed}{$queue$}
 \State $energy$ $\gets$ \Call{DecideEnergy}{$seed$}
 \For{$i$ $\gets$ $1$ \textbf{to} $energy$}
  \State $input, mutation, batch\_size$ 
  \State $\gets$ \Call{RandomMutation}{$seed, instance_{mut}, instances_{bat}$}
  \State $result$ $\gets$ \Call{ExecutePUT}{$program, input$}
  \State $b$ $\gets$ \Call{WasInputValuable}{$result$}
  \State \Call{RewardArm}{$mutation, b$}
  \State \Call{RewardArm}{$batch\_size, b$}
  \State \Call{SaveInputIfValuable}{$queue, input, result$}
  \State \Call{SaveInputIfCrash}{$crashes, input, result$}
 \EndFor
\EndWhile
\EndFunction

%\end{adjustwidth}

\end{algorithmic}
\end{algorithm}



%\section{Example Appendix}
%\label{sec:appendix}

%This is an appendix.

\end{document}
