
\subsection*{Data preprocessing}
We follow the training data preprocessing of \citet{bianchi-etal-2021-cross} for the BoW input: removing stopwords and retaining the 2000 most frequent words of each language as our vocabularies. We use the English and German stopword lists from NLTK\footnote{\url{https://www.nltk.org/}}.

\subsection*{Hyperparameters}
The neural topic models are trained on a single Nvidia V100 GPU (35 minutes) while PLTM is trained on a single Intel Xeon CPU (3 hours). During testing, we averaged the inferred topic distributions for each article/image from 20 samples. For all the neural models we used Adam optimizer with a learning rate of $2^{-3}$. We use a batch size of 64 except for M3L-Contrast. For M3L-Contrast, we set the temperature $\tau$ to 0.07 following \cite{guo2022multilingual}. We set the contrastive weight $s$ to 50 based on initial experiments. Tuning $\tau$ and $s$ are saved for future work.

\subsection*{Inference network}
We use the same inference network structure as ZeroshotTM~\cite{bianchi-etal-2021-cross}: one fully-connected hidden layer followed by softplus layer with 100 dimensions. We save the investigation of other inference network structures for future work.


\subsection*{Encoder Details}
We use SentenceBERT to encode all our data~\cite{reimers-gurevych-2020-making}~\footnote{\url{https://www.sbert.net/docs/pretrained_models.html}}. For a fairer comparison, we set the maximum sequence length of all text encoders to 128 tokens. The multilingual text encoder is \textit{paraphrase-multilingual-mpnet-base-v2}. For the monolingual encoders, the English encoder is \textit{all-mpnet-base-v2} and the German encoder is \textit{T-Systems-onsite/erman-roberta-sentence-transformer-v2}. ResNet embeddings are provided in this Kaggle challenge: \url{https://www.kaggle.com/competitions/wikipedia-image-caption}.

%\subsection*{Impact Statement}
\subsection*{Potential impact and risks}
Our models are currently for research purposes only. We do not advise that it be used in production settings. Our models might associate images of people and objects with negative and insensitive stereotypes if the training data has these associations. Since we use CLIP to encode texts and images in our experiments, our models might also perpetuate the harmful stereotypes found in the CLIP training data discussed in~\cite{birhane2021multimodal}. The same issue applies to the other pretrained encoders we use in our experiments.

%Furthermore, our models might associate images with topics that are inconsistent with the image content. Therefore these outputs must be used with caution in applications such as image tagging and image clustering.

%\subsection*{Intended use of datasets}
%The Wikipedia Image Text (WIT) dataset is for multimodal multilingual models and tasks. The Wikipedia Comparable Corpora is for multilingual/cross-lingual tasks. Since we link the images in WIT with full articles from the Comparable Corpus using article titles, we want to emphasize that these linked articles and images should not be used outside of research settings because we cannot guarantee that all the article-image associations are correct.

