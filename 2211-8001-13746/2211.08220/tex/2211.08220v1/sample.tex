%% if you are submitting an initial manuscript then you should have submission as an option here
%% if you are submitting a revised manuscript then you should have revision as an option here
%% otherwise options taken by the article class will be accepted
\documentclass[submission]{FPSAC2023}
%% but DO NOT pass any options (or change anything else anywhere) which alters page size / layout / font size etc

%% note that the class file already loads {amsmath, amsthm, amssymb}

\newtheorem{thm}{Theorem}
\newtheorem{lem}{Lemma}

\usepackage{lipsum}

%% define your title in the usual way
\title[Lozenge tilings of hexagons with intrusions]{Lozenge tilings of hexagons with intrusions}

%% define your authors in the usual way
%% use \addressmark{1}, \addressmark{2} etc for the institutions, and use \thanks{} for contact details
\author[Seok Hyun Byun and Tri Lai]{Seok Hyun Byun\thanks{\href{mailto:sbyun@clemson.edu}{sbyun@clemson.edu}.}\addressmark{1}, \and Tri Lai\thanks{\href{mailto:tlai3@unl.edu}{tlai3@unl.edu}.}\addressmark{2}}

%\thanks{\href{mailto:hello@world.c}{hello@world.c}. Longer names me was partially supported by Grant 2017.11.14.$\partial$\;supp.}

%% then use \addressmark to match authors to institutions here
\address{\addressmark{1}School of Mathematical and Statistical Sciences, Clemson University, SC \\ \addressmark{2} Department of Mathematics, University of Nebraska–Lincoln, NE}

%% put the date of submission here
\received{\today}

%% leave this blank until submitting a revised version
%\revised{}

%% put your English abstract here, or comment this out if you don't have one yet
%% please don't use custom commands in your abstract / resume, as these will be displayed online
%% likewise for citations -- please don't use \cite, and instead write out your citation as something like (author year)
\abstract{MacMahon's classical theorem on the number of boxed plane partitions has been generalized in several directions. One way to generalize the theorem is to view boxed plane partitions as lozenge tilings of a hexagonal region, then generalize it by making some holes in the region and counting its tilings. In this extended abstract, we provide new regions whose numbers of lozenges tilings are given by simple product formulas. The regions we consider can be obtained from hexagons by removing structures called \textit{intrusions}. In fact, we show that tiling generating functions of those regions under certain weights are given by similar formulas. These give $q$-analogue of the enumeration results.}

%% put your French abstract here, or comment this out if you don't have one
%\resume{\lipsum[2]}

%% put your keywords here, or comment this out if you don't have them yet
\keywords{enumeration, lozenge tilings, MacMahon's theorem, plane partitions, tiling generating function.}

%% you can include your bibliography however you want, but using an external .bib file is STRONGLY RECOMMENDED and will make the editor's life much easier
%% regardless of how you do it, please use numerical citations; i.e., [xx, yy] in the text

%% this sample uses biblatex, which (among other things) takes care of URLs in a more flexible way than bibtex
%% but you can use bibtex if you want
\usepackage[backend=bibtex]{biblatex}
\addbibresource{sample.bib}
%% note the \printbibliography command at the end of the file which goes with these biblatex commands
\begin{document}
\maketitle
%% note that you DO NOT have to put your abstract here -- it is generated by \maketitle and the \abstract and \resume commands above

\section{Introduction}

The enumeration of lozenge tilings received considerable attention during the last three decades. Most of the works in this area were motivated by the classical result of MacMahon~\cite{M}. Via the bijection of David and Tomei~\cite{DT}, MacMahon's theorem on boxed plane partitions can be rephrased as follow. The number of lozenge tilings of hexagons with sides of length $x,y,z,x,y,$ and $z$ (in clockwise order) is given by the following product formula:

\begin{equation}\label{eqn:eq11}
    \prod_{i=1}^{x}\prod_{j=1}^{y}\prod_{k=1}^{z}\frac{i+j+k-1}{i+j+k-2}=\frac{H(x)H(y)H(z)H(x+y+z)}{H(x+y)H(y+z)H(z+x)},
\end{equation}
where $H(n):=\prod_{i=0}^{n-1}i!$ for $n\geq 1$ and $H(0):=1$.

In 1998, Krattenthaler and Okada~\cite{KO} enumerated the lozenge tilings of a hexagon with the central unit triangle removed. Soon later, in 2001, this result was generalized by Ciucu et al.~\cite{CEKZ}. They generalized Krattenthaler and Okada's result by considering a hexagonal region with a triangle of arbitrary size removed from the center. This more general result was then further generalized by Rosengren. In his paper~\cite{R}, Rosengren considered a hexagon with a triangular hole in an arbitrary position and found a formula that enumerated its lozenge tilings. In fact, he proved a more general statement. There, he considered certain weights on lozenges and found a tiling generating function of the region under the weight assignment (see the three pictures on the top in Figure \ref{fig:Figure_1} that illustrate these regions).
\begin{figure}
    \centering
    \includegraphics[width=0.6\textwidth]{Figure_1.pdf}
    \caption{Various regions obtained from hexagons by making some holes in them.}
    \label{fig:Figure_1}
\end{figure}

Recently, the first author~\cite{B} considered a hexagonal region with several unit triangles removed\footnote{In that paper, the collection of these removed unit triangles is called "intrusion". In this extended abstract, the definition of  "intrusion" will be more general than that.} and enumerated the lozenge tilings of the region (see the bottom left picture in Figure \ref{fig:Figure_1} for an example). Motivated by the result of Ciucu et al.~\cite{CEKZ} and Rosengren~\cite{R}, we generalize this result by allowing removed left-pointing triangles to have arbitrary odd size (except the rightmost one, which also allows having an arbitrary even size) and by considering tiling generating functions under a similar weight assignment (see the bottom right picture in Figure \ref{fig:Figure_1} that illustrate these regions). A precise description of the regions will be given in the next section.

This extended abstract is organized as follows. First, section 2 introduces the regions of our interest and states the main result. Then, in section 3, we provide a sketch of the proof with some details.

\section{New regions and statement of the main results}

\begin{figure}
    \centering
    \includegraphics[width=0.5454\textwidth]{Figure_2.pdf}
    \caption{A lozenge tiling of the region $H_{5,3,4}$ and the weight assignment on its lozenges. Horizontal lozenges labeled by an integer $n$ are given a weight $\frac{q^{n}+q^{-n}}{2}$ and all other lozenges are given a weight 1.}
    \label{fig:Figure_2}
\end{figure}

We consider a triangular lattice such that one family of lattice lines is vertical. On the lattice, we draw a hexagon with sides of length $x,y,z,x,y,$ and $z$ clockwise from the left and denote the region by $H_{x,y,z}$. We first provide a weighted generalization of MacMahon's theorem since it is needed to understand the main result better. To give weights on lozenges, we put the region $H_{x,y,z}$ on $(i,j)-$coordinate system\footnote{We set the unit length of the $i$-axis on this plane equals the half of the side length of unit triangles. We use this $(i,j)-$coordinate system throughout this extended abstract.} as indicated in Figure \ref{fig:Figure_2}. As shown in the picture, we put the region so that the $j-$axis overlaps with the perpendicular bisector of the left side. Now, we give a weight $\frac{q^i+q^{-i}}{2}$ to all horizontal lozenges whose centers have $i-$coordinate $i$ and give a weight $1$ to all non-horizontal lozenges. When certain weights are assigned on lozenges and a tiling of a region is given, \textit{a weight of the tiling} of the region is the product of weights of all lozenges that constitute the tiling. \textit{A tiling generating function (TGF)} of the region is the sum of weights of all its lozenge tilings. Under this weight assignment, let $M_{q}(H_{x,y,z})$ be the TGF of $H_{x,y,z}$. To state the lemma, we introduce several notations. Let $\langle n\rangle_q:=\frac{q^n-q^{-n}}{q-q^{-1}}$ for a positive integer $n$. Since $\langle n\rangle_{q}\rightarrow 1$ as $q\rightarrow 1$, $\langle n\rangle_{q}$ can be considered as a $q$-analogue of a positive integer $n$. Also, we set $\langle n\rangle_q!:=\langle1\rangle_q\cdots\langle n\rangle_q$ for a positive integer $n$ and $\langle0\rangle_q!:=1$. 

\begin{lem}
For non-negative integers $x,y,$ and $z,$ the TGF of the region $H_{x,y,z}$ is given by the following formula:
    \begin{equation}\label{eqn:eq21}
        M_q(H_{x,y,z})=\Bigg[\prod_{i=1}^{y}\prod_{j=1}^{z}\frac{q^{i-j}+q^{j-i}}{2}\Bigg]\cdot\frac{H_{q}(x)H_{q}(y)H_{q}(z)H_{q}(x+y+x)}{H_{q}(x+y)H_{q}(y+z)H_{q}(z+x)}
    \end{equation}
\end{lem}
where $H_{q}(n):=\langle0\rangle_{q}!\langle1\rangle_{q}!\cdots\langle n-1\rangle_{q}!$ for positive integer $n$ and $H_{q}(0):=1$.

We generalize this region $H_{x,y,z}$ by putting some triangular holes in it. For any non-negative integers $a,m,n,k, a_1, a_2,\cdots,a_{k-1},$ and $a_k$, we set $b_i=\sum_{j=i}^{k}a_j$ for all positive integers $i=1,\cdots,k$. We now define the two regions $H_{e}(a,m,n,k:a_1,\cdots,a_k)$ and $H_{o}(a,m,n,k:a_1,\cdots,a_k)$.

\begin{figure}
    \centering
    \includegraphics[width=0.97\textwidth]{Figure_3.pdf}
    \caption{Two regions $H_{e}(2,4,8,3:2,1,2)$ (left) and $H_{o}(2,4,8,3:2,1,2)$ (right) on the $(i,j)-$coordinate planes.}
    \label{fig:Figure_3}
\end{figure}

When $k=0$, we set $H_{e}(a,m,n,0:\cdot):=H_{2a,m,n}$. When $k\geq 1$, we first consider a hexagon with sides of length $2a+1, m+2b_1+k-1, n+k, 2a+2b_1, m+k,$ and $n+2b_1+k-1$ clockwise from the left. Then, we delete the consecutive triangles of side lengths $1, 2a_1+1, 1, 2a_2+1,\cdots,1, 2a_{k-1}+1, 1, 2a_{k}$ on the perpendicular bisector of the left side from the left so that they alternate the orientation, where the first one is right-pointing. We denote this region by $H_{e}(a,m,n,k:a_1,\cdots,a_k)$ (see the left picture in Figure \ref{fig:Figure_3}). The collection of $2k$ triangles that we removed is an \textit{intrusion} of the region.

Similarly, when $k=0$, we set $H_{o}(a,m,n,0:\cdot):=H_{2a+1, m, n}$. When $k\geq1$, we consider a hexagon with sides of length $2a+1, m+2b_1+k, n+k, 2a+2b_1+1, m+k$, and $n+2b_1+k$ clockwise from the left. Then, we delete the consecutive triangles of side lengths $1, 2a_1+1, 1, 2a_2+1,\cdots,1, 2a_{k-1}+1, 1, 2a_k+1$ on the perpendicular bisector of the left side from the left so that they alternate the orientation, where the first one is right-pointing. We denote this region by $H_{o}(a,m,n,k:a_1,\cdots,a_k)$ (see the right picture in Figure \ref{fig:Figure_3}). Again, an \textit{intrusion} of this region is the collection of $2k$ triangles we removed.

We also assign certain weights to lozenges. Following the weight assignment on $H_{x,y,z}$, we put the regions $H_{e}(a,m,n,k:a_1,\cdots,a_k)$ and $H_{o}(a,m,n,k:a_1,\cdots,a_k)$ on $(i,j)-$plane as indicated in Figure \ref{fig:Figure_3}. As before, we put the regions so that the $j-$axis overlaps with the perpendicular bisectors of the left sides. Then, we give a weight $\frac{q^i+q^{-i}}{2}$ to all horizontal lozenges whose centers have $i-$coordinate $i$ and give a weight $1$ to all non-horizontal lozenges. Under this weight assignment, let $M_q(H_{e}(a,m,n,k:a_1,\cdots,a_k))$ and $M_q(H_{o}(a,m,n,k:a_1,\cdots,a_k))$ be the TGFs of the two regions. Note that if we interchange $m$ and $n$, the TGFs are invariant. This is because we assigned weights on lozenges so that it is symmetric with respect to the $j-$axis. Hence, without loss of generality, we will assume $n\geq m$ in the theorem below.

\begin{thm}\label{thm:Theorem1}
For any non-negative integers $a,m,n,k, a_1, a_2,\cdots,a_{k-1},$ and $a_k$ such that $n\geq m$, we have

\begin{equation}\label{eqn:eq22}
\begin{aligned}
    &\frac{M_q(H_{e}(a,m,n,k:a_1,\cdots,a_k))}{M_q(H_{2a+2b_1, m+k, n+k})}\\
    &
\begin{aligned}
    =\prod_{i=0}^{k-1}\Bigg[&\frac{(\langle b_{1}+i+\frac{1}{2}\rangle_{q^2})_{m+k-b_{1}-2i}(\langle a+b_1-b_{i+1}+i+1\rangle_{q^2})_{m+k+2b_{i+1}-2i-1}}{(\langle a+b_1+i+\frac{1}{2}\rangle_{q^2})_{m+k-2i}(\langle i+1\rangle_{q^2})_{m+k+b_{i+1}-2i-1}}\\
    &\cdot \frac{(\langle a+n+k+b_1-i\rangle_{q^2})_{b_{i+1}}}{2^{4b_{i+1}}(\langle i+1\rangle_{q^2})_{b_1}(\langle n+k-i-\frac{1}{2}\rangle_{q^2})_{b_{i+1}}(\langle a+b_1+i+1\rangle_{q^2})_{-b_{i+1}}}\\
    &\cdot \frac{(\langle m+k-i+\frac{1}{2}\rangle_{q^2})_{\lfloor\frac{n-m}{2}\rfloor}(\langle n+k-i\rangle_{q^2})_{-\lfloor\frac{n-m}{2}\rfloor}}{(\langle a+m+k+b_{1}-i+\frac{1}{2}\rangle_{q^2})_{\lfloor\frac{n-m}{2}\rfloor}(\langle a+n+k+b_{1}-i\rangle_{q^2})_{-\lfloor\frac{n-m}{2}\rfloor}}\\
    &\cdot \frac{(\langle m+k-i\rangle_{q^2})_{\lfloor\frac{n-m}{2}\rfloor}(\langle n+k-i-\frac{1}{2}\rangle_{q^2})_{-\lfloor\frac{n-m}{2}\rfloor}}{(\langle m+k+b_{i+1}-i\rangle_{q^2})_{\lfloor\frac{n-m}{2}\rfloor}(\langle n+k+b_{i+1}-i-\frac{1}{2}\rangle_{q^2})_{-\lfloor\frac{n-m}{2}\rfloor}}
\end{aligned}
\\
    &\cdot \prod_{j=1}^{i}\Big[
    \frac{(\langle b_j-b_{i+1}+i+1-j\rangle_{q^2})^2}{(\langle i+1-j\rangle_{q^2})^2}\cdot(\langle b_{j+1}+i-j+\frac{1}{2}\rangle_{q^2})_{a_j}(\langle b_{j+1}+i-j+1\rangle_{q^2})_{a_j}\Big]\Bigg]
\end{aligned}
\end{equation}

and

\begin{equation}\label{eqn:eq23}
\begin{aligned}
    &\frac{M_q(H_{o}(a,m,n,k:a_1,\cdots,a_k))}{M_q(H_{2a+2b_1+1, m+k, n+k})}\\
    &
\begin{aligned}
    =\prod_{i=0}^{k-1}\Bigg[&\frac{(\langle a+b_1-b_{i+1}+i+1\rangle_{q^2})_{n+k+2b_{i+1}-2i}(\langle b_1+i+\frac{3}{2}\rangle_{q^2})_{n+k-b_1-2i-2}}{(\langle i+1\rangle_{q^2})_{n+k+b_{i+1}-2i-1}(\langle a+b_1+i+\frac{3}{2}\rangle_{q^2})_{n+k-2i-1}}\\
    &\cdot \frac{(\langle b_{i+1}+1\rangle_{q^2})(\langle a+m+k+b_1-i+1\rangle_{q^2})_{b_{i+1}}}{2^{4b_{i+1}+2}(\langle i+2\rangle_{q^2})_{b_1}(\langle m+k-i+\frac{1}{2}\rangle_{q^2})_{b_{i+1}}(\langle a+b_1+i+1\rangle_{q^2})_{-b_{i+1}}}\\
    &\cdot \frac{(\langle m+k-i\rangle_{q^2})_{\lfloor\frac{n-m}{2}\rfloor}(\langle n+k-i-\frac{1}{2}\rangle_{q^2})_{-\lfloor\frac{n-m}{2}\rfloor}}{(\langle a+m+k+b_1-i+1\rangle_{q^2})_{\lfloor\frac{n-m}{2}\rfloor}(\langle a+n+k+b_1-i+\frac{1}{2}\rangle_{q^2})_{-\lfloor\frac{n-m}{2}\rfloor}}\\
    &\cdot \frac{(\langle m+k-i+\frac{1}{2}\rangle_{q^2})_{\lfloor\frac{n-m}{2}\rfloor}(\langle n+k-i\rangle_{q^2})_{-\lfloor\frac{n-m}{2}\rfloor}}{(\langle m+k+b_{i+1}-i+\frac{1}{2}\rangle_{q^2})_{\lfloor\frac{n-m}{2}\rfloor}(\langle n+k+b_{i+1}-i\rangle_{q^2})_{-\lfloor\frac{n-m}{2}\rfloor}}\\
    &
    \begin{aligned}
        \cdot \prod_{j=1}^{i}&\Big[\frac{(\langle b_j-b_{i+1}+i+1-j\rangle_{q^2})^2}{(\langle i+1-j\rangle_{q^2})(\langle b_j+i+1-j\rangle_{q^2})}\frac{(\langle b_j+i+2-j\rangle_{q^2})}{(\langle i+2-j\rangle_{q^2})}\\
        &\cdot (\langle b_{j+1}+i-j+\frac{3}{2}\rangle_{q^2})_{a_j}(\langle b_{j+1}+i-j+2\rangle_{q^2})_{a_j}\Big]\Bigg]
    \end{aligned}
\end{aligned}
\end{aligned}
\end{equation}
\end{thm}

where $\langle n\rangle_q:=\frac{q^n-q^{-n}}{q-q^{-1}}$ for positive integer (or half-integer) $n$ and 

\begin{equation*}
    (\langle a\rangle_{q})_n:=
    \begin{cases}
        \langle a\rangle_{q}\langle a+1\rangle_{q}\cdots\langle a+n-1\rangle_{q}  & \text{if $n$ is positive,}\\
        1                    & \text{if $n$ is $0$,}\\
        \frac{1}{\langle a-1\rangle_{q}\langle a-2\rangle_{q}\cdots\langle a+n\rangle_{q}} & \text{if $n$ is negative,}
    \end{cases}
\end{equation*}
for any positive integer (or half-integer) $a$ and any integer $n$ such that $a+n>0$.

Note that by combining \eqref{eqn:eq21} and \eqref{eqn:eq22} (or \eqref{eqn:eq23}), one can obtain a product formula for the tiling generating function of the region $H_{e}(a,m,n,k:a_1,\cdots,a_k)$ (or $H_{o}(a,m,n,k:a_1,\cdots,a_k)$). Hence, one can view our results as tiling generating functions of these two regions. The case when $q=0$ and $k=0$ gives MacMahon's formula~\cite{M} and the case when $q=0$ and $a_{i}=0$ for all $i$ was considered by the first author~\cite{B}.

\section{Schematic proof of the main theorem}

\begin{proof}
The proof mainly consists of three parts: 1) the case when $n=m$, 2) the case when $n=m+1$, and 3) the case when $n\geq m+2$.

\begin{figure}
    \centering
    \includegraphics[width=0.97\textwidth]{Figure_4.pdf}
    \caption{Applying Matching Factorization Theorem on two regions in Figure 3.}
    \label{fig:Figure_4}
\end{figure}

The proof of the first case is based on the Matching Factorization Theorem of Ciucu~\cite{C} and results of Lai and Rohatgi~\cite{LR} on tiling generating functions of certain regions (they called these regions \textit{halved hexagons with dents}). First, we split the region $H_{e}(a,m,m,k:a_1,\cdots,a_k)$ along the zigzag line as shown in the left picture\footnote{In the picture, shaded ellipses denote lozenges weighted by $\frac{1}{2}$.} in Figure \ref{fig:Figure_4}. One can see that the region is split into two subregions by the zigzag line. Then, the Matching Factorization Theorem claims that the tiling generating function $M_q(H_{e}(a,m,m,k:a_1,\cdots,a_k))$ can be expressed as $2^{m}$ times the product of tiling generating functions of the two subregions. Using a simple combinatorial argument, one can show that these two subregions have the same tiling generating functions as the regions studied by Lai and Rohatgi~\cite{LR}. Thus, using their formulas, we can obtain the tiling generating function of $H_{e}(a,m,m,k:a_1,\cdots,a_k)$. A similar argument can also prove that the tiling generating function of $H_{o}(a,m,m,k:a_1,\cdots,a_k)$ is given by \eqref{eqn:eq23} with $n$ replaced by $m$.


The proof of the remaining two cases, on the other hand, is an inductive proof based on Kuo's graphical condensation method~\cite{K}. Using this, we make three recurrence relations and prove the main theorem using induction. To state the theorem, denote a bipartite graph by $G=(V_1, V_2, E)$, where $E$ is the set of edges of the graph and $(V_1, V_2)$ is the partition of the vertex set of the graph $G$ such that every edge in $E$ connects a vertex in $V_1$ and a vertex in $V_2$. Also, for any set of vertices $\{x_1, x_2,...,x_n\}$, let $G-\{x_1, x_2,...,x_n\}$ be the graph obtained from $G$ by deleting $n$ vertices $x_1, x_2,...,x_n$ and all edges that are adjacent to at least one of $x_1, x_2,...,x_n$. If all the edges of a graph are weighted, \textit{the weight of a perfect matching}\footnote{\textit{A perfect matching} is a subset of edges of a bipartite graph such that every vertex is incident to precisely one edge.} is defined to be the product of weights of all weighted edges that constitute the perfect matching. For any weighted graph $G$, let $M(G)$ be the sum of weights of all perfect matchings of it.

\begin{thm}[Kuo, ~\cite{K}]\label{thm:Theorem2}
Let $G=(V_1, V_2, E)$ be a plane bipartite graph in which $|V_1|=|V_2|$. Let vertices $x, y, z,$ and $w$ appear in a cyclic order on a face of $G$. If $x,z \in V_1$ and $y,w \in V_2$, then
\begin{equation}\label{eqn:eq31}
\begin{aligned}
    &M(G)M(G-\{x,y,z,w\})\\
    &=M(G-\{x,y\})M(G-\{z,w\})+M(G-\{x,w\})M(G-\{y,z\}).    
\end{aligned}
\end{equation}
\end{thm}

\begin{thm}[Kuo, ~\cite{K}]\label{thm:Theorem3}
Let $G=(V_1, V_2, E)$ be a plane bipartite graph in which $|V_1|=|V_2|+1$. Let vertices $x, y, z,$ and $w$ appear in a cyclic order on a face of $G$. If $x,y,z \in V_1$ and $w \in V_2$, then
\begin{equation}\label{eqn:eq32}
\begin{aligned}
    &M(G-\{y\})M(G-\{x,z,w\})\\
    &=M(G-\{x\})M(G-\{y,z,w\})+M(G-\{z\})M(G-\{x,y,w\}).
\end{aligned}
\end{equation}
\end{thm}

\begin{figure}
    \centering
    \includegraphics[width=0.73\textwidth]{Figure_5.pdf}
    \caption{Application of Kuo's graphical condensation on the three regions.}
    \label{fig:Figure_5}
\end{figure}

We first consider the region $H_{o}(a,m+1,n,k-1:a_1,\cdots,a_{k-1})$ on the $(i,j)-$plane and choose four unit triangles $x,y,z,$ and $w$ as described in the top left picture in Figure \ref{fig:Figure_5}. We consider this region's dual graph and four vertices corresponding to the four chosen unit triangles. Then, Theorem \ref{thm:Theorem2} can be applied to the dual graph\footnote{This idea will be used two more times when we obtain two more recurrence relations below.} and we obtain the following recurrence relation (see Figure \ref{fig:Figure_6} that shows how six regions are obtained):

\begin{equation}\label{eqn:eq33}
\begin{aligned}
    &M_{q}(H_{o}(a,m+1,n,k-1:a_1,\cdots,a_{k-1}))M_{q}(H_{e}(a,m,n-1,k:a_1,\cdots,a_{k-1},0))\\
    &
\begin{aligned}
    =&M_{q}(H_{e}(a,m+1,n-1,k:a_1,\cdots,a_{k-1},0))M_{q}(H_{o}(a,m,n,k-1:a_1,\cdots,a_{k-1}))\\
    &+M_{q}(H_{e}(a,m,n,k:a_1,\cdots,a_{k-1},0))M_{q}(H_{o}(a,m+1,n-1,k-1:a_1,\cdots,a_{k-1})).
\end{aligned}
\end{aligned}
\end{equation}

\begin{figure}
    \centering
    \includegraphics[width=0.79\textwidth]{Figure_6.pdf}
    \caption{Application of Kuo's graphical condensation on the top left region in Figure 5. Lozenges that are always contained in any tilings are shaded.}
    \label{fig:Figure_6}
\end{figure}

Now we consider the region $H_{e}(a,m+1,n,k:a_1,\cdots,a_k)$ on the $(i,j)-$plane with $a_k\geq 1$. From the region, we replace the removed left-pointing triangle of side length $2a_k$ with that of side length $2a_k-1$. Note that this new region contains one more left-pointing unit triangle than right-pointing. From the new region, we choose four unit triangles $x,y,z,$ and $w$ as described in the top right picture in Figure \ref{fig:Figure_5}. Then, Theorem \ref{thm:Theorem3} can be applied to the dual graph of the region, and we obtain the following recurrence relation:

\begin{equation}\label{eqn:eq34}
\begin{aligned}
    &M_{q}(H_{e}(a,m+1,n,k:a_1,\cdots,a_k))M_{q}(H_{o}(a,m+1,n,k:a_1,\cdots,a_k-1))\\
    &
\begin{aligned}
    =&M_{q}(H_{o}(a,m+2,n,k:a_1,\cdots,a_k-1))M_{q}(H_{e}(a,m,n,k:a_1,\cdots,a_k))\\
    &+M_{q}(H_{o}(a,m+1,n+1,k:a_1,\cdots,a_k-1))M_{q}(H_{e}(a,m+1,n-1,k:a_1,\cdots,a_k)).
\end{aligned}
\end{aligned}
\end{equation}

For the last recurrence relation, we consider the region $H_{o}(a,m+1,n,k:a_1,\cdots,a_k)$ on the $(i,j)-$plane (This time, we allow $a_k$ to be $0$). From the region, we replace the removed left-pointing triangle of side length $2a_k+1$ with that of side length $2a_k$. From the new region, we choose four unit triangles $x,y,z,$ and $w$ as described in the bottom picture in Figure \ref{fig:Figure_5}. Again, Theorem \ref{thm:Theorem3} gives the following recurrence relation:

\begin{equation}\label{eqn:eq35}
\begin{aligned}
    &M_{q}(H_{o}(a,m+1,n,k:a_1,\cdots,a_k))M_{q}(H_{e}(a,m+1,n,k:a_1,\cdots,a_k))\\
    &
\begin{aligned}
    =&M_{q}(H_{e}(a,m+2,n,k:a_1,\cdots,a_k))M_{q}(H_{o}(a,m,n,k:a_1,\cdots,a_k))\\
    &+M_{q}(H_{e}(a,m+1,n+1,k:a_1,\cdots,a_k))M_{q}(H_{o}(a,m+1,n-1,k:a_1,\cdots,a_k)).
\end{aligned}
\end{aligned}
\end{equation}

If we replace $n$ by $m+1$ in \eqref{eqn:eq33}-\eqref{eqn:eq35} and use the simple fact that two regions that are symmetric with respect to $j$-axis have the same tiling generating functions, we get

\begin{equation}\label{eqn:eq36}
\begin{aligned}
    &M_{q}(H_{o}(a,m+1,m+1,k-1:a_1,\cdots,a_{k-1}))M_{q}(H_{e}(a,m,m,k:a_1,\cdots,a_{k-1},0))\\
    &=2M_{q}(H_{e}(a,m,m+1,k:a_1,\cdots,a_{k-1},0))M_{q}(H_{o}(a,m,m+1,k-1:a_1,\cdots,a_{k-1})),
\end{aligned}
\end{equation}

\begin{equation}\label{eqn:eq37}
\begin{aligned}
    &M_{q}(H_{e}(a,m+1,m+1,k:a_1,\cdots,a_k))M_{q}(H_{o}(a,m+1,m+1,k:a_1,\cdots,a_k-1))\\
    &=2M_{q}(H_{o}(a,m+1,m+2,k:a_1,\cdots,a_k-1))M_{q}(H_{e}(a,m,m+1,k:a_1,\cdots,a_k)),
\end{aligned}
\end{equation}

and

\begin{equation}\label{eqn:eq38}
\begin{aligned}
    &M_{q}(H_{o}(a,m+1,m+1,k:a_1,\cdots,a_k))M_{q}(H_{e}(a,m+1,m+1,k:a_1,\cdots,a_k))\\
    &=2M_{q}(H_{e}(a,m+1,m+2,k:a_1,\cdots,a_k))M_{q}(H_{o}(a,m,m+1,k:a_1,\cdots,a_k)).
\end{aligned}
\end{equation}

Before we show the proof of two remaining cases, we first define a concept of \textit{depth} (denoted by $d$) on two families of regions $H_{e}(a,m,n,k:a_1,\cdots,a_k))$ and $H_{o}(a,m,n,k:a_1,\cdots,a_k))$. The depth of the both regions $H_{e}(a,m,n,k:a_1,\cdots,a_k))$ and $H_{o}(a,m,n,k:a_1,\cdots,a_k))$ is $b_1+k=(\sum_{j=1}^{k}a_j)+k$.

We prove the case when $n=m+1$ using induction on the depth $d$ we just defined. The identity is trivial when $d=0$. Suppose \eqref{eqn:eq22} and \eqref{eqn:eq23} are true for $n=m+1$ and $d<D$ for some positive integer $D$. In \eqref{eqn:eq36} (and \eqref{eqn:eq37}), the two terms on the left side are known to be given by \eqref{eqn:eq22} and \eqref{eqn:eq23} (together with \eqref{eqn:eq21}) due to the proof of the first case (case when $n=m$). Also, note that the depth of $H_{e}(a,m,m+1,k:a_1,\cdots,a_{k-1},0)$ is one greater than that of $H_{o}(a,m,m+1,k-1:a_1,\cdots,a_{k-1})$ and the depth of $H_{e}(a,m,m+1,k:a_1,\cdots,a_k)$ is one greater than that of $H_{o}(a,m+1,m+2,k:a_1,\cdots,a_k-1)$ in \eqref{eqn:eq36} and \eqref{eqn:eq37}, respectively. Hence, using \eqref{eqn:eq36} and \eqref{eqn:eq37}, we can show that $M_{q}(H_{e}(a,m,m+1,k:a_1,\cdots,a_k))$ is given by \eqref{eqn:eq22} and \eqref{eqn:eq21} for any $a_k\geq0$. If we see \eqref{eqn:eq38}, the two terms on the left side are again given by \eqref{eqn:eq22} and \eqref{eqn:eq23} (together with \eqref{eqn:eq21}). This time, the two regions appeared on the right side, $H_{e}(a,m+1,m+2,k:a_1,\cdots,a_k)$ and $H_{o}(a,m,m+1,k:a_1,\cdots,a_k)$, have the equal depth. However, since we already verified that \eqref{eqn:eq22} is true when $n=m+1$ and $d=D$, we can keep using mathematical induction to show that $M_{q}(H_{o}(a,m,m+1,k:a_1,\cdots,a_k))$ is given by \eqref{eqn:eq23} and \eqref{eqn:eq21}.

The proof of the general case when $n\geq m+2$ is very similar to that of the previous case. We use double induction to give proof. First, we use induction on $n-m$. The case when $n-m=0$ and $1$ were already proven. Now, assume that Theorem \ref{thm:Theorem1} is true when $n-m<E$ for some positive integer $E$. To complete the proof, we need to show that the theorem still holds when $n-m=E$. We verify this induction step using another induction on the depth. There is nothing to prove when $n-m=E$ and the depth is $0$. Then, using a similar argument as the proof of the case when $n-m=1$, we can show that the theorem is true for the case when $n-m=E$ with arbitrary depth (instead of \eqref{eqn:eq36}, \eqref{eqn:eq37}, and \eqref{eqn:eq38}, we need to use \eqref{eqn:eq33}, \eqref{eqn:eq34}, and \eqref{eqn:eq35}). Since we show that Theorem \ref{thm:Theorem1} holds when $n-m=E$, by mathematical induction, the theorem is true for any $m$ and $n$ with $n\geq m$. This completes the proof.
\end{proof}

%\subsection{A subsection}

%\lipsum[5]

%\begin{align*}
%1 + 1 + 1 &= 2 + 1 \\ &= 3.
%\end{align*}

%\subsubsection{A subsubsection}

%\lipsum[17]

%equation~%\eqref{eqn:eq1}. And a reference to %Figure~\ref{fig:plot}.

%\acknowledgements

%% if you use biblatex then this generates the bibliography
%% if you use some other method then remove this and do it your own way

\printbibliography

\end{document}
