\documentclass[11pt, reqno]{amsart}
%\usepackage{e-jc}
\usepackage{amsthm,amsmath,amssymb}
\usepackage{mathtools}
\usepackage{graphicx}
\usepackage{caption}
\usepackage{ragged2e}
\graphicspath{{Figure/}}
\numberwithin{equation}{section}
\numberwithin{figure}{section}
\theoremstyle{plain}
\newtheorem{definition}{Definition}[section]
\newtheorem{thm}{Theorem}[section]
\newtheorem{lem}[thm]{Lemma}
\newtheorem{cor}[thm]{Corollary}
\newtheorem{prop}[thm]{Proposition}
\newtheorem{con}[thm]{Conjecture}
\theoremstyle{definition}
\theoremstyle{remark}
\newtheorem{rmk}[thm]{Remark}

\textwidth16.5cm
\topmargin0cm
\oddsidemargin0cm
\evensidemargin0cm
\textheight22.4cm

\newcommand{\M}{\operatorname{M}}
\newcommand{\Hf}{\operatorname{H}}
\newcommand{\wt}{\operatorname{wt}}
\newcommand{\B}{\operatorname{Box}}
\newcommand{\RomanNumeralCaps}[1]
    {\MakeUppercase{\romannumeral #1}}
\begin{document}
\setlength{\abovedisplayskip}{10pt}
\setlength{\belowdisplayskip}{10pt}

\title{Lozenge Tilings of Hexagons with Intrusions I: Generalized Intrusion}

\author{Seok Hyun Byun}
\address{School of Mathematical and Statistical Sciences, Clemson University, Clemson, SC 29634, U.S.A.}
\email{sbyun@clemson.edu}
\thanks{S.B. is supported by the AMS-Simons Travel Grant.}

\author{Tri Lai}
\address{Department of Mathematics, University of Nebraska – Lincoln, Lincoln, NE 68588, U.S.A.}
\email{tlai3@unl.edu}
\thanks{T.L. was supported in part by Simons Collaboration Grant (\#585923).}

\maketitle
\begin{abstract}
MacMahon's classical theorem on the number of boxed plane partitions has been generalized in several directions. One way to generalize the theorem is to view boxed plane partitions as lozenge tilings of a hexagonal region, then generalize it by making some holes in the region and counting its tilings. In this paper, we provide new regions whose numbers of lozenges tilings are given by simple product formulas. The regions we consider can be obtained from hexagons by removing structures called \textit{intrusions}. In fact, we show that tiling-generating functions of those regions under certain weights are given by similar formulas. These give $q$-analogue of the enumeration results.
\end{abstract}

\section{Introduction}

The enumeration of lozenge tilings has received considerable attention during the last three decades. Most of the works in this area were motivated by the classical result of MacMahon~\cite{M}. Via the bijection of David and Tomei~\cite{DT}, MacMahon's theorem on boxed plane partitions can be rephrased as follows. The number of lozenge tilings of hexagons with sides of length $x,y,z,x,y,$ and $z$ (in clockwise order) is given by the following product formula:

\begin{equation}\label{eqn:eq11}
    \prod_{i=1}^{x}\prod_{j=1}^{y}\prod_{k=1}^{z}\frac{i+j+k-1}{i+j+k-2}=\frac{H(x)H(y)H(z)H(x+y+z)}{H(x+y)H(y+z)H(z+x)},
\end{equation}
where $H(n):=\prod_{i=0}^{n-1}i!$ for $n\geq 1$ and $H(0):=1$.

From lozenge tilings point of view, one way to generalize MacMahon's theorem is to make some holes in the hexagonal region and to count its tilings (see ~\cite{B,C2, C3, CEKZ, CK3, CL, Fu2, L1, L2, L3, L4, LR1, R} for various results that generalized MacMahon's theorem in this direction.). In 1998, Krattenthaler and Okada~\cite{KO} enumerated the lozenge tilings of a hexagon with the central unit triangle removed. Soon later, in 2001, this result was generalized by Ciucu et al.~\cite{CEKZ}. They generalized Krattenthaler and Okada's result by considering a hexagonal region with a triangle of arbitrary size removed from the center. This more general result was then further generalized by Rosengren. In his paper~\cite{R}, Rosengren considered a hexagon with a triangular hole in an arbitrary position and found a formula that enumerated its lozenge tilings. In fact, he proved a more general statement. He considered certain weights on lozenges and found a tiling-generating function of the region under the weight assignment (see the three pictures on the top in Figure \ref{fig:Figure_1} that illustrate these regions).
\begin{figure}
    \centering
    \includegraphics[width=0.6\textwidth]{Figure1.pdf}
    \caption{Various regions obtained from hexagons by making some holes in them.}
    \label{fig:Figure_1}
\end{figure}

Recently, the first author~\cite{B} considered a hexagonal region with several unit triangles removed\footnote{In that paper, the collection of these removed unit triangles is called ``an intrusion". In this paper, the definition of  ``an intrusion" will be more general than that.} and enumerated the lozenge tilings of the region (see the bottom left picture in Figure \ref{fig:Figure_1} for an example). Motivated by the result of Ciucu et al.~\cite{CEKZ} and Rosengren~\cite{R}, we generalize this result by allowing removed left-pointing triangles to have arbitrary odd size (except the rightmost one, which also allows having an arbitrary even size) and by considering tiling generating functions under a similar weight assignment (see the bottom right picture in Figure \ref{fig:Figure_1} that illustrates these regions). A precise description of the regions will be given in the next section. Recently, Fulmek~\cite{Fu2} also considered lozenge tilings enumeration of hexagonal regions that generalize the region considered in \cite{B}. While Fulmek considered new regions by translating the intrusions considered in ~\cite{B}, we generalized the intrusions in ~\cite{B} by inflating some left-pointing unit triangles. We also obtained the generalization of the first author's work in different directions, and they will be presented in the forthcoming papers~\cite{BL1,BL2}.

This paper is organized as follows. 
In Section 2, we first state a weighted generalization of MacMahon's theorem. Then, we describe our new regions and present the main result.
In Section 3, we gather several results that are needed in the next section.
In Section 4, we give a proof of the main theorem, Theorem 2.2.

\section{New regions and statement of the main results}


\begin{figure}
    \centering
    \includegraphics[width=0.55\textwidth]{Figure3.pdf}
    \caption{A lozenge tiling of the region $H_{5,3,4}$ and the weight assignment on its lozenges. Horizontal lozenges labeled by an integer $n$ are given a weight $\frac{q^{n}+q^{-n}}{2}$ and all other lozenges are given a weight 1.}
    \label{fig:Figure_2}
\end{figure}

We consider a triangular lattice such that one family of lattice lines is vertical. On the lattice, we draw a hexagon with sides of length $x,y,z,x,y,$ and $z$ clockwise from the left and denote the region by $H_{x,y,z}$. We first provide a weighted generalization of MacMahon's theorem since it is needed to understand the main result better. To give weights on lozenges, we put the region $H_{x,y,z}$ on $(i,j)-$coordinate system\footnote{We set the unit length of the $i$-axis on this plane equals the half of the side length of unit triangles. We use this $(i,j)-$coordinate system throughout this paper.} as indicated in Figure \ref{fig:Figure_2}. As shown in the picture, we put the region so that the $j-$axis overlaps with the perpendicular bisector of the left side. Now, we give a weight $\frac{q^i+q^{-i}}{2}$ to all horizontal lozenges whose centers have $i-$coordinate $i$ and give a weight $1$ to all non-horizontal lozenges. \textit{A lozenge tiling} of a region is a collection of unit lozenges that covers the region without gaps or overlaps. When certain weights are assigned on lozenges, and a tiling of a region is given, \textit{a weight of the tiling} of the region is the product of weights of all lozenges that constitute the tiling. \textit{A tiling generating function (TGF)} of the region is the sum of weights of all its lozenge tilings. Under this weight assignment, let $M_{q}(H_{x,y,z})$ be the TGF of $H_{x,y,z}$. To state the lemma, we introduce several notations. Let $\langle n\rangle_q:=\frac{q^n-q^{-n}}{q-q^{-1}}$ for a positive integer $n$. Since $\langle n\rangle_{q}\rightarrow 1$ as $q\rightarrow 1$, $\langle n\rangle_{q}$ can be considered as a $q$-analogue of a positive integer $n$. Also, we set $\langle n\rangle_q!:=\langle1\rangle_q\cdots\langle n\rangle_q$ for a positive integer $n$ and $\langle0\rangle_q!:=1$. The following lemma is not new. This is equivalent to a special case of Rosengren's result (Theorem 2.1 in ~\cite{R}) when a triangle of size $0$ is removed from a hexagonal region. The second author also proved a result (Lemma 2.7 in ~\cite{L5}) that can be specialized to almost the same result as the following lemma\footnote{In \cite{L5}, the second author proved a TGF of dented semi-hexagons (that can be specialized to hexagons by making suitable choices of positions of dents) with a different choice of $j$-axis. However, the proof presented in that paper can also be employed to give proof of the above Lemma 2.1 in the current paper. A more general result that generalized both Lemma 2.1 in this paper and Lemma 2.7 in ~\cite{L5} and its proof will be presented in the forthcoming paper of the authors \cite{BL1}.}.

\begin{lem}
For non-negative integers $x,y,$ and $z,$ the TGF of the region $H_{x,y,z}$ is given by the following formula:
    \begin{equation}\label{eqn:eq21}
        M_q(H_{x,y,z})=\Bigg[\prod_{i=1}^{y}\prod_{j=1}^{z}\frac{q^{i-j}+q^{j-i}}{2}\Bigg]\cdot\frac{H_{q}(x)H_{q}(y)H_{q}(z)H_{q}(x+y+x)}{H_{q}(x+y)H_{q}(y+z)H_{q}(z+x)}
    \end{equation}
where $H_{q}(n):=\langle0\rangle_{q}!\langle1\rangle_{q}!\cdots\langle n-1\rangle_{q}!$ for positive integer $n$ and $H_{q}(0):=1$.
\end{lem}

We generalize this region $H_{x,y,z}$ by putting some triangular holes in it. For any non-negative integers $a,m,n,k, a_1, a_2,\cdots,a_{k-1},$ and $a_k$, we set $b_i=\sum_{j=i}^{k}a_j$ for all positive integers $i=1,\cdots,k$. We now define the two regions $H_{e}(a,m,n,k:a_1,\cdots,a_k)$ and $H_{o}(a,m,n,k:a_1,\cdots,a_k)$.

\begin{figure}
    \centering
    \includegraphics[width=1.0\textwidth]{Figure4.pdf}
    \caption{Two regions $H_{e}(2,4,8,3:2,1,2)$ (left) and $H_{o}(2,4,8,3:2,1,2)$ (right) on the $(i,j)-$coordinate planes.}
    \label{fig:Figure_3}
\end{figure}

When $k=0$, we set $H_{e}(a,m,n,0:\cdot):=H_{2a,m,n}$. When $k\geq 1$, we first consider a hexagon with sides of length $2a+1, m+2b_1+k-1, n+k, 2a+2b_1, m+k,$ and $n+2b_1+k-1$ clockwise from the left. Then, we delete the consecutive triangles of side lengths $1, 2a_1+1, 1, 2a_2+1,\cdots,1, 2a_{k-1}+1, 1, 2a_{k}$ on the perpendicular bisector of the left side from the left so that they alternate the orientation, where the first one is right-pointing. We denote this region by $H_{e}(a,m,n,k:a_1,\cdots,a_k)$ (see the left picture in Figure \ref{fig:Figure_3}). The collection of $2k$ triangles we removed is an \textit{intrusion} of the region.

Similarly, when $k=0$, we set $H_{o}(a,m,n,0:\cdot):=H_{2a+1, m, n}$. When $k\geq1$, we consider a hexagon with sides of length $2a+1, m+2b_1+k, n+k, 2a+2b_1+1, m+k$, and $n+2b_1+k$ clockwise from the left. Then, we delete the consecutive triangles of side lengths $1, 2a_1+1, 1, 2a_2+1,\cdots,1, 2a_{k-1}+1, 1, 2a_k+1$ on the perpendicular bisector of the left side from the left so that they alternate the orientation, where the first one is right-pointing. We denote this region by $H_{o}(a,m,n,k:a_1,\cdots,a_k)$ (see the right picture in Figure \ref{fig:Figure_3}). Again, an \textit{intrusion} of this region is the collection of $2k$ triangles we removed.

We also assign certain weights to lozenges. Following the weight assignment on $H_{x,y,z}$, we put the regions $H_{e}(a,m,n,k:a_1,\cdots,a_k)$ and $H_{o}(a,m,n,k:a_1,\cdots,a_k)$ on $(i,j)-$plane as indicated in Figure \ref{fig:Figure_3}. As before, we put the regions so that the $j-$axis overlaps with the perpendicular bisectors of the left sides. Then, we give a weight $\frac{q^i+q^{-i}}{2}$ to all horizontal lozenges whose centers have $i-$coordinate $i$ and give a weight $1$ to all non-horizontal lozenges. Under this weight assignment, let $M_q(H_{e}(a,m,n,k:a_1,\cdots,a_k))$ and $M_q(H_{o}(a,m,n,k:a_1,\cdots,a_k))$ be the TGFs of the two regions. Note that if we interchange $m$ and $n$, the TGFs are invariant. This is because we assigned weights on lozenges so that it is symmetric with respect to the $j-$axis. Hence, without loss of generality, we will assume $n\geq m$ in the theorem below.

\begin{thm}\label{thm:Theorem1}
For any non-negative integers $a,m,n,k, a_1, a_2,\cdots,a_{k-1},$ and $a_k$ such that $n\geq m$, we have

\begin{equation}\label{eqn:eq22}
\begin{aligned}
    &\frac{M_q(H_{e}(a,m,n,k:a_1,\cdots,a_k))}{M_q(H_{2a+2b_1, m+k, n+k})}\\
    &
\begin{aligned}
    =\prod_{i=0}^{k-1}\Bigg[&\frac{(\langle b_{1}+i+\frac{1}{2}\rangle_{q^2})_{m+k-b_{1}-2i}(\langle a+b_1-b_{i+1}+i+1\rangle_{q^2})_{m+k+2b_{i+1}-2i-1}}{(\langle a+b_1+i+\frac{1}{2}\rangle_{q^2})_{m+k-2i}(\langle i+1\rangle_{q^2})_{m+k+b_{i+1}-2i-1}}\\
    &\cdot \frac{(\langle a+n+k+b_1-i\rangle_{q^2})_{b_{i+1}}}{2^{4b_{i+1}}(\langle i+1\rangle_{q^2})_{b_1}(\langle n+k-i-\frac{1}{2}\rangle_{q^2})_{b_{i+1}}(\langle a+b_1+i+1\rangle_{q^2})_{-b_{i+1}}}\\
    &\cdot \frac{(\langle m+k-i+\frac{1}{2}\rangle_{q^2})_{\lfloor\frac{n-m}{2}\rfloor}(\langle n+k-i\rangle_{q^2})_{-\lfloor\frac{n-m}{2}\rfloor}}{(\langle a+m+k+b_{1}-i+\frac{1}{2}\rangle_{q^2})_{\lfloor\frac{n-m}{2}\rfloor}(\langle a+n+k+b_{1}-i\rangle_{q^2})_{-\lfloor\frac{n-m}{2}\rfloor}}\\
    &\cdot \frac{(\langle m+k-i\rangle_{q^2})_{\lfloor\frac{n-m}{2}\rfloor}(\langle n+k-i-\frac{1}{2}\rangle_{q^2})_{-\lfloor\frac{n-m}{2}\rfloor}}{(\langle m+k+b_{i+1}-i\rangle_{q^2})_{\lfloor\frac{n-m}{2}\rfloor}(\langle n+k+b_{i+1}-i-\frac{1}{2}\rangle_{q^2})_{-\lfloor\frac{n-m}{2}\rfloor}}
\end{aligned}
\\
    &\cdot \prod_{j=1}^{i}\Big[
    \frac{(\langle b_j-b_{i+1}+i+1-j\rangle_{q^2})^2}{(\langle i+1-j\rangle_{q^2})^2}\cdot(\langle b_{j+1}+i-j+\frac{1}{2}\rangle_{q^2})_{a_j}(\langle b_{j+1}+i-j+1\rangle_{q^2})_{a_j}\Big]\Bigg]
\end{aligned}
\end{equation}

and

\begin{equation}\label{eqn:eq23}
\begin{aligned}
    &\frac{M_q(H_{o}(a,m,n,k:a_1,\cdots,a_k))}{M_q(H_{2a+2b_1+1, m+k, n+k})}\\
    &
\begin{aligned}
    =\prod_{i=0}^{k-1}\Bigg[&\frac{(\langle a+b_1-b_{i+1}+i+1\rangle_{q^2})_{n+k+2b_{i+1}-2i}(\langle b_1+i+\frac{3}{2}\rangle_{q^2})_{n+k-b_1-2i-2}}{(\langle i+1\rangle_{q^2})_{n+k+b_{i+1}-2i-1}(\langle a+b_1+i+\frac{3}{2}\rangle_{q^2})_{n+k-2i-1}}\\
    &\cdot \frac{(\langle b_{i+1}+1\rangle_{q^2})(\langle a+m+k+b_1-i+1\rangle_{q^2})_{b_{i+1}}}{2^{4b_{i+1}+2}(\langle i+2\rangle_{q^2})_{b_1}(\langle m+k-i+\frac{1}{2}\rangle_{q^2})_{b_{i+1}}(\langle a+b_1+i+1\rangle_{q^2})_{-b_{i+1}}}\\
    &\cdot \frac{(\langle m+k-i\rangle_{q^2})_{\lfloor\frac{n-m}{2}\rfloor}(\langle n+k-i-\frac{1}{2}\rangle_{q^2})_{-\lfloor\frac{n-m}{2}\rfloor}}{(\langle a+m+k+b_1-i+1\rangle_{q^2})_{\lfloor\frac{n-m}{2}\rfloor}(\langle a+n+k+b_1-i+\frac{1}{2}\rangle_{q^2})_{-\lfloor\frac{n-m}{2}\rfloor}}\\
    &\cdot \frac{(\langle m+k-i+\frac{1}{2}\rangle_{q^2})_{\lfloor\frac{n-m}{2}\rfloor}(\langle n+k-i\rangle_{q^2})_{-\lfloor\frac{n-m}{2}\rfloor}}{(\langle m+k+b_{i+1}-i+\frac{1}{2}\rangle_{q^2})_{\lfloor\frac{n-m}{2}\rfloor}(\langle n+k+b_{i+1}-i\rangle_{q^2})_{-\lfloor\frac{n-m}{2}\rfloor}}\\
    &
    \begin{aligned}
        \cdot \prod_{j=1}^{i}&\Big[\frac{(\langle b_j-b_{i+1}+i+1-j\rangle_{q^2})^2}{(\langle i+1-j\rangle_{q^2})(\langle b_j+i+1-j\rangle_{q^2})}\frac{(\langle b_j+i+2-j\rangle_{q^2})}{(\langle i+2-j\rangle_{q^2})}\\
        &\cdot (\langle b_{j+1}+i-j+\frac{3}{2}\rangle_{q^2})_{a_j}(\langle b_{j+1}+i-j+2\rangle_{q^2})_{a_j}\Big]\Bigg]
    \end{aligned}
\end{aligned}
\end{aligned}
\end{equation}
\end{thm}

where $\langle n\rangle_q:=\frac{q^n-q^{-n}}{q-q^{-1}}$ for positive integer (or half-integer) $n$ and 

\begin{equation*}
    (\langle a\rangle_{q})_n:=
    \begin{cases}
        \langle a\rangle_{q}\langle a+1\rangle_{q}\cdots\langle a+n-1\rangle_{q}  & \text{if $n$ is positive,}\\
        1                    & \text{if $n$ is $0$,}\\
        \frac{1}{\langle a-1\rangle_{q}\langle a-2\rangle_{q}\cdots\langle a+n\rangle_{q}} & \text{if $n$ is negative,}
    \end{cases}
\end{equation*}
for any positive integer (or half-integer) $a$ and any integer $n$ such that $a+n>0$.

Note that by combining \eqref{eqn:eq21} and \eqref{eqn:eq22} (or \eqref{eqn:eq23}), one can obtain a product formula for the tiling generating function of the region $H_{e}(a,m,n,k:a_1,\cdots,a_k)$ (or $H_{o}(a,m,n,k:a_1,\cdots,a_k)$). Hence, one can view our results as tiling generating functions of these two regions. The case when $q=0$ and $k=0$ gives MacMahon's formula~\cite{M} and the case when $q=0$ and $a_{i}=0$ for all $i$ was considered by the first author~\cite{B}.

\section{Some previous results}

In this section, we recall the result of Lai and Rohatgi that gives TGFs of the two families of regions $R_{\mathbf{l},n,x}$ and $\overline{R}_{\mathbf{l},n,x}$ that we define below. In their notations, $\mathbf{l}=\{l_1,l_2,\cdots,l_m\}$ is a set of positive integers or an empty set, where elements are written in increasing order, $n$ is a non-negative integer, and $x$ is an integer that makes all the side lengths of the regions non-negative.

We first define the region $R_{\mathbf{l},n,x}$. Let \emph{O} be any lattice point on the triangular lattice and \emph{$\Bar{O}$} be the lattice point that is one unit to the northwest from \emph{O}. Consider the horizontal line \emph{l} through \emph{O} (which is not a lattice line) and let \emph{A} be the $n$th lattice point right to \emph{O}, which is on \emph{l} (when $n=0$, \emph{A}=\emph{O}). Similarly, consider the horizontal line $\Bar{\emph{l}}$ through \emph{$\Bar{O}$} and let \emph{B} be the $l_m$th lattice point left to \emph{$\Bar{O}$} which is on $\Bar{\emph{l}}$ (when $\mathbf{l}=\emptyset$, \emph{B}=\emph{$\Bar{O}$}). Then, we label left-pointing unit-triangles on $\Bar{\emph{l}}$ left to \emph{$\Bar{O}$} by $1,2,3,\cdots$ from the right to the left. The region $R_{\mathbf{l},n,x}$ is defined differently, depending on whether $\mathbf{l}$ is an empty set or not (see Figure 3.1 for two examples).

\begin{figure}
    \centering
    \includegraphics[width=16.5cm]{Figure5.pdf}
    \caption{The regions $R_{\emptyset,8,3}$ (left) and $R_{\{1,3,6\},2,3}$ (right) on the $(i,j)-$planes.}
\end{figure}

\begin{figure}
    \centering
    \includegraphics[width=16.5cm]{Figure6.pdf}
    \caption{The regions $\overline{R}_{\mathbf\{1,2,5,7,8\},0,3}$ (left) and $\overline{R}_{\mathbf\{1,3,6\},2,3}$ (right) on the $(i,j)-$planes.}
\end{figure}

When $\mathbf{l}=\emptyset$, the region $R_{\emptyset,n,x}$ is defined as follows without using \emph{$\Bar{O}$} and \emph{B}. From \emph{A} to \emph{O}, we follow the zigzag line along \emph{l}, by alternating moving one unit to the southwest and one unit to the northwest. Then, from \emph{O}, we move $x+1$ units to the north, $n$ units to the northeast, $n$ units to the southeast, and then $x+1$ units to the south until we reach \emph{A}. For any integer $x\geq-1$, $R_{\emptyset,n,x}$ is defined to be the bounded region enclosed by the path described above.

When $\mathbf{l}\neq\emptyset$, from \emph{A} to \emph{O}, we follow the zigzag line along \emph{l}, as we did in the previous case. Then, from \emph{O} to \emph{$\Bar{O}$}, we move one unit to the north and then one unit to the southwest. Now, we connect \emph{$\Bar{O}$} and \emph{B} using the same type of zigzag line along $\Bar{\emph{l}}$. From \emph{B}, we move $x$ units to the north, $2l_m-m+n+1$ units to the northeast, $m+n$ units to the southeast, and then $x+l_m-m+1$ units to the south until we reach \emph{A}. From the bounded region enclosed by the path described above, we delete the labeled left-pointing unit triangles on $\Bar{\emph{l}}$ whose labels are in $[l_m]\setminus\mathbf{l}$.\footnote{For any positive integer $k$, $[k]:=\{1,2,\cdots,k\}$.} For any non-negative integer $x$, $R_{\mathbf{l},n,x}$ is defined to be the resulting region.

The region $\overline{R}_{\mathbf{l},n,x}$ is defined differently, depending on whether $n$ is zero or not (see Figure 3.2). Its construction is almost identical to $R_{\mathbf{l},n,x}$. The differences are

1) we connect \emph{O} and \emph{$\Bar{O}$} using the unit segment connecting them, and

2-1) when $n=0$, from \emph{B}, we move $x$ units to the north, $2l_m-m$ units to the northeast, $m$ units to the southeast, and then $x+l_m-m$ units to the south until we reach \emph{$\Bar{O}$}. Then, we removed the labeled left-pointing unit-triangles on $\Bar{\emph{l}}$ whose labels are in $[l_m]\setminus\mathbf{l}$ (so \emph{O} and \emph{A} are not used when we define the region $\overline{R}_{\mathbf{l},0,x}$),

2-2) when $n\geq1$, from \emph{B}, we move $x$ units to the north, $2l_m-m+n$ units to the northeast, $m+n+1$ units to the southeast, and then $x+l_m-m$ units to the south until we reach \emph{A}. Then, we removed the labeled left-pointing unit triangles on $\Bar{\emph{l}}$ whose labels are in $[l_m]\setminus\mathbf{l}$.

To assign weights on lozenges, we put the regions $R_{\mathbf{l},n,x}$ and $\overline{R}_{\mathbf{l},n,x}$ on $(i,j)$-coordinate system as indicated in Figures 3.1 and 3.2. Then, we assign weights on lozenges using the same criteria as before (giving weight $\frac{q^i+q^{-i}}{2}$ to all horizontal lozenges whose centers have $i-$coordinate $i$ and giving weight $1$ to all non-horizontal lozenges) except the lozenges on the $j$-axis. According to the criteria, those lozenges on the $j$-axis are given a weight $\frac{q^{0}+q^{0}}{2}=1$. However, instead of giving a weight $1$, we assign a weight $\frac{1}{2}$ to those lozenges. The lozenges weighted by $\frac{1}{2}$ are marked by shaded ellipses on the pictures in this paper. Under this weight assignment, let $M_{q}(R_{\mathbf{l},n,x})$ and $M_{q}(\overline{R}_{\mathbf{l},n,x})$ be the TGFs of the regions $R_{\mathbf{l},n,x}$ and $\overline{R}_{\mathbf{l},n,x}$, respectively. Ciucu first studied the lozenge tilings of these regions in ~\cite{
C2}. Recently, Lai and Rohatgi ~\cite{LR2} found the TGFs of these regions under the weight assignment as described above\footnote{The regions $R_{\mathbf{l},n,x}$ and $\overline{R}_{\mathbf{l},n,x}$ correspond to the regions $T_{x,n,l_{m}}(\mathbf{l};[n])$ and $S_{x,n,l_{m}}(\mathbf{l};[n])$ in their paper.}.
%In the below formulas, when $\mathbf{l}=\emptyset$, $l_0$ is understood as $0$.

\begin{thm} [\cite{LR2}, Theorem 2.12 and 2.13]
    The TGFs of the regions $R_{\mathbf{l},n,x}$ and $\overline{R}_{\mathbf{l},n,x}$ are given by the following product formulas:\footnote{The formulas presented in this paper looks different from the original formulas of Lai and Rohatgi and it is because two papers are using different $q$-analogue of integers. While they used $[n]_{q}\coloneqq\frac{1-q^{n}}{1-q}$, we are using $<n>_{q}\coloneqq\frac{q^{n}-q^{-n}}{q-q^{-1}}$.}
    \begin{equation}
    \begin{aligned}
        M_{q}(R_{\mathbf{l},n,x})=&2^{-e}\frac{\prod_{1\leq i<j\leq m}<2(l_{j}-l_{i})>_{q}\prod_{1\leq i<j\leq n}<2(j-i)>_{q}}{\prod_{i=1}^{m}\prod_{j=1}^{n}<2(l_{i}+j)>_{q}\prod_{i=1}^{m}<2l_{i}>_{q}!\prod_{j=1}^{n}<2j-1>_{q}!}\\
        &\cdot\prod_{i=1}^{\lceil n/2\rceil}\prod_{j=1}^{2n-4i+3}<2x+2l_{m}-2m+2i+j>_{q}\\
        &\cdot\prod_{i=1}^{m}<2(x+l_{m}-m+n+i)>_{q}<2(x+l_{m}-m+n+i+1)>_{q}\\
        &\cdot\prod_{i=1}^{m}\prod_{j=1}^{n+i-1}<2x+2l_{m}-2m+n+i+j+1>_{q}\\
        &\cdot\prod_{i=1}^{n}\prod_{j=1}^{m}\frac{<2(x+l_{m}-m+i+j-1)>_{q}}{<2(x+l_{m}-m+i+j-1)+1>_{q}}\\
        &\cdot\prod_{i=1}^{m}\prod_{j=1}^{l_{i}-i}<2(x+l_{m}-m+n+i+j+1)>_{q}<2(x+l_{m}-i-j+1)>_{q}
    \end{aligned}
    \end{equation}
    and
    \begin{equation}
    \begin{aligned}
        M_{q}(\overline{R}_{\mathbf{l},n,x})=&2^{-\overline{e}}\frac{\prod_{1\leq i<j\leq m}<2(l_{j}-l_{i})>_{q}\prod_{1\leq i<j\leq n}<2(j-i)>_{q}}{\prod_{i=1}^{m}\prod_{j=1}^{n}<2(l_{i}+j)>_{q}\prod_{i=1}^{m}<2l_{i}-1>_{q}!\prod_{j=1}^{n}<2j>_{q}!}\\
        &\cdot\prod_{i=1}^{\lceil m/2\rceil}\prod_{j=1}^{2m-4i+3}<2x+2l_{m}-2m+2i+j-1>_{q}\\
        &\cdot\prod_{i=1}^{n}<2(x+l_{m}+i)>_{q}\prod_{i=1}^{n}\prod_{j=1}^{m+i}<2x+2l_{m}-m+i+j>_{q}\\
        &\cdot\prod_{i=1}^{m}\prod_{j=1}^{n}\frac{<2(x+l_{m}-m+i+j-1)>_{q}}{<2(x+l_{m}-m+i+j-1)+1>_{q}}\\
        &\cdot\prod_{i=1}^{m}\prod_{j=1}^{l_{i}-i}<2(x+l_{m}-m+i+j+n)>_{q}<2(x+l_{m}-i-j+1)>_{q}
    \end{aligned}
    \end{equation}
where
$e=\frac{n(n+1)}{2}+(n+1)m+\sum^{m}_{i=1}(2l_i-i)$ \text{and} $\overline{e}=\frac{n(n+1)}{2}+n(m+1)+\sum^{m}_{i=1}(2l_i-i)$.\\
Also, when $\mathbf{l}=\emptyset$, we set $m=0$ and $l_{0}=0$ by convention.
\end{thm}
%$e=\frac{(n+1)m(4x+4l_{m}-3m+n+3)}{2}+\sum_{i=1}^{m}\sum^{2x+1+2l_m-m-(m-i+1)}_{j=2x+1+2(l_m-l_i)-2(m-i)}j+\sum_{i=1}^{n}\sum_{j=2(x+l_m-m+1)}^{2(x+l_m-m+1)+i-1}j$\\and $\overline{e}=\frac{n(m+1)(4x+4l_{m}-3m+n+1)}{2}+\sum_{i=1}^{m}\sum_{j=2x+1+2(l_m-l_i)-2(m-i)}^{2x+1+2l_m-2m+i-1}j+\sum_{i=1}^{n}\sum_{j=2(x+l_{m}-m)}^{2(x+l_{m}-m)+i-1}j$.\\ Also, when $\mathbf{l}=\emptyset$, we set $m=0$ and $l_{0}=0$ by convention.

We also recall the Matching Factorization Theorem of Ciucu~\cite{C1}. Instead of stating the theorem in full generality, we will present how the theorem is applied to the regions $H_{e}(a,m,m,k:a_1,\cdots,a_k)$ and $H_{o}(a,m,m,k:a_1,\cdots,a_k)$. Note that the region $H_{e}(a,m,m,k:a_1,\cdots,a_k)$ is symmetric and has a horizontal symmetry axis. Furthermore, since $\frac{q^{i}+q^{-i}}{2}=\frac{q^{-i}+q^{i}}{2}$, the weight assignment on the lozenges in these regions are symmetric (in other words, two lozenges that are in symmetric positions have the same weight). On this region, we consider the midpoint of the left-pointing deleted triangle of side length $2a_k$ (the rightmost triangle of the intrusion). Starting from the midpoint, we follow the zigzag line along the symmetry axis, alternating its direction between the northeast and the southeast (see the left picture in Figure 3.3). The region is then split into two subregions by the zigzag line. We denote the region on the top by $H_{e}^{-}(a,m,m,k:a_1,\cdots,a_k)$. On the bottom region, we additionally put a weight $\frac{1}{2}$ to the horizontal lozenges that are placed on the symmetry axis of $H_{e}(a,m,m,k:a_1,\cdots,a_k)$ (originally, these lozenges were weighted by $\frac{q^{0}+q^{0}}{2}=1$). Then, we denote the region on the bottom by $H_{e}^{+}(a,m,m,k:a_1,\cdots,a_k)$. The Matching Factorization Theorem provides the following factorization of the TGF of $H_{e}(a,m,m,k:a_1,\cdots,a_k)$:

\begin{figure}
    \centering
    \includegraphics[width=16.5cm]{Figure7.pdf}
    \caption{Pictures that show how the Factorization Theorem is applied on the regions $H_{e}(2,4,4,3:2,1,2)$ and $H_{o}(2,4,4,3:2,1,2)$. Left picture shows how the Factorization Theorem split the region $H_{e}(2,4,4,3:2,1,2)$ into two subregions $H_{e}^{+}(2,4,4,3:2,1,2)$ (bottom) and $H_{e}^{-}(2,4,4,3:2,1,2)$ (top). The picture on the right presents $H_{o}^{+}(2,4,4,3:2,1,2)$ (bottom) and $H_{o}^{-}(2,4,4,3:2,1,2)$ (top) that can be obtained from $H_{o}(2,4,4,3:2,1,2)$ by applying the theorem.}
\end{figure}

\begin{equation}
\begin{aligned}
    &M_{q}(H_{e}(a,m,m,k:a_1,\cdots,a_k))\\
    &=2^{m}M_{q}(H_{e}^{+}(a,m,m,k:a_1,\cdots,a_k))M_{q}(H_{e}^{-}(a,m,m,k:a_1,\cdots,a_k)).
\end{aligned}
\end{equation}

\begin{figure}
    \centering
    \includegraphics[width=16.5cm]{Figure8.pdf}
    \caption{Pictures that show how the Factorization Theorem is applied on the regions $H_{6,9,9}$ and $H_{7,9,9}$. Left picture shows how the Factorization Theorem split the region $H_{6,9,9}$ into two subregions $H_{6,9,9}^{+}$ (bottom) and $H_{6,9,9}^{-}$ (top). The picture on the right present $H_{7,9,9}^{+}$ (bottom) and $H_{7,9,9}^{-}$ (top) that can be obtained from $H_{7,9,9}$ by applying the theorem.}
\end{figure}

We split the region $H_{o}(a,m,m,k:a_1,\cdots,a_k)$ in a similar manner (see the right picture in Figure 3.3. Since the midpoint of the left-pointing deleted triangle of side length $2a_k+1$ is not a lattice point, our zigzag line starts at the lattice point that lies just above the midpoint). Then, the Factorization Theorem provides us the following factorization of $M_{q}(H_{o}(a,m,m,k:a_1,\cdots,a_k))$:

\begin{equation}
\begin{aligned}
    &M_{q}(H_{o}(a,m,m,k:a_1,\cdots,a_k))\\
    &=2^{m}M_{q}(H_{o}^{+}(a,m,m,k:a_1,\cdots,a_k))M_{q}(H_{o}^{-}(a,m,m,k:a_1,\cdots,a_k)).
\end{aligned}
\end{equation}

We again apply the Factorization Theorem to the hexagonal region $H_{x,y,y}$. We split the region $H_{x,y,y}$ into two subregions using a zigzag line and give weight $\frac{1}{2}$ to lozenges on the symmetry axis (see Figure 3.4). Then $M_{q}(H_{x,y,y})$ has the following factorization:

\begin{equation}
    M_{q}(H_{x,y,y})=2^{y}M_{q}(H_{x,y,y}^{+})M_{q}(H_{x,y,y}^{-})
\end{equation}
where $H_{x,y,y}^{-}$ is a subregion above the zigzag line and $H_{x,y,y}^{+}$ is a subregion below the zigzag line with additional $\frac{1}{2}$ weight on horizontal lozenges on the symmetry axis.

The last theorems we need are two versions of Kuo's graphical condensation method~\cite{Kuo}. This method is an alternative combinatorial interpretation of the Kasteleyn-Percus method~\cite{Ka, Pe}, as pointed out by Fulmek~\cite{Fu1}. To state the theorem, denote a bipartite graph by $G=(V_1, V_2, E)$, where $E$ is the set of edges of the graph and $(V_1, V_2)$ is the partition of the vertex set of the graph $G$ such that every edge in $E$ connects a vertex in $V_1$ and a vertex in $V_2$. Also, for any set of vertices $\{x_1, x_2,...,x_n\}$, let $G-\{x_1, x_2,...,x_n\}$ be the graph obtained from $G$ by deleting $n$ vertices $x_1, x_2,...,x_n$ and all edges that are adjacent to at least one of $x_1, x_2,...,x_n$. If all the edges of a graph are weighted, \textit{the weight of a perfect matching}\footnote{\textit{A perfect matching} of a bipartite graph is a subset of edges such that every vertex is incident to precisely one edge.} is defined to be the product of weights of all edges that constitute the perfect matching. For any weighted graph $G$, let $M(G)$ be the sum of the weights of all perfect matchings.

\begin{thm}[\cite{Kuo}, Theorem 5.1]
Let $G=(V_1, V_2, E)$ be a weighted plane bipartite graph in which $|V_1|=|V_2|$. Let vertices $x, y, z,$ and $w$ appear in a cyclic order on a face of $G$. If $x,z \in V_1$ and $y,w \in V_2$, then
\begin{equation}
    M(G)M(G-\{x,y,z,w\})=M(G-\{x,y\})M(G-\{z,w\})+M(G-\{x,w\})M(G-\{y,z\}).
\end{equation}
\end{thm}

\begin{thm}[\cite{Kuo}, Theorem 5.3]
Let $G=(V_1, V_2, E)$ be a weighted plane bipartite graph in which $|V_1|=|V_2|+1$. Let vertices $x, y, z,$ and $w$ appear in a cyclic order on a face of $G$. If $x,y,z \in V_1$ and $w \in V_2$, then
\begin{equation}
    M(G-\{y\})M(G-\{x,z,w\})=M(G-\{x\})M(G-\{y,z,w\})+M(G-\{z\})M(G-\{x,y,w\}).
\end{equation}
\end{thm}

Using a well-known bijection between the set of lozenge tilings of a region and the set of perfect matchings on its dual graph, we will apply the above two theorems on the dual graphs of certain regions on a triangular grid. As a result, we will obtain recurrence relations that involve TGFs of lozenges tilings of six related regions on the grid. We will use these recurrence relations to give an inductive proof of our main theorem.

\section{Proof of Theorem 2.2}

The poof is organized as follows: first, using Ciucu's Matching Factorization Theorem and Lai and Rohatgi's result mentioned in Section 3 (Theorem 3.1), we prove the special case of the theorem when $n=m$. Then, using Kuo's graphical condensation method, we construct three recurrence relations and specialize them to prove another special case when $n=m+1$. We end the proof by proving the general case when $n\geq m+2$ using general forms of the three recurrence relations. When $k=0$, there is nothing to prove because (2.2) and (2.3) become $1=1$. Thus, in the proof, we assume that $k\geq1$.

We first prove the case when $n=m$. To prove this case, we need to show that the following equations (4.1) and (4.2) hold:

\begin{equation}
\begin{aligned}
    &\frac{M_q(H_{e}(a,m,m,k:a_1,\cdots,a_k))}{M_q(H_{2a+2b_1, m+k, m+k})}\\
    &
\begin{aligned}
    =\prod_{i=0}^{k-1}\Bigg[&\frac{(<b_{1}+i+\frac{1}{2}>_{q^2})_{m+k-b_{1}-2i}(<a+b_1-b_{i+1}+i+1>_{q^2})_{m+k+2b_{i+1}-2i-1}}{(<a+b_1+i+\frac{1}{2}>_{q^2})_{m+k-2i}(<i+1>_{q^2})_{m+k+b_{i+1}-2i-1}}\\
    &\cdot \frac{(<a+m+k+b_1-i>_{q^2})_{b_{i+1}}}{2^{4b_{i+1}}(<i+1>_{q^2})_{b_1}(<m+k-i-\frac{1}{2}>_{q^2})_{b_{i+1}}(<a+b_1+i+1>_{q^2})_{-b_{i+1}}}\\
    &\cdot \prod_{j=1}^{i}\Big[\frac{(<b_j-b_{i+1}+i+1-j>_{q^2})^2}{(<i+1-j>_{q^2})^2}\cdot(<b_{j+1}+i-j+\frac{1}{2}>_{q^2})_{a_j}(<b_{j+1}+i-j+1>_{q^2})_{a_j}\Big]\Bigg]
\end{aligned}
\end{aligned}
\end{equation}

and

\begin{equation}
\begin{aligned}
    &\frac{M_q(H_{o}(a,m,m,k:a_1,\cdots,a_k))}{M_q(H_{2a+2b_1+1, m+k, m+k})}\\
    &
\begin{aligned}
    =\prod_{i=0}^{k-1}\Bigg[&\frac{(<a+b_1-b_{i+1}+i+1>_{q^2})_{m+k+2b_{i+1}-2i}(<b_1+i+\frac{3}{2}>_{q^2})_{m+k-b_1-2i-2}}{(<i+1>_{q^2})_{m+k+b_{i+1}-2i-1}(<a+b_1+i+\frac{3}{2}>_{q^2})_{m+k-2i-1}}\\
    &\cdot \frac{(<b_{i+1}+1>_{q^2})(<a+m+k+b_1-i+1>_{q^2})_{b_{i+1}}}{2^{4b_{i+1}+2}(<i+2>_{q^2})_{b_1}(<m+k-i+\frac{1}{2}>_{q^2})_{b_{i+1}}(<a+b_1+i+1>_{q^2})_{-b_{i+1}}}\\
    &
    \begin{aligned}
        \cdot \prod_{j=1}^{i}&\Big[\frac{(<b_j-b_{i+1}+i+1-j>_{q^2})^2}{(<i+1-j>_{q^2})(<b_j+i+1-j>_{q^2})}\frac{(<b_j+i+2-j>_{q^2})}{(<i+2-j>_{q^2})}\\
        &\cdot (<b_{j+1}+i-j+\frac{3}{2}>_{q^2})_{a_j}(<b_{j+1}+i-j+2>_{q^2})_{a_j}\Big]\Bigg].
    \end{aligned}
\end{aligned}
\end{aligned}
\end{equation}

We first show (4.1). The left side of (4.1) can be expressed as follows\footnote{When $i=1$, the sequence $(a_1,\cdots,a_{i-1}, b_{i})$ is understood as $(b_{1})$}:

\begin{equation}
\begin{aligned}
    &\frac{M_{q}(H_{e}(a,m,m,k:a_1,\cdots,a_k))}{M_{q}(H_{2a+2b_1, m+k, m+k})}\\
    &
\begin{aligned}
    =&\Bigg[\prod_{i=1}^{k-1}\frac{M_{q}(H_{e}(a,m+k-i-1,m+k-i-1,i+1:a_1,\cdots,a_{i},b_{i+1} ))}{M_{q}(H_{e}(a,m+k-i,m+k-i,i:a_1,\cdots,a_{i-1}, b_{i}))}\Bigg]\\
    &\cdot\frac{M_{q}(H_{e}(a,m+k-1,m+k-1,1:b_1))}{M_{q}(H_{2a+2b_1, m+k, m+k})}    
\end{aligned}
\end{aligned}
\end{equation}

We investigate each ratio on the right side of (4.3). For any $i\in[k]$, by the Matching Factorization Theorem, we know that the TGF $M_{q}(H_{e}(a,m+k-i,m+k-i,i:a_1,\cdots,a_{i-1}, b_{i}))$ can be factorized as follows:

\begin{equation}
\begin{aligned}
    &M_{q}(H_{e}(a,m+k-i,m+k-i,i:a_1,\cdots,a_{i-1}, b_{i}))\\
    &
\begin{aligned}    
    =2^{m+k-i}&M_{q}(H_{e}^{+}(a,m+k-i,m+k-i,i:a_1,\cdots,a_{i-1}, b_{i}))\\
    &\cdot M_{q}(H_{e}^{-}(a,m+k-i,m+k-i,i:a_1,\cdots,a_{i-1}, b_{i})).
\end{aligned}
\end{aligned}
\end{equation}

Similarly, the TGF $M_{q}(H_{2a+2b_1, m+k, m+k})$ also has a similar factorization as below:

\begin{equation}
    M_{q}(H_{2a+2b_1, m+k, m+k})=2^{m+k}M_{q}(H_{2a+2b_1, m+k, m+k}^{+})M_{q}(H_{2a+2b_1, m+k, m+k}^{-}).
\end{equation}

Now, we claim that the TGFs of the regions $H_{e}^{+}(a,m+k-i,m+k-i,i:a_1,\cdots,a_{i-1}, b_{i})$, $H_{e}^{-}(a,m+k-i,m+k-i,i:a_1,\cdots,a_{i-1}, b_{i})$, $H_{2a+2b_1, m+k, m+k}^{+}$, and $H_{2a+2b_1, m+k, m+k}^{-}$ are the same as that of certain $R$ regions and $\overline{R}$ regions that we mentioned in the previous section. For any $i\in[k]$, we have

\begin{equation}
\begin{aligned}
    &M_{q}(H_{e}^{+}(a,m+k-i,m+k-i,i:a_1,\cdots,a_{i-1}, b_{i}))\\
    &=M_{q}(R_{\{b_{i}, b_{i-1}+1, \cdots, b_{1}+i-1\},m+k-i,a}),
\end{aligned}
\end{equation}

\begin{equation}
\begin{aligned}
    &M_{q}(H_{e}^{-}(a,m+k-i,m+k-i,i:a_1,\cdots,a_{i-1}, b_{i}))\\
    &=M_{q}(\overline{R}_{\{1,2,\cdots,m+k-i-1,m+k-i+b_{i},m+k-i+b_{i-1}+1,\cdots,m+k+b_{1}-1\},0,a}),
\end{aligned}
\end{equation}

\begin{equation}
    M_{q}(H_{2a+2b_1, m+k, m+k}^{+})=M_{q}(R_{\emptyset,m+k,a+b_1-1}),
\end{equation}

and

\begin{equation}
    M_{q}(H_{2a+2b_1, m+k, m+k}^{-})=M_{q}(\overline{R}_{[m+k-1],0,a+b_1}).
\end{equation}

To check the above identities, we first need to observe that one can obtain identical regions from the corresponding regions after removing forced lozenges\footnote{A \textit{forced lozenge} on a region $R$ is a lozenge-shaped tile that is contained in all tilings of $R$. In this paper, we indicate forced lozenges by shading them. Suppose the region $R$ has a forced lozenge, denoted by $L$, that has weight $wt(L)$. Since $L$ is always part of the tilings of $R$, one can check that $M(R-L)=M(R)/wt(L)$ holds. Especially, we have $M(R-L)=M(R)$ when $wt(L)=1$. In this paper, all forced lozenges have a weight $1$, so TGFs are unchanged after removing them from regions.}  (see four pictures in the first column in Figure 4.1 that illustrate (4.6) and (4.7)). Furthermore, all the removed lozenges are not horizontal; hence, they have a weight of $1$, so the TGFs are unchanged after removing them. Also, since the weight assignment is symmetric about $j$-axis, all the regions and their mirror images about $j$-axis have the same TGFs. Therefore, the TGFs of the corresponding regions are the same.

\begin{figure}
    \centering
    \includegraphics[width=16.5cm]{Figure9.pdf}
    \caption{The first and the second pictures in the first column show $\overline{R}_{\{1,2,3,6,8,11\},0,2}$ and $H_{e}^{-}(2,4,4,3:2,1,2)$. The third and the fourth pictures in the first column show $H_{e}^{+}(2,4,4,3:2,1,2)$ and a mirror image of $R_{\{2,4,7\},4,2}$. The first and the second pictures in the second column show $\overline{R}_{\{1,2,3,4,7,9,12\},0,2}$ and $H_{o}^{-}(2,4,4,3:2,1,2)$. The third and the fourth pictures in the second column show $H_{o}^{+}(2,4,4,3:2,1,2)$ and a mirror image of $R_{\{3,5,8\},3,2}$. Forced lozenges are represented by shaded lozenges. One can compare the corresponding regions and check that they have the same TGFs.}
\end{figure}

Using Theorem 3.1, each factor on the right side of (4.3) could be simplified as follows:

\begin{equation}
\begin{aligned}
    &\frac{M_{q}(H_{e}(a,m+k-i-1,m+k-i-1,i+1:a_1,\cdots,a_{i},b_{i+1}))}{M_{q}(H_{e}(a,m+k-i,m+k-i,i:a_1,\cdots,a_{i-1}, b_{i}))}\\
    &
\begin{aligned}
    =&\frac{2^{m+k-i-1}}{2^{m+k-i}}\frac{M_{q}(H_{e}^{+}(a,m+k-i-1,m+k-i-1,i+1:a_1,\cdots,a_{i},b_{i+1}))}{M_{q}(H_{e}^{+}(a,m+k-i,m+k-i,i:a_1,\cdots,a_{i-1}, b_{i}))}\\
    &\cdot\frac{M_{q}(H_{e}^{-}(a,m+k-i-1,m+k-i-1,i+1:a_1,\cdots,a_{i},b_{i+1}))}{M_{q}(H_{e}^{-}(a,m+k-i,m+k-i,i:a_1,\cdots,a_{i-1}, b_{i}))}
\end{aligned}
    \\
    &
\begin{aligned}
    =&\frac{1}{2}\frac{M_{q}(R_{\{b_{i+1}, b_{i}+1, \cdots, b_{1}+i\},m+k-i-1,a})}{M_{q}(R_{\{b_{i}, b_{i-1}+1, \cdots, b_{1}+i-1\},m+k-i,a})}\\
    &\cdot\frac{M_{q}(\overline{R}_{\{1,2,\cdots,m+k-i-2,m+k-i+b_{i+1}-1,m+k-i+b_{i},\cdots,m+k+b_{1}-1\},0,a})}{M_{q}(\overline{R}_{\{1,2,\cdots,m+k-i-1,m+k-i+b_{i},m+k-i+b_{i-1}+1,\cdots,m+k+b_{1}-1\},0,a})}
\end{aligned}    
    \\
    &
\begin{aligned}
    =&\frac{(<b_{1}+i+\frac{1}{2}>_{q^2})_{m+k-b_{1}-2i}(<a+b_1-b_{i+1}+i+1>_{q^2})_{m+k+2b_{i+1}-2i-1}}{(<a+b_1+i+\frac{1}{2}>_{q^2})_{m+k-2i}(<i+1>_{q^2})_{m+k+b_{i+1}-2i-1}}\\
    &\frac{(<a+m+k+b_1-i>_{q^2})_{b_{i+1}}}{2^{4b_{i+1}}(<i+1>_{q^2})_{b_1}(<m+k-i-\frac{1}{2}>_{q^2})_{b_{i+1}}(<a+b_1+i+1>_{q^2})_{-b_{i+1}}}\\
    &\prod_{j=1}^{i}\Big[\frac{(<b_j-b_{i+1}+i+1-j>_{q^2})^2}{(<i+1-j>_{q^2})^2}\cdot(<b_{j+1}+i-j+\frac{1}{2}>_{q^2})_{a_j}(<b_{j+1}+i-j+1>_{q^2})_{a_j}\Big]    
\end{aligned}
\end{aligned}
\end{equation}

and

\begin{equation}
\begin{aligned}
    &\frac{M_{q}(H_{e}(a,m+k-1,m+k-1,1:b_1))}{M_{q}(H_{2a+2b_1, m+k, m+k})}\\
    &=\frac{2^{m+k-1}}{2^{m+k}}\frac{M_{q}(H_{e}^{+}(a,m+k-1,m+k-1,1:b_1))}{M_{q}(H_{2a+2b_1, m+k, m+k}^{+})}\frac{M_{q}(H_{e}^{-}(a,m+k-1,m+k-1,1:b_1))}{M_{q}(H_{2a+2b_1, m+k, m+k}^{-})}\\
    &=\frac{1}{2}\frac{M_{q}(R_{\{b_{1}\},m+k-1,a})}{M_{q}(R_{\emptyset,m+k,a+b_1-1})}\frac{M_{q}(\overline{R}_{\{1,2,\cdots,m+k-2,m+k+b_{1}-1\},0,a})}{M_{q}(\overline{R}_{[m+k-1],0,a+b_1})}\\
    &
    \begin{aligned}
        =&\frac{(<b_{1}+\frac{1}{2}>_{q^2})_{m+k-b_{1}}(<a+1>_{q^2})_{m+k+2b_{1}-1}}{(<a+b_1+\frac{1}{2}>_{q^2})_{m+k}(<1>_{q^2})_{m+k+b_{1}-1}}\\
        &\cdot\frac{(<a+m+k+b_1>_{q^2})_{b_{1}}}{2^{4b_{1}}(<1>_{q^2})_{b_1}(<m+k-\frac{1}{2}>_{q^2})_{b_{1}}(<a+b_1+1>_{q^2})_{-b_{1}}}.
    \end{aligned}
\end{aligned}
\end{equation}

Thus, by combining (4.3), (4.10), and (4.11), we get (4.1).

The proof of (4.2) is almost the same as that of (4.1). Instead of (4.6)-(4.9), we use the following identities (see four pictures in the second column in Figure 4.1 that illustrate (4.12) and (4.13)):

\begin{equation}
\begin{aligned}
    &M_{q}(H_{o}^{+}(a,m+k-i,m+k-i,i:a_1,\cdots,a_{i-1}, b_{i}))\\
    &=M_{q}(R_{\{b_{i}+1, b_{i-1}+2, \cdots, b_{1}+i\},m+k-i-1,a}),
\end{aligned}
\end{equation}

\begin{equation}
\begin{aligned}
    &M_{q}(H_{o}^{-}(a,m+k-i,m+k-i,i:a_1,\cdots,a_{i-1}, b_{i}))\\
    &=M_{q}(\overline{R}_{\{1,2,\cdots,m+k-i,m+k-i+b_{i}+1,m+k-i+b_{i-1}+2,\cdots,m+k+b_{1}\},0,a}),
\end{aligned}
\end{equation}

\begin{equation}
    M_{q}(H_{2a+2b_1+1, m+k, m+k}^{+})=M_{q}(R_{\emptyset,m+k-1,a+b_1}),
\end{equation}

and

\begin{equation}
    M_{q}(H_{2a+2b_1+1, m+k, m+k}^{-})=M_{q}(\overline{R}_{[m+k],0,a+b_1}).
\end{equation}


If one follows the proof of (4.1) (applying the Matching Factorization Theorem and using Lai and Rohatgi's result, Theorem 3.1), the proof of (4.2) follows. This completes the proof of (4.1) and (4.2), which are special cases of Theorem 2.2 when $n=m$.

Now we move on to the proof of the following case when $n=m+1$. Our goal is to prove the following two equations:

\begin{equation}
\begin{aligned}
    &\frac{M_{q}(H_{e}(a,m,m+1,k:a_1,\cdots,a_k))}{M_{q}(H_{2a+2b_1, m+k, m+k+1})}\\
    &
\begin{aligned}
    =\prod_{i=0}^{k-1}\Bigg[&\frac{(<b_{1}+i+\frac{1}{2}>_{q^2})_{m+k-b_{1}-2i}(<a+b_1-b_{i+1}+i+1>_{q^2})_{m+k+2b_{i+1}-2i-1}}{(<a+b_1+i+\frac{1}{2}>_{q^2})_{m+k-2i}(<i+1>_{q^2})_{m+k+b_{i+1}-2i-1}}\\
    &\cdot \frac{(<a+m+k+b_1-i+1>_{q^2})_{b_{i+1}}}{2^{4b_{i+1}}(<i+1>_{q^2})_{b_1}(<m+k-i+\frac{1}{2}>_{q^2})_{b_{i+1}}(<a+b_1+i+1>_{q^2})_{-b_{i+1}}}\\
    &\cdot \prod_{j=1}^{i}\Big[\frac{(<b_j-b_{i+1}+i+1-j>_{q^2})^2}{(<i+1-j>_{q^2})^2}\cdot(<b_{j+1}+i-j+\frac{1}{2}>_{q^2})_{a_j}(<b_{j+1}+i-j+1>_{q^2})_{a_j}\Big]\Bigg]
\end{aligned}
\end{aligned}
\end{equation}
and
\begin{equation}
\begin{aligned}
    &\frac{M_{q}(H_{o}(a,m,m+1,k:a_1,\cdots,a_k))}{M_{q}(H_{2a+2b_1+1, m+k, m+k+1})}\\
    &
    \begin{aligned}
    =\prod_{i=0}^{k-1}\Bigg[&\frac{(<a+b_1-b_{i+1}+i+1>_{q^2})_{m+k+2b_{i+1}-2i+1}(<b_1+i+\frac{3}{2}>_{q^2})_{m+k-b_1-2i-1}}{(<i+1>_{q^2})_{m+k+b_{i+1}-2i}(<a+b_1+i+\frac{3}{2}>_{q^2})_{m+k-2i}}\\
    &\cdot \frac{(<b_{i+1}+1>_{q^2})(<a+m+k+b_1-i+1>_{q^2})_{b_{i+1}}}{2^{4b_{i+1}+2}(<i+2>_{q^2})_{b_1}(<m+k-i+\frac{1}{2}>_{q^2})_{b_{i+1}}(<a+b_1+i+1>_{q^2})_{-b_{i+1}}}\\
    &
    \begin{aligned}
        \cdot \prod_{j=1}^{i}&\Big[\frac{(<b_j-b_{i+1}+i+1-j>_{q^2})^2}{(<i+1-j>_{q^2})(<b_j+i+1-j>_{q^2})}\frac{(<b_j+i+2-j>_{q^2})}{(<i+2-j>_{q^2})}\\
        &\cdot (<b_{j+1}+i-j+\frac{3}{2}>_{q^2})_{a_j}(<b_{j+1}+i-j+2>_{q^2})_{a_j}\Big]\Bigg].
    \end{aligned}
\end{aligned}
\end{aligned}
\end{equation}

\begin{figure}
    \centering
    \includegraphics[width=15cm]{Figure10.pdf}
    \caption{Pictures that show three regions and choices of four vertices $x,y,z,$ and $w$ on each region.}
\end{figure}

We will construct three recurrence relations using Kuo's graphical condensation method to give an inductive proof. First, we consider a region $H_{o}(a,m+1,n,k-1:a_1,\cdots,a_{k-1})$ on the $(i,j)-$plane and choose four unit-triangles $x,y,z,$ and $w$ as described in Figure 4.2 (see the top-left picture in the figure). Then, we consider the dual graph of the region and four vertices of the dual graph that correspond to four unit-triangles $x,y,z,$ and $w$. They satisfy the conditions in Theorem 3.2, so we can apply the theorem on the dual graph. However, using a well-known bijection between the set of lozenge tilings of a region and the set of perfect matchings on its dual graph, we also obtain an identity that involves TGFs of $H_{o}(a,m+1,n,k-1:a_1,\cdots,a_{k-1})$ and the same region with some unit triangles removed (see ~\cite{B} for more explanation about this bijection). As one can see in Figure 4.3, the removal of each unit triangle generates some forced lozenges. However, since all those forced lozenges are not horizontal, they are weighted by $1$, and removing them from the regions does not change TGFs\footnote{This argument will be used again when we construct the remaining two recurrence relations.}. Therefore, we obtain the following recurrence relation (see Figure 4.3, which shows how we obtain the six regions):

\begin{figure}
    \centering
    \includegraphics[width=13.5cm]{Figure11.pdf}
    \caption{Six regions appeared in the recurrence relation (4.18).}
\end{figure}

\begin{equation}
\begin{aligned}
    &M_{q}(H_{o}(a,m+1,n,k-1:a_1,\cdots,a_{k-1}))M_{q}(H_{e}(a,m,n-1,k:a_1,\cdots,a_{k-1},0))\\
    &
\begin{aligned}
    =&M_{q}(H_{e}(a,m+1,n-1,k:a_1,\cdots,a_{k-1},0))M_{q}(H_{o}(a,m,n,k-1:a_1,\cdots,a_{k-1}))\\
    &+M_{q}(H_{e}(a,m,n,k:a_1,\cdots,a_{k-1},0))M_{q}(H_{o}(a,m+1,n-1,k-1:a_1,\cdots,a_{k-1})).
\end{aligned}
\end{aligned}
\end{equation}

Now we consider the region $H_{e}(a,m+1,n,k:a_1,\cdots,a_k)$ on the $(i,j)-$plane with $a_k\geq 1$. From the region, we replace the removed left-pointing triangle of side length $2a_k$ with that of side length $2a_k-1$. Note that this new region contains one more left-pointing unit triangles than right-pointing unit triangles. From the new region, we choose four unit triangles $x,y,z,$ and $w$ as described in Figure 4.2 (see the top-right picture in the figure). Note that this time, the dual graph of the region and four vertices of the dual graph that correspond to four unit-triangles $x,y,z,$ and $w$ satisfy the conditions in Theorem 3.3. If we again use the bijection between the set of lozenge tilings of a region and the set of perfect matchings on its dual graph and remove forced lozenges weighted by 1 from regions, by applying Theorem 3.3, we have the following recurrence relation (see Figure 4.4, which shows how we obtain the six regions):

\begin{figure}
    \centering
    \includegraphics[width=13.5cm]{Figure12.pdf}
    \caption{Six regions appeared in the recurrence relation (4.19).}
\end{figure}

\begin{equation}
\begin{aligned}
    &M_{q}(H_{e}(a,m+1,n,k:a_1,\cdots,a_k))M_{q}(H_{o}(a,m+1,n,k:a_1,\cdots,a_k-1))\\
    &
\begin{aligned}
    =&M_{q}(H_{o}(a,m+2,n,k:a_1,\cdots,a_k-1))M_{q}(H_{e}(a,m,n,k:a_1,\cdots,a_k))\\
    &+M_{q}(H_{o}(a,m+1,n+1,k:a_1,\cdots,a_k-1))M_{q}(H_{e}(a,m+1,n-1,k:a_1,\cdots,a_k)).
\end{aligned}
\end{aligned}
\end{equation}

For the last recurrence relation, we consider the region $H_{o}(a,m+1,n,k:a_1,\cdots,a_k)$ on the $(i,j)-$plane (This time, we allow $a_k$ to be $0$). From the region, we replace the removed left-pointing triangle of side length $2a_k+1$ with that of side length $2a_k$. From the new region, we choose four unit triangles $x,y,z,$ and $w$ as described in Figure 4.2 (see the bottom picture in the figure). Suppose we apply the same argument in the previous paragraph on the new region and four unit triangles $x,y,z,$ and $w$. In that case, we can generate the following recurrence relation using Theorem 3.3 (see Figure 4.5, which shows how we obtain the six regions):

\begin{figure}
    \centering
    \includegraphics[width=13.5cm]{Figure13.pdf}
    \caption{Six regions appeared in the recurrence relation (4.20).}
\end{figure}

\begin{equation}
\begin{aligned}
    &M_{q}(H_{o}(a,m+1,n,k:a_1,\cdots,a_k))M_{q}(H_{e}(a,m+1,n,k:a_1,\cdots,a_k))\\
    &
\begin{aligned}
    =&M_{q}(H_{e}(a,m+2,n,k:a_1,\cdots,a_k))M_{q}(H_{o}(a,m,n,k:a_1,\cdots,a_k))\\
    &+M_{q}(H_{e}(a,m+1,n+1,k:a_1,\cdots,a_k))M_{q}(H_{o}(a,m+1,n-1,k:a_1,\cdots,a_k)).
\end{aligned}
\end{aligned}
\end{equation}

To convert these recurrence relations into suitable forms to prove the case when $n=m+1$, we will specialize $m$ and $n$ in the above three recurrence relations. Note that the original forms of the three recurrence relations (4.18)-(4.20) will be used later when we prove the general case (when $n\geq m+2$).
We first specialize $n=m+1$ in (4.18). Then, we get

\begin{equation}
\begin{aligned}
    &M_{q}(H_{o}(a,m+1,m+1,k-1:a_1,\cdots,a_{k-1}))M_{q}(H_{e}(a,m,m,k:a_1,\cdots,a_{k-1},0))\\
    &
\begin{aligned}
    =&M_{q}(H_{e}(a,m+1,m,k:a_1,\cdots,a_{k-1},0))M_{q}(H_{o}(a,m,m+1,k-1:a_1,\cdots,a_{k-1}))\\
    &+M_{q}(H_{e}(a,m,m+1,k:a_1,\cdots,a_{k-1},0))M_{q}(H_{o}(a,m+1,m,k-1:a_1,\cdots,a_{k-1})).
\end{aligned}
\end{aligned}
\end{equation}

Note that the regions $H_{e}(a,m+1,m,k:a_1,\cdots,a_{k-1},0)$ and $H_{e}(a,m,m+1,k:a_1,\cdots,a_{k-1},0)$ are symmetric about the $j-$axis, so they have the same TGFs\footnote{This is because our weight assignment on lozenges is symmetric about the $j$-axis}. By the same reasoning, the two regions $H_{o}(a,m,m+1,k-1:a_1,\cdots,a_{k-1})$ and $H_{o}(a,m+1,m,k-1:a_1,\cdots,a_{k-1})$ also have the same TGFs. Thus, the above identity can be simplified as follows:

\begin{equation}
\begin{aligned}
    &M_{q}(H_{o}(a,m+1,m+1,k-1:a_1,\cdots,a_{k-1}))M_{q}(H_{e}(a,m,m,k:a_1,\cdots,a_{k-1},0))\\
    &=2M_{q}(H_{e}(a,m,m+1,k:a_1,\cdots,a_{k-1},0))M_{q}(H_{o}(a,m,m+1,k-1:a_1,\cdots,a_{k-1})).
\end{aligned}
\end{equation}

Using similar specializations ($n=m+1$), one can obtain the following identities from (4.19) and (4.20):

\begin{equation}
\begin{aligned}
    &M_{q}(H_{e}(a,m+1,m+1,k:a_1,\cdots,a_k))M_{q}(H_{o}(a,m+1,m+1,k:a_1,\cdots,a_k-1))\\
    &=2M_{q}(H_{o}(a,m+1,m+2,k:a_1,\cdots,a_k-1))M_{q}(H_{e}(a,m,m+1,k:a_1,\cdots,a_k))
\end{aligned}
\end{equation}

and

\begin{equation}
\begin{aligned}
    &M_{q}(H_{o}(a,m+1,m+1,k:a_1,\cdots,a_k))M_{q}(H_{e}(a,m+1,m+1,k:a_1,\cdots,a_k))\\
    &=2M_{q}(H_{e}(a,m+1,m+2,k:a_1,\cdots,a_k))M_{q}(H_{o}(a,m,m+1,k:a_1,\cdots,a_k)).
\end{aligned}
\end{equation}

Using the three identities we obtained, we now prove the case when $n=m+1$. To give a proof, we first define a concept of \textit{depth} (denoted by $d$) on two families of regions $H_{e}(a,m,n,k:a_1,\cdots,a_k))$ and $H_{o}(a,m,n,k:a_1,\cdots,a_k))$. The depth of the regions $H_{e}(a,m,n,k:a_1,\cdots,a_k))$ and $H_{o}(a,m,n,k:a_1,\cdots,a_k))$ is defined to be $b_1+k=(\sum_{j=1}^{k}a_j)+k$. We will prove the case when $n=m+1$ using an induction on the depth $d$ we just defined. When $d=0$, there is nothing to prove because $d=0$ implies $k=0$ and the regions $H_{e}(a,m,m+1,0:\cdot)$ and $H_{o}(a,m,m+1,0:\cdot)$ are the same as $H_{2a,m,m+1}$ and $H_{2a+1,m,m+1}$, respectively. Thus, the identities (4.16) and (4.17) become $1=1$, which verifies the case $d=0$. Now suppose (4.16) and (4.17) are true for any regions with $d<D$ for some positive integer $D$. Under this assumption, we show that the same is true for the regions with $d=D$. We will complete this by following the three steps described below:

(Step 1) Show that $M_{q}(H_{e}(a,m,m+1,k:a_1,\cdots,a_{k-1},0))$ is given by (4.16) when the depth of the region is $D$. In the proof of this step, we only use the induction hypothesis.

(Step 2) Show that for any $a_k\geq1$, $M_{q}(H_{e}(a,m,m+1,k:a_1,\cdots,a_{k-1},a_k))$ is given by (4.16) when the depth of the region is $D$. In the proof of this step, we only use the induction hypothesis.

(Step 3) Show that for any $a_k\geq0$, $M_{q}(H_{o}(a,m,m+1,k:a_1,\cdots,a_k))$ is given by (4.17) when the depth of the region is $D$. In the proof of this step, we use both the induction hypothesis and the results from Step 1 and Step 2.

We first check Step 1. We use the recurrence relation (4.22) to verify Step 1. In (4.22), the two terms on the left side are known to be given by the formulas (4.1) and (4.2) because $m+1=m+1$ and $m=m$. Also, the depths of $H_{e}(a,m,m+1,k:a_1,\cdots,a_{k-1},0)$ and $H_{o}(a,m,m+1,k-1:a_1,\cdots,a_{k-1})$ are $D$ and $D-1$, respectively. Thus, if we check that the formulas (4.1),(4.2),(4.16), (4.17), and (2.1) satisfy the recurrence relation (4.22), we can conclude that $M_{q}(H_{e}(a,m,m+1,k:a_1,\cdots,a_{k-1},0))$ is given by the formula (4.16). Note that (4.22) is equivalent to the following equation:

\begin{equation}
    \frac{1}{2}=\frac{M_{q}(H_{e}(a,m,m+1,k:a_1,\cdots,a_{k-1},0))}{M_{q}(H_{e}(a,m,m,k:a_1,\cdots,a_{k-1},0))}\frac{M_{q}(H_{o}(a,m,m+1,k-1:a_1,\cdots,a_{k-1}))}{M_{q}(H_{o}(a,m+1,m+1,k-1:a_1,\cdots,a_{k-1}))}.
\end{equation}

From (4.1),(4.2),(4.16), (4.17), and (2.1), we have

\begin{equation}
\begin{aligned}
    &\frac{M_{q}(H_{e}(a,m,m+1,k:a_1,\cdots,a_{k-1},0))}{M_{q}(H_{e}(a,m,m,k:a_1,\cdots,a_{k-1},0))}\\
    &
    \begin{aligned}
        =&\prod_{i=1}^{m+k}\Bigg[\frac{q^{(m+1+k)-i}+q^{i-(m+1+k)}}{2}\Bigg]\cdot\frac{<m+k>_{q}!<2a+2b_1+2m+2k>_{q}!}{<2m+2k>_{q}!<2a+2b_1+m+k>_{q}!}\\
    &\cdot\prod_{i=0}^{k-1}\Bigg[\frac{<m+k-i-\frac{1}{2}>_{q^2}<a+m+k+b_1+b_{i+1}-i>_{q^2}}{<b_{i+1}+m+k-i-\frac{1}{2}>_{q^2}<a+m+k+b_1-i>_{q^2}}\Bigg]
    \end{aligned}
\end{aligned}
\end{equation}

and

\begin{equation}
\begin{aligned}
    &\frac{M_{q}(H_{o}(a,m,m+1,k-1:a_1,\cdots,a_{k-1}))}{M_{q}(H_{o}(a,m+1,m+1,k-1:a_1,\cdots,a_{k-1}))}\\
    &
    \begin{aligned}
        =&\prod_{i=1}^{m+k}\Bigg[\frac{2}{q^{(m+k)-i}+q^{i-(m+k)}}\Bigg]\cdot\frac{<2m+2k-1>_{q}!<2a+2b_1+m+k>_{q}!}{<m+k-1>_{q}!<2a+2b_1+2m+2k>_{q}!}\\
        &\cdot\prod_{i=0}^{k-2}\Bigg[\frac{<b_{i+1}+m+k-i-\frac{1}{2}>_{q^2}<a+m+k+b_1-i>_{q^2}}{<m+k-i-\frac{1}{2}>_{q^2}<a+m+k+b_1+b_{i+1}-i>_{q^2}}\Bigg].
    \end{aligned}
\end{aligned}
\end{equation}

By putting (4.26) and (4.27) together and using the fact that $b_k=a_k=0$, we have

\begin{equation}
\begin{aligned}
    &\frac{M_{q}(H_{e}(a,m,m+1,k:a_1,\cdots,a_{k-1},0))}{M_{q}(H_{e}(a,m,m,k:a_1,\cdots,a_{k-1},0))}\frac{M_{q}(H_{o}(a,m,m+1,k-1:a_1,\cdots,a_{k-1}))}{M_{q}(H_{o}(a,m+1,m+1,k-1:a_1,\cdots,a_{k-1}))}\\
    &=\frac{q^{m+k}+q^{-(m+k)}}{2}\cdot\frac{<m+k>_{q}}{<2m+2k>_{q}}\\
    &=\frac{1}{2}.
\end{aligned}
\end{equation}
and thus Step 1 is checked.

We move on to Step 2. Here, we use the recurrence relation (4.23). In (4.23), the two terms on the left side are given by the formulas (4.1) and (4.2) because $m+1=m+1$. The depth of the two regions on the right, $H_{o}(a,m+1,m+2,k:a_1,\cdots,a_k-1)$ and $H_{e}(a,m,m+1,k:a_1,\cdots,a_k)$, are $D-1$ and $D$, respectively. By the induction hypothesis, we know that $M_{q}(H_{o}(a,m+1,m+2,k:a_1,\cdots,a_k-1))$ is given by (4.17). Thus, to show that $M_{q}(H_{e}(a,m,m+1,k:a_1,\cdots,a_k))$ is also given by the formula (4.16) when $a_k\geq 1$, it is enough to show that (4.1), (4.2), (4.16), (4.17), and (2.1) satisfy the recurrence relations (4.23). Note that (4.23) is equivalent to the following equation:6). Note that (4.22) is equivalent to the following equation:

\begin{equation}
    \frac{1}{2}=\frac{M_{q}(H_{e}(a,m,m+1,k:a_1,\cdots,a_k))}{M_{q}(H_{e}(a,m+1,m+1,k:a_1,\cdots,a_k))}\frac{M_{q}(H_{o}(a,m+1,m+2,k:a_1,\cdots,a_k-1))}{M_{q}(H_{o}(a,m+1,m+1,k:a_1,\cdots,a_k-1))}.
\end{equation}

From (4.1),(4.2),(4.16), (4.17), and (2.1), we have

\begin{equation}
\begin{aligned}
    &\frac{M_{q}(H_{e}(a,m,m+1,k:a_1,\cdots,a_k))}{M_{q}(H_{e}(a,m+1,m+1,k:a_1,\cdots,a_k))}\\
    &
    \begin{aligned}
        =&\prod_{i=1}^{m+k+1}\Bigg[\frac{2}{q^{(m+k+1)-i}+q^{i-(m+k+1)}}\Bigg]\cdot\frac{<2m+2k+1>_{q}!<2a+2b_1+m+k>_{q}!}{<m+k>_{q}!<2a+2b_1+2m+2k+1>_{q}!}\\
        &\cdot\prod_{i=0}^{k-1}\Bigg[\frac{<b_{i+1}+m+k-i>_{q^2}<a+b_1+m+k-i+\frac{1}{2}>_{q^2}}{<m+k-i+\frac{1}{2}>_{q^2}<a+b_1+b_{i+1}+m+k-i>_{q^2}}\Bigg]
    \end{aligned}
\end{aligned}
\end{equation}

and

\begin{equation}
\begin{aligned}
    &\frac{M_{q}(H_{o}(a,m+1,m+2,k:a_1,\cdots,a_k-1))}{M_{q}(H_{o}(a,m+1,m+1,k:a_1,\cdots,a_k-1))}\\
    &
    \begin{aligned}
        =&\prod_{i=1}^{m+k+1}\Bigg[\frac{q^{(m+k+2)-i}+q^{i-(m+k+2)}}{2}\Bigg]\cdot\frac{<m+k+1>_{q}!<2a+2b_1+2m+2k+1>_{q}!}{<2m+2k+2>_{q}!<2a+2b_1+m+k>_{q}!}\\
        &\cdot\prod_{i=0}^{k-1}\Bigg[\frac{<m+k-i+\frac{1}{2}>_{q^2}<a+b_1+b_{i+1}+m+k-i>_{q^2}}{<b_{i+1}+m+k-i>_{q^2}<a+b_1+m+k-i+\frac{1}{2}>_{q^2}}\Bigg].
    \end{aligned}
\end{aligned}
\end{equation}

By putting (4.30) and (4.31) together, we have

\begin{equation}
\begin{aligned}
    &\frac{M_{q}(H_{e}(a,m,m+1,k:a_1,\cdots,a_k))}{M_{q}(H_{e}(a,m+1,m+1,k:a_1,\cdots,a_k))}\frac{M_{q}(H_{o}(a,m+1,m+2,k:a_1,\cdots,a_k-1))}{M_{q}(H_{o}(a,m+1,m+1,k:a_1,\cdots,a_k-1))}\\
    &=\frac{q^{m+k+1}+q^{-(m+k+1)}}{2}\cdot\frac{<m+k+1>_{q}}{<2m+2k+2>_{q}}\\
    &=\frac{1}{2}.
\end{aligned}
\end{equation}
and thus Step 2 is done.

A slightly different argument is needed to check Step 3. As we already mentioned, we will use both the induction hypothesis and the results from Step 1 and Step 2 to show that $M_{q}(H_{o}(a,m,m+1,k:a_1,\cdots,a_k))$ is given by (4.17) when its depth is $D$. We use the recurrence relation (4.24) in this case. In (4.24), the left side is known to be given by the formulas (4.1) and (4.2) because $m+1=m+1$. On the right side, the depth of two regions appeared there ($H_{e}(a,m+1,m+2,k:a_1,\cdots,a_k)$ and $H_{o}(a,m,m+1,k:a_1,\cdots,a_k)$) are both $D$. However, in Step 1 and Step 2, we showed that $H_{e}(a,m+1,m+2,k:a_1,\cdots,a_k)$ is given by (4.16) under the induction hypothesis (This is because $m+2=(m+1)+1$ and the depth is $D$). Thus, to prove that $M_{q}(H_{o}(a,m,m+1,k:a_1,\cdots,a_k))$ is given by (4.17), we only have to check that the formulas (4.1), (4.2), (4.16), (4.17), and (2.1) satisfy (4.24). Note that (4.24) is equivalent to the following equation:

\begin{equation}
    \frac{1}{2}=\frac{M_{q}(H_{e}(a,m+1,m+2,k:a_1,\cdots,a_k))}{M_{q}(H_{e}(a,m+1,m+1,k:a_1,\cdots,a_k))}\frac{M_{q}(H_{o}(a,m,m+1,k:a_1,\cdots,a_k))}{M_{q}(H_{o}(a,m+1,m+1,k:a_1,\cdots,a_k))}.
\end{equation}

From (4.1), (4.2), (4.16), (4.17), and (2.1), we have
\begin{equation}
\begin{aligned}
    &\frac{M_{q}(H_{e}(a,m+1,m+2,k:a_1,\cdots,a_k))}{M_{q}(H_{e}(a,m+1,m+1,k:a_1,\cdots,a_k))}\\
    &
    \begin{aligned}
        =&\prod_{i=1}^{m+k+1}\Bigg[\frac{q^{(m+k+2)-i}+q^{i-(m+k+2)}}{2}\Bigg]\cdot\frac{<m+k+1>_{q}!<2a+2b_1+2m+2k+2>_{q}!}{<2m+2k+2>_{q}!<2a+2b_1+m+k+1>_{q}!}\\
        &\cdot\prod_{i=0}^{k-1}\Bigg[\frac{<m+k-i+\frac{1}{2}>_{q^2}<a+b_1+b_{i+1}+m+k-i+1>_{q^2}}{<b_{i+1}+m+k-i+\frac{1}{2}>_{q^2}<a+b_1+m+k-i+1>_{q^2}}\Bigg],
    \end{aligned}
\end{aligned}
\end{equation}

and

\begin{equation}
\begin{aligned}
    &\frac{M_{q}(H_{o}(a,m,m+1,k:a_1,\cdots,a_k))}{M_{q}(H_{o}(a,m+1,m+1,k:a_1,\cdots,a_k))}\\
    &
    \begin{aligned}
        =&\prod_{i=1}^{m+k+1}\Bigg[\frac{2}{q^{(m+k+1)-i}+q^{i-(m+k+1)}}\Bigg]\cdot\frac{<2m+2k+1>_{q}!<2a+2b_1+m+k+1>_{q}!}{<m+k>_{q}!<2a+2b_1+2m+2k+2>_{q}!}\\
        &\cdot\prod_{i=0}^{k-1}\Bigg[\frac{<b_{i+1}+m+k-i+\frac{1}{2}>_{q^2}<a+b_1+m+k-i+1>_{q^2}}{<m+k-i+\frac{1}{2}>_{q^2}<a+b_1+b_{i+1}+m+k-i+1>_{q^2}}\Bigg].
    \end{aligned}
\end{aligned}
\end{equation}

By putting (4.34) and (4.35) together, we have

\begin{equation}
\begin{aligned}
    &\frac{M_{q}(H_{e}(a,m+1,m+2,k:a_1,\cdots,a_k))}{M_{q}(H_{e}(a,m+1,m+1,k:a_1,\cdots,a_k))}\frac{M_{q}(H_{o}(a,m,m+1,k:a_1,\cdots,a_k))}{M_{q}(H_{o}(a,m+1,m+1,k:a_1,\cdots,a_k))}\\
    &=\frac{q^{m+k+1}+q^{-(m+k+1)}}{2}\cdot\frac{<m+k+1>_{q}}{<2m+2k+2>_{q}}\\
    &=\frac{1}{2}.
\end{aligned}    
\end{equation}

Thus, Step 3 is checked, and this completes the proof of the case when $n=m+1$. The two cases when $n=m$ and $n=m+1$ will serve as base cases when we prove the general case using mathematical induction.\\

Finally, we finish the proof by showing that the statement is true for any $m$ and $n$ with $n\geq m$. The proof is based on mathematical induction, and we will use induction on the value $n-m$. We call this induction \textit{an outer induction}. The cases when $n-m=0$ and $1$ have been checked, so now we assume that (2.2) and (2.3) are true when $n-m<E$ for some positive integer $E\geq2$. Under this assumption, we must show that (2.2) and (2.3) are still true when $n-m=E$.

Strictly speaking, what we have to show is that (2.2) and (2.3) hold for the cases when $n-m=E$, while the depth of the intrusion can be \textit{arbitrary}. Thus, We show the induction step by employing another induction on the depth $d$. We call this induction \textit{an inner induction}. There is nothing to prove when $n-m=E$ and $d=0$ because (2.2) and (2.3) become $1=1$. Suppose (2.2) and (2.3) are true when $n-m=E$ and $d<D$ for some positive integer $D$. Under this assumption, we must show that (2.2) and (2.3) hold for the regions with $n-m=E$ and $d=D$. Like the proof of the case when $n=m+1$, we will show this by checking the following three steps:

(Step 1$'$) Show that $M_{q}(H_{e}(a,m,n,k:a_1,\cdots,a_{k-1},0))$ is given by (2.2) when $n-m=E$ and the depth of the region is $D$. In the proof of this step, we use the induction hypothesis of both outer and inner induction.

(Step 2$'$) Show that for any $a_k\geq1$, $M_{q}(H_{e}(a,m,n,k:a_1,\cdots,a_{k-1},a_k))$ is given by (2.2) when $n-m=E$ and the depth of the region is $D$. In the proof of this step, we use the induction hypothesis of both outer and inner induction.

(Step 3$'$) Show that for any $a_k\geq0$, $M_{q}(H_{o}(a,m,n,k:a_1,\cdots,a_k))$ is given by (2.3) when $n-m=E$ and the dept of the region is $D$. In the proof of this step, we use the induction hypothesis of both outer and inner induction and the result from Step 1$'$ and Step 2$'$.

We first check Step 1$'$. To do this, we analyze the regions that appeared in the recurrence relation (4.18).

In (4.18), there are six regions appeared in the recurrence relation, and now we claim that five of them are known to be given by (2.2) and (2.3) by the (either outer or inner) induction hypothesis. $M_{q}(H_{o}(a,m+1,n,k-1:a_1,\cdots,a_{k-1}))$, $M_{q}(H_{e}(a,m,n-1,k:a_1,\cdots,a_{k-1},0))$, $M_{q}(H_{e}(a,m+1,n-1,k:a_1,\cdots,a_{k-1},0))$, and $M_{q}(H_{o}(a,m+1,n-1,k-1:a_1,\cdots,a_{k-1}))$ are given by (2.2) and (2.3). It is due to the induction hypothesis of the outer induction because $n-(m+1)=(n-1)-m=n-m-1=E-1<E$ and $(n-1)-(m+1)=n-m-2=E-2<E$. Also, $M_{q}(H_{o}(a,m,n,k-1:a_1,\cdots,a_{k-1}))$ is given by (2.3). This time, it is due to the induction hypothesis of the inner induction because $n-m=E$ and the depth of the region is $D-1$, which is less than $D$.

Thus, to show that $M_{q}(H_{e}(a,m,n,k:a_1,\cdots,a_{k-1},0))$ is given by (2.2) when $n-m=E$ and the depth of the region is $D$, it is enough to check that the formulas (2.2), (2.3), and (2.1) satisfy the recurrence relation (4.18). Note that (4.18) is equivalent to the following identity:

\begin{equation}
\begin{aligned}
    &\frac{M_{q}(H_{e}(a,m+1,n-1,k:a_1,\cdots,a_{k-1},0))}{M_{q}(H_{e}(a,m,n-1,k:a_1,\cdots,a_{k-1},0))}\cdot\frac{M_{q}(H_{o}(a,m,n,k-1:a_1,\cdots,a_{k-1}))}{M_{q}(H_{o}(a,m+1,n,k-1:a_1,\cdots,a_{k-1}))}\\
    &+\frac{M_{q}(H_{e}(a,m,n,k:a_1,\cdots,a_{k-1},0))}{M_{q}(H_{e}(a,m,n-1,k:a_1,\cdots,a_{k-1},0))}\cdot\frac{M_{q}(H_{o}(a,m+1,n-1,k-1:a_1,\cdots,a_{k-1}))}{M_{q}(H_{o}(a,m+1,n,k-1:a_1,\cdots,a_{k-1}))}\\
    &=1.
\end{aligned}    
\end{equation}

These four ratios appeared in the above equation (4.37) can be easily simplified using the formulas (2.2), (2.3), and (2.1). After the simplification, one gets

\begin{equation}
\begin{aligned}
    &\frac{M_{q}(H_{e}(a,m+1,n-1,k:a_1,\cdots,a_{k-1},0))}{M_{q}(H_{e}(a,m,n-1,k:a_1,\cdots,a_{k-1},0))}\\
    &
    \begin{aligned}
        =&\prod_{i=1}^{n+k-1}\Bigg[\frac{q^{(m+k+1)-i}+q^{i-(m+k+1)}}{2}\Bigg]\cdot\frac{<m+k>_{q}!<2a+2b_1+m+n+2k-1>_{q}!}{<m+n+2k-1>_{q}!<2a+2b_1+m+k>_{q}!}\\
        &\cdot\prod_{i=0}^{k-1}\Bigg[\frac{<a+b_1+b_{i+1}+m+k-i>_{q^2}<\frac{1}{2} m+\frac{1}{2}n+k-i-\frac{1}{2}>_{q^2}<\frac{1}{2}m+\frac{1}{2}n+k-i-1>_{q^2}}{<m+k-i>_{q^2}<a+b_1+\frac{1}{2}m+\frac{1}{2}n+k-i-\frac{1}{2}>_{q^2}<b_{i+1}+\frac{1}{2}m+\frac{1}{2}n+k-i-1>_{q^2}}\Bigg],
    \end{aligned}
\end{aligned}
\end{equation}

\begin{equation}
\begin{aligned}
    &\frac{M_{q}(H_{o}(a,m,n,k-1:a_1,\cdots,a_{k-1}))}{M_{q}(H_{o}(a,m+1,n,k-1:a_1,\cdots,a_{k-1}))}\\
    &
    \begin{aligned}
        =&\prod_{i=1}^{n+k-1}\Bigg[\frac{2}{q^{(m+k)-i}+q^{i-(m+k)}}\Bigg]\cdot\frac{<m+n+2k-2>_{q}!<2a+2b_1+m+k>_{q}!}{<m+k-1>_{q}!<2a+2b_1+m+n+2k-1>_{q}!}\\
        &\cdot\prod_{i=0}^{k-2}\Bigg[\frac{<m+k-i-1>_{q^2}<a+b_1+\frac{1}{2}m+\frac{1}{2}n+k-i-\frac{1}{2}>_{q^2}<b_{i+1}+\frac{1}{2}m+\frac{1}{2}n+k-i-1>_{q^2}}{<a+b_1+b_{i+1}+m+k-i>_{q^2}<\frac{1}{2}m+\frac{1}{2}n+k-i-\frac{3}{2}>_{q^2}<\frac{1}{2}m+\frac{1}{2}n+k-i-1>_{q^2}}\Bigg],
    \end{aligned}
\end{aligned}
\end{equation}

\begin{equation}
\begin{aligned}
    &\frac{M_{q}(H_{e}(a,m,n,k:a_1,\cdots,a_{k-1},0))}{M_{q}(H_{e}(a,m,n-1,k:a_1,\cdots,a_{k-1},0))}\\
    &
    \begin{aligned}
        =&\prod_{i=1}^{m+k}\Bigg[\frac{q^{(n+k)-i}+q^{i-(n+k)}}{2}\Bigg]\cdot\frac{<n+k-1>_{q}!<2a+2b_1+m+n+2k-1>_{q}!}{<m+n+2k-1>_{q}!<2a+2b_1+n+k-1>_{q}!}\\
        &\cdot\prod_{i=0}^{k-1}\Bigg[\frac{<a+b_1+b_{i+1}+n+k-i-1>_{q^2}<\frac{1}{2}m+\frac{1}{2}n+k-i-\frac{1}{2}>_{q^2}<\frac{1}{2}m+\frac{1}{2}n+k-i-1>_{q^2}}{<n+k-i-1>_{q^2}<a+b_1+\frac{1}{2}m+\frac{1}{2}n+k-i-\frac{1}{2}>_{q^2}<b_{i+1}+\frac{1}{2}m+\frac{1}{2}n+k-i-1>_{q^2}}\Bigg],
    \end{aligned}
\end{aligned}
\end{equation}

and

\begin{equation}
\begin{aligned}
    &\frac{M_{q}(H_{o}(a,m+1,n-1,k-1:a_1,\cdots,a_{k-1}))}{M_{q}(H_{o}(a,m+1,n,k-1:a_1,\cdots,a_{k-1}))}\\
    &
    \begin{aligned}
        =&\prod_{i=1}^{m+k}\Bigg[\frac{2}{q^{(n+k-1)-i}+q^{i-(n+k-1)}}\Bigg]\cdot\frac{<m+n+2k-2>_{q}!<2a+2b_1+n+k-1>_{q}!}{<n+k-2>_{q}!<2a+2b_1+m+n+2k-1>_{q}!}\\
        &\cdot\prod_{i=0}^{k-2}\Bigg[\frac{<n+k-i-2>_{q^2}<a+b_1+\frac{1}{2}m+\frac{1}{2}n+k-i-\frac{1}{2}>_{q^2}<b_{i+1}+\frac{1}{2}m+\frac{1}{2}n+k-i-1>_{q^2}}{<a+b_1+b_{i+1}+n+k-i-1>_{q^2}<\frac{1}{2}m+\frac{1}{2}n+k-i-\frac{3}{2}>_{q^2}<\frac{1}{2}m+\frac{1}{2}n+k-i-1>_{q^2}}\Bigg].
    \end{aligned}
\end{aligned}
\end{equation}

By putting (4.38)-(4.41) together in (4.37) and using $b_k=a_k=0$, one can check that 

\begin{equation}
\begin{aligned}
    &\frac{M_{q}(H_{e}(a,m+1,n-1,k:a_1,\cdots,a_{k-1},0))}{M_{q}(H_{e}(a,m,n-1,k:a_1,\cdots,a_{k-1},0))}\cdot\frac{M_{q}(H_{o}(a,m,n,k-1:a_1,\cdots,a_{k-1}))}{M_{q}(H_{o}(a,m+1,n,k-1:a_1,\cdots,a_{k-1}))}\\
    &+\frac{M_{q}(H_{e}(a,m,n,k:a_1,\cdots,a_{k-1},0))}{M_{q}(H_{e}(a,m,n-1,k:a_1,\cdots,a_{k-1},0))}\cdot\frac{M_{q}(H_{o}(a,m+1,n-1,k-1:a_1,\cdots,a_{k-1}))}{M_{q}(H_{o}(a,m+1,n,k-1:a_1,\cdots,a_{k-1}))}\\
    &
    \begin{aligned}
        =&\frac{q^{m+k}+q^{-(m+k)}}{q^{n-m-1}+q^{-(n-m-1)}}\cdot\frac{<m+k>_{q}}{<m+n+2k-1>_{q}}\cdot\frac{<a+b_1+m+1>_{q^2}<\frac{1}{2}m+\frac{1}{2}n+k-\frac{1}{2}>_{q^2}}{<m+k>_{q^2}<a+b_1+\frac{1}{2}m+\frac{1}{2}n+\frac{1}{2}>_{q^2}}\\
        &+\frac{q^{n+k-1}+q^{-(n+k-1)}}{q^{n-m-1}+q^{-(n-m-1)}}\cdot\frac{<n+k-1>_{q}}{<m+n+2k-1>_{q}}\cdot\frac{<a+b_1+n>_{q^2}<\frac{1}{2}m+\frac{1}{2}n+k-\frac{1}{2}>_{q^2}}{<n+k-1>_{q^2}<a+b_1+\frac{1}{2}m+\frac{1}{2}n+\frac{1}{2}>_{q^2}}
    \end{aligned}
    \\
    &=\frac{<a+b_1+m+1>_{q^2}+<a+b_1+n>_{q^2}}{(q^{n-m-1}+q^{-(n-m-1)})<a+b_1+\frac{1}{2}m+\frac{1}{2}n+\frac{1}{2}>_{q^2}}\\
    &=1.
\end{aligned}    
\end{equation}

Thus, we just check that $M_{q}(H_{e}(a,m,n,k:a_1,\cdots,a_{k-1},0))$ is given by the formula (2.2) when $n-m=E$ and the depth of the region is $D$. It verifies Step 1$'$.

We move on to the verification of Step 2$'$. In this step, we will use the recurrence relation (4.19). Among the six terms in (4.19), we claim that five of them are known to be given by (2.2), (2.3), and (2.1) by the induction hypotheses. Indeed, $M_{q}(H_{e}(a,m+1,n,k:a_1,\cdots,a_k))$, $M_{q}(H_{o}(a,m+1,n,k:a_1,\cdots,a_k-1))$, $M_{q}(H_{o}(a,m+2,n,k:a_1,\cdots,a_k-1))$, and $M_{q}(H_{e}(a,m+1,n-1,k:a_1,\cdots,a_k))$ are given by (2.2) and (2.3). It is due to the induction hypothesis of the outer induction because $n-(m+1)=n-m-1=E-1<E$ and $n-(m+2)=(n-1)-(m+1)=n-m-2=E-2<E$. Also, $M_{q}(H_{o}(a,m+1,n+1,k:a_1,\cdots,a_k-1))$ is given by (2.3). It is due to the induction hypothesis of the inner induction because $(n+1)-(m+1)=n-m=E$ and the depth of the region is $D-1<D$.

Hence, to show that $M_{q}(H_{e}(a,m,n,k:a_1,\cdots,a_{k-1},a_k))$ with $a_k\geq1$ is given by (2.2), it is enough to check that (2.2), (2.3), and (2.1) satisfy (4.19). Note that (4.19) is equivalent to the following identity:

\begin{equation}
\begin{aligned}
    &\frac{M_{q}(H_{o}(a,m+2,n,k:a_1,\cdots,a_k-1))}{M_{q}(H_{o}(a,m+1,n,k:a_1,\cdots,a_k-1))}\cdot\frac{M_{q}(H_{e}(a,m,n,k:a_1,\cdots,a_k))}{M_{q}(H_{e}(a,m+1,n,k:a_1,\cdots,a_k))}\\
    &+\frac{M_{q}(H_{o}(a,m+1,n+1,k:a_1,\cdots,a_k-1))}{M_{q}(H_{o}(a,m+1,n,k:a_1,\cdots,a_k-1))}\frac{M_{q}(H_{e}(a,m+1,n-1,k:a_1,\cdots,a_k))}{M_{q}(H_{e}(a,m+1,n,k:a_1,\cdots,a_k))}\\
    &=1.
\end{aligned}
\end{equation}

As before, we can use (2.2), (2.3), and (2.1) to obtain the following identities:

\begin{equation}
\begin{aligned}
    &\frac{M_{q}(H_{o}(a,m+2,n,k:a_1,\cdots,a_k-1))}{M_{q}(H_{o}(a,m+1,n,k:a_1,\cdots,a_k-1))}\\
    &
    \begin{aligned}
        =&\prod_{i=1}^{n+k}\Bigg[\frac{q^{(m+k+2)-i}+q^{i-(m+k+2)}}{2}\Bigg]\cdot\frac{<m+k+1>_{q}!<2a+2b_1+m+n+2k>_{q}!}{<m+n+2k+1>_{q}!<2a+2b_1+m+k>_{q}!}\\
        &\cdot\prod_{i=0}^{k-1}\Bigg[\frac{<a+b_1+b_{i+1}+m+k-i>_{q^2}<\frac{1}{2}m+\frac{1}{2}n+k-i>_{q^2}<\frac{1}{2}m+\frac{1}{2}n+k-i+\frac{1}{2}>_{q^2}}{<m+k-i+1>_{q^2}<a+b_1+\frac{1}{2}m+\frac{1}{2}n+k-i>_{q^2}<b_{i+1}+\frac{1}{2}m+\frac{1}{2}n+k-i-\frac{1}{2}>_{q^2}}\Bigg],        
    \end{aligned}
\end{aligned}
\end{equation}

\begin{equation}
\begin{aligned}
    &\frac{M_{q}(H_{e}(a,m,n,k:a_1,\cdots,a_k))}{M_{q}(H_{e}(a,m+1,n,k:a_1,\cdots,a_k))}\\
    &
    \begin{aligned}
        =&\prod_{i=1}^{n+k}\Bigg[\frac{2}{q^{(m+k+1)-i}+q^{i-(m+k+1)}}\Bigg]\cdot\frac{<m+n+2k>_{q}!<2a+2b_1+m+k>_{q}!}{<m+k>_{q}!<2a+2b_1+m+n+2k>_{q}!}\\
        &\cdot\prod_{i=0}^{k-1}\Bigg[\frac{<m+k-i>_{q^2}<a+b_1+\frac{1}{2}m+\frac{1}{2}n+k-i>_{q^2}<b_{i+1}+\frac{1}{2}m+\frac{1}{2}n+k-i-\frac{1}{2}>_{q^2}}{<a+b_1+b_{i+1}+m+k-i>_{q^2}<\frac{1}{2}m+\frac{1}{2}n+k-i>_{q^2}<\frac{1}{2}m+\frac{1}{2}n+k-i-\frac{1}{2}>_{q^2}}\Bigg],    
    \end{aligned}
\end{aligned}
\end{equation}

\begin{equation}
\begin{aligned}
    &\frac{M_{q}(H_{o}(a,m+1,n+1,k:a_1,\cdots,a_k-1))}{M_{q}(H_{o}(a,m+1,n,k:a_1,\cdots,a_k-1))}\\
    &
    \begin{aligned}
        =&\prod_{i=1}^{m+k+1}\Bigg[\frac{q^{(n+k+1)-i}+q^{i-(n+k+1)}}{2}\Bigg]\cdot\frac{<n+k>_{q}!<2a+2b_1+m+n+2k>_{q}!}{<m+n+2k+1>_{q}!<2a+2b_1+n+k-1>_{q}!}\\
        &\cdot\prod_{i=0}^{k-1}\Bigg[\frac{<a+b_1+b_{i+1}+n+k-i-1>_{q^2}<\frac{1}{2}m+\frac{1}{2}n+k-i>_{q^2}<\frac{1}{2}m+\frac{1}{2}n+k-i+\frac{1}{2}>_{q^2}}{<n+k-i>_{q^2}<a+b_1+\frac{1}{2}m+\frac{1}{2}n+k-i>_{q^2}<b_{i+1}+\frac{1}{2}m+\frac{1}{2}n+k-i-\frac{1}{2}>_{q^2}}\Bigg],    
    \end{aligned}
\end{aligned}
\end{equation}

and

\begin{equation}
\begin{aligned}
    &\frac{M_{q}(H_{e}(a,m+1,n-1,k:a_1,\cdots,a_k))}{M_{q}(H_{e}(a,m+1,n,k:a_1,\cdots,a_k))}\\
    &
    \begin{aligned}
        =&\prod_{i=1}^{m+k+1}\Bigg[\frac{2}{q^{(n+k)-i}+q^{i-(n+k)}}\Bigg]\cdot\frac{<m+n+2k>_{q}!<2a+2b_1+n+k-1>_{q}!}{<n+k-1>_{q}!<2a+2b_1+m+n+2k>_{q}!}\\
        &\cdot\prod_{i=0}^{k-1}\Bigg[\frac{<n+k-i-1>_{q^2}<a+b_1+\frac{1}{2}m+\frac{1}{2}n+k-i>_{q^2}<b_{i+1}+\frac{1}{2}m+\frac{1}{2}n+k-i-\frac{1}{2}>_{q^2}}{<a+b_1+b_{i+1}+n+k-i-1>_{q^2}<\frac{1}{2}m+\frac{1}{2}n+k-i>_{q^2}<\frac{1}{2}m+\frac{1}{2}n+k-i-\frac{1}{2}>_{q^2}}\Bigg].    
    \end{aligned}
\end{aligned}
\end{equation}

By putting (4.44)-(4.47) together in (4.43), we get

\begin{equation}
\begin{aligned}
    &\frac{M_{q}(H_{o}(a,m+2,n,k:a_1,\cdots,a_k-1))}{M_{q}(H_{o}(a,m+1,n,k:a_1,\cdots,a_k-1))}\cdot\frac{M_{q}(H_{e}(a,m,n,k:a_1,\cdots,a_k))}{M_{q}(H_{e}(a,m+1,n,k:a_1,\cdots,a_k))}\\
    &+\frac{M_{q}(H_{o}(a,m+1,n+1,k:a_1,\cdots,a_k-1))}{M_{q}(H_{o}(a,m+1,n,k:a_1,\cdots,a_k-1))}\cdot\frac{M_{q}(H_{e}(a,m+1,n-1,k:a_1,\cdots,a_k))}{M_{q}(H_{e}(a,m+1,n,k:a_1,\cdots,a_k))}\\
    &
    \begin{aligned}
        =&\frac{q^{m+k+1}+q^{-(m+k+1)}}{q^{n-m-1}+q^{-(n-m-1)}}\cdot\frac{<m+k+1>_{q}}{<m+n+2k+1>_{q}}\cdot\frac{<m+1>_{q^2}<\frac{1}{2}m+\frac{1}{2}n+k+\frac{1}{2}>_{q^2}}{<m+k+1>_{q^2}<\frac{1}{2}m+\frac{1}{2}n+\frac{1}{2}>_{q^2}}\\
        &+\frac{q^{n+k}+q^{-(n+k)}}{q^{n-m-1}+q^{-(n-m-1)}}\cdot\frac{<n+k>_{q}}{<m+n+2k+1>_{q}}\cdot\frac{<n>_{q^2}<\frac{1}{2}m+\frac{1}{2}n+k+\frac{1}{2}>_{q^2}}{<n+k>_{q^2}<\frac{1}{2}m+\frac{1}{2}n+\frac{1}{2}>_{q^2}}\\
    \end{aligned}
    \\
    &=\frac{<m+1>_{q^2}+<n>_{q^2}}{(q^{n-m-1}+q^{-(n-m-1)})<\frac{1}{2}m+\frac{1}{2}n+\frac{1}{2}>_{q^2}}\\
    &=1.
\end{aligned}
\end{equation}
and this verifies Step 2$'$.

Lastly, we will use (4.20) to check Step 3$'$. Among the six terms in (4.20), we claim that four of them are known to be given by (2.2) and (2.3) by the induction hypotheses, and one of them is known to be given by (2.2) due to the results of Step 1$'$ and Step 2$'$. In fact, $M_{q}(H_{o}(a,m+1,n,k:a_1,\cdots,a_k))$, $M_{q}(H_{e}(a,m+1,n,k:a_1,\cdots,a_k))$, $M_{q}(H_{e}(a,m+2,n,k:a_1,\cdots,a_k))$, and $M_{q}(H_{o}(a,m+1,n-1,k:a_1,\cdots,a_k))$ are given by (2.2) and (2.3). It is due to the induction hypothesis of the outer induction because $n-(m+1)=n-m-1=E-1<E$ and $n-(m+2)=(n-1)-(m+1)=n-m-2=E-2<E$. Also, $M_{q}(H_{e}(a,m+1,n+1,k:a_1,\cdots,a_k))$ is given by (2.2). It is due to the results of Step 1$'$ and Step 2$'$ because $(n+1)-(m+1)=n-m=E$ and the depth of the region is $D$. Thus, to show that $M_{q}(H_{o}(a,m,n,k:a_1,\cdots,a_k))$ is given by (2.3), we have to check that (2.2), (2.3), and (2.1) satisfy the recurrence relation (4.20). Note that the recurrence relation (4.20) is equivalent to the following identity:

\begin{equation}
\begin{aligned}
    &\frac{M_{q}(H_{e}(a,m+2,n,k:a_1,\cdots,a_k))}{M_{q}(H_{e}(a,m+1,n,k:a_1,\cdots,a_k))}\cdot\frac{M_{q}(H_{o}(a,m,n,k:a_1,\cdots,a_k))}{M_{q}(H_{o}(a,m+1,n,k:a_1,\cdots,a_k))}\\
    &+\frac{M_{q}(H_{e}(a,m+1,n+1,k:a_1,\cdots,a_k))}{M_{q}(H_{e}(a,m+1,n,k:a_1,\cdots,a_k))}\cdot\frac{M_{q}(H_{o}(a,m+1,n-1,k:a_1,\cdots,a_k))}{M_{q}(H_{o}(a,m+1,n,k:a_1,\cdots,a_k))}\\
    &=1.
\end{aligned}
\end{equation}

Again, using (2.2), (2.3) and (2.1), we have

\begin{equation}
\begin{aligned}
    &\frac{M_{q}(H_{e}(a,m+2,n,k:a_1,\cdots,a_k))}{M_{q}(H_{e}(a,m+1,n,k:a_1,\cdots,a_k))}\\
    &
    \begin{aligned}
        =&\prod_{i=1}^{n+k}\Bigg[\frac{q^{(m+k+2)-i}+q^{i-(m+k+2)}}{2}\Bigg]\cdot\frac{<m+k+1>_{q}!<2a+2b_1+m+n+2k+1>_{q}!}{<m+n+2k+1>_{q}!<2a+2b_1+m+k+1>_{q}!}\\
        &\cdot\prod_{i=0}^{k-1}\Bigg[\frac{<a+b_1+b_{i+1}+m+k-i+1>_{q^2}<\frac{1}{2}m+\frac{1}{2}n+k-i+\frac{1}{2}>_{q^2}<\frac{1}{2}m+\frac{1}{2}n+k-i>_{q^2}}{<m+k-i+1>_{q^2}<a+b_1+\frac{1}{2}m+\frac{1}{2}n+k-i+\frac{1}{2}>_{q^2}<b_{i+1}+\frac{1}{2}m+\frac{1}{2}n+k-i>_{q^2}}\Bigg],    
    \end{aligned}
\end{aligned}
\end{equation}

\begin{equation}
\begin{aligned}
    &\frac{M_{q}(H_{o}(a,m,n,k:a_1,\cdots,a_k))}{M_{q}(H_{o}(a,m+1,n,k:a_1,\cdots,a_k))}\\
    &
    \begin{aligned}
        =&\prod_{i=1}^{n+k}\Bigg[\frac{2}{q^{(m+k+1)-i}+q^{i-(m+k+1)}}\Bigg]\cdot\frac{<m+n+2k>_{q}!<2a+2b_1+m+k+1>_{q}!}{<m+k>_{q}!<2a+2b_1+m+n+2k+1>_{q}!}\\
        &\cdot\prod_{i=0}^{k-1}\Bigg[\frac{<m+k-i>_{q^2}<a+b_1+\frac{1}{2}m+\frac{1}{2}n+k-i+\frac{1}{2}>_{q^2}<b_{i+1}+\frac{1}{2}m+\frac{1}{2}n+k-i>_{q^2}}{<a+b_1+b_{i+1}+m+k-i+1>_{q^2}<\frac{1}{2}m+\frac{1}{2}n+k-i-\frac{1}{2}>_{q^2}<\frac{1}{2}m+\frac{1}{2}n+k-i>_{q^2}}\Bigg],    
    \end{aligned}
\end{aligned}
\end{equation}

\begin{equation}
\begin{aligned}
    &\frac{M_{q}(H_{e}(a,m+1,n+1,k:a_1,\cdots,a_k))}{M_{q}(H_{e}(a,m+1,n,k:a_1,\cdots,a_k))}\\
    &
    \begin{aligned}
        =&\prod_{i=1}^{m+k+1}\Bigg[\frac{q^{(n+k+1)-i}+q^{i-(n+k+1)}}{2}\Bigg]\cdot\frac{<n+k>_{q}!<2a+2b_1+m+n+2k+1>_{q}!}{<m+n+2k+1>_{q}!<2a+2b_1+n+k>_{q}!}\\
        &\cdot\prod_{i=0}^{k-1}\Bigg[\frac{<a+b_1+b_{i+1}+n+k-i>_{q^2}<\frac{1}{2}m+\frac{1}{2}n+k-i+\frac{1}{2}>_{q^2}<\frac{1}{2}m+\frac{1}{2}n+k-i>_{q^2}}{<n+k-i>_{q^2}<a+b_1+\frac{1}{2}m+\frac{1}{2}n+k-i+\frac{1}{2}>_{q^2}<b_{i+1}+\frac{1}{2}m+\frac{1}{2}n+k-i>_{q^2}}\Bigg],    
    \end{aligned}
\end{aligned}
\end{equation}

and

\begin{equation}
\begin{aligned}
    &\frac{M_{q}(H_{o}(a,m+1,n-1,k:a_1,\cdots,a_k))}{M_{q}(H_{o}(a,m+1,n,k:a_1,\cdots,a_k))}\\
    &
    \begin{aligned}
        =&\prod_{i=1}^{m+k+1}\Bigg[\frac{2}{q^{(n+k)-i}+q^{i-(n+k)}}\Bigg]\cdot\frac{<m+n+2k>_{q}!<2a+2b_1+n+k>_{q}!}{<n+k-1>_{q}!<2a+2b_1+m+n+2k+1>_{q}!}\\
        &\cdot\prod_{i=0}^{k-1}\Bigg[\frac{<n+k-i-1>_{q^2}<a+b_1+\frac{1}{2}m+\frac{1}{2}n+k-i+\frac{1}{2}>_{q^2}<b_{i+1}+\frac{1}{2}m+\frac{1}{2}n+k-i>_{q^2}}{<a+b_1+b_{i+1}+n+k-i>_{q^2}<\frac{1}{2}m+\frac{1}{2}n+k-i-\frac{1}{2}>_{q^2}<\frac{1}{2}m+\frac{1}{2}n+k-i>_{q^2}}\Bigg].
    \end{aligned}
\end{aligned}
\end{equation}

Hence,

\begin{equation}
\begin{aligned}
    &\frac{M_{q}(H_{e}(a,m+2,n,k:a_1,\cdots,a_k))}{M_{q}(H_{e}(a,m+1,n,k:a_1,\cdots,a_k))}\cdot\frac{M_{q}(H_{o}(a,m,n,k:a_1,\cdots,a_k))}{M_{q}(H_{o}(a,m+1,n,k:a_1,\cdots,a_k))}\\
    &+\frac{M_{q}(H_{e}(a,m+1,n+1,k:a_1,\cdots,a_k))}{M_{q}(H_{e}(a,m+1,n,k:a_1,\cdots,a_k))}\frac{M_{q}(H_{o}(a,m+1,n-1,k:a_1,\cdots,a_k))}{M_{q}(H_{o}(a,m+1,n,k:a_1,\cdots,a_k))}\\
    &
    \begin{aligned}
        =&\frac{q^{m+k+1}+q^{-(m+k+1)}}{q^{n-m-1}+q^{-(n-m-1)}}\cdot\frac{<m+k+1>_{q}}{<m+n+2k+1>_{q}}\cdot\frac{<m+1>_{q^2}<\frac{1}{2}m+\frac{1}{2}n+k+\frac{1}{2}>_{q^2}}{<m+k+1>_{q^2}<\frac{1}{2}m+\frac{1}{2}n+\frac{1}{2}>_{q^2}}\\
        &+\frac{q^{n+k}+q^{-(n+k)}}{q^{n-m-1}+q^{-(n-m-1)}}\cdot\frac{<n+k>_{q}}{<m+n+2k+1>_{q}}\cdot\frac{<n>_{q^2}<\frac{1}{2}m+\frac{1}{2}n+k+\frac{1}{2}>_{q^2}}{<n+k>_{q^2}<\frac{1}{2}m+\frac{1}{2}n+\frac{1}{2}>_{q^2}}
    \end{aligned}
    \\
    &=\frac{<m+1>_{q^2}+<n>_{q^2}}{(q^{n-m-1}+q^{-(n-m-1)})<\frac{1}{2}m+\frac{1}{2}n+\frac{1}{2}>_{q^2}}\\
    &=1.
\end{aligned}
\end{equation}
and this verifies Step 3$'$.

We just showed $M_{q}(H_{e}(a,m,n,k:a_1,\cdots,a_{k-1},a_k))$ and $M_{q}(H_{o}(a,m,n,k:a_1,\cdots,a_{k-1},a_k))$ are given by (2.2) and (2.3) when $n-m=E$ and the depths of the regions are $D$. Thus, by the inner induction, we can conclude that the TGFs of the two families of regions are given by (2.2) and (2.3) when $n-m=E$. This verifies the induction step of the outer induction, so we can conclude that (2.2) and (2.3) are true for any $m$ and $n$ with $n\geq m$. This completes the proof of Theorem 2.2. \qedsymbol

\begin{thebibliography}{999}

\bibitem{B}
  S. Byun,
  \emph{Lozenge tilings of a hexagon with a horizontal intrusion},
  Ann. Comb. \textbf{26} (2022), no. 4,
  943-970.

\bibitem{BL1}
  S. Byun, T. Lai
  \emph{Lozenge Tilings of Hexagons with Intrusions II.},
  in preparation.
  
\bibitem{BL2}
  S. Byun, T. Lai
  \emph{Lozenge Tilings of Hexagons with Intrusions III.},
  in preparation.  

\bibitem{C1}
  M. Ciucu,
  \emph{Enumeration of perfect matchings in graphs with reflective symmetry},
  J. Combin. Theory Ser. A \textbf{77} (1997),
  67-97.

\bibitem{C2}
  M. Ciucu,
  \emph{Plane Partition I: A generalization of MacMahon's formula},
  Mem. Amer. Math. Soc. \textbf{178} (2005), no. 839,
  107-144.

\bibitem{C3}
  M. Ciucu,
  \emph{The other dual of MacMahon's theorem on plane partitions},
  Adv. Math. \textbf{306} (2017),
  427-450.

\bibitem{CEKZ}
  M. Ciucu, T. Eisenkolbl, C. Krattenthaler, D. Zare
  \emph{Enumeration of lozenge tilings of hexagons with a central triangular hole},
  J. Combin. Theory Ser. A \textbf{95} (2001), no. 2,
  251–334.

%\bibitem{CK1}
%  M. Ciucu and C. Krattenthaler,
%  \emph{The number of centered lozenge tilings of a symmetric hexagon},
%  J. Combin. Theory Ser. A \textbf{86}, no. 1, (1999),
%  103-126.

%\bibitem{CK2}
%  M. Ciucu and C. Krattenthaler,
%  \emph{Enumeration of Lozenge tilings of hexagons with cut-off corners},
%  J. Combin. Theory Ser. A \textbf{100} (2002),
%  201-231.  
  
\bibitem{CK3}
  M. Ciucu and C. Krattenthaler,
  \emph{A dual of MacMahon's theorem on plane partitions},
  Proc. Natl. Acad. Sci. USA \textbf{110} (2013), no. 12,
  4518–4523. 

\bibitem{CL}
  M. Ciucu and T. Lai,
  \emph{Lozenge tilings of doubly-intruded hexagons},
  J. Combin. Theory Ser. A \textbf{167} (2019),
  294–339.

\bibitem{DT}
  G. David, C. Tomei,
  \emph{The problem of the calissons},
  Amer. Math. Monthly, \textbf{96} (1989),
  429-431.

%\bibitem{Fi}
%  I. Fischer,
%  \emph{Enumeration of rhombus tilings of a hexagon which contain a fixed rhombus in the centre},
%  J. Combin. Theory Ser. A \textbf{96} (2001),
%  31-88.
  
\bibitem{Fu1}
  M. Fulmek,
  \emph{Graphical condensation, overlapping Pfaffians and superpositions of matchings},
  Electron. J. Combin. \textbf{17} (2010), no. 1, Research Paper 83,
  42 pp.  

\bibitem{Fu2}
  M. Fulmek,
  \emph{Tilings of damaged hexagons},
  preprint arXiv:2302.00959. 

%\bibitem{FK1}
%  M. Fulmek and C. Krattenthaler,
%  \emph{The number of rhombus tilings of a symmetric hexagon which contain a fixed rhombus on the symmetry axis, I},
%  Ann. Comb. \textbf{2} (1998),
%  19-41.

%\bibitem{FK2}
%  M. Fulmek and C. Krattenthaler,
%  \emph{The number of rhombus tilings of a symmetric hexagon which contain a fixed rhombus on the symmetry axis, II},
%  European J. Combin. \textbf{21} (2000),
%  601-640.
  
%\bibitem{GH}
%  I. Gessel and H. Helfgott,
%  \emph{Enumeration of tilings of diamonds and hexagons with defects.},
%  Electron. J. Combin. \textbf{6} (1999),
%  \#R16, 26 pp.   

\bibitem{Ka}
  P.W. Kasteleyn,
  \emph{Dimer statistics and phase transitions},
  J. Mathematical Phys., \textbf{4} (1963)
  287–293.

\bibitem{KO}
  C. Krattenthaler and S. Okada,
  \emph{The number of rhombus tilings of a "punctured'' hexagon and the minor summation formula},
  Adv. in Appl. Math., \textbf{21} (1998)
  381–404.

\bibitem{Kuo}
  E. H. Kuo,
  \emph{Applications of graphical condensation for enumerating matchings and tilings},
  Theoret. Comput. Sci, \textbf{319} (2004),
  29-57.

%\bibitem{Kup}
%  G. Kuperberg,
%  \emph{Symmetries of plane partitions and the permanent-determinant method},
%  J. Combin. Theory Ser. A \textbf{68} (1994), no. 1,
%  115–151.

\bibitem{L1}
  T. Lai,
  \emph{A q-enumeration of lozenge tilings of a hexagon with three dents},
  Adv. in Appl. Math. \textbf{82} (2017),
  23–57.

\bibitem{L2}
  T. Lai,
  \emph{A q-enumeration of lozenge tilings of a hexagon with four adjacent triangles removed from the boundary},
  European J. Combin. \textbf{64} (2017),
  66–87.

\bibitem{L3}
  T. Lai,
  \emph{Lozenge tilings of hexagons with central holes and dents},
  Electron. J. Combin. \textbf{27} (2020), no. 1, Paper No. 1.61,
  63 pp.

\bibitem{L4}
  T. Lai,
  \emph{Tiling Enumeration of Hexagons with Off-central Holes},
  Electron. J. Combin. \textbf{29} (2022), no. 1, Paper No. 1.41,
  76 pp.

\bibitem{L5}
  T. Lai,
  \emph{Ratio of tiling generating functions of semi-hexagons and quartered hexagons with dents},
  Enumer. Comb. Appl. \textbf{2} (2022), no. 1, Paper No. S2R5, 14.

\bibitem{LR1}
  T. Lai, R. Rohatgi,
  \emph{Enumeration of lozenge tilings of a hexagon with a shamrock missing on the symmetry axis},
  Discrete Math. \textbf{342} (2019), no. 2,
  451–472.

\bibitem{LR2}
  T. Lai, R. Rohatgi,
  \emph{Tiling generating functions of halved hexagons and quartered hexagons},
  Ann. Comb. \textbf{25} (2021), no. 2,
  471–493.
  
\bibitem{M}
  P. A. MacMahon,
  \emph{Combinatory Analysis},
  vol. 2, Cambridge (1916), reprinted in by Chelsea, New York (1960).

\bibitem{Pe}
  J. K. Percus,
  \emph{One more technique for the dimer problem},
  J. Mathematical Phys. \textbf{10} (1969),
  1881–1888.

%\bibitem{Pr}
%  J. Propp,
%  \emph{Enumeration of matchings: problems and progress},
%  New perspectives in algebraic combinatorics, Cambridge Univ. Press, 1999,
%  255–291.
  
\bibitem{R}
  H. Rosengren,
  \emph{Selberg integrals, Askey-Wilson polynomials and lozenge tilings of a hexagon with a triangular hole},
  J. Combin. Theory Ser. A \textbf{138} (2016),
  29–59.

\end{thebibliography}
\end{document}