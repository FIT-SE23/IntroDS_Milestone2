%%
%% This is file `sample-xelatex.tex',
%% generated with the docstrip utility.
%%
%% The original source files were:
%%
%% samples.dtx  (with options: `sigconf')
%% 
%% IMPORTANT NOTICE:
%% 
%% For the copyright see the source file.
%% 
%% Any modified versions of this file must be renamed
%% with new filenames distinct from sample-sigconf.tex.
%% 
%% For distribution of the original source see the terms
%% for copying and modification in the file samples.dtx.
%% 
%% This generated file may be distributed as long as the
%% original source files, as listed above, are part of the
%% same distribution. (The sources need not necessarily be
%% in the same archive or directory.)
%%
%% Commands for TeXCount
%TC:macro \cite [option:text,text]
%TC:macro \citep [option:text,text]
%TC:macro \citet [option:text,text]
%TC:envir table 0 1
%TC:envir table* 0 1
%TC:envir tabular [ignore] word
%TC:envir displaymath 0 word
%TC:envir math 0 word
%TC:envir comment 0 0
%%
%%
%% The first command in your LaTeX source must be the \documentclass command.
\documentclass[sigconf,anonymous,review]{acmart}
%% NOTE that a single column version may be required for 
%% submission and peer review. This can be done by changing
%% the \doucmentclass[...]{acmart} in this template to 
%% \documentclass[manuscript,screen]{acmart}
%% 
%% To ensure 100% compatibility, please check the white list of
%% approved LaTeX packages to be used with the Master Article Template at
%% https://www.acm.org/publications/taps/whitelist-of-latex-packages 
%% before creating your document. The white list page provides 
%% information on how to submit additional LaTeX packages for 
%% review and adoption.
%% Fonts used in the template cannot be substituted; margin 
%% adjustments are not allowed.
%%
%%
%% \BibTeX command to typeset BibTeX logo in the docs

%
%  Uncomment \acmBooktitle if th title of the proceedings is different
%  from ``Proceedings of ...''!
%
%\acmBooktitle{Woodstock '18: ACM Symposium on Neural Gaze Detection,
%  June 03--05, 2018, Woodstock, NY} 


%%
%% Submission ID.
%% Use this when submitting an article to a sponsored event. You'll
%% receive a unique submission ID from the organizers
%% of the event, and this ID should be used as the parameter to this command.
%%\acmSubmissionID{123-A56-BU3}

%%
%% For managing citations, it is recommended to use bibliography
%% files in BibTeX format.
%%
%% You can then either use BibTeX with the ACM-Reference-Format style,
%% or BibLaTeX with the acmnumeric or acmauthoryear sytles, that include
%% support for advanced citation of software artefact from the
%% biblatex-software package, also separately available on CTAN.
%%
%% Look at the sample-*-biblatex.tex files for templates showcasing
%% the biblatex styles.
%%

%%
%% The majority of ACM publications use numbered citations and
%% references.  The command \citestyle{authoryear} switches to the
%% "author year" style.
%%
%% If you are preparing content for an event
%% sponsored by ACM SIGGRAPH, you must use the "author year" style of
%% citations and references.
%% Uncommenting
%% the next command will enable that style.
%%\citestyle{acmauthoryear}


\setcopyright{none}
\settopmatter{printacmref=false} % Removes citation information below abstract
\renewcommand\footnotetextcopyrightpermission[1]{} % removes footnote with conference information in first column

%%
%% end of the preamble, start of the body of the document source.
\begin{document}

%%
%% The "title" command has an optional parameter,
%% allowing the author to define a "short title" to be used in page headers.
\title{Appendix for WSDM2023 Submission: FolkScope}

%%
%% The "author" command and its associated commands are used to define
%% the authors and their affiliations.
%% Of note is the shared affiliation of the first two authors, and the
%% "authornote" and "authornotemark" commands
%% used to denote shared contribution to the research.

%%
%% By default, the full list of authors will be used in the page
%% headers. Often, this list is too long, and will overlap
%% other information printed in the page headers. This command allows
%% the author to define a more concise list
%% of authors' names for this purpose.

%%
%% The abstract is a short summary of the work to be presented in the
%% article.

%%
%% The code below is generated by the tool at http://dl.acm.org/ccs.cfm.
%% Please copy and paste the code instead of the example below.
%%

\settopmatter{printacmref=false, printccs=false, printfolios=false}

%%
%% This command processes the author and affiliation and title
%% information and builds the first part of the formatted document.
\maketitle

\appendix

\section{Annotation Guideline}

Workers satisfying the following three requirements are invited to participate: (1) at least 90\% lifelong HITs approval rate, (2) at least 1,000 HITs approved, and (3) achieving 80\% accuracy on at least 10 qualification questions, which are carefully selected by authors of this paper. Qualified workers will be further invited to annotate 16 tricky assertions. Based on workers' annotations, they will receive personalized feedback containing explanations of the errors they made along with advice to improve their annotation accuracy. Workers surpassing these two rounds are deemed qualified for main round annotations. To avoid spamming, experts will provide feedback for all workers based on a sample of their main rounds' annotations from time to time.


We conduct human annotations and evaluations on the Amazon Mechanical Turk using the Figure~\ref{fig:validity_question} for the first-stage plausibility annotation and the Figure~\ref{fig:quality_question} for the second-stage typicality annotation.

\begin{figure}[t]
      \centering
       \includegraphics[scale=0.30]{figure/image_card.pdf}
       \caption{The item card in our annotation template. Three item's images and item's name and category information are provided for both items. Clicking the item's name (in red) will jump to the item's shopping webpage.}
       \label{fig:image_card}
\end{figure}


\begin{figure}[t]
      \centering
      \includegraphics[scale=0.4]{figure/validity_question.pdf}
      \caption{The question card in our plausibility annotation round. A prompted assertion with its corresponding relation is presented to Turk worker. Workers can choose one from valid, invalid, and unfamiliar.}
      \label{fig:validity_question}
\end{figure}


\begin{figure}[t]
      \centering
      \includegraphics[scale=0.4]{figure/quality_question.pdf}
      \caption{The question card in our typicality annotation round. A prompted assertion with its corresponding relation is presented to Turk worker. Workers can choose one from valid, invalid, and unfamiliar.}
      \label{fig:quality_question}
\end{figure}


\section{Knowledge Population}

Using different confidence cutting-off thresholds leads to trade-offs between the accuracy of generation and the size of the corpus. 
Higher values result in conservative selections that favor precision over recall, whereas lower ones tend to recall more plausible assertions.
We plotted four cutoff points in Figure~\ref{fig:prcurve}.


\begin{figure}
     \centering
     \includegraphics[scale=0.4]{figure/prcurve.pdf}
     \caption{The precision-recall curve of our plausibility population classifier on the human-labeled validation set. The annotated points show the different thresholds~(cutoffs) to filter the generated assertions, i.e. from left to right: 0.9, 0.8, 0.7, 0.5 respectively.}
     \label{fig:prcurve}
\end{figure}


\begin{table}[h]\small
\centering
\caption{Frequent pattern coverage on human-annotated knowledge.}\label{tab:coverage}
\setlength\tabcolsep{4pt}
\begin{tabular}{l|c|c|c}
     \toprule
     Type & Relation & \# of Patterns & Coverage \\
     \midrule
     \multirow{10}{*}{\begin{minipage}{0.4in}Item\end{minipage}}& \textit{RelatedTo} & 14 & 96.94 \\
     & \textit{IsA} & 15 & 97.20 \\
     & \textit{HasA} & 12 & 99.30 \\
     & \textit{PartOf} & 8 & 99.83 \\
     & \textit{MadeOf} & 13 & 99.45 \\
     & \textit{SimilarTo} & 7 & 22.22 \\
     & \textit{CreatedBy} & 14 & 98.59 \\
     & \textit{HasProperty} & 16 & 63.20 \\
     & \textit{DistinctFrom} & 9 & 97.30 \\
     & \textit{DerivedFrom} & 20 & 100.00 \\
     \midrule
     \multirow{5}{*}{\begin{minipage}{0.4in}Function\end{minipage}}& \textit{UsedFor} & 2 & 96.57 \\
     & \textit{CapableOf} & 13 & 74.68 \\
     & \textit{DefinedAs} & 27 & 95.99 \\
     & \textit{SymbolOf} & 9 & 99.76\\
     & \textit{MannerOf} & 34 & 98.56 \\
     \midrule
     \multirow{4}{*}{\begin{minipage}{0.4in}Human\end{minipage}}& \textit{Cause} & 21 & 93.68 \\
     & \textit{Result} & 0 & 0 \\
     %& \textit{MotivatedBy} & \\
     & \textit{CauseDesire} & 0 & 0 \\
     \midrule 
     \textit{Overall} & / & 256 & 80.77 \\
     \bottomrule
\end{tabular}
\end{table}


% \section{CCS Concepts and User-Defined Keywords}

% The ACM Computing Classification System ---
% \url{https://www.acm.org/publications/class-2012} --- is a set of
% classifiers and concepts that describe the computing
% discipline. Authors can select entries from this classification
% system, via \url{https://dl.acm.org/ccs/ccs.cfm}, and generate the
% commands to be included in the \LaTeX\ source.



% \section{Math Equations}
% You may want to display math equations in three distinct styles:
% inline, numbered or non-numbered display.  Each of the three are
% discussed in the next sections.

% \begin{displaymath}
%   \sum_{i=0}^{\infty} x + 1
% \end{displaymath}
% and follow it with another numbered equation:
% \begin{equation}
%   \sum_{i=0}^{\infty}x_i=\int_{0}^{\pi+2} f
% \end{equation}
% just to demonstrate \LaTeX's able handling of numbering.








%%
%% The acknowledgments section is defined using the "acks" environment
%% (and NOT an unnumbered section). This ensures the proper
%% identification of the section in the article metadata, and the
%% consistent spelling of the heading.


%%
%% The next two lines define the bibliography style to be used, and
%% the bibliography file.
\bibliographystyle{ACM-Reference-Format}
\bibliography{wsdm}

%%
%% If your work has an appendix, this is the place to put it.
%\appendix





\end{document}
\endinput
%%
%% End of file `sample-xelatex.tex'.
