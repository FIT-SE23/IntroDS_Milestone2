\documentclass[a4paper,twoside]{article}

\usepackage{epsfig}
\usepackage{subfigure}
\usepackage{calc}
\usepackage{amssymb}
\usepackage{amstext}
\usepackage{amsmath}
\usepackage{amsthm}
\usepackage{multicol}
\usepackage{pslatex}
\usepackage{apalike}
\usepackage{graphicx}
\usepackage{color}
\usepackage{SCITEPRESS}     % Please add other packages that you may need BEFORE the SCITEPRESS.sty package.

\subfigtopskip=0pt
\subfigcapskip=0pt
\subfigbottomskip=0pt

\begin{document}

\title{Phase Distribution in Probabilistic Movement Primitives, Representing Time Variability for the Recognition and Reproduction of Human Movements}

%\author{\authorname{Vittorio Lippi, Raphael Deimel}
%\affiliation{Technische Universit{\"a}t Berlin, Fachgebiet Regelungssysteme, Sekretariat EN11, Einsteinufer 17, D-10587 Berlin, Germany}
%\email{\{vittorio.lippi@tu-berlin.de, raphael.deimel@tu-berlin.de}
%}

\keywords{ProMP, Human Movement, Prediction, Recognition, Time Warping, Phase}

\abstract{Probabilistic Movement Primitives (ProMPs) are widely used as a representation of movement for human-robot interaction. ProMPs are also suited to represent movements subject to temporal scaling. The aim of this paper is to investigate such property specifically. In this work ProMP will be used to model movement distributions based on recorded human movements, in particular, hand trajectories in a two-dimensional reaching task. Movements can be performed at different velocities and such temporal modulation can be taken into account by a phase variable. A system for the simultaneous recognition of movement and phase is proposed. The performance will be discussed both in respect to movement recognition and movement reproduction.}

\onecolumn \maketitle \normalsize \vfill

\section{\uppercase{Introduction}}
\paragraph{Overview.} Probabilistic movement primitives (ProMP) \cite{paraschos2013probabilistic} are a representation of movement widely used in robot control and human-robot interaction (HRI) applications (see for example \cite{maeda2017probabilistic}). ProMP are designed with several desirable properties that make them suited to represent distributions of movements that can be used to model tasks for robot control and HRI\cite{paraschos2018using}. One of these properties is temporal scaling: the movement trajectory is not a direct function of time but function of a \textit{phase} variable, $\phi(t)$. The temporal evolution of $\phi$ determines the the velocity of the movement independently of the shape of its trajectory. 

\begin{figure}[htbp]
	\centering
		\includegraphics[width=1.00\columnwidth]{./Figures/Fig1BasisNphase.eps}
	\caption{Features and Phase. In (a) the features $\Phi_i(\phi)$ are represented as a function of the phase. In (b) a phase profile, based on the \textsl{beta function} as a function of time. In (c) the features are represented as a function of the time.}
	\label{fig:Fig1BasisNphase}
\end{figure}

A movement primitive representing a sample movement $y(t)$ can be expressed as
\begin{equation}
	y(t)=\left[ 
	\begin{array}[pos]{c}
		q \\
		\dot{q}
	\end{array}\right]
	=\Phi(\phi(t))w+\epsilon
	\label{eqdef}
\end{equation}
where $q$ is the vector of variables describing the movement (e.g. joint angles or hand effector position and orientation) and $\Phi$ is a $Nx2$ vector of bell-shaped functions and their derivatives defined as
\begin{equation}
	\Phi_i = \frac{exp^{(\phi(t)-c_i)^2/h_i}}{\sum^{N}_{k=1} exp^{(\phi(t)-c_k)^2/h_k}}
	\label{feateq}
\end{equation}
Where $c_i$ represents the \textit{center}  and $h_i$ expresses the spread of the bell-shaped feature. The model is probabilistic, the vector $w$ is drawn from the distribution, defined by the parameters $\theta$, 
\begin{equation}
p(w|\theta)=\mathcal{N}(\mu_w,\Sigma_w)
	\label{wdist}
\end{equation}
In a general formulation (see for example \cite{colome2014dimensionality}) the noise process can be taken into account, i.e., the $\epsilon$ in Eq. \ref{eqw}  is characterized by a distribution such as $p(\epsilon)=\mathcal{N}(0,\Sigma_y)$. In this work we assume that all the variability observed on $y$ can only be accounted by the distribution of $w$ and that $\epsilon$ can be neglected. In this way we can represent each observed movement $y_i$ with a weight vector $w_i$. Similarly, the vector $w$ can be seen as a compressed representation of the movement trajectory, a projection on a low dimensional subspace:
\begin{equation}
	w=(\Phi(\phi)^T\Phi(\phi))^{-1} \Phi^T(\phi)y
	\label{eqw}
\end{equation}
where dependence on phase (and time) has been omitted for simplicity. The choice of the criteria used for the projection may vary, e.g. it can take into account jerk minimization as regularization principle \cite{paraschos2018using}, alternatively features for movement representation can be designed to guarantee some properties, for example perfect tracking of final position and speed \cite{lippi2012method}. In general the probabilistic nature of the ProMP framework provides a way to impose constrains on generated trajectories by means of conditioning the probability distribution\cite{paraschos2018using}. 
In Figure \ref{fig:Fig1BasisNphase} the features are shown as a function of time and phase. The time modulation performed by the phase function has the effect of modulating the resolution of the low dimensional representation during the movement execution, e.g. in figure \ref{fig:Fig1BasisNphase} (c) features are more dense in the middle of the movement and more sparse at the beginning and at the end. Time modulation is based on the assumption that part of the variability observed among the repetitions of a movement depends on the different velocity at which the movement is performed. The time scaling can be performed by means of a linear constant, i.e. $\phi= \alpha t$, as shown, for example, in \cite{maeda2017probabilistic} and \cite{ewerton2015modeling}. In this work the phase profile assigned to each observation of a given movement will be defined as the one that minimizes the variances of $w$ (or, equivalently of the response in phase domain), under the constraint of an assumed structure for the phase profile. 
In particular we propose a parametric phase profile:
\begin{equation}
\phi_{\Delta_1,\Delta_2}(t)=\int_0^t \beta_{2,2}(\frac{\tau-\Delta_1}{\Delta_1-\Delta_2}) d\tau
\label{betaeq}
\end{equation}
where $\beta_{2,2}$ is the beta function distribution with parameters $a=b=2$. This function, shown in \ref{fig:Fig1BasisNphase} (b), has the characteristic of being differentiable, monotonic and saturating at $0$ and $1$. In this framework the phase associated with the single trial depends on the whole distribution of the sample set. Notice that in general the weights $w$ are dependent on the chosen phase profile (see Eq. \ref{eqw}). The average of the sample set distribution is defined as:
\begin{equation}
	\overline{y}= argmin_{\widetilde{y}} \sum_{i=1}^{N}\int_{0}^{1}(y_i(\phi)-\widetilde{y}(\phi))^2d\phi
	\label{mina}
\end{equation}
For the $i^{th}$ sample the phase is computed as:

\begin{equation}
	\phi_i = arg\,min_{\widetilde{\phi}} \int_{0}^{1}(y_i(\widetilde{\phi}(\tau(\phi)))-\overline{y}(\phi))^2d\phi
  \label{minb}
\end{equation}
Where the alignment between the average $\overline{y}$ and the $y_i$ sample is defined by the comparison of the respective phases, i.e:
\begin{equation}
  \tau(\phi)=arg_t \overline{y}(\phi)=\widetilde{\phi}(t)
\end{equation}
As $\overline{y}$ depends on all the phase profiles, Eq. \ref{mina} and Eq.\ref{minb} represent a single problem of finding the optimal phase profiles. The problem is solved iterating the computation of $\overline{\phi}(t)$ and the $\phi_i$. This results in a phase profile for each sample movement. The phase profile of the average trajectory is fixed, because the time scaling when comparing two or more movements (e.g. when computing the average) depends on the relationship between the respective phase profiles and hence there is a degree of redundancy. In this work the phase profile of the average trajectory is set to the one in Eq.\ref{betaeq} with $\Delta_1 =\Delta_2 = 0$. The described time scaling can be performed on time domain values. In order to take into account the movement primitive representation the $y$ in Eq. \ref{mina} and Eq. \ref{minb} can be projected on the MP representation $y_{mp}=\Phi w$.
Once the time scaling of the sample set is performed, the probability distributions describing the movements can be obtained with empirical estimators. In the following paragraphs we address the following points: (a) how to identify a model for the movements, (b) how to recognize an observed movement given a set of movement models, (c) how to recognize the current phase, (d) how to integrate perception in the phase recognition process and (e) how to generate a movement.
%
\paragraph{Model identification.}  The transformation of a sample set into a ProMP model can be based on different principles, e.g. linear fit of each single observation or maximizing the likelihood of the observed dataset\cite{paraschos2018using}. As introduced in the overview in this work we assumed that we can neglect $\epsilon$ in Eq. \ref{eqw} and the phase profile can be chosen to minimize the variability on $w$. This leads to a two step procedure: firstly each observation is transformed into a set of parameters $y_i(t) \rightarrow (w_i,phi_i(t))$. This is done using Eq. \ref{mina} and Eq. \ref{minb} to obtain the phase profile, and Eq. \ref{eqw} to obtain $w$, then the distributions for ($w$ 	Eq. \ref{feateq}) and $\phi(t)$ can be obtained using empirical estimators. In particular also the phase is assumed to have a normal distribution $p(\phi(t))=\mathcal{N}(\mu_{\phi}(t),\sigma_{phi}(t))$. An example of model identification is shown in Fig.\ref{lineexample}. The training set is generated artificially and consists of planar motion with random time profile (with the form specified in Eq.\ref{betaeq}) and non-intersecting parabolic trajectories. In Fig.\ref{lineexample} (b) movements generated with the identified distribution are shown. Notice that the assumption of normal distribution of $w$ produces a distribution of trajectory that is different from the more ``uniform'' training set.

\paragraph{Movement recognition} An observed movement $y$ can be labeled by applying a maximum likelihood approach. The probabilities of observing the movement given the ProMP models are computed and the movement is classified as belonging to the most likely model, i.e. $L = argmax_k p(y|\theta_k)$. The probability of the observation $y(t_i)$ conditioned to the $k^{th}$ movement is given by 
\begin{equation}
\small
	p(y(t_i)|\theta_k)=\int_0^1 p(y(t_i)|\Phi(\phi) \mu_k, \Phi(\phi) \Sigma_k \Phi(\phi)^T) p(\phi|t_i) d \phi
	\label{movid}
\end{equation}
Notice that the probability of $y(t_i)$ is considered independent of the previous observations, i.e. $p(y(t_i)|\theta_k,y(t_{i-1})) = p(y(t_i)|\theta_k)$. The probability of an observed sequence can be obtained by multiplying the probabilities of the observed samples, $p(y|\theta_k) = \prod_{0}^{t_f} p(y_t|\theta_k)$.
In the described examples we will include all the past observations of $y_t$, in the general case not all points of the trajectory may be available, for example due to an occlusion, the proMP framework can be applied to such cases as shown in \cite{maeda2017probabilistic}.

\paragraph{Phase recognition} In order to estimate the most likely phase given an observed sample the distribution is computed: 
\begin{equation}
p(\phi|y(t_i))=\frac{p(y(t_i)|\Phi(\phi)\mu_k,\Phi(\phi) \Sigma_k \Phi(\phi)^T) p(\phi|t_i)}{\sum_k^M p(\theta_k) p(\phi|y(t),\theta_k)}
\label{phaseid}
\end{equation}
Notice that $p(\phi|t_i)$ is assumed to have a normal distribution and hence it is represented as mean $\mu_{\phi}(t)$ and standard deviation $\sigma_{\phi}(t)$ that are function of time. The distribution of $\Delta_1$ and $\Delta_2$ are not normal due to the nonlinear relationship in Eq. \ref{betaeq}. In most of the previous work on the topic, $\phi$ is assumed to depend linearly on time both the characterizing parameter and $\phi(t)$ can be described with a normal distribution, e.g. in \cite{ewerton2015modeling}, recently it has been proposed to use a general dynamic time warping (DTW) together with ProMPs \cite{ewerton2018assisting}.
\paragraph{Perception.}  Examples of  integrations of ProMP and perception have been proposed, see for example \cite{dermy2019multi}, where ProMPs are used in the context of predicting human intentions. In this work perception will be focused on the identification of phase from the observed profiles. The phase identification proposed in Eq \ref{phaseid} is based on the empirical distributions obtained from the training set, it can be extended assuming an on-line estimation of the phase based on the sensor input. In general the sensor input can include a larger number of variables with respect to $y$. As an example we will consider an estimator based on previous values of $y$ over a time window. The output of such estimator consists in the current phase $\phi^*(t_i)$. Using this estimation Eq. \ref{movid} becomes:
\begin{equation}
\small
	p(y(t_i)|\theta_k)=p(y(t_i)|\Phi(\phi^*(t_i)) \mu_k, \Phi(\phi^*(t_i)) \Sigma_k \Phi(\phi^*(t_i))^T) 
	\label{percphase}
\end{equation}
%
\paragraph{Movement Generation.} The model can be used to generate movements. Depending on the task the movements can be generated deterministically using Eq. \ref{eqdef} or using a $w$ drawn from the distribution. In certain cases the term $\epsilon$ in Eq. \ref{eqdef} can be due to actuation and external noise, which is not the case in the presented example. Generating movements from ProMP can be used both for robot control and to predict human movements in HRI, see, for example, \cite{oguz2017progressive}. The probabilistic nature of the ProMP representation allows for a representation of target positions (or velocities) constraints and via-points using a conditioned distribution. In particular considering that in the present work the ProMP is represented by the parameters $\mu_w$,$\Sigma_w$ (i.e. $\epsilon$ in Eq. \ref{eqdef} is neglected) the conditioning consists of updating such parameters. A constraint can be expressed as a target point to be reached at a given phase $y^*(\phi_i)$, exploiting the properties of normal distributions and the linear dependency between $w$ and $y$ the updated parameters for $p(w|y_{\phi_i}=y^*(\phi_i))$ becomes
\begin{equation}
	\mu^*_{w}=\mu_w +\Sigma_w \Phi^T_(\phi) \left( \Phi(\phi)\Sigma_w \Phi^T(\phi)\right)^{1} (y^*-\Phi(\phi)\mu_w)
	 \label{munew}
\end{equation}
 \begin{equation}
	\Sigma^*_{w}=\Sigma_w +\Sigma_w \Phi^T_(\phi) \left( \Phi(\phi)\Sigma_w \Phi^T(\phi)\right)^{1} \Phi(\phi)\Sigma_w
	\label{sigmanew}
\end{equation}

An example of generated movements is shown in Fig.\ref{lineexample} (b,c,d). Particularly in (c) the conditioned distributions exhibit a very small $\Sigma_w$ (i.e. all the trajectories starting from a given point are indistinguishable in shape) capturing the regularity of the training set. In Fig.\ref{lineexample} (d) the distribution is conditioned to pass through two via-points that are unlikely to be observed in the same movement, the trajectory does not resemble the samples presented in the training set.  

\begin{figure*}[htb]
	\centering
	 \begin{tabular}{cc}

		\includegraphics[width=1.00\columnwidth]{./Figures/exampleA.eps} &
		\includegraphics[width=1.00\columnwidth]{./Figures/exampleB.eps} \\
		\includegraphics[width=1.00\columnwidth]{./Figures/exampleC.eps} &
		\includegraphics[width=1.00\columnwidth]{./Figures/exampleD.eps} \\
		
		\end{tabular}
	\caption{Parabolic lines. In (a) an artificial data-set of trajectories with a parabolic shape and random duration is shown, only 10 of the 100 produced samples, covering more intermediate positions,  are shown. The color gradient shows the temporal evolution of the trajectory, the black line represents the average trajectory in time, the resulting shape does not match the one of the presented examples, especially in the end when the number of involved samples changes. The data-set has been used as a training set for a proMP model. In (b) lines generated with the proMP distribution are shown, the black line represent the average (computed respect to phase). Notice that while the model captured very accurately the shape of the movements, the gaussian distribution of parameters results in a distribution of trajectory that is different from the one presented in the training set. In (c) conditioned trajectories are shown, with imposed via-points a time $t=0$ represented as black stars. The conditioned $\Sigma_w$ is so small that the trajectories generated for each via-point are not distinguishable. In (c) the trajectory is conditioned to pass through two via-points that are unlikely to appear in the same movement, the resulting shape is not similar to the ones presented in the training set, trajectories from the training set are shown in gray for comparison}
	\label{lineexample}
\end{figure*}


\section{Dataset and Benchmark Problem}

\begin{figure}
	\centering
		\includegraphics[width=1.00\columnwidth]{./Figures/Fig1dataset.eps}
	\caption{The sample data set. A two-dimensional reaching task performed by a subject on a computer screen using a mouse. The user was required to perform the movement as fast as possible to reach circular targets that were appearing in a position randomly decided among $4$. The resulting set is composed by $25$ repetitions of $4$ movements.}
	\label{dataset}
\end{figure}

The sample dataset consists in a reaching task: Users have been asked to move the mouse cursor from a starting point to targets appearing in 4 different positions. Each target identifies a different movement, associated to a different ProMP model. The samples are shown in Fig. \ref{dataset}. Given the nature of the task, consisting in reaching movements, it comes natural to perform the segmentation on the basis of events, in particular: (a) a movement starts when the mouse cursor is in the starting position and the target is shown on the screen and (b) it stops when a target is reached. In general different heuristics are possible for segmentation, notably the ProMP framework provides the possibility to segment the signal exploiting an \textit{expectation-maximization} algorithm, treating segmentation as a latent variable to be optimized together with the movement models parameters\cite{lioutikov2017learning}. The dataset includes 100 movement samples used as training-set for the \textit{model identification} and $100$ samples used as test set for the \textit{movement recognition} and \textit{phase recognition}. Movements are sampled at $100$ Hz. The number of features is set to $N=9$, with parameters, in Eq. \ref{feateq}, $c_k=(k-1)/(N-1)$ and $h_k=0.15$. 


\section{Results}
\begin{figure}[htbp]
	\centering
		\includegraphics[width=0.90\columnwidth]{Figures/Fig3Fiteps.eps}
	\caption{Dataset "mov 1" (see Fig. \ref{dataset}). In (a) vertical and horizontal components together, gray lines represent single repetition, black line represents the average, as a function of time. In (b) the distribution of phase profiles, in gray the ones associated with the observations, in black the average one. In (c) the movements and the average movement are represented in phase domain.}
	\label{fig:Fig3Fiteps}
\end{figure}

\begin{figure}[htbp]
	\centering
		\includegraphics[width=0.90\columnwidth]{Figures/ResultsLabelPhase.eps}
	\caption{Classification and phase evaluation accuracy evaluated on a test set of 100 samples (25 for each of the 4 recognized movements). In (a) the relative error is shown as a function of time (blue). As the different test samples have different durations. The number of total movement samples compared at a given time is shown (orange). The classification accuracy increases as the sample movements are observed. In (b) the phase identified phase profiles are shown (gray). Notice that the recognized phase is computed at each time using Eq. \ref{phaseid} and selecting the most likely phase: the resulting phase profile is not the one expressed in the training set (Eq. \ref{betaeq}), and in general it is not guaranteed to be monotonic. Nevertheless, the average phase (black) shows the typical sigmoid profile.}
	\label{fig:ResultsLabelPhase}
\end{figure}

\paragraph{Model identification, movement recognition and phase recognition.} The distribution of phase profiles obtained with Eq. \ref{mina}, Eq. \ref{minb} is shown in Fig \ref{fig:Fig3Fiteps} (b). Each sample is transformed in an MP, i.e. the parameters $(w,\phi)$ are extracted, $4$ \textit{ProMP} models representing the $4$ movements are estimated. The observed movement is identified as belonging to the most likely model to produce it according to Eq. \ref{movid}. The movement identification is performed for each time step, the classification accuracy increases with the time as shown in Fig. \ref{fig:ResultsLabelPhase} (a). Not surprisingly the classification accuracy is $100\%$ at the end of the movement when the movements are unambiguously differentiated by the position. Phase recognition is also performed for each time step using Eq. \ref{phaseid}. The resulting profile is obtained sample-by-sample and may not be monotonic as shown in \ref{fig:ResultsLabelPhase} (b). 

\paragraph{Movement Generation.}

\begin{figure*}[htb]
	\centering
	
	\begin{tabular}{cc}
		\includegraphics[width=1.00\columnwidth]{Figures/GeneratedHumanA.eps} &
		\includegraphics[width=1.00\columnwidth]{Figures/GeneratedHumanB.eps} \\
		\includegraphics[width=1.00\columnwidth]{Figures/GeneratedHumanC.eps} &
		\includegraphics[width=1.00\columnwidth]{Figures/GeneratedHumanD.eps} \\
		\end{tabular}	
			
	\caption{Movements generated using the ProMP model for the set "mov 1" (see Fig. \ref{dataset}). The trajectories in red are randomly generated by sampling from the $w$ distribution. The trajectories in blue are constrained to pass through a via-point at phase $\phi^*=0.5$ represented as a black star. The darker thick lines represent the average for the two groups of generated movements. The red circle is the final target. In (a) the imposed via-point is within the space of the presented examples and hence the trajectories resemble thoses presented in the training set. In (b) the via-point is more distant from the positions presented in the training set and hence the conditioned trajectories are less representative of the original distribution. In (c) the conditioned distribution is used together with the classification and the phase identification to produce a prediction of an observed movement (black dotted line). The most likely trajectory is plotted with the blue thick line, the thin blue lines are generated from the conditional distribution probability. In (d) the prediction is performed using a neural network to estimate the phase.}
	\label{fig:Generated}
\end{figure*}

In order to generate movements the parameters $w$ are sampled from the distribution. Fig. \ref{fig:Generated} shows movements  generated with and without constraints. The constraint is expressed as a via-point to be reached at a given phase. The constrained movements are generated from the conditioned distribution with parameters from Eq. \ref{munew}, Eq. \ref{sigmanew}. In Fig. \ref{fig:Generated}(a) the imposed via-point is within the space of the presented examples and hence the trajectories resemble those presented in the training set. In Fig. \ref{fig:Generated} (b) the via-point is more distant from the positions presented in the training set and hence the conditioned trajectories are less representative of the original distribution, e.g. forming loops and missing the final target. In (c) the conditioned distribution is used  to produce a prediction an observed movement. The resulting prediction is also affected by the uncertainty on the current phase that is estimated according to Eq. \ref{phaseid}. 
\paragraph{Perception.} The phase estimation was implemented as a neural network with three layers (respectively with 40, 20 and 10 neurons) taking the current time and a vector of 20 previous samples as an input. The results are shown in Fig. \ref{fig:Resultsperc} where the classification task (shown in Fig. \ref{fig:Resultsperc}) is repeated by exploiting the estimated phase, according to Eq. \ref{percphase}. The performance of the classification using the neural network is comparableto that which obtained by integrating the phase distribution (Eq. \ref{movid}). In Fig. \ref{fig:Generated} (d) the phase estimated with the neural network is used to perform a prediction on the basis of the observed trajectory. The result is similar to that which is shown in Fig. \ref{fig:Generated} (c).  

\begin{figure}[htbp]
	\centering
	\includegraphics[width=0.90\columnwidth]{Figures/Resultsperc.eps}
	\caption{Classification and phase evaluation accuracy using a neural network for the identification of phase. The training set and the test set arethe same used in Fig. \ref{fig:ResultsLabelPhase}. In (a) the relative error is shown as a function of time (blue). As the different test samples have different durations, the number of total movement samples compared at a given time is shown (orange). The classification accuracy increases as the sample movements are observed. In (b) the phase identified phase profiles are shown (gray). The phase is computed at each time step. The average phase (black) shows the typical sigmoid profile. Overall, although in some cases the phase looks more inaccurate using the neural network estimator, the classification performance is comparable to that which is shown in Fig. \ref{fig:ResultsLabelPhase}. }		
	\label{fig:Resultsperc}
\end{figure}
%%%% HERE :)
%%%%
%%%%
\paragraph{A 6DoF robot arm example.} In this additional example the model identification is applied to the movements performed with a 6DoF robotic arm. 

\begin{figure}[htbp]
	\centering
		\includegraphics[width=0.70\columnwidth]{Figures/RobotSetup.jpg}
	\caption{The robot task configuration (A).The robot's gripper is holding an hook (B) used to pick up the object (C).}
	\label{robot}
\end{figure}

The task consists in picking up and object and handing it to an user with a single movement. The used configuration is shown in Fig. \ref{robot}. The robot is holding an hook used to pick up the object. The object represents a constraint in space for the trajectory, since it is always picked up in the same position. The trajectory examples were provided moving the robot manually, producing a total of $11$ examples. The task is represented in joint space. The number of features was set to $9$, hence the number of parameters representing the ProMP are $63$.
We can compare the covariance matrix for all the components with and without the phase scaling as shown in Fig. \ref{covmat}. 
The use of the phase modulation produced a decrease in the variance of the most of the ProMP parameters because the different speed was modeled by the phase distribution.
\begin{figure}[htbp]
	\centering
		\begin{tabular}{cc}
		\includegraphics[height=0.35\columnwidth]{Figures/CovNoScalingCROP.pdf}	&	
		\includegraphics[height=0.35\columnwidth]{Figures/CovWithScalingCROP.pdf}
		\end{tabular}
		
	\caption{Color representation of the covariance matrix for $w$, for the robot arm movements set without phase modulation (left) and with phase scaling (right). Most of the values in the covariance matrix are smaller.}
	\label{covmat}
\end{figure}


%%%% HERE :)
%%%%
%%%%

\section{Discussion and Future Work}
In this work we used ProMPs to model human movements in order to perform movement recognition and movement generation with particular emphasis on the possibility of using a parametric phase profile to modulate movement execution speed. The approach, tested on a data-set of reaching movements proved to properly classify and generalize the presented movements. The proposed parametric phase profile has a \textit{sigmoid} shape that is particularly suitable to represent reaching tasks such as the one presented into the example and present in "real world" tasks, such as pick and place operations where it is relevant to the control of initial and final position. However, such phase profile choice assumes that the derivative of the phase profile is zero at the beginning and at the end of the movement. Therefore, in tasks where final velocity must be controlled, other profiles should be considered. In general it should be considered that time scaling of human movements can work differently depending on the task and the working conditions\cite{zhang1999effects}. The presented reaching task is represented as a library of $4$ ProMPs describing $4$ possible movements. Therefore, the movement recognition classifier works with an arbitrary number of time samples as input. In general, the variety of movements expressed in a task can be represented with a manifold in the space of hyper-parameters and mixture distributions \cite{rueckert2015extracting}, or specifically with gaussian mixtures describing the distribution of $w$ \cite{ewerton2015modeling}. The parameters can be constrained with linear relationships for dimensionality reduction, as recently shown in \cite{colome2018dimensionality}. In the case of human movements the dependencies between target positions and the moving parameters can be represented by a linear model\cite{avizzano2011regression,lippi2012method}. In all cases, the phase can be estimated while observing the movement by modifying Eq. \ref{phaseid} in order to take into account the chosen distribution.
    
The introduction of a specific perception component oriented to the identification of phase was described and implemented. Such estimator led to a performance, in terms of classification and prediction accuracy, that was comparable to that of the system exploiting the empirical probability distribution phase. The advantage of using a neural network may consist in a simplification of the system: avoiding solving the integral in Eq.\ref{movid} at each integration step.   

Future work on phase recognition will be oriented to the identification of the state of dynamic system models underlying the process describing the task, e.g. state machines or \textit{stable heteroclinic channel} models \cite{rabinovich2008transient}. Such models could provide phase profiles and could be used in conjunction with a library of proMPs to control a task. Furthermore, the current research on ProMPs includes the design of feedback control systems considering physical interaction with the environment and the users  \cite{paraschos2018using,paraschos2013probabilistic}. Specifically, time scaling does not apply in the same way to the control of arm kinematics and the external contact forces (this holds for each kind of rescaling, also the linear one). A general framework that can scale  the kinematics and  the contact instants in time should be integrated with a consistent control of applied forces and torques.     

%\section*{\uppercase{Acknowledgements}}
%We gratefully acknowledge financial support for the project MTI-engAge (16SV7109) by BMBF

%\vfill
\bibliographystyle{apalike}
{\small
\bibliography{lippi}}


%\section*{\uppercase{Appendix}}

\vfill
\end{document}

