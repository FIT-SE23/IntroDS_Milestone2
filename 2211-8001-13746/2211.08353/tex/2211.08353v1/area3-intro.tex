
The LHC effective field theory working group (LHC EFT WG)~\cite{lhceftwg} gathers members of the LHC experiments 
and the theory community to provide a framework for the interpretation of LHC data in the context of 
effective field theories (EFTs). The LHC EFT WG studies the physics requirements needed to facilitate 
an interpretation commensurate with the available measurements performed in a wide range of different 
processes with Higgs bosons, top quarks, and electroweak bosons. It provides recommendations 
for the use of EFT by the experiments to interpret their data, and a forum for theoretical discussions of 
EFT issues. This includes recommendations on the theory setup as well as Monte Carlo simulation and 
other tools needed for EFT analyses. It focuses on recommendations, developments, 
and combinations that require coordination between the existing WGs (Higgs, Top, Electroweak), in order 
to allow global EFT analyses inside and outside experimental collaborations. 
The following six areas of activity have been identified:
\begin{enumerate}
\item EFT Formalism;
\item Predictions and Tools;
\item Experimental Measurements and Observables;
\item Fits and Related Systematics;
\item Benchmark Scenarios from UV Models;
\item Flavour.
\end{enumerate}

This note is focussed on the topics from Area 3 devoted to {\it experimental measurements and observables}.
This activity area covers how observables relate to operators, which measurements are important 
for a given operator or set of operators, differential/fiducial measurements vs. dedicated ones, identification
of optimal observables, machine learning, re-interpretation vs. static, presentation of results: covariance, 
multi-D likelihood, etc., compatibility with global fits (i.e. assumptions used in deriving measurement and 
reporting results).

The goal is to study observable, channel, process sensitivities and complementarities:
\begin{itemize}
\item Experimental targets: survey of the sensitive channels and corresponding operators;
\item Differential distributions, optimal observables, including machine learning, and dedicated EFT measurements, 
         spin density matrices, EFT-optimized fiducial regions, amplitude analyses, angular distributions (e.g. for $CP$), pseudo observables...
\item Agreement across experiments (for fiducial regions in particular);
\item What observables are most sensitive to new physics? Exploit energy growing effects, non-interferences, and other theoretical knowledge;
\item Expected uncertainties: systematics or statistics dominated.
\end{itemize}

The goal is also to study analysis strategies and experimental outputs, also with a view at legacy measurements and their possible reinterpretation:
\begin{itemize}
\item Dedicated EFT extractions by collaborations;
\item Differential measurements and the best choice of observables for re-interpretation;
\item Presentation of measurements: cross sections, correlations/covariance, multi-D likelihood...
\item Experimental systematics related to EFT (e.g. accounting for detector effects);
\item Detector effects: unfolding, forward folding, efficiency maps, recasting through re-weighting...
\item EFT in backgrounds: final-state driven instead of signal-background, statistical model.
\end{itemize}

%Within the LHC EFT Working Group, we have identified the so-called Area-3 of the WG activities devoted to 
%experimental measurements and observables. Here we relate EFT operators, defined in Area-1 and modeled in Area-2, 
%to experimentally observable effects. Therefore, we survey the sensitive channels, or physical processes, and the
%corresponding operators. Once this connection is established, we examine strategies of experimental analysis of
%the LHC data using the sensitive channels, and determine experimental outputs, or measurements. These 
%measurements become the input to the global EFT fits, further considered in Area-4 of the WG activities. 

This note serves as a guide to experimental measurements leading to EFT fits, but does not establish authoritative 
guidelines how those measurements should be performed. There is a spectrum of experimental approaches. 
% covering differential distributions, optimal observables, including machine learning, dedicated EFT measurements, 
% spin density matrices, EFT-optimized fiducial regions, amplitude analyses, angular distributions, pseudo observables, etc.
Each approach has its own stronger and weaker sides, and none of the approaches has been established as the
universally best approach to perform the measurements. Therefore, one of the goals of this note is to survey these
approaches and identify their key features. 

In order to discuss experimental approaches, we will use the following notation.
We will denote a {\it channel} to be a process used to perform a measurement, for example production
of a top-antitop pair in association with the Higgs boson $t\bar{t}H$ in pp collisions at the LHC. 
We will denote an {\it observable} to be an experimentally defined quantity in such a process, 
for example transverse momentum of the Higgs boson $p_T^H$. 
We will denote a {\it measurement} to be an experimentally delivered quantitative result, 
for example differential cross section in bins of $p_T^H$ in the $t\bar{t}H$ process.  
Experimental {\it measurements} using the {\it observables} sensitive to EFT effects in a given set of 
{\it channels} at the LHC become the input to EFT fits which provide constraints on the EFT operator coefficients.  
