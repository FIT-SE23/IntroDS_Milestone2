One of the goals of this activity area on Experimental Measurements and Observables is to survey which 
experimental channels are sensitive to which EFT operators. The aim is to establish a map between operators
and experimental measurements. 
As a first step we want to determine which operators are relevant and as a second step to 
examine the amount of information that each process can provide for a given operator, given the accuracy of any
given experimental process. The second step can be 
achieved by considering some appropriate metric such as the Fisher information. 
 
A valuable source of information in this endeavour are the global fits which exist in the literature. 
The set of operators relevant for each fit is determined by the processes included in the fit as well as the choice of 
 flavour assumption which determines the relevant degrees of freedom. As an example, in Fig.~\ref{fig:schematic}
 we show a schematic representation of the datasets and their overlapping dependences on the 34 Wilson coefficients 
 included in the global Higgs, top, diboson and EWPO analysis of \verb|fitmaker|~\cite{Ellis:2020unq}. 
% The relevant operators form 
% part of the Warsaw basis, with the ones involving the top quark following the conventions set in the Top Working Group 
% EFT note.  We note that such a simple representation can be further refined by considering different 
% channels in each sector, e.g. examining each production and decay channel of the Higgs separately etc 

\begin{figure}[htbp]
  \begin{center}
    \includegraphics[width=0.5\linewidth]{venn.pdf}
    \caption{\small 
     \label{venn} Schematic representation of the datasets and their overlapping dependences on the 34 Wilson coefficients included in the analysis of Ref.~\cite{Ellis:2020unq}}
    \label{fig:schematic}
  \end{center}
\end{figure}

% ------

A major part of global interpretations of HEP data in the SMEFT framework is establishing the connection between operators and experimental measurements and eventually the generation of predictions for the dependence of each measurement on the Wilson Coefficients of interest. The first step involves starting from a particular flavour assumption and then creating a list of operators entering a given process at some perturbative order. The second step is more quantitative and typically involves a significant undertaking, requiring large scale numerical calculations that take into account details of the experimental analyses. The map from operators to measurements contains a great deal of information in itself, indicating both how the data may collectively constrain each degree of freedom and the origin of potential correlations among them, post-fit. 

Such results only make up half of the required ingredients of a global fit, the other half being the experimental data and covariance matrix. Nevertheless, it is worthwhile to present and dissect them in a systematic way for a number of reasons. From a fundamental perspective, we may understand the potential impacts and interplay that the data imposes on the allowed parameter space of the theory, which can guide our expectations before doing any elaborate fits and also help us to understand the results of such an exercise. Moreover, it serves as a useful validation exercise, where we can verify the consistency of the predictions both internally and against other existing calculations. This is especially important given the fact that such a large scale generation of predictions is inevitably prone to human error. Finally, making the results public allows them to be re-used in future interpretations, avoiding a duplication of effort and therefore hastening progress towards maximal indirect sensitivity to new physics from precision measurements. 


In this note, we establish a map between operators and experimental measurements and quantify the sensitivity of the Higgs, top, and Electroweak measurements that went into the recently published global analyses of \verb|fitmaker|~\cite{Ellis:2020unq} and SMEFiT~\cite{Ethier:2021bye} collaborations.  Whilst the two collaborations follow similar approaches, some differences exist and these will be mentioned in the note where relevant. 

%performed using \verb|fitmaker|. All of the numerical results shown can be found on the \verb|fitmaker| public repository at {\tt https://gitlab.com/kenmimasu/fitrepo}.

\subsection{Calculational framework}

In this note we work in the Warsaw basis of SMEFT dimension-6 operators ~\cite{Grzadkowski:2010es}.
The operator coefficients are normalised as
%
\begin{equation}
    \mathcal{L}_{\text{Dim-6}} = \sum_{i} \frac{C_i}{\Lambda^2} \mathcal{O}_i \, .
\end{equation}
%


%\paragraph{Operator basis.} In the following we adopt the notations
%of the  Warsaw basis.
%
%
Before discussing the operators entering each analysis we briefly 
mention the corresponding flavour assumptions as employed by the
two fitting collaborations. SMEFiT~\cite{Ethier:2021bye} follows  the strategy presented
in~\cite{Hartland:2019bjb,AguilarSaavedra:2018nen}, namely
minimal flavour violation (MFV) hypothesis~\cite{DAmbrosio:2002vsn} in the
quark sector as the baseline scenario. 
 \verb|fitmaker|~\cite{Ellis:2020unq} considers both a flavour universal 
and an MFV scenario. We discuss in detail the operators considered in each analysis below. 
%

\subsubsection{Operator basis in SMEFiT 2021 analysis}
The SMEFiT analysis adopts a  $U(2)_q\times U(2)_u \times U(3)_d$ symmetry
where the Yukawa couplings are nonzero only for the top quark,
consistent with the {\tt SMEFT@NLO} model~\cite{Degrande:2020evl}.
Whilst strictly not consistent with the flavour assumption, the bottom and
charm quark Yukawa operators are included in the fit, to account for the current LHC
sensitivity to these parameters, while all other Yukawas are set to zero.
%
A universal $(U(1)_\ell \times
U(1)_e)^3$ symmetry is adopted in the lepton sector,
which sets  all the lepton masses as well as their Yukawa couplings to
zero. Again, a non-zero $\tau$ Yukawa operator is allowed.

%%%%%%%%%%%%%%%%%%%%%%%%%%%%%%%%%%%%%%%%%%%%%%%%
\input{tables/table-operatorbasis.tex}
%%%%%%%%%%%%%%%%%%%%%%%%%%%%%%%%%%%%%%%%%%%%%%%%

%%%%%%%%%%%%%%%%%%%%%%%%%%%%%%%%%%%%%%
\input{tables/table-bosonicoperators.tex}
%%%%%%%%%%%%%%%%%%%%%%%%%%%%%%%%%%%%%%
      


With this considerations, one ends up with 50 dimension-six EFT degrees of freedom
to be constrained by experimental data.
%
These are summarised in Table~\ref{tab:operatorbasis},
categorised into five disjoint classes, from top to bottom: four-quark (two-light-two-heavy), four-quark (four-heavy), four-lepton, two-fermion, and purely bosonic DoFs. Flavour indices are labelled by $i,j,k$ and $l$; left-handed
quark and lepton fermion SU(2)$_L$ doublets are denoted by $q_i$, $\ell_i$;
the right-handed quark singlets by $u_i$, $d_i$, while
the right-handed lepton singlets are denoted by $e$, $\mu$, $\tau$ without using
flavor index.
%
We  use $Q$ and $t$ to denote the left-handed top-bottom doublet and the right-handed
top singlet, and the Higgs doublet is denoted by $\varphi$.
%
Of these 50 EFT coefficients, 36 are independent fit parameters
while 14 are indirectly constrained by the LEP EWPOs, following
the procedure described in~\cite{Ethier:2021bye}. 

%
When presenting results for operator sensitivity,
the SMEFiT analysis selects the purely bosonic operators $c_{\varphi W B}$ and $c_{\varphi D}$,
for illustration purposes.


Table~\ref{tab:oper_bos} provides the definition of the 
purely bosonic dimension-six operators considered in the analysis, which
modify the production and decay of Higgs bosons as well as the interactions
of the electroweak gauge bosons.
%
In addition to the information provided by the input dataset,
the  operators $O_{\varphi WB}$ and $O_{\varphi D}$
are also severely
constrained by the EWPOs.
%
The triple gauge operator $O_{W}$ generates a TGC coupling modification which is purely transversal
and is hence constrained only by diboson data.

Table~\ref{tab:oper_ferm_bos} collects, using the same format
as in Table~\ref{tab:oper_bos}, the relevant Warsaw-basis operators
that contain two fermion fields, either quarks or leptons,
plus a single four-lepton operator.
%
From top to bottom, the table lists the two-fermion operators involving 3rd generation quarks,
those involving 1st and 2nd generation quarks, and
operators containing two leptonic fields (of any generation).
%
In this list the four-lepton operator $\mathcal{O}_{\ell\ell}$ is also included.
%
The operators that involve a top-quark field, either $Q$ (left-handed doublet) or $t$
(right-handed singlet),
are crucial for the interpretation of LHC top-quark measurements
and all of them involve at least one Higgs-boson field, which
introduces an interplay between the top and Higgs sectors of the SMEFT.
%
We point out that
most of the operator coefficients defined in Table~\ref{tab:oper_ferm_bos} correspond
directly to degrees of freedom used in
the fit, except for three of them, which are
indicated with a (*) in the second column,
for which 
three additional degrees of freedom are defined from the linear
combinations  following ~\cite{AguilarSaavedra:2018nen}, see~\cite{Ethier:2021ydt}.

%%%%%%%%%%%%%%%%%%%%%%%%%%%%%%%%%%%%%%%%
\input{tables/table-2foperators.tex}
%%%%%%%%%%%%%%%%%%%%%%%%%%%%%%%%%%%%%%

\clearpage

Finally, SMEFiT considers four-quark operators which involve the top quark 
fields and thus modify the production of top quarks at hadron colliders.
%
These  can be classified into
 operators composed by four heavy quark fields (top and/or bottom quarks) and 
 operators composed by two light and two heavy quark fields.
 %
 The EFT coefficients  are
constructed in terms of suitable linear combinations of the four fermion
coefficients in the Warsaw basis, 
\begin{align}
	\qq{1}{qq}{ijkl}
	&= (\bar q_i \gamma^\mu q_j)(\bar q_k\gamma_\mu q_l)
	 \nonumber
	,\\
	\qq{3}{qq}{ijkl}
	&= (\bar q_i \gamma^\mu \tau^I q_j)(\bar q_k\gamma_\mu \tau^I q_l)
 \nonumber
	,\\
	\qq{1}{qu}{ijkl}
	&= (\bar q_i \gamma^\mu q_j)(\bar u_k\gamma_\mu u_l)
         \nonumber
	,\\
	\qq{8}{qu}{ijkl}
	&= (\bar q_i \gamma^\mu T^A q_j)(\bar u_k\gamma_\mu T^A u_l)
         \nonumber
	,\\
	\qq{1}{qd}{ijkl}
	&= (\bar q_i \gamma^\mu q_j)(\bar d_k\gamma_\mu d_l)
         \nonumber
	,\\
	\qq{8}{qd}{ijkl}
	&= (\bar q_i \gamma^\mu T^A q_j)(\bar d_k\gamma_\mu T^A d_l)
        \label{eq:FourQuarkOp} %%%%%%%%%%%%%%%%%%%%%%%%
	,\\
	\qq{}{uu}{ijkl}
	&=(\bar u_i\gamma^\mu u_j)(\bar u_k\gamma_\mu u_l)
         \nonumber
	,\\
	\qq{1}{ud}{ijkl}
	&=(\bar u_i\gamma^\mu u_j)(\bar d_k\gamma_\mu d_l)
         \nonumber
	,\\
	\qq{8}{ud}{ijkl}
	&=(\bar u_i\gamma^\mu T^A u_j)(\bar d_k\gamma_\mu T^A d_l)
         \nonumber \, ,
\end{align}
%
In Table~\ref{eq:summaryOperatorsTop} we provide the definition of all four-fermion EFT
degrees of freedom
that enter the fit
in terms of the coefficients of Warsaw basis operators of Eq.~(\ref{eq:FourQuarkOp}).
%
Recall that within our flavour assumptions, the coefficients associated to different values of
the generation indices $i$ ($i=1,2$) or $j$ ($j=1,2,3$) will be the same.

%%%%%%%%%%%%%%%%%%%%%%%%%%%%%%%%%%%%%%%%%%%%%%%%
\input{tables/table-fourfermion.tex}
%%%%%%%%%%%%%%%%%%%%%%%%%%%%%%%%%%%%%%%%%%%%%%%%

\subsubsection{Operators in the fitmaker analysis}

There are 20 operators relevant for Higgs, diboson and electroweak measurements in the flavour universal scenario. In the notation of Ref.~\cite{Grzadkowski:2010es}, these are
%
\begin{align}
    &\mathcal{O}_{H W B}, \mathcal{O}_{H D}, \mathcal{O}_{u}, \mathcal{O}_{H l}^{(3)}, \mathcal{O}_{H l}^{(1)}, \mathcal{O}_{H e}, \mathcal{O}_{H q}^{(3)}, \mathcal{O}_{H q}^{(1)}, \mathcal{O}_{H d}, \mathcal{O}_{H u}, \\
    &\mathcal{O}_{H \square}, \mathcal{O}_{H G}, \mathcal{O}_{H W}, \mathcal{O}_{H B}, \mathcal{O}_{W}, \mathcal{O}_{G},
    \mathcal{O}_{\tau H}, \mathcal{O}_{\mu H}, \mathcal{O}_{b H}, \mathcal{O}_{t H} \,. 
\end{align}
We note here the slightly different notation used as $H$ denotes the Higgs doublet for which Tables~\ref{tab:operatorbasis} and~\ref{tab:oper_ferm_bos} use a $\varphi$. Otherwise, the above operator choice corresponds to a subset of the SMEFiT operators with the following differences:
\begin{itemize}
    \item $\mathcal{O}_{H\Box}$ is equivalent to $\mathcal{O}_{\phi d}$ in SMEFiT up to a minus sign from integration-by-parts
    \item $C_W$ and the SMEFiT $c_{WWW}$ differ by a conventional minus sign
    \item \verb|fitmaker| additionally includes the muon Yukawa operator, $\mathcal{O}_{\mu H}$
\end{itemize}
As mentioned, for the purposes of this note we focus on the Higgs, diboson and electroweak measurements, and relevant operators. We refer the reader to Ref.~\cite{Ellis:2020unq} for the global analysis, which also includes top physics measurements.
%with the $\{\alpha_{EW}, G_F, m_Z\}$ input scheme


\subsection{Experimental measurements}

\subsubsection{Input experimental data in SMEFiT 2021 analysis}
%
We discuss next the experimental data used
in~\cite{Ethier:2021bye}.
%
%
In Table~\ref{eq:table_dataset_overview} we summarise the number of data points in the baseline dataset
for each of the data categories and processes considered in this analysis, as well
as the total per category and the overall total.
%
SMEFiT include 150, 97, and 70 cross-sections from top-quark production, Higgs boson production
and decay, and diboson production
cross-sections from LEP and the LHC respectively in the baseline dataset,
for a total of 317 cross-section measurements.
%
For all processes, we consider only parton-level measurements, since the theoretical EFT
interpretation and simulation of particle-level measurements is more challenging.

%%%%%%%%%%%%%%%%%%%%%%%%%%%%%%%%%%%%%%%%%%%%%%%%%%%%%%%%
\input{tables/table-dataset-overview.tex}
%%%%%%%%%%%%%%%%%%%%%%%%%%%%%%%%%%%%%%%%%%%%%%%%%%%%%%%%

Concerning top-quark production measurements,
we consider four
different categories: inclusive top-quark pair production, top-quark pair
production in association with vector bosons or heavy quarks, inclusive single
top-quark production, and single top-quark production in association with
vector bosons.
%
Top-quark pair production in association
with a Higgs boson is considered part of the Higgs processes.
%
The bulk of the top quark measurements corresponds to
inclusive top-quark pair production, with measurements
of single and double differential distributions in the dilepton
and lepton+jets final states from ATLAS and CMS
~\cite{Aad:2015mbv,Khachatryan:2015oqa,Sirunyan:2017azo,Aaboud:2016hsq,Khachatryan:2016fky,Khachatryan:2016mnb,Sirunyan:2018wem,Sirunyan:2017mzl,Aaboud:2016iot,Aad:2019ntk,Sirunyan:2018ucr}.
%
In general, the $m_{t\bar{t}}$ invariant mass distributions are found
to be the most constraining observables.
%
We also include top-quark pair charge asymmetry measurements:
the ATLAS and CMS combined dataset at 8~TeV~\cite{Sirunyan:2017lvd},
and the ATLAS dataset at 13~TeV~\cite{ATLAS:2019czt}.

For associated top-quark pair production together with gauge bosons
and heavy quarks, we consider
the ATLAS and CMS measurements of the total cross-sections for $t\bar{t}t\bar{t}$
and $b\bar{b}b\bar{b}$ production~\cite{Sirunyan:2017snr,Sirunyan:2017roi,Sirunyan:2019wxt,Aad:2020klt,Aaboud:2018eki,Sirunyan:2019jud},
in the ATLAS and CMS measurements of inclusive $tW$ and $tZ$ production at
8~TeV and 13~TeV~\cite{Khachatryan:2015sha,Sirunyan:2017uzs,Aad:2015eua,
  Aaboud:2016xve,Aaboud:2019njj,CMS:2019too}.
%
In addition to inclusive measurements, we also consider the measurement
of the $p_T^Z$ differential distribution in $t\bar{t}Z$ production from CMS.
%
The $t\bar{t}V$ measurements are
especially useful to constrain EFT effects that modify the electroweak
couplings of the top-quark.
%
We note that the  $t\bar{t}t\bar{t}$
and $b\bar{b}b\bar{b}$  measurements can only be meaningfully described
within a EFT analysis that accounts for the quadratic $\mathcal{O}\lp \Lambda^{-4}\rp$
corrections.

In the case of inclusive single top-quark production
both in the $t$-channel and in the $s$-channel,
we account for total cross-sections and rapidity distributions
from ATLAS and CMS~\cite{Khachatryan:2014iya,CMS-PAS-TOP-14-004,Aaboud:2017pdi,Aad:2015upn,Khachatryan:2016ewo,Aaboud:2016ymp,CMS:2016xnv,Sirunyan:2016cdg,Sirunyan:2019hqb}.
%
We also consider the associated single top-quark production with weak bosons,
with the $tW$ and $tZ$ measurements from ATLAS and CMS
at 8 and 13 TeV~\cite{Aad:2015eto,Chatrchyan:2014tua,Aaboud:2016lpj,Sirunyan:2018lcp,Sirunyan:2017nbr,Aaboud:2017ylb,Aad:2020zhd,Aad:2015eto,Aad:2020wog,Sirunyan:2018zgs}.

Concerning  Higgs boson production and decay measurements,
we consider  inclusive fiducial cross-section measurements (signal
strengths) as well as  differential
distributions and STXS measurements.
%
For the LHC Run I, we take into account the inclusive
measurements of  Higgs boson production and decay rates from the ATLAS and CMS
combination based on the full 7 and 8~TeV datasets~\cite{Khachatryan:2016vau}.
%
For the LHC Run II, we consider the ATLAS measurement of signal strengths
corresponding to an integrated luminosity of
$\mathcal{L}=80$~fb$^{-1}$~\cite{Aad:2019mbh}, and the CMS measurement
corresponding to an integrated luminosity of
$\mathcal{L}=35.9$~fb$^{-1}$~\cite{Sirunyan:2018koj}.
As in the case of the Run I signal strengths, we keep into account
correlations between the various production and final state combinations.

In the case of differential measurements, we consider the ATLAS and CMS differential distributions
in the Higgs boson kinematic variables obtained from the combination of the
$h\to \gamma\gamma$, $h\to ZZ$, and (in the CMS case)
$h \to b\bar{b}$ final states at 13~TeV based
on $\mathcal{L}=36$~fb$^{-1}$~\cite{Aaboud:2018ezd,Sirunyan:2018sgc}.
Specifically, we consider the differential distributions in the Higgs boson
transverse momentum $p_T^h$.
%
We also include the ATLAS measurement of the associated production of Higgs
bosons, $Vh$, in the $h\to b\bar{b}$ final state at
13~TeV~\cite{Aaboud:2019nan}.
%
Then we also include selected differential measurements presented in the ATLAS
Run II Higgs combination paper~\cite{Aad:2019mbh}.
Specifically, we include the measurements of Higgs production
in gluon fusion, $gg \to h$, in different bins of
$p_T^h$ and in the number of jets in the event.
%
Furthermore, we consider the differential STXS Higgs boson production measurements
presented by CMS at 13~TeV based on an integrated luminosity of
$\mathcal{L}=77.4$~fb$^{-1}$ and corresponding to the final state
$\gamma\gamma$~\cite{CMS:1900lgv}.
%
In all differential measurements, whenever available the
information on the experimental correlated systematic uncertainties is included.

Finally we consider diboson production cross-sections measured by LEP and the LHC.
%
To begin with, we consider the LEP-2
legacy measurements of $WW$ production~\cite{Schael:2013ita}.
Specifically, we include the cross-sections differential
in $\cos\theta_W$ in four different bins in the center of
mass energy, from $\sqrt{s}=182$ GeV up to $\sqrt{s}=206$ GeV.
%
Concerning the LHC datasets, we include measurements of the differential
distributions for $W^{\pm}Z$ production at 13~TeV from
ATLAS~\cite{ATLAS-CONF-2018-034} and CMS~\cite{Sirunyan:2019bez} based on a
luminosity of $\mathcal{L}=36.1$ fb$^{-1}$. In both cases, the two gauge bosons
are reconstructed by means of the fully leptonic final state.
%
Our baseline choice will be to include the $m_{T}^{WZ}$
distribution for the ATLAS measurement, which extends up to transverse masses of $m_{T}^{WZ}=600$ GeV,
while for  the corresponding CMS measurement, the normalised
differential distributions in $p_T^Z$ is chosen.
%
In addition, we also consider
the differential distributions for $WW$ production from ATLAS at 13~TeV
based on a luminosity of $\mathcal{L}=36.1$ fb$^{-1}$~\cite{Aaboud:2019nkz}.
%
We note that diboson measurements are the only ones providing direct information
on the triple gauge operator $c_{WWW}$.

To finalise this discussion of the experimental data, we mention that
the {\sc\small SMEFiT} analysis framework was used in~\cite{Ethier:2021ydt}
for the EFT interpretation of vector-boson scattering (VBS) measurements
from the LHC Run II, together with diboson data.
%
The constraints on the electroweak EFT operators provided by VBS
were found to be compatible with those derived in the global fit.


\subsubsection{Input experimental data in fitmaker analysis}

The \verb|fitmaker| code contains a database of 342 measurements. We will consider here a subset pertaining to Higgs, diboson and electroweak precision data. \verb|fitmaker| is publicly available such that the maps produced here can be extended to include Top data and different assumptions on the flavour structure.

The \verb|fitmaker| dataset for the flavour universal scenario fit consists of the electroweak pseudo-observables reported by LEP/SLD, together with the $W$ mass measurements at ATLAS and Tevatron, a range of diboson measurements from LEP and ATLAS, as well as a large number of Higgs measurements, both inclusive and differential including STXS bins. It is constructed to avoid potential statistical overlap as much as possible, while favouring experimental combinations that provide access to correlation information. For example, only one type of Higgs measurement is included per experiment, i.e., an STXS combination for ATLAS and a signal strength combination for CMS, both of which publish correlation matrices. The datasets are further discussed below when the linear dependence of operators and measurements is presented. For more details, we refer the reader to Ref.~\cite{Ellis:2020unq}. This study precedes the latest CDF $W$-mass measurement, and we refer interested readers to our recent dedicated study of the impact of this measurement in Ref~\cite{Bagnaschi:2022whn}.


\subsection{Mapping between data and EFT coefficients.}
In this section we discuss which operators enter which measurements following two different approaches. First we present the linear dependence of various measurements on the Wilson coefficients as presented in the \verb|fitmaker| analysis. We then present the Fisher information table which determines the impact of various operators on different measurements taking into account the experimental precision. 
\subsubsection{Linear dependences of measurements on operators (Fitmaker)}

In the \verb|fitmaker| analysis the linear dependence, $a_i^X$, on the Wilson coefficients, $C_i$, is computed using MadGraph5 with the SMEFTsim UFO model for a measured quantity $X$ relative to the SM value $X_{SM}$,  
%
\begin{equation}
\mu_{X} \equiv \frac{X}{X_{S M}}=1+\sum_{i} a_{i}^{X} \frac{C_{i}}{\Lambda^{2}}+\mathcal{O}\left(\frac{1}{\Lambda^{4}}\right) \, .
\end{equation}
%
In other words,   $a_i^X$ indicates the linear EFT cross-section for process $X$ and operator $\mathcal{O}_i$ once the corresponding Wilson coefficient $C_i$ has been factored out.
This is done at tree-level for all measurements except for the Higgs coupling to gluons which is calculated at one loop using SMEFT@NLO.  Wherever relevant, measurements are of unfolded observables, such that the dependences are computed at parton-level in the unfolded phase space, not including any showering or detector effects. This means that using this set of dependences neglects any acceptance differences between the SM and the SMEFT predictions,  which, in some cases, have been shown to be relevant (See, \emph{e.g.}, the impact of $\mathcal{O}_{HWB}$ on the measurement of $h\to 4\ell$ presented in Ref.~\cite{ATLAS:2020rej}).  See Ref.~\cite{Ellis:2020unq} for more details on our extraction of the linear dependences. 
After normalising the $a_i$ linear dependences of $\mu_{X}$ on the Wilson coefficients by dividing by the largest absolute value of the $a_i$ for a given measurement $X$, the resulting normalised dependence is colour coded on a log scale in Fig.~\ref{fig:EWPO}. Each row corresponds to a measurement while each column corresponds to a coefficient. The darkest red colours represent the strongest linear dependencies while the uncoloured white blocks indicate no dependence at linear order. We see that the right half of the grid is uncoloured since they correspond mostly to operators that can only be constrained in Higgs physics or diboson measurements. We can also read off that the operator coefficients most relevant for a large number of measurements is $C_{HWB}$, though the operators $C_{Hu}$ and $C_{Hd}$ have the largest dependencies in the $A_c$ and $A_b$ asymmetries respectively. For the $Z$ decay width, $\Gamma_Z$, it is the $Z$ coupling to fermions modified by the operators corresponding to $C_{Hl}^{(3)}$ and $C_{Hq}^{(3)}$ that are the most important contributions to this measurement. It is well known that, of the 10 Warsaw basis operators that impact these measurements, only 8 linear combinations are actually constrained by a fit to this data alone~\cite{Falkowski:2014tna}.

Similarly, Fig.~\ref{fig:diboson} shows the mapping of linear dependences for a variety of diboson measurements in the same normalisation and colour scheme. These are differential distribution measurements that include either several kinematic bins or cross section measurements are different collider energy; we show here only the linear dependence in the last bin for simplicity. Just as in the $Z$ decay width case, $C_{Hl}^{(3)}$ and $C_{Hq}^{(3)}$ are the largest contributions to these measurements. In this case, however, this is expected from the fact that these operators directly modify the interactions of the $W$ boson, while the others enter via $Z$-coupling modifications and input parameter shifts. Crucially, these measurements probe different linear combinations of the 10 operators constrained by the $Z/W$ pole data. Moreover, we also see that the operator coefficient $C_W$ can now be constrained, in particular by the new $Zjj$ data that was found by ATLAS to eliminate blind directions and be particularly sensitive to linear contributions from this operator in a global fit. Since the operator $\mathcal{O}_W$ involves three gauge field strengths, it can only be constrained by diboson data and not by electroweak or Higgs processes. In Fig.~\ref{fig:Zjj} we show the linear dependences for all the bins of this differential measurement, where the linear dependencies are now normalised to 1 with respect to the strongest $a_i$ in all the bins. The non-trivial shape induced by $C_W$ is evident. This is in contrast to other, differential measurements such as $W/Z$ boson $p_T$'s that suffer from a suppressed interference between the SM and $C_W$ that leads to relatively weaker sensitivity at linear level.

Finally in Figs.~\ref{fig:Higgs} and \ref{fig:STXS} are the maps for Higgs signal strengths and stage 1.1 Simplified Template Cross-section (STXS) bins, respectively. The operators that can only be constrained by Higgs physics are now populated. As expected, the gluon fusion channels depend on $C_{HG}$ the strongest while the one-loop calculation using SMEFT@NLO~\cite{Degrande:2020evl} allows this to be properly calculated. This is also crucial to account for the effects of the triple gluon field strength operator coefficient, $C_G$ as well as the top quark chromomagnetic operator $C_{tG}$, which we considered in a generalised flavour assumption employed when combining with top quark data. The EW Higgs production modes provide additional handles on the current operators probed by the $W/Z$ pole and diboson data. In general, the new kinematic granularity provided by the STXS bins breaks degeneracies among many operators present in the signal strengths. 
%We see that the colour patterns in the maps differ between the signal strengths and STXS figures, indicating the importance of differential information for probing different directions in SMEFT parameter space and lifting flat directions. 

\begin{figure}[t]
    \centering
    \includegraphics[width=1.0\textwidth]{EWPO_dependencies_log.pdf}
    \caption{Logarithm of normalised linear dependences for electroweak measurements. The entries are normalised by dividing each one by the largest operator dependence of a given measurement, $a_{\mathrm{max}}^X$, such that the colour map depicts $\log( a_i^X /a_{\mathrm{max}}^X )$.}
    \label{fig:EWPO}
\end{figure}

\begin{figure}[t]
    \centering
    \includegraphics[width=1.0\textwidth]{diboson_dependencies_log.pdf}
    \caption{Logarithm of normalised linear dependences for diboson measurements, as in Fig.~\ref{fig:EWPO}.}
    \label{fig:diboson}
\end{figure}

\begin{figure}[t]
    \centering
    \includegraphics[width=1.0\textwidth]{Zjj_dependencies_log.pdf}
    \caption{Logarithm of normalised linear dependences for each $\Delta\phi_{jj}$ bin from $-\pi$ to $\pi$ in the differential $Zjj$ measurement. The normalisation here is with respect to the strongest linear dependence across all bins of the measurement.}
    \label{fig:Zjj}
\end{figure}

\begin{figure}[t]
    \centering
    \includegraphics[width=1.0\textwidth]{Higgs_dependencies_log.pdf}
    \caption{Logarithm of normalised linear dependences for Higgs signal strength measurements, as in Fig.~\ref{fig:EWPO}. Dependences include effects in both production and decay.}
    \label{fig:Higgs}
\end{figure}

\begin{figure}[t]
    \centering
    \includegraphics[width=1.0\textwidth]{STXS_dependencies_log.pdf}
    \caption{Logarithm of normalised linear dependences for Higgs STXS measurements, as in Fig.~\ref{fig:EWPO}. Although the figure is labelled ``ATLAS'', since our analysis makes use of those specific measurements, STXS definitions, and therefore SMEFT dependences, are universal. Dependences include effects in both production and the nominal, $h\to 4\ell$ decay, except the first four. These quantify the dependences of the ratio of the other Higgs decay branching fractions to the nominal one.}
    \label{fig:STXS}
\end{figure}

\clearpage

\subsubsection{Fisher information matrix analysis (SMEFiT)}
A clean mapping between the input experimental measurements
is provided by the tools underlying information geometry~\cite{Brehmer:2017lrt}, specifically
by means of the Fisher information matrix~\cite{Ethier:2021bye}. defined as
\be
\label{eq:FisherDef}
I_{ij}\lp {\boldsymbol c} \rp = -{\rm E}\lc \frac{\partial^2 \ln f \lp {\boldsymbol \sigma}_{\rm exp}|
{\boldsymbol c} \rp}{\partial c_i \partial c_j} \rc \, , \qquad i,j=1,\ldots,n_{\rm op} \, ,
\ee
%
 Here $E$ indicates the expectation value, $f$ is the functional dependence of the cross-section with the EFT coefficients $\boldsymbol{c}$
and ${\boldsymbol \sigma}_{\rm exp}$ is the array of central experimental
cross-section measurements.
%
The  functional dependence $f \lp {\boldsymbol \sigma}_{\rm exp}|
{\boldsymbol c} \rp$ corresponds to the full likelihood function,
which is most cases is provided as a multi-Gaussian distribution.
%
It is important to emphasize however that the absolute size of the entries of the Fisher matrix does not
contain physical information: one is always allowed to redefine the overall normalisation
of an operator such that $c_i\sigma^{(\rm eft)}_{m,i} = c_i'\sigma'^{(\rm eft)}_{m,i}$, with
$c_i' = B_i c_i$ and $\sigma^{(\rm eft)}_{m,i} =\sigma'^{(\rm eft)}_{m,i}/B_i$
with $B_i$ being arbitrary constants.
%
For a given operator the relative value of $I_{ii}$ between two groups of processes is independent
of this choice of normalisation and
thus conveys meaningful information.
%
For this reason, in the following we present results for the Fisher information matrix normalised
such that 
the sum of the diagonal entries associated to a given EFT coefficient adds up to a fixed reference value
which is taken to be 100.

%%%%%%%%%%%%%%%%%%%%%%%%%%%%%%%%%%%%%%%%%%%%%%%%%%%%%%%%%%%%%%%%%%%%%
\begin{figure}[htbp]
  \begin{center}
    \includegraphics[width=0.85\linewidth]{plots_v2/Fisher_heat_NS_GLOBAL_NLO_HO.pdf}
    \caption{\small The values of the diagonal entries of the Fisher
      information matrix evaluated for the dataset of the global linear
      (left) and quadratic (right panel) SMEFiT analysis
      %
      The normalisation here is such that the sum of the entries associated to each EFT
      coefficient adds up to 100.
      For entries in the heat map larger than 10, we also indicate the corresponding
      numerical values.}
     \label{fig:FisherMatrix} 
  \end{center}
\end{figure}
%%%%%%%%%%%%%%%%%%%%%%%%%%%%%%%%%%%%%%%%%%%%%%%%%%%%%%%%%%%%%%%%%%%%%%

Fig.~\ref{fig:FisherMatrix} displays the values of the diagonal entries of the Fisher
information matrix both at the linear and at the quadratic level.
%
We note that at linear order in the EFT  expansion the dependence on the coefficients cancels out and the Fisher information
matrix is strictly fit-independent. At the quadratic level one
has some dependence on the Wilson coeffictions and
the Fisher information needs to be computed in an iterative manner.
%
One can identify, for each EFT coefficient, which datasets
provide the dominant constraints.
%
For instance, one observes that the two-light-two-heavy operators are overwhelmingly constrained
by inclusive top quark pair production data, except for $c_{Qq}^{3,1}$ for
which single top is the most important set of processes.
%
At the linear level, the information on the two-light-two-heavy coefficients provided
by the differential distributions and by the charge asymmetry $A_C$ data is comparable,
while the latter is less important in the quadratic fits.
%
In the case of the two-fermion operators, the leading constraints typically arise from Higgs data, in particular
from the Run II signal strengths measurements, and then to a lesser extent from the Run I data
and the Run II differential distributions.
%
Two  exceptions are $c_{\varphi t}$, which at the linear level (but not at the quadratic one)
is dominated by $t\bar{t}V$, and the chromo-magnetic operator $c_{tG}$, for which inclusive
$t\bar{t}$ production is most important.
%
Also for the purely bosonic operators the Higgs data provides most of the information,
except for $c_{WWW}$, as expected since this operator is only accessible in
diboson processes.
%
One observes that the $\mathcal{O}\lp \Lambda^{-4}\rp$ corrections induce in most
cases a moderate change in the Fisher information
matrix, but in others they can significantly alter
the balance between processes.
  
By fine-graining the calculation of the Fisher information matrix,
one can gain insight about not only which group of processeses dominates
the constraints on a given EFT coefficient, but also within a given group
of processes which is the specific dataset that dominates.
%
To illustrate this, \cref{tab:FisherMatrix_AC,tab:FisherMatrix_tt8,tab:FisherMatrix_tt13} display
a similar comparison as that shown for Fig.~\ref{fig:FisherMatrix} now in a fine-grained version
and restricted to the inclusive top quark pair production datasets.
%
Among various interesting observations,
we see that in the quadratic EFT fit the constraints on the 2-light-2-heavy
operators are dominated by the CMS measurements at 13 TeV in the dilepton and lepton+jets
channels based on the 2016 dataset.
%
This finding is not completely surprising, since these two datasets are the only
ones based on 36 fb$^{-1}$ of luminosity, yet it is reassuring that we can identify
this dominance {\it a priori}, without needing to redo the actual fit.

%%%%%%%%%%%%%%%%%%%%%%%%%%%%%%%%%%%%%%%%%%%%%%%%%%%%%%%%%%%%%%%%%%%%%
%\begin{figure}[htbp]
%  \begin{center}
%    \includegraphics[width=0.99\linewidth]{plots_v2/Fisher_tt_extended.pdf}
%    \caption{\small Same as Fig.~\ref{fig:FisherMatrix} now in a fine-grained version
%      and restricted to the inclusive top quark pair production datasets.
%      The number outside (inside) brackets corresponds to the value of the Fisher
%      information matrix in the linear (quadratic) fit.
%      %
%      See text for more details.
%    }
%     \label{fig:FisherMatrix_extended} 
%  \end{center}
%\end{figure}

%%%%%%%%%%%%%%%%%%%%%%%%%%%%%%%%%%%%%%%%%%%%%%%%%%%%%%%%%%%%%%%%%%%%%
\begin{table}
\scriptsize
\centering
\begin{tabular}{|c|c|c|c|}
\hline
 \multicolumn{2}{|c|}{} & \multicolumn{2}{c|}{Processes} \\ \hline
 Class & Coefficient & \href{https://arxiv.org/abs/1709.05327}{${\rm ATLAS\_CMS\_tt\_AC\_8TeV}$} & \href{https://cds.cern.ch/record/2682109}{${\rm ATLAS\_tt\_AC\_13TeV}$} \\ \hline
\multirow{10}{*}{2FB}
 & $c_{\varphi Q}^{(-)}$ & { \color{black} 0.000}({\color{black} 0.000}) & { \color{black} 0.000}({\color{black} 0.000})\\ \cline{2-4} 
 & $c_{\varphi Q}^{3}$ & { \color{black} 0.000}({\color{black} 0.000}) & { \color{black} 0.000}({\color{black} 0.000})\\ \cline{2-4} 
 & $c_{\varphi t}$ & { \color{black} 0.000}({\color{black} 0.000}) & { \color{black} 0.000}({\color{black} 0.000})\\ \cline{2-4} 
 & $c_{tW}$ & { \color{black} 0.000}({\color{black} 0.000}) & { \color{black} 0.000}({\color{black} 0.000})\\ \cline{2-4} 
 & $c_{tG}$ & { \color{black} 0.000}({\color{black} -0.000}) & { \color{black} 0.000}({\color{black} 0.002})\\ \cline{2-4} 
 & $c_{t \varphi}$ & { \color{black} 0.000}({\color{black} 0.000}) & { \color{black} 0.000}({\color{black} 0.000})\\ \cline{2-4} 
 & $c_{tZ}$ & { \color{black} 0.000}({\color{black} 0.000}) & { \color{black} 0.000}({\color{black} 0.000})\\ \cline{2-4} 
 & $c_{b \varphi}$ & { \color{black} 0.000}({\color{black} 0.000}) & { \color{black} 0.000}({\color{black} 0.000})\\ \cline{2-4} 
 & $c_{c \varphi}$ & { \color{black} 0.000}({\color{black} 0.000}) & { \color{black} 0.000}({\color{black} 0.000})\\ \cline{2-4} 
 & $c_{\tau \varphi}$ & { \color{black} 0.000}({\color{black} 0.000}) & { \color{black} 0.000}({\color{black} 0.000})\\ \hline
\multirow{14}{*}{2L2H}
 & $c_{qq}^{1,8}$ & { \color{black} 3.656}({\color{black} 1.068}) & { \color{blue} 29.418}({\color{blue} 10.975})\\ \cline{2-4} 
 & $c_{qq}^{1,1}$ & { \color{black} 6.186}({\color{black} -1.948}) & { \color{blue} 67.425}({\color{black} -0.434})\\ \cline{2-4} 
 & $c_{qq}^{8,3}$ & { \color{black} 5.417}({\color{black} -0.492}) & { \color{blue} 43.636}({\color{black} 2.502})\\ \cline{2-4} 
 & $c_{qq}^{1,3}$ & { \color{black} 0.005}({\color{black} -0.386}) & { \color{black} 0.063}({\color{black} -0.447})\\ \cline{2-4} 
 & $c_{qt}^{8}$ & { \color{black} 3.607}({\color{black} 1.915}) & { \color{blue} 23.810}({\color{black} 9.620})\\ \cline{2-4} 
 & $c_{qt}^{1}$ & { \color{black} 6.626}({\color{black} 2.269}) & { \color{blue} 71.322}({\color{blue} 10.718})\\ \cline{2-4} 
 & $c_{ut}^{8}$ & { \color{black} 5.775}({\color{black} 1.058}) & { \color{blue} 48.661}({\color{blue} 13.818})\\ \cline{2-4} 
 & $c_{ut}^{1}$ & { \color{black} 6.527}({\color{black} -2.794}) & { \color{blue} 73.039}({\color{black} -2.238})\\ \cline{2-4} 
 & $c_{qu}^{8}$ & { \color{black} 7.163}({\color{black} 3.653}) & { \color{blue} 51.370}({\color{blue} 18.553})\\ \cline{2-4} 
 & $c_{qu}^{1}$ & { \color{black} 4.991}({\color{black} 1.881}) & { \color{blue} 56.302}({\color{black} 8.139})\\ \cline{2-4} 
 & $c_{dt}^{8}$ & { \color{black} 4.623}({\color{black} -0.268}) & { \color{blue} 38.021}({\color{black} 4.517})\\ \cline{2-4} 
 & $c_{dt}^{1}$ & { \color{black} 5.293}({\color{black} -1.517}) & { \color{blue} 52.588}({\color{black} -2.034})\\ \cline{2-4} 
 & $c_{qd}^{8}$ & { \color{black} 4.445}({\color{black} 2.603}) & { \color{blue} 31.046}({\color{blue} 11.473})\\ \cline{2-4} 
 & $c_{qd}^{1}$ & { \color{black} 5.504}({\color{black} 2.325}) & { \color{blue} 55.378}({\color{black} 9.071})\\ \hline
\multirow{5}{*}{4H}
 & $c_{QQ}^{1}$ & { \color{black} 0.000}({\color{black} 0.000}) & { \color{black} 0.000}({\color{black} 0.000})\\ \cline{2-4} 
 & $c_{QQ}^{8}$ & { \color{black} 0.000}({\color{black} 0.000}) & { \color{black} 0.000}({\color{black} 0.000})\\ \cline{2-4} 
 & $c_{Qt}^{1}$ & { \color{black} 0.000}({\color{black} 0.000}) & { \color{black} 0.000}({\color{black} 0.000})\\ \cline{2-4} 
 & $c_{Qt}^{8}$ & { \color{black} 0.000}({\color{black} 0.000}) & { \color{black} 0.000}({\color{black} 0.000})\\ \cline{2-4} 
 & $c_{tt}^{1}$ & { \color{black} 0.000}({\color{black} 0.000}) & { \color{black} 0.000}({\color{black} 0.000})\\ \hline
\multirow{7}{*}{B}
 & $c_{\varphi G}$ & { \color{black} 0.000}({\color{black} 0.000}) & { \color{black} 0.000}({\color{black} 0.000})\\ \cline{2-4} 
 & $c_{\varphi B}$ & { \color{black} 0.000}({\color{black} 0.000}) & { \color{black} 0.000}({\color{black} 0.000})\\ \cline{2-4} 
 & $c_{\varphi W}$ & { \color{black} 0.000}({\color{black} 0.000}) & { \color{black} 0.000}({\color{black} 0.000})\\ \cline{2-4} 
 & $c_{\varphi WB}$ & { \color{black} 0.000}({\color{black} 0.000}) & { \color{black} 0.000}({\color{black} 0.000})\\ \cline{2-4} 
 & $c_{\varphi \Box}$ & { \color{black} 0.000}({\color{black} 0.000}) & { \color{black} 0.000}({\color{black} 0.000})\\ \cline{2-4} 
 & $c_{\varphi D}$ & { \color{black} 0.000}({\color{black} 0.000}) & { \color{black} 0.000}({\color{black} 0.000})\\ \cline{2-4} 
 & $c_{WWW}$ & { \color{black} 0.000}({\color{black} 0.000}) & { \color{black} 0.000}({\color{black} 0.000})\\ \hline
\end{tabular}
\caption{\small Same as Fig.~\ref{fig:FisherMatrix} now in a fine-grained version
      and restricted to the datasets measuring the charge asymmetry in top quark pair production.
      The number outside (inside) brackets corresponds to the value of the Fisher
      information matrix in the linear (quadratic) fit.
      %
      See text for more details.
    }
\label{tab:FisherMatrix_AC}
\end{table}
\begin{table}
\scriptsize
\centering
\begin{tabular}{|c|c|c|c|c|c|}
\hline
 \multicolumn{2}{|c|}{} & \multicolumn{4}{c|}{Processes} \\ \hline
 Class & Coefficient & \rot{\href{https://arxiv.org/abs/1511.04716}{${\rm ATLAS\_tt\_8TeV\_ljets\_Mtt}$}} & \rot{\href{https://arxiv.org/abs/1607.07281}{${\rm ATLAS\_tt\_8TeV\_dilep\_Mtt}$}} & \rot{\href{https://arxiv.org/abs/1505.04480}{${\rm CMS\_tt\_8TeV\_ljets\_Ytt}$}} & \rot{\href{https://arxiv.org/abs/1703.01630}{${\rm CMS\_tt2D\_8TeV\_dilep\_MttYtt}$}} \\ \hline
\multirow{10}{*}{2FB}
 & $c_{\varphi Q}^{(-)}$ & { \color{black} 0.000}({\color{black} 0.000}) & { \color{black} 0.000}({\color{black} 0.000}) & { \color{black} 0.000}({\color{black} 0.000}) & { \color{black} 0.000}({\color{black} 0.000})\\ \cline{2-6} 
 & $c_{\varphi Q}^{3}$ & { \color{black} 0.000}({\color{black} 0.000}) & { \color{black} 0.000}({\color{black} 0.000}) & { \color{black} 0.000}({\color{black} 0.000}) & { \color{black} 0.000}({\color{black} 0.000})\\ \cline{2-6} 
 & $c_{\varphi t}$ & { \color{black} 0.000}({\color{black} 0.000}) & { \color{black} 0.000}({\color{black} 0.000}) & { \color{black} 0.000}({\color{black} 0.000}) & { \color{black} 0.000}({\color{black} 0.000})\\ \cline{2-6} 
 & $c_{tW}$ & { \color{black} 0.000}({\color{black} 0.000}) & { \color{black} 0.000}({\color{black} 0.000}) & { \color{black} 0.000}({\color{black} 0.000}) & { \color{black} 0.000}({\color{black} 0.000})\\ \cline{2-6} 
 & $c_{tG}$ & { \color{black} 1.002}({\color{black} 0.976}) & { \color{black} 3.025}({\color{black} 3.120}) & { \color{blue} 17.981}({\color{blue} 18.770}) & { \color{blue} 10.349}({\color{blue} 10.159})\\ \cline{2-6} 
 & $c_{t \varphi}$ & { \color{black} 0.000}({\color{black} 0.000}) & { \color{black} 0.000}({\color{black} 0.000}) & { \color{black} 0.000}({\color{black} 0.000}) & { \color{black} 0.000}({\color{black} 0.000})\\ \cline{2-6} 
 & $c_{tZ}$ & { \color{black} 0.000}({\color{black} 0.000}) & { \color{black} 0.000}({\color{black} 0.000}) & { \color{black} 0.000}({\color{black} 0.000}) & { \color{black} 0.000}({\color{black} 0.000})\\ \cline{2-6} 
 & $c_{b \varphi}$ & { \color{black} 0.000}({\color{black} 0.000}) & { \color{black} 0.000}({\color{black} 0.000}) & { \color{black} 0.000}({\color{black} 0.000}) & { \color{black} 0.000}({\color{black} 0.000})\\ \cline{2-6} 
 & $c_{c \varphi}$ & { \color{black} 0.000}({\color{black} 0.000}) & { \color{black} 0.000}({\color{black} 0.000}) & { \color{black} 0.000}({\color{black} 0.000}) & { \color{black} 0.000}({\color{black} 0.000})\\ \cline{2-6} 
 & $c_{\tau \varphi}$ & { \color{black} 0.000}({\color{black} 0.000}) & { \color{black} 0.000}({\color{black} 0.000}) & { \color{black} 0.000}({\color{black} 0.000}) & { \color{black} 0.000}({\color{black} 0.000})\\ \hline
\multirow{14}{*}{2L2H}
 & $c_{qq}^{1,8}$ & { \color{black} 0.989}({\color{black} 0.116}) & { \color{black} 0.910}({\color{black} 0.499}) & { \color{black} 3.939}({\color{black} 2.715}) & { \color{black} 2.907}({\color{black} 0.308})\\ \cline{2-6} 
 & $c_{qq}^{1,1}$ & { \color{black} 0.746}({\color{black} -1.338}) & { \color{black} 0.809}({\color{black} 0.769}) & { \color{black} 0.594}({\color{black} 5.602}) & { \color{black} 2.204}({\color{black} -4.524})\\ \cline{2-6} 
 & $c_{qq}^{8,3}$ & { \color{black} 0.568}({\color{black} -0.514}) & { \color{black} 0.552}({\color{black} 0.335}) & { \color{black} 2.638}({\color{black} 2.356}) & { \color{black} 1.823}({\color{black} -1.572})\\ \cline{2-6} 
 & $c_{qq}^{1,3}$ & { \color{black} 0.000}({\color{black} -0.271}) & { \color{black} 0.000}({\color{black} 0.440}) & { \color{black} 0.002}({\color{black} 1.051}) & { \color{black} 0.005}({\color{black} -0.862})\\ \cline{2-6} 
 & $c_{qt}^{8}$ & { \color{black} 0.968}({\color{black} -0.029}) & { \color{black} 1.177}({\color{black} 0.662}) & { \color{black} 6.098}({\color{black} 4.034}) & { \color{black} 3.923}({\color{black} 0.290})\\ \cline{2-6} 
 & $c_{qt}^{1}$ & { \color{black} 1.008}({\color{black} -1.219}) & { \color{black} 0.840}({\color{black} 0.749}) & { \color{black} 0.584}({\color{black} 5.045}) & { \color{black} 0.722}({\color{black} -4.150})\\ \cline{2-6} 
 & $c_{ut}^{8}$ & { \color{black} 1.324}({\color{black} -0.321}) & { \color{black} 1.260}({\color{black} 0.987}) & { \color{black} 4.812}({\color{black} 5.256}) & { \color{black} 3.462}({\color{black} -0.637})\\ \cline{2-6} 
 & $c_{ut}^{1}$ & { \color{black} 0.466}({\color{black} -1.945}) & { \color{black} 0.594}({\color{black} 1.070}) & { \color{black} 0.815}({\color{black} 8.012}) & { \color{black} 1.001}({\color{black} -5.962})\\ \cline{2-6} 
 & $c_{qu}^{8}$ & { \color{black} 1.741}({\color{black} -0.569}) & { \color{black} 1.890}({\color{black} 1.267}) & { \color{blue} 10.610}({\color{black} 7.625}) & { \color{black} 6.911}({\color{black} -0.784})\\ \cline{2-6} 
 & $c_{qu}^{1}$ & { \color{black} 0.516}({\color{black} -1.112}) & { \color{black} 0.495}({\color{black} 0.704}) & { \color{black} 0.870}({\color{black} 4.351}) & { \color{black} 1.406}({\color{black} -3.318})\\ \cline{2-6} 
 & $c_{dt}^{8}$ & { \color{black} 1.722}({\color{black} -0.370}) & { \color{black} 1.774}({\color{black} 0.678}) & { \color{black} 7.789}({\color{black} 3.768}) & { \color{black} 5.872}({\color{black} -1.217})\\ \cline{2-6} 
 & $c_{dt}^{1}$ & { \color{black} 0.803}({\color{black} -1.097}) & { \color{black} 0.713}({\color{black} 0.770}) & { \color{black} 2.627}({\color{black} 4.482}) & { \color{black} 2.696}({\color{black} -4.220})\\ \cline{2-6} 
 & $c_{qd}^{8}$ & { \color{black} 2.681}({\color{black} -0.397}) & { \color{black} 3.058}({\color{black} 2.103}) & { \color{blue} 13.187}({\color{black} 7.253}) & { \color{black} 8.826}({\color{black} -1.971})\\ \cline{2-6} 
 & $c_{qd}^{1}$ & { \color{black} 0.610}({\color{black} -1.607}) & { \color{black} 0.692}({\color{black} 0.845}) & { \color{black} 1.493}({\color{black} 5.918}) & { \color{black} 2.854}({\color{black} -5.613})\\ \hline
\multirow{5}{*}{4H}
 & $c_{QQ}^{1}$ & { \color{black} 0.000}({\color{black} 0.000}) & { \color{black} 0.000}({\color{black} 0.000}) & { \color{black} 0.000}({\color{black} 0.000}) & { \color{black} 0.000}({\color{black} 0.000})\\ \cline{2-6} 
 & $c_{QQ}^{8}$ & { \color{black} 0.000}({\color{black} 0.000}) & { \color{black} 0.000}({\color{black} 0.000}) & { \color{black} 0.000}({\color{black} 0.000}) & { \color{black} 0.000}({\color{black} 0.000})\\ \cline{2-6} 
 & $c_{Qt}^{1}$ & { \color{black} 0.000}({\color{black} 0.000}) & { \color{black} 0.000}({\color{black} 0.000}) & { \color{black} 0.000}({\color{black} 0.000}) & { \color{black} 0.000}({\color{black} 0.000})\\ \cline{2-6} 
 & $c_{Qt}^{8}$ & { \color{black} 0.000}({\color{black} 0.000}) & { \color{black} 0.000}({\color{black} 0.000}) & { \color{black} 0.000}({\color{black} 0.000}) & { \color{black} 0.000}({\color{black} 0.000})\\ \cline{2-6} 
 & $c_{tt}^{1}$ & { \color{black} 0.000}({\color{black} 0.000}) & { \color{black} 0.000}({\color{black} 0.000}) & { \color{black} 0.000}({\color{black} 0.000}) & { \color{black} 0.000}({\color{black} 0.000})\\ \hline
\multirow{7}{*}{B}
 & $c_{\varphi G}$ & { \color{black} 0.000}({\color{black} 0.000}) & { \color{black} 0.000}({\color{black} 0.000}) & { \color{black} 0.000}({\color{black} 0.000}) & { \color{black} 0.000}({\color{black} 0.000})\\ \cline{2-6} 
 & $c_{\varphi B}$ & { \color{black} 0.000}({\color{black} 0.000}) & { \color{black} 0.000}({\color{black} 0.000}) & { \color{black} 0.000}({\color{black} 0.000}) & { \color{black} 0.000}({\color{black} 0.000})\\ \cline{2-6} 
 & $c_{\varphi W}$ & { \color{black} 0.000}({\color{black} 0.000}) & { \color{black} 0.000}({\color{black} 0.000}) & { \color{black} 0.000}({\color{black} 0.000}) & { \color{black} 0.000}({\color{black} 0.000})\\ \cline{2-6} 
 & $c_{\varphi WB}$ & { \color{black} 0.000}({\color{black} 0.000}) & { \color{black} 0.000}({\color{black} 0.000}) & { \color{black} 0.000}({\color{black} 0.000}) & { \color{black} 0.000}({\color{black} 0.000})\\ \cline{2-6} 
 & $c_{\varphi \Box}$ & { \color{black} 0.000}({\color{black} 0.000}) & { \color{black} 0.000}({\color{black} 0.000}) & { \color{black} 0.000}({\color{black} 0.000}) & { \color{black} 0.000}({\color{black} 0.000})\\ \cline{2-6} 
 & $c_{\varphi D}$ & { \color{black} 0.000}({\color{black} 0.000}) & { \color{black} 0.000}({\color{black} 0.000}) & { \color{black} 0.000}({\color{black} 0.000}) & { \color{black} 0.000}({\color{black} 0.000})\\ \cline{2-6} 
 & $c_{WWW}$ & { \color{black} 0.000}({\color{black} 0.000}) & { \color{black} 0.000}({\color{black} 0.000}) & { \color{black} 0.000}({\color{black} 0.000}) & { \color{black} 0.000}({\color{black} 0.000})\\ \hline
\end{tabular}
\caption{\small Same as Tab.~\ref{tab:FisherMatrix_AC} now for differential parton-level distributions in inclusive top quark pair production datasets at $\sqrt{s}=8\ TeV$.
}
\label{tab:FisherMatrix_tt8}
\end{table}
\begin{table}
\scriptsize
\centering
\begin{tabular}{|c|c|c|c|c|c|c|}
\hline
 \multicolumn{2}{|c|}{} & \multicolumn{5}{c|}{Processes} \\ \hline
 Class & Coefficient & \rot{\href{https://arxiv.org/abs/1610.04191}{${\rm CMS\_tt\_13TeV\_ljets\_2015\_Mtt}$}}& \rot{\href{https://arxiv.org/abs/1708.07638}{${\rm CMS\_tt\_13TeV\_dilep\_2015\_Mtt}$}} & \rot{\href{https://arxiv.org/abs/1803.08856}{${\rm CMS\_tt\_13TeV\_ljets\_2016\_Mtt}$}} & \rot{\href{https://arxiv.org/abs/1811.06625}{${\rm CMS\_tt\_13TeV\_dilep\_2016\_Mtt}$}} & \rot{\href{https://arxiv.org/abs/1908.07305}{${\rm ATLAS\_tt\_13TeV\_ljets\_2016\_Mtt}$}} \\ \hline
\multirow{10}{*}{2FB}
 & $c_{\varphi Q}^{(-)}$ & { \color{black} 0.000}({\color{black} 0.000}) & { \color{black} 0.000}({\color{black} 0.000}) & { \color{black} 0.000}({\color{black} 0.000}) & { \color{black} 0.000}({\color{black} 0.000}) & { \color{black} 0.000}({\color{black} 0.000})\\ \cline{2-7} 
 & $c_{\varphi Q}^{3}$ & { \color{black} 0.000}({\color{black} 0.000}) & { \color{black} 0.000}({\color{black} 0.000}) & { \color{black} 0.000}({\color{black} 0.000}) & { \color{black} 0.000}({\color{black} 0.000}) & { \color{black} 0.000}({\color{black} 0.000})\\ \cline{2-7} 
 & $c_{\varphi t}$ & { \color{black} 0.000}({\color{black} 0.000}) & { \color{black} 0.000}({\color{black} 0.000}) & { \color{black} 0.000}({\color{black} 0.000}) & { \color{black} 0.000}({\color{black} 0.000}) & { \color{black} 0.000}({\color{black} 0.000})\\ \cline{2-7} 
 & $c_{tW}$ & { \color{black} 0.000}({\color{black} 0.000}) & { \color{black} 0.000}({\color{black} 0.000}) & { \color{black} 0.000}({\color{black} 0.000}) & { \color{black} 0.000}({\color{black} 0.000}) & { \color{black} 0.000}({\color{black} 0.000})\\ \cline{2-7} 
 & $c_{tG}$ & { \color{black} 0.939}({\color{black} 1.078}) & { \color{black} 1.423}({\color{black} 1.579}) & { \color{black} 3.431}({\color{black} 4.254}) & { \color{black} 2.072}({\color{black} 2.486}) & { \color{black} 1.207}({\color{black} 1.349})\\ \cline{2-7} 
 & $c_{t \varphi}$ & { \color{black} 0.000}({\color{black} 0.000}) & { \color{black} 0.000}({\color{black} 0.000}) & { \color{black} 0.000}({\color{black} 0.000}) & { \color{black} 0.000}({\color{black} 0.000}) & { \color{black} 0.000}({\color{black} 0.000})\\ \cline{2-7} 
 & $c_{tZ}$ & { \color{black} 0.000}({\color{black} 0.000}) & { \color{black} 0.000}({\color{black} 0.000}) & { \color{black} 0.000}({\color{black} 0.000}) & { \color{black} 0.000}({\color{black} 0.000}) & { \color{black} 0.000}({\color{black} 0.000})\\ \cline{2-7} 
 & $c_{b \varphi}$ & { \color{black} 0.000}({\color{black} 0.000}) & { \color{black} 0.000}({\color{black} 0.000}) & { \color{black} 0.000}({\color{black} 0.000}) & { \color{black} 0.000}({\color{black} 0.000}) & { \color{black} 0.000}({\color{black} 0.000})\\ \cline{2-7} 
 & $c_{c \varphi}$ & { \color{black} 0.000}({\color{black} 0.000}) & { \color{black} 0.000}({\color{black} 0.000}) & { \color{black} 0.000}({\color{black} 0.000}) & { \color{black} 0.000}({\color{black} 0.000}) & { \color{black} 0.000}({\color{black} 0.000})\\ \cline{2-7} 
 & $c_{\tau \varphi}$ & { \color{black} 0.000}({\color{black} 0.000}) & { \color{black} 0.000}({\color{black} 0.000}) & { \color{black} 0.000}({\color{black} 0.000}) & { \color{black} 0.000}({\color{black} 0.000}) & { \color{black} 0.000}({\color{black} 0.000})\\ \hline
\multirow{14}{*}{2L2H}
 & $c_{qq}^{1,8}$ & { \color{black} 1.003}({\color{black} 1.973}) & { \color{black} 0.333}({\color{black} 0.560}) & { \color{black} 8.035}({\color{blue} 16.655}) & { \color{blue} 13.744}({\color{blue} 39.714}) & { \color{black} 7.522}({\color{blue} 14.983})\\ \cline{2-7} 
 & $c_{qq}^{1,1}$ & { \color{black} 1.441}({\color{black} 6.796}) & { \color{black} 0.227}({\color{black} 2.201}) & { \color{black} 9.698}({\color{blue} 52.347}) & { \color{black} 6.597}({\color{blue} 34.891}) & { \color{black} 3.678}({\color{black} 3.338})\\ \cline{2-7} 
 & $c_{qq}^{8,3}$ & { \color{black} 0.430}({\color{black} 2.986}) & { \color{black} 0.130}({\color{black} 0.874}) & { \color{black} 2.441}({\color{blue} 37.461}) & { \color{black} 0.937}({\color{blue} 18.134}) & { \color{black} 0.564}({\color{black} 3.590})\\ \cline{2-7} 
 & $c_{qq}^{1,3}$ & { \color{black} 0.001}({\color{black} 1.243}) & { \color{black} 0.000}({\color{black} 0.420}) & { \color{black} 0.007}({\color{black} 7.540}) & { \color{black} 0.003}({\color{black} 4.673}) & { \color{black} 0.002}({\color{black} -0.152})\\ \cline{2-7} 
 & $c_{qt}^{8}$ & { \color{black} 1.095}({\color{black} 3.119}) & { \color{black} 0.401}({\color{black} 0.779}) & { \color{blue} 12.938}({\color{blue} 43.148}) & { \color{black} 3.043}({\color{blue} 16.920}) & { \color{black} 1.733}({\color{black} 4.893})\\ \cline{2-7} 
 & $c_{qt}^{1}$ & { \color{black} 1.148}({\color{black} 5.566}) & { \color{black} 0.215}({\color{black} 1.966}) & { \color{blue} 10.525}({\color{blue} 36.109}) & { \color{black} 4.370}({\color{blue} 36.495}) & { \color{black} 2.456}({\color{black} 5.531})\\ \cline{2-7} 
 & $c_{ut}^{8}$ & { \color{black} 1.407}({\color{black} 5.182}) & { \color{black} 0.377}({\color{black} 1.296}) & { \color{blue} 17.865}({\color{blue} 40.618}) & { \color{black} 7.932}({\color{blue} 27.325}) & { \color{black} 4.398}({\color{black} 3.189})\\ \cline{2-7} 
 & $c_{ut}^{1}$ & { \color{black} 0.878}({\color{black} 8.309}) & { \color{black} 0.154}({\color{black} 3.035}) & { \color{black} 6.966}({\color{blue} 44.848}) & { \color{black} 5.931}({\color{blue} 40.126}) & { \color{black} 3.270}({\color{black} 3.230})\\ \cline{2-7} 
 & $c_{qu}^{8}$ & { \color{black} 1.060}({\color{black} 4.667}) & { \color{black} 0.670}({\color{black} 1.851}) & { \color{black} 4.893}({\color{blue} 43.891}) & { \color{black} 2.067}({\color{blue} 12.474}) & { \color{black} 1.293}({\color{black} 2.146})\\ \cline{2-7} 
 & $c_{qu}^{1}$ & { \color{black} 1.523}({\color{black} 3.869}) & { \color{black} 0.143}({\color{black} 1.609}) & { \color{blue} 20.696}({\color{blue} 48.683}) & { \color{black} 8.364}({\color{blue} 26.455}) & { \color{black} 4.580}({\color{black} 6.013})\\ \cline{2-7} 
 & $c_{dt}^{8}$ & { \color{black} 2.273}({\color{black} 4.198}) & { \color{black} 0.779}({\color{black} 1.207}) & { \color{blue} 17.028}({\color{blue} 44.740}) & { \color{black} 9.665}({\color{blue} 31.876}) & { \color{black} 5.477}({\color{black} 9.261})\\ \cline{2-7} 
 & $c_{dt}^{1}$ & { \color{black} 1.816}({\color{black} 6.135}) & { \color{black} 0.339}({\color{black} 1.948}) & { \color{blue} 16.044}({\color{blue} 53.957}) & { \color{blue} 10.761}({\color{blue} 35.154}) & { \color{black} 5.924}({\color{black} 4.508})\\ \cline{2-7} 
 & $c_{qd}^{8}$ & { \color{black} 1.601}({\color{black} 3.070}) & { \color{black} 1.207}({\color{black} 2.155}) & { \color{black} 6.384}({\color{blue} 32.430}) & { \color{black} 7.078}({\color{blue} 29.956}) & { \color{black} 4.157}({\color{black} 5.835})\\ \cline{2-7} 
 & $c_{qd}^{1}$ & { \color{black} 2.438}({\color{black} 6.856}) & { \color{black} 0.410}({\color{black} 2.536}) & { \color{blue} 18.526}({\color{blue} 37.766}) & { \color{black} 7.595}({\color{blue} 36.886}) & { \color{black} 4.297}({\color{black} 2.158})\\ \hline
\multirow{5}{*}{4H}
 & $c_{QQ}^{1}$ & { \color{black} 0.000}({\color{black} 0.000}) & { \color{black} 0.000}({\color{black} 0.000}) & { \color{black} 0.000}({\color{black} 0.000}) & { \color{black} 0.000}({\color{black} 0.000}) & { \color{black} 0.000}({\color{black} 0.000})\\ \cline{2-7} 
 & $c_{QQ}^{8}$ & { \color{black} 0.000}({\color{black} 0.000}) & { \color{black} 0.000}({\color{black} 0.000}) & { \color{black} 0.000}({\color{black} 0.000}) & { \color{black} 0.000}({\color{black} 0.000}) & { \color{black} 0.000}({\color{black} 0.000})\\ \cline{2-7} 
 & $c_{Qt}^{1}$ & { \color{black} 0.000}({\color{black} 0.000}) & { \color{black} 0.000}({\color{black} 0.000}) & { \color{black} 0.000}({\color{black} 0.000}) & { \color{black} 0.000}({\color{black} 0.000}) & { \color{black} 0.000}({\color{black} 0.000})\\ \cline{2-7} 
 & $c_{Qt}^{8}$ & { \color{black} 0.000}({\color{black} 0.000}) & { \color{black} 0.000}({\color{black} 0.000}) & { \color{black} 0.000}({\color{black} 0.000}) & { \color{black} 0.000}({\color{black} 0.000}) & { \color{black} 0.000}({\color{black} 0.000})\\ \cline{2-7} 
 & $c_{tt}^{1}$ & { \color{black} 0.000}({\color{black} 0.000}) & { \color{black} 0.000}({\color{black} 0.000}) & { \color{black} 0.000}({\color{black} 0.000}) & { \color{black} 0.000}({\color{black} 0.000}) & { \color{black} 0.000}({\color{black} 0.000})\\ \hline
\multirow{7}{*}{B}
 & $c_{\varphi G}$ & { \color{black} 0.000}({\color{black} 0.000}) & { \color{black} 0.000}({\color{black} 0.000}) & { \color{black} 0.000}({\color{black} 0.000}) & { \color{black} 0.000}({\color{black} 0.000}) & { \color{black} 0.000}({\color{black} 0.000})\\ \cline{2-7} 
 & $c_{\varphi B}$ & { \color{black} 0.000}({\color{black} 0.000}) & { \color{black} 0.000}({\color{black} 0.000}) & { \color{black} 0.000}({\color{black} 0.000}) & { \color{black} 0.000}({\color{black} 0.000}) & { \color{black} 0.000}({\color{black} 0.000})\\ \cline{2-7} 
 & $c_{\varphi W}$ & { \color{black} 0.000}({\color{black} 0.000}) & { \color{black} 0.000}({\color{black} 0.000}) & { \color{black} 0.000}({\color{black} 0.000}) & { \color{black} 0.000}({\color{black} 0.000}) & { \color{black} 0.000}({\color{black} 0.000})\\ \cline{2-7} 
 & $c_{\varphi WB}$ & { \color{black} 0.000}({\color{black} 0.000}) & { \color{black} 0.000}({\color{black} 0.000}) & { \color{black} 0.000}({\color{black} 0.000}) & { \color{black} 0.000}({\color{black} 0.000}) & { \color{black} 0.000}({\color{black} 0.000})\\ \cline{2-7} 
 & $c_{\varphi \Box}$ & { \color{black} 0.000}({\color{black} 0.000}) & { \color{black} 0.000}({\color{black} 0.000}) & { \color{black} 0.000}({\color{black} 0.000}) & { \color{black} 0.000}({\color{black} 0.000}) & { \color{black} 0.000}({\color{black} 0.000})\\ \cline{2-7} 
 & $c_{\varphi D}$ & { \color{black} 0.000}({\color{black} 0.000}) & { \color{black} 0.000}({\color{black} 0.000}) & { \color{black} 0.000}({\color{black} 0.000}) & { \color{black} 0.000}({\color{black} 0.000}) & { \color{black} 0.000}({\color{black} 0.000})\\ \cline{2-7} 
 & $c_{WWW}$ & { \color{black} 0.000}({\color{black} 0.000}) & { \color{black} 0.000}({\color{black} 0.000}) & { \color{black} 0.000}({\color{black} 0.000}) & { \color{black} 0.000}({\color{black} 0.000}) & { \color{black} 0.000}({\color{black} 0.000})\\ \hline
\end{tabular}
\caption{\small Same as Tab.~\ref{tab:FisherMatrix_AC} now for differential parton-level distributions in inclusive top quark pair production datasets at $\sqrt{s}=13\ TeV$.
}
\label{tab:FisherMatrix_tt13}
\end{table}

%%%%%%%%%%%%%%%%%%%%%%%%%%%%%%%%%%%%%%%%%%%%%%%%%%%%%%%%%%%%%%%%%%%%%

\clearpage

\subsection{Impact of measurements on individual constraints (Fitmaker)}

The linear dependence maps from measurements to Wilson coefficients provide important information to understand the result of a fit. However, the final constraint on a Wilson coefficient depends not only on how strongly a measurement varies with the corresponding coefficient but also on the sensitivity of those measurements. Some information in this direction is thus provide by the Fisher information tables shown in the previous section.  This can be further quantified by extracting individual constraints obtained by switching on operators one by one in the fit while setting all others to zero. 

In Fig.~\ref{fig:EWPO_constraints} we show a table of the 95\% CL individual limits for electroweak measurements in each row and the corresponding constraint on a Wilson coefficient in each row. The limit is interpreted in terms of a scale $\Lambda/\sqrt{C_i}$ in units of TeV, colour coded in blue. We see that the bounds extend up to $\Lambda \sim 12$ TeV for $C_i \sim 1$, though this would be lower by a loop factor for loop-induced new physics and further lowered by a weak coupling. By picking out the darkest blue squares across a column, we can easily identify the measurement responsible for setting the strongest constraint on a particular operator coefficient. For example, $C_{Hl}^{(3)}$ and $C_{Hq}^{(3)}$ obtain some of their strongest constraints from $\Gamma_Z$, as expected from the fact that this measurement has a large linear dependence on them. 

Fig.~\ref{fig:diboson_constraints} shows the individual bounds from diboson measurements. The scale sensitivity here is limited to about $2$ TeV, though this could be increased for new physics that is more strongly coupled. We see here the importance of the $Zjj$ measurement in constraining $C_W$. This provides the highest scale sensitivity, reflecting the linear dependence for $C_W$ in Fig.~\ref{fig:diboson} showing a larger relative linear dependence here than in $WW$ (note though that the $a_i$'s are normalised to 1 across each row). 

The Higgs signal strengths and STXS bounds in Figs.~\ref{fig:Higgs_constraints} and \ref{fig:STXS_constraints}, respectively, demonstrate a sensitivity to scales above 10 TeV (again, this would be lowered for weakly coupled and loop-induced new physics, while the $C_{\mu H}$ operator should be lowered by a tiny muon Yukawa factor due to this particular operator's normalisation). However, we see that the operator coefficients in the first half of the table columns are sensitive to much lower scales than the sensitivity coming from electroweak precision measurements, with the exception of $C_{HWB}$. For the operators most relevant for the Higgs in the second half of the table, the STXS constraints are also less sensitive than those coming from Higgs signal strengths. The STXS measurements are more important for marginalised limits where all operator coefficients are allowed to vary simultaneously. 


\begin{figure}
    \centering
    \includegraphics[width=0.8\textwidth]{EWPO_indiv.pdf}
    \caption{95\% CL individual limits from electroweak measurements. }
    \label{fig:EWPO_constraints}
\end{figure}

\begin{figure}
    \centering
    \includegraphics[width=0.8\textwidth]{diboson_indiv.pdf}
    \caption{95\% CL individual limits from diboson measurements.}
    \label{fig:diboson_constraints}
\end{figure}

\begin{figure}
    \centering
    \includegraphics[width=0.8\textwidth]{Higgs_indiv.pdf}
    \caption{95\% CL individual limits from Higgs measurements.}
    \label{fig:Higgs_constraints}
\end{figure}

\begin{figure}
    \centering
    \includegraphics[width=0.8\textwidth]{STXS_indiv.pdf}
    \caption{95\% CL individual limits from Higgs STXS measurements.}
    \label{fig:STXS_constraints}
\end{figure}

\clearpage
\subsection{Global sensitivity from dataset variations (SMEFiT)}
%%%%%%%%%%%%%%%%%%%%%%%%%%%%%%%%%%%%%%%%%%%%%%%%%%%%%%%%%%%%%%%%%%%%%%

Finally, we should mention that ultimately the cleanest method to quantify the impact
of a specific dataset or group of processes is to repeat the fit by removing them
and studying what are the differences at the level of EFT coefficients.\footnote{One can
also deploy the methods of Bayesian reweighting~\cite{vanBeek:2019evb} to achieve the same goal.}
% - addition in v2:
In this Section, we illustrate the method with results of the SMEFiT global fit reported in Ref.~\cite{Ethier:2021bye} 
without producing new results or recommending any EFT analysis. 
%
For example, Fig.~\ref{fig:global_vs_toponly} shows a 
comparison of the size of the 95\% CL bounds
on the Wilson coefficients considered in this analysis obtained in the global fit
with those of fits to restricted datasets: a top-only, a Higgs-only, and a no-diboson fit.
%
This way one can identify which coefficients are most sensitive to which datasets
or groups of processes.
%
Some observations that can be derived from these plots are that diboson data
provides the unique handle on $c_{WWW}$; that Higgs processes provide some sensitivity
on the top electroweak couplings $c_{tW}$ and $c_{tZ}$ but much less than the top data itself;
and that most of the purely bosonic operators are completely unconstrained unless
Higgs measurements are included in the analysis.

%%%%%%%%%%%%%%%%%%%%%%%%%%%%%%%%%%%%%%%%%%%%%%%%%%%%%%%%%%%%%%%%%%%%%
\begin{figure}[t]
  \begin{center}
    \includegraphics[width=0.8\linewidth]{plots_v2/Coeffs_Bar_Top_only.pdf}
    \includegraphics[width=0.8\linewidth]{plots_v2/Coeffs_Bar_Higgs_only.pdf}
     \includegraphics[width=0.8\linewidth]{plots_v2/Coeffs_Bar_noVV.pdf}
     \caption{\label{fig:global_vs_toponly} Comparison of the magnitude of the 95\% CL bounds
       on the Wilson coefficients considered in the SMEFiT~\cite{Ethier:2021bye} analysis obtained in the global fit
     with those of fits to restricted datasets: a top-only, a Higgs-only, and a no-diboson fit.}
  \end{center}
\end{figure}


