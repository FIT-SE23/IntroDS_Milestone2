\begin{figure*}[t]
    \centering
    \begin{tikzpicture}[font=\sffamily]
        \node [rectangle, draw=ENCODER_COLOR, fill=LimeGreen!20, minimum height=1.5cm, minimum width=4.5cm] (ENCODER) {\textcolor{ENCODER_COLOR}{\footnotesize Transformer Encoder}};
        
        \node [rectangle, draw=DECODER_COLOR, fill=Peach!20, minimum height=1.5cm, minimum width=4.5cm, right=of ENCODER] (DECODER) {\textcolor{DECODER_COLOR}{\footnotesize Transformer Decoder}};

        % Encoder input
        \node [rectangle, draw=ENCODER_COLOR, align=center, minimum width=0.4cm, minimum height=0.8cm, below=of ENCODER, below=0.6cm] (ENCODER_INPUT_4) {\rotatebox{90}{\textcolor{ENCODER_COLOR}{\scriptsize x}}};
        \node [rectangle, draw=ENCODER_COLOR, align=center, minimum width=0.4cm, minimum height=0.8cm, left=of ENCODER_INPUT_4, anchor=west, left=0.2cm] (ENCODER_INPUT_3) {\rotatebox{90}{\textcolor{ENCODER_COLOR}{\scriptsize sin}}};
        \node [rectangle, draw=ENCODER_COLOR, align=center, minimum width=0.4cm, minimum height=0.8cm, left=of ENCODER_INPUT_3, anchor=west, left=0.2cm] (ENCODER_INPUT_2) {\rotatebox{90}{\textcolor{ENCODER_COLOR}{\scriptsize div}}};
        \node [rectangle, draw=ENCODER_COLOR, align=center, minimum width=0.4cm, minimum height=0.8cm, left=of ENCODER_INPUT_2, anchor=west, left=0.2cm] (ENCODER_INPUT_1) {\rotatebox{90}{\textcolor{ENCODER_COLOR}{\scriptsize SOE}}};
        \node [rectangle, draw=ENCODER_COLOR, align=center, minimum width=0.4cm, minimum height=0.8cm, right=of ENCODER_INPUT_4, anchor=east, right=0.2cm] (ENCODER_INPUT_5) {\rotatebox{90}{\textcolor{ENCODER_COLOR}{\scriptsize cos}}};
        \node [rectangle, draw=ENCODER_COLOR, align=center, minimum width=0.4cm, minimum height=0.8cm, right=of ENCODER_INPUT_5, anchor=east, right=0.2cm] (ENCODER_INPUT_6) {\rotatebox{90}{\textcolor{ENCODER_COLOR}{\scriptsize x}}};
        \node [rectangle, draw=ENCODER_COLOR, align=center, minimum width=0.4cm, minimum height=0.8cm, right=of ENCODER_INPUT_6, anchor=east, right=0.2cm] (ENCODER_INPUT_7) {\rotatebox{90}{\textcolor{ENCODER_COLOR}{\scriptsize EOE}}};
		
		% Decoder input
		\node [rectangle, draw=DECODER_COLOR, align=center, minimum width=0.4cm, minimum height=0.8cm, below=of DECODER, below=0.6cm] (DECODER_INPUT_2) {\rotatebox{90}{\textcolor{DECODER_COLOR}{\scriptsize tan}}};
		\node [rectangle, draw=DECODER_COLOR, align=center, minimum width=0.4cm, minimum height=0.8cm, left=of DECODER_INPUT_2, anchor=west, left=0.2cm] (DECODER_INPUT_1) {\rotatebox{90}{\textcolor{DECODER_COLOR}{\scriptsize SOE}}};
		\node [rectangle, draw=DECODER_COLOR, align=center, minimum width=0.4cm, minimum height=0.8cm, right=of DECODER_INPUT_2, anchor=east, right=0.2cm] (DECODER_INPUT_3) {\rotatebox{90}{\textcolor{DECODER_COLOR}{\scriptsize x}}};

		% Decoder output
		\node [rectangle, draw=DECODER_COLOR, align=center, minimum width=0.4cm, minimum height=0.8cm, above=of DECODER, above=0.6cm] (DECODER_OUTPUT_2) {\rotatebox{90}{\textcolor{DECODER_COLOR}{\scriptsize x}}};
		\node [rectangle, draw=DECODER_COLOR, align=center, minimum width=0.4cm, minimum height=0.8cm, left=of DECODER_OUTPUT_2, anchor=west, left=0.2cm] (DECODER_OUTPUT_1) {\rotatebox{90}{\textcolor{DECODER_COLOR}{\scriptsize tan}}};
		\node [rectangle, draw=DECODER_COLOR, align=center, minimum width=0.4cm, minimum height=0.8cm, right=of DECODER_OUTPUT_2, anchor=east, right=0.2cm] (DECODER_OUTPUT_3) {\rotatebox{90}{\textcolor{DECODER_COLOR}{\scriptsize EOE}}};
		
		% Encoder-decoder arrow
		\draw [-Stealth, ENCODER_COLOR] (ENCODER.east) -- (DECODER.west);
        
        % Encoder input arrows
        \draw [-Stealth, ENCODER_COLOR] (ENCODER_INPUT_1.north) -- ++(0, 0.6cm);
        \draw [-Stealth, ENCODER_COLOR] (ENCODER_INPUT_2.north) -- ++(0, 0.6cm);
        \draw [-Stealth, ENCODER_COLOR] (ENCODER_INPUT_3.north) -- ++(0, 0.6cm);
        \draw [-Stealth, ENCODER_COLOR] (ENCODER_INPUT_4.north) -- ++(0, 0.6cm);
        \draw [-Stealth, ENCODER_COLOR] (ENCODER_INPUT_5.north) -- ++(0, 0.6cm);
        \draw [-Stealth, ENCODER_COLOR] (ENCODER_INPUT_6.north) -- ++(0, 0.6cm);
        \draw [-Stealth, ENCODER_COLOR] (ENCODER_INPUT_7.north) -- ++(0, 0.6cm);
        
        % Decoder input arrows
        \draw [-Stealth, DECODER_COLOR] (DECODER_INPUT_1.north) -- ++(0, 0.6cm);
        \draw [-Stealth, DECODER_COLOR] (DECODER_INPUT_2.north) -- ++(0, 0.6cm);
        \draw [-Stealth, DECODER_COLOR] (DECODER_INPUT_3.north) -- ++(0, 0.6cm);
        
        % Decoder output arrows
        \draw [Stealth-, DECODER_COLOR] (DECODER_OUTPUT_1.south) -- ++(0, -0.6cm);
        \draw [Stealth-, DECODER_COLOR] (DECODER_OUTPUT_2.south) -- ++(0, -0.6cm);
        \draw [Stealth-, DECODER_COLOR] (DECODER_OUTPUT_3.south) -- ++(0, -0.6cm);
        
        % Expression and tree
        \node [text width=1.9cm, left=of ENCODER, anchor=east, left=2cm] (TREE) {{\scriptsize
        \begin{forest}
            for tree={s sep=10mm, inner sep=0, l=0}
            [div,
                [sin, [x]],
                [cos, [x]]
            ]
        \end{forest}
        }};
        
        \node [text width=1cm, left=of TREE] (EXP) {
            \begin{tabular}{@{}c@{}}
                 \scriptsize sin(x) \\
                 \hline
                 \scriptsize cos(x) \\
            \end{tabular}
        };
        
        \draw [-Stealth] (EXP.east) -- (TREE.west);
        
        \draw [-Stealth] (TREE.south) -- ++(0, -0.75cm) -- ++(2.75cm, 0) node [pos=0.5, above] {\scriptsize Polish Notation};
        
        \begin{scope}[on background layer]
            \draw [fill=Gray!20] ($(ENCODER.north west) + (-0.3cm, 0.3cm)$) rectangle ($(DECODER.south east) + (0.3cm, -0.3cm)$);
        \end{scope}
        
        
    \end{tikzpicture}
    \caption{An overview of our approach. The model is trained to generate an expression mathematically equivalent to the input. For example, given an input expression $\frac{\sin(x)}{\cos(x)}$, the model learns to generate $\tan(x)$. We use max pooling on the hidden states of the last encoder layer corresponding to the input tokens and use the result as the \textit{continuous vector representation} of the input expression. Here, ``{\sffamily \footnotesize SOE}'', ``{\sffamily \footnotesize EOE}'', and ``{\sffamily \footnotesize div}'' represent the start token, the end token, and the division operation, respectively.}
    \label{fig:equiv_exp_example}
\end{figure*}