\begin{figure}[t]
    % \vskip 0.2in
    \begin{center}
    % \begin{sc}
    \begin{subfigure}{0.49\textwidth}
        \centering
        \fontsize{7}{8}\selectfont\sffamily
        \includesvg[scale=0.28]{img/expemba_pca_plot.svg}
        \caption{\expemba{}}
    \end{subfigure}
    \hfill
    \begin{subfigure}{0.49\textwidth}
        \centering
        \fontsize{7}{8}\selectfont\sffamily
        \includesvg[scale=0.28]{img/expembe_pca_plot.svg}
        \caption{\expembe{}}
    \end{subfigure}
    \caption{Plots for the embedding vectors generated by \expemba{} (autoencoder) and \expembe{} (proposed equivalent-expression method). PCA is used to reduce the dimensionality from 512 to 2. \expembe{} groups expressions with operators of the same type together, indicating its ability to understand semantics, whereas \expemba{} groups expressions mainly based on their visual structure. The interactive versions of these plots are available on our project page.}
    \label{fig:pca_plots}
    % \end{sc}
    \end{center}
    % \vskip -0.2in
\end{figure}