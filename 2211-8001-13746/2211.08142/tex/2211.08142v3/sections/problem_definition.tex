% \section{Problem Definition}
% Though mathematical expressions represent the information in an accurate and precise manner, these expressions can be ambiguous when looked in isolation. Equations are generally accompanied with a context or a description and only convey an idea clearly when they are interpreted with the help of this context or description. The previous work has looked at . This approach does not work well for mathematical equations in practice. \citet{krstovski2018equation} treat an equation as a single token similar to words. \citet{greiner-petterWhyMachinesCannot2019} treat equations as single token, streams of tokens, and semantic group of tokens. We argue that You shall know a word by the company it keeps (Firth, J. R. 1957:11) does not apply to mathematical equations as well as it applies to natural text. The contents of an equation are as important as the context or description.