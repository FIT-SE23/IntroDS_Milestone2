\begin{abstract}
    % The recent success in Natural Language Processing (NLP) can be largely attributed to deep learning methods. One important task in NLP has been to represent words, sentences, or documents in a continuous vector space. These representations have played a crucial role in many of these state-of-the-art NLP systems. However, it is still unclear how these systems should handle mathematical equations. The word embedding approaches like word2vec do not work well for mathematical equations. This is partly due to the fact that the context assumptions that hold for natural text may not hold for mathematical equations. In this paper, we present an approach for representing mathematical equations in a continuous vector space. We use an encoder-decoder-based architecture for this problem. At the training time, we train the network to produce equivalent equations for a given input equation. After training, we discard the decoder and use the last hidden state of the encoder as equation embedding. In the absence of a suitable downstream task, we qualitatively analyze these embeddings using t-SNE plots. We also show that these embeddings learn analogies to some extent.
    
    Mathematical notation makes up a large portion of STEM literature, yet, finding semantic representations for formulae remains a challenging problem. Because mathematical notation is precise and its meaning changes significantly with small character shifts, the methods that work for natural text do not necessarily work well for mathematical expressions. In this work, we describe an approach for representing mathematical expressions in a continuous vector space. We use the encoder of a sequence-to-sequence architecture, trained on visually different but mathematically equivalent expressions, to generate vector representations (embeddings). We compare this approach with an autoencoder and show that the former is better at capturing mathematical semantics. Finally, to expedite future projects, we publish a corpus of equivalent transcendental and algebraic expression pairs.
\end{abstract}