\section{Conclusion}
\label{sec:conclusion}
In this work, we developed a framework to represent mathematical expressions in a continuous vector space. We showed that a \textsc{Seq2Seq} model could be trained to generate expressions that are mathematically equivalent to the input. The encoder of this model could be used to generate vector representations of mathematical expressions. We performed quantitative and qualitative experiments and demonstrated that these representations were semantically rich. Our experiments also showed that these representations are better at grouping similar expressions and perform better on the analogy task than the representations generated by an autoencoder.

We also note certain limitations. There is a need for downstream tasks or better evaluation metrics to evaluate the quality of the learned representations of mathematical expressions. Though we evaluated our approach on the \semvec{} datasets, we could not perform a similar evaluation on the Equivalent Expressions Dataset due to a limited number of equivalent expressions per expression. Furthermore, the dataset presented in this work consists of univariate expressions that contain simple algebraic and transcendental operators. It may be extended to include expressions with multiple variables and more complex operators. We leave these avenues for future research.
