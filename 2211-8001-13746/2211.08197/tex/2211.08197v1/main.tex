\documentclass[prl,twoside,twocolumn,10pt,floatfix,showpacs,citeautoscript,superscriptaddress]{revtex4-2}

\bibliographystyle{apsrev4-2}

\usepackage{amsmath}
\usepackage{amssymb}
\usepackage{booktabs}
\usepackage{comment}
\usepackage{glossaries}
\usepackage{graphicx}
\usepackage[version=4]{mhchem}
\usepackage{siunitx}
\usepackage{tabularx}
\usepackage[dvipsnames]{xcolor}

\usepackage[colorlinks=true]{hyperref}
\hypersetup{
    pdffitwindow=false,
    pdfstartview={FitH},
    pdfnewwindow=true,
    colorlinks=true,
    linkcolor=NavyBlue,
    citecolor=NavyBlue,
    filecolor=NavyBlue,
    urlcolor=NavyBlue,
}

\hypersetup{pdfauthor={Someone}}
\hypersetup{pdftitle={Some title}}

\graphicspath{{figures/}}

% acronyms
\setacronymstyle{long-short}
\newacronym{acf}{ACF}{autocorrelation function}
\newacronym{dft}{DFT}{density functional theory}
\newacronym{dho}{DHO}{damped harmonic oscillator}
\newacronym{ehm}{EHM}{effective harmonic model}
\newacronym{fc}{FC}{force constant}
\newacronym{fcp}{FCP}{force constant potential}
\newacronym{gpu}{GPU}{graphical processing unit}
\newacronym{md}{MD}{molecular dynamics}
\newacronym{mlp}{MLP}{machine learning potential}
\newacronym{nep}{NEP}{neuroevolution potential}
\newacronym{pes}{PES}{potential energy surface}
\newacronym{rfe}{RFE}{recursive feature elimination}
\newacronym{rmse}{RMSE}{root-mean-square error}
\newacronym{scan}{SCAN}{strongly constrained and appropriately normed}
\newacronym{sscha}{SSCHA}{stochastic self-consistent harmonic approximation}
\newacronym{scp}{SCP}{self-consistent phonon}
\newacronym{sed}{SED}{spectral energy density}
\newacronym{soap}{SOAP}{smooth overlap of atomic positions}

% autoref
\def\sectionautorefname{Sect.}
\def\figureautorefname{Fig.}
\def\tableautorefname{Table}
\def\equationautorefname{Eq.}
\def\noteautorefname{Note}

% commands
\DeclareSIUnit\angstrom{\text {Å}}
\renewcommand{\vec}[1]{\ensuremath\boldsymbol{#1}}
\renewcommand{\epsilon}[0]{\varepsilon}
\newcommand{\cmt}[1]{\emph{\color{red}#1}}

\global\let\oldnewlabel\newlabel
\gdef\newlabel#1#2{\newlabelxx{#1}#2}
\gdef\newlabelxx#1#2#3#4#5#6{\oldnewlabel{#1}{{#2}{#3}}}
\let\newlabel\oldnewlabel
\newlabel{snote:dft}{{S1}{2}{Supplemental Notes}{note.1}{}}
\newlabel{snote:model_validation}{{S2}{2}{Supplemental Note \thenote : Density functional theory calculations}{note.2}{}}
\newlabel{snote:md}{{S3}{2}{Supplemental Note \thenote : Training and validation of potential}{note.3}{}}
\newlabel{snote:fcp}{{S4}{2}{Supplemental Note \thenote : Molecular dynamics details}{note.4}{}}
\newlabel{snote:scp}{{S5}{2}{Supplemental Note \thenote : Force constant expansions}{note.5}{}}
\newlabel{snote:ehm}{{S6}{3}{Supplemental Note \thenote : Self-consistent phonons}{note.6}{}}
\newlabel{sfig:model_validation}{{S1}{3}{Parity plots for energies, forces, and virials for training (top) and test (bottom) data}{figure.1}{}}
\newlabel{sfig:energy_volume}{{S2}{4}{Energy-volume curves for the cubic phase of \ce {CsPbBr3} from \gls {nep} and \gls {dft} calculations}{figure.2}{}}
\newlabel{sfig:phonon_PES}{{S3}{4}{Phonon mode \gls {pes} along the M and R tilt modes with \gls {nep} and \gls {dft}. The top x-axis indicates the displacement of each Br atoms involved in the mode in units of Å}{figure.3}{}}
\newlabel{sfig:phonon_dispersion}{{S4}{5}{Phonon dispersions for the cubic phase of \ce {CsPbBr3} with the finite temperature dispersions from \gls {ehm} for \gls {dft} and \gls {nep} constructed from the same training structures (\gls {md} at \SI {400}{K})}{figure.4}{}}
\newlabel{sfig:scph_QM_CL_NEP_FCP}{{S5}{5}{(a) The M-tilt mode frequency vs temperature with the fourth order \gls {fcp} using \gls {scp}-hiphive quantum (QM) and classical (CL) sampling. (b) The M-tilt mode frequency vs temperature with compared between fourth-order \gls {fcp} and using the \gls {nep} model directly}{figure.5}{}}
\newlabel{sfig:acf_time_convergence}{{S6}{6}{The \gls {acf}, $C_Q(t)$, of the M-tilt mode at \SI {350}{\kelvin } with different total number of time steps considered in the ensemble averaging. A well converged \gls {acf} is obtained for \SI {50}{ns}}{figure.6}{}}
\newlabel{sfig:acf_size_convergence}{{S7}{6}{The \gls {acf}, $C_Q(t)$, of the M-tilt mode at \SI {350}{\kelvin } for different system size (size $N$ refers to a supercell of $N\times N\times N$ conventional cubic cells). Here, the sizes 12, 16, 20 produce almost identical \glspl {acf}}{figure.7}{}}
\newlabel{sfig:acf_exp_splits}{{S8}{7}{(a) The \gls {acf} of the mode coordinate, $C_Q(t)$, of the M-tilt mode at \SI {350}{\kelvin }. (b) The \gls {acf} of the mode velocity, $C_P(t)$, of the M-tilt mode at \SI {350}{\kelvin}. The timescales are $\tau_\text {L} = $\SI {5.22}{ps} and $\tau_\text{S} = $\SI {0.31}{ps}. Here, the solid lines corresponds to the raw \gls {acf} obtained from \gls {md}, the black dashed line corresponds to the fitted \gls {dho}, the colored dashed line corresponds to the long-timescale exponential decay in of the overdamped \gls {dho} and the dotted line corresponds to the short-time scale exponential decay}{figure.8}{}}
\newlabel{sfig:R15_frequency}{{S9}{7}{Temperature dependent frequency for the highest optical mode at R with different methods}{figure.9}{}}
\newlabel{sfig:Mtilt_powerspectra}{{S10}{8}{Power spectra of the M-tilt mode at various temperatures for the position ($Q$) and velocity ($P$) of the mode. The solid lines are the raw spectra obtained from \gls {md} simulations, the dashed colored lines correspond to the fits to the \gls {dho}, the dashed vertical lines correspond to $\omega _0$ from the \gls {dho} and the dotted lines correspond to the frequency obtained from \gls {scp}}{figure.10}{}}
\newlabel{sfig:dispersion_G2X_G2M}{{S11}{9}{Phonon frequencies and phonon linewidths for $\Gamma \to $X (left) and $\Gamma \to $M (right) calculated using the \gls {sed} method from \gls {md} simulations compared to experimental data \cite {Songvilay2019, Lanigan2021} at \SI {420}{\kelvin }. The point marked with a star correspond to the values obtained through phonon mode-projection. Solid lines serve as guides to the eyes (polynomial fits to the \gls {md} data)}{figure.11}{}}


% Affiliations
\newcommand{\addchalmers}{
    Department of Physics,
    Chalmers University of Technology,
    SE-41296, Gothenburg, Sweden
}
\newcommand{\addnims}{
    Research Center for Magnetic and Spintronic Materials,
    National Institute for Materials Science (NIMS),
    1-2-1 Sengen, Tsukuba, Ibaraki 305-0047, Japan
}

\begin{document}

\title{
    Probing the limits of the phonon quasi-particle picture:\texorpdfstring{\\}{}
    The transition from underdamped to overdamped dynamics in \texorpdfstring{\ce{CsPbBr3}}{CsPbBr3}
}


\author{Erik Fransson}
\author{Petter Rosander}
\author{Fredrik Eriksson}
\author{Magnus Rahm}
\affiliation{\addchalmers}
\author{Terumasa Tadano}
\affiliation{\addnims}
\author{Paul Erhart}
\affiliation{\addchalmers}
\email{erhart@chalmers.se}

\begin{abstract}
The soft modes associated with continuous-order phase transitions are commonly associated with particularly strong anharmonicity.
Here, we show that this can lead to overdamped behavior already far above the actual transition temperature using molecular dynamics simulations and a machine-learned potential.
While in the overdamped limit the interpretation of lattice vibrations as phonon quasi-particles with a specified frequency and relaxation time becomes physically questionable the mathematical description in terms of damped oscillators still holds.
A physically more intuitive picture can be obtained by considering the relaxation times of the mode coordinate and its conjugate momentum, which at the instability approach infinity and the inverse damping factor, respectively.
We demonstrate this behavior quantitatively for the prototypical case of the cubic-to-tetragonal phase transition of the inorganic halide perovskite \ce{CsPbBr3}, and show that the overdamped region extends almost 200 K (or about 60\%) above the transition temperature.
\end{abstract}

\maketitle


% ===============
%% Introduction
%% ==============

The vibrational properties of solids are pivotal for many physical phenomena, including but not limited to phase stability and thermal conduction.
In crystalline solids, the vibrational spectrum is commonly described in terms of phonons as quasi-particle representations of the lattice vibrations.
The phonon frequency, $\omega_0$, is typically much larger than the damping $\Gamma$; the phonon relaxation time $\tau = 2/\Gamma$ is thus much longer than the oscillation period, such that the quasi-particle picture is well motivated \cite{Zim60, SunAll2010, SunSheAll2010, ZhaSunWen2014, Lv2016, Isaeva2019}.
In this so-called underdamped limit, the relaxation time \emph{decreases} as the damping $\Gamma$ increases.

By comparison, there are far fewer cases when phonon modes become overdamped, i.e., $\omega_0 \tau < 1$ \cite{Silverman1974, SchSto1978}.
This can occur either due to large damping or for very soft modes, usually in the immediate vicinity of a phase transition, as for example in the case of body-centered cubic Ti \cite{PetHeiTra91, FraErh20, FraSlaErhWah2021}, rotationally disordered 2D materials \cite{Kim2021}, in ferroelectrics such as \ce{BaTiO3} \cite{Nakamura1975, Nakamura1992, Dove1997, Ehsan2021} or in halide perovskites \cite{Songvilay2019, Lanigan2021}.
Notably, in the overdamped limit, the relaxation time \emph{increases} with increasing damping $\Gamma$, which calls into the question the picture of a well defined phonon mode with a frequency and relaxation time.
Overdamped phonon dynamics is, however, usually limited to a rather narrow temperature window and under these circumstances the inversion of the relation between relaxation time and damping cannot be readily observed.
Here, we demonstrate that the soft phonons modes associated with the phase transitions in the prototypical halide perovskite \ce{CsPbBr3} are, however, outstanding manifestations of this exact behavior as the overdamped region extends almost \SI{200}{\kelvin} above the tetragonal-cubic phase transition.

Halide perovskites are promising materials for photovoltaic and optoelectronic applications.
Specifically, \ce{CsPbBr3} has received a lot of attention in recent years \cite{Stoumpos2013}.
With increasing temperature it undergoes phase transitions from an orthorhombic (Pnma) to a tetragonal (P4/mbm) and eventually a cubic phase (Pm$\bar{3}$m) \cite{Hirotsu1974, Sharma1991, Rodov2003, Lopez2020, Malyshkin2020}.
These phase transitions are connected to specific phonon modes and arise due to tilting of the \ce{PbBr6} octahedra, corresponding to phonon modes at the R and M points (\autoref{fig:phonons_and_phase_transitions}a) \cite{Huang2014, daSilva2015, Yang2017, Yang2020}.
Experimentally, these modes have been shown to exhibit overdamped characteristics in the vicinity of the phase transitions \cite{Songvilay2019, Lanigan2021, Cohen2022}.
The phase transitions have also been studied from first-principles and via \gls{md} simulations, see, e.g., Refs.~\onlinecite{ZhuEgger2022, Lahnsteiner2022, TadWis2022}.

\begin{figure*}[bt]
\centering
\includegraphics{fig1a.pdf}
\includegraphics{fig1b.pdf}
\caption{
    (a) Phonon dispersion for the cubic phases of \ce{CsPbBr3} obtained using the \gls{mlp} in the harmonic approximation (\SI{0}{\kelvin}) and from an \gls{ehm} at \SI{500}{\kelvin}.
    (b) Potential energy landscape along the unstable M-mode calculated with \gls{mlp} and \gls{dft}.
    The inset shows the \ce{CsPbBr3} crystal structure (Cs purple, Pb gray, Br red) in the energy minima, for which the \ce{PbBr6} have been tilted in-phase (visualization made with \textsc{ovito} \cite{Stukowski2010}).
    (c, d, e) Lattice parameters and mode coordinates obtained from a cooling run based on the isothermal-isobaric ($NpT$) ensemble with phase transitions at approximately \SI{300}{\kelvin} and \SI{265}{\kelvin}.
}
\label{fig:phonons_and_phase_transitions}
\end{figure*}

Here, we reveal the dynamics of the octahedral tilt modes in \ce{CsPbBr3} over a wide temperature range, which as shown below requires both large large systems (comprising at least several 10,000 atoms) and sufficiently long times scales ($\sim\,\SI{50}{\nano\second}$ to \SI{100}{\nano\second}) in order to achieve converged results (\autoref{sfig:acf_size_convergence} and \autoref{sfig:acf_time_convergence}) \footnote{
    See Supplemental Material for additional figures and further details.
}.
To this end, we employ a \gls{mlp} that achieves close to \gls{dft} accuracy (\autoref{snote:model_validation}) \cite{FanZenZha21, FanWanYin22} and extensive \gls{md} simulations enabled by an efficient implementation on \glspl{gpu} \footnote{
    The \gls{dft} data and the \gls{mlp} models are provided in a zenodo dataset \cite{zenodo_dataset}.
}.
Reference data for the construction of the \gls{mlp} was generated by \gls{dft} calculations \cite{KreHaf93, Blo94, KreFur96} using the \gls{scan} exchange-correlation functional \cite{SunRuzPer15} (\autoref{snote:dft}).
Simulations and atomic structures were handled via the \textsc{ase} \cite{Larsen2017} and \textsc{calorine} packages \cite{calorine}.
The dynamics were analyzed directly from \gls{md} simulations via normal mode projections.
The obtained phonon frequencies and relaxation times with the \gls{mlp} are in good agreement with experimental work for multiple phonon modes (see \autoref{sfig:dispersion_G2X_G2M}).
In addition, we consider several different
\gls{scp} renormalization  methods \cite{Esfarjani2020} as well as \glspl{ehm} \cite{Kong2009, Kong2011, And12, HelSteAbr13} using the \textsc{hiphive} \cite{EriFraErh19}, \textsc{alamode} \cite{TadGohTsu14}, and \textsc{sscha} packages \cite{MonBiaChe21}.


% ===============
%% Methods
%% ==============
%\textit{Mode projections.}
To analyze phonon modes directly from \gls{md} simulations we employ phonon mode projection \cite{SunSheAll2010, CarTogTan2017, RohLiLuoHen2022}.
The atomic displacements $\vec{u}(t)$ and velocities $\vec{v}(t)$ can be projected on a mode $\lambda$, with the supercell eigenvector $\vec{e}_\lambda$ via
\begin{align*}
    Q_\lambda (t) = \vec{u}(t) \cdot \vec{e}_\lambda \quad\text{and}\quad
    P_\lambda (t) = \vec{v}(t) \cdot \vec{e}_\lambda.
\end{align*}
The \glspl{acf} of $Q$ and $P$ are calculated in order to analyze the dynamics of the modes of interest as
\begin{align}
    \label{eq:mode_acf}
    C_Q(t) =& \left < Q_\lambda (t') Q_\lambda (t+t') \right >,
\end{align}
which can be modeled as the \gls{acf} of a \gls{dho}.
The \gls{dho} is driven by a stochastic force and has a natural frequency $\omega_0$ and a damping $\Gamma$.
The \gls{acf} of the \gls{dho} splits into an underdamped regime ($\omega_0 > \Gamma/2$) and an overdamped regime ($\omega_0 < \Gamma/2$).
In the underdamped regime the solution of the \gls{dho} is
\begin{align}
    \label{eq:acf_Q_underdamped}
    C^\text{DHO}_Q(t) = A \mathrm{e}^{-t/\tau} \left ( \cos{\omega_e t} + \frac{\Gamma}{2\omega_e}\sin{\omega_e t} \right ),
\end{align}
where $\omega_e = \sqrt{\omega_0^2 - \frac{\Gamma^2}{4}}$, the relaxation time is $\tau = 2 / \Gamma$, and $A$ is the amplitude \cite{FraSlaErhWah2021}.
In the overdamped limit the solution becomes the sum of two exponential decays as
\begin{align}
    \label{eq:acf_Q_overdamped}
    C^\text{DHO}_Q(t) = \frac{A}{\tau_\text{L} - \tau_\text{S}} \left(  \tau_\text{L} \text{e}^{-t / \tau_\text{L}}  - \tau_\text{S} \text{e}^{-t / \tau_\text{S}} \right )
\end{align}
where
\begin{align*}
    \tau_\text{S,L} = \frac{\tau}{1 \pm \sqrt{1-(\omega_0\tau)^2}}.
\end{align*}
Here, $\tau_\text{S}$ is a short timescale and $\tau_\text{L}$ is the long timescale.
If the natural frequency approaches zero (e.g., for continuous phase transitions driven by a soft-mode) we thus expect  $\tau_\text{L} \to \infty$ and $\tau_\text{S} \to \tau/2$.
In this limit the resulting \gls{acf}, $C^\text{DHO}_Q(t)$, would only consist of a single exponential decay, with a decay time approaching infinity, which corresponds to the behavior seen in Brownian motion.

Similar expressions are obtained for the \gls{acf} of the phonon velocity, which is $C^\text{DHO}_P(t) = - \frac{\mathrm{d}^2 }{\mathrm{d} t^2} C^\text{DHO}_Q(t)$.
For the overdamped case it becomes
\begin{align*}
C^\text{DHO}_P(t) =\frac{A}{\tau_\text{L}-\tau_\text{S}} \left(  \frac{1}{\tau_\text{S}} \text{e}^{-t / \tau_\text{S}} - \frac{1}{\tau_\text{L}} \text{e}^{-t / \tau_\text{L}}  \right )
\end{align*}
The \glspl{acf} for $Q$ and $P$ are fitted simultaneously to the \gls{dho} model in order to extract $\omega_0$ and $\Gamma$.



\begin{figure}
\centering
\includegraphics{fig2.pdf}
\caption{
    Mode coordinate $Q(t)$ for the M-tilt mode at \SI{500}{\kelvin} (well above $T_C$), at \SI{350}{\kelvin} (close to $T_C$) and at \SI{280}{\kelvin} (below above $T_C$).
    The M-tilt mode is three-fold degenerate ($x$, $y$, $z$) but here we only show $M_z$.
    At \SI{280}{\kelvin}, the system switches the tilt axis at irregular intervals.
    As a result one observes $M_z \approx 0$ at certain times.
}
\label{fig:mode_coordinates}
\end{figure}


% ===============
%% Results
%% ==============

%% Tilt-modes and phase transitions
The phase transitions in \ce{CsPbBr3}, and similarly in many other perovskites, are driven by the modes that correspond to the tilting of the \ce{PbBr6} octahedra.
These modes are located at the M (in-phase tilting) and R (out of phase tilting) points in the phonon dispersion for the cubic structure (\autoref{fig:phonons_and_phase_transitions}a).
They are three-fold degenerate corresponding to tilting around the three Cartesian directions.
These tilt-modes exhibit a double-well \gls{pes}, which the \gls{mlp} reproduces  perfectly compared to \gls{dft} (\autoref{fig:phonons_and_phase_transitions}b).

The \gls{mlp} predicts the cubic-tetragonal and tetragonal-orthorhombic transitions at \SI{300}{\kelvin} and \SI{265}{\kelvin}, respectively (\autoref{fig:phonons_and_phase_transitions}c), which is lower than the experimental values of \SI{400}{\kelvin} and \SI{360}{\kelvin} \cite{Sharma1991, Rodov2003, Stoumpos2013, Malyshkin2020}, a discrepancy that can be primarily attributed to the underlying exchange-correlation functional.

The mode coordinates of the tilt-modes are very useful order parameters for analyzing the phase transitions (\autoref{fig:phonons_and_phase_transitions}d,e).
At \SI{300}{K} the system transitions from the cubic to the tetragonal phase as seen in both the lattice parameters and in the freezing in of one of the three M-tilt modes (M$_z$).
For the tetragonal phase two R-modes (R$_x$ and R$_y$) start to show larger fluctuations, and at \SI{265}{K} the system transitions to the orthorhombic phase.
Here, we also note the slight difference in character between these two phase transitions.
For the cubic-tetragonal transition the order parameter and lattice parameter changes rather sharply at the transition temperature $T_C$ (closer in character to a displacive transition), whereas for the tetragonal-orthorhombic transition the order parameter and lattice parameter change more gradually around $T_C$ (exhibiting more order-disorder character) in agreement with experimental observations of the transition character \cite{Stoumpos2013}.

%% mode coordinate dynamics at high T, close to Tc and below Tc
The mode coordinates exhibit interesting dynamical behavior already in the cubic phase, which can be conveniently observed in the time domain (\autoref{fig:mode_coordinates}).
At \SI{500}{K} regular (phonon) oscillator behavior is observed, whereas for \SI{350}{K} (closer but still above $T_C$) a slower dynamic component becomes evident.
Finally, at \SI{280}{K} and thus below the phase transition, one observes the common oscillatory motion superimposed on a long timescale hopping motion between the two minima, corresponding to the (degenerate) tetragonal phase (\autoref{fig:phonons_and_phase_transitions}b) \footnote{
    We note that the hopping frequency depends strongly on the system size, and is thus not a good observable on its own.
}.


\begin{figure}
\centering
\includegraphics{fig3.pdf}
\caption{
    Auto-correlation functions (solid lines) of (a) the highest optical mode at the R-point as well as the M-tilt mode at (b) \SI{500}{\kelvin} and (c) \SI{350}{\kelvin}, along with fits to the \gls{dho} model (dashed lines).
}
\label{fig:mode_ACF}
\end{figure}

%% Autocorrelation and damped-harmonic oscillator
The mode coordinate can be analyzed by fitting the respective \glspl{acf} to a \gls{dho} model (\autoref{fig:mode_ACF}).
The \gls{acf} for a regular (underdamped) mode shows a clear oscillatory pattern as illustrated here by the highest optical mode at the R-point with a typical relaxation time of about \SI{0.37}{ps} (\autoref{fig:mode_ACF}a).
For the M-tilt mode at \SI{500}{K} the mode is quite strongly damped (yet underdamped) and the \gls{acf} decays quickly with a relaxation time of about \SI{0.58}{ps} (\autoref{fig:mode_ACF}b).
At \SI{350}{K}, however, the same mode is overdamped and in this case the \gls{dho} model becomes the sum of two exponential decays, see \autoref{eq:acf_Q_overdamped}, with relaxation  times $\tau_\text{L} = $\SI{5.22}{ps} and $\tau_\text{S} = $\SI{0.31}{ps}.
It is noteworthy that the relaxation time of the \gls{acf} at \SI{350}{K} is about ten-times longer than at \SI{500}{K}.
The \gls{dho} fits still match the data surprisingly well for both the underdamped and overdamped cases (see \autoref{sfig:acf_exp_splits} for how the two exponential decays behave for $Q$ and $P$ in the overdamped case).
When $\Gamma / \omega_0$ increases and the system becomes overdamped the dynamics of the modes is moving towards the diffusive Brownian motion regime.
For overdamped modes the relaxation time of the \gls{acf} increases as $\Gamma / \omega_0$ increases, opposite to the underdamped behavior.
While this is a well known feature of a simple one-dimensional \gls{dho}, here we demonstrate this behavior for phonon modes in a complex system.
This phenomenon arises due to the \emph{free} energy landscape being very flat close to the transition.% to resembling a bathtub.
As a result of the high friction it therefore takes a long time for the \gls{dho} to move back and forth around zero (\autoref{fig:mode_coordinates}c; also see \autoref{sfig:Mtilt_powerspectra} for the power spectra) \cite{Volpe2013}.


\begin{figure}
\centering
\includegraphics{fig4.pdf}
\caption{
    Frequencies and relaxation times obtained from \glspl{acf} of a) M-tilt and b) R-tilt modes as a function of temperature.
    Here, markers are data-points and lines are interpolations to guide the eye.
    The shaded region indicates the overdamped regime.
}
\label{fig:lifetimes_vs_temperature}
\end{figure}

%% Frequency and lifetimes vs temperature
The frequencies and relaxation times of the M-tilt and R-tilt modes are summarized as a function of temperature in \autoref{fig:lifetimes_vs_temperature}.
The frequency $\omega_0$ softens significantly with decreasing temperature for both modes, whereas the relaxation times $\tau$ is more or less constant.
We note that the frequency $\omega_0$ decreases almost linearly across the entire temperature range and does not follow $\omega_0^2 \sim (T-T_C)$ as sometimes seen for displacive phase transitions \cite{Cochran1960, PytteFeder1969, Silverman1974}.
The softening of the frequency thus drives the modes to the overdamped limit with decreasing temperature.
Remarkably, the M-tilt and R-tilt modes only become underdamped above \SI{480}{\kelvin} and \SI{410}{\kelvin}, respectively, well above the transition temperature to the tetragonal phase at \SI{300}{\kelvin}.
At the cross-over from the underdamped to the overdamped regime, the two time scales $\tau_\text{S}$ and $\tau_\text{L}$ emerge. 
When approaching $T_C$  we see that $\tau_\text{L}$ increases exponentially, whereas  $\tau_\text{S} \to \tau/2$.



\begin{figure}
\centering
\includegraphics{fig5.pdf}
\caption{
    Frequency of (a) M-tilt and (b) R-tilt as a function of temperature from several \gls{scp} schemes as well as \glspl{ehm}.
}
\label{fig:frequencies_vs_temperature}
\end{figure}

%% SCP methods and EHMs
Lastly, we analyze the representation of these strongly anharmonic modes by commonly used phonon renormalization techniques, specifically different \gls{scp} schemes and \glspl{ehm} (\autoref{fig:frequencies_vs_temperature}) (see \autoref{snote:scp} and \autoref{snote:ehm} for a more detailed description of the methods).
There are several variations of \glspl{scp} \cite{Esfarjani2020}, and here we employ three different ones \gls{scp}-alamode, \gls{sscha}, \gls{scp}-hiphive.
In \gls{scp}-alamode the Green's functions approach is employed as implemented in  the \textsc{alamode} software \cite{TadGohTsu14}.
In \gls{sscha} the harmonic free energy is minimized using gradient methods, as implemented in the software package \textsc{sscha} \cite{MonBiaChe21}.
In \gls{scp}-hiphive second order force-constants are obtained from fitting with displacement sampled from the harmonic model and forces obtained from the \gls{mlp} as implemented in the \textsc{hiphive} package \cite{EriFraErh19}.
The \glspl{ehm} (in this field also referred to as temperature-dependent potentials) are constructed from fitting second order force-constants to displacement and force data obtained from \gls{md} with the \gls{mlp}.

Here, we find a very similar behavior for both M-tilt and R-tilt modes.
The three \gls{scp} (\gls{scp}-hiphive, \gls{sscha}, \gls{scp}-alamode) methods employed here are in remarkable agreement with each other given the differences in theory and implementation between them.
The \gls{scp} frequencies systematically overestimate the frequency $\omega_0$ obtained from the \glspl{acf} by almost a factor of two (see \autoref{snote:scp} for a more detailed description of the \gls{scp} methods).
The \glspl{ehm} constructed by fitting the  forces from \gls{md} trajectories show good agreement with the mode projection results.
We note here that the trend for \glspl{scp} and \glspl{ehm} to over and underestimate frequencies, respectively, appears to hold for all modes in the system, which is in line with previous studies \cite{KorBelYan2018, MetKli2019, TadWis2022, Tolborg2022}.
However,  while \glspl{ehm} from \gls{md} yield a better frequency for the tilt mode compared to \gls{scp}, this is not in general true (see \autoref{sfig:R15_frequency} for details).


% ===============
%% Discussion
%% ==============
To conclude, we have carried out a detailed computational analysis of the tilt-modes in \ce{CsPbBr3}.
Most strikingly, we observe overdamped modes for the cubic phase almost \SI{200}{K} above the phase transition temperatures.
These overdamped modes exhibit correlation on very long time scales, much longer than the typical relaxation time  or period of the mode.
This is in line with the dynamics transitioning toward Brownian motion behavior.
What we find here is that these modes can, however, still be mathematically well described by a \gls{dho}, from which one can formally obtain a phonon frequency and relaxation time compliant with a quasi-particle picture.
A physically more intuitive description is, however, obtained if the \gls{dho} model is described by two relaxation times, which can be approximately associated with mode coordinate and momentum, respectively.

In addition, we demonstrated that commonly used computational phonon renormalization methods agree very well with each other but exhibit systematic errors in describing the frequency of anharmonic modes.
Understanding the single-point frequency obtained from such methods and its relation to the full dynamical spectra is thus very important when, e.g., comparing with experimental measurements.

\begin{acknowledgments}
This work was funded by the Swedish Research Council (grant numbers 2018-06482, 2020-04935, 2021-05072), the Swedish Energy Agency (grant No. 45410-1), the Excellence Initiative Nano at Chalmers, and the Chalmers Initiative for Advancement of Neutron and Synchrotron Techniques.
T.\ T. was supported by JSPS KAKENHI Grant No. 21K03424.
The computations were enabled by resources provided by the Swedish National Infrastructure for Computing (SNIC) at NSC, C3SE, and PDC partially funded by the Swedish Research Council (grant number 2018-05973).
\end{acknowledgments}

%apsrev4-2.bst 2019-01-14 (MD) hand-edited version of apsrev4-1.bst
%Control: key (0)
%Control: author (72) initials jnrlst
%Control: editor formatted (1) identically to author
%Control: production of article title (-1) disabled
%Control: page (0) single
%Control: year (1) truncated
%Control: production of eprint (0) enabled
\begin{thebibliography}{61}%
\makeatletter
\providecommand \@ifxundefined [1]{%
 \@ifx{#1\undefined}
}%
\providecommand \@ifnum [1]{%
 \ifnum #1\expandafter \@firstoftwo
 \else \expandafter \@secondoftwo
 \fi
}%
\providecommand \@ifx [1]{%
 \ifx #1\expandafter \@firstoftwo
 \else \expandafter \@secondoftwo
 \fi
}%
\providecommand \natexlab [1]{#1}%
\providecommand \enquote  [1]{``#1''}%
\providecommand \bibnamefont  [1]{#1}%
\providecommand \bibfnamefont [1]{#1}%
\providecommand \citenamefont [1]{#1}%
\providecommand \href@noop [0]{\@secondoftwo}%
\providecommand \href [0]{\begingroup \@sanitize@url \@href}%
\providecommand \@href[1]{\@@startlink{#1}\@@href}%
\providecommand \@@href[1]{\endgroup#1\@@endlink}%
\providecommand \@sanitize@url [0]{\catcode `\\12\catcode `\$12\catcode
  `\&12\catcode `\#12\catcode `\^12\catcode `\_12\catcode `\%12\relax}%
\providecommand \@@startlink[1]{}%
\providecommand \@@endlink[0]{}%
\providecommand \url  [0]{\begingroup\@sanitize@url \@url }%
\providecommand \@url [1]{\endgroup\@href {#1}{\urlprefix }}%
\providecommand \urlprefix  [0]{URL }%
\providecommand \Eprint [0]{\href }%
\providecommand \doibase [0]{https://doi.org/}%
\providecommand \selectlanguage [0]{\@gobble}%
\providecommand \bibinfo  [0]{\@secondoftwo}%
\providecommand \bibfield  [0]{\@secondoftwo}%
\providecommand \translation [1]{[#1]}%
\providecommand \BibitemOpen [0]{}%
\providecommand \bibitemStop [0]{}%
\providecommand \bibitemNoStop [0]{.\EOS\space}%
\providecommand \EOS [0]{\spacefactor3000\relax}%
\providecommand \BibitemShut  [1]{\csname bibitem#1\endcsname}%
\let\auto@bib@innerbib\@empty
%</preamble>
\bibitem [{\citenamefont {Ziman}(1960)}]{Zim60}%
  \BibitemOpen
  \bibfield  {author} {\bibinfo {author} {\bibfnamefont {J.~M.}\ \bibnamefont
  {Ziman}},\ }\href@noop {} {\emph {\bibinfo {title} {Electrons and Phonons}}}\
  (\bibinfo  {publisher} {Oxford University Press, London},\ \bibinfo {year}
  {1960})\BibitemShut {NoStop}%
\bibitem [{\citenamefont {Sun}\ and\ \citenamefont {Allen}(2010)}]{SunAll2010}%
  \BibitemOpen
  \bibfield  {author} {\bibinfo {author} {\bibfnamefont {T.}~\bibnamefont
  {Sun}}\ and\ \bibinfo {author} {\bibfnamefont {P.~B.}\ \bibnamefont
  {Allen}},\ }\href {https://doi.org/10.1103/PhysRevB.82.224305} {\bibfield
  {journal} {\bibinfo  {journal} {Phys. Rev. B}\ }\textbf {\bibinfo {volume}
  {82}},\ \bibinfo {pages} {224305} (\bibinfo {year} {2010})}\BibitemShut
  {NoStop}%
\bibitem [{\citenamefont {Sun}\ \emph {et~al.}(2010)\citenamefont {Sun},
  \citenamefont {Shen},\ and\ \citenamefont {Allen}}]{SunSheAll2010}%
  \BibitemOpen
  \bibfield  {author} {\bibinfo {author} {\bibfnamefont {T.}~\bibnamefont
  {Sun}}, \bibinfo {author} {\bibfnamefont {X.}~\bibnamefont {Shen}},\ and\
  \bibinfo {author} {\bibfnamefont {P.~B.}\ \bibnamefont {Allen}},\ }\href
  {https://doi.org/10.1103/PhysRevB.82.224304} {\bibfield  {journal} {\bibinfo
  {journal} {Phys. Rev. B}\ }\textbf {\bibinfo {volume} {82}},\ \bibinfo
  {pages} {224304} (\bibinfo {year} {2010})}\BibitemShut {NoStop}%
\bibitem [{\citenamefont {Zhang}\ \emph {et~al.}(2014)\citenamefont {Zhang},
  \citenamefont {Sun},\ and\ \citenamefont {Wentzcovitch}}]{ZhaSunWen2014}%
  \BibitemOpen
  \bibfield  {author} {\bibinfo {author} {\bibfnamefont {D.-B.}\ \bibnamefont
  {Zhang}}, \bibinfo {author} {\bibfnamefont {T.}~\bibnamefont {Sun}},\ and\
  \bibinfo {author} {\bibfnamefont {R.~M.}\ \bibnamefont {Wentzcovitch}},\
  }\href {https://doi.org/10.1103/PhysRevLett.112.058501} {\bibfield  {journal}
  {\bibinfo  {journal} {Phys. Rev. Lett.}\ }\textbf {\bibinfo {volume} {112}},\
  \bibinfo {pages} {058501} (\bibinfo {year} {2014})}\BibitemShut {NoStop}%
\bibitem [{\citenamefont {Lv}\ and\ \citenamefont {Henry}(2016)}]{Lv2016}%
  \BibitemOpen
  \bibfield  {author} {\bibinfo {author} {\bibfnamefont {W.}~\bibnamefont
  {Lv}}\ and\ \bibinfo {author} {\bibfnamefont {A.}~\bibnamefont {Henry}},\
  }\bibfield  {journal} {\bibinfo  {journal} {Scientific Reports}\ }\textbf
  {\bibinfo {volume} {6}},\ \href {https://doi.org/10.1038/srep37675}
  {10.1038/srep37675} (\bibinfo {year} {2016})\BibitemShut {NoStop}%
\bibitem [{\citenamefont {Isaeva}\ \emph {et~al.}(2019)\citenamefont {Isaeva},
  \citenamefont {Barbalinardo}, \citenamefont {Donadio},\ and\ \citenamefont
  {Baroni}}]{Isaeva2019}%
  \BibitemOpen
  \bibfield  {author} {\bibinfo {author} {\bibfnamefont {L.}~\bibnamefont
  {Isaeva}}, \bibinfo {author} {\bibfnamefont {G.}~\bibnamefont
  {Barbalinardo}}, \bibinfo {author} {\bibfnamefont {D.}~\bibnamefont
  {Donadio}},\ and\ \bibinfo {author} {\bibfnamefont {S.}~\bibnamefont
  {Baroni}},\ }\bibfield  {journal} {\bibinfo  {journal} {Nature
  Communications}\ }\textbf {\bibinfo {volume} {10}},\ \href
  {https://doi.org/10.1038/s41467-019-11572-4} {10.1038/s41467-019-11572-4}
  (\bibinfo {year} {2019})\BibitemShut {NoStop}%
\bibitem [{\citenamefont {Silverman}(1974)}]{Silverman1974}%
  \BibitemOpen
  \bibfield  {author} {\bibinfo {author} {\bibfnamefont {B.~D.}\ \bibnamefont
  {Silverman}},\ }\href {https://doi.org/10.1103/PhysRevB.9.203} {\bibfield
  {journal} {\bibinfo  {journal} {Phys. Rev. B}\ }\textbf {\bibinfo {volume}
  {9}},\ \bibinfo {pages} {203} (\bibinfo {year} {1974})}\BibitemShut {NoStop}%
\bibitem [{\citenamefont {Schneider}\ and\ \citenamefont
  {Stoll}(1978)}]{SchSto1978}%
  \BibitemOpen
  \bibfield  {author} {\bibinfo {author} {\bibfnamefont {T.}~\bibnamefont
  {Schneider}}\ and\ \bibinfo {author} {\bibfnamefont {E.}~\bibnamefont
  {Stoll}},\ }\href {https://doi.org/10.1103/PhysRevB.17.1302} {\bibfield
  {journal} {\bibinfo  {journal} {Phys. Rev. B}\ }\textbf {\bibinfo {volume}
  {17}},\ \bibinfo {pages} {1302} (\bibinfo {year} {1978})}\BibitemShut
  {NoStop}%
\bibitem [{\citenamefont {Petry}\ \emph {et~al.}(1991)\citenamefont {Petry},
  \citenamefont {Heiming}, \citenamefont {Trampenau}, \citenamefont {Alba},
  \citenamefont {Herzig}, \citenamefont {Schober},\ and\ \citenamefont
  {Vogl}}]{PetHeiTra91}%
  \BibitemOpen
  \bibfield  {author} {\bibinfo {author} {\bibfnamefont {W.}~\bibnamefont
  {Petry}}, \bibinfo {author} {\bibfnamefont {A.}~\bibnamefont {Heiming}},
  \bibinfo {author} {\bibfnamefont {J.}~\bibnamefont {Trampenau}}, \bibinfo
  {author} {\bibfnamefont {M.}~\bibnamefont {Alba}}, \bibinfo {author}
  {\bibfnamefont {C.}~\bibnamefont {Herzig}}, \bibinfo {author} {\bibfnamefont
  {H.~R.}\ \bibnamefont {Schober}},\ and\ \bibinfo {author} {\bibfnamefont
  {G.}~\bibnamefont {Vogl}},\ }\href
  {https://doi.org/10.1103/PhysRevB.43.10933} {\bibfield  {journal} {\bibinfo
  {journal} {Physical Review B}\ }\textbf {\bibinfo {volume} {43}},\ \bibinfo
  {pages} {10933} (\bibinfo {year} {1991})}\BibitemShut {NoStop}%
\bibitem [{\citenamefont {Fransson}\ and\ \citenamefont
  {Erhart}(2020)}]{FraErh20}%
  \BibitemOpen
  \bibfield  {author} {\bibinfo {author} {\bibfnamefont {E.}~\bibnamefont
  {Fransson}}\ and\ \bibinfo {author} {\bibfnamefont {P.}~\bibnamefont
  {Erhart}},\ }\href {https://doi.org/10.1016/j.actamat.2020.06.040} {\bibfield
   {journal} {\bibinfo  {journal} {Acta Materialia}\ }\textbf {\bibinfo
  {volume} {196}},\ \bibinfo {pages} {770} (\bibinfo {year}
  {2020})}\BibitemShut {NoStop}%
\bibitem [{\citenamefont {Fransson}\ \emph {et~al.}(2021)\citenamefont
  {Fransson}, \citenamefont {Slabanja}, \citenamefont {Erhart},\ and\
  \citenamefont {Wahnstr\"{o}m}}]{FraSlaErhWah2021}%
  \BibitemOpen
  \bibfield  {author} {\bibinfo {author} {\bibfnamefont {E.}~\bibnamefont
  {Fransson}}, \bibinfo {author} {\bibfnamefont {M.}~\bibnamefont {Slabanja}},
  \bibinfo {author} {\bibfnamefont {P.}~\bibnamefont {Erhart}},\ and\ \bibinfo
  {author} {\bibfnamefont {G.}~\bibnamefont {Wahnstr\"{o}m}},\ }\href
  {https://doi.org/10.1002/adts.202000240} {\bibfield  {journal} {\bibinfo
  {journal} {Advanced Theory and Simulations}\ }\textbf {\bibinfo {volume}
  {4}},\ \bibinfo {pages} {2000240} (\bibinfo {year} {2021})}\BibitemShut
  {NoStop}%
\bibitem [{\citenamefont {Kim}\ \emph {et~al.}(2021)\citenamefont {Kim},
  \citenamefont {Mujid}, \citenamefont {Rai}, \citenamefont {Eriksson},
  \citenamefont {Suh}, \citenamefont {Poddar}, \citenamefont {Ray},
  \citenamefont {Park}, \citenamefont {Fransson}, \citenamefont {Zhong},
  \citenamefont {Muller}, \citenamefont {Erhart}, \citenamefont {Cahill},\ and\
  \citenamefont {Park}}]{Kim2021}%
  \BibitemOpen
  \bibfield  {author} {\bibinfo {author} {\bibfnamefont {S.~E.}\ \bibnamefont
  {Kim}}, \bibinfo {author} {\bibfnamefont {F.}~\bibnamefont {Mujid}}, \bibinfo
  {author} {\bibfnamefont {A.}~\bibnamefont {Rai}}, \bibinfo {author}
  {\bibfnamefont {F.}~\bibnamefont {Eriksson}}, \bibinfo {author}
  {\bibfnamefont {J.}~\bibnamefont {Suh}}, \bibinfo {author} {\bibfnamefont
  {P.}~\bibnamefont {Poddar}}, \bibinfo {author} {\bibfnamefont
  {A.}~\bibnamefont {Ray}}, \bibinfo {author} {\bibfnamefont {C.}~\bibnamefont
  {Park}}, \bibinfo {author} {\bibfnamefont {E.}~\bibnamefont {Fransson}},
  \bibinfo {author} {\bibfnamefont {Y.}~\bibnamefont {Zhong}}, \bibinfo
  {author} {\bibfnamefont {D.~A.}\ \bibnamefont {Muller}}, \bibinfo {author}
  {\bibfnamefont {P.}~\bibnamefont {Erhart}}, \bibinfo {author} {\bibfnamefont
  {D.~G.}\ \bibnamefont {Cahill}},\ and\ \bibinfo {author} {\bibfnamefont
  {J.}~\bibnamefont {Park}},\ }\href
  {https://doi.org/10.1038/s41586-021-03867-8} {\bibfield  {journal} {\bibinfo
  {journal} {Nature}\ }\textbf {\bibinfo {volume} {597}},\ \bibinfo {pages}
  {660} (\bibinfo {year} {2021})}\BibitemShut {NoStop}%
\bibitem [{\citenamefont {Nakamura}(1975)}]{Nakamura1975}%
  \BibitemOpen
  \bibfield  {author} {\bibinfo {author} {\bibfnamefont {T.}~\bibnamefont
  {Nakamura}},\ }\href {https://doi.org/10.1080/00150197508237719} {\bibfield
  {journal} {\bibinfo  {journal} {Ferroelectrics}\ }\textbf {\bibinfo {volume}
  {9}},\ \bibinfo {pages} {159} (\bibinfo {year} {1975})}\BibitemShut {NoStop}%
\bibitem [{\citenamefont {Nakamura}(1992)}]{Nakamura1992}%
  \BibitemOpen
  \bibfield  {author} {\bibinfo {author} {\bibfnamefont {T.}~\bibnamefont
  {Nakamura}},\ }\href {https://doi.org/10.1080/00150199208015939} {\bibfield
  {journal} {\bibinfo  {journal} {Ferroelectrics}\ }\textbf {\bibinfo {volume}
  {137}},\ \bibinfo {pages} {65} (\bibinfo {year} {1992})}\BibitemShut
  {NoStop}%
\bibitem [{\citenamefont {Dove}(1997)}]{Dove1997}%
  \BibitemOpen
  \bibfield  {author} {\bibinfo {author} {\bibfnamefont {M.~T.}\ \bibnamefont
  {Dove}},\ }\href {https://doi.org/10.2138/am-1997-3-401} {\bibfield
  {journal} {\bibinfo  {journal} {American Mineralogist}\ }\textbf {\bibinfo
  {volume} {82}},\ \bibinfo {pages} {213} (\bibinfo {year} {1997})}\BibitemShut
  {NoStop}%
\bibitem [{\citenamefont {Ehsan}\ \emph {et~al.}(2021)\citenamefont {Ehsan},
  \citenamefont {Arrigoni}, \citenamefont {Madsen}, \citenamefont {Blaha},\
  and\ \citenamefont {Tr\"{o}ster}}]{Ehsan2021}%
  \BibitemOpen
  \bibfield  {author} {\bibinfo {author} {\bibfnamefont {S.}~\bibnamefont
  {Ehsan}}, \bibinfo {author} {\bibfnamefont {M.}~\bibnamefont {Arrigoni}},
  \bibinfo {author} {\bibfnamefont {G.~K.~H.}\ \bibnamefont {Madsen}}, \bibinfo
  {author} {\bibfnamefont {P.}~\bibnamefont {Blaha}},\ and\ \bibinfo {author}
  {\bibfnamefont {A.}~\bibnamefont {Tr\"{o}ster}},\ }\bibfield  {journal}
  {\bibinfo  {journal} {Physical Review B}\ }\textbf {\bibinfo {volume}
  {103}},\ \href {https://doi.org/10.1103/physrevb.103.094108}
  {10.1103/physrevb.103.094108} (\bibinfo {year} {2021})\BibitemShut {NoStop}%
\bibitem [{\citenamefont {Songvilay}\ \emph {et~al.}(2019)\citenamefont
  {Songvilay}, \citenamefont {Giles-Donovan}, \citenamefont {Bari},
  \citenamefont {Ye}, \citenamefont {Minns}, \citenamefont {Green},
  \citenamefont {Xu}, \citenamefont {Gehring}, \citenamefont {Schmalzl},
  \citenamefont {Ratcliff}, \citenamefont {Brown}, \citenamefont {Chernyshov},
  \citenamefont {van Beek}, \citenamefont {Cochran},\ and\ \citenamefont
  {Stock}}]{Songvilay2019}%
  \BibitemOpen
  \bibfield  {author} {\bibinfo {author} {\bibfnamefont {M.}~\bibnamefont
  {Songvilay}}, \bibinfo {author} {\bibfnamefont {N.}~\bibnamefont
  {Giles-Donovan}}, \bibinfo {author} {\bibfnamefont {M.}~\bibnamefont {Bari}},
  \bibinfo {author} {\bibfnamefont {Z.-G.}\ \bibnamefont {Ye}}, \bibinfo
  {author} {\bibfnamefont {J.~L.}\ \bibnamefont {Minns}}, \bibinfo {author}
  {\bibfnamefont {M.~A.}\ \bibnamefont {Green}}, \bibinfo {author}
  {\bibfnamefont {G.}~\bibnamefont {Xu}}, \bibinfo {author} {\bibfnamefont
  {P.~M.}\ \bibnamefont {Gehring}}, \bibinfo {author} {\bibfnamefont
  {K.}~\bibnamefont {Schmalzl}}, \bibinfo {author} {\bibfnamefont {W.~D.}\
  \bibnamefont {Ratcliff}}, \bibinfo {author} {\bibfnamefont {C.~M.}\
  \bibnamefont {Brown}}, \bibinfo {author} {\bibfnamefont {D.}~\bibnamefont
  {Chernyshov}}, \bibinfo {author} {\bibfnamefont {W.}~\bibnamefont {van
  Beek}}, \bibinfo {author} {\bibfnamefont {S.}~\bibnamefont {Cochran}},\ and\
  \bibinfo {author} {\bibfnamefont {C.}~\bibnamefont {Stock}},\ }\href
  {https://doi.org/10.1103/PhysRevMaterials.3.093602} {\bibfield  {journal}
  {\bibinfo  {journal} {Phys. Rev. Materials}\ }\textbf {\bibinfo {volume}
  {3}},\ \bibinfo {pages} {093602} (\bibinfo {year} {2019})}\BibitemShut
  {NoStop}%
\bibitem [{\citenamefont {Lanigan-Atkins}\ \emph {et~al.}(2021)\citenamefont
  {Lanigan-Atkins}, \citenamefont {He}, \citenamefont {Krogstad}, \citenamefont
  {Pajerowski}, \citenamefont {Abernathy}, \citenamefont {Xu}, \citenamefont
  {Xu}, \citenamefont {Chung}, \citenamefont {Kanatzidis}, \citenamefont
  {Rosenkranz}, \citenamefont {Osborn},\ and\ \citenamefont
  {Delaire}}]{Lanigan2021}%
  \BibitemOpen
  \bibfield  {author} {\bibinfo {author} {\bibfnamefont {T.}~\bibnamefont
  {Lanigan-Atkins}}, \bibinfo {author} {\bibfnamefont {X.}~\bibnamefont {He}},
  \bibinfo {author} {\bibfnamefont {M.}~\bibnamefont {Krogstad}}, \bibinfo
  {author} {\bibfnamefont {D.}~\bibnamefont {Pajerowski}}, \bibinfo {author}
  {\bibfnamefont {D.}~\bibnamefont {Abernathy}}, \bibinfo {author}
  {\bibfnamefont {G.~N.}\ \bibnamefont {Xu}}, \bibinfo {author} {\bibfnamefont
  {Z.}~\bibnamefont {Xu}}, \bibinfo {author} {\bibfnamefont {D.-Y.}\
  \bibnamefont {Chung}}, \bibinfo {author} {\bibfnamefont {M.}~\bibnamefont
  {Kanatzidis}}, \bibinfo {author} {\bibfnamefont {S.}~\bibnamefont
  {Rosenkranz}}, \bibinfo {author} {\bibfnamefont {R.}~\bibnamefont {Osborn}},\
  and\ \bibinfo {author} {\bibfnamefont {O.}~\bibnamefont {Delaire}},\
  }\href@noop {} {\bibfield  {journal} {\bibinfo  {journal} {Nature Materials}\
  }\textbf {\bibinfo {volume} {20}},\ \bibinfo {pages} {977} (\bibinfo {year}
  {2021})}\BibitemShut {NoStop}%
\bibitem [{\citenamefont {Stoumpos}\ \emph {et~al.}(2013)\citenamefont
  {Stoumpos}, \citenamefont {Malliakas}, \citenamefont {Peters}, \citenamefont
  {Liu}, \citenamefont {Sebastian}, \citenamefont {Im}, \citenamefont
  {Chasapis}, \citenamefont {Wibowo}, \citenamefont {Chung}, \citenamefont
  {Freeman}, \citenamefont {Wessels},\ and\ \citenamefont
  {Kanatzidis}}]{Stoumpos2013}%
  \BibitemOpen
  \bibfield  {author} {\bibinfo {author} {\bibfnamefont {C.~C.}\ \bibnamefont
  {Stoumpos}}, \bibinfo {author} {\bibfnamefont {C.~D.}\ \bibnamefont
  {Malliakas}}, \bibinfo {author} {\bibfnamefont {J.~A.}\ \bibnamefont
  {Peters}}, \bibinfo {author} {\bibfnamefont {Z.}~\bibnamefont {Liu}},
  \bibinfo {author} {\bibfnamefont {M.}~\bibnamefont {Sebastian}}, \bibinfo
  {author} {\bibfnamefont {J.}~\bibnamefont {Im}}, \bibinfo {author}
  {\bibfnamefont {T.~C.}\ \bibnamefont {Chasapis}}, \bibinfo {author}
  {\bibfnamefont {A.~C.}\ \bibnamefont {Wibowo}}, \bibinfo {author}
  {\bibfnamefont {D.~Y.}\ \bibnamefont {Chung}}, \bibinfo {author}
  {\bibfnamefont {A.~J.}\ \bibnamefont {Freeman}}, \bibinfo {author}
  {\bibfnamefont {B.~W.}\ \bibnamefont {Wessels}},\ and\ \bibinfo {author}
  {\bibfnamefont {M.~G.}\ \bibnamefont {Kanatzidis}},\ }\href
  {https://doi.org/10.1021/cg400645t} {\bibfield  {journal} {\bibinfo
  {journal} {Crystal Growth {\&} Design}\ }\textbf {\bibinfo {volume} {13}},\
  \bibinfo {pages} {2722} (\bibinfo {year} {2013})}\BibitemShut {NoStop}%
\bibitem [{\citenamefont {Hirotsu}\ \emph {et~al.}(1974)\citenamefont
  {Hirotsu}, \citenamefont {Harada}, \citenamefont {Iizumi},\ and\
  \citenamefont {Gesi}}]{Hirotsu1974}%
  \BibitemOpen
  \bibfield  {author} {\bibinfo {author} {\bibfnamefont {S.}~\bibnamefont
  {Hirotsu}}, \bibinfo {author} {\bibfnamefont {J.}~\bibnamefont {Harada}},
  \bibinfo {author} {\bibfnamefont {M.}~\bibnamefont {Iizumi}},\ and\ \bibinfo
  {author} {\bibfnamefont {K.}~\bibnamefont {Gesi}},\ }\href
  {https://doi.org/10.1143/JPSJ.37.1393} {\bibfield  {journal} {\bibinfo
  {journal} {Journal of the Physical Society of Japan}\ }\textbf {\bibinfo
  {volume} {37}},\ \bibinfo {pages} {1393} (\bibinfo {year}
  {1974})}\BibitemShut {NoStop}%
\bibitem [{\citenamefont {Sharma}\ \emph {et~al.}(1991)\citenamefont {Sharma},
  \citenamefont {Weiden},\ and\ \citenamefont {Weiss}}]{Sharma1991}%
  \BibitemOpen
  \bibfield  {author} {\bibinfo {author} {\bibfnamefont {S.}~\bibnamefont
  {Sharma}}, \bibinfo {author} {\bibfnamefont {N.}~\bibnamefont {Weiden}},\
  and\ \bibinfo {author} {\bibfnamefont {A.}~\bibnamefont {Weiss}},\ }\href
  {https://doi.org/10.1515/zna-1991-0406} {\bibfield  {journal} {\bibinfo
  {journal} {Zeitschrift f\"{u}r Naturforschung A}\ }\textbf {\bibinfo {volume}
  {46}},\ \bibinfo {pages} {329} (\bibinfo {year} {1991})}\BibitemShut
  {NoStop}%
\bibitem [{\citenamefont {Rodov{\'{a}}}\ \emph {et~al.}(2003)\citenamefont
  {Rodov{\'{a}}}, \citenamefont {Bro{\v{z}}ek}, \citenamefont
  {Kn{\'{\i}}{\v{z}}ek},\ and\ \citenamefont {Nitsch}}]{Rodov2003}%
  \BibitemOpen
  \bibfield  {author} {\bibinfo {author} {\bibfnamefont {M.}~\bibnamefont
  {Rodov{\'{a}}}}, \bibinfo {author} {\bibfnamefont {J.}~\bibnamefont
  {Bro{\v{z}}ek}}, \bibinfo {author} {\bibfnamefont {K.}~\bibnamefont
  {Kn{\'{\i}}{\v{z}}ek}},\ and\ \bibinfo {author} {\bibfnamefont
  {K.}~\bibnamefont {Nitsch}},\ }\href
  {https://doi.org/10.1023/a:1022836800820} {\bibfield  {journal} {\bibinfo
  {journal} {Journal of Thermal Analysis and Calorimetry}\ }\textbf {\bibinfo
  {volume} {71}},\ \bibinfo {pages} {667} (\bibinfo {year} {2003})}\BibitemShut
  {NoStop}%
\bibitem [{\citenamefont {L{\'{o}}pez}\ \emph {et~al.}(2020)\citenamefont
  {L{\'{o}}pez}, \citenamefont {Abia}, \citenamefont {Alvarez-Galv{\'{a}}n},
  \citenamefont {Hong}, \citenamefont {Mart{\'{\i}}nez-Huerta}, \citenamefont
  {Serrano-S{\'{a}}nchez}, \citenamefont {Carrascoso}, \citenamefont
  {Castellanos-G{\'{o}}mez}, \citenamefont {Fern{\'{a}}ndez-D{\'{\i}}az},\ and\
  \citenamefont {Alonso}}]{Lopez2020}%
  \BibitemOpen
  \bibfield  {author} {\bibinfo {author} {\bibfnamefont {C.~A.}\ \bibnamefont
  {L{\'{o}}pez}}, \bibinfo {author} {\bibfnamefont {C.}~\bibnamefont {Abia}},
  \bibinfo {author} {\bibfnamefont {M.~C.}\ \bibnamefont
  {Alvarez-Galv{\'{a}}n}}, \bibinfo {author} {\bibfnamefont {B.-K.}\
  \bibnamefont {Hong}}, \bibinfo {author} {\bibfnamefont {M.~V.}\ \bibnamefont
  {Mart{\'{\i}}nez-Huerta}}, \bibinfo {author} {\bibfnamefont {F.}~\bibnamefont
  {Serrano-S{\'{a}}nchez}}, \bibinfo {author} {\bibfnamefont {F.}~\bibnamefont
  {Carrascoso}}, \bibinfo {author} {\bibfnamefont {A.}~\bibnamefont
  {Castellanos-G{\'{o}}mez}}, \bibinfo {author} {\bibfnamefont {M.~T.}\
  \bibnamefont {Fern{\'{a}}ndez-D{\'{\i}}az}},\ and\ \bibinfo {author}
  {\bibfnamefont {J.~A.}\ \bibnamefont {Alonso}},\ }\href
  {https://doi.org/10.1021/acsomega.9b04248} {\bibfield  {journal} {\bibinfo
  {journal} {{ACS} Omega}\ }\textbf {\bibinfo {volume} {5}},\ \bibinfo {pages}
  {5931} (\bibinfo {year} {2020})}\BibitemShut {NoStop}%
\bibitem [{\citenamefont {Malyshkin}\ \emph {et~al.}(2020)\citenamefont
  {Malyshkin}, \citenamefont {Sereda}, \citenamefont {Ivanov}, \citenamefont
  {Mazurin}, \citenamefont {Sednev-Lugovets}, \citenamefont {Tsvetkov},\ and\
  \citenamefont {Zuev}}]{Malyshkin2020}%
  \BibitemOpen
  \bibfield  {author} {\bibinfo {author} {\bibfnamefont {D.}~\bibnamefont
  {Malyshkin}}, \bibinfo {author} {\bibfnamefont {V.}~\bibnamefont {Sereda}},
  \bibinfo {author} {\bibfnamefont {I.}~\bibnamefont {Ivanov}}, \bibinfo
  {author} {\bibfnamefont {M.}~\bibnamefont {Mazurin}}, \bibinfo {author}
  {\bibfnamefont {A.}~\bibnamefont {Sednev-Lugovets}}, \bibinfo {author}
  {\bibfnamefont {D.}~\bibnamefont {Tsvetkov}},\ and\ \bibinfo {author}
  {\bibfnamefont {A.}~\bibnamefont {Zuev}},\ }\href
  {https://doi.org/10.1016/j.matlet.2020.128458} {\bibfield  {journal}
  {\bibinfo  {journal} {Materials Letters}\ }\textbf {\bibinfo {volume}
  {278}},\ \bibinfo {pages} {128458} (\bibinfo {year} {2020})}\BibitemShut
  {NoStop}%
\bibitem [{\citenamefont {Huang}\ and\ \citenamefont
  {Lambrecht}(2014)}]{Huang2014}%
  \BibitemOpen
  \bibfield  {author} {\bibinfo {author} {\bibfnamefont {L.-y.}\ \bibnamefont
  {Huang}}\ and\ \bibinfo {author} {\bibfnamefont {W.~R.~L.}\ \bibnamefont
  {Lambrecht}},\ }\href {https://doi.org/10.1103/PhysRevB.90.195201} {\bibfield
   {journal} {\bibinfo  {journal} {Phys. Rev. B}\ }\textbf {\bibinfo {volume}
  {90}},\ \bibinfo {pages} {195201} (\bibinfo {year} {2014})}\BibitemShut
  {NoStop}%
\bibitem [{\citenamefont {da~Silva}\ \emph {et~al.}(2015)\citenamefont
  {da~Silva}, \citenamefont {Skelton}, \citenamefont {Parker},\ and\
  \citenamefont {Walsh}}]{daSilva2015}%
  \BibitemOpen
  \bibfield  {author} {\bibinfo {author} {\bibfnamefont {E.~L.}\ \bibnamefont
  {da~Silva}}, \bibinfo {author} {\bibfnamefont {J.~M.}\ \bibnamefont
  {Skelton}}, \bibinfo {author} {\bibfnamefont {S.~C.}\ \bibnamefont
  {Parker}},\ and\ \bibinfo {author} {\bibfnamefont {A.}~\bibnamefont
  {Walsh}},\ }\bibfield  {journal} {\bibinfo  {journal} {Physical Review B}\
  }\textbf {\bibinfo {volume} {91}},\ \href
  {https://doi.org/10.1103/physrevb.91.144107} {10.1103/physrevb.91.144107}
  (\bibinfo {year} {2015})\BibitemShut {NoStop}%
\bibitem [{\citenamefont {Yang}\ \emph {et~al.}(2017)\citenamefont {Yang},
  \citenamefont {Skelton}, \citenamefont {da~Silva}, \citenamefont {Frost},\
  and\ \citenamefont {Walsh}}]{Yang2017}%
  \BibitemOpen
  \bibfield  {author} {\bibinfo {author} {\bibfnamefont {R.~X.}\ \bibnamefont
  {Yang}}, \bibinfo {author} {\bibfnamefont {J.~M.}\ \bibnamefont {Skelton}},
  \bibinfo {author} {\bibfnamefont {E.~L.}\ \bibnamefont {da~Silva}}, \bibinfo
  {author} {\bibfnamefont {J.~M.}\ \bibnamefont {Frost}},\ and\ \bibinfo
  {author} {\bibfnamefont {A.}~\bibnamefont {Walsh}},\ }\href
  {https://doi.org/10.1021/acs.jpclett.7b02423} {\bibfield  {journal} {\bibinfo
   {journal} {The Journal of Physical Chemistry Letters}\ }\textbf {\bibinfo
  {volume} {8}},\ \bibinfo {pages} {4720} (\bibinfo {year} {2017})}\BibitemShut
  {NoStop}%
\bibitem [{\citenamefont {Yang}\ \emph {et~al.}(2020)\citenamefont {Yang},
  \citenamefont {Skelton}, \citenamefont {da~Silva}, \citenamefont {Frost},\
  and\ \citenamefont {Walsh}}]{Yang2020}%
  \BibitemOpen
  \bibfield  {author} {\bibinfo {author} {\bibfnamefont {R.~X.}\ \bibnamefont
  {Yang}}, \bibinfo {author} {\bibfnamefont {J.~M.}\ \bibnamefont {Skelton}},
  \bibinfo {author} {\bibfnamefont {E.~L.}\ \bibnamefont {da~Silva}}, \bibinfo
  {author} {\bibfnamefont {J.~M.}\ \bibnamefont {Frost}},\ and\ \bibinfo
  {author} {\bibfnamefont {A.}~\bibnamefont {Walsh}},\ }\href
  {https://doi.org/10.1063/1.5131575} {\bibfield  {journal} {\bibinfo
  {journal} {The Journal of Chemical Physics}\ }\textbf {\bibinfo {volume}
  {152}},\ \bibinfo {pages} {024703} (\bibinfo {year} {2020})}\BibitemShut
  {NoStop}%
\bibitem [{\citenamefont {Cohen}\ \emph {et~al.}(2022)\citenamefont {Cohen},
  \citenamefont {Brenner}, \citenamefont {Klarbring}, \citenamefont {Sharma},
  \citenamefont {Fabini}, \citenamefont {Korobko}, \citenamefont {Nayak},
  \citenamefont {Hellman},\ and\ \citenamefont {Yaffe}}]{Cohen2022}%
  \BibitemOpen
  \bibfield  {author} {\bibinfo {author} {\bibfnamefont {A.}~\bibnamefont
  {Cohen}}, \bibinfo {author} {\bibfnamefont {T.~M.}\ \bibnamefont {Brenner}},
  \bibinfo {author} {\bibfnamefont {J.}~\bibnamefont {Klarbring}}, \bibinfo
  {author} {\bibfnamefont {R.}~\bibnamefont {Sharma}}, \bibinfo {author}
  {\bibfnamefont {D.~H.}\ \bibnamefont {Fabini}}, \bibinfo {author}
  {\bibfnamefont {R.}~\bibnamefont {Korobko}}, \bibinfo {author} {\bibfnamefont
  {P.~K.}\ \bibnamefont {Nayak}}, \bibinfo {author} {\bibfnamefont
  {O.}~\bibnamefont {Hellman}},\ and\ \bibinfo {author} {\bibfnamefont
  {O.}~\bibnamefont {Yaffe}},\ }\href {https://doi.org/10.1002/adma.202107932}
  {\bibfield  {journal} {\bibinfo  {journal} {Advanced Materials}\ }\textbf
  {\bibinfo {volume} {34}},\ \bibinfo {pages} {2107932} (\bibinfo {year}
  {2022})}\BibitemShut {NoStop}%
\bibitem [{\citenamefont {Zhu}\ \emph {et~al.}(2022)\citenamefont {Zhu},
  \citenamefont {Caicedo-D{\'a}vila}, \citenamefont {Gehrmann},\ and\
  \citenamefont {Egger}}]{ZhuEgger2022}%
  \BibitemOpen
  \bibfield  {author} {\bibinfo {author} {\bibfnamefont {X.}~\bibnamefont
  {Zhu}}, \bibinfo {author} {\bibfnamefont {S.}~\bibnamefont
  {Caicedo-D{\'a}vila}}, \bibinfo {author} {\bibfnamefont {C.}~\bibnamefont
  {Gehrmann}},\ and\ \bibinfo {author} {\bibfnamefont {D.~A.}\ \bibnamefont
  {Egger}},\ }\href@noop {} {\bibfield  {journal} {\bibinfo  {journal} {ACS
  Applied Materials \& Interfaces}\ }\textbf {\bibinfo {volume} {14}},\
  \bibinfo {pages} {22973} (\bibinfo {year} {2022})}\BibitemShut {NoStop}%
\bibitem [{\citenamefont {Lahnsteiner}\ and\ \citenamefont
  {Bokdam}(2022)}]{Lahnsteiner2022}%
  \BibitemOpen
  \bibfield  {author} {\bibinfo {author} {\bibfnamefont {J.}~\bibnamefont
  {Lahnsteiner}}\ and\ \bibinfo {author} {\bibfnamefont {M.}~\bibnamefont
  {Bokdam}},\ }\href {https://doi.org/10.1103/PhysRevB.105.024302} {\bibfield
  {journal} {\bibinfo  {journal} {Phys. Rev. B}\ }\textbf {\bibinfo {volume}
  {105}},\ \bibinfo {pages} {024302} (\bibinfo {year} {2022})}\BibitemShut
  {NoStop}%
\bibitem [{\citenamefont {Tadano}\ and\ \citenamefont
  {Saidi}(2022)}]{TadWis2022}%
  \BibitemOpen
  \bibfield  {author} {\bibinfo {author} {\bibfnamefont {T.}~\bibnamefont
  {Tadano}}\ and\ \bibinfo {author} {\bibfnamefont {W.~A.}\ \bibnamefont
  {Saidi}},\ }\href {https://doi.org/10.1103/PhysRevLett.129.185901} {\bibfield
   {journal} {\bibinfo  {journal} {Phys. Rev. Lett.}\ }\textbf {\bibinfo
  {volume} {129}},\ \bibinfo {pages} {185901} (\bibinfo {year}
  {2022})}\BibitemShut {NoStop}%
\bibitem [{\citenamefont {Stukowski}(2009)}]{Stukowski2010}%
  \BibitemOpen
  \bibfield  {author} {\bibinfo {author} {\bibfnamefont {A.}~\bibnamefont
  {Stukowski}},\ }\href {https://doi.org/10.1088/0965-0393/18/1/015012}
  {\bibfield  {journal} {\bibinfo  {journal} {Modelling and Simulation in
  Materials Science and Engineering}\ }\textbf {\bibinfo {volume} {18}},\
  \bibinfo {pages} {015012} (\bibinfo {year} {2009})}\BibitemShut {NoStop}%
\bibitem [{Note1()}]{Note1}%
  \BibitemOpen
  \bibinfo {note} {See Supplemental Material at [URL will be inserted by
  publisher] for additional figures and further details.}\BibitemShut {Stop}%
\bibitem [{\citenamefont {Fan}\ \emph {et~al.}(2021)\citenamefont {Fan},
  \citenamefont {Zeng}, \citenamefont {Zhang}, \citenamefont {Wang},
  \citenamefont {Song}, \citenamefont {Dong}, \citenamefont {Chen},\ and\
  \citenamefont {Ala-Nissila}}]{FanZenZha21}%
  \BibitemOpen
  \bibfield  {author} {\bibinfo {author} {\bibfnamefont {Z.}~\bibnamefont
  {Fan}}, \bibinfo {author} {\bibfnamefont {Z.}~\bibnamefont {Zeng}}, \bibinfo
  {author} {\bibfnamefont {C.}~\bibnamefont {Zhang}}, \bibinfo {author}
  {\bibfnamefont {Y.}~\bibnamefont {Wang}}, \bibinfo {author} {\bibfnamefont
  {K.}~\bibnamefont {Song}}, \bibinfo {author} {\bibfnamefont {H.}~\bibnamefont
  {Dong}}, \bibinfo {author} {\bibfnamefont {Y.}~\bibnamefont {Chen}},\ and\
  \bibinfo {author} {\bibfnamefont {T.}~\bibnamefont {Ala-Nissila}},\ }\href
  {https://doi.org/10.1103/PhysRevB.104.104309} {\bibfield  {journal} {\bibinfo
   {journal} {Physical Review B}\ }\textbf {\bibinfo {volume} {104}},\ \bibinfo
  {pages} {104309} (\bibinfo {year} {2021})}\BibitemShut {NoStop}%
\bibitem [{\citenamefont {Fan}\ \emph {et~al.}(2022)\citenamefont {Fan},
  \citenamefont {Wang}, \citenamefont {Ying}, \citenamefont {Song},
  \citenamefont {Wang}, \citenamefont {Wang}, \citenamefont {Zeng},
  \citenamefont {Xu}, \citenamefont {Lindgren}, \citenamefont {Rahm},
  \citenamefont {Gabourie}, \citenamefont {Liu}, \citenamefont {Dong},
  \citenamefont {Wu}, \citenamefont {Chen}, \citenamefont {Zhong},
  \citenamefont {Sun}, \citenamefont {Erhart}, \citenamefont {Su},\ and\
  \citenamefont {Ala-Nissila}}]{FanWanYin22}%
  \BibitemOpen
  \bibfield  {author} {\bibinfo {author} {\bibfnamefont {Z.}~\bibnamefont
  {Fan}}, \bibinfo {author} {\bibfnamefont {Y.}~\bibnamefont {Wang}}, \bibinfo
  {author} {\bibfnamefont {P.}~\bibnamefont {Ying}}, \bibinfo {author}
  {\bibfnamefont {K.}~\bibnamefont {Song}}, \bibinfo {author} {\bibfnamefont
  {J.}~\bibnamefont {Wang}}, \bibinfo {author} {\bibfnamefont {Y.}~\bibnamefont
  {Wang}}, \bibinfo {author} {\bibfnamefont {Z.}~\bibnamefont {Zeng}}, \bibinfo
  {author} {\bibfnamefont {K.}~\bibnamefont {Xu}}, \bibinfo {author}
  {\bibfnamefont {E.}~\bibnamefont {Lindgren}}, \bibinfo {author}
  {\bibfnamefont {J.~M.}\ \bibnamefont {Rahm}}, \bibinfo {author}
  {\bibfnamefont {A.~J.}\ \bibnamefont {Gabourie}}, \bibinfo {author}
  {\bibfnamefont {J.}~\bibnamefont {Liu}}, \bibinfo {author} {\bibfnamefont
  {H.}~\bibnamefont {Dong}}, \bibinfo {author} {\bibfnamefont {J.}~\bibnamefont
  {Wu}}, \bibinfo {author} {\bibfnamefont {Y.}~\bibnamefont {Chen}}, \bibinfo
  {author} {\bibfnamefont {Z.}~\bibnamefont {Zhong}}, \bibinfo {author}
  {\bibfnamefont {J.}~\bibnamefont {Sun}}, \bibinfo {author} {\bibfnamefont
  {P.}~\bibnamefont {Erhart}}, \bibinfo {author} {\bibfnamefont
  {Y.}~\bibnamefont {Su}},\ and\ \bibinfo {author} {\bibfnamefont
  {T.}~\bibnamefont {Ala-Nissila}},\ }\href {https://doi.org/10.1063/5.0106617}
  {\bibfield  {journal} {\bibinfo  {journal} {Journal of Chemical Physics}\
  }\textbf {\bibinfo {volume} {157}},\ \bibinfo {pages} {114801} (\bibinfo
  {year} {2022})}\BibitemShut {NoStop}%
\bibitem [{Note2()}]{Note2}%
  \BibitemOpen
  \bibinfo {note} {The \gls {dft} data and the \gls {mlp} models are provided
  in a zenodo dataset \cite {zenodo_dataset}.}\BibitemShut {Stop}%
\bibitem [{\citenamefont {Kresse}\ and\ \citenamefont
  {Hafner}(1993)}]{KreHaf93}%
  \BibitemOpen
  \bibfield  {author} {\bibinfo {author} {\bibfnamefont {G.}~\bibnamefont
  {Kresse}}\ and\ \bibinfo {author} {\bibfnamefont {J.}~\bibnamefont
  {Hafner}},\ }\href {https://doi.org/10.1103/PhysRevB.47.558} {\bibfield
  {journal} {\bibinfo  {journal} {Physical Review B}\ }\textbf {\bibinfo
  {volume} {47}},\ \bibinfo {pages} {558} (\bibinfo {year} {1993})}\BibitemShut
  {NoStop}%
\bibitem [{\citenamefont {Bl\"ochl}(1994)}]{Blo94}%
  \BibitemOpen
  \bibfield  {author} {\bibinfo {author} {\bibfnamefont {P.~E.}\ \bibnamefont
  {Bl\"ochl}},\ }\href {https://doi.org/10.1103/PhysRevB.50.17953} {\bibfield
  {journal} {\bibinfo  {journal} {Physical Review B}\ }\textbf {\bibinfo
  {volume} {50}},\ \bibinfo {pages} {17953} (\bibinfo {year}
  {1994})}\BibitemShut {NoStop}%
\bibitem [{\citenamefont {Kresse}\ and\ \citenamefont
  {Furthm\"uller}(1996)}]{KreFur96}%
  \BibitemOpen
  \bibfield  {author} {\bibinfo {author} {\bibfnamefont {G.}~\bibnamefont
  {Kresse}}\ and\ \bibinfo {author} {\bibfnamefont {J.}~\bibnamefont
  {Furthm\"uller}},\ }\href {https://doi.org/10.1016/0927-0256(96)00008-0}
  {\bibfield  {journal} {\bibinfo  {journal} {Computational Materials Science}\
  }\textbf {\bibinfo {volume} {6}},\ \bibinfo {pages} {15} (\bibinfo {year}
  {1996})}\BibitemShut {NoStop}%
\bibitem [{\citenamefont {Sun}\ \emph {et~al.}(2015)\citenamefont {Sun},
  \citenamefont {Ruzsinszky},\ and\ \citenamefont {Perdew}}]{SunRuzPer15}%
  \BibitemOpen
  \bibfield  {author} {\bibinfo {author} {\bibfnamefont {J.}~\bibnamefont
  {Sun}}, \bibinfo {author} {\bibfnamefont {A.}~\bibnamefont {Ruzsinszky}},\
  and\ \bibinfo {author} {\bibfnamefont {J.~P.}\ \bibnamefont {Perdew}},\
  }\href {https://doi.org/10.1103/PhysRevLett.115.036402} {\bibfield  {journal}
  {\bibinfo  {journal} {Physical Review Letters}\ }\textbf {\bibinfo {volume}
  {115}},\ \bibinfo {pages} {036402} (\bibinfo {year} {2015})}\BibitemShut
  {NoStop}%
\bibitem [{\citenamefont {Larsen}\ \emph {et~al.}(2017)\citenamefont {Larsen},
  \citenamefont {Mortensen}, \citenamefont {Blomqvist}, \citenamefont
  {Castelli}, \citenamefont {Christensen}, \citenamefont {Dułak},
  \citenamefont {Friis}, \citenamefont {Groves}, \citenamefont {Hammer},
  \citenamefont {Hargus}, \citenamefont {Hermes}, \citenamefont {Jennings},
  \citenamefont {Jensen}, \citenamefont {Kermode}, \citenamefont {Kitchin},
  \citenamefont {Kolsbjerg}, \citenamefont {Kubal}, \citenamefont {Kaasbjerg},
  \citenamefont {Lysgaard}, \citenamefont {Maronsson}, \citenamefont {Maxson},
  \citenamefont {Olsen}, \citenamefont {Pastewka}, \citenamefont {Peterson},
  \citenamefont {Rostgaard}, \citenamefont {Schiøtz}, \citenamefont {Schütt},
  \citenamefont {Strange}, \citenamefont {Thygesen}, \citenamefont {Vegge},
  \citenamefont {Vilhelmsen}, \citenamefont {Walter}, \citenamefont {Zeng},\
  and\ \citenamefont {Jacobsen}}]{Larsen2017}%
  \BibitemOpen
  \bibfield  {author} {\bibinfo {author} {\bibfnamefont {A.~H.}\ \bibnamefont
  {Larsen}}, \bibinfo {author} {\bibfnamefont {J.~J.}\ \bibnamefont
  {Mortensen}}, \bibinfo {author} {\bibfnamefont {J.}~\bibnamefont
  {Blomqvist}}, \bibinfo {author} {\bibfnamefont {I.~E.}\ \bibnamefont
  {Castelli}}, \bibinfo {author} {\bibfnamefont {R.}~\bibnamefont
  {Christensen}}, \bibinfo {author} {\bibfnamefont {M.}~\bibnamefont {Dułak}},
  \bibinfo {author} {\bibfnamefont {J.}~\bibnamefont {Friis}}, \bibinfo
  {author} {\bibfnamefont {M.~N.}\ \bibnamefont {Groves}}, \bibinfo {author}
  {\bibfnamefont {B.}~\bibnamefont {Hammer}}, \bibinfo {author} {\bibfnamefont
  {C.}~\bibnamefont {Hargus}}, \bibinfo {author} {\bibfnamefont {E.~D.}\
  \bibnamefont {Hermes}}, \bibinfo {author} {\bibfnamefont {P.~C.}\
  \bibnamefont {Jennings}}, \bibinfo {author} {\bibfnamefont {P.~B.}\
  \bibnamefont {Jensen}}, \bibinfo {author} {\bibfnamefont {J.}~\bibnamefont
  {Kermode}}, \bibinfo {author} {\bibfnamefont {J.~R.}\ \bibnamefont
  {Kitchin}}, \bibinfo {author} {\bibfnamefont {E.~L.}\ \bibnamefont
  {Kolsbjerg}}, \bibinfo {author} {\bibfnamefont {J.}~\bibnamefont {Kubal}},
  \bibinfo {author} {\bibfnamefont {K.}~\bibnamefont {Kaasbjerg}}, \bibinfo
  {author} {\bibfnamefont {S.}~\bibnamefont {Lysgaard}}, \bibinfo {author}
  {\bibfnamefont {J.~B.}\ \bibnamefont {Maronsson}}, \bibinfo {author}
  {\bibfnamefont {T.}~\bibnamefont {Maxson}}, \bibinfo {author} {\bibfnamefont
  {T.}~\bibnamefont {Olsen}}, \bibinfo {author} {\bibfnamefont
  {L.}~\bibnamefont {Pastewka}}, \bibinfo {author} {\bibfnamefont
  {A.}~\bibnamefont {Peterson}}, \bibinfo {author} {\bibfnamefont
  {C.}~\bibnamefont {Rostgaard}}, \bibinfo {author} {\bibfnamefont
  {J.}~\bibnamefont {Schiøtz}}, \bibinfo {author} {\bibfnamefont
  {O.}~\bibnamefont {Schütt}}, \bibinfo {author} {\bibfnamefont
  {M.}~\bibnamefont {Strange}}, \bibinfo {author} {\bibfnamefont {K.~S.}\
  \bibnamefont {Thygesen}}, \bibinfo {author} {\bibfnamefont {T.}~\bibnamefont
  {Vegge}}, \bibinfo {author} {\bibfnamefont {L.}~\bibnamefont {Vilhelmsen}},
  \bibinfo {author} {\bibfnamefont {M.}~\bibnamefont {Walter}}, \bibinfo
  {author} {\bibfnamefont {Z.}~\bibnamefont {Zeng}},\ and\ \bibinfo {author}
  {\bibfnamefont {K.~W.}\ \bibnamefont {Jacobsen}},\ }\href
  {https://doi.org/10.1088/1361-648X/aa680e} {\bibfield  {journal} {\bibinfo
  {journal} {Journal of Physics: Condensed Matter}\ }\textbf {\bibinfo {volume}
  {29}},\ \bibinfo {pages} {273002} (\bibinfo {year} {2017})}\BibitemShut
  {NoStop}%
\bibitem [{cal(2022)}]{calorine}%
  \BibitemOpen
  \href@noop {} {\bibinfo {title} {\textsc{calorine}}},\ \bibinfo
  {howpublished} {\url{https://gitlab.com/materials-modeling/calorine}}
  (\bibinfo {year} {2022}),\ \bibinfo {note} {accessed: 2022-10-31}\BibitemShut
  {NoStop}%
\bibitem [{\citenamefont {Esfarjani}\ and\ \citenamefont
  {Liang}(2020)}]{Esfarjani2020}%
  \BibitemOpen
  \bibfield  {author} {\bibinfo {author} {\bibfnamefont {K.}~\bibnamefont
  {Esfarjani}}\ and\ \bibinfo {author} {\bibfnamefont {Y.}~\bibnamefont
  {Liang}},\ }in\ \href {https://doi.org/10.1088/978-0-7503-1738-2ch7} {\emph
  {\bibinfo {booktitle} {Nanoscale Energy Transport}}},\ \bibinfo {series and
  number} {2053-2563}\ (\bibinfo  {publisher} {IOP Publishing},\ \bibinfo
  {address} {Bristol England},\ \bibinfo {year} {2020})\ pp.\ \bibinfo {pages}
  {7--1 to 7--35}\BibitemShut {NoStop}%
\bibitem [{\citenamefont {Kong}\ \emph {et~al.}(2009)\citenamefont {Kong},
  \citenamefont {Bartels}, \citenamefont {Campa{\~{n}}{\'{a}}}, \citenamefont
  {Denniston},\ and\ \citenamefont {M\"{u}ser}}]{Kong2009}%
  \BibitemOpen
  \bibfield  {author} {\bibinfo {author} {\bibfnamefont {L.~T.}\ \bibnamefont
  {Kong}}, \bibinfo {author} {\bibfnamefont {G.}~\bibnamefont {Bartels}},
  \bibinfo {author} {\bibfnamefont {C.}~\bibnamefont {Campa{\~{n}}{\'{a}}}},
  \bibinfo {author} {\bibfnamefont {C.}~\bibnamefont {Denniston}},\ and\
  \bibinfo {author} {\bibfnamefont {M.~H.}\ \bibnamefont {M\"{u}ser}},\ }\href
  {https://doi.org/10.1016/j.cpc.2008.12.035} {\bibfield  {journal} {\bibinfo
  {journal} {Computer Physics Communications}\ }\textbf {\bibinfo {volume}
  {180}},\ \bibinfo {pages} {1004} (\bibinfo {year} {2009})}\BibitemShut
  {NoStop}%
\bibitem [{\citenamefont {Kong}(2011)}]{Kong2011}%
  \BibitemOpen
  \bibfield  {author} {\bibinfo {author} {\bibfnamefont {L.~T.}\ \bibnamefont
  {Kong}},\ }\href {https://doi.org/10.1016/j.cpc.2011.04.019} {\bibfield
  {journal} {\bibinfo  {journal} {Computer Physics Communications}\ }\textbf
  {\bibinfo {volume} {182}},\ \bibinfo {pages} {2201} (\bibinfo {year}
  {2011})}\BibitemShut {NoStop}%
\bibitem [{\citenamefont {Andersson}(2012)}]{And12}%
  \BibitemOpen
  \bibfield  {author} {\bibinfo {author} {\bibfnamefont {T.}~\bibnamefont
  {Andersson}},\ }\emph {\bibinfo {title} {One-shot free energy calculations
  for crystalline materials}},\ \href {https://doi.org/20.500.12380/158974}
  {Master's thesis},\ \bibinfo  {school} {Chalmers University of Technology}
  (\bibinfo {year} {2012})\BibitemShut {NoStop}%
\bibitem [{\citenamefont {Hellman}\ \emph {et~al.}(2013)\citenamefont
  {Hellman}, \citenamefont {Steneteg}, \citenamefont {Abrikosov},\ and\
  \citenamefont {Simak}}]{HelSteAbr13}%
  \BibitemOpen
  \bibfield  {author} {\bibinfo {author} {\bibfnamefont {O.}~\bibnamefont
  {Hellman}}, \bibinfo {author} {\bibfnamefont {P.}~\bibnamefont {Steneteg}},
  \bibinfo {author} {\bibfnamefont {I.~A.}\ \bibnamefont {Abrikosov}},\ and\
  \bibinfo {author} {\bibfnamefont {S.~I.}\ \bibnamefont {Simak}},\ }\href
  {https://doi.org/10.1103/PhysRevB.87.104111} {\bibfield  {journal} {\bibinfo
  {journal} {Physical Review B}\ }\textbf {\bibinfo {volume} {87}},\ \bibinfo
  {pages} {104111} (\bibinfo {year} {2013})}\BibitemShut {NoStop}%
\bibitem [{\citenamefont {Eriksson}\ \emph {et~al.}(2019)\citenamefont
  {Eriksson}, \citenamefont {Fransson},\ and\ \citenamefont
  {Erhart}}]{EriFraErh19}%
  \BibitemOpen
  \bibfield  {author} {\bibinfo {author} {\bibfnamefont {F.}~\bibnamefont
  {Eriksson}}, \bibinfo {author} {\bibfnamefont {E.}~\bibnamefont {Fransson}},\
  and\ \bibinfo {author} {\bibfnamefont {P.}~\bibnamefont {Erhart}},\ }\href
  {https://doi.org/10.1002/adts.201800184} {\bibfield  {journal} {\bibinfo
  {journal} {Adv. Theory Simul.}\ }\textbf {\bibinfo {volume} {2}},\ \bibinfo
  {pages} {1800184} (\bibinfo {year} {2019})}\BibitemShut {NoStop}%
\bibitem [{\citenamefont {Tadano}\ \emph {et~al.}(2014)\citenamefont {Tadano},
  \citenamefont {Gohda},\ and\ \citenamefont {Tsuneyuki}}]{TadGohTsu14}%
  \BibitemOpen
  \bibfield  {author} {\bibinfo {author} {\bibfnamefont {T.}~\bibnamefont
  {Tadano}}, \bibinfo {author} {\bibfnamefont {Y.}~\bibnamefont {Gohda}},\ and\
  \bibinfo {author} {\bibfnamefont {S.}~\bibnamefont {Tsuneyuki}},\ }\href
  {https://doi.org/10.1088/0953-8984/26/22/225402} {\bibfield  {journal}
  {\bibinfo  {journal} {Journal of Physics: Condensed Matter}\ }\textbf
  {\bibinfo {volume} {26}},\ \bibinfo {pages} {225402} (\bibinfo {year}
  {2014})}\BibitemShut {NoStop}%
\bibitem [{\citenamefont {Monacelli}\ \emph {et~al.}(2021)\citenamefont
  {Monacelli}, \citenamefont {Bianco}, \citenamefont {Cherubini}, \citenamefont
  {Calandra}, \citenamefont {Errea},\ and\ \citenamefont
  {Mauri}}]{MonBiaChe21}%
  \BibitemOpen
  \bibfield  {author} {\bibinfo {author} {\bibfnamefont {L.}~\bibnamefont
  {Monacelli}}, \bibinfo {author} {\bibfnamefont {R.}~\bibnamefont {Bianco}},
  \bibinfo {author} {\bibfnamefont {M.}~\bibnamefont {Cherubini}}, \bibinfo
  {author} {\bibfnamefont {M.}~\bibnamefont {Calandra}}, \bibinfo {author}
  {\bibfnamefont {I.}~\bibnamefont {Errea}},\ and\ \bibinfo {author}
  {\bibfnamefont {F.}~\bibnamefont {Mauri}},\ }\href
  {https://doi.org/10.1088/1361-648x/ac066b} {\bibfield  {journal} {\bibinfo
  {journal} {Journal of Physics: Condensed Matter}\ }\textbf {\bibinfo {volume}
  {33}},\ \bibinfo {pages} {363001} (\bibinfo {year} {2021})}\BibitemShut
  {NoStop}%
\bibitem [{\citenamefont {Carreras}\ \emph {et~al.}(2017)\citenamefont
  {Carreras}, \citenamefont {Togo},\ and\ \citenamefont
  {Tanaka}}]{CarTogTan2017}%
  \BibitemOpen
  \bibfield  {author} {\bibinfo {author} {\bibfnamefont {A.}~\bibnamefont
  {Carreras}}, \bibinfo {author} {\bibfnamefont {A.}~\bibnamefont {Togo}},\
  and\ \bibinfo {author} {\bibfnamefont {I.}~\bibnamefont {Tanaka}},\ }\href
  {https://doi.org/https://doi.org/10.1016/j.cpc.2017.08.017} {\bibfield
  {journal} {\bibinfo  {journal} {Computer Physics Communications}\ }\textbf
  {\bibinfo {volume} {221}},\ \bibinfo {pages} {221} (\bibinfo {year}
  {2017})}\BibitemShut {NoStop}%
\bibitem [{\citenamefont {Rohskopf}\ \emph {et~al.}(2022)\citenamefont
  {Rohskopf}, \citenamefont {Li}, \citenamefont {Luo},\ and\ \citenamefont
  {Henry}}]{RohLiLuoHen2022}%
  \BibitemOpen
  \bibfield  {author} {\bibinfo {author} {\bibfnamefont {A.}~\bibnamefont
  {Rohskopf}}, \bibinfo {author} {\bibfnamefont {R.}~\bibnamefont {Li}},
  \bibinfo {author} {\bibfnamefont {T.}~\bibnamefont {Luo}},\ and\ \bibinfo
  {author} {\bibfnamefont {A.}~\bibnamefont {Henry}},\ }\href
  {https://doi.org/10.1088/1361-651X/ac5ebb} {\bibfield  {journal} {\bibinfo
  {journal} {Modelling and Simulation in Materials Science and Engineering}\
  }\textbf {\bibinfo {volume} {30}},\ \bibinfo {pages} {045010} (\bibinfo
  {year} {2022})}\BibitemShut {NoStop}%
\bibitem [{Note3()}]{Note3}%
  \BibitemOpen
  \bibinfo {note} {We note that the hopping frequency depends strongly on the
  system size, and is thus not a good observable on its own.}\BibitemShut
  {Stop}%
\bibitem [{\citenamefont {Volpe}\ and\ \citenamefont
  {Volpe}(2013)}]{Volpe2013}%
  \BibitemOpen
  \bibfield  {author} {\bibinfo {author} {\bibfnamefont {G.}~\bibnamefont
  {Volpe}}\ and\ \bibinfo {author} {\bibfnamefont {G.}~\bibnamefont {Volpe}},\
  }\href {https://doi.org/10.1119/1.4772632} {\bibfield  {journal} {\bibinfo
  {journal} {American Journal of Physics}\ }\textbf {\bibinfo {volume} {81}},\
  \bibinfo {pages} {224} (\bibinfo {year} {2013})}\BibitemShut {NoStop}%
\bibitem [{\citenamefont {Cochran}(1960)}]{Cochran1960}%
  \BibitemOpen
  \bibfield  {author} {\bibinfo {author} {\bibfnamefont {W.}~\bibnamefont
  {Cochran}},\ }\href {https://doi.org/10.1080/00018736000101229} {\bibfield
  {journal} {\bibinfo  {journal} {Advances in Physics}\ }\textbf {\bibinfo
  {volume} {9}},\ \bibinfo {pages} {387} (\bibinfo {year} {1960})}\BibitemShut
  {NoStop}%
\bibitem [{\citenamefont {Pytte}\ and\ \citenamefont
  {Feder}(1969)}]{PytteFeder1969}%
  \BibitemOpen
  \bibfield  {author} {\bibinfo {author} {\bibfnamefont {E.}~\bibnamefont
  {Pytte}}\ and\ \bibinfo {author} {\bibfnamefont {J.}~\bibnamefont {Feder}},\
  }\href {https://doi.org/10.1103/PhysRev.187.1077} {\bibfield  {journal}
  {\bibinfo  {journal} {Phys. Rev.}\ }\textbf {\bibinfo {volume} {187}},\
  \bibinfo {pages} {1077} (\bibinfo {year} {1969})}\BibitemShut {NoStop}%
\bibitem [{\citenamefont {Korotaev}\ \emph {et~al.}(2018)\citenamefont
  {Korotaev}, \citenamefont {Belov},\ and\ \citenamefont
  {Yanilkin}}]{KorBelYan2018}%
  \BibitemOpen
  \bibfield  {author} {\bibinfo {author} {\bibfnamefont {P.}~\bibnamefont
  {Korotaev}}, \bibinfo {author} {\bibfnamefont {M.}~\bibnamefont {Belov}},\
  and\ \bibinfo {author} {\bibfnamefont {A.}~\bibnamefont {Yanilkin}},\ }\href
  {https://doi.org/https://doi.org/10.1016/j.commatsci.2018.03.057} {\bibfield
  {journal} {\bibinfo  {journal} {Computational Materials Science}\ }\textbf
  {\bibinfo {volume} {150}},\ \bibinfo {pages} {47} (\bibinfo {year}
  {2018})}\BibitemShut {NoStop}%
\bibitem [{\citenamefont {Metsanurk}\ and\ \citenamefont
  {Klintenberg}(2019)}]{MetKli2019}%
  \BibitemOpen
  \bibfield  {author} {\bibinfo {author} {\bibfnamefont {E.}~\bibnamefont
  {Metsanurk}}\ and\ \bibinfo {author} {\bibfnamefont {M.}~\bibnamefont
  {Klintenberg}},\ }\href {https://doi.org/10.1103/PhysRevB.99.184304}
  {\bibfield  {journal} {\bibinfo  {journal} {Physical Review B}\ }\textbf
  {\bibinfo {volume} {99}},\ \bibinfo {pages} {184304} (\bibinfo {year}
  {2019})}\BibitemShut {NoStop}%
\bibitem [{\citenamefont {Tolborg}\ and\ \citenamefont
  {Walsh}(2022)}]{Tolborg2022}%
  \BibitemOpen
  \bibfield  {author} {\bibinfo {author} {\bibfnamefont {K.}~\bibnamefont
  {Tolborg}}\ and\ \bibinfo {author} {\bibfnamefont {A.}~\bibnamefont
  {Walsh}},\ }\bibfield  {journal} {\bibinfo  {journal} {ChemRxiv}\ }\href
  {https://doi.org/10.26434/chemrxiv-2022-lkzm9} {10.26434/chemrxiv-2022-lkzm9}
  (\bibinfo {year} {2022})\BibitemShut {NoStop}%
\bibitem [{zen(2022)}]{zenodo_dataset}%
  \BibitemOpen
  \href@noop {} {\bibinfo {title} {Zenodo dataset}},\ \bibinfo {howpublished}
  {\url{10.5281/zenodo.7313504}} (\bibinfo {year} {2022})\BibitemShut {NoStop}%
\end{thebibliography}%

\end{document}
