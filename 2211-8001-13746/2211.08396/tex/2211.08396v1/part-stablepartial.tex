\subsection{Length of the series of a derivative of a monomial}

\subsubsection{Case $\epsilon\preceq\ln u$}

\begin{lemma}
	\label{lem:formeEtaPhiOmegaA}
	Assume $x=\omega^a=\partial u\exp\epsilon$ with $\epsilon=r\ln u+\eta$ and $u=\ln_n\kappa_{-\alpha}$. 
	Let $b\in\supp\Phi(\omega^a)$. Then, there is a path $P\in\Pcal_\Lbb(\eta)$ such that
	\begin{calculs}
		& \omega^b &\asymp& \partial u\exp\pa{r\ln u + \eta -\Sum{m=n+2}{\pinf}\ln_m\kappa_{-\alpha}
			-\hspace{-2.5em}\Sum{\footnotesize\begin{array}{c}
					\beta>\alpha, m\in\Nbb^*\\ \beta\tq\kappa_{-\beta}\succeq^KP(k_P)
			\end{array}}{}\hspace{-2.5em} \ln_m\kappa_{-\beta} + \Sum{i=0}{\pinf}\ln |P(i)|}
	\end{calculs}
\end{lemma}

\begin{proof}
	It is just a calculation. First notice that $\f{\omega^a}{r+1}\f u{\partial u}$ is a term as a product of terms. Then, let $b\in\supp \Phi(\omega^a)$. There is path $P$ of $\eta$ such that
	\centers{$\omega^b\asymp\omega^a\f u{\partial u}\partial P = u\partial P\exp(r\ln u + \eta)$}
	\lc{write}{$\partial P=P(0)\cdots P(k_P-1)\partial_\Lbb P(k_P)$}
	
	Since $P(0)$ is a term of $\eta\prec\ln u$, we also have $P(0)\prec\ln u$. Moreover since $\eta$ consists in
	purely infinite  term, so is $P(0)$ and then  $\ln |P(0)|\prec P(0)$. Since $P(1)$ is a purely infinite 
	term of $\ln|P(0)|$, we get that $P(1)\prec P(0)$. By induction, for all $i$, $P(i+1)\preceq P(i)\leq P(0)$.
	In particular, $P(k_P)\preceq P(0)\preceq^k\kappa_{-\alpha}$. Then, $\kappa_{-\alpha}\succeq^K P(k_P)$. That leads to
	\begin{calculs}
		& \partial_\Lbb(P(k_P)) &=& \exp\pa{-\Sum{\beta\leq\alpha,\ m\in\Nbb^*}{}\ln_m\kappa_{-\beta} 
			-\hspace{-2.5em}\Sum{\footnotesize\begin{array}{c}
					\beta>\alpha, m\in\Nbb^*\\ \beta\tq\kappa_{-\beta}\succeq^KP(k_P)
			\end{array}}{}\hspace{-2.5em} \ln_m\kappa_{-\beta} + \Sum{m=1}{\pinf}\ln_mP(k_P)} \\
		& \partial_\Lbb(P(k_P)) &=& \partial u \exp\pa{-\Sum{m=n+1}{\pinf}\ln_m{\kappa_{-\alpha}}
			-\hspace{-2.5em}\Sum{\footnotesize\begin{array}{c}
					\beta>\alpha, m\in\Nbb^*\\ \beta\tq\kappa_{-\beta}\succeq^KP(k_P)
			\end{array}}{}\hspace{-2.5em} \ln_m\kappa_{-\beta} + \Sum{m=1}{\pinf}\ln_mP(k_P)}\\
		Since $P(k_P)\in\Lbb$,\\
		& \partial_\Lbb(P(k_P)) &=& \partial u \exp\pa{-\Sum{m=n+1}{\pinf}\ln_m{\kappa_{-\alpha}}
			-\hspace{-2.5em}\Sum{\footnotesize\begin{array}{c}
					\beta>\alpha, m\in\Nbb^*\\ \beta\tq\kappa_{-\beta}\succeq^KP(k_P)
			\end{array}}{}\hspace{-2.5em} \ln_m\kappa_{-\beta} + \Sum{i=k_P}{\pinf}\ln |P(i)|}\\
		Then & \partial P &=& \partial u\exp\pa{-\Sum{m=n+1}{\pinf}\ln_m{\kappa_{-\alpha}}
			-\hspace{-2.5em}\Sum{\footnotesize\begin{array}{c}
					\beta>\alpha, m\in\Nbb^*\\ \beta\tq\kappa_{-\beta}\succeq^KP(k_P)
			\end{array}}{}\hspace{-2.5em} \ln_m\kappa_{-\beta} + \Sum{i=0}{\pinf}\ln |P(i)|}\\ \end{calculs}Finally, \begin{calculs}
		& \omega^b &\asymp & \partial u\exp(r\ln u+\eta)u\\ &&& \qquad\times\exp\pa{-\Sum{m=n+1}{\pinf}\ln_m{\kappa_{-\alpha}}
			-\hspace{-2.5em}\Sum{\footnotesize\begin{array}{c}
					\beta>\alpha, m\in\Nbb^*\\ \beta\tq\kappa_{-\beta}\succeq^KP(k_P)
			\end{array}}{} \hspace{-2.5em}\ln_m\kappa_{-\beta} + \Sum{i=0}{\pinf}\ln |P(i)|}\\
		& \omega^b &\asymp & \partial u\exp\pa{r\ln u+\eta-\Sum{m=n+2}{\pinf}\ln_m{\kappa_{-\alpha}}
			-\hspace{-2.5em}\Sum{\footnotesize\begin{array}{c}
					\beta>\alpha, m\in\Nbb^*\\ \beta\tq\kappa_{-\beta}\succeq^KP(k_P)
			\end{array}}{}\hspace{-2.5em} \ln_m\kappa_{-\beta} + \Sum{i=0}{\pinf}\ln |P(i)|}\\
	\end{calculs}
\end{proof}

\begin{proposition}\label{prop:supportPhi1}	
	Assume $x=\omega^a=\partial u\exp\epsilon$ with $\epsilon=r\ln u+\eta$ and $u=\ln_n\kappa_{-\alpha}$. 
	We denote for $P_0,\dots,P_k\in\Pcal_\Lbb(\eta)$ and $i_1,\dots,i_k\in\Nbb^*$,
	\begin{calculs}
		&e\left(\begin{array}{c}P_0,\dots,P_k\\ i_1,\dots,i_k\end{array}\right) &=& 
		-(k+1)\Sum{m=n+2}{\pinf}\ln_m{\kappa_{-\alpha}}\\&&&\qquad
		-\Sum{j=0}{k}\hspace{-1em}\Sum{\footnotesize\begin{array}{c}
				\beta>\alpha, m\in\Nbb^*\\ \beta\tq\kappa_{-\beta}\succeq^KP_j(k_{P_j})
		\end{array}}{}\hspace{-2em} \ln_m\kappa_{-\beta}
		+ \Sum{j=0}{k}\Sum{i=i_j}{\pinf}\ln |P_j(i)|
	\end{calculs}
	with $i_0=0$. For $k\in\Nbb$ define $E_{1,k}$ by:
	\begin{calculs}
		& e\left(\begin{array}{c}P_0,\dots,P_k\\ i_1,\dots,i_k\end{array}\right)\in E_{1,k}&\Leftrightarrow& P_0,\dots,P_k\in\Pcal_\Lbb(\eta)\quad\wedge\quad i_1,\dots,i_k\in\Nbb^*\\
		&&&\quad\wedge\quad \forall j\in\intn1k\quad \exists j'\in\intn 0{j-1}\\
		&&&\qquad\qquad \forall i\in\intn0{i_j-1}\qquad P_{j'}(i)=P_j(i)\\
		&&&\quad\wedge\quad \forall j \in \intn1k\\ &&&\qquad\quad \supp P_j(i_j)\subseteq\supp e\left(\begin{array}{c}P_0,\dots,P_{k-1}\\ i_1,\dots,i_{k-1}\end{array}\right)
			
	\end{calculs}
	\begin{calculs}
		& E_1 &=& \Union{k\in\Nbb}{} E_{1,k} \\ [.5cm]
		&E_2 &=& \enstq{\begin{array}{c}-\Sum{m=n+2}{\pinf}\ln_m\kappa_{-\alpha}
			\\-\hspace{-1em}\Sum{\gamma<\beta,\ m\in\Nbb^*}{}\hspace{-1em}\ln_m\kappa_{-\gamma} - \Sum{m=1}{p}\ln_m\kappa_{-\beta}\end{array}}{
			\begin{array}{c}
				\beta>\alpha \\ \exists P\in\Pcal_\Lbb(\eta)\quad \kappa_{-\beta}\succeq^K P(k_P) \\ p\in\Nbb
		\end{array}}\\ [1cm]
		&E_3 &=& \enstq{-\Sum{m=n+2}{p}\ln_m{\kappa_{-\alpha}}}{p\geq n+2}\\ [.4cm]
		& E &=& E_1\cup E_2\cup E_3
	\end{calculs}
	and $\inner E$ be the monoid it generates. Let $b\in\Union{\ell=0}{\pinf}\supp\Phi^\ell(\omega^a)$. Then, there is $y\in\inner E$ such that
	\centre{$\omega^b\asymp\partial u\exp(r\ln u + \eta + y)$}
\end{proposition}

\begin{proof}
	We prove it by induction on $\ell$.
	\begin{itemize}
		\item If $b\in\supp\omega^a$, then $y=0$ works.
		
		\item Assume the property for $\ell\in\Nbb$ and let $b\in\sup\Phi^{\ell+1}(\omega^a)$. Then there is 
		$c\in\supp\Phi^\ell(\omega^a)$ such that $b\in\supp\Phi(\omega^c)$. Apply the induction hypothesis on $c$
		and on $y$ associated to $c$. Since any element $e\in E$ is such that $e\prec\ln u$, we have $y\prec\ln u$
		then 
		Apply Lemma \ref{lem:formeEtaPhiOmegaA} to get that there is $P\in\Pcal_\Lbb(\eta+y)$ such that
		\begin{calculs}
			& \omega^b &\asymp& \partial u\exp\pa{r\ln u + \eta + y -\Sum{m=n+2}{\pinf}\ln_m\kappa_{-\alpha}
				\vphantom{\Sum{\footnotesize\begin{array}{c}
							\beta>\alpha, m\in\Nbb^*\\ \beta\tq\kappa_{-\beta}\succeq^KP(k_P)
					\end{array}}{}}\right.\\
			&&&\qquad\qquad\left. -\hspace{-2em}\Sum{\footnotesize\begin{array}{c}
					\beta>\alpha, m\in\Nbb^*\\ \beta\tq\kappa_{-\beta}\succeq^KP(k_P)
			\end{array}}{}\hspace{-2em} \ln_m\kappa_{-\beta} + \Sum{i=0}{\pinf}\ln |P(i)|}
		\end{calculs}
		If $P(0)$ a term of $\eta$, up to some real factor, then there is a real number $s$ and some $e\in E_{1,0}$ such that
		\centers{$\exp\pa{-\Sum{m=n+2}{\pinf}\ln_m\kappa_{-\alpha} -\hspace{-2em}\Sum{\footnotesize\begin{array}{c}
						\beta>\alpha, m\in\Nbb^*\\ \beta\tq\kappa_{-\beta}\succeq^KP(k_P)
				\end{array}}{}\hspace{-2em} \ln_m\kappa_{-\beta} + \Sum{i=0}{\pinf}\ln |P(i)|} = s\exp e$} 
		Then $y+e\in \inner E$ and $\omega^b\asymp\partial u\exp(r\ln u+\eta+y+e)$. If not, then $P(0)$ is a term
		of $y$ (not up to a real factor, an actual term). Hence, we have the following cases :
		\begin{itemize}
			\item $P(0)=s\ln_p\kappa_{-\alpha}$ for some $s\in\Rbb^*_-$ and $p\geq n+2$. Then,
			\begin{calculs}
				&-\Sum{m=n+2}{\pinf}\ln_m\kappa_{-\alpha} -\hspace{-2em}\Sum{\footnotesize\begin{array}{c}
						\beta>\alpha, m\in\Nbb^*\\ \beta\tq\kappa_{-\beta}\succeq^KP(k_P)
				\end{array}}{} \hspace{-2em}\ln_m\kappa_{-\beta} + \Sum{i=0}{\pinf}\ln |P(i)| &=&  
				\ln|s| - \Sum{m=n+2}{p}\ln_m\kappa_{-\alpha} \\ &&\in& \ln|s|+E_3
			\end{calculs}
			Then,
			\centre{$y-\Sum{m=n+2}{\pinf}\ln_m\kappa_{-\alpha} -\hspace{-2em}\Sum{\footnotesize\begin{array}{c}
						\beta>\alpha, m\in\Nbb^*\\ \beta\tq\kappa_{-\beta}\succeq^KP(k_P)
				\end{array}}{} \hspace{-2em}\ln_m\kappa_{-\beta} + \Sum{i=0}{\pinf}\ln |P(i)| \in\Rbb+\inner E$}
			
			\item $P(0)=s\ln_p\kappa_{-\beta}$ with $\beta>\alpha$ and $p\in\Nbb^*$ such that there is some path $Q\in\Pcal_\Lbb(\eta)$ such that $\kappa_{-\beta}\succeq^K Q(k_Q)$. Then
			\centre{$-\Sum{m=n+2}{\pinf}\ln_m\kappa_{-\alpha} -\hspace{-2em}\Sum{\footnotesize\begin{array}{c}
						\beta>\alpha, m\in\Nbb^*\\ \beta\tq\kappa_{-\beta}\succeq^KP(k_P)
				\end{array}}{} \hspace{-2em}\ln_m\kappa_{-\beta} + \Sum{i=0}{\pinf}\ln |P(i)| \in \ln|s| + E_2$}
			Then,
			\centre{$y-\Sum{m=n+2}{\pinf}\ln_m\kappa_{-\alpha} -\hspace{-2em}\Sum{\footnotesize\begin{array}{c}
						\beta>\alpha, m\in\Nbb^*\\ \beta\tq\kappa_{-\beta}\succeq^KP(k_P)
				\end{array}}{} \hspace{-2em}\ln_m\kappa_{-\beta} + \Sum{i=0}{\pinf}\ln |P(i)| \in\Rbb+\inner E$}
			
			\item There are some paths $P_0,\dots,P_k\in \Pcal_\Lbb(\eta)$ and some non-zero integers $i_1,\dots,i_k$ such that 
			\centers{$\forall j\in\intn1k\ \exists j'\in\intn 0{j-1}\ \forall i\in\intn0{i_j-1}\quad P_{j'}(i)=P_j(i)$}
			and
			\begin{calculs}
				&\exists y'\in\inner E\quad y&=&y'-(k+1)\Sum{m=n+2}{\pinf}\ln_m{\kappa_{-\alpha}}\\
				&&& \quad-\Sum{j=0}{k}\Sum{\footnotesize\begin{array}{c}
						\beta>\alpha, m\in\Nbb^*\\ \beta\tq\kappa_{-\beta}\succeq^KP_j(k_{P_j})
				\end{array}}{}\hspace{-2em} \ln_m\kappa_{-\beta} + \Sum{j=0}{k}\Sum{i=i_j}{\pinf}\ln |P_j(i)|
			\end{calculs}
			and such that $P(0)\in\Rbb z$ for some $z$ a term of some $\ln|P_j({i_{k+1}}')|$ with $j\in\intn0k$ and ${i_{k+1}}'\geq i_j$. 
			Let $P_{k+1}$ be the following path :
			\centre{$P_{k+1}(i) = \begin{accolade}
					P_j(i) & i\leq {i_{k+1}}' \\ z & i={i_{k+1}}'+1 \\ P(i-{i_{k+1}}'-1) & i>{i_{k+1}}'+1 
				\end{accolade}$}
			Then, $P_{k+1}\in\Pcal(\eta)$. Moreover, $\partial P_{k+1}=\underbrace{P_j(0)\cdots P_j({i_{k+1}}')}_{\neq 0}\underbrace{\partial P}_{\neq 0}$. Then $P_{k+1}\in\Pcal_\Lbb(\eta)$. Note also that for all $\beta$, 
			
			\centre{$\kappa_{-\beta}\succeq^K P_{k+1}(k_{P_{k+1}})\Longleftrightarrow\kappa_{-\beta}\succeq^KP(k_P)$}
			Finally,
			
			\begin{calculs}
				&&&-\Sum{m=n+2}{\pinf}\ln_m\kappa_{-\alpha} -\hspace{-2.5em}\Sum{\footnotesize\begin{array}{c}
						\beta>\alpha, m\in\Nbb^*\\ \beta\tq\kappa_{-\beta}\succeq^KP(k_P)
				\end{array}}{} \hspace{-2.5em}\ln_m\kappa_{-\beta} + \Sum{i=0}{\pinf}\ln |P(i)|\\
				&&=&-\Sum{m=n+2}{\pinf}\ln_m\kappa_{-\alpha} -\hspace{-2.5em}\Sum{\footnotesize\begin{array}{c}
						\beta>\alpha, m\in\Nbb^*\\ \beta\tq\kappa_{-\beta}\succeq^KP_{k+1}(k_{P_{k+1}})
				\end{array}}{}\hspace{-2.5em} \ln_m\kappa_{-\beta} + \Sum{i={i_{k+1}}'+1}{\pinf}\ln |P_{k+1}(i)| + \ln\underbrace{\left|\f{P(0)}{z}\right|}_{\in\Rbb^*_+}\\
			\end{calculs}
			From that we derive that
			\begin{calculs}
				&&& y-\Sum{m=n+2}{\pinf}\ln_m\kappa_{-\alpha} -\hspace{-2em}\Sum{\footnotesize\begin{array}{c}
						\beta>\alpha, m\in\Nbb^*\\ \beta\tq\kappa_{-\beta}\succeq^KP(k_P)
				\end{array}}{} \hspace{-2em}\ln_m\kappa_{-\beta} + \Sum{i=0}{\pinf}\ln |P(i)| \\
				&&=& y'-(k+2)\Sum{m=n+2}{\pinf}\ln_m{\kappa_{-\alpha}}\\
				&&&\quad-\Sum{j=0}{k+1}\Sum{\footnotesize\begin{array}{c}
						\beta>\alpha, m\in\Nbb^*\\ \beta\tq\kappa_{-\beta}\succeq^KP_j(k_{P_j})
				\end{array}}{}\hspace{-2em} \ln_m\kappa_{-\beta} + \Sum{j=0}{k+1}\Sum{i=i_j}{\pinf}\ln |P_j(i)| 
				+ \ln\left|\f{P(0)}{z}\right|\\
				&&\in& \Rbb + \inner E
			\end{calculs}
			where $i_{k+1}={i_{k+1}}'+1$ and $P_{k+1}(i_k)=z$ has indeed its support (which is reduced to a singleton) included
			in the one of $e\left(\begin{array}{c}P_0,\dots,P_k\\ i_1,\dots,i_k\end{array}\right)$.
		\end{itemize}
		Then there is a real number $s$, and $e\in\inner E$ such that 
		\centre{$\omega^b\asymp\partial u\exp(r\ln u + \eta + e + s)\asymp \partial u\exp(r\ln u + \eta + e)$}
		Then we get the property at rank $\ell+1$.
	\end{itemize}
	By the induction principle, we conclude that the proposition is true for any $\ell\in\Nbb$.
\end{proof}

\begin{corollary}\label{cor:propsupportPhi1}	Let $x=\aSurreal$ such that
	\centre{$\exists u=\ln_n\kappa_{-\alpha}\ \exists r\in\Rbb\ \forall a\in\supp x\ \exists \eta\prec\ln u\quad \omega^a=\partial(u)\exp(r\ln u+\eta)$}
	We denote for $P_0,\dots,P_k\in\Pcal_\Lbb(x)$ and $i_1,\dots,i_k\in\Nbb\setminus\{0,1\}$,
	\begin{calculs}
		&e\left(\begin{array}{c}P_0,\dots,P_k\\ i_1,\dots,i_k\end{array}\right) &=& 
		-(k+1)\Sum{m=n+2}{\pinf}\ln_m{\kappa_{-\alpha}}\\
		&&&\quad-\Sum{j=0}{k}\Sum{\footnotesize\begin{array}{c}
				\beta>\alpha, m\in\Nbb^*\\ \beta\tq\kappa_{-\beta}\succeq^KP_j(k_{P_j})
		\end{array}}{} \hspace{-2em}\ln_m\kappa_{-\beta}
		+ \Sum{j=0}{k}\Sum{i=i_j}{\pinf}\ln |P_j(i)|
	\end{calculs}
	with $i_0=0$. For $k\in\Nbb$ define $E_{1,k}$ by:
	\begin{calculs}
		& e\left(\begin{array}{c}P_0,\dots,P_k\\ i_1,\dots,i_k\end{array}\right)\in E_{1,k}&\Leftrightarrow& P_0,\dots,P_k\in\Pcal_\Lbb(\eta)\quad\wedge\quad i_1,\dots,i_k\in\Nbb\setminus\{0,1\}\\
		&&&\quad\wedge\quad \forall i\in\intn0k\qquad P_i(1)\prec\ln u\\
		&&&\quad\wedge\quad \forall j\in\intn1k\quad \exists j'\in\intn 0{j-1}\\
		&&&\qquad\qquad \forall i\in\intn0{i_j-1}\qquad P_{j'}(i)=P_j(i)\\
		&&&\quad\wedge\quad \forall j \in \intn1k\\ &&&\qquad\quad \supp P_j(i_j)\subseteq\supp e\left(\begin{array}{c}P_0,\dots,P_{k-1}\\ i_1,\dots,i_{k-1}\end{array}\right)
		
	\end{calculs}
	Let also
	\begin{calculs}
		& E_1 &=& \Union{k\in\Nbb}{} E_{1,k} \\ [.5cm]
		&E_2 &=& \enstq{\begin{array}{c}
				-\Sum{m=n+2}{\pinf}\ln_m\kappa_{-\alpha}
			-\hspace{-1em}\Sum{\gamma<\beta,\ m\in\Nbb^*}{}\hspace{-1em}\ln_m\kappa_{-\gamma}\\ \quad - \Sum{m=1}{p}\ln_m\kappa_{-\beta}\end{array}\!\!}{\!\!
			\begin{array}{c}
				\beta>\alpha \\ \exists P\in\Pcal_\Lbb(x)\ \kappa_{-\beta}\succeq^K P(k_P) \\ p\in\Nbb
		\end{array}}\\ [1cm]
		&E_3 &=& \enstq{-\Sum{m=n+2}{p}\ln_m{\kappa_{-\alpha}}}{p\geq n+2}\\ [.4cm]
		& E &=& E_1\cup E_2\cup E_3
	\end{calculs}
	and $\inner E$ be the monoid it generates. Let $b\in\Union{\ell=0}{\pinf}\supp\Phi^\ell(x)$. Then, there is $y\in\inner E$ such that
	\centre{$\omega^b\asymp\exp(y)$}
\end{corollary}

\begin{proof}
	Since $\Phi$ is strongly linear\index{Strongly linear function}, we just need to apply Proposition \ref{prop:supportPhi1} to each term of $x$. For each term, $\partial u \exp(r\ln u + \eta)$ is term we add at the beginning $P_0$. Each path involved is shifted one rank. 
\end{proof}

\begin{proposition}
	\label{prop:bonOrderPhi1}
	Let $x=\aSurreal$ such that
	\centre{$\exists u=\ln_n\kappa_{-\alpha}\ \exists r\in\Rbb\ \forall a\in\supp x\ \exists \eta\prec\ln u\quad \omega^a=\partial(u)\exp(r\ln u+\eta)$} 
	Consider $E_1$, $E_2$ and $E_3$ as defined Corollary \ref{cor:propsupportPhi1}.
	Let $\gamma$ be the smallest ordinal such that $\kappa_{-\gamma}\prec^K P(k_P)$ for all path $P\in\Pcal_\Lbb(\eta)$.
	Let $\lambda$ the least $\epsilon$-number greater than $\NR(x)$ and $\gamma$. Then $E=E_1\cup E_2\cup E_3$ is reverse well-ordered with order type at most $2\lambda+\omega(\gamma+1)$.
\end{proposition}

\begin{proof}
	First notice that $E_3$ is reverse well-ordered with order type $\omega$. $E_2$ is also reverse well-ordered with order at most $\omega+\omega\otimes\gamma+n\leq \omega\otimes (\gamma+1)$. We then focus on $E_1$. We denote again
	\begin{calculs}
		&e\left(\begin{array}{c}P_0,\dots,P_k\\ i_1,\dots,i_k\end{array}\right) &=& 
		-(k+1)\Sum{m=n+2}{\pinf}\ln_m{\kappa_{-\alpha}}\\
		&&&\qquad-\Sum{j=0}{k}\Sum{\footnotesize\begin{array}{c}
				\beta>\alpha, m\in\Nbb^*\\ \beta\tq\kappa_{-\beta}\succeq^KP_j(k_{P_j})
		\end{array}}{} \ln_m\kappa_{-\beta}
		+ \Sum{j=0}{k}\Sum{i=i_j}{\pinf}\ln |P_j(i)|
	\end{calculs}
	
	\begin{enumerate}[label=(\roman*)]
		\item We first claim that for all $i\geq 3$ and all path $P\in\Pcal(x)$ such that $P(1)\prec\ln u$, $P(i)\prec P(2)\preceq \ln_2 u$. Let $P\in\Pcal(x)$ such that $P(1)\prec\ln u$. Assume $P(2)\succ\ln_2 u$. Then, since $P(2)$ is a term of $\ln |P(1)|$, we also have $\ln|P(1)|\succ\ln_2(u)$. Then, either $\ln|P(1)|<-m\ln_2 u$ for all $m\in\Nbb$, or $\ln|P(1)|>m\ln_2(u)$ for all $m\in\Nbb$. By definition, $P(1)$ is purely infinite. In particular, $\ln |P(1)|$ cannot be negative. Then,
		\centre{$\forall m\in\Nbb\qquad \ln|P(1)|>m\ln_2 u$}
		\lcr{and}{$\forall m\in\Nbb\qquad |P(1)| > \pa{\ln u}^m$}{($\exp$ is increasing)}
		which is a contradiction with $P(1)\prec\ln u$, since $\ln u$ is infinitely large. Since, for $i\geq 2$, $P(i)$ is infinitely large, $\ln|P(i)|\prec P(i)$, and since $P(i+1)\preceq\ln|P(i)|$, we have for all $i\geq 1$, $P(i+1)\prec P(i)$.
		By induction, we get
		\centers{$\forall i\geq3\qquad P(i)\prec P(2) \preceq\ln_2 u$}
		
		\item We claim that for all path $P\in\Pcal(x)$ such that $P(1)\prec\ln u$, if $P(2)\asymp\ln_2u$, then, denoting $r$  the real number such that 
		$P(2)\sim r\ln_2 u$, we have $0<r\leq1$ . Let $P\in\Pcal(x)$ such that $P(1)\prec\ln u$ and
		assume $P(2)\asymp\ln_2u$. Since $P(2)$ is a term there is a non-zero real number $r$ such that $P(1)=r\ln_2u$. From (i), we know that $P(2)$ is the dominant term of $\ln|P(1)|$ so that
		\centre{$\ln|P(1)|\sim r\ln_2 u$}
		If $r<0$, Proposition \ref{prop:formeExpXPurelyInfiniteOmegaAg} ensures that $|P(1)|\prec 1$ what is impossible since $P(1)$ is infinite. Then $r>0$. If now $r>1$ then again with Proposition \ref{prop:formeExpXPurelyInfiniteOmegaAg}, $|P(1)|\succ\ln u$ what is not true. Then, $0<r\leq1$.
		
		\item For all $j$ and $i\geq 2$, $\ln |P_j(i)|\preceq\ln_3 u\prec \ln_2u$. Indeed, using (i), we know that $P_j(i)\preceq \ln_2 u$. Then, there is a natural number $m\geq 1$ such that $|P_j(i)|\leq m\ln_2u$. Using the fact that $\ln$ is increasing,
		\centre{$\ln|P_j(i)| \leq \ln_3u+\ln m\preceq \ln_3u\prec \ln_2u$} 
		
		\item We now claim that $E_{1,k}>E_{1,k+2}$. Indeed, using (ii) and (iii) if $e_1\in E_{1,k}$, then there is $s\in\intff{-(k+1)}{-k}$ such that 
		$e_1\sim s\ln_2 u$. Similarly, for $e_2\in E_{1,k+2}$, there is $s'\in\intff{-(k+3)}{-(k+2)}$ such that $e_2\sim s'\ln_2u$. 
		
		\item We define the following sequence : 
		\begin{itemize}
			\item $a_0=\omega^{\omega^{\omega(\NR(x)+1)}}$
			\item $a_{k+1} = \omega^{\omega^{\omega(\omega(\NR(x)+\gamma+4)a_k+1)}}$
		\end{itemize}
		We show that $E_{1,k}$ is reverse well-ordered with order type less than $a_k$. We also claim that the equivalence classes of $E_{1,k}/{\asymp}$ are finite and that 
		\centre{$\NR\pa{\Sum{t\in E_{1,k}}{}\exp t}\leq \omega (\NR(x)+\gamma+4)a_k$}
		We show it by induction on $k\in\Nbb$.
		\begin{itemize}
			\item For $k=0$, let $t\in E_{1,0}$. Take $P\in\Pcal_\Lbb(x)$, minimal for $<_{lex}$ such that $P(1)\prec\ln u$ and $t=e(P;)$. Then
			\begin{calculs}
				&\partial(\ln u)\exp t &=& |P(0)\cdots P(k_P-1)|\exp\pa{
					-\Sum{\footnotesize\begin{array}{c}
							\beta\tq\kappa_{-\beta}\succeq^KP(k_P)\\ m\in\Nbb^*
					\end{array}}{} \ln_m\kappa_{-\beta}\right.\\ &&&\hspace{15em}\left. \vphantom{\Sum{\footnotesize\begin{array}{c}
						\beta\tq\kappa_{-\beta}\succeq^KP(k_P)\\ m\in\Nbb^*
				\end{array}}{}}
					+\Sum{i=k_P}{\pinf}\ln |P(i)|}\\ 
				&&=&|\partial P|
			\end{calculs}
			Since there are finitely many paths $Q\in\Pcal_\Lbb(x)$ such that $\partial P\asymp\partial Q$, there are finitely many $t'\in E_{1,0}$ such that
			\centre{$\partial (\ln u)\exp t \asymp\partial (\ln u)\exp t'$}
			Since $\exp$ is an increasing function and $\partial(\ln u)>0$, we get , using Proposition \ref{prop:majorationNuPartial}, that $E_{1,0}$ is reverse well-ordered with order type less than $\omega\otimes\omega^{\omega^{\omega(\NR(x)+1)}}=\omega^{\omega^{\omega(\NR(x)+1)}}=a_0$.
			Finally, it remains to compute the nested rank of $\Sum{t\in E_{1,0}}{}\exp t$. Write	
			\centre{$t=
				-\Sum{m=n+2}{\pinf}\ln_m{\kappa_{-\alpha}}
				-\Sum{\footnotesize\begin{array}{c}
						\beta>\alpha, m\in\Nbb^*\\ \beta\tq\kappa_{-\beta}\succeq^KP_0(k_{P_0})
				\end{array}}{} \ln_m\kappa_{-\beta}
				+\Sum{i=0}{\pinf}\ln |P_0(i)|$}
			\begin{calculs}
				& \NR\pa{t} &=& \NR\pa{
					-\Sum{m=n+2}{\pinf}\ln_m{\kappa_{-\alpha}}
					-\hspace{-2.5em}\Sum{\footnotesize\begin{array}{c}
							\beta>\alpha, m\in\Nbb^*\\ \beta\tq\kappa_{-\beta}\succeq^KP_0(k_{P_0})
					\end{array}}{}\hspace{-2.5em} \ln_m\kappa_{-\beta}
					+\Sum{i=0}{\pinf}\ln |P_0(i)|}\\
				&&\leq& \NR\pa{
					-\Sum{m=n+2}{\pinf}\ln_m{\kappa_{-\alpha}}
					-\hspace{-2.5em}\Sum{\footnotesize\begin{array}{c}
							\beta>\alpha, m\in\Nbb^*\\ \beta\tq\kappa_{-\beta}\succeq^KP_0(k_{P_0})
					\end{array}}{} \hspace{-2.5em}\ln_m\kappa_{-\beta}
					+\Sum{i=k_{P_0}}{\pinf}\ln |P_0(i)|}\\
				&&&+\Sum{i=0}{k_{P_0}-1}\NR\pa{\ln|P_0(i)|}+k_{P_0} & (Lemma \ref{lem:NRsum})\\
				&&\leq& (\omega\oplus \omega\otimes\gamma \oplus\omega)
				+\Sum{i=0}{k_{P_0}-1}\NR\pa{\ln|P_0(i)|}+k_{P_0}&(Lemma \ref{lem:NRSommeLogAtomiques})\\
				&&\leq& (\omega\oplus \omega\otimes\gamma \oplus\omega)+k_{P_0}(\NR(x)+1) & (using Proposition \ref{prop:NRComparaisonTerme})\\
				&&\leq& \omega (\NR(x)+\gamma+4)
			\end{calculs} 
			Then, since the equivalence classes of $E_{1,0}/{\asymp}$ are finite,
			\centre{$\NR\pa{\Sum{t\in E_{1,0}}{}\exp t}\leq \omega (\NR(x)+\gamma+4)a_0$}
			
			\item Assume the property for some $k\in\Nbb$. Let $t\in E_{1,k+1}$. Let \linebreak 
			$(P_0,0),\dots,(P_{k+1},i_{k+1})$ minimal for the order $(<_{lex},<)_{lex}$ such that $t=e\left(\begin{array}{c}P_0,\dots,P_{k+1}\\ i_1,\dots,i_{k+1}\end{array}\right)$. Then,
			\begin{calculs}
				&t&=&e\left(\begin{array}{c}P_0,\dots,P_k\\ i_1,\dots,i_k\end{array}\right)
				-\Sum{m=n+2}{\pinf}\ln_m{\kappa_{-\alpha}}\\
				&&&\qquad-\Sum{\footnotesize\begin{array}{c}
						\beta>\alpha, m\in\Nbb^*\\ \beta\tq\kappa_{-\beta}\succeq^KP_{k+1}(k_{P_{k+1}})
				\end{array}}{}\hspace{-2em} \ln_m\kappa_{-\beta}
				+ \Sum{i=i_{k+1}}{\pinf}\ln |P_{k+1}(i)|
			\end{calculs}
			Write $s=e\left(\begin{array}{c}P_0,\dots,P_k\\ i_1,\dots,i_k\end{array}\right)$. We then have,
			\begin{calculs}
				&\partial(\ln u)\exp t &=& 
					\exp(s)\exp\pa{-\Sum{\footnotesize\begin{array}{c}
							\beta\tq\kappa_{-\beta}\succeq^KP_{k+1}(k_{P_{k+1}})\\ m\in\Nbb^*
					\end{array}}{} \ln_m\kappa_{-\beta}\right.\\
					&&&\hspace{12em}\left.+\Sum{i=i_{k+1}}{\pinf}\ln |P_{k+1}(i)| \vphantom{\Sum{\footnotesize\begin{array}{c}
								\beta\tq\kappa_{-\beta}\succeq^KP_{k+1}(k_{P_{k+1}})\\ m\in\Nbb^*
						\end{array}}{}}}
			\end{calculs}
			Consider the following path:
			\centre{$\begin{accolade}
					R(0)=\exp s \\
					R(i)= P_{k+1}(i-1+i_{k+1}) & i>0
				\end{accolade}$}
			It is indeed a path since, by definition of $E_{1,k+1}$, $\supp P_{k+1}(i_{k+1})$ must be contained in $\supp s$. Then,
			\centre{$\partial(\ln u)\exp t=\partial R$} 
			Moreover, $R\in\Pcal_\Lbb\pa{\Sum{s\in E_{1,k}}{}\exp s}$. By induction hypothesis and Proposition \ref{prop:majorationNuPartial}, $E_{1,k+1}$ has order type less than 
			\centre{$\omega^{\omega^{\omega(\omega(\NR(x)+\gamma+4)a_k+1)}}=a_{k+1}$}	
			Since the equivalences classes of $\Pcal_\Lbb\pa{\Sum{s\in E_{1,k}}{}\exp s}/{\asymp}$ are finite, the 
			ones of $E_{1,k+1}/{\asymp}$ are also finite. Finally, using Lemmas \ref{lem:NRsum} and \ref{lem:NRSommeLogAtomiques},
			\begin{calculs}
				& \NR(t) &\leq& (\omega\oplus\omega\otimes\gamma\oplus\omega) + 
				\Sum{j=0}{k+1}\Sum{i=i_j}{k_{P_j}-1}\NR\pa{\ln|P_j(i)|}+\Sum{j=0}{k+1}\max(0,k_{P_j}-i_j)\\
				&&\leq& \omega(\NR(x)+\gamma+4)
			\end{calculs}
			\lc{Then,}{$\NR\pa{\Sum{t'\in E_{1,k+1}}{}\exp t'}\leq \omega(\NR(x)+\gamma+4)a_{k+1}$}
		\end{itemize}
		We conclude thanks to the induction principle.
		
		\item By easy induction, for all $k\in\Nbb$, $a_k<\lambda$.
		
		\item Using (iv), we get that for all $N\in\Nbb$, $\Union{k=0}{N}E_{1,2k}$ is an initial segment of
		$\Union{k\in\Nbb}{}E_{1,2k}$. We also have that $\Union{k=0}{N}E_{1,2k+1}$ is an initial segment of
		$\Union{k\in\Nbb}{}E_{1,2k+1}$. Using (v), we get that $\Union{k\in\Nbb}{}E_{1,2k}$ has order type at most
		\centre{$\sup\enstq{\Oplus{k=0}{N}a_{2k}}{N\in\Nbb} = \sup\enstq{a_{2N}}{N\in\Nbb}\underset{\text{by (vi)}}{\leq}\lambda$}
		Similarly,  $\Union{k\in\Nbb}{}E_{1,2k+1}$ has order type at most $\lambda$. Using Proposition \ref{prop:unionEnsBienOrd}, we conclude that $E_1$ has order type at most~$2\lambda$.
	\end{enumerate}
	Using again proposition \ref{prop:unionEnsBienOrd}, point (vii) above and the properties of $E_2$ and $E_3$ mentioned in the beginning of this proof, we get that $E$ is reverse well-ordered with order type at most $2\lambda+\omega(\gamma+1)$.
\end{proof}

\begin{Cor}
	\label{cor:supportPhi1} Let $x=\aSurreal$ such that
	\centre{$\exists u=\ln_n\kappa_{-\alpha}\quad\exists r\in\Rbb\quad \forall a\in\supp x\quad \exists \eta\prec\ln u\qquad \omega^a=\partial u\exp(r\ln u+\eta)$} 
	Let $\gamma$ be the smallest ordinal such that $\kappa_{-\gamma}\prec^K P(k_P)$ for all path $P\in\Pcal_\Lbb(\eta)$.
	Let $\lambda$ the least $\epsilon$-number greater than $\NR(x)$ and $\gamma$. Then $\Union{\ell=0}{\pinf}\supp\Phi^\ell(x)$ is reverse well-ordered with order type less at most $\omega^{\omega\pa{2\lambda+\omega(\gamma+1)+1}}$
\end{Cor}

\begin{proof}
	Just use Propositions \ref{prop:supportPhi1}, \ref{prop:bonOrderPhi1} and \ref{prop:orderTypeMonoid}.
\end{proof}

\subsubsection{Case $\epsilon\succ\ln u$}

\begin{Lem}
	\label{lem:comparePkPQkQdominantPath}
	Let $x$ be a surreal number. Let $P$ be the dominant path of $x$ and $Q\in\Pcal_\Lbb(x)$. Then, $P(k_P)\succeq^K Q(k_Q)$. In particular, for all ordinal $\beta$, if $\kappa_{-\beta}\succeq^K P(k_P)$, then $\kappa_{-\beta}\succeq^K Q(k_Q)$.
\end{Lem}

\begin{proof}
	\begin{enumerate}[label=(\roman*)]
		\item We first claim that for all $i\in\Nbb$, $P(i)\succeq Q(i)$. We prove it by induction.
		\begin{itemize}
			\item For $i=0$, $P(0)$ is the leading term of $x$ and $Q(0)$ is some term of $x$. Therefore, $P(0)\succeq Q(0)$.
			
			\item Assume $P(i)\succeq Q(i)$. $P(i+1)$ is the leading term of $\ln|P(i)|$. $P(i)$ and $Q(i)$ are both infinitely large. Then $\ln|P(i)|$ and $\ln|Q(i)|$ are both positive infinitely large. If $Q(i+1)\succ P(i+1)$ then, in particular, $\ln |Q(i)|\succ\ln|P(i)|$ what is impossible since $P(i)\succeq Q(i)$. Then $P(i+1)\succeq Q(i+1)$.
		\end{itemize}
		We conclude thanks to induction principle.
		
		\item Take $k=\max(k_P,k_Q)$. Using (i), we have :
		\centre{$P(k_P)\asymp^KP(k)\succeq Q(k)\asymp^K Q(k_Q)$}
		Hence, $P(k_P)\succeq^K Q(k_Q)$.
	\end{enumerate}
\end{proof}

\begin{Lem}
	\label{lem:formeEpsilonPhiOmegaA}
	Assume $x=\omega^a=\partial u\exp\epsilon$ with $\epsilon\succ\ln u$ and $u=\ln_n\kappa_{-\alpha}$. 
	Let $b\in\supp\Phi(\omega^a)$. Then, we have one of theses cases~:
	\begin{itemize}
		\item there is a path $P\in\Pcal(\eta)$ and $i\in\Nbb$ such that
		\centers{$\omega^b \asymp \partial u\exp\pa{\epsilon -\hspace{-2em}\Sum{\footnotesize\begin{array}{c}
						\beta\geq\alpha, m\in\Nbb^*\\ \beta\tq P_0(k_{P_0})\succ^K \kappa_{-\beta}\succeq^KP(k_P)
				\end{array}}{}\hspace{-2em} \ln_m\kappa_{-\beta} + \Sum{j=0}{\pinf}\ln \left|\f{P(i+j)}{P_0(j)}\right|}$}
		\lc{and}{$\forall j\in\intn0{i-1}\qquad P(j)=P_0(j)$}
		
		\item There is some $(\beta,m)<_{lex}(\alpha,n)$ such that there is some $\eta\prec\ln_m\kappa_{-\beta}$ such that
		\centre{$\omega^b\asymp\partial(\ln_m\kappa_{-\beta})\exp\eta$}
		where $\eta=\epsilon+\eta'$ and $\eta'$ only depends on $\alpha,\beta,n,m$ and $P_0$, the dominant path of $\epsilon$ :
		\centre{$\eta' = \Sum{\footnotesize\begin{array}{c}
					(\zeta,p)>_{lex}(\beta,m)\\ \zeta\tq \kappa_{-\zeta}\succeq^KP_0(k_{P_0})
			\end{array}}{}\hspace{-2em}\hspace{-1em} \ln_p\kappa_{-\zeta} - \Sum{(\beta,m)<_{lex}(\zeta,p)<_{lex}(\alpha,n)}{} \hspace{-1em}\ln_p\kappa_{-\zeta} - \Sum{i=0}{\pinf}\ln |P_0(i)|$}
		\lc{or}{$\eta' = \Sum{\footnotesize\begin{array}{c}
					(\zeta,p)\geq_{lex}(\alpha,n)\\ \zeta\tq\kappa_{-\zeta}\succeq^KP_0(k_{P_0})
			\end{array}}{}\hspace{-1em} \ln_p\kappa_{-\zeta}-\Sum{i=0}{\pinf}\ln|P_0(i)|$}
	\end{itemize}
\end{Lem}

\begin{proof}
	\lc{We have}{$\Phi(\omega^a)=\pa{1-\f{\partial\epsilon}s}\omega^a - \partial\pa{\f{\partial u}s}\exp\epsilon $}
	Let $b\in\supp\Phi(\omega^a)$. Then either 
	\centre{$b\in\supp\pa{\pa{1-\f{\partial\epsilon}s}\omega^a}$}
	\lc{or}{$b\in\supp\pa{\partial\pa{\f{\partial u}s}\exp\epsilon}$}
	\begin{itemize}
		\item \underline{First case} : $b\in\supp\pa{\pa{1-\f{\partial\epsilon}s}\omega^a}$. Then there is a path $P$, which is not the dominant path, such that
		\centre{$\omega^b\asymp \f{\partial P}{s}\omega^a
			\asymp \f{\exp\pa{
					-\Sum{\footnotesize\begin{array}{c}
							\beta\geq\alpha, m\in\Nbb^*\\ \beta\tq\kappa_{-\beta}\succeq^KP(k_P)
					\end{array}}{} \ln_m\kappa_{-\beta} + \Sum{i=0}{\pinf}\ln |P(i)|
			}}{
				\exp\pa{
					-\Sum{\footnotesize\begin{array}{c}
							\beta\geq\alpha, m\in\Nbb^*\\ \beta\tq\kappa_{-\beta}\succeq^KP_0(k_{P_0})
					\end{array}}{} \ln_m\kappa_{-\beta} + \Sum{i=0}{\pinf}\ln |P_0(i)|
			}}\omega^a$}
		where $P_0$ is the dominant path of $\epsilon$. Using Lemma \ref{lem:comparePkPQkQdominantPath}, we get
		\centre{$\omega^b\asymp \omega^a\exp\pa{-\Sum{\footnotesize\begin{array}{c}
						\beta\geq\alpha, m\in\Nbb^*\\ \beta\tq P_0(k_{P_0})\succ^K \kappa_{-\beta}\succeq^KP(k_P)
				\end{array}}{} \ln_m\kappa_{-\beta} + \Sum{i=0}{\pinf}\ln \left|\f{P(i)}{P_0(i)}\right|}$}
		
		\item\underline{Second case} : $b\in\supp\pa{\partial\pa{\f{\partial u}s}\exp\epsilon}$. First notice that $\partial\partial u=S_u\partial u$ where 
		\centre{$S_u = -\Sum{\beta<\alpha\  m\in\Nbb^*}{}\exp\pa{-\Sum{\zeta<\beta\ p\in\Nbb^*}{}\ln_p\kappa_{-\zeta} - \Sum{p=1}{m-1}\ln_p\kappa_{-\beta}} - \Sum{m=1}{n-1}\exp\pa{-\Sum{\beta<\alpha\ p\in\Nbb^*}{}\ln_p\kappa_{-\beta} - \Sum{p=1}{m-1}\ln_p\kappa_{-\alpha}}$}
		Hence, if $b\in\supp\pa{\f{\partial\partial u}{s}\exp\epsilon}$, there is some $(\beta,m)<_{lex}(\alpha,n)$ such that
		\centre{$\omega^b\asymp \omega^a\f{\exp\pa{-\Sum{\zeta<\beta\ p\in\Nbb^*}{}\ln_p\kappa_{-\zeta} - \Sum{p=1}{m-1}\ln_p\kappa_{-\beta}}}{
				\exp\pa{-\Sum{\footnotesize\begin{array}{c}
							p\in\Nbb^*\\ \zeta\tq \kappa_{-\zeta}\succeq^KP_0(k_{P_0})
					\end{array}}{} \ln_p\kappa_{-\zeta} + \Sum{i=0}{\pinf}\ln |P_0(i)|}}$}
		\lc{Therefor,}{$\omega^b\asymp\omega^a\exp\pa{\Sum{\footnotesize\begin{array}{c}
						(\zeta,p)\geq_{lex}(\beta,m)\\ \zeta\tq \kappa_{-\zeta}\succeq^KP_0(k_{P_0})
				\end{array}}{} \ln_p\kappa_{-\zeta} - \Sum{i=0}{\pinf}\ln |P_0(i)|}$}
		
		\lc{Notice that}{$\Sum{\footnotesize\begin{array}{c}
					(\zeta,p)\geq_{lex}(\beta,m)\\ \zeta\tq \kappa_{-\zeta}\succeq^KP_0(k_{P_0})
			\end{array}}{} \ln_p\kappa_{-\zeta} \sim \ln_m\kappa_{-\beta}\succ P_0(0)$}
		and then
		\begin{calculs}
			&\omega^a &\asymp& \partial(\ln_m\kappa_{-\beta})\exp\pa{
				\epsilon + \Sum{\footnotesize\begin{array}{c}
						(\zeta,p)>_{lex}(\beta,m)\\ \zeta\tq \kappa_{-\zeta}\succeq^KP_0(k_{P_0})
				\end{array}}{} \ln_p\kappa_{-\zeta}\right.\\&&&\left. \qquad\vphantom{\Sum{\footnotesize\begin{array}{c}
					(\zeta,p)>_{lex}(\beta,m)\\ \zeta\tq \kappa_{-\zeta}\succeq^KP_0(k_{P_0})
			\end{array}}{}}
				- \Sum{(\beta,m)<_{lex}(\zeta,p)<_{lex}(\alpha,n)}{} \ln_p\kappa_{-\zeta} - \Sum{i=0}{\pinf}\ln |P_0(i)|
			}
		\end{calculs}
		
		\lc{Since}{$\epsilon-\Sum{i=0}{\pinf}\ln|P(i)| \sim\epsilon\prec\ln_m\kappa_{-\beta}$}
		\lc{and}{$\Sum{\footnotesize\begin{array}{c}
					(\zeta,p)>_{lex}(\beta,m)\\ \zeta\tq \kappa_{-\zeta}\succeq^KP_0(k_{P_0})
			\end{array}}{} \ln_p\kappa_{-\zeta} - \Sum{(\beta,m)<_{lex}(\zeta,p)<_{lex}(\alpha,n)}{} \ln_p\kappa_{-\zeta}\prec\ln_m\kappa_{-\beta}$}
		Moreover,
		\centre{$\NR\pa{\ln\partial(\ln_m\kappa_{-\beta}) + \epsilon -\Sum{(\beta,m)<_{lex}(\zeta,p)<_{lex}(\alpha,n)}{} \ln_p\kappa_{-\zeta}}\leq \NR(x)$}
		and using Proposition \ref{prop:majorationNRPartial},
		\begin{calculs}
			&\NR\pa{\Sum{\footnotesize\begin{array}{c}
						(\zeta,p)>_{lex}(\beta,m)\\ \zeta\tq \kappa_{-\zeta}\succeq^KP_0(k_{P_0})
				\end{array}}{} \hspace{-2em}\ln_p\kappa_{-\zeta}- \Sum{i=0}{\pinf}\ln |P_0(i)|}&\leq& \NR(\partial P_0)\\
			&&\leq& k_{P_0}(\NR(x)+1)+\omega(\gamma+1)
		\end{calculs}
		We then conclude that there is some $\eta\prec\ln_m\kappa_{-\beta}$ such that
		\centre{$\omega^b = \partial(\ln_m\kappa_{-\beta})\exp\eta$}
		and by Corollary \ref{cor:NRprod},
		\centre{$\NR(\omega^b)\leq(k_{P_0}+1)(\NR(x)+1) + \omega(\gamma+1)$}
		
		Now assume $b\in\supp\pa{\f{\partial s}{s^2}\omega^a}$. 
		Notice that
		\begin{calculs}
			& \partial s &=& s\pa{	-\Sum{\footnotesize\begin{array}{c}
						m\in\Nbb^*\\ \beta\tq\kappa_{-\beta}\succeq^KP_0(k_{P_0})
				\end{array}}{} \partial\ln_m\kappa_{-\beta} + \Sum{i=0}{\pinf}\partial\ln |P_0(i)|}\\
			&&=& s\pa{-\Sum{\footnotesize\begin{array}{c}
						m\in\Nbb^*\\ \beta\tq\kappa_{-\beta}\succeq^KP_0(k_{P_0})
				\end{array}}{}\exp\pa{-\Sum{\zeta<\beta\ p\in\Nbb^*}{}\ln_p\kappa_{-\zeta} - \Sum{p=1}{m-1}\ln_p\kappa_{-\beta}} \right.\\ &&&\left. \vphantom{\Sum{\footnotesize\begin{array}{c}
					m\in\Nbb^*\\ \beta\tq\kappa_{-\beta}\succeq^KP_0(k_{P_0})
			\end{array}}{}}
				\qquad+\Sum{i=0}{\pinf}\partial\ln |P_0(i)|}
		\end{calculs}
		We then have the following sub-cases :
		\begin{itemize}
			\item There is some $m\in\Nbb^*$ and some ordinal $\beta$ such that $\kappa_{-\beta}\succeq^K P_0(k_{P_0})$ such that
			\begin{calculs}
				& \omega^b &\asymp& \f{\exp\pa{-\Sum{\zeta<\beta\ 				
							p\in\Nbb^*}{}\ln_p\kappa_{-\zeta} - \Sum{p=1}{m-1}\ln_p\kappa_{-\beta}}}{\exp\pa{-\Sum{\footnotesize\begin{array}{c}
								p\in\Nbb^*\\ \zeta\tq\kappa_{-\zeta}\succeq^KP_0(k_{P_0})
						\end{array}}{} \ln_p\kappa_{-\zeta}+\Sum{i=0}{\pinf}\ln|P_0(i)|}}\omega^a\\
				&&\asymp& \partial(\ln_m\kappa_{-\beta})\exp\pa{
					\epsilon +\Sum{\footnotesize\begin{array}{c}
							(\zeta,p)\geq_{lex}(\alpha,n)\\ \zeta\tq\kappa_{-\zeta}\succeq^KP_0(k_{P_0})
					\end{array}}{} \ln_p\kappa_{-\zeta}-\Sum{i=0}{\pinf}\ln|P_0(i)|}
			\end{calculs}
			with
			\centre{$\epsilon - \Sum{\footnotesize\begin{array}{c}
						\zeta\geq\alpha\ p\in\Nbb^*\\ \zeta\tq\kappa_{-\zeta}\succeq^KP_0(k_{P_0})
				\end{array}}{} \ln_p\kappa_{-\zeta}-\Sum{i=0}{\pinf}\ln|P_0(i)|\sim\epsilon\prec\ln_m\kappa_{-\beta}$}
			
			We then conclude that there is some $\eta\prec\ln_m\kappa_{-\beta}$ such that
			\centre{$\omega^b = \partial(\ln_m\kappa_{-\beta})\exp\eta$}
			and by Corollary \ref{cor:NRprod},
			\centre{$\NR(\omega^b)\leq(k_{P_0}+1)(\NR(x)+1) + \omega(\gamma+1)$}
			
			\item There is some path $P\in\Pcal_\Lbb(\epsilon)$ and some $i\geq 1$ such that for all $j<i$, $P(j)=P_0(j)$ and
			\centre{$\omega^b\asymp \f{\exp\pa{-\Sum{\footnotesize\begin{array}{c}
								p\in\Nbb^*\\ \zeta\tq\kappa_{-\zeta}\succeq^KP(k_P)
						\end{array}}{} \ln_p\kappa_{-\zeta}+\Sum{j=i}{\pinf}\ln|P(j)|}}{\exp\pa{-\Sum{\footnotesize\begin{array}{c}
								p\in\Nbb^*\\ \zeta\tq\kappa_{-\zeta}\succeq^KP_0(k_{P_0})
						\end{array}}{} \ln_p\kappa_{-\zeta}+\Sum{j=0}{\pinf}\ln|P_0(j)|}}\omega^a$}
			As in the first case, we get
			\centre{$\omega^b\asymp \omega^a\exp\pa{-\hspace{-1em}\Sum{\footnotesize\begin{array}{c}
							\beta\geq\alpha, m\in\Nbb^*\\ \beta\tq P_0(k_{P_0})\succ^K \kappa_{-\beta}\succeq^KP(k_P)
					\end{array}}{} \hspace{-3em}\ln_m\kappa_{-\beta} + \Sum{j=0}{\pinf}\ln \left|\f{P(i+j)}{P_0(j)}\right|}$}
		\end{itemize}
	\end{itemize}
\end{proof}

\begin{Prop} \label{prop:supportPhi2}
	Assume $x=\omega^a=\partial u\exp\epsilon$ with $\epsilon\succ\ln u$ and $u=\ln_n\kappa_{-\alpha}$. 
	Let $P_0$ be the dominant path of $\epsilon$.
	We denote for $P_1,\dots,P_{k+k'}\in\Pcal_\Lbb(\epsilon)$, $i_1,\dots,i_{k+k'}\in\Nbb^*$ and $(\beta,m)\leq_{lex}(\alpha,n)$,
	\begin{calculs}
		&e^{(\beta,m)}\left(\begin{matrix}
			P_1,\dots,P_k\\
			P_{k+1},\dots, P_{k+k'}\\
			i_1,\dots,i_{k+k'}
		\end{matrix}\right) &=&
		-k\Sum{i=0}{\infty}\ln|P_0(i)| + \Sum{j=1}{k}\Sum{i=i_j}{\infty}\ln|P_j(i)|\\ &&&
		-\Sum{j=1}{k}\Sum{\footnotesize\begin{array}{c}
				\gamma\geq\alpha, \ell\in\Nbb^*\\ \gamma\tq P_0(k_{P_0})\succ^K \kappa_{-\gamma}\succeq^KP_j(k_{P_j})
		\end{array}}{}\hspace{-2em} \ln_\ell\kappa_{-\gamma} \\ &&&
		-k'\Sum{\ell=m+2}{\pinf}\ln_\ell{\kappa_{-\beta}} + \Sum{j=k+1}{k+k'}\Sum{i=i_j}{\pinf}\ln |P_j(i)|\\ &&&
		-\Sum{j=k+1}{k+k'}\Sum{\footnotesize\begin{array}{c}
				\gamma>\beta, \ell\in\Nbb^*\\ \gamma\tq\kappa_{-\gamma}\succeq^KP_j(k_{P_j})
		\end{array}}{} \ln_\ell\kappa_{-\gamma}
		
	\end{calculs}
	We now define $E_{1,k,k'}^{(\beta,m)}$ as follows:
	\begin{calculs}
		&&& e^{(\beta,m)}\left(\begin{matrix}
			P_1,\dots,P_k\\
			P_{k+1},\dots, P_{k+k'}\\
			i_1,\dots,i_{k+k'}
		\end{matrix}\right)\in E_{1,k,k'}^{(\beta,m)}\\&&\Leftrightarrow&
			P_1,\dots,P_k\in\Pcal_\Lbb(\epsilon)\setminus\{P_0\}\\
		&&&\wedge\quad P_{k+1},\dots,P_{k+k'}\in\Pcal_\Lbb(\epsilon)\\
		&&&\wedge\quad i_1,\dots,i_k\in\Nbb\\
		&&&\wedge\quad i_{k+1},\dots,i_{k+k'}\in\Nbb^*\\
		&&&\wedge\quad 	\forall j\in\intn{1}{k+k'}\ \exists j'\in\intn 0{j-1}\\ &&&\qquad\qquad \forall i\in\intn0{i_j-1}\quad P_{j'}(i)=P_j(i)\\
		&&&\wedge\quad \forall j \in \intn{k+1}{k+k'}\\
			&&&\qquad\qquad\supp P_j(i_j)\subseteq\supp e^{(\beta,m)}\left(\begin{matrix}
			P_1,\dots,P_k\\ P_{k+1},\dots,P_j\\ i_1,\dots, i_j
		\end{matrix}\right)
	\end{calculs}
	\centre{$E_1^{(\beta,m)} = \begin{accolade}
			\Union{k\in\Nbb,\ k'\in\Nbb^*}{} E_{1,k,k'}^{(\beta,m)} & (\beta,m)\neq (\alpha,n)\\
			\Union{k\in\Nbb}{} E_{1,k,0}^{(\beta,m)} & (\beta,m)=(\alpha,n) 
		\end{accolade}$}
	Define sets $E_2^{(\beta,m)}$ as follows:
	\begin{itemize}
		\item If $(\beta,m)\neq (\alpha,n)$, then
		\centre{$
				-\Sum{\ell=m+2}{\pinf}\ln_\ell\kappa_{-\beta}
				-\hspace{-1em}\Sum{\gamma'<\gamma,\ \ell\in\Nbb^*}{}\hspace{-1em}\ln_\ell\kappa_{-\gamma'} - \Sum{\ell=1}{p}\ln_\ell\kappa_{-\gamma} \in  E_2^{(\beta,m)} $}
		\tiff $\gamma>\beta$, $p\in\Nbb$ and there is some $P\in\Pcal_\Lbb(\epsilon)$ such that $\kappa_{-\gamma}\succeq^K P(k_P)$
		\item If $(\beta,m)=(\alpha,n)$, then
		\centre{$-\Sum{j=0}{\infty}\ln|P_0(j)| - \Sum{\gamma>\zeta>\alpha,\ \ell\in\Nbb^*}{}\ln_\ell\kappa_{-\zeta} - \Sum{\ell=1}{p}\ln_\ell\kappa_{-\gamma} \in  E_2^{(\beta,m)}$}
		\tiff $\gamma>\alpha$, $p\in\Nbb$ and there is some $P\in\Pcal_\Lbb(\epsilon)$ such that $\kappa_{-\gamma}\succeq^K P(k_P)$.
	\end{itemize}
	Let also
	\begin{calculs}
		&E_3^{(\beta,m)} &=& \begin{accolade}
			\enstq{-\Sum{\ell=m+2}{p}\ln_\ell{\kappa_{-\beta}}}{p\geq m+2} & (\beta,m)\neq(\alpha,n)\\
			\emptyset & (\beta,m)=(\alpha,n)
		\end{accolade} \\ [.4cm]
		& E^{(\beta,m)} &=& E_1^{(\beta,m)}\cup E_2^{(\beta,m)}\cup E_3^{(\beta,m)}
	\end{calculs}
	and $\inner {E^{(\beta,m)}}$ be the monoid it generates. Finally, let $H^{(\beta,m)}$ defined by cases as follows:
	\centre{$\begin{accolade}
			\left\{\Sum{\footnotesize\begin{array}{c}
					(\zeta,p)>_{lex}(\beta,m)\\ \zeta\tq \kappa_{-\zeta}\succeq^KP_0(k_{P_0})
			\end{array}}{} \ln_p\kappa_{-\zeta}\hspace{8em}\right.\\ - \Sum{(\beta,m)<_{lex}(\zeta,p)<_{lex}(\alpha,n)}{} \hspace{-2em}\ln_p\kappa_{-\zeta} - \Sum{i=0}{\pinf}\ln |P_0(i)|,\\
			\left.\Sum{\footnotesize\begin{array}{c}
					(\zeta,p)\geq_{lex}(\alpha,n)\\ \zeta\tq\kappa_{-\zeta}\succeq^KP_0(k_{P_0})
			\end{array}}{} \hspace{-2em}\ln_p\kappa_{-\zeta}-\Sum{i=0}{\pinf}\ln|P_0(i)|\right\} & (\beta,m)\neq (\alpha,n) \\
			\{0\} & (\beta,m)=(\alpha,n)
		\end{accolade}$}
	Let $b\in\Union{q=0}{\pinf}\supp\Phi^q(\omega^a)$. Then, there are $\eta\in H^{(\beta,m)}$ and $y\in \inner {E^{(\beta,m)}}$ such that
	\centre{$\omega^b\asymp\partial (\ln_m\kappa_{-\beta})\exp(\epsilon + \eta + y)$}
\end{Prop}

\begin{proof}
	We prove it by induction on $q$.
	\begin{itemize}
		\item If $b\in\supp\Phi^0(\omega^a)$, then $b=a$ and $y=0$ with $(\beta,m)=(\alpha,n)$ and $\eta=0$ works.
		
		\item Assume the property for some $q\in\Nbb$. Let $b\in\supp\Phi^{q+1}(\omega^b)$. Then there is 
		$c\in\supp\Phi^q(\omega^a)$ such that $b\in\supp\Phi(\omega^c)$. Apply the induction hypothesis on $c$. Take $(\beta,m)$, $\eta\in H^{(\beta,m)}$ and $y\in\inner{E^{(\beta,m)}}$ such that
		\centre{$\omega^c\asymp\partial(\ln_m\beta)\exp(\epsilon+\eta+y)$}
		\begin{itemize}
			\item If $(\beta,m)<_{lex}(\alpha,n)$, then $y,\epsilon\prec\ln_{n+1}\kappa_{-\beta}$. Hence, using Lemma \ref{lem:formeEtaPhiOmegaA}, we get that there is $P\in\Pcal_\Lbb(\epsilon+\eta+y)$ such that
			\begin{calculs}
				&\omega^b &\asymp& \partial(\ln_m\kappa_{-\beta})\exp\pa{\epsilon+\eta+y
					-\Sum{\ell=m+2}{\pinf}\ln_\ell\kappa_{-\beta} \vphantom{\Sum{\footnotesize\begin{array}{c}
								\gamma>\beta, \ell\in\Nbb^*\\ \gamma\tq\kappa_{-\gamma}\succeq^KP(k_P)
						\end{array}}{}}\right. \\ &&& \left.\qquad
					-\hspace{-2em}\Sum{\footnotesize\begin{array}{c}
							\gamma>\beta, \ell\in\Nbb^*\\ \gamma\tq\kappa_{-\gamma}\succeq^KP(k_P)
					\end{array}}{}\hspace{-2em} \ln_\ell\kappa_{-\gamma} + \Sum{i=0}{\pinf}\ln |P(i)|}
			\end{calculs}
			If $P(0)$ a term of $\epsilon$, up to some real factor, then there is a real number $s$ and some $e\in E_{1,0,1}^{(\beta,m)}$ such that
			\begin{calculs}
				 & s\exp e &=& \exp\pa{-\Sum{\ell=m+2}{\pinf}\ln_\ell\kappa_{-\beta}
				 	-\hspace{-2em}\Sum{\footnotesize\begin{array}{c}
				 			\gamma>\beta, \ell\in\Nbb^*\\ \gamma\tq\kappa_{-\gamma}\succeq^KP(k_P)
				 	\end{array}}{}\hspace{-2em} \ln_\ell\kappa_{-\gamma} + \Sum{i=0}{\pinf}\ln |P(i)|}
			\end{calculs}
			Then $y+e\in \inner {E^{(\beta,m)}}$ and $\omega^b\asymp\partial(\ln_m\kappa_{-\beta})\exp(\epsilon+y+e)$. If not, then $P(0)$ is a term
			of $\eta+y$. Hence, we have the following cases:
			\begin{itemize}
				\item $P(0)=s\ln_p\kappa_{-\beta}$ for some $s\in\Rbb^*_-$ and $p\geq m+2$. Then,
				\begin{calculs}
					&&&-\Sum{\ell=m+2}{\pinf}\ln_\ell\kappa_{-\beta}
					-\Sum{\footnotesize\begin{array}{c}
							\gamma>\beta, \ell\in\Nbb^*\\ \gamma\tq\kappa_{-\gamma}\succeq^KP(k_P)
					\end{array}}{} \ln_\ell\kappa_{-\gamma} + \Sum{i=0}{\pinf}\ln |P(i)| \\
					&&=&  \ln|s| - \Sum{\ell=m+2}{p}\ln_\ell\kappa_{-\beta} \in \ln|s|+E_3^{(\beta,m)}
				\end{calculs}
				Then,
				\centre{$y-\Sum{\ell=m+2}{\pinf}\ln_\ell\kappa_{-\beta}
					-\hspace{-2em}\Sum{\footnotesize\begin{array}{c}
							\gamma>\beta, \ell\in\Nbb^*\\ \gamma\tq\kappa_{-\gamma}\succeq^KP(k_P)
					\end{array}}{}\hspace{-2em} \ln_\ell\kappa_{-\gamma} + \Sum{i=0}{\pinf}\ln |P(i)| \in\Rbb+\inner {E^{(\beta,m)}}$}
				
				\item $P(0)=s\ln_p\kappa_{-\gamma}$ with $\gamma>\beta$ and $p\in\Nbb^*$ such that there is some path $Q\in\Pcal_\Lbb(\epsilon)$ such that $\kappa_{-\beta}\succeq^K Q(k_Q)$. Then
				\centre{$-\Sum{\ell=m+2}{\pinf}\ln_\ell\kappa_{-\beta}
					-\hspace{-2em}\Sum{\footnotesize\begin{array}{c}
							\gamma>\beta, \ell\in\Nbb^*\\ \gamma\tq\kappa_{-\gamma}\succeq^KP(k_P)
					\end{array}}{} \hspace{-2em}\ln_\ell\kappa_{-\gamma} + \Sum{i=0}{\pinf}\ln |P(i)| \in \ln|s| + E_2^{(\beta,m)}$}
				Then,
				\centre{$y-\Sum{\ell=m+2}{\pinf}\ln_\ell\kappa_{-\beta}
					-\Sum{\footnotesize\begin{array}{c}
							\gamma>\beta, \ell\in\Nbb^*\\ \gamma\tq\kappa_{-\gamma}\succeq^KP(k_P)
					\end{array}}{} \ln_\ell\kappa_{-\gamma} + \Sum{i=0}{\pinf}\ln |P(i)| \in\Rbb+\inner {E^{(\beta,m)}}$}
				
				\item There are some paths $P_1,\dots,P_{k+k'}\in \Pcal_\Lbb(\epsilon)$ and some integers $i_1,\dots,i_{k+k'}$ such that 
				\centers{$e^{(\beta,m)}\left(\begin{matrix}
						P_1,\dots,P_k \\ P_{k+1},\dots,P_{k+k'}\\ i_1,\dots,i_{k+k'}
					\end{matrix}\right)\in E_{1,k,k'}^{(\beta,m)}$}
				\lc{and}{$\exists y'\in\inner E\qquad y=y'+e^{(\beta,m)}\left(\begin{matrix}
						P_1,\dots,P_k \\ P_{k+1},\dots,P_{k+k'}\\ i_1,\dots,i_{k+k'}
					\end{matrix}\right)$}
				and finally such that $P(0)\in\Rbb z$ for some $z$ a term of some $\ln|P_j({i_{k+k'+1}}')|$ with $j\in\intn0{k+k'}$ and ${i_{k+k'+1}}'\geq i_j$. 
				Let $P_{k+k'+1}$ be the following path :
				\centre{$P_{k+k'+1}(i) = \begin{accolade}
						P_j(i) & i\leq {i_{k+1}}' \\ z & i={i_{k+1}}'+1 \\ P(i-{i_{k+1}}'-1) & i>{i_{k+1}}'+1 
					\end{accolade}$}
				Then, $P_{k+k'+1}\in\Pcal(\epsilon)$. Moreover, 
				$$\partial P_{k+k'+1}=\underbrace{P_j(0)\cdots P_j({i_{k+1}}')}_{\neq 0}\underbrace{\partial P}_{\neq 0}$$
				Then $P_{k+k'+1}\in\Pcal_\Lbb(\epsilon)$. Note also that for all $\beta$, 
				
				\centre{$\kappa_{-\beta}\succeq^K P_{k+k'+1}(k_{P_{k+k'+1}})\Longleftrightarrow\kappa_{-\beta}\succeq^KP(k_{P})$}
				Finally,
				
				\begin{calculs}
					&&&-\Sum{\ell=m+2}{\pinf}\ln_\ell\kappa_{-\beta}
					-\Sum{\footnotesize\begin{array}{c}
							\gamma>\beta, \ell\in\Nbb^*\\ \gamma\tq\kappa_{-\gamma}\succeq^KP(k_P)
					\end{array}}{} \ln_\ell\kappa_{-\gamma} + \Sum{i=0}{\pinf}\ln |P(i)|\\
					&&=&-\Sum{\ell=m+2}{\pinf}\ln_\ell\kappa_{-\beta}	-\Sum{\footnotesize\begin{array}{c}
							\gamma>\beta, \ell\in\Nbb^*\\ \gamma\tq\kappa_{-\gamma}\succeq^KP_{k+k'+1}(k_{P_{k+k'+1}})
					\end{array}}{} \ln_\ell\kappa_{-\gamma}\\ &&&\qquad\qquad + \Sum{i={i_{k+1}}'+1}{\pinf}\ln |P_{k+k'+1}(i)| + \ln\underbrace{\left|\f{P(0)}{z}\right|}_{\in\Rbb^*_+}\\
				\end{calculs}
				From that we derive that
				\begin{calculs}
					&&& y-\Sum{\ell=m+2}{\pinf}\ln_\ell\kappa_{-\beta}
					-\Sum{\footnotesize\begin{array}{c}
							\gamma>\beta, \ell\in\Nbb^*\\ \gamma\tq\kappa_{-\gamma}\succeq^KP(k_P)
					\end{array}}{} \ln_\ell\kappa_{-\gamma} + \Sum{i=0}{\pinf}\ln |P(i)| \\
					&&=& y' +  e^{(\beta,m)}\left(\begin{matrix}
						P_1,\dots,P_k \\ P_{k+1},\dots,P_{k+k'+1}\\ i_1,\dots,i_{k+k'+1}
					\end{matrix}\right)
					+\ln\left|\f{P(0)}{z}\right| \in\Rbb + \inner {E^{(\beta,m)}}
				\end{calculs}
				where $i_{k+k'+1}={i_{k+k'+1}}'+1$ and $P_{k+k'+1}(i_{k+k'})=z$ has indeed its support (which is reduced to a singleton) included
				in the one of $e^{(\beta,m)}\left(\begin{matrix}
					P_1,\dots,P_k\\ P_{k+k'},\dots,P_{k+k'}\\ i_1,\dots, i_{k+k'}
				\end{matrix}\right)$.
			\end{itemize}
			Then there is a real number $s$, and $e\in\inner{E^{(\beta,m)}}$ such that 
			\centre{$\omega^b\asymp\partial (\ln_m\beta)\exp(\epsilon + \eta + e + s)\asymp \partial (\ln_m\beta)\exp(\epsilon + \eta + e)$}
			Then we get the property at rank $q+1$.
			
			
			\item If $(\beta,m)=(\alpha,n)$, we have $\eta=0$ and write
			\centre{$y=y'+e^{(\alpha,n)}\left(\begin{matrix}
					P_1,\dots, P_k\\ \emptyset\\ i_1,\dots,i_{k+k'}
				\end{matrix}\right)$}
			with, $y'\in\inner{E^{(\beta,m)}}$ and, $k,k'\in\Nbb$.
			Using Lemma \ref{lem:formeEpsilonPhiOmegaA}, we have
			
			\begin{itemize}
				\item \lc{\underline{First case} :}{$\omega^b\asymp\partial u\exp(\epsilon + y + e)$}
				where
				\centre{$e=-\Sum{\footnotesize\begin{array}{c}
							\gamma\geq\alpha, \ell\in\Nbb^*\\ \gamma\tq P_0(k_{P_0})\succ^K \kappa_{-\gamma}\succeq^KP(k_P)
					\end{array}}{} \ln_\ell\kappa_{-\gamma} + \Sum{j=0}{\pinf}\ln \left|\f{P(i+j)}{P_0(j)}\right|$}
				for some path $P\in\Pcal_\Lbb(\epsilon + y)$ and some $i\in\Nbb$ such that
				\centre{$\forall j\in\intn0{i-1}\qquad P(j)=P_0(j)$}
				Indeed, $y\in\inner{E^{(\alpha,n)}}$. In particular, $y\prec\epsilon$ and then $\epsilon+y\sim\epsilon$ so that $P_0$ is also the dominant path of $\epsilon+y$. 
				
				\begin{itemize}
					\item If $P(0)$ is, up to a real factor, a term of $\epsilon$, 
					then we get that there is some path $Q\in\Pcal_\Lbb(\epsilon)$ and a real number $s$ such that 
					\centre{$y+e=y'+e^{(\beta,m)}\left(\begin{matrix}
							P_1,\dots,P_k,Q\\ \emptyset\\ i_1,\dots, i_k,i
						\end{matrix}\right)+s$}
					Since $y\prec\epsilon$, and $P\neq P_0$, we also have $Q\neq P_0$. Then $y+e\in \inner{E^{(\beta,m)}}+E_{1,k+1,k'}^{(\beta,m)}+s$. Let  \centre{$y''=y+e-s\in\inner{E^{(\beta,m)}}$} 
					\lc{then,}{$\omega^b\asymp\partial u\exp(\epsilon + y'')$}
					In particular, $y''\prec\epsilon$.
					
					\item If $P(0)$ is a term of $y$, and more precisely if it can be written as $P(0)=s\ln_p\kappa_{-\gamma}$ for $s\in\Rbb$, $p\in\Nbb$ and $\gamma\geq\alpha$ such that 
					\centre{$P_0(k_{P_0})\succ^K \kappa_{-\gamma}\succeq^K Q(k_Q)$}
					for some path $Q\in\Pcal_\Lbb(\epsilon)\setminus\{P_0\}$. Then,
					\centre{$e=-\Sum{j=0}{\infty}\ln|P_0(j)| - \Sum{\gamma>\zeta>\alpha,\ \ell\in\Nbb^*}{}\ln_\ell\kappa_{-\zeta} - \Sum{\ell=1}{p+i}\ln_\ell\kappa_{-\gamma} + \ind_{i=0}\ln|s| \in E_2^{(\beta,m)}+\Rbb$}
					Then $y+e-\ln|s|\in\inner{E^{(\beta,m)}}$ and  since $e\prec\epsilon$, $y+e-s\prec\epsilon$ and 
					\centre{$\omega^b\asymp\partial u\exp(\epsilon+y+e-\ln|s|)$}
					
					\item If $P(0)$ is a term of $y$, and more precisely if it can be written as $P(0)=s\ln|P_\ell(j)|$ for some $s\in\Rbb$ and some $\ell\in\intn0{k+k'}$ (actually it is true if we have chosen well the $y'$ in the beginning, but up to a renaming, it is true). Consider the following path
					\centre{$ Q(p) = \begin{accolade}
							P_\ell(p) & p\leq j\\ P(p-j) & p>j
						\end{accolade}$}
					We have $Q\in\Pcal_\Lbb(\epsilon)$ and
					\centre{$y+e=y'+e^{(\beta,m)}\left(\begin{matrix}
							P_1,\dots,P_k,Q\\ \emptyset\\ i_1,\dots,i_{k},j
						\end{matrix}\right) + \ln|s|$}
					Then $y+e-\ln|s|\in\inner{E^{(\beta,m)}}$ and  since $e\prec\epsilon$, $y+e-s\prec\epsilon$ and 
					\centre{$\omega^b\asymp\partial u\exp(\epsilon+y+e-\ln|s|)$}
				\end{itemize}
				This concludes the first case.
				
				\item \underline{Second case} : There are $(\beta',m')<_{lex}(\alpha,n)$ and $\eta'\in H^{(\beta,m)}$ such that $\omega^b\asymp\partial(\ln_{m'}\kappa_{-\beta'})\exp(\epsilon+\eta'+y)$. This immediately conclude the second case.
			\end{itemize}
			We then have the property at rank $q+1$.
		\end{itemize}
	\end{itemize}
	Thanks to the induction principle, we conclude that the property holds for any $q\in\Nbb$.
\end{proof}


\begin{corollary} \label{cor:propsupportPhi2} Let $x$ be a surreal number such that
	\centre{$\exists u=\ln_n\kappa_{-\alpha}\quad \exists r_0\in\Rbb^* \quad \exists a_0\in\Nobf \quad \forall a\in\supp x\quad\exists\epsilon\sim r\omega^{a_0}\succ\ln u\qquad \omega^a\asymp\partial u\exp\epsilon$}
	\lc{Let}{$\Pcal_0(x)=\enstq{P\in\Pcal_\Lbb(x)}{\begin{array}{c}
				P(1)=r\omega^{a_0} \\ \forall i\geq 1 \quad P(i+1)\sim\ln|P(i)
		\end{array}}$}
	It is the set of all the possible dominant paths of the epsilon to which we add the corresponding term of $x$ at the beginning.
	We denote for $P_0\in\Pcal_0(x)$, $P_1,\dots,P_{k+k'}\in\Pcal_\Lbb(x)$, $i_1,\dots,i_{k+k'}\in\Nbb^*$ and $(\beta,m)\leq_{lex}(\alpha,n)$,
	\begin{calculs}
		&e^{(\beta,m)}\left(\begin{matrix}
			P_0; P_1,\dots,P_k\\
			P_{k+1},\dots, P_{k+k'}\\
			i_1,\dots,i_{k+k'}
		\end{matrix}\right) &=&
		-k\Sum{i=1}{\infty}\ln|P_0(i)| -k'\Sum{\ell=m+2}{\pinf}\ln_\ell{\kappa_{-\beta}}\\ &&& 
		-\Sum{j=1}{k}\hspace{-1em}\Sum{\scriptsize\begin{array}{c}
				\gamma\geq\alpha, \ell\in\Nbb^*\\ \gamma\tq P_0(k_{P_0})\succ^K \kappa_{-\gamma}\succeq^KP_j(k_{P_j})
		\end{array}}{}\hspace{-3em} \ln_\ell\kappa_{-\gamma}\\
		&&&
		+ \Sum{j=1}{k}\Sum{i=i_j}{\infty}\ln|P_j(i)|\\
		&&&	
		-\Sum{j=k+1}{k+k'}\Sum{\footnotesize\begin{array}{c}
				\gamma>\beta, \ell\in\Nbb^*\\ \gamma\tq\kappa_{-\gamma}\succeq^KP_j(k_{P_j})
		\end{array}}{}\hspace{-2em} \ln_\ell\kappa_{-\gamma}\\
		&&&
		+ \Sum{j=k+1}{k+k'}\Sum{i=i_j}{\pinf}\ln |P_j(i)|
	\end{calculs}

	We now define $E_{1,k,k'}^{(\beta,m)}$ as follows:
	\begin{calculs}
		&&& e^{(\beta,m)}\left(\begin{matrix}
			P_0;P_1,\dots,P_k\\
			P_{k+1},\dots, P_{k+k'}\\
			i_1,\dots,i_{k+k'}
		\end{matrix}\right)\in E_{1,k,k'}^{(\beta,m)} \\&&\Leftrightarrow&
		P_0\in\Pcal_0(x) \wedge P_1,\dots,P_k\in\Pcal_\Lbb(\epsilon)\setminus\{P_0\}\\
		&&&\wedge\quad P_{k+1},\dots,P_{k+k'}\in\Pcal_\Lbb(x)\\
		&&&\wedge\quad i_1,\dots,i_k\in\Nbb^*\\
		&&&\wedge\quad i_{k+1},\dots,i_{k+k'}\in\Nbb\setminus \{0,1\}\\
		&&&\wedge\quad 	\forall j\in\intn{1}{k+k'}\ \exists j'\in\intn 0{j-1}\\ &&&\qquad\qquad \forall i\in\intn0{i_j-1}\quad P_{j'}(i)=P_j(i)\\
		&&&\wedge\quad \forall j \in \intn{k+1}{k+k'}\\
		&&&\qquad\qquad\supp P_j(i_j)\subseteq\supp e^{(\beta,m)}\left(\begin{matrix}
			P_0; P_1,\dots,P_k\\ P_{k+1},\dots,P_j\\ i_1,\dots, i_j
		\end{matrix}\right)
	\end{calculs}
	\centre{$E_1^{(\beta,m)} = \begin{accolade}
			\Union{k\in\Nbb,\ k'\in\Nbb^*}{} E_{1,k,k'}^{(\beta,m)} & (\beta,m)\neq (\alpha,n)\\
			\Union{k\in\Nbb}{} E_{1,k,0}^{(\beta,m)} & (\beta,m)=(\alpha,n) 
		\end{accolade}$}
	Define sets $E_2^{(\beta,m)}$ as follows:
	\begin{itemize}
		\item If $(\beta,m)\neq (\alpha,n)$, then
		\centre{$
			-\Sum{\ell=m+2}{\pinf}\ln_\ell\kappa_{-\beta}
			-\hspace{-1em}\Sum{\gamma'<\gamma,\ \ell\in\Nbb^*}{}\hspace{-1em}\ln_\ell\kappa_{-\gamma'} - \Sum{\ell=1}{p}\ln_\ell\kappa_{-\gamma} \in  E_2^{(\beta,m)} $}
		\tiff $\gamma>\beta$, $p\in\Nbb$ and there is some $P\in\Pcal_\Lbb(\epsilon)$ such that $\kappa_{-\gamma}\succeq^K P(k_P)$
		\item If $(\beta,m)=(\alpha,n)$, then
		\centre{$-\Sum{j=0}{\infty}\ln|P_0(j)| - \Sum{\gamma>\zeta>\alpha,\ \ell\in\Nbb^*}{}\ln_\ell\kappa_{-\zeta} - \Sum{\ell=1}{p}\ln_\ell\kappa_{-\gamma} \in  E_2^{(\beta,m)}$}
		\tiff $\gamma>\alpha$, $p\in\Nbb$, $P_0\in\Pcal_0(x)$, $P_0(k_{P_0}) \succ^K\kappa_{-\gamma}$ and there is some $P\in\Pcal_\Lbb(\epsilon)$ such that $\kappa_{-\gamma}\succeq^K P(k_P)$.
	\end{itemize}

	Let also
	\begin{calculs}
		&E_3^{(\beta,m)} &=& \begin{accolade}
			\enstq{-\Sum{\ell=m+2}{p}\ln_\ell{\kappa_{-\beta}}}{p\geq m+2} & (\beta,m)\neq(\alpha,n)\\
			\emptyset & (\beta,m)=(\alpha,n)
		\end{accolade} \\ [.4cm]
		& E^{(\beta,m)} &=& E_1^{(\beta,m)}\cup E_2^{(\beta,m)}\cup E_3^{(\beta,m)}
	\end{calculs}
	and $\inner {E^{(\beta,m)}}$ be the monoid it generates. Finally, let
	$H^{(\beta,m)}$ defined by cases as follows:
	\begin{itemize}
		\item If $(\beta,m)\neq (\alpha,n)$, then
		\begin{calculs}
			& H^{(\beta,m)} &=& \enstq{\begin{array}{r}
					\Sum{\footnotesize\begin{array}{c}
						(\zeta,p)>_{lex}(\beta,m)\\ \zeta\tq \kappa_{-\zeta}\succeq^KP_0(k_{P_0})
				\end{array}}{}\hspace{-2em} \ln_p\kappa_{-\zeta} - \Sum{(\beta,m)<_{lex}(\zeta,p)<_{lex}(\alpha,n)}{} \hspace{-2em}\ln_p\kappa_{-\zeta}\\ - \Sum{i=0}{\pinf}\ln |P_0(i)|\end{array}}{P_0\in\Pcal_0(x)}\\
			&&&\qquad\bigcup\enstq{\Sum{\footnotesize\begin{array}{c}
						(\zeta,p)\geq_{lex}(\alpha,n)\\ \zeta\tq\kappa_{-\zeta}\succeq^KP_0(k_{P_0})
				\end{array}}{}\hspace{-2em} \ln_p\kappa_{-\zeta}-\Sum{i=0}{\pinf}\ln|P_0(i)|}{P_0\in\Pcal_0(x)}
		\end{calculs}
		\item If $(\beta,m) = (\alpha,n)$, then
		\centre{$H^{(\beta,m)} = \enstq{-\ln|P_0(x)|}{P_0\in\Pcal_0(x)}$}
	\end{itemize}
	Let $b\in\Union{q=0}{\pinf}\supp\Phi^q(x)$. Then, there are $\eta\in H^{(\beta,m)}$ and $y\in \inner {E^{(\beta,m)}}$ such that
	\centre{$\omega^b\asymp\f{\partial \ln_m\kappa_{-\beta}}{\partial u}\exp(\eta + y)$}
\end{corollary}

\begin{proof}
	Since $\Phi$ is strongly linear\index{Strongly linear function}, we just need to apply Proposition \ref{prop:supportPhi2} to each term of $x$. Each path of $\Pcal_0(x)$ involved is shifted one rank. In $H^{(\beta,m)}$ we the add $\ln |P_0(0)|$ compare to Proposition \ref{prop:supportPhi2}. Then $\exp(\eta)$ gives also $|\partial u\exp\epsilon|$. We just remove it so that it does not appear twice. 
\end{proof}

\begin{Prop}
	\label{prop:bonOrderPhi2}
	Let $x$ be a surreal number such that there $u$ of the form $u=\ln_n\kappa_{-\alpha}$, $ r\in\Rbb^*$ and $a_0\in\Nobf$ such that $r\omega^{a_0}\succ \ln u$ and
	\centre{$\forall a\in\supp x\quad \exists\epsilon\sim r\omega^{a_0}\qquad \omega^a\asymp\partial u\exp\epsilon$}
	\lc{Let}{$\Pcal_0(x)=\enstq{P\in\Pcal_\Lbb(x)}{\begin{array}{c}
				P(1)=r\omega^{a_0} \\ \forall i\geq 1 \quad P(i+1)\sim\ln|P(i)
		\end{array}}$}
	Consider $E_1^{(\beta,m)}$, $E_2^{(\beta,m)}$
	and $E_3^{(\beta,m)}$ as defined in Corollary \ref{cor:propsupportPhi2}.
	Let $\xi$ be the smallest ordinal such that $\kappa_{-\xi}\prec^K P(k_P)$ for all path $P\in\Pcal_\Lbb(x)$.
	Let $\lambda$ the least $\epsilon$-number greater than $\NR(x)$ and $\xi$. Then $E^{(\beta,m)}=E_1^{(\beta,m)}\cup E_2^{(\beta,m)}\cup E_3^{(\beta,m)}$ is reverse well-ordered with order type at most $2\lambda+\omega(\xi+1)$.
\end{Prop}

\begin{proof}
	First notice that $E_3^{(\beta,m)}$ is reverse well-ordered with order type at most $\omega$. $E_2^{(\beta,m)}$ is also reverse well-ordered with order at most $\omega\otimes\xi$. We then focus on $E_1^{(\beta,m)}$. For the moment, we will assume $(\beta,m)<_{lex}(\alpha,n)$.
	
	\begin{enumerate}[label=(\roman*)]
		\item We first claim that for all $i\geq 3$ and all path $P\in\Pcal(x)$, $P(i)\prec P(2)\preceq \ln_{m+2} \kappa_{-\beta}$. It is indeed the same proof as the point (i) of the proof of Proposition \ref{prop:bonOrderPhi1}. 
		
		\item We claim that for all path $P\in\Pcal(x)$, if $P(2)\asymp\ln_{m+2}\kappa_{-\beta}$, then, if $r$ is the real number such that 
		$P(2)\sim r\ln_{m+2}\kappa_{-\beta}$, we have $0<r\leq1$. It is indeed the same proof as the point (ii) of the proof of Proposition \ref{prop:bonOrderPhi1}.
		
		\item For all $j$ and $i\geq 2$, $\ln |P_j(i)|\preceq\ln_{m+3} \kappa_{-\beta}\prec \ln_{m+2}\kappa_{-\beta}$. Indeed, using (i), we know that $P_j(i)\preceq \ln_{m+2} \kappa_{-\beta}$. Then, there is a natural number $m\geq 1$ such that $|P_j(i)|\leq m\ln_{m+2}\kappa_{-\beta}$. Using the fact that $\ln$ is increasing,
		\centre{$\ln|P_j(i)| \leq \ln_{m+3}\kappa_{-\beta}+\ln m\preceq \ln_{m+3}\kappa_{-\beta}\prec \ln_{m+2}\kappa_{-\beta}$} 
		
		\item We now claim that $\Unionin k\Nbb E_{1,k,k'}^{(\beta,m)}>\Unionin k\Nbb E_{1,k,k'+2}^{(\beta,m)}$. Indeed, using (ii) and (iii) if $e_1\in \Unionin k\Nbb E_{1,k,k'}^{(\beta,m)}$, then there is $s\in\intff{-(k+1)}{-k}$ such that 
		$e_1\sim s\ln_{m+2} \kappa_{-\beta}$. Similarly, for $e_2\in\Unionin k\Nbb E_{1,k,k'+2}^{(\beta,m)}$, there is \linebreak $s'\in\intff{-(k+3)}{-(k+2)}$ such that $e_2\sim s'\ln_{m+2}\kappa_{-\beta}$. 
		
		\item We define the following sequence : 
		\begin{itemize}
			\item $a_0=1$
			\item $a_{k+1} = \omega^{\omega^{\omega(\omega(\NR(x)+\xi+1)a_k+1)}}$
		\end{itemize}
		We show that $E_{1,k,0}^{(\beta,m)}$ is reverse well-ordered with order type less than $a_k$. We also claim that the equivalence classes of $E_{1,k,0}^{(\beta,m)}/{\asymp}$ are finite and that 
		\centre{$\NR\pa{\Sum{t\in E_{1,k,0}^{(\beta,m)}}{}\exp t}\leq \omega (\NR(x)+\xi+1)a_k$}
		We show it by induction on $k\in\Nbb$.
		\begin{itemize}
			\item For $k=0$, $E_{1,0,0}^{(\beta,m)}=\{0\}$. Then it is reverse well-ordered with order type $1$. We also have
			\centre{$\NR\pa{\Sum{t\in E_{1,0,0}^{(\beta,m)}}{}\exp t}=\NR(1)=1\leq \omega (\NR(x)+\xi+1)$}
			
			\item Assume the property for some $k\in\Nbb$. Let $t\in E_{1,k+1,0}^{(\beta,m)}$. Let \linebreak $(P_0,0),(P_1,i_1),\dots,(P_{k+1},i_{k+1})$ minimal for the order $(<_{lex},<)_{lex}$ such that 
			\centre{$t=e^{(\beta,m)}\left(\begin{matrix}
					P_0;P_1,\dots,P_{k+1}\\
					\emptyset\\
					i_1,\dots,i_{k+1}
				\end{matrix}\right)$}
			Then,
			\begin{calculs}
				& t&=&e^{(\beta,m)}\left(\begin{matrix}
					P_0;P_1\dots,P_k\\ \emptyset\\i_1,\dots,i_k
				\end{matrix}\right)
				-\Sum{i=1}{\pinf}\ln|P_0(i)|\\
				&&&\qquad-\Sum{\footnotesize\begin{array}{c}
						\gamma\geq\alpha, m\in\Nbb^*\\ \gamma\tq P_0(k_{P_0})\succ^K\kappa_{-\gamma}\succeq^KP_{k+1}(k_{P_{k+1}})
				\end{array}}{}\hspace{-5em} \ln_m\kappa_{-\gamma}
				+ \Sum{i=i_{k+1}}{\pinf}\ln |P_{k+1}(i)|
			\end{calculs}
			Write $s=e^{(\beta,m)}\left(\begin{matrix}
				P_0;P_1,\dots,P_k\\ \emptyset\\i_1,\dots,i_k
			\end{matrix}\right)$ and consider the following path: 
			\centre{$\begin{accolade}
					R(0)=\exp s \\
					R(i)= P_{k+1}(i-1+i_{k+1}) & i>0
				\end{accolade}$}
			It is indeed a path since, by definition of $E_{1,k+1,0}^{(\beta,m)}$, $\supp P_{k+1}(i_{k+1})$ must be contained in $\supp s$. We then have,
			\centre{$\exp t = \f{\partial R}{\partial P_0[1:]}$}
			Moreover, $R\in\Pcal_\Lbb\pa{\Sum{s\in E_{1,k,0}^{(\beta,m)}}{}\exp s}$. By assumption on $x$, the set $\enstq{P_0[1:]}{P_0\in\Pcal_0(x)}$ is a singleton. Therefore, so is the set $\enstq{\partial P_0[1:]}{P_0\in\Pcal_0(x)}$. By induction hypothesis and Proposition \ref{prop:majorationNuPartial}, $E_{1,k+1,0}^{(\beta,m)}$ has order type less than 
			\centre{$\omega^{\omega^{\omega(\omega(\NR(x)+\xi+1)a_k+1)}}=a_{k+1}$}	
			Since the equivalences classes of $\Pcal_\Lbb\pa{\Sum{s\in E_{1,k,0}^{(\beta,m)}}{}\exp s}/{\asymp}$ are finite, the 
			ones of $E_{1,k+1,0}^{(\beta,m)}/{\asymp}$ are also finite. Finally, using Lemmas \ref{lem:NRsum} and \ref{lem:NRSommeLogAtomiques},
			\begin{calculs}
				& \NR(t) &\leq& (\omega\otimes\xi) + 
				\Sum{i=1}{k_{P_0}-1} \NR(\ln|P_0(i)|) + k_{P_0} \\ &&& \qquad +\Sum{j=1}{k+1}\Sum{i=i_j}{k_{P_j}-1}\NR\pa{\ln|P_j(i)|}+\Sum{j=0}{k+1}\max(0,k_{P_j}-i_j)+4\\
				&&\leq& \omega(\NR(x)+\xi+1)
			\end{calculs}
			\lc{Then,}{$\NR\pa{\Sum{t'\in E_{1,k+1}^{(\beta,m)}}{}\exp t'}\leq \omega(\NR(x)+\xi+1)a_{k+1}$}
		\end{itemize}
		We conclude thanks to the induction principle.
		
		\item We have $\Unionin k\Nbb E_{1,k,0}^{(\beta,m)}\subseteq\inner {E_{1,1,0}^{(\beta,m)}}$. Then, using (v) and applying Proposition \ref{prop:orderTypeMonoid}, it has order type at most $\omega^{\hat{a_1}}\leq\omega^{\omega a_1}$.
		
		\item We define the following sequence : 
		\begin{itemize}
			\item $b_0=\omega^{\hat{a_1}}$
			\item $b_{k'+1} = \omega^{\omega^{\omega(\omega(\NR(x)+\xi+4)b_{k'}+1)}}$
		\end{itemize}
		We show that $\Unionin k\Nbb E_{1,k,k'}^{(\beta,m)}$ is reverse well-ordered with order type less than $b_{k'}$. We also claim that the equivalence classes of $\Unionin k\Nbb E_{1,k,k'}^{(\beta,m)}/{\asymp}$ are finite and that 
		\centre{$\NR\pa{\Sum{t\in \Unionin k\Nbb E_{1,k,k'}^{(\beta,m)}}{}\exp t}\leq \omega (\NR(x)+\xi+4)b_{k'}$}
		We show it by induction on $k'\in\Nbb$.
		\begin{itemize}
			\item For $k'=0$, we just apply (vi).
			
			\item Assume the property for some $k'\in\Nbb$. Let $t\in \Unionin k\Nbb E_{1,k,k'+1}^{(\beta,m)}$. Let $(P_0,0)(P_1,i_1),\dots,(P_{k+k'+1},i_{k+k'+1})$ minimal for the order \linebreak $(<_{lex},<)_{lex}$ such that $t=e^{(\beta,m)}\left(\begin{matrix}
				P_0;P_1,\dots,P_k\\ P_{k+1},\dots, P_{k+k'+1}\\ i_1,\dots,i_{k+k'+1}
			\end{matrix}\right)$. Then,
			\begin{calculs}
				&t &=& e^{(\beta,m)}\left(\begin{matrix}
					P_0;P_1,\dots,P_k\\ P_{k+1},\dots, P_{k+k'}\\ i_1,\dots,i_{k+k'}
				\end{matrix}\right)
				-\Sum{\ell=m+2}{\pinf}\ln_\ell{\kappa_{-\beta}}\\
				&&&\qquad -\Sum{\footnotesize\begin{array}{c}
						\gamma>\beta, \ell\in\Nbb^*\\ \gamma\tq\kappa_{-\gamma}\succeq^KP_{k+k'+1}(k_{P_{k+k'+1}})
				\end{array}}{} \ln_m\kappa_{-\beta}
				+ \Sum{i=i_{k+1}}{\pinf}\ln |P_{k+1}(i)|
			\end{calculs}
			Write $s=e^{(\beta,m)}\left(\begin{matrix}
				P_0;P_1,\dots,P_k\\ P_{k+1},\dots, P_{k+k'}\\ i_1,\dots,i_{k+k'}
			\end{matrix}\right)$. We then have,
			\begin{calculs}
				&\partial(\ln_{m+1}\kappa_{-\beta})\exp t &=& 
				\exp(s)\exp\pa{
					\Sum{i=i_{k+1}}{\pinf}\ln |P_{k+1}(i)|\right.\\&&&\left.\qquad
					-\Sum{\footnotesize\begin{array}{c}
							\ell\in\Nbb^*\\ \gamma\tq\kappa_{-\gamma}\succeq^KP_{k+k'+1}(k_{P_{k+k'+1}})
					\end{array}}{}\hspace{-3em} \ln_m\kappa_{-\beta}}
			\end{calculs}
			Consider the following path:
			\centre{$\begin{accolade}
					R(0)=\exp s \\
					R(i)= P_{k+k'+1}(i-1+i_{k+1}) & i>0
				\end{accolade}$}
			It is indeed a path since, by definition of $E_{1,k,k'+1}^{(\beta,m)}$, $\supp P_{k+k'+1}(i_{k+k'+1})$ must be contained in $\supp s$. Then,
			\centre{$\partial(\ln_{m+1}\kappa_{-\beta})\exp t=\partial R$} 
			Moreover, $R\in\Pcal_\Lbb\pa{\Sum{s\in \Unionin k\Nbb E_{1,k,k'}^{(\beta,m)}}{}\exp s}$. By induction hypothesis and Proposition \ref{prop:majorationNuPartial}, $\Unionin k\Nbb E_{1,k,k'+1}^{(\beta,m)}$ has order type less than 
			\centre{$\omega^{\omega^{\omega(\omega(\NR(x)+\xi+4)b_{k'}+1)}}=b_{k'+1}$}	
			Since the equivalences classes of $\Pcal_\Lbb\pa{\Sum{s\in \Unionin k\Nbb E_{1,k,k'}^{(\beta,m)}}{}\exp s}/{\asymp}$ are finite, the 
			ones of $\Unionin k\Nbb E_{1,k,k'+1}^{(\beta,m)}/{\asymp}$ are also finite. Finally, using Lemmas \ref{lem:NRsum} and \ref{lem:NRSommeLogAtomiques},
			\begin{calculs}
				& \NR(t) &\leq& (\omega\oplus\omega\otimes\xi\oplus\omega) + 
				\Sum{j=0}{k+1}\Sum{i=i_j}{k_{P_j}-1}\NR\pa{\ln|P_j(i)|}+\Sum{j=0}{k+1}\max(0,k_{P_j}-i_j)\\
				&&\leq& \omega(\NR(x)+\xi+4)
			\end{calculs}
			\lc{Then,}{$\NR\pa{\Sum{t'\in \Unionin k\Nbb E_{1,k,k'+1}^{(\beta,m)}}{}\exp t'}\leq \omega(\NR(x)+\xi+4)b_{k'+1}$}
		\end{itemize}
		We conclude thanks to the induction principle.
		
		\item By easy induction, for all $k\in\Nbb$, $b_{k'}<\lambda$.
		
		\item Using (iv), we get that for all $N\in\Nbb$, $\Union{k'=0}{N}\Unionin k\Nbb E_{1,,k,2k'}^{(\beta,m)}$ is an initial segment of
		$\Unionin{k'}\Nbb\Unionin k\Nbb E_{1,k,2k'}^{(\beta,m)}$. We also have that $\Union{k'=0}{N}\Unionin k\Nbb E_{1,k,2k'+1}^{(\beta,m)}$ is an initial segment of
		$\Unionin {k'}\Nbb\Unionin k\Nbb E_{1,k,2k'+1}^{(\beta,m)}$. Using (vii), we get that $\Unionin{k'}\Nbb \Unionin k\Nbb E_{1,k,2k'}^{(\beta,m)}$ has order type at most
		\centre{$\sup\enstq{\Oplus{k=0}{N}b_{2k'}}{N\in\Nbb} = \sup\enstq{b_{2N}}{N\in\Nbb}\underset{\text{by (viii)}}{\leq}\lambda$}
		Similarly,  $\Unionin{k'}\Nbb\Union{k\in\Nbb}{}E_{1,k,2k'+1}^{(\beta,m)}$ has order type at most $\lambda$. Using Proposition \ref{prop:unionEnsBienOrd}, we conclude that $E_1^{(\beta,m)}$ has order type at most~$2\lambda$.
	\end{enumerate}
	Now we deal with the case $(\beta,m)=(\alpha,n)$. A close looking at point (v) above reveals that the property it shows does not depend on $(\beta,m)$. Then we have, using a similar argument as in points (viii) and (ix), that $\Unionin k\Nbb E_{1,k,0}^{(\alpha,n)}$ has order type at most~$2\lambda$. Then, for any $(\beta,m)\leq_{lex}(\alpha,n)$, $E_1^{(\beta,m)}$ is reverse well-ordered with order type at most $2\lambda$. Using again Proposition \ref{prop:unionEnsBienOrd} and the properties of $E_2^{(\beta,m)}$ and $E_3^{(\beta,m)}$ mentioned in the beginning of this proof, we get that $E^{(\beta,m)}$ is reverse well-ordered with order type at most $2\lambda+\omega(\xi+1)$.
\end{proof}

\subsection{Length of the series of the anti-derivative of an arbitrary surreal number}

%\begin{proposition}
%	\label{prop:supportPhi}
%	Let $x$ be a surreal number. Let $\gamma$ be the smallest ordinal such that $\kappa_{-\gamma}\prec^K P(k_P)$ for all path $P\in\Pcal_\Lbb(x)$. Let $\lambda$ be the least $\epsilon$-number greater than $\NR(x)$ and $\gamma$. Then $\Unionin i\Nbb \supp\Phi^i(x)$ is reverse well-ordered with order type less than $\omega^{\omega^{\lambda+2}}$.
%\end{proposition}
\propsupportPhi*

\begin{proof}
	Let $\alpha<\gamma$ and $n\in\Nbb$. Write $x=\Sumin a{\supp x}r_a\omega^a$. For any ordinal $\alpha<\gamma$, $n\in\Nbb$, $r\in\Rbb\setminus\{-1\}$ and any term $s\omega^{a_0}$, define $S_{\alpha,n,1,s\omega^{a_0}}$ to be the set
	\centre{$\enstq{a\in\supp x}{\exists\epsilon\in\Nobf_\infty\ \forall (\beta,m)<_{lex}(\alpha,n)\quad
	\begin{accolade}
		\ln_n\kappa_{-\alpha} \prec\epsilon\\
		\epsilon\sim s\omega^{a_0}\\
		\epsilon\ln_m\kappa_{-\beta}\\
		\omega^a\asymp\partial (\ln_n\kappa_{-\alpha})\exp\epsilon
	\end{accolade}
	}$}
	and consider also
	\centre{$S_{\alpha,n,2,r}=\enstq{a\in\supp x}{\exists\epsilon\in\Nobf_\infty\quad \epsilon\sim r\ln_n\kappa_{-\alpha}\wedge \omega^a\asymp\partial (\ln_n\kappa_{-\alpha})\exp\epsilon}$}
	\centre{$x_{\alpha,n,1,s\omega^{a_0}}=\Sumin a{S_{\alpha,n,1,s\omega^{a_0}}}r_a\omega^a\qqandqq x_{\alpha,n,2,r}=\Sumin a{S_{\alpha,n,2,r}}r_a\omega^a$}
	All theses surreal numbers have disjoint supports and 
	$$x=\Sumin{s\omega^{a_0}}{\Rbb\omega^\Nobf}\Sumlt\alpha\gamma\Sumin n\Nbb x_{\alpha,n,1,s\omega^{a_0}} + \Sumin r{\Rbb\setminus\{-1\}} \Sumlt\alpha\gamma\Sumin n\Nbb x_{\alpha,n,2,r}$$
	We then study both sums of the above equality.
	
	\begin{itemize}
		\item The set $\enstq{r\in\Rbb\setminus\{-1\}}{S_{\alpha,n,2,r}\neq\emptyset}$ is reverse well-ordered with order type at most $\nu(x)$. Let
		\centre{$S_0=\Unionin i\Nbb \supp\Phi^i\pa{ \Sumin r{\Rbb\setminus\{-1\}} \Sumlt\alpha\gamma\Sumin n\Nbb x_{\alpha,n,2,r}}$}
		Since $\Phi$ is strongly linear, 
		\centre{$S_0\subseteq\Unionin r{\Rbb\setminus\{-1\}}\Unionlt\alpha\gamma\Unionin n\Nbb\Unionin i\Nbb \supp\Phi^i(x_{\alpha,n,2,r})$}
		Using Corollary \ref{cor:supportPhi1}, $\Unionin i\Nbb \supp\Phi^i(x_{\alpha,n,2,r})$ is reverse well-ordered with order type at most $\omega^{\omega(2\lambda+\omega(\gamma+1)+1)}$. Moreover, Lemma \ref{lem:formeEtaPhiOmegaA} ensure that is $(\alpha,n,r)>_{lex}(\alpha',n',r')$, then $\Unionin i\Nbb \supp\Phi^i(x_{\alpha,n,2,r})<\Unionin i\Nbb \supp\Phi^i(x_{\alpha',n',2,r'})$. We end up with the fact that $\Unionin i\Nbb \supp\Phi^i\pa{ \Sumin r{\Rbb\setminus\{-1\}} \Sumlt\alpha\gamma\Sumin n\Nbb x_{\alpha,n,2,r}}$ is reverse well-ordered with order type at most $\omega^{\omega(2\lambda+\omega(\gamma+1)+1)}\nu(x)\gamma$.
		
		\item \lc{Let}{$S_1=\Unionin i\Nbb \supp\Phi^i\pa{ \Sumin{s\omega^{a_0}}{\Rbb\omega^\Nobf}\Sumlt\alpha\gamma\Sumin n\Nbb x_{\alpha,n,1,s\omega^{a_0}} }$}
		Since $\Phi$ is strongly linear,
		\centre{$S_1\subseteq\Unionin {s\omega^{a_0}}{\Rbb\omega^\Nobf}\Unionlt\alpha\gamma\Unionin n\Nbb\Unionin i\Nbb \supp\Phi^i(x_{\alpha,n,1,s\omega^{a_0}})$}
		Denote $H^{(\beta,m)}(x_{\alpha,n,1,s\omega^{a_0}})$, $E^{(\beta,m)}(x_{\alpha,n,1,s\omega^{a_0}})$ the sets defined as in Corollary \ref{cor:propsupportPhi2} for $x_{\alpha,n,1,s\omega^{a_0}}$. Then, using this corollary, $S_1$ is contained in the set
		\centre{$
			\Union{\tiny \begin{array}{c}
					\beta<\gamma \\ m\in\Nbb \\ s\omega^{a_0}\in \Rbb\omega^\Nobf
			\end{array}}{}
			\Union{\tiny \begin{array}{c}
					\alpha,n\tq (\beta,m)\leq_{lex}(\alpha,n)\\\alpha<\gamma
			\end{array}}{}
			\Union{\tiny \begin{array}{c}
					\eta\in H^{(\beta,m)}(x_{\alpha,n,1,s\omega^{a_0}})\\
					y\in \inner{E^{(\beta,m)}(x_{\alpha,n,1,s\omega^{a_0}})}
			\end{array}}{}\hspace{-3em}\supp\pa{ \f{\partial\ln_m\kappa_{-\beta}}{\partial\ln_n\kappa_{-\alpha}}\exp(\eta+y)}
			$}
		We also know that $(\alpha,n,s\omega^{a_0})>_{lex}(\alpha',n',s'\omega^{a_0'})$, then the set
		\centre{$\Union{\tiny \begin{array}{c}
					\eta\in H^{(\beta,m)}(x_{\alpha,n,1,s\omega^{a_0}})\\
					y\in \inner{E^{(\beta,m)}(x_{\alpha,n,1,s\omega^{a_0}})}
			\end{array}}{}\hspace{-3em}\supp\pa{ \f{\partial\ln_m\kappa_{-\beta}}{\partial\ln_n\kappa_{-\alpha}}\exp(\eta+y)}$}
		is contained in the set
		\centre{$\Union{\tiny \begin{array}{c}
					\eta\in H^{(\beta,m)}\pa{x_{\alpha',n',1,s'\omega^{a_0'}}}\\
					y\in \inner{E^{(\beta,m)}\pa{x_{\alpha',n',1,s'\omega^{a_0'}}}}
			\end{array}}{}\hspace{-3em}\supp\pa{ \f{\partial\ln_m\kappa_{-\beta}}{\partial\ln_n\kappa_{-\alpha}}\exp(\eta+y)}$}
		Propositions \ref{prop:bonOrderPhi2} and \ref{prop:orderTypeMonoid} guarantee that all of theses sets are reverse well-ordered with order type less than~$\omega^{2\omega\lambda}$. Let
		\centre{$S_{\beta,m} = \Union{\tiny \begin{array}{c}
					\alpha,n\tq (\beta,m)\leq_{lex}(\alpha,n)\\\alpha<\gamma\\ s\omega^{a_0}\in \Rbb\omega^\Nobf
			\end{array}}{}\Union{\tiny \begin{array}{c}
					\eta\in H^{(\beta,m)}(x_{\alpha,n,1,s\omega^{a_0}})\\
					y\in \inner{E^{(\beta,m)}(x_{\alpha,n,1,s\omega^{a_0}})}
			\end{array}}{}\hspace{-3em}\supp\pa{ \f{\partial\ln_m\kappa_{-\beta}}{\partial\ln_n\kappa_{-\alpha}}\exp(\eta+y)}$}
		The set of possible $s\omega^{a_0}$ is reverse well-ordered with order type at most $\nu(x)$. Moreover, $\alpha$ and $n$ are determined from $s\omega^{a_0}$. Then $S_{\beta,m}$ is reverse well-ordered with order type at most $\omega^{2\omega^{\lambda+1}}\nu(x)$. Finally, if $(\beta,m)>_{lex}(\beta',m')$, then $S_{\beta,m}\subsetneq S_{\beta',m'}$ and there are at most $\omega\gamma$ such couples. Then, $S_1$ is reverse well-ordered with order type at most $\omega^{2\omega\lambda+1}\nu(x)\gamma$.
	\end{itemize}
	
	Both sets have order type less than $\omega^{\omega^{\lambda+2}}$, which is a multiplicative ordinal. Using Proposition \ref{prop:sommeEnsBienOrd}, $\Unionin i\Nbb\supp\Phi^i(x)$ has order type less than $\omega^{\omega^{\lambda+2}}$.
	
	
	
\end{proof}


\subsection{Stability of some surreal fields}

We are ready to exhibit a surreal field that is stable under $\exp, \ln, \partial$ and anti-derivation and that is not $\Nobf$ itself. We actually have a lot of such fields.

To get a field stable under derivation and anti-derivation, it sufficient that the field is stable under $\exp$ and $\ln$, and, if for all $\beta<\alpha$, $\kappa_{-\beta}$ is in the field, and if $\omega\otimes\alpha$ is less than the authorized length of a serie, then $\kappa_{-\alpha}$ must also be in the field. 

%\begin{theorem}
%	\label{thm:corpsStable}
%	Let $\alpha$ be a limit ordinal %Let $\eta<\epsilon_\alpha$. 
%	and $\suitelt{\Gamma_\beta}\beta\alpha$ be a sequence of Abelian subgroups of $\Nobf$ such that
%	\begin{itemize}
%		\item $\forall\beta<\alpha\quad \forall\gamma<\beta\qquad \Gamma_\gamma\subseteq\Gamma_\beta$
%		
%		\item $\forall\beta<\alpha\qquad \omega^{(\Gamma_\beta)^*_+}\succ^K\kappa_{-\epsilon_\beta}$
%		
%		\item  $\forall\beta<\alpha\quad \forall\gamma<\epsilon_\beta\qquad \kappa_{-\gamma}\in\omega^{\Gamma_\beta}$
%		
%		\item $\forall\beta<\alpha\quad\exists\eta_\beta<\epsilon_\beta\quad \forall x\in\omega^{\Gamma_\beta}\qquad \NR(x)<\eta_\beta$
%	\end{itemize}
%	Then $\Unionlt\beta\alpha \SRF{\epsilon_\beta}{\Gamma_\beta^{\uparrow\epsilon_\beta}}$ is stable under $\exp$, $\ln$, $\partial$ and anti-derivation.
%\end{theorem}

\thmcorpsStable*

\begin{proof}
	Let $\Kbb=\Unionlt\beta\alpha\SRF{\epsilon_\beta}{\Gamma_\beta^{\uparrow\epsilon_\beta}}$
	As an increasing union of fields, $\Kbb$ is indeed a field.
	
	\begin{enumerate}[label=(\roman*)]
		\item Using Theorem \ref{thm:SRFGammaUpStableExpLn}, each field $\SRF{\epsilon_\beta}{\Gamma_\beta^{\uparrow\epsilon_\beta}}$ is stable under $\exp$ and $\ln$, then so is $\Kbb$.
		
		\item Write $\Gamma_\beta^{\uparrow\epsilon_\beta}=\suitelt{\Gamma_{i,\beta}}i{\gamma_{\epsilon_{\beta}}}$. We use the notation introduce in the beginning Definition \ref{def:uparrow}. We prove by induction on $i<\gamma_{\epsilon_\beta}$ that for $x\in\Gamma_{i,\beta}$, $\NR(\omega^x)<\eta_\beta e_i$.
		\begin{itemize}
			\item For $i=0$ we have $e_0=1$ and $\Gamma_{0,\beta}=\Gamma_\beta$. By assumption on $\Gamma_\beta$, for all $x\in\Gamma_{0,\beta}$, $\NR(\omega^x)<\eta_\beta=\eta_\beta e_0$.
			
			\item Assume the property for some ordinal $i$. Then let $x\in\Gamma_{i+1,\beta}$. Write $x=u+v+\Sum{k=1}{p}h(w_k)$ with $u\in\Gamma_{i,\beta}$, $v\in\SRF{e_i}{g\pa{(\Gamma_{i,\beta})_+^*}}$ and $w_k$ such $r\omega^{w_k}$ is a term of some element $y_k\in\Gamma_{i,\beta}$, for some $r\in\Rbb$. Using Corollary \ref{cor:NRprod},
			\centre{$\NR(\omega^x)\leq\NR(\omega^u)+\NR(\omega^v)+\Sum{k=1}{p}\NR(\omega^{h(w_k)})+p+1$}
			From induction hypothesis,
			\centre{$\NR(\omega^u)<\eta_\beta e_i$}
			Write $v=\Sum{j<\nu}{}r_j\omega^{g(a_j)}$. Then $\omega^v=\exp\pa{\Sum{j<\nu}{}r_j\omega^{a_j}}$. From induction hypothesis, $\NR(\omega^{a_j})<\eta_\beta e_i$. Then $\NR(r_j\omega^{a_j})<\eta_\beta e_i+1$. Then
			\centre{$\NR(\omega^v)=\NR\pa{\Sum{j<\nu}{}r_j\omega^{a_j}}\leq(\eta_\beta e_i+ 1)\otimes\nu\leq(\eta_\beta e_i +1)\otimes e_i\leq\eta_\beta e_i^2$}
			We also have
			\centre{$\NR(\omega^{h(w_k)})=\NR(\omega^{\omega^{w_k}})\leq\NR(\omega^{y_k})<\eta_\beta e_i$}
			\lc{Finally,}{$\NR(\omega^x)\leq(p+1)(\eta_\beta e_i+1)+\eta_\beta e_i^2<\eta_\beta e_{i+1}$}
			
			\item Assume $i$ is a limit ordinal. Then by definition of $\Gamma_{i,\beta}$ for any $x\in\Gamma_{i,\beta}$ there is some $j<i$ such that $x\in\Gamma_{j,\beta}$. Then induction hypothesis concludes.
		\end{itemize}
		\item Let $x\in\Kbb$ and $\beta<\alpha$ such that $x\in\SRF{\epsilon_\beta}{\Gamma_\beta^{\uparrow\epsilon_\beta}}$. Using (ii), there is $i<\gamma_{\epsilon_\beta}$ such that \centre{$\NR(x)\leq(\eta_\beta e_i+1)\otimes\nu(x)<\eta_\beta\otimes\epsilon_\beta=\epsilon_\beta$}
		Since $\eta_\beta\otimes\epsilon_\beta$ is a limit ordinal, then we also have $\NR(x)+1<\eta_\beta\otimes \epsilon_\beta=\epsilon_\beta$.
		
		\item Let $x\in\Kbb$ and $\beta<\alpha$ such that $x\in\SRF{\epsilon_\beta}{\Gamma_\beta^{\uparrow\epsilon_\beta}}$. Using (iii) and Proposition \ref{prop:majorationNuPartial}, $\nu(\partial x)<\omega^{\omega^{\omega(\NR(x)+1)}}<\epsilon_\beta$. Using Corollary \ref{cor:LogAtomicDeGammaUp} and (i), we also have for all $P\in\Pcal(x)$, $\partial P\in\Rbb\omega^{\Gamma_\beta^{\uparrow\epsilon_\beta}}$. Then, 
		\centre{$\partial x\in\SRF{\epsilon_\beta}{\Gamma_\beta^{\uparrow\epsilon_\beta}}\subseteq\Kbb$}
		Then $\Kbb$ is stable under $\partial$.
		
		\item Let $x\in\Kbb$ and $\beta<\alpha$ such that $x\in\SRF{\epsilon_\beta}{\Gamma_\beta^{\uparrow\epsilon_\beta}}$. Using Proposition \ref{prop:supportPhi} and the definition of $\Acal$, 
		\centre{$\nu\pa{\Acal\circ\pa{\Sumin i\Nbb\Phi^i}(x)}<\omega^{\omega^{\lambda+2}}$}
		where $\lambda$ is least $\epsilon$-number greater $\NR(x)$ and such that
		\centre{$\forall P\in\Pcal_\Lbb(x)\qquad \kappa_{-\lambda}\prec^K P(k_P)$}
		Using (iii), $\NR(x)<\epsilon_\beta$. Let $P\in\Pcal_\Lbb(x)$. Using (i), $\SRF{\epsilon_\beta}{\Gamma_\beta^{\uparrow\epsilon_\beta}}$ is stable under $\exp$ and $ln$. Since $P(i+1)$ is a term of $\ln|P(i)|$, if $P(i)\in\omega^{\Gamma_\beta}$, then $P(i+1)\in\omega^{\Gamma_\beta}$. By induction, $P(k_P)\in\omega^{\Gamma_\beta}$. Since $P(k_P)$ is infinitely large, $P(k_P)\in\omega^{\pa{\Gamma_\beta}^*_+}$. By assumption on $\Gamma_\beta$, $P(k_P)\succ^K\kappa_{-\epsilon_\beta}$. Finally, $\lambda\leq\epsilon_\beta$ and
		\centre{$\nu\pa{\Acal\circ\pa{\Sumin i\Nbb\Phi^i}(x)}<\omega^{\omega^{\epsilon_\beta+2}}<\epsilon_{\beta+1}$}
		Propositions \ref{prop:supportPhi1} and \ref{prop:supportPhi2} and the third assumption about $\Gamma_\beta$ ensure that each term of $\Acal\circ\pa{\Sumin i\Nbb\Phi^i}(x)$ is in $\omega^{\Gamma_\beta}\subseteq\omega^{\Gamma_{\beta+1}}$. Then
		\centre{$\Acal\circ\pa{\Sumin i\Nbb\Phi^i}(x)\in\SRF{\epsilon_{\beta+1}}{\Gamma_{\beta+1}^{\uparrow\epsilon_{\beta+1}}}$}
		Application of Corollary \ref{cor:existencePrimitive} gives that $\Kbb$ is stable under anti-derivation.
	\end{enumerate}
\end{proof}

The previous theorem may seem have a lot of strong hypothesis but we can actually give a non-trivial application.


	Take $\alpha=\omega$ and for $n<\omega$, $\Gamma_n=\enstq{x\in\Nolt{\epsilon_n}}{\NR(\omega^x)<\epsilon_{n-1}}$, with $\epsilon_{-1}:=\omega$.
	We first recall that from Lemma \ref{lem:kappaMoinsAlpha}, for any ordinal $\alpha$, 
	\centre{$\kappa_{-\alpha}=\omega^{\omega^{-\omega\otimes\alpha}}$}
	\lc{in particular}{$\kappa_{-\epsilon_n} = \omega^{\omega^{-\omega\otimes\epsilon_n}} = \omega^{\omega^{-\epsilon_n}} = \omega^{\f1{\epsilon_n}}$}
	From Theorem \ref{thm:serieToSignExp}, we know that the sign sequence of $\omega^{-\omega\otimes\alpha}$ is $(+)(-)^{\omega\otimes\alpha}$, which has length $1\oplus\omega\otimes\alpha$.
	\begin{itemize}
		\item Since $\epsilon_n$ is an $\epsilon$-number, hence an additive ordinal, for any $n\in\Nbb$, $\Gamma_n$ is an abelian group.
		\item Of course for any $n\leq m$, $\Gamma_n\subseteq\Gamma_m$.	
		\item Since $\length{\omega^{-\epsilon_n}}=1\oplus\omega\otimes\epsilon_n=\epsilon_n$, we have $\kappa_{-\epsilon_n}\notin\omega^{\Gamma_\beta}$. However, for $\alpha<\epsilon_n$, $\length{\omega^{-\omega\otimes\alpha}}=1\oplus\omega\otimes\alpha<\epsilon_n$, and, since $\kappa_{-\alpha}\in\Lbb$, $\NR(\kappa_{-\alpha})=0<\epsilon_{n-1}$. Therefor,  we have $\kappa_{-\alpha}\in\omega^{\Gamma_\beta}$. Since, $\kappa_{-\epsilon_\beta}\in\omega^\Nobf$, is $x\in\omega^{\Gamma_\beta}$ is such that $x\preceq^K\kappa_{-\epsilon_\beta}$ then $\kappa_{-\epsilon_\beta}$ is a prefix of $x$ and $\length x\geq\epsilon_\beta$ what is impossible from Lemma \ref{lem:lengthExpLog}.
		\item We can take $\eta_\beta=\epsilon_{\beta-1}<\epsilon_\eta$.
	\end{itemize}
	Theorem \ref{thm:corpsStable} applies and $\Unionin n\Nbb \SRF{\epsilon_n}{\Gamma_n^{\uparrow\epsilon_n}}$ is stable under $\exp$, $\ln$, $\partial$ and anti-derivation. As a final note, we can notice that
	\centre{$\Unionin n\Nbb \SRF{\epsilon_n}{\Gamma_n^{\uparrow\epsilon_n}} = \Unionin n\Nbb \SRF{\epsilon_n}{\Nolt{\epsilon_n}^{\uparrow\epsilon_n}}$}