%!TEX root = paper-stabilite-exp-log.tex
 


%\SOURCETXT{Le blabla qui suit vient de soumission CIE}

%\olivierpourlui{Radote sur longueur}
%Recall that the length of a surreal number $x \in \No$, denoted by $\length{x}$,  is  $dom(x)$, i.e. the ordinal $\lambda$ considering $x$ as $s : \alpha \to 2 = \{-,+\}$.   Let $\No_\lambda$ denote the set of
% sequences of ordinal length $\lambda$.
%
  Let $\Nolambda$ denote the set surreal number whose signs 
  sequences have length less than $\lambda$ where $\lambda$ is some ordinal.
  We have of course $\Nobf = \bigcup_{\lambda \in \On} \Nolambda$.  
  
Van den Dries and Ehrlich have proved the following: 

  \begin{theorem}[\cite{DriesEhrlich01,van2001erratum}]
   The ordinals $\lambda$
  such that $\Nolambda$ is closed under the various fields operations of $\No$ can be
  characterised as follows:
  \begin{itemize}
  \item $\Nolambda$ is an additive subgroup of $\No$ iff
    $\lambda=\omega^\alpha$ for some ordinal $\alpha$.
  \item $\Nolambda$ is a subring of $\No$ iff
    $\lambda=\omega^{\omega^\alpha}$ for some ordinal $\alpha$.
  \item $\Nolambda$ is a subfield of $\No$ iff
    $\omega^\lambda=\lambda$.
  \label{ou}
\end{itemize}
\end{theorem}
The ordinals $\lambda$ satisfying first (respectively: second) item
are often said to be additively (resp. multiplicatively)
indecomposable but for the sake of brevity we shall just call them
\textbf{additive} (resp. \textbf{multiplicative}). Multiplicative ordinals are exactly
the ordinals $\lambda>1$ such that $\mu \nu <\lambda$ whenever
$\mu,\nu < \lambda$. The ordinal satisfying third item are called
\textbf{$\epsilon$-numbers}. The smallest $\epsilon$-number is usually denoted
by $\epsilon_0$ and is given by
$$\epsilon_0:=\sup\{\omega,\omega^\omega,\omega^{\omega^\omega},\dots\}.$$

\begin{remark} \label{rqdouze}
	Since rational numbers have length at most $\omega$, we have that if $\lambda$ is multiplicative, then $\Nolt\lambda$ is a divisible group.
\end{remark}

If $\lambda$ is an $\epsilon$-number, $\Nolambda$ is actually more than only a field: 

\begin{theorem}[\cite{DriesEhrlich01,van2001erratum}]
Let $\lambda$ be any $\epsilon$-number. Then 
  $\Nolambda$ is a real closed field. 
\end{theorem}



