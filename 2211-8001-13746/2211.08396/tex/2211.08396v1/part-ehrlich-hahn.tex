%!TEX root = paper-stabilite-exp-log.tex
 
%\olivierpourlui{vérifier que parfaitement défini ce truc}
%
%We have some classical real sub-fields when we want to restrict the length of the ordinal sum. We introduce  the following notation :
%\begin{definition}
%	$\Kbb_{\lambda}^\Gamma=\enstq{x\in\Kbb((t^\Gamma))}{\supp x < \lambda}$
%\end{definition}
%
%\noindent To enable the previous set to be field, it is sufficient to assume that $\lambda$ is an $\epsilon$-number so that the inverse operation steel works and is well-defined. Namely, we have:

%Following Remark \ref{rqdouze}, we introduce the following.

As we will often play with exponents of formal power series consided in the Hahn series, we propose to introduce the following notation: 

\begin{definition}
	If $\lambda$ is an $\epsilon$-number and $\Gamma$ a divisible Abelian group, we denote  $$\SRF{\lambda}{\Gamma}=\HahnFieldOrd\Rbb{\Gamma}\lambda$$
\end{definition}
As a consequence of Proposition \ref{prop:hahnFieldRealClosed} we have 

\begin{corollary}
$\SRF\lambda{\Nolt\mu}$ is a real-closed field when $\mu$ is a multiplicative ordinal and $\lambda$ an $\epsilon$-number. 
\end{corollary}
This fields are somehow the atoms constituting the fields $\Nolambda$. 


\thEhrlichquatresept*

%\begin{Theorem}[{\cite[Proposition 4.7]{DriesEhrlich01,van2001erratum}}] \label{th:Ehrlichquatresept}
%Let $\lambda$ be an $\epsilon$-number. Then
%\begin{enumerate}
%\item $\Nolambda=\bigcup_{\mu} \SRF{\lambda}{\Nolt\mu}$, where $\mu$ ranges over the additive ordinals $<\lambda$ (equivalently, $\mu$ ranges over the multiplicative ordinals $<\lambda$ ).
%\item  $\Nolambda$ is a real closed subfield of $\No$, and is closed under the restricted analytic functions of $\No$.
%\item $\Nolambda=\SRF{\lambda}{\Nolt\lambda}$ if and only if $\lambda$ is a regular cardinal.
%\end{enumerate}
%\end{Theorem}

\begin{remark}
The fact that if $\lambda$ is not a regular cardinal, then $\Nolambda \neq\SRF\lambda{\Nolt\lambda}$ can be seen as follows: Suppose that $\lambda$ is not a regular cardinal. This means that we can take some strictly increasing sequence $(\mu_{\alpha})_{\alpha < \beta}$ that is cofinal in $\lambda$ with $\beta<\lambda$. Then $\sum_{\alpha <\beta} \omega^{-\mu_{\alpha}}$ is in $\SRF{\lambda}{\Nolt\lambda}$ by definition, but is not in $\Nolambda$.
\end{remark}



