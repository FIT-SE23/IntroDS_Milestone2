%!TEX root = ms.tex
\subsection{A hierarchy of subfields of $\No$ stable by exponential and logarithm}
In this subsection we recall our previous work on a hierarchy of surreal subfields stable under exponential and logarithm.

We start by Theorem \ref{thm:SRFGammaUpStableExpLn} repeated here for readability:

\thmSRFGammaUpStableExpLn*

This result is actually a consequence of a more general proposition which is the following.
\begin{proposition}[{\cite[Proposition 5.1]{bournez2022surreal}}]
	\label{prop:UnionSRFStableExpLn}
	Let $\lambda$ be an $\epsilon$-number and $\suite{\Gamma_i}iI$ be a family of Abelian subgroups of $\Nobf$. Then
	$\RlGI$ is stable under $\exp$ and $\ln$ if and only if $$\Unionin iI\Gamma_i=\Unionin iI\SRF\lambda{g\pa{\pa{\Gamma_i}^*_+}}$$
\end{proposition}



%\begin{proof}
%	
%	\begin{itemize}
%		\item \textbf{Proof of direct implication}: We assume that $\RlGI$ is stable under both exponential and logarithm. Then for any $x=\aSurreal$ a purely infinite number, we have 
%		
%		$$
%		\exp x = \omega^{\Sumlt i\nu r_i\omega^{g(a_i)}}\in\RlGI
%		$$
%		and therefore 
%		
%		$$
%		\Sumlt i\nu r_i\omega^{g(a_i)}\in\Unionin jI \Gamma_j
%		$$
%		This being true for any family $(a_i)_{i<\nu}$ of $\Gamma_j$, for any $j\in I$. Hence, $\Unionin iI \SRF\lambda{g\pa{\pa{\Gamma_i}^*_+}}\subseteq\Unionin iI\Gamma_i$.
%		
%		Conversely, for any $j\in I$ and any $a\in\Gamma_j$ then we have $\ln\omega^a\in\RlGI$. Writing $a=\aSurreal$, we get $\Sumlt i\nu r_i\omega^{h(a_i)}\in\RlGI$. To say it another way,
%		$$\exists k\in I\quad \forall i<\nu\qquad h(a_i)\in\Gamma_k$$
%		or
%		$$\exists k\in I\quad \forall i<\nu\qquad a_i\in g\left(\left(\Gamma_k\right)^*_+\right)$$
%		Then, there is some $k\in I$ such that $a\in\SRF\lambda{g\pa{\pa{\Gamma_k}^*_+}}$. Hence, for all $j\in I$, $\Gamma_j\subseteq \Unionin iI \SRF\lambda{g\pa{\pa{\Gamma_i}^*_+}}$. 
%		Finally, $\Unionin iI \SRF\lambda{g\pa{\pa{\Gamma_i}^*_+}}\supseteq\Unionin iI\Gamma_i$.
%		
%		We have both inclusions, then, $\Unionin iI \SRF\lambda{g\pa{\pa{\Gamma_i}^*_+}} = \Unionin iI\Gamma_i$
%		
%		\item \textbf{Proof of indirect implication}: We assume that $\Unionin iI \SRF\lambda{g\pa{\pa{\Gamma_i}^*_+}} = \Unionin iI\Gamma_i$. We distinguish the proof in several steps
%		
%		\begin{enumerate}[label=(\roman*)]
%			\item\label{it:casExpApp} %cas exp pour les appreciables 
%			First take $x= \aSurreal\in\RlGI$ being appreciable, i.e. $a_i\leq 0$ for all $i<\nu$. By definition there is some $j\in I$ such that $x$ is an element of $\SRF\lambda{\Gamma_j}$. Since from Theorem \ref{thm:expAppreciables} for any appreciable number 
%			$$\supp\exp x \subseteq \inner{\supp x} $$
%			\noindent where $\inner{\supp x}$ is the monoid generated by $\supp x$ in $\Gamma_j$. In particular $\supp\exp x\subseteq\Gamma_j$. Then, Proposition \ref{prop:orderTypeMonoid} ensures that the order type of $\supp\exp x$ is less than $\lambda$. Hence $\exp x \in\RlGI$. 
%			
%			\item\label{it:casExpInf} %cas exp pour les purement infiniment grands
%			
%			Let $x=\aSurreal\in\RlGI$ a purely infinite number. Let $j\in I$ such that $x\in\SRF\lambda{\Gamma_j}$ that is that $a_i\in\left(\Gamma_j\right)^*_+$ for all $i<\nu$.  We have $\exp x = \omega^{\Sumlt i\nu r_i\omega^{g(a_i)}}$ and 
%			$$
%			\Sumlt i\nu r_i\omega^{g(a_i)}\in \SRF\lambda{g\left(\left(\Gamma_j\right)^*_+\right)}
%			$$
%			
%			By assumption, $\SRF\lambda{g\left(\left(\Gamma_j\right)^*_+\right)}\subseteq \Unionin iI\Gamma_i$. Then $\exp x\in \RlGI$.
%			
%			\item\label{it:casExpG} %cas général exp
%			We now make use of both Items \ref{it:casExpApp} and \ref{it:casExpInf}. Let $x\in\RlGI$ be arbitrary. Let $x_\infty$ its purely infinite part and $x_a$ its appreciable part. Then $x=x_\infty+x_a$ and $\exp x = \exp(x_\infty)\exp(x_a)$. Using \ref{it:casExpInf} and \ref{it:casExpApp} respectively, we have $\exp x_\infty\in\RlGI$ and $\exp x_a\in\RlGI$. Then since $\RlGI$ is a field, $\exp x\in\RlGI$.
%			
%			\item\label{it:casLnInf} %cas des infinitésimaux et de ln
%			Similarly to Point \ref{it:casExpApp}, if $x=\aSurreal\in\RlGI$ is infinitesimal, i.e. $a_i<0$ for all $i<\nu$, then $\ln(1+x)=\Sum{k=1}{\infty}\f{x^k}{k}\in\RlGI$
%			
%			\item\label{it:casLnOmega} %cas ln pour les omega^a
%			Let $a\in\Unionin iI\Gamma_i$. By assumption there is $j\in I$ such that $a\in\SRF\lambda{g\left(\left(\Gamma_j\right)^*_+\right)}$.  Hence, we can write $a=\Sumlt i\nu r_i\omega^{g(a_i)}$ where $\nu<\lambda$ and $a_i\in \left(\Gamma_j\right)^*_+$ for all $i<\nu$. Then, $\ln\omega^a=\aSurreal$. Hence $\ln\omega^a\in\SRF\lambda{\Gamma_j}\subseteq\RlGI$.
%			
%			\item\label{it:casLnG} %cas général pour ln
%			Let $x\in\left(\RlGI\right)^*_+$ be arbitrary and write it as $x=r\omega^a(1+\epsilon)$ where $\epsilon$ is infinitesimal, $r$ is a positive real number and $a$ a surreal number. Then, 
%			$\ln x = \ln\omega^a+\ln r+\ln(1+\epsilon)$. Then since $\RlGI$ is a field, $\exp x\in\RlGI$. Using 
%			\ref{it:casLnOmega} and \ref{it:casLnInf} respectively, we have $\ln \omega^a\in\RlGI$ and $\ln(1+\epsilon)\in\RlGI$. Then since $\RlGI$ is a field containing $\R$, $\ln x\in\RlGI$.
%			
%			\item Item \ref{it:casExpG} proves that $\RlGI$ is stable under exponential and Item \ref{it:casLnG} that $\RlGI$ is stable under logarithm. This is what was announced.
%			
%		\end{enumerate}
%		
%	\end{itemize}
%\end{proof}

%We are now ready to prove the theorem. We use the notations of Definition~\ref{def:uparrow}.
%
%\begin{proof}[Proof of Theorem \ref{thm:SRFGammaUpStableExpLn}]
%	We write $\Gamma^{\uparrow\lambda}=(\Gamma_\beta)_{\beta<\gamma_\lambda}$. Using Proposition \ref{prop:UnionSRFStableExpLn}, we just need to show
%	$$\Unionlt \beta{\gamma_\lambda}\Gamma_\beta = \Unionlt \beta{\gamma_\lambda}\SRF\lambda{g\pa{\pa{\Gamma_\beta}^*_+}}$$
%	\begin{itemize}
%		\item[\CSsubset] Let $x\in\SRF\lambda{g\pa{\pa{\Gamma_\beta}^*_+}}$. Let $n<\gamma_\lambda$ minimal such that $\nu(x)<e_n$. Then $x\in\Gamma_{\max(n,\beta)}$.
%		
%		\item[\CNsubset] Let $x\in\Gamma_\beta$. Write $x=\aSurreal$. We also have $x=\Sumlt i\nu r_i\omega^{g(h(a_i))}$ and $h(a_i)\in\Gamma_{\beta+1}$.  Then $x\in\SRF\lambda{g\pa{\pa{\Gamma_{\beta+1}}^*_+}}$.
%	\end{itemize}
%\end{proof}

Note that a consequence of Proposition \ref{prop:UnionSRFStableExpLn} is also the following:

\begin{corollary}[{\cite[Corollary 5.2]{bournez2022surreal}}]
	Let $\lambda$ be an $\epsilon$-number and $\Gamma$ be an abelian subgroup of $\Nobf$. Then
	$\RlG$ is stable under $\exp$ and $\ln$ if and only if $\Gamma=\SRF\lambda{g\pa{\Gamma^*_+}}$.
\end{corollary}
This result is quite similar to Theorem \ref{thm:SRFGammaUpStableExpLn} but in the particular very particular case where $\Unionin G{\Gamma^{\uparrow\lambda}}G=\Gamma$. This apply for instance when $\Gamma=\{0\}$. In this case, we get $\RlG=\Rbb$. If $\lambda$ is a regular cardinal we get an other example considering $\SRF\lambda\Gamma=\Gamma=\Nolt\lambda$.



%To prove the Theorem \ref{thm:NolambdaDecompCorpsStables}, we first prove a proposition to ensure inclusion of the fields in the union.
%
%\begin{proposition}
%	\label{prop:stableExpLnContenuNoLambda}
%	Let $\lambda$ be an $\epsilon$-number and $\mu<\lambda$ an additive (or multiplicative) ordinal. If $\Gamma\subseteq\Nolt\mu$ then $\RlGup\subseteq\Nolt\lambda$ 
%\end{proposition}
%
%\begin{proof}
%	Write $\Gamma^{\uparrow\lambda}=\suitelt{\Gamma_\beta}\beta{\gamma_\lambda}$. What we have to prove is that for all $i<\gamma_\lambda$, $\Gamma_i\subseteq\Nolt{\mu_i}$ for some $\mu_i<\lambda$. We will even prove that $\mu_i=e_{k\oplus 2\otimes i}$ works for some fixed ordinal $k$. We prove it by induction on $i$.
%	
%	\begin{itemize}
%		\item For $i=0$, $\mu_0=e_k$ with $k$ the least ordinal such that $\mu\leq e_k$ works. 
%		
%		\item Assume $i=j+1$ and that the property is true for $j$. Therefore $\Gamma_{i}$ is the group generated by $\Gamma_j$, $\SRF{e_j}{g\pa{(\Gamma_j)^*_+}}$ and $\enstq{h(a_k)}{\Sumlt k\nu r_k\omega^{a_k}\in\Gamma_j}$. Thanks to the induction hypothesis and Lemma \ref{lem:lengthGA}, $g\pa{(\Gamma_j)^*_+}\subseteq \Nolt{\mu_j}$, since $\mu_j$ is an additive ordinal. Hence, thanks to Lemma \ref{lem:lengthOmegaA}, $\SRF{e_j}{g\pa{(\Gamma_j)^*_+}}\subseteq \Nolt{\omega^{\mu_j\otimes\omega}\otimes e_j}$. Finally, from Lemmas \ref{lem:lengthH} and \ref{lem:lengthOmegaA}, $h(a_k)\in\Nolt{\omega^{\mu_j}}$. Thus, $\Gamma_i\subseteq \Nolt{\omega^{\mu_j\otimes\omega}\otimes e_j}$. Since  $\omega^{\mu_j\otimes\omega},e_j<e_{k\oplus2\otimes i}$, and $e_{k\oplus2\otimes i}$ is multiplicative,  taking, $\mu_i=e_{k\oplus 2\otimes i}$ works.
%		
%		\item If $i<\gamma_\lambda$ is a limit ordinal, for all $j<i$, $\lambda > e_{k\oplus2\otimes i}>e_{k\oplus 2\otimes j}$. Then, by the induction hypothesis on all $j<i$, $\Gamma_i\subseteq\Nolt{e_{k\oplus2\otimes i}}$.
%	\end{itemize}
%\end{proof}

%
%With the previous proposition, we have all what we need to prove Theorem \ref{thm:NolambdaDecompCorpsStables}, that we repeat here for readability:

Theorem \ref{thm:SRFGammaUpStableExpLn} enables us to consider a lot of fields stable under exponential and logarithm and enabled us to prove that we can express $\Nolt\lambda$ as a strictly increasing hierarchy of fields stable under $\exp$ and $\ln$.

\thmNolambdaDecompCorpsStables*


%\begin{proof}[Proof of Theorem \ref{thm:NolambdaDecompCorpsStables}]
%	Using Theorem \ref{th:Ehrlichquatresept}, we know that 
%	$$\Nolambda=\Unionin\mu{\enstq{\mu<\lambda}{\mu\text{ additive ordinal}}}\SRF\lambda{\Nolt\mu}$$ 
%	By definition of  $\SRF\lambda{{\Nolt\mu}^{\uparrow\lambda}}$, it must contain $\SRF\lambda{\Nolt\mu}$ and then
%	$$\Nolambda\subseteq \Unionin\mu{\enstq{\mu<\lambda}{\mu\text{ additive ordinal}}}\SRF\lambda{{\Nolt\mu}^{\uparrow\lambda}}$$
%	On the other hand, applying Proposition \ref{prop:stableExpLnContenuNoLambda} gives
%	$$\Unionin\mu{\enstq{\mu<\lambda}{\mu\text{ additive ordinal}}}\SRF\lambda{{\Nolt\mu}^{\uparrow\lambda}} \subseteq \Nolt\lambda$$
%	and this concludes the proof.
%\end{proof}
%
%\subsection{Strictness of the Hierarchy} 
%
%The hierarchy is strict (the theorem is reformulated here to help readability):

\thmhierarchieUparrow*