In this section, we quickly take a look at some useful lemma about order type of well ordered sets. In all the following, circled operators ($\oplus,\otimes$) stand for usual operations over ordinal numbers. The usual symbols ($+,\times$) stand for natural operations, which are commutative.

Our first proposition is about the union of well ordered sets. This result is already knows but we still provide a proof since it is hard to find it in the literature.

\begin{lemma}[Folklore]
	\label{lem:ajoutDUnElementEnsBienOrd}
	Let $\Gamma$ be a totally ordered set, $A\subseteq\Gamma$ be a well-ordered subset with order type $\alpha$. Let $g\in\Gamma$. Then the
	set $A\cup \{g\}$ is well ordered with order type at most $\alpha+1$.
\end{lemma}

\begin{proof}
	We prove it by induction on $\alpha$.
	\begin{itemize}
		\item If $\alpha=0$ then $A\cup\{g\}$ has only one element, and then has order type $1=\alpha+1$.
		
		\item If $\alpha=\gamma+1$ is a successor ordinal. Let $u$ the largest element in $A$. If $u\leq g$ then $A\cup\{g\}$ has indeed order type at most $\alpha+1$. If not, then, by induction hypothesis, $\pa{A\setminus\{u\}}\cup\{g\}$ has order type at most $\gamma+1=\alpha$. Then $A\cup\{g\}=\pa{\pa{A\setminus\{u\}}\cup\{g\}}\cup\{u\}$ has order type at most $\alpha+1$.
		
		\item If $\alpha$ is a limit ordinal. If $g$ is larger than any element of $A$, then $A\cup\{g\}$ has order type $\alpha+1$. If not, let $a_0\in A$ such that $a_0\geq g$. For $a\in A$ such that $a> a_0$ set 
		\centre{$B_a=\{g\}\cup\enstq{a'\in A}{a'< a}$}
		\lc{Since $\alpha$ is limit, we have}{$A\cup \{g\} = \Union{a>a_0}{}B_a$}
		and each of the element in the union is an initial segment of $A\cup\{g\}$. 
		We also denote $\alpha_a$ the order type of the set~$\enstq{a'\in A}{a'< a}$. In particular, $\alpha_a<\alpha$. Using induction hypothesis, $B_a$ has order type at most $\alpha_a+1$. Then, since we have an increasing union of initial segments, the order type of $A\cup\{g\}$ is at most 
		\centre{$\sup\enstq{\alpha_a+1}{a>a_0}=\sup\enstq{\alpha'+1}{\alpha'<\alpha}= \alpha$} 
		since $\alpha$ is a limit ordinal.
	\end{itemize}
	We conclude thanks to the induction principle.
\end{proof}

\begin{proposition}[Union of well-ordered sets, folklore]
	\label{prop:unionEnsBienOrd}
	Let $\Gamma$ be a totally ordered set $A,B\subseteq\Gamma$ be non-empty well-ordered subsets with respective order types $\alpha$ and $\beta$. Then the
	subset $A\cup B$ is well ordered with order type at most $\alpha+\beta$.
\end{proposition}

\begin{proof} $A\cup B$ is well-ordered. Indeed, if we have an infinite decreasing	 	
	sequence of $A\cup B$, then we can extract either an infinite one for either $A$ or $B$ which is not possible. It remains to show the bound on its order type.
	We do it by induction over $\alpha$ and $\beta$.
	\begin{itemize}
		\item If $\alpha=\beta=1$, then $A\cup B$ has at most two elements. Then, its order type is at most $2=\alpha+\beta$.
		
		\item If $\alpha$ or $\beta$ is a successor ordinal. Since both cases are symmetric, we assume without loss of generality that $\beta=\gamma+1$. Let $u$ be the largest element of $B$ and $C=B\setminus\{u\}$. Then, by induction hypothesis, $A\cup C$ has order type at most $\alpha+\gamma$. Using Lemma \ref{lem:ajoutDUnElementEnsBienOrd}, we get that the order type of $A\cup B$ is at most $\alpha+\gamma+1=\alpha+\beta$.
		
		\item If $\alpha$ and $\beta$ are limit ordinal. $A$ or $B$ must be cofinal with $A\cup B$. For instance say it is $A$. For $a\in A$, let
		\centre{$A_a=\enstq{a'\in A}{a'<a}\qqandqq B_a=\enstq{b\in B}{b<a}$}
		\lc{We have}{$A\cup B = \Union{a\in A}{}A_a\cup B_a$}
		Since $A$ is cofinal with $A\cup B$, it is an increasing union of initial segments. Let $\alpha_a$ be the order type of $A_a$ and $\beta_a$ the one of $B_a$. We have $\alpha_a<\alpha$ and $\beta_a\leq\beta$. By induction hypothesis, $A_a\cup B_a$ has order type at most $\alpha_a+\beta_a$. Then $A\cup B$ has order type at most
		\centre{$\sup\enstq{\alpha_a+\beta_a}{a\in A}\leq \alpha+\beta$}
	\end{itemize}
	We conclude the proof using the induction principle.
\end{proof}

We know move to addition of well ordered subset of a group. Again this result in know but its proof is not easily findable in the literature. 

\begin{proposition}[Folklore]
	\label{prop:sommeEnsBienOrd} 
	Let $\Gamma$ be an ordered Abelian additive monoid and $A,B\subseteq\Gamma$ be non-empty well-ordered subsets with respective order types $\alpha$ and $\beta$. Then the
	subset $A+B=\enstq{a+b}{a\in A\quad B\in B}$ is well ordered with order type at most $\alpha\beta$.
\end{proposition}

\begin{proof}
	We do it by induction over $\alpha$ and $\beta$.
	\begin{itemize}
		\item If $\alpha=\beta=1$, then $A+B$ has only one element, then has order type $1=\alpha\beta$.
		
		\item If $\alpha$ or $\beta$ is not an additive ordinal\index{Ordinal number!additive ordinal}. Let say $\beta=\gamma + \delta$ with $\gamma,\delta<\beta$. We choose $\gamma,\delta$ such that $\gamma+\delta=\gamma\oplus\delta$. Let $B_1$ the initial segment of length $\gamma$ of $B$. Let $B_2=B\setminus B_1$. $B_2$ has order type $\delta$. Then, by induction hypothesis, $A+B_1$ has order type at most $\alpha\gamma$ and $A+B_2$ has order type at most $\alpha\delta$. Then, using Proposition \ref{prop:unionEnsBienOrd}, $A+B$ has order type at most $\alpha\gamma+\alpha\delta=\alpha\beta$.
		
		\item If both $\alpha$ and $\beta$ are additive ordinals. Assume $A+B$ has order type more than $\alpha\beta$. Let $a+b\in A+B$ such that the set $C$ defined by
		$$C:=\enstq{c\in A+B}{c< a+b}$$ 
		has order type $\alpha\beta$. Let 
		\centre{$A_0=\enstq{a'\in A}{a'<a}$ and $B_0=\enstq{b'\in B}{b'<b}$}
		and $\alpha_0$ and $\beta_0$ their respective order types. We have
		\centre{$C\subseteq \pa{A_0+B}\cup\pa{A+B_0}$}
		Using induction hypothesis and Proposition \ref{prop:unionEnsBienOrd}, $C$ has order type at most $\alpha_0\beta+\alpha\beta_0$. Since $\alpha_0<\alpha$ and $\beta_0<\beta$, we have $\alpha_0\beta<\alpha\beta$ and $\alpha\beta_0<\alpha\beta$. $\alpha$ and $\beta$ being additive ordinal, $\alpha\beta$ is itself an additive ordinal and then $C$ has order type less than $\alpha\beta$, what is a contradiction. Then $A+B$ has order type at most $\alpha\beta$.
	\end{itemize}
	We conclude thanks to the induction principle.
\end{proof}

In the same idea, we can take a look at a well ordered non-negative subset of an ordered group. The proof is less easy so we refer to \cite{weiermannMaximalOrderType} for the details.

\begin{proposition}[{\cite[Corollary 1]{weiermannMaximalOrderType}}]
	\label{prop:orderTypeMonoid}
	Let $\Gamma$ be an ordered Abelian group and $S\subseteq\Gamma_+$ be a well-ordered subset with order type $\alpha$. Then, $\inner S$, the monoid generated by $S$ in $\Gamma$ is itself well-ordered with order type at most $\omega^{\hat{\alpha}}$
	where, if the Cantor normal form of $\alpha$ is
	\centre{$\alpha=\Sum{i=1}{n}\omega^{\alpha_i}n_i$}
	\lc{then}{$\hat\alpha = \Sum{i=1}{n}\omega^{\alpha_i'}n_i$}
	\lc{and}{$\beta'=\begin{accolade}
			\beta+1 & \text{if $\beta$ is an $\epsilon$-number}\\
			\beta & \text{otherwise}
		\end{accolade}$}
	In particular, $\inner S$ has order type at most $\omega^{\omega\alpha}$ (commutative multiplication).
\end{proposition}

Finally, we consider finite sequences over a well ordered set.

\begin{theorem}[{\cite[Theorem 3.11]{DEJONGH1977195}} and  {\cite[Theorem 2.9]{SchmidtOrderTypes}}]
	\label{thm:borneTypeOrdreSuitesFinies}
	Let $(X,\leq)$ be a well ordered set with order type $\alpha$. Let $X^*$ be the set of finite sequences over $X$. Let $\beta$ the order type of $X^*$. We have
	\centre{$\beta \leq \begin{accolade}
			\omega^{\omega^{\alpha-1}} & \text{if }\alpha\text{ is finite}\\
			\omega^{\omega^{\alpha+1}} & \textit{if }\epsilon\leq\alpha<\epsilon+\omega\text{ for some $\epsilon$-number }\epsilon\\
			\omega^{\omega^\alpha} & \text{ otherwise}
		\end{accolade}$}
\end{theorem}