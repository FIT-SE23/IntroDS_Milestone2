%!TEX root = impan-main.tex
 
 
%\SOURCE{berarducci2018surreal}
%\olivierpourlui{copié collé de BM}

We assume some familiarity with the ordered field of surreal numbers (refer to  \cite{conway2000numbers,gonshor1986introduction} for presentations) which we denote by $\No$. In this section we give a brief presentation of the basic definitions and results, and we fix the notations that will be used in the rest of the paper.



\subsection{Order and simplicity}


The class $\No$ of surreal numbers can be defined either by transfinite recursion, as in  \cite{conway2000numbers} or by transfinite length sequences of $+$ and $-$ as done in 
\cite{gonshor1986introduction}. We  will mostly follow \cite{gonshor1986introduction}, as well as \cite{berarducci2018surreal} for their presentation.

We introduce the class $\No = 2^{<\On}$ of all binary sequences of some ordinal length $\alpha \in \On$, where $\On$ denotes the class of the ordinals. In other words,  $\No$ corresponds to functions of the form $x : \alpha \to \{-,+\}$. The \textbf{length} (sometimes also called \textbf{birthday} in  literature) of a surreal number $x$ is the ordinal number $\alpha = \dom(x)$. We will also write $\alpha=\length{x}$ (the point of this notation is to ``count'' the number of pluses and minuses).
Note that $\No$ is not a set but a proper class, and all the relations and functions we shall define on $\No$ are going to be class-relations and class-functions, usually constructed by transfinite induction. 




We say that $x \in \No$ is \textbf{simpler} than $y \in \No$, denoted $x \simpler y$, i.e., if $x$ is a strict \textbf{initial segment} (also called \textbf{prefix}) of $y$ as a binary sequence. We say that $x$ is simpler than or equal to $y$, written $x \simplereq y$, if $x \simpler y$ or $x = y$ i.e., $x$ is an initial segment of $y$. The simplicity relation is a binary tree-like partial order on $\No$, with the immediate successors of a node $x\in\No$ being the sequences $x_-$ and $x_+$ obtained by appending $-$ or $+$ at the end of the signs sequence of $x$. Observe in particular that the simplicity relation $\simpler $ is well-founded, and the empty sequence, which will play the role of the number zero, is simpler than any other surreal number. 

We can introduce a total order $<$ on $\No$ which is basically the lexicographic order over the corresponding sequences: More precisely,  we consider the order  $-<\square<+$ where $\square$ is the blank symbol. Now to compare two signs sequences,  append blank symbols to the shortest so that they have the same length. Then,  just compare them with the corresponding lexicographic order to get the total order $<$.

Given two sets $A \subseteq \No$ and $B \subseteq \No$ with $A < B$ (meaning that $a < b$ for all $a \in A$ and $b \in B$),
it is quite easy to understand why there is a simplest surreal number, denoted $\crotq AB$ such that $A<\crotq AB < B$.  However, a formal proof is long. See \cite[Theorem 2.1]{gonshor1986introduction} for details.
If $x=\crotq AB$, we say that Such a pair $\crotq{A}{B}$ is \textbf{representation} of $x$.% , and we call $(A; B)$ the associated convex class.

Every surreal number $x$ has several different representations $x = \crotq{A}{ B} = \crotq{A'}{ B'}$, for instance, if $A$ is cofinal with $A'$ and $B$ is coinitial with $B'$. In this situation, we shall say that $\crotq{A}{ B} = \crotq{A'}{ B'}$ by cofinality. On the other hand, as discussed in \cite{berarducci2018surreal}, 
it may well happen that $\crotq{A}{ B} = \crotq{A'}{ B'}$ even if $A$ is not cofinal with $A'$ or $B$ is not coinitial with $B'$. The \textbf{canonical representation} $x = \crotq{A}{ B}$ is the unique one such that $A \cup B$ is exactly the set of all surreal numbers strictly simpler than $x$. Indeed it turns out that is $A=\enstq{y\sqsubset x}{y<x}$ and $B=\enstq{y\sqsubset x}{y>x}$, then $x=\crotq AB$.

\begin{remark} 
By definition, if $x = \crotq{A}{ B}$ and $A < y < B$, then $x \simplereq y$.
\end{remark}
To make the reading easier we may forget $\{\}$ when writing explicitly $A$ and $B$. For instance $\crotq xy$ will often stand for $\crotq{\{x\}}{\{y\}}$ when $x,y\in\No$.



%\olivierpourlui{squizé des trucs}


\subsection{Field operations}

%\olivierpourlui{Attention, vraiment copié collé}
%\olivierpourlui{En plus, lui emphatique des trucs}
%\olivier{Check que unifié représentation du membre gauche, droit}

Ring operations $+$, $·$ on $\No$ are defined by transfinite induction on simplicity as follows:
$$x+y:=\crotq{x' +y, x+y'}{x'' +y, x+y''}$$
$$
xy := \crotq{\begin{array}{c}
		x'y + xy' - x'y' \\ x''y + xy'' - x''y''
	\end{array}}{\begin{array}{c}
		x'y + xy'' - x'y'' \\ x''y + xy' - x''y'
	\end{array}}
$$
where $x'$ (resp. $y'$) ranges over the numbers simpler than $x$ (resp. $y$) such that $x' < x$ (resp. $y'<y$) and $x''$ (resp. $y''$) ranges over the numbers simpler than $x$ (resp. $y$) such that $x < x''$ (resp. $y<y''$); in other words, when $x = \crotq{x'}{x''}$ and $y = \crotq{y'} {y''}$ are the canonical representations of $x$ and $y$ respectively. The expression for the product may seem not intuitive, but actually, it is basically inspired by the fact that we expect $(x-x')(y-y')>0$, $(x-x'')(y-y'')>0$, $(x-x')(y-y'')<0$ and $(x-x'')(y-y')<0$.

\begin{remark}
The definitions of sum and product are uniform in the sense of \cite[page 15]{gonshor1986introduction}. Namely the equations that define $x + y$ and $xy$ does not require the canonical representations of $x$ and $y$ but any representation. In particular, if $x=\crotq AB$ and $y=\crotq CD$, the variables $x', x'', y', y''$ may range over $A$, $B$, $C$, $D$ respectively.
\end{remark}

It is an early result that these operations, together with the order, give $\No$ a structure of ordered field, and even a structure of real closed field (see \cite[Theorem 5.10]{gonshor1986introduction}). 
Consequently, there is a unique embedding of the rational numbers in $\No$ so we can
identify $\Qbb$ with a subfield of $\No$. 
%
Actually, the subgroup of the dyadic rationals $m/2^{n}\in \Qbb$, 
with $m\in\Zbb$ and $n \in \Nbb$, correspond exactly to the surreal numbers $s : k \to \{-, +\}$ of finite length $k \in \Nbb.$

The field $\Rbb$ can be isomorphically identified with a subfield of $\No$ by sending $x\in\Rbb$ to the number $\crotq{A}{ B}$ where $A\subseteq\No$ is the set of rationals (equivalently: dyadics) lower than $x$ and $B\subseteq\No$ is the set of (equivalently: dyadics) greater than $x$. This embedding is consistent with the one of $\Qbb$ into $\No$. We may thus write $\Qbb\subseteq\Rbb\subseteq\No$. By \cite[page 33]{gonshor1986introduction}, the length of a real number is at most $\omega$ (the least infinite ordinal). There are however surreal numbers of length $\omega$ which are not real numbers, such as $\omega$ itself or its inverse that is a positive infinitesimal.  

The ordinal numbers can be identified with a subclass of $\No$ by sending the ordinal $\alpha$ to the sequence $s : \alpha \rightarrow\{+,-\}$ with constant value $+$. Under this identification, the ring operations of $\No$, when restricted to the ordinals $\Ord \subseteq \No$, coincide with the Hessenberg sum and product (also called natural operations) of ordinal numbers. Similarly, the sequence $s : \alpha\rightarrow\{+,-\}$ with constant value $-$ corresponds to the opposite (inverse for the additive law) of the ordinal $\alpha$, namely $-\alpha$. We remark that $x \in \Ord$ if and only if $x$ admits a representation of the form $x = \crotq AB$ with $B=\emptyset$, and similarly $x \in -\Ord$ if and only if we can write $x = \crotq AB$ with $A=\emptyset$.

Under the above identification of $\Qbb$ as a subfield of $\No$, the natural numbers $\Nbb \subseteq\Qbb$ are exactly the finite ordinals.



