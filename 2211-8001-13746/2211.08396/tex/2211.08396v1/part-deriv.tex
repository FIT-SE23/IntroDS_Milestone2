\subsection{Log-atomic numbers}

We now introduce the concept of \textbf{log-atomic numbers}. Log-atomic numbers were first introduced by Schmeling in \cite[page 30]{schmeling2001corps} about transseries. Such number are basically number whose series of iterated logarithm have all length $1$.

\begin{definition}[Log-atomic]
	A positive surreal number $x\in\No^*_+$ is said \textbf{log-atomic} \tiff for all $n\in\Nbb$, there is a surreal number $a_n$ such that $\ln_nx=\omega^{a_n}$. We denote $\Lbb$ the class of log-atomic numbers.
\end{definition}

For instance, $\omega$ is a log-atomic number and we can check that for all $n\in\Nbb$, $\ln_n \omega = \omega^{\f1{\omega^n}}$. Log-atomic number are the number we cannot divide into simpler numbers when considering exponential and logarithm and are the fundamental blocs we end up with when writing $x=\aSurreal$ and then each $\omega^{a_i}$ as $\omega^{a_i}=\exp x_i$ with $x_i$ a purely infinite number and then doing the same thing with each of the $x_i$s. The use of the word ``simpler'' is not innocent. Indeed, log-atomic numbers are also the simplest elements for some equivalence relation introduced by Beraducci and Mantova \cite{Berarducci_2018}.

\begin{definition}[{\cite[Definition 5.2]{Berarducci_2018}}]
	\label{def:asympPrecL}
	Let $x,y$ be two positive infinite surreal numbers. We write
	\begin{itemize}
		\item $x\asymp^L y$\index{$\asymp^L$, Definition \ref{def:asympPrecL}} \tiff there are some natural numbers $n,k$ such that $$\exp_n\pa{\f1k\ln_n y}\leq x\leq \exp_n\pa{k\ln_ny}$$
		Equivalently, we ask that the is a natural number $n$ such that  $\ln_n x \asymp \ln_n y$. For such $n$ we notice that $\ln_{n+1}x\sim\ln_{n+1}y$.
		\item $x\prec^L y$\index{$\prec^L,\preceq^L$, Definition \ref{def:asympPrecL}} \tiff for all natural numbers $n$ and $k$, 
		$$x<\exp_n\pa{\f1k\ln_n y}$$
		Equivalently, we ask that for all $n\in\Nbb$, $\ln_n x\prec\ln_n y$.
		\item $a\preceq^L b$ \tiff there are some natural numbers $n$ and $k$, 
		$$x\leq\exp_n\pa{\f1k\ln_n y}$$
		Equivalently, we ask that for some $n\in\Nbb$, $\ln_n x\preceq\ln_n y$.
	\end{itemize}
\end{definition}

Log-atomic number are closely related to this equivalence relation since they representatives of each equivalence classes.

\begin{proposition}[{\cite[Propositions 5.6 and 5.8]{Berarducci_2018}}]
	\label{prop:lamdaMapLogAtomic}
	For all positive infinite $x$ there is unique log-atomic number $y\in\Lbb$ such that $y\sqsubseteq x$ and such that $y\asymp^L x$. In particular, if $x,y\in\Lbb$ with $x<y$ then $x\prec^L y$.
\end{proposition}

This proposition shows in particular that not even log-atomic are representative of the equivalence classes of $\asymp^L$, they also are the simplest element (\ie{} the shortest in terms of length) in their respective equivalence classes. This make them a canonic class of representatives.

As we can parametrize additive ordinal, multiplicative ordinal or even $\epsilon$-numbers (for which a generalization for surreal numbers exists in Gonshor's book \cite{gonshor1986introduction}), we can parametrized epsilon numbers by a an increasing function $\lambda_\cdot$. A first conjecture was to consider $\kappa$-numbers which are defined by Kuhlmann and Matusinski as follows:

\begin{definition}[{\cite[Definition 3.1]{Kuhlmann_2014}}]\label{def:kappaMap}
	Let $x$ be a surreal number and write it in canonical representation as $x=\crotq{x'}{x''}$. Then we define
	$$
	\kappa_x=\crotq{\Rbb,\exp_n \kappa_{x'}}{\ln_n \kappa_{x''}}
	$$
\end{definition}

Intuitively, $x<y$ \tiff every iterated exponential of $\kappa_x$ is less than $\kappa_y$ and we try to build them as simple as possible. As an example, it is quite easy to see that $\kappa_0=\omega$, $\kappa_{-1}=\omega^{\omega^{-\omega}}$ and $\kappa_1=\epsilon_0$. It was conjectured that $\Lbb$ consists in $\kappa$-number and there iterated exponentials and logarithms. As shown by Berarducci and Mantova, it turns out that it is not true. They then suggest a more general map which is the following:

\begin{definition}[{\cite[Definition 5.12]{Berarducci_2018}}]\label{def:lambdaMap}
	Let $x$ be a surreal number and write it in canonical representation $x=\crotq{x'}{x''}$. Then we define
	$$
	\lambda_x=\crotq{\Rbb,\exp_n\pa{k\ln_n \lambda_{x'}}
	}{\exp_n\pa{\f1k\ln_n\lambda_{x''}}}
	$$
	where $n,k\in\Nbb^*$.
\end{definition}

\begin{proposition}[{\cite[Proposition 5.13 and Corollary 5.15]{Berarducci_2018}}]
	The function $x\mapsto\lambda_x$ is well defined, increasing, satisfies the uniformity property\index{Unifromity property} and if $x<y$ then $\lambda_x\prec^L\lambda_y$.
\end{proposition}

\begin{proposition}[{\cite[Proposition 5.16]{Berarducci_2018}}]
	For every $x\in \No$ with $x > \Rbb$ there is a unique $y\in\No$ such
	that $x\asymp^L\lambda_y$ and $\lambda_y\sqsubseteq x$. In particular, $\lambda_y$ is the simplest number in its equivalence class for $\asymp^L$. As a consequence, $\lambda_\No=\Lbb$.
\end{proposition}

Moreover, the $\lambda$ map behaves very nicely with exponential and logarithm.

\begin{proposition}[{\cite[Proposition 2.5]{aschenbrenner:hal-02350421}}]
	For all surreal number $x$, 
	$$
	\exp\lambda_x=\lambda_{x+1}\qqandqq\ln\lambda_x=\lambda_{x-1}
	$$
\end{proposition}

\begin{lemma}[{\cite[Lemma 2.6]{aschenbrenner:hal-02350421}}]
	For all ordinal $\alpha$, $\lambda_{-\alpha}=\omega^{\omega^{-\alpha}}$.
\end{lemma}

\begin{lemma}[{\cite[Aschenbrenner, van den Dries and  van der Hoeven, Corollary 2.9]{aschenbrenner:hal-02350421}}]
	\label{lem:kappaMoinsAlpha}
	For all ordinal number $\alpha$, \centre{$\kappa_{-\alpha}=\lambda_{-\omega\otimes\alpha}=\omega^{\omega^{-\omega\otimes\alpha}}$}
\end{lemma}
\subsection{Nested truncation rank}
\subsubsection{Definition}
Log-atomic number are the base case (up to minor changes) of a notion of rank over surreal numbers, the \textbf{nested truncation rank}. As expected, it is based on some well partial order. This one has been defined by Berarducci and Mantova as follows:

\begin{definition}[{\cite[Definition 4.3]{Berarducci_2018}}]
	\label{def:nestedn}
	For all natural number $n\in\Nbb$, we define the relation $\nestedneq$ as follows:
	\begin{itemize}
		\item Writing $y\nestedneq[0] x$ if any only if $y=\aSurrealPrefix$ and $x=\aSurreal$ with $\nu'\leq\nu$. We say that $y$ is a \textbf{truncation} of $x$.
		
		\item Let $x=\aSurreal$  Since $\omega^\Nobf=\exp(\No_\infty)$, we can write 
		$$
		x=\Sumlt i\nu r_i\exp(x_i)
		$$
		where $\exp(x_i)=\omega^{a_i}$. For a surreal number $y$, we say $y\nestedneq[n+1] x$ if there is $\nu'<\nu$ and $y'\nestedneq x_{\nu'}$ such that
		$$
		y=\Sumlt i{\nu'}r_i\exp(x_i) + \sign(r_{\nu'})\exp y'
		$$ 
		We say that $y$ is a \textbf{nested truncation} of $x$.
	\end{itemize}
	We also write $y\nestedeq x$ is there is some natural number $n$ such that $y\nestedneq x$. We also introduce the corresponding strict relations $\nestedn$ and $\nested$.
\end{definition}

\begin{definition}[Nested truncation rank {\cite[Definition 4.27]{Berarducci_2018}}]
	\label{def:nestedTruncRk}
	The \textbf{nested truncation rank} of $x\in\Nobf^*$ is defined by
	$$
	\NR(x) = \sup\enstq{\NR y + 1}{y\nested x}
	$$
	By convention, we also set $\NR(0)=0$.
\end{definition}

\subsubsection{Properties}
We know investigate some properties of the nested truncation rank. More precisely, we provide compatibility properties with the operations over surreal numbers and bounds on some particular nested truncation ranks. First of all, the nested truncation rank is unaffected by the exponential.

\begin{proposition}[{\cite[Proposition 4.28]{Berarducci_2018}}]
	\label{prop:NRStableExp}
	If $\gamma\in\Nobf_\infty$, then $$\NR(\pm\exp \gamma) = \NR(\gamma)$$
\end{proposition}

\begin{corollary}
	\label{cor:NRminus}
	For all $a\in\Nobf^*$, $\NR(a)=\NR\pa{-a}$
\end{corollary}
\begin{proof} Without loss of generality, we assume that $a>0$. Then
	\begin{calculs}
		& \NR\pa{a} &=& \NR\pa{\ln a} & (Proposition \ref{prop:NRStableExp})\\
		&&=& \NR(-\exp\ln a) &(Proposition \ref{prop:NRStableExp}) \\
		&&=& \NR\pa{-a}\\
	\end{calculs}
\end{proof}

\begin{corollary}
	\label{cor:NRinverse}
	For all $a\in\Nobf^*$, $\NR(a)=\NR\pa{\f1a}$
\end{corollary}
\begin{proof}\ 
	\begin{calculs}
		& \NR\pa{\f1a} &=& \NR\pa{\ln\f1a} & (Proposition \ref{prop:NRStableExp})\\
		&&=& \NR(-\ln a)\\
		&&=& \NR\pa{\ln a} &(Corollary \ref{cor:NRminus})\\\\
		&&=& \NR(a) & (Proposition \ref{prop:NRStableExp})\\
	\end{calculs}
\end{proof}

\begin{lemma}\label{lem:NR0}
	For all $x\in\No$, $\NR(x)=0$ \tiff either $x\in\Rbb$ or $x=\pm\lambda^{\pm1}$ for some log-atomic number $\lambda$.
\end{lemma}

\begin{proof}
	\begin{itemize}
		\item[\CS] Note that if $x\in\Rbb$ then there is no $y\in\Nobf$ such that $y\nested x$. Therefore $\NR(x)=0$. 
		Now assume that there is some $x=\pm\lambda^{\pm 1}$ with $\lambda\in\Lbb$ such that $\NR(x)\neq 0$. Therefore there is some $y\in\Nobf$ such that $y\nested x$. Let $n\in\Nbb$ minimal such that there is $y\in\Nobf$ and $\lambda\in\Lbb$ such that $y\nestedn \pm\lambda^{\pm 1}$.
		Note that since $\pm\lambda^{\pm1}$ is a term, $n>0$. Then $y=\pm\exp(\pm y')$ with $y'\nestedn[n-1]\ln\lambda\in\Lbb$. But this contradicts the minimality of $n$. hence, for all $\lambda\in\Lbb$, $\NR\pa{\pm\lambda^{\pm1}}=0$.
		
		\item[\CN] Assume $\NR(x)=0$ and $x$ is not a real number. If $x$ is not a term, then there is $y\nestedn[0] x$ and in particular $\NR(x)\geq 1$, what is impossible. Therefore there is some $r\in\Rbb^*$ and some $x_1\in\Jbb$ such that $x=r\exp(x_1)$. If $r\neq \pm1$ then $\sign(x)\exp(x')\nested x$ what is again impossible. Hence, $x=\pm\exp(x_1)$. Proposition \ref{prop:NRStableExp} ensures that $\NR(x_1)=0$. We then can apply the same work to $x_1$ so that there is some $x_2\in\Jbb$ such that $x_1=\pm\exp(x_2)$. By induction, we can always define $x_n=\pm\exp(x_{n+1})$ with $x_{n+1}\in\Jbb$. For $n\geq 1$ we have $x_n\in\Jbb$, therefore $x_{n+1}>0$. In particular
		$$
		\forall n\geq 2\qquad x_n=\exp(x_{n+1})
		$$
		So, for all $n\in\Nbb$,  $\ln_n x_2$ is a monomial, this means that $x_2\in\Lbb$. We also have 
		$$x=\pm\exp\pa{\pm\exp x_2}=\pm\pa{\exp_2(x_2)}^{\pm1}$$
		Since $\exp_2 x_2\in\Lbb$, we have the expected result.
	\end{itemize}
\end{proof}

\begin{lemma}
	\label{lem:NRAjoutNouveauTerme}
	Let $x=\aSurreal$ and $r\in\Rbb^*, a\in\Nobf$ such that for all $i<\nu$, $r\omega^a\prec\omega^{a_i}$. Then 
	\centre{$\NR(x+r\omega^a) = \NR(x)\oplus1\oplus\NR(\omega^a)\oplus\ind_{r\neq\pm1}$}
	where the $\oplus$ is the usual sum on ordinal numbers.
\end{lemma}

\begin{proof}
	Let $y\nested x+r\omega^a$. Then $y\nestedeq x$ or $y=x+\sign(r)\exp(\delta)$ with $\delta\nested\ln\omega^a$ or, if $r\neq\pm 1$,  $y=x+\sign(r)\omega^a$.
	Let \centre{$A=\enstq{y}{y\nestedeq x}\qqandqq B=\enstq{x+\sign(r)\exp(\delta)}{\delta\nested \ln\omega^a}$}
	\lc{and}{$C=\begin{accolade}
			\emptyset & r=\pm 1 \\
			x+\sign(r)\omega^a & r\neq\pm 1
		\end{accolade}$}
	One can easily see that
	\centre{$\forall y\in A\quad \forall y'\in B\quad\forall y''\in C \qquad y\nested y'\wedge y\nested y'' \wedge y'\nested y''$}
	We now proceed by induction on $\NR(\omega^a)$.
	\begin{itemize}
		\item If $\NR(\omega^a)=0$, using Lemma \ref{lem:NR0}, either $\omega^a=\pm\lambda^{\pm 1}$ for some log-atomic number $\lambda$ or $a=0$. In both cases, there is no $\delta\nested\ln\omega^a$. 
		\begin{calculs}
			& \NR(x+r\omega^a) &=& \sup\enstq{\NR(y)+1}{y\in A\cup C}\\
			&&=& \sup\pa{\enstq{\underbrace{\NR(y)+1}_{\leq\NR(x)}}{y\nested x}\right.\\&&&\left.\qquad\vphantom{\underbrace{\NR(y)+1}_{\leq\NR(x)}}\cup\{\NR(x)+1\} \cup\enstq{\NR(y)+1}{y\in C}} \\
			&&=& \begin{accolade}
				\NR(x) + 1 & r=\pm 1 \\
				\NR(x+\sign(r)\omega^a) & r\neq \pm 1 
			\end{accolade}\\
			&&=& \begin{accolade}
				\NR(x) + 1 & r=\pm 1 \\
				\NR(x)+2 & r\neq \pm 1 
			\end{accolade}\\
			& \NR(x+r\omega^a) &=& \NR(x)+1+\NR(\omega^a) + \ind_{r\neq\pm 1}
		\end{calculs}
		\item For heredity now. Let $\delta\nested\ln\omega^a$. Since $\ln\omega^a$ is a purely infinite number, so is $\delta$. Then $\exp\delta$ is of the form $\omega^b$ for some surreal $b\in\Nobf$. Moreover
		\centre{$\NR(\omega^b) \underset{\text{Proposition }\ref{prop:NRStableExp}}{=} \NR(\delta) < \NR(\ln\omega^a) \underset{\text{Proposition }\ref{prop:NRStableExp}}{=} \NR(\omega^a)$}
		From the induction hypothesis, we have that for any $\delta\nested\ln\omega^a$
		\centre{$\NR(x+\sign(r)\exp(\delta)) = \NR(x)\oplus1\oplus\NR(\exp\delta)$}
		\begin{calculs}
			& \NR(x+r\omega^a) &=& \sup\enstq{\NR(y)+1}{y\in B\cup C}\\
			&&=& \sup\pa{\enstq{\underbrace{\NR(x+\sign(r)\exp\delta)+1}_{\leq\NR(x+\sign(r)\omega^a)}}{\delta\nested \ln\omega^a}\right. \\ &&&\left.\qquad\cup\enstq{\NR(y)+1}{y\in C} \vphantom{\underbrace{\NR(x+\sign(r)\exp\delta)+1}_{\leq\NR(x+\sign(r)\omega^a)}} } \\
			&&=& \sup\enstq{\NR(x)\oplus1\oplus\NR(\exp\delta)\oplus1}{\delta\nested\ln\omega^a}+\ind_{r\neq\pm1}\\
			&&=& \NR(x)\oplus1\oplus\sup\enstq{\NR(\exp\delta)+1}{\delta\nested\ln\omega^a}\oplus\ind_{r\neq\pm1}\\
			& \NR(x+r\omega^a) &=& \NR(x)\oplus1\oplus\NR(\omega^a)\oplus\ind_{r\neq\pm1}
		\end{calculs}
	\end{itemize}
\end{proof}

\begin{lemma}
	\label{lem:NRSommeLogAtomiques}
	Let $x=\aSurreal$ such that for all $i<\nu$, $r_i=\pm 1$ and $\omega^{a_i}=\lambda_i^{\pm1}$ for some $\lambda\in\Lbb$. Then
	\centre{$\NR(x) = \begin{accolade}
			\nu +1 & \nu<\omega\\ \nu & \nu\geq\omega
		\end{accolade}$}
\end{lemma}
\begin{proof}
	If $\nu<\omega$, we just proceed by induction using Lemma \ref{lem:NRAjoutNouveauTerme}. Now we prove by induction the remaining. 
	\begin{itemize}
		\item If $\nu=\omega$. Then 
		\centre{$\NR(x)=\sup\enstq{\NR\pa{\aSurrealPrefix}+1}{\nu'<\nu}=\sup\enstq{\nu'+2}{\nu'<\omega}=\omega$}
		
		\item Assume for $\omega\leq\nu'<\nu$, $\NR\pa{\aSurrealPrefix} = \nu'$. If $\nu$ is a non-limit ordinal, then Lemma \ref{lem:NRAjoutNouveauTerme} concludes. Otherwise
		\centre{$\NR(x)=\sup\enstq{\NR\pa{\aSurrealPrefix}+1}{\nu'<\nu}=\sup\enstq{\nu'+^1}{\omega\leq\nu'<\nu}=\nu$}
	\end{itemize}
\end{proof}

\begin{lemma}
	\label{lem:NRvsNu}
	Let $x=\Sumlt i\nu r_i(x)\omega^{a_i(x)}\in\Nobf$. Then $\nu\leq\NR(x)+1$. The equality stands \tiff $x$ is a finite sum of numbers of the form $\pm y^{\pm1}$ with $y\in\Lbb$ and possibly one non-zero real number.
\end{lemma}

\begin{proof}
	Using induction on $\nu$ it is trivial. For $0$, $\nu=0=\NR(0)$. Now assume $\nu\neq 0$. Then, by definition
	\centre{$\NR(x)+1\geq\sup\enstq{\NR(y)+1}{y\nestedn[0]x\quad y\neq 0}+1\underset{\text{induction hypothesis}}{\geq}
		\sup\enstq{\nu(y)}{y\nestedn[0]x\quad y\neq0} +1 \geq \nu(x)$}
	
	Now assume $\nu(x)=\NR(x)+1$ and write $x=\Sumlt{i}{\nu(x)}r_i\omega^{a_i}$. We use induction on $\Nobf^*$ with the well partial order $\nestedn[0]$.
	
	\begin{itemize}
		\item If $x$ is a monomial, $\nu(x)=1$ and $\NR(x)=0$. That is $x=\pm y^{\pm1}$ for some $y\in\Lbb$ or $x\in\Rbb$ (using Lemma \ref{lem:NR0}).
		
		\item If $x$ is not a monomial.  Assume $r_i\omega^{a_i}\notin\pm\Lbb^{\pm 1}\cup\Rbb^*$ with $i$ minimal for that property. Let $x'=\Sumlt ji r_j\omega^{a_j}$. 
		\begin{itemize}
			\item If $i=0$ then $\NR(r_0\omega^{a_0})\geq 1$. A simple induction shows that $\NR\pa{\Sumlt i{\nu'}r_i\omega^{a_i}}\geq \nu'$ for all $\nu'\leq\nu$. What is a contradiction.
			\item Otherwise $x'\neq 0$ and $x'\nestedn[0]x$. If $\NR(x')+1\neq i$ then $\NR(x')\geq i$ and 
			\centre{$\NR(x)\geq\NR(x') \oplus (\nu\ominus i)\geq \nu$}
			where $\nu\ominus i$ is the ordinal such that $i\oplus(\nu\ominus i)=\nu$.
			what is a contraction. Then by induction hypothesis, $i=\NR(x')+1$ is finite. Now consider
			$y\nested x'+r_i\omega^{a_i}$. Then $y\nestedneq[0] x'$ ($y\nestedn x'$ with $n\geq 1$ is impossible since $x'$ has only terms in $\pm\Lbb^{\pm 1}\cup\Rbb$) or $y=x'+\sign(r_i)\exp(\delta)$ with $\delta\nestedeq \ln(\omega^{a_i})$. Since $r_i\omega^{a_i}\notin\pm\Lbb^{\pm 1}\cup\Rbb$, there is such a $y$ of the later form such that $y\neq x'+r_i\omega^{a_i}$. From Lemma \ref{lem:NRAjoutNouveauTerme}, we have $\NR(y)\geq\NR(x')+1$. Then $\NR(x'+r_i\omega^{a_i})\geq\NR(y)+1\geq\NR(x')+2$. By induction we then can show that 
			\centre{$\NR(x)\geq \NR(x'+r_i\omega^{a_i})\oplus(\nu-(i+1))\geq\NR(x')\oplus2\oplus(\nu\ominus(i\oplus 1)) = i\oplus1+(\nu\ominus(i\oplus1))=\nu$}
			and we get a contradiction.
		\end{itemize}
		Then, every term of $x$ is in $\pm\Lbb^{\pm1}\cup\Rbb$ and by definition only one can be a non-zero real number. It remains to show that there are finitely many terms, what follows from Lemma \ref{lem:NRSommeLogAtomiques}.
	\end{itemize}
\end{proof}

\begin{remark}
	\label{rk:NRvsLength}
	For all $x\in\Nobf$, $\NR(x)\leq \length x$
\end{remark}

\begin{proof}
	Assume the converse and take $x$ with minimal length that contradicts the property then there is $y\nested x$ such that $\NR(y)\geq \length x$. Since $\length x > \length y$, then $y$ reaches contradiction with the minimality of $x$.
\end{proof}


\begin{proposition}[{\cite[Berarducci and Mantova, Proposition 4.29]{Berarducci_2018}}]
	\label{prop:NRComparaisonCoefReel}
	For all $a\in\Nobf^*$, for all $r\in\Rbb\setminus\{\pm1\}$, we have $\NR(r\omega^a)=\NR(\omega^a)+1$.
\end{proposition}
\begin{proposition}[{\cite[Berarducci and Mantova, Proposition 4.30]{Berarducci_2018}}]
	\label{prop:NRComparaisonTerme}
	Let $x=\aSurreal\in\Nobf^*$. Then 
	\begin{itemize}
		\item $\forall i<\nu\qquad \NR(r_i\omega^{a_i})\leq\NR(x)$
		\item $\forall i<\nu\qquad i+1<\nu\Rightarrow \NR(r_i\omega^{a_i})<\NR(x)$
	\end{itemize}
\end{proposition}

We can also say something about the nested truncation rank of a sum of surreal number.

\begin{lemma}
	\label{lem:NRsum}
	For $a,b\in\Nobf$, $\NR(a+b)\leq\NR(a)+\NR(b)+1$ (natural sum of ordinal, which correspond to the surreal sum).
\end{lemma}

\begin{proof}
	We prove it by induction on the couple $(\NR(a),\NR(b))$.
	\begin{itemize}
		\item If $\NR(a)=\NR(b)=0$ then, by Lemma \ref{lem:NR0} both $a,b$ are in $\pm\Lbb^{\pm1}\cup\Rbb$. If $a\in\Rbb$ or $b\in\Rbb$ then $\NR(a+b)\leq 1$ by Lemmas \ref{lem:NRAjoutNouveauTerme} and \ref{lem:NR0}. Otherwise, either $a=\pm b$ and then $\NR(a+b)=0$ or $a\neq\pm b$ and Lemma \ref{lem:NRvsNu} ensure that $\NR(a+b)=1$.
		
		\item Assume the property for all $x,y$ such that 
		$$(\NR(x),\NR(y))<_{lex}(\NR(a),\NR(b))$$
		Then, consider $y\nested a+b$. Write $a+b=\aSurreal$.
		\begin{itemize}
			\item If $y=\aSurrealPrefix$ with $\nu'<\nu$. Let $z_a$ be the series constituted of the terms of $a$ which asolute value is infintely larger than $\omega^{a_\nu'}$. We define the same way $z_b$. Then $y=z_a+z_b$. We have $(\NR(z_a),\NR(z_b))<_{lex}(\NR(a),\NR(b))$ since there is term with order of magnitude $\omega^{a_{\nu'}}$ in either $a$ or $b$. Then, applying induction hypothesis, 
			\centre{$\NR(y)\leq\NR(z_a)+\NR(z_b)+1$}
			
			Since we have at least one of the following inequalities $z_a\nestedn[0]a$ or $z_b\nestedn[0]b$, then $\NR(z_a)+1\leq \NR(a)$ or $\NR(z_b)+1\leq \NR(b)$. In all cases
			\centre{$\NR(y)+1\leq\NR(a)+\NR(b)+1$}
			
			\item If $y=\aSurrealPrefix+\sign(r_{\nu'})\exp(y')$ with $\nu'<\nu$ and $y'\nestedeq \ln\omega^{a_{\nu'}}$ (and $y\nested\ln\omega^{a_{\nu'}}$ if $r_{\nu'}=\pm1$). Let $z_a$ be the series constituted of the terms of $a$ which absolute value is infinitely larger than $\omega^{a_\nu'}$. We define the same way $z_b$. Then $y=z_a+z_b+\sign(r_{\nu'})\omega^{a_{\nu'}}$. 
			Since there is term with order of magnitude $\omega^{a_{\nu'}}$ with the same sign as $r_{\nu'}$ in either $a$ or $b$. Without loss of generality, assume it is $a$. Then $z_a+\sign(r_{\nu'})\exp y'\nestedeq a$. 
			We have 
			$$(\NR(z_a+\sign(r_{\nu'})\exp y'),\NR(z_b))<_{lex}(\NR(a),\NR(b))$$
			otherwise $y=a+b$ what is not the case. Then, applying induction hypothesis, 
			\centre{$\NR(y)\leq\NR(z_a+\sign(r_{\nu'})\exp y')+\NR(z_b)+1$}
			Since we have at least one of the following inequalities $z_a+\sign(r_{\nu'})\exp y'\nested a$ or $z_b\nestedn[0]b$, then we have either 
			\centre{$\NR(z_a+\sign(r_{\nu'})\exp y')+1\leq \NR(a)$}
			\lc{or}{$\NR(z_b)+1\leq \NR(b)$}
			In all cases
			\centre{$\NR(y)+1\leq\NR(a)+\NR(b)+1$}
		\end{itemize}
		Then, for any $y\nested a+b$, $\NR(y)+1\leq\NR(a)+\NR(b)+1$. This proves that
		\centre{$\NR(a+b)\leq\NR(a)+\NR(b)+1$}
	\end{itemize}
\end{proof}

\begin{corollary}
	\label{cor:NRprod}
	For all $a,b\in\Nobf$, $\NR(ab)\leq\NR(a)+\NR(b)+1$.
\end{corollary}

\begin{proof}\ 
	\begin{calculs}
		We have & \NR(ab) &=& \NR(\ln\pa{ab}) & (Proposition \ref{prop:NRStableExp})\\
		&&=&\NR(\ln a + \ln b)\\
		&&\leq&\NR(\ln a) + \NR(\ln b)+1 & (Lemma \ref{lem:NRsum}) \\
		&&\leq&\NR(a)+\NR(b)+1& (Proposition \ref{prop:NRStableExp})\\
	\end{calculs}
\end{proof}

\subsubsection{Paths}
Surreal numbers can be seen as trees. More precisely, it is possible to associate to each surreal number a tree (with an ordinal numbers of node at each layers) whose leaves are labeled by log-atomic numbers or $0$. This gives us some information about the structure of the surreal number. With this notion of tree we can look at the \textbf{paths} from the root (labeled by the surreal number itself) to the leaves that are not labeled by $0$ (actually there is at most one such a leaf). More precisely, the tree associated to a surreal number $x$ is built as follows:
\begin{itemize}
	\item Base case: if $x\in\Lbb$ or $x=0$ just create a node labeled by $x$ and stop the construction.
	
	\item Otherwise: 
		\begin{enumerate}
			\item Put a node at the root and label is t $x$. Write $x$ under the form $x=\Sumlt i\nu r_i\exp(x_i)$ where $r_i\in\Rbb^*$, $\nu$ is an ordinal and $x_i\in\Nobf_\infty$ form a decreasing sequence.
			
			\item For all $i<\nu$ create built the tree for $x_i$ and link its root to $x$ by an edge labeled by $r_i$.
		\end{enumerate}
\end{itemize}

With the a notion, it is possible to have a geometric interpretation of the well partial order $\nested$.

\centre{\includegraphics[scale=.6]{ordreTronc.pdf}}

The dotted arrows from ``$\sign$'' are to be understood by the fact that we can apply the sign function or not to this arrow. The plain one means that we must apply it. Thanks to this figure we can understand $y\nested x$ by the fact that the tree representation of $y$ is a left-part of the tree representation of $x$.

\begin{remark}
	The reason why we stop the construction on log-atomic numbers is because if we proceed the construction, we would get an infinite path where each node as exactly one child and where every edge is labeled by $1$.
\end{remark}

This notion of tree comes with a notion of path\index{Path} inside the tree.

\begin{definition}\label{def:path}
	Let $x$ be a surreal number. A path $P$ of $x$ is sequence $P:\Nbb\to\Nobf$ such that
	\begin{itemize}
		\item $P(0)$ is a term of $x$
		\item For all $i\in\Nbb$, $P(i+1)$ is an infinite term of $\ln |P(i)|$
	\end{itemize}
	
	We denote $\Pcal(x)$ the set of all paths of $x$.
	
	We also denote $\ell(x)$ to be the purely infinite part of $\ln|x|$. Then $P(i+1)$ is an infinite term of $\ell(P(i))$.
\end{definition}

\begin{definition} The \textbf{dominant path} of $x$ is the path such that
	\begin{itemize}
		\item $P(0)$ is the leading term of $x$
		\item $P(i+1)$ is the leading term of $\ln|P(i)|$.
	\end{itemize}
\end{definition}

In a more graphical point of view, the dominant path of $x$ is the left most path in the tree of $x$ that does not end on the lead $0$. This reduce to the left most path if $x\not\asymp 1$.

\begin{proposition}
	\label{prop:cheminMajorationNu}
	Let $x\in\Nobf$ and $P\in\Pcal(x)$. Then for any $n\in\Nbb$, the length of the serie of $\ell(P(n))$, $\nu(\ell(P(n)))$ satisfies
	\centre{$\nu(\ell(P(n)))\leq \NR(x)+1$}
\end{proposition}

\begin{proof}
	For any $x\in\Nobf$ we write $x=\Sumlt i{\nu(x)}r_i(x)\omega^{a_i(x)}$ in Gonshor's normal form. Now fix $x\in\Nobf$.
	Let $P\in\Pcal(x)$. We set $x_0=x$, and $\alpha_0<\nu(x)$ such $P(0)=r_{\alpha_0}(x)\omega^{a_{\alpha_0}(x_0)}$ and for any natural number $n$,
	\centre{$x_{n+1}=\ln\omega^{a_{\alpha_n}(x_n)} = \ell(P(n))$}
	\lc{and}{$P(n+1)=r_{\alpha_{n+1}}\omega^{a_{\alpha_{n+1}}(x_{n+1})}$}
	Using Proposition \ref{prop:NRStableExp}, we get
	\centre{$\NR(x_{n+1})=\NR\pa{\omega^{a_{\alpha_n}(x_n)}}$}
	By definition $x_{n+1}$ is purely infinite. Then $a_{\alpha_{n+1}}(x_{n+1})>0$ for all natural number $n$. Since $P$ is path, $P(0)\notin\Rbb$ (otherwise $P(1)$ is not defined) and then $a_{\alpha_0}(x_0)\neq 0$. We then can apply Proposition \ref{prop:NRComparaisonCoefReel} and get for all natural number $n$
	\centre{$\NR(x_{n+1})\leq \NR\pa{r_{\alpha_n}(x_n)\omega^{a_{\alpha_n}(x_n)}}$}
	\lc{Now using Proposition \ref{prop:NRComparaisonTerme},}{$\NR(x_{n+1})\leq\NR(x_n)$}
	Then for any natural number $n$ we have $\NR(x_n)\leq \NR(x_0)=\NR(x)$. Applying Lemma \ref{lem:NRvsNu}, we get
	\centre{$\forall n\in\Nbb\qquad \nu(x_n)\leq\NR(x_n)+1\leq\NR(x)+1$}
\end{proof}

\begin{remark}
	Actually, we often have $\nu(\ell(P(n)))\leq\NR(x)$. Indeed, using the notations of the proof and assuming that~$\nu(x_{n+1})=\NR(x)+1$, we have
	\centre{$\NR(x)+1=\nu(x_{n+1})\underset{\text{Proposition \ref{prop:cheminMajorationNu}}}{\leq}\NR(x_{n+1})+1 \leq\cdots\leq \NR(x)+1$}
	Then, all the inequalities are equalities and from Proposition \ref{prop:cheminMajorationNu} we get that $x_{n+1}$ is a finite sum of terms of the form $\pm\Lbb^{\pm1}$, in particular $\nu(x_{n+1})<\omega$ and $\NR(x)$ is finite.
\end{remark}

\subsection{Derivative of a surreal number}

\begin{definition}[Summable family]
	\label{def:summableFam}\index{Summable family!Definition \ref{def:summableFam}}
	Let $\famille{x_i}iI$ be a family of surreal numbers. For $i\in I$ write 
	$$
	x_i=\Sumin a\Nobf r_{i,a}\omega^a
	$$The family $\famille{x_i}iI$ is \textbf{summable} \tiff
	\begin{enumerate}[label=(\roman*)]
		\item $\Unionin iI\supp x_i$ is a reverse well ordered set.
		\item For all $a\in\Unionin iI\supp x_i$, $\enstq{i\in I}{a\in\supp x_i}$ is a finite set.
	\end{enumerate}
	In this case, its sum is defined as $\Sumin iI x_i=\Sumin a\Nobf s_a\omega^a$ where for all $a\in\Nobf$,
	$$
	s_a=\Sum{i\in I\tq a\in\supp x_i}{} r_{i,a}
	$$
	which is a finite sum.
\end{definition}

\begin{definition}[{\cite[Berarducci and Mantova, Definition 6.1]{Berarducci_2018}}]
	A derivation $D$ over a totally ordered exponential (class)-field $\Kbb\supseteq\Rbb$ is a function $D:\Kbb\to\Kbb$ such that
	\begin{enumerate}[label={\textbf{D\arabic*.}}]
		\item\label{ax:D1} \lcr{It satisfies}{$\forall x,y\in\Kbb\qquad D(xy)=xD(y)+D(x)y$}{(Liebniz Rule)}
		
		\item\label{ax:D2} \lcr{If $\famille{x_i}iI$ is summable,}{$D\pa{\Sum{i\in I}{}x_i} = \Sum{i\in I}{}D(x_i)$}{(Strong additivity)\index{Strong additivity}}
		
		\item\label{ax:D3} $\forall x\in\Kbb\qquad D(\exp x)=D(x)\exp x$ 
		
		\item\label{ax:D4} $\ker D = \Rbb$
		
		\item $\forall x>\Nbb\qquad D(x)>0$
	\end{enumerate}
\end{definition}

\begin{remark}
	We can replace Axiom \ref{ax:D2} by
	\begin{enumerate}[label={\textbf{D\arabic*'.}}]
		\setcounter{enumi}{1}
		\item\label{ax:D2p} If $\famille{x_i}iI$ is summable and $\famille{r_i}iI$ is a family of real numbers,
		\lcr{}{$D\pa{\Sum{i\in I}{}r_i x_i} = \Sum{i\in I}{}r_iD(x_i)$}{(Strong lineraity)\index{Strong lineraity}}
	\end{enumerate}
	Indeed, we have
	\lc{}{$
		\ref{ax:D2p}\implies\ref{ax:D2}\qqandqq \ref{ax:D1}\wedge \ref{ax:D2}\wedge \ref{ax:D4}\implies\ref{ax:D2p}$}
\end{remark}

Berarducci and Mantova \cite{Berarducci_2018} provided a general way to define derivation over the class-field $\Nobf$. We recall quickly some of their results.

\begin{proposition}[{\cite[Berarducci and Mantova, Proposition 6.4]{Berarducci_2018}}]
	\label{prop:compareDerivative}
	We have the following properties for a derivation $D$:
	\begin{itemize}
		\item $\forall x,y\in\Kbb\qquad 1\not\asymp x\succ y\Rightarrow D(x)\succ D(y)$
		\item $\forall x,y\in\Kbb\qquad 1\not\asymp x\sim y\Rightarrow D(x)\sim D(y)$
		\item $\forall x,y\in\Kbb\qquad 1\not\asymp x\asymp y\Rightarrow D(x)\asymp D(y)$
	\end{itemize}
\end{proposition}

If $\Kbb\subseteq\Nobf$ is stable under $\exp$ and $\ln$, we can get a nice property satisfied by a general derivation.

\begin{proposition}[{\cite[Berarducci and Mantova, Proposition 6.5]{Berarducci_2018}}]
	Let $\Kbb\subseteq\Nobf$ be a field of surreal number stable by $\exp$ and $\ln$. Let $D$ be a derivation over $\Kbb$. For all $x,y>\Nbb$ such that $x-y>\Nbb$, 
	$$
	\ln D(x) - \ln D(y) \prec x-y \preceq \max(x,y)
	$$
\end{proposition}

To define the derivation, Berarducci and Mantova started by defining it on log-atomic numbers and then extending it on all surreal numbers. More precisely, a derivation on log-atomic number must satisfy the following:

\begin{definition}[{\cite[Berarducci and Mantova, Definition 9.1]{Berarducci_2018}} Prederivation]
	Let $\Kbb$ be a field of surreal numbers stale under $\exp$ and $\ln$ and  such that for all $x\in\Kbb$, for all path $P\in\Pcal(x)$, for all $k\in\Nbb$, if $P(k)\in\Lbb$, then $P(k)\in\Kbb$. A \textbf{prederivation} over $\Kbb$ is a function $D_\Lbb:\Lbb\cap\Kbb\to\Kbb$ such that
	\begin{itemize}
		\item[\textbf{D3.}] $\forall \lambda\in\Lbb\cap\Kbb\qquad 
		D_\Lbb\exp\lambda = (D_\Lbb\lambda)\exp\lambda$
		\item[\textbf{PD1.}] For all $\lambda\in\Lbb\cap\Kbb$, $D_\Lbb\lambda$ is a positive term.
		
		\item[\textbf{PD2.}] $\forall\lambda,\mu\in\Lbb\cap\Kbb\qquad 
		\ln D_\Lbb\lambda - \ln D_\Lbb\ln\mu \prec \max(\lambda,\mu)$
	\end{itemize}
\end{definition}

They key notion to define the derivation from the the prederivation is the notion of \textbf{path derivative}. This notion look at all the paths of the surreal number to say how it contributes to the derivative of the surreal number.

\begin{definition}[{\cite[Berarducci and Mantova, Definition 6.13]{Berarducci_2018}} Path derivative]
	\label{def:pathDeriv}\index{Path derivative!Definition  \ref{def:pathDeriv}}
	Let $P$ be a path. We define the \textbf{path derivative} $\partial P\in\Rbb\omega^\Nobf$ by
	\centre{$\partial P = \begin{accolade}
			P(0)\cdots P(k-1)D_\Lbb P(k) & P(k)\in\Lbb \\ 0 & \forall k\in\Nbb\quad P(k)\notin\Lbb
		\end{accolade} $}
	We denote $\Pcal_\Lbb(x)=\enstq{P\in\Pcal(x)}{\partial P\neq 0}$, which is the set of paths that indeed reach log-atomic numbers at some point.
\end{definition}

One can notice that for any $P\in\Pcal_\Lbb(x)$, $\partial P=r\omega^a$ for some $r\in\Rbb^*$ and $a\in\Nobf$. Indeed, every $P(k)$ is a term and $D_{\Lbb}P(k)$, when $P(k)\in\Lbb$ is an exponential of a purely infinite number, hence, it is a monomial.
For $P\in\Pcal_\Lbb(x)$ there is a minimum $k_P\in\Nbb$ such that $P(k_P)\in\Lbb$. Then $P$ is entirely determined by $P(0),\dots,P(k_P)$. 
We then define $\alpha_0(P),\dots,\alpha_{k_P}(P)$ as follows :
\begin{itemize}
	\item Writing $x=\Sumlt i{\nu(x)} r_i(x)\omega^{a_i(x)}$, then define $\alpha_0(P)<\nu(x)$ such that $P(0)=r_{\alpha_0(P)}(x)\omega^{a_{\alpha_0(P)}}$.
	
	\item For $0\leq i<k$, write $P(i)=r\omega^a$. Then $P(i+1)$ is a term of $\ln\omega^{a}$. Write $\ln\omega^a=\Sumlt i{\nu(a)}r_i(a)\omega^{h(a_i(a))}$. Then set $\alpha_{i+1}(P)$ such that $P(i+1)=r_{\alpha_{i+1}(P)}(a)\omega^{h(a_{\alpha_{i+1}(P)}(a))}$
\end{itemize}

Using Proposition \ref{prop:cheminMajorationNu}, we get that $\suite{\alpha_i(P)}i{\intn0{k_P}}$ is a finite sequence over ordinal less than $\NR(x)+1$. In particular, we can give $\Pcal_\Lbb(x)$ a lexicographic order inherited from the one over finite sequences. 

\begin{definition}
	We define the order $<_{lex}$ on paths by
	\centre{$P<_{lex}Q\Longleftrightarrow 
		(\alpha_0(P),\dots,\alpha_{k_P}(P))<_{lex} (\alpha_0(Q),\dots,\alpha_{k_Q}(Q))$}
\end{definition}

This order will be useful later when we will try to understand better what is going on to get some bounds about the derivatives. For now, the path-derivative being defined, we can recall a theorem by Berarducci and Mantova which explains how to build a general derivation from a prederivation.

\begin{lemma}[{\cite[Berarducci and Mantova, Corollary 6.17]{Berarducci_2018}}]
	\label{lem:partialPQprec}
	Let $P,Q\in\Pcal(x)$ such that $\partial P,\partial Q\neq0$. If there is $i\in\Nbb$ such that
	\begin{enumerate}
		\item $\forall j\leq i\qquad P(i)\preceq Q(i)$
		\item $P(i+1)$ is not a term of $\ell(Q(i))$, 
	\end{enumerate}
	\lc{then}{$\partial P\prec\partial Q$}
\end{lemma}

\begin{lemma}[{\cite[Berarducci and Mantova, Lemma 6.18]{Berarducci_2018}}]
	\label{lem:NRpath}
	Given $P\in\Pcal(x)$ a path of $x$ we have for all $i$ $\NR(P(i+1))\leq\NR(P(i))$ with equality if and only if $P(i)$ is the last term of $\ell(P(i))$. We also have $\NR(P(0)\leq\NR(x)$ with equality if and only if $P(0)$ is the last term of $x$.)
\end{lemma}

\begin{theorem}[{\cite[Berarducci and Mantova, Proposition 6.20, Theorem 6.32]{Berarducci_2018}}]\label{thm:prederivToDeriv}
	Let $D_\Lbb$ be a prederivation over a surreal field $\Kbb$ stable under $\exp$ and $\ln$. Then $D_\Lbb$ extends to a derivation $\partial:\Kbb\to\Nobf$ such that 
	$$
	\forall x\in\Kbb\qquad  \partial x = \Sumin P{\Pcal(x)}\partial P
	$$
	In particular, $\famille{\partial P}P{\Pcal(x)}$ is summable (see Definition \ref{def:summableFam}).
\end{theorem}

The study would not be complete without an example. Berarducci and Mantova provided such a derivation and even more: it is the simplest in some sense.

\begin{definition}[{\cite[Berarducci and Mantova, Definition 6.7]{Berarducci_2018}}]
	\label{def:partialLbb}
	We define $\partial_\Lbb:\Lbb\to\Nobf$ by
	$$	
	\forall\lambda\in\Lbb\qquad \partial_\Lbb\lambda = \exp\pa{-\Sum{\alpha\in\Ord|\kappa_{-\alpha}\succeq^K\lambda}{}\ \Sum{n=1}{\pinf}\ln_n\kappa_\alpha + \Sum{n=1}\pinf \ln_n\lambda}
	$$
\end{definition}

For example, we have:
	\begin{align*}
		\partial_\Lbb \omega &= 1 &\partial_\Lbb \exp\omega &= \exp\omega\\
		\partial_\Lbb \ln\omega &= \exp(-\ln\omega) = \f1\omega 
		&\partial_\Lbb\ln_n\omega &= \f1{\Prod{k=0}{n-1}\ln_k\omega}\\
		\partial_\Lbb\kappa_1=\partial_\Lbb\epsilon_0 &= \exp\pa{\Sum{n=1}\pinf\ln_n\kappa_1} &\partial_\Lbb\kappa_{-1} &= \exp\pa{-\Sum{n=1}\pinf \ln_n\omega}
	\end{align*}
In fact, $\kappa_1$ is intuitively $\exp_\omega\omega$. Therefore it is also quite intuitive that $\partial_\Lbb\kappa_1 = \kappa_1\ln(\kappa_1)\ln\ln(\kappa_1)\cdots$. The same happens for $\kappa_{-1}$ which is intuitively $\ln_\omega\omega$. We indeed have $\partial_\Lbb\kappa_{-1} = \f1{\omega\ln(\omega)\ln\ln(\omega)\cdots}$.

\begin{proposition}[{\cite[Berarducci and Mantova, Propositions 6.9 and 6.10]{Berarducci_2018}}]
	$\partial_\Lbb$ is a prederivation. 
\end{proposition}

The previous proposition ensures that the associated function $\partial$ defined by Theorem \ref{thm:prederivToDeriv} is indeed a derivation over surreal numbers. It turns out that it the simplest for the order $\sqsubseteq$.

We now explain what is meant when saying that $\partial$ is the simplest derivation. In fact, we mean that $\partial_\Lbb$ is the simplest prederivation with respect to the order $\sqsubseteq$.

\begin{theorem}[{Berarducci and Mantova\cite[Theorem 9.6]{Berarducci_2018}}]
	Let $D_\lambda$ be a prederivation. Let $\lambda\in\Lbb$, minimal (in $\Lbb$) for $\sqsubseteq$ such that $D_\Lbb\lambda \neq \partial_\Lbb\lambda$. Then $\partial_\Lbb\lambda\sqsubset D_\Lbb\lambda$. 
\end{theorem}

\subsection{A first bound about the derivative}
We give here some bound on the length of the series of a derivative.

%\begin{proposition}
%	\label{prop:majorationNuPartial}
%	For any $x\in\Nobf$, the set $\Pcal_\Lbb(x)$ is well-ordered with order type
%	$\beta<\omega^{\omega^{\omega(\NR(x)+1)}}$. In particular,
%	\centre{$\nu(\partial x)<\omega^{\omega^{\omega(\NR(x)+1)}}$}
%\end{proposition}
\propmajorationNuPartial*

\begin{proof}
	We know that $\famille{\partial P}P{\Pcal(x)}$ is summable (see Definition \ref{def:summableFam})\index{Summable family}. In particular $\famille{\partial P}P{\Pcal_\Lbb(x)}$ is summable. By definition of summability (in this context) for any $P\in\Pcal_\Lbb(x)$, there are finitely many $Q\in\Pcal_\Lbb(x)$ such that $\partial P\asymp \partial Q$. 
	
	By definition of summability, $<_\Pcal$ is a well total order over $\Pcal_\Lbb(x)$ and if $\beta$ is its order type, then $\omega\otimes\nu(\partial x)<\beta$ (usual ordinal product). Then, to complete the proof, we just need to show that $\beta<\omega^{\omega^{\omega(\NR(x)+1)}}$. We proceed by induction on $\NR(x)$.	
	\begin{itemize}
		\item \underline{$\NR(x)$=0} : then $x=0$ or $x=\pm y^{\pm 1}$ for some $y\in\Lbb$ and $\nu\pa{\partial x} \leq 1 <\omega^{\omega^\omega}$ and we conclude the proof.
		
		\item Assume that for any $y$ such that $\NR(y)<\NR(x)$, $\Pcal_\Lbb(y)$ has order type less than $\omega^{\omega^{\omega(\NR(y)+1)}}$. Assume for contradiction that $\beta\geq\omega^{\omega^{\omega(\NR(x)+1)}}$. Then for any multiplicative ordinal $\mu<\omega^{\omega^{\omega(\NR(x)+1)}}$, there is some $P_\mu\in\Pcal_\Lbb(x)$, minimum with respect to $<_{lex}$, such that the set
		\centre{$\Ecal_\mu(x)=\enstq{Q\in\Pcal_\Lbb(x)}{Q<_\Pcal P_\mu}$}
		has order type $\beta_\mu\geq\mu$. Let us select any $\mu$ such that $\mu\geq\omega^{\omega^{\omega\NR(x)+1}}$. Now define
		\begin{calculs}
			& \Ecal_\mu^{(1)}(x) &=& \enstq{Q\in\Pcal_\Lbb(x)}{Q<_\Pcal P_\mu\quad Q<_{lex} P_\mu}\\ [.4cm]
			& \Ecal_\mu^{(2)} &=& \enstq{Q\in\Pcal_\Lbb(x)}{Q >_{lex} P_\mu}
		\end{calculs}
		\lc{Theses sets are disjoints and}{$\Ecal_\mu=\Ecal_\mu^{(1)}\cup\Ecal_\mu^{(2)}$}
		Let $\beta_\mu^{(i)}$ be the order type of $\Ecal_\mu^{(i)}$. We then have
		\centre{$\mu\leq\beta_\mu\leq\beta_\mu^{(1)}+\beta_\mu^{(2)}$}
		where the addition is the surreal addition of ordinal numbers. Since $\mu$ is multiplicative ordinal, hence, an additive one, at least one of the $\beta_\mu^{(i)}\geq\mu$.
		\begin{itemize}
			\item \underline{First case }: $\beta_\mu^{(2)}\geq\mu$. Since $\mu$ is additive, there is an $i\in\intn0{k_P}$ such that the well ordered set
			\centre{$\Ecal_\mu^{(2,i)}=\enstq{Q\in\Ecal_\mu^{(2)}}{\forall j<i\ Q(j)=P_\mu(j)\quad Q(i)\prec P_\mu(i)}$}
			has order type at least $\mu$. We take such an $i$. For $Q\in\Ecal_\mu^{(2,i)}$, we consider the path $Q'(n)=Q(n+i+1)$. Since $\partial Q\succeq \partial P_\mu$, Lemma \ref{lem:partialPQprec} gives us that $Q(i+1)$ is a term of $\ell(P_\mu(i))$.  We then have $Q'\in\Pcal\pa{\ell(P_\mu(i))}$ and 
			\centre{$\partial Q' = \f{\partial Q}{Q(0)\cdots Q(i)}=\f{\partial Q}{P_\mu(0)\cdots P_\mu(i-1)Q(i)}$}
			%\lc{Similarely}{$\partial P_\mu' = \f{\partial P_\mu}{P_\mu(0)\cdots P_\mu(i)}$}
			%\lc{and}{$P_\mu(0)\cdots P_\mu(i)\partial Q' \succ Q(0)\cdots Q(i)\partial Q' = \partial Q\sqsubset \partial P_\mu = P_\mu(0)\cdots P_\mu(i)\partial P_\mu'$}
			In particular $Q'\in\Pcal_\Lbb\pa{\ell(P_\mu(i))}$. 
			%Hence, if $\partial Q'\preceq \partial P_\mu'$, we get that $\partial P_\mu'\succ\partial P_\mu'$ which is a contradiction. Then $\partial Q'\succ \partial P_\mu'$, in particular, $Q'<_\Pcal P_\mu'$. 
			Since $Q(i)\prec P_\mu(i)$, $P_\mu(i)$ is not the last term of $\ell(P_\mu(i-1))$ (or $x$ if $i=0$). Then Proposition \ref{prop:NRComparaisonTerme} ensures that 
			\centre{$\NR(\ell(P_\mu(i)))\leq \NR(P_\mu(i))<\NR(x)$}
			Applying the induction hypothesis on $\ell(P_\mu(i))$, the order type of $\Pcal_\Lbb(\pa{\ell(P_\mu(i)))}$ has order type $\gamma$ such that
			\centre{$\gamma<\omega^{\omega^{\omega\pa{\NR(\ell(P(i)))+1}}}\leq\omega^{\omega^{\omega\NR(x)}}<\omega^{\omega^{\omega\NR(x)+1}}\leq\mu$}
			For $Q,R\in\Ecal_\mu^{(2,i)}$, $Q<_\Pcal R$ \tiff 
			\begin{calculs}
				& (Q(i)\partial Q' \succ R(i)\partial R') &\vee& (Q(i)\partial Q'\asymp R(i)\partial R' \wedge Q(i)\partial Q' > R(i)\partial R')\\
				&&& \vee (Q(i)\partial Q' = R(i)\partial R\wedge Q<_{lex}R)
			\end{calculs}
			what we can also write
			\begin{calculs}
				& Q<_\Pcal R &\Leftrightarrow&
				\big(\ell(Q(i)) + \ell(\partial Q') > \ell(R(i))+\ell(\partial R')\big)\\
				&&&\vee \big(\ell(Q(i))+\ell(\partial Q')=\ell(R(i))+\ell(\partial R') \wedge Q(i)\partial Q'>R(i)\partial R'\big)\\
				&&&\vee (Q(i)\partial Q' = R(i)\partial R\wedge Q<_{lex}R) 
			\end{calculs}
			where the two later cases occur finitely may times for $Q$ or $R$ fixed. 
			Let $\delta$ denote the order type of the possible values for $Q(i)$ and $\beta_\mu^{(2,i)}$ the order type of $\Ecal_\mu^{(2,i)}$. Since $\ell$ is non-decreasing, the set $\enstq{\ell(\partial Q')}{Q\in\Ecal_\mu^{(2,i)}}$ has order type at most $\gamma$ and the set $\enstq{\ell(Q(i))}{Q\in\Ecal_\mu^{(2,i)}}$ has order type at most $\NR(x)$. Using Proposition \ref{prop:sommeEnsBienOrd},
			
			
			\centre{$\beta_\mu^{(2,i)}\leq (\gamma\NR(x))\otimes\omega<\mu$}
			\lc{Finally}{$\mu\leq\beta_\mu^{(2,i)}<\mu$}
			and we reach the contradiction.
			
			\item \underline{Second case }: $\beta_\mu^{(2)}<\mu$. Then $\beta_\mu^{(1)}\geq\mu$.  Let us define for $i\in\intn0{k_P}$ 
			\centre{$\Ecal_\mu^{(1,i)}=\enstq{Q\in\Ecal_\mu^{(1)}}{\forall j<i\ P_\mu(j)=Q(j)\quad P_\mu(i)\prec Q(i)}$}
			Since there are finitely many of them, that they form a partition of $\Ecal_\mu^{(1)}$ and $\mu$ is multiplicative, hence additive, there is at least one of them which has order type at least $\mu$. We consider such an $i\in\intn0{k_P}$. Now define
			\centre{$x_j = \begin{accolade}
					x & i=j \\ \ell(P(i-j-1)) & j<i 
				\end{accolade}$} 
			Writing $x_0=\Sumlt n{\nu(x_0)} r_n(x_0)\omega^{a_n(x_0)}$ and $P_\mu(i)=r_{\alpha_0}(x_0)\omega^{a_{\alpha_0}(x_0)}$ we set
			\centre{$y_0 = \Sumlt n{\alpha_0} r_n(x_0)\omega^{a_n(x_0)}$}
			Now for $0\leq j<i$, we define $y_{j+1}$ has follows. $P_\mu(i-j-1)$ is a term of $x_{j+1}$. Write $P_\mu(i-j-1)=r_{\alpha_{j+1}}(x_{j+1})\omega^{a_{\alpha_{j+1}}(x_{j+1})}$ for some ${\alpha_{j+1}}<\nu(x_{j+1})$. Then set
			\centre{$y_{j+1} = \Sumlt n{\alpha_{j+1}} r_n(x_{j+1})\omega^{a_n(x_{j+1})} + \sign(r_{\alpha_{j+1}}(x_{j+1})) \exp(y_j)$}
			Denote $y=y_i$. For $Q\in\Ecal_\mu^{(1,i)}$. For any $Q\in\Ecal_\mu^{(1,i)}$ we will build $Q'\in\Pcal_\Lbb(y)$. We expect to use the induction hypothesis on $y$. First we prove that $\NR(y)<\NR(x)$. In fact, by trivial induction, we have $y_j\nestedn[j] x_j$. So $y\nestedn[i] x$ and by definition of $\NR$ we have $\NR(y)<\NR(x)$. Now consider the path $Q'$ defined as follows :
			\begin{itemize} 
				\item $\forall j<i\qquad  Q'(j)=\sign(r_{\alpha_j}(x_{i-j}))\exp(y_{i-j-1})$
				\item $\forall j\geq i\qquad Q'(j)=Q(j)$
			\end{itemize}
			
			We then have $Q'\in\Pcal(y)$. We can even say $Q'\in\Pcal_\Lbb(y)$. Moreover, since we change only the common terms of the path, and the changes do not depend on $Q$, we have
			\centre{$\forall Q,R\in\Ecal_\mu^{(1,i)}\qquad Q<_\Pcal R\Leftrightarrow Q'<_\Pcal R'$}
			We then have an increasing function
			\centre{$\fct{\Phi}{\Ecal_\mu^{(1,i)}}{\Pcal_\Lbb(y)}{Q}{Q'}$}
			The induction hypothesis give that the order type of $P'(y)$ is less than $\omega^{\omega^{\omega(\NR(y)+1)}}$. Then
			\centre{$\omega^{\omega^{\omega\NR(x)}}\leq \mu < \omega^{\omega^{\omega(\NR(y)+1)}}\leq\omega^{\omega^{\omega\NR(x)}}$}
			and we get the contradiction.
		\end{itemize}
		This completes the proof.
	\end{itemize}
\end{proof}

\begin{corollary}
	\label{cor:majorationNuPartial}
	If $\NR(x)<\lambda$ then $\nu(\partial x)<\lambda$
\end{corollary} 

\begin{proposition}
	\label{prop:majorationNRPartial}
	For all $x\in\Nobf$, let $\alpha$ the minimum ordinal such that $\kappa_{-\alpha}\prec^K t$ for all log-atomic $t$ such that there is some path $P\in\Pcal_\Lbb(x)$ and some index $k\in\Nbb$ such that $P(k)=t$. Then, for all path $P$, 
	\centre{$\NR(\partial P)\leq  k(\NR(x)+1)+\omega(\alpha+1) $}
	\lc{and}{$\NR(\partial x)\leq \omega(\NR(x)+\alpha+2)\otimes \nu(\partial x) \leq\omega^{\omega^{\omega(\NR(x)+1)}+\alpha}$} 
\end{proposition}

\begin{proof}
	Let $P$ be a path of such that $\partial P\neq 0$. Then there is some $k\in\Nbb$ such that $\partial P=P(0)\cdots P(k-1)\partial_\Lbb P(k)$. With Corollary \ref{cor:NRprod}, we get
	\begin{calculs}
		& \NR(\partial P) &\leq& \Sum{i=0}{k-1}\NR(P(i)) + \NR(\partial_\Lbb P(k))+k\\
		&&\leq& k\NR(x)+\NR(\partial_\Lbb P(k))+k&(Lemma \ref{lem:NRpath})\\
		&&\leq& k\NR(x)+k+\NR\pa{\exp\pa{-\Sum{\kappa_{-\beta}\succeq^K P(k)}{}\Sum{n\geq 1}{} \ln_n\kappa_{-\beta} + \Sum{n\geq 1}{}\ln_nP(k)}}\\
		&&\leq& k\NR(x)+k+\NR\pa{-\Sum{\kappa_{-\beta}\succeq^K P(k)}{}\Sum{n\geq 1}{} \ln_n\kappa_{-\beta} + \Sum{n\geq 1}{}\ln_nP(k)} & \\&&&&(Proposition \ref{prop:NRStableExp})\\
		&&\leq& k\NR(x)+k+(\omega\otimes(\alpha\oplus 1)) &(Lemma \ref{lem:NRSommeLogAtomiques})\\
		&&\leq& \omega(\NR(x)+1)+\omega(\alpha+1)
	\end{calculs}
	This bound does not depend on $P$. Then applying Proposition \ref{prop:majorationNuPartial} and Lemma \ref{lem:NRAjoutNouveauTerme} we get
	\begin{calculs}
		&\NR(\partial x) &\leq& (\omega(\NR(x)+\alpha+2))\otimes \nu(\partial x)\\
		&&<& (\omega(\NR(x)+\alpha+2)) \otimes \omega^{\omega^{\omega(\NR(x)+1)}}\\
		&&\leq&\omega^{\omega^{\omega(\NR(x)+1)}+\alpha}
	\end{calculs}
\end{proof}



\subsection{Anti-derivative of a surreal number}

Berarducci and Mantova provided a derivation, $\partial$. An other strong property of this derivation is that is as a compositional inverse, an anti-derivation. The first thing to prove it is to prove that there is an asymptotic anti derivation.

\begin{proposition}[{\cite[Berarducci and Mantova, Proposition 7.4]{Berarducci_2018}}]
	There is a class function $A : \Nobf^*\to \Rbb\omega^{\Nobf^*}$ such that 
	\centre{$\forall x\in\Nobf^*\qquad x\sim\partial A(x)$}
\end{proposition}

Basically, $A(x)$ is the leading term of $x\f{xu/\partial u}{\partial(xu/\partial u)}$ where $u=\kappa_\alpha$ for $\alpha$ a sufficiently large ordinal. Actually we can be even more precise and give a more explicit formula for Berarducci and Mantova's asymptotic anti-derivation \cite{Berarducci_2018}.

\begin{lemma}
	\label{lem:antiDerivee1}
	Let $u=\ln_n\kappa_{-\alpha}$ for some $n\in\Nbb$ and some ordinal $\alpha$. Let $x=\partial u\exp\epsilon$. If $\epsilon\succ \ln u$, then
	\centre{$\partial\pa{\f x{\partial\epsilon}}\sim x$}
\end{lemma}

\begin{proof}
	Let $y=\f x{\partial\epsilon} = \f{\partial u}{\partial\epsilon}\exp(\epsilon)$. Since $\epsilon\succ\ln u$, Proposition \ref{prop:compareDerivative} ensures that $\partial\epsilon \succ \f{\partial u} u$.  Then, $\f{\partial u}{\partial\epsilon}\prec u\not\asymp 1 $
	\begin{calculs}
		& \partial y &=& \f{\partial u}{\partial\epsilon}\partial\epsilon\exp(\epsilon) + \partial\pa{\f{\partial u}{\partial\epsilon}}\exp(\epsilon) \\
		
		&&=& x + \partial\pa{\f{\partial u}{\partial\epsilon}}\exp(\epsilon)\\ 
	\end{calculs}
	
	Proposition \ref{prop:compareDerivative} gives that $\partial\pa{\f{\partial u}{\partial\epsilon}}\prec\partial u$. Then
	\centre{$\partial y\sim x$}
\end{proof}

\begin{lemma}
	\label{lem:antiDerivee2}
	Let $u=\ln_n\kappa_{-\alpha}$ for some $n\in\Nbb$ and some ordinal $\alpha$. Let $x=\partial u\exp(\epsilon)$. If $\epsilon\sim r\ln u$ for some $r\in\Rbb\setminus\{0,-1\}$, then
	\centre{$\partial\pa{\f1{r+1}\f {ux}{\partial u}}\sim x$}
\end{lemma}

\begin{proof}
	Let us compute the above derivative.
	\centre{$\partial\pa{\f1{r+1}\f {ux}{\partial u}} =
		\partial\left(\f{u\exp(\epsilon)}{r+1}\right) = 
		\f x{r+1} + \f{u\partial\epsilon\exp(\epsilon)}{r+1}$}
	Using Proposition \ref{prop:compareDerivative}, we get that $\partial\epsilon \sim\partial(r\ln u) =  r\f{\partial u}u$. Then, since $r\neq -1$, we get that
	\centre{$\partial\pa{\f1{r+1}\f {ux}{\partial u}}\sim x$}
\end{proof}

\begin{lemma}
	\label{lem:antiDerivee3}
	Let $u=\ln_n\kappa_{-\alpha}$ for some $n\in\Nbb$ and some ordinal $\alpha$. Let $x=\partial u\exp(\epsilon)$. If $\epsilon\prec \ln u$, then
	\centre{$\partial\pa{\f {ux}{\partial u}}\sim x$}
\end{lemma}

\begin{proof}
	Let us compute the above derivative.
	\centre{$\partial\pa{\f {ux}{\partial u}} =
		\partial\left(u\exp(\epsilon)\right) = 
		x + u\partial\epsilon\exp(\epsilon)$}
	Using Proposition \ref{prop:compareDerivative}, we get that $\partial\epsilon\prec\partial\ln u =  \f{\partial u}u$. Then, $u\partial\epsilon\exp(\epsilon)\prec x$ and we get that
	\centre{$\partial\pa{\f {ux}{\partial u}}\sim x$}
\end{proof}

\begin{theorem}
	\label{thm:formeAntiDeriveeAsymp}
	Let $x$ be a term. Write $|x|=\partial u\exp(\epsilon)$ with $u=\ln_n\kappa_{-\alpha}=\lambda_{-\omega\alpha-n}$ with $\omega\alpha+n$ such minimal that $\epsilon\not\sim-\ln u$. Then,
	\centre{$A(x)\sim\begin{accolade}
			\f x{\partial\epsilon} & \epsilon \succ\ln u\\
			\f{ux}{(r+1)\partial u} & \epsilon\sim r\ln u\quad r\neq 0,-1\\
			\f{ux}{\partial u} & \epsilon\prec\ln u
		\end{accolade}$}
\end{theorem}

In this theorem, the quantities $\kappa_a$ and $\lambda_a$ are defined in Definitions \ref{def:kappaMap} and \ref{def:lambdaMap}. 

\begin{proof}
	Since $A(x)=-A(-x)$, we may assume that $x>0$. Then, we just need to apply Lemmas \ref{lem:antiDerivee1}, \ref{lem:antiDerivee2}, and \ref{lem:antiDerivee3}.
\end{proof}

\begin{corollary}
	\label{cor:formeAntiDeriveeAsymp}
	Let $x$ be a non-zero surreal number. Write $|x|=\partial u\exp(\epsilon)$ with $u=\ln_n\kappa_{-\alpha}=\lambda_{-\omega\alpha-n}$ with $\omega\alpha+n$ such minimal that $\epsilon\not\sim-\ln u$. Then,
	\centre{$A(x)= \begin{accolade}
			\f{t}{s} & \epsilon \succ\ln u\\ [.4cm]
			\f{ut}{(r+1)\partial u} & \epsilon=r\ln u + \eta \quad r\neq -1, \eta\prec\ln u
		\end{accolade}$}
	\vskip .2cm
	where $t$ is the leading term of $x$ and $s$ the leading term of $\partial\epsilon$.
\end{corollary}

\begin{proof}
	Just use Theorem \ref{thm:formeAntiDeriveeAsymp} and the definition of $A$.
\end{proof}

We are now ready to build the anti-derivation for surreal numbers.
We start with a useful lemma due to Aschenbrenner, van den Dries and van der Hoeven. We give it in a form that matches our notations.

\begin{definition}\label{def:strongLin}\index{Strongly linear function!Definition \ref{def:strongLin}}
	A function $\Phi$ is strongly linear is for all summable family $\famille{x_i}iI$, 
	$$
	\Phi\pa{\Sumin iI x_i} = \Sumin iI \Phi(x_i)
	$$
\end{definition}

\begin{lemma}[{\cite[Aschenbrenner, van den Dries, van der Hoeven, Corollary 1.4]{vdH:dagap}}]
	\label{lem:inversePhi}
	Let $\Phi$ a strongly linear map defined over a field $\Kbb$ of surreal numbers.  Assume that for any monomial
	$\omega^a\in\Kbb$, we have $\Phi(\omega^a)\prec\omega^a$. Then $\Sumin n\Nbb\Phi^n(x)$ makes sense as a surreal number (\ie{} $\famille{\Phi^n(x)}n\Nbb$ is summable\index{Summable family}) and if it belongs to $\Kbb$ for all $x$, we have
	\centers{$(\id-\Phi)^{-1}=\Sumin n\Nbb\Phi^n$}
\end{lemma}

\begin{definition}
	\label{def:phi}
	We define an extension of $A$, denoted $\Acal$, to all surreal numbers by
	
	\centre{$\Acal\pa{\aSurreal} = \Sumlt i\nu r_i A(\omega^{a_i})$}
	
	We also introduce the function $\Phi=\id-\partial\circ\Acal$.
\end{definition}

Proposition \ref{prop:compareDerivative} ensures that the function $\Acal$ is well defined. Moreover, this function is obviously strongly linear\index{Strongly linear function}. We now consider, given a surreal number $x$, the sequence
\centre{$\begin{accolade}
		&x_0 = x\\
		&x_{n+1} = x_n-\partial\Acal(x_n) = \Phi(x_n)
	\end{accolade}$}

Note that if $\omega^a=\partial u\exp\epsilon$ with $u=\lambda_{-\omega\otimes\alpha-n} =\ln_n\kappa_{-\alpha}$ and $\epsilon\prec \ln \lambda_{-\omega\otimes\beta-m}$ for $\omega\otimes\beta+m<\omega\otimes\alpha+n$, and $\omega\otimes\alpha+n$
maximum for that property, we have
\centre{$\Phi(\omega^a) = \begin{accolade}
		\pa{1-\f{\partial\epsilon}s}\omega^a - \partial\pa{\f{\partial u}s}\exp\epsilon 
		& \epsilon\succ\ln u\quad s\text{ dominant term of }\partial\epsilon\\
		\f{\omega^a}{r+1}\partial\eta\f u{\partial u} & \epsilon=r\ln u+\eta\quad r\neq -1
	\end{accolade}$}

\begin{corollary}
	\label{cor:existencePrimitive}
	The operator $\id-\Phi$ is invertible with inverse $\Sum{i\in\Nbb}{}\Phi^{i}$. Moreover $\Acal\circ\Sum{i\in\Nbb}{}\Phi^i$ is an operator that sends every $x$ to some anti-derivative of $x$.
\end{corollary}

\begin{proof}
	Lemma \ref{lem:inversePhi} ensure that $\id-\Phi$ has a inverse expressed by $\Sumin i\Nbb\Phi^i$. We also have that $\id-\Phi=\partial\circ\Acal$. Then,
	\centre{$\partial\circ\pa{\Acal\circ\Sumin i\Nbb\Phi^i} = (\partial\circ\Acal)\circ\pa{\partial\circ\Acal}^{-1}=\id$}
	In particular, for all $x$, $\pa{\Acal\circ\Sumin i\Nbb\Phi^i}(x)$ is a anti-derivative of $x$.
\end{proof}