% This must be in the first 5 lines to tell arXiv to use pdfLaTeX, which is strongly recommended.
\pdfoutput=1
% In particular, the hyperref package requires pdfLaTeX in order to break URLs across lines.

\documentclass[11pt]{article}

% Remove the "review" option to generate the final version.
\usepackage[]{EMNLP2022}

% Standard package includes
\usepackage{times}
\usepackage{latexsym}
\usepackage{graphicx}
\usepackage{algorithm}
\usepackage{algpseudocode}
% For proper rendering and hyphenation of words containing Latin characters (including in bib files)
\usepackage[T1]{fontenc}
% For Vietnamese characters
% \usepackage[T5]{fontenc}
% See https://www.latex-project.org/help/documentation/encguide.pdf for other character sets

% This assumes your files are encoded as UTF8
\usepackage[utf8]{inputenc}

% This is not strictly necessary, and may be commented out.
% However, it will improve the layout of the manuscript,
% and will typically save some space.
\usepackage{microtype}

% This is also not strictly necessary, and may be commented out.
% However, it will improve the aesthetics of text in
% the typewriter font.
\usepackage{inconsolata}

\usepackage{multirow}
\usepackage{float}
\usepackage[subtle]{savetrees}

%Path relative to the .tex file containing the \includegraphics command
\graphicspath{ {images/} }

% If the title and author information does not fit in the area allocated, uncomment the following
%
%\setlength\titlebox{<dim>}
%
% and set <dim> to something 5cm or larger.

\title{An Efficient Active Learning Pipeline for Legal Text Classification}

% Author information can be set in various styles:
% For several authors from the same institution:
\author{Sepideh Mamooler \and Rémi Lebret \and Stephane Massonnet \and Karl Aberer  \\
        School of Computer and Communication Sciences, EPFL, Switzerland}
% if the names do not fit well on one line use
%         Author 1 \\ {\bf Author 2} \\ ... \\ {\bf Author n} \\
% For authors from different institutions:
% \author{Author 1 \\ Address line \\  ... \\ Address line
%         \And  ... \And
%         Author n \\ Address line \\ ... \\ Address line}
% To start a seperate ``row'' of authors use \AND, as in
% \author{Author 1 \\ Address line \\  ... \\ Address line
%         \AND
%         Author 2 \\ Address line \\ ... \\ Address line \And
%         Author 3 \\ Address line \\ ... \\ Address line}

% \author{First Author \\
%   Affiliation / Address line 1 \\
%   Affiliation / Address line 2 \\
%   Affiliation / Address line 3 \\
%   \texttt{email@domain} \\\And
%   Second Author \\
%   Affiliation / Address line 1 \\
%   Affiliation / Address line 2 \\
%   Affiliation / Address line 3 \\
%   \texttt{email@domain} \\}

\begin{document}
\maketitle
\begin{abstract}
Active Learning (AL) is a powerful tool for learning with less labeled data, in particular, for specialized domains, like legal documents, where unlabeled data is abundant, but the annotation requires domain expertise and is thus expensive. Recent works have shown the effectiveness of AL strategies for pre-trained language models. However, most AL strategies require a set of labeled samples to start with, which is expensive to acquire. In addition, pre-trained language models have been shown unstable during fine-tuning with small datasets, and their embeddings are not semantically meaningful. In this work, we propose a pipeline for effectively using active learning with pre-trained language models in the legal domain. To this end, we leverage the available \textit{unlabeled} data in three phases. First, we continue pre-training the model to adapt it to the downstream task. Second, we use knowledge distillation to guide the model's embeddings to a semantically meaningful space. Finally, we propose a simple, yet effective, strategy to find the initial set of labeled samples with fewer actions compared to existing methods. Our experiments on Contract-NLI, adapted to the classification task, and LEDGAR benchmarks show that our approach outperforms standard AL strategies, and is more efficient. Furthermore, our pipeline reaches comparable results to the fully-supervised approach with a small performance gap, and dramatically reduced annotation cost. Code and the adapted data will be made available.
\end{abstract}
  
\section{Introduction}

With the advent of pre-trained transformer-based language models (\citealp{Devlin2019BERTPO,Liu2019RoBERTaAR,He2021DeBERTaDB}), training models from scratch has been outperformed by fine-tuning pre-trained language models for several tasks in natural language processing, including text classification \citep{Howard2018UniversalLM}. However, fine-tuning these models still needs large labeled datasets to perform well on the downstream task \citep{Dodge2020FineTuningPL,Zhang2021RevisitingFB,Mosbach2021OnTS}. Collecting a large annotated dataset is a highly expensive and time-consuming process in specialized domains, where annotation can only be performed by the domain experts, such as the legal domain \citep{Hendrycks2021CUADAE}.

Active Learning (AL) has been proved effective for data-efficient fine-tuning of pre-trained language models in non-specialized domains like news, emotions, and movies \citep{EinDor2020ActiveLF,Margatina2022OnTI}. In addition, \citet{Margatina2022OnTI} have shown that the unlabeled data can be used to adapt the pre-trained language model to the downstream task, thereby improving the active learning performance with no extra annotation cost. On the specialized domains, \citet{Chhatwal2017EmpiricalEO} have evaluated multiple AL strategies in the legal domain before the emergence of pre-trained language models. Nevertheless, to the best of our knowledge, the effectiveness of active learning in fine-tuning pre-trained language models in the legal domain has been poorly studied. 

In this work, we focus on efficient legal text classification with RoBERTa \citep{Liu2019RoBERTaAR} by leveraging existing AL strategies. We identify two challenges in deploying AL strategies in the legal domain; First, legal texts contain a specialized vocabulary that is not common in other domains, including the ones on which pre-trained language models are trained. Second, the annotation of legal texts is highly expensive and time-consuming due to the necessity of specialized training for understanding these texts. For example, \citet{Hendrycks2021CUADAE} reported a cost of over $\$2$ million for the annotation of the Contract Understanding Atticus Dataset (CUAD) consisting of around $500$ contracts. 

To account for the specialized vocabulary, inspired by \citeposs{Margatina2022OnTI} work, we leverage the available \textit{unlabeled} data to adapt the pre-trained language model to the downstream task. In addition, considering the limitations of pre-trained language models like BERT and RoBERTa in capturing semantics \citep{Reimers2019SentenceBERTSE}, we use knowledge distillation to further improve the task-adapted model by mapping its embedding space to a semantically meaningful space. Our experiments demonstrate that AL strategies can benefit from semantically meaningful embeddings. 


Concerning the cost and time constraints, we focus on the fact that many AL strategies \citep{Lewis1994ASA,Gal2016DropoutAA,Gissin2019DiscriminativeAL} require an annotated set of $N$ positive and negative samples to start with. In practice, acquiring this set is expensive for large and skewed datasets. We propose a strategy to make the first iteration more efficient by clustering the unlabeled samples and limiting the pull of candidates to the cluster medoids. Our experiments demonstrate we can achieve comparable results with the standard initial sampling approach with up to $63\%$, and $25\%$ fewer actions on the skewed Contract-NLI \citep{Koreeda2021ContractNLIAD}, and balanced LEDGAR benchmarks \citep{Tuggener2020LEDGARAL} respectively. 

Our contributions can be summarized as follows:

\begin{itemize}
    \item[1.] We design an efficient and effective active learning pipeline for legal text classification by leveraging the available unlabeled data using task-adaptation and knowledge distillation, which obtains comparable performance to fully-supervised fine-tuning with considerably reduced annotation effort.
    \item[2.] We propose a strategy to reduce the number of actions in the first iteration of active learning by clustering the unlabeled data, and collecting the samples from cluster medoids, further increasing the efficiency of our approach.
    \item[3.] We evaluate our approach over Contract-NLI and LEDGAR benchmarks. Our results illustrate an increase of $0.3346$, and $0.1658$ in the best obtained F1-score, compared to standard active learning strategies, for Contract-NLI and LEDGAR respectively.
    
\end{itemize}
 

\section{Related Work}

\paragraph{Active learning with pre-trained language models}
Multiple works have studied active learning for pre-trained language models like BERT. \citet{EinDor2020ActiveLF} have evaluated various AL strategies for fine-tuning BERT for text classification, and showed that AL can boost BERT's performance especially for skewed datasets. However, they do not leverage the available unlabeled data to adapt the pre-trained language model to the task at hand, and only focus on non-specialized domains like news and sentiment analysis that do not require experts' knowledge.

\citet{Gururangan2020DontSP} have shown that task-adaptive pre-training using the available unlabeled data leads to performance gain when using pre-trained language models like BERT. Following this observation, \citet{Margatina2022OnTI} demonstrated the importance of task-adaptation for active learning for non-specialized texts like news, movies and sentiment analysis. 

Inspired by these works, we leverage the available unlabeled data to effectively adapt RoBERTa to legal text classification, where the annotation demands experts' knowledge. In addition, we propose an additional step to map the embedding space of the task-adapted RoBERTa to a semantically meaningful space using sentence transformers. 

\paragraph{Sentence transformers} 
\citet{Reimers2019SentenceBERTSE} have shown that the embedding space of off-the-shelf pre-trained language models like BERT \citep{Devlin2019BERTPO} and RoBERTa \citep{Liu2019RoBERTaAR} is not semantically meaningful, and thus, is not suitable for common sentence comparison measures like cosine similarity. To overcome this limitation, they propose sentence transformers, obtained by adding a pooling layer on top of pre-trained language models, and fine-tuning them in a Siamese network architecture with pairs of similar sentences. In this work, we use a RoBERTa-based sentence transformer as a teacher model and distill its knowledge to the task-adapted RoBERTa to produce sentence embeddings that capture the semantics and can be compared using cosine similarity. 

\paragraph{Active learning strategies}
Numerous methods have been proposed to find proper labeling candidates for active learning. Majority of them belong to one or both of two categories: diversity-sampling, and uncertainty-sampling. Diversity-based methods (\citealp{Sener2018ActiveLF,Gissin2019DiscriminativeAL,Wang2017IncorporatingDA}) aim to find labeling candidates that best represent the dataset, whereas uncertainty-based methods (\citealp{Gal2016DropoutAA,Kirsch2019BatchBALDEA,Zhang2021CartographyAL}) target candidates about which the model is uncertain. 
BADGE \citep{Ash2020DeepBA} is a cluster-based AL strategy that belongs to both of these categories. It transforms data into gradient embeddings that encode model confidence and sentence feature at the same time. 
By applying kmeans++ on the gradient embeddings it can find samples that differ both in terms of semantics and predictive uncertainty. ALPS \citep{Yuan2020ColdstartAL} is another cluster-based AL strategy that leverages both uncertainty and diversity using the surprisal embeddings obtained by passing the sentences to the MLM head of the pre-trained language model, and computing the cross entropy loss for a random set of tokens against the target labels. 

Existing AL strategies often require a set of labeled samples to start with, which is expensive to acquire. To overcome this high cost, we propose a clustering-based strategy to reduce the effort required to create the initial set of annotated samples.

\section{Notation and Setting}
In this section, we explain the structure shared between all AL strategies used in this work and fix the notation.

Active learning is an iterative process aiming to obtain a desired performance given an annotation budget. Here, we consider the annotation budget to be the number of actions performed by the annotator. In addition, we assume all annotators are legal experts, and that each annotator assigns perfect labels to text segments. Let $U_0$ and $L_0$ be the starting pool of unlabeled and labeled samples respectively. Initially, $L_0 = \emptyset$. At the first iteration, the annotator labels $N$ sample, $P$ positive and $N-P$ negative, to obtained $L_1$. Then, at each iteration $i$, the model is fine-tuned using $L_i$, and the AL strategy recommends a set of samples $C_i$ for annotation. These samples are labeled and $U_i$ and $L_i$ are updated as $U_{i+1} = U_i \setminus C_i$, and $L_{i+1} = L_i \cup C_i $. The procedure is repeated until the annotation budget is exhausted, or the desired performance is achieved. 


We base our work on the Low-Resource Text Classification Framework introduced by \citet{EinDor2020ActiveLF}. 
Following this work, we focus on binary text classification, given a small annotation budget and a potentially imbalanced dataset. This scenario matches common use cases in the legal domain, where the goal is to find phrases that correspond to a specific category, with the lowest possible number of actions, given a pool of unlabeled, imbalanced data. We perform $5$ AL iterations, and assume a more restricted annotation budget compared to \citet{EinDor2020ActiveLF}, allowing only $10$ annotations per iteration. For the first AL iteration, we assume that $5$ positive and $5$ negative samples need to be annotated.


\section{Methodology}

\begin{algorithm*}[ht]
\caption{AL pipeline for text classification}\label{alg:cap}
\hspace*{\algorithmicindent} \textbf{Input:} {unlabeled samples $U_0$, PT RoBERTa, PT Sentence-RoBERTa, AL strategy $\alpha$, \# iterations $T$}
\hspace*{\algorithmicindent} \textbf{Output:} {text classifier CLS RoBERTa, acquired labeled dataset $L_T$}
\begin{algorithmic}
\State $L_0 \gets \emptyset$
\State \textbf{\textit{Phase 1: Task-adaptation with Masked Language Modeling (MLM)}} \State TAPT RoBERTa $\gets$ MLM(PT RoBERTa, $U_0$) 
\State \textbf{\textit{Phase 2: Knowledge distillation}} 
\State DisTAPT RoBERTa $\gets$ Distill(TAPT RoBERTa, PT Sentence-RoBERTa, $U_0$)
\State \textbf{\textit{Phase 3: Initial sampling}} 
\State cluster medoids $\gets$ KMeans(DisTAPT RoBERTa, $U_0$)
\State $L_1 \gets$ Sample(cluster medoids)
\State $U_1 \gets U_0 \setminus L_1$
\State \textbf{\textit{Phase 4: Active learning}} 
\For{$i \gets 1$ to $T$ }
    \State CLS RoBERTa $\gets$ Train(DisTAPT RoBERTa, $L_i$)
    \State $C_i \gets \alpha$(CLS RoBERTa, $U_i$)
    \State $L_{i+1} \gets L_i \cup C_i$
    \State $U_{i+1} \gets U_i \setminus C_i$
\EndFor
\end{algorithmic}
\label{alg:algorithm}
\end{algorithm*}

We propose an efficient active learning pipeline for fine-tuning pre-trained language models for legal text classification. Our approach leverages available unlabeled data in three phases to adapt the pre-trained model to the downstream task (Sec.~\ref{sec:task-adaptation}), guide its embedding space to a semantically meaningful and comparable space (Sec.~\ref{sec:distillation}), and reduce the number of actions required to collect the initial labeled set (Sec.~\ref{sec:sampling}). Finally, it leverages existing AL strategies to efficiently fine-tune a classifier (Sec.~\ref{sec:meth_al}). We now explain each step in detail. An overview of this pipeline can be found in Algorithm~\ref{alg:algorithm}.




\subsection{Task-Adaptation}\label{sec:task-adaptation}

It has been shown that fine-tuning off-the-shelf pre-trained language models with standard approaches is unstable for small training sets \citep{Zhang2021RevisitingFB,Dodge2020FineTuningPL,Mosbach2021OnTS}. As shown by \citet{Margatina2022OnTI}, this can lead to poor performance when fine-tuning pre-trained language models with AL. In addition, existing pre-trained language models are often trained on texts that do not need specialized training to be understood. However, legal texts contain specialized words that are not common in other domains. Thus, task-adaptation is crucial for the effectiveness of active learning in legal text classification. In the first step of our proposed pipeline, we obtain the task-adapted pre-trained (TAPT) RoBERTa by continuing pre-training the model with unlabeled samples for the Masked Language Modeling (MLM) task, as suggested by \citet{Gururangan2020DontSP} and \citet{Margatina2022OnTI}. 
 
\subsection{Knowledge Distillation}\label{sec:distillation}

Previous works \citep{Reimers2019SentenceBERTSE,Li2020OnTS,Su2021WhiteningSR} have shown that, without fine-tuning, the sentence embeddings produced by pre-trained language models poorly capture semantic meaning of sentences, and are not comparable using cosine similarity. To overcome this shortcoming, \citet{Reimers2019SentenceBERTSE} introduced sentence transformers by adding a pooling layer on top of pre-trained transformer-based language models, and training them in a Siamese network architecture with pairs of similar sentences. Compared to out-of-the-box RoBERTa, a RoBERTa-based sentence transformer drives semantically comparable sentence embeddings. 

As we will explain in Sec.~\ref{sec:sampling}, we cluster the normalized sentence embeddings based on their Euclidean distance to efficiently acquire the labeled samples for the initial iteration of AL. The Euclidean distance between normalized embeddings can be driven from their cosine similarity. Hence, sentence embeddings that are comparable with cosine similarity can result in clusters with higher quality. In addition, semantically meaningful sentence embeddings give a better initialization of the \texttt{[CLS]} token, thereby obtaining better classification performance with a smaller training set.

We use a pre-trained RoBERTa-based sentence transformer (PT Sentence-RoBERTa) as a teacher model, and distill its knowledge to the TAPT RoBERTa. The resulting distilled task-adapted pre-trained (DisTAPT) RoBERTa produces semantically meaningful embeddings that are comparable via cosine similarity, and, as shown by our experiments (Sec.~\ref{sec:exp_distillaition}), benefit the classification task. 



\subsection{Initial Sampling}\label{sec:sampling}

Many AL strategies (\citealp{Gissin2019DiscriminativeAL,Gal2016DropoutAA}) require an initial set of $N$ labeled samples containing $P$ positive and $N-P$ negative sentences, which is either assumed to be available, or obtained by randomly sampling the entire dataset until the desired number of positive and negative samples are found. This approach is highly expensive for large and skewed datasets. We propose a simple, yet effective, strategy to efficiently acquire the initial labeled set. To this end, we leverage the distilled task-adapted pre-trained RoBERTa to cluster the unlabeled samples using KMeans algorithm \citep{MacQueen1967SomeMF}. The labeled set for the initial iteration is then driven from the cluster medoids. As a result, we shrink the pool of candidates from the entire dataset to the cluster medoids, therefore, reduce the number of actions for obtaining the initial annotated set, while achieving comparable performance with the standard approach for initial sampling.

\subsection{Active Learning}\label{sec:meth_al}
In the last phase, we iteratively fine-tune the DisTAPT RoBERTa for the downstream task. The initial labeled set is used at the first iteration. Then, more samples are labeled in the following rounds using an AL acquisition strategy until the annotation budget is exhausted, or the classifier satisfies the expected performance. 

Our proposed pipeline can be used with existing AL strategies and, as demonstrated by our experiments (Sec.~\ref{sec:exp_distillaition}), consistently outperforms standard AL approaches, regardless of the AL strategy used. 


\section{Experimental Setup}\label{sec:setup}

We evaluate our approach against four standard active learning strategies provided in the Low-Resource Text Classification Framework \citep{EinDor2020ActiveLF}:

\begin{itemize}
    \item \textbf{Random} At each iteration, this approach randomly chooses samples for annotation.
    \item \textbf{Hard-Mining} Selects instances that the model is  uncertain about, based on the absolute difference of prediction score and $0.5$. 
    \item \textbf{Perceptron Dropout} \citep{Gal2016DropoutAA} Also selects instances for which the model is least certain. The uncertainty is calculated using Monte Carlo Dropout on $10$ inference cycles. 
    \item \textbf{Discriminative Active Learning (DAL)} \citep{Gissin2019DiscriminativeAL} Deploys a binary classifier to select instances that best represent the entire unlabeled samples. 
\end{itemize}

We consider pre-trained RoBERTa and LEGAL-BERT~\citep{Chalkidis2020LEGALBERTTM} as the baselines. Note that our goal is not to rely on domain-adapted models like LEGAL-BERT since they might not always be available. For example, if the data is in German, we can find a pre-trained RoBERTa in German, but the LEGAL-BERT is pre-trained on English text only.

\subsection{Datasets}\label{sec:datasets}
We evaluate our framework on Contract-NLI \citep{Koreeda2021ContractNLIAD} and LEDGAR \citep{Tuggener2020LEDGARAL} benchmarks. 

Contract-NLI \citep{Koreeda2021ContractNLIAD} is a dataset for document-level natural language inference. It consists of $607$ documents with $77.8$ spans per document on average. Each span is checked against $17$ hypotheses and classified as contradiction, entailment, or not mentioned. In this work, we adapt this dataset to the classification task by considering each hypothesis as a category. If a span is classified as contradiction or entailment for a hypothesis, we label it with the corresponding category. Following this approach, we end up with a classification dataset with $4,371$ train, $614$ development, and $1,188$ test samples within $17$ classes.

LEDGAR \citep{Tuggener2020LEDGARAL} is a text classification benchmark consisting of a corpus of legal provisions in contracts. The entire dataset consists of $846,274$ provisions and $12,608$ labels. We only consider a subset of this dataset that corresponds to provisions with labels that appeared at least $10,000$ times in the corpus, resulting in $44,249$ train, $7,375$ development, and $12,907$ test samples across $5$ categories. Similar to \citet{Tuggener2020LEDGARAL}, we perform a $70\%-10\%-20\%$ random split to obtain the train, development and test sets. 

The class distributions of both datasets can be found in the appendix (Sec.~\ref{sec:app_class_dist}). Compared to Contract-NLI, LEDGAR has fewer categories, is an order of magnitude bigger, and is more balanced. 


\subsection{Implementation Details}

We base our implementation on the Low-Resource Text Classification Framework provided by \citet{EinDor2020ActiveLF}\footnote{\url{https://github.com/IBM/low-resource-text-classification-framework}}, and augment it with the task-adaptation, knowledge distillation, and initial sampling steps. 

As the pre-trained model, we use \texttt{roberta-base}\footnote{\url{https://huggingface.co/roberta-base}} (with $125$M parameters), the RoBERTa \citep{Liu2019RoBERTaAR} language model trained on the union of $5$ datatsets: Book corpus \citep{Zhu2015AligningBA}, English Wikipedia\footnote{\url{https://dumps.wikimedia.org}}, CC-News \citep{Mackenzie2020CCNewsEnAL}, OpenWebText Corpus \citep{Gokaslan2019OpenWeb}, and Stories \citep{Trinh2018ASM}, none of which belong to the legal domain. 

For LEGAL-BERT, we use the \texttt{nlpaueb/legal-bert-base-uncased}\footnote{\url{https://huggingface.co/nlpaueb/legal-bert-base-uncased}} (with $110$M parameters), trained on six datasets containing legal docments across Europe and the US.

For task-adaptation, we continue pre-training RoBERTa for the MLM task using the available unlabeled data. We train for $10$ epochs with batch-size $64$, and the learning rate set to $3\mathrm{e}{-4}$. The task-adapted model has perplexity $4.9706$ for Contract-NLI and $2.1628$ for LEDGAR.
 

For model distillation, we use \texttt{stsb-roberta-base-v2} (with $125$M parameters), a RoBERTa-based sentence transformer trained on the STS benchmark \citep{Cer2017SemEval2017T1}, as the teacher model, and the task-adapted RoBERTa as the student model. Mean Squared Error (MSE) is used as the loss function. The student model is trained for $10$ epochs, with $10$K warmup steps, $1\mathrm{e}{-4}$ learning rate and no bias correction. The final MSE ($\times 100$) is $6.8607$ for Contract-NLI, and $7.2003$ for LEDGAR.

For clustering the normalized sentence embeddings we use the KMeans implementation by \texttt{scikit-learn}. We cluster the Contract-NLI and LEDGAR sentence embeddings into $437$, and $442$ groups respectively. The number of clusters are chosen based on the dataset size, and the number of categories, and to make initial sampling with cluster medoids manageable for experts. 

In all the active learning experiments, we perform $5$ AL iterations, starting with $10$ initial samples, and increasing the size of the annotated data by $10$ at each iteration. Adam optimizer \citep{Kingma2015AdamAM} is used with learning rate set to $5\mathrm{e}{-5}$. The model is trained for $100$ epochs and early stopping is used with patience set to $10$. To account for randomization, we repeat each experiment three times.

To compare our approach with standard AL methods, we use F1-score as the evaluation metric as it captures both precision and recall and is sensitive to data distribution.

\section{Results and Discussion}

In this section, we provide the results of our experiments and explain them in detail. We start by comparing our approach with and without the initial medoid sampling against standard AL strategies (Sec.~\ref{sec:ours-vs-baseline}). Then, we show the effectiveness of knowledge distillation on top of task-adaptation (Sec.~\ref{sec:exp_distillaition}). In addition, we demonstrate the efficiency of the initial sampling with cluster medoids (Sec.~\ref{sec:exp_initial-sampling}). Finally, we evaluate how well our approach performs for different AL strategies (Sec.~\ref{sec:exp_AL_effect}). 

\begin{figure}[t]
    \centering
    \includegraphics[width=0.5\textwidth]{Contract-NLI_DAL_avg_with_legalbert.pdf}
    \includegraphics[width=0.5\textwidth]{LEDGAR_DAL_avg_with_legalbert.pdf}
    \caption{Test F1-score for \textbf{DAL} during AL iterations. The F1-score for the fully supervised fine-tuning is $0.6990$ for Contract-NLI and $0.9538$ for LEDGAR. The figure is best viewed in color.}
    \label{fig:ours-vs-baseline}
\end{figure}

\subsection{Efficient AL Pipeline}\label{sec:ours-vs-baseline}

Figure~\ref{fig:ours-vs-baseline} compares our approach with and without the initial sampling phase (DisTAPT with IS, and DisTAPT) to standard DAL with pre-trained (PT) RoBERTa, LEGALBERT, and TAPT RoBERTa for Contract-NLI and LEDGAR benchmarks. We report the average F1-score over all categories. DAL is chosen due to its better performance, as shown in Figure~\ref{fig:all_ALs}. 
The results for other AL strategies can be found in the appendix (Sec.~\ref{sec:app_tabs}). 

Our experiments show the importance of task-adaptation and knowledge distillation for pre-trained language models prior to fine-tuning with active learning. Figure~\ref{fig:ours-vs-baseline} illustrates that, for the same size of annotated data, our pipeline consistently achieves better performance than standard AL approaches even for LEGAL-BERT. 



For the Contract-NLI dataset, the F1-score obtained by fully-supervised fine-tuning (with $4,371$ labeled samples) is $0.6990$ for \texttt{roberta-base} and $0.7152$ for \texttt{legal-bert-base-uncased}. DisTAPT RoBERTa reaches a F1-score as high as $0.6508$ with only $40$ labeled samples. The best F1-score obtained using pre-trained RoBERTa is $0.3162$ with $30$ labeled samples, which is $0.3165$ lower than DisTAPT RoBERTa's F1-score for the same size of annotated data.

For the LEDGAR dataset, the F1-score obtained by the fully-supervised fine-tuning (with $44,249$ labeled samples) is $0.9538$ for \texttt{roberta-base} and $0.9588$ for \texttt{legal-bert-base-uncased}. DisTAPT RoBERTa reaches a very close performance of $0.9321$ F1-score with merely $60$ labeled samples. The highest F1-score that pre-trained RoBERT reaches is $0.7663$ with $20$ annotated samples, which is $0.0904$ lower than DisTAPT's performance with the same size of labeled data.

These results show that, for both datasets, there is only a small performance gap between our approach and the fully-supervised approach, indicating that our AL pipeline dramatically reduces the annotation cost, while achieving comparable performance with the fully-supervised fine-tuning.


In addition, It is observed that standard AL with off-the-shelf pre-trained RoBERTa is unstable. This is aligned with the previous works' observations \citep{Mosbach2021OnTS,Zhang2021RevisitingFB,Dodge2020FineTuningPL}. During fine-tuning, the pre-trained model should perform two tasks: adaptation to the legal domain with the new vocabulary, and classification. By performing task-adaptation and knowledge distillation before fine-tuning, we train the model in a curriculum learning approach, making the model stable even for small training sets.


\subsection{Effect of Knowledge Distillation}\label{sec:exp_distillaition}

To evaluate the effectiveness of knowledge distillation on the quality of obtained clusters, we compare the distribution of the Dunn Index of the clusters before and after knowledge distillation. For both datasets, after knowledge distillation, most of the clusters have higher Dunn Index which indicates that they are more compact and better separated than the clusters before knowledge distillation step. The results are provided in the appendix~\ref{sec:app_knowldge_distill} due to space constraints.


In addition, we evaluate the effect of knowledge distillation on the task-adapted pre-trained RoBERTa, and report the average F1-score over all classes for each dataset. Figure~\ref{fig:ours-vs-baseline} shows that, for both datasets, DisTAPT RoBERTa outperforms TAPT RoBERTa at early iterations of active learning, and as the size of the labeled set increases, the two models' performance converge. This can be explained by the fact that, initially, DisTAPT RoBERTa's embeddings better capture the semantics of sentences, and thus result in better classification performance. As the labeled data grows, TAPT RoBERTa is fine-tuned and can produce semantically meaningful embeddings as well. Hence, for a highly restricted annotation budget, distilling the knowledge of a sentence transformer to the TAPT language model can lead to performance gain. 


\subsection{Efficiency of Initial Medoid Sampling} \label{sec:exp_initial-sampling}

It was shown in Figure~\ref{fig:ours-vs-baseline} that DisTAPT with IS obtains comparable performance with DisTAPT without IS. In this section, we evaluate the \textit{efficiency} of the proposed sampling strategy for the initial iteration of AL. 

To this end, we simulate the standard sampling strategy by randomly sampling text segments from the full dataset until $5$ positive and $5$ negative samples are found. The number of iterations is then considered as the number of annotations required to collect the labeled set for the initial AL iteration. Similarly, to simulate our proposed initial sampling, we randomly sample from cluster medoids until $5$ positive and $5$ negative samples are obtained. To account for randomness, we repeat the simulations $1000$ times and report the median and the $90^{th}$ percentile over all runs.  

Table~\ref{tab:exp_medoid} illustrates the results of our simulations for Contract-NLI and LEDGAR. Due to the high number of classes in Contract-NLI, only eight categories of this dataset are presented in this table, and the results for other categories can be found in the appendix (Sec.~\ref{sec:app_medoid}). For each class, in addition to the median and $90^{th}$ percentile over $1000$ runs, the difference in the $90^{th}$ percentile between standard approach and our strategy (in $\%$) is reported as the gain in annotation effort. For example, for the \texttt{Sharing with third-parties} class in Contract-NLI, the $90^{th}$ percentile is $62\%$ less when using medoids for initial sampling, meaning that, with $90\%$ confidence, the annotators perform $62\%$ fewer actions to acquire the initial labeled set using our approach. 

It is observed that, for the skewed Contract-NLI dataset, our proposed initial sampling strategy reduces the number of actions performed by the annotator up to $63\%$. For LEDGAR however, which consists of balanced categories, the highest effort gain in sampling from cluster medoids is $25\%$. There are also few cases where using the entire dataset is more efficient than sampling from medoids. This happens when the class' frequency is higher in the full dataset than its frequency in the cluster medoids. 

Overall, our results demonstrate the advantage of using the cluster medoids for collecting the initial annotated samples for a skewed dataset like Contract-NLI, which is a realistic use-case in the legal domain. It is noteworthy that the original version of LEDGAR dataset is also imbalanced, but as explained in Sec.~\ref{sec:datasets}, due to the drastically high number of classes, and for the sake of comparison with skewed datasets, only the most dominant categories are kept in this work. 

Thanks to the semantically meaningful and comparable sentence embeddings obtained after the knowledge distillation step, the cluster medoids well represent the entire dataset, and thus sampling among them drastically reduces the annotation effort without harming the performance. 
As a real life scenario, consider a company with hundreds of legal contracts aiming to classify their sentences into multiple categories, under a restricted budget. Reducing the annotation effort means lowering down the financial costs of annotation, which can be highly expensive in the legal domain (over $\$2$ million for annotating around $500$ contracts according to~\citet{Hendrycks2021CUADAE}). 

% We study the best and worst case scenarios considering an expert trying to find $P$ positive samples for a target category given the initial pool of unlabeled samples $U_0$. In the best case, the first $P$ samples found by the expert belong to the target category, and the number of actions is the same with and without initial medoid sampling. In the worst case however, the expert will go through all samples from other categories until they find the first positive sample. In this case, the number of actions decreases by a great factor when sampling from cluster medoids instead of the entire dataset. Although these extreme cases have low probability of happening in practice, studying them can provide an approximate understanding of the effectiveness of using cluster medoids for initial sampling. In addition, for small classes in skewed datasets, the reality is closer to the worst case scenario.  

\begin{table*}[h]
\centering
\resizebox{\textwidth}{!}{
\begin{tabular}{clccccc}
\hline
\multirow{2}{*}{Dataset} & \multirow{2}{*}{Category} & \multicolumn{2}{c}{full dataset} & \multicolumn{2}{c}{medoids} & \multirow{2}{*}{gain(\%)} \\
& & median & $90^{th}\%$tile & median & $90^{th}\%$tile &  \\
\hline
\multirow{8}{*}{\small{\texttt{\rotatebox[origin=c]{90}{Contract-NLI}}}} 
& \small{\texttt{Inclusion of verbally conveyed information}} & 75.0 & 125.0 & 35.5 & 59.0 & 52.8\\
& \small{\texttt{No licensing}} & 64.0 & 108.0 & 68.5 & 109.1 & -1.0 \\
& \small{\texttt{No reverse engineering}} & 342.0 & 568.0 & 144.0 & 209.1 & 63.2\\
& \small{\texttt{Notice on compelled disclosure}} & 74.5 & 122.0 & 99.0 & 155.0 & -27.0\\
& \small{\texttt{Sharing with employees}} & 57.0 & 90.0 & 21.0 & 34.1 & 62.1 \\
& \small{\texttt{Sharing with third-parties}} & 54.0 & 92.1 & 21.0 & 35.0 & 62.0\\
& \small{\texttt{Survival of obligations}} & 64.0 & 106.0 & 36.0 & 57.0 & 46.2\\
& \small{\texttt{Return of confidential information}} & 116.0 & 189.0 & 61.0 & 99.0 & 47.6\\
\hline
\multirow{5}{*}{\small{\texttt{\rotatebox[origin=c]{90}{LEDGAR}}}} & \small{\texttt{Amendments}} & 23.0 & 37.1 & 21.0 & 33.0 & 10.8\\
& \small{\texttt{Counterparts}} & 26.0 & 42.0 & 34.0 & 54.1 & -28.8 \\
& \small{\texttt{Entire agreements}} & 26.0 & 42.0 & 33.0 & 55.0 & -30.9 \\
& \small{\texttt{Governing laws}} & 17.5 & 28.0 & 14.0 & 21.0 & 25.0\\
& \small{\texttt{Notices}} & 29.0 & 49.0 & 26.0 & 44.0 & 10.2\\
\hline
\end{tabular}}
\caption{\label{tab:exp_medoid}
Number of actions to acquire the initial labeled set for $8$ categories of Contract-NLI, and LEDGAR when sampling from the full dataset (standard approach), and sampling from the cluster medoids (our approach).}
\end{table*}

% \begin{table*}[h!]
% \centering
% \begin{tabular}{lcccc}
% \hline
% Category &  \multicolumn{2}{c}{full dataset} & \multicolumn{2}{c}{medoids} \\
%  & median & 90-percentile & median & 90-percentile \\
% \hline
% \small{\texttt{Amendments}} & 23 & 37.1 &21 & 33\\
% \small{\texttt{Counterparts}} & 26 & 42 & 34 & 54.1 \\
% \small{\texttt{Entire agreements}} & 26 & 42 & 33 & 55 \\
% \small{\texttt{Governing laws}} & 17.5 & 28 & 14 & 21\\
% \small{\texttt{Notices}} & 29 & 49 & 26 & 44\\
% \hline
% \end{tabular}
% \caption{\label{tab:ledgar_medoid}
% Number of actions to acquire the initial labeled set for categories of \textbf{LEDGAR} when sampling from the full dataset (standard approach), and sampling from the cluster medoids.}
% \end{table*}

% Tables \ref{tab:contract_nli_medoid} and \ref{tab:ledgar_medoid} show the number of positive samples among the entire dataset and the cluster medoids for $5$ categories of Contract-NLI and LEDGAR respectively. For a small class like {\small\texttt{No reverse engineering}} in the Contract-NLI dataset, in the worst case, $4,311$ samples should be verified before finding the first positive sentence when sampling from the entire dataset. This number is reduced to $423$ using our proposed initial sampling strategy. For LEDGAR, the number of actions in the worst case drops from $37,185$ to $364$ with our initial sampling strategy. However, as this dataset is more balanced than Contract-NLI, the worst case scenario is further from reality. 

\begin{figure}[h!]
    \centering
    \includegraphics[width=0.5\textwidth]{Contract-NLI_avg.pdf}
    \includegraphics[width=0.5\textwidth]{LEDGAR_avg.pdf}
    \caption{Comparison of four AL strategies when used with DisTAPT RoBERTa with IS.}
    \label{fig:all_ALs}
\end{figure}

\subsection{Effect of AL strategy}\label{sec:exp_AL_effect}

Finally, we evaluate the generalizability of our approach over the four AL strategies mentioned in Sec.~\ref{sec:setup}: DAL, Random, Hard-Mining, and Perceptron Dropout. As shown in Figure~\ref{fig:all_ALs}, DAL results in the best performance with at most $0.08$ higher F1-score than other strategies with $60$ labeled samples for Contract-NLI, and less than $0.04$ higher F1-score with $40$ annotated samples for LEDGAR. The small performance gap of these four AL methods in our pipeline indicates the generalizability of this approach to various AL strategies. 



\section{Conclusion}

We propose an efficient active learning pipeline for legal text classification. Our approach leverages the available unlabeled data to adapt the pre-trained language model to the downstream task, and guide its embeddings to a semantically meaningful space before fine-tuning. We use model distillation to produce semantically comparable embeddings. A future work can study the effect of other approaches like BERT-Flow \citep{Li2020OnTS} and whitening \citep{Su2021WhiteningSR} on AL with this pipeline. Moreover, we design a simple strategy to efficiently acquire a labeled set of positive and negative samples for the initial iteration of active learning. 

Our experiments over Contract-NLI and LEDGAR benchmarks demonstrate the effectiveness of our approach compared to standard active learning strategies. Our results also show that our pipeline obtains very close performance to the fully-supervised approach with considerably less annotation cost. We test our methodology in the legal domain, and for four AL strategies, but we expect it to generalize to other strategies like ALPS and BADGE, and other specialized domains, like medicine. We leave this evaluation as a future work. 



\section*{Limitations}

In this work, we have shown the importance of task-adaptation and knowledge distillation, and that we can leverage the available unlabeled data to perform efficient fine-tuning via active learning and obtain better performance. The price to pay for this performance gain is time and computational power. The time taken by task-adaptation and distillation scales with the size of unlabeled data. On the other hand, more unlabeled samples result in more effective adaptation to the downstream task. Therefore, the user of this approach needs to find the best trade-off given their data, annotation budget, time and computational power. For, LEDGAR, the larger dataset used in this work, we performed the adaptation and distillation steps in $4$ and $1$ hour(s) respectively, using a single Nvidia GeForce GTX TITAN X GPU.

Moreover, we showed that by clustering the sentence embeddings produced by DisTAPT RoBERTa, the initial labeled set can be acquired more efficiently. Nevertheless, this approach inherits the limitations of clustering. Namely, the time complexity of clustering the embeddings scales with the data, and the number of clusters should be empirically chosen. In our experiments we spent $10$ minutes to cluster the $44,249$ samples belonging to LEDGAR dataset into $442$ groups. 


\section*{Ethics Statement}
Industries have hundreds of contracts with tens of thousands of sentences that belong to various topics. Labeling all of these samples is a highly expensive and time-consuming process. In this work, we aim to reduce the resources spent on this task by leveraging recent advances in natural language processing, while keeping the human expert in the loop. The goal is to reduce the human effort in annotation so that the legal experts' time and knowledge can be used in another task at which humans are better than machines. 

% \section*{Acknowledgements}

% \begin{table}
% \centering
% \begin{tabular}{lc}
% \hline
% \textbf{Command} & \textbf{Output}\\
% \hline
% \verb|{\"a}| & {\"a} \\
% \verb|{\^e}| & {\^e} \\
% \verb|{\`i}| & {\`i} \\ 
% \verb|{\.I}| & {\.I} \\ 
% \verb|{\o}| & {\o} \\
% \verb|{\'u}| & {\'u}  \\ 
% \verb|{\aa}| & {\aa}  \\\hline
% \end{tabular}
% \begin{tabular}{lc}
% \hline
% \textbf{Command} & \textbf{Output}\\
% \hline
% \verb|{\c c}| & {\c c} \\ 
% \verb|{\u g}| & {\u g} \\ 
% \verb|{\l}| & {\l} \\ 
% \verb|{\~n}| & {\~n} \\ 
% \verb|{\H o}| & {\H o} \\ 
% \verb|{\v r}| & {\v r} \\ 
% \verb|{\ss}| & {\ss} \\
% \hline
% \end{tabular}
% \caption{Example commands for accented characters, to be used in, \emph{e.g.}, Bib\TeX{} entries.}
% \label{tab:accents}
% \end{table}

% \begin{table*}
% \centering
% \begin{tabular}{lll}
% \hline
% \textbf{Output} & \textbf{natbib command} & \textbf{Old ACL-style command}\\
% \hline
% \citep{ct1965} & \verb|\citep| & \verb|\cite| \\
% \citealp{ct1965} & \verb|\citealp| & no equivalent \\
% \citet{ct1965} & \verb|\citet| & \verb|\newcite| \\
% \citeyearpar{ct1965} & \verb|\citeyearpar| & \verb|\shortcite| \\
% \citeposs{ct1965} & \verb|\citeposs| & no equivalent \\
% \citep[FFT;][]{ct1965} &  \verb|\citep[FFT;][]| & no equivalent\\
% \hline
% \end{tabular}
% \caption{\label{citation-guide}
% Citation commands supported by the style file.
% The style is based on the natbib package and supports all natbib citation commands.
% It also supports commands defined in previous ACL style files for compatibility.
% }
% \end{table*}



% Entries for the entire Anthology, followed by custom entries
% This must be in the first 5 lines to tell arXiv to use pdfLaTeX, which is strongly recommended.
\pdfoutput=1
% In particular, the hyperref package requires pdfLaTeX in order to break URLs across lines.

\documentclass[11pt]{article}

% Remove the "review" option to generate the final version.
\usepackage{EMNLP2022}
\usepackage{times}
\usepackage{latexsym}
\usepackage{amsmath}
\usepackage{amssymb}
\usepackage{enumitem}
\usepackage{mathtools}
\usepackage{multirow}
\usepackage{booktabs}
\usepackage{multicol}
\usepackage{multirow}
\newtheorem{theorem}{Theorem}
% For proper rendering and hyphenation of words containing Latin characters (including in bib files)
\usepackage[T1]{fontenc}
% For Vietnamese characters
% \usepackage[T5]{fontenc}
% See https://www.latex-project.org/help/documentation/encguide.pdf for other character sets

% This assumes your files are encoded as UTF8
\usepackage[utf8]{inputenc}
% Standard package includes
\usepackage{times}
\usepackage{latexsym}

% For proper rendering and hyphenation of words containing Latin characters (including in bib files)
\usepackage[T1]{fontenc}
% For Vietnamese characters
% \usepackage[T5]{fontenc}
% See https://www.latex-project.org/help/documentation/encguide.pdf for other character sets

% This assumes your files are encoded as UTF8
\usepackage[utf8]{inputenc}

% This is not strictly necessary, and may be commented out.
% However, it will improve the layout of the manuscript,
% and will typically save some space.
\usepackage{microtype}

% This is also not strictly necessary, and may be commented out.
% However, it will improve the aesthetics of text in
% the typewriter font.
\usepackage{inconsolata}


% If the title and author information does not fit in the area allocated, uncomment the following
%
%\setlength\titlebox{<dim>}
%
% and set <dim> to something 5cm or larger.

\title{Adaptive Contrastive Learning on Multimodal Transformer for \\ Review Helpfulness Predictions}


\author{Thong Nguyen$^{1, 2}$,~~Xiaobao Wu$^{3}$,~~Anh Tuan Luu$^{3}$\thanks{~~Corresponding Author}, \\
~~\textbf{Cong-Duy Nguyen}$^{3}$, \textbf{Zhen Hai}$^{4}$,~~\textbf{Lidong Bing}$^{4}$ \\
  $^1$National University of Singapore, Singapore \\
  $^2$VinAI Research, Vietnam \\
  $^3$Nanyang Technological University, Singapore \\
  $^4$DAMO Academy, Alibaba Group\\
  \texttt{\small e0998147@u.nus.edu, anhtuan.luu@ntu.edu.sg} \\}

\begin{document}
\maketitle
\begin{abstract}
Modern Review Helpfulness Prediction systems are dependent upon multiple modalities, typically texts and images. Unfortunately, those contemporary approaches pay scarce attention to polish representations of cross-modal relations and tend to suffer from inferior optimization. This might cause harm to model’s predictions in numerous cases. To overcome the aforementioned issues, we propose Multi-modal Contrastive Learning for Multimodal Review Helpfulness Prediction (MRHP) problem, concentrating on mutual information between input modalities to explicitly elaborate cross-modal relations. In addition, we introduce Adaptive Weighting scheme for our contrastive learning approach in order to increase flexibility in optimization. Lastly, we propose Multimodal Interaction module to address the unalignment nature of multimodal data, thereby assisting the model in producing more reasonable multimodal representations. Experimental results show that our method outperforms prior baselines and achieves state-of-the-art results on two publicly available benchmark datasets for MRHP problem. 
\end{abstract}
\section{Introduction}
Current e-commerce sites such as Amazon, Ebay, etc., construct review platforms to collect user feedback concerning their products. These platforms play a fundamental role in online transactions since they help future consumers collect useful reviews which assist them in deciding whether to make the purchase or not. Unfortunately, nowadays the number of user-generated reviews is overwhelming, raising doubts related to the relevance and veracity of reviews. Therefore, there is a need to verify the quality of reviews before publishing them to prospective customers. As a result, this inspires a recent surge of interest targeting the Review Helpfulness Prediction (RHP) problem.
\begin{table}[h!]
\centering
\begin{tabular}{p{0.9\linewidth}}
% \toprule
\textbf{Product Information} \\
\small{
The Cooks Standard 6-Quart Stainless Steel Stockpot with Lid is made with 18/10 stainless steel with an aluminum disc layered in the bottom. The aluminum disc bottom provides even heat distribution and prevents hot spots. Tempered glass lid with steam hole vent makes viewing food easy. Stainless steel riveted handles offer durability. Induction compatible. Works on gas, electric, glass, ceramic, etc. Oven safe to 500F, glass lid to 350F. Dishwasher safe.} \\
\includegraphics[width=0.3\linewidth]{images/product_images/pic_4_0.jpg} \includegraphics[width=0.3\linewidth]{images/product_images/pic_4_1.jpg} \\
\midrule
\textbf{Review 1} \\
\small{
I needed a stainless steel pot for canning my tomatoes.  I learned the hard way that you have to use a non-reactive pot or else your end result will be inedible (I thought I was using stainless steel but quickly realized it wasnt)  I headed to Amazon and came across this Cooks Standard SS Cookpot with cover and bought it after reading the reviews.  I have had it for just under a year and it still looks just as good as the day I bought it.  I couldn't be happier with my purchase!  Oh, and by the way, this one actually is stainless steel unlike the other pot I bought that said it was and wasn't.} \\
% Label score: \textbf{4} \\
% MCR score: \textbf{0.168} \\
% Our Model score: \textbf{4.651} \\
\midrule
\textbf{Review 2} \\
\small{
I ordered it on May 21st. What a waste of time and money.} \\
\includegraphics[width=0.3\linewidth]{images/review_images/pic_1_0.jpg} \\
% Label score: \textbf{1} \\
% MCR score: \textbf{3.637} \\
% Our Model score: \textbf{0.743} \\
% \bottomrule
\end{tabular}
\small
\begin{tabular}{ccc}
\toprule
& Review 1 & Review 2 \\
\midrule
Label score & 4 & 1 \\
MCR score & 0.168 & 3.637 \\
Our Model score & \textbf{4.651} & \textbf{0.743} \\
\bottomrule
\end{tabular}
\caption{Example of unreasonable predictions in the Multimodal Review Helpfulness Prediction task.}
\label{table:example}
\vspace{-5mm}
\end{table}

Two principal groups of early efforts focus on purely textual data. The first group follows feature engineering techniques, retrieving argument-based features \cite{liu2017using}, lexical features \cite{krishnamoorthy2015linguistic}, and semantic features \cite{kim2006automatically}, as input to their classifier. Inherently, their methods are labor-intensive and vulnerable to the typical issues of conventional machine learning methods. Instead of relying on manual features, the second group leverages deep neural models, for instance, RNN \cite{alsmadi2020predicting} and CNN \cite{chen2018cross}, to learn rich features automatically. Nonetheless, their approach is ineffective because the helpfulness of a review is not only contingent upon textual information but also other modalities.

To cope with the above issues, recent works \cite{liu2021multi,han2022sancl} proposed to utilize multi-modality via the Multi-perspective Coherent Reasoning (MCR) model. Hypothesizing that a review is helpful if it exhibits coherent text and images with the product information, those works take into account both textual and visual modality of the inputs, then estimate their coherence level to discern whether the reviews are \emph{helpful} or \emph{unhelpful}. However, the MCR model contains a detrimental drawback. Particularly, it aims to maximize the scores $s_p$ of positive (helpful) product-review pairs while minimizing those $s_n$ of negative (unhelpful) pairs. Hence, it was assumed that following the aforementioned manner would project features with similar semantics to stay close and those with disparate ones to be distant apart. Unfortunately, in multimodal learning, this was shown not to be the case, causing the model to learn ad-hoc representations \cite{zolfaghari2021crossclr}. This is one reason leading to unreasonable predictions of MCR in Table \ref{table:example}. As it can be seen, even though Review 1 closely relates to the product of ``\emph{6-Quart Stainless Steel Stockpot}'', the model classifies it as \emph{unhelpful}. In addition, the target of Review 2’s text content is vague because it does not specifically correspond to the ``\emph{Stockpot}''. In fact, it can be used for any product. Moreover, the image does not clearly show any hint of the ``\emph{Stockpot}'' as well. Despite such vagueness, the output of MCR for Review 2 is still \emph{helpful}. 

As a remedy to this problem, we propose Cross-modal Contrastive Learning to mine the mutual information of cross-modal relations in the input to capture more sensible representations. Nonetheless, plainly applying symmetric gradient pattern, which is similar to MCR that they assign equivalent penalty to $s_n$ and $s_p$, is inflexible. In cases that $s_p$ is small and $s_n$ is already negatively skewed, or both $s_p$ and $s_n$ are positively skewed, it is irrational to assign equivalent penalties to both $s_p$ and $s_n$. Last but not least, MCR directly leverages Coherent Reasoning, repeatedly enforcing alignment among modalities in the input. This ignores the unaligned nature of multimodal input, for example, images might only refer to a particular section in the text, hence do not completely align with the textual content. In consequence, strictly forming alignment can make the model learn inefficient multimodal representations  \cite{tsai2019multimodal}.

To overcome the above problems, we propose an adaptive scheme to accomplish the flexibility in the optimization of our contrastive learning stage. Finally, we propose to adopt a multimodal attention module that reinforces one modality’s high-level features with low-level ones of other modalities. This not only relaxes the alignment assumption but also informs one modality of information of others, encouraging refined representation learning.

In sum, our contributions are three-fold:
\begin{itemize}
    \item We propose an Adaptive Cross-modal Contrastive Learning for Review Helpfulness Prediction task by polishing cross-modal relation representations.
    \item We propose a Multimodal Interaction module which correlates modalities’ features without depending upon the alignment assumption.
    \item We conducted extensive experiments on two datasets for the RHP problem and found that our method outperforms other baselines which are both textual-only and multimodal, and obtains state-of-the-art results on those benchmarks.
\end{itemize}
\section{Model Architecture}
\begin{figure*}[t]
    \centering
    \includegraphics[width=\linewidth]{figures/fig_model.pdf}
    \caption{Diagram of our Multimodal Review Helpfulness Prediction model.}
    \label{fig:model}
\end{figure*}

In this section we delineate the overall architecture of our MRHP model. Particular modules of our system are depicted in Figure \ref{fig:model}.
\subsection{Problem Definition}
Given a product item $p$, which consists of a description $T^p$ and images $I^p$, and a set of reviews $R = \{r_1,…, r_N\}$, where each review is composed of user-generated text $T^r_i$ and images $I^r_i$, RHP model’s task is to generate the scores
\begin{equation}
    s_i = f(p, r_i), \quad 1 \leq i \leq N
\end{equation}
where $N$ is the number of reviews for product $p$ and $f$ is the scoring function of the RHP model. Empirically, each score estimated by $f$ indicates the helpfulness level of each review, and the ground-truth is the descending sort order of helpfulness scores.
\subsection{Encoding Modules}
Our model accepts product description $T^p$, product images $I^p$, review text $T^r_i$, and review images $I^r_i$ as input. The encoding process of those elements is described as follows.

\noindent\textbf{Text Encoding} Product description and review text are sequences of words. Each sequence is indexed into the word embedding layer and then passed into the respective LSTM layer for product or review.
\begin{gather}
    K^p = \text{LSTM}^p (\mathbf{W}_{\textbf{emb}} (T^p)) \\
    K^r = \text{LSTM}^r (\mathbf{W}_{\textbf{emb}} (T^r)) 
\end{gather}
where $K^p \in \mathbb{R}^{l_p \times d}$, $K^r \in \mathbb{R}^{l_r \times d}$, $l_p$ and $l_r$ are the sequence lengths of product and review text respectively, and $d$ is the hidden size.

\noindent\textbf{Image Encoding} We follow \citet{anderson2018bottom} to take detected objects as embeddings of the image. In particular, a pre-trained Faster R-CNN is applied to extract ROI features for $m$ objects $\{\mathbf{a}_1, \mathbf{a}_2, …, \mathbf{a}_m\}$ from the product and review images. Subsequently, we encode extracted features using the self-attention module (SelfAttn) \cite{vaswani2017attention}
\begin{equation}
    A = \text{SelfAttn}(\{\mathbf{a}_1, \mathbf{a}_2, ..., \mathbf{a}_m\})
\end{equation}
where $A \in \mathbb{R}^{m \times d}$ and $d$ is the hidden size. Here we use $A^p$ and $A^r$ to indicate product and review image features, respectively.

\subsection{Multimodal Interaction Module}
We consider two components $\gamma$, $\eta$ with their inputs $X_\gamma$, $X_\eta$, where $\eta$ is the concatenation of input elements apart from the one in $\gamma$. For instance, if $\gamma = {K^p}$, then $\eta = [K^r, A^p, A^r]$, where $[., .]$ indicates the concatenation operation. We define each cross-modal attention block to have three components $Q$, $K$, and $V$:
\begin{gather}
    Q_\gamma = X_\gamma \cdot W_{Q_\gamma} \\
    K_\eta = X_\eta \cdot W_{K_\eta} \\
    V_\eta = X_\eta \cdot W_{V_\eta}
\end{gather}
where $W_{Q_\gamma} \in \mathbb{R}^{d_\gamma \times d_k}$, $W_{K_\eta} \in \mathbb{R}^{d_\eta \times d_k}$, and $W_{V_\eta} \in \mathbb{R}^{d_\eta \times d_v}$ are weight matrices. The interaction between $\gamma$ and $\eta$ is computed in the cross-attention manner
\begin{equation}
    \begin{split}
        Z_\gamma = \text{CM}_{\gamma} (X_\gamma, X_\eta)
        = \text{softmax} \left(\frac{Q_\gamma \cdot K^T_{\eta}}{\sqrt{d_k}}\right) \cdot V_\eta
    \end{split}
\end{equation}
Our full module comprises $D$ layers of the above-mentioned attention block, as indicated in the right part of Figure \ref{fig:model}. Theoretically, the computation is carried out as follows
\begin{gather}
    Q_{\gamma}[0] = X_\gamma\\
    T[i] = \text{CM}_{\gamma}[i] (\text{LN}(Q_{\gamma}[i-1]), \text{LN}(X_\eta)) \\
    U_{\gamma}[i] = T[i] + Q_{\gamma}[i-1] \\
    Q_{\gamma}[i] = \text{GeLU}(\text{Linear}(U_{\gamma}[i])) 
\end{gather}
where $\textit{LN}$ denotes layer normalization operator. We iteratively estimate cross-modal features for product text, product images, review text, and review images with a view to obtaining $H^p$, $V^p$, $H^r$, and $V^r$.
\begin{gather}
    H^p = Q^p_k[D], \quad V^p = Q^p_a[D] \\
    H^r = Q^r_k[D], \quad V^r = Q^r_a[D] 
\end{gather}

After our cross-modal interaction module, we proceed to pass features to undertake relation fusion in three paths: intra-modal, inter-modal, and intra-review.

\noindent\textbf{Intra-modal Fusion} The intra-modal alignment is calculated for two relation kinds: (1) product text - review text and (2) product image - review image. Firstly, we learn alignment among intra-modal features via self-attention modules
\begin{gather}
    H^\text{intraM} = \text{SelfAttn}([H^p, H^r]) \\
    V^\text{intraM} = \text{SelfAttn}([V^p, V^r])
\end{gather}
Then intra-modal hidden representations are fed to a CNN, and continuously a max-pooling layer to attain salient entries
\begin{equation}
    \mathbf{z}^\text{intraM} = \text{MaxPool} (\text{CNN}([H^{\text{intraM}}, V^{\text{intraM}}]))
\end{equation}
\noindent\textbf{Inter-modal Fusion} Similar to intra-modal alignment, inter-modal one is calculated for two types of relations as well: (1) product text - review image and (2) product image - review text. The first step is also to relate feature components using self-attention modules
\begin{gather}
    H^{\text{prd\_txt - rvw\_img}} = \text{SelfAttn}([H^p, V^r]) \\
    H^{\text{prd\_img - rvw\_txt}} = \text{SelfAttn}([V^p, H^r])
\end{gather}
We adopt a mean-pool layer to aggregate inter-modal features and then concatenate the pooled vectors to construct the final inter-modal representation
\begin{gather}
    I^{\text{prd\_txt - rev\_img}} = \text{MeanPool}(H^{\text{prd\_txt - rvw\_img}})\\
    I^{\text{prd\_img - rev\_txt}} = \text{MeanPool}(H^{\text{prd\_img - rvw\_txt}})\\
    \mathbf{z}^{\text{interM}} = [I^{\text{prd\_txt - rvw\_img}}, I^{\text{prd\_img - rvw\_txt}}]
\end{gather}
\noindent\textbf{Intra-review Fusion} The estimation of intra-review module completely mimics the inter-modal manner. The only discrimination is that the estimation is taken upon two different relations: (1) product text - product image and (2) review text - review image.
\begin{gather}
    H^{\text{prd\_txt - prd\_img}} = \text{SelfAttn}([H^p, V^p]) \\
    H^{\text{rvw\_txt - rev\_img}} = \text{SelfAttn}([H^r, V^r]) \\
    G^{\text{prd\_txt - prd\_img}} = \text{MeanPool}(H^{\text{prd\_txt - prd\_img}}) \\
    G^{\text{rvw\_txt - rvw\_img}} = \text{MeanPool}(H^{\text{rvw\_txt - rvw\_img}}) \\
    \mathbf{z}^{\text{intraR}} = [G^{\text{prd\_txt - prd\_img}}, G^{\text{rvw\_txt - rvw\_img}}]
\end{gather}
Finally, we concatenate intra-modal, inter-modal, and intra-review output, and then feed the concatenated vector to the linear layer to obtain the ranking score:
\begin{gather}
\mathbf{z}^{\text{final}} = [\mathbf{z}^{\text{intraM}}, \mathbf{z}^{\text{interM}}, \mathbf{z}^{\text{intraR}}] \\
f(p,r_i) = \text{Linear}(\mathbf{z}^{\text{final}})
\end{gather}
\section{Training Strategies}
\subsection{Adaptive Cross-modal Contrastive Learning}
In this section, we explain the formulation and adaptive pattern along with its derivation of our Cross-modal Contrastive Learning.

\noindent\textbf{Cross-modal Contrastive Learning} First of all, we extract hidden states of helpful product-review pairs. Second of all, hidden features are max-pooled to extract meaningful entries.
\begin{gather}
    \mathbf{h}^p = \text{MaxPool}(H^p), \, \mathbf{h}^r = \text{MaxPool}(H^r) \\
    \mathbf{v}^p = \text{MaxPool}(V^p), \, \mathbf{v}^r = \text{MaxPool}(V^r) 
\end{gather}
We formulate our contrastive learning framework taking positive and negative pairs from the above-mentioned cross-modal features. In our framework, we hypothesize that pairs established by modalities of the same sample are positive, whereas those formed by modalities of distinct ones are negative. 
\begin{equation}
    \mathcal{L}_{\text{CE}} = -\sum_{i=1}^{B} \text{sim}(\mathbf{t}^1_i, \mathbf{t}^2_i) +  \sum_{j=1, k=1, j \neq k}^{B} \text{sim}(\mathbf{t}_j^{1}, \mathbf{t}_k^{2})
\end{equation}
where $\mathbf{t}^1, \mathbf{t}^2 \in \{\mathbf{h}^p, \mathbf{h}^r, \mathbf{v}^p, \mathbf{v}^r\}$, and $B$ denotes the batch size in the training process.

\noindent\textbf{Adaptive Weighting} The standard contrastive objective suffers from inflexible optimization due to irrational gradient assignment to positive and negative pairs. As a result, to tackle the problem, we propose the Adaptive Weighting Strategy for our contrastive framework. Initially, we introduce weights $\epsilon^p$ and $\epsilon^n$ to represent distances from the optimum, then integrate them into positive and negative terms of our loss.
\begin{equation}
\begin{split}
    &\mathcal{L}_{\text{AdaptiveCE}} = -\sum_{i=1}^{B} \epsilon^p_i \cdot \text{sim}(\mathbf{t}^1_i, \mathbf{t}^2_i) \\
    &+ \sum_{j=1, k=1, j \neq k}^{B} \epsilon_{j,k}^n \cdot \text{sim}(\mathbf{t}_j^{1}, \mathbf{t}_k^{2})
\end{split}
\label{eq:adaptive_ce}
\end{equation}
where $\epsilon_i^p = [o^p - \text{sim}(\mathbf{t}^1_i, \mathbf{t}^2_i)]_+$ and  $\epsilon_{j,k}^n = [\text{sim}(\mathbf{t}^1_j, \mathbf{t}^2_k) - o^n]_+$. Investigating the intuition to determine the values for $o^p$ and $o^n$, we continue to conduct derivation and arrive in the following theorem

\begin{theorem} Adaptive Contrastive Loss (\ref{eq:adaptive_ce}) has the hyperspherical form: 
\begin{equation*}
    \begin{split}
       &\mathcal{L}_{\text{AdaptiveCE}} = \sum_{i=1}^{B} \left(\text{sim} (\mathbf{t}^1_i, \mathbf{t}^2_i) - \frac{o^p}{2}\right)^2 \\
       &+ \sum_{j=1, k=1, j \neq k}^{B} \left(\text{sim} (\mathbf{t}^1_j, \mathbf{t}^2_k) - \frac{o^n}{2}\right)^2 - C, \\
       &\quad \text{where} \, C > 0         
    \end{split}
\end{equation*}
\label{theorem:spherical}
\end{theorem}
We provide the proof for Theorem (\ref{theorem:spherical}) in the Appendix section. As a consequence, theoretically the contrastive objective arrives in the optimum when $\text{sim}(\mathbf{t}_i^1, \mathbf{t}_i^2) = \frac{o^p}{2}$ and $\text{sim}(\mathbf{t}_j^1, \mathbf{t}_k^2) = \frac{o^n}{2}$. Based upon this observation, in our experiments we set $o^p = 2$ and $o^n = 0$.

\subsection{Training Objective}
For the Review Helpfulness Prediction problem, the model’s parameters are updated according to the pairwise ranking loss as follows
\begin{equation}
    \mathcal{L}_{\text{ranking}} = \sum_i \text{max} (0, \beta - f(p_i, r^{+}) + f(p_i, r^{-}))
\end{equation}
where $r^{+}$and $r^{-}$ are random reviews in which $r^{+}$ possesses a higher helpfulness level than $r^{-}$. We jointly combine the contrastive goal with the ranking objective of the Review Helpfulness Prediction problem to train our model
\begin{equation}
    \mathcal{L} = \mathcal{L}_\text{AdaptiveCE} + \mathcal{L}_\text{ranking}
\end{equation}
\section{Experiments}
\begin{table}[ht]
\centering
\resizebox{\linewidth}{!}{
\begin{tabular}{l|l|ccc}
\toprule
\multirow{2}{*}{\textbf{Dataset}} & \multirow{2}{*}{\textbf{Split}} & \multicolumn{3}{c}{\textbf{Category (Product / Review)}} \\ 
& & Clothing & Electronics. & Home \\
\midrule
 \multirow{2}{*}{Lazada} & Train \& Dev & 8K/130K & 5K/52K & 4K/16K  \\
  & Test & 2K/32K & 1K/13K & 1K/13K \\
\midrule
 \multirow{2}{*}{Amazon} & Train \& Dev & 16K/349K & 13K/325K & 18K/462K  \\
  & Test & 4K/87K & 3K/80K & 5K/111K \\
 \bottomrule
\end{tabular} }
\caption{
Statistics of MRHP datasets.}
\label{table:datasets}
\end{table}

\begin{table*}[t]
\centering
\resizebox{1\textwidth}{!}{
\begin{tabular}{|l|l|ccc|ccc|ccc|}
\toprule
\multirow{2}{*}{\textbf{Type}} & \multirow{2}{*}{\textbf{Method}} & \multicolumn{3}{c|}{\textbf{Clothing}} & \multicolumn{3}{c|}{\textbf{Electronics}} & \multicolumn{3}{c|}{\textbf{Home}} \\ 
 &  & \textbf{MAP} & \textbf{N@3} & \textbf{N@5} & \textbf{MAP} & \textbf{N@3} & \textbf{N@5} & \textbf{MAP} & \textbf{N@3} & \textbf{N@5} \\
\midrule
\multirow{4}{*}{Text-only} & BiMPM & 60.0 & 52.4 & 57.7 & 74.4 & 67.3 & 72.2 & 70.6 & 64.7 & 69.1 \\
 & EG-CNN & 60.4 & 51.7 & 57.5 & 73.5 & 66.3 & 70.8 & 70.7 & 63.4 & 68.5 \\
 & Conv-KNRM & 62.1 & 54.3 & 59.9 & 74.1 & 67.1 & 71.9 & 71.4 & 65.7 & 70.5 \\
 & PRH-Net & 62.1 & 54.9 & 59.9 & 74.3 & 67.0 & 72.2 & 71.6 & 65.2 & 70.0 \\
\midrule
\multirow{4}{*}{Multimodal} & SSE-Cross & 66.1 & 59.7 & 64.8 & 76.0 & 68.9 & 73.8 & 72.2 & 66.0 & 71.0 \\
 & DR-Net & 66.5 & 60.7 & 65.3 & 76.1 & 69.2 & 74.0 & 72.4 & 66.3 & 71.4 \\
 & MCR & 68.8 & 62.3 & 67.0 & 76.8 & 70.7 & 75.0 & 73.8 & 67.0 & 72.2 \\
 & \textbf{Our Model} & \textbf{70.3} & \textbf{64.7} & \textbf{69.0} & \textbf{78.2} & \textbf{72.4} & \textbf{76.5} & \textbf{75.2} & \textbf{68.8} & \textbf{73.7} \\
\bottomrule
\end{tabular} }
\caption{
Helpfulness Prediction results on Lazada-MRHP dataset.}
\label{table:lazada_results}
\end{table*}

\begin{table*}[t]
\centering
\resizebox{1\textwidth}{!}{
\begin{tabular}{|l|l|ccc|ccc|ccc|}
\toprule
\multirow{2}{*}{\textbf{Type}} & \multirow{2}{*}{\textbf{Method}} & \multicolumn{3}{c|}{\textbf{Clothing}} & \multicolumn{3}{c|}{\textbf{Electronics}} & \multicolumn{3}{c|}{\textbf{Home}} \\ 
 &  & \textbf{MAP} & \textbf{N@3} & \textbf{N@5} & \textbf{MAP} & \textbf{N@3} & \textbf{N@5} & \textbf{MAP} & \textbf{N@3} & \textbf{N@5} \\
\midrule
\multirow{4}{*}{Text-only} & BiMPM & 57.7 & 41.8 & 46.0 & 52.3 & 40.5 & 44.1 & 56.6 & 43.6 & 47.6 \\
 & EG-CNN & 56.4 & 40.6 & 44.7 & 51.5 & 39.4 & 42.1 & 55.3 & 42.4 & 46.7 \\
 & Conv-KNRM & 57.2 & 41.2 & 45.6 & 52.6 & 40.5 & 44.2 & 57.4 & 44.5 & 48.4 \\
 & PRH-Net & 58.3 & 42.2 & 46.5 & 52.4 & 40.1 & 43.9 & 57.1 & 44.3 & 48.1 \\
\midrule
\multirow{4}{*}{Multimodal} & SSE-Cross & 65.0 & 56.0 & 59.1 & 53.7 & 43.8 & 47.2 & 60.8 & 51.0 & 54.0 \\
 & DR-Net & 65.2 & 56.1 & 59.2 & 53.9 & 44.2 & 47.5 & 61.2 & 51.8 & 54.6 \\
 & MCR & 66.4 & 57.3 & 60.2 & 54.4 & 45.0 & 48.1 & 62.6 & 53.5 & 56.6 \\
 & \textbf{Our Model} & \textbf{67.4} & \textbf{58.6} & \textbf{61.6} & \textbf{56.5} & \textbf{47.6} & \textbf{50.8} & \textbf{63.5} & \textbf{54.6} & \textbf{57.8} \\
\bottomrule
\end{tabular} }
\caption{
Helpfulness Prediction results on Amazon-MRHP dataset.}
\label{table:amazon_results}
\end{table*}
\subsection{Datasets}
We evaluate our methods on two publicly available benchmark datasets for MRHP task: Lazada-MRHP and Amazon-MRHP.

\noindent\textbf{Lazada-MRHP} \cite{liu2021multi} consists of product items and artificial reviews on Lazada.com, an e-commerce platform in Southest Asia. All of the texts in the dataset are expressed in Indonesian.

\noindent\textbf{Amazon-MRHP} \cite{liu2021multi} is collected from Amazon.com, the large-scale international e-commerce platform. Product information and associated reviews are in English and extracted between 2016 and 2018.

Both datasets comprise 3 categories: (i) Clothing, Shoes \& Jewelry (Clothing), (ii) Electronics (Electronics), and (iii) Home \& Kitchen (Home). We present the statistics of them in Table \ref{table:datasets}.
\subsection{Implementation Details}
We use a 1-layer LSTM with hidden dimension size of 128. We initialize our word embedding with fastText embedding \cite{bojanowski2017enriching} for Lazada-MRHP dataset and 300-dimensional GloVe pretrained word vectors \cite{pennington2014glove} for Amazon-MRHP dataset. We set our multimodal attention module to have $D = 5$ attention layers. For the visual modality, we extract 2048-dimensional ROI features from each image and encode them into 128-dimensional vectors. Our entire model is trained end-to-end with Adam optimizer \cite{kingma2014adam} and batch size of 32. For the training objective, we set the value of the margin in the ranking loss to be 1.

\subsection{Baselines}
We compare our proposed architecture against the following baselines:
\begin{itemize}
    \item \textbf{BiMPM} \cite{wang2017bilateral}: a ranking model which encodes input sentences in two directions to ascertain the matching result.
    \item \textbf{Conv-KNRM} \cite{dai2018convolutional}: a CNN-based model which encodes n-gram of multiple lengths and uses kernel pooling to generate the final ranking score.
    \item \textbf{EG-CNN} \cite{chen2018cross}: a CNN-based model targeting data scarcity and OOV problem in RHP task via taking advantage of character-based representations and domain discriminators.
    \item \textbf{PRH-Net} \cite{fan2019product}: a baseline to predict helpfulness of a review by taking into consideration both product text and product metadata.
    \item \textbf{DR-Net} \cite{xu2020reasoning}: a cross-modality approach that models contrast in associated contexts by leveraging decomposition and relation modules.
    \item \textbf{SSE-Cross} \cite{abavisani2020multimodal}: multimodal model to fuse different modalities with stochastic shared embeddings.
    \item \textbf{MCR} \cite{liu2021multi}: a baseline model focusing on coherent reasoning.
\end{itemize}

\begin{table*}[ht]
\centering
\resizebox{\textwidth}{!}{
\begin{tabular}{|l|ccc|ccc|ccc|}
\toprule
\multirow{2}{*}{\textbf{Dataset}} & \multicolumn{3}{c|}{\textbf{Clothing}} & \multicolumn{3}{c|}{\textbf{Electronics}} & \multicolumn{3}{c|}{\textbf{Home}} \\ 
 & \textbf{MAP} & \textbf{N@3} & \textbf{N@5} & \textbf{MAP} & \textbf{N@3} & \textbf{N@5} & \textbf{MAP} & \textbf{N@3} & \textbf{N@5} \\
\midrule
 Lazada & $4.48 \cdot 10^{-2}$ & $1.55 \cdot 10^{-2}$ & $3.93 \cdot 10^{-2}$ & $4.54 \cdot 10^{-3}$ & $1.05 \cdot 10^{-4}$ & $2.63 \cdot 10^{-3}$ & $1.09 \cdot 10^{-3}$ & $3.40 \cdot 10^{-2}$ & $3.68 \cdot 10^{-3}$ \\
 Amazon & $3.45 \cdot 10^{-2}$ & $4.22 \cdot 10^{-2}$ & $1.86 \cdot 10^{-2}$ & $4.37 \cdot 10^{-3}$ & $2.81 \cdot 10^{-2}$ & $3.04 \cdot 10^{-2}$ & $2.04 \cdot 10^{-3}$ & $3.30 \cdot 10^{-3}$ & $6.50 \cdot 10^{-3}$ \\
\bottomrule
\end{tabular} }
\caption{
Significance test of the results of our model against MCR model. }
\label{table:sig_tests}
\end{table*}

\subsection{Automatic Evaluation}
In Table \ref{table:lazada_results} and \ref{table:amazon_results}, we follow previous work \cite{liu2021multi} to report Mean Average Precision (MAP), Normalized Discounted Cumulative Gain (NDCG@N) \cite{jarvelin2017ir} where $N = 3$ and $N = 5$. As it can be seen, multimodal approaches achieve better performance than text-only ones.

For Lazada-MRHP dataset, we achieve an absolute improvement of NDCG@3 of 2.4 points in Clothing, NDCG@5 of $1.5$ points in Electronics, and MAP of $1.4$ points in Home over the previous best method, which is MCR. In addition, our model also obtains better results than the best text-only RHP model, which is PRH-Net, with a gain of NDCG@3 of $9.8$ points in Clothing, NDCG@5 of $4.3$ points in Electronics, and MAP of $3.6$ points in Home. Those results prove that our method can produce reasonable rankings for associated reviews.

For Amazon dataset, which is written in English, our model outperforms MCR on all 3 categories, by NDCG@5 of $1.4$ points in Clothing, $2.7$ points in Electronics, and $1.2$ points in Home, respectively. These results have verified that our interaction module and optimization approach can come up with more useful multimodal fusion than previous state-of-the-art baselines, not only in English context but other language one as well. 

We also perform significance tests to evaluate the statistical significance of our improvement on two datasets Amazon-MRHP and Lazada-MRHP, and note p-values in Table \ref{table:sig_tests}. As shown in the table, all of the p-values are smaller than $0.05$, verifying the statistical significance in the enhancement of our method against prior best MRHP model, MCR \cite{liu2021multi}.

\subsection{Case Study}
In Table \ref{table:example}, we introduce an example of one product item and two reviews extracted from Electronics category of Amazon-MRHP dataset. Whereas MCR fails to predict relevant helpfulness scores, our model successfully produces sensible rankings for both of them. We hypothesize that our Multimodal Interaction module learns more meaningful representations and Adaptive Contrastive Learning framework acquires more logical hidden states of relations among input elements. Thus, our model is able to generate more rational outcomes.

\subsection{Ablation Study}
In this section, we proceed to study the impact of (1) Adaptive Contrastive Learning framework and (2) Cross-modal Interaction module.

\noindent\textbf{Adaptive Contrastive Learning} It is worth noting from Table \ref{table:ablation} that plainly integrating contrastive learning brings less enhancement to the performance, with the improvement of NDCG@3 dropping $0.53$ points in Lazada-MRHP dataset, NDCG@5 waning $0.84$ points in Amazon-MRHP dataset. Furthermore, completely removing contrastive objective hurts performance, as NDCG@3 score decreasing $0.77$ points in Lazada-MRHP, and MAP score declining $1.06$ points in Amazon-MRHP. We hypothesize that the model loses the ability to learn efficient representations for cross-modal relations.

\noindent\textbf{Cross-modal Interaction} In this ablation, we eliminate the cross-modal interaction module. As shown in Table \ref{table:ablation}, without the module, the improvement is downgraded, for instance, N@3 drops $1.89$ points in Lazada-MRHP dataset, MAP shrinks $1.39$ points in Amazon-MRHP dataset. It is hypothesized that without the module, the model is rigidly dependent upon the alignment nature among multimodal input elements, which brings about insensible modeling because in most cases, cross-modal elements are irrelevant to be bijectively mapped together.

\begin{table}[ht]
\centering
\resizebox{\linewidth}{!}{
\begin{tabular}{|l|l|ccc|}
\toprule
\textbf{Dataset} & \textbf{Model} & \textbf{MAP} & \textbf{N@3} & \textbf{N@5} \\ 
\midrule
 \multirow{4}{*}{Lazada} & Our Model & \textbf{78.15} & \textbf{72.43} & \textbf{76.49}  \\
  & - w/o Adaptive Weighting & 77.90 & 71.90 & 75.97 \\
  & - w/o Contrastive Objective & 77.69 & 71.66 & 75.85 \\
  & - w/o Cross-modal Module & 77.32 & 70.54 & 74.86 \\
\midrule
 \multirow{4}{*}{Amazon} & Our Model & \textbf{56.49} & \textbf{47.62} & \textbf{50.79}  \\
  & - w/o Adaptive Weighting & 56.03 & 46.98 & 49.95 \\
  & - w/o Contrastive Objective & 55.43 & 46.30 & 49.02 \\
  & - w/o Cross-modal Module & 55.10 & 45.67 & 48.50 \\
 \bottomrule
\end{tabular} }
\caption{
Ablation study in Electronics category of Lazada-MRHP and Amazon-MRHP datasets.}
\label{table:ablation}
\vspace{-10pt}
\end{table}
\subsection{Impact of Contrastive Learning on Cross-modal Relations}
\begin{table*}[t]
\centering
\resizebox{\linewidth}{!}{
\begin{tabular}{|l|l|cc|cc|cc|}
\toprule
\multirow{2}{*}{\textbf{Label}} & \multirow{2}{*}{\textbf{Model}} & \multicolumn{2}{c|}{\textbf{Intra-modal}} & \multicolumn{2}{c|}{\textbf{Inter-modal}} & \multicolumn{2}{c|}{\textbf{Intra-review}} \\ 
 & & \textbf{CS} & \textbf{L2} & \textbf{CS} & \textbf{L2} & \textbf{CS} & \textbf{L2} \\
\midrule
\multirow{2}{*}{1} & MCR & 0.785 $\pm$ 0.002 & 3.852 $\pm$ 0.067 & 0.843 $\pm$ 0.002 & 11.719 $\pm$ 0.001 & 0.845 $\pm$ 0.002 & 14.631 $\pm$ 0.001 \\
 & Our Model & 0.875 $\pm$ 0.002 & 6.545 $\pm$ 0.007 & 0.957 $\pm$ 0.002 & 13.934 $\pm$ 0.027 & 0.953 $\pm$ 0.002 & 15.160 $\pm$ 0.036  \\
 \midrule
 \multirow{2}{*}{4} & MCR & 0.533 $\pm$ 0.004 & 1.014 $\pm$ 0.051 & 0.712 $\pm$ 0.010 & 9.476 $\pm$ 0.001 & 0.617 $\pm$ 0.001 & 8.519 $\pm$ 0.001 \\
 & Our Model & 0.433 $\pm$ 0.001 & 0.981 $\pm$ 0.005 & 0.564 $\pm$ 0.001 & 4.179 $\pm$ 0.017 & 0.538 $\pm$ 0.001 & 3.827 $\pm$ 0.020  \\
\bottomrule
\end{tabular} }
\caption{Intra-modal, Inter-modal, and Intra-review distances in Home category of Lazada-MRHP dataset.}
\label{table:lazada_cs_mse_dist}
\end{table*}

\begin{table*}[t]
\centering
\resizebox{\linewidth}{!}{
\begin{tabular}{|l|l|cc|cc|cc|}
\toprule
\multirow{2}{*}{\textbf{Label}} & \multirow{2}{*}{\textbf{Model}} & \multicolumn{2}{c|}{\textbf{Intra-modal}} & \multicolumn{2}{c|}{\textbf{Inter-modal}} & \multicolumn{2}{c|}{\textbf{Intra-review}} \\ 
 & & \textbf{CS} & \textbf{L2} & \textbf{CS} & \textbf{L2} & \textbf{CS} & \textbf{L2} \\
\midrule
\multirow{2}{*}{1} & MCR & 0.785 $\pm$ 0.006 & 8.532 $\pm$ 0.292 & 0.686 $\pm$ 0.001 & 9.696 $\pm$ 0.300 & 0.880 $\pm$ 0.002 & 9.620 $\pm$ 0.217 \\
 & Our Model & 0.971 $\pm$ 0.001 & 10.663 $\pm$ 0.770 & 0.976 $\pm$ 0.001 & 13.234 $\pm$ 0.493 & 0.970 $\pm$ 0.001 & 12.222 $\pm$ 0.431 \\
 \midrule
 \multirow{2}{*}{4} & MCR & 0.697 $\pm$ 0.009 & 3.045 $\pm$ 0.139 & 0.624 $\pm$ 0.001 & 3.179 $\pm$ 0.830 & 0.781 $\pm$ 0.001	& 5.098 $\pm$ 0.636 \\
 & Our Model & 0.571 +- 0.001 & 1.572 +- 0.037 & 0.488 +- 0.001 & 1.460 +- 0.008 & 0.487 +- 0.001 & 3.555 +- 0.001 \\
\bottomrule
\end{tabular} }
\caption{Intra-modal, Inter-modal, and Intra-review distances in Home category of Amazon-MRHP dataset.}
\label{table:amazon_cs_mse_dist}
\end{table*}

Despite improved performances, it remains a quandary that whether the enhancement stems from more meaningful representations of input samples, which we hypothesize as a significant benefit of our contrastive learning framework. For deeper investigation, we decide to statistically measure distances among input samples using standard distance functions. Table \ref{table:lazada_cs_mse_dist} and \ref{table:amazon_cs_mse_dist} reveal the results of our experiment. In particular, we estimate the cosine distance (CS) and L2 distance (L2) between tokens of (1) product text - review text and product image - review image (intra-modal), (2) product text - review image and product image - review text (inter-modal), and (3) product text - product image and review text - review image (intra-review), then calculate the mean value of all samples. As it can be seen, our frameworks are more efficient in attracting elements of helpful pairs and repelling those of unhelpful pairs.
\section{Related Work}
\subsection{Review Helpfulness Prediction}
Past works that pursue Review Helpfulness Prediction (RHP) dilemma follow text-only approaches. In general, they extract salient information, for instance lexical \cite{krishnamoorthy2015linguistic}, argument \cite{liu2017using}, and emotional features \cite{martin2014prediction} from reviews. Subsequently, these features are fed to a standard classifier such as Random Forest \cite{louppe2014understanding} in order to produce the output score. Inspired by the meteoric development of computation resources, contemporary approaches seek to take advantage of deep learning techniques to tackle the RHP problem. For instance, \citet{wang2017bilateral} propose multi-perspective matching between review and product information via applying attention mechanism. Furthermore, \citet{chen2018cross, dai2018convolutional} adapt CNN models to learn textual representations in various views. 

In reality, review content are not only determined by texts but also other modalities. As a consequence, \citet{fan2019product} integrate metadata information of the target product into the prediction model. \citet{abavisani2020multimodal} filter out uninformative signals before fusing various modalities. Moreover, \citet{liu2021multi} perform coherent reasoning to ascertain the matching level between product and numerous review items. 

\subsection{Contrastive Estimation}
Different from architectural techniques such as Knowledge Distillation \cite{hinton2015distilling, hahn2019self, nguyen2022improving} or Variational AutoEncoder \cite{zhao2020neural, nguyen2021enriching, nguyen2021contrastive, wang2019topic}, Contrastive Learning has been introduced as a representation-based but universal mechanism to enhance natural language processing performance. Proposed by \citet{chopra2005learning}, Contrastive Learning has been widely adopted in myriad problems of Natural Language Processing (NLP). 

As an approach to polish text representations, \citet{gao2021simcse, zhang2021pairwise, liu2021dialoguecse, nguyen2021contrastive} employ contrastive loss to advance sentence embeddings and topic representations. For downstream tasks, \citet{cao2021cliff} propose negative sampling strategies to generate noisy output so that the model can learn to distinguish correct summaries from incorrect ones in Document Summarization. For Spoken Question Answering (SQA), \citet{you2021self} introduce augmentation algorithms in their contrastive learning stage so as to capture noisy-invariant representations of utterances. Additionally, \citet{ke2021classic} inherit the formulation of the contrastive objective to construct distillation loss which transfers knowledge of the previous task to the current one. Their proposals are to improve tasks in the Aspect Sentiment Classification domain. Unfortunately, despite the surge of interest in exercising contrastive learning for NLP, research works to adapt the method to the MRHP task have been scant.
\section{Conclusion}
In this paper, we propose methods to polish representation learning for the Multimodal Review Helpfulness Prediction task.  In particular, we aim to advance cross-modal relation representations by learning mutual information through contrastive learning. In order to further enhance our framework, we propose an adaptive weighting strategy to encourage flexibility in optimization. Moreover, we integrate a cross-modal interaction module to loose the model’s reliance on unalignment nature among modalities, continuing to refine multimodal representations. Our framework is able to outperform prior baselines and achieve state-of-the-art results on the MRHP problem. 
\section{Limitations}
Despite the novelty and benefits of our method for Multimodal Review Helpfulness Prediction (MRHP) problem, it does include some drawbacks. Firstly, even though empirical results demonstrate that our approach not only works in English contexts, we have not conducted the verification in multilingual circumstances, in which product or review texts are written in different languages. If a model is corroborated to work efficiaciously in such contexts, it is capable of providing myriad benefits for practical implementation, for example, e-commerce applications can leverage such one single model for multiple cross-lingual scenarios. Furthermore, our work can also be extended to other domains. For instance, in movie assessment, we need to determine whether the review suits the material in the film, or visual scenes in the comment are consistent with the textual content. These would form our prospective future directions.

Secondly, in the MRHP problem, there are several relationships that contrastive learning could exploit to burnish the performance. In particular, performing contrastive discrimination between two sets of reviews is able to furnish the model with useful set-based representations, which consolidate general knowledge for better helpfulness prediction. Similar insights are applicable for two sets of product information. At the moment, we leave such promising perspectives for future work.
\section{Acknowledgement}
This work was supported by Alibaba Innovative Research (AIR) programme with research grant AN-GC-2021-005.



% Entries for the entire Anthology, followed by custom entries
\bibliography{emnlp2022}
\bibliographystyle{acl_natbib}

\appendix
\newpage
\appendix

\section{Supplemental Tables}

%\section{Hyperparameters of Other Bandit Algorithms}
%\label{sec:bandit_hyperparams}
%Table~\ref{tab:hyperparams} lists the hyperparameters for bandit algorithms other than dBE.

\newcommand\topmidheader[2]{\multicolumn{#1}{c}{\textbf{#2}}\\%
                \addlinespace[1ex]}

\newcommand{\midheader}[2]{%
        \midrule\topmidheader{#1}{#2}}

\newcommand{\specialcell}[3][c]{% 
        \begin{tabular}[#1]{@{}#2@{}}#3\end{tabular}}%

\aptLtoX[graphic=no,type=env]{\begin{table}[htb]
  \centering
  \caption{Hyperparameters of bandit algorithms}
  \label{tab:hyperparams}
  \begin{tabular}{llc}
    \toprule
    Sign & Description & Value \\
    \multicolumn{3}{c}{\textbf{UCB1}}\\
    $c$ & Parameter to control the confidence level used in $\sqrt{c \cdot {\log{t}}/{N_t(arm)}}$ & 0.5  \\
    \multicolumn{3}{c}{\textbf{Thompson Sampling}}\\
    $p(\theta)$ & Prior Distribution & $\mathcal{B}(1, 1)$ \\
    \multicolumn{3}{c}{\textbf{discounted Thompson Sampling}}\\
    $\gamma$ & Discount factor & $1-10^{-8}$ \\
    \multicolumn{3}{c}{\textbf{discounted Thompson Samplingadaptive shrinking Thompson Sampling}}\\
    $M$ & Parameter to control memory usage in a data structure ADWIN2 \cite{ADWIN} & 10 \\
    $\delta$ & Parameter to control the confidence level in a data structure ADWIN2 & $1-10^{-7}$ \\
    \multicolumn{3}{c}{\textbf{EXP-IX}}\\
    $\eta_t$ & Parameter used for weights of arms & $\sqrt{\frac{2 \cdot \log{K}}{K \cdot t}}$ \\
    \addlinespace[1ex]
    $\gamma_t$ & Parameter used for loss estimates & $\frac{\eta_t}{2}$ \\
    \multicolumn{3}{c}{\textbf{EXP3++}}\\
    $\alpha$ & Constant used in calculating $\xi_t(a)$ & $3$ \\
    $\beta$ & Constant used in calculating $\xi_t(a)$ & $256$ \\
    \bottomrule
  \end{tabular}
\end{table}}{\begin{table}[htb]
  \centering
  \caption{Hyperparameters of bandit algorithms}
  \label{tab:hyperparams}
  \begin{tabular}{llc}
    \toprule
    Sign & Description & Value \\
    \midheader{3}{UCB1}
    $c$ & \specialcell{l}{Parameter to control the confidence \\ level used in $\sqrt{c \cdot {\log{t}}/{N_t(arm)}}$} & 0.5  \\
    \midheader{3}{Thompson Sampling}
    $p(\theta)$ & Prior Distribution & $\mathcal{B}(1, 1)$ \\
    \midheader{3}{discounted Thompson Sampling}
    $\gamma$ & Discount factor & $1-10^{-8}$ \\
    \midheader{3}{adaptive shrinking Thompson Sampling}
    $M$ & \specialcell{l}{Parameter to control memory usage \\ in a data structure ADWIN2 \cite{ADWIN}} & 10 \\
    $\delta$ & \specialcell{l}{ Parameter to control the confidence \\ level in a data structure ADWIN2} & $1-10^{-7}$ \\
    \midheader{3}{EXP-IX}
    $\eta_t$ & Parameter used for weights of arms & $\sqrt{\frac{2 \cdot \log{K}}{K \cdot t}}$ \\
    \addlinespace[1ex]
    $\gamma_t$ & Parameter used for loss estimates & $\frac{\eta_t}{2}$ \\
    \midheader{3}{EXP3++}
    $\alpha$ & Constant used in calculating $\xi_t(a)$ & $3$ \\
    $\beta$ & Constant used in calculating $\xi_t(a)$ & $256$ \\
    \bottomrule
  \end{tabular}
\end{table}}

\begin{table}[htb]
  \centering
  \caption{Commit IDs of the PUTs used in our vulnerability discovery and AFL++ used as the baseline.}
  \begin{tabular}{lc}
    \toprule
    Program & Commit \\
    \midrule

    AFL++ & 32a0d6ac315 (ver ++3.14c) \\
    Bloaty &  60209eb \\
    HarfBuzz & 77eeec5 \\
    libarchive & 86c9361 \\
       libxml2 & dea91c9 \\
    MuPDF & ef3d68d \\
   PHP & fdf0455f \\
    Poppler & 6d72d82 \\
    PROJ & 76dfefe \\
    QPDF &  3794f8e \\
    libtpm2 & bc3bb26 \\
    Wireshark  & 1fc621e \\
    Xpdf & N/A (ver 4.03) \\

    \bottomrule
  \end{tabular}
\label{tab:commit-ids}
\end{table}


\begin{table}[htb]
  \centering
  \caption{Initial and theoretical maximum values of code coverage of the PUTs in FuzzBench. 
           Initial values were investigated only in the PUTs used.}
  \begin{tabular}{lcc}
    \toprule
    PUT & Initial & Maximum \\
    \midrule

bloaty\_fuzz\_target & N/A & 83114 \\
curl\_curl\_fuzzer\_http & N/A & 78362 \\
freetype2-2017 & 1517 & 26262 \\
harfbuzz-1.3.2 & N/A & 12212 \\
jsoncpp\_jsoncpp\_fuzzer & N/A & 2114 \\
lcms-2017-03-21 & 149 & 7036 \\
libjpeg-turbo-07-2017 & N/A & 9384 \\
libpcap\_fuzz\_both & 2 & 7294 \\
libpng-1.2.56 & 138 & 3736 \\
libxml2-v2.9.2 & 258 & 67994 \\
libxslt\_xpath & N/A & 51456 \\
mbedtls\_fuzz\_dtlsclient & N/A & 12888 \\
openssl\_x509 & 6026 & 54116 \\
openthread-2019-12-23 & N/A & 19846 \\
php\_php-fuzz-parser & N/A & 215210 \\
proj4-2017-08-14 & 46 & 6534 \\
re2-2014-12-09 & 1 & 3982 \\
sqlite3\_ossfuzz & 4767 & 28766 \\
systemd\_fuzz-link-parser & N/A & 1798 \\
vorbis-2017-12-11 & 410 & 4082 \\
woff2-2016-05-06 & N/A & 5708 \\
zlib\_zlib\_uncompress\_fuzzer & N/A & 910 \\

    \bottomrule
  \end{tabular}
\label{tab:fuzzbench_max_cov}
\end{table}

\begin{table}[htb]
\centering
\caption{List of unique bugs found in the 7-day trial (manually triaged).}
\begin{minipage}{\columnwidth}

\centering
\begin{tabular}{lll}
\toprule

ID & PUT & Bug Type \\
\midrule
Bug-A & bloaty & NULL Pointer Deref \\
Bug-B & harfbuzz & Out-of-bounds Read \\
Bug-C & mupdf & Assertion Fail \\
Bug-D & mupdf & NULL pointer deref \\
Bug-E & xpdf & Stack Overflow \\
Bug-F & xpdf & NULL Pointer Deref \\
Bug-G \footnote{CVE-2022-24106 is issued.} & xpdf & Use of Uninitialized Value \\
Bug-H \footnote{CVE-2022-24107 is issued.} & xpdf & Integer Overflow \\
Bug-I & php & Use-After-Free \\
Bug-J & php & Use-After-Free \\
Bug-K & php & NULL Pointer Deref \\
Bug-L & php & Use-After-Free \\ 
Bug-M & php & NULL Pointer Deref \\
Bug-N & php & Assertion Fail \\
Bug-O & php & Use-After-Free \\
Bug-P & php & Use-After-Free \\
Bug-Q \footnote{CVE-2022-23308 is issued.} & libxml2 & Use-After-Free \\
\bottomrule
\end{tabular}

\label{tab:7d-bug}
\end{minipage}
\end{table}

\begin{table*}[htb]
  \centering
  \caption{List of the PUTs used in Section~\ref{sec:banditcomparison}. If the source code of a PUT was maintained in Git, the latest version at the time of the experiment in the master (or main) branch was used for the build. The `+' sign in a version indicates that the used source code is not the official release version of the source code.}
  \renewcommand\tabularxcolumn[1]{m{#1}}
  \renewcommand{\arraystretch}{1.2}
  \begin{tabularx}{\textwidth}{lXllXc}
    \toprule
    Project & Version & Commit ID & PUT & Format of Initial Seeds & Initial Edge Coverage \\
    \midrule
    Bloaty & v1.1+ & 60209eb & fuzz\_target & Executable (e.g., ELF, PE, Mach-O) & 4773\\
    libmpeg2 & N/A & 5432dc1 & mpeg2\_dec\_fuzzer & MPEG2 & 2428 \\
    PHP & 8.0+ & fdf0455f & php-fuzz-execute & PHP source code & 25241 \\
    HarfBuzz & 3.1.0 & 77eeec5 & hb-shape-fuzzer & Font (e.g., TrueType, OpenType) & 15298 \\
    Xpdf & 4.03 & N/A & fuzz\_pdfload & PDF & 4755 \\
    libtpm2 & N/A & bc3bb26 & tpm2\_execute\_command\_fuzzer & TPM command & 3884\\
    libyaml & v0.2.5+ & f8f760f & libyaml\_dumper\_fuzzer & YAML & 1310 \\
    libzip & 1.8.0+ & bff2eb9 & zip\_read\_fuzzer & ZIP & 805 \\
    libgit2 & v1.3.0+ & 50b4d53 & download\_refs\_fuzzer & Git packet & 3911 \\
    file & 5.41+ & fcbb5d8 & magic\_fuzzer & any (e.g., Zstd compressed file) & 1171 \\
%    MuPDF & 1.19.0+ & ef3d68d & pdf\_fuzzer & PDF & 16936 \\
%    libxml2 & 2.9.12+ & dea91c9 & xml & XML & 7027 \\
    \bottomrule
  \end{tabularx}
\label{tab:put_details}
\end{table*}

%\section{Full Results of Some Experiments}
%\label{sec:full_result}

%Table~\ref{tab:alg_cmp_all}, Figure \ref{fig:vis_bandits} and Figure \ref{fig:full_ablation_time_vs_cov} show the omitted results.

\begin{table*}[htb]
\centering
\caption{Median edge coverage obtained by AFL++ and 8 versions of \OurMethodName-AFL++ in 10 PUTs after 24 h. }

\begin{tabular}{lccccccccc}
\toprule

PUT & AFL++ & UCB1 & KLUCB & TS & dTS & dBE & ADS-TS & EXP3-IX & EXP3++ \\
\midrule

bloaty & \textit{1845.5} & 2198.5 & 2246.0 & 2232.5 & 2191.0 & 2292.0 & \textbf{2340.0} & 2181.5 & 2231.5 \\
harfbuzz & \textit{13497.5} & 14031.5 & 14247.5 & 14360.5 & \textbf{14374.0} & 14067.5 & 14149.0 & 13883.0 & 13891.0 \\
xpdf & \textit{3384.0} & 3494.0 & 3812.5 & \textbf{4618.5} & 4166.5 & 3791.5 & 3902.0 & 3860.0 & 3615.0 \\
libzip & \textit{267.5} & 272.0 & 274.0 & 268.0 & 268.5 & 271.5 & \textbf{276.0} & 271.5 & 268.0 \\
libgit2 & 898.0 & 888.5 & 890.5 & 906.5 & \textbf{916.0} & 884.0 & 914.0 & 899.5 & \textit{881.0} \\
php & \textit{9841.5} & 11861.0 & 13551.5 & \textbf{14324.0} & 14187.5 & 12657.5 & 13408.0 & 11423.5 & 11828.5 \\
libmpeg2 & \textit{1873.5} & 1900.5 & 1905.0 & 1905.5 & \textbf{1906.5} & 1903.0 & \textbf{1906.5} & 1897.0 & 1902.0 \\
tpm2 & \textit{281.5} & 299.5 & 313.0 & 317.0 & \textbf{317.5} & 305.0 & 311.0 & 298.5 & 291.0 \\
libyaml & 2811.5 & 2841.0 & \textbf{2841.5} & \textit{2800.5} & 2837.0 & 2827.5 & 2831.5 & 2828.0 & 2834.5 \\
file & 830.5 & 829.5 & 828.0 & 827.0 & 827.5 & 833.5 & \textbf{840.5} & 826.5 & \textit{826.0} \\

\bottomrule

\end{tabular}

\label{tab:alg_cmp_all}
\end{table*}

\begin{table*}[htb]
\centering
\caption{P-value of Mann-Whitney's U test (Holm-Bonferroni corrected) and Vargha-Delaney's $\hat{A}_{12}$ between AFL++ and the fuzzer in the column for the evaluation conducted in Section~\ref{subsec:eval-vs-existing}. If the p-value is bold, the difference is significant in the test ($p < 0.01$). The characters `L', `M', `S' and `N' in parentheses indicate that the effect size is large, medium, small, and none, respectively, according to \cite{A12}. The `+' sign means the fuzzer in the column is superior to AFL++ when compared by rank sum as well as $\hat{A}_{12}$, and the `-' sign means the opposite.}
\begin{tabular}{lllllllllllll}
 \toprule

  & \multicolumn{2}{c}{MOpt} & \multicolumn{2}{c}{CMFuzz} & \multicolumn{2}{c}{Karamcheti} & \multicolumn{2}{c}{\HavocMAB{}} & \multicolumn{2}{c}{SLOPT} \\
  \cmidrule(r){2-3}\cmidrule(r){4-5}\cmidrule(r){6-7} \cmidrule(r){8-9} \cmidrule(r){10-11}
  PUT & $p$ & $\hat{A}_{12}$ & $p$ & $\hat{A}_{12}$ & $p$ & $\hat{A}_{12}$ & $p$ & $\hat{A}_{12}$ & $p$ & $\hat{A}_{12}$ \\
\midrule

openssl\_x509 & \textbf{ < 0.001 } & 0.82 (+L) & \textbf{ 0.023 } & 0.71 (+L) & \textbf{ < 0.001 } & 0.92 (+L) & \textbf{ < 0.001 } & 0.82 (+L) & \textbf{ < 0.001 } & 0.91 (+L) \\
re2-2014-12-09 & \textbf{ < 0.001 } & 0.18 (-L) & > 0.1 & 0.37 (-S) & > 0.1 & 0.38 (-S) & > 0.1 & 0.47 (-N) & > 0.1 & 0.52 (+N) \\
proj4-2017-08-14 & \textbf{ < 0.001 } & 0.08 (-L) & \textbf{ < 0.001 } & 0.86 (+L) & \textbf{ < 0.001 } & 0.99 (+L) & > 0.1 & 0.54 (+N) & \textbf{ < 0.001 } & 0.92 (+L) \\
sqlite3\_ossfuzz & > 0.1 & 0.55 (+N) & \textbf{ < 0.001 } & 0.85 (+L) & \textbf{ < 0.001 } & 0.93 (+L) & 0.1 & 0.68 (+M) & \textbf{ < 0.001 } & 1.00 (+L) \\
libxml2-v2.9.2 & \textbf{ < 0.001 } & 0.08 (-L) & \textbf{ < 0.001 } & 0.93 (+L) & \textbf{ < 0.001 } & 0.98 (+L) & \textbf{ < 0.001 } & 0.97 (+L) & \textbf{ < 0.001 } & 0.84 (+L) \\
freetype2-2017 & \textbf{ < 0.001 } & 0.08 (-L) & 0.094 & 0.33 (-M) & > 0.1 & 0.54 (+N) & > 0.1 & 0.52 (+N) & \textbf{ < 0.001 } & 0.79 (+L) \\
libpcap\_fuzz\_both & > 0.1 & 0.57 (+S) & \textbf{ < 0.001 } & 0.79 (+L) & \textbf{ < 0.001 } & 0.80 (+L) & \textbf{ < 0.001 } & 0.87 (+L) & \textbf{ < 0.001 } & 0.81 (+L) \\
libpng-1.2.56 & > 0.1 & 0.42 (-S) & > 0.1 & 0.36 (-M) & > 0.1 & 0.49 (-N) & > 0.1 & 0.56 (+S) & 0.049 & 0.68 (+M) \\
lcms-2017-03-21 & > 0.1 & 0.45 (-N) & \textbf{ 0.037 } & 0.70 (+M) & \textbf{ < 0.001 } & 0.85 (+L) & > 0.1 & 0.37 (-S) & \textbf{ < 0.001 } & 0.88 (+L) \\
vorbis-2017-12-11 & > 0.1 & 0.39 (-S) & > 0.1 & 0.56 (+S) & \textbf{ < 0.001 } & 0.20 (-L) & > 0.1 & 0.62 (+S) & 0.092 & 0.65 (+M) \\

\bottomrule
\end{tabular}
\label{tab:statistics}
\end{table*}

\clearpage

\section{Algorithm Overview}

\begin{algorithm}[H]

\centering
\caption{Pseudocode of \OurMethodName{}}
\label{alg:slopt}

\begin{algorithmic}[0]

\Require{\mbox{}\\
    $initial\_seeds$ -- a set of initial test cases \\
    $program$ -- a PUT to be fuzzed
}

\Ensure{\mbox{}\\
    $queue$ -- a set of valuable test cases \\
    $crashes$ -- a set of test cases that trigger crashes
}

%\begin{adjustwidth}{-9pt}{}
%\setstretch{0.85}
\vspace{5pt}

\Function{RandomMutation}{$seed, instance_{mut}, instances_{bat}$}
\State $input$ $\gets$ \Call{CopyBytesFromSeed}{$seed$}
\State $mutation$ $\gets$ \Call{SelectArm}{$instance_{mut}$}
\State $idx$ $\gets$ \Call{GetGroupIndex}{$len(input)$}
\State $batch\_size$ $\gets$ \Call{SelectArm}{$instances_{bat}[idx][mutation]$}
\For{$i$ $\gets$ $1$ \textbf{to} $batch\_size$}
    \State $pos$ $\gets$ \Call{SelectPosition}{$input$}
    \State $input$ $\gets$ \Call{ApplyOperator}{$mutation, input, pos$}
\EndFor
\State \textbf{return} $input, mutation, batch\_size$
\EndFunction

%\end{adjustwidth}

%\vspace{-6pt}

%\begin{adjustwidth}{-9pt}{}
%\setstretch{0.85}

\vspace{5pt}

\Function{MutationFuzzing}{$initial\_seeds, program$}

\State $crashes$ $\gets$ $\varnothing$
\State $queue$ $\gets$ \Call{ConstructQueue}{$initial\_seeds$}
\State $instance_{mut}$ $\gets$ \Call{CreateBanditArms}{$number\_of\_mutations$}
\For{$i$ $\gets$ $1$ \textbf{to} $5$}
 \For{$j$ $\gets$ $1$ \textbf{to} $number\_of\_mutations$}
  \State $instances_{bat}[i][j]$ $\gets$ \Call{CreateBanditInstance}{$7$}
 \EndFor
\EndFor

\State

\While{ $\neg$ \Call{UserWantsStop}{\null}}
 \State $seed$ $\gets$ \Call{SelectSeed}{$queue$}
 \State $energy$ $\gets$ \Call{DecideEnergy}{$seed$}
 \For{$i$ $\gets$ $1$ \textbf{to} $energy$}
  \State $input, mutation, batch\_size$ 
  \State $\gets$ \Call{RandomMutation}{$seed, instance_{mut}, instances_{bat}$}
  \State $result$ $\gets$ \Call{ExecutePUT}{$program, input$}
  \State $b$ $\gets$ \Call{WasInputValuable}{$result$}
  \State \Call{RewardArm}{$mutation, b$}
  \State \Call{RewardArm}{$batch\_size, b$}
  \State \Call{SaveInputIfValuable}{$queue, input, result$}
  \State \Call{SaveInputIfCrash}{$crashes, input, result$}
 \EndFor
\EndWhile
\EndFunction

%\end{adjustwidth}

\end{algorithmic}
\end{algorithm}



\end{document}

\bibliographystyle{acl_natbib}

\appendix

\section{Appendix}\label{sec:appendix}

\subsection{Dataset Distributions}\label{sec:app_class_dist}

We provide the details of class distributions for Contract-NLI and LEDGAR benchmarks in Table~\ref{tab:app_freq}. As shown in this table, LEDGAR contains considerably larger categories compared to Contract-NLI and is more balanced. 



\begin{table*}[h]
\centering
\begin{tabular}{clccc}
\hline
Dataset & Category & Train Size & Dev Size & Test Size\\
\hline
\multirow{17}{*}{\texttt{Contract-NLI}} & \small{\texttt{Confidentiality of Agreement}} & 161 & 29 & 46\\
& \small{\texttt{Explicit identification}} & 203 & 29 & 60\\
& \small{\texttt{Inclusion of verbally conveyed information}} & 274 & 45 & 76\\
& \small{\texttt{Limited use}} & 371 & 53 & 110\\
& \small{\texttt{No licensing}} & 327 & 39 & 86\\
& \small{\texttt{No reverse engineering}} & 60 & 8 & 13\\
& \small{\texttt{No solicitation}} & 93 & 11 & 28\\
& \small{\texttt{None-inclusion of non-technical information}} & 332 & 50 & 94\\
& \small{\texttt{Notice on compelled disclosure}} & 276 & 45 & 77\\
& \small{\texttt{Permissible acquirement of similar information}} & 311 & 47 & 96\\
& \small{\texttt{Permissible copy}} & 167 & 17 & 49\\
& \small{\texttt{Permissible development of similar information}} & 263 & 40 & 73\\
& \small{\texttt{Permissible post-agreement possession}} & 312 & 25 & 63\\
& \small{\texttt{Return of confidential information}} & 182 & 24 & 38\\
& \small{\texttt{Sharing with employees}} & 358 & 56 & 94\\
& \small{\texttt{Sharing with third-parties}} & 370 & 53 & 102\\
& \small{\texttt{Survival of obligations}} & 311 & 43 & 83\\
\hline

\multirow{5}{*}{\texttt{LEDGAR}} & \small{\texttt{Amendments}} & 9,132 & 1,515 & 2,615 \\
& \small{\texttt{Counterparts}} & 8,033 & 1,312 & 2,363 \\
& \small{\texttt{Entire agreements}} & 8,094 & 1,361 & 2,370 \\
& \small{\texttt{Governing laws}} & 11,926 & 1,997 & 3,454 \\
& \small{\texttt{Notices}} & 7,064 & 1,190 & 2,105 \\
\hline
\end{tabular}
\caption{\label{tab:app_freq}
Category frequency for Contract-NLI adapted to classification task, and LEDGAR benchmarks.}
\end{table*}

% \begin{table*}[h]
% \centering
% \begin{tabular}{lccc}
% \hline
% Category & Train Size & Dev Size & Test Size\\
% \hline
% \small{\texttt{Amendments}} & 9,132 & 1,515 & 2,615 \\
% \small{\texttt{Counterparts}} & 8,033 & 1,312 & 2,363 \\
% \small{\texttt{Entire agreements}} & 8,094 & 1,361 & 2,370 \\
% \small{\texttt{Governing laws}} & 11,926 & 1,997 & 3,454 \\
% \small{\texttt{Notices}} & 7,064 & 1,190 & 2,105 \\
% \hline
% \end{tabular}
% \caption{\label{tab:ledgar_freq}
% Category frequency for \textbf{LEDGAR} benchmark.}
% \end{table*}

% \begin{table*}
% \centering
% \begin{tabular}{llllll}
% \hline
% {Method} & \multicolumn{5}{c}{Acquired dataset size} \\ 
% & {20} & {30} & {40} & {50} & {60} \\
% \hline
% Standard AL & 0.3591 & 0.3138 & 0.2766 & 0.2159 & 0.1319  \\
% Ours w/o initial sampling  & 0.5551 & 0.6126 & 0.6288 & 0.5937 & 0.6037 \\
% Ours & 0.5242 & 0.5453 & 0.5611 & 0.5825 & 0.5401 \\
% \hline
% \end{tabular}
% \caption{\textbf{Contract-NLI} test f1-score for \textbf{perceptron dropout} AL.}
% \label{tab:contract_nli_tab2}
% \end{table*}

% \begin{table*}
% \centering
% \begin{tabular}{llllll}
% \hline
% {Method} & \multicolumn{5}{c}{Acquired dataset size} \\ 
% & {20} & {30} & {40} & {50} & {60} \\
% \hline
% Standard AL & 0.3678 & 0.3667 & 0.3068 & 0.1895 & 0.1694  \\
% Ours w/o initial sampling & 0.5790 & 0.6044 & 0.6377 & 0.6468 & 0.5712 \\
% Ours & 0.5322 & 0.5660 & 0.5963 & 0.6032 & 0.5912  \\
% \hline
% \end{tabular}
% \caption{\textbf{Contract-NLI} test f1-score for \textbf{hard mining} AL.}
% \label{tab:contract_nli_tab3}
% \end{table*}


% \begin{table*}
% \centering
% \begin{tabular}{llllll}
% \hline
% {Method} & \multicolumn{5}{c}{Acquired dataset size} \\ 
% & {20} & {30} & {40} & {50} & {60}  \\
% \hline
% Standard AL & 0.3769 & 0.2893 & 0.2445 & 0.2192 & 0.1580\\
% Ours w/o initial sampling & 0.5300 & 0.5727 & 0.6012 & 0.5781 & 0.5996  \\
% Ours & 0.4975 & 0.5424 & 0.5582 & 0.5723 & 0.5672 \\
% \hline
% \end{tabular}
% \caption{\textbf{Contract-NLI} test f1-score for \textbf{random acquisition}.}
% \label{tab:contract_nli_tab4}
% \end{table*}


% \begin{table*}
% \centering
% \begin{tabular}{llllll}
% \hline
% {Method} & \multicolumn{5}{c}{Acquired dataset size} \\ 
% & {20} & {30} & {40} & {50} & {60}  \\
% \hline
% Standard AL & 0.8192 & 0.8733 & 0.7804 & 0.7797 & 0.6698  \\
% Ours w/o initial sampling & 0.8362 & 0.8758 & 0.8851 & 0.9081 & 0.9073  \\
% Ours & 0.8547 & 0.8879 & 0.9131 & 0.9016 & 0.9117  \\
% \hline
% \end{tabular}
% \caption{\textbf{LEDGAR} test f1-score for \textbf{perceptron dropout} AL. }
% \label{tab:ledgar_tab2}
% \end{table*}

% \begin{table*}
% \centering
% \begin{tabular}{llllll}
% \hline
% {Method} & \multicolumn{5}{c}{Acquired dataset size} \\ 
% & {20} & {30} & {40} & {50} & {60} \\
% \hline
% Standard AL & 0.8176 & 0.8186 & 0.5843 & 0.3328 & 0.5197 \\
% Ours w/o initial sampling & 0.8478 & 0.8827 & 0.9036 & 0.9213 & 0.9304  \\
% Ours & 0.8173 & 0.8962 & 0.9189 & 0.9302 & 0.9330 \\
% \hline
% \end{tabular}
% \caption{\textbf{LEDGAR} test f1-score for \textbf{hard mining} AL.}
% \label{tab:ledgar_tab3}
% \end{table*}


% \begin{table*}
% \centering
% \begin{tabular}{llllll}
% \hline
% {Method} & \multicolumn{5}{c}{Acquired dataset size} \\ 
% & {20} & {30} & {40} & {50} & {60}  \\
% \hline
% Standard AL & 0.7968 & 0.7644 & 0.7579 & 0.7350 & 0.7934 \\
% Ours w/o Initial Sampling & 0.8525 & 0.8733 & 0.8830 & 0.9083 & 0.9092 \\
% Ours & 0.8572 & 0.8768 & 0.8880 & 0.9149 & 0.9132 \\
% \hline
% \end{tabular}
% \caption{\textbf{LEDGAR} test f1-score for \textbf{random acquisition}.}
% \label{tab:ledgar_tab4}
% \end{table*}


\subsection{Effective Fine-Tuning}\label{sec:app_tabs}

Here we present the results of standard active learning and our approach for four AL strategies discussed in Sec.~\ref{sec:setup} including Random, Hard-Mining, and Perceptron Dropout. As before, we report the average F1-score over three runs. Figure~\ref{fig:app_contract_nli} corresponds to Contract-NLI and Figure~\ref{fig:app_ledgar} illustrates the results for the LEDGAR dataset.


\begin{figure}[t]
    \centering
    \includegraphics[width=0.5\textwidth]{Contract-NLI_DROPOUT_PERCEPTRON_avg.pdf}
    \includegraphics[width=0.5\textwidth]{Contract-NLI_HARD_MINING_avg.pdf}
    \includegraphics[width=0.5\textwidth]{Contract-NLI_RANDOM_avg.pdf}
    \caption{Test F1-score for \textbf{Contract-NLI} during AL iterations. The F1-score for the fully supervised fine-tuning is $0.6990$.}
    \label{fig:app_contract_nli}
\end{figure}

\begin{figure}[t]
    \centering
    \includegraphics[width=0.5\textwidth]{LEDGAR_DROPOUT_PERCEPTRON_avg.pdf}
    \includegraphics[width=0.5\textwidth]{LEDGAR_HARD_MINING_avg.pdf}
    \includegraphics[width=0.5\textwidth]{LEDGAR_RANDOM_avg.pdf}
    \caption{Test F1-score for \textbf{LEDGAR} during AL iterations. The F1-score for the fully supervised fine-tuning is $0.9538$.}
    \label{fig:app_ledgar}
\end{figure}


\subsection{Effect of Knowledge Distillation}\label{sec:app_knowldge_distill}

Figures~\ref{fig:app_contract-nli-distill-effect-clusters} and~\ref{fig:app_ledgar-distill-effect-clusters} illustrate the comparison of the Dunn Index distribution that were not presented in the main paper. 

\begin{figure}[t]
    \centering
    \includegraphics[width=0.4\textwidth]{dunn/contract_nli/Survival_of_obligations_six-clusters-dunn-histogram.pdf}
    \includegraphics[width=0.4\textwidth]{dunn/contract_nli/Explicit_identification_four-clusters-dunn-histogram.pdf}
    \includegraphics[width=0.4\textwidth]{dunn/contract_nli/Sharing_with_employees_seven-clusters-dunn-histogram.pdf}
    \includegraphics[width=0.4\textwidth]{dunn/contract_nli/Inclusion_of_verbally_conveyed_information_five-clusters-dunn-histogram.pdf}
    \includegraphics[width=0.4\textwidth]{dunn/contract_nli/Permissible_development_of_similar_information_five-clusters-dunn-histogram.pdf}
    \caption{Comparison of the Dunn Index distribution before (TAPT RoBERTa) and after knowledge distillation (DisTAPT RoBERTa) for \textbf{Contract-NLI} dataset.}
    \label{fig:app_contract-nli-distill-effect-clusters}
\end{figure}

\begin{figure}[t]
    \centering
    \includegraphics[width=0.4\textwidth]{dunn/ledgar/amendments_ten-clusters-dunn-histogram.pdf}
    \includegraphics[width=0.4\textwidth]{dunn/ledgar/counterparts_nine-clusters-dunn-histogram.pdf}
    \includegraphics[width=0.4\textwidth]{dunn/ledgar/entire_agreements_nine-clusters-dunn-histogram.pdf}
    \includegraphics[width=0.4\textwidth]{dunn/ledgar/governing_laws_thirteen-clusters-dunn-histogram.pdf}
    \includegraphics[width=0.4\textwidth]{dunn/ledgar/notices_seven-clusters-dunn-histogram.pdf}
    \caption{Comparison of the Dunn Index distribution before (TAPT RoBERTa) and after knowledge distillation (DisTAPT RoBERTa) for \textbf{LEDGAR} dataset.}
    \label{fig:app_ledgar-distill-effect-clusters}
\end{figure}


% \begin{figure}[t]
%     \centering
%     \includegraphics[width=0.4\textwidth]{images/dunn/contract_nli/Confidentiality of Agreement_3-clusters-dunn-histogram.pdf}
%     \includegraphics[width=0.4\textwidth]{images/dunn/contract_nli/Limited use_7-clusters-dunn-histogram.pdf}
%     \includegraphics[width=0.4\textwidth]{images/dunn/contract_nli/Sharing with third-parties_7-clusters-dunn-histogram.pdf}
%     \includegraphics[width=0.4\textwidth]{images/dunn/contract_nli/None-inclusion of non-technical information_6-clusters-dunn-histogram.pdf}
%     \includegraphics[width=0.4\textwidth]{images/dunn/contract_nli/Permissible development of similar information_5-clusters-dunn-histogram.pdf}
%     \caption{Comparison of the Dunn Index distribution of the clusters obtained before and after knowledge distillation for \textbf{Contract-NLI} dataset.}
%     \label{fig:app_contract-nli-distill-effect-clusters}
% \end{figure}

\subsection{Efficiency of Initial Sampling with Medoids}\label{sec:app_medoid}

In Table~\ref{tab:app_medoid} we provide the median and $90^{th}$ percentile of number of actions performed to collect the initial labeled set, for the standard sampling approach, and our proposed strategy using cluster medoids, for nine categories of Contract-NLI that were not included in Table~\ref{tab:exp_medoid} in Sec.~\ref{sec:exp_initial-sampling}. It is observed that, for most categories, there is a considerable reduction in the number of actions performed to acquire the annotated data for the initial AL iteration. 

\begin{table*}[h]
\centering
\begin{tabular}{lccccc}
\hline
\multirow{2}{*}{Category} &  \multicolumn{2}{c}{full dataset} & \multicolumn{2}{c}{medoids} & \multirow{2}{*}{gain($\%$)}\\
 & median & $90^{th}\%$tile & median & $90^{th}\%$tile \\
\hline
\small{\texttt{Confidentiality of Agreement}} & 125.0 & 215.1 & 120.0 & 178.2 & 17.1\\
\small{\texttt{Explicit identification}} & 100.0 & 161.1 & 48.0 & 77.0 & 52.2\\
\small{\texttt{Limited use}} & 56.0 & 90.1 & 37.0 & 58.0 & 35.6\\
\small{\texttt{No solicitation}} & 227.0 & 383.0 & 178.0 & 261.0 & 31.8\\
\small{\texttt{None-inclusion of non-technical information}} & 61.0 & 101.1 & 39.0 & 64.0 & 36.7\\
\small{\texttt{Permissible acquirement of similar information}} & 65.0 & 107.0 & 91.0 & 145.0 & -35.5\\
\small{\texttt{Permissible copy}} & 121.0 & 197.0 & 68.0 & 108.0 & 45.2\\
\small{\texttt{Permissible development of similar information}} & 77.0 & 129.1 & 82.0 & 129.0 & 0.1\\
\small{\texttt{Permissible post-agreement possession}} & 66.0 & 108.1 & 41.0 & 66.0 & 38.9\\
\hline
\end{tabular}
\caption{\label{tab:app_medoid}
Number of actions to acquire the initial labeled set for $9$ categories of Contract-NLI when sampling from the full dataset (standard approach), and sampling from the cluster medoids.}
\end{table*}

\end{document}
