\documentclass[final]{siamart220329}
\newcommand{\creflastconjunction}{, and~}
\usepackage{amssymb}
\usepackage{cleveref}
\newsiamremark{remark}{Remark}
\usepackage[utf8]{inputenc}
\usepackage{amsmath}
\usepackage{amsfonts}
\usepackage{array}
\usepackage[english]{babel}
\usepackage{float}
\newcommand{\enstq}[2]{\left\{#1\mathrel{}\middle|\mathrel{}#2\right\}}
\newcommand{\norm}[1]{\left\|#1\right\|}
\newcommand{\N}{\mathbb{N}}
\newcommand{\Z}{\mathbb{Z}}
\newcommand{\D}{\mathbb{D}}
\newcommand{\R}{\mathbb{R}}
\newcommand{\duality}[2]{\left\langle #1,#2\right\rangle}
\newcommand{\inner}[2]{\left( #1,#2\right)}
\newcommand{\abs}[1]{\left\lvert #1 \right\rvert}
\newcommand{\toDo}[1]{{\color{red}#1}}
\newcommand{\Cinf}{C^{\infty}}
\newcommand{\isdef}{\mathrel{\mathop:}=}
\newcommand{\opFromTo}[3]{#1 : #2 \longrightarrow #3}
\usepackage{todonotes}
%\usepackage{subcaption}
\DeclareMathOperator{\DL}{\textup{DL}}
\DeclareMathOperator{\W}{\textup{W}}
\DeclareMathOperator{\curl}{\mathbf{curl}} 
\let\div\undefined
\DeclareMathOperator{\div}{\textup{div}} 
\makeatletter
\newsavebox{\@brx}
\newcommand{\llangle}[1][]{\savebox{\@brx}{\(\m@th{#1\langle}\)}%
	\mathopen{\copy\@brx\kern-0.5\wd\@brx\usebox{\@brx}}}
\newcommand{\rrangle}[1][]{\savebox{\@brx}{\(\m@th{#1\rangle}\)}%
	\mathclose{\copy\@brx\kern-0.5\wd\@brx\usebox{\@brx}}}
\makeatother
\newcommand{\dduality}[2]{\llangle#1\,, #2\rrangle}
\makeatletter    
\renewcommand*{\vec}[1]{\boldsymbol{#1}}
\makeatother

\headers{A stable and jump-aware projection onto discrete multi-traces}{M. Averseng}
\title{A stable and jump-aware projection onto a discrete multi-trace space.}
\author{Martin Averseng\thanks{Department of Mathematical Sciences, University of Bath, BATH BA2 7AY, UK. (\email{ma3092@bath.ac.uk}, \url{https://people.bath.ac.uk/ma3092})}}


\begin{document}
\maketitle

\begin{abstract}
	This work is concerned with boundary element methods on singular geometries, specifically, those falling in the framework of ``multi-screens" by Claeys and Hiptmair. We construct a stable quasi-interpolant which preserves piecewise linear jumps on the multi-trace space. This operator is the boundary element analog of the Scott-Zhang quasi-interpolant used in the analysis of finite-element methods. 
	More precisely, let $\Gamma$ be a multi-screen resolved by a triangulation $(\mathcal{M}_{\Gamma,h})$, and let $\mathbb{V}_h(\Gamma)$ be the space of continuous piecewise-linear multi-traces on $\Gamma$. We construct a linear operator $\Pi_h: \mathbb{H}^{1/2}(\Gamma) \to \mathbb{V}_h(\Gamma)$ with the following properties: (i) $\norm{\Pi_h u}_{\mathbb{H}^{1/2}} \leq C_h \norm{u}_{\mathbb{H}^{1/2}(\Gamma)}$ for all $u \in \mathbb{H}^{1/2}(\Gamma)$, (ii) $\Pi_h u_h = u_h$ for $u_h \in \mathbb{V}_h(\Gamma)$ and, (iii) $[\Pi_h u] = 0$ for every single trace $u \in H^{1/2}([\Gamma])$. The stability constant $C_h$ only depends on the aspect ratio of the elements of $\mathcal{M}_{\Omega,h}$, where $\mathcal{M}_{\Omega,h}$ is a tetrahedral mesh of $\Omega$ extending $\mathcal{M}_{\Gamma,h}$. We deduce uniform bounds for the stability of the discrete jump lifting, and the equivalence of the $\widetilde{H}^{1/2}$ norm with a discrete quotient norm.
\end{abstract}

\begin{keywords}
	Boundary Element Methods, Singularities, Interpolation
\end{keywords}
\begin{MSCcodes}
	 65N12, 65N38
\end{MSCcodes}


\section{Introduction}

The motivation for this work is the numerical analysis of boundary element methods for applications involving geometric singularities. Integral equations, and their resolution by the boundary element methods, are by now well developed on Lipschitz domains, see e.g.  \cite{mclean2000strongly,sauter2011bem}. In the past 30 years, a lot of effort has been dedicated to extend the range of geometries that one may tackle, both theoretically and numerically. Initially ``screens" and ``cracks" were considered, \cite{buffa2003electric,stephan1987boundary,wendland1990hypersingular} and more complex geometries were studied in recent works \cite{chandler2021boundary,claeys2013integral,claeys2016integral}. For a selection of real-life studies using such complex geometric models in various types of applications, see \cite{alad2013capacitance,chen1970transmission,fan2020high,glisson1980simple,lenti2003bem,sladek1993nonsingular,zhao2020iterative} and references therein. 

The numerical analysis of boundary element methods in such singular geometries involves at least three main challenges, compared to the case of a Lipschitz regular obstacle. First, at the theory level, one has to give a suitable definition for the function spaces in which the boundary integral operators naturally act. For instance, when the obstacle $\Gamma$ is an infinitely thin screen, the spaces $H^{1/2}(\Gamma)$ and $H^{-1/2}(\Gamma)$ are no longer dual to each other, and the weakly singular operator (resp. hypersingular operator) act in $\widetilde{H}^{-1/2}(\Gamma)$ (resp. $\widetilde{H}^{1/2}(\Gamma)$). We refer to \cite{chandler2017sobolev} for a survey on Sobolev spaces on ``rough" (possibly fractal) sets. The second challenge, which is connected to the first one, is that second-kind formulations are difficult to design for singular obstacles, and, when it comes to first-kind integral equations, the known preconditioning methods, with Calder\'{o}n preconditioning as a prominent example \cite{christiansen2002preconditioner,steinbach1998construction}, must be adapted to take the singularity into account \cite{alouges2021new,averseng2022quasi,bruno2012second,bruno2013high,hiptmair2020optimal,ramaciotti2017about}. Finally, for the a priori analysis and convergence theory, one must develop a good understanding of the singularity of the solutions of the scattering problems (see e.g. \cite{costabel2003asymptotics,grisvard2011elliptic}). This knowledge is important in order to choose the proper mesh refinement method (e.g. $h$-$p$ refinement \cite{babuvska1990hp,stephan1996hp}), and to quantify the order of convergence of the numerical solutions. 


In this work, we make a step in addressing those challenges for a class of singular geometric models called ``multi-screens" \cite{claeys2013integral,claeys2016integral}. Essentially, multi-screens are arrangements of two-dimensional surfaces in $\R^3$, which may intersect each other in complex ways. In particular, they combine two types of singularities: sharp edges (the boundary of the screen) and junction lines and points. Because of the latter, a multi-screen may fail to be a manifold at some locations. For such geometries, an adapted functional framework has been recently established by Claeys and Hiptmair \cite{claeys2013integral,claeys2016integral}, and was applied with a lot of success in the context of domain decomposition methods \cite{claeys2022robust}. In parallel, the first numerical implementation of a first-kind boundary element method for multi-screens has recently appeared in \cite{claeys2021quotient}, and ideas for corresponding preconditioners are emerging \cite{averseng2022ddm,cools2022preconditioners}. 

The aim of this paper is to construct a stable quasi-interpolant which preserves piecewise linear jumps on the multi-trace space. The properties of the operator that we construct are completely analogous to those of the celebrated Scott-Zhang quasi-interpolant of \cite{scott1990finite} (see also the recent work \cite{gawlik2021local} on a related interpolant for discrete differential forms). The Scott-Zhang operator has a tremendous importance for the numerical analysis of strongly elliptic equations in Lipschitz domains. Its main use is to analyze the approximation of functions by piecewise polynomials (see \cite[Thm 1.1]{ainsworth1997posteriori}). It can also be used to derive uniform bounds on discrete jump liftings, see e.g. \cite[Lemma 1.56]{pechstein2013finite}, which are commonly used in domain decomposition methods. The operator we construct here (restricted to piecewise linear functions, and not general polynomial order as in \cite{scott1990finite}) similarly implies uniform bounds for a discrete jump lifting on the jump space $\widetilde{H}^{1/2}([\Gamma])$. Its properties are used to analyze some new preconditioners for the boundary element methods on multi-screens in \cite{averseng2022ddm,cools2022preconditioners}. We also expect that it will be useful for the a priori analysis of the convergence of the boundary element solution to the true solution. 

The remainder of this work is organized as follows. In \Cref{sec:notationMainResult}, we introduce the necessary notation to state our main result. The construction of the quasi-interpolant involves a (primal) basis of the discrete multi-trace space, which is introduced in \Cref{sec:BasisMulti}. This allows to give the definition of the quasi-interpolant $\Pi_h$ and prove its properties in \Cref{sec:defQuasiInterp}. A central role in those proofs is played by a set of ``dual" basis functions, whose construction is presented in \Cref{sec:construction}. This construction is the main novelty of this work. 


\section{Notations and main result}
\label{sec:notationMainResult}
\paragraph{Simplices and meshes} An {\em $n$-simplex} ($n = 2$ for a triangle or $3$ for a tetrahedron) is the closed convex hull of $n+1$ affinely independent points in $\R^3$ called its vertices. A {\em face} of a simplex $S$ is a $(n-1)$-simplex spanned by $n$ vertices of $S$. 

A $n$-dimensional {\em mesh} $\mathcal{M}$ is a finite set of $n$-simplices such that if $K,K' \in \mathcal{M}$, then the intersection $K \cap K'$ is either empty, or equal to a common subsimplex (i.e. a vertex, edge, or face) of both $K$ and $K'$. The set of faces of a mesh $\mathcal{M}$, denoted by $\mathcal{F}(\mathcal{M})$, is the set of faces of the simplices of $\mathcal{M}$. The boundary $\partial \mathcal{M}$ of an $n$-dimensional mesh $\mathcal{M}$ is defined as the subset of $\mathcal{F}(\mathcal{M})$ whose elements are the face of exactly one $n$-simplex in  $\mathcal{M}$. The {\em geometry} of $\mathcal{M}$, denoted by $\abs{\mathcal{M}}$, is the union of all of its elements, i.e. 
\[\abs{\mathcal{M}} = \bigcup_{K \in \mathcal{M}} K\,.\]
Given $\gamma > 0$, we say that a mesh $\mathcal{M}$ is $\gamma$-{\em shape-regular} if it satisfies
\[\frac{h_K}{\rho_K} \geq \gamma\,, \quad \forall K \in \mathcal{M}\,,\] 
where $h_K$ is the diameter of $K$ and $\rho_K$ is the radius of the largest ball contained in $K$.  
A mesh $\mathcal{M}$ is {\em regular} if its geometry is an $n$-dimensional (piecewise linear) manifold.  

In what follows, we fix a Lipschitz polyhedron $\Omega \subset \R^3$, that is, a connected open set such that there holds $\overline{\Omega} = \abs{\mathcal{M}_{\Omega}}\,,$
for some regular tetrahedral mesh $\mathcal{M}_{\Omega}$. We also fix a set $\Gamma = \abs{\mathcal{M}_{\Gamma}}$, where $\mathcal{M}_{\Gamma}$ is a triangular mesh satisfying
\[\mathcal{M}_{\Gamma} \subset \mathcal{F}(\mathcal{M}_{\Omega}) \setminus \partial \mathcal{M}_{\Omega}\,.\]
We do not require $\mathcal{M}_{\Gamma}$ to be regular, however, we impose that $\Gamma$ be a {\em multi-screen}, in the sense of Claeys and Hiptmair \cite{claeys2013integral}. For instance, $\mathcal{M}_\Gamma$ may be as in \Cref{junction2}.
\begin{figure}[H]
	\centering
	\includegraphics[width=0.3\textwidth]{junction2}
	\caption{Possible choice of mesh $\mathcal{M}_\Gamma$.}
	\label{junction2}
\end{figure}

\paragraph{Function spaces} For an open set $U$, let $C^\infty_c(U)$ be the set of real-valued functions $u$ that are infinitely differentiable and compactly supported on $U$. We denote by $H^1(U)$ the Sobolev space of real-valued functions $u$ which are square-integrable on $U$ and such that there exists a square-integrable vector field $\vec p \in (L^2(U))^3$ satisfying
\[\int_{U} u \div \vec \phi  = - \int_{U} \vec p \cdot \vec \phi\,, \quad \forall \vec \phi \in (C^\infty_c(U))^3\,.\]
Writing $\nabla u \isdef \vec p$ the {\em weak gradient} of $u$ on $U$, a norm on $H^1(U)$ is defined by
\[\norm{u}^2_{H^1(U)} \isdef \norm{u}^2_{L^2(U)} + \norm{\nabla u}^2_{L^2(U)}\,.\]

%The Hilbert space $\textup{H}(\div,U) \subset (L^2(U))^3$ is defined similarly, interchanging the roles of the gradient and the divergence in the definition above. 
%(resp. $H_{0,\Gamma}(\div,\Omega)$)  (resp. $H(\div,\Omega)$).
% \quad \mathbb{H}^{-1/2}(\Gamma) \isdef H^1(\div,\Omega \setminus \Gamma) / H^1_{0,\Gamma}(\div,\Omega)
%  \to \mathbb{H}^{1/2}(\Gamma)$ and $\pi_N: H(\div,\Omega \setminus \Gamma) \to \mathbb{H}^{-1/2}(\Gamma)$ 

Let $H^1_{0,\Gamma}(\Omega)$ be the closure of $C^\infty_c(\Omega \setminus \Gamma)$ in $H^1(\Omega)$. The {\em multi-trace space} $\mathbb{H}^{1/2}(\Gamma)$ (see \cite{claeys2013integral}) is the Hilbert space defined by the quotient
\[\mathbb{H}^{1/2}(\Gamma) \isdef H^1(\Omega \setminus \Gamma) / H^1_{0,\Gamma}(\Omega) \,.\]
Let $\pi_D: H^1(\Omega \setminus \Gamma)$ the corresponding canonical surjection, and $H^{1/2}([\Gamma])$ the {\em single-trace space}, which is the closed subspace of $\mathbb{H}^{1/2}(\Gamma)$ defined by 
\[H^{1/2}([\Gamma]) \isdef \pi_D(H^1(\Omega))\,.\] 
In turn, the {\em jump space} $\widetilde{H}^{1/2}(\Gamma)$ is the Hilbert space defined by the quotient
\[\widetilde{H}^{1/2}(\Gamma) \isdef \mathbb{H}^{1/2}(\Gamma) / H^{1/2}([\Gamma])\,,\]
and $[\cdot]$ will denote the corresponding canonical surjection. 

\paragraph{Finite-dimensional subspaces} If $\mathcal{M}_{\Omega,h}$ and $\mathcal{M}_{\Gamma,h}$ are meshes of $\Omega$ and $\Gamma$ (possibly different from $\mathcal{M}_\Omega$ and $\mathcal{M}_\Gamma$), we say that the pair $(\mathcal{M}_{\Omega,h},\mathcal{M}_{\Gamma,h})$ is {\em trace-compatible} if $\mathcal{M}_{\Gamma,h} \subset \mathcal{F}(\mathcal{M}_{\Omega,h}) \setminus \partial \mathcal{M}_{\Omega,h}$. Given a trace-compatible pair $(\mathcal{M}_{\Omega,h},\mathcal{M}_{\Gamma,h})$, let $V_h(\Omega \setminus \Gamma)$ be the finite-dimensional subspace of $H^1(\Omega \setminus \Gamma)$ consisting of piecewise linear functions on $\mathcal{M}_\Omega$, that is,
\begin{equation*}
	V_h(\Omega \setminus \Gamma) \isdef \enstq{u \in \textup{H}^1(\Omega \setminus \Gamma)}{u_{|K} \textup{ is affine } \forall K \in \mathcal{M}_{\Omega}}\,.
\end{equation*}
Finally, define 
$$\mathbb{V}_h(\Gamma) \isdef \pi_D(V_h(\Omega \setminus \Gamma))\,, \quad \widetilde{V}_h(\Gamma) \isdef [\mathbb{V}_h(\Gamma)]\,.$$



\paragraph{Main result} The goal of this work is to prove the following result. 
\begin{theorem}
	\label{thm}
 	For each $\gamma_0 > 0$, there exists a constant $C(\gamma_0) > 0$ such that the following holds. Suppose that $(\mathcal{M}_{\Omega,h},\mathcal{M}_{\Gamma,h})$ is a trace-compatible pair of meshes of $\Omega$ and $\Gamma$, and assume that $\mathcal{M}_{\Omega,h}$ and $\mathcal{M}_{\Gamma,h}$ are $\gamma_0$-shape-regular. Then there exists a linear operator $\Pi_h: \mathbb{H}^{1/2}(\Gamma) \to \mathbb{V}_h(\Gamma)$ such that
	\begin{itemize}
		\item[(i)] $\norm{\Pi_h u}_{\mathbb{H}^{1/2}} \leq C(\gamma_0) \norm{u}_{\mathbb{H}^{1/2}}$ for all $u \in \mathbb{H}^{1/2}(\Gamma)$,
		\item[(ii)] $\Pi_h u_h = u_h$ for all $u_h \in \mathbb{V}_h(\Gamma)$, that is, $\Pi_h$ is a projection,
		\item[(iii)] $u \in H^{1/2}([\Gamma]) \implies \Pi_h u \in H^{1/2}([\Gamma])$. 
	\end{itemize}
\end{theorem}
Notice the analogy with \cite{scott1990finite}: $\mathbb{H}^{1/2}(\Gamma)$ plays the role of $H^1(\Omega)$, and jumps play the role of boundary values. The definition of $\Pi_h$ is given in \Cref{defPih}. The proof of \Cref{thm} is also inspired by \cite{scott1990finite}, but the essential difficulty is the construction of suitable ``dual basis functions". 
\begin{remark}
	The stability constant depends on the aspect ratio of the elements of the mesh $\mathcal{M}_{\Omega,h}$, and not just $\mathcal{M}_{\Gamma,h}$. Although our proof requires this condition, we conjecture that the result still holds if we only assume that $\mathcal{M}_{\Gamma,h}$ is $\gamma_0$-shape-regular and that its elements are sufficiently small. 
\end{remark}


\section{Applications}

Theorem \ref{thm} has the following important consequences. First, it gives a uniform bound for the stability of the {\em discrete jump lifting}
$\Phi_h: \widetilde{V}_h(\Gamma) \to \mathbb{V}_h(\Gamma)$, defined by 
$$\Phi_h: [w_h] \mapsto u_h$$
where $u_{h}$ is the unique minimizer of the $\mathbb{H}^{1/2}$ norm over the set of $w_h + H^{1/2}([\Gamma]) \cap \mathbb{V}_h(\Gamma)$. 
\begin{corollary}[Uniform bound for the discrete jump lifting]
	\label{discHarmLift}
	Let the assumptions of \Cref{thm} be satisfied. Then there holds
	\begin{equation}
		\label{discreteStabPhih}
		\quad \norm{\Phi_h \widetilde{\varphi}_h}_{\mathbb{H}^{1/2}} \leq C(\gamma_0) \norm{\widetilde{\varphi}_h}_{\widetilde{H}^{1/2}}\,, \quad \forall \widetilde{\varphi}_h \in \widetilde{V}_h(\Gamma)\,.
	\end{equation}
\end{corollary}
\begin{proof}
	By definition of the quotient norm on $\widetilde{H}^{1/2}(\Gamma)$, and due to the Hilbert structure on this space, there exists a {\em continuous harmonic lifting} $\Phi$, that is, an isometry
	\[\Phi: \widetilde{H}^{1/2}(\Gamma) \to \mathbb{H}^{1/2}(\Gamma)\]
	such that $[\Phi \widetilde{\varphi}] = \widetilde{\varphi}$ and
	\[\norm{\Phi \widetilde{\varphi}}_{\mathbb{H}^{1/2}} = \norm{\widetilde{\varphi}}_{\widetilde{H}^{1/2}}\,, \quad \forall \widetilde{\varphi} \in \widetilde{H}^{1/2}(\Gamma)\,.\]
	We claim that the operator $\Psi_h: \widetilde{V}_h(\Gamma) \to \mathbb{H}^{1/2}(\Gamma)$, defined by $\Psi_h \isdef \Pi_h \circ \Phi$, satisfies the property 
	\begin{equation}
		\label{PsiRightInv}
		\forall \widetilde{\varphi}_h \in \widetilde{V}_h(\Gamma)\,, \quad [\Psi_h \widetilde{\varphi}_h] = \widetilde{\varphi}_h\,.
	\end{equation}
	In other words, $\Psi_h$ is a right-inverse of the jump operator on $\widetilde{V}_h(\Gamma)$. Provided that this holds, we have by the minimization property of $\Phi_h$, and using \Cref{thm} property (i):
	\[\begin{aligned}
		\forall \widetilde{\varphi}_h \in \widetilde{V}_h(\Gamma)\,, \quad \norm{\Phi_h \widetilde{\varphi}_h}_{\mathbb{H}^{1/2}(\Gamma)} &\leq \norm{\Psi_h \widetilde{\varphi}_h}_{\mathbb{H}^{1/2}}\\
		& \leq C(\gamma_0) \norm{\Phi \widetilde{\varphi}_h}_{\mathbb{H}^{1/2}}= C(\gamma_0) \norm{\widetilde{\varphi}_h}_{\widetilde{H}^{1/2}}\,.
	\end{aligned}\]
	It remains to show the property eq.\eqref{PsiRightInv}. For this, fix $\widetilde{\varphi}_h \in \widetilde{V}_h(\Gamma)$ and let
	\[v_h \isdef \Psi_h \widetilde{\varphi}_h\,.\] 
	Furthermore, pick $w_h \in \mathbb{V}_h(\Gamma)$ such that $[w_h] = \widetilde{\varphi}_h$. Since $[\Phi \widetilde{\varphi}_h - w_h] = 0$, it follows by Theorem \ref{thm} (ii) and (iii) that
	$$0 = [\Pi_h (\Phi \widetilde{\varphi}_h - w_h)] = [\Pi_h (\Phi\widetilde{\varphi}_h) - \Pi_h w_h] = [v_h] - [w_h] = [v_h] - \widetilde{\varphi}_h\,.$$
	This shows that $[v_h] = \widetilde{\varphi}_h$, concluding the proof.
\end{proof}

\Cref{thm} can also be used to prove the equivalence on $\widetilde{V}_h(\Gamma)$ of the $\widetilde{H}^{1/2}$ norm and the {\em discrete quotient norm} $N_h$, defined by 
\[\forall \widetilde{\varphi}_h \in \widetilde{V}_h(\Gamma)\,, \quad N_h(\widetilde{\varphi}_h) \isdef \inf \enstq{\norm{u_h}_{\mathbb{H}^{1/2}}}{u_h \in \mathbb{V}_h(\Gamma) \,\, \textup{s.t. } [u_h] = \widetilde{\varphi}_h}\,.\]
\begin{corollary}
	\label{discJumpNorm}
	Let the assumptions of \Cref{thm} be satisfied. Then for all $\widetilde{\varphi}_h \in \widetilde{V}_h(\Gamma)$, there holds
	\begin{multline*}
		\inf \enstq{\norm{v_h}_{\mathbb{H}^{1/2}}}{v_h \in \mathbb{V}_h(\Gamma)\,\,\,\textup{s.t.} \,\, [v_h] = \widetilde{\varphi}_h}\\ \leq C(\gamma_0) \inf \enstq{\norm{v}_{\mathbb{H}^{1/2}}}{v \in \mathbb{H}^{1/2}(\Gamma)\,\,\,\textup{s.t.} \,\, [v] = \widetilde{\varphi}_h}\,,	
	\end{multline*}
	or in other words, one has the norm equivalence
	\begin{equation}
		\label{discreteJumpNorm}
		\frac{1}{C(\gamma_0)} N_h(\widetilde{\varphi}_h) \leq  \norm{\widetilde{\varphi}_h}_{\widetilde{H}^{1/2}} \leq  N_h(\widetilde{\varphi}_h)\,.
	\end{equation}
\end{corollary}
\begin{proof}
	The right inequality in eq.~\eqref{discreteJumpNorm} follows from the definition of the quotient norm on $\widetilde{H}^{1/2}(\Gamma)$. Reciprocally, if $\widetilde{\varphi}_h \in \widetilde{V}_h(\Gamma)$, then by \Cref{discHarmLift}, we have
	\[\inf \enstq{\norm{v_h}_{\mathbb{H}^{1/2}(\Gamma)}}{v_h \in \mathbb{V}_h(\Gamma)\,\,\,\textup{s.t.} \,\, [v_h] = \widetilde{\varphi}_h} \leq \norm{\Phi_h \widetilde{\varphi}_h}_{\mathbb{H}^{1/2}(\Gamma)} \leq C_h \norm{\widetilde{\varphi}_h}_{\widetilde{H}^{1/2}}\,,\]
	giving the left inequality. 
\end{proof}

The rest of this work is dedicated to prove \Cref{thm}. From now on, we fix a constant $\gamma_0 > 0$ and a $\gamma_0$-regular pair of trace-compatible meshes $\mathcal{M}_{\Gamma,h}$ and $\mathcal{M}_{\Omega,h}$. The dependence in $h$ will be omitted from some of the notation. We shall also use the letter $C$ to denote a generic positive constants which only depends on $\Gamma$, $\Omega$ and $\gamma_0$, but is, in particular, independent on the size of the elements of $\mathcal{M}_{\Gamma,h}$ and $\mathcal{M}_{\Omega,h}$. 




\section{Basis of the discrete multi-trace space}
\label{sec:BasisMulti}
In order to construct the operator $\Pi_h$ of \Cref{thm}, we first construct a basis of $\mathbb{V}_h(\Gamma)$. The main ideas are from \cite{averseng2022fractured}. Let us denote by $\{\vec{x}_1\,,\ldots\,,\vec{x}_N\}$ be the vertices of $\mathcal{M}_{\Omega,h}$ with the common vertices of $\mathcal{M}_{\Omega,h}$ and $\mathcal{M}_{\Gamma,h}$ given by $\vec{x}_1,\ldots,\vec{x}_M$. Let $V_h(\Omega)$ be the space of functions in $V_h(\Omega \setminus \Gamma)$, which are continuous. Notice that
\[V_h(\Omega) = H^1(\Omega) \cap V_h(\Omega \setminus \Gamma)\,.\]
Let $\{\phi_i\}_{1 \leq i \leq N}$ be the nodal basis of $V_h(\Omega)$, that is, the set of elements of $V_h(\Omega)$ defined by
\[\phi_i(\vec{x}_{i'}) = \delta_{i,i'}\,, \quad 1 \leq i,i' \leq N\,.\] 
For each $i \in \{1,\ldots,N\}$, the {\em star} of $\vec{x}_i$, denoted by $\textup{st}(\vec{x}_i,\mathcal{M}_{\Omega,h})$, is the set of tetrahedra $K \in \mathcal{M}_{\Omega,h}$ containing $\vec{x}_i$ as a vertex. We define a graph $\mathcal{G}(\vec{x}_i)$ with 
\begin{itemize}
	\item {\bf Nodes:} The elements of $\textup{st}(\vec{x}_i,\mathcal{M}_{\Omega,h})$
	\item {\bf Edges:} The pairs $\{K,K'\} \subset \textup{st}(\vec x_i,\mathcal{M}_\Omega)$ such that $K$ and $K'$ share a face not contained in $\mathcal{M}_\Gamma$. 
\end{itemize}
Let $\gamma_{i,1}, \ldots, \gamma_{i,q_i}$ be the connected components of $\mathcal{G}(\vec{x}_i)$, and let 
\[\abs{\gamma_{i,j}} \isdef \bigcup_{K \in \gamma_{i,j}} \overline{K}\,.\]
Let us write
\begin{equation*}
	\mathcal{H}(\Omega) \isdef \enstq{(i,j) \in \N^2}{1 \leq i \leq N\,,\,\, 1 \leq j \leq q_i}\,.
\end{equation*}

For $(i,j) \in \mathcal{H}(\Omega)$, we denote by $\phi_{i,j}$ the {\em split basis function} of $V_h(\Omega \setminus \Gamma)$ on $\gamma_{i,j}$, which is defined by 
\[\phi_{i,j}(\vec{x}) \isdef \begin{cases}\phi_i( \vec x) & \textup{for } \vec x \in \textup{int}(\abs{\gamma_{i,j}})\,, \\
	0 & \textup{otherwise}.	
\end{cases}\]
where $\textup{int}(\abs{\gamma_{i,j}})$ is the interior of $\abs{\gamma_{i,j}}$. 
\begin{lemma}
	\label{basisVh}
	The split basis functions $\{\phi_{i,j}\}_{(i,j) \in \mathcal{H}(\Omega)}$ form a basis of $V_h(\Omega \setminus \Gamma)$.
\end{lemma}
\begin{proof}
	Suppose that there exist real coefficients $\lambda_{i,j}$ such that 
	\begin{equation}
		\label{linDepPhiIJ}
		\forall \vec x \in \Omega\,, \quad \sum_{(i,j) \in \mathcal{H}(\Omega)} \lambda_{i,j} \phi_{i,j}(\vec{x}) = 0\,.
	\end{equation}
	Fix $(i_0,j_0) \in \mathcal{H}(\Omega)$ and let $K \in \gamma_{i,j}$. Let $(\vec y_{n})_{n \in \N}$ be a sequence of points in the interior of $K$, such that 
	\[\lim_{n \to \infty} \vec y_n = \vec x_{i_0}\,.\]
	It is easy to show that $\lim_{n \to \infty} \phi_{i,j}(\vec y_n) \to \delta_{i,i_0} \delta_{j,j_0}$. Therefore, applying eq.~\eqref{linDepPhiIJ} to $\vec y_n$ and passing to the limit, we conclude that $\lambda_{i_0,j_0} = 0$. This proves that the functions $\phi_{i,j}$ are linearly independent. 
	
	Next, we consider $u_h \in V_h(\Omega \setminus \Gamma)$. For each $(i,j) \in \mathcal{H}(\Omega)$, we fix a tetrahedron $K_{i,j} \in \gamma_{i,j}$ and a sequence $(\vec y^{i,j}_{n})_{n \in \N}$ converging to $\vec x_i$ from $K_{i,j}$, as above. Let $\lambda_{i,j} = \lim_{n \to \infty} u_h(\vec y^{i,j}_n)$. We prove that 
	\begin{equation}
		\label{decompUhPhiIJ}
		u_h = \sum_{(i,j) \in \mathcal{H}(\Omega)} \lambda_{i,j} \phi_{i,j}\,,	
	\end{equation}
	by showing that this equality holds on the interior of each tetrahedron $K \in \mathcal{M}_{\Omega,h}$. Suppose that the vertices of $K$ are $\vec x_{i_0}, \ldots, \vec x_{i_3}$, and for each $p \in \{0,\ldots,3\}$, let $j_p$ be such that $K \in \gamma_{i_p,j_p}$. Let $\vec y_n^p$ be a sequence of points in the interior of $K$ converging to $\vec x_{i_p}$. We claim that 
	\begin{equation}
		\label{limitUh}
		\lim_{n \to \infty} \phi_{i,j}(\vec y_n^p) = \delta_{i,i_p} \delta_{j,j_p}\,, \quad 
		\lim_{n \to \infty} u_h(\vec y^{p}_n) = \lambda_{i_p,j_p}\,.
	\end{equation}
	As before, the first limit can be shown easily. The second one is obvious if $K = K_{i_p,j_p}$. If $K$ shares a face $F \notin \mathcal{M}_\Gamma$ with $K_{i_p,j_p}$, then it is a consequence of the well-known property that for $u \in H^1(\Omega \setminus \Gamma)$, the traces of $u$ of $F$ from $K$ and $K_{i_p,j_p}$ must agree. Otherwise, by definition of $\mathcal{G}(\vec{x}_i)$, one can consider a face-connected path of tetrahedra from $K$ to $K_{i_p,j_p}$, and the desired limit is established by repeating the previous argument for each pair of consecutive tetrahedra in this path.
	
	Having shown the property \eqref{limitUh}, we conclude that the linear functions defined on $K$ by each side of eq.~\eqref{decompUhPhiIJ} have a common limit at $4$ affinely independent points, thus they are equal on $K$. This concludes the proof of the lemma.
\end{proof}
For $u_h \in V_h(\Omega \setminus \Gamma)$, we will denote by $u_h(\gamma_{i,j})$ the coefficient of $u_h$ on the split basis function $\phi_{i,j}$. 
On $V_h(\Omega \setminus \Gamma)$, we introduce the discrete $l^2$ scalar product 
\begin{equation*}
	\forall (u,v) \in {V}_h(\Omega \setminus \Gamma) \times {V}_h(\Omega \setminus \Gamma)\,, \quad [u,v]_{l^2} \isdef \sum_{i = 1}^{N} \sum_{j = 1}^{q_i} u(\gamma_{i,j}) v(\gamma_{i,j})\,,
\end{equation*}
Let $Y_h(\Omega)$ be the $[\cdot,\cdot]_{l^2}$ orthogonal complement of $V_h(\Omega)$. Let 
\[\widetilde{\mathcal{H}} \isdef \enstq{(i,j) \in \mathcal{H}(\Omega)}{q_i > 1 \textup{ and } j \leq q_i - 1}\,,\]
and for $(i,j) \in \widetilde{\mathcal{H}}$, define 
\[\widetilde{\phi}_{i,j} \isdef \phi_{i,j} - \phi_{i,q_i}\,.\]
Using that $\{\phi_i\}_{1 \leq i \leq N}$ is a basis of $V_h(\Omega)$, together with the property
\[\phi_{i} = \sum_{j = 1}^{q_i} \phi_{i,j}\,,\]
and \Cref{basisVh}, a simple algebraic reasoning shows that $\{\widetilde{\phi}_{i,j}\}_{(i,j) \in \widetilde{\mathcal{H}}}$ is a basis of $Y_h(\Omega)$. 
Define
\[\eta_{i} \isdef \pi_D(\phi_{i}) \,, \quad \widetilde{\eta}_{i,j} \isdef \pi_D(\widetilde{\phi}_{i,j})\,.\]
Let $V_h([\Gamma]) \isdef \mathbb{V}_h \cap H^{1/2}([\Gamma])$ and $Y_h(\Gamma) \isdef \pi_D(Y_h(\Omega))$. In summary, we have the following result:
\begin{lemma}
	There holds
	\[\mathbb{V}_h(\Gamma) = V_h([\Gamma]) \oplus Y_h(\Gamma)\,.\]
	Moreover, $\{\eta_i\}_{1 \leq i \leq M}$ is a basis of $V_h([\Gamma])$ and $\{\widetilde{\eta}_{i,j}\}_{(i,j) \in \widetilde{\mathcal{H}}}$ is a basis of $Y_h(\Gamma)$.  
\end{lemma}
%
%
%where, for any $u \in V_h(\Omega \setminus \Gamma)$, the coefficients $u(\vec{x}_{i,j})$ are uniquely defined by
%\begin{equation*}
%	u = \sum_{i = 1}^{N} \sum_{j=1}^{q_j} u(\vec{x}_{i,j})\phi_{i,j}\,.
%\end{equation*}
%Let $Y_h(\Omega)$ (resp.~$Y_h(\Gamma)$) be the $[\cdot,\cdot]$-orthogonal complement of $V_h(\Omega)$ in $V_h(\Omega \setminus \Gamma)$ (resp.~of $V_h(\Gamma)$ in $\mathbb{V}_h(\Gamma)$) for the $[\cdot,\cdot]_{l^2}$ scalar product (resp.~for the scalar product induced by $[\cdot,\cdot]_{l^2}$). Let 
%\begin{equation*}
%	\mathcal{B} \isdef \enstq{(i,j)\in \mathcal{H}(\Gamma)}{q_i > 1 \textup{ and } j \leq q_i - 1}\,.
%\end{equation*}
%and, for $(i,j) \in \mathcal{B}$, let 
%\begin{equation*}
%	\widetilde{\phi}_{i,j} \isdef \phi_{i,j} - \phi_{i,q_i}\,, \quad \widetilde{\eta}_{i,j} \isdef \pi_D(\widetilde{\phi}_{i,j}) =\eta_{i,j} - \eta_{i,q_i}\,.
%\end{equation*}
%
%The elementary properties of the spaces and basis functions introduced above are recapped in the following Lemma.
%\begin{lemma}
%	\label{lemBasis}
%	There holds
%	\[V_h(\Omega \setminus \Gamma) = V_h(\Omega) \oplus Y_h(\Omega)\]
%	\[\mathbb{V}_h(\Gamma) = V_h([\Gamma]) \oplus Y_h(\Gamma)\]
%	and
%	\[\pi_D(V_h(\Omega)) = V_h([\Gamma])\,, \quad \pi_D(Y_h(\Omega)) = Y_h(\Gamma)\,.\]
%	Furthermore, one has
%	\[V_h(\Omega) = \textup{Span}(\enstq{\phi_i}{i \in \{1\,,\ldots\,,N\}})\,, \quad Y_h(\Omega) = \textup{Span}(\enstq{\widetilde{\phi}_{i,j}}{(i,j) \in \mathcal{B}})\,,\]
%	\[V_h(\Gamma) = \textup{Span}\left(\enstq{\eta_i}{i \in \{1\,,\ldots\,,M\}}\right)\,, \quad Y_h(\Gamma) = \textup{Span}(\enstq{\widetilde{\eta}_{i,j}}{(i,j) \in \mathcal{B}})\,.\]
%\end{lemma}


\section{Definition of the quasi-interpolant}
\label{sec:defQuasiInterp}

We now define the quasi-interpolant $\Pi_h: \mathbb{H}^{1/2}(\Gamma) \to \mathbb{V}_h(\Gamma)$, much in the same way as in \cite{scott1990finite}. We refer to \cite{claeys2013integral} for the definition of the space $\mathbb{H}^{-1/2}(\Gamma)$ and the duality pairing $\dduality{\cdot}{\cdot}: \mathbb{H}^{1/2}(\Gamma) \times \mathbb{H}^{-1/2}(\Gamma) \to \R$. 
Also recall the operator $\pi_N: H(\div,\Omega \setminus \Gamma) \to \mathbb{H}^{-1/2}(\Gamma)$. 
\begin{definition}
	\label{defPih}
	Let $\Pi_h: \mathbb{H}^{1/2}(\Gamma) \to \mathbb{V}_h(\Gamma)$ be the operator defined by
	\[\forall u \in \mathbb{H}^{1/2}(\Gamma)\,, \quad \Pi_h u \isdef \sum_{k = 1}^M \dduality{u}{\vec \psi_k}\eta_k \,\,\, + \sum_{(i,j) \in \widetilde{\mathcal{H}}} \dduality{u}{\vec \psi_{i,j}} \widetilde{\eta}_{i,j}\,.\]
	where $\vec \psi_k = \pi_N(\vec w_k)$, $\vec \psi_{i,j} = \pi_N(\vec w_{i,j})$, and the vector fields $\vec w_k$ and $\vec w_{i,j}$ are given in \Cref{prop1}.
\end{definition}

\begin{proposition}[Dual basis functions]
\label{prop1}
There exists two sets of vector fields $\{\vec w_{k}\}_{1 \leq k \leq N}$ and $\{\vec w_{i,j}\}_{(i,j) \in \widetilde{\mathcal{H}}}$ satisfying the following properties:
\[\vec w_{k} \in H(\div,\Omega \setminus \Gamma)\,, \quad \vec w_{i,j} \in H(\div,\Omega)\,,\]
\[\textup{supp}(\vec w_k) \subset \abs{\textup{st}(\vec x_k,\mathcal{M}_{\Omega,h})}\,, \quad  \textup{supp}(\vec w_{i,j}) \subset \abs{\textup{st}(\vec x_i,\mathcal{M}_{\Omega,h})}\,,\]
\[\norm{\vec w_k}_{\infty} \leq C h_k^{-2}\,, \quad \norm{\div \vec w_k}_{\infty} \leq C h_k^{-3}\,,\]
\[\norm{\vec w_{i,j}}_{\infty} \leq C h_i^{-2}\,, \quad \norm{\div \vec w_{i,j}}_{\infty} \leq C h_i^{-3}\,,\]
and the orthogonality relations
\begin{equation}
	\label{dualityWiWij}
	\begin{cases}
		\displaystyle\int_{\Omega \setminus \Gamma} \nabla \phi_k \cdot \vec w_{k'} + \phi_k \div \vec w_{k'} = \delta_{k,k'}\,, & \displaystyle \int_{\Omega \setminus \Gamma} \nabla \widetilde{\phi}_{i,j} \cdot \vec w_{k'} + \phi_{i,j} \div \vec w_{k'}  = 0 \\[2em]
		\displaystyle\int_{\Omega \setminus \Gamma} \nabla \phi_k \cdot \vec w_{i',j'} + \phi_k \div \vec w_{i',j'} = 0\,, & \displaystyle\int_{\Omega \setminus \Gamma} \nabla \widetilde{\phi}_{i,j} \cdot \vec w_{i',j'} + \widetilde{\phi}_{i,j} \div \vec w_{i',j'}  = \delta_{i,i'}\delta_{j,j'}
	\end{cases}
\end{equation}
for all $(k,k') \in \{1,\ldots,N\}^2$ and $((i,j),(i',j')) \in \widetilde{\mathcal{H}}^2$, where $h_i$ is the diameter of $\abs{\textup{st}(\vec{x}_i,\mathcal{M}_{\Omega,h})}$.
\end{proposition}
The proof of this proposition is the object of the next section. From the relations \eqref{dualityWiWij}, it follows that 
\[\begin{cases}
	\dduality{\eta_{k}}{\vec \psi_{k'}}  = \delta_{k,k'}\,,& \dduality{\eta_k}{\vec \psi_{i',j'}} = 0\,, \\
	\dduality{\widetilde{\eta}_{i,j}}{\vec \psi_{k'}} = 0\,,& \dduality{\widetilde{\eta}_{i,j}}{\vec \psi_{i',j'}} = \delta_{i,i'}\delta_{j,j'} \,,
\end{cases}\]
which implies that $\Pi_h$ is indeed a projection on $\mathbb{V}_h(\Gamma)$. Moreover, since $\vec w_{i,j} \in H(\div,\Omega)$, we have $\vec \psi_{i,j} \in H^{-1/2}([\Gamma])$ (where $H^{-1/2}([\Gamma])$ is the Neumann single trace space, cf. \cite{claeys2013integral}), ensuring that the property 
\begin{equation*}
	\forall u \in {H}^{1/2}([\Gamma])\,, \quad \Pi_h u \in V_h([\Gamma])\,,
\end{equation*}
is satisfied, due to the polarity of $H^{1/2}([\Gamma])$ and $H^{-1/2}([\Gamma])$ \cite[Prop. 6.3]{claeys2013integral}. This proves properties (ii) and (iii) of Theorem \ref{thm}, so it remains to prove the uniform $\mathbb{H}^{1/2}$ stability of $\Pi_h$. For this, we mostly follow the method of \cite{scott1990finite}. We consider another projection $\mathcal{Z}_h: H^1(\Omega \setminus \Gamma) \to V_h(\Omega \setminus \Gamma)$ with the property that 
\begin{equation}
	\label{commutZh}
	\pi_D\circ \mathcal{Z}_h = \Pi_h \circ \pi_D\,.	
\end{equation}
We shall need yet another set of dual basis functions $(\psi_i)_{M+1 \leq i \leq N}$, which are defined as in \cite{scott1990finite}. Namely, for each $i \in \{M+1,\ldots,N\}$, we pick a tetrahedron $K_i$ such that $\vec x_i$ is a vertex of $K_i$. Then, $\psi_i$ is defined as the $L^2$ dual affine function on $K_i$ associated to $\vec x_i$. Let
\[\mathcal{Z}_hf \isdef \sum_{i = M+1}^N  \left(\int_{K_i}\psi_i(x) f(x)dx\right)\phi_i + \sum_{i = 1}^{M} \dduality{\pi_D(f)}{\vec \psi_i} \phi_{i} + \sum_{(i,j) \in \widetilde{\mathcal{H}}} \dduality{\pi_D(f)}{\vec \psi_{i,j}} \widetilde{\phi}_{i,j}\,.\]
It is immediate that $\mathcal{Z}_h$ is a projection and that eq.~\eqref{commutZh} is satisfied.
For a tetrahedron $K \in \mathcal{M}_{\Omega,h}$, we denote by  $\textup{st}(K,\mathcal{M}_{\Omega,h})$ the set of tetrahedra of $\mathcal{M}_{\Omega,h}$ sharing at least a vertex with $K$. The estimate of \cite[Thm 3.1]{scott1990finite} carries over for $\mathcal{Z}_h$ with minor modifications as we show now.
\begin{lemma}
	\label{lemUK}
	For $f \in \textup{H}^{1}(\Omega \setminus \Gamma)$ and $K \in \mathcal{M}_{\Omega,h}$, there holds
	\begin{equation*}
		\norm{\mathcal{Z}_hf}_{{H}^1(K)} \leq C \left(h_K^{-1}\norm{f}_{L^2(S_K)} + \norm{\nabla f}_{L^2(S_K)}\right)\,, 
	\end{equation*}
	where $S_K \isdef \abs{\textup{st}(K,\mathcal{M}_{\Omega,h})} \setminus \Gamma$.
\end{lemma}
\begin{proof}
	We focus on the case $K\cap \Gamma\neq \emptyset$, because $K \cap \Gamma = \emptyset$ is settled by \cite[Thm. 3.1]{scott1990finite}. Given a tetrahedron $K$ incident to a face $F$ of $\mathcal{M}_{\Gamma,h}$, we can write
	\begin{equation*}
		\begin{split}
			\norm{\mathcal{Z}_hf}_{{H}^1(K)} &\leq \sum_{i = M+1}^N \abs{\int_{K_i} \psi_i(x) f(x) dx} \norm{\phi_i}_{H^1(K)} \\
			 &+ \sum_{i = 1}^M \abs{\dduality{\pi_D(f)}{\pi_N(\vec{w}_i)}} \norm{\phi_i}_{{H}^1(K)} \\
			 &+ \sum_{(i,j) \in \widetilde{\mathcal{H}}} \abs{\dduality{\pi_D(f)}{\pi_N(\vec{w}_{i,j})}} \norm{\widetilde{\phi}_{i,j}}_{{H}^1(K)}\,
		\end{split}
	\end{equation*}
	by the triangular inequality. The first line of the right hand side is estimated as in \cite[Thm. 3.1]{scott1990finite}. We furthermore recall the classical inequality
	\[\norm{\phi_i}_{H^1(K)} \leq C h_K^{1/2}\,,\]
	using the shape regularity assumption. A similar estimate holds for $\widetilde{\phi}_{i,j}$, since $\phi_{i,j}$ is either equal to $0$ or $\phi_i$ on $K$. On the other hand, by definition of $\dduality{\cdot}{\cdot}$, we have the expression 
	\begin{equation*}
		\label{tmp1}
		\dduality{\pi_D(f)}{\vec \psi_i} = \int_{\Omega} \nabla f \cdot \vec{w}_i + \div(\vec{w}_i) f = \int_{S_K}\nabla f \cdot \vec{w}_i + \div(\vec{w}_i) f \,,
	\end{equation*}
	since $\vec{w}_i$ is zero outside $\textup{st}(\vec{x}_i,\mathcal{M}_{\Omega,h}) \subset S_K$.  By the previous estimates on $\vec{w}_i$, using the shape-regularity of $\mathcal{M}_{\Omega,h}$ and some elementary inequalities, this leads to
	\[\abs{\dduality{u}{\pi_N(\vec{w}_i}} \leq C \left(h_K^{1/2} \norm{\nabla f}_{L^2(S_K)} + h_K^{-1/2} \norm{f}_{L^2(S_K)} \right)\,.\]
	The coefficients involving $\vec{w}_{i,j}$ are treated similarly. The rest of the proof follows is as in \cite[Thm. 3.1]{scott1990finite}.
\end{proof}
Using now exactly the same arguments as in \cite[Thm 4.1]{scott1990finite} and \cite[Cor 4.1]{scott1990finite}, we deduce 
\begin{equation}
	\label{brambleIneq}
	\norm{\mathcal{Z}_h f}_{{H}^1(\Omega \setminus \Gamma)} \leq C\norm{f}_{{H}^1(\Omega \setminus \Gamma)}\,.
\end{equation}
\begin{remark}
	The main tool, to derive \eqref{brambleIneq} from \Cref{lemUK}, is a Bramble-Hilbert inequality in domains $U$ consisting of face-connected and shape-regular unions of tetrahedra. To apply this result, a crucial requirement is that the restrictions of elements $V_h(\Omega \setminus \Gamma)$ to each tetrahedron $K \in \mathcal{M}_\Omega$ span all polynomial functions of degree $1$ in $K$.  
\end{remark}
\begin{corollary}
	There exists a constant $C > 0$ only depending on shape-regularity such that
	\[\forall u \in \mathbb{H}^{1/2}(\Gamma)\,, \quad \norm{\Pi_h u}_{\mathbb{H}^{1/2}} \leq C \norm{u}_{\mathbb{H}^{1/2}}\]
\end{corollary}
\begin{proof}
	Fix $f \in H^1(\Omega \setminus \Gamma)$ such that $u = \pi_D(f)$. Then we have $\Pi_h u = \pi_D(\mathcal{Z}_h f)$. 
	Therefore, it holds that $\norm{\Pi_h u}_{\mathbb{H}^{1/2}(\Gamma)} \leq \norm{\mathcal{Z}_h f}_{\textup{H}^1(\Omega \setminus \Gamma)}$ by definition of the $\mathbb{H}^{1/2}$ norm. By the previous lemma, this gives
	\[\norm{\Pi_h u}_{\mathbb{H}^{1/2}} \leq C \norm{f}_{{H}^1(\Omega \setminus \Gamma)}\,.\]
	This holds for any $f \in \textup{H}^1(\Omega \setminus \Gamma)$ such that $\pi_D(f) = u$. Hence, passing to the infimum, we conclude
	\[\forall u \in \mathbb{H}^{1/2}(\Gamma)\,, \quad \norm{\Pi_h u}_{\mathbb{H}^{1/2}} \leq C \norm{u}_{\mathbb{H}^{1/2}}\,,\]
	which proves the claim.
\end{proof}

We address the construction of the ``dual basis functions" $\vec{w}_k$ and $\vec{w}_{i,j}$ of \Cref{prop1} in the next section.

\section{Construction of the dual basis functions}
\label{sec:construction}
The main tool for constructing the required vector fields is the set of bridge functions that we define now. Let $i \in \{1,\ldots,N\}$ and $1 \leq j,k \leq q_i$. We say that $\{j,k\}$, with $j \neq k$, is a {\em bridge} around $\vec{x}_i$ if there exist two tetrahedra $K \in \gamma_{i,j}$, $K' \in \gamma_{i,k}$ such that $K$ and $K'$ have a common face $F \in \mathcal{M}_{\Gamma,h}$. The set of bridges around $\vec{x}_i$ is denoted by $\mathcal{B}(i)$. 
\begin{lemma}[Bridge functions]
	\label{lemBridge}
	For each bridge $\{j,k\} \in \mathcal{B}(i)$ with $j < k$, there exists a vector field $\vec{b}_{i,\{j,k\}} \in H(\div,\Omega)$ such that 
	\[\textup{supp}(\vec{b}_{i,\{j,k\}}) \subset \abs{\textup{st}(\vec{x}_i,\mathcal{M}_{\Omega,h})}\,,\]
	\begin{equation}
		\label{boundBridge}
		\norm{\vec{b}_{i,\{j,k\}}} \leq C h_i^{-2}\,, \quad \norm{\div \vec{b}_{i,\{j,k\}}} \leq C h_i^{-3}\,,
	\end{equation}
	and 
	\begin{equation}
		\label{propertyOfBridgeFunctions}
		\forall u_h \in V_h(\Omega \setminus \Gamma)\,, \quad \int_{\Omega \setminus \Gamma} \vec{b}_{i,\{j,k\}}\nabla u_h + \div(\vec{b}_{i,\{j,k\}})\,u_h = u_h(\gamma_{i,j}) - u_h(\gamma_{i,k})\,.
	\end{equation}
\end{lemma}
\begin{proof}
	Given a tetrahedron $K$, a face $F$ of $K$ and an affine function $\psi$ defined on $F$, we define the vector field $\vec b_{K,F,\psi}$ on $K$ by 
	\[\vec b_{K,F,\psi}(\vec y) \isdef \frac{1}{h(K,F)} \sum_{m = 0}^2\psi(\vec x_m) \lambda_m(\vec y)(\vec x_m - \vec x_3)\,, \quad \vec y \in K\,,\]
	where $\{\vec x_0,\vec x_1, \vec x_2\}$ are the vertices of $F$ and $\lambda_m$ is the barycentric coordinate of $K$ associated to $\vec x_m$, and $h(K,F)$ is the height of $K$ from $F$. Given a face $F'$ of $K$, we denote by $\vec n_{K,F'}$ the outward pointing normal vector on $F'$. With this definition, we have $h(K,F) = \vec n_{K,F} \cdot (\vec x_m - \vec x_3)$, where $m$ is any number in $\{0,1,2\}$ and $\vec x_3$ is the vertex of $K$ opposite to $F$. One can check that 
	\[\vec b_{K,F,\psi}\cdot \vec n_{K,F'} = \begin{cases}
		\psi & \textup{if } F' = F\,. \\
		0 & \textup{otherwise}
	\end{cases}\]
	If $K \in \mathcal{M}_{\Omega_h}$, we denote the extension of $\vec b_{K,F,\psi}$ by $0$ outside $K$ again by $\vec b_{K,F,\psi}$. By what precedes, $\vec b_{K,F,\psi} \in H(\div,\Omega \setminus \Gamma)$ if $F \in \mathcal{M}_{\Gamma,h}$. 
	
	Fix a bridge $\{j,k\} \in \mathcal{B}(i)$, and let $K_j \in \gamma_{i,j}$ and $K_k \in \gamma_{i,k}$ be two tetrahedra sharing a face $F_{jk} \in \mathcal{M}_{\Gamma,h}$. The face $F_{jk}$ contains the vertex $\vec x_i$. Following \cite{scott1990finite}, we define an affine function $\psi$ such that, for any affine function $\rho$, 
	\[\int_{F} \psi(\vec y)\rho(\vec y) \,d \vec y = \rho(\vec x_i)\,. \]
	Using pullback and scaling arguments (cf. \cite[Lemma 3.1]{scott1990finite}), one has
	\begin{equation}
		\label{boundMaxPhi}
		\max_{\vec y \in F_{jk}}\abs{\psi(\vec y)}\leq C h_{F_{jk}}^{-2}\,,
	\end{equation}
	where $h_F$ is the diameter of $F_{jk}$. With this choice of function $\psi$, let
	\[\vec b_{i,\{j,k\}} \isdef \vec b_{K_j,F_{jk},\psi} - \vec b_{K_k,F_{jk},\psi}\,.\]
	This time, $\vec b_{i,\{j,k\}} \in H(\div,\Omega)$. It is also clear that $\textup{supp}(\vec b_{i,\{j,k\}})\subset \abs{\textup{st}(\vec x_i,\mathcal{M}_{\Omega,h})}$. The estimate of the $L^\infty$ norm of $\vec b_{i,\{j,k\}}$ in eq.~\eqref{boundBridge} is obtained from the bound on $\abs{\psi}$ \eqref{boundMaxPhi}, the bound $\gamma_0$ on the aspect ratio of $K_j$ and $K_k$, together with the simple estimate $\abs{\lambda_m}_{\infty} \leq 1$. The bound on $\norm{\div \vec (b_{i,\{j,k\}})}_{\infty}$ follows from the previous one, since $\vec b_{i,\{j,k\}}$ is linear on each tetrahedron $K_j$ and $K_k$ (using again the control over the aspect ratio of $K_j$ and $K_k$). 
	
	Fix $u_h \in V_h(\Omega \setminus \Gamma)$. We have
	\begin{equation}
		\label{eqSplit1}
		\begin{split}
			\int_{\Omega \setminus \Gamma} \vec{b}_{i,\{j,k\}}\nabla u_h &+ \div(\vec{b}_{i,\{j,k\}})\,u_h \\
			= & \sum_{(i',j') \in \mathcal{H}(\Omega)}  u_h(\gamma_{i',j'})\int_{K_j} \vec{b}_{K_j,F_{jk},\psi}\nabla \phi_{i',j'} + \div(\vec{b}_{K_j,F_{jk},\psi})\,\phi_{i',j'}\\
			& - \sum_{(i',j') \in \mathcal{H}(\Omega)}  u_h(\gamma_{i',j'})\int_{K_k} \vec{b}_{K_k,F_{jk},\psi}\nabla \phi_{i',j'} + \div(\vec{b}_{K_k,F_{jk},\psi})\,\phi_{i',j'}\,.
		\end{split}
	\end{equation} 
	Let $(i',j') \in \mathcal{H}(\Omega)$ and let $r_{i',j'}$ be the trace of $\phi_{i',j'}$ on $F_{jk}$ from $K_j$. There holds
	\[\int_{K_j} \vec{b}_{K_j,F_{jk},\psi}\nabla \phi_{i',j'} + \div(\vec{b}_{K_j,F_{jk},\psi})\,\phi_{i',j'} = \int_{F_{jk}} \psi(\vec y) r_{i',j'}(\vec y) d\vec y = r_{i',j'}(\vec x_i)\,.\]
	If $i' \neq i$, then $r_{i',j'}(\vec x_i) = 0$. On the other hand, if $i' = i$ but $j' \neq j$, then $\phi_{i,j'}$ is identically $0$ on $\gamma_{i,j}$, and in particular on $K_j$. Finally, if $i = i'$ and $j = j'$, one has $r_{i,j}(\vec x_i) = 1$. In summary, 
	\[\int_{K_j} \vec{b}_{K_j,F_{jk},\psi}\nabla \phi_{i',j'} + \div(\vec{b}_{K_j,F_{jk},\psi})\,\phi_{i',j'} = \delta_{i,i'}\delta_{j,j'}\]
	and by the same reasoning, 
	\[\int_{K_k} \vec{b}_{K_k,F_{jk},\psi}\nabla \phi_{i',j'} + \div(\vec{b}_{K_k,F_{jk},\psi})\,\phi_{i',j'} = \delta_{i,i'}\delta_{k,j'}\,.\]
	The identity \eqref{propertyOfBridgeFunctions} follows by injecting the two relations above in eq.~\eqref{eqSplit1}. 
\end{proof}

If $q_i = 2$, there is a unique bridge function $\vec b_{i,1}$ at $\vec x_i$, and one can immediately see that $\vec w_{i,1} \isdef \vec b_{i,1}$ fulfills the requirements of \Cref{prop1}. In general, the idea is that one can construct the vector fields $\vec{w}_{i,j}$ of \Cref{prop1} as a suitable linear combination of the bridge functions at $\vec{x}_i$. To see this, we fix $i \in \{1,\ldots,N\}$ and introduce the vector spaces
\[F_i \isdef  \textup{Span}(\{\widetilde{\phi}_{i,j}\}_{1 \leq j \leq q_i - 1})\,, \quad G_i \isdef \textup{Span}(\{\vec b_{i,\{j,k\}}\}_{\{j,k\} \in \mathcal{B}(i)})\,.\]
Note that $G_i \subset H(\div,\Omega)$. Let $F_i^*$ be the dual of $F_i$, i.e. the set of linear forms on $F_i$ and define the operator $\mathcal{A}: G_i \to F_i^*$ by 
\[\forall \vec (b,u_h) \in G_i \times F_i\,, \quad (\mathcal{A}\vec b)(u_h) \isdef \int_{\Omega \setminus \Gamma} \nabla u_h \cdot \vec b + u_h \div\vec b\,. \]
\begin{lemma}
	\label{lemSurject}
	The operator $\mathcal{A}: G_i \to F_i^*$ is surjective. 
\end{lemma}
\begin{proof}
	We start by introducing the graph $\mathcal{G}^*(\vec{x}_i)$ defined by
	\begin{itemize}
		\item {\bf Nodes}: the numbers $1,\ldots,q_i$,
		\item {\bf Edges}: the bridges $\{k,k'\}$.
	\end{itemize} 
	The key point is that $\mathcal{G}^*(\vec x_i)$ is connected (see e.g. \cite[Lemma 1.5]{averseng2022fractured}). By \cite[Lemma 3.9]{rudin1991functional}, to prove the lemma, it suffices to show that if $u_h \in F_i$ is such that 
	\begin{equation}
		\label{assum1}
		(\mathcal{A}\vec b_{i,\{j,k\}})u_h = 0 \quad \forall \{j,k\} \in \mathcal{B}(i)\,,
	\end{equation}
	then $u_h = 0$. Every element $v_h$ of $F_i$ satisfies
	\[\sum_{j = 1}^{q_i} v_h(\gamma_{i,j}) = 0\,.\]
	Hence, it remains to show that for $1 \leq j,k \leq N$, 
	\begin{equation}
		\label{uhgammaij=uhgammaik}
		u_h(\gamma_{i,j}) = u_h(\gamma_{i,k})\,.
	\end{equation}
	When $\{j,k\} \in \mathcal{B}(i)$, this relation is a consequence of the assumption \eqref{assum1}. In general, we see that \eqref{uhgammaij=uhgammaik} holds by considering a path from $j$ to $k$ in $\mathcal{G}^*(\vec x_i)$. 
\end{proof}
Let $\{w^*_{i,j}\}_{1 \leq j \leq q_i - 1}$ be the basis of $F_i^*$ defined by the relations
\[w^*_{i,j}(\widetilde{\phi}_{i,j'}) = \delta_{j,j'}\,.\]
By the previous lemma, there exists an element $\vec w_{i,j} \in G_i$ such that $\mathcal{A}(\vec w_{i,j}) = w^*_{i,j}$. One has the properties
\[\vec w_{i,j} \in H(\div,\Omega)\,, \quad \textup{supp}(\vec w_{i,j}) \subset \abs{\textup{st}(\vec x_i,\mathcal{M}_{\Omega,h})}\,,\] 
by linearity since those properties hold for all bridge functions at $\vec x_i$.
The same argument allows to check that for all $i' \in \{1,\ldots,N\}$ and $(i'',j'') \in \widetilde{\mathcal{H}}$ with $i'' \neq i$, there holds
\[\int_{\Omega \setminus \Gamma} \nabla {\phi}_{i'} \cdot \vec w_{i,j} + \phi_{i'} \div \vec w_{i,j} = 0\,, \quad \int_{\Omega \setminus \Gamma} \nabla \widetilde{\phi}_{i'',j''} \cdot \vec w_{i,j} + \widetilde{\phi}_{i'',j''} \div \vec w_{i,j} = 0\,.\] 
Finally, we have 
\begin{equation}
	\label{temp1}
	\int_{\Omega \setminus \Gamma} \nabla \widetilde{\phi}_{i,j'} \cdot \vec w_{i,j} + \widetilde{\phi}_{i,j'} \div \vec w_{i,j} = \delta_{j,j'}\,,
\end{equation}
by construction. This proves the relations \eqref{dualityWiWij} for $\vec w_{i,j}$, and it remains to estimate its $L^\infty$ norm as well as that of its divergence. This is addressed in the following lemma. 
\begin{lemma}
	One can choose $\vec w_{i,j}$ such that 
	\[\norm{\vec w_{i,j}}_{\infty} \leq Ch_i^{-2}\,, \quad \norm{\div \vec w_{i,j}}_{\infty} \leq Ch_i^{-3}\,.\]
\end{lemma}
\begin{proof}
	First, we remark that the number $q_i$ is bounded independently of the meshes $\mathcal{M}_{\Omega,h}$ and $\mathcal{M}_{\Gamma,h}$. Indeed, one can check that $q_i$ is equal to the number of connected components of $B\setminus \Gamma$, where $B$ is a small enough ball centered at $\vec x_i$. We shall not prove this assertion in detail, for the sake of conciseness. Let us denote by $Q$ an upper bound for $q_i - 1$. 
	
	Next, since $\mathcal{A}$ is surjective and since $\textup{dim}(F_i^*) = \textup{dim}(F_i) = q_i - 1$, one can find a set of pairs $\{\{j_p,k_p\}\}_{1 \leq p \leq q_i - 1}$ such that $\{\mathcal{A} \vec b_{i,\{j_p,k_p\}}\}_{1 \leq p \leq q_i - 1}$ is a basis of $F_i^*$. In this basis, $w^*_{i,j}$ is expressed by 
	\[w^*_{i,j} = \sum_{p = 1}^{q_i - 1} \lambda_p (\mathcal{A} \vec b_{i,\{j_p,k_p\}})\,. \]
	The coefficients $\lambda_p$ are found by solving the linear system
	\[\mathbf{A} \begin{bmatrix}
		\lambda_1\\
		\vdots\\
		\lambda_{q_i - 1}
	\end{bmatrix} = \vec e_j\,,\]
 	where $\vec e_j$ is the $j$-th vector of the canonical basis of $\R^{q_i - 1}$ and the matrix coefficients 
	\[(\mathbf{A})_{p,p'} \isdef \widetilde{\phi}_{i,p'}(\gamma_{i,j_p}) - \widetilde{\phi}_{i,p'}(\gamma_{i,k_p})\]
	are integer between $-2$ and $2$. The set $\mathcal{S}_{Q}$ of invertible square matrices $\mathbf{A}$ of size at most $Q$ with coefficients in $\{-2,-1,0,1,2\}$ is finite. Therefore, we can write 
	\begin{equation}
		\label{indepBound}
		\abs{\lambda_p} \leq \sup_{\mathbf{A} \in \mathcal{S}_{Q}}|||\mathbf{A}^{-1}|||_{\infty}\,, \quad \forall p \in \{1\,,\ldots\,,q_i - 1\}\,,
	\end{equation} 
	where, for an element $v$ of $\R^r$ and a square $r \times r$ matrix $M$, we have denoted 
	\[\norm{v}_{\infty} \isdef \max_{1 \leq q \leq r} \abs{v_q}\,, \quad ||| M |||_{\infty} \isdef \sup_{v \in R^{r}} \frac{\norm{M v}_{\infty}}{\norm{v}_{\infty}} \,.\] 
	The bound \eqref{indepBound} only depends on $\mathcal{M}_\Gamma$ (through the number $Q$). The proof is concluded by using the triangular inequality and the bounds \eqref{boundBridge} from \Cref{lemBridge}. 
\end{proof}

The construction of the vector fields $\{\vec w_k\}_{1 \leq k \leq M}$ of \Cref{prop1} is easier; we present this now. For each $(i,j) \in \mathcal{H}(\Omega)$, one can construct a vector field $\vec c_{i,j} \in H(\div,\Omega \setminus \Gamma)$ such that 
\[\forall u_h \in V_h(\Omega \setminus \Gamma)\,, \quad \int_{\Omega \setminus \Gamma} \nabla u_h \cdot \vec c_{i,j} + u_h \div(\vec c_{i,j}) = u_h(\gamma_{i,j})\,.\]
With the notation of the proof of \Cref{lemBridge}, it suffices to pick any tetrahedron $K$ in $\gamma_{i,j}$ with a face $F$ in $\mathcal{M}_{\Gamma,h}$. We then let $\vec c_{i,j} \isdef \vec b_{K,F,\psi}$, where $\psi$ is again the $L^2$ dual linear function associated to $\vec x_i$. Then, for $k \in \{1\,,\ldots\,, M\}$, we set
\[\vec w_k \isdef \frac{1}{q_k} \sum_{l = 1}^{q_k} \vec{c}_{k,l}\,,\]
and this definition fulfills the requirements of \Cref{prop1}. 

We have thus constructed the vector fields $\vec w_k$ and $\vec w_{i,j}$ and shown that they satisfy the properties stated in \Cref{prop1}, and hence the proof of \Cref{thm} is concluded.


\bibliographystyle{siamplain}
\bibliography{biblioJSZ.bib}


\end{document}
