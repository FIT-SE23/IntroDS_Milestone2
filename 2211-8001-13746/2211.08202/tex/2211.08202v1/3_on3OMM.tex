\section{Memory for Pareto points of \threeOMM}
%ARX \simon{rethink title}
Before analyzing the optimization time of the \NSGA on \threeOMM, we show that, by employing sufficiently many reference points, once the population covers a point on the Pareto front, it is covered for all future iterations.
To this end, we first analyze how the normalization shifts the points on the Pareto front to then conclude that every reference point is associated with at most one point of the Pareto front.
With this, we argue that already sampled points on the Pareto front are never lost again. 

Our analysis assumes that the population is non-degenerated, i.e., that the 3 extreme vectors in each objective span a plane to make the normalization as described by \cite{Deb_Jain_2014} possible. 
However, w.h.p. this holds for the first iteration and thus, by applying our arguments, also for every following iteration.
%ARX \merk{Ev. spaeter mal huebscher machen}

\begin{lemma}\label{lem:normalizedPlane}
Let $Z$ be a set of solutions for \threeOMM and $z_j$ be the minimum value along the $j$th objective in $Z$, $j\in\{1,2,3\}$, and $\zmin=z_1+z_2+z_3$.
Then every objective value $v$ before normalization corresponds to $\frac{1}{n-\zmin}(v-\colvec{z_1\\z_2\\z_3})$ after normalization with respect to $Z$.
\end{lemma}
\begin{proof}
The first normalization step simply moves the plane by subtracting $\colvec{z_1\\z_2\\z_3}$ from every point. 
We call the plane of points on the Pareto front after this step $E'\colon v_1+v_2+v_3 = n - \zmin$.

In the second step, a maximal point for each objective is computed. 
Then, for each point on the Pareto front, each objective value is divided by the intercept of the plane that is spanned by the extreme points with the respective objective axis.
As the complete Pareto front lies in $E'$ after the first step, the plane spanned by the extreme point is exactly $E'$ for all non-degenerated populations.
Computing the intercepts of $E'$ with the objective axes gives that each objective value is divided by $n-\zmin$.
We note that $n-\zmin>0$ for all non-degenerated instances.
\end{proof}

\begin{lemma}\label{lem:uniqueAssociations}
By employing more than $\frac{2n}{\sqrt{3}}$ divisions along each objective, all individuals that are associated with the same reference point have the same objective value.
\end{lemma}
\begin{proof}
Before normalization, the Pareto front lies in the plane $E\colon v_1+v_2+v_3=n$.
Recall that every solution is a Pareto element and thereby has rank 1, so we normalize with respect to the complete population.
Let $z_1,z_2,z_3$ be the minimum values in each objective in the population and let $z_{\min} = z_1+z_2+z_3$.
By Lemma~\ref{lem:normalizedPlane}, for each point we first subtract $\zmin$ from $v_1+v_2+v_3$ and then divide by $n-\zmin$, so the normalized Pareto front lies in the non-negative domain of the plane
$E^n\colon v_1+v_2+v_3=1$.

This is the exact domain in which the reference points are placed. We note that the Voronoi cells, that is, the set of points in the plane which are closest to a given reference point, are all regular hexagons (possibly intersected with the positive quadrant $\R_{\ge 0}^3$).
We prove that at most one point of the normalized Pareto front lies in each cell, which yields the desired statement.
To this end, we show that the diameter of the hexagons is smaller than the distance between any pair of points on the normalized Pareto front.

Let $v \neq w$ be two points on the Pareto front before normalization.
Then they differ in at least 2 objectives by at least 1, so $\norm{v-w}\ge \sqrt{1^2+1^2}=\sqrt{2}$.
By applying Lemma~\ref{lem:normalizedPlane}, this distance of the points after normalization is
\begin{align*}
    \norm{\frac{1}{n-\zmin}\left(v-\colvec{z_1\\z_2\\z_3}\right)-\frac{1}{n-\zmin}\left(w-\colvec{z_1\\z_2\\z_3}\right)}
    = \frac{\norm{v-w}}{n-\zmin} \ge \frac{\sqrt{2}}{n}.
\end{align*}
%
Next, we compute the diameter of the Voronoi cells provided we employed $p$ divisions along each objective. Observe that the length $s$ of the short diagonal of the hexagons is exactly the length of the divisions, so $s= \frac{\sqrt{2}}{p}$.
Let $a$ denote the length of the sides of the hexagons. Since $s = \sqrt 3 a$, we have $a = \sqrt{2/3} / p$. Since the diameter, that is, the length of the long diagonal, satisfies $\ell = 2a$, we conclude that for $p>\frac{2n}{\sqrt{3}}$ we have 
\begin{equation*}
    \ell = 2 \sqrt{2/3} \cdot \frac{1}{p} > \frac{\sqrt{2}}{n}.
\end{equation*}
Then, each cell contains at most one point of the Pareto front.
\end{proof}

We note that $\frac{2n}{\sqrt{3}}+1 > \ceil{\frac{2n}{\sqrt{3}}}$ divisions correspond to 
\begin{align*}
    \binom{3+\frac{2n}{\sqrt{3}}+1-1}{\frac{2n}{\sqrt{3}}+1}
    =\binom{\frac{2n}{\sqrt{3}}+3}{2}\\
   = \frac{1}{2}\left(\frac{2n}{\sqrt{3}}+3\right)\cdot \left(\frac{2n}{\sqrt{3}}+2\right)
    = \frac{2n^2}{3}+\frac{5n}{\sqrt{3}}+3
\end{align*} 
reference points in total.
The Pareto front in comparison contains $(\frac{n}{2}+1)^2=\frac{n^2}{4}+n+1$ points.

\begin{lemma}\label{lem:notLooseSolution}
By employing a population of size $N \ge (\frac{n}{2}+1)^2$ and more than $\frac{2n}{\sqrt{3}}$ divisions along each objective for the reference points, once the population contains a solution for a certain point on the Pareto front it will always contain a solution for this point.
\end{lemma}
\begin{proof}
Observe that for \threeOMM, every solution is non-dominated.
Thus, after the recombination and mutation step, the complete population of old solutions and offsprings contains $2N$ solutions in the first non-dominated level.
After normalizing the objectives, the solutions are associated with their nearest reference point.
All solutions with the same multi-objective values are associated with the same reference point, as they have the same position in the solution space.
At the same time, by Lemma~\ref{lem:uniqueAssociations}, each reference point is associated only with solutions with the same multi-objective value.
Thereby, each reference point with at least one associated solution corresponds to exactly one unique point on the Pareto front.
Therefore, there at most $ (\frac{n}{2}+1)^2$ such reference points.
Because $N\ge (\frac{n}{2}+1)^2$, at least one solution is selected from each reference point with non-empty association set.
Thereby, for each already sampled point on the Pareto front, a solution is selected for the next iteration.
\end{proof}

\section{Runtime of the NSGA-III on \threeOMM}
Knowing that we do not lose points on the Pareto front once sampled, we are able to give a first upper bound on the expected optimization time of the \NSGA on \threeOMM. 

We assume the recombination and mutation step to be such that each individual with a Hamming distance of 1 to any individual in the population is created with a probability in $\Omega(\frac{1}{n})$. 
For example, this is achieved if each individual in the population has at least a constant probability to produce an offspring by applying standard uniform mutation in each iteration.

\begin{theorem}\label{thm:runTime3OMM}
Consider the \NSGA with population size $N \ge (\frac{n}{2}+1)^2$ and more than $\frac{2n}{\sqrt{3}}$ divisions along each objective  optimizing \threeOMM.
Let $T$ denote the number of iterations until the population covers the Pareto front.
Then, $E[T]\in O(n^3)$.
\end{theorem}
\begin{proof}
As long as the complete Pareto front is not yet sampled, the population contains at least one bitstring $s$ with a Hamming distance of 1 to a bitstring $s'$ of a not yet sampled point on the Pareto front in each iteration.
By our assumption on the recombination and mutation step, $s'$ is sampled with a probability in $\Omega(\frac{1}{n})$.
Thus, it takes $O(n)$ iterations in expectation to sample the next point on the Pareto front.
By Lemma~\ref{lem:notLooseSolution} the population will never loose a sampled point on the Pareto front.
Thereby, after an expected number of at most $O(n(\frac{n}{2}+1)^2)=O(n^3)$ iterations, the complete Pareto front is sampled.
\end{proof}

We note that this analysis pessimistically assumes that the Pareto front is sampled in some defined order. 
We are optimistic to improve the bound by employing more sophisticated arguments.
%ARX \simon{vlg. prev. work}