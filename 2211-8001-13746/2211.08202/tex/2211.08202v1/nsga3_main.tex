%new 14.6.21: \left..\right trick from https://tex.stackexchange.com/questions/2607/spacing-around-left-and-right
%    07.08.22: added \usepackage{lmodern} as it sometimes gives much better printouts.
%    20.08.22: redefined mathbbm -> mathbb to have type-1 fonts
%    16.10.22: 

\documentclass[english,a4paper,12pt]{article}

\let\mathbbm\mathbb

%ARX \usepackage{showkeys}
%\usepackage{fullpage}% with \begin{large}...\end{large} below

\hyphenation{analysis onemax Doerr parameter leadingones Hoeffding develop-ment Krejca Carola}

\usepackage{amsxtra, amsfonts, amssymb, amstext, amsmath, mathtools}%, 
\usepackage{amsthm}
\usepackage{booktabs}
\usepackage{nicefrac}
\usepackage{xspace}
\usepackage{url}\urlstyle{rm}
\usepackage{graphics,color}
\usepackage[algo2e,ruled,vlined,linesnumbered]{algorithm2e}
\usepackage{wrapfig}
\usepackage{lmodern}


\renewcommand{\labelenumi}{\theenumi}
\renewcommand{\theenumi}{(\roman{enumi})}

%\allowdisplaybreaks[4]
\clubpenalty=10000
\widowpenalty=10000
\frenchspacing 

\newtheorem{theorem}{Theorem}
\newtheorem{lemma}[theorem]{Lemma}
\newtheorem{corollary}[theorem]{Corollary}
\newtheorem{definition}[theorem]{Definition}



% Mathematic abbreviations
\newcommand{\oea}{\mbox{${(1 + 1)}$~EA}\xspace}
\newcommand{\oeamu}{${(1 + 1)}$~EA$_{\mu,p}$\xspace}
\newcommand{\ooea}{\oea}
\newcommand{\olea}{${(1\overset{+}{,}\lambda)}$~EA\xspace}
\newcommand{\oplea}{\mbox{${(1+\lambda)}$~EA}\xspace}
\newcommand{\mpoea}{\mbox{${(\mu+1)}$~EA}\xspace}
\newcommand{\mplea}{\mbox{${(\mu+\lambda)}$~EA}\xspace}
\newcommand{\mclea}{\mbox{${(\mu,\lambda)}$~EA}\xspace}
\newcommand{\oclea}{\mbox{${(1,\lambda)}$~EA}\xspace}
\newcommand{\opllga}{\mbox{${(1+(\lambda,\lambda))}$~GA}\xspace}
\newcommand{\ollga}{\opllga}
\newcommand{\NSGAtwo}{NSGA\nobreakdash-II\xspace}
\newcommand{\NSGA}{NSGA\nobreakdash-III\xspace}

\newcommand{\OM}{\textsc{OM}\xspace}
\newcommand{\onemax}{\textsc{OneMax}\xspace}
\newcommand{\zeromax}{\textsc{ZeroMax}\xspace}
\newcommand{\LO}{\textsc{Leading\-Ones}\xspace}
\newcommand{\leadingones}{\LO}
\newcommand{\needle}{\textsc{Needle}\xspace}
\newcommand{\cliff}{\textsc{Cliff}\xspace}
\newcommand{\binval}{\textsc{BinVal}\xspace}
\newcommand{\plateau}{\textsc{Plateau}\xspace}
\newcommand{\jump}{\textsc{Jump}\xspace}
\newcommand{\oneminmax}{\textsc{OneMinMax}\xspace}
\newcommand{\OMM}[1]{\text{$#1$-\textsc{OMM}}\xspace}
\newcommand{\threeOMM}{\textsc{3-OMM}\xspace}
\newcommand{\cocz}{\textsc{COCZ}\xspace}
\newcommand{\lotz}{\textsc{LOTZ}\xspace}
\newcommand{\ojzj}{\textsc{OneJumpZeroJump}\xspace}
\newcommand{\mlotz}{$m$\textsc{LOTZ}\xspace}
\newcommand{\mcocz}{$m$\textsc{COCZ}\xspace}


\DeclareMathOperator{\rand}{rand}
\DeclareMathOperator{\Sample}{Sample}
\DeclareMathOperator{\minmax}{minmax}
\DeclareMathOperator{\paral}{par}
\DeclareMathOperator{\poly}{poly}
\newcommand{\gmax}{g_{\max}}
\newcommand{\Ymax}{Y_{\max}}
\newcommand{\X}{\mathcal{X}}

%\newcommand{\C}{\ensuremath{\mathbb{C}}}
\newcommand{\R}{\ensuremath{\mathbb{R}}}
\newcommand{\Q}{\ensuremath{\mathbb{Q}}}
\newcommand{\N}{\ensuremath{\mathbb{N}}} % ohne Null!!!
\newcommand{\Z}{\ensuremath{\mathbb{Z}}}
\newcommand{\bbone}{{\mathbbm{1}}}


\newcommand{\calA}{\ensuremath{\mathcal{A}}} 
\newcommand{\calE}{\ensuremath{\mathcal{E}}} 
\newcommand{\calF}{\ensuremath{\mathcal{F}}} 
\newcommand{\calP}{\ensuremath{\mathcal{P}}} 
\newcommand{\calS}{\ensuremath{\mathcal{S}}} 
\newcommand{\calT}{\ensuremath{\mathcal{T}}} 
% no \calO for big-Oh	
	
\DeclareMathOperator{\Bin}{Bin}
\DeclareMathOperator{\Geom}{Geom}
\DeclareMathOperator{\arcsinh}{arcsinh}
\DeclareMathOperator{\mutate}{mutate}
\newcommand{\fmin}{f_{\mathrm{min}}}
\newcommand{\fmax}{f_{\mathrm{max}}}
\newcommand{\xmin}{x_{\mathrm{min}}}
\newcommand{\xmax}{x_{\mathrm{max}}}
\newcommand{\zmin}{z_{\min}}

% use \Pr[...] for probability
% use E[...] for expectation
\newcommand{\Var}{\mathrm{Var}\xspace} %use with [...]
\newcommand{\Cov}{\mathrm{Cov}\xspace} %use with [...] 
\newcommand{\eps}{\varepsilon} 

\newcommand{\merk}[1]{\textbf{\textcolor{red}{#1}}}

\newcommand{\assign}{\leftarrow}

\let\originalleft\left
\let\originalright\right
\renewcommand{\left}{\mathopen{}\mathclose\bgroup\originalleft}
\renewcommand{\right}{\aftergroup\egroup\originalright}

\newcommand{\simon}[1]{\textcolor{blue}{SW: #1}}
\newcommand{\mtodo}[1]{\textcolor{red}{#1}}
\newcommand{\nootherflip}{\left(1-\frac{1}{n}\right)^{n-1}}
\newcommand{\colvec}[2][.6]{%
  \scalebox{#1}{%
    \renewcommand{\arraystretch}{.6}%
    $\begin{pmatrix}#2\end{pmatrix}$%
  }
}
\DeclarePairedDelimiter\floor{\lfloor}{\rfloor}
\DeclarePairedDelimiter\ceil{\lceil}{\rceil}
\newcommand{\abs}[1]{\left| #1\right|\xspace}
\newcommand{\norm}[1]{\left\lVert #1 \right\rVert_2}

\usepackage{hyperref}

\begin{document}
%\begin{large}
\title{A Mathematical Runtime Analysis of the Non-dominated Sorting Genetic Algorithm III (\NSGA)}

%\author{Benjamin Doerr\\ Laboratoire d'Informatique (LIX)\\ CNRS\\ \'Ecole Polytechnique\\ Institut Polytechnique de Paris\\ Palaiseau\\ France%\\ email: {\tt doerr@mpi-inf.mpg.de}}
\author{Benjamin Doerr\setcounter{footnote}{6}\thanks{Laboratoire d'Informatique (LIX), CNRS, \'Ecole Polytechnique, Institut Polytechnique de Paris, Palaiseau, France} \and Simon Wietheger\thanks{Hasso Plattner Institute, Potsdam, Germany}}

%\author{Benjamin Doerr\thanks{Laboratoire d'Informatique (LIX), CNRS, \'Ecole Polytechnique, Institut Polytechnique de Paris, Palaiseau, France}}

%
%\title{Multiplicative Up-Drift\thanks{Extended and improved version of a paper that appeared in the proceedings of GECCO 2019~\cite{DoerrK19}. }
%}
%%\titlerunning{Multiplicative Up-Drift}        % if too long for running head



\maketitle

\sloppy{
\begin{abstract}
\begin{abstract}
% Modern ConvNets
Since the recent success of Vision Transformers (ViTs), explorations toward ViT-style architectures have triggered the resurgence of ConvNets.
% Novel view: interaction
In this work, we explore the representation ability of modern ConvNets from a novel view of multi-order game-theoretic interaction, which reflects inter-variable interaction effects w.r.t.~contexts of different scales based on game theory.
% MogaNet
Within the modern ConvNet framework, we tailor the two feature mixers with conceptually simple yet effective depthwise convolutions to facilitate middle-order information across spatial and channel spaces respectively.
% Experiments illustration
In this light, a new family of pure ConvNet architecture, dubbed MogaNet, is proposed, which shows excellent scalability and attains competitive results among state-of-the-art models with more efficient use of parameters on ImageNet and multifarious typical vision benchmarks, including COCO object detection, ADE20K semantic segmentation, 2D\&3D human pose estimation, and video prediction.
% Highlight results
Typically, MogaNet hits 80.0\% and 87.8\% top-1 accuracy with 5.2M and 181M parameters on ImageNet, outperforming ParC-Net-S and ConvNeXt-L while saving 59\% FLOPs and 17M parameters.
% code (arxiv & final version)
The source code is available at \url{https://github.com/Westlake-AI/MogaNet}.
\vspace{-1.0em}


\end{abstract}

\end{abstract}

\section{Introduction}
\section{Introduction}
\label{sec:introduction}
Reliable, fast, and efficient data processing is crucial given the growing volumes of data in both industry and research.
These needs are often addressed by using distributed dataflow frameworks like Spark~\cite{Zaharia2010}, and Flink~\cite{Carbone2015}.
As these frameworks' handle parallelism, distribution, and fault tolerance, they make it easier for users to create scalable data-parallel programs.
The resulting applications can use a variety of compute clusters for data processing.

However, it is still difficult to choose and configure resources in a way that closely meets user-specific goals and constraints~\cite{RajanKCK16,cloudcomputingchallenges2018}.
Numerous strategies have been put forth to assist users, and they can be grouped into two categories:
Model-based techniques~\cite{MaoAMK16,RajanKCK16,ShahAKW19,AlSayehS19,KirchoffXMR19,ChenLLWZ21silhouette,ScheinertTZWAWK21,WillTSBK21,AlSayehMJPS22} often rely on access to historical performance data, however, historical workload execution data is not always available.
Search-based techniques~\cite{AlipourfardLCVY17,HsuNFM18,bilal2020finding,klimovic2018selecta,fekry2020accelerating,MendesCRG20,LiuXL20,SongZLSFDS21} conduct costly profiling runs prior to executing the actual workload utilizing all, or a fraction, of the input data to iteratively create performance models.

Often enough though, the optimized resource configuration is only relevant for the workload at hand. 
Information about the underlying infrastructure are solely obtained implicitly, i.e., by measuring the performance of the target workload in one execution context.
As a consequence, a thorough understanding of utilized resources and their capabilities is lacking and insights gained cannot be easily transferred to other contexts, for instance, when profiling new workloads with different resource demands. 
This requires repeated profiling overhead for reoccurring or comparable workloads that could be avoided, rendering current approaches less resource-efficient than they could be.

Addressing these limitations, we present \emph{Perona}, a novel approach to explicit and robust infrastructure fingerprinting. 
It motivates the usage of common sets and configurations of benchmarking tools to assess the full capabilities of target infrastructures and to make the obtained benchmarking metrics directly comparable.
This explicit fingerprinting operation transparently reveals the characteristics of resources and allows ranking them.
Perona discards irrelevant benchmarking metrics in a data-driven manner by learning a dense, low-dimensional representation of input metric vectors. 
With these, more sophisticated resource decisions can be made for big data analytics, e.g., with regard to scheduling or resource allocations.
To be able to assess a recent benchmark execution, our approach incorporates results of prior benchmark executions, which is particularly useful for detecting resource degradation. 

\emph{Contributions}. The contributions of this paper are:

\begin{itemize}
    \item A novel approach for incorporating infrastructure fingerprinting into model-based methods for optimized resource configuration of workloads through ranking of resources and detection of degrading resource behavior.
    \item A method for context-aware representation learning of benchmark metrics, thereby not only discarding insignificant features but also taking prior benchmark runs and corresponding machine metrics into account. 
    \item An openly available implementation\footnote{\url{https://github.com/dos-group/perona-infrastructure-fingerprinting}} of Perona which we evaluated with regard to common metrics and in interplay with resource configuration methods for distributed dataflows and scientific workflows. 
\end{itemize}

\emph{Outline}. \autoref{sec:related_work} discusses the related work.
\autoref{sec:approach} describes the three main aspects of our approach in detail. 
\autoref{sec:evaluation} presents the results of our evaluation.
\autoref{sec:conclusion} concludes the paper and gives an outlook on future work.

\section{Preliminaries}
We now define the required notation for multi-objective optimization as well as the considered objective functions and give an introduction to the \NSGA.
For $m\in \N$, an \emph{$m$-objective function} $f$ is a tuple $(f_1,\ldots, f_m)$, where $f_i\colon \Omega\rightarrow \R$ for some search space $\Omega$.
For all $x\in\Omega$, we define $f(x)=(f_1(x),\ldots, f_m(x))$.
Other than in single-objective optimization, there is usually no solution that maximizes all $m$ objective functions simultaneously.
For two solutions $x,y$, we say that $x$ \emph{dominates} $y$ and write $x\succeq y$ if and only if $f_j(x)\ge f_j(y)$ for all $1\le j\le m$.
If additionally there is a $j_0$ such that $f_{j_0}(x)>f_{j_0}(y)$, we say that $x$ \emph{strictly dominates} $y$, denoted by $x\succ y$.
A solution is \emph{Pareto-optimal} if it is not strictly dominated by any other solution.
We refer to the set of objective values of Pareto-optimal solutions as the \emph{Pareto front}.
In our analyses, we analyze the number of function evaluations until the population covers the Pareto front, i.e., until for each value $p$ on the Pareto front the population contains a solution $x$ with $f(x)=p$.

\subsection{\threeOMM}
We are interested in studying the \NSGA on a 3-objective function.
The \oneminmax function, first proposed by \cite{GielL10}, translates the well-established \onemax benchmark into a bi-objective setting. 
It is defined as $\oneminmax\colon \{0,1\}^n\rightarrow \N\times \N$ by 
\begin{equation*}
    \oneminmax(x) = (\zeromax(x),\onemax(x)) = \left(n-\sum_{i=1}^nx_i, \sum_{i=1}^nx_i\right)
\end{equation*}
for all $x=(x_1,\ldots,x_n)\in\{0,1\}^n$.

We call its translation into a 3-objective setting \threeOMM (for 3-\oneminmax).
For even $n$, we define $\threeOMM\colon \{0,1\}^n\rightarrow \N^3$ by
\begin{align*}
 \threeOMM(x)
 %  = (\zeromax(x), \onemax(x_{1\ldots n/2}),
  %                 \onemax(x_{n/2+1\ldots n}))
   = \left(n-\sum_{i=1}^n x_i,
          \sum_{i=1}^{n/2} x_i,
          \sum_{i=n/2+1}^{n} x_i\right)
\end{align*}
for all $x=(x_1,\ldots,x_n)\in\{0,1\}^n$.

\subsection{The NSGA-III}
The main structure of the \NSGA \cite{Deb_Jain_2014} is identical to the one of the \NSGAtwo \cite{DebPAM02}.
It is initialized with a random population of size $N$.
In each iteration, the user applies mutation and/or crossover operators to generate an offspring population of size $N$.
As the NSGA framework is an MOEA with a fixed population size, out of this total of $2N$ individuals, $N$ have to be selected for the next iteration.

Because non-dominated solutions are to be preferred, the following ranking scheme is used to set the dominance relation as the predominant criterion for the survival of individuals. 
Individuals that are not strictly dominated by any other individual in the population obtain rank 1.
Recursively, the other ranks are defined. Each individual that has not yet been ranked and is only strictly dominated by individuals of rank \(1,\ldots,k-1\) is assigned rank $k$. 
Clearly, an individual is more interesting the lower its rank is. 
Let $F_i$ denote the set of individuals with rank $i$ and let $i_0$ be minimal such that $\sum_{i=1}^{i_0} |F_i| \ge N$.
All individuals with rank at most $i_0-1$ survive into the next generation.
Further, $0<k\le N$ individuals of rank $i_0$ have to be selected for the next generation such that the new population is again of size $N$, and the next iteration can begin. 
The only difference between the \NSGAtwo and the \NSGA is the procedure of selecting the $k$ individuals of rank $i_0$.
While the \NSGAtwo employs crowding-distance, the \NSGA uses reference points, typically distributed in some structured manner on the normalized hyper-plane, in order to select a diverse population.
For the whole framework, see Algorithm~\ref{alg:nsga3}. 
Note that whenever we refer to sets of individuals, we are actually referring to multi-sets as each solution might be represented multiple times in the population.
%
\begin{algorithm2e}%
Let the initial population $P_0$ be composed of $N$ individuals chosen independently and uniformly at random from $\{0,1\}^n$.

\For{$t = 0, 1, 2, \ldots$}{
Generate the offspring population $Q_t$ with size $N$

Use fast-non-dominated-sort() from Deb et al.\ \cite{DebPAM02}) to divide $R_t = P_t \cup Q_t$ into $F_1, F_2, \ldots$

Find $i^* \ge 1$ such that $\sum_{i=1}^{i^*-1} |F_i| < N$ and $\sum_{i=1}^{i^*} |F_i| \ge N$

$Z_t = \bigcup_{i=1}^{i_0-1}F_i$

Select $\Tilde{F_{i^*}}\subseteq F_{i^*}$ such that $|Z_t\cup\Tilde{F_{i^*}}| = N$ (use crowding-distance for \NSGAtwo and Algorithm~\ref{alg:selection} for \NSGA)

$P_{t+1} = Z_t \cup \Tilde{F_{i^*}}$
}
\caption{NSGA-II and NSGA-III}
\label{alg:nsga3}
\end{algorithm2e}%

In order to select individuals from the critical rank $i_0$, the \NSGA normalizes the objective functions and associates each individual with a reference point.

Regarding the normalization, let $z_j$ be the minimum value in the $j$th objective among all individuals of rank at most $i_0$.
Then, $z_j$ is subtracted from each objective function $f_j$ to obtain $f_j'$.
For each objective $j$, let $x_j$ be an individual with rank at most $i_0$ that maximizes $f'_j(x)$ and let $z_j^{\max} = f'(x)$.
We refer to the $j$th objective value of the point at the intersection of a hyper-plane and the $j$th objective axis as an \emph{intercept}.
Then, each objective is divided by the intercept of the hyper-plane spanned by $z_1^{\max},z_2^{\max},\ldots$ with the respective objective axis to obtain the normalized objective function $f^n$.
Algorithm~\ref{alg:normalize} formalizes the normalization procedure.
%
\begin{algorithm2e}%
\SetKwInOut{Input}{Input}
\SetKwInOut{Output}{Output}

\Input{%
$f=(f_1,\ldots, f_M)$: objective function\newline
$Z\subseteq \{0,1\}^n$: a multi-set of individuals}
\Output{$f^n=(f_1^n,\ldots, f_M^n)$: normalized objective function}
\For{$j=1$ \KwTo $M$}{
    $z_j = \min_{z\in Z} f_j(z)$\\
    $f_j'(x) = f_j(x) - z_j \quad \forall x \in Z$\\
    $z_j^{\max} = f(\arg\max_{x\in Z} f'_j(x))$
}
Let $H$ be the hyper-plane spanned by $z_1^{\max},\ldots, z_M^{\max}$  

\For{$j=1$ \KwTo $M$}{
    $a_j = $ the intercept of $H$ with the $j$th objective axis
    
    $f_j^n(x) = \frac{f_j'(x)}{a_j} \quad \forall x\in Z$
}
\caption{Normalization}
\label{alg:normalize}
\end{algorithm2e}%

After the normalization, each individual of rank at most $i_0$ is associated with its closest reference point with respect to the normalized objectives.
Then, one iterates through the reference points, always selecting the one with the fewest associated individuals that are already selected for the next generation. Ties are resolved randomly.
If the reference point has no associated individuals that are not yet selected, it is skipped.
Otherwise, the individual that is closest to the reference point (with respect to the normalized objective function) is selected for the next generation. 
Once more, ties are resolved randomly.
If the next generation already contains an individual that is associated with the reference point, other measures than the distance to the reference point can be considered.
The selection terminates as soon as the required number of individuals is reached.
This procedure is formalized in Algorithm~\ref{alg:selection}.
%
\begin{algorithm2e}%
\SetKwInOut{Input}{Input}
\SetKwInOut{Output}{Output}
\SetKw{Breakall}{break all}

\Input{$Z_t$: the multi-set of already selected individuals\newline
$F_{i^*}$: the multi-set of individuals to choose from}
\Output{$\Tilde{F_{i^*}}$ with $|Z_t \cup \Tilde{F_{i^*}}| = N$}
$\Tilde{F_{i^*}} = \emptyset$

$f^n = \textsc{Normalize}(f, Z=Z_t\cup F_{i^*})$ using Algorithm~\ref{alg:normalize}

Associate each individual $x\in Z_t \cup F_{i^*}$ to the reference point $r$ for which the line passing through the origin and $r$ is closest to $f^n(x)$ 

For each reference point $r\in R$, let $\rho_j$ denote the number of individuals in $Z_t$ associated with $r$

\While{true}{
    Let $r_{\min}$ be such that $\rho_{r_{\min}}$ is minimal (break ties randomly) 
    
    Let $x_{r_{\min}}$ be the individual that is associated with $r_{\min}$ and minimizes the distance between $f^n(x_{r_{\min}})$ and $r_{\min}$ (break ties randomly)\footnotemark
    
    \If{$x_{r_{\min}}$ exists}{
        $\Tilde{F_{i^*}} = \Tilde{F_{i^*}} \cup \{x_{r_{\min}}\}$
        
        $\rho_{r_{\min}} = \rho_{r_{\min}} +1 $
    
        \If{$|S_t|+|\Tilde{F_{i^*}}|=N$}{
            \Breakall and \Return $\Tilde{F_{i^*}}$
        }
    }
}
\caption{Selection based on a set $R$ of reference points when maximizing the function $f$}
\label{alg:selection}
\end{algorithm2e}%
\footnotetext{If $\rho_{r_{\min}}>0$, $x_{r_{\min}}$ can be selected in any other diversity-preserving manner from the associated individuals.}
%

For our analyses, we assume that the \NSGA employs a set of structured reference points in the normalized hyper-plane as proposed by Deb and Jain \cite{Deb_Jain_2014}.
In the case of 3 objectives, this corresponds to a set of points in the triangle spanned by $\colvec{1\\0\\0}, \colvec{0\\1\\0},$ and $\colvec{0\\0\\0}$.
Divide the lines between two pairs of these points into $p$ divisions of equal length.
Consider the lines that pass through the start and end points of all divisions and are orthogonal to the respective side. 
Then, a reference point is placed at every intersection of these lines, see Figure~\ref{fig:referencePoints}.
By \cite[Equation~3]{Deb_Jain_2014}, this creates $\binom{3+p-1}{p}=\binom{p+2}{2}$ reference points.
Observe that these reference points partition the non-negative domain of the spanned triangle in regular hexagonal Voronoi cells.
%
\begin{figure}
    \centering
    \includegraphics[width=.5\textwidth]{DebJain2014_Fig1_referencePoints.PNG}
    \caption{Structured set of reference points for 3 objectives with $p=4$ divisions (\cite[Figure~1]{Deb_Jain_2014})}
    \label{fig:referencePoints}
\end{figure}



\section{Memory for Pareto points of \threeOMM}
%ARX \simon{rethink title}
Before analyzing the optimization time of the \NSGA on \threeOMM, we show that, by employing sufficiently many reference points, once the population covers a point on the Pareto front, it is covered for all future iterations.
To this end, we first analyze how the normalization shifts the points on the Pareto front to then conclude that every reference point is associated with at most one point of the Pareto front.
With this, we argue that already sampled points on the Pareto front are never lost again. 

Our analysis assumes that the population is non-degenerated, i.e., that the 3 extreme vectors in each objective span a plane to make the normalization as described by \cite{Deb_Jain_2014} possible. 
However, w.h.p. this holds for the first iteration and thus, by applying our arguments, also for every following iteration.
%ARX \merk{Ev. spaeter mal huebscher machen}

\begin{lemma}\label{lem:normalizedPlane}
Let $Z$ be a set of solutions for \threeOMM and $z_j$ be the minimum value along the $j$th objective in $Z$, $j\in\{1,2,3\}$, and $\zmin=z_1+z_2+z_3$.
Then every objective value $v$ before normalization corresponds to $\frac{1}{n-\zmin}(v-\colvec{z_1\\z_2\\z_3})$ after normalization with respect to $Z$.
\end{lemma}
\begin{proof}
The first normalization step simply moves the plane by subtracting $\colvec{z_1\\z_2\\z_3}$ from every point. 
We call the plane of points on the Pareto front after this step $E'\colon v_1+v_2+v_3 = n - \zmin$.

In the second step, a maximal point for each objective is computed. 
Then, for each point on the Pareto front, each objective value is divided by the intercept of the plane that is spanned by the extreme points with the respective objective axis.
As the complete Pareto front lies in $E'$ after the first step, the plane spanned by the extreme point is exactly $E'$ for all non-degenerated populations.
Computing the intercepts of $E'$ with the objective axes gives that each objective value is divided by $n-\zmin$.
We note that $n-\zmin>0$ for all non-degenerated instances.
\end{proof}

\begin{lemma}\label{lem:uniqueAssociations}
By employing more than $\frac{2n}{\sqrt{3}}$ divisions along each objective, all individuals that are associated with the same reference point have the same objective value.
\end{lemma}
\begin{proof}
Before normalization, the Pareto front lies in the plane $E\colon v_1+v_2+v_3=n$.
Recall that every solution is a Pareto element and thereby has rank 1, so we normalize with respect to the complete population.
Let $z_1,z_2,z_3$ be the minimum values in each objective in the population and let $z_{\min} = z_1+z_2+z_3$.
By Lemma~\ref{lem:normalizedPlane}, for each point we first subtract $\zmin$ from $v_1+v_2+v_3$ and then divide by $n-\zmin$, so the normalized Pareto front lies in the non-negative domain of the plane
$E^n\colon v_1+v_2+v_3=1$.

This is the exact domain in which the reference points are placed. We note that the Voronoi cells, that is, the set of points in the plane which are closest to a given reference point, are all regular hexagons (possibly intersected with the positive quadrant $\R_{\ge 0}^3$).
We prove that at most one point of the normalized Pareto front lies in each cell, which yields the desired statement.
To this end, we show that the diameter of the hexagons is smaller than the distance between any pair of points on the normalized Pareto front.

Let $v \neq w$ be two points on the Pareto front before normalization.
Then they differ in at least 2 objectives by at least 1, so $\norm{v-w}\ge \sqrt{1^2+1^2}=\sqrt{2}$.
By applying Lemma~\ref{lem:normalizedPlane}, this distance of the points after normalization is
\begin{align*}
    \norm{\frac{1}{n-\zmin}\left(v-\colvec{z_1\\z_2\\z_3}\right)-\frac{1}{n-\zmin}\left(w-\colvec{z_1\\z_2\\z_3}\right)}
    = \frac{\norm{v-w}}{n-\zmin} \ge \frac{\sqrt{2}}{n}.
\end{align*}
%
Next, we compute the diameter of the Voronoi cells provided we employed $p$ divisions along each objective. Observe that the length $s$ of the short diagonal of the hexagons is exactly the length of the divisions, so $s= \frac{\sqrt{2}}{p}$.
Let $a$ denote the length of the sides of the hexagons. Since $s = \sqrt 3 a$, we have $a = \sqrt{2/3} / p$. Since the diameter, that is, the length of the long diagonal, satisfies $\ell = 2a$, we conclude that for $p>\frac{2n}{\sqrt{3}}$ we have 
\begin{equation*}
    \ell = 2 \sqrt{2/3} \cdot \frac{1}{p} > \frac{\sqrt{2}}{n}.
\end{equation*}
Then, each cell contains at most one point of the Pareto front.
\end{proof}

We note that $\frac{2n}{\sqrt{3}}+1 > \ceil{\frac{2n}{\sqrt{3}}}$ divisions correspond to 
\begin{align*}
    \binom{3+\frac{2n}{\sqrt{3}}+1-1}{\frac{2n}{\sqrt{3}}+1}
    =\binom{\frac{2n}{\sqrt{3}}+3}{2}\\
   = \frac{1}{2}\left(\frac{2n}{\sqrt{3}}+3\right)\cdot \left(\frac{2n}{\sqrt{3}}+2\right)
    = \frac{2n^2}{3}+\frac{5n}{\sqrt{3}}+3
\end{align*} 
reference points in total.
The Pareto front in comparison contains $(\frac{n}{2}+1)^2=\frac{n^2}{4}+n+1$ points.

\begin{lemma}\label{lem:notLooseSolution}
By employing a population of size $N \ge (\frac{n}{2}+1)^2$ and more than $\frac{2n}{\sqrt{3}}$ divisions along each objective for the reference points, once the population contains a solution for a certain point on the Pareto front it will always contain a solution for this point.
\end{lemma}
\begin{proof}
Observe that for \threeOMM, every solution is non-dominated.
Thus, after the recombination and mutation step, the complete population of old solutions and offsprings contains $2N$ solutions in the first non-dominated level.
After normalizing the objectives, the solutions are associated with their nearest reference point.
All solutions with the same multi-objective values are associated with the same reference point, as they have the same position in the solution space.
At the same time, by Lemma~\ref{lem:uniqueAssociations}, each reference point is associated only with solutions with the same multi-objective value.
Thereby, each reference point with at least one associated solution corresponds to exactly one unique point on the Pareto front.
Therefore, there at most $ (\frac{n}{2}+1)^2$ such reference points.
Because $N\ge (\frac{n}{2}+1)^2$, at least one solution is selected from each reference point with non-empty association set.
Thereby, for each already sampled point on the Pareto front, a solution is selected for the next iteration.
\end{proof}

\section{Runtime of the NSGA-III on \threeOMM}
Knowing that we do not lose points on the Pareto front once sampled, we are able to give a first upper bound on the expected optimization time of the \NSGA on \threeOMM. 

We assume the recombination and mutation step to be such that each individual with a Hamming distance of 1 to any individual in the population is created with a probability in $\Omega(\frac{1}{n})$. 
For example, this is achieved if each individual in the population has at least a constant probability to produce an offspring by applying standard uniform mutation in each iteration.

\begin{theorem}\label{thm:runTime3OMM}
Consider the \NSGA with population size $N \ge (\frac{n}{2}+1)^2$ and more than $\frac{2n}{\sqrt{3}}$ divisions along each objective  optimizing \threeOMM.
Let $T$ denote the number of iterations until the population covers the Pareto front.
Then, $E[T]\in O(n^3)$.
\end{theorem}
\begin{proof}
As long as the complete Pareto front is not yet sampled, the population contains at least one bitstring $s$ with a Hamming distance of 1 to a bitstring $s'$ of a not yet sampled point on the Pareto front in each iteration.
By our assumption on the recombination and mutation step, $s'$ is sampled with a probability in $\Omega(\frac{1}{n})$.
Thus, it takes $O(n)$ iterations in expectation to sample the next point on the Pareto front.
By Lemma~\ref{lem:notLooseSolution} the population will never loose a sampled point on the Pareto front.
Thereby, after an expected number of at most $O(n(\frac{n}{2}+1)^2)=O(n^3)$ iterations, the complete Pareto front is sampled.
\end{proof}

We note that this analysis pessimistically assumes that the Pareto front is sampled in some defined order. 
We are optimistic to improve the bound by employing more sophisticated arguments.
%ARX \simon{vlg. prev. work}

%ARX \mtodo{Investigate how \cite{BianQT18ijcaigeneral} are able to get rid of the log-factor for $m>4$ of SEMO on \mcocz and see whether we can do sth similar}

	%
%\begin{algorithm2e}%
	%Initialize $P_0$ with $\mu$ individuals chosen independently and uniformly at random from $\{0,1\}^n$\;
	%\For{$t = 1, 2, \ldots$}{
    %\For{$i \in [1..\lambda]$}{
      %Select $x_i \in P_{t-1}$ uniformly at random\;
      %Generate $y_i$ from $x_i$ via standard bit mutation\;
      %}
    %Select $P_t$ from the multi-set $\{y_1, \ldots, y_\lambda\}$ by choosing $\mu$ best individuals (breaking ties arbitrarily)\;
  %}
%\caption{The \mclea to maximize a function $f : \{0,1\}^n \to \R$.}
%\label{alg:algo}
%\end{algorithm2e}
%
%
%
%Now the $n$-dimensional \emph{jump function} with \emph{jump parameter (jump size)} $k \in [1..n]$ is defined by
%\[
%\jump_{nk}(x) = 
%\begin{cases}
%\|x\|_1+k & \mbox{if $\|x\|_1 \in [0..n-k] \cup \{n\}$,}\\
%n - \|x\|_1 & \mbox{if $\|x\|_1 \in [n-k+1\, ..\, n-1]$}.
%\end{cases}
%\]
%


}%sloppy, please do not remove

\bibliographystyle{alphaurl}
\bibliography{ich_master,alles_ea_master,rest}

%\end{large}
\end{document}