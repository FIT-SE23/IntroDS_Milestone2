\subsection{The NSGA-III}
The main structure of the \NSGA \cite{Deb_Jain_2014} is identical to the one of the \NSGAtwo \cite{DebPAM02}.
It is initialized with a random population of size $N$.
In each iteration, the user applies mutation and/or crossover operators to generate an offspring population of size $N$.
As the NSGA framework is an MOEA with a fixed population size, out of this total of $2N$ individuals, $N$ have to be selected for the next iteration.

Because non-dominated solutions are to be preferred, the following ranking scheme is used to set the dominance relation as the predominant criterion for the survival of individuals. 
Individuals that are not strictly dominated by any other individual in the population obtain rank 1.
Recursively, the other ranks are defined. Each individual that has not yet been ranked and is only strictly dominated by individuals of rank \(1,\ldots,k-1\) is assigned rank $k$. 
Clearly, an individual is more interesting the lower its rank is. 
Let $F_i$ denote the set of individuals with rank $i$ and let $i_0$ be minimal such that $\sum_{i=1}^{i_0} |F_i| \ge N$.
All individuals with rank at most $i_0-1$ survive into the next generation.
Further, $0<k\le N$ individuals of rank $i_0$ have to be selected for the next generation such that the new population is again of size $N$, and the next iteration can begin. 
The only difference between the \NSGAtwo and the \NSGA is the procedure of selecting the $k$ individuals of rank $i_0$.
While the \NSGAtwo employs crowding-distance, the \NSGA uses reference points, typically distributed in some structured manner on the normalized hyper-plane, in order to select a diverse population.
For the whole framework, see Algorithm~\ref{alg:nsga3}. 
Note that whenever we refer to sets of individuals, we are actually referring to multi-sets as each solution might be represented multiple times in the population.
%
\begin{algorithm2e}%
Let the initial population $P_0$ be composed of $N$ individuals chosen independently and uniformly at random from $\{0,1\}^n$.

\For{$t = 0, 1, 2, \ldots$}{
Generate the offspring population $Q_t$ with size $N$

Use fast-non-dominated-sort() from Deb et al.\ \cite{DebPAM02}) to divide $R_t = P_t \cup Q_t$ into $F_1, F_2, \ldots$

Find $i^* \ge 1$ such that $\sum_{i=1}^{i^*-1} |F_i| < N$ and $\sum_{i=1}^{i^*} |F_i| \ge N$

$Z_t = \bigcup_{i=1}^{i_0-1}F_i$

Select $\Tilde{F_{i^*}}\subseteq F_{i^*}$ such that $|Z_t\cup\Tilde{F_{i^*}}| = N$ (use crowding-distance for \NSGAtwo and Algorithm~\ref{alg:selection} for \NSGA)

$P_{t+1} = Z_t \cup \Tilde{F_{i^*}}$
}
\caption{NSGA-II and NSGA-III}
\label{alg:nsga3}
\end{algorithm2e}%

In order to select individuals from the critical rank $i_0$, the \NSGA normalizes the objective functions and associates each individual with a reference point.

Regarding the normalization, let $z_j$ be the minimum value in the $j$th objective among all individuals of rank at most $i_0$.
Then, $z_j$ is subtracted from each objective function $f_j$ to obtain $f_j'$.
For each objective $j$, let $x_j$ be an individual with rank at most $i_0$ that maximizes $f'_j(x)$ and let $z_j^{\max} = f'(x)$.
We refer to the $j$th objective value of the point at the intersection of a hyper-plane and the $j$th objective axis as an \emph{intercept}.
Then, each objective is divided by the intercept of the hyper-plane spanned by $z_1^{\max},z_2^{\max},\ldots$ with the respective objective axis to obtain the normalized objective function $f^n$.
Algorithm~\ref{alg:normalize} formalizes the normalization procedure.
%
\begin{algorithm2e}%
\SetKwInOut{Input}{Input}
\SetKwInOut{Output}{Output}

\Input{%
$f=(f_1,\ldots, f_M)$: objective function\newline
$Z\subseteq \{0,1\}^n$: a multi-set of individuals}
\Output{$f^n=(f_1^n,\ldots, f_M^n)$: normalized objective function}
\For{$j=1$ \KwTo $M$}{
    $z_j = \min_{z\in Z} f_j(z)$\\
    $f_j'(x) = f_j(x) - z_j \quad \forall x \in Z$\\
    $z_j^{\max} = f(\arg\max_{x\in Z} f'_j(x))$
}
Let $H$ be the hyper-plane spanned by $z_1^{\max},\ldots, z_M^{\max}$  

\For{$j=1$ \KwTo $M$}{
    $a_j = $ the intercept of $H$ with the $j$th objective axis
    
    $f_j^n(x) = \frac{f_j'(x)}{a_j} \quad \forall x\in Z$
}
\caption{Normalization}
\label{alg:normalize}
\end{algorithm2e}%

After the normalization, each individual of rank at most $i_0$ is associated with its closest reference point with respect to the normalized objectives.
Then, one iterates through the reference points, always selecting the one with the fewest associated individuals that are already selected for the next generation. Ties are resolved randomly.
If the reference point has no associated individuals that are not yet selected, it is skipped.
Otherwise, the individual that is closest to the reference point (with respect to the normalized objective function) is selected for the next generation. 
Once more, ties are resolved randomly.
If the next generation already contains an individual that is associated with the reference point, other measures than the distance to the reference point can be considered.
The selection terminates as soon as the required number of individuals is reached.
This procedure is formalized in Algorithm~\ref{alg:selection}.
%
\begin{algorithm2e}%
\SetKwInOut{Input}{Input}
\SetKwInOut{Output}{Output}
\SetKw{Breakall}{break all}

\Input{$Z_t$: the multi-set of already selected individuals\newline
$F_{i^*}$: the multi-set of individuals to choose from}
\Output{$\Tilde{F_{i^*}}$ with $|Z_t \cup \Tilde{F_{i^*}}| = N$}
$\Tilde{F_{i^*}} = \emptyset$

$f^n = \textsc{Normalize}(f, Z=Z_t\cup F_{i^*})$ using Algorithm~\ref{alg:normalize}

Associate each individual $x\in Z_t \cup F_{i^*}$ to the reference point $r$ for which the line passing through the origin and $r$ is closest to $f^n(x)$ 

For each reference point $r\in R$, let $\rho_j$ denote the number of individuals in $Z_t$ associated with $r$

\While{true}{
    Let $r_{\min}$ be such that $\rho_{r_{\min}}$ is minimal (break ties randomly) 
    
    Let $x_{r_{\min}}$ be the individual that is associated with $r_{\min}$ and minimizes the distance between $f^n(x_{r_{\min}})$ and $r_{\min}$ (break ties randomly)\footnotemark
    
    \If{$x_{r_{\min}}$ exists}{
        $\Tilde{F_{i^*}} = \Tilde{F_{i^*}} \cup \{x_{r_{\min}}\}$
        
        $\rho_{r_{\min}} = \rho_{r_{\min}} +1 $
    
        \If{$|S_t|+|\Tilde{F_{i^*}}|=N$}{
            \Breakall and \Return $\Tilde{F_{i^*}}$
        }
    }
}
\caption{Selection based on a set $R$ of reference points when maximizing the function $f$}
\label{alg:selection}
\end{algorithm2e}%
\footnotetext{If $\rho_{r_{\min}}>0$, $x_{r_{\min}}$ can be selected in any other diversity-preserving manner from the associated individuals.}
%

For our analyses, we assume that the \NSGA employs a set of structured reference points in the normalized hyper-plane as proposed by Deb and Jain \cite{Deb_Jain_2014}.
In the case of 3 objectives, this corresponds to a set of points in the triangle spanned by $\colvec{1\\0\\0}, \colvec{0\\1\\0},$ and $\colvec{0\\0\\0}$.
Divide the lines between two pairs of these points into $p$ divisions of equal length.
Consider the lines that pass through the start and end points of all divisions and are orthogonal to the respective side. 
Then, a reference point is placed at every intersection of these lines, see Figure~\ref{fig:referencePoints}.
By \cite[Equation~3]{Deb_Jain_2014}, this creates $\binom{3+p-1}{p}=\binom{p+2}{2}$ reference points.
Observe that these reference points partition the non-negative domain of the spanned triangle in regular hexagonal Voronoi cells.
%
\begin{figure}
    \centering
    \includegraphics[width=.5\textwidth]{DebJain2014_Fig1_referencePoints.PNG}
    \caption{Structured set of reference points for 3 objectives with $p=4$ divisions (\cite[Figure~1]{Deb_Jain_2014})}
    \label{fig:referencePoints}
\end{figure}

