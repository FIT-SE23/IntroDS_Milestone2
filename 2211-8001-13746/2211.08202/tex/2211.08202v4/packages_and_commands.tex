
\hyphenation{analysis onemax Doerr parameter leadingones Hoeffding develop-ment Krejca Carola}

\usepackage{amsxtra, amsfonts, amssymb, amstext, amsmath, mathtools}%, 
\usepackage{amsthm}
\usepackage{booktabs}
\usepackage{nicefrac}
\usepackage{xspace}
\usepackage{url}\urlstyle{rm}
\usepackage{graphics,color}
\usepackage[algo2e,ruled,vlined,linesnumbered]{algorithm2e}
    \SetKwInOut{Input}{Input}
    \SetKwInOut{Output}{Output}
    \ResetInOut{output}
    \SetKw{Break}{break}
    \SetKw{Breakall}{break all}
\usepackage{wrapfig}
\usepackage{lmodern}


\renewcommand{\labelenumi}{\theenumi}
\renewcommand{\theenumi}{(\roman{enumi})}

%\allowdisplaybreaks[4]
\clubpenalty=10000
\widowpenalty=10000
\frenchspacing 

\newtheorem{theorem}{Theorem}
\newtheorem{lemma}[theorem]{Lemma}
\newtheorem{corollary}[theorem]{Corollary}
\newtheorem{definition}[theorem]{Definition}



% Mathematic abbreviations
\newcommand{\oea}{\mbox{${(1 + 1)}$~EA}\xspace}
\newcommand{\oeamu}{${(1 + 1)}$~EA$_{\mu,p}$\xspace}
\newcommand{\ooea}{\oea}
\newcommand{\olea}{${(1\overset{+}{,}\lambda)}$~EA\xspace}
\newcommand{\oplea}{\mbox{${(1+\lambda)}$~EA}\xspace}
\newcommand{\mpoea}{\mbox{${(\mu+1)}$~EA}\xspace}
\newcommand{\mplea}{\mbox{${(\mu+\lambda)}$~EA}\xspace}
\newcommand{\mclea}{\mbox{${(\mu,\lambda)}$~EA}\xspace}
\newcommand{\oclea}{\mbox{${(1,\lambda)}$~EA}\xspace}
\newcommand{\opllga}{\mbox{${(1+(\lambda,\lambda))}$~GA}\xspace}
\newcommand{\ollga}{\opllga}
\newcommand{\NSGAtwo}{NSGA\nobreakdash-II\xspace}
\newcommand{\NSGA}{NSGA\nobreakdash-III\xspace}

\newcommand{\OM}{\textsc{OM}\xspace}
\newcommand{\onemax}{\textsc{OneMax}\xspace}
\newcommand{\zeromax}{\textsc{ZeroMax}\xspace}
\newcommand{\LO}{\textsc{Leading\-Ones}\xspace}
\newcommand{\leadingones}{\LO}
\newcommand{\needle}{\textsc{Needle}\xspace}
\newcommand{\cliff}{\textsc{Cliff}\xspace}
\newcommand{\binval}{\textsc{BinVal}\xspace}
\newcommand{\plateau}{\textsc{Plateau}\xspace}
\newcommand{\jump}{\textsc{Jump}\xspace}
\newcommand{\oneminmax}{\textsc{OneMinMax}\xspace}
\newcommand{\OMM}[1]{\text{$#1$-\textsc{OMM}}\xspace}
\newcommand{\threeOMM}{\textsc{3-OMM}\xspace}
\newcommand{\cocz}{\textsc{COCZ}\xspace}
\newcommand{\lotz}{\textsc{LOTZ}\xspace}
\newcommand{\ojzj}{\textsc{OneJumpZeroJump}\xspace}
\newcommand{\mlotz}{$m$\textsc{LOTZ}\xspace}
\newcommand{\mcocz}{$m$\textsc{COCZ}\xspace}


\DeclareMathOperator{\rand}{rand}
\DeclareMathOperator{\Sample}{Sample}
\DeclareMathOperator{\minmax}{minmax}
\DeclareMathOperator{\paral}{par}
\DeclareMathOperator{\poly}{poly}

\newcommand{\gmax}{g_{\max}}
\newcommand{\Ymax}{Y_{\max}}
\newcommand{\X}{\mathcal{X}}

%\newcommand{\C}{\ensuremath{\mathbb{C}}}
\newcommand{\R}{\ensuremath{\mathbb{R}}}
\newcommand{\Q}{\ensuremath{\mathbb{Q}}}
\newcommand{\N}{\ensuremath{\mathbb{N}}} % ohne Null!!!
\newcommand{\Z}{\ensuremath{\mathbb{Z}}}
\newcommand{\bbone}{{\mathbbm{1}}}


\newcommand{\calA}{\ensuremath{\mathcal{A}}} 
\newcommand{\calE}{\ensuremath{\mathcal{E}}} 
\newcommand{\calF}{\ensuremath{\mathcal{F}}} 
\newcommand{\calP}{\ensuremath{\mathcal{P}}} 
\newcommand{\calS}{\ensuremath{\mathcal{S}}} 
\newcommand{\calT}{\ensuremath{\mathcal{T}}} 
\newcommand{\calX}{\ensuremath{\mathcal{X}}} 
\newcommand{\calY}{\ensuremath{\mathcal{Y}}} 
% no \calO for big-Oh	
	
\DeclareMathOperator{\Bin}{Bin}
\DeclareMathOperator{\Geom}{Geom}
\DeclareMathOperator{\arcsinh}{arcsinh}
\DeclareMathOperator{\mutate}{mutate}
\newcommand{\fmin}{f_{\mathrm{min}}}
\newcommand{\fmax}{f_{\mathrm{max}}}
\newcommand{\xmin}{x_{\mathrm{min}}}
\newcommand{\xmax}{x_{\mathrm{max}}}
\newcommand{\zmin}{z^{\min}}
\newcommand{\zmax}{z^{\max}}

% use \Pr[...] for probability
% use E[...] for expectation
\newcommand{\Var}{\mathrm{Var}\xspace} %use with [...]
\newcommand{\Cov}{\mathrm{Cov}\xspace} %use with [...] 
\newcommand{\eps}{\varepsilon} 

\newcommand{\merk}[1]{\textbf{\textcolor{red}{#1}}}

\newcommand{\assign}{\leftarrow}

\let\originalleft\left
\let\originalright\right
\renewcommand{\left}{\mathopen{}\mathclose\bgroup\originalleft}
\renewcommand{\right}{\aftergroup\egroup\originalright}

\newcommand{\simon}[1]{\textcolor{blue}{SW: #1}}
\newcommand{\mtodo}[1]{\textcolor{red}{#1}}
\newcommand{\nootherflip}{\left(1-\frac{1}{n}\right)^{n-1}}
\newcommand{\colvec}[2][.6]{%
  \scalebox{#1}{%
    \renewcommand{\arraystretch}{.6}%
    $\begin{pmatrix}#2\end{pmatrix}$%
  }
}
\DeclarePairedDelimiter\floor{\lfloor}{\rfloor}
\DeclarePairedDelimiter\ceil{\lceil}{\rceil}
\newcommand{\abs}[1]{\left| #1\right|\xspace}
\newcommand{\norm}[1]{\left\lVert #1 \right\rVert_2}

\let\oldsqrt\sqrt
\def\hksqrt{\mathpalette\DHLhksqrt}
\def\DHLhksqrt#1#2{\setbox0=\hbox{$#1\oldsqrt{#2\,}$}\dimen0=\ht0
   \advance\dimen0-0.2\ht0
   \setbox2=\hbox{\vrule height\ht0 depth -\dimen0}%
   {\box0\lower0.4pt\box2}}
\renewcommand\sqrt\hksqrt

\DeclareMathOperator{\refp}{rp}
\DeclareMathOperator{\nad}{nad}
\DeclareMathOperator{\mTrue}{True}
\DeclareMathOperator{\mFalse}{False}
\DeclareMathOperator{\valid}{valid}


\usepackage{hyperref}


\hyphenation{analysis onemax oneminmax Doerr parameter leadingones leadingones-trailingzeroes Hoeffding develop-ment Krejca Carola NSGA}


\renewcommand{\labelenumi}{\theenumi}
\renewcommand{\theenumi}{(\roman{enumi})}

%\allowdisplaybreaks[4]
\clubpenalty=10000
\widowpenalty=10000
\frenchspacing 

\usepackage{amsxtra, amsthm, amsfonts, amssymb, amstext, amsmath, mathtools}%, 
\usepackage{booktabs}
\usepackage{cite}
\usepackage{nicefrac}
\usepackage{xspace}
\usepackage{url}\urlstyle{rm}
\usepackage{graphics,color}
\usepackage[algo2e,ruled,vlined,linesnumbered]{algorithm2e}
    \SetKwInOut{Input}{Input}
    \SetKwInOut{Output}{Output}
   % \ResetInOut{output}
    \SetKw{Break}{break}
    \SetKw{Breakall}{break all}
\usepackage{wrapfig}
\usepackage{lmodern}
\usepackage{pgf}


\usepackage{placeins}

\renewcommand{\labelenumi}{\theenumi}
\renewcommand{\theenumi}{(\roman{enumi})}

%\allowdisplaybreaks[4]
\clubpenalty=10000
\widowpenalty=10000
\frenchspacing 

