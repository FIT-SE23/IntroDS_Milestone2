\documentclass[12pt]{article}
 \usepackage[margin=1in]{geometry}
\usepackage{amsmath,amssymb,amsthm,graphicx}
%\graphicspath{ {./static/} }
\usepackage{url}


\usepackage{comment}
\usepackage[utf8]{inputenc}

\usepackage{makecell}
\usepackage{xcolor}
\usepackage{caption}
\usepackage{subcaption}
\usepackage[ruled,vlined]{algorithm2e}

\usepackage[shortlabels]{enumitem}
%\usepackage[authoryear]{natbib}
\usepackage{bbm}
%\usepackage[natbib=true,style=apa,backend=biber]{biblatex}
\usepackage{natbib}
\setcitestyle{authoryear, open={(}, close={)}}


\DeclareMathOperator*{\argmax}{arg\,max}
\DeclareMathOperator*{\argmin}{arg\,min}

\newcommand{\Real}{\hbox{{I}\kern-.1667em\hbox{R}}}  
\newcommand{\vect}[1]{\mbox{\boldmath $ #1$}}   
\newcommand{\combos}[2]{\left(\mbox{\small $\begin{array}{c}  
        {#1}\\{#2} \end{array}$} \right)}   
\newcommand{\Var}{{\rm Var}\,}  
\newcommand{\Pb}{\mbox{{I}\kern-.1667em\mbox{P}}}  
\newcommand{\Ex}{\mbox{{I}\kern-.1667em\mbox{E}}}  



\theoremstyle{definition}
\newtheorem{definition}{Definition}


% NOTE: To produce blinded version, replace "1" with "0" below.
\newcommand{\blind}{1}

\begin{document}


\def\spacingset#1{\renewcommand{\baselinestretch}%
{#1}\small\normalsize} \spacingset{1}


%%%%%%%%%%%%%%%%%%%%%%%%%%%%%%%%%%%%%%%%%%%%%%%%%%%%%%%%%%%%%%%%%%%%%%%%%%%%%%

\if1\blind
{
  \title{\bf Time-varying Bayesian Network Meta-Analysis}
  \author{Patrick LeBlanc\thanks{Email: patrick.leblanc@duke.edu}\\
    Department of Statistical Sciences, Duke University\\
    and \\
    David Banks\\
    Department of Statistical Sciences,Duke University}
  \maketitle
} \fi

\if0\blind
{
  \bigskip
  \bigskip
  \bigskip
  \begin{center}
    {\LARGE\bf Title}
\end{center}
  \medskip
} \fi

\bigskip
\begin{abstract}
The presence of methicillin-resistant \textit{Staphylococus Aureus} (MRSA) in complicated skin and soft structure infections (cSSSI) is associated with greater health risks and economic costs to patients.  There is concern that MRSA is becoming resistant to other ``gold standard” treatments such as vancomycin.  While there are a number of review papers employing Bayesian Network Meta-Analyses (BNMAs) to investigate which treatments are best used to treat MRSA related cSSSIs, none have investigated whether the efficacy of treatments changes over time.  This paper proposes two novel BNMA methods: Sig-BNMA, which allows treatments to follow a biologically-plausible sigmoidal time curve, and GP-BNMA, which models time effects non-parametrically.  In a simulation environment, both proposed methods can detect time-varying trends which existing methods cannot.  A dataset was agglomerated from nine existing review MRSA cSSSI review papers.  It contains $58$ studies comparing $19$ treatments over $19$ years.  Sig-BNMA and GP-BNMA found all treatments to be approximately as effective at the end of the time-period as at the beginning.  However, GP-BNMA found evidence of non-linear trends for linezolid, tedizolid, telavancin, and tigecycline; their efficacy relative to vancomycin increased until $2010$, after which it declined.  This is consistent with observations about vancomycin resistant MRSA in the literature. 
\end{abstract}

\noindent%
{\it Keywords:}  Bayesian inference, Bayesian Network Meta-Analysis (BNMA), Gaussian Process, MRSA
\vfill

\newpage
\spacingset{2}

\section{Introduction}

Methicilin-resistant \textit{Staphylococcus aureus} (MRSA) infections are a threat to public health.  MRSA causes increased mortality, longer hospital stays, and greater cost burden compared to non-methicillin-resistant \text{S. aureus} \citep{2006_Crum_etal, 2007_McCollum_etal, 2012_Shorr}.  The incidence of MRSA rose globally in the late $1900$'s and early $2000$'s \citep{2008_Hersh_etal}; the SENTRY antimicrobial surveillance program, for instance, observed increasing prevalence of MRSA in complicated skin and soft structure infections (cSSSI) \citep{2007_Moet_etal}.  More recent findings suggest that MRSA prevalence peaked in $2008$ and has been declining since in the EU and the United States (US) \citep{2017_Klein_etal, 2019_Diekema_etal}: see Figure \ref{fig:2019_diekema_etal_mrsa_prev} for a plot of MRSA prevalence among bloodstream infections (BSIs) over time. Yet MRSA is the second most common cause of antibiotic-resistant bacterial infections in the European Union (EU) \citep{2019_Gasser_Etal} and remains stable in the Asia-Pacific region \citep{2019_Lim_etal}.  The prevalence of MRSA is still concerning.

Moreover, growing antibiotic resistance in MRSA is a potential problem \citep{2009_Wilcox, 2009_Nathwani}.  \textit{S. aureus} is possibly developing resistance to other treatments, such as fusidic acid and mupirocin \citep{2021_Brown_etal}.  In particular, the Infectious Disease Society of America (IDSA) has long recommended vancomycin as a treatment for MRSA \citep{2012_Gould_etal}, and vancomycin can be regarded as the ``gold standard" of MRSA treatments \citep{2012_Shorr}.   It has been reported that the increase in MRSA prevalence resulted in the increasing use of vancomycin and and the emergence of vancomycin resistant \textit{S. aureus} \citep{2007_Daum, 2004_Cosgrove_etal}.  \cite{2019_Diekema_etal}, in contrast, finds that there was no increase in vancomycin-resistant MRSA from 2013-2016.  There remains an ``evidence gap" with respect to vancomycin-resistant \textit{S. aureus} \citep{2021_Brown_etal}.  

A number of randomized controlled trials (RCT) have been conducted assessing the effectiveness of treatments for MRSA-related cSSSIs.  These studies provide a mix of direct and indirect evidence for a number of treatments, so Bayesian network meta-analyses (BNMA) have been employed to infer treatment effects \citep{2015_Thom_etal, 2016_Liu_etal, 2017_Guest_etal, 2017_McCool_etal, 2018_Li_Xu, 2019_Zhang_etal, 2019_Lan_etal, 2021_Brown_etal, 2021_Feng_etal}.  If MRSA is developing antibiotic resistance, however, treatment effects for different treatments would vary over time: a type of design consistency which no studies have explicitly addressed.  

To investigate whether \textit{S. aureus} is developing vancomycin-resistance, this paper will develop a class of Bayesian Network Meta-Analysis (BNMA) models which can detect time-varying treatment effects: two such models are developed.  The first, Sigmoidal BNMA (Sig-BNMA) automatically detects whether a time-varying effect follows a sigmoidal pattern or a constant one.  The second, Gaussian Process BNMA (GP-BNMA) fits a Gaussian Process regression over time, taking the average treatment effect as the mean.  In simulations, both of these models detect time-varying treatment effects which standard BNMA and BNMA with a meta-regression on time effects (here termed Meta-BNMA) cannot adequately detect.

The datasets of \cite{2015_Thom_etal, 2016_Liu_etal, 2017_Guest_etal, 2017_McCool_etal, 2018_Li_Xu, 2019_Zhang_etal, 2019_Lan_etal, 2021_Brown_etal, 2021_Feng_etal} are combined to form one agglomerated MRSA-cSSSI dataset including $58$ studies comparing $19$ treatments from $2000$ to $2019$.  Standard BNMA, Meta-BNMA, Sig-BNMA, and GP-BNMA are fit to this data.  There is no evidence of a linear or sigmoidal trend, but GP-BNMA detects that vancomyicn-resistance in MRSA peaked in $2010$. 

\section{Bayesian Network Meta-Analysis}

There are a large variety of treatment options available for any medical condition.  In any given randomized conrolled trial (RCT), researchers can compare only a subset of those possible treatments.  To know whether a given treatment, A, is more or less effective than another treatment, B, then, there is a mix of direct and indirect evidence.  When there are only three treatments with two pairwise comparisons --- $A$ compared to $B$ and $B$ compared to $C$ --- then analysis is straightforward \citep{1997_Bucher_etal}. However, situations of greater complexity arise: one trial might compare treatments $A$ and $B$, a second might compare $B$ and $C$, a third might compare $C$ and $D$, and a fourth might compare $A$, $B$, and $D$.  This induces a network of comparisons amongst the treatments; models developed to estimate the treatment effects are referred to as Network Meta-Analyses (NMA). Frequentist NMA's have been developed by \cite{1996_Higgins_Whitehead, 2002_Lumley,2008_Chootrakool_Shi} while Bayesian NMA's have been developed by \cite{2003_Ades,2004_Lu_Ades, 2006_Lu_Ades}. The formulation of \cite{2011_Dias_Tech_Sup} for BNMAs with binomial data is followed here. 

Let there be $I$ studies comparing $K$ total treatments.  The response variable, the number of successes, for study $i$ and treatment $k$ is $y_{ik}$.  Each $y_{ik}$ will have a probability of success $p_{ik}$ and number of patients $n_{ik}$, and follow a binomial likelihood: $y_{ik} \, | \, p_{ik}, n_{ik} \sim \text{Bin}(p_{ik},n_{ik})$. The number of trials $n_{ik}$ is given for each study $i$.  The probability is modelled with a logit-link function: $\text{logit}(p_{ik}) = \mu_{i} + \delta_{i,b_i,k}1_{b_i\neq k}$. Here, $b_i$ is the baseline treatment in study $i$.  If desirable, all studies would have the same baseline, $b$,  but this is not always be possible and so each study, $i$, will have its own baseline, $b_i$.  The most common treatment is a natural choice for the baseline.  The baseline effects of trial $i$ are captured by $\mu_{i}$.  These are nuisance parameters and are modelled as random effects, $\mu_i \sim N(m_\mu,\sigma_\mu^2)$, for some prior mean $m_\mu$ and variance $\sigma_\mu^2$.  The $\mu_i$ allow BNMA to accurately find the mean effect of each treatment $d_{1k}$ even when there are significant unknown confounding effects between studies, such as time-varying effects.

The difference in efficacy between treatment $k$ and treatment $b_i$ in study $i$ is modelled by $\delta_{i,b_i,k}$.  In a random effects model, it is drawn from a normal distribution, $\delta_{i,b_i,k} \, | \, d_{b_i,k},\sigma^2 \sim \text{N}(d_{b_i,k},\sigma^2)$. Homogeneity of variance is assumed because there is not enough data to learn heterogenous variance \citep{1996_Higgins_Whitehead}: every baseline $b_i$ and treatment $k$ have the same variance across studies ($\sigma_{b_i,k}^2 = \sigma^2)$.   Setting $\sigma^2 = 0$ instead corresponds to a fixed effects model.  \cite{2021_Rosenberger_etal} compared different commonly used prior specifications --- inverse-gamma, uniform, and half-normal --- and found that the prior choice had little effect on point estimates.  A uniform prior is thus placed on $\sigma^2$, $\sigma^2 \sim \text{Unif}(0,5)$.   In multi-arm trials, it is desirable to have the correlation between different treatments be equal to $0.5$.   This induces the following multivariate normal:
\[
\begin{bmatrix}
\delta_{i,b_i,2}\\
\delta_{i,b_i,3}\\
\dots\\
\delta_{ib_i,k-1}
\end{bmatrix}
\sim 
N\bigg(
\begin{bmatrix}
d_{b_i,2}\\
d_{b_i,3}\\
\dots\\
d_{b_i,k-1}
\end{bmatrix},
\begin{bmatrix}
\sigma^2 & \frac{\sigma^2}{2} & \dots & \frac{\sigma^2}{2}\\
\frac{\sigma^2}{2} & \sigma^2 & \dots & \frac{\sigma^2}{2}\\
\dots & \dots & \dots & \dots\\
\frac{\sigma^2}{2} & \frac{\sigma^2}{2} & \dots & \sigma^2
\end{bmatrix}
\bigg)
\]
It is more efficient to decompose this joint likelihood into a product of conditional likelihoods:
\[
\begin{aligned}
\delta_{i,b_i,k} \, &| \, \delta_{i,b_i,2},\dots,\delta_{i,b_i,k-1},
d_{b_i,2},\dots,d_{b_i,k-1}, \sigma^2 \\
& \sim \text{N}\bigg(d_{b_i,k} + \frac{1}{k-1}\sum_{j=1}^{k-1}[\delta_{i,b_i,j} - d_{b_i,j}],\frac{k}{(k-1)}\sigma^2\bigg)
\end{aligned}
\]
The mean difference between the baseline $b_i$ and treatment $k$ is $d_{b_i,k}$.  Under the consistency assumption \citep{2006_Lu_Ades} (also called coherence in \cite{2002_Lumley}), the difference in treatment between $b_i$ and $k$ can be decomposed into the difference between $b_i$ and treatment $1$ (which may be taken as the general baseline $b$) and the difference between $k$ and $1$.  That is, $d_{b_i,k} = d_{1k} - d_{1b_i}$.  Further, these baseline differences are drawn from a normal distribution: $d_{1k} \sim N(m_d,\sigma_d^2)$, for some prior mean $m_d$ and variance $\sigma_d^2$.  $d_{11},d_{12},\dots,d_{1k}$ are called basic parameters while the $d_{b_i,k}$ are called functional parameters.

Taken together, the contrast-based BNMA model with binomial outcomes for each arm is
\[
\begin{aligned}
y_{ik} \, | \, p_{ik}, n_{ik} &\sim \text{Bin}(p_{ik},n_{ik}) 
&\text{logit}(p_{ik}) &= \mu_{i} + \delta_{i,b_i,k}1_{b_i\neq k} \\
\delta_{i,b_i,k} \, &| \, \delta_{i,b_i,2},\dots,\delta_{i,b_i,k-1},
d_{b_i,2},\dots,d_{b_i,k-1}, \sigma^2 \\
&\sim \text{N}\bigg(d_{b_i,k} + \frac{1}{k-1}\sum_{j=1}^{k-1}[\delta_{i,b_i,j} - d_{b_i,j}],\frac{k}{(k-1)}\sigma^2\bigg) \\
\mu_i \, | \, m_\mu, sd_\mu &\sim N(m_\mu,\sigma_\mu^2) 
&m_\mu &\sim \text{N}(0,10000)  \\
\sigma_\mu &\sim \text{Unif}(0,5) 
&d_{b_i,k} &= d_{1k} - d_{1b_i} \\
\sigma^2 &\sim  \text{Unif}(0,5) 
&d_{1k} &\sim \text{N}(m_d,\sigma_d^2) \\
m_d &\sim \text{N}(0,10000) 
&\sigma_d &\sim \text{Unif}(0,5)
\end{aligned}
\]

\section{Time-Varying Bayesian Network Meta-Analysis}

\cite{2012_Higgins_etal} outlines two situations in which the consistency assumption could be violated: loop inconsistency occurs when direct and indirect evidence disagree while design inconsistency occurs when the choice of treatments in a study is associated with different effect sizes. An unexplored type of design inconsistency is time-varying effects, e.g. if a disease develops antibiotic resistance.  Moreover, different treatments might have different trends in time.  Thus far, there has been little literature explicitly examining time-varying effects, though such design inconsistencies could be addressed with standard meta-regression techniques \citep{2012_White_etal}. \cite{2009_Salanti_etal} employs a time meta-regression in a BNMA to study the effectiveness of oral health interventions and found that the placebo treatments became more effective over time.  There is an important constraint to this meta-regression, however --- \cite{2009_Salanti_etal} only allows for the baseline treatments become more equally effective relative to other treatments.  Moreover, existing meta-regression BNMA techniques are limited to linear effects.  If treatments varied in time according to another pattern, such as biologically plausible sigmoidal functions or higher level polynomials, meta-regression techniques will be of limited utility.  A model which allows all treatments effects to vary in time and which is capable of detecting nonlinear trends is needed. 

To this end, notation for time-varying effects will be introduced.  Recall that $d_{1k}$ is the basic parameter capturing the effect of treatment $k$ relative to the baseline treatment $1$, and that under consistency the collection of parameters $\{d_{12},\dots,d_{1k}\}$ fully characterize treatment effects.  The set of studies present in a dataset can be indexed by $i\in\{1,2,\dots,I\}$.  The timepoint that study $i$ occurred in is $t_i$, so that the list of possibly non-unique timepoints is $t_1,t_2,\dots,t_I$.  Treatment $k$ occurs in $I_k$ studies, and the list of studies it occurs in can be indexed by $i_k$.  The timepoints that treatment $k$ occurs in can thus be indexed by $t_{ik}$.  We model a time-specific value of $d_{1k}$, $d_{1k}^{t_{ik}}$, at each of these timepoints. Note further that it is only worth modelling treatments as time-varying if there are enough datapoints to make meaningful inference.  Let $\mathcal{T}_0$ be the set of treatments which appear less than $5$ times in the dataset, and let $\mathcal{T}_1$ be the set of treatments which appear at least $5$ times.  Treatments $k \in \mathcal{T}_0$ are modelled as constant and treatments $k\in\mathcal{T}_1$ as time-varying.  The introduction of the $d_{1k}^{tk}$ further necessitates a modification of the $\delta_{i,b_i,k}$.  That is, for $d_{b_i,k} = d_{1k}^{t_{ik}} - d_{1b_i}^{t_{ib_i}}$, 
\[
\begin{aligned}
\delta_{i,b_i,k} \, &| \, \delta_{i,b_i,2},\dots,\delta_{i,b_i,k-1},
d_{b_i,2},\dots,d_{b_i,k-1}, \sigma^2 \\
&\sim \text{N}\bigg(d_{b_i,k} + \frac{1}{k-1}\sum_{j=1}^{k-1}[\delta_{i,b_i,j} - d_{b_i,j}],\frac{k}{(k-1)}\sigma^2\bigg) \\
\end{aligned}
\]

\subsection{Sigmoidal Regression}

The prevalence of antibiotic resistant microbes is commonly modelled with compartmental models \citep{2013_Spicknall_etal,2019_Niewiadomska_etal}, e.g. in \cite{2002_Levin}.  These models imply that the prevalence of antibiotic resistant microbes will be sigmoidal in time.  \cite{1999_Austin_etal} analyzed the prevalence of $\beta$-lactamase producing strains of \textit{M. catarrhalis} in Finnish children and of penicillin-resistant pneumococci in Iceland and found that the proportion of antibiotic resistant microbes was approximately sigmoidal in the absence of health interventions.  Non-antiobitic-resistant microbes will also thus follow a sigmoidal trend.  If the efficacy of a treatment is directly proportion to the proportion of microbes which do not resist that treatment, then the efficacy of a drug targeting a growing community of resistant microbes ought to be sigmoidal in time.  

Sig-BNMA is a novel BNMA model which employs sigmoidal regression to model time-varying effects.  Fitting a sigmoidal curve on a constant time series, however, can lead to poor inference.  A set of latent variables, $z_k$, are thus introduced, one for every treatment in $\mathcal{T}_1$.  If $z_k = 1$, then the $d_{1k}^{t_{ik}}$ are modelled as following a sigmoidal curve in time with mean value $d_{1k}$, distance from mean value to asymptotes of $a_k$, scale $b_k$, and center $c_k$.; else, they are constant. A Gibbs sampler is implemented in JAGS.  The average treatment effect $d_{1k}$ is used as the average response of the sigmoidal curve because standard BNMA is effective at estimating the $d_{1k}$ even in the presence of time-varying effects.  Moreover, it improves Gibbs sampling --- if the values of $d_{1k}^{t_{ik}}$ when $z_k = 0$ and $z_k = 1$ use unrelated parameterizations, then mixing is poor and convergence rates are lower.  $\pi$ is set \textit{a priori}, usually a value of $0.1$ is adopted. 
\[
\begin{aligned}
y_{ik} \, | \, p_{ik}, n_{ik} &\sim \text{Bin}(p_{ik},n_{ik}) 
&\text{logit}(p_{ik}) &= \mu_{i} + \delta_{i,b_i,k}1_{b_i\neq k} \\
\delta_{i,b_i,k} \, &| \, \delta_{i,b_i,2},\dots,\delta_{i,b_i,k-1},
d_{b_i,2},\dots,d_{b_i,k-1}, \sigma^2 \\
&\sim \text{N}\bigg(d_{b_i,k} + \frac{1}{k-1}\sum_{j=1}^{k-1}[\delta_{i,b_i,j} - d_{b_i,j}],\frac{k}{(k-1)}\sigma^2\bigg) \\
\mu_i \, | \, m_\mu, sd_\mu &\sim N(m_\mu,\sigma_\mu^2) 
&m_\mu &\sim \text{N}(0,10000) \\
\sigma_\mu &\sim \text{Unif}(0,5) 
& d_{b_i,k} &= d_{1k}^{t_{ik}} - d_{1b_i}^{t_{ib_i}} \\
d_{1k}^{t_{ik}} \, &| \, k\in\mathcal{T}_1, a_k, b_k, c_k, d_k, z_k, d_{1k} \\
&\sim z_{k}\text{N}\bigg( (d_{1k} - a_k) + \frac{2a_k}{1 + \exp[-b_k(t_n^k - c_k)]}, \psi^2 \bigg) && + (1-z_k) d_{1k}\\
\sigma^2 &\sim \text{Unif}(0,5) 
&d_{1k}^{t_{ik}} \, | \, k\in\mathcal{T}_0, d_{1k} &= d_{1k}\\
\psi &\sim \text{Unif}(0,5) 
&a_k &\sim \text{N}(0,1) \\
b_k &\sim \Gamma(1,1) 
&c_k &\sim \text{Unif}(0,T) \\
z_k &\sim \text{Bern}(\pi) 
&d_{1k} &\sim \text{N}(m_d,\sigma_d^2) \\
m_d &\sim \text{N}(0,10000) 
&\sigma_d &\sim \text{Unif}(0,5)
\end{aligned}
\]

\subsection{GP-BNMA}

While sigmoidal curves are biologically plausible, there are many plausible forms a time-varying treatment effect curve could take.  Health interventions, such as in \cite{1999_Austin_etal}, can alter the time prevalence of antibiotic resistant microbes into a highly nonlinear curve.  For these situations, a nonparametric solution is necessary.  To this end, the $d_{1k}^{t_{ik}}$ are modelled as arising from a Gaussian Process (GP) kernel.  Let $d_{1k}^{t_{1}} \sim \text{GP}(d_{1k},\psi^2\mathbb{I}_{nk} + K(\cdot,\cdot))$ represent the following distribution:
\[
\begin{bmatrix}
d_{1k}^{t_{1k}} \\
d_{1k}^{t_{2k}} \\
\dots \\
d_{1k}^{t_{I_kk}} 
\end{bmatrix}
\sim 
N \bigg(
\begin{bmatrix}
d_{1k}\\
d_{1k}\\
\dots \\
d_{1k}
\end{bmatrix},
\psi^2 \mathbb{I}_{I_k} + K(\cdot,\cdot)
\bigg),
\]
where $\mathbb{I}_{I_k}$ is the $I_k\times I_k$ identity matrix and $K(\cdot,\cdot)$ is the square exponential kernel: $K(n,m) = \phi^2 \exp(-\rho |t_{nk} - t_{mk}|^2)$.  As BNMA is effective at finding the average values $d_{1k}$, these are taken as the mean.  The introduction of the GP kernel gives BNMA sufficient flexibility to capture any time-varying trend, and the resulting model is referred to as GP-BNMA. 
\[
\begin{aligned}
y_{ik} \, | \, p_{ik}, n_{ik} &\sim \text{Bin}(p_{ik},n_{ik}) 
&\text{logit}(p_{ik}) &= \mu_{i} + \delta_{i,b_i,k}1_{b_i\neq k} \\
\delta_{i,b_i,k} \, &| \, \delta_{i,b_i,2},\dots,\delta_{i,b_i,k-1},
d_{b_i,2},\dots,d_{b_i,k-1}, \sigma^2\\
&\sim \text{N}\bigg(d_{b_i,k} + \frac{1}{k-1}\sum_{j=1}^{k-1}[\delta_{i,b_i,j} - d_{b_i,j}],\frac{k}{(k-1)}\sigma^2\bigg) \\
\mu_i \, | \, m_\mu, \sigma_\mu &\sim N(m_\mu,\sigma_\mu^2) 
&m_\mu &\sim \text{N}(0,10000) \\
\sigma_\mu &\sim \text{Unif}(0,5) 
&d_{b_i,k} &= d_{1k}^{t_{ik}} - d_{1b_i}^{t_{ib_i}} \\
d_{1k}^{t_{ik}} \, &| \, k\in\mathcal{T}_0, d_{1k} = d_{1k}
&\sigma^2 &\sim \text{Unif}(0,5) \\
d_{1k}^{t_{ik}} \, &| \, k\in\mathcal{T}_1, d_{1k}, \psi, \phi, \rho \sim \text{GP}(d_{1k},\psi^2\mathbb{I}_{nk} + K(\cdot,\cdot)))
&\psi &\sim \text{Unif}(0,5) \\
K(n,m) &= \phi_k^2 \exp(-\rho_k |t_{nk} - t_{mk}|^2)
&\phi_k &\sim \text{Unif}(0,5) \\
\rho_k &\sim \text{Unif}(0.001,5)
&d_{1k} &\sim \text{N}(m_d,\sigma_d^2) \\
m_d &\sim \text{N}(0,10000) 
&\sigma_d &\sim \text{Unif}(0,5)
\end{aligned}
\]
A Gibbs sampler is implemented in JAGS.  


\section{Data, Simulations, and Analysis}

A dataset is formed by agglomerating datasets from a number of existing reviews employing BNMA to analyze MRSA-related cSSSI treatments.  Using the network, treatment arms, and timepoints from this agglomerated dataset, simulated data is generated with significant time effects on one treatment.  The performance of four BNMA methods on this simulate dataset is assessed. Finally, the models are fit on the agglomerated dataset to determine the potential time-varying effects of MRSA treatments.  

\subsection{Data}

The data is agglomerated from the datasets used by reviews employing BNMA to investigate the efficacy of various treatments on MRSA-related cSSSI's: \cite{2015_Thom_etal, 2016_Liu_etal, 2017_Guest_etal,2017_McCool_etal,2018_Li_Xu,2019_Zhang_etal,2019_Lan_etal,2021_Brown_etal,2021_Feng_etal}.  A potential concern with combining datasets from multiple studies is that they will be incompatible --- different experimental designs, for instance, may give rise to RCTs implemented on significantly different populations --- and violate the consistency assumption.  The reviews were all conducted according to PRISM or Cochrane standards, so there is a measure of similarity in how they collected studies.  Moreover, in all of these reviews, the vast majority of studies appeared in at least one other review; this implies a sort of transitive consistency.  Given the lack of data on MRSA-related cSSSI's \citep{2021_Brown_etal}, it is better to be expansive when deciding which studies to include.  Moreover, the random effects allow the models to compensate for any inconsistencies introduced by combining datasets from different reviews.

These reviews contribute a total of $58$  studies comparing $19$ treatments from $2000$ to $2019$.  A plot of the network is provided in Figure \ref{fig:data_network_plot}.  Four studies had $3$ treatment arms; the rest had $2$.  The most prevalent treatments are vancomycin (VAN), which appeared $46$ times, and linezolid (LIN), which appeared $27$ times.  There are $13$ direct comparisons of the two.  Both vancomycin and linezolid have comparisons with dalbavancin (DAL) and delafloxacin (DEL), but otherwise have no common comparators and the network structure can be thought of as having two poorly connected halves.  Vancomycin has additional comparisons with ceftaroline (CEF1),ceftobiprole (CEF2), oritavancin (ORI), daptomycin (DAP), telavancin (TEL),  tigecycline (TIG), iclaprim (ICL), and lefamulin (LEF).  Linezolid has additional comparisons with rifampicin (SXT/RIF), teicoplanin (TEI),  omadacycline (OMA), a novel fluoroquinolone  (JNJ-Q2), fusidic acid (CLEM-102), tedizolid (TED), and oxacillin-dicloxacililn (OXA).  Daptomycin and telavancin have one comparison with each other while tigecycline and delafloxin have two.  There are no other comparisons in the network.   

\subsection{Simulations}

One dataset was generated with no time-varying effects --- BNMA was taken as ``truth" using the network, comparisons, number of trials, and time points of the agglomerated dataset.  Four models were fit on the data: standard BNMA, Meta-BNMA, Sig-BNMA, and GP-BNMA.  Meta-BNMA is standard BNMA with a meta-regression on time; it fits linear time trends on $d_{1k}$ for treatments $k$ which appeared at least $5$ times.  For Meta-BNMA, Sig-BNMA, and GP-BNMA, there were $5$ treatments modelled with time effects: linezolid, daptomycin, telavancin, tigecycline, and tedizolid.  The proposed time-varying methods performed similarly to BNMA when there were no time effects. For more detail, see the Supporting Information S1.  

Another dataset was generated according to the above procedure, with the exception that the treatment effect for linezolid varied in time according to a sigmoidal relationship with $a_k = 2$, $b_k = 1$, and $c_k = 9.5$.  The other treatments are constant with respect to time: $d_{1k}^{t_{ik}} = d_{1k}$ for all $t_{ik}$.  This dataset may thus be thought of as being generated from ``true" Sig-BNMA dataset with $z_{k} = 1$ for linezolid and $z_k =0$ for all other treatments.  The posterior mean estimates and $95\%$ credible intervals for the treatment effects $d_{1k}$ for BNMA, Meta-BNMA, Sig-BNMA, and GP-BNMA are presented in Figure \ref{fig:sim_gpbnma_mean_te_ci} as well as the true treatment effects. 

BNMA provided accurate point estimates for the true average treatment effects $d_{1k}$.  However, it has wider intervals for every treatment than it did when there were no time effects present (CF Supporting Information S1).  BNMA is an accurate estimator of the these average effects because the random effects incorporated into the model allow the $\delta_{i,b_i,k}$ and $\mu_i$ to vary across study and ``absorb" the time-varying effects as cross-study heterogeneity.  It therefore finds correct values for the $d_{1k}$ at the cost of inflated uncertainty.

Meta-BNMA finds that the posterior probabilities that $\{\text{linezolid}$, $\text{daptomycin}$, $\text{telavancin}$, $\text{tigecycline}$, $\text{tedizolid}\}$ have time effects are$\{100$,$99.6$,$7.2$,$86.1$,$96.3\}$, respectively.  Thus, it detects significant time effects not only for linezolid, but for daptomycin and tedizolid as well. The linear effects built into Meta-BNMA cannot capture the true sigmoidal nature of the data, and Meta-BNMA can compensate only by modelling the other treatments as having linear time-varying effects.  Because of this, Meta-BNMA misestimates some of the treatments which are not modelled as having time effects: fusidic acid, the novel fluoroquinolone, omadacycline, oxacillin-dicloxacillin, and rifampicin in particular.  These treatments appear in relatively few studies, and so modelling a set of other treatments with incorrect time trends negatively effects the estimation of their treatment effect. 

Sig-BNMA provides accurate estimation to the mean $d_{1k}$, and does so with the narrowest set of credible intervals.  The posterior probability that $\{\text{linezolid}$, $\text{daptomycin}$, $\text{telavancin}$, $\text{tigecycline}$, $\text{tedizolid}\}$ have time effects are $\{100,24.5,12.0,3.5,5.4\}$, respectively.  Sig-BNMA correctly finds that linezolid has a sigmoidal time-varying effect, but places non-negligible probability on the event that either daptomycin or telavancin do as well.  This is confounding from linezolid --- Sig-BNMA is confusing a time trend on one treatment with time trends on multiple.  Despite this, the posterior trends for these treatments are approximately constant.

GP-BNMA correctly estimates the posterior means, but does so with wider credible intervals than Sig-BNMA does --- especially on treatments modelled with time effects.  Figure \ref{fig:sim_gpbnma_run_gpbnma_meancred} plots the the posterior mean credible intervals for the $d_{1k}$, and shows that GP-BNMA detects the sigmoidal time trend in linezolid.  Most of the rest of the treatments are relatively constant with respect to time --- the exceptions are daptomycin and tedizolid, where there is slight evidence of time effects.  However, these time effects are less than the posterior uncertainty present, and so can be disregarded.  

Inference to the average treatment effect $d_{1k}$, however, is of limited practical use.  Of greater interest is the estimate for the treatment effect at the end of the time period in question --- the treatment effect at the timepoint closest to the present.  For clinical purposes, this is the most valuable estimate, as it is the most informative estimate for guiding treatment.  The true value is $2.31$.  BNMA finds a posterior mean value of $0.011$ with a credible interval of $(-0.66,0.68)$.  Meta-BNMA finds a posterior mean of $3.86$ with a credible interval of $(2.97,4.79)$.  Sig-BNMA finds a posterior mean estimate of $2.62$ with a credible interval of $(2.03,3.40)$.  GP-BNMA finds a posterior mean of $4.06$ with a credible interval of $(3.70,4.43)$.  Only Sig-BNMA estimates this value correctly: BNMA underestimates this value while Meta-BNMA and GP-BNMA overestimate it. 

\subsection{Implementation on MRSA Data}

To detect any possible time effects, BNMA, Meta-BNMA, Sig-BNMA, and GP-BNMA were run on the agglomerated dataset.  All methods took vancomyicn as the baseline because it was the treatment which appeared the most; taking another treatment, e.g. linezolid, as a baseline does not significantly alter results. Treatments which appeared at least five times --- linezolid, daptomycin, telavancin, tigecycline, and tedizolid --- were allowed to have time-varying effects in Meta-BNMA, Sig-BNMA, and GP-BNMA.  All other treatment effects were fixed with respect to time.  No covariates asides from time were considered.  One reason was that the lack of covariate information for some studies --- a possible extension of this work would follow in the direction of \cite{2012_Jansen} or \cite{2020_Phillippo_etal} and employ a meta-regression model to ``balance" studies with individual patient data (IPD) to those without.  However, such methods are data-intensive and may not be able to be effectively employed simultaneously with our time-varying models.  

The mean treatment effects and $95\%$ credible intervals for each of the treatments, compared to vancomycin are presented in Figure \ref{fig:data_te_mean_cred}.  The four methods produce similar posterior mean estimates of the $d_{1k}$.  Meta-BNMA and GP-BNMA had wider credible intervals for those treatments which were modelled with time-varying effects.  

Meta-BNMA found that the percent chance that $\{\text{linezolid}$, $\text{daptomycin}$, $\text{telavancin}$, $\text{tigecycline}$, $\text{tedizolid}\}$ had significant positive time-varying linear effects are $\{36.8,39.3,51.6,97.7,26.5\}$, respectively.  That is, it finds that tigecycline has a positive linear time-varying relationship with respect to vancomycin.  The posterior mean estimate of the regression coefficient is $0.199$ with credible interval $(0.005,0.413)$.  Tigecycline appears to be increasing in effectiveness compared to vancomycin from $2005$ to $2008$, where it features in $4$ studies; however, tigecycline also appears in a high-leverage study in $2015$ where it appears to have decreased in effectiveness.  Meta-BNMA fits a linear trend to these small and arguably nonlinear datapoints. 

Sig-BNMA found that the percent chance that $\{\text{linezolid}$, $\text{daptomycin}$, $\text{telavancin}$, $\text{tigecycline}$, $\text{tedizolid}\}$ had significant time-varying sigmoidal effects are $\{5.6$,$0.4$,$0.8$,$2.1$,$0.8\}$, respectively --- that is, it did not find the presence of significant sigmoidal time-varying effects.  Given the small probabilities, Sig-BNMA defaulted to BNMA and produced results which were almost exactly the same.  

GP-BNMA finds weak evidence of time-varying treatment effects.   Figure \ref{fig:data_gpbnma_credible} displays the posterior mean and credible intervals for the $d_{k}^{t_{ik}}$.  All treatments had periods of non-overlapping intervals.  That is, for each treatment, the lower bound of the $95\%$ credible interval at one timepoint was higher than the upper bound of the $95\%$ credible interval at another timepoint.  The difference between these points is largest for linezolid and tigecycline, which also appear to have the strongest time effects.  Moreoever, linezolid, tedizolid, telavancin, and tigecycline have similar trends; they became more effective relative to vancomycin until approximately $2010$, after which their relative efficacy lessened.  Note also that linezolid, daptomycin, telavancin, and tigecycline all have direct comparisons with vancomycin (CF Figure \ref{fig:data_network_plot}), so this trend is observed across multiple independent treatments and is not the result of indirect comparison alone.  

BNMA, Meta-BNMA, Sig-BNMA, and GP-BNMA found similar estimates of the average treatment effects $d_{1k}$.  Moreover, Meta-BNMA and Sig-BNMA indicate that there is neither a linear nor a sigmoidal trend for any of the treatments present, with the exception of a positive linear trend for tigecycline.  GP-BNMA, however, finds evidence of a non-linear trend --- multiple treatments increased in efficacy relative to vancomycin until $2010$, after which they decreased.  These results are consistent with the findings of \cite{2007_Daum, 2004_Cosgrove_etal}, which indicate that growing vancomycin resistance was an issue before $2010$, as well as with \cite{2019_Diekema_etal}, which indicates that vancomycin resistance had plateaued during $2013-2016$.  It is interesting to observe that this vancomycin resistance \textit{S. Aureus} trend is similar to MRSA trend outlined in Figure \ref{fig:2019_diekema_etal_mrsa_prev} --- perhaps the prevalence of vancomyicn-resistance \textit{S. Aureus} and MRSA are correlated.  Regardless, all four methods found that almost all treatments had the same relative efficacy compared to vancomycin at the end of the time period considered in the dataset as they did in the beginning: MRSA was no more resistant to vancomycin in $2019$ than it was in $2000$.  

\section{Discussion}

Two novel BNMA models are proposed which seek to account for design inconsistencies in network meta-analyses of RCTs introduced via time-varying treatment effects.  Sig-BNMA decides, based off of the data, whether to model each treatment's time trend as constant or as a biologically plausible sigmoidal function.  GP-BNMA nonparametrically models the time trend for each treatment about that treatment's average treatment effect.  Both models are fully Bayesian and allow for posterior uncertainty quantification; posterior computation for both models proceeds via a Gibbs sampler implemented in JAGS.  Incorporating these time-varying methods into standard BNMA allows them to capture time-varying effects that standard BNMA is unable to detect.  In simulations, both proposed time-varying models outperformed existing BNMA methods in the presence of significant sigmoidal time-varying effects.

A dataset was agglomerated from a collection of review paper investigating MRSA related cSSSI and analyzed using existing BNMA techniques as well as the proposed time-varying methods.  Sig-BNMA and GP-BNMA find that MRSA is not more resistant to vancomycin at the end of the period in question than it was at the beggining.  However, GP-BNMA finds evidence that vancomycin resistance in MRSA grew from approximately $2000$ until $2010$, when it peaked, and declined thereafter.  The dataset is sparse, and more studies would improve inference.  

The time-varying methods presented in this paper could be expanded upon.  One such extension would follow \cite{2012_Jansen} or \cite{2020_Phillippo_etal} and employ a meta-regression model to ``balance" studies with covariate information to those without.  Alternate kernels for modelling the time-varying effects could also be explored.  

\bibliographystyle{agsm}
\bibliography{sample}

\clearpage

\begin{figure}
    \centering
    \includegraphics[width=\textwidth]{Figures/2019_Diekema_etal_MRSA_prev.png}
    \caption{``SENTRY Program 20-year trends in percentage of \textit{Staphylococcus aureus} BSI isolates that are MRSA." \citep{2019_Diekema_etal} }
    \label{fig:2019_diekema_etal_mrsa_prev}
\end{figure}

\clearpage

\begin{figure}
    \centering
    \includegraphics[width = \textwidth]{Figures/Data_Network_plot.png}
    \caption{The network of treatments found in the agglomerated dataset.  Treatments in larger nodes appeared more often, and the thicker the line between two nodes the more often they were compared. }
    \label{fig:data_network_plot}
\end{figure}

\clearpage

\begin{figure}
    \centering
    \includegraphics[width = \textwidth]{Figures/Sim_GPBNMA_Mean_TE_CI.png}
    \caption{Posterior mean and credible intervals for $d_{1k}$ by model when there is a time effect on LIN.}
    \label{fig:sim_gpbnma_mean_te_ci}
\end{figure}

\clearpage

\begin{figure}
    \centering
    \includegraphics[width = \textwidth]{Figures/Sim_GPBNMA_Bin_Run_GPBNMA_MeanCred.png}
    \caption{GP-BNMA posterior mean estimates and credible intervals when there is a sigmoidal time effect on LIN.}
    \label{fig:sim_gpbnma_run_gpbnma_meancred}
\end{figure}

\clearpage 

\begin{figure}
    \centering
    \includegraphics[width = \textwidth]{Figures/Data_TE_Mean_Cred.png}
    \caption{Posterior mean and credible intervals for average treatment effects $d_{1k}$ compared to VAN.}
    \label{fig:data_te_mean_cred}
\end{figure}

\clearpage 

\begin{figure}
    \centering
    \includegraphics[width = \textwidth]{Figures/Data_GPBNMA_Credible.png}
    \caption{Posterior mean and credible intervals for nine treatments compared to vancomycin for GP-BNMA.}
    \label{fig:data_gpbnma_credible}
\end{figure}

\clearpage 

\section*{Software}

Reproducible code and data for this paper is available at \url{https://github.com/PatrickLeBlanc/tBNMA}.

\clearpage

\section*{Supporting Information}

\subsubsection*{S1: If there are no time effects than how do the models compare?}

To simulate a dataset with no time-varying treatment effects, we simulate a dataset with BNMA using the network, comparisons, number of trials, and time points of the agglomerated dataset.  

We fit standard BNMA, GP-BNMA, sig-BNMA, and a BNMA with a meta regression.  The meta-regression was linear effects of time on all $d_{1k}$ for all treatments $k \geq 2$ that occurred at least $5$ times in the data.  For Sig-BNMA and GP-BNMA, we set $C = 5$, so that we only fit time effect.  This implies that we model $5$ treatments with time effects: DAP, LIN, TED, TEL, and TIG.  

The posterior mean estimates and credible intervals for Sig-BNMA, Meta-BNMA, GP-BNMA, and BNMA are presented in Figure \ref{fig:sim_bnma_mean_te_ci} as well as the true treatment effects $d_{1k}$ for each treatment.  For those treatments which we did not model with time effects, all methods reproduced the results produced by standard BNMA.  

Sig-BNMA found at most a $0.4\%$ chance that any of the treatment effects modelled with time-varying effects actually had a sigmoidal function.  It thus reduced to BNMA and correctly deduced the absence of time effects.  

Meta-BNMA provides posterior mean estimates and credible intervals similar to the other models for treatments which are not modelled with time effects.  However, for most treatments modelled with treatment effects --- DAP, TED, TEL, and TIG --- Meta-BNMA finds large credible intervals and can find slightly different means.  This is because we are modelling a linear time effect which is not, in fact, present, increasing our uncertainty.  This is also related to the small sample size of these treatments, which appear either $5$, $6$, or $7$ times.  Moreover, Meta-BNMA's posterior $95\%$ credible interval for TEL does not contain the true value. This is because it finds a significant negative linear time effect with at a $95\%$ level --- the posterior mean estimate for this effect is $-0.164$ and the interval is $(-0.321, -0.020)$.  Because of this spurious finding, it overestimates the average treatment effect $d_{1k}$ for TEL.  For LIN, which appears $26$ times, Meta-BNMA finds estimates for $d_{1k}$ similar to those from BNMA; moreover, it finds that there are no time effects.  

GP-BNMA produced similar mean estimates as BNMA in all treatments.  However, for DAP, TED, TEL, and TIG, it found considerably wider credible intervals than BNMA does --- we pay a price in uncertainty quantification for modelling possible time effects.  This is also possibly a function of small amounts of data: these four treatments occur either $5$, $6$, or $7$ times in the data.  Note that GP-BNMA does not find a wider credible interval for LIN, which appears $26$ times.  A plot of mean estimates and uncertainty for the $d_{1k}^t$ for the treatments we modelled as time-varying are presented in Figure \ref{fig:sim_bnma_run_gpbnma_uncertainty}.  (Credible intervals for the mean are presented in Figure \ref{fig:sim_bnma_run_gpbnma_meancred}.)  The $d_{1k}$ for LIN are stable in time with fairly narrow bounds, but the estimates for the other treatments are much more prone to uncertainty.  The only treatment which GP-BNMA might have found a time effect for is TEL, though there is too much uncertainty to draw any conclusions, and there is a relative lack of data.  

\begin{figure}
    \centering
    \includegraphics[width = \textwidth]{Figures/Sim_BNMA_Mean_TE_CI.png}
    \caption{Posterior mean and credible intervals for $d_{1k}$ by model when BNMA is true and there are no time-varying effects.}
    \label{fig:sim_bnma_mean_te_ci}
\end{figure}

\begin{figure}
    \centering
    \includegraphics[width = \textwidth]{Figures/Sim_BNMA_Bin_Run_GPBNMA_Uncertainty.png}
    \caption{GP-BNMA posterior mean estimates and uncertainty for when the truth is BNMA.}
    \label{fig:sim_bnma_run_gpbnma_uncertainty}
\end{figure}

\begin{figure}
    \centering
    \includegraphics[width = \textwidth]{Figures/Sim_BNMA_Bin_Run_GPBNMA_MeanCred.png}
    \caption{GP-BNMA posterior mean estimates credible intervals when the truth is BNMA.}
    \label{fig:sim_bnma_run_gpbnma_meancred}
\end{figure}



\end{document}
