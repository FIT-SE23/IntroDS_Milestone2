%=================================================================
\documentclass[universe,article,accept,pdftex,oneauthor ]{Definitions/mdpi} 
%\documentclass[preprints,article,accept,moreauthors,pdftex]{mdpi}.

%--------------------
% Class Options:
%--------------------
%----------
% journal
%----------3
% Choose between the following MDPI journals:
% universe
%---------
% article
%---------
% The default type of manuscript is "article", but can be replaced by: 
%----------
% submit
%----------
% The class option "submit" will be changed to "accept" by the Editorial Office when the paper is accepted. This will only make changes to the frontpage (e.g., the logo of the journal will get visible), the headings, and the copyright information. Also, line numbering will be removed. Journal info and pagination for accepted papers will also be assigned by the Editorial Office.

%------------------
% moreauthors
%------------------
% If there is only one author the class option oneauthor should be used. Otherwise use the class option moreauthors.

%---------
% pdftex
%---------
% The option pdftex is for use with pdfLaTeX. If eps figures are used, remove the option pdftex and use LaTeX and dvi2pdf.

%=================================================================
% MDPI internal commands
\firstpage{1} 
\makeatletter 
\setcounter{page}{\@firstpage} 
\makeatother
\pubvolume{1}
\issuenum{1}
\articlenumber{0}
\pubyear{2022}
\copyrightyear{2022}
\externaleditor{Academic Editors: Dr. Eleonora Di Valentino, Prof. Dr. Leandros Perivolaropoulos, Dr. Jackson Levi Said %Please provide Academic Editor.
}
\datereceived{1 November 2022} 
%\daterevised{} % Only for the journal Acoustics
\dateaccepted{24 November 2022} 
\datepublished{} 
%\datecorrected{} % Corrected papers include a "Corrected: XXX" date in the original paper.
%\dateretracted{} % Corrected papers include a "Retracted: XXX" date in the original paper.
\hreflink{https://doi.org/} % If needed use \linebreak
%\doinum{}
%------------------------------------------------------------------
% The following line should be uncommented if the LaTeX file is uploaded to arXiv.org
%\pdfoutput=1

%=================================================================
% Add packages and commands here. The following packages are loaded in our class file: fontenc, inputenc, calc, indentfirst, fancyhdr, graphicx, epstopdf, lastpage, ifthen, lineno, float, amsmath, setspace, enumitem, mathpazo, booktabs, titlesec, etoolbox, tabto, xcolor, soul, multirow, microtype, tikz, totcount, changepage, attrib, upgreek, cleveref, amsthm, hyphenat, natbib, hyperref, footmisc, url, geometry, newfloat, caption

%\usepackage{verbatim}
%\usepackage{comment}
%\usepackage{graphicx}
%\usepackage[T1]{fontenc}
%\usepackage{graphicx}
\usepackage{epsfig}
\usepackage{epstopdf}
%\usepackage{bm}
%\usepackage{tensor}
\usepackage{amsfonts}
%\usepackage[T1]{fontenc}
%\usepackage[latin9]{inputenc}
%\usepackage{fouriernc1} % sharper font for on-screen
\usepackage{amssymb}
%https://www.overleaf.com/project/6006ae558abbb08625a2e4bb
%\usepackage{float}
%\usepackage{amsmath}
\makeatletter
\let\c@lofdepth\relax
\let\c@lotdepth\relax
\makeatother
\usepackage{subfigure}
\usepackage{tabularx}
%\usepackage{newcent}
%\usepackage{dcolumn}

%\usepackage[normalem]{ulem}
%\usepackage[utf8]{inputenc}
%\usepackage{cancel}
%\usepackage[colorlinks]{hyperref}
%\usepackage[usenames,dvipsnames]{color}
%\usepackage{hyperref}


%=================================================================
% Full title of the paper (Capitalized)
\Title{DE Models with Combined $H_0 \cdot r_d $ from BAO and CMB Dataset and Friends}

% MDPI internal command: Title for citation in the left column
\TitleCitation{DE Models with Combined $H_0 \cdot r_d $ from BAO and CMB Dataset and Friends}

% Author Orchid ID: enter ID or remove command
\newcommand{\orcidauthorA}{0000-0001-6139-3125} % Add \orcidA{} behind the author's name
%\newcommand{\orcidauthorB}{0000-0000-0000-000X} % Add \orcidB{} behind the author's name

% Authors, for the paper (add full first names)
\Author{{Denitsa} %Please carefully check the accuracy of names and affiliations. Changes will not be possible after proofreading.
 Staicova \orcidA{0000-0001-6139-3125}}

%\longauthorlist{yes}

% MDPI internal command: Authors, for metadata in PDF
\AuthorNames{Denitsa Staicova}

% MDPI internal command: Authors, for citation in the left column
\AuthorCitation{Staicova, D.}
% If this is a Chicago style journal: Lastname, Firstname, Firstname Lastname, and Firstname Lastname.

% Affiliations / Addresses (Add [1] after \address if there is only one affiliation.)
\address[1]{%
Institute for Nuclear Research and Nuclear Energy, Bulgarian Academy of Sciences, Sofia, 1784 %Please provide post code.
, Bulgaria; dstaicova@inrne.bas.bg\\}

% Contact information of the corresponding author
%\corres{Correspondence: dstaicova@inrne.bas.bg;}

%\documentclass[reprint,aps,amsmath,amssymb,nofootinbib,twoside,prd,showkeys,superscriptaddress]{revtex4-1}
%\pdfoutput=1

\abstract{It has been theorized that dynamical dark energy (DDE) could be a possible solution to Hubble tension. To avoid degeneracy between Hubble parameter $H_0$ and sound horizon scale $r_d$, in this article, we use their multiplication as one parameter $c/\left(H_0 r_d\right)$, and we use it to infer cosmological parameters for 6 models---$\Lambda$CDM and 5 DDE parametrizations---the Chevallier--Polarski--Linder (CPL), the  Barboza--Alcaniz (BA), the  low correlation (LC), the  Jassal--Bagla--Padmanabhan (JBP) and  the  Feng--Shen--Li-Li models. We choose a dataset that treats this combination as one parameter, which includes the baryon acoustic oscillation (BAO) data $0.11 \le z \le 2.40$ and additional points from the cosmic microwave background (CMB) peaks ($z \simeq 1090$). To them, we add the marginalized Pantheon dataset and GRB dataset. We see that the tension is moved from $H_0$ and $r_d$ to $c/\left(H_0 r_d\right)$ and $\Omega_m$. There is only one model that satisfies the Planck 2018 constraints on both parameters, and this is LC with a huge error. The rest cannot fit into both constraints. $\Lambda$CDM is preferred, with respect to the statistical measures.
}

% Keywords
\keyword{cosmological tensions; dynamical dark energy; baryon acoustic oscillation; Pantheon dataset; gamma-ray bursts} 

%\maketitle

%\tableofcontents
\begin{document}


\section{Introduction}
The quest for understanding cosmological tensions  has driven research for years now. It seems that the tension between the direct measurements of the Hubble constant ($H_0$) from the late universe (\cite{Freedman:2000cf,Riess:1998cb,Perlmutter:1998np, Riess:2020fzl, Riess:2022mme}) and that from the early universe (i.e., from measuring the temperature and polarization anisotropies in the cosmic microwave background~\cite{Troxel:2017xyo,Aghanim:2018eyx,Ade:2015xua,Dainotti:2021pqg}) is only aggravated with the increase in the precision and knowledge of the systematics of the data and has reached 5 $\sigma$.  This discrepancy has spurred many works, trying to resolve whether the dark energy is a constant energy density or with a dynamical behavior, and if so, of what origin, leading to many different theories and possible explanations~\cite{Benisty:2021wxi,Capozziello:2011et,Bull:2015stt,DiValentino:2021izs,Yang:2021flj,Schoneberg:2019wmt,DiValentino:2017gzb,DiValentino:2020zio,Perivolaropoulos:2021jda,Lucca:2021dxo, Colgain:2022nlb, Colgain:2022rxy}.


There are many dark energy (DE) parametrizations~\cite{Wang:2018fng,Reyes:2021owe,Colgain:2021pmf, 2108.04188} that can be used in the search for deviations from the cosmological constant, $\Lambda$.
Some of them fall in the group of early dark energy models~\cite{Pettorino:2013ia, Poulin:2018cxd, Lin:2020jcb, Smith:2022hwi, Smith:2020rxx}, which modify physics of the early universe. Others modify the late-time universe physics, such as in the phantom dark energy~\cite{DiValentino:2020vnx, Haridasu:2020pms} models, and emergent dark energy~\cite{Li:2019yem,Yang:2020ope},~add interaction in the DE sector, as in the interacting dark energy~\cite{Kumar:2017dnp,DiValentino:2019ffd, Yang:2019uzo}, or~add exotic species or scalar fields~\cite{Gogoi:2020qif,Sakstein:2019fmf,Tian:2021omz,Nojiri:2021dze,Seto:2021xua}. For~a review on the taxonomy of DE models, see Refs.~\cite{Escamilla-Rivera:2021boq, Motta:2021hvl, Yang:2021eud}. Finally, there is the generalized emergent dark energy (GEDE)~\cite{Yang:2021eud} found to be able to compete with $\Lambda$CDM for some BAO datasets~\cite{Staicova:2021ntm}. In~this work, we take the so-called dynamical dark energy parametrizations, which allow for a non-constant DE contribution, regardless of the origin behind it. We take a number of models, namely the Chevallier--Polarski--Linder (CPL), the~ Barboza--Alcaniz (BA), the~low correlation (LC), the~ Jassal--Bagla--Padmanabhan (JBP) and  the  Feng--Shen--Li--Li (FSLLI) model, and we use statistical measures to judge their performance in fitting the~data.

An important part of the tensions debate revolves around the role of the sound horizon at drag epoch $r_d$. At~recombination, after~the onset of CMB at $z_*  \simeq  1090$, the~baryons escape the drag of photons at the drag epoch, $z_d  \simeq  1059$ (Planck 2018~\cite{Aghanim:2018eyx}). This sets the standard ruler for the baryon acoustic oscillations  (BAO)---the distance ($r_d$) at which the baryon--photon plasma waves oscillating in the hot  universe froze at $z=z_d$. The~sound horizon at drag epoch is given by
\begin{equation}
r_d = \int_{z_d}^{\infty} \frac{c_s(z)}{H(z)} dz
,\end{equation}
where $c_s \approx c \left(3 + 9\rho_b /(4\rho_\gamma) \right)^{-0.5}$ is the speed of sound in the baryon--photon fluid with the baryon $\rho_b(z)$ and the photon $\rho_\gamma(z)$ densities, respectively~\cite{Aubourg:2014yra,Arendse:2019itb}.

Many papers discuss the relation between the $H_0$ and the sound horizon scale $r_d$ for different models~\cite{Aylor:2018drw,Pogosian:2020ded,Aizpuru:2021vhd}. Any DE model claiming to resolve the $H_0$ tension should also be able to resolve the $r_d$ tension since they are strongly connected~\cite{Jedamzik:2020zmd,Aizpuru:2021vhd,delaMacorra:2021hoh}. In~other words, setting a prior on $r_d$ has a very strong effect on $H_0$ and vice~versa. In~this paper, to~avoid this problem, we combine the $H_0$ and $r_d$ into one parameter. We choose measurements that combine the $H_0 $ and $r_d$ from the BAO and the prior distance from the CMB peaks~\cite{Wang:2013mha, Mamon:2016wow, Grandon:2018uoe, Chen:2018dbv, daSilva:2018ehn, Zhai:2018vmm,DiValentino:2020hov, Nilsson:2021ute, Yao:2022kub} and we use them to infer the cosmological parameters for $\Lambda$CDM and 5 DDE models. To~the BAO+CMB dataset, we add the gamma-ray bursts (GRB) dataset and the Pantheon dataset with similarly marginalized dependence on $H_0$ (and $M_B$). We do this to expand the redshift considered by the models. In~a previous work~\cite{Staicova:2021ntm}, we used a similar approach in which we integrated $H_0 r_d$ in the $\chi^2$ of the model, while here, we use them as one single quantity without modifying the $\chi^2$. In~the marginalized version, we saw an interesting possibility for some DE model to fit the data better than $\Lambda$CDM. We continue this investigation with new models and a new approach in this~paper.

Historically, the~approach of using the combination $H_0 r_d$ is not new.  It has been used in~\cite{LHuillier:2016mtc} with BAO and SN data to find consistency with the Planck 2015 best-fit $\Lambda$CDM cosmology; Ref.~\cite{Shafieloo:2018gin} used the BAO data to fit the growth measurement, again finding consistency with the Planck 2015; Ref.~\cite{Arendse:2019hev} used the Cepheids and the Tip of the Red Branch measurements to calibrate BAO and SN measurements and find significant tension in both $H_0$ and $r_d$, despite testing the   $\Lambda CDM$ and DE models ($EDE$, $wCDM$, pEDE). The~implication is that  modifications of the physics after recombination fail to solve both tensions. The~overall conclusion is that the $H_0$ tension should not be considered separately from the $r_d$ measurement implied by it~\cite{Knox:2019rjx}. In~the current work, we choose a different approach. We repeat the analysis on $H_0 r_d$ used in earlier works, but we also take the ratio $r_*/r_d$ as an independent parameter. This means that we do not use the known analytical formulas for them, but~instead we use MCMC to infer them. This avoids using explicit prior knowledge on the baryon load of the universe. This way, we avoid both the degeneracy on $H_0 r_d$ from the BAO data, but~also we do not use as a hidden prior the Planck~measurements. 

The plan of the work is as follows: Section~\ref{sec:theory} formulates the relevant theory.  Section~\ref{sec:method} describes the method. Section~\ref{sec:res} shows the results, and Section~\ref{sec:sum} summarizes the~results.


\section{Theory}
\label{sec:theory}
A Friedmann--Lema\^itre--Robertson--Walker metric with the scale parameter\linebreak $a = 1/(1+z)$ is considered, where $z$ is the redshift. The~evolution of the universe for it is governed by the Friedmann equation, which connects the equation of the state for the $\Lambda$CDM~background:
\begin{equation}
    E(z)^2 = \Omega_{r} (1+z)^4 + \Omega_{m} (1+z)^3 + \Omega_{k} (1+z)^2 + \Omega_{DE}(z),
    \label{eq:hzlcdm}
\end{equation}

\noindent where in standard $\Lambda$CDM, $\Omega_{DE}(z)\to \Omega_\Lambda$, with~the expansion of the universe\linebreak $E(z)= H(z)/H_0$, where $H(z) := \dot{a}/a$ is the Hubble parameter at redshift $z$, and $H_0$ is the Hubble parameter today. $\Omega_{r}$, $\Omega_{m}$, $\Omega_{DE}$ and $\Omega_{k}$ are the fractional densities of radiation, matter, dark energy and the spatial curvature at redshift $z=0$. We take into account the radiation energy density as $\Omega_r = 1 - \Omega_m - \Omega_{\Lambda} - \Omega_{k}$. The~spatial curvature is expected to be zero for a flat universe, $\Omega_k=0$, and we set it to zero because we focus on DE~models.

We will consider a number of different DE models, all of which will feature a dark energy component depending on $z$. This can be done with a generalization of the Chevallier--Polarski--Linder (CPL) parametrization~\cite{Chevallier:2000qy,Linder:2005ne,Barger:2005sb}:
\begin{equation}
\Omega_{DE} \left(z\right) = \Omega_{\Lambda}  \exp\left[\int_0^{z} \frac{3(1+w(z')) dz'}{1+z'}\right]
\label{eq:ol}
\end{equation}
which allows for three possible models from which we will consider only the CPL:
\begin{equation}
w(z)=w_0 + w_a \frac{z}{z+1}
\end{equation}
and $\Lambda$CDM is recovered for $w_0=-1, w_a=0$. 

\textls[-25]{To this parametrization, we add another model~\cite{Barboza:2008rh, Escamilla-Rivera:2021boq}, which is the Barboza--Alcaniz (BA) model~with}
%MDPI: Newly added information, please confirm. %ok

\begin{equation}
    w(z)=w_0+z\frac{1+z}{1+z^2}w_1
\end{equation}

This model is good for describing the whole universe history because~it does not diverge for $z \to -1$. It gives
\begin{equation}
    \Omega_{DE}=\Omega_{\Lambda}(1+z)^{3(1+w_0)}{(1+z^2)}^{\frac{3w_1}{2}}.
\end{equation}

Next, we use the low correlation model (LC) \cite{Wang:2008zh,  Escamilla-Rivera:2021boq} with
\begin{equation}
    w(z)=\frac{(-z+z_c)w_0+z(1+z_c)w_c}{(1+z)z_c}
\end{equation}
where $w_0=w(0)$ and $w_c=w(z_c)$ where $z_c$ is the redshift at which $w_0$ and $w_z$ are uncorrelated. The~effective entry into the EOS is
\begin{equation}
\Omega_{DE}=
\Omega_\Lambda(1+z)^{(3(1-2w_0+3wa))} e^{\frac{9(w_0-wa)z}{(1+z))}}
\end{equation}
where, here, are replaced $w_c$ with $w_a$ for consistency with the other~models.

The Jassal--Bagla--Padmanabhan (JBP) parametrization~\cite{Jassal:2004ej, Motta:2021hvl}
\begin{equation}
    w(z)=w_0+w_1\frac{z}{(1+z)^2}
\end{equation}
which gives
\begin{equation}
    \Omega_{DE}=\Omega_\Lambda (1+z)^{3(1+w_0)}e^{\frac{3w_1z^2}{2(1+z)^2}}
\end{equation}
with $w_0=w(z=0)$ and  $w_1 = (dw/dz)_{|(z=0)}$.

Finally, we will also test the Feng--Shen--Li--Li parametrization~\cite{Feng:2012gf, Motta:2021hvl} which is divergence-free for the entire history of the universe. It has two cases:
\begin{align}
&w(z)^+=w_0+w_1\frac{z}{1+z^2}\\
&w(z)^-=w_0+w_1\frac{z^2}{1+z^2}
\end{align}
with the final contribution to the EOS of each of them being, accordingly,
\begin{equation}
\Omega_{DE}^\pm =\Omega_\Lambda(1+z)^{3(1+w_0)}e^{\pm \frac{3w_1}{2}\arctan(z)}(1+z^2)^{\frac{3w_1}{4}}(1+z)^{\mp \frac{3}{2}w_1}
\end{equation}

In this work, the~plus case (i.e., $\Omega_{DE}^+$) is denoted as FSLLI, and the minus case (i.e., $\Omega_{DE}^-$) is denoted as~FSLLII.

The distance priors provide effective information of the CMB power spectrum in two aspects: the acoustic scale $l_\textrm{A}$ characterizes the CMB temperature power spectrum in the transverse direction, leading to the variation of the peak spacing, and~the ``shift parameter'' $R$ influences the CMB temperature spectrum along the line-of-sight direction, affecting the heights of the peaks. The~popular definitions of the distance priors are~\cite{Komatsu:2008hk}
\begin{equation}
\begin{split}
l_\textrm{A} =(1+z_*)\frac{\pi D_\textrm{A}(z_*)}{r_s(z_*)} ,\\
R\equiv(1+z_*)\frac{D_\textrm{A}(z_*) \sqrt{\Omega_m } H_0}{c},
\end{split}
\label{la:Rz}
\end{equation}
where $z_*$ is the redshift at the photon decoupling epoch with $z_*  \simeq  1089$ according to the $Planck$ 2018 results~\cite{Aghanim:2018eyx}. $r_*$ is the co-moving sound horizon at $z=z_*$. {Ref.} \cite{Chen:2018dbv}
%MDPI: Newly added information, please confirm.%ok
 derives the distance priors in several different models using $Planck$ 2018 TT,TE,EE $+$ lowE which is the latest CMB data from the final full-mission Planck measurement~\cite{Aghanim:2018eyx}.  {We use the correlation matrices given in {Table~1%Please  confirm that this is citation of Table from ref 57.
} in~\cite{Chen:2018dbv} to obtain the covariance matrices for $l_A$ and $R$  corresponding to each model.}


 {The angular diameter distance,  $D_\textrm{A}$,  needed for both the distance priors and the BAO points, is given by}
\begin{align}\label{}
D_\textrm{A}
%=\frac{D_\textrm{M}}{1+z}  &
=\frac{c}{(1+z) H_0 \sqrt{|\Omega_{k}|}  } \textrm{sinn}\left[|\Omega_{k}|^{1/2}\int_0^z \frac {dz'} {E(z')}\right]\ ,
\end{align}
where $\textrm{sinn}(x) \equiv \textrm{sin}(x)$, $x$, $\textrm{sinh}(x)$ for $\Omega_{k}<0$, $\Omega_{k}=0$, $\Omega_{k}>0$, respectively. We see that for the measured $D_A/r_d$, one can isolate the variable $b=c/(H_0 r_d)$. Below, we set $\Omega_k=0$, so this formula simplifies to
\begin{equation}
\frac{D_\textrm{A}}{r_d}= \frac{b}{(1+z)}\int_0^z \frac{dz'}{E(z')}
 \label{v_model}
\end{equation}


Finally, for~the SN and GRB datasets, we define the distance modulus $\mu(z)$, which is related to the luminosity distance ($d_L = D_A(1+z)^2$), through
\begin{equation}
        \mu_B (z) - M_B = 5 \log_{10} \left[ d_L(z)\right] + 25  \,,
\label{eq:dist_mod_def}
\end{equation}
where $d_L$ is measured in units of Mpc, and~$M_B$ is the absolute~magnitude.



\section{Methods}
\label{sec:method}
In this paper, we use three datasets, which we treat differently. For~the BAO dataset, the~definition of  $\chi^2$, which we minimize, is the standard one since we do not use the covariance matrix for it.
\begin{equation}
\begin{split}
\chi^2_{BAO} = \sum_{i} \frac{\left(\vec{v}_{obs} - \vec{v}_{model}\right)^2}{\sigma^2},
\end{split}
\end{equation}
%where ${D_A}_{D}^{i}/r_d$
where $\vec{v}_{obs}$ is a vector of the observed points (i.e., the values of $D_A/r_d$ at each $z$ in {Table}~\ref{tab:data}), $\vec{v}_{model}$ is the theoretical prediction of the model calculated with Equation~(\ref{v_model}) and $\sigma$ is the error of each~measurement.

Additionally, we use the SN and the GRB datasets to further constrain the models. For~them, we use the following marginalized over $H_0$ and $M_B$ formula, taken from~\cite{Staicova:2021ntm} so that we avoid setting priors on $H_0$ and $M_B$.

Following the approach used in (\cite{DiPietro:2002cz,Nesseris:2004wj,Perivolaropoulos:2004yr,Lazkoz:2005sp}),  the~integrated $\chi^2$ is
\begin{equation}
\tilde{\chi}^2_{SN,  GRB} = D-\frac{E^2}{F} + \ln\frac{F}{2\pi},
\end{equation}
{for%Please check and confirm format of subequations. Same below.
}
\begin{subequations}
\begin{equation}
D = \sum_i \left( \Delta\mu \, C^{-1}_{cov} \, \Delta\mu^T \right)^2,
\end{equation}
\begin{equation}
E = \sum_i \left( \Delta\mu \, C^{-1}_{cov} \, E \right),
\end{equation}
\begin{equation}
F = \sum_i  C^{-1}_{cov}  ,
\end{equation}
\end{subequations}
where  $\mu_{}^{i}$ is the observed luminosity, $\sigma_i$ is its error,~$d_L(z)$ is the luminosity distance, $\Delta\mu =\mu_{}^{i} - 5 \log_{10}\left[d_L(z_i)\right)$, $E$ is the unit matrix, and~$C^{-1}_{cov}$ is the inverse covariance matrix of the dataset. For~the GRB dataset, $C^{-1}\to 1/\sigma_i^2$ since there is no known covariance matrix for it.  For~the Pantheon dataset, the~total covariance is defined as $C_{cov}=D_{stat}+C_{sys}$, where $D_{stat}=\sigma_i^2$ comes from the measurement and $C_{sys}$ is provided separately~\cite{Deng:2018jrp}. 
Note, in~the so-defined marginalized $\chi^2$, the values of $M$ and $H_0$ do not change the marginalized $\tilde{\chi}^2_{SN}$.

The final $\chi^2$ is
$$\chi^2=\chi^2_{BAO}+\chi^2_{CMB}+\chi^2_{SN}+\chi^2_{GRB}.$$

\section{Datasets}
The dataset we are using is a collection of points from different BAO\linebreak observations~\cite{BOSS:2016goe,BOSS:2016wmc, BOSS:2016hvq,Blake:2012pj,Carvalho:2015ica,Seo:2012xy,Sridhar:2020czy,DES:2017rfo,Tamone:2020qrl,Zhu:2018edv,Hou:2020rse,Blomqvist:2019rah,duMasdesBourboux:2017mrl}, to~which we add the CMB distant prior~\cite{Chen:2018dbv} and the data from the binned Pantheon dataset, which contain $1048$ supernovae luminosity measurements in the redshift range $z\in (0.01,2.3)$ \cite{Pan-STARRS1:2017jku,Pan-STARRS1:2017jku} binned into 40 points. The~GRB dataset~\cite{Demianski:2016zxi} consists of 162 measurements in the range $z\in [0.03351,9.3]$. %???

To estimate the possible correlations in the BAO dataset, we use the methodology in~\cite{Kazantzidis:2018rnb,Benisty:2020otr}. This method avoids the use of N-body mocks to find the covariance matrices due to systematic errors and replaces it with an evaluation of the effect of possible small correlation on the final result. We add to the covariance matrix for uncorrelated points $C_{ii} = \sigma_i^2$ symmetrically a number of randomly selected nondiagonal elements $C_{ij}$. Their magnitudes are set to $C_{ij} =0.5 \sigma_i \sigma_j$, where $\sigma_i \sigma_j$ are the published $1\sigma$ errors of the data points $i,j$. We introduce positive correlations in up to 6 pairs of randomly selected data points (more than $25\%$ of the data). {Figure%Changed to Figure A1 according to layout rules, please check and confirm. Same below.
}~\ref{fig:checkCov} in {Appendix} {\ref{Appendix}} shows the corner plots with different randomized points for all the models we employ in this article. From~the plots, one can see that the effect from adding the correlations is below $10\%$ on average. {This indicates that we can consider the chosen set of BAO points for being effectively~uncorrelated.}%Please check that intended meaning is retained.

To run the inference, we use a Monte Carlo Markov Chain (MCMC) nested sampler to find the best fit. We use the open-source {package%Please check if italic is necessary? Same in text below.
} $Polychord$ \cite{Handley:2015fda} with the $GetDist$ package~\cite{Lewis:2019xzd} to present the~results. 

The prior is a uniform distribution for all the {quantities}:
%MDPI: Please check 0.  1. ; is it correct? Yes it is
 $\Omega_{m} \in [0, 1.]$, $\Omega_{\Lambda}\in[0, 1 - \Omega_{m}]$, $\Omega_r\in[0,1 - \Omega_{m} - \Omega_{\Lambda}]$, $c/ (H_0 r_d) \in [25, 35]$, $w_0 \in [-1.5, -0.5]$ and $w_a \in [-0.5,0.5]$. Since the distance prior is defined at the decoupling epoch ($z_*$) and the BAO---at drag epoch ($z_d$), we parametrize the difference between $r_s(z_*)$ and $r_s(z_d)$ as $rat= r_*/r_d$, where the prior for the ratio is $rat \in [0.9, 1.1]$.


\section{Results}
\label{sec:res}







Figure~\ref{fig:bwwa_BAO1} (as well as the figures in the Appendix \ref{Appendix})
show the final values obtained by running MCMC on the selected priors, the~numbers being in Table~\ref{results} in the {Appendix \ref{Appendix}}, where also the corner plots can be found. We see that the models differ seriously in their estimations for the physical quantities $c/(H_0 r_d), \Omega_m$ and $r_d/r_s$, probably due to the very wide prior imposed on $\Omega_m$. 

\begin{figure}[H]
\begin{tabularx}{\textwidth}{|c|c|}
\hline 
&\\[-1ex]
\includegraphics[width=0.468\textwidth]{omb_new.pdf}&
\includegraphics[width=0.468\textwidth]{omb_newSN.pdf}
\\
\includegraphics[width=0.468\textwidth]{wwa_inv.pdf}&
\includegraphics[width=0.468\textwidth]{wwa_invSN.pdf}\\
\hline
\end{tabularx}
\caption{{The 2D contour plot for the different DE parametrizations for the BAO+CMB dataset to the left and for the BAO+CMB+SN+GRB to the right. The~upper panel shows the results for $\Omega_m$ vs. $c/(H_0 r_d)$ and the lower panel shows the results for $w$ and $w_a$. $\Lambda$CDM corresponds to $w_0 = -1$ and $w_a = 0$. The~grey lines show the $1 \sigma$ and $2\sigma$ of $\Omega_m$ and $c/(H_0 r_d)$ as measured by Planck 2018, while on the bottom plot the gray cross shows where we recover $\Lambda$CDM.}}
\label{fig:bwwa_BAO1}
\end{figure}

Since we avoid the degeneracy between $r_d$ and $H_0$ by considering the combined quantity $c/(H_0 r_d)$ this leads to an explicit correlation with $\Omega_m$ for some models and rather strict bounds on the error. The~values of $\Omega_m$ closest to the ones published by Planck 2018~\cite{Aghanim:2018eyx} $\Omega_m = 0.315\pm 0.007$ are for the BA, JPB and FSLLI models for the BAO dataset {and%Footnote format changed into Notes format, according to Journal rules, please check.
} $\Lambda$CDM\endnote{In Sections~\ref{sec:res} and \ref{sec:sum} we discuss only the {\bf {flat} }$\Lambda$CDM model.
%MDPI: Is the bold necessary? % yes
The~effect of the spatial curvature on DE models has been considered recently in~\cite{Yang:2022kho}.}  and BA for the BAO+SN+GRB. The~rest significantly overestimate $\Omega_m$ . For~the ratio $r_*/r_d$ Planck 2018 gives $0.98$, the~closest models are BA, JPB and FSLLI/FSLLII models for the BAO dataset and (flat) $\Lambda$CDM, JPB and FSLLI/FSLLII for the BAO+SN+GRB. For~$c/(H_0 r_d)$, the~Planck 2018 values is $30.26\pm0.06$. Here, the models closest to this value are $\Lambda$CDM, CPL and LC for the BAO dataset and $\Lambda$CDM, CPL and LC for the BAO+SN+GRB.

The DE parameters seem to be constrained to different level for the different models. As~a whole, the~trend to better constrain $w_0$ than $w_a$, which we observed in~\cite{Staicova:2021ntm} (and the referenced inside other works), is confirmed in this case as well. Notable exceptions are the BA and LC models, where the error of $w_a$ is much smaller. For them, however, the~other parameters seem to be outside of the expected boundaries.~$\Lambda$CDM performs as expected under both~datasets.

To compare the different models, we use well-known statistical measures. The~results can be seen in  Table~\ref{stats_BAO}. In~it, we publish four selection criteria:  Akaike information criterion (AIC), Bayesian information criterion (BIC), deviance information criterion (DIC) and the Bayes factor (BF). Since  for small datasets, both AIC and BIC are dominated by the number of parameters in the model (which are 3 for $\Lambda$CDM, and 5 for the DE models), we emphasize here on the DIC and the BF which rely on the numerically evaluated likelihood and evidence, making them more unbiased. The~DIC criterion, just like the AIC, selects the best model to be the one with the minimal value of the DIC measurement. The~reference table we use for DIC is
$\Delta DIC>10$ shows strong support for the model with lower DIC , $\Delta DIC = 5\text{--}10$ shows substantial support for the model with lower DIC, and $\Delta DIC<5$ gives ambiguous support for the model with lower DIC. Here, we use the logarithmic scale for the BF, for~which $ln(BF)>1$ shows support for the base model ($\Lambda$CDM), while $ln(BF)<-1$ for the other hypothesis. $|ln(BF)|<1$ shows an inconclusive~result.

\begin{table}[H] 
\caption{Selection criteria of different models in a comparison to the $\Lambda$CDM model for the BAO dataset and the BAO+SN+GRB {dataset%Please check if this Table should have header. Please note that we changed hyphens into minus sign in whole table body. Table format revised, please check, same below.
}.\label{stats_BAO}}
\newcolumntype{C}{>{\centering\arraybackslash}X}
\begin{tabularx}{\textwidth}{CCCCCCCC}
\toprule
\multicolumn{8}{c}{BAO+CMB}\\
			\midrule
			Model & AIC & $\Delta$AIC & BIC & $\Delta BIC$ & DIC & $\Delta$DIC & ln(BF) \\
			\midrule
			$\Lambda$CDM & 22.0 &  & 24.5 &  & 16.8 &  &  \\
			\midrule
			CPL & 25.7 & $-$3.7 & 29.9 & $-$5.4 & 16.5 & 0.3 & 0.6 \\
			\midrule
			BA & 25.3 & $-$3.3 & 29.5 & $-$4.9 & 16.2 & 0.65 & $-$5.3 \\
			\midrule
			LC & 56.0 & $-$33.9 & 60.2 & $-$35.6 & 51.1 & $-$34.3 & 38.5 \\
			\midrule
			JPB & 27.8 & $-$5.8 & 31.9 & $-$7.4 & 18.6 & $-$1.8 & $-$3.5 \\
			\midrule
			FSLLI & 27.1 & $-$5.1 & 31.3 & $-$6.9 & 17.9 & $-$1.1 & $-$3.8 \\
			\midrule
			FSLLII & 26.6 & $-$4.6 & 30.8 & $-$6.3 & 17.4 & $-$0.65 & $-$4.0\\
			\midrule
			\multicolumn{8}{c}{BAO+CMB+SN+GRB}\\
			\midrule
			$\Lambda$CDM & 228.1 &  & 238.3 &  & 222.7 &  &  \\
			\midrule
			CPL & 229.2 & $-$1.1 & 246.1 & $-$7.8 & 219.9 & 2.8 & $-$1.2 \\
			\midrule
			BA & 229.0 & $-$0.9 & 246.0 & $-$7.8 & 219.8 & 2.9 & $-$5.9 \\
			\midrule
			LC & 436.8 & $-$208.7 & 453.7 & $-$215.5 & 427.6 & $-$204.9 & 208.9 \\
			\midrule
			JPB & 232.2 & $-$4.1 & 249.2 & $-$10.9 & 222.9 & $-$0.2 & $-$4.0 \\
			\midrule
			FSLLI & 231.1 & $-$2.9 & 248.0 & $-$9.7 & 221.9 & 0.9 & $-$3.7 \\
			\midrule
			FSLLII & 230.5 & $-$2.4 & 247.4 & $-$9.2 & 221.3 & 1.5 & $-$4.8 \\
			\bottomrule
\end{tabularx}
\end{table}

%\begin{table}[H]
%	\begin{center}
%		\begin{tabular}{|c|c|c|c|c|c|c|c|}
%			\toprule
%			\multicolumn{8}{|c|}{BAO+CMB}\\
%			\midrule
%			Model & AIC & $\Delta$AIC & BIC & $\Delta BIC$ & DIC & $\Delta$DIC & ln(BF) \\
%			\midrule
%			$\Lambda$CDM & 22.0 &  & 24.5 &  & 16.8 &  &  \\
%			\midrule
%			CPL & 25.7 & -3.7 & 29.9 & -5.4 & 16.5 & 0.3 & 0.6 \\
%			\midrule
%			BA & 25.3 & -3.3 & 29.5 & -4.9 & 16.2 & 0.65 & -5.3 \\
%			\midrule
%			LC & 56.0 & -33.9 & 60.2 & -35.6 & 51.1 & -34.3 & 38.5 \\
%			\midrule
%			JPB & 27.8 & -5.8 & 31.9 & -7.4 & 18.6 & -1.8 & -3.5 \\
%			\midrule
%			FSLLI & 27.1 & -5.1 & 31.3 & -6.9 & 17.9 & -1.1 & -3.8 \\
%			\midrule
%			FSLLII & 26.6 & -4.6 & 30.8 & -6.3 & 17.4 & -0.65 & -4.0\\
%			\midrule
%			\multicolumn{8}{|c|}{BAO + CMB + SN + GRB}\\
%			\midrule
%			$\Lambda$CDM & 228.1 &  & 238.3 &  & 222.7 &  &  \\
%			\midrule
%			CPL & 229.2 & -1.1 & 246.1 & -7.8 & 219.9 & 2.8 & -1.2 \\
%			\midrule
%			BA & 229.0 & -0.9 & 246.0 & -7.8 & 219.8 & 2.9 & -5.9 \\
%			\midrule
%			LC & 436.8 & -208.7 & 453.7 & -215.5 & 427.6 & -204.9 & 208.9 \\
%			\midrule
%			JPB & 232.2 & -4.1 & 249.2 & -10.9 & 222.9 & -0.2 & -4.0 \\
%			\midrule
%			FSLLI & 231.1 & -2.9 & 248.0 & -9.7 & 221.9 & 0.9 & -3.7 \\
%			\midrule
%			FSLLII & 230.5 & -2.4 & 247.4 & -9.2 & 221.3 & 1.5 & -4.8 \\
%			\bottomrule
%		\end{tabular}
%	\end{center}
%	\caption{{ Selection Criteria of different models in a comparison to the $\Lambda$CDM model for the BAO dataset and the BAO + SN + GRB dataset. }}
%\label{stats_BAO}
%\end{table}


From Table~\ref{stats_BAO}, we see that the AIC and BIC for all models show a preference for $\Lambda$CDM. For~the DIC criterion, we see a slight possibility for a preference for other models in the case of the CPL and BA models for both tested datasets. For~the BF, we see that there is some possible preference for BA, JPN and FSLLI/FSLLII for the BAO+CMB case and for CPL and BA, JPN and FSLLI/FSLLII in the BAO+CMB+SN+GRB case. The~results of the LC model show that it is underfitting the data (from the $\chi^2/dof$$\sim$$2$) and the statistics for it is not reliable. This demonstrates another benefit of performing the statistical~analysis.





The preference for the BA and LC models which we observe was also observed in the results of~\cite{Escamilla-Rivera:2021boq}, where the authors studied a dataset consisting of  SN, cosmic chronometers and gravitational~waves.

The BAO dataset we use combines the $H_0$ and the $r_d$ into one quantity. Therefore, we estimate the new variable $c/(H_0 r_d)$$\sim$$30$. Figure~\ref{fig:H0rdvalGRB} shows the values of the $c/(H_0 r_d)$ for different models vs. the result from Planck 2018: $30.24 \pm 0.08$. For~comparison, the~most recent local measurement by SH0ES is $30.19 \pm 0.53$, corresponding to $H_0=73.01 \pm 0.99 \; \text{km s}^{-1} \text{Mpc}^{-1}$ \cite{Riess:2022mme}. We do not put it on the plot, because~the $r_d$ used to obtain it is the indirect result from inference on the H0LiCOW+SN+BAO+SH0ES dataset~\cite{Arendse:2019hev}. It is, however, clearly very close to the Planck value, as~expected. 



On Figure~\ref{fig:H0rdvalGRB}, we superimpose the BAO+CMB-only result with the BAO+CMB+SN+GRB one. This figure enables us to visually track the tension between the Planck 2018 results and the datasets we use, which are mostly local universe ones (except for the 2 CMB points). We see that the tension is now between $c/(H_0 r_d)$ and $\Omega_m$. The~models whose bounds cross with the Planck 2018 one for $c/(H_0 r_d)$ are $\Lambda$CDM, CPL and LC for BAO+CMB and only LC for the BAO+CMB+SN+GRB dataset. For~$\Omega_m$, the models that enter the interval are all but $\Lambda$CDM and $CPL$ for the BAO+CMB dataset and  $\Lambda$CDM, BA, LC for the BAO+CMB+SN+GRB dataset. We see that the inclusion of the new datasets decreases the number of models satisfying the constraints. The~only model that is not in tension is LC because of its huge error. Notably, in~this approach, $\Lambda$CDM, while satisfying the bounds for $c/(H_0 r_d)$, does not satisfy them for $\Omega_m$. % ($\Omega_m^{Pl18}=0.3153 \pm 0.0073$) 

\vspace{-6pt}
\begin{figure}[H]
 	%\centering
\includegraphics[width=0.495\textwidth]{b_compareGRBBAO.pdf}
\includegraphics[width=0.495\textwidth]{om_compareGRBBAO.pdf}

\caption{{ {The final values of the $c/(H_0 r_d)$ from different DE models, compared to the values from Planck for the BAO+CMB+SN+GRB dataset. The~smaller, darker, errorbox are for BAO+CMB, the~lighter, bigger errorbox--for SN+GRB.}}}
 	\label{fig:H0rdvalGRB}
\end{figure}

From the plot, one can see that in general adding the new datasets decrease the errors, but~they do not move the mean values in the same direction and the overall effect is not very big. This may be due to unknown errors in the SN+GRB dataset or to the fact that this dataset is not sensitive toward the combined variable $c/(H_0 r_d)$ since we have marginalized over $H_0$ so that we do not have to impose a prior on $r_d$. Because~of this, the~only effect the SN+GRB dataset has on the combined variable is indirect, through $\Omega_m$ and the other parameters. It could also point to some inconsistency in the $\mu(M_B)$ relation such as the ones considered in~\cite{Benisty:2022psx,Ferramacho:2008ap,Linden:2009vh,Tutusaus:2017ibk,DiValentino:2020evt,Perivolaropoulos:2022khd}) questioning the assumption that $M_B=const$.





\section{Discussion}
\label{sec:sum}
This paper uses the combination $H_0\cdot r_d$ to avoid the degeneracy between $H_0$ and $r_d$ which has plagued the use of BAO measurements and could be part of the resolution of cosmological tensions. The~use of a combined parameter avoids imposing separate priors on $H_0$ and $r_d$ and thus it avoids additional assumptions on them. We use points from the late universe, the~BAO dataset ($z<2.4$)), few points from the early universe (the CMB distant priors, ($z  \simeq  1089$)), to~which we add SN data and GRB datasets, properly marginalized, to~make a statistical comparison between different DE~models.

The results show that the tension is now between the new parameter $c/(H_0 r_d)$ and $\Omega_m$---the only model that fits in the constraints set by Planck 2018 is LC, which comes with the biggest error. For~the rest of the models, one of the two parameters do not fit the constraints, even if some of them somewhat reduce the tension. Statistically, there is a preference for the $\Lambda$CDM model over the DE models in most cases. It is worth noting that there is strong evidence in support of $\Lambda$CDM compared to all other models only when using AIC and BIC, while from DIC and BF, the~support is not substantial, and it even slightly favors other models. This result raises the question of the use of different statistical measures when comparing DE models, and~also it opens the possibility that a better DE model may eventually help in reducing both the $H_0$ tension and the $r_d$ tension. 

Another interesting point is that for some models, the~known impossibility to constrain $w_a$ is eliminated and $w_a$ has very tight bounds. These models, LC and BA and somewhat FSLLI, show interesting new possibilities for DE models. Furthermore, the~choice of datasets and models make explicit the degeneracy between $H_0\cdot r_d$ and $\Omega_m$, emphasizing the need to find a way to disentangle the three quantities---$H_0, r_d$ and $\Omega_m$---if we are to understand the cosmological tensions. The~results show that adding the SN and GRB datasets decrease the errors on the constrained parameters, but~they do not move them in the same direction for each model. We see that combining different datasets and different marginalization techniques, along with the use of statistical measures, is a promising tool to study new cosmological~models.

\vspace{6pt}

\funding{{Bulgarian National Science Fund research grants KP-06-N58/5/19.11.2021.}%Please add: ``This research received no external funding'' or ``This research was funded by NAME OF FUNDER grant number XXX.'' and  and ``The APC was funded by XXX''. Check carefully that the details given are accurate and use the standard spelling of funding agency names at \url{https://search.crossref.org/funding}, any errors may affect your future funding.}
}


\dataavailability{All the data we used in this paper were taken from the corresponding citations and available to~use.} 

\acknowledgments{D.S. thanks David Benisty for the useful comments and discussions. D.S. is thankful to Bulgarian National Science Fund for support via research grants KP-06-N58/5.}

\conflictsofinterest{{The authors declare no conflict of interest.}%Declare conflicts of interest or state ``The authors declare no conflict of interest.'' Authors must identify and declare any personal circumstances or interest that may be perceived as inappropriately influencing the representation or interpretation of reported research results. Any role of the funders in the design of the study; in the collection, analyses or interpretation of data; in the writing of the manuscript, or in the decision to publish the results must be declared in this section. If there is no role, please state ``The funders had no role in the design of the study; in the collection, analyses, or interpretation of data; in the writing of the manuscript, or in the decision to publish the~results''.}

}







%\bibliographystyle{apsrev4-1}
%\bibliography{ref}

%\newpage
%%%%%%%%%%%%%%%%%%%%%%%%%%%%%%%%%%%%%%%%%%%%%%%%%%

%%%%%%%%%%%%%%%%% APPENDICES %%%%%%%%%%%%%%%%%%%%%

\appendixtitles{yes} % Leave argument "no" if all appendix headings stay EMPTY (then no dot is printed after "Appendix A"). If the appendix sections contain a heading then change the argument to "yes".
\appendixstart
\appendix
\section[\appendixname~\thesection]{Some Extra~{Material}\label{Appendix}}
%\subsection[\appendixname~\thesubsection]{}

%\section{}


\begin{table}[H] 
\caption{{The} uncorrelated dataset used in this paper. For~each redshift, the table presents the parameter, the~mean value, and~the corresponding error bar. The~reference and the collaboration are also reported.\label{tab:data}}


\begin{adjustwidth}{-\extralength}{0cm}
%\centering %% If there is a figure in wide page, please release command \centering

\newcolumntype{C}{>{\centering\arraybackslash}X}
\begin{tabularx}{\fulllength}{CCCCm{8cm}<{\centering}C}
\toprule
$\mathbold{z}$   & $\mathbold{D_A/{r_d}}$ & \textbf{Error} & \textbf{Year}  & \textbf{Survey} &  \textbf{Ref.} \\
\midrule
$0.11$  & $2.607$ & $0.138$&  $2021$ & SDSS blue galaxies & \cite{deCarvalho:2021azj}\\
$0.24$   & $5.594$& $0.305$&  $2016$ &BOSS-DR12 RSD of LOWZ and CMASS & \cite{BOSS:2016goe}\\
$0.32$    &  $6.636$  &  $0.11$   &  $2016$  &  SDSS-DR9+DR10+DR11+DR12 +covariance  & \cite{BOSS:2016wmc} \\
$0.38$    &  $7.389$  &  $0.122$  &  $2019$  &  BOSS-DR12 power spectrum & \cite{BOSS:2016hvq}\\
$0.44$    &  $8.19$  &  $0.77 $  &   $2012$  &  WiggleZ (galaxy clustering)  & \cite{Blake:2012pj}\\
%$0.51$   &  $7.893$  &  $0.279$  &   $2015$  &  BOSS-DR10 ang. gal. clust. & \cite{Carvalho:2015ica}	\\
$0.54$    &  $9.212$  &  $0.41$    &  $2012$  &  SDSS-III DR8 (luminous galaxies) & \cite{Seo:2012xy}\\
$0.6$    &  $9.37$  &  $0.65$   &  $2012$  &   WiggleZ (galaxy clustering) & \cite{Blake:2012pj}\\
$0.697$    &  $10.18$  &  $0.52$   &  $2020$  &  DECals DR8 (LRG)  & \cite{Sridhar:2020czy}\\
$0.73$    &  $10.42$  &  $0.73$  &   $2012$  &  Wiggle (galaxy clustering) & \cite{Blake:2012pj}\\
$0.81$    &  $10.75$  &  $0.43$   &  $2017$  &  DES Year1  (galaxy clustering) & \cite{DES:2017rfo}\\
$0.85$    &  $10.76$  &  $  0.54$   &  $ 2020$  &  eBOSS DR16 ELG & \cite{Tamone:2020qrl}\\
$0.874$    &  $11.41$  &  $0.74$    &  $2020$  &  DECals DR8 (LRG) & \cite{Sridhar:2020czy}\\
$1.00$    &  $11.521$  &  $1.032$    &  $2019$  & eBOSS DR14 quasar clustering & \cite{Zhu:2018edv}\\
%$1.480$    &  $12.18$  &  $0.32$  &  $2020$  &  eBOSS DR16 BAO+RSD consensus  & \cite{Hou:2020rse}\\
$2.00$    &  $12.011$  &  $0.562$  &   $2019$  &  eBOSS DR14 quasars clustering & \cite{Zhu:2018edv}\\
$2.35$    &  $10.83$  &  $0.54$  &  $2019$  &  BOSS DR14 Lya and quasars & \cite{Blomqvist:2019rah}  \\
$2.4$    &  $10.5$  &  $0.34$  & 2017 & SDSS-III/DR12 & \cite{duMasdesBourboux:2017mrl}\\
\bottomrule
\end{tabularx}

\end{adjustwidth}
\end{table}



%\begin{table}[H]
%\scalebox{1.1}{
%\begin{tabular}{ccccccc}
%\toprule\hline
%
%$z$   & $D_A/{r_d}$ & Error & year  & Survey &  Ref. \\ \midrule\hline
%$0.11$  & $2.607$ & $0.138$&  $2021$ & SDSS blue galaxies & \cite{deCarvalho:2021azj}\\
%$0.24$   & $5.594$& $0.305$&  $2016$ &BOSS-DR12 RSD of LOWZ and CMASS & \cite{BOSS:2016goe}\\
%$0.32$    &  $6.636$  &  $0.11$   &  $2016$  &  SDSS-DR9+DR10+DR11+DR12+covariance  & \cite{BOSS:2016wmc} \\
%$0.38$    &  $7.389$  &  $0.122$  &  $2019$  &  BOSS-DR12 power spectrum & \cite{BOSS:2016hvq}\\
%$0.44$    &  $8.19$  &  $0.77 $  &   $2012$  &  WiggleZ (galaxy clustering)  & \cite{Blake:2012pj}\\
%%$0.51$   &  $7.893$  &  $0.279$  &   $2015$  &  BOSS-DR10 ang. gal. clust. & \cite{Carvalho:2015ica}	\\
%$0.54$    &  $9.212$  &  $0.41$    &  $2012$  &  SDSS-III DR8 (luminous galaxies) & \cite{ Seo:2012xy}\\
%$0.6$    &  $9.37$  &  $0.65$   &  $2012$  &   WiggleZ (galaxy clustering) & \cite{Blake:2012pj}\\
%$0.697$    &  $10.18$  &  $0.52$   &  $2020$  &  DECals DR8 (LRG)  & \cite{Sridhar:2020czy}\\
%$0.73$    &  $10.42$  &  $0.73$  &   $2012$  &  Wiggle (galaxy clustering) & \cite{Blake:2012pj}\\
%$0.81$    &  $10.75$  &  $0.43$   &  $2017$  &  DES Year1  (galaxy clustering) & \cite{DES:2017rfo}\\
%$0.85$    &  $10.76$  &  $  0.54$   &  $ 2020$  &  eBOSS DR16 ELG & \cite{Tamone:2020qrl}\\
%$0.874$    &  $11.41$  &  $0.74$    &  $2020$  &  DECals DR8 (LRG) & \cite{Sridhar:2020czy}\\
%$1.00$    &  $11.521$  &  $1.032$    &  $2019$  & eBOSS DR14 quasar clustering & \cite{Zhu:2018edv}\\
%%$1.480$    &  $12.18$  &  $0.32$  &  $2020$  &  eBOSS DR16 BAO+RSD consensus  & \cite{Hou:2020rse}\\
%$2.00$    &  $12.011$  &  $0.562$  &   $2019$  &  eBOSS DR14 quasars clustering & \cite{Zhu:2018edv}\\
%$2.35$    &  $10.83$  &  $0.54$  &  $2019$  &  BOSS DR14 Lya and quasars & \cite{Blomqvist:2019rah}  \\
%$2.4$    &  $10.5$  &  $0.34$  & 2017 & SDSS-III/DR12 & \cite{duMasdesBourboux:2017mrl}\\
%
%
%\midrule\hline
%\end{tabular}
%}
%\caption{{The uncorrelated dataset used in this paper. For~each redshift the table presents the parameter, the~mean value, and~the corresponding error bar. The~reference and the collaboration is also reported.}}
%\label{tab:data}
%\end{table}
%\unskip
\vspace{-10pt}
\begin{table}[H] 
\caption{{The} 68 $\% $ C.L. limits for $R$, $l_A$, in~different cosmological models and their correlation matrix for from Planck $2018$ $TT,TE,EE+lowE$; see the text for details.\label{distancePr}}
\newcolumntype{C}{>{\centering\arraybackslash}X}
\begin{tabularx}{\textwidth}{Cm{5cm}<{\centering}CC}
\toprule
$\Lambda$CDM & $Planck~ \textrm{TT,TE,EE}+\textrm{lowE} $ & $R$ & $l_\textrm{A}$ \\
\midrule
$R$  & $1.7502\pm0.0046$ & $1.0$&    $0.46$ \\
$l_\textrm{A}$  & $301.471^{+0.089}_{-0.090}$ & $0.46$&    $1.0$  \\
\midrule
  $w$CDM & $Planck~ \textrm{TT,TE,EE}+\textrm{lowE} $ & $R$ & $l_\textrm{A}$  \\
\midrule
$R$ & $1.7493^{+0.0046}_{-0.0047}$  & $1.0$&    $0.47$\\ \specialrule{0em}{1pt}{1pt}
$l_\textrm{A}$ & $301.462^{+0.089}_{-0.090}$ & $0.47$&    $1.0$ \\
\midrule
   $\Omega_k \Lambda$CDM & $Planck~ \textrm{TT,TE,EE}+\textrm{lowE} $ & $R$ & $l_\textrm{A}$\\
\midrule
$R$ & $1.7429\pm0.0051$ & $1.0$ & $0.54$  \\
$l_\textrm{A}$ & $301.409\pm0.091$ & $0.54$&   $1.0$\\
\bottomrule
\end{tabularx}
\end{table}

%
%\begin{table}[H]
%\centering
%\renewcommand{\arraystretch}{1.5}
%%\tiny
%\begin{tabular}{|c|c|cc|}
%\toprule\hline
%  $\Lambda$CDM & $Planck~ \textrm{TT,TE,EE}+\textrm{lowE} $ & $R$ & $l_\textrm{A}$ \\
%\midrule
%$R$  & $1.7502\pm0.0046$ & $1.0$&    $0.46$ \\
%$l_\textrm{A}$  & $301.471^{+0.089}_{-0.090}$ & $0.46$&    $1.0$  \\
%\midrule\hline
%  $w$CDM & $Planck~ \textrm{TT,TE,EE}+\textrm{lowE} $ & $R$ & $l_\textrm{A}$  \\
%\midrule
%$R$ & $1.7493^{+0.0046}_{-0.0047}$  & $1.0$&    $0.47$\\
%$l_\textrm{A}$ & $301.462^{+0.089}_{-0.090}$ & $0.47$&    $1.0$ \\
%\midrule\hline
%   $\Omega_k \Lambda$CDM & $Planck~ \textrm{TT,TE,EE}+\textrm{lowE} $ & $R$ & $l_\textrm{A}$\\
%\midrule
%$R$ & $1.7429\pm0.0051$ & $1.0$ & $0.54$  \\
%$l_\textrm{A}$ & $301.409\pm0.091$ & $0.54$&   $1.0$\\
%\midrule\hline
%\end{tabular}
%\caption{The 68 $\% $ C.L. limits for $R$, $l_A$, in~different cosmological models and their correlation matrix for from Planck $2018$ $TT,TE,EE+lowE$, see the text for details % \cite{Aghanim:2018eyx}. The~data is taken from the Table~1 of~\cite{Chen:2018dbv}.
%}
%\label{distancePr}
%\end{table}
%\unskip
%%this error seems to be related to the citations, I don't know why.

\begin{table}[H] 
\caption{{The} posterior values for $c/(H_0 r_d)$, $\Omega_m$, $r_*/r_d$ and $w_0,w_a$ for different parametrization of DE for the BAO+CMB dataset (top) and for the BAO+CMB+SN+GRB (bottom).\label{results}}
 \resizebox{\textwidth}{!}{  
    \begin{tabular}{cccccc}
\toprule
\multicolumn{6}{c}{BAO+CMB}\\
			\midrule
			Model & $c/(H_0 r_d)$ & $\Omega_m$ & $r_*/r_d$ & w & $w_a$ \\
			\midrule
			$\Lambda$CDM & $29.96\pm 0.3$ & $0.36\pm 0.02$ & $0.92\pm 0.01$ & $-$1.000 & 0.000 \\
			\midrule
			CPL & $29.42\pm 0.85$ & $0.35\pm 0.02$ & $0.91\pm 0.01$ & $-1.14\pm 0.19$ & $0.13\pm 0.31$ \\
			\midrule
			BA & $27.58\pm 1.74$ & $0.29\pm 0.04$ & $0.93\pm 0.02$ & $-1.06\pm 0.11$ & $0.35\pm 0.12$ \\
			\midrule
			LC & $30.64\pm 2.87$ & $0.38\pm 0.07$ & $0.901\pm 0.0009$ & $-0.5082\pm 0.0072$ & $-0.4979\pm 0.0018$ \\
			\midrule
			JPB & $28.03\pm 1.76$ & $0.29\pm 0.04$ & $0.94\pm 0.02$ & $-0.86\pm 0.09$ & $0.08\pm 0.31$ \\
			\midrule
			FSLLI & $27.97\pm 1.81$ & $0.29\pm 0.04$ & $0.94\pm 0.02$ & $-0.92\pm 0.1$ & $0.22\pm 0.24$ \\
			\midrule
			FSLLII & $28.1\pm 1.79$ & $0.3\pm 0.04$ & $0.94\pm 0.02$ & $-0.91\pm 0.09$ & $0.26\pm 0.2$ \\
			\midrule
		    \multicolumn{6}{c}{BAO+CMB+SN+GRB}\\
			\midrule
			$\Lambda$CDM & $29.51\pm 0.24$ & $0.32\pm 0.01$ & $0.95\pm 0.01$ & $-$1.000 & 0.000 \\
			\midrule
			CPL & $29.27\pm 0.25$ & $0.34\pm 0.01$ & $0.91\pm 0.01$ & $-1.15\pm 0.06$ & $0.09\pm 0.31$ \\
			\midrule
			BA & $27.44\pm 1.36$ & $0.28\pm 0.03$ & $0.94\pm 0.01$ & $-1.13\pm 0.04$ & $0.37\pm 0.1$ \\
			\midrule
			LC & $31.11\pm 2.86$ & $0.39\pm 0.07$ & $0.9009\pm 0.0007$ & $-0.55\pm 0.02$ & $-0.4992\pm 0.0007$ \\
			\midrule
			JPB & $27.58\pm 1.38$ & $0.27\pm 0.03$ & $0.9657\pm 0.0084$ & $-1.02\pm 0.04$ & $0.22\pm 0.23$ \\
			\midrule
			FSLLI & $27.37\pm 1.47$ & $0.27\pm 0.03$ & $0.9614\pm 0.0082$ & $-1.06\pm 0.04$ & $0.32\pm 0.14$ \\
			\midrule
			FSLLII & $27.35\pm 1.36$ & $0.27\pm 0.03$ & $0.9556\pm 0.0099$ & $-1.04\pm 0.03$ & $0.35\pm 0.13$ \\
			\bottomrule
 \end{tabular}%
    }
\end{table}

\begin{figure}[H]
 	
(\textbf{a}) \includegraphics[width=0.25\textwidth]{LCDM_cov.pdf}
(\textbf{b}) \includegraphics[width=0.25\textwidth]{CPL_cov.pdf}
(\textbf{c}) \includegraphics[width=0.25\textwidth]{BA_cov.pdf}\\
(\textbf{d}) \includegraphics[width=0.25\textwidth]{JBP_cov.pdf}
(\textbf{e}) \includegraphics[width=0.25\textwidth]{FSLLI_cov.pdf}
 \caption{{{The} covariance test plot for the considered models: (\textbf{a}) $\Lambda$CDM, (\textbf{b}) $CPL$, (\textbf{c}) $BA$, (\textbf{d}) $JPB$, (\textbf{e}) $FSLLI$ model.}}
\label{fig:checkCov}
\end{figure}
\unskip








\begin{figure}[H]
 	
\includegraphics[width=0.45\textwidth]{bwwa_inv_all.pdf}
\includegraphics[width=0.45\textwidth]{bwwa_inv_allSN.pdf}
	\caption{{The} posterior distribution for $c/(H_0 r_d)$, $\Omega_m$, $r_*/r_d$ and $w_0,w_a$ for different parametrization of DE for the BAO+CMB dataset to the left and for the BAO+CMB+SN+GRB to the~right. }
\label{results_BAO}
\end{figure}






%
%\begin{table}[H]
% 	\begin{tabular}{|c|c|c|c|c|c|}
% 	\toprule
%			\multicolumn{6}{|c|}{BAO + CMB}\\
%			\midrule
%			Model & $c/(H_0 r_d)$ & $\Omega_m$ & $r_*/r_d$ & w & $w_a$ \\
%			\midrule
%			$\Lambda$CDM & $29.96\pm 0.3$ & $0.36\pm 0.02$ & $0.92\pm 0.01$ & -1.000 & 0.000 \\
%			\midrule
%			CPL & $29.42\pm 0.85$ & $0.35\pm 0.02$ & $0.91\pm 0.01$ & $-1.14\pm 0.19$ & $0.13\pm 0.31$ \\
%			\midrule
%			BA & $27.58\pm 1.74$ & $0.29\pm 0.04$ & $0.93\pm 0.02$ & $-1.06\pm 0.11$ & $0.35\pm 0.12$ \\
%			\midrule
%			LC & $30.64\pm 2.87$ & $0.38\pm 0.07$ & $0.901\pm 0.0009$ & $-0.5082\pm 0.0072$ & $-0.4979\pm 0.0018$ \\
%			\midrule
%			JPB & $28.03\pm 1.76$ & $0.29\pm 0.04$ & $0.94\pm 0.02$ & $-0.86\pm 0.09$ & $0.08\pm 0.31$ \\
%			\midrule
%			FSLLI & $27.97\pm 1.81$ & $0.29\pm 0.04$ & $0.94\pm 0.02$ & $-0.92\pm 0.1$ & $0.22\pm 0.24$ \\
%			\midrule
%			FSLLII & $28.1\pm 1.79$ & $0.3\pm 0.04$ & $0.94\pm 0.02$ & $-0.91\pm 0.09$ & $0.26\pm 0.2$ \\
%			\midrule
%		    \multicolumn{6}{|c|}{BAO + CMB + SN + GRB}\\
%			\midrule
%			$\Lambda$CDM & $29.51\pm 0.24$ & $0.32\pm 0.01$ & $0.95\pm 0.01$ & -1.000 & 0.000 \\
%			\midrule
%			CPL & $29.27\pm 0.25$ & $0.34\pm 0.01$ & $0.91\pm 0.01$ & $-1.15\pm 0.06$ & $0.09\pm 0.31$ \\
%			\midrule
%			BA & $27.44\pm 1.36$ & $0.28\pm 0.03$ & $0.94\pm 0.01$ & $-1.13\pm 0.04$ & $0.37\pm 0.1$ \\
%			\midrule
%			LC & $31.11\pm 2.86$ & $0.39\pm 0.07$ & $0.9009\pm 0.0007$ & $-0.55\pm 0.02$ & $-0.4992\pm 0.0007$ \\
%			\midrule
%			JPB & $27.58\pm 1.38$ & $0.27\pm 0.03$ & $0.9657\pm 0.0084$ & $-1.02\pm 0.04$ & $0.22\pm 0.23$ \\
%			\midrule
%			FSLLI & $27.37\pm 1.47$ & $0.27\pm 0.03$ & $0.9614\pm 0.0082$ & $-1.06\pm 0.04$ & $0.32\pm 0.14$ \\
%			\midrule
%			FSLLII & $27.35\pm 1.36$ & $0.27\pm 0.03$ & $0.9556\pm 0.0099$ & $-1.04\pm 0.03$ & $0.35\pm 0.13$ \\
%			\bottomrule
%		\end{tabular}
%		\caption{The posterior values for $c/(H_0 r_d)$, $\Omega_m$, $r_*/r_d$ and $w_0,w_a$ for different parametrization of DE for the BAO + CMB dataset (top) and for the BAO+CMB+SN+GRB (bottom).}
%\label{results}
%\end{table}


\begin{adjustwidth}{-\extralength}{0cm}
%\printendnotes[custom] % Un-comment to print a list of endnotes
\printendnotes[custom]

\reftitle{References}

% Please provide either the correct journal abbreviation (e.g. according to the “List of Title Word Abbreviations” http://www.issn.org/services/online-services/access-to-the-ltwa/) or the full name of the journal.
% Citations and References in Supplementary files are permitted provided that they also appear in the reference list here. 

%=====================================
% References, variant A: external bibliography
%=====================================
%\bibliography{your_external_BibTeX_file}

%=====================================
% References, variant B: internal bibliography
%=====================================
\begin{thebibliography}{999}

\bibitem[Freedman \em{et~al.}(2001)Freedman et~al.]{Freedman:2000cf}
Freedman, W.L.; {Madore, B.F.; Gibson, B.K.; Ferrarese, L.; Kelson, D.D.; Sakai, S.; Mould, J.R.; Kennicutt, J.R.C.; Ford, H.C.; Graham, J.A.;} et al.
%MDPI: Newly added author name information, please confirm. Same for ref. 2, 3, 6, 47, 65, 74, 80--83, 85, 87--90, 92--94, 108--112. 
%MDPI: Ref. 7, 8, 12, 17: Please include the first ten authors' names before using “et al.” in the references. 
% These are all collaboration papers,
%MDPI: References 80&108;  81&109;  87=>111;  92=>112;  82=>110;  93=>94; are the same. Please check whether it should be replaced with another publication or not. If not, please delete one of them and renumber the references, and revise citation in main text.
\newblock {Final results from the Hubble Space Telescope key project to measure
  the Hubble constant}.
\newblock {\em Astrophys. J.} {\bf 2001}, {\em 553},~47--72.
  %\href{http://xxx.lanl.gov/abs/astro-ph/0012376}{{\normalfont
  %[astro-ph/0012376]}}.
\newblock
{\changeurlcolor{black}\href{https://doi.org/10.1086/320638}{\detokenize{https://doi.org/10.1086/320638}}}.

\bibitem[Riess \em{et~al.}(1998)Riess et~al.]{Riess:1998cb}
Riess, A.G.; {Filippenko, A.V.; Challis, P.; Clocchiatti, A.; Diercks, A.; Garnavich, P.M.; Gilliland, R.L.; Hogan, C.J.; Jha, S.; Kirshner, R.P.;} et al.
\newblock {Observational evidence from supernovae for an accelerating universe
  and a cosmological constant}.
\newblock {\em Astron. J.} {\bf 1998}, {\em 116},~1009--1038.
  %\href{http://xxx.lanl.gov/abs/astro-ph/9805201}{{\normalfont
  %[astro-ph/9805201]}}.
\newblock
{\changeurlcolor{black}\href{https://doi.org/10.1086/300499}{\detokenize{https://doi.org/10.1086/300499}}}.

\bibitem[Perlmutter \em{et~al.}(1999)Perlmutter et~al.]{Perlmutter:1998np}
Perlmutter, S.; {Aldering, G.; Goldhaber, G.; Knop, R.A.; Nugent, P.; Castro, P.G.; Deustua, S.; Fabbro, S.; Goobar, A.; Groom, D.E.;} et~al.
\newblock {Measurements of $\Omega$ and $\Lambda$ from 42 high redshift
  supernovae}.
\newblock {\em Astrophys. J.} {\bf 1999}, {\em 517},~565--586.
  %\href{http://xxx.lanl.gov/abs/astro-ph/9812133}{{\normalfont
  %[astro-ph/9812133]}}.
\newblock
{\changeurlcolor{black}\href{https://doi.org/10.1086/307221}{\detokenize{https://doi.org/10.1086/307221}}}.

\bibitem[Riess \em{et~al.}(2021)Riess, Casertano, Yuan, Bowers, Macri, Zinn,
  and Scolnic]{Riess:2020fzl}
Riess, A.G.; Casertano, S.; Yuan, W.; Bowers, J.B.; Macri, L.; Zinn, J.C.;
  Scolnic, D.
\newblock {Cosmic Distances Calibrated to 1\% Precision with Gaia EDR3
  Parallaxes and Hubble Space Telescope Photometry of 75 Milky Way Cepheids
  Confirm Tension with $\Lambda$CDM}.
\newblock {\em Astrophys. J. Lett.} {\bf 2021}, {\em 908},~L6.
  %\href{http://xxx.lanl.gov/abs/2012.08534}{{\normalfont
  %[arXiv:astro-ph.CO/2012.08534]}}.
\newblock
{\changeurlcolor{black}\href{https://doi.org/10.3847/2041-8213/abdbaf}{\detokenize{https://doi.org/10.3847/2041-8213/abdbaf}}}.

\bibitem[Riess \em{et~al.}(2022)Riess, Breuval, Yuan, Casertano,
  \textasciitilde{}Macri, Scolnic, Cantat-Gaudin, Anderson, and
  Reyes]{Riess:2022mme}
Riess, A.G.; Breuval, L.; Yuan, W.; Casertano, S.; Macri,
  L.M.; Scolnic, D.; Cantat-Gaudin, T.; Anderson, R.I.; Reyes, M.C.
\newblock {Cluster Cepheids with High Precision Gaia Parallaxes, Low Zeropoint
  Uncertainties, and Hubble Space Telescope {Photometry.}} \emph{arXiv} {\bf 2022}, arXiv:2208.01045.
\newblock  %\href{http://xxx.lanl.gov/abs/2208.01045}{{\normalfont
  %[arXiv:astro-ph.CO/2208.01045]}}.

\bibitem[Troxel \em{et~al.}(2018)Troxel et~al.]{Troxel:2017xyo}
Troxel, M.A.; {MacCrann, N.; Zuntz, J.; Eifler, T.F.; Krause, E.; Dodelson, S.; Gruen, D.; Blazek, J.; Friedrich, O.; Samuroff, S.;} et al.
\newblock {Dark Energy Survey Year 1 results: Cosmological constraints from
  cosmic shear}.
\newblock {\em Phys. Rev. D} {\bf 2018}, {\em 98},~043528.
  %\href{http://xxx.lanl.gov/abs/1708.01538}{{\normalfont
  %[arXiv:astro-ph.CO/1708.01538]}}.
\newblock
{\changeurlcolor{black}\href{https://doi.org/10.1103/PhysRevD.98.043528}{\detokenize{https://doi.org/10.1103/PhysRevD.98.043528}}}.

\bibitem[Aghanim \em{et~al.}(2020)Aghanim et~al.]{Aghanim:2018eyx}
Aghanim, N.;  {et~al.}
%MDPI: Please include the first ten authors' names before using “et al.” in the references. 
\newblock {Planck 2018 results. VI. Cosmological parameters}.
\newblock {\em Astron. Astrophys.} {\bf 2020}, {\em 641},~A6.
  %\href{http://xxx.lanl.gov/abs/1807.06209}{{\normalfont
  %[arXiv:astro-ph.CO/1807.06209]}}.
\linebreak \newblock
{\changeurlcolor{black}\href{https://doi.org/10.1051/0004-6361/201833910}{\detokenize{https://doi.org/10.1051/0004-6361/201833910}}}.

\bibitem[Ade \em{et~al.}(2016)Ade et~al.]{Ade:2015xua}
Ade, P.A.R.;  {et~al.}
\newblock {Planck 2015 results. XIII. Cosmological parameters}.
\newblock {\em Astron. Astrophys.} {\bf 2016}, {\em 594},~A13.
  %\href{http://xxx.lanl.gov/abs/1502.01589}{{\normalfont
  %[arXiv:astro-ph.CO/1502.01589]}}.
\linebreak \newblock
{\changeurlcolor{black}\href{https://doi.org/10.1051/0004-6361/201525830}{\detokenize{https://doi.org/10.1051/0004-6361/201525830}}}.

\bibitem[Dainotti \em{et~al.}(2021)Dainotti, De~Simone, Schiavone, Montani,
  Rinaldi, and Lambiase]{Dainotti:2021pqg}
Dainotti, M.G.; De~Simone, B.; Schiavone, T.; Montani, G.; Rinaldi, E.;
  Lambiase, G.
\newblock {On the Hubble constant tension in the SNe Ia Pantheon sample}.
\newblock {\em Astrophys. J.} {\bf 2021}, {\em 912},~150.
  %\href{http://xxx.lanl.gov/abs/2103.02117}{{\normalfont
  %[arXiv:astro-ph.CO/2103.02117]}}.
\newblock
{\changeurlcolor{black}\href{https://doi.org/10.3847/1538-4357/abeb73}{\detokenize{https://doi.org/10.3847/1538-4357/abeb73}}}.

\bibitem[Benisty \em{et~al.}(2021)Benisty, Vasak, Kirsch, and
  Struckmeier]{Benisty:2021wxi}
Benisty, D.; Vasak, D.; Kirsch, J.; Struckmeier, J.
\newblock {Low-redshift constraints on covariant canonical Gauge theory of
  gravity}.
\newblock {\em Eur. Phys. J. C} {\bf 2021}, {\em 81},~125.
  %\href{http://xxx.lanl.gov/abs/2101.07566}{{\normalfont
  %[arXiv:gr-qc/2101.07566]}}.
\newblock
{\changeurlcolor{black}\href{https://doi.org/10.1140/epjc/s10052-021-08924-0}{\detokenize{https://doi.org/10.1140/epjc/s10052-021-08924-0}}}.

\bibitem[Capozziello and De~Laurentis(2011)]{Capozziello:2011et}
Capozziello, S.; De~Laurentis, M.
\newblock {Extended Theories of Gravity}.
\newblock {\em Phys. Rept.} {\bf 2011}, {\em 509},~167--321.
  %\href{http://xxx.lanl.gov/abs/1108.6266}{{\normalfont
  %[arXiv:gr-qc/1108.6266]}}.
\newblock
{\changeurlcolor{black}\href{https://doi.org/10.1016/j.physrep.2011.09.003}{\detokenize{https://doi.org/10.1016/j.physrep. 2011.09.003}}}.

\bibitem[Bull \em{et~al.}(2016)Bull et~al.]{Bull:2015stt}
Bull, P.;  {et~al.}
\newblock {Beyond $\Lambda$CDM: Problems, solutions, and the road ahead}.
\newblock {\em Phys. Dark Univ.} {\bf 2016}, {\em 12},~56--99.
  %\href{http://xxx.lanl.gov/abs/1512.05356}{{\normalfont
  %[arXiv:astro-ph.CO/1512.05356]}}.
\linebreak\newblock 
{\changeurlcolor{black}\href{https://doi.org/10.1016/j.dark.2016.02.001}{\detokenize{https://doi.org/10.1016/j.dark.2016.02.001}}}.

\bibitem[Di~Valentino \em{et~al.}(2021)Di~Valentino, Mena, Pan, Visinelli,
  Yang, Melchiorri, Mota, Riess, and Silk]{DiValentino:2021izs}
Di~Valentino, E.; Mena, O.; Pan, S.; Visinelli, L.; Yang, W.; Melchiorri, A.;
  Mota, D.F.; Riess, A.G.; Silk, J.
\newblock {In the Realm of the Hubble tension---A Review of {Solutions}.} \emph{arXiv} {\bf
  2021}, arXiv:2103.01183
\newblock  %\href{http://xxx.lanl.gov/abs/2103.01183}{{\normalfont
  %[arXiv:astro-ph.CO/2103.01183]}}.

\bibitem[Yang \em{et~al.}(2021)Yang, Di~Valentino, Pan, Wu, and
  Lu]{Yang:2021flj}
Yang, W.; Di~Valentino, E.; Pan, S.; Wu, Y.; Lu, J.
\newblock {Dynamical dark energy after Planck CMB final release and $H_0$
  tension}.
\newblock {\em Mon. Not. Roy. Astron. Soc.} {\bf 2021}, {\em 501},~5845--5858.
  %\href{http://xxx.lanl.gov/abs/2101.02168}{{\normalfont
  %[arXiv:astro-ph.CO/2101.02168]}}.
\newblock
{\changeurlcolor{black}\href{https://doi.org/10.1093/mnras/staa3914}{\detokenize{https://doi.org/10.1093/mnras/staa3914}}}.

\bibitem[Sch\"oneberg \em{et~al.}(2019)Sch\"oneberg, Lesgourgues, and
  Hooper]{Schoneberg:2019wmt}
Sch\"oneberg, N.; Lesgourgues, J.; Hooper, D.C.
\newblock {The BAO+BBN take on the Hubble tension}.
\newblock {\em JCAP} {\bf 2019}, {\em 10},~029.
  %\href{http://xxx.lanl.gov/abs/1907.11594}{{\normalfont
  %[arXiv:astro-ph.CO/1907.11594]}}.
\newblock
{\changeurlcolor{black}\href{https://doi.org/10.1088/1475-7516/2019/10/029}{\detokenize{https://doi.org/10.1088/1475-7516/2019/10/029}}}.

\bibitem[Di~Valentino(2017)]{DiValentino:2017gzb}
Di~Valentino, E.
\newblock {Crack in the cosmological paradigm}.
\newblock {\em Nat. Astron.} {\bf 2017}, {\em 1},~569--570.
  %\href{http://xxx.lanl.gov/abs/1709.04046}{{\normalfont
  %[arXiv:physics.pop-ph/1709.04046]}}.
\newblock
{\changeurlcolor{black}\href{https://doi.org/10.1038/s41550-017-0236-8}{\detokenize{https://doi.org/10.1038/s41550-017-0236-8}}}.

\bibitem[Di~Valentino \em{et~al.}(2020)Di~Valentino
  et~al.]{DiValentino:2020zio}
Di~Valentino, E.;  {et~al.}
\newblock {Cosmology Intertwined II: The Hubble Constant {Tension}.} \emph{arXiv} {\bf 2020}, arXiv:2008.11284.
\newblock  %\href{http://xxx.lanl.gov/abs/2008.11284}{{\normalfont
  %[arXiv:astro-ph.CO/2008.11284]}}.

\bibitem[Perivolaropoulos and Skara(2021)]{Perivolaropoulos:2021jda}
Perivolaropoulos, L.; Skara, F.
\newblock {Challenges for $\Lambda$CDM: An {update}.} \emph{arXiv} {\bf 2021}, arXiv:2105.05208.
\newblock  %\href{http://xxx.lanl.gov/abs/2105.05208}{{\normalfont
  %[arXiv:astro-ph.CO/2105.05208]}}.

\bibitem[Lucca(2021)]{Lucca:2021dxo}
Lucca, M.
\newblock {Dark energy-dark matter interactions as a solution to the $S_8$
  {tension}.} \emph{arXiv} {\bf 2021}, arXiv:2105.09249.
\newblock  %\href{http://xxx.lanl.gov/abs/2105.09249}{{\normalfont
  %[arXiv:astro-ph.CO/2105.09249]}}.

\bibitem[Colg\'ain \em{et~al.}(2022{\natexlab{a}})Colg\'ain, Sheikh-Jabbari,
  Solomon, Bargiacchi, Capozziello, Dainotti, and Stojkovic]{Colgain:2022nlb}
Colg\'ain, E.O.; Sheikh-Jabbari, M.M.; Solomon, R.; Bargiacchi, G.;
  Capozziello, S.; Dainotti, M.G.; Stojkovic, D.
\newblock {Revealing intrinsic flat \ensuremath{\Lambda}CDM biases with
  standardizable candles}.
\newblock {\em Phys. Rev. D} {\bf 2022}, {\em 106},~L041301.
  %\href{http://xxx.lanl.gov/abs/2203.10558}{{\normalfont
  %[arXiv:astro-ph.CO/2203.10558]}}.
\newblock
{\changeurlcolor{black}\href{https://doi.org/10.1103/PhysRevD.106.L041301}{\detokenize{https://doi.org/10.1103/PhysRevD.106.L041301}}}.

\bibitem[Colg\'ain \em{et~al.}(2022{\natexlab{b}})Colg\'ain, Sheikh-Jabbari,
  Solomon, Dainotti, and Stojkovic]{Colgain:2022rxy}
Colg\'ain, E.O.; Sheikh-Jabbari, M.M.; Solomon, R.; Dainotti, M.G.; Stojkovic,
  D.
\newblock {Putting Flat $\Lambda$CDM In The (Redshift) Bin} \emph{arXiv} {\bf 2022}, arXiv:2206.11447.
\newblock  %\href{http://xxx.lanl.gov/abs/2206.11447}{{\normalfont
  %[arXiv:astro-ph.CO/2206.11447]}}.

\bibitem[Wang \em{et~al.}(2018)Wang, Pogosian, Zhao, and Zucca]{Wang:2018fng}
Wang, Y.; Pogosian, L.; Zhao, G.B.; Zucca, A.
\newblock {Evolution of dark energy reconstructed from the latest
  observations}.
\newblock {\em Astrophys. J. Lett.} {\bf 2018}, {\em 869},~L8.
  %\href{http://xxx.lanl.gov/abs/1807.03772}{{\normalfont
  %[arXiv:astro-ph.CO/1807.03772]}}.
\newblock
{\changeurlcolor{black}\href{https://doi.org/10.3847/2041-8213/aaf238}{\detokenize{https://doi.org/10.3847/2041-8213/aaf238}}}.

\bibitem[Reyes and Escamilla-Rivera(2021)]{Reyes:2021owe}
Reyes, M.; Escamilla-Rivera, C.
\newblock {Improving data-driven model-independent reconstructions and new
  constraints in Horndeski {cosmology}.} \emph{arXiv} {\bf 2021}, arXiv:2104.04484.
\newblock  %\href{http://xxx.lanl.gov/abs/2104.04484}{{\normalfont
  %[arXiv:astro-ph.CO/2104.04484]}}.

\bibitem[{Colg\'ain, Eoin\'o and Sheikh-Jabbari, M. M. and Yin,
  Lu}(2021)]{Colgain:2021pmf}
{Colg\'ain, Eoin\'o and Sheikh-Jabbari, M. M. and Yin, Lu}.
\newblock {Can dark energy be {dynamical?}} \emph{arXiv} {\bf 2021}, arXiv:2104.01930.
\newblock  %\href{http://xxx.lanl.gov/abs/2104.01930}{{\normalfont
  %[arXiv:astro-ph.CO/2104.01930]}}.

\bibitem[Liu \em{et~al.}(2021)Liu, Anchordoqui, Di~Valentino, Pan, Wu, and
  Yang]{2108.04188}
Liu, W.; Anchordoqui, L.A.; Di~Valentino, E.; Pan, S.; Wu, Y.; Yang, W.
\newblock {Constraints from High-Precision Measurements of the Cosmic Microwave
  Background: The Case of Disintegrating Dark Matter with ${\Lambda}$ or
  Dynamical Dark {Energy}.}  \emph{arXiv} {\bf 2021}, arXiv:2108.04188.
\newblock  %\href{http://xxx.lanl.gov/abs/2108.04188}{{\normalfont
  %[arXiv:astro-ph.CO/2108.04188]}}.

\bibitem[Pettorino \em{et~al.}(2013)Pettorino, Amendola, and
  Wetterich]{Pettorino:2013ia}
Pettorino, V.; Amendola, L.; Wetterich, C.
\newblock {How early is early dark energy?}
\newblock {\em Phys. Rev. D} {\bf 2013}, {\em 87},~083009.
  %\href{http://xxx.lanl.gov/abs/1301.5279}{{\normalfont
  %[arXiv:astro-ph.CO/1301.5279]}}.
\newblock
{\changeurlcolor{black}\href{https://doi.org/10.1103/PhysRevD.87.083009}{\detokenize{https://doi.org/10.1103/ PhysRevD.87.083009}}}.

\bibitem[Poulin \em{et~al.}(2019)Poulin, Smith, Karwal, and
  Kamionkowski]{Poulin:2018cxd}
Poulin, V.; Smith, T.L.; Karwal, T.; Kamionkowski, M.
\newblock {Early Dark Energy Can Resolve The Hubble Tension}.
\newblock {\em Phys. Rev. Lett.} {\bf 2019}, {\em 122},~221301.
  %\href{http://xxx.lanl.gov/abs/1811.04083}{{\normalfont
  %[arXiv:astro-ph.CO/1811.04083]}}.
\newblock
{\changeurlcolor{black}\href{https://doi.org/10.1103/PhysRevLett.122.221301}{\detokenize{https://doi.org/10.1103/PhysRevLett.122.221301}}}.

\bibitem[Lin \em{et~al.}(2020)Lin, Hu, and Raveri]{Lin:2020jcb}
Lin, M.X.; Hu, W.; Raveri, M.
\newblock {Testing $H_0$ in Acoustic Dark Energy with Planck and ACT
  Polarization}.
\newblock {\em Phys. Rev. D} {\bf 2020}, {\em 102},~123523.
  %\href{http://xxx.lanl.gov/abs/2009.08974}{{\normalfont
  %[arXiv:astro-ph.CO/2009.08974]}}.
\newblock
{\changeurlcolor{black}\href{https://doi.org/10.1103/PhysRevD.102.123523}{\detokenize{https://doi.org/10.1103/PhysRevD.102.123523}}}.

\bibitem[Smith \em{et~al.}(2022)Smith, Lucca, Poulin, Abellan, Balkenhol,
  Benabed, Galli, and Murgia]{Smith:2022hwi}
Smith, T.L.; Lucca, M.; Poulin, V.; Abellan, G.F.; Balkenhol, L.; Benabed, K.;
  Galli, S.; Murgia, R.
\newblock {Hints of early dark energy in Planck, SPT, and ACT data: New physics
  or systematics?}
\newblock {\em Phys. Rev. D} {\bf 2022}, {\em 106},~043526.
  %\href{http://xxx.lanl.gov/abs/2202.09379}{{\normalfont
  %[arXiv:astro-ph.CO/2202.09379]}}.
\newblock
{\changeurlcolor{black}\href{https://doi.org/10.1103/PhysRevD.106.043526}{\detokenize{https://doi.org/10.1103/PhysRevD.106.043526}}}.

\bibitem[Smith \em{et~al.}(2021)Smith, Poulin, Bernal, Boddy, Kamionkowski, and
  Murgia]{Smith:2020rxx}
Smith, T.L.; Poulin, V.; Bernal, J.L.; Boddy, K.K.; Kamionkowski, M.; Murgia,
  R.
\newblock {Early dark energy is not excluded by current large-scale structure
  data}.
\newblock {\em Phys. Rev. D} {\bf 2021}, {\em 103},~123542.
  %\href{http://xxx.lanl.gov/abs/2009.10740}{{\normalfont
  %[arXiv:astro-ph.CO/2009.10740]}}.
\newblock
{\changeurlcolor{black}\href{https://doi.org/10.1103/PhysRevD.103.123542}{\detokenize{https://doi.org/10.1103/PhysRevD.103.123542}}}.

\bibitem[Di~Valentino(2021)]{DiValentino:2020vnx}
Di~Valentino, E.
\newblock {A combined analysis of the $H_0$ late time direct measurements and
  the impact on the Dark Energy sector}.
\newblock {\em Mon. Not. Roy. Astron. Soc.} {\bf 2021}, {\em 502},~2065--2073.
  %\href{http://xxx.lanl.gov/abs/2011.00246}{{\normalfont
  %[arXiv:astro-ph.CO/2011.00246]}}.
\newblock
{\changeurlcolor{black}\href{https://doi.org/10.1093/mnras/stab187}{\detokenize{https://doi.org/10.1093/mnras/stab187}}}.

\bibitem[Haridasu \em{et~al.}(2021)Haridasu, Viel, and
  Vittorio]{Haridasu:2020pms}
Haridasu, B.S.; Viel, M.; Vittorio, N.
\newblock {Sources of $H_0$-tension in dark energy scenarios}.
\newblock {\em Phys. Rev. D} {\bf 2021}, {\em 103},~063539.
  %\href{http://xxx.lanl.gov/abs/2012.10324}{{\normalfont
  %[arXiv:astro-ph.CO/2012.10324]}}.
\newblock
{\changeurlcolor{black}\href{https://doi.org/10.1103/PhysRevD.103.063539}{\detokenize{https://doi.org/10.1103/PhysRevD.103.063539}}}.

\bibitem[Li and Shafieloo(2019)]{Li:2019yem}
Li, X.; Shafieloo, A.
\newblock {A Simple Phenomenological Emergent Dark Energy Model can Resolve the
  Hubble Tension}.
\newblock {\em Astrophys. J. Lett.} {\bf 2019}, {\em 883},~L3.
  %\href{http://xxx.lanl.gov/abs/1906.08275}{{\normalfont
  %[arXiv:astro-ph.CO/1906.08275]}}.
\newblock
{\changeurlcolor{black}\href{https://doi.org/10.3847/2041-8213/ab3e09}{\detokenize{https://doi.org/10.3847/2041-8213/ab3e09}}}.

\bibitem[Yang \em{et~al.}(2021)Yang, Di~Valentino, Pan, and Mena]{Yang:2020ope}
Yang, W.; Di~Valentino, E.; Pan, S.; Mena, O.
\newblock {Emergent Dark Energy, neutrinos and cosmological tensions}.
\newblock {\em Phys. Dark Univ.} {\bf 2021}, {\em 31},~100762.
  %\href{http://xxx.lanl.gov/abs/2007.02927}{{\normalfont
  %[arXiv:astro-ph.CO/2007.02927]}}.
\newblock
{\changeurlcolor{black}\href{https://doi.org/10.1016/j.dark.2020.100762}{\detokenize{https://doi.org/10.1016/j.dark.2020.100762}}}.

\bibitem[Kumar and Nunes(2017)]{Kumar:2017dnp}
Kumar, S.; Nunes, R.C.
\newblock {Echo of interactions in the dark sector}.
\newblock {\em Phys. Rev. D} {\bf 2017}, {\em 96},~103511.
  %\href{http://xxx.lanl.gov/abs/1702.02143}{{\normalfont
  %[arXiv:astro-ph.CO/1702.02143]}}.
\newblock
{\changeurlcolor{black}\href{https://doi.org/10.1103/PhysRevD.96.103511}{\detokenize{https://doi.org/10.1103/PhysRevD. 96.103511}}}.

\bibitem[Di~Valentino \em{et~al.}(2020)Di~Valentino, Melchiorri, Mena, and
  Vagnozzi]{DiValentino:2019ffd}
Di~Valentino, E.; Melchiorri, A.; Mena, O.; Vagnozzi, S.
\newblock {Interacting dark energy in the early 2020s: A promising solution to
  the $H_0$ and cosmic shear tensions}.
\newblock {\em Phys. Dark Univ.} {\bf 2020}, {\em 30},~100666.
  %\href{http://xxx.lanl.gov/abs/1908.04281}{{\normalfont
  %[arXiv:astro-ph.CO/1908.04281]}}.
\newblock
{\changeurlcolor{black}\href{https://doi.org/10.1016/j.dark.2020.100666}{\detokenize{https://doi.org/10.1016/j.dark.2020.100666}}}.

\bibitem[Yang \em{et~al.}(2019)Yang, Mena, Pan, and Di~Valentino]{Yang:2019uzo}
Yang, W.; Mena, O.; Pan, S.; Di~Valentino, E.
\newblock {Dark sectors with dynamical coupling}.
\newblock {\em Phys. Rev. D} {\bf 2019}, {\em 100},~083509.
  %\href{http://xxx.lanl.gov/abs/1906.11697}{{\normalfont
  %[arXiv:astro-ph.CO/1906.11697]}}.
\newblock
{\changeurlcolor{black}\href{https://doi.org/10.1103/PhysRevD.100.083509}{\detokenize{https://doi.org/10.1103/PhysRevD.100.083509}}}.

\bibitem[Gogoi \em{et~al.}(2021)Gogoi, Sharma, Chanda, and Das]{Gogoi:2020qif}
Gogoi, A.; Sharma, R.K.; Chanda, P.; Das, S.
\newblock {Early Mass-varying Neutrino Dark Energy: Nugget Formation and Hubble
  Anomaly}.
\newblock {\em Astrophys. J.} {\bf 2021}, {\em 915},~132.
  %\href{http://xxx.lanl.gov/abs/2005.11889}{{\normalfont
  %[arXiv:astro-ph.CO/2005.11889]}}.
\newblock
{\changeurlcolor{black}\href{https://doi.org/10.3847/1538-4357/abfe5b}{\detokenize{https://doi.org/10.3847/1538-4357/abfe5b}}}.

\bibitem[Sakstein and Trodden(2020)]{Sakstein:2019fmf}
Sakstein, J.; Trodden, M.
\newblock {Early Dark Energy from Massive Neutrinos as a Natural Resolution of
  the Hubble Tension}.
\newblock {\em Phys. Rev. Lett.} {\bf 2020}, {\em 124},~161301.
  %\href{http://xxx.lanl.gov/abs/1911.11760}{{\normalfont
  %[arXiv:astro-ph.CO/1911.11760]}}.
\newblock
{\changeurlcolor{black}\href{https://doi.org/10.1103/PhysRevLett.124.161301}{\detokenize{https://doi.org/10.1103/PhysRevLett.124.161301}}}.

\bibitem[Tian and Zhu(2021)]{Tian:2021omz}
Tian, S.X.; Zhu, Z.H.
\newblock {Early dark energy in $k$-essence}.
\newblock {\em Phys. Rev.} {\bf 2021}, {\em D103},~043518.
  %\href{http://xxx.lanl.gov/abs/2102.06399}{{\normalfont
  %[arXiv:gr-qc/2102.06399]}}.
\newblock
{\changeurlcolor{black}\href{https://doi.org/10.1103/PhysRevD.103.043518}{\detokenize{https://doi.org/10.1103/PhysRevD.103.043518}}}.

\bibitem[Nojiri \em{et~al.}(2021)Nojiri, Odintsov, Saez-Chillon~Gomez, and
  Sharov]{Nojiri:2021dze}
Nojiri, S.; Odintsov, S.D.; Saez-Chillon~Gomez, D.; Sharov, G.S.
\newblock {Modelling and testing the equation of state for (Early) dark {energy}.} \emph{arXiv}
  {\bf 2021}, arXiv:2103.05304.
\newblock  %\href{http://xxx.lanl.gov/abs/2103.05304}{{\normalfont
  %[arXiv:gr-qc/2103.05304]}}.

\bibitem[Seto and Toda(2021)]{Seto:2021xua}
Seto, O.; Toda, Y.
\newblock {Comparing early dark energy and extra radiation solutions to the
  Hubble tension with BBN}.
\newblock {\em Phys. Rev. D} {\bf 2021}, {\em 103},~123501.
  %\href{http://xxx.lanl.gov/abs/2101.03740}{{\normalfont
  %[arXiv:astro-ph.CO/2101.03740]}}.
\newblock
{\changeurlcolor{black}\href{https://doi.org/10.1103/PhysRevD.103.123501}{\detokenize{https://doi.org/10.1103/PhysRevD.103.123501}}}.

\bibitem[Escamilla-Rivera and N\'ajera(2022)]{Escamilla-Rivera:2021boq}
Escamilla-Rivera, C.; N\'ajera, A.
\newblock {Dynamical dark energy models in the light of gravitational-wave
  transient catalogues}.
\newblock {\em JCAP} {\bf 2022}, {\em 03},~060.
  %\href{http://xxx.lanl.gov/abs/2103.02097}{{\normalfont
  %[arXiv:gr-qc/2103.02097]}}.
\newblock
{\changeurlcolor{black}\href{https://doi.org/10.1088/1475-7516/2022/03/060}{\detokenize{https://doi.org/10.1088/1475-7516/2022/03/060}}}.

\bibitem[Motta \em{et~al.}(2021)Motta, Garc\'\i{}a-Aspeitia,
  Hern\'andez-Almada, Maga\~na, and Verdugo]{Motta:2021hvl}
Motta, V.; Garc\'\i{}a-Aspeitia, M.A.; Hern\'andez-Almada, A.; Maga\~na, J.;
  Verdugo, T.
\newblock {Taxonomy of Dark Energy Models}.
\newblock {\em Universe} {\bf 2021}, {\em 7},~163.
  %\href{http://xxx.lanl.gov/abs/2104.04642}{{\normalfont
  %[arXiv:astro-ph.CO/2104.04642]}}.
\newblock
{\changeurlcolor{black}\href{https://doi.org/10.3390/universe7060163}{\detokenize{https://doi.org/10.3390/universe7060163}}}.

\bibitem[Yang \em{et~al.}(2021)Yang, Di~Valentino, Pan, Shafieloo, and
  Li]{Yang:2021eud}
Yang, W.; Di~Valentino, E.; Pan, S.; Shafieloo, A.; Li, X.
\newblock {Generalized emergent dark energy model and the Hubble constant
  tension}.
\newblock {\em Phys. Rev. D} {\bf 2021}, {\em 104},~063521.
  %\href{http://xxx.lanl.gov/abs/2103.03815}{{\normalfont
  %[arXiv:astro-ph.CO/2103.03815]}}.
\newblock
{\changeurlcolor{black}\href{https://doi.org/10.1103/PhysRevD.104.063521}{\detokenize{https://doi.org/10.1103/PhysRevD.104.063521}}}.

\bibitem[Staicova and Benisty(2021)]{Staicova:2021ntm}
Staicova, D.; Benisty, D.
\newblock {Constraining the dark energy models using Baryon Acoustic
  Oscillations: An approach independent of $H_0 \cdot r_d$}.  \emph{arXiv} {\bf 2021}, arXiv:2107.14129.
\newblock  %\href{http://xxx.lanl.gov/abs/2107.14129}{{\normalfont
  %[arXiv:astro-ph.CO/2107.14129]}}.
%\newblock
%{\changeurlcolor{black}\href{https://doi.org/10.1051/0004-6361/202244366}{\detokenize{https://doi.org/10.1051/0004-6361/202244366}}}.

\bibitem[Aubourg \em{et~al.}(2015)Aubourg et~al.]{Aubourg:2014yra}
Aubourg, E.; {Bailey, S.; Bautista, J.E.; Beutler, F.; Bhardwaj, V.; Bizyaev, D.; Blanton, M.; Blomqvist, M.; Bolton, A.S.; Bovy, J.;} et al.
\newblock {Cosmological implications of baryon acoustic oscillation
  measurements}.
\newblock {\em Phys. Rev. D} {\bf 2015}, {\em 92},~123516.
  %\href{http://xxx.lanl.gov/abs/1411.1074}{{\normalfont
  %[arXiv:astro-ph.CO/1411.1074]}}.
\newblock
{\changeurlcolor{black}\href{https://doi.org/10.1103/PhysRevD.92.123516}{\detokenize{https://doi.org/10.1103/PhysRevD.92.123516}}}.

\bibitem[Arendse \em{et~al.}(2019)Arendse, Agnello, and
  Wojtak]{Arendse:2019itb}
Arendse, N.; Agnello, A.; Wojtak, R.
\newblock {Low-redshift measurement of the sound horizon through gravitational
  time-delays}.
\newblock {\em Astron. Astrophys.} {\bf 2019}, {\em 632},~A91.
  %\href{http://xxx.lanl.gov/abs/1905.12000}{{\normalfont
  %[arXiv:astro-ph.CO/1905.12000]}}.
\newblock
{\changeurlcolor{black}\href{https://doi.org/10.1051/0004-6361/201935972}{\detokenize{https://doi.org/10.1051/0004-6361/201935972}}}.

\bibitem[Aylor \em{et~al.}(2019)Aylor, Joy, Knox, Millea, Raghunathan, and
  Wu]{Aylor:2018drw}
Aylor, K.; Joy, M.; Knox, L.; Millea, M.; Raghunathan, S.; Wu, W.L.K.
\newblock {Sounds Discordant: Classical Distance Ladder \textbackslash{}\&
  $\Lambda$CDM -based Determinations of the Cosmological Sound Horizon}.
\newblock {\em Astrophys. J.} {\bf 2019}, {\em 874},~4.
  %\href{http://xxx.lanl.gov/abs/1811.00537}{{\normalfont
  %[arXiv:astro-ph.CO/1811.00537]}}.
\newblock
{\changeurlcolor{black}\href{https://doi.org/10.3847/1538-4357/ab0898}{\detokenize{https://doi.org/10.3847/1538-4357/ab0898}}}.

\bibitem[Pogosian \em{et~al.}(2020)Pogosian, Zhao, and
  Jedamzik]{Pogosian:2020ded}
Pogosian, L.; Zhao, G.B.; Jedamzik, K.
\newblock {Recombination-independent determination of the sound horizon and the
  Hubble constant from BAO}.
\newblock {\em Astrophys. J. Lett.} {\bf 2020}, {\em 904},~L17.
  %\href{http://xxx.lanl.gov/abs/2009.08455}{{\normalfont
  %[arXiv:astro-ph.CO/2009.08455]}}.
\newblock
{\changeurlcolor{black}\href{https://doi.org/10.3847/2041-8213/abc6a8}{\detokenize{https://doi.org/10.3847/2041-8213/abc6a8}}}.

\bibitem[Aizpuru \em{et~al.}(2021)Aizpuru, Arjona, and
  Nesseris]{Aizpuru:2021vhd}
Aizpuru, A.; Arjona, R.; Nesseris, S.
\newblock {Machine learning improved fits of the sound horizon at the baryon
  drag epoch}.
\newblock {\em Phys. Rev. D} {\bf 2021}, {\em 104},~043521.
  %\href{http://xxx.lanl.gov/abs/2106.00428}{{\normalfont
  %[arXiv:astro-ph.CO/2106.00428]}}.
\newblock
{\changeurlcolor{black}\href{https://doi.org/10.1103/PhysRevD.104.043521}{\detokenize{https://doi.org/10.1103/PhysRevD.104.043521}}}.

\bibitem[Jedamzik \em{et~al.}(2021)Jedamzik, Pogosian, and
  Zhao]{Jedamzik:2020zmd}
Jedamzik, K.; Pogosian, L.; Zhao, G.B.
\newblock {Why reducing the cosmic sound horizon alone can not fully resolve
  the Hubble tension}.
\newblock {\em Commun. Phys.} {\bf 2021}, {\em 4},~123.
  %\href{http://xxx.lanl.gov/abs/2010.04158}{{\normalfont
  %[arXiv:astro-ph.CO/2010.04158]}}.
\newblock
{\changeurlcolor{black}\href{https://doi.org/10.1038/s42005-021-00628-x}{\detokenize{https://doi.org/10.1038/s42005-021-00628-x}}}.

\bibitem[de~la Macorra \em{et~al.}(2021)de~la Macorra, Almaraz, and
  Garrido]{delaMacorra:2021hoh}
de~la Macorra, A.; Almaraz, E.; Garrido, J.
\newblock {Towards a Solution to the H0 Tension: The Price to {Pay}.} \emph{arXiv} {\bf 2021}, arXiv:2106.12116.
\newblock  %\href{http://xxx.lanl.gov/abs/2106.12116}{{\normalfont
  %[arXiv:astro-ph.CO/2106.12116]}}.

\bibitem[Wang and Wang(2013)]{Wang:2013mha}
Wang, Y.; Wang, S.
\newblock {Distance Priors from Planck and Dark Energy Constraints from Current
  Data}.
\newblock {\em Phys. Rev. D} {\bf 2013}, {\em 88},~043522.
  %\href{http://xxx.lanl.gov/abs/1304.4514}{{\normalfont
  %[arXiv:astro-ph.CO/1304.4514]}}.
\newblock [Erratum: Phys.Rev.D 88, 069903 (2013)],
{\changeurlcolor{black}\href{https://doi.org/10.1103/PhysRevD.88.043522}{\detokenize{https://doi.org/10.1103/PhysRevD.88.043522}}}.

\bibitem[Mamon \em{et~al.}(2017)Mamon, Bamba, and Das]{Mamon:2016wow}
Mamon, A.A.; Bamba, K.; Das, S.
\newblock {Constraints on reconstructed dark energy model from SN Ia and
  BAO/CMB observations}.
\newblock {\em Eur. Phys. J. C} {\bf 2017}, {\em 77},~29.
  %\href{http://xxx.lanl.gov/abs/1607.06631}{{\normalfont
  %[arXiv:gr-qc/1607.06631]}}.
\newblock
{\changeurlcolor{black}\href{https://doi.org/10.1140/epjc/s10052-016-4590-y}{\detokenize{https://doi.org/10.1140/epjc/s10052-016-4590-y}}}.

\bibitem[Grandon and Cardenas(2018)]{Grandon:2018uoe}
Grandon, D.; Cardenas, V.H.
\newblock {Exploring evidence of interaction between dark energy and dark
  {matter}.} \emph{arXiv} {\bf 2018}, arXiv:1804.03296.
\newblock  %\href{http://xxx.lanl.gov/abs/1804.03296}{{\normalfont
  %[arXiv:astro-ph.CO/1804.03296]}}.
\newblock
{\changeurlcolor{black}\href{https://doi.org/10.1007/s10714-019-2526-1}{\detokenize{https://doi.org/ 10.1007/s10714-019-2526-1}}}.

\bibitem[Chen \em{et~al.}(2019)Chen, Huang, and Wang]{Chen:2018dbv}
Chen, L.; Huang, Q.G.; Wang, K.
\newblock {Distance Priors from Planck Final Release}.
\newblock {\em JCAP} {\bf 2019}, {\em 02},~028.
  %\href{http://xxx.lanl.gov/abs/1808.05724}{{\normalfont
  %[arXiv:astro-ph.CO/1808.05724]}}.
\newblock
{\changeurlcolor{black}\href{https://doi.org/10.1088/1475-7516/2019/02/028}{\detokenize{https://doi.org/10.1088/1475-7516/2019/02/028}}}.

\bibitem[da~Silva and Silva(2019)]{daSilva:2018ehn}
da~Silva, W.J.C.; Silva, R.
\newblock {Extended $\Lambda$CDM model and viscous dark energy: A Bayesian
  analysis}.
\newblock {\em JCAP} {\bf 2019}, {\em 05},~036.
  %\href{http://xxx.lanl.gov/abs/1810.03759}{{\normalfont
  %[arXiv:astro-ph.CO/1810.03759]}}.
\newblock
{\changeurlcolor{black}\href{https://doi.org/10.1088/1475-7516/2019/05/036}{\detokenize{https://doi.org/10.1088/1475-7516/2019/05/036}}}.

\bibitem[Zhai and Wang(2019)]{Zhai:2018vmm}
Zhai, Z.; Wang, Y.
\newblock {Robust and model-independent cosmological constraints from distance
  measurements}.
\newblock {\em JCAP} {\bf 2019}, {\em 07},~005.
  %\href{http://xxx.lanl.gov/abs/1811.07425}{{\normalfont
  %[arXiv:astro-ph.CO/1811.07425]}}.
\newblock
{\changeurlcolor{black}\href{https://doi.org/10.1088/1475-7516/2019/07/005}{\detokenize{https://doi.org/10.1088/1475-7516/2019/07/005}}}.

\bibitem[Di~Valentino \em{et~al.}(2021)Di~Valentino, Melchiorri, and
  Silk]{DiValentino:2020hov}
Di~Valentino, E.; Melchiorri, A.; Silk, J.
\newblock {Investigating Cosmic Discordance}.
\newblock {\em Astrophys. J. Lett.} {\bf 2021}, {\em 908},~L9.
  %\href{http://xxx.lanl.gov/abs/2003.04935}{{\normalfont
  %[arXiv:astro-ph.CO/2003.04935]}}.
\newblock
{\changeurlcolor{black}\href{https://doi.org/10.3847/2041-8213/abe1c4}{\detokenize{https://doi.org/10.3847/ 2041-8213/abe1c4}}}.

\bibitem[Nilsson and Park(2022)]{Nilsson:2021ute}
Nilsson, N.A.; Park, M.I.
\newblock {Tests of standard cosmology in Ho\v{r}ava gravity, Bayesian evidence
  for a closed universe, and the Hubble tension}.
\newblock {\em Eur. Phys. J. C} {\bf 2022}, {\em 82},~873.
  %\href{http://xxx.lanl.gov/abs/2108.07986}{{\normalfont
  %[arXiv:hep-th/2108.07986]}}.
\newblock
{\changeurlcolor{black}\href{https://doi.org/10.1140/epjc/s10052-022-10839-3}{\detokenize{https://doi.org/10.1140/epjc/s10052-022-10839-3}}}.

\bibitem[Yao and Meng(2022)]{Yao:2022kub}
Yao, Y.H.; Meng, X.H.
\newblock {Can interacting dark energy with dynamical coupling resolve the
  Hubble {tension}.} \emph{arXiv} {\bf 2022}, arXiv:2207.05955.
\newblock  %\href{http://xxx.lanl.gov/abs/2207.05955}{{\normalfont
  %[arXiv:astro-ph.CO/2207.05955]}}.

\bibitem[L'Huillier and Shafieloo(2017)]{LHuillier:2016mtc}
L'Huillier, B.; Shafieloo, A.
\newblock {Model-independent test of the FLRW metric, the flatness of the
  Universe, and non-local measurement of $H_0r_\mathrm{d}$}.
\newblock {\em JCAP} {\bf 2017}, {\em 01},~015.
  %\href{http://xxx.lanl.gov/abs/1606.06832}{{\normalfont
  %[arXiv:astro-ph.CO/1606.06832]}}.
\newblock
{\changeurlcolor{black}\href{https://doi.org/10.1088/1475-7516/2017/01/015}{\detokenize{https://doi.org/10.1088/1475-7516/2017/01/015}}}.

\bibitem[Shafieloo \em{et~al.}(2018)Shafieloo, L'Huillier, and
  Starobinsky]{Shafieloo:2018gin}
Shafieloo, A.; L'Huillier, B.; Starobinsky, A.A.
\newblock {Falsifying $\Lambda$CDM: Model-independent tests of the concordance
  model with eBOSS DR14Q and Pantheon}.
\newblock {\em Phys. Rev. D} {\bf 2018}, {\em 98},~083526.
  %\href{http://xxx.lanl.gov/abs/1804.04320}{{\normalfont
  %[arXiv:astro-ph.CO/1804.04320]}}.
\newblock
{\changeurlcolor{black}\href{https://doi.org/10.1103/PhysRevD.98.083526}{\detokenize{https://doi.org/10.1103/PhysRevD.98.083526}}}.

\bibitem[Arendse \em{et~al.}(2020)Arendse et~al.]{Arendse:2019hev}
Arendse, N.; {Wojtak, R.; Agnello, A.; Chen, G.C.-F.; Fassnacht, C.D.; Sluse, D.; Hilbert, S.; Millon, M.; Bonvin, V.; Wong, K.C.;} et al.
\newblock {Cosmic dissonance: Are new physics or systematics behind a short
  sound horizon?}
\newblock {\em Astron. Astrophys.} {\bf 2020}, {\em 639},~A57.
  %\href{http://xxx.lanl.gov/abs/1909.07986}{{\normalfont
  %[arXiv:astro-ph.CO/1909.07986]}}.
\newblock
{\changeurlcolor{black}\href{https://doi.org/10.1051/0004-6361/201936720}{\detokenize{https://doi.org/10.1051/0004-6361/201936720}}}.

\bibitem[Knox and Millea(2020)]{Knox:2019rjx}
Knox, L.; Millea, M.
\newblock {Hubble constant hunter\textquoteright{}s guide}.
\newblock {\em Phys. Rev. D} {\bf 2020}, {\em 101},~043533.
  %\href{http://xxx.lanl.gov/abs/1908.03663}{{\normalfont
  %[arXiv:astro-ph.CO/1908.03663]}}.
\newblock
{\changeurlcolor{black}\href{https://doi.org/10.1103/PhysRevD.101.043533}{\detokenize{https://doi.org/10.1103/PhysRevD.101. 043533}}}.

\bibitem[Chevallier and Polarski(2001)]{Chevallier:2000qy}
Chevallier, M.; Polarski, D.
\newblock {Accelerating universes with scaling dark matter}.
\newblock {\em Int. J. Mod. Phys. D} {\bf 2001}, {\em 10},~213--224.
  %\href{http://xxx.lanl.gov/abs/gr-qc/0009008}{{\normalfont [gr-qc/0009008]}}.
\newblock
{\changeurlcolor{black}\href{https://doi.org/10.1142/S0218271801000822}{\detokenize{https://doi.org/10.1142/S0218271801000822}}}.

\bibitem[Linder and Huterer(2005)]{Linder:2005ne}
Linder, E.V.; Huterer, D.
\newblock {How many dark energy parameters?}
\newblock {\em Phys. Rev. D} {\bf 2005}, {\em 72},~043509.
  %\href{http://xxx.lanl.gov/abs/astro-ph/0505330}{{\normalfont
  %[astro-ph/0505330]}}.
\newblock
{\changeurlcolor{black}\href{https://doi.org/10.1103/PhysRevD.72.043509}{\detokenize{https://doi.org/10.1103/PhysRevD.72. 043509}}}.

\bibitem[Barger \em{et~al.}(2006)Barger, Guarnaccia, and
  Marfatia]{Barger:2005sb}
Barger, V.; Guarnaccia, E.; Marfatia, D.
\newblock {Classification of dark energy models in the (w(0), w(a)) plane}.
\newblock {\em Phys. Lett. B} {\bf 2006}, {\em 635},~61--65.
  %\href{http://xxx.lanl.gov/abs/hep-ph/0512320}{{\normalfont
  %[hep-ph/0512320]}}.
\newblock
{\changeurlcolor{black}\href{https://doi.org/10.1016/j.physletb.2006.02.018}{\detokenize{https://doi.org/10.1016/j.physletb.2006.02.018}}}.

\bibitem[Barboza and Alcaniz(2008)]{Barboza:2008rh}
Barboza, Jr., E.M.; Alcaniz, J.S.
\newblock {A parametric model for dark energy}.
\newblock {\em Phys. Lett. B} {\bf 2008}, {\em 666},~415--419.
  %\href{http://xxx.lanl.gov/abs/0805.1713}{{\normalfont
  %[arXiv:astro-ph/0805.1713]}}.
\linebreak \newblock
{\changeurlcolor{black}\href{https://doi.org/10.1016/j.physletb.2008.08.012}{\detokenize{https://doi.org/10.1016/j.physletb.2008.08.012}}}.

\bibitem[Wang(2008)]{Wang:2008zh}
Wang, Y.
\newblock {Figure of Merit for Dark Energy Constraints from Current
  Observational Data}.
\newblock {\em Phys. Rev. D} {\bf 2008}, {\em 77},~123525.
  %\href{http://xxx.lanl.gov/abs/0803.4295}{{\normalfont
  %[arXiv:astro-ph/0803.4295]}}.
\newblock
{\changeurlcolor{black}\href{https://doi.org/10.1103/PhysRevD.77.123525}{\detokenize{https://doi.org/10.1103/PhysRevD.77.123525}}}.

\bibitem[Jassal \em{et~al.}(2005)Jassal, Bagla, and Padmanabhan]{Jassal:2004ej}
Jassal, H.K.; Bagla, J.S.; Padmanabhan, T.
\newblock {WMAP constraints on low redshift evolution of dark energy}.
\newblock {\em Mon. Not. Roy. Astron. Soc.} {\bf 2005}, {\em 356},~L11--L16.
  %\href{http://xxx.lanl.gov/abs/astro-ph/0404378}{{\normalfont
  %[astro-ph/0404378]}}.
\newblock
{\changeurlcolor{black}\href{https://doi.org/10.1111/j.1745-3933.2005.08577.x}{\detokenize{https://doi.org/10.1111/j.1745-3933.2005.08577.x}}}.

\bibitem[Feng \em{et~al.}(2012)Feng, Shen, Li, and Li]{Feng:2012gf}
Feng, C.J.; Shen, X.Y.; Li, P.; Li, X.Z.
\newblock {A New Class of Parametrization for Dark Energy without Divergence}.
\newblock {\em JCAP} {\bf 2012}, {\em 09},~023.
  %\href{http://xxx.lanl.gov/abs/1206.0063}{{\normalfont
  %[arXiv:astro-ph.CO/1206.0063]}}.
\newblock
{\changeurlcolor{black}\href{https://doi.org/10.1088/1475-7516/2012/09/023}{\detokenize{https://doi.org/10.1088/1475-7516/2012/09/023}}}.

\bibitem[Komatsu \em{et~al.}(2009)Komatsu et~al.]{Komatsu:2008hk}
Komatsu, E.; {Dunkley, J.; Nolta, M.R.; Bennett, C.L.; Gold, B.; Hinshaw, G.; Jarosik, N.; Larson, D.; Limon, M.; Page, L.; }et al.
\newblock {Five-Year Wilkinson Microwave Anisotropy Probe (WMAP) Observations:
  Cosmological Interpretation}.
\newblock {\em Astrophys. J. Suppl.} {\bf 2009}, {\em 180},~330--376.
  %\href{http://xxx.lanl.gov/abs/0803.0547}{{\normalfont
  %[arXiv:astro-ph/0803.0547]}}.
\newblock
{\changeurlcolor{black}\href{https://doi.org/10.1088/0067-0049/180/2/330}{\detokenize{https://doi.org/10.1088/0067-0049/180/2/330}}}.

\bibitem[Di~Pietro and Claeskens(2003)]{DiPietro:2002cz}
Di~Pietro, E.; Claeskens, J.F.
\newblock {Future supernovae data and quintessence models}.
\newblock {\em Mon. Not. Roy. Astron. Soc.} {\bf 2003}, {\em 341},~1299.
  %\href{http://xxx.lanl.gov/abs/astro-ph/0207332}{{\normalfont
  %[astro-ph/0207332]}}.
\newblock
{\changeurlcolor{black}\href{https://doi.org/10.1046/j.1365-8711.2003.06508.x}{\detokenize{https://doi.org/10.1046/j.1365-8711.2003.06508.x}}}.

\bibitem[Nesseris and Perivolaropoulos(2004)]{Nesseris:2004wj}
Nesseris, S.; Perivolaropoulos, L.
\newblock {A Comparison of cosmological models using recent supernova data}.
\newblock {\em Phys. Rev. D} {\bf 2004}, {\em 70},~043531.
  %\href{http://xxx.lanl.gov/abs/astro-ph/0401556}{{\normalfont
  %[astro-ph/0401556]}}.
\newblock
{\changeurlcolor{black}\href{https://doi.org/10.1103/PhysRevD.70.043531}{\detokenize{https://doi.org/10.1103/PhysRevD.70.043531}}}.

\bibitem[Perivolaropoulos(2005)]{Perivolaropoulos:2004yr}
Perivolaropoulos, L.
\newblock {Constraints on linear negative potentials in quintessence and
  phantom models from recent supernova data}.
\newblock {\em Phys. Rev. D} {\bf 2005}, {\em 71},~063503.
  %\href{http://xxx.lanl.gov/abs/astro-ph/0412308}{{\normalfont
  %[astro-ph/0412308]}}.
\newblock
{\changeurlcolor{black}\href{https://doi.org/10.1103/PhysRevD.71.063503}{\detokenize{https://doi.org/10.1103/PhysRevD.71.063503}}}.

\bibitem[Lazkoz \em{et~al.}(2005)Lazkoz, Nesseris, and
  Perivolaropoulos]{Lazkoz:2005sp}
Lazkoz, R.; Nesseris, S.; Perivolaropoulos, L.
\newblock {Exploring Cosmological Expansion Parametrizations with the Gold SnIa
  Dataset}.
\newblock {\em JCAP} {\bf 2005}, {\em 11},~010.
  %\href{http://xxx.lanl.gov/abs/astro-ph/0503230}{{\normalfont
  %[astro-ph/0503230]}}.
\newblock
{\changeurlcolor{black}\href{https://doi.org/10.1088/1475-7516/2005/11/010}{\detokenize{https://doi.org/10.1088/1475-7516/2005/11/010}}}.

\bibitem[Deng and Wei(2018)]{Deng:2018jrp}
Deng, H.K.; Wei, H.
\newblock {Null signal for the cosmic anisotropy in the Pantheon supernovae
  data}.
\newblock {\em Eur. Phys. J. C} {\bf 2018}, {\em 78},~755.
  %\href{http://xxx.lanl.gov/abs/1806.02773}{{\normalfont
  %[arXiv:astro-ph.CO/1806.02773]}}.
\newblock
{\changeurlcolor{black}\href{https://doi.org/10.1140/epjc/s10052-018-6159-4}{\detokenize{https://doi.org/10.1140/epjc/s10052-018-6159-4}}}.

\bibitem[Blake \em{et~al.}(2012)Blake et~al.]{Blake:2012pj}
Blake, C.;  {Blake, C.; Brough, S.; Colless, M.; Contreras, C.; Couch, W.; Croom, S.; Croton, D.; Davis, T.M.; Drinkwater, M.J.; Forster, K.;} et al.
\newblock {The WiggleZ Dark Energy Survey: Joint measurements of the expansion
  and growth history at z \ensuremath{<} 1}.
\newblock {\em Mon. Not. Roy. Astron. Soc.} {\bf 2012}, {\em 425},~405--414.
  %\href{http://xxx.lanl.gov/abs/1204.3674}{{\normalfont
  %[arXiv:astro-ph.CO/1204.3674]}}.
\newblock
{\changeurlcolor{black}\href{https://doi.org/10.1111/j.1365-2966.2012.21473.x}{\detokenize{https://doi.org/10.1111/j.1365-2966.2012.21473.x}}}.

\bibitem[Carvalho \em{et~al.}(2016)Carvalho, Bernui, Benetti, Carvalho, and
  Alcaniz]{Carvalho:2015ica}
Carvalho, G.C.; Bernui, A.; Benetti, M.; Carvalho, J.C.; Alcaniz, J.S.
\newblock {Baryon Acoustic Oscillations from the SDSS DR10 galaxies angular
  correlation function}.
\newblock {\em Phys. Rev. D} {\bf 2016}, {\em 93},~023530.
  %\href{http://xxx.lanl.gov/abs/1507.08972}{{\normalfont
  %[arXiv:astro-ph.CO/1507.08972]}}.
\newblock
{\changeurlcolor{black}\href{https://doi.org/10.1103/PhysRevD.93.023530}{\detokenize{https://doi.org/10.1103/PhysRevD.93.023530}}}.

\bibitem[Seo \em{et~al.}(2012)Seo et~al.]{Seo:2012xy}
Seo, H.-J.; {Ho, S.; White, M.; Cuesta, A.J.; Ross, A.; Saito, S.; Reid, B.; Padmanabhan, N.; Percival, W.J.; De Putter, R.; et al.}
\newblock {Acoustic scale from the angular power spectra of SDSS-III DR8
  photometric luminous galaxies}.
\newblock {\em Astrophys. J.} {\bf 2012}, {\em 761},~13.
  %\href{http://xxx.lanl.gov/abs/1201.2172}{{\normalfont
  %[arXiv:astro-ph.CO/1201.2172]}}.
\newblock
{\changeurlcolor{black}\href{https://doi.org/10.1088/0004-637X/761/1/13}{\detokenize{https://doi.org/10.1088/0004-637X/761/1/13}}}.

\bibitem[Sridhar \em{et~al.}(2020)Sridhar, Song, Ross, Zhou, Newman, Chuang,
  Prada, Blum, Gazta\~naga, and Landriau]{Sridhar:2020czy}
Sridhar, S.; Song, Y.S.; Ross, A.J.; Zhou, R.; Newman, J.A.; Chuang, C.H.;
  Prada, F.; Blum, R.; Gazta\~naga, E.; Landriau, M.
\newblock {Clustering of LRGs in the DECaLS DR8 Footprint: Distance Constraints
  from Baryon Acoustic Oscillations Using Photometric Redshifts}.
\newblock {\em Astrophys. J.} {\bf 2020}, {\em 904},~69.
  %\href{http://xxx.lanl.gov/abs/2005.13126}{{\normalfont
  %[arXiv:astro-ph.CO/2005.13126]}}.
\newblock
{\changeurlcolor{black}\href{https://doi.org/10.3847/1538-4357/abc0f0}{\detokenize{https://doi.org/10.3847/1538-4357/abc0f0}}}.


\bibitem[Tamone \em{et~al.}(2020)Tamone et~al.]{Tamone:2020qrl}
Tamone, A.; {Raichoor, A.; Zhao, C.; de Mattia, A.; Gorgoni, C.; Burtin, E.; Ruhlmann-Kleider, V.; Ross, A.J.; Alam, S.; Percival, W.J.;} et al.
\newblock {The Completed SDSS-IV extended Baryon Oscillation Spectroscopic
  Survey: Growth rate of structure measurement from anisotropic clustering
  analysis in configuration space between redshift 0.6 and 1.1 for the Emission
  Line Galaxy sample}.
\newblock {\em Mon. Not. Roy. Astron. Soc.} {\bf 2020}, {\em 499},~5527--5546.
  %\href{http://xxx.lanl.gov/abs/2007.09009}{{\normalfont
  %[arXiv:astro-ph.CO/2007.09009]}}.
\newblock
{\changeurlcolor{black}\href{https://doi.org/10.1093/mnras/staa3050}{\detokenize{https://doi.org/10.1093/mnras/staa3050}}}.

\bibitem[Zhu \em{et~al.}(2018)Zhu et~al.]{Zhu:2018edv}
Zhu, F.; {Padmanabhan, N.; Ross, A.J.; White, M.; Percival, W.J.; Ruggeri, R.; Zhao, G.; Wang, D.; Mueller, E.-M.; Burtin, E.;} et al.
\newblock {The clustering of theSDSS-IV extended Baryon Oscillation
  Spectroscopic Survey DR14 quasar sample: Measuring the anisotropic baryon
  acoustic oscillations with redshift weights}.
\newblock {\em Mon. Not. Roy. Astron. Soc.} {\bf 2018}, {\em 480},~1096--1105.
  %\href{http://xxx.lanl.gov/abs/1801.03038}{{\normalfont
  %[arXiv:astro-ph.CO/1801.03038]}}.
\newblock
{\changeurlcolor{black}\href{https://doi.org/10.1093/mnras/sty1955}{\detokenize{https://doi.org/10.1093/mnras/sty1955}}}.

\bibitem[Hou \em{et~al.}(2020)Hou et~al.]{Hou:2020rse}
Hou, J.; Sánchez, {A.G.; Ross, A.J.; Smith, A.; Neveux, R.; Bautista, J.; Burtin, E.; Zhao, C.; Scoccimarro, R.; Dawson, K.S.;} et al.
\newblock {The Completed SDSS-IV extended Baryon Oscillation Spectroscopic
  Survey: BAO and RSD measurements from anisotropic clustering analysis of the
  Quasar Sample in configuration space between redshift 0.8 and 2.2}.
\newblock {\em Mon. Not. Roy. Astron. Soc.} {\bf 2020}, {\em 500},~1201--1221.
  %\href{http://xxx.lanl.gov/abs/2007.08998}{{\normalfont
  %[arXiv:astro-ph.CO/2007.08998]}}.
\newblock
{\changeurlcolor{black}\href{https://doi.org/10.1093/mnras/staa3234}{\detokenize{https://doi.org/10.1093/mnras/staa3234}}}.

\bibitem[Blomqvist \em{et~al.}(2019)Blomqvist et~al.]{Blomqvist:2019rah}
Blomqvist, M.;  et~al.
\newblock {Baryon acoustic oscillations from the cross-correlation of
  Ly$\alpha$ absorption and quasars in eBOSS DR14}.
\newblock {\em Astron. Astrophys.} {\bf 2019}, {\em 629},~A86.
  %\href{http://xxx.lanl.gov/abs/1904.03430}{{\normalfont
  %[arXiv:astro-ph.CO/1904.03430]}}.
\newblock
{\changeurlcolor{black}\href{https://doi.org/10.1051/0004-6361/201935641}{\detokenize{https://doi.org/10.1051/0004-6361/201935641}}}.


\bibitem[Scolnic \em{et~al.}(2018{\natexlab{a}})Scolnic
  et~al.]{Pan-STARRS1:2017jku}
{Scolnic%References 93 and 94 are identical, please revise.
}, D.M.;  {Scolnic, D.M.; Jones, D.O.; Rest, A.; Pan, Y.C.; Chornock, R.; Foley, R.J.; Huber, M.E.; Kessler, R.; Narayan, G.; Riess, A.G.;} et al.
\newblock {The Complete Light-curve Sample of Spectroscopically Confirmed SNe
  Ia from Pan-STARRS1 and Cosmological Constraints from the Combined Pantheon
  Sample}.
\newblock {\em Astrophys. J.} {\bf 2018}, {\em 859},~101.
  %\href{http://xxx.lanl.gov/abs/1710.00845}{{\normalfont
  %[arXiv:astro-ph.CO/1710.00845]}}.
\newblock
{\changeurlcolor{black}\href{https://doi.org/10.3847/1538-4357/aab9bb}{\detokenize{https://doi.org/10.3847/1538-4357/aab9bb}}}.


\bibitem[Demianski \em{et~al.}(2017)Demianski, Piedipalumbo, Sawant, and
  Amati]{Demianski:2016zxi}
Demianski, M.; Piedipalumbo, E.; Sawant, D.; Amati, L.
\newblock {Cosmology with gamma-ray bursts: I. The Hubble diagram through the
  calibrated $E_{\rm p,i}$ - $E_{\rm iso}$ correlation}.
\newblock {\em Astron. Astrophys.} {\bf 2017}, {\em 598},~A112.
  %\href{http://xxx.lanl.gov/abs/1610.00854}{{\normalfont
  %[arXiv:astro-ph.CO/1610.00854]}}.
\newblock
{\changeurlcolor{black}\href{https://doi.org/10.1051/0004-6361/201628909}{\detokenize{https://doi.org/10.1051/0004-6361/201628909}}}.

\bibitem[Kazantzidis and Perivolaropoulos(2018)]{Kazantzidis:2018rnb}
Kazantzidis, L.; Perivolaropoulos, L.
\newblock {Evolution of the $f\sigma_8$ tension with the Planck15/$\Lambda$CDM
  determination and implications for modified gravity theories}.
\newblock {\em Phys. Rev. D} {\bf 2018}, {\em 97},~103503.
  %\href{http://xxx.lanl.gov/abs/1803.01337}{{\normalfont
  %[arXiv:astro-ph.CO/1803.01337]}}.
\newblock
{\changeurlcolor{black}\href{https://doi.org/10.1103/PhysRevD.97.103503}{\detokenize{https://doi.org/10.1103/PhysRevD.97.103503}}}.

\bibitem[Benisty and Staicova(2021)]{Benisty:2020otr}
Benisty, D.; Staicova, D.
\newblock {Testing late-time cosmic acceleration with uncorrelated baryon
  acoustic oscillation dataset}.
\newblock {\em Astron. Astrophys.} {\bf 2021}, {\em 647},~A38.
  %\href{http://xxx.lanl.gov/abs/2009.10701}{{\normalfont
  %[arXiv:astro-ph.CO/2009.10701]}}.
\newblock
{\changeurlcolor{black}\href{https://doi.org/10.1051/0004-6361/202039502}{\detokenize{https://doi.org/10.1051/0004-6361/202039502}}}.

\bibitem[Handley \em{et~al.}(2015)Handley, Hobson, and
  Lasenby]{Handley:2015fda}
Handley, W.J.; Hobson, M.P.; Lasenby, A.N.
\newblock {PolyChord: Nested sampling for cosmology}.
\newblock {\em Mon. Not. Roy. Astron. Soc.} {\bf 2015}, {\em 450},~L61--L65.
  %\href{http://xxx.lanl.gov/abs/1502.01856}{{\normalfont
  %[arXiv:astro-ph.CO/1502.01856]}}.
\newblock
{\changeurlcolor{black}\href{https://doi.org/10.1093/mnrasl/slv047}{\detokenize{https://doi.org/10.1093/mnrasl/slv047}}}.

\bibitem[Lewis(2019)]{Lewis:2019xzd}
Lewis, A.
\newblock {GetDist: A Python package for analysing Monte Carlo {samples}.} \emph{arXiv}  {\bf
  2019}, arXiv:1910.13970.
\newblock  %\href{http://xxx.lanl.gov/abs/1910.13970}{{\normalfont
  %[arXiv:astro-ph.IM/1910.13970]}}.

\bibitem[Yang \em{et~al.}(2022)Yang, Giar\`e, Pan, Di~Valentino, Melchiorri,
  and Silk]{Yang:2022kho}
Yang, W.; Giar\`e, W.; Pan, S.; Di~Valentino, E.; Melchiorri, A.; Silk, J.
\newblock {Revealing the effects of curvature on the cosmological {models}.} \emph{arXiv} {\bf
  2022}, arXiv:2210.09865.
\newblock  %\href{http://xxx.lanl.gov/abs/2210.09865}{{\normalfont
  %[arXiv:astro-ph.CO/2210.09865]}}.

\bibitem[Benisty \em{et~al.}(2022)Benisty, Mifsud, Said, and
  Staicova]{Benisty:2022psx}
Benisty, D.; Mifsud, J.; Said, J.L.; Staicova, D.
\newblock {On the Robustness of the Constancy of the Supernova Absolute
  Magnitude: Non-parametric Reconstruction \& Bayesian {approaches}.} \emph{arXiv} {\bf 2022}, arXiv:2202.04677.
\newblock  %\href{http://xxx.lanl.gov/abs/2202.04677}{{\normalfont
  %[arXiv:astro-ph.CO/2202.04677]}}.

\bibitem[Ferramacho \em{et~al.}(2009)Ferramacho, Blanchard, and
  Zolnierowski]{Ferramacho:2008ap}
Ferramacho, L.D.; Blanchard, A.; Zolnierowski, Y.
\newblock {Constraints on C.D.M. cosmology from galaxy power spectrum, CMB and
  SNIa evolution}.
\newblock {\em Astron. Astrophys.} {\bf 2009}, {\em 499},~21.
  %\href{http://xxx.lanl.gov/abs/0807.4608}{{\normalfont
  %[arXiv:astro-ph/0807.4608]}}.
\newblock
{\changeurlcolor{black}\href{https://doi.org/10.1051/0004-6361/200810693}{\detokenize{https://doi.org/10.1051/0004-6361/200810693}}}.

\bibitem[Linden \em{et~al.}(2009)Linden, Virey, and Tilquin]{Linden:2009vh}
Linden, S.; Virey, J.M.; Tilquin, A.
\newblock {Cosmological Parameter Extraction and Biases from Type Ia Supernova
  Magnitude Evolution}.
\newblock {\em Astron. Astrophys.} {\bf 2009}, {\em 50},~1095--1105.
  %\href{http://xxx.lanl.gov/abs/0907.4495}{{\normalfont
  %[arXiv:astro-ph.CO/0907.4495]}}.
\newblock
{\changeurlcolor{black}\href{https://doi.org/10.1051/0004-6361/200912811}{\detokenize{https://doi.org/10.1051/0004-6361/200912811}}}.

\bibitem[Tutusaus \em{et~al.}(2017)Tutusaus, Lamine, Dupays, and
  Blanchard]{Tutusaus:2017ibk}
Tutusaus, I.; Lamine, B.; Dupays, A.; Blanchard, A.
\newblock {Is cosmic acceleration proven by local cosmological probes?}
\newblock {\em Astron. Astrophys.} {\bf 2017}, {\em 602},~A73.
  %\href{http://xxx.lanl.gov/abs/1706.05036}{{\normalfont
  %[arXiv:astro-ph.CO/1706.05036]}}.
\newblock
{\changeurlcolor{black}\href{https://doi.org/10.1051/0004-6361/201630289}{\detokenize{https://doi.org/10.1051/0004-6361/201630289}}}.

\bibitem[Di~Valentino \em{et~al.}(2020)Di~Valentino, Gariazzo, Mena, and
  Vagnozzi]{DiValentino:2020evt}
Di~Valentino, E.; Gariazzo, S.; Mena, O.; Vagnozzi, S.
\newblock {Soundness of Dark Energy properties}.
\newblock {\em JCAP} {\bf 2020}, {\em 07},~045.
  %\href{http://xxx.lanl.gov/abs/2005.02062}{{\normalfont
  %[arXiv:astro-ph.CO/2005.02062]}}.
\newblock
{\changeurlcolor{black}\href{https://doi.org/10.1088/1475-7516/2020/07/045}{\detokenize{https://doi.org/10.1088/1475-7516/2020/07/045}}}.

\bibitem[Perivolaropoulos and Skara(2022)]{Perivolaropoulos:2022khd}
Perivolaropoulos, L.; Skara, F.
\newblock {A reanalysis of the latest SH0ES data for $H_0$: Effects of new
  degrees of freedom on the Hubble tension}.
\newblock {\em Universe} {\bf 2022}, {\em 8}, {502}.
  %\href{http://xxx.lanl.gov/abs/2208.11169}{{\normalfont
  %[arXiv:astro-ph.CO/2208.11169]}}.
\newblock
{\changeurlcolor{black}\href{https://doi.org/10.3390/universe8100502}{\detokenize{https://doi.org/10.3390/universe8100502}}}.

\bibitem[de~Carvalho \em{et~al.}(2021)de~Carvalho, Bernui, Avila, Novaes, and
  Nogueira-Cavalcante]{deCarvalho:2021azj}
de~Carvalho, E.; Bernui, A.; Avila, F.; Novaes, C.P.; Nogueira-Cavalcante, J.P.
\newblock {BAO angular scale at zeff = 0.11 with the SDSS blue galaxies}.
\newblock {\em Astron. Astrophys.} {\bf 2021}, {\em 649},~A20.
  %\href{http://xxx.lanl.gov/abs/2103.14121}{{\normalfont
  %[arXiv:astro-ph.CO/2103.14121]}}.
\newblock
{\changeurlcolor{black}\href{https://doi.org/10.1051/0004-6361/202039936}{\detokenize{https://doi.org/10.1051/0004-6361/202039936}}}.

\bibitem[Chuang \em{et~al.}(2017)Chuang et~al.]{BOSS:2016goe}
{Chuang}, C.H.;  {Pellejero-Ibanez, M.; Rodríguez-Torres, S.; Ross, A.J.; Zhao, G.; Wang, Y.; Cuesta, A.J.; Rubiño-Martin, J.A.; Prada, F.; Alam, S.; }et al.
\newblock {The clustering of galaxies in the completed SDSS-III Baryon
  Oscillation Spectroscopic Survey: Single-probe measurements from DR12 galaxy
  clustering \textendash{} towards an accurate model}.
\newblock {\em Mon. Not. Roy. Astron. Soc.} {\bf 2017}, {\em 471},~2370--2390.
  %\href{http://xxx.lanl.gov/abs/1607.03151}{{\normalfont
  %[arXiv:astro-ph.CO/1607.03151]}}.
\newblock
{\changeurlcolor{black}\href{https://doi.org/10.1093/mnras/stx1641}{\detokenize{https://doi.org/10.1093/mnras/stx1641}}}.

\bibitem[Alam \em{et~al.}(2017)Alam et~al.]{BOSS:2016wmc}
{Alam}, S.;  Ata, M.; Bailey, S.; Beutler, F.; Bizyaev, D.; Blazek, J.A.; Bolton, A.S.; Brownstein, J.R.; Burden, A.; Chuang, C.-H.; et al.
\newblock {The clustering of galaxies in the completed SDSS-III Baryon
  Oscillation Spectroscopic Survey: Cosmological analysis of the DR12 galaxy
  sample}.
\newblock {\em Mon. Not. Roy. Astron. Soc.} {\bf 2017}, {\em 470},~2617--2652.
  %\href{http://xxx.lanl.gov/abs/1607.03155}{{\normalfont
  %[arXiv:astro-ph.CO/1607.03155]}}.
\newblock
{\changeurlcolor{black}\href{https://doi.org/10.1093/mnras/stx721}{\detokenize{https://doi.org/10.1093/mnras/stx721}}}.

\bibitem[Beutler \em{et~al.}(2017)Beutler et~al.]{BOSS:2016hvq}
{Beutler}, F.;  {Seo, H.-J.; Ross, A.J.; McDonald, P.; Saito, S.; Bolton, A.S.; Brownstein, J.R.; Chuang, C.-H.; Cuesta, A.J.; Eisenstein, D.J.;} et al.
\newblock {The clustering of galaxies in the completed SDSS-III Baryon
  Oscillation Spectroscopic Survey: Baryon acoustic oscillations in the Fourier
  space}.
\newblock {\em Mon. Not. Roy. Astron. Soc.} {\bf 2017}, {\em 464},~3409--3430.
  %\href{http://xxx.lanl.gov/abs/1607.03149}{{\normalfont
  %[arXiv:astro-ph.CO/1607.03149]}}.
\newblock
{\changeurlcolor{black}\href{https://doi.org/10.1093/mnras/stw2373}{\detokenize{https://doi.org/10.1093/mnras/stw2373}}}.

\bibitem[Abbott \em{et~al.}(2019)Abbott et~al.]{DES:2017rfo}
{Abbott}, T.M.C.;  {Abdalla, F.B.; Alarcon, A.; Allam, S.; Andrade-Oliveira, F.; Annis, J.; Avila, S.; Banerji, M.; Banik, N.; Bechtol, K.;} et al.
\newblock {Dark Energy Survey Year 1 Results: Measurement of the Baryon
  Acoustic Oscillation scale in the distribution of galaxies to redshift 1}.
\newblock {\em Mon. Not. Roy. Astron. Soc.} {\bf 2019}, {\em 483},~4866--4883.
  %\href{http://xxx.lanl.gov/abs/1712.06209}{{\normalfont
  %[arXiv:astro-ph.CO/1712.06209]}}.
\newblock
{\changeurlcolor{black}\href{https://doi.org/10.1093/mnras/sty3351}{\detokenize{https://doi.org/10.1093/mnras/sty3351}}}.

\bibitem[du~Mas~des Bourboux \em{et~al.}(2017)du~Mas~des Bourboux
  et~al.]{duMasdesBourboux:2017mrl}
{du~Mas~des Bourboux, H}.;  {Le Goff, J.-M.; Blomqvist, M.; Busca, N.G.; Guy, J.; Rich, J.; Yèche, C.; Bautista, J.E.; Burtin, E.; Dawson, K.S.;} et al.
\newblock {Baryon acoustic oscillations from the complete SDSS-III
  Ly$\alpha$-quasar cross-correlation function at $z=2.4$}.
\newblock {\em Astron. Astrophys.} {\bf 2017}, {\em 608},~A130.
  %\href{http://xxx.lanl.gov/abs/1708.02225}{{\normalfont
  %[arXiv:astro-ph.CO/1708.02225]}}.
\newblock
{\changeurlcolor{black}\href{https://doi.org/10.1051/0004-6361/201731731}{\detokenize{https://doi.org/10.1051/0004-6361/201731731}}}.

\end{thebibliography}


\end{adjustwidth}
\end{document}
%
% ****** End of file apssamp.tex ******
