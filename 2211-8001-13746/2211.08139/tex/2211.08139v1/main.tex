%=================================================================
\documentclass[preprtints,article,accept,pdftex,oneauthor ]{Definitions/mdpi} 
%\documentclass[preprints,article,accept,moreauthors,pdftex]{mdpi}.

%--------------------
% Class Options:
%--------------------
%----------
% journal
%----------3
% Choose between the following MDPI journals:
% universe
%---------
% article
%---------
% The default type of manuscript is "article", but can be replaced by: 
%----------
% submit
%----------
% The class option "submit" will be changed to "accept" by the Editorial Office when the paper is accepted. This will only make changes to the frontpage (e.g., the logo of the journal will get visible), the headings, and the copyright information. Also, line numbering will be removed. Journal info and pagination for accepted papers will also be assigned by the Editorial Office.

%------------------
% moreauthors
%------------------
% If there is only one author the class option oneauthor should be used. Otherwise use the class option moreauthors.

%---------
% pdftex
%---------
% The option pdftex is for use with pdfLaTeX. If eps figures are used, remove the option pdftex and use LaTeX and dvi2pdf.

%=================================================================
% MDPI internal commands
\firstpage{1} 
\makeatletter 
\setcounter{page}{\@firstpage} 
\makeatother
\pubvolume{1}
\issuenum{1}
\articlenumber{0}
\pubyear{2022}
\copyrightyear{2022}
%\externaleditor{Academic Editor: Firstname Lastname}
\datereceived{} 
%\daterevised{} % Only for the journal Acoustics
\dateaccepted{} 
\datepublished{} 
%\datecorrected{} % Corrected papers include a "Corrected: XXX" date in the original paper.
%\dateretracted{} % Corrected papers include a "Retracted: XXX" date in the original paper.
\hreflink{https://doi.org/} % If needed use \linebreak
%\doinum{}
%------------------------------------------------------------------
% The following line should be uncommented if the LaTeX file is uploaded to arXiv.org
%\pdfoutput=1

%=================================================================
% Add packages and commands here. The following packages are loaded in our class file: fontenc, inputenc, calc, indentfirst, fancyhdr, graphicx, epstopdf, lastpage, ifthen, lineno, float, amsmath, setspace, enumitem, mathpazo, booktabs, titlesec, etoolbox, tabto, xcolor, soul, multirow, microtype, tikz, totcount, changepage, attrib, upgreek, cleveref, amsthm, hyphenat, natbib, hyperref, footmisc, url, geometry, newfloat, caption

%\usepackage{verbatim}
%\usepackage{comment}
%\usepackage{graphicx}
%\usepackage[T1]{fontenc}
%\usepackage{graphicx}
\usepackage{epsfig}
\usepackage{epstopdf}
%\usepackage{bm}
%\usepackage{tensor}
\usepackage{amsfonts}
%\usepackage[T1]{fontenc}
%\usepackage[latin9]{inputenc}
%\usepackage{fouriernc1} % sharper font for on-screen
\usepackage{amssymb}
%https://www.overleaf.com/project/6006ae558abbb08625a2e4bb
%\usepackage{float}
%\usepackage{amsmath}
\makeatletter
\let\c@lofdepth\relax
\let\c@lotdepth\relax
\makeatother
\usepackage{subfigure}
\usepackage{tabularx}
%\usepackage{newcent}
%\usepackage{dcolumn}

%\usepackage[normalem]{ulem}
%\usepackage[utf8]{inputenc}
%\usepackage{cancel}
%\usepackage[colorlinks]{hyperref}
%\usepackage[usenames,dvipsnames]{color}
%\usepackage{hyperref}


%=================================================================
% Full title of the paper (Capitalized)
\Title{DE models with combined $H_0 \cdot r_d $ from BAO and CMB dataset and friends}

% MDPI internal command: Title for citation in the left column
\TitleCitation{DE models with combined $H_0 \cdot r_d $ from BAO and CMB dataset and friends}

% Author Orchid ID: enter ID or remove command
\newcommand{\orcidauthorA}{0000-0001-6139-3125} % Add \orcidA{} behind the author's name
%\newcommand{\orcidauthorB}{0000-0000-0000-000X} % Add \orcidB{} behind the author's name

% Authors, for the paper (add full first names)
\Author{Denitsa Staicova $^{1,}$\orcidA{}}

%\longauthorlist{yes}

% MDPI internal command: Authors, for metadata in PDF
\AuthorNames{Denitsa Staicova}

% MDPI internal command: Authors, for citation in the left column
\AuthorCitation{Staicova, D.}
% If this is a Chicago style journal: Lastname, Firstname, Firstname Lastname, and Firstname Lastname.

% Affiliations / Addresses (Add [1] after \address if there is only one affiliation.)
\address{%
$^{1}$ \quad Institute for Nuclear Research and Nuclear Energy, Bulgarian Academy of Sciences, Sofia, Bulgaria; dstaicova@inrne.bas.bg\\}

% Contact information of the corresponding author
\corres{Correspondence: dstaicova@inrne.bas.bg;}

%\documentclass[reprint,aps,amsmath,amssymb,nofootinbib,twoside,prd,showkeys,superscriptaddress]{revtex4-1}
%\pdfoutput=1
\begin{document}

\begin{abstract}
It has been theorized that Dynamical Dark Energy (DDE) could be a possible solution to the Hubble tension. To avoid the degeneracy between the Hubble parameter $H_0$ and the sound horizon scale $r_d$, in this article we use their multiplication as one parameter $c/\left(H_0 r_d\right)$ and we use it to infer cosmological parameters for 6 different models - $\Lambda$CDM and 5 DDE parametrizations -- the Chevallier-Polarski-Linder (CPL), the  Barboza-Alcaniz (BA), the  Low correlation (LC), the  Jassal-Bagla-Padmanabhan (JBP) and  the  Feng-Shen-Li-Li model. We choose a dataset that treats this combination as one parameter, that includes the Baryon Acoustic Oscillation (BAO) data $0.11 \le z \le 2.40$ and additional points from the Cosmic Microwave Background (CMB) Peaks ($z \approx 1090$). To them, we add the marginalized Pantehon dataset and GRB dataset. We see that the tension is moved from $H_0$ and $r_d$ to $c/\left(H_0 r_d\right)$ and $\Omega_m$. There is only one model that satisfies the Planck 2018 constraints on both parameters and this is LC with a huge error. The rest cannot fit into both constraints. $\Lambda$CDM is preferred with respect to the statistical measures.
\end{abstract}

% Keywords
\keyword{Cosmological tensions; dynamical dark energy; Baryon Acoustic Oscillation; Pantheon dataset; Gamma-Ray Bursts;} 

%\maketitle

%\tableofcontents


\section{Introduction}
The quest for understanding the cosmological tensions  has driven research for years now. It seems that the tension between the direct measurements of the Hubble constant ($H_0$) from the late universe (\cite{Freedman:2000cf,Riess:1998cb,Perlmutter:1998np, Riess:2020fzl, Riess:2022mme}) and that from the early universe (i.e. from measuring the temperature and polarization anisotropies in the Cosmic Microwave Background  \cite{Troxel:2017xyo,Aghanim:2018eyx,Ade:2015xua,Dainotti:2021pqg}) seems to only aggravate with the increase of the precision and knowledge of systematics of the data and has reached $5\sigma$.  This discrepancy has spurred a lot of works, trying to resolve whether the dark energy is a constant energy density or with a dynamical behavior and if so of what origin, leading to many different theories and possible explanations \cite{Benisty:2021wxi,Capozziello:2011et,Bull:2015stt,DiValentino:2021izs,Yang:2021flj, Schoneberg:2019wmt,DiValentino:2017gzb,DiValentino:2020zio,DiValentino:2021izs,Perivolaropoulos:2021jda,Lucca:2021dxo}.


There are many Dark Energy (DE) parametrisations \cite{Wang:2018fng,Reyes:2021owe,Colgain:2021pmf, 2108.04188} that can be used in the search for deviations from the cosmological constant, $\Lambda$.
Some of them fall in the group of Early Dark Energy models \cite{Pettorino:2013ia, Poulin:2018cxd, Lin:2020jcb, Smith:2022hwi, Smith:2020rxx}, which modify physics of the early universe. Others modify the late-time universe physics such as in the Phantom Dark Energy \cite{DiValentino:2020vnx, Haridasu:2020pms} models, Emergent Dark Energy \cite{Li:2019yem,Yang:2020ope}, or add interaction in the DE sector as in the Interacting Dark Energy \cite{Kumar:2017dnp,DiValentino:2019ffd, Yang:2019uzo}, or add exotic species or scalar fields\cite{Gogoi:2020qif,Sakstein:2019fmf,Tian:2021omz,Nojiri:2021dze,Seto:2021xua}. For review on the taxonomy of DE models see Ref. \cite{Escamilla-Rivera:2021boq, Motta:2021hvl, Yang:2021eud}. Finally there is the Generalized Emergent Dark Energy (GEDE) \cite{Yang:2021eud} found to be able to compete with $\Lambda$CDM for some BAO datasets \cite{Staicova:2021ntm}. In this work, we take the so called dynamical dark energy parametrizations which allow for a non-constant DE contribution regardless of the origin behind it. We take a number of models, namely the Chevallier-Polarski-Linder (CPL), the  Barboza-Alcaniz (BA), the  Low correlation (LC), the  Jassal-Bagla-Padmanabhan (JBP) and  the  Feng-Shen-Li-Li (FSLII) model and we use statistical measures to judge their performance in fitting the data. 

An important part of the tensions debate revolves around the role of the sound horizon at drag epoch $r_d$. At recombination, after the onset of CMB at $z_* \approx 1090$, the baryons escape the drag of photons at the drag epoch, $z_d \approx 1059$ (Planck 2018 \cite{Aghanim:2018eyx}). This sets the standard ruler for the Baryon Acoustic Oscillations  (BAO) -- the distance ($r_d$) at which the baryon-photon plasma waves oscillating in the hot  Universe froze at $z=z_d$. The sound horizon at drag epoch is given by:
\begin{equation}
r_d = \int_{z_d}^{\infty} \frac{c_s(z)}{H(z)} dz
,\end{equation}
where $c_s \approx c \left(3 + 9\rho_b /(4\rho_\gamma) \right)^{-0.5}$ is the speed of sound in the baryon-photon fluid with the baryon $\rho_b(z)$ and the photon $\rho_\gamma(z)$ densities, respectively \cite{Aubourg:2014yra,Arendse:2019itb}.

Many papers discuss the relation between the $H_0$ and the sound horizon scale $r_d$ for different models \cite{Aylor:2018drw,Pogosian:2020ded,Aizpuru:2021vhd}. Any DE model claiming to resolve the $H_0$ tension, should also be able to resolve the $r_d$ tension since they are strongly connected \cite{Jedamzik:2020zmd,Aizpuru:2021vhd,delaMacorra:2021hoh}. In another words, setting a prior on $r_d$ has a very strong effect on $H_0$ and vice versa. In this paper, to avoid this problem, we combine the $H_0$ and $r_d$ into one parameter. We choose measurements that combines the $H_0 $ and $r_d$ from the BAO and the distance prior from the CMB peaks \cite{Wang:2013mha, Mamon:2016wow, Grandon:2018uoe, Chen:2018dbv, daSilva:2018ehn, Zhai:2018vmm,DiValentino:2020hov, Nilsson:2021ute, Yao:2022kub} and we use them to infer cosmological parameters for $\Lambda$CDM and 5 DDE models. To the BAO+CMB dataset we add the gamma-ray bursts (GRB) dataset and the Pantheon dataset with similarly marginalized dependence on $H_0$ (and $M_B$). We do this to expand the redshift considered by the models. In a previous work \cite{Staicova:2021ntm}, we used similar approach in which, we integrated $H_0 r_d$ in the $\chi_2$ of the model, while here, we use them as one single quantity without modifying the $\chi_2$. In the marginalized version, we saw interesting possibility for some DE model to fit the data better than $\Lambda$CDM. We continue this investigation with new models and new approach in this paper.

Historically, the approach of using the combination $H_0 r_d$ is not new.  It has been used in \cite{LHuillier:2016mtc} with BAO and SN data to find consistency with the Planck 2015 best-fit $\Lambda$CDM cosmology, \cite{Shafieloo:2018gin} use the BAO data to fit the growth measurement, again finding consistency with the Planck 2015. \cite{Arendse:2019hev} use the Cepheids and the Tip of the Red Branch measurements to calibrate BAO and SN measurements and find significant tension in both $H_0$ and $r_d$, despite testing the   $\Lambda CDM$ and DE models ($EDE$, $wCDM$, pEDE). The implication is that  modifications of the physics after recombination fail to solve both tensions. The over-all conclusion is that the $H_0$ tension should not be considered separately from the $r_d$ measurement implied by it \cite{Knox:2019rjx}. In the current work, we choose a different approach. We repeat the analysis on $H_0 r_d$ used in earlier works but we also take the ratio $r_*/r_d$ as independent parameters. This means, we do not use the known analytical formulas for them, but instead we use MCMC to infer them. This avoids using explicit prior knowledge on the baryon load of the Universe. This way we avoid both the degeneracy on $H_0 r_d$ from the BAO data, but also we do not use as a hidden prior the Planck measurements. 

The plan of the work is as follows: Section \ref{sec:theory} formulates the relevant theory.  Section \ref{sec:method} describes the method. Section \ref{sec:res} shows the results and section \ref{sec:sum} summarizes the results.


\section{Theory}
\label{sec:theory}
A Friedmann - Lema\^itre - Robertson - Walker metric with the scale parameter $a = 1/(1+z)$ is considered, where $z$ is the redshift. The evolution of the universe for it is governed by the Friedmann equation which connects the equation of the state for $\Lambda$CDM background:
\begin{equation}
    E(z)^2 = \Omega_{r} (1+z)^4 + \Omega_{m} (1+z)^3 + \Omega_{k} (1+z)^2 + \Omega_{DE}(z),
    \label{eq:hzlcdm}
\end{equation}

\noindent where in standard $\Lambda$CDM, $\Omega_{DE}(z)\to \Omega_\Lambda$, with the expansion of the universe $E(z)= H(z)/H_0$, where $H(z) := \dot{a}/a$ is the Hubble parameter at redshift $z$ and $H_0$ is the Hubble parameter today. $\Omega_{r}$, $\Omega_{m}$, $\Omega_{DE}$ and $\Omega_{k}$ are the fractional densities of radiation, matter, dark energy and the spatial curvature at redshift $z=0$. We take into account the radiation energy density as $\Omega_r = 1 - \Omega_m - \Omega_{\Lambda} - \Omega_{k}$. The spatial curvature is expected to be zero for a flat Universe, $\Omega_k=0$ and we set it to zero because we focus on DE models.

We will consider a number of different DE models all of which will feature a dark energy component depending on $z$. This can be done with a generalization of the Chevallier-Polarski-Linder (CPL) parametrization \cite{Chevallier:2000qy,Linder:2005ne,Barger:2005sb}:
\begin{equation}
\Omega_{DE} \left(z\right) = \Omega_{\Lambda}  \exp\left[\int_0^{z} \frac{3(1+w(z')) dz'}{1+z'}\right]
\label{eq:ol}
\end{equation}
which allows for three possible models from which we will consider only the CPL: 
\begin{equation}
w(z)=w_0 + w_a \frac{z}{z+1}
\end{equation}
and $\Lambda$CDM is recovered for $w_0=-1, w_a=0$. 

To this parametrization we add another model, \cite{Barboza:2008rh, Escamilla-Rivera:2021boq} is Barboza-Alcaniz (BA) model with:

\begin{equation}
    w(z)=w_0+z\frac{1+z}{1+z^2}w_1
\end{equation}

This model is good for describing the whole universe history, because it does not diverge for $z \to -1$. It gives
\begin{equation}
    \Omega_{DE}=\Omega_{DE}(1+z)^{3(1+w_0)}{(1+z^2)}^{\frac{3w_1}{2}}.
\end{equation}

Next, we use the Low correlation model (LC) \cite{Wang:2008zh,  Escamilla-Rivera:2021boq} with
\begin{equation}
    w(z)=\frac{(-z+z_c)w_0+z(1+z_c)w_c}{(1+z)z_c}
\end{equation}

where $w_0=w(0)$ and $w_c=w(z_c)$ where $z_c$ is the redshift at which $w_0$ and $w_z$ are uncorrelated. The effective entry into the EOS is:
\begin{equation}
\Omega_{DE}=
\Omega_\Lambda(1+z)^{(3(1-2w_0+3wa))} e^{\frac{9(w_0-wa)z}{(1+z))}}
\end{equation}

where here we are replaced $w_c$ with $w_a$ for consistency with the other models.

The Jassal-Bagla-Padmanabhan (JBP) parametrization \cite{Jassal:2004ej, Motta:2021hvl}

\begin{equation}
    w(z)=w_0+w_1\frac{z}{(1+z)^2}
\end{equation}
which gives
\begin{equation}
    \Omega_{DE}=(1+z)^{3(1+w_0)}e^{\frac{3w_1z^2}{2(1+z)^2}}
\end{equation}

with $w_0=w(z=0)$ and  $w_1 = (dw/dz)_{|(z=0)}$.

Finally, we will also test the Feng-Shen-Li-Li parametrization \cite{Feng:2012gf, Motta:2021hvl} which is divergence-free for the entire history of the universe. It has two cases:
\begin{equation}
w(z)=w_0+w_1\frac{z}{1+z^2}\\
w(z)=w_0+w_1\frac{z^2}{1+z^2}
\end{equation}

with the final contribution to the EOS of each of them being:
\begin{equation}
\Omega_{DE} =(1+z)^{3(1+w_0)}e^{\pm \frac{3w_1}{2}\arctan(z)}(1+z^2)^{\frac{3w_1}{4}}(1+z)^{\mp \frac{3}{2}w_1}
\end{equation}
   Here, the plus and the minus case will be denoted as  FSLLIp and FSLLIm. 

The distance priors provide effective information of CMB power spectrum in two aspects: the acoustic scale $l_\textrm{A}$ characterizes the CMB temperature power spectrum in the transverse direction, leading to the variation of the peak spacing, and the "shift parameter" $R$ influences the CMB temperature spectrum along the line-of-sight direction, affecting the heights of the peaks. The popular definitions of the distance priors are \cite{Komatsu:2008hk}:
\begin{equation}
\begin{split}
l_\textrm{A} =(1+z_*)\frac{\pi D_\textrm{A}(z_*)}{r_s(z_*)} ,\\
R\equiv(1+z_*)\frac{D_\textrm{A}(z_*) \sqrt{\Omega_m } H_0}{c},
\end{split}
\label{la:Rz}
\end{equation}
where $z_*$ is the redshift at the photon decoupling epoch with $z_* \approx 1089$ according to the $Planck$ 2018 results \cite{Aghanim:2018eyx}. $r_*$ is the co-moving sound horizon at $z=z_*$. \cite{Chen:2018dbv} derives the distance priors in several different models using $Planck$ 2018 TT,TE,EE $+$ lowE which is the latest CMB data from the final full-mission Planck measurement \cite{Aghanim:2018eyx}.  {We use the correlation matrices given in Table 1 in \cite{Chen:2018dbv} to obtain the covariance matrices for $l_A$ and $R$  corresponding to each model.}


 {The angular diameter distance,  $D_\textrm{A}$,  needed for both the distance priors and the BAO points is given by:}
\begin{align}\label{}
D_\textrm{A}
%=\frac{D_\textrm{M}}{1+z}  &
=\frac{c}{(1+z) H_0 \sqrt{|\Omega_{k}|}  } \textrm{sinn}\left[|\Omega_{k}|^{1/2}\int_0^z \frac {dz'} {E(z')}\right]\ ,
\end{align}
where $\textrm{sinn}(x) \equiv \textrm{sin}(x)$, $x$, $\textrm{sinh}(x)$ for $\Omega_{k}<0$, $\Omega_{k}=0$, $\Omega_{k}>0$ respectively. We see that for the measured $D_A/r_d$ one can isolate the variable $b=c/(H_0 r_d)$.

Finally, for the SN and GRB datasets we will define the distance modulus $\mu(z)$ which is related to the luminosity distance ($d_L = D_A(1+z)^2$) through:
\begin{equation}
        \mu_B (z) - M_B = 5 \log_{10} \left[ d_L(z)\right] + 25  \,,
\label{eq:dist_mod_def}
\end{equation}
where $d_L$ is measured in units of Mpc, and $M_B$ is the absolute magnitude.



\section{Methods}
\label{sec:method}
In this paper we will use three different datasets, which we will treat differently. For the BAO dataset, the definition of the $\chi^2$ which we will minimize is the standard one, since we do not use covariance matrix for it.
\begin{equation}
\begin{split}
\chi^2_{BAO} = \sum_{i} \frac{\left(\vec{v}_{obs} - \vec{v}_{model}\right)^2}{\sigma^2},
\end{split}
\end{equation}
%where ${D_A}_{D}^{i}/r_d$
where $\vec{v}_{obs}$ is a vector of the observed points, $\vec{v}_{model}$ is the theoretical prediction of the model and $\sigma$ is the error of each measurement.

Additionally, we use the SN and the GRB datasets to constrain furthermore the models. For them, we use the following marginalized over $H_0$ and $M_B$ formula, taken from \cite{Staicova:2021ntm} so that we avoid setting priors on $H_0$ and $M_B$.

Following the approach used in (\cite{DiPietro:2002cz,Nesseris:2004wj,Perivolaropoulos:2004yr,Lazkoz:2005sp}),  the integrated $\chi^2$ is:
\begin{equation}
\tilde{\chi}^2_{SN,  GRB} = D-\frac{E^2}{F} + \ln\frac{F}{2\pi},
\end{equation}
for
\begin{subequations}
\begin{equation}
D = \sum_i \left( \Delta\mu \, C^{-1}_{cov} \, \Delta\mu^T \right)^2,
\end{equation}
\begin{equation}
E = \sum_i \left( \Delta\mu \, C^{-1}_{cov} \, E \right),
\end{equation}
\begin{equation}
F = \sum_i  C^{-1}_{cov}  ,
\end{equation}
\end{subequations}
where  $\mu_{}^{i}$ is the observed luminosity, $\sigma_i$ is its error, and the $d_L(z)$ is the luminosity distance, $\Delta\mu =\mu_{}^{i} - 5 \log_{10}\left[d_L(z_i)\right)$, $E$ is the unit matrix, and $C^{-1}_{cov}$ is the inverse covariance matrix of the dataset. For the GRB dataset, $C^{-1}\to 1/\sigma_i^2$ since there is no known covariance matrix for it.  For the Pantheon dataset, the total covariance is defined as $C_{cov}=D_{stat}+C_{sys}$, where $D_{stat}=\sigma_i^2$ comes from the measurement and $C_{sys}$ is provided separately \cite{Deng:2018jrp}. 
Note, in the so defined marginalized $\chi^2$ the values of $M$ and $H_0$ don't change the marginalized $\tilde{\chi}^2_{SN}$.

The final $\chi^2$ is:
$$\chi^2=\chi^2_{BAO}+\chi^2_{CMB}+\chi^2_{SN}+\chi^2_{GRB}.$$

\section{Datasets}
The dataset we are using is a collection of points from different BAO observations \cite{Chuang:2016uuz,Alam:2016hwk, Beutler:2016ixs,Blake:2012pj,Carvalho:2015ica,Seo:2012xy,Sridhar:2020czy,Abbott:2017wcz,Tamone:2020qrl,Zhu:2018edv,Hou:2020rse,Blomqvist:2019rah,Bourboux:2017cbm}, to which we add the CMB distant prior \cite{Chen:2018dbv} and the data from the binned Pantheon dataset, which contains $1048$ supernovae luminosity measurements in the redshift range $z\in (0.01,2.3)$ \cite{Pan-STARRS1:2017jku,Scolnic:2017caz} binned into 40 points. The GRB dataset \cite{Demianski:2016zxi} consists of 162 measurements in the range $z\in [0.03351,9.3]$. %???

To avoid the possible correlations in the BAO dataset, we use the methodology in \cite{Kazantzidis:2018rnb,Benisty:2020otr}. This method avoids the use of N-body mocks to find the covariance matrices due to systematic errors and replaces it with an evaluation of the effect of possible small correlation on the final result. We add to the covariance matrix for uncorrelated points $C_{ii} = \sigma_i^2$ symmetrically a number of randomly selected nondiagonal elements $C_{ij}$. Their magnitudes are set to $C_{ij} =0.5 \sigma_i \sigma_j$, where $\sigma_i \sigma_j$ are the published $1\sigma$ errors of the data points $i,j$. We introduce positive correlations in up to 6 pairs of randomly selected data points (more than $25\%$ of the data). Figure \ref{fig:checkCov} in the Appendix show the corner plots with different randomized points for all the models we employ in this article. From the plots, one can see that the effect from adding the correlations is below $10\%$ on average. This indicates that we can consider the chosen set of BAO points for effectively uncorrelated.

To run the inference, we use a Monte Carlo Markov Chain (MCMC) nested sampler to find the best fit. We use the open-source package $Polychord$ \cite{Handley:2015fda} with the $GetDist$ package \cite{Lewis:2019xzd} to present the results. 

The prior is a uniform distribution for all the quantities: $\Omega_{m} \in [0., 1.]$, $\Omega_{\Lambda}\in[0.;1 - \Omega_{m}]$, $\Omega_r\in[0.;1 - \Omega_{m} - \Omega_{\Lambda}]$, $c/ (H_0 r_d) \in [25, 35]$, $w_0 \in [-1.5, -0.5]$ and $w_a \in [-0.5,0.5]$. Since the distance prior is defined at the decoupling epoch ($z_*$) and the BAO -- at drag epoch ($z_d$), we parametrize the difference between $r_s(z_*)$ and $r_s(z_d)$ as $rat= r_*/r_d$, where the prior for the ratio is $rat \in [0.9, 1.1]$.


\section{Results}
\label{sec:res}


\begin{figure*}[ht]
\begin{tabularx}{\textwidth}{|c|c|}
\hline \\
\includegraphics[width=0.47\textwidth]{omb_new.pdf}&
\includegraphics[width=0.47\textwidth]{omb_newSN.pdf}
\\
\includegraphics[width=0.47\textwidth]{wwa_inv.pdf}&
\includegraphics[width=0.47\textwidth]{wwa_invSN.pdf}\\
\hline
\end{tabularx}
\caption{\it{The 2D contour plot for the different DE parametrizations for the BAO+CMB dataset to the left anf for the BAO+CMB+SN+GRB to the right. The upper panel shows the results for $\Omega_m$ vs. $c/(H_0 r_d)$ and the lower panel shows the results for $w$ and $w_a$. $\Lambda$CDM corresponds to $w_0 = -1$ and $w_a = 0$. The grey lines show the $1 \sigma$ and $2\sigma$ of $\Omega_m$ and $c/(H_0 r_d)$ as measured by Planck 2018, while on the bottom plot the gray cross shows where we recover $\Lambda$CDM.}}
\label{fig:bwwa_BAO1}
\end{figure*}




Fig \ref{fig:DDE} and \ref{fig:DDESN}
show the final values obtained by running MCMC on the selected priors, the numbers being in table \ref{results} in the Appendix where also the corner plots can be found. We see that the models differ seriously in their estimations for the physical quantities $c/(H_0 r_d), \Omega_m$ and $r_d/r_s$, probably due to the very wide prior imposed on $\Omega_m$. 

Since we avoid the degeneracy between $r_d$ and $H_0$ by considering the combined quantity $c/(H_0 r_d)$ this leads to an explicit correlation with $\Omega_m$ for some models and rather strict bounds on the error. The values of $\Omega_m$ closest to the ones published by Planck 2018 \cite{Aghanim:2018eyx} $\Omega_m = 0.315\pm 0.007$ are for the BA, JPB and FLIIp models for the BAO dataset and $\Lambda$CDM and BA for the BAO+SN+GRB. The rest significantly overestimate $\Omega_m$ . For the ratio $r_*/r_d$ Planck 2018 gives $0.98$, the closest models are BA, JPB and FLIIp models for the BAO dataset and $\Lambda$CDM, JPB and FSLIIp for the BAO+SN+GRB. For $c/(H_0 r_d)$, the Planck 2018 values is $30.26\pm0.06$. Here closest to this value models are $\Lambda$CDM, CPL and LC for the BAO dataset and $\Lambda$CDM, CPL and LC for the BAO+SN+GRB.

The DE parameters seem to be constrained to different level for the different models. As a whole, the trend to constrain much better $w_0$ than $w_a$ which we observed in \cite{Staicova:2021ntm} (and the referenced inside other works) is confirmed in this case as well. Notable exceptions are the BA and LC models, where the error of $w_a$ is much smaller. Fro them, however, the other parameters seem to be outside of the expected boundaries. The $\Lambda$CDM performs as expected under both datasets.
\begin{table}
	\begin{center}
		\begin{tabular}{|c|c|c|c|c|c|c|c|}
			\hline
			\multicolumn{8}{|c|}{BAO+CMB}\\
			\hline
			Model & AIC & $\Delta$AIC & BIC & $\Delta BIC$ & DIC & $\Delta$DIC & ln(BF) \\
			\hline
			$\Lambda$CDM & 22.0 &  & 24.5 &  & 16.8 &  &  \\
			\hline
			CPL & 25.7 & -3.7 & 29.9 & -5.4 & 16.5 & 0.3 & 0.6 \\
			\hline
			BA & 25.3 & -3.3 & 29.5 & -4.9 & 16.2 & 0.65 & -5.3 \\
			\hline
			LC & 56.0 & -33.9 & 60.2 & -35.6 & 51.1 & -34.3 & 38.5 \\
			\hline
			JPB & 27.8 & -5.8 & 31.9 & -7.4 & 18.6 & -1.8 & -3.5 \\
			\hline
			FSLIIp & 27.1 & -5.1 & 31.3 & -6.9 & 17.9 & -1.1 & -3.8 \\
			\hline
			\multicolumn{8}{|c|}{BAO + CMB + SN + GRB}\\
			\hline
			$\Lambda$CDM & 228.1 &  & 238.3 &  & 222.7 &  &  \\
			\hline
			CPL & 229.2 & -1.1 & 246.1 & -7.8 & 219.9 & 2.8 & -1.2 \\
			\hline
			BA & 229.0 & -0.9 & 246.0 & -7.8 & 219.8 & 2.9 & -5.9 \\
			\hline
			LC & 436.8 & -208.7 & 453.7 & -215.5 & 427.6 & -204.9 & 208.9 \\
			\hline
			JPB & 232.2 & -4.1 & 249.2 & -10.9 & 222.9 & -0.2 & -4.0 \\
			\hline
			FSLIIp & 231.1 & -2.9 & 248.0 & -9.7 & 221.9 & 0.9 & -3.7 \\
			\hline
		\end{tabular}
	\end{center}
	\caption{{ Selection Criteria of different models in a comparison to the $\Lambda$CDM model for the BAO dataset and the BAO + SN + GRB dataset. }}
\label{stats_BAO}
\end{table}
To compare the different models, we use well-known statistical measures. The results can be seen in the Table \ref{stats_BAO}. In it we publish four selection criteria:  Akaike information criterion (AIC), Bayesian information criterion (BIC), deviance information criterion (DIC) and the Bayes Factor (BF). Since  for small datasets both AIC and BIC are dominated by the number of parameters in the model (which are 3 for $\Lambda$CDM, 5 for the DE models), we emphasize here on the DIC and the BF which rely on the numerically evaluated likelihood and evidence, making them more unbiased. The DIC criterion, just like the AIC, selects the best model to be the one with the minimal value of the DIC measurement. The reference table we use for DIC is:
$\Delta DIC>10$ shows strong support for the model with lower DIC , $\Delta DIC = 5-10$ shows substantial support for the model with lower DIC, $\Delta DIC<5$ gives ambiguous support for the model with lower DIC. Here we use the logarithmic scale for the BF, for which $ln(BF)>1$ shows support for the base model ($\Lambda$CDM), while $ln(BF)<-1$ for the other hypothesis. $|ln(BF)|<1$ shows inconclusive result.

From the table \ref{stats_BAO}  we see that the AIC and BIC for all models show a preference for $\Lambda$CDM. For the DIC criterion, we see a slight possibility for a preference for other models in the case of the CPL and BA models for both tested datasets. For the BF, we see that there is some possible preference for BA, JPN and FSLIIp for the BAO+CMB case and for CPL and BA, JPN and FSLIIp in the BAO+CMB+SN+GRB case. The results of the LC model show that it is underfitting the data (from the $\chi^2/dof\sim 2$) and the statistics for it is not reliable. This demonstrates another benefit from performing the statistical analysis.




\begin{figure}
 	\centering
\includegraphics[width=0.495\textwidth]{b_compareGRBBAO.pdf}
\includegraphics[width=0.495\textwidth]{om_compareGRBBAO.pdf}

\caption{{ {The final values of the $c/(H_0 r_d)$ from different DE models, compared to the values from Planck for the BAO+CMB+SN+GRB dataset. The smaller, darker, errorbox are for BAO+CMB, the lighter, bigger errorbox -- for SN+GRB}}}
 	\label{fig:H0rdvalGRB}
\end{figure}
The preference for BA and LC we observe matches the results of \cite{Escamilla-Rivera:2021boq} where it has been observed in a dataset consisting of  SN and cosmic chronometers and gravitational waves.

The BAO dataset we use combines the $H_0$ and the $r_d$ in one quantity. Therefore we estimate the new variable $c/(H_0 r_d) \sim 30$. Fig \ref{fig:H0rdvalGRB} shows the values of the $c/(H_0 r_d)$ for different models vs. the result from Planck 2018: $30.24 \pm 0.08$. For comparison, the most recent local measurement by SH0ES is: $30.19 \pm 0.53$, corresponding to $H_0=73.01 \pm 0.99 km s^{-1} Mpc^{-1}$ \cite{Riess:2022mme}. We do not put it on the plot, because the $r_d$ used to obtain it is indirect result from inference on a H0LiCOW+SN+BAO+SH0ES dataset \cite{Arendse:2019hev}. It is, however, clearly very close to the Planck value, as expected. 

On Fig. \ref{fig:H0rdvalGRB} we superimpose the BAO+CMB-only result with the BAO+CMB+SN+GRB one. This figure enables us to visually track the tension between the Planck 2018 results and the datasets we use, which are mostly local Universe ones (except for the 2 CMB points). We see that the tension is now between $c/(H_0 r_d)$ and $\Omega_m$. The models whose bounds cross with the Planck 2018 one for $c/(H_0 r_d)$ are $\Lambda$CDM, CPL and LC for BAO+CMB and only LC for the BAO+CMB+SN+GRB dataset. For $\Omega_m$ the models that enter the interval are all but $\Lambda$CDM and $CPL$ for the BAO+CMB dataset and  $\Lambda$CDM, BA, LC for the BAO+CMB+SN+GRB dataset. We see that the inclusion of the new datasets decrease the number of models satisfying the constraints. The only model that is not in tension is LC because of its huge error. Notably, in this approach, $\Lambda$CDM while satisfying the bounds for $c/(H_0 r_d)$, it does not satisfy them for $\Omega_m$. % ($\Omega_m^{Pl18}=0.3153 \pm 0.0073$) 

From the plot one can see that in general adding the new datasets decrease the errors, but they do not move the mean values in the same direction and the overall effect is not very big. This may be due to unknown errors in the SN+GRB dataset or to the fact that this dataset is not sensitive towards the combined variable $c/(H_0 r_d)$ since we have marginalized over $H_0$ so that we do not have to impose a prior on $r_d$. Because of this, the only effect the SN+GRB dataset has on the combined variable is indirect, trough $\Omega_m$ and the other parameters. It could also point to some inconsistency in the $\mu(M_B)$ relation such as the ones considered in \cite{Benisty:2022psx,Ferramacho:2008ap,Linden:2009vh,Tutusaus:2017ibk,DiValentino:2020evt,Perivolaropoulos:2022khd}) questioning the assumption that $M_B=const$.





\section{Discussion}
\label{sec:sum}
This paper uses the combination $H_0. r_d$ to avoid the degeneracy between $H_0$ and $r_d$ which has plagued the use of BAO measurements and could be part of the resolution of cosmological tensions. The use of a combined parameter avoids imposing separate priors on $H_0$ and $r_d$ and thus it avoids additional assumptions on them. We use points from the late universe, the BAO dataset ($z<2.4$)), few points from the early universe -- the CMB distant priors, ($z \approx 1089$), to which we add SN data and GRB datasets, properly marginalized, to make a statistical comparison between different DE models. 

The results show that the tension is now between the new parameter $c/(H_0 r_d)$ and $\Omega_m$ -- the only model that fits in the constraints set by Planck 2018 is LC, which comes with the biggest error. For the rest of the models one of the two parameters do not fit the constraints, even if some of them reduce somewhat the tension. Statistically there is a preference for the $\Lambda$CDM model over the DE models in most cases. It is worth noting that there are strong evidence in support of $\Lambda$CDM compared to all other models only when using AIC and BIC, while from DIC and BF, the support is not substantial and it even slightly favors other models. This result raises the question of the use of the different statistical measures when comparing DE models, and also it opens the possibility that a better DE model may eventually help reducing both the $H_0$ tension and the $r_d$ tension. 

Another interesting point is that for some models, the known impossibility to constrain $w_a$ is eliminated and $w_a$ has very tight bounds. These models, LC and BA and somewhat FSLIIp, show interesting new possibilities for DE models. Furthermore, the choice of datasets and models make explicit the degeneracy between $H_0.r_d$ and $\Omega_m$ emphasizing on the need to find a way to disentangle the 3 quantities -- $H_0, r_d$ and $\Omega_m$ -- if we are to understand the cosmological tensions. The results show that adding the SN and GRB datasets decrease the errors on the constrained parameters, but they do not move them in the same direction for each model. We see that combining different datasets and different marginalization techniques, along with the use of statistical measures, is a promising tool to study new cosmological models.  



\section*{Acknowledgements}
D.S. thanks David Benisty for the useful comments and discussions. D.S. is thankful to Bulgarian National Science Fund for support via research grants KP-06-N58/5. %We have received partial support from European COST actions CA15117 and CA18108.


\section*{Data availability}
All the data we use in this paper is taken from the corresponding citations and available to use.







%\bibliographystyle{apsrev4-1}
\bibliography{ref}

\newpage
%%%%%%%%%%%%%%%%%%%%%%%%%%%%%%%%%%%%%%%%%%%%%%%%%%

%%%%%%%%%%%%%%%%% APPENDICES %%%%%%%%%%%%%%%%%%%%%

\appendix

\section{Some extra material}


\begin{table*}
\scalebox{1.1}{
\begin{tabular}{ccccccc}
\hline\hline

$z$   & $D_A/{r_d}$ & Error & year  & Survey &  Ref. \\ \hline\hline
$0.11$  & $2.607$ & $0.138$&  $2021$ & SDSS blue galaxies & \cite{deCarvalho:2021azj}\\
$0.24$   & $5.594$& $0.305$&  $2016$ &BOSS-DR12 RSD of LOWZ and CMASS & \cite{BOSS:2016goe}\\
$0.32$    &  $6.636$  &  $0.11$   &  $2016$  &  SDSS-DR9+DR10+DR11+DR12+covariance  & \cite{BOSS:2016wmc} \\
$0.38$    &  $7.389$  &  $0.122$  &  $2019$  &  BOSS-DR12 power spectrum & \cite{BOSS:2016hvq}\\
$0.44$    &  $8.19$  &  $0.77 $  &   $2012$  &  WiggleZ (galaxy clustering)  & \cite{Blake:2012pj}\\
$0.51$   &  $7.893$  &  $0.279$  &   $2015$  &  BOSS-DR10 ang. gal. clust. & \cite{Carvalho:2015ica}	\\
$0.54$    &  $9.212$  &  $0.41$    &  $2012$  &  SDSS-III DR8 (luminous galaxies) & \cite{ Seo:2012xy}\\
$0.6$    &  $9.37$  &  $0.65$   &  $2012$  &   WiggleZ (galaxy clustering) & \cite{Blake:2012pj}\\
$0.697$    &  $10.18$  &  $0.52$   &  $2020$  &  DECals DR8 (LRG)  & \cite{Sridhar:2020czy}\\
$0.73$    &  $10.42$  &  $0.73$  &   $2012$  &  Wiggle (galaxy clustering) & \cite{Blake:2012pj}\\
$0.81$    &  $10.75$  &  $0.43$   &  $2017$  &  DES Year1  (galaxy clustering) & \cite{DES:2017rfo}\\
$0.85$    &  $10.76$  &  $  0.54$   &  $ 2020$  &  eBOSS DR16 ELG & \cite{Tamone:2020qrl}\\
$0.874$    &  $11.41$  &  $0.74$    &  $2020$  &  DECals DR8 (LRG) & \cite{Sridhar:2020czy}\\
$1.00$    &  $11.521$  &  $1.032$    &  $2019$  & eBOSS DR14 quasar clustering & \cite{Zhu:2018edv}\\
$1.480$    &  $12.18$  &  $0.32$  &  $2020$  &  eBOSS DR16 BAO+RSD consensus  & \cite{Hou:2020rse}\\
$2.00$    &  $12.011$  &  $0.562$  &   $2019$  &  eBOSS DR14 quasars clustering & \cite{Zhu:2018edv}\\
$2.35$    &  $10.83$  &  $0.54$  &  $2019$  &  BOSS DR14 Lya and quasars & \cite{Blomqvist:2019rah}  \\
$2.4$    &  $10.5$  &  $0.34$  & 2017 & SDSS-III/DR12 & \cite{duMasdesBourboux:2017mrl}\\


\hline\hline
\end{tabular}
}
\caption{{The uncorrelated dataset used in this paper. For each redshift the table presents the parameter, the mean value, and the corresponding error bar. The reference and the collaboration is also reported.}}
\label{tab:data}
\end{table*}


\begin{table}
\centering
\renewcommand{\arraystretch}{1.5}
%\tiny
\begin{tabular}{|c|c|cc|}
\hline\hline
  $\Lambda$CDM & $Planck~ \textrm{TT,TE,EE}+\textrm{lowE} $ & $R$ & $l_\textrm{A}$ \\
\hline
$R$  & $1.7502\pm0.0046$ & $1.0$&    $0.46$ \\
$l_\textrm{A}$  & $301.471^{+0.089}_{-0.090}$ & $0.46$&    $1.0$  \\
\hline\hline
  $w$CDM & $Planck~ \textrm{TT,TE,EE}+\textrm{lowE} $ & $R$ & $l_\textrm{A}$  \\
\hline
$R$ & $1.7493^{+0.0046}_{-0.0047}$  & $1.0$&    $0.47$\\
$l_\textrm{A}$ & $301.462^{+0.089}_{-0.090}$ & $0.47$&    $1.0$ \\
\hline\hline
   $\Omega_k \Lambda$CDM & $Planck~ \textrm{TT,TE,EE}+\textrm{lowE} $ & $R$ & $l_\textrm{A}$\\
\hline
$R$ & $1.7429\pm0.0051$ & $1.0$ & $0.54$  \\
$l_\textrm{A}$ & $301.409\pm0.091$ & $0.54$&   $1.0$\\
\hline\hline
\end{tabular}
\caption{The 68 $\% $ C.L. limits for $R$, $l_A$, in different cosmological models and their correlation matrix for from Planck $2018$ $TT,TE,EE+lowE$, see the text for details % \cite{Aghanim:2018eyx}. The data is taken from the table 1 of \cite{Chen:2018dbv}.
}
\label{distancePr}
\end{table}
%this error seems to be related to the citations, I don't know why.


\begin{figure*}
 	\centering
a) \includegraphics[width=0.25\textwidth]{LCDM_cov.pdf}
b) \includegraphics[width=0.25\textwidth]{CPL_cov.pdf}
c) \includegraphics[width=0.25\textwidth]{BA_cov.pdf}\\
d) \includegraphics[width=0.25\textwidth]{JBP_cov.pdf}
e) \includegraphics[width=0.25\textwidth]{FSLIIp_cov.pdf}
 \caption{{The covariance test plot for the considered models: a) $\Lambda$CDM, b) $CPL$, c) $BA$, d) $JPB$, e) $FSLIIp$ model.}}
\label{fig:checkCov}
\end{figure*}








\begin{figure*}
 	\centering
\includegraphics[width=0.45\textwidth]{bwwa_inv_all.pdf}
\includegraphics[width=0.45\textwidth]{bwwa_inv_allSN.pdf}
	\caption{The posterior distribution for $c/(H_0 r_d)$, $\Omega_m$, $r_*/r_d$ and $w_0,w_a$ for different parametrization of DE for the BAO + CMB dataset to the left and for the BAO+CMB+SN+GRB to the right. }
\label{results_BAO}
\end{figure*}


\begin{table*}
 	\begin{tabular}{|c|c|c|c|c|c|}
 	\hline
			\multicolumn{6}{|c|}{BAO + CMB}\\
			\hline
			Model & $c/(H_0 r_d)$ & $\Omega_m$ & $r_*/r_d$ & w & $w_a$ \\
			\hline
			$\Lambda$CDM & $29.96\pm 0.3$ & $0.36\pm 0.02$ & $0.92\pm 0.01$ & -1.000 & 0.000 \\
			\hline
			CPL & $29.42\pm 0.85$ & $0.35\pm 0.02$ & $0.91\pm 0.01$ & $-1.14\pm 0.19$ & $0.13\pm 0.31$ \\
			\hline
			BA & $27.58\pm 1.74$ & $0.29\pm 0.04$ & $0.93\pm 0.02$ & $-1.06\pm 0.11$ & $0.35\pm 0.12$ \\
			\hline
			LC & $30.64\pm 2.87$ & $0.38\pm 0.07$ & $0.901\pm 0.0009$ & $-0.5082\pm 0.0072$ & $-0.4979\pm 0.0018$ \\
			\hline
			JPB & $28.03\pm 1.76$ & $0.29\pm 0.04$ & $0.94\pm 0.02$ & $-0.86\pm 0.09$ & $0.08\pm 0.31$ \\
			\hline
			FSLIIp & $27.97\pm 1.81$ & $0.29\pm 0.04$ & $0.94\pm 0.02$ & $-0.92\pm 0.1$ & $0.22\pm 0.24$ \\
			\hline
		    \multicolumn{6}{|c|}{BAO + CMB + SN + GRB}\\
			\hline
			$\Lambda$CDM & $29.51\pm 0.24$ & $0.32\pm 0.01$ & $0.95\pm 0.01$ & -1.000 & 0.000 \\
			\hline
			CPL & $29.27\pm 0.25$ & $0.34\pm 0.01$ & $0.91\pm 0.01$ & $-1.15\pm 0.06$ & $0.09\pm 0.31$ \\
			\hline
			BA & $27.44\pm 1.36$ & $0.28\pm 0.03$ & $0.94\pm 0.01$ & $-1.13\pm 0.04$ & $0.37\pm 0.1$ \\
			\hline
			LC & $31.11\pm 2.86$ & $0.39\pm 0.07$ & $0.9009\pm 0.0007$ & $-0.55\pm 0.02$ & $-0.4992\pm 0.0007$ \\
			\hline
			JPB & $27.58\pm 1.38$ & $0.27\pm 0.03$ & $0.9657\pm 0.0084$ & $-1.02\pm 0.04$ & $0.22\pm 0.23$ \\
			\hline
			FSLIIp & $27.37\pm 1.47$ & $0.27\pm 0.03$ & $0.9614\pm 0.0082$ & $-1.06\pm 0.04$ & $0.32\pm 0.14$ \\
			\hline
		\end{tabular}
		\caption{The posterior values for $c/(H_0 r_d)$, $\Omega_m$, $r_*/r_d$ and $w_0,w_a$ for different parametrization of DE for the BAO + CMB dataset (top) and for the BAO+CMB+SN+GRB (bottom).}
\label{results}
\end{table*}


\end{document}
%
% ****** End of file apssamp.tex ******
