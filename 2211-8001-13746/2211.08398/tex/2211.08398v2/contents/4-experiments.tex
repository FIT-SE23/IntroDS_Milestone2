% experiments that we plan to do
% 1. three kinds of students
%   - small bev query + small backbone + low resolution
%   - small bev query + small backbone + large resolution
%   - large bev query + small backbone + large resolution

% 2. compared works
%   - some other kd methods, including Hinton KD, FitNet, non-local KD, Attention Guided KD, CRD, RelationKD, SimilairyKD and so on
%   firstly conduct these experiments on the first kind of students. do more on the second and third kinds if possible

% 3. ablation study
%   - only temporal
%   - only spatial
%   - only weight inhering
%   - spatial + temporal
%   - spatial + temporal + weight inheriting 
%   
\vspace{0.4cm}
\begin{table}
    \centering
      \caption{Student-teacher settings in our experiments. Please refer to the supplementary material for more details.}
    \vspace{-0.3cm}
      \resizebox{\linewidth}{!}{
    \begin{tabular}{c|clccccc}
    \toprule
        Model& FPS&Params&2D Backbone & BEV Query & Decoder Depth\\
        \midrule
        Student-1& 14.5&40.45&ResNet50&(150, 150)& 3 \\
        Teacher-1& 10.2&56.57&ResNet101& (150, 150) & 3\\ \midrule
        Student-2& 14.5&40.45&ResNet50&(150, 150)& 3 \\
        Teacher-2& 5.3&201.20&ResNeXt-Large& (150, 150) & 3\\ \midrule
        Student-3& 5.2&47.56&ResNet50&(200, 200)& 6 \\
        Teacher-3& 3.5&65.93&ResNet101& (200, 200) & 6\\ \bottomrule


    \end{tabular}}

    \label{tab:stu_tea}
    \vspace{-1em}
\end{table}


\begin{table*}[t]
    \caption{Comparison with other knowledge distillation methods on the nuScenes~\cite{caesar2020nuscenes} dataset with BEVFormer. Note that a higher mAP and NDS, as well as a lower ATE, ASE, AOE, and AAE indicate better performance. Params: the number of parameters (M). FPS: Frame per second. FPS is measured with one A100 GPU. Please refer to ~\cite{caesar2020nuscenes} for detailed metrics definitions.\label{tab:nuscene}}
    \vspace{-0.25cm}
    \begin{center}
  \resizebox{\linewidth}{!}{\begin{tabular}{lccl|cccccccccccccc}
    \toprule
        Backbone&FPS&Params& KD Method  &mAP($\uparrow$)&NDS($\uparrow$)&mATE($\downarrow$)&mASE($\downarrow$)&mAOE($\downarrow$)&mAVE($\downarrow$)&mAAE($\downarrow$)\\
        \midrule
        ResNet101&10.2&56.57& Teacher w/o KD&36.31&47.49&69.21&28.16&46.08&43.87&19.32& \\
        \midrule
        \multirow{10}{*}{ResNet50}&\multirow{10}{*}{14.5}&\multirow{10}{*}{40.45}& Student w/o KD&33.56&44.61&71.41&28.65&54.17&46.44&21.03 \\
        &&& + Hinton~\emph{et al.}&        33.57&45.23&71.17&28.50&49.04&46.52&20.33 \\
        &&& + Zagoruyko~\emph{et al.}& 
33.68&45.69&70.13&\textbf{27.74}&47.87&45.45&20.26\\
        &&& + Heo~\emph{et al.}&         33.87&45.82&69.92&27.79&47.78&45.55&20.09\\
&&& + Park~\emph{et al.}&33.77&45.87&70.88&27.78&48.18&43.47&19.83 \\
        &&& + Pung~\emph{et al.}&34.01&45.36&71.21&28.06&50.49&45.77&20.88 \\
        &&& + Ahn~\emph{et al.}& 34.11&46.36&70.69&28.02&46.16&42.09&20.04\\
&&& + Zhang~\emph{et al.}& 34.25&46.34&70.84&28.44&47.06&\textbf{41.68}&19.82\\

&&& + Guo~\emph{et al.}& 34.10&46.22&70.39&28.39&46.75&42.52&20.22\\
&&& \textbf{+ Ours}\cellcolor{linecolor}&\textbf{34.91}\cellcolor{linecolor}&\textbf{46.87}\cellcolor{linecolor}&\textbf{69.77}\cellcolor{linecolor}&28.07\cellcolor{linecolor}&\textbf{46.31}\cellcolor{linecolor}&42.23\cellcolor{linecolor}&\textbf{19.43}\cellcolor{linecolor} \\
         \midrule
        ResNeXt-Large&5.3&201.2& Teacher w/o KD&37.69&46.67&70.44&28.52&56.89&45.81&20.12 \\
        \midrule
        \multirow{10}{*}{ResNet50}& \multirow{10}{*}{14.5}&\multirow{10}{*}{40.45} & Student w/o KD&33.56&44.61&71.41&28.65&54.17&46.44&21.03 \\
        &&& + Hinton~\emph{et al.}&33.84&45.68&72.72&28.16&46.54&44.50&20.48\\
        &&& + Zagoruyko~\emph{et al.}&34.10&46.26&70.99&28.24&46.12&42.45&20.05\\
        &&& + Heo~\emph{et al.}& 34.30&46.50&70.36&27.94&\textbf{44.78}&43.06&20.39\\
        &&& + Park~\emph{et al.}&33.98&46.40&71.82&28.07&45.84&39.86&20.24 \\
        &&& + Pung~\emph{et al.}&34.23&46.23&70.05&28.32&47.33&43.13&20.04 \\
        &&& + Ahn~\emph{et al.}& 34.16&46.25&70.37&28.08&46.43&42.73&20.66\\
        &&& + Zhang~\emph{et al.}&34.56&46.61&70.11&28.01&46.00&42.39&20.14 \\
        &&& + Guo~\emph{et al.}&34.35&46.06&69.92&\textbf{27.79}&47.78&45.55&20.09 \\
        &&& \textbf{+ Ours}\cellcolor{linecolor}& \textbf{35.58}\cellcolor{linecolor}&\textbf{47.39}\cellcolor{linecolor}&\textbf{68.97}\cellcolor{linecolor}&28.25\cellcolor{linecolor}&48.06\cellcolor{linecolor}&\textbf{39.79}\cellcolor{linecolor}&\textbf{18.93}\cellcolor{linecolor}\\
        \midrule
        ResNet101&3.5&65.93& Teacher w/o KD&41.01&51.88&67.45&27.36&34.92&37.57&18.97 \\
        \midrule
        \multirow{10}{*}{ResNet50}&\multirow{10}{*}{5.2}&\multirow{10}{*}{47.56}& Student w/o KD&35.77&46.74&73.61&28.26&45.85&43.79&19.94 \\
        &&& + Hinton~\emph{et al.}&35.89&46.93&73.45&\textbf{28.02}&45.46&43.66&19.58\\
        &&& + Zagoruyko~\emph{et al.}&35.98&46.98&73.30&28.22&45.32&43.68&19.60\\ 
        &&& + Heo~\emph{et al.}&36.23&47.16&73.09&28.18&45.28&43.34&19.69\\ 
        &&& + Park~\emph{et al.}& 36.30&47.18&72.94&28.17&45.48&43.43&19.64\\
        &&& + Pung~\emph{et al.}&36.42&47.26&72.96&28.23&45.48&43.37&19.51\\ 
        &&& + Ahn~\emph{et al.}& 36.38&47.20&73.02&28.25&45.51&43.50&19.60\\
        &&& + Zhang~\emph{et al.}&36.64&47.38&73.12&28.15&\textbf{45.28}&43.11&19.53 \\
        &&& + Guo~\emph{et al.}&36.77&47.40&73.14&28.25&45.34&43.43&19.74\\ 
        &&& \textbf{+ Ours}\cellcolor{linecolor}&\textbf{38.88}\cellcolor{linecolor}&\textbf{48.52}\cellcolor{linecolor}&\textbf{71.53}\cellcolor{linecolor}&28.24\cellcolor{linecolor}&47.34\cellcolor{linecolor}&\textbf{42.91}\cellcolor{linecolor}&\textbf{19.17}\cellcolor{linecolor} \\
         \bottomrule 
    \end{tabular}}
    \vspace{-0.2cm}
    \end{center}
\end{table*}

\vspace{-0.4cm}
\section{Experiment}
\vspace{-0.1cm}
\label{sec:experiment}
\subsection{Experimental Setting}
\paragraph{Dataset:} The nuScenes dataset is a large-scale autonomous driving dataset, which has 3D bounding boxes for 1000 scenes collected from six cameras~\cite{caesar2020nuscenes}. The scenes are officially split into 700, 150, and 150 scenes for training, validation, and testing, respectively, including 1.4 million annotated 3D bounding boxes belonging to 10 classes.

%\vspace{-0.40cm}
%\paragraph{Evaluation Metric:} Seven kinds of performance metrics are utilized in our experiments, including mean Average Precision (mAP), Average Translation Error (ATE), Average Scale Error (ASE), Average Orientation Error (AOE), Average Velocity Error (AVE), Average Attribute Error (AAE) and nuScenes Detection Score (NDS). Note that a higher mAP and NDS, as well as a lower ATE, ASE, AOE, and AAE indicate better performance. The reported FPS is measured on a single NVIDIA A100 GPU.

\vspace{-0.40cm}
\paragraph{Model Architecture:} BEVFormer models of different sizes are utilized as the student and teacher detectors in our experiments. As shown in Table~\ref{tab:stu_tea}, We mainly reduce the model size by using fewer BEV queries and smaller 2D backbones. Please refer to the supplementary material for more details on the models and training settings.


\vspace{-0.40cm}
\paragraph{Comparison Method:}
Eight previous knowledge distillation methods have been utilized for comparison. Six of the comparison methods are feature-based knowledge distillation for general vision tasks, including methods from Hinton~\emph{et al.}~\cite{distill_hinton}, Zagoruyko~\emph{et al.}~\cite{attentiondistillation}, Heo~\emph{et al.}~\cite{kd_comprehensive}, Park~\emph{et~al.}~\cite{relational_kd}, Pung~\emph{et al.}~\cite{relational_kd2} and Ahn~\emph{et al.}~\cite{kd_variational}. 
We adopt them to multi-view detection by using them to distill the features of both the 2D backbone and self-attention layers. Besides, two of the comparison methods are proposed for 2D object detection, including methods from Zhang~\emph{et~al}~\cite{detectiondistillation} and Guo~\emph{et~al.}~\cite{DBLP:conf/cvpr/Guo00W0X021}. Please refer to the supplementary material on our detailed implementation.

% fitent, attention kd, non-local kd, attetention-guided kd, hinton kd, 

\subsection{Experiment Results}
\vspace{-0.1cm}
Experimental results of our method and eight previous knowledge distillation methods in three different student-teacher settings are shown in Table~\ref{tab:nuscene}. It is observed that: (i)~On average, 2.16 mAP and 2.27 NDS improvements can be observed with our method in the three student-teacher settings, which are 1.26 mAP and 0.80 NDS higher than the second-best knowledge distillation methods. (ii) In all three student-teacher settings, our method leads to performance improvements in terms of most of the performance metrics, including mAP, NDS, mATE, mATE, mASE, mAOE, mAVE, and mAAE, indicating that our method benefits students in estimating the translation, scale, orientation, velocity and attributes of the objects. (iii) The performance of our method in different categories is shown in Table~\ref{tab:class}. It is observed that our method leads to consistent improvements in most of the categories. (iv) The first student achieves 0.67 higher mAP than the second student, indicating that our method benefits from a strong teacher.
%(iv) With our method, the student-1 achieves 0.67 higher mAP than the student-2, indicating our method benefits from a stronger teacher
%(iv) Although the students in the first two groups have the same architecture, the student in the second group achieves 0.67 and 0.52 higher improvements on mAP and NDS, respectively, indicating a good teacher can further improve the effectiveness of our method.

%0.67 0.52

\begin{table}[t]
  \caption{Average precision in different classes on nuScenes. ``KD'' indicates whether our method is applied. Experiments of the three groups are conducted with student-teacher settings in Table~\ref{tab:stu_tea}. \label{tab:class}}
  \vspace{-0.3cm}
  \begin{center}
    \resizebox{\linewidth}{!}{\setlength{\tabcolsep}{0.50mm}{\begin{tabular}{c|ccccccccccccccccc}
      \toprule
         KD  &~~~Car~~~&Truck&~~~Bus~~~&Trailer&Con.Veh.& Pedest.&Motor.&Bicycle&Barrier&Tra.Cone\\
           \midrule
          $\times$&54.3&26.0&32.3&8.9&7.4&41.8&31.8&28.2&53.8&51.0\\
          $\checkmark$&55.1&27.4&34.2&10.1&6.8&43.4&34.2&31.1&53.9&52.9\\
          \midrule[0.1pt]
          $\times$&54.3&26.0&32.3&8.9&7.4&41.8&31.8&28.2&53.8&51.0\\
          $\checkmark$&56.5&29.3&37.5&13.3&10.3&45.6&34.4&34.4&43.8&50.8\\
          \midrule[0.1pt]
          $\times$&55.8&28.7&35.0&9.7&6.5&46.6&37.5&37.3&54.6&46.0\\
          $\checkmark$&58.7&33.1&36.2&12.8&10.1&47.4&40.4&40.8&57.6&51.9\\
           \bottomrule 
      \end{tabular}}}
      \end{center}
      \vspace{-1.3em}
  \end{table}



\begin{table*}[h]
  \caption{Ablation study of different modules in our method. ``Spatial-Temporal'', ``BEV Response'', ``Weight-Inherit'' indicates spatial-temporal distillation, BEV response distillation, and the weight-inheriting scheme, respectively.
  \label{tab:ablation}}
      \vspace{-0.25cm}
%  \vspace{-0.5cm}
  \begin{center}
  \small
\resizebox{\linewidth}{!}{\begin{tabular}{cccccccccccccccccc}
  \toprule
       \multicolumn{3}{c}{Modules in Our Method} &  &\multirow{2}{*}{mAP($\uparrow$)}&\multirow{2}{*}{NDS($\uparrow$)}&\multirow{2}{*}{mATE($\downarrow$)}&\multirow{2}{*}{mASE($\downarrow$)}&\multirow{2}{*}{mAOE($\downarrow$)}&\multirow{2}{*}{mAVE($\downarrow$)}&\multirow{2}{*}{mAAE($\downarrow$)}\\
       \cmidrule{1-3}
       {{\small Spatial-Temporal}} &{{\small BEV Response}}& {{\small Weight-Inherit}} \\
       \midrule
      $\times$&$\times$&$\times$&&33.56&44.61&71.41&28.65&54.17&46.44&21.03\\
      $\times$&$\times$&$\checkmark$&&34.52&46.60&70.97&28.05&46.00&41.74&19.86\\
      $\times$&$\checkmark$&$\checkmark$&&34.99&47.17&70.22&27.75&46.58&39.47&19.24\\
      $\checkmark$&$\times$&$\checkmark$&&34.91&47.02&70.62&27.90&47.26&39.60&18.93 \\
      $\checkmark$&$\checkmark$&$\times$&&35.00&46.68&71.07&28.43&46.09&42.46&20.19& \\
      $\checkmark$&$\checkmark$&$\checkmark$&& 35.58&47.39&68.97&28.25&48.06&39.79&18.93\\
       \bottomrule 
  \end{tabular}}
  \end{center}
\end{table*}


%\begin{table}[h]
 %   \caption{Ablation study on different schemes on weight inheriting in the first student-teacher setting. Experiments are conducted without other knowledge distillation loss. \label{tab:ablation_weight}}
  %  \begin{center}
  %\resizebox{\linewidth}{!}{\begin{tabular}{lcccccccccccccccc}
   % \toprule
    %   Method &mAP($\uparrow$)&NDS($\uparrow$)&mATE($\downarrow$)&mASE($\downarrow$)&mAOE($\downarrow$)&mAVE($\downarrow$)&mAAE($\downarrow$)\\
     %    \midrule
  %     Student w/o KD &\\
   %    Init &\\
    %   Loss &\\
     %  Init + Loss &\\
      % Init + Freeze (Ours) &\\
       %  \bottomrule 
%    \end{tabular}}
 %   \end{center}
%\end{table}



