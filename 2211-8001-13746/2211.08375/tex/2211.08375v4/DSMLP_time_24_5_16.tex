% 工作日志
% 2024-03-22 修改了 Lemma 3.4 证明中的一些符号错误; Lemma 3.1 中增加了
% independent of $ \tau $ 的表述
% 2024-04-12 检查了一遍证明
% 2024-05-08 Theorem 3.2 证明中关于 Upsilon_1, Upsilon_2 有严重错误
% 2024-05-16 做了比较大的修改,目前是一个较为满意的版本
% 2024-05-16 几乎使用AI工具润色了每一个证明

\documentclass[a4paper,10pt, oneside]{article}
\usepackage{amsmath,amssymb,amsthm,bm,bbm,mathrsfs,mathtools,graphicx}
\usepackage{algorithm,algorithmic}
%\usepackage[affil-it]{authblk}
\usepackage{authblk}
%\usepackage[inline]{enumitem}
\usepackage[font=small,labelfont=md,textfont=it]{caption}
%\usepackage{floatrow}
\usepackage[titletoc, title]{appendix}
%\usepackage[pdftex,linkcolor=blue,citecolor=blue,backref=page]{hyperref}
\usepackage{enumitem}
\usepackage[colorlinks,linkcolor=blue,citecolor=blue]{hyperref}
\usepackage{etoolbox}
\usepackage{longtable}
\usepackage{diagbox}
\usepackage{booktabs,makecell,multirow}
\usepackage[capitalise,nosort]{cleveref}
\usepackage{cases,color}
\crefname{equation}{}{}
\crefname{lemma}{Lemma}{Lemmas}
\crefname{theorem}{Theorem}{Theorems}
\crefname{discr}{Discretization}{Discretizations}


\DeclareMathOperator{\D}{D}
\DeclareMathOperator{\I}{I}

\apptocmd{\sloppy}{\hbadness 10000\relax}{}{}

%\newcommand{\dual}[1]{\langle {#1} \rangle}
\newcommand{\dual}[1]{\langle {#1} \rangle}
\newcommand{\dualb}[1]{\big\langle {#1} \big\rangle}
\newcommand{\dualB}[1]{\Big\langle {#1} \Big\rangle}
\newcommand{\Dual}[1]{\left\langle {#1} \right\rangle}
\newcommand{\jmp}[1]{{[\![ {#1} ]\!]}}
\newcommand{\nm}[1]{\lVert {#1} \rVert}
\newcommand{\nmb}[1]{\big\lVert {#1} \big\rVert}
\newcommand{\nmB}[1]{\Big\lVert {#1} \Big\rVert}
\newcommand{\Nm}[1]{\left\lVert {#1} \right\rVert}
\newcommand{\snm}[1]{\lvert {#1} \rvert}
\newcommand{\snmb}[1]{\big\lvert {#1} \big\rvert}
\newcommand{\snmB}[1]{\Big\lvert {#1} \Big\rvert}
\newcommand{\Snm}[1]{\left\lvert {#1} \right\rvert}
\newcommand{\ssnm}[1]
{
	\left\vert\kern-0.25ex
	\left\vert\kern-0.25ex
	\left\vert
	{#1}
	\right\vert\kern-0.25ex
	\right\vert\kern-0.25ex
	\right\vert
}
\newcommand{\lsnm}[1]{\snm{#1}^{\text{\tiny L}}}
\newcommand{\rsnm}[1]{\snm{#1}^{\text{\tiny R}}}


\makeatletter
\def\spher@harm#1{%
	\vbox{\hbox{%
			\offinterlineskip
			\valign{&\hb@xt@2\p@{\hss$##$\hss}\vskip.2ex\cr#1\crcr}%
		}\vskip-.36ex}%
}
\def\gshone{\spher@harm{.}}
\def\gshtwo{\spher@harm{.&.}}
\def\gshthree{\spher@harm{.&.&.}}
\let\gsh\spher@harm
\makeatother


\newtheorem{algor}{Algoritheorem}
\newtheorem{discr}{Discretization}
\newtheorem{coro}{Corollary}[section]
\newtheorem{Def}{Definition}[section]
\newtheorem{hypothesis}{Hypothesis}[section]
\newtheorem{lemma}{Lemma}[section]
\newtheorem{remark}{Remark}[section]
\newtheorem{theorem}{Theorem}[section]



\renewcommand\qedsymbol{\hfill\ensuremath{\blacksquare}}

\makeatletter\def\@captype{table}\makeatother



\begin{document}
\title{
  \Large\bf Stability and convergence of the Euler scheme for stochastic linear evolution
  equations in Banach spaces
}
% \author{}

\author{Binjie Li\thanks{libinjie@scu.edu.cn} } 
\author{Xiaoping Xie\thanks{Corresponding author: xpxie@scu.edu.cn}}
\affil{School of Mathematics, Sichuan University, Chengdu 610064, China}




\date{}
\maketitle
\begin{abstract}
  %This paper studies the stability and convergence of the
  %Euler scheme for the stochastic linear evolution equations.
  %Discrete maximal $ \ell^p $-regularity estimate is derived, and
  %sharp error estimate in the norm $ \nm{\cdot}_{\ell^p((0,T)\times\Omega;L^q(\mathcal O))} $,
  %$ p \in (2,\infty) $, $ q \in [2,\infty) $, is established via a duality argument.
  % 2022-11-09
  For the Euler scheme of the stochastic linear evolution
  equations, discrete stochastic maximal $ L^p $-regularity
  estimate is established, and a sharp error estimate in the norm
  $ \nm{\cdot}_{L^p((0,T)\times\Omega;L^q(\mathcal O))} $,
  $ p,q \in [2,\infty) $, is derived via a duality argument.
\end{abstract}

\medskip\noindent{\bf Keywords:} stochastic evolution equations, Euler scheme,
discrete stochastic maximal $ L^p $-regularity, convergence




\section{Introduction}
% 2022-11-09
The numerical methods of stochastic partial differential equations have been
extensively studied in the past decades, and by now it is still an active
research area; see, e.g., \cite{Bessaih2019,Breit2021,CarelliProhl2012,Prohl2013,
Prohl2012,Cui_Hong_2019,Kruse2014book,Yan2005,Zhang2017book}.
%The numerical methods of stochastic evolution equations have been extensively
%studied in the past forty years; see \cite{Kruse2014book,Zhang2017book} and the
%huge references therein.
However, the numerical analysis in this field so far mainly focuses on the
Hilbert space setting; the numerical analysis of
the Banach space setting is rather limited. This motivates us to analyze the
stability and convergence of the Euler scheme for the stochastic linear
evolution equations in Banach spaces, which is one of the most popular
temporal discretization scheme in this realm.

% 2022-11-09
%Our main contribution is twofold. Firstly,
%Firstly, we study the stability
%\cite{Amann2004,Denk2020,Denk2013book,Kunstmann2004,Pruss2003,Pruss2016,Weis2001}
Firstly, we establish a discrete stochastic maximal $ L^p $-regularity estimate.
Maximal $ L^p $-regularity is of fundamental importance for the deterministic
evolution equations; see, e.g., \cite{Denk2013book,Kunstmann2004,Pruss2016,Weis2001}.
In the past twenty years, the discrete maximal $ L^p $-regularity of deterministic
evolution equations has also attracted great attention; see, e.g.,
\cite{Blunck2001,Kemmochi2016, Lubich2016,Kemmochi2018,Vexler_Lp_2017,
LiB2017Math,LiB2017SIAM}. Using the techniques of $ H^\infty $-calculus,
$ \mathcal R $-bundedness, and square function estimates, in the case of
$ p \in (2,\infty) $ and $ q \in [2,\infty) $, Van Neerven et al.~\cite{Neerven2012}
established the stochastic maximal $ L^p $-regularity estimate
\begin{align*}
  \nm{A^{1/2}y}_{L^p(\mathbb R_{+} \times \Omega; L^q(\mathcal O))}
  \leqslant c \nm{f}_{L^p(\mathbb R_{+}\times\Omega;L^q(\mathcal O;H))},
\end{align*}
for the stochastic linear evolution equation
\[ % 2022-11-09
  \begin{cases}
    \mathrm{d}y + Ay(t) \, \mathrm{d}t = f(t) \, \mathrm{d}W(t),
    \quad t > 0, \\
    y(0) = 0.
  \end{cases}
\]
Following the idea in \cite{Neerven2012}, for the Euler scheme
\[ % 2022-10-22 2022-10-24 2022-10-28
  \begin{cases}
    Y_{j+1} - Y_j + \tau A Y_{j+1} = f_j \delta W_j,
    \quad j \in \mathbb N, \\
    Y_0 = 0,
  \end{cases}
\]
we obtain the discrete stochastic maximal $ L^p $-regularity estimate
\begin{align*}
  & \Big(
    \sum_{j=0}^\infty \nmB{
      \frac{Y_{j+1} - Y_j}{\sqrt\tau}
    }_{L^p(\Omega;L^q(\mathcal O))}^p
  \Big)^{1/p} + \Big(
    \sum_{j=0}^\infty \nm{A^{1/2}Y_j}_{L^p(\Omega;L^q(\mathcal O))}^p
  \Big)^{1/p} \\
  \leqslant{} &
  c \Big(
    \sum_{j=0}^\infty \nm{f_j}_{L^p(\Omega;L^q(\mathcal O;H))}^p
  \Big)^{1/p}.
\end{align*}
Under the condition that $ p=q=2 $ and $ {-A} $ is the realization of the
Laplace operator in $ L^q(\mathcal O) $ with homogeneous Dirichlet boundary
condition, the above estimate can be proved by a straightforward energy
argument. Although our numerical analysis assumes that $ A $ is a sectorial
operator on $ L^q(\mathcal O) $, it can also be extended to the Stokes operator.
%We note that the numerical analysis can also be extended to the stochastic
%Stokes equation.

%Recently, Van Neerven, Veraar and Weis \cite{Neerven2012,Neerven2012b}
%proposed the concept of stochastic maximal $ L^p $-regularity for
%the stochastic evolution equations.
%To our best knowledge, so far there is no
%discrete maximal $ L^p $-regularity theory available.

%The numerical analysis of stochastic evolution equations has attracted
%been extensively studied in the past
%forty years; see \cite{Kruse2014book,Zhang2017book} and the huge references
%therein. So far, the numerical analysis of stochastic evolution equations mainly focus
%on the Hilbert space setting. The numerical analysis of the Banach space setting
%is very limited.



% 2022-11-09
Secondly, we derive a sharp error estimate in the norm $ \nm{\cdot}
_{L^p((0,T)\times\Omega;L^q(\mathcal O))} $, $ p,q \in [2,\infty) $.
So far, the numerical analysis in the literature mainly considers the
convergence in a Hilbert space at some given points of time; the convergence 
in the norm $ L^p((0, T)\times\Omega;L^q(\mathcal O)) $ has rarely been analyzed.
Error estimates of this type not only characterize intrinsically the convergence of
the Euler scheme, but also will be indispensable for the numerical analysis
of optimal control problems governed by the stochastic evolution equations.
%Combing the numerical analysis of deterministic evolution equations,
We use a duality argument, together with the convergence result of
a discrete deterministic evolution equation, to derive a sharp error estimate
\[ % 2022-11-09
  \Big(
    \sum_{j=0}^{J-1} \nm{y-Y_j}_{
      L^p((t_j,t_{j+1})\times\Omega;L^q(\mathcal O))
    }^p
  \Big)^{1/p} \leqslant c \tau^{1/2}
  \nm{f}_{L^p((0,T)\times\Omega;L^q(\mathcal O;H))}.
\]
As far as we know, in the case of $ p=q=2 $ (i.e., a Hilbert space setting),
the above error estimate is still new.




%This motivates us to study the numerical analysis of the Euler scheme for the
%stochastic evolution equations in Banach spaces. Firstly, we establish the discrete
%maximal $ \ell^p $-regularity of the Euler scheme. Secondly, we derive a sharp
%convergence rate in the norm $ \nm{\cdot}_{\ell^p((0,T)\times\Omega;L^q(\mathcal
%O))} $.



%Maximal $ \ell^p $-regularity is not only of fundamental importance for the theory of
%deterministic, evolution equations, but also has wide applications, such as
%moving interface problems, optimal control problems governed by evolution
%equations. see \cite{Weis2001,Pruss2016} and the
%references therein. Zhang \cite{ZhangSPDE2006}
%\cite{Brzezniak1995}
%\cite{Krylov1996SPDEs,Krylov2000SPDEsI}




% 2022-11-09
The rest of this paper is organized as follows. \cref{sec:pre} introduces
some notations and the concepts of $ \gamma $-radonifying operators,
$ \mathcal R $-boundedness, $ H^\infty $-calculus and stochastic integral.
\cref{sec:stability} establishes the discrete maximal $ L^p $-regularity.
\cref{sec:convergence} derives a sharp convergence rate.



\section{Preliminaries}
\label{sec:pre}
\medskip\noindent\textbf{Conventions}.
Throughout this paper, we will use the following conventions:
the notation \( \dual{\cdot,\cdot}_E \) represents the duality
pairing between a Banach space \( E \) and its dual space \( E^* \);
for any Banach spaces $ E_1 $ and $ E_2 $, $ \mathcal L(E_1,E_2) $
denotes the space of all bounded linear operators from $ E_1 $ to $ E_2 $,
and $ \mathcal L(E_1,E_1) $ is abbreviated to $ \mathcal L(E_1) $;
the symbol $ I $ denotes the identity operator;
for each \( p \in [1,\infty) \), its conjugate exponent is denoted by \( p' \);
$ \mathcal O \subset \mathbb R^d \, (d\geqslant 2) $ is a bounded
domain with Lipschitz boundary;
$ i $ denotes the imaginary unit;
for any \( z \in \mathbb C \setminus \{0\} \), its argument
\( \operatorname{Arg} z \) is restricted to the interval \( (-\pi, \pi] \);
by $ c $ we mean a generic positive constant, which is independent
of the time step $ \tau $ but may differ in different places.
In addition, for any $ \theta \in (0,\pi) $, define
\[ 
  \Sigma_{\theta} := \{
    z \in \mathbb C \setminus \{0\} \mid
    -\theta < \operatorname{Arg} z < \theta
  \}.
\]


% 2022-11-03
\medskip\noindent\textbf{$\gamma$-Radonifying operators}.
For any Banach space $ E $ and Hilbert space $ U $ with inner product $
(\cdot,\cdot)_U $, define
\[ % 2022-10-29
   \mathcal S(U,E) := \text{span}\big\{
   u \otimes e \mid \, u \in U, \, e \in E
   \},
\]
where $ u \otimes e \in \mathcal L(U,E) $ is defined by
\[
  (u \otimes e)(v) := (v,u)_U e, \quad \forall v \in U.
\]
Let $ \gamma(U,E) $ denote the completion of $ \mathcal S(U,E) $
with respect to the norm
\[
  \nmB{\sum_{n=1}^N \phi_n \otimes e_n}_{\gamma(U,E)} := \Big(
  \mathbb E \nmB{ \sum_{n=1}^N \gamma_n e_n }_E^2
  \Big)^{1/2}
\]
for all $ N \in \mathbb N_{>0} $, all orthonormal systems
$ (\phi_n)_{n=1}^N $ of $ U $, all sequences $ (e_n)_{n=1}^N $ in $ E $,
and all sequences $ (\gamma_n)_{n=1}^N $ of independent standard Gaussian
random variables, where $ \mathbb E $ denotes the expectation operator
associated with the probability space on which $ \gamma_1, \ldots, \gamma_N $
are defined.



% 2022-10-22 2022-10-26 2022-11-03 2023-08-24
\medskip\noindent\textbf{$ \mathcal R $-boundedness}. For any two Banach spaces
$ E_1 $ and $ E_2 $, an operator family $ \mathcal A \subset \mathcal L(E_1,
E_2) $ is said to be $ \mathcal R $-bounded if there exists a constant $ C > 0 $
such that
\[
  \int_0^1 \nmB{ \sum_{n=1}^N r_n(t) B_n x_n  }_{E_2}^2 \, \mathrm{d}t
  \leqslant C \int_0^1 \nmB{ \sum_{n=1}^N r_n(t) x_n }_{E_1}^2 \, \mathrm{d}t
\]
for all $ N \geqslant 1 $, all sequences $ (B_n)_{n=1}^N $ in $\mathcal A $,
all sequences $ (x_n)_{n=1}^N $ in $ E_1 $, and all sequences $ (r_n)_{n=1}^N $
of independent symmetric $ \{-1,1\} $-valued random variables on $ [0,1] $.
We denote by $ \mathcal R(\mathcal A) $ the infimum of these $ C $'s.


%% 2022-11-06
\medskip\noindent\textbf{$ H^\infty $-calculus}.
A sectorial operator $ A $ on some Banach space $ E $ is said to have a bounded
$ H^\infty $-calculus if there exists $ \sigma \in (0,\pi] $ such that
\[
  \nmB{
    \int_{\partial\Sigma_{\sigma}} \varphi(z) (z-A)^{-1} \, \mathrm{d}z
  }_{\mathcal L(E)} \leqslant
  C \sup_{z \in \Sigma_{\sigma}} \snm{\varphi(z)}
\]
for all $ \varphi \in \mathcal H_0^\infty(\Sigma_\sigma) $, where $ C $ is a
positive constant independent of $ \varphi $ and
%\begin{align*}
  %\mathcal H_0^\infty(\Sigma_\sigma) &:= \Big\{
    %\varphi: \Sigma_{\sigma} \to \mathbb C \mid \,
    %\text{$\varphi$ is analytic, } \sup_{z \in \Sigma_{\sigma}}
    %\snm{\Psi(z)} < \infty, \\
    %&
    %\qquad\qquad \text{there exists $ c,\varepsilon > 0 $ such that}
    %\sup_{z \in \Sigma_\sigma} \Big( \frac{1+\snm{z}^2}{\snm{z}}\Big)^\varepsilon \snm{\Psi(z)} < \infty
  %\Big\}
%\end{align*}
\begin{align*}
  \mathcal H_0^\infty(\Sigma_\sigma)
  &:= \Big\{
    \varphi: \Sigma_{\sigma} \to \mathbb C \mid \,
    \text{$\varphi$ is analytic and there exists $ \varepsilon > 0 $ such that} \\
  & \qquad\qquad\qquad\qquad\qquad
  \sup_{z \in \Sigma_\sigma}
  \Big( \frac{1+\snm{z}^2}{\snm{z}} \Big)^\varepsilon
  \snm{\varphi(z)} < \infty
  \Big\}.
\end{align*}
The infimum of these $ \sigma $'s is called the angle of the $ H^\infty
$-calculus of $ A $.


% 2022-10-22 2023-08-24
\medskip\noindent\textbf{Stochastic integral}.
Assume that $ (\Omega, \mathcal F, \mathbb P) $ is a given
complete probability space with a filtration $ \mathbb F :=
(\mathcal F_t)_{t \geqslant 0} $, on which a sequence of
independent $ \mathbb F $-adapted Brownian motions
$ (\beta_n)_{n \in \mathbb N} $ are defined. In the sequel, we will
use $ \mathbb E $ to denote the expectation of a random
variable on $ \Omega $. Let $ H $ be a separable Hilbert
space with inner product $ (\cdot,\cdot)_H $ and an orthonormal
basis $ (h_n)_{n \in \mathbb N} $.
%Define
%\[ % 2022-10-22 2022-10-24
  %W(t) := \sum_{n \in \mathbb N} \beta_n(t) h_n,
  %\quad t \in \mathbb R_{+},
%\]
%where $ W(t) $ is understood as a bounded linear operator from $ H $ to $
%L^2(\Omega) $ as follows:
%\[ % 2022-10-24
  %W(t)h := \sum_{n \in \mathbb N} \beta_n(t) (h,h_n)_H,
  %\quad \forall h \in H.
%\]
For each $ t \in \mathbb R_{+} $, define $ W(t) \in \mathcal L(H, L^2(\Omega)) $
by
\[ % 2022-10-24
  W(t)h := \sum_{n \in \mathbb N} \beta_n(t) (h,h_n)_H,
  \quad \forall h \in H.
\]
%For any $ p,q, \in (1,\infty) $, define
%\begin{align*}
  %\mathcal S_{p,q}
  %&:= \Big\{
    %\sum_{l=1}^L \sum_{m=1}^M \sum_{n=1}^N
    %I_{(t_{l-1},t_{l}]} g_{lmn} h_n\mid \,
    %L, M, L \in \mathbb N_{>0}, \,
    %0 \leqslant t_0 < \ldots < t_L < \infty, \\
  %& \qquad\qquad\qquad\qquad\qquad
  %g_{lmn} \in L^p(\Omega,\mathcal F_{t_{l-1}}, \mathbb P; L^q(\mathcal O))
  %\Big\},
%\end{align*}
%where $ I_{(t_{l-1},t_l]} $ is the indicator function of the time
%interval $ (t_{l-1},t_l] $.
Assume that $ p,q \in (1,\infty) $. For any
\[
  f = \sum_{l=1}^L \sum_{m=1}^M \sum_{n=1}^N
  I_{(t_{l-1},t_{l}]} g_{lmn} h_n,
\]
define
\[
  \int_{\mathbb R_{+}} f(t) \, \mathrm{d}W(t) :=
  \sum_{l=1}^L \sum_{m=1}^M \sum_{n=1}^N
  \big( \beta_n(t_{l}) - \beta_n(t_{l-1}) \big) g_{lmn},
\]
where $ L,M,N $ are positive integers,
$ 0 \leqslant t_0 \leqslant t_1 \ldots \leqslant t_L < \infty $,
$ I_{(t_{l-1},t_l]} $ is the indicator function of the time
interval $ (t_{l-1},t_l] $, and
\[
  g_{lmn} \in L^p(\Omega, \mathcal F_{t_{l-1}}, \mathbb P;L^q(\mathcal O)).
\]
We denote by $ \mathcal S_{pq} $ the set of all such functions $ f $'s.
%There exists a positive constant $ c $ independent of $ f $ such that
%\[
  %\nmB{
    %\int_{\mathbb R_{+}} f(t) dW(t)
  %}_{L^p(\Omega;\ell^{q}(\mathcal O)} \leqslant
  %c \mathbb E \nm{f}_{\ell^{q}(\mathcal O;L^2(\mathbb R_{+};H))}.
%\]
%Hence, the above stochastic integral can be uniquely extended to
%%$ \ell_{\mathbb F}^p(\mathbb R_{+}\times\Omega; L^q(\mathcal O;H)) $,
%$ \ell_{\mathbb F}^p(\Omega;L^q(D;L^2(\mathbb R_{+};H))) $,
%defined as the completion of these $ f $'s with respect to the norm
%$ \nm{\cdot}_{\ell^p(\mathbb R_{+} \times \Omega; L^q(\mathcal O;H))} $.
%More precisely, we have the following isomorphism;
The above integral has the following essential isomorphism feature;
see, e.g., \cite[Theorem 2.3]{Neerven2012b}.
\begin{lemma}
  \label{lem:integral}
  For any $ p,q \in (1,\infty) $, there exist two positive constants
  $ c_0 $ and $ c_1 $ such that
  \begin{equation} % 2022-10-24
    c_0 \mathbb E\nm{f}_{L^q(\mathcal O;L^2(\mathbb R_{+};H))}^p
    \leqslant \mathbb E\nmB{
      \int_{\mathbb R_{+}} f(t) dW(t)
    }_{L^q(\mathcal O)}^p \leqslant
    c_1 \mathbb E\nm{f}_{L^q(\mathcal O;L^2(\mathbb R_{+};H))}^p
  \end{equation}
  %\begin{equation} % 2022-10-24
    %c_0 \nm{f}_{L^p(\Omega;L^q(\mathcal O;L^2(R_{+};H))}
    %\leqslant \nmB{
      %\int_{\mathbb R_{+}} g(t) dW(t)
    %}_{L^p(\Omega;L^q(\mathcal O))} \leqslant
    %c_1 \nm{f}_{L^p(\Omega;L^q(\mathcal O;L^2(R_{+};H))}
  %\end{equation}
  %\begin{equation} % 2022-10-24
    %c_0 \nm{f}_{\ell^p(\mathbb R_{+}\times\Omega;L^q(\mathcal O;H)}
    %\leqslant \nmB{
      %\int_{\mathbb R_{+}} g(t) dW(t)
    %}_{L^p(\Omega;L^q(\mathcal O))} \leqslant
    %c_1 \nm{g}_{\ell^p(\mathbb R_{+}\times\Omega;L^q(\mathcal O;H))}
  %\end{equation}
  %\begin{equation}
    %c_0 \mathbb E \nm{g}_{\ell^{p_1}(\mathcal O;L^2(\mathbb R_{+};H))}^{p_0}
    %\leqslant \mathbb E \nmB{
      %\int_{\mathbb R_{+}} g(t) dW_h(t)
    %}_{\ell^{p_1}(\mathcal O)}^{p_0} \leqslant
    %c_1 \mathbb E \nm{g}_{\ell^{p_1}(\mathcal O;L^2(\mathbb R_{+};H))}^{p_0}.
  %\end{equation}
  for all $ f \in \mathcal S_{pq} $.
\end{lemma}
% 2022-11-09
By virtue of this lemma, for any $ p,q \in (1,\infty) $, the above integral
can be uniquely extended to the space
$ L_\mathbb F^p(\Omega;L^q(\mathcal O;L^2(\mathbb R_{+};H))) $,
defined as the closure of $ \mathcal S_{pq} $ in
$ L^p(\Omega;L^q(\mathcal O; L^2(\mathbb R_{+};H))) $.
For any $ p \in (1,\infty) $ and $ q \in [2,\infty) $, since Minkowski's
inequality implies
\[ % 2022-10-26 2022-11-03
  \nm{f}_{
    L^p(\Omega;L^q(\mathcal O;L^2(\mathbb R_{+};H)))
  } \leqslant \nm{f}_{
    L^p(\Omega;L^2(\mathbb R_{+};L^q(\mathcal O;H))
  } \quad \forall f \in \mathcal S_{pq},
\]
the above integral can also be uniquely extended to
$ L_\mathbb F^p(\Omega;L^2(\mathbb R_{+};L^q(\mathcal O;H))) $,
defined as the closure of $ \mathcal S_{pq} $ in $ L^p(\Omega;L^2(\mathbb
R_{+};L^q(\mathcal O;H))) $.
% 2022-11-04
%\begin{remark}
  %Let $ j \in \mathbb N $ and $ B \in \mathcal L(L^q(\mathcal O)) $. Define
  %\begin{align*}
    %\mathcal S_j &:= \text{span}\bigg\{
      %\sum_{k=1}^{\mathscr K}
      %\sum_{l=1}^{\mathscr L}
      %\sum_{m=1}^{\mathscr M}
      %I_{C_{k}} v_{klm} h_m: \,
      %C_k \in \mathcal F_{t_j}, \,
      %v_{klm} \in L^q(\mathcal O), \,
      %\text{$ (h_m)_{m=1}^{\mathscr M} $ is an} \\
      %& \qquad\qquad\qquad
      %\text{orthonormal system in $ H $}, \,
      %\mathscr K \in \mathbb N_{>0}, \,
      %\mathscr L \in \mathbb N_{>0}, \,
      %\mathscr M \in \mathbb N_{>0}
    %\bigg\},
  %\end{align*}
  %where $ I_{C_k} $ is the indicator function of $ C_k $.
  %For any
  %\[
    %w := \sum_{k=1}^{\mathscr K} \sum_{l=1}^{\mathscr L}
    %\sum_{m=1}^{\mathscr M} u_{klm} v_{lm} h_m
    %\in \mathcal S_j,
  %\]
  %by $ Bw $ we mean
  %\[
    %Bw := \sum_{k=1}^{\mathscr K} \sum_{l=1}^{\mathscr L}
    %\sum_{m=1}^{\mathscr M} I_{C_k} (Bv_{klm}) h_m.
  %\]
  %We can verify that in this sense $ B $ is a bounded linear operator on $
  %\mathcal S_j $. Since $ \mathcal S_j $ is dense in $ L^p(\Omega, \mathcal
  %F_{t_j}, \mathbb P; L^q(\mathcal O;H)) $, $ B $ can be uniquely extended to a bounded
  %linear operator on
  %\[
    %L^p(\Omega, \mathcal F_{t_j}, \mathbb P; L^q(\mathcal O;H)).
  %\]
%\end{remark}


%Let $ \mathcal L(L^q(\mathcal O)) $ denote the set of all bounded linear operators on $
%L^q(\mathcal O) $.

\medskip\noindent\textbf{Discrete spaces}.
% 2022-10-17 2022-11-09
For any Banach space $ E $ and $ p \in [1,\infty) $, define
\[ % 2022-11-03 2022-11-09
  \ell^p(E) := \Big\{
    (v_j)_{j \in \mathbb N}  \Bigl| \,\,
    \sum_{j \in \mathbb N} \nm{v_j}_E^p < \infty
  \Big\},
\]
and endow this space with the norm
\[ % 2022-10-17 2022-11-09
  \nm{(v_j)_{j\in\mathbb N}}_{\ell^p(E)} :=
  \Big( \sum_{j \in \mathbb N} \nm{v_j}_E^p \Big)^{1/p}
  \quad\text{for all } (v_j)_{j \in \mathbb N} \in \ell^p(E).
\]
For any $ v \in \ell^p(E) $, we use $ v_j $, $ j \in \mathbb N $, to denote
its $ j $-th element.
%In particular, for any $ p,q \in [1,\infty) $, define
%\[ % 2022-11-03
  %\ell_{\mathbb F}^p(L^q(\Omega;E) := \big\{
    %v \in \ell^p(L^q(\Omega;E)) \mid \,
    %\text{$v_j$ is $ \mathcal F_{t_j} $-measurable}
  %\big\}.
%\]






%\begin{hypothesis}
  %\label{hypo:A}
  %The operator $ A $ satisfies the following conditions:
  %\begin{itemize} % 2022-10-11
    %\item $ -A $ generates an analytic semigroup on $ L^q(\mathcal O) $;
    %\item the spectrum of $ A $ is contained in
      %\[
        %\Sigma_{\theta_A} := \{
          %z \in \mathbb C \setminus \{0\}: \, \snm{ \arg z } < \theta_A
        %\},
      %\]
      %where $ \theta_A \in (0,\pi/2) $;
      %%\item there exists a positive constant $ \mathcal M $ such that
      %%\begin{equation}
      %%\label{eq:z-A-inv}
      %%\nm{(z-A)^{-1}}_{\mathcal L(L^q(\mathcal O))} \leqslant
      %%\frac{\mathcal M}{1+\snm{z}}, \quad
      %%\forall z \not\in \Sigma_{\theta_A};
      %%\end{equation}
    %\item $ A $ admits a bounded $ H^\infty $-calculus of angle less than $
      %\theta_A $.
  %\end{itemize}
%\end{hypothesis}
%\medskip\textbf{Definition of $ A $}. Let $ A $ be the realization of a
%second-order elliptic operator in $ L^q(\mathcal O) $ satisfying the following conditions:
%\begin{itemize} % 2022-10-11 2022-10-22
  %\item $ -A $ generates an analytic semigroup on $ L^q(\mathcal O) $;
  %\item the spectrum of $ A $ is contained in
    %\[
      %\Sigma_{\theta_A} := \{
        %z \in \mathbb C \setminus \{0\}: \, \snm{ \arg z } < \theta_A
      %\},
    %\]
    %where $ \theta_A \in (0,\pi/2) $;
    %%\item there exists a positive constant $ \mathcal M $ such that
    %%\begin{equation}
    %%\label{eq:z-A-inv}
    %%\nm{(z-A)^{-1}}_{\mathcal L(L^q(\mathcal O))} \leqslant
    %%\frac{\mathcal M}{1+\snm{z}}, \quad
    %%\forall z \not\in \Sigma_{\theta_A};
    %%\end{equation}
  %\item $ A $ admits a bounded $ H^\infty $-calculus (see, e.g.,
    %\cite[Chapter~10]{HytonenWeis2017}) of angle less than $
    %\theta_A $.
  %\item the set $ \{z(z+A)^{-1} \mid z \in \Sigma_{\pi-\theta_A} \} $ is
    %$ \mathcal R $-bounded.
%\end{itemize}
%There exists a positive constant $ C_A $ such that
%\begin{equation}
  %\label{eq:z-A-inv}
  %\nm{(z - A)^{-1}}_{\mathcal L(L^q(D))}
  %\leqslant \frac{C_A}{1+\snm{z}}
  %\quad \forall z \not\in \sigma_{\theta_A}.
%\end{equation}
%Let $ D(A) $ denote the domain of $ A $, equipped with the usual norm that
%\[
  %\nm{v}_{D(A)} := \nm{Av}_{L^q(D)}
  %\quad \forall v \in D(A).
%\]

%Let $ \ell^{p,\mathrm{c}}(E) $ be the space of all sequences in $ \ell^p(E) $
%with compact supports.
%For any $ 1 \leqslant p,q < \infty $, define
%\begin{align*} % 2022-10-17
  %\ell_{\mathbb F}^p(L^q(\Omega;E)) :=
  %\big\{
    %v \in \ell^p(L^q(\Omega;E)): \,
    %\text{$ v_j $ is $ \mathcal F_{t_j} $-measurable}
  %\big\}.
%\end{align*}
%It is standard that $ \ell_{\mathbb F}^p(L^q(\Omega;E)) $ is a Banach space with
%respect to the norm $ \nm{\cdot}_{\ell^p(L^q(\Omega;E))} $.








\section{Stability estimates}
\label{sec:stability}
% 2022-10-11 2022-10-22 2022-10-23 2022-10-24 2022-10-27
% 2022-11-03 2022-11-04
Fix $ 0 < \tau < 1 $ and let $ t_j := j \tau $ for each $ j \in \mathbb N $.
Define
\[ % 2022-10-23 2022-10-24 2022-10-28
  \delta W_j := W(t_{j+1}) - W(t_j),
  \quad j \in \mathbb N.
\]
%For any Banach space $ E $ and $ 1 \leqslant p,q < \infty $, define
%\begin{align*} % 2022-10-17
  %\ell_{\mathbb F}^p(L^q(\Omega;E)) :=
  %\big\{
    %v \in \ell^p(L^q(\Omega;E)): \,
    %\text{$ v_j $ is $ \mathcal F_{t_j} $-measurable}
  %\big\}.
%\end{align*}
For any $ p,q,r \in [1,\infty) $, define
\[ % 2022-10-24
  \ell_{\mathbb F}^p(L^r(\Omega;L^q(\mathcal O;H))) :=
  \big\{
    v \in \ell^p(L^r(\Omega;L^q(\mathcal O;H))) \mid \,
    v_j \text{ is $ \mathcal F_{t_j} $-measurable}
  \big\}.
\]
It is standard that $ \ell_{\mathbb F}^p(L^r(\Omega;L^q(\mathcal O;H))) $ is
a Banach space with respect to the norm $ \nm{\cdot}_{\ell^p(L^r(\Omega;
L^q(\mathcal O;H)))} $.
This section studies the stability of the following Euler scheme: seek
$ Y := (Y_j)_{j\in\mathbb N} $ such that
\begin{subequations} % 2022-10-22 2022-10-24 2022-10-28
  \label{eq:Y-def}
  \begin{numcases}{}
    Y_{j+1} - Y_j + \tau A Y_{j+1} = f_j \delta W_j,
    \quad j \in \mathbb N, \\
    Y_0 = 0,
  \end{numcases}
\end{subequations}
where $ f := (f_j)_{j\in \mathbb N} $ is given.
%In the rest of the paper, $ c $ denotes a generic positive constant, which is
%independent of $ \tau $ but may differ in different places.
The main result of this section are the following two theorems.
\begin{theorem} % 2022-10-22 2022-10-28 2022-11-04
  \label{thm:time-regu}
  Let $ p,q,r \in (1,\infty) $. Assume that $ A $ is a densely defined
  sectorial operator on $ L^q(\mathcal O) $ and
  \(\{z(z-A)^{-1} \mid z \in \mathbb C\setminus\overline{\Sigma_{\theta_A}}\} \)
  is $ \mathcal R $-bounded in $ \mathcal L(L^q(\mathcal O)) $,
  where $ \theta_A \in (0,\pi/2) $.
  Let $ Y $ be the solution to \cref{eq:Y-def} with
  \[% 2022-10-24
    f \in \ell_{\mathbb F}^p(L^r(\Omega; L^q(\mathcal O;H))).
  \]
  Then
  \begin{equation}
    \label{eq:time-regu}
    \Big(
      \sum_{j=0}^\infty  \nmB{
        \frac{Y_{j+1} - Y_j}{\sqrt\tau}
      }_{L^r(\Omega;L^q(\mathcal O))}^p
    \Big)^{1/p} \leqslant
    c \nm{f}_{\ell^p(L^r(\Omega;L^q(\mathcal O;H)))}.
  \end{equation}
\end{theorem}


%\begin{proof}
%Let $ \widetilde A $ be the natural extension of $ A $ in $ L^r(\Omega;L^q(\Omega)) $.
%It is easily to verify that $ \widetilde A $ is a sectorial operator on $
%L^r(\Omega;L^q(\Omega)) $, and
%\[
%\mathcal R\big(
%\{ z(z+\widetilde A)^{-1} \mid z \in \Sigma_{\pi-\theta_A} \}
%\big) \leqslant
%\mathcal R\big(
%\{ z(z+A)^{-1} \mid z \in \Sigma_{\pi-\theta_A} \}
%\big) < \infty.
%\]
%Define $ U,G \in l(L^p(\Omega;L^q(\mathcal O))) $ by
%\begin{align*} % 2022-10-22
%U_j &:= \frac{Y_{j+1} - Y_j}{\sqrt\tau},
%\quad j \in \mathbb N, \\
%G_j &:= f_j \delta W_j /\sqrt\tau,
%\quad \forall j \in \mathbb N.
%\end{align*}
%By the discrete Fourier transform, we obtain
%\begin{align*}
%U = \mathcal F^{-1}(M \mathcal F G),
%\end{align*}
%where
%\begin{align*}
%M(\xi) := (1-e^{-i\xi}) (1 - e^{-i\xi} + \tau A)^{-1},
%\quad \xi \in [-\pi,\pi].
%\end{align*}
%Hence, by \cite[Theorem~1.3]{Blunck2001} and the estimate
%\begin{align*} % 2022-10-11 2022-10-12 2022-10-22
%\nm{G}_{\ell^p(L^p(\Omega;L^q(\mathcal O)))}
%&= \Big(
%\sum_{j=0}^\infty \nm{f_j\delta W_j/\sqrt\tau}_{
%L^p(\Omega;L^q(\mathcal O))
%}^p
%\Big)^{1/p} \\
%& \leqslant c\Big(
%\sum_{j=0}^\infty \nm{f_j}_{L^p(\Omega;L^q(\mathcal O;H))}^p
%\Big)^{1/p} \quad\text{(by \cref{lem:integral})} \\
%& = c \nm{f}_{\ell_{+}^p(L^p(\Omega;L^q(\mathcal O;H)))},
%\end{align*}
%it remains to prove
%\begin{equation}
%\label{eq:M-theta}
%\mathcal R(\mathcal T) \leqslant c,
%\end{equation}
%where
%\begin{align*}
%\mathcal T := \big\{
%M(\xi) \mid\, \xi \in [-\pi, \pi]
%\big\} \cup \big\{
%(e^{i\xi}-1)(e^{i\xi} + 1) M'(\xi) \mid \, \xi \in [-\pi, \pi]
%\big\}.
%\end{align*}

%To this end, we proceed as follows. Since
%\[
%M(\xi) = \frac{1-e^{-i\xi}}\tau
%\Big(
%\frac{1-e^{-i\xi}}\tau + A
%\Big)^{-1}, \quad \xi \in [-\pi,\pi],
%\]
%by the definition of $ A $ we conclude that
%\[
%\mathcal R\big( \{M(\xi)\mid \xi \in [-\pi,\pi] \big) \leqslant c.
%\]
%A direct computation gives, for any $ \xi
%\in [-\pi,\pi] $,
%\[
%(e^{i\xi}-1)(e^{i\xi}+1)M'(\xi) = M_1(\xi) + M_2(\xi),
%\]
%where
%\begin{align*} % 2022-10-22
%M_1(\xi) &:= i(e^{i\xi}+1)\frac{1-e^{-i\xi}}\tau
%\Big( \frac{1 - e^{-i\xi}}\tau + A \Big)^{-1}, \\
%M_2(\xi) &:= -i(e^{i\xi}+1) \bigg(
%\frac{1-e^{-i\xi}}\tau
%\Big( \frac{1 - e^{-i\xi}}\tau +  A \Big)^{-1}
%\bigg)^2
%\end{align*}
%%\begin{align*}
%%& (e^{i\xi}-1)(e^{i\xi}+1)M'(\xi) \\
%%={} &
%%ie^{-i\xi}(1-e^{-i\xi} + \tau A)^{-1} -
%%ie^{-i\xi}(1-e^{-i\xi})(1-e^{-i\xi}+\tau A)^{-2}.
%%\end{align*}
%We have
%\begin{align*}
%\mathcal R(\{M_1(\xi)\mid \xi \in [-\pi,\pi]\})
%\end{align*}
%By \cref{eq:z-A-inv} we have
%\begin{align*} % 2022-10-22
%\sup_{\xi \in [-\pi,\pi]}
%\nm{M'(\xi)}_{\mathcal L(L^r(\Omega;L^q(\mathcal O)))}
%\leqslant c.
%\end{align*}
%By \cite[Proposition~8.5.7]{HytonenWeis2017} we then obtain
%\[
%\mathcal R\big(
%\{M(\xi) \mid \xi \in [-\pi,\pi]\}
%\big) \leqslant c.
%\]
%\begin{align*}
%& \mathcal R\big\{
%A^\theta (z+A)^{-1}: \, \Re z \geqslant 0
%\big\} \\
%={}& \mathcal R\big\{
%A^{\theta-1} A (z+A)^{-1}: \, \Re z \geqslant 0
%\big\} \\
%\leqslant{} &
%\nm{A^{\theta-1}}_{\mathcal L(L^q(\mathcal O))}
%\mathcal R\big\{
%A(z+A)^{-1}: \, \Re z \geqslant 0
%\big\} \\
%\leqslant{} &
%c.
%\end{align*}
%Hence,
%\begin{equation}
%\mathcal R\big\{ M_\theta(t): \, t \in [-\pi,\pi] \big\}
%\leqslant c.
%\end{equation}
%\end{proof}


%\begin{remark}
  %Let $ Y $ be the solution to \cref{eq:Y-def} with
  %\[
    %f \in \ell_{\mathbb F}^{p_0}(\ell^{p_1}(\Omega;\ell^{p_2}(\mathcal O;H))),
  %\]
  %where $ p_0,p_1,p_2 \in (1,\infty) $. A simple modification of the proof
  %of \cref{eq:time-regu} yields
  %\[
    %\Big(
      %\sum_{j=0}^\infty  \nmB{
        %\frac{Y_{j+1} - Y_j}{\sqrt\tau}
      %}_{\ell^{p_1}(\Omega;\ell^{p_2}(\mathcal O))}^{p_0}
    %\Big)^{1/p_0} \leqslant
    %c \nm{f}_{\ell^{p_0}(\ell^{p_1}(\Omega;\ell^{p_2}(\mathcal O;H)))}.
  %\]
%\end{remark}
% 2022-10-12
%\begin{remark}
  %By \cite[Proposition 4.2.15]{HytonenWeis2017}, the space
  %\[
    %\ell^p(L^r(\Omega;L^q(\mathcal O)))
  %\]
  %is a UMD space for all $ 1 < p,q,r < \infty $.
%\end{remark}

\begin{theorem} % 2022-10-22 2022-11-04 2023-08-28
  \label{thm:space-regu}
  Let $ p \in (2,\infty) $ and $ q \in [2,\infty) $. Assume that $ A $ is a
  densely defined sectorial operator on $ L^q(\mathcal O) $ satisfying the
  following conditions:
  %assumptions on $ A $ in \cref{thm:time-regu} also hold, and in addition
  \begin{itemize}
    \item $ A $ has a dense range in $ L^q(\mathcal O) $;
    % \item the range of $ A $ is dense in $ L^q(\mathcal O) $;
    % \item the resolvent set of $ A $ contains
    %   $ \{z \in \mathbb C \setminus \{0\} \mid \snm{\operatorname{Arg} z} \geqslant \theta_A  \}$ with
    %   $ \theta_A \in (0,\pi/2) $;
    \item  there exists $ \theta_A \in (0,\pi/2) $ such that
      \begin{equation}
        \label{eq:z-A-inv}
        \sup_{z \in \mathbb C \setminus \{0\}, \, \snm{\operatorname{Arg} z} \geqslant \theta_A} \,
        \snm{z} \nm{(z-A)^{-1}}_{\mathcal L(L^q(\mathcal O))}
        < \infty;
      \end{equation}
    \item $ A $ admits a bounded $ H^\infty $-calculus of angle less than $ \theta_A $.
      % \begin{equation}
      %   \label{eq:z-A-inv}
      %   \sup_{z \in \overline{\Sigma_{\pi-\theta_A}} \setminus \{0\}} \,
      %   \snm{z} \nm{(z+A)^{-1}}_{\mathcal L(L^q(\mathcal O))}
      %   < \infty.
      % \end{equation}
    % \, \mathrm{d}\snm{\lambda} \nm{f}_X \quad\text{(by \cref{eq:z-A-inv})}.
      %\item $ A $ admits a bounded $ H^\infty $-calculus (see, e.g.,
      %\cite[Chapter~10]{HytonenWeis2017}) of angle less than $
      %\theta_A $.
      % \item $ A $ admits a bounded $ H^\infty $-calculus of angle less than $
      %   \theta_A $.
  \end{itemize}
  Let $ Y $ be the solution to \cref{eq:Y-def} with
  \[
    f \in \ell_{\mathbb F}^p(L^p(\Omega; L^q(\mathcal O;H))).
  \]
  Then
  \begin{equation}
    \label{eq:space-regu}
    \nm{A^{1/2}Y}_{\ell^p(L^p(\Omega;L^q(\mathcal O)))} \leqslant
    c \nm{f}_{\ell^p(L^p(\Omega;L^q(\mathcal O;H)))}.
  \end{equation}
\end{theorem}

% \begin{remark}
%   Under the conditions in \cref{thm:space-regu}, a modification of
%   the proof of \cref{thm:space-regu} yields that
%   \[
%     \sup_{j \in \mathbb N} \, \nm{A^{1/2-1/p} Y_j}_{L^p(\Omega;L^q(\mathcal O))}
%     \leqslant c \tau^{1/p} \nm{f}_{\ell^p(L^p(\Omega;L^q(\mathcal O;H)))}.
%   \]
%   We leave the details to the interested readers.
% \end{remark}

%% 2022-10-31
%\begin{theorem}
  %\label{thm:time-inf-regu}
  %Under the condition of \cref{thm:space-regu}, we have
  %\begin{equation}
    %\mathbb E\sup_{j \in \mathbb N} \,
    %\nm{A^{1/2-1/p}Y_j}_{L^q(\mathcal O)} \leqslant
    %c \nm{f}_{\ell^p(L^p(\Omega;L^q(\mathcal O;H)))}.
  %\end{equation}
%\end{theorem}



\subsection{Proof of \texorpdfstring{\cref{thm:time-regu}}{}}
% 2022-10-23 2022-11-04
Assume that each $ f_j $ is of the form
\[
  f_j = \sum_{m=1}^M \sum_{n=1}^N g_{jmn} h_n,
\]
where $ M, N $ are positive integers and $ g_{jmn} \in L^p(\Omega, \mathcal
F_{t_j}, \mathbb P; L^q(\mathcal O)) $; the general case can be proved by
a density argument.
Let $ \widetilde A $ be the natural extension of $ A $ in $
L^r(\Omega;L^q(\mathcal O)) $. It is easily verified that $ \widetilde A $ is a
sectorial operator on $ L^r(\Omega;L^q(\mathcal O)) $. Let $ (r_n)_{n=1}^\infty $ be
a sequence of independent symmetric $ \{-1,1\} $-valued random variables on
$ [0,1] $. For any $ N \geqslant 1 $, $ (z_n)_{n=1}^N \subset
\mathbb C \setminus \overline{\Sigma_{\theta_A}} $ and $ (v_n)_{n=1}^N \subset L^r(\Omega;L^q(\mathcal O)) $, 
we obtain
\begin{align*} % 2022-10-22
    & \Big(
      \int_0^1 \nmB{
        \sum_{n=1}^N r_n(t) z_n(z_n - \widetilde A)^{-1}v_n
      }_{L^r(\Omega;L^q(\mathcal O))}^2 \, \mathrm{d}t
    \Big)^{1/2} \\
  \stackrel{\text{(i)}}{\leqslant} &
  c \Big(
    \int_0^1 \nmB{
      \sum_{n=1}^N r_n(t) z_n(z_n - \widetilde A)^{-1}v_n
    }_{L^r(\Omega;L^q(\mathcal O))}^r \, \mathrm{d}t
  \Big)^{1/r} \\
  %={} &
  %c \Big(
  %\int_0^1 \mathbb E \nmB{
  %\sum_{n=1}^N r_n(t) z_n(z_n+\widetilde A)^{-1}v_n
  %}_{L^q(\mathcal O)}^r
  %\, \mathrm{d}t
  %\Big)^{1/r} \\
  ={} &
  c \Big(
    \mathbb E\int_0^1 \nmB{
      \sum_{n=1}^N r_n(t) z_n(z_n - \widetilde A)^{-1}v_n
    }_{L^q(\mathcal O)}^r
    \, \mathrm{d}t
  \Big)^{1/r} \\
  ={} &
  c \Big(
    \mathbb E\int_0^1 \nmB{
      \sum_{n=1}^N r_n(t) z_n(z_n - A)^{-1}v_n
    }_{L^q(\mathcal O)}^r
    \, \mathrm{d}t
  \Big)^{1/r} \\
  % \leqslant{} &
  \stackrel{\text{(ii)}}{\leqslant} &
  c \bigg(
    \mathbb E \Big(
      \int_0^1
      \nmB{
        \sum_{n=1}^N r_n(t) z_n(z_n - A)^{-1}v_n
      }_{L^q(\mathcal O)}^2
      \, \mathrm{d}t
    \Big)^{r/2}
  \bigg)^{1/r} \\
  % \leqslant{} &
  \stackrel{\text{(iii)}}{\leqslant} &
  c\mathcal R\big(\{z(z - A)^{-1}\mid z \in \mathbb C \setminus\overline{\Sigma_{\theta_A}}\}\big)
  \bigg(
    \mathbb E  \Big(
      \int_0^1
      \nmB{
        \sum_{n=1}^N r_n(t) v_n
      }_{L^q(\mathcal O)}^2
      \, \mathrm{d}t
    \Big)^{r/2}
  \bigg)^{1/r} \\
  % \leqslant{} &
  \stackrel{\text{(iv)}}{\leqslant} &
  c\mathcal R\big(\{z(z - A)^{-1}\mid z \in \mathbb C \setminus\overline{\Sigma_{\theta_A}}\}\big)
  \Big(
    \mathbb E  \int_0^1 \nmB{
      \sum_{n=1}^N r_n(t) v_n
    }_{L^q(\mathcal O)}^r
    \, \mathrm{d}t
  \Big)^{1/r} \\
  ={} &
  c\mathcal R\big(\{z(z-A)^{-1}\mid z \in \mathbb C \setminus\overline{\Sigma_{\theta_A}}\}\big)
  \Big(
    \int_0^1 \nmB{
      \sum_{n=1}^N r_n(t) v_n
    }_{L^r(\Omega;L^q(\mathcal O))}^r
    \, \mathrm{d}t
  \Big)^{1/r} \\
  % \leqslant{} &
  \stackrel{\text{(v)}}{\leqslant} &
  c\mathcal R\big(\{z(z-A)^{-1}\mid z \in \mathbb C \setminus \overline{\Sigma_{\theta_A}}\}\big)
  \Big(
    \int_0^1 \nmB{
      \sum_{n=1}^N r_n(t) v_n
    }_{L^r(\Omega;L^q(\mathcal O))}^2
    \, \mathrm{d}t
  \Big)^{1/2},
\end{align*}
where in (i), (ii), (iv) and (v) we used the Kahane-Khintchine inequality (see, e.g.,
\cite[Theorem~6.2.4]{HytonenWeis2017}), and in (iii) we used the $ \mathcal R $-boundedness of
$ \{z(z-A) \mid z \in \mathbb C \setminus\overline{\Sigma_{\theta_A}}\}$. It follows that
$ \{z(z-\widetilde A)^{-1} \mid z \in \mathbb C \setminus\overline{\Sigma_{\theta_A}}\} $
is $ \mathcal R $-bounded. Moreover,
\cite[Proposition~4.2.15]{HytonenWeis2016} implies that $
L^r(\Omega;L^q(\mathcal O)) $ is a UMD space. Therefore,
we use \cite[Theorem 3.2]{Kemmochi2016} to conclude that
% 2022-10-22 2022-10-23
\begin{align*}
  \Big(
    \sum_{j=0}^\infty \nmB{
      \frac{Y_{j+1} - Y_j
    }\tau }_{L^r(\Omega;L^q(\mathcal O))}^p
  \Big)^{1/p} \leqslant c \Big(
    \sum_{j=0}^\infty \nmB{
      \frac{f_j}\tau \delta W_j
    }_{L^r(\Omega;L^q(\mathcal O))}^p
  \Big)^{1/p},
\end{align*}
which implies
\begin{align*} % 2022-10-22
  \Big(
    \sum_{j=0}^\infty
    \nmB{ \frac{Y_{j+1} - Y_j }{\sqrt\tau}}_{L^r(\Omega;L^q(\mathcal O))}^p
  \Big)^{1/p} \leqslant c \Big(
    \sum_{j=0}^\infty \nmB{
      \frac{f_j}{\sqrt\tau} \delta W_j
    }_{L^r(\Omega;L^q(\mathcal O))}^p
  \Big)^{1/p}.
\end{align*}
Consequently, the desired inequality \cref{eq:time-regu} follows from the
estimate
% 2022-10-17 2022-10-19 2022-10-22 2022-11-04
\begin{align*}
    & \bigg(
      \sum_{j=0}^\infty \nmB{
        \frac{f_j}{\sqrt\tau} \delta W_j
      }_{L^r(\Omega;L^q(\mathcal O))}^p
    \bigg)^{1/p} \\
  \leqslant{} &
  c \bigg(
    \sum_{j=0}^\infty
    \nm{f_j}_{L^r(\Omega;L^q(\mathcal O;H))}^p
  \bigg)^{1/p} \quad\text{(by \cref{lem:integral})} \\
  ={} &
  c \nm{f}_{\ell^p(L^r(\Omega;L^q(\mathcal O;H)))}.
\end{align*}
This completes the proof of \cref{thm:time-regu}.


\subsection{Proof of \texorpdfstring{\cref{thm:space-regu}}{}}
\label{ssec:space-regu}
% 2022-11-04 2023-08-25
Throughout this subsection, we will assume that the conditions in
\cref{thm:space-regu} are always satisfied. Firstly, let us introduce
some notations. Let $ A^* $ be the dual operator of $ A $ on $ L^{q'}(\mathcal O) $.
It is standard that $ A^* $ is a sectorial operator on $ L^{q'}(\mathcal O) $;
see, e.g., \cite[Theorem~2.4.1]{Sinha2017book}.
Moreover, $ A^* $ has a bounded $ H^\infty $-calculus 
as $ A $; see \cite[Proposition 10.2.20]{HytonenWeis2017}.


\begin{figure}[ht]
  \centering
  \includegraphics[width=0.5\textwidth]{upsilon4.png}
  \caption{The orientations of $ \Upsilon_1 $, $ \Upsilon_2 $ and $ \partial\Sigma_{\theta_A} $.}
  \label{fig:upsilon_sigma}
\end{figure}


Suppose the complex logarithm, denoted by \( \log \), is confined to the
strip \( \{z \in \mathbb{C} \mid -\pi < \operatorname{Im} z \leq \pi\} \).
Choose any \( 0 < \epsilon_A < \cot\theta_A \). 
A simple calculation gives the existence of a positive real number \( k_A \)
such that
\[
e^{-x} \cos y - 1 \leqslant \epsilon_A e^{-x} |\sin y|
\]
holds for all \( x \geqslant -k_A |y| \) with \( y \in [-\pi, \pi] \).
This leads to
\[
  \operatorname{Re}(e^{-(x+iy)}-1) \leqslant
  \epsilon_A |\operatorname{Im}(e^{-(x+iy)}-1)|
\]
for the same conditions on \( x \) and \( y \). Consequently,
\begin{equation}
  \label{eq:lost}
  |\operatorname{Arg}(e^{-z}-1)| > \theta_A
  \quad\text{for all $ z \in \Sigma_{\beta_A} $ with $ -\pi \leqslant \operatorname{Im} z \leqslant \pi $},
\end{equation}
where \( \beta_A := \pi/2 + \arctan k_A \).
By selecting \( \alpha_A \) sufficiently large within the interval
\( (\theta_A,\pi/2) \), we can establish the existence of \( \alpha_A \)
such that the curve described by \( \{-\log(1+re^{i\alpha_A}) \mid r > 0\} \)
intersects the boundary \( \partial\Sigma_{\beta_A} \)
at a unique point. We then define
\begin{align*}
  \Upsilon
  &:= \Upsilon_1 \cup \Upsilon_2, \\
  \Upsilon_1
  &:= \left\{
    -\log(1 + re^{i\alpha_A}) \big| \, 0 \leqslant r \leqslant r_A
  \right\} \bigcup \left\{
    -\log(1 + re^{-i\alpha_A}) \big| \, 0 \leqslant r \leqslant r_A
  \right\}, \\
  \Upsilon_2
  &:= \left\{
    re^{-i\beta_A} \Big| \, |\log(1+r_Ae^{i\alpha_A})| \leqslant r \leqslant \frac{\pi}{\cos(\beta_A-\pi/2)}
  \right\} \\
  & \qquad \bigcup \left\{
    re^{i\beta_A} \Big| \, |\log(1+r_Ae^{i\alpha_A})| \leqslant r \leqslant \frac{\pi}{\cos(\beta_A-\pi/2)}
  \right\},
\end{align*}
with \( r_A \) being the positive value at the aforementioned intersection point.
Given the compactness of \( \{e^{-z}-1 : z \in \Upsilon_2\} \) and its
non-intersection with \( \partial\Sigma_{\theta_A} \), it follows that
\begin{equation}
  \label{eq:upsilon2}
  \inf_{z \in \Upsilon_2} \inf_{\lambda \in \Sigma_{\theta_A}} |e^{-z} - 1 - \lambda| > 0.
\end{equation}
For any $ z \in \Sigma_{\theta_A} $, define
\begin{align}
  \varphi_{+}(z) &:= (2\pi i)^{-1/2} e^{i\alpha_A/4} z^{1/4}
  (-e^{i\alpha_A} + z)^{-1/2},
  \label{eq:varphi-} \\
  \varphi_{+}^*(z) &:= (-2\pi i)^{-1/2} e^{-i\alpha_A/4}
  z^{1/4} (-e^{-i\alpha_A} + z)^{-1/2},
  \label{eq:varphi-*} \\
  \varphi_{-}(z) &:= (-2\pi i)^{-1/2} e^{-i\alpha_A/4}
  z^{1/4} (-e^{-i\alpha_A} + z)^{-1/2}, 
  \label{eq:varphi+} \\
  \varphi_{-}^*(z) &:= (2\pi i)^{-1/2} e^{i\alpha_A/4}
  z^{1/4} (-e^{i\alpha_A} + z)^{-1/2}.
  \label{eq:vaprhi+*}
\end{align}
For any \( z \in \Upsilon \setminus \{0\} \),
two additional functions, \( \Psi \) and \( \Psi^* \), are defined:
\begin{align}
  \Psi(z) &:= (2\pi i)^{-1/2} (e^{-z}-1)^{-1/4} (\tau A)^{1/4} (1 - e^{-z} + \tau A)^{-1/2}, \label{eq:varphi} \\
  \Psi^*(z) &:= (-2\pi i)^{-1/2} (e^{-\bar z}-1)^{-1/4} (\tau A^*)^{1/4} (1 - e^{-\bar z} + \tau A^*)^{-1/2}.
  \label{eq:varphi*}
\end{align}
Lastly, we introduce a family of operators, \( \{\mathcal I(z) \mid z \in \Upsilon\} \),
which map between \( \ell_{\mathbb F}^p(L^p(\Omega;L^q(\mathcal O;H))) \)
and \( \ell^p(L^p(\Omega;L^q(\mathcal O))) \), as follows: for any $ z \in \Upsilon $ and
$ g \in \ell_{\mathbb F}^p(L^p(\Omega;L^q(\mathcal O;H))) $, 
$ \mathcal I(z)g $ is defined by
\begin{subequations}
  \label{eq:I-def}
  \begin{numcases}{}
    \big( \mathcal I(z)g \big)_0 := 0, \\
    \big( \mathcal I(z)g \big)_j := \sum_{k=0}^{j-1} 
    (e^{-z} - 1)^{1/2} e^{(j-k-1)z} g_k \frac{\delta W_k}{\sqrt\tau},
    \quad j \geqslant 1.
  \end{numcases}
\end{subequations}
In \cref{sec:some_proofs} we demonstrate that the operators
$ \mathcal I(z) $, $ z \in \Upsilon $, are indeed bounded linear
operators from $ \ell_{\mathbb F}^p(L^p(\Omega;L^q(\mathcal O;H))) $ to
$ \ell^p(L^p(\Omega;L^q(\mathcal O))) $.


% Lastly, we introduce a family of operators, \( \{\mathcal I(z) : z \in \Upsilon\} \),
% which map between sequence spaces \( \ell_{\mathbb F}^p(L^p(\Omega;L^q(\mathcal O;H))) \)
% and \( \ell^p(L^p(\Omega;L^q(\mathcal O))) \).
% For any \( z \in \Upsilon \) and \( g \) in the former space, these operators are defined by
% \begin{align*}
% (\mathcal I(z)g)_0 &= 0, \\
% (\mathcal I(z)g)_j &= \sum_{k=0}^{j-1} (e^{-z} - 1)^{1/2} e^{(j-k-1)z} g_k \frac{\delta W_k}{\sqrt\tau}, \quad j \geqslant 1.
% \end{align*}
% This framework intertwines complex analysis, operator theory, and potentially stochastic calculus, providing a rigorous foundation for further investigations into the behavior of exponential functions and their implications within a specified domain.





%For any $ z \in \Upsilon $ and $ g \in \ell_{\mathbb F}^p(L^p(\Omega;L^q(\mathcal O;H))) $,
%define
%\[ % 2022-10-17
  %\mathcal I(z)g \in E_1
%\]
%by
%\begin{subequations} % 2022-10-17
  %\begin{numcases}{}
    %(\mathcal I(z)g)_0 := 0, \\
    %\big( \mathcal I(z)g \big)_j := \sum_{k=0}^{j-1} (e^{-z}-1)^{1/2}
    %e^{(j-k-1)z} g_k \delta W_k/\sqrt\tau,
    %\quad j \geqslant 1.
  %\end{numcases}
%\end{subequations}
%For each $ m \in \mathbb N $ and $ g \in \ell_{\mathbb F}^p(L^p(\Omega;L^q(\mathcal O;H))) $,
%define
%\[
  %\mathcal {II}_mg \in E_1
%\]
%by
%% 2022-10-17
%\begin{subequations}
  %\begin{numcases}{}
    %(\mathcal {II}_mg)_0 := 0, \\
    %(\mathcal {II}_mg)_j := \frac1{\sqrt{m+1}}
    %\sum_{k=(j-1-m)\vee 0}^{j-1} g_k \delta W_k / \sqrt\tau,
    %\quad j \geqslant 1.
  %\end{numcases}
%\end{subequations}

%Define
%\begin{equation}
  %\label{eq:I-def}
  %\big(\mathcal I(z)g\big)_j :=
  %\sum_{k=0}^{j-1} z^{1/2} (1+z)^{k+1-j} f_k \delta W_k / \sqrt\tau.
%\end{equation}


Secondly, let us introduce the $ \mathcal R $-boundedness of $ \{\mathcal I(z) \mid z \in \Upsilon \}  $ and
the square function bounds associated with $ \varphi_{+} $, $ \varphi_{+}^* $, $ \varphi_{-} $, 
$ \varphi_{-}^* $, $ \Psi $ and $ \Psi^* $.
For the clarity of the presentation of the main idea of the proof of
\cref{thm:space-regu}, we put some technical lemmas in \cref{sec:some_proofs}.

% 2022-10-31 2022-11-01 2022-11-02 2022-11-06
\begin{lemma}
  \label{lem:Pi_m}
  Define
  \[ 
    \{ \Pi_m \mid \, m \in \mathbb N \} \subset
    \mathcal L\big(
      \ell_{\mathbb F}^p(L^p(\Omega;L^q(\mathcal O;H))), \,
      \ell^p(L^p(\Omega;L^q(\mathcal O)))
    \big)
  \]
  as follows: for each $ m \in \mathbb N $ and
  $ g \in \ell_{\mathbb F}^p(L^p(\Omega;L^q(\mathcal O;H))) $,
  \begin{subequations}
    \label{eq:calII-def}
    \begin{numcases}{}
      (\Pi_mg)_0 := 0, \\
      (\Pi_mg)_j := \frac1{\sqrt{m+1}}
      \sum_{k=(j-1-m)\vee 0}^{j-1} g_k \frac{\delta W_k}{\sqrt\tau},
      \quad j \geqslant 1.
    \end{numcases}
  \end{subequations}
  Then $ \mathcal R(\{\Pi_m \mid m \in \mathbb N\}) $ is uniformly bounded
  with respect to the time step $ \tau $.
\end{lemma}
\begin{proof} % 2022-10-31 2022-11-01 2022-11-06
  Following the proof of \cite[Theorem~3.1]{Neerven2012}, we only present
  a brief proof. Let $ N $ be an arbitrary positive integer. Let $
  (r_n)_{n=1}^N $ be a sequence of independent symmetric $ \{-1,1\} $-valued
  random variables on a probability space $ (\Omega_r, \mathcal F_r, \mathbb
  P_r) $, and we use $ \mathbb E_r $ to denote the expectation of a random
  variable on this probability space. Fix any sequence $ (g^n)_{n=1}^N $ in $
  \ell_\mathbb F^p(L^p(\Omega;L^q(\mathcal O;H))) $ and any sequence
  $ (m_n)_{n=1}^N $ in $ \mathbb N $.
  By the Kahane-Khintchine inequality (see, e.g.,
  \cite[Theorem~6.2.4]{HytonenWeis2017}) we have
  \begin{equation}
    \label{eq:425}
    \begin{aligned}
      \Big( \mathbb{E}_r \big\| \sum_{n=1}^N r_n g^n \big\|_{\ell^p(L^p(\Omega; L^q(\mathcal{O}; H)))}^2 \Big)^{1/2} \geqslant c \Big( \sum_{j=0}^{\infty} \big\| \sum_{n=1}^N \|g_j^n\|_H^2 \big\|_{L^{p/2}(\Omega; L^{q/2}(\mathcal{O}))}^{p/2} \Big)^{1/p}
    \end{aligned}
  \end{equation}
  Further, we establish the following bound:
  \begin{align*}
& \Big(
  \mathbb{E}_r \big\| \sum_{n=1}^N r_n \Pi_{m_n} g^n \big\|_{\ell^p(L^p(\Omega; L^q(\mathcal{O})))}^2
\Big)^{1/2} \\
    \leqslant{} & c \Big( \mathbb{E}_r \big\| \sum_{n=1}^N r_n \Pi_{m_n} g^n \big\|_{\ell^p(L^p(\Omega; L^q(\mathcal{O})))}^p \Big)^{1/p} \\
    = {} & c \Big(
      \sum_{j=0}^{\infty} \mathbb{E} \mathbb{E}_r \big\|
      \sum_{n=1}^N \frac{r_n}{\sqrt{1+m_n}} \sum_{k=j-1-m_n\vee 0}^{j-1} g_k^n \delta W_k / \sqrt{\tau} \big\|_{L^q(\mathcal{O})}^p 
    \Big)^{1/p} \\
      \leqslant{} & c \Big(
        \sum_{j=0}^{\infty} \mathbb{E}_r \mathbb{E} \Big[ \int_{\mathcal{O}} \Big( \sum_{k=0}^{j-1} \big\| \sum_{n=1}^N \frac{r_n}{\sqrt{1+m_n}} \mathbbm{1}_{j-1-m_n \leqslant k} g_k^n \big\|_H^2 \Big)^{q/2} \, \mathrm{d}\mu \Big]^{p/q} 
      \Big)^{1/p},
      \end{align*}
      where the first inequality utilizes the Kahane-Khintchine inequality,
      and the final inequality employs Lemma \ref{lem:integral}.
      Here, \( \mathbbm{1}_{j-1-m_n \leqslant k} \) denotes the indicator 
      function, which equals \(1\) if \( j-1-m_n \leqslant k \), and \(0\) otherwise.
      Additionally, applying the Kahane-Khintchine inequality, we obtain,
      for any \(j \in \mathbb N\),
      \begin{align*}
& \mathbb{E}_r \Big[
  \int_{\mathcal{O}} \Big(
    \sum_{k=0}^{j-1} \big\| \sum_{n=1}^N \frac{r_n}{\sqrt{1+m_n}} \mathbbm{1}_{j-1-m_n \leqslant k} g_k^n \big\|_H^2
  \Big)^{q/2} \, \mathrm{d}\mu
\Big]^{p/q} \\
        \leqslant{} & c \big\| \sum_{n=1}^N \frac{1}{1+m_n} \sum_{k=j-1-m_n\vee 0}^{j-1} \|g_k^n\|_H^2 \big\|_{L^{q/2}(\mathcal{O})}^{p/2}.
      \end{align*}
      Combining the above estimates, we obtain
      \begin{equation}
        \label{eq:426}
        \begin{aligned}
      & \Big(
        \mathbb E_r \nmB{\sum_{n=1}^N r_n \Pi_{m_n}g^n}_{
          \ell^p(L^p(\Omega;L^q(\mathcal O)))
        }^2
      \Big)^{1/2} \\
      \leqslant{} &
      c \Big(
        \sum_{j=0}^\infty \nmB{
          \sum_{n=1}^N \frac1{1+m_n} \sum_{k=j-1-m_n\vee 0}^{j-1}
          \nm{g_k^n}_H^2
        }_{L^{p/2}(\Omega;L^{q/2}(\mathcal O))}^{p/2}
      \Big)^{1/p}.
    \end{aligned}
  \end{equation}
  For any $ Z \in \ell^{(p/2)'}(L^{(p/2)'}(\Omega;L^{(q/2)'}(\mathcal O))) $,
  we have
  \begin{align*}
    & \sum_{j=0}^\infty \dualB{
      \sum_{n=1}^N \frac1{1+m_n} \sum_{k=j-1-m_n\vee 0}^{j-1}
      \nm{g_k^n}_H^2, \, Z_j
    }_{L^{p/2}(\Omega;L^{q/2}(\mathcal O))} \\
    ={} &
    \sum_{k=0}^\infty \sum_{n=1}^N \sum_{j=k+1}^{k+1+m_n}
    \frac1{1+m_n} \dualB{
      \nm{g_k^n}_H^2, \, Z_j
    }_{L^{p/2}(\Omega;L^{q/2}(\mathcal O))} \\
    ={} &
    \sum_{k=0}^\infty \sum_{n=1}^N \dualB{
      \nm{g_k^n}_H^2, \, \frac1{1+m_n}
      \sum_{j=k+1}^{k+1+m_n} Z_j
    }_{L^{p/2}(\Omega;L^{q/2}(\mathcal O))} \\
    \leqslant{} &
    \sum_{k=0}^\infty \sum_{n=1}^N \dualB{
      \nm{g_k^n}_H^2, \, \sup_{m\in\mathbb N}\frac1{1+m}
      \sum_{j=k+1}^{k+1+m} \snm{Z_j}
    }_{L^{p/2}(\Omega;L^{q/2}(\mathcal O))} \\
    ={} &
    \sum_{k=0}^\infty \dualB{
      \sum_{n=1}^N \nm{g_k^n}_H^2,
      \, \sup_{m\in\mathbb N}\frac1{1+m}
      \sum_{j=k+1}^{k+1+m} \snm{Z_j}
    }_{L^{p/2}(\Omega;L^{q/2}(\mathcal O))} \\
    \leqslant{} &
    \Big(
      \sum_{k=0}^\infty \nmB{
        \sum_{n=1}^N \nm{g_k^n}_H^2
      }_{L^{p/2}(\Omega;L^{q/2}(\mathcal O))}^{p/2}
    \Big)^{2/p} \times {} \\
                & \qquad \Big(
                  \sum_{k=0}^\infty \nmB{
                    \sup_{m\in\mathbb N} \frac1{1+m}
                    \sum_{j=k+1}^{k+1+m}\snm{Z_j}
                  }_{L^{(p/2)'}(\Omega;L^{(q/2)'}(\mathcal O))}^{(p/2)'}
                \Big)^{1/(p/2)'}.
  \end{align*}
  From \cref{lem:Fefferman-Stein} it follows that
  \begin{align*}
    & \sum_{j=0}^\infty \dualB{
      \sum_{n=1}^N \frac1{1+m_n}
      \sum_{k=j-1-m_n\vee 0}^{j-1}
      \nm{g_k^n}_H^2, \, Z_j
    }_{L^{p/2}(\Omega;L^{q/2}(\mathcal O))} \\
    \leqslant{} &
    c\Big(
      \sum_{k=0}^\infty \nmB{
        \sum_{n=1}^N \nm{g_k^n}_H^2
      }_{L^{p/2}(\Omega;L^{q/2}(\mathcal O))}^{p/2}
    \Big)^{2/p} \times {} \\
                & \qquad \nm{Z}_{\ell^{(p/2)'}(L^{(p/2)'}(\Omega;L^{(q/2)'}(\mathcal O)))}.
  \end{align*}
  Invoking the duality principle then yields
  \begin{align*}
    & \Big(
      \sum_{j=0}^\infty \nmB{
        \sum_{n=1}^N \frac1{1+m_n} \sum_{k=j-1-m_n\vee 0}^{j-1}
        \nm{g_k^n}_H^2
      }_{L^{p/2}(\Omega;L^{q/2}(\mathcal O))}^{p/2}
    \Big)^{2/p} \\
    \leqslant{} &
    c\Big(
      \sum_{k=0}^\infty \nmB{
        \sum_{n=1}^N \nm{g_k^n}_H^2
      }_{L^{p/2}(\Omega;L^{q/2}(\mathcal O))}^{p/2}
    \Big)^{2/p},
  \end{align*}
  which, in conjunction with \cref{eq:425,eq:426}, leads to the inequality
  \begin{align*}
    \Bigl(
      \mathbb E_r \nmB{\sum_{n=1}^N r_n \Pi_{m_n}g^n}_{
        \ell^p(L^p(\Omega;L^q(\mathcal O)))
      }^2
    \Bigr)^{1/2} \leqslant
    c \Bigl(
      \mathbb E_r \nmB{\sum_{n=1}^N r_n g^n}_{
        \ell^p(L^p(\Omega;L^q(\mathcal O;H)))
      }^2
    \Bigr)^{1/2}.
  \end{align*}
  Since the above generic positive constant \( c \) is independent of \( \tau \),
  and given that \( N \) is an arbitrary positive integer
  while \( (g^n)_{n=1}^N \) is an arbitrary sequence in
  \( \ell_{\mathbb F}^p(L^p(\Omega; L^q(\mathcal O; H))) \),
  it follows that \( \mathcal R(\{\Pi_m \mid m \in \mathbb N\}) \)
  is uniformly bounded with respect to \( \tau \).
  This completes the proof.
\end{proof}



% % Proof of the 425

  % proof of 426
% \end{proof}



\begin{lemma}
  \label{lem:I-R-bounded}
  $ \mathcal R(\{\mathcal I(z) \mid z \in \Upsilon\}) $
  is uniformly bounded with respect to the time step $ \tau $.
\end{lemma}
\begin{proof}
  Let $ \{\Pi_m \mid m \in \mathbb N \} $ be defined by
  \cref{eq:calII-def}.
  For any $ z \in \Upsilon\setminus\{0\} $ and
  $ g \in \ell_{\mathbb F}^{p}(L^p(\Omega;L^q(\mathcal O;H))) $,
  by \cref{eq:I-def} and the identity
  \(e^{(j-k-1)z} = (1-e^z) \sum_{m=j-k-1}^\infty e^{mz} \),
  we obtain
  \begin{align*} 
    \big(\mathcal I(z)g\big)_j
      &= \sum_{k=0}^{j-1}
      (e^{-z}-1)^{1/2} (1-e^z) \sum_{m=j-k-1}^\infty e^{mz}
      \, g_k \delta W_k / \sqrt\tau \\
      &=
      \sum_{m=0}^\infty (e^{-z}-1)^{1/2} (1-e^z) e^{mz}
      \sum_{k= (j-1-m)\vee 0}^{j-1} g_k \delta W_k / \sqrt\tau \\
      &=
      \sum_{m=0}^\infty (e^{-z}-1)^{3/2} e^{(m+1)z}
      \sum_{k= (j-1-m)\vee 0}^{j-1} g_k \delta W_k / \sqrt\tau \\
      &=
      \sum_{m=0}^\infty \sqrt{1+m} \, (e^{-z}-1)^{3/2}
      e^{(m+1)z} \frac1{\sqrt{1+m}} \sum_{k= (j-1-m)\vee 0}^{j-1}
      g_k \delta W_k / \sqrt\tau \\
      &=
      \sum_{m=0}^\infty \sqrt{1+m} \, (e^{-z}-1)^{3/2}
      e^{(m+1)z} (\Pi_{m}g)_j,
      \quad \forall j \geqslant 1,
  \end{align*}
  Considering further that $ (\mathcal I(z)g)_0 = 0 $
  for all $ z \in \Upsilon\setminus\{0\} $ and
  $ (\Pi_mg)_0 = 0 $ for each $ m \in \mathbb N $, we arrive
  at the representation
  \begin{equation}
    \label{eq:I-IIm}
    \mathcal I(z)  = \sum_{m=0}^\infty
    \sqrt{1+m} \, (e^{-z}-1)^{3/2} e^{(m+1)z} \,
    \Pi_{m}, \quad \forall z \in \Upsilon\setminus\{0\}.
  \end{equation}
  For any $ z \in \Upsilon \setminus \{0\} $, an elementary calculation yields
  \begin{align*} 
    & \sum_{m=0}^\infty
    \sqrt{m+1} \, \snmb{ (e^{-z}-1)^{3/2} e^{(m+1)z} } \\
    ={} &
    \snm{e^{-z}-1}^{3/2}
    \sum_{m=0}^\infty \sqrt{m+1} \, \snm{e^{z}}^{m+1} \\
    \leqslant{} &
    \snm{e^{-z}-1}^{3/2}
    \int_0^\infty \sqrt{x+1} \, \snm{e^z}^{x} \, \mathrm{d}x,
  \end{align*}
  by the fact $ \operatorname{Re}z < 0 $. Upon introducing the change of variable
  $ y = -x\ln\snm{e^z} $, we further obtain
  \begin{align*} 
    & \sum_{m=0}^\infty
    \sqrt{m+1} \, \snmb{ (e^{-z}-1)^{3/2} e^{(m+1)z} } \\
    \leqslant{} &
    \frac{\snm{e^{-z}-1}^{3/2}}{\ln\snm{e^{-z}}}
    \int_0^\infty \sqrt{1+ \frac{y}{\ln\snm{e^{-z}}}} \, e^{-y} \, \mathrm{d}y \\
    ={} &
    \frac{\snm{e^{-z}-1}}{\ln\snm{e^{-z}}}
    \int_0^\infty \sqrt{
      \snm{e^{-z}-1}+ \frac{\snm{e^{-z}-1}}{\ln\snm{e^{-z}}} y 
    } \, \, e^{-y} \, \mathrm{d}y.
  \end{align*}
  Hence, from the inequality
  \[
    \sup_{z \in \Upsilon\setminus\{0\}}
    \frac{\snm{e^{-z}-1}}{\ln\snm{e^{-z}}} < \infty.
  \]
  which is easily verified by definition, we deduce that the supremum
  \begin{align*} 
    \mathcal M := \sup_{z \in \Upsilon \setminus\{0\}} \, \sum_{m=0}^\infty
    \sqrt{m+1} \, \snmb{ (e^{-z}-1)^{3/2} e^{(m+1)z} } 
  \end{align*}
  is finite. For each $ m \in \mathbb N $, define
  \begin{align*}
    \theta_m(z) := \begin{cases}
      \operatorname{Arg}\big( (e^{-z}-1)^{3/2}e^{(m+1)z} \big),
      &\text{ if } z \in \Upsilon\setminus\{0\}, \\
      0 & \text{ if } z = 0.
    \end{cases}
  \end{align*}
  By Kahane's contraction principle (see \cite[Proposition~6.1.13]{HytonenWeis2017})
  and \cref{lem:Pi_m}, we infer that the $ \mathcal R $-bound of the set
  \[
    \left\{e^{i\theta_m(z)}
      \mathcal M \Pi_m \mid z \in \Upsilon, \, m \in \mathbb N
    \right\}
  \]
  remains uniformly bounded with respect to $ \tau $.
  Given that the set $ \{\mathcal I(z) \mid z \in \Upsilon\} $,
  according to \cref{eq:I-IIm} and acknowledging $ \mathcal I(0) = 0 $,
  resides within the convex hull of 
  the aforementioned set, we use the convexity of $ \mathcal R$-bounds (see \cite[Proposition~8.1.21]{HytonenWeis2017}) to conclude that
  the $ \mathcal R$-bound of \( \{\mathcal I(z) \mid z \in \Upsilon\} \) is also uniformly
  bounded with respect to the time step $ \tau $.
  This completes the proof.
\end{proof}

% \begin{remark}
%   By definition, we have
%   \begin{align*}
%     & \lim_{
%       \Upsilon\cap\{z\in\mathbb C \mid \operatorname{Im}z>0\}
%       \ni x+iy \to {0}
%     } \frac{\mathrm{d}x}{\mathrm{d}y} =  -\frac1{\tan\alpha_A}, \\
%     & \lim_{
%       \Upsilon\cap\{z\in\mathbb C \mid \operatorname{Im}z>0\}
%       \ni x+iy \to {0}
%     } \frac{\mathrm{d}x}{\mathrm{d}y} =  \frac1{\tan\alpha_A}.
%   \end{align*}
%   This implies that 
%   \begin{align*}
%     \sup_{z \in \Upsilon\setminus\{0\}}
%   \end{align*}
% \end{remark}



% 2022-10-19 2022-10-29
  %Then we turn to the proof of \cref{eq:Mf2}. Let $ \widetilde A $ be
  %the natural extension of $ A $ to $ \gamma(H,L^q(D)) $.
  %Then $ A $ is a densely defined sectorial operator on $ \gamma(H,L^q(D)) $ satisfying
  %the following conditions:
  %\begin{itemize}
    %\item the spectrum of $ \widetilde A $ is contained in
       %$ \Sigma_{\theta_A} $;
    %\item $ \widetilde A $ admits a bounded $ H^\infty $-calculus of
    %angle less than $ \theta_A $.
  %\end{itemize}


% By \cite[Theorem~10.4.16]{HytonenWeis2017}, we have the following square function estimates.
% 2024-05-07
\begin{lemma}
  \label{lem:varphi+-}
  For any $ g \in \ell^p(L^p(\Omega;L^q(\mathcal O;H))) $, we have
  \begin{align}
    \nm{\varphi_{-}(rA)g}_{
      \gamma(
      L^2(\mathbb R_{+},\frac{\mathrm{d}r}r), \,
      \ell^p(L^p(\Omega;L^q(\mathcal O;H)))
      )
    } \leqslant c \nm{ g }_{
      \ell^p(L^p(\Omega;L^q(\mathcal O;H)))
    }, \label{eq:varphi+A} \\
    \nm{\varphi_{+}(rA)g}_{
      \gamma(
      L^2(\mathbb R_{+},\frac{\mathrm{d}r}r), \,
      \ell^p(L^p(\Omega;L^q(\mathcal O;H)))
      )
    } \leqslant c \nm{ g }_{
      \ell^p(L^p(\Omega;L^q(\mathcal O;H)))
    }. \label{eq:varphi-A}
  \end{align}
\end{lemma}
\begin{lemma} % 2022-11-02
  \label{lem:varphi+-*}
  For any $ g \in \ell^{p'}(L^{p'}(\Omega;L^{q'}(\mathcal O))) $, we have
  \begin{align}
    \nm{\varphi_{-}^*(rA^*)g}_{
      \gamma(L^2(\mathbb R_{+}, \frac{\mathrm{d}r}r), \ell^{p'}(L^{p'}(\Omega;L^{q'}
      (\mathcal O))))
    } \leqslant c \nm{g}_{
      \ell^{p'}(L^{p'}(\Omega;L^{q'}(\mathcal O)))
    }, \label{eq:varphi*+A} \\
    \nm{\varphi_{+}^*(rA^*)g}_{
      \gamma(L^2(\mathbb R_{+}, \frac{\mathrm{d}r}r), \ell^{p'}(L^{p'}(\Omega;L^{q'}
      (\mathcal O))))
    } \leqslant c \nm{g}_{
      \ell^{p'}(L^{p'}(\Omega;L^{q'}(\mathcal O)))
    }. \label{eq:varphi*-A}
  \end{align}
\end{lemma}

% 2023-08-25 2023-08-30 2023-08-31 2023-09-01
% 2024-05-07
\begin{lemma}
  \label{lem:varphi}
  For any $ f \in \ell^p(L^p(\Omega;L^q(\mathcal O;H))) $, we have
  \begin{equation}
    \label{eq:varphi-esti}
    \nm{\Psi f}_{
      \gamma(L^2(\Upsilon_2,\snm{\mathrm{d}z}), \ell^p(L^p(\Omega;L^q(\mathcal O;H))))
    } \leqslant c \nm{f}_{\ell^p(L^p(\Omega;L^q(\mathcal O;H)))}.
  \end{equation}
\end{lemma}
\begin{proof}
  We denote the space $ \ell^p(L^p(\Omega;L^q(\mathcal O;H))) $ by $ X $ for brevity.
  For any $ z \in \Upsilon_2 $, the inequality \cref{eq:upsilon2} implies that
  \[
    \left( \cdot \right)^{1/4} \left( 1 - e^{-z} + \tau \cdot \right)
    \in \mathcal H_0^\infty(\Sigma_{\theta_A}),
  \]
  and so by the theory of the Dunford functional calculus
  (see, e.g., \cite[Chapter~10]{HytonenWeis2017}) we obtain
  \[
    A^{1/4}(1 - e^{-z} + \tau A)^{-1/2} = \frac{1}{2\pi i}
    \int_{\partial\Sigma_{\theta_A}} \lambda^{1/4}(1 - e^{-z} + \tau \lambda)^{-1/2}
    (\lambda - A)^{-1} \, \mathrm{d}\lambda.
  \]
  Substituting this expression into \cref{eq:varphi} yields, for any $ z \in \Upsilon_2 $,
  \begin{align*}
    \Psi(z) &= \frac{1}{(2\pi i)^{3/2}} \int_{\partial\Sigma_{\theta_A}}
    (e^{-z} - 1)^{-1/4} (\tau \lambda)^{1/4} (1 - e^{-z} + \tau \lambda)^{-1/2}
    (\lambda - A)^{-1} \, \mathrm{d}\lambda \\
            &= \int_{\partial\Sigma_{\theta_A}}
            G(z,\lambda) (\tau \lambda)^{1/4}
            (\lambda-A)^{-1} \, \mathrm{d}\lambda,
  \end{align*}
  where 
  \begin{align*}
    G(z,\lambda) := (2\pi i)^{-3/2}
    (e^{-z}-1)^{-1/4}(1-e^{-z}+\tau\lambda)^{-1/2},
    \quad z \in \Upsilon_2, \, \lambda \in \partial\Sigma_{\theta_A}.
  \end{align*}
  Hence,
  \begin{align*}
    & \nm{ \Psi f }_{\gamma(L^2(\Upsilon_2,\snm{\mathrm{d}z}), \, X)} \\
    ={}
    & \nmB{
      \int_{\partial\Sigma_{\theta_A}}
      G(z,\lambda) (\tau \lambda)^{1/4} (\lambda - A)^{-1} f \, \mathrm{d}\lambda 
    }_{\gamma(L^2(\Upsilon_2,\snm{\mathrm{d}z}), \, X)} \\
    \leqslant{}
    & \int_{\partial\Sigma_{\theta_A}}
    \nmB{
      G(z,\lambda) (\tau \lambda)^{1/4} 
      (\lambda - A)^{-1} f
    }_{\gamma(L^2(\Upsilon_2,\snm{\mathrm{d}z}), \, X)}
    \, \snm{\mathrm{d}\lambda} \\
    \stackrel{\text{(i)}}{\leqslant}{}
    & c \int_{\partial\Sigma_{\theta_A}}
    \nmB{
      G(z,\lambda) (\tau \lambda)^{1/4} (\lambda - A)^{-1} f
    }_{
      \ell^p( L^p( \Omega;L^q( \mathcal O; \gamma(L^2(\Upsilon_2,\snm{\mathrm{d}z});H) )))
    } \, \snm{\mathrm{d}\lambda} \\
    \stackrel{\text{(ii)}}{=}{}
    & c \int_{\partial\Sigma_{\theta_A}}
    \nmB{
      (\tau \lambda)^{1/4} (\lambda - A)^{-1} f
    }_{
      \ell^p( L^p( \Omega;L^q( \mathcal O; H)))
    } \nm{G(z,\lambda)}_{L^2(\Upsilon_2,\snm{\mathrm{d}z})}
    \, \snm{\mathrm{d}\lambda} \\
    ={} &
    c \int_{\partial\Sigma_{\theta_A}}
    \snmb{\tau \lambda}^{1/4} 
    \nmb{ (\lambda - A)^{-1} f }_X
    \nm{G(z,\lambda)}_{L^2(\Upsilon_2,\snm{\mathrm{d}z})}
    \, \snm{\mathrm{d}\lambda},
  \end{align*}
  where in (i) we used the $ \gamma $-Fubini isomorphism
  (see \cite[Theorem 9.4.8]{HytonenWeis2017}) and in (ii) we used the 
  \cite[Proposition~9.2.9]{HytonenWeis2017}.
  Since \cref{eq:upsilon2} implies
  \[
    \nmb{ G(z,\lambda) }_{L^2(\Upsilon_2,\snm{\mathrm{d}z})} \leqslant \frac{c}{ 1 + \snm{\tau \lambda}^{1/2} }
    \quad \text{ for all $ \lambda \in \partial\Sigma_{\theta_A} $},
  \]
  it follows that
  \begin{align*} % 2023-08-31
    & \nm{ \Psi f }_{\gamma(L^2(\Upsilon_2, \snm{\mathrm{d}z}), \, X)} \\
    \leqslant{} & c \int_{\partial\Sigma_{\theta_A}}
    \frac{\snm{\tau \lambda}^{1/4}}{1+\snm{\tau\lambda}^{1/2}}
    \nm{(\lambda - A)^{-1}f}_{X} \, \snm{\mathrm{d}\lambda} \\
    \leqslant{} & c \int_{\partial\Sigma_{\theta_A}}
    \frac{\snm{\tau\lambda}^{1/4}}{1+\snm{\tau\lambda}^{1/2}}
    \, \frac{\snm{\mathrm{d}\lambda}}{\snm{\lambda}} \nm{f}_X \quad\text{(by \cref{eq:z-A-inv})} \\
    ={}&
    c \int_{\partial\Sigma_{\theta_A}}
    \frac{\snm{\eta}^{1/4}}{1+\snm{\eta}^{1/2}}
    \, \frac{\snm{\mathrm{d}\eta}}{\snm{\eta}} \, \nm{f}_X
  \end{align*}
  by the change of variable $ \eta:= \tau\lambda $.
  Given that the above integral over \( \partial\Sigma_{\theta_A} \) is convergent,
  we can assert the validity of the desired estimate \eqref{eq:varphi-esti},
  thereby completing the proof.
\end{proof}


% Similarly, we have the following square function estimate for $ \varphi^* $.
\begin{lemma}
  \label{lem:varphi*}
  For any $ g \in \ell^{p'}(L^{p'}(\Omega;L^{q'}(\mathcal O))) $, we have
  \begin{equation}
    \nm{\Psi^* g}_{
      \gamma(L^2(\Upsilon_2,\snm{\mathrm{d}z}), \, \ell^{p'}(L^{p'}(\Omega;L^{q'}(\mathcal O))))
    } \leqslant c \nm{g}_{\ell^{p'}(L^{p'}(\Omega;L^{q'}(\mathcal O)))}.
  \end{equation}
\end{lemma}

\begin{remark}
  For the proof of \cref{lem:varphi+-,lem:varphi+-*}, we refer the reader to
  \cite[Theorem~10.4.16]{HytonenWeis2017}. The proof of \cref{lem:varphi*} is similar to
  that of \cref{lem:varphi}.
\end{remark}



Thirdly, let us introduce a representation formula of the solution to
\cref{eq:Y-def}. For any \( r \geqslant \tau/r_A \), we define 
\begin{align}
  \mathscr I_{+}(r) &:=
  \mathcal I\left( -\log\left(1 + \frac{\tau e^{i\alpha_A}}r\right) \right),
  \label{eq:scrI-} \\
  \mathscr I_{-}(r) &:=
  \mathcal I\left( -\log\left(1 + \frac{\tau e^{-i\alpha_A}}r \right) \right).
  \label{eq:scrI+}
\end{align}
\begin{lemma} % 2022-10-31 2022-11-04 2023-08-26 2023-08-28 2023-09-01
  Let $ Y $ be the solution to \cref{eq:Y-def} with
  \[
    f \in \ell_{\mathbb F}^p(L^p(\Omega;L^q(\mathcal O;H))).
  \]
  Then
  \begin{small}
  \begin{equation}
    \label{eq:A12Y}
    A^{1/2} Y = \int_{\tau/r_A}^\infty \Big(
      \frac{
        \varphi_{+}(rA)^2 \mathscr I_{+}(r)
      }{
        r+\tau e^{i\alpha_A}
      } + \frac{
        \varphi_{-}(rA)^2 \mathscr I_{-}(r) 
      }{
        r+\tau e^{-i\alpha_A}
      }
    \Big) f \, \mathrm{d}r + \int_{\Upsilon_2}
    \Psi(z)^2 \mathcal I(z) f \, \mathrm{d}z.
  \end{equation}
  \end{small}
\end{lemma}
\begin{proof}
  \textit{Step 1}. For any $ \epsilon \in (0,1) $, define
  \[
    A_\epsilon := (\epsilon + A)(1 + \epsilon A)^{-1}.
  \]
  According to \cite[Propositions~3.1.4 and 3.1.9]{Pruss2016},
  the operator \( A_\epsilon \) has the following properties:
  \begin{enumerate}
    \item[(a)] $ A_\epsilon $ has a bounded inverse for all $ \epsilon \in (0,1) $;
    \item[(b)] \(
      \nm{z(z-A_\epsilon)^{-1}}_{\mathcal L(L^q(\mathcal O))}
      \) is uniformly bounded with respect to $ \epsilon \in (0,1) $
      and $ z \in \mathbb C \setminus\{0\} $ satisfying
      $ \snm{\operatorname{Arg}z} \geqslant \theta_A $;
    \item[(c)] 
      for any $ m \in \mathbb N_{>0} $ and for any $ z \in \mathbb C \setminus\{0\} $ with
      $ \snm{\operatorname{Arg}z} \geqslant \theta_A $,
      the operator \( A_\epsilon^{1/2}(z-A_\epsilon)^{-m} \) converges to
      $ A^{1/2}(z-A)^{-m} $ in $\mathcal L(L^q(\mathcal O))$
      as $ \epsilon \to {0+} $.
  \end{enumerate}
  Furthermore, for any \( \epsilon \in (0,1) \) and
  \( z \in \mathbb C \setminus \{0\} \) with
  \( \snm{\operatorname{Arg}z} > \theta_A \), since
  \begin{align*}
    & \nm{z^{1/2}A_\epsilon^{1/2}(z-A_\epsilon)^{-1}}_{\mathcal L(L^q(\mathcal O))} \\
    ={} 
    & \nmB{
      \frac1{2\pi i} \int_{\partial\Sigma_{\theta_A}}
      z^{1/2} \lambda^{1/2} (z-\lambda)^{-1} (\lambda - A_\epsilon)^{-1} \, \mathrm{d}\lambda
    }_{\mathcal L(L^q(\mathcal O))} \\
    \leqslant{}
    & \frac{1}{2\pi} \int_{\partial\Sigma_{\theta_A}}
    \snm{z}^{1/2} \snm{\lambda}^{-1/2} \snm{z-\lambda}^{-1}
    \nm{\lambda(\lambda-A_\epsilon)^{-1}}_{\mathcal L(L^q(\mathcal O))}
    \, \snm{\mathrm{d}\lambda},
  \end{align*}
  we use property (b) to conclude the following property:
  \begin{enumerate}
    \item[(d)] 
      for any given $ \theta \in (\theta_A,\pi) $,
      the norm $ \nm{z^{1/2}A_\epsilon^{1/2}(z-A_\epsilon)^{-1}}_{\mathcal L(L^q(\mathcal O))} $
      is uniformly bounded with respect to \( \epsilon \in (0,1) \) and
      $ z \in \mathbb C \setminus\{0\} $ with $ \snm{\operatorname{Arg}z} \geqslant \theta $.
  \end{enumerate}

  \textit{Step 2}. For any $ \epsilon \in (0,1) $, define $ (Y_{j,\epsilon})_{j\in\mathbb N} $ by
  \[
    \begin{cases}
      Y_{j+1,\epsilon} - Y_{j,\epsilon} +
      \tau A_\epsilon Y_{j+1,\epsilon} = f_j \delta W_j,
      \quad j \in \mathbb N, \\
      Y_{0,\epsilon} = 0.
    \end{cases}
  \]
  By definition we have, for any $ j \geqslant 1 $,
  \begin{align*}
    A^{1/2}Y_j &= \sum_{k=0}^{j-1} A^{1/2}(I + \tau A)^{k-j} f_k \delta W_k, \\
    A_\epsilon^{1/2}Y_{j,\epsilon} &=
    \sum_{k=0}^{j-1} A_{\epsilon}^{1/2}(I + \tau A_\epsilon)^{k-j} f_k \delta W_k.
  \end{align*}
  Hence, using property (c) from Step 1 gives
  \begin{equation}
    \label{eq:Yeps-Y}
    \lim_{\epsilon \to {0+}}
    \nm{A_\epsilon^{1/2} Y_{j,\epsilon} - A^{1/2}Y_j}_{L^q(\mathcal O)}
    = 0, \quad \forall j \geqslant 1.
  \end{equation}


  \textit{Step 3}. Let us prove the decomposition
  \begin{equation}
    \label{eq:Aeps12Y}
    A_\epsilon^{1/2}Y_{j,\epsilon} = I_{j,\epsilon}^{(1)} +
    I_{j,\epsilon}^{(2)} + I_{j,\epsilon}^{(3)},
    \quad \forall j \geqslant 1,
    \, \forall \epsilon \in (0,1),
  \end{equation}
  where
  \begin{align*}
    I_{j,\epsilon}^{(1)}
    &:= \frac1{2\pi i}\int_{
      \Upsilon_1 \cap \{z \in \mathbb C \mid \operatorname{Im}z < 0\}
    } A_\epsilon^{1/2}(1-e^{-z}+\tau A_\epsilon)^{-1}
    \eta_j(z) \, \mathrm{d}z, \\
    I_{j,\epsilon}^{(2)}
    &:= \frac1{2\pi i}\int_{
      \Upsilon_1 \cap \{z \in \mathbb C \mid \operatorname{Im}z > 0\}
    } A_\epsilon^{1/2}(1-e^{-z}+\tau A_\epsilon)^{-1}
    \eta_j(z) \, \mathrm{d}z, \\
    I_{j,\epsilon}^{(3)}
    &:= \frac1{2\pi i} \int_{\Upsilon_2}
    A_\epsilon^{1/2}(1-e^{-z}+\tau A_\epsilon)^{-1} \eta_j(z) \, \mathrm{d}z,
  \end{align*}
  with \( \eta_j(z) := \sum_{k=0}^{j-1} e^{(j-k-1)z} f_k \delta W_k \).
  To this end, fix any $ \epsilon \in (0,1) $ and $ j \geqslant 1 $.
  Using the standard discrete Laplace transform method yields
  \begin{align*}
    Y_{j,\epsilon} &= \frac1{2\pi i} \int_{(1-i\pi, 1+i\pi)} 
    e^{(j-1)z} (1-e^{-z} + \tau A_\epsilon)^{-1}
    \sum_{k=0}^\infty e^{-kz} f_k \delta W_k \, \mathrm{d}z \\
                   &= \frac1{2\pi i} \int_{(1-i\pi, 1+i\pi)} 
                   (1-e^{-z} + \tau A_\epsilon)^{-1}
                   \sum_{k=0}^\infty e^{(j-k-1)z} f_k \delta W_k \, \mathrm{d}z.
  \end{align*}
  Applying Cauchy's theorem, which is justified by property (b) from Step 1, 
  we obtain the identity
  \[
    \int_{(1-i\pi, 1+i\pi)} \left(1 - e^{-z} + \tau A_{\epsilon}\right)^{-1} e^{-mz}
    \, \mathrm{d}z = 0, \quad \forall m \in \mathbb{N}_{>0},
  \]
  which simplifies the expression for \( Y_{j,\epsilon} \) to
  \[
    Y_{j,\epsilon} = \frac{1}{2\pi i} \int_{(1-i\pi, 1+i\pi)}
    \left(1 - e^{-z} + \tau A_{\epsilon}\right)^{-1}
    \eta_j(z) \, \mathrm{d}z,
  \]
  Given that the integrand in the above integral is analytic over
  the closure of the set \( \{z \in \Sigma_{\beta_A} : -\pi \leq \operatorname{Im} z \leq \pi\} \),
  as certified by \cref{eq:lost} and properties (a) and (b) from Step 1, 
  we can invoke Cauchy's theorem to recast \( Y_{j,\epsilon} \) as
  an integral over \( \Upsilon \):
  \[
    Y_{j,\epsilon} = \frac{1}{2\pi i} \int_{\Upsilon}
    \left(1 - e^{-z} + \tau A_{\epsilon}\right)^{-1} \eta_j(z) \, \mathrm{d}z.
  \]
  Applying $ A_{\epsilon}^{1/2} $ to both sides of the above equality and partitoning
  the integral curve $ \Upsilon $ into three parts
  $ \Upsilon_1 \cap \{z \in \mathbb C \mid \operatorname{Im}z < 0\} $,
  $ \Upsilon_1 \cap \{z \in \mathbb C \mid \operatorname{Im}z > 0\} $
  and $ \Upsilon_2 $, we immediately arrive at the desired decomposition \cref{eq:Aeps12Y}.

 % partitioning the integral according to the sign of the imaginary part of \( z \) along contours \( \Upsilon_1 \) and a distinct segment \( \Upsilon_2 \).
  \textit{Step 4}.
  Fix any $ j \geqslant 1 $. For $ I_{j,\epsilon}^{(1)} $ we have
  \begin{align*}
    I_{j,\epsilon}^{(1)}
    \stackrel{\text{(i)}}{=}{}
    & \frac1{2\pi i}\int_0^{r_A} \frac{
      e^{i\alpha_A} A_\epsilon^{1/2}(-re^{i\alpha_A} + \tau A_\epsilon)^{-1}
      \eta_j\big( -\log(1+re^{i\alpha_A}) \big)
    }{1 + re^{i\alpha_A}}
    \, \mathrm{d}r \\
    \stackrel{\text{(ii)}}{=}{}
    & \frac1{2\pi i} \int_{\tau/r_A}^\infty
    \frac{
      e^{i\alpha_A} A_\epsilon^{1/2}
      \big( -e^{i\alpha_A} + rA_\epsilon \big)^{-1}
      \eta_j\left(-\log\left( 1+\frac{\tau e^{i\alpha_A}}r \right)\right)
    }{r+\tau e^{i\alpha_A}}\, \mathrm{d}r,
  \end{align*}
  where in (i) we used the change of variable
  \[
    r = \big(e^{-z} - 1\big)e^{-i\alpha_A}
    \text{ for $ z \in \Upsilon_1 $ with $ \operatorname{Im} z < 0 $}, 
  \]
  and in (ii) we used the change of variable $ r := \tau/r $.
In light of properties (c) and (d) outlined in Step 1, and noting the bound
\[
  \nmB{
    \eta_j\left(-\log\left(1+\frac{\tau e^{i\alpha_A}}{r}\right)\right)
  }_{L^q(\mathcal{O})}
  \leqslant \sum_{k=0}^{j-1} \nm{f_k\delta W_k}_{L^q(\mathcal{O})},
  \quad \text{for all } r \geq \frac{\tau}{r_A},
\]
the application of Lebesgue's dominated convergence theorem yields
\[
  \lim_{\epsilon \to 0^+} I_{j,\epsilon}^{(1)} = 
\frac{1}{2\pi i} \int_{\tau/r_A}^{\infty}
\frac{e^{i\alpha_A} A^{1/2}(-e^{i\alpha_A}+rA)^{-1}
\eta_j\left(-\log\left(1+\frac{\tau e^{i\alpha_A}}{r}\right)\right)}
{r+\tau e^{i\alpha_A}} \, \mathrm{d}r.
\]
Utilizing equations \cref{eq:varphi+,eq:I-def,eq:scrI+}, this limit can be succinctly rewritten as
\[
  \lim_{\epsilon \to 0^+} I_{j,\epsilon}^{(1)} = 
\int_{\tau/r_A}^{\infty} \frac{\varphi_{+}(rA)^2}{r+\tau e^{i\alpha_A}}
\big( \mathscr{I}_{+}(r)f \big)_j \, \mathrm{d}r.
\]
Similarly, for \( I_{j,\epsilon}^{(2)} \), we deduce
\[
  \lim_{\epsilon \to 0^+} I_{j,\epsilon}^{(2)} = 
\int_{\tau/r_A}^{\infty} \frac{\varphi_{-}(rA)^2}{r+\tau e^{-i\alpha_A}}
\big(\mathscr{I}_{-}(r)f\big)_j \, \mathrm{d}r.
\]
Proceeding analogously for \( I_{j,\epsilon}^{(3)} \),
given the analogous bound holds for \( z \in \Upsilon_2 \), we find
\[
  \lim_{\epsilon \to 0^+} I_{j,\epsilon}^{(3)} = 
  \frac{1}{2\pi i} \int_{\Upsilon_2}
  A^{1/2}(1-e^{-z}+\tau A)^{-1} \eta_j(z) \, \mathrm{d}z,
\]
which, with reference to \cref{eq:varphi,eq:I-def}, translates into
\[
  \lim_{\epsilon \to 0^+} I_{j,\epsilon}^{(3)} = 
\int_{\Upsilon_2} \Psi(z)^2 \big( \mathcal{I}(z)f \big)_j \, \mathrm{d}z.
\]
Combining these limits with \cref{eq:Aeps12Y}, we attain
\begin{align*}
\lim_{\epsilon \to 0^+} A_\epsilon^{1/2} Y_{j,\epsilon}
& = \int_{\tau/r_A}^{\infty} \left(
\frac{\varphi_{+}(rA)^2}{r+\tau e^{i\alpha_A}} \big( \mathscr{I}_{+}(r)f \big)_j +
\frac{\varphi_{-}(rA)^2}{r+\tau e^{-i\alpha_A}} \big( \mathscr{I}_{-}(r)f \big)_j
\right) \, \mathrm{d}r \\
& \quad + \int_{\Upsilon_2} \Psi(z)^2 \big( \mathcal{I}(z)f \big)_j \, \mathrm{d}z,
\end{align*}
which, combined with \cref{eq:Yeps-Y}, gives 
\begin{align*}
  A^{1/2} Y_j 
  &= \int_{\tau/r_A}^{\infty} \left(
    \frac{\varphi_{+}(rA)^2}{r+\tau e^{i\alpha_A}} \big( \mathscr{I}_{+}(r)f \big)_j +
    \frac{\varphi_{-}(rA)^2}{r+\tau e^{-i\alpha_A}} \big( \mathscr{I}_{-}(r)f \big)_j
  \right) \, \mathrm{d}r \\
  & \quad {}+ \int_{\Upsilon_2}
  \Psi(z)^2 \big( \mathcal{I}(z)f \big)_j \, \mathrm{d}z.
\end{align*}
Since the above equality holds for all $j \geqslant 1 $ and holds trivially
for \(j=0\), the desired equality \cref{eq:A12Y} then follows.
This completes the proof.
\end{proof}

% This identity, holding for all \( j \geqslant 1 \) and trivially for \( j = 0 \),
% completes the demonstration of the sought-after equality in Equation \eqref{eq:A12Y},
% affirming the intricate connection between the square root of the operator \( A \),
% the sequence \( Y_j \), and the involved transforms and contours.

  % the desired equality \cref{eq:A12Y} then follows
  % by Lebesgue's dominated convergence theorem.

  % Then by the result in Step 1, the desired equality \cref{eq:A12Y} follows
  % by Lebesgue's dominated convergence theorem.
  % This completes the proof.

  % \begin{align*}
  %   \lim_{\epsilon \to {0+}} 
  %   \int_{\tau/r_A}^\infty \frac{
  %     \varphi_{+}(rA_{\epsilon})^2 \mathscr I_{+}(r)
  %   }{r+\tau e^{i\alpha_A}} \, \mathrm{d}r =
  %   \int_{\tau/r_A}^\infty \frac{
  %     \varphi_{+}(rA)^2 \mathscr I_{+}(r)
  %   }{r+\tau e^{i\alpha_A}} \, \mathrm{d}r.
  % \end{align*}
  % \begin{align*}
  %   & \lim_{\epsilon \to {0+}}
  %   (e^{-z}-1)^{-1/2} (\tau A_\epsilon)^{1/2} (1-e^{-z}+\tau A_\epsilon)^{-1} \\
  %   ={} & (e^{-z}-1)^{-1/2} (\tau A)^{1/2} (1-e^{-z}+\tau A)^{-1}
  % \end{align*}
%   To this end, we proceed as follows. By the change of variables, we obtain
%   \[
%     \int_{\Upsilon_1} \Psi(z)^2 \mathcal I(z) \, \mathrm{d}z = I_1 + I_2,
%   \]
%   where
%   \begin{small}
%     \begin{align*} % 2022-10-20 2022-10-29 2022-10-30 2023-08-31
%       I_1 &:=
%       \frac1{2\pi i} \int_0^{r_A}
%       e^{i\alpha_A} (r e^{i\alpha_A})^{-1/2} (\tau A)^{1/2}
%       (-re^{i\alpha_A} \!+\! \tau A)^{-1} \mathcal I(-\log(re^{i\alpha_A}+1))
%       \, \frac{\mathrm{d}r}{1 + re^{i\alpha_A}}, \\
%       I_2 &:=
%       -\frac1{2\pi i} \int_0^{r_A}
%       e^{-i\alpha_A} (re^{-i\alpha_A})^{-1/2} (\tau A)^{1/2}
%       (-re^{-i\alpha_A} + \tau A)^{-1} \mathcal I(-\log(re^{-i\alpha_A}+1))
%       \, \frac{\mathrm{d}r}{1 + re^{-i\alpha_A}}.
%     \end{align*}
%   \end{small}
%   For $ I_1 $ we have
%   \begin{small}
%     \begin{align*} % 2022-10-30 2023-08-31 2023-08-31
%       I_1 &=
%       \frac1{2\pi i} \! \int_{\frac1{r_A}}^\infty
%       e^{i\alpha_A} (e^{i\alpha_A}/r)^{-1/2}  (\tau A)^{1/2}
%       \big(-e^{i\alpha_A}/r \!+\! \tau A\big)^{-1}
%       \mathcal I(-\log(e^{i\alpha_A}/r+1)) \frac{\mathrm{d}r}{
%         r^2\big(1\!+\!\frac1r e^{i\alpha_A}\big)
%       } \\
%       &=
%       \frac1{2\pi i} \int_{\frac1{r_A}}^\infty
%       e^{i\alpha_A} (e^{i\alpha_A}/r)^{-1/2} (\tau A)^{1/2} (-e^{i\alpha_A} + r\tau A)^{-1}
%       \mathcal I(-\log(e^{i\alpha_A}/r+1))
%       \frac{\mathrm{d}r}{r+e^{i\alpha_A}} \\
%       &=
%       \frac1{2\pi i} \int_{\frac1{r_A}}^\infty
%       (r\tau e^{i\alpha_A} A)^{1/2} (-e^{i\alpha_A} + r\tau A)^{-1}
%       \mathcal I(-\log(e^{i\alpha_A}/r+1))
%       \frac{\mathrm{d}r}{r+e^{i\alpha_A}} \\
%       &=
%       \frac1{2\pi i} \int_{\frac{\tau}{r_A}}^\infty
%       (r e^{i\alpha_A} A)^{1/2} (-e^{i\alpha_A} + r A)^{-1}
%       \mathcal I(-\log(\tau e^{i\alpha_A}/r+1))
%       \frac{\mathrm{d}r}{r+\tau e^{i\alpha_A}} \\
%       &=
%       \int_{\frac{\tau}{r_A}}^\infty \varphi_{+}(rA)^2 \mathscr I_{+}(r)
%       \frac{\mathrm{d}r}{r+\tau e^{i\alpha_A}}
%       \quad\text{(by \cref{eq:varphi-,eq:scrI-})}.
%   \end{align*}
% \end{small}
%   For $ I_2 $ we have
%   \begin{small}
%     \begin{align*} % 2022-10-30
%       I_2 &=
%       \frac{-1}{2\pi i} \int_{1/r_A}^\infty
%       e^{-i\alpha_A} (e^{-i\alpha_A}/r)^{-1/2}
%       (\tau A)^{1/2} \big(-e^{-i\alpha_A}/r + \tau A\big)^{-1}
%       \mathcal I(-\log(e^{-i\alpha_A}/r+1)) \frac{\mathrm{d}r}{r^2\big(1+\frac1r e^{-i\alpha_A}\big)} \\
%       &=
%       \frac{-1}{2\pi i} \int_{1/r_A}^\infty
%       e^{-i\alpha_A} (e^{-i\alpha_A}/r)^{-1/2} (\tau A)^{1/2} (-e^{-i\alpha_A} + r\tau A)^{-1}
%       \mathcal I(-\log(e^{-i\alpha_A}/r+1)) \frac{\mathrm{d}r}{r+e^{i\alpha_A}} \\
%       &=
%       \frac{-1}{2\pi i} \int_{1/r_A}^\infty
%       (r\tau e^{-i\alpha_A} A)^{1/2} (-e^{-i\alpha_A} \!+\! r\tau A)^{-1}
%       \mathcal I(-\log(e^{-i\alpha_A}/r+1))
%       \frac{\mathrm{d}r}{r+e^{-i\alpha_A}} \\
%       &=
%       \frac{-1}{2\pi i} \int_{\tau/r_A}^\infty
%       (re^{-i\alpha_A} A)^{1/2} (-e^{-i\alpha_A} + r A)^{-1}
%       \mathcal I(-\log(\tau e^{-i\alpha_A}/r+1))
%       \frac{\mathrm{d}r}{r+\tau e^{-i\alpha_A}} \\
%       &=
%       \int_{\tau/r_A}^\infty \varphi_{-}(rA)^2 \mathscr I_{-}(r)
%       \frac{\mathrm{d}r}{r+\tau e^{-i\alpha_A}}
%       \quad \text{(by \cref{eq:varphi+,eq:scrI+})}.
%     \end{align*}
%   \end{small}
%   Combining the above equalities gives \cref{eq:22} and thus completes the proof.
% % \end{proof}
% % \begin{lemma}
% %   For any $ f \in \ell_\mathbb F^p(L^p(\Omega;L^q(\mathcal O;H))) $, we have
% %   \begin{equation}
% %     \nm{\mathcal I(\cdot) \varphi_1(\cdot) f}_{
% %       \ell^p(L^p(\Omega;L^q(\mathcal O;L^2(\Upsilon_1,\snm{\mathrm{d}z}))))
% %     } \leqslant c \nm{f}_{\ell^p(L^p(\Omega;L^q(\mathcal O;H)))}.
% %   \end{equation}
% % \end{lemma}



% 2023-09-01
Finally, we conclude the proof of \cref{thm:space-regu} as follows.
We introduce the following notation for brevity:
\begin{align*}
  X_0 &:= \ell^p(L^p(\Omega; L^q(\mathcal O; H))), \\
  X_1 &:= \ell^p(L^p(\Omega; L^q(\mathcal O))), \\
  X_2 &:= \ell^{p'}(L^{p'}(\Omega; L^{q'}(\mathcal O))), \\
  X_3 &:= \ell^p\left(L^p\left(\Omega; L^q\left(\mathcal O; L^2\left(\mathbb R_{+}, \frac{\mathrm{d}r}{r}\right)\right)\right)\right), \\
  X_4 &:= \ell^{p'}\left(L^{p'}\left(\Omega; L^{q'}\left(\mathcal O; L^2\left(\mathbb R_{+}, \frac{\mathrm{d}r}{r}\right)\right)\right)\right).
\end{align*}
Fix any $ g \in X_2 $. We have
\begin{align*}  
  & \dual{A^{1/2}Y,g}_{X_1} \\
  ={} &
  \int_{\tau/r_A}^\infty \dualB{
    \frac{r}{r+\tau e^{i\alpha_A}} \varphi_{+}(rA)^2 \mathscr I_{+}(r) f +
    \frac{r}{r+\tau e^{-i\alpha_A}} \varphi_{-}(rA)^2 \mathscr I_{-}(r), \, g
  }_{X_1} \frac{\mathrm{d}r}r \\ 
      & \qquad {} + \int_{\Upsilon_2}
      \dualB{\Psi(z)^2 \mathcal I(z) f, \, g}_{X_1} \, \mathrm{d}z \\
  ={} &
  \int_{\tau/r_A}^\infty \dualB{
    \frac{r}{r+\tau e^{i\alpha_A}} \varphi_{+}(rA) \mathscr I_{+}(r) f, \, \varphi_{+}^*(rA^*)g
  }_{X_1} \frac{\mathrm{d}r}r \\ 
      & \qquad {} + \int_{\tau/r_A}^\infty \dualB{
        \frac{r}{r+\tau e^{-i\alpha_A}} \varphi_{-}(rA) \mathscr I_{-}(r) f, \, \varphi_{-}^*(rA^*) g
      }_{X_1} \frac{\mathrm{d}r}r \\ 
      & \qquad {} + \int_{\Upsilon_2} 
      \dualB{\Psi(z) \mathcal I(z) f, \, \Psi^*(z)g}_{X_1} \, \mathrm{d}z \\
  ={} &
  \int_{\tau/r_A}^\infty \dualB{
    \frac{r}{r+\tau e^{i\alpha_A}} \mathscr I_{+}(r) \varphi_{+}(rA) f, \, \varphi_{+}^*(rA^*)g
  }_{X_1} \frac{\mathrm{d}r}r \\ 
      & \qquad {} + \int_{\tau/r_A}^\infty \dualB{
        \frac{r}{r+\tau e^{-i\alpha_A}} \mathscr I_{-}(r) \varphi_{-}(rA) f, \, \varphi_{-}^*(rA^*) g
      }_{X_1} \frac{\mathrm{d}r}r \\ 
      & \qquad {} + \int_{\Upsilon_2}
      \dualB{\mathcal I(z) \Psi(z) f, \, \Psi^*(z)g}_{X_1} \, \mathrm{d}z,
\end{align*}
where in the first equality we used the equality \cref{eq:A12Y},
in the second equality we used the fact that
\( \varphi_{\pm}^*(rA^*) \) and \( \Psi^*(z) \) are the adjoint operators
of \( \varphi_{\pm}(rA) \) and \( \Psi(z) \), respectively,
and in the third equality we applied the commutative properties between
$ \varphi_{\pm}(rA) $ and $ \mathscr I_{\pm}(r) $,
as well as the commutativity between $ \Psi(z) $ and $ \mathcal I(z) $.
By H\"older's inequality and the uniform boundedness of
$ \frac{r}{r+\tau e^{\pm i\alpha_A}} $ with respect to $ r \in [\tau/r_A,\infty) $,
we further obtain
\begin{align*}
  & \snmB{\dual{A^{1/2}Y,g}_{X_1}} \\
  \leqslant{}
  & \nmB{
    \frac{r}{r+\tau e^{i\alpha_A}}\mathscr I_{+}(r) \varphi_{+}(rA)f
  }_{X_3}
  \nm{\varphi_{+}^*(rA^*)g}_{X_4} \\
  & \,
  {} + \nmB{
    \frac{r}{r+\tau e^{-i\alpha_A}}\mathscr I_{-}(r) \varphi_{-}(rA)f
  }_{X_3} \nm{\varphi_{-}^*(rA^*)g}_{ X_4 } \\
  & \,
  {} + \nm{\mathcal I(\cdot)\Psi(\cdot)f}_{
    \ell^p(L^p(\Omega;L^q(\mathcal O;L^2(\Upsilon_2,\snm{\mathrm{d}z}))))
  } \nm{\Psi^*(\cdot)g}_{
    \ell^{p'}(L^{p'}(\Omega;L^{q'}(\mathcal O;L^2(\Upsilon_2,\snm{\mathrm{d}z}))))
  } \\
  \leqslant{}
  & \nmB{
    \mathscr I_{+}(r) \varphi_{+}(rA)f
  }_{X_3} \nm{\varphi_{+}^*(rA^*)g}_{X_4} + \nmB{
    \mathscr I_{-}(r) \varphi_{-}(rA)f
  }_{X_3} \nm{\varphi_{-}^*(rA^*)g}_{X_4} \\
  & \,
  {} + \nm{\mathcal I(\cdot)\Psi(\cdot)f}_{
    \ell^p(L^p(\Omega;L^q(\mathcal O;L^2(\Upsilon_2,\snm{\mathrm{d}z}))))
  } \nm{\Psi^*(\cdot)g}_{
    \ell^{p'}(L^{p'}(\Omega;L^{q'}(\mathcal O;L^2(\Upsilon_2,\snm{\mathrm{d}z}))))
  } .
\end{align*}
Hence, using \cite[Theorem~9.4.8]{HytonenWeis2017} gives
\begin{align*}
  & \snmB{\dual{A^{1/2}Y,g}_{X_1}} \\
  \leqslant{}&
  c\nmB{
    \mathscr I_{+}(r) \varphi_{+}(rA)f
  }_{
    \gamma(L^2(\mathbb R_{+},\frac{\mathrm{d}r}r), X_1)
  } \nm{\varphi_{+}^*(rA^*)g}_{
    \gamma(L^2(\mathbb R_{+},\frac{\mathrm{d}r}r), X_2)
  } \\
  & \,
  {} + c\nmB{
    \mathscr I_{-}(r) \varphi_{-}(rA)f
  }_{
    \gamma(L^2(\mathbb R_{+},\frac{\mathrm{d}r}r),  X_1)
  } \nm{\varphi_{-}^*(rA^*)g}_{
    \gamma(L^2(\mathbb R_{+},\frac{\mathrm{d}r}r), X_2)
  } \\
  & \,
  {} + c\nm{\mathcal I(\cdot)\Psi(\cdot)f}_{
    \gamma(L^2(\Upsilon_2,\snm{\mathrm{d}z}),X_1)
  } \nm{\Psi^*(\cdot)g}_{
    \gamma(L^2(\Upsilon_2,\snm{\mathrm{d}z}),X_2)
  }.
\end{align*}
Given that the collection of operators
\( \{\mathscr{I}_{+}(r) \mid r \geq \tau/r_A\} \cup \{\mathscr{I}_{-}(r) \mid r \geq \tau/r_A\} \)
is a subset of \( \{\mathcal{I}(z) \mid z \in \Upsilon\} \),
\cref{lem:I-R-bounded} ensures that the \( \mathcal{R} \)-bound of this
operator set is uniformly bounded with respect to \( \tau \).
Therefore, invoking~\cite[Theorems~8.1.3 and 9.5.1]{HytonenWeis2017}
yields
\begin{align*}
  \snmB{\dual{A^{1/2}Y,g}_{X_1}} 
  &\leqslant
  c\nmB{
    \varphi_{+}(rA)f
  }_{
    \gamma(L^2(\mathbb R_+,\frac{\mathrm{d}r}r), X_0)
  } \nm{\varphi_{+}^*(rA^*)g}_{
    \gamma(L^2(\mathbb R_+,\frac{\mathrm{d}r}r), X_2)
  } \\
   & \,
   {} + c\nmB{
     \varphi_{-}(rA)f
   }_{\gamma(L^2(\mathbb R_+,\frac{\mathrm{d}r}r),X_0)}
   \nm{\varphi_{-}^*(rA^*)g}_{
     \gamma(L^2(\mathbb R_+,\frac{\mathrm{d}r}r),X_2)
   } \\
     & \,
     {} + c\nm{\Psi(\cdot)f}_{
       \gamma(L^2(\Upsilon_2,\snm{\mathrm{d}z}),X_0)
     } \nm{\Psi^*(\cdot)g}_{
       \gamma(L^2(\Upsilon_2,\snm{\mathrm{d}z}),X_2)
     }.
\end{align*}
This inequality, combined with \cref{lem:varphi+-,lem:varphi+-*,lem:varphi,lem:varphi*},
leads to
\begin{align*}
  \snmB{ \dual{A^{1/2}Y,g}_{X_1} } 
  \leqslant c  \nm{f}_{X_0} \nm{g}_{X_2}.
\end{align*}
Since \( X_2 \) is the dual of \( X_1 \) and \( g \) is arbitrarily chosen
from \( X_2 \), the desired estimate \eqref{eq:space-regu} is established,
thereby completing the proof of Theorem \ref{thm:space-regu}.






% \begin{align*}
%   & \dual{A^{1/2}Y_j,v}_{L^p(\Omega;L^q(\mathcal O))} \\
%   ={} & \frac1{2\pi i} \int_\Upsilon 
%   \dualB{\Psi(z)^2 \Pi^j(z) f, \, g}_{L^p(\Omega;L^q(\mathcal O))} \, \mathrm{d}z \\
%   ={} & \frac1{2\pi i} \int_\Upsilon 
%   \dualB{\Psi(z) \Pi^j(z) f, \, \Psi^*(z)g}_{
%     L^p(\Omega;L^q(\mathcal O))
%   } \, \mathrm{d}z \\
%   ={} & \frac1{2\pi i} \int_\Upsilon 
%   \dualB{\Pi^j(z) \Psi(z)f, \, \Psi^*(z)g}_{
%     L^p(\Omega;L^q(\mathcal O))
%   } \, \mathrm{d}z.
% \end{align*}
% It follows that
% \begin{align*}
%   & \snmB{ \dual{A^{1/2-1/p}Y_j, g}_{L^p(\Omega;L^q(\mathcal O))} } \\
%   \leqslant{}& \frac1{2\pi}
%   \nm{\Pi^j(\cdot) \Psi(\cdot) f}_{L^p(\Omega;L^q(\mathcal O;L^2(\Upsilon,\snm{\mathrm{d}z})))}
%   \nm{\Psi^*(\cdot) g}_{L^{p'}(\Omega;L^{q'}(\mathcal O;L^2(\Upsilon,\snm{\mathrm{d}z})))} \\
%   \stackrel{\text{(i)}}{\leqslant}{}& c
%   \nm{\Pi^j(\cdot) \Psi(\cdot) f}_{\gamma(L^2(\Upsilon,\snm{\mathrm{d}z}), L^p(\Omega;L^q(\mathcal O)))}
%   \nm{\Psi^*(\cdot) g}_{\gamma(L^2(\Upsilon,\snm{\mathrm{d}z}),L^{p'}(\Omega;L^{q'}(\mathcal O)))} \\
%   \stackrel{\text{(ii)}}{\leqslant}{}& c
%   \nm{\Psi(\cdot) f}_{\gamma(L^2(\Upsilon,\snm{\mathrm{d}z}), \ell^p(L^p(\Omega;L^q(\mathcal O;H))))}
%   \nm{\Psi^*(\cdot) g}_{\gamma(L^2(\Upsilon,\snm{\mathrm{d}z}),L^{p'}(\Omega;L^{q'}(\mathcal O)))} \\
%   \stackrel{\text{(ii)}}{\leqslant}{}& c
%   % \nm{\Psi(\cdot) f}_{\ell^p(L^p(\Omega;L^q(\mathcal O;L^2(\Upsilon,\snm{\mathrm{d}z}))))}
%   \nm{(\mathcal I(\cdot) f)_j}_{\gamma(L^2(\Upsilon,\snm{\mathrm{d}z}), L^p(\Omega;L^q(\mathcal O)))}
%   \nm{g}_{L^{p'}(\Omega;L^{q'}(\mathcal O))} \\
%   % \nm{\Psi^*(\cdot) g}_{\ell^{p'}(L^{p'}(\Omega;L^{q'}(\mathcal O;L^2(\Upsilon,\snm{\mathrm{d}z}))))} \\
%   % \stackrel{\text{(ii)}}{\leqslant}{}& c
%   % \nm{\Psi(\cdot) f}_{\gamma(L^2(\Upsilon,\snm{\mathrm{d}z}), \ell^p(L^p(\Omega;L^q(\mathcal O;H))))}
%   % \nm{\Psi^*(\cdot) g}_{\gamma(L^2(\Upsilon,\snm{\mathrm{d}z}),\ell^{p'}(L^{p'}(\Omega;L^{q'}(\mathcal O))))} \\
%   % \stackrel{\text{(iii)}}{\leqslant}{}& c
%   % \nm{\Psi(\cdot) f}_{\gamma(L^2(\Upsilon,\snm{\mathrm{d}z}), \ell^p(L^p(\Omega;L^q(\mathcal O;H))))}
%   % \nm{\Psi^*(\cdot) g}_{\gamma(L^2(\Upsilon,\snm{\mathrm{d}z}),\ell^{p'}(L^{p'}(\Omega;L^{q'}(\mathcal O))))} \\
%   \stackrel{\text{(iii)}}{\leqslant}{}& c
%   \nm{f}_{\ell^p(L^p(\Omega;L^q(\mathcal O;H)))}
%   \nm{g}_{L^{p'}(\Omega;L^{q'}(\mathcal O))},
% \end{align*}



% For any $ j \geqslant 1 $ and $ z \in \Upsilon $, define
% \[
%  \Pi^j(z) \in \mathcal L\big( \ell_\mathbb F^p(L^p(\Omega;L^q(\mathcal O;H))), \,
%  L^p(\Omega;L^q(\mathcal O)) \big) 
% \] 
% by
% \[
%   \Pi^j(z) f := (\mathcal I(z) f)_j,
%   \quad \forall f \in \ell_\mathbb F^p(L^p(\Omega;L^q(\mathcal O;H))).
% \] 

% Let
% \[
%   \rho := \sup_{z \in \Upsilon} \, \snm{e^{z}}.
% \]
% \begin{align*}
%   & \Big(
%     \mathbb E_r \nmB{\sum_{n=1}^N r_n \Pi^j(z_n)g^n}_{L^p(\Omega;L^q(\mathcal O))}^2
%   \Big)^{1/2} \\
%   \leqslant{} & c \Big(
%     \mathbb E_r \nmB{\sum_{n=1}^N r_n \Pi^j(z_n)g^n}_{L^p(\Omega;L^q(\mathcal O))}^p
%   \Big)^{1/p} \\
%   \leqslant{} &
%   c \Big(
%     \mathbb E_r \nmB{
%       \Big(
%         \sum_{k=0}^{j-1}  \nmB{
%           \sum_{n=1}^N r_n(e^{-z_n}-1)^{1/2}
%           e^{(j-k-1)z_n} g_k^n
%         }_H^2
%       \Big)^{1/2}
%     }_{L^{p}(\Omega;L^{q}(\mathcal O))}^{p}
%   \Big)^{1/p} \\
%   \leqslant{}&
%   c\Big(
%     \mathbb E \Big(
%       \int_{\mathcal O} \mathbb E_r \Big(
%         \sum_{k=0}^{j-1} \nmB{
%           \sum_{n=1}^N r_n(e^{-z_n}-1)^{1/2} e^{(j-k-1)z_n} g_k^n
%         }_H^2
%       \Big)^{q/2}
%       \, \mathrm{d}\mu
%     \Big)^{p/q}
%   \Big)^{1/p} \\
%   \leqslant{}&
%   c\Big(
%     \mathbb E \Big(
%       \int_{\mathcal O} \Big(
%         \mathbb E_r
%         \sum_{k=0}^{j-1} \nmB{
%           \sum_{n=1}^N r_n(e^{-z_n}-1)^{1/2} e^{(j-k-1)z_n} g_k^n
%         }_H^2
%       \Big)^{q/2}
%       \, \mathrm{d}\mu
%     \Big)^{p/q}
%   \Big)^{1/p} \\
%   ={}&
%   c\Big(
%     \mathbb E \Big(
%       \int_{\mathcal O} \Big(
%         \sum_{k=0}^{j-1} \sum_{n=1}^N \nmB{
%           (e^{-z_n}-1)^{1/2} e^{(j-k-1)z_n} g_k^n
%         }_H^2
%       \Big)^{q/2}
%       \, \mathrm{d}\mu
%     \Big)^{p/q}
%   \Big)^{1/p} \\
%   \leqslant{}&
%   c\Big(
%     \mathbb E \Big(
%       \int_{\mathcal O} \Big(
%         \sum_{k=0}^{j-1} \sum_{n=1}^N \nmB{
%           e^{(j-k-1)z_n} g_k^n
%         }_H^2
%       \Big)^{q/2}
%       \, \mathrm{d}\mu
%     \Big)^{p/q}
%   \Big)^{1/p} \\
%   \leqslant{}&
%   c\Big(
%     \mathbb E \Big(
%       \int_{\mathcal O} \Big(
%         \sum_{k=0}^{j-1} \rho^{2(j-k-1)} \sum_{n=1}^N \nmb{ g_k^n }_H^2
%       \Big)^{q/2}
%       \, \mathrm{d}\mu
%     \Big)^{p/q}
%   \Big)^{1/p} \\
%   \leqslant{}&
%   c \Big(
%     \mathbb E \Big(
%       \sum_{k=0}^{j-1} \rho^{2(j-k-1)} \Big(
%         \int_{\mathcal O} \Big(\sum_{n=1}^N \nmb{ g_k^n }_H^2\Big)^{q/2}
%         \, \mathrm{d}\mu
%       \Big)^{2/q}
%     \Big)^{p/2}
%   \Big)^{1/p} \\
%   ={}&
%   c \Big(
%     \mathbb E \Big(
%       \sum_{k=0}^{j-1} \rho^{2(j-k-1)} \Big(
%         \nmB{\sum_{n=1}^N \nmb{ g_k^n }_H^2}_{L^{q/2}(\mathcal O)}
%     \Big)^{p/2}
%   \Big)^{1/p} \\
%   \leqslant{}&
%   c \Big(
%     \sum_{k=0}^{j-1} \rho^{2(j-k-1)}
%     \Big(\mathbb E \nmB{\sum_{n=1}^N \nmb{ g_k^n }_H^2}_{L^{q/2}(\mathcal O)}^{p/2}\Big)^{2/p}
%   \Big)^{1/2} \\
%   ={}&
%   c \Big(
%     \sum_{k=0}^{j-1} \rho^{2(j-k-1)}
%     \nmB{\sum_{n=1}^N \nmb{ g_k^n }_H^2}_{L^{p/2}(\Omega;L^{q/2}(\mathcal O))}
%   \Big)^{1/2} 
% \end{align*}
% \begin{align*}
%   & \Big(
%     \sum_{k=0}^{j-1} \rho^{2(j-k-1)}
%     \nmB{\sum_{n=1}^N \nmb{ g_k^n }_H^2}_{L^{p/2}(\Omega;L^{q/2}(\mathcal O))}
%   \Big)^{1/2}  \\
%   \leqslant{}&
%   \Big(
%     \sum_{k=0}^{j-1} \rho^{(j-k-1)2p/(p-2)}
%   \Big)^{1/2-1/p}
%   \Big(
%     \sum_{k=0}^{j-1} 
%     \nmB{\sum_{n=1}^N \nmb{ g_k^n }_H^2}_{L^{p/2}(\Omega;L^{q/2}(\mathcal O))}^{p/2}
%   \Big)^{1/p}  \\
% \end{align*}





% for all $ f \in \ell_\mathbb F^p(L^p(\Omega;L^q(\mathcal O;H))) $.

%\subsection{Proof of \texorpdfstring{\cref{thm:time-inf-regu}}{}}
%Define
%\[
  %B_\tau Y := (Y_{j+1} - Y_j)_{j \in \mathbb N}.
%\]
%We have that
%\[
  %\mathcal F \big( h(B_\tau) Y \big)(x) =
  %h(e^{ix}-1) (\mathcal FY)(x)
%\]

%Define
%\begin{align} % 2022-10-31
  %\phi_{+}(z) &:= (-2\pi i)^{-1/2} e^{i\theta_A(1/2+1/p)/2}
  %z^{1/4-1/(2p)} (e^{i\theta_A} - z)^{-1/2}, \\
  %\phi_{-}(z) &:= (2\pi i)^{-1/2} e^{-i\theta_A(1/2+1/p)/2}
  %z^{1/4-1/(2p)} (e^{-i\theta_A} - z)^{-1/2}.
%\end{align}
%Fix $ j \in \mathbb N_{>0} $. Define
%\[ % 2022-10-31
  %\{\mathscr I_j(\xi)\mid \xi \in \Gamma_{\theta_A}\} \subset
  %\mathcal L\big(
    %\ell^p(L^q(\mathcal O)), \, L^q(\mathcal O)
  %\big)
%\]
%as follows: for any $ \xi \in \Gamma_{\theta_A} $ and
%$ g \in \ell^p(L^q(\mathcal O)) $, let
%\begin{equation} % 2022-10-31
  %\label{eq:II-def}
  %\mathscr I_j(\xi)g:= \sum_{k=0}^{j-1}
  %\xi^{1/2-1/p} (1+\xi)^{k+1-j} g_k.
%\end{equation}
%We have $ \mathbb P $-a.s.
%\begin{align*} % 2022-10-31
  %A^{1/2-1/p} Y_j = \tau^{1/p} \int_{
    %\frac{2\tau}{\snm{\Gamma_{\theta_A}}}
  %}^\infty \Big(
    %\frac{
      %\phi_{-}(rA)^2 \mathcal I(\tau e^{-i\theta_A}/r)
    %}{
      %r + \tau e^{-i\theta_A}
    %} + \frac{
      %\phi_{+}(rA)^2 \mathcal i(\tau e^{i\theta_A}/r)
    %}{
      %r + \tau e^{i\theta_A}
    %}
  %\Big) F \, \mathrm{d}r,
%\end{align*}
%where $ F \in L^p(\Omega;\ell^p(L^q(\mathcal O))) $ is defined by
%\begin{equation} % 2022-10-31
  %F = (f_j \delta W_j/\sqrt\tau)_{j\in \mathbb N},
  %\quad \text{ $\mathbb P$-a.s.}
%\end{equation}
%Hence, for any $ Z \in \ell^{q'}(\mathcal O))) $ we have $ \mathbb P $-a.s.
%\begin{equation}
  %\dualb{A^{1/2-1/p}Y_j, Z}_{L^q(\mathcal O)} =
  %I_1 + I_2,
%\end{equation}
%where
%\begin{align*}
  %I_1 &:= \tau^{1/p} \int_{\frac{2\tau}{\snm{\Gamma_{\theta_A}}}}^\infty
  %\dualB{
    %\frac{r}{r+\tau e^{-i\theta_A}} \phi_{-}(rA)^2
    %\mathscr I_j(\tau e^{-i\theta_A}/r) f, \, Z
  %}_{L^q(\mathcal O)} \frac{\mathrm{d}r}r, \\
  %I_2 &:= \tau^{1/p} \int_{\frac{2\tau}{\snm{\Gamma_{\theta_A}}}}^\infty
  %\dualB{
    %\frac{r}{r+\tau e^{i\theta_A}} \phi_{-}(rA)^2
    %\mathscr I_j(\tau e^{i\theta_A}/r) f, \, Z
  %}_{L^q(\mathcal O)} \frac{\mathrm{d}r}r, \\
%\end{align*}
%For $ I_1 $ we have $ \mathbb P $-a.s.
%\begin{align*}
  %I_1 & =
  %\tau^{1/p} \int_{\frac{2\tau}{\snm{\Gamma_{\theta_A}}}}^\infty
%\end{align*}
%\begin{align*} % 2022-10-29
  %& \snmB{
    %\dualb{(A^{1/2-1/p}Y_j), \,  Z}_{\ell^{q}(\mathcal O)}
  %} \\
  %={} &
  %\frac1{2\pi} \snmB{
    %\dualB{
      %\int_{\Upsilon}
      %\tau^{1/4+1/(2p)}
      %N(z) \big( \mathcal I(z) N(z)f \big)_j
      %\, \mathrm{d}z, \, Z
    %}_{\ell^{q}(\mathcal O)}
  %} \\
  %={} &
  %\frac1{2\pi} \snmB{
    %\int_{\Upsilon} \dualB{
      %\tau^{1/4+1/(2p)}
      %N(z) \big( \mathcal I(z) N(z)f \big)_j, \, Z
    %}_{\ell^{q}(\mathcal O)} \, \mathrm{d}z
  %} \\
  %={} &
  %\frac1{2\pi} \snmB{
    %\int_{\Upsilon} \dualB{
      %\tau^{1/4+1/(2p)}
      %\big( \mathcal I(z) N(z)f \big)_j,
      %\, N^*(z)Z
    %}_{\ell^{q}(\mathcal O)} \, \mathrm{d}z
  %} \\
  %\leqslant{} &
  %\frac1{2\pi} \tau^{1/4+1/(2p)} \nm{\mathcal INf}_{
    %L^q(\mathcal O;L^2(\Upsilon,\snm{\mathrm{d}z}))
  %} \nm{N^*Z}_{
    %\ell^{q'}(\mathcal O;L^2(\Upsilon,\snm{\mathrm{d}z}))
  %} \\
  %\leqslant{} &
  %c\nm{\mathcal INf}_{
    %L^q(\mathcal O;L^2(\Upsilon,\snm{\mathrm{d}z}))
  %} \nm{Z}_{
    %\ell^{q'}(\mathcal O)
  %} \quad \text{(by \cref{eq:M2*f})}.
%\end{align*}
%Since $ \ell^{q'}(\mathcal O) $ is the dual space of $ L^q(\mathcal O) $,
%it follows that
%\begin{align*}
  %\nm{A^{1/2-1/p}Y_j}_{L^q(\mathcal O)}
  %&\leqslant
  %c \nm{\mathcal I N f}_{
    %L^q(\mathcal O;L^2(\Upsilon,\snm{\mathrm{d}z}))
  %}             \\
  %&\leqslant
  %c\mathcal R\{\mathcal I(z): \, z \in \Upsilon\}
  %\nm{(Mf)_j}_{
    %L^q(\mathcal O;L^2(\Upsilon,\snm{\mathrm{d}z}))
  %} \\
  %&\leqslant
  %c\mathcal R\{\mathcal I(z): \, z \in \Upsilon\}
  %\nm{Mf}_{
    %\ell^p(L^q(\mathcal O;L^2(\Upsilon,\snm{\mathrm{d}z})))
  %},
%\end{align*}
%by \cite[Theorems~9.4.8 and 9.5.1]{HytonenWeis2017}. It follows that
%\begin{align*}
  %\sup_{j \in \mathbb N} \nm{A^{1/2-1/p}Y_j}_{L^q(\mathcal O)}
  %\leqslant c \nm{\mathcal I \mathcal M f}_{
    %\ell^p(L^q(\mathcal O;L^2(\Upsilon,\snm{\mathrm{d}z})))
  %}.
%\end{align*}

\section{Convergence estimate}
\label{sec:convergence}
% 2022-10-26 2022-11-04 2022-11-09
This section considers the convergence of the Euler scheme for the
stochastic linear evolution equation:
\begin{equation} % 2022-10-22 2022-10-23 2022-10-26 2022-11-09
  \label{eq:y}
  \begin{cases}
    \mathrm{d}y(t) + Ay(t) \, \mathrm{d}t =
    f(t) \, \mathrm{d}W(t), \quad 0 \leqslant t \leqslant T, \\
    y(0) = 0,
  \end{cases}
\end{equation}
where $ 0 < T < \infty $ and $ f $ is given. The mild solution of
the above equation is given by (see, e.g., \cite{Neerven2012b})
\begin{equation} % 2022-10-21 2022-10-23 2022-10-26 2022-11-09
  \label{eq:y-mild}
  y(t) = \int_0^t S(t-s) f(s) \, \mathrm{d}W(s),
  \quad t \in [0,T],
\end{equation}
where $ S(\cdot) $ denotes the analytic semigroup generated by $ {-A} $.


Let $ J $ be a positive integer and define the time step $ \tau := T/J $.
We introduce the temporal nodes $ t_j = j \tau $ for each $ 0 \leqslant j \leqslant J $, 
and the increment $ \delta W_j $ of the Wiener process $ W $ as
$ \delta W_j := W(t_{j+1}) - W(t_j) $
for each $ 0 \leqslant j < J $, consistent with the definitions
in \cref{sec:stability}. For any $ p, q \in [1,\infty) $, let 
\( L_{\mathbb F,\tau}^p((0,T)\times\Omega;L^q(\mathcal O;H)) \) denote
the space of all functions \( f: [0,T]\times\Omega\to L^q(\mathcal O;H) \)
that are piecewise constant over each time interval $ [t_j,t_{j+1}) $ and 
satisfy
\[
  f(t_j) \in L^p(\Omega, \mathcal{F}_{t_j}, \mathbb{P}; L^q(\mathcal{O}; H))
\]
for all \( 0 \leq j < J \).
The main result of this section is the following error estimate.
\begin{theorem}
  \label{thm:conv}
  Let $ p,q \in [2,\infty) $. Assume that $ A $ is a densely defined
  sectorial operator on $ L^q(\mathcal O) $, which possesses a bounded inverse,
  and that the set
  \(
  \left\{ z(z-A)^{-1} \mid z \in \mathbb C \setminus \overline{\Sigma_{\theta_A}} \right\}
  \) is $ \mathcal R $-bounded in $ \mathcal L(L^q(\mathcal O)) $,
  where $ \theta_A \in (0,\pi/2) $. Let $ y $ be the mild solution to
  equation \cref{eq:y} with
  \[
    f \in L_{\mathbb F,\tau}^p((0,T)\times\Omega;L^q(\mathcal O;H)).
  \]
  Define $ (Y_j)_{j=0}^J \subset L^p(\Omega;L^q(\mathcal O)) $ by
  \begin{subequations}
    \label{eq:Y-J-def}
    \begin{numcases}{}
      Y_{j+1} - Y_j + \tau A Y_{j+1} = f(t_j) \delta W_j,
      \quad 0 \leqslant j < J, \\
      Y_0 = 0.
    \end{numcases}
  \end{subequations}
  Then
  \begin{equation} 
    \label{eq:conv}
    \Big(
      \sum_{j=0}^{J-1} \nm{y-Y_j}_{
        L^p((t_j,t_{j+1}) \times \Omega; L^q(\mathcal O))
      }^p
    \Big)^{1/p} \leqslant
    c \tau^{1/2} \nm{f}_{L^p((0,T) \times \Omega;L^q(\mathcal O;H))}.
  \end{equation}
\end{theorem}






% \begin{remark}
%   Assume that the conditions in \cref{thm:space-regu} hold. Let $ f \in  Since $ \ell^p(L^p(\Omega;L^q(\mathcal O))) $
%   is reflexive,
% \end{remark}


% 2022-11-04
\begin{remark}
  Since equation \cref{eq:y} does not possess the regularity estimate
  (see \cite[pp.~1382]{Neerven2012b})
  \[ % 2022-10-25
    \nm{y}_{H^{1/2,p}(0,T;L^p(\Omega;L^q(\mathcal O)))}
    \leqslant c \nm{f}_{L^p((0,T)\times\Omega; L^q(\mathcal O;H))},
  \]
  it appears to be unusual that in the error estimate \cref{eq:conv}
  the convergence rate $ O(\tau^{1/2}) $ is still available.
\end{remark}



The remaining task of this section is to prove the aforementioned theorem.
For the sake of convenience, throughout the remainder of this section,
we will consistently assume that the conditions outlined in Theorem \cref{thm:conv}
are satisfied. We define the space
\[
  {}^0 H^{1,p'}(0,T; L^{q'}(\mathcal{O})) := \left\{
    v \in H^{1,p'}(0,T; L^{q'}(\mathcal{O})) \,\middle|\, v(T) = 0
  \right\}
\]
and endow it with the norm
\[
  \nm{v}_{{}^0 H^{1,p'}(0,T; L^{q'}(\mathcal{O}))} := \nm{v'}_{L^{p'}(0,T; L^{q'}(\mathcal{O}))}
  \quad \forall v \in {}^0 H^{1,p'}(0,T; L^{q'}(\mathcal{O})),
\]
where \( H^{1,p'}(0,T; L^{q'}(\mathcal{O})) \) denotes a standard vector-valued Sobolev
space. As discussed in \cref{ssec:space-regu}, let \( A^* \) denote the dual operator of \( A \),
which is a sectorial operator on \( L^{q'}(\mathcal{O}) \). Let \( D(A^*) \) denote
the domain of \( A^* \), equipped with the standard graph norm
\[
  \nm{v}_{D(A^*)} := \nm{A^*v}_{L^{q'}(\mathcal{O})},
  \quad \forall v \in D(A^*).
\]
Additionally, by \cite[Proposition~8.4.1]{HytonenWeis2017}, the set of operators
\[
  \left\{ z(z - A^*)^{-1} \mid z \in \mathbb{C} \setminus \overline{\Sigma_{\theta_A}} \right\}
\]
is \( \mathcal{R} \)-bounded in \( \mathcal{L}(L^{q'}(\mathcal{O})) \).



%By \cite[Theorem~4.2]{Weis2001}, we have
%\begin{equation}
  %\nm{w}_{{}^0W^{1,p}(0,T;L^q(\mathcal O))} +
  %\nm{Aw}_{\ell^p(0,T;L^q(\mathcal O))} \leqslant
  %c \nm{w' + Aw}_{\ell^p(0,T;L^q(\mathcal O))}
%\end{equation}



% 2022-10-25 2022-10-26 2022-11-04 2022-11-09
\begin{lemma}[see \cite{Weis2001}]
  \label{lem:z-regu}
  For any $ g \in L^{p'}(0,T;L^{q'}(\mathcal O)) $, there exists a unique
  \[
    z \in {}^0H^{1,p'}(0,T;L^{q'}(\mathcal O)) \cap
    L^{p'}(0,T;D(A^*))
  \]
  such that
  \[ % 2022-10-26
    -z' + A^*z = g \quad \text{ in }
    L^{p'}(0,T;L^{q'}(\mathcal O)).
  \]
  Moreover,
  \[
    \nm{z'}_{L^{p'}(0,T;L^{q'}(\mathcal O))} +
    \nm{A^*z}_{L^{p'}(0,T;L^{q'}(\mathcal O))}
    \leqslant c \nm{g}_{L^{p'}(0,T;L^{q'}(\mathcal O))}.
  \]
\end{lemma}

% 2022-10-25 2022-10-26 2022-11-04
\begin{lemma}
  \label{lem:z-Z}
  For any
  \[
    z \in {}^0H^{1,p'}(0,T;L^{q'}(\mathcal O)) \cap
    L^{p'}(0,T;D(A^*)),
  \]
  define $ (Z_j)_{j=0}^J \subset L^{q'}(\mathcal O) $ by
  \begin{subequations}
    \label{eq:Z}
    \begin{numcases}{}
      Z_j - Z_{j+1} + \tau A^*Z_j = \int_{t_j}^{t_{j+1}}
      g(t) \, \mathrm{d}t, \quad 0 \leqslant j < J, \\
      Z_J = 0,
    \end{numcases}
  \end{subequations}
  where $ g := -z' + A^*z $. Then
  \begin{equation}
    \label{eq:Z-conv}
    \Big(
      \sum_{j=1}^{J-1} \tau
      \nm{z(t_j) - Z_j}_{L^{q'}(\mathcal O)}^{p'}
    \Big)^{1/p'} \leqslant c
    \tau \nm{g}_{L^{p'}(0,T;L^{q'}(\mathcal O))}.
  \end{equation}
\end{lemma}
%\begin{lemma}
  %\label{lem:z-Z}
  %Under the condition of \cref{thm:conv},
  %for any
  %\[
    %g \in L^{p'}((0,T);\ell^{q'}(\mathcal O)),
  %\]
  %let
  %\[
    %z(t) := \int_t^T S^*(s-t) g(s) \, \mathrm{d}s,
    %\quad 0 \leqslant t \leqslant T,
  %\]
  %and define $ (Z_j)_{j=0}^J \subset L^{q'}(\mathcal O) $ by
  %\begin{subequations}
    %\begin{numcases}{}
      %Z_j - Z_{j+1} + \tau A^*Z_j = \int_{t_j}^{t_{j+1}}
      %g(t) \, \mathrm{d}t, \quad 0 \leqslant j < J, \\
      %Z_J = 0.
    %\end{numcases}
  %\end{subequations}
  %%where $ A^* $ is the dual of $ A $ and $ S^*(\cdot) $ is the analytic
  %%semigroup generated by $ -A^* $. Then
  %Then
  %\begin{equation}
    %\label{eq:Z-conv}
    %\Big(
      %\sum_{j=0}^{J-1} \tau
      %\nm{z(t_j) - Z_j}_{L^{q'}(\mathcal O)}^{p'}
    %\Big)^{1/p'} \leqslant c
    %\tau \nm{g}_{L^{p'}(0,T;\ell^{q'}(\mathcal O))}.
  %\end{equation}
%\end{lemma}
\begin{proof}
  It is standard that
  \[
    \Big(
      \sum_{j=0}^{J-1} \nm{z - z(t_{j+1})}_{
        L^{p'}(t_j,t_{j+1};L^{q'}(\mathcal O))
      }^{p'}
    \Big)^{1/p'} \leqslant c \tau \nm{z'}_{L^{p'}(0,T;L^{q'}(\mathcal O))}.
  \]
  Following a similar approach to that in~\cite[Theorem III]{Kemmochi2018}, we 
  can establish the inequality
  \begin{align*}
    & \Big(
      \sum_{j=0}^{J-1} \nm{z - Z_{j+1}}_{L^{p'}(t_j,t_{j+1};L^{q'}(\mathcal O))}^{p'}
    \Big)^{1/p'} \\
    \leqslant{}
    & c\tau \Big(
      \nm{z'}_{L^{p'}(0,T;L^{q'}(\mathcal O))} +
      \nm{A^*z}_{L^{p'}(0,T;L^{q'}(\mathcal O))}
    \Big).
  \end{align*}
  Consequently, by applying Minkowski's inequality we deduce that
  \begin{align*}
    & \Big(
      \sum_{j=0}^{J-1} \tau
      \nm{z(t_{j+1}) - Z_{j+1}}_{L^{q'}(\mathcal O)}^{p'}
    \Big)^{1/p'} \\
    \leqslant{}
    & \Big(
      \sum_{j=0}^{J-1} 
      \nm{z - z(t_{j+1})}_{L^{p'}(t_j,t_{j+1};L^{q'}(\mathcal O))}^{p'}
    \Big)^{1/p'} \\
    &\quad + \Bigg(
      \sum_{j=0}^{J-1} 
      \nm{z - Z_{j+1}}_{L^{p'}(t_j,t_{j+1};L^{q'}(\mathcal O))}^{p'}
    \Big)^{1/p'} \\
    \leqslant{}
    & c\tau \Big(
      \nm{z}_{L^{p'}(0,T;L^{q'}(\mathcal O))} +
      \nm{A^*z}_{L^{p'}(0,T;L^{q'}(\mathcal O))}
    \Big).
  \end{align*}
  The desired estimate \cref{eq:Z-conv} is then a direct consequence
  of Lemma \ref{lem:z-regu}. This completes the proof.
\end{proof}



\begin{lemma}
  \label{lem:y-z-dual}
  Let $ y $ be the mild solution to equation \cref{eq:y} with
  \[
    f \in L_{\mathbb F,\tau}^p((0,T)\times\Omega;L^q(\mathcal O;H)).
  \]
  Then
  \begin{equation}
    \label{eq:y-z}
    \begin{aligned}
      & \dualb{y, \,-z'+A^*z}_{
        L^{p}((0,T) \times \Omega; L^{q}(\mathcal O))
      } \\
      ={} &
      \sum_{j=0}^{J-2} \dualb{f(t_j) \delta W_j, \, z(t_{j+1})}_{
        L^{p}(\Omega;L^{q}(\mathcal O))
      } - {} \\
      & \qquad \sum_{j=0}^{J-1} \dualb{
        f(t_j) \big( W - W(t_j) \big), \, z'
      }_{
        L^{p}((t_j,t_{j+1}) \times \Omega;L^{q}(\mathcal O))
      }
    \end{aligned}
\end{equation}
for all
\[ % 2022-10-21
  z \in L^{p'}\big(
    \Omega;
    {}^0H^{1,p'}(0,T;L^{q'}(\mathcal O)) \cap
    L^{p'}(0,T;D(A^*))
  \big).
\]
\end{lemma}
\begin{proof}
  Let \( R \) denote the standard isometric isomorphism between
  \( L^{q'}(\mathcal O) \) and the dual space of \( L^q(\mathcal O) \).
  Let \( v \) be an arbitrary but fixed element of
  \( {}^0H^{1,p'}(0,T;L^{q'}(\mathcal O)) \cap L^{p'}(0,T;D(A^*)) \).
  For convenience, we will implicitly utilize the stochastic Fubini theorem in what follows.
  We have almost surely (\(\mathbb P\)-a.s.)
  \begin{align*} 
    & \int_0^T \big(Rv'(t)\big) y(t) \, \mathrm{d}t \\
    ={} &
    \int_0^T \big(Rv'(t)\big)
    \int_0^t S(t-s) f(s) \, \mathrm{d}W(s)
    \, \mathrm{d}t \quad\text{(by \cref{eq:y-mild})} \\
    ={} &
    \int_0^T \int_s^T \big( Rv'(t) \big) S(t-s)f(s)
    \, \mathrm{d}t \, \mathrm{d}W(s) \\
    ={} &
    \int_0^T \int_s^T \big( Rv'(t) \big) S(t-s)
    \, \mathrm{d}t \, f(s) \, \mathrm{d}W(s).
  \end{align*}
  Since
  \begin{align*}
    \int_s^T \big( Rv'(t) \big) S(t-s) \, \mathrm{d}t
    &= -Rv(s) - \int_s^T \big( Rv(t) \big)
    \frac{\mathrm{d}}{dt} S(t-s) \, \mathrm{d}t \\
    &= -Rv(s) + \int_s^T \big( Rv(t) \big) A S(t-s) \, \mathrm{d}t,
  \end{align*}
  it follows that, $ \mathbb P $-a.s.,
  \begin{align*}
    & \int_0^T \big( Rv'(t) \big) y(t) \, \mathrm{d}t \\
    ={}& \int_0^T \Big(
      -Rv(s) + \int_s^T \big( Rv(t) \big) AS(t-s) \, \mathrm{d}t
    \Big) f(s) \, \mathrm{d}W(s) \\
      ={}& \int_0^T \Big(
      -Rv(s) + \int_s^T \big( RA^*v(t) \big) S(t-s) \, \mathrm{d}t
    \Big) f(s) \, \mathrm{d}W(s) \\
    ={}& - \int_0^T \big( Rv(s) \big) f(s) \, \mathrm{d}W(s) +
    \int_0^T \big( RA^*v(t) \big)
    \int_0^t S(t-s) f(s) \, \mathrm{d}W(s) \, \mathrm{d}t \\
    ={}& - \int_0^T \big( Rv(s) \big) f(s) \, \mathrm{d}W(s) +
    \int_0^T \big( RA^*v(t) \big) y(t) \, \mathrm{d}t
    \quad\text{(by \cref{eq:y-mild})}.
  \end{align*}
  This further implies that, $ \mathbb P $-a.s.,
  \[
    \int_0^T \Big( R \big( -v'(t) + A^*v(t) \big) \Big) y(t)
    \, \mathrm{d}t =
    \int_0^T \big( Rv(s) \big) f(s) \, \mathrm{d}W(s).
  \]
  Then by the equality, $ \mathbb P $-a.s.,
  \begin{align*}
    & \int_0^T \big( Rv(s) \big) f(s) \, \mathrm{d}W(s) =
    \sum_{j=0}^{J-1} \int_{t_j}^{t_{j+1}} \big( Rv(s) \big)
    f(t_j) \, \mathrm{d}W(s) \\
    ={} &
    \sum_{j=0}^{J-1} \big( Rv(t_{j+1}) \big) f(t_j) \delta W_j -
    \sum_{j=0}^{J-1} \int_{t_j}^{t_{j+1}}
    \int_{s}^{t_{j+1}} \big(R v'(t)\big) \, \mathrm{d}t \, f(t_j) \, \mathrm{d}W(s) \\
    ={} &
    \sum_{j=0}^{J-1} \big( Rv(t_{j+1}) \big) f(t_j) \delta W_j -
    \sum_{j=0}^{J-1} \int_{t_j}^{t_{j+1}}
    \int_{t_{j}}^{t} \big(R v'(t)\big)f(t_j) \, \mathrm{d}W(s)
    \, \mathrm{d}t \\
    ={} &
    \sum_{j=0}^{J-1} \big( Rv(t_{j+1}) \big) f(t_j) \delta W_j -
    \sum_{j=0}^{J-1} \int_{t_j}^{t_{j+1}}
    \big(R v'(t)\big)f(t_j) (W(t) - W(t_j))
    \, \mathrm{d}t \\
    ={} &
    \sum_{j=0}^{J-2} \big( Rv(t_{j+1}) \big) f(t_j) \delta W_j -
    \sum_{j=0}^{J-1} \int_{t_j}^{t_{j+1}}
    \big(R v'(t)\big)f(t_j) (W(t) - W(t_j))
    \, \mathrm{d}t,
  \end{align*}
  we obtain $ \mathbb P $-a.s.
  \[
    \begin{aligned}
      \int_0^T \Big( R \big( -v'(t) + A^*v(t) \big) \Big) y(t) \, \mathrm{d}t
      = \sum_{j=0}^{J-1}
     \big(Rv(t_{j+1}\big)f(t_j)\delta W_j - {} \\
     \sum_{j=0}^{J-1} \int_{t_j}^{t_{j+1}}
     \big( Rv'(t) \big) f(t_j) (W(t) - W(t_j)) \, \mathrm{d}t.
    \end{aligned}
  \]
  For any $ C \in \mathcal F $, multiplying both sides of the above equation
  by the indicator function $ \mathbbm{1}_C $ for \( C \) and taking expectations,
  we deduce that \cref{eq:y-z} holds with $ z $ replaced by $ v\mathbbm{1}_C $.
  Given that both sides of \cref{eq:y-z} are bounded linear functionals
  on 
  \[
    L^{p'}(\Omega; {}^0H^{1,p'}(0,T;L^{q'}(\mathcal{O})) \cap L^{p'}(0,T;D(A^*))) 
  \]
  with respect to the variable $ z $, and considering the density of the linear span 
  \[ 
    \text{span}\big\{
      vI_C \mid v \in {}^0H^{1,p'}(0,T;L^{q'}(\mathcal{O})) \cap L^{p'}(0,T;D(A^*)), C \in \mathcal{F}
    \big\}
  \]
  within this space, we use a density argument to establish that \cref{eq:y-z} holds for all
  \[
    z \in L^{p'}( \Omega; {}^0H^{1,p'}(0,T;L^{q'}(\mathcal O)) \cap L^{p'}(0,T;D(A^*)) ).
  \]
  This completes the proof.
\end{proof}





%\begin{align*}
  %& \sum_{j=0}^{J-1} \dual{f(t_j)\delta W_j, z(t_{j+1})}_{\ell^{p}(\Omega;\ell^{q}(\mathcal O))} \\
  %\leqslant{} &
  %\sum_{j=0}^{J-1} \nm{f(t_j)\delta W_j}_{\ell^{p}(\Omega;L^q(\mathcal O))}
  %\nm{z(t_{j+1})}_{\ell^{p}(\Omega;\ell^{q}(\mathcal O))} \\
  %\leqslant{} &
  %\sum_{j=0}^{J-1} \tau^{1/2} \nm{f(t_j)}_{L^p(\Omega;L^q(\mathcal O;H))}
  %\nm{z(t_{j+1})}_{\ell^{p}(\Omega;\ell^{q}(\mathcal O))}
  %\quad\text{(by \cref{lem:integral})} \\
  %\leqslant{} &
%\end{align*}

Finally, we are in a position to prove \cref{thm:conv} as follows.

\medskip\noindent\textbf{Proof of \cref{thm:conv}.}
Let \( g \in L^{p'}((0,T) \times \Omega;L^{q'}(\mathcal O)) \) be
arbitrarily chosen. By Lemma \cref{lem:z-regu}, there exists
\( z \in L^{p'}\big( \Omega;{}^0H^{1,p'}(0,T;L^{q'}(\mathcal O)) \cap L^{p'}(0,T;D(A^*)) \big) \)
such that
\begin{equation} 
  \label{eq:z-regu}
  \begin{aligned}
    & \nm{z}_{L^{p'}(\Omega;{}^0H^{1,p'}(0,T;L^{q'}(\mathcal O)))} +
    \nm{A^*z}_{L^{p'}(\Omega;L^{p'}(0,T;L^{q'}(\mathcal O)))} \\
    \leqslant{} &
    c \nm{g}_{L^{p'}((0,T)\times\Omega;L^{q'}(\mathcal O))}
  \end{aligned}
\end{equation}
and, \(\mathbb P\)-a.s.,
\[ 
  \begin{cases}
    -z'(t) + A^*z(t) = g(t), \quad 0 \leqslant t \leqslant T, \\
    z(T) = 0.
  \end{cases}
\]
Let \( (Z_j)_{j=0}^J \) be defined \(\mathbb P\)-a.s.~by \cref{eq:Z}.
From Lemma \cref{lem:z-Z}, it follows that
\begin{equation}
  \label{eq:z-Z-omega}
  \Big(
    \sum_{j=1}^{J-1} \tau \nm{z(t_j) - Z_j}_{L^{p'}(\Omega;L^{q'}(\mathcal O))}^{p'}
  \Big)^{1/p'} \leqslant
  c \tau \nm{g}_{L^{p'}((0,T)\times\Omega;L^{q'}(\mathcal O))}.
\end{equation}
We will proceed with the proof in the following three steps.



  \textit{Step 1}.
  Let us establish the following identity:
  \begin{equation} 
    \label{eq:y-Y-g}
    \sum_{j=0}^{J-1} \dualb{ y - Y_j, \, g }_{
      L^{p}( (t_j,t_{j+1}) \times \Omega; L^{q}(\mathcal O) )
    } = I_1 + I_2,
\end{equation}
where
\begin{align*} 
  I_1
    &:= \sum_{j=0}^{J-2} \dualb{
      f(t_j)\delta W_j, \, z(t_{j+1}) - Z_{j+1}
    }_{L^{p}(\Omega;L^{q}(\mathcal O))}, \\
    I_2
    &:= -\sum_{j=0}^{J-1}
    \dualb{
      f(t_j)(W-W(t_j)), \, z'
    }_{
      L^{p}((t_j,t_{j+1}) \times \Omega; L^{q}(\mathcal O))
    }.
  \end{align*}
  A straightforward computation gives that
  \begin{align*} 
    & \sum_{j=0}^{J-1} \int_{t_j}^{t_{j+1}}
    \dualb{ Y_j, \, g(t) }_{L^{p}(\Omega;L^{q}(\mathcal O))} \, \mathrm{d}t \\
    &= \sum_{j=0}^{J-1} \dualb{
      Y_j, \, Z_j - Z_{j+1} + \tau A^* Z_j
    }_{L^{p}(\Omega;L^{q}(\mathcal O))}
    \quad\text{(by \cref{eq:Z})} \\
    &= \sum_{j=1}^{J-1} \dualb{
      Y_j - Y_{j-1} + \tau AY_j, \, Z_j
    }_{L^{p}(\Omega;L^{q}(\mathcal O))}
    \quad\text{(given that $Y_0 = 0$ and $Z_J = 0$)} \\
    &= \sum_{j=0}^{J-2} \dualb{
      Y_{j+1} - Y_{j} + \tau AY_{j+1}, \, Z_{j+1}
    }_{L^{p}(\Omega;L^{q}(\mathcal O))} \\
    &= \sum_{j=0}^{J-2} \dualb{
      f(t_j) \delta W_j, \, Z_{j+1}
    }_{L^{p}(\Omega;L^{q} (\mathcal O))}
    \quad\text{(by \cref{eq:Y-J-def})}.
  \end{align*}
  Furthermore, \cref{lem:y-z-dual} implies that
  \begin{align*} 
    & \dualb{y, \, g}_{L^{p}((0,T)\times\Omega;L^{q}(\mathcal O))} \\
    ={}& \sum_{j=0}^{J-2} \dualb{
      f(t_j)\delta W_j, \, z(t_{j+1})
    }_{L^{p}(\Omega; L^{q}(\mathcal O))} \\
    &\quad - \sum_{j=0}^{J-1} \dualb{f(t_j)(W - W(t_j)), \, z'}_{
      L^{p}((t_j,t_{j+1}) \times\Omega; L^{q}(\mathcal O))
    }.
\end{align*}
The desired equality \cref{eq:y-Y-g} is thus established by
combining the above equalities.


  \medskip\textit{Step 2}.
  Let us estimate \( I_1 \) and \( I_2 \). For \( I_1 \), we have
  \begin{align*}
    I_1
  & \leqslant \sum_{j=0}^{J-2}
  \left\| f(t_j) \delta W_j \right\|_{L^p(\Omega; L^q(\mathcal O))}
  \left\| z(t_{j+1}) - Z_{j+1} \right\|_{L^{p'}(\Omega; L^{q'}(\mathcal O))} \\
  & \leqslant \left(
    \sum_{j=0}^{J-2} \left\| f(t_j) \delta W_j \right\|_{L^p(\Omega; L^q(\mathcal O))}^p
  \right)^{\frac{1}{p}} \left(
    \sum_{j=0}^{J-2} \left\| z(t_{j+1}) - Z_{j+1} \right\|_{
      L^{p'}(\Omega; L^{q'}(\mathcal O))
    }^{p'}
  \right)^{\frac{1}{p'}} \\
   & \leqslant c \tau^{1 - \frac{1}{p'}} \left(
     \sum_{j=0}^{J-2} \left\| f(t_j) \delta W_j \right\|_{L^p(\Omega; L^q(\mathcal O))}^p
   \right)^{\frac{1}{p}} \left\| g \right\|_{L^{p'}((0,T) \times \Omega; L^{q'}(\mathcal O))}
   \quad \text{(by \cref{eq:z-Z-omega})} \\
   & = c \tau^{\frac{1}{p}} \left(
     \sum_{j=0}^{J-2} \left\| f(t_j) \delta W_j \right\|_{L^p(\Omega; L^q(\mathcal O))}^p
   \right)^{\frac{1}{p}} \left\| g \right\|_{L^{p'}((0,T) \times \Omega; L^{q'}(\mathcal O))},
  \end{align*}
  which, together with
  \begin{align*}
  & \left(
    \sum_{j=0}^{J-2} \left\| f(t_j) \delta W_j \right\|_{L^p(\Omega; L^q(\mathcal O))}^p
  \right)^{\frac{1}{p}} \\
  & \leqslant c \left(
    \sum_{j=0}^{J-2} \tau^{\frac{p}{2}} \left\| f(t_j) \right\|_{L^p(\Omega; L^q(\mathcal O; H))}^p
  \right)^{\frac{1}{p}}
  \quad \text{(by \cref{lem:integral})} \\
  & \leqslant c \tau^{\frac{1}{2} - \frac{1}{p}} \left\| f \right\|_{L^p((0,T) \times \Omega; L^q(\mathcal O; H))},
  \end{align*}
  yields
  \begin{equation}
    \label{eq:I1-bound}
    I_1 \leqslant  c \tau^{1/2}
    \left\| f \right\|_{L^p((0,T) \times \Omega; L^q(\mathcal O))}
    \left\| g \right\|_{L^{p'}((0,T) \times \Omega; L^{q'}(\mathcal O))}.
  \end{equation}
  For \( I_2 \), we have
  \begin{align*}
    I_2
  & = -\sum_{j=0}^{J-1} \int_{t_j}^{t_{j+1}} \dualB{
    f(t_j) \big( W(t) - W(t_j) \big), \, z'(t)
  }_{L^{p}(\Omega; L^{q}(\mathcal O))} \, \mathrm{d}t \\
  & \leqslant \left(
    \sum_{j=0}^{J-1} \int_{t_j}^{t_{j+1}}
    \left\| f(t_j) (W(t) - W(t_j)) \right\|_{L^p(\Omega; L^q(\mathcal O))}^p \, \mathrm{d}t
  \right)^{\frac{1}{p}}
  \left\| z' \right\|_{L^{p'}((0,T) \times \Omega; L^{q'}(\mathcal O))} \\
  & \leqslant c \left(
    \sum_{j=0}^{J-1} \int_{t_j}^{t_{j+1}}
    \left\| f(t_j) (W(t) - W(t_j)) \right\|_{L^p(\Omega; L^q(\mathcal O))}^p \, \mathrm{d}t
  \right)^{\frac{1}{p}}
  \left\| g \right\|_{L^{p'}((0,T) \times \Omega; L^{q'}(\mathcal O))},
  \end{align*}
  by \cref{eq:z-regu}. Hence, from
  \begin{align*}
  & \sum_{j=0}^{J-1} \int_{t_j}^{t_{j+1}}
  \left\| f(t_j) (W(t) - W(t_j)) \right\|_{L^p(\Omega; L^q(\mathcal O))}^p \, \mathrm{d}t \\
  & \leqslant c \sum_{j=0}^{J-1} \int_{t_j}^{t_{j+1}}
  (t-t_j)^{\frac{p}{2}} \left\| f(t_j) \right\|_{L^p(\Omega; L^q(\mathcal O; H))}^p \, \mathrm{d}t
  \quad \text{(by \cref{lem:integral})} \\
  & \leqslant c \tau^{\frac{p}{2}} \left\| f \right\|_{L^p((0,T) \times \Omega; L^q(\mathcal O; H))}^p,
  \end{align*}
  it follows that
  \begin{equation}
    \label{eq:I2-bound}
    I_2 \leqslant c \tau^{1/2}
    \left\| f \right\|_{L^p((0,T) \times \Omega; L^q(\mathcal O; H))}
    \left\| g \right\|_{L^{p'}((0,T) \times \Omega; L^{q'}(\mathcal O))}.
  \end{equation}



  % \medskip\textit{Step 2}.
  % Let us estimate $ I_1 $ and $ I_2 $.
  % For $ I_1 $ we have
  % % 2022-10-21 2022-10-26
  % \begin{align*}
  %   I_1
  %   & \leqslant \sum_{j=0}^{J-1}
  %   \nm{f(t_j)\delta W_j}_{L^p(\Omega;L^q(\mathcal O))}
  %   \nm{z(t_{j+1}) - Z_{j+1}}_{L^{p'}(\Omega;L^{q'}(\mathcal O))} \\
  %   & \leqslant \Big(
  %     \sum_{j=0}^{J-1} \nm{f(t_j)\delta W_j}_{L^p(\Omega;L^q(\mathcal O))}^p
  %   \Big)^{1/p} \Big(
  %     \sum_{j=0}^{J-1} \nm{z(t_{j+1}) - Z_{j+1}}_{L^{p'}(\Omega;L^{q'}(\mathcal O))}^{p'}
  %   \Big)^{1/p'} \\
  %   & \leqslant c \tau^{1-1/p'}\Big(
  %     \sum_{j=0}^{J-1} \nm{f(t_j)\delta W_j}_{L^p(\Omega;L^q(\mathcal O))}^p
  %   \Big)^{1/p} \nm{g}_{L^{p'}((0,T)\times\Omega;L^{q'}(\mathcal O))}
  %   \quad\text{(by \cref{eq:z-Z-omega})},
  %   %\quad\text{(by \cref{lem:z-Z})},
  % \end{align*}
  % which, together with
  % \begin{align*} % 2022-10-26
  %   & \Big(
  %     \sum_{j=0}^{J-1} \nm{f(t_j)\delta W_j}_{L^p(\Omega;L^q(\mathcal O))}^p
  %   \Big)^{1/p} \\
  %   \leqslant{} &
  %   c \Big(
  %     \sum_{j=0}^{J-1}
  %     \tau^{p/2}\nm{f(t_j)}_{L^p(\Omega;L^q(\mathcal O;H))}^p
  %   \Big)^{1/p} \quad\text{(by \cref{lem:integral})} \\
  %   \leqslant{} &
  %   c \tau^{1/2-1/p} \nm{f}_{L^p((0,T)\times\Omega;L^q(\mathcal O;H))},
  % \end{align*}
  % yields
  % \begin{align} % 2022-10-26
  %   I_1 &\leqslant c \tau^{3/2-1/p-1/p'}
  %   \nm{f}_{L^p((0,T)\times\Omega;L^q(\mathcal O))}
  %   \nm{g}_{L^{p'}((0,T)\times\Omega;L^{q'}(\mathcal O))} \notag \\
  %   &= c \tau^{1/2}
  %   \nm{f}_{L^p((0,T)\times\Omega;L^q(\mathcal O))}
  %   \nm{g}_{L^{p'}((0,T)\times\Omega;L^{q'}(\mathcal O))}.
  %   \label{eq:I1-bound}
  % \end{align}
  % For $ I_2 $ we have
  % \begin{align*} % 2022-10-21 2022-10-26
  %   I_2
  %   &= -\sum_{j=0}^{J-1} \int_{t_j}^{t_{j+1}} \dualB{
  %     f(t_j)\big( W(t) - W(t_j) \big), \, z'(t)
  %   }_{L^{p}(\Omega;L^{q}(\mathcal O))} \, \mathrm{d}t \\
  %   &\leqslant \Big(
  %     \sum_{j=0}^{J-1} \int_{t_j}^{t_{j+1}}
  %     \nm{f(t_j)(W(t) - W(t_j))}_{L^p(\Omega;L^q(\mathcal O))}^p \, \mathrm{d}t
  %   \Big)^{1/p}
  %   \nm{z'}_{L^{p'}((0,T) \times \Omega; L^{q'}(\mathcal O))} \\
  %   &\leqslant c\Big(
  %     \sum_{j=0}^{J-1} \int_{t_j}^{t_{j+1}}
  %     \nm{f(t_j)(W(t) - W(t_j))}_{L^p(\Omega;L^q(\mathcal O))}^p \, \mathrm{d}t
  %   \Big)^{1/p}
  %   \nm{g}_{L^{p'}((0,T) \times \Omega; L^{q'}(\mathcal O))},
  % \end{align*}
  % by \cref{eq:z-regu}. Hence, from
  % \begin{align*} % 2022-10-21
  %   & \sum_{j=0}^{J-1} \int_{t_j}^{t_{j+1}}
  %   \nm{f(t_j)(W(t) - W(t_j))}_{L^p(\Omega;L^q(\mathcal O))}^p \, \mathrm{d}t \\
  %   \leqslant{} &
  %   c \sum_{j=0}^{J-1} \int_{t_j}^{t_{j+1}}
  %   (t-t_j)^{p/2} \nm{f(t_j)}_{L^p(\Omega;L^q(\mathcal O;H))}^p \, \mathrm{d}t
  %   \quad\text{(by \cref{lem:integral})} \\
  %   \leqslant{} &
  %   c \tau^{p/2} \nm{f}_{L^p((0,T) \times \Omega;L^q(\mathcal O;H))}^p,
  % \end{align*}
  % it follows that
  % \begin{equation} % 2022-11-04
  %   \label{eq:I2-bound}
  %   I_2 \leqslant c\tau^{1/2} \nm{f}_{
  %     L^p((0,T)\times\Omega;L^q(\mathcal O;H))
  %   } \nm{g}_{L^{p'}((0,T)\times\Omega;L^{q'}(\mathcal O))}.
  % \end{equation}


  \medskip\textit{Step 3}. Combining \cref{eq:y-Y-g,eq:I1-bound,eq:I2-bound}
  leads to
  \begin{align*}
    & \sum_{j=0}^{J-1} \dualb{ y-Y_j, g }_{
      L^{p}((t_j,t_{j+1})\times\Omega;L^{q}(\mathcal O))
    } \\
    \leqslant{} &
    c\tau^{1/2} \nm{f}_{
      L^p((0,T)\times\Omega;L^q(\mathcal O))
    } \nm{g}_{
      L^{p'}((0,T)\times\Omega;L^{q'}(\mathcal O))
    }.
  \end{align*}
  Given that $ L^{p'}((0,T)\times\Omega;L^{q'}(\mathcal O)) $ is the dual space of
  $ L^p((0,T)\times\Omega;L^q(\mathcal O)) $, and considering that
  $ g \in L^{p'}((0,T)\times\Omega;L^{q'}(\mathcal O)) $ is arbitrarily chosen,
  we readily obtain the desired error estimate \cref{eq:conv}.
  This completes the proof of \cref{thm:conv}.

\hfill\ensuremath{\blacksquare}










\appendix
\section{Some technical estimates}
\label{sec:some_proofs}
In the section we shall use the notations introduced in \cref{ssec:space-regu}.
Let $ \mu $ denote the Lebesgue measure on $ \mathbb R^d $.
\begin{lemma}
  \label{lem:calI}
  Assume that $ p, q\in [2,\infty) $, and let $ \mathcal I(z) $, $ z \in \Upsilon $, be defined by \cref{eq:I-def}.
  Then
  \[ % 2022-10-24 2022-11-01 2022-11-06
    \sup_{z \in \Upsilon} \,
    \nm{\mathcal I(z)}_{
      \mathcal L\big(
        \ell_{\mathbb F}^p(L^p(\Omega;L^q(\mathcal O;H))),
        \, \ell^p(L^p(\Omega;L^q(\mathcal O)))
      \big)
    } < \infty.
  \]
\end{lemma}
\begin{proof}
  Fix any $ z \in \Upsilon $ and let $ \xi = e^{-z} - 1 $.
  Using \cref{lem:integral}, Minkowski's inequality and H\"older's inequality,
  we obtain, for any $ g \in \ell_\mathbb F^p(L^p(\Omega;L^q(\mathcal O;H))) $,
  \begin{align*}
    & \nm{\mathcal I(z)g}_{\ell^p(L^p(\Omega;L^q(\mathcal O)))}^p \\
    ={} &
    \sum_{j=1}^\infty \nmB{
      \sum_{k=0}^{j-1} \xi^{1/2} (1 + \xi)^{k+1-j}
      g_k \delta W_k/\sqrt\tau
    }_{L^p(\Omega;L^q(\mathcal O))}^p \\
    \leqslant{} &
    c\sum_{j=1}^\infty \mathbb E \bigg(
      \int_\mathcal O \Big(
        \sum_{k=0}^{j-1} \nm{\xi^{1/2}(1+\xi)^{k+1-j} g_k}_H^2
      \Big)^{q/2} \, \mathrm{d}\mu
    \bigg)^{p/q} \\
    \leqslant{} &
    c \sum_{j=1}^\infty \mathbb E \bigg(
      \sum_{k=0}^{j-1} \Big(
        \int_\mathcal O
        \nm{\xi^{1/2}(1+\xi)^{k+1-j}g_k}_H^q \, \mathrm{d}\mu
      \Big)^{2/q}
    \bigg)^{p/2} \\
    ={} &
    c \sum_{j=1}^\infty \mathbb E \bigg(
      \sum_{k=0}^{j-1}
      \snmb{ \xi^{1/2}(1+\xi)^{k+1-j} }^2
      \nm{g_k}_{L^q(\mathcal O;H)}^2
    \bigg)^{p/2} \\
    \leqslant{} &
    c\sum_{j=1}^\infty \mathbb E \sum_{k=0}^{j-1}
    \snmb{\xi^{1/2} (1+\xi)^{k+1-j}}^2
    \nm{g_k}_{L^q(\mathcal O;H)}^p \big( I(j,\xi) \big)^{p/2-1} \\
    ={} &
    c\sum_{j=1}^\infty \sum_{k=0}^{j-1}
    \snmb{\xi^{1/2} (1+\xi)^{k+1-j}}^2
    \nm{g_k}_{L^p(\Omega;L^q(\mathcal O;H))}^p \big( I(j,\xi) \big)^{p/2-1},
  \end{align*}
  where
  \begin{align*}
    I(j,\xi) := \sum_{k=0}^{j-1} \snmb{
      \xi^{1/2} (1+\xi)^{k+1-j}
    }^2 = \sum_{k=0}^{j-1} \snmb{
      \xi^{1/2}(1+\xi)^{-k}
    }^2.
  \end{align*}
  It follows that, for any $ g \in \ell_\mathbb F^p(L^p(\Omega;L^q(\mathcal O;H))) $,
  \begin{align*}
    & \nm{\mathcal I(z)g}_{\ell^p(L^p(\Omega;L^q(\mathcal O)))}^p \\
    \leqslant{} &
    c I(\infty,\xi)^{p/2-1} \sum_{j=1}^\infty \sum_{k=0}^{j-1}
    \snmb{ \xi^{1/2} (1+\xi)^{k+1-j} }^2
    \nm{g_k}_{L^p(\Omega;L^q(\mathcal O;H))}^p \\
    ={} &
    c I(\infty,\xi)^{p/2-1} \sum_{k=0}^\infty \sum_{j=k+1}^\infty
    \snmb{ \xi^{1/2} (1+\xi)^{k+1-j} }^2
    \nm{g_k}_{L^p(\Omega;L^q(\mathcal O;H))}^p \\
    \leqslant{} &
    c I(\infty,\xi)^{p/2} \nm{g}_{\ell^p(L^p(\Omega;L^q(\mathcal O;H)))}^p \\
    ={} &
    c I(\infty,e^{-z}-1)^{p/2} \nm{g}_{\ell^p(L^p(\Omega;L^q(\mathcal O;H)))}^p,
  \end{align*}
  by the fact $ \xi = e^{-z} - 1 $. This leads to the bound
  \begin{align*}
    \nm{\mathcal I(z)}_{
      \mathcal L\big(
        \ell_\mathbb F^p(L^p(\Omega;L^q(\mathcal O;H))), \,
        \ell^p(L^p(\Omega;L^q(\mathcal O)))
      \big)
    } \leqslant c I(\infty,e^{-z}-1)^{1/2}.
  \end{align*}
  Therefore, the desired claim follows from the fact $ I(\infty,0) = 0 $ and the estimate
  \begin{align*} 
    \sup_{z \in \Upsilon\setminus\{0\}} \, I(\infty,e^{-z}-1)
    &= \sup_{z \in \Upsilon\setminus\{0\}} \,
    \sum_{k=0}^\infty \snmb{
      (e^{-z}-1)^{1/2} e^{kz}
    }^2 = \sup_{z \in \Upsilon\setminus\{0\}} \, \frac{\snm{e^{-z}-1}}{1 - \snm{e^{2z}}} < \infty,
  \end{align*}
  which is easily verified by the construction of $ \Upsilon $ as detailed in \cref{ssec:space-regu}.
\end{proof}


% % 2022-11-01 2022-11-06
% \begin{lemma}
%   \label{lem:Pim}
%   Assume that $ p, q \in [2,\infty) $,
%   and let $ \Pi_m $, $ m \in \mathbb N $, be defined by
%   \cref{eq:calII-def}. Then
%   \begin{equation} % 2022-11-01
%     \label{eq:100}
%     \sup_{m \in \mathbb N}
%     \, \nm{\Pi_m}_{
%       \mathcal L\big(
%         \ell_{\mathbb F}^p(L^p(\Omega;L^q(\mathcal O;H))), \,
%         \ell^p(L^p(\Omega;L^q(\mathcal O)))
%       \big)
%     } < \infty.
%   \end{equation}
% \end{lemma}
% \begin{proof}
%   For any $ m \in \mathbb N $ and $ g \in \ell_\mathbb F^p(L^p(\Omega;L^q(\mathcal O;H))) $,
%   by \cref{lem:integral}, Minkowski's inequality and H\"older's inequality, we obtain
%   % 2022-10-17 2022-10-25 2022-11-01
%   \begin{align*}
%     & \nm{\Pi_mg}_{\ell^p(L^p(\Omega;L^q(\mathcal O)))}^p \\
%     ={} &
%     (m+1)^{-p/2} \sum_{j=1}^\infty
%     \mathbb E \nmB{
%       \sum_{k=j-1-m \vee 0}^{j-1} g_k \delta W_k / \sqrt\tau
%     }_{L^q(\mathcal O)}^p \\
%     \leqslant{} &
%     c(m+1)^{-p/2} \sum_{j=1}^\infty
%     \mathbb E \Big(
%       \int_{\mathcal O}
%       \Big( \sum_{k=j-1-m\vee 0}^{j-1} \nm{g_k}_H^2 \Big)^{q/2}
%       \, \mathrm{d}\mu
%     \Big)^{p/q} \\
%     \leqslant{} &
%     c (m+1)^{-p/2} \sum_{j=1}^\infty \mathbb E \Big(
%       \sum_{k=j-1-m \vee 0}^{j-1} \nm{g_k}_{L^q(\mathcal O;H)}^2
%     \Big)^{p/2} \\
%     \leqslant{} &
%     c(m+1)^{-p/2} \sum_{j=1}^\infty \mathbb E\bigg(
%       \sum_{k=j-1-m\vee 0}^{j-1} \nm{g_k}_{L^q(\mathcal O;H)}^p
%       ( m+1)^{p/2-1}
%     \bigg) \\
%     ={} &
%     c(m+1)^{-1} \sum_{j=1}^\infty
%     \sum_{k=j-1-m\vee 0}^{j-1} \nm{g_k}_{L^p(\Omega;L^q(\mathcal O;H))}^p \\
%     ={} &
%     c(m+1)^{-1} \sum_{k=0}^\infty \nm{g_k}_{L^p(\Omega;L^q(\mathcal O;H))}^p
%     \sum_{j=k+1}^{k+1+m} 1 \\
%     ={} &
%     c \nm{g}_{\ell^p(L^p(\Omega;L^q(\mathcal O;H)))}^p.
%   \end{align*}
%   This implies the desired inequality \cref{eq:100}.
% \end{proof}

% 2022-11-06 2023-08-26
\begin{lemma}
  \label{lem:Fefferman-Stein}
  Let $ r \in (1,\infty) $ and $ s \in (1,\infty] $. For any $ g \in \ell^r(L^s(\mathcal O)) $,
  we have
  \[
    \sum_{j=1}^\infty  \nmB{
      \sup_{m \in \mathbb N} \frac1{1+m}
      \sum_{k=j}^{j+m} \snm{g_k}
    }_{L^s(\mathcal O)}^r \leqslant
    c \nm{g}_{\ell^r(L^s(\mathcal O))}^r.
  \]
\end{lemma}
\begin{proof}
  Define
  \[ 
    G(t) := \sup_{m \in \mathbb N}
    \frac1{(1+m)\tau} \int_t^{t+(1+m) \tau}
    \widetilde g(\beta) \, \mathrm{d}\beta.
  \]
  where $ \widetilde g(t) := \snm{g_j} $ for all $ t \in [t_j,t_{j+1}) $
  and $ j \in \mathbb N $. For any $ j \geqslant 1 $, it is easily verified that
  \begin{align*}
    G(t_j)(x) \leqslant \frac{c}\tau
    \int_{t_{j-1}}^{t_{j+1}} G(t)(x)
    \, \mathrm{d}t, \quad \forall x \in \mathcal O,
  \end{align*}
  which implies, through Minkowski's inequality and H\"older's inequality, that
  \begin{align*} 
    \nm{G(t_j)}_{L^s(\mathcal O)}^r
    & \leqslant
    c \tau^{-r} \Big(
      \int_{\mathcal O}
      \Big(
        \int_{t_{j-1}}^{t_{j+1}} G(t) \, \mathrm{d}t
      \Big)^s \, \mathrm{d}\mu
    \Big)^{r/s} \\
    & \leqslant
    c \tau^{-r} \Big(
      \int_{t_{j-1}}^{t_{j+1}} \nm{G(t)}_{L^s(\mathcal O)}
      \, \mathrm{d}t
    \Big)^r \\
    & \leqslant
    c \tau^{-1} \int_{t_{j-1}}^{t_{j+1}}
    \nm{G(t)}_{L^s(\mathcal O)}^r
    \, \mathrm{d}t.
  \end{align*}
  Summing over $j$, we find
  \begin{align*} 
    \sum_{j = 1}^\infty \nm{G(t_j)}_{L^s(\mathcal O)}^r
    &\leqslant c \tau^{-1} \int_{\mathbb R_{+}}
    \nm{G(t)}_{L^s(\mathcal O)}^r \, \mathrm{d}t \\
    &\leqslant c \tau^{-1} \nm{\widetilde g}_{L^r(\mathbb R_{+};L^s(\mathcal O))}^r
    \quad\text{(by \cite[Proposition~3.4]{Neerven2012})} \\
    & = c \nm{g}_{\ell^r(L^s(\mathcal O))}^r.
  \end{align*}
  The desired inequality then follows from  the fact
  \[
    G(t_j) = \sup_{m \in \mathbb N}
    \frac{1}{1+m} \sum_{k=j}^{j+m}
    \snm{g_k}, \quad \forall j \geqslant 1.
  \]
  This completes the proof.
\end{proof}



  %Assume that $ g \in \ell^p(L^p(\Omega;L^q(\mathcal O;H))) $ is of the form
  %\[
    %g_j = \sum_{n=1}^N g^n_j h_n, \quad \forall j \in \mathbb N,
  %\]
  %where $ g^n := (g^n_j)_{j \in \mathbb N} \in \ell^p(L^p(\Omega;L^q(\mathcal O))) $
  %for each $ 1 \leqslant n \leqslant N $. Note that $ L^q(\mathcal O;L^2(\Upsilon,\snm{\mathrm{d}z})) $
  %has type
  %\begin{align*}
    %& \nm{Mf}_{
      %\ell^p(L^p(\Omega;L^q(\mathcal O;L^2(\Upsilon,\snm{\mathrm{d}z};H))))
    %} \\
    %={} &
    %\sum_{j\in\mathbb N} \mathbb E \bigg(
      %\int_{\mathcal O} \Big(
        %\mathbb E_r \nmB{ \sum_{n=1}^N r_n Mg_j^n}_{
          %L^2(\Upsilon,\snm{\mathrm{d}z})
        %}^2
      %\Big)^{q/2} \, \mathrm{d}\mu
    %\bigg)^{p/q} \\
    %\leqslant{} &
    %c\sum_{j\in\mathbb N} \mathbb E \bigg(
      %\int_{\mathcal O}
        %\mathbb E_r \nmB{ \sum_{n=1}^N r_n Mg_j^n}_{
          %L^2(\Upsilon,\snm{\mathrm{d}z})
        %}^q \, \mathrm{d}\mu
    %\bigg)^{p/q} \\
    %={} &
    %c\sum_{j\in\mathbb N} \mathbb E \bigg(
      %\mathbb E_r \nmB{ \sum_{n=1}^N r_n Mg_j^n}_{
        %L^q(\mathcal O; L^2(\Upsilon,\snm{\mathrm{d}z}))
    %}^q
    %\bigg)^{p/q} \\
    %\leqslant{} &
    %c\sum_{j\in\mathbb N} \mathbb E
      %\mathbb E_r \nmB{ \sum_{n=1}^N r_n Mg_j^n}_{
        %L^q(\mathcal O; L^2(\Upsilon,\snm{\mathrm{d}z}))
    %}^p \\
    %={} &
    %c\sum_{j\in\mathbb N}
      %\mathbb E_r \nmB{ \sum_{n=1}^N r_n Mg_j^n}_{
        %L^p(\Omega;L^q(\mathcal O; L^2(\Upsilon,\snm{\mathrm{d}z})))
    %}^p \\
    %\leqslant{} &
    %c\sum_{j\in\mathbb N} \Big(
      %\mathbb E_r \nmB{ \sum_{n=1}^N r_n Mg_j^n}_{
        %L^p(\Omega;L^q(\mathcal O; L^2(\Upsilon,\snm{\mathrm{d}z})))
      %}^2
    %\Big)^{p/2} \\
    %\leqslant{} &
    %c\sum_{j\in\mathbb N} \Big(
      %\sum_{n=1}^N \nmB{ Mg_j^n}_{
        %L^p(\Omega;L^q(\mathcal O; L^2(\Upsilon,\snm{\mathrm{d}z})))
      %}^2
    %\Big)^{p/2}.
    %% \leqslant{} &
    %% c\sum_{j\in\mathbb N} \mathbb E
    %%   \sum_{n=1}^N \nmB{ Mg_j^n}_{
    %%     L^q(\mathcal O; L^2(\Upsilon,\snm{\mathrm{d}z}))
    %%   }^p \\
    %% ={} &
    %% c \sum_{n=1}^N \nm{Mg^n}_{
    %%   \ell^p(L^p(\Omega;L^q(\mathcal O;L^2(\Upsilon,\snm{\mathrm{d}z}))))
    %% } \\
    %% \leqslant{} &
    %% c \sum_{n=1}^N\nm{g_n}_{\ell^p(L^p(\Omega;L^q(\mathcal O)))}^p.
%\end{align*}
%By the Minkowski's inequality, we then obtain
%\begin{align*}
  %& \nm{Mg}_{\ell^p(L^p(\Omega;L^q(\mathcal O;L^2(\Upsilon,\snm{\mathrm{d}z};H))))} \\
  %\leqslant{} &
  %c \bigg(
  %\sum_{n=1}^N \Big(
  %\sum_{j \in \mathbb N} \nm{Mg_j^n}_{
  %L^p(\Omega;L^q(\mathcal O;L^2(\Upsilon,\snm{\mathrm{d}z})))
  %}^p
  %\Big)^{2/p}
  %\bigg)^{p/2}                                                                               \\
  %={}         &
  %c \bigg(
  %\sum_{n=1}^N \nm{Mg^n}_{
  %\ell^p(L^p(\Omega;L^q(\mathcal O;L^2(\Upsilon,\snm{\mathrm{d}z}))))
  %}^2
  %\bigg)^{p/2}                                                                               \\
  %\leqslant{} &
  %c \bigg(
  %\sum_{n=1}^N \nm{g^n}_{
  %\ell^p(L^p(\Omega;L^q(\mathcal O)))
  %}^2
   %\bigg)^{p/2}    \\
 %\end{align*}
%We have that $ L^p(\Omega;L^q(\mathcal O)) $ has cotype $ p \vee q $.
%We also have

%% 2022-10-29
%\begin{align*}
    %& \nm{g}_{\ell^p(L^p(\Omega;L^q(\mathcal O;H)))}^p \\
    %={} &
    %\sum_{j\in\mathbb N} \mathbb E \bigg(
      %\mathbb E_r \nmB{\sum_{n=1}^N r_n g_j^n}_{L^q(\mathcal O)}^2
    %\bigg)^{p/2} \\
    %\geqslant{} &
    %c\sum_{j\in\mathbb N} \mathbb E \mathbb E_r
    %\nmB{\sum_{n=1}^N r_n g_j^n}_{L^q(\mathcal O)}^p \\
    %={} &
    %c\sum_{j\in\mathbb N} \mathbb E_r
    %\nmB{\sum_{n=1}^N r_n g_j^n}_{L^p(\Omega;L^q(\mathcal O))}^p \\
    %\geqslant{} &
    %c \sum_{n=1}^N\nm{g_n}_{\ell^p(L^p(\Omega;L^q(\mathcal O)))}^p.
%\end{align*}
%\begin{align*}
  %& \nm{f}_{\ell^p(L^p(\Omega;L^q(\mathcal O;H)))}^p \\
%\end{align*}
%%\begin{theorem}
  %%Assume that
  %%\[
    %%\sigma:(t,\omega, x) \to \sigma(t,\omega,x)
  %%\]
  %%is a measurable mapping from $ (\Omega \times) $
  %%\begin{equation}
    %%\int_E \nm{\sigma(t,\cdot,\cdot)}_{L^q((0))}
  %%\end{equation}
  %%We have $ \mathbb P $-a.s.
  %%\begin{equation}
    %%\int_0^T \int_0^t \sigma(t,s) \, \mathrm{d}W(s) \, \mathrm{d}t
    %%= \int_0^T \int_s^T \sigma(t,s) \, \mathrm{d}t \, \mathrm{d}W(s).
  %%\end{equation}
%%\end{theorem}
%%\begin{proof}

%\end{proof}


%\begin{thebibliography}{10}
%\end{thebibliography}


%\bibliographystyle{plain}
%\bibliography{stochastic_optim}

\begin{thebibliography}{10}

\bibitem{Bessaih2019}
H.~Bessaih.
\newblock {Strong $ L^2 $ convergence of time numerical schemes for the
  stochastic two-dimensional Navier-Stokes equations}.
\newblock {\em IMA J. Numer. Anal.}, 39:2135--2167, 2019.

\bibitem{Blunck2001}
S.~Blunck.
\newblock Maximal regularity of discrete and continuous time evolution
  equations.
\newblock {\em Studia Math.}, 146:157--176, 2001.

\bibitem{Breit2021}
D.~Breit and A.~Dodgson.
\newblock Convergence rates for the numerical approximation of the 2d
  stochastic {N}avier-{S}tokes equations.
\newblock {\em Numer. Math.}, 147:553--578, 2021.

\bibitem{CarelliProhl2012}
Z.~Brz\'ezniak, E.~Carelli, and A.~Prohl.
\newblock Time-splitting methods to solve the stochastic incompressible
  {S}tokes equation.
\newblock {\em SIAM J. Numer. Anal.}, 50:2917--2939, 2012.

\bibitem{Prohl2013}
Z.~Brz\'ezniak, E.~Carelli, and A.~Prohl.
\newblock Finite-element-based discretizations of the incompressible
  {N}avier–{S}tokes equations with multiplicative random forcing.
\newblock {\em IMA J. Numer. Anal.}, 33:771--824, 2013.

\bibitem{Prohl2012}
E.~Carelli and A.~Prohl.
\newblock Rates of convergence for discretizations of the stochastic
  incompressible {N}avier-{S}tokes equations.
\newblock {\em SIAM J. Numer. Anal.}, 50:2467--2496, 2012.

\bibitem{Cui_Hong_2019}
J.~Cui and J.~Hong.
\newblock Strong and weak convergence rates of a spatial approximation for
  stochastic partial differential equation with one-sided lipschitz
  coefficient.
\newblock {\em SIAM J. Numer. Anal.}, 57:1815--1841, 2019.

\bibitem{Denk2013book}
R.~Denk and M.~Kaip.
\newblock {\em {General Parabolic Mixed Order Systems in $ L_p $ and
  Applications, Operator Theory: Advances and Applications, vol. 239}}.
\newblock Springer, Cham, 2013.


\bibitem{HytonenWeis2016}
T.~Hyt\"onen, J.~van Neerven, M.~Veraar, and L.~Weis.
\newblock {\em Analysis in Banach spaces, Volume I: Martingales and Littlewood-Paley Theory}.
\newblock Springer, Cham, 2016.

\bibitem{HytonenWeis2017}
T.~Hyt\"onen, J.~van Neerven, M.~Veraar, and L.~Weis.
\newblock {\em Analysis in Banach spaces, Volume II: Probabilistic Methods and Operator Theory}.
\newblock Springer, Cham, 2017.

\bibitem{Kemmochi2016}
T.~Kemmochi.
\newblock Discrete maximal regularity for abstract cauchy problems.
\newblock {\em Studia Math.}, 234:241--263, 2016.

\bibitem{Kemmochi2018}
T.~Kemmochi and N.~Saito.
\newblock Discrete maximal regularity and the finite element method for
  parabolic equations.
\newblock {\em Numer. Math.}, 138:905--937, 2018.

\bibitem{Lubich2016}
B.~Kov\'acs, B.~Li, and C.~Lubich.
\newblock A-stable time discretizations preserve maximal parabolic regularity.
\newblock {\em SIAM J. Numer. Anal.}, 54:3600--3624, 2016.

\bibitem{Kruse2014book}
R.~Kruse.
\newblock {\em Strong and weak approximation of semilinear stochastic evolution
  equations}.
\newblock Springer, Cham, 2014.

\bibitem{Kunstmann2004}
P.C. Kunstmann and L.~Weis.
\newblock {\em {Maximal $ L_p $-regularity for Parabolic Equations, Fourier
  Multiplier Theorems and $H^\infty$-functional Calculus}}, pages 65--311.
\newblock Springer, Berlin, 2004.

\bibitem{Vexler_Lp_2017}
D.~Leykekhman and B.~Vexler.
\newblock Discrete maximal parabolic regularity for galerkin finite element
  methods.
\newblock {\em Numer. Math.}, 135:923--952, 2017.

\bibitem{LiB2017Math}
B.~Li and W.~Sun.
\newblock {Maximal $ L^p $ analysis of finite element solutions for parabolic
  equations with nonsmooth coefficients in convex polyhedra}.
\newblock {\em Math. Comp.}, 86:1071--1102, 2017.

\bibitem{LiB2017SIAM}
B.~Li and W.~Sun.
\newblock Maximal regularity of fully discrete finite element solutions of
  parabolic equations.
\newblock {\em SIAM J. Numer. Anal.}, 55:521--542, 2017.

\bibitem{Pruss2016}
J.~Pr\"uss and G.~Simonett.
\newblock {\em Moving interfaces and quasilinear parabolic evolution
  equations}.
\newblock Birkh\"auser Basel, 2016.

\bibitem{Sinha2017book}
K.~B. Sinha and S.~Srivastava.
\newblock {\em Theory of semigroups and applications}.
\newblock Springer, Singapore, 2017.

\bibitem{Neerven2012b}
J.~van Neerven, M.~Veraar, and L.~Weis.
\newblock {Maximal $ L^p $-regularity for stochastic evolution equations}.
\newblock {\em SIAM J. Math. Anal.}, 44:1372--1414, 2012.

\bibitem{Neerven2012}
J.~van Neerven, M.~Veraar, and L.~Weis.
\newblock {Stochastic maximal $ L^p $-regularity}.
\newblock {\em Ann. Probab.}, 40:788--812, 2012.

\bibitem{Weis2001}
L.~Weis.
\newblock {Operator-valued Fourier multiplier theorems and maximal $ L_p
  $-regularity}.
\newblock {\em Math. Ann.}, 319:735--758, 2001.

\bibitem{Yan2005}
Y.~Yan.
\newblock Galerkin finite element methods for stochastic parabolic partial
  differential equations.
\newblock {\em SIAM J. Numer. Anal.}, 43:1363--1384, 2005.

\bibitem{Zhang2017book}
Z.~Zhang and G.~E. Karniadakis.
\newblock {\em Numerical methods for stochastic partial differential equations
  with white noise}.
\newblock Springer, Cham, 2017.

\end{thebibliography}


\end{document}
