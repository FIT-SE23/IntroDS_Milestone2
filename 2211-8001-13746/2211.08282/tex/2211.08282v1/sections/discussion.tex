\section{Discussion}
\label{sec:discussion}
\looseness=-1
In this work we have studied the impact of combining augmentation-homomorphic feature extractors with augmentation-based SSL objectives. In doing so, we have introduced a new framework which we call Homomorphic-SSL which illustrates an equivalence between previously distinct SSL methods when the homomorphism constraint is satisfied. Since it is not currently known how to construct neural networks which are analytically equivariant with respect to all input augmentations used in modern SSL, this constraint is precisely the greatest current limitation of this framework, and we expand on this limitation in Appendix \ref{appendix:A}. We therefore propose this work not as an improvment to the state of the art, but rather as a new perspective on SSL which provides a bridge to previously distant literature. Specifically, one field of research which appears particularly promising for future work is the integration of learned homomorphisms \cite{keller2021topographic, keurti2022homomorphism, connor2021variational, lconv, nptn} with H-SSL. In the H-SSL framework, a learned homomorphism can be seen as equivalent to a learned augmentation, providing a potential new avenue for approaching the extremely challenging \cite{pmlr-v163-blaas22a} but fruitful \cite{shi2022adversarial} goal of learned image augmentations. 

We additionally present this work as an attempt to renew interest in SSL objectives which operate without multiple inferences of a transformed image, such as Deep InfoMax \cite{dim} and Greedy InfoMax \cite{gim}, by allowing them to exploit the theoretical foundations developed for multi-view SSL \cite{cl_mv, ddn, SSL_MV, ssl_content_style, marco_MV, theory_cl}. Although DIM-like methods have to-date not yielded the same performance as their A-SSL counterparts, we believe the coupling between objective and network architecture is likely to yield more parallelizable algorithms which are therefore more scalable and biologically plausible \cite{gim}.
