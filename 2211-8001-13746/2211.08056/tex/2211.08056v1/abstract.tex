Motivated by developer productivity, serverless computing, and microservices
have become the de facto development model in the cloud. Microservices decompose
monolithic applications into separate functional units deployed individually.
This deployment model, however, costs CSPs a large infrastructure tax of more
than 25\%~\cite{warehousescalecomputing, fbcommunication}. To overcome these
limitations, CSPs shift workloads to Infrastructure Processing Units (IPUs) like
Amazon's Nitro~\cite{amazon-nitro} or, complementary, innovate by building on
memory-safe languages and novel software
abstractions~\cite{fastly-wasm,cloudflare-workers}.
%Memory-safe languages aid these novel approaches by providing a secure and
%performant runtime environment allowing to specialize functionality to their
%workload and improve performance avoiding costly context switches and bypassing
%OS network stacks via kernel-bypasses.

Based on these trends, we hypothesize a \arch providing a general-purpose
runtime environment to specialize functionality when needed and strongly isolate
components. To achieve this goal, we investigate building a single address space
OS or a multi-application library OS, possible hardware implications, and
demonstrate their capabilities, drawbacks and requirements. The goal is to bring
the advantages to all application workloads including legacy and memory-unsafe
applications, and analyze how hardware may improve the efficiency and security.

% Balancing recent advances in memory-safe languages
% and runtimes with hardware-based optimizations creates opportunities for
% research and industry.
%
\if 0
% paragraph of new center
We are seeing a tremendous shift
in datacenter technologies while, increasingly, secure services are migrating to
the public cloud and datacenters. Motivated by developer productivity,
microservices and Function and a Service (FaaS) have become de facto development
practices in the cloud. This shift is resulting in software architectures where
Cloud Service Providers (CSPs) provide the infrastructure software to run and
connect microservices and functions. Developers favor microservices and FaaS for
their reusability and logical decomposition, however this comes at a cost with
CSPs reporting a large infrastructure tax of more than 25%. To overcome these
limitations CSPs are responding with a shift to Infrastructure Processing Units
(IPUs), offloading the burden of infrastructure services to more cost effective
hardware resulting in a Disaggregated Datacenter (DDC). As a complementary
approach, CSPs are developing custom software abstractions building on
innovations in memory-safe languages. These new abstractions aim to reduce the
infrastructure tax while delivering 100x higher density and elasticity.  The
TDCoF collaborative research center will conduct innovative research on the
security implications of these technology trends and the necessary evolution of
Confidential Computing to address these shifts in both datacenter architecture
and software technologies . TDCoF research will be structured around three key
research vectors, 1) Secure Hardware Architectures, 2) Secure System
Architectures, 3) High Performance Secure Communication. Intel will be deeply
engaged with the center and will assign partner technologists/collaborators
across research vectors to interact with the academic community to produce a
stream of innovation proof-points, publications, demonstrations, and technology
transfers into Intel and the broader industry throughout the duration of the
program.

\fi


\if 0
Motivated by developer productivity, the microservices and memory-safe languages
have become the de facto development practice in the cloud. This shift is
resulting in a software architecture where CSPs provide the infrastructure
software to run and connect microservices. Developers favor microservices for
their reusability and logical decomposition, but CSPs report a large
infrastructure tax of more than 25\% and high elasticity requirements. To cope
with this new class of workloads CSPs respond in two forms. First, they slimmed
hypervisor and OS kernels to increase elasticity. As a complementary approach,
CSPs developed custom memory-safe runtimes that rely on software-fault isolation
to achieve 100x higher density and elasticity.

In this paper we will argue to generalize both approaches and design a novel
memory-safe software architecture. In this architecture all components restrict
memory accesses to allow operating system and workloads to co-exist. We discuss
the requirements, build a design using common memory-safe languages, compilers
and runtimes, and highlight uses and optimizations. Using this software
architecture, we drastically reduce the complexity of the virtual memory and
remove hardware boundaries to reduce the infrastructure tax of microservice
workloads and improve performance of memory-bandwidth limited applications such
as machine learning.

\fi