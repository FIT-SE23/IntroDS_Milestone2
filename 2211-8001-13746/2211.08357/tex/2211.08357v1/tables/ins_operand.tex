\begin{table*}[t]
    \small
    \centering
    \begin{tabular}{llllllll}
    \hline
    \#         & Type & Length & Colorized o0           & Colorized o1 & Original o0 & Original o1 & Executions \\ \hline
    \textbf{1} & INS  & 8      & 120 (2001)             & e8 (e800)    & 640 (4006)  & 614 (1406)  & 0          \\
    \textbf{2} & INS  & 8      & 1c8 (c801)             & 190 (9001)   & 40 (4000)   & 00 (00)     & 7087       \\
    \textbf{3} & INS  & 4      & 1010111 (1101)         & 08 (0800)    & 0e (0e00)   & 08(0800)    & 20400      \\
    \textbf{4} & INS  & 8      & 101010101010101 (0101) & 00 (00)      & 01 (100)    & 00 (00)     & 38668      \\ \hline
    \end{tabular}
    \caption{Examples of operand values of INS comparisons with the number of produced executions, showed is both the left and right operand.}
    \label{tab:ins-operand}
\end{table*}
% \begin{table}[]
%     \centering
%     \begin{tabular}{lllllllll}
%     \hline
%     \# & Type & Length & Colorized o0           & Colorized o1 & Original o0 & Original o1 & Executions &                                                                                                       \\ \hline
%     \textbf{1}  & INS  & 8      & 120 (2001)             & e8 (e800)    & 640 (4006)  & 614 (1406)  & 0          &                                                                                                       \\
%     \textbf{2}  & INS  & 8      & 1c8 (c801)             & 190 (9001)   & 40 (4000)   & 00 (00)     & 7087       & 01 = 01 orig 00 == 00 Repl = 01 !                                                                   \\
%     \textbf{3}  & INS  & 4      & 1010111 (1101)         & 08 (0800)    & 0e (0e00)   & 08(0800)    & 20400      & \begin{tabular}[c]{@{}l@{}}01 == 01 orig 00 == 00 Repl: 00\\ 101 == 101 \& orig 00 == 00 ! \end{tabular} \\
%     \textbf{4}  & INS  & 8      & 101010101010101 (0101) & 00 (00)      & 01 (100)    & 00 (00)     & 38668      & \begin{tabular}[c]{@{}l@{}}1010101 == 1010101\\ 101 == 101 \\ 01 == 01\end{tabular}                   \\ \hline
%     \end{tabular}
%     \caption{}
%     \label{tab:ins-operand}
% \end{table}