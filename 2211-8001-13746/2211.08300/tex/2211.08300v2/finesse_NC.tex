\documentclass[
 reprint, 10pt,
 amsmath,amssymb,
 aps, superscriptaddress
]{revtex4-1}

\usepackage{graphicx}
\usepackage{dcolumn}
\usepackage{hyperref}		% to include hyperlinks
\hypersetup{
  colorlinks   = true,		%Colours links instead of ugly boxes
  urlcolor     = blue,		%Colour for external hyperlinks
  linkcolor    = blue,		%Colour of internal links
  citecolor   = blue		%Colour of citations
}

\usepackage[dvipsnames]{xcolor}
\usepackage{bm}
\usepackage{bbold}
\usepackage{bbm}
\usepackage{ulem}
\usepackage{braket}

%\usepackage[normalem]{ulem}
%\usepackage{fullpage}

\definecolor{marie}{RGB}{0,128,128}
\definecolor{todo}{RGB}{102,51,153}
\definecolor{emph}{RGB}{0,150,150}

\begin{document}

\title{Molecular Strong Coupling and Cavity Finesse}

\author{Kishan S Menghrajani}
\email{kishan.menghrajani@monash.edu}
\affiliation{Department of Physics and Astronomy, Stocker Road, University of Exeter, Devon EX4 4QL, United Kingdom}
\affiliation{School of Physics and Astronomy, Monash University, Wellington Raid, Clayton, 3800, Victoria, Australia}
\author{Adarsh B Vasista}
\affiliation{Department of Physics, Indian Institute of Science Education and Research, Bhopal 462066, India}
\author{Wai Jue Tan}
\affiliation{Department of Physics and Astronomy, Stocker Road, University of Exeter, Devon EX4 4QL, United Kingdom}
\author{Philip A Thomas}
\affiliation{Department of Physics and Astronomy, Stocker Road, University of Exeter, Devon EX4 4QL, United Kingdom}
\author{Felipe Herrera}
\affiliation{Department of Physics, Universidad de Santiago de Chile, Av. Victor Jara 3493, Santiago, Chile}
\affiliation{Millennium Institute for Research in Optics, Concepción, Chile}
\author{William L Barnes}
\email{W.L.Barnes@exeter.ac.uk}
\affiliation{Department of Physics and Astronomy, Stocker Road, University of Exeter, Devon EX4 4QL, United Kingdom}

%
\date{\today}
%
\begin{abstract}
%Open cavities have been put forward as a convenient platform with which to undertake molecular strong coupling. Whilst a number of demonstrations of Rabi-splitting in open cavities have been reported by monitoring reflectivity and transmission, it remains unclear whether such structures really drive a molecular resonance into the strong coupling regime. Here we explore a more stringent monitor of strong coupling, the modification of photoluminescence. We examine the emission from a range of dye-doped open, half and full optical microcavities. For each configuration an analysis of the reflectivity data indicates the presence of strong coupling. It is only for the full-cavity, for which the dielectric layer is bound both above and below by a metallic mirror, that we find significant modification, with the photoluminescence clearly tracking the lower polariton. Based on our observations we suggest the usual strong coupling criterion, based on the coupling strength, may need to be supplemented by an additional condition based on the cavity finesse.
Molecular strong coupling offers exciting prospects in physics, chemistry and materials science. Whilst attention has been focused on including realistic models for the molecular systems involved, the important role played by the entire mode structure of the optical cavities employed so far has been largely overlooked. We show that the extent and effectiveness of molecular strong coupling is critically dependent on cavity finesse. Low finesse cavities can show strong coupling as judged from the presence of Rabi splitting in reflectivity measurements, but give  photoluminescence signals equivalent to that of bare molecules outside the cavity. The emission of light is shown to involve hybridized light-matter polaritonic states only for cavities with high finesse. By developing an analytical model of cavity photoluminescence in multimode structures, we clarify the role of finite finesse in polariton formation. The detailed nature of the modes supported by a cavity and how these modes interact with the molecular system – whether on resonance or not – will be as important in developing a coherent framework for molecular strong coupling as the inclusion of realistic molecular models.
\end{abstract}

\keywords{Strong-Coupling, Photoluminescence, Optical Microcavity, Polaritons}


%%\pacs[JEL Classification]{D8, H51}

%%\pacs[MSC Classification]{35A01, 65L10, 65L12, 65L20, 65L70}

\maketitle

\section{Introduction}\label{sec:introduction}

When molecules are placed inside an optical microcavity, the strong interaction between molecular resonances and cavity modes leads to the formation of hybrid states called polaritons - states that inherit characteristics of both the optical cavity modes and the molecules from which they are formed~\cite{Ebbesen_ACSAccounts_2016_49_2403,Herrera_JCP_2020_152_100902}.
This process, known as molecular strong coupling, has been extensively explored in the context of both excitonic~\cite{Lidzey_Nature_1998_395_53,Schwartz_PRL_2011_106_196405,Polak_ChemSci_2020_11_343} and vibrational resonances~\cite{Shalabney_NatComm_2015_6_5981,Long_ACSPhot_2014_2_130,Takele_PCCP_2021_23_16837}. In the context of vibrational strong coupling there is much excitement at present owing to the prospect of modifying chemical processes~\cite{Ebbesen_ACSAccounts_2016_49_2403,Yuen-Zhou_PNAS_2019_116_5214,Herrera_JCP_2020_152_100902,Garci-Vidal_Science_2021_373_eabd0336,Hirai_CPC_2020_85_1981,Thomas_Science_2019_363_615,Ahn_Science_2023_380_6650}, despite an incomplete understanding of the underlying science~\cite{Vurgaftman_JPCL_2020_11_3557,Imperatore_JCP_2021_154_191103,Vurgaftman_JCP_2022_156_034110}. 

A range of `cavity' structures have been explored, most frequently planar (Fabry-Perot) optical microcavities in which the molecules are located between two closely spaced metal or dielectric mirrors. Planar microcavities have dominated molecular strong coupling studies for many years, in both excitonic and vibrational regimes. However, such structures do not offer good access to the molecules involved, thereby limiting their applicability to cavity modified chemistry. Alternative `open' geometries have been explored, including surface plasmon modes~\cite{Baieva_JCP_2013_138_044707,Torma_RepProgPhys_2015_78_013901}, dielectric microspheres \cite{Vasista_NL_2020_20_1766}, and surface lattice resonances~\cite{Yadav_NL_2020_20_5043,Verdelli_JPCC_2022_126_7143}. More recently so-called `cavity-free' geometries have been investigated~\cite{Georgiou_JPCL_2020_11_9893,Thomas_JPCL_2021_12_6914,Canales_JCP_2021_154_024701,georgiou2023strong}, and extensive mode splitting observed. These cavities do not use metallic or dielectric multi-layer (DBR) mirrors, instead they rely on reflection from the interface of the molecular material with another dielectric to produce optical modes. Whilst some of these reports concerning open cavities have noted changes to the molecular absorption, it remains to be seen whether such structures can be used to control chemistry effectively. Since modification of photoluminescence is a more stringent measure of strong coupling than reflectance, transmittance, absorbance and scattering~\cite{Wersall_NL_2016_16_551,Wersall_ACSPhot_2019_6_2570,Vasista_Nanoscale_2021_13_14497,arXiv_2302.00023}, here we chose to explore the photoluminescence process for open, half and full cavities, in an attempt to gain better insight into cavity-free strong coupling. In doing so we identify an additional requirement that needs to be met for effective molecular strong coupling, one that highlights the vital role of cavity finesse.

\vspace{0.3cm}
In the work reported here we made use of three different planar cavity structures, shown in the top row of figure \ref{fig:schematic of cavities and PL}: (left) an open cavity, i.e. a layer of polymer containing dye molecules supported by a silicon substrate; (centre) a half-cavity, similar to (a), but with the addition of a metallic (gold) mirror between the substrate and the dye-doped polymer, and; (right) a full-cavity, similar to the half-cavity but now with a second metallic mirror added to the top of the structure. For a more extended description of the optical modes supported by the different structures see Supplementary Information (SI) sections 4-6. We made use of the J-aggregated dye TDBC (5,5$^\prime$,6,6$^\prime$-tetrachloro-1,1$^\prime$-diethyl-3,3$^\prime$-di(4-sulfobutyl)-benzimidazolocarbocyanine), either dispersed in the polymer PVA, or deposited using a layer-by-layer approach~\cite{Vasista_Nanoscale_2021_13_14497}.
We used a silicon substrate, and made use of gold for the metallic mirrors; further details of fabrication are given in the Methods section. We measured photoluminescence and reflectance spectra as a function of polar angle, thereby enabling us to construct dispersion diagrams. We analysed our experimental data using a transfer matrix model to calculate the reflectance, transmittance and absorption, whilst we made use of a coupled oscillator model to determine the modes of each system; again, details of both are given in the SI.

To enter the strong coupling regime the collective Rabi splitting, $\Omega_R$, should be greater than the mean of the cavity and molecular spectral widths, $K$ and $\Gamma$ respectively (see also note in SI), which we can write as~\cite{Rider_CP_2021_62_217},
%
\begin{equation}
\Omega_R > (K+\Gamma)/2.
\end{equation} 
\label{eq:usual_condition}
%
However, as we will see below, satisfying this condition does not guarantee strong coupling as witnessed by photoluminescence. Instead we find that we need to place another condition on the finesse of the cavity modes.

%%%%%%%%%%%%%%
\begin{figure*}
\includegraphics[width=1.9\columnwidth]{new_attempt_v3.pdf}
\caption{\textbf{Schematic of Cavity Structures and Photoluminescence dispersion:} \textbf{Left: Open Cavity} consisting of a TDBC-doped polymer (PVA) film on a silicon substrate, doped-PVA thickness $\sim$340 nm. \textbf{Centre: Half Cavity}, as (a) but now a thin gold film is included between the substrate and the dye-doped polymer, doped-PVA thickness $\sim$600 nm. \textbf{Right: Full Cavity}, similar to (b) but with the addition of a top gold layer to form the second mirror of a closed cavity, the cavity thickness is $\sim$400 nm. Photoluminescence spectra were acquired from each sample as a function of collection angle and the data are plotted here in the form of a dispersion diagram. The white dashed lines in each PL plot show the positions of the polaritons determined from the coupled oscillator model. Further data for these samples are shown in the SI.}
\label{fig:schematic of cavities and PL}
\end{figure*}
%%%%%%%%%%%%%%%%%

\section{Results and Discussion}

Photoluminescence (PL) and reflectance spectra were acquired as a function of wavelength and angle. For reflectance measurements a white-light source was coupled to an objective lens (100x, 0.8 NA) and focused on to the sample. The reflected light was then collected using the same objective lens and projected onto the Fourier plane~\cite{43}. For PL measurements, a 532 nm (green) diode-laser source was focused onto the sample and the PL was collected by the same objective lens in the back-scattering configuration. Details of the optical setup are provided in section 3 of the SI.

For each system we determine the extent of the anti-crossing (Rabi-splitting), $\Omega_R$, and the width (FWHM) of the cavity mode, $K$. Both $\Omega_R$ and $K$ were estimated on the basis of matching a simple coupled oscillator model simultaneously to PL and reflectance data ($K$ was taken from the calculated reflectance of `no-resonance' systems, where `no-resonance' means that the oscillator strength was set to zero), see Methods.

In the second row of figure \ref{fig:schematic of cavities and PL}, we show examples of the collected photoluminescence from the three types of cavity in the form of dispersion diagrams.
Here the PL spectra have been plotted as a function of frequency and in-plane wavevector $k_{\parallel} = \frac{2\pi}{\lambda} \sin\theta$ (where $\lambda$ is the wavelength of light and $\theta$ is the angle of incidence; the plane is that of the sample). Also shown as white dashed lines are the energies of the hybrid modes determined via the coupled oscillator model, further details are given in the Methods section. We next discuss the PL data for each cavity configuration in turn.

%\subsection*{\textit{Open Cavity}}

In the second row, left column of figure \ref{fig:schematic of cavities and PL} we show the measured photoluminescence dispersion. We observe a strong peak at 2.07 eV and a weaker peak at 1.97 eV. For reference a PL spectrum from a very thin (20 nm) TDBC film on Si is also shown in Supplementary Figure S9. The reference spectrum is very similar to that of the open cavity: for the open cavity case, the 2.07 eV peak is slightly broader and the 1.97 eV shoulder slightly stronger. The photoluminescence of the open cavity is also non-dispersive. Therefore, there is little if any sign that strong coupling has influenced the dye photoluminescence of this open cavity. One might argue that when the lower polariton mode at $k_{\parallel}=0$ is so far in energy from the un-modified photoluminescence that no change would be expected. However, we have observed elsewhere that this need not be the case~\cite{Vasista_Nanoscale_2021_13_14497}, provided there are phonon/vibrational modes that can scatter emission via the polariton.
Note that there is no dispersion of the PL in the vicinity of the dispersion curve where anti-crossing might be anticipated.
We repeated this experiment for a thicker ($\sim$1030 nm) TDBC film (Supplementary Figure S3) and once again found that the PL is only marginally modified (if at all) in the open cavity configuration.
To investigate the absorption in the TDBC layer we made further use of transfer matrix modelling, the results are shown in figure S2 of the SI. Although there is a significant change in the absorption (which in the bulk would be a single peak) it is clear that the distortion is not due to the presence of polaritons. The doublet feature in absorption in this seemingly simple sample arises from the complex interplay between absorption and the impedance that the TDBC-layer presents to incoming light~\cite{Tan_JCP_2021_154_024704}.
This is consistent with our previous modelling of the absorption of cavity-free strong coupling with a broad spectrum dye~\cite{Thomas_JPCL_2021_12_6914}, which showed modified absorption but no clearly resolvable polariton modes. 



%\subsection*{\textit{Half Cavity}}

The half cavity case (centre column of figure \ref{fig:schematic of cavities and PL}) appears very similar to the open cavity case (a larger PL peak at 2.07 eV and a smaller PL peak at 1.97 eV), but with a smaller difference between the magnitudes of the two peaks. Again, there is no clear mapping of the PL onto the position of the polariton modes. Calculated data for the absorption in the TDBC layer are shown in the SI, figure S4 panel (e). There is again a significant change in the absorption compared to that of the bulk, and further, compared to the case of the open cavity, there is now some indication of the absorption tracking the lower polariton mode, at least to some limited extent. Data collected from a 1630 nm thick half cavity sample (supplementary figure S5) also show a somewhat modified PL spectrum.


%\subsection*{\textit{Full Cavity}}

The measured photoluminescence dispersion for the full cavity (right hand column of figure 1) is significantly different from the open and half cavity cases, the PL clearly tracking the lower polariton mode. Calculated data for the absorption in the TDBC are shown in SI figure S8, panel (e). As for the PL, there is now a very significant change in the absorption that also clearly tracks the polariton modes.


\vspace{0.2cm}
\noindent It is useful at this point to compare the line spectra for the PL, for which we have chosen $k_{\parallel}=0$.  Figure \ref{fig:PL_comparison} shows the PL spectra for another set of open, half and open cavities with different thicknesses than those in fig. \ref{fig:schematic of cavities and PL}, but with equivalent spectral behaviour (full dispersion data in supplementary figures S3, S4, and S8). We have indicated the position of the lower polariton of the least detuned cavity mode at $k_{\parallel}=0$. %These line data encapsulate much of what we see in the dispersion data.  


%%%%%%%%%%%%%%%%%
\begin{figure}[htb!]
\centering
\includegraphics[width=0.9\linewidth]{PL_K_zero_comparison_v3.pdf}
\caption{\textbf{PL line spectra}.
PL for normal emission for an open cavity (1030 nm sample, red line), a half cavity (600 nm sample, blue line), and full cavity (355 nm sample, black line). The PL data have been normalised and scaled to lie between values of 0 and 1.
The position of the lower polariton modes for $k_{\parallel}=0$ are shown as vertical dashed lines, the lower polaritons shown here are associated with the mode that crosses the TDBC absorption energy in the zero-oscillator strength dispersion, see panels (c) in supplementary figures  S3, S4, and S8. The thin film reference PL data set (dotted line) was acquired from a thin film of TDBC on a silicon substrate, see supp info section S7. Note the difference in the peak position of the `bare' PL for the full cavity data when compared to the open and half cavity data. Part of this difference can be attributed to the fact that the TDBC for this full cavity was made using the layer-by-layer technique (see supp info), whilst for the other two data sets the TDBC was on the polymer host PVA, again, see supp info section S7.}
\label{fig:PL_comparison}
\end{figure}
%%%%%%%%%%%%%%%%%


%\noindent {}

%\noindent \textbf{Open cavity:} 
As a reference `uncoupled' case, a thin film of TDBC (20 nm) on Si is used, which has a strong PL peak at 2.07 eV with a weak shoulder around 1.96 eV. 

For the {\it open cavity}, the 1.97 eV PL shoulder is clearer than in the thin film and the 2.07 eV peak is broader and the PL remains non-dispersive. For the {\it half cavity}, the 1.97 eV PL peak is substantially enhanced compared to the open cavity, but there is still no clear PL dispersion. The thicker half-cavity PL is slightly dispersive. The changes at 1.97 eV might be weakly linked to the (lower) polariton modes supported by this structure, see figure S4 (d,f). The absorption spectrum shows signs of being modified.

In contrast, for {\it full cavity} both  PL and absorption clearly track the lower polariton. We also note that, as commonly found~\cite{Bellessa_PRL_2004_93_036404,Agranovich_PRB_2003_67_085311,Schwartz_CPC_2013_14_125}, PL is observed from polaritons at energies lower than the molecular resonance energy, i.e. we do not see any PL associated with the upper polariton. Looking at the dispersion of the PL from the lower polariton we see that it is not uniform: PL is typically produced by the relaxation of reservoir states through the loss of vibrational energy~\cite{Coles_JPCA_2010_114_11920} so that PL emission is strongest when the difference between the bare molecular resonance energy (reservoir) and the polariton branch are equal to the energy of a vibrational mode~\cite{Vasista_Nanoscale_2021_13_14497}.


\vspace{0.3cm}
\subsection*{Strong Coupling and Finesse}
We have compared the photoluminescence from dye molecules (TDBC aggregates) located in three different cavity configurations: open, half and full cavities. In all three cases the reflectivity data indicate the strong coupling regime has been reached, see the fifth column of Table 1. The calculated absorption shows a somewhat different picture, with changes in the absorption by the TDBC for all three cavity types, but only in the case of the full cavity does the absorption track the (lower) polariton fully.
We also observe changes in PL for all three samples, but again it is only for the full cavity that the PL maps onto the lower polariton. 
The behaviour we observe in PL, figure \ref{fig:PL_comparison}, is reminiscent of the transition from weak to strong coupling observed in reflection measurements~\cite{Thomas_NanoLett_2020}, where an initially uncoupled peak broadens before splitting into two clearly resolvable peaks.
In PL the upper polariton is not observed due to non-radiative relaxation, so instead a lower polariton branch eventually becomes distinct from the uncoupled PL peak.
Ordinarily, the transition from weak to strong coupling is observed by increasing the number of molecules in a cavity, for example by using photochromic molecules~\cite{Schwartz_PRL_2011_106_196405,Thomas_NanoLett_2020}. Here, however, similar behaviour has instead been observed by modifying the cavity structure. This leads to a number of questions: how can we quantify the change in these structures that has caused this transition into the strong coupling regime, and what is the threshold for the observation of strong coupling in PL?


\vspace{0.3cm}
Previously we have looked at whether absorption by the dye is modified in the strong coupling regime, and -- as here -- we found that for an open cavity there was some modification~\cite{Thomas_JPCL_2021_12_6914}. However, that study made use of a broad spectrum dye (whereas TDBC is narrow-band -- with a spectral width of $\Gamma$ = 0.07 eV, FWHM of TDBC extinction) -- and no PL measurements were undertaken. Other work looking at strong coupling between dye molecules and particle plasmon modes found that there was clear PL arising from the lower polariton in the strong coupling regime~\cite{Wersall_ACSPhot_2019_6_2570}.




\begin{table*}[htb!]
\centering
\begin{tabular}{|p{1.1cm}| p{1.7cm}|p{1.7cm}| p{1.7cm} | p{1.9cm} | p{1.7cm} | p{1.6cm} |}
\hline
\hline
Cavity\newline(nm)&$\Delta\omega\,$(eV)\newline{(FSR)}&$K\,$(eV)\newline{(mode-width)} & $\Omega_R\,$(eV)\newline{(Rabi splitting)}& $2\Omega_R/(K+\Gamma)$\newline{($>$ 1 for SC )}&$Q$\newline{(Q-factor)}& $\mathcal{F}$\newline{\textbf{(Finesse)}}\\
\hline
\hline
\textbf{Open}& & & & & & \\
\hline
340 &1.20$\pm0.02$&0.50$\pm0.02$&0.50$\pm0.03$&1.79$\pm0.12$&4.2$\pm0.2$&\textbf{2.4$\pm0.2$}\\
\hline
1030 &0.38$\pm0.01$&0.17$\pm0.01$&0.67$\pm0.04$&5.58$\pm0.57$&12.4$\pm0.7$&\textbf{2.2$\pm0.2$}\\
\hline
\hline
\textbf{Half}&&&&&&\\
\hline
600 &0.64$\pm0.01$&0.25$\pm0.01$&0.50$\pm0.03$&3.13$\pm0.27$&8.4$\pm0.4$&\textbf{2.6$\pm0.1$}\\
\hline
1630 &0.25$\pm0.01$&0.10$\pm0.01$&0.35$\pm0.02$&4.12$\pm0.40$&21$\pm2$&\textbf{2.5$\pm0.4$}\\
\hline
\hline
\textbf{Full}&&&&&&\\
\hline
330 &0.73$\pm0.01$&0.13$\pm0.01$&0.16$\pm0.01$&1.60$\pm0.19$&14$\pm4$&\textbf{5.6$\pm0.6$}\\
\hline
355 &0.75$\pm0.01$&0.11$\pm0.01$&0.20$\pm0.01$&2.22$\pm0.27$&19$\pm2$&\textbf{8.2$\pm0.6$}\\
\hline
400 &0.76$\pm0.01$&0.09$\pm0.01$&0.27$\pm0.02$&3.38$\pm0.49$&23$\pm2$&\textbf{9.9$\pm1.2$}\\
\hline
\hline
\end{tabular}

\caption{\textbf{Spectral parameters for the different cavities}. The relevant figures are as follows: open cavity, figures S2 and S3; half cavities, figures S4 and S5; full cavities, S6 - S8.}
\label{tab:spectral_parameters}
\end{table*}

\vspace{0.3cm}
What are we to make of our results? To address this question we looked for a correlation between our findings and the cavity parameters, e.g. $Q$-factor. In table ~\ref{tab:spectral_parameters} we have brought together the parameters for our different structures. As noted in the introduction, a conventional criterion for strong coupling is that $2\Omega_R>(\Gamma + K)$. Based on the data in table \ref{tab:spectral_parameters} and the fact that for TDBC, the spectral width is $\Gamma$ = 0.07 eV, all of ours samples meet this criterion (see fifth column of table~\ref{tab:spectral_parameters}).  

We next focus our attention on a previously ignored parameter, the cavity finesse, $\mathcal{F}$, given by $\mathcal{F}=\Delta\omega/K$, where $\Delta\omega$ is the free spectral range (FSR), i.e. the frequency separation of adjacent modes, see also SI section 8. Looking now at the penultimate column in table~\ref{tab:spectral_parameters}, there appears to be a clear correlation between the behaviour we see in our PL results and the cavity finesse. What we find is that for low finesse structures, $\mathcal{F}\sim 2$, the PL does not track the (reflectance) lower polariton, even when the Rabi splitting exceeds the linewidths. For higher finesse values, $\mathcal{F}> 4$, the PL does track the (reflectance) lower polariton.

This observation leads us to suggest a two-dimensional criterion for strong coupling. Based on our analysis of the cavity finesse (see table \ref{tab:spectral_parameters}) we suggest that in addition to the usual strong coupling criterion, \eqref{eq:usual_condition}, we need to constrain the finesse. We thus suggest a new criterion,

\begin{equation}
2\Omega_R > (K+\Gamma),\, \textrm{and},\, \mathcal{F}>\sim 4. 
\end{equation} 
\label{eq:new_condition}

\noindent We have plotted our data in this form in figure \ref{fig:KRW}; solid and grey lines indicate the two criteria.\\

%%%%%%%%%%%%%%%%%
\begin{figure}[t]
\centering
\includegraphics[width=0.9\linewidth]{KRW_annot_rev3.pdf}
\caption{\textbf{Strong value of coupling parameter space}.
Values of $2\Omega_R/(K+\Gamma)$ and $\Delta \omega /K$ for each of the cavities we investigated. The error bars (red) are derived from the data in table \ref{tab:spectral_parameters}. Data from open cavities are indicated by an open square, from half cavities by a half-filled square, and from full cavities by a filled square.
Also shown are data points (together with associated error bars, in this case blue) based on other reports, for details see main text.
%Point 1 is for the liquid filled cavity reported by Hertzog and Borjesson \textit{et al.}~\cite{Hertzog_ChemPhotChem_2020_4_612}. Point 2 is for the open dielectric cavity of Thomas \textit{et al.}~\cite{Thomas_JPCL_2021_12_6914}. Point 3 is for the dielectric microsphere of Vasista \textit{et al.}~\cite{Vasista_NL_2020_20_1766}. 
%Points 4 and 5 are planar Fabry-Perot cavities used in two high profile studies that report modifications to chemical reactions due to vibrational strong coupling. Point 4 corresponds to the work of Thomas \textit{et al.}~\cite{Thomas_Science_2019_363_615}, whilst point 5 corresponds to the work of Ahn \textit{et al.}~\cite{Ahn_Science_2023_380_6650}. The error bars associated with the data extracted from the literature are likely to be dominated by errors in extracting data from graphs presented in these literature reports.
%The diagonal line indicates where $\Omega_R /K$ = $\Delta \omega /K$, whilst the
The horizontal line at $2\Omega_R /(K+\Gamma) = 1$ indicates the `usual' strong coupling condition. The vertical dashed line indicates our suggested criterion based on cavity finesse. Our 2D criterion is satisfied in the region of white background.} 
\label{fig:KRW}
\end{figure}
%%%%%%%%%%%%%%%%%

%To take this idea of finesse a little further we extend our results by calculating the effect of different thickness mirrors for our full cavity structures. The results of this preliminary work are shown in Supplementary Information section 8. We find that there is a transition somewhere between a finesse of $\mathcal{F}=2.7$ and $\mathcal{F}=4.7$ that in some way marks a boundary that needs to be crossed for strong coupling to have a significant effect on cavity absorption and, we might speculate, a significant effect on photoluminescence.


%First, we have added two lines to help better understand our results. The diagonal line indicates where $\Omega_R /K$ = $\Delta \omega /K$, whilst the horizontal line at $\Omega_R /K = 1$ indicates the `usual' strong coupling condition. We suggest that for effective strong coupling in a system, the parameters will need to be such that the finesse places the data point to the right of the diagonal line, and a value of $\Omega_R /K$ that is greater than 1. Second, concerning
For the data associated with the cavities that we have investigated here (data points with red error bars), we see that the open and half cavities are all such that the free spectral range is too low. In contrast, the values for the three full cavities have both sufficient finesse and sufficient coupling strength for effective strong coupling.
%(One full cavity, the one with a finesse value of 5.6, looks to be marginal in terms of $\Omega_R /K$. An examination of the PL data in figure S6 shows that in this case the lower polariton position coincides with the 1.97 eV PL feature, so that an assessment of whether the strong coupling condition has been met can not really be made because this is a two electronic state problem.)
%Based on these observations, we suggest that for a clear and effective signature of strong coupling in photoluminescence, the following needs to hold,

%\begin{equation}
%K<\Omega_R <\Delta \omega.
%\end{equation} 
%\label{eq:new_condition}

To explore these ideas further we plot on figure \ref{fig:KRW} data extracted from a number of reports in the literature. Point 1 is for the dye-coated plasmonic nano-prism reported by Wersall \textit{et al.}~\cite{Wersall_NL_2016_16_551}. It is clear that for the plasmonic particle system investigated by Wersall \textit{et al.} effective strong coupling was achieved, and their PL data confirm this. Point 2 is for the open dielectric cavity of Thomas \textit{et al.}~\cite{Thomas_JPCL_2021_12_6914}. As for the open cavities we have explored here, it is perhaps marginal whether this system has attained the effective strong coupling regime. In this case, coupling to a higher-order electromagnetic mode would greatly lower $K$, potentially pushing this system into the effective strong coupling regime.
Point 3 is for the dielectric microsphere of Vasista \textit{et al.}~\cite{Vasista_NL_2020_20_1766}. It is clear in this case that whilst the coupling strength is good, the finesse is too low. It may be that a reduction in microsphere size (resulting in an increase in $\Delta\omega$) would allow the effective strong coupling regime to be reached. 
Points 4 and 5 are planar Fabry-Perot cavities used in two high profile studies that report modifications to chemical reactions due to vibrational strong coupling. Point 4 corresponds to the work of Thomas \textit{et al.}~\cite{Thomas_Science_2019_363_615}, whilst point 5 corresponds to the work of Ahn \textit{et al.}~\cite{Ahn_Science_2023_380_6650}. In both case the effective strong coupling regime is comfortably reached. More information on these data is given in the SI.


%%%%%%%%%%%%%%%%%
%\begin{figure}[htb!]
%\centering
%\includegraphics[width=0.9\linewidth]{figure4_annot.pdf}
%\caption{\textbf{Cavity mode widths}.
%\textcolor{gray}{\sout{We show the}} Calculated transmittance (empty cavity, i.e. zero oscillator strength) for two of our samples, the 600 nm half cavity (dotted line) and the 330 nm full cavity (full line). For each sample we label the half (full) cavity mode that primarily interacts with the molecules as $m(p)$, the mode at higher frequency as $m+1(p+1)$, and the mode at lower frequency as $m-1(p-1)$.} 
%\label{fig:widths}
%\end{figure}
%%%%%%%%%%%%%%%%%


\subsection*{Microscopic Quantum Theory}



%%%%%%%%%%%%%%%%%
\begin{figure*}[t]
\centering
\includegraphics[width=\linewidth]{Figure4-PL-v2.png}
\caption{\textbf{PL in a two-mode cavity}.
(a) Calculated polariton spectrum of a two-mode cavity as a function of the mode energy separation $\Delta$. The central $q=m$ mode strongly couples to the molecular resonance at $\omega_e=2.1$ eV and is kept fixed in energy. The lower $q=m-1$ mode approaches the molecular resonance with decreasing $\Delta$. (b) Simulated PL spectra of the two-mode cavity in panel (a), for an ensemble of $N$ inhomogeneously broadened molecular dipoles centered at 2.1 eV (vertical dashed line, $\sigma=0.021$ eV). Curves are labelled by the value of $\Delta$. LP denotes lower polariton. (c) Calculated PL spectral weight $X_{\rm LP}C_{\rm LP}$ at the LP energy as a function of $\Delta$, for the same parameters in panel (b). In all cases we set $N=150$, $\sqrt{N}g=0.2$ eV, $K=0.1$ eV, and $NW/K=10^{-4}$. Curves are averaged over 650 disorder configurations.} 
\label{fig:PL multimode}
\end{figure*}
%%%%%%%%%%%%%%%%%

How might we understand this requirement of a lower limit on the finesse in multi-mode cavities for effective strong coupling? Molecular strong coupling relies on the coherent exchange of energy between a set of molecular resonators and a cavity mode. If the finesse is too low then the molecular resonators can interact with multiple cavity modes simultaneously, thus changing the properties of the lower and upper polariton states around the molecular resonance. 

Cavity PL is a multi-step process in which molecular dipoles are effectively pumped incoherently from the ground state $S_0$ to the lowest excited state $S_1$, via UV excitation $S_0\rightarrow S_n$ followed by ultrafast radiationless relaxation $S_n\rightarrow S_1$ \cite{Muccini_2000}. Due to light-matter coupling inside the cavity, excited dipoles give out their energy to produce individual cavity photons, which then rapidly decay through the cavity mirrors at rate $K$ to generate the PL signal. 

To understand PL in a multi-mode cavity with tunable finesse, we expand the quantum mechanical analysis in \cite{Herrera_PRA_2017_95_053867,Herrera2017-prl}, by explicitly modelling incoherent pumping of molecular dipoles at rate $W$ and include multiple discrete cavity modes. Assuming that the coupled light-matter density matrix can be approximated by the mixed-state $\hat \rho\approx \ket{G}\bra{G}+\sum_j \rho_j^{(1)}\ket{\epsilon_j}\bra{\epsilon_j}$, where $\ket{G}$ is the coupled ground state (no molecular or cavity excitations) and $\ket{\epsilon_j}$ is the $j$-th polariton eigenstate in the single excitation manifold, the stationary PL spectrum is given by
%
\begin{equation}\label{eq:PL spectrum}
S_{PL}(\omega)= \pi W \,\sum_j \frac{ X_j^T\,C_j^q}{NW X_j^T+(NW+K)C_j^T}\,L_j(\omega),
\end{equation}
%
where $N$ is the number of molecular dipoles, $X_j^T$ is the total exciton content of the $j$-th light-matter eigenstate (including the dark manifold~\cite{Herrera_PRA_2017_95_053867}), $C_j^q$ is the photon content of the $j$-th state in the $q$-th cavity mode and $C_j^T=\sum_q C_j^q$ is the total photon content. $L_j(\omega)$ is a normalized Lorentzian response function with central eigenfrequencies $\omega_j$ and bandwidth $\Gamma_j$ (FWHM), which for simplicity we set to $\Gamma_j=K$.  The derivation of Eq. (\ref{eq:PL spectrum}) is given in Section 10 of the SI.

Figure \ref{fig:PL multimode}(a) shows the spectrum of an idealized two-mode cavity with tunable finesse. The central mode $q=m$ is kept at exact resonance with a molecular transition at $2.1$ eV, and the energy separation $\Delta$ to the lower $q=m-1$ mode is varied. The positions of the LP, UP and exciton lines are marked. An ensemble of $N$ molecular dipoles is equally coupled to both cavity modes with local coupling strength $g=0.2/\sqrt{N}$ eV, which gives a Rabi spliting of $\Omega_R=0.4$ eV for large $\Delta$. Strong coupling is thus established for the $q=m$ mode ($K=0.1$ eV, $\Gamma=0.05$ eV).


Figure \ref{fig:PL multimode}(b) shows the suppression of the PL signal at the LP energy as the mode separation energy $\Delta$ decreases, while the central emission feature near the bare molecular resonance remains relatively unaltered,  in qualitative agreement with the experimental comparison in figure \ref{fig:PL_comparison}. A Gaussian distribution of molecular transition frequencies is assumed ($\bar\omega_e$ = 2.1 eV, $\sigma = 21$ meV,  FWHM = 0.05 eV). The emergence of the lower polariton feature in PL with increasing intermode separation $\Delta$ is controlled by the spectral weight product $X_{\rm LP}^T\,C_{\rm LP}^{q}$ in Eq. (\ref{eq:PL spectrum}), with $q=m$. This quantity can be described as the degree of admixing between light and matter at a particular energy $\omega_j$. If the state at $\omega_j$ is either purely photonic ($X_j^T=0$) or purely excitonic ($X_j^q=0$), the PL signal is strongly suppressed. 

For a single-mode resonant cavity with a spectrally homogeneous ensemble of molecules ($\sigma=0$), the lower polariton state in strong coupling has $C_{\rm LP}^T\approx 1/2$ and $C_{\rm LP}\approx 1/2$, which sets the optimal mixing limit for the PL spectral weight $C_{\rm LP}X_{\rm LP}\approx 1/4$. Figure \ref{fig:PL multimode}(c) shows how the PL spectral weight of the LP state monotonically decreases from this asymptotic upper limit as the energy separation between modes decreases. In addition to the change in spectral weight, for intermode energy separation $\Delta$ comparable or smaller than the collective Rabi coupling $\sqrt{N}g$, we also expect level pushing of the LP from below towards the molecular resonance, for constant  light-matter coupling strength. The combination of reduced spectral weight and level pushing gives the spectral progression shown in figure \ref{fig:PL multimode}(b).



%We can perhaps gain some insight by looking at the mode spectra for some of our samples. In figure \ref{fig:widths} we show the calculated transmittance for two empty (zero oscillator strength) cavities. We see that for the half cavity (low finesse) there is significant overlap of adjacent modes, whilst for the full cavity (higher finesse) the modes are much better separated. It is perhaps now not so surprising that the low finesse system is not as effective as the higher finesse system\textcolor{magenta}{:} the overlapping modes are likely to lead to a loss of coherence in the cavity/molecule interaction.

%We have begun to develop a theory to model strong coupling in multimode cavities of low finesse. Preliminary calculations indicate that indeed, in low finesse situations, the effective Rabi splitting decreases and the the exciton content is reduced. A full report of this theory will be the focus of a future publication.

\section{Conclusion and Outlook}

%The work reported here enables us to identify some areas for further investigation, so as to check the conclusions we have drawn regarding our effective strong coupling criterion that involves finesse as well as coupling strength. One starting point would be to look at full cavities of different finesse (as discussed above) and to focus in particular on the PL emission from such structures. In particular, lower finesse structures based on thinner mirrors might enable the transition from low to high finesse to be explored.
%A check should also be made to see whether the same observations hold for s-polarised modes. A preliminary calculation indicates the behaviour of s-polarised modes is consistent with that of the p-polarised modes discussed above, see Supplementary Information section 9. This is potentially valuable since s-polarised modes in many planar structures are typically sharper than p-polarised modes.
%It would also be interesting to try and probe PL from modes that are strongly coupled beyond the light line, perhaps via some grating-scattering approach~\cite{Menghrajani_ACSPhot_2020_7_2448}. 
%How should think about cavity finesse for non planar cavities such as a small plasmonic nanoparticle that supports only one localised surface plasmon resonance? In such single (cavity) mode situations  $\mathcal{F}$ and Q are identical, so that it is not surprising that Q has typically been a focus of interest, rather than $\mathcal{F}$.
%Similarly, how should we think about the propagating surface plasmon supported by a metal surface? We note that luminescence from the lower polariton branch involving dye molecules coupled to surface plasmons is well established~\cite{Bellessa_PRL_2004_93_036404}, as it is for plasmonic nanoparticles~\cite{Zhengin_JPCC_2016_120_20588,arXiv_2302.00023}. Then there are surface lattice resonances to consider etc. \textbf{Third}, input from theory would be very valuable here. It would be very beneficial if more complete theoretical models -- ones that go beyond the simple coupled oscillator model we have used here -- could be developed to explore the absorption and especially the PL in these structures, perhaps along the lines of previously of reported models~\cite{Herrera_PRA_2017_95_053867}.\\

In summary, we investigated the photoluminescence, reflectance, and absorption associated with a range of dye-doped cavities. We compared a variety of spectral parameters with the behaviour we observed in photoluminescence, and found a correlation with cavity finesse. Thus, whilst our aim was to better understand strong coupling in open cavities, we have arrived at a more general conclusion about effective strong coupling: that in addition to the usual condition on the coupling strength, an additional condition based on the cavity finesse needs to be met. By considering available experimental data together and quantum mechanical modelling, we expect the influence of finesse on the strong coupling criterion to apply for optical microcavities, plasmonic nanocavities as well as infrared resonators. Future experiments using novel molecular cavity designs, as well as realistic microscopic quantum theory that includes the entire cavity mode profile, molecular dephasing and collective relaxation effects will further refine our fundamental understanding of molecular strong coupling at room temperature and enable new applications.

%This additional condition based on finesse allows a clear transition to the bulk regime~\cite{Canales_JCP_2021_154_024701} to be readily appreciated. Although our open cavities did not show effective strong coupling, our analysis suggests that such coupling will be possible, and our findings should enable the design of appropriate open cavities.


%\noindent \textcolor{red}{[to think about the 1.97 peak and the Raman/vibrational bond]}\\

%\noindent \textcolor{red}{[to say something about a wide range of oscillator strengths (0.02 - 0.3?, but that this does not correlate with the emergence of LP PL]}\\

%\noindent \textcolor{red}{[to say something about the very simple one-oscillator model we have used, and that we have also modelled a the LBL stuff as though it fills the cavity.]}





\section*{Acknowledgements}
The authors would like to acknowledge useful discussions and with Marie Rider and William Wardley. K.S.M. acknowledges financial support from the Leverhulme Trust research grant ``Synthetic biological control of quantum optics''. K.S.M. also acknowledges the support of Royal Society International Exchange grant (119893R).
KSM, AV, PAT and WAB acknowledge the support of European Research Council through the photmat project 
(ERC-2016-AdG-742222 :www.photmat.eu). FH acknowledges the support of the Royal Society through the award of a Royal Society Wolfson Visiting Fellowship, and through grants Fondecyt Regular 1221420 and Millennium Science Initiative Program ICN17\_012. For the purpose of open access, the authors have applied a Creative Commons Attribution (CC BY) licence to any Author Accepted Manuscript version arising.
Data associated with these results can be found at XXXXXXXXXX.

\section*{Methods}

\subsection{\textit{Sample preparation}}

\noindent {\textbf{Solution preparation:}}
Poly Vinyl Acetate (PVA, molar weight 450 000) was used as a host matrix for TDBC.
PVA was dissolved in water. TDBC was then dissolved in the PVA-water solution.
%with a weight ratio of 3:2 SPI to PMMA.
TDBC/PVA films were deposited on a silicon wafer by spin-coating. This produced film thicknesses over the range 300-1700 nm.

For the open-cavity structures, a TDBC/PVA solution was spun on top of a silicon substrate. For the half-cavity a thin gold layer of 30 nm was deposited on top of silicon substrate by thermal evaporation. Later, TDBC/PVA was spun on top of gold thin film. For the full-cavity, the lower mirror was prepared by thermally evaporating 40 nm of gold onto a silicon substrate. Then, following deposition of the TDBC/PVA layer, a second 40 nm gold mirror was deposited, again by thermal evaporation. 

\subsection{\textit{Reflectance Measurements}}

Whilst substrate-only spectra for the reflectance should provide a means to normalise the reflectance data, in our samples a non-trivial level of scattering meant that the normalisation process was not as effective as hoped. Accordingly the reflectance spectra were scaled to provide a better match with calculated data.

\subsection{\textit{Coupled Oscillator Models}}

We made use of a standard coupled oscillator model to help us interpret our data~\cite{Richter_APL_2015_107_231104}. The number of cavity resonances included in each model depended on the details of the sample (thickness etc.). For a molecular resonance at a frequency (angular) of  $\omega_{\textrm{mol}}$ interacting with several cavity resonances (whose energy depend on in-plane wavevector, i.e. $\omega_{\textrm{cav}}(k_{\parallel})$), the coupled oscillator model allows the dispersion of the new hybrid (polariton) modes to be determined. Since all of our samples supported more than one cavity modes we made use of a 2N multi-photonic-mode coupled oscillator model~\cite{Richter_APL_2015_107_231104,Balasubrahmaniyam_PRB_2021_103_L241407} to find the polariton frequencies by solving for those frequencies $\omega$ for which the determinant of the coupling matrix is zero, i.e. $|\mathbf{M}|=0$. As an example, the coupling matrix for one molecular resonance interacting with two cavity (photonic) modes is given by,

\begin{align}
    \mathbf{M} = \begin{pmatrix}\omega_{\textrm{mol}}- \omega & \Omega_R/2 &0&0\\
    \Omega_R/2 & \omega_{1}(k_{\parallel})-\omega&0&0\\
    0&0&\omega_{\textrm{mol}}- \omega & \Omega_R/2
    \\
    0&0&\Omega_R/2 & \omega_{2}(k_{\parallel})-\omega\end{pmatrix}
\label{CO_matrix}
\end{align}


\noindent where $\Omega_R$ is the extent of the interaction (assumed to be the same for all  modes), and is equal to the frequency splitting (anti-crossing) at the value of $k_{\parallel}$ where the uncoupled resonances cross. For the $n^{th}$ cavity resonance we described the $k_{\parallel}$ dependence, $\omega_n (k_{\parallel})$, as,

\begin{equation}
\omega_n = \frac{\omega_{n\,{k_0 = 0}}}{k_0} \sqrt{{k_0}^2 + \beta_n\, {k_{\parallel}}^2/{\varepsilon_b}},
\end{equation}
\label{eq:cavity_k}

\noindent where $\omega_{n\,{k_0 = 0}}$ is the angular frequency of the n$^{th}$ resonance at $k_{\parallel} = 0$. The factor $\beta_n$ was used to match the dispersion of the bare cavity modes with the calculated bare reflectance (needed because the coupled oscillator model ignores the angle-dependant phase change on reflection from an interface). We chose to work with the simplest model that we could, and therefore assumed no damping of the modes. The values of the $\omega_{n\,{k_0 = 0}}$, of $\Omega_R$ and of $\beta_n$ were determined by simultaneously trying to match the eigenvalues to the calculated and measured reflection data.
The values for the Rabi splitting we report here are based on the parameters used in the coupled oscillator model that best match our data.

\subsection{\textit{Transfer Matrix modelling}}

We made use of a standard multi-layer optics approach to calculations of the reflectance and absorption~\cite{Smith_JMO_2008_55_2929,NandH,WJTan_PhD}, and assumed all media were isotropic. The reflectance was calculated by taking the square of the modulus of the amplitude reflection coefficient. For the permittivities, we used the following:

\vspace{0.2cm}
\noindent \textbf{\textit{Silicon substrate}}: we took the permittivity to be 16.0 + 3.0i.

\vspace{0.2cm}
\noindent \textbf{\textit{Gold films}}: we used a Drude-Lorentz model for the frequency/energy dependant permittivity of gold, $\varepsilon_{\rm{Au}}(\omega)$, given by~\cite{Rakic_AO_1998_37_5271},

\begin{equation}
\varepsilon_{\rm{Au}}(\omega) = \frac{f_d\,\omega_p^2}{\omega^2+i\gamma_d\omega} + \sum_j{\frac{f_j\omega_p^2}{\omega_j^2-\omega^2-i\gamma_j\omega}},
\end{equation}
\label{eq:eps_Au}

\noindent where the plasma frequency $\omega_p$ was 9.03 eV,  the Drude damping rate $\gamma_d$ was 0.053 eV, and the strength, $f_d$ was 0.76. The Lorentz oscillator parameters we used are,

\begin{table}[htb!]
\centering
\begin{tabular}{| c| c| c| c |}
\hline
$j$   & $f_j$ &$\omega_j (eV)$ & $\gamma_j(eV)$ \\
\hline
1 &$0.024$ & $0.451$ & $0.241$ \\
\hline
2 &$0.010$ & $0.830$ & $0.345$ \\
\hline
3 &$0.071$ & $2.969$ & $0.870$ \\
\hline
4 &$0.601$ & $4.304$ & $2.494$ \\
\hline
5 &$4.385$ & $13.32$ & $2.214$ \\
\hline
\end{tabular}
\caption{Lorentz oscillator model parameters for gold films, as determined from ellipsometry}
\label{tab:eps_Au_parameters}
\end{table}

\vspace{0.2cm}
\noindent \textbf{\textit{TDBC J-aggregates}}: we used a Lorentz model, with the permittivity $\varepsilon_{\rm{TDBC}}(E)$ given by,

\begin{equation}
\label{eq:eps_TDBC}
\varepsilon_{\rm{TDBC}}(\omega) = \varepsilon_b + \frac{f\,\omega^2_0}{\omega^2_0-\omega^2-i\gamma\omega},
\end{equation}


\noindent where we took: $\omega_0 = 2.10$ eV, $\gamma=0.056$ eV, and $\varepsilon_b = 2.292$. The value of the reduced oscillator strength, $f$, was determined for each sample by seeking a best `by eye' match of the calculated and measured reflectance data. 

\bibliography{finesse_NC.bib}

\end{document}