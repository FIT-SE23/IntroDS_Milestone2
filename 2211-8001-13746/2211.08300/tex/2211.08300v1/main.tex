%\documentclass[12pt]{article}
\documentclass[journal=jacsat,manuscript=article]{achemso}
\usepackage[utf8]{inputenc}
\usepackage{graphicx}% Include figure files
%\usepackage{dcolumn}% Align table columns on decimal point
\usepackage{bm}% bold math
%\usepackage[mathlines]{lineno}% Enable numbering of text and display math
%\linenumbers\relax % Commence numbering lines
\usepackage{amsmath}
\usepackage[utf8]{inputenc}
\usepackage[T1]{fontenc}
\usepackage{mathptmx}
\usepackage{color}
%\usepackage{hyperref} 
%\usepackage{epstopdf}
\usepackage{gensymb}
\usepackage{comment}
\usepackage{subcaption}
\usepackage{braket}
\usepackage{float}
%\newcommand*\mycommand[1]{\texttt{\emph{#1}}}
%\DeclareUnicodeCharacter{2212}{-}  
%\epstopdfDeclareGraphicsRule{.tiff}{png}{.png}{convert #1 \OutputFile}
%\AppendGraphicsExtensions{.tiff}
%\usepackage[
%backend=biber,
%style=chem-acs,
%]{biblatex}
%\addbibresource{reff1.bib}

\author{Kishan S. Menghrajani}
%\altaffiliation{A shared footnote}
\affiliation[University of Exeter]
{Department of Physics and Astronomy, University of Exeter, Exeter, EX4 4QL, United Kingdom}
\email{K.S.Menghrajani@exeter.ac.uk}

\author{Adarsh B. Vasista}
\affiliation[ETH Zurich]
{Nanophotonic Systems Laboratory, ETH Zurich, Switzerland}

\author{Wai Jue Tan}
%\altaffiliation{A shared footnote}
\affiliation[University of Exeter]
{Department of Physics and Astronomy, University of Exeter, Exeter, EX4 4QL, United Kingdom}

\author{William L. Barnes}
%\altaffiliation{A shared footnote}
\affiliation[University of Exeter]
{Department of Physics and Astronomy, University of Exeter, Exeter, EX4 4QL, United Kingdom}

%%%%%%%%%%%%%%%%%%%%%%%%%%%%%%%%%%%%%%%%%%%%%%%%%%%%%%%%%%%%%%%%%%%%%
%% The document title should be given as usual. Some journals require
%% a running title from the author: this should be supplied as an
%% optional argument to \title.
%%%%%%%%%%%%%%%%%%%%%%%%%%%%%%%%%%%%%%%%%%%%%%%%%%%%%%%%%%%%%%%%%%%%%


\title{Molecular strong coupling: an exploration employing open, half- and full planar cavities}


\begin{document}
%\title{Origin of an anti-crossing between a leaky photonic mode and epsilon-near-zero point of silver}
%\date{}


%\author{Wai Jue Tan*$^1$, Philip A. Thomas$^2$ \\ and William L. Barnes$^2$\\ \\
%\small{$^{1,2}$Department of Physics and Astronomy, University of Exeter,} \\ %\small{Exeter, EX4 4QL, United Kingdom}\\
%\small{$^1$wjt206@exeter.ac.uk}}
%\small{wjt206@exeter.ac.uk}
%\small{$^2$College of Engineering, Mathematics and Physical Sciences, University of Exeter,} \\ \small{Exeter, EX4 4QL, United Kingdom}}
%\maketitle

\begin{abstract}
We explore the modification of photoluminescence from dye-doped open, half and full optical microcavities. For each configuration, an analysis of the reflectivty data indicates the presence of strong coupling. 
We find that an open cavity, i.e. a dye-doped dielectric slab placed on a silicon substrate, shows little if any modification of the photoluminescence spectrum. For the half-cavity, in which a thin metallic film is placed between the dye-doped dielectric and the substrate, we find a limited degree of modification of the photoluminescence. For the full-cavity, for which the dielectric layer is bound both above and below by a metallic mirror, we find very significant modification to the photoluminescence emission, the photoluminescence clearly tracking the lower polariton.
To learn more about the photo-physics we compare reflectivity, absorption and photoluminescence dispersion data, making use where appropriate of numerical modelling. We discuss our results in the context of harnessing open cavities to modify molecular photo-physics through molecular strong coupling .
\end{abstract}

\vspace{1cm}

\noindent {\textbf{keywords:}} Strong-Coupling, Photoluminescence, Optical Microcavity, Polaritons.
\vspace{1cm}

\section{Introduction}\label{sec:introduction}

When molecules are placed inside an optical microcavity, the interaction of molecular resonances with the cavity mode may lead to the formation of hybrid states called polaritons - states that inherit characteristics of both the optical cavity modes and the molecules from which they are formed~\cite{Ebbesen_ACSAccounts_2016_49_2403}.
This process, known as molecular strong coupling, has been extensively explored in the context of both excitonic~\cite{Lidzey_Nature_1998_395_53,Schwartz_PRL_2011_106_196405,Polak_ChemSci_2020_11_343} and vibrational resonances ~\cite{Shalabney_NatComm_2015_6_5981,Long_ACSPhot_2014_2_130,Takele_PCCP_2021_23_16837}. In the context of vibrational strong coupling there is much excitement at present owing to the prospect of modifying chemical processes~\cite{Ebbesen_ACSAccounts_2016_49_2403,Yuen-Zhou_PNAS_2019_116_5214,Herrera_JCP_2020_152_100902,Garci-Vidal_Science_2021_373_eabd0336,Hirai_CPC_2020_85_1981}, a prospect that is still somewhat controversial~\cite{Vurgaftman_JPCL_2020_11_3557,Imperatore_JCP_2021_154_191103,Vurgaftman_JCP_2022_156_034110}. A range of `cavity' structures have been explored, most frequently planar (Fabry-Perot) optical microcavities in which the molecules are located between two closely spaced metal or dielectric mirrors. Indeed, planar microcavities have been the work horse for molecular strong coupling studies for many years, in both excitonic and vibrational regimes. However, such structures do not offer good access to the molecules involved, thereby limiting their applicability to modified chemistry. As a result alternative `open' geometries have been explored, including surface plasmon modes~\cite{Torma_RepProgPhys_2015_78_013901}, dielectric microspheres \cite{Vasista_NL_2020_20_1766}, and surface lattice resonances~\cite{Verdelli_JPCC_2022_126_7143}. More recently so-called `cavity-free' geometries have been investigated~\cite{Georgiou_JPCL_2020_11_9893,Thomas_JPCL_2021_12_6914,Canales_JCP_2021_154_024701}, and extensive mode splitting observed, these cavities do not use metallic or dielectric multi-layer (DBR) mirrors, instead they rely on reflection from the interface of the molecular material with another dielectric, including air, to produce optical modes. Whilst some of these works noted changes to the molecular absorption, it remains to be seen whether such structures can be used to control chemistry effectively. Since modification of photoluminescence is a more stringent measure of strong coupling than reflectance, transmittance and scattering~\cite{Wersall_NL_2016_16_551,Wersall_ACSPhot_2019_6_2570,Vasista_Nanoscale_2021_13_14497}, here we chose to explore the photoluminescence process for open, half and full cavities, in an attempt to gain better insight into cavity free strong coupling.

\vspace{0.3cm}
In the work reported here we made use of three different planar cavity structures in our study, shown in figure \ref{fig:schematic of cavities}, these were: (a) an open cavity, i.e. a layer of polymer containing dye molecules supported by a silicon substrate; (b) a half-cavity, similar to (a), but with the addition of a metallic (gold) mirror between the substrate and the dye-doped polymer, and; (c) a full-cavity, similar to (b) but now with a second metallic mirror added to the top of the structure. In the work reported here we made use of the J-aggregated dye TDBC (5,5$^\prime$,6,6$^\prime$-tetrachloro-1,1$^\prime$-diethyl-3,3$^\prime$-di(4-sulfobutyl)-benzimidazolocarbocyanine) in two forms, one simply dispersed in the polymer PVA for the open and half cavities, and the other assembled using a layer-by-layer technique (see methods) for the full cavity. 
Our choice of two different forms of TDBC for these experiments was motivated by our desire to produce comparable levels of Rabi splitting in the (reflectivity-based) anti-crossing for the three structures, something we only partially accomplished.
We used a silicon substrate, and made use of gold for the metallic mirrors; further details of fabrication etc. are given in the methods section. We analysed the experimental data using a transfer matrix model to calculate the reflectance, transmittance and absorption, whilst we made use of a coupled oscillator model to determine the modes of each system; again, details of both are given in the methods.

%%%%%%%%%%%%%%
\begin{figure}[htb!]
\centering
\includegraphics[width=0.85\linewidth]{fig/fig_schematic_wlb.pdf}
\caption{\textbf{Schematic of Cavity Structures:} \textbf{(a) Open Cavity}
consisting of a TDBC-doped polymer (PVA) film on a silicon substrate, doped-PVA thickness $\sim$340 nm. \textbf{(b) Half Cavity}, as (a) but now a thin gold film is included between the substrate and the dye-doped polymer, doped-PVA thickness $\sim$600 nm. \textbf{(c) Full Cavity}, similar to (b) but with the addition of a top gold layer to form the second mirror of a closed cavity. Here the TDBC/PVA has been replaced by a combination of polymer (PMMA) layers and TDBC in the form of molecular layers deposited by the layer-by-layer technique, thickness of cavity $\sim$375 nm.}
\label{fig:schematic of cavities}
\end{figure}
%%%%%%%%%%%%%%%%%

\section{Results and Discussion}

Experimental data were acquired from the different cavity structures so as to produce dispersion plots. Specifically reflectance and photoluminescence (PL) were acquired as a function of wavelength and angle. For reflectance measurements a white-light source was coupled to an objective lens (100x, 0.8 NA) and focused on to the sample. The reflected light was then collected using the same objective lens and projected onto the Fourier plane~\cite{43}. For PL measurements, a 532 nm green diode-laser source was focused onto the sample and the PL was collected by the same objective lens in the back-scattering configuration. Details of the optical setup are provided in section S1 of the supplementary information.


\subsection*{\textit{Open Cavity}}

%%%%%%%%%%%%%%%%%
\begin{figure}[htb!]
\centering
\includegraphics[width=1.00\linewidth]{fig/open_cavity_340nm}
\caption{{\textbf{Open Cavity}.
In panel (a) we show an experimentally measured dispersion, based on p-polarised reflectance, indicated as a colour plot (see methods re: normalisation).
In panel (b) we show the calculated reflectance, the white dotted lines represent the limit of the collection numerical aperture used in the experiment. Panel (c) is the same as panel (b) except that in the calculation we have set the oscillator strength of the dye to zero.
In panel (d) we show an experimentally measured dispersion plot acquired from photoluminescence data. In panel (e) we show the calculated absorption in the TDBC-doped layer. In panels (a-e) the polariton dispersion is shown, as determined from the coupled oscillator model described in the methods section. The polariton positions are shown as white dashed lines, and the molecular resonance (2.08 eV) by a black dotted line; just one cavity mode was used in the model (see panel (c)), and the polymer thickness was $\sim$340 nm.} In panel (f) the polariton positions (normal incidence) are indicated by black dashed lines.}
\label{fig:open_data}
\end{figure}
%%%%%%%%%%%%%%%%%

\vspace{0.3cm}
In figure \ref{fig:open_data} we show, in the upper row, dispersion data based on reflectance for the open cavity, i.e. for a $\sim$340 nm TDBC-doped PVA layer on a silicon substrate. On the left (a) we show the measured reflectance, in the middle the calculated reflectance (b), and in (c) we again show the calculated reflectance, but this time we set the oscillator strength of the dye to zero. In the lower row  we again show a dispersion plot, panel (d), this time based on the measured photoluminescence. In panel (e) we show the calculated absorption in the TDBC-doped PVA layer, whilst in panel (f) we show normal incidence line spectra for the measured reflectance, the measured photoluminescence and the calculated absorption in the TDBC. The white dashed lines in panels (a-e) indicate the positions of the polariton modes as determined from our coupled oscillator model (see methods), the polariton positions at normal incidence are indicated in panel (f) by black dashed lines. Further details are given in the figure caption.

\vspace{0.3cm}
Looking first at the reflectance, panels (a) and (b), the presence of modes of the open cavity system is indicated by regions of low reflectivity (blue). Also shown are the positions of the polaritons as determined by our coupled oscillator model (details in methods). We see that overall there is a reasonable match between the experimental data, the calculated data, and the coupled oscillator model (which here employed a single cavity resonance). The calculated dispersion for a `bare' cavity and the `bare' coupled oscillator model is shown in panel (c). Here `bare' refers to the oscillator strength in equation~\eqref{eq:eps_TDBC} being set to zero. From the modelling we determine that in this case the reduced oscillator strength is 0.15, the splitting is $\Omega = 0.50\pm0.04$ eV, and $\beta = 1.3$ ($\beta$ is a parameter that ensures the bare cavity dispersion is correctly included in the coupled oscillator model, see methods). The splitting exceeds the sum of the TDBC linewidth, $0.053\pm0.003$ eV, and the width of the cavity modes, $0.18\pm0.04$ eV, so, if we accept reflectance as a good indicator of the allowed modes, then we are in the strong coupling regime. Panel (d) shows the measured photoluminescence dispersion, we see that there is no sign of any influence of polaritons on the photoluminescence for this open cavity. One might argue that when the lower polariton mode is this far in energy from the un-modified photoluminescence that no change would be expected. However, we have observed before that this is not the case~\cite{Vasista_Nanoscale_2021_13_14497}, provided there are phonon/vibrational modes that can scatter emission via the polariton. Nonetheless, to investigate this aspect further we looked at a thicker open cavity, where the lower polariton is much closer in energy to the molecular resonance (less de-tuned). The results of that investigation, shown in figure S2 of the Supplementary Information, lead to the same observation, i.e. that the PL is not really modified in the open cavity configuration. Finally, to investigate the absorption we made further use of transfer matrix modelling, the results are shown in panel (e). Although there is a significant change in the absorption (which in the bulk would be a single peak) it is clear from panel (e) that the distortion is not due to the presence of polaritons. The `double' nature of the absorption in this seemingly simple sample arises from the complex interplay between absorption and the impedance the TDBC-layer presents to incoming light~\cite{Tan_JCP_2021_154_024704}. In summary, whilst the reflectivity is strongly modified in the open cavity system, the photoluminescence is unaffected, and the absorption shows little if any sign of the influence of polaritons. As noted above, we reached a very similar conclusion by looking at a second, thicker, open cavity sample, see supplementary information section S2. We defer a fuller discussion of these results until we have looked at data for the half and full cavities.

\subsection*{\textit{Half Cavity}}

%%%%%%%%%%%%%%%%%
\begin{figure}[htb!]
\centering
\includegraphics[width=1.00\linewidth]{fig/half_cavity_600nm}
\caption{{\textbf{Half Cavity}.
Data for the half cavity are shown in figure~\ref{fig:half_data}, where the layout is similar to that used for the open cavity data shown in figure~\ref{fig:open_data}. In panel (a) we show an experimentally measured dispersion, based on p-polarised reflectance, indicated as a colour plot (see methods re: normalisation).
In panel (b) we show the calculated reflectance, the white dotted lines represent the limit of the collection numerical aperture used in the experiment. Panel (c) is the same as panel (b) except that in the calculation we have set the oscillator strength of the dye to zero.
In panel (d) we show an experimentally measured dispersion plot acquired from photoluminescence data. In panel (e) we show the calculated absorption in the TDBC-doped layer. In panels (a-e) the polariton dispersion is shown, as determined from the coupled oscillator model described in the methods section. The polariton positions are shown as white dashed lines, and the molecular resonance (2.08 eV) by a black dotted line; just one cavity mode was used in the model (see panel (c)), and the polymer thickness was $\sim$600 nm. In panel (f) we show normal incidence line spectra for the measured reflectance, the measured photoluminescence and the calculated absorption in the TDBC.} In panel (f) the polariton positions (normal incidence) are indicated by black dashed lines.}
\label{fig:half_data}
\end{figure}
%%%%%%%%%%%%%%%%%

\vspace{0.3cm}
Looking first at the reflectance, panels (a) and (b
), the presence of modes of the half cavity system are again indicated by regions of low reflectivity (blue). Also shown are the positions of the polaritons as determined by our coupled oscillator model, where we made use of two cavity modes. We note that overall the match between the experimental data, the calculated data, and the coupled oscillator model are not as good as they were for the open cavity. The calculated dispersion for a `bare' cavity and the `bare' coupled oscillator model is shown in panel (c). From the modelling we determine that in this case the reduced oscillator strength is 0.15, the splitting is $\Omega = 0.50\pm0.03$ eV, and $\beta = 1.1$. The splitting exceeds the sum of the TDBC linewidth, $0.053\pm0.003$ eV, and the width of the cavity modes, $0.11\pm0.02$ eV, so that we are again, on the basis of reflectivity data anyway, in the strong coupling regime.
% Note that the splitting we find for the half cavity is very similar to that found for the open cavity discussed above. In figure S3 of the supplementary information we show data from a 1630 nm half cavity, for which we used a lower concentration of dye molecules, and observed a lower Rabi splitting.... here, 0.5 eV, is less than that we found for the thinner open cavity, 0.67 eV. This is contrary to what one might expect~\cite{Vasista_Nanoscale_2021_13_14497} but arises because we used different concentration solutions for these different cavity types; we did this to try and achieve a similar degree of Rabi splitting in the different cavities. It is for this reason that we needed to use different oscillator strengths to model the open and half-cavities.
Panel (d) shows the measured photoluminescence dispersion, things are now slightly more interesting than they were for the open cavity. The PL spectrum is significantly different from the open cavity case, but we see that there is barely any mapping of the PL onto the position of the polariton modes. Calculated data for the absorption in the TDBC are shown in panel (e). There is again a significant change in the absorption compared to that of the bulk (a single absorption peak), and further, compared to the case of the open cavity, there is now an indication of the absorption tracking the lower polariton mode, at least to some limited extent. In summary, the reflectivity is strongly modified in the half cavity system, as is the absorption, however the photoluminescence shows only very modest signs of change. Similar conclusions can be drawn from data based on a 1630 nm thick half cavity sample, figure S3 in the supplementary information.

\subsection*{\textit{Full Cavity}}

%%%%%%%%%%%%%%%%%
\begin{figure}[htb!]
\centering
\includegraphics[width=1.00\linewidth]{fig/full_cavity_355nm}
\caption{{\textbf{Full Cavity}.
In panel (a) we again show an experimentally measured dispersion plot, based on p-polarised reflectance, indicated as a colour plot. In panel (b) we show the corresponding calculated reflectance, the white dotted lines again represent the limit of the collection optics used in the experiment. In panel (d) we show the photoluminescence, whilst the calculated absorption in the TDBC is shown in panel (e). In all plots the polariton dispersion is shown, as determined from the coupled oscillator model described in the methods section. The polariton positions are shown as white dashed lines, and the molecular resonance, this time at 2.13 eV, by a black dashed line; only one cavity mode was used in the model. The cavity was 355 nm thick. In panel (f) we again show normal incidence line spectra for the measured reflectance, the measured photoluminescence and the calculated absorption in the TDBC.} In panel (f) the polariton positions (normal incidence) are indicated by black dashed lines.}
\label{fig:full_data}
\end{figure}
%%%%%%%%%%%%%%%%%

\vspace{0.3cm}
Again, as for the open and half cavities, in figure \ref{fig:full_data} 
we show the reflectance, panels (a) and (b), the presence of modes of the half cavity system are again indicated by regions of low reflectivity (blue). Also shown are the positions of the polaritons as determined by our coupled oscillator model, where we made use of two cavity modes. We see that overall there is a reasonable match between the experimental data, the calculated data, and the coupled oscillator model. The calculated dispersion for a `bare' cavity and the `bare' coupled oscillator model is shown in panel (c). In panel (f) we show normal incidence line spectra for the measured reflectance, the measured photoluminescence and the calculated absorption in the TDBC.


\vspace{0.3cm}
Looking first at the reflectance, panels (a) and (c), the presence of modes of the full cavity system are, as before, indicated by regions of low reflectivity (blue). Also shown are the positions of the polaritons as determined by our coupled oscillator model, showing a good match between the experimental data, the calculated data, and the coupled oscillator model. Notice that for the full cavity the molecular resonance is at 2.13 eV rather than 2.08 eV for the open and half cavities; this difference is due to the different nature of the systems, TDBC-doped PVA in the open and half cavities, TDBC in molecular layer-by-layer films in the full cavity~\cite{Vasista_Nanoscale_2021_13_14497}. The calculated dispersion for a `bare' cavity and the `bare' coupled oscillator model is shown in panel (c). From the modelling we determine that in this case the reduced oscillator strength is 0.03, the splitting is $\Omega = 0.23\pm0.03$ eV, and $\beta = 0.65$. The splitting exceeds the sum of the TDBC linewidth, $0.053\pm0.003$ eV, and the width of the cavity modes, $0.11\pm0.02$ eV, so that we are again, on the basis of reflectivity data, in the strong coupling regime. Panel (d) shows the measured photoluminescence dispersion, and now the influence of polaritons is very clear. The PL spectrum is significantly different from the open and half cavity cases, the the PL clearly tracking the lower polariton. Calculated data for the absorption in the TDBC are shown in panel (e). As for the PL, there is now a very significant change in the absorption, and it also clearly tracks the polariton modes. 
Experimental data for a slightly thicker (390 nm) full cavity are shown in figure S4 of the supplementary information, they show similar trends to those described above.

\vspace{0.3cm}
It is useful at this point to compare the line spectra for the PL, the absorption and the reflectance at normal incidence (i.e. $k_{\parallel}=0$ ) for each cavity type, i.e. panels (f) in figures \ref{fig:open_data}, \ref{fig:half_data} and \ref{fig:full_data}, we also refer the reader to figures S2, S3 and S4 of the supplementary information.
These data encapsulate much of what we see in the full dispersion data.

\noindent \textbf{Open cavity:} For the open cavity the absorption is modified, but doesn't track well with polariton position. The PL is unperturbed. 
Indeed, a comparison of the PL data in figure \ref{fig:open_data}(f) with the data in figure S5 of the Supplementary Information shows that the weak PL shoulder at $\sim$1.97 eV is present in the thin film and is thus unlikely to be a consequence of the open cavity structure.

\noindent \textbf{Half cavity:} For the half cavity the absorption shows clear signs of being modified by the polaritons, and the PL is also very clearly modified, although does not track the polariton position, there is only very weak dispersion of some of the PL features, see figure \ref{fig:half_data}(d). 
The thicker half-cavity does show extra features in the PL, and these may be linked to the (lower) polariton modes supported by this structure, see figure S4(d,f).

\noindent \textbf{Full cavity:} For the full cavity the absorption and the PL both track the lower polariton very nicely.
We also note that, as commonly found~\cite{Bellessa_PRL_2004_93_036404,Agranovich_PRB_2003_67_085311,Schwartz_CPC_2013_14_125}, we observe PL from polaritons at energies lower than the molecular resonance energy, see figures \ref{fig:full_data}(e) and  \ref{fig:full_data}(f), i.e. we do not see any PL associated with the upper polariton. Looking at the dispersion of the PL from the lower polariton we see that it is not uniform, PL is typically produced by the relaxation of reservoir states through the loss of vibrational energy~\cite{Coles_JPCA_2010_114_11920} so that PL emission is strongest when the difference between the bare molecular resonance energy (reservoir) and the polariton branch are equal to the energy of a vibrational mode~\cite{Vasista_Nanoscale_2021_13_14497}.


\vspace{0.3cm}
\section*{Conclusions}
We have compared the photoluminescence from TDBC aggregates located in three different cavity structures: open, half and full cavities. In all three case the reflectivity data indicate the strong coupling regime has been reached. The calculated absorption shows a somewhat different picture, with significant changes in the absorption by the TDBC for all three cavity types, but only in the case of the full cavity does the absorption track the (lower) polariton fully. The situation for photoluminescence is that there is no modification for the open cavity, some modification for the half cavity, and -- as for the absorption -- in the case of the full cavity the PL maps onto the lower polariton very well.

\vspace{0.3cm}
Previously we have looked at whether absorption by the dye is modified in the strong coupling regime, and for an open cavity we found that there is some modification. However, that study~\cite{Thomas_JPCL_2021_12_6914} made use of a broad spectrum dye (whereas TDBC is narrow-band), and no PL measurements were undertaken. Other work looking at strong coupling between dye molecules and particle plasmon modes found that there was clear PL arising from the lower polariton in the strong coupling regime~\cite{Wersall_ACSPhot_2019_6_2570}.

\vspace{0.3cm}
What are we to make of these results? Our work raises a number questions concerning the use of planar open cavities for molecular strong coupling. First, whilst it is well known that measurements of reflectance on their own are not sufficient to confirm the presence of strong coupling~\cite{Antosiewicz_ACSPhotonics_2014_1_454,Zhengin_JPCC_2016_120_20588}, we are left with the question of what level of modification of the absorption would make a convincing case? Second, and perhaps more importantly given the results reported here, does an absence of any sign of modification to the PL spectrum indicate that the changes seen in reflectivity, albeit very substantial, are not due to strong coupling? The data from our intermediate half cavity system perhaps provides a clue, showing that changes to absorption for this system \textit{\textbf{do}} show signs of the presence of polaritons whilst the PL, although not showing clear polaritonic behaviour, is nonetheless significantly modified. For the full cavity the situation is clear, absorption and PL are determined by the polariton modes of the system, as observed previously~\cite{Bellessa_PRL_2004_93_036404,Agranovich_PRB_2003_67_085311,Schwartz_CPC_2013_14_125}. It would be very beneficial if more complete theoretical models, ones that go beyond the simple coupled oscillator model we have used here, could be developed to explore the absorption and especially the PL in these structures, perhaps along the lines of previously reported models~\cite{Herrera_PRA_2017_95_053867}. It would also be useful if a future investigation looked to see if different PL excitation wavelengths lead to similar conclusions~\cite{Vasista_Nanoscale_2021_13_14497}; such an investigation was beyond the scope of the present study.\\

In summary, whilst further work is clearly needed, our investigation prompts us to question the effectiveness of planar open cavities as a means to harness strong coupling. Our results from planar half cavities lead us to suggest that these structures perhaps deserve greater attention/investigation; they are still `open' yet seem to offer a greater degree of `cavity' control. It would be especially interesting to see whether similar observations arise if these different cavity structures are explored in the vibrational (IR) regime~\cite{Verdelli_JPCC_2022_126_7143}, and perhaps using metamaterial analogues~\cite{Baraclough_ACSPhot_2021_8_2997}. Before closing we should emphasise that here we have looked at planar cavities, and have question whether planar cavity-free structures offer an effective means of providing open cavity strong coupling. We note that there are other open cavity systems, for example those based on the plasmon modes of metallic nanostructures, that do exhibit clear signatures of strong coupling via photoluminescence~\cite{Wersall_ACSPhot_2019_6_2570}.


\section*{Methods}
\noindent {\textbf{Solution preparation:}}
Poly Vinyl Acetate (PVA, molar weight 450 000) was used as a host matrix for TDBC.
PVA was dissolved in water. TDBC was then dissolved in the PVA-water solution.
%with a weight ratio of 3:2 SPI to PMMA.
TDBC/PVA films were deposited on a silicon wafer by spin-coating. This produced film thicknesses over the range 300-1500 nm.

\subsection*{\textit{Sample preparation}}

For the open-cavity a TDBC/PVA solution was spun on top of a silicon substrate. For the half-cavity a thin gold layer of 30 nm was deposited on top of silicon substrate by thermal evaporation. Later, TDBC/PVA was spun on top of gold thin film. For the full-cavity, the lower mirror was prepared by thermally evaporating 40 nm of gold onto a silicon substrate. A 100 nm thcik layer of PMMA was then added by spin coating to create part of the cavity. Then the TDBC film was deposited on top of this PMMA layer using a layer-by-layer approach~\cite{27}. Briefly, we used a cationic poly(diallyldimethylammonium chloride) (PDAC) solution as the polyelectrolyte binder for anionic TDBC J - aggregate solution. A typical deposition step consists of subsequent dipping the substrate inside a beaker of PDAC solution (20$\%$ by weight in water - diluted 1:1000) and TDBC solution in water (0.01M  diluted 1:10) for 15 minutes each. The substrate was washed with DI water after each immersion and same steps were repeated to deposit multiple layers of PDAC/TDBC. To increase the adhesion we first coat one layer of anionic polystyrene sulfonate (PSS) using the above mentioned process and continue with PDAC - TDBC. Finally the TDBC layer was protected by depositing a layer of PDAC molecules. The superstrate was then prepared by spinning a second layer of PMMA and finally the top mirror was prepared by thermally evaporating a further 40 nm of gold. 

\subsection*{\textit{Reflectance Measurements}}

Whilst substrate-only spectra for the reflectance should provide normalisation, in our samples a non-trivial level of scattering meant that the normalisation process was not as effective as hoped. Accordingly the reflectance spectra were scaled to provide a better match with calculated data.

\subsection*{\textit{Coupled Oscillator Models}}

We made use of a standard coupled oscillator model to help us interpret our data. The number of cavity resonances included in each model depended on the details of the sample (thickness etc.).
For a molecular resonance at a frequency (angular) of  $\omega_{\textrm{mol}}$ interacting with a single cavity resonance (whose energy depends on in-plane wavevector, i.e. $\omega_{\textrm{cav}}(k_{\parallel})$, the coupled oscillator model allows the new dispersion of the new hybrid (polariton) modes to be determined. They are found by solving for those frequencies $\omega$ for which the determinant of the coupling matrix is zero, i.e. $|\mathbf{M}|=0$. The coupling matrix for a single molecular resonance interacting with a single cavity (photonic) mode is given by,

\begin{align}
    \mathbf{M} = \begin{pmatrix}\omega_{\textrm{mol}}- \omega & \Omega/2 \\ \Omega/2 & \omega_{\textrm{cav}}(k_{\parallel})-\omega\end{pmatrix}
\label{CO_matrix}
\end{align}


\noindent where $\Omega$ is the extent of the interaction, and is equal to the frequency splitting (anti-crossing) at the value of $k_{\parallel}$ where the uncoupled resonances cross. For the $n^{th}$ cavity resonance we described the $k_{\parallel}$ dependence, $\omega^n_{\textrm{cav}}(k_{\parallel})$, as,

\begin{equation}
\omega^n_{\textrm{cav}} = \frac{\omega^n_{k_0 = 0}}{k_0} \sqrt{{k_0}^2 + \beta {k_{\parallel}}^2/{\varepsilon_b}},
\end{equation}
\label{eq:cavity_k}

\noindent where $\omega^n_{k_0 = 0}$ is the angular frequency of the resonance at $k_{\parallel} = 0$. The factor $\beta$ was used to match the dispersion of the bare cavity modes with the calculated bare reflectance (needed because the coupled oscillator model ignores the angle-dependant phase change on reflection from an interface). We chose to work with the simplest model that we could, and therefore assumed no damping of the modes. The values of the $\omega^n_{k_0 = 0}$, of $\Omega$ and of $\beta$ were determined by simultaneously trying to match the eigenvalues to the calculated and measured reflection data. When more thean one cavity mode was involved we made use of the 2N coupled oscillator model~\cite{Richter_APL_2015_107_231104,Balasubrahmaniyam_PRB_2021_103_L241407}. 
The values for the Rabi splitting we report here are based on the parameters used in the coupled oscillator model that best fit our data.

\subsection*{\textit{Transfer Matrix modelling}}

We made use of a standard multi-layer optics approach to calculations of the reflectance and absorption~\cite{Smith_JMO_2008_55_2929,NandH}, and assumed all media were isotropic. The reflectance was calculated by taking the square of the modulus of the amplitude reflection coefficient. For the permittivities, we used the following:

\vspace{0.2cm}
\noindent \textbf{\textit{Silicon substrate}}: we took the permittivity to be 11.76 + 0.001i.

\vspace{0.2cm}
\noindent \textbf{\textit{Gold films}}: we used a Drude-Lorentz model for the frequency/energy dependant permittivity of gold, $\varepsilon_{\rm{Au}}(\omega)$, given by,

\begin{equation}
\varepsilon_{\rm{Au}}(\omega) = \frac{f_d\,\omega_p^2}{\omega^2+i\gamma_d\omega} + \sum_j{\frac{f_j\omega_p^2}{\omega_j^2-\omega^2-i\gamma_j\omega}},
\end{equation}
\label{eq:eps_Au}

\noindent with the Lorentz oscillator parameters given by,

\begin{table}[htb!]
\centering
\begin{tabular}{| c| c| c| c |}
\hline
$j$   & $f_j$ &$\omega_j (eV)$ & $\gamma_j(eV)$ \\
\hline
1 &$0.024$ & $0.451$ & $0.241$ \\
\hline
2 &$0.010$ & $0.830$ & $0.345$ \\
\hline
3 &$0.071$ & $2.969$ & $0.870$ \\
\hline
4 &$0.601$ & $4.304$ & $2.494$ \\
\hline
5 &$4.385$ & $13.32$ & $2.214$ \\
\hline
\end{tabular}
\caption{Lorentz oscillator model parameters for gold films, as determined from ellipsometry}
\label{tab:eps_Au_parameters}
\end{table}

\vspace{0.2cm}
\noindent \textbf{\textit{TDBC j-aggregates}}: we used a Lorentz model, with the energy dependant permittivity $\varepsilon_{\rm{TDBC}}(E)$ being given by,

\begin{equation}
\label{eq:eps_TDBC}
\varepsilon_{\rm{TDBC}}(\omega) = \varepsilon_b + \frac{f\,\omega^2_0}{\omega^2_0-\omega^2-i\gamma\omega},
\end{equation}


\noindent where for the open and half cavities (TDBC dispersed in PVA) we took: $\omega_0 = 2.08$ eV, $\gamma=0.0459$ eV, and $\varepsilon_b = 2.292$; for the full cavity (TDBC layer-by-layer and PMMA) we took $\omega_0 = 2.13$ eV. The value of the reduced oscillator strength, , was determined for each sample by seeking a best 'by eye' match of the calculated and measured reflectance data. 


\section*{\label{sec:level1}Acknowledgements}
K.S.M. acknowledges financial support from the Leverhulme Trust research grant ``Synthetic biological control of quantum optics''. K.S.M. also acknowledges the support of Royal Society International Exchange grant (119893R).
The authors acknowledge the support of European Research Council through the photmat project 
(ERC-2016-AdG-742222 :www.photmat.eu).
Data associated with these results can be found at XXXXXXXXXX.   

\noindent Supporting Information Available: Optical Fourier setup; second samples/data sets for open, half and full cavities; PL spectra of thin films of TDBC, and comparison of PL spectra from open, half and full cavities.
This material is available free of charge via the Internet at *************


% +\section*{For table of contents use only}
% \begin{figure} [H] 
%   \centering
% \includegraphics[width=1   \columnwidth]{fig/PL_open_full_comparison.pdf}
% \label{fig:toc}
% \end{figure}
\newpage
\appendix
\section{\textbf{S1: Optical Fourier setup}}
Figure \ref{fig:Fourier_setup} shows a schematic of the experimental setup. To measure angle resolved reflectance, a white light source was focused onto the sample using a 0.8 NA 100x objective lens and the reflected signal was collected using the same lens. Lens L4 and L5, together, project the back-focal plane of the objective lens onto either the spectrometer or the camera, depending on the position of the flip mirror FM2. Lens L6 was a flip-lens used to project the real plane to the spectrometer/camera. For Photoluminescence (PL) measurements, a beam-expanded diode-pumped 532 nm source was focused onto the sample using the objective lens, and the PL was collected in the back-scattering configuration. The laser line was then rejected using spectral edge filters placed after lens L5. For k$\sim$0 excitation (normal incidence), we place a lens, L7, in the input path such that the laser was focused onto the back aperture of the objective lens, resulting in an approximate parallel beam with k$\sim$0 at the sample plane. 

\begin{figure}[h!]
    \centering
    \includegraphics[width=\linewidth]{fig/setup_2.png}
    \caption{Schematic of the Fourier optical setup}
    \label{fig:Fourier_setup}
\end{figure}

%\newpage
\clearpage

\section{\textbf{S2: Open Cavity (1030 nm)}}

%%%%%%%%%%%%%%%%%
\begin{figure}[h!]
\centering
\includegraphics[width=1.00\linewidth]{fig/open_cavity_1030nm}
\caption{{\textbf{Open Cavity (1030 nm)}.
In panel (a) we show an experimentally measured dispersion, based on p-polarised reflectance, indicated as a colour plot (see methods re: normalisation).
In panel (b) we show the calculated reflectance, the white dotted lines represent the limit of the collection numerical aperture used in the experiment. Panel (c) is the same as panel (b) except that in the calculation we have set the oscillator strength of the dye to zero.
In panel (d) we show an experimentally measured dispersion plot acquired from photoluminescence data. In panel (e) we show the calculated absorption in the TDBC-doped layer. In panels (a-e) the polariton dispersion is shown, as determined from the coupled oscillator model described in the methods section. The polariton positions are shown as white dashed lines, and the molecular resonance (2.08 eV) by a black dotted line; three cavity modes were used in the model (see panel (c)), and the polymer thickness was $\sim$1030 nm.} In panel (f) the polariton positions (normal incidence) are indicated by black dashed lines.}
\label{fig:SI_open_data}
\end{figure}

\vspace{0.3cm}
In figure \ref{fig:SI_open_data} we show, in the upper row, dispersion data based on reflectance for the open cavity, i.e. for a $\sim$1030 nm TDBC-doped PVA layer on a silicon substrate. The sample was similar to that used for the open cavity data shown in figure 2 of the main manuscript, the primary difference being that here the thickness of the cavity was 1030 nm rather than 340 nm used for the sample discussed in the main manuscript. On the left (a) we show the measured reflectance, in the middle the calculated reflectance (b), and in (c) we again show the calculated reflectance, but this time set the oscillator strength of the dye to zero. In the lower row  we again show a dispersion plot, panel (d), this time based on the measured photoluminescence. In panel (e) we show the calculated absorption in the TDBC-doped PVA layer, whilst in panel (f) we show normal incidence line spectra for the measured reflectance, the measured photoluminescence and the calculated absorption in the TDBC. The white dashed lines in panels (a-e) are indicate the positions of the polariton modes as determined from our coupled oscillator model (see methods), the polariton positions at normal incidence are indicated in panel (f) by black dashed lines. Further details are given in the figure caption.

\vspace{0.3cm}
Looking first at the reflectance, panels (a) and (b), the presence of modes of the open cavity system is indicated by regions of low reflectivity (blue). Also shown are the positions of the polaritons as determined by our coupled oscillator model (details in methods). We see that overall there is a reasonable match between the experimental data, the calculated data, and the coupled oscillator model (which here employed three cavity resonances). The calculated dispersion for a `bare' cavity and the `bare' coupled oscillator model is shown in panel (c). Here `bare' refers to the oscillator strength in the Lorentz oscillator model being set to zero. From the modelling we determine that in this case the reduced oscillator strength is 0.28, the splitting is $\Omega = 0.67\pm0.04$ eV, and $\beta = 1.1$ ($\beta$ is a parameter that ensures the bare cavity dispersion is correctly included in the coupled oscillator model, see methods). The splitting exceeds the sum of the TDBC linewidth, $0.053\pm0.003$ eV, and the width of the cavity modes, $0.18\pm0.04$ eV, so, if we accept reflectance as a good indicator of the allowed modes, then we are in the strong coupling regime. Panel (d) shows the measured photoluminescence dispersion, we see that there is no sign of any influence of polaritons on the photoluminescence for this open cavity. Finally, to investigate the absorption we made further use of transfer matrix modelling, the results are shown in panel (e).
%%%%%%%%%%%%%%%%%

\clearpage

\section{\textbf{S3: Half Cavity (1630 nm)}}

\begin{figure}[h!]
\centering
\includegraphics[width=1.00\linewidth]{fig/half_cavity_1630nm}
\caption{{\textbf{Half Cavity (1630 nm)}.
In panel (a) we show an experimentally measured dispersion, based on p-polarised reflectance, indicated as a colour plot.
In panel (b) we show the calculated reflectance, the white dotted lines represent the limit of the collection numerical aperture used in the experiment. Panel (c) is the same as panel (b) except that in the calculation we have set the oscillator strength of the dye to zero.
In panel (d) we show an experimentally measured dispersion plot acquired from photoluminescence data. In panel (e) we show the calculated absorption in the TDBC-doped layer. In panels (a-e) the polariton dispersion is shown, as determined from the coupled oscillator model described in the methods section. The polariton positions are shown as white dashed lines, and the molecular resonance (2.08 eV) by a black dotted line; three cavity modes were used in the model (see panel (c)), and the polymer thickness was $\sim$1630 nm.} In panel (f) the polariton positions (normal incidence) are indicated by black dashed lines.}
\label{fig:SI_half_data}
\end{figure}

\vspace{0.3cm}
The sample used here was similar to that used for the half cavity data shown in figure 3 of the main manuscript, the primary difference is that here the thickness of the cavity was 1630 nm rather than 600 nm used for the sample discussed in the main manuscript. Looking first at the reflectance, panels (a) and (b), the presence of modes of the half cavity system are again indicated by regions of low reflectivity (blue). Also shown are the positions of the polaritons as determined by our coupled oscillator model, where we made use of three cavity modes. We see that overall there is a reasonable match between the experimental data, the calculated data, and the coupled oscillator model. The calculated dispersion for a `bare' cavity and the `bare' coupled oscillator model is shown in panel (c). From the modelling we determine that in this case the reduced oscillator strength is 0.07, the splitting is $\Omega = 0.23\pm0.03$ eV, and $\beta = 1.1$. The splitting exceeds the sum of the TDBC linewidth, $0.053\pm0.003$ eV, and the width of the cavity modes, $0.11\pm0.02$ eV, so that we are again, on the basis of reflectivity data, in the strong coupling regime.
Panel (d) shows the measured photoluminescence dispersion. As for the 600 nm thick half cavity, things are now slightly more interesting than they were for the open cavity. The PL spectrum is somewhat different from the open cavity case, and there is perhaps some slight mapping of the PL onto the position of the lower polariton modes. Calculated data for the absorption in the TDBC are shown in panel (e). There is again a significant change in the absorption compared to that of the bulk (a single absorption peak), and further there is now an indication of the absorption tracking the polariton modes, at least in some very limited way.

\clearpage

\section{\textbf{S4: Full Cavity (390 nm)}}

\begin{figure}[h!]
\centering
\includegraphics[width=1.00\linewidth]{fig/full_cavity_390nm}
\caption{{\textbf{Full Cavity (390 nm)}.
In panel (a) we show an experimentally measured dispersion, based on p-polarised reflectance, indicated as a colour plot.
In panel (b) we show the calculated reflectance, the white dotted lines represent the limit of the collection numerical aperture used in the experiment. Panel (c) is the same as panel (b) except that in the calculation we have set the oscillator strength of the dye to zero.
In panel (d) we show an experimentally measured dispersion plot acquired from photoluminescence data. In panel (e) we show the calculated absorption in the TDBC-doped layer. In panels (a-e) the polariton dispersion is shown, as determined from the coupled oscillator model described in the methods section. The polariton positions are shown as white dashed lines, and the molecular resonance (2.13 eV) by a black dotted line; just one cavity mode was used in the model (see panel (c)), and the cavity thickness was $\sim$390 nm.} In panel (f) the polariton positions (normal incidence) are indicated by black dashed lines.}
\label{fig:SI_full_data}
\end{figure}

\vspace{0.3cm}
The sample used here was similar to that used for the full cavity data shown in figure 4 of the main manuscript, the primary difference being that here the thickness of the cavity was 390 nm rather than 355 nm used for the sample discussed in the main manuscript. Looking first at the reflectance, panels (a) and (b), the presence of modes of the full cavity system are again indicated by regions of low reflectivity (blue). Also shown are the positions of the polaritons as determined by our coupled oscillator model, where we made use of one cavity mode (see panel (c)). We see that overall there is a reasonable match between the experimental data, the calculated data, and the coupled oscillator model. The calculated dispersion for a `bare' cavity and the `bare' coupled oscillator model is shown in panel (c). From the modelling we determine that in this case the reduced oscillator strength is 0.04, the splitting is $\Omega = 0.27\pm0.03$ eV, and $\beta = 0.7$. The splitting exceeds the sum of the TDBC linewidth, $0.053\pm0.003$ eV, and the width of the cavity modes, $0.11\pm0.02$ eV, so that we are again, on the basis of reflectivity data, in the strong coupling regime.
Panel (d) shows the measured photoluminescence dispersion. As for the 355 nm thick full cavity, the PL spectrum is very clearly different from the open and half cavity cases, the PL nicely mapping onto the position of the lower polariton mode. Calculated data for the absorption in the TDBC are shown in panel (e). There is again a very significant change in the absorption compared to that of the bulk (a single absorption peak), the absorption now tracking the polariton modes.

\clearpage

\section{\textbf{S5: Photoluminescence from different thin TDBC layers}}

\begin{figure}[h!]
\centering
\includegraphics[width=0.9\linewidth]{fig/tdbc_comparision_all.png}
\caption{\textbf{Photoluminescence from thin TDBC layers}. Three spectra are shown: PL from TDBC produced by the layer-by-layer technique, deposited on Si; PL from TDBC-doped PVA on Si; PL from TDBC-doped PVA on Au coated Si.}
\label{fig:PL_data}
\end{figure}

\vspace{0.3cm}
Note that the PL for TDBC-doped PVA peaks at a lower energy ($\sim$2.08 eV) than the TDBC produced using the layer-by-layer technique ($\sim$2.13 eV). Note also that for the TDBC-doped PVA on Au coated Si, the low energy shoulder of the TDBC PL is enhanced in relative strength. This is consistent with a recent observation based on squarine dyes~\cite{Quenzel_ACSNano_2022_16_4693}.

%\bibliographystyle{MSP}
\bibliography{main}
%\end{sloppypar}
\end{document}


%\bibliography{references}

