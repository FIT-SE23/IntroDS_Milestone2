% \section{Software and Benchmark}
% \label{sec:software}
% % \hangx{Yichi}

% As a burgeoning research field, physics-informed machine learning needs comprehensive and well-developed software and benchmarks to promote the efficiency of its research process, including data generation, algorithm iteration and experimental verification. A brief summary of the existing tools, libraries and benchmarks is listed below, organized by their purposes.

% \subsection{Software for Numerical Solution and Simulation}

% There have been some packages and applications for multiphysics simulation with traditional numerical methods, which can provide high-precision solutions for either data-driven training or performance evaluation.

% Existing libraries for scientific computing like scipy~\cite{virtanen2020scipy} and sympy~\cite{simpy} provide simple interfaces for ordinary and first-order linear partial differential equations, which are not sufficient for the simulation of complex systems. Thus, some packages such as PolyFEM~\cite{polyfem}, py-pde~\cite{py-pde} and PyPDE~\cite{PyPDE} are proposed, which are specially designed for PDE-related computations and based on numerical methods like FEM and ADER-WENO method~\cite{dumbser2013ader}. Their capabilities of solving PDEs are still limited in the variety of equations (\textit{e.g.,} hyperbolic or parabolic PDEs) and the flexibility of spatial domain (\textit{e.g.,} rectangular and circular shapes). 

% FEniCS Project~\cite{alnaes2015fenics} is a relatively well-developed open-source software aiming at automating the solution of partial differential equations with variational formulations . It is composed of DOLFIN~\cite{logg2012dolfin}, a C++/Python library as user interface for problem solving, and a set of scientific-computing tools for computational meshes, finite element analysis, etc. FEniCS can handle equations under different scenarios and has been utilized to generate reference solutions in current research. Besides packages for programming languages, commercial software also enables analysis of complex, physics systems. 

% COMSOL~\cite{multiphysics1998comsol} is a cross-platform software for solving and simulating multiphysics systems of PDEs using finite element analysis, which provides a user-friendly IDE and off-the-shelf modules categorized according to the application areas, including electrical, fluids, acoustic, chemical engineering. It achieves a unified workflow from geometric modeling to postprocessing of the simulation results.

% \subsection{Software for Physics-Informed Machine Learning}

% As physics-informed ML is developing, specialized software libraries are proposed to support further research in the field, commonly based on the automatic differentiation mechanism in mainstream ML frameworks such as Tensorflow~\cite{tensorflow2015-whitepaper}, PyTorch~\cite{pytorch} and JAX~\cite{jax2018github}. With the rapid development of PINNs and other deep learning methods for solving PDE-driven systems, some libraries, including DeepXDE~\cite{lu2021deepxde}, SimNet~\cite{hennigh2021nvidia}, PyDEns~\cite{koryagin2019pydens}, NeuroDiffEq~\cite{chen2020neurodiffeq} and NeuralPDE~\cite{zubov2021neuralpde}, serve as neural solvers, which provide user interfaces to define the physical systems and solve the problems with encapsulated methods. Other libraries such as SciANN~\cite{haghighat2021sciann} and ADCME~\cite{xu2020adcme} provide some high-level functions to facilitate implementing scientific computing and developing physics-informed ML algorithms, which are categorized as wrappers. 

% \subsection{Benchmarks for Physics-informed Machine Learning}

% Benchmarking platforms and datasets evaluate the performance of the algorithms and further promote the development of the field. For physics-informed machine learning, equation selection and experimental settings to verify the effectiveness of the proposed methods vary widely among different studies. The variety and complexity of dynamic physical  systems makes it difficult to compare methods. Several studies have been published and may inspire more comprehensive and well-developed benchmarks to push the research forward.

% PSSBench~\cite{otness2021extensible} introduces a set of benchmark problems for physical system simulation. The problems include a single oscillating spring, a 1-d linear wave equation, a 2-d Navier-Stokes flow problem and a 2-d mesh of damped springs, while the framework can be extended to accommodate other learning tasks. In this work, the accuracy of some traditional methods such as kernel-based methods, MLP, CNN and KNN are tested on the benchmark with respect to training set size, out-of-distribution prediction, step and derivative prediction, etc. This  proposes several representative benchmark problems and examines the design with traditional ML methods. However, the test problems are limited to oscillation, wave equation and Navier-Stokes equation, the evaluation metrics mainly focus on MSE and computational overhead, and the methods do not involve recently-proposed physics-informed algorithms.

% PDEBench~\cite{takamotopdebench} was proposed later, composed of pre-computed datasets for 11 various PDE problems ranging from simple 1D equations to challenging 3D coupled systems, several ML methods for physics including PINN, FNO and U-Net, tasks of both forward learning and inverse inference, and a set of metrics beyond standard MSE to better evaluate the prediction accuracy. Nevertheless, there is still some room for improvement. For instance, the baseline methods are limited and comparison among PINN, neural operator and their variants may inspire further refinement. Besides, the evaluation metrics still only emphasize the accuracy and inference time, though a set of MSE variants are introduced. Other facets of machine learning algorithms like training dynamics and convergence should also be considered, since the training process of methods like PINN in some complex systems has been shown to be ill-conditioned and instable.

% SciMLBenchmarks~\cite{ScimlBenchmarks} provides benchmarking of different equation solver implementations, methods for parameter estimation or inverse problems and other algorithms related to differential equation, made to be a comprehensive open source benchmark for scientific machine learning (SciML) implemented in Julia. It also evaluates the influence of different components in the training of physics-informed neural networks, including cost functions and optimizers. 

% Other than the benchmarking systems for physics-informed ML introduced above, there are several benchmarks for numerical methods solving PDE-based problems that may contribute to the related research. NIST AMR Benchmark~\cite{NISTAMRBench} offers benchmarking problems and reference solutions for testing the methods of adaptive mesh refinement, along with a user-friendly filter to select the equations that interest the researchers. COMBS~\cite{COMBS} is an open-source multi-physics benchmark suite for high performance computing, where the problems may inspire the benchmarking in complex physics systems. OPT-PDE~\cite{OPTPDE} is a collection of problems in PDE-constrained optimization, showing various kinds of features and difficulties.