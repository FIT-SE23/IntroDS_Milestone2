\section{Conclusion}
\label{sec:conclusion}
% \hangx{Zhongkai, hang}


In this review, we have systematically investigated and summarized the field of physics-informed machine learning as seen through the eyes of machine learning researchers. First, we have identified and introduced the general concept of physics-informed machine learning. We suggest that there are several types of physical prior , i.e. PDEs/ODEs/SDEs, symmetry constraints and intuitive physics. They could be embedded into different parts of machine learning models, i.e. data, architecture, loss functions, optimization methods and inference algorithms. 
Then, we exhaustively presented existing methods, challenges, and future directions for these problems. Most of existing works focus on using neural networks for solving or identifying systems governed by PDEs/ODEs, i.e. neural simulation and inverse design. We have summarized the progress of these methods in detail.


From a methodological perspective, there are many open challenges for problems of physics-informed machine learning.
\begin{itemize}
    \item How to design a standardized dataset for problems of different physical priors is an open challenge. Datasets and benchmarks provide a fair environment to compare different algorithms and inspire researchers to discover new ones. In the field of physics-informed machine learning, such datasets are lacking due to the diversity of physical priors. A valuable, realistic dataset or benchmark containing either a single level or multiple levels of physical prior will play important role in boosting the machine learning community.
    
    \item Designing better optimization and inference algorithms incorporating or informed by physical priors is a valuable topic. While there are many works exploring model architectures inspired or to represent physical prior, optimization methods and inference algorithms receive less attention. Different from constraining the hypothesis space, novel optimization methods incorporating physical prior might be another choice to train physics-informed machine learning models. Moreover, when models are pre-trained, we might also need better inference algorithms to ensure the output satisfies physical laws.
    

    \item Scalable algorithms incorporating data with real-world physical prior or intuitive physics is a basic issue for building intelligent systems capable of real-world interaction. It might be a prevailing trend for developing machine learning models driven by both large data and physical prior with a broad range of applications in computer vision and robotics control. 
\end{itemize}

From the perspective of tasks of physics-informed machine learning, we also suggest several research directions and opportunities.
\begin{itemize}
    \item For neural simulation, existing optimization techniques and architectures are far from optimal. There is still a gap between PINNs and highly specialized traditional numerical methods like FEM/FVM/FDM/Spectral methods on both speed and accuracy. Promising future research directions include inventing novel and effective optimization targets, learning paradigms and neural architectures. 
    \item Inverse problems are fundamental problems in scientific discovery, computer vision as well as many other engineering domains. Many works show that methods based on PINNs and DeepONets achieve better results on inverse problems compared with traditional methods. However, the complexity and ill-posedness requires learning algorithms that better utilize data and physical knowledge. Moreover, inverse design, as a specific type of inverse problem has many  promising application scenarios in structural/topology optimization, optimal control and molecular/drug discovery. 
    \item For computer vision and reinforcement learning in real world, we need better algorithms for inference stage or training stages that incorporates physical prior. A possible direction is to model the real world environment with physical prior. The emergence of concepts like world models or NERF provides a possible method that we could learns the real world environment from data with the help of physical prior. Such models could then be used for training downstream models interacting with the world.
    
    \item  Theoretical analysis like convergence and generalization ability for physics-informed machine learning algorithms are still at a beginning. It is still a challenging task due to the difficulty of analyzing the training process of neural networks. We even don't know what is the theoretical benefits of introducing physical prior into machine learning.
\end{itemize}

We conclude that physics-informed machine learning will be a fundamental and essential topic of AI. There is still much potential for improving current methods of physics-informed machine learning.


