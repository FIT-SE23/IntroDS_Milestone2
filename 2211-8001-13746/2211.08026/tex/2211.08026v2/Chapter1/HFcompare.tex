Floer cohomology of a symplectomorphism $f$ can be viewed as
Lagrangian Floer cohomology of the pair $(\Delta, \Gamma_{f})$.
This isomorphism is well-known, see for instance \cite{WW}, \cite{MakWu} and \cite[section 2.7]{leclercq-zapolsky}.
Namely we have
\begin{prop}\label{HFcompare:prop}
There is a canonical graded isomorphism $\Psi_{f} \colon \HF(f) \to \HF(\Delta, \Gamma_{f})$.
\end{prop}
\noindent
For the convenience of the reader we include a sketch of the proof in section \ref{sec:ADD}.

Let $\varphi \colon M \to M$ be an anti-symplectic involution.
Consider the symplectomorphism
\begin{align*}
     \Phi^{\varphi} \colon M \times M^- \longrightarrow M \times M^- \\
     (x,y) \longmapsto (\varphi(y), \varphi(x)).
\end{align*}
The map $(\varphi f^{-1})_*$ on $\HF^*(\tau_S^{-1})$ corresponds to $\Phi^{\varphi}_*$
under the isomorphism of Proposition \ref{HFcompare:prop}, i.e. the following diagram commutes:
\begin{align*}
    \xymatrix{
&\HF^*(f^{-1}) \ar[r]^{\left(\varphi f^{-1}\right)_*} \ar[d]^{\Psi_{f^{-1}}} &\HF^*(\varphi f \varphi^{-1}) \ar[d]^{\Psi_{\varphi f \varphi ^{-1}}}\\
&\HF^*(\Delta, \Gamma_{f^{-1}}) \ar[r]^{\Phi^{\varphi}_*} &\HF^*(\Delta, \Gamma_{\varphi f \varphi ^{-1}})
}
\end{align*}
As a special case, we recover the commutative diagram (\ref{Ioutline:diag:phi_c}) by setting $f = \tau_S$ and $\varphi = \tilde{c}$.
