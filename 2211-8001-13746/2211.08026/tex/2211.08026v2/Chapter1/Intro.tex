Let $(M,\omega)$ be a closed symplectic manifold and $S\subset M$ a Lagrangian sphere with a parametrization $\iota\colon S^n \xrightarrow{\approx} S$.
Associated to $(S,\iota)$ there exists a distinguished symplectic isotopy class represented by the \textit{Dehn twist}. The Dehn twist $\tau_S$ is a symplectomorphism compactly supported in a neighbourhood of $S$.
Seidel proved that the square of the Dehn twist, in some cases, is not symplectically, but only smoothly isotopic to the identity \cite{seidelthesis97}, \cite{seidel_lectures}.
To prove this result Seidel established a Floer homology exact sequence
\begin{align}\label{I:eqLES}
    \dots \to (\HF^*(S,N) \otimes \HF^*(Q,S))^{k} \to \HF^{k}(Q,N) \to 
    \HF^k(Q,\tau_s(N)) \to \dots
\end{align}
for admissible Lagrangian submanifolds $Q$ and $N$ in $M$
  \cite{seidel03},\cite{seidelbook}.
There is a distinguished element $A\in \HF^*(\tau^{-1}_S)$ that characterizes the map $\HF^k(Q,N) \to HF^k(Q,\tau_S(N))$ that occurs in the sequence. 
\begin{comment}
By work of Mak-Wu in \cite{MakWu} the sequence (\ref{I:eqLES}) is a special case of a long exact sequence of Floer cohomology groups of Lagrangian submanifolds in the product manifold $M\times M^- := \left( M\times M, \omega \oplus -\omega\right)$:
\begin{align}\label{I:eqLES2}
    \dots \to \HF^k(K,S\times S) \to \HF^k(K,\Delta) \to \HF^k(K,\Gamma_{\tau_S^{-1}}) \to \HF^{k+1}(K,S\times S) \to \dots,
\end{align}
where $K$ is an admissible Lagrangian submanifold in $M\times M^-$.
For the special case $K=Q\times N$, this sequence reduces to the long exact sequence (\ref{I:eqLES}).
The middle map in the sequence (\ref{I:eqLES2}) can be understood as $\mu^2(A,-)$ for a certain element $$A\in \HF^0(\Delta, \Gamma_{\tau_S^{-1}}) \cong \HF^0(\tau_S^{-1}).$$
\end{comment}

Due to the relevance of the above exact sequence it is thus natural to investigate properties of the element $A$.
The goal of this paper is to study the element $A$ in the situation,
where there exists an anti-symplectic involution that preserves $S$.

We work in the following setting. $(M, \omega)$ is a closed
symplectically aspherical symplectic manifold. Unless otherwise explicitely stated, all involved Lagrangian submanifolds are assumed to be closed, oriented and relatively symplectically aspherical. 
Floer cohomology groups are $\mathbb{Z}_2$-graded with coefficients in 
the universal Novikov field over $\mathbb{Z}_2$. More details about these assumptions are given in section \ref{subsec:HFsetting}.

Let $c\colon M \to M$ be an anti-symplectic involution satisfying $c(S)=S$. Consider the smooth involution $\iota^* c:=\iota^{-1}c\iota \colon S^n \to S^n$. We assume that $\iota^*c$ is either smoothly isotopic to the identity or to the reflection
$r(x_1,x_2,\dots,x_{n+1})= (-x_1,x_2,\dots,x_{n+1}).$
This assumption is satisfied in the important geometric setting where $(M,c)$ is a real fiber of a real Lefschetz fibration with one critical point and $S$ is the corresponding vanishing sphere.

Under this assumption, our main result is

\begin{thmx}\label{I:thm:main}
    $c$ induces an automorphism 
    $c_* \colon \HF^*(\tau_S^{-1}) \to \HF^*(\tau_S^{-1})$
    and $c_*(A) = A$.
\end{thmx}

\begin{rem}
$c_* \colon \HF^*(\tau_S^{-1}) \to \HF^*(\tau_S^{-1})$ is an involution
of a vector space over a field with characteristic $2$. Any such map
has a fixed point because $(c_*-\mathrm{id})^2=0$, hence $\mathrm{ker}(c_*-\mathrm{id})\neq 0$. The relevance of the second part of Theorem \ref{I:thm:main} is therefore not merely the existence of a fixed point. It should rather be understood as a special property of the element $A$.
\end{rem}