The rest of this paper is organised as follows.
In section \ref{sec:REALlefschetz} 
we explain the construction of real Lefschetz fibrations and the decomposition of the monodromy into two anti-symplectic involuions as stated in Propositions \ref{prop:Lefschetz} and \ref{R:prop}.
In section \ref{sec:HF} we fix the setting and collect the properties of Floer cohomology we need.
In section \ref{sec:LC} we briefly recall Biran-Cornea's Lagrangian cobordism framework and how cobordisms induce cone decompositions.
Section \ref{sec:MW} recalls the construction of the Mak-Wu cobordism. In section 
\ref{sec:INV} we prove Theorem \ref{Ioutline_of_proof:thm:invariance} about the symmetry of the cobordism. 
Section \ref{sec:ADD} contains some more background material on Floer cohomology
for the convenience of the reader.
The \hyperref[appendix]{appendix} contains some algebraic background on Fukaya categories.