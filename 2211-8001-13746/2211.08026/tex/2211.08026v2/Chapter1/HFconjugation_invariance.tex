Let $f$ be a symplectomorphism on $X$ and $\varphi$ be an antisymplectic diffeomorphism on $X$.
We will make substancial use of the following fact, which is an anti-symplectic version of the well-known conjugation invariance of Floer cohomology (see e.g. \cite[section 3]{seidel_lectures}).
We include a proof in section \ref{subsec:AdddefHF}.
\begin{prop}\label{HFconjugation_invariance:thm:main}
There is a canonical graded isomorphism
$$(\varphi f^{-1})_*\colon \HF^*(f^{-1}) \to \HF^*(\varphi f \varphi^{-1}).$$
\end{prop}

\begin{comment}
\begin{proof}
Let $(\mathcal{J},H)$ be a Floer datum for $f^{-1}$.
Then $(\mathcal{J}', K)$, defined by
\[
    K_s := H_{1-s} \circ \varphi^{-1}
\]
and
\[
    J_s' := -(\varphi^{-1})^*J_{1-s}
\]
is an admissible Floer datum for $\varphi f \varphi ^{-1}$.
A straight-forward calculation shows that
\begin{align*}\label{HFconjugation_invariance:eq}
    CF^*(f^{-1}; \mathcal{J},H) &\longrightarrow 
    CF^*(\varphi f \varphi ^{-1}; \mathcal{J}', K) \\
    P_{f^{-1}}(H) \rotatebox[origin=c]{180}{\in} x &\longmapsto \varphi (f^{-1}(x))
\end{align*}
defines a $\Lambda$-linear isomorphism of $\Z/2$-graded cochain complexes.
Moreover, the degree in preserved.
%%%%%%%%%%%%%%%%%%
Moreover, the degree is preserved:
Let $y\in P_{f^{-1}}(H)$. Then
\begin{align*}
    \det(\mathrm{Id} - D((\psi_1^K)^{-1}\varphi f \varphi^{-1})_{\varphi(f^{-1}(y(0)))})
    &= \det(\mathrm{Id} - D(\varphi \psi_1^H f \varphi^{-1})_{\varphi(f^{-1}(y(0)))})\\
    &=
    \det(\mathrm{Id} - D(\psi_1^H f )_{f^{-1}(y(0))})\\
    &= \det(\mathrm{Id} - D(f\psi_1^H )_{y(0)})\\
    &= \det(\mathrm{Id} - D(f\psi_1^H )_{y(0)}^{-1})\\
    &= \det(\mathrm{Id} - D((\psi_1^H)^{-1}f^{-1})_{y(0)}).
\end{align*}
In the first equality, we used
$(\psi^K_1)^{-1} = \varphi \psi_1^H \varphi^{-1}$,
the second follows from factoring out $D\varphi$ and $D\varphi^{-1}$
and using multiplicativity of $\det$, the third follows from multiplying with $Df$ and $Df^{-1}$ from left resp right,
the fourth used that $D(f\psi_1^H)$ is symplectic.

(For a symplectic matrix $A$ one has
\begin{align*}
 \det(\mathrm{Id} - A^{-1}) &= \det (\mathrm{Id} - J_0^{-1}A^T J_0)\\
 &= \det (\mathrm{Id} - A^T)\\
 &= \det (\mathrm{Id}-A).)
\end{align*}

This clearly implies $\mathrm{deg}(y) = \mathrm{deg}(\varphi(f^{-1}(y))$, which means that the isomorphism above preserves the degree.
%%%%%%%%%%%%%%%%%%%%%%

Concatenation of this chain-level isomorphism with a continuation map shows
Theorem \ref{HFconjugation_invariance:thm:main}.
\end{proof}
\end{comment}

If $\tau_S = c \circ \tilde{c}$ we can apply this result to $\varphi = \tilde{c}$ and $f= \tau_S$.
We get an automorphism
\begin{align*}
c_* \colon \HF^*(\tau_S^{-1}) \longrightarrow \HF^*(\tilde{c}\tau_S \tilde{c}) \cong \HF^*(\tau_S^{-1}).
%(c_-)_* \colon HF^*(\tau) \longrightarrow HF^*(c_+ \tau^{-1} c_+) \cong HF^*(\tau).
\end{align*}
This is induced by the chain-level map sending a generator $x$ to $\tilde{c}\tau_S^{-1}(x) = c(x)$, concatenated with a continuation map.

%%%%%%%%%%%%%%%%%%%%%%%%%%%%%%%%%%%%%%%%%%%%%%%%%%%%%%%%%%%%%%%%%%
%%%%%%%%%%%%%%%%%%%%%%%%%%%%%%%%%%%%%%%%%%%%%%%%%%%%%%%%%%%%%%%%%%
%%%%%%%%%%%%%%%%%%%Long version of proof
%%%%%%%%%%%%%%%%%%%%%%%%%%%%%%%%%%%%%%%%%%%%%%%%%%%%%%%%%%%%%%%%%%
%%%%%%%%%%%%%%%%%%%%%%%%%%%%%%%%%%%%%%%%%%%%%%%%%%%%%%%%%%%%%%%%%%
\begin{comment}
\begin{proof}
To prove this, choose admissible Floer datum for $f^{-1}$:
Hamiltonian functions $H_s$
with $H_s = H_{s+1} \circ f^{-1}$ and $\omega$-compatible almost complex structures $J_s$ with $J_s = (f^{-1})^*J_{s+1}$. Denote the Hamiltonian vector field of $H_s$ by $X_s^H$ and its flow by $\Psi_t^H$.
We call $g:= \varphi f \varphi^{-1}$.
As Floer datum for $HF^*(\varphi f \varphi^{-1})$ we use
$$K_s:= H_{1-s}\circ \varphi^{-1}$$
and
$$\tilde{J}_s := - (\varphi^{-1})^* J_{1-s}.$$
Note that the choices make sure that
\begin{align*}
 K_{s+1} \circ g  &= H_{-s} \varphi^{-1} \varphi f \varphi^{-1} \\
                  &= H_{1-s} \varphi^{-1} = K_s
\end{align*}
and 
\begin{align*}
 g^* \tilde{J}_{s+1} &= - g^* (\varphi^{-1})^* J_{-s} \\
                     &= - (\varphi^{-1})^*f^*J_{-s}\\
                     &= - (\varphi^{-1})^* J_{1-s} = \tilde{J}_s.
\end{align*}
The Hamiltonian vector field of $K$ is given by
$$X_t^K(z) = - D\varphi X_{1-t}^H (\varphi^{-1}(z))$$
and its flow relates to $\psi^K$ by
$$\Psi_t^K = \varphi \Psi_{1-t}^H (\Psi_1^H)^{-1} \varphi^{-1}.$$
We define a map
\begin{align*}
  \alpha \colon CF^*(g; K, \tilde{J}) \longrightarrow CF^*(f^{-1}; H, J)
\end{align*}
by sending generators $x$ of $CF^*(g; K, \tilde{J})$
to $f(\varphi^{-1}(x))$, which is a generator of $CF^*(f^{-1};H,J)$.
Indeed, $x$ satisfies $g(x)=\Psi^K_1(x)$ and hence
\begin{align*}
  f \Psi_1^H f \varphi^{-1}(x) = 
  f \varphi ^{-1} (\Psi_1^K)^{-1} \varphi f \varphi^{-1}(x)
  = f \varphi ^{-1} (\Psi_1^K)^{-1} g (x)
  = f \varphi ^{-1} (x).
\end{align*}
So $\alpha (x) = f \varphi ^{-1} (x)$ is a generator.

\noindent
This is clearly an isomorphism of vectorspaces
with inverse $\beta(x) = \varphi(f(x))$.
In fact, it is an isomorphism of chain complexes.
This follows from the bijection
\begin{align*}
 \mathcal{M}(\beta(x), \beta(y)) &\leftrightarrow \mathcal{M}(x,y)\\
 v(s,t) = \varphi(u(1-s,t)) &\leftmapsto u(s,t)
\end{align*}

\noindent
\textit{Check that this is well-defined:}

\begin{align*}
  &\frac{\partial v}{\partial t}\vert_{(s,t)} = D \varphi \frac{\partial u}{\partial t}
     \vert_{(1-s,t)}\\
  &\frac{\partial v}{\partial s}\vert_{(s,t)}  = - D \varphi \frac{\partial u}{\partial s} \vert_{(1-s,t)}\\
  &\tilde{J}_s(v)\vert_{(s,t)}  = - D\varphi J_{1-s}D\varphi^{-1}(\varphi(u)) \vert_{(1-s,t)}\\
  &X_s^K(v)\vert_{(s,t)} = -D\varphi X_{1-s}^H(u) \vert_{(1-s,t)}
\end{align*}
and therefore,
\begin{align*}
  \frac{\partial v}{\partial t} + \tilde{J}_s(v) 
     \left(\frac{\partial v}{\partial s} - X_s^K(v)\right)\vert_{(s,t)}
   = D\varphi \left \{ 
   \frac{\partial u}{\partial t} + J_{1-s}(u) 
     \left(\frac{\partial u}{\partial s} - X_{1-s}^H(u)\right) \right
     \vert_{(1-s,t)} \}
   = 0.
\end{align*}

\end{proof}
\end{comment}
 
