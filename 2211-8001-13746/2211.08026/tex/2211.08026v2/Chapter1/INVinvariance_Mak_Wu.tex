%20.12_Summary(2)
Let $V\subset M$ be a Weinstein neighbourhood of $S$ and
$\varphi\colon V \to T_{\delta}^*S^n$ a symplectomorphism
for some $\delta>0$.
Let $0 < \epsilon < \delta$ small enough, such that $c(U) \subset U$ 
for $U:= \varphi ^{-1}(T_{\epsilon}^*S^n)$.
Consider 
$\Phi$
as a map
\[
 U \times U \times \mathbb{C} \to V \times V \times \mathbb{C}.
\]
This induces via $\widetilde{\varphi}\times \mathrm{id}$ the map
\begin{align*}
\Phi \colon T_\epsilon^*S^n \times T_\epsilon^*S^n  \times \mathbb{C} &\to T_\delta^*S^n  \times T_\delta^*S^n \times \mathbb{C} \\
(\xi_1,\xi_2,z) &\mapsto (c(-\xi_2),c(\xi_1),z).
\end{align*}
Let $\psi_t \colon T^*S^n \to T^*S^n$ be the Hamiltonian isotopy from Proposition \ref{INVlinear:prop:main}. Consider the Hamiltonian isotopy
\begin{align*}
\Psi_t\colon T_\delta^*S^n \times T_\delta^*S^n \times T^*\mathbb{R} &\to T_\delta^*S^n \times T_\delta^*S^n \times T^*\mathbb{R} \\
(\xi_1,\xi_2, p) &\mapsto (\psi_t(\xi_1), -\psi_t(-\xi_2),p).
\end{align*}
We consider surgery in $T_{\epsilon}^*S^n$, so that the handle
$\hat{H}_{\nu}$ is contained in $T_{\epsilon}^*S^n$.
We claim that
\begin{itemize}
\item $\Psi_1(\hat{H}_{\nu}) = \Phi(\hat{H}_{\nu})$,
\item $\Psi_t\vert_{S \times S \times \mathbb{R}} = \mathrm{id}$,
\item $\Psi_t(N_{\Delta_S}^* \times\{p\}) = N_{\Delta_S}^* \times \{p\}$ for any $p\in i \R$,
\item $\pi_\C \circ \Psi_t = \pi_\C$.
\end{itemize}
We check these properties:
\begin{itemize}
\item 
Let $\xi \in T^*S$ and $(q,p) \in T^*\R$ such that $$\sqrt{\norm{ \xi } ^2 + p^2}< \epsilon.$$
We introduce the following abbreviations:
\[
    s(\norm{ \xi }, \lvert p \rvert ) = \nu\left(\sqrt{\norm{ \xi } ^2 + p^2}\right) \frac{\norm{ \xi }}{\sqrt{\norm{ \xi } ^2 + p^2}}
\]
and
\[
    r(\norm{ \xi }, \lvert p \rvert ) = \nu\left(\sqrt{\norm{ \xi } ^2 + p^2}\right) \frac{\lvert p \rvert}{\sqrt{\norm{ \xi } ^2 + p^2}}.
\]
Elements of $\hat{H}_{\nu}$ are of the form 
$$\alpha :=(\xi, \psi_{s(\norm{\xi}, \lvert p \rvert)}^\sigma (-\xi), (r(\norm{ \xi }, \lvert p \rvert) + q,p)).$$
Therefore, elements of $\Psi_1(\hat{H}_{\nu})$ are of the form
$$\Psi_1(\alpha) = (cc_0^* (\xi), -cc_0^*(-\psi_{s(\norm{ \xi})}^\sigma (- \xi)), (r(\norm{ \xi }, \lvert p \rvert) + q,p)).$$
Put
$$\zeta:= c_0^*(-\phi_s^\sigma(-\xi)).$$
Then $\norm{ \zeta } = \norm{ \xi }$
by part $2$ of Lemma \ref{INVlinear:lem:properties}. By part $1$ of Lemma \ref{INVlinear:lem:properties} we have the equality
\[
    c_0^*(\xi) = -\psi_s^\sigma(-c_0^*(-\psi_s^\sigma(-\xi)))= - \psi_s^{\sigma}(-\zeta).
\]
Thus
\[
    \Psi_1(\alpha) = (c(-\psi_{s(\norm{ \zeta })}^\sigma (-\zeta)), -c(\zeta), (r(\norm{ \zeta }, \lvert p \rvert ) + q, p))
\]
which are precisely the elements of $\Phi(\hat{H}_{\nu})$. 
\item ${\Psi_t}(S\times S \times \mathbb{R}) = S\times S \times \mathbb{R}$: For $(x,y)\in S \times S$
\begin{align*}
\Psi_t(x,y,p) &= (\psi_t(x), -\psi_t(-y),p) \in S \times S \times \R
\end{align*}
because $\psi_t(S) = S$.
\item $\Psi_t(\xi, -\xi,p) = (\psi_t(\xi),-\psi_t(\xi),p)$ and thus 
 $\Psi_t(N_{\Delta_S}^* \times\{p\}) = N_{\Delta_S}^* \times \{p\}$.
\item $\pi_\C \circ \Psi_t = \pi_\C$ is clear.
\end{itemize}

Thus the surgery model $(S\times S \times \R) \#_{\Delta_S}N^*_{\Delta_S \times \{0\}}$
is Hamiltonian isotopic to the image of itself under $\Phi$.
The smoothing and the extension to a cobordism with ends
$S\times S$, $\Delta$ and $(S\times S) \#_{\Delta_S} \Delta$ can be done while keeping the Hamiltonian isotopy type.
Hence the surgery part of the cobordism is Hamiltonian isotopic to the image of itself under $\Phi$.

\begin{comment}
Let $S\subset M$ be a Lagrangian sphere, $c\colon M \to M$ an anti-symplectic involution with $c(S) =S$.
Choose a Riemannian metric $g$ on $S$ such that $c\vert_S \colon S \to S$ is an isometry. Let $U\supset S$ be a neighbourhood and $\varphi\colon U \to T_\delta^*S$ be a symplectomorphism with $\varphi(S) = 0_S$. The symplectomorphism 
$$\varphi \times (-\varphi) \colon (U \times U, \omega \oplus -\omega) \to
(T_\delta^*S \oplus T_\delta^* S, \omega_0 \oplus \omega_0) \subset T^*(S \times S)$$
gives a tubular neighbourhood of $S\times S \subset M \times M^-$, and
$$(\varphi \times (-\varphi))^{-1}(N_{\Delta_S}^* \cap \left(T_\delta^*S \oplus T_\delta^*S)\right)) = \Delta.$$
Choose $\epsilon>0$ with $\epsilon <\delta$ small enough such that
$$c(\varphi^{-1}(T_\epsilon^*S)) \subset V.$$
Set $V_0:= \varphi^{-1}(T_\epsilon^*S)$.
Then $c\colon V_0 \to V$ induces $c\colon T_\epsilon^* \to T_\delta^*S$.

Consider 
\begin{align*}
\Phi \colon V_0 \times V_0 \times \mathbb{C} &\to V \times V \times \mathbb{C} \\
(x,y,z) &\mapsto (c(y),c(x),z).
\end{align*}
This induces \begin{align*}
\Phi \colon T_\epsilon^*S \times T_\epsilon^*S  \times \mathbb{C} &\to T_\delta^*S  \times T_\delta^*S \times \mathbb{C} \\
(\xi_1,\xi_2,z) &\mapsto (c(-\xi_2),c(\xi_1),z).
\end{align*}
Consider the $E_2$-flow handle $H^{E_2}$ within $T_\epsilon^*S$.
We have to find a compactly supported Hamiltonian isotopy
$$\Psi_t \colon T_\delta^*S \times T_\delta^*S \times T^*\mathbb{R} \to T_\delta^*S \times T_\delta^*S \times T^*\mathbb{R}$$
such that 
\begin{itemize}
\item $\Psi_1(H^{E_2}) = \Phi(H^{E_2})$,
\item $\Psi_t\vert_{S \times S \times \mathbb{R}} = id$,
\item $\Psi_t(N_{\Delta_S}^* \times\{p\}) = N_{\Delta_S}^* \times \{p\}$,
\item $\pi_\C \circ \Psi_t = \pi_\C$.
\end{itemize}

\begin{lem}
Suppose there is a compactly supported Hamiltonian isotopy
$$\phi_t^F \colon T_\delta^*S \to T_\delta^*S$$
with $\phi_0^F = id$ with $c_t:= c\phi_t^F$ and $\phi_t^F \vert_S = id$
and $\tilde{c}$ is the dual of the map $-D(c\vert_S)$.
\end{lem}
Define 
\begin{align*}
\Psi_t\colon T_\delta^*S \times T_\delta^*S \times T^*\mathbb{R} &\to T_\delta^*S \times T_\delta^*S \times T^*\mathbb{R} \\
(\xi_1,\xi_2, p) &\mapsto (cc_t(\xi_1), -cc_t(-\xi_2)p).
\end{align*}
This is a Hamiltonian isotopy generated by
$$K_t(\xi_1,\xi_2,p) = F_t(\xi_1) + F_t(-\xi_2).$$
We check the properties:
\begin{itemize}
\item Elements of $H^{E_2}$ are of the form 
$$\alpha :=(\xi, \phi_{s(\vert \vert \xi \vert \vert, \vert \vert p \vert \vert)}^\sigma (-\xi), (r(\vert \vert \xi \vert \vert, \vert \vert p \vert \vert) + q,p))$$
for $\xi \in T^*S$ and $\vert \vert \xi \vert \vert < \epsilon$.
Elements of $\Psi_1(H^{E_2})$ are of the form
$$\Psi_1(\alpha) = (c\tilde{c} (\xi), -c\tilde{c}(-\psi_{s(\vert\vert \xi\vert \vert)}^\sigma (- \xi)), (r(\vert \vert \xi \vert \vert, \vert \vert p \vert \vert) + q,p))$$
for $\xi$ as above.
Elements of $\Phi(H^{E_2})$ are of the form
$$(c(-\psi_{s(\vert \vert \zeta \vert \vert)}^\sigma (-\zeta)), -c(\zeta), (r(\vert \vert \zeta\vert \vert, \vert \vert p \vert \vert ) + q, p))$$
for $\zeta \in T^*S$ and $\vert \vert \xi \vert \vert < \epsilon$.
Note that
$\tilde{c}(\xi) = -\phi_s^\sigma(-\tilde{c}(-\phi_s^\sigma(-\xi)))$.
Put
$\zeta:= \tilde{c}(-\phi_s^\sigma(-\xi))$.
Then $\vert \vert \zeta \vert \vert = \vert \vert \xi \vert \vert$.
This shows that $\Psi_1(\alpha)$ are precisely the elements in $\Phi(H^{E_2})$.
\item ${\Psi_t}\vert_{S\times S \times \mathbb{R}} = id$: 
\begin{align*}
\Psi_t(x,y,p) &= (cc_t(x), -cc_t(-y),p) \\
&=(\phi_t^F(x), -\Phi_t^F(-y),p) \\
&= (x,y,p)
\end{align*}
because $\phi_t^F \vert _ S = id$.
\item $\Psi_t(\xi, -\xi,p) = (cc_t(\xi),-cc_t(\xi),p)$ and thus 
 $\Psi_t(N_{\Delta_S}^* \times\{p\}) = N_{\Delta_S}^* \times \{p\}$.
\item $\pi_\C \circ \Psi_t = \pi_\C$ is clear.
\end{itemize}

\end{comment}
