Mak-Wu \cite{MakWu} explain how to put gradings on $S\times S$, $\Delta$,
$\Gamma_{\tau_S^{-1}}$ and on $V_{MW}$ such that $V_{MW}$ becomes a graded cobordism from $\Gamma_{\tau_S^{-1}}$ to $(S\times S[1], \Delta)$.
Here, we only use $\Z/2$-gradings, but it follows from their proof, that $V_{MW}$ is an oriented cobordism.

In the situation of symplectically aspherical manifolds, $V_{MW}$
is a relatively symplectically aspherical Lagrangian in $M\times M^- \times \C$ with relatively symplectically aspherical ends.
More precisely, assuming $\omega \vert _ {\pi_2(M)} \equiv 0$
and $\omega \vert _ {\pi_2(M,S)} \equiv 0$ implies that 
$(\omega \oplus -\omega)\vert _{\pi_2(M\times M^-, S\times S)} \equiv 0$,
$(\omega \oplus -\omega)\vert _{\pi_2(M\times M^-, \Delta)} \equiv 0$,
$(\omega \oplus -\omega)\vert _{\pi_2(M\times M^-, \Gamma_{\tau_S^{-1}})} \equiv 0$ and
 $(\omega \oplus -\omega \oplus \omega_{\C}) \vert _ {\pi_2(M\times M^- \times \C, V_{MV})} \equiv 0$.
The latter follows from an argument very similar to the proof of the corresponding result on exactness and monotonicity in \cite[Lemma 6.2, 6.3]{MakWu}.
\begin{comment}
If $n\geq 2$, $\pi_1(S\times S\times \R, \Delta_S \times \{0\}) = 0$ and $\pi_2(M\times M^- \times i \R, \Delta \times \{0\}) = 0$ enure that any element of $\pi_2(M\times M^- \times \C, V)$ can be represented 
by $v\colon \mathbb{D}^2 \to M\times M^-\times \C$ with
$v(\partial \mathbb{D}^2) \subset (\Delta\times i \R) \cup \hat{H}_{\nu}$. The latter is relatively symplectically aspherical, because it is Hamiltonian isotopic to $\Delta \times i \R$. Therefore also $V$ is relatively symplectically aspherical. The rest of the construction of $V_{MW}$ does not change this property.

For $n=1$ we can use the Mayer-Vietoris exact sequence to show that $\omega\oplus -\omega \oplus \omega_{\C}$ vanishes on $H_2(M\times M^- \times \C ,  V_{MW}$.
\end{comment}

Floer theory for $V_{MW}$ and the ends is therefore well-defined.
The cone-decomposition result
\ref{subsec:LCconedecomposition}
from Biran-Cornea \cite{BC1}, \cite{BC2} therefore yields a long exact sequence of graded
Lagrangian Floer cohomology groups \cite[Theorem 6.4]{MakWu}:
\begin{align*}
\dots \to \HF^k(K,&S\times S) \xrightarrow{\mu^2(B,-)} \HF^{k}(K,\Delta) \\ &\xrightarrow{\mu^2(A,-)} \HF^{k}(K,\Gamma_{\tau^{-1}}) 
\xrightarrow{\mu^2(C,-)} \HF^{k+1}(K,S\times S) \to \dots
\end{align*}
for any admissible Lagrangian submanifold $K\subset \left(M\times M, \omega \oplus -\omega\right)$.
This is precisely the sequence (\ref{I:eqLES2}).
As indicated, the maps are given by $\mu^2$ operations with elements
$A \in \HF^0(\Delta, \Gamma_{\tau^{-1}})$,
$B \in \HF^0(S\times S, \Delta)$
and $C \in \HF^1(\Gamma_{\tau_S^{-1}},S\times S)$.
$A, B$ and $C$ are independent of $K$.

\begin{prop}
    If $2c_1(M) = 0$ in $\Homol^2(M; \Z)$ and $2c_1(M,S) = 0$
    in $\Homol^2(M,S)$ then $A\neq 0$.
    \footnote{The condition $2c_1(M,S) = 0$ is automatic for $n\geq 2$.
    For $n=1$ it's equivalent to $S$ being a non-contractible circle.}
\end{prop}
\begin{comment}
Indeed, consider the LES
\[
H^1(M) \to H^1(S) \to H^2(M,S) \to H^2(M)
\]
The last map sends $c_1(M,S)$ to $c_1(M)$. It's enough to show that this map is injective. For $n\geq 2$, this follows from $H^1(S) = 0$.
For $n=1$ this happens iff $H^1(M) \to H^1(S)$ is surjective. That's equivalent to $S$ being non-contractible in a surface $M\sim \Sigma_g$.
\end{comment}
\begin{proof}
Under the condition on the Chern class, everything becomes $\Z$-graded,
see \cite{seidel00}.
For $K=\Delta$, the sequence becomes
\begin{align*}
\dots \to \HF^{k}(S,S) \to \HF^k(\mathrm{id}) \xrightarrow{\Psi} \HF^k(\Delta, \Gamma_{\tau^{-1}})
\to \HF^{k+1}(S,S) \to \dots.
\end{align*}
Assume by contradiction that $A=0$. Then $\Psi=0$ and hence
we get $\Z$-graded isomorphisms
\begin{align*}
\Homol^*(S;\Lambda) \cong QH^*(S) \cong \HF^*(S,S) \cong \HF^*(\mathrm{id}) \cong \mathrm{QH}^*(M) \cong \Homol^*(M; \Lambda).
\end{align*}
This is a impossible.
We conclude that $A\neq 0$.
\end{proof}
\begin{comment}
The iso between singular and quantum only holds under the asphericity assumption in general, as rings! However it's always true as groups. And so, the proof also works for monotone mfds, as long as we have enough grading.
\end{comment}

\begin{comment}
\begin{rem}
    Under the isomorphisms $HF^0(S\times S, \Delta) \cong HF^0(S,S)
    \cong H^0(S; \mathbb{Z}/2\mathbb{Z})$,
    $B$ corresponds to the unit $\mu_S$.
    Under the isomorphisms $HF^1(\Gamma_{\tau_S^{-1}},S\times S) \cong HF^1(S,S) \cong H^1(S;\mathbb{Z}/2\mathbb{Z})$,
    $C$ corresponds to the dual of the point class $[pt^*]$.
    This follows from $B,C\neq 0$.
    For $B$, this follows from an argument in \cite[Appendix A]{MakWu}.
    For $C$???
\end{rem}
\end{comment}

\begin{comment}
For $K= Q\times N$ the Mak-Wu sequence yields the sequence: 
\begin{align}\label{seidelseq}
\dots \to \left( HF_*(Q,S) \otimes HF^*(S,N)\right)^{k} \to HF^k(Q, N) \to HF^k(Q,\tau_S(N)) \to 
\left( HF^*(Q,S) \otimes HF^*(S,N)\right)^{k+1} \to \dots
\end{align}
It seems to be expected that this sequence is isomorphic to the famous Seidel sequence (\cite{seidel03}). However, I couldn't find a proof.
\end{comment}



\begin{comment}
The baby-case $Q=N=S$, if $\dim M = 2n \geq 4$ might be instructive.
Note that $\tau_S(S)=S[1-n]$ (\cite[Lemma 5.7]{seidel00}) and hence
$HF_*(S,\tau_S(S)) = HF_*(S,S[1-n]) = HF_*(S,S)[1-n]$.
Therefore, (\ref{seidelseq}) becomes
\begin{align*}
H_{k+1-n}(S) \to H_k(S) \to \left(HF_*(S) \otimes H_*(S)\right)_{k-1} \to H_{k-n}(S) \to 
H_{k-1}(S)..
\end{align*}
The non-zero terms of this sequence occur for $k=n$, $k=0$ and $k+1=2n$:
\begin{align*}
0 \to \Lambda\mu_S \to \Lambda \mu_S \oplus \Lambda \mu_S \to \Lambda [pt] \to 0 \\
0 \to \Lambda [pt] \to \Lambda [pt] \to 0\\
0 \to \Lambda (\mu_S \otimes \mu_S) \to \Lambda \mu_S \to 0.
\end{align*}
\end{comment}

%%%%%%%%%%%%%%%%%%%%%%%%%%%%%%%%%%%%%%%%%%%%%%%%
%%%%%%%%%%%%%%%%%%%%%%%%%%%%%%%%%%%%%%%%%%%%%%%%
%%%%%%%%%%%%%%%%%%%%%%%%%%%%%%%%%%%%%%%%%%%%%%%%
\begin{comment}
%cf Juli2021, 21.6_Summary, 7.6_Summary, 8.6_Seidel_sequence
The Mak-Wu cobordism together with the cone-decomposition machinery from Biran-Cornea yields a long exact sequence of Lagrangian Floer homology groups:
\begin{align*}
\dots \to HF^k(K,S\times S) \to HF^k(K,\Delta) \xrightarrow{\mu^2(A,-)} HF^k(K,\Gamma_{\tau^{-1}}) \to HF^{k+1}(K,S\times S) \to \dots
\end{align*}
for a Lagrangian submanifold $K\subset \left(M\times M, \omega \oplus -\omega\right)$
and an element
$A \in HF^0(\Delta, \Gamma_{\tau^{-1}}) \cong HF^0(\tau^{-1})$
that is independent of $K$.

For $K=\Delta$, the sequence becomes
\begin{align*}
\dots \to HF^k(S,S) \to HF^k(\mathrm{id}) \xrightarrow{\Psi} HF^k(\tau^{-1})
\to HF^{k+1}(S,S) \to \dots
\end{align*}
We can deduce from it, that $A\neq 0$ if $(M,\omega)$ is exact:
assume by contradiction that $A=0$. Then $\Psi=0$ and hence
we get graded isomorphisms
\begin{align*}
H^*(S,\Lambda) \cong HF^*(S,S) \cong HF^*(\mathrm{id}) \cong H^*(M;\Lambda).
\end{align*}
This is a impossible for $n\geq 1$.
We conclude that $A\neq 0$.


For $K= Q\times N$ the Mak-Wu sequence yields the famous Seidel sequence (\cite{seidel03}): 
\begin{align}\label{seidelseq}
\left( HF^*(S,N) \otimes HF^*(Q,S)\right)^k \to 
HF^k(Q,N) \to HF^k(Q,\tau_S(N)) \to \left( HF^*(S,N) \otimes HF^*(Q,S)\right)^{k+1}.
\end{align}

The baby-case $Q=N=S$, if $\dim M = 2n \geq 4$ might be instructive.
Note that $\tau_S(S)=S[1-n]$ (\cite[Lemma 5.7]{seidel00}) and hence
$HF^*(S,\tau_S(S)) = HF^*(S,S[1-n]) = HF^*(S,S)[1-n]$.
Therefore, (\ref{seidelseq}) becomes
\begin{align*}
\left(HF^*(S) \otimes H^*(S)\right)^k \to H^k(S) \to 
H^{k+1-n}(S) \to \left( H^*(S) \otimes H^*(S)\right)^{k+1}.
\end{align*}
The non-zero terms of this sequence occur for $k=n$, $k=0$ and $k+1=2n$:
\begin{align*}
0 \to \Lambda[pt] \to \Lambda \mu_S \oplus \Lambda \mu_S \to \Lambda \mu_S \to 0 \\
0 \to \Lambda [pt] \to \Lambda [pt] \to 0\\
0 \to \Lambda \mu_S \to \Lambda (\mu_S \otimes \mu_S) \to 0.
\end{align*}

\end{comment}