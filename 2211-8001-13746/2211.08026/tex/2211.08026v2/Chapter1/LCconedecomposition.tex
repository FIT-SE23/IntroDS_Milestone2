We recall here how a cobordism gives rise to cone decompositions of its ends in $\mathcal{DF}uk(M)$.
% (For the moment, Theorem A and Cor 1.1.2 from Lag Cob und Fuk cat is enough.)
Since we work with cohomology, rather than homology, we write here a cohomological reformulation of Theorem A from \cite{BC2}. 
%See also \cite{MakWu}.
\begin{thm}[Theorem A in \cite{BC2}]
Let $V$ be an oriented cobordism from $L$ to the family $(L_1[l-1], L_2[l-2] \dots , L_l)$.
Assume that all Lagrangians involved (including $V$) are uniformly monotone.
Then there exists a graded quasi-isomorphism
$$L \cong \mathrm{Cone}\left( \dots \mathrm{Cone}(\mathrm{Cone}(L_1 \to L_{2} ) \to L_3)\to \dots \to L_l\right)$$
in the derived Fukaya category $\mathcal{DF}uk(M).$
\end{thm}
Here, we denote by $L[k], k\in \Z$ the Lagrangian $L$ with the same orientation for even $k$, and with oppostite orientation for odd $k$.
(The theorem also holds in the context of $\Z$-gradings, see also \cite{MakWu}.)
 
A special case occurs when there are only three Lagrangians involved, namely $V$ has one right end, $L$, and two left ends, $L_1[1]$ and $L_2$.
Then we get 
\[
L \cong Cone(L_1 \xrightarrow{\varphi} L_2).
\]
As we explain further in the \hyperref[appendix]{appendix}, the morphism $\varphi$ is determined by a unique element $\alpha_V \in HF^0(L_1,L_2)$. In particular, for any Lagrangian $K$ we get a quasi-isomorphism of chain complexes
\[
 \CF^*(K, L) \cong Cone \left( \CF^*(K,L_1) \xrightarrow{\mu^2(\alpha_V,-)} \CF^*(K,L_2) \right).
\]
Note that $\alpha_V$ is independent of $K$.

The associate long exact sequence in cohomology is
\[
\dots \to \HF^{k-1}(K,L) \to \HF^{k}(K,L_1) \xrightarrow{\mu^2(\alpha_V,-)} \HF^k(K,L_2) \to \HF^k(K,L) \to \dots
\]

\begin{comment}
Then there is an exact triangle in $DFuk(M)$ as follows:
\begin{eqnarray*}
	\begin{tikzpicture}[node distance=4cm, auto]
		\node (A) {$L_2$};
		\node (H) [below of = A] [node distance=1.7cm]{};
		\node (B) [right of=H] [node distance=3.4cm] {$L$};
		\node (C) [below of=H] [node distance=1.7cm] {$L_1$};
		%\node (D)[right of=C] {$HF_*(0_N,0_N;(H',J))$};
		\draw[->] (B) to node[above, font=\fontsize{9}{0}] {$\mathcal{F}(V)$} (A);
		\draw[->] (A) to node[left, font=\fontsize{9}{0}] {$\mathcal{F}(V'')$} (C);
		\draw[->] (C) to node[below,font=\fontsize{9}{0}] {$\mathcal{F}(V')$} (B);
		%\draw[->] (C) to node {$(j')_*^{\lambda+ \epsilon}$} (D);
	\end{tikzpicture}
	\end{eqnarray*}	
Here, $V'$ and $V''$ are the cobordisms obtained from $V$ by shuffling the ends:
$V'$ has $L_1$ as right end and $V'$ has $L_2$ as right end.
The morphism $\mathcal{F}(V)$ can be described in three equivalent ways, as explained in 
\cite[Section 4.8]{BC2}. 

\textit{Maybe recall it here. It might be nice to show that the sequence induced by the Mak-Wu cobordism actually IS the Seidel triangle.}

HOMOLOGICAL VERSION:
We recall here how cobordism give rise to cone decompositions of its ends in $DFuk(M)$.
% (For the moment, Theorem A and Cor 1.1.2 from Lag Cob und Fuk cat is enough.)
Theorem A from \cite{BC2}:
\begin{thm}[Theorem A in \cite{BC2}]
Let $V$ be a cobordism from $L$ to $L_1, \dots , L_l$.
Assume that the Lagrangians are all "admissible" (e.g. monotone or exact).
Then there exists an isomorphism
$$L \cong \mathrm{itCone}(L_l \to L_{l-1} \to \dots \to L_1)$$
in the derived Fukaya category $DFuk(M).$
\end{thm}

A special case occurs when there are only three Lagrangians involved, namely $V$ has one right end, $L$, and two left ends, $L_1$ and $L_2$.
Then we get 
\[
L \cong Cone(L_2[1] \xrightarrow{\varphi} L_1).
\]
Here, in contrast to the statement of the theorem, we include the grading. 
By the appendix, the morphism $\varphi$ is determined by a unique element $\alpha_V \in HF_{2n}(L_2[1],L_1) \cong HF_{2n-1}(L_2,L_1)$. In particular, for any Lagrangian $K$ we get a quasi-isomorphism of chain complexes
\[
 CF_k(K, L) \cong Cone \left( CF_{k+1}(K,L_2) \xrightarrow{\mu_2(-,\alpha_V)} CF_k(K,L_1) \right).
\]
Note that $\alpha_V$ is independent of $K$.

We get a long exact sequence in homology:
\[
\dots \to HF_{k+1}(K,L_2) \xrightarrow{\mu_2(-,\alpha_V)} HF_k(K,L_1) \to HF_k(K,L) \to HF_{k}(K,L_2) \to \dots
\]

Then there is an exact triangle in $DFuk(M)$ as follows:
\begin{eqnarray*}
	\begin{tikzpicture}[node distance=4cm, auto]
		\node (A) {$L_2$};
		\node (H) [below of = A] [node distance=1.7cm]{};
		\node (B) [right of=H] [node distance=3.4cm] {$L$};
		\node (C) [below of=H] [node distance=1.7cm] {$L_1$};
		%\node (D)[right of=C] {$HF_*(0_N,0_N;(H',J))$};
		\draw[->] (B) to node[above, font=\fontsize{9}{0}] {$\mathcal{F}(V)$} (A);
		\draw[->] (A) to node[left, font=\fontsize{9}{0}] {$\mathcal{F}(V'')$} (C);
		\draw[->] (C) to node[below,font=\fontsize{9}{0}] {$\mathcal{F}(V')$} (B);
		%\draw[->] (C) to node {$(j')_*^{\lambda+ \epsilon}$} (D);
	\end{tikzpicture}
	\end{eqnarray*}	
Here, $V'$ and $V''$ are the cobordisms obtained from $V$ by shuffling the ends:
$V'$ has $L_1$ as right end and $V'$ has $L_2$ as right end.
The morphism $\mathcal{F}(V)$ can be described in three equivalent ways, as explained in 
\cite[Section 4.8]{BC2}. 

\textit{Maybe recall it here. It might be nice to show that the sequence induced by the Mak-Wu cobordism actually IS the Seidel triangle.}

\end{comment}