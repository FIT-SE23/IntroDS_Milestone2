%See 6.4.20_Splitting_of_Dehn_twist_2.pdf
The assumption on the isotopy class of $\iota^*c$ is automatically satisfied for $n=1,2,3$.
As already mentioned, the assumption is equivalent to $M$ being a real fiber of a real Lefschetz fibration. This is the content of the following
\begin{propx}\label{prop:Lefschetz}
Let $M$ be a symplectic manifold, $S\subset M$ a Lagrangian sphere with parametrization $\iota$ and $c\colon M \to M$ an anti-symplectic involution.
Then the following statements are equivalent:
\begin{enumerate}
    \item[(i)]$c(S)=S$ and $\iota^*c\simeq \mathrm{id}$ or $\iota^*c\simeq r$. \item[(ii)] There exists a real Lefschetz fibration (see Sections \ref{subsec:REALmonodromy} and \ref{subsec:REALdef}) $\pi\colon E \to \mathbb{D}^2$ with real structure $c_E\colon E \to E$, real fiber $M=\pi^{-1}(1)$ and vanishing sphere $(S,\iota)$ such that
$c_E$ restricts to a real structure on $M$ that is Hamiltonian isotopic to $c$.
\end{enumerate}
\end{propx}

Seidel computed Floer cohomology of products of disjoint Dehn twists on surfaces of genus $\geq 2$ in \cite{seidel96}. 
As a special case, his result yields a $\Z$-graded isomorphism
\begin{align}\label{Iexamples:thm:pedrotti}
\HF^*(\tau_S^{-1}) \cong \mathrm{H}^*(M \backslash S; \Lambda).
\end{align}

Later, Gautschi \cite{gautschi} generalised Seidel's result to diffeomorphisms of finite type, still on surfaces.
Recently Pedrotti \cite{pedrotti} proved a $\Z_2$-graded version of (\ref{Iexamples:thm:pedrotti}) for rational, $W^+$-monotone symplectic manifolds of dimension at least $4$.
The $W^+$-condition is explained in Seidel \cite{seidel97}. It is immediate that symplectically aspherical manifolds are $W^+$-monotone.

It turns out that the automorphism $c_*$ on $HF(\tau_S^{-1})$ corresponds to the (topologically induced) map $c^*$ on singular cohomology $\mathrm{H}^*(M\backslash S; \Lambda)$. 
Namely, under the assumption that $M$ is $W^+$-monotone and that 
$c(S)=S$ the following diagram commutes:
\begin{align}\label{HFDehnTwist:diag:action}
\begin{split}
\xymatrix{
&\HF^*(\tau^{-1}_S) \ar[r]^{\cong} \ar[d]^{c_*} &\HF^*(M,S) \ar[d]^{c^*}\\
&\HF^*(\tau_S^{-1}) \ar[r]^{\cong} &\HF^*(M,S).
}
\end{split}
\end{align}
Together with Theorem \ref{I:thm:main} this allows us to deduce topological restrictions on the element $A\in \HF^*(\tau^{-1}_S)$ and sometimes enables us to compute $A$. More concrete examples are explained in section \ref{subsec:REAL2Dexamples}.

\begin{comment}
\begin{rem}
The fact that there exists a fixed point of $(c_+)_*$
is trivial because any linear involution $c$ of a vector space $V$ over a field with characteristic $2$ has a fixed point. This follows from the Jordan-Normal form.
So Theorem \ref{I:thm:main} should really be understood as a property of the element $A$.


In characteristic $\neq 2$, each involution $c \colon V \to V$
of a vectorspace $V$ is diagonalizable with eigenvalues $1$ and $-1$.

(This follows from the Jordan normal form together with $c^2=id$.
Jonny Evans used a calculation 
\begin{align*}
(c-id)^k = -2k*(c-id)
\end{align*} 
which has to vanish at some point because for $c$ in its normal form,
$c- id$ is nilpotent. This works for eigenvalues all $=1$ and can be adapted easily to the general case.)

Here, even the existence of a fixed point is non-trivial.
Generalizing the theory developed here to coefficients in $\Z$ might yield a stronger statement.
\end{rem}
\end{comment}
