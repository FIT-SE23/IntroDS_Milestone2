Any map $$c_0 \colon S^n \to S^n$$
induces an anti-symplectic involution
$$c_0^* \colon T^*S^n \to T^*S^n$$
via
$c_0^*(q,p) = (c_0(q), -p\circ (Dc_0)_{c_0(q)})$.
We assume that $c_0 = \mathrm{id}$ or $c_0=r$ (or more generally that $c_0$
is any isometry).
Note that we secretly identify $T^*S^n$ and $T_*S^n$ via the canonical isomorphism $\alpha \colon T_*S^n \to T^*S^n$ coming from the standard Riemannian metric.
The following diagram commutes:
\begin{equation*}
\xymatrix{
	&T^*S \ar[r]^{c_0^*}  &T^*S \\
	&TS \ar[r]^{-Dc_0} \ar[u]_{\alpha}^{\cong} & TS \ar[u]_{\alpha}^{\cong}
}
\end{equation*}
\begin{comment}
Indeed,
\begin{align*}
\alpha (-Dc_0)(q,v) = \alpha (c_0(q), -(Dc_0)_q(v)) = (c_0(q), -g((Dc_0)_q(v), -)) \\
c_0^* \circ \alpha (q,v) = (c_0(q,0), -g(v,(Dc_0)_{c_0(q)}(-))
\end{align*}
and
\begin{align*}
g((Dc_0)_q(v), -) &= g((Dc_0)_{c_0(q)}, (Dc_0)_q(v), (Dc_0)_{c_0(q)}(-)) \\
 &= g(v, (Dc_0)_{c_0(q)}(-))
\end{align*}
\end{comment}
The following Lemma collects some properties of $c_0^*$.
\begin{lem}\label{INVlinear:lem:properties}
Assume that $c_0 = \mathrm{id}$ or $c_0=r$ (or more generally that $c_0$
is any isometry). Then
$c_0^* \colon T^*S \to T^*S$ satisfies
\begin{itemize}
\item $c_0^*(\xi) = - \phi_s^\sigma (-c_0^*(-\phi_s^\sigma (- \xi))) $
\item $\norm{ c_0^* ( - \phi_s^\sigma (-\xi)) } = \norm{ \xi }.$
\end{itemize}
\end{lem}

\begin{proof}
Let $\xi\in T_x^*S\cong T_xS$. Let $\gamma$ be the unique geodesic in $S$ with
$\gamma(0)=x$ and $\gamma'(0)=-\xi$.
Then $\phi_s^{\sigma}(-\xi) = \gamma'(s)$.
Note that
$c_0^*(\gamma'(s)) = -dc(\gamma'(s)) =- (c\circ \gamma)'(s)$.
Moreover, 
$(c\circ \gamma)'(0) = dc(-\xi)= -c_0^*(-\xi)$, 
hence 
\begin{align*}
\phi_s^\sigma ( c_0^* \phi_s^{\sigma}(-\xi))
&= \phi_s^{\sigma}(-\phi_s^{\sigma}(- c_0^*(-\xi)))\\
&= c_0^*(-\xi).
\end{align*}
$c_0^*$ commutes with the minus sign because it is linear.
So the first claim follows.
For the second, note that both $\phi_s^{\sigma}$ and $c_0^*$ both preserve
the length induced by $g$. The latter follows from $c_0$ being an isometry.
\end{proof}

The next proposition shows that any anti-symplectic involution $c$ on $T^*S^n$
as before locally looks like $c_0^*$ for $c_0=\mathrm{id}$
or $c_0= r$.
\begin{prop}\label{INVlinear:prop:main}
    Let $c\colon T^*S^n \rightarrow T^*S^n$ be an anti-symplectic involution restricting to $\sigma\colon S^n \longrightarrow S^n$ where either $\sigma \simeq \mathrm{id}$ or $\sigma \simeq r$. Then for every $\eta_2>0$ there exists 
    $\eta_2 > \eta_1 >0$ and
    a Hamiltonian isotopy
    \[
        \psi_t^H \colon T^*S^n \longrightarrow T^*S^n
    \]
    with $\operatorname{supp}H \subset T^*_{\eta_2}S^n$ such that 
    $$c \psi^H_1 = c_0^* \; \text{ on } \; T^*_{\eta_1}S^n,$$
    where $c_0= \mathrm{id}$ or $c_0=r$.
\end{prop}
\begin{proof}
    Consider the symplectomorphism $\psi:= c \circ \sigma^*$. Write in local coordinates $\psi(q,p)=(u(q,p), v(q,p))$ with $u(q,p) \in S^n$ and
$v(q,p)\in T_{u(q,p)}S^n$. Since $c=T^*\sigma$ on $S^n$ we have
$u(q,0) = q$.
    Consider the following isotopy of symplectomorphisms $\psi_t \colon T^*S^n \to T^*S^n$ between
$\psi_0 = \mathrm{id}$ and $\psi_1(q,p)= \psi$:
\begin{align*}
\psi_{t}(q,p) =
\begin{cases}
 	(u(q,tp), \frac{v(q,tp)}{t}) \qquad \qquad \; t\neq 0\\
 	(u(q,0), (\partial_p v(q,0))p) \qquad t=0
\end{cases}
\end{align*}
$\psi_0(q,p) = (q,p)$
because
as it can be seen from writing $D\psi_{(q,0)}$ in local coordinates:
\begin{align*}
D\psi_{(q,0)} =
    \begin{pmatrix}
        &\mathrm{id} & \partial_q v(q,0)\\
        &\partial_p u(q,0) &\partial_p v(q,0)
    \end{pmatrix}
\end{align*}
and using that $D\psi_{(q,0)}$ is a symplectic matrix, we get
\[
\partial_p v(q,0) = \mathrm{id}.
\]
$\psi_t$ is a Hamiltonian isotopy:
For $n\geq 2$ this is automatic. For $n=1$ it follows from $\psi_t(S^n) = S^n$.
Concatenate $\psi_t$ with a Hamiltonian isotopy $\sigma^* \simeq c_0^*$.
Finally cut off the Hamiltonian so that the resulting Hamiltonian $H$ has support in $T^*_{\eta_2}S^n$. 
Clearly, $\psi_1^H = c \circ c_0^*$ on $T^*_{\eta_1}S^n$ for $\eta_1$
 small enough.
 In particular, $c\psi_1^H=c_0^*$ on $T^*_{\eta_1}S^n$.
\end{proof}