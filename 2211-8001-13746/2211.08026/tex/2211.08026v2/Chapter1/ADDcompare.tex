Choose Floer datum $H_s$ and $J_s$ as in Section \ref{subsec:AdddefHF}.
The generators of $\CF(\phi; H_s, J_s)$ are points
$x\in M$ such that $\phi(x)=\phi_1^H(x)$.
For the Lagrangian Floer complex, we choose the following Floer data:
\begin{align*}
K_s(x,y)= -\frac{1}{2} H_{\frac{1-s}{2}}(x) - \frac{1}{2}H_{\frac{s+1}{2}}(y).
\end{align*}
and
\begin{align*}
\tilde{J}_s := \tilde{J}_s^{(1)} \oplus \tilde{J}_s^{(2)} := J_{\frac{1-s}{2}} \oplus (-J_{\frac{s+1}{2}}).
\end{align*}
Generators of $\CF(\Delta, \Gamma_{\phi}; K_s, \tilde{J}_s)$ are of the form
$(x,\phi(x))\in \psi_1^K (\Gamma_{\mathrm{id}})$.
We show that the map
\begin{align*}
\CF(\phi; H_s, J_s) &\longrightarrow \CF(\Delta, \Gamma_{\phi}; K_s, \tilde{J}_s)\\
x &\longmapsto (x, \phi(x))
\end{align*}
is a chain isomorphism.
This follows from checking that generators get mapped to generators, and
solutions to 
\begin{equation*}
    \begin{cases}
      \frac{\partial v}{\partial t} + \tilde{J}_s(v) \left(
      \frac{\partial v}{\partial s} - X_s^K(v) \right) = 0\\
      v(0,t) \in \Delta \text{ and } v(1,t) \in \Gamma_{\phi}\\
      \lim_{t\to -\infty} v(s,t) = \psi_s^K(z) \text{ for some } z\in \Delta \\
      \lim_{t\to \infty} v(s,t) = \psi_s^K(w) \text{ for some } w\in \Delta \\
    \end{cases}
\end{equation*}
are in one to one correspondence to solutions of
\begin{equation*}
    \begin{cases}
      \frac{\partial u}{\partial t} + J_s(u) \left(
      \frac{\partial u}{\partial s} - X_s^H(u) \right) = 0\\
      u(1,t) = \phi(u(0,t)\\
      \lim_{t\to -\infty} u(s,t) = \psi_s^K(x) \\
      \lim_{t\to \infty} u(s,t) = \psi_s^K(y).  \\
    \end{cases}
\end{equation*}
The correspondence is
given by
\begin{align*}
v(s,t) = (v_1(s,t), v_2(s,t)) \longleftrightarrow u(s,t)=
\begin{cases}
v_1(1-2s), -2t) \qquad s\in [0, \frac{1}{2}]\\
v_2(2s-1,-2t) \qquad s\in [\frac{1}{2}, 1].
\end{cases}
\end{align*}

For the grading: Let $(x,x)\in \Delta \cap \Gamma_{\phi}$.
Let $\mathcal{B}^M$ be a basis of $T_xM$ and consider
the bases $\mathcal{B}^{\Delta}$ and $\mathcal{B}^{\Gamma_{\phi}}$ of 
$T_{(x,x)}\Delta$ and $T_{(x,x)}\Gamma_{\phi}$ associated to $\mathcal{B}^M$.
Note that $\mathcal{B}^{\Delta}$ and $\mathcal{B}^{\Gamma_{\phi}}$ are either both positive or both negative. Hence $\nu(x,x)=1$ if and only if the basis $\mathcal{B}=\left(\mathcal{B}^{\Delta}, \mathcal{B}^{\Gamma_{\phi}} \right)$ is a positive of $T_{(x,x)}M\times M^-$
One computes
\[
        \mathcal{B} = 
        \begin{pmatrix}
            \mathrm{Id} & \mathrm{Id} \\
            \mathrm{Id} & D\phi 
        \end{pmatrix}
        \mathcal{B}_0,
        \]
where $\mathcal{B}_0 = \left( (\mathcal{B}^M,0), (0, \mathcal{B}^M) \right)$.
$\mathcal{B}_0$ is positively oriented if and only if $n$ is even. The determinant of the matrix
is $\det (D\phi - \mathrm{Id}) = \det(\mathrm{Id}- D\phi)$. Hence $$\nu(x,x) = (-1)^n\mathrm{sign} \det (\mathrm{Id}- D\phi)$$ and
\begin{align*}
    (-1)^{\mathrm{deg}(x,x)} &= (-1)^n (-1)^{\frac{2n(2n+1)}{2}}\nu(x,x) \\
    &=(-1)^n (-1)^{\frac{2n(2n+1)}{2}}(-1)^n\mathrm{sign} \det (\mathrm{Id}- D\phi) \\
    &= (-1)^n (-1)^{\frac{2n(2n+1)}{2}}(-1)^n (-1)^{\mathrm{deg}(x)}\\
    &= (-1)^{\mathrm{deg}(x)}.
\end{align*}
This shows that the isomorphism above indeed preserves the grading.