We outline the proof of Theorem \ref{I:thm:main}.

We view the Dehn twist as a monodromy in the Lefschetz fibration
$\pi\colon E \to \mathbb{D}^2$ from Proposition \ref{prop:Lefschetz}.
Carrying a result by Salepci \cite{Salepci} over to the symplectic setting, one gets
\begin{propx}\label{R:prop}
$\tau$ is Hamiltonian isotopic to $c \circ \tilde{c}$ for some anti-symplectic involution $\tilde{c}
\colon M \to M$. In particular, $c\tau_S c$ is Hamiltonian isotopic to $\tau_S^{-1}$.
\end{propx}

Floer-theoretic considerations yield a homomorphism
    \[
        c_* \colon \HF^*(\tau_S^{-1}) \to \HF^*(\tilde{c}\tau_S \tilde{c}).
    \]
Proposition \ref{R:prop} implies that $\tilde{c}\tau_S \tilde{c} \simeq \tau_S^{-1}$
and therefore $\HF^*(\tilde{c}\tau_S \tilde{c}) \cong \HF^*(\tau_S^{-1})$.
It follows that $c$ induces an automorphism of $\HF^*(\tau_S^{-1})$,
which proves the first part of Theorem \ref{I:thm:main}.

To show that $c_*(A) = A$, we adopt the framework of Biran-Cornea \cite{BC1}, \cite{BC2}, \cite{BC3}
and Mak-Wu \cite{MakWu} about Lagrangian cobordisms.

Let $M^-$ be the symplectic manifold $(M,-\omega)$.
We denote by $\Gamma_{\phi} \subset M \times M^-$ the graph of $\phi$ for a symplectomorphism $\phi$ on $M$. This is a Lagrangian submanifold of $M\times M^-$.
For $\phi=\mathrm{id}$ it is the diagonal and we write $\Delta:= \Gamma_{\mathrm{id}}$.
In \cite{MakWu} the authors construct a Lagrangian cobordism $V_{MW} \subset M\times M^- \times \C$
that has three ends: $S\times S, \Delta$ and $\Gamma_{\tau_S^{-1}}$.
We recall the construction of $V_{MW}$ in section \ref{sec:MW}.
By general results on Lagrangian cobordisms due to Biran-Cornea this cobordism induces an exact triangle in $D\mathcal{F}uk(M\times M^-)$:

\begin{center}
\begin{tikzpicture}[node distance=1cm, auto]
\node (P) {$S\times S$};
\node (phantom)[right of =P] {};
\node(Q)[right of=phantom, below of=phantom] {$\Delta$};
\node (qhantom)[left of =Q] {};
\node (B) [left of =qhantom, below of=qhantom] {$\Gamma_{\tau_S^{-1}}$};

\draw[->](P) to node {}(Q);
\draw[->](Q) to node {}(B);
\draw[->] (B) to node {} (P);
\end{tikzpicture}
\end{center}
The associated long exact sequence is 
\begin{align}\label{I:eqLES2}
    \dots \to \HF^k(K,S\times S) \to \HF^k(K,\Delta) \to \HF^k(K,\Gamma_{\tau_S^{-1}}) \to \HF^{k+1}(K,S\times S) \to \dots,
\end{align}
where $K$ is an admissible Lagrangian submanifold in $M\times M^-$.
For the special case $K=Q\times N$, this sequence reduces to Seidel's long exact sequence (\ref{I:eqLES}).
The middle map in sequence (\ref{I:eqLES2}) can be understood as $\mu^2(A,-)$ for the element $$A\in \HF^0(\Delta, \Gamma_{\tau_S^{-1}}) \cong \HF^0(\tau_S^{-1}).$$

Consider the symplectomorphism
\begin{align*}
\Phi \colon M \times M^- \times \C &\longrightarrow M \times M^- \times \C \\
(x,y,z) &\longmapsto (c(y),c(x),z).
\end{align*}
$\Phi$ preserves the ends of the cobordisms $V_{MW}$. In particular,
$\Phi$ induces an automorphism 
$$\Phi_* \colon \HF^*(\Delta, \Gamma_{\tau^{-1}_S}) \to \HF^*(\Delta, \Gamma_{\tau^{-1}_S}).$$
This automorphism corresponds to the action of $c$ on $\HF(\tau^{-1}_S)$, namely
the following diagram commutes
\begin{align}\label{Ioutline:diag:phi_c}
\xymatrix{
&\HF^*(\tau^{-1}_S) \ar[r]^{c_*} \ar[d]^{\cong} &\HF^*(\tau^{-1}_S) \ar[d]^{\cong}\\
&\HF^*(\Delta, \Gamma_{\tau^{-1}_S}) \ar[r]^{\Phi_*} &\HF^*(\Delta, \Gamma_{\tau^{-1}_S}).
}
\end{align}
We explain these isomorphisms and the commutativity of the diagram in section \ref{sec:HF}.
A major step in the proof is the following 
\begin{thmx}\label{Ioutline_of_proof:thm:invariance}
$\Phi(V_{MW})$ is Hamiltonian isotopic to $V_{MW}$.
\end{thmx}

We show how this implies Theorem \ref{I:thm:main}.
Denote by $\bar{A}\in \HF^*(\Delta, \Gamma_{\tau^{-1}_S})$ the element corresponding
to $A\in \HF^*(\tau^{-1}_S)$ under the natural isomorphism
$\HF(\tau^{-1}_S) \cong \HF(\Delta, \Gamma_{\tau^{-1}_S})$.
As a consequence of Theorem \ref{Ioutline_of_proof:thm:invariance}, the cobordisms $V_{MW}$ and $\Phi(V_{MW})$
induce isomorphic triangles. In particular, the following diagram commutes:
\begin{align*}
\xymatrix@C+2pc{
&\HF^*(K,\Delta) \ar[r]^{\mu^2(\bar{A}, -)} \ar[d]^{\Phi_*} & \HF^*(K, \Gamma_{\tau^{-1}_S}) \ar[d]^{\Phi_*}\\
&\HF^*(K, \Delta) \ar[r]^{\mu^2(\Phi_*(\bar{A}), -)} & \HF^*(K, \Gamma_{\tau^{-1}_S})
}
\end{align*}
for all $K$. 
It follows that $\Phi_*(\bar{A})= \bar{A}$ and hence $c_*(A) =A$ by commutativity of diagram (\ref{Ioutline:diag:phi_c}).

\begin{rem}\label{Ioutline_of_proof:rmk:setting}
\begin{enumerate}
    \item 
It can be seen from the proof that all is needed are well-defined Floer cohomology groups, and applicability of Biran-Cornea's \cite{BC1},\cite{BC2} and Mak-Wu's \cite{MakWu} framework.
One could therefore easily weaken the asphericity assumption to monotonicity conditions. 
    \item The assumption that $M$ is closed is important for our arguments:
    The version of Floer cohomology we use only works for compactly supported symplectomorphisms. In general however, the monodromy in a Lefschetz fibration with non-compact fibers, if it exists, is not compactly supported. 
    We expect that the results generalize to a non-compact framework,
    when working with an appropriate version of Floer theory.
    \item In general we have a symplectic isotopy
\[
 c\circ \tau_{(S,\iota)} \circ c \simeq \tau_{(S,c\circ \iota)}^{-1}.
\]
However, it is unknown how the Dehn twist depends on the parametrization of the sphere. It is only known that if $\iota^*c$ is isotopic to an isometry, then the Dehn twist associated to $c\circ \iota$ is symplectically isotopic to the Dehn twist associated to $\iota$
\cite[Remark 3.1]{seidelthesis97}.
This explains why we make the assumption on the mapping class of 
$\iota^*c$.
    \item     
  %  Theorem \ref{I:thm:main} has an analogue interpretation in the framework of Seidel \cite{seidelbook}:
The second map in the long exact sequence (\ref{I:eqLES}) is 
\[
\mu^2(a_N,-) \colon \HF^k(Q,N) \to \HF^k(Q,\tau_S(N))
\]
for some element $a_N\in \HF^0(N,\tau_S(N))$.
$a_N$ and $A\in \HF^*(\tau_S^{-1})$ are related as follows.
There is an operation
\[
    * \colon \HF^*(\tau_S^{-1}) \otimes \HF^*(N,N) \to \HF^*(N, \tau_S(N)).
\]
If $e_N \in HF^*(N,N)$ denotes the unit, we have $A * e_N = a_N$.
The fixed point property $c_*(A) = A$ then implies 
\begin{align}\label{eq:seidel}
     \gamma(a_N) = a_{c(N)},
\end{align}
where $\gamma$ is the isomorphism
\begin{align*}
\HF^*(N,\tau_S(N)) %\xlongrightarrow^{c_*} 
\cong 
\HF^*(\tilde{c}(N), c(N)) \cong
\HF^*(c(N), \tau_S(c(N)).
\end{align*}
The construction of $a_N$ is explained in \cite[Sections 17a-17c]{seidelbook}.
$a_N$ comes from counting the number of holomorphic sections of a Lefschetz fibration with moving boundary condition coming from moving $N$ via parallel transport. The invariance property (\ref{eq:seidel}) can be proven directly in Seidel's
framework, by observing that the holomorphic sections for boundary conditions coming from $N$ and $c(N)$ are in bijection.
\end{enumerate}
\end{rem}