A Lefschetz fibration $\pi\colon E \to \mathbb{D}^2$ is called real, if the total space $E$ is endowed with an anti-symplectic involution $c_E\colon E \to E$ that covers complex conjugation $c_{\C}\colon \mathbb{D}^2 \to \mathbb{D}^2$, meaning the diagram
\begin{align}\label{REALdef:diag:conj}
\begin{split}
\xymatrix{
&E \ar[r]^{c_E} \ar[d]^{\pi} &E \ar[d]^{\pi}\\
&\mathbb{D}^2 \ar[r]^{c_{\C}} &\mathbb{D}^2
}
\end{split}
\end{align}
commutes.
Consider the fiber $M=E_1$ over $1$. $c_E$ induces an anti-symplectic involution on $c$ on $M$. The following Lemma shows that the assumption of Theorem
\ref{I:thm:main} is satisfied.
\begin{lem}\label{lem:lefschetz}
    $c(S)=S$ and $\iota^{-1}c\iota \simeq \mathrm{id}$ or $\iota^{-1}c\iota \simeq r$,
    where $\iota$ is the canonical framing of the vanishing sphere $S$.
\end{lem}
\begin{proof}
    $c(S)=S$ follows from $c_E(0)=0$ and the fact that $c_E$ commutes with parallel transport.
    For the second part, note that it is enough to consider the model
    $Q \colon \C^{n+1} \rightarrow \C$.
    In that case, $S=S^{n}\subset \C^{n+1}$ is a standard sphere.
    Note that $c_E$ restricted to the thimble
    $\Sigma= B^{n+1}(0)$ is a smooth extension of the sphere $c\vert_{S^n}$ to the ball.
    Moreover, since parallel transport commutes with $c_E$, it is a linear extension, in the sense that
    \[
        c_E(x)=c\left(\frac{x}{\norm{x}}\right)\norm{x} .
    \]
    It follows that $c_E\vert_{\R^{n+1}}$ is an orthogonal linear transformation and hence
    $c\vert_{S^n}$ is an isometry. In particular, $c\vert_{S^n}$ is smoothly isotopic to $\mathrm{id}$
    or $r$.    
\end{proof}

Proposition \ref{prop:Lefschetz} states that the condition on $\iota^*c$ is equivalent to $M$ being the fiber of a real Lefschetz fibration. One direction is proven in Lemma \ref{lem:lefschetz} above. We now prove the other direction.
\begin{proof}[Proof of Proposition \ref{prop:Lefschetz}.]
Suppose the tuple $(M,S,\iota,c)$ satisfy the conclusion of Lemma \ref{lem:lefschetz}. We want to construct a real Lefschetz fibration whose fiber is $M$, whose vanishing sphere is $(S,\iota)$ and whose real structure restricts to a real structure Hamiltonian isotopic to $c$. 

First we endow the Lefschetz fibration $\pi^0_{\epsilon} \colon E^0_{\epsilon} \to \mathbb{D}^2$ from the previous section with a real structure.
We consider two options:
\[
    c_1(z_1,\dots,z_{n+1}) = (\overline{z_1}, \dots,\overline{z_{n+1}})
\]
and 
\[
     c_2(z_1,z_2,\dots,z_{n+1}) = (-\overline{z_1},\overline{z_2},\dots, \overline{z_{n+1}}).
\]
These are real structures on $E^0_{\epsilon}$.

By Proposition \ref{INVlinear:prop:main} there exists a Hamiltonian isotopy $\psi_t$ on $M$
supported in $V$
such that in the model $T^*_{\delta}S^n$ ($\delta<\epsilon$ small enough) one has 
\[
    \psi_1 c (q,p) = (q,-p)
\]
if $\varphi^*c\simeq \mathrm{id}$
and
\[
    \psi_1 c (q,p) = (r(q),-r(p))
\]
if $\varphi^*c \simeq r$.
These two maps exactly correspond to $c_1$ and $c_2$ on the fiber $(\pi^0_{\epsilon})^{-1}(1)$. 

We now glue the fibration $\pi \colon E^0 \to \mathbb{D}^2$ from two parts: the trivial fibration 
$$\mathbb{D}^2 \times (M \backslash \varphi^{-1}(T_{\delta}^*S^n))$$ and the local model fibration $E_{\delta}^0$.
On the first part, we define
$c(z, x) := (\overline{z},\psi_1c(x))$. On $E_{\delta}^0$ we define $c(z):= c_1(z)$ or $c(z)=c_2(z)$.
These definitions are compatible on the glued region and hence descend to a real structure $c_{E^0}$ on $E^0$ satisfying $c_E\vert_{E_1}=\psi_1c$.
\end{proof}