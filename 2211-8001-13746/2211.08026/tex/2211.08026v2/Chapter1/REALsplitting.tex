Let $$\pi \colon E \to \mathbb{D}^2$$ be a real Lefschetz fibration with real structure $c_E \colon E \to E$ as above. 
We assume that $p\in E$ is the unique critical point of $\pi$ and $\pi(p)=0$.
Let $M:=E_1:=\pi^{-1}(\{1\})$ and denote
by $\tau\colon M \to M$ the monodromy along the boundary loop 
$\gamma(t) = e^{2\pi i t}, t\in [0,1].$
The following result is due to Salepci \cite{Salepci} in the smooth category.
Here we adapt it to the symplectic framework.
\begin{lem}\label{EXlem:splitting}
$\tau$ splits into a product of two anti-symplectic involutions on $M$.
More concretely, $\tau = c_+ \circ c_-$
for two anti-symplectic involutions $c_{\pm}\colon M \longrightarrow M$, where $c_+= (c_E)\vert _{E_1}$. 
Equivalently, we have $c\tau c = \tau^{-1}$.
\end{lem}
\begin{proof}
$\Omega_E$ defines a symplectic connection on the smooth part of $E$.
Let us denote by 
\[
    P_{\gamma(s);t} \colon E_{\gamma(s)} \to E_{\gamma(s+t)}
\]
the parallel transport for time $t$ along $\gamma$. 
Let $v\in E_{-1}$. Consider the parallel lift $w(t)\in E_{e^{\pi i - \pi i t}}$
of $x:=c_E(v)$ along the upper half $\eta^+$ of $\gamma$. 
%i.e. $$\pi(w(t)) =  \eta^+(t), w'(t) \in H_{w(t)}\subset T_{w(t)}E.$$
Note that 
$$c_E\circ (P_{1;\frac{1}{2}})^{-1} \circ c_E(v) =c_E(w(1)).$$
It is straight-forward to check that $v(t):= c_E(w(t))$ is actually a 
parallel lift of $v$ along the lower half $\eta^-$ of $\gamma$. This uses $(dc_E)(H_{w(t)}) = H_{c_E(w(t))}$.
Hence,
$$c_E \circ (P_{1;\frac{1}{2}})^{-1} \circ c_E = P_{-1,\frac{1}{2}}$$
and the lemma follows:
$$\tau = P_{-1;\frac{1}{2}} \circ P_{1;\frac{1}{2}} = 
c_E \circ (P_{1;\frac{1}{2}})^{-1} \circ c_E \circ P_{1;\frac{1}{2}}
= c_+ \circ c_-,$$
where $c_+ = (c_E)\vert_M$ and $c_- = (P_{1;\frac{1}{2}})^{-1} \circ c_E \circ P_{1;\frac{1}{2}}$.
\end{proof}

This proves Proposition \ref{R:prop}.
Alternatively, Proposition \ref{R:prop} can also be shown directly from the definition of a model Dehn twist without going through real Lefschetz fibrations.