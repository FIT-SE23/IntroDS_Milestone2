This is very similar to the preceeding surgery part.
Again, it is enough to show the statement for the surgery model.
Let $(S\times S) \#_{\Delta_S}^{\nu_t} \Delta$, $t\in [0,1]$ be a Hamiltonian isotopy, where
all $\nu_t$ are admissible, except for $\nu_1$ which coincides with $\nu_\epsilon^{\mathrm{Dehn}}$.
The Hamiltonian $K_t \colon T^*(S\times S) \to T^*(S\times S)$ generating the isotopy can be chosen to be of the form
$K_t(\xi_1, \xi_2) = K_t(\vert \vert \xi_1 \vert \vert,\vert \vert \xi_2 \vert \vert)$,
see \cite[Lemma 3.6]{MakWu}. Moreover, $K_t$ can be chosen to be zero near $0$ and near $1$.
The suspension cobordism is the cylindrical extension of the Lagrangian
\[
    \mathcal{S}:= \left \{ (\psi_t^K(x), t-iK_t(\psi_t^K(x))) \in M \times M^- \times \C \, \big \vert \, x\in H^{\nu_0}, t\in [0,1] \right \}.
\]
Consider the Hamiltonian isotopy $\Psi_t$ from before.
We claim that
\begin{itemize}
    \item $\Psi_1(\mathcal{S}) = \Phi(\mathcal{S})$
    \item $\Psi_t \left( {\left((S\times S) \#_{\Delta_S}^{\nu_0} \Delta \right) \times \{p\} } \right) = {\left((S\times S) \#_{\Delta_S}^{\nu_0} \Delta \right) \times \{p\} }$ for $p\in \R_{<0}$
    \item$\Psi_t \left( {\left((S\times S) \#_{\Delta_S}^{\nu_0} \Delta \right) \times \{p\} } \right) = {\left((S\times S) \#_{\Delta_S}^{\nu_1} \Delta \right) \times \{p\} }$ for $p\in \R_{>1}$
    \item $\pi_{\C} \circ \Psi_t = \pi_{\C}$
\end{itemize}
Let us check these properties.
\begin{itemize}
    \item Elements of ``the handle part" of $\mathcal{S}$ can be written as
    $$\alpha =(\xi, \psi^{\sigma}_{\nu(\vert \vert \xi \vert \vert)}(-\xi), t-iK_t(\vert \vert \xi \vert \vert))$$
    for some $\xi\in T_{\epsilon}^*S$.
    Hence elements of the corresponding part of $\Psi_1(\mathcal{S})$ are of the form
    $$\Psi_1(\alpha) = (cc_0^*(\xi), -cc_0^*(-\psi^{\sigma}_{\nu(\vert \vert \xi \vert \vert)}(-\xi), t-iK_t(\vert \vert \xi \vert \vert). $$
    Elements of the corresponding part of 
    $\Phi(\mathcal{S})$ are of the form 
    \[
    (c(-\psi^{\sigma}_{\nu(\vert \vert \xi \vert \vert)}(-\zeta), -c(\zeta), t-iK(\vert \vert \eta \vert \vert)
    \]
    for $\zeta \in T_{\epsilon}^*S$. As before,
    using Lemma \ref{INVlinear:lem:properties},
    the elements are in $1:1$-correspondence via 
    $\zeta = c_0^*(-\phi_{\nu(\vert \vert \xi \vert \vert)}(-\xi))$.
    
    \item It is very similar, but simpler to see that $\Psi_t$ preserves $H^{\nu_0} \times \{t\}$ for $t\in \R_{<0}$, and also $H^{\nu_1} \times \{t\}$ for $t\in \R_{>1}$.
    
    \item The last item is obvious.
    
\end{itemize}
Therefore, the suspension part $\mathcal{S}$ of the cobordism is Hamiltonian isotopic to 
$\Phi(\mathcal{S})$.

The symmetry of the surgery part shown in section \ref{subsec:INVinvariance_Mak_Wu} and the symmetry of the suspension part shown above together prove Theorem \ref{Ioutline_of_proof:thm:invariance}.