We adopt here the definition used in \cite{BC3}. We denote by $\mathbb{D}^2$ the closed unit disc viewed as a subset of $\C$. A Lefschetz fibration with base $\mathbb{D}^2$ consists of 
\begin{enumerate}
    \item a closed symplectic manifold $(E,\Omega_E)$ endowed with an almost complex structure $J_E$,
    \item a proper $(J_E,i)$-holomorphic map $\pi \colon E \to \mathbb{D}^2$
\end{enumerate}
such that 
\begin{enumerate}
    \item $\pi$ has only finitely many critical points with distinct critical values,
    \item all the critical points of $\pi$ are ordinary double points, that is for every critical point $p\in E$, there exists $J_E$-holomorphic coordinates around $p$ such that in these coordinates $\pi(z_1, \dots, z_n) = z_1^2 + \dots + z_n^2$ holds.
\end{enumerate}
For $p\in \mathbb{D}^2$ we denote by $E_p:= \pi^{-1}(\{p\})$ the fiber above $p$.
All regular fibers of $\pi$ are symplectic manifolds with symplectic form induced from $\Omega_E$.

Given a symplectic manifold $(M,\omega)$ and a Lagrangian sphere $S$, one can construct a Lefschetz fibration with smooth fiber $M$ such that the Dehn twist is symplectically isotopic to the monodromy around a critical point.
We refer the reader to \cite[Section 1]{seidel03} and \cite[Section (16e)]{seidelbook} for a detailed explanation.
We only include a very brief outline of the construction here.
Consider the following local model for $\epsilon > 0$: Let $Q \colon \C^{n+1} \to \C, Q(z_1, \dots, z_{n+1}) = z_1^2 + \dots + z_{n+1}^2$ and define the total space of the fibration to be
\[
    E^0_{\epsilon} := \left \{ z\in \C^{n+1} \bigg \vert \vert Q(z) \vert \leq 1,
    \frac{\vert z \vert ^4 - \vert Q(z) \vert ^2}{4} < \epsilon \right \}.
\]
The fibration then is $\pi_{\epsilon}^0 \colon E^0_{\epsilon} \to \mathbb{D}^2, \pi(z) = Q(z).$
The smooth fibers are symplectomorphic to $T^*_{\epsilon}S^n$.
Consider the family of Lagrangian spheres
\[
    \Sigma_r = \sqrt{r}S^{n} = \{ (\sqrt{r}z_1, \dots , \sqrt{r}z_{n+1}) \big \vert z \in S^n \subset \R^{n+1}\}  \subset \left(E_{\epsilon}^0\right)_{r}
\]
for $r>0$. They are called vanishing cycles.
The union $\Sigma = \left(\cup_{r>0} \Sigma_r\right) \cup \{0\}$ is a Lagrangian disc in $E^0_{\epsilon}$, called a Lefschetz thimble.
There is an isomorphism 
\[
    \Phi \colon E^0_{\epsilon}\backslash \Sigma \to \mathbb{D}^2 \times
    (T_{\epsilon}^*S^n \backslash S^n).
\]
The monodromy $\tau \colon (\pi_{\epsilon}^0)^{-1}({1}) \to (\pi_{\epsilon}^0)^{-1}({1})$ along $\partial \mathbb{D}^2$ is the Dehn twist along the vanishing cycle $\Sigma_1$ \cite[Lemma 1.10]{seidel03}.
To get the claimed Lefschetz fibration $\pi^0 \colon E^0 \to \mathbb{D}^2$, one glues $E^0_{\epsilon}$ together with the trivial fibration $\mathbb{D}^2 \times (M \backslash S)$ using $\Phi$ by identifying a tubular neighbourhood of $S\subset M$ with $T^*_{\epsilon}S^n$ for small enough $\epsilon$.

Locally, each Lefschetz fibration looks like a model Lefschetz fibration $E^0$.
In particular, there is a notion of vanishing spheres in any Lefschetz fibration.
The monodromy $\tau \colon E_p \to E_p$ along a path around the singularity is the Dehn twist along a vanishing cycle in $E_p$:
Usually, the monodromy in not supported near $S$. However, $\tau$ is symplectically isotopic to the Dehn twist as defined
in section \ref{subsec:REALDehntwist}.
