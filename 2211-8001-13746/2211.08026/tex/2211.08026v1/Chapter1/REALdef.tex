A Lefschetz fibration $\pi\colon E \to \mathbb{D}^2$ is called real, if the total space $E$ is endowed with an anti-symplectic involution $c_E\colon E \to E$ that covers complex conjugation $c_{\C}\colon \mathbb{D}^2 \to \mathbb{D}^2$, meaning the diagram
\begin{align}\label{REALdef:diag:conj}
\begin{split}
\xymatrix{
&E \ar[r]^{c_E} \ar[d]^{\pi} &E \ar[d]^{\pi}\\
&\mathbb{D}^2 \ar[r]^{c_{\C}} &\mathbb{D}^2.
}
\end{split}
\end{align}
commutes.
Suppose now that $M$ admits an anti-symplectic involution $c\colon M \to M$ such that $c(S)=S$. We show that we can endow the Lefschetz fibration $\pi^0\colon E^0 \to \mathbb{D}^2$ from the previous section with a real structure $c_E$ such that $(c_E)\vert_{E_1}= c$.

The fibration $\pi \colon E^0\to \mathbb{D}^2$ is glued from two parts: the trivial fibration 
$\mathbb{D}^2 \times (M \backslash S)$ and the local model fibration $E_{\epsilon}^0$.
On the first part, we simply define
$c_1(z, x) := (\overline{z},c(x))$. On $E_{\epsilon}^0$ we use the following explicit trivialization \cite[section 1.2]{seidel03}
\begin{align*}
    \Phi \colon E_{\epsilon}^0 \backslash \Sigma &\to \mathbb{D}^2 \times (T^*_{\epsilon}S^n \backslash S^n),\\
    y & \mapsto \left(Q(x), \sigma_{\frac{\alpha}{2}}\left(\frac{\mathrm{Re}(\hat{x})}{\vert \vert \mathrm{Re}(\hat{x})
    \vert \vert}, - \mathrm{im} (\hat{x}) \vert \vert \mathrm{re}(\hat{x}) \vert \vert\right) \right),
\end{align*}
where $Q(x) = e^{i\alpha}$ and $\hat{x} = e^{-i\frac{\alpha}{2}}x$.
On $E_{\epsilon}^0 \backslash \Sigma$ we define $c_2(\Phi^{-1}(z,x)):= \Phi^{-1}(\overline{z}, c(x))$.
$c_2$ extends smoothly to $E_{\epsilon}^0$ and endows $\pi^0_{\epsilon}$ with a real structure. $c_1$ and $c_2$ are compatible on the glued region and hence descend to a real structure $c_E$ on $E$ satisfying $c_E\vert_{E_1}=c$.