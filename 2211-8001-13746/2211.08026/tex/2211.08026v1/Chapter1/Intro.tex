Let $(M,\omega)$ be a closed symplectic manifold and $S\subset M$ a Lagrangian sphere.
Associated to $S$ there exists a distinguished symplectic isotopy class represented by the \textit{Dehn twist}. The Dehn twist $\tau_S$ is a symplectomorphism compactly supported in a neighbourhood of $S$.
Seidel proved that the square of the Dehn twist, in some cases, is not symplectically, but only smoothly isotopic to the identity \cite{seidelthesis97}, \cite{seidel_lectures}.
To prove this result Seidel established a Floer homology exact sequence
\begin{align}\label{I:eqLES}
    \dots \to (\HF^*(S,N) \otimes \HF^*(Q,S))^{k} \to \HF^{k}(Q,N) \to 
    \HF^k(Q,\tau_s(N)) \to \dots
\end{align}
for admissible Lagrangian submanifolds $Q$ and $N$ in $M$
  \cite{seidel03},\cite{seidelbook}.
There is a distinguished element $A\in \HF^*(\tau^{-1}_S)$ that characterizes the map $\HF^k(Q,N) \to HF^k(Q,\tau_S(N))$ that occurs in the sequence. 
\begin{comment}
By work of Mak-Wu in \cite{MakWu} the sequence (\ref{I:eqLES}) is a special case of a long exact sequence of Floer cohomology groups of Lagrangian submanifolds in the product manifold $M\times M^- := \left( M\times M, \omega \oplus -\omega\right)$:
\begin{align}\label{I:eqLES2}
    \dots \to \HF^k(K,S\times S) \to \HF^k(K,\Delta) \to \HF^k(K,\Gamma_{\tau_S^{-1}}) \to \HF^{k+1}(K,S\times S) \to \dots,
\end{align}
where $K$ is an admissible Lagrangian submanifold in $M\times M^-$.
For the special case $K=Q\times N$, this sequence reduces to the long exact sequence (\ref{I:eqLES}).
The middle map in the sequence (\ref{I:eqLES2}) can be understood as $\mu^2(A,-)$ for a certain element $$A\in \HF^0(\Delta, \Gamma_{\tau_S^{-1}}) \cong \HF^0(\tau_S^{-1}).$$
\end{comment}

Due to the relevance of the above exact sequence it is thus natural to investigate properties of the element $A$.
The goal of this paper is to study the element $A$ in the situation,
where there exists an anti-symplectic involution that preserves $S$.

We work in the following setting. $(M, \omega)$ is a closed
symplectically aspherical symplectic manifold. Unless otherwise explicitely stated, all involved Lagrangian submanifolds are assumed to be closed, oriented and relatively symplectically aspherical. 
Floer cohomology groups are $\mathbb{Z}_2$-graded with coefficients in 
the universal Novikov field over $\mathbb{Z}_2$. More details about these assumptions are given in section \ref{subsec:HFsetting}.

Let $c\colon M \to M$ be an anti-symplectic involution satisfying $c(S)=S$.
We only assume that $S$ is invariant under $c$, but $S$ does not have to be
pointwise fixed by $c$.
Our main result is

\begin{thmx}\label{I:thm:main}
    $c$ induces an automorphism 
    $c_* \colon \HF^*(\tau_S^{-1}) \to \HF^*(\tau_S^{-1})$
    and $c_*(A) = A$.
\end{thmx}

\begin{rem}
$c_* \colon \HF^*(\tau_S^{-1}) \to \HF^*(\tau_S^{-1})$ is an involution
of a vector space over a field with characteristic $2$. Any such map
has a fixed point because $(c_*-\mathrm{id})^2=0$, hence $\mathrm{ker}(c_*-\mathrm{id})\neq 0$. The relevance of the second part of Theorem \ref{I:thm:main} is therefore not merely the existence of a fixed point. It should rather be understood as a special property of the element $A$.
\end{rem}

Along the proof we show
\begin{propx}\label{I:prop:main}
  $\tau_S$ is Hamiltonian isotopic to a symplectomorphism $\tau$ such that $c\tau$ is an anti-symplectic involution.
\end{propx}

\begin{comment}
$S$ $\tau_S$ splits into a composition of two anti-symplectic involutions $c_+$ and $c_-$ that preserve the sphere.
More precisely, the situation we consider is the following:
\begin{equation}\label{Intro:Assumption}\tag{\ddag}
\begin{split}
    &\text{There exist anti-symplectic involutions } c_{\pm} \colon M \to M,\\
    &\text{such that } \tau_S \text{ is symplectically isotopic to } c_+ \circ c_- \\
    &\text{and } c_+(S) = c_-(S)=S.
\end{split}
\end{equation}
This situation naturally occurs when $(M,\omega)$ is the fiber of a real Lefschetz fibration with vanishing sphere $S$. The argument is based on \cite{Salepci}, where Salepci studied topological real Lefschetz fibrations. We explain this further in sections \ref{subsec:Iexamples} and \ref{sec:REALlefschetz}.

We work in the following setting. $(M, \omega)$ is a closed
symplectically aspherical symplectic manifold. Unless otherwise explicitely stated, all involved Lagrangian submanifolds are assumed to be closed and relatively symplectically aspherical. In particular, $S\subset M$ is relatively symplectically aspherical. 
\footnote{
This is automatic if $M$ is symplectically aspherical, unless $S$ has dimension $1$. In the latter case, the condition is equivalent to $S$ being a non-contractible circle.
}

\footnote{This is automatically true for $\dim M \geq 4$. For $\dim M = 2$
the condition is equivalent to $S$ being a non-contractible circle.}
While this is by far not the most general setting, we chose it for sake of readability of the paper.
We comment in Remark \ref{Ioutline_of_proof:rmk:setting} on possible weaker assumptions.
Our Floer cohomology groups will be defined over the universal Novikov field $\Lambda$ with coefficients in $\Z_2$. They are always $\Z_2$-graded. If they admit a $\Z$-grading, the results are still valid in a graded version.

In situation (\ref{Intro:Assumption}) $c_+$ induces an automorphism $(c_+)_* \colon 
\HF^*(\tau^{-1}_S) \to \HF^*(\tau^{-1}_S)$.
Our main result is
\begin{thmx}\label{I:thm:main}
    Under the assumption (\ref{Intro:Assumption}) the element $A\in \HF^*(\tau^{-1}_S)$ is a fixed point of $(c_+)_*$.
\end{thmx}

\end{comment}