%Graph of Dehn twist as E_2-flow surgery
Following the principle that surgeries provide cobordisms with three ends \cite[Section 6]{BC1}, the Mak-Wu cobordism also arises as the trace of a surgery. The first step therefore is to understand $\Gamma_{\tau_S^{-1}}$ as the result of a surgery between 
$S \times S \subset M\times M^-$ and the diagonal $\Delta \subset M\times M^-$
along the clean intersection $\Delta_S := (S\times S)\cap \Delta$.
The surgery construction takes place locally in a Weinstein neighbourhood of $S\times S$. We choose a very specific neighbourhood, so that we can later compare it to $\Gamma_{\tau_S^{-1}}$.
Namely, consider the symplectic embedding
\begin{align*}
\widetilde{\varphi} \colon V \times V &\longrightarrow T_\epsilon^*S^n \oplus T_\epsilon^*S^n \subset T^*(S^n \times S^n) \\ 
(x,y) &\longmapsto (\varphi(x), -\varphi(y))
\end{align*}
that identifies $S\times S$ with the zero-section in $T^*(S^n \times S^n)$. Note that 
$$\widetilde{\varphi}^{-1}(N_{\Delta_S}^*) = \Delta \cap (V\times V),$$
where 
\[
    N_{\Delta_S}^*:= \left \{ \alpha \in T^*(S^n \times S^n) \vert 
    \forall v\in \Delta_S \colon \alpha(v) = 0 \right \}.
\]
%Admissible functions
We will define a surgery model in $T^*(S^n \times S^n)$ for surgery of the zero-section and $N_{\Delta_S}^*$ along their intersection $\Delta_S$.
Then we will glue the surgery model into $V\times V$ via $\tilde{\varphi}$. 
To define the surgery model, we need some auxiliary functions:
\begin{defi}\label{MWGraphDTadm}
A \textit{$\lambda$-admissible} function $\nu_{\lambda}\colon \mathbb{R}_{\geq 0}
\longrightarrow [0, \lambda]$ is a smooth function satisfying
\begin{align*}
\begin{cases}
	 \nu_{\lambda}(0) = \lambda,\\
	 \nu_{\lambda}^{-1} \text{has vanishing derivatives of all orders at } \lambda,\\
	 0 < \nu_{\lambda}(r) < \lambda \text{ and strictly decreasing}
	 									 &\text{for } 0< r < \epsilon,\\
	 \nu_{\lambda}(r)=0                   &\text{for } r\geq \epsilon.
\end{cases}
\end{align*}
\end{defi}
\begin{comment}
\begin{center}
\includegraphics[scale=0.2]{Pictures/MWGraphDT_pic1.pdf}
\end{center}
\end{comment}

Let $\pi_2 \colon T^*(S^n \times S^n) \cong T^*S^n \oplus T^*S^n \to T^*S^n$ be the projection
to the second summand. Consider 
$\sigma_\pi \colon T^*(S^n \times S^n) \to \R$ defined by
$\sigma_\pi(\xi) = \vert \vert \pi_2(\xi)\vert \vert$.
This has a well-defined Hamiltonian flow on $T^*(S^n \times S^n)\backslash \Delta_S$.
Let $\lambda < \pi$. Consider a $\lambda$-admissible function $\nu = \nu_{\lambda}$,
and define the following \textit{flow handle}:
\begin{align*}
H_{\nu} = \left\{ \psi_{\nu(\sigma_\pi(\xi))}^{\sigma_\pi}(\xi) \in T^*(S^n \times S^n) \, \big\vert \,
\xi \in N_{\Delta_S}^* \backslash \Delta_S, \sigma_\pi(\xi) \leq \epsilon\right\}.
\end{align*}
$H_\nu$ can be glued to a part of $S\times S$ and $N^*_{\Delta_S}$, resulting in a smooth Lagrangian in $T^*(S^n \times S^n)$ that coincides with $N_{\Delta_S}^*$ outside of $T_\epsilon^* S \oplus T_\epsilon^*S^n$. 
We denote the resulting Lagrangian by 
\[
(S^n \times S^n) \#_{\Delta_S}^{\nu} N_{\Delta S}^*.
\]
We finally glue this model surgery into $V\times V$:
$$(S\times S) \#_{\Delta_S}^{\nu} \Delta := (\tilde{\varphi})^{-1}
\left((S^n \times S^n) \#_{\Delta_S}^{\nu} N_{\Delta_S}^*\right)
\cup \left(\Delta \backslash (V\times V^-)\right).$$

Mak-Wu \cite[Lemma 3.4]{MakWu} show that all such surgeries are Hamiltonian isotopic for different choices of $\nu$. Moreover, the same construction works for $\nu= \nu_\epsilon^{\text{Dehn}}$ (even though this is \textit{not} admissible)
and the result is again Hamiltonian isotopic to any of the other surgeries.
It's straight-forward to see that
$$(S\times S) \#_{\Delta_S}^{\nu_\epsilon^{\text{Dehn}}} \Delta = \Gamma_{\tau_S^{-1}}$$
and so any of the above surgeries is Hamiltonian isotopic to $\Gamma_{\tau_S^{-1}}$.
In particular, since $\Gamma_{\tau_S^{-1}}$ is relatively symplectically aspherical, so is the surgery $(S\times S)\#_{\Delta_S}^{\nu} \Delta$.


\begin{rem}
This version of surgery is a special case of $E_2$-flow surgery, introduced in \cite[section 2.3]{MakWu} in more general situations.
\end{rem}