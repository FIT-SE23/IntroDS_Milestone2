For convenience of the reader we briefly collect the basic ideas and notation for Floer cohomology of a symplectomorphism following \cite{DostSal}. For more detailed expositions, we refer the reader to 
\cite{DostSal} for the monotone case, and to \cite{seidel97}  and \cite{lee} for $W^+$-symplectic manifolds. 
%\cite{tonkonog}

Let $(M,\omega)$ be a closed symplectically aspherical symplectic manifold.
Let $f\in \Symp(M)$ be a symplectomorphism. We first need to choose a Hamiltonian perturbation, namely a family of Hamiltonian functions $\{H_s\colon X \to \mathbb{R}\}_{s \in \mathbb{R}}$.
It should be $f$-periodic, in the sense that
$$H_s=H_{s+1} \circ f.$$
Roughly speaking, Floer cohomology of $f$ is Morse cohomology on the twisted loop space
\[
    \Omega_{f} := \{ x \in C^{\infty}(\R, X) \vert x(s+1) = f(x(s))\}
\]
with the closed $1$-form
\[
    \lambda_H(x)(\xi) = \int_0^1 \omega\left(\dot{x}(s)-X^H_s(x(s)), \xi(s) \right) \, ds.
\]
Here, $X^H_s$ denotes the Hamiltonian vector field of $H_s$.
We write $P_{f}(H)$ for the set of $x\in \Omega_{f}$ satisfying $\dot{x}(s) = X_s^H(x(s))$.
For a generic choice of $H$, $P_{f}(H)$ is a finite set. 
%(i.e. $\ker (\mathrm{Id}-df_H(x))=0$ for any fixed point $x\in M$.
%Consider $f_H := (\psi_1^H)^{-1} \circ f \in Symp(M)$.
The vector space underlying the Floer complex is the $\Lambda$-vector space
generated by $P_{f}(H)$:
\begin{align*}
\CF^*(f;H) = \bigoplus_{x \in P_{f}(H)} \Lambda x.
\end{align*}


\begin{comment}
(In \textit{\cite{tonkonog}: $\mathbb{F}=\mathbb{C}$ according to \cite{tonkonog}.
In \cite{DostSal}, under the monotonicity assumption, they use $\mathbb{Z}$ instead of $\Lambda$! In \cite{seidel97}, he uses $\mathbb{Z}_2$ instead of $\Lambda$.)
Here, $\mathbb{F}= \mathbb{Z}_2$}.
\end{comment}


To define the differential, we need to choose a family of almost complex structures $\mathcal{J}= \{J_s\}_{s\in \R}$ on $M$, compatible with $\omega$ and $f$-periodic, meaning
$J_s= {f}^*(J_{s+1})$.
One considers finite-energy solutions
$u\colon \R \times \R \to X, (s,t) \mapsto u(s,t)$
of Floer's equation
$$\frac{\partial u}{\partial t} + J_s(u) \left( \frac{\partial u}{\partial s} - X^H_s(u) \right) = 0,$$
which are $f$-periodic in $s$, $u(s+1,t) = f(u(s,t))$,
and satisfy the asymptotic conditions
$$\lim_{t\to -\infty} u(s,t) = x(s)
\text{  and  }
\lim_{t \to \infty} u(s,t) = y(s)
$$
for some Hamiltonian chords $x,y$.
Consider the moduli space $\mathcal{M}(x,y;\mathcal{J},H)$ of all such solutions $u$.
For regular $(\mathcal{J},H)$, the moduli space is a smooth manifold.
$\mathbb{R}$ acts on the one-dimensional component $\mathcal{M}^1(x,y;\mathcal{J},H)$ by translation, and the quotient set $\hat{\mathcal{M}}^1(x,y;\mathcal{J},H) = \mathcal{M}^1(x,y;\mathcal{J},H)/ \mathbb{R}$ is discrete.

The Floer differential $\partial \colon \CF^*(f; \mathcal{J},H) \to \CF^*(f; \mathcal{J},H)$
is defined by
\begin{align*}
\partial(x) = \sum\limits_{y \in P_{\varphi}(H)} \sum\limits_{u\in \hat{\mathcal{M}}^1(x,y;\mathcal{J},H)} y.
\end{align*}
$\CF^*(f)$ is $\Z/2$-graded as follows. A generator $x\in P_{f}(H)$ corresponds to a fixed point $x(0)$ of ${f}_H:= (\Psi_1^H)^{-1}f$. The degree 
$\mathrm{deg}(x) \in \Z/2$ of $x$
is related to the index of $x(0)$ by
\[
    (-1)^{\mathrm{deg}(x)} =
      \mathrm{sign} \left( \det ( \mathrm{id} - (Df_H)_{x(0)} ) \right).
\]
There are also graded continuation maps for different choices of Floer datum:
Suppose $(H, \mathcal{J})$ and $(H', \mathcal{J}')$ are regular Floer data as above. Choose a family
$(H_{s,t}, J_{s,t})$ 
that satisfies the periodicity assumptions
\[
    J_s = f^*(J_{s+1}) \text{ and } J_s' = f^*(J_{s+1}')
\]
and interpolate between $(H_s, J_s)$ and $(H_s', J_s')$, i.e.
\begin{align*}
    H_{s,t} = H_s', J_{s,t} = J_t' \qquad &\text{for $t$ near $-\infty$},\\
    H_{s,t} = H_s, J_{s,t} = J_t  \qquad &\text{for $t$ near $\infty$}.
\end{align*}
We denote by $\mathcal{M}(x,y; J_{s,t}, H_{s,t})$ the moduli space of solutions to the $1$-parametric Floer equation
\[
\frac{\partial u}{\partial t} + J_{s,t}(u) \left( \frac{\partial u}{\partial s} - X^H_{s,t}(u) \right) = 0
\]
that are $f$-periodic in $s$ and tend to $x$ and $y$ as $t\to \pm \infty$.
For generic choice of $(H_{s,t}, J_{s,t})$ the moduli space is a manifold
and its zero-dimensional component $\mathcal{M}^0(x,y; J_{s,t}, H_{s,t})$
is discrete.
The chain-level continuation map is the chain map
\begin{align*}
    C_{H_{s,t},J_{s,t}} \colon \CF(f; \mathcal{J}, H) &\longrightarrow \CF(f,\mathcal{J}',H)\\
    x &\longmapsto \sum\limits_{y \in P_{\varphi}(H)} \sum\limits_{u\in \mathcal{M}^0(x,y;J_{s,t},H_{s,t})} y.
\end{align*}
The map induced in cohomology is independent of the choice of homotopy $(H_{s,t}, J_{s,t})$. This allows us to identify the cohomology groups $\HF(f,\mathcal{J}, H)$ and $\HF(f,\mathcal{J}', H')$ and simply write
$\HF(f)$ for the cohomology group.

Here is a proof for the result on conjugation by an anti-symplectic involution.
\begin{proof}[Proof of Proposition \ref{HFconjugation_invariance:thm:main}]
Let $(\mathcal{J},H)$ be a Floer datum for $f^{-1}$.
Then $(\mathcal{J}', K)$, defined by
\[
    K_s := H_{1-s} \circ \varphi^{-1}
\]
and
\[
    J_s' := -(\varphi^{-1})^*J_{1-s}
\]
is an admissible Floer datum for $\varphi f \varphi ^{-1}$.
A straight-forward calculation shows that
\begin{align*}\label{HFconjugation_invariance:eq}
    \CF^*(f^{-1}; \mathcal{J},H) &\longrightarrow 
    \CF^*(\varphi f \varphi ^{-1}; \mathcal{J}', K) \\
   P_{f^{-1}}(H) \, \ni 
   \, x &\longmapsto \varphi (f^{-1}(x))
\end{align*}
defines a $\Lambda$-linear isomorphism of $\Z_2$-graded cochain complexes.
Moreover, the degree in preserved.
\begin{comment}
Moreover, the degree is preserved:
Let $y\in P_{f^{-1}}(H)$. Then
\begin{align*}
    \det(\mathrm{Id} - D((\psi_1^K)^{-1}\varphi f \varphi^{-1})_{\varphi(f^{-1}(y(0)))})
    &= \det(\mathrm{Id} - D(\varphi \psi_1^H f \varphi^{-1})_{\varphi(f^{-1}(y(0)))})\\
    &=
    \det(\mathrm{Id} - D(\psi_1^H f )_{f^{-1}(y(0))})\\
    &= \det(\mathrm{Id} - D(f\psi_1^H )_{y(0)})\\
    &= \det(\mathrm{Id} - D(f\psi_1^H )_{y(0)}^{-1})\\
    &= \det(\mathrm{Id} - D((\psi_1^H)^{-1}f^{-1})_{y(0)}).
\end{align*}
In the first equality, we used
$(\psi^K_1)^{-1} = \varphi \psi_1^H \varphi^{-1}$,
the second follows from factoring out $D\varphi$ and $D\varphi^{-1}$
and using multiplicativity of $\det$, the third follows from multiplying with $Df$ and $Df^{-1}$ from left resp right,
the fourth used that $D(f\psi_1^H)$ is symplectic.

(For a symplectic matrix $A$ one has
\begin{align*}
 \det(\mathrm{Id} - A^{-1}) &= \det (\mathrm{Id} - J_0^{-1}A^T J_0)\\
 &= \det (\mathrm{Id} - A^T)\\
 &= \det (\mathrm{Id}-A).)
\end{align*}

This clearly implies $\mathrm{deg}(y) = \mathrm{deg}(\varphi(f^{-1}(y))$, which means that the isomorphism above preserves the degree.
\end{comment}
Concatenation of this chain-level isomorphism with a continuation map shows
Proposition \ref{HFconjugation_invariance:thm:main}.
\end{proof}