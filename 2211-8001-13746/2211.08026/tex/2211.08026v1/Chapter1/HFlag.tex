Note that for any symplectomorphism $f$ on a symplectically aspherical symplectic manifold $M$, the graph $\Gamma_{f}$ is a relatively aspherical Lagrangian manifold in $M\times M^{-}$. Also, products of relatively aspherical Lagrangians in $M$ are relatively aspherical Lagrangians in $M\times M^-$.

We endow the graph $\Gamma_{f}$ with the following orientation:
Given a positive basis $v_1, \dots , v_{2n}$ of $T_xM$, then the basis
$(v_1,Df_x(v_1)), \dots , (v_{2n}, Df_x(v_{2n}))$ of $T_x\Delta \subset T_xM \oplus T_xM$ is defined to be positive
if $(-1)^{\frac{n(n-1)}{2}}=1$ and negative otherwise, see \cite{WW}.
Moreover, given an oriented Lagrangian $N$, note that $f(N)$ has a canonical orientation.

Let $Q$ and $N$ be oriented Lagrangians in $M$. There are the following canonical graded isomorphisms between
Floer cohomology groups for Lagrangians in $M\times M^{-}$ and Lagrangians in $M$:
\begin{enumerate}
    %\item $HF^*(Q\times N, \Delta) \cong HF^*(Q,N)$
    \item $\HF^*(Q\times N, \Gamma_{f^{-1}}) \cong \HF^*(Q, f(N))$
    \item $\HF^*(Q\times N, Q'\times N') \cong \HF^*(Q,Q') \otimes \HF^*(N',N)$
\end{enumerate}
\begin{comment}
\begin{proof}
    \begin{enumerate}
        \item We assume that $Q$ intersects $f(N)$ transversely.
        Let $x\in Q \cap f(N)$.
        Let $\mathcal{B}^Q$ and $\mathcal{B}^N$ be positive bases for $T_xQ$ and $T_{x}N$.
        Then $\nu(x) = 1$ if and only if $\mathcal{B}^M := \left(\mathcal{B}^Q, Df(\mathcal{B}^N) \right)$ is a positive basis for $T_xM$.
        Consider the bases 
        $\mathcal{B}^{Q\times N^-}:= \left( (\mathcal{B}^Q, 0),
        (0,\mathcal{B}^N) \right)$ for $Q\times N^-$
        and $\mathcal{B}^{\Gamma_{f^{-1}}}$ associated to $\mathcal{B}^M$.
        $\mathcal{B}^{Q\times N^-}$ is always positive. $\mathcal{B}^{\Gamma_{f^{-1}}}$ is positive if and only if $(-1)^{\frac{n(n-1)}{2}}\nu(x) = 1$.
        We have to understand $\mathcal{B} = \left(\mathcal{B}^{Q\times N^-},
        \mathcal{B}^{\Gamma_{f^{-1}}} \right).$
        We have
        \[
        \mathcal{B} = 
        \begin{pmatrix}
            \mathrm{Id} & 0 &0 &0 \\
            0 & 0 & 0 & Df \vert_N^{-1}\\
            \mathrm{Id} & 0 & Df \vert_Q^{-1} & 0\\
            0 & \mathrm{Id} & 0 &  Df \vert_N^{-1}
        \end{pmatrix}
        \mathcal{B}_0,
        \]
        where $\mathcal{B}_0= \left((\mathcal{B}^M,0), (0, \mathcal{B}^M \right)$.
        This basis is positively oriented if and only if $n$ is even.
        The determinant of the matrix is $(-1)^n$.
        It therefore follows that $\mathcal{B}$ is always positively oriented.
        We deduce that
        $\nu(x,f^{-1}(x))=(-1)^{\frac{n(n-1)}{2}}\nu(x)$.
        It then follows
        \begin{align*}
            (-1)^{\mathrm{deg}(x,f^{-1}(x))} &= (-1)^{\frac{2n(2n+1)}{2}}\nu(x,f^{-1}(x))\\
            &=(-1)^{\frac{2n(2n+1)}{2}}(-1)^{\frac{n(n-1)}{2}}\nu(x)\\
            &= (-1)^{\frac{2n(2n+1)}{2}}(-1)^{\frac{n(n-1)}{2}}
            (-1)^{\frac{n(n+1)}{2}}(-1)^n \mathrm{deg}(x)\\
            &= (-1)^{\mathrm{deg}(x)}.
        \end{align*}
        \item We assume that $Q$ resp. $N$ intersects $Q'$ resp. $N'$ transversely and consider $(x,y) \in \left(Q \cap Q' \right) \times
        \left( N' \cap N \right)$. We abbreviate
        \begin{align*}
        \nu(x,y)&:= \nu((x,y); Q\times N^-, Q'\times N'^-), \\
        \nu(x) &:= \nu(x; Q,Q'), \\
        \nu(y)&:= \nu(y;N',N).
        \end{align*}
        Similarly for 
        \begin{align*}
             \mathrm{deg}(x,y)= (-1)^{\frac{2n(2n+1)}{2}}\nu(x,y), \\
        \mathrm{deg}(x) = (-1)^{\frac{n(n+1)}{2}}\nu(x),\\ \mathrm{deg}(y)=(-1)^{\frac{n(n+1)}{2}}
        \nu(y).
        \end{align*}
        We have to show that $\mathrm{deg}(x,y) = \mathrm{deg}(x) + \mathrm{deg}(y)$.
        Let $\mathcal{B}^Q$, $\mathcal{B}^{Q'}$, $\mathcal{B}^N$, $\mathcal{B}^{N'}$ be positive bases for $T_xQ$, $T_xQ'$, $T_yN$ and $T_yN'$. Then $\mathcal{B}^{Q\times N^-}:= \left( (\mathcal{B}^Q, 0),
        (0,\mathcal{B}^N) \right)$ is a positively oriented basis.
        Similar for $\mathcal{B}^{Q'\times N'^-}$.
        Note that $\nu(x,y)=1$ if and only if $\mathcal{B}:=\left( \mathcal{B}^{Q'\times N^-},
        \mathcal{B}^{Q'\times N'^-} \right)$ is positively oriented.
        Now 
        \[
        \mathcal{B} = 
        \begin{pmatrix}
            \mathrm{Id} & 0 &0 &0 \\
            0 & 0 & \mathrm{Id} & 0\\
            0 & \mathrm{Id} & 0 & 0\\
            0 & 0 & 0 &  \mathrm{Id}
        \end{pmatrix}
        \mathcal{B}_0,
        \]
        where $\mathcal{B}_0 = \left( (\mathcal{B}^Q, 0), (\mathcal{B}^{Q'}, 0),
        (0,\mathcal{B}^N), (0,\mathcal{B}^{N'}) \right)$.
        Note that $\mathcal{B}_0$ is positively oriented if and only if 
        $\nu(x; Q, N) = \nu(y; N', N)$. Since the determinant of the matrix above is $(-1)^n$, it follows that $\mathcal{B}$ is positive if and only if
        $(-1)^n \nu(x;Q,N) \nu(y;N',N)=1$. Hence 
        \begin{align*}
            (-1)^{\mathrm{deg}(x,y)} = (-1)^{\frac{2n(2n+1)}{2}}\nu(x,y)
           & = (-1)^{\frac{2n(2n+1)}{2}}(-1)^{n}\nu(x;Q,N) \nu(y,N',N)\\
           & =\nu(x;Q,N) \nu(y,N',N)\\
           & =(-1)^{\frac{n(n+1)}{2}}\nu(x;Q,N)(-1)^{\frac{n(n+1)}{2}} \nu(y,N',N)\\
            &= (-1)^{\mathrm{deg}(x) + \mathrm{deg}(y)}.
        \end{align*}
        We conclude that $\mathrm{deg}(x,y) = \mathrm{deg}(x) + \mathrm{deg}(y)$.
    \end{enumerate}
\end{proof}
\end{comment}
