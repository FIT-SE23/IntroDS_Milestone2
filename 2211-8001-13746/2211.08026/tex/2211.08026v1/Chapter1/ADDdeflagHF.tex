We recall here Lagrangian Floer cohomology
for relatively aspherical Lagrangians.
Given two closed Lagrangians $L_0, L_1 \subset M$, choose $H$ so that
$\psi_1^H(L_0) \cap L_1$ is a transverse intersection at finitely many points. Then the underlying $\Lambda$-vectorspace of $CF(L_0,L_1; H,J)$ is generated by those points.
The differential is defined by counting $J$-holomorphic strips, using a $w$-compatible almost complex structure
$J$ on $M$.
Floer's equation reads:
\begin{equation*}
    \begin{cases}
      \frac{\partial u}{\partial t} + J_s(u) \left(
      \frac{\partial u}{\partial s} - X_s^H(u) \right) = 0\\
      u(0,t) \in L_0, \qquad u(1,t) \in L_1\\
      \lim_{t\to -\infty} u(s,t) = \psi_s^H(z) \text{ for some } z\in L_0 \\
      \lim_{t\to \infty} u(s,t) = \psi_s^H(w) \text{ for some } w\in L_0 \\
    \end{cases}\,.
\end{equation*}

If $L_0$ and $L_1$ are oriented, we define the degree of $x$ as follows:
\[
 (-1)^{\mathrm{deg}(x)} = (-1)^{\frac{n(n+1)}{2}}\nu(x; L_0,L_1), 
\]
where $\nu(x;L_0,L_1)\in \{\pm 1\}$ denotes the intersection index of $L_0$ and $L_1$ at $x$. This number is defined to be $+1$ if $v_1, \dots , v_{2n}$ is a positive basis
for $T_xM$ whenever $v_1,\dots , v_n$ is a positive basis for $T_xL_0$ and 
$v_{n+1}, \dots , v_{2n}$ is a positive basis for $T_xL_1$.
See \cite[Section 2d]{seidel00} for the grading, and \cite{robbin_salamon}
for the intersection index.