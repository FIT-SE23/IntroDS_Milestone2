%Juli2021 9.7_Shadows
$(S\times S) \#_{\Delta_S}^{\nu} \Delta$ is related to $S\times S$ and $\Delta$ via
a cobordism. 
\begin{comment}
This, and the construction, doesn't come as a surprise,
since Biran-Cornea \cite{BC1} introduced the trace of a surgery.
\end{comment}
This follows from a construction called "trace of a surgery", which is a surgery construction in one dimension higher. This was first introduced in \cite{BC1} for the case of a transverse surgery in a point.
As shown in \cite{MakWu}, exactly the same construction works for the $E_2$-surgery along clean intersections. We recall the construction in our special case.

Consider the symplectomorphism
\begin{align*}
\tilde{\varphi} \times \mathrm{id} \colon V \times V\times T^*\R &\longrightarrow T_\epsilon^*S^n \oplus T_\epsilon^*S^n \oplus T^*\R \subset T^*(S^n \times S^n \times \R) 
\end{align*}
and define the handle in the model $T^*(S^n \times S^n \times \R)$ 
%$T_\epsilon^*S^n \oplus T_\epsilon^*S^n \oplus T^*\R$
as follows:
\[
\hat{H}_\nu = \left \{ \psi_{\nu(\sigma_{\hat{\pi}}(\xi))}^{\sigma_{\hat{\pi}}}(\xi) \in T^*(S^n \times S^n\times \R) \, \big \vert \,
\xi \in N_{\Delta_S \times \{0\}}^* \backslash (\Delta_S \times \{0\}), \sigma_{\hat{\pi}}(\xi) \leq \epsilon \right \},
\]
where $\sigma_{\hat{\pi}} \colon T^*(S^n \times S^n\times \R) \to \R$ is given by
$\sigma_{\hat{\pi}}(\xi_1, \xi_2, p) = \vert \vert ((\xi_2,p)\vert \vert$.
Here, $\nu= \nu_\lambda$ is a $\lambda$-admissible function, as defined in 
Definition \ref{MWGraphDTadm}.
One computes
\begin{align*}
\psi_t^{\sigma{\hat{\pi}}}(\xi_1, \xi_2,p) = 
\left( \xi_1, \psi^{\sigma}_{\frac{t \vert \vert \xi \vert \vert}{\sqrt{\vert \vert \xi \vert \vert ^2 + p^2}}} (\xi_2),
\psi^{\sigma^{\mathbb{R}}}_{\frac{t \vert \vert p \vert \vert}{\sqrt{\vert \vert \xi \vert \vert ^2 + p^2}}} (p) \right)
\end{align*}
So more concretely,
$\hat{H}_\nu$ can be described as follows:
\begin{align*}
\hat{H}_{\nu} = \left \{ \left(\xi, \psi^{\sigma}_{\nu\left(\sqrt{\vert \vert \xi \vert \vert ^2 + p^2}\right) \frac{\vert \vert \xi \vert \vert}{\sqrt{\vert \vert \xi \vert \vert ^2 + p^2}}} (\xi),
 \psi^{\sigma^{\mathbb{R}}}_{\nu\left(\sqrt{\vert \vert \xi \vert \vert ^2 + p^2}\right) \frac{\vert \vert p \vert \vert}{\sqrt{\vert \vert \xi \vert \vert ^2 + p^2}}} (p)\right) 
 \, \Big \vert \,
 \begin{array}{l}
      \xi \in T_{\epsilon}^*S, p \in \mathbb{R},\\
      \sqrt{\vert \vert \xi \vert \vert ^2 + p^2}< \epsilon
 \end{array}
  \right\}.
\end{align*}
Here, $\sigma\colon T^*S \longrightarrow \mathbb{R}$ is the Hamiltonian function
$\sigma(\xi) = \vert \vert \xi \vert \vert$ we used earlier to define $\tau_S$
and $\sigma^{\mathbb{R}}\colon T^*\mathbb{R} \longrightarrow \mathbb{R}$ is given by
$\sigma^{\mathbb{R}}(p) = \vert p \vert$.
\begin{comment}
So for example, the projection of this cobordism looks like
\begin{align}\label{MWproj}
\pi_{\mathbb{C}}(\hat{H}_\nu) = \left \{ \nu(s) \frac{p}{s} - ip \, \big \vert \,
s \in (\vert p \vert, \epsilon), p\in (-\epsilon, \epsilon)\right\}
\subset \mathbb{C}.
\end{align}
This is the region below the rotated curve $\nu$.
\begin{center}
\includegraphics[scale=0.5]{Pictures/MWcob_pic1.pdf}
\end{center}
\end{comment}

The model handle $\hat{H}_\nu$ glues to a part of $(S^n\times S^n \times \R)\backslash \partial H$, which yields the model surgery trace
$$(S^n \times S^n \times \R) \#_{\Delta_S \times \{0\}} N_{\Delta_S \times \{0\}}^*.$$
Gluing this into $M\times M^- \times T^*\R$ via $\tilde{\varphi} \times \mathrm{id}$ we get
\[
V:=
(S\times S \times \R) \#_{\Delta_S \times \{0\}} 
\left( \Delta \times i\R \right):=
(\tilde{\varphi}\times id)^{-1}
\left((S^n \times S^n \times \R) \#_{\Delta_S \times \{0\}} N_{\Delta S \times \{0\}}^*\right). 
\]
$V \subset M\times M^- \times T^*\R$ is a Lagrangian submanifold. Under the identification $T^*\R \cong \C$ via $(q,p) \leftrightarrow q-ip$,
$V$ satisfies
\begin{align*}
&V\cap \pi_{\C}^{-1}(\epsilon) = S \times S \times \{\epsilon\},\\
&V\cap \pi_{\C}^{-1}(i\epsilon) = \Delta \times \{i \epsilon\},\\
&V\cap \pi_{\C}^{-1}(0) = (S \times S) \#_{\Delta_S}^{\nu} \Delta.
\end{align*}
By taking half of $V$, extending it by a ray of $(S \times S) \#_{\Delta_S}^{\nu} \Delta$ at $0$ and smoothing it, and bending the ends, as explained in \cite[Section 6.1]{BC1}, we get a 
cobordism 
$$\tilde{V}\colon (S \times S) \#_{\Delta_S}^{\nu} \Delta \rightsquigarrow 
(S\times S, \Delta) .$$

As discussed in section \ref{subsec:MWGraphDT}, $(S \times S) \#_{\Delta_S}^{\nu} \Delta$
is Hamiltonian isotopic to $\Gamma_{\tau_S^{-1}}$. Gluing a corresponing suspension to $\tilde{V}$ finally gives us the claimed cobordism 
\[
V_{MW}\colon \Gamma_{\tau_S^{-1}} \rightsquigarrow 
(S\times S, \Delta) .
\]

