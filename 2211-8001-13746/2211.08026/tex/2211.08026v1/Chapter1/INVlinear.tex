The map $$c_0:= c\vert _S \colon S \to S$$
induces an anti-symplectic involution
$$c_0^* \colon T^*S \to T^*S$$
via
$c_0^*(q,p) = (c_0(q), -p\circ (Dc_0)_{c_0(q)})$.
We choose a Riemannian metric $g$ on $S$ such that $c_0$ is an isometry with respect to $g$. 
The metric $g$ induces a canonical isomorphism $\alpha \colon TS \to T^*S$.
The following diagram commutes:
\begin{equation*}
\xymatrix{
	&T^*S \ar[r]^{c_0^*}  &T^*S \\
	&TS \ar[r]^{-Dc_0} \ar[u]_{\alpha}^{\cong} & TS \ar[u]_{\alpha}^{\cong}
}
\end{equation*}
\begin{comment}
Indeed,
\begin{align*}
\alpha (-Dc_0)(q,v) = \alpha (c_0(q), -(Dc_0)_q(v)) = (c_0(q), -g((Dc_0)_q(v), -)) \\
c_0^* \circ \alpha (q,v) = (c_0(q,0), -g(v,(Dc_0)_{c_0(q)}(-))
\end{align*}
and
\begin{align*}
g((Dc_0)_q(v), -) &= g((Dc_0)_{c_0(q)}, (Dc_0)_q(v), (Dc_0)_{c_0(q)}(-)) \\
 &= g(v, (Dc_0)_{c_0(q)}(-))
\end{align*}
\end{comment}
The following Lemma collects some properties of $c_0^*$.
\begin{lem}\label{INVlinear:lem:properties}
$c_0^* \colon T^*S \to T^*S$ satisfies
\begin{itemize}
\item $c_0^*(\xi) = - \phi_s^\sigma (-c_0^*(-\phi_s^\sigma (- \xi))) $
\item $\vert \vert c_0^* ( - \phi_s^\sigma (-\xi)) \vert \vert = \vert \vert \xi \vert \vert.$
\end{itemize}
\end{lem}

\begin{proof}
Let $\xi\in T_x^*S\cong T_xS$. Let $\gamma$ be the unique geodesic in $S$ with
$\gamma(0)=x$ and $\gamma'(0)=-\xi$.
Then $\phi_s^{\sigma}(-\xi) = \gamma'(s)$.
Note that
$c_0^*(\gamma'(s)) = -dc(\gamma'(s)) =- (c\circ \gamma)'(s)$.
Moreover, 
$(c\circ \gamma)'(0) = dc(-\xi)= -c_0^*(-\xi)$, 
hence 
\begin{align*}
\phi_s^\sigma ( c_0^* \phi_s^{\sigma}(-\xi))
&= \phi_s^{\sigma}(-\phi_s^{\sigma}(- c_0^*(-\xi)))\\
&= c_0^*(-\xi).
\end{align*}
$c_0^*$ commutes with the minus sign because it is linear.
So the first claim follows.
For the second, note that both $\phi_s^{\sigma}$ and $c_0^*$ both preserve
the length induced by $g$. The latter follows from $c$ being an isometry.

\end{proof}

We now show that there exists an isotopy $\sigma_t \colon T^*S \to T^*S$ between
$\sigma_0 = c$ and $\sigma_1= c_0^*$.
Write $c(q,p)=(c_1(q,p), c_2(q,p))$ with $c_1(q,p) \in S$ and
$c_2(q,p)\in T_{c_1(q,p)}S$.
Set
\begin{align*}
\sigma_{1-t}(q,p) =
\begin{cases}
 	(c_1(q,tp), \frac{c_2(1,tp)}{t}) \qquad t\neq 0\\
 	(c_1(q,0), (\partial _p c_2(q,0))p) \qquad t=0
\end{cases}
\end{align*}
Clearly, $\sigma_0 = c$. Moreover, $\sigma_1 \vert _S = c_0$ and $\sigma_1$ preserves the fibers. Recall the following standard result.
\begin{lem}
Let $f\colon M \to M$ be a diffeomorphism, $F\colon T^*M \to T^*M$ an anti-symplectic map with $F\vert_M = f$, fiber-preserving and $f^*\theta = -\theta$ for the Liouville form $\theta = p\mathrm{d}q$.
Then $F=-f^*$.
\end{lem}
\begin{proof}
Consider the symplectomorphism $G:= F \circ (-f^*)\colon T^*M \to T^*M$.
Then $G^*\theta = \theta$, $G = \mathrm{id}$ on $M$ and $\mathrm{d}\pi \circ \mathrm{d}G=\mathrm{d}\pi$ for the projection $\pi\colon T^*M \to T^*M$.
For $(q,p)\in T^*M$ and $v\in T_{(q,p)}T^*M$ we calculate
\begin{align*}
    p\mathrm{d}\pi(v) = \theta_{(q,p)}(v) = G^*\theta_{(q,p)}(v)
    = \theta_{(q,G_q(p))}\mathrm{d}G(v) =
    G_q(p)(\mathrm{d}\pi(\mathrm{d}G(v))) =
    G_q(p)\mathrm{d}\pi(v).
\end{align*}
We deduce $G_q(p)=p$, which implies $F=-f^*$.
\end{proof}

It is therefore enough to show that
$\sigma_1^*(\theta) = -\theta$,
where $\theta = pdq$.
Indeed, this property is satisfied:
\begin{align*}
(\sigma_1^* \theta)_{(q,p)}(v) &= (\partial_p c_2(q,0))(p) (D\pi (D\sigma_1)_{(q,p)}(v)) \\
&= (\partial_p c_2(q,0))(p)((Dc_0)_q D\pi(v)) \\
&= -p(D\pi(v))
\end{align*}
because
$$(\partial_p c_2(q,0))(p) \circ (Dc_0)_q = -p$$
coming from $c^*\omega = \omega$.
So $\sigma_t$ is an isotopy of anti-symplectic maps from $c$ to $c_0^*$.

Given constants $\eta_2 > \eta_1 > 0$, we cut off $\sigma_t$
such that the resulting isotopy $\sigma_t'$ is still anti-symplectic and satisfies 
\begin{align*}
    \sigma_t' = \begin{cases}
    \sigma_t \qquad \text{ on } T_{\eta_1}^*S,\\
    c \qquad \, \, \, \text{  on } T^*S \backslash T^*_{\eta_2}S.
    \end{cases}
\end{align*}
Then $\sigma'_0 = c$ on $T^*S$, $\sigma_1' = c_0^*$ on $T^*_{\eta_1}S$.

We claim that the compactly supported symplectic isotopy $\phi_t = c\circ \sigma_t' \colon T^*S \to T^*S$ is a Hamiltonian isotopy. Clearly, it is automatically Hamiltonian for every $n\geq 2$. For $n=1$, this follows from the fact, that $\phi_t$ preserves the zero-section $S$ for every time $t$:

\begin{lem}
Let $\phi_t\colon T^*{S^1} \to T^*{S^1}$ be a compactly supported symplectic isotopy. Assume that $\phi_t(S) = S$ for every $t$.
Then $\phi_t$ is a Hamiltonian isotopy.
\end{lem}
\begin{proof}
 By [MS] Theorem 10.2.5 and Exercise 10.2.6, it holds that
 $$\text{Flux}(\{\phi_t\}) = 0 \Rightarrow \phi_t \simeq \text{ a Ham. isotopy with fixed endpoints}.$$
 By inspecting the proof, one sees that
 if $\text{Flux}(\{\phi_t\}_{0\leq t\leq T}) = 0$ for every $T$, then 
 $\phi_t$ is a Hamiltonian isotopy. This condition is satisfied if 
 $\phi_t(S) = S$ for every $t$.
\end{proof}
We therefore proved:
\begin{prop}\label{INVlinear:prop:main}
For every $\eta_2 > \eta_1 >0$
there exists a Hamiltonian isotopy $$\phi_t \colon T^*S \to T^*S$$ with $\phi_0=\mathrm{id}$, $\phi_1 = c c_0^*$ on $T_{\eta_1}^*S$ and compact support in $T_{\eta_2}^*S$.
\end{prop}


