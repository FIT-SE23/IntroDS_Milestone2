Power side-channel (PSC) attacks are well-known threats to sensitive hardware like
advanced encryption standard (AES) crypto cores.
Given the significant impact of supply voltages (VCCs) on power profiles, various countermeasures 
based on VCC tuning have been proposed, among other defense strategies.
Driver strengths of cells, however, have been largely overlooked, despite having direct and significant impact
on power profiles as well.

For the first time, we thoroughly explore
the prospects of jointly tuning driver strengths and VCCs as novel working principle for PSC-attack countermeasures.
Toward this end, we take the following steps:
1)~we develop a simple circuit-level scheme for tuning;
2)~we implement a CAD flow for design-time evaluation of
ASICs,
enabling security assessment of ICs before tape-out;
3)~we implement a correlation power analysis (CPA) framework for thorough and comparative security analysis;
4)~we conduct an extensive experimental study of a regular AES design, implemented in ASIC as well as FPGA fabrics, under
various tuning scenarios;
5)~we summarize design guidelines for secure and efficient joint tuning.


In our experiments, we observe that runtime tuning is more effective than static tuning, for both ASIC and FPGA
implementations. For the latter, the AES core is rendered >11.8x (i.e., at least 11.8 times) as resilient as the
untuned baseline design.
Layout overheads can be considered acceptable, with, e.g., around +10\% 
critical-path delay for the most resilient tuning scenario in FPGA.

We will release source codes for our methodology, as well as artifacts from the experimental study, post peer-review.
