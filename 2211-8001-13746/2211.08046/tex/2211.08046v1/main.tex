
%%
%% This is file `sample-authordraft.tex',
%% generated with the docstrip utility.
%%
%% The original source files were:
%%
%% samples.dtx  (with options: `authordraft')
%% 
%% IMPORTANT NOTICE:
%% 
%% For the copyright see the source file.
%% 
%% Any modified versions of this file must be renamed
%% with new filenames distinct from sample-authordraft.tex.
%% 
%% For distribution of the original source see the terms
%% for copying and modification in the file samples.dtx.
%% 
%% This generated file may be distributed as long as the
%% original source files, as listed above, are part of the
%% same distribution. (The sources need not necessarily be
%% in the same archive or directory.)
%%
%% Commands for TeXCount
%TC:macro \cite [option:text,text]
%TC:macro \citep [option:text,text]
%TC:macro \citet [option:text,text]
%TC:envir table 0 1
%TC:envir table* 0 1
%TC:envir tabular [ignore] word
%TC:envir displaymath 0 word
%TC:envir math 0 word
%TC:envir comment 0 0
%%
%%
%% The first command in your LaTeX source must be the \documentclass command.
\documentclass[sigconf,authorversion=true]{acmart}
%% NOTE that a single column version may required for 
%% submission and peer review. This can be done by changing
%% the \doucmentclass[...]{acmart} in this template to 
%% \documentclass[manuscript,screen]{acmart}
%% 
%% To ensure 100% compatibility, please check the white list of
%% approved LaTeX packages to be used with the Master Article Template at
%% https://www.acm.org/publications/taps/whitelist-of-latex-packages 
%% before creating your document. The white list page provides 
%% information on how to submit additional LaTeX packages for 
%% review and adoption.
%% Fonts used in the template cannot be substituted; margin 
%% adjustments are not allowed.

%%
%% \BibTeX command to typeset BibTeX logo in the docs
\AtBeginDocument{%
  \providecommand\BibTeX{{%
    \normalfont B\kern-0.5em{\scshape i\kern-0.25em b}\kern-0.8em\TeX}}}

%% Rights management information.  This information is sent to you
%% when you complete the rights form.  These commands have SAMPLE
%% values in them; it is your responsibility as an author to replace
%% the commands and values with those provided to you when you
%% complete the rights form.
\setcopyright{acmcopyright}
\copyrightyear{2023}
\acmYear{2023}
\acmDOI{XXXXXXX.XXXXXXX}
\acmDOI{}

%% These commands are for a PROCEEDINGS abstract or paper.
\acmConference[ISPD 2023]{International Symposium on Physical Design}{March 26--29, 2023}{Online}
%
%  Uncomment \acmBooktitle if th title of the proceedings is different
%  from ``Proceedings of ...''!
%
%\acmBooktitle{Woodstock '18: ACM Symposium on Neural Gaze Detection,
%  June 03--05, 2018, Woodstock, NY} 
%\acmPrice{15.00}
%\acmISBN{978-1-4503-XXXX-X/18/06}


%%
%% Submission ID.
%% Use this when submitting an article to a sponsored event. You'll
%% receive a unique submission ID from the organizers
%% of the event, and this ID should be used as the parameter to this command.
%%\acmSubmissionID{123-A56-BU3}

%%
%% For managing citations, it is recommended to use bibliography
%% files in BibTeX format.
%%
%% You can then either use BibTeX with the ACM-Reference-Format style,
%% or BibLaTeX with the acmnumeric or acmauthoryear sytles, that include
%% support for advanced citation of software artefact from the
%% biblatex-software package, also separately available on CTAN.
%%
%% Look at the sample-*-biblatex.tex files for templates showcasing
%% the biblatex styles.
%%

%%
%% For managing citations, it is recommended to use bibliography
%% files in BibTeX format.
%%
%% You can then either use BibTeX with the ACM-Reference-Format style,
%% or BibLaTeX with the acmnumeric or acmauthoryear sytles, that include
%% support for advanced citation of software artefact from the
%% biblatex-software package, also separately available on CTAN.
%%
%% Look at the sample-*-biblatex.tex files for templates showcasing
%% the biblatex styles.
%%

%%
%% The majority of ACM publications use numbered citations and
%% references.  The command \citestyle{authoryear} switches to the
%% "author year" style.
%%
%% If you are preparing content for an event
%% sponsored by ACM SIGGRAPH, you must use the "author year" style of
%% citations and references.
%% Uncommenting
%% the next command will enable that style.
%%\citestyle{acmauthoryear}

%%%%%%%%%%%%%%%%%%%%
%%% own settings %%%
%%%%%%%%%%%%%%%%%%%%
\settopmatter{printacmref=false}
%\renewcommand\footnotetextcopyrightpermission[1]{}
%\captionsetup[subfigure]{labelformat=empty}
\newcommand{\smallerspace}{\vspace{-0.8em}}
\newcommand{\littlesmallerspace}{\vspace{-0.5em}}    
%\renewcommand{\smallerspace}{}
%\renewcommand{\littlesmallerspace}{}

%%%%%%%%%%%%%%%%%%%%
%%% own packages %%%
%%%%%%%%%%%%%%%%%%%%
\usepackage{soul}
\usepackage{multirow}
\usepackage{ragged2e}
\usepackage[noend]{algpseudocode}     
\usepackage{algorithm}            
\usepackage{threeparttable}
\usepackage{subfig}
\algrenewcommand\algorithmicforall{\textbf{foreach}}
\renewcommand{\algorithmicrequire}{\textbf{Input:}}
\renewcommand{\algorithmicensure}{\textbf{Output:}}

%\usepackage{setspace}
%\setstretch{0.9816}

%%
%% end of the preamble, start of the body of the document source.
\begin{document}

%%%
%%% The "author" command and its associated commands are used to define
%%% the authors and their affiliations.
%%% Of note is the shared affiliation of the first two authors, and the
%%% "authornote" and "authornotemark" commands
%%% used to denote shared contribution to the research.
%%%
\author{Saideep Sreekumar, Mohammed Ashraf, Mohammed Nabeel, Ozgur Sinanoglu, Johann Knechtel}
\email{{sds710, ma199, mtn2, ozgursin, johann}@nyu.edu}
\affiliation{%
  \country{New York University Abu Dhabi, UAE}
  %\streetaddress{P.O. Box 1212}
  %\city{Dublin}
  %\state{Ohio}
  %\country{UAE}
  %\postcode{43017-6221}
}

%\author{Lars Th{\o}rv{\"a}ld}
%\affiliation{%
%  \institution{The Th{\o}rv{\"a}ld Group}
%  \streetaddress{1 Th{\o}rv{\"a}ld Circle}
%  \city{Hekla}
%  \country{Iceland}}
%\email{larst@affiliation.org}
%
%\author{Valerie B\'eranger}
%\affiliation{%
%  \institution{Inria Paris-Rocquencourt}
%  \city{Rocquencourt}
%  \country{France}
%}
%
%\author{Aparna Patel}
%\affiliation{%
% \institution{Rajiv Gandhi University}
% \streetaddress{Rono-Hills}
% \city{Doimukh}
% \state{Arunachal Pradesh}
% \country{India}}
%
%\author{Huifen Chan}
%\affiliation{%
%  \institution{Tsinghua University}
%  \streetaddress{30 Shuangqing Rd}
%  \city{Haidian Qu}
%  \state{Beijing Shi}
%  \country{China}}
%
%\author{Charles Palmer}
%\affiliation{%
%  \institution{Palmer Research Laboratories}
%  \streetaddress{8600 Datapoint Drive}
%  \city{San Antonio}
%  \state{Texas}
%  \country{USA}
%  \postcode{78229}}
%\email{cpalmer@prl.com}
%
%\author{John Smith}
%\affiliation{%
%  \institution{The Th{\o}rv{\"a}ld Group}
%  \streetaddress{1 Th{\o}rv{\"a}ld Circle}
%  \city{Hekla}
%  \country{Iceland}}
%\email{jsmith@affiliation.org}
%
%\author{Julius P. Kumquat}
%\affiliation{%
%  \institution{The Kumquat Consortium}
%  \city{New York}
%  \country{USA}}
%\email{jpkumquat@consortium.net}

%%%
%%% By default, the full list of authors will be used in the page
%%% headers. Often, this list is too long, and will overlap
%%% other information printed in the page headers. This command allows
%%% the author to define a more concise list
%%% of authors' names for this purpose.
%\renewcommand{\shortauthors}{Trovato and Tobin, et al.}

%%%%
%%%% The code below is generated by the tool at http://dl.acm.org/ccs.cfm.
%%%% Please copy and paste the code instead of the example below.
%%%%
%\begin{CCSXML}
%<ccs2012>
%<concept>
%<concept_id>10002978.10003001.10010777.10011702</concept_id>
%<concept_desc>Security and privacy~Side-channel analysis and countermeasures</concept_desc>
%<concept_significance>500</concept_significance>
%</concept>
%<concept>
%<concept_id>10010583.10010662</concept_id>
%<concept_desc>Hardware~Power and energy</concept_desc>
%<concept_significance>500</concept_significance>
%</concept>
%<concept>
%<concept_id>10010583.10010682.10010712</concept_id>
%<concept_desc>Hardware~Methodologies for EDA</concept_desc>
%<concept_significance>300</concept_significance>
%</concept>
%<concept>
%<concept_id>10010583.10010600</concept_id>
%<concept_desc>Hardware~Integrated circuits</concept_desc>
%<concept_significance>300</concept_significance>
%</concept>
%<concept>
%<concept_id>10010583.10010600.10010628</concept_id>
%<concept_desc>Hardware~Reconfigurable logic and FPGAs</concept_desc>
%<concept_significance>300</concept_significance>
%</concept>
%</ccs2012>
%\end{CCSXML}
%
%\ccsdesc[500]{Security and privacy~Side-channel analysis and countermeasures}
%\ccsdesc[500]{Hardware~Power and energy}
%\ccsdesc[300]{Hardware~Methodologies for EDA}
%\ccsdesc[300]{Hardware~Integrated circuits}
%\ccsdesc[300]{Hardware~Reconfigurable logic and FPGAs}

%%%
%%% Keywords. The author(s) should pick words that accurately describe
%%% the work being presented. Separate the keywords with commas.
\keywords{power side-channel (PSC), correlation power analysis (CPA), driver strength, supply voltage,
	application-specific integrated circuit (ASIC), field-programmable gate array (FPGA), computer-aided design
	(CAD), power simulation, power measurement}

%%% A "teaser" image appears between the author and affiliation
%%% information and the body of the document, and typically spans the
%%% page.
%\begin{teaserfigure}
%  \includegraphics[width=\textwidth]{template/sampleteaser}
%  \caption{Seattle Mariners at Spring Training, 2010.}
%  \Description{Enjoying the baseball game from the third-base
%  seats. Ichiro Suzuki preparing to bat.}
%  \label{fig:teaser}
%\end{teaserfigure}

\newcommand{\jk}[1]{{\color{orange}JK: {#1}}}

\newcommand{\T}[2]{\#TTD(${#1}\%$, {#2})}

\title{X-Volt: Tuning Driver Strengths and Supply Voltages to Defend Power Side-Channel Attacks}
\title{X-Volt:
		Tuning Driver Strengths and Supply Voltages for an\\
		Efficient Defense against Power Side-Channel Attacks}
\title{X-Volt:
		Joint Tuning of Driver Strengths and Supply Voltages Against Power Side-Channel Attacks}

\begin{abstract}
\begin{abstract}

This paper presents a learning framework to estimate an agent capability and task requirement model for multi-agent task allocation.
With a set of team configurations and the corresponding task performances as the training data, linear task constraints can be learned to be embedded in many existing optimization-based task allocation frameworks.
Comprehensive computational evaluations are conducted to test the scalability and prediction accuracy of the learning framework with a limited number of team configurations and performance pairs.
A ROS and Gazebo-based simulation environment is developed to validate the proposed requirements learning and task allocation framework in practical multi-agent exploration and manipulation tasks.
Results show that the learning process for scenarios with 40 tasks and 6 types of agents uses around 12 seconds, ending up with prediction errors in the range of 0.5-2\%.



\end{abstract}

% \begin{IEEEkeywords}
% Some keywords
% \end{IEEEkeywords}

\end{abstract}

\maketitle

\section{Introduction}
\label{sec:intro}

\textbf{Background:}
To protect sensitive data handled within integrated circuits (ICs), the use of cryptographic (crypto) modules is widely
adopted. Such modules are
based on provably secure algorithms for encryption/decryption of data.  Still, once attackers have access to ICs, direct or
even only remote/indirect, they can monitor
the runtime behaviour and physical interactions with the environment, e.g.,
via measurements (direct) or via software interfaces to embedded sensors (remote/indirect).
Such observations enable so-called \textit{side-channel attacks}~\cite{zhou05}, which can
serve to infer the secret key used for crypto modules, etc.

\textbf{Limitation of Prior Art:}
Power side-channel (PSC) attacks are a well-known and effective type of side-channel
attacks~\cite{brier04,skorobogatov12}.
Thus, a plethora of PSC countermeasures have been proposed, e.g., masking and hiding~\cite{li17_SCA%,9015513
}, voltage
switching~\cite{8361769,9006696%,10.1145/2345770.2345774%,DBLP:journals/corr/abs-1907-09440
}, noise injection~\cite{bellizia18%,8351968
}, etc.
Given the direct impact of supply voltages (VCCs) on power profiles, various countermeasures are
based on some kind of VCC tuning.
Driver strengths of cells, however, have been largely overlooked, despite significant
impact on power profiles as well.

\begin{figure}[tb]
\includegraphics[width=.85\columnwidth]{incl/power_hist_1.pdf}\\[3mm]
\includegraphics[width=.85\columnwidth]{incl/power_hist_2.pdf}\\[3mm]
\includegraphics[width=\columnwidth]{incl/power_hist_3.pdf}\\[3mm]
\smallerspace
	\caption{Motivational example for the impact of tuning.
		Histograms of power profiles, for a regular AES core implemented in a GlobalFoundries 55nm
		technology.
			The two ``VCC Tuning'' scenarios at the top are for
			two different cases of driver strengths assigned to all AES
			  flip-flops. These scenarios demonstrate two aspects of tuning:
			  (i) VCC tuning results in power profiles (red) that largely overlap with
			  the baseline profiles (blue and yellow) and (ii) the overlap or rather shape/distribution of
			  the tuned profile depends on the driver strengths.
			The ``X-Volt'' scenario at the bottom demonstrates how joint tuning of driver strengths and VCCs renders the
			resulting profile (black) even more interspersed.}
	\label{fig:power_profiles}
\smallerspace
\end{figure}

\textbf{Motivation -- Impact of Tuning:}
In Fig.~\ref{fig:power_profiles}, we show the power profiles for a regular advanced encryption standard (AES) core
when operating the core under different driver strengths and VCCs.
We observe that dynamic tuning---that is, runtime reconfiguration of switching drivers and/or utilized
VCCs---results in profiles that largely
overlap with the baseline profiles.

{Such interspersion of power profiles as shown in Fig.~\ref{fig:power_profiles} represents a major challenge for PSC
attacks} as follows. Assuming the same
text is processed, using the same key, but under varying tuning settings, a large number of different power values can arise within the
resulting power profiles. The reverse also applies: for the same power value observed, a large range of different
possible texts and/or different possible keys may be underlining of the crypto computation.
Naturally, such ambiguity can be quite misleading for analytical models that are at the heart of PSC attacks.

\textbf{Contributions:}
{As indicated, prior art did overlook the potential of jointly tuning driver strengths and VCCs in general,
let alone for dynamic runtime modes.} In this work, we
address this gap.
We build up a multi-part methodology to thoroughly study various 
scenarios for joint tuning of driver strengths and VCCs, applicable for ASIC as well as FPGA fabrics.

We take the following steps:
\begin{enumerate}
\item We develop a simple circuit-level scheme for tuning, which is applicable for ASIC as well as FPGA fabrics.
\item We implement a CAD flow for design-time evaluation of ASIC power profiles at runtime,
enabling proper security assessment of ICs before tape-out.
\item We implement a correlation power analysis (CPA) framework for a thorough and comparative security analysis of tuning.
\item As key contribution of this work, we conduct an extensive experimental study of a regular AES design, implemented in ASIC as
well as FPGA fabrics, under various tuning scenarios.
\item Finally, we derive design guidelines for secure and efficient joint tuning in ASIC and FPGA fabrics.
\end{enumerate}
We emphasize that our work is not meant to compete with or replace prior art for PSC 
countermeasures, but rather to extend the landscape of available options at a foundational level. Joint tuning
can be either used as stand-alone measure, as it is done in this work at hand, or to further complement prior countermeasures.

\textbf{Findings:}
In our experimental study, we observe the following.
\begin{enumerate}

\item For the ASIC design based on a GlobalFoundries 55nm technology,
we find that a)~static design-time tuning
may improve the resilience, but only to limited degrees, and can sometimes even counteract it.
In contrast,
b)~dynamic runtime tuning always renders the AES core more resilient, namely up to 235\% as resilient as the untuned
baseline.\footnote{%
	This quantitative finding is conservative, as it is based on design-time evaluation
	without impact of layout effects, let alone measurement noises. Thus,
	for attacks on real hardware, we can assume a larger impact---we confirm this via FPGA implementation.}

\item We confirm our key finding---that dynamic runtime tuning is more effective---in the field, using the Sakura-X
FPGA board based on a 28nm technology.
Here, the AES core is rendered >11.8x (i.e., at least 11.8 times) as resilient.

\item
Regarding trade-offs for resilience versus layout overheads, we find them
reasonable with, e.g., $\approx$10\% impact on critical-path delay for the
most resilient FPGA tuning scenario.

\end{enumerate}

\textbf{Release:}
We will release source codes for our methodology, as well as empirical artifacts, post peer-review in~\cite{anon_web}.

\section{Background}
\label{sec:bg}

\subsection{Side-Channel Attacks}
\label{sec:bg:SCA}

Side-channel attacks infer sensitive information by observing and analysing physical channels
established by ICs during operation~\cite{zhou05}. These channels are leaking some kind of information due to the
basic workings of the underlying circuitry, but also due to micro-architectural implementation decisions. For the
latter, e.g.,
timing behavior and speculative execution in modern processors has been demonstrated as vulnerability~\cite{osvik05}.
For the classical PSC, e.g.,
it is well-known that the secret key for AES
can be inferred by analysing the data-dependent power
consumption~\cite{brier04,skorobogatov12}.
	
Different PSC attacks have been demonstrated, like
correlation power analysis
(CPA)~\cite{brier04}, mutual information analysis~\cite{gierlichs2008mutual}, 
or machine learning-based techniques~\cite{picek2017side}.
Furthermore, there are more generic, analytical
approaches
	like test vector
leakage assessment (TVLA)~\cite{schneider15},
	architecture correlation~\cite{yao20},
etc.

Without loss of generality (w/o.l.o.g.), we focus on the CPA attack in this work.
CPA is well-established and used widely throughout the literature.
CPA is effective, e.g.,
CPA requires on average fewer traces than DPA~\cite{brier04,massimo08,Fei2015}.
More details are explained in Sec.~\ref{sec:method:CPA_framework}.

\subsection{Prior Art for Countermeasures}
\label{sec:bg:prior}

Various countermeasures against PSC attacks {have} been proposed over the years,
including masking and hiding~\cite{li17_SCA%,9015513
}, voltage
switching~\cite{8361769,9006696%,10.1145/2345770.2345774%,DBLP:journals/corr/abs-1907-09440
}, noise injection~\cite{bellizia18%,8351968
}, etc.
Essentially, these countermeasures seek to de-correlate the observable
power consumption from the sensitive crypto operations.

More specifically, masking and hiding approaches do restructure and reimplement the design such that
sensitive operations are decomposed/split at the functional as well as the circuit level.
However, the overheads for such schemes can scale quadratically with the related
security requirements, making efficient implementations quite challenging~\cite{10.1007/978-3-319-52153-4_6}.
Voltage switching can be enabled by, e.g., multiple voltage domains and related control circuitry, or by integrated voltage
regulators (IVRs)~\cite{8361769}.
Note that IVRs are commonly available in modern IC designs, as they can enable significant power savings.
Noise injection, like interposing random data into
redundantly designed register paths~\cite{bellizia18}, is effective but can also {incur} considerable area and power costs.

Driver strengths, while directly impacting power profiles and thus the resilience against PSC attacks, 
have been largely overlooked in prior art. To the best of our knowledge, ``Karna''~\cite{karna}
is the only recent work that has explicitly studied the role of driver strengths (known as gate sizes in~\cite{karna}), along with
VCCs and threshold voltages. The authors found that varying strengths has also varying impact on the resilience.
Tuning of these parameters, however, was limited to static design-time tuning for ASICs.\footnote{%
In contrast, we study both static and dynamic tuning, for ASIC as well as FPGA fabrics. We find that static tuning
has limited effects; this is in agreement with findings in ``Karna.'' We also find that static
tuning is even counteractive for FPGA implementation, whereas ``Karna'' did only study ASIC implementation.}


\section{Threat Model}
\label{sec:tm}

We consider a stringent threat model for PSC attacks as follows.

Attackers can only act as passive observers.
That is, attackers do have direct/indirect access to the ASIC or FPGA, but only for monitoring the power
consumption and the cipher-texts. Attackers have no control of plain-texts and no control over the power supply.

We assume that attackers are fully aware of our countermeasure's working principle.
However, given that operation of the tuning implementation (Sec.~\ref{sec:method:tuning})
is randomized, randomly switching between different tuning scenarios, and given that power profiles for
different tuning scenarios are considerably interspersed (recall Fig.~\ref{fig:power_profiles}),
attackers cannot ascertain the specific driver strength and VCC underlying for any particular point in
time or operation. Accordingly, attackers cannot explicitly separate the multiple distributions underlying the power
profiles, which will hinder any analytical attack model.

\section{Methodology}

\subsection{Runtime Tuning of Driver Strengths, VCCs}
\label{sec:method:tuning}

This section outlines implementation options for the key idea of our work, that is dynamic tuning of driver strengths
and VCCs. Other options could be devised as well, e.g., toward a more optimized, cell-level integration of
different driver strengths.

Registers in general, and those holding AES texts in particular, are most
relevant for PSC attacks, since they build up considerable correlation between the processed data and the observable
power consumption;
see also
 Sec.~\ref{sec:method:CAD_flow} and Sec.~\ref{sec:method:CPA_framework}.
Thus, for both ASIC and FPGA implementations, we focus on registers.

\textbf{Implementation in ASICs:}
For static driver-strength tuning during design time, we simply reconfigure the strength for each register
of choice. We randomly select, w/o.l.o.g., either the lowest or highest available strength. To maintain the selected
strengths throughout the design flow, we mark these register as ``don't touch.''

For dynamic tuning of driver strengths, we implement tunable registers
as outlined in Fig.~\ref{fig:tune_imp} and described next.
For each register of choice, we replace it with a pair of registers, again w/o.l.o.g.~one with lowest and one with
highest available strengths, respectively.
Register pairs are marked as ``don't touch''
such that the strengths are maintained.
We reconnect the original register's nets through two additional multiplexers
(MUXes) such that only one of the two registers is randomly selected for operation at a time; the other register is
feeding back itself the current data, i.e., is guaranteed to not toggle at that point in time.

\begin{figure}[tb]
\includegraphics[width=0.90\columnwidth%, height=5.5cm%, keepaspectratio
]{incl/tune_imp.pdf}
\littlesmallerspace
	\caption{Implementation principle of dynamic tuning of driver strengths.}
	\label{fig:tune_imp}
\smallerspace
\smallerspace
\end{figure}

For design-time evaluation, VCC tuning is mimicked through cell and library configurations.
For actual ASIC implementations, we assume IVRs or other tuning features to be
available.
Note that
requirements for such
would be reasonable; dynamic VCC tuning is required only once per
full AES round, not every clock cycle. This is because the CPA attack focuses on the last (or first) intermediate
round~\cite{brier04}; other attacks follow similar principles of attacking specific parts of the AES operation.

\textbf{Implementation in FPGAs:}
Note that common FPGA fabrics do not provide the option for reconfiguring cell driver strengths.
However, IO pins can be reconfigured for driver
strengths and other parameters.
Thus, we implement tuning on FPGAs as follows.

For each register of choice, we additionally connect its output with two IO pins, again w/o.l.o.g.~one with lowest and one with highest
available IO driver strengths, respectively.
Similar to the ASIC implementation, we use additional MUXes to randomly select only one IO pin to be driven at
a time.

For static tuning scenarios, we simplify the above implementation by additionally connecting and hard-wiring each register of choice to only one
IO pin of lowest/highest IO driver strength.

For VCC tuning, we assume some tuning features are available, like FPGA on-board voltage regulators.

\subsection{CAD Flow for Design-Time Evaluation of ASIC Power Profiles}
\label{sec:method:CAD_flow}


The CAD flow described here serves to investigate the role that joint tuning of driver strengths and VCCs plays
against PSC attacks early on, in a design-time simulation environment, without need for FPGA implementation
or even IC tape-out and measurements.\footnote{%
We still conduct FPGA implementation, measurements,
and related analysis later on, to verify our findings for practical attack scenarios and across hardware
fabrics.}
Note that the idea of such CAD-flow-based investigation is not new; similar approaches
have been taken in, e.g., \cite{karna,Sadhukhan2019}. Still, the flow described here has been devised
     independently of prior art, and also verified within other studies
[omitted for blind review].

Our flow (Fig.~\ref{fig:CAD_flow}) takes as inputs: (i) the register-transfer level (RTL) code of the design to be
evaluated, e.g., a regular AES crypto core, (ii) the
standard-cell library of choice, and (iii) sets of plain-texts and keys. The latter can also be randomly generated
within the flow itself.
The flow provides the zero-delay power values,
i.e. power values without any impact of layout effects or noises, for the design's circuit-level computation as
triggered by the plain-text and key inputs.
These power values are then utilized for PSC evaluation/security analysis (Sec.~\ref{sec:method:CPA_framework}).

\begin{figure}[tb]
\includegraphics[width=.95\columnwidth]{incl/CAD_flow.pdf}
\littlesmallerspace
	\caption{CAD flow for design-time evaluation of zero-delay power profiles of ASICs.}
	\label{fig:CAD_flow}
\smallerspace
\smallerspace
\end{figure}

Next, we describe the flow in some more detail.
For the implementation, we employ regular commercial tools (Sec.~\ref{sec:setup:tools}).
We will release source codes for our CAD flow post peer-review in~\cite{anon_web}.

\textbf{Step 1:}
We synthesize the AES core's RTL. We verify the functionality of the obtained gate-level netlist
using a Verilog testbench, with randomized sets of plain-texts and keys.
We also confirm the functionality of the design, using software simulation of the crypto operations and cross-checking of
the two sets of cipher-texts.

\textbf{Step 2:}
We perform zero-delay gate-level simulation of the design, to generate a value change dump (VCD) file.
Note that VCD files are
well-established for simulation purposes.

\textbf{Step 3:}
The VCD file is then used for power simulation of the synthesized gate-level netlist.
To limit simulation efforts---without comprising the accuracy for the PSC attack evaluation---we focus only on the relevant
time intervals, i.e., the last (or first round) of AES, which are the ones sufficient to attack~\cite{brier04}.

\textbf{Scope of Simulations:}
Instead of performing full-scale transient simulations, which would also capture noises induced by glitching activities, here we leverage
noise-free, zero-delay simulations.
For our notion of tuning driver strengths and VCCs, glitches are less relevant; tuning has significant impact
on power profiles overall (recall Fig.~\ref{fig:power_profiles}), not only on
glitching activities.

For such zero-delay simulation, all power-consuming transitions occur simultaneously for
the clock edge. Thus, peak-power values, which are of particular relevance for PSC attacks, can be easily extracted.
Further, note that register are generally contributing the largest shares of dynamic power consumption.
While the registers holding the secret key itself are not switching, thus not providing any leverage for PSC attacks,
other registers do switch. In fact, those register that are holding the texts of intermediate AES rounds
incur considerable switching activities
      by design, due to the confusion and diffusion properties of the AES crypto algorithm, and can thus be well
      correlated against.


\subsection{CPA Framework for Security Analysis}
\label{sec:method:CPA_framework}

The CPA framework described here serves for an empirical security analysis, yet in a thorough manner and backed by
solid analytical formalism.
We will release source codes
post peer-review in~\cite{anon_web}.

As indicated, CPA is known to be effective~\cite{brier04,massimo08,Fei2015}.
At the heart of CPA is the linear {Pearson correlation coefficient (PCC)}, used to quantify the
relationship between actual power profiles and hypothetical power profiles.
The latter are typically built up by enumeration of all
byte-wise possible keys~\cite{brier04}.
After building up correlation over a number of traces---obtained in any way, e.g., via design-time power
simulations using the above CAD flow or via measurements---the most promising
candidates for all bytes are concatenated to form the guess of the correct key.


We take the following steps in our framework.

\textbf{Step 1:}
Note that, since registers consume a significant share of
dynamic power during signal transitions, the Hamming distance (HD) for the registers'
data before and after switching operations is established as simple, yet effective, \textit{HD power model}~\cite{brier04}.

Now, as indicated, sets of hypothetical power values are to be derived for all possible key values.
This is done using the HD power model, namely by reverting the AES last-round operation using the observed cipher-texts, and
computing and memorizing the HD when considering all possible key values for that reverse operation.

\textbf{Step 2:}
Using the PCC formalism, the actual power traces---again, can obtained in any way---are
correlated against all hypothetical power profiles. The profile resulting in the
highest PCC value across a number of traces is assumed to represent the correct key.
As indicated, the correlation analysis can be conducted at the byte level (instead of bit level)~\cite{brier04},
which is essential to manage complexity for exploring
the search space of all possible keys.

Instead of considering all available traces at once for this correlation analysis, we thoroughly and step-wise explore
the range of how many traces are needed at least until disclosure of the correct key with certain
confidence. See Sec.~\ref{sec:metrics:sec} for more details.


\textbf{Attack Versus Security Analysis:}
Acting as designers, we can readily verify the key guess for any CPA run during the security analysis.
An attacker, however, has to monitor the progression of PCC values for all the possible key hypotheses throughout a more or less
large number of traces; only once the best candidate shows a significant PCC outlier among all other candidates,
can the attacker assume to have successfully inferred the correct key.

We take the attacker's approach for parts of our study as well, namely for realistic pre-processing (i.e.,
without relying on the actual correct key) of noisy power traces obtained for FPGA measurements. There, any sub-set of
traces that does not exhibit a sufficiently significant PCC outlier is rejected as too noisy.

\section{Empirical Study: Setup}
\label{sec:setup}


\subsection{Experimental Setup}

\begin{figure}[tb]
\includegraphics[width=0.70\columnwidth, height=4.4cm%, keepaspectratio
]{incl/lab_setup.png}
\littlesmallerspace
	\caption{FPGA measurement setup.}
	\label{fig:lab_setup}
\smallerspace
\smallerspace
\end{figure}

\begin{figure*}[tb]
\includegraphics[width=\textwidth]{incl/ASIC_boxplots.pdf}
	\smallerspace
	\smallerspace
	\caption{CPA results for the ASIC implementation. See the main text for description of the different
		scenarios. Also recall that, for each data point underlying each box,
			there is a robust and thorough sampling process underlying (Sec.~\ref{sec:metrics:sec};
					Footnote~\ref{fn:sampling}).}
	\label{fig:ASIC_CPA}
	\littlesmallerspace
\end{figure*}

\textbf{Tools:}
\label{sec:setup:tools}
We devise the CAD flow and perform ASIC implementation using standard commercial tools, i.e., Synopsys DC for logic
synthesis and Synopsys VCS for gate-level power simulation. We devise custom tcl scripts for the CAD flow integration
and bash scripts for data management and processing.
We use Xilinx ISE Webpack suite for FPGA implementation.
We implement the CPA framework in C++, based on the release in~\cite{CPA_yunsi}.

\textbf{Design:}
We utilize a regular AES core, with 128-bit keys and 128-bit texts processed in electronic code book (ECB) mode.
We release the RTL post peer-review in~\cite{anon_web}.

\textbf{Implementations:}
For the ASIC implementation, we employ a commercial 55nm technology by GlobalFoundries, for logic synthesis and
zero-delay, gate-level power simulation.
For the FPGA implementation, we use a Sakura-X board, specifically its Kintex-7 FPGA chip, which is manufactured
in a 28nm technology. We build up a common FPGA measurement setup (Fig.~\ref{fig:lab_setup}).
We tune VCCs using the FPGA's on-board core-voltage regulator.

Naturally, the ASIC and FPGA implementations differ considerably in terms of (i) available driver strengths and VCCs,
(ii) technology nodes and hardware fabrics, and (iii) noise profiles.
Such diversity is essential to confirm and generalize our findings.

\textbf{Metrics and Workflow for Security Analysis:}
\label{sec:metrics:sec}
We report the minimal number of traces needed to disclosure 
as \T{c}{t}, i.e., for a confidence value $c$ across $t$ randomized CPA trials.

Throughout all experiments, we report \T{90}{1k} which means that $\geq$900 out of 1,000 randomized CPA trials succeed
for the reported number of traces.
To determine \T{90}{1k} values accurately, we conduct multiple CPA campaigns as follows.
Each campaign is run independently in steps, where an increasing number of randomly selected plaint-texts and corresponding
power traces are made available to the CPA framework. More specifically, for each campaign step, 1,000 randomized CPA
trials are conducted on 1,000 different sets of randomly selected texts and corresponding traces.
The success rate is tracked and more and more steps are taken, until the point of 90\% confidence is reached, i.e.,
900/1,000 trials succeed.
While such workflow is computationally intensive, it is trivial to parallelize, and essential for a robust security analysis.



For the exploration of different tuning scenarios,
we repeatedly conduct CPA campaigns with many trials, as outlined above, all while
maintaining the overall sets/pool of plain-texts and keys. Doing so is
important for fair comparison across tuning scenarios.

We consider sets of 5k traces for the ASIC implementation and 15k--170k traces for the FPGA implementation.\footnote{%
For the FPGA implementation, we observe that 15k traces are sufficient for breaking less resilient scenarios, whereas around 200k traces
are still insufficient for breaking the more resilient scenarios. As indicated, we pre-process measurement traces,
similar to what an attacker would do, to
reject noisy traces. After measuring 200k traces, we split them into 20 by 10k sets, and had to reject three sets;
thus, 170k traces remain.}
   For each step in any CPA campaign, we increase the number of available traces by 5 and by 15,
   respectively, for the ASIC and the FPGA implementation.

\textbf{Metrics and Workflow for Layout Analysis:}
We report power, performance, and area (PPA) numbers for ASIC and FPGA implementations.
For ASIC implementation, performance and area is reported from logic synthesis using DC.
Power
is reported as average peak power, derived from the same gate-level simulations used for security analysis.
For FPGA implementation, performance and area---the latter in terms of utilization of flip-flops (FFs) and look-up
tables (LUTs)---are reported from ISE runs.
Power is reported as average peak power from measurements.
We also report additional IO pins used for implementing tuning in FPGA.

\subsection{Tuning Settings}
We consider the following tuning settings for our study.
Each setting comprises different scenarios, in terms of static versus dynamic tuning
and in terms of driver strengths, VCCs available for tuning in the ASIC versus FPGA
implementation.
\begin{enumerate}

\item[(I)] All FFs are tuned to the same driver strength and VCC.

	\begin{enumerate}

	\item[(Ia)] Static tuning only.

	\item[(Ib)] Static and dynamic tuning, covering all combinations of static/dynamic tuning for driver strengths and VCCs.

	\item[(Ic)] Dynamic tuning only.

	\end{enumerate}

\item[(II)] FFs holding AES texts versus all other FFs are separated into two groups.
Groups of FFs are tuned differently, whereas all FFs within a group are tuned the same.
This setting is motivated by the potential need for a more limited and less costly implementation; 
see also Sec.~\ref{sec:experiments:layout}.
	\begin{enumerate}

	\item[(IIa)] Static tuning only.

	\item[(IIb)] Dynamic tuning of FFs holding AES texts; static tuning of all other FFs.

	\end{enumerate}

\end{enumerate}
Note that, for Setting (II), we refrain from considering further possible scenarios, like dynamic tuning of FFs holding AES
texts along with different dynamic tuning of all other FFs.
This is w/o.l.o.g., due to practical limitations on the number of available IO pins for dynamic tuning on our FPGA
of choice.

\section{Empirical Study: Security Analysis}
\label{sec:experiments:security}



\subsection{ASIC Implementation}

Detailed results are provided in Fig.~\ref{fig:ASIC_CPA}; related observations are presented next.
Note that the numbering of scenarios below matches that in Fig.~\ref{fig:ASIC_CPA}.
Also note that, in Sec.~\ref{sec:experiments:security:summary}, we streamline and summarize
findings for both ASIC and FPGA implementations.

\begin{enumerate}

    \item \textit{Untuned Baseline:}
The regular AES design, without any tuning. VCC is set to 1.08V for all FFs and all other gates.
Driver strengths are set automatically by logic synthesis.
Here, 100 CPA campaigns are conducted,
resulting in $N=100$ data points for the \T{90}{1k} metric (Sec.~\ref{sec:metrics:sec}), but recall that
many more CPA trials are underlying for a robust analysis.\footnote{%
	\label{fn:sampling}
As indicated, each campaign progresses in steps of
more traces becoming available (w/o.l.o.g., 5 traces for the ASIC implementation), and for each step 1,000 trials are run.
To avoid running many trials with too few
traces to begin with, we initially conduct some exploratory sampling, to determine a reasonable starting point for all
campaigns; we found 600 traces suitable for this scenario. Thus, there are 100 campaigns run, with 16--25 steps per
campaign (covering the observed min and max points of 680 and 725 for \T{90}{1k}, respectively), with 1,000 trials per step, resulting in 1.6--2.5 million CPA trials in total, just for exploring this tuning
scenario. All other scenarios are explored in the same thorough manner.}


\end{enumerate}

\textit{Tuning Setting (I):} For static or dynamic tuning of all FFs,
	we consider the following scenarios.

\begin{enumerate}
\setcounter{enumi}{1}

    \item \textit{Static X, VCC:}
    All combinations for all five available driver strengths, ranging from X0.5 to X4, as well as three
    available VCCs, ranging from 0.9V to 1.08V, are considered, resulting in 15 tuning configurations and, across
    three CPA campaigns, in $N=45$ data points.

    This is the least resilient scenario across Setting (I), with little difference to the untuned baseline.
    Along with low standard deviation (SD) across all 45 combinations, this indicates that
    \ul{exclusively static tuning is not effective}.


    \item \textit{Static X0.5, Dynamic VCC:}
    Here, 100 runs for randomized, dynamic VCC tuning across 0.9--1.08V, are considered in three CPA campaigns,
    resulting in $N=300$ data points.

    This scenario is more resilient than static tuning (2), but
    less resilient than dynamic VCC tuning for higher driver strengths (4), and also less resilient than
    driver-strength tuning for static VCCs (5), (6).
    This indicates that dynamic tuning can be beneficial, when applied thoughtfully.


    \item \textit{Static X4, Dynamic VCC:}
    Same setup as in (3), except driver strengths are set to X4 for all FFs.

    As indicated, this scenario is more resilient than VCC tuning for lower driver strength (3). It is also somewhat
    more resilient than driver-strength tuning for low, static VCC (5). These observations, together with those for
    (2), imply that \ul{high driver strengths
	    can be beneficial}
	    and \ul{dynamic VCC
	    tuning can be beneficial}.
    

    \item \textit{Dynamic X, Static 0.9V:}
    Here, 100 runs for randomized, dynamic driver-strength tuning across X0.5--X4, are considered in three CPA campaigns, resulting in
		    $N=300$ data points.

    As indicated, this scenario is on average more resilient than VCC tuning for low driver strength (3), but
    somewhat less resilient than VCC tuning for high driver strength (4).


    \item \textit{Dynamic X, Static 1.08V:}
    Same setup as in (5), except VCCs are set to 1.08V for all FFs.

    On average, this scenario is more resilient than all prior ones. Considered along with (5), this
    implies that, \ul{for dynamic driver-strength tuning, high VCCs can be beneficial}.
    

    \item \textit{Dynamic X, Dynamic VCC:}
    Here, 100 runs for randomized, dynamic driver-strength tuning across X0.5--X4 and dynamic VCC tuning across 0.9--1.08V,
    are considered in three CPA campaigns, resulting in
		    $N=300$ data points.

    This is the most resilient scenario. This implies that \ul{dynamic tuning of both driver strengths
    and VCCs provides superior resilience}, namely on average 196\% and up to 235\% as resilient as both static tuning and the
	    untuned baseline.


\end{enumerate}


For \textit{Tuning Setting (II)}, separate tuning of FFs holding AES texts versus all other FFs, we consider the following scenarios.

\begin{enumerate}
\setcounter{enumi}{7}

    \item \textit{Static Only:}
    For each of the two groups, all combinations of five driver strengths, ranging from X0.5 to X4, as well as two VCCs, 
    0.9V and 1.08V, are considered, resulting in 100 configurations, $N=300$ data points for three CPA campaigns.


    Here, we observe a similarly low resilience as with static-tuning Scenario (2), again with low SD across all
    different configurations.
    This re-iterates that \ul{static tuning alone is not effective, not even in any particular tuning configuration}.


   
    \item \textit{Static and Dynamic:}
    All combinations of two corner-case driver strengths, X0.5 and X4, and VCCs, 0.9V and 1.08V, are
    considered for static tuning of other FFs. At the same time, all six possible, non-redundant combinations for dynamic
    tuning using different configurations for driver strengths and VCCs are considered for FFs holding AES texts.
    In other words, this scenario explores dynamic tuning of FFs holding AES texts, while statically tuning all
    other FFs.
    All combinations are explored via ten CPA campaigns, resulting in
    $N=4*6*10=
    240$ data points.

    Also here, we observe a low resilience, similar to
    (2) and (8), albeit with
    a higher SD, implying that some particular configurations are more promising.
    Still, \ul{limiting dynamic tuning to only the FFs holding AES texts is not effective}.\footnote{While this
	    applies for the ASIC implementation, we observe that, for the FPGA implementation, such limited tuning can
		    still be effective.}


\end{enumerate}

\subsection{FPGA Implementation}

Next, we elaborate on our findings for the FPGA implementation.
In Sec.~\ref{sec:experiments:security:summary}, we streamline and summarize all
findings.

Given that available driver strengths and VCCs, as well as noise profiles, differ vastly from those of the ASIC
implementation, quantitative results cannot be compared across these implementations.
More importantly, however, observations are verified across hardware fabrics and even
technology nodes, rendering our insights on the prospects of tuning robust.

Detailed results
are provided in Fig.~\ref{fig:FPGA_TTD}; related observations are discussed next.
The scenario numbering matches that in Fig.~\ref{fig:FPGA_TTD}.

\begin{figure*}[tb]
\includegraphics[width=.90\textwidth]{incl/FPGA_traces.pdf}
\littlesmallerspace
	\caption{CPA results for the FPGA implementation. See the main text for description of the different
		scenarios.
			Note the varying ranges for \T{90}{1k} across the different tuning scenarios.
			Also recall that, for each data point underlying each box,
			there is a robust and thorough sampling process underlying (Sec.~\ref{sec:metrics:sec};
					Footnote~\ref{fn:sampling}).}
	\label{fig:FPGA_TTD}
\littlesmallerspace
\end{figure*}

\begin{enumerate}

	\item \textit{Untuned Baseline, 0.955V:}
	The regular AES design, without any tuning. VCC is set to 0.955V for all FFs.

	\item \textit{Untuned Baseline, 1.055V:}
	The regular AES design, without any tuning. VCC is set to 1.055V for all FFs.
This scenario is less resilient then (1), suggesting that
	\ul{lower VCCs \textit{can} be beneficial}, especially as long as driver strengths are not tuned dynamically; see
		also the remaining scenarios.

\end{enumerate}

Recall that we use additional IO pins for
driver-strength tuning in the FPGA implementation (Sec.~\ref{sec:method:tuning}).
Since IO pins are limited, we also want/need to limit the number of FFs that are tuned.
Thus, counterparts for promising configurations observed for the ASIC implementation, namely Setting (I) in general and the most resilient Scenario (7) in
particular---dynamic tuning of both driver strengths and VCCs for all FFs---are impractical for the FPGA implementation.

Accordingly, we skip directly to \textit{Tuning Setting (II)}, separate tuning of FFs holding AES texts versus all other FFs.
Again considering limited numbers of IO pins, we do not specifically tune other FFs here, but only FFs
	holding AES texts.
	We consider the following resulting scenarios.
   30 CPA campaigns are conducted
(i.e., $N=30$ data points)
	for each scenario unless stated otherwise.

\begin{enumerate}
\setcounter{enumi}{2}


    \item \textit{Static X4, 0.955V:}
This scenario is only slightly more resilient than the untuned baseline for the same VCC, indicating that static
tuning with low driver strength is not effective.

    \item \textit{Static X4, 1.055V:}
Here, there is a significant \textit{drop} in resilience.
Considering together with (3) and the remaining scenarios for static tuning, i.e., (5) and (6), this indicates that
\ul{static tuning is most often counterproductive}, due to increased information leakage
incurred via toggling IO pins according to AES texts held in the related FFs.

    \item \textit{Static X16, 0.955V}

    \item \textit{Static X16, 1.055V}


    \item \textit{Dynamic X, Static 0.955V:}
This scenario represents a strong turning point.
On average, when compared to the corresponding untuned and statically-tuned baselines,
resilience is increased to 199\% and 197--297\%, respectively.
This indicates that \ul{dynamic driver-strength tuning is effective}.
Unlike with static tuning, where any related changes in power profiles can still be well correlated against, such dynamic tuning
significantly interrupts the correlation working principle, by interspersing power
profiles \textit{in a randomized manner} such that different texts may well be related to the same power values and vice versa, as motivated in
Sec.~\ref{sec:intro}.

    \item \textit{Dynamic X, Static 1.055V:}
This scenario represents another turning point, as in high a VCC notably improving resilience again, unlike for the untuned
baselines or static-tuning scenarios. This implies that \ul{high VCCs can be beneficial in combination with
dynamic driver-strength tuning}. 

    \item \textit{Dynamic X, Dynamic VCC:}
This is the most resilient scenario by far.
Here, we consider even 100 CPA campaigns for robust sampling, and find that
none can break the tuning-induced resilience, even when considering all 170k available traces at once.\footnote{%
	Accordingly, there are no corresponding \T{90}{1k} data points included in Fig.~\ref{fig:FPGA_TTD}.}
This implies that resilience is increased by \textit{at least} 11.8x over the untuned baseline (1).
This clearly shows that \ul{joint and dynamic tuning is by far most resilient}.

\end{enumerate}

\subsection{Summary}
\label{sec:experiments:security:summary}

Our findings for both the ASIC and FPGA implementations are:
\begin{enumerate}
\item Dynamic VCC tuning is promising, but limited on its own;
\item Dynamic driver-strength tuning, along with high VCCs or dynamic VCC tuning, is most
effective;
\item Tuning of all FFs is promising, but is also limited in practice (by available IO pins for the FPGA implementation and
by overheads for the ASIC implementation; see Table~\ref{tab:ASIC_PPA} below);
\item Static tuning is least effective in general and even counterproductive for the FPGA implementation (where implicit masking by
environmental noises can be nullified when using high VCCs and/or high driver strengths for tuning).
\end{enumerate}


\section{Empirical Study: Layout Analysis}
\label{sec:experiments:layout}


\begin{table}
\setlength\tabcolsep{2.0pt}
	\small
	\caption{Layout Analysis for ASIC Implementations}
	\smallerspace
	\label{tab:ASIC_PPA}
	\begin{tabular}{cccc}
		\toprule
		\multirow{3}{*}{Design}& Avg. Peak & Critical-Path & Std.-Cell \\
		& Power [mW]
		& Delay [ns] & Area [$\mu m^2$] \\
		& 0.9V / 1.08V & 0.9V / 1.08V & 0.9V / 1.08V \\
		\toprule
		Baseline & 2.709 / 3.100 & 9.64 / 9.79 & 54,928 / 43,639 \\
		\midrule
		All FFs & 3.134 / 3.764 & 14.13 / 11.86 & 67,873 / 57,300 \\
		Tunable & (+15.69\% / +21.42\%) & (+46.68\% / +21.14\%) & (+23.57\% / +31.30\%) \\
		\midrule
		AES-Text & 2.779 / 3.237 & 14.13 / 11.68 & 57,272 / 46,324 \\
		FFs Tunable & (+02.58\% / +04.42\%) & (+46.68\% / +19.31\%) & (+04.27\% / +06.15\%) \\
		\bottomrule
	\end{tabular}%\\[1mm] % only needed if table footnote is used
\littlesmallerspace
\end{table}


\textbf{ASIC Implementation:}
See Table~\ref{tab:ASIC_PPA}.
Naturally, layout costs are larger when all FFs are tunable, whereas costs are reasonable if only FFs holding
AES texts are tunable.

We argue that costs may well be amortized for large-scale ASIC designs with many modules. In
contrast, to study upper limits of overheads, here we consider a stand-alone AES core.
Besides, some optimized cell-level integration of different driver strengths and tuning peripherals may be attainable
in future work.

\begin{table}
	\small
	\caption{Layout Analysis for FPGA Implementations}
	\smallerspace
	\label{tab:FPGA_PPA}
	\begin{tabular}{cccc}
		\toprule
		\multirow{3}{*}{Design}& Avg. Peak & \multirow{2}{*}{Critical-Path} & \multirow{2}{*}{FFs / LUTs} \\
		& Power [mW]
		& \multirow{2}{*}{Delay [ns]} & \multirow{2}{*}{Util. [\# / \#]} \\
		& 0.9V / 1.08V & & \\
		\toprule
		Baseline & 0.969357 / 1.07049 & 9.052 & 952 / 3,137 \\
		\midrule
		\multirow{2}{*}{Static X4} & 0.972771 / 1.07352 & 10.352 & 965 / 3,118 \\
		& (+00.35\% / +00.28\%) & (+14.36\%) & (+01.37\% / -00.61\%) \\
		\midrule
		\multirow{2}{*}{Static X16} & 0.971117 / 1.07347 & 10.352 & 965 / 3,118 \\
		& (+00.18\% / +00.27\%) & (+14.36\%) & (+01.37\% / -00.61\%) \\
		\midrule
		\multirow{2}{*}{Dynamic} & 0.973676 / 1.07214 & 9.994 & 1,028 / 3,183 \\
		& (+00.45\% / +00.15\%) & (+10.41\%) & (+07.98\% / +01.47\%) \\
		\bottomrule
	\end{tabular}%\\[1mm] % only needed if table footnote is used
\littlesmallerspace
\end{table}


\textbf{FPGA Implementation:}
See Table~\ref{tab:FPGA_PPA}.
Note that, for both static-tuning designs, critical-path delays and utilization are the same; this is because only the
driver-strength configurations for the additional IO pins differs here, whereas the core circuitry remains the same.
Overall, we observe marginal impact on power as well as utilization,
along with some overheads for
delays.

Delay overheads are due to large-scale changes imposed on placement and routing,
after connecting the circuitry to the additional IO pins used for tuning, as we have observed via ISE PlanAhead.
An iterative design strategy
might reduce overheads;\footnote{%
	For example, the strategy could be (i) placement and
routing, (ii) selection of nearby IO pins for assignment to tuned FFs, (iii) evaluation of layout overheads, and (iv)
	repeated selection/assignment of IO pins, guided by worst-case timing overheads, etc.}
	currently, we assign from available IO pins arbitrarily to FFs to be tuned.

\section{Design Guidelines for Tuning}

Recall the key findings for the security analysis, summarized in Sec.~\ref{sec:experiments:security:summary}.
Considering these together with the layout analysis in Sec.~\ref{sec:experiments:layout}, we propose the following design
guidelines.


Static tuning is discouraged.
Dynamic and joint tuning should be applied whenever possible. Otherwise, dynamic tuning of driver strengths is
preferred as simple, yet effective, alternative. This is because (i)~VCC tuning requires some IVR or other
tuning features, whereas driver-strength tuning can be implemented at circuit level at its own,
and (ii)~dynamic driver-strength tuning is more resilient.

Tuning of all FFs can be considered when the relatively high layout overheads for an ASIC implementation
are acceptable---which should be true for actual ASICs where crypto cores are only a small part---or as
long as sufficient IO pins are available for an FPGA implementation. Otherwise, tuning only the FFs that are holding
the AES text is still more resilient than untuned baselines, especially in the field where other noise profiles are
coming into play as well.

\section{Conclusions and Future Work}
\label{sec:conc}

In this paper,
we have explored
joint tuning
of driver strengths and VCCs as countermeasure against PSC attacks.
Toward this end, we have proposed a simple implementation scheme, devised a CAD flow for design-time exploration of
ASICs, devised a CPA framework for thorough
and robust security analysis, and conducted a comprehensive experimental study considering both ASIC and FPGA
fabrics under various tuning scenarios.
We find that dynamic tuning is particularly effective, increasing resilience considerable for ASIC and FPGA fabrics
   along with acceptable overheads.


For future work, we plan to study joint tuning in more detail as follows.
First,
	we shall explore more efficient means for tuning, e.g., circuit-level primitives for
	ASIC implementations or an iterative strategy for IO-pin assignment for FPGA implementations.
Second,
	we shall study tuning also in the context of leakage-power attacks, given that driver strengths and
	VCCs do impact leakage-power profiles as well.
Third,
	besides using a CPA attack, we shall also utilize generic approaches for security assessment, e.g., TVLA,
	to more rigorously study possible limitations for tuning.

\balance

%\bibliographystyle{ACM-Reference-Format}
%\bibliography{main}
\documentclass[journal]{IEEEtran}
\usepackage{cite}
\usepackage{amsmath} 

\usepackage{subfigure}
\ifCLASSINFOpdf
\usepackage[pdftex]{graphicx}
  % declare the path(s) where your graphic files are
  \graphicspath{{../pdf/}{../jpeg/}}
  % and their extensions so you won't have to specify these with
  % every instance of \includegraphics
  \DeclareGraphicsExtensions{.pdf,.jpeg,.png}
\else
  % or other class option (dvipsone, dvipdf, if not using dvips). graphicx
  % will default to the driver specified in the system graphics.cfg if no
  % driver is specified.
  \usepackage[dvips]{graphicx}
  % declare the path(s) where your graphic files are
  \graphicspath{{../eps/}}
  % and their extensions so you won't have to specify these with
  % every instance of \includegraphics
  \DeclareGraphicsExtensions{.eps}
\fi  
\usepackage{amsmath}
\usepackage{cases}
\usepackage{stfloats}
\usepackage{amsfonts}
\usepackage{subeqnarray}
\usepackage{longtable}
\usepackage{supertabular}
\usepackage{setspace}
\usepackage{multirow}
\usepackage{booktabs}
% \usepackage{algorithm}
% \usepackage{algorithmic}
\usepackage[ruled,linesnumbered]{algorithm2e}
\makeatletter
\newcommand{\nosemic}{\renewcommand{\@endalgocfline}{\relax}}% Drop semi-colon ;
\newcommand{\dosemic}{\renewcommand{\@endalgocfline}{\algocf@endline}}% Reinstate semi-colon ;
\newcommand{\pushline}{\InDPP}% Indent
\newcommand{\popline}{\Indm\dosemic}% Undent
\let\oldnl\nl% Store \nl in \oldnl
\newcommand{\nonl}{\renewcommand{\nl}{\let\nl\oldnl}}% Remove line number for one line
\makeatother
\usepackage[export]{adjustbox} 
\usepackage{booktabs}
\usepackage{setspace}
\usepackage{xcolor}
\usepackage{mathrsfs}
\usepackage{amsmath}
\usepackage{array}
\usepackage{amssymb}
\usepackage{amsthm}
\usepackage{microtype}
\usepackage{url}
\usepackage{amsfonts,amssymb}
% \usepackage{bbm}
\usepackage{dsfont}
\usepackage{mathtools}
\usepackage{xcolor,colortbl}
\usepackage{colortbl}
\usepackage{graphicx}
% \usepackage{tabularray}

\newcommand{\mc}[2]{\multicolumn{#1}{c}{#2}}
\definecolor{Gray}{gray}{0.85}
\definecolor{Whitecolor}{rgb}{1,1,1}


\hyphenation{op-tical net-works semi-conduc-tor}

\setlength{\textfloatsep}{5pt}
\allowdisplaybreaks
\begin{document}
\setstretch{1}
\title{\textls[-25]{The Design of By-product Hydrogen Supply Chain Considering Large-scale Storage and Chemical Plants: A Game Theory Perspective}}
\author{Qianni~Cao,~\IEEEmembership{Student~Member,~IEEE},
Boda~Li,~\IEEEmembership{Student~Member,~IEEE},
Mengshuo~Jia,~\IEEEmembership{Member,~IEEE}, and 
Chen~Shen,~\IEEEmembership{Senior~Member,~IEEE}



%\thanks{M. Jia and C. Shen are with the State Key Laboratory of Power Systems, Tsinghua University, 100084 Beijing, China. Y. Wang and G. Hug are with the Power Systems Laboratory, ETH Zurich, 8092 Zurich, Switzerland.}
}
        
%\thanks{This work was supported in part by the Joint Funds of the National Natural Science Foundation of China under Grant U1766206 (Correspondence to Chen Shen).}
%\thanks{M. Jia, C. Shen and Z. Wang are affiliated with the State Key Laboratory of Power Systems, Department of Electrical Engineering, Tsinghua University, Beijing 100084, China (e-mail addresses: jms16@mails.tsinghua.edu.cn, shenchen@mail.tsinghua.edu.cn,
%    wangzhaojian@mail.tsinghua.edu.cn).}% <-this % stops a space
% \thanks{Manuscript received April 19, 2005; revised August 26, 2015.}

%\markboth{Submitted to IEEE Trans. Smart Grid}%
%{Shell \MakeLowercase{\textit{et al.}}: Bare Demo of IEEEtran.cls for IEEE Journals}
\maketitle


\begin{abstract}
Hydrogen, an essential resource in the decarbonized economy, is commonly produced as a by-product of chemical plants. To promote the use of by-product hydrogen, this paper proposes a supply chain model among chemical plants, hydrogen-storage salt caverns, and end users, considering time-of-use (TOU) hydrogen price, coalition strategies of suppliers, and road transportation of liquefied and compressed hydrogen. The transport route planning problem among multiple chemical plants is modeled through a cooperative game, while the hydrogen market among the salt cavern and chemical plants is modeled through a Stackelberg game. The equilibrium of the supply chain model gives the transportation and trading strategies of individual stakeholders. Simulation results demonstrate that the proposed method can provide useful insights on by-product hydrogen market design and analysis.
\end{abstract}
%Although the historical data of renewable generations could be assumed as publicly known
% Note that keywords are not normally used for peerreview papers.
\begin{IEEEkeywords}
%   Wind power, chance constraint, OPF, distributed computing, confidentiality preservation
Hydrogen market, large-scale storage, Stackelberg game, cooperative game, supply chain
\end{IEEEkeywords}
\IEEEpeerreviewmaketitle

\section*{Nomenclature}
\addcontentsline{toc}{section}{Nomenclature}

\subsection*{Indices} 
\begin{IEEEdescription}[\IEEEusemathlabelsep\IEEEsetlabelwidth{$aaaaaaaa$}]
	\item[$i,j$]		Index of chemical plants.
	\item[$t$]		Index of time periods during the day.
	\item[$n$]		Index of hydrogen processing equipment, including liquefiers and compressors.
	\item[$I+1$]		Index of the salt cavern.
\end{IEEEdescription}
\subsection*{Parameters} 
\begin{IEEEdescription}[\IEEEusemathlabelsep\IEEEsetlabelwidth{$aaaaaaaa$}]
	\item[$I$]		Number of chemical plants.
	\item[$T$]		Number of time periods.
	\item[$p_{o}$]		Retail price purchased by customers from the salt cavern.
	\item[$\underline{p}_{t},\overline{p}_{t}$]	    Lower and upper bound of the buying price offered by the salt cavern to chemical plants.
	\item[$Q_{trans}$]		Maximal injection rate of the salt cavern.
	\item[$N_\mathcal{C}$]		Number of compressors with different capacity.
	\item[$N_\mathcal{D}$]		Number of liquefiers with different capacity.
	\item[$Q_{i,t}$]		By-product hydrogen quantity produced by chemical plant $i$ in period $t$.
	\item[$\boldsymbol{Q_{pr}}$]        Capacity set of hydrogen processing equipment(kg/h), $\boldsymbol{Q_{pr}}=\{Q_{pr}^{n}\}, \forall n$.
	\item[$Q_\mathcal{C},Q_\mathcal{D}$]        Capacity of a tube trailer and a tanker truck (kg/trip).
	\item[$w_{t}$]      Electricity price in period $t$.
	\item[$\gamma_{c},\gamma_{d}$]      Electricity consumption for unit compressed hydrogen and liquefied hydrogen (kwh/kg).
	\item[$\boldsymbol{K_{1}}$]     Initial investment set of hydrogen processing equipment, $\boldsymbol{K_{1}}=\{K_{1}^{n}\}, \forall n$.
	\item[$K_{2}^{c},K_{2}^{d}$]        Initial investment cost of a tube trailer and a tanker truck.
	\item[$K_{3}$]      Operation cost of a tube trailer (or a tanker truck) in each period.
	\item[$\boldsymbol{T_{a}}$]     $T_{a}^{i,j}$ represents duration from chemical plant $i$ to $j$ ($j= I+1$ represents the salt cavern).
	\item[$\beta_{L1}$]      1 - hourly evaporation rate during the tanker truck loading.
	\item[$\beta_{L2}$]      1 - hourly evaporation rate during transit by a tanker truck.
\end{IEEEdescription}
\subsection*{Decision variables of the salt cavern} 
\begin{IEEEdescription}[\IEEEusemathlabelsep\IEEEsetlabelwidth{$aaaaaaaa$}]
	\item[$p_{t}$]		Buying price the salt cavern offers to chemical plants in period $t$.
	\item[$q_{i,t}^{trans}$]		Hydrogen transaction amount of chemical plant $i$ in period $t$, measured as hydrogen shipped from chemical plant $i$ at the end of the time period $t$.
	\item[$u_{i,I+1}$]		Binary variables. Equals to 1 when products from chemical plant $i$ is shipped directly to the salt cavern. Otherwise, $u_{i,I+1}$ equals to 0.
\end{IEEEdescription}
\subsection*{Decision variables of chemical plants} 
\begin{IEEEdescription}[\IEEEusemathlabelsep\IEEEsetlabelwidth{$aaaaaaaa$}]
	\item[$q_{i,t}^{pr}$]		Hydrogen quantity chemical plant $i$ compressed/liquified in period $t$.
	\item[$\boldsymbol{x_{i}}$]		$\boldsymbol{x_{i}}=\{x_{i}\}, \forall n$ is a set of binary variables. $x_{i}^{n}=1$ when the type of hydrogen processing equipment is selected to purchase. Otherwise, $x_{i}^{n}=0$.
	\item[$N_{i}^{cars}$]		Integer variables of number of tube trailers (or tanker trucks) purchased by chemical plant $i$.
	\item[$u_{i,j}$]		Binary variables. Equals to 1 when products from chemical plant $i$ is shipped to chemical plant $j$. Otherwise, $u_{i,j}$ equals to 0.
	\item[$q_{i,t}^{store}$]		Hydrogen quality in the tube trailer (or tanker truck) left at chemical plant $i$ before filled to capacity in period $t$.
	\item[$q_{i,t}^{unpr}$]		Hydrogen quantity temporarily stored in low-pressure storage tanks before compression or liquefication in period $t$.
	\item[$n_{i,t}^{cars}$]		Integer variables of tube trailers (or tanker trucks) leave chemical plant $i$ in period $t$.
\end{IEEEdescription}



\section{Introduction}
\subsection{Motivation}
\IEEEPARstart{I}{n} 
the context of emission peak and carbon neutrality, hydrogen is not only regarded as a critical alternative to fossil fuel to achieve carbon neutrality but offers versatility and flexibility that renewables cannot reach\cite{Allan2021}. As one of the most cost-effective options, hydrogen produced as a by-product from many chemical plants serves as a cheap and large-scale source of hydrogen. Moreover, by-product hydrogen is usually sufficiently clean and well suited for a wide range of applications, such as fuel cell (FC)-based cogeneration, FC vehicles, domestic heating, and so on\cite{CAMPANARI2020335}. However, the potential of by-product hydrogen has yet to be realized, which is emitted and thus wasted in most cases. Therefore, it presents opportunities as a new revenue stream for chemical plants and promisingly delivers on announced pledges of energy conversion nationwide in the mid-term. However, the lack of infrastructure development such as large-scale storage, logistical supply chain establishment and unexplored market have slowed down its further development.

Salt cavern storage is one of the most promising technologies to achieve large-scale, fast and secure hydrogen storage\cite{ANDERSSON201911901},which offers the most promising option owing to their low investment cost, high sealing potential and low cushion gas requirement\cite{CAGLAYAN20206793}. Notable projects are the salt cavity storages for hydrogen in Teeside, UK, and Texas, USA\cite{Gregoire2019}, demonstrating the operation feasibility on a full industrial scale. However, the business of acquiring, storing and selling by-product hydrogen has not yet been presented as an option by salt cavern operators, which inspires the work to design by-product hydrogen supply chain considering large-scale storage and chemical plants in this paper.

\subsection{Literature Review}

As demand and production capacity for hydrogen grows robustly in recent years, the outlines of hydrogen markets are starting to emerge worldwide. Initial trade and market price discoveries come first on a regional and local basis\cite{James2021}. Infrastructure development, transparent pricing benchmark and logistical supply chain establishment are key growth challenges faced by this new traded commodity just becoming established in energy commodity markets\cite{Allan2021}. 

Presently,  the hydrogen market is far from mature but is showing great potential. Many researchers focus on the planning of the hydrogen supply chain, considering various market scales, hydrogen sources and transportation modes. Life cycle analysis to estimate the economic and environmental benefits was conducted on global\cite{BRANDLE2021117481}, regional\cite{OBARA2019848} or national\cite{REN2020118482} scales. For different hydrogen sources, steam methane reforming (SMR)\cite{CARRERA2021107386} , coal gasification (CG)\cite{LI202027979}, biomass gasification (BG)\cite{CHO2019527,LUMMEN2020118996} and electrolysis (ELE)\cite{WANG2022122194} are common production technologies in recent researches. Considering hydrogen production based on different feedstocks and energy sources, an optimal structure of the hydrogen, biomass and {$\rm CO_{2}$} networks were determined in \cite{GABRIELLI2020115245}. To make comparisons of different transportation modes, Ref. \cite{FAZLIKHALAF202034503} considered four common options with various criteria and scenarios. Ref. \cite{GIM20121162} introduced a method for comparing different transport possibilities of tube or liquid trailer vs. pipeline delivery. The results showed that each transportation technology had a maximally cost-efficient niche and there was no single perfect solution for the entire system. Recently, large-scale storage for liquid hydrogen is of great attention. Ref. \cite{SEO2020114452}  considered  integrated bulk storage of hydrogen and concluded that a centralized storage structure and liquefication in central production plants can reduce the overall cost. Similarly, the status and key gaps for the commercialization of hydrogen liquefication technology with large-scale storage were discussed in \cite{RATNAKAR202124149}. A combination of the hydrogen supply chain with other energy sources has also attracted the attention of many researchers. Ref. \cite{xiao2018} established a local energy market for electricity and hydrogen. Ref. \cite{CARRERA2021116861} proposed a methodological design framework for hydrogen and methane supply chains based on Power-to-Gas systems.

In particular, by-product hydrogen has seen growing attention these years. Ref. \cite{YANEZ2018777} for the first time assessed the economic advantages, the techno-economic feasibility and the central role of reusing by-product hydrogen in the early phase of hydrogen infrastructure in the northern Spain region. A multi-period programming was designed in \cite{YOON2022112083} to make use of existing infrastructure for by-product hydrogen and natural gas (NG) pipelines, which demonstrated the economic benefits of by-product hydrogen. Even though, the potential of by-product hydrogen remains to be discovered.

Meanwhile, most of the literature focuses on maximizing the total benefit of the whole hydrogen supply chain. Ref. \cite{HAN20125328} aimed to maximize social welfare in Korea by planning both capacity and technology of production, storage as well as transportation in an envisioned nationwide hydrogen supply chain. Ref. \cite{WICKHAM2022117740} assessed the effects that hydrogen grades play in the development of a cost-effective hydrogen supply chain. Ref. \cite{EHRENSTEIN2020115486} incorporated the concept of biophysical limits of the planet to address the optimal design of the hydrogen supply chain. An optimization method was proposed in \cite{QUARTON2020113936} for an integrated value chain of carbon dioxide and hydrogen. Individual rationality was introduced in \cite{GUO2021119608}, where the peer-to-peer transaction, endogenous market-clearing price, and uncertainties in hydrogen production were considered in detail. However, most works failed to consider the strategic behaviors and the profit of individual participants, which differed from the usual practice that suppliers and retailers are private companies and operate with a profit-driven mode.

The research gaps for the existing works are: 
\begin{enumerate}
	\item The potential of by-product hydrogen is yet to be realized and its corresponding market is waiting for further exploration.
	\item The dynamic process of chemical plants and salt caverns considering hydrogen generation, compression (or liquefaction), and the transaction is waiting to be modeled.
	\item The interactions and dynamic strategic behaviors of each stakeholder desire a more dedicated modeling framework that captures profits and rationality of individual participants.
\end{enumerate}

\subsection{Contribution}
In this work, we study the by-product hydrogen supply chain considering large-scale storage and multiple chemical plants. The main contributions are threefold:
\begin{enumerate}
	\item We establish a business model for salt caverns to acquire and store by-product hydrogen from chemical plants and sell them to end-users. The by-product hydrogen supply chain composed of each stakeholder in the business model is investigated.
	\item The hour-by-hour decision-making process of each stakeholder, i.e., chemical plants and the salt cavern, is investigated and mathematically modeled under the proposed business model, providing a foundation for the TOU hydrogen pricing strategy. 
	\item The by-product hydrogen market is formulated as a game, considering the individual rationality of each stakeholder. The planning problem among multiple chemical plants is modeled through a cooperative game. The hydrogen market among the salt cavern and chemical plants is modeled through a Stackelberg game, in which the salt cavern is the leader and chemical plants are the followers. 
\end{enumerate}

\section{A business model of salt caverns and chemical plants}
In this section, we develop a business model for salt caverns to acquire by-product hydrogen from chemical plants and sell them to end-users. Generation, large-scale storage, and consumptive way of by-product hydrogen in the business model is introduced first. Then, the comparison between the by-product hydrogen supply chain and the present hydrogen supply chain is made. Followed by this, the structure of the by-product hydrogen market under the proposed business model is introduced in the following section. 

\subsection{Generation, Large-scale Storage and Consumptive Way of By-product Hydrogen}
By-product hydrogen is a cost-competitive and widely distributed source of hydrogen.
The process of generation of by-product hydrogen and its consumptive ways are illustrated in Fig.\ref{fig:The process of generation of by-product hydrogen and its consumptive ways}.  
\begin{figure}[h] %可选参数 h t b p,代表允许图片出现的位置,h表示此处附近,t表示顶部,b表示底部,p表示单独一页,H表示固定此处
    \centering
    \includegraphics[width=8.5cm]{fig/Fig.1_The_process_of_generation_of_by-product_hydrogen_and_its_consumptive_ways.png}
    \caption{The process of by-product hydrogen generation and its consumptive ways}\label{fig:The process of generation of by-product hydrogen and its consumptive ways}
\end{figure}

Electrochemical processes, such as the industrial production of steel, caustic soda and chlorine, produce hydrogen as a by-product, burnt or emitted as the current practice. However, they can be made available for applications outside chemical plants as a future consumptive way. To transport products from the production facilities to storage sites, by-product hydrogen should be compressed or liquified in advance, which collectively 
are referred to as “hydrogen secondary processing”. Two common transportation modes are compressed gaseous hydrogen via tube trailers (CH2) and liquid hydrogen via tanker trucks (LH2). To alleviate the imbalance between supply and demand of hydrogen, underground cavities like salt caverns are potential to offer natural infrastructure to realize cost-effective and reliable hydrogen storage. At the last link in the supply chain, by-product hydrogen is sold and distributed to various end-users. The proposed generation, storage and consumptive way of hydrogen give rise to a promising by-product hydrogen business model consisting of chemical plants as suppliers, a salt cavern as a retailer and end-users as consumers. 

\subsection{Characteristics of By-product Hydrogen Supply Chain} \label{subsection: Characteristics of by-product hydrogen supply chain}

Differences between the by-product hydrogen supply chain under the proposed business model and most hydrogen supply chains found in literature can be mainly concluded as twofold: 1) composition of major costs; 2) flexibility to coordinate between planning and scheduling. These differences will lead to a distinct focus and a smaller timescale for the formulation of the by-product hydrogen supply chain, which is analyzed as follows:
% Please add the following required packages to your document preamble:
% \usepackage{multirow}
\begin{table}[h]
\centering
\caption{Major costs of hydrogen supply chain} \label{tab:Major costs of hydrogen supply chain}
\footnotesize
\begin{tabular}{cccc}
\hline\toprule
\multicolumn{2}{c}{\multirow{2}{*}{Major costs}}                                              & \multicolumn{2}{c}{Hydrogen Supply Chain}                        \\ \cline{3-4} 
\multicolumn{2}{c}{}                                                                          & \multicolumn{1}{l}{Traditional} & \multicolumn{1}{l}{By-product} \\ \hline
\multirow{2}{*}{Production}     & Investment & \checkmark &  \\
                                & Operation  & \checkmark &  \\ \hline
\multirow{2}{*}{Storage}        & Investment & \checkmark &  \\
                                & Operation  & \checkmark &  \\ \hline
\multirow{2}{*}{Transportation} & Investment & \checkmark  & \checkmark  \\
                                & Operation  & \checkmark & \checkmark \\ \hline
\multirow{2}{*}{\begin{tabular}[c]{@{}c@{}}Secondary \\ processing\end{tabular}} & Investment &                               & \checkmark                             \\
                                & Operation  &  & \checkmark \\ \hline
\end{tabular}
\end{table}
% Please add the following required packages to your document preamble:
% \usepackage{multirow}
\begin{table}[h]
\centering
\caption{Major costs and the influence factors}
\footnotesize
\label{tab:Major costs and the influence factors}
% \footnotesize
\begin{tabular}{cll}
\hline\toprule
\multirow{2}{*}{Major costs} &
  \multicolumn{2}{c}{\multirow{2}{*}{Influence factors}} \\
                                & \multicolumn{2}{c}{}                         \\ \hline
\multirow{2}{*}{Production}     & \multicolumn{2}{l}{1) Production technology} \\
                                & \multicolumn{2}{l}{2) Scale of production}   \\ \hline
\multirow{2}{*}{Storage}        & \multicolumn{2}{l}{1) Storage technology}    \\
                                & \multicolumn{2}{l}{2) Storage capacity}       \\ \hline
\multirow{3}{*}{Transportation} & \multicolumn{2}{l}{1) Transportation mode}   \\
 &
  \multicolumn{2}{l}{\multirow{2}{*}{\begin{tabular}[c]{@{}l@{}}2) Hydrogen volume\\ 3) Transport distance\end{tabular}}} \\
                                & \multicolumn{2}{l}{}                         \\ \hline
\multirow{3}{*}{\begin{tabular}[c]{@{}c@{}}Secondary \\ processing\end{tabular}} &
  \multicolumn{2}{l}{1) Type of processing equipment} \\
 &
  \multicolumn{2}{l}{\multirow{2}{*}{\begin{tabular}[c]{@{}l@{}}2) TOU electricity price\\ 3) Hydrogen volume\end{tabular}}} \\
                                & \multicolumn{2}{l}{}                         \\ \hline
\end{tabular}
\end{table}

\subsubsection{Different composition of major costs}
Major costs of hydrogen supply chain and their influence factors are demonstrated in Table \ref{tab:Major costs of hydrogen supply chain} and \ref{tab:Major costs and the influence factors}, respectively. Unlike the present hydrogen supply chain, producers in the by-product hydrogen supply chain benefit from very low-cost generation. Thus, the major cost comes from secondary processing and transportation.

Power is the major cost for secondary processing. If the liquefier or compressor operates at low-price periods, it may potentially reduce operating costs. Since electricity price fluctuates by hours, the strategic behaviors of each stakeholder should also be modeled by hour.

Transport cost is determined by transportation mode, hydrogen volume and the transport distance. For two transportation modes considered in this paper, LH2 features large transport capacity (often 10-20 times as CH2), high initial investment cost (several times as CH2) and hourly volatile losses. On the contrary, CH2 features low transport capacity, low initial investment cost and zero loss. Usually, for long-distance transportation of a large amount of hydrogen, CH2 is less economical since it requires long rides of much more vehicles than LH2. However, for mid- or short-distance of a small amount of hydrogen, CH2 is more economical since there is no volatile loss. Obviously, a reasonable decision of transportation mode would largely reduce the cost of each chemical plant.
\subsubsection{Less flexibility to coordinate between planning and scheduling }
For suppliers in the by-product hydrogen supply chain, the generation scale of hydrogen is limited by the production plan of their main products. Moreover, their location is less likely to be optimized for the transportation of by-product hydrogen.

Therefore, there may be a mismatch between each supplier's location and generation scale. Specifically, for distant (to the salt cavern) and medium-yield chemical plants, if CH2 is adopted, long-distance transport of more tube trailers may result in high transportation costs. Nevertheless, if LH2 is adopted, substantial volatile losses would happen due to hours of filling time. This situation results in a dilemma since both transportation mode leads to a revenue decline in some way. Therefore, we envision a scenario where several chemical plants in proximity to each other form a coalition and select a transit hub between them to lower transportation costs, instead of shipping individually to the salt cavern. Two examples of envisioned transportation routes are highlighted in color in Fig.\ref{fig:Possible routes for the salt cavern to acquire hydrogen from the chemical plants}. Moreover, to lower transportation costs, chemical plants destined for the transit hub adopt the CH2 transportation mode, while the transit hub destined for the salt cavern adopt the LH2 transportation mode. In this way, the dilemma between high transportation costs of CH2 and large volatile loss of LH2 is mitigated. 
\begin{figure}[h] %可选参数 h t b p,代表允许图片出现的位置,h表示此处附近,t表示顶部,b表示底部,p表示单独一页,H表示固定此处
    \centering
    \includegraphics[width=8cm]{fig/Fig.2_Possible_routes_for_the_salt_cavern_to_acquire_hydrogen_from_the_chemical_plants.png}
    \caption{Possible routes for the salt cavern to acquire hydrogen from the chemical plants} \label{fig:Possible routes for the salt cavern to acquire hydrogen from the chemical plants}
\end{figure}

To sum up, cost structure differences and the lack of flexibility to coordinate between production scale and location lead to a gap between the by-product hydrogen supply chain and the present ones. Therefore, it is essential to model the by-product hydrogen supply chain according to its characteristics rather than simply applying the model of the traditional hydrogen supply chain. 

\subsection{The Structure of By-product Hydrogen Market}

The structure of the proposed by-product hydrogen market is provided in this subsection, followed by the basic assumptions.

\begin{figure}[h] %可选参数 h t b p,代表允许图片出现的位置,h表示此处附近,t表示顶部,b表示底部,p表示单独一页,H表示固定此处
    \centering
    \includegraphics[width=8cm]{fig/Fig.3_The_structure_of_the_by-hydrogen_market_under_investigation.png}
    \caption{The structure of the by-hydrogen market under investigation} \label{fig:The structure of the by-product hydrogen market under investigation}
\end{figure}
The by-hydrogen market under the proposed business model has the structure illustrated in Fig.\ref{fig:The structure of the by-product hydrogen market under investigation}. Suppliers, namely chemical plants, process by-product hydrogen by liquefiers or compressors (depends on the decision results of each supplier) and deliver it to the retailers. The retailers, namely salt caverns, sell hydrogen to the customers. To simplify the problem, salt caverns are regarded as an entity owned by a single company.

This paper focuses on the transaction between suppliers and retailers. The following assumptions are made without loss of generality:

\begin{enumerate}
	\item The end-users buy all the hydrogen from the retailer at a fixed price. This may happen when the injection-production rate of the salt cavern is higher than the market demand in a region. In order to alleviate the supplier’s market power to drive up prices, we assume that the salt cavern and suppliers have reached such an
    agreement to bring a fixed price into effect. 
	\item The secondary processing cost and transport cost is undertaken by suppliers.  
	\item The production cost is neglected since hydrogen is a by-product of the industrial process of chemical plants. 
	\item Chemical plants would not adjust their production schedule of their main product for the revenue generated by by-product hydrogen.
\end{enumerate}

Based on the above assumptions, the retailer’s and suppliers’ problems can be described as follows. To maximize profits, the salt cavern intends to purchase as much hydrogen as possible from chemical plants at the lowest cost. If the price is too low, chemical plants are less likely to be attracted by this new revenue stream and may waste them as before, which reduces profits of the salt cavern. On the contrary, if the price is too high, the purchasing cost would increase. Therefore, it is important for the salt cavern to strike a balance between the attraction of chemical plants and the purchasing cost. To maximize profits, chemical plants upstream would like to sell more hydrogen when the selling price is high on the one hand, and to reduce processing costs and transport costs on the other hand.

Taking into account the analysis in the last subsection, the challenges of modeling the by-product hydrogen supply chain under the proposed structure are mainly twofold: 1) to explicitly consider possible coalition structures and transport route strategies in the timescale of transport duration, electricity price fluctuation and volatile losses; 2) and to allocate the payoff among the producers in some fairway.

\section{Strategies and decision-making process of stakeholders}

In this section, the decision-making process of each stakeholder is investigated and mathematically modeled under the proposed business model.

\subsection{The Retailer’s Problem}

In the price-setting problem of the salt cavern, the retailer decides its buying price $p_{t}$ (offered to the suppliers), while considering the reactions ${q_{i,t}^{trans}}$ from suppliers. The problem can be formulated as
\setlength{\abovedisplayskip}{3pt}
\begin{align}
    \max\limits_{p_{t}}\ p_{o}\sum_{i=1}^{I}\sum_{t=1}^{T}q_{i,t}^{trans}u_{i,I+1}-\sum_{i=1}^{I}\sum_{t=1}^{T}p_{t}q_{i,t}^{trans}u_{i,I+1}  \label{eq:constraint1}
\end{align}
\begin{align}
    s.t.\ \underline{p}_{t}\le p_{t}\le \overline{p}_{t},\forall t\label{eq:constraint2}
\end{align}
\begin{align}
    \sum\limits_{i=1}^{I}q_{i,t-T_{a}^{i,I+1}}^{trans}\le Q_{trans},\forall t\label{eq:constraint3}
\end{align}

Objective \eqref{eq:constraint1} is the retailer’s profit in which the first term is the selling income, and the second term is the purchasing cost. Inequality \eqref{eq:constraint2} restricts the price offered to suppliers to be within the interval $[\underline{p}_{t},\overline{p}_{t}]$ in each period. Here we assume that the retailer and suppliers have already reached an agreement to bring this constraint into effect. Inequality \eqref{eq:constraint3} prescribes maximal transaction quantity in each period by maximal injection rate of the salt cavern. $q_{i,t}^{trans}$ and $u_{i,I+1}$ are the optimal solution to the suppliers’ problem.

\subsection{The Suppliers’ Problem}

For the suppliers, the planning of the type of processing equipment, transportation mode, and the transport route as well as scheduling of transaction quantity, is formulated in this subsection. To capture the dynamic process of hydrogen transactions between each stakeholder in detail, as well as investigating dynamic strategic behaviors of each stakeholder, the loading process is elaborately taken into consideration. Specifically, hydrogen is produced as a by-product along with main products and has three possible disposal ways: 
\begin{enumerate}
    \item Hydrogen can be loaded to a tube trailer (or a tanker truck) after compression (or liquefaction). At the end of period $t$, tube trailers (or tanker trucks) filled to maximum capacity should depart from chemical plants. Otherwise, they stay until filled up in the following periods. Therefore, the transaction quantity sequence $q_{i,t}^{trans}$ depends on hydrogen processing quantity sequence $q_{i,t}^{pr}$ and capacity of the vehicle ($Q_\mathcal{C}$ for a tube trailer and $Q_\mathcal{D}$ for a tanker truck).
    \item Hydrogen can also be temporarily stored in low-pressure storage tanks before liquefication or achieving an adequate compression rate. It will further be loaded into tube trailers (or tanker trucks) after being compressed (or liquified) in the following periods.
    \item Hydrogen may also be discarded by being emitted or burnt as the current practice, which may happen when buying price offered by the salt cavern is too low or low-pressure storage tanks are filled up.
\end{enumerate}

The above three disposal ways offer multiple options for chemical plants during planning and scheduling. For example, a chemical plant with a generation volume of 100kg per hour, may purchase processing equipment of 100kg per hour. Thus, hydrogen can be processed hour-by-hour. An alternative is to purchase processing equipment of 1000kg per hour. In this case, by-product hydrogen can be temporarily stored in low-pressure storage tanks and will be processed every 10 hours. The suppliers’ problem is to find optimal solutions for planning and scheduling while considering possible coalitions with each other.

In the suppliers’ problem, if the destinations of all suppliers for hydrogen shipment are the salt cavern, decision variables should be the type of processing equipment $\boldsymbol{x_{i}}$ and hydrogen processing amount $q_{i,t}^{pr}$; if the scenario of coalitions of suppliers is taken into account, transport route $u_{i,j}$ of chemical plants $i$ and $j$, which form a coalition.

The decision-making problem, including constraints and objectives of supplier $i$, is given as follows.
\subsubsection{Constraints on transit shipment pattern}
\begin{gather}
    \sum\limits_{j=1}^{I+1}u_{i,j}=1,\forall i\label{eq:constraint4}\\
    u_{i,j}+u_{j,i}\le 1,\forall i,j \label{eq:constraint5}
\end{gather}

Constraint \eqref{eq:constraint4} denotes that the destination of each chemical plant is unique. Constraint \eqref{eq:constraint5} defines that any pairs of the chemical plant $(i,j)$ wouldn’t select each other as the transit destination simultaneously.

\subsubsection{Constraints on hydrogen processing and transport scheduling}

Chemical plants adopting CH2 satisfy: 
\begin{gather}
    n_{i,t}^{cars}\le(q_{i,t}^{pr}+q_{i,t-1}^{store})/Q_c^{car}\le n_{i,t}^{cars}+1,\forall t \label{eq:constraint6}\\
    q_{i,t}^{trans}=n_{i,t}^{cars}Q_\mathcal{C},\forall t \label{eq:constraint7}\\
    q_{i,t}^{store}=q_{i,t-1}^{store}+q_{i,t}^{pr}-q_{i,t}^{trans}, \forall t\in\{2,...T\} \label{eq:constraint8}
\end{gather}

Constraints \eqref{eq:constraint6} and \eqref{eq:constraint7} indicate that hydrogen transaction amount in each period is an integer multiple of the capacity of a tube trailer since only tube trailers filled to maximum capacity will depart from chemical plants. Constraint \eqref{eq:constraint8} denotes variations of hydrogen quantity stored in low-pressure storage tanks.

With the remaining proportion of hydrogen after being shipped from chemical plant $i$ to $j$ ($j= I+1$ represents the salt cavern) written as $\beta_{L2}^{i,j}=\beta_{L2} T_{a}^{i,j}$, chemical plants adopting LH2 satisfy
\begin{gather}
    n_{i,t}^{cars}\le(q_{i,t}^{pr}+\beta_{L1}q_{i,t-1}^{store})/Q_d^{car}\le n_{i,t}^{cars}+1,\forall t \label{eq:constraint9}\\  
    q_{i,t}^{trans}=n_{i,t}^{cars}Q_\mathcal{D}\sum_{j=1}^{I+1}u_{i,j}\beta_{L2}^{i,j},\forall t \label{eq:constraint10}
\end{gather}
\setlength{\abovedisplayskip}{-10pt}
\begin{multline}
    q_{i,t}^{store}=\beta_{L1}q_{i,t-1}^{store}+q_{i,t}^{pr}-{q_{i,t}^{trans}}/{\sum_{j=1}^{I+1}u_{i,j}\beta_{L2}^{i,j}},\\ \forall t \in \{2,...T\} \label{eq:constraint11}
\end{multline}

Constraints \eqref{eq:constraint9} and \eqref{eq:constraint10} indicate that the hydrogen transaction amount in each period is an integer multiple of the capacity of a tanker truck. Constraint \eqref{eq:constraint11} denotes variations of hydrogen quantity stored in low-pressure storage tanks.

Constraints irrelevant to transportation modes are given in \eqref{eq:constraint12}-\eqref{eq:constraint15}, in which the transport duration for chemical plant $i$ is written as $t_{ar}^{i}=\sum_{j=1}^{I+1}u_{i,j}T_{a}^{i,j}$.
\setlength{\abovedisplayskip}{3pt}
\begin{gather}
{\sum_{t - 2\times t_{ar}^{i}}^{t}n_{i,t}^{cars}} \leq N_{i}^{cars}, \forall t \in \left\{2\times t_{ar}^{i},...T \right\} \label{eq:constraint12} \\
q_{i,t}^{pr} \leq {\sum_{n = 1}^{N_{\mathcal{C}} + N_{\mathcal{D}}}x_{i}^{n}}Q_{type}^{n}, \forall t \label{eq:constraint13} \\
q_{i,t}^{unpr} \leq \sum_{n = 1}^{N_{\mathcal{C}} + N_{\mathcal{D}}} x_{i}^{n}Q_{type}^{n}, \forall t \label{eq:constraint14}
\end{gather}
\setlength{\abovedisplayskip}{-3pt}
\begin{multline}
q_{i,t}^{unpr} \leq q_{i,t - 1}^{unprocess} + Q_{i,t} - q_{i,t}^{pr} + {\sum_{j = 1}^{I}{u_{j,i}q_{j,trans}^{t - T_{a}^{i,j}}}},\\\forall t \in \left\{ \max{({1,T_{a}^{i,j}})},\ldots,T \}\right. \label{eq:constraint15} 
\end{multline}
\setlength{\belowdisplayskip}{5pt}

Constraint \eqref{eq:constraint12} imposes the total number of tube trailers (or tanker trucks) purchased by chemical plant $i$ as the upper bound of tube trailers (or tanker trucks) in the round trip during the time period $\left\lbrack t - 2\times t_{ar}^{i} \right\rbrack$. Constraint \eqref{eq:constraint13} prescribes the processing capability of each chemical plant. Constraint \eqref{eq:constraint14} restricts the upper bound of hydrogen stored locally, and the bound parameter is chosen as $\sum_{n=1}^{N_\mathcal{C}+N_\mathcal{D}}x_i^nQ_{type}^n$. Constraint \eqref{eq:constraint15} represents variations of hydrogen stored locally, in which `$\le$' indicates that hydrogen as a by-product can be stored temporarily or directly discarded.

If destinations of all suppliers for hydrogen shipment are the salt cavern, the objective of each chemical plant is to maximize its daily profit and is given in \eqref{eq:constraint16}, in which the income by selling hydrogen to the consumers, initial investment cost and operation cost are considered.
\begin{gather}
\max~\pi_{Fi} = {\sum_{t = 1}^{T}( p_{t}q_{i,t}^{trans} - C_{O}^{i} - C_{T}^{i} )} - C_{INV1}^{i} - C_{INV2}^{i} \label{eq:constraint16} 
\end{gather}
where
\begin{gather}
% \setlength{\belowdisplayskip}{8pt}
C_{O}^{i} = q_{i,t}^{pr}{\sum\limits_{n = 1}^{N_{\mathcal{C}} + N_{\mathcal{D}}}x_{i}^{n}}w_{t}\left( \gamma_{c}x_{i}^{c} + \gamma_{d}x_{i}^{d} \right) \label{eq:constraint17}\\
C_{T}^{i} = n_{i,t}^{cars}{\sum_{j = 1}^{I + 1}{u_{i,j}{K_{3}T}_{a}^{i,j}}} \label{eq:constraint18}\\
C_{INV1}^{i} = {\sum_{n = 1}^{N_{\mathcal{C}} + N_{\mathcal{D}}}{x_{i}^{n}K_{1}^{n}}} \label{eq:constraint19}\\
C_{INV2}^{i} = N_{i}^{cars}{({x_{i}^{c}K_{2}^{c} + x_{i}^{d}K_{2}^{d}})} \label{eq:constraint20}
\end{gather}
where transportation mode is written as $x_i^c=\sum_{n=1}^{N_\mathcal{C}}x_i^n$ and $x_i^d=\sum_{n=N_\mathcal{C}}^{N_\mathcal{C}+N_\mathcal{D}}x_i^n$; $C_O^i , C_T^i$ represent hourly processing and transport cost respectively; $C_{INV1}^i$ , $C_{INV2}^i$ represent investment cost of processing equipment and tube trailers (or tanker trucks) after converted into daily cost with a discount rate, respectively.
If the scenario where coalitions of suppliers are considered, we denote chemical plants in a coalition as $\Gamma$. For the chemical plant $i$, $\forall i\in\Gamma$, the objective is to maximize the daily profit of the coalition and is given as \eqref{eq:constraint21}.
\setlength{\belowdisplayskip}{6pt}
\begin{multline}
\max~~\pi_{F\tau}=\sum_{i \in \Gamma}{{\sum_{t = 1}^{T}\left( p_{t}q_{i,t}^{trans}u_{i,I + 1} - C_{O}^{i} - C_{T}^{i} \right)}} \\{ - C_{INV1}^{i} - C_{INV2}^{i}} \label{eq:constraint21}
\end{multline}
where $u_{i,I+1}=1$ when chemical plant $i$ is chosen as a transit hub. Otherwise $u_{i,I+1}=0$.


\section{Game formulation and solution}
\subsection{Game Formulation for By-product Hydrogen Supply Chain}

In this section, the by-product hydrogen market is formulated as a game, considering the individual rationality of each stakeholder. 

The decision-making process of each individual can be concluded as follows. The suppliers plan their initial equipment investment, coalition structure and transport routes in the planning stage. Then, the hydrogen transaction problem, including the retailer's pricing problem and suppliers' scheduling problem, is optimized in the scheduling stage.

The overall framework of the game models is illustrated in Fig.\ref{fig:Game models involved in by-product hydrogen supply chain including salt cavern and chemical plants}. Specifically, the planning problem of multiple chemical plants is formulated as a cooperative game, in which a binding coalition could be formed to reduce transport costs. The hydrogen transaction problem between the salt cavern and chemical plants is formulated as a Stackelberg game, in which the salt cavern is the leader and chemical plants are the followers.

\begin{figure}[] %可选参数 h t b p,代表允许图片出现的位置,h表示此处附近,t表示顶部,b表示底部,p表示单独一页,H表示固定此处
    \centering
    \includegraphics[width=8.5cm]{fig/Fig.4_Game_models_involved_in_by-product_hydrogen_supply_chain_including_salt_cavern_and_chemical_plants.png}
    \caption{Game models involved in by-product hydrogen supply chain including salt cavern and chemical plants} \label{fig:Game models involved in by-product hydrogen supply chain including salt cavern and chemical plants}
\end{figure}
\subsubsection{Coorperative game in the planning stage} \label{subsubsection: first-stage problem}
As previously analyzed in subsection \ref{subsection: Characteristics of by-product hydrogen supply chain}, coalitions between chemical plants would potentially lower transportation costs, thus bringing collective payoffs. Moreover, to fairly allocate the payoff $\pi_{F\tau}$ among the players, the Shapley value is adopted.

The following assumptions are made without loss of generality when considering possible coalition structures:

i) Chemical plants in each coalition select one of them as a transit hub to which other chemical plants in the coalition transport hydrogen. Since reducing transport costs is considered as the key factor behind the coalition, we assume that two chemical plants destined for the salt cavern lack the motivation to form a coalition.

ii)	The influence of hydrogen price variations on the coalition structure is neglected since the salt cavern's buying price is unknown at the planning stage. Moreover, the driving force in forming a coalition is to reduce costs rather than to increase the selling income.

Generally, the planning problem of chemical plants is based on the cooperative game, where players are the chemical plants. For chemical plant $i$, decision variables are the type of processing equipment, $\textit{\textbf{x}}_\textit{\textbf{i}}=\left\{x_i^n\right\},\forall n$, hydrogen processing amount $q_{i,t}^{pr}$ and transport route $u_{i,j}$ of chemical plants in the coalition. Payoffs are described as \eqref{eq:constraint16} and \eqref{eq:constraint21} for self-sufficient chemical plants and coalitions respectively.

Note that in the planning stage, the optimal solution $q_{i,t}^{pr}$ is to roughly estimate operation cost under different transportation mode and processing equipment type decisions, thus helping the decision of transport route $u_{i,j}$. Therefore, the solution of $q_{i,t}^{pr}$ here neglects the influence of hydrogen price variations. Actual hydrogen processing quantity sequence $q_{i,t}^{pr}$ will be obtained by equilibrium analysis in the scheduling stage.

\subsubsection{Stackelberg game in the scheduling stage}
The problem in the scheduling is the hydrogen transaction problem between the retailer and the suppliers. After formulating transport route decisions of suppliers as a cooperative game, the interaction between the salt cavern and multiple chemical plants is formulated as a Stackelberg game, where the salt cavern is the leader, whose strategy is the TOU hydrogen price, and chemical plants are followers, whose strategies are hourly transaction. 

At this stage, the retailer’s and suppliers’ problem can be formulated as a bilevel optimization. The retailer determines the hydrogen price sequence $v_t$ in the upper level, and the suppliers decide their optimal transaction pattern $q_{i,t}^{trans}$ in the lower level, with respect to the hydrogen price sequence $v_t$. The optimal transaction pattern $q_{i,t}^{trans}$ would in turn influences hydrogen price sequence $v_t$ determined by the retailer in the upper level. Assume that the information of each chemical plant, such as transit transport routes, processing equipment type and by-product hydrogen generation quantities, are accessible to the salt cavern. Therefore, the optimal solution of $q_{i,t}^{trans}$ can be predicted by the salt cavern under any given hydrogen price sequence $v_t$. The suppliers’ dispatching problem \eqref{eq:constraint3}-\eqref{eq:constraint21} can be regarded as constraints of the retailer’s pricing problem.

According to the analysis above, the interactions between the salt cavern and the chemical plants constitute a Stackelberg competition. In this competition, the salt cavern is the leader, whose strategy is the TOU hydrogen price sequence. Chemical plants are the followers, whose strategy is the hourly hydrogen transaction quantity. The leader’s pricing problem maximizes its profit, subject to the bounds of hydrogen price (Eq.\eqref{eq:constraint2}) and maximal injection rate (Eq.\eqref{eq:constraint3}). The followers’ scheduling problem maximizes individual profits or coalition profits, subject to constraints given in \eqref{eq:constraint9}-\eqref{eq:constraint18}.
\subsection{Solution of the Problem}
In this section, we introduce the solution of the game formulation of the by-product hydrogen supply chain. 

Tractable reformulations of the suppliers’ problem are made to efficiently calculate the equilibrium in the lower level for both the planning and scheduling problems. Specifically, for the suppliers’ problem in both stages, the objective of each individual player (or coalition) is irrelevant to the strategies of other individual players (or coalitions), while the strategy set is influenced by the strategies of other individual players (or coalitions). According to the potential game theory, the suppliers' problem can be regarded as a potential game. The sum of the objectives of each individual player (or coalition) is the potential function. Besides, the pure-strategy equilibrium exists in the transport route planning problem of the suppliers since there exists at least one pure-strategy equilibrium in an infinite potential game. Thus, the suppliers’ problem is formulated as a potential game that can be solved as an optimization problem.

After the reformulation of the suppliers' problem, the planning stage problem is reformulated to a mixed integer nonlinear program (MINLP) with ${\textit{\textbf{x}}_\textit{\textbf{i}},u_{i,j}, q_{i,t}^{pr},N_i^{cars}},\forall i\in{1,\ldots I}$ as decision variables, \eqref{eq:constraint22} as the objective and \eqref{eq:constraint4}-\eqref{eq:constraint15} as constraints. Commercial solvers such as Baron can be used to solve the problem. The solved optimal strategy $\textit{\textbf{x}}_\textit{\textbf{i}}$ and $u_{i,j}$ will be adopted at the scheduling stage.
\setlength{\abovedisplayskip}{3pt}
\begin{multline}
\max~~\pi_{F}~ = ~\sum_{i=1}^{I}{{\sum_{t = 1}^{T}\left( p_{t}q_{i,t}^{trans}u_{i,I + 1} - C_{O}^{i} - C_{T}^{i} \right)}}\\{-C_{INV1}^{i} - C_{INV2}^{i}}  \label{eq:constraint22}
\end{multline}

To solve the bi-level problem at the scheduling stage, Genetic Algorithm (GA) is adopted. First, for the salt cavern in the upper level, pieces of hydrogen price sequences are generated and regarded as individuals. Second, to acquire the fitness of each individual, the suppliers’ scheduling problems in the lower level are solved. Since the transit transport routes $\textit{\textbf{x}}_\textit{\textbf{i}}$ and the processing equipment type $u_{i,j}$ are known at the scheduling stage, the suppliers’ problem becomes a mixed-integer linear program (MILP), which can be solved efficiently by off-the-shelf commercial solvers. Thus, daily profits of the salt cavern, considering the best response of the suppliers, can thus be calculated and regarded as finesses for given price sequences. 

\section{Case Study}
To validate the effectiveness of the proposed model and algorithm, numeric experiments on a by-product hydrogen supply chain composed of three chemical plants and a salt cavern are carried out. All of the following tests are conducted on PCs with Intel Xeon W-2255 processor, 3.70 GHz primary frequency, and 128GB memory. CPLEX 2.16 is used to solve related MILP problems.

\subsection{System Configuration}
Scenario parameters of the envisioned by-product hydrogen supply chain are given in Table \ref{tab:Scenario parameters}. $Q_{i,t}$ are hydrogen generation sequences of a typical day produced by a Gaussian distribution with a mean value of 1000 for the 1st chemical plant (1500 for the 2nd and 3000 for the 3rd) and a variance of 100. Moreover, in the envisioned by-product hydrogen supply chain, $\boldsymbol{Q_{pr}}$ are a vector consisting of 1200, 2000, 4000 and 8000, the first two and the last two of which are the compressor capacity and liquefier capacity to choose from, respectively. Parameters of different processing equipment and transportation modes refer to \cite{HAN20125328} and \cite{Argonne2021} and are given in Table \ref{tab:Parameters of hydrogen transportation}. $\boldsymbol{K_{1}}$ are a vector consisting of 774.29, 126612, 18977.17 and 34757.99, corresponding to each element in $\boldsymbol{Q_{pr}}$. Note that the time scale involved in the problem is one day. Initial investment costs of the liquefier, the compressor, and the transportation vehicles are converted into daily investment costs with a discount rate. The operation cost of a tube trailer (or a tanker truck) in each period includes fuel price, driver wage, and maintenance expenses. 

\begin{table}[h]
\centering
\caption{Scenario parameters of the by-product hydrogen supply chain}
% \captionsetup{font={footnotesize}}
% \resizebox{0.4\textwidth}{!}{
\label{tab:Scenario parameters}
\footnotesize
\begin{tabular}{lllll}
\hline\toprule
\multicolumn{5}{l}{Parameters}                                                           \\ \hline
$I$                      & \multicolumn{2}{l}{3}  & $N_\mathcal{D}$     & 2                             \\
$T$                      & \multicolumn{2}{l}{12} & $T_{a}$     & {[}0,0,0,4;0,0,0,4;0,0,0,4{]} \\
\multicolumn{1}{c}{$p_{o}$} & \multicolumn{2}{l}{15} & $\underline{p}_{t},\overline{p}_{t}$  & 5/13                          \\
$N_\mathcal{C}$                     & \multicolumn{2}{l}{2}  & $Q_{trans}$ & 9000                          \\ \hline
\end{tabular}
% }
\end{table}

\begin{table}[]
\centering
\caption{Parameters of hydrogen transportation} \label{tab:Parameters of hydrogen transportation}
\footnotesize
\begin{tabular}{lllllll}
\hline\toprule
\multicolumn{7}{l}{Parameters}                                                                             \\ \hline
$Q_\mathcal{C}$                        & \multicolumn{4}{l}{200}          & $K_{3}(\$/h)$       & {[}0,0,0,4;0,0,0,4;0,0,0,4{]} \\
$Q_\mathcal{D}$                        & \multicolumn{4}{l}{4000}         & $\beta_{L1}$ & 5/13                          \\
\multicolumn{1}{c}{$\gamma_{c}/\gamma_{d}(kwh/kg)$} & \multicolumn{4}{l}{1/8.18}       & $\beta_{L2}$ & 9000                          \\
$K_{2}^c/K_{2}^d(\$)$                      & \multicolumn{4}{l}{82.20/219.18} &          & \multicolumn{1}{c}{}          \\ \hline
\end{tabular}
\end{table}
\subsubsection{Equilibrium of possible coalition structures of the suppliers}
With three chemical plants, there are five possible coalition structures: no cooperation, cooperation between two players with the third being self-sufficient (there are three ways this could occur) and complete cooperation among all the three chemical plants. The benefits of individual participants or coalitions are shown in Table \ref{tab:Participants/alliance}, in which $M$ represents the benefit, and the benefit of each chemical plant and the sum of them are denoted by $M_{1},M_{2},M_{3}$ and $M_{total}$ respectively. ‘\{\}’ indicates a cooperation, and the chemical plant serving as the transit hub is marked by a ‘*’. 

% Please add the following required packages to your document preamble:
% \usepackage{multirow}
\begin{table}[]
\centering
\caption{Participants/alliance optimal income under non-cooperative and cooperative game models} \label{tab:Participants/alliance}
\footnotesize
\begin{tabular}{llll}
\hline\toprule
\multirow{2}{*}{Number} &
  \multirow{2}{*}{\begin{tabular}[c]{@{}l@{}}Coalition\\ structure\end{tabular}} &
  \multicolumn{2}{c}{Profits(\$/day)} \\ \cline{3-4} 
  &                & \begin{tabular}[c]{@{}l@{}}Individual or\\ a coalition\end{tabular}        & $M_{total}$ \\ \hline
1 &
  \{1\},\{2\},\{3\} &
  \begin{tabular}[c]{@{}l@{}}$M_{1} = 54052$\\ $M_{2} = 81060$\\ $M_{3} = 236814$\end{tabular} &
  371926 \\
2 & \{1,2*\},\{3\} & \begin{tabular}[c]{@{}l@{}}$M_{\{1,2\}}$ = 170589\\ $M_3$ = 236814\end{tabular} & 407403   \\
3 & \{1,3*\},\{2\} & \begin{tabular}[c]{@{}l@{}}$M_{\{1,3\}}$ = 286531\\ $M_2$ = 107868\end{tabular} & 394399   \\
4 & \{1\},\{2,3*\} & \begin{tabular}[c]{@{}l@{}}$M_1$ = 53562\\ $M_{\{2,3\}}$ = 323154\end{tabular}  & 376716   \\
5 & \{1,2,3*\}     & $M_{\{1,2,3\}}$ = 383925                                                       & 383925   \\ \hline
\end{tabular}
\end{table}

It can be analyzed from Table \ref{tab:Participants/alliance} that:

i)	In the 1st coalition structure with no cooperation at all, the total benefit of the three chemical plants is the lowest among all coalition structures, indicating a potential collective payoff gained by forming coalitions between chemical plants. 

ii)  In the 3rd coalition structure, the benefit of the coalition $\{1, 3^{*}\}$ denoted as $M_{\{1,3^{*}\}}$ equals to 286531 and is lower than the sum of benefits that they could get on their own, which is calculated as $M_{1}+M_{3}=290866$, violating collective rationality.

iii)  In the 5th coalition structure, although collective benefit is higher than the sum of benefits each coalition member could get on their own, the total benefit of the 5th coalition structure $M_{total}\{1,2,3^{*}\}$ is lower than that of the 2nd coalition structure $M_{total}(\{1,2^{*}\},\{3\})$. Therefore, the grand coalition is not stable since there is a preferred alternative. The analysis of the 4th coalition structure is analogous.

iv)  In the 2nd coalition structure, $M_{\{1,2^{*}\}}$, the benefit of the coalition $\{1,2^{*}\}$, equals to 170589 and is higher than the sum of benefits they could get on their own, which satisfies $M_{1}+M_{2}=135112$. Moreover, the total benefit of the 2nd coalition structure is the highest among the five possible structures, so there exists no preferred alternatives. Therefore, the coalition of the chemical plants $\{1,2^{*}\}$ is stable.

The insights provided by different coalition structures above is that for several chemical plants in proximity to each other, those chemical plants with low or medium generation scale (chemical plant 1 and 2 in our case) tends to form a coalition, and to compete with those with larger generation scale.

In order to realize a fair imputation of the collective payoff of chemical plants $\{1,2^{*}\}$, the Shapley value is adopted. The allocation result is {71790.5,98798.5}\$, which is higher than the benefit they could get on their own, which are \{\$54052, \$81060\}. The coalition between chemical plant 1 and 2 increase their profits by 24.7\% and 18.0\% respectively.

\subsection{Equilibrium of Hydrogen Pricing and Scheduling}

In this case, the fixed price at which consumers purchase is set as 15 \$/kg. The equilibrium of the buying price offered by the salt cavern $p_t$ and the hydrogen transaction quantity $q_{i,t}^{pr}$ are illustrated in Fig.5. The minimal price takes value at its lower bound 5\$/kg, and the maximal value is 11.9 \$/kg .
\begin{figure}[] %可选参数 h t b p,代表允许图片出现的位置,h表示此处附近,t表示顶部,b表示底部,p表示单独一页,H表示固定此处
    \centering
    \includegraphics[width=7cm]{fig/Fig_5._Hydrogen_price_of_salt_cavern_and_transaction_quantity_of_chemical_plant.png}
    \caption{Hydrogen price of salt cavern and transaction quantity of chemical plant} \label{fig:Hydrogen price of salt cavern and transaction quantity of chemical plant}
\end{figure}
It can be observed from Fig.\ref{fig:Hydrogen price of salt cavern and transaction quantity of chemical plant} that the variation trend of the hydrogen transaction quantity goes with the buying price. The higher the buying price, the higher the transaction quantity. This can be attributed to the storage capacity of chemical plants, which can temporarily store by-product hydrogen in low-pressure storage tanks or tube trailers (or tanker trucks) before filled to maximal capacity. Therefore, the chemical plants can choose to sell hydrogen at a higher price.

Moreover, due to the influence of the TOU electricity price, the operating cost of the processing equipment fluctuates. The TOU electricity price and the equilibrium of the total processing quantity are plotted in Fig.\ref{fig:Time of use electricity price and processing mass of chemical plant}.

\begin{figure}[] %可选参数 h t b p,代表允许图片出现的位置,h表示此处附近,t表示顶部,b表示底部,p表示单独一页,H表示固定此处
    \centering
    \includegraphics[width=7.5cm]{fig/Fig_6._Time_of_use_electricity_price_and_processing_mass_of_chemical_plant.png}
    \caption{Time of use electricity price and processing mass of chemical plant} \label{fig:Time of use electricity price and processing mass of chemical plant}
\end{figure}

It can be observed from Fig.\ref{fig:Time of use electricity price and processing mass of chemical plant} that the variation trend of the hydrogen processing quantity and the TOU electricity price go oppositely. This is because chemical plants tend to process hydrogen when the electricity price is low, thus reducing the processing cost of hydrogen.

According to the above results, it can be noted that the equilibrium of salt cave pricing encourages chemical plants to process and trade hydrogen when the electricity price is lower. As a result, the salt cavern can purchase hydrogen with lower processing cost, thus reducing the purchase cost of hydrogen per unit. For chemical plants, the hydrogen price is higher during 1-2 periods after periods with lower electricity prices than in other periods, thus reducing the hydrogen processing cost.

The result of profits and total transaction quantities are plotted in Fig.\ref{fig:The result of profits and total transaction quantities with time-invariant hydrogen price} considering different fixed prices. The optimal price offered by the salt cavern is about 9\$/kg, and its profit is \$287884.8 for a day. However, the profit of the salt cavern reaches to \$343947.16 at the optimal TOU hydrogen price. Hence, a TOU hydrogen price strategy for the salt cavern increases its profit by 19.5\%.

\begin{figure}[] %可选参数 h t b p,代表允许图片出现的位置,h表示此处附近,t表示顶部,b表示底部,p表示单独一页,H表示固定此处
    \centering
    \includegraphics[width=7.5cm]{fig/Fig.7_The_result_of_profits_and_total_transaction_quantities_with_time-invariant_hydrogen_price.png}
    \caption{The result of profits and total transaction quantities with time-invariant hydrogen price}\label{fig:The result of profits and total transaction quantities with time-invariant hydrogen price}
\end{figure}

Generally, the equilibrium of the Stackelberg game between the salt cavern and the chemical plants benefits all the players. It also indicates the positive response of the salt cavern and chemical plants to TOU electricity price, and reflects the role of chemical plants in peak shaving and valley filling, which benefits the safe and stable operation of power grid.

\subsection{Sensitivity Analysis}
\subsubsection{Impact of per period transportation operation cost}

The reduction in operation cost of a tube trailer (or a tanker truck) per period $K_{3}$ reduces the transport cost, thus bringing down the collective payoff brought by coalitions of chemical plants. Based on the first assumption in section \ref{subsubsection: first-stage problem}, each coalition must take one of them as a transit hub, and two chemical plants destined for the salt cavern lack the motivation to form a coalition. Consequently, the collective payoff declines as the transport cost reduces, until collective rationality no longer holds when the benefits of the coalition are less than the sum of benefits each individual could get on their own. As shown in Fig.\ref{fig:Impact of running cost of single vehicle of single period on the profit of chemical plant 1 and 2}, when $K_{3}$ decreases from \$390 to \$382, the sum of benefits of chemical plant 1 and 2 under the equilibrium of the 1st and 2nd coalition structure, denoted by $M_{total}^{\{1,2\}},M_{total}^{\{1\},\{2\}}$ respectively, gradually increases. 

\begin{figure}[] %可选参数 h t b p,代表允许图片出现的位置,h表示此处附近,t表示顶部,b表示底部,p表示单独一页,H表示固定此处
    \centering
    \includegraphics[width=7cm]{fig/Fig_8._Impact_of_running_cost_of_single_vehicle_of_single_period_on_the_profit_of_chemical_plant_1_and_2.png}
    \caption{The result of profits and total transaction quantities with time-invariant hydrogen price}\label{fig:Impact of running cost of single vehicle of single period on the profit of chemical plant 1 and 2}
\end{figure}

As illustrated in Fig.\ref{fig:Impact of running cost of single vehicle of single period on the profit of chemical plant 1 and 2}, the coalition benefit is more sensitive to $K_3$ than individual benefits. When $K_3$ decreases to about \$386, the coalition $\{1,2^{*}\}$ no longer bring additional benefits to individuals, resulting in a breakdown of the coalition. 
\subsubsection{Impact of maximal injection rate of the salt cavern}

The maximal injection rate $Q_{trans}$ of the salt cavern directly limits the total transaction quantity per period between the salt cavern and the chemical plants. Table \ref{tab:Individual income} demonstrates the impact of $Q_{trans}$ to the equilibrium of the second-stage problem.

\begin{table}[]
\centering
\caption{Individual income of the equilibrium under different maximum transportation quality of salt cavern gas pipeline in single period} \label{tab:Individual income}
\footnotesize
\begin{tabular}{lllll}
\hline\toprule
\multirow{3}{*}{$Q_{trans}$} &
  \multirow{3}{*}{$M_{total}^{\{1,2\}}$/kg} &
  \multirow{3}{*}{$M_{total}^{\{3\}}$/kg} &
  \multirow{3}{*}{\begin{tabular}[c]{@{}l@{}}$M_{total}$\\(chemical\\ plants)/\$\end{tabular}} &
  \multirow{3}{*}{\begin{tabular}[c]{@{}l@{}}$M_{total}$\\(the salt\\\ cavern)/\$\end{tabular}} \\
      &                              &          &          &           \\
      &                              &          &          &           \\ \hline
12000 & 23103.96                     & 21387.07 & 44491.03 & 342814.83 \\
9000  & \multicolumn{1}{c}{26203.97} & 22762.82 & 48966.80 & 325379.79 \\
6000  & 10528.29                     & 1139.11  & 11667.40 & 278396.59 \\ \hline
\end{tabular}
\end{table}

It can be analyzed from Table \ref{tab:Individual income} that $Q_{trans}$ has different impacts on the participants: the daily income of chemical plants does not necessarily increase with the increase of $Q_{trans}$, whereas the daily income of the salt cavern increases with the increase of $Q_{trans}$. Therefore, the salt cavern will be motivated to determine an appropriate $Q_{trans}$ according to the generation scale of by-product hydrogen of the chemical plants so as to increase individual benefits.

\section{Conclusion}
This paper proposes an equilibrium model of a by-product hydrogen market with the salt cavern as the retailer and chemical plants as the suppliers. A business model for large-scale storage to acquire by-product hydrogen from chemical plants and sell them to end-users is established for the first time. The decision-making process of each stakeholder, i.e., chemical plants and the salt cavern, is investigated and mathematically modeled considering different transportation modes, locations of chemical plants and TOU electricity price. To consider the individual rationality of each stakeholder, the by-product hydrogen market is formulated as games. The transport route planning problem between multiple chemical plants is formulated as a cooperative game. The hydrogen transaction problem between the salt cavern and chemical plants is formulated as a Stackelberg game. Numeric experiments on a by-product hydrogen supply chain composed of three chemical plants and a salt cavern are carried out. The results show that a coalition between chemical plants potentially increases their profits. Moreover, the adoption of TOU hydrogen price in a Stackelberg formulation also increases the profit of the salt cavern. The proposed business model and the optimization of the by-product hydrogen supply chain management not only presents a new revenue stream for both chemical plants and salt caverns but increases resource efficiency and accelerates energy conversion.





% \section*{Appendix A}
% \vspace{-0.2cm}
% \section*{Proof of Proposition 1}



% \section*{Appendix B}
% \vspace{-0.2cm}
% \section*{Proof of Proposition 2}



\bibliographystyle{IEEEtran}
\bibliography{ref}
\end{document}






\end{document}
\endinput
