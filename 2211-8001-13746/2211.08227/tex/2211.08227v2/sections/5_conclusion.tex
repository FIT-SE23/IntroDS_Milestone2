\section{Conclusion}
\label{sec:conclusion}
This paper presents Perona, a novel approach to holistic infrastructure fingerprinting for improving resource efficiency of big data analytics.
Perona automates the assessment of target infrastructures via standardized sets and configurations of benchmarking tools and subsequent data preprocessing.
Our approach then utilizes representation learning and graph modeling for robust information extraction and detection of potential performance degradation. 
Learned representations can be directly used within existing resource configuration approaches, where they allow for similar or even superior optimization results.
Our evaluation on gathered data and in conjunction with existing methods shows that Perona successfully extracts and preserves relevant information, which in turn can be used in downstream optimization tasks. 

In the future, we plan to provide a unified framework for the management of data processing jobs by integrating Perona with recent resource configuration optimization methods.
