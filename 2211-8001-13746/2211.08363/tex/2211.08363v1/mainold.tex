\documentclass[aps, twocolumn, amsfonts, amsmath, amssymb, superscriptaddress, showkeys]{revtex4-1}

\usepackage{graphicx}

%\usepackage{caption,subcaption}
\usepackage[normalem]{ulem}
\usepackage{physics}
\usepackage{tikz,lipsum,lmodern}
\usepackage[export]{adjustbox}
\usepackage{soul}


\usepackage[unicode=true,pdfusetitle,
 bookmarks=true,bookmarksnumbered=false,bookmarksopen=false,
 breaklinks=false,pdfborder={0 0 0},backref=false,colorlinks=true,citecolor=blue,urlcolor=violet 
]{hyperref}
\usepackage{xcolor}
\newcommand{\orcid}[1]{\href{https://orcid.org/#1}{\includegraphics[width=10pt]{orcid}}}
\newcommand{\eop}{\hfill{\rule{2.2mm}{2.2mm}}}
\newcommand{\inp}[2]{\left( {#1} ,\,{#2} \right)}
\newcommand{\dip}[2]{\left< {#1} ,\,{#2} \right>}


\begin{document}
\title[]{Emergence of Classicality in Stern-Gerlach Experiment via Self-Gravity}
\author{Sourav Kesharee Sahoo\orcid{0000-0003-1812-0417}}
\email{sourav.sahoo1490@gmail.com.} 
\affiliation{Department of Physics, BITS-Pilani K K Birla Goa Campus, Goa-403726, India.}
\author{Radhika Vathsan\orcid{0000-0001-5892-9275}}
\email{radhika@goa.bits-pilani.ac.in} 
\affiliation{Department of Physics, BITS-Pilani K K Birla Goa Campus, Goa-403726, India.}

\author{Tabish Qureshi\orcid{0000-0002-8452-1078}}
\email{tabish@ctp-jamia.res.in}
\affiliation{Center for Theoretical Physics, Jamia Millia Islamia, New Delhi 110025.}
\date{\today}

\begin{abstract}
Emergence of classicality from quantum mechanics is a hotly debated topic, where several approaches have been suggested but none provides a satisfactory picture. In the present work, the Schr\"odinger–Newton equation has been used to study the role of self-gravitational interaction in a Stern-Gerlach experiment. For a small mass spin-1/2 particle, the results of quantum state evolution show a superposition of two trajectories virtually identical to those seen in a conventional Stern-Gerlach experiment, in agreement with standard quantum theory. For a large enough mass, the results show only a single trajectory of the particle, indicative of an \emph{apparently} unquantized spin. This single trajectory of the particle follows the expected classical path of a particle carrying a \emph{classical magnetic moment}. This classicality emerges simply with the increasing mass of
the particle, due to self-gravitational interaction. In contrast, decoherence in a Stern-Gerlach experiment leads to a mixed state density matrix of two trajectories. Decoherence is thus, unable to explain the classically expected path of the particle.
\end{abstract}
\keywords{ Stern-Gerlach experiment, Schr\"odinger-Newton equation, Self-gravitational interaction, Semi-classical gravity.
}
\maketitle
\section{Introduction}

Quantum theory, in the modern perspective, is an extremely successful theory. Even those aspects of the theory which made eminent physicists like Einstein uncomfortable, have been experimentally verified \cite{Aspect_2015}. However, there still remain two issues, which are not really independent, which are nowhere close to resolution. One is the so-called measurement problem, namely explaining how a particular outcome emerges from a measurement process \cite{Schlosshauer2011}. Quantum theory only provides a prescription to calculate the probabilities associated with the various outcomes. It provides no mechanism for a single outcome in a measurement process, a process which appears to be non-unitary in character. The other is the issue of emergence of the classical world from quantum theory \cite{schlosshauer2007decoherence,joos2013decoherence}. Quantum theory, in principle, allows superposition of two or many states. On the other hand, in our familiar classical world, superpositions of macroscopically distinct states is never observed. For example, a pebble is never seen to be in a superposition of being two well separated locations. If one believes that quantum theory is the more fundamental theory, the classical world should emerge from quantum theory, in some limit which we may call the classical limit. Again, quantum theory provides no such mechanism. While there is absolutely no convergence on how the measurement problem could be solved, a lot of people have come believe that decoherence is a very plausible mechanism for the emergence of classicality from quantum mechanics \cite{joos2013decoherence,schlosshauer2007decoherence}. Decoherence emphasizes the  unavoidable role the environment plays in the evolution of a quantum system. It stresses that real systems, especially macroscopic ones, can never be shielded from the environmental interactions. It is demonstrated that even the weakest of interactions is enough to destroy the quantum coherence of a system, and drives it into a \emph{mixed state}. The various components of this mixed state are associated with different outcomes with specific probabilities.  Although the use of probabilities in an essential way makes many people uncomfortable, decoherence has emerged as a popular mechanism for the emergence of classical world from quantum theory. However, we will argue in the following that there are certain simple situations in which decoherence offers no explanation of the classical behavior. 
\begin{figure}[t]         
\includegraphics[width=\columnwidth]{sgsetup.pdf}
         \caption{Schematic representation of a typical Stern–Gerlach setup: atoms travel through an inhomogeneous magnetic field and are deflected up or down depending on their spin. (CC BY-NC; Ümit Kaya via LibreTexts)}
         \label{sgsetup}
\end{figure}

Consider a typical Stern-Gerlach experiment where a massive spin-1/2 particle travels towards a screen, and passes through an inhomogeneous magnetic field on the way (see Fig. \ref{sgsetup}). The spin is assumed to be in an initial state $\cos\frac{\theta}{2}|\uparrow\rangle + \sin\frac{\theta}{2}|\downarrow\rangle$, 
$|\uparrow\rangle,|\downarrow\rangle$ being the eigen-states of the z component of the spin, $S_z$. According to quantum mechanics, the particle is driven into a superposition of two wave-packets, whose time evolution correspond to two paths which lead to two different positions on the screen, corresponding to the spin states $|\uparrow\rangle,|\downarrow\rangle$. On hitting the screen, which is believed to be a classical object, the state of the particle reduces to one of the two positions on the screen, with probabilities $\cos^2\frac{\theta}{2}$ and $\sin^2\frac{\theta}{2}$, respectively. Now suppose that the particle is massive enough to behave like a classical particle. How does decoherence explain the emergence of classicality of the particle in this experiment? The ``reduced" density matrix of the particle, obtained after tracing out the environment, has an almost diagonal form which has two components corresponding to two spatially localized distributions, with weights $\cos^2\frac{\theta}{2}$ and $\sin^2\frac{\theta}{2}$, respectively \cite{Venugopalan_Kumar_Ghosh_1995}. The time evolution of this density matrix corresponds to two trajectories. Thus, according to decoherence the emergent classical behavior will show the particle following two trajectories, corresponding to spin states $|\uparrow\rangle,|\downarrow\rangle$, with different probabilities. 

Now, what does one expect from the classical dynamics of particle carrying a magnetic moment, traveling through an in-homogeneous  magnetic field? The quantum state $\cos\frac{\theta}{2}|\uparrow\rangle + \sin\frac{\theta}{2}|\downarrow\rangle$ is an eigen-state of the spin component 
$S_{\theta}=\vec{S}\cdot \vec{n}$, where $\vec{n}$ is a unit vector making an angle $\theta$ with the z-axis. Thus the classical angular momentum can be assumed to be pointing in a direction making an angle $\theta$ with the z-axis. A simple classical analysis shows that the particle will experience a uniform force and its trajectory will be somewhere in between the two trajectories in the quantum picture. This particle will hit the screen somewhere in between the two spots corresponding to the spin states $|\uparrow\rangle$ and $|\downarrow\rangle$. In fact, there is a continuous set of points on the screen, between the spots mentioned above, where the particle will land, depending on the value of $\theta$. It is clear that decoherence will never lead to any trajectory other than the two extreme top and bottom ones. So decoherence is unable to explain the behaviour of a \emph{classical} particle passing through a Stern-Gerlach setup. One then needs to look for alternative approaches for emergence of classicality.

Roger Penrose \cite{penrose1996gravity} proposed 
that gravity could play a role in the emergence of classicality in
massive particles. He argued that macroscopic
gravity could be the driving force behind the reduction of the wave function as the wave packet responds to its own gravity.  He used the Schrödinger-Newton (S-N) equation, earlier introduced by Diosi \cite{DIOSI1984199}, to explore the quantum-state reduction phenomenon \cite{penrose1996gravity,penrose2014gravitization}. Subsequently, several authors investigated the effect of gravity and self-gravity on quantum systems in various ways \cite{colella1975observation,grossardt2016effects,grossardt2016approximations,singh2015possible,yang2013macroscopic,kumar2000single}. 
Could self-gravity of a massive spin 1/2 particle lead to emergence of classical behaviour in the Stern-Gerlach experiment?



\section{S-N equation for a particle emerging from a stern-gerlach machine}

\subsection{The Schr\"odinger-Newton equation}

The seeds of the S-N equation came from semi-classical gravity independently considered by M\"oller \cite{moller1962theories} and Rosenfeld \cite{rosenfeld1963quantization}. In this approach, quantized matter is assumed to be coupled to the classical gravitational field \cite{mattingly2005quantum,kibble1981semi,bahrami2014schrodinger}, and the Einstein field equations get modified as, 
\begin{equation}
          R_{\mu \nu} + \frac{1}{2} g_{\mu \nu} R = \frac{8 \pi G}{c^4} \langle \Psi | \hat{T}_{\mu \nu}| \Psi \rangle \label{e2}
          \end{equation}
where the term on the right hand side is the expectation value of the energy-momentum tensor with respect to the quantum state $|\Psi\rangle$ of matter.  
The question whether it is really necessary to quantize gravity or would a semi-classical treatment suffice, was addressed by studying this semi-classical modification  \cite{eppley1977necessity,page1981indirect}. 
This modification to the Einstein field equation then naturally leads to the Schr\"odinger-Newton equation \cite{bahrami2014schrodinger,van2011schrodinger,salzman2005investigation,giulini2012schrodinger},
\vphantom{ Could we replace this}
  \begin{equation}
          \Bigg[ - \frac{\hslash^2}{2 m} \nabla^2 - G m^2 \int \frac{|\Psi (r',t)|^2}{|r - r'|} d^3 r'\Bigg] \Psi(r,t) = i \hslash \frac{\partial \Psi (r,t)}{\partial t}. \label{e1}
      \end{equation}
 which introduces an additional potential
  \begin{equation} V_{G} = - G m^2 \int \frac{|\Psi (r',t)|^2}{|r - r'|} d^3 r' .
  \label{VG}
  \end{equation}
 \vphantom{ We could, but don't you think the first time we talk of the S-N eqn, we should show the full form of it? Maybe it can followed with, where the additional potential .... represents the gravitational self-interaction... }
 This is no ordinary potential, as it depends on the wavefunction itself, which makes the Schr\"odinger-Newton equation basically a \emph{non-linear} modification
of the Schr\"odinger equation. The non-linearity breaks the unitarity of  Schr\"odinger evolution, and opens up the potentialities of all those effects which were not possible because of the linearity of quantum dynamics. One then hopes to see a dynamical reduction of the wavefunction, what is often called the wavefunction collapse. However, this breaking of linearity, which is considered sacred in quantum mechanics, also has the potentiality to bring up new problems  \cite{anastopoulos2014problems}. However, the hope behind this approach is that at the scales at which quantum mechanics has been successfully tested, the dynamics will be linear for all practical purposes, and the non-linearity will show up only when one approaches the classical limit. There have also
been several  other approaches in which non-linearity has been introduced in order to obtain wavefunction collapse \cite{bassi2013models,ghirardi1990markov,adler2007collapse}. However, an elegant aspect of the  Schr\"odinger-Newton equation is that there is no tunable parameter that has been introduced, and the scale at which classicality might appear, should naturally emerge from this equation.

\subsection{Formulation of the Stern-Gerlach problem}

\vphantom{
Stern-Gerlach experiment was first performed with a beam of unpolarized silver atoms passing through an in-homogeneous magnetic field. Because of the intrinsic angular momentum that we now know as spin the atoms got deflected in opposite directions. Quantum mechanics from its beginning has sown several aspects of it but spin and entanglement among them have no classical analogue or counterpart. In this work we have shown the emergence of classicality from a stern-gerlach setup which is purely a spin dependant setup. We also observed Several other deviations from the usual predictions of a stern-gerlach experiment. So far the experiment is not prepared for extremely heavier particles and hence such anomalies might not have been come into picture if it is true. We have added the self-gravitational potential to the Schrodinger's equation which is well known as Schrodinger-Newton equation. The equation came into picture by Penrose and Diosi individually during their attempt to understand the role of gravity in wave-function collapse. Later by several authors it was shown that gravity does not make the wave function to collapse into one of its eigenstates. In our previous work we showed how there is an observable effect on the fringewidth because of the self-gravity potential. 
}

In a typical Stern-Gerlach experiment, a spin-1/2 particle travels along (say) y axis, as shown in Fig. \ref{sgsetup}. It passes through an inhomogeneous magnetic field. In order to satisfy Maxwell equations, we assume the field to be parallel to z axis, but inhomogeneous along the x axis. That particle travels along the y axis, and experiences a force in the x direction, depending on the eigenvalue of the z component of the spin. For simplicity the field is assumed to vary linearly with the $x$ position, such that it is given by  $\vec{B} = B_0x\hat{z}$.  The potential experienced by the particle, due to the magnetic field, is given by
\begin{equation}
 V_B(x) = -\frac{e\hbar}{2m_ec}\vec{B}\cdot\vec{\sigma}
     = -\frac{e\hbar}{2m_ec} x B_0\sigma_z
     = -\gamma x \sigma_z,
     \label{VB}
\end{equation}
where $\vec{\sigma}$ is the Pauli matrix corresponding to the particle spin, and $\gamma$ is a parameter denoting the effective strength of the magnetic field. The dynamics of the particle along the y axis is trivial, and just serves to translate the y position of the particle from the source to the screen, in a given time. Consequently we ignore the explicit dynamics of the particle in the y direction, and assume it to be traveling with a constant velocity of magnitude $v$ in the y direction. The y position of the particle, at any given time $t$, will be assumed to be $y = vt$. The quantum dynamics of the particle in the x direction is then given by one-dimensional Schr\"odinger equation of the particle, under the influence of the potential $V_B(x)$ given by (\ref{VB}).

In order to investigate the role of self-gravitational interaction on the dynamics of a particle of mass $m$, in a Stern-Gerlach experiment, our strategy is to consider its quantum dynamics with the potential given by (\ref{VB}), and an additional self-gravitational potential given by (). 
The resultant Schrodinger-Newton equation in one dimension is \cite{sahoo2022}

\begin{eqnarray}
          \Bigg[ - \frac{\hslash^2}{2 m} \frac{\partial^2}{\partial x^2} 
          -\gamma x \sigma_z
          - G m^2 \int \frac{|\Psi (x',t)|^2}{|x - x'|} d x'\Bigg] \Psi(x,t)
          \nonumber\\
          = i \hslash \frac{\partial \Psi (x,t)}{\partial t}, \label{e1}
      \end{eqnarray}
where the state $|\Psi\rangle$ of the particle now has a spatial part and a spin part:
$ |\Psi\rangle = |\chi\rangle\otimes|s\rangle$. The spatial part of the wavefunction is then a spinor with two components $\chi_{\pm}(x,t)$ corresponding to the two eigenstates of the z component of the spin $\sigma_z|\uparrow\rangle = |\uparrow\rangle,
\sigma_z|\downarrow\rangle = -|\downarrow\rangle$.

\vphantom{We analyze the effect of self gravity on the particles emerging from a stern-gerlach experimental setup. The stern-gerlach machine is oriented in the z-axis and the in-homogeneity is taken along the x-axis. The magnetic field we have considered here is $\vec{B}=B_0x \,\hat{z}.$ the reason behind considering such complicated magnetic field is to not mess around with Maxwell's equations, i.e, magnetic field is always divergence-less. A magnetic field that is linear in $x$ is a well-behaved magnetic field that could be produced in laboratory. Such magnetic field will cause the particles having spin to deflect in the x-axis ($\uparrow$ and $\downarrow$).  }


Due to the presence of the magnetic field, the up spin component of the wave function $\ket{\chi_+}$ will move along the up and the down spin component $\ket{\chi_-}$ will move along down driven by the magnetic field. 
In general this will lead to a state entangled in the position and the spin degrees of freedom, $\ket{\Psi(x,t)}=\alpha \ket{\chi_+(x,t)} \otimes \ket{\uparrow}+\beta \ket{\chi_-(x,t)}\otimes \ket{\downarrow}$. ($\alpha$, $\beta$ are the normalization constants)
In our problem, we can reduce the S-N equation in radial coordinates to a cartesian effective 1-D equation as we have shown in our previous work \cite{sahoo2022}.
The Schr\"odinger-Newton equation for the given problem with an added magnetic potential is,


\begin{equation}
\begin{pmatrix}
\frac{p^2}{2m}-\gamma x } + V_G & 0\\
0 & \frac{p^2}{2m}+\gamma x + V_G 
\end{pmatrix} \begin{pmatrix}\chi_+ \\ \chi_- \end{pmatrix} = i\hslash \begin{pmatrix} \Dot{\chi}_+ \\ \Dot{\chi}_-
\end{pmatrix}
\label{SGSN}
\end{equation}
%\begin{widetext}
\vphantom{ \begin{equation}
\begin{pmatrix}
\frac{p^2}{2m}-\frac{\gamma B_0 \hslash x}{2}+V_G(\Psi) & 0\\
0 & \frac{p^2}{2m}+\frac{\gamma B_0 \hslash x}{2}+V_G(\Psi) 
\end{pmatrix} \begin{pmatrix}\chi_+ \\ \chi_- \end{pmatrix} = i\hslash \begin{pmatrix} \Dot{\chi}_+ \\ \Dot{\chi}_-
\end{pmatrix}
\end{equation}}
%\end{widetext}
where $V_G(\Psi)$ is the self-gravity potential depending on $|\Psi\rangle$. The probability distribution of the particle is given by
\begin{eqnarray}
|\Psi (x,t)|^2 &=& |\alpha \ket{\chi_+(x,t)} \otimes \ket{\uparrow}  + \beta \ket{\chi_-(x,t)}\otimes \ket{\downarrow}|^2 \nonumber\\
&=& |\alpha|^2 |\chi_+(x,t)|^2 + |\beta|^2|\chi_-(x,t)|^2 ,
\end{eqnarray}
where the cross terms between $\chi_{\pm}(x,t)$  vanish due to the orthogonality of $\ket{\uparrow},\ket{\downarrow}$. Using (\ref{VG}), the above yields
\begin{align*}
 V_G(\Psi) &=V_G(\chi_+)+V_G(\chi_-)\nonumber \\
 &=- \frac{Gm^2}{2}\Bigg[ \int \frac{|\chi_+ (x',t)|^2}{|x-x'|} dx'+ \int \frac{|\chi_- (x',t)|^2}{|x-x'|} dx'\Bigg]\nonumber
\end{align*}
Eqn. (\ref{SGSN}) represents a set of two differential equation describing the time evolution of $\chi_{\pm}$.
The appearance of $V_G(\chi_-)$ in the $\chi_+$ equation and vice versa couples the two equations and we end up with two coupled non-linear integro-differential equations.

We introduce dimensionless variables
\[
\tilde{r}=r/\sigma_r,~~~
\tilde{m}=m/m_r~,~~~
\tilde{t}=t/t_r~ \]
using a length scale $\sigma_r$, and time and mass scales
\begin{equation}
t_r=\left(\frac{\sigma_r^5}{G \hslash}\right)^{\frac{1}{3}},~~~
m_r=\left(\frac{\hslash^2}{G \, {\sigma_r}}\right)^{\frac{1}{3}}.
\label{eq:scalefactors}
\end{equation}
%For the simulation we have fixed the value of $\sigma _r$ equals to $1.1806\times 10^{-5}$ \text{m}.\\
Consider equation(1),
\begin{widetext}
\begin{eqnarray}
           - \frac{\hslash^2}{2m} \frac{\partial^2 \chi_+}{\partial x^2} +\frac{gqB_0 \hslash x}{4m_e} \chi_+ - \frac{Gm^2}{2}\Bigg[ \int \frac{|\chi_+ (x',t)|^2}{|x-x'|} dx'+ \int \frac{|\chi_- (x',t)|^2}{|x-x'|} dx'\Bigg] \chi _+(x,t) = i \hslash \frac{\partial }{\partial t}\chi_+ (x,t) \nonumber
      \end{eqnarray}
      

 

\begin{eqnarray}
       \Rightarrow  - \frac{\hslash^2}{2m_r \tilde{m}\sigma_r^2} \frac{\partial^2 \tilde{\chi}_+}{\partial \tilde{x}^2} +\frac{gqB_0 \hslash \sigma_r \tilde{x}}{4m_r \tilde{m}} \tilde{\chi}_+ - \frac{Gm_r^2\tilde{m}^2}{2\sigma_r}\Bigg[ \int \frac{|\tilde{\chi}_+ (\tilde{x}',\tilde{t})|^2}{|\tilde{x}-\tilde{x}'|} d\tilde{x}'+ \int \frac{|\tilde{\chi}_- (\tilde{x}',\tilde{t})|^2}{|\tilde{x}-\tilde{x}'|} d\tilde{x}'\Bigg] \tilde{\chi} _+(\tilde{x}',\tilde{t}) = i \hslash \frac{\partial }{\partial t_r\tilde{t}}\tilde{\chi}_+ (\tilde{x},\tilde{t}) \nonumber
      \end{eqnarray}
      
\begin{eqnarray}
         \Rightarrow - \frac{\hslash t_r}{2m_r \tilde{m}\sigma_r^2} \frac{\partial^2 \tilde{\chi}_+}{\partial \tilde{x}^2} +\frac{gqB_0 t_r \sigma_r \tilde{x}}{4m_r \tilde{m}} \tilde{\chi}_+ - \frac{Gm_r^2\tilde{m}^2t_r}{2\sigma_r \hslash}\Bigg[ \int \frac{|\tilde{\chi}_+ (\tilde{x}',\tilde{t})|^2}{|\tilde{x}-\tilde{x}'|} d\tilde{x}'+ \int \frac{|\tilde{\chi}_- (\tilde{x}',\tilde{t})|^2}{|\tilde{x}-\tilde{x}'|} d\tilde{x}'\Bigg] \tilde{\chi} _+(\tilde{x},\tilde{t}) = i  \frac{\partial }{\partial \tilde{t}}\tilde{\chi}_+ (\tilde{x},\tilde{t}) \nonumber
      \end{eqnarray}    
\end{widetext}
The coefficient of $\frac{\partial^2 \tilde{\chi}_+}{\partial \tilde{x}^2}$in the above expression is, 
$$- \frac{\hslash t_r}{2m_r \tilde{m}\sigma_r^2} =\frac{-\hslash}{2m_r\tilde{m}\sigma_r^2}\times \frac{\sigma^{\frac{5}{3}}}{G^{\frac{1}{3}}\hslash^{\frac{1}{3}}}=-\frac{1}{2\tilde{m}}$$

The coefficient of $\tilde{\chi}(\tilde{x},\tilde{t})$ is, 

$$\frac{gqB_0 t_r \sigma_r \tilde{x}}{4m_r \tilde{m}}=\frac{gqB_0}{4} \frac{(\frac{\sigma_r^5}{G\hslash})^{\frac{1}{3}}\,\sigma_r}{(\frac{\hslash^2}{G\sigma_r})^{\frac{1}{3}}}\frac{\tilde{x}}{\tilde{m}}=\frac{gqB_0}{4}\frac{\sigma_r^3}{\hslash}\frac{\tilde{x}}{\tilde{m}}$$

The coefficient of $\Bigg[ \int \frac{|\tilde{\chi}_+ (\tilde{x}',\tilde{t})|^2}{|\tilde{x}-\tilde{x}'|} d\tilde{x}'+ \int \frac{|\tilde{\chi}_- (\tilde{x}',\tilde{t})|^2}{|\tilde{x}-\tilde{x}'|} d\tilde{x}'\Bigg]$ is

$$\frac{Gm_r^2\tilde{m}^2t_r}{2\sigma_r \hslash}=\frac{\tilde{m}^2}{2}$$


So, equation (\ref{SGSN}) could be expressed in terms of dimensionless variables as,
\begin{eqnarray}
          - \frac{1}{2\tilde{m}} \frac{\partial^2 \tilde{\chi}_\pm}{\partial \tilde{x}^2} \pm \tilde{\gamma}\tilde{x} \tilde{\chi}_\pm - \frac{\tilde{m}^2}{2}\Bigg[ \int \frac{|\tilde{\chi}_+ (\tilde{x}',\tilde{t})|^2}{|\tilde{x}-\tilde{x}'|} d\tilde{x}'+ \nonumber\\ \int \frac{|\tilde{\chi}_- (\tilde{x}',\tilde{t})|^2}{|\tilde{x}-\tilde{x}'|} d\tilde{x}'\Bigg] \tilde{\chi}_\pm(\tilde{x},\tilde{t}) = i  \frac{\partial }{\partial \tilde{t}}\tilde{\chi}_\pm  
      \end{eqnarray} 
where $\tilde{\gamma}=\frac{gqB_0\sigma^3}{4\hslash\tilde{m_e}}$ is a dimensional parameter quantifying the strength of the magnetic force on the particle.

%The term $\frac{gqB_0}{4}\frac{\sigma^3}{\hslash}=\frac{g}{4}\frac{CT\sigma^2}{\hslash}=C\,\frac{Kg }{C\, s}\frac{\sigma^2 \, s}{m^2 \, Kg}$. \\ 


Where $C$ is coulomb, $T$ is Tesla, $s$ is second, $m$ is meter. Equation(5) is the dimensionless Schrodinger-Newton equation with an added magnetic potential term.\\
To make the dimensionless magnetic parameter $\tilde{\gamma}=\frac{gqB_0\sigma^3}{4\hslash}$ equal to $0.1$, we have set the value of $\sigma_r$ equals to $1.1806\times 10^{-5}$ \text{m}.


The initial state for the problem is,
$$\Psi(x,0)=\tfrac{1}{\sigma\sqrt{\pi}}}e^{-x^2/2\sigma^2}\otimes (\cos\tfrac{\theta}{2}|\uparrow\rangle + \sin\tfrac{\theta}{2}|\downarrow\rangle),$$
which represents the particle localized at $x=0$, with the spin in a state $\cos\tfrac{\theta}{2}|\uparrow\rangle + \sin\tfrac{\theta}{2}|\downarrow\rangle$.
Where the width of the gaussian $\sigma$ is equal to $2\sigma_r$. The Spatial and temporal grid size is taken as $0.05$ and $0.01$ respectively. The numerical infinity which is also the boundary considered here is taken at $-100$ to $100$ so that when the Gaussians move away from center, there won't be any reflections from the boundary. In the time frame considered for the problem, there is no reflection of the solutions from the boundary is observed.
\section{Results and Discussion}
\subsection{Case I: Equal superposition}

If we take the initial state as an equal superposition of $ \ket{\uparrow}$ and $\ket{\downarrow}$ states,

$$\Psi(x,0)=\frac{1}{\sqrt{2}}e^{\frac{-x^2}{2\, \sigma^2}}\otimes \ket{\uparrow}+\frac{1}{\sqrt{2}}e^{\frac{-x^2}{2\, \sigma^2}}\otimes \ket{\downarrow}$$

the results are as follows. We see the position vs time plots for both $|\chi_+(x,t)|^2$ and $|\chi_-(x,t)|^2$ with and without the self-gravity potential and compare them for different values of $\tilde{m}$. 



\begin{figure*}[t]
     \centering
     \begin{tabular}{cc}
         \includegraphics[width=\columnwidth]{position_vs_time_equal_theta_mr_0.3.pdf} & 
         \includegraphics[width=\columnwidth]{position_vs_time_equal_theta_mr_0.5.pdf}\\
         $\tilde{m}=0.3$ &  $\tilde{m}=0.5$
     \end{tabular}
      \caption{Position vs Time for equal superpositions for $\tilde{m}$ $0.3$ and $0.5$}
      \label{fig:fringewidth}
     \end{figure*}


It is evident from the plot that, the paths with and without self-gravity potential are parabolic in nature but the separation between the two beams is less compared to the free case. For lower values of $\tilde{m}$ the solutions to the S-N equation is almost same as without the self-gravity potential term, where as for higher values of $\tilde{m}$, the separation gets lesser and as we go on, eventually there is no separation at all. The two trajectories merge with the line $\tilde{x}=0$. This is an interesting fact that would not be observed in stern-gerlach experiments in the usual way. The non-linearity in the evolution equation dictates it and makes the trajectories behave in such a way. The two peaks become a single peak which cannot be attributed in decoherence kind of phenomenon.

\begin{figure}[h]
     \centering
    \includegraphics[width=\columnwidth]{position_vs_time_equal_theta_mr_0.8.pdf}
    \caption{$\tilde{m}=0.8$}
     \end{figure}

\subsection{Case II: Unequal superposition}

As we got the merging of two trajectories at the center for higher values of $\tilde{m}$ , we expect the same thing to happen but at an angle $\theta$ if we take the initial stat as an unequal superposition. So, we take the initial state as an unequal superposition of $\ket{\uparrow}$ and $\ket{\downarrow}$ states,

$$\Psi(x,0)=\frac{\sqrt{3}}{2}e^{\frac{-x^2}{2\, \sigma^2}}\otimes \ket{\uparrow}+\frac{1}{2}e^{\frac{-x^2}{2\, \sigma^2}}\otimes \ket{\downarrow}$$

The results ar as follows. Again we see the position vs time plots for both $|\chi_+(x,t)|^2$ and $|\chi_-(x,t)|^2$ with and without the self-gravity potential and compare them for different values of $\tilde{m}$. 

\begin{figure*}[t]
     
     \begin{tabular}{cc}

         \includegraphics[width=\columnwidth]{position_vs_time_unequal_theta_mr_0.3.pdf} & 
          \includegraphics[width=\columnwidth]{position_vs_time_unequal_theta_new_mr_0.5.pdf} \\
          $\tilde{m}=0.3$ & 
          $\tilde{m}=0.5$ \\
     \end{tabular}
      \caption{Position vs Time for unequal superpositions for $\tilde{m}$ $0.3$ and $0.5$}
      
     \end{figure*}
The plots show that the trajectories of the two solutions $|\chi_+(x,t)|$ and $|\chi_-(x,t)|$ are parabolic and the separation between them is less compared to the free case. We also observed an interesting result here that is there is an asymmetry between the two trajectories that was not before when we had considered equal superposition. The discrepancy between the relative positions of the maximas we found in this investigation. The asymmetry due to the unequal superposition shows that the trajectory of $\chi_+(\tilde{x},\tilde{t})$ with a higher weight-age would be close to origin compared to the trajectory of $\chi_-(\tilde{x},\tilde{t})$ with lesser weight-age. This result is completely in disagreement with the actual stern-gerlach experiment's prediction which does not have a coupling term between the governing equations of $\chi_+(\tilde{x},\tilde{t})$ and $\chi_-(\tilde{x},\tilde{t})$. Even if we start with an unequal superposition, without the self-gravitational interaction the asymmetry is never expected and it is a clear signature of the effect of self-gravity. 




\subsection{Case III: Emergence of Classicality}

As we go on with more unequal superposition compared to the previous case the  and increase the value of $\tilde{m}$ we see that the two trajectories merge but not on the line $\tilde{x}=0$ but at an angle. The initial state considered is,

$$\Psi(x,0)=0.95e^{\frac{-x^2}{2\, \sigma^2}}\otimes \ket{\uparrow}+0.32e^{\frac{-x^2}{2\, \sigma^2}}\otimes \ket{\downarrow}$$
The above state is simulated for $\tilde{m}=0.8$
The plots below show the results.\\


The angle we calculated from Ehernfest's theorem that the classical path is the average of the quantum paths, i.e, $(\cos^2\theta-\sin ^2 \theta)\chi_+(\tilde{x},\tilde{t})$ or $-\frac{1}{2}at^2$. Our simulation shows that the angle at which the two trajectories merge is same as the classical angle predicted by Ehernfest's theorem. This is an interesting result that could describe emergence of classicality which is a long standing issue with Quantum theory. This result cannot be attributed to decoherence as decoherence mechanism would show two distinguished separate paths anyway.

Simulation for one more state is considered here, i.e,

$$\Psi(x,0)=\frac{\sqrt{3}}{2}e^{\frac{-x^2}{2\, \sigma^2}}\otimes \ket{\uparrow}+\frac{1}{2} e^{\frac{-x^2}{2\, \sigma^2}}\otimes \ket{\downarrow}$$

The above state is simulated for $\tilde{m}=0.6$
The plots below show the results.\\

\maketitle
\section{Results and Discussions}













\section{Conclusion}

The macroscopic theory that describes gravity is a nonlinear in nature as matter describes the geometry of spacetime and the geometry decides how the matter should move. In contrast, the microscopic theory that describes the dynamics at very small length scale is a linear theory. If the S-N equation could describe physical phenomenons after experimental tests, we could have both the fundamental theories those are nonlinear in nature. As we showed, for small mass range the nonlinear S-N equation describes the physical phenomenon exactly as quantum mechanics does and for large mass it shows some new physics emerging, quantum theory that we know could be considered as a special case of the nonlinear theory. For high mass range the deviation of the trajectories from free case, emergence of classicality, the deviation of the straight line path for fringe width vs inverse mass in two slit interference are the new emerging phenomena that should be tested in experiments and if found true then it could put some light on the theory of quantum gravity.  
































     
\begin{figure*}[t]
     
     \begin{tabular}{cc}
         
         \includegraphics[width=\columnwidth]{Position_vs_time_new_more_unequal_theta_mr_0.8.pdf}&
          \includegraphics[width=\columnwidth]{Position_vs_time_new_unequal_theta_mr_0.6.pdf}\\
         
     \end{tabular}
      \caption{Position vs Time for unequal superpositions for $\tilde{m}$ $0.8$ and $\tilde{m}$ $0.6$}
      \label{fig:Classical vs quantum}
     \end{figure*}





\bibliographystyle{ieeetr}
\bibliography{Reference.bib}
\end{document}
