\documentclass[aps, twocolumn, amsfonts, amsmath, bbm, amssymb, superscriptaddress, showkeys]{revtex4-1}

\usepackage{graphicx}

%\usepackage{caption,subcaption}
\usepackage[normalem]{ulem}
\usepackage{physics}
\usepackage{tikz,lipsum,lmodern}
\usepackage[export]{adjustbox}
\usepackage{soul}


\usepackage[unicode=true,pdfusetitle,
 bookmarks=true,bookmarksnumbered=false,bookmarksopen=false,
 breaklinks=false,pdfborder={0 0 0},backref=false,colorlinks=true,citecolor=blue,urlcolor=violet 
]{hyperref}
\usepackage{xcolor}
\newcommand{\orcid}[1]{\href{https://orcid.org/#1}{\includegraphics[width=10pt]{orcid}}}
\newcommand{\eop}{\hfill{\rule{2.2mm}{2.2mm}}}
\newcommand{\inp}[2]{\left( {#1} ,\,{#2} \right)}
\newcommand{\dip}[2]{\left< {#1} ,\,{#2} \right>}

%%%%%%%%Dirac Notation %%%%%%%m
\newcommand{\bket}[1]{\mbox{$|{#1}\rangle$}}
\newcommand{\aket}[1]{\mbox{$\langle{#1}|$}}
\newcommand{\bracket}[2]{\mbox{$\langle{#1}|{#2}\rangle$}}
\newcommand{\qup}{\mbox{$\bket{\uparrow}$}}
\newcommand{\qdwn}{\mbox{$\bket{\downarrow}$}}
\newcommand{\qst}{\mbox{$\bket{\psi}$}}

%%%%%%%%%%%%%%%%%%%%%%%%%%%%%%%%%%%%%%%%%%%%%%%%%


\begin{document}
\title[]{Emergence of Classicality in Stern-Gerlach Experiment via Self-Gravity}
\author{Sourav Kesharee Sahoo\orcid{0000-0003-1812-0417}}
\email{sourav.sahoo1490@gmail.com.} 
\affiliation{Department of Physics, BITS-Pilani K K Birla Goa Campus, Goa-403726, India.}
\author{Radhika Vathsan\orcid{0000-0001-5892-9275}}
\email{radhika@goa.bits-pilani.ac.in} 
\affiliation{Department of Physics, BITS-Pilani K K Birla Goa Campus, Goa-403726, India.}

\author{Tabish Qureshi\orcid{0000-0002-8452-1078}}
\email{tabish@ctp-jamia.res.in}
\affiliation{Center for Theoretical Physics, Jamia Millia Islamia, New Delhi 110025.}
%\date{\today}

\begin{abstract}
Emergence of classicality from quantum mechanics is a hotly debated topic, where several approaches have been suggested but none provides a satisfactory resolution. In the present work, the Schr\"odinger–Newton equation has been used to study the role of self-gravitational interaction in a Stern-Gerlach experiment with massive spin-1/2 particles. For small mass, the results of state evolution show a superposition of two wave-packets, following trajectories virtually identical to those seen in a conventional Stern-Gerlach experiment, in agreement with standard quantum theory. For large enough mass, the results show only a single trajectory of the wave-packet, indicative of an \emph{apparently} unquantized spin. This single trajectory of the wave-packet follows the expected \emph{classical} path of a particle carrying a classical magnetic moment. This classicality emerges simply with the increasing mass of
the particle, due to self-gravitational interaction. This is in contrast to the picture emerging from decoherence, which is a \emph{mixed state} of two trajectories corresponding to the spin "up" and spin "down" states. Decoherence is thus, unable to explain the classically expected path of the particle.
The classically expected path of the particle, in the Stern-Gerlach experiment, probably cannot be explained even in 
the many worlds interpretation of quantum mechanics. 
Stern-Gerlach experiments in the macroscopic domain are needed to settle this question.
\end{abstract}
\keywords{ Stern-Gerlach experiment, Schr\"odinger-Newton equation, Self-gravitational interaction, Semi-classical gravity.
}
\maketitle
\section{Introduction}

Quantum theory, in the modern perspective, is an extremely successful theory. Even those aspects of the theory that made eminent physicists like Einstein uncomfortable, have been experimentally verified \cite{Aspect_2015}. However, there still remain two issues, which are not really independent, that are nowhere close to resolution. One is the so-called measurement problem, namely explaining how a particular outcome emerges from all the possibilities in a measurement process \cite{Schlosshauer2011}. Quantum theory only provides a prescription to calculate the probabilities associated with the various outcomes. It provides no mechanism for a single outcome in a measurement process,  which appears to be non-unitary in character. The other is the issue of emergence of the classical world from quantum theory \cite{schlosshauer2007decoherence,joos2013decoherence}. Quantum theory, in principle, allows superposition of two or many states. On the other hand, in our familiar classical world, superpositions of macroscopically distinct states are never observed. For example, a pebble is never seen to be in a superposition of being at two well-separated locations. If one believes that quantum theory is the more fundamental theory, the classical world should emerge from quantum theory, in some limit that we may call the classical limit. Again, quantum theory provides no such mechanism. While there is absolutely no convergence on how the measurement problem could be solved, one major contender is  decoherence as a very plausible mechanism for the emergence of classicality from quantum mechanics \cite{joos2013decoherence,schlosshauer2007decoherence}. Decoherence emphasizes the  unavoidable role played by the environment in the evolution of a quantum system. It stresses that real systems, especially macroscopic ones, can never be shielded from environmental interactions. It is demonstrated that even the weakest of interactions is enough to destroy the quantum coherence of a system, and drives it into a \emph{mixed state}, that predicts the probabilities associated with different outcomes but contains much less information about the coherence between the components.  The inevitability of using probabilities in an essential way makes many people uncomfortable but decoherence is still a popular mechanism for the emergence of classical world from quantum theory. 

Consider the Stern-Gerlach experiment (see Fig. \ref{sgsetup}), that is considered to be a prototype example of a quantum measurement \cite{bohm1989quantum}.
A spin-1/2 particle emerges from the source and travels through an inhomogeneous magnetic field, to the screen. In the experiment, the particle always lands at one of the two spots on the screen, corresponding to spin ``up" or ``down". That was the evidence for \emph{space quantization} of angular momentum. Quantum mechanically, the z-component of the spin angular momentum can have only discrete values, and none in between. If one has a picture of quantum measurement process, or if one just assumes the collapse postulate, this experimental result can be understood quite well. However, we want to address the question, what happens if the particle is classical. Here we need not invoke a collapse postulate corresponding to the particle hitting the screen.
What we wish to ask is how does a particle in a Stern-Gerlach experiment behave, if it is massive enough to be considered almost classical.
This experiment has been done with quantum particles. To our knowledge it
has not yet been performed using classical particles. So, as of now we do not know what actually happens in a Stern-Gerlach experiment with particles massive enough to be considered classical.
\begin{figure}[t]         
\includegraphics[width=0.845\columnwidth]{SG.pdf}
         \caption{Schematic representation of a typical Stern–Gerlach setup: atoms travel through an inhomogeneous magnetic field and are deflected up or down depending on the value of the $z$ component of their spin. }
         \label{sgsetup}
\end{figure}

However, what does one expect from the classical dynamics of a particle carrying a magnetic moment, traveling through an in-homogeneous  magnetic field? The quantum state $\cos\frac{\theta}{2}\qup + \sin\frac{\theta}{2}\qdwn$ is an eigenstate of the spin component 
$S_{\theta}=\vec{S}\cdot \hat{n}$, where $\hat{n}$ is a unit vector at an angle $\theta$ to the $z$-axis. In the classical picture this would represent an angular momentum  vector pointing along $\hat{n}$. A simple classical analysis shows that the particle experiences a  force driving its trajectory somewhere in between the two trajectories in the quantum picture. This particle will hit the screen at a spot in between the two spots corresponding to the spin states $\qup$ and $\qdwn$. In fact, depending on the value of $\theta$, there is a continuous set of possible points on the screen, between the spots mentioned above, where the particle will land.

How would decoherence explain the emergence of classical behavior in a Stern-Gerlach experiment? The effect of decoherence in the Stern-Gerlach experiment leads to an approximately mixed state of two trajectories corresponding to the two eigenstates of the spin component in the direction of the field \cite{Venugopalan_Kumar_Ghosh_1995}. This point is discussed in detail in section \ref{dec}. It is clear that decoherence will never lead to any trajectory other than the two extreme top and bottom ones. So decoherence is unable to explain the \emph{classically expected} behaviour of a \emph{classical} particle passing through a Stern-Gerlach setup.

We wish to emphasize here that in various other experiments, the classically expected results can be explained using either the collapse postulate or by invoking decoherence. For example, in the case of a massive particle passing through a double-slit, classical dynamics says that the particle will follow one of the two possible paths. Quantum theory allows the particle to pass through both the slits together, and be in a superposition of two spatially separated positions (of the slits). Invoking decoherence, one can argue that a \emph{classical} particle will follow one the two paths, and the superpositions will be suppressed. So decoherence satisfactorily explains the classically expected behavior in the two-slit experiment. The Stern-Gerlach experiment appears to be a curious case where the classically expected result cannot be explained either by decoherence, or by invoking the collapse postulate. One then needs to look for alternative approaches for emergence of classicality.

Roger Penrose \cite{penrose1996gravity} proposed 
that gravity could play a role in the emergence of classicality in
massive particles. He argued that macroscopic
gravity could be the driving force behind the reduction of the wave function as the wave packet responds to its own gravity.  He used the Schr\"{o}dinger-Newton (S-N) equation, earlier introduced by Di\'{o}si \cite{DIOSI1984199}, to explore the quantum-state reduction phenomenon \cite{penrose1996gravity,penrose2014gravitization}. Subsequently, several authors investigated the effect of gravity and self-gravity on quantum systems in various ways \cite{colella1975observation,grossardt2016effects,grossardt2016approximations,singh2015possible,yang2013macroscopic,kumar2000single}. 
Could self-gravity of a massive spin 1/2 particle lead to emergence of classical behaviour in the Stern-Gerlach experiment?



\section{The Stern-Gerlach experiment and self-gravity}

\subsection{The Schr\"odinger-Newton equation}

The seeds of the S-N equation came from semi-classical gravity independently considered by M\"oller \cite{moller1962theories} and Rosenfeld \cite{rosenfeld1963quantization}. In this approach, quantized matter is assumed to be coupled to the classical gravitational field \cite{mattingly2005quantum,kibble1981semi,bahrami2014schrodinger}, and the Einstein field equations get modified as, 
\begin{equation}
          R_{\mu \nu} + \frac{1}{2} g_{\mu \nu} R = \frac{8 \pi G}{c^4} \langle \Psi | \hat{T}_{\mu \nu}| \Psi \rangle, \label{e2}
          \end{equation}
where the term on the right hand side is the expectation value of the energy-momentum tensor with respect to the quantum state $|\Psi\rangle$ of matter.  %This semi-classical modification has been studied in the context of the question whether it is really necessary to quantize gravity or would a semi-classical treatment suffice\cite{eppley1977necessity,page1981indirect}. 
This modification to the Einstein field equation then naturally leads to the Schr\"odinger-Newton equation \cite{bahrami2014schrodinger,van2011schrodinger,salzman2005investigation,giulini2012schrodinger},
 \begin{equation}
          \Bigg[ \frac{p^2}{2m} - G m^2 \int \frac{|\Psi (r',t)|^2}{|r - r'|} d^3 r'\Bigg] \Psi(r,t) = i \hbar \frac{\partial \Psi (r,t)}{\partial t}. \label{eq:SN}
      \end{equation}
The  additional potential,
  \begin{equation} V_{G} = - G m^2 \int \frac{|\Psi (r',t)|^2}{|r - r'|} d^3 r',
  \label{VG}
  \end{equation}
  represents the gravitational self-interaction, which is derived from the spatial probability distribution of the  massive particle.
Since it depends on the wavefunction itself, it makes the S-N equation basically a \emph{non-linear} modification
of the Schr\"odinger equation. The non-linearity breaks the unitarity of  Schr\"odinger evolution, and opens up the potentialities of all those effects which were not possible because of the linearity of quantum dynamics. One then hopes to see a dynamical reduction of the wavefunction, what is often called  wavefunction collapse. Breaking of  the sacred quantum-mechanical property of linearity potentially  brings up new problems  \cite{anastopoulos2014problems}. However, the hope behind this approach is that at the scales at which quantum mechanics has been successfully tested, the dynamics will be linear for all practical purposes, and the non-linearity will show up only when one approaches the classical limit. There have also
been several  other approaches in which non-linearity has been introduced in order to obtain wave-function collapse \cite{bassi2013models,ghirardi1990markov,adler2007collapse}. However, an elegant aspect of the  S-N equation is that there is no tunable parameter introduced, and the scale at which classicality might appear should naturally emerge from this equation.

\subsection{Formulation of the Stern-Gerlach problem}
Consider a Stern-Gerlach experiment as shown in Fig.~\ref{sgsetup}, in which a spin-1/2 particle  of mass $m$ travels along the $x$ axis,  experiencing an inhomogeneous magnetic field.  Suppose the  field  is along the  $z$ axis. 
For simplicity the field is assumed to vary linearly with $z$: $\vec{B} \approx B_0z\hat{z}$
\footnote{In reality the magnetic field should satisfy $\frac{\partial B_y}{\partial y} + \frac{\partial B_z}{\partial z} = 0$. So there should be an inhomogeneous component in the $y$ direction too. In addition there should be a homogeneous component in the $z$ direction. Since these are of no consequence in the dynamics of the particle in the Stern-Gerlach experiment, we ignore them in the analysis here.}.
If we assume that the spin-1/2 magnetic moment $\vec{\mu}$ is due to a single unpaired electron in the particle, then the potential experienced by the particle is given by
\begin{equation}
 V_B(z) = -\vec{\mu}\cdot\vec{B}
     = -\frac{eB_0}{m_e} ~ ~z S_z
     = -\gamma z \hat{\sigma}_z.
     \label{eq:VB}
\end{equation}
Here $S_z= \frac{\hbar}{2}\hat{\sigma}_z$, the $z$-component of the spin and the parameter  $\gamma=\mu_B B_0$ where $\mu_B$ is the Bohr magneton. This potential causes a spin-dependent deviation along  the $z$-direction. The  dynamics of the particle along the $x$-axis is trivial, and just serves to translate the $x$-position of the particle by $x = vt$  in a given time $t$,  assuming an initial constant velocity of magnitude $v$ in the $x$-direction. The quantum dynamics of the particle in the $z$ direction is  given by the one-dimensional Schr\"odinger equation  under the influence of the potential $V_B(z)$ given by Eq.~(\ref{eq:VB}).

We now impose an additional self-gravitational potential given by Eq.~(\ref{VG}). The effective dynamics in the $z$ direction is given by
the Schrodinger-Newton equation reduced to one dimension\cite{sahoo2022}, along with the magnetic interaction potential:
\begin{eqnarray}
          \Bigg[   \frac{p_z^2}{2 m} 
          - G m^2 \int \frac{|\Psi (z',t)|^2}{|z - z'|} d z'  + V_B(z) \Bigg] \Psi(z,t)
          \nonumber\\
          = i \hbar \frac{\partial \Psi (z,t)}{\partial t}. \label{eq:SNB}
      \end{eqnarray}
The state of the particle in general is given by a spinor $ \bket{\Psi}$, which in the position basis, is a two-component wave-function that can be  expressed  as 
\[\ip{z}{\Psi(t)} =
\chi_+(z,t) \qup +  \chi_-(z,t) \qdwn,\] 
where $\chi_{\pm}$ are the  wave-functions corresponding to the eigenstates $\qup$ and $\qdwn$ of the $S_z$ operator.
The position-space probability distribution of the particle  is required to calculate the gravitational potential and is given by
\begin{eqnarray}
|\Psi (z,t)|^2 &=& |\ip{ z}{\Psi}|^2 \nonumber \\
   &=& |\chi_+(z,t)|^2 + |\chi_-(z,t)|^2,
\end{eqnarray}
where the cross terms between $\chi_{\pm}(z,t)$  vanish due to the orthogonality of $ \qup$ and $\qdwn$. Using Eq.~(\ref{VG}), the self-gravitational potential becomes
\begin{align*}
 V_G(\Psi) &=V_G(\chi_+)+V_G(\chi_-)\nonumber \\
 {\text{ where }}V_G(\chi_\pm) &=- Gm^2\int \frac{|\chi_\pm (z',t)|^2}{|z-z'|} dz'.
\end{align*}
The S-N equation for our system can now be expressed as
\begin{equation}
\begin{pmatrix}
\frac{p_z^2}{2m}+V_G - \gamma z  & 0\\
0 & \frac{p_z^2}{2m}+V_G + \gamma z  
\end{pmatrix}\begin{pmatrix}\chi_+ \\ \chi_- \end{pmatrix} = i\hbar \begin{pmatrix} \Dot{\chi}_+ \\ \Dot{\chi}_-
\end{pmatrix},
\label{SGSN}
\end{equation}
which is a pair of  coupled integro-differential equations for the time evolution of the two components $\chi_{\pm}$.
We now rescale the variables to convert this equation to dimensionless form, by  introducing a length scale $\sigma_r$  in terms of which  time and mass scales are defined as 
\begin{equation}
t_r=\left(\frac{\sigma_r^5}{G \hbar}\right)^{\frac{1}{3}},~~~
m_r=\left(\frac{\hbar^2}{G \, {\sigma_r}}\right)^{\frac{1}{3}}.
\label{eq:scalefactors}
\end{equation}
We can now work in terms of the dimensionless variables
\[
\tilde{z}=z/\sigma_r,~~~
\tilde{m}=m/m_r~,~~~
\tilde{t}=t/t_r~ , \]
and  rescaled wavefunctions
$ \tilde{\psi} = \sqrt{\sigma_r} \psi,~~
\tilde{\chi}_{\pm} = \sqrt{\sigma_r} \chi_{\pm}. $
%For the simulation we have fixed the value of $\sigma _r$ equals to $1.1806\times 10^{-5}$ \text{m}.\\
 The rescaled  magnetic field parameter is $\tilde{\gamma}=\gamma\sigma_r$, and Eq.~(\ref{SGSN}) can be rewritten in dimensionless terms  as 
\begin{eqnarray}
          - \frac{1}{2\tilde{m}}\frac{\partial^2\tilde{\chi}_\pm}{\partial\tilde{z}^2}  
          \mp \tilde{\gamma}\tilde{z} \tilde{\chi}_\pm 
          - \tilde{m}^2\tilde{\chi}_\pm \int \tfrac{|\tilde{\chi}_+ (\tilde{z}',\tilde{t})|^2 + |\tilde{\chi}_- (\tilde{z}',\tilde{t})|^2}{|\tilde{z}-\tilde{z}'|} d\tilde{z}' \nonumber \\ 
          = i  \frac{\partial \tilde{\chi}_\pm}{\partial \tilde{t}} 
          ~~~~
          \label{SGSN-d}
      \end{eqnarray} 
\phantom{To make the dimensionless magnetic parameter $\tilde{\gamma}=\frac{gqB_0\sigma^3}{4\hbar}$ equal to $0.1$, we have set the value of $\sigma_r$ equals to $1.1806\times 10^{-5}$ \text{m}.}


The  initial state of the particle is taken to be a Gaussian wave-packet of width $\epsilon$, localized at $\tilde{z}=0$,  with the spin state in a general superposition $\cos\frac{\theta}{2} \qup + \sin\frac{\theta}{2}\qdwn $, so that we have
\[
\langle z|\Psi(0)\rangle = N e^{-\tilde{z}^2/2\epsilon^2}
\left(\cos\tfrac{\theta}{2} \qup + \sin\tfrac{\theta}{2}\qdwn\right)
%\tilde{\chi}_+(\tilde{z},0) = N\cos\tfrac{\theta}{2} %e^{-\tilde{z}^2/2\epsilon^2}, ~~
%\tilde{\chi}_-(\tilde{z},0) = N\sin\tfrac{\theta}{2} %e^{-\tilde{z}^2/2\epsilon^2}
\]
where $N$ is a normalization constant. This is the quantum state for a particle that is spin-polarized at an angle $\theta$ with the $z$ axis.
Consequently $\tilde{\chi}_+(\tilde{z},0) = N\cos\tfrac{\theta}{2} e^{-\tilde{z}^2/2\epsilon^2}, ~~
\tilde{\chi}_-(\tilde{z},0) = N\sin\tfrac{\theta}{2} e^{-\tilde{z}^2/2\epsilon^2}$.

We used the Crank-Nicolson method \cite{Crank_1947,smith1985numerical} for solving Eqs. (\ref{SGSN-d}) numerically.  We chose the spatial and temporal grid sizes as $\Delta \tilde{z} = 0.05$ and $\Delta \tilde{t} = 0.01$ respectively, and   the CFL (Courant–Friedrichs–Lewy) condition for stability of the solutions is satisfied since $\frac{\Delta\tilde{t}}{\Delta \tilde{z}} < 1$. Numerical  simulations of the evolution are plagued by reflections from the boundaries, to avoid which we take the spatial numerical infinity  at $\pm 100$. In the time frames we simulated, we observed no reflection of the solutions from these boundaries.

\section{Results and Analysis}

\subsection{Dynamics of wavepackets}

We simulated the evolution of the initial state with $\theta=\pi/3$. 
%As the particle evolves with time, we observe the peaks of the spin-up and spin-down wavepackets deflected in opposite directions. 
The time evolution of the wave-function with $\tilde{t}$ for a relatively small mass $\tilde{m}=0.1$ is portrayed 
in the surface plot in Fig.~\ref{fig:3Dm0.1}.
An initial single wave-packet, of width $\epsilon=10$, splits into a superposition of two wave-packets traveling in different directions. This behavior is not different from that expected from  the standard Stern-Gerlach experiment.
\begin{figure}
\includegraphics[width=1.0\columnwidth]{psi_m_0.1.pdf}
%\includegraphics[width=1.0\columnwidth]{3D_psi_mr_0.1_gravity_unequal_2.pdf}
\caption{Time evolution of $|\tilde{\Psi}|^2$ for $\tilde{m} = 0.1$ and $\theta=\pi/3$.} The initial wave-packet, localized at $\tilde{z}=0$, evolves into a superposition two wave-packets of unequal heights.
\label{fig:3Dm0.1}
\end{figure}


A contour plot of 
$|\tilde{\Psi}(\tilde{z},\tilde{t})|^2$ as it evolves in time 
is shown in Fig.~\ref{fig:peaksm0.1}.
%Since the motion of the wavepacket along the $x$ axis, not explicitly considered here, is assumed to be uniform, the peak position versus time curve can be interpreted as the trajectory of the particle.
%The dynamics of the peaks of the two wave-packets is compared for the case with and without the self-gravity potential for different values of $\tilde{m}$. 
The overlay shows the trajectory of the peaks of the spin-up and spin-down wavepackets under pure Schr\"odinger dynamics. It is evident that for low enough mass ($\tilde{m}=0.1$) the presence of self-gravity does little to affect the  pure quantum dynamics,  and quantum superpositions persist.
\begin{figure}
\includegraphics[width=1.0\columnwidth]{3D_m_0.1_gravity.pdf}
%\includegraphics[width=1.0\columnwidth]{new_position_vs_time_unequal_theta_mr_0.1.pdf}
\caption{Evolution of $|\tilde{\Psi}|^2$ with $\tilde{t}$ for $\tilde{m} = 0.1$ and $\theta=\pi/3$, with self-gravity, represented as a contour plot. The red lines represent the position of the peaks of the two wavepackets without self-gravity. The time evolution in the presence of self-gravity is indistinguishable from pure Schr\"odinger dynamics.}
\label{fig:peaksm0.1}
\end{figure}


As we increase the mass, the trajectories of the  peaks of the wave-packets begin to deviate from those corresponding to pure Schr\"odinger dynamics. For example,  Fig. \ref{fig:peaksm0.3}) is a comparison of the trajectories (with and without self-gravity) for $\tilde{m}=0.5$. This behaviour is expected due to the gravitational pull between the two parts of the wavefunction. Interestingly, this deviation of the two trajectories is not symmetric if the spin superposition is unequal. This asymmetry of the two trajectories about the $\tilde{z}=0$ line is a signature of the self-gravity term, that may be experimentally tested.
\begin{figure}[t]
%        \includegraphics[width=1.0\columnwidth]{new_position_vs_time_unequal_theta_mr_0.3.pdf}\\
          \includegraphics[width=1.0\columnwidth]{new_position_vs_time_unequal_theta_mr_0.5.pdf}
                \caption{Evolution of the position of the peaks of the wavepacket ($\tilde{z}_p$) for unequal superpositions ($\theta=\pi/3$) for $\tilde{m} = 0.5$. Self-gravity leads to an ``attraction" between trajectories.}
      \label{fig:peaksm0.3}
     \end{figure}

\phantom{when compared to the evolution without the self-gravity interaction, the separation between the two peaks is less, as is expected due to the gravitational pull between the two parts of the wavefunction. The most interesting behaviour as  mass increases is that the deflection of the two wave-packets is proportional to the  weightage in the superposition of the initial state. This asymmetry due to the unequal superposition is in complete contradiction to  the behaviour  of the system without self-gravity: the {\em paths} of the wave-packets are not affected by the probability amplitudes in the superposition, only the intensities of the spots observed on the screen are.}
%These features are beautifully brought out in the surface plots \ref{fig:surface-plots}.


\subsection{Emergence of Classicality}

\begin{figure}[t]
\includegraphics[width=1.0\columnwidth]{psi_m_0.6.pdf}
%\includegraphics[width=1.0\columnwidth]{3D_psi_mr_0.6_gravity.pdf}
\caption{Evolution of $|\tilde{\Psi}|^2$ with $\tilde{t}$ for $\tilde{m} = 0.6$ and $\theta=\pi/3$. The initial wave-packet, localized at $\tilde{z}=0$, does not evolve into a superposition of two wave-packets. Instead, it continues as a single wave-packet partially deflected due to the magnetic field.}
\label{fig:psim0.6}
\end{figure}

\begin{figure}[t]
\includegraphics[width=1.0\columnwidth]{3D_m_0.6_gravity.pdf}
%\includegraphics[width=1.0\columnwidth]{3D_psi_mr_0.6_gravity.pdf}
\caption{Contour plot of the time evolution of $|\tilde{\Psi}|^2$ for $\tilde{m} = 0.6$ and $\theta=\pi/3$. The initial wave-packet, localized at $\tilde{z}=0$, does not evolve into a superposition of two wave-packets. Instead, it continues as a single wave-packet whose center follows the classical trajectory.}
\label{fig:3Dm0.6}
\end{figure}
The most unexpected results are seen as we increase the mass still further. For values of $\tilde{m}$ greater than around $0.6$,  the initial wavepacket centered at $\tilde{z}=0$, does not separate into two, as is expected in a Stern-Gerlach experiment, instead is deflected along a particular path (see Figs. \ref{fig:psim0.6} and \ref{fig:3Dm0.6}). On the screen we would see a single spot instead of two. 
%The simulation results for $\tilde{m}=0.6$ and  $\tilde{m}=0.8$, for $theta=\pi/3$ are displayed in Fig. ??.

The trajectory of a classical particle with magnetic moment at an angle $\theta$ with the $z$ axis, is one with acceleration $a = ({\tilde{\gamma}}/{\tilde{m}}) \cos\theta$.
The peak of the wavepacket for $\tilde{m}=0.6$ follows this trajectory, and so does  the expectation value of position ($ \langle\tilde{z}\rangle$)(see Fig. \ref{fig:peaksm0.6}). This seems to signal the emergence of classicality. A single trajectory of the particle, in between the two trajectories corresponding to the two spin states $\qup$ and $\qdwn$, signifies absence of space quantization. Interestingly, although the spin is treated as fully quantized in our analysis,  the particle behaviour due to self-gravity is as though the spin \emph{appears} not quantized. 
\begin{figure}[t]
           \includegraphics[width=1.0\columnwidth]{new_position_vs_time_unequal_theta_mr_0.6.pdf}\\
                \caption{Evolution of the expectation value of position. $\langle\tilde{z}\rangle$, for unequal superposition ($\theta=\pi/3$) for $\tilde{m}=0.6$. Self-gravity results in a single trajectory, which agrees well with the classical trajectory.}
      \label{fig:peaksm0.6}
     \end{figure}

We tested this effect for various values of the asymmetry in  superposition of the spin states, i.e., for various values of $\theta$. The results summarized in Fig. \ref{fig:peaksm0.7} show that for high mass, the S-N equation recovers the classically expected result in the Stern-Gerlach experiment. The quantum average of the position follows the classical trajectory reasonably well in all the cases studied here. Classical magnetic moments polarized at different angles to the $z$-axis would hit the screen at different locations  between the two points corresponding to the spin state $\qup$ and $\qdwn$ and that is exactly what is seen in  Fig. \ref{fig:peaksm0.7}.
\phantom{According to classical mechanics, an ensemble of randomly polarized spin particles are likely to hit the screen anywhere between the two points corresponding to the spin state $\qup$ and $\qdwn$.}

\phantom{The trajectories obtained from the simulations show slight oscillatory deviations from the classical path. We interpret these as an interplay between the magnetic force trying to pull the two wave-packets in different directions, and the self-gravity force trying to keep them together. }

Following these results, a high mass particle  initially in an equal superposition of $\qup$ and $\qdwn$ states ($\theta = \pi/2$),  should not show any deflection in the magnetic field,  since the classical force on the particle, in the $z$ direction, is zero in this case. Indeed, that is what we observed in Fig. \ref{fig:peaksm0.7}.

Another interesting aspect of dynamics under the S-N equation is regarding the behaviour of the expetation value of the position of the particle.  In pure Schr\"odinger dynamics, Ehrenfest's theorem connects the time evolution of the expectation values of position and momentum to the classical equations of motion \cite{Ehrenfest_1927}.  Our results show 
that the expectation value of position closely follows the classical path for all values of mass considered. Thus, the Ehrenfest theorem continues to hold for the S-N dynamics, despite the nonlinear, self-gravitational potential term. This has not yet been shown so far.  

%\begin{figure*}[t]
%      \begin{tabular}{cc}
%        \includegraphics[width=0.8\columnwidth]{new_position_vs_time_unequal_theta_mr_0.3.pdf} & 
%          \includegraphics[width=0.8\columnwidth]{new_position_vs_time_unequal_theta_mr_0.5.pdf} \\
 %         (a)&(b) \\
%            \includegraphics[width=0.8\columnwidth]{new_position_vs_time_unequal_theta_mr_0.6.pdf}&
%          \includegraphics[width=0.8\columnwidth]{new2_position_vs_time_unequal_theta_mr_0.8.pdf}\\
%          (c) & (d)

%     \end{tabular}
%      \caption{Evolution of the position of the peaks of the %wavepacket ($\tilde{z}_p$) for unequal superpositions %($\theta=\pi/3$ for $\tilde{m}$ $0.3, 0.5, 0.6$ and $0.8$}
%      \label{fig:graph2}
%     \end{figure*}

\phantom{We tested this effect for a higher asymmetry in the superposition: $\cos \theta/2 = 0.95, \sin\theta/2 = 0.32$.
Fig.~(\ref{fig:graph2})(c) and (d) show behaviour that corroborates this effect: namely that the two spin wave-packets separate for low enough mass, but for higher mass $\tilde{m} = 0.8$, they merge on a line that makes an angle $\theta$ with the horizontal. This is exactly the behaviour expected of a classical spin polarized at an angle $\theta$. We verified that the line to which the ``path" tends is actually  the average of the quantum paths, i.e, the maximum of $(\cos^2\theta-\sin ^2 \theta)\chi_+(\tilde{z},\tilde{t})$. We verified that this is exactly the classical trajectory of the spin $\frac{1}{2}at^2$, where $a = \frac{\gamma}{m} \cos \theta$. }

\phantom{While the spin-up and spin-down  never merge in pure quantum mechanics, neither does such behaviour emerge from decoherence-like scenarios, where both the quantum paths are supported with different probabilities. The self-gravitational interaction signals the emergence of classicality at high enough masses.}
 

\begin{figure}[t]
     \centering
%     \begin{tabular}{ccc}
%         \includegraphics[width=0.33\textwidth]{new_position_vs_time_equal_theta_mr_0.3.pdf} & 
%         \includegraphics[width=0.33\textwidth]{new_position_vs_time_equal_theta_mr_0.5.pdf}&
%         \includegraphics[width=1.0\columnwidth]{new_position_vs_time_equal_theta_mr_0.8.pdf} \\
%     \end{tabular}
          \includegraphics[width=1.0\columnwidth]{position_vs_time_unequal_theta_mr_0.7.pdf}
      \caption{Expectation value of position vs time for  $\tilde{m}=0.7$, for various asymmetries of superposition (various values of $\theta$). In all cases the trajectory of the peak of the wave-packet follows the classical path reasonably well. This should be construed as the emergence of classicality, driven by self-gravity.}
      \label{fig:peaksm0.7}
     \end{figure}
     
%\subsection{Equal superposition}
\phantom{
In an attempt to quantify the emergence of classicality with mass, we  focus on the case where 
the initial state is in an equal superposition of $ \ket{\uparrow}$ and $\ket{\downarrow}$ states, $\theta = \pi/2$.
Fig.~(\ref{fig:graph1}) shows that the separation of the peaks of the up-spin and down-spin components 
reduces as the mass increases. This is expected as an effect of self-gravity for different values of $\tilde{m}$. 
For low  values of $\tilde{m}$,  the separation between the wavepackets is nearly the same with and without self-gravity.  The separation reduces for higher mass values, but there appears to be a transition at around $\tilde{m}=0.6$ where the   separation vanishes. The particle evolves  without deviating, along  the line $\tilde{z}=0$. 
For higher masses there is really no separation of the two spin components: this agrees with the fact that the $z$ component of a classical magnetic moment is zero in this case, and hence the particle experiences no force due to the magnetic field along $z$ direction.}


\section{Discussion}
\subsection{Emergence of classicality in quantum mechanics}\label{dec}

In the light of the results obtained here, showing emergence of classicality in the Stern-Gerlach experiment via self-gravity, let us now discuss how classicality emerges in a Stern-Gerlach experiment via environment-induced decoherence.

\phantom{Consider a typical Stern-Gerlach experiment where a massive spin-1/2 particle travels  through an inhomogeneous magnetic field towards a screen where it is detected (see Fig.~\ref{sgsetup}). The spin is in general  in an initial state $\cos\frac{\theta}{2}\qup + \sin\frac{\theta}{2}\qdwn$,  in the basis 
$\{\qup, \qdwn\}$ which are the eigenstates of the $z$ component of spin, $S_z$. If the magnetic field  direction is taken as $\hat{z}$, then the wave-packet of the spatially localized particle splits, corresponding to the two spin components, along the direction of inhomogeneity of the magnetic field, into two wave-packets which  evolve along two separate paths, arriving at  two separate positions on the screen. On hitting the screen, which is believed to be a classical object, the state of the particle reduces to one of the two positions on the screen, with probabilities $\cos^2\frac{\theta}{2}$ and $\sin^2\frac{\theta}{2}$, respectively. Now suppose that the particle is massive enough to behave like a classical particle. How does decoherence explain the emergence of classicality of the particle in this experiment? }

The particle in this picture is considered to be interacting with the environment while it traverses the magnetic field. The reduced density matrix of the particle, obtained after tracing out the environment, has an almost diagonal form, with the two components corresponding to  spatially localized probability distributions, with weights $\cos^2\tfrac{\theta}{2}$ and $\sin^2\tfrac{\theta}{2}$, respectively \cite{Venugopalan_Kumar_Ghosh_1995}. The time evolution of this density matrix corresponds to two trajectories. Thus, according to decoherence, the emergent ``classical" behavior will show the particle following two trajectories, corresponding to spin states $\qup,\qdwn$, with different probabilities. This is at variance with what we expect from the classical laws of motion.

%\subsection{The Stern-Gerlach experiment in the Many-Worlds Interpretation}\label{mwi}

An interpretation of quantum mechanics which is popular among many, and does not invoke any wavefunction collapse or reduction, is the \emph{many worlds interpretation} \cite{mwi1973}. According to this interpretation, the particle in the Stern-Gerlach experiment splits into a superposition of two wave-packets, correlated with the two eigenstates of the $z$ component of the spin. The two wave-packets, and the spin states associated with them, are believed to belong to \emph{two independent worlds}. In each world, the particle is localized in position and has a distinct value of the $z$ component of the spin. However, in the many worlds interpretation too, the classically expected path of the particle is nowhere in the picture. 

\subsection{Need for Stern-Gerlach experiment in macroscopic domain}\label{expts}

The picture that emerges from the preceding analysis is that the classically expected path of the particle can be explained via the effect of self-gravity. On the other hand, it cannot be explained by decoherence and the many worlds interpretation of quantum mechanics. The question that naturally arises then is, if the Stern-Gerlach experiment is carried out with particles which are classical, does nature really show the classically expected path of the particle? This question can be settled by experiments.
The Stern-Gerlach experiment has traditionally been carried out with light particles which are expected to follow quantum dynamics. Such experiments bring out the space quantization of angular momentum. Magnetic experiments have been carried out with large atomic clusters, and nano-size particles \cite{Cobalt1991,Arndt2022}. However, in such experiments the magnetic moment of the atomic cluster is also large, and the space quantization may not be apparent. In order to experimentally probe the question addressed above, one needs Stern-Gerlach experiment with particles in the macroscopic domain, but with a small magnetic moment, ideally a spin-1/2 or spin-1. Coming up with large particles with small, stable magnetic moments may be challenging as spin relaxation is often observed in such situations
\cite{Amirav_Navon_1983}.

Stern-Gerlach experiments with almost macroscopic particles, but with small spin, can settle the question if space quantization, as observed in the Stern-Gerlach experiment, survives in the macroscopic domain (in terms of mass), or is it not seen as classical dynamics would tell us. If indeed the space quantization disappears in the macroscopic domain, it would indicate a success for the self-gravity model governed by the S-N equation. It would also pose difficult questions for some theories of emergence of classicality, like decoherence and many worlds interpretation, as these theories cannot explain the classically expected result.

As seen from our results, the effect of self-gravity in the Stern-Gerlach experiment are visible
for $\tilde{m} \sim 0.6$ and $\tilde{t} \sim 10$. From (\ref{eq:scalefactors}) one can see that $m_r$
has a $\sigma_r^{-1/3}$ dependence whereas $t_r$ has a $\sigma_r^{5/3}$ dependence.
This implies that if one chooses a small $\sigma_r$, one would see a noticeable self-gravity effect for large mass, after a
short time evolution. Thus $\sigma_r$ can be chosen such that it gives an experimentally feasible mass of the particle which
will pass through the Stern-Gerlach field in a time
of the order of $t_r$. For example, choosing $\sigma_r=0.371$ nm leads us to $m_r=46.05\times 10^9$ u and $t_r=0.1$ s. This implies that a particle of mass $27.63\times 10^9$ u ($\tilde{m}=0.6$), passing through a suitable inhomogeneous magnetic field, will show the effect of self-gravity, and emergent classical behavior in a time $t=1$ s ($\tilde{t}=10$). The good part here is that, unlike the interference experiments \cite{sahoo2022} one need not worry about maintaining any coherent superpositions. The only challenge, we believe, would be coming up with a particle of such high mass, but having a spin-1/2. Our simulations show that any particle below a mass of about $5\times 10^9$ u will not show any effect of self-gravity in a Stern-Gerlach experiment.

%\begin{figure}[b]
%     \centering
%    \includegraphics[width=0.8\columnwidth]{separation_vs%_time_different_mass.pdf}
%    \caption{Difference between the quantum path and S-N %path of the maximum of the probability distribution,  for %different masses}
%    \label{fig:separation-vs-mass}
%     \end{figure}
     
\section{Conclusion}

Including semiclassical gravitational self-interaction into the Schr\"odinger equation yields the interesting outcome of emergence of classicality for high masses in the Stern-Gerlach experiment. This is a feature that is amenable to testing in the laboratory, were it feasible to create spin-1/2 particles of high enough mass.
The emergence of classicality in this picture is more appealing  than decoherence-like pictures where a single classical outcome does not emerge, and one always ends up with two possible outcomes with different probabilities.

The question which remains is whether nature shows the expected classical outcome for large mass particles, or does it continue to follow standard quantum mechanics all the way to the classical regime. It is not clear if the classically expected outcome in a Stern-Gerlach experiment can be explained by any linear theory. Thus this experiment may turn out to be a crucial one in deciding if nonlinear theories like the S-N equation should be retained or discarded. If the experiments do not show the classically expected result even for classical particles, it would mean that  self-gravity plays no role in the emergence of classicality. If, on the other hand, the experiments show that for large masses the classically expected result is recovered, it would indicate that there is something lacking in the linear quantum mechanics, including the ideas of decoherence and the many worlds interpretation.

We believe this experimental test should be easier to implement compared to the interference experiments which are being attempted on the mesoscopic scale \cite{arndt2014testing,Fein2019,Bassi_Cacciapuot_2022}.

\phantom{
* Rules out the many worlds interpretation as well.
* experimental regime: value of mass, magnetic field and $\sigma_r $}

\phantom{The macroscopic theory that describes gravity is a nonlinear in nature as matter describes the geometry of spacetime and the geometry decides how the matter should move. In contrast, the microscopic theory that describes the dynamics at very small length scale is a linear theory. If the S-N equation could be proved as  describing physical phenomena after experimental tests, we could have both the fundamental theories that are nonlinear in nature. As we showed, for small mass range the nonlinear S-N equation describes the physical phenomenon exactly as quantum mechanics does and for large mass it predicts classical behaviour, quantum theory as we know could be considered as a special case of the nonlinear theory. For high mass range the deviation of the trajectories from free case, emergence of classicality, the deviation of the straight line path for fringe width vs inverse mass in two slit interference are the new emerging phenomena that should be tested in experiments and if found true then it could put some light on the theory of quantum gravity.  }


\section*{Acknowledgements}

This work was partially supported by the Department of Science and Technology, India through the grant DST/ICPS/QuST/Theme-3/2019/Q109.
SKS would also like to thank Dr. Ashutosh Dash for fruitful discussions related to the work and N K Patra for the help in 3D plotting. 

\bibliographystyle{ieeetr}
\bibliography{Reference.bib}
\end{document}
