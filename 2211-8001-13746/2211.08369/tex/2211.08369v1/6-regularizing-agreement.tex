\section{Improving Agreement Through Regularization}
\label{sec:regularizing-agreement}

Now that we know on which instance categories models tend to disagree on, our next goal is to attempt to fix these disagremeents.
Most saliency methods are based on gradient information -- either directly using gradient information \cite{denil2014extraction,sundararajan2017axiomatic} or modifying the rules of backpropagation \cite{bach2015pixel,shrikumar2017learning}.
It is our hypothesis that because easy-to-classify instances, in particular, are correctly classified with high confidence from the first time the model encounters them, they receive small updates from backpropagation in subsequent epochs.
We believe this causes the local representation space around that concrete instance to possibly be sharp -- which is problematic for methods that leverage gradient-based information as each incorporates different notions of locality.
To alleviate these issues, we leverage two methods that have been shown to improve faithfulness of attention-based explanations in conicity \cite{mohankumar2020towards} and tying \cite{tutek2020staying}.
Our motivation is that although the cross entropy loss will still produce little signal to the model, the signal from the regularization term will be enough to sufficiently explore the local space, making it smoother and thus increasing agreement.

\todo{Popuniti tekst bulleta}
\begin{enumerate}
    \item Describe regularization methods
    \item Results: improved agreement (on average)
    \begin{itemize}
        \item In addition to graphs, also add improvement <on average> for each reg. method wrt. baseline
    \end{itemize}
    \item Results: improved agreement wrt. cartography
    \item Result analysis
\end{enumerate}


\begin{table}
\small
\caption{Top $F_1$ score across epochs \todo{caption}}
\label{tab:repr_dbert}
\begin{center}
\begin{tabular}{lrrrr}
\toprule
& & Non-reg & Conicity & Tying \\
\midrule
\multirow{4}{*}{\rotatebox[origin=c]{90}{\textsc{dbert}}}
& SUBJ & $.93_{.01}$ & $.90_{.02}$ & $.93_{.00}$ \\
& SST & $.83_{.00}$ & $.83_{.01}$ & $.82_{.01}$ \\
& TREC & $.92_{.01}$ & $.92_{.01}$ & $.91_{.01}$ \\
& IMDB & $.86_{.01}$ & $.86_{.01}$ & $.88_{.00}$ \\
\midrule
\multirow{4}{*}{\rotatebox[origin=c]{90}{\textsc{jwa}}}
& SUBJ & $.92_{.00}$ & $.90_{.00}$ & $.89_{.00}$ \\
& SST & $.78_{.04}$ & $.76_{.02}$ & $.78_{.02}$ \\
& TREC & $.89_{.02}$ & $.86_{.01}$ & $.89_{.01}$ \\
& IMDB & $.89_{.00}$ & $.88_{.00}$ & $.86_{.00}$ \\
\bottomrule
\end{tabular}
\end{center}
\end{table}