\section{Summary}
\label{sec:summary}

% \meiyi{replace this para with a 'summary':}
% This paper focuses on exploring and analyzing the performance of our model and exploiting current technologies. Furthermore, we believe this area has many potentials, including 1. using the self-supervised method because of the limited data collection, 2. unsupervised learning to generate features that professionals could use, 3. reinforcement learning method to recommend different sets of exercises to perform depends on the users' short-term and long-term goals, and 4. creating an innovative method to segment exercises of different kinds through deep learning methods dynamically. 

% \meiyi{below is fine}
We develop an innovative system, PhysiQ, to quantitatively digitalize the quality of exercises through new metrics on a commodity smartwatch. We scrutinize and verify that different metrics and exercises have unique characteristics that can be recognized and understood by our deep learning model. By developing such a model based on Siamese Neural Network with additional spatiotemporal representation encoding, our model can achieve 95 percent R-square correlation and 90 percent accuracy in classification. Moreover, a comprehensive evaluation and user studies are performed to show the effects on our metrics in range of motion, stability, and repetition. 
The end goal is to improve the prediction and assessment of people who needs therapy to improve their quality of life. In addition, we envision that current technologies and relevant professions can be benefited from it by expanding usability, generalizability, and model learnability. By combining deep learning and the physical therapy method, we believe that our framework is the tool to lead people to improve their quality of life.