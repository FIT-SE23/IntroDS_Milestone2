% \pgfplotsset{width=7cm,compat=1.8}
\begin{figure}
    \centering
    \includegraphics[width=\textwidth]{Data_figures/Figure2.png}
% \begin{tikzpicture}
% \begin{axis}[
%     ybar,
%     title= \change{LOOCV Accuracy  Among Different Models for Classifying Range of Motions Labels},
% % 	x tick label style={
% % 		/pgf/number format/1000 sep=},
% 	xlabel=subject ID,
% 	ylabel=Accuracy Percentage,
% xtick={1,2,3,4,5,6,7,8,9,10,11,12,13,14,15,16,17,18,19,20,21,22,23,24,25,26,27,28,29,30,31},
% ytick={ 0, .25,  .5, 1.0},
% % axis equal,
% bar width=1.5pt,
% height=6cm,width=16cm,
% % 	enlargelimits=.1,
% xmin=0,xmax=31.8, ymin=0, ymax=1,
% 	legend style={at={(0.5,-.35
% 	)},
% 	anchor=center,legend columns=-2},
% % 	nodes near coords,  
% %     nodes near coords align={vertical},  
% % 	ybar interval=.4,
% ]
% \addplot[draw=none, fill=red!55]  
% % python .\classification.py --exercise e1 --metrics rom  --lr .001 --hidden_size 256 --epochs 50 --dropout .2 --filename PHYSIQ_E1
% 	coordinates {
% (1, 0.9)
% (2, 0.98)
% (3, 0.98)
% (4, 0.92)
% (5, 0.92)
% (6, 0.82)
% (7, 0.96)
% (8, 0.94)
% (9, 0.96)
% (10, 0.88)
% (11, 0.9)
% (12, 0.76)
% (13, 0.94)
% (14, 0.98)
% (15, 0.9)
% (16, 0.96)
% (17, 0.88)
% (18, 0.64)
% (19, 0.9)
% (20, 1.0)
% (21, 0.96)
% (22, 0.86)
% (23, 1.0)
% (24, 0.86)
% (25, 0.84)
% (26, 1.0)
% (27, 0.88)
% (28, 0.78)
% (29, 1.0)
% (30, 0.94)
% (31, 0.98)


% };
% % python .\classification.py --exercise e1 --metrics rom  --lr .001 --hidden_size 256 --epochs 50 --dropout .2 --filename BASELINE_LOG_E1 --baseline log

% \addplot[draw=none, fill=green!30]
% 	coordinates {
% (1, 0.22)
% (2, 0.58)
% (3, 0.82)
% (4, 0.96)
% (5, 0.84)
% (6, 0.88)
% (7, 0.56)
% (8, 0.9)
% (9, 0.9)
% (10, 0.7)
% (11, 0.92)
% (12, 0.68)
% (13, 0.68)
% (14, 0.94)
% (15, 0.98)
% (16, 0.7)
% (17, 0.48)
% (18, 0.32)
% (19, 0.82)
% (20, 0.76)
% (21, 0.92)
% (22, 0.6)
% (23, 0.62)
% (24, 0.78)
% (25, 0.64)
% (26, 0.88)
% (27, 1.0)
% (28, 0.58)
% (29, 0.94)
% (30, 0.82)
% (31, 0.8)
% };
% \addplot[draw=none, fill=blue!30]
% coordinates {%python .\classification.py --exercise e1 --metrics rom  --lr .001 --hidden_size 256 --epochs 50 --dropout .2 --filename BASELINE_CNN_E1 --baseline cnn
% (1, 0.34)
% (2, 0.6)
% (3, 0.78)
% (4, 0.84)
% (5, 0.64)
% (6, 0.82)
% (7, 0.48)
% (8, 0.82)
% (9, 0.9)
% (10, 0.66)
% (11, 0.9)
% (12, 0.7)
% (13, 0.62)
% (14, 0.84)
% (15, 0.92)
% (16, 0.62)
% (17, 0.7)
% (18, 0.38)
% (19, 0.8)
% (20, 0.72)
% (21, 0.88)
% (22, 0.5)
% (23, 0.64)
% (24, 0.76)
% (25, 0.6)
% (26, 0.76)
% (27, 0.92)
% (28, 0.54)
% (29, 0.88)
% (30, 0.54)
% (31, 0.68)

%  };
% \addplot[draw=none, fill=yellow!50] 
% 	coordinates {%python .\classification.py --exercise e1 --metrics rom  --lr .001 --hidden_size 256 --epochs 50 --dropout .2 --filename BASELINE_RNN_E1 --baseline rnn
% (1, 0.48)
% (2, 0.58)
% (3, 0.72)
% (4, 0.66)
% (5, 0.62)
% (6, 0.52)
% (7, 0.76)
% (8, 0.36)
% (9, 0.74)
% (10, 0.44)
% (11, 0.56)
% (12, 0.6)
% (13, 0.62)
% (14, 0.8)
% (15, 0.6)
% (16, 0.64)
% (17, 0.52)
% (18, 0.24)
% (19, 0.4)
% (20, 0.8)
% (21, 0.52)
% (22, 0.38)
% (23, 0.56)
% (24, 0.42)
% (25, 0.32)
% (26, 0.68)
% (27, 0.48)
% (28, 0.32)
% (29, 0.32)
% (30, 0.56)
% (31, 0.56)

%  };
	
% \legend{PhysiQ, Logistic Regression, CNN,LSTM}

% \end{axis}
% \end{tikzpicture}
\caption{This figure shows the accuracy of our model with output of classification of \textit{range of motion}. As showed in this figure, our PhysiQ (shown in red) performs mostly better than the other models in this diagrams with a percentage accuracy on y-axis.}
\label{figure:loocv_class}
\end{figure}