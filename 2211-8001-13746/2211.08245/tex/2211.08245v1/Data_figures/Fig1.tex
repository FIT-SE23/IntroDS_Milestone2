% \pgfplotsset{width=7cm,compat=1.8}
\begin{figure}
    \centering
    \includegraphics[width=\textwidth]{Data_figures/Figure1.png}
% \begin{tikzpicture}
% \begin{axis}[
%     ybar,
%     title= \change{LOOCV Different Models for Similarity Comparison of Range of Motions},
% % 	x tick label style={
% % 		/pgf/number format/1000 sep=},
% 	xlabel=subject ID,
% 	ylabel=R-Squared,
% xtick={1,2,3,4,5,6,7,8,9,10,11,12,13,14,15,16,17,18,19, 20,21,22,23,24,25,26,27,28,29,30,31},
% ytick={-.5,  -.25, 0, .25,  .5, 1.0},
% % axis equal,
% bar width=1.5pt,
% height=6cm,width=15cm,
% % 	enlargelimits=.1,
% xmin=0, xmax=31.7, ymin=-.4, ymax=1,
% 	legend style={at={(0.5,-.35
% 	)},
% 	anchor=center,legend columns=-2},
% % 	nodes near coords,  
% %     nodes near coords align={vertical},  
% % 	ybar interval=.4,
% ]
% \addplot[draw=none, fill=red!50]  
% %PhysiQ with attention
% 	coordinates {
% 	(1, 0.907)
% (2, 0.97)
% (3, 0.975)
% (4, 0.936)
% (5, 0.913)
% (6, 0.936)
% (7, 0.874)
% (8, 0.939)
% (9, 0.984 )
% (10, 0.87)
% (11, 0.964 )
% (12, 0.744 )
% (13, 0.804)
% (14, 0.916 )
% (15, 0.986 )
% (16, 0.926)
% (17, 0.8)
% (18, 0.886)
% (19, 0.862)
% (20, 0.997)
% (21, 0.975)
% (22, 0.793)
% (23, 0.987)
% (24, 0.959)
% (25, 0.975)
% (26, 0.991)
% (27, 0.791)
% (28, 0.641)
% (29, 0.985)
% (30, 0.922)
% (31, 0.953)
% };
% \addplot[draw=none, fill=green!30] %PhysiQ without attention
% 	coordinates {
% 	(1, -0.0382) 
% (2, 0.575 ) 
% (3, 0.838 ) 
% (4, 0.853 ) 
% (5, 0.209 ) 
% (6, 0.626 ) 
% (7, 0.763 ) 
% (8, 0.732 ) 
% (9, 0.131 ) 
% (10, 0.631 ) 
% (11, 0.778 ) 
% (12, 0.695 ) 
% (13, 0.191 ) 
% (14, 0.836 ) 
% (15, 0.908 ) 
% (16, 0.773 ) 
% (17, -0.0739 ) 
% (18, -0.273 ) 
% (19, 0.696 ) 
% (20, 0.864 )
% (21, 0.765 )
% (22, -0.13)
% (23, 0.692 )
% (24, 0.751 )
% (25, 0.474 )
% (26, 0.878 )
% (27, 0.851 )
% (28, 0.423 )
% (29, 0.564 )
% (30, 0.448 )
% (31, 0.826 )
% };
% \addplot[draw=none, fill=blue!30]
% 	coordinates {
% (1, 0.373 ) 
% (2, 0.717 ) 
% (3, 0.61 ) 
% (4, 0.739 ) 
% (5, 0.798 ) 
% (6, 0.734 ) 
% (7, 0.476 ) 
% (8, 0.311 ) 
% (9, 0.842 ) 
% (10, 0.683 ) 
% (11, 0.409 ) 
% (12, 0.489 ) 
% (13, 0.334 ) 
% (14, 0.659 ) 
% (15, 0.567 ) 
% (16, 0.478 ) 
% (17, 0.686 ) 
% (18, 0.404 ) 
% (19, 0.584 )
% (20, 0.655 )
% (21, 0.489 )
% (22, 0.762 )
% (23, 0.829 )
% (24, 0.749 )
% (25, 0.73 )
% (26, 0.866 )
% (27, 0.713 )
% (28, 0.353 )
% (29, 0.848 )
% (30, 0.601 )
% (31, 0.528 )};
% \addplot[draw=none, fill=yellow!50]
% 	coordinates {
% (1, -0.0227) 
% (2, 0.0596) 
% (3, -0.0883) 
% (4, -0.112) 
% (5, -0.104) 
% (6, -0.006) 
% (7, 0.0840) 
% (8, -0.0143) 
% (9, 0.0398 ) 
% (10, -0.0391 ) 
% (11, 0.145 ) 
% (12, -0.293 ) 
% (13, -0.0352 ) 
% (14, 0.183 ) 
% (15, -0.0795 ) 
% (16, -0.177 ) 
% (17, -0.0256 ) 
% (18, 0.0817 ) 
% (19, -0.112 )
% (20, 0.0239 )
% (21, 0.0962 )
% (22, -0.155 )
% (23, 0.156 )
% (24, -0.107 )
% (25, -0.00873 )
% (26, -0.00594 )
% (27, 0.291 )
% (28, 0.0636 )
% (29, -0.0706 )
% (30, 0.0566 )
% (31, -0.3)};
	
% \legend{PhysiQ, RNN, SimCLR,VGG11}


% % python .\siamese_cross_validation.py --exercise sa --metrics rom  --lr .001 --hidden_size 256 --epochs 25 --dropout .2 --filename BASELINE_RNN_E1_LOOCV_ROM_REP1_AUG_14 --baseline rnn --batch_size 4096
% % python .\siamese_cross_validation.py --exercise sa --metrics rom  --lr .001 --hidden_size 256 --epochs 25 --dropout .2 --filename BASELINE_CNN_E1_LOOCV_ROM_REP1_AUG_14 --baseline cnn --batch_size 4096

% \end{axis}
% \end{tikzpicture}
\caption{This figure demonstrates that our framework, PhysiQ, outperforms many other architectures by capturing the spatiotemporal information. As shown above, the red line represents our framework, PhysiQ with R-Squared result on y-axis. Noted, the higher the R-Squared, the better the result as it explains how fitted our model is given the unseen data. Moreover, interestingly, as shown in green line, vanilla RNN captures the temporal information very well but it has problems of generalizability as in subject ID 1 while some of subjects the RNN model can predict very well.}
\label{figure:loocv_similar}
\end{figure}