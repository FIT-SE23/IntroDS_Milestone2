\newpage
% \Large
\noindent\textbf{Response Letter} \\
\label{rev:response}
We truly appreciate all the valuable comments from the reviewers. We have done a major revision on the paper to address all the comments. We highlighted the revised parts using blue text in the paper. 
To answer the reviewer’s questions, we write a response to each comment below. 

Additionally, we summarize the changes below:
\change{
\begin{itemize}
    \item We significantly increased our collected dataset. (1) We nearly doubled the participants in all exercises and added a third exercise, forward flexion. (2) The third exercise underwent the same procedure and evaluation as the first two, and significantly outperformed the baselines, which further confirmed the generalizability of our model, PhysiQ. (3) We provided figures and explanation on why we choose this exercise, forward flexion in Fig. \ref{fig:Exercises}, Page \pageref{fig:Exercises}. (4) With a total of three exercises and three metrics, we expanded the age range between 19 and 44 with more survey completion. 
    \item (1) Based on our in-depth analysis of muscular system and difference among arm-spans, we appended a new ground truth section to explain our labeling procedure for the range of motion and stability. We explained the reasoning for how we get stability and range of motion in Sec. \ref{subsec:solution:digitalizingExerMetrics}, Page \pageref{subsec:solution:digitalizingExerMetrics}. (2) With the modification of stability ground truth, we conducted our evaluation on all exercises. As a result, in all three exercises metrics, our model significantly outperformed all baselines and showed that with a sophisticated design of metrics, our model learns the underlying structure of the signals.
    \item We added a problem formulation section to define our problem of supervised learning. We explained the classification and similarity comparison problems separately in Sec. \ref{subsec:solutions:problem}, Page \pageref{subsec:solutions:problem}. 
    \item We improved our mobile application to demonstrate the structure of the PhysiQ framework with additional screenshots in Fig. \ref{fig:PhysiQ_GUI}, Page \pageref{fig:PhysiQ_GUI}. We also showed the potential of our framework in real-life scenarios between patients and doctors.
    \item We re-conducted the evaluation section with expanded datasets and re-evaluated the results with extensive tables and figures. Our model demonstrated the generalizability in different participants, metrics, and exercises. PhysiQ also outperformed the baselines as a backbone to classification problem of range of motion. 
    \item We carefully checked the manuscript and adjusted the description accordingly, as suggested by the reviewers.
\end{itemize}
}
\vspace{0.5cm}
\noindent\textbf{1AE's meta-review (Reviewer 2)}
\begin{itemize}
    \item Increasing the experimental data sample size 
    
    \change{We collected 11 more participants data, aged between 19 and 44. Additionally, we also added another exercise, forward flexion, to be part of this work.}
    \item Formal problem definition and important representation: we believe that using deep learning methods, it can helps patients and therapists to understand the quality of exercises through defined metrics and fathom the nuances of exercises they performed. 
    
    \change{We added a section of formal problem definition in Sec. \ref{subsec:solutions:problem}, Page \pageref{subsec:solutions:problem}}
    \item New results on "range of motion" and "stability" 
    
    \change{New results and evaluation has been updated in Sec. \ref{sec:evaluation}. We re-ran most of the evaluation because of the expansion of the dataset.}
    \item Clear description of the ground truth 
    
    \change{We added a ground truth section in Page \pageref{subsec:solution:digitalizingExerMetrics}, Sec. \ref{subsec:solution:digitalizingExerMetrics}.}
    \change{And we modified the metrics of \textit{stability} to make our model more accessible.}
    \item Improving the writing quality

    
    \change{We re-checked and re-read everything. Thank you for all the suggestions.}
\end{itemize}


\large
\noindent\textbf{2AE (Reviewer 1)}
\begin{itemize}
    \item The manuscript has some detailed description errors
    
    \change{We re-read and correct as many errors as we think we have made. Thank you for the suggestions!}
    \item Is the dataset too small? Since smartwatch is a convenient device to collect data, it should collect more datasets of participants who are injured or increase normal participants who are not injured and simulate the stability. 
    
    \change{We collected 12 more participants and re-defined the metrics for stability to be useful across all dataset.}
    \item There is a lack of description of shape of input data and problem definition in the methodology. Problem definition with data dimension transformation will make the process of spatio-temporal feature representation more specific.  
    
    \change{We clarified our dimensions, problem definitions, and methodology in Sec \ref{sec:data collection} and Sec \ref{sec:solutions}}
    
    \item Following the weak point 1, for instance, there is a grammar error in the
    caption of Fig. 3: “Exercises Metrics: we first identify the type of exercise to
    perform and that model measures the quality of.”. Additionally, The description of
    Table. 2 are mis-presented. More details need to be check in the manuscript.
    
    \change{We fixed it accordingly, and re-visited all the manuscripts. Thank you for your suggestion!}
    \item Following the weak point 2, considering that people in Previously Shoulder
    Injures are not easy to find, this can increase the dataset as much as possible.
    
    \change{We re-conducted and re-visited the data collection process, and collected 12 participants with 3 exercises. Based on that, we ran the evaluation with all three metrics of range of motion, stability, and repetition.}

\end{itemize}
\large

\large
\noindent\textbf{Reviewer 3}
\begin{itemize}
    \item How does the method distinguish "range of motion" and "stability"? Does it utilize the energy method? If yes, can you show the experiment result of it? If not, please clarify this. 
    
    \change{It does not utilize the energy method; we clarified it in the ground truth section in Page \pageref{subsec:solution:digitalizingExerMetrics}, Sec \ref{subsec:solution:digitalizingExerMetrics}. Energy method is used to segment repetitions in a session. Range of motion is labeled with physical movement, and stability is measured through coefficient of variation.}
    \item The description of the Spatio-temporal Feature Representation is not clear. What is signal exercise and anchor exercise? Are they two consecutive exercises?
    
    \change{The signal exercise is user's new exercise that is been done and used to compared the "perfect" anchor exercise. These two are not consecutive exercises.}
    \item How is the ground truth of similarity score defined? Please provide an example for clarification.
    
    \change{The ground truth of similarity score, stability and range of motion is defined in Page \pageref{eq:rom} and Page \pageref{eq:stb} in Equation \ref{eq:rom}, \ref{eq:stb}.}
    \item In the classification experiment (section 5.4.2), there is no figure for stability. Please provide the description and figure for it. 
    \change{We modified our ground truth and stability is now between 0 and 1. We have shown our result of three metrics in table \ref{table:sa_701020}. }
    \item The sample size is small and the range of age is also small. It will be more representative if the dataset is larger. 
    
    \change{We increased our dataset by adding 12 participants and add another exercise to evaluate our model.}
\end{itemize}

\large
\noindent\textbf{Reviewer 4}
\begin{itemize}
    \item Paragraph 1 of Introduction: it would be better to motivate this work and explain the impacts by citing some numbers, e.g., how many people could benefit from PT. 
    
    \change{We added such motivation in the introduction. Thank you for your suggestions!}
    \item line 38, this claim came out of nowhere, i.e., why do we start directly from deep learning? Why don't naive solutions work here? 
    
    \change{We re-structured and re-phrased the motivation section along with the second part of the introduction due to inconsistency of the context.}
    \item Section 1.1, it's unclear how this work differs from the existing works mentioned in previous paragraphs, e.g., no related work is cited in the section.  
    
    \change{We added the reference of works and added our work is focus on to let patients' see their quality of exercises through muscular metrics.}
    
    \item line 69-70, grammar mistake: "there do not have" 
    
    \change{We fixed it accordingly. Thank you for the correction.}
    \item line 79, similarly, there is no justification for starting with deep learning models. 
    
    \change{We re-structured and re-phrased the motivation section along with the second part of the introduction due to inconsistency of the context.}
    \item It would be good to mention how does this work address these mentioned challenges.
    
    \change{We re-stated how our work helps to target the aforementioned challenge in the introduction. The list of contributions corresponds to the challenges. }
    \item In introduction, there should be a lot of motion tracking papers at least, which may not be focusing on professional metrics but they should be mentioned.
    
    \change{We mentioned more papers of HAR in the introduction as suggested, including ones on how to improve the quality of life through smart HAR.}
    \item  Line 97, it's hard to justify this novelty since there are already mature products doing this such as Apple Watch or Switch. They may have not have paper published but there should be plenty of patents. 
    
    \change{We double checked that there is no product on recognizing the quality of exercises but number of steps, calories, and potentially arm position. But these results do not consider as quality of exercises through muscular metrics. First bulletin point in contribution in introduction updated to mention our work is not just about similarity, but also a comparison of quality of metrics.}
    \item The contribution list is not well organized, e.g., collecting data is trivial considering the size of the data. 
    
    \change{To address this comment, we added another 12 participants across the existing two types of exercises, and added one additional exercise, forward flexion. The challenge and novelty of our data collection is that we collect and annotate the data with different levels of quality per different metrics.}
    \item line 197, the muscular system is not even mentioned in the introduction? It should be since it seems to be a major novelty of this work.
    
    \change{Thank you very much for this detailed suggestion. We presented our idea of the muscular system in Sec. \ref{sec:solutions}, and we also added muscular system as a basis to our exercises metrics in the contribution list. }
    \item  line 310, "we use our intuition to decide on the cutting point for segmentation", this is so not convincing. 
    
    \change{We explained accordingly in the Sec. \ref{subsec:solution:digitalizingExerMetrics}, that we used the previous and next repetition as a neighbor energy to amplify the cutting point.}
    \item line 314, how are the labels collected? 
    \change{We clarified it in the ground truth section in Page \pageref{subsec:solution:digitalizingExerMetrics}, Sec \ref{subsec:solution:digitalizingExerMetrics}.  Range of motion is labeled with physical movement, and stability is measured through coefficient of variation.}
    
    \item line 317, is "shoulder abduction exercise" used as an example or is it the only exercise considered? 
    
    \change{We used shoulder abduction as the primary example to test our model but added two more exercises to assess more.}
    
    \item line 345, figure 6 is far away from this line.
    
    \change{We modified the location of figure 6 to be closer to the actual paragraph.}
    \item line 346, citation for the mentioned similar work. Also, it is worth explaining the difference. 
    
    \change{We added citations and explained the differences in the later section in SNN.}
    \item line 354, CNN is used without justification.
    
    \change{We justified CNN as our spatial compression model to represent information in each sliding windows.}
    \item line 399, reference missing.
    
    \change{We added the reference of LSTM as suggested.}
    \item line 549, the two exercises should be mentioned in the introduction.
    
    \change{We mentioned all three shoulder exercises in the introduction section.}
    \item Section 5.4.1, the choice of baselines should be better explained at the beginning of the paragraph. 
    
    \change{We added our reasoning for RNN, CNN and SimCLR with its reference. RNN and CNN because RNN is known for arbitrary input with time-varied data, and CNN learns well in spatial and temporal features. Additionally, we choose SimCLR as third baseline because its architecture enables useful representation learning in contrastive learning.}
    \item Section 5.6, why not show the ablation for the attention layer?
    
    \change{We added an attention ablation study for our model.}
    \item line 725, how will you deploy and update your model to the app? E.g., based on Figure 1.
    
    \change{We envision to deploy our deep learning model into cloud server to support API call directly from our mobile application. It is also easy for update and version control.}
\end{itemize}