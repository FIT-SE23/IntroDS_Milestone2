\section{A Survey on User Experience}
\label{sec:survey}

To investigate PhysiQ's application in practice, we surveyed the participants who used our system for data collection. Users wear an Apple Watch with a connected iPhone with our PhysiQ apps installed during data collection. In the beginning, users have the exercise instruction on the phone to start. 
% Next, users tap the start button on the smartwatch, and the smartwatch counts 3 seconds down and gives a vibration as a signal to the users to start exercising. Afterward, they press the stop button to transfer the session data to the iPhone. 
PhysiQ on the phone visualizes the signal to see the repetition quality and feedback to the users (as shown in Fig. \ref{fig:PhysiQ_GUI}). 

After collecting the data, we distribute a questionnaire to all users. The questionnaire asks six questions on different scales. 
We designed the survey questions for three purposes. First, we get users' feedback on our current design (Q1). Secondly, we explore users' preferences in alternative platforms and recommendations to improve our current design (Q2 and Q3). Last, we gather users' demographic information (presented in Section \ref{sec:data collection}) and ask behavioral questions (Q4-Q6). We study the correlation between users' behaviors and the performance of our algorithms in measuring their activities. 
We attach the questions and summary of results below because we believe our findings are valuable for our future work and the research community. 

\begin{itemize}
    \item \textbf{Q1: How do you like the current feedback and recommendation system?} 
        \begin{itemize}
        \item scale of 1 to 5, with 1 being the lowest, and 5 being the highest
        \end{itemize}
    \item \textbf{Q2: If we have a different platform, what platform will you like to have recommendation system of the exercises on?}
        \begin{itemize}
        \item Smartwatch
        \item Smartphone
        \item Smartglasses such as VR, AR glass
        \end{itemize}
    \item \textbf{Q3: During what period of the exercise do you like such feedback? For example, for today's exercise session, you have 5 different exercises to perform and for each exercise, you have 10 repetitions of 5 sets? }
        \begin{itemize}
        \item After a set of exercises  (during this particular exercise, after 10 repetitions)
        \item After the particular exercise of 5 sets
        \item After the entire exercise session (after 5 exercises)
        \end{itemize}
    
     \item \textbf{Q4: During the time of collecting your exercise data, do you drink coffee or any caffeinated drinks regularly? If so, how often? }
        \begin{itemize}
        \item Every day
        \item A few times a week
        \item About once a week
        \item A few times a month
        \item Once a month
        \item Less than once a month
        \end{itemize}
    
    \item \textbf{Q5: During the time of collecting your exercise data, how many hours do you sleep? } 
        \begin{itemize}
        \item Time in hour
        \end{itemize}
        
    \item \textbf{Q6: This question is regarding your medical history and you do not need to specify the medication. At the time of collecting your exercise data, do you know and take any medication at the time that would affect your ability to perform exercise or activity? }
        \begin{itemize}
        \item Yes or no
        \end{itemize}
    \end{itemize}
    
Out of 31 participants, we have 27 participants who complete the follow-up survey. Overall, we have an average of 4.26 rating on how much the users like our system. Interestingly, we have 48.1\% of the users who want to have a recommendation on a smartwatch, 51.9\% of the users on a smartphone, and no one wants to use smart glass to get recommendation feedback. Additionally, 59.3\% of our users want to have their feedback after a set of exercises, 22.2\% after the particular exercises, and 14.8\% after an entire session. Lastly, one participant suggests they should be able to see the feedback anytime they want.

Moreover, we have 18.5\%, 33.3\%, 14.8\%, 18.5\%, 3.7\%, and 11.1\%, respectively, on the frequency of coffee consumption based on the answer order above. At the same time, on average, our participants sleep 7.63 hours with a minimum of 6 and a maximum of 9. Lastly, we only have 1 participant possibly on medication that affected their performance overall, but we do not find that medication was affecting the performance of our model testing on this participants' exercises.



    


