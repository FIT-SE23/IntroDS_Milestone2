\section{Related Work}
\label{sec:related work}
% \david{change the deep learning network for similarity comparison.  fundamental techniques, how does it this compare to state of art }
In this section, we present existing literature on state-of-the-art measuring the quality of activity and applications, and deep learning methods for these applications through similarity comparison and other methods.
% Human Activity Recognition (HAR).\\  % different approaches of Qualitative Assessment in different input sources

\subsection{Measuring Quality of Activity}
% \subsubsection{Human Activity Quality Recognition}
Though there does not exist a field of such, Human Activity Quality Recognition (HAQR) is a critical research question for real-world application. Therefore, we have gathered and scrutinized related works in state-of-the-arts. For example, one quality in exercises is repetition counting. Work, such as \cite{stromback2020mm}, focuses on multi-modality to provide a more accurate result of repetitions counting and exercise recognition. A fascinating work done by Radhakrishnan et al. focuses on using in-ear devices such as wireless headphones fusing with inertial measurement units to quantify insights and feedback in gym exercises \cite{radhakrishnan2019can}.

Additionally, one work uses smart speakers to analyze the duration, intensity, continuity, and smoothness of exercises at home \cite{xie2021hearfit}. However, such work requires people to have knowledge about exercises and some level of proximity to the actual devices. Additionally, IMUTube introduces how to simulate virtual IMU data from video \cite{kwon2020imutube}. Kwon et al. utilizes video to simulate IMU through a number of off-the-shelf computer vision and graphics techniques. Furthermore, Radhakrishnan et al. suggest a system that uses a magnetic accelerometer sensor. The device is mounted on the weight stack of a gym machine to infer exercise behavior using multiple machine learning models to identify the person, amount of weights, type of exercise, and mistakes \cite{radhakrishnan2021w8}.

% \subsubsection{Sleep Quality}
Like the quality of exercises, sleep quality analyzes humans' brain activity through Electroencephalography (EEG) signal. It categorizes the level of sleep activity in Rapid Eye Movement (REM), Non-REM, S2 (light sleep), and S3 (slow-wave sleep). Additionally, polysomnography is considered the standard methodology for detecting the sleep pattern with carefully analyzed records of epochs \cite{crivello2019meaning}. There are two leading technologies for sleep monitoring: 
portable and contact. Portable devices such as smartwatches or smartphones seamlessly collect users' data through passive means. Contact devices are medical-level tools to collect reliable data through less passive means \cite{crivello2019meaning}. 
 
Portable devices require stationary devices to record the daily routine of the subjects \cite{sleep:chang2018sleepguard, sleep:mehrabadi2020sleep, sleep:scherz2017heart, sleep:sun2017sleepmonitor,ma2017m}. These works include signals of accelerometer data to approximate respiration rate and heart rate through, for example, sound recording to validate sleep time and duration, and electrocardiogram (ECG) signal to proximate heart rate and distinguish threshold for sleeping and waking. Contact devices, which require direct contact with the subjects, are the cases of PPG, EEG, and Actigraphy. These assessment technologies have the main advantage of accurately sampling the human body's physiological and mechanical signals. However, such devices are perceived as obtrusive due to their limited portability and transparency. Sleep quality recognition and detection using contact devices have been well studied. One research targets whether having additional information such as age and sleep stage information can distinguish abnormality using the deep learning method on EEG signal data \cite{sleep:contact:van2019detecting}. Other research focuses on developing an automatic sleep staging method in EGG signals, in which the author proposes multi-epoch methods to segment and encodes the feature and re-concatenate to predict its sleep stage \cite{sleep:contact:li2021end}. Lastly, one aims to develop a sleep scoring toolbox with the competency of multi-signal processes, feature extraction, and classification and prediction but only using a simple logic-feature-based method to differentiate sleep stage \cite{sleep:contact:yan2019automatic}.

% \subsection{Qualitative Assessment}
% First of all, there has been papers that support that IMU has the ability to differentiate the quality of exercises in different exercises conditions \cite{IMU_evaluation_2013}. This present a single IMU has the ability to identify the conditions of poor techniques. Moreover, there are different type of works related to Qualitative Assessment, such as automatic segmentation to do repetition counting \cite{AUTOMATIC_SEGMENTATION_2018}, Motion sensor (Kinect V2) detecting joints in human skeleton to compare with a reference to output exercise performance using DTW and HMM \cite{HAGHIGHIOSGOUEI2020}, and 




\subsection{Deep Learning Models for Quality Measurements} 
% \subsubsection{Simaese Neural Network}
Siamese Neural Network (SNN), utilizing two identical networks, is commonly applied to compare if two subjects are same or not. 
% For example, image classification requires categorizing images into classes, determined by a similarity score. SNN can quantify the similarity between two instances using two or more identical sub-network sharing the same weights. 
Typical tasks include image classification, object recognition, and object tracking \cite{he2018twofold, dong2018triplet, shen2019visual, wang2018learning, guo2017learning, leal2016learning}. However, it only returns a binary result (i.e., True or False) without a quantitative measurement.  

Furthermore, recognizing similarities in 1-D signal data, such as radar, speech, and natural language process (NLP) \cite{govalkar2021siamese, mittag2020full, neculoiu2016learning, benajiba2019siamese}, has also gained many usages in different applications. 
An intriguing application stands out using semantic similarity between sentences, supplementing recurrent neural networks with synonymic encoding. Mueller et al. use LSTM to encode the different length inputs with its positional encoder to analyze the semantic similarity with an outstandingly high Pearson correlation of 0.8822 \cite{mueller2016siamese}. Furthermore, A deep dive evaluation of the SNN proposes a residual module to reduce learning biases caused by padding \cite{zhang2019deeper}. The determinant factors, such as stride, padding, and receptive field size (the size of the region that produces the feature), are crucial to the performance of the SNN. Additionally, Lawrence et al. suggest an innovative way of utilizing spatial and temporal aspects from videos to recognize human emotions is inspiring. \cite{lawrance2021emotion}. Lastly, Agrawal et al. use an inventive way to calculate articles' similarity distance for political stance using discrete labeling of relatedness \cite{agrawal2017cosine}.  

% \subsubsection{Contrastive Learning}
Additionally, it is worth mentioning a self-supervised method called contrastive learning. It is a technique to learn the general features of a dataset without labels by telling what data points are similar or dissimilar. The reason why this is important to us is that the similarity of the task is to recognize the similarity. However, the goal might differ for contrastive learning due to their limited dataset or no handcrafted labeling. Several interesting works related to SimCLR, SIMCLRv2, and MoCo (momentum contrast) are fine-tuning with a few labeled examples to achieve high accuracy \cite{chen2020big, he2020momentum, chen2020simple}. These works are essential in unsupervised learning fields and provide highly accurate and efficient solutions. Additionally, an audio similarity work is being introduced to assign high similarity to the audio segment from the same recording while assigning low similarity to different segments from di qfferent recordings \cite{saeed2021contrastive}.

% \david{try revise this: }
In summary, these systems and applications measure the quality of words, speech, sleep, and emotion. 
% In addition, a few papers attempt to approach the quality of exercise through machine learning. 
However, none of them standardizes the metrics of exercises through the muscular system. PhysiQ recognizes therapeutic exercises and digitalizes the exercises to provide feedback through the metrics and compare exercises using deep learning methods.
