\section{Discussion}
\label{sec:discussion}

\subsection{Applications}
\emph{Physical Therapy Home Assessment}. This application is intended to work for patients and people injured, postoperative, or mentally traumatized. While performing at-home exercises, our application can provide real-time feedback and assess the quality of the exercises to provide better interaction and supervision while clinics are not accessible. At the same time, we believe our model is capable of analyzing and predicting people who might suffer from a different illness or potential injuries. One example is that people with heavy usage of handcrafting might suffer from carpal tunnel syndrome, which is caused by pressure on the median nerve. 

\emph{Daily Exercises Assessment}. This application can also support working with people who enjoy exercise. People can benefit from it by closely assessing how they perform specific repetitions of exercises. Moreover, this application could also apply to people who are playing sports. We envision that our model can eventually support people who play sports like tennis to predict a player's direction, speed, or posture.


\subsection{Assumption and Limitation}
This paper discusses the potential of using a smartwatch to support users assess their exercises with a quantitative measure of the quality. However, there are a few assumptions made to support this application. First of all, we assume that the users' posture should have some level of decency. For example, suppose users perform the exercise of shoulder abduction while bending their neck or wiggling around as not in a straight form. In that case, the application likely can still give a good score on the exercises simply because the users might still perform the exercise correctly. Because the smartwatch cannot capture all the movement within the body, we do not have the luxury of analyzing all the posture. Secondly, 
we assume that the primary segmentation is performed upon accelerometer data, and its energy is only extracted from the accelerometer. Thirdly, 
we assume there is 5 level of quality of exercises in \textit{range of motion} in shoulder abduction. These are our metrics to digitalize from body movement to computational numbers. The five levels can vary based on different professionals' metrics, and our goal is to standardize a metric in our model that can measure and understand.


% \subsection{Limitation}
% PhysiQ is a new framework that has the potential to generalize to different tasks in various exercises. However, there are some limitations to our project. 
We notice that the energy plot does not have a clear pattern in some exercises for some subjects. Such limitation is due to our little knowledge regarding the data collection. Applying our energy plot when collecting data from the users could be more rigorous and use it as a checkpoint to verify if the subjects have correctly performed the exercises.
% Moreover, there is a problem of imbalanced data among the actual patients and healthy participants due to the limited resource and time. We do not have the luxury of collecting data from actual patients. Thus, our model is limited to testing only on the people who did not have injuries at the time. 
% We also observe that the participants' speed in performing the exercise can affect the performance of the models. One potential solution is to augment the dataset to expand or shorten the length of segments. Additionally, we define a metrics of \textit{range of motion} in scale of 30, 60, 90, 120, 150. Discrete classification can be a limitation due to the inconsistency of participants' movements. We observe and judge the participants' range of motion visually. However, we could standardize the method by using a regression method from a 3-D motion projectile. 
In the future work, we will improve our framework to better handle above assumptions and limitations.  