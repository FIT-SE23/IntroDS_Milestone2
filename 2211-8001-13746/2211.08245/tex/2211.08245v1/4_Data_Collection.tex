\section{Data Collection}
\label{sec:data collection}
% mean, std in demographic! and data collection section

\subsection{Type of Exercises}
% \meiyi{This section is confusing. We should just talk about the activities that we actually collect data from. You can talk about the other activities in the discussion/future work section.}
Selecting well-represented exercises is very crucial to our problem because the exercise itself should have the quality of repetitiveness, singularity (a defined beginning and ending), reconstructiveness (an exercise that can reshape part of a person's body as an improvement), and representativeness (can cover many parts of the muscles). Therefore, considering these factors, we conduct our evaluation of these three exercises, as shown in Fig. \ref{fig:Exercises}:
\begin{itemize}
    % \item Pendulum (Shoulder): Pendulum is an exercise that requires the subject to lean over on a table or chair with one hand pressing on the table for support. Subjects relax the injured arm, letting it hang straight down, and slowly swing the relaxed arm with your body in a circle for both directions. The benefits of this exercises are relaxation of the shoulder muscles and neck and allowing for passive range of motion in the shoulder joint. However, It is limited in the definition of singularity because it does not clarify the start and end of the recording for quantitative measurement.
    \item Shoulder Abduction: The shoulder abduction is an exercise that requires subjects to stand straight with both hands tucking on the side of the legs as the starting point. Subjects perform the exercise by raising the instructed arm to a certain degree of motion and straightening the arm. Once the subjects reach the stopping point, they drop their arm steadily, as it is similar to raising their arms. 
    \item External Rotation: The external rotation has two components. First, the upper arm (bicep and tricep) tucks in the armpit while rotating the arm externally and keeping the lower arm raised to about 90 degrees, perpendicular to the chest or abdominal. Secondly, the arm should move horizontally and stop at a certain degree of motion or to the full ROM, where the subject's shoulder should feel a sense of blocking by the joints. 
    \item Forward Flexion: The forward flexion is similar to shoulder abduction but different in direction. Subjects perform the exercise by raising the arm slowly forward, reaching the point of ROM, then consistently lowering the arm to the beginning position.
    % \item Side-lying Leg Raise (Lower Body): This is a type of leg exercise that requires patients to fully lie down on one side while keep the body straight. Their core and gluten strength are required to keep their form unchanged. While starting that, the instructed leg will slowly raise certain degree of motions and fall back down.
    % \item Standing Leg Lateral Raise (Lower Body): This exercise requires the subjects to stand straight and perform a side leg raise as the subjects tilting to the other side for balance. 
\end{itemize}

% \begin{figure}[hbt!]
%     \centering
%     \includegraphics[scale=.1]{Figures/ShoulderAbduction.jpg}
%     \caption{Visualization of Shoulder Abduction }
%     \label{fig:SystemStructure}
% \end{figure}

\begin{figure}%
    \centering
    \subfloat[\centering Shoulder Abduction]
    {{\includegraphics[scale=.15]{Figures/E1.jpg} }}%
    \qquad
    \subfloat[\centering External Rotation]
    {{\includegraphics[scale=.15]{Figures/E2.jpg} }}%
    \qquad
    \subfloat[\centering Forward Flexion]
    {{\includegraphics[scale=.15]{Figures/E3.jpg} }}%
    \caption{Three Exercises: The exercise on the top left is one repetition of shoulder abduction, the exercise on the top right is one repetition of external rotation, and the exercise on the bottom is forward flexion. Also noted is that we define greater equal than 150 degrees of motion as 150 on the shoulder abduction and forward flexion.}%
    \label{fig:Exercises}%
\end{figure}
We simplify our problem as a preliminary examination from the description we mentioned above. Each exercise has its advantage and weakness. As a result, shoulder abduction, external rotation, and forward flexion have aspects that we look for in exercises, including subjects not needing to lie down and wear a smartwatch on the leg to perform exercises in initial modeling. Additionally, we choose shoulder abduction as our first exercise to test our framework, PhysiQ, because shoulder abduction has all the factors we considered: repeating in sets, singular in the beginning and ending, reconstructing people's bodies, and targeting many muscles around the shoulder. It requires extensive time and resources to find the participants during the pandemic, and the quality of data is varied based on the subjects because we minimize how much we tell the subjects to perform the exercises while giving enough details on reaching different stop points of degrees in motions and instability. In order to perform the metric of \textit{range of motion}, \textit{stability}, and \textit{repetition}, we see how robust our model is and in what scenario it can handle and fail.

% \subsection{Preprocess}
% To learn every possible scenario and relationship between two sets of exercises with different \textit{range of motion} (ROM), \textit{stability}, and \textit{repetition}, we collect 5 different sets of repetition of 10 exercises, 3 different sets of stability exercises using resistance band. Then we have segmented it into one repetition pair using our novel energy method presented in Section \ref{sec:solutions}. We have created more than 1,700 of one repetition for the dataset of shoulder abduction. Something to be noted here in terms of how to simulate stability: stability is difficult to simulate for normal participants who are not injured. Normal participants are difficult to produce weakness and fatigue, typically similar to the unstable exercises. We attempt to use three different approaches: 
% \begin{itemize}
%     \item Exhaustive exercises: overloading the muscles by adding repetitions without weights to reach fatigue in muscle and creating an exhaustive repetition of exercises
%     \item Exhaustive weighted exercises: overloading the muscles by combining additional dumbbell weights with repetitions
%     \item Two sets of resistance bands
% \end{itemize}
% After carefully considering the ethical and convenient aspects of simulating instability in exercises, we decide to go with resistance bands; however, with one twist in using a resistance band in shoulder abduction, when the participants raise the arm/shoulder, the resistance increases along with the travel distance. However, there is a tendency to help the participants drop the arm as participants lower the arm. So a particular instruction is informed that participants should "resist" such force to keep a steady singularity in the exercise when lowering the arm.

\subsection{Data Statistics}
In total, we have 1550 segmented one-repetition exercises of \textit{range of motion (ROM)} and 1170 segmented one-repetition exercises of \textit{stability} for shoulder abduction. At the same time, we have 600 segmented one-repetition exercises of ROMs for external rotation. Additionally, the third exercise, forward flexion, has 650 segmented one-repetition data.
In total, we have collected 31 participants for all data collection in different evaluation periods.
In addition, we have 31 participants from shoulder abduction, 24 participants from external rotation, and 11 participants from forward flexion. The metrics of \textit{range of motion} are labeled as we collect the data, and the \textit{stability} is generated using our method as shown in Equation \ref{eq:gt-stb}. The metric of \textit{repetition} is utilized through our energy segmentation and merged based on the number of repetitions for evaluation. We can form any number of \textit{repetition} by combining the adjacent neighbors of segmented repetitions.

In SNN, we create pairs of inputs for similarity comparison in one repetition exercise of \textit{range of motion}, \textit{stability}, and \textit{repetition}. Therefore, we define the problem as no comparison between subjects, only the comparison of exercises within one particular subject at a time, because with the presumably perfect anchor exercise, it is a relative measure of the similarity of the signal exercise on a particular user.

%Additionally, we exclude a particular subject as validation and testing data so that model does not see the distribution. As a result, there are about \textbf{\change{34,000} training samples} and about \textbf{\change{3,000} validating and testing} by statistically doing a whole combination of all training datasets. Similarly, we have listed the dataset size in Table. \ref{table:datasetpair} \footnote{The annotated dataset on the quality of exercise will be published with the paper.}. 

% \begin{table}[h!]
% \centering
% \begin{tabular}{| c ||c | c | c |} 
%  \hline
%   Repetition & Train & Valid & Test  \\ [0.5ex] 
%  \hline
%  \hline
%  1 & \change{34,912} & 1,838 & 1,225\\
%   \hline
%  2 & \change{28,215} & 1,485 & 990 \\
%   \hline
%  3 & \change{22,230} & 1,170 & 780 \\
%   \hline
%  4 & \change{16,957} & 893 & 595 \\
%  \hline
% \end{tabular}
% \caption{Dataset Pair Information}
% \label{table:datasetpair}
% \end{table}


% \begin{table}[]
%     \centering
%     \begin{tabular}{||c|c||}
%     \hline
%          &  \\
%          & 
%     \hline
%     \end{tabular}
%     \caption{Caption}
%     \label{tab:my_label}
% \end{table}

\begin{table}[h!]
\centering

\begin{tabular}{||c || c | c |c |c||} 
 \hline
  Attribute & Male & Female & avg & std\\ [0.5ex] 
 \hline
 \hline
 Gender & 17 & 14&N.A.&N.A. \\
  \hline
 Age & 18-25 & 18-44 &22.32&4.77\\
  \hline
 Height (cm) &  167.6-190.5     & 149.8-182.9&173.6&10.27 \\
  \hline
 Weight (kg) &  54.4-108.8      & 45.3-88.9&67.31&15.90\\
  \hline
 Previously Shoulder Injures & 2 & 1&N.A.&N.A. \\
 \hline
\end{tabular}

\caption{Participants information}
\label{table:datastats}
\end{table}

% \textbf{Shoulder Abduction}:
% The mean of age for shoulder abduction is 22.21 and std is 2.57; the mean of height is 174.54 cm and std is 11.04 cm; the mean of weight is 70.01 kg and std 17.70 kg. Out of 19 participants, 7 are females and 12 are males.\\
% \textbf{External Rotation}:
% The mean of age for external rotation is 22.25 and std is 3.11; the mean of height is 174.4 cm and std is 13.36 cm; the mean of weight is 74.16 kg and std 21.87 kg. Out of 8 participants, 2 are females and 6 are males.