\section{Introduction}
\label{sec:introduction}
{Physical therapy} (PT) is a process in which patients regain their strength through exercises after surgery, incidents, or illness. It benefits patients by reducing pain, improving mobility, preventing further injury, and improving muscle balance. Patients undergo challenges, endeavors, and struggles with lasting benefits with well-prescribed instruction and supervision. 
Currently, there are more than 5.1 billion Years Lived with Disability (YLDs)\footnote{This measures the impact of an illness before it resolves or leads to death} growth per year \cite{jesus2019global}.  In the U.S., there are 38,800 physical therapy clinics operating and an average of 150 patients in each clinic each week, with approximately 300 million sessions for patients each year \cite{salazar_2019}. Usually, patients require extensive sets of exercises to return to regular activities. However, they usually have limited time in clinics with supervision under physical therapists, and are required to perform exercises by themselves \textbf{at home}. 
Nevertheless, patients have very little knowledge of how well they perform without monitoring or supervision. Moreover, they do not have the flexibility and convenience to set up a camera to self-monitor~\cite{stankovic2021challenges}. Therefore, having a quantitative measurement of the quality of exercises with wearable devices for patients and therapists is crucial to support the patients get their wellness back.

\subsection{Motivation}
To improve effective rehabilitation, self-efficacy, self-motivation, social support, intentions, and previous adherence to physical therapies can help patients perform exercises in home-based physical therapy \cite{essery2017predictors}. 
However, there is a considerable gap existing between how patients perform self-monitored offsite therapeutic and clinically supervised exercises. Patients and their therapists have no effective and convenient way to track exercises quantitatively at home.

Human Activity Recognition (HAR) in wearable devices is a prevalent research topic that includes many day-to-day locational, behavioral, and planning recognition. For instance, handwashing \cite{wang2021you}, finger gestures \cite{chen2021vifin}, eating behavior \cite{bi2018auracle}, and daily activities (writing, cooking, and cleaning) \cite{bhattacharya2022leveraging} improve to recognize activities through wearable technologies. However, although these works meet the need for daily human activities, a limited attempt exists to help therapeutic rehabilitation for patients. 

Moreover, existing works on the quality of exercise utilize vision-based devices, such as cameras and K2 Kinect \cite{NESHOV2019, HAGHIGHIOSGOUEI2020}, to track the quality of exercises and provide simple feedback. However, users must set them up physically and potentially interfere with the occlusion of the camera. Moreover, calibrating and adjusting vision-based devices are costly and inconvenient for injured or immobile people. Therefore, there is a need to provide portable and wearable devices to measure the quality of exercises.

Lastly, attempts on wearable devices to track quality, such as calorie intake \cite{hussain2022smart} and gait authentication \cite{papavasileiou2021gaitcode}, suggest the endeavors to use wearable sensors to track the quality of exercises. Ghanashyama et al. attempt to use deep learning models to recognize and count the repetitions of exercises using a single sensor \cite{PRABHU2021}. However, qualitative information is not a therapeutic metric to help rehabilitation. 

In summary, there is a high demand to improve how patients and users quantitatively measure their exercises offsite with meaningful feedback from wearable devices.
% \change{
% % In order to understand the exercises' quality, we must first understand \change{\st{how deep learning} different ways digitalize exercises.} 
% % not technical parts, just talk about the exercise and activity that have been done. 
% % video based: quality does have, intrusive and inconvienent 
% % wearable base: has quality, such as step, not enough.
% Human Activity Recognition (HAR) has been a trending and ongoing research for many years. For instance, handwashing \cite{wang2021you}, finger gestures \cite{chen2021vifin}, and daily activities (writing, cooking, and cleaning) \cite{bhattacharya2022leveraging} improve the quality of life through wearable technologies. Although these works meet the need for daily human activities, a limited attempt exists to help therapeutic rehabilitation. As a result, we highlight two primary reasons for this research for such a work in the qualitative assessment of exercises.
% }

% \change{
% First, a considerable gap exists between how patients perform self-monitored offsite therapeutic and clinically supervised exercises. To improve effective rehabilitation, Essery et al. suggest that greater self-efficacy, self-motivation, social support, intentions, and previous adherence to physical therapies can help patients perform exercises in Home-based Physical Therapy \cite{essery2017predictors}.
% In state-of-the-arts, researchers attempt to alleviate Parkinson's disease \cite{WEI2019} and post-injury physiotherapy \cite{WEI2019}. Yurtman et al. use IMU sensors to analyze the similarity of previous and subsequent exercises of patients \cite{YURTMAN2014}. Burns et al. use machine learning to recognize shoulder physical therapeutic exercises through a smartwatch \cite{BURNS2018}. \meiyi{unclear why these works are limited} 
% However, limited technologies motivate and guide patients to rehabilitate. }

% % these works attempt to alleviate Parkinson's disease \cite{WEI2019} and post-injury physiotherapy \cite{WEI2019}. Moreover, Yurtman et al. use 5 IMU sensors to analyze the similarity of previous and subsequent exercises of patients \cite{YURTMAN2014}. Burns et al. attempt to use machine learning to recognize shoulder physical therapeutic exercises through a smartwatch \cite{BURNS2018}.}

% \change{Secondly, patients and their therapists have no effective and convenient way to track their exercises quantitatively at home. Physical therapy requires patients' perseverance and dedication to go through long-term rehabilitation to regain their physical activities in daily life. However, this process is sometimes tedious and time-consuming, and many people cannot persist to on-site clinics to perform hourly exercises for a while. Even though existing work and models utilize cameras and K2 Kinect \cite{NESHOV2019, HAGHIGHIOSGOUEI2020} and vision-based devices to track motion, users must set them up physically and potentially occlusion interference. In addition, calibrating and moving are costly for injured or immobile people. On the other hand, Ghanashyama et al. attempt to use deep learning models to recognize and count the repetitions of exercises using a single sensor \cite{PRABHU2021}. However, this work is limited to repetition counting and not sufficient to measure physical therapeutic quality.}

% \change{In summary, there is a high demand to improve how patients and users quantitatively measure their exercises offsite with meaningful feedback with wearable devices. However, to the best of our knowledge, there is no current work on building a framework capable of generalizing therapeutic and anaerobic exercises in professionally understood metrics. We refer to Section \ref{sec:related work} for more detailed discussion of the related work.}


% Exercises recognition, also known as Human Activity Recognition (HAR), has been trending and ongoing research for many years. \meiyi{what HAR/exercise} 
% \change{For example, deep learning methods with attention is a prevalent topic to recognizing human activity \cite{murahari2018attention, zeng2018understanding, ma2019attnsense}. Moreover, diverse interests in improving people's quality of life are widespread using different deep learning methods \cite{chen2021vifin, bhattacharya2022leveraging, wang2021you}.} Researchers have developed different vision-based and sensor-based devices to collect participants' motions and exercises for various purposes, including Parkinson's disease \cite{WEI2019} and post-injury physiotherapy \cite{WEI2019}. For example, 5 IMUs are used to analyze the similarity of previous and subsequent exercises \cite{YURTMAN2014}. In addition, David Burns applies machine learning to recognize shoulder physical therapeutic exercises through a smartwatch \cite{BURNS2018}.
% Secondly, we need to understand how quality works in terms of exercises. Ghanashyama et al. targets how to use the deep learning models to recognize and count the repetitions of exercises using a single sensor \cite{PRABHU2021}. Neshov and Haghighi focus on how to use comparison to understand the quality of activities using vision-based devices, including cameras and K2 Kinect \cite{NESHOV2019, HAGHIGHIOSGOUEI2020}, which are intrusive and inconvenient for patients to use in the real world. \change{As a result, we proceed to use deep learning as a technique that focuses on different aspects of signals; in terms of similarity comparison, we do not want to compare the difference (such as Dynamic Time Warping) between the two, but based on different metrics, the model helps us to understand a professional metrics of the quality of exercises. }




% There are two primary reasons for this research for such a work in the qualitative assessment of exercises. First, a considerable gap exists between how patients perform self-monitored therapeutic exercises and clinically supervised exercises. \change{Specifically, Essery suggests that greater self-efficacy, self-motivation, social support, intentions and previous adherence to physical therapies can help patients to perform exercises in Home-based Physical Therapy \cite{essery2017predictors}. Such qualities are challenging for patients to maintain and keep them self-motivated. Therefore, a simple and portable device to track the improvement as a self-reward system is necessary for patients to feel motivated.} Second, patients do not have the applications in affordable and portable devices to conveniently perform exercises at home. Physical therapy requires patients' perseverance and dedication to go through a long-term rehabilitation to regain their physical activities in daily life. However, this process sometimes is tedious and time-consuming, and many people cannot persist through to on-site clinics to perform hourly exercises for a while. Even though existing work and models utilize cameras and vision-based devices to track motion, users must set them up physically. The process of calibrating and moving is costly for injured or immobile people. With our limited knowledge, there \change{does} not have any current work on building a framework capable of generalizing therapeutic and anaerobic exercises in professionally understood metrics. Thus, there is a demand to improve how patients and users quantitatively measure their exercises with meaningful feedback and recommendations with commodity devices.   

\subsection{Challenges}
There are three significant challenges in designing such a framework. 
First, how to digitize physical metrics is an open question.
To the best of our knowledge, there are no existing systems or models measuring the quality of an activity, and existing models are not sophisticated enough to directly return a quantitative measurement.  
Secondly, the quality of exercise varies for different people of different ages, heights, weights, and gender. For example, a tall person has a long traveling distance for shoulder abduction exercise because of his height and arm lengths. Suppose someone of average height performs the same PT exercise with the same range of motion, the model is supposed to be able to tell the difference and similarities even though their heights are different. Similarly, suppose an elder performs differently than a teenager in a PT exercise; even though their objective quality differs, quality should remain the same if they raise their arm to 90 degrees compared to 60 degrees.
Lastly, there are no existing datasets with annotated quality of exercises.     

% are no existing applications and resources on data and metrics on how to collect data to the best of our knowledge.


% Current technology and state-of-art algorithms focus on how to recognize the exercises in various setting of both outdoor and indoor activity. However, the quality of the exercise has not been focused on and measured with. Specifically, for instance, a shoulder extension can be performed in various range of motion and all of these variation do consider as shoulder extension in physical therapy; the nuance of the exercises should be recorded and fathomed by the model in order to help patients to measure the quality numerically. Having a higher standard of granularity will facilitate patients' recovery as a secondary supervision to provide additional sensory information about their performances. Not only it will help the clinical therapists to understand the patients better, but also provide significant level of qualitative assessment when patients have no access to professionals. At the same time, there is no standardized metrics of how to quantitatively measure quality of exercises without additional stationary tests, such as Crank Tests and Lift-off Test for Frozen Shoulder Test. Additionally, this is a research question on how to connect between deep learning model and qualitative assessment of exercises and how do we translate from sensory data to qualitative metrics to digital measurements. More specifically, for example, how can we convert range of motion into numerical representation of the quality of exercises in Shoulder Abduction? Lastly, to the best of our knowledge, there is no data collected with the focus of the quality of exercises. Collecting data with a new and scrutinized metrics is a tremendous challenge in order to test our model.
% \begin{itemize}
%     \item Different choices of sensors and models:
%     \begin{itemize}
%         \item Currently, many devices are used to record patients' sensory data but balancing between portability, cost, and accuracy remained a central issue. For patients to use a device as their main source of evaluation on both activity recognition and qualitative assessment, it has to be portable, relatively cheap and has a good accuracy overall. And to choose and balance the benefits and disadvantages need to be well considered.
%         \item Different deep learning models should also be considered to achieve a great performance; however, due to limited space in portable devices such as a smartwatch, the complexity of an algorithm should also be considered as if a real-time application is required.
%     \end{itemize}
%     \item Usability in real-world setting in people: 
%     \begin{itemize}
%         \item The application needs to be generalized in different variations of exercises, such that a lateral raise is considered as a lateral raise even if patients perform such an exercise in a different range of motion.
%         \item The application and models are necessary to apply in a heterogeneous population such that people from different gender, age, and people with different weights and heights are capable of picking up the devices and activating them quickly.
%         \item The main problem in performing therapeutic exercises is the difference in short-term and long-term goals. Many researchers are unfamiliar with the methodology that applies in terms of time-sensitive goals. More specifically, the short-term quality of exercise focuses on the difficulties and doability, while the long-term quality of exercise concentrates on the efficacy of improvement of a given combination of exercises.
%     \end{itemize}
% \end{itemize}

\subsection{Contributions}
Targeting these challenges, this paper introduces a novel framework, PhysiQ, that continuously tracks and quantitatively measures people's off-site exercise activity through passive sensory detection. 
The framework is robust and general to handle users who perform exercises with different speeds, positions, and postures. We summarize our main contributions below:
% \david{TODO: add collecting data as a contribution 19 partcipants with different quality}
% \david{we design and build an app}
% \david{extensive evaluation with performance better than baselin by how many percentages.}
\begin{itemize}
    \item To the best of our knowledge, this is the first framework for quantitative measurement of exercises using a smartwatch.  
    % with respect to the anchor/reference exercises from previous performance for patients' improvement.
    The framework identifies and digitalizes three key exercise metrics of \textit{range of motion}, \textit{stability}, \textit{repetition}, which are built upon the muscular system for understanding the functionality of the skeletal system.
    \item We create a novel multi-task spatio-temporal Siamese Neural Network that  
    measures both absolute quality and relative quality based on an individual's PT progress through similarity comparison. It enables patients to understand the quality of their offsite exercises over time.
    \item We build an application collecting users' motion data in a smartwatch and giving explainable feedback with recommendations based on their quality of exercises in real-time. 
    % \change{Having a portable device, patients are capable of self-supervising their exercises readily.}
    \item We collect and annotate 31 participants' motion data with different levels of stability, range of motion, and repetition, in three shoulder exercises, which are shoulder abduction, external rotation, and forward flexion.
    \item We perform an extensive evaluation using real user's data. Results show that our framework performance outperforms the baselines by 47.67\% on average in R-Squared for all exercises and all three metrics.  %originally: 41.76
    We also provide insights of how user's behaviors influence the framework through a user experience study. 

\end{itemize}


\subsection{Paper Organization}
% \meiyi{don't use WILL} 
In the rest of the paper, we discuss the related work in  Section \ref{sec:related work}. We present our framework PhysiQ in  Section \ref{sec:solutions}. Next, we show how we collect the exercise data with different quality in  Section \ref{sec:data collection}, and evaluation results in  Section \ref{sec:evaluation}. Furthermore, we present a survey and discuss user experience using our app and implications in  Section \ref{sec:survey}, followed by a discussion and summary in  Section \ref{sec:discussion} and  Section \ref{sec:summary}, respectively.


