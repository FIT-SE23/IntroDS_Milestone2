% \vspace{-5pt}
\section{Experimental Results}
% \vspace{-5pt}
\label{sec:experiments}

\subsection{Experimental Setup}

Our experiments include adaptive token pruning and hardware implementation for ViTs with different pruning settings. Note that after token pruning we will apply 8-bit quantization on weight and activation, and all the quantization processes do not lose accuracy, except for 1.2\% on DeiT-T.

\subsubsection{Training Setup for ViT Pruning}

The baseline models with 32-bit floating-point precision are from the TorchVision library~\cite{Torchvision}.
Our experiments are conducted on the ImageNet-1K dataset with various transformers backbones, including DeiT-T, DeiT-S, DeiT-B, LV-ViT-S, and LV-ViT-M, as shown in Table~\ref{tab:vit_backbone}.
We follow the training settings in DeiT.
Training on one selector insertion costs 30 epochs on 8 NVIDIA A100-SXM4-40GB GPUs.
The training effort on block-to-stage training is listed in Table~\ref{tab:vit_backbone} which illustrates that the training effort of the entire block-to-stage pipeline is equivalent to the train-from-scratch of the backbone ViT.
Through our training pipeline, we observe that 3$\sim$4 token selectors are suitable for most of the models.


% Please add the following required packages to your document preamble:
% \usepackage{multirow}
\begin{table}[htb]
\centering
%\vspace{-0.1in}
\caption{Training effort for ViTs with different backbones.}
\label{tab:vit_backbone}
\vspace{-0.1in}
\begin{tabular}{c|c|c|c|cc}
\hline
\multirow{2}{*}{\textbf{Model}} & \multirow{2}{*}{\textbf{\#Heads}} & \multirow{2}{*}{\textbf{\begin{tabular}[c]{@{}c@{}}Embed. \\ Dim.\end{tabular}}} & \multirow{2}{*}{\textbf{Depth}} & \multicolumn{2}{c}{\textbf{\#Epochs for Training}}                   \\ \cline{5-6} 
                                &                                   &                                                                                  &                                 & \multicolumn{1}{c|}{\textbf{Baseline}} & \textbf{Ours} \\ \hline
DeiT-T                          & 3                                 & 192                                                                              & 12                              & \multicolumn{1}{c|}{300}               & 270                            \\ \hline
DeiT-S                          & 6                                 & 384                                                                              & 12                              & \multicolumn{1}{c|}{300}               & 270                            \\ \hline
DeiT-B                          & 12                                & 768                                                                              & 12                              & \multicolumn{1}{c|}{300}               & 270                            \\ \hline
LV-ViT-S                        & 6                                 & 384                                                                              & 16                              & \multicolumn{1}{c|}{400}               & 390                            \\ \hline
LV-ViT-M                        & 8                                 & 512                                                                              & 20                              & \multicolumn{1}{c|}{400}               & 390                            \\ \hline
\end{tabular}
\end{table}

\subsubsection{Hardware Platform}

Our hardware accelerator designs are evaluated on the Xilinx ZCU102 platform~\cite{ZCU102} with Zynq UltraScale+ MPSoC, containing 2520 DSPs, 912 BRAM blocks, and 274.1k LUTs.
We use Vitis HLS to generate the FPGA accelerators.
The synthesis and implementation of the designs are performed with the Xilinx Vitis 2020.1 tool~\cite{vitis}. The working frequency is set to 150 MHz.
The data in all models are represented in an 8-bit fixed-point format.
The hardware design for \M~is built based on state-of-the-art FPGA design for ViT~\cite{lizh2022logic} with the GEMM engine described in Section~\ref{sec:loop_GEMM}. Additionally, the \M~hardware design incorporates a token selector and polynomial approximation of nonlinear functions
%into the GEMM engine to execute the token selection procedure 
explained in Section~\ref{sec:token_selection} and~\ref{sec:approx}.

\subsection{Accuracy and GMACs Results}

Fig.~\ref{fig:user-trade-off} demonstrates that our models achieve better accuracy-computation trade-offs compared to other pruned or scaled models.
Our \M~reduces the computation cost by $16.1\% {\sim} 42.6\%$ for various backbones with negligible $\leq 0.75\%$ accuracy degradation, which surpasses existing methods on both accuracy and efficiency.
% On the lightweight DeiT-T, the proposed \M~ still reduces GMACs by $31\%$ with a negligible $0.1\% $ decrease of accuracy (72.10\% vs. 72.20\%).
To explore model scaling on ViT, we train more DeiT models with the embedding dimension of 160/256/288/320 as our baselines.
The accuracy improvement of HeatViT is 4\% (72.1\% vs. 68.1\% with 0.9 GMACs) over DeiT-T-160, 4.67\% (76.87\% vs. 72.20\% with 1.3 GMACs) over DeiT-T, and 0.81\% (79.34\% vs. 78.53\% with 2.64 GMACs) over DeiT-S-288.
Additionally, our method can prune up to 23.1\% on DeiT-T and 16.1\% on DeiT-S without any accuracy degradation.

% \begin{table}[htb]
\centering
\tabcolsep 5pt
\vspace{-0.1in}
\caption{Comparison between representative ViT pruning methods.}
\vspace{-0.1in}\label{tab:pruning_works}
\begin{tabular}{c|c|c|c}
\toprule
\textbf{Method}                      & \textbf{\makecell{Design \\ Scheme}} & \textbf{Method} & \textbf{\makecell{Design \\ Scheme}}  \\
\midrule
DyViT~\cite{rao2021dynamicvit}        & \ding{172}                          & ATS~\cite{fayyaz2021}      & \ding{172}           \\ \hline
IA        ~\cite{pan2021iared2} 	  & \ding{173}                          & PS-ViT~\cite{tang2021patch}& \ding{172}           \\ \hline
VTP~\cite{zhu2021visual}              & \ding{174}                          & Evo-ViT~\cite{xu2021evo}   & \ding{173}          \\ \hline
S2ViTE~\cite{chen2021chasing}         & \ding{172}\ding{174}\ding{175} 		& UP-DeiT~\cite{yu2021unified}& \ding{174}\ding{175} \\ \bottomrule
% UVC~\cite{yu2022unified}              & \ding{173}\ding{174}\ding{175} 		& UP-DeiT~\cite{yu2021unified}& \ding{174}\ding{175} \\ \bottomrule
\end{tabular}
{\\ \footnotesize \raggedright \ding{172} -- Static Token Pruning; \ding{173} -- Adaptive Token Pruning; \ding{174} -- Head Pruning; \\ \ding{175} -- Token Channel Pruning.\par}
\vspace{-0.2cm}
\end{table}
% \begin{table}[htb]
% \small
\centering
\tabcolsep 3pt

\vspace{0.05in}
\caption{Comparison to state-of-the-art ViT pruning methods on ImageNet-1K. %\zf{update this table as discussed, highlight the HeatViT row with highest GMACs reduction for each model while having less than 1\% accuracy loss than baseline}
}
\label{tab:main_results}
\vspace{-0.1in}
% \resizebox{1.4\columnwidth}{!}{
\begin{tabular}{clcccc}
\toprule
\bf Model  &\bf Method &\bf \makecell{Params \\ (M)}  & \makecell{\textbf{\#GMACs} \\ (Reduction)} & \bf \makecell{Top-1 \\ Acc. (\%)} %&\multicolumn{1}{c}{\bf Top-5 Acc. (\%)}
\\
\hline%{5mm}
                 &Baseline/192   &5.60         &1.30 (0.00\%)         &72.20                \\
                 &Baseline/160*  &4.00         &0.90 (30.77\%)        &68.10                \\
                 &UP-DeiT-T      &5.70         &1.30 (0.00\%)         &75.79           \\
                 &DynamicViT     &5.90         &0.91 (30.00\%)        &71.85            \\
                 &PS-ViT         &5.60         &0.93 (28.46\%)        &72.00                 \\
                 &S$^2$ViTE      &4.20         &0.95 (26.92\%)        &70.12                 \\
DeiT-T           &ATS+DeiT/258   &10.13        &1.50 (-15.38\%)       &76.90            \\
                 &HeatViT-256 (Ours)  &10.16                 &1.29 (0.77\%)         &76.87                \\
                 &HeatViT-320 (Ours)  &15.44                 &1.30 (0.00\%)         &77.02                \\
                 &HeatViT (Ours)      &5.70                  &0.54 (58.46\%)        &70.80              \\
                 &\textbf{HeatViT (Ours)}      &5.70                  &0.90 (30.77\%)        &\textbf{72.10}                \\
                 &HeatViT (Ours)      &5.70                  &1.00 (23.08\%)        &72.20               \\
 \hline 
                 &Baseline/384   &22.10       &4.60 (0.00\%)          &79.80          \\
                 &Baseline/320*  &15.40       &3.25 (29.35\%)         &79.00          \\
                 &Baseline/288*  &12.60       &2.65 (42.39\%)         &78.53          \\
                 &Baseline/256*  &10.00       &2.14 (53.48\%)         &77.21                \\
                 &S$^2$ViTE      &14.60       &3.14 (31.74\%)         &79.22                 \\
 DeiT-S          &IA-RED$^2$     &-           &3.15 (31.52\%)         &79.10           \\ 
                 &DynamicViT     &22.80       &2.91 (36.74\%)         &79.30              \\   
                 &DynamicViT*    &22.80       &2.71 (41.09\%)         &79.12                \\
                 &HeatViT-320 (Ours) &15.44        &2.01 (56.52\%)       &79.00             \\
                 &HeatViT (Ours)     &22.13        &1.36 (70.43\%)      &78.05            \\
                 &\textbf{HeatViT (Ours)}     &22.13        &2.64 (42.61\%)       &\textbf{79.34}            \\
                 &HeatViT (Ours)     &22.13        &3.86 (16.09\%)       &79.80             \\
 \hline 
                 &Baseline/768   &86.50      &17.60 (0.00\%)          &81.80          \\
 DeiT-B          &IA-RED$^2$     &-          &11.80 (32.96\%)         &80.30           \\ 
                 &DynamicViT     &89.50      &11.02 (37.39\%)         &80.70              \\   
                 &HeatViT (Ours)       &86.60        &3.70 (78.98\%)       &80.04             \\
                 &\textbf{HeatViT (Ours)}       &86.60        &10.49 (40.40\%)       &\textbf{81.05}             \\
 \hline 
\multirow{4}{*}{LV-ViT-S}        &Baseline/384      &26.15       &6.55 (0.00\%)                  &83.30      \\
~ &ATS+CvT-21        &32.00       &5.10 (22.14\%)              &82.30    \\
                 &DynamicViT        &26.90       &4.57 (30.22\%)              &83.00     \\
                 &HeatViT (Ours)     &26.17       &1.61 (75.47\%)    &81.72            \\
                 &\textbf{HeatViT (Ours)}     &26.17       &4.28 (34.65\%)     &\textbf{83.10}             \\
\hline 
\multirow{10}{*}{LV-ViT-M}     &Baseline/512      &55.83       &12.67 (0.00\%)                  &84.00          \\
    &CvT-21/224        &32.00        &7.10 (43.96\%)               &82.50    \\
                 &DynamicViT*   &57.10        &7.35 (41.99\%)           &83.61        \\
                 &DynamicViT    &57.10        &8.45 (33.31\%)           &83.80         \\
                 &Evo-ViT       &87.30        &9.07 (28.41\%)               &81.11         \\
                 &VTP           &48.00        &10.00 (21.07\%)               &80.70    \\
                 &ATS+CvT-13/384&20.00        &11.70 (7.66\%)              &82.90         \\
                 &ATS+CvT-21/384&32.00        &17.40 (-37.33\%)               &83.10         \\
                 &DeiT-B        &86.60        &17.60 (-38.91\%)               &81.80        \\
                 &HeatViT (Ours)       &55.85               & 3.46 (77.69\%)      &82.18  \\ 
                 &\textbf{HeatViT (Ours)}       &55.85               &7.32 (42.23\%)      &\textbf{83.71}  \\ 
\bottomrule
\end{tabular}
% }

{\footnotesize \raggedright
``*'' means our reproduced results to obtain models with similar GMACs. Negative values in MACs reduction mean MAC increases. HeatViT-256/320 indicates pruning from DeiT scaling model of DeiT-256/320.\par}

% \vspace{-0.4cm}

\end{table}
% \begin{figure*}[htb]
% \centering
% \includegraphics[width=1.5\columnwidth]{Figs/fpga_main_flops_acc_test.pdf}
% \caption{Computation (GMACs) and top-1 accuracy trade-offs on ImageNet. Our models can achieve better trade-offs compared to other pruned or scaled models.}% GMACs and accuracy reductions are calculated from their corresponding dense model.} %Note that ``*'' refers to our reproduced results to obtain models with similar FLOPs for comparison. Baseline/160/192/288/384/512/768 indicates the embedding dimensions.}
% \label{fig:main_flops}
% \end{figure*}https://www.overleaf.com/project/6338bee4ec1dd54d2be36bee
% \begin{table}[htb]
% \centering
% \tabcolsep 4pt
% \caption{Effect of multi-heads in DeiT-T.}
% \vspace{-0.1in}\label{tab:multihead}
% % \resizebox{0.95\columnwidth}{!}{
% \begin{tabular}{ccccc}
% \toprule
%     \bf Method & \bf Pruning & \bf Multi-Head &\bf \#GMACs &\bf Top-1 Acc. (\%)   \\
% \midrule
% \M & Yes & Yes  & 0.77 & 71.60 \\
% \M & Yes & No  & 0.77 & 71.06 \\
% \hline 
% \M & Yes & Yes & 0.54 & 70.30 \\
% \M & Yes & No  & 0.54 & 69.32\\
% \bottomrule
% \end{tabular}
% \vspace{-0.4cm}
% % }
% \end{table}



% \subsection{Ablation Studies--Software Side}
% \subsubsection{Token Selector with Multi-Heads or Single-Head}

% We employ our method both with multi-head and single-head structure respectively.
% As shown in Table~\ref{tab:multihead} our multi-head design can improve the accuracy by 0.54\% (71.6\% vs. 71.06\%), and 0.98\% (70.30\% vs. 68.32\%).
\begin{figure}[tb]
\centering
%\vspace{-0.15inch}
\includegraphics[width=1 \columnwidth]{Figs/token_selector.pdf}
\vspace{-0.3in}
\caption{Comparison of different token selector structures. We use DeiT-T (72.2\%, 1.3 GMACs) as the baseline.}
\label{fig:selector_structure}
%\vspace{-0.1in}
\end{figure}
% \begin{table*}[htb]
% \centering
% \tabcolsep 3pt

% \vspace{0.05in}
% \caption{Accuracy and hardware results under different pruning settings for various ViTs on ImageNet dataset.}
% \label{tab:result_hardware}
% \vspace{-0.1in}
% \begin{tabular}{c|c|ccc|cccc|c|cc}
% \toprule
% \multirow{2}{*}{\bf Design} & \multirow{2}{*}{\bf Model} & \bf Keep Ratio & \bf \#GMACs & \multirow{2}{*}{\bf Bitwidth} & \multicolumn{4}{c|}{\bf Resource Utilization} & \bf Power & \bf FPS & \bf Energy Effi. \\
% \cline{6-9}
% ~ & ~ & (Stage 1/2/3) & (Pruning Rate) & ~ & kLUT & kFF & BRAM36 & DSP & (W) & (Accl. Rate) & (FPS/W) \\
% \midrule
% \multirow{4}{*}{Baseline} & DeiT-T & 1/1/1 & 1.30 (1$\times$) & 16 & \makecell{115.6 \\ (42\%)} & \makecell{101.5 \\ (19\%)} & \makecell{288.5 \\ (32\%)} & \makecell{1739 \\ (69\%)} & 8.012 & 78.3 (1$\times$) & 9.77 \\
% \cline{2-12}

% ~ & DeiT-S & 1/1/1 & 4.60 (1$\times$) & 16 & \multirow{2}{*}{\makecell{130.3 \\ (48\%)}} & \multirow{2}{*}{\makecell{102.8 \\ (19\%)}} & \multirow{2}{*}{\makecell{492.5 \\ (54\%)}} & \multirow{2}{*}{\makecell{1754 \\ (70\%)}} & \multirow{2}{*}{10.095} & 25.9 (1$\times$) & 2.57 \\
% \cline{2-5} \cline{11-12}

% ~ & LV-ViT-S & 1/1/1 & 6.55 (1$\times$) & 16 & ~ & ~ & ~ & ~ & ~ & 19.4 (1$\times$) & 1.92 \\
% \cline{2-12}

% ~ & DeiT-B & 1/1/1 & 17.60 (1$\times$) & 16 & \makecell{144.5 \\ (53\%)} & \makecell{103.9 \\ (19\%)} & \makecell{664.3 \\ (73\%)} & \makecell{1786 \\ (71\%)} & 11.041 & 11.2 (1$\times$) & 1.01 \\

% \midrule

% \multirow{12}{*}{\makecell{\textbf{\M~} \\ with Token \\ Selector}} & \multirow{3}{*}{DeiT-T} & 0.85/0.79/0.51 & 1.00 (1.30$\times$) & 8 & \multirow{3}{*}{\makecell{137.6 \\ (50\%)}} & \multirow{3}{*}{\makecell{126 \\ (23\%)}} & \multirow{3}{*}{\makecell{355.5 \\ (39\%)}} & \multirow{3}{*}{\makecell{1968 \\ (78\%)}} & \multirow{3}{*}{\makecell{9.453}} & 184.7 (2.36$\times$) & 19.4 \\
% \cline{3-5} \cline{11-12}
% ~ & ~ & 0.76/0.70/0.41 & 0.90 (1.44$\times$) & 8 & ~ & ~ & ~ & ~ & ~ & 198.8 (2.53$\times$) & 21.1 \\
% \cline{3-5} \cline{11-12}
% ~ & ~ & 0.70/0.39/0.21 & 0.75 (1.74$\times$) & 8 & ~ & ~ & ~ & ~ & ~ & 231.1 (2.95$\times$) & 24.4 \\
% \cline{3-5} \cline{11-12}
% % ~ & ~ & 0.42/0.21/0.09 & 0.75 (1.74$\times$) & 70.30 & ~ & ~ & ~ & ~ & ~ & 146.4 (1.87$\times$) & 15.49 \\
% \cline{2-12}

% ~ & \multirow{3}{*}{DeiT-S} & 0.90/0.84/0.61 & 3.86 (1.19$\times$) & 8 & \multirow{6}{*}{\makecell{145.0 \\ (53\%)}} & \multirow{6}{*}{\makecell{100.4 \\ (18\%)}} & \multirow{6}{*}{\makecell{338.5 \\ (37\%)}} & \multirow{6}{*}{\makecell{1955 \\ (78\%)}} & \multirow{6}{*}{10.697} & 57.0 (57.02.20$\times$) & 5.33 \\
% \cline{3-5} \cline{11-12}
% % ~ & ~ & 0.69/0.43/0.20 & 2.64 (1.74$\times$) & 79.34 & ~ & ~ & ~ & ~ & ~ & 40.9 (1.58$\times$) & 3.82 \\
% \cline{3-5} \cline{11-12}
% ~ & ~ & 0.70/0.39/0.21 & 2.64 (1.74$\times$) & 8 & ~ & ~ & ~ & ~ & ~ & 78.3 (3.02$\times$) & 7.3 \\
% \cline{3-5} \cline{11-12}
% ~ & ~ & 0.42/0.21/0.13 & 2.02 (2.27$\times$) & 8 & ~ & ~ & ~ & ~ & ~ & 99.0 (3.82$\times$) & 9.3 \\
% \cline{2-5} \cline{11-12}

% ~ & \multirow{3}{*}{LV-ViT-S} & 0.90/0.84/0.61 & 5.49 (1.19$\times$) & 8 & ~ & ~ & ~ & ~ & ~ & 41.7 (57.02.20$\times$) & 3.9 \\
% \cline{3-5} \cline{11-12}
% ~ & ~ & 0.70/0.39/0.21 & 3.77 (1.74$\times$) & 8 & ~ & ~ & ~ & ~ & ~ & 58.7 (3.03$\times$) & 5.5 \\
% \cline{3-5} \cline{11-12}
% ~ & ~ & 0.42/0.21/0.13 & 2.88 (2.27$\times$) & 8 & ~ & ~ & ~ & ~ & ~ & 74.2 (3.83$\times$) & 6.9 \\

% \cline{2-12}
% ~ & \multirow{3}{*}{DeiT-B} & 0.90/0.84/0.61 & 14.79 (1.19$\times$) & 8 & \multirow{3}{*}{\makecell{161.4 \\ (59\%)}} & \multirow{3}{*}{\makecell{101.8 \\ (19\%)}} & \multirow{3}{*}{\makecell{528.6 \\ (58\%)}} & \multirow{3}{*}{\makecell{2066 \\ (82\%)}} & \multirow{3}{*}{11.352} & 24.0 (2.14$\times$) & 2.1 \\
% \cline{3-5} \cline{11-12}
% ~ & ~ & 0.70/0.39/0.21 & 10.11 (1.74$\times$) & 8 & ~ & ~ & ~ & ~ & ~ & 33.9 (3.02$\times$) & 3.0 \\
% \cline{3-5} \cline{11-12}
% ~ & ~ & 0.42/0.21/0.13 & 7.75 (2.27$\times$) & 8 & ~ & ~ & ~ & ~ & ~ & 42.9 (3.83$\times$) & 3.8 \\

% \bottomrule
% \end{tabular}
% \vspace{-0.1in}
% \end{table*}





% \begin{table*}[htb]
% \centering
% \tabcolsep 3pt

% \vspace{0.05in}
% \caption{Accuracy and hardware results under different pruning settings for various ViTs on ImageNet dataset.}
% \label{tab:result_hardware}
% \vspace{-0.1in}
% \begin{tabular}{c|c|ccc|cccc|c|cc}
% \toprule
% \multirow{2}{*}{\bf Design} & \multirow{2}{*}{\bf Model} & \bf Keep Ratio & \bf \#GMACs & \multirow{2}{*}{\bf Bitwidth} & \multicolumn{4}{c|}{\bf Resource Utilization} & \bf Power & \bf FPS & \bf Energy Effi. \\
% \cline{6-9}
% ~ & ~ & (Stage 1/2/3) & (Pruning Rate) & ~ & kLUT & kFF & BRAM36 & DSP & (W) & (Accl. Rate) & (FPS/W) \\
% \midrule
% \multirow{4}{*}{Baseline} & DeiT-T & 1/1/1 & 1.30 (1$\times$) & 16 & \makecell{115.6 \\ (42\%)} & \makecell{101.5 \\ (19\%)} & \makecell{288.5 \\ (32\%)} & \makecell{1739 \\ (69\%)} & 8.012 & 78.3 (1$\times$) & 9.77 \\
% \cline{2-12}

% ~ & DeiT-S & 1/1/1 & 4.60 (1$\times$) & 16 & \multirow{2}{*}{\makecell{130.3 \\ (48\%)}} & \multirow{2}{*}{\makecell{102.8 \\ (19\%)}} & \multirow{2}{*}{\makecell{492.5 \\ (54\%)}} & \multirow{2}{*}{\makecell{1754 \\ (70\%)}} & \multirow{2}{*}{10.095} & 25.9 (1$\times$) & 2.57 \\
% \cline{2-5} \cline{11-12}

% ~ & LV-ViT-S & 1/1/1 & 6.55 (1$\times$) & 16 & ~ & ~ & ~ & ~ & ~ & 19.4 (1$\times$) & 1.92 \\
% \cline{2-12}

% ~ & DeiT-B & 1/1/1 & 17.60 (1$\times$) & 16 & \makecell{144.5 \\ (53\%)} & \makecell{103.9 \\ (19\%)} & \makecell{664.3 \\ (73\%)} & \makecell{1786 \\ (71\%)} & 11.041 & 11.15 (1$\times$) & 1.01 \\

% \midrule

% \multirow{12}{*}{\makecell{\textbf{\M~} \\ with Token \\ Selector}} & \multirow{3}{*}{DeiT-T} & 0.85/0.79/0.51 & 1.00 (1.30$\times$) & 8 & \multirow{3}{*}{\makecell{137.6 \\ (47\%)}} & \multirow{3}{*}{\makecell{126 \\ (23\%)}} & \multirow{3}{*}{\makecell{355.5 \\ (39\%)}} & \multirow{3}{*}{\makecell{1968 \\ (71\%)}} & \multirow{3}{*}{\makecell{9.453}} & 222.3 (2.84$\times$) & 23.5  \\
% \cline{3-5} \cline{11-12}
% ~ & ~ & 0.76/0.70/0.41 & 0.90 (1.44$\times$) & 8 & ~ & ~ & ~ & ~ & ~ & 238.5 (3.05$\times$) & 25.33 \\
% \cline{3-5} \cline{11-12}
% ~ & ~ & 0.70/0.39/0.21 & 0.90 (2.41$\times$) & 8 & ~ & ~ & ~ & ~ & ~ & 456.0 (5.31$\times$) & 44.0 \\
% \cline{3-5} \cline{11-12}
% % ~ & ~ & 0.42/0.21/0.09 & 0.54 (2.41$\times$) & 70.30 & ~ & ~ & ~ & ~ & ~ & 146.4 (1.87$\times$) & 15.49 \\
% \cline{2-12}

% ~ & \multirow{3}{*}{DeiT-S} & 0.90/0.84/0.61 & 3.86 (1.19$\times$) & 8 & \multirow{6}{*}{\makecell{145 \\ (50\%)}} & \multirow{6}{*}{\makecell{100.4 \\ (18\%)}} & \multirow{6}{*}{\makecell{338.5 \\ (37\%)}} & \multirow{6}{*}{\makecell{1955 \\ (70\%)}} & \multirow{6}{*}{10.697} & 67.3 (2.60$\times$) & 6.3 \\
% \cline{3-5} \cline{11-12}
% % ~ & ~ & 0.69/0.43/0.20 & 2.64 (1.74$\times$) & 79.34 & ~ & ~ & ~ & ~ & ~ & 40.9 (1.58$\times$) & 3.82 \\
% \cline{3-5} \cline{11-12}
% ~ & ~ & 0.70/0.39/0.21 & 2.01 (2.29$\times$) & 8 & ~ & ~ & ~ & ~ & ~ & 121.5 (4.69$\times$) & 11.4 \\
% \cline{3-5} \cline{11-12}
% ~ & ~ & 0.42/0.21/0.13 & 1.36 (3.38$\times$) & 8 & ~ & ~ & ~ & ~ & ~ & 194.8 (7.52$\times$) & 18.2 \\
% \cline{2-5} \cline{11-12}

% ~ & \multirow{3}{*}{LV-ViT-S} & 0.90/0.84/0.61 & 3.74 (1.75$\times$) & 8 & ~ & ~ & ~ & ~ & ~ & 72.3 (3.73$\times$) & 6.8 \\
% \cline{3-5} \cline{11-12}
% ~ & ~ & 0.70/0.39/0.21 & 2.84 (2.31$\times$) & 8 & ~ & ~ & ~ & ~ & ~ & 92.7 (3.87$\times$) & 8.7 \\
% \cline{3-5} \cline{11-12}
% ~ & ~ & 0.42/0.21/0.13  & 1.61 (4.08$\times$) & 8 & ~ & ~ & ~ & ~ & ~ & 173.6 (8.95$\times$) & 16.2 \\

% \cline{2-12}
% ~ & \multirow{3}{*}{DeiT-B} & 0.90/0.84/0.61 & 9.62 (1.83$\times$) & 8 & \multirow{3}{*}{\makecell{161.4 \\ (59\%)}} & \multirow{3}{*}{\makecell{101.8 \\ (19\%)}} & \multirow{3}{*}{\makecell{528.6 \\ (58\%)}} & \multirow{3}{*}{\makecell{2066 \\ (82\%)}} & \multirow{3}{*}{11.352} & 27.8 (3.73$\times$) & 6.8 \\
% \cline{3-5} \cline{11-12}
% ~ & ~ & 0.70/0.39/0.21 & 7.46 (2.36$\times$) & 8 & ~ & ~ & ~ & ~ & ~ & 68.9 (3.87$\times$) & 8.7 \\
% \cline{3-5} \cline{11-12}
% ~ & ~ & 0.42/0.21/0.13  & 3.70 (4.76$\times$) & 8 & ~ & ~ & ~ & ~ & ~ & 104.0 (8.95$\times$) & 9.2 \\

% \bottomrule
% \end{tabular}
% \end{table*}



% \begin{table*}[htb]
% \centering
% % \tabcolsep 3pt

% \vspace{0.05in}
% \caption{Accuracy and hardware results under different pruning settings for various ViTs on ImageNet dataset.}
% \label{tab:result_hardware}
% \vspace{-0.1in}
% \begin{tabular}{c|c|cc|cccc|c|cc}
% \toprule
% \multirow{2}{*}{\bf Design} & \multirow{2}{*}{\bf Model} & \bf Keep Ratio & \bf \#GMACs & \multicolumn{4}{c|}{\bf Resource Utilization} & \bf Power & \bf FPS & \bf Energy Effi. \\
% \cline{6-9}
% ~ & ~ & (Stage 1/2/3) & (Pruning Rate) & kLUT & kFF & BRAM36 & DSP & (W) & (Accl. Rate) & (FPS/W) \\
% \midrule
% \multirow{5}{*}{\makecell{Baseline \\ \rev{(16-bit)}}} & DeiT-T & 1/1/1 & 1.30 (1$\times$) & \makecell{115.6 \\ (42\%)} & \makecell{101.5 \\ (19\%)} & \makecell{288.5 \\ (32\%)} & \makecell{1739 \\ (69\%)} & 8.012 & 78.3 (1$\times$) & 9.77 \\
% \cline{2-11}

% ~ & DeiT-S & 1/1/1 & 4.60 (1$\times$) & \multirow{2}{*}{\makecell{130.3 \\ (48\%)}} & \multirow{2}{*}{\makecell{102.8 \\ (19\%)}} & \multirow{2}{*}{\makecell{492.5 \\ (54\%)}} & \multirow{2}{*}{\makecell{1754 \\ (70\%)}} & \multirow{2}{*}{10.095} & 25.9 (1$\times$) & 2.57 \\
% \cline{2-4} \cline{10-11}

% ~ & LV-ViT-S & 1/1/1 & 6.55 (1$\times$) & ~ & ~ & ~ & ~ & ~ & 19.4 (1$\times$) & 1.92 \\
% \cline{2-11}

% ~ & DeiT-B & 1/1/1 & 17.60 (1$\times$) & \makecell{144.5 \\ (53\%)} & \makecell{103.9 \\ (19\%)} & \makecell{664.3 \\ (73\%)} & \makecell{1786 \\ (71\%)} & 11.041 & 11.2 (1$\times$) & 1.01 \\

% \midrule

% \tbrev \multirow{12}{*}{\makecell{\textbf{\M~} \\ with Token \\ Selector \\ \rev{(8-bit)}}} & \multirow{3}{*}{DeiT-T} & 0.85/0.79/0.51 & 1.00 (1.30$\times$) & \multirow{3}{*}{\makecell{137.6 \\ (47\%)}} & \multirow{3}{*}{\makecell{126 \\ (23\%)}} & \multirow{3}{*}{\makecell{355.5 \\ (39\%)}} & \multirow{3}{*}{\makecell{1968 \\ (71\%)}} & \multirow{3}{*}{\makecell{9.453}} & 184.7 (2.36$\times$) & 19.4 \\
% \cline{3-4} \cline{10-11}
% ~ & ~ & 0.76/0.70/0.41 & 0.90 (1.44$\times$) & ~ & ~ & ~ & ~ & ~ & 198.8 (2.53$\times$) & 21.1 \\
% \cline{3-4} \cline{10-11}
% ~ & ~ & 0.70/0.39/0.21 & 0.54 (2.41$\times$) & ~ & ~ & ~ & ~ & ~ & 322.6 (4.12$\times$) & 28.7 \\
% \cline{3-4} \cline{10-11}
% % ~ & ~ & 0.42/0.21/0.09 & 0.54 (2.41$\times$) & 70.30 & ~ & ~ & ~ & ~ & ~ & 146.4 (1.87$\times$) & 15.49 \\
% \cline{2-11}

% ~ & \multirow{3}{*}{DeiT-S} & 0.90/0.84/0.61 & 3.86 (1.19$\times$) & \multirow{6}{*}{\makecell{145 \\ (50\%)}} & \multirow{6}{*}{\makecell{100.4 \\ (18\%)}} & \multirow{6}{*}{\makecell{338.5 \\ (37\%)}} & \multirow{6}{*}{\makecell{1955 \\ (70\%)}} & \multirow{6}{*}{10.697} & 57.0 (57.02.20$\times$) & 5.33 \\
% \cline{3-4} \cline{10-11}
% % ~ & ~ & 0.69/0.43/0.20 & 2.64 (1.74$\times$) & 79.34 & ~ & ~ & ~ & ~ & ~ & 40.9 (1.58$\times$) & 3.82 \\
% \cline{3-4} \cline{10-11}
% ~ & ~ & 0.70/0.39/0.21 & 2.01 (2.29$\times$) & ~ & ~ & ~ & ~ & ~ & 97.1 (3.75$\times$) & 9.07 \\
% \cline{3-4} \cline{10-11}
% ~ & ~ & 0.42/0.21/0.13 & 1.36 (3.38$\times$) & ~ & ~ & ~ & ~ & ~ & 109.2 (4.21$\times$) & 10.2 \\
% \cline{2-4} \cline{10-11}

% ~ & \multirow{3}{*}{LV-ViT-S} & 0.90/0.84/0.61 & 3.74 (1.75$\times$) & ~ & ~ & ~ & ~ & ~ & 62.8 (3.24$\times$) & 5.5 \\
% \cline{3-4} \cline{10-11}
% ~ & ~ & 0.70/0.39/0.21 & 2.84 (2.31$\times$) & ~ & ~ & ~ & ~ & ~ & 72.8 (3.75$\times$) & 7.1 \\
% \cline{3-4} \cline{10-11}
% ~ & ~ & 0.42/0.21/0.13 & 1.61 (4.08$\times$) & ~ & ~ & ~ & ~ & ~ & 89.1 (4.59$\times$) & 8.33 \\
% \cline{2-11}

% ~ & \multirow{3}{*}{DeiT-B} & 0.90/0.84/0.61 & 9.62 (1.83$\times$) & \multirow{3}{*}{\makecell{161.4 \\ (59\%)}} & \multirow{3}{*}{\makecell{101.8 \\ (19\%)}} & \multirow{3}{*}{\makecell{528.6 \\ (58\%)}} & \multirow{3}{*}{\makecell{1985 \\ (71\%)}} & \multirow{3}{*}{11.352} & 36.1 (3.22$\times$) & 3.1 \\
% \cline{3-4} \cline{10-11}
% ~ & ~ & 0.70/0.39/0.21 & 7.46 (2.36$\times$) & ~ & ~ & ~ & ~ & ~ & 43.3 (3.87$\times$) & 4.7 \\
% \cline{3-4} \cline{10-11}
% ~ & ~ & 0.42/0.21/0.13 & 3.70 (4.76$\times$) & ~ & ~ & ~ & ~ & ~ & 54.8 (4.89$\times$) & 7.1 \\

% \bottomrule
% \end{tabular}
% \end{table*}
% last time table

% \begin{table*}[htb]
% \centering
% % \tabcolsep 3pt

% \vspace{0.05in}
% \caption{Accuracy and hardware results under different pruning settings for various ViTs on ImageNet dataset.}
% \label{tab:result_hardware}
% \vspace{-0.1in}
% \begin{tabular}{c|c|cc|cccc|c}
% \toprule
% \multirow{2}{*}{\bf Design} & \multirow{2}{*}{\bf Model} & \bf Keep Ratio & \bf \#GMACs & \multicolumn{4}{c|}{\bf Resource Utilization} & \bf Power  \\
% \cline{5-9}
% ~ & ~ & (Stage 1/2/3) & (Pruning Rate) & kLUT & kFF & BRAM36 & DSP & (W)  \\
% \midrule
% \multirow{5}{*}{\makecell{Baseline \\ \rev{(16-bit)}}} & DeiT-T & 1/1/1 & 1.30 (1$\times$) & \makecell{115.6 \\ (42\%)} & \makecell{101.5 \\ (19\%)} & \makecell{288.5 \\ (32\%)} & \makecell{1739 \\ (69\%)} & 8.012  \\
% \cline{2-9}

% ~ & DeiT-S & 1/1/1 & 4.60 (1$\times$) & \multirow{2}{*}{\makecell{130.3 \\ (48\%)}} & \multirow{2}{*}{\makecell{102.8 \\ (19\%)}} & \multirow{2}{*}{\makecell{492.5 \\ (54\%)}} & \multirow{2}{*}{\makecell{1754 \\ (70\%)}} & \multirow{2}{*}{10.095} \\
% \cline{2-4} 

% ~ & LV-ViT-S & 1/1/1 & 6.55 (1$\times$) & ~ & ~ & ~ & ~ & ~  \\
% \cline{2-9}

% ~ & DeiT-B & 1/1/1 & 17.60 (1$\times$) & \makecell{144.5 \\ (53\%)} & \makecell{103.9 \\ (19\%)} & \makecell{664.3 \\ (73\%)} & \makecell{1786 \\ (71\%)} & 11.041 \\

% \midrule

% \tbrev \multirow{12}{*}{\makecell{\textbf{\M~} \\ with Token \\ Selector \\ \rev{(8-bit)}}} & \multirow{3}{*}{DeiT-T} & 0.85/0.79/0.51 & 1.00 (1.30$\times$) & \multirow{3}{*}{\makecell{137.6 \\ (47\%)}} & \multirow{3}{*}{\makecell{126 \\ (23\%)}} & \multirow{3}{*}{\makecell{355.5 \\ (39\%)}} & \multirow{3}{*}{\makecell{1968 \\ (71\%)}} & \multirow{3}{*}{\makecell{9.0753}} \\
% \cline{3-4} 
% ~ & ~ & 0.76/0.70/0.41 & 0.90 (1.44$\times$) & ~ & ~ & ~ & ~ & ~ \\
% \cline{3-4} 
% ~ & ~ & 0.70/0.39/0.21 & 0.54 (2.41$\times$) & ~ & ~ & ~ & ~ & ~ \\
% \cline{3-4} 
% % ~ & ~ & 0.42/0.21/0.09 & 0.54 (2.41$\times$) & 70.30 & ~ & ~ & ~ & ~ & ~ & 146.4 (1.87$\times$) & 15.49 \\
% \cline{2-9}

% ~ & \multirow{3}{*}{DeiT-S} & 0.90/0.84/0.61 & 3.86 (1.19$\times$) & \multirow{6}{*}{\makecell{145 \\ (50\%)}} & \multirow{6}{*}{\makecell{100.4 \\ (18\%)}} & \multirow{6}{*}{\makecell{338.5 \\ (37\%)}} & \multirow{6}{*}{\makecell{1955 \\ (70\%)}} & \multirow{6}{*}{10.697}  \\
% \cline{3-4} 
% % ~ & ~ & 0.69/0.43/0.20 & 2.64 (1.74$\times$) & 79.34 & ~ & ~ & ~ & ~ & ~ & 40.9 (1.58$\times$) & 3.82 \\
% \cline{3-4}
% ~ & ~ & 0.70/0.39/0.21 & 2.01 (2.29$\times$) & ~ & ~ & ~ & ~ & ~  \\
% \cline{3-4} 
% ~ & ~ & 0.42/0.21/0.13 & 1.36 (3.38$\times$) & ~ & ~ & ~ & ~ & ~  \\
% \cline{2-4} 

% ~ & \multirow{3}{*}{LV-ViT-S} & 0.90/0.84/0.61 & 3.74 (1.75$\times$) & ~ & ~ & ~ & ~ & ~  \\
% \cline{3-4} 
% ~ & ~ & 0.70/0.39/0.21 & 2.84 (2.31$\times$) & ~ & ~ & ~ & ~ & ~ \\
% \cline{3-4} 
% ~ & ~ & 0.42/0.21/0.13 & 1.61 (4.08$\times$) & ~ & ~ & ~ & ~ & ~  \\
% \cline{2-9}

% ~ & \multirow{3}{*}{DeiT-B} & 0.90/0.84/0.61 & 9.62 (1.83$\times$) & \multirow{3}{*}{\makecell{161.4 \\ (59\%)}} & \multirow{3}{*}{\makecell{101.8 \\ (19\%)}} & \multirow{3}{*}{\makecell{528.6 \\ (58\%)}} & \multirow{3}{*}{\makecell{1985 \\ (71\%)}} & \multirow{3}{*}{11.352} \\
% \cline{3-4} 
% ~ & ~ & 0.70/0.39/0.21 & 7.46 (2.36$\times$) & ~ & ~ & ~ & ~ & ~ \\
% \cline{3-4} 
% ~ & ~ & 0.42/0.21/0.13 & 3.70 (4.76$\times$) & ~ & ~ & ~ & ~ & ~  \\

% \bottomrule
% \end{tabular}
% \end{table*}
% not remove the keep ratio

% \begin{table*}[htb]
% \centering
% % \tabcolsep 3pt

% \vspace{0.05in}
% \caption{Hardware results under different pruning settings for various ViTs on ImageNet.}
% \label{tab:result_hardware}
% \vspace{-0.1in}
% \begin{tabular}{c|c|c|cccc|c c}
% \toprule
% \multirow{2}{*}{\bf Design} & \multirow{2}{*}{\bf Model}  & \bf \#GMACs & \multicolumn{4}{c|}{\bf Resource Utilization} & \bf Power & \multirow{6}{*}{\includegraphics[width=1.95 in]{Figs/fps.pdf}}\\
% \cline{4-7}
% ~ & ~ & (Pruning Rate) & kLUT & kFF & BRAM36 & DSP & (W)  \\

% % \midrule
% \cline{1-8}

% \multirow{5}{*}{\makecell{Baseline \\ \rev{(16-bit)}}} & DeiT-T & 1.30 (1$\times$) & \makecell{115.6 \\ (42\%)} & \makecell{101.5 \\ (19\%)} & \makecell{288.5 \\ (32\%)} & \makecell{1739 \\ (69\%)} & 8.012 &  \\
% \cline{2-8}

% ~ & DeiT-S & 4.60 (1$\times$) & \multirow{2}{*}{\makecell{130.3 \\ (48\%)}} & \multirow{2}{*}{\makecell{102.8 \\ (19\%)}} & \multirow{2}{*}{\makecell{492.5 \\ (54\%)}} & \multirow{2}{*}{\makecell{1754 \\ (70\%)}} & \multirow{2}{*}{10.095} \\
% \cline{2-3} 

% ~ & LV-ViT-S & 6.55 (1$\times$) & ~ & ~ & ~ & ~ & ~  \\
% \cline{2-8}

% ~ & DeiT-B & 17.60 (1$\times$) & \makecell{144.5 \\ (53\%)} & \makecell{103.9 \\ (19\%)} & \makecell{664.3 \\ (73\%)} & \makecell{1786 \\ (71\%)} & 11.041 \\

% % \midrule
% \cline{1-8}

%  \multirow{12}{*}{\makecell{\textbf{\M~} \\ with Token \\ Selector \\ \rev{(8-bit)}}} & \multirow{3}{*}{DeiT-T}  & 1.00 (1.30$\times$) & \multirow{3}{*}{\makecell{137.6 \\ (50\%)}} & \multirow{3}{*}{\makecell{126 \\ (23\%)}} & \multirow{3}{*}{\makecell{355.5 \\ (39\%)}} & \multirow{3}{*}{\makecell{1968 \\ (78\%)}} & \multirow{3}{*}{\makecell{9.0753}} \\
% \cline{3-3} 
% ~ & ~  & 0.90 (1.44$\times$) & ~ & ~ & ~ & ~ & ~  \\
% \cline{3-3} 
% ~ & ~  & 0.75 (1.74$\times$) & ~ & ~ & ~ & ~ & ~ & \multirow{6}{*}{\includegraphics[width=1.95 in]{Figs/efficiency.pdf}} \\
% \cline{3-3} 
% % ~ & ~ & 0.42/0.21/0.09 & 0.54 (2.41$\times$) & 70.30 & ~ & ~ & ~ & ~ & ~ & 146.4 (1.87$\times$) & 15.49 \\
% \cline{2-8}

% ~ & \multirow{3}{*}{DeiT-S} & 3.86 (1.19$\times$) & \multirow{6}{*}{\makecell{145.0 \\ (53\%)}} & \multirow{6}{*}{\makecell{100.4 \\ (18\%)}} & \multirow{6}{*}{\makecell{338.5 \\ (37\%)}} & \multirow{6}{*}{\makecell{1955 \\ (78\%)}} & \multirow{6}{*}{10.697} &  \\
% \cline{3-3} 
% % ~ & ~ & 0.69/0.43/0.20 & 2.64 (1.74$\times$) & 79.34 & ~ & ~ & ~ & ~ & ~ & 40.9 (1.58$\times$) & 3.82 \\
% \cline{3-3}
% ~ & ~  & 2.64 (1.74$\times$) & ~ & ~ & ~ & ~ & ~  \\
% \cline{3-3} 
% ~ & ~  & 2.02 (2.27$\times$) & ~ & ~ & ~ & ~ & ~  \\
% \cline{2-3} 

% ~ & \multirow{3}{*}{LV-ViT-S} & 5.49 (1.19$\times$) & ~ & ~ & ~ & ~ & ~  \\
% \cline{3-3} 
% ~ & ~  & 3.77 (1.74$\times$) & ~ & ~ & ~ & ~ & ~ \\
% \cline{3-3} 
% ~ & ~  & 2.88 (2.27$\times$) & ~ & ~ & ~ & ~ & ~  \\
% \cline{2-8}

% ~ & \multirow{3}{*}{DeiT-B} & 14.79 (1.19$\times$) & \multirow{3}{*}{\makecell{161.4 \\ (59\%)}} & \multirow{3}{*}{\makecell{101.8 \\ (19\%)}} & \multirow{3}{*}{\makecell{528.6 \\ (58\%)}} & \multirow{3}{*}{\makecell{2066 \\ (82\%)}} & \multirow{3}{*}{11.352} \\
% \cline{3-3} 
% ~ & ~  & 10.11 (1.74$\times$) & ~ & ~ & ~ & ~ & ~ \\
% \cline{3-3} 
% ~ & ~  & 7.75 (2.27$\times$) & ~ & ~ & ~ & ~ & ~  \\

% \bottomrule
% \end{tabular}
% \end{table*}
% with figure


\begin{table*}[htb]
\centering
\tabcolsep 3pt

\vspace{0.05in}
\caption{Hardware results under different pruning settings for various ViTs on ImageNet dataset.}
\label{tab:result_hardware}
\vspace{-0.1in}
\begin{tabular}{c|c|ccc|cccc|c|cc}
\toprule
\multirow{2}{*}{\bf Design} & \multirow{2}{*}{\bf Model} & \bf Keep Ratio & \bf \#GMACs & \multirow{2}{*}{\bf Bitwidth} & \multicolumn{4}{c|}{\bf Resource Utilization} & \bf Power & \bf FPS & \bf Energy Effi. \\
\cline{6-9}
~ & ~ & (Stage 1/2/3) & (Pruning Rate) & ~ & kLUT & kFF & BRAM36 & DSP & (W) & (Accl. Rate) & (FPS/W) \\
\midrule
\multirow{4}{*}{Baseline} & DeiT-T & 1/1/1 & 1.30 (1$\times$) & 16 & \makecell{115.6 \\ (42\%)} & \makecell{101.5 \\ (19\%)} & \makecell{288.5 \\ (32\%)} & \makecell{1739 \\ (69\%)} & 8.012 & 78.3 (1$\times$) & 9.77 \\
\cline{2-12}

~ & DeiT-S & 1/1/1 & 4.60 (1$\times$) & 16 & \multirow{2}{*}{\makecell{130.3 \\ (48\%)}} & \multirow{2}{*}{\makecell{102.8 \\ (19\%)}} & \multirow{2}{*}{\makecell{492.5 \\ (54\%)}} & \multirow{2}{*}{\makecell{1754 \\ (70\%)}} & \multirow{2}{*}{10.095} & 25.9 (1$\times$) & 2.57 \\
\cline{2-5} \cline{11-12}

~ & LV-ViT-S & 1/1/1 & 6.55 (1$\times$) & 16 & ~ & ~ & ~ & ~ & ~ & 19.4 (1$\times$) & 1.92 \\
\cline{2-12}

~ & DeiT-B & 1/1/1 & 17.60 (1$\times$) & 16 & \makecell{144.5 \\ (53\%)} & \makecell{103.9 \\ (19\%)} & \makecell{664.3 \\ (73\%)} & \makecell{1786 \\ (71\%)} & 11.041 & 11.2 (1$\times$) & 1.01 \\

\midrule

\multirow{12}{*}{\makecell{\textbf{\M~} \\ with Token \\ Selector}} & \multirow{3}{*}{DeiT-T} & 0.85/0.79/0.51 & 1.00 (1.30$\times$) & 8 & \multirow{3}{*}{\makecell{137.6 \\ (50\%)}} & \multirow{3}{*}{\makecell{126 \\ (23\%)}} & \multirow{3}{*}{\makecell{355.5 \\ (39\%)}} & \multirow{3}{*}{\makecell{1968 \\ (78\%)}} & \multirow{3}{*}{\makecell{9.453}} & 183.4 (2.34$\times$) & 19.4  \\
\cline{3-5} \cline{11-12}
~ & ~ & 0.76/0.70/0.41 & 0.90 (1.44$\times$) & 8 & ~ & ~ & ~ & ~ & ~ & 198.8 (2.54$\times$) & 21.0 \\
\cline{3-5} \cline{11-12}
~ & ~ & 0.70/0.39/0.21 & 0.75 (1.74$\times$) & 8 & ~ & ~ & ~ & ~ & ~ & 271.2 (3.46$\times$) & 28.7 \\
\cline{3-5} \cline{11-12}
% ~ & ~ & 0.42/0.21/0.09 & 0.54 (2.41$\times$) & 70.30 & ~ & ~ & ~ & ~ & ~ & 146.4 (1.87$\times$) & 15.49 \\
\cline{2-12}

~ & \multirow{3}{*}{DeiT-S} & 0.90/0.84/0.61 & 3.86 (1.19$\times$) & 8 & \multirow{6}{*}{\makecell{145 \\ (53\%)}} & \multirow{6}{*}{\makecell{100.4 \\ (18\%)}} & \multirow{6}{*}{\makecell{338.5 \\ (37\%)}} & \multirow{6}{*}{\makecell{1955 \\ (78\%)}} & \multirow{6}{*}{10.697} & 57.0 (2.20$\times$) & 5.33 \\
\cline{3-5} \cline{11-12}
% ~ & ~ & 0.69/0.43/0.20 & 2.64 (1.74$\times$) & 79.34 & ~ & ~ & ~ & ~ & ~ & 40.9 (1.58$\times$) & 3.82 \\
\cline{3-5} \cline{11-12}
~ & ~ & 0.70/0.39/0.21 & 2.64 (1.74$\times$) & 8 & ~ & ~ & ~ & ~ & ~ & 97.1 (3.75$\times$) & 9.08 \\
\cline{3-5} \cline{11-12}
~ & ~ & 0.42/0.21/0.13 & 2.02 (2.27$\times$) & 8 & ~ & ~ & ~ & ~ & ~ & 109.2 (4.22$\times$) & 10.2 \\
\cline{2-5} \cline{11-12}

~ & \multirow{3}{*}{LV-ViT-S} & 0.90/0.84/0.61 & 5.49 (1.19$\times$) & 8 & ~ & ~ & ~ & ~ & ~ & 62.8 (3.24$\times$) & 5.87 \\
\cline{3-5} \cline{11-12}
~ & ~ & 0.70/0.39/0.21 & 3.77 (1.74$\times$) & 8 & ~ & ~ & ~ & ~ & ~ & 72.8 (3.75$\times$) & 6.81 \\
\cline{3-5} \cline{11-12}
~ & ~ & 0.42/0.21/0.13 & 2.88 (2.27$\times$) & 8 & ~ & ~ & ~ & ~ & ~ & 89.1 (4.59$\times$) & 8.33 \\

\cline{2-12}
~ & \multirow{3}{*}{DeiT-B} & 0.90/0.84/0.61 & 14.79 (1.19$\times$) & 8 & \multirow{3}{*}{\makecell{161.4 \\ (59\%)}} & \multirow{3}{*}{\makecell{101.8 \\ (19\%)}} & \multirow{3}{*}{\makecell{528.6 \\ (58\%)}} & \multirow{3}{*}{\makecell{2066 \\ (82\%)}} & \multirow{3}{*}{11.352} & 36.1 (3.22$\times$) & 3.18 \\
\cline{3-5} \cline{11-12}
~ & ~ & 0.70/0.39/0.21 & 10.11 (1.74$\times$) & 8 & ~ & ~ & ~ & ~ & ~ & 43.3 (3.87$\times$) & 3.81 \\
\cline{3-5} \cline{11-12}
~ & ~ & 0.42/0.21/0.13 & 7.75 (2.27$\times$) & 8 & ~ & ~ & ~ & ~ & ~ & 54.8 (4.89$\times$) & 4.83\\

\bottomrule
\end{tabular}
\vspace{-0.1in}
\end{table*}

\subsubsection{Operations in Token Selector}
We compare different selector structures with the same computation cost of 0.9 GMACs in Fig.~\ref{fig:selector_structure}.
MLP-based token selectors outperform convolution-based ones under different combinations. 
In addition, MLP-based selectors bring more benefits than convolution-based ones, since we reuse the GEMM on the FC operation of original ViTs without opening additional resources for the new computation design.

For the activation functions, GELU outperforms ReLU~\cite{agarap2018deep} and Hardswish~\cite{howard2019searching} continuously.
Even though ReLU and Hardswish can be directly deployed on FPGAs, the resource utilization of GELU can be improved to $35\times$$\sim$$572\times$ by the polynomial approximation (Section~\ref{sec:approx}).



% \subsubsection{Effect of the Token Packager}

% We apply our method both with and without the token packager.
% As shown in Table~\ref{tab:token_pack}, our token packager can improve the accuracy by 0.52\% (72.10\% vs. 71.58\% under 0.9 GMACs), and 1.74\% (70.30\% vs. 68.56\% under 0.54 GMACs). 
% \begin{table}[htb]
% \centering
% \tabcolsep 4pt

% \caption{Effect of token pruning and packaging for DeiT-T.}
% \label{tab:token_pack}
% \vspace{-0.1in}
% % \resizebox{0.95\columnwidth}{!}{
% \begin{tabular}{ccccc}
% \toprule
% \bf Method & \bf Pruning & \bf Packaging &\bf \#GMACs &\bf Top-1 Acc. (\%)   \\
% \midrule
% Baseline & No & Yes & 1.30  & 72.20 \\\hline 
% \M & Yes & Yes & 0.90 & 72.10 \\
% \M & Yes & No  & 0.90 & 71.58 \\\hline 
% \M & Yes & Yes & 0.54 & 70.30 \\
% \M & Yes & No  & 0.54 & 68.56\\
% \bottomrule
% \end{tabular}
% \vspace{-0.4cm}
% % }
% \end{table}

% \subsubsection{Effect of Token Selector Number and Location}

% Based on the trend of Figure~\ref{fig:model_location}, we divide the evolution of the pruning rate into 2, 3, and 4 stages.
% We keep the appropriate selectors and re-finetune the whole model (10$\sim$15 epochs).
% In Table~\ref{selector_location}, the 3-6-9 division style has the highest accuracy and the lowest computation cost, just like 3-5-7-9.
% In order to achieve a hardware-efficient design, we choose 3-6-9 as the best.
% For another 3-stage style, 1-6-9, the accuracy and computation cost are both not ideal.
% Due to insufficient encoding, it is difficult to perform token pruning in the front blocks of ViTs.
% Meanwhile, for the 3-6-11 style, both the accuracy and computation cost are slightly inferior to the 3-6-9 style.
% The possible reason is the pruning rate of the second stage should be smaller than the third stage and the coverage of the second stage is too wide.
% The similar reason can support the 2-stage style, 6-9.
% \begin{table}[htb]
% % \small
% \centering
% \tabcolsep 4pt

% \caption{Comparison between different token selector insertion strategies on DeiT-S with 12 blocks.}
% \label{selector_location}
% \vspace{-0.1in}
% % \resizebox{1.0\columnwidth}{!}{
% \begin{tabular}{lccc}
% \toprule
% \textbf{\makecell[l]{Token Selector \\ Location}} &\bf Params (M) &\bf \#GMACs &\bf Top-1 Acc. (\%)   \\
%  \hline
%   3-6-9          &22.13     &\textbf{2.65}      & \textbf{79.34} \\
%   1-6-9          &22.13     &2.70      & 76.10 \\
%   3-6-11         &22.13     &2.72      &78.76  \\\hline 
%   6-9            &22.10     &2.71      &78.53  \\\hline 
%   3-5-7-9        &22.16     &2.66      & 79.34 \\
% \bottomrule
% \end{tabular}
% \vspace{-0.4cm}
% % }
% \end{table}

% % \begin{table*}[htb]
% \centering
% \tabcolsep 3pt

% \vspace{0.05in}
% \caption{Accuracy and hardware results under different pruning settings for various ViTs on ImageNet dataset.}
% \label{tab:result_hardware}
% \vspace{-0.1in}
% \begin{tabular}{c|c|ccc|cccc|c|cc}
% \toprule
% \multirow{2}{*}{\bf Design} & \multirow{2}{*}{\bf Model} & \bf Keep Ratio & \bf \#GMACs & \multirow{2}{*}{\bf Bitwidth} & \multicolumn{4}{c|}{\bf Resource Utilization} & \bf Power & \bf FPS & \bf Energy Effi. \\
% \cline{6-9}
% ~ & ~ & (Stage 1/2/3) & (Pruning Rate) & ~ & kLUT & kFF & BRAM36 & DSP & (W) & (Accl. Rate) & (FPS/W) \\
% \midrule
% \multirow{4}{*}{Baseline} & DeiT-T & 1/1/1 & 1.30 (1$\times$) & 16 & \makecell{115.6 \\ (42\%)} & \makecell{101.5 \\ (19\%)} & \makecell{288.5 \\ (32\%)} & \makecell{1739 \\ (69\%)} & 8.012 & 78.3 (1$\times$) & 9.77 \\
% \cline{2-12}

% ~ & DeiT-S & 1/1/1 & 4.60 (1$\times$) & 16 & \multirow{2}{*}{\makecell{130.3 \\ (48\%)}} & \multirow{2}{*}{\makecell{102.8 \\ (19\%)}} & \multirow{2}{*}{\makecell{492.5 \\ (54\%)}} & \multirow{2}{*}{\makecell{1754 \\ (70\%)}} & \multirow{2}{*}{10.095} & 25.9 (1$\times$) & 2.57 \\
% \cline{2-5} \cline{11-12}

% ~ & LV-ViT-S & 1/1/1 & 6.55 (1$\times$) & 16 & ~ & ~ & ~ & ~ & ~ & 19.4 (1$\times$) & 1.92 \\
% \cline{2-12}

% ~ & DeiT-B & 1/1/1 & 17.60 (1$\times$) & 16 & \makecell{144.5 \\ (53\%)} & \makecell{103.9 \\ (19\%)} & \makecell{664.3 \\ (73\%)} & \makecell{1786 \\ (71\%)} & 11.041 & 11.2 (1$\times$) & 1.01 \\

% \midrule

% \multirow{12}{*}{\makecell{\textbf{\M~} \\ with Token \\ Selector}} & \multirow{3}{*}{DeiT-T} & 0.85/0.79/0.51 & 1.00 (1.30$\times$) & 8 & \multirow{3}{*}{\makecell{137.6 \\ (50\%)}} & \multirow{3}{*}{\makecell{126 \\ (23\%)}} & \multirow{3}{*}{\makecell{355.5 \\ (39\%)}} & \multirow{3}{*}{\makecell{1968 \\ (78\%)}} & \multirow{3}{*}{\makecell{9.453}} & 184.7 (2.36$\times$) & 19.4 \\
% \cline{3-5} \cline{11-12}
% ~ & ~ & 0.76/0.70/0.41 & 0.90 (1.44$\times$) & 8 & ~ & ~ & ~ & ~ & ~ & 198.8 (2.53$\times$) & 21.1 \\
% \cline{3-5} \cline{11-12}
% ~ & ~ & 0.70/0.39/0.21 & 0.75 (1.74$\times$) & 8 & ~ & ~ & ~ & ~ & ~ & 231.1 (2.95$\times$) & 24.4 \\
% \cline{3-5} \cline{11-12}
% % ~ & ~ & 0.42/0.21/0.09 & 0.75 (1.74$\times$) & 70.30 & ~ & ~ & ~ & ~ & ~ & 146.4 (1.87$\times$) & 15.49 \\
% \cline{2-12}

% ~ & \multirow{3}{*}{DeiT-S} & 0.90/0.84/0.61 & 3.86 (1.19$\times$) & 8 & \multirow{6}{*}{\makecell{145.0 \\ (53\%)}} & \multirow{6}{*}{\makecell{100.4 \\ (18\%)}} & \multirow{6}{*}{\makecell{338.5 \\ (37\%)}} & \multirow{6}{*}{\makecell{1955 \\ (78\%)}} & \multirow{6}{*}{10.697} & 57.0 (57.02.20$\times$) & 5.33 \\
% \cline{3-5} \cline{11-12}
% % ~ & ~ & 0.69/0.43/0.20 & 2.64 (1.74$\times$) & 79.34 & ~ & ~ & ~ & ~ & ~ & 40.9 (1.58$\times$) & 3.82 \\
% \cline{3-5} \cline{11-12}
% ~ & ~ & 0.70/0.39/0.21 & 2.64 (1.74$\times$) & 8 & ~ & ~ & ~ & ~ & ~ & 78.3 (3.02$\times$) & 7.3 \\
% \cline{3-5} \cline{11-12}
% ~ & ~ & 0.42/0.21/0.13 & 2.02 (2.27$\times$) & 8 & ~ & ~ & ~ & ~ & ~ & 99.0 (3.82$\times$) & 9.3 \\
% \cline{2-5} \cline{11-12}

% ~ & \multirow{3}{*}{LV-ViT-S} & 0.90/0.84/0.61 & 5.49 (1.19$\times$) & 8 & ~ & ~ & ~ & ~ & ~ & 41.7 (57.02.20$\times$) & 3.9 \\
% \cline{3-5} \cline{11-12}
% ~ & ~ & 0.70/0.39/0.21 & 3.77 (1.74$\times$) & 8 & ~ & ~ & ~ & ~ & ~ & 58.7 (3.03$\times$) & 5.5 \\
% \cline{3-5} \cline{11-12}
% ~ & ~ & 0.42/0.21/0.13 & 2.88 (2.27$\times$) & 8 & ~ & ~ & ~ & ~ & ~ & 74.2 (3.83$\times$) & 6.9 \\

% \cline{2-12}
% ~ & \multirow{3}{*}{DeiT-B} & 0.90/0.84/0.61 & 14.79 (1.19$\times$) & 8 & \multirow{3}{*}{\makecell{161.4 \\ (59\%)}} & \multirow{3}{*}{\makecell{101.8 \\ (19\%)}} & \multirow{3}{*}{\makecell{528.6 \\ (58\%)}} & \multirow{3}{*}{\makecell{2066 \\ (82\%)}} & \multirow{3}{*}{11.352} & 24.0 (2.14$\times$) & 2.1 \\
% \cline{3-5} \cline{11-12}
% ~ & ~ & 0.70/0.39/0.21 & 10.11 (1.74$\times$) & 8 & ~ & ~ & ~ & ~ & ~ & 33.9 (3.02$\times$) & 3.0 \\
% \cline{3-5} \cline{11-12}
% ~ & ~ & 0.42/0.21/0.13 & 7.75 (2.27$\times$) & 8 & ~ & ~ & ~ & ~ & ~ & 42.9 (3.83$\times$) & 3.8 \\

% \bottomrule
% \end{tabular}
% \vspace{-0.1in}
% \end{table*}





% \begin{table*}[htb]
% \centering
% \tabcolsep 3pt

% \vspace{0.05in}
% \caption{Accuracy and hardware results under different pruning settings for various ViTs on ImageNet dataset.}
% \label{tab:result_hardware}
% \vspace{-0.1in}
% \begin{tabular}{c|c|ccc|cccc|c|cc}
% \toprule
% \multirow{2}{*}{\bf Design} & \multirow{2}{*}{\bf Model} & \bf Keep Ratio & \bf \#GMACs & \multirow{2}{*}{\bf Bitwidth} & \multicolumn{4}{c|}{\bf Resource Utilization} & \bf Power & \bf FPS & \bf Energy Effi. \\
% \cline{6-9}
% ~ & ~ & (Stage 1/2/3) & (Pruning Rate) & ~ & kLUT & kFF & BRAM36 & DSP & (W) & (Accl. Rate) & (FPS/W) \\
% \midrule
% \multirow{4}{*}{Baseline} & DeiT-T & 1/1/1 & 1.30 (1$\times$) & 16 & \makecell{115.6 \\ (42\%)} & \makecell{101.5 \\ (19\%)} & \makecell{288.5 \\ (32\%)} & \makecell{1739 \\ (69\%)} & 8.012 & 78.3 (1$\times$) & 9.77 \\
% \cline{2-12}

% ~ & DeiT-S & 1/1/1 & 4.60 (1$\times$) & 16 & \multirow{2}{*}{\makecell{130.3 \\ (48\%)}} & \multirow{2}{*}{\makecell{102.8 \\ (19\%)}} & \multirow{2}{*}{\makecell{492.5 \\ (54\%)}} & \multirow{2}{*}{\makecell{1754 \\ (70\%)}} & \multirow{2}{*}{10.095} & 25.9 (1$\times$) & 2.57 \\
% \cline{2-5} \cline{11-12}

% ~ & LV-ViT-S & 1/1/1 & 6.55 (1$\times$) & 16 & ~ & ~ & ~ & ~ & ~ & 19.4 (1$\times$) & 1.92 \\
% \cline{2-12}

% ~ & DeiT-B & 1/1/1 & 17.60 (1$\times$) & 16 & \makecell{144.5 \\ (53\%)} & \makecell{103.9 \\ (19\%)} & \makecell{664.3 \\ (73\%)} & \makecell{1786 \\ (71\%)} & 11.041 & 11.15 (1$\times$) & 1.01 \\

% \midrule

% \multirow{12}{*}{\makecell{\textbf{\M~} \\ with Token \\ Selector}} & \multirow{3}{*}{DeiT-T} & 0.85/0.79/0.51 & 1.00 (1.30$\times$) & 8 & \multirow{3}{*}{\makecell{137.6 \\ (47\%)}} & \multirow{3}{*}{\makecell{126 \\ (23\%)}} & \multirow{3}{*}{\makecell{355.5 \\ (39\%)}} & \multirow{3}{*}{\makecell{1968 \\ (71\%)}} & \multirow{3}{*}{\makecell{9.453}} & 222.3 (2.84$\times$) & 23.5  \\
% \cline{3-5} \cline{11-12}
% ~ & ~ & 0.76/0.70/0.41 & 0.90 (1.44$\times$) & 8 & ~ & ~ & ~ & ~ & ~ & 238.5 (3.05$\times$) & 25.33 \\
% \cline{3-5} \cline{11-12}
% ~ & ~ & 0.70/0.39/0.21 & 0.90 (2.41$\times$) & 8 & ~ & ~ & ~ & ~ & ~ & 456.0 (5.31$\times$) & 44.0 \\
% \cline{3-5} \cline{11-12}
% % ~ & ~ & 0.42/0.21/0.09 & 0.54 (2.41$\times$) & 70.30 & ~ & ~ & ~ & ~ & ~ & 146.4 (1.87$\times$) & 15.49 \\
% \cline{2-12}

% ~ & \multirow{3}{*}{DeiT-S} & 0.90/0.84/0.61 & 3.86 (1.19$\times$) & 8 & \multirow{6}{*}{\makecell{145 \\ (50\%)}} & \multirow{6}{*}{\makecell{100.4 \\ (18\%)}} & \multirow{6}{*}{\makecell{338.5 \\ (37\%)}} & \multirow{6}{*}{\makecell{1955 \\ (70\%)}} & \multirow{6}{*}{10.697} & 67.3 (2.60$\times$) & 6.3 \\
% \cline{3-5} \cline{11-12}
% % ~ & ~ & 0.69/0.43/0.20 & 2.64 (1.74$\times$) & 79.34 & ~ & ~ & ~ & ~ & ~ & 40.9 (1.58$\times$) & 3.82 \\
% \cline{3-5} \cline{11-12}
% ~ & ~ & 0.70/0.39/0.21 & 2.01 (2.29$\times$) & 8 & ~ & ~ & ~ & ~ & ~ & 121.5 (4.69$\times$) & 11.4 \\
% \cline{3-5} \cline{11-12}
% ~ & ~ & 0.42/0.21/0.13 & 1.36 (3.38$\times$) & 8 & ~ & ~ & ~ & ~ & ~ & 194.8 (7.52$\times$) & 18.2 \\
% \cline{2-5} \cline{11-12}

% ~ & \multirow{3}{*}{LV-ViT-S} & 0.90/0.84/0.61 & 3.74 (1.75$\times$) & 8 & ~ & ~ & ~ & ~ & ~ & 72.3 (3.73$\times$) & 6.8 \\
% \cline{3-5} \cline{11-12}
% ~ & ~ & 0.70/0.39/0.21 & 2.84 (2.31$\times$) & 8 & ~ & ~ & ~ & ~ & ~ & 92.7 (3.87$\times$) & 8.7 \\
% \cline{3-5} \cline{11-12}
% ~ & ~ & 0.42/0.21/0.13  & 1.61 (4.08$\times$) & 8 & ~ & ~ & ~ & ~ & ~ & 173.6 (8.95$\times$) & 16.2 \\

% \cline{2-12}
% ~ & \multirow{3}{*}{DeiT-B} & 0.90/0.84/0.61 & 9.62 (1.83$\times$) & 8 & \multirow{3}{*}{\makecell{161.4 \\ (59\%)}} & \multirow{3}{*}{\makecell{101.8 \\ (19\%)}} & \multirow{3}{*}{\makecell{528.6 \\ (58\%)}} & \multirow{3}{*}{\makecell{2066 \\ (82\%)}} & \multirow{3}{*}{11.352} & 27.8 (3.73$\times$) & 6.8 \\
% \cline{3-5} \cline{11-12}
% ~ & ~ & 0.70/0.39/0.21 & 7.46 (2.36$\times$) & 8 & ~ & ~ & ~ & ~ & ~ & 68.9 (3.87$\times$) & 8.7 \\
% \cline{3-5} \cline{11-12}
% ~ & ~ & 0.42/0.21/0.13  & 3.70 (4.76$\times$) & 8 & ~ & ~ & ~ & ~ & ~ & 104.0 (8.95$\times$) & 9.2 \\

% \bottomrule
% \end{tabular}
% \end{table*}



% \begin{table*}[htb]
% \centering
% % \tabcolsep 3pt

% \vspace{0.05in}
% \caption{Accuracy and hardware results under different pruning settings for various ViTs on ImageNet dataset.}
% \label{tab:result_hardware}
% \vspace{-0.1in}
% \begin{tabular}{c|c|cc|cccc|c|cc}
% \toprule
% \multirow{2}{*}{\bf Design} & \multirow{2}{*}{\bf Model} & \bf Keep Ratio & \bf \#GMACs & \multicolumn{4}{c|}{\bf Resource Utilization} & \bf Power & \bf FPS & \bf Energy Effi. \\
% \cline{6-9}
% ~ & ~ & (Stage 1/2/3) & (Pruning Rate) & kLUT & kFF & BRAM36 & DSP & (W) & (Accl. Rate) & (FPS/W) \\
% \midrule
% \multirow{5}{*}{\makecell{Baseline \\ \rev{(16-bit)}}} & DeiT-T & 1/1/1 & 1.30 (1$\times$) & \makecell{115.6 \\ (42\%)} & \makecell{101.5 \\ (19\%)} & \makecell{288.5 \\ (32\%)} & \makecell{1739 \\ (69\%)} & 8.012 & 78.3 (1$\times$) & 9.77 \\
% \cline{2-11}

% ~ & DeiT-S & 1/1/1 & 4.60 (1$\times$) & \multirow{2}{*}{\makecell{130.3 \\ (48\%)}} & \multirow{2}{*}{\makecell{102.8 \\ (19\%)}} & \multirow{2}{*}{\makecell{492.5 \\ (54\%)}} & \multirow{2}{*}{\makecell{1754 \\ (70\%)}} & \multirow{2}{*}{10.095} & 25.9 (1$\times$) & 2.57 \\
% \cline{2-4} \cline{10-11}

% ~ & LV-ViT-S & 1/1/1 & 6.55 (1$\times$) & ~ & ~ & ~ & ~ & ~ & 19.4 (1$\times$) & 1.92 \\
% \cline{2-11}

% ~ & DeiT-B & 1/1/1 & 17.60 (1$\times$) & \makecell{144.5 \\ (53\%)} & \makecell{103.9 \\ (19\%)} & \makecell{664.3 \\ (73\%)} & \makecell{1786 \\ (71\%)} & 11.041 & 11.2 (1$\times$) & 1.01 \\

% \midrule

% \tbrev \multirow{12}{*}{\makecell{\textbf{\M~} \\ with Token \\ Selector \\ \rev{(8-bit)}}} & \multirow{3}{*}{DeiT-T} & 0.85/0.79/0.51 & 1.00 (1.30$\times$) & \multirow{3}{*}{\makecell{137.6 \\ (47\%)}} & \multirow{3}{*}{\makecell{126 \\ (23\%)}} & \multirow{3}{*}{\makecell{355.5 \\ (39\%)}} & \multirow{3}{*}{\makecell{1968 \\ (71\%)}} & \multirow{3}{*}{\makecell{9.453}} & 184.7 (2.36$\times$) & 19.4 \\
% \cline{3-4} \cline{10-11}
% ~ & ~ & 0.76/0.70/0.41 & 0.90 (1.44$\times$) & ~ & ~ & ~ & ~ & ~ & 198.8 (2.53$\times$) & 21.1 \\
% \cline{3-4} \cline{10-11}
% ~ & ~ & 0.70/0.39/0.21 & 0.54 (2.41$\times$) & ~ & ~ & ~ & ~ & ~ & 322.6 (4.12$\times$) & 28.7 \\
% \cline{3-4} \cline{10-11}
% % ~ & ~ & 0.42/0.21/0.09 & 0.54 (2.41$\times$) & 70.30 & ~ & ~ & ~ & ~ & ~ & 146.4 (1.87$\times$) & 15.49 \\
% \cline{2-11}

% ~ & \multirow{3}{*}{DeiT-S} & 0.90/0.84/0.61 & 3.86 (1.19$\times$) & \multirow{6}{*}{\makecell{145 \\ (50\%)}} & \multirow{6}{*}{\makecell{100.4 \\ (18\%)}} & \multirow{6}{*}{\makecell{338.5 \\ (37\%)}} & \multirow{6}{*}{\makecell{1955 \\ (70\%)}} & \multirow{6}{*}{10.697} & 57.0 (57.02.20$\times$) & 5.33 \\
% \cline{3-4} \cline{10-11}
% % ~ & ~ & 0.69/0.43/0.20 & 2.64 (1.74$\times$) & 79.34 & ~ & ~ & ~ & ~ & ~ & 40.9 (1.58$\times$) & 3.82 \\
% \cline{3-4} \cline{10-11}
% ~ & ~ & 0.70/0.39/0.21 & 2.01 (2.29$\times$) & ~ & ~ & ~ & ~ & ~ & 97.1 (3.75$\times$) & 9.07 \\
% \cline{3-4} \cline{10-11}
% ~ & ~ & 0.42/0.21/0.13 & 1.36 (3.38$\times$) & ~ & ~ & ~ & ~ & ~ & 109.2 (4.21$\times$) & 10.2 \\
% \cline{2-4} \cline{10-11}

% ~ & \multirow{3}{*}{LV-ViT-S} & 0.90/0.84/0.61 & 3.74 (1.75$\times$) & ~ & ~ & ~ & ~ & ~ & 62.8 (3.24$\times$) & 5.5 \\
% \cline{3-4} \cline{10-11}
% ~ & ~ & 0.70/0.39/0.21 & 2.84 (2.31$\times$) & ~ & ~ & ~ & ~ & ~ & 72.8 (3.75$\times$) & 7.1 \\
% \cline{3-4} \cline{10-11}
% ~ & ~ & 0.42/0.21/0.13 & 1.61 (4.08$\times$) & ~ & ~ & ~ & ~ & ~ & 89.1 (4.59$\times$) & 8.33 \\
% \cline{2-11}

% ~ & \multirow{3}{*}{DeiT-B} & 0.90/0.84/0.61 & 9.62 (1.83$\times$) & \multirow{3}{*}{\makecell{161.4 \\ (59\%)}} & \multirow{3}{*}{\makecell{101.8 \\ (19\%)}} & \multirow{3}{*}{\makecell{528.6 \\ (58\%)}} & \multirow{3}{*}{\makecell{1985 \\ (71\%)}} & \multirow{3}{*}{11.352} & 36.1 (3.22$\times$) & 3.1 \\
% \cline{3-4} \cline{10-11}
% ~ & ~ & 0.70/0.39/0.21 & 7.46 (2.36$\times$) & ~ & ~ & ~ & ~ & ~ & 43.3 (3.87$\times$) & 4.7 \\
% \cline{3-4} \cline{10-11}
% ~ & ~ & 0.42/0.21/0.13 & 3.70 (4.76$\times$) & ~ & ~ & ~ & ~ & ~ & 54.8 (4.89$\times$) & 7.1 \\

% \bottomrule
% \end{tabular}
% \end{table*}
% last time table

% \begin{table*}[htb]
% \centering
% % \tabcolsep 3pt

% \vspace{0.05in}
% \caption{Accuracy and hardware results under different pruning settings for various ViTs on ImageNet dataset.}
% \label{tab:result_hardware}
% \vspace{-0.1in}
% \begin{tabular}{c|c|cc|cccc|c}
% \toprule
% \multirow{2}{*}{\bf Design} & \multirow{2}{*}{\bf Model} & \bf Keep Ratio & \bf \#GMACs & \multicolumn{4}{c|}{\bf Resource Utilization} & \bf Power  \\
% \cline{5-9}
% ~ & ~ & (Stage 1/2/3) & (Pruning Rate) & kLUT & kFF & BRAM36 & DSP & (W)  \\
% \midrule
% \multirow{5}{*}{\makecell{Baseline \\ \rev{(16-bit)}}} & DeiT-T & 1/1/1 & 1.30 (1$\times$) & \makecell{115.6 \\ (42\%)} & \makecell{101.5 \\ (19\%)} & \makecell{288.5 \\ (32\%)} & \makecell{1739 \\ (69\%)} & 8.012  \\
% \cline{2-9}

% ~ & DeiT-S & 1/1/1 & 4.60 (1$\times$) & \multirow{2}{*}{\makecell{130.3 \\ (48\%)}} & \multirow{2}{*}{\makecell{102.8 \\ (19\%)}} & \multirow{2}{*}{\makecell{492.5 \\ (54\%)}} & \multirow{2}{*}{\makecell{1754 \\ (70\%)}} & \multirow{2}{*}{10.095} \\
% \cline{2-4} 

% ~ & LV-ViT-S & 1/1/1 & 6.55 (1$\times$) & ~ & ~ & ~ & ~ & ~  \\
% \cline{2-9}

% ~ & DeiT-B & 1/1/1 & 17.60 (1$\times$) & \makecell{144.5 \\ (53\%)} & \makecell{103.9 \\ (19\%)} & \makecell{664.3 \\ (73\%)} & \makecell{1786 \\ (71\%)} & 11.041 \\

% \midrule

% \tbrev \multirow{12}{*}{\makecell{\textbf{\M~} \\ with Token \\ Selector \\ \rev{(8-bit)}}} & \multirow{3}{*}{DeiT-T} & 0.85/0.79/0.51 & 1.00 (1.30$\times$) & \multirow{3}{*}{\makecell{137.6 \\ (47\%)}} & \multirow{3}{*}{\makecell{126 \\ (23\%)}} & \multirow{3}{*}{\makecell{355.5 \\ (39\%)}} & \multirow{3}{*}{\makecell{1968 \\ (71\%)}} & \multirow{3}{*}{\makecell{9.0753}} \\
% \cline{3-4} 
% ~ & ~ & 0.76/0.70/0.41 & 0.90 (1.44$\times$) & ~ & ~ & ~ & ~ & ~ \\
% \cline{3-4} 
% ~ & ~ & 0.70/0.39/0.21 & 0.54 (2.41$\times$) & ~ & ~ & ~ & ~ & ~ \\
% \cline{3-4} 
% % ~ & ~ & 0.42/0.21/0.09 & 0.54 (2.41$\times$) & 70.30 & ~ & ~ & ~ & ~ & ~ & 146.4 (1.87$\times$) & 15.49 \\
% \cline{2-9}

% ~ & \multirow{3}{*}{DeiT-S} & 0.90/0.84/0.61 & 3.86 (1.19$\times$) & \multirow{6}{*}{\makecell{145 \\ (50\%)}} & \multirow{6}{*}{\makecell{100.4 \\ (18\%)}} & \multirow{6}{*}{\makecell{338.5 \\ (37\%)}} & \multirow{6}{*}{\makecell{1955 \\ (70\%)}} & \multirow{6}{*}{10.697}  \\
% \cline{3-4} 
% % ~ & ~ & 0.69/0.43/0.20 & 2.64 (1.74$\times$) & 79.34 & ~ & ~ & ~ & ~ & ~ & 40.9 (1.58$\times$) & 3.82 \\
% \cline{3-4}
% ~ & ~ & 0.70/0.39/0.21 & 2.01 (2.29$\times$) & ~ & ~ & ~ & ~ & ~  \\
% \cline{3-4} 
% ~ & ~ & 0.42/0.21/0.13 & 1.36 (3.38$\times$) & ~ & ~ & ~ & ~ & ~  \\
% \cline{2-4} 

% ~ & \multirow{3}{*}{LV-ViT-S} & 0.90/0.84/0.61 & 3.74 (1.75$\times$) & ~ & ~ & ~ & ~ & ~  \\
% \cline{3-4} 
% ~ & ~ & 0.70/0.39/0.21 & 2.84 (2.31$\times$) & ~ & ~ & ~ & ~ & ~ \\
% \cline{3-4} 
% ~ & ~ & 0.42/0.21/0.13 & 1.61 (4.08$\times$) & ~ & ~ & ~ & ~ & ~  \\
% \cline{2-9}

% ~ & \multirow{3}{*}{DeiT-B} & 0.90/0.84/0.61 & 9.62 (1.83$\times$) & \multirow{3}{*}{\makecell{161.4 \\ (59\%)}} & \multirow{3}{*}{\makecell{101.8 \\ (19\%)}} & \multirow{3}{*}{\makecell{528.6 \\ (58\%)}} & \multirow{3}{*}{\makecell{1985 \\ (71\%)}} & \multirow{3}{*}{11.352} \\
% \cline{3-4} 
% ~ & ~ & 0.70/0.39/0.21 & 7.46 (2.36$\times$) & ~ & ~ & ~ & ~ & ~ \\
% \cline{3-4} 
% ~ & ~ & 0.42/0.21/0.13 & 3.70 (4.76$\times$) & ~ & ~ & ~ & ~ & ~  \\

% \bottomrule
% \end{tabular}
% \end{table*}
% not remove the keep ratio

% \begin{table*}[htb]
% \centering
% % \tabcolsep 3pt

% \vspace{0.05in}
% \caption{Hardware results under different pruning settings for various ViTs on ImageNet.}
% \label{tab:result_hardware}
% \vspace{-0.1in}
% \begin{tabular}{c|c|c|cccc|c c}
% \toprule
% \multirow{2}{*}{\bf Design} & \multirow{2}{*}{\bf Model}  & \bf \#GMACs & \multicolumn{4}{c|}{\bf Resource Utilization} & \bf Power & \multirow{6}{*}{\includegraphics[width=1.95 in]{Figs/fps.pdf}}\\
% \cline{4-7}
% ~ & ~ & (Pruning Rate) & kLUT & kFF & BRAM36 & DSP & (W)  \\

% % \midrule
% \cline{1-8}

% \multirow{5}{*}{\makecell{Baseline \\ \rev{(16-bit)}}} & DeiT-T & 1.30 (1$\times$) & \makecell{115.6 \\ (42\%)} & \makecell{101.5 \\ (19\%)} & \makecell{288.5 \\ (32\%)} & \makecell{1739 \\ (69\%)} & 8.012 &  \\
% \cline{2-8}

% ~ & DeiT-S & 4.60 (1$\times$) & \multirow{2}{*}{\makecell{130.3 \\ (48\%)}} & \multirow{2}{*}{\makecell{102.8 \\ (19\%)}} & \multirow{2}{*}{\makecell{492.5 \\ (54\%)}} & \multirow{2}{*}{\makecell{1754 \\ (70\%)}} & \multirow{2}{*}{10.095} \\
% \cline{2-3} 

% ~ & LV-ViT-S & 6.55 (1$\times$) & ~ & ~ & ~ & ~ & ~  \\
% \cline{2-8}

% ~ & DeiT-B & 17.60 (1$\times$) & \makecell{144.5 \\ (53\%)} & \makecell{103.9 \\ (19\%)} & \makecell{664.3 \\ (73\%)} & \makecell{1786 \\ (71\%)} & 11.041 \\

% % \midrule
% \cline{1-8}

%  \multirow{12}{*}{\makecell{\textbf{\M~} \\ with Token \\ Selector \\ \rev{(8-bit)}}} & \multirow{3}{*}{DeiT-T}  & 1.00 (1.30$\times$) & \multirow{3}{*}{\makecell{137.6 \\ (50\%)}} & \multirow{3}{*}{\makecell{126 \\ (23\%)}} & \multirow{3}{*}{\makecell{355.5 \\ (39\%)}} & \multirow{3}{*}{\makecell{1968 \\ (78\%)}} & \multirow{3}{*}{\makecell{9.0753}} \\
% \cline{3-3} 
% ~ & ~  & 0.90 (1.44$\times$) & ~ & ~ & ~ & ~ & ~  \\
% \cline{3-3} 
% ~ & ~  & 0.75 (1.74$\times$) & ~ & ~ & ~ & ~ & ~ & \multirow{6}{*}{\includegraphics[width=1.95 in]{Figs/efficiency.pdf}} \\
% \cline{3-3} 
% % ~ & ~ & 0.42/0.21/0.09 & 0.54 (2.41$\times$) & 70.30 & ~ & ~ & ~ & ~ & ~ & 146.4 (1.87$\times$) & 15.49 \\
% \cline{2-8}

% ~ & \multirow{3}{*}{DeiT-S} & 3.86 (1.19$\times$) & \multirow{6}{*}{\makecell{145.0 \\ (53\%)}} & \multirow{6}{*}{\makecell{100.4 \\ (18\%)}} & \multirow{6}{*}{\makecell{338.5 \\ (37\%)}} & \multirow{6}{*}{\makecell{1955 \\ (78\%)}} & \multirow{6}{*}{10.697} &  \\
% \cline{3-3} 
% % ~ & ~ & 0.69/0.43/0.20 & 2.64 (1.74$\times$) & 79.34 & ~ & ~ & ~ & ~ & ~ & 40.9 (1.58$\times$) & 3.82 \\
% \cline{3-3}
% ~ & ~  & 2.64 (1.74$\times$) & ~ & ~ & ~ & ~ & ~  \\
% \cline{3-3} 
% ~ & ~  & 2.02 (2.27$\times$) & ~ & ~ & ~ & ~ & ~  \\
% \cline{2-3} 

% ~ & \multirow{3}{*}{LV-ViT-S} & 5.49 (1.19$\times$) & ~ & ~ & ~ & ~ & ~  \\
% \cline{3-3} 
% ~ & ~  & 3.77 (1.74$\times$) & ~ & ~ & ~ & ~ & ~ \\
% \cline{3-3} 
% ~ & ~  & 2.88 (2.27$\times$) & ~ & ~ & ~ & ~ & ~  \\
% \cline{2-8}

% ~ & \multirow{3}{*}{DeiT-B} & 14.79 (1.19$\times$) & \multirow{3}{*}{\makecell{161.4 \\ (59\%)}} & \multirow{3}{*}{\makecell{101.8 \\ (19\%)}} & \multirow{3}{*}{\makecell{528.6 \\ (58\%)}} & \multirow{3}{*}{\makecell{2066 \\ (82\%)}} & \multirow{3}{*}{11.352} \\
% \cline{3-3} 
% ~ & ~  & 10.11 (1.74$\times$) & ~ & ~ & ~ & ~ & ~ \\
% \cline{3-3} 
% ~ & ~  & 7.75 (2.27$\times$) & ~ & ~ & ~ & ~ & ~  \\

% \bottomrule
% \end{tabular}
% \end{table*}
% with figure


\begin{table*}[htb]
\centering
\tabcolsep 3pt

\vspace{0.05in}
\caption{Hardware results under different pruning settings for various ViTs on ImageNet dataset.}
\label{tab:result_hardware}
\vspace{-0.1in}
\begin{tabular}{c|c|ccc|cccc|c|cc}
\toprule
\multirow{2}{*}{\bf Design} & \multirow{2}{*}{\bf Model} & \bf Keep Ratio & \bf \#GMACs & \multirow{2}{*}{\bf Bitwidth} & \multicolumn{4}{c|}{\bf Resource Utilization} & \bf Power & \bf FPS & \bf Energy Effi. \\
\cline{6-9}
~ & ~ & (Stage 1/2/3) & (Pruning Rate) & ~ & kLUT & kFF & BRAM36 & DSP & (W) & (Accl. Rate) & (FPS/W) \\
\midrule
\multirow{4}{*}{Baseline} & DeiT-T & 1/1/1 & 1.30 (1$\times$) & 16 & \makecell{115.6 \\ (42\%)} & \makecell{101.5 \\ (19\%)} & \makecell{288.5 \\ (32\%)} & \makecell{1739 \\ (69\%)} & 8.012 & 78.3 (1$\times$) & 9.77 \\
\cline{2-12}

~ & DeiT-S & 1/1/1 & 4.60 (1$\times$) & 16 & \multirow{2}{*}{\makecell{130.3 \\ (48\%)}} & \multirow{2}{*}{\makecell{102.8 \\ (19\%)}} & \multirow{2}{*}{\makecell{492.5 \\ (54\%)}} & \multirow{2}{*}{\makecell{1754 \\ (70\%)}} & \multirow{2}{*}{10.095} & 25.9 (1$\times$) & 2.57 \\
\cline{2-5} \cline{11-12}

~ & LV-ViT-S & 1/1/1 & 6.55 (1$\times$) & 16 & ~ & ~ & ~ & ~ & ~ & 19.4 (1$\times$) & 1.92 \\
\cline{2-12}

~ & DeiT-B & 1/1/1 & 17.60 (1$\times$) & 16 & \makecell{144.5 \\ (53\%)} & \makecell{103.9 \\ (19\%)} & \makecell{664.3 \\ (73\%)} & \makecell{1786 \\ (71\%)} & 11.041 & 11.2 (1$\times$) & 1.01 \\

\midrule

\multirow{12}{*}{\makecell{\textbf{\M~} \\ with Token \\ Selector}} & \multirow{3}{*}{DeiT-T} & 0.85/0.79/0.51 & 1.00 (1.30$\times$) & 8 & \multirow{3}{*}{\makecell{137.6 \\ (50\%)}} & \multirow{3}{*}{\makecell{126 \\ (23\%)}} & \multirow{3}{*}{\makecell{355.5 \\ (39\%)}} & \multirow{3}{*}{\makecell{1968 \\ (78\%)}} & \multirow{3}{*}{\makecell{9.453}} & 183.4 (2.34$\times$) & 19.4  \\
\cline{3-5} \cline{11-12}
~ & ~ & 0.76/0.70/0.41 & 0.90 (1.44$\times$) & 8 & ~ & ~ & ~ & ~ & ~ & 198.8 (2.54$\times$) & 21.0 \\
\cline{3-5} \cline{11-12}
~ & ~ & 0.70/0.39/0.21 & 0.75 (1.74$\times$) & 8 & ~ & ~ & ~ & ~ & ~ & 271.2 (3.46$\times$) & 28.7 \\
\cline{3-5} \cline{11-12}
% ~ & ~ & 0.42/0.21/0.09 & 0.54 (2.41$\times$) & 70.30 & ~ & ~ & ~ & ~ & ~ & 146.4 (1.87$\times$) & 15.49 \\
\cline{2-12}

~ & \multirow{3}{*}{DeiT-S} & 0.90/0.84/0.61 & 3.86 (1.19$\times$) & 8 & \multirow{6}{*}{\makecell{145 \\ (53\%)}} & \multirow{6}{*}{\makecell{100.4 \\ (18\%)}} & \multirow{6}{*}{\makecell{338.5 \\ (37\%)}} & \multirow{6}{*}{\makecell{1955 \\ (78\%)}} & \multirow{6}{*}{10.697} & 57.0 (2.20$\times$) & 5.33 \\
\cline{3-5} \cline{11-12}
% ~ & ~ & 0.69/0.43/0.20 & 2.64 (1.74$\times$) & 79.34 & ~ & ~ & ~ & ~ & ~ & 40.9 (1.58$\times$) & 3.82 \\
\cline{3-5} \cline{11-12}
~ & ~ & 0.70/0.39/0.21 & 2.64 (1.74$\times$) & 8 & ~ & ~ & ~ & ~ & ~ & 97.1 (3.75$\times$) & 9.08 \\
\cline{3-5} \cline{11-12}
~ & ~ & 0.42/0.21/0.13 & 2.02 (2.27$\times$) & 8 & ~ & ~ & ~ & ~ & ~ & 109.2 (4.22$\times$) & 10.2 \\
\cline{2-5} \cline{11-12}

~ & \multirow{3}{*}{LV-ViT-S} & 0.90/0.84/0.61 & 5.49 (1.19$\times$) & 8 & ~ & ~ & ~ & ~ & ~ & 62.8 (3.24$\times$) & 5.87 \\
\cline{3-5} \cline{11-12}
~ & ~ & 0.70/0.39/0.21 & 3.77 (1.74$\times$) & 8 & ~ & ~ & ~ & ~ & ~ & 72.8 (3.75$\times$) & 6.81 \\
\cline{3-5} \cline{11-12}
~ & ~ & 0.42/0.21/0.13 & 2.88 (2.27$\times$) & 8 & ~ & ~ & ~ & ~ & ~ & 89.1 (4.59$\times$) & 8.33 \\

\cline{2-12}
~ & \multirow{3}{*}{DeiT-B} & 0.90/0.84/0.61 & 14.79 (1.19$\times$) & 8 & \multirow{3}{*}{\makecell{161.4 \\ (59\%)}} & \multirow{3}{*}{\makecell{101.8 \\ (19\%)}} & \multirow{3}{*}{\makecell{528.6 \\ (58\%)}} & \multirow{3}{*}{\makecell{2066 \\ (82\%)}} & \multirow{3}{*}{11.352} & 36.1 (3.22$\times$) & 3.18 \\
\cline{3-5} \cline{11-12}
~ & ~ & 0.70/0.39/0.21 & 10.11 (1.74$\times$) & 8 & ~ & ~ & ~ & ~ & ~ & 43.3 (3.87$\times$) & 3.81 \\
\cline{3-5} \cline{11-12}
~ & ~ & 0.42/0.21/0.13 & 7.75 (2.27$\times$) & 8 & ~ & ~ & ~ & ~ & ~ & 54.8 (4.89$\times$) & 4.83\\

\bottomrule
\end{tabular}
\vspace{-0.1in}
\end{table*}
% \begin{figure*}[tb]
% \centering
% \includegraphics[width=1.8\columnwidth]{Figs/hardware_tmp.pdf}
% \vspace{-0.15in}
% \caption{\rev{Accuracy and hardware results under different pruning settings for various ViTs on ImageNet dataset.}}
% \label{fig:hd_tmp_results}
% \vspace{-0.3cm}
% \end{figure*}


\subsection{Hardware Results} 

Multiple hardware accelerators are designed according to the number of heads in a specific ViT. As shown in Table~\ref{tab:result_hardware}, with the same total degree of computation parallelism, the resource utilization and power of DeiT-S and LV-ViT-S designs are higher than those of DeiT-T ones, since DeiT-T has 3 heads while DeiT-S and LV-ViT-S all have 6 heads, requiring more BRAM space to accommodate data of all the attention heads. This trend is similar for DeiT-B (12 heads).

Compared with the baseline hardware designs (16-bit and no token pruning), the accelerators with token selector in \M~framework utilize 9\% more DSPs and 8\% more LUTs for DeiT-T, 8\% more DSPs and 5\% more LUTs for DeiT-S and LV-ViT-S, 11\% more DSPs and 6\% more LUTs for DeiT-B.
This demonstrates that the control flow to support adaptive token pruning introduces negligible resource utilization overhead.
After the token pruning, the frame rate increases from 78.3 FPS to 142.7 FPS (1.82$\times$) for DeiT-T,  from 25.9 FPS to 57.6 FPS (2.22$\times$) for DeiT-S, from 19.4 FPS to 46.9 FPS (2.42$\times$) for LV-ViT-S, and from 11.2 FPS to 28.9 FPS (2.58$\times$) for DeiT-B.
Furthermore, we deploy the 8-bit fixed-point quantization on models to achieve another 1.90$\times$ speedup, ending up the final speedup with 3.46$\times$ (271.2 FPS) for DeiT-T, 4.22$\times$ (109.2 FPS) for DeiT-S, 4.59$\times$ (89.1 FPS) for LV-ViT-S, and 4.89$\times$ (54.8 FPS) for the DeiT-B.
% Note that all the quantization process do not lose accuracy, except for 1.2\% on DeiT-T.

\subsubsection{Comparisons with CPUs and GPUs}

% \begin{figure}[htbp]
% \centering
% % \vspace{-0.2in}
% \includegraphics[width=1\columnwidth]{Figs/cpu-gpu.pdf}
% \vspace{-0.1in}
% \caption{Comparison of energy efficiency and speedup between \M~over and embedded CPU/GPU. %\zf{update this figure with the correct reults. Right figure, add another breakdown of speedup from quantization, i.e., FPGA-16bit, FPGA-8-bit.}
% }
% \label{fig:speedup_energy}
% \vspace{-0.2cm}
% \end{figure}

% \begin{figure}[tb]
% \vspace{-0.1cm}
% \begin{minipage}{0.50\linewidth}
%     \centering
%     % \captionsetup{type=figure}
%     \includegraphics[width=1.1\columnwidth]{Figs/energy_efficiency_breakdown.pdf}
%     % \caption{}
%     \label{fig:headattention}
% 	\end{minipage}\hfill
% 	\begin{minipage}{0.50\linewidth}
%         \centering
%         % \captionsetup{type=figure}
%         \includegraphics[width=1.1\columnwidth]{Figs/speedup_breakdown.pdf}
% % \caption{The CKA between the final CLS token and other tokens.}
% 	\end{minipage}
% \vspace{-0.4cm}
% \caption{\rev{Comparison with energy efficiency between HeatViT over and embedded CPU/GPU. Speedup by pruning is shown here.}}
% \label{fig:comparison}
% \end{figure}

\begin{figure}[tb]
\centering
\includegraphics[width=1\columnwidth]{Figs/speedup_breakdown.pdf}
\vspace{-0.3in}
\caption{Comparison of energy efficiency between HeatViT and TX2 CPU/GPU with the improvement breakdown of different techniques.}
\label{fig:comparison}
%\vspace{-0.1in}
\end{figure}

We also test DeiT-T, DeiT-S, LV-ViT-S, and DeiT-B on Jetson TX2 with 4-core ARM CPU and NVIDIA Pascal GPU, and compared them with our FPGA (ZCU102) implementation.
Since MSA and FFN computations are reduced by token pruning, CPUs and GPUs can also be accelerated.
And TX2 CPU/GPU does not support low-bit computation,  so we only present the full precision model for them with adaptive token pruning as shown in Fig.~\ref{fig:comparison}.
All the results are normalized against the original model on TX2 CPU without token pruning. 
First, our final FPGA implementation achieves the highest $1827\times\sim3013\times$ speedup with 9.453W, 10.697W, and 11.352W power for different designs. 
While the baseline FPGA design~\cite{lizh2022logic} (16-bit and no token pruning) achieves 373$\times$$\sim$870$\times$ speedup, token pruning can bring 1.82$\times$$\sim$2.58$\times$ speedup and ambitious 8-bit quantization can contribute another 1.90$\times$ speedup.
Second, with token pruning, TX2 GPU achieves 647$\times$$\sim$814$\times$ speedup with 12W power and TX2 CPU achieves 1.78$\times$$\sim$2.67$\times$ speedup with 4W power.
For the energy efficiency, our FPGA implementation achieves 4.8 FPS/W$\sim$28.7 FPS/W, which is 242.6$\times$$\sim$719.0$\times$ higher than TX2 CPU and 3.0$\times$$\sim$4.7$\times$ higher than TX2 GPU (with token pruning).
%Eventually, the speed of \M~is significantly improved, demonstrating the effective combination of data-level pruning and 8-bit fixed-point quantization techniques in \M.
%Moreover, it has a highly parallelized structure and all the computation components are friendly to be accelerated.


% \subsubsection{End-to-End Performance with Approximation}
% \textbf{Roofline Analysis;}
% \subsubsection{Breakdown of Speedup}
% \textbf{Efficiency-Accuracy Trade-offs.}