\section{introduction}
Hyperspectral remote sensing (HSRS) has attracted wide attention in recent years, which is able to capture abundant information about the landform with hundreds of fine-grained and continuous spectral bands~\cite{landgrebe2002hyperspectral,ghamisi2017advances,bioucas2013hyperspectral}. 
As one of the most important tasks in HSRS, hyperspectral image classification (HSIC) aims to assign each pixel with one label from a pre-defined set, such as gravel and water~\cite{he2017recent,zhu2020residual}. 
HSIC has been widely employed in various situations including crop discrimination~\cite{eddy2014weed}, military reconnaissance~\cite{khan2018modern}, and geological examination~\cite{villa2011hyperspectral}.




\begin{figure}[t]
\centering
\includegraphics[width=0.475\textwidth]{figures/motivation.pdf}
\caption{The motivation of the proposed PDML framework. 
Existing HSIC methods take as input an image patch centered at the concerned pixel and output a single classification score for the center pixel, which suffers from the inaccuracy of the training signals caused by the spectral uncertainty (i.e., intraclass variability and interclass similarity) and the label uncertainty (i.e., mixed pixels and lack of labels for surrounding pixels). 
The proposed PDML framework employs an encoder-decoder to model the uncertainty of each pixel by a probabilistic distribution with different variances. 
We further adopt a probabilistic deep metric learning method to constrain the distances between distributions, which can suppress the influence of the uncertainty in training samples to obtain a model more tolerant to the possible inconsistency in the training data.
(Best viewed in color.)
}
\label{fig:motivation}
\vspace{-5mm}
\end{figure}

A core issue for hyperspectral image classification is the spectral uncertainty including the spectral variability between intraclass materials and the spectral similarity between interclass materials~\cite{zare2013endmember, chang2000information}. 
The spectral uncertainty results from a number of factors, such as natural spectrum variation, instrument noises, atmospheric effects, etc.~\cite{he2017recent, shaw2002signal}. 
Under such circumstances, the spectral characteristics alone are incapable of accurately and robustly discriminating a certain pixel in a hyperspectral image.
This motivates the recent spectral-spatial classification methods to additionally take spatial information into consideration in order to reduce the effect of spectral uncertainty~\cite{sun2014supervised,paoletti2019deep,fang2014spectral,kang2013spectral}. 
They employ spatial neighborhoods of the concerned pixel as well as their spectral features and take the resulting 3D patch as the input with a Convolutional Neural Network (CNN) to process hyperspectral images~\cite{hamida20183,chen2016deep,article,li2017spectral}. 
Essentially, these methods perform multiple sampling of adjacent pixels and employ a voting scheme to enhance the confidence of the overall prediction, where the voting weights are implicitly learned in the CNNs.
The improvement is effective assuming that adjacent pixels capture the same type of materials. 
However, this assumption might not hold due to the native low spatial resolution of hyperspectral sensors~\cite{audebert2019deep,zhang2018diverse}, making surrounding pixels possible to encoder features from different materials.
Furthermore, even a single pixel might contain different materials (i.e., mixed pixel)~\cite{rajabi2015sparsity,hsieh2001effect,xie2018unsupervised}.
This brings uncertainty to the labels of both surrounding pixels and center pixels (i.e., label uncertainty), leading to the inaccuracy of the supervisory signal.



In this paper, we propose a probabilistic deep metric learning (PDML) framework to address the aforementioned spectral uncertainty and label uncertainty, as demonstrated in Figure~\ref{fig:motivation}. 
We employ a high-dimension probabilistic distribution to represent a pixel to directly model the uncertainty. 
Different from existing methods which learn a single deterministic representation only for the concerned center pixel, we use an encoder-decoder architecture to learn a probabilistic representation for each pixel in the patch.
We similarly assign the same label to all the pixels due to the lack of ground truth labels for the surrounding pixels but accompany them with different degrees of uncertainty.
We achieve this by enforcing a larger variance for a further pixel since it is more probable to capture a different material from the center pixel. 
We further propose a deep probabilistic metric learning method to constrain the similarities between probabilistic distributions to increase interclass variance and decrease intraclass variance.
We employ Monte Carlo sampling to generate a number of samples following the corresponding distribution to represent each pixel and apply a metric learning loss to them. 
The overall framework can be learned simultaneously in an end-to-end manner during training, and we only extract the mean vector of the center pixel for classification during inference, introducing no additional workload compared with existing methods.
The proposed PDML framework directly models the implicit uncertainty within the hyperspectral image patches, enabling our model to adaptively rectify the influence of possible inconsistent training samples.
Furthermore, we treat all the pixels in a patch as training samples with the same label but different variances and impose constraints on all of them to exploit full information from the patch, compensating for the lack of training data which is a common problem in HSIC.
Extensive experiments on four widely used datasets demonstrate the effectiveness of the proposed PDML framework.

In general, we summarize our key contributions as follows:

1) To the best of our knowledge, we are the first to consider both the spectral uncertainty and label uncertainty in a hyperspectral image and propose to use a probabilistic embedding to represent each pixel. 


2) We propose a probabilistic deep metric learning method, where we generate various samples following the distribution of each pixel by Monte Carlo sampling and impose a metric learning loss to enlarge the interclass distances as well as reduce the intraclass distances in the embedding space.

3) We conduct experiments on four widely used hyperspectral datasets, which demonstrates that our proposed framework can be applied to various existing methods to boost their performance and further achieve the state-of-the-art.