 \documentclass[10pt,preprint]{aastex}
%\documentclass[10pt,linenumbers]{aastex63}
 \usepackage{natbib}
 \bibliographystyle{apj}

%-----------------------------------------------------------------------
%                       TeX - Special symbols
%----------------------------------------------------------------------
\def\a{\"a}
\def\o{\"o}
\def\u{\"u}
\def\A{\"A}
\def\O{\"O}
\def\U{\"U}
\def\etal{et al.~}
\def\oo{\char'034}
\def\OO{\char'037}
\def\EE{\mathchar"3245}
\def\ang{\AA}
\def\arcsec{\hbox{$^{\prime\prime}$}}
\def\arcmin{\hbox{$^{\prime}$}}
\def\gapprox{\lower.4ex\hbox{$\;\buildrel >\over{\scriptstyle\sim}\;$}}
\def\lapprox{\lower.4ex\hbox{$\;\buildrel <\over{\scriptstyle\sim}\;$}}
\def\captio#1{\caption{\small {#1} \normalsize}}

%----------------------------------------------------------------------
%                       AASTeX - LATEX - file
%----------------------------------------------------------------------
\shortauthors{Aschwanden 2022}
\shorttitle{Universality of Power Law Slopes: HMI and IRIS}

\begin{document}
\renewcommand{\topfraction}{0.95}
\renewcommand{\bottomfraction}{0.95}
\renewcommand{\textfraction}{0.05}
\renewcommand{\floatpagefraction}{0.95}
\renewcommand{\dbltopfraction}{0.95}
\renewcommand{\dblfloatpagefraction}{0.95}

%{\sl  Manuscript, version ....; accepted ... }

\title{ The Universality of Power Law Slopes 
	in the Solar Photosphere and Transition Region
	Observed with HMI and IRIS}

\author{Markus J. Aschwanden}
\affil{Lockheed Martin, Solar and Astrophysics Laboratory (LMSAL),
       Advanced Technology Center (ATC),
       A021S, Bldg.252, 3251 Hanover St.,
       Palo Alto, CA 94304, USA;
       e-mail: aschwanden@lmsal.com}

\and 
\author{Nived Vilangot Nhalil}
\affil{Armagh Observatory and Planetarium, College Hill, Armagh BT61 9DG, UK}

\begin{abstract}
We compare the size distributions of {\sl self-organized criticality (SOC)}
systems in the solar photosphere and the transition region, using magnetogram
data from {\sl Helioseismic and Magnetic Imager (HMI)} and {\sl Interface 
Region Imaging Spectrograph (IRIS)} data. For each dataset we fit a
combination of a Gaussian and a power law size distribution function, 
which yields information on four different physical processes: 
(i) photosopheric granulation convection dynamics (explaining the Gaussian
random noise distribution in IRIS data);
(ii) spicular plage events in the transition region 
(explaining the power law size distribution in IRIS data);
(iii) salt-and-pepper small-scale magnetic structures
(explaining the random noise distributions in HMI magnetograms);
and (iv) magnetic reconnection processes in flares and nanoflares
(explaining the power law size distribution in HMI data).
We find a high correlation (CCC=0.97) between IRIS and HMI data.
Datasets with magnetic flux balance are found to match the 
SOC-predicted power law slope $\alpha_F=1.80$ for mean fluxes, 
which confirms the universality of SOC-inferred flux 
size distributions, and agrees with the results of
Parnell et al.~(2009), $\alpha_F=1.85\pm0.14$.
\end{abstract}

\keywords{methods: statistical --- fractal dimension --- Sun: transition region --- 
	  solar granulation --- solar photosphere ---}

\section{	INTRODUCTION 		}  

The atmospheric structure of the Sun consists of the photospheric
layer on the solar surface, the chromosphere, the transition region,
the corona, and solar wind regions, which all host different physical
processes, characterized by the electron density, the electron 
temperature, and the magnetic field strength. In this study we 
sample very diverse temperature structures, from $T_e \approx 5800$ K
observed in photospheric magnetograms with the {\sl Helioseismic 
and Magnetic Imager (HMI)}, to $T_e \approx 10^4-10^6$ K, observed
in Slitjaw images (SJI) of the 1400 \ang\ channel of IRIS,
which are dominated by the Si IV 1394 \ang\ and 1403 \ang\
resonance line, and form in the transition region
(Rathore and Carlsson 2015; Rathore et al.~2015).
Due to this huge temperature range, different physical processes
are dominant in the various temperature regimes
(Gallagher et al.~1998; Warren et al.~2016), and thus we do
not know {\sl a priori} whether the concept of {\sl self-organized criticality 
(SOC)} systems is applicable (Aschwanden 2011; 2014; 2016;
McAteer et al.~2016; Warren et al.~2016). More specifically, 
we want to understand the functional shapes of observed occurrence frequency 
(size) distributions, and whether they exhibit power law function (slopes) 
with universal validity in different temperature and wavelength regimes.

There is an ongoing debate on the functional form of size distributions
in avalanching SOC processes, such as: 
a power law function, 
a log-normal distribution (Verbeeck et al.~2019), 
a Pareto distribution (Hosking and Wallis 1987), 
a Lomax distribution (Lomax 1954; Giles et al.~2011),
or a Weibull distribution (Weibull 1951), for instance. 
Since all these 
functional forms are close to a power law function on the right-hand 
side of the size distribution, which is also called the ``fat-tail'',  
various linear combinations of these functional forms have been
found to fit the observed size distributions to a comparable degree  
(Munoz-Jaramillo et al.~2015). In this study we use a combination of
(Gaussian) linear random and (power law) nonlinear random structures.
The linear (Gaussian) component describes the granulation dynamics 
(visible in IRIS data), as well as the salt-and-pepper structure 
(visible in HMI magnetograms). On the other side, the nonlinear 
power-law component may be produced by the spicular dynamics 
(visible in IRIS data), or by magnetic reconnection dynamics 
of small-scale features and nanoflares (visible in HMI magnetograms).
Gaussian distributions have been tested with Yohkoh soft X-ray data
(Katsukawa and Tsuneta 2001).
Lognormal distributions, which are closest to our Gaussian-plus-power-law
method used here, have been previously studied for 
Quiet-Sun FUV emission (Fontenla et al.~2007),
solar flares (Verbeek et al. 2019),
the solar wind (Burlaga and Lazarus 2000),
accretion disks (Kunjaya et al.~2011), 
and are discussed also in Ceva and Luzuriaga (1998), 
Mitzenmacher (2004), 
and Scargle (2020).

There are at least two approaches to model SOC systems: (i) the microscopic
concept that is based on avalanches driven by next-neighbor interactions
in a lattice grid (Bak et al.~1987; Bak et al.~1988; Bak 1996), 
pioneered for solar flares by Lu and Hamilton (1991);  
and (ii) the macroscopic concept, where
scaling laws and correlations between SOC parameters determine the
statistical size distributions (Aschwanden 2012; 2016; 2022a), 
and waiting time distributions.
A new aspect of this study is the invention of a single-image
algorithm to derive approximate size distributions
$N(F) \propto F^{-\alpha_F}$. One of our tests consists of
comparing the slope $\alpha_F$ with the SOC-predicted value, 
i.e., $\alpha_F=1.80$, which agrees also with observations
of a more than the five decades extending size distribution
inferred by Parnell et al.~(2009), i.e., $\alpha_F=1.85\pm0.14$.
Another crucial test is the power law slope $\alpha_E$ of nanoflare 
energies, which is decisive for testing the viability of coronal 
heating energetics (Hudson 1991; Krucker and Benz 1998;
Vilangot Nhalil et al.~2020; Aschwanden 2022b).

The content of this paper includes data analysis (Section 2),
a discussion (Section 3), and conclusions (Section 4). 

\section{	DATA ANALYSIS 		}  

When we observe solar emission at {\sl near ultra-violet (NUV)} and
{\sl far ultra-violet (FUV)} wavelengths, we may gather photons from
photospheric granulation structures (granules), as well as from
spicules in plages in the transition region (at formation temperatures
of $T_e \approx 10^4-10^6$). Consequently, we have to deal with
multiple size distribution functions, which may include 
linear random processes (Gaussian noise) and/or nonlinear avalanche 
processes with power law distribution functions, also known as 
``fat-tail'' distribution functions, which occur natually in 
{\sl self-organized criticality (SOC)} systems.

\subsection{	Definitions of Flux Distributions		}

In the following we attempt to model event statistics
with a combination of (i) a Gaussian distribution (originating from
random processes, such as granular convection), and (ii) a power law 
distribution, e.g., created by spicular activity in the transition 
region, (Fig.~1). The Gaussian noise is defined in the standard way,
\begin{equation}
	N(F)\ dF = N_0 \exp{\left(-{(F-F_0)^2 \over 2 \sigma_F^2}\right)} \ dF \ ,
\end{equation}
where $F$ is the flux averaged over the duration of an event 
(measured here at a wavelength of 1400 \ang ), $N(F)$ is the number 
of observed structures, $F_0$ is the
mean value, $\sigma_F$ is one standard deviation, and
$N_0$ is the normalized number of events accumulated over time and area.

The second distribution we employ in our analysis is a
power law distribution function, which is defined in the
simplest way by,
\begin{equation}
	N(F)\ dF = N_0 \left({F \over F_0}\right)^{-\alpha_F} \ dF \ ,
\end{equation}
where $\alpha_F$ is the power law slope of the relevant part
of the distribution function.

The flux $F(t)$ of an event at time $t$ is defined for IRIS data by, 
\begin{equation}
	F_{\rm IRIS}(t) = {4\ \pi\ f(t) \ E_{\lambda}\ k \over A(t)\ \Omega} , \quad  
	[{\rm erg}\ {\rm cm}^{-2} {\rm s}^{-1}] 
\end{equation}
where $f(t)$ is the mean observed flux at time $(t)$ in $DN/s$ (data number per second), 
$E_{\lambda}$ is the energy of the photon, 
$k$ is the factor that converts the DN to the number of photons, 
$\Omega$ is the {\sl slit-jaw image (SJI)} in steradian, 
$A$ [cm$^{-2}$] is the area, 
and the background is subtracted (Vilangot Nhalil et al.~2020).
In order to convert IRIS-observed fluxes into a SOC-inferred
fluxes $F$, which has the physical units of $[{\rm erg}\ {\rm s}^{-1}]$.
spatial integration and temporal averaging is required,
\begin{equation}
	F_{SOC} = {\int_t^{t+T} F_{\rm IRIS}(t) \ A(t)\ dt \over T}  , \quad  
	[{\rm erg}\ {\rm s}^{-1}] \ .
\end{equation}

\subsection{	Analysis of IRIS Data 	}

The 12 analyzed 1400 \ang\ SJI images $F(x,y)$ of IRIS are shown in 
Fig.~2, which are identical in time and FOV (field-of-view) with
those of Vilangot Nhalil et al.~(2020), and are also identical
with those used in the study on fractal dimension measurements
(Aschwanden and  Vilangot Nhalil~2022). 
The 12 IRIS maps shown in Fig.~2 have the following
color code: The Gaussian distribution with values 
$F(x,y) < F_{thr}$ below a threshold of $F_{thr}=F_{avg}+F_{sig}$
is rendered with orange-to-red colors, while the power law
function with the fat-tail $F(x,y) > F_{thr}$ is masked out 
with white color. In other words, all the orange-to-red regions
in the IRIS maps visualize the locations of linear random noise
(produced by granulation dynamics), while the white regions
mark the location of SOC-driven nonlinear avalanches
(probably produced by spicular dynamics in the transition region).
An even crispier representation of the spicular component 
$F(x,y) > P_{thr}$, is displayed with a black-and-white rendering
(Fig.~3), where black depicts locations with a power law 
distribution, and white demarcates locations with a Gaussian 
distribution

The information content of an IRIS image can be described
with a 2-D array of flux values $F(x,y)$ at a given time $t$,
or alternatively with a 1-D histogram $N(F)$. From the
histogram $N(F)$ we calculate the average value $F_{avg}$ 
and the standard deviation $\pm F_{sig}$, which corresponds 
to the Gaussian function, with a peak at $F_0$ and a standard 
deviation $F_{sig}=(F_2-F_0)$ in Fig.~(1) top panel. 
These averages and standard deviations, as well as
the maximum flux values $F_{max}$ [DN/s] are listed in Table 1.
We see that the maximum flux values vary from $F_{max}=26$
to 501 {\sl DN/s}. Since we want to fit a two-component distribution
function (i.e., with a Gaussian and a power law), we need to
introduce a separator between the two distributions,
which we choose to be one standard deviation above the
average (see Fig.~1). We fit then both distribution functions 
(Eqs.~1 and 2) separately, the Gaussian function in the range
of $[F_1, F_2]$, and the power law function in the range
of $[F_2, F_3]$, as depicted in Fig.~1. We are fitting 
the distribution functions with a standard Gaussian fit
method, and with a standard linear regression fit for the 
logarithmic flux function. Note that the power law function $N(S)$
appears to be a straight line in a logarithmic display only
(Fig.~1 bottom panel), i.e., log(N)-log(S), but not in a 
linear representation (Fig.~1 top panel), i.e., lin(N)-lin(S), 
as used here.

The results of the fitting of the observed histograms are 
shown for all 12 datasets in Fig.~(4), where the Gaussian fit 
is rendered with a blue color, and the power law fit with a red color. 
We see that our two-component model for the distribution
function does produce acceptable fits to all analyzed
IRIS data (histograms in Fig.~4), at least in the left
half of the histograms. 

If we would assume that all fluxes are generated by linear
random noise, we would not be able to fit the nonlinear
data at all. Obviously, we would under-predict most of the 
fluxes substantially (blue dashed curves in Fig.~4), which
underscores that the ``fat-tail'' power law function, 
a hallmark of SOC processes, is highly relevant for
fitting the observed IRIS 1400 \ang\ data here. 

In a next step we investigate the numerical values of
the power law slopes $\alpha_F$ of the flux distribution
parameters $F$, which are listed in the third column
of Table 1. At a first glance, it appears that these
values vary wildly in a range of $\alpha_F=0.55$ to
2.15. However, Vilangot Nhalil et al.~(2020) classified
the 12 analyzed datasets into 4 cases containing a
sunspot, and 8 cases containing plages in the transition
region without sunspots. From this bimodal behavior
it was concluded that the power law index of the energy
distribution is larger in plages ($\alpha_E > 2$),
compared with sunspot-dominated active regions 
($\alpha_{E} < 2$), (Vilangot Nhalil et al.~2020). 
In our investigation here, the 4 cases with sunspots
exhibit substantially flatter power law slopes 
$\alpha_F$ (except \#3),
which indicates that sunspot-dominant
distributions are indeed significantly different
from those without sunspots (Table 1). Actually,
we find an even better predictor of this bimodal
behavior, by using the maximum flux $F_{max}$
(Column 6 in Table 1). We find that
flux distributions $N(F) \propto F^{-\alpha_F}$ 
with maximum fluxes less than $F_{max} \lapprox
50$ [DN/s] exhibit a power law value of
\begin{equation}
	\alpha_F^{obs} \approx 1.67 \pm 0.14,
	\quad  F_{max} < 50\ {\rm DN/s} \ ,
\end{equation}
which includes the five datasets \#6, 7, 8, 9, 11.
In contrast, the seven other datasets \#1, 2, 3,
4, 5, 10, 12 have consistently higher maximum
values, $F_{max} \gapprox 50$ DN/s. 
Instead of using the maximum values $F_{max}$,
we can also use the average fluxes
$F_{avg}$ and find the same bimodal behavior.

Even more significant is that this power law value
is consistent with the theoretical prediction of
the power law slopes (Aschwanden 2012; 2016; 2022a),
\begin{equation}
	\alpha_{\rm F,SOC} = {9 \over 5} = 1.80 \ .
\end{equation}
Thus we can conclude that flux distributions have
a power law slope that agrees with the theoretial
prediction for small maximum fluxes, while
distributions with large fluxes or with sunspots
display flatter slopes. Apparently, more complex
spatial structures produce flatter slopes.

\subsection{	Avalanche Detection Approximations	}

The standard method to sample size distributions $N(F)$ of
SOC avalanches is generally carried out by an algorithm
that detects an avalanche event above some given threshold
$F > F_{thr}$, traces its spatial and temporal evolution,
and determines the time-evolving envelope of the saturated
avalanche.
This has been accomplished for the same 12 IRIS datasets
in the study of Vilangot Nhalil et al.~(2020). Because
the development of an automated feature recognition code
is a complex and time-consuming task and needs extensive
testing, we explore here two
new methods that are much simpler to apply and require
much less data to determine the underlying power law slopes
$\alpha_F$. 

Our Method 1 samples all pixels of the 12 IRIS maps
$F(x_i,y_j,t_i), i=0,...,N_x, j=0,...,N_y, k=1,...,12$,
Essentially, this strategy assumes that every single
pixel represents an avalanche, differing from each other
by the peak flux $F(x_i,y_j,t_k)$ only. This Method 1
was used in the calculations of the values $\alpha_{\rm F1}$ 
listed in Table 1. 

In order to test the robustness of Method 1 we conceive
a Method 2, where only those pixels were sampled that
represent a local peak relative to the 4 next neighbors,,
\begin{equation}
	F(x_i,y_j) > [F(x_{i-1},y_j), F(x_{i+1},y_j),
		      F(x_i,y_{j-1}), F(x_i,y_{j+1}] \ .
\end{equation}
From the 12 IRIS maps and the corresponding power law
slopes $\alpha_F$ we compare the values of the
power law slopes for Method 1 and Method 2 in Fig.~5,
from which we find a high cross-correlation coefficient
of CCC=0.99, ignoring one outlier event (\#3). 
A linear regression fit between the two methods yields (Fig.~5),
\begin{equation}
	\alpha_{\rm F2} = 0.14 + 1.32 \times \alpha_{\rm F1} \ .
\end{equation}
so the two distributions of power law slopes
are almost proportional, except for a slight
bias in the slope. This deviation stems from the
different approximations of avalanche geometries,
one using all pixels (Method 1), and one using the
peak pixels only (Method 2). 
We can consider Method 1 as an upper limit,
and Method 2 as a lower limit of the ``true''
size distribution, i.e., 
$(\alpha_{\rm F2} < \alpha_{\rm F} < \alpha_{\rm F1})$. 

\subsection{	HMI Magnetogram Analysis	}

In order to test the universality of the results we repeat
the same analysis for 12 coincident HMI magnetograms onboard 
the {Solar Dynamics Observatory (SDO)}, which
have simultaneous times and identical spatial field-of-views.
The 12 analyzed HMI images are shown in Fig.~6, where
black features indicate negative magnetic polarity, and 
white features indicate positive magnetic polarity. We see sunspots
in at least 4 magnetograms (\#1, 2, 3, 10), with 
sunspots having a negative magnetic polarity (\#1, 2),
and two cases with positive magnetic polarity (\#3, 10).
All 12 magnetograms show mixed polarities, but some are
heavily unbalanced (\#1-5, 10-12). 

We quantify the magnetic flux balance with the ratio $q_{pos}$,
\begin{equation}
	q_{pos} = \left( {\sum_{pos} F_{ij} \over
		\sum_{pos} F_{ij} + |\sum_{neg} F_{ij} |} \right) \ .
\end{equation}
If the magnetic flux (line-of-sight) component is
well-balanced, we would expect a value of $q_{pos}=0.5$,
assuming $\sum_{pos}=\sum_{neg}$. 
Only 4 cases have approximately balanced fluxes (\#6, 7, 8, 9), 
namely $q_{pos}=[0.44, 0.43, 0.38, 0.44]$, while the other
6 cases have large flux imbalances, from 
$q_{pos}=0.04$ to 0.99 (Table 2 and Fig.~7).
The associated power law slopes of the 4 well-balanced
cases are $\alpha_F=[1.57,1.71,1.89,1.81]=1.75 \pm 0.12$
which coincides with those cases that have a power law slope
close to the theoretical SOC-prediction of
$\alpha_F \approx 1.80$ (Aschwanden 2012; 2016; 2022a). 

We analyze the HMI data in the same way as the IRIS data,
by fitting joint Gaussian distributions (blue curves in Fig.~7)
and power law distribution functions (red curves in Fig.~7),
which clearly show a ``fat-tail'' feature that is far in
excess of the Gaussian function (blue dashed curves in
Fig.~7). We compare the power law slopes $\alpha_F$ obtained
with the two completely different datasets from IRIS and
HMI in Fig.~8, using the all-pixels Method 1. The two
datasets are found to be highly correlated (with
CCC=0.97, if we ignore the outlier \#3). The scaling of
the two correlated datasets can be described by a
linear regression fit,
\begin{equation}
	\alpha_{\rm F,HMI} = 0.71 + 0.64 \ \alpha_{\rm F,IRIS} \ ,
\end{equation} 
which indicates an (unknown) more complex scaling law.
Nevertheless, the power law slopes $\alpha_F$ are
concentrated in two regimes, one that is consistent
with our theoretical SOC prediction of
$\alpha_{\rm F,IRIS} = \alpha_{\rm F,HMI} \approx 1.80$,
while a second cluster is centered around  
$\alpha_{\rm F,IRIS} \approx 0.6-1.0$ and
$\alpha_{\rm F,HMI} \approx 1.0-1.4$ (Fig.~8).
In essence, we find 5 datasets (\# 6, 7, 8, 9, 11)
that are consistent with the SOC prediction for
events with well-balanced flux $q_{pos} \approx 0.5$, 
while a second group cannot 
reproduce the SOC model, but can be characterized with
large unblanced magnetic fluxes 
(\# 1, 2, 3, 4, 5, 10, 12).
Apparently, the magnetic flux balance is decisive
whether the calculated power law slopes are
universally consistent with SOC models or not.

The physical interpretation of the IRIS data is,
of course, different for the HMI data. In the
previous analysis of IRIS data we interpreted the linear 
random distribution due to the granulation dynamics, 
and the nonlinear avalanche statistics in terms of 
SOC-controlled power law functions due to spicules 
in the transition region. In contrast, using the HMI
data, which provides the magnetic field 
line-of-sight component $B_z$,
we can interpret the statistics of linear random
distributions in terms of ``salt-and-pepper'' small-scale
magnetic fields in the photosphere, and the nonlinear 
avalanche statistics in terms of SOC-controlled magnetic 
reconnection processes in nanoflares and larger flares
(Table 3).
Note that the two parameters $\alpha_{\rm F,IRIS}$ amd
$\alpha_{\rm F,HMI}$ are observed independently from
different spacecraft, as well as in markedly
different wavelength bands, i.e., 
$\lambda \approx 1400$ \ang\ for IRIS, and 
$\lambda = 6173$ \ang\ for HMI/SDO magnetograms,
which measures the mean flux $F$ from the
line-of-sight magnetic field component $B_z(x,y)$.
Despite of the very different instruments and
wavelengths, the power law slope $\alpha_F$ 
of the mean flux appears to be universally
valid and consistent with the theoretical 
SOC prediction for datasets with approximate
magnetic flux balance (Fig.~8). However we learned that
the magnetic flux balance and the absence of sunspots
represent additional requirements to warrant the
universality of the SOC slopes. This yields a testable
prediction: If the field-of-view of each HMI magnetogram
is readjusted so that the enclosed magnetic flux becomes
more balanced and no sunspot appears in the FOV, the 
power law slope is expected to approach the theoretical 
universal value of $\alpha_{\rm F,IRIS} \approx 
\alpha_{\rm F,HMI} \approx 1.80$.

\section{	DISCUSSION		} 

In the following we discuss two linear random processes
(convective granulation, salt-and-pepper small-scale
scale magnetic features), and two nonlinear random 
processes (spicular dynamics, magnetic reconnection, 
flaring, and nanoflaring), which relate to each other 
as depicted in Table 3.  

\subsection{	Magnetic Flux Distribution		}

The most extensive statistical study on the size distribution
of magnetic field features on the solar surface has been undertaken
by Parnell et al.~(2009). Combining magnetic field data from
three instruments (SOT/Hinode, MDI/NFI, and MDI/FD on SOHO,
a combined occurrence frequency size distribution was
synthesized that extends over five decades, in the range of 
$\Phi =2 \times 10^{17}-10^{23}$ Mx (Parnell et al.~2009),
\begin{equation}
	N(\Phi ) \propto \left( \Phi_0 \right)^{-1.85\pm 0.14} \quad 
	[{\rm Mx}^{-1} {\rm cm}^{-2}] \ ,
\end{equation}
where the magnetic flux $\Phi$ is obtained from integration
of the magnetic field $B(x,y)$ over a thresholded area 
$A = \int\ dx\ dy$,
\begin{equation}
	\Phi = \int B(x,y)\ dx\ dy \quad [{\rm Mx}]\ .
\end{equation}
If we equate the magnetic (feature) flux $\Phi$ with the
mean flux $F$ of an event in standard SOC models, we predict
a power law slope of (Aschwanden 2012; 2016; 2022a),
\begin{equation}
        \alpha_{\rm F,SOC} = 1 + {(d-1) \over (\gamma D_V)} = {9 \over 5}
        = 1.80 \ ,
\end{equation}
which agrees well with the result 
(Eq.~11) observed by Parnell et al.~(2009).
A lower value was found from cellular automaton simulations,
$N(\Phi)\approx \Phi^{-1.5\pm0.05}$ (Fragos et al.~2004),
where flux emergence is driven by a percolation rule,
similar to the percolation model of Seiden and Wentzel (1996),
or Balke et al.~(1993).
Mathematical models have been developed to model the percolation
phenomenon, based on combinatorial and statistical concepts
of connectedness that exhibit universality in form of
powerlaw distributions.

\subsection{	Universality of SOC Size Distributions  	}

Power-law-like size distributions are the hallmark
of self-organized criticality systems. Statistical
studies in the past have collected SOC parameters
such as length scales $L$, time scales $T$, peak flux
rates $P$, mean fluxes $F$, fluences and energies
$E=F\times T$, mono-fractal and multi-fractal dimensions
(Mandelbrot 1977), in order to test whether the theoretically
expected power law size distributions, or the power law
slopes of waiting times, agree with the observed
distributions (mostly observed in astrophysical systems). 
The universality of SOC models 
(Aschwanden 2012; 2016; 2022a) is based on four scaling laws:
the scale-free probability conjecture
$N(L) \propto L^{-d}$, classical diffusion
$L\propto T^{\beta/2}$, the flux-volume relationship
$F \propto V^{\gamma}$, and the Euclidean 
scaling law, $P \propto L^{\gamma d}$, where
$d=3$ is the Euclidean dimension, $\beta \approx 1$ is the
classical diffusion coefficient, $\gamma \approx 1$ the
flux-volume proportionality, while $D_A=3/2$ and $D_V=5/2$
are the mean fractal dimensions in 2-D and 3-D Euclidean
space. The standard SOC model is expressed in terms of
these universal constants: $d=3$, $\gamma=1$, $\beta=1$, 
and the mean fractal dimension $D_v=5/2$. Consequently,
the four basic scaling laws reduce to 
$N(L) \propto L^{-3}$, $L \propto T^{1/2}$,  
and $F \propto L^{2.5}$. Since we measure the
mean flux $F$ in this study, our main test of the
universality of SOC models if formulated in terms of
the flux-volume relationship $F \propto V^{\gamma}$, 
leading to the 
power law slope $\alpha_{\rm F,SOC}=1.80$ (Eq.~13).

The SOC-inferred scaling laws hold for a large number 
of phenomena. This implies that our SOC formalism is 
universal in the sense that the statistical size distributions 
are identical for each phenomenon, displaying a universal
power law slope of $\alpha_{\rm F,SOC}=1.80$.   
When we conclude that the power law slope $\alpha_F$ is universal,
the SOC model implies that the flux-volume proportionality
($\gamma \approx 1$) as well as the mean fractal dimension ($D_V \approx 2.5$,
$d=3$) are universal too. 

\subsection{	Physical Processes in SOC Systems	}

Once we establish the self-consistency of power law slopes 
between theoretical (SOC) and observed size distributions, the
next question is what physical processes are at work. 
We envision four different physical processes (Table 3): 
(i) photosopheric granulation dynamics (to explain the Gaussian
random noise distribution in IRIS data);
(ii) spicular plage events in the transition region 
(to explain the power law size distribution in IRIS data);
(iii) salt-and-pepper small-scale magnetic structures
(to explain the random noise distributions in HMI magnetograms);
and (iv) magnetic reconnection proceses in flares and nanoflares
(to explain the power law size distribution in HMI data).
However, there are deviations from these rules. We found that
the power law 
distributions are modified in the presence of sunspots, or when 
the magnetic flux is unbalanced. Under ideal conditions, 
the SOC scaling laws are fulfilled universally, independent
of the wavelength or plasma temperature.
Magnetic field data (from HMI/SDO) or $\lambda \approx 1400$ \ang\ 
(from IRIS) produce emission in volumes that are proportional in 
the photosphere or transition zone, even when they are formed at 
quite different temperatures, i.e., $T_{\rm phot} \approx 5800$ K 
in the photosphere and $T_{\rm TR} \approx 10^4-10^6$ K in the 
transition region.  

Another ingredient of the SOC model is the scale-free probability
conjecture, i.e., $N(L) \propto L^{-d} = L^{-3}$, which
cannot be uniquely linked to a particular  physical process. Parnell
et al.~(2009) conclude that a combination of emergence, 
coalescence, cancellation, and fragmentation may possibly produce
power law size distributions of spatial scales $L$. 
Alternative models include the
turbulence, the magneto-chemistry equations (Schrijver et al.~1997), 
and Weibull distributions (Parnell 2002). 
Munoz-Jaramillo et al.~(2015) study the best-fitting distribution
functions for 11 different databases of sunspot areas, sunspot
group areas, sunspot umbral areas, and magnetic fluxes. 
They find that a linear combination of Weibull and log-normal
distributions fit the data best (Munoz-Jaramillo et al.~2015). 
Weibull and log-normal distributions combine two distribution
functions, similar to our synthesis of a Gaussian-plus-power-law 
distribution.

A general physical scenario of a
power law size distribution is the evolution of
avalanches by exponential growth (Rosner and Vaiana 1978), 
with subsequent saturation (logistic growth) after a random 
time interval, which produces an exact power law function  
(Aschwanden et al.~1998). Our approach to model the size
distribution of solar phenomena with two different functions,
employing a Gaussian noise and a power law tail, reflects the
duality of a linear and a nonlinear random component, in both
the data from IRIS and HMI (Table 3). In summary, 
linear random components 
include granulation, super-granulation, and salt-and-pepper 
small-scale magnetic features, while nonlinear components
include spicular avalanches and magnetic reconnection
avalanches from nanoflares to large flares. 

\subsection{	Granular Dynamics	}

The physical understanding of solar (or stellar) granulation
has been advanced by numerical magneto-convection models and
N-body dynamic simulations, which predict the evolution of
small-scale (granules) into large-scale
features (meso- or super-granulation), organized by surface
flows that sweep up small-scale structures and form clusters of
recurrent and stable granular features (Hathaway et al.~2000;
Berrilli et al.~1998, 2005; Rieutord et al.~2008, 2010;
Cheung and Isobe 2014; Martinez-Sykora et al.~2008).
The fractal multi-scale dynamics has been found to be operational
in the Quiet-Sun photosphere, in quiescent non-flaring states, as
well as during flares (Uritsky et al.~2007, 2013; 
Uritsky and Davila 2012).
A common origin of granulation and super-granulation network cells
is assumed (Berrilli et al.~1998).
The fractal structure of the solar granulation is obviously a
self-organizing pattern that is created by a combination of
subphotospheric magneto-convection and surface flows, which are
turbulence-type phenomena. The spatial structure of photospheric 
granulation has also been explained in terms of a self-organization
(SO) process (without criticality !), where order is created out of 
randomness (Aschwanden et al.~2018).

\subsection{	Spicular Dynamics			}

One prominent feature in the transition region is the phenomenon
of {\sl ``moss''}, which appears as a bright dynamic pattern with
dark inclusions, on spatial scales of $L \approx 1-3$ Mm, which
has been interpreted as the upper transition region above active
region plages, and below relatively hot loops 
(De Pontieu et al.~1999; 2014).
Our measurement of structures in the IRIS 1400 \ang\ channel
is sensitive to a temperature range of $T_e \approx 10^4-10^6$ K,
and thus is likely to include chromospheric and transition region
phenomena such as: spicules II (De Pontieu et al.~2007), 
macro-spicules, dark mottles, dynamic fibrils, surges, miniature 
filament eruptions, etc. Theoretical models include 
the rebound shock model (Sterling and Hollweg 1988), 
pressure-pulses in the high atmosphere (Singh et al.~2019), 
Alfv'enic resonances (Sterling 1998),
magnetic reconnection models for type II spicules (De Pontieu et al.~2007),
ion-neutral collisional damping (De Pontieu 1999), 
leakage of global p-mode oscillations (De Pontieu et al.~2004), 
MHD kink waves (Zaqarashvili and Erdelyi 2009), 
vortical flow models (Kitiashvili et al.~2013), and
magneto-convective driving by shock waves (De Pontieu et al.~2007). 

The fact that we obtain a power law size distribution 
($\alpha_F=1.67\pm0.14$, Table 1), which is very similar 
to solar flares in general, $\alpha_{\rm F,SOC}=1.80$,
implies the universality of the SOC framework. Furthermore
we find power law-like size distributions for spicular events,
rather than a Gaussian distribution, which tells us that 
spicule events need to be modeled in terms of SOC-driven 
avalanches, instead of with Gaussian random distributions.
The difference between linear and nonlinear random processes 
is the property of coherent growth, which results into
exponential (or logistic) growth, after triggered by an
instability. This does not happen for incoherent growth.

\subsection{	Salt-and-Pepper Magnetic Field	        }

We interepret the random noise Gaussian distribution of
magnetic fluxes in Quiet-Sun regions as small-scale
magnetic field ``pepper-and-salt'' structures, also
called {\sl magnetic carpet} (Priest et al.~2002), 
where the black and white
color in magnetograms corresponds to negative and
magnetic polarity. According
to our analysis, the magnetic flux has an average of
$F_{avg} \approx 10-200$ DN/s, and a similar standard 
deviation of $F_{sigm} \approx 10-200$ DN/s (Table 1). 
The fact that we obtain two distinctly different 
size distributions (Gaussian vs. power law) indicates
at least two different physical mechanisms, one being
a linear random (Gaussian) process, the other one
being a nonlinear (power law) avalanche process.
The salt-and-pepper structure is generated apparently
by a linear random process, rather than by an
nonlinear avalanching process, according to our fits. 
This may constrain
the origin of the solar magnetic field, being created 
by emergence, coalescence, cancellation, and/or
fragmentation. 

\subsection{	Magnetic Reconnection 		}	

The re-arrangement of the stress-induced solar 
magnetic field requires ubiquitous and permanent 
(but intermittent) magnetic reconnection processes 
on all spatial and temporal scales.
Our study finds power law size distributions,
with a slope of $\alpha_F=1.75\pm 0.12$ from HMI 
magnetograms, which is similar to flares in general
(see Aschwanden et al.~2016 for a review of all 
wavelengths, gamma-rays, hard X-rays, soft X-rays, 
EUV, etc). This tells us that there is a strong correlation 
between the photospheric field (in HMI images) 
and the transition region (in IRIS images), as evident 
from the cross-correlation coefficient of CCC=0.97 
shown in Fig.~8. The fractal multi-scale 
dynamics apparently operates in the quiet photosphere, 
in the quiescent non-flaring state, as well as during 
flares in active regions (Uritsky and Davila 2012).

\section{	CONCLUSIONS			}

Solar and stellar flares, pulsar glitches, auroras,
lunar craters, as well as earthquakes, landslides,
wildfires, snow avalanches, and sandpile avalanches
are all driven by self-organized criticality (SOC),
which predicts power law-like occurrence 
frequency (size) distributions and waiting time 
distribution functions. What is new in our studies
of SOC systems is that we are now able to calculate 
the slope $\alpha_x$ of power law functions, which 
allows us to test SOC models by comparing the
observed (and fitted) distribution functions with
the theoretically predicted values. In this study
we compare statistical distributions of SOC parameters
from different wavelengths and different instruments
(UV emission observed with IRIS and magnetograms
with HMI). The results of our study are:

\begin{enumerate}

\item{The histogrammed distribution of fluxes $N(F)$
obtained from an IRIS 1400 \ang\ image, or from a HMI 
magnetogram, cannot be fitted solely by a Gaussian 
function, but requires a two-component function, 
such as a combination of a Gaussian and a power law
function, i.e., a ``fat-tail'' extension above about
one standard deviation. We define a separator between
the two functions at one standard deviation and
obtain power law slopes of $\alpha_F=1.67\pm 0.14$
from the IRIS data, and $\alpha_F=1.75\pm 0.12$
from the HMI data, which agree with the theoretical
SOC prediction of $\alpha_F=1.80$, and thus 
demonstrate universality across UV wavelengths and
magnetograms. Moreover, it agrees with the five
order of magnitude extending power law distribution
sampled by Parnell et al.~(2009), $\alpha_F 
= 1.85\pm0.14$.}

\item{The phenomenology of the 12 analyzed dataset
consists of 4 active regions with sunspots, while
the other 8 cases were selected in plage regions
and in the Quiet-Sun. It turns out that regions with 
large average fluxes ($F_{avg} > 50$) DN/s, have
a large unbalanced flux, and contain sunspots.
Therefore, matching the theoretical SOC prediction
requires magnetic flux balance and the absence of
sunspots, in order to warrant universality of
the flux power law distribution function.}

\item{We designed two algorithms that produce
size distributions from a single image (e.g.,
from a UV image or a magnetogram), by sampling a
histogram using all pixels (Method 1), or a
histogram using only pixels with a local
peak (Method 2). The two methods agree
closely with each other (with a cross-correlation
coefficient of CCC=0.99. One can consider 
Method 1 as an upper limit, and Method 2 as 
a lower limit to the ``true'' size distributions, 
i.e., $(\alpha_{\rm F2} < \alpha_{\rm F} < 
\alpha_{\rm F1})$. Both methods are computationally
very fast and do not require any particular
automated pattern recognition code.}

\item{We can interpret the analyzed size 
distributions in terms of four distinctly different 
physical processes: (i) photospheric granulation 
dynamics (explaining the Gaussian
random noise distribution in IRIS data);
(ii) spicular plage events in the transition region 
(explaining the power law size distribution in IRIS data);
(iii) salt-and-pepper small-scale magnetic structures
(explaining the random noise distributions in HMI magnetograms);
and (iv) magnetic reconnection processes in flares and nanoflares
(explaining the power law size distribution in HMI data).}

\end{enumerate}

Future work may include: (i) testing of the SOC-predicted
size distributions with power law slopes $\alpha_F$ for
all available (mean) fluxes $F$ (in HXR, SXR, EUV, etc.); 
(ii) testing the selection of different FOV sizes in 
the absence or existence of sunspots, and magnetic 
flux balance; (iii) testing and cross-comparing the
approximations of the all-pixel (Method 1) and peak-pixel
(Method 2). Ultimately these methods should help us
to converge the numerical values in SOC models.

\medskip
\acknowledgments
{\sl Acknowledgements:}
We acknowledge constructive comments of reviewers
and stimulating discussions (in alphabetical order) with 
Paul Charbonneau, Adam Kowalski, Karel Schrijver, Vadim Uritsky, 
and Nived Vilangot Nhalil.
This work was partially supported by NASA contract NNX11A099G
``Self-organized criticality in solar physics'', NASA contract
NNG04EA00C of the SDO/AIA instrument, and the IRIS contract 
NNG09FA40C to LMSAL.

\clearpage

%%%%%%%%%%%%%%%%%%%%%%%%%%%%%%%% REFERENCES &&&&&&&&&&&&&&&&&&&&&&&&
\def\ref#1{\par\noindent\hangindent1cm {#1}}

\section*{	References	}

\ref{Aschwanden, M.J., Dennis, B.R., and Benz, A.O. 1998,
 	{\sl Logistic avalanche processes, elementary time structures, 
	and frequency distributions in solar flares},
 	ApJ 497, 972}
\ref{Aschwanden, M.J. 2011,
        {\sl Self-Organized Criticality in Astrophysics. The Statistics
        of Nonlinear Processes in the Universe}, ISBN 978-3-642-15000-5,
        Springer-Praxis: New York, 416p.}
\ref{Aschwanden, M.J. 2012,
        {\sl A statistical fractal-diffusive avalanche model of a
        slowly-driven self-organized criticality system},
        A\&A 539, A2, (15 p)}
\ref{Aschwanden, M.J. 2014,
        {\sl A macroscopic description of self-organized systems and
        astrophysical applications}, ApJ 782, 54}
\ref{Aschwanden,M.J., Crosby,N., Dimitropoulou,M., Georgoulis,M.K.,
        Hergarten,S., McAteer,J., Milovanov,A., Mineshige,S., Morales,L.,
        Nishizuka,N., Pruessner,G., Sanchez,R., Sharma,S., Strugarek,A.,
        and Uritsky, V. 2016,
        {\sl 25 Years of Self-Organized Criticality: Solar and Astrophysics}
        Space Science Reviews 198, 47-166.}
\ref{Aschwanden, M.J., Scholkmann, F., Bethune, W., Schmutz, W., 
	Abramenko,W., Cheung,M.C.M., Mueller,D., Benz,A.O., 
	Chernov,G., Kritsuk,A.G., Scargle,J.D., Melatos,A., 
	Wagoner,R.V., Trimble,V., Green,W. 2018, 
	{\sl Order out of randomness: Self-organization processes 
	in astrophysics}, Space Science Reviews 214:55}
\ref{Aschwanden, M.J. 2022a,
	{\sl The fractality and size distributions of astrophysical
	self-organized criticality systems},
	ApJ 934:33}
\ref{Aschwanden, M.J. 2022b,
	{\sl Reconciling power-law slopes in solar flare and nanoflare
	size distributions}, ApJL 934:L3}
\ref{Aschwanden, M.J. and Vilangot Nhalil, N. 2022,
	{\sl Interface region imaging spectrograph (IRIS) observations 
	of the fractal dimension in the solar atmosphere},
	Frontiers in Astronomy and Space Sciences, Manuscript ID 999329}
\ref{Balke,A.C., Schrijver, C.J., Zwaan,C., and Tarbell,T.D. 1993,
        {\sl Percolation theory and the geometry of photospheric
        magnetic flux concentrations}, Solar Phys. 143, 215.}
\ref{Bak, P., Tang, C., and Wiesenfeld, K. 1987,
        {\sl Self-organized criticality: An explanation of 1/f noise},
        Physical Review Lett. 59(27), 381}
\ref{Bak, P., Tang, C., and Wiesenfeld, K. 1988,
        {\sl Self-organized criticality},
        Physical Rev. A 38(1), 364}
\ref{Bak, P. 1996,
        {\sl How Nature Works. The Science of Self-Organized Criticality},
        New York: Copernicus}
\ref{Berrilli, F., Florio, A., and Ermolli, I. 1998,
	{\sl On the geometrical properties of the chromospheric network},
	Sol.Phys. 180, 29-45}
\ref{Berrilli, F., Del Moro, D., Russo, S., et al. 2005,
        {\sl Spatial clustering of photospheric structures},
        ApJ 632, 677}
\ref{Burlaga, L.F. and Lazarus, A.J. 2000, 
 	{\sl Lognormal distributions and spectra of solar wind plasma 
	fluctuations: Wind 1995-1998},
 	JGR 105, 2357}
\ref{Ceva, H. and Luzuriaga, J. 1998,
 	{\sl Correlations in the sand pile model: From the log-normal 
	distribution to self-organized criticality},
 	Physics Letters A 250, 275}
\ref{Cheung, M.C.M. and Isobe, H. 2014,
	{\sl Flux emergence (Theory)},
	LRSP 11, 3.}
\ref{De Pontieu, B. 1999,
 	{\sl Numerical simulations of spicules driven by weakly-damped 
	Alfven waves I. WKB approach},
 	A\&A 347, 696}
\ref{De Pontieu, B., Berger, T.E., Schrijver, C.J., and Title, A.M. 1999,
	{\sl Dynamics of transition region 'moss' at high time resolution}.
	Sol.Phys. 190, 419}
\ref{De Pontieu, B., Erdelyi, R., and James, S.P. 2004,
 	{\sl Solar chromospheric spicules from the leakage of photospheric 
	oscillations and flows},
 	Nature 430, 536}
\ref{De Pontieu, B., Title, A.M., Lemen, J.R., Kushner, G.D., 
	Akin, D.J., Allard, B., Berger, T., Boerner, P., 2014,
	{The Interface Region Imaging Spectrograph (IRIS)},
	Sol.Phys. 289, 2733}
\ref{De Pontieu,B., McIntosh,S., Hansteen,V.H., Carlsson,M., 
	Schrijver,C.J., et al. 2007,
 	{\sl A tale of two spicules: The impact of spicules 
	on the magnetic chromosphere},
 	PASJ 59, S655}
\ref{Fontenla, J.M., Curdt, W., Avrett, E.H., and Harder, J. 2007, 
 	{\sl Log-normal intensity distribution of the quiet-Sun FUV 
	continuum observed by SUMER},
 	A\&A 468, 695}
\ref{Fragos, T., Rantsiou, E., and Vlahos, L. 2004,
 	{\sl On the distribution of magnetic energy storage in solar 
	active regions},
 	A\&A 420, 719.}
\ref{Gallagher, P.T., Phillips, K.J.H., Harra-Murnion, L.K., et al.
        1998, {\sl Properties of the Quiet Sun EUV network},
        A\&A 335, 733}
\ref{Giles, D.E., Feng,H., and Godwin,R.T. 2011,
 	{\sl On the bias of the maximum likelihood estimator for the 
	two-parameter Lomax distribution},
 	Econometrics Workshop Paper EWP1104, ISSN 1485}
\ref{Hathaway, D.H., Beck, J.G., Bogart, R.S., et al. 2000,
        {\sl The photospheric convection spectrum}
        SoPh 193, 299}     
\ref{Hosking, J.R.M. and Wallis, J.R. 1987,
 	{\sl Parameter and quantile estimation for the generalized 
	Pareto distribution},
 	Technometrics, 29(3), 339}
\ref{Hudson, H.S. 1991,
	{\sl Solar flares, microflares, nanoflares, and coronal heating},
	Sol.Phys. 133, 357}
\ref{Katsukawa, Y. and Tsuneta, S. 2001,
 	{\sl Small fluctuation of coronal X-ray intensity and a
	signature of nanoflares},
 	ApJ 557, 343}
\ref{Kitiashvili, I.N., Kosovichev, A.G., Lele, S.K., Mansour, N.N.
	and Wray, A.A. 2013,
	{\sl Ubiquitous solar eruptions driven by magnetized
	vortex tubes},
	ApJ 770, 37}
\ref{Krucker, S. and Benz, A.O. 1998,
	{\sl Energy distribution of heating processes in the quiet
	solar corona},
	ApJ 501, L213}
\ref{Kunjaya, C., Mahasena, P., Vierdayanti, K., and Herlie, S. 2011,
	{\sl Can self-organized critical accretion disks generate a log-normal 
	emission variability in AGN?},
 	ApSS 336, 455}
\ref{Lomax, K.S. 1954,
	J. Am. Stat. Assoc. 49, 847}
\ref{Lu, E.T. and Hamilton, R.J. 1991,
 	{\sl Avalanches and the distribution of solar flares}, 
	ApJ 380, L89}
\ref{Mandelbrot, B.B. 1977, {\sl The Fractal Geometry of Nature}.
        W.H.Freeman and Company: New York}
\ref{Martinez-Sykora, J., Hansteen,V., and Carlsson, M. 2008,
 	{\sl Twisted Flux Tube Emergence From the Convection Zone to 
	the Corona}, ApJ 679, 871}
\ref{McAteer,R.T.J., Aschwanden,M.J., Dimitropoulou,M., Georgoulis,M.K.,
        Pruessner, G., Morales, L., Ireland, J., and Abramenko,V. 2016,
        {\sl 25 Years of Self-Organized Criticality: Numerical Detection Methods},
        SSRv 198 217-266.}
\ref{Mitzenmacher, M. 2004,
 	{\sl A brief history for generative models for power law and 
	lognormal distributions},
 	Internet mathematics 1(2), 226}
\ref{Munoz-Jaramillo, A., Senkpeil, R.R., Windmueller, J.C., Amouzou, E.C., et al. 2015,
 	{\sl Small-scale and Global Dynamos and the Area and Flux Distributions 
	of Active Regions, Sunspot Groups, and Sunspots: A Multi-database Study}, 
 	ApJ 800, 48}
\ref{Parnell, C.E. 2002,
	{\sl Nature of the magnetic carpet - 1. Distribution of
	magnetic fluxes},
	MNRAS 335/2, 398.}
\ref{Parnell, C.E., DeForest, C.E., Hagenaar, H.J., Johnston, B.A., 
	Lamb, D.A., and Welsch, B.T. 2009, 
 	{\sl A Power-Law Distribution of Solar Magnetic Fields Over More 
	Than Five Decades in Flux},
	ApJ 698, 75-82}
\ref{Priest,E.R., Heyvaerts, J.F., and Title, A.M. 2002,
 	{\sl A Flux-Tube Tectonics Model for Solar Coronal Heating 
	Driven by the Magnetic Carpet},
 	ApJ 576, 533}
\ref{Rathore, B. and Carlsson, M. 2015, 
	{\sl The Formation of IRIS Diagnostics. VI. The Diagnostic Potential of the 
	C II Lines at 133.5 nm in the Solar Atmosphere},
	ApJ 811, 80}
\ref{Rathore, B., Carlsson, M., Leenaarts, J., De Pontieu B. 2015, 
	{\sl  The Formation of Iris Diagnostics. VIII. Iris Observations in the 
	C II 133.5 nm Multiplet},
	ApJ 811, 81}
\ref{Rieutord, M., Meunier, N., Roudier, T., et al. 2008,
        {\sl Solar super-granulation revealed by granule tracking},
        A\&A 479, L17}
\ref{Rieutord, M., Roudier, T., Rincon, F., 2010,
        {\sl On the power spectrum of solar surface flows},
        A\&A 512, A4}
\ref{Rosner, R., and Vaiana, G.S. 1978,
 	{\sl Cosmic flare transients: constraints upon models for energy storage 
	and release derived from the event frequency distribution},
 	ApJ 222, 1104}
\ref{Scargle, J.D. 2020,
 	{\sl Studies in astronomical time-series analysis. VII. An enquiry 
	concerning nonlinearity, the rms-mean flux relation, and lognormal flux 
	distributions},
 	ApJ 895, 90}
\ref{Schrijver, C.J., Hagenaar, H.J., and Title, A.M.  1997
 	{\sl On the patterns of the solar granulation and super-granulation},
 	ApJ 475, 328-337}
\ref{Seiden, P.E. and Wentzel, D.G. 1996,
 	{\sl Solar active regions as a percolation phenomenon II.}
 	ApJ 460, 522}
\ref{Singh, B., Sharma, K., and Srivastava, A.K. 2019,
	{\sl On modelling the kinematics and evolutionary properties
	of pressure=puls-driven impulsive solar jets},
	Ann.Geophys. 37, 891}
\ref{Sterling,A.C. and Hollweg,J.V. 1988,
 	{\sl The rebound shock model for solar spicules: Dynamics at long 
	times}
 	ApJ 327, 950}
\ref{Sterling, A.C. 1998,
	{\sl Alfv\'enic resonances on ultraviolet spicules},
	ApJ 508, 916}
\ref{Uritsky, V.M., Paczuski, M., Davila, J.M., and Jones, S.I. 2007,
        {\sl Coexistence of self-organized criticality and intermittent
        turbulence in the solar corona},
        Phys.~Rev.~Lett. 99(2), id. 025001}
\ref{Uritsky, V.M., and Davila, J.M. 2012,
 	{\sl Multiscale Dynamics of Solar Magnetic Structures},
 	ApJ 748, 60}
\ref{Uritsky, V.M., Davila, J.M., Ofman, L., and Coyner, A.J. 2013,
        {\sl Stochastic coupling of solar photosphere and corona},
        ApJ 769, 62}
\ref{Verbeeck, C., Kraaikamp, E., Ryan, D.F., and Podladchikova, O. 2019,
 	{\sl Solar Flare Distributions: Lognormal Instead of Power Law?},
 	ApJ 884, 50}
\ref{Vilangot Nhalil, N.V., Nelson, C.J., Mathioudakis, M., and Doyle, G.J. 2020,
 	{\sl Power-law energy distributions of small-scale impulsive events 
	on the active Sun: results from IRIS},
 	MNRAS 499, 1385}
\ref{Warren, H.P., Reep, J.W., Crump, N.A., and Simoes, P.J.A. 2016,
	{\sl Transition region and chromospheric signatures of	
	impulsive heating events. I. Observations},
	ApJ 829:35}
\ref{Weibull, W. 1951,
	{\sl A statistical distribution function of wide applicability},
	J. Appl. Mech 18(3) 293}
\ref{Zaqarashvili, T.V. and Erdelyi, R. 2009,
	{\sl Oscillations and waves in solar spicules},
	SSRv 149, 355}
\clearpage

%%%%%%%%%%%%%%%%%%%%%%%%%%%%% TABLE1 %%%%%%%%%%%%%%%%%%%%%%%%%%%%%%%%

\begin{table}
\begin{center}
\small	%\normalsize
\caption{Results of 12 datasets obtained with IRIS 1400 \ang , 
including the power law slope $\alpha_{F1}$ of the flux distribution,
the maximum flux $F_{max}$, and fractal dimension $D_A$. Note
that the power law slope $\alpha_F$ agrees with the theoretical
prediction of $\alpha_F=9/5= 1.8$, whenver there is no
sunspot and the maximum flux $F_{max}$ amounts to less than
a critical value of $F_{max} \lapprox 50$ DN/s. The values of
$\alpha_{F1}$ in parenthesis indicate outliers.} 
\medskip
\begin{tabular}{cccccccc}
\hline
\hline
Number	& Phenomenon 	& Power law 	& Agrees with	  & Average              & Maximum   & Max.flux   & Fractal   \\
Dataset	& 1400 \ang     & slope fit     & prediction      & flux                 & flux      & criterion  & dimension \\
IRIS	&		& $\alpha_{\rm F1}$ & $\alpha_{\rm F1} \approx 1.8$ & $F_{avg}\pm F_{sig}$ & $F_{max}$ & $<50$ DN/s & $D_A$     \\
\#	&		&  	        &                 & [DN/s]               & [DN/s]    &            &           \\
\hline
1	& Sunspot	& (0.87$\pm$0.04) & NO            &  47$\pm$ 73 & 121 & NO  & 1.44 \\
2	& Sunspot	& (0.85$\pm$0.01) & NO	          &  77$\pm$112 & 190 & NO  & 1.56 \\
3	& Sunspot	& (2.15$\pm$0.07) & NO	          & 129$\pm$113 & 243 & NO  & 1.53 \\
4	& Plage		& (0.93$\pm$0.01) & NO	          &  57$\pm$ 49 & 108 & NO  & 1.67 \\
5	& Plage		& (1.02$\pm$0.01) & NO	          & 103$\pm$ 95 & 199 & NO  & 1.67 \\
6	& Plage		&  1.57$\pm$0.02  & YES	          & 29 $\pm$ 22 &  50 & YES & 1.66 \\
7	& Plage		&  1.57$\pm$0.02  & YES           & 14 $\pm$ 12 &  26 & YES & 1.64 \\
8	& Plage		&  1.87$\pm$0.03  & YES           & 12 $\pm$ 15 &  28 & YES & 1.56 \\
9	& Plage		&  1.78$\pm$0.02  & YES           & 22 $\pm$ 19 &  42 & YES & 1.64 \\
10	& Sunspot	& (0.55$\pm$0.01) & NO 	          & 233$\pm$268 & 501 & NO  & 1.60 \\
11	& Plage		&  1.58$\pm$0.02  & YES	          & 16 $\pm$ 15 &  31 & YES & 1.65 \\
12	& Plage		& (0.68$\pm$0.08) & NO            & 30 $\pm$ 23 &  54 & NO  & 1.60 \\
\hline
	& Observations	& 1.67$\pm$0.14   &               &		&		&	& 1.60$\pm$0.07 \\		
	& Theory  	& 1.80            &               &		&		&	& 1.50  \\		
\hline
\end{tabular}
\end{center}
\end{table}

%%%%%%%%%%%%%%%%%%%%%%%%%%%%% TABLE2 %%%%%%%%%%%%%%%%%%%%%%%%%%%%%%%%

\begin{table}
\begin{center}
\small	%\normalsize
\caption{Results of 12 datasets obtained with HMI/SDO, 
including the power law slope $\alpha_{F2}$ of the flux distribution,
the maximum flux $F_{max}$, and fractal dimension $D_A$. Note
that the power law slope $\alpha_{F2}$ agrees with the theoretical
prediction of $\alpha_F=9/5\approx 1.8$, when there is no
sunspot and the magnetic flux is balanced. The values of
$\alpha_{F2}$ in parenthesis indicate outliers and are ignored
in the averages.} 
\medskip
\begin{tabular}{ccccccccc}
\hline
\hline
Number	& Phenomenon 	& Power law 	& Matching	  & Average              & Magnetic  & Matching & Matching & Fractal \\
Dataset	&  		& slope fit     & prediction      & flux                 & field     & balance  & balance  & dimension \\
HMI	&		& $\alpha_{\rm F2}$ & $\alpha_{\rm F2}\approx 1.8$ & $F_{avg}\pm F_{sig}$ &   & $q_{pos} $ & $q_{pos}\approx 0.50$ & $D_A$\\

\#	&		&  	        &                 & [DN/s]      &        &        &           \\
\hline
1	& Sunspot	& (1.34$\pm$0.05) & NO            & 156$\pm$295 & +1073 & (0.04) & NO  & 1.54 \\
2	& Sunspot	& (1.28$\pm$0.01) & NO	          & 102$\pm$232 & -1729 & (0.16) & NO  & 1.55 \\
3	& Sunspot	& (1.05$\pm$0.03) & NO	          & 241$\pm$306 & -2076 & (0.99) & NO  & 1.59 \\
4	& Plage		& (1.32$\pm$0.02) & NO	          & 63 $\pm$123 & +1785 & (0.29) & NO  & 1.58 \\
5	& Plage		& (1.28$\pm$0.02) & NO	          & 67 $\pm$143 & -1186 & (0.81) & NO  & 1.57 \\
6	& Plage		&  1.57$\pm$0.03  & YES	          & 29 $\pm$ 69 & +1854 & 0.44 & YES & 1.51 \\
7	& Plage		&  1.71$\pm$0.03  & YES           & 29 $\pm$ 70 & -1011 & 0.43 & YES & 1.51 \\
8	& Plage		&  1.89$\pm$0.03  & YES           & 28 $\pm$ 73 & -1022 & 0.38 & YES & 1.49 \\
9	& Plage		&  1.81$\pm$0.02  & YES           & 25 $\pm$ 65 &  +955 & 0.44 & YES & 1.50 \\
10	& Sunspot	& (0.96$\pm$0.01) & NO 	          & 228$\pm$326 & -1055 & (0.34) & NO  & 1.66 \\
11	& Plage		&  1.79$\pm$0.03  & YES	          & 29 $\pm$ 68 & +2058 & (0.92) & NO  & 1.51 \\
12	& Plage		& (1.29$\pm$0.04) & NO            & 76 $\pm$137 & +1036 & (0.88) & NO  & 1.52 \\
\hline
	& Observations	& 1.75$\pm$0.12   &               &		& & 0.42 $\pm$ 0.03 &	& 1.79$\pm$0.07 \\		
	& Theory  	& 1.80            &               &		& & 0.50 	    &	& 1.50  \\		
\hline
\end{tabular}
\end{center}
\end{table}

%%%%%%%%%%%%%%%%%%%%%%%%%%%%% TABLE3 %%%%%%%%%%%%%%%%%%%%%%%%%%%%%%%%

\begin{table}
\begin{center}
\normalsize
\caption{Diagram of phenomena observed with different instruments
(IRIS, HMI), different wavelengths (columns), for linear and nonlinear
random processes (rows).}
\medskip
\begin{tabular}{|l|c|c|}
\hline
\hline
 		     	 & IRIS		& HMI	        	  \\
 		     	 & 1400 \ang\ \ & 6173 \ang\ 		  \\
\hline
linear random process 	 & granules	& salt-and-pepper 	  \\
(Gaussian function)  	 &              & small-scale magnetic fields \\
\hline
nonlinear random process & spicules 	& flares, nanoflares      \\
(power law function) 	 &		& magnetic reconnection   \\
\hline
\end{tabular}
\end{center}
\end{table}


%%%%%%%%%%%%%%%%%%%%%%%%%%%%%% FIGURES %%%%%%%%%%%%%%%%%%%%%%%%%
\begin{figure}
\centerline{\includegraphics[width=1.0\textwidth]{f1.eps}}
\caption{A schematic of the two size distributions is shown:
a Gaussian function for the linear random statistics, 
and a power law function (also called fat-tail) for the 
statistics of nonlinear, avalanching events, separated at 
a critical value $F_2$. The upper panel shows a linear
(LIN-LIN) representation, the lower panel a logarithmic
(LOG-LOG) representation.} 
\end{figure}

\begin{figure}
\centerline{\includegraphics[width=1.0\textwidth]{f2.eps}}
\caption{Intensity maps of 12 different active regions
and Quiet-Sun regions, observed with IRIS SJI 1400 \ang . 
Granules are rendered in orange-to-red color, while 
spicules and network cells are masked out with white color.}
\end{figure}

\begin{figure}
\centerline{\includegraphics[width=0.9\textwidth]{f3.eps}}
\caption{Intensity maps of 12 different active regions
and Quiet-Sun regions, observed with IRIS SJI 1400 \ang . 
Granular structures are masked out (with peak fluxes 
$F(x,y) < F_{thr}$), while network cells and spicules 
are rendered in black.}
\end{figure}

\begin{figure}
\centerline{\includegraphics[width=0.9\textwidth]{f4.eps}}
\caption{Mean flux histograms of 12 different regions in plages
of transition regions, observed with IRIS SJI 1400 \ang\ . 
The flux distribution of granules is fitted with a Gaussian
function (blue curve, $F < F_2$), and extrapolated
with dashed blue curves. The flux distribution of spicules is 
fitted with a power law distribution function (thick red 
curve. The separation between the two 
distributions is marked with a vertical thin line.}
\end{figure}

\begin{figure}
\centerline{\includegraphics[width=0.9\textwidth]{f5.eps}}
\caption{Power law index $\alpha_F$ calculated with two
different methods of IRIS 1400 \ang\ data: 
with the all-pixel Method 1 (x-axis),
and the peak-pixel Method 2 (y-axis). 
A linear regression fit is shown (solid line), 
the cross-correlation coefficient CCC=0.99, 
the equivalence (dashed diagonal line),
and the outlier (\# 3) is ignored.}
\end{figure}

\begin{figure}
\centerline{\includegraphics[width=0.9\textwidth]{f6.eps}}
\caption{Magnetograms of 12 different active regions
and plage regions, observed with HMI/SDO . The black 
color indicates negative magnetic polarity, and the 
white color indicates positive magnetic polarity.} 
\end{figure}

\begin{figure}
\centerline{\includegraphics[width=0.9\textwidth]{f7.eps}}
\caption{Histograms of different solar regions, 
observed in magnetograms with HMI/SDO. The size
distribution of salt-and-pepper magnetic noise is fitted
with a Gaussian function (blue curve), while the 
distribution of magnetic features are fitted with a 
power function (red curves).}
\end{figure}

\begin{figure}
\centerline{\includegraphics[width=0.9\textwidth]{f8.eps}}
\caption{Power law slope $\alpha_F$ is calculated for two
datasets observed in different wavelengths: from IRIS data 
(x-axis) and from HMI/SDO data (y-axis). 
A linear regression fit is shown (solid line), 
along with the cross-correlation coefficient CCC=0.97, 
the equivalence (dotted diagonal line), the theoretical
prediction $\alpha_F=1.80$ (vertical and horizontal 
dotted lines), and the outlier (\# 3) is ignored.}
\end{figure}

\end{document}
