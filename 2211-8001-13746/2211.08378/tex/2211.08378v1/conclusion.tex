\vspace{-1ex}
\section{Conclusion, Limitations, and Future Work}\label{sec:conclusion}
We present \model{}, an end-to-end unsupervised framework for detecting edge anomalies in dynamic multiplex networks. \model{} is based on a new architecture that employs GNN and GRU cells to take advantage of both temporal and structural properties and adds an attention mechanism that effectively incorporates information across different types of connections. Finally, it uses a negative sampling approach in training to overcome the lack of ground truth data. Extensive experiments show the power of \model{} to effectively identify temporal and structural anomalies in both single-layer and multiplex networks. Our case studies on brain networks and blockchain transaction networks show the usefulness of \model{} in a wide array of applications and domains. The success raises many interesting directions for future studies: \textbf{(1)} \model{} shows the potential to detect anomalous connections in human brains, which can help predict brain diseases or disorders. One future direction is to design an end-to-end model based on the \model{} architecture and directly optimize it to predict the presence of disease. \textbf{(2)} \model{} assigns the same importance to all snapshots, while earlier information might have less impact and importance than more recent information. 