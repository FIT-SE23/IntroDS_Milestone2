\section{Introduction}\label{sec:introduction}

% Anamaly detection in simple and dynamic networks
Identifying anomalous activities in networks is a long-standing and vital problem with a wide variety of applications in different domains, e.g., finance, social networks, security, and public health~\cite{anomaly-fraud, anomaly-first, anomaly-survey2, survey_anomaly_detection, anomaly-survey-dynamic}. While several anomaly detection approaches focus on the topological properties of networks~\cite{anomaly_static1, anomaly_static2, anomaly_static3, anomaly_static4, anomaly_static5, anomaly_static6}, detecting anomalies in real-world networks also requires attention to their dynamic nature~\cite{anomaly-survey-dynamic}. Anomalies might appear as malware in computer systems~\cite{unicorn}, social bots and social spammers in social networks~\cite{social_bots}, or financial fraud in financial systems~\cite{anomaly-fraud, anomaly_financial}. Accordingly, anomaly detection in dynamic (evolving) complex systems has recently~attracted~much~attention.


% Multilayer networks
Most prior work effort focuses on detecting anomalies in dynamic networks whose edges are all of the same type~\cite{anomaly_prob1, anomaly_distance1, F-FADE, anomaly_distance3, AddGraph, NetWalk, anomaly_dynamic1, anomaly-survey-dynamic}; these networks are called single-layer, dynamic networks. However, in many complex dynamic systems,
%such as social, transportation, and financial networks,
there are many different kinds of interactions between objects. For example, interactions between people can be social or professional, and professional interactions can differ according to topics.
We model graphs with different kinds of edges as Multilayer or Multiplex networks~\cite{main-ML}.
%, where nodes can interact in multiple layers, have been proposed to accurately model such applications. 
In these networks, the different types of connections are complementary to each other, providing more complex and richer information than simple graphs. Surprisingly, anomaly detection in multiplex networks is relatively less explored and has only recently attracted attention.


% drawbacks of current approaches
Existing approaches to anomaly detection in multiplex networks suffer from three main limitations: \textbf{(1)} Structure and feature inflexibility: existing methods assume pre-defined anomaly patterns or man-made features. Such approaches are application dependent and do not easily generalize to different domains. Moreover, in the real-world networks, anomalies might be more complex in nature, and it is nearly impossible to detect anomalies with high accuracy using pre-defined patterns/roles. \textbf{(2)} Same importance for all type of connections: these methods treat each relation type (i.e., layer) identically, assigning the same importance to each layer. However, real-world multiplex networks can contain noisy/insignificant layers~\cite{FirmCore, ml-core-journal}. Moreover, all vertices might not participate equally in all layers, so which layers are noisy/insignificant can be different for each vertex~\cite{FirmCore, ml-core-journal}. \textbf{(3)} Lack of edge anomaly detection: previous methods for anomaly detection in multiplex networks focus on identifying anomalous nodes, subgraphs, or events. However, in many real-world applications, a connection between two vertices might be an anomaly~\cite{NetWalk, F-FADE, AddGraph, anomaly_prob1}. This anomalous connection might be a suspicious transaction in a financial network, a fake follower in a social network, or an abnormal functional correlation between two regions of the brain.

% drawbacks of current approaches in simple graph
Existing methods for anomaly detection in single-layer dynamic graphs also exhibit limitations. \textbf{(1)} Structure inflexibility: even in single layer networks, most existing anomaly detection methods for dynamic networks rely on pre-defined patterns or heuristic rules (see \cite{anomaly_distance5, anomaly_distance4}). These heuristic rules are usually content features or long-term temporal factors. However, due the complex nature of real-world anomalies, these factors are not flexible and are restricted in a specific patterns. 
\textbf{(2)} Memory usage: deep learning based methods~\cite{AddGraph, NetWalk}, which are commonly proposed, requiring storing entire snapshots of the network at each time window, consuming large~and~increasing~amounts~of~memory. 


To mitigate the above limitations, we introduce \model{} (\textbf{\underline{Ano}}maly Detection in \textbf{\underline{Mul}}tiplex D\textbf{\underline{y}}namic Networks). To take advantage of both temporal properties and complementary information present in multiple relation (edge) types, \model{} extends the idea of \textit{hierarchical node states}~\cite{roland} to multiplex dynamic networks by using an attention mechanism that incorporates information about different relation types. Next, it uses selective negative sampling to learn anomalous edges in an unsupervised manner. To the best of our knowledge, \model{} is the first edge-anomaly detection method for multiplex networks.
Further, when it is possible to model a simple network as a multiplex network, \model{} outperform existing simple network approaches, because the multiplex network provides richer and more complex information than does a simple network~\cite{survey_anomaly_detection, anomaly-survey2, anomaly-survey-dynamic}.


% Applications:
%Since multiplex networks provide richer and more complex information than do simple networks, they can benefit typical applications of anomaly detection in simple dynamic graphs~\cite{survey_anomaly_detection, anomaly-survey2, anomaly-survey-dynamic}, delivering better solutions. 
Consider the following two applications for anomaly detection in dynamic multiplex networks:

\head{Applications: Brain Networks}
Monitoring functional systems in the human brain is a fundamental task in neuroscience~\cite{functional_system_brain, functional_system_brain2}. Each node in a brain network represents a region of interest (ROI), which is responsible for a specific function, and edges represent high functional correlation between two ROIs. A temporal brain network is usually derived from functional magnetic resonance imaging (fMRI), which lets us measure the statistical association between the functionality of ROIs over time. Since a (dynamic) brain network generated from an individual can be noisy and incomplete~\cite{Brain_network_fmri, FirmTruss}, prior work used the average of brain networks from many individuals~\cite{anomaly_brain1, brain_dataset}. However, these methods ignore the complex relationships in each individual's brain.
%which causes missing information~\cite{ml_brain_first}. 
We can capture these missing relationships
%in To monitor functional systems more accurately, one can construct
by modeling the network as a multiplex (dynamic) network~\cite{ml_brain_first, FirmTruss}, where each layer represents an individual's brain network. We show that edge anomaly detection approaches in a multiplex brain network can be used to detect abnormal connections in the brains of people who live with a brain disease or disorder (see~\autoref{sec:experiments}).

\head{Applications: Fraud Detection in Multiple Blockchain Networks}
Anomaly detection in (dynamic) blockchain transaction networks has recently attracted enormous attention~\cite{blockchain_anomaly1, blockchain_anomaly2, blockchain_anomaly3, blockchain_anomaly4, blockchain_anomaly5, blockchain_anomaly_survey}, due to the emergence of a huge assortment of financial systems' applications~\cite{blockchain_application1, blockchain_application2, blockchain_application3}.  While most existing work focuses on detecting illicit activity in a single blockchain network, recent research shows that cryptocurrency criminals increasingly employ cross-cryptocurrency trades to hide their identity~\cite{crypto_criminals, ofori2021topological}. Accordingly, \citet{blockchain_ml_first} have recently shown that analyzing links across several blockchain networks is critical for identifying emerging criminal activity on the blockchain. An edge anomaly detection approach in multiplex networks can be employed to detect suspicious transactions and identify criminal activities across several blockchain~transaction~networks~more~accurately.

The contributions of this work are: 
\textbf{(1)} We present a novel layer-aware node embedding approach in multiplex dynamic networks, \encoder, which uses an attention mechanism to incorporate both temporal and structural information on different relation types.
\textbf{(2)} We present \model, a general end-to-end unsupervised learning method for anomalous edge detection in multiplex dynamic networks, using a GRU cell to incorporate the outputs of \encoder{} for different snapshots. \textbf{(3)} We demonstrate a new application of edge-anomaly detection in dynamic multiplex networks and present a case study on brain networks of people living with attention deficit hyperactivity disorder (ADHD). Our results show the effectiveness and usefulness of \model{} in identifying abnormal connections of different ROIs of the human brain. This approach could be employed as a new tool to understand abnormal brain activity that might reveal a disease or disorder.
\textbf{(4)} We conduct extensive experiments on nine real-world multiplex and simple networks. Results show the superior performance of \model{} in both single-layer and multiplex networks. 