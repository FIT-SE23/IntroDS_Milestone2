\section{Related Work}
\label{sec:related-work}
We begin with a brief review of anomaly detection algorithms in dynamic simple networks, then methods for dynamic and multiplex graph learning, and finally anomaly detection in multiplex networks. To situate our research in a broader context, we discuss work on anomaly detection in brain and blockchain networks (~\autoref{app:additional-related-work}). For additional related work, we refer the reader to the extensive survey by \citet{survey_anomaly_detection}.


\head{Anomaly Detection in Dynamic Networks}
There has been much work in anomaly detection for single-layer  dynamic networks. That work falls in five categories: \textbf{(1)} Probabilistic methods~\cite{anomaly_prob1, anomaly_prob2, anomaly_prob3, anomaly_prob4, anomaly_prob5, anomaly_prob6, anomaly_prob7} that identify anomalies based on the pattern deviation from the regular communication patterns. \textbf{(2)} Distance-based methods~\cite{anomaly_distance1, anomaly_distance2, anomaly_distance3, anomaly_distance4, anomaly_distance5} that use certain time-evolving measures of dynamic network structures and use their changes to detect anomalies. \textbf{(3)} Density-based methods~\cite{anomaly_density1, anomaly_density2, anomaly_density3} that view anomalies as subgraphs with high density or as subgraphs with sudden changes in their density. \textbf{(4)} Matrix factorization methods~\cite{matrix_factorization1, matrix_factorization2, matrix_factorization3, matrix_factorization4, matrix_factorization5} that use the low-rank property of network structures and define anomalies as breakers of this property. \textbf{(5)} Learning-based methods~\cite{AddGraph, NetWalk, anomaly_dynamic_transformer, anomaly_dynamic_learning1, anomaly_dynamic_learning2, anomaly_learning3}, that combine the graph embedding method into the anomaly detection approach. These learning-based models must store the entire snapshot, which requires large memory,~limiting~their~scalability. To show the effectiveness of \model{} in even single-layer networks, we compare it with \textsc{NetWalk}\cite{NetWalk}, and \textsc{AddGraph}~\cite{AddGraph} in \autoref{sec:experiments}. All these methods apply to single-layer networks only and do not naturally extend to~multiplex~networks.
We further discuss the novelty of the architecture of our approach in~\autoref{app:additional-related-work}.

%\citet{NetWalk} proposed a dynamic graph embedding approach, \textsc{NetWalk}, based on random walk. It uses dynamic clustering over obtained nodes' embeddings to identify anomalies. \textsc{AddGraph}~\cite{AddGraph} is an end-to-end approach that extends GCN to temporal networks to capture both structural and temporal features. 
%
%In dynamic networks, several methods have been proposed in the context of edge streams~\cite{F-FADE, anomaly_prob1, anomaly_distance1, anomaly_distance2} and graph streams~\cite{NetWalk, AddGraph, anomaly_temporal1, anomaly_temporal2, anomaly_distance3, anomaly_prob4}. We review this works in five main categories:
%
%As an example, Bhatia et al. [] proposed \textsc{MIDAS} to find suddenly arriving groups of suspiciously similar edges by employing chi-squared goodness-of-fit test. 
 
%For example, Ranshous et al. [] proposed \textsc{CM-Sketch}, a sketch-based method, which uses the local structural information and historical behavior of an edge neighbourhood to decide whether the edge is anomalous. 
%Spot-Light [] is an embedding-based method that randomly samples a series of node sets, and encodes the the graph at each timestamp to a vector by computing the overlap between these sets and the nodes of current edge set. It then cluster these vectors to find anomalies.

%As an example, Shin et al. [] proposed \textsc{DenseAlert} for anomalous subgraph detection, which identifies dense subtensors created within a short time.
 
%As an example, Chang et al. [] proposed a novel frequency factorization algorithm, \textsc{F-FADE}, that spots anomalous incoming edges based on their likelihood of observed frequency. 


%All approaches in the abovementioned groups are based on a pre-defined patterns/roles and cannot learn the anomaly patters in the data, which limits their application. To address this issue learning-based methods have been introduced. \textbf{(5)} Learning-based methods~\cite{AddGraph, NetWalk}, that combine the graph embedding methods into the anomaly detection approaches. \citet{NetWalk} proposed a dynamic graph embedding approach, \textsc{NetWalk}, based on random walk. Then it uses dynamic clustering over obtained nodes' embedding, to identify anomalies. \textsc{AddGraph}~\cite{AddGraph} is an end-to-end approach that extends GCN to temporal networks to capture both structural and temporal features. These learning-based models need to store the entire snapshot, which requires large memory,~limiting~their~scalability.

%Specifically, Ranshous et al. [] proposed an anomaly detection method based on edge scoring, which is calculated based on historical evidence and node neighbourhood. Chang et al. [] proposed F-FADE, a frequency factorization approach for anomaly detection. Bhatia et al. [] proposed MIDAS to find suddenly arriving groups of suspiciously similar edges by employing chi-squared goodness-of-fit test. \textsc{HotSpot} [] is proposed to detect nodes whose neighbourhood suddenly and significantly change. Ranshous et al. [] proposed CM-Sketch,  a sketch-based method, which uses the local structural information and historical behavior of an edge neighbourhood to decide whether the edge is anomalous. Spot-Light [] is an embedding-based method that randomly samples a series of node sets, and encodes the the graph at each timestamp to a vector by computing the overlap between these sets and the nodes of current edge set. It then cluster these vectors to find anomalies. 



\head{Dynamic Graph Neural Networks}
The problem of learning from dynamic networks has been extensively studied in the literature~\cite{time-then-graph, dynamic_gnn1, dynamic_learning1, dynamic_learning2, dynamic_learning3, dynamic_learning4, dynamic_rnn1, yang2022few}. The first group of existing methods use Recurrent Neural Networks (RNN) and then replace its linear layer with graph convolution layer~\cite{dynamic_rnn1, dynamic_rnn2, dynamic_rnn3}.  
The second group uses a GNN as a feature encoder and then deploys a sequence model on top of the GNN to encode temporal properties~\cite{dynamic_gnn1, dynamic_gnn2, dynamic_gnn3}. However, all these models have limitations in both model design and training strategy~\cite{roland}. To address these limitations, \citet{roland} proposed \textsc{ROLAND}, a graph learning framework for dynamic graphs that can re-purpose any static GNN to dynamic graphs. However, this framework cannot be used for graphs with different types of edges (multiplex networks). Our work extends  \textsc{ROLAND} to multiplex networks and introduces an attention mechanism that incorporates the relation-specific hierarchical node states in each snapshot, taking advantage of additional information present in multiplex networks.



\head{Multiplex Graph Learning}
In a multiplex network, also known as multilayer, multi-view, or multi-dimensional networks, all nodes have the same type, but edges (relations) have multiple types~\cite{main-ML}. Several methods have been proposed to learn network embeddings on multiplex networks by integrating information from individual relation type~\cite{multiplex_int1, multiplex_int2, multiplex_int3, hetro1, hetro2, wang2020dynamic}. Other work proposed Graph Convolutional Networks (GCN) methods for multiplex networks~\cite{CS-MLGCN, multiplex_gcn1, multiplex_gcn2, multiplex_gcn1}. Inspired by Deep Graph Infomax~\cite{dgi}, \citet{unsupervisedML-2020} and \citet{multiplex_dgi} proposed unsupervised approaches to learn node embeddings by maximizing the mutual information between local patches and the global representation of the entire graph. \citet{scalable-ml-embedding} proposed a method that uses a latent space to integrate the information across multiple views. Recently, \citet{deep-partial-ml} proposed \textsc{DPMNE} to learn from incomplete multiplex networks. 

% Several methods have been proposed for learning on heterogeneous networks~\cite{hetro1, hetro2, hetro_survey}. Although multiplex networks can be seen as a special case of heterogeneous (with single type of nodes), these learning methods on heterogeneous graphs emphasize heterogeneous types of entities connected by different relationships, which is different from the concept of multiplex networks~\cite{unsupervisedML-2020}. 

% All existing methods for multiplex networks are designed for the static setting, which ignores the temporal properties of the graph. To the best of our knowledge, our \textit{Snapshot Encoder}, which is inspired by \textsc{ROLAND}~\cite{roland}, is the first method to learn the node embedding on multiplex dynamic~networks.


\head{Anomaly Detection in Multiplex Networks}
The problem of anomaly detection in multiplex networks has recently attracted attention. \citet{mittal2018anomaly} use eigenvector centrality, page rank centrality, and degree centrality as handcrafted features for nodes to detect anomalies in static multiplex networks. \citet{Bindu2017} proposed a node anomaly detection algorithm in static multiplex networks that uses handcrafted features based on  clique/near-clique and star/near-star structures. \citet{Bansal2020} defined a quality measure, Multi-Normality, which employs the structure and attributes together of each layer to detect attribute coherence in neighborhoods between layers. \citet{Centrality-Based-Anomaly} use 
centrality of all nodes in each layer and apply many-objective optimization with full enumeration based on minimization to obtain Pareto Front. Then, they use Pareto Front as a basis for finding suspected anomaly nodes. \citet{AnoMAN} proposed \textsc{AnomMAN} that uses an auto-encoder module and a GCN-based decoder to detect node anomalies in static multiplex networks. Although this model can learn from the data, it is limited to static networks, and it treats each layer equally in the \textit{Structure Reconstruction} step. 
Finally, \citet{ofori2021topological} developed a new persistence summary and utilized it to detect events in dynamic multiplex blockchain networks.

All of these approaches are designed to detect topological anomalous subgraphs, nodes, or events, and cannot identify anomalous edges. Moreover, as we discussed in \autoref{sec:introduction}, these methods, except \textsc{AnoMAN}~\cite{AnoMAN}, are based on pre-defined patterns/roles or handcrafted features, while in real-world networks anomalies have complex nature. Therefore, these models cannot be generalized to different domains, limiting their application. 


