%-------------------------------------------------------------------
\section{Apéndice: Predicciones con el Modelo SIR que considera la vacunación}

El análisis de las secciones pasadas es válido si no se considera el proceso de vacunación. Si se utiliza una modificación en el modelo que considere las personas \textbf{vacunadas} $V$ y una \textbf{velocidad de vacunación} estimada que varía linealmente con el tiempo $\tilde{\beta}(t)$, se producen los siguientes resultados.

Se reportan sólo las gráficas de las predicciones de los infectados activos (gráficas \ref{fig:Inf7_vac} y \ref{fig:Inf21_vac}), pues es donde se puede apreciar una diferencia notable entre las predicciones realizadas con el modelo que considera los vacunados y el modelo que no los toma en cuenta.

\subsection{Predicciones para 7 días}

El número de \textbf{infectados activos} predichos para dentro de 7 días, al 17.05.21, es de $\mathbf{715}$ (con una \textit{cota superior} de $1,017$ y una \textit{cota inferior} de $517$).
\begin{figure}[H]
    \centering
    \includegraphics[scale=0.58]{infectados7dias_vac.png}
    \caption{En los puntos rojos se muestran los casos infectados activos calculados, día a día, desde el 11 de abril del 2021 hasta el 10 de mayo del 2021. En la región azul se encuentran los valores predichos en los siguientes siete días usando un modelo que toma en cuenta la vacunación.}
    \label{fig:Inf7_vac}
\end{figure}

Para el número de \textbf{recuperados acumulados}, dentro de 7 días, al 17.05.21, se tiene una predicción de $\mathbf{120,920}$ (con una \textit{cota superior} de $121,326$ y una \textit{cota inferior} de $120,559$).

%\newpage

Para el número de \textbf{difuntos acumulados}, se tiene una predicción al 17.05.21 de $\mathbf{10,990}$ (con una \textit{cota superior} de $11,068$ y una \textit{cota inferior} de $10,916$).

Para el número de \textbf{infectados acumulados}, se tiene una predicción al 17.05.21 de $\mathbf{132,625}$ (con una \textit{cota superior} de $133,412$ y una \textit{cota inferior} de $131,992$).

%\vfill
%-------------------------------------------------------------------
\subsection{    Predicciones para 21 días.}

La predicción del número de \textbf{infectados activos} para dentro de 21 días, al 31.05.21, es de $\mathbf{404}$ (con una \textit{cota superior} de $751$ y una \textit{cota inferior} de  $238$).
\begin{figure}[H]
    \centering
    \includegraphics[scale=0.58]{infectados21dias_vac.png}
    \caption{En los puntos rojos se muestran los casos infectados activos calculados, día a día, desde el 11 de abril hasta el 10 de mayo del 2021. En la región azul se encuentran los valores predichos para los siguientes veintiún días usando un modelo que toma en cuenta la vacunación.}
    \label{fig:Inf21_vac}
\end{figure}
Para los \textbf{recuperados acumulados}, para dentro de 21 días, al 23.05.21, se tiene una predicción de $\mathbf{121,511}$ (con una \textit{cota superior} de $122,272$ y una \textit{cota inferior} de $120,940$).


Para los \textbf{difuntos acumulados}, se tiene una predicción a 21 días, al 31.05.21, de $\mathbf{11,048}$ (con una \textit{cota superior} de $11,161$ y una \textit{cota inferior} de $10,953$).


Para el número de \textbf{infectados acumulados}, se tiene una predicción al 31.05.21 de $\mathbf{132,964}$ (con una \textit{cota superior} de $134,186$ y una \textit{cota inferior} de $132,131$).

\subsection{La velocidad de vacunación}
En la página de la Secretaría de Salud del Estado de Guanajuato se reportan las vacunas aplicadas. Por la ambigüedad que se encontró en esa definición, se optó por suponer que el número de personas vacunadas es el número de vacunas aplicadas entre dos.

Dado que hay días en los que el número de personas vacunadas no cambia, para el los siguientes cálculos sólo se usaron los datos de los días en los que hubo un cambio en el número (los datos que en la figura \ref{fig:datosvac} aparecen en color naranja).

\begin{figure}[H]
    \centering
    \includegraphics[scale=0.58]{datosvac.png}
    \caption{El número de vacunados acumulados desde el 17 de marzo del 2021 (el día en el que se empezó a reportar el número de vacunas) hasta el 10 de mayo el 2021. Los puntos naranjas son aquellos que se usan para calcular la velocidad de vacunación}
    \label{fig:datosvac}
\end{figure}

La \textbf{velocidad de vacunación} dada en unidades de $[\tilde{\beta}]=$personas$/$d\'{\i}as, se estima con los datos desde el 17 de marzo del 2021 y se ajusta a una función lineal:
\begin{equation*}
    \tilde{\beta} = (-4 \pm 3) t +8,000 \pm 1,000
\end{equation*}                      

Su ajuste en función del día se muestra en la \textbf{figura \ref{fig:velvac}}. Note que esta tasa se construye como el cociente de la cantidad de nuevas personas vacunadas entre el número de días que hay entre cada dato usado (las diferencias en el eje horizontal entre cada punto naranja consecutivo en la figura \ref{fig:datosvac}).

\begin{figure}[H]
    \centering
    \includegraphics[scale=0.58]{betatilde.png}
    \caption{El ajuste lineal entre el valor del la valocidad de vacunación contra el día en el que se obtiene tal valor.}
    \label{fig:velvac}
\end{figure}
