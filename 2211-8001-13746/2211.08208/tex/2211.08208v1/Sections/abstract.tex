En el presente reporte se analizan datos de infectados, recuperados y fallecidos del estado de Guanajuato empleando un modelo SIR (Suceptibles, Infectados y Recuperados). Se considera una tasa de contagio variable ($\beta$) y se ajusta con datos al {\bf 10.05.21}. Se eval\'ua la taza de recuperaci\'on local ($\gamma$). Se realizan estimados de infectados activos, acumulados, recuperados y fallecidos a una  y tres semanas incluyendo rangos de error.


Hagamos algunas acotaciones:
\begin{itemize}
\item El $4$ de abril del $2020$ es el día 1 a partir del que comienza el conteo que aparece en las \textbf{gráficas de la \ref{fig:Inf7} a la \ref{fig:Dif21} y la \ref{fig:logbeta}}.

\item El \textbf{número de infectados activos} en un día dado se calcula a partir de los datos reportados en el mismo día como el número de infectados acumulados menos los recuperados acumulados y los difuntos acumulados.

\item La \textbf{incertidumbre superior} de cada valor predicho (infectados activos, recuperados acumulados y los difuntos acumulados) es la cota superior para ese valor menos el valor predicho.

\item La \textbf{incertidumbre inferior} es el valor predicho menos la cota inferior.

\item El \textbf{número de infectados acumulados predichos} se calcula como los infectados activos predichos más los recuperados acumulados predichos y los difuntos acumulados predichos.

\item La \textbf{cota superior de la predicción de infectados acumulados} se calcula como los infectados acumulados más la suma de las incertidumbres superiores de infectados activos, recuperados acumulados y difuntos acumulados.

\item La \textbf{cota inferior de la predicción de infectados acumulados} son los infectados acumulados menos la suma de las incertidumbres inferiores de infectados activos, recuperados acumulados y difuntos acumulados.

\item Se referirá el número de difuntos acumulados más recuperados acumulados como los \textbf{inmunes} y esa cifra se denotará como $\tilde{R}$.

\item El número de recuperados fue ajustado linealmente en función de los inmunes (tanto en  la proyecci\'on a 7 y a 21 días). 

\item El número de difuntos fue ajustado linealmente en función de los inmunes (tanto en  la proyecci\'on a 7 y a 21 días). 

\item  En la  \textbf{seccci\'on \ref{sec:pred7}} se reportan las proyecciones a 7 d\'{\i}as.
En la \textbf{sección \ref{sec:pred21}} se reportan las proyecciones a 21 d\'{\i}as.
En la \textbf{ secci\'on \ref{sec:ajustes}} se reporta la tasa de recuperaci\'on $\gamma$, la tasa de 
contagio $\beta$, as\'{\i} como el ajuste de fallecidos contra recuperados.

\end{itemize}  