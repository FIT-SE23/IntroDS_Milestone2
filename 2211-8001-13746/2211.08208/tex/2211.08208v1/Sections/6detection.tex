\section{How to obtain the effective detection rate}\label{sec 6}
The \textbf{detection rate} is the proportion of the reported COVID-related statistics versus the real ones. Naturally, this number cannot be directly measured and, because of this, the estimation of this rate comes with a lot of uncertainty. This, of course, is a problem even if we don't take into consideration factors like the vaccination process.

Due to the \textit{vaccination independence} of the $\beta_S$ model, it is an useful tool in the estimation of the detection rate with only the most elemental factors: the reported infections, the recovered cases and the reported deaths. Nonetheless, it is important to say that the following method does not necessarily captures the real detection rate, but instead it gives an \textbf{effective detection rate}, which may come in handy only during an analysis based on the SIR model. However, in the following, we will limit to write \textit{detection rate} but it should be understood that it actually refers to the \textit{effective detection rate}.

The estimation of the detection rate goes as following:
\begin{enumerate}
    \item Choose an interval of time (in our case, we used the thirty most recent dairy registers on the database) such that even the most recent date plus the time window in which you make predictions is still in the dataset (in our case, such time window is 21 days). These set of registers from 52-21 days before the final date will be named \textbf{T-data}.
    \item On the interval $(0,1]$, which constitutes the possible values of the detection rate $k$, choose how many equidistant points you want to analyze (we used $100$ points, ranging from $0.01$ to $1.00$). The effective detection rate will arise from this set of values $k_{eff}\in (0,1]$. For simplicity, this set will be named the \textbf{k-interval}.
    \item For the oldest register in the T-data, make the predictions for the time window selected (21 days) for the different points on the k-interval. For each prediction, given a $k$ value, compare it with the real value using your preferred metric. For example, we used the sum of the relative errors between the prediction and the real point squared for the infected cases and the recovered ones. From all of those values, choose the value of $k$ which produces the least absolute difference between the predictions and the real value. This will be the characteristic $k$ value for this register.
    \item Repeat the last step for every register on the T-data.
    \item From the set of all characteristic $k$ values, we obtain the mean $k$ value and its respective standard deviation. This is the detection rate $k_{eff}$ and its associated estimation error $\Delta k_{eff}$.
\end{enumerate}

Note that it is possible to find some cases where the minimum $k$ on a given day is $1$. We ignored such cases for the calculation of $k_{eff}$ and $\Delta k_{eff}$, because we found that the reasonable values (by this, we mean the most frequent minimum $k$) are more close to $0.1$ than they are to $1$ (and, physically, this makes sense). As a matter of fact, those \textit{non-converging} values of $k$ \textbf{do} converge to a reasonable value if we increase the precision of the search. That is, studying smaller orders of magnitude for the decimals of $k$.

For México, analyzing the 5 months prior to 10-16-2021 with a precision up to 8 decimal places (using an optimized algorithm for searching around $k\sim0.1$), we obtain an effective detection rate of $0.015\pm0.022$. All the values of the time series range from $5.5560\times10^{-4}$ to $0.10000862$.

For Guanajuato, analyzing the 5 months prior to 10-18-2021 with a precision up to 8 decimal places (using said optimized algorithm for $k\sim0.1$), we obtain an effective detection rate of $0.027\pm0.019$. All the values of the time series range from $2.500\times10^{-4}$ to $0.0970$.

We also estimate the $k$ values by a different metric, which judges better the
global features. We calculate the difference between the population prediction of a given day
and the real population value, an sum the squares of the differences over the following 21 days to the T-data point. The populations are the infected and the active cases. We select the $k$ which minimizes this differences. There are ore details about it in \textbf{Section \ref{sec 10}}. For this method of estimation, for the same time period, we obtain $k_{global}=0.15$ for México and $k_{global}=0.04$ for Guanajuato.

It should be noted that for Guanajuato, $k_{global}\sim k$. This property can be attributed to the distribution that the detection rates at different times show. In the \textbf{Figure \ref{fig:hist_gto}} it can be seen that the distribution is approximately normal. The normality of the distribution justifies the use of a mean value and a standard deviation to describe the overall behaviour of the sample. Hence, the global rate is similar to the mean rate.

\begin{figure}[H]
    \centering
    \includegraphics[scale=.55]{Images/hist_gto.png}
    \caption{The histogram of the local detection rates for Guanajuato on the 5 months prior to 10-18-2021.}
    \label{fig:hist_gto}
\end{figure}

For México, $k_{global}$ is not similar to $k$. The reason for this may be similar to that of Guanajuato, but in this case, it is because of the non-normality of the distribution shown in \textbf{Figure \ref{fig:hist_mex}}.
\begin{figure}[H]
    \centering
    \includegraphics[scale=.55]{Images/hist_mex.png}
    \caption{The histogram of the local detection rates for México on the 5 months prior to 10-16-2021.}
    \label{fig:hist_mex}
\end{figure}
Point is, if someone found a method for constructing a normal distribution for the historical detection rates of México, then $k_{global}\sim k$.


%There should be a method for approximating a normal distribution for México, for example, increasing the search interval from $(0,1]$ to $[-1,1]-{0}$.