\section{La tasa de transmisión ($\beta$), la tasa de recuperación ($\gamma$) y el ajuste de los difuntos.}\label{sec:ajustes}

En las tres gráficas siguientes, los \textbf{puntos azules} son los datos usados, la \textbf{recta roja} es el ajuste de interés en cada caso, la \textbf{región azul} sombreada es el intervalo de confianza del $95$ $\%$ del modelo y las \textbf{curvas grises} en torno al modelo y el intervalo de confianza contienen al intervalo de predicción del $95$ $\%$.

La \textbf{tasa de recuperación efectiva} dada en unidades de $[\gamma_{eff}]=$1/d\'{\i}as, se estima para los últimos 150 días mediante un ajuste lineal como se muestra en la \textbf{figura \ref{fig:gamma}}. Esta tasa es la pendiente de la recta, $\gamma_{eff}$ = ({8.5} $\pm$ {0.3}) $\times$ \num{e-2}.
%Tasa de recuperación efectiva: 0.08517 +/- 0.00282


\begin{figure}[H]
    \centering
    \includegraphics[scale=0.6]{gamma.png}
    \caption{El ajuste lineal entre los infectados activos al día contra los nuevos casos inmunes. La pendiente de este ajuste es la tasa de recuperación (calculada con la información de los últimos $150$ días).}
    \label{fig:gamma}
\end{figure}

La \textbf{tasa de transmisión} dada en unidades de $[\beta]=$1/(d\'{\i}as$\times$ personas), se estima de los últimos 30 días se ajusta a una exponencial con los siguientes parámetros:
\begin{equation*}
    \beta =  e ^  {(0.0027 \pm 0.0004) t -2.8 \pm 0.1}. 
%Tasa de infección como función del tiempo: exp(0.00289 t + -3.8059) * exp(+/- 0.00038 t +/- 0.14447)
\end{equation*}                      

Su ajuste en función del día se muestra en la \textbf{figura \ref{fig:logbeta}}. Note que esta tasa se construye, día por día, como la suma de $\gamma_{eff}$ con el cociente la cantidad de infecciones nuevas entre las infecciones activas ese día.

\begin{figure}[H]
    \centering
    \includegraphics[scale=0.58]{beta.png}
    \caption{El ajuste lineal entre el valor del logaritmo de la tasa de transmisión contra el día en el que se obtiene tal valor. El exponencial de este ajuste es la expresión que se busca de la tasa de transmisión como función del tiempo.}
    \label{fig:logbeta}
\end{figure}

%esto creo falta redactar que no es solo función de los difuntos  aun----------------------------------------
El \textbf{ajuste de los recuperados en función de los inmunes} de los últimos 274 días se muestra en la \textbf{figura \ref{fig:delta}}. Los parámetros se presentan a continuación.
\begin{equation*}
    R = (0.9107 \pm 0.0002) \tilde{R} +800 \pm 20.
\end{equation*}  

\begin{figure}[H]
    \centering
    \includegraphics[scale=0.58]{ajusterecuperados.png}
    \caption{El ajuste lineal entre el número de recuperados acumulados y los inmunes (calculado con la información de los últimos $274$ días).}
    \label{fig:rho}
\end{figure}

El \textbf{ajuste de los difuntos en función de los inmunes} de los últimos 274 días se muestra en la \textbf{figura \ref{fig:delta}}. Los parámetros se presentan a continuación.
%cambie la palabra recuperados por inmunes 
\begin{equation*}
    D = (0.0893 \pm 0.0002) \tilde{R} -800 \pm 20.
    %Ajuste de difuntos en función de inmunes: (0.09042 Rtilde + -904.57) +/- (0.00019 Rtilde +/- 16.7945)
\end{equation*}          



\begin{figure}[H]
    \centering
    \includegraphics[scale=0.58]{ajustedifuntos.png}
    \caption{El ajuste lineal entre el número de difuntos acumulados y los inmunes  (calculado con la información de los últimos $274$ días).}
    \label{fig:delta}
\end{figure}
