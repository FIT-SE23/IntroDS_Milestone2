

\section{Estimating the contagion rate via the $\beta_S$ model}\label{sec 4}

Given the SIR equations, it is possible to simplify the system of equations by fitting the $\beta(t) S(t)$ terms into a single function, $\beta_S(t)$. By doing this, the SIR system reduces to the following pair of equations:

\begin{equation}
    \dv{I}{t}=\beta_S I-\gamma_{eff}I
\end{equation}
\begin{equation}
    \dv{\Tilde{R}}{t}=\gamma_{eff}I
\end{equation}
The $\beta_S$ function is relevant because it captures the impact that the vaccination process has in the evolution of the pandemic without the need of introducing that information into the model. This happens
because the vaccinated population decreases the \textit{total} susceptible population (although it has been proved that vaccines don't necessarily limit the spread of the virus, they limit the severity of the infections; however, it can be hypothesized that this has an impact in the \textit{net} susceptible populations and in the \textit{net} transmission of the virus).

The fitting of $\beta_S$ is made by plotting $\gamma_{eff}+\dot{I}/I$ versus the time axis and fitting the points to an exponential function. We use as input the constant $\gamma_{eff}$ obtained in earlier sections.
Given that the numbers come from a daily report, $\dot{I}$ represent the new daily infected cases reported and $I$ represent the active cases of the day. By this, it can be noted that numerically, $\beta_{approx}$ and $\beta_S$ are identical but their meanings are completely different.

%Depending if one works with the total infections $I_{tot}$ or the 

detected infections $I$, the $\beta_S$ terms need to be multiplied by a scaling factor. Let's say, if you go from the detected infections to the total infections $I\rightarrow I_{tot}=I/k$ and the inmune cases $\Tilde{R}\rightarrow \Tilde{R}_{tot}=\Tilde{R}/k$ (where $k$ is the detection rate), then $\beta_S\rightarrow \beta_{S,tot}=\lambda(k)\beta_S$, such that $\lambda(k)=S_{tot}/S$, where $S=1-(I+\Tilde{R})$ and $S_{tot}=1-(I_{tot}+\Tilde{R}_{tot})$ (the last expressions are valid using the units where the total population $N$ is set to 
$1$). Note that $\beta_S$ accounts for the change in the 
transformation between the \textit{reported} susceptible 
population and the real one. It is important to note that we use the term \textit{reported} when referring to infected because it 
is indirectly obtained by taking into consideration only the reported cases and comparing it to the total population, which is admittedly tricky and can be avoided -if wanted- by scaling the total population along the infected cases and the recovered ones. This is not what we do in this work for reasons that will become apparent when we estimate the detection rate. In other words, we construct quantities which are not all invariant by scaling the reported quantities to the real ones precisely to estimate the detection rate. In any case, the latter considerations cannot be applied to a model where precisely measured terms are considered (like the vaccination rate, for example), therefore, the simplification mentioned can only be used in the simpler models.

