\section{Vaccination effects on the model}\label{sec 7}
The vaccination process intends to produce a decrease in the susceptibility of the population to the virus. By this, supposing that the vaccines are applied at a rate $\Tilde{\beta}$ and that they produce immediate immunity, then the SIR model can be modified as follows:
\begin{equation}
    \dv{S}{t}=-\beta SI-\Tilde{\beta}
\end{equation}
\begin{equation}
    \dv{I}{t}=\beta SI-\gamma_{eff}I
\end{equation}
\begin{equation}
    \dv{\Tilde{R}}{t}=\gamma_{eff}I
\end{equation}
\begin{equation}
    \dv{V}{t}=\Tilde{\beta}
\end{equation}
Where it can be easily noticed that the equations for the evolution of $I$ and $\Tilde{R}$ remain the same, while the $S$ equation adds a term considering the vaccination rate, which is defined on the fourth equation, where $V$ is precisely the vaccinated population. Notice that it is supposed that the set of people recovered and the vaccinated population are disjoint sets (which may be a valid supposition when $S>>\Tilde{R}$ because even if $S>>V$ or if $S\sim V$, then the contribution of $\Tilde{R}$ becomes either irrelevant together with the $V$ contribution or irrelevant when compared with the $V$ contribution).

It is important to mention that $V$ remains the same when $I$ and $\Tilde{R}$ are scaled by the detection rate because it is reasonable to assume that there is a strict control of the vaccines applied, such that all -or, in the worst case scenario, almost all- of the vaccinated people is properly taken into consideration when the statistics are reported. Therefore, this model isn't invariant when the aforementioned scaling is applied, just as discussed for the $\beta_S$ model.

\subsection{Comparison with $\beta_S$}
As we mentioned earlier, the $\beta_S$ model is interesting because it captures the effects of a lot of factors that may not be taken into consideration in all of the other discussed models. It is the case for the vaccination process. The $\Tilde{\beta}$ model for the vaccination works under the assumptions that the vaccine -whatever vaccine it is- produces an immediate perfect immunity and prevents completely the spread of the virus. Whoever, it is possibly to check if those assumptions are too much for it to make good predictions or not by, precisely, comparing a bunch of predictions using both models and then checking if those  predictions are adequate.