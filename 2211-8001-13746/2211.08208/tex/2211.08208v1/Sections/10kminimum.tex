\section{\textbf{Appendix}: Another method for estimating the detection rate}\label{sec 10}

There is another possible way to estimate the detection rate. However, it isn't as efficient as the first method introduced and it doesn't lead to an easy estimation of the uncertainty of the rate.

The alternative estimation of the detection rate goes as following:
\begin{enumerate}
    \item On the interval $(0,1]$, which constitutes the possible values of the detection rate $k$, choose how many equidistant points you want to analyze (we used $100$ points, ranging from $0.01$ to $1.00$). The effective detection rate will arise from this set of values $k_{eff}\in (0,1]$. For simplicity, this set will be named the \textbf{k-interval}.
    \item Choose an interval of time (in our case, we used the thirty most recent daily registers on the dataset) such that even the most recent date plus the time window in which you make predictions is still in the dataset (in our case, such time window is 21 days). These set of registers from 52-21 days before the final date will be named \textbf{T-data}.
    \item For the first $k$ value in the k-interval, make the predictions for the time window selected (21 days) for every register in the T-data. For each prediction, given a register in the T-data, compare it with the real value using your preferred metric. For example, we used the sum of the relative errors between the prediction and the real point squared for the infected cases and the recovered ones. Sum over all of those error values.
    \item Repeat the last step for every $k$ value on the k-interval.
    \item Compare the sum of errors associated with every $k$ values. The $k$ value with the least global error is the estimated detection rate $k_{eff}$.
\end{enumerate}

Even if this method doesn't gives an estimation error, it is a good way to check if the effective detection rate obtained using the method presented in the \textbf{Section \ref{sec 6}} \textit{makes sense}.