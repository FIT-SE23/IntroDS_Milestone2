\section{\textbf{Appendixi}: Lambda Factor and $\frac{dI}{dt}$}\label{sec 12}
To make predictions we need data so we can compute quantities like infection rate or recovery rate but if we compare the predictions made with reality we observe that predictions always are below reality, this is well know because data recollected is just an approximation to real quantities that is why we use an the detection rate k but this have implications in SIR model, we refer to reported or observed quantities to data that is given by the government or different organizations and real quantities to an estimation of how much people is infected or has recovered in reality.\\
 Working with real quantities requires using the detection rate, we can transform between both if $I_o$ represents observed infected people then $I_r=\frac{I_o}{k}$ where $I_r$ are real infected people.
\\
If we ignore births $S_r+I_r+\tilde{R_r}=1$ where the subscript r refers to real quantities and the sum is one because we work with normalize population, $\tilde{R}$ represent the sum of recovered and dead people then we can write.
\begin{equation}\label{ec lambda factor 1}
    \frac{\tilde{R_o}}{k}+\frac{I_o}{k}+S_r=1
\end{equation}
From equation \ref{ec lambda factor 1}.
\begin{equation}\label{ec lambda factor 2}
    S_o=kS_{r}-\tilde{R}_o+1
\end{equation}
From equation \ref{ec lambda factor 2}.
\begin{equation}\label{lambda 4}
    \frac{S_r}{S_o} = \frac{1}{k}+\frac{\tilde{R}_r-\frac{1}{k} }{S_o}
\end{equation}
Working with equation (\ref{lambda 4}) and defining $\lambda=\frac{S_r}{S_o}$.
\begin{equation}\label{lambda final}
    \lambda=\frac{1-I_r-\tilde{R_r}}{1-k(I_r+\tilde{R_r})}
\end{equation}
With (\ref{lambda final}) the second equation in the SIR model its.
\begin{equation}\label{lambda factor 3}
    \frac{dI_r}{dt}=(\beta_s \lambda-\gamma_{eff})I
\end{equation}
Where $\beta_s=\beta S_o=\beta \lambda S_r$
\begin{figure}[H]
    \centering
    \includegraphics[scale=0.5]{Images/rlambdafact.png}
    \caption{$\lambda$ factor in the function of time as we can see it is always decreasing with time.}
    \label{lambda graph}
\end{figure}
As we can see from definition of $\lambda$ its always decreasing with time and is bounded between 0 and 1 see figure ( \ref{lambda graph} ). This implies than even when $\beta_s$> $ \gamma_{eff}$ we can have $\frac{dI}{dt}$<0 see figure(\ref{dIdt}).
\begin{figure}[H]
    \centering
    \includegraphics[scale=0.5]{Images/compdidt.png}
    \caption{Comparison between dI/dt and $\beta_s-\gamma$ as we can see even when $\beta$-$\gamma$ > 0 dI/dt can be negative.}
    \label{dIdt}
\end{figure}