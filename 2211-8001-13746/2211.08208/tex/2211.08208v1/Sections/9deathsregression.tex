\section{\textbf{Appendix}: Deaths linear regression}\label{sec 9}
It is necessary to remember that we can't directly make predictions about the number of recovered $R$ or deceased $D$ cases with any of the discussed models, that is because we used the \textit{immune} category, $\Tilde{R}=R+D$, which is the union of both sets. This means that it is necessary to make complementary adjustments of $R$ and $D$ both as functions of $\Tilde{R}$. We opted to use linear regressions, that is:
\begin{equation}
    R=\rho \Tilde{R}+b_\rho
\end{equation}
\begin{equation}
    D=\delta \Tilde{R}+b_\delta
\end{equation}
Where $\rho$ and $\delta$ are the slopes for the $R$ and $D$ fit and, likewise, $b_\rho$ and $b_\delta$ are the y-axis intersection for their respective straight line.

Intuitively, one should hope that both lines pass through the origin -and therefore, both $b_\rho$ and $b_\delta\approx 0$-, because that means that when there are $0$ immune people there are $0$ dead and recovered cases. However, this is usually not the case, because the y-axis intersections tend to \textit{not be that close to $0$}. There are a few of possible reasons.

The first possible explanation may be that the linear adjustment may not be a good fit and there may be a polynomial such that, for that expression, the intersection with the y-axis is, indeed, \textit{close to $0$}. It is possible to see if that's a possibility by doing just that for polynomials of various degrees and comparing the values of their y-axis intersections.

To check the first explanation, we ran an analysis running from April $4$th, $2020$ to the June $15$th, $2021$ (that is, the entire database until the day of the analysis). We $R$ as a function of $\Tilde{R}$ by fitting polynomials of different order and then checked the value of the intersection. The graph \ref{fig:orderR} shows the $y-axis$ intersection as a function of the degree of the fitted polynomial. Note that we fitted a function (using \textit{curve\_fit}, from \textbf{scipy}) to predict the behaviour of the intersection as the polynomial degree tends to infinity. We did the same procedure for $D$ as a function of $\Tilde{R}$, it is represented in the figure \ref{fig:orderD}.
The function fitted for \ref{fig:orderR} is (noting the intersection as $y_R$ and the degree as $n$):
\begin{equation*}
    y_R=-e^{-0.53n}\bigg(1,700cos(0.15n+1.5)+
\end{equation*}
\begin{equation}
    +440sin(1.9n-0.36)\bigg)+3.4
\end{equation}
With an error in the last term of $7.4$

\begin{figure}[H]
    \centering
    \includegraphics[scale=.55]{Images/order_RtildeVR.png}
    Optimal control of a SIR epidemic model with general incidence function and a time delays
    \caption{The intersection of the fit of $R(\Tilde{R})$ and the origin as a function of the degree of the polynomial fitted. This set of blue points is interpolated by a function (the blue curve) that behaves like a damped oscillator that tends to the equilibrium position shown. This shows that said intersection goes to zero with better polynomial approximations of real $R(\Tilde{R})$.}
    \label{fig:orderR}
\end{figure}

The function fitted for \ref{fig:orderD} is (noting the intersection as $y_D$ and the degree as $n$):
\begin{equation*}
    y_D=e^{-0.33n}\bigg(5,300cos(0.66n+0.76)+
\end{equation*}
\begin{equation}
    +3,000sin(1.7n+0.31)\bigg)-19
\end{equation}
With an error in the last term of $35$.

\begin{figure}[H]
    \centering
    \includegraphics[scale=.55]{Images/order_RtildeVD.png}
    \caption{The intersection of the fit of $D(\Tilde{R})$ and the origin as a function of the degree of the polynomial fitted. This set of blue points is interpolated by a function (the blue curve) that behaves like a damped oscillator that tends to the equilibrium position shown. This shows that said intersection goes to zero with better polynomial approximations of real $R(\Tilde{R})$.}
    \label{fig:orderD}
\end{figure}

By that analysis, note that only the last terms of the fit are relevant because as $n$ grows to infinity, the negative exponential will decrease until it is equal to zero. It can be noted that the terms that will survive aren't exactly zero, but it can be noted that zero is inside the interval of uncertainty. Therefore, we can see that the first explanation is plausible. 

%%%% HAY QUE HACER UNAS GRÁFICAS PARA DESCARTAR LA EXPLICACIÓN DE QUE EL ERROR DE LA INTERSECCIÓN ES INDEPENDIENTE DEL MODELO

The second explanation is complement to the first one. It may be that it is a consequence of the oscillations along the trend in both the recovered and deceased cases versus the total immune population. These oscillations may be interpreted as the back and forth between the population ignoring the safety measures when the things \textit{get better} and the population obeying such measures when things \textit{get worse}. Such perturbations may be the cause.

As a tangential mention, one could argue that there always can be lag and other error-inducing processes that happen between the hospital reports to the government and the reports that the government releases to the public, hence the variations along the linear fit.

%The third possible explanation that the authors can come up with may be that this disagreement is caused by a less than ideal treatment of data. This too can be seen in the scatter plot for both the recovered and deceased cases versus the total immune population as the oscillations of points along the trend. Maybe not all the hospital reports came in time to the daily cut or maybe not all institutions inform daily their local status about the cases or maybe not all the cases are getting reported by the authorities or maybe there are some cases that get on hold for whatever reason and don't get released until after some time passes. All the speculation about the data treatment that one can come up with may have its affect on those oscillations.