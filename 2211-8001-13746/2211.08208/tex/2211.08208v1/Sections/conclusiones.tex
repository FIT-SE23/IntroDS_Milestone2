\section{Conclusiones}
Los datos actualizados de la epidemia en el estado de Guanajuato, analizados con la suposici\'on de un modelo SIR muestran una tendencia a la disminuci\'on de casos activos, sin embargo se proyecta que seguir\'an en el orden de los miles en las pr\'oximas semanas. Esperamos que estas estimaciones sirvan como gu\'{\i}a para el an\'alisis de la situaci\'on actual. Este reporte es un estudio matem\'atico, y sus proyecciones son responsabilidad \'unica de los autores.

\section{Agradecimientos}

 Agradecemos al Programa de Servicio Social de la Universidad de Guanajuato, que nos permiti\'o realizar el proyecto ``Modelación matemática de la epidemia del COVID-19" del cual este reporte es parte, agradecemos a Gabriel Am\'ezquita, Oscar Esaul Cervantes, Juan Carlos God\'{\i}nez e Iv\'an Yebra por su colaboraci\'on en dicho proyecto. Agradecemos a los investigadores: Argelia Bernal, Juan Barranco, Alejandro Cabo, Alma Gonz\'alez, Dami\'an Mayorga, Gustavo Niz y Luis Ure\~na por \'utiles discusiones en esta tem\'atica.  Agradecemos al Laboratorio de Datos de la DCI, UG, el apoyo del proyecto  CIIC 264/2022 UG, y del proyecto CONACyT A1-S-37752.

\newpage
\newpage