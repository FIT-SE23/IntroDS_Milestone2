\section{Introduction}
\label{sec:intro}

Brain aging involves complex biological processes, such as cortical thinning, that are highly heterogeneous across individuals, suggesting that people do not age in the same manner. Accurately modeling brain aging at the subject-level is a long-standing goal in neuroscience as it could enhance our understanding of age-related diseases such as neurodegenerative disorders. To this end, brain-age predictors linking neuroanatomy to chronological age have been proposed using Deep Learning (DL)~\cite{peng2021}. 
In order to build accurate biomarker of aging, DL models need large-scale neuroimaging dataset for training, which often involves multi-site studies, partly because of the high cost per patient in each study. 
\begin{figure}
    \centering
    \begin{subfigure}[b]{0.45\linewidth}
        \includegraphics[width=\linewidth]{img/viz-yaware.pdf}
        \caption{y-aware}
        \label{fig:viz-yaware}
    \end{subfigure}
    \begin{subfigure}[b]{0.45\linewidth}
        \includegraphics[width=\linewidth]{img/viz-threshold.pdf}
        \caption{threshold}
        \label{fig:viz-threshold}
    \end{subfigure}
    \begin{subfigure}[b]{0.45\linewidth}
        \includegraphics[width=\linewidth]{img/viz-expw.pdf}
        \caption{exp}
        \label{fig:viz-expw}
    \end{subfigure}
    \caption{Comparison between different contrastive learning regression losses and their effect on the representations. Samples are aligned ($\gg$ $\ll$) and repelled ($\ll \gg$) with varying strength (line thickness) based on the continuous label $y$. 
    }
    \label{fig:viz-losses}
\end{figure}
Recent works have shown that DL models, and in particular Deep Neural Networks (DNN), largely over-fit site-related noise when trained on such multi-site datasets, notably due to the difference in acquisition protocols, scanner constructors, physical properties such as permanent magnetic field~\cite{wachinger2021detect, glocker2019machine}. This also implies poor generalization performance on data from new incoming sites, highly limiting the applicability of these models to real-life scenarios. 
In order to build more robust and accurate brain age models insensitive to site, the OpenBHB challenge~\cite{dufumier2022openbhb} has been recently released. 
While most DNN used to derive brain age gap are usually trained as standard regressors with the optimization of mean absolute error~\cite{cole2017predicting, Jonsson2019}, Ridge or cross-entropy loss~\cite{peng2021} (if age is binarized), these frameworks do not pay particular care about site-related information during training to produce robust representations of brain imaging data. 
On the other hand, contrastive learning paradigms for DNN training have been recently proposed in various contexts such as supervised~\cite{khosla2020supervised}, weakly-supervised~\cite{tsai2022conditional, dufumier2022rethinking} and unsupervised representation learning ~\cite{chen2020simCLR,barbano2023unbiased}.
More importantly, contrastive learning has been shown to be more robust than traditional end-to-end approaches, such as cross-entropy, against noise in the data or the labels, resulting in better generalizing models~\cite{khosla2020supervised,Graf2021}.
For this reason, in this work, we propose a novel contrastive learning loss for regression in the context of the OpenBHB challenge, where chronological age must be learned without being affected by site-related noise. With our method, we obtain the best results in the official leaderboard. 
Our contributions are twofold:
\begin{itemize}
    \item We propose a novel contrastive learning regression loss for brain age prediction;
    \item We achieve state-of-the-art performance in brain age prediction on the OpenBHB challenge.
\end{itemize}


