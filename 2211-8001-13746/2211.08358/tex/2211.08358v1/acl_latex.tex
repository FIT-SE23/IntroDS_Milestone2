% This must be in the first 5 lines to tell arXiv to use pdfLaTeX, which is strongly recommended.
\pdfoutput=1
% In particular, the hyperref package requires pdfLaTeX in order to break URLs across lines.

\documentclass[11pt]{article}
\usepackage{authblk}


% Remove the "review" option to generate the final version.
%\usepackage[review]{acl}
\usepackage[]{acl}
\usepackage{times,latexsym}
\usepackage{url}
\usepackage[T1]{fontenc}

%% Our packages
\usepackage{graphicx}
\usepackage{amsmath}
\usepackage{tabularx}
\usepackage{multicol}
\usepackage{multirow}
\usepackage{algorithm}
\usepackage{algpseudocode}
\usepackage{inconsolata}
\usepackage{tikz}
\usepackage{tcolorbox}
\usepackage{enumitem}
\usepackage{booktabs}

\usepackage{xspace,mfirstuc,tabulary}
%% Our packages

\DeclareMathOperator*{\argmax}{\textbf{argmax}}
\DeclareMathOperator*{\argmin}{\textbf{argmin}}

% This assumes your files are encoded as UTF8
\usepackage[utf8]{inputenc}

% This is not strictly necessary, and may be commented out,
% but it will improve the layout of the manuscript,
% and will typically save some space.
\usepackage{microtype}


% If the title and author information does not fit in the area allocated, uncomment the following
%
%\setlength\titlebox{<dim>}
%
% and set <dim> to something 5cm or larger.

%% Our macro
\newcounter{notecounter}
\newcommand{\enotesoff}{\long\gdef\enote##1##2{}}
\newcommand{\enoteson}{\long\gdef\enote##1##2{{
			\stepcounter{notecounter}
			{\large\textbf{ \hspace{1cm}\arabic{notecounter} $<<<$ ##1: ##2 $>>>$\hspace{1cm}}}}}}
% \enoteson
\enotesoff



\newcommand{\smallpm}{\scriptstyle\pm}
\newcommand{\tinypm}{\scriptscriptstyle\pm}

\newtcbox{\inlinepattern}{on line,colback=c0_new!10,colframe=white,size=fbox,arc=3pt, box align=base,before upper=\strut,
	top=-4pt, bottom=-4pt, boxrule=0pt}
\newtcbox{\pattern}{on line,colback=c0_new!10,colframe=white,size=fbox,arc=3pt, box align=base,before upper=\strut,
	top=-2pt, bottom=-2pt, boxrule=0pt}
\newtcolorbox{multipattern}{on line,colback=c0_new!10,colframe=white,size=fbox,arc=3pt, box align=base, top=-2pt, bottom=0pt, boxrule=0pt, before=\adjustbox{valign=c}\bgroup, after=\egroup, before upper=\strut}

\definecolor{c0}{cmyk}{1,0.3968,0,0.2588}
\definecolor{c0_new}{cmyk}{0.38,0.0,0.28,0.15}
\definecolor{c1}{cmyk}{0,0.6175,0.8848,0.1490} 
\definecolor{c2}{cmyk}{0.1127,0.6690,0,0.4431} 
\definecolor{c3}{cmyk}{0.6765,0.2017,0,0.0667} 
\definecolor{c4}{cmyk}{0.3081,0,0.7209,0.3255} 
\definecolor{c5}{cmyk}{0,0.8765,0.7099,0.3647} 
\definecolor{cwhite}{cmyk}{0,0,0,0}
\definecolor{darkgrey}{RGB}{180,180,180}
\definecolor{decentgrey}{RGB}{220,220,220}
\usetikzlibrary{calc,fit,positioning,arrows,intersections}
\newcommand\mask{\textit{MASK}}

\newcommand\ourmethod{MEAL\xspace}

\title{\ourmethod: Stable and Active Learning for Few-Shot Prompting}

\author[*$\diamond$]{Abdullatif Köksal}
\author[$\dag$]{Timo Schick}
\author[*$\diamond$]{Hinrich Sch\"utze}


\affil[*]{Center for Information and Language Processing, LMU Munich}
\affil[$\diamond$]{Munich Center of Machine Learning}
\affil[$\dag$]{Meta AI Research \protect\\
	\texttt{akoksal@cis.lmu.de}}


\begin{document}
\maketitle

\begin{abstract}
	Few-shot classification in NLP has recently made great
	strides due to the availability of large foundation models
	that, through priming and prompting, are highly effective
	few-shot learners. However, this approach has high variance
	across different sets of few shots and across different
	finetuning runs. For example, we find that validation
	accuracy on RTE can vary by as much as 27 points.  In this
	context, we make two contributions for more effective
	few-shot learning. First, we propose novel ensembling
	methods and show that they substantially reduce variance.
	Second, since performance depends a lot on the set of few
	shots selected, active learning is promising for few-shot
	classification. Based on our stable ensembling method, we
	build on existing work on active learning  and
	introduce a new criterion: inter-prompt uncertainty 
	sampling with diversity. We present the
	first active learning based approach to select
	training examples for prompt-based learning and show
	that it outperforms prior work on active learning. Finally,
	we show that our combined method, \ourmethod
	(\textbf{M}ultiprompt finetuning and prediction
	\textbf{E}nsembling with \textbf{A}ctive \textbf{L}earning), 
	improves overall performance of prompt-based finetuning
	by 2.3 absolute points on five different tasks.
\end{abstract}



\section{Introduction}


Accurate estimates of posterior probabilities are crucial for neural networks in various Natural Language Processing (NLP) tasks~\cite{icml17,DBLP:conf/nips/Lakshminarayanan17}. For example, it would be helpful for humans if the models deployed in practice abstain or interact when they cannot make a decision with high confidence~\cite{DBLP:journals/jamia/JiangOKO12}. While Pre-trained Language Models (PLMs) have improved the performance of many NLP tasks~\cite{bert,roberta}, how to better avoid miscalibration is still an open research problem ~\cite{calibration_emnlp20,dan_roth_emnlp21}. 
\begin{table}[t!]
    \centering
    \begin{tabular}{l|p{0.65\columnwidth}}
    \hline

    %  Example 1: & It is \hlc[cyan!10]{a} \hlc[red!40]{warm} \hlc[red!60]{funny} \hlc[red!40]{engaging} \hlc[cyan!20]{film} . \\ \hline
     Positive & a fast \hlc[green!10]{funny} \hlc[green!40]{highly} \hlc[green!80]{enjoyable} movie.\\ \hline
    %  like a south of the border melrose place
     
     Negative & It's about \hlc[red!5]{following} your \hlc[green!10]{dreams} \hlc[red!10]{no} matter \hlc[red!5]{what} your \hlc[green!5]{parents} think.\\
    \hline
  \end{tabular}
    \caption{Two motivating examples with highlight explanations~\cite{SST}. The saturation of the colors signifies the magnitude. The confidence of the model should be easily recognized by looking at token attributions.}
    % \vspace{-4mm}
    \label{tab:example-m}
\end{table}
In this paper, we investigate if and how model explanations can help calibrate the model. 

Explanation methods have attracted considerable research interest in recent years for revealing the internal reasoning processes behind models~\cite{IG,Uncertainty_Aware_Attention,deeplift}. Token attribution scores generated by explanation methods represent the contribution to the prediction~\cite{diagnostic}. Intuitively, one can draw some insight for analyzing and debugging neural models from these scores if they are correctly attributed, as shown in Table~\ref{tab:example-m}. For example, when the model identifies a highly indicative pattern, the tokens involved would have high attribution scores for the predicted label and low attribution scores for other labels. Similarly, if the model has difficulty recognizing the inductive information of any class (i.e., the attribution scores are not high for any label), the model should not be highly confident. As such, the computed explanation of an instance could indicate the confidence of the model in its prediction to some extent.
 
Inspired by this, we propose a simple and effective method named \textbf{CME} that can be applied at training time and improve the performance of the confidence estimates. The estimated confidence measures how confident the model is for a specific example. Ideally, reasonable confidence estimates should have higher confidence for correctly classified examples resulting in higher attributions than incorrect ones. Hence, given an example pair during training with an inverse classification relationship, we regularize the classifier by comparing the wrong example's attribution magnitude and the correct example's attribution magnitude.

Our work is related to recent works on incorporating explanations into learning. Different from previous studies that leverage explanations to help users predict model decisions~\cite{DBLP:journals/corr/abs-2102-02201} or improve the accuracy~\cite{DBLP:conf/icml/RiegerSMY20}, we focus on answering the following question: \textit{are these explanations of black-box models useful for calibration?} If so, how should we exploit the predictive power of these explanations? Considering the model may be uninterpretable due to the nature of neural networks and limitations of explanation method~\cite{Fragile,DBLP:conf/nips/YehHSIR19}, a calibrated model by CME at least can output the unbiased confidence. Moreover, we exploit intrinsic explanation during training, which does not require designing heuristics~\cite{xiye1} and additional data augmentation~\cite{mixup21acl}.
% Are these explanations useful for calibrating the model?

We conduct extensive experiments using BERT~\cite{bert} and RoBERTa~\cite{roberta} to show the efficacy of our approach on three natural language understanding tasks (i.e., natural language inference, paraphrase detection, and commonsense reasoning) under In-Domain (ID) and Out-of-Domain (OD) settings. CME achieves the lowest expected calibration error without accuracy drops compared with strong SOTA methods, e.g.,~\citet{mixup21acl}. When combined with Temperature Scaling (TS)~\cite{icml17}, the expected calibration errors are further reduced as better calibrated posterior estimates under these two settings.



\section{Related Work}

\paragraph{Inverse Rendering of Indoor Scenes} Inverse rendering attempts to reconstruct geometry and spatially-varying material and lighting information from monocular (which is our case) or multiple RGB images. Most previous methods only recognize one or part of the above attributes. Geometry reconstructions, including depth estimation and surface normal reconstruction, has been widely studied \cite{eigen2015predicting,liu2019planercnn}.
Most material reconstruction methods are only able to either estimate diffuse albedo~\cite{li2018cgintrinsics, barron2013intrinsic, karsch2014automatic} or classify material categories~\cite{bell2015material}.
For lighting estimation, recent deep learning methods have made progress in estimating global~\cite{gardner2017learning,gardner2019deep} and even spatially-varying~\cite{garon2019fast,song2019neural,li2020inverse} lighting conditions.
Recent works attempt to predict multiple intrinsics jointly by a holistic inverse rendering framework. Li et al.~\shortcite{li2020inverse} proposed a method to reconstruct disentangled geometry, spatially-varying reflectance and lighting from a single RGB indoor scene image.


\paragraph{Lighting Estimation and Relighting.}
Light estimation is one of the sub-tasks of inverse rendering. Most previous works ignore spatially-varying effects and predict a single environment map for the whole scene \cite{gardner2017learning,sengupta2019neural,munkberg2022extracting}. Indoor scenes suffer from spatial variations, thus recent work explores spatially-varying lighting estimation for indoor scenes. The representation of spatially-varying illumination includes environment maps, per-pixel spherical lobes~\cite{li2020inverse} (spherical Harmonics/Gaussians), or 3D voxel grids~\cite{wang2021learning}. Relighting is also a widely-studied relevant  task. \citet{griffiths2022outcast} leverages screen-space method to detect occlusion and cast shadows to relight an outdoor image. \citet{li2022physically} proposed a novel pipeline to modify the light conditions within an indoor scene.


\paragraph{Neural Scene Representations.} 
Neural representations are a rapidly growing area of research. Recent advances include  voxels~\cite{yu2021plenoxels,sun2021direct}, hashgrids~\cite{muller2022instant}, point clouds~\cite{aliev2020neural}, and neural implicit functions~\cite{mildenhall2020nerf,wang2021neus,yariv2021volume,yariv2020multiview}. 
Neural radiance fields (NeRFs)~\cite{mildenhall2020nerf} represents scenes as neural implicit functions, encoding a scene as a continuous volumetric radiance field of color and density. With volume rendering, a NeRF can synthesize novel view images with promising results. Our proposed method uses a NeRF as the representation of the out-of-view area of the scene (Sec.~\ref{sec:background}).

\paragraph{Differentiable Rendering.} A number of recent inverse rendering works utilize differentiable rendering to recover complex light transport effects. Some recent works have proposed general-purpose physically-based differentiable renderers~\cite{Li:2018:DMC,NimierDavidVicini2019Mitsuba2}. \citet{Zhang:2020:PSDR} and \citet{Zeltner2021MonteCarlo} discussed a rigurous theory of differentiable light transport and Monte-Carlo combinations. These physically-based methods achieve high-quality global illumination effects at the cost of substantial performance overhead. Some differentiable rendering techniques are customized for specific purpose such as texture~\cite{nimier2021material}, split-sum lighting and mesh extraction~\cite{munkberg2022extracting}. Our method designs a Monte-Carlo based in-network differentiable rendering layer to recover the appearance of indoor scenes (Sec.~\ref{sec:render}).

\paragraph{Indoor Scene Datasets.} 
Supervised learning requires a large database of indoor scene images and their corresponding ground truth geometry, material, and lighting for network training. Datasets include 3D shape models~\cite{chang2015shapenet}, real-world scans~\cite{chang2017matterport3d, dai2017scannet}, and scene datasets~\cite{song2017semantic,savva2017minos,li2018interiornet,li2021openrooms}, which can be classified as either real or synthetic data. Real datasets provide real-world images and geometry, while synthetic datasets provide arbitrary scene annotations for inverse rendering, some of which, such as materials and illumination, are difficult to acquire from real world. To the best of our knowledge, InteriorNet~\cite{li2018interiornet} and OpenRooms~\cite{li2021openrooms} are so far the highest-quality public indoor datasets with spatially-varying photorealistic material and illumination annotations. Unfortunately, InteriorNet provides only LDR results, while OpenRooms provides only lighting information on the scene surface (instead of at any 3D location), and lacks the complexity of material and furniture variations. We present a new indoor scene HDR dataset to tackle their shortcomings.

	

\section{Multiprompt-based Finetuning}
\label{sec:multiprompt}
Let $M$ be a masked language model, $T$ its
vocabulary, and $\textit{MASK} \in T$ the mask token. We use
Pattern-Exploiting Pattern (PET)
\cite{schick-schutze-2021-exploiting}
for our prompt-based finetuning experiments on few-shot  classification, but without knowledge distillation and unlabeled data. Patterns ($\mathcal{P}$) transform a given input $x$ into cloze-style phrase $x_p$, containing a single mask. Verbalizers ($V$) convert each label $l\in L$ into a single token $s_l \in T$ representing the task-specific meaning of the output label. 

Our prediction for a label is its probability,
according to the language model,
as a substitution for the mask:
\begin{equation}
\label{eq:pet}
P(y|x) = \frac{\exp
s_m(V(y)|x_p)}{\sum_{y^*\in L}\exp  s_m(V(y^{*})|x_p)}
\end{equation}
where $s_m$ gives the raw score of $V(y)$ from a pretrained language model $M$ for the \textit{MASK} position in the cloze-style phrase of the input.

Using the cross-entropy loss of $P$, PET trains a separate
model for each prompt (i.e., single prompt finetuning). In inference, it ensembles model predictions by logit averaging. 

We propose \textbf{multiprompt-based finetuning} with a
single model M that is trained with all prompts for a given
task. During inference time, we also use ensembling with
logit averaging for each prompt. However, our approach generates 
a single finetuned model regardless of the number of
prompts. Compared to PET, this reduces runtime, memory, and
overall complexity. Our results
suggest that multiprompt training is better than or
comparable to single-prompt training.
See also \cite{schick-schutze-2021-shot} where 
similar behavior was observed for text
generation.

\section{Stability of Few-Shot Classification}

\begin{figure*}
\centering
\includegraphics[width=\linewidth]{Figures/loss_surface_form_equal.pdf}
\caption{Loss and validation accuracy surface visualizations
  for two RTE runs; the training set is the same for the two runs.
Left (training loss):
The two models
$\theta_s$ and $\theta_f$
have similar loss
-- they are both located in the upper right blue zero-loss triangle.
Right (validation accuracy):
The successful model $\theta_s$ performs much better than the failed model
$\theta_f$.
}
\label{fig:loss_surface_form}
\end{figure*}

\label{sec:stability}
In few-shot classification, finetuning PLMs such as
ALBERT \cite{Lan2020ALBERT} with  an MLM objective on samples converted into cloze-style phrases \cite{schick-schutze-2021-just} 
achieves comparable performance to prompting with much larger models like GPT-3
\cite{gpt3}. 
Just as prompting methods are sensitive to data order
\cite{lu-etal-2022-fantastically} and label distributions
\cite{pmlr-v139-zhao21c}, finetuning PLMs also exhibits
sensitivity and instability as shown by
\citet{DBLP:journals/corr/abs-2002-06305} for a fully supervised setting.

We show that the instability of finetuning PLMs also exists
in few-shot prompt finetuning. Even though prompt finetuning
does not introduce new parameters like classifier heads as
in fully supervised classification, there is variance from
dropout and training data order. We conduct experiments with
multiprompt-based finetuning with default PET
\cite{schick-schutze-2021-exploiting} settings without
knowledge distillation. In Figure \ref{fig:instability}, we
show that runs with different random seeds for the same
training set can vary by as much as 13
points.

\citet{DBLP:conf/iclr/MosbachAK21} suggest that longer
training with a low learning rate and warmup improves the
stability of finetuning PLMs.  Their main motivation is to
avoid models ending up in suboptimal training loss
regions. However, this is not valid in few-shot prompt
tuning as the number of training examples is low, and
finetuning achieves almost zero training loss quickly. Our
initial experiments show that longer training does reduce
the standard deviation between different runs, but that it
also causes lower accuracy for most tasks. For example, it
reduces the average standard deviation by more than 70\%
for both single and multi-prompt finetuning while causing
many datasets to have lower mean accuracy by up to 7.3 points
as shown in Table \ref{tab:stability_results}.

In Figure \ref{fig:loss_surface_form},
we analyze this instability by creating a training
loss and validation accuracy surface visualization of two
RTE runs with \emph{the same training set} and multiprompt-based
finetuning. The failed model
$\theta_f$ (red)
achieves 58.5\% validation
accuracy while the successful model
$\theta_s$ (green)
achieves 71.5\%. The two
models only differ by random seed of finetuning. The figure
illustrates the training loss and validation accuracy
surfaces for combinations of the model weights of the
pretrained model ($\theta_p$), the failed model
($\theta_f$), and the successful model ($\theta_s$). We create a two-dimensional space based on $f(a, b) = F(\theta_p + a\delta_f + b\delta_s)$, where $\delta_f=\theta_f-\theta_p$, $\delta_s=\theta_s-\theta_p$, and $F$ is the loss or accuracy function, depending on the graph. We use 16 values for both $a$ and $b$ to plot training loss and validation accuracy surface forms.

Figure \ref{fig:loss_surface_form} shows that there
is a large region with $\leq1e-4$ training loss (left
graph, dark blue) that includes
$\theta_f$ and $\theta_s$. However, most of this region is
suboptimal in terms of validation accuracy (right
graph). This indicates that our instability problem differs
from fully supervised finetuning where large learning
rates often result in suboptimal training loss while we observe almost 0 training loss for each run including failed ones. Therefore, longer training with a low learning rate and warmup only makes finetuned models end up in a similar region with lower variance, but it causes suboptimal validation accuracy scores, as further analysis in Section \ref{sec:res_instability} suggests.

To overcome the instability issue, we propose
two ensemble models: we ensemble the logits of
different runs in \textbf{ENSEMBLE}\textsubscript{Prediction}, and we take the average of parameters of different runs in \textbf{ENSEMBLE}\textsubscript{Parameter}. We will show that these reduce the effect of failed runs while achieving a higher accuracy score than the mean of different runs for five tasks. Our experiments show that these ensembles yields better or comparable performance than average accuracy both for single and multiprompt finetuning. The final prediction of \textbf{ENSEMBLE}\textsubscript{Prediction} for input $x$ is:
\[P(y|x) = \sigma(\sum\limits_{r=1}^{R}[\sum\limits_{p\in\mathcal{P}}F_{r}(y|x_{p})]/(R*|\mathcal{P}|))\]
where $\sigma$ is softmax, R is the number of runs, $\mathcal{P}$ is the
set of prompts, and $F_{r}$ gives,
for the finetuned model in run $r$,
the logit of each class for the input $x$ with  prompt
$p$.

Following  recent work on averaging of deep neural
networks \cite{izmailov2018averaging},
we  average  each parameter of the
finetuned language models across runs, resulting in a single
averaged model.
The prediction of
\textbf{ENSEMBLE}\textsubscript{Parameter}
for input $x$ is the prediction of this single model.

\section{Active Learning}
\label{sec:active_learning}

\begin{algorithm*}
	\caption{Inter-prompt uncertainty sampling with diversity}\label{alg:algorithm}
	\begin{algorithmic}[1]
		\Require Pretrained Language Model $M$, Unlabeled Pool
		$\mathcal{U}$, Size of Training Set $\mathcal{N}$, Pattern Set $\mathcal{P}$
		\State Compute logits representation for each sample in the unlabeled pool, $L(u)$ by concatenating each pattern's logits: $M(X=x_p)$ for $x\in \mathcal{U}$ and $p\in \mathcal{P}$  
		\State Train k-means from logits representation with 8 clusters
		\For{\texttt{$iter=1,2,...,1000$}}
		\State Select $\mathcal{N}$ samples, uniformly distributed in $8$ clusters with random seed $iter$, called $X_{iter}$
		\State $\texttt{Scores}_{iter} = \sum\limits_{x_i\in X_{iter}}\sum\limits_{(p,q)\in \mathcal{P} \times \mathcal{P}}\mbox{KL}(P(y|x_{i,p}, M)||P(y|x_{i,q},M))$
		\EndFor
		\State $\texttt{MostUncertainIteration} = \underset{iter=1,2,...,1000}{\mathrm{\textbf{argmax}}} \texttt{Scores}_{iter} $
		\State \textbf{return} $X_{\texttt{MostUncertainIteration}}$
	\end{algorithmic}
\end{algorithm*}

Another important source of variance for few-shot
classification is the selection of training examples. Figure
\ref{fig:instability} shows the effect of the selected examples: the validation accuracy significantly varies in all tasks. For example, the difference is up to 13.7 points for RTE and 6.7 points for MRPC.

We adapt active learning algorithms for few-shot
prompt-based finetuning; we select all training examples at
once.
In the first step, we use a PLM without finetuning
to get contextual embeddings, logits, and probabilities for
each sample in \emph{a zero-shot setting}.
(We exploit here that, due to the cloze-style format, PLMs can make predictions
before any finetuning step.)
In the second step, we apply modified active learning
algorithms for prompts. We select all examples at once to simplify the selection process. For each task, we select $16*L$ training samples where $L$ is the number of labels.

\subsection{Active Learning Algorithms}
\textbf{Random} selection draws random samples from an unlabeled pool. We report random selection results with five different seeds.

\noindent\textbf{Entropy} \cite{entropy} finds the probability entropy of each sample by summing the entropy of each prompt. We then select samples with highest entropy scores.
\begin{equation*}
	\resizebox{\hsize}{!}
	{
		$\mbox{s}(x_i) = \sum \limits_{j=1}^{j=L}\sum\limits_{p\in \mathcal{P}}-P(y=l_j|x_{i,p})\ln P(y=l_j|x_{i,p})$
	}
\end{equation*}
where $L$ is the number of labels, $\mathcal{P}$ is the set of prompts, and $x_{i,p}$ is the input $x_{i}$ with pattern $p$.

\noindent\textbf{Breaking Ties (BT)} \cite{breaking_ties} selects samples with minimum difference between the highest two probability classes.
\[
\mbox{bt}(x_i) = 
\sum\limits_{p\in \mathcal{P}}P(y=l_{1}|x_{i,p})-P(y=l_{2}|x_{i,p})
\]
where $l_{1}$ and $l_{2}$ are the labels with highest
and second highest probability for $x_{i,p}$.

\noindent\textbf{Lowest Confidence
  (LC)} \cite{lowest_confidence} calculates
\mbox{lc} as
the sum of probability scores for the predicted
class for each prompt. Then, it selects samples with lowest
\mbox{lc} scores.
Lowest confidence algorithm produces the same order as breaking ties algorithm when the number of labels is two.
\[\mbox{lc}(x_i) = \sum\limits_{p\in \mathcal{P}}\max\{P(y=l_j|x_{i,p}):j=1..L\}\]

Specifically for prompt-based learning, we propose the
\textbf{KL Divergence} selection
algorithm. For each sample, we calculate
$\mbox{kl}(x_i)$
as the sum
of KL divergence scores of probabilities from prompt pairs
and then select samples with the highest
$\mbox{kl}(x_i)$.
This gives a high score to a sample
for which there is a high
variance in the model's predictions across the different
prompts.
By including such a high-scoring sample, we want
the model to learn more information from different prompts
for a given sample.
\[
\mbox{kl}(x_i)
= \sum\limits_{(p, q) \in \mathcal{P} \times \mathcal{P}}\mbox{KL}(P(y|x_{i,p})||P(y|x_{i,q}))
\]
\noindent\textbf{Contrastive Active Learning (CAL)}
\cite{cal} selects samples with the highest KL divergence
between the sample and its $M$ nearest
neighbors in the PLM contextual embedding space.
\[
\mbox{cal}(x_i) = \sum\limits_{m=1}^{m=M}\sum\limits_{p\in\mathcal{P}}\mbox{KL}(P(y|x_{m,p})||P(y|x_{i,p}))
\]

\noindent\textbf{Batch Active learning by Diverse Gradient
  Embeddings (BADGE)} \cite{badge} represents samples with
the gradient of the cross entropy loss, conditioned on the one-hot
encoding of the predicted label, with respect to the parameters of
the final (output) layer. For prompt-based finetuning, we
concatenate the gradient vectors of each prompt for a given
sample by using the decoder of the masked language model head as
the final layer. Then we find as many cluster centers
as the number of training samples via kmeans++ algorithm with gradient vectors.
Then corresponding sample for each cluster is selected as a training example.
As k-means++  depends on its random initialization,
we report BADGE results with
five different random seeds.

\noindent\textbf{Inter-prompt uncertainty sampling with
  diversity (IPUSD)} is our active learning
algorithm that combines uncertainty and diversity sampling
for prompt-based finetuning and focuses on the prediction variance of PLMs when
different prompts are used. As described in Algorithm
\ref{alg:algorithm}, it first
represents each sample $x$ as a vector of dimensionality
$|\mathcal{P}|\cdot|L|$, the concatenation
of the $L$ logits for $x$ for each of the patterns in
$\mathcal{P}$ -- where
L1 = line 1. We
utilize logits as representations because they can represent
the model's probability distribution, certainty, and
divergence across different prompts. Based on these
representations, we run k-means clustering with 8
clusters (L2). In the following iteration,
we sample a  training set,
uniformly distributed from the 8 clusters (L4). The KL
divergence of a sample is calculated by summation of the KL
divergence score of each prompt pair (L5); this indicates
the divergence of probability distributions across
prompts.
We repeat
the iteration loop 1000 times 
and then select 
the most uncertain training
set based on
inter-prompt uncertainty sampling with
  diversity.
We select based on 1000 iterations (as opposed to finding the most uncertain
samples from each cluster) to ensure a balance between randomization
and uncertainty. Otherwise, our initial experiments suggest that this
strategy would only focus on outlier samples with suboptimal performance.
As  k-means  and sampling depend on the
random seed, we repeat this algorithm five times during
experimentation (as for random and BADGE).


\section{Experiments and Results}
\label{sec:experiments}
\subsection{Setup}
\label{sec:setup}
We use a diverse set of classification tasks to analyze the
stability of prompt tuning, compare single to
multiprompt-based finetuning, and evaluate active learning
algorithms. For each dataset, we use four prompts, described
in detail in \S\ref{sec:datasets}. We report results on the
validation set as we conducted all experiments without 
hyperparameter tuning by assuming a realistic few-shot
scenario in which no dev set is available for tuning.\footnote{Our initial experiments
	show the same patterns for all
	experiments with similar performance gains for the test set.}
For single prompt- and multiprompt-based finetuning,
we use the default values from PET
\cite{schick-schutze-2021-exploiting}. For longer training,
we compare the default values from PET to adapted settings
from \citet{DBLP:conf/iclr/MosbachAK21} for stability.
The default values of PET have a 1e-5 learning
rate for 10 epochs without warmup while the settings for
stability have a 1e-6 learning rate for 50 epochs with
linear scheduled warmup with a 0.1 ratio.

\begin{table*}
	\centering
	\setlength\tabcolsep{4.75pt}
	\begin{tabular}{lccccccc}
		\toprule
		\textbf{Finetuning} & \textbf{Stability Technique} & \textbf{RTE} & \textbf{SST-2} & \textbf{SST-5} & \textbf{TREC} & \textbf{MRPC} & \textbf{Average} \\
		\cmidrule(lr){1-2} \cmidrule(lr){3-7} \cmidrule(lr){8-8} 
		
		\multirow{4}{*}{Single Prompt} & - (Default) & 63.7$\tinypm$\tiny{1.91} &93.1$\tinypm$\tiny{0.54} &51.6$\tinypm$\tiny{0.73} &81.3$\tinypm$\tiny{1.11} &67.7$\tinypm$\tiny{1.07} &71.5$\tinypm$\tiny{1.07} \\
		& Longer Training &  63.4$\tinypm$\tiny{0.42} &92.1$\tinypm$\tiny{0.10} &50.2$\tinypm$\tiny{0.20} &74.0$\tinypm$\tiny{0.45} &67.0$\tinypm$\tiny{0.20} &69.3$\tinypm$\tiny{0.27} \\
		& ENSEMBLE\textsubscript{Parameter} & 64.0$\tinypm$\tiny{1.35} &93.2$\tinypm$\tiny{0.27} &51.9$\tinypm$\tiny{0.40} &80.6$\tinypm$\tiny{0.56} &67.5$\tinypm$\tiny{0.70} & 71.4$\tinypm$\tiny{0.66}\\
		& ENSEMBLE\textsubscript{Prediction} & 64.0$\tinypm$\tiny{0.95} &93.2$\tinypm$\tiny{0.30} &52.0$\tinypm$\tiny{0.51} &81.9$\tinypm$\tiny{0.72} &67.9$\tinypm$\tiny{0.58} &71.8$\tinypm$\tiny{0.61} \\
		\midrule
		\multirow{4}{*}{Multiprompt} & - (Default) & 64.2$\tinypm$\tiny{4.88} &91.7$\tinypm$\tiny{1.55} &52.1$\tinypm$\tiny{0.92} &82.9$\tinypm$\tiny{1.77} &67.9$\tinypm$\tiny{2.25} &71.8$\tinypm$\tiny{2.28}\\
		& Longer Training & 66.2$\tinypm$\tiny{0.45} &93.0$\tinypm$\tiny{0.08} &50.4$\tinypm$\tiny{0.26} &79.0$\tinypm$\tiny{0.34} &66.8$\tinypm$\tiny{0.36} &71.1$\tinypm$\tiny{0.30} \\
		& ENSEMBLE\textsubscript{Parameter} & 64.4$\tinypm$\tiny{2.53} &92.5$\tinypm$\tiny{0.31} &52.9$\tinypm$\tiny{0.47} &83.7$\tinypm$\tiny{1.00} &68.3$\tinypm$\tiny{1.34} & 72.4$\tinypm$\tiny{1.13}\\
		& ENSEMBLE\textsubscript{Prediction} & 65.4$\tinypm$\tiny{2.86} &92.4$\tinypm$\tiny{0.43} &52.8$\tinypm$\tiny{0.42} &84.4$\tinypm$\tiny{0.87} &68.9$\tinypm$\tiny{1.03} & \textbf{72.8}$\tinypm$\tiny{1.12}\\
		\bottomrule
	\end{tabular}
	\caption{Comparing different stability techniques for prompt-based finetuning with single and multiple prompts with ALBERT  on randomly selected training sets. Multiprompt finetuning improves the overall performance compared to PET. ENSEMBLE\textsubscript{Prediction} improves stability while achieving higher performance for both finetuning techniques.}
	\label{tab:stability_results}
\end{table*}

\subsection{Datasets and Evaluation Metrics}
\label{sec:datasets}
\textbf{RTE} \cite{rte} is a textual entailment dataset that contains text pairs of premise and hypothesis with the objective of detecting entailment and contradiction. For a premise-hypothesis pair ($p$, $h$), we use

\[\pattern{$h$\textsf{\small?$\,|\,$\mask{}, }$p$}, \pattern{\textsf{\small``}$h$\textsf{\small''?$\,|\,$\mask{}, ``}$p$\textsf{\small''}},\]
\[\pattern{$h$\textsf{\small?$\,|\,$\mask{}. }$p$},
\pattern{\textsf{\small``}$h$\textsf{\small''?$\,|\,$\mask{}. ``}$p$\textsf{\small''}} \]


\noindent patterns and a verbalizer \textsf{\small yes} and \textsf{\small no} for entailment and no entailment labels.


\textbf{SST-2} and \textbf{SST-5} \cite{sst} is a sentiment
analysis dataset of movie reviews. SST-2 contains two
classes: positive and negative. SST-5 has five labels;
very positive, positive, neutral, negative, and very
negative. For a given movie review $t$, we use
\[\pattern{$t$ \textsf{\small It was \mask{}.}}, \pattern{$t$ \textsf{\small A \mask{} one.}},\]
\[\pattern{$t$ \textsf{\small The movie is \mask{}.}},
\pattern{$t$ \textsf{\small All in all \mask{}.}}\]
\noindent patterns and verbalizers ``great'', ``terrible''
(SST-2) and  ``great'', ``good'', ``okay'', ``bad'',  ``terrible'' (SST-5).

\textbf{TREC} \cite{trec} is a question classification dataset. We use six coarse classes: abbreviation, entity, description, human, location, numeric value. For a  question $t$, we use
\[ \pattern{\textsf{\small[Question Category: \mask{}] }$t$}\text{,} \pattern{\textsf{\small[Category: \mask{}] }$t$},\]
\[ \pattern{$t$ \textsf{\small This question is related to \mask{} category.}},\]
\[ \pattern{\textsf{\small I'd like to ask a question about \mask{}. } $t$}\]

\noindent patterns and verbalizers ``abbreviated'', ``entity'', ``description'', ``human'', ``location'', ``number''.

\textbf{MRPC} \cite{mrpc} is a paraphrase identification dataset of sentence pairs. For two sentences $(t_1, t_2)$, the task is to decide whether they are semantically equivalent or not. We use the same pattern and verbalizer as for RTE.


We report the average of accuracies, run standard
deviations, and training set standard deviations for a given
dataset and active learning algorithm over five training
sets and five runs for each training set. For AL algorithms
without variance (e.g., entropy and KL), we do not report
training set standard deviation as the algorithm outputs a
single training set. We increase the number of runs
from 5 to 20 for stability experiments and report run variance over
20 runs for default and longer training. For ENSEMBLE techniques,
we report variance over 4 trials where we ensemble 5 different runs
with the given technique.


For active learning, we treat training examples of each dataset as unlabeled
data and pick few-shot samples from there. We report the
average accuracy of five datasets and the average ranking of
active learning algorithms for each dataset following
\citet{schroder-etal-2022-revisiting} with additional
analysis of diversity \cite{zhdanov2019diverse}, representativeness \cite{dor2020active},
and label entropy \cite{prabhu-etal-2019-sampling} as explained in Section \ref{sec:al_additional_analysis}.

\subsection{Stability}
\label{sec:res_instability}
We analyze both single and multiprompt-based finetuning with
the proposed solutions for the stability experiments over five randomly 
selected training sets. We
compare
ENSEMBLE
with the default setup (default hyperparameters given in
\S\ref{sec:setup}) and 
with \citet{DBLP:conf/iclr/MosbachAK21}'s proposal:
longer training with a lower learning rate and warmup
training.

The results
in Table \ref{tab:stability_results}
suggest that \citet{DBLP:conf/iclr/MosbachAK21}'s
longer training approach   reduces the run standard
deviation for each dataset, but causes suboptimal accuracy
results for SST-5, TREC, and MRPC in multiprompt-based
finetuning, and for all datasets in single prompt-based
finetuning. Therefore, using a longer training approach is
not feasible for practical scenarios; for example, even the
worst runs in the default approach achieve better results than every run in the longer training approach for TREC.

Table \ref{tab:stability_results} shows that
ENSEMBLE\textsubscript{Prediction} consistently produces
better results than default settings for each dataset both
for single prompt and multiprompt finetuning. On the other
hand, ENSEMBLE\textsubscript{Parameter} performs better than
default settings only in multiprompt finetuning
consistently, but speeding up the prediction process during
inference time with a single model.  On top of that, both
ENSEMBLE techniques avoid failed runs. For example, the
default approach gets 66.1\% average accuracy for RTE with
one of the five random training sets. However, two runs
result in 61.7\% and 64.3\% validation accuracy, and
ENSEMBLE\textsubscript{Prediction} ensures that we get
better average accuracy (67.9\%) and it is more useful in
practical scenarios as it will not end up with suboptimal
runs. ENSEMBLE\textsubscript{Parameter} is an alternative
approach to increase stability and performance while
providing a single model and lower time complexity during inference
time.


\begin{table*}
	\centering
	\begin{tabular}{lccccc}
		\toprule
		\textbf{Algorithms} & \textbf{RTE} & \textbf{SST-2} & \textbf{SST-5} & \textbf{TREC} & \textbf{MRPC} \\
		\midrule
		Random & 65.3$\pm$5.8 & 92.1$\pm$2.4 & 52.8$\pm$\textbf{0.7} & 83.8$\pm$2.3 & 69.3$\pm$2.7\\
		\midrule
		Entropy & \underline{71.1} & 89.3 & 49.0 & 76.2 & 68.9 \\
		Lowest Confidence & \textbf{\underline{71.8}} & 91.4 & 48.8 & 72.6 & \underline{70.1}\\
		Breaking Ties &\textbf{\underline{71.8}} & 91.4 & 49.9 & 77.2 & \underline{70.1} \\
		KL (Ours) & 59.6 & 89.8 & \textbf{\underline{53.5}} & 77.4 & 65.4 \\
		Contrastive AL \cite{cal} & 56.7 & \underline{92.9} & 49.0 & 81.6 & \textbf{\underline{71.8}}\\
		BADGE \cite{badge} & \underline{68.7}$\pm$8.9 & \textbf{\underline{93.2}}$\pm$\textbf{\underline{0.8}} & 51.2$\pm$2.8 & 82.7$\pm$3.1 & \underline{70.3}$\pm$\textbf{\underline{0.9}}\\

		IPUSD (Ours) & \underline{70.1}$\pm$\textbf{\underline{3.8}} & \underline{92.9}$\pm$\textbf{\underline{0.9}} & 51.4$\pm$1.4 & \textbf{\underline{85.0}}$\pm$\textbf{\underline{2.2}} & \underline{69.8}$\pm$3.3 \\
		
		\bottomrule
	\end{tabular}
	\caption{Comparison of active learning methods. Random,
	  BADGE, IPUSD (inter-prompt uncertainty sampling
                with diversity)
		are non-deterministic. We run these algorithms for five
		different random seeds and
		report average accuracy and standard deviation. Best results are indicated in bold, results better than random are underlined.}
	\label{tab:al_results}
\end{table*}


\subsection{\mbox{Single and Multiprompt-based Finetuning}}
\label{sec:res_single_multi}
In Table \ref{tab:stability_results}, we also compare single
prompt to multiprompt-based finetuning.
For the three stability techniques
(longer training, ENSEMBLE\textsubscript{Parameter}, ENSEMBLE\textsubscript{Prediction}),
multiprompt finetuning achieves better average accuracy than
single prompt tuning for almost all datasets. Also,
multiprompt-based finetuning in the default setup produces a higher overall
accuracy with a higher run standard deviation but ENSEMBLE techniques overcome
this. Thus, multiprompt achieves better overall performance than
single prompt for each stability technique. Furthermore, multiprompt simplifies 
training and deployment in practice because it outputs a single finetuned model,
compared to
one model per prompt for the single prompt method.


\subsection{Active Learning}


We compare our active learning algorithms with a variety of
uncertainty and diversity-based algorithms in
Table \ref{tab:al_results}. To provide more stable results and fair comparison by reducing noise from different runs, we employ \emph{multiprompt finetuning with ENSEMBLE\textsubscript{Prediction}} for each active learning algorithm in this section.
Our results show that all uncertainty-based
algorithms -- entropy, lowest confidence, breaking ties and
KL -- perform worse than random selection in terms of 
accuracy averaged over five datasets.
Our interpretation of this result is that, considering that we are
finetuning a PLM with few examples, finetuning with the
highest uncertainty examples is not enough to generalize for
the task. In contrast,
\citet{schroder-etal-2022-revisiting} found that
uncertainty-based active learning performs
consistently better than random selection for
fully supervised settings in PLMs.



More complex active learning algorithms that combine
uncertainty and diversity sampling such as CAL \cite{cal}
and BADGE \cite{badge} perform better than uncertainty-based
algorithms. Furthermore, BADGE outperforms random selection
on three out of five datasets in terms of average accuracy.
However,
BADGE has higher standard deviation than random selection: its average
over the five datasets is 3.0 compared to an average of 2.5
for random selection.

Finally, our proposed algorithm, IPUSD, performs
better than random and better than other active learning algorithms
with higher average accuracy and lower standard
deviation across different selections on five datasets.
However, we perform a more in-depth analysis of active
learning algorithms to understand their relative
performance better to understand failure cases
like SST-5. We believe that these insights
will lead to improved active learning strategies in future
work.


\subsubsection{Additional Analysis}
\label{sec:al_additional_analysis}
To
provide a more in-depth analysis of AL algorithms,
we compare three aspects of active learning selection:
\emph{diversity}, \emph{representativeness} and \emph{label entropy}.

\textbf{Diversity} \cite{zhdanov2019diverse} aims to identify whether selected training examples are redundant or similar. Following \citet{dor2020active}'s interpretation, we calculate the diversity of selected examples as the reciprocal of the average Euclidean distance between each unlabeled sample and the nearest training example. The representation of each sample is an embedding of the [CLS] token from the corresponding finetuned model with selected training examples. Therefore, a higher score means that training examples are selected from a diverse area by ensuring a smaller average distance to unlabeled example space.

\textbf{Representativeness} \cite{dor2020active} is a metric that refers to the well-known issue of selecting outlier examples in AL, especially in cases of uncertainty-based algorithms. It is calculated as the reciprocal of the average distance between selected training examples and their nearest K (10) neighbors from unlabeled examples. A higher score means that a training example is more likely to be located in a more densely populated region and therefore less likely to be an outlier.

\textbf{Label Entropy} \cite{prabhu-etal-2019-sampling} is the KL divergence of the class distribution difference between unlabeled data and selected training examples. 



\begin{table}[t]
	\small
	\setlength\tabcolsep{3.6pt}
	\centering
	\begin{tabular}{lccccc}
		& {\textbf{Acc. $\uparrow$}} & {\textbf{Rank $\downarrow$}} & {\textbf{Div. $\uparrow$}} & {\textbf{Repr.$\uparrow$}} & {\textbf{Ent.$\downarrow$}} \\
	    \cmidrule(lr){2-3} \cmidrule(lr){4-6}
		Random & 72.6{$\tinypm$\tiny{2.8}} & 4.0 & \textbf{13.6} & \textbf{17.6} & \textbf{2.0} \\
		\midrule
		Entropy & 70.9 & 6.4 & 13.3 & 16.9 & 6.1 \\
		LC & 70.9 & 5.6 & 13.5 & 17.2 & 5.3  \\
		BT & 72.1 & 4.0 & 13.4 &  17.1 & 5.6 \\
		KL & 69.1 & 5.6 & 13.4 & 16.9 & 9.0 \\
		CAL  & 70.4 & 4.4 & 13.1 & 17.1 & 23.5 \\
		\midrule
		BADGE & 73.2$\tinypm$\tiny{3.3} & \textbf{3.0} & \textbf{13.6}  & \textbf{17.6} & 2.2 \\
		\textbf{IPUSD} & \textbf{73.9}$\tinypm$\tiny{\textbf{2.3}} & \textbf{3.0} &13.5 & \textbf{17.6} & \textbf{2.0} \\
		\bottomrule
	\end{tabular}

\caption{Comparing average accuracy, rankings, diversity, representativeness, 
	and label entropy scores for 7 active learning algorithms.
	IPUSD outperforms random and other
	algorithms by providing more balanced selections of three aspects. Uncertainty-based
	active learning algorithms have worse scores for each feature, and thus lower performance.
	Arrows indicate whether higher ($\uparrow$)
	or lower ($\downarrow$) is better.}  \label{tab:al_ranking}
\end{table}


Table \ref{tab:al_ranking} shows that our proposed active
learning algorithm outperforms all other active learning
algorithms with higher average accuracy, better ranking
among all algorithms, and lower standard
deviation across different selections. Inter-prompt
uncertainty sampling with diversity
(IPUSD) gets better performance than random by 1.3\%
(73.9-72.6) average absolute improvement.


Further analysis shows uncertainty-based active learning algorithms usually have higher label entropy values which indicate different class distribution than the unlabeled data distribution in addition to lower representativeness and diversity scores. Random selection provides a strong baseline in terms of three aspects. While BADGE and IPUSD have similar scores for three features, a slight change in label entropy (i.e., selecting a training set with different distribution than unlabeled data) can cause significant changes in overall performance. Therefore, our algorithm, IPUSD, achieves the best balance between diversity, representativeness, and label entropy features while achieving higher performance thanks to the diversity, uncertainty, and randomness aspect of our algorithm. We observe that designing a training data selection strategy for few-shot finetuning is a more challenging task than designing an active learning strategy for the fully-supervised setting as these algorithms can only rely on zero-shot information gathered from PLM. Therefore, we share our insights on the limitations of IPUSD in SST-5 and MRPC datasets in the next section to guide future work in this field.



\subsubsection{IPUSD: Underlying assumptions}
\label{sec:al_limitations}
Table \ref{tab:al_results} shows that IPUSD performs worse than random only on the SST-5 dataset. Our initial analysis with diversity, representativeness, and label entropy features does not show any difference between random and IPUSD on SST-5 but further manual investigation shows that IPUSD selects examples that might belong to multiple classes. As the distinction between classes is not fully clear in this dataset (e.g., very negative and negative), selecting challenging examples as a training set may result in suboptimal finetuned models. To test this, we finetuned a fully-supervised RoBERTA\textsubscript{LARGE} model on a fine-grained sentiment analysis task with YELP \cite{zhang2015character}. This model achieves $47.7\%$ average accuracy on selected examples with random while achieving $39.2\%$ accuracy on selected examples with IPUSD which shows that selected examples with IPUSD are more challenging (i.e., not clear which class they belong to). Therefore, it might not be optimal to use IPUSD when the distinction between labels is not clear. 

Thus, IPUSD makes the assumption that discrimination between
classes can be learned well. If that is not the case, then
it underperforms.

Table \ref{tab:al_results} also illustrates a rather small
improvement in MRPC with a higher standard deviation than
random. We observe that unlabeled data of MRPC have a
non-uniform distribution of labels as $68\%$ belong to the
equivalent class while $32\%$ belong to the non-equivalent
class. Usually, IPUSD selects training examples similar to
the original distribution because of its clustering
mechanism with zero-shot logits as suggested by the label
entropy score; however, IPUSD failed to select samples with
similar distribution in one of the five selections. That
selection has a $(53\%, 47\%)$ distribution of equivalent and
non-equivalent classes, therefore it achieves lower accuracy
($64.2$) while the other four selections in IPUSD achieve
much higher average accuracy ($71.2\pm1.3$).

Thus, IPUSD makes the assumption that
selected training sets have a label distribution similar to
the overall distribution. If this assumption is not true, it
underperforms.



\begin{table}[t]
	\setlength\tabcolsep{3.2pt}
	\begin{tabular}{lcc}
		& {\textbf{ALBERT}} & {\textbf{RoBERTa}}\\
		\midrule
		\ourmethod & \textbf{73.9} & \textbf{72.7} \\
		\quad--\textbf{AL} (IPUSD) & 72.6 & 71.8 \\
		\quad--\textbf{E}NSEMBLE\textsubscript{Prediction} & 72.0 & 71.1 \\
		\quad--\textbf{M}ultiprompt & 71.6 & 70.7 \\
		
		\bottomrule
	\end{tabular}
	
	\caption{Ablation study of \ourmethod by comparing average performance over five tasks with ALBERT-xxlarge-v2 and RoBERTa\textsubscript{LARGE} with \textit{5} runs. Each row shows the cumulative performance drop when the relevant module is changed with the default setting (IPUSD to random, ENSEMBLE\textsubscript{Prediction} to default, Multiprompt to single prompt). }  \label{tab:ablation_study}
\end{table}


\subsection{Ablation Study of \ourmethod}
We summarize \ourmethod's performance with its modules
and test its robustness with different pretrained language
models. In addition to providing more stable results, Table
\ref{tab:ablation_study} shows that \ourmethod increases
overall performance by 2.3\% and 2.0\% points over default
prompt-based finetuning for ALBERT \cite{Lan2020ALBERT}
and RoBERTa$_{\text{LARGE}}$ \cite{roberta}, consecutively.
The active learning module of \ourmethod, IPUSD, gives the highest
performance improvement by $1.3$ and $0.9$ absolute points,
showing the potential of AL in few-shot learning. Following that,
ENSEMBLE\textsubscript{Prediction} improves the overall
performance by $0.6$ and $0.7$ points for ALBERT and RoBERTA 
on top of its improvement for stability as discussed in Section
 \ref{sec:res_instability}. Finally, we see an overall improvement
 with multiprompt finetuning by $0.4$ points for both LM on top
 of its improvement of model space complexity. Multiprompt
 finetuning reduces the number of models from the number of
 prompts to one during inference time. Finally, we see consistent
 improvements for each module over different language
 models which illustrate \ourmethod's robustness.		
	
	\section{Conclusion}
	\label{sec:conclusion}
We demonstrate two problems with stability in few-shot
classification with prompt-based finetuning -- instability due
to finetuning proper and due to training set selection --
and show that
existing solutions for instability fail.
To address instability,
we first
propose finetuning a single model with multiple prompts;
this results in
better performance and less model space complexity than
finetuning several models with single prompts.
We then propose run ensemble techniques that improve stability
and overall performance.

Our more stable setup allows us
to explore active learning for prompt-based finetuning. 
We compare a set of active
learning algorithms to reduce training data selection
instability and improve overall performance. Our active
learning algorithm, inter-prompt  uncertainty sampling
with diversity, outperforms prior active learning algorithms
(and random selection) for both ALBERT and
RoBERTa$_\text{LARGE}$.

Finally, we believe that our findings and proposals on
measuring and reducing variance for stable and active
learning for few-shot classification will be useful for the
community by enabling fair comparison and helping
researchers track progress on NLP tasks.
	
	\section*{Limitations}
	We exploit information in Pretrained Language Models (PLMs) to effectively use them in prompt-based finetuning for few-shot classification. Therefore prompt-based few-shot classification might be more open to reflecting biases in pretrained language models. Furthermore, our ENSEMBLE\textsubscript{Prediction} technique increases the model complexity during inference time to provide better performance and stability. However, ENSEMBLE\textsubscript{Parameter} can be an alternative as it also improves performance and stability over the default setting while maintaining model complexity same. Furthermore, this work is only demonstrated for tasks in the English language. 
	
	

% Entries for the entire Anthology, followed by custom entries
\bibliography{anthology,custom}
\bibliographystyle{acl_natbib}

\appendix

\section{Appendix}
\label{sec:appendix}
\subsection{Implementation Details}
We implement our contributions on top of the PET library \footnote{\href{https://github.com/timoschick/pet}{https://github.com/timoschick/pet}}. For prompt-based finetuning with single and multiple prompts, we use $10$ epochs, $1e-5$ learning rate, and also the default parameters for other hyperparameters. For a longer training approach, we use $50$ epochs with a $1e-6$ learning rate and warmup with a ratio of $0.1$. We report results with five different runs on five different training data. 

We conduct our experiments with NVIDIA GeForce GTX 1080 Ti, and one run of prompt-based finetuning with a single prompt takes approximately 20 minutes while one run of prompt-based finetuning with a multiprompt approach takes approximately 18 minutes for 32 examples, 4 prompts, ALBERT-xxlarge-v2 model with 223M parameters. As described in the Section \ref{sec:active_learning}, we select training sets with $16*L$ samples where $L$ is the number of labels. Therefore, RTE, SST-2, and MRPC have 32 training samples, SST-5 has 80 training samples, and TREC has 96 training samples. RTE, SST-2, SST-5, TREC, and MRPC contain 277, 872, 1101, 500, and 408 validation examples respectively.

\end{document}
