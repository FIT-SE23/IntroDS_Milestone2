% \documentclass[twocolumn,superscriptaddress]{revtex4-2}
%\setcitestyle{super}
\documentclass[aip,prl,reprint]{revtex4-2}
%\documentclass[prl, pre, twocolumn, reprint]{revtex4-1}
\usepackage[T1]{fontenc}
%\usepackage{fouriernc}
\usepackage{times}
%\usepackage{garamondlibre}
%\documentclass[prl,twocolumn]{revtex4-2}
%[aip,apl,reprint]
\usepackage{graphicx}
%\usepackage{dcolumn}
%\usepackage{bm}
\usepackage{booktabs}

\usepackage{amsfonts, amsmath,amssymb, bm, graphicx}
\usepackage{siunitx}

%\usepackage{times}
%
\usepackage{amssymb}
\usepackage[english]{babel}
\usepackage{lipsum}

\def\vec#1{{\bm{#1}}}
\def\mat#1{{\hat{\vec{#1}}}}
\def\H{{\mathcal{H}}}
\newcommand{\figref}[2]{\hyperref[#1]{\ref{#1}(#2)}}
\newcommand{\figrefsub}[3]{\hyperref[#1]{\ref{#1}(#2)#3}}

\usepackage{physics}
\usepackage{lipsum}
\usepackage{changes}

\usepackage[colorlinks=true,allcolors=blue]{hyperref}
\usepackage{footmisc}

%\newcommand{\todo}[1]{\textbf{\textcolor{black}{#1}}}

\usepackage{upgreek}

\usepackage{soul}
\newenvironment{rcases}
  {\left.\begin{aligned}}
  {\end{aligned}\right\rbrace}

\makeatletter
\let\ORIbbl@fixname\bbl@fixname
\def\bbl@fixname#1{%
  \@ifundefined{languagealias@\expandafter\string#1}
    {\ORIbbl@fixname#1}
    {\edef\languagename{\@nameuse{languagealias@#1}}}%
}
\newcommand{\definelanguagealias}[2]{%
  \@namedef{languagealias@#1}{#2}%
}
\makeatother

\definelanguagealias{en}{english}

\begin{document}

\title{Modification of three-magnon splitting in a flexed magnetic vortex}

\author{L. K\"orber}\email{l.koerber@hzdr.de}
\affiliation{Helmholtz-Zentrum Dresden--Rossendorf, Institut f\"ur Ionenstrahlphysik und Materialforschung, D-01328 Dresden, Germany}
\affiliation{Fakultät Physik, Technische Universität Dresden, D-01062 Dresden, Germany}


\author{C. Heins}
\affiliation{Helmholtz-Zentrum Dresden--Rossendorf, Institut f\"ur Ionenstrahlphysik und Materialforschung, D-01328 Dresden, Germany}
\affiliation{Fakultät Physik, Technische Universität Dresden, D-01062 Dresden, Germany}

\author{I. Soldatov}
\affiliation{Institute for Integrative Nanosciences, Leibniz Institute for Solid State and Materials Science (IFW) Dresden, Helmholtzstrasse 20, 01069 Dresden, Germany}
%\affiliation{Institute of Natural Sciences, Ural Federal University, 620002 Ekaterinburg, Russia}

\author{R. Schäfer}
\affiliation{Institute for Integrative Nanosciences, Leibniz Institute for Solid State and Materials Science (IFW) Dresden, Helmholtzstrasse 20, 01069 Dresden, Germany}
\affiliation{Institut für Materialphysik, Technische Universität Dresden, D-01062 Dresden, Germany}


\author{A. Kákay}
\affiliation{Helmholtz-Zentrum Dresden--Rossendorf, Institut f\"ur Ionenstrahlphysik und Materialforschung, D-01328 Dresden, Germany}

\author{H. Schultheiss}
\affiliation{Helmholtz-Zentrum Dresden--Rossendorf, Institut f\"ur Ionenstrahlphysik und Materialforschung, D-01328 Dresden, Germany}
\affiliation{Fakultät Physik, Technische Universität Dresden, D-01062 Dresden, Germany}

\author{K. Schultheiss}
\affiliation{Helmholtz-Zentrum Dresden--Rossendorf, Institut f\"ur Ionenstrahlphysik und Materialforschung, D-01328 Dresden, Germany}







%\author{R. Verba}
%\affiliation{Institute of Magnetism, National Academy of Sciences of Ukraine, Kyiv 03142, Ukraine}





%\author{T. Hache}
%\affiliation{Helmholtz-Zentrum Dresden--Rossendorf, Institut f\"ur Ionenstrahlphysik und Materialforschung, D-01328 Dresden, Germany}
%\affiliation{TU Chemnitz, Germany}

%\author{L. Bischoff}
%\affiliation{Helmholtz-Zentrum Dresden--Rossendorf, Institut f\"ur Ionenstrahlphysik und Materialforschung, D-01328 Dresden, Germany}

%\author{A. Awad}
%\affiliation{Department of Physics, University of Gothenburg, 412 96 Gothenburg, Sweden}

%\author{V. Tiberkevich}
%\affiliation{Department of Physics, Oakland University, Rochester, MI 48309, USA}

%\author{A.N. Slavin}
%\affiliation{Department of Physics, Oakland University, Rochester, MI 48309, USA}

%\author{J. Fassbender}
%\affiliation{Helmholtz-Zentrum Dresden--Rossendorf, Institut f\"ur Ionenstrahlphysik und Materialforschung, D-01328 Dresden, Germany}
%\affiliation{TU Dresden, D-01062 Dresden, Germany}




\date{\today}

%Letter
%A Letter reports an important novel research study. Letters typically occupy four printed journal pages. The text is limited to 2,000 words, including the introductory paragraph, but excluding Methods, references and figure legends. Letters should have no more than 4 display items (figures and/or tables). As a guideline, Letters allow up to 30 references (excluding those cited exclusively in Methods).
%This format begins with a title of, at most, 15 words, followed by an introductory paragraph (not abstract) of approximately 150 words, summarizing the background, rationale, main results (introduced by "Here we show" or some equivalent phrase) and implications of the study. This paragraph should be referenced, as in Nature style, and should be considered part of the main text, so that any subsequent introductory material avoids too much redundancy with the introductory paragraph.

%Article
%An article is a substantial novel research study of high quality and general interest to the physics community. The main text (excluding abstract, Methods, references and figure legends) is 2,000-3,000 words. Articles have 4-6 display items (figures and/or tables). As a guideline, Articles allow up to 50 references (excluding those cited exclusively in Methods).
%The maximum title length is 15 words. The abstract is typically 150 words and is unreferenced; it contains a brief account of the background and rationale of the work, followed by a statement of the main conclusions introduced by the phrase "Here we show" or some equivalent. An introduction (without heading) of up to 500 words of referenced text expands on the background of the work (some overlap with the summary is acceptable), followed by a concise, focused account of the findings, ending with one or two short paragraphs of discussion.


\begin{abstract}

We present an experimental and numerical study of three-magnon splitting in a micrometer-sized magnetic disk with the vortex state  strongly deformed by static in-plane magnetic fields. Excited with a large enough power at frequency $f_\mathrm{RF}$, the primary radial magnon modes of a cylindrical magnetic vortex can decay into secondary azimuthal modes via spontaneous three-magnon splitting. This nonlinear process exhibits selection rules leading to well-defined and distinct frequencies $f_\mathrm{RF}/2\pm \Delta f$ of the secondary modes.
Here, we demonstrate that three-magnon splitting in vortices can be significantly modified by deforming the magnetic vortex with in-plane magnetic fields, leading to a much richer three-magnon response. We find that, with increasing field, an additional class of secondary modes is excited which are localized to the highly-flexed regions adjacent to the displaced vortex core. While these modes satisfy the same selection rules of three-magnon splitting, they exhibit a much lower three-magnon threshold power compared to regular secondary modes of a centered vortex. The applied static magnetic fields are small ($\simeq \SI{10}{\milli\tesla}$), providing an effective parameter to control the nonlinear spectral response of confined vortices. Our work expands the understanding of nonlinear magnon dynamics in vortices and advertises these for potential neuromorphic applications based on magnons.

\end{abstract}

\maketitle

Magnetic vortices are non-uniform magnetization states which naturally appear in condensed matter and are of interest for both fundamental as well as applied research. In thin easy-plane ferromagnets, they are comprised by an outer region, the vortex \textit{skirt}, where the direction of magnetization $\bm{m}=\bm{M}/M_\mathrm{s}$ (with $M_\mathrm{s}$ being the saturation magnetization of the material) performs a full $2\pi$ rotation in the plane of the ferromagnet, and an inner region, the vortex \textit{core}, where the magnetization points out of the plane at the center of rotation. Topologically, vortices are merons, closely related to Skyrmions, with a half-integer topological charge.\cite{kosevichMagneticSolitons1990,metlovTwodimensionalTopologicalSolitons2001} Isolated vortices can be observed, for example, in nano-to-micrometer-sized flat magnetic disks. In such systems, vortices have been proposed for example as storage devices, encoding information into the vortex-core polarity (up or down).\cite{shinjoMagneticVortexCore2000,vanwaeyenbergeMagneticVortexCore2006,hertelUltrafastNanomagneticToggle2007} Other works propose vortex-resonators based on the gyrotropic motion of the core.\cite{gaidideiMagneticVortexDynamics2010,guslienkoEigenfrequenciesVortexState2002,pribiagMagneticVortexOscillator2007,mistralCurrentDrivenVortexOscillations2008,ruotoloPhaselockingMagneticVortices2009} Apart from the statics and dynamics related to the core, vortices also bear a rich spectrum of azimuthal and radial magnon modes, which live, predominantly, in the vortex skirt.\cite{ivanovMagnonModesMagnonvortex1998,shekaAmplitudesMagnonScattering2004,ivanovHighFrequencyModes2005,buessExcitationsNegativeDispersion2005} These modes exhibit interesting linear as well as nonlinear dynamics. For disk radii in the micrometer regime, the magnon spectrum allows for resonant three-magnon splitting (3MS), where one primary magnon, excited above a certain power threshold, splits into two secondary magnons under conservation of energy and angular momentum.\cite{lvovWaveTurbulenceParametric1994} Recently, it has been shown how 3MS can be exploited to excite modes with unprecedented large azimuthal mode number,\cite{schultheissExcitationWhisperingGallery} which can potentially couple to photon modes in optical cavities. In magnetic vortices, 3MS exhibits special selection rules, leading to secondary modes with distinct frequencies,\cite{verbaTheoryThreemagnonInteraction2021} and can even be stimulated non-locally below its nonlinear threshold.\cite{korberNonlocalStimulationThreemagnon2020} These qualities make confined vortices attractive for nonlinear networks and reservoir computing, for example, as a magnon-scattering reservoir.\cite{korberNonlocalStimulationThreemagnon2020,korberPatternRecognitionMagnonscattering2022}

So far, 3MS in confined vortices has been studied in the absence of external fields, where the vortex core is centered within the magnetic disk. In this work, we study the influence on the nonlinear magnon dynamics of an in-plane magnetic field which displaces the vortex core from its center position and leads to a flexing of the vortex skirt. Our study is carried out numerically using micromagnetic simulations and experimentally using Kerr microscopy and micro-focused Brillouin-light-scattering spectroscopy. We find that the spectrum of secondary modes acquires additional distinct frequency features, originating to two different kinds of magnons, localized either to the quasi-homogeneous (field-aligned) or the highly-flexed regions of the skirt. The latter ones, which only exist in a flexed vortex, are found to exhibit a significantly lower three-magnon threshold power compared to the regular modes in the field-free case.
Despite breaking the cylindrical symmetry of the magnetic state, the selection rules of 3MS in vortices persist, providing secondary modes with well-defined frequencies. The in-plane magnetic fields necessary to achieve a sufficient vortex deformation are small ($\simeq \SI{10}{\milli\tesla}$) and could be generated with on-chip stripline antennae.\cite{schultheissTimeRefractionSpin2021} This work contributes to the understanding of magnon modes in a magnetic vortex and explores an effective parameter to modify their nonlinear characteristics \textit{in-situ}.

For our study, we consider a magnetic disk of \SI{50}{\nano\meter} thick Ni$_{81}$Fe$_{19}$ (permalloy, Py) with a diameter of \SI{5.1}{\micro\meter},\cite{schultheissExcitationWhisperingGallery} as seen in Fig.~\figref{fig:FIG1}{a}. In such a system, a magnetic vortex is stable and corresponds to the state which minimizes magnetic strays fields everywhere except in the core region. The presence of the vortex state at zero field is confirmed with micromagnetic simulations\cite{schultheissExcitationWhisperingGallery,korberNonlocalStimulationThreemagnon2020,vansteenkisteDesignVerificationMuMax32014} and Kerr microscopy.\cite{hubertMagneticDomainsAnalysis1998,soldatovSelectiveSensitivityKerr2017} In Fig.~\figref{fig:FIG1}{a}, we present the in-plane angle of the magnetization obtained with both methods as a color map superimposed on the top surface of the magnetic disk.

To discuss the deformation of the vortex under application of an in-plane external magnetic field $\bm{B}$ (here, applied in $y$ direction), it is useful to consider the contour lines of the magnetization component $m_x=\mathrm{const}.$ perpendicular to the magnetic field. In the following, we only show the contour lines (as dotted lines) obtained from the numerical data. At zero field, these contours are straight lines, in agreement with an absence of magnetic volume charges $\varrho=-M_\mathrm{s}\bm{\nabla}\bm{m}$. With increasing vortex-displacement, these charges become nonzero and play an important role for the magnon dynamics. In field-free case, the magnon modes can be characterized by their number of nodal lines $n=0,1,2,...$ in radial direction and their number of periods $m=0,\pm 1,\pm 2$ in azimuthal direction. An $\Omega$-shaped microwave antenna is used to excite radial magnon modes with $m=0$ at frequency $f_\mathrm{RF}$. When the input power is high enough, this primary radial mode splits into two secondary modes with opposite azimuthal indices $\pm m$ (conservation of angular momentum), distributed around half the excitation frequency $f_{1,2}=f_\mathrm{RF}/2\pm \Delta f$ (conservation of energy), as seen in Fig.~\figref{fig:FIG1}{b}. An additional selection rule requires the two secondary modes to exhibit different radial indices, resulting in a frequency split $\Delta f \neq 0$ between them, as shown by~\textcite{schultheissExcitationWhisperingGallery} \textcolor{black}{In first order of perturbation (with respect to excitation power), the split $\Delta f$ is determined by the band gap at given azimuthal index $m$ between modes with different radial index $n$. For larger excitation powers, this split is altered by nonlinear frequency shift.\cite{krivosikHamiltonianFormulationNonlinear2010}}

\begin{figure}
    \centering
    \includegraphics{FIG1}
     \caption{(a) Schematics of a magnetic Py disk with \SI{5.1}{\micro\meter} diameter and \SI{50}{\nano\meter} thickness and of the antenna design used in the experiments. At remanence, the disk exhibits a stable vortex state, confirmed with micromagnetic simulations and, experimentally, with Kerr microscopy. The color code depicts the in-plane angle of the magnetization, obtained with both methods. A white circle indicates the position of the vortex core where the magnetization tilts out-of-plane. Dashed lines denote the contour lines of constant $m_x$ ($x$ component of the normalized magnetization) obtained from the simulations. (b) Exciting a radial mode above the threshold triggers nonlinear three-magnon splitting, and pairs of modes with frequencies $f_{1,2}=f_\mathrm{RF}/2 \pm \Delta f$ and different radial profiles are  excited. The shown spatial mode profiles have been obtained from micromagnetic simulations. (c) Measured (dotted line) and simulated (solid line) hysteresis loop segment of the vortex disk with in-plane external field is shown. The magnetization distribution for certain field values is added as insets, in which the color code represents the experimentally and numerically obtained in-plane angle of the magnetization states. In particular, the lower half of each inset shows the respective quantitative Kerr image.}
    \label{fig:FIG1}
\end{figure}


Before we explore the influence of vortex deformation on 3MS, we study the effect of an externally applied magnetic field on the vortex state itself, in particular, its in-plane hysteresis loop. For this, we sweep the external field from 0 to \SI{40}{\milli\tesla} (and back), tracking the vortex state with micromagnetic simulations and Kerr microscopy. In Fig.~\figref{fig:FIG1}{c}, we show the evolution of the magnetization component $m_y$ parallel to the applied field together with two-dimensional maps of the vortex state at certain points of the loop. For moderate in-plane fields (below \SI{15}{\milli\tesla}), the vortex core is displaced from its center position which leads to a growth of the region of the vortex skirt magnetized parallel to the field (green area in the angle maps). However, in parallel to this displacement of the core, the skirt of the vortex is not simply displaced transversally to the field (which would correspond to a rigid-vortex model). Instead, the vortex skirt flexes in order to keep the magnetization parallel to the sides of the disk which avoids magnetic edge charges and, thus, minimizes stray fields. This can be seen nicely in Fig.~\figref{fig:FIG1}{d}, as the contour lines of $m_x$ clearly bend (deviating from rigid core displacement). Such behavior is described by the two-vortex model\cite{guslienkoEvolutionStabilityMagnetic2001,metlovStabilityMagneticVortex2002} for zero field, or, more appropriate for the case of applied fields, by a deformable-vortex pinning model.\cite{burgessAnalyticalModelVortex2014} A description of the vortex displacement under in-plane magnetic fields in similar magnetic elements is found, \textit{e.g.} in Refs.~\citenum{schaeferHysteresisSoftFerromagnetic2002,desimoneLowEnergyDomain2002}.

Finally, after further increasing the applied magnetic field, the vortex core reaches the boundary of the disk and annihilates, allowing for a saturation of the magnetic sample. When again decreasing the external field from saturation, two vortices nucleate at the boundary, which move together and finally merge to a single vortex close to zero applied field. We highlight that, in Fig.~\figref{fig:FIG1}{d}, this two-vortex nucleation is seen in both experiment and simulation, attesting to the good agreement between both methods.

For the following study of nonlinear magnon interaction, we shall restrict ourselves to the field range from $0\rightarrow\SI{10}{\milli\tesla}$ in which the vortex core is merely displaced, as marked in Fig.~\figref{fig:FIG1}{c}. We start again at zero field and excite the confined vortex with an out-of-plane microwave field at $f_\mathrm{RF}=\SI{6.1}{\giga\hertz}$. In the micromagnetic simulations, the strength of the field is set to $b_\mathrm{RF}=\SI{2.8}{\milli\tesla}$. In our experiments, we set the microwave output power to $L_P = 17$\,dBm. Even though it is experimentally cumbersome to determine the exact microwave power arriving at the sample and, thus, comparing it with the field applied in the simulations, in both cases, the radial mode $(n=0, m=0)$ is excited above its 3MS threshold. In experiments, the spectral response of the system is probed by means of micro-focused Brillouin-light-scattering spectroscopy\cite{sebastianMicrofocusedBrillouinLight2015} and in simulations, by means of Fourier analysis. As seen in Fig.~\figref{fig:FIG2}{a}, nonlinear splitting of the excited primary mode leads to two secondary modes around half the excitation frequency. Additional frequency contributions can be attributed to higher-order processes (\textit{e.g.} four-magnon-scattering between the secondary modes) and, for simplicity, will not be considered in the following discussion.

\begin{figure}
    \centering
    \includegraphics{FIG2}
    \caption{(a) Numerically and (b) experimentally obtained frequency response of the magnetic vortex excited at $f_\mathrm{RF}=\SI{6.1}{\giga\hertz}$ above the threshold for 3MS as a function of the applied in-plane magnetic field. At zero field, only one pair of secondary modes is created around $f_\mathrm{RF}/2$ by 3MS (other signals at zero field are only higher-harmonics\cite{schultheissExcitationWhisperingGallery}). With increasing in-plane magnetic field, the vortex is deformed which leads to a splitting of the three-magnon response into several branches, corresponding to the \textit{regular} vortex modes and additional \textit{butterfly} modes, as seen in the mode profiles. (c) Numerical and (d) experimental profiles of all modes at a given vortex-displacement (in-plane field), showing that regular and butterfly modes are arranged in pairs which satisfy the selection rules of 3MS in vortices.}
    \label{fig:FIG2}
\end{figure}


To investigate the field dependence of the spectral response, we increase the external field to displace the vortex core from its center position, while maintaining the microwave excitation at \SI{6.1}{\giga\hertz}. The corresponding spectral response as a function of applied field is shown as colormaps in Fig.~\figref{fig:FIG2}{a} (simulation) and Fig.~\figref{fig:FIG2}{b} (experiment). Discrepancies between simulation and experiment can be attributed to the previously discussed uncertainty in the microwave-power levels as well as a small uncertainty ($< \SI{1}{\milli\tesla}$) in the experimentally set in-plane fields. For very small fields ($\lesssim \SI{2.5}{\milli\tesla}$) we see that the frequency split $\Delta f$ between the secondary modes decreases slightly. As a first observation, overall, 3MS is found to be robust against vortex deformation. However, with increasing in-plane field [see again Figs.~\figref{fig:FIG2}{a,b}], we observe a splitting of the secondary modes into several branches. This splitting continues for even higher in-plane fields. To identify the different branches, we obtain the spatial modes profiles at $B=\SI{4.5}{\milli\tesla}$ in the simulations by means of inverse Fourier transform at the respective frequencies. In our experiments, the spatial mode profiles are obtained at $B=\SI{5.8}{\milli\tesla}$ by scanning the whole sample with BLS microscopy.\cite{sebastianMicrofocusedBrillouinLight2015} \textcolor{black}{These two external fields were chosen to achieve well-separated peaks in the frequency spectra obtained with each method. Their difference can again be justified by the uncertainty in microwave power and external field mentioned above, as well as by the lower spectral resolution in the experiments.} \textcolor{black}{The numerical mode magnitude corresponds to the local magnitude $\vert \Tilde{m}_z(\bm{r},f)\vert$ of the corresponding Fourier component of the dynamic magnetization, which is proportional to the photon counts we measure in our BLS experiments.}


The profiles of all modes (directly excited and secondary) are shown for both methods in Figs.~\figref{fig:FIG2}{c,d}. We find that the different branches correspond to two qualitatively distinct classes of modes. The first class of modes resembles the \textit{regular} modes of a magnetic vortex and are merely deformed versions of the modes in a centered vortex~\cite{schultheissExcitationWhisperingGallery}. For the chosen magnetic field, these modes oscillate at $\SI{2.7}{\giga\hertz}$ and at $\SI{3.4}{\giga\hertz}$. The second class of modes is characterized by unconventional spatial profiles, resembling the shape of a butterfly with the vortex core at its center ($\SI{2.4}{\giga\hertz}$ and $\SI{3.7}{\giga\hertz}$). Clearly, these modes are arranged in separate pairs, a pair of regular secondary modes and a pair of butterfly secondary modes. Each of these pairs satisfies the three-magnon resonance condition $f_{1,2}=f_\mathrm{RF}/2\pm \Delta f$. It is quite remarkable that, even-though the cylindrical symmetry of the system is broken and a characterization of the modes in terms of radial and azimuthal mode numbers becomes ambiguous, the secondary modes within one scattering channel (regular or butterfly) still obey to certain selection rules which lead to a non-zero frequency split between them. In the numerically obtained mode profiles in Fig.~\figref{fig:FIG2}{c}, it can be nicely seen how the modes in each pair still have a different number of nodal lines along the new "radial" direction. This retaining asymmetry between the spatial profiles of the secondary modes is not surprising, as the vortex still inherits a mirror symmetry even when deformed into a flexed state. Such mirror symmetry alone can lead to selection rules for three-magnon splitting, as it is the case in magnetic films.\cite{lvovWaveTurbulenceParametric1994}


\begin{figure}
    \centering
    \includegraphics{FIG3}
    \caption{Static magnetic-volume charges in a (a) centered vortex at zero in-plane field and a (b) flexed vortex at non-zero in-plane field. Whereas at zero field, only one class of modes is present, at finite field, the flexing of the vortex skirt leads to plateaus and wells in the internal dipolar energy of the vortex, leading to two different classes of modes (\textit{regular} on the plateau and \textit{butterfly}-shaped in the wells).}
    \label{fig:FIG3}
\end{figure}

\begin{figure*}
    \centering
    \includegraphics{FIG4}
    \caption{Evolution of the numerically and experimentally obtained frequency response of the magnetic vortex excited at a fixed microwave-excitation frequency of \SI{6.1}{\giga\hertz} with increasing magnitudes/powers of the driving microwave source for in-plane fields between 0 to \SI{10}{\milli\tesla}. For low excitation (a) and (b) only the primary magnon mode can be seen. At slightly higher excitations, panel (c) and (d) secondary modes appear for static fields for which the vortex is strongly distorted. The mode profiles obtained both from simulations and experiments are shown as insets and demonstrate that these modes are butterfly-shaped modes localized to the highly flexed regions of the vortex skirt. Further increasing the excitation strength, shown in  (e) and (f), leads to the appearance of the 3MS for regular vortex modes close the zero in-plane field, confirming that the three-magnon splitting into butterfly modes has a lower threshold power.  Finally, at large excitation strengths a rich and complex frequency response is measured, summarized on panels (i)-(k).
%    Here we show that the butterfly modes have a lower threshold, how the two mode kinds grow together, and how rich the whole spectrum is in general \textbf{TODO}
}
    \label{fig:FIG4}
\end{figure*}

The splitting of the spectrum of secondary modes into two classes can be understood in a simple picture by considering the landscape of magnetic volume charges created by deforming the magnetic vortex. To illustrate this, in Fig.~\ref{fig:FIG3}, we show the volume charges for a centered and a deformed vortex obtained from our simulations. At zero field [Fig.~\figref{fig:FIG3}{a}], the vortex achieves to suppress volume charges almost everywhere by forming a perfectly azimuthal rotation of in-plane magnetization, except close to the vortex core. As a result, the spectrum is only comprised of the well-known regular vortex modes described by azimuthal and radial mode numbers. For finite static in-plane fields, the flexing of the vortex skirt leads to an accumulation of magnetic volume charges adjacent to the core and at the boundaries of the magnetic disk close to the core, as seen in Fig.~\figref{fig:FIG3}{b}. As shown in Fig.~\figref{fig:FIG3}{c}, this charge distribution leads to potential wells for bound magnon state similar to the channeled magnons in Néel-type domain walls.\cite{wagnerMagneticDomainWalls2016} These bound modes have already been observed with low number of periods around the vortex core in Refs.~\citenum{alievSpinWavesCircular2009,jenkinsElectricalCharacterisationHigher2021}. In said works, the authors used linear excitation with in-plane microwave fields or direct spin-polarized currents to excite these modes. Both works, however, did not experimentally verify the spatial character of these modes. Apart from the flexed part of the vortex skirt, an (almost) charge-free region remains at the opposite side of the disk, leading to a potential plateau for the regular "free" vortex modes.




In the following we will see that the observed bound/butterfly modes exhibit a much lower three-magnon power threshold than the regular modes in a centered vortex. Again, we sweep the static in-plane field from 0 to \SI{10}{\milli\tesla}, however, this time, for different magnitudes/powers of the driving microwave field. We start with a microwave power below the nonlinear 3MS threshold, at $b_\mathrm{RF}=\SI{0.8}{\milli\tesla}$. We see in Fig.~\figref{fig:FIG4}{a,b} (simulation and experiment, respectively) that only the primary magnon at $f_\mathrm{RF}=\SI{6.1}{\giga\hertz}$ is excited over the whole field range and no additional secondary modes. With increasing microwave power [Fig.~\figref{fig:FIG4}{c,d}], secondary modes appear for static fields $\gtrsim\SI{4.5}{\milli\tesla}$ at which the vortex is deformed considerably. Analyzing the spatial mode profiles of these modes from experiments and simulation reveals that all of them are indeed modes localized to the highly flexed regions of the vortex skirt [shown as insets in Fig.~\figref{fig:FIG4}{c,d}]. Note again, that all of them are arranged in pairs with frequency $f_{1,2}=f_\mathrm{RF}/2 \pm \Delta f$ and different "radial" profiles. Only when increasing the microwave power even further, the 3MS threshold of the regular vortex modes is reached [Fig.~\figref{fig:FIG4}{e,f}], and their frequency contribution is visible at small in-plane fields at which the vortex is not deformed at all. Finally, with further increasing the microwave power, the branches of the two classes of modes approach each other until their overlap in in-plane-field range and ultimately, the behavior discussed in context of Fig.~\ref{fig:FIG2} is recovered.


The reason of the butterfly modes to inherit a lower three-magnon threshold, compared to the regular vortex modes, can be two-fold: First, their intrinsic damping rates $\Gamma_{1,2}$~\cite{} could be significantly lower which would point towards lower mode ellipticities, or, second, their three-magnon scattering coefficients $V_{0,12}$~\cite{} with respect to the directly-excited (primary) magnon are significantly larger.\cite{verbaTheoryThreemagnonInteraction2021} A combination of both factors finally results in lower threshold microwave fields $b_{\mathrm{RF},\mathrm{crit}}\propto \Gamma_1\Gamma_2/\abs{V_{0,12}}$. To answer this open question, an analysis of both factors separately would be necessary, which would be possible, \textit{e.g.} within the frame of a vector Hamiltonian formalism for nonlinear magnon dynamics\cite{tyberkevychVectorHamiltonianFormalism2020} after having numerically calculated the spatial mode profiles, \textit{e.g.} with a dynamic-matrix method.\cite{grimsditchMagneticNormalModes2004} This would, however, go beyond the scope of this work.

With our study of three-magnon splitting in a flexed magnetic vortex, confined to a micrometer-sized magnetic disk,  we have shown that this nonlinear process is stable with respect to displacement of the vortex core by in-plane magnetic bias fields. Application of such fields leads to a flexing of the vortex skirt, which, in return, leads to a separation of the secondary modes produced by three-magnon splitting into modes corresponding to the regular radial and azimuthal modes of a symmetric vortex, and, into additional \textit{butterfly} modes which are confined to the highly flexed regions adjacent to the vortex core. Splitting into these additional secondary modes exhibits a much lower power threshold than into the regular vortex modes. This work expands the understanding of three-magnon splitting in confined magnetic systems, providing a way to excite unconventional magnon modes in flexed magnetic vortices. Furthermore, small in-plane bias fields, in the range of a few tens of \si{\milli\tesla} are shown to be a powerful parameter to tune the characteristics of a microscopic nonlinear systems in place, which is attractive, for example, for neuromorphic applications such as reservoir computing.


\section*{Author declarations}

\subsection*{Conflict of Interest}
The authors have no conflicts of interest to disclose.

\subsection*{Author's contributions}

\textit{Will be inserted via submission form.}


\section*{Acknowledgements}

The authors acknowledge fruitful discussions with V. Tyberkevych. Financial support by the Deutsche Forschungsgemeinschaft (DFG) within the programs SCHU 2922/1-1, KA 5069/1-1 and KA 5069/3-1 is gratefully acknowledged as well as from the EU Research and Innovation Programme Horizon Europe under grant agreement no. 101070290 (NIMFEIA). Support by the Nanofabrication Facilities Rossendorf (NanoFaRo) at the IBC is gratefully acknowledged.


\section*{Data availability}
The data that support the findings of this study are openly available in RODARE.\footnote{L. Körber, C. Heins, I. Soldatov, R. Schäfer, A. Kákay, H. Schultheiss, K. Schultheiss, `` Data publication: Modification of three-magnon splitting in a flexed magnetic vortex,``  RODARE \url{http://doi.org/10.14278/rodare.2064}, version 1 2022}

%\section*{References}

%\bibliography{references.bib}
%\documentclass[journal]{IEEEtran}
\usepackage{cite}
\usepackage{amsmath} 

\usepackage{subfigure}
\ifCLASSINFOpdf
\usepackage[pdftex]{graphicx}
  % declare the path(s) where your graphic files are
  \graphicspath{{../pdf/}{../jpeg/}}
  % and their extensions so you won't have to specify these with
  % every instance of \includegraphics
  \DeclareGraphicsExtensions{.pdf,.jpeg,.png}
\else
  % or other class option (dvipsone, dvipdf, if not using dvips). graphicx
  % will default to the driver specified in the system graphics.cfg if no
  % driver is specified.
  \usepackage[dvips]{graphicx}
  % declare the path(s) where your graphic files are
  \graphicspath{{../eps/}}
  % and their extensions so you won't have to specify these with
  % every instance of \includegraphics
  \DeclareGraphicsExtensions{.eps}
\fi  
\usepackage{amsmath}
\usepackage{cases}
\usepackage{stfloats}
\usepackage{amsfonts}
\usepackage{subeqnarray}
\usepackage{longtable}
\usepackage{supertabular}
\usepackage{setspace}
\usepackage{multirow}
\usepackage{booktabs}
% \usepackage{algorithm}
% \usepackage{algorithmic}
\usepackage[ruled,linesnumbered]{algorithm2e}
\makeatletter
\newcommand{\nosemic}{\renewcommand{\@endalgocfline}{\relax}}% Drop semi-colon ;
\newcommand{\dosemic}{\renewcommand{\@endalgocfline}{\algocf@endline}}% Reinstate semi-colon ;
\newcommand{\pushline}{\InDPP}% Indent
\newcommand{\popline}{\Indm\dosemic}% Undent
\let\oldnl\nl% Store \nl in \oldnl
\newcommand{\nonl}{\renewcommand{\nl}{\let\nl\oldnl}}% Remove line number for one line
\makeatother
\usepackage[export]{adjustbox} 
\usepackage{booktabs}
\usepackage{setspace}
\usepackage{xcolor}
\usepackage{mathrsfs}
\usepackage{amsmath}
\usepackage{array}
\usepackage{amssymb}
\usepackage{amsthm}
\usepackage{microtype}
\usepackage{url}
\usepackage{amsfonts,amssymb}
% \usepackage{bbm}
\usepackage{dsfont}
\usepackage{mathtools}
\usepackage{xcolor,colortbl}
\usepackage{colortbl}
\usepackage{graphicx}
% \usepackage{tabularray}

\newcommand{\mc}[2]{\multicolumn{#1}{c}{#2}}
\definecolor{Gray}{gray}{0.85}
\definecolor{Whitecolor}{rgb}{1,1,1}


\hyphenation{op-tical net-works semi-conduc-tor}

\setlength{\textfloatsep}{5pt}
\allowdisplaybreaks
\begin{document}
\setstretch{1}
\title{\textls[-25]{The Design of By-product Hydrogen Supply Chain Considering Large-scale Storage and Chemical Plants: A Game Theory Perspective}}
\author{Qianni~Cao,~\IEEEmembership{Student~Member,~IEEE},
Boda~Li,~\IEEEmembership{Student~Member,~IEEE},
Mengshuo~Jia,~\IEEEmembership{Member,~IEEE}, and 
Chen~Shen,~\IEEEmembership{Senior~Member,~IEEE}



%\thanks{M. Jia and C. Shen are with the State Key Laboratory of Power Systems, Tsinghua University, 100084 Beijing, China. Y. Wang and G. Hug are with the Power Systems Laboratory, ETH Zurich, 8092 Zurich, Switzerland.}
}
        
%\thanks{This work was supported in part by the Joint Funds of the National Natural Science Foundation of China under Grant U1766206 (Correspondence to Chen Shen).}
%\thanks{M. Jia, C. Shen and Z. Wang are affiliated with the State Key Laboratory of Power Systems, Department of Electrical Engineering, Tsinghua University, Beijing 100084, China (e-mail addresses: jms16@mails.tsinghua.edu.cn, shenchen@mail.tsinghua.edu.cn,
%    wangzhaojian@mail.tsinghua.edu.cn).}% <-this % stops a space
% \thanks{Manuscript received April 19, 2005; revised August 26, 2015.}

%\markboth{Submitted to IEEE Trans. Smart Grid}%
%{Shell \MakeLowercase{\textit{et al.}}: Bare Demo of IEEEtran.cls for IEEE Journals}
\maketitle


\begin{abstract}
Hydrogen, an essential resource in the decarbonized economy, is commonly produced as a by-product of chemical plants. To promote the use of by-product hydrogen, this paper proposes a supply chain model among chemical plants, hydrogen-storage salt caverns, and end users, considering time-of-use (TOU) hydrogen price, coalition strategies of suppliers, and road transportation of liquefied and compressed hydrogen. The transport route planning problem among multiple chemical plants is modeled through a cooperative game, while the hydrogen market among the salt cavern and chemical plants is modeled through a Stackelberg game. The equilibrium of the supply chain model gives the transportation and trading strategies of individual stakeholders. Simulation results demonstrate that the proposed method can provide useful insights on by-product hydrogen market design and analysis.
\end{abstract}
%Although the historical data of renewable generations could be assumed as publicly known
% Note that keywords are not normally used for peerreview papers.
\begin{IEEEkeywords}
%   Wind power, chance constraint, OPF, distributed computing, confidentiality preservation
Hydrogen market, large-scale storage, Stackelberg game, cooperative game, supply chain
\end{IEEEkeywords}
\IEEEpeerreviewmaketitle

\section*{Nomenclature}
\addcontentsline{toc}{section}{Nomenclature}

\subsection*{Indices} 
\begin{IEEEdescription}[\IEEEusemathlabelsep\IEEEsetlabelwidth{$aaaaaaaa$}]
	\item[$i,j$]		Index of chemical plants.
	\item[$t$]		Index of time periods during the day.
	\item[$n$]		Index of hydrogen processing equipment, including liquefiers and compressors.
	\item[$I+1$]		Index of the salt cavern.
\end{IEEEdescription}
\subsection*{Parameters} 
\begin{IEEEdescription}[\IEEEusemathlabelsep\IEEEsetlabelwidth{$aaaaaaaa$}]
	\item[$I$]		Number of chemical plants.
	\item[$T$]		Number of time periods.
	\item[$p_{o}$]		Retail price purchased by customers from the salt cavern.
	\item[$\underline{p}_{t},\overline{p}_{t}$]	    Lower and upper bound of the buying price offered by the salt cavern to chemical plants.
	\item[$Q_{trans}$]		Maximal injection rate of the salt cavern.
	\item[$N_\mathcal{C}$]		Number of compressors with different capacity.
	\item[$N_\mathcal{D}$]		Number of liquefiers with different capacity.
	\item[$Q_{i,t}$]		By-product hydrogen quantity produced by chemical plant $i$ in period $t$.
	\item[$\boldsymbol{Q_{pr}}$]        Capacity set of hydrogen processing equipment(kg/h), $\boldsymbol{Q_{pr}}=\{Q_{pr}^{n}\}, \forall n$.
	\item[$Q_\mathcal{C},Q_\mathcal{D}$]        Capacity of a tube trailer and a tanker truck (kg/trip).
	\item[$w_{t}$]      Electricity price in period $t$.
	\item[$\gamma_{c},\gamma_{d}$]      Electricity consumption for unit compressed hydrogen and liquefied hydrogen (kwh/kg).
	\item[$\boldsymbol{K_{1}}$]     Initial investment set of hydrogen processing equipment, $\boldsymbol{K_{1}}=\{K_{1}^{n}\}, \forall n$.
	\item[$K_{2}^{c},K_{2}^{d}$]        Initial investment cost of a tube trailer and a tanker truck.
	\item[$K_{3}$]      Operation cost of a tube trailer (or a tanker truck) in each period.
	\item[$\boldsymbol{T_{a}}$]     $T_{a}^{i,j}$ represents duration from chemical plant $i$ to $j$ ($j= I+1$ represents the salt cavern).
	\item[$\beta_{L1}$]      1 - hourly evaporation rate during the tanker truck loading.
	\item[$\beta_{L2}$]      1 - hourly evaporation rate during transit by a tanker truck.
\end{IEEEdescription}
\subsection*{Decision variables of the salt cavern} 
\begin{IEEEdescription}[\IEEEusemathlabelsep\IEEEsetlabelwidth{$aaaaaaaa$}]
	\item[$p_{t}$]		Buying price the salt cavern offers to chemical plants in period $t$.
	\item[$q_{i,t}^{trans}$]		Hydrogen transaction amount of chemical plant $i$ in period $t$, measured as hydrogen shipped from chemical plant $i$ at the end of the time period $t$.
	\item[$u_{i,I+1}$]		Binary variables. Equals to 1 when products from chemical plant $i$ is shipped directly to the salt cavern. Otherwise, $u_{i,I+1}$ equals to 0.
\end{IEEEdescription}
\subsection*{Decision variables of chemical plants} 
\begin{IEEEdescription}[\IEEEusemathlabelsep\IEEEsetlabelwidth{$aaaaaaaa$}]
	\item[$q_{i,t}^{pr}$]		Hydrogen quantity chemical plant $i$ compressed/liquified in period $t$.
	\item[$\boldsymbol{x_{i}}$]		$\boldsymbol{x_{i}}=\{x_{i}\}, \forall n$ is a set of binary variables. $x_{i}^{n}=1$ when the type of hydrogen processing equipment is selected to purchase. Otherwise, $x_{i}^{n}=0$.
	\item[$N_{i}^{cars}$]		Integer variables of number of tube trailers (or tanker trucks) purchased by chemical plant $i$.
	\item[$u_{i,j}$]		Binary variables. Equals to 1 when products from chemical plant $i$ is shipped to chemical plant $j$. Otherwise, $u_{i,j}$ equals to 0.
	\item[$q_{i,t}^{store}$]		Hydrogen quality in the tube trailer (or tanker truck) left at chemical plant $i$ before filled to capacity in period $t$.
	\item[$q_{i,t}^{unpr}$]		Hydrogen quantity temporarily stored in low-pressure storage tanks before compression or liquefication in period $t$.
	\item[$n_{i,t}^{cars}$]		Integer variables of tube trailers (or tanker trucks) leave chemical plant $i$ in period $t$.
\end{IEEEdescription}



\section{Introduction}
\subsection{Motivation}
\IEEEPARstart{I}{n} 
the context of emission peak and carbon neutrality, hydrogen is not only regarded as a critical alternative to fossil fuel to achieve carbon neutrality but offers versatility and flexibility that renewables cannot reach\cite{Allan2021}. As one of the most cost-effective options, hydrogen produced as a by-product from many chemical plants serves as a cheap and large-scale source of hydrogen. Moreover, by-product hydrogen is usually sufficiently clean and well suited for a wide range of applications, such as fuel cell (FC)-based cogeneration, FC vehicles, domestic heating, and so on\cite{CAMPANARI2020335}. However, the potential of by-product hydrogen has yet to be realized, which is emitted and thus wasted in most cases. Therefore, it presents opportunities as a new revenue stream for chemical plants and promisingly delivers on announced pledges of energy conversion nationwide in the mid-term. However, the lack of infrastructure development such as large-scale storage, logistical supply chain establishment and unexplored market have slowed down its further development.

Salt cavern storage is one of the most promising technologies to achieve large-scale, fast and secure hydrogen storage\cite{ANDERSSON201911901},which offers the most promising option owing to their low investment cost, high sealing potential and low cushion gas requirement\cite{CAGLAYAN20206793}. Notable projects are the salt cavity storages for hydrogen in Teeside, UK, and Texas, USA\cite{Gregoire2019}, demonstrating the operation feasibility on a full industrial scale. However, the business of acquiring, storing and selling by-product hydrogen has not yet been presented as an option by salt cavern operators, which inspires the work to design by-product hydrogen supply chain considering large-scale storage and chemical plants in this paper.

\subsection{Literature Review}

As demand and production capacity for hydrogen grows robustly in recent years, the outlines of hydrogen markets are starting to emerge worldwide. Initial trade and market price discoveries come first on a regional and local basis\cite{James2021}. Infrastructure development, transparent pricing benchmark and logistical supply chain establishment are key growth challenges faced by this new traded commodity just becoming established in energy commodity markets\cite{Allan2021}. 

Presently,  the hydrogen market is far from mature but is showing great potential. Many researchers focus on the planning of the hydrogen supply chain, considering various market scales, hydrogen sources and transportation modes. Life cycle analysis to estimate the economic and environmental benefits was conducted on global\cite{BRANDLE2021117481}, regional\cite{OBARA2019848} or national\cite{REN2020118482} scales. For different hydrogen sources, steam methane reforming (SMR)\cite{CARRERA2021107386} , coal gasification (CG)\cite{LI202027979}, biomass gasification (BG)\cite{CHO2019527,LUMMEN2020118996} and electrolysis (ELE)\cite{WANG2022122194} are common production technologies in recent researches. Considering hydrogen production based on different feedstocks and energy sources, an optimal structure of the hydrogen, biomass and {$\rm CO_{2}$} networks were determined in \cite{GABRIELLI2020115245}. To make comparisons of different transportation modes, Ref. \cite{FAZLIKHALAF202034503} considered four common options with various criteria and scenarios. Ref. \cite{GIM20121162} introduced a method for comparing different transport possibilities of tube or liquid trailer vs. pipeline delivery. The results showed that each transportation technology had a maximally cost-efficient niche and there was no single perfect solution for the entire system. Recently, large-scale storage for liquid hydrogen is of great attention. Ref. \cite{SEO2020114452}  considered  integrated bulk storage of hydrogen and concluded that a centralized storage structure and liquefication in central production plants can reduce the overall cost. Similarly, the status and key gaps for the commercialization of hydrogen liquefication technology with large-scale storage were discussed in \cite{RATNAKAR202124149}. A combination of the hydrogen supply chain with other energy sources has also attracted the attention of many researchers. Ref. \cite{xiao2018} established a local energy market for electricity and hydrogen. Ref. \cite{CARRERA2021116861} proposed a methodological design framework for hydrogen and methane supply chains based on Power-to-Gas systems.

In particular, by-product hydrogen has seen growing attention these years. Ref. \cite{YANEZ2018777} for the first time assessed the economic advantages, the techno-economic feasibility and the central role of reusing by-product hydrogen in the early phase of hydrogen infrastructure in the northern Spain region. A multi-period programming was designed in \cite{YOON2022112083} to make use of existing infrastructure for by-product hydrogen and natural gas (NG) pipelines, which demonstrated the economic benefits of by-product hydrogen. Even though, the potential of by-product hydrogen remains to be discovered.

Meanwhile, most of the literature focuses on maximizing the total benefit of the whole hydrogen supply chain. Ref. \cite{HAN20125328} aimed to maximize social welfare in Korea by planning both capacity and technology of production, storage as well as transportation in an envisioned nationwide hydrogen supply chain. Ref. \cite{WICKHAM2022117740} assessed the effects that hydrogen grades play in the development of a cost-effective hydrogen supply chain. Ref. \cite{EHRENSTEIN2020115486} incorporated the concept of biophysical limits of the planet to address the optimal design of the hydrogen supply chain. An optimization method was proposed in \cite{QUARTON2020113936} for an integrated value chain of carbon dioxide and hydrogen. Individual rationality was introduced in \cite{GUO2021119608}, where the peer-to-peer transaction, endogenous market-clearing price, and uncertainties in hydrogen production were considered in detail. However, most works failed to consider the strategic behaviors and the profit of individual participants, which differed from the usual practice that suppliers and retailers are private companies and operate with a profit-driven mode.

The research gaps for the existing works are: 
\begin{enumerate}
	\item The potential of by-product hydrogen is yet to be realized and its corresponding market is waiting for further exploration.
	\item The dynamic process of chemical plants and salt caverns considering hydrogen generation, compression (or liquefaction), and the transaction is waiting to be modeled.
	\item The interactions and dynamic strategic behaviors of each stakeholder desire a more dedicated modeling framework that captures profits and rationality of individual participants.
\end{enumerate}

\subsection{Contribution}
In this work, we study the by-product hydrogen supply chain considering large-scale storage and multiple chemical plants. The main contributions are threefold:
\begin{enumerate}
	\item We establish a business model for salt caverns to acquire and store by-product hydrogen from chemical plants and sell them to end-users. The by-product hydrogen supply chain composed of each stakeholder in the business model is investigated.
	\item The hour-by-hour decision-making process of each stakeholder, i.e., chemical plants and the salt cavern, is investigated and mathematically modeled under the proposed business model, providing a foundation for the TOU hydrogen pricing strategy. 
	\item The by-product hydrogen market is formulated as a game, considering the individual rationality of each stakeholder. The planning problem among multiple chemical plants is modeled through a cooperative game. The hydrogen market among the salt cavern and chemical plants is modeled through a Stackelberg game, in which the salt cavern is the leader and chemical plants are the followers. 
\end{enumerate}

\section{A business model of salt caverns and chemical plants}
In this section, we develop a business model for salt caverns to acquire by-product hydrogen from chemical plants and sell them to end-users. Generation, large-scale storage, and consumptive way of by-product hydrogen in the business model is introduced first. Then, the comparison between the by-product hydrogen supply chain and the present hydrogen supply chain is made. Followed by this, the structure of the by-product hydrogen market under the proposed business model is introduced in the following section. 

\subsection{Generation, Large-scale Storage and Consumptive Way of By-product Hydrogen}
By-product hydrogen is a cost-competitive and widely distributed source of hydrogen.
The process of generation of by-product hydrogen and its consumptive ways are illustrated in Fig.\ref{fig:The process of generation of by-product hydrogen and its consumptive ways}.  
\begin{figure}[h] %可选参数 h t b p,代表允许图片出现的位置,h表示此处附近,t表示顶部,b表示底部,p表示单独一页,H表示固定此处
    \centering
    \includegraphics[width=8.5cm]{fig/Fig.1_The_process_of_generation_of_by-product_hydrogen_and_its_consumptive_ways.png}
    \caption{The process of by-product hydrogen generation and its consumptive ways}\label{fig:The process of generation of by-product hydrogen and its consumptive ways}
\end{figure}

Electrochemical processes, such as the industrial production of steel, caustic soda and chlorine, produce hydrogen as a by-product, burnt or emitted as the current practice. However, they can be made available for applications outside chemical plants as a future consumptive way. To transport products from the production facilities to storage sites, by-product hydrogen should be compressed or liquified in advance, which collectively 
are referred to as “hydrogen secondary processing”. Two common transportation modes are compressed gaseous hydrogen via tube trailers (CH2) and liquid hydrogen via tanker trucks (LH2). To alleviate the imbalance between supply and demand of hydrogen, underground cavities like salt caverns are potential to offer natural infrastructure to realize cost-effective and reliable hydrogen storage. At the last link in the supply chain, by-product hydrogen is sold and distributed to various end-users. The proposed generation, storage and consumptive way of hydrogen give rise to a promising by-product hydrogen business model consisting of chemical plants as suppliers, a salt cavern as a retailer and end-users as consumers. 

\subsection{Characteristics of By-product Hydrogen Supply Chain} \label{subsection: Characteristics of by-product hydrogen supply chain}

Differences between the by-product hydrogen supply chain under the proposed business model and most hydrogen supply chains found in literature can be mainly concluded as twofold: 1) composition of major costs; 2) flexibility to coordinate between planning and scheduling. These differences will lead to a distinct focus and a smaller timescale for the formulation of the by-product hydrogen supply chain, which is analyzed as follows:
% Please add the following required packages to your document preamble:
% \usepackage{multirow}
\begin{table}[h]
\centering
\caption{Major costs of hydrogen supply chain} \label{tab:Major costs of hydrogen supply chain}
\footnotesize
\begin{tabular}{cccc}
\hline\toprule
\multicolumn{2}{c}{\multirow{2}{*}{Major costs}}                                              & \multicolumn{2}{c}{Hydrogen Supply Chain}                        \\ \cline{3-4} 
\multicolumn{2}{c}{}                                                                          & \multicolumn{1}{l}{Traditional} & \multicolumn{1}{l}{By-product} \\ \hline
\multirow{2}{*}{Production}     & Investment & \checkmark &  \\
                                & Operation  & \checkmark &  \\ \hline
\multirow{2}{*}{Storage}        & Investment & \checkmark &  \\
                                & Operation  & \checkmark &  \\ \hline
\multirow{2}{*}{Transportation} & Investment & \checkmark  & \checkmark  \\
                                & Operation  & \checkmark & \checkmark \\ \hline
\multirow{2}{*}{\begin{tabular}[c]{@{}c@{}}Secondary \\ processing\end{tabular}} & Investment &                               & \checkmark                             \\
                                & Operation  &  & \checkmark \\ \hline
\end{tabular}
\end{table}
% Please add the following required packages to your document preamble:
% \usepackage{multirow}
\begin{table}[h]
\centering
\caption{Major costs and the influence factors}
\footnotesize
\label{tab:Major costs and the influence factors}
% \footnotesize
\begin{tabular}{cll}
\hline\toprule
\multirow{2}{*}{Major costs} &
  \multicolumn{2}{c}{\multirow{2}{*}{Influence factors}} \\
                                & \multicolumn{2}{c}{}                         \\ \hline
\multirow{2}{*}{Production}     & \multicolumn{2}{l}{1) Production technology} \\
                                & \multicolumn{2}{l}{2) Scale of production}   \\ \hline
\multirow{2}{*}{Storage}        & \multicolumn{2}{l}{1) Storage technology}    \\
                                & \multicolumn{2}{l}{2) Storage capacity}       \\ \hline
\multirow{3}{*}{Transportation} & \multicolumn{2}{l}{1) Transportation mode}   \\
 &
  \multicolumn{2}{l}{\multirow{2}{*}{\begin{tabular}[c]{@{}l@{}}2) Hydrogen volume\\ 3) Transport distance\end{tabular}}} \\
                                & \multicolumn{2}{l}{}                         \\ \hline
\multirow{3}{*}{\begin{tabular}[c]{@{}c@{}}Secondary \\ processing\end{tabular}} &
  \multicolumn{2}{l}{1) Type of processing equipment} \\
 &
  \multicolumn{2}{l}{\multirow{2}{*}{\begin{tabular}[c]{@{}l@{}}2) TOU electricity price\\ 3) Hydrogen volume\end{tabular}}} \\
                                & \multicolumn{2}{l}{}                         \\ \hline
\end{tabular}
\end{table}

\subsubsection{Different composition of major costs}
Major costs of hydrogen supply chain and their influence factors are demonstrated in Table \ref{tab:Major costs of hydrogen supply chain} and \ref{tab:Major costs and the influence factors}, respectively. Unlike the present hydrogen supply chain, producers in the by-product hydrogen supply chain benefit from very low-cost generation. Thus, the major cost comes from secondary processing and transportation.

Power is the major cost for secondary processing. If the liquefier or compressor operates at low-price periods, it may potentially reduce operating costs. Since electricity price fluctuates by hours, the strategic behaviors of each stakeholder should also be modeled by hour.

Transport cost is determined by transportation mode, hydrogen volume and the transport distance. For two transportation modes considered in this paper, LH2 features large transport capacity (often 10-20 times as CH2), high initial investment cost (several times as CH2) and hourly volatile losses. On the contrary, CH2 features low transport capacity, low initial investment cost and zero loss. Usually, for long-distance transportation of a large amount of hydrogen, CH2 is less economical since it requires long rides of much more vehicles than LH2. However, for mid- or short-distance of a small amount of hydrogen, CH2 is more economical since there is no volatile loss. Obviously, a reasonable decision of transportation mode would largely reduce the cost of each chemical plant.
\subsubsection{Less flexibility to coordinate between planning and scheduling }
For suppliers in the by-product hydrogen supply chain, the generation scale of hydrogen is limited by the production plan of their main products. Moreover, their location is less likely to be optimized for the transportation of by-product hydrogen.

Therefore, there may be a mismatch between each supplier's location and generation scale. Specifically, for distant (to the salt cavern) and medium-yield chemical plants, if CH2 is adopted, long-distance transport of more tube trailers may result in high transportation costs. Nevertheless, if LH2 is adopted, substantial volatile losses would happen due to hours of filling time. This situation results in a dilemma since both transportation mode leads to a revenue decline in some way. Therefore, we envision a scenario where several chemical plants in proximity to each other form a coalition and select a transit hub between them to lower transportation costs, instead of shipping individually to the salt cavern. Two examples of envisioned transportation routes are highlighted in color in Fig.\ref{fig:Possible routes for the salt cavern to acquire hydrogen from the chemical plants}. Moreover, to lower transportation costs, chemical plants destined for the transit hub adopt the CH2 transportation mode, while the transit hub destined for the salt cavern adopt the LH2 transportation mode. In this way, the dilemma between high transportation costs of CH2 and large volatile loss of LH2 is mitigated. 
\begin{figure}[h] %可选参数 h t b p,代表允许图片出现的位置,h表示此处附近,t表示顶部,b表示底部,p表示单独一页,H表示固定此处
    \centering
    \includegraphics[width=8cm]{fig/Fig.2_Possible_routes_for_the_salt_cavern_to_acquire_hydrogen_from_the_chemical_plants.png}
    \caption{Possible routes for the salt cavern to acquire hydrogen from the chemical plants} \label{fig:Possible routes for the salt cavern to acquire hydrogen from the chemical plants}
\end{figure}

To sum up, cost structure differences and the lack of flexibility to coordinate between production scale and location lead to a gap between the by-product hydrogen supply chain and the present ones. Therefore, it is essential to model the by-product hydrogen supply chain according to its characteristics rather than simply applying the model of the traditional hydrogen supply chain. 

\subsection{The Structure of By-product Hydrogen Market}

The structure of the proposed by-product hydrogen market is provided in this subsection, followed by the basic assumptions.

\begin{figure}[h] %可选参数 h t b p,代表允许图片出现的位置,h表示此处附近,t表示顶部,b表示底部,p表示单独一页,H表示固定此处
    \centering
    \includegraphics[width=8cm]{fig/Fig.3_The_structure_of_the_by-hydrogen_market_under_investigation.png}
    \caption{The structure of the by-hydrogen market under investigation} \label{fig:The structure of the by-product hydrogen market under investigation}
\end{figure}
The by-hydrogen market under the proposed business model has the structure illustrated in Fig.\ref{fig:The structure of the by-product hydrogen market under investigation}. Suppliers, namely chemical plants, process by-product hydrogen by liquefiers or compressors (depends on the decision results of each supplier) and deliver it to the retailers. The retailers, namely salt caverns, sell hydrogen to the customers. To simplify the problem, salt caverns are regarded as an entity owned by a single company.

This paper focuses on the transaction between suppliers and retailers. The following assumptions are made without loss of generality:

\begin{enumerate}
	\item The end-users buy all the hydrogen from the retailer at a fixed price. This may happen when the injection-production rate of the salt cavern is higher than the market demand in a region. In order to alleviate the supplier’s market power to drive up prices, we assume that the salt cavern and suppliers have reached such an
    agreement to bring a fixed price into effect. 
	\item The secondary processing cost and transport cost is undertaken by suppliers.  
	\item The production cost is neglected since hydrogen is a by-product of the industrial process of chemical plants. 
	\item Chemical plants would not adjust their production schedule of their main product for the revenue generated by by-product hydrogen.
\end{enumerate}

Based on the above assumptions, the retailer’s and suppliers’ problems can be described as follows. To maximize profits, the salt cavern intends to purchase as much hydrogen as possible from chemical plants at the lowest cost. If the price is too low, chemical plants are less likely to be attracted by this new revenue stream and may waste them as before, which reduces profits of the salt cavern. On the contrary, if the price is too high, the purchasing cost would increase. Therefore, it is important for the salt cavern to strike a balance between the attraction of chemical plants and the purchasing cost. To maximize profits, chemical plants upstream would like to sell more hydrogen when the selling price is high on the one hand, and to reduce processing costs and transport costs on the other hand.

Taking into account the analysis in the last subsection, the challenges of modeling the by-product hydrogen supply chain under the proposed structure are mainly twofold: 1) to explicitly consider possible coalition structures and transport route strategies in the timescale of transport duration, electricity price fluctuation and volatile losses; 2) and to allocate the payoff among the producers in some fairway.

\section{Strategies and decision-making process of stakeholders}

In this section, the decision-making process of each stakeholder is investigated and mathematically modeled under the proposed business model.

\subsection{The Retailer’s Problem}

In the price-setting problem of the salt cavern, the retailer decides its buying price $p_{t}$ (offered to the suppliers), while considering the reactions ${q_{i,t}^{trans}}$ from suppliers. The problem can be formulated as
\setlength{\abovedisplayskip}{3pt}
\begin{align}
    \max\limits_{p_{t}}\ p_{o}\sum_{i=1}^{I}\sum_{t=1}^{T}q_{i,t}^{trans}u_{i,I+1}-\sum_{i=1}^{I}\sum_{t=1}^{T}p_{t}q_{i,t}^{trans}u_{i,I+1}  \label{eq:constraint1}
\end{align}
\begin{align}
    s.t.\ \underline{p}_{t}\le p_{t}\le \overline{p}_{t},\forall t\label{eq:constraint2}
\end{align}
\begin{align}
    \sum\limits_{i=1}^{I}q_{i,t-T_{a}^{i,I+1}}^{trans}\le Q_{trans},\forall t\label{eq:constraint3}
\end{align}

Objective \eqref{eq:constraint1} is the retailer’s profit in which the first term is the selling income, and the second term is the purchasing cost. Inequality \eqref{eq:constraint2} restricts the price offered to suppliers to be within the interval $[\underline{p}_{t},\overline{p}_{t}]$ in each period. Here we assume that the retailer and suppliers have already reached an agreement to bring this constraint into effect. Inequality \eqref{eq:constraint3} prescribes maximal transaction quantity in each period by maximal injection rate of the salt cavern. $q_{i,t}^{trans}$ and $u_{i,I+1}$ are the optimal solution to the suppliers’ problem.

\subsection{The Suppliers’ Problem}

For the suppliers, the planning of the type of processing equipment, transportation mode, and the transport route as well as scheduling of transaction quantity, is formulated in this subsection. To capture the dynamic process of hydrogen transactions between each stakeholder in detail, as well as investigating dynamic strategic behaviors of each stakeholder, the loading process is elaborately taken into consideration. Specifically, hydrogen is produced as a by-product along with main products and has three possible disposal ways: 
\begin{enumerate}
    \item Hydrogen can be loaded to a tube trailer (or a tanker truck) after compression (or liquefaction). At the end of period $t$, tube trailers (or tanker trucks) filled to maximum capacity should depart from chemical plants. Otherwise, they stay until filled up in the following periods. Therefore, the transaction quantity sequence $q_{i,t}^{trans}$ depends on hydrogen processing quantity sequence $q_{i,t}^{pr}$ and capacity of the vehicle ($Q_\mathcal{C}$ for a tube trailer and $Q_\mathcal{D}$ for a tanker truck).
    \item Hydrogen can also be temporarily stored in low-pressure storage tanks before liquefication or achieving an adequate compression rate. It will further be loaded into tube trailers (or tanker trucks) after being compressed (or liquified) in the following periods.
    \item Hydrogen may also be discarded by being emitted or burnt as the current practice, which may happen when buying price offered by the salt cavern is too low or low-pressure storage tanks are filled up.
\end{enumerate}

The above three disposal ways offer multiple options for chemical plants during planning and scheduling. For example, a chemical plant with a generation volume of 100kg per hour, may purchase processing equipment of 100kg per hour. Thus, hydrogen can be processed hour-by-hour. An alternative is to purchase processing equipment of 1000kg per hour. In this case, by-product hydrogen can be temporarily stored in low-pressure storage tanks and will be processed every 10 hours. The suppliers’ problem is to find optimal solutions for planning and scheduling while considering possible coalitions with each other.

In the suppliers’ problem, if the destinations of all suppliers for hydrogen shipment are the salt cavern, decision variables should be the type of processing equipment $\boldsymbol{x_{i}}$ and hydrogen processing amount $q_{i,t}^{pr}$; if the scenario of coalitions of suppliers is taken into account, transport route $u_{i,j}$ of chemical plants $i$ and $j$, which form a coalition.

The decision-making problem, including constraints and objectives of supplier $i$, is given as follows.
\subsubsection{Constraints on transit shipment pattern}
\begin{gather}
    \sum\limits_{j=1}^{I+1}u_{i,j}=1,\forall i\label{eq:constraint4}\\
    u_{i,j}+u_{j,i}\le 1,\forall i,j \label{eq:constraint5}
\end{gather}

Constraint \eqref{eq:constraint4} denotes that the destination of each chemical plant is unique. Constraint \eqref{eq:constraint5} defines that any pairs of the chemical plant $(i,j)$ wouldn’t select each other as the transit destination simultaneously.

\subsubsection{Constraints on hydrogen processing and transport scheduling}

Chemical plants adopting CH2 satisfy: 
\begin{gather}
    n_{i,t}^{cars}\le(q_{i,t}^{pr}+q_{i,t-1}^{store})/Q_c^{car}\le n_{i,t}^{cars}+1,\forall t \label{eq:constraint6}\\
    q_{i,t}^{trans}=n_{i,t}^{cars}Q_\mathcal{C},\forall t \label{eq:constraint7}\\
    q_{i,t}^{store}=q_{i,t-1}^{store}+q_{i,t}^{pr}-q_{i,t}^{trans}, \forall t\in\{2,...T\} \label{eq:constraint8}
\end{gather}

Constraints \eqref{eq:constraint6} and \eqref{eq:constraint7} indicate that hydrogen transaction amount in each period is an integer multiple of the capacity of a tube trailer since only tube trailers filled to maximum capacity will depart from chemical plants. Constraint \eqref{eq:constraint8} denotes variations of hydrogen quantity stored in low-pressure storage tanks.

With the remaining proportion of hydrogen after being shipped from chemical plant $i$ to $j$ ($j= I+1$ represents the salt cavern) written as $\beta_{L2}^{i,j}=\beta_{L2} T_{a}^{i,j}$, chemical plants adopting LH2 satisfy
\begin{gather}
    n_{i,t}^{cars}\le(q_{i,t}^{pr}+\beta_{L1}q_{i,t-1}^{store})/Q_d^{car}\le n_{i,t}^{cars}+1,\forall t \label{eq:constraint9}\\  
    q_{i,t}^{trans}=n_{i,t}^{cars}Q_\mathcal{D}\sum_{j=1}^{I+1}u_{i,j}\beta_{L2}^{i,j},\forall t \label{eq:constraint10}
\end{gather}
\setlength{\abovedisplayskip}{-10pt}
\begin{multline}
    q_{i,t}^{store}=\beta_{L1}q_{i,t-1}^{store}+q_{i,t}^{pr}-{q_{i,t}^{trans}}/{\sum_{j=1}^{I+1}u_{i,j}\beta_{L2}^{i,j}},\\ \forall t \in \{2,...T\} \label{eq:constraint11}
\end{multline}

Constraints \eqref{eq:constraint9} and \eqref{eq:constraint10} indicate that the hydrogen transaction amount in each period is an integer multiple of the capacity of a tanker truck. Constraint \eqref{eq:constraint11} denotes variations of hydrogen quantity stored in low-pressure storage tanks.

Constraints irrelevant to transportation modes are given in \eqref{eq:constraint12}-\eqref{eq:constraint15}, in which the transport duration for chemical plant $i$ is written as $t_{ar}^{i}=\sum_{j=1}^{I+1}u_{i,j}T_{a}^{i,j}$.
\setlength{\abovedisplayskip}{3pt}
\begin{gather}
{\sum_{t - 2\times t_{ar}^{i}}^{t}n_{i,t}^{cars}} \leq N_{i}^{cars}, \forall t \in \left\{2\times t_{ar}^{i},...T \right\} \label{eq:constraint12} \\
q_{i,t}^{pr} \leq {\sum_{n = 1}^{N_{\mathcal{C}} + N_{\mathcal{D}}}x_{i}^{n}}Q_{type}^{n}, \forall t \label{eq:constraint13} \\
q_{i,t}^{unpr} \leq \sum_{n = 1}^{N_{\mathcal{C}} + N_{\mathcal{D}}} x_{i}^{n}Q_{type}^{n}, \forall t \label{eq:constraint14}
\end{gather}
\setlength{\abovedisplayskip}{-3pt}
\begin{multline}
q_{i,t}^{unpr} \leq q_{i,t - 1}^{unprocess} + Q_{i,t} - q_{i,t}^{pr} + {\sum_{j = 1}^{I}{u_{j,i}q_{j,trans}^{t - T_{a}^{i,j}}}},\\\forall t \in \left\{ \max{({1,T_{a}^{i,j}})},\ldots,T \}\right. \label{eq:constraint15} 
\end{multline}
\setlength{\belowdisplayskip}{5pt}

Constraint \eqref{eq:constraint12} imposes the total number of tube trailers (or tanker trucks) purchased by chemical plant $i$ as the upper bound of tube trailers (or tanker trucks) in the round trip during the time period $\left\lbrack t - 2\times t_{ar}^{i} \right\rbrack$. Constraint \eqref{eq:constraint13} prescribes the processing capability of each chemical plant. Constraint \eqref{eq:constraint14} restricts the upper bound of hydrogen stored locally, and the bound parameter is chosen as $\sum_{n=1}^{N_\mathcal{C}+N_\mathcal{D}}x_i^nQ_{type}^n$. Constraint \eqref{eq:constraint15} represents variations of hydrogen stored locally, in which `$\le$' indicates that hydrogen as a by-product can be stored temporarily or directly discarded.

If destinations of all suppliers for hydrogen shipment are the salt cavern, the objective of each chemical plant is to maximize its daily profit and is given in \eqref{eq:constraint16}, in which the income by selling hydrogen to the consumers, initial investment cost and operation cost are considered.
\begin{gather}
\max~\pi_{Fi} = {\sum_{t = 1}^{T}( p_{t}q_{i,t}^{trans} - C_{O}^{i} - C_{T}^{i} )} - C_{INV1}^{i} - C_{INV2}^{i} \label{eq:constraint16} 
\end{gather}
where
\begin{gather}
% \setlength{\belowdisplayskip}{8pt}
C_{O}^{i} = q_{i,t}^{pr}{\sum\limits_{n = 1}^{N_{\mathcal{C}} + N_{\mathcal{D}}}x_{i}^{n}}w_{t}\left( \gamma_{c}x_{i}^{c} + \gamma_{d}x_{i}^{d} \right) \label{eq:constraint17}\\
C_{T}^{i} = n_{i,t}^{cars}{\sum_{j = 1}^{I + 1}{u_{i,j}{K_{3}T}_{a}^{i,j}}} \label{eq:constraint18}\\
C_{INV1}^{i} = {\sum_{n = 1}^{N_{\mathcal{C}} + N_{\mathcal{D}}}{x_{i}^{n}K_{1}^{n}}} \label{eq:constraint19}\\
C_{INV2}^{i} = N_{i}^{cars}{({x_{i}^{c}K_{2}^{c} + x_{i}^{d}K_{2}^{d}})} \label{eq:constraint20}
\end{gather}
where transportation mode is written as $x_i^c=\sum_{n=1}^{N_\mathcal{C}}x_i^n$ and $x_i^d=\sum_{n=N_\mathcal{C}}^{N_\mathcal{C}+N_\mathcal{D}}x_i^n$; $C_O^i , C_T^i$ represent hourly processing and transport cost respectively; $C_{INV1}^i$ , $C_{INV2}^i$ represent investment cost of processing equipment and tube trailers (or tanker trucks) after converted into daily cost with a discount rate, respectively.
If the scenario where coalitions of suppliers are considered, we denote chemical plants in a coalition as $\Gamma$. For the chemical plant $i$, $\forall i\in\Gamma$, the objective is to maximize the daily profit of the coalition and is given as \eqref{eq:constraint21}.
\setlength{\belowdisplayskip}{6pt}
\begin{multline}
\max~~\pi_{F\tau}=\sum_{i \in \Gamma}{{\sum_{t = 1}^{T}\left( p_{t}q_{i,t}^{trans}u_{i,I + 1} - C_{O}^{i} - C_{T}^{i} \right)}} \\{ - C_{INV1}^{i} - C_{INV2}^{i}} \label{eq:constraint21}
\end{multline}
where $u_{i,I+1}=1$ when chemical plant $i$ is chosen as a transit hub. Otherwise $u_{i,I+1}=0$.


\section{Game formulation and solution}
\subsection{Game Formulation for By-product Hydrogen Supply Chain}

In this section, the by-product hydrogen market is formulated as a game, considering the individual rationality of each stakeholder. 

The decision-making process of each individual can be concluded as follows. The suppliers plan their initial equipment investment, coalition structure and transport routes in the planning stage. Then, the hydrogen transaction problem, including the retailer's pricing problem and suppliers' scheduling problem, is optimized in the scheduling stage.

The overall framework of the game models is illustrated in Fig.\ref{fig:Game models involved in by-product hydrogen supply chain including salt cavern and chemical plants}. Specifically, the planning problem of multiple chemical plants is formulated as a cooperative game, in which a binding coalition could be formed to reduce transport costs. The hydrogen transaction problem between the salt cavern and chemical plants is formulated as a Stackelberg game, in which the salt cavern is the leader and chemical plants are the followers.

\begin{figure}[] %可选参数 h t b p,代表允许图片出现的位置,h表示此处附近,t表示顶部,b表示底部,p表示单独一页,H表示固定此处
    \centering
    \includegraphics[width=8.5cm]{fig/Fig.4_Game_models_involved_in_by-product_hydrogen_supply_chain_including_salt_cavern_and_chemical_plants.png}
    \caption{Game models involved in by-product hydrogen supply chain including salt cavern and chemical plants} \label{fig:Game models involved in by-product hydrogen supply chain including salt cavern and chemical plants}
\end{figure}
\subsubsection{Coorperative game in the planning stage} \label{subsubsection: first-stage problem}
As previously analyzed in subsection \ref{subsection: Characteristics of by-product hydrogen supply chain}, coalitions between chemical plants would potentially lower transportation costs, thus bringing collective payoffs. Moreover, to fairly allocate the payoff $\pi_{F\tau}$ among the players, the Shapley value is adopted.

The following assumptions are made without loss of generality when considering possible coalition structures:

i) Chemical plants in each coalition select one of them as a transit hub to which other chemical plants in the coalition transport hydrogen. Since reducing transport costs is considered as the key factor behind the coalition, we assume that two chemical plants destined for the salt cavern lack the motivation to form a coalition.

ii)	The influence of hydrogen price variations on the coalition structure is neglected since the salt cavern's buying price is unknown at the planning stage. Moreover, the driving force in forming a coalition is to reduce costs rather than to increase the selling income.

Generally, the planning problem of chemical plants is based on the cooperative game, where players are the chemical plants. For chemical plant $i$, decision variables are the type of processing equipment, $\textit{\textbf{x}}_\textit{\textbf{i}}=\left\{x_i^n\right\},\forall n$, hydrogen processing amount $q_{i,t}^{pr}$ and transport route $u_{i,j}$ of chemical plants in the coalition. Payoffs are described as \eqref{eq:constraint16} and \eqref{eq:constraint21} for self-sufficient chemical plants and coalitions respectively.

Note that in the planning stage, the optimal solution $q_{i,t}^{pr}$ is to roughly estimate operation cost under different transportation mode and processing equipment type decisions, thus helping the decision of transport route $u_{i,j}$. Therefore, the solution of $q_{i,t}^{pr}$ here neglects the influence of hydrogen price variations. Actual hydrogen processing quantity sequence $q_{i,t}^{pr}$ will be obtained by equilibrium analysis in the scheduling stage.

\subsubsection{Stackelberg game in the scheduling stage}
The problem in the scheduling is the hydrogen transaction problem between the retailer and the suppliers. After formulating transport route decisions of suppliers as a cooperative game, the interaction between the salt cavern and multiple chemical plants is formulated as a Stackelberg game, where the salt cavern is the leader, whose strategy is the TOU hydrogen price, and chemical plants are followers, whose strategies are hourly transaction. 

At this stage, the retailer’s and suppliers’ problem can be formulated as a bilevel optimization. The retailer determines the hydrogen price sequence $v_t$ in the upper level, and the suppliers decide their optimal transaction pattern $q_{i,t}^{trans}$ in the lower level, with respect to the hydrogen price sequence $v_t$. The optimal transaction pattern $q_{i,t}^{trans}$ would in turn influences hydrogen price sequence $v_t$ determined by the retailer in the upper level. Assume that the information of each chemical plant, such as transit transport routes, processing equipment type and by-product hydrogen generation quantities, are accessible to the salt cavern. Therefore, the optimal solution of $q_{i,t}^{trans}$ can be predicted by the salt cavern under any given hydrogen price sequence $v_t$. The suppliers’ dispatching problem \eqref{eq:constraint3}-\eqref{eq:constraint21} can be regarded as constraints of the retailer’s pricing problem.

According to the analysis above, the interactions between the salt cavern and the chemical plants constitute a Stackelberg competition. In this competition, the salt cavern is the leader, whose strategy is the TOU hydrogen price sequence. Chemical plants are the followers, whose strategy is the hourly hydrogen transaction quantity. The leader’s pricing problem maximizes its profit, subject to the bounds of hydrogen price (Eq.\eqref{eq:constraint2}) and maximal injection rate (Eq.\eqref{eq:constraint3}). The followers’ scheduling problem maximizes individual profits or coalition profits, subject to constraints given in \eqref{eq:constraint9}-\eqref{eq:constraint18}.
\subsection{Solution of the Problem}
In this section, we introduce the solution of the game formulation of the by-product hydrogen supply chain. 

Tractable reformulations of the suppliers’ problem are made to efficiently calculate the equilibrium in the lower level for both the planning and scheduling problems. Specifically, for the suppliers’ problem in both stages, the objective of each individual player (or coalition) is irrelevant to the strategies of other individual players (or coalitions), while the strategy set is influenced by the strategies of other individual players (or coalitions). According to the potential game theory, the suppliers' problem can be regarded as a potential game. The sum of the objectives of each individual player (or coalition) is the potential function. Besides, the pure-strategy equilibrium exists in the transport route planning problem of the suppliers since there exists at least one pure-strategy equilibrium in an infinite potential game. Thus, the suppliers’ problem is formulated as a potential game that can be solved as an optimization problem.

After the reformulation of the suppliers' problem, the planning stage problem is reformulated to a mixed integer nonlinear program (MINLP) with ${\textit{\textbf{x}}_\textit{\textbf{i}},u_{i,j}, q_{i,t}^{pr},N_i^{cars}},\forall i\in{1,\ldots I}$ as decision variables, \eqref{eq:constraint22} as the objective and \eqref{eq:constraint4}-\eqref{eq:constraint15} as constraints. Commercial solvers such as Baron can be used to solve the problem. The solved optimal strategy $\textit{\textbf{x}}_\textit{\textbf{i}}$ and $u_{i,j}$ will be adopted at the scheduling stage.
\setlength{\abovedisplayskip}{3pt}
\begin{multline}
\max~~\pi_{F}~ = ~\sum_{i=1}^{I}{{\sum_{t = 1}^{T}\left( p_{t}q_{i,t}^{trans}u_{i,I + 1} - C_{O}^{i} - C_{T}^{i} \right)}}\\{-C_{INV1}^{i} - C_{INV2}^{i}}  \label{eq:constraint22}
\end{multline}

To solve the bi-level problem at the scheduling stage, Genetic Algorithm (GA) is adopted. First, for the salt cavern in the upper level, pieces of hydrogen price sequences are generated and regarded as individuals. Second, to acquire the fitness of each individual, the suppliers’ scheduling problems in the lower level are solved. Since the transit transport routes $\textit{\textbf{x}}_\textit{\textbf{i}}$ and the processing equipment type $u_{i,j}$ are known at the scheduling stage, the suppliers’ problem becomes a mixed-integer linear program (MILP), which can be solved efficiently by off-the-shelf commercial solvers. Thus, daily profits of the salt cavern, considering the best response of the suppliers, can thus be calculated and regarded as finesses for given price sequences. 

\section{Case Study}
To validate the effectiveness of the proposed model and algorithm, numeric experiments on a by-product hydrogen supply chain composed of three chemical plants and a salt cavern are carried out. All of the following tests are conducted on PCs with Intel Xeon W-2255 processor, 3.70 GHz primary frequency, and 128GB memory. CPLEX 2.16 is used to solve related MILP problems.

\subsection{System Configuration}
Scenario parameters of the envisioned by-product hydrogen supply chain are given in Table \ref{tab:Scenario parameters}. $Q_{i,t}$ are hydrogen generation sequences of a typical day produced by a Gaussian distribution with a mean value of 1000 for the 1st chemical plant (1500 for the 2nd and 3000 for the 3rd) and a variance of 100. Moreover, in the envisioned by-product hydrogen supply chain, $\boldsymbol{Q_{pr}}$ are a vector consisting of 1200, 2000, 4000 and 8000, the first two and the last two of which are the compressor capacity and liquefier capacity to choose from, respectively. Parameters of different processing equipment and transportation modes refer to \cite{HAN20125328} and \cite{Argonne2021} and are given in Table \ref{tab:Parameters of hydrogen transportation}. $\boldsymbol{K_{1}}$ are a vector consisting of 774.29, 126612, 18977.17 and 34757.99, corresponding to each element in $\boldsymbol{Q_{pr}}$. Note that the time scale involved in the problem is one day. Initial investment costs of the liquefier, the compressor, and the transportation vehicles are converted into daily investment costs with a discount rate. The operation cost of a tube trailer (or a tanker truck) in each period includes fuel price, driver wage, and maintenance expenses. 

\begin{table}[h]
\centering
\caption{Scenario parameters of the by-product hydrogen supply chain}
% \captionsetup{font={footnotesize}}
% \resizebox{0.4\textwidth}{!}{
\label{tab:Scenario parameters}
\footnotesize
\begin{tabular}{lllll}
\hline\toprule
\multicolumn{5}{l}{Parameters}                                                           \\ \hline
$I$                      & \multicolumn{2}{l}{3}  & $N_\mathcal{D}$     & 2                             \\
$T$                      & \multicolumn{2}{l}{12} & $T_{a}$     & {[}0,0,0,4;0,0,0,4;0,0,0,4{]} \\
\multicolumn{1}{c}{$p_{o}$} & \multicolumn{2}{l}{15} & $\underline{p}_{t},\overline{p}_{t}$  & 5/13                          \\
$N_\mathcal{C}$                     & \multicolumn{2}{l}{2}  & $Q_{trans}$ & 9000                          \\ \hline
\end{tabular}
% }
\end{table}

\begin{table}[]
\centering
\caption{Parameters of hydrogen transportation} \label{tab:Parameters of hydrogen transportation}
\footnotesize
\begin{tabular}{lllllll}
\hline\toprule
\multicolumn{7}{l}{Parameters}                                                                             \\ \hline
$Q_\mathcal{C}$                        & \multicolumn{4}{l}{200}          & $K_{3}(\$/h)$       & {[}0,0,0,4;0,0,0,4;0,0,0,4{]} \\
$Q_\mathcal{D}$                        & \multicolumn{4}{l}{4000}         & $\beta_{L1}$ & 5/13                          \\
\multicolumn{1}{c}{$\gamma_{c}/\gamma_{d}(kwh/kg)$} & \multicolumn{4}{l}{1/8.18}       & $\beta_{L2}$ & 9000                          \\
$K_{2}^c/K_{2}^d(\$)$                      & \multicolumn{4}{l}{82.20/219.18} &          & \multicolumn{1}{c}{}          \\ \hline
\end{tabular}
\end{table}
\subsubsection{Equilibrium of possible coalition structures of the suppliers}
With three chemical plants, there are five possible coalition structures: no cooperation, cooperation between two players with the third being self-sufficient (there are three ways this could occur) and complete cooperation among all the three chemical plants. The benefits of individual participants or coalitions are shown in Table \ref{tab:Participants/alliance}, in which $M$ represents the benefit, and the benefit of each chemical plant and the sum of them are denoted by $M_{1},M_{2},M_{3}$ and $M_{total}$ respectively. ‘\{\}’ indicates a cooperation, and the chemical plant serving as the transit hub is marked by a ‘*’. 

% Please add the following required packages to your document preamble:
% \usepackage{multirow}
\begin{table}[]
\centering
\caption{Participants/alliance optimal income under non-cooperative and cooperative game models} \label{tab:Participants/alliance}
\footnotesize
\begin{tabular}{llll}
\hline\toprule
\multirow{2}{*}{Number} &
  \multirow{2}{*}{\begin{tabular}[c]{@{}l@{}}Coalition\\ structure\end{tabular}} &
  \multicolumn{2}{c}{Profits(\$/day)} \\ \cline{3-4} 
  &                & \begin{tabular}[c]{@{}l@{}}Individual or\\ a coalition\end{tabular}        & $M_{total}$ \\ \hline
1 &
  \{1\},\{2\},\{3\} &
  \begin{tabular}[c]{@{}l@{}}$M_{1} = 54052$\\ $M_{2} = 81060$\\ $M_{3} = 236814$\end{tabular} &
  371926 \\
2 & \{1,2*\},\{3\} & \begin{tabular}[c]{@{}l@{}}$M_{\{1,2\}}$ = 170589\\ $M_3$ = 236814\end{tabular} & 407403   \\
3 & \{1,3*\},\{2\} & \begin{tabular}[c]{@{}l@{}}$M_{\{1,3\}}$ = 286531\\ $M_2$ = 107868\end{tabular} & 394399   \\
4 & \{1\},\{2,3*\} & \begin{tabular}[c]{@{}l@{}}$M_1$ = 53562\\ $M_{\{2,3\}}$ = 323154\end{tabular}  & 376716   \\
5 & \{1,2,3*\}     & $M_{\{1,2,3\}}$ = 383925                                                       & 383925   \\ \hline
\end{tabular}
\end{table}

It can be analyzed from Table \ref{tab:Participants/alliance} that:

i)	In the 1st coalition structure with no cooperation at all, the total benefit of the three chemical plants is the lowest among all coalition structures, indicating a potential collective payoff gained by forming coalitions between chemical plants. 

ii)  In the 3rd coalition structure, the benefit of the coalition $\{1, 3^{*}\}$ denoted as $M_{\{1,3^{*}\}}$ equals to 286531 and is lower than the sum of benefits that they could get on their own, which is calculated as $M_{1}+M_{3}=290866$, violating collective rationality.

iii)  In the 5th coalition structure, although collective benefit is higher than the sum of benefits each coalition member could get on their own, the total benefit of the 5th coalition structure $M_{total}\{1,2,3^{*}\}$ is lower than that of the 2nd coalition structure $M_{total}(\{1,2^{*}\},\{3\})$. Therefore, the grand coalition is not stable since there is a preferred alternative. The analysis of the 4th coalition structure is analogous.

iv)  In the 2nd coalition structure, $M_{\{1,2^{*}\}}$, the benefit of the coalition $\{1,2^{*}\}$, equals to 170589 and is higher than the sum of benefits they could get on their own, which satisfies $M_{1}+M_{2}=135112$. Moreover, the total benefit of the 2nd coalition structure is the highest among the five possible structures, so there exists no preferred alternatives. Therefore, the coalition of the chemical plants $\{1,2^{*}\}$ is stable.

The insights provided by different coalition structures above is that for several chemical plants in proximity to each other, those chemical plants with low or medium generation scale (chemical plant 1 and 2 in our case) tends to form a coalition, and to compete with those with larger generation scale.

In order to realize a fair imputation of the collective payoff of chemical plants $\{1,2^{*}\}$, the Shapley value is adopted. The allocation result is {71790.5,98798.5}\$, which is higher than the benefit they could get on their own, which are \{\$54052, \$81060\}. The coalition between chemical plant 1 and 2 increase their profits by 24.7\% and 18.0\% respectively.

\subsection{Equilibrium of Hydrogen Pricing and Scheduling}

In this case, the fixed price at which consumers purchase is set as 15 \$/kg. The equilibrium of the buying price offered by the salt cavern $p_t$ and the hydrogen transaction quantity $q_{i,t}^{pr}$ are illustrated in Fig.5. The minimal price takes value at its lower bound 5\$/kg, and the maximal value is 11.9 \$/kg .
\begin{figure}[] %可选参数 h t b p,代表允许图片出现的位置,h表示此处附近,t表示顶部,b表示底部,p表示单独一页,H表示固定此处
    \centering
    \includegraphics[width=7cm]{fig/Fig_5._Hydrogen_price_of_salt_cavern_and_transaction_quantity_of_chemical_plant.png}
    \caption{Hydrogen price of salt cavern and transaction quantity of chemical plant} \label{fig:Hydrogen price of salt cavern and transaction quantity of chemical plant}
\end{figure}
It can be observed from Fig.\ref{fig:Hydrogen price of salt cavern and transaction quantity of chemical plant} that the variation trend of the hydrogen transaction quantity goes with the buying price. The higher the buying price, the higher the transaction quantity. This can be attributed to the storage capacity of chemical plants, which can temporarily store by-product hydrogen in low-pressure storage tanks or tube trailers (or tanker trucks) before filled to maximal capacity. Therefore, the chemical plants can choose to sell hydrogen at a higher price.

Moreover, due to the influence of the TOU electricity price, the operating cost of the processing equipment fluctuates. The TOU electricity price and the equilibrium of the total processing quantity are plotted in Fig.\ref{fig:Time of use electricity price and processing mass of chemical plant}.

\begin{figure}[] %可选参数 h t b p,代表允许图片出现的位置,h表示此处附近,t表示顶部,b表示底部,p表示单独一页,H表示固定此处
    \centering
    \includegraphics[width=7.5cm]{fig/Fig_6._Time_of_use_electricity_price_and_processing_mass_of_chemical_plant.png}
    \caption{Time of use electricity price and processing mass of chemical plant} \label{fig:Time of use electricity price and processing mass of chemical plant}
\end{figure}

It can be observed from Fig.\ref{fig:Time of use electricity price and processing mass of chemical plant} that the variation trend of the hydrogen processing quantity and the TOU electricity price go oppositely. This is because chemical plants tend to process hydrogen when the electricity price is low, thus reducing the processing cost of hydrogen.

According to the above results, it can be noted that the equilibrium of salt cave pricing encourages chemical plants to process and trade hydrogen when the electricity price is lower. As a result, the salt cavern can purchase hydrogen with lower processing cost, thus reducing the purchase cost of hydrogen per unit. For chemical plants, the hydrogen price is higher during 1-2 periods after periods with lower electricity prices than in other periods, thus reducing the hydrogen processing cost.

The result of profits and total transaction quantities are plotted in Fig.\ref{fig:The result of profits and total transaction quantities with time-invariant hydrogen price} considering different fixed prices. The optimal price offered by the salt cavern is about 9\$/kg, and its profit is \$287884.8 for a day. However, the profit of the salt cavern reaches to \$343947.16 at the optimal TOU hydrogen price. Hence, a TOU hydrogen price strategy for the salt cavern increases its profit by 19.5\%.

\begin{figure}[] %可选参数 h t b p,代表允许图片出现的位置,h表示此处附近,t表示顶部,b表示底部,p表示单独一页,H表示固定此处
    \centering
    \includegraphics[width=7.5cm]{fig/Fig.7_The_result_of_profits_and_total_transaction_quantities_with_time-invariant_hydrogen_price.png}
    \caption{The result of profits and total transaction quantities with time-invariant hydrogen price}\label{fig:The result of profits and total transaction quantities with time-invariant hydrogen price}
\end{figure}

Generally, the equilibrium of the Stackelberg game between the salt cavern and the chemical plants benefits all the players. It also indicates the positive response of the salt cavern and chemical plants to TOU electricity price, and reflects the role of chemical plants in peak shaving and valley filling, which benefits the safe and stable operation of power grid.

\subsection{Sensitivity Analysis}
\subsubsection{Impact of per period transportation operation cost}

The reduction in operation cost of a tube trailer (or a tanker truck) per period $K_{3}$ reduces the transport cost, thus bringing down the collective payoff brought by coalitions of chemical plants. Based on the first assumption in section \ref{subsubsection: first-stage problem}, each coalition must take one of them as a transit hub, and two chemical plants destined for the salt cavern lack the motivation to form a coalition. Consequently, the collective payoff declines as the transport cost reduces, until collective rationality no longer holds when the benefits of the coalition are less than the sum of benefits each individual could get on their own. As shown in Fig.\ref{fig:Impact of running cost of single vehicle of single period on the profit of chemical plant 1 and 2}, when $K_{3}$ decreases from \$390 to \$382, the sum of benefits of chemical plant 1 and 2 under the equilibrium of the 1st and 2nd coalition structure, denoted by $M_{total}^{\{1,2\}},M_{total}^{\{1\},\{2\}}$ respectively, gradually increases. 

\begin{figure}[] %可选参数 h t b p,代表允许图片出现的位置,h表示此处附近,t表示顶部,b表示底部,p表示单独一页,H表示固定此处
    \centering
    \includegraphics[width=7cm]{fig/Fig_8._Impact_of_running_cost_of_single_vehicle_of_single_period_on_the_profit_of_chemical_plant_1_and_2.png}
    \caption{The result of profits and total transaction quantities with time-invariant hydrogen price}\label{fig:Impact of running cost of single vehicle of single period on the profit of chemical plant 1 and 2}
\end{figure}

As illustrated in Fig.\ref{fig:Impact of running cost of single vehicle of single period on the profit of chemical plant 1 and 2}, the coalition benefit is more sensitive to $K_3$ than individual benefits. When $K_3$ decreases to about \$386, the coalition $\{1,2^{*}\}$ no longer bring additional benefits to individuals, resulting in a breakdown of the coalition. 
\subsubsection{Impact of maximal injection rate of the salt cavern}

The maximal injection rate $Q_{trans}$ of the salt cavern directly limits the total transaction quantity per period between the salt cavern and the chemical plants. Table \ref{tab:Individual income} demonstrates the impact of $Q_{trans}$ to the equilibrium of the second-stage problem.

\begin{table}[]
\centering
\caption{Individual income of the equilibrium under different maximum transportation quality of salt cavern gas pipeline in single period} \label{tab:Individual income}
\footnotesize
\begin{tabular}{lllll}
\hline\toprule
\multirow{3}{*}{$Q_{trans}$} &
  \multirow{3}{*}{$M_{total}^{\{1,2\}}$/kg} &
  \multirow{3}{*}{$M_{total}^{\{3\}}$/kg} &
  \multirow{3}{*}{\begin{tabular}[c]{@{}l@{}}$M_{total}$\\(chemical\\ plants)/\$\end{tabular}} &
  \multirow{3}{*}{\begin{tabular}[c]{@{}l@{}}$M_{total}$\\(the salt\\\ cavern)/\$\end{tabular}} \\
      &                              &          &          &           \\
      &                              &          &          &           \\ \hline
12000 & 23103.96                     & 21387.07 & 44491.03 & 342814.83 \\
9000  & \multicolumn{1}{c}{26203.97} & 22762.82 & 48966.80 & 325379.79 \\
6000  & 10528.29                     & 1139.11  & 11667.40 & 278396.59 \\ \hline
\end{tabular}
\end{table}

It can be analyzed from Table \ref{tab:Individual income} that $Q_{trans}$ has different impacts on the participants: the daily income of chemical plants does not necessarily increase with the increase of $Q_{trans}$, whereas the daily income of the salt cavern increases with the increase of $Q_{trans}$. Therefore, the salt cavern will be motivated to determine an appropriate $Q_{trans}$ according to the generation scale of by-product hydrogen of the chemical plants so as to increase individual benefits.

\section{Conclusion}
This paper proposes an equilibrium model of a by-product hydrogen market with the salt cavern as the retailer and chemical plants as the suppliers. A business model for large-scale storage to acquire by-product hydrogen from chemical plants and sell them to end-users is established for the first time. The decision-making process of each stakeholder, i.e., chemical plants and the salt cavern, is investigated and mathematically modeled considering different transportation modes, locations of chemical plants and TOU electricity price. To consider the individual rationality of each stakeholder, the by-product hydrogen market is formulated as games. The transport route planning problem between multiple chemical plants is formulated as a cooperative game. The hydrogen transaction problem between the salt cavern and chemical plants is formulated as a Stackelberg game. Numeric experiments on a by-product hydrogen supply chain composed of three chemical plants and a salt cavern are carried out. The results show that a coalition between chemical plants potentially increases their profits. Moreover, the adoption of TOU hydrogen price in a Stackelberg formulation also increases the profit of the salt cavern. The proposed business model and the optimization of the by-product hydrogen supply chain management not only presents a new revenue stream for both chemical plants and salt caverns but increases resource efficiency and accelerates energy conversion.





% \section*{Appendix A}
% \vspace{-0.2cm}
% \section*{Proof of Proposition 1}



% \section*{Appendix B}
% \vspace{-0.2cm}
% \section*{Proof of Proposition 2}



\bibliographystyle{IEEEtran}
\bibliography{ref}
\end{document}







%aipnum4-2.bst 2019-01-14 (MD) hand-edited version of apsrev4-1.bst
%Control: key (0)
%Control: author (8) initials jnrlst
%Control: editor formatted (1) identically to author
%Control: production of article title (0) allowed
%Control: page (1) range
%Control: year (1) truncated
%Control: production of eprint (0) enabled
\begin{thebibliography}{36}%
\makeatletter
\providecommand \@ifxundefined [1]{%
 \@ifx{#1\undefined}
}%
\providecommand \@ifnum [1]{%
 \ifnum #1\expandafter \@firstoftwo
 \else \expandafter \@secondoftwo
 \fi
}%
\providecommand \@ifx [1]{%
 \ifx #1\expandafter \@firstoftwo
 \else \expandafter \@secondoftwo
 \fi
}%
\providecommand \natexlab [1]{#1}%
\providecommand \enquote  [1]{``#1''}%
\providecommand \bibnamefont  [1]{#1}%
\providecommand \bibfnamefont [1]{#1}%
\providecommand \citenamefont [1]{#1}%
\providecommand \href@noop [0]{\@secondoftwo}%
\providecommand \href [0]{\begingroup \@sanitize@url \@href}%
\providecommand \@href[1]{\@@startlink{#1}\@@href}%
\providecommand \@@href[1]{\endgroup#1\@@endlink}%
\providecommand \@sanitize@url [0]{\catcode `\\12\catcode `\$12\catcode
  `\&12\catcode `\#12\catcode `\^12\catcode `\_12\catcode `\%12\relax}%
\providecommand \@@startlink[1]{}%
\providecommand \@@endlink[0]{}%
\providecommand \url  [0]{\begingroup\@sanitize@url \@url }%
\providecommand \@url [1]{\endgroup\@href {#1}{\urlprefix }}%
\providecommand \urlprefix  [0]{URL }%
\providecommand \Eprint [0]{\href }%
\providecommand \doibase [0]{https://doi.org/}%
\providecommand \selectlanguage [0]{\@gobble}%
\providecommand \bibinfo  [0]{\@secondoftwo}%
\providecommand \bibfield  [0]{\@secondoftwo}%
\providecommand \translation [1]{[#1]}%
\providecommand \BibitemOpen [0]{}%
\providecommand \bibitemStop [0]{}%
\providecommand \bibitemNoStop [0]{.\EOS\space}%
\providecommand \EOS [0]{\spacefactor3000\relax}%
\providecommand \BibitemShut  [1]{\csname bibitem#1\endcsname}%
\let\auto@bib@innerbib\@empty
%</preamble>
\bibitem [{\citenamefont {Kosevich}, \citenamefont {Ivanov},\ and\
  \citenamefont {Kovalev}(1990)}]{kosevichMagneticSolitons1990}%
  \BibitemOpen
  \bibfield  {author} {\bibinfo {author} {\bibfnamefont {A.~M.}\ \bibnamefont
  {Kosevich}}, \bibinfo {author} {\bibfnamefont {B.~A.}\ \bibnamefont
  {Ivanov}},\ and\ \bibinfo {author} {\bibfnamefont {A.~S.}\ \bibnamefont
  {Kovalev}},\ }\bibfield  {title} {\enquote {\bibinfo {title} {Magnetic
  solitons},}\ }\href@noop {} {\bibfield  {journal} {\bibinfo  {journal}
  {Physics Reports}\ }\textbf {\bibinfo {volume} {194}},\ \bibinfo {pages}
  {117--238} (\bibinfo {year} {1990})}\BibitemShut {NoStop}%
\bibitem [{\citenamefont
  {Metlov}(2001)}]{metlovTwodimensionalTopologicalSolitons2001}%
  \BibitemOpen
  \bibfield  {author} {\bibinfo {author} {\bibfnamefont {K.~L.}\ \bibnamefont
  {Metlov}},\ }\href@noop {} {\enquote {\bibinfo {title} {Two-dimensional
  topological solitons in soft ferromagnetic cylinders},}\ }\bibinfo {type}
  {Tech. Rep.}\ (\bibinfo {year} {2001})\BibitemShut {NoStop}%
\bibitem [{\citenamefont {Shinjo}\ \emph {et~al.}(2000)\citenamefont {Shinjo},
  \citenamefont {Okuno}, \citenamefont {Hassdorf}, \citenamefont {Shigeto},\
  and\ \citenamefont {Ono}}]{shinjoMagneticVortexCore2000}%
  \BibitemOpen
  \bibfield  {author} {\bibinfo {author} {\bibfnamefont {T.}~\bibnamefont
  {Shinjo}}, \bibinfo {author} {\bibfnamefont {T.}~\bibnamefont {Okuno}},
  \bibinfo {author} {\bibfnamefont {R.}~\bibnamefont {Hassdorf}}, \bibinfo
  {author} {\bibfnamefont {K.}~\bibnamefont {Shigeto}},\ and\ \bibinfo {author}
  {\bibfnamefont {T.}~\bibnamefont {Ono}},\ }\bibfield  {title} {\enquote
  {\bibinfo {title} {Magnetic {{Vortex Core Observation}} in {{Circular Dots}}
  of {{Permalloy}}},}\ }\href {https://doi.org/10.1126/science.289.5481.930}
  {\bibfield  {journal} {\bibinfo  {journal} {Science}\ }\textbf {\bibinfo
  {volume} {289}},\ \bibinfo {pages} {930--932} (\bibinfo {year}
  {2000})}\BibitemShut {NoStop}%
\bibitem [{\citenamefont {Van~Waeyenberge}\ \emph {et~al.}(2006)\citenamefont
  {Van~Waeyenberge}, \citenamefont {Puzic}, \citenamefont {Stoll},
  \citenamefont {Chou}, \citenamefont {Tyliszczak}, \citenamefont {Hertel},
  \citenamefont {F{\"a}hnle}, \citenamefont {Br{\"u}ckl}, \citenamefont {Rott},
  \citenamefont {Reiss}, \citenamefont {Neudecker}, \citenamefont {Weiss},
  \citenamefont {Back},\ and\ \citenamefont
  {Sch{\"u}tz}}]{vanwaeyenbergeMagneticVortexCore2006}%
  \BibitemOpen
  \bibfield  {author} {\bibinfo {author} {\bibfnamefont {B.}~\bibnamefont
  {Van~Waeyenberge}}, \bibinfo {author} {\bibfnamefont {a.}~\bibnamefont
  {Puzic}}, \bibinfo {author} {\bibfnamefont {H.}~\bibnamefont {Stoll}},
  \bibinfo {author} {\bibfnamefont {K.~W.}\ \bibnamefont {Chou}}, \bibinfo
  {author} {\bibfnamefont {T.}~\bibnamefont {Tyliszczak}}, \bibinfo {author}
  {\bibfnamefont {R.}~\bibnamefont {Hertel}}, \bibinfo {author} {\bibfnamefont
  {M.}~\bibnamefont {F{\"a}hnle}}, \bibinfo {author} {\bibfnamefont
  {H.}~\bibnamefont {Br{\"u}ckl}}, \bibinfo {author} {\bibfnamefont
  {K.}~\bibnamefont {Rott}}, \bibinfo {author} {\bibfnamefont {G.}~\bibnamefont
  {Reiss}}, \bibinfo {author} {\bibfnamefont {I.}~\bibnamefont {Neudecker}},
  \bibinfo {author} {\bibfnamefont {D.}~\bibnamefont {Weiss}}, \bibinfo
  {author} {\bibfnamefont {C.~H.}\ \bibnamefont {Back}},\ and\ \bibinfo
  {author} {\bibfnamefont {G.}~\bibnamefont {Sch{\"u}tz}},\ }\bibfield  {title}
  {\enquote {\bibinfo {title} {Magnetic vortex core reversal by excitation with
  short bursts of an alternating field.}}\ }\href
  {https://doi.org/10.1038/nature05240} {\bibfield  {journal} {\bibinfo
  {journal} {Nature}\ }\textbf {\bibinfo {volume} {444}},\ \bibinfo {pages}
  {461--464} (\bibinfo {year} {2006})}\BibitemShut {NoStop}%
\bibitem [{\citenamefont {Hertel}\ \emph {et~al.}(2007)\citenamefont {Hertel},
  \citenamefont {Gliga}, \citenamefont {F{\"a}hnle},\ and\ \citenamefont
  {Schneider}}]{hertelUltrafastNanomagneticToggle2007}%
  \BibitemOpen
  \bibfield  {author} {\bibinfo {author} {\bibfnamefont {R.}~\bibnamefont
  {Hertel}}, \bibinfo {author} {\bibfnamefont {S.}~\bibnamefont {Gliga}},
  \bibinfo {author} {\bibfnamefont {M.}~\bibnamefont {F{\"a}hnle}},\ and\
  \bibinfo {author} {\bibfnamefont {C.~M.}\ \bibnamefont {Schneider}},\
  }\bibfield  {title} {\enquote {\bibinfo {title} {Ultrafast {{Nanomagnetic
  Toggle Switching}} of {{Vortex Cores}}},}\ }\href@noop {} {\bibfield
  {journal} {\bibinfo  {journal} {Phys. Rev. Lett.}\ }\textbf {\bibinfo
  {volume} {98}},\ \bibinfo {pages} {117201} (\bibinfo {year}
  {2007})}\BibitemShut {NoStop}%
\bibitem [{\citenamefont {Gaididei}, \citenamefont {Kravchuk},\ and\
  \citenamefont {Sheka}(2010)}]{gaidideiMagneticVortexDynamics2010}%
  \BibitemOpen
  \bibfield  {author} {\bibinfo {author} {\bibfnamefont {Y.}~\bibnamefont
  {Gaididei}}, \bibinfo {author} {\bibfnamefont {V.~P.}\ \bibnamefont
  {Kravchuk}},\ and\ \bibinfo {author} {\bibfnamefont {D.~D.}\ \bibnamefont
  {Sheka}},\ }\bibfield  {title} {\enquote {\bibinfo {title} {Magnetic vortex
  dynamics induced by an electrical current},}\ }\href@noop {} {\bibfield
  {journal} {\bibinfo  {journal} {International Journal of Quantum Chemistry}\
  }\textbf {\bibinfo {volume} {110}},\ \bibinfo {pages} {83--97} (\bibinfo
  {year} {2010})}\BibitemShut {NoStop}%
\bibitem [{\citenamefont {Guslienko}\ \emph {et~al.}(2002)\citenamefont
  {Guslienko}, \citenamefont {Ivanov}, \citenamefont {Novosad}, \citenamefont
  {Otani}, \citenamefont {Shima},\ and\ \citenamefont
  {Fukamichi}}]{guslienkoEigenfrequenciesVortexState2002}%
  \BibitemOpen
  \bibfield  {author} {\bibinfo {author} {\bibfnamefont {K.~Y.}\ \bibnamefont
  {Guslienko}}, \bibinfo {author} {\bibfnamefont {B.~A.}\ \bibnamefont
  {Ivanov}}, \bibinfo {author} {\bibfnamefont {V.}~\bibnamefont {Novosad}},
  \bibinfo {author} {\bibfnamefont {Y.}~\bibnamefont {Otani}}, \bibinfo
  {author} {\bibfnamefont {H.}~\bibnamefont {Shima}},\ and\ \bibinfo {author}
  {\bibfnamefont {K.}~\bibnamefont {Fukamichi}},\ }\bibfield  {title} {\enquote
  {\bibinfo {title} {Eigenfrequencies of vortex state excitations in magnetic
  submicron-size disks},}\ }\href {https://doi.org/10.1063/1.1450816}
  {\bibfield  {journal} {\bibinfo  {journal} {Journal of Applied Physics}\
  }\textbf {\bibinfo {volume} {91}},\ \bibinfo {pages} {8037} (\bibinfo {year}
  {2002})}\BibitemShut {NoStop}%
\bibitem [{\citenamefont {Pribiag}\ \emph {et~al.}(2007)\citenamefont
  {Pribiag}, \citenamefont {Krivorotov}, \citenamefont {Fuchs}, \citenamefont
  {Braganca}, \citenamefont {Ozatay}, \citenamefont {Sankey}, \citenamefont
  {Ralph},\ and\ \citenamefont
  {Buhrman}}]{pribiagMagneticVortexOscillator2007}%
  \BibitemOpen
  \bibfield  {author} {\bibinfo {author} {\bibfnamefont {V.~S.}\ \bibnamefont
  {Pribiag}}, \bibinfo {author} {\bibfnamefont {I.~N.}\ \bibnamefont
  {Krivorotov}}, \bibinfo {author} {\bibfnamefont {G.~D.}\ \bibnamefont
  {Fuchs}}, \bibinfo {author} {\bibfnamefont {P.~M.}\ \bibnamefont {Braganca}},
  \bibinfo {author} {\bibfnamefont {O.}~\bibnamefont {Ozatay}}, \bibinfo
  {author} {\bibfnamefont {J.~C.}\ \bibnamefont {Sankey}}, \bibinfo {author}
  {\bibfnamefont {D.~C.}\ \bibnamefont {Ralph}},\ and\ \bibinfo {author}
  {\bibfnamefont {R.~A.}\ \bibnamefont {Buhrman}},\ }\bibfield  {title}
  {\enquote {\bibinfo {title} {Magnetic vortex oscillator driven by d.c.
  spin-polarized current},}\ }\href@noop {} {\bibfield  {journal} {\bibinfo
  {journal} {Nat Phys}\ }\textbf {\bibinfo {volume} {3}},\ \bibinfo {pages}
  {498--503} (\bibinfo {year} {2007})}\BibitemShut {NoStop}%
\bibitem [{\citenamefont {Mistral}\ \emph {et~al.}(2008)\citenamefont
  {Mistral}, \citenamefont {{van Kampen}}, \citenamefont {Hrkac}, \citenamefont
  {Kim}, \citenamefont {Devolder}, \citenamefont {Crozat}, \citenamefont
  {Chappert}, \citenamefont {Lagae},\ and\ \citenamefont
  {Schrefl}}]{mistralCurrentDrivenVortexOscillations2008}%
  \BibitemOpen
  \bibfield  {author} {\bibinfo {author} {\bibfnamefont {Q.}~\bibnamefont
  {Mistral}}, \bibinfo {author} {\bibfnamefont {M.}~\bibnamefont {{van
  Kampen}}}, \bibinfo {author} {\bibfnamefont {G.}~\bibnamefont {Hrkac}},
  \bibinfo {author} {\bibfnamefont {J.-V.}\ \bibnamefont {Kim}}, \bibinfo
  {author} {\bibfnamefont {T.}~\bibnamefont {Devolder}}, \bibinfo {author}
  {\bibfnamefont {P.}~\bibnamefont {Crozat}}, \bibinfo {author} {\bibfnamefont
  {C.}~\bibnamefont {Chappert}}, \bibinfo {author} {\bibfnamefont
  {L.}~\bibnamefont {Lagae}},\ and\ \bibinfo {author} {\bibfnamefont
  {T.}~\bibnamefont {Schrefl}},\ }\bibfield  {title} {\enquote {\bibinfo
  {title} {Current-{{Driven Vortex Oscillations}} in {{Metallic
  Nanocontacts}}},}\ }\href {https://doi.org/10.1103/PhysRevLett.100.257201}
  {\bibfield  {journal} {\bibinfo  {journal} {Phys. Rev. Lett.}\ }\textbf
  {\bibinfo {volume} {100}},\ \bibinfo {pages} {257201} (\bibinfo {year}
  {2008})}\BibitemShut {NoStop}%
\bibitem [{\citenamefont {Ruotolo}\ \emph {et~al.}(2009)\citenamefont
  {Ruotolo}, \citenamefont {Cros}, \citenamefont {Georges}, \citenamefont
  {Dussaux}, \citenamefont {Grollier}, \citenamefont {Deranlot}, \citenamefont
  {Guillemet}, \citenamefont {Bouzehouane}, \citenamefont {Fusil},\ and\
  \citenamefont {Fert}}]{ruotoloPhaselockingMagneticVortices2009}%
  \BibitemOpen
  \bibfield  {author} {\bibinfo {author} {\bibfnamefont {A.}~\bibnamefont
  {Ruotolo}}, \bibinfo {author} {\bibfnamefont {V.}~\bibnamefont {Cros}},
  \bibinfo {author} {\bibfnamefont {B.}~\bibnamefont {Georges}}, \bibinfo
  {author} {\bibfnamefont {A.}~\bibnamefont {Dussaux}}, \bibinfo {author}
  {\bibfnamefont {J.}~\bibnamefont {Grollier}}, \bibinfo {author}
  {\bibfnamefont {C.}~\bibnamefont {Deranlot}}, \bibinfo {author}
  {\bibfnamefont {R.}~\bibnamefont {Guillemet}}, \bibinfo {author}
  {\bibfnamefont {K.}~\bibnamefont {Bouzehouane}}, \bibinfo {author}
  {\bibfnamefont {S.}~\bibnamefont {Fusil}},\ and\ \bibinfo {author}
  {\bibfnamefont {A.}~\bibnamefont {Fert}},\ }\bibfield  {title} {\enquote
  {\bibinfo {title} {Phase-locking of magnetic vortices mediated by
  antivortices},}\ }\href {https://doi.org/10.1038/nnano.2009.143} {\bibfield
  {journal} {\bibinfo  {journal} {Nature Nanotechnology}\ }\textbf {\bibinfo
  {volume} {4}},\ \bibinfo {pages} {528--532} (\bibinfo {year}
  {2009})}\BibitemShut {NoStop}%
\bibitem [{\citenamefont {Ivanov}\ \emph {et~al.}(1998)\citenamefont {Ivanov},
  \citenamefont {Schnitzer}, \citenamefont {Mertens},\ and\ \citenamefont
  {Wysin}}]{ivanovMagnonModesMagnonvortex1998}%
  \BibitemOpen
  \bibfield  {author} {\bibinfo {author} {\bibfnamefont {B.~A.}\ \bibnamefont
  {Ivanov}}, \bibinfo {author} {\bibfnamefont {H.~J.}\ \bibnamefont
  {Schnitzer}}, \bibinfo {author} {\bibfnamefont {F.~G.}\ \bibnamefont
  {Mertens}},\ and\ \bibinfo {author} {\bibfnamefont {G.~M.}\ \bibnamefont
  {Wysin}},\ }\bibfield  {title} {\enquote {\bibinfo {title} {Magnon modes and
  magnon-vortex scattering in two-dimensional easy-plane ferromagnets},}\
  }\href {https://doi.org/10.1103/PhysRevB.58.8464} {\bibfield  {journal}
  {\bibinfo  {journal} {Physical Review B}\ }\textbf {\bibinfo {volume} {58}},\
  \bibinfo {pages} {8464--8474} (\bibinfo {year} {1998})}\BibitemShut {NoStop}%
\bibitem [{\citenamefont {Sheka}\ \emph {et~al.}(2004)\citenamefont {Sheka},
  \citenamefont {Yastremsky}, \citenamefont {Ivanov}, \citenamefont {Wysin},\
  and\ \citenamefont {Mertens}}]{shekaAmplitudesMagnonScattering2004}%
  \BibitemOpen
  \bibfield  {author} {\bibinfo {author} {\bibfnamefont {D.~D.}\ \bibnamefont
  {Sheka}}, \bibinfo {author} {\bibfnamefont {I.~A.}\ \bibnamefont
  {Yastremsky}}, \bibinfo {author} {\bibfnamefont {B.~A.}\ \bibnamefont
  {Ivanov}}, \bibinfo {author} {\bibfnamefont {G.~M.}\ \bibnamefont {Wysin}},\
  and\ \bibinfo {author} {\bibfnamefont {F.~G.}\ \bibnamefont {Mertens}},\
  }\bibfield  {title} {\enquote {\bibinfo {title} {Amplitudes for magnon
  scattering by vortices in two-dimensional weakly easy-plane ferromagnets},}\
  }\href {https://doi.org/10.1103/PhysRevB.69.054429} {\bibfield  {journal}
  {\bibinfo  {journal} {Physical Review B}\ }\textbf {\bibinfo {volume} {69}},\
  \bibinfo {pages} {054429} (\bibinfo {year} {2004})}\BibitemShut {NoStop}%
\bibitem [{\citenamefont {Ivanov}\ and\ \citenamefont
  {Zaspel}(2005)}]{ivanovHighFrequencyModes2005}%
  \BibitemOpen
  \bibfield  {author} {\bibinfo {author} {\bibfnamefont {B.~A.}\ \bibnamefont
  {Ivanov}}\ and\ \bibinfo {author} {\bibfnamefont {C.~E.}\ \bibnamefont
  {Zaspel}},\ }\bibfield  {title} {\enquote {\bibinfo {title} {High {{Frequency
  Modes}} in {{Vortex-State Nanomagnets}}},}\ }\href
  {https://doi.org/10.1103/PhysRevLett.94.027205} {\bibfield  {journal}
  {\bibinfo  {journal} {Physical Review Letters}\ }\textbf {\bibinfo {volume}
  {94}},\ \bibinfo {pages} {27205} (\bibinfo {year} {2005})}\BibitemShut
  {NoStop}%
\bibitem [{\citenamefont {Buess}\ \emph {et~al.}(2005)\citenamefont {Buess},
  \citenamefont {Knowles}, \citenamefont {H{\"o}llinger}, \citenamefont {Haug},
  \citenamefont {Krey}, \citenamefont {Weiss}, \citenamefont {Pescia},
  \citenamefont {Scheinfein},\ and\ \citenamefont
  {Back}}]{buessExcitationsNegativeDispersion2005}%
  \BibitemOpen
  \bibfield  {author} {\bibinfo {author} {\bibfnamefont {M.}~\bibnamefont
  {Buess}}, \bibinfo {author} {\bibfnamefont {T.~P.~J.}\ \bibnamefont
  {Knowles}}, \bibinfo {author} {\bibfnamefont {R.}~\bibnamefont
  {H{\"o}llinger}}, \bibinfo {author} {\bibfnamefont {T.}~\bibnamefont {Haug}},
  \bibinfo {author} {\bibfnamefont {U.}~\bibnamefont {Krey}}, \bibinfo {author}
  {\bibfnamefont {D.}~\bibnamefont {Weiss}}, \bibinfo {author} {\bibfnamefont
  {D.}~\bibnamefont {Pescia}}, \bibinfo {author} {\bibfnamefont {M.~R.}\
  \bibnamefont {Scheinfein}},\ and\ \bibinfo {author} {\bibfnamefont {C.~H.}\
  \bibnamefont {Back}},\ }\bibfield  {title} {\enquote {\bibinfo {title}
  {Excitations with negative dispersion in a spin vortex},}\ }\href
  {https://doi.org/10.1103/PhysRevB.71.104415} {\bibfield  {journal} {\bibinfo
  {journal} {Physical Review B}\ }\textbf {\bibinfo {volume} {71}},\ \bibinfo
  {pages} {104415} (\bibinfo {year} {2005})}\BibitemShut {NoStop}%
\bibitem [{\citenamefont {L'vov}(1994)}]{lvovWaveTurbulenceParametric1994}%
  \BibitemOpen
  \bibfield  {author} {\bibinfo {author} {\bibfnamefont {V.~S.}\ \bibnamefont
  {L'vov}},\ }\href@noop {} {\emph {\bibinfo {title} {Wave {{Turbulence Under
  Parametric Excitation}} : {{Applications}} to {{Magnets}}}}}\ (\bibinfo
  {publisher} {{Springer Berlin Heidelberg}},\ \bibinfo {year}
  {1994})\BibitemShut {NoStop}%
\bibitem [{\citenamefont {Schultheiss}\ \emph {et~al.}()\citenamefont
  {Schultheiss}, \citenamefont {Verba}, \citenamefont {Wehrmann}, \citenamefont
  {Wagner}, \citenamefont {K{\"o}rber}, \citenamefont {Hula}, \citenamefont
  {Hache}, \citenamefont {Kakay}, \citenamefont {Awad}, \citenamefont
  {Tiberkevich}, \citenamefont {Slavin}, \citenamefont {Fassbender},\ and\
  \citenamefont {Schultheiss}}]{schultheissExcitationWhisperingGallery}%
  \BibitemOpen
  \bibfield  {author} {\bibinfo {author} {\bibfnamefont {K.}~\bibnamefont
  {Schultheiss}}, \bibinfo {author} {\bibfnamefont {R.}~\bibnamefont {Verba}},
  \bibinfo {author} {\bibfnamefont {F.}~\bibnamefont {Wehrmann}}, \bibinfo
  {author} {\bibfnamefont {K.}~\bibnamefont {Wagner}}, \bibinfo {author}
  {\bibfnamefont {L.}~\bibnamefont {K{\"o}rber}}, \bibinfo {author}
  {\bibfnamefont {T.}~\bibnamefont {Hula}}, \bibinfo {author} {\bibfnamefont
  {T.}~\bibnamefont {Hache}}, \bibinfo {author} {\bibfnamefont
  {A.}~\bibnamefont {Kakay}}, \bibinfo {author} {\bibfnamefont {A.~A.}\
  \bibnamefont {Awad}}, \bibinfo {author} {\bibfnamefont {V.}~\bibnamefont
  {Tiberkevich}}, \bibinfo {author} {\bibfnamefont {A.~N.}\ \bibnamefont
  {Slavin}}, \bibinfo {author} {\bibfnamefont {J.}~\bibnamefont {Fassbender}},\
  and\ \bibinfo {author} {\bibfnamefont {H.}~\bibnamefont {Schultheiss}},\
  }\bibfield  {title} {\enquote {\bibinfo {title} {Excitation of whispering
  gallery magnons in a magnetic vortex},}\ }\href@noop {} {\bibinfo  {journal}
  {ArXiv e-prints}\ }\BibitemShut {NoStop}%
\bibitem [{\citenamefont {Verba}\ \emph {et~al.}(2021)\citenamefont {Verba},
  \citenamefont {K{\"o}rber}, \citenamefont {Schultheiss}, \citenamefont
  {Schultheiss}, \citenamefont {Tiberkevich},\ and\ \citenamefont
  {Slavin}}]{verbaTheoryThreemagnonInteraction2021}%
  \BibitemOpen
\bibfield  {journal} {  }\bibfield  {author} {\bibinfo {author} {\bibfnamefont
  {R.}~\bibnamefont {Verba}}, \bibinfo {author} {\bibfnamefont
  {L.}~\bibnamefont {K{\"o}rber}}, \bibinfo {author} {\bibfnamefont
  {K.}~\bibnamefont {Schultheiss}}, \bibinfo {author} {\bibfnamefont
  {H.}~\bibnamefont {Schultheiss}}, \bibinfo {author} {\bibfnamefont
  {V.}~\bibnamefont {Tiberkevich}},\ and\ \bibinfo {author} {\bibfnamefont
  {A.}~\bibnamefont {Slavin}},\ }\bibfield  {title} {\enquote {\bibinfo {title}
  {Theory of three-magnon interaction in a vortex-state magnetic nanodot},}\
  }\href {https://doi.org/10.1103/PhysRevB.103.014413} {\bibfield  {journal}
  {\bibinfo  {journal} {Physical Review B}\ }\textbf {\bibinfo {volume}
  {103}},\ \bibinfo {pages} {014413} (\bibinfo {year} {2021})},\ \Eprint
  {https://arxiv.org/abs/2008.11812} {arXiv:2008.11812} \BibitemShut {NoStop}%
\bibitem [{\citenamefont {K{\"o}rber}\ \emph {et~al.}(2020)\citenamefont
  {K{\"o}rber}, \citenamefont {Schultheiss}, \citenamefont {Hula},
  \citenamefont {Verba}, \citenamefont {Fa{\ss}bender}, \citenamefont
  {K{\'a}kay},\ and\ \citenamefont
  {Schultheiss}}]{korberNonlocalStimulationThreemagnon2020}%
  \BibitemOpen
  \bibfield  {author} {\bibinfo {author} {\bibfnamefont {L.}~\bibnamefont
  {K{\"o}rber}}, \bibinfo {author} {\bibfnamefont {K.}~\bibnamefont
  {Schultheiss}}, \bibinfo {author} {\bibfnamefont {T.}~\bibnamefont {Hula}},
  \bibinfo {author} {\bibfnamefont {R.}~\bibnamefont {Verba}}, \bibinfo
  {author} {\bibfnamefont {J.}~\bibnamefont {Fa{\ss}bender}}, \bibinfo {author}
  {\bibfnamefont {A.}~\bibnamefont {K{\'a}kay}},\ and\ \bibinfo {author}
  {\bibfnamefont {H.}~\bibnamefont {Schultheiss}},\ }\bibfield  {title}
  {\enquote {\bibinfo {title} {Nonlocal stimulation of three-magnon splitting
  in a magnetic vortex},}\ }\href
  {https://doi.org/10.1103/PhysRevLett.125.207203} {\bibfield  {journal}
  {\bibinfo  {journal} {Physical Review Letters}\ }\textbf {\bibinfo {volume}
  {125}},\ \bibinfo {pages} {207203} (\bibinfo {year} {2020})}\BibitemShut
  {NoStop}%
\bibitem [{\citenamefont {K{\"o}rber}\ \emph {et~al.}(2022)\citenamefont
  {K{\"o}rber}, \citenamefont {Heins}, \citenamefont {Hula}, \citenamefont
  {Kim}, \citenamefont {Schultheiss}, \citenamefont {Fassbender},\ and\
  \citenamefont {Schultheiss}}]{korberPatternRecognitionMagnonscattering2022}%
  \BibitemOpen
  \bibfield  {author} {\bibinfo {author} {\bibfnamefont {L.}~\bibnamefont
  {K{\"o}rber}}, \bibinfo {author} {\bibfnamefont {C.}~\bibnamefont {Heins}},
  \bibinfo {author} {\bibfnamefont {T.}~\bibnamefont {Hula}}, \bibinfo {author}
  {\bibfnamefont {J.-V.}\ \bibnamefont {Kim}}, \bibinfo {author} {\bibfnamefont
  {H.}~\bibnamefont {Schultheiss}}, \bibinfo {author} {\bibfnamefont
  {J.}~\bibnamefont {Fassbender}},\ and\ \bibinfo {author} {\bibfnamefont
  {K.}~\bibnamefont {Schultheiss}},\ }\bibfield  {title} {\enquote {\bibinfo
  {title} {Pattern recognition with a magnon-scattering reservoir},}\ }\href
  {https://doi.org/10.48550/ARXIV.2211.02328} {\  (\bibinfo {year} {2022}),\
  10.48550/ARXIV.2211.02328}\BibitemShut {NoStop}%
\bibitem [{\citenamefont {Schultheiss}\ \emph {et~al.}(2021)\citenamefont
  {Schultheiss}, \citenamefont {Sato}, \citenamefont {Matthies}, \citenamefont
  {K{\"o}rber}, \citenamefont {Wagner}, \citenamefont {Hula}, \citenamefont
  {Gladii}, \citenamefont {Pearson}, \citenamefont {Hoffmann}, \citenamefont
  {Helm}, \citenamefont {Fassbender},\ and\ \citenamefont
  {Schultheiss}}]{schultheissTimeRefractionSpin2021}%
  \BibitemOpen
  \bibfield  {author} {\bibinfo {author} {\bibfnamefont {K.}~\bibnamefont
  {Schultheiss}}, \bibinfo {author} {\bibfnamefont {N.}~\bibnamefont {Sato}},
  \bibinfo {author} {\bibfnamefont {P.}~\bibnamefont {Matthies}}, \bibinfo
  {author} {\bibfnamefont {L.}~\bibnamefont {K{\"o}rber}}, \bibinfo {author}
  {\bibfnamefont {K.}~\bibnamefont {Wagner}}, \bibinfo {author} {\bibfnamefont
  {T.}~\bibnamefont {Hula}}, \bibinfo {author} {\bibfnamefont {O.}~\bibnamefont
  {Gladii}}, \bibinfo {author} {\bibfnamefont {J.~E.}\ \bibnamefont {Pearson}},
  \bibinfo {author} {\bibfnamefont {A.}~\bibnamefont {Hoffmann}}, \bibinfo
  {author} {\bibfnamefont {M.}~\bibnamefont {Helm}}, \bibinfo {author}
  {\bibfnamefont {J.}~\bibnamefont {Fassbender}},\ and\ \bibinfo {author}
  {\bibfnamefont {H.}~\bibnamefont {Schultheiss}},\ }\bibfield  {title}
  {\enquote {\bibinfo {title} {Time {{Refraction}} of {{Spin Waves}}},}\ }\href
  {https://doi.org/10.1103/PhysRevLett.126.137201} {\bibfield  {journal}
  {\bibinfo  {journal} {Physical Review Letters}\ }\textbf {\bibinfo {volume}
  {126}},\ \bibinfo {pages} {137201} (\bibinfo {year} {2021})}\BibitemShut
  {NoStop}%
\bibitem [{\citenamefont {Vansteenkiste}\ \emph {et~al.}(2014)\citenamefont
  {Vansteenkiste}, \citenamefont {Leliaert}, \citenamefont {Dvornik},
  \citenamefont {Helsen}, \citenamefont {{Garcia-Sanchez}},\ and\ \citenamefont
  {Van~Waeyenberge}}]{vansteenkisteDesignVerificationMuMax32014}%
  \BibitemOpen
  \bibfield  {author} {\bibinfo {author} {\bibfnamefont {A.}~\bibnamefont
  {Vansteenkiste}}, \bibinfo {author} {\bibfnamefont {J.}~\bibnamefont
  {Leliaert}}, \bibinfo {author} {\bibfnamefont {M.}~\bibnamefont {Dvornik}},
  \bibinfo {author} {\bibfnamefont {M.}~\bibnamefont {Helsen}}, \bibinfo
  {author} {\bibfnamefont {F.}~\bibnamefont {{Garcia-Sanchez}}},\ and\ \bibinfo
  {author} {\bibfnamefont {B.}~\bibnamefont {Van~Waeyenberge}},\ }\bibfield
  {title} {\enquote {\bibinfo {title} {The design and verification of
  {{MuMax3}}},}\ }\href {https://doi.org/10.1063/1.4899186} {\bibfield
  {journal} {\bibinfo  {journal} {AIP Advances}\ }\textbf {\bibinfo {volume}
  {4}},\ \bibinfo {pages} {107133} (\bibinfo {year} {2014})}\BibitemShut
  {NoStop}%
\bibitem [{\citenamefont {Hubert}\ and\ \citenamefont
  {Sch{\"a}fer}(1998)}]{hubertMagneticDomainsAnalysis1998}%
  \BibitemOpen
  \bibfield  {author} {\bibinfo {author} {\bibfnamefont {A.}~\bibnamefont
  {Hubert}}\ and\ \bibinfo {author} {\bibfnamefont {R.}~\bibnamefont
  {Sch{\"a}fer}},\ }\href@noop {} {\emph {\bibinfo {title} {Magnetic Domains:
  The Analysis of Magnetic Microstructures}}}\ (\bibinfo  {publisher}
  {{Springer}},\ \bibinfo {address} {{Berlin ; New York}},\ \bibinfo {year}
  {1998})\BibitemShut {NoStop}%
\bibitem [{\citenamefont {Soldatov}\ and\ \citenamefont
  {Sch{\"a}fer}(2017)}]{soldatovSelectiveSensitivityKerr2017}%
  \BibitemOpen
  \bibfield  {author} {\bibinfo {author} {\bibfnamefont {I.~V.}\ \bibnamefont
  {Soldatov}}\ and\ \bibinfo {author} {\bibfnamefont {R.}~\bibnamefont
  {Sch{\"a}fer}},\ }\bibfield  {title} {\enquote {\bibinfo {title} {Selective
  sensitivity in {{Kerr}} microscopy},}\ }\href
  {https://doi.org/10.1063/1.4991820} {\bibfield  {journal} {\bibinfo
  {journal} {Review of Scientific Instruments}\ }\textbf {\bibinfo {volume}
  {88}},\ \bibinfo {pages} {073701} (\bibinfo {year} {2017})}\BibitemShut
  {NoStop}%
\bibitem [{\citenamefont {Krivosik}\ and\ \citenamefont
  {Patton}(2010)}]{krivosikHamiltonianFormulationNonlinear2010}%
  \BibitemOpen
  \bibfield  {author} {\bibinfo {author} {\bibfnamefont {P.}~\bibnamefont
  {Krivosik}}\ and\ \bibinfo {author} {\bibfnamefont {C.~E.}\ \bibnamefont
  {Patton}},\ }\bibfield  {title} {\enquote {\bibinfo {title} {Hamiltonian
  formulation of nonlinear spin-wave dynamics: {{Theory}} and applications},}\
  }\href {https://doi.org/10.1103/PhysRevB.82.184428} {\bibfield  {journal}
  {\bibinfo  {journal} {Physical Review B}\ }\textbf {\bibinfo {volume} {82}},\
  \bibinfo {pages} {184428} (\bibinfo {year} {2010})}\BibitemShut {NoStop}%
\bibitem [{\citenamefont {Guslienko}\ and\ \citenamefont
  {Metlov}(2001)}]{guslienkoEvolutionStabilityMagnetic2001}%
  \BibitemOpen
  \bibfield  {author} {\bibinfo {author} {\bibfnamefont {K.~Y.}\ \bibnamefont
  {Guslienko}}\ and\ \bibinfo {author} {\bibfnamefont {K.~L.}\ \bibnamefont
  {Metlov}},\ }\bibfield  {title} {\enquote {\bibinfo {title} {Evolution and
  stability of a magnetic vortex in a small cylindrical ferromagnetic particle
  under applied field},}\ }\href {https://doi.org/10.1103/PhysRevB.63.100403}
  {\bibfield  {journal} {\bibinfo  {journal} {Physical Review B}\ }\textbf
  {\bibinfo {volume} {63}},\ \bibinfo {pages} {100403} (\bibinfo {year}
  {2001})}\BibitemShut {NoStop}%
\bibitem [{\citenamefont {Metlov}\ and\ \citenamefont
  {Guslienko}(2002)}]{metlovStabilityMagneticVortex2002}%
  \BibitemOpen
  \bibfield  {author} {\bibinfo {author} {\bibfnamefont {K.~L.}\ \bibnamefont
  {Metlov}}\ and\ \bibinfo {author} {\bibfnamefont {K.~Y.}\ \bibnamefont
  {Guslienko}},\ }\bibfield  {title} {\enquote {\bibinfo {title} {Stability of
  magnetic vortex in soft magnetic nano-sized circular cylinder},}\ }\href
  {https://doi.org/10.1016/S0304-8853(01)01360-9} {\bibfield  {journal}
  {\bibinfo  {journal} {Journal of Magnetism and Magnetic Materials}\ }\textbf
  {\bibinfo {volume} {242--245}},\ \bibinfo {pages} {1015--1017} (\bibinfo
  {year} {2002})}\BibitemShut {NoStop}%
\bibitem [{\citenamefont {Burgess}, \citenamefont {Losby},\ and\ \citenamefont
  {Freeman}(2014)}]{burgessAnalyticalModelVortex2014}%
  \BibitemOpen
  \bibfield  {author} {\bibinfo {author} {\bibfnamefont {J.}~\bibnamefont
  {Burgess}}, \bibinfo {author} {\bibfnamefont {J.}~\bibnamefont {Losby}},\
  and\ \bibinfo {author} {\bibfnamefont {M.}~\bibnamefont {Freeman}},\
  }\bibfield  {title} {\enquote {\bibinfo {title} {An analytical model for
  vortex core pinning in a micromagnetic disk},}\ }\href
  {https://doi.org/10.1016/j.jmmm.2014.02.078} {\bibfield  {journal} {\bibinfo
  {journal} {Journal of Magnetism and Magnetic Materials}\ }\textbf {\bibinfo
  {volume} {361}},\ \bibinfo {pages} {140--149} (\bibinfo {year}
  {2014})}\BibitemShut {NoStop}%
\bibitem [{\citenamefont {Schaefer}\ and\ \citenamefont
  {Desimone}(2002)}]{schaeferHysteresisSoftFerromagnetic2002}%
  \BibitemOpen
  \bibfield  {author} {\bibinfo {author} {\bibfnamefont {R.}~\bibnamefont
  {Schaefer}}\ and\ \bibinfo {author} {\bibfnamefont {A.}~\bibnamefont
  {Desimone}},\ }\bibfield  {title} {\enquote {\bibinfo {title} {Hysteresis in
  soft ferromagnetic films: Experimental observations and micromagnetic
  analysis},}\ }\href {https://doi.org/10.1109/TMAG.2002.803592} {\bibfield
  {journal} {\bibinfo  {journal} {IEEE Transactions on Magnetics}\ }\textbf
  {\bibinfo {volume} {38}},\ \bibinfo {pages} {2391--2393} (\bibinfo {year}
  {2002})}\BibitemShut {NoStop}%
\bibitem [{\citenamefont {DeSimone}\ \emph {et~al.}(2002)\citenamefont
  {DeSimone}, \citenamefont {Kohn}, \citenamefont {M{\"u}ller}, \citenamefont
  {Otto},\ and\ \citenamefont {Sch{\"a}fer}}]{desimoneLowEnergyDomain2002}%
  \BibitemOpen
  \bibfield  {author} {\bibinfo {author} {\bibfnamefont {A.}~\bibnamefont
  {DeSimone}}, \bibinfo {author} {\bibfnamefont {R.}~\bibnamefont {Kohn}},
  \bibinfo {author} {\bibfnamefont {S.}~\bibnamefont {M{\"u}ller}}, \bibinfo
  {author} {\bibfnamefont {F.}~\bibnamefont {Otto}},\ and\ \bibinfo {author}
  {\bibfnamefont {R.}~\bibnamefont {Sch{\"a}fer}},\ }\bibfield  {title}
  {\enquote {\bibinfo {title} {Low energy domain patterns in soft ferromagnetic
  films},}\ }\href {https://doi.org/10.1016/S0304-8853(01)01356-7} {\bibfield
  {journal} {\bibinfo  {journal} {Journal of Magnetism and Magnetic Materials}\
  }\textbf {\bibinfo {volume} {242--245}},\ \bibinfo {pages} {1047--1051}
  (\bibinfo {year} {2002})}\BibitemShut {NoStop}%
\bibitem [{\citenamefont {Sebastian}\ \emph {et~al.}(2015)\citenamefont
  {Sebastian}, \citenamefont {Schultheiss}, \citenamefont {Obry}, \citenamefont
  {Hillebrands},\ and\ \citenamefont
  {Schultheiss}}]{sebastianMicrofocusedBrillouinLight2015}%
  \BibitemOpen
  \bibfield  {author} {\bibinfo {author} {\bibfnamefont {T.}~\bibnamefont
  {Sebastian}}, \bibinfo {author} {\bibfnamefont {K.}~\bibnamefont
  {Schultheiss}}, \bibinfo {author} {\bibfnamefont {B.}~\bibnamefont {Obry}},
  \bibinfo {author} {\bibfnamefont {B.}~\bibnamefont {Hillebrands}},\ and\
  \bibinfo {author} {\bibfnamefont {H.}~\bibnamefont {Schultheiss}},\
  }\bibfield  {title} {\enquote {\bibinfo {title} {Micro-focused {{Brillouin}}
  light scattering: Imaging spin waves at the nanoscale},}\ }\href
  {https://doi.org/10.3389/fphy.2015.00035} {\bibfield  {journal} {\bibinfo
  {journal} {Frontiers in Physics}\ }\textbf {\bibinfo {volume} {3}} (\bibinfo
  {year} {2015}),\ 10.3389/fphy.2015.00035}\BibitemShut {NoStop}%
\bibitem [{\citenamefont {Wagner}\ \emph {et~al.}(2016)\citenamefont {Wagner},
  \citenamefont {K{\'a}kay}, \citenamefont {Schultheiss}, \citenamefont
  {Henschke}, \citenamefont {Sebastian},\ and\ \citenamefont
  {Schultheiss}}]{wagnerMagneticDomainWalls2016}%
  \BibitemOpen
  \bibfield  {author} {\bibinfo {author} {\bibfnamefont {K.}~\bibnamefont
  {Wagner}}, \bibinfo {author} {\bibfnamefont {A.}~\bibnamefont {K{\'a}kay}},
  \bibinfo {author} {\bibfnamefont {K.}~\bibnamefont {Schultheiss}}, \bibinfo
  {author} {\bibfnamefont {A.}~\bibnamefont {Henschke}}, \bibinfo {author}
  {\bibfnamefont {T.}~\bibnamefont {Sebastian}},\ and\ \bibinfo {author}
  {\bibfnamefont {H.}~\bibnamefont {Schultheiss}},\ }\bibfield  {title}
  {\enquote {\bibinfo {title} {Magnetic domain walls as reconfigurable
  spin-wave nanochannels},}\ }\href {https://doi.org/10.1038/nnano.2015.339}
  {\bibfield  {journal} {\bibinfo  {journal} {Nature Nanotechnology}\ }\textbf
  {\bibinfo {volume} {11}},\ \bibinfo {pages} {432--436} (\bibinfo {year}
  {2016})}\BibitemShut {NoStop}%
\bibitem [{\citenamefont {Aliev}\ \emph {et~al.}(2009)\citenamefont {Aliev},
  \citenamefont {Sierra}, \citenamefont {Awad}, \citenamefont {Kakazei},
  \citenamefont {Han}, \citenamefont {Kim}, \citenamefont {Metlushko},
  \citenamefont {Ilic},\ and\ \citenamefont
  {Guslienko}}]{alievSpinWavesCircular2009}%
  \BibitemOpen
  \bibfield  {author} {\bibinfo {author} {\bibfnamefont {F.~G.}\ \bibnamefont
  {Aliev}}, \bibinfo {author} {\bibfnamefont {J.~F.}\ \bibnamefont {Sierra}},
  \bibinfo {author} {\bibfnamefont {A.~A.}\ \bibnamefont {Awad}}, \bibinfo
  {author} {\bibfnamefont {G.~N.}\ \bibnamefont {Kakazei}}, \bibinfo {author}
  {\bibfnamefont {D.-S.}\ \bibnamefont {Han}}, \bibinfo {author} {\bibfnamefont
  {S.-K.}\ \bibnamefont {Kim}}, \bibinfo {author} {\bibfnamefont
  {V.}~\bibnamefont {Metlushko}}, \bibinfo {author} {\bibfnamefont
  {B.}~\bibnamefont {Ilic}},\ and\ \bibinfo {author} {\bibfnamefont {K.~Y.}\
  \bibnamefont {Guslienko}},\ }\bibfield  {title} {\enquote {\bibinfo {title}
  {Spin waves in circular soft magnetic dots at the crossover between vortex
  and single domain state},}\ }\href
  {https://doi.org/10.1103/PhysRevB.79.174433} {\bibfield  {journal} {\bibinfo
  {journal} {Phys. Rev. B}\ }\textbf {\bibinfo {volume} {79}},\ \bibinfo
  {pages} {174433} (\bibinfo {year} {2009})}\BibitemShut {NoStop}%
\bibitem [{\citenamefont {Jenkins}\ \emph {et~al.}(2021)\citenamefont
  {Jenkins}, \citenamefont {Alvarez}, \citenamefont {Memshawy}, \citenamefont
  {Bortolotti}, \citenamefont {Cros}, \citenamefont {Freitas},\ and\
  \citenamefont {Ferreira}}]{jenkinsElectricalCharacterisationHigher2021}%
  \BibitemOpen
  \bibfield  {author} {\bibinfo {author} {\bibfnamefont {A.~S.}\ \bibnamefont
  {Jenkins}}, \bibinfo {author} {\bibfnamefont {L.~S.~E.}\ \bibnamefont
  {Alvarez}}, \bibinfo {author} {\bibfnamefont {S.}~\bibnamefont {Memshawy}},
  \bibinfo {author} {\bibfnamefont {P.}~\bibnamefont {Bortolotti}}, \bibinfo
  {author} {\bibfnamefont {V.}~\bibnamefont {Cros}}, \bibinfo {author}
  {\bibfnamefont {P.~P.}\ \bibnamefont {Freitas}},\ and\ \bibinfo {author}
  {\bibfnamefont {R.}~\bibnamefont {Ferreira}},\ }\bibfield  {title} {\enquote
  {\bibinfo {title} {Electrical characterisation of higher order spin wave
  modes in vortex-based magnetic tunnel junctions},}\ }\href
  {https://doi.org/10.1038/s42005-021-00614-3} {\bibfield  {journal} {\bibinfo
  {journal} {Communications Physics}\ }\textbf {\bibinfo {volume} {4}},\
  \bibinfo {pages} {107} (\bibinfo {year} {2021})}\BibitemShut {NoStop}%
\bibitem [{\citenamefont {Tyberkevych}\ \emph {et~al.}(2020)\citenamefont
  {Tyberkevych}, \citenamefont {Slavin}, \citenamefont {Artemchuk},\ and\
  \citenamefont {Rowlands}}]{tyberkevychVectorHamiltonianFormalism2020}%
  \BibitemOpen
  \bibfield  {author} {\bibinfo {author} {\bibfnamefont {V.}~\bibnamefont
  {Tyberkevych}}, \bibinfo {author} {\bibfnamefont {A.}~\bibnamefont {Slavin}},
  \bibinfo {author} {\bibfnamefont {P.}~\bibnamefont {Artemchuk}},\ and\
  \bibinfo {author} {\bibfnamefont {G.}~\bibnamefont {Rowlands}},\ }\bibfield
  {title} {\enquote {\bibinfo {title} {Vector {{Hamiltonian Formalism}} for
  {{Nonlinear Magnetization Dynamics}}},}\ }\href@noop {} {\bibfield  {journal}
  {\bibinfo  {journal} {arXiv:2011.13562 [cond-mat]}\ } (\bibinfo {year}
  {2020})},\ \Eprint {https://arxiv.org/abs/2011.13562} {arXiv:2011.13562
  [cond-mat]} \BibitemShut {NoStop}%
\bibitem [{\citenamefont {Grimsditch}\ \emph {et~al.}(2004)\citenamefont
  {Grimsditch}, \citenamefont {Giovannini}, \citenamefont {Montoncello},
  \citenamefont {Nizzoli}, \citenamefont {Leaf},\ and\ \citenamefont
  {Kaper}}]{grimsditchMagneticNormalModes2004}%
  \BibitemOpen
  \bibfield  {author} {\bibinfo {author} {\bibfnamefont {M.}~\bibnamefont
  {Grimsditch}}, \bibinfo {author} {\bibfnamefont {L.}~\bibnamefont
  {Giovannini}}, \bibinfo {author} {\bibfnamefont {F.}~\bibnamefont
  {Montoncello}}, \bibinfo {author} {\bibfnamefont {F.}~\bibnamefont
  {Nizzoli}}, \bibinfo {author} {\bibfnamefont {G.}~\bibnamefont {Leaf}},\ and\
  \bibinfo {author} {\bibfnamefont {H.~G.}\ \bibnamefont {Kaper}},\ }\bibfield
  {title} {\enquote {\bibinfo {title} {Magnetic normal modes in ferromagnetic
  nanoparticles: A dynamical matrixapproach},}\ }\href
  {https://doi.org/10.1103/PhysRevB.70.054409} {\bibfield  {journal} {\bibinfo
  {journal} {Phys. Rev. B}\ }\textbf {\bibinfo {volume} {70}},\ \bibinfo
  {pages} {54409} (\bibinfo {year} {2004})}\BibitemShut {NoStop}%
\bibitem [{Note1()}]{Note1}%
  \BibitemOpen
  \bibinfo {note} {L. Körber, C. Heins, I. Soldatov, R. Schäfer, A. Kákay,
  H. Schultheiss, K. Schultheiss, `` Data publication: Modification of
  three-magnon splitting in a flexed magnetic vortex,`` RODARE \protect \url
  {http://doi.org/10.14278/rodare.2064}, version 1 2022}\BibitemShut {NoStop}%
\end{thebibliography}%


\end{document}
