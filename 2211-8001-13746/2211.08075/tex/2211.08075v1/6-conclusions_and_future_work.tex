\section{Conclusions and future work}\label{sec:conclusions_and_future_work}

This paper presents an extension model for the \ac*{OSLC} standard to support \ac*{ECA}-based automation. More specifically, it provides the concepts to make \ac*{ECA} automation possible in an interoperable environment. The end goal of this proposal is to use \ac*{OSLC} to improve the adoption of DevOps by facilitating integration and automation between tools from different vendors or domains.

The proposed extension model has been validated by studying the main issues in the DevOps field, and the architectural characteristics deemed beneficial by practitioners. Next, a set of requirements has been established based on the said issues and used for validation. Then, a prototype architecture was designed to assess the requirements of the motivational environment. Finally, a worked example has been conducted that involves two major software development tools, GitHub and Bugzilla, to detail the application of the proposed model in a real-world workflow.

% The paper also presents an extension to the \ac{OSLC} Automation specification.
% Its goal is to serve as a model for a DevOps system that supports service integration and rule-based automation, linking the \ac{OSLC} and \ac{EWE} vocabularies.
% Although both the \ac{OSLC} standard and the \ac{TRS} protocol allow easy and flexible integration between services, the \ac{EWE} ontology enables the definition of semantic rules following the \ac{ECA} model.
% Together, they have served as the foundation for an architecture capable of executing semantically defined workflows between services using standardized interfaces.

% In conclusion, this paper attempts to address some challenges faced when implementing DevOps by combining them with semantic technologies.
% Some of these technologies are still in the early stages of their development and implementing architectures like the one proposed here will become easier as they continue to grow.
% However, if the Linked Data technologies and \ac{OSLC} increase in attractiveness and incentives vendors to integrate them into their services, proposals like this will become even more relevant.

The contribution is expected to be helpful in automated environments where interoperability is a priority. Moreover, the semantic model could potentially contribute to the open project of the \ac*{OSLC} standard.

For future work, a methodology for adopting \ac*{OSLC} and \ac*{ECA} automation for DevOps professionals will be developed. This will make the architecture easier to translate into a real-world scenario. Also, other kinds of service (beyond the change management domain) will be integrated into the architecture to test more characteristics relevant to DevOps and to provide a more complex and interesting worked example. For example, tools used to manage infrastructures, such as Docker~\citep{Docker} or Kubernetes~\citep{Kubernetes}, are very popular in DevOps environments. They can help manage large architectures efficiently and accelerate the deployment of services.
Standardizing their interfaces with \ac*{OSLC} could help integrate them with even more tools and enable event-based automation.

Another aspect of \ac*{ECA} that will be explored in the future is automation rules. Semantic web technologies have also been proposed in this domain. A bridge between one \ac*{ECA} ontology, such as \ac*{EWE}~\citep{coronadoModellingRulesAutomating2015}, and the proposed model would allow for more powerful automation features.

Furthermore, other issues concerning DevOps are explored in the systematic review used to establish the requirements of the prototype architecture~\citep{bolscherDesigningSoftwareArchitecture2019}. These challenges could be explored in future work, such as testing and monolithic databases~\citep{bolscherDesigningSoftwareArchitecture2019}. In addition, the model can still be extended to support new features and domains. For example, automation infrastructures often provide a login system for users that has not been implemented in this use case. The vocabulary can be extended further with new concepts to cover all these aspects.