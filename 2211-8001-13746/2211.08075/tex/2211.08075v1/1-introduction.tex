\section{Introduction}\label{sec:introduction}

% introducción a DevOps
In recent years, a set of tools, practices, and philosophies in the software development industry have become increasingly popular. Their goal is to improve efficiency in the application lifecycle process, resulting in faster production and delivery. This paradigm, known as DevOps~\citep{jabbariWhatDevOpsSystematic2016}, requires an infrastructure capable of rapidly scaling and adapting to changes as the market moves constantly. It also needs to accelerate the delivery of services and applications by automating tasks in phases such as testing and deployment, a practice known as Continuous Delivery~\citep{chenContinuousDeliveryOvercoming2017}.

% conceptos en DevOps
Therefore, the line between development and operations teams is becoming thinner as many companies begin to interact with their infrastructure programmatically, using the APIs that their tools and services provide, seeking a more versatile and scalable way of managing them~\citep{artacDevOpsIntroducingInfrastructureasCode2017}. Integrating these tools provides flexibility and allows practicioners to automate processes, resulting in faster and more efficient production. Such automation enhancements are becoming more attractive to the point where tools that work themselves as automation providers for other services are being implemented.

% TAS y automatización
The tools that provide automation capabilities are known as \ac{TAS}~\citep{coronadoTaskAutomationServices2016}, and most of them are based on the \ac{ECA} model. This model consists of three phases: an event triggers a rule, then a condition is evaluated, and finally, an action is executed accordingly. The concept implemented by \ac{TAS}s is known as \ac{AaaS}. Examples of such technologies, which focus on giving users the ability to define automation for their applications and processes, are IFTTT~\citep{iftttIFTTT} or Zapier~\citep{ZapierAutomationThat}. In the context of software development, tools such as StackStorm~\citep{StackStorm} apply the same ideas to automate production and delivery tasks. These tools differ from other workflow execution frameworks like Apache Airflow or GitHub Actions in approaching rule definition. In these frameworks, users are provided with tools to build their workflows over the specific platform they are using. For example, in GitHub Actions, even though it can be integrated with external tools, users are meant to build their workflows based on the activity of a GitHub repository. Instead, a \ac{TAS} focuses on integration with any services that are able to generate events and execute actions on demand. The users can use the \ac{TAS} to build rules based on these external resources.

Nevertheless, there are still two critical challenges associated with the adoption of DevOps in general and \ac{AaaS} in particular: the potential elevated cost of integrating new services~\citep{eversEvaluatingCloudAutomation2015} and the lack of flexibility when automating processes~\citep{kargerSemanticWebEnd2014}. They are analyzed in more depth in later sections. A potential solution to the integration problem found in the literature~\citep{wettingerStandardsBasedDevOpsAutomation2014} would be standardizing the interfaces between the different tools, making the migration process fast and seamless. However, this is not a simple task. Every tool has its particular model, architecture, and requisites, so it would be difficult for any possible standard to adjust appropriately to all of them. Furthermore, even if such a standard was implemented, forcing vendors to adapt their services to a rigid model could create more difficulties. They would lose their freedom to experiment with new features and functionalities.

Standardizing how the involved services are modeled could also be beneficial in terms of possible flexibility improvements for automation solutions. Again, the challenge is to design a sufficiently flexible model so that new services can be directly integrated without requiring them to adjust to a restrictive protocol.

% posible solución: Linked Data
In this context of interoperability between services, solutions based on semantic web technologies have been considered before~\citep{mcilraithSemanticWebServices2001}. Following the principles of Linked Data, a service can get heterogeneous data from different sources and still process it, a feature that could simplify integrations between services.

% Linked Data como estandar (OSLC)
There are already projects attempting to standardize the interfaces between lifecycle tools using Linked Data, the most relevant for our purposes being \ac{OSLC}~\citep{amsdenOSLCCoreVersion2021}. Because its models are not fixed and can be expanded, various tools can define their specific concepts and vocabularies. At the same time, other services can still understand and work with their resources. For example, suppose that a vendor implements a new feature that cannot be represented with its current classes. In that case, they can create an ontology and expand their resources with new definitions. Other services can discover this feature by following the links in the resource definitions, avoiding many of the usual integration issues.

% Linked Data en TAS (EWE)
% Similarly, semantic web technologies have also been applied in the context of \ac{TAS}. For example, \ac{EWE} (\citet{coronadoModellingRulesAutomating2015}) is an ontology that models the concepts used in the \ac{TAS} domain. Moreover, because \ac{EWE} defines \ac{ECA} rules with Linked Data, it allows generating events and actions beyond those implemented by the vendor and using reasoning techniques to decide when to trigger or execute them.

Similarly, semantic web technologies have also been applied in the context of \ac{TAS}. \ac{OSLC} has a specific domain for automation tools. However, it has some limitation when it comes to modeling \ac{ECA}-based environments. Later sections explore this topic in more detail. The main issue is that \ac{OSLC} lacks a proper semantic model for events and actions.

% Resumen DevOps
The contribution of this paper is to define a new semantic model for \ac{OSLC} that extends the Automation domain to support events and actions. The model provides standard interaction with \acp{TAS} based on \ac*{ECA} rules. More specifically, it has been designed to apply to real-world DevOps scenarios. To validate this last claim, an architecture prototype is presented based on the proposed semantic model to face some usual challenges DevOps practitioners face. Finally, a worked example has been conducted to evaluate this environment.

The remainder of this paper is structured as follows. Section~\ref{sec:related_work} presents related works that have laid the basis for the presented proposal. Section~\ref{sec:background} explains the concepts defined by \ac*{OSLC} and their limitation when modeling an \ac*{ECA} environment. In Section~\ref{sec:semantic_model} the proposed semantic model for extending \ac{OSLC} is proposed. Section~\ref{sec:evaluation} analyzes the main challenges in DevOps adoption and defines an architecture to face them based on the \ac*{OSLC} extension. It also presents a worked example to evaluate the proposal's validity in a real-world scenario. Finally, the conclusions and possible future work are described in Section~\ref{sec:conclusions_and_future_work}.
