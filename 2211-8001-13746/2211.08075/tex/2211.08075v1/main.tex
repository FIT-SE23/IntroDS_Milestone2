\documentclass[preprint,1p,times,longtitle]{elsarticle}

\usepackage[utf8x]{inputenc}
\usepackage{cite}
\usepackage{url}
\usepackage{amsmath,amssymb,amsfonts}
\usepackage{algorithmic}
\usepackage{graphicx}
\usepackage{textcomp}
\usepackage{xcolor}
\def\BibTeX{{\rm B\kern-.05em{\sc i\kern-.025em b}\kern-.08em
    T\kern-.1667em\lower.7ex\hbox{E}\kern-.125emX}}

\usepackage[nolist]{acronym}
\usepackage{todonotes}
\usepackage{dirtytalk}
\usepackage{multirow}
\usepackage{amsthm}
\usepackage{booktabs}
\usepackage{listings}
\usepackage{hyperref}


\usepackage[T1]{fontenc}
\usepackage[scaled=0.9]{beramono}
\usepackage{microtype}

\newcommand\Small{\fontsize{8}{10}\selectfont}
\newcommand*\LSTfont{%
  \Small\ttfamily\SetTracking{encoding=*}{-50}\lsstyle}

\lstset{
  basicstyle=\LSTfont,
  columns=fullflexible,
  xleftmargin=5mm,
  framexleftmargin=5mm,
  breakatwhitespace=false,
  tabsize=2,
  captionpos=b,
  numbers=left,
  frame=shadowbox
}

\presetkeys{todonotes}{fancyline,color=green}{}

\newcommand{\cm}[1]{\todo[author=Comentario Álvaro,color=orange]{#1}}

\newcommand{\oslc}[1]{\emph{oslc:#1}}
\newcommand{\trs}[1]{\emph{trs:#1}}
\newcommand{\auto}[1]{\emph{oslc\_auto:#1}}
\newcommand{\oslccm}[1]{\emph{oslc\_cm:#1}}
\newcommand{\actions}[1]{\emph{oslc\_eca:#1}}
\newcommand{\events}[1]{\emph{oslc\_eca:#1}}
\newcommand{\ewe}[1]{\emph{ewe:#1}}
\newcommand{\kafka}[1]{\emph{kafka:#1}}
\newcommand{\eca}[1]{\emph{oslc\_eca:#1}}

\hyphenation{Ser-vice-Pro-vi-der}
\hyphenation{Au-to-ma-tion-Plan}
\hyphenation{Au-to-ma-tion-Re-quest}
\hyphenation{Au-to-ma-tion-Re-sult}

\newpageafter{author}

\begin{document}

\begin{frontmatter}

%\title{Towards an architecture for automation of DevOps based on the OSLC standard}
%\title{Extension of OSLC standard for ECA model}
%\title{Including Events and Actions in OSLC standard for ECA-based automation}
%\title{Extending OSLC automation domain for ECA-based automation}
%\title{Extending OSLC standard for ECA-based DevOps automation}
\title{Extending the OSLC standard for ECA-based automation in DevOps environments}


% \author{\IEEEauthorblockN{Guillermo García Grao}}
\author[gsi]{Guillermo García-Grao\corref{cor1}}
\ead{g.ggrao@upm.es}

\author[gsi]{Álvaro Carrera}
\ead{a.carrera@upm.es}

\cortext[cor1]{Corresponding author}
\address[gsi]{Universidad Politécnica de Madrid, Avenida Complutense 30, Ciudad Universitaria, 28040, Madrid}

% \maketitle

\begin{abstract}

The DevOps paradigm is taking over software development systems, helping businesses increase efficiency, accelerate production, and adapt quickly to market changes.
However, adopting these principles can be challenging.
Practitioners often face an important issue known as vendor lock-in caused by the cost of tool replacement.
In addition, automating the processes that involve these tools also requires investment.
These issues could be addressed by standardizing service interfaces to facilitate their integration.
Linked Data is an attractive choice for implementing such a standard without sacrificing versatility.
An exciting and promising proposal in this direction is the \ac{OSLC} standard specification.
Its purpose is to build an environment where services can interoperate using standard Linked Data models.
However, the current specification version still lacks standard definitions for concepts that are critical to automating the execution of actions in fast-changing environments.
Therefore, this paper proposes a new specification to extend \ac*{OSLC}, based on the \ac*{ECA} model, for event-based interoperable automation, especially for DevOps environments, which are our motivational scenario.
A simple DevOps architecture is built as a prototype to validate the proposed model.
Using that architecture, the proposed model is validated in a real-world workflow to prove its contribution to the OSLC standard and the DevOps field.
% The architecture and the \ac*{OSLC} extension are the main contributions presented in this work.
% Finally, a case study is implemented to demonstrate how the proposal addresses the described challenges in adopting DevOps practices.

\end{abstract}

% \begin{IEEEkeywords}
\begin{keyword}
OSLC \sep DevOps \sep Linked Data \sep semantic \sep ontology \sep rule-based automation
\end{keyword}
% \end{IEEEkeywords}

\end{frontmatter}

\begin{acronym}
 
 \acro{OWL}[OWL]{Web Ontology Language} 
 \acro{RDF}[RDF]{Resource Description Framework}
 \acro{RDFS}[RDFS]{Resource Description Framework Schema}
 \acro{OSLC}[OSLC]{Open Services for Lifecycle Collaboration}
 \acro{EWE}[EWE]{Evented WEb}
 \acro{TAS}[TAS]{Task Automation Server}
 \acro{ECA}[ECA]{Event-Condition-Action}
 \acro{AaaS}[AaaS]{Automation as a Service}
 \acro{W3C}[W3C]{World Wide Web Consortium}
 \acro{MQTT}[MQTT]{Message Queuing Telemetry Transport}
 \acro{TRS}[TRS]{Tracked Resource Set}
 \acro{EYE}[EYE]{Euler Yet another proof Engine}
 \acro{CAD}[CAD]{Computer-Aided Design}
 \acro{BPMN}[BPMN]{Business Process Modeling Notation}
 \acro{CRUD}[CRUD]{Create, Read, Update and Delete}
 \acro{FOL}[FOL]{First Order Logic}
 \acro{CI}[CI]{Continuous Integration}
 \acro{DML}[DML]{Distributed Messaging Layer}

\end{acronym}

\section*{Acknowledgment}

This research work is supported by the Ministry of Science and Innovation (Spain) through the SmartDevOps project (RTC2019-007326-7). 


\section{Introduction}
\label{sec:intro}

% Intro: NN popularity
Following the meteoric rise of the popularity of neural NLP models during the neural revolution, they have found practical usage across a plethora of domains and tasks.
% Problem of trust
However, in a number of high-stakes domains such as law \citep{kehl2017algorithms}, finance \citep{grath2018interpretable}, and medicine \citep{caruana2015intelligible}, the opacity of deep learning methods needs to be addressed.
% Goals of XAI
In the area of explainable artificial intelligence (XAI), one of the major recent efforts is to unveil the neural black box and produce explanations for the end-user.
% Explanation methods
There are various approaches to rationalizing model predictions, such as using the attention mechanism \cite{bahdanau2014neural}, saliency methods \cite{denil2014extraction,bach2015pixel,ribeiro2016should,lundberg2016unexpected,shrikumar2017learning,sundararajan2017axiomatic}, rationale generation by-design \cite{lei-etal-2016-rationalizing,bastings2019interpretable,jain2020learning}, or self-rationalizing models \cite{marasovic2021few}.
% Faithfulness + plausibility
These methods have to simultaneously satisfy numerous desiderata to have practical application in high-stakes scenarios: they have to be \textit{faithful} -- an accurate representation of the inner reasoning process of the model, and \textit{plausible} -- convincing to human stakeholders.

% Introduce agreement as evaluation
When evaluating faithfulness in using attention as explanations, \citet{jain2019attention} have shown that attention importance scores do not correlate well with gradient-based measures of feature importance.
The authors state that although gradient-based measures of feature importance should not be taken as ground truth, one would still expect importance measures to be highly agreeable, bringing forth the \textit{agrement-as-evaluation} paradigm \cite{abnar2020quantifying,meister2021sparse}.
% Saliency methods don't disagree well between themselves
While the imperfect agreement is something one could expect as interpretability methods differ in their formulation, and it is reasonable to observe differences in importance scores, subsequent work has shown that saliency methods exhibit low agreement scores even when applied to the \textit{same model instance} \cite{neely2021order}.
Since a single trained model instance can only have a single feature importance ranking for its decision, disagreement of saliency methods implies that at least one, if not all methods, do not produce faithful explanations -- placing doubt on their practical relevance.
It has been hypothesized that unfaithfulness of attention is caused by input entanglement in the hidden space \cite{jain2019attention}.
This claim has later been experimentally verified through results showing that regularization techniques targeted to reduce entanglement significantly improve the faithfulness of attention-based explanations \cite{mohankumar2020towards,tutek2020staying}.
While entanglement in the hidden space is clearly a problem in the case of attention explanations, where attention weights directly pertain to hidden states, we also hypothesize that representation entanglement could cause similar issues for gradient- and propagation-based explainability methods -- which might not be able to adequately disentangle importance when propagating toward the inputs.

% Rank correlation
In our work, we first take a closer look at whether the rank correlation is an appropriate method for evaluating agreement and confirm that, as hypothesized in previous work, small differences in values of saliency scores significantly affect agreement scores.
We argue that a linear correlation method such as Pearson-$r$ is a better-motivated choice since the exact ranking order of features is not as crucial for agreement as the relative importance values, which Pearson-$r$ adequately captures.
% Improving agreement through regularization
We hypothesize that the cause of saliency method disagreements is rooted in representation entanglement and experimentally show that agreement can be significantly improved by regularization techniques such as tying \cite{tutek2020staying} and conicity \cite{mohankumar2020towards}.
The fact that regularization methods, which were originally aimed at improving faithfulness of attention, also improve agreement between saliency methods suggests that the two problems have the same underlying cause.
Taking the analysis deeper, we apply techniques from dataset cartography \cite{swayamdipta2020dataset} and show that, surprisingly, the explanations of easy-to-learn instances exhibit a lower agreement than of ambiguous instances.
We further analyze how local curvature of the representation space morphs when regularization techniques are applied, paving the way for further analysis of (dis)agreements between interpretability methods. 
\section{Related work}\label{sec:related_work}

This section presents research work in the field of knowledge related to this paper. It starts by reviewing ontologies proposed in the literature for DevOps environments. Special attention is payed to those that focus on the challanges explained in Section~\ref{sec:introduction}. \ac{OSLC} stands out for its approach to interoperability. Then, a study is conveyed to find architectures where the \ac{OSLC} standard has been used, emphasising those where automation is the main focus.

% First, in Section~\ref{devops_ontologies}, semantic technologies that intersect with DevOps are explored to learn from concepts defined in other ontologies. Then, Section~\ref{oslc_architectures} reviews how \ac{OSLC}-based architectures have been used before to solve integration problems in multiple domains.
% Finally, Section~\ref{event-based_automation_rules} studies examples of automation capabilities using event-based rules.

% \subsection{DevOps ontologies}\label{devops_ontologies}

As mentioned before, an interesting approach to the problem of integrating tools in a DevOps environment is to use semantic technologies~\citep{mcilraithSemanticWebServices2001}. These techniques can provide a flexible representation of resource data that different programs and devices can seamlessly understand. However, defining an ontology model for DevOps is not a simple task, as it is a concept related to different domains. Moreover, the very definition of DevOps is still imprecise and a subject of many debates, so the attempts to create such a vocabulary are often either excessively abstract and general or too specific for a particular tool. A definition based on the concepts found in the literature that has been proposed is the following: \say{DevOps is a development methodology aimed at bridging the gap between Development (Dev) and Operations, emphasizing communication and collaboration, continuous integration, quality assurance, and delivery with automated deployment utilizing a set of development practices}~\citep{jabbariWhatDevOpsSystematic2016}.

There are case studies where the use of Linked Data in a system of systems has been used for a road construction project~\citep{axelssonExperiencesUsingLinked2019}. A core ontology was defined and made extensible using word models. In another work~\citep{johngHarmoniaContinuousService2019}, a DevOps ontology extracted from Jenkins logs was used. Concepts where defined like Deployment Task or Infrastructure Change, but it is warned that this is just a reference ontology and might not be generalizable to other scenarios.

Another example of semantic technologies used with DevOps~\citep{mccarthyComposableDevOpsAutomated2015} establishes a common language for operations and development teams to use and defines some ontologies for the maturity of DevOps. More examples include a framework based on a DevOps ontology~\citep{guerreroSystematicMappingStudy2020} built on top of the PrMO (Ontology of Process-reference Models)~\citep{pardo-calvacheReferenceOntologyHarmonizing2014} and the SMO (Software Measurement Ontology)~\citep{barcellosWellFoundedSoftwareMeasurement2010}. Finally, the PaaSport semantic model~\citep{bassiliadesPaaSportSemanticModel2018}, an extension of the DUL (DOLCE+DnS Ultralite) ontology~\citep{gangemiUnderstandingSemanticWeb2003}, has been reviewed. However, it is more focused on compatibility between cloud computing platform providers.

Different Linked Data approaches to the tool interoperability problem in software development projects have been analyzed before~\citet{rodriguezFormalOntologiesData2019}. One of them was \ac{OSLC}. The \ac{OSLC} project~\citep{amsdenOSLCCoreVersion2021} tries to define a standardized interface for software development tools that applies the semantic technologies provided by \ac{W3C}. \ac{OSLC} is a set of open specifications for tool integration using the Linked Data initiative. The project consists of a Core specification that every \ac{OSLC} compliant tool must follow and a series of domain-specific protocols that are more suited for the particular needs of some groups of tools.

% There have been different approaches to the problem of modeling \ac{OSLC} resources. \citet{zhangClassModelingOSLC2013} \citep{zhangModelingToolIntegration2014} explains how \ac{OSLC} Resources can be modeled using Class Models and how Artifacts and Roles improve tool integration capabilities.
% In \citet{rymanOSLCResourceShape2013}, it is demonstrated that \ac{RDFS} and \ac{OWL} are not suitable languages to define constraints for \ac{OSLC} data.
% Instead, \ac{OSLC} Resource Shapes are proposed.

Based on \ac{OSLC}, a modeling tool for the specification of Linked Data resources focused on the development toolchain has been proposed~\citep{el-khouryModellingSupportLinked2016}. However, the authors also state that their modeling tool should be extended to cover other aspects of DevOps, like requirements analysis and automated testing.

% From the different Linked Data solutions found in the literature that could be used to model DevOps services, the one selected for the architecture proposed in this paper is \ac*{OSLC}.
The combination of standardization and flexibility found in \ac{OSLC} made it suited for the purpose of this paper. To learn more about previous applications of it, different architectures based on \ac{OSLC} from the literature are explore. The following case studies help illustrate how versatile \ac{OSLC} is and how different domains could benefit from it being more widely adopted.

% The following section explores some practical applications of \ac*{OSLC} in different projects gathered from the literature.

% \subsection{Application of \ac{OSLC} in practical scenarios}\label{oslc_architectures}

Beyond the software engineering field, architectures integrating IoT devices using \ac{OSLC} have been proposed~\citep{chenOpenSourceLifecycle2019}. Because of the heterogeneity of these IoT devices and the data they generate, getting them to interoperate becomes a problem. Using Linked Data and a standard protocol is helpful in this situation. 
% Every hardware interface has an \ac{OSLC} adapter that serves the data in the cloud so that other services can interact with it.

Another \ac{OSLC}-based architecture is used to integrate testing and requirements tools~\citep{nardoneOSLCbasedEnvironmentSystemlevel2020} in the context of the European Rail Traffic Management System/European Train Control System. Furthermore, the implications that \ac{OSLC} architectures have on implementation in the context of space systems have also been discussed~\citep{hoppeRequirementsSharedData2015}. Finally, \ac{OSLC} has been used to connect \ac{CAD} design tools and product visualization~\citep{ebelingOSLCBasedApproach2017}.

Modeling also presents interoperability challenges because system models are usually composed of submodels from different tools. For example, combinations of \ac{OSLC} and OpenModelica for model management and traceability~\citep{mengistTraceabilitySupportOpenModelica2017} or uses of the \ac{OSLC} Knowledge Management specification to allow lifecycle artifacts to be reused~\citep{alvarez-rodriguezElevatingMeaningData2019}.
In addition, the PROMIS project~\citep{aoyamaPROMISManagementPlatform2013} uses \ac{OSLC} to allow for an interorganizational software development model with a Software Supply Network (SSN).

% \todo[inline]{Aquí he metido lo de Crystal:}

A previous work that implemented an interoperability platform based on \ac*{OSLC} is the Crystal project~\citep{leitner_lessons_2016}. It was tested in a larger and more complex case study than the one presented in this paper. However, although \ac{OSLC} is used as the foundation for the project, their models go far beyond the proposed \ac*{OSLC} standards. This paper aims to define a model that could serve as a potential new specification in \ac{OSLC}. Also, the semantic models they proposed are not explicitly designed for \ac*{ECA} automation, which is the primary purpose of the present work.

There is a previous article that presents an architecture that includes something similar to event-based automation with \ac*{OSLC}~\citep{berezovskyiImprovingLifecycleQuery2018}. It uses it in the context of an integrated toolchain for IoT and uses the \ac{TRS}~\citep{crossleyOSLCTrackedResource2021}, an \ac*{OSLC}-based protocol, to automate different tasks.
% \ac{TRS} is a specification defined as part of the \ac{OSLC} project and allows a server to expose a log consisting of the changes data resources suffer so that a client can access them.
They affirm that this approach would also be very beneficial in Big Data environments because \say{using microservices and Linked Data to integrate systems within the toolchain eliminates the need for multiple transformations}, achieving better velocity and variety than previous techniques. They propose a case study of a robotic warehouse optimized for logistics automation. It uses a few tools from the change management and requirements management domains, each with its own \ac{OSLC} adapter. To communicate changes in these tools, instead of using HTTP, they resort to the \ac{MQTT} protocol~\citep{banksMQTTVersion2019} to have everything synchronized.

The proposal is exciting because it integrates various tools and devices using semantic technologies and creates a centralized communication network between these tools, which offers the possibility to automate their behavior.
They even make an effort to deliver messages asynchronously, essential for faster and more efficient communication.
However, the semantic models it uses have not been designed explicitly for \ac*{ECA}, as \ac*{TRS} is a change tracking protocol, but it was not created with events in particular.
The following section explores \ac*{OSLC} in more detail to provide a better context and to highlight the limitations found when it comes to automation based on event-based models.

% This section shown the versatility of \ac*{OSLC} and the possibilities it offers for automation environments. Although it does not support automated events and actions at the moment, this paper presents an extension to the standard to fill this gap. The following section explores other ontologies used in event-based automation scenarios, focusing on \ac*{EWE}, also used in the paper's proposed architecture.

% \subsection{Event-based automation rules and \ac{EWE}}\label{event-based_automation_rules}

% The \ac{ECA} model allows a system to define events that trigger actions based on certain conditions.
% This technique allows databases to be active sources of information in the system (\citet{gohECARulebasedSupport2001}).
% These automation rules have been utilized in Workflow Management Systems (WfMS), \citet{casatiEnvironmentDesigningExceptions1999} and \citet{baeAutomaticControlWorkflow2004}.

% Some commercial solutions that apply this model to provide automation capabilities have become very popular in recent years.
% For example, services like IFTTT \citep{iftttIFTTT} or Zapier \citep{ZapierAutomationThat} allow users to generate automation rules between applications, while StackStorm \citep{StackStorm} or Clarive \citep{Clarive} can be used to automate DevOps workflows.
% There are also examples in the literature of this model being used to automate business workflows, like in \citet{baeAutomaticControlWorkflow2004}.
% The concept implemented here is \ac{AaaS}.

% This service can be enhanced using semantic technologies.
% Two examples of such enhancements are the \ac{EYE} reasoner and the \ac{EWE} ontology.
% The \ac{EYE} reasoning engine (\citet{verborghDrawingConclusionsLinked2015}) allows defining semantical rules using the Notation3 format, which can then be executed using advanced reasoning techniques.
% The \ac{EWE} ontology was created to model the \ac{TAS} knowledge domain (\citet{coronadoModellingRulesAutomating2015}).
% It defines a vocabulary with the necessary concepts for event-based service automation.

% \ac{EWE} Tasker (\citet{munozSmartOfficeAutomation2016}) is a task automation server that combines the \ac{EWE} data schema and the \ac{EYE} reasoner to perform service automation based on semantically defined rules.
% It defines Channels that generate Events used to execute Actions if some Rule is triggered, following the \ac{ECA} model.
% \ac{EWE} Tasker offers flexibility, as many software tools and devices can be connected to the server by defining semantic representations as channels, which allows for the definition of events and actions.

% The \ac*{EWE} ontology is suited for defining of automated workflows in DevOps environments.
% Furthermore, because the architecture proposed in this paper uses \ac*{OSLC} for interoperability between its services,
% the \ac*{TAS} implemented should be accessible via \ac*{OSLC}.
% Therefore, a bridge between \ac*{OSLC} and \ac*{EWE} is defined along with the extension of the \ac*{OSLC} standard to support events and actions.

\section{Background}\label{sec:background}

% intro
This section explains how the \ac*{OSLC} Core Specifiaction works in more detail. It also presents \ac{OSLC} Automation and \ac{TRS}, domains that are relevant in the motivational DevOps scenarios. First, Section~\ref{sec:service_modelling} introduces the Core specifaction of \ac*{OSLC} which is the foundation for every service modeling. Then, Section~\ref{sec:auto_modelling} showcases the Automation domain which would be the best suited to model a \ac*{TAS} in the current state of the project. Next, Section~\ref{sec:logging_modelling} presents the \ac*{TRS} protocol, useful for logging and change tracking. Finally, Section~\ref{sec:oslc_limitations} discusses some of the limitations found in these vocabularies when trying to use them to model an \ac*{ECA} environment.

% resumen
% Services in the proposed architecture are modeled using \acf{OSLC} (\citet{amsdenOSLCCoreVersion2021}).
% Resources in each service are defined using the appropriate \ac{OSLC} domain, while the \ac{OSLC} Core vocabulary ensures interoperability.
% This is explained in more detail in Section~\ref{sec:service_modelling}.

% The \ac*{OSLC} Automation Server is modeled using the concepts from the \ac{OSLC} Automation domain (\citet{amsdenOSLCAutomationVersion2021}).
% They are explained in Section~\ref{sec:auto_modelling}.
% At the moment of writing this, the Automation domain is not in its final version, and it does not support events and actions.
% An extension to the domain is proposed to provide the necessary concepts for events (Section~\ref{sec:event_modelling}) and actions (Section~\ref{sec:action_modelling}).

% To model the automation rules used internally by the \ac*{TAS}, the \acf*{EWE} ontology (\citet{coronadoEWEOntologySpecification2017}) is used.
% The concepts from this ontology are presented in Section~\ref{sec:rule_modelling}.

% Finally, changes in the resources are exposed via \acf{TRS} (\citet{crossleyOSLCTrackedResource2021}) to be used for logging and monitoring.
% The \ac*{TRS} protocol is a specification provided by the \ac*{OSLC} family and is explained in more detail in Section~\ref{sec:logging_modelling}.

\subsection{OSLC Core overview}\label{sec:service_modelling}

% oslc general
% As briefly mentioned in Section~\ref{sec:state_of_the_art}, 
\ac{OSLC} is a family of open specifications for integrating tools. Its purpose is to integrate product/application lifecycle services regardless of their internal model. This goal makes the standard suitable for modeling the services in a DevOps infrastructure. In \ac{OSLC}, different topics such as change management, test management, or requirements management define specifications for their respective resources. They are known as domains. These domains are built on top of the \ac*{OSLC} Core Specification~\citep{amsdenOSLCCoreVersion2021} to ensure their compatibility.

% oslc detalle
In \ac{OSLC}, data is represented using \ac{RDF} and retrieved via HTTP\@. Therefore, each service must provide a catalog of its Service Providers, represented by the \oslc{ServiceProviderCatalog} and \oslc{ServiceProvider} classes. The \ac{OSLC} primer defines Service Providers as \say{organizing concepts that partition the overall space of artifacts in the tool into smaller containers. Examples of common partitioning concepts offered by tools include \say{projects}, \say{modules}, \say{user databases}, and so on}~\citep{speicherOSLCPrimer2019}. Hence, the meaning of Service Provider changes for each service, but the main idea is that it classifies the resources in that service.

% oslc core
An \ac*{OSLC} API offers clients the possibility of discovering all its resources and capabilities based on its root resource, the \oslc{ServiceProviderCatalog}. This resource allows clients to obtain its list of Service Providers responsible for providing the URLs needed to send creation and query requests to the server. These endpoints can be discovered through the \oslc{CreationFactory} and \oslc{QueryCapabilities} services. They may also provide an \oslc
{ResourceShape} that describes the properties a resource from that service is expected to have. An overview of the \ac{OSLC} Core specification from \citet{amsdenOSLCCoreVersion2021} can be found in Figure~\ref{fig:oslc_core_overview}.

% Figure~\ref{fig:oslc_execution} shows how a client could interact with an \ac{OSLC} service in a \ac{BPMN} diagram.

% \begin{figure}[!ht]
%     \centering
%     \includegraphics[width=0.9\textwidth]{figures/bpmn/oslc.png}
%     \caption{Execution process to modify an \ac{OSLC} resource.}\label{fig:oslc_execution}
% \end{figure}

\begin{figure}[!ht]
    \centering
    \includegraphics[width=0.9\textwidth]{figures/ontologies/oslc.png}
    \caption{\ac{OSLC} Core overview.}\label{fig:oslc_core_overview}
\end{figure}

\subsection{OSLC Automation}\label{sec:auto_modelling}

% oslc automation
Regarding the process automation field, \ac{OSLC} defines a specific domain called \ac{OSLC} Automation~\citep{amsdenOSLCAutomationVersion2021}. This domain aims to model an interface so that an automation provider can interact with other \ac{OSLC} services. It defines three different resources a Service Provider exposes following the Core standard. The first of these resources is the \auto{AutomationPlan} which defines a unit of automation available for execution. Next, the \auto{AutomationRequest} resource provides the information required to execute an Automation Plan. Finally, the standard defines an \auto{AutomationResult} resource to track an Automation Request status and contributions. Figure~\ref{fig:oslc_automation_overview} displays a diagram of these resources and their relationships.

\begin{figure}
    \centering
    \includegraphics[width=0.8\textwidth]{figures/ontologies/oslc-auto.png}
    \caption{\ac{OSLC} Automation diagram.}\label{fig:oslc_automation_overview}
\end{figure}


% \subsection{Rule modeling}\label{sec:rule_modelling}

% % ewe intro
% Besides \ac*{OSLC}, the architecture uses another vocabulary to model its automation rules: the \ac{EWE} ontology.
% As briefly explained in Section~\ref{sec:state_of_the_art}, \ac{EWE} is a vocabulary designed to describe \ac{TAS}s.
% According to its specification (\citet{coronadoEWEOntologySpecification2017}), one of the main goals of \ac{EWE} is to \say{provide a base vocabulary for building domain-specific vocabularies, e.g., Twitter Task Ontology or Evernote Task Ontology}.
% Because \ac{OSLC} can model services from different domains, this feature is essential to enable interoperability.

% % ewe detalle
% The core concepts in \ac{EWE} are \ewe{Rule}, \ewe{Channel}, \ewe{Event}, and \ewe{Action}.
% Every \ewe{Rule} follows the \ac{ECA} model; if an \ewe{Event} is received and certain conditions are met, an \ewe{Action} is executed.
% The concept of \ewe{Channel} represents \say{individuals that either generate Events, provide Actions, or both.
% In the context we refer to, Channel mainly defines Web Services} \citet{coronadoEWEOntologySpecification2017}.
% Events and Actions can have different \ewe{Parameter} resources used for \ewe{Rule} evaluation and execution.
% Diagrams showing the execution process followed to evaluate rules can be observed in Figure~\ref{fig:ewe_execution}.
% The main concepts from \ac{EWE} are shown in Figure~\ref{fig:ewe_main_entities}.

% \begin{figure}[!ht]
%     \centering
%     \includegraphics[width=0.8\textwidth]{figures/bpmn/ewe.png}
%     \caption{Description of the execution process after an event is received in \ac{EWE}.}\label{fig:ewe_execution}
% \end{figure}


% \begin{figure}[!ht]
%     \centering
%     \includegraphics[width=0.9\textwidth]{figures/ontologies/ewe.png}
%     \caption{Main concepts defined in \ac{EWE} ontology, from \citet{munozSmartOfficeAutomation2016}.}\label{fig:ewe_main_entities}
% \end{figure}

% % resumen
% In summary, services in the proposed architecture are modeled using \ac{OSLC}.
% Resources in each service are defined using the appropriate \ac{OSLC} domain while the \ac{OSLC} Core vocabulary ensures interoperability.
% Changes in the resources are exposed via \ac{TRS} to be used for logging and monitoring.
% To provide the whole environment with automation capabilities, a semantic \ac{TAS} is utilized.
% Internally, it is modeled with \ac{EWE}, while \ac{OSLC} Automation defines the external interface, so it can be interacted with in a standardized way.

\subsection{The TRS protocol}\label{sec:logging_modelling}

% trs
% To fulfill requirement RQ3, \ac{OSLC} services need a model to generate logs. 
As briefly mentioned in Section~\ref{sec:related_work}, the \ac{TRS} protocol~\citep{crossleyOSLCTrackedResource2021} is specified as part of the \ac{OSLC} project. It was conceived to track changes in the resource set of an \ac{OSLC} service and expose them as \ac{RDF} HTTP resources themselves. These exposed changes have been used before to trigger automated actions~\citep{berezovskyiImprovingLifecycleQuery2018}.

\ac{TRS} defines a list of \trs{ChangeLog} resources. Each of these Change Logs contains numbered entries representing every creation, modification, or deletion suffered by any resource in the set. Because this list can be accessed via HTTP, a client can periodically check what changes have occurred in the resource set to monitor the behavior of the service. The protocol also offers the \ac{TRS} Patch mechanism to allow a \trs{ChangeEvent} to carry more detailed information about the modifications suffered by the resources.

\subsection{OSLC limitations for ECA-based automation}\label{sec:oslc_limitations}

After reviewing the current state of the \ac*{OSLC} project, the goal is to introduce it in a DevOps environment. More specifically, an environment where a \ac*{TAS} provides automation for other services using a system of \ac*{ECA} rules. To achieve this, \ac*{OSLC} needs a standardize way for services to provide event detection and actions execution capabilities.

As previously mentioned, there is a specific \ac*{OSLC} domain for Automation. However, key concepts are still missing for this model to support event detection. For example, to run an \auto{AutomationPlan}, \ac*{OSLC} Automation requires a consumer to create an \auto{AutomationRequest} by making a POST request to the Automation server. The ideal scenario would be to have the \ac*{OSLC} Automation service listening for events generated in another service by itself. This idea is the starting point for defining events in the proposed extension model.
% This is what the Event Listener component described in the architecture does. To model this component with \ac*{OSLC}, an extension to the Automation domain is proposed.

On the other hand, \ac{OSLC} resources already support basic CRUD actions (create, read, update and delete). It was already mentioned how Service Providers enable resource creation and querying. The resources in \ac{OSLC} can also be updated or deleted using the corresponding HTTP methods (PUT and DELETE, respectively). The \ac{OSLC} community has shown efforts to support the execution of more concrete actions~\citep{painOSLCActions2020} over resources. The proposal allows a service to perform actions on other service resources without knowing their type and properties. The document is a potential candidate for a future specification. It even suggests ideas of introducing actions in the \ac{OSLC} Core Specification. This draft serves as foundation for the action model proposed in this paper, extending to improve some key aspects still missing.
Next section presents the extension to \ac*{OSLC} for event and action support. It details all the properties that have been defined and explians how it is suposed to be used for interoperability.

\section{OSLC Extension for ECA-based Automation}\label{sec:semantic_model}

This section showcases the main contribution of the paper which is the semantic model that extends \ac*{OSLC} to support \ac*{ECA}-based automation~\citep{OSLCExtensionECAbased}. Section~\ref{sec:event_modeling} details how events are proposed to work. Section~\ref{sec:action_modeling} presents an overview of the new actions vocabulary. Finally, Section~\ref{sec:rules} explains how automation rules are defined and provides a whole overview of the propoed \ac*{ECA} model. For the state of clarity, every new concept proposed in the paper uses the \textit{oscl\_eca} prefix.

\subsection{Event modeling}\label{sec:event_modeling}

% falta eventos ext
% As previously mentioned, there is a specific \ac*{OSLC} domain for Automation.
% However, key concepts are still missing for this model to meet the requirements of the proposed architecture.
% To run an \auto{AutomationPlan}, \ac*{OSLC} Automation requires a consumer to create an \auto{AutomationRequest} by making a POST request to the Automation server.
% The ideal scenario would be to have the Automation Server listening for events generated in another service by itself.
% This is what the Event Listener component described in the architecture does.
% To model this component with \ac*{OSLC}, an extension to the Automation domain is proposed.

This section details the psoposed model for integration events in \ac{OSLC}. A diagram is shown in Figure~\ref{fig:oslc_events} with these newly defined resources.
% Listing~\ref{listing:events} shows an example of an \events{Event} and its corresponding \events{EventListener} in Notation3 language.

\begin{figure}
    \centering
    \includegraphics[width=\textwidth]{figures/ontologies/oslc-events.png}
    \caption{Diagram of classes and properties for Events - OSLC ECA proposal.}\label{fig:oslc_events}
\end{figure}

% events
The philosophy followed by the \ac*{OSLC} standard is not to make assumptions about its tools and services. Therefore, the focus has to be on the concepts enabling interoperability. In the context of events, a concept needs to be defined so consumers can discover how to listen to a service generating them. For this purpose, the \events{EventListener} resource is defined.

An \events{EventListener} resource represents a backgorund process insida the service that listens to a source of events. Consumers can create an \events{EventListener} so it notifies them when an event occur. The source of these events could be many things depending on the service and is signaled by the \events{source} property. Other properties, required or optional, can be described by an \oslc{ResourceShape}, as in any other \ac*{OSLC} resource.

However, unlike other \ac*{OSLC} resources, interacting with it using CRUD actions is not enough. Although a consumer could periodically poll the service for new events, the optimal result would be for the service to asynchronously notify consumers. There are multiple options to do this, among others: using webhooks, providing a URL to the service to make POST request with the new events; with a publish-subscriber protocol, like MQTT which has been used before in a similar context~\citep{berezovskyiImprovingLifecycleQuery2018}; or using some messaging service that can deliver the events to their intended receivers.

There are too many possiblities to standardize all of them. To solve this issue, a concept known as interacion patterns can be borrow from the \ac*{OSLC} Actions proposal~\citep{painOSLCActions2020}. The idea is to determine which protocol to use depending on the type of resource pointed by a \events{binding} property. The simplest example is with a webhook. When the \events{binding} property is supposed to point at a \textit{http:Request}, then it means the service uses webhooks. A consumer could then create an \events{EventListener} providing a description of the \textit{http:Request} that wants to receive. This leaves the door open to new interaction patterns to join the standard in the future.

The events themselves are represented by an \ac{OSLC} resource named \events{Event}. Whenever an \events{EventListener} detects an event triggered by its sources, it creates and exposes an \events{Event}. Depending on the interaction pattern used by the service, it may also send the \events{Event} asynchronously (inside POST request, for example). Besides the \ac*{OSLC} stabdard properties and typical metadata annotations (like a title or a timestamp), an \events{Event} resource can have other service specific properties. An \oslc{ResourceShape} should describe all of the possible properties the event can contain. An \events{Event} also has the \events{generatedBy} and \events{causedBy} properties pointing at the \events{EventListener} and \ac*{OSLC} resource that produced it, respectively.

There is a relation between these events and the \ac*{OSLC} Automation domain. For example, an \events{Event} could be used to trigger \auto{AutomationPlans}. Hence, there should be a relationship between \events{Events} and \auto{AutomationRequests}. An \auto{AutomationRequest} could be started when the automation service receives an \events{Event}, so a new \events{triggeredBy} property is proposed for addition.

Before going into the next secton, another interaction pattern more suitable for DevOps scenarios needs to be explored. The webhook example served to illustrate how a simple interaction pattern could be defined but is not ideal for environments where scalability is critical. Using a message broker to deliver the events to consumers is a more appropriate solution. Tools like Apache Kafka~\citep{ApacheKafka} provide this kind of service and allow for more scalable architectures. For this reason, this interaction pattern is used in the architecture and worked example presented in Section~\ref{sec:evaluation} for the evaluation of the model.

Such an interaction pattern requires the \events{binding} to point at a resource representing the broker, with properties signaling a URL for providers to POST messages and a topic for conumers to subscribe. The \textit{kafka} prefix is used to model this service. The \kafka{Broker} resource is proposed to model the broker, along with the the \kafka{providerURL} and \kafka{topic} properties.

% \begin{lstlisting}[caption={Example of an \events{Event} and its \events{EventListener}.}, label={listing:events}]
%     @prefix oslc: <http://open-services.net/ns/core#> .
%     @prefix oslc_auto: <http://open-services.net/ns/auto#> .
%     @prefix oslc_events: <http://open-services.net/ns/events#> .
%     @prefix dcterms: <http://purl.org/dc/terms/> .
%     @prefix rdf: <http://www.w3.org/1999/02/22-rdf-syntax-ns#> .

%     <http://example.com/Event/1> a oslc_events:Event,
%             oslc_auto:AutomationRequest,
%         oslc:serviceProvider <http://example.com/ServiceProvider/1> ;
%         oslc_events:generatedBy <http://example.com/EventListener/1> ;
%         dcterms:title "Event generated by service X"^^rdf:XMLLiteral .
    
%     <http://example.com/EventListener/1> a oslc_events:EventListener ;
%         oslc_events:source <http://example.com/source/of/events> ;
%         dcterms:title "Listener detecting events from service X"^^rdf:XMLLiteral .
% \end{lstlisting}

\subsection{Action modeling}\label{sec:action_modeling}

This section presents the proposed model for actions in \ac{OSLC}.
Figure~\ref{fig:oslc_actions} shows a diagram representing its classes and properties, relations with other \ac{OSLC} concepts.

\begin{figure}
    \centering
    \includegraphics[width=\textwidth]{figures/ontologies/oslc-actions.png}
    \caption{Diagram of classes and properties for Actions - OSLC ECA proposal.}\label{fig:oslc_actions}
\end{figure}

% oslc acciones crud
% In addition to the Event Listener, the Action Dispatcher on an \ac{OSLC} adapter also needs modeling.
% \ac{OSLC} resources already support basic CRUD actions (create, read, update and delete).
% It was already mentioned how Service Providers enable resource creation and querying.
% The resources in \ac{OSLC} can also be updated or deleted using the corresponding HTTP methods (PUT and DELETE, respectively).

% % oslc actions
% The \ac{OSLC} community has shown efforts to support the execution of more concrete Actions (\citet{painOSLCActions2020}) over resources.
% This specification allows a service to perform actions on other service resources without knowing their type and properties.
% The document is a potential candidate for a future specification.
% It even suggests ideas of introducing Actions in the \ac{OSLC} Core Specification.

% oslc actions detalle
The starting point for the action model is the \ac{OSLC} Actions proposal, previously mentioned in Section~\ref{sec:background}. According to it, resources may have an \actions{action} property advertising possible actions that can be performed on them.
This property leads to \actions{Action} resources that describe the action and provide instructions to execute it and determine its result.
The information is provided through action bindings which can support multiple interaction patterns (i.e., HTTP request or UI dialog).
This is the same method mentioned for the events model proposed in the previous section. Making each resource expose its own available actions facilitates interoperability.

% oslc actions problemas
However, in many cases, resources from the same \oslc{ServiceProvider} and same domain have the same available actions.
It would be interesting to provide a way of discovering available actions for a whole set of resources in a Service Provider, instead of having to query specific resources to find them.
Also, the binding and interaction pattern mechanisms seem overly complicated in this case, as \ac*{OSLC} already has the concept of \oslc{ResourceShape} to indicate to consumers how to work with resources like \actions{Actions}.

% oslc actions ext
A different approach is proposed to solve these issues. The \actions{ActionDispatcher} resource is defined to represent the capability of the service to execute a specific action. Inspired by the \oslc{CreationFactory} and \oslc{QueryCapabilities} concepts, an \actions{ActionDispatcher} provides a URL where consumers can send post requests to execute the action it dispatches. The \actions{execution} data property signals this URL. Actions sent for execution can be queried using another URL provided by the \actions{ActionDispatcher} through the \actions{queryActions} predicate. These URLs are represented by \textit{ldp:Container} objects, like in the \ac*{OSLC} Core specification.

\actions{ActionDispatchers} also provide an \oslc{ResourceShape} to let consumers know the necessary properties to execute its \actions{Actions}. Individual resources can still indicate which actions are available for execution by pointing with an \actions{availableAction} to the corresponding \actions{ActionDispatcher}. Furthermore, an executed \actions{Action} points at its target resource with the \actions{executedOn} property. Finally, the \actions{Actions} contain a \actions{status} and \actions{verdict} properties indicating the execution result, similar to an \auto{AutomationRequest}.

% oslc automation + actions
The standard establishes a relation between the Automation and Actions specifications. Specific \actions{Actions} can become available (or be automatically executed) when an \auto{AutomationPlan} successfully finishes running. The \actions{futureAction} property was defined to identify them. In this extended model, \actions{futureAction} points to an \actions{ActionDispatcher} instead. The corresponding \auto{AutomationResult} indicates that the \actions{Action} is available for immediate execution.


% Next section evaluates the concepts defined in the proposed model by establishing a set of architectural requirements and testing them in a worked example.
% Listing~\ref{listing:actions} shows an example of an \actions{Action} and the \actions{ActionDispatcher} that executes it in Notation3 language.

% \begin{lstlisting}[caption={Example of an \actions{Action} and its \actions{ActionDispatcher}.}, label={listing:actions}]
%     @prefix oslc: <http://open-services.net/ns/core#> .
%     @prefix oslc_actions: <http://open-services.net/ns/actions#> .
%     @prefix dcterms: <http://purl.org/dc/terms/> .
%     @prefix rdf: <http://www.w3.org/1999/02/22-rdf-syntax-ns#> .

%     <http://example.com/Action/1> a oslc_actions:Action,
%             oslc_actions:CustomAction,
%         oslc:serviceProvider <http://example.com/ServiceProvider/1> ;
%         oslc_actions:executedBy <http://example.com/ActionDispatcher/1> ;
%         oslc_actions:executedOn <http://example.com/OSLCResource/1> ;
%         dcterms:title "Action executed on service Y"^^rdf:XMLLiteral .

%     <http://example.com/ActionDispatcher/1> a oslc_actions:ActionDispatcher ;
%         oslc:serviceProvider <http://example.com/ServiceProvider/1> ;
%         oslc:resourceShape <http://example.com/CustomAction/resourceShape> ;
%         oslc_actions:execute <http://example.com/url/to/post/actions/to/execute> ;
%         dcterms:title "Dispatcher that exposes actions from service Y"^^rdf:XMLLiteral .
% \end{lstlisting}

\subsection{Rule modeling}\label{sec:rules}

Besides events and actions, it is necessary to have a model for automation rules to fully represent an \ac*{ECA} environment.
These rules represent the conditions that need to be met for an event to trigger an automated action.
Figure~\ref{fig:oslc_rules} shows the whole model proposed in this work.

\begin{figure}
    \centering
    \includegraphics[width=\textwidth]{figures/ontologies/oslc-rules.png}
    \caption{Diagram of classes and properties for Rules - OSLC ECA proposal.}\label{fig:oslc_rules}
\end{figure}

An \eca{Rule} represents an automation rule.
It can point to an \events{Event} with an \eca{triggeredBy} property and to an \actions{Action} with an \eca{executedBy} predicate.
The \eca{consequenceOf} property is also defined for actions to point at events that caused them to be executed.

The \eca{AssertedCondtion} resource is defined to provide a means of discovering if some evaluation of a rule was successful.
When the rule is evaluated, an \eca{AssertedCondtion} is created pointing at the specific \actions{Action} generated with the \eca{produces} property.
The \eca{asserts} predicate links it to the \eca{Rule}.
A boolean linked by the \eca{asserted} property indicates whether the evaluation was successful. 

Because \ac*{OSLC} is focus on interoperability, the proposal is not concerned with the way automation rules work internally.
Depending on the service managing the automation, rules can be evaluated in many ways.
They can even use one of the frameworks that exist for semantic rules, like SPIN~\citep{knublauch2011spin} or Notation 3 Logic~\citep{berners2005notation}.
This would mean further interoperability, as they are also Linked Data standards.
Therefore, the proposed rule modeling aims to establish a framework able to cope any rules based on the \ac{ECA} model without specifying the method applied during the inference process.
% Future work will focus on filling the gap between \ac*{OSLC} and one of these semantic rule models.

\section{Validation}\label{sec:evaluation}

This section discusses the validity of the results obtained by the proposed semantic model.
The need for these extension to the \ac*{OSLC} standard was justified in Section~\ref{sec:introduction} with the benefits it provides to the DevOps field.
Section~\ref{sec:problem_overview} explores the literature to find notable issues faced by DevOps practitioners.
Based on these challenges, requirements are defined for a DevOps architecture.
Meeting these requirements means the architecture improves practitioners' experience regarding the issues found in the literature.

To prove that \ac*{OSLC} and, more specifically, the proposed semantic model help in fulfilling the requirements, a DevOps architecture is presented in Section~\ref{sec:architecture}.
The concepts defined in Section~\ref{sec:semantic_model} are used to build an automation environment based on the \ac{ECA} model.
To ensure the architecture actually follows the DevOps philosophy, a set of characteristic for such architectures are gathered from the literature.
The most relevant are included in the design process.

Finally, to validate the proposed semantic model, the architecture enabled by it is put to the test with a worked example conducted in Section~\ref{sec:case_study}.
It presents a real-world scenario where two popular commercial services are integrated into the \ac*{ECA} environment.
It proves that the model and the architecture it made possible meet the established requirements.

\subsection{Problem overview}\label{sec:problem_overview}

% A precise understanding of the problem is the key to making design decisions.
This section focuses on establishing a set of requirements to face the challenges identified in the DevOps domain. The goal is to show how the proposed model can be used to build an architecture that meets the requirements and, therefore, helps improving DevOps practices. The list challenges has been gathered from a systematic review of the literature~\citep{bolscherDesigningSoftwareArchitecture2019}. The ones addressed by this proposal have been selected and are shown in Table~\ref{tab:issues}. They also put together a list of characteristics desirable for DevOps environments that will be useful for the design of the architecture.

% \renewcommand{\arraystretch}{2}
\begin{table}[!ht]
    \centering
    \begin{tabular}{ccc}
        \toprule
        \textbf{ID} & \textbf{Issue} & \textbf{Description} \\ \midrule
        IS1 & Methods and tools &
        \begin{tabular}[c]{@{}c@{}}
          Complex tooling and lack \\
          of proper support
        \end{tabular} \\ \midrule

        IS2 &
        \begin{tabular}[c]{@{}c@{}}
          Ever-changing operational \\
          environments and tools
        \end{tabular} &
        \begin{tabular}[c]{@{}c@{}}
          Deploying in heterogeneous \\
          operations environments
        \end{tabular} \\ \midrule

        IS3 & Logging & Bug traceability with many services \\ \midrule

        IS4 & Monitoring & Complex and sophisticated \\ \midrule

        IS5 & Scaling &
        \begin{tabular}[c]{@{}c@{}}
          Hard to implement DevOps if the \\
          architecture does not scale
        \end{tabular} \\ \bottomrule
    \end{tabular} 
\caption{List of issues/challenges for DevOps environments~\citep{bolscherDesigningSoftwareArchitecture2019}.}\label{tab:issues}
\end{table}
% \renewcommand{\arraystretch}{1}

% problema: de integración de herramientas
Section~\ref{sec:introduction} mentions some issues found when adopting DevOps practices. First, the problem known as vendor lock-in appears when software companies become dependent on the tools they are using, not being able to substitute them when they need to~\citep{opara-martinsCriticalAnalysisVendor2016}. Companies seeking to adopt DevOps practices like \ac{CI} could face this challenge due to the complexity of the required tools and the effort needed to integrate them into their workflows~\citep{stahlCindersContinuousIntegration2017} (issue IS1). Even those companies that have already successfully transitioned to DevOps could suffer from this issue, as the environments and tools are changing fast and constantly~\citep{shahinIntersectionContinuousDeployment2016} (issue IS2). This issue results in a lack of flexibility that is incompatible with the DevOps idea of quickly adapting to changes in the market.

% problema: falta de flexibilidad en los TAS
Regarding \ac{AaaS}, designing flexible solutions can be a problematic task~\citep{coronadoTaskAutomationServices2016}. For most \ac{TAS} platforms, workflows are defined based on available features. Regardless of how extensive they make this set of features, there can be situations where users need some specific functionality that the vendor does not offer. Users might, for example, want to integrate a recently released service into their infrastructure, but it may not be yet supported by the \ac{TAS} they are currently using~\citep{coronadoTaskAutomationServices2016}.

Even when users implement such integrations themselves, they might want to switch to another \ac{TAS} later. However, because they have all of their automated tasks built on top of the \ac{TAS} they are currently using, they might find it too expensive to make the migration, getting again to a position of vendor lock-in. These issues could lead to companies not adapting and evolving fast enough, so their infrastructure quickly becomes obsolete.

% problema: trazabilidad
Another issue that arises when adopting DevOps is managing logging and monitoring (issues IS3 and IS4). In a microservices architecture, traceability becomes increasingly hard to handle at scale. Failing to address traceability adequately could delay finding bugs which means less efficient \ac{CI}.

% problema: perder otros beneficios de DevOps en el proceso
While facing the previous challenges, other DevOps concepts should not be overlooked. Flexible implementations and a short time to market are key DevOps aspects, but the scalability of the infrastructure (issue IS5) should not be sacrificed to achieve them.

These concepts will be the foundation for the defined requirements. Other works presenting DevOps architectures make quantitative analyses to evaluate the performance gained by their proposed approach~\citep{vergoriDevOpsPerformanceEngineering2017}. However, in this case, it is hard to find values that can be measured to make such assessments. Therefore, the proposed model is evaluated by defining a set of requirements, presenting an architecture based on such a model, and implementing a test case to verify that they are correctly satisfied.

% \begin{figure}[!ht]
%     \centering
%     \includegraphics[width=\textwidth]{figures/sysml/requirements.png}
%     \caption{SysML requirements diagram.}\label{fig:requirements}
% \end{figure}


% RQ1
% To tackle the service integration issue (composed of IS1 and IS2), requirement \textbf{RQ1} is defined in Table~\ref{tab:rq1}.
To tackle the service integration issue (composed of IS1 and IS2), requirement \textbf{RQ1} is defined. Because changes in the development and business model can happen over time, tools will need to be added, updated, or removed from the infrastructure. The integration between these tools should be as independent of their internal structure as possible to maximize versatility. Tools need to provide a means to interact with them in a standardized way, either directly or through an adapter. For the architecture proposed in this paper, that standard will be \ac{OSLC}. This requirement can be validated by successfully executing a complete workflow that detects an event in one service, triggers a rule in a \ac{TAS}, and executes an action in a different service. Such a workflow should be implemented without changing the system interfaces. They have to work regardless of the internal structure of the integrated service.

% \begin{table}[!ht]
%   \begin{tabular}{ll}
%     \toprule
%     \textbf{RQ1} &
%     \begin{tabular}[c]{@{}l@{}}
%       The infrastructure shall seamlessly integrate any tool, as long\\
%       as it follows specific standards.
%     \end{tabular} \\ \midrule
%     \textbf{Precondition} &
%     \begin{tabular}[c]{@{}l@{}}
%       The service that needs to be integrated into the system is already\\
%       compliant with \ac{OSLC}. No changes to the system were required for\\
%       this particular service.
%     \end{tabular} \\ \midrule
%     \textbf{Postcondition} &
%     \begin{tabular}[c]{@{}l@{}}
%       The system detects events in the new service and actions take\\
%       effect in its resources. Rules can be defined utilizing these events and\\
%       actions.
%     \end{tabular} \\ \midrule
%     \textbf{Validation} &
%     \begin{tabular}[c]{@{}l@{}}
%       To consider the integration process sufficiently fast and seamless,\\
%       rule definition needs to be possible without coding API calls.\\
%       For two different \ac{OSLC} compliant services, if a set of rules involving\\
%       one event and one action from each service can be executed; the requirement\\
%       will be considered satisfied.
%     \end{tabular} \\ \bottomrule
%   \end{tabular}
% \caption{Seamless integration of services (RQ1).}\label{tab:rq1}
% \end{table}

% RQ2
% It is subdivided into requirement RQ21 from Table~\ref{tab:rq21} and requirement RQ22 from Table~\ref{tab:rq22}.
Requirement \textbf{RQ2} focuses on the flexibility challenges faced by \ac{AaaS} platforms. To avoid the vendor lock-in problem, it should be possible to interact with the \ac*{TAS} using a standardized protocol. In this case, that standard is \ac*{OSLC}. In particular, it uses the concepts from the \ac*{OSLC} Automation domain.
% Rules should also be standardized, otherwise every automated workflow would have to be reimplemented when migrating to a new \ac*{TAS}.

% \begin{table}[!ht]
%   \begin{tabular}{ll}
%     \toprule
%     \textbf{RQ21} &
%     \begin{tabular}[c]{@{}l@{}}
%       The system shall use a standardized interface to communicate with automation\\
%       services.
%     \end{tabular} \\ \midrule
%     \textbf{Precondition} &
%     \begin{tabular}[c]{@{}l@{}}
%       The automation service used complies with the\ac*{OSLC} Automation standard.
%     \end{tabular} \\ \midrule
%     \textbf{Postcondition} &
%       Interaction with the automation service is standardized. \\ \midrule
%     \textbf{Validation} &
%     \begin{tabular}[c]{@{}l@{}}
%       When the automation service can be interacted with using \ac*{OSLC} patterns,\\
%       then it is easier to integrate with new services and replacing it with another\\
%       \ac*{OSLC}-compliant \ac*{TAS} would take less effort.
%     \end{tabular} \\ \bottomrule
%   \end{tabular}
%   \caption{Standardized automation service interface (RQ21).}\label{tab:rq21}
% \end{table}

% \begin{table}[!ht]
%   \begin{tabular}{ll}
%     \toprule
%     \textbf{RQ22} &
%     \begin{tabular}[c]{@{}l@{}}
%       The system shall support the standardized definition of automation rules.
%     \end{tabular} \\ \midrule
%     \textbf{Precondition} &
%     \begin{tabular}[c]{@{}l@{}}
%       A \ac*{TAS} with support for the definition of semantic rules.
%     \end{tabular} \\ \midrule
%     \textbf{Postcondition} &
%       Automation features implemented using a semantic standard for automation rules. \\ \midrule
%     \textbf{Validation} &
%     \begin{tabular}[c]{@{}l@{}}
%       Automation rules should follow a standard based on Linked Data to be interoperable.
%     \end{tabular} \\ \bottomrule
%   \end{tabular}
%   \caption{Interoperable automation rules (RQ22).}\label{tab:rq22}
% \end{table}

% RQ3
% \textbf{RQ3}, shown in Table~\ref{tab:rq3}, addresses the problem of logging and monitoring (IS3 and IS4) the components of the architecture.
\textbf{RQ3} addresses the problem of logging and monitoring (IS3 and IS4) the components of the architecture. Once again, using a standard would simplify this task. In additon, traceability becomes easier if every service follows the same model for its logs. The \ac*{TRS} standard fits because of its compatibility with \ac*{OSLC}.

% \begin{table}[!ht]
%   \begin{tabular}{ll}
%     \toprule
%     \textbf{RQ3}           & The system shall use a standard for logging and monitoring.\\ \midrule
%     \textbf{Precondition}  & \ac*{OSLC} support by the components of the system.\\ \midrule
%     \textbf{Postcondition} & 
%     \begin{tabular}[c]{@{}l@{}}
%       Logs can be retrieved using the \ac*{TRS} standard.
%     \end{tabular}\\ \midrule
%     \textbf{Validation}    &
%     \begin{tabular}[c]{@{}l@{}}
%       Every component of the system uses the same Linked Data model for logging.
%     \end{tabular} \\ \bottomrule
%   \end{tabular}
%   \caption{Traceability (RQ3).}\label{tab:rq3}
% \end{table}

% RQ4
% \textbf{RQ4} is defined in Table~\ref{tab:rq4} to guarantee that the architecture can scale and adapt to such changes without being restructured (IS5).
Growth in the complexity of the infrastructure is supported as the number of tools integrated increases. \textbf{RQ4} guarantees that the architecture can scale and adapt to such changes without being restructured (IS5). To satisfy this requirement, the system should rely on technologies able to scale horizontally. In other words, as long as the different modules can be containerized and provide APIs to access them, the infrastructure can be considered scalable.

% \begin{table}[!ht]
%   \begin{tabular}{ll}
%     \toprule
%     \textbf{RQ4}           & The infrastructure shall be scalable and support larger scenarios.\\ \midrule
%     \textbf{Precondition}  & Follow DevOps design principles.\\ \midrule
%     \textbf{Postcondition} & 
%     \begin{tabular}[c]{@{}l@{}}
%       Several services are integrated into the system communicating via \ac{OSLC}\\
%       compatible REST APIs.
%     \end{tabular}\\ \midrule
%     \textbf{Validation}    &
%     \begin{tabular}[c]{@{}l@{}}
%       The technologies used in implementing the system must allow it to scale\\
%       horizontally.
%     \end{tabular} \\ \bottomrule
%   \end{tabular}
%   \caption{Scalability (RQ4).}\label{tab:rq4}
% \end{table}

% Once the requirements for the architecture have been laid out, it is possible to make the appropriate design decisions.

% Table ? showcases all the requirements along with the issues that justify them and their validation methods.
The following section presents an architecture based on the proposed semantic model to fulfill these requirements, summarised in Table~\ref{table:reqs}.
%It also possesses a series of characteristics essential for a DevOps architecture.

\begin{table}
    \centering
    \begin{tabular}{lp{0.5\textwidth}l} 
        \toprule
        \textbf{RQ ID} & \textbf{Description} &\textbf{Related Issues}\\ 
        \hline
        RQ1         & The infrastructure shall seamlessly integrate any tool, as long as it follows specific standards. & IS1, IS2 \\ 
        \hline
        RQ2         & The system shall use a standardized interface to communicate with automation services. & IS1, IS2 \\ 
        \hline
        RQ3         & The system shall use a standard for logging and monitoring. & IS3, IS4 \\ 
        \hline
        RQ4         & The infrastructure shall be scalable and support larger scenarios. & IS5 \\
        \bottomrule
    \end{tabular}
    \caption{Summary of defined requirements and their relation to the issues found in the literature for DevOps.}\label{table:reqs}
\end{table}


% \begin{table}
%     \centering
%     \caption{Requirements and their relation to the issues found in the literature for DevOps.}
%     \begin{tabular}{llll} 
%     \hline
%     \textbf{RQ} & \textbf{Description}                                                                                                                         & \textbf{Validation}                                                                                                                                                                                                                                                                                                                                                & \begin{tabular}[c]{@{}l@{}}\textbf{Related}\\\textbf{issues}\end{tabular}  \\ 
%     \hline
%     RQ1         & \begin{tabular}[c]{@{}l@{}}The infrastructure shall\\seamlessly integrate any\\tool, as long as it follows\\specific standards.\end{tabular} & \begin{tabular}[c]{@{}l@{}}To consider the integration process\\sufficiently fast and seamless, rule\\definition needs to be possible without\\coding API calls. For two different OSLC\\compliant services, if a set of rules\\involving one event and one action from\\each service can be executed; the\\requirement will be considered satisfied.\end{tabular} & IS1 and IS2                                                                \\ 
%     \hline
%     RQ2         & \begin{tabular}[c]{@{}l@{}}The system shall use a\\standardized interface\\to communicate with\\automation services.\end{tabular}            & \begin{tabular}[c]{@{}l@{}}When the automation service can be\\interacted with using OSLC patterns,\\then it is easier to integrate with new\\services and replacing it with another \\OSLC-compliant TAS would take less\\effort.\end{tabular}                                                                                                                    &                                                                            \\ 
%     \hline
%     RQ3         & \begin{tabular}[c]{@{}l@{}}The system shall use a\\standard for logging and\\monitoring.\end{tabular}                                        & \begin{tabular}[c]{@{}l@{}}Every component of the system uses\\the same Linked Data model for logging.\end{tabular}                                                                                                                                                                                                                                                & IS3 and IS4                                                                \\ 
%     \hline
%     RQ4         & \begin{tabular}[c]{@{}l@{}}The infrastructure shall\\be scalable and support\\larger scenarios.\end{tabular}                                 & \begin{tabular}[c]{@{}l@{}}The technologies used in implementing\\the system must allow it to scale\\horizontally.\end{tabular}                                                                                                                                                                                                                                    & IS5                                                                        \\
%     \hline
%     \end{tabular}
% \end{table}

\subsection{Prototype Architecture}\label{sec:architecture}

This section presents an architecture that uses the \ac*{OSLC} extension model as a foundation. 
% First, Section~\ref{sec:characteristics} showcases a list of characteristics found beneficial for DevOps architectures in the literature. Then, Section~\ref{sec:overview} presents a general view of the architecture.
% Finally, Section~\ref{sec:building_blocks} explores the details of its internal components.

% \subsubsection{Beneficial characteristics for DevOps architectures}\label{sec:characteristics}

This section presents an architecture founded on the \ac*{OSLC} extension model. The decisions made in the architecture design are aimed to meet the requirements established in Section~\ref{sec:problem_overview}. Because it is meant to be applicable in DevOps scenarios, it also includes the essential characteristics deemed beneficial for such an architecture in the literature~\citep{bolscherDesigningSoftwareArchitecture2019}. The most relevant of these characteristics have been selected to guide its design. The criteria used are based on the number of references and the relation to the requirements of this proposal. These chosen characteristics are listed in Table~\ref{tab:characteristics}. Some characteristics not selected could be included in future work.

% \renewcommand{\arraystretch}{2}
\begin{table}[!ht]
    \centering
    \begin{tabular}{cccl}
        \toprule
        \textbf{ID} & \textbf{Characteristic} & \textbf{Related RQs} & \textbf{Description} \\ \midrule

        CH1 & Deployability & RQ1 and RQ4 &
        \begin{tabular}[l]{@{}l@{}}
            The architecture deploying downtime is \\
            minimized and moving between different \\
            environments is fast.
        \end{tabular} \\ \midrule

        CH2 & Modularity & RQ1 and RQ4 &
        \begin{tabular}[l]{@{}l@{}}
            Services in the architecture have minimized \\
            dependencies and changes are isolated.
        \end{tabular} \\ \midrule

        CH3 & Loosely coupled & RQ1 and RQ4 &
        \begin{tabular}[l]{@{}l@{}}
            Decreasing application and inter-team \\
            dependencies.
        \end{tabular} \\ \midrule

        CH4 & Agility/Modifiability & RQ1 &
        \begin{tabular}[l]{@{}l@{}}
            Architectural changes can be performed fast.
        \end{tabular} \\ \midrule

        CH5 & Automation & RQ2 &
        \begin{tabular}[l]{@{}l@{}}
            Processes like testing and deployment \\
            are automated as much as possible.
        \end{tabular} \\ \midrule

        CH6 & Monitoring & RQ3 &
        \begin{tabular}[l]{@{}l@{}}
            Services can be monitored.
        \end{tabular} \\ \midrule

        CH7 & Logging & RQ3 &
        \begin{tabular}[l]{@{}l@{}}
            Services produce and store meaningful logs.
        \end{tabular} \\ \bottomrule
    \end{tabular} 
\caption{List of beneficial characteristics for DevOps architectures by \citet{bolscherDesigningSoftwareArchitecture2019}.}\label{tab:characteristics}
\end{table}
% \renewcommand{\arraystretch}{1}

% \subsubsection{Architecture overview}\label{sec:overview}

The architecture comprises five layers, as shown in Figure~\ref{fig:basic_architecture}. These five layers are the Service Layer, the \ac*{OSLC} Adaption Layer, the Distributed Messaging Layer, The \ac{OSLC} Automation Layer, and the \ac{OSLC} Logging Layer. In the following paragraphs, the inclusion of each layer is justified by the defined requirements and the selected characteristics.

\begin{figure}[!ht]
    \centering
    \includegraphics[width=0.9\textwidth]{figures/architecture/basic.png}
    \caption{Diagram of the prototype architecture for validation.}\label{fig:basic_architecture}
\end{figure}

The most frequently required characteristic in a DevOps environment is deployability (CH1). A deployable architecture needs to be easily moved between environments. This is often achieved by dividing it into smaller components that can be deployed independently (microservices). Modularity (CH2) and loosely coupled (CH3) are two more characteristics this pattern meets. In the proposed architecture, these components form the \textbf{Service Layer}.

To meet requirement RQ1, services in the architecture need to be able to communicate via \ac*{OSLC}. This would also achieve modifiability (CH4), as services would be easier to substitute if their interfaces are standardized. Some of these services might be natively compliant with \ac*{OSLC}, but those not following the standard require adapters. These adapters are independent component, so the previously mentioned characteristics are not sacrificed. They form the \textbf{\ac*{OSLC} Adaptation Layer}.

Automation (CH5) is another crucial characteristic for DevOps practitioners. It is fulfilled, along with requirement RQ2, by the \textbf{\ac{OSLC} Automation Layer}. To keep the architecture modular, automation is separated into its own service. It is compliant with \ac*{OSLC} and communicates with the other services through \ac*{OSLC} adapters. However, having the \ac{OSLC} Automation Layer directly communicate via HTTP requests with the adapters would cause issues as the system scales and complexity grows.

To avoid conflicts with requirement RQ4, a \textbf{\ac{DML}} is included. This service is responsible for delivering the messages generated in the services to the \ac{OSLC} Automation Layer and vice versa. Any service in the architecture can push messages to the \ac*{DML} that will be received by those subscribed to the appropriate topic.

Finally, to meet requirement RQ3, every \ac*{OSLC}-compliant service implements the \ac*{TRS} specification. \ac*{TRS} provides tracking capabilities for changes in a service's resource set. This capability can be used for monitoring (CH6) and logging (CH7). The \ac*{TRS} functionality is separated from the service into its own component to keep a modular architecture. The collection of \ac*{TRS} services form the \textbf{\ac{OSLC} Logging Layer}.

% \subsubsection{Architecture components}\label{sec:building_blocks}

% Several layers of the architecture include components that require further discussion.
% The Service Layer depends on the particular tools used when implementing the architecture in practice and is explored in Section~\ref{sec:service_layer}.
% The \ac*{OSLC} Adaption Layer includes an adapter component for each service.
% The elements inside this adapter are presented in Section~\ref{sec:oslc_adapter}.

% Section~\ref{sec:dml} shows the internal architecture of the \ac*{DML}.
% The \ac{OSLC} Automation Layer needs more detail as well.
% Its components are explained in Section~\ref{sec:auto_layer}.
% Finally, the \ac{OSLC} Logging Layer consists of servers implementing the \ac*{TRS} specification, so it is omitted from the explanation.

% % \subsubsubsection{The Service Layer}\label{sec:service_layer}

% The particular services used in actual architecture implementations live in the Service Layer.
% This layer has several properties.
% On the one hand, because of the deployability, modularity, and loosely coupled characteristics (CH1, CH2, and CH3, respectively), services should be isolated and not have dependencies on each other.
% This often leads to a very heterogeneous set of services.
% On the other hand, the services should be easy to change and replace because of the agility/modifiability characteristic (CH4).
% The \ac*{OSLC} Adaptation Layer standardizes the communication between the services using \ac*{OSLC} adapters to simultaneously support all of these characteristics.

% % \subsubsubsection{The \ac*{OSLC} Adaption Layer}\label{sec:oslc_adapter}

% The main components in the \ac{OSLC} Adaption Layer are the \ac*{OSLC} adapters for the services in the Service Layer.
% In most cases, an \ac*{OSLC} adapter only needs an HTTP server to expose its resources and core capabilities and a backdoor connection to its adapting service.
% In the case of supporting \ac*{TRS}, it also needs to expose those resources via HTTP\@.
% However, for the architecture proposed in this paper, \ac*{OSLC} adapters also need to generate events and execute actions.
% Therefore, they have two extra elements: an \textbf{Event Listener} and an \textbf{Action Dispatcher}.

% The \textbf{Event Listener} has several roles.
% First, it substitutes the backdoor connection between the central \ac*{OSLC} server and the adapted service.
% How the Event Listener detects the events generated at the service depends on the service's implementation (it could be via webhooks, for example).
% Once an event is received, the Event Listener models it using the \ac*{OSLC} vocabulary from the corresponding domain.
% The standardized event is then sent to the \ac*{DML} to be delivered to the \ac{OSLC} Automation Layer.
% They are also logged at the \ac*{TRS} server.

% The \textbf{Action Dispatcher} is subscribed to the \ac*{DML} to receive the resources containing the actions sent by the \ac{OSLC} Automation Layer.
% The actions are described using the appropriate \ac*{OSLC} vocabulary and are meant to be executed on the adapted service.
% The execution mechanism depends on the service (an HTTP request would be the most common example).
% The Action Dispatcher is responsible for understanding the action's standardized description and executing it on the adapted service.
% The resource containing the action is logged at the \ac*{TRS} server.
% Any changes to its execution status (pending, failed, success) are also logged.

% Figure~\ref{fig:oslc_adapter} shows the internal architecture of an adapter like this.
% Modeling events and actions in \ac*{OSLC} requires extending the specification.
% This is addressed in Section~\ref{sec:semantic_model}.

% \begin{figure}[!ht]
%     \centering
%     \includegraphics[width=0.8\textwidth]{figures/architecture/adapter.png}
%     \caption{\ac*{OSLC} adapter architecture.}\label{fig:oslc_adapter}
% \end{figure}

% % \subsubsubsection{The Distributed Messaging Layer}\label{sec:dml}

% The role of the \ac*{DML} is to handle the delivery of messages between the services and the \ac*{OSLC} Automation Layer.
% The main reason for having a separate tool for this functionality is better \textbf{scalability (RQ4)}.
% If the services communicated directly with the \ac*{OSLC} Automation Layer, it could produce a bottleneck.
% Therefore, the \ac*{DML} has to be distributed, which means it scales horizontally as the architecture gets bigger and incorporates more services.
% The distributedness of the \ac*{DML} also satisfies characteristics CH1, Ch3, and CH4.
% The architecture of the \ac*{DML} consists of a server and two types of clients.

% The server acts as a \textbf{message broker}.
% Its role is to have open connections with the clients, which they can use to push messages or receive them asynchronously.
% As previously stated, to keep it scalable, it has to be distributed.

% The clients can act as \textbf{producers and consumers}.
% Producers can use an open connection with the broker to push messages indexed by a particular topic.
% Consumers can use their connections to subscribe to topics and receive a message asynchronously when it is pushed to that topic.

% There are topics specific for events generated by certain services in the architecture so the \ac*{OSLC} Automation Layer can receive them.
% In addition, there are topics for actions that are sent to the \ac*{OSLC} adapters for execution.

% % \subsubsubsection{The \ac{OSLC} Automation Layer}\label{sec:auto_layer}

% The \ac{OSLC} Automation Layer needs to fulfill requirements RQ21 and RQ22 from Section~\ref{sec:problem_overview}.
% Requirement RQ21 stated that the automation service has to offer the means to interact with it via \ac*{OSLC}.
% On the other hand, requirement RQ22 established that the automation rules managed by the service need a semantic model for interoperability.
% Two submodules are included in the \ac{OSLC} Automation Layer to meet these requirements: an \textbf{\ac*{OSLC} Automation Server} and a \textbf{Semantic \ac*{TAS}}.
% Figure~\ref{fig:automation_module} shows the architecture of the whole \ac{OSLC} Automation Layer.

% \begin{figure}[!ht]
%     \centering
%     \includegraphics[width=0.8\textwidth]{figures/architecture/automation.png}
%     \caption{\ac{OSLC} Automation Layer architecture.}\label{fig:automation_module}
% \end{figure}

% The \textbf{\ac*{OSLC} Automation Server} is basically an \ac*{OSLC} adapter for the Semantic \ac*{TAS}.
% It contains the same elements as any other adapter.
% An Event Listener listens to events from the \ac*{DML} that come from all the different services in the architecture.
% These are translated appropriately and sent to the \ac*{TAS}.
% If an action is generated due to the execution of a rule, an Action Dispatcher standardizes it using the semantic model from Section~\ref{sec:semantic_model}.
% It then publishes the action to the \ac*{DML} to be delivered to the corresponding service.

% Regarding the \textbf{Semantic \ac*{TAS}}, the \ac*{EWE} ontology has been selected to model the automation rules.
% The implementation of the \ac*{TAS} needs to have an API so that the adapter can communicate with it and a rule engine capable of evaluating events and returning actions accordingly.
% A \ac*{TAS} implementation that uses this model is \ac*{EWE} Tasker (\citet{munozSmartOfficeAutomation2016}).
% In \ac*{EWE} Tasker, rule evaluation is performed by an engine capable of making inferences using Notation3 formulas (\citet{berners-leeNotation3N3Readable2011}).
% This rule engine is based on the \ac{EYE} reasoner (\citet{verborghDrawingConclusionsLinked2015}).
% A \ac*{TAS} like this meets the requirements to be included in the architecture.

% Figure~\ref{fig:exec_process} shows a BPMN diagram representing the execution process of an automated workflow.
% It follows every step from an event generated in one service until an action is executed in another.

% \begin{figure}[!ht]
%     \centering
%     \includegraphics[width=0.9\textwidth]{figures/bpmn/workflow.png}
%     \caption{\ac*{BPMN} diagram showing the execution process.}\label{fig:exec_process}
% \end{figure}

% When a service generates an event at the Service Layer, it is notified to the Event Listener in that service's \ac*{OSLC} adapter.
% The system used for this notification process depends on the implementation of the service.
% The Event Listener translates it to \ac*{OSLC} vocabulary, exposes it as an HTTP resource, and saves a log in the \ac*{TRS}.
% Finally, it sends the event to the \ac*{DML} using the appropriate topic.

% The Event Listener in the \ac*{OSLC} Automation Server receives the event and repeats the same process.
% The event triples are sent to the Semantic \ac*{TAS}, which evaluates it with its Rule Engine.
% If it finds a rule triggered by that specific event, an action is returned to the \ac*{OSLC} Automation Server.
% The Action Dispatcher processes it and sends it back to the related topic of the \ac*{DML}.

% The \ac*{OSLC} adapter of the corresponding service receives the action.
% Its Action Dispatcher is in charge of logging, processing, and executing it.
% This completes the execution process.

\subsection{Worked Example of an ECA-based automation in the motivational scenario}\label{sec:case_study}

% intro
This section presents a worked example with a practical architecture implementation involving two real-world services.
Its goal is to demonstrate that the proposed model enabled an architecture that can meet the requirements defined in Section~\ref{sec:problem_overview}.
The issues and requirements defined in that section are addressed throughout this worked example.
The chosen tools are Bugzilla~\citep{Bugzilla}, a bug tracking web service, and GitHub \citep{GitHub}, a git-based version control tool.
Both tools are well known and widely used in the software development industry.
Also, \ac*{OSLC} already provides a Change Management domain to model bug tracking tools.
This presents an opportunity to show how the model integrates with other parts of \ac*{OSLC}.
All the code is available in a public repository\footnote{Worked example source code: \url{https://github.com/gsi-upm/oslc-eca_environment}}.

% historieta
A practical example is introduced where this integration could be helpful. A company that has already adopted some DevOps practices. The development team uses GitHub repositories to manage the code of the applications they are working on. They are also using GitHub issues for change management. GitHub offers integration with many tools, and the development team uses some of them for CI/CD\@. For example, they have an automated workflow that deploys a new version of an application when a pull request to the main branch is accepted. This workflow is only executed if there are no open issues with a \say{critical} label.

Working on the same project, another team is in charge of testing the application. This team wants to use Bugzilla instead of GitHub issues to keep track of the bugs and errors they find. A reason they want this could be, for example, because Bugzilla is open source and can be hosted by them. The development team needs to be aware of this new tool and integrate it into their automated workflows, so the application is not deployed while there are unresolved bugs (issue \textbf{IS1}). To save the work of manually doing this, they could use some already developed solutions to integrate GitHub and Bugzilla, such as GitZilla~\citep{geraGitZilla2021}. However, this approach makes them more dependent on these tools. As established by issue \textbf{IS2}, the tools and environments in DevOps change all the time. The testing team might want to move to a different bug tracking tool in the future. To avoid these issues and meet the requirement \textbf{RQ1}, the company decides to use \ac*{OSLC} standardized interfaces to integrate both tools.

% bugzilla adapter
Adapters must be built because none of these services are natively integrated with \ac{OSLC} or \ac{TRS}.
This tool is based on the \ac*{ECA} model: it receives events from other services, evaluates internal rules, and sends back actions for execution.
An adapter has already been implemented for Bugzilla as part of a tutorial on \ac{OSLC} made by some Core committee members\footnote{OSLC Bugzilla tutorial: \url{https://oslc.github.io/developing-oslc-applications/integrating_products_with_oslc/running_the_examples}}. Bugzilla classifies its \say{bugs} into \say{products}. Hence, the adapter chooses these elements as its \ac{OSLC} resources and \oslc{ServiceProviders}. It operates in the Change Management domain. Whenever a bug is created or updated, it is reflected in the \ac{TRS} server provided by the adapter.

% github adapter
GitHub has no \ac{OSLC} capabilities implemented but provides a REST API with all the necessary features to implement an \ac{OSLC} server. For this worked example, the resources of interest on GitHub are its \say{issues}, as they are closely related to bugs in Bugzilla. The concept of \say{repository} is chosen as \oslc{ServiceProvider} and Change Management as the domain. GitHub also offers a webhook-based notification system whenever an event occurs. The \ac{TRS} server can be updated without periodically polling the API to check for changes with this feature.

% They are concerned about being locked into a specific \ac*{TAS}, so they choose \ac*{EWE} Tasker for its support of interoperable semantically modeled rules.
The development team uses a \ac*{TAS} to implement its automation rules keeping its architecture modular.
Because the rest of the tools they are using are integrated with \ac*{OSLC}, they implement an \ac*{OSLC} interface to interact with the \ac*{TAS} in a standardized way. These decisions are consistent with the requirement \textbf{RQ2}. Three workflows are defined to synchronize the services using these events and actions. Table~\ref{tab:table_workflows} illustrates how these workflows and their respective rules operate.

% \renewcommand{\arraystretch}{2}
\begin{table}[!ht]
    \centering
    \begin{tabular}{cccc} 
        \toprule
        \multicolumn{2}{l}{\multirow{2}{*}{}} & \multicolumn{2}{c}{Workflows}                                                               \\ 
        \cmidrule(lr){3-4}
        \multicolumn{2}{l}{}                   & Bugzilla to GitHub                         & GitHub to Bugzilla                            \\ 
        \midrule
        \multirow{3}{*}{Rules} & Create        & \(Bug\:created \rightarrow Create\:issue\) & \(Issue\:created \rightarrow Create\:bug\)    \\ 
        \cmidrule(lr){2-4}
                               & Update        & \(Bug\:updated \rightarrow Update\:issue\) & \(Issue\:updated \rightarrow Update\:bug\)  \\
        \cmidrule(lr){2-4}
                               & Resolve        & \(Bug\:resolved \rightarrow Close\:issue\) & \(Issue\:closed \rightarrow Resolve\:bug\)  \\
        \bottomrule
    \end{tabular}
    \caption{Workflows and rules defined for the worked example.}\label{tab:table_workflows}
\end{table}
% \renewcommand{\arraystretch}{1}

To support standardized logging and monitoring of their services (\textbf{RQ3}), they use \ac*{TRS} servers to expose changes in every \ac*{OSLC} resource. In addition, they use Apache Kafka to send messages between servicest to maintain scalability as their automation environment grows (\textbf{RQ4}).

Figure~\ref{fig:case_study} shows the complete implementation of the architecture for the exposed worked example. The diagram shows the workflow where a bug is created using Bugzilla, and an issue is automatically generated on GitHub following the \ac*{OSLC} \ac*{ECA} proposed extension.

\begin{figure}[!ht]
    \centering
    \includegraphics[width=\textwidth]{figures/architecture/case_study.png}
    \caption{Workflow detailed in the worked example, where the creation of a bug in Bugzilla triggers the generation of an issue in GitHub.}\label{fig:case_study}
\end{figure}

% ejecución
Following the diagram steps (Figure~\ref{fig:case_study}), the workflow starts with a bug creation in Bugzilla (Step 1).
To describe resources for bug tracking like this, \ac*{OSLC} provides the Change Management domain~\citep{amsdenOSLCChangeManagement2020}.
The \ac*{OSLC} Bugzilla adapter uses this domain's \textit{oslc\_cm:ChangeRequest} class to model bugs.
Listing~\ref{listing:bug} shows the example resource generated from the created bug in Notation 3 language.

\begin{lstlisting}[caption={\ac{RDF} representing the bug created in Bugzilla.}, label={listing:bug}]
@prefix bgz: <http://www.bugzilla.org/rdf#> .
@prefix oslc: <http://open-services.net/ns/core#> .
@prefix dcterms: <http://purl.org/dc/terms/> .
@prefix oslc_cm: <http://open-services.net/ns/cm#> .
@prefix xsd: <http://www.w3.org/2001/XMLSchema#> .

<http://bugzilla.example.com/ServiceProvider/1/bugs/1> a oslc_cm:ChangeRequest ;
    oslc_cm:severity "enhancement"^^xsd:string ;
    oslc_cm:status "CONFIRMED"^^xsd:string ;
    oslc:serviceProvider <http://bugzilla.example.com/ServiceProvider/1> ;
    dcterms:contributor "admin"^^xsd:string ;
    dcterms:created "2021-12-10T07:07:33+00:00"^^xsd:dateTime ;
    dcterms:identifier "1"^^xsd:integer ;
    dcterms:modified "2021-12-10T07:07:33+00:00"^^xsd:dateTime ;
    dcterms:title "Testing bug to issue automation"^^xsd:string ;
    bgz:component "TestComponent"^^xsd:string ;
    bgz:operatingSystem "Linux"^^xsd:string ;
    bgz:platform "PC"^^xsd:string ;
    bgz:priority "---"^^xsd:string ;
    bgz:version "unspecified"^^xsd:string .    
\end{lstlisting}

The creation is notified to the Event Listener of its \ac*{OSLC} adapter, which sends an \ac*{OSLC} modeled event containing all the bug's properties to the Apache Kafka server (Step 2).
Listing~\ref{listing:bugzilla_event} shows an example of such an \events{Event} and its corresponding \events{EventListener}.

\begin{lstlisting}[caption={\ac{RDF} representing an event generated by the creation of a bug.}, label={listing:bugzilla_event}]
    @prefix oslc: <http://open-services.net/ns/core#> .
    @prefix oslc_auto: <http://open-services.net/ns/auto#> .
    @prefix oslc_eca: <http://open-services.net/ns/eca#> .
    @prefix dcterms: <http://purl.org/dc/terms/> .
    @prefix xsd: <http://www.w3.org/2001/XMLSchema#> .
    @prefix example: <http://subclass.example.com#> .

    <http://bugzilla.example.com/ServiceProvider/1/Events/1> a oslc_eca:Event ,
            a example:BugCreationEvent ;
        oslc:serviceProvider <http://bugzilla.example.com/ServiceProvider/1> ;
        oslc_eca:generatedBy <http://bugzilla.example/EventListener/1> ;
        oslc_eca:causedBy <http://bugzilla.example.com/ServiceProvider/1/bugs/1> ;
        dcterms:created "2021-07-14T07:07:33+00:00"^^xsd:dateTime ;
        dcterms:title "Event generated by the creation of a bug"^^xsd:string .
\end{lstlisting}

%     <http://bugzilla.example.com/EventListener/1> a oslc_eca:EventListener ;
%         oslc_eca:source <http://bugzilla.example.com> ;
%         dcterms:title "Listener detecting events from  Bugzilla"^^rdf:XMLLiteral .
% \end{lstlisting}

The \ac*{OSLC} Bugzilla adapter also generates a changelog for the \ac*{TRS} server (Step 3).
It contains two \trs{Creation} resources pointing at the bug created and the event resource generated by the event listener.
The changelog is shown in Listing~\ref{listing:trs}.

\begin{lstlisting}[caption={\ac{RDF} representing the \ac*{TRS} logs that register the creation of the bug and its consequent event.}, label={listing:trs}]
    @prefix trs: <http://open-services.net/ns/core/trs#> .
    @prefix xsd: <http://www.w3.org/2001/XMLSchema#> .

    <http://bugzilla.example.com/trs/ChangeLog/1> a trs:ChangeLog ;
        trs:change [
            a trs:Creation ;
            trs:order "1"^^xsd:integer ;
            trs:changed <http://bugzilla.example.com/ServiceProvider/1/bugs/1> .
        ] ;
        trs:change [
            a trs:Creation ;
            trs:order "2"^^xsd:integer ;
            trs:changed <http://bugzilla.example.com/ServiceProvider/1/Events/1> .
        ] .
\end{lstlisting}

The event is then pushed to the \ac*{OSLC} Automation Server (Step 4), which sends it to the \ac*{TAS} in the corresponding format.
The \ac*{TAS} evaluates its automation rules and returns an action to the \ac*{OSLC} Automation Server.
The \ac*{OSLC} Automation Server stores a changelog in its \ac*{TRS} server, registering the creation of the corresponding AssertedCondition (Step 5).
Listing~\ref{listing:trs_auto} shows an \ac{RDF} representation of this changelog.

\begin{lstlisting}[caption={\ac{RDF} representing the \ac*{TRS} log message generated at the Automation Server.}, label={listing:trs_auto}]
    @prefix oslc_eca: <http://open-services.net/ns/eca#> .
    @prefix trs: <http://open-services.net/ns/core/trs#> .
    @prefix xsd: <http://www.w3.org/2001/XMLSchema#> .

    <http://auto.server/AssertedCondition/1> a oslc_eca:AssertedCondition ;
            oslc_eca:asserts <http://auto.server/ServiceProvider/1/Rule/1> ;
            oslc_eca:produces <http://auto.server/Action/1> ;
            oslc_eca:asserted "true"^^xsd:boolean .

    <http://auto.server/trs/ChangeLog/1> a trs:ChangeLog ;
        trs:change [
            a trs:Creation ;
            trs:order "1"^^xsd:interger ;
            trs:changed <http://auto.server/AssertedCondition/1> .
        ] .
\end{lstlisting}

%     <http://auto.server/ServiceProvider/1/AssertedCondition/1> a oslc_eca:AssertedCondition ;
%         oslc_eca:asserts <http://auto.server/ServiceProvider/1/Rule/1> ;
%         oslc_eca:produced <http://auto.server/Action/1> .
% \end{lstlisting}

Then, the \ac*{OSLC} Automation Server sends an action back to the Apache Kafka server (Step 6).
The \auto{status} and \auto{verdict} properties also indicate that it is still in process of being executed.
For this specific \textit{example:CreateIssueAction}, which is described by a ResourceShape not included for the sake of brevity, the \textit{example:originalChangeRequest} property links to the bug created previously in Bugzilla, which will be the source of information when GitHub's Action Dispatcher would execute the action.
Listing~\ref{listing:github_action} shows the Notation 3 code for this example \actions{Action}.

\begin{lstlisting}[caption={\ac{RDF} representing an action sent for execution to create a GitHub issue.}, label={listing:github_action}]
    @prefix oslc: <http://open-services.net/ns/core#> .
    @prefix oslc_eca: <http://open-services.net/ns/eca#> .
    @prefix oslc_auto: <http://open-services.net/ns/auto#> .
    @prefix oslc_cm: <http://open-services.net/ns/cm#> .
    @prefix dcterms: <http://purl.org/dc/terms/> .
    @prefix xsd: <http://www.w3.org/2001/XMLSchema#> .
    @prefix example: <http://subclass.example.com#> .

    <http://auto.server/Action/1> a oslc_eca:Action,
            example:CreateIssueAction ;
        oslc:serviceProvider <http://github.example.com/ServiceProvider/1> ;
        oslc_actions:executedBy <http://github.example.com/ActionDispatcher/1> ;
        oslc_auto:status oslc_auto:new ;
        oslc_auto:veredict oslc_auto:unavailable ;
        example:originalChangeRequest <http://bugzilla.example.com/ServiceProvider/1/bugs/1> ;
        dcterms:created "2021-07-14T07:07:33+00:00"^^xsd:dateTime ;
        dcterms:title "Action to be executed on GitHub"^^xsd:string .
        
    <http://auto.server/Action/1/versions/1>  dcterms:isVersionOf    
            <http://auto.server/Action/1> .
\end{lstlisting}

%         oslc_actions:executedOn <http://github.example.com/OSLCResource/1> ;
%         dcterms:created "2021-07-14"^^xsd:dateTime ;
%         dcterms:title "Action executed on GitHub"^^rdf:XMLLiteral .

%     <http://github.example.com/ActionDispatcher/1> a oslc_eca:ActionDispatcher ;
%         oslc:serviceProvider <http://github.example.com/ServiceProvider/1> ;
%         oslc:resourceShape <http://github.example.com/CustomAction/resourceShape> ;
%         oslc_actions:execute <http://github.example.com/url/to/post/actions/to/execute> ;
%         dcterms:title "Dispatcher that exposes actions from GitHub"^^rdf:XMLLiteral .
% \end{lstlisting}

The action is delivered to the Action Dispatcher of the GitHub \ac*{OSLC} adapter (Step 7).
It then modifies the action's status and sets it to queued.
A \ac{TRS} changelog is generated representing this update.
It uses \ac{TRS} patch to show what properties changed and is displayed in Listing~\ref{listing:action_update}.

\begin{lstlisting}[caption={\ac{RDF} of a \ac{TRS} changelog showing the changes in an action status.}, label={listing:action_update}]
    @prefix xsd: <http://www.w3.org/2001/XMLSchema#>.
    @prefix dcterms: <http://purl.org/dc/terms/> .
    @prefix trs: <http://open-services.net/ns/core/trs#>.
    @prefix trspatch: <http://open-services.net/ns/core/trspatch#>.
    @prefix oslc_auto: <http://open-services.net/ns/auto#> .
    
    <http://github.example.com/trs/ChangeLog/1> a trs:ChangeLog ;
        trs:change [
            a trs:Creation;
            trs:changed <http://auto.server/Action/1/version/2>;
            trs:order "2"^^xsd:integer;
            trspatch:createdFrom <http://auto.server/Action/1/version/1>;
            trspatch:rdfPatch
                """
                D <http://auto.server/Action/1/versions/1>  dcterms:isVersionOf    
                        <http://auto.server/Action/1> .
                A <http://auto.server/Action/1/versions/2>  dcterms:isVersionOf          
                        <http://auto.server/Action/1> .
                D <http://auto.server/Action/1> oslc_auto:status oslc_auto:new.
                A <http://auto.server/Action/1> oslc_auto:status oslc_auto:queued.
                """ .
            ] .
\end{lstlisting}

The adapter then uses GitHub's API to create the issue described by the action parameters (Step 8).
Listing~\ref{listing:issue} showcases the newly created issue \ac*{RDF} representation.

% % \todo{RDF issue example}
\begin{lstlisting}[caption={\ac{RDF} representing the issues created in GitHub.}, label={listing:issue}]
    @prefix dcterms <http://purl.org/dc/terms/> .
    @prefix oslc_cm: <http://open-services.net/ns/cm#> .
    @prefix oslc: <http://open-services.net/ns/core#> .
    @prefix xsd: <http://www.w3.org/2001/XMLSchema#> .

    <http://github.example.com/ServiceProvider/1/issue/1> a oslc_cm:ChangeRequest ;
        oslc_cm:status "open"^^xsd:string ;
        oslc:serviceProvider <http://github.example.com/ServiceProvider/1 ;
        dctermscontributor "admin"^^xsd:string ;
        dcterms:created "2021-12-10T07:07:33+00:00"^^xsd:dateTime ;
        dctermsidentifier "1"^^xsd:integer ;
        dcterms:modified "2021-12-10T07:07:33+00:00"^^xsd:dateTime ;
        dcterms:title "Testing bug to issue automation"^^xsd:string .
\end{lstlisting}

Finally, the adapter sends the corresponding changelog to its \ac*{TRS} server (Step 9), registering the creation of the issue, as shown in Listing~\ref{listing:issue_trs}.
It also signals the update of the action to have its \actions{executedOn} property pointing to the appropriate resource, which was just created.

\begin{lstlisting}[caption={\ac*{RDF} representing a \ac*{TRS} log registering the correct execution of the action.}, label={listing:issue_trs}]
    @prefix trs: <http://open-services.net/ns/core/trs#> .
    @prefix oslc_auto: <http://open-services.net/ns/auto#> .
    @prefix xsd: <http://www.w3.org/2001/XMLSchema#> .

    <http://github.example.com/trs/ChangeLog/2> a trs:ChangeLog ;
        trs:change [
            a trs:Creation ;
            trs:order "1"^^xsd:integer ;
            trs:changed <http://github.example.com/ServiceProvider/1/issue/1> .
        ] ;
        trs:change [
            a trs:Modification ;
            trs:order "2"^^xsd:integer ;
            trs:changed <http://auto.server/Action/1> ;
            trspatch:createdFrom <http://auto.server/Action/1/version/2>;
            trspatch:rdfPatch
                """
                D <http://auto.server/Action/1/versions/2>  dcterms:isVersionOf    
                        <http://auto.server/Action/1> .
                A <http://auto.server/Action/1/versions/3>  dcterms:isVersionOf          
                        <http://auto.server/Action/1> .
                D <http://auto.server/Action/1> oslc_auto:status oslc_auto:queued.
                A <http://auto.server/Action/1> oslc_auto:status oslc_auto:completed.
                D <http://auto.server/Action/1>      oslc_auto:veredict oslc_auto:unavailable ;
                A <http://auto.server/Action/1>      oslc_auto:veredict oslc_auto:passed ;
                A <http://auto.server/Action/1> oslc_eca:executedOn
                        <http://github.example.com/ServiceProvider/1/issue/1>.
                """ .
            ] .
        ] .
\end{lstlisting}

% resumen final
To summarize, the worked example presents a practical implementation of the architecture based on the proposed \ac*{OSLC} extension to support events and actions. It meets the requirements to address the common issues faced when adopting DevOps. Because both services are integrated using \ac*{OSLC} interfaces, requirement RQ1 is fulfilled, making integrating new services more straightforward and efficient. Automation is separated into its own service accessible via \ac*{OSLC}, satisfying requirement RQ2. Traceability is managed with the \ac*{TRS} protocol, a standard within \ac*{OSLC}, and meets requirement RQ3. Finally, requirement RQ4 is fulfilled because the components are loosely coupled, and a distributed service handles messaging, allowing the architecture to scale horizontally.












% inicialización EventListeners
% Before the workflow can be executed, the \ac{OSLC} Automation Server needs to be initialized.
% There are two \ac{TRS} servers it needs to listen to, so a \events{EventListener} is created for each one.
% The \ac{OSLC} Automation Server creation URL request contains the following \events{EventListener}.
% Creating each \events{EventListener} initializes an \ewe{Channel} in the \ac{TAS} representing the events interface from each tool.

% \begin{lstlisting}[caption={\ac{RDF} representation of the \events{EventListener} that tracks changes from the Bugzilla \ac{TRS} server.}]
% @prefix oslc: <http://open-services.net/ns/core#> .
% @prefix oslc_events: <http://open-services.net/ns/events#> .
% @prefix trs: <http://open-services.net/ns/core/trs#> .
% @prefix dcterms: <http://purl.org/dc/terms/> .
% @prefix rdf: <http://www.w3.org/1999/02/22-rdf-syntax-ns#> .

% <bugzillaEventListener_url> a oslc_events:EventListener ;
%     oslc:serviceProvider <oslcAutomationServiceProvider_url> ;
%     oslc_events:source <bugzillaTrackedResourceSet_url> ;
%     dcterms:title "Listener detecting events from Bugzilla"^^rdf:XMLLiteral .
% \end{lstlisting}

% % inicialización ActionDispatchers
% Once the \events{EventListeners} are created, the Automation Server starts listening for events in both services.
% An \actions{ActionDispatcher} has to be created for each service, so the Automation Server can discover the available \actions{Actions} they provide.
% The following \ac{RDF} represents the \actions{ActionDispatcher} after discovering the actions provided by GitHub.
% Besides the basic creation and update actions through regular HTTP requests, an \actions{PotentialAction} is found, the \actions{ResolveChangeRequest} action.
% Creating these \actions{ActionDispatchers} updates the \ewe{Channels} representing the services with their actions interfaces.

% \begin{lstlisting}[caption={\ac{RDF} representation of the \events{EventListener} that tracks changes from the Bugzilla \ac{TRS} server.}]
% @prefix oslc: <http://open-services.net/ns/core#> .
% @prefix oslc_actions: <http://open-services.net/ns/actions#> .
% @prefix dcterms: <http://purl.org/dc/terms/> .
% @prefix rdf: <http://www.w3.org/1999/02/22-rdf-syntax-ns#> .

% <githubActionDispatcher_url> a oslc_actions:ActionDispatcher ;
%     oslc:serviceProvider <oslcAutomationServiceProvider_url> ;
%     oslc_actions:actionProvider <githubServiceProvider_url> ;
%     oslc_actions:supportedAction <bugzillaResolveChangeRequestPotentialAction_url> ;
%     dcterms:title "Action dispatcher that discovered actions from GitHub"^^rdf:XMLLiteral .
% \end{lstlisting}

% % definición de reglas
% After the \ac{OSLC} Automation Server has started listening for events and has discovered the available actions, rule definition is possible via \ac{EWE} Tasker.
% \ewe{Rules} are written in Notation 3, a compact and readable alternative to RDF's XML syntax. \ac{EWE} Tasker uses the \ac{EYE} reasoner from \citet{verborghDrawingConclusionsLinked2015} to evaluate the events.
% \ac{EYE} stands for \say{Euler Yet another proof Engine}, and it is a further incremental development of Euler, an inference engine supporting logic-based proofs.
% It uses \ac{FOL} to perform the reasoning, expressed with Notation 3 logic.
% This allows us to make an inference like \({  ?x a :Person }  => { ?x  :mother [ a :Person] }\).
% In this case, the evaluated event is on the left side of the formula and the resulting action on the right side.
% Following this method, an \ewe{Rule} can be defined.
% It will be triggered by a bug creation event and execute an issue creation action in consequence.
% The corresponding \auto{AutomationPlan} is also created, allowing \ac{OSLC} services to discover the workflow.

% % paso 1: se crea el bug
% When a bug is created in the Bugzilla client application, an \ac{OSLC} resource representation is exposed in the server.
% This resource is of the type \oslccm{ChangeRequest}, one of the domains defined by the \ac{OSLC} specification.
% It has a set of properties, some of them are present in every \oslccm{ChangeRequest} and others are specific to Bugzilla bugs.
% The following \ac{RDF} represents a generic bug.

% \begin{lstlisting}[caption={Bug \ac{RDF} example.}]
% @prefix bgz: <http://www.bugzilla.org/rdf#> .
% @prefix oslc: <http://open-services.net/ns/core#> .
% @prefix dcterms: <http://purl.org/dc/terms/> .
% @prefix oslc_cm: <http://open-services.net/ns/cm#> .
% @prefix rdf: <http://www.w3.org/1999/02/22-rdf-syntax-ns#> .
% @prefix xsd: <http://www.w3.org/2001/XMLSchema#> .

% <bug_url> a oslc_cm:ChangeRequest ;
%     oslc_cm:severity "enhancement" ;
%     oslc_cm:status "CONFIRMED" ;
%     oslc:serviceProvider <bugzillaServiceProvider_url> ;
%     dcterms:contributor "admin" ;
%     dcterms:created "2021-12-10T07:07:33+00:00"^^xsd:dateTime ;
%     dcterms:identifier 1 ;
%     dcterms:modified "2021-12-10T07:07:33+00:00"^^xsd:dateTime ;
%     dcterms:title "Testing bug to issue automation"^^rdf:XMLLiteral ;
%     bgz:component "TestComponent" ;
%     bgz:operatingSystem "Linux" ;
%     bgz:platform "PC" ;
%     bgz:priority "---" ;
%     bgz:version "unspecified" .    
% \end{lstlisting}

% % paso 2: se detecta el evento
% In the Bugzilla \ac{TRS} server, a new \trs{ChangeEvent} is added to the \trs{ChangeLog}.
% The \ac{TRS} client in the \ac{OSLC} Automation Server discovers this new modification and reads the URI of the \ac{OSLC} resource that has suffered the change.
% The type of modification is a \trs{Creation}.
% The moment it is detected, an \auto{AutomationRequest} (which is a subclass of \events{Event}) is created and exposed in the Automation Server.
% This resource is also an \ewe{Event} and is sent to \ac{EWE} Tasker for evaluation.

% \begin{lstlisting}[caption={\ac{RDF} representing an event generated by the creation of a bug.}]
% @prefix oslc: <http://open-services.net/ns/core#> .
% @prefix oslc_auto: <http://open-services.net/ns/auto#> .
% @prefix oslc_events: <http://open-services.net/ns/events#> .
% @prefix trs: <http://open-services.net/ns/core/trs#> .
% @prefix ewe: <http://gsi.dit.upm.es/ontologies/ewe/ns/> .
% @prefix dcterms: <http://purl.org/dc/terms/> .
% @prefix rdf: <http://www.w3.org/1999/02/22-rdf-syntax-ns#> .

% <event_url> a ewe:Event,
%         oslc_events:Event,
%         oslc_auto:AutoamtionRequest,
%         trs:Creation ;
%     oslc:serviceProvider <oslcAutomationServiceProvider_url> ;
%     trs:changed <bug_url> ;
%     oslc_events:generatedBy <bugzillaEventListener_url> ;
%     dcterms:title "Event generated by the creation of a bug"^^rdf:XMLLiteral .
% \end{lstlisting}

% % paso 3: se evalúan las reglas
% \ac{EWE} Tasker receives the event.
% Every \ewe{Rule} stored in the \ac{TAS} is evaluated using the \ac{EYE} reasoning engine.
% The \ewe{Rule} defined in the initialization step (which \ac{RDF} is shown below) is triggered.
% This results in the execution of a creation action.

% \begin{lstlisting}[caption={\ac{RDF} representing an \ewe{Rule} which sends a creation action to GitHub when it receives a creation event from Bugzilla.}]
% @prefix oslc: <http://open-services.net/ns/core#> .
% @prefix oslc_cm: <http://open-services.net/ns/cm#> .
% @prefix oslc_auto: <http://open-services.net/ns/auto#> .
% @prefix oslc_events: <http://open-services.net/ns/events#> .
% @prefix oslc_actions: <http://open-services.net/ns/actions#> .
% @prefix trs: <http://open-services.net/ns/core/trs#> .
% @prefix ewe: <http://gsi.dit.upm.es/ontologies/ewe/ns/> .
% @prefix dcterms <http://purl.org/dc/terms/> .
% @prefix rdf: <http://www.w3.org/1999/02/22-rdf-syntax-ns#> .
% @prefix http: <http://www.w3.org/2011/http#> .
% @prefix http-methods: <http://www.w3.org/2011/http-methods#> .

% {
%     ?event a ewe:Event,
%             oslc_events:Event,
%             oslc_auto:AutoamtionRequest,
%             trs:Creation ;
%         oslc:serviceProvider <oslcAutomationServiceProvider_url> ;
%         trs:changed <bug_url> ;
%         oslc_events:generatedBy <bugzillaEventListener_url> .
%     <bug_url> a oslc_cm:ChangeRequest ;
%         dcterms:title ?title ;
%         dcterms:contributor ?contributor ;
%         oslc:serviceProvider <bugzillaServiceProvider_url> ;
%         oslc_cm:status ?status .
% } => {
%     oslc_actions:action1 a oslc_actions:Action,
%         a oslc_actions:CreateChangeRequestAction,
%         ewe:Action ;
%         oslc_actions:executes <bugzillaResolveChangeRequestPotentialAction_url> ;
%         oslc_actions:binding [
%             a http:Request ;
%             http:mthd http-methods:POST ;
%             http:requestURI <githubCreation_url> ;
%             http:body [
%                 a oslc_cm:ChangeRequest
%                 dcterms:title ?title ;
%                 dcterms:contributor ?contributor ;
%                 oslc:serviceProvider <githubServiceProvider_url> ;
%                 oslc_cm:status ?status .
%             ]
%         ]        
% }.
% \end{lstlisting}

% % paso 4: se ejecuta la acción
% The triples in the first formula determine that the rule is executed when the subject is an \ewe{Event} and an \auto{AutomationRequest}.
% This event is also a \trs{Creation} and points at the newly created bug URL\@.
% The values from some properties, like title and status, are stored in variables and retrieved when generating the action.
% In the second formula, an \ewe{Action} of the type \actions{CreateChangeRequestActions} is obtained due to the inference. 
% This action uses the previously stored variables to generate the corresponding GitHub issue. 
% The \actions{binding} on the action represents the HTTP request sent to create this issue.
% The \ac{OSLC} adapter exposes the \ac{RDF} representing the issue while creating the actual issue via the GitHub API\@.

% % \todo{RDF issue example}
% \begin{lstlisting}[caption={Issue \ac{RDF}.}]
% @prefix dcterms <http://purl.org/dc/terms/> .
% @prefix oslc_cm: <http://open-services.net/ns/cm#> .
% @prefix oslc: <http://open-services.net/ns/core#> .
% @prefix rdf: <http://www.w3.org/1999/02/22-rdf-syntax-ns#> .
% @prefix xsd: <http://www.w3.org/2001/XMLSchema#> .

% <issue_url> a oslc_cm:ChangeRequest ;
%     oslc_cm:status "open" ;
%     oslc:serviceProvider <githubServiceProvider_url> ;
%     dctermscontributor "admin" ;
%     dcterms:created "2021-12-10T07:07:33+00:00"^^xsd:dateTime ;
%     dctermsidentifier 1 ;
%     dcterms:modified "2021-12-10T07:07:33+00:00"^^xsd:dateTime ;
%     dcterms:title "Testing bug to issue automation"^^rdf:XMLLiteral .
% \end{lstlisting}

% % fin del worflow
% To complete the workflow, both the \events{Event} and \actions{Action} involved are exposed by the \ac{OSLC} Automation Server.
% An \auto{AutomationResult} is also exposed, pointing at both resources and the \auto{AutomationPlan} executed.
% Its \auto{verdict} property shows an \auto{passed} \ac{RDF} value, which means the \auto{AutomationPlan} it reports on was run successfully.
%     oslc_actions:action1 a oslc_actions:Action,
%         a oslc_actions:CreateChangeRequestAction,
%         ewe:Action ;
%         oslc_actions:executes <bugzillaResolveChangeRequestPotentialAction_url> ;
%         oslc_actions:binding [
%             a http:Request ;
%             http:mthd http-methods:POST ;
%             http:requestURI <githubCreation_url> ;
%             http:body [
%                 a oslc_cm:ChangeRequest
%                 dcterms:title ?title ;
%                 dcterms:contributor ?contributor ;
%                 oslc:serviceProvider <githubServiceProvider_url> ;
%                 oslc_cm:status ?status .
%             ]
%         ]        
% }.
% \end{lstlisting}

% % paso 4: se ejecuta la acción
% The triples in the first formula determine that the rule is executed when the subject is an \ewe{Event} and an \auto{AutomationRequest}.
% This event is also a \trs{Creation} and points at the newly created bug URL\@.
% The values from some properties, like title and status, are stored in variables and retrieved when generating the action.
% In the second formula, an \ewe{Action} of the type \actions{CreateChangeRequestActions} is obtained due to the inference. 
% This action uses the previously stored variables to generate the corresponding GitHub issue. 
% The \actions{binding} on the action represents the HTTP request sent to create this issue.
% The \ac{OSLC} adapter exposes the \ac{RDF} representing the issue while creating the actual issue via the GitHub API\@.

% % \todo{RDF issue example}
% \begin{lstlisting}[caption={Issue \ac{RDF}.}]
% @prefix dcterms <http://purl.org/dc/terms/> .
% @prefix oslc_cm: <http://open-services.net/ns/cm#> .
% @prefix oslc: <http://open-services.net/ns/core#> .
% @prefix rdf: <http://www.w3.org/1999/02/22-rdf-syntax-ns#> .
% @prefix xsd: <http://www.w3.org/2001/XMLSchema#> .

% <issue_url> a oslc_cm:ChangeRequest ;
%     oslc_cm:status "open" ;
%     oslc:serviceProvider <githubServiceProvider_url> ;
%     dctermscontributor "admin" ;
%     dcterms:created "2021-12-10T07:07:33+00:00"^^xsd:dateTime ;
%     dctermsidentifier 1 ;
%     dcterms:modified "2021-12-10T07:07:33+00:00"^^xsd:dateTime ;
%     dcterms:title "Testing bug to issue automation"^^rdf:XMLLiteral .
% \end{lstlisting}

% % fin del worflow
% To complete the workflow, both the \events{Event} and \actions{Action} involved are exposed by the \ac{OSLC} Automation Server.
% An \auto{AutomationResult} is also exposed, pointing at both resources and the \auto{AutomationPlan} executed.
% Its \auto{verdict} property shows an \auto{passed} \ac{RDF} value, which means the \auto{AutomationPlan} it reports on was run successfully.ction} of the type \actions{CreateChangeRequestActions} is obtained due to the inference. 
% This action uses the previously stored variables to generate the corresponding GitHub issue. 
% The \actions{binding} on the action represents the HTTP request sent to create this issue.
% The \ac{OSLC} adapter exposes the \ac{RDF} representing the issue while creating the actual issue via the GitHub API\@.

% % \todo{RDF issue example}
% \begin{lstlisting}[caption={Issue \ac{RDF}.}]
% @prefix dcterms <http://purl.org/dc/terms/> .
% @prefix oslc_cm: <http://open-services.net/ns/cm#> .
% @prefix oslc: <http://open-services.net/ns/core#> .
% @prefix rdf: <http://www.w3.org/1999/02/22-rdf-syntax-ns#> .
% @prefix xsd: <http://www.w3.org/2001/XMLSchema#> .

% <issue_url> a oslc_cm:ChangeRequest ;
%     oslc_cm:status "open" ;
%     oslc:serviceProvider <githubServiceProvider_url> ;
%     dctermscontributor "admin" ;
%     dcterms:created "2021-12-10T07:07:33+00:00"^^xsd:dateTime ;
%     dctermsidentifier 1 ;
%     dcterms:modified "2021-12-10T07:07:33+00:00"^^xsd:dateTime ;
%     dcterms:title "Testing bug to issue automation"^^rdf:XMLLiteral .
% \end{lstlisting}

% % fin del worflow
% To complete the workflow, both the \events{Event} and \actions{Action} involved are exposed by the \ac{OSLC} Automation Server.
% An \auto{AutomationResult} is also exposed, pointing at both resources and the \auto{AutomationPlan} executed.
% Its \auto{verdict} property shows an \auto{passed} \ac{RDF} value, which means the \auto{AutomationPlan} it reports on was run successfully.


\section{Conclusions and future work}\label{sec:conclusions_and_future_work}

This paper presents an extension model for the \ac*{OSLC} standard to support \ac*{ECA}-based automation. More specifically, it provides the concepts to make \ac*{ECA} automation possible in an interoperable environment. The end goal of this proposal is to use \ac*{OSLC} to improve the adoption of DevOps by facilitating integration and automation between tools from different vendors or domains.

The proposed extension model has been validated by studying the main issues in the DevOps field, and the architectural characteristics deemed beneficial by practitioners. Next, a set of requirements has been established based on the said issues and used for validation. Then, a prototype architecture was designed to assess the requirements of the motivational environment. Finally, a worked example has been conducted that involves two major software development tools, GitHub and Bugzilla, to detail the application of the proposed model in a real-world workflow.

% The paper also presents an extension to the \ac{OSLC} Automation specification.
% Its goal is to serve as a model for a DevOps system that supports service integration and rule-based automation, linking the \ac{OSLC} and \ac{EWE} vocabularies.
% Although both the \ac{OSLC} standard and the \ac{TRS} protocol allow easy and flexible integration between services, the \ac{EWE} ontology enables the definition of semantic rules following the \ac{ECA} model.
% Together, they have served as the foundation for an architecture capable of executing semantically defined workflows between services using standardized interfaces.

% In conclusion, this paper attempts to address some challenges faced when implementing DevOps by combining them with semantic technologies.
% Some of these technologies are still in the early stages of their development and implementing architectures like the one proposed here will become easier as they continue to grow.
% However, if the Linked Data technologies and \ac{OSLC} increase in attractiveness and incentives vendors to integrate them into their services, proposals like this will become even more relevant.

The contribution is expected to be helpful in automated environments where interoperability is a priority. Moreover, the semantic model could potentially contribute to the open project of the \ac*{OSLC} standard.

For future work, a methodology for adopting \ac*{OSLC} and \ac*{ECA} automation for DevOps professionals will be developed. This will make the architecture easier to translate into a real-world scenario. Also, other kinds of service (beyond the change management domain) will be integrated into the architecture to test more characteristics relevant to DevOps and to provide a more complex and interesting worked example. For example, tools used to manage infrastructures, such as Docker~\citep{Docker} or Kubernetes~\citep{Kubernetes}, are very popular in DevOps environments. They can help manage large architectures efficiently and accelerate the deployment of services.
Standardizing their interfaces with \ac*{OSLC} could help integrate them with even more tools and enable event-based automation.

Another aspect of \ac*{ECA} that will be explored in the future is automation rules. Semantic web technologies have also been proposed in this domain. A bridge between one \ac*{ECA} ontology, such as \ac*{EWE}~\citep{coronadoModellingRulesAutomating2015}, and the proposed model would allow for more powerful automation features.

Furthermore, other issues concerning DevOps are explored in the systematic review used to establish the requirements of the prototype architecture~\citep{bolscherDesigningSoftwareArchitecture2019}. These challenges could be explored in future work, such as testing and monolithic databases~\citep{bolscherDesigningSoftwareArchitecture2019}. In addition, the model can still be extended to support new features and domains. For example, automation infrastructures often provide a login system for users that has not been implemented in this use case. The vocabulary can be extended further with new concepts to cover all these aspects.
% \input{problem_overview.tex}
% \input{architecture.tex}
% 
\newcommand{\wronglabel}[1]{\textcolor{red}{\textit{(#1)}}}
\newcommand{\correctlabel}[1]{\textcolor{blue}{\textit{(#1)}}}
\begin{table*}[!t]
% \renewcommand\arraystretch{1.2}
\caption{
(For MNLI) Examples of generated second sequence (hypothesis) by generators tuned with three different objectives conditioned on a given first sequence (premise) ``\textit{In 2009, hair samples from 1,137 Taiji residents were tested for mercury by the National Institute for Minamata Disease (NIMD)}''. The true label of the generated sequence is marked at the end of the sequence (if the generated sequence correctly pertains to the target label, it is marked in \textcolor{blue}{blue}; otherwise, it is in \textcolor{red}{red}).
}
\vspace{-0.5em}
\centering
\small 
\resizebox{\textwidth}{!}{
\begin{tabular}{lll}
\toprule
\textbf{Objective} & \textbf{Label} & \textbf{Generated Second Sequence} \\
\midrule
\multirow{4}{*}{$\mathcal{L}_{\text{gen}}$} & \makecell[l]{entailment\\  } & \makecell[l]{The NIMD found that hair samples from 1,137 Taiji residents had mercury in their hair. \wronglabel{neutral}} \\
& \makecell[l]{neutral\\ } & \makecell[l]{The NIMD found that there was no evidence of a link between exposure to high levels \\ of mercury and thyroid cancer. \correctlabel{neutral}} \\
& \makecell[l]{contradiction\\ } & \makecell[l]{There was no evidence of mercury in hair samples from Taiji. \wronglabel{neutral}}\\
\midrule
\multirow{3}{*}{$\mathcal{L}_{\text{gen}}+\mathcal{L}_{\text{disc}}$} & {entailment} & The number of hairs in a sample is equal to the number of people who lived in Taiji. \wronglabel{neutral} \\
& {neutral} & The results showed that there was no significant difference in levels of mercury. \correctlabel{neutral} \\
& contradiction & Hair samples from 1,137 Taiji residents were not tested. \correctlabel{contradiction} \\
\midrule
\multirow{4}{*}{$\mathcal{L}_{\text{w-gen}}$} & {entailment} & \makecell[l]{The NIMD tested hair samples from 1,137 residents of Taiji. \correctlabel{entailment}} \\
& {neutral} & \makecell[l]{There was no significant difference in levels between people who lived near a nickel mine \\ and those living far away. \correctlabel{neutral}} \\
& contradiction & The NIMD did not test any of the hair samples. \correctlabel{contradiction} \\
\bottomrule
\end{tabular}
}
\vspace{-.5em}
\label{tab:case_studies}
\end{table*}

% \section{Performance and Usability}
~\label{testing} 
To gain insights into our design and distill some lessons, we evaluated the performance and usability of our off-the-shelf, experimental framework in two ways; i) System Testing (frequently carried out by our team), and ii) Quality Assurance Testing (carried out by prospective users).  
Note that the framework's security analysis requires sending a zero-click exploit to the victim and forensically investigating if the victim's device has been infected or not.
We could not test the framework from a security perspective owing to the absence of zero-click binaries. 
However, the fact that the chat applications were sandboxed/isolated means the design principle of separation of privilege secures them.  

\begin{table}[t]
\centering
\caption{Evaluating connection establishment phase with different client OS, server deployments, and networks.}
\label{tab:conntime}
\resizebox{\columnwidth}{!}{%
\begin{tabular}{l|l|l|c|c|c|c|c} 
\hline
\multicolumn{1}{c|}{\multirow{2}{*}{\begin{tabular}[c]{@{}c@{}}\textbf{Client }\\\textbf{OS}\end{tabular}}} & \multicolumn{1}{c|}{\multirow{2}{*}{\begin{tabular}[c]{@{}c@{}}\textbf{Access}\\\textbf{Approach}\end{tabular}}} & \multicolumn{1}{c|}{\multirow{2}{*}{\begin{tabular}[c]{@{}c@{}}\textbf{Connection}\\\textbf{Status}\end{tabular}}} & \multicolumn{3}{c|}{\textbf{Wi-Fi}}  & \multicolumn{2}{c}{\textbf{Cellular}}  \\ 
\cline{4-8}
\multicolumn{1}{c|}{}                                                                                       & \multicolumn{1}{c|}{}                                                                                            & \multicolumn{1}{c|}{}                                                                                              & \textbf{A} & \textbf{B} & \textbf{C} & \textbf{A} & \textbf{B}                \\ 
%\hline
%\multicolumn{8}{c}{\textbf{GCP Cloud Server}}                                                                                                                                                                                                                                                                                                                                                                                       \\ 
\hline
\multirow{4}{*}{Android}                                                                                & \multirow{4}{*}{\begin{tabular}[c]{@{}l@{}}WebRTC\end{tabular}}                                          & Successful                                                                                                         & 25         & 25         & 25         & 25         & 25                        \\ 
\cline{3-8}
                                                                                                            &                                                                                                                  & Failed                                                                                                             & 0          & 0          & 0          & 0          & 0                         \\ 
\cline{3-8}
                                                                              &                                                                                                                  & Re-tried                                                                                                           & 0          & 0          & 0          & 0        & 0                         \\ 
\cline{3-8}
                                                                                              &                                                                                                                  & Terminated                                                                                                         & 0          & 0          & 0          & 0          & 0                         \\ 
\hline
\multirow{4}{*}{iOS}                                                                                        & \multirow{4}{*}{\begin{tabular}[c]{@{}l@{}}Screen\\Capturing\end{tabular}}                                           & Successful                                                                                                         & 25         & 25         & 25        & 25         & 25                        \\ 
\cline{3-8}
                                                                              &                                                                                                                  & Failed                                                                                                            & 0          & 0          & 0          & 0          & 0                         \\ 
\cline{3-8}
                                                                                  &                                                                                                                  & Re-tried                                                                                                           & 0          & 0          & 0          & 0          & 0                         \\ 
\cline{3-8}
                                                                                                  &                                                                                                                  & Terminated                                                                                                         & 0          & 0          & 0          & 0          & 0                         \\ 
\hline
%\multicolumn{8}{c}{\textbf{Dedicated Server}}                                                                                                                                                                                                          \\ 
%\hline
%\multirow{4}{*}{Android}                                                                                    & \multirow{4}{*}{WebRTC}                                                                                          & Successful                                                                                                         & 25         & 25         & 25         & 25         & 25                        \\ 
%\cline{3-8}
%                                                                                                            &                                                                                                                  & Failed                                                                                                             & 0          & 0          & 0          & 0          & 0                         \\ 
%\cline{3-8}
                                                  %                                                          &                                                                                                                  & Re-attempts                                                                                                           & 0          & 0          & 0          & 0          & 0                         \\ 
%\cline{3-8}
%                                                                                                            &                                                                                                                  & Terminated                                                                                                         & 0          & 0          & 0          & 0          & 0                         \\ 
%\hline
%\multirow{4}{*}{iOS}                                                                                        & \multirow{4}{*}{\begin{tabular}[c]{@{}l@{}}Screen \\Capturing\end{tabular}}                                      & Successful                                                                                                         & 25         & 25         & 25         & 25         & 25                        \\ 
%\cline{3-8}
                                                                                                    %        &                                                                                                                  & Failed                                                                                                             & 0          & 0          & 0          & 0          & 0                         \\ 
%\cline{3-8}
 %                                                                                                           &                                                                                                                  & Re-attempts                                                                                                           & 0          & 0          & 0          & 0          & 0                         \\ 
%\cline{3-8}
                                                                                                    %        &                                                                                                                  & Terminated                                                                                                         & 0          & 0          & 0          & 0          & 0                         \\
%\hline
\end{tabular}}
\end{table} 
\subsection{Setup} We performed tests on eight different smartphones to ensure that our implementation was independent of the screen sizes, device manufacturers and web browsers. 
The Android phones included LG V40 ThinQ, Motorolla Nexus 6, Samsung Galaxy S10E, Oppo A5 and Vivo U1, whereas the iOS devices included iPhone XR, iPhone X and iPhone 7. 
In addition, we also tested other device models by creating virtual devices in the Android Emulator~\cite{androidstudio}.  
We set up the server side on a Linux-based VM instance on GCP. 
This instance had nested virtualization enabled and was accessible over the Internet using public IP. 
We ran three Docker containers on GCP, one for each application (WhatsApp, Signal and regular messaging application) and configured the applications using Google Voice number.  
For testing, we set up three more GCP accounts and configured the applications with different Google voice numbers. 
This essentially allowed the participants of the user study to test the framework without linking their personal accounts on the server before they were fully satisfied. 
To encourage active involvement from the participants, group chats were also created where random texts, images, gifs, etc., were frequently shared.

\subsection{System Testing} 
\label{performance} 
Here, we assessed our system's connection time, usability, introduced latency, and scalability. 

\paragraph{Connection Time:} 
To evaluate the connection establishment phase, we accessed the remote server on several phones at different times, using various networks. 
For accessing the remote server, we specifically used the WebRTC-based screen sharing approach for Android clients and the PNG-based screen sharing approach for iOS clients.  
We made 25 connection attempts from the client device to the remote server and observed how often the connection fails, how many attempts are required to reconnect, and if the connection ever terminates itself while the system is being used. 
Each round of the experiment lasted for an hour. Our results (Table~\ref{tab:conntime}) indicate that the client was able to connect to the server in the first attempt seamlessly, and the connection was stable throughout the experiment. 

\paragraph{Usability:} 
To evaluate our experimental system's usability, we accessed the remote server from smartphones of different screen sizes, device
manufacturers, and web browsers (notably Google Chrome, Firefox, Microsoft Edge, Opera, UC and Oppo browser). In all instances, the text was readable, the remote display was of high quality and covered maximum screen size. 
However, as our framework added an additional layer to the communication path, it introduced a lag which sometimes created usability challenges. 

\paragraph{Measuring Lag:}
In practice, it is difficult to measure the introduced lag precisely, as the clocks of the two computing devices (phone and server) are not in sync. 
Therefore, we measured the additional transmission time between client and server as half of the Round Trip Time (RTT) between these two nodes. 
For this, we developed client and server Android applications, whereby the client sends a test message to the server, which replies with an acknowledgement (ACK).
The client then measures RTT as the difference between the time the message was sent, and the ACK was received. 
To minimize the measurement error, the client sends ten messages per second and calculates the average lag. 
Although the RTT measurement is influenced by other factors such as network speed, traffic load, etc., the measurements provided a quantitative value to the introduced lag.
Table~\ref{tab:delay} reports the average lag in seconds (s) over different networks. 
One important question here was, given the security benefits of our solution and that the user experience was not always significantly affected, whether the 0.49 seconds lag is acceptable?  \begin{table}
\centering
\caption{Measuring lag introduced by COTS components at different times and network connections.}
\label{tab:delay}
\begin{tabular}{l|c|c|c|c|c} 
\hline
\multicolumn{1}{c|}{\multirow{3}{*}{\textbf{Client Device}}} & \multicolumn{5}{c}{\textbf{Introduced Lag (s)}}                                                                                           \\ 
\cline{2-6}
\multicolumn{1}{c|}{}                                        & \multicolumn{3}{c|}{\textbf{Wi-Fi}}                                               & \multicolumn{2}{l}{\textbf{Cellular}}                 \\ 
\cline{2-6}
\multicolumn{1}{c|}{}                                        & \textbf{A}                & \textbf{B}                & \textbf{C}                & \textbf{A}                & \textbf{B}                \\ 
\hline
LG V40 ThinQ                                                 & 0.44                      & 0.47                      & 0.42                      & 0.49                      & 0.51                      \\ 
\hline
Motorolla Nexus 6                                            & 0.45                      & 0.51                      & 0.43                      & 0.50                      & 0.53                      \\ 
\hline
Samsung Galaxy S10E                                          & 0.47                      & 0.52                      & 0.49                      & 0.53                      & 0.56                      \\ 
\hline
Oppo A5                                                      & 0.42                      & 0.53                      & 0.54                      & 0.54                      & 0.55                      \\ 
\hline
Vivo U1                                                      & \multicolumn{1}{l|}{0.45} & \multicolumn{1}{l|}{0.48} & \multicolumn{1}{l|}{0.49} & \multicolumn{1}{l|}{0.48} & \multicolumn{1}{l}{0.50}  \\ 
\hline\hline
\textbf{Average Lag (s)        }                                      & \multicolumn{5}{c}{ 0.49}                                                                                                                 \\
\hline
\end{tabular}
\end{table} 

\paragraph{Resource Requirement and Scalability:} 
Realistically, the user might wish to port additional chat applications (e.g., Viber, Telegram) remotely. 
To determine if our COTs-based framework was scalable, we performed cost analysis for a single user with respect to the increase in the number of remote applications.  
We observed that for efficient performance, the bare-minimum requirement for each Android emulator to run a chat application is 4GB RAM and 2GB disk space.  
Hence, we kept the GCP cloud instance with the Intel Haswell CPU Platform, 8GB RAM and 100GB disk space as the baseline for running one chat application.
As disk space is not a hard constraint, we noted the monthly cost as we increased the required RAM (16, 32 and 64 GB) corresponding to the number of applications to be hosted.  
Note that the user can also self-deploy the server side to lessen the recurring expenditure. 
To provide cost analysis for a dedicated server, we kept a Dell PowerEdge server with 8GB RAM and 1TB disk space (\$780) as the base server for running one chat application and kept increasing the RAM as the number of applications increased. 
The results in Figure~\ref{fig:plot} suggest that despite high initial investment, the dedicated server is more cost-effective than a cloud setup in the long run.
However, our rationale was that the GCP cloud setup is practically more secure in terms of the server's physical and logical security.

\begin{figure}[t]
\begin{center}
\includegraphics[width=\linewidth]{figures/plot.pdf}
\caption{Cost vs resource analysis per user for deploying server end on Google Cloud Platform and dedicated server.}
\label{fig:plot}
\end{center}
\end{figure}

\subsection{Quality Assurance Testing} \label{usability}
To determine our envisioned system's usability in practice, we got the system evaluated by potential users and received feedback.

\paragraph{Ethical Considerations:} 
We note that our usability study was exempted under the IRB Exemption Category 3 - Benign Behavioral Interventions by our institution's IRB.  
Although our study involves human subjects, it does not collect sensitive and personal information or perform deception, attacks, etc.

To recruit participants for the user study, we sent out invitation emails to 30 individuals from our personal and professional circle belonging to different regions of the world.  
On a high level, all participants were above 18 years and included i) potential targets of zero-click attacks (including three journalists) and ii) privacy-conscious smartphone users.  
Of 30, 27 individuals showed willingness to participate in the user study, whereas 3 individuals did not respond.
As the user study was voluntary, we did not send follow-up emails. 
We scheduled Zoom meetings with the participants to obtain informed consent, demonstrate the working of our system and provide instructions regarding the user study.  
Since participants might initially have privacy concerns regarding setting up their personal chat accounts on the remote server, we pre-configured the applications on each GCP test account using Google voice numbers.
We did not collect personally identifiable information (e.g., name or email of the participants) or other related data like IP addresses during the study.
To maintain the confidentiality of the results, we analyzed and reported the responses in our work as group data without identifying any individual.
Finally, after the user study, we deleted all the conversations from the server. 

\paragraph{User Study and Results:}
The tasks involved accessing the remote server by typing the provided public IP of the server in the mobile web browser.
The participants were given temporary login credentials to prevent unauthorized access to the remote server. 
Once logged in, each participant was asked to exchange random text messages, images, gifs, etc., with the saved contacts or engage in group chats.  
Our team controlled the other contacts and group chats to let the participant actively communicate using the system and observe if they experienced any possible delay or performance degradation.
After using the system for an hour, participants shared their feedback on the following via a brief Google survey form (provided in the Appendix).  

\begin{enumerate}

\item Prior know-how of zero-click attacks, 

\item Experience connecting to the remote server, i.e., Is it accessible? Is the connection stable or terminates frequently? 

\item Experience of sending test messages, i.e., Is the system user-friendly? Is the lag acceptable? 

\item Suggestions to improve the system.

\end{enumerate}

The results indicated that 21 out of 27 participants had a fair idea about zero-click attacks before starting the user study.  
Figure~\ref{fig:conn} shows the results of the participant's experience while connecting to the remote server.
As shown, for all 27 participants accessing the server from different regions at different timings, the remote server was accessible in the first attempt, and the connection remained stable throughout the study (i.e., it did not terminate at all).
  
\begin{figure}[t] \begin{center}
\includegraphics[width=\linewidth]{figures/conn.pdf} \caption{Evaluating Server's Accessibility and Connection Status via User Study: All 27 participants found the server accessible in the first attempt and the connection stable.}
\label{fig:conn} \end{center} \end{figure}

Figure~\ref{fig:usability} indicates the results of the participant's experience while sending test messages. 
For this, we specifically evaluated if the participants found the system user-friendly (i.e., text was readable, the screen was full size, etc.) and whether the lag introduced by the COTS-based components was acceptable considering the security benefits of the solution. 
The results indicated that 21 out of 27 participants found the system to be user-friendly and the lag (interestingly) acceptable.
Six participants found the system to be user-friendly but complained that the lag was significant. 
Overall, the participants gave constructive feedback, such as keeping the native keyboard on-screen while controlling the remote screen and ensuring that the user does not experience any lag.

\begin{figure}[t]
\begin{center}
\includegraphics[width=\linewidth]{figures/usability.pdf}
\caption{Evaluating System's Usability via User Study: Here, usable = user-friendly and acceptable lag, partially usable = user-friendly but unacceptable lag, unusable = not friendly and unacceptable lag. Result: 21 participants found the system usable, while 6 participants deemed it partially usable.}
\label{fig:usability}
\end{center}
\end{figure}



% \bibliographystyle{elsarticle-harv}
% \bibliography{introduction,state_of_the_art,problem_overview,semantic_model,architecture,case_study,evaluation,conclusion}
\bibliography{references}
\bibliographystyle{plainnat}


\end{document}
