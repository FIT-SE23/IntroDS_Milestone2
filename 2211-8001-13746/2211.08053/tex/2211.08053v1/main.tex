\documentclass[a4paper,11pt]{article}
\usepackage[utf8]{inputenc}

\usepackage{amsmath, amssymb, amsthm}
\usepackage[english]{babel}
\usepackage{xcolor}
\usepackage[top=2.5cm,bottom=2.5cm,left=2.5cm,right=2.5cm]{geometry}
\usepackage{graphicx}
\usepackage{hyperref}
\usepackage{url}

\makeatletter
\g@addto@macro\bfseries{\boldmath}
\makeatother

\title{A higgledy-piggledy set of planes\\ based on the ABB-representation of linear sets}
\author{Lins Denaux \\ {\it Ghent University} \and Jozefien D'haeseleer \\ {\it Ghent University} \and Geertrui Van de Voorde \\ {\it University of Canterbury}}
\date{}

\newtheorem{thm}{Theorem}[section]
\newtheorem{lm}[thm]{Lemma}
\newtheorem{res}[thm]{Result}
\newtheorem{crl}[thm]{Corollary}
\newtheorem{prop}[thm]{Proposition}

\theoremstyle{definition}
\newtheorem{rmk}[thm]{Remark}
\newtheorem{df}[thm]{Definition}
\newtheorem{constr}[thm]{Construction}

\newcommand{\NN}{\mathbb{N}}
\newcommand{\FF}{\mathbb{F}}

\newcommand{\vspan}[1]{\left\langle#1\right\rangle}
\newcommand{\vspanq}[2]{\vspan{#1}_{#2}}
\newcommand{\set}[1]{\left\{#1\right\}}
\newcommand{\sett}[2]{\left\{#1\,:\,#2\right\}}
\newcommand{\V}[2]{\mathrm{V}\!\mspace{-2mu}\left(#1,#2\right)}
\newcommand{\pg}[2]{\textnormal{PG}\!\left(#1,#2\right)}
\newcommand{\pgTitle}[2]{\textnormal{\textbf{PG}}\!\left(#1,#2\right)}
\newcommand{\ag}[2]{\textnormal{AG}\!\left(#1,#2\right)}
\newcommand{\spread}{\mathcal{D}}
\newcommand{\linset}{\mathcal{L}}
\newcommand{\pointset}{\mathcal{P}}
\newcommand{\mS}{\mathcal{S}}
\newcommand{\qbin}[2]{\genfrac{[}{]}{0pt}{}{#1}{#2}_q}

\renewcommand{\geq}{\geqslant}
\renewcommand{\leq}{\leqslant}

\setlength{\parindent}{0pt}

\begin{document}

\maketitle

\begin{abstract} %The André/Bruck-Bose correspondence represents points of the projective plane $\pg{2}{q^t}$ in $\pg{2t}{q}$. 

In this paper, we investigate the André/Bruck-Bose representation of certain $\FF_q$-linear sets contained in a line of $\pg{2}{q^t}$. We show that {\em scattered $\FF_q$-linear sets of rank $3$} in $\pg{1}{q^3}$ correspond to particular hyperbolic quadrics and that {\em $\FF_q$-linear clubs} in $\pg{1}{q^t}$ are linked to subspaces of a certain $2$-design based on normal rational curves; this design extends the notion of a {\em circumscribed bundle of conics}. Finally, we use these results to construct optimal higgledy-piggledy sets of planes in $\pg{5}{q}$.



%    We consider the André/Bruck-Bose representation of the projective plane $\pg{2}{q^t}$ in $\pg{2t}{q}$. We investigate the representation of a linear set of rank $k$ on a line, different from the line at infinity in $\pg{2}{q^t}$. More precisely, we characterize the representation of tangent scattered linear sets on a line for $k=3, t=3$, tangent clubs with head point contained in the line at infinity for $k\geq 3, t\geq k,$ and tangent clubs with head point not contained in the line at infinity for $k=3, t\geq 3$.
%    This characterisation of the the André/Bruck-Bose representation of linear sets can be used to find optimal Higgledy-piggledy sets.
   % \textcolor{magenta}{Misschien/hopelijk kunnen we ook nog iets zeggen voor $k\geq 3$.}
\end{abstract}

{\it Keywords:} André/Bruck-Bose representation, linear set, club, scattered linear set, normal rational curve, circumscribed bundle, higgledy-piggledy set

{\it Mathematics Subject Classification:} 51E20.




\section{Introduction}
\subsection{Motivation and overview}

{\em Linear sets} are particular point sets in a finite projective space. They are of interest in finite geometry, and have been studied in recent years through their connections with other topics such as {\em blocking sets}, and their applications in coding theory (see e.g.\ \cite{olga,LavrauwVandeVoordeFieldRed,olgaZullo}). Linear sets generalise the concept of a subgeometry as it has been shown that every linear set is either a subgeometry or the projection of a subgeometry \cite{lunardon}.

The {\em Andr\'e/Bruck-Bose representation} is a way to represent the projective plane over the field $\FF_{q^t}$ with $q^t$ elements, as an incidence structure defined over the subfield $\FF_q$. It is a natural question to study the ABB-representation of certain `nice' sets in the plane, and this has previously been done for sets such as sublines and subplanes \cite{RotteySheekeyVandeVoorde}, (sub)conics \cite{ABBconics} and Hermitian unitals \cite{ABB1}. As such, one can ask the same question about the ABB-representation of $\FF_q$-linear sets; we will give a partial answer in this paper.



We will see that the ABB-representation of a certain type of linear set gives rise to an interesting point set which can be described by using a subspace of a {\em design} of certain {\em normal rational curves}. This design is a generalisation of a well-known design based on the conics of a {\em circumscribed bundle of conics} \cite{BakerBrownEbertFisher}. 

After having introduced the necessary background and definitions in Section \ref{sec1}, we will show in Section \ref{sec2} how to construct this design in a geometric way, and use coordinates to show that the obtained design is, in fact, isomorphic to the design of points and lines in a projective space.

In Sections \ref{sec3} and \ref{sec4}, we will turn our attention towards the ABB-representation of clubs of rank $k$ in $\pg{1}{q^t}$ (Theorem \ref{thmclub}) and scattered linear sets of rank $3$ in $\pg{1}{q^3}$ (Theorem \ref{thmscattered}), both tangent to the line at infinity $\ell_\infty$. 

In Section \textcolor{cyan}{\ref{sec5}}, we first provide the necessary background on higgledy-piggledy sets, and then use the results of Sections \ref{sec3} and \ref{sec4} to show the existence and give explicit constructions of sets of seven planes in $\pg{5}{q}$ in higgledy-piggledy arrangment. This answers an open question of \cite{Denaux}. It was this link which provided the incentive to consider the problem of determining the ABB-representation of linear sets in $\pg{1}{q^3}$.



%
%\textcolor{red}{Introductie over het belang van linear sets en het reeds gedane onderzoek over de ABB-representatie van deelrechten en -vlakken; de uitbreiding naar linear sets lijkt een logische stap gezien deze vaak worden beschreven als een veralgemening van deelmeetkundes.}
%
%\section{Preliminaries}
%
%Within this article, we assume $q$ to be an arbitrary prime power.
%Correspondingly, the Galois field $\textnormal{GF}(q)$ of order $q$ will be denoted by $\FF_q$ and the Desarguesian projective space of (projective) dimension $n\in\NN$ over $\FF_q$ will be denoted by $\pg{n}{q}$.
%By omitting a hyperplane $H_\infty$ in $\pg{n}{q}$, we naturally obtain the Desarguesian affine space of dimension $n$ over $\FF_q$, which we will denote by $\ag{n}{q}$, its subspaces being \emph{affine} subspaces.
%Generally, if a subspace is mentioned, this subspace is assumed to be a \emph{projective} subspace.
%Whenever we emphasize a subspace $\pi$ to be affine, its unique \emph{projective completion} will be denoted by $\bar{\pi}$ (naturally, this definition depends on the choice of hyperplane $H_\infty$).
%
%
%Throughout this article, besides $n\in\NN$, we consider natural numbers $k,r\in\NN$ and $t\in\NN\setminus\set{0}$.
%
%\bigskip
%A key concept throughout this article is the notion of an $\FF_q$-subgeometry.
%
%\begin{df}\label{Def_Subgeometry}
%	Let $m\in\NN$, $m\leq n$.
%	An $m$\emph{-dimensional} $\FF_q$\emph{-subgeometry} $\mathcal{B}$ of $\pg{n}{q^t}$ is a set of subspaces (points, lines, \ldots, $(m-1)$-dimensional subspaces) of $\pg{n}{q^t}$, together with the incidence relation inherited from $\pg{n}{q^t}$, such that $\mathcal{B}$ is isomorphic to $\pg{m}{q}$.
%	
%	If $m=1$ or $m=2$, we will often call $\mathcal{B}$ an \emph{$\FF_q$-subline} or an \emph{$\FF_q$-subplane} of $\pg{n}{q^t}$, respectively.
%	Moreover, we will denote the $m$-dimensional subspace of $\pg{n}{q^t}$ spanned by the points of $\mathcal{B}$ by $\vspanq{\mathcal{B}}{q^t}$.
%	If $m=n$, we will omit the dimension and simply call $\mathcal{B}$ an $\FF_q$-subgeometry of $\pg{n}{q^t}$.
%	Lastly, whenever $q$ is clear from context, the prefix `$\FF_q$-' will often be omitted.
%	
%	\textcolor{red}{Waarschijnlijk is meeste van dit laatste blokje niet nodig, mag dan weggelaten worden.}
%\end{df}
%
%\subsection{Field reduction, Desarguesian spreads, indicator spaces and linear sets}
%
%This subsection aims to give a summary on field reduction, linear sets and other relevant topics.
%A great survey on this can be found in \cite{LavrauwVandeVoordeFieldRed}.
%
%\subsubsection{Field reduction, Desarguesian spreads and indicator spaces}
%
%The idea behind \emph{field reduction} is interpreting a projective geometry $\pg{r-1}{q^t}$ as its underlying vector space $\V{r}{q^t}$, which is known to be isomorphic to $\V{rt}{q}$, which in turn naturally translates to $\pg{rt-1}{q}$.
%Thus, one obtains a correspondence between subspaces of $\pg{r-1}{q^t}$ and subspaces of $\pg{rt-1}{q}$ by `reducing' the underlying field.
%In this way, the point set of $\pg{r-1}{q^t}$ corresponds to a set $\spread$ of $(t-1)$-dimensional subspaces of $\pg{rt-1}{q}$ that partition the point set of $\pg{rt-1}{q}$.
%Hence, this set of $(t-1)$-subspaces forms a $(t-1)$-spread of $\pg{rt-1}{q}$; any spread isomorphic to $\spread$ is called a \emph{Desarguesian $(t-1)$-spread}.
%
%Alternatively, one can describe this correspondence using coordinates.
%A point $P$ with coordinates $(x_0,x_1,\dots,x_{r-1})\in\FF_{q^t}^r$ corresponds to the point set in $\pg{rt-1}{q}$ with coordinates $\sett{(\alpha x_0,\alpha x_1,\dots,\alpha x_{r-1})_q}{\alpha\in\FF_{q^t}}$.
%Note that we have used $r$ coordinates from $\FF_{q^t}$, defined up to $\FF_q$-scalar multiple, to define points of $\pg{rt-1}{q}$, and the set point set consists of $\frac{q^t-1}{q-1}$ different points forming a $(t-1)$-dimensional space over $\FF_q$.
%Hence, we find that $\spread$ is given by the set of $(t-1)$-spaces 
%\[
%    \sett{(\alpha x_0,\alpha x_1,\dots,\alpha x_{r-1})_q}{\alpha\in\FF_{q^t}}\quad\textnormal{for all }(x_0,x_1,\ldots,x_{r-1})\in\V{r}{q^t}\textnormal{.}
%\]
%These coordinates for points of $\pg{rt-1}{q}$ can be transformed to the usual coordinates consisting of $rt$ elements of $\FF_q$ by representing the elements of $\FF_{q^t}$ as the $t$ coordinates with respect to a fixed basis of $\FF_{q^t}$ over $\FF_q$.
%
%\textcolor{magenta}{Do we need to specify something about reguli?}
%
%\textcolor{red}{Hier komt nog Segre's interpretatie van een Desarguesiaanse spread, samen met de notie van indicator spaces.}
%
%\subsubsection{Linear sets}
%
%Linear sets are a central object in finite geometry and have been studied intensively, mainly due to the connection with other objects such as semifield planes, blocking sets and, more recently, MRD-codes (see e.g.\ \cite{Lavrauw,LavrauwVandeVoordeFieldRed,Polverino}).
%
%Let $V$ be an $r$-dimensional vector space over $\FF_{q^t}$ and let $\Omega$ be the corresponding projective space $\pg{r-1}{q^t}$.
%A set $\linset$ is said to be an \emph{$\FF_q$-linear set} of $\Omega$ of rank $k$ if it is defined by the non-zero vectors
%of an $\FF_q$-vector subspace $U$ of $V$ of vector dimension $k$, i.e.\
%\[
%    \linset = \sett{\vspanq{u}{q^t}}{u\in U\setminus\set{0}}\textnormal{.}
%\]
%
%One can also approach the notion of a linear set from a geometric point of view.
%An $\FF_q$-linear set of rank $k$ is a set $\linset$ of points of $\pg{r-1}{q^t}$ for which there exists a subspace $\pi$ of (projective) dimension $k-1$ in $\pg{rt-1}{q}$ such that the points of $\linset$ correspond to the elements of $\spread$ that have a non-empty intersection with $\pi$.
%If the subspace $\pi$ intersects each spread element in at most a point, then $\pi$ is called \emph{scattered} with respect to $\spread$ and the associated linear set is called a \emph{scattered} linear set.
%Note that $\pi$ is $(k-1)$-dimensional and scattered if and only if the associated $\FF_q$-linear set has rank $k$ and has exactly $\frac{q^k-1}{q-1}$ points.
%Furthermore, if the subspace $\pi$ intersects precisely one spread element $E\in\spread$ in a $(k-2)$-space and all other spread elements in at most a point, then $\linset$ is called a \emph{club}, and the point of $\linset$ corresponding to $E$ is called the \emph{head} of the club.
%Convince yourself that $\pi$ is $(k-1)$-dimensional and corresponds to a club if and only if the associated $\FF_q$-linear set has rank $k$ and has exactly $q^{k-1}+1$ points.
%
%\begin{res}[{\cite[Theorem 4.3]{BlokhuisLavrauw}}]
%    The rank of a scattered $\FF_q$-linear set in $\pg{r-1}{q^t}$ is at most $rt/2$.
%\end{res}
%
%A \emph{maximum scattered linear set} is a scattered $\FF_q$-linear set in $\pg{r-1}{q^t}$ with rank $rt/2$.
%
%\bigskip
%A subgeometry is a linear set, but the converse is generally untrue.
%However, the following result of Lunardon and Polverino shows that every linear set is equal to the projection of a subgeometry.
%This gives rise to a third perspective on linear sets as the quotient geometry of a subgeometry\footnote{\textcolor{magenta}{Moet deze 'canonical' zijn? Zie artikel \cite{LavrauwVandeVoordeLinSetLine}.}}.
%
%\begin{res}[{\cite[Theorems $1$ and $2$]{LunardonPolverino}}]\label{Res_LinSetsAreProjs}
%    Let $m\in\NN$, $m\geq r$.
%    Consider an $\FF_q$-subgeometry $\mSigma\cong\pg{m}{q}$ of $\mSigma^*\cong \pg{m}{q^t}$ and suppose there exists an $(m-r)$-dimensional subspace $\Omega^*$ of $\mSigma^*$ disjoint to $\mSigma$.
%    Let $\Omega\cong\pg{r-1}{q^t}$ be an $(r-1)$-dimensional subspace of $\mSigma^*$ disjoint to $\Omega^*$, and let $\Gamma$ be the projection of $\mSigma$ from $\Omega^*$ onto $\Omega$.
%    Let $p_{\Omega^*,\Omega}$ denote the map defined by $x\mapsto\vspan{\Omega^*,x}\cap\Omega$ for each point $x\in\mSigma^*\setminus\Omega^*$.
%    
%    If $\Gamma$ is a projection of $\pg{m}{q}$ into $\Omega\cong\pg{r-1}{q^t}$, then $\Gamma$ is an $\FF_q$-linear set of rank $m+1$ and $\vspanq{\Gamma}{q^t}=\Omega$.
%    Conversely, if $\linset$ is an $\FF_q$-linear set of $\Omega$ of rank $m+1$ and $\vspanq{\linset}{q^t}=\Omega\cong\pg{r-1}{q^t}$, then either $\linset$ is a canonical subgeometry of $\Omega$ or there is an $(m-r)$-dimensional subspace $\Omega^*$ of $\mSigma^*\cong\pg{m}{q^t}$ disjoint to $\Omega$ and a canonical subgeometry $\mSigma$ of $\mSigma^*$ disjoint to $\Omega^*$ such that $\linset = p_{\Omega^*,\Omega}(\mSigma)$.
%\end{res}
%
%\textcolor{red}{Bovenstaande resultaat kan eventueel vereenvoudigd in dit artikel gestoken worden? $m=r=2$?}

\subsection{Preliminaries}\label{sec1}
The topics introduced in the following subsections are interrelated; for more information, we refer to \cite{LavrauwVandeVoordeFieldRed}, \cite{RotteySheekeyVandeVoorde} and \cite{IndicatorSet}, respectively.

\subsubsection{Field reduction and Desarguesian spreads}
It is well-known that the vector space $\V{r}{q^t}$ is isomorphic to $\V{rt}{q}$; this isomorphism translates to a correspondence between the associated projective spaces $\pg{r-1}{q^t}$ and $\pg{rt-1}{q}$. Every point of $\pg{r-1}{q^t}$ corresponds to a $1$-dimensional vector space over $\FF_{q^t}$, which is a $t$-dimensional vector space over $\FF_q$, and hence, corresponds to a $(t-1)$-dimensional subspace of $\pg{rt-1}{q}$. In this way, the point set of $\pg{r-1}{q^t}$ gives rise to a set $\spread$ of $(t-1)$-dimensional subspaces of $\pg{rt-1}{q}$ partitioning the point set of $\pg{rt-1}{q}$, that is, they form a {\em $(t-1)$-spread} of $\pg{rt-1}{q}$. Any spread isomorphic to $\spread$ is called a \emph{Desarguesian $(t-1)$-spread}. Similarly, a $(k-1)$-dimensional subspace of $\pg{r-1}{q^t}$ corresponds to a $(kt-1)$-dimensional subspace of $\pg{rt-1}{q}$, spanned by elements of $\spread$.
More formally, we can define the field reduction map $\mathcal{F}_{q,r,t}$ which maps a $(k-1)$-dimensional subspace of $\pg{r-1}{q^t}$ to its associated $(kt-1)$-dimensional subspace of $\pg{rt-1}{q}$. We will omit the subscript of $\mathcal{F}_{q,r,t}$ if the field size and dimensions are clear. If $\mathcal{S}$ is a point set,  we use $\mathcal{F}(\mathcal{S})$ to denote the union of the images of the points in $\mathcal{S}$ under $\mathcal{F}$.

%For more information about field reduction, we refer to \cite{LavrauwVandeVoordeFieldRed}. 
   \subsubsection{The Andr\'e/Bruck-Bose representation}\label{abbintro}
Andr\'e \cite{Andre} and Bruck and Bose \cite{BruckBose} independently derived a representation of a projective plane of order $q^t$ in the projective space $\pg{2t}{q}$.
We refer to this correspondence as the \emph{André/Bruck-Bose representation} or the \emph{ABB-representation}.

Let $H_\infty$ be a hyperplane in $\pg{2t}{q}$ and let $\spread$ be a $(t-1)$-spread in $H_\infty$.
Let $\pointset$ be the set of {\em affine} points (i.e. those of $\pg{2t}{q}$, not contained in $H_\infty$), together with the $q^t+1$ spread elements of $\spread$.
Let $\linset$ be the set of $t$-spaces in $\pg{2t}{q}$ meeting $H_\infty$ in an element of $\spread$, together with the hyperplane at infinity $H_\infty$.
The incidence structure $(\pointset,\linset,I)$, with $I$ the natural incidence relation, is isomorphic to a projective plane of order $q^t$, which is called the \emph{Andr\'e/Bruck-Bose plane} corresponding to the spread $\spread$.
The Andr\'e/Bruck-Bose plane corresponding to a spread $\spread$ is Desarguesian if and only if the spread $\spread$ is Desarguesian.

Now consider $\pg{2}{q^t}$ and let $\ell_\infty$ be a designated line at infinity. Let $H_\infty=\mathcal{F}\left(\ell_\infty\right)$ be a $(2t-1)$-dimensional subspace of $\pg{3t-1}{q}=\mathcal{F}(\pg{2}{q^t})$. Fix a $2t$-space $\mu$ through $H_\infty$. 
It is not hard to see that the Andr\'e/Bruck-Bose representation of an affine point $P$ of $\pg{2}{q^t}$ in $\mu\cong\pg{2t}{q}$ is the point $\mathcal{F}(P)\cap \mu$. We let $\phi$ denote the Andr\'e/Bruck-Bose map on affine points:
$$\phi(P):=\mathcal{F}(P)\cap \mu.$$ The ABB-representation of a point $Q\in\ell_\infty$ is the $(t-1)$-space $\mathcal{F}(Q)$.



\subsubsection{Indicator spaces and Desarguesian subspreads}\label{subsubind}
Finally, we recall the construction of a spread as introduced by Segre \cite{Segre64}. Embed $\Lambda \simeq \pg{rt-1}{q}$ as a subgeometry of $\Lambda^*\simeq \pg{rt-1}{q^t}$. The subgroup of $\mathrm{P}\Gamma\mathrm{L}(rt,q^t)$ fixing $\Lambda$ pointwise is isomorphic to $\mathrm{Aut}(\FF_{q^t}/\FF_q)$. Consider a generator $g$ of this group. One can prove that that there exists an $(r - 1)$-space $\nu$ skew to the subgeometry $\Lambda$ and that a subspace of $\pg{rt-1}{q^t}$ of dimension $s$ is fixed by $g$  if and only if it intersects the subgeometry $\Lambda$ in a subspace of dimension $s$  (see \cite{IndicatorSet}). Let $P$ be a point of $\nu$ and let $L(P)$ denote the $(t-1)$-dimensional subspace generated by the {\em conjugates} of $P$, i.e., $L(P) = \langle P,P^g,\ldots,P^{g^{t-1}}\rangle$. Then $L(P)$ is fixed by $g$ and hence it intersects $\pg{rt-1}{q}$ in a $(t-1)$-dimensional subspace. Repeating this for every point of $\nu$, one obtains a set $\spread$ of $(t-1)$-spaces of the subgeometry $\Gamma$ forming a spread. This spread $\spread$ can be shown to be a Desarguesian spread and $\{\nu,\nu^{g},\ldots,\nu^{g^{t-1}}\}$ is called the {\it indicator set} of $\spread$. An indicator set is also called a set of {\em director spaces} \cite{Segre64}.
It is known from \cite[Theorem 6.1]{IndicatorSet} that for any Desarguesian $(t-1)$-spread of $\pg{rt-1}{q}$ there exist a unique indicator set in $\pg{rt-1}{q^t}$. 

In this paper, {\color{black} we will make use of a particular coordinate system describing a subgeometry $\pi \simeq \pg{t-1}{q}$ in $\pg{t-1}{q^t}$, and for each $s|t$, we will define an $(s-1)$-spread denoted by $\spread_s$ of $\pi$. In the case that $s=t$, this `spread' of $\pi$ is the subspace $\pi$ itself. To describe the set-up,}


%we will frequently make a particular choice for $\Lambda$ and $g$:

let $\sigma$ denote the collineation of $\pg{t-1}{q^t}$ which maps a point with homogeneous coordinates $(x_0,x_1,x_2,\ldots,x_{t-1})$, $x_i\in \FF_{q^t}$, not all zero, onto the point with homogeneous coordinates $(x_{t-1}^q,x_0^q,x_1^q,\ldots,\ldots,x_{t-2}^q)$. The fixed points of $\sigma$ then form a 
subgeometry $\pi\simeq \pg{t-1}{q}$, consisting of all points with homogeneous coordinates $(x,x^q,x^{q^2},\ldots,x^{q^{t-1}})$ for $x\in \FF_{q^t}$. Let $R$ denote the point with coordinates $(1,0,\ldots,0)$, then we see that $R^\sigma=(0,1,\ldots,0)$, $R^{\sigma^2}=(0,0,1,\ldots,0)$ $\ldots$, $R^{\sigma^{t-1}}=(0,0,\ldots,1)$.
Given $R$, every positive divisor $s$ of $t$ induces a unique Desarguesian $(s-1)$-spread $\spread_s$ of $\pi$: consider $\Lambda_s=\mathrm{Fix}(\sigma^s)\simeq \pg{t-1}{q^s}$ and let $\Pi=\langle R,R^{\sigma^s},R^{\sigma^{2s}},\ldots,R^{\sigma^{t-s}}\rangle \cap \Lambda_s$. Then $\{\Pi,\Pi^\sigma,\ldots,\Pi^{\sigma^{s-1}}\}$ is a set of director spaces for $\spread_s$ in $\pg{t-1}{q}$. 

We denote the extension of an element $D$ of $\spread_s$ to $\pg{t-1}{q^t}$ by $\overline{D}$. 

For ease of notation in the case $s=t$, we define the `spread' $\spread_t$ to be equal to $\pi$ and the indicator set of $\pi$ to be the point set $\{R,R^\sigma,\ldots,R^{\sigma^{t-1}}\}$.


\begin{df} Let $$P_x:=\left(\frac{1}{x},\frac{1}{x^q},\frac{1}{x^{q^2}},\ldots,\frac{1}{x^{q^{t-1}}}\right)$$ denote the point of $\pi\simeq \pg{t-1}{q}$ corresponding to $\frac{1}{x}\in \FF_{q^t}^\ast$. 
\end{df}

Note that $P_x=P_y$ if and only if $x/y\in \FF_q$. Furthermore, it is easy to see that $P_x$ is contained in the element $D$ of $\spread_s$ spanned by the points $X,X^\sigma,\ldots,X^{\sigma^{s-1}}$ where $X$ is stabilised by $\sigma^s$ and given by $X=\left(\frac{1}{x},0,\ldots,\frac{1}{x^{q^s}},0,\ldots,\frac{1}{x^{q^{2s}}},0,\ldots,\frac{1}{x^{q^{t-s}}},0,\ldots,0\right)$. Geometrically, the point $X$ is the intersection point of $\overline{D}$ with $\Pi$, where the latter is the director space defining the spread $\spread_s$. It now easily follows that two different points $P_x$ and $P_y$ lie in the same element of $\spread_s$ if and only if $x/y\in \FF_{q^s}$. 






%The idea behind \emph{field reduction} is interpreting a projective geometry $\pg{r-1}{q^t}$ as its underlying vector space $\V{r}{q^t}$, which is known to be isomorphic to $\V{rt}{q}$, which in turn naturally translates to $\pg{rt-1}{q}$. is well





%We adopt the way of giving coordinates to the points of $\pg{2}{q^t}$ and $\pg{2t}{q}$ from \cite{RotteySheekeyVandeVoorde}.
%In this way, one obtains the explicit map $\varphi:\pg{2}{q^t}\rightarrow\pg{2t}{q}$ mapping each point to
%its corresponding element in the ABB-representation as follows.
%A point $P\in\pg{2}{q^t}$ defined by the vector $(a,b,c)\in\FF_{q^t}^3$ is denoted by $(a,b,c)_{q^t}$.
%W.l.o.g.\ we suppose that the line $\ell_\infty$ at infinity in $\pg{2}{q^t}$ is defined by
%\begin{align*}
%    \ell_\infty = \sett{(a,b,0)_{q^t}}{a,b\in\FF_{q^t},(a,b)\neq(0,0)}\textnormal{.}
%\end{align*}
%The points in $\pg{2}{q^t}\setminus \ell_\infty$ are the affine points, which can be written as $(a,b,1)_{q^t}$, $a,b\in\FF_{q^t}$.
%On the other hand, each point of $\pg{2t}{q}$ can be described by $(a,b,c)_q$, where $a,b\in\FF_{q^t}$ and $c\in\FF_q$.
%We may assume that the hyperplane $H_\infty$ of $\pg{2t}{q}$ is defined by
%\begin{align*}
%    H_\infty = \sett{(a,b,0)_q}{a,b\in\FF_{q^t},(a,b)\neq(0,0)}\textnormal{.}
%\end{align*}
%Note that $H_\infty$ contains the Desarguesian $(t-1)$-spread $\spread$ defined by
%\begin{align*}
%    \spread=\sett{\sett{(ax,bx,0)_q}{x\in\FF_{q^t}^*}}{a,b\in\FF_{q^t},(a,b)\neq(0,0)}\textnormal{.}
%\end{align*}
%From this, one can conclude that the following map $\varphi$, for $a,b\in\FF_{q^t}$, $(a,b)\neq(0,0)$, corresponds to the ABB-representation:
%\[
%    \varphi:\pg{2}{q^t}\rightarrow\pg{2t}{q}:\begin{cases}(a,b,0)_{q^t}&\mapsto\sett{(ax,bx,0)_q}{x\in\FF_{q^t}^*}\textnormal{,}\\
%    (a,b,1)_{q^t}&\mapsto(a,b,1)_q\textnormal{.}\end{cases}
%\]
%
%The map $\varphi$ will from this point onward we referred to as the \emph{ABB-map}.
%
%\begin{lm}\label{Lm_Projection}
%Let $\varphi$ be the ABB-map.
%Interpret a line $\ell\neq \ell_\infty$ of $\pg{2}{q^t}$ as the projective line $\pg{1}{q^t}$ and consider the points $P_\infty:=\ell\cap \ell_\infty$ and $P\in \ell_\infty\setminus\{P_\infty\}$.
%Then a point set $\linset\subseteq\ell\setminus\set{P_\infty}$ is the projection of an affine point set $A$ of $\pg{2}{q^t}$ from $P$ onto $\ell$ if and only if its corresponding point set $\varphi(\linset)$ is the projection of the affine point set $\varphi(A)$ of $\pg{2t}{q}$ from the $(t-1)$-space $\varphi(P)$ onto the $t$-space $\varphi(\ell)$.
%\end{lm}
%\begin{proof}
%   Let $Q$ be a point of $\linset$.
%   Then there exists a point $R\in A$ such that $Q=\ell\cap\vspan{P,R}$; in particular, the points $P$, $Q$ and $R$ lie on a line $r$.
%   As the ABB-map preserves incidence, the $(t-1)$-space $\varphi(P)$ and the points $\varphi(Q)$ and $\varphi(R)$ lie in the $t$-space $\varphi(r)$.
%   Moreover, the point $\varphi(Q)$ is the unique point within the intersection $\varphi(\ell)\cap\varphi(r)$.
%   Hence, it is clear that $\varphi(Q)$ is precisely the projection of $\varphi(R)$ from $\varphi(P)$ onto $\varphi(\ell)$.
%   The proof of the converse is analogous by making use of $\varphi^{-1}$.
%\end{proof}

\subsubsection{Arcs and normal rational curves}
For any $m\in\NN$ and $k\geq1$, an \emph{$m$-arc} of $\pg{k}{q}$ is a set of $m$ points \emph{in general position}, i.e.\ every $k+1$ points of this point set span $\pg{k}{q}$.

\begin{df}
    Let $1\leq k\leq q$.
    A \emph{normal rational curve} in $\pg{k}{q}$ is a $(q+1)$-arc projectively equivalent to the $(q+1)$-arc corresponding to the coordinates
    \[
        \set{(0,0,\dots,0,1)}\cup\sett{(1,t,t^2,t^3,\dots,t^k)}{t\in\FF_q}\textnormal{.}
    \]
    A point set $\mathcal{C}$ of $\pg{n}{q}$ is a normal rational curve \emph{of degree $k$} if and only if it is a normal rational curve in a $k$-dimensional subspace of $\pg{n}{q}$.
    Note that a normal rational curve of degree $1$ is a line, while one of degree $2$ is a non-degenerate conic.
    %A normal rational curve of degree $3$ is called a \emph{twisted cubic}.}
\end{df}

\begin{res} [{\cite[Theorem 1.18]{Harris}}] \label{uniqueNRC} Consider a $(k+2)$-arc $\mathcal{A}$ in $\pg{k-1}{q}$, $k+1\leq q$, then there exists a unique normal rational curve of degree $k-1$ through all points of $\mathcal{A}$.
\end{res}

\begin{res} [{\cite[Lemma 27.5.2(i)]{Hirschfeldgalois}}] \label{projNRC} Let $\mathcal{C}$ be a normal rational curve \emph{of degree $k-1$} in $\pg{k-1}{q}$, and let $P\in \mathcal{C}$. The projection of $\mathcal{C}\setminus\{P\}$ from $P$ onto a $(k-2)$-space disjoint from $P$ is a point set of size $q$ contained in a normal rational curve of degree $k-2$. If $k+1\leq q$, then this normal rational curve is unique.
\end{res}

\subsubsection{The ABB-representation of sublines and subplanes}

The ABB-represention of $\FF_{q^k}$-sublines and tangent subplanes of $\pg{2}{q^t}$ was studied in \cite{RotteySheekeyVandeVoorde}. 


In this paper, we will make use of the following cases tackled there:

\begin{res}[\cite{RotteySheekeyVandeVoorde}]\label{Res_SublinesTangent}
    \begin{itemize}
        \item[(a)] The affine points of an $\FF_{q}$-subline in $\pg{2}{q^t}$ tangent to $\ell_\infty$ correspond to the points of an affine line in the ABB-representation and vice versa.
        \item[(b)] Suppose that $q\geq t$ and $k\mid t$. Let $m$ be an $\FF_q$-subline of $\pg{2}{q^t}$ external to $\ell_\infty$ where the smallest subline containing $m$ and tangent to $\ell_\infty$ is an $\FF_{q^k}$-subline. Then the ABB-representation of $m$ is a set of points $\mathcal{C}$ in $\pg{2t}{q}$ such that     \begin{enumerate}
        \item $\mathcal{C}$ is a normal rational curve of degree $k$ contained in a $k$-space intersecting $H_\infty$ in an element of $\spread_k$.
        \item its $\FF_{q^t}$-extension $\mathcal{C}^*$ to $\pg{2t}{q^t}$ intersects the indicator set $\set{\Pi,\Pi^\sigma,\dots,\Pi^{\sigma^{k-1}}}$ of $\spread_k$ in $k$ conjugate points.
    \end{enumerate} and vice versa, any set $\mathcal{C}$ with those properties gives rise to the point set of an $\FF_q$-subline, external to $\ell_\infty$.
        
    \end{itemize}
\end{res}


%\begin{res}[{\cite[Theorem $3.6$]{RotteySheekeyVandeVoorde}}]\label{Res_SublinesExternal}
%    Suppose that $q\geq t$ and $k\mid t$.
%    A set of points $\mathcal{C}$ in $\pg{2t}{q}$ is the ABB-representation of an $\FF_q$-subline $s$ of $\pg{2}{q^t}$ external to $\ell_\infty$ if and only if
%    \begin{enumerate}
%        \item $\mathcal{C}$ is a normal rational curve of degree $k$ contained in a $k$-space intersecting $H_\infty$ in an element of $\spread_k$.
%        \item its $\FF_{q^t}$-extension $\mathcal{C}^*$ to $\pg{2t}{q^t}$ intersects the indicator set $\set{\Pi,\Pi^\sigma,\dots,\Pi^{\sigma^{k-1}}}$ of $\spread_k$ in $k$ conjugate points.
%    \end{enumerate}
%    Moreover, the smallest subline containing $s$ and tangent to $\ell_\infty$ is an $\FF_{q^k}$-subline.
%\end{res}

%\textcolor{red}{TBD: afwachten of het nodig is om de algemenere versie van bovenstaande te gebruiken.}
%
%\begin{res}[{\cite[Theorem $4.1$]{RotteySheekeyVandeVoorde}}]\label{Res_SubplanesSecant}
%    Suppose that $q>2$\textcolor{red}{[$q>2$ wordt niet vermeldt in het oorspronkelijk resultaat maar lijkt me noodzakelijk wegens hun verwijzing naar Theorem $3.10$.]} and $k\mid t$.
%    A set $\Pi$ of affine points in $\pg{2t}{q}$ is the ABB-representation of the affine points of an $\FF_{q^k}$-subplane in $\pg{2}{q^t}$ secant to $\ell_\infty$ if and only if
%    \begin{enumerate}
%        \item $\Pi$ is a $2k$-dimensional affine space,
%        \item its projective completion $\bar{\Pi}$ intersects $H_\infty$ in a $(2k-1)$-space which intersects $q^k+1$ elements of $\spread_t$ in exactly a $(k-1)$-space.
%    \end{enumerate}
%    Moreover, this $(2k-1)$-space intersects each of the $q^k+1$ spread elements of $\spread_t$ in a $(k-1)$-space of $\spread_k$.
%\end{res}
%
%\textcolor{red}{Hier komt nog de definitie van een Normal Rational Scroll.}
%
%\begin{res}[{\cite[Theorem $4.5$]{RotteySheekeyVandeVoorde}}]\label{Res_SubplanesTangent}
%    Suppose that $q\geq t$ and let $i\in\NN\setminus\set{0}$, $i\mid k$ and $k\mid t$.
%    A set of affine points $\mathcal{S}$ in $\pg{2t}{q}$ is the ABB-representation of an $\FF_{q^i}$-subplane $\mu$ of $\pg{2}{q^t}$ tangent to $\ell_\infty$ if and only if $\mathcal{S}$ consists of the affine points of a normal rational scroll defined by curves $\mathcal{C}$, $\mathcal{N}$ such that
%    \begin{enumerate}
%        \item $\mathcal{C}$ is a normal rational curve of degree $k$ contained in an affine $k$-space $\pi$, for which $\pi\cap H_\infty$ is an element $E_1$ of $\spread_k$, such that its $\FF_{q^t}$-extension $\mathcal{C}^*$ contains all conjugate points $\set{P,P^\sigma,\dots,P^{\sigma^{k-1}}}$ generating the spread element $E_1$,
%        \item $\mathcal{N}$ is a normal rational curve of degree $k-1$ contained in an element $E_2$ of $\spread_k$, where $E_1$ and $E_2$ are not contained in the same element of $\spread_t$, such that its $\FF_{q^t}$-extension $\mathcal{N}^*$ contains all conjugate points $\set{Q,Q^\sigma,\dots,Q^{\sigma^{k-1}}}$ generating the spread element $E_2$, and
%        \item the $\FF_{q^t}$-extension of the normal rational scroll contains the lines $\vspan{P^{\sigma^j},Q^{\sigma^j}}$, each line contained in an indicator space $\Pi^{\sigma^j}$ of $\spread_k$, for all $j\in\set{0,1,\dots,k-1}$.
%    \end{enumerate}
%    Moreover, in that case the smallest subplane containing $\mu$ and secant to $\ell_\infty$ is an $\FF_{q^k}$-subplane.
%\end{res}
%
%\textcolor{red}{Misschien kan ook dit resultaat vereenvoudigd worden `to fit our needs'.}


\subsubsection{Linear sets}
For a more thorough introduction to linear sets, we refer to \cite{LavrauwVandeVoordeFieldRed,olga}.
In this paper, we will only be concerned with linear sets on a projective line, and we will use the geometrical point of view on linear sets using Desarguesian spreads. Let $\spread$ be the Desarguesian spread in $\pg{2t-1}{q}$ obtained as the image of the field reduction map on points of $\pg{1}{q^t}$. Then a set $\mathcal{S}$ in $\pg{1}{q^t}$ is an $\FF_q$-linear set of rank $k$ if and only if there is a $(k-1)$-dimensional subspace $\pi$ of $\pg{2t-1}{q}$ such that $$\mathcal{F}(\mathcal{S})=\mathcal{B}(\pi),$$ where $\mathcal{B}(\pi)$ is the set of elements of $\spread$ meeting $\pi$ in at least a point.

\begin{df} We denote the $\FF_q$-linear set $\mathcal{S}$ such that $\mathcal{F}(\mathcal{S})=\mathcal{B}(\pi)$ by $L_{\pi}$.
\end{df}

The {\em weight} of a point $P$ in $L_\pi$ is $w+1$ if $w$ is the dimension of $\mathcal{F}(P)\cap\pi$. Note that the weight of a point in a linear set is only well-defined if we specify the subspace $\pi$ defining $L_\pi$. 


In this article, we focus on {\em scattered $\FF_q$-linear sets} in $\pg{1}{q^3}$ and {\em clubs} in $\pg{1}{q^t}$. A scattered linear set of rank $k$ in $\pg{1}{q^t}$ is an $\FF_q$-linear set of rank $k$ consisting of $\frac{q^k-1}{q-1}$ points. We see that all the points of a scattered linear set have weight one. If $L_\pi$ is a scattered linear set, then the subspace $\pi$ is called {\em scattered} (with respect to the Desarguesian spread $\spread$).
A {\em $t$-club} of rank $k$ is an $\FF_q$-linear set $L_\pi$ such that there is one point of weight $t$ and all other points have weight one; if $t=k-1$, this set is simply called a {\em club}. The point of weight $t$ is called the {\em head} of the club. As for the weight of the points in the linear set, we see that the head of the club is only well-defined with respect to the subspace $\pi$.


%By the above result, we can embed the projective line into $\pg{2}{q^t}$ and view linear sets of rank $3$ as projections of an $\FF_q$-subplane.
%
%\bigskip
%Lavrauw and Van de Voorde investigated equivalence and intersection properties of clubs and scattered linear sets on a projective line; we summarise some of their results which will prove to be useful in our proofs.
%
%The following result was already presented in \cite{FancsaliSziklai} but, as the authors of \cite{LavrauwVandeVoordeLinSetLine} point out, the proof was incomplete as they assumed the projective equivalence of clubs, which wasn't proven up to that point and is generally not true for clubs of $\pg{1}{q^t}$, $t\geq4$ \cite[Theorem $5$]{LavrauwVandeVoordeLinSetLine}.
%

We have the following result about the possible intersection of an $\FF_q$-linear set and an $\FF_q$-subline.
\begin{res}[{\cite[Theorem $8$]{LavrauwVandeVoordeLinSetLine}}]\label{Res_LinearSetIntersectionSubline}
    An $\FF_q$-subline intersects an $\FF_q$-linear set of rank $k$ of $\pg{1}{q^t}$ in at most $k$ or precisely $q+1$ points.
\end{res}

The following results on clubs and scattered linear sets on a projective line reveal some useful geometric properties. Note that the authors of \cite{LavrauwVandeVoordeLinSetLine} did not include the necessary condition that $q\geq 3$.

\begin{res}[{\cite[Corollary $13$ and $15$]{LavrauwVandeVoordeLinSetLine},\cite[Theorem 3.7.4]{mijnthesis}}]\label{Res_LinearSetClub}
    Suppose that $q\geq 3$.   \begin{itemize}\item[(a)] If $\mS$ is a club of $\pg{1}{q^t}$, $\mS\not\simeq\pg{1}{q^2}$, then through two distinct non-head points of $\mS$, there exists exactly one $\FF_q$-subline contained in $\mS$, which necessarily contains the head of the club.
    \item[(b)]  If $\mS$ is a scattered linear set of rank $3$ of $\pg{1}{q^3}$, then through two distinct points of $\mS$, there are exactly two $\FF_q$-sublines contained in $S$.
 \item[(c)] Let $q\geq 5$. Consider a scattered plane $\pi$ with respect to the Desarguesian $2$-spread $\spread$ in $\pg{5}{q}$ and let $r\in \pi$. Then there is exactly one plane $\pi'\neq \pi$ through $r$ such that $\mathcal{B}(\pi)=\mathcal{B}(\pi')$.
    \end{itemize}
\end{res}

%\begin{res}[{\cite[Corollary $15$]{LavrauwVandeVoordeLinSetLine}}]\label{Res_LinearSetScattered}
%    Suppose that $q>2$.
%    If $\mS$ is a scattered linear set of rank $3$ of $\pg{1}{q^t}$, then through two distinct points of $\mS$, there exist at most two $\FF_q$-sublines contained in $\mS$.
%    If $t=3$, then through two distinct points of $\mS$, there are exactly two $\FF_q$-sublines contained in $S$.
%\end{res}


\section{Generalising the circumscribed bundle of conics}\label{sec2}

In order to characterise the ABB-representation of clubs, tangent to $\ell_\infty$, we will introduce a block design $\mathcal{H}$ embedded in $\pg{t-1}{q}$, where blocks are certain normal rational curves. In the particular case when $t=3$, this design is known as the design arising from a {\em circumscribed} bundle of conics. In \cite{BakerBrownEbertFisher}, the authors describe three types of {\em projective bundles}, which they define to be a collection of $q^2+q+1$ conics mutually intersecting in exactly one point. The circumscribed bundles are {\em bundles} in the classical algebraic sense: given three conics in the bundle defined by equations $f=0$, $g=0$, $h=0$ where $h$ is not an $\FF_q$-linear combination of $f$ and $g$, every conic in the bundle is defined by $\lambda f+\mu g+\nu h=0$ for some $\lambda,\mu,\nu \in \FF_q$.

We see that the design $(\mathcal{P},\mathcal{B})$ where points $\mathcal{P}$ are the points of $\pg{2}{q}$, blocks $\mathcal{B}$ are the conics of the projective bundle, and incidence is inherited, forms a projective plane. The {\em circumscribed} bundle consists of all conics in $\pg{2}{q}$ whose extension to $\pg{2}{q^3}$ contains three fixed conjugate points $R,R^q,R^{q^2}$ spanning $\pg{2}{q^3}$. It can be deduced from \cite{LavrauwVandeVoordeLinSetLine} that the projective plane constructed via the circumscribed bundle is the Desarguesian plane $\pg{2}{q}$. The design here will be a natural generalisation of this construction; for $t$ prime, its definition is straightforward but for $t$ non-prime, extra care must be taken. 





Let $e_0,e_1,\ldots,e_{t-1}$ be the standard basis vectors of length $t$ (with $1$ in the $(i+1)$-th position and zero elsewhere) and let $\langle v\rangle$ denote the projective point of $\pg{t-1}{q^t}$ with homogeneous coordinates given by $v$.

%\textcolor{orange}{Ik ben wat in de war over de `wereld' die beschouwd wordt in onderstaande lemma -- in mijn hoofd is onderstaande situatie een `deel' van een grote Desarguesiaanse spread (namelijk, wat later zal blijken te zijn, heel ons ABB-hypervlak op oneindig). De notatie $\spread$ zoals hieronder gebruikt slaat dan op een `grote' $(t-1)$-spread waarvan $\pg{t-1}{q^t}$ de extensie is van slechts \'e\'en spreadelement.}

%\textcolor{orange}{Echter kijken wij in onderstaande lemma enkel naar (de extensie van) dat ene spreadelement, vandaar dat het voor mij wat raar is om te spreken van de `grote' $(t-1)$-spread $\spread$.}

%\textcolor{orange}{Kortom: ik vermoed de beschrijving van het lemma duidelijker wordt als we inderdaad starten van de notaties van \ref{subsubind}, vervolgens \'e\'en spreadelement van de grote spread belichten, waarvan we de extensie identificeren als $\pg{t-1}{q^t}$ en we zijn vertrokken. Akkoord?}
%{\color{magenta} Ik denk dat ik niet helemaal snap wat je bedoelt (is het dat je nu met 1 spreadelement ipv een spread werkt?), maar als je akkoord bent met de notaties van 1.2.3, pas dan gerust hieronder aan wat je duidelijker vindt!}
%\textcolor{cyan}{ok voor mij}
%\textcolor{orange}{Ik probeerde 't aan te passen, maar grr, 't toch niet helemaal duidelijk voor mij, maar misschien ligt mijn probleem niet hier, maar net bij \ref{subsubind}. Vooral vanaf het punt `In this paper, we will frequently make a particular choice for $\Lambda$ and $g$'. Plots heeft men het daar over $\pg{t-1}{q^t}$ in plaats van $\pg{rt-1}{q^t}$. Betekent dit dat men $r=1$ koos? Maar men blijft verder praten over `de spread $\spread$', wat in deze context gewoon bestaat uit één element, namelijk $\pi$? Ik vind dit echt verwarrend. In onderstaande lemma wordt $\spread$ vermeldt, maar hoe ziet die er dan uit, mits we duidelijk enkel spelen in de `wereld' $\pg{t-1}{q^t}$?}

\begin{lm} \label{NRCs}  (Using the notations introduced in \ref{subsubind}) Consider the points $R^{\sigma^i}=\langle e_i\rangle$, $i=0,\ldots,t-1$, in $\pg{t-1}{q^t}$ and two points $P_a\neq P_b$ in $\pi \simeq \pg{t-1}{q}$. Let $s$ be the smallest integer such that $a/b\in \FF_{q^s}$ and let $D$ be the element of the Desarguesian $(s-1)$-spread  $\spread_{s}$ containing $P_a$ and $P_b$. Then 
\begin{enumerate}
\item there is a unique normal rational curve $\mathcal{C}^{a,b}$ of degree $s-1$ through $P_a$ and $P_b$, contained in $\overline{D}$, and meeting the indicator spaces $\{\Pi,\Pi^\sigma,\ldots,\Pi^{\sigma^{s-1}}\}$ in $s$ conjugate points. 
\item the points of $\mathcal{C}^{a,b}$ are given by $\{K^{a,b}_{u,v}|u,v\in \FF_{q^t}\}$ where 
$$K^{a,b}_{u,v}:=\left\langle \sum_{i=0}^{s-1}\prod_{j=0,j\neq i}^{s-1} (a^{q^j}u-b^{q^j}v)w_i\right\rangle;$$
and the conjugate points are $Q,Q^\sigma,\ldots,Q^{\sigma^{s-1}}$ where $Q^{\sigma^{i-1}}=\langle w_i \rangle$ with

%\textcolor{orange}{Waarom de factor $a^{q^i}$ v\'o\'or elke onderstaande vector? Om het eerste niet-nul element per se $1$ te krijgen?}{\color{magenta} Omdat  volgens mij de exacte uitdrukking zoals hierboven beschreven anders niet klopt:) Je wil dat die $K^{a,b}_{u,v}$ door precies de juiste vectoren beschreven wordt, niet op veelvoud na, om door de juiste punten te gaan, en $P_{au-bv}$ te geven als doorsnede. Maar reken dat dus nog eens na, voor pakweg $t=6$, $s=3$ of zo. Voor $s=t$ werkte de eerste coefficient gelijk aan $1$ nemen inderdaad prima, en dit is de uitbreiding denk ik.}
%\textcolor{cyan}{akkoord}
\begin{align} w_0&=a(\frac{1}{a},0,\ldots,0,\frac{1}{a^{q^s}},0,\ldots,0,\frac{1}{a^{q^{2s}}},\ldots,\frac{1}{a^{q^{t-s}}},0,\ldots,0)\nonumber\\
w_1&=a^q(0,\frac{1}{a^q},\ldots,0,\frac{1}{a^{q^{s+1}}},0,\ldots,0,\frac{1}{a^{q^{2s+1}}},\ldots,\frac{1}{a^{q^{t-s+1}}},0,\ldots,0)\nonumber\\
&\;\;\vdots\nonumber\\
 w_{s-1}&=a^{q^{s-1}}(0,\ldots,\frac{1}{a^{q^{s-1}}},0,\ldots,0,\frac{1}{a^{q^{t-1}}}).\label{conj}\end{align}

\item $\mathcal{C}^{a,b}$ meets $\pi$ in $q+1$ points, determined by the points $P_{au-bv}$ where $u,v\in \FF_q$.
\end{enumerate}
\end{lm}



\begin{proof} Recall that, given $D$, the set of $s$ conjugate points contained in both the indicator spaces and in $\overline{D}$ is fixed. As discussed in Section \ref{subsubind}, it is easy to check that the coordinates corresponding to this set $\{Q,Q^\sigma,\ldots,Q^{\sigma^{s-1}}\}$ of conjugate points is given by the vectors in \eqref{conj}. By Result \ref{uniqueNRC}, we know that there is a unique normal rational curve of degree $s-1$ containing the $s$ conjugate points and the points $P_a$ and $P_b$.


It is well-known (see e.g.\ \cite[Example 1.17]{Harris}) that $\mathcal{C}^{a,b}$ as given in the statement of the lemma defines a normal rational curve; the degree of this curve is $d$ if the point set $\{(a^{q^i},b^{q^i})|i=0,\ldots,t-1\}$ in $\pg{1}{q^t}$ consists of $d+1$ different points. Recall that $s$ is the smallest integer such that $a/b\in \FF_{q^s}$, and hence, $s$ is the smallest integer for which $(\frac{a}{b})^{q^s}=\frac{a}{b}$. This means that the point set $\{(a^{q^i},b^{q^i})|i=0,\ldots,t-1\}$ consists of $s$ different points, implying that the degree of $\mathcal{C}^{a,b}$ is indeed $s-1$.


Now consider the point $K^{a,b}_{0,1}=\langle (-1)^{s-1}\sum_{i=0}^{s-1}(\prod_{j=0,j\neq  i}^{s-1}b^{q^j})w_i\rangle$. By dividing by $(-1)^{s-1}\prod_{j=0}^{s-1}b^{q^j}$, we find that this point has coordinates $(\frac{1}{b},\frac{1}{b^q},\ldots,\frac{1}{b^{q^{t-1}}})$, and hence, is the point $P_b$. Similarly, $K^{a,b}_{1,0}$ is the point $P_a$, and we see that $\mathcal{C}^{a,b}$ indeed passes through $P_a$ and $P_b$.




Note that $K_{b^{q^{i'}},a^{q^{i'}}}^{a,b}=\langle w_{i'}\rangle$, $i'=0,1,\dots,s-1$. In other words, $\mathcal{C}^{a,b}$ indeed contains the $s$ conjugate points $Q,Q^\sigma,\ldots,Q^{\sigma^{s-1}}$.



{\color{black} 
Finally, if $u,v\in \FF_q$, and using that $b/a\in \FF_{q^s}$, it can be checked that $P_{au-bv}=K_{u,v}^{a,b}$, and vice versa, if a point $K_{u,v}^{a,b}$ lies in $\pi$, then it follows that $u,v\in \FF_q$. This means that the $q+1$ different points of the form $P_{au-bv}$, where $u,v\in \FF_q$, are precisely those in $\mathcal{C}^{a,b}\cap\pi$; the normal rational curve $\mathcal{C}^{a,b}$ meets $\pi$ in a normal rational curve of $\pi$.}




%{\color{magenta} Moet er hier een bewijs van zijn? Het is ambetant om op te schrijven. Tenzij voor $s=t$. Dan wordt het gewoon
 % \begin{align*}\left(\frac{1}{au-bv},\frac{1}{a^qu-b^qv},\ldots,\frac{1}{a^{q^{t-1}}u-b^{q^{t-1}}v}\right)&=\\
%\left(\frac{1}{au-bv},\left(\frac{1}{au-bv}\right)^q,\ldots,\left(\frac{1}{au-bv}\right)^{q^{t-1}}\right)&=P_{au-bv},\end{align*}} 

%\textcolor{orange}{Goeie vraag. We kunnen de wensen van de referee(s) afwachten... en desnoods in een appendix zetten mochten de uitdrukkingen te bombastisch worden.\\}\textcolor{cyan}{akkoord}{\color{magenta} Akkoord, ik laat het hier gewoon nog even staan voor Jozefien}


%{\color{brown}Vice versa, if a point $K_{u,v}^{a,b}$ lies in $\pi$, then it follows that $u,v\in \FF_q$} \textcolor{cyan}{Waarschijnlijk domme vraag, maar hoe kunnen we dit snel zien? Of volgt dit ook door te rekenen?}{\color{magenta} Reken je rot gaat ook, maar een snellere intuitie is dat je NRC hebt over $\FF_{q^t}$ waarvan we al weten dat als $u,v$ in het deelveld liggen, het punt erdoor bepaald in de deelmeetkunde ligt. Dus we hebben dat de grote NRC de deelmeetkunde minstens in $q+1$ punten snijdt. Die vormen ook een NRC. Maar de grote arc kan ook niet in meer snijden natuurlijk: een arc snijdt een deelmeetkunde in een arc van die deelmeetkunde. En een NRC kan niet meer uitgebreid worden naar een grotere arc.} \textcolor{orange}{Ik vindt dit deel in het roze een veel duidelijkere redenering in plaats van het zwarte deeltje `Vice versa...'. Ik zou het houden op de onderstaande zwarte zin, en eraan toevoegen dat je inderdaad niet meer dan $q+1$ punten kunt hebben aangezien je die niet kunt uitbreiden?}. {\color{brown}Since there are $q+1$ different points of the form $P_{au-bv}$, where $u,v\in \FF_q$, we find that there are precisely $q+1$ points in $\mathcal{C}^{a,b}\cap\pi$.}
\end{proof}

\begin{rmk} The fact that $P_{au-bv}$ defines a normal rational curve in the subgeometry $\pi$ as seen in Lemma \ref{NRCs} also follows by considering the cyclic model of $\pg{t-1}{q}$ (see e.g.\ \cite{cyclicmodel}): it is well-known that the inverse of a line in this model is a normal rational curve. In Lemma \ref{NRCs}, we have described the extension of this normal rational curve to $\pg{t-1}{q^t}$.
\end{rmk}

\begin{df} Consider a subgeometry $\pi$ $\simeq\pg{t-1}{q}$ arising as the set of fixed points of a collineation $\sigma$ of $\pg{t-1}{q^t}$, and let $R$ be a point such that the points $R,R^\sigma,R^{\sigma^2},\ldots,R^{\sigma^{t-1}}$ span $\pg{t-1}{q^t}$. Consider the Desarguesian subspreads $\spread_s$ for every $1<s\leq t$, $s|t$, as defined in Subsection \ref{subsubind}. Let $\mathcal{H}$ denote the following incidence structure:
\begin{itemize}
\item Points $\mathcal{P}$ are the points of $\pi$;
\item Let $P$ and $Q$ be two distinct points of $\pi$, and $s$ be the smallest integer such that $P,Q$ are contained in the same element of $\spread_s$, say $D$. Then the unique block through $P$ and $Q$ is the set of points of $\pi$ contained in the normal rational curve of degree $s-1$ through $P,Q$ and the intersection points of $\overline{D}$ with the indicator spaces $\Pi,\Pi^\sigma,\ldots,\Pi^{\sigma^{s-1}}$.
\end{itemize}
\end{df}
In the case $t=3$, the above construction reproduces the design obtained from the circumscribed bundle of conics; we have $q^2+q+1$ points in $\mathcal{H}$. Since $t$ is prime, necessarily $s=3$ for all pairs of points. Recall that a normal rational curve of degree $2$ is a conic, and hence, the block through two points $P$ and $Q$ is simply the intersection of $\pg{2}{q}$ with the unique conic through $P,Q,R,R^\sigma$ and $R^{\sigma^2}$. We see that indeed, these five points are in general position, and that the unique conic through these $5$ points intersects $\pi$ in a subconic. 




In the following Lemma, we will use the axiom of Veblen-Young to deduce that the point-line incidence geometry $\mathcal{H}$ is isomorphic to the point-line incidence geometry of a projective space, which is necessarily $\pg{t-1}{q}$. Note that this approach does not reprove the case $t=3$. %The fact that the circumscribed bundle of conics gives rise to a Desarguesian projective plane can be deduced from \cite{LavrauwVandeVoordeLinSetLine}.

\begin{thm} \label{Hisprojective} Let $t>3$. The incidence structure $\mathcal{H}$ is a $2$-$(\theta_{t-1},q+1,1)$ design, isomorphic to the design of points and lines in $\pg{t-1}{q}$
\end{thm}
\begin{proof}  The fact that $\mathcal{H}$ determines a $2$-$(\theta_{t-1},q+1,1)$ design follows directly from Lemma \ref{NRCs} and the fact that there are $\theta_{t-1}$ points in $\pg{t-1}{q}$. In order to show that it is isomorphic to the design of points and lines in $\pg{t-1}{q}$, we will verify that the Veblen-Young axiom holds in $\mathcal{H}$. More precisely, we will show that if the block through two points $A$ and $B$ (denoted by $AB$) has a point in common with the  block $CD$,  then the block $AD$ has a point in common with the block $BC$.

Let $A=P_a$, $B=P_b$, $C=P_c$ and $D=P_d$ be four different points of $\pi$ and assume that there is a point $P$ on $AB$ and $CD$. By Lemma \ref{NRCs}, $P=P_{au_0-bv_0}$ for some $u_0,v_0\in \FF_q$. Similarly, $P=P_{cu_1-dv_1}$ for some $u_1,v_1\in\FF_q$. Since $P=P_{au_0-bv_0}=P_{cu_1-dv_1}$, it follows that $(au_0-bv_0)/(cu_1-dv_1)\in \FF_q$, so there exists an element $\lambda\in \FF_q$ with
$$au_0-bv_0=\lambda(cu_1-dv_1),$$
or equivalently,
$$au_0+\lambda dv_1=bv_0+\lambda cu_1.$$
This implies that $P_{au_0+\lambda dv_1}=P_{bv_0+\lambda cu_1}$. Since $\lambda, u_0,v_0,u_1,v_1\in \FF_q$, the left hand side is a point of $\mathcal{C}^{a,d}$ in $\pi$, and the right hand side is a point of $\mathcal{C}^{b,c}$ in $\pi$. Hence, the blocks $AD$ and $BC$ have a point in common.
\end{proof}

It follows that $\mathcal{H}$ admits {\em subspaces}, and that we can talk about the dimension of this subspace. To avoid confusing with subspaces of $\pg{n}{q}$, we will denote subspaces of $\mathcal{H}$ by $\mathcal{H}$-subspaces.
These $\mathcal{H}$-subspaces will appear in the characterisation of the ABB-representation of a club, tangent to $\ell_\infty$ and with head different from $P_\infty$.

\section{Tangent clubs of rank \texorpdfstring{$k$}{k} in \texorpdfstring{$\pgTitle{1}{q^t}$}{PG(1,qt)}} \label{sec3}

%In this section, we will characterise all linear sets of rank $3$ on the projective line $\pg{1}{q^t}$ (embedded in $\pg{2}{q^t}$) which are tangent to $\ell_\infty$.
%Let the $t$-subspace $\Pi$ be the ABB-representation of the projective line $\pg{1}{q^t}$ and denote with $P_\infty$ the unique point of $\pg{1}{q^t}$ contained in $\ell_\infty$; its ABB-representation is a $(t-1)$-subspace of $H_\infty$ which we will denote by $\pi_\infty\subseteq\Pi$.

As in Subsection \ref{abbintro}, we let $\ell_\infty$ be the line of $\pg{2}{q^t}$ such that the ABB-representation of $\pg{2}{q^t}$ has $H_\infty=\mathcal{F}(\ell_\infty)$ as the hyperplane at infinity of $\mu=\pg{2t}{q}$. In this section, we will consider the ABB-representation of a linear set contained in a line $\ell\neq \ell_\infty$ of $\pg{2}{q^t}$. We will denote $P_\infty=\ell\cap \ell_\infty$ and the corresponding spread element by $\pi_\infty=\mathcal{F}(P_\infty)$. Let $\Pi$ be the $t$-space in $\pg{2t}{q}$  through $\pi_\infty$ containing all the points of $\phi(\ell\setminus\{P_\infty\})$.

\begin{rmk} The different perspectives on linear sets lead to different possible approaches for studying their ABB-representation. The (affine part of) the ABB-representation of a linear set $L_\pi$ on a projective line $\pg{1}{q^t}$ can be seen as the intersection of the set $\mathcal{B}(\pi)$ with a $t$-dimensional subspace containing a fixed spread element of $\spread$. Furthermore, since a linear set of rank $3$ can be seen as the projection of a subplane, and the ABB-representation of tangent and secant subplanes is understood (see \cite{RotteySheekeyVandeVoorde}), in Theorem \ref{thmscattered} we are looking to characterise the projection of certain normal rational scrolls.
The two above approaches make it possible to give a description of the ABB-representation of a linear set; for example, the ABB-representation of a scattered linear set of rank 3 tangent to the line at infinity is the projection of a normal rational scroll. However, we found these descriptions insufficient to be able to fully characterise the ABB-representation of the linear sets as done with the approach of our paper.
\end{rmk}


%\subsection{Combinatorial/geometric versus projective arguments}
%
%\begin{lm}
%    Suppose that $S$ is a linear set of rank $3$ in $\pg{1}{q^t}$ containing $P_\infty$, and let $P_1,P_2,P_3\in S\setminus\set{P_\infty}$ be three distinct points such that $\varphi(P_1)$, $\varphi(P_2)$ and $\varphi(P_3)$ lie on an affine line $\ell$.
%    Then $\varphi(S)$ contains all points of $\ell$.
%\end{lm}
%\begin{proof}
%    By Result \ref{Res_SublinesTangent}$(2.)$, the points of $\ell$ correspond to the affine points of an $\FF_q$-subline $\mathfrak{L}$ of $\pg{1}{q^t}$ tangent to $\ell_\infty$.
%    This subline $\mathfrak{L}$ contains the points $P_1$, $P_2$, $P_3$ and $P_\infty$, hence $\mathfrak{L}$ contains at least $4$ points of $S$.
%    As $S$ is a linear set of rank $3$, by Lemma \ref{Res_LinearSetIntersectionSubline}, $\mathfrak{L}$ contains at least $q+1$ points of $S$ hence $\mathfrak{L}\subseteq S$.
%    Translating this statement back to its ABB-representation (Result \ref{Res_SublinesTangent}$(1.)$) finishes the proof.
%\end{proof}
%

\subsection{Counting clubs of \texorpdfstring{$\pgTitle{1}{q^t}$}{PG(1,q\textasciicircum t)} }
In order to characterise the ABB-representation of clubs, we will count the number of different clubs with a fixed head. Note that we are not dealing with {\em (in)-equivalence} nor {\em simplicity} here; in general, clubs of rank $t$ in $\pg{1}{q^t}$ are equivalent but the same is not true for clubs of rank $k<t$ (see e.g.\ \cite{equivalence} and \cite{NPSZ}). Furthermore, in general, clubs are not necessarily {\em simple}: if $\mathcal{B}(\pi)=\mathcal{B}(\pi')$ is a club for two subspaces $\pi$ and $\pi'$ sharing a point, then it is not true that necessarily $\pi=\pi'$, nor is the head of the club determined by the point set itself (this was already noted in \cite{clubs}). However, if we specify the head of the club, we can show the following statement:
\begin{lm} \label{heads}Let $L_\pi=L_{\pi'}$ be two clubs of rank $k$ in $\pg{1}{q^t}$ with head $P$ (that is, $\pi$ and $\pi'$ are $(k-1)$-dimensional spaces and $\pi\cap \mathcal{F}(P)$ and $\pi'\cap \mathcal{F}(P)$ are $(k-2)$-dimensional). If there is a point $r$ in $\pi\cap\pi'$, and not in $\mathcal{F}(P)$, then $\pi=\pi'$. Hence, there are $\frac{q^t-1}{q-1}$ subspaces $\pi'$ such that $L_\pi=L_{\pi'}$ is a club with head $P$.
\end{lm}
\begin{proof} Let $\pi$ and $\pi'$ be as in the statement of the lemma and assume that that $\pi\neq\pi'$. Then there exists a point $s\in \pi$, not in $\pi'$, nor in $\mathcal{F}(P)$; since $\mathcal{B}(\pi)=\mathcal{B}(\pi')$, it follows that $\mathcal{B}(s)$ intersects $\pi'$ in a point $s'$. The line through $r$ and $s$ meets $\mathcal{F}(P)$ in a point, as does the line through $r$ and $s'$; hence, both define the unique $\FF_q$-subline through $\mathcal{F}^{-1}(\mathcal{B}(r))$,  $\mathcal{F}^{-1}(\mathcal{B}(s))$ and $P$ in $L_\pi$. But there is a unique transversal line through $r$ to the regulus defined by the elements $\mathcal{B}(r),\mathcal{B}(s), \mathcal{F}(P)$, a contradiction.
Finally, it is well-known that the elementwise stabiliser of the Desarguesian spread $\spread$ acts transitively on the points inside a spread element (see e.g.\ \cite[Lemma 4.3]{LavrauwVandeVoordeFieldRed}). Hence, for all $\frac{q^t-1}{q-1}$ points $u$ in $\mathcal{B}(r)$ we find a unique subspace $\pi''$ through $u$ with $\mathcal{B}(\pi'')=\mathcal{B}(\pi)$ and $\pi''\cap \mathcal{F}(P)$ a $(k-2)$-dimensional space, so the statement follows.
\end{proof}



\subsection{Clubs with head \texorpdfstring{$P_\infty$}{P-infty}}
The characterisation of the ABB-representation of clubs with head $P_\infty$ easily follows by using the different perspectives on linear sets.
\begin{prop}\label{trivial}
    Suppose that $q\geq 3$.
    A point set $\mathcal{S}$ of $\pg{1}{q^t}$ is an $\FF_q$-linear club of rank $k$ with head $P_\infty$ if and only if the ABB-representation of $S\setminus\set{P_\infty}$ is an affine $(k-1)$-space of $\Pi$.
\end{prop}
\begin{proof}% Let $L$ be the line of $\pg{2}{q^t}$ containing the point set $\mathcal{S}$ and let $\mathcal{F}$ denote the field reduction map from $\pg{2}{q^t}$ to $\pg{3t-1}{q}$.
Let $M$ be an affine point set contained in the line $\ell\neq \ell_\infty$ of $\pg{2}{q^t}$. Recall that the ABB-representation of $M$ can be obtained from intersecting the image of $M$ under the field reduction map with the subspace $\mu$ of dimension $2t$ through $H_\infty$, where $H_\infty$ is the $(2t-1)$-dimensional space $\mathcal{F}(\ell_\infty)$. We denote the subspace $\mathcal{F}(\ell)\cap \mu$ containing the ABB-representation of the affine points of $\ell$ by $\Pi$. The ABB-representation of $M$ is the intersection of spread elements $\mathcal{F}(P)$, where $P\in M$, with $\Pi$. We claim that if $M$ is the affine point set of a club with head $P_\infty$, the points of this intersection form a subspace and vice versa.

First note that if $\nu$ is an affine $(k-1)$-space of $\Pi$, and $\bar{\nu}$ denotes its projective completion, trivially, $\mathcal{B}(\bar{\nu})$ is the set of elements of the Desarguesian spread meeting a $(k-1)$-space and intersecting $P_\infty$ in a $(k-2)$-space; that is, it defines a club of rank $k$ with head $P_\infty$. 

Vice versa, suppose that $M$ is the affine point set of a club with head $P_\infty=\ell\cap \ell_\infty$. By definition, there is a $(k-1)$-dimensional subspace $\pi$ contained in $\mathcal{F}(\ell)$ such that $\mathcal{S}=\mathcal{B}(\pi)$, and furthermore, such that $\pi$ meets $H_\infty$ in a $(k-2)$-dimensional space. If $\pi$ is a subspace of $\Pi$, then we are done. Otherwise, let $v$ be a point of $\Pi$ lying in a spread element of $\mathcal{B}(\pi)$, different from $\mathcal{F}(P_\infty)=\pi_\infty$, then by Lemma \ref{heads}, there is a subspace $\pi'$ through $v$ such $\mathcal{B}(\pi')=\mathcal{B}(\pi)$. Since $\pi'$ lies in $\Pi$, we find that $\pi'$ is the intersection of $\mathcal{B}(\pi)$ with $\Pi$ and the statement follows.
\end{proof} 

%{\color{magenta} Het rode stuk is wel degelijk nodig--het gaat erom dat je een als en slechts als wilt bewijzen. Als je een club neemt in field reduction zijn dat de spread elementen die de ruimte pi snijden. Als de ruimte die je ABB definieert pi bevat, ja, dan is het triviaal dat de ABB representie van die club essentieel de deelruimte pi is. Maar het omgekeerde is niet triviaal: je vertrekt van de spreadelementen die een vaste deelruimte pi snijden, en snijdt die verzameling spreadelementen dan met een zekere ruimte mu die de ABB-representatie bepaalt. Die ruimte mu bevat niet per se pi. Elk spreadelement bepaalt een punt in mu, maar wie zegt er dat die punten samen een deelruimte bepalen?. Daarvoor heb je dat tweede geval nodig: er is altijd een deelruimte pi' te vinden in mu die dezelfde spread elementen snijdt als pi.}
%\textcolor{cyan}{Aha, ik denk dat ik snap wat je bedoelt. Toen ik het eerst las leek het alsof je twee keer dezelfde richting zou bewijzen (want in het begin van het rode deel start ook met S een club). Misschien moeten we dit toch nog wat verduidelijken? }{\color{magenta} Akkoord, het was slecht opgeschreven, hopelijk nu duidelijker.}

Let $\qbin{n}{k}$ denote the number of $(k-1)$-dimensional subspaces of $\pg{n-1}{q}$, that is, $$\qbin{n}{k}=\frac{(q^n-1)(q^{n-1}-1)\cdots (q-1)}{(q^k-1)(q^{k-1}-1)\cdots(q-1)},$$
and let $\theta_m$ be the number of points in $\pg{m-1}{q}$, that is,
\[
    \theta_m=\frac{q^m-1}{q-1}.
\]

\begin{prop}\label{heads2} There are 
$q^{t-k+1}\qbin{t}{k-1}$
clubs $L_\pi$ of rank $k$ with head $P_\infty$. 
\end{prop}
\begin{proof} There are $\qbin{t}{k-1}$ {\color{black} subspaces of dimension $k-2$} in $\pi_\infty=\mathcal{F}(P_\infty)$, and each of them lies on  $\frac{q^{2t-k+1}-1}{q-1}-\frac{q^{t-k+1}-1}{q-1}$ subspaces of dimension $k-1$, not contained in $\pi_\infty$. By Lemma \ref{heads}, there are $\theta_{t-1}$ of such $(k-1)$-spaces $\pi$ giving rise to the same club. Hence, we find that there are
$$\frac{\qbin{t}{k-1}(\frac{q^{2t-k+1}-1}{q-1}-\frac{q^{t-k+1}-1}{q-1} )}{\frac{q^t-1}{q-1}} =q^{t-k+1}\qbin{t}{k-1}$$ clubs with head $P_\infty$.

\end{proof}
\subsection{Clubs with head different from \texorpdfstring{$P_\infty$}{P-infty} }
\begin{prop}\label{Prop_NumberClubsHeadNotInftyevenmoregeneral} Let $H$ and $P_\infty$ be two different points of $\pg{1}{q^t}$. Then there exist $\qbin{t}{k-1}$ clubs $L_\pi$ through $P_\infty$ with head $H$, {where $\pi$ is a $(k-1)$-space}.
    
    %There exist $\qbin{t}{k-1}$ clubs of rank $k$ in $\pg{1}{q^t}$ with head $H$ containing $P_\infty$.
     Furthermore, there are $q^t\qbin{t}{k-1}$ clubs $L_\pi$, where $\pi$ is a $(k-1)$-space, containing $P_\infty$, with head different from $P_\infty$.
%    And there are $q^t\qbin{t}{k-1}$ clubs of rank $k$ in $\pg{1}{q^t}$ containing $P_\infty$ and have a head different from $P_\infty$.
\end{prop}
\begin{proof}
    Let $\gamma:=\mathcal{F}(H)$.
    A $(k-2)$-space $g$ in $\gamma$ and a point $P$ in $\pi_\infty$ span a $(k-1)$-space $\langle g,P\rangle$ which defines a club with head $H$ and containing $P_\infty$.
   By Lemma \ref{heads}, every club with head $H$ and containing $P_\infty$ is defined by exactly $\theta_{t-1}$ such $(k-1)$-spaces, so the total 
 number of clubs through a fixed head point $H\neq P_\infty$ and containing $P_\infty$ is
    \[
        \frac{\qbin{t}{k-1}\theta_{t-1}}{\theta_{t-1}}\textnormal{.}
    \]
    There are $q^t$ choices for a point $H\neq P_\infty$, and each subspace $\pi$ defines a unique $H$, so there are $q^t\qbin{t}{k-1}$ clubs $L_\pi$, where $\pi$ is a $(k-1)$-space  and the head is different from $P_\infty$.
\end{proof}

%\begin{prop}\label{Prop_Numbercones}
%    There exists $\theta_2 q^3$ cones in $\Pi$ with vertex a point $H\notin \pi_\infty$ and basis a non-degenerate conic in $\pi_\infty$ such that its $\mathbb{F}_{q^t}-$extension contains the $3$ conjugate points that generate the spreadelement $\pi_\infty$.
%\end{prop}
%\begin{proof}
%   We count the number of cones spanned by a point $H\in \Pi\setminus \pi_\infty$ and a non-degenerate conic in $\pi_\infty$ through its three conjugate points. It is known that all non-degenerate conics in $\pi_\infty$, such that it contains three fixed conjugated points, together with all points in $\pi_\infty$ forms a projective plane \textcolor{magenta}{or $2-(\theta_2, q+1,1)$-design or circumscribed bundle of conics } \cite{}. Hence, there are $\theta_2$ possibilities for the conic in $\pi_\infty$. Through each conic, there are $q^3$ cones; one for each vertex point $H\notin \pi_\infty$.
%\end{proof}
\begin{prop}\label{Prop_Numbercones}
    There exists $q^t\qbin{t}{k-1}$ cones in $\Pi$ with vertex a point $H\notin \pi_\infty$ and base a $(k-2)$-dimensional subspace of the $2$-design $\mathcal{H}$.
    
    %basis a non-degenerate conic in $\pi_\infty$ such that its $\mathbb{F}_{q^t}-$extension contains the $3$ conjugate points that generate the spreadelement $\pi_\infty$.
\end{prop}
\begin{proof}
From Theorem \ref{Hisprojective}, it follows that the number of $(k-2)$-dimensional subspaces of $\mathcal{H}$ equals the number of $(k-2)$-spaces in $\pg{t-1}{q}$, that is, $\qbin{t}{k-1}$. Furthermore, there are $q^t$ points in $\Pi$, not in $\pi_\infty$, each of which defines a unique cone with vertex that point and base a $(k-2)$-dimensional subspace of $\mathcal{H}$. 
\end{proof}

%%%%HIER
In order to characterise the ABB-representation of a club with head, different from the point at infinity, we need the following Lemma from \cite{sam}.
\begin{lm}[{\cite[Lemma 5.7]{sam}}] \label{lemmahyp}  Assume that $\mS$ is a point set in $\pg{n}{q}$, $q\geq 4$, with the property that every line intersects $\mS$ in $0,1,q$ or $q+1$ points. Then there exists a hyperplane $H$ in $\pg{n}{q}$ such that either $\mS\subseteq H$ or $\mS^c\subset H$, where $\mS^c$ denotes the complement of $\mS$ in $\pg{n}{q}$.

\end{lm}
%%
\begin{thm} \label{thmclub}A set $\mathcal{S}$ is an $\FF_q$-linear club of rank $k$ in $\pg{1}{q^t}$ containing $P_\infty$ and with head $H\neq P_\infty$, if and only if $\phi(\mathcal{S}\setminus\{P_\infty\})$, the ABB-representation of $\mathcal{S}\setminus\{P_\infty\}$ in $\pg{2t}{q}$, is the affine point set of a cone with vertex $\phi(H)$ and base an $\mathcal{H}$-subspace of dimension $(k-2)$  in $\mathcal{F}(P_\infty)$ (the spread element corresponding to $P_\infty$).

\end{thm}
\begin{proof} Let $\mathcal{S}$ be an $\FF_q$-linear club of rank $k$ containing $P_\infty$ and with head $H\neq P_\infty$, and let $\phi(H)$ be the ABB-representation of the head $H$. Let $Q\notin\{H,P_\infty\}$ be a point of $\mathcal{S}$. By Result \ref{Res_LinearSetClub}(a), we know that the subline through $H,Q,P_\infty$ is contained in $\mathcal{S}$. By Result \ref{Res_SublinesTangent}(a), the ABB-representation of the points, different from $P_\infty$, of this subline are the affine points of the line through $\phi(H)$ and $\phi(Q)$. In other words, the $q^{k-1}-1$ points of $\mS\setminus\{H,P_\infty\}$ are contained in $\frac{q^{k-1}-1}{q-1}$ lines through $\phi(H)$, that is, they form a cone with vertex $\phi(H)$. The projective completions of those lines meet $\mathcal{F}(P_\infty)$ in a set $\mathcal{K}$ of $\frac{q^{k-1}-1}{q-1}$ points. 


%To show that $\mathcal{K}$ is a subspace of dimension $(k-2)$ of $\mathcal{H}$, we will show that for every two points of $\mathcal{K}$, the unique block of $\mathcal{H}$ through them is again contained in $\mathcal{H}$.
Let $R_i$, $i=1,2$, be two different points of $\mathcal{K}$, and let $Q_i$ be a point on the line through $\phi(H)$ and $R_i$, different from $\phi(H)$ and $R_i$. We have that $Q_i=\phi(S_i)$ for some point $S_i\in \mathcal{S}$. Moreover, from Result \ref{Res_LinearSetClub}(a), we know that the subline $m$ through $H,S_1,S_2$ is contained in $\mathcal{S}$. Let $s$ be the integer such that the smallest subline containing $m$ and tangent to $\ell_\infty$ is an $\FF_{q^s}$-subline. Then by Result \ref{Res_SublinesTangent}(b), we know that the affine points of this subline correspond to a normal rational curve $\mathcal{C}$ through $\phi(H),Q_1, Q_2$, contained in an $s$-space meeting $\mathcal{F}(P_\infty)$ in an element $D$ of $\spread_s$, whose $\FF_{q^t}$-extension intersects the indicator set of $\spread_s$ in $s$ conjugate points. Note that $R_1,R_2$ are contained in $D$, and hence, $D$ is the unique element of $\spread_s$ containing $R_1,R_2$.

By Result \ref{projNRC}, the projection of the normal rational curve $\mathcal{C}$ from the point $\phi(H)\in \mathcal{C}$ onto $H_\infty$ is contained in a normal rational curve; this curve is contained in $\pi_\infty$, goes through $R_1$, $R_2$ and the extension contains the same points in $H_\infty$ as $\mathcal{C}$ did. Hence, the block of the design $\mathcal{H}$ through $R_1,R_2$ contains $q$ points of $\mathcal{K}$. It follows that $\mathcal{K}$ is a point set meeting every block in $0,1,q$ (or $q+1$) points. By Theorem \ref{Hisprojective}, $\mathcal{H}$ is isomorphic to the point-line design of  $\pg{t-1}{q}$ so we may use Lemma \ref{lemmahyp} to conclude that $\mathcal{K}$ or its complement must be contained in a hyperplane $\mu$ of the design $\mathcal{H}$. Since $\frac{q^t-1}{q-1}-|\mathcal{K}|>\frac{q^{t-1}-1}{q-1}$, the latter possibility does not occur.
We can repeat the same reasoning in the $(t-2)$-dimensional $\mathcal{H}$-subspace $\mu$: all blocks of $\mu$ meet $\mathcal{K}$ in $0,1,q$ or $q+1$ points, and since $\frac{q^{t-1}-1}{q-1}-|\mathcal{K}|>\frac{q^{t-2}-1}{q-1}$, $\mathcal{K}$ is contained in a hyperplane of $\mu$, that is, a $(t-3)$-dimensional $\mathcal{H}$-subspace. Continuing in this fashion, we conclude that $\mathcal{K}$ is contained in a $(k-2)$-dimensional $\mathcal{H}$-subspace . Since $|\mathcal{K}|=\frac{q^{k-1}-1}{q-1}$, equality holds.


Furthermore, by Propositions \ref{Prop_Numbercones} and \ref{Prop_NumberClubsHeadNotInftyevenmoregeneral}, the number of such cones equals the number of $\FF_q$-linear club of rank $k$ containing $P_\infty$ and with head $H\neq P_\infty$, and the theorem follows.
\end{proof}



\section{Tangent scattered linear sets of rank \texorpdfstring{$3$}{3} in \texorpdfstring{$\pgTitle{1}{q^3}$}{PG(1,q\textasciicircum3)} }\label{sec4}


We continue to use the same notations as in the previous section, as introduced in Subsection \ref{abbintro}.

%From Result \ref{} we know that there are $2$ scattered planes through a fixed point of $D$ determining the same linear set such that the linear set $\mathcal{F}^{-1}\B(\pi)$ contains some fixed point. 
%
%
%
%Furthermore, there are $(q^3+1)(q^2+q+1)(q^3+q^2+q)$ planes intersecting some element of $\spread$ in a line: there are $(q^3+1)(q^2+q+1)$ choices for an element of $\spread$
%    All linear sets of rank $3$ are defined by spread elements of $\pg{5}{q}$ intersecting a fixed plane.
%    Of all planes in $\pg{5}{q}$, precisely $q^3+1$ planes (the spread elements) will define a linear set consisting of one point.
%    Furthermore, $(q^3+1)\theta_2(\theta_3-1)$ other planes will define clubs; after all, of the $q^3+1$ spread elements, one will correspond to the head of the club, and for each of the $\theta_2$ lines in this plane, one can choose $\theta_3-1$ possible planes that will define a club.
%    In conclusion, as there exist a total of $(q^3+1)(q^2+1)\theta_4$ planes in $\pg{5}{q}$, we are left with precisely
%    \[
%        (q^3+1)(q^2+1)\theta_4-(q^3+1)-(q^3+1)\theta_2(\theta_3-1)=(q^3+1)q^3(q^3-1)
%    \]
%    planes in $\pg{5}{q}$ that define a scattered linear set.
%    Nevertheless, not all such planes will define a scattered linear set that contains $P_\infty$ (i.e.\ through a fixed point).
%    As we know the total number of planes that define a scattered linear set and know that each of these planes intersect precisely $q^2+q+1$ of the $q^3+1$ spread elements, we can perform a double counting to obtain that there are
%    \[
%        \frac{(q^3+1)q^3(q^3-1)\cdot(q^2+q+1)}{q^3+1}=q^3(q^3-1)\theta_2
%    \]
%    planes that define a scattered linear set containing $P_\infty$.
%    
%    However, in this way, each scattered linear set will be counted $2\theta_2$ times \textcolor{red}{[Geertrui, waarom was dit weer precies? (of toch, vanwaar komt de $\theta_2$ weer) (3/3)]}.
%    Hence, the total number of scattered linear sets through $P_\infty$ is
%    \[
%        \frac{q^3(q^3-1)\theta_2}{2\theta_2}\textnormal{.}\qedhere
%    \]




\begin{prop}\label{combprop}
    Suppose that $q\geq 5$. 
    Let $\mathcal{U}$ be a point set of $\ag{3}{q}$ with the following three properties:
    \begin{enumerate}
        \item for each line $\ell$ holds that $|\ell\cap\mathcal{U}|\in\{0,1,2,q\}$,
        \item through each point of $\mathcal{U}$, there exist precisely two lines that are contained in $\mathcal{U}$, and
        \item $|\mathcal{U}|=q^2+q$.
    \end{enumerate}
    Let $\pi_\infty$ be the plane at infinity when embedding $\ag{3}{q}$ in $\pg{3}{q}$.
    Then $\mathcal{U}$ is the affine part of a hyperbolic quadric in $\pg{3}{q}$ that intersects $\pi_\infty$ in a non-degenerate conic.
\end{prop}
\begin{proof}
    We claim that the intersection of a plane $\sigma$ with $\mathcal{U}$ is either a cap or the union of two distinct lines. 
    First note that it impossible for $\sigma\cap\mathcal{U}$ to contain two lines $\ell_1$, $\ell_2$ and a point $R\in\mathcal{U}\setminus\left(\ell_1\cup\ell_2\right)$: in this case, since $q\geq 5$, we find that there are at least $3$ lines through $R$ meeting $\ell_1$ and $\ell_2$ in distinct points, which forces those lines to be contained in $\mathcal{U}$ by Property 1., contradicting Property 2.
    
   
    Suppose that $\sigma\cap\mathcal{U}$ is not a cap, then there exists a line $r$ in $\sigma$ with at least three points of $\mathcal{U}$. By Property 1., $r$ is contained in $\mathcal{U}$.
    By Property $2.$, there exists another line contained in $\mathcal{U}$ through each of the $q$ points on $r$; let $\ell_1,\ldots,\ell_q$ denote those lines. They are necessarily pairwise disjoint since otherwise, we would find a plane with three lines of $\mathcal{U}$. Hence, the $q$ distinct planes $\langle r,\ell_j\rangle$, $j=1,\ldots,q$, intersect $\mathcal{U}$ precisely in $\ell_j$ and $r$, and the lines $\ell_j$ meet $r$ each in a different point. As $|\mathcal{U}|=q^2+q$ (Property $3.$), the remaining plane $\tau$ through $r$ contains precisely $q$ points of $\mathcal{U}$ not on the line $r$.
    Let $Q_1$ and $Q_2$ be two distinct such points.
    If $\langle Q_1,Q_2\rangle$ intersects $r$, then $\langle Q_1,Q_2\rangle$ contains three distinct points of $\mathcal{U}$ and hence, by Property $1.$, is contained in $\mathcal{U}$, which implies that $\langle Q_1,Q_2\rangle\cap r$ is a point of $\mathcal{U}$ through which there exist at least three lines fully contained in $\mathcal{U}$, contradicting Property $2$. We find that the $q$ points of $(\tau\cap\mathcal{U})\setminus r$ are precisely those of an affine line, parallel with $r$ (*).
    
    Let $\mu(\mathcal{U})$ denote the set of projective lines of $\pg{3}{q}$ whose affine points are contained in the set $\mathcal{U}$, and let $\mathcal{U}_\infty$ be the set of points in $\pi_\infty$ which are contained in a line of $\mu(\mathcal{U})$.
    Let $\Tilde{\mathcal{U}}:=\mathcal{U}\cup \mathcal{U}_\infty$. 
    Now we prove that $\Tilde{\mathcal{U}}$, together with the set of projective lines $\mu(\mathcal{U})$, form a generalised quadrangle with parameters $(s,t)=(q,1)$ embedded in $\pg{3}{q}$, and hence, a hyperbolic quadric $Q^+(3,q)$.    As $\mu(\mathcal{U})$ is a set of projective lines, each one contains $q+1=s+1$ points.
    
    Moreover, by Property $2.$, we know that every affine point is contained in precisely $2=t+1$ lines. 
    Hence let $P\in\mathcal{U}_\infty$ be a point at infinity incident with a line $\ell_P\in\mu(\mathcal{U})$. From ($\ast$), we have that there is precisely one line in $\mu(\mathcal{U})$, different from $\ell_P$ whose extension is $P$. % If there would be three lines $m_1,m_2,m_3$ contained in $\mathcal{U}$ whose extensions contain the point $P$, then there are two different planes through $m_1$ containing a line parallel to $m_1$, a contradiction. 
    Since there are $q^2+q$ points in $\mathcal{U}$, each on exactly $2$ lines, we have that there are $2(q+1)$ lines contained in $\mathcal{U}$, giving rise to $q+1$ points in $\pi_\infty$. Furthermore, it follows from the fact that there are no planes with more than $2$ lines that there are no triangles in $\Tilde{\mathcal{U}}$. Hence, $\Tilde{\mathcal{U}}$ is indeed a generalised quadrangle of order $(q,1)$ embedded in $\pg{3}{q}$. Since it has $q^2+q$ affine points by Proposition 3, it meets $\pi_\infty$ in $q+1$ points forming a non-degenerate conic.
    \end{proof}

\begin{lm}\label{oneway}
    Suppose that $q\geq 5$.
    If $\mS\ni P_\infty$ is a scattered linear set of rank $3$ of $\pg{1}{q^3}$, then the ABB-representation of $\mS\setminus\set{P_\infty}$ is the affine part of a hyperbolic quadric $\mathcal{Q}$ intersecting the plane $\pi_\infty$ in a non-degenerate conic. Furthermore, the extension of this conic contains the $3$ conjugate points defining the spread element $\pi_\infty$.
\end{lm}
\begin{proof}
Let  $\mS\ni P_\infty$ be a point set of $\pg{1}{q^3}$, which is a scattered linear set of rank $3$ and let $T$ be the ABB-representation of $S\setminus\{P_\infty \}$.

We see that the three conditions of Proposition \ref{combprop} hold for $\mathcal{U}=T$:

\begin{enumerate}
    \item An affine line $\ell\in \Pi$ corresponds to a tangent subline of $\pg{1}{q^3}$. Condition $1$ follows from Result \ref{Res_LinearSetIntersectionSubline}.
    \item By Result \ref{Res_SublinesTangent} we know that through every two distinct points $P_1, P_2$ of $S$ there are precisely two $\mathbb{F}_q$-sublines contained in $S$. Let $P_1$ be the point at infinity $P_\infty$ and let $P_2$ be a random affine point in $S$. Then we know that $P_2$ is contained in precisely two tangent $\mathbb{F}_q$-sublines.  Hence, we know by Result $\ref{Res_SublinesTangent}$ that $\varphi(P_2)$ is contained in precisely two lines fully contained in $T$.
    \item The scattered linear set contains $q^2+q+1$ points, of which $q^2+q$ affine ones. 
\end{enumerate}
This implies that $T$ is the affine point set of a hyperbolic quadric. Now consider $\mathcal{Q}$, the extension to $\FF_{q^t}$ of the projective completion of $T$.

By Proposition \ref{Res_LinearSetClub}, through two points of $\mS\setminus\{P_\infty\}$, there are two sublines contained in $\mS$, at least one of which, say $m$, does not contain $P_\infty$. 
By Result \ref{Res_SublinesTangent}, we know that the $\FF_q$-subline $m$, corresponds to a normal rational curve $\mathcal{C}$ whose extension to $\FF_{q^t}$ contains the $3$ conjugate points defining the spread element $\pi_\infty$. Since $m\subseteq\mS$, the extension of $\mathcal{C}$ is contained in $\mathcal{Q}$, and hence, $\mathcal{Q}$ contains the $3$ conjugate points defining $\pi_\infty$.
\end{proof}
\begin{rmk} The first part of Lemma 4.3 can also be proven using the coordinate description of $\mathcal{B}(\pi)$, where $\pi$ is a scattered plane in $\pg{5}{q}$ with respect to the Desarguesian plane spread $\spread$, derived in \cite{LSZ}. If we intersect the hypersurface, whose coordinates are explicitly described there, with a $3$-dimensional subspace containing a spread element $S$ of $\spread$, we find the union of a hyperbolic quadric with the points of $S$. To show that the extension of this hyperbolic quadric contains the $3$ conjugate points, one could then use the coordinates for the indicator sets derived in \cite{BarwickJackson}.
\end{rmk}





\begin{prop}\label{Prop_Numberhyperbol}
    There exists $\frac{1}{2} q^3(q^3-1)$ hyperbolic quadrics $\mathcal{Q}$ in $\Pi$, intersecting the plane $\pi_\infty$ in a non-degenerate conic $\mathcal{C}$ such that its $\mathbb{F}_{q^t}-$extension contains the $3$ conjugate points generated by the spreadelement $\pi_\infty$.
\end{prop}
\begin{proof}
    We again use the fact that all non-degenerate conics in $\pi_\infty$, such that its extension contains three fixed conjugated points, together with all points in $\pi_\infty$ form a $2-(\theta_2, q+1,1)$-design as shown in \cite{BakerBrownEbertFisher}. Hence, there are $\theta_2$ possibilities for choosing an appropriate conic in $\pi_\infty$. It is known that the total number of hyperbolic quadrics in $\Pi$ is $\frac{1}{2}q^4(q^2+1)(q^3-1)$, the number of non-degenerate conics contained in a fixed hyperbolic quadric is $\theta_3-(q+1)^2=q(q^2-1)$ and the number of non-degenerate conics in a solid is $\theta_3 q^2(q^3-1)$ \cite{Hirschfeldgalois}. We can now perform a double counting to obtain that there exist 
    \begin{align*}
        \frac{\frac{1}{2}q^4(q^2+1)(q^3-1)q(q^2-1)}{\theta_3 q^2(q^3-1)}=\frac{1}{2}q^3(q-1)
    \end{align*}
    hyperbolic quadrics containing a fixed non-degenerate conic. Hence, in total, there are $\frac{1}{2} q^3(q^3-1)$ hyperbolic quadrics $\mathcal{Q}$ in $\Pi$, intersecting the plane $\pi_\infty$ in a non-degenerate conic $\mathcal{C}$ such that its $\mathbb{F}_{q^t}-$extension contains the $3$ conjugate points generated by the spreadelement $\pi_\infty$. 
\end{proof}


\begin{prop}\label{Prop_NumberScattered}
    Let $q\geq 5$. There exists $\frac{1}{2}q^3(q^3-1)$ scattered linear sets of rank $3$  in $\pg{1}{q^3}$ which contain $P_\infty$.
\end{prop}
\begin{proof} We will first count the number of scattered planes in $\pg{5}{q}$ with respect to the Desarguesian plane spread $\spread$. There are $\qbin{6}{3}$ planes in $\pg{5}{q}$, of which $q^3+1$ are elements of $\spread$. Now consider  triples $(S,L,\pi)$, where $S$ is an element of $\spread$, $L$ is a line in $S$, and $\pi$ is a plane containing $L$, different from $S$. It easily follows that there are $(q^3+1)(q^2+q+1)(q^3+q^2+q)$ such triples, and since the choice of the plane $\pi$ defines $S$ and $L$ in a unique way, we find $(q^3+1)(q^2+q+1)(q^3+q^2+q)$ planes meeting some spread element in exactly a line. We conclude that there are $\qbin{6}{3}-(q^3+1)-(q^3+1)(q^2+q+1)(q^3+q^2+q)=(q^3+1)q^3(q^3-1)$ scattered planes.
%Each of those scattered planes determines $(q^2+q+1)$ elements of $\spread$, and there $q^3+1$ points in $\spread$, so there are $(q^3+1)q^3(q^3-1)(q^2+q+1)/(q^3+1)=q^3(q^3-1)(q^2+q+1)$ planes meeting a fixed element $D$ of $\spread$. 
Now count $(\pi,r,S)$ where $r$ is a point of the scattered plane $\pi$ such that $L_\pi$ is the scattered linear set $S$.
On one hand, we have $(q^3+1)q^3(q^3-1)$ scattered planes $\pi$ determining a unique linear set $S$, and $q^2+q+1$ points $r$. On the other hand, by Result \ref{Res_LinearSetClub}(c), we have that given $S$ and $r$, there are exactly $2$ planes $\pi$ through $r$ with $L_\pi=S$. It follows that $|S|(q^2+q+1)2=(q^3+1)q^3(q^3-1)(q^2+q+1)$, and hence, $|S|=\frac{(q^3+1)q^3(q^3-1)}{2}$. The number of scattered linear sets through each of the $q^3+1$ points of $\pg{1}{q^3}$ is a constant, so there are $\frac{q^3(q^3-1)}{2}$ scattered linear sets through $P_\infty$. \end{proof}
\begin{thm} \label{thmscattered} A set $\mathcal{S}$ is the ABB-representation of the affine point set of a scattered linear set of rank $3$ in $\pg{1}{q^3}$, containing $P_\infty$ if and only if it is the affine point set of a hyperbolic quadric intersecting the plane $\pi_\infty$ in a non-degenerate conic $\mathcal{C}$ such that its $\mathbb{F}_{q^t}-$extension contains the $3$ conjugate points generated by the spreadelement $\pi_\infty$.
\end{thm}
\begin{proof} Lemma \ref{oneway} proves that the ABB-representation of the affine point set of a scattered linear set of rank $3$ in $\pg{1}{q^3}$, containing $P_\infty$ is a hyperbolic quadric intersecting the plane $\pi_\infty$ in a non-degenerate conic $\mathcal{C}$ whose extension contains the $3$ conjugate points generating the spreadelement $\pi_\infty$. For the other direction, it suffices to note that the number of such hyperbolic quadrics found in Proposition  \ref{Prop_Numberhyperbol} is precisely the number of scattered linear sets containing $P_\infty$ counted in Proposition \ref{Prop_NumberScattered}.
\end{proof}

%\begin{crl} The representation is...
%
%\end{crl}














\section{The optimal case of seven planes of \texorpdfstring{$\pgTitle{5}{q}$}{PG(5,q)} in higgledy-piggledy arrangement}\label{sec5}

%{\em Blocking sets} are one of the central topics in finite geometry.
%%Within the research domain of finite geometry, the notion of \emph{blocking sets} is one of great interest.
%%We adopt the definition used in \cite{DeBeuleStorme}.
%
%\begin{df}\label{Def_BlockingSet}
%    Let $k\in\set{0,1,\dots,n-1}$.
%    A $k$\emph{-blocking set} of $\pg{n}{q}$ is a point set that meets every $(n-k)$-dimensional subspace.
%\end{df}
%
%A $k$-subspace is the easiest (and smallest) example of a $k$-blocking set of $\pg{n}{q}$ \cite{BoseBurton}.
%A natural generalisation of a $k$-blocking set of $\pg{n}{q}$ is the concept of a $t$-fold $k$-blocking set, $t\in\NN$, which is a point set of $\pg{n}{q}$ that meets every $(n-k)$-dimensional subspace in at least $t$ points.
%Obviously, any set of $t$ pairwise disjoint $k$-subspaces is a $t$-fold $k$-blocking set.

In order to define higgledy-piggledy sets, we need the concept of a {\em strong $k$-blocking set}, which was introduced in \cite[Definition $3.1$]{DavydovEtAl}. They have also appeared in the literature under the terminology \emph{generator sets} and \emph{cutting blocking sets}.

\begin{df}
    Let $k\in\set{0,1,\dots,n-1}$.
    A \emph{strong $k$-blocking set} in $\pg{n}{q}$ is a point set that meets every $(n-k)$-dimensional subspace $\kappa$ in a set of points spanning $\kappa$.
\end{df}

%Note that any strong $k$-blocking set is necessarily an $(n-k+1)$-fold $k$-blocking set, although the converse is generally false.
%Following this line of thought, one could try to construct a strong $k$-blocking set by considering all points lying in the union of a certain number of well-chosen $k$-subspaces. Although sporadic examples of such point sets were already presented in \cite{DavydovEtAl}, this idea was thoroughly investigated in \cite{FancsaliSziklai2,HegerPatkosTakats} for $k=1$ and later generalised in \cite{FancsaliSziklai3} to arbitrary $k$.

\begin{df}
    Let $k\in\set{0,1,\dots,n-1}$ and suppose that $\mathcal{K}$ is a set of $k$-subspaces in $\pg{n}{q}$.
    If the union of points contained in at least one subspace of $\mathcal{K}$ is a strong $k$-blocking set, then the elements of $\mathcal{K}$ are said to be in \emph{higgledy-piggledy arrangement} and the set $\mathcal{K}$ itself is said to be a \emph{higgledy-piggledy set of $k$-subspaces}.
\end{df}
%
%As one generally wishes to construct higgledy-piggledy sets of small size, \textbf{lower bounds} on the size of such sets were determined to reveal which sizes would (theoretically) be optimal.
%Based on the reasoning behind \cite[Theorem $14$]{FancsaliSziklai2}, the authors of \cite[Theorem $20$]{FancsaliSziklai3} inductively determined a lower bound on the size of general higgledy-piggledy sets of $k$-subspaces.
%Consequently, by taking the duality of the projective space into account, the author of \cite{Denaux} slightly refined this bound.
%

The goal is to construct higgledy-piggledy sets of small size. The following particular cases follow from the known lower bounds (see \cite{FancsaliSziklai3}, and \cite{Denaux} for a slight improvement):
%\begin{thm}[{\cite{Denaux,FancsaliSziklai3}}]\label{Res_LowerBound}
%    Let $k\in\set{0,1,\dots,n-1}$.
%    A higgledy-piggledy set of $k$-subspaces in $\pg{n}{q}$ contains at least
%    \[
%        \min\set{q,\max\set{(k+1)+\sum_{i=1}^{k+1}\left\lfloor\frac{n-k-1}{i}\right\rfloor,(n-k)+\sum_{i=1}^{n-k}\left\lfloor\frac{k}{i}\right\rfloor}}+1
%    \]
%    elements.
%\end{thm}
%
%Focusing on ambient projective geometries of small dimension, the above theorem implies the following (excluding the trivial cases $k\in\set{0,n-1}$).

\begin{crl}\label{Crl_LowerBound}
    If $0<k<n-1$ and $q\geq 7$, then a higgledy-piggledy set of $k$-subspaces
    \begin{enumerate}
        \item contains at least $4$ elements if $n=3$,
        \item contains at least $6$ elements if $n=4$, and
        \item contains at least $7$ elements if $n=5$.
    \end{enumerate}
\end{crl}

The above lower bounds are sharp (\cite[Theorem $3.7$, Example $9$]{DavydovEtAl,FancsaliSziklai2}, \cite[Proposition $12$]{BartoliKissMarcuginiPambianco}, \cite[Theorem $3.15$]{BartoliCossidenteMarinoPavese}, \cite[Theorem $33$ and $39$, Corollary $34$ and $35$]{Denaux}), except for the case $(n,k)=(5,2)$. Concerning the latter case, the author of \cite{Denaux} used the following construction to find $8$ planes in higgledy-piggledy arrangment.
%A natural research question that arises asks whether there exist higgledy-piggledy sets of $k$-subspaces in $\pg{n}{q}$ with sizes equal to the lower bounds mentioned above.
%General existing results close to this bound do exist (\cite[Proposition $10$ and Corollary $29$]{FancsaliSziklai3}), but sporadic examples of higgledy-piggledy sets in projective space of small dimension hint that these results might be open to improvement.
%In fact, the author of \cite{Denaux} summarizes all best-known smallest sizes of higgledy-piggledy sets in $\pg{n}{q}$, $n\leq5$, and concludes that there exist such sets of minimal size for all constraint parameters, with one exception.
%
%\begin{thm}[{\cite[Theorem $39$]{Denaux}}]\label{Res_EightHigPigPlanes5}
%    There exist eight pairwise disjoint planes of $\pg{5}{q}$ in higgledy-piggledy arrangement.
%\end{thm}
%
%Corollary \ref{Crl_LowerBound} does not prohibit the existence of a higgledy-piggledy plane set in $\pg{5}{q}$ of size $7$.
%By using field reduction, one can try to find such a set of size $7$ as part of a Desarguesian plane spread; the same method was used to obtain Theorem \ref{Res_EightHigPigPlanes5}:

\begin{crl}\label{Crl_ConstructionMethod}
    Suppose that $\mathcal{P}$ is a point set of $\pg{1}{q^3}$ that is not contained in any $\FF_q$-linear set of rank at most $3$.
    Then $\mathcal{F}(\mathcal{P})$ is a higgledy-piggledy set of pairwise disjoint planes in $\pg{5}{q}$.
\end{crl}
\begin{proof}
    This is a special case of \cite[Theorem $16$]{Denaux}.
\end{proof}

%If $q>5$, a higgledy-piggledy plane set in $\pg{5}{q}$ constructed in this way is optimal in the following sense:
Any higgledy-piggledy set of planes constructed in this way consists of disjoint planes; however, it is worth noting that this is not a restriction:
\begin{prop}[{\cite[Proposition $40$]{Denaux}}]
    If $q\geq 7$, then any seven planes of $\pg{5}{q}$ in higgledy-piggledy arrangement are pairwise disjoint.
\end{prop}

Using the results obtained in previous sections, we are able to show that the lower bound of Corollary \ref{Crl_LowerBound} is sharp in the case $n=5$:
%It is no wonder that the author of \cite[Open Problem $42$]{Denaux} proposed this as an open problem.
%Fortunately, thanks to the results obtained in previous sections, we are able to prove the existence of such an optimal set.

\begin{thm}
    There exist seven planes of $\pg{5}{q}$ in higgledy-piggledy arrangement.
\end{thm}
\begin{proof}
If $q\leq5$, we can easily verify the statement using a computer package such as GAP (see e.g.\ \cite[Code Snippet $56$]{Denaux})\footnote{In fact, using similar code, one can check that there exist in fact $6$ planes of $\pg{5}{3}$ and $5$ planes of $\pg{5}{2}$ in higgledy-piggledy arrangement.}.
Hence, assume that $q\geq5$ for the remainder of this proof.
By Corollary \ref{Crl_ConstructionMethod}, it is sufficient to pick $7$ points in $\pg{1}{q^3}$ such that no linear set of rank at most $3$ contains all these $7$ points. First note that if $7$ points are contained in a linear set of rank $<3$, they are also contained in a linear set of rank $3$. Hence, we only need to show that it is possible to pick $7$ points, not contained in a linear set of rank $3$.

Pick a point $P_\infty$ in $\pg{1}{q^3}$. Then we know from Proposition \ref{heads2} that there are $q^3+q^2+q$ clubs with head $P_\infty$, from Proposition \ref{Prop_NumberClubsHeadNotInftyevenmoregeneral} that there are $q^3(q^2+q+1)$ clubs through $P_\infty$ with head different from $P_\infty$, and from Proposition \ref{Prop_NumberScattered} that there are $\frac{1}{2}q^3(q^3-1)$ scattered linear sets containing $P_\infty$.

We will count the set $S=\{(P_1,P_2,P_3,P_4,P_5,P_6,L)\}$ where $P_i\neq P_\infty$ are different points of $\pg{1}{q^3}$ and $L$ is a linear set of rank $3$ containing $P_\infty$ and $P_i$, $i=1,\ldots,6$.
We have that $$|S|=(q^3+q^2+q)c+q^3(q^2+q+1)c+\frac{1}{2}q^3(q^3-1)d,$$ where $c=q^2(q^2-1)(q^2-2)(q^2-3)(q^2-4)(q^2-5)$ is the number of ways to pick $6$ different points different from $P_\infty$ in a club through $P_\infty$, and $d=(q^2+q)(q^2+q-1)(q^2+q-2)(q^2+q-3)(q^2+q-4)(q^2+q-5)$ is the number of ways to pick $6$ points different from $P_\infty$ in a scattered linear set through $P_\infty$.

If all choices of $6$ points $P_1,\ldots,P_6$ would be contained in at least one linear set of rank $3$ through $P_\infty$, then $|S|\geq q^3(q^3-1)(q^3-2)(q^3-3)(q^3-4)(q^3-5)$, a contradiction for $q\geq 3$.


%By Corollary \ref{Crl_ConstructionMethod}, we need to pick $7$ points in $\pg{1}{q^3}$ such that no linear set of rank at most $3$ contains all these $7$ points. Embed the line $L=\pg{1}{q^3}$ in $\pg{2}{q^3}$ and select one point $P_\infty$ on $L$. Let $\ell_\infty$ be a line of $\pg{2}{q^3}$ through $P_\infty$, different from $L$ and consider the ABB-representation of $\pg{2}{q^3}$ with $\ell_\infty$ as line at infinity. Then the set of points $\phi(P)$, where $P\neq P_\infty$ is a point of $L$ forms a $3$-dimensional subspace $\Pi$. In $\Pi$, we pick an affine point $P$, and two sets of affine points $\{Q_1,R_1\}$ and $\{Q_2,R_2\}$ collinear with $P$, where the lines $Q_1R_1$ and $Q_2R_2$ are distinct. Let $\pi=\mathcal{F}(P_\infty)$ and let $S_i=\pi\cap Q_i$, $i=1,2$.
%%We pick
%%    In this space, choose five coplanar affine points as follows: one arbitrary affine point $P$ and two sets of affine points $Q_1,Q_2$ and $R_1,R_2$ such that $\vspan{P,Q_1,Q_2}$ and $\vspan{P,R_1,R_2}$ are two distinct lines.
%%    
%
%    Now, consider the ABB-representations of all linear sets of rank at most $3$ that contain the six chosen points.
%    Such a linear set cannot be of rank smaller than three, or else the points $P$, $Q_1$, $Q_2$, $R_1$ and $R_2$ would be collinear. Hence, the linear set has rank $3$ and is either a club or a scattered linear set.
%    If the club has head $P_\infty$, the ABB-representation is the affine plane containing $P$, $Q_1,Q_2,R_1,R_2$, which also implies that there is a unique such club. If the club has head $H\neq P_\infty$, by Theorem \ref{}, its ABB-representation is a cone with vertex $\phi(H)$ and base a conic $\mathcal{C}$ in $\pi$ whose extension to $\FF_{q^t}$ contains the conjugate points $R,R^\sigma,R^\sigma$ lying on the indicator set of $\spread$. Since $Q_1Q_2$ and $R_1R_2$ are two concurrent lines, contained in this cone, it follows that $P$ is the head of this club and that there is a unique club with head, different from $P_\infty$ containing those 6 points.
%    
%    
%    In the case of a scattered linear set, by Theorem \ref{}, its ABB-representation is a hyperbolic quadric meeting $\pi$ in $\mathcal{C}$. Since there is a unique hyperbolic quadric containing a fixed conic $\mathcal{C}$ and two intersecting lines, there is a unique scattered linear set containing the six points. 
%
%    
%   % Moreover, as such linear set of rank $3$ contains the point at infinity, its affine part in $\pg{3}{q}$ is either an affine plane, a cone or a hyperbolic quadric, of which the latter two intersect the plane at infinity in a conic of the `bundle of conics'. (\textcolor{red}{ref naar relevante resultaten})
%    Note that there must exist precisely three linear sets of rank $3$ containing the six points: one affine plane that is uniquely defined by five coplanar affine points, one cone and one hyperbolic quadric.
%    After all, such cone or hyperbolic quadric necessarily contains the concurrent affine lines $\vspan{P,Q_1,Q_2}$ and $\vspan{P,R_1,R_2}$, hence two points at infinity are fixed.
%    As any cone of the `bundle of conics' is uniquely defined by two points, the base at infinity of this cone/hyperbolic quadric is fixed (\textcolor{red}{yet to prove that two fixed concurrent lines + non-singular conic uniquely defines a hyperbolic quadric}).
%    
%    The affine plane contains $q^2$ affine points of $\pg{3}{q}$.
%    The cone contains $(q-1)^2$ extra affine points.
%    The hyperbolic quadric contains $(q+1)^2-(q+1)-(2q-1)=q^2-q+1$ extra affine points.
%    In conclusion, the union of the plane, cone and hyperbolic quadric cover precisely
%    \[
%        q^2+(q-1)^2+q^2-q+1=3q^2-3q+2
%    \]
%    affine points of $\pg{3}{q}$, which is less than $q^3$ for every $q>2$.
%    This implies that we can take a seventh point not contained in any of these three linear sets.
\end{proof}

We will now use the results of this paper to explicitely construct a set of $7$ planes in $\pg{5}{q}$ in higgledy-piggledy arrangement. We start by writing down explicit equations of the set of conics in $\pg{2}{q}$ containing $3$ fixed conjugate points.

\begin{lm}\label{juistevorm} Let $\omega\in \FF_{q^3}\setminus\FF_q$ be a generator of $(\FF_{q^3}^*,.)$ satisfying $\omega^3+\lambda_1\omega^2+\lambda_2\omega+\lambda_3=0$. Then the conics in $\pg{2}{q}$ whose extension to $\FF_{q^3}$ contains the points $(1,\omega,\omega^2)$, $(1,\omega^q,\omega^{2q})$, $(1,\omega^{q^2},\omega^{2q^2})$ are given by
\begin{align}
g_{d,e,f}(X_0,X_1,X_2):=(\lambda_3e-\lambda_1\lambda_3f)X_0^2+(\lambda_2e+(\lambda_3-\lambda_1\lambda_2)f)X_0X_1&+\nonumber\\(\lambda_1e+(\lambda_2-\lambda_{1}^2)f-d)X_0X_2+dX_1^2+eX_1X_2+fX_2^2&=0,\label{specialconics}\end{align}
with $d,e,f\in \FF_q$ not all zero. 
%That is,
%\begin{itemize}
%\item[(a)] $f_{d,e}:=(\lambda_3e-\lambda_1\lambda_3)X_0^2+(\lambda_2e+\lambda_3-\lambda_1\lambda_2)X_0X_1+(\lambda_1e+\lambda_2-\lambda_{1}^2-d)X_0X_2+dX_1^2+eX_1X_2+X_2^2=0,$ where $d,e\in \FF_q$;
%\item[(b)] $f_d:=\lambda_3X_0^2+\lambda_2X_0X_1+(\lambda_1-d)X_0X_2+X_1X_1=0$, $d\in \FF_q$;
%\item[(c)] $f_\infty:=X_0X_2-X_1^2=0$.
%\end{itemize}
\end{lm}
\begin{proof} An arbitrary conic $\mathcal{C}$ in $\pg{2}{q}$ has equation $aX_0^2+bX_0X_2+cX_0X_2+dX_1^2+eX_1X_2+fX_2^2=0$ where $a,b,c,d,e,f\in \FF_q$. Note that if $(1,\omega,\omega^2)$ lies on the extension of $\mathcal{C}$ to $\pg{2}{q^3}$, then $(1,\omega^q,\omega^{2q})$ and $(1,\omega^{q^2},\omega^{2q^2})$ also lie on this extension. Expressing that $(1,\omega,\omega^2)$ lies on $\mathcal{C}$, using that $\omega^4=(\lambda_1^2-\lambda_2)\omega^2+(\lambda_1\lambda_2-\lambda_3)\omega+\lambda_1\lambda_3$, and that $1,\omega,\omega^2$ are $\FF_q$-independent, we find the following system of equations:
\begin{align*} a-\lambda_3e+\lambda_1\lambda_3f&=0\\
b-\lambda_2e+(\lambda_1\lambda_2-\lambda_3)f&=0\\
c+d-\lambda_1e+(\lambda_1^2-\lambda_2)f&=0.\qedhere
\end{align*}
%If we let $f=1$, we find the $q^2$ conics $f_{d,e}$ of the form (a), if $f=0,e=1$, we find the $q$ conics $f_d$ of the form  $(b)$, and if $f=e=0$, $d=1$, we find the conic $f_\infty$ given by (c).
\end{proof}



%Consider a conic $\mathcal{C}$ with equation $f(X_0,X_1,X_2)=0$ in the plane $\pi$ and embed $\pi$ as a hyperplane $X_3=0$ in $\pg{3}{q}$. Then all quadrics intersecting $\pi$ in $\mathcal{C}$ are given by $f(X_0,X_1,X_2)+X_3(uX_0+vX_1+wX_2+tX_3)=0$ for some $u,v,w,t\in \FF_q$.


\begin{prop} \label{construction} Let $P_i(x_0^{(i)},x_1^{(i)},x_2^{(i)},1)$, $i=1,\ldots, 6$ be six non-coplanar points contained in a non-degenerate elliptic quadric intersecting the plane $\pi:X_3=0$ in the conic $X_0X_2-X_1^2=0$. Consider the quadrics  \begin{align} \mathcal{Q}(d,e,f,u,v,w,t,X_0,X_1,X_2,X_3):=g_{d,e,f}(X_0,X_1,X_2)+X_3(uX_0+vX_1+wX_2+tX_3)=0. \label{quadrics}\end{align} 
Let $A$ be the $(6\times 7)$-matrix whose $i$-th row $(A)_i$ satisfies $$(A)_i[d,e,f,u,v,w,t]^T=\mathcal{Q}(d,e,f,u,v,w,t,x_0^{(i)},x_1^{(i)},x_2^{(i)},1).$$ If $rk(A)=6$, then the points $P_1,\ldots,P_6$, together with $P_\infty$, are the ABB-representation of a set of seven points in $\pg{1}{q^3}$ such that, under field reduction, these seven points form a higgledy-piggledy set of $7$ planes in $\pg{5}{q}$. That is, $\{\mathcal{F}(\phi^{-1}(P_i))\mid 1\leq i\leq 6\}\cup \mathcal{F}(P_\infty)$ is a set of seven planes in $\pg{5}{q}$ in higgledy-piggledy arrangement.
\end{prop}

\begin{proof} By Corollary \ref{Crl_ConstructionMethod}, it is sufficient to construct a set of $7$ points in $\pg{1}{q^3}$ such that no linear set of rank at most $3$ contains all these $7$ points. Embed the line $L=\pg{1}{q^3}$ in $\pg{2}{q^3}$ and select one point $P_\infty$ on $L$. Let $\ell_\infty$ be a line of $\pg{2}{q^3}$ through $P_\infty$, different from $L$ and consider the ABB-representation of $\pg{2}{q^3}$ with $\ell_\infty$ as line at infinity. Then the set of points $\mathcal{F}(P)$, with $P$ a point of $L$ different from $P_\infty$, defines a $3$-dimensional subspace $\Pi$. We coordinatise in such way that the points in $\Pi$ have coordinates $(x_0,x_1,x_2,x_3)$ such that the points with $x_3=0$ are the points in the plane $\pi=\mathcal{F}(P_\infty)$ and the three conjugate points defining $\pi$ are $(1,\omega,\omega^2)$, $(1,\omega^q,\omega^{2q})$, $(1,\omega^{q^2},\omega^{2q^2})$.
In view of Proposition \ref{trivial}, Theorem \ref{thmclub}, and Theorem \ref{thmscattered}, we need to find six affine points of $\Pi$ such that these are not contained in a plane, nor a cone with vertex not in $\pi$ and base a conic whose extension contains the $3$ conjugate points, nor a hyperbolic quadric through such a conic. All (possibly degenerate) quadrics meeting in a conic of the form \eqref{specialconics} are given by an equation of the form \begin{align} f_{d,e,f}(X_0,X_1,X_2)+X_3(uX_0+vX_1+wX_2+tX_3)=0. \label{quadricsform}\end{align}
So if we pick six points, contained in an elliptic quadric $\mathcal{E}$ meeting $\pi$ in the conic  $X_0X_2-X_1^2=0$, we simply need to show that $\mathcal{E}$ is the only quadric with equation of the form \eqref{quadricsform} through those $6$ points. This happens if and only if the homogeneous system of $6$ equations in the variables $d,e,f,u,v,w,t$ that arises from substituting the coordinates of the six points has a unique solution up to scalar multiple, which happens if and only if its coefficient matrix $A$ has $rk(A)=6$.
\end{proof}

\bigskip
In order to give an explicit construction of six such points and make the computations easier, we will restrict ourselves to those values of $q$ such that there is a primitive cubic polynomial of a particular form.

\begin{thm} \begin{itemize} \item[(a)] Let $q$ be odd, $q\equiv 1\pmod{3}$. Let $a$ be a non-square in $\FF_q$, where $a\neq \frac{1}{2}$. The six points $(1,0,-a,1)$,$(1,0,-a,-1)$,$(1,1,1-a,1)$,$(1,-1,1-a,1)$,$(1,1,1-a,-1)$, $(1,-1,1-a,-1)$ give rise to a higgledy-piggledy set of $7$ planes in $\pg{5}{q}$.
\item[(b)] Let $q$ be even such that there is an irreducible polynomial of the form $\omega^3+\omega+1=0$. Let $a\in \FF_q$ with $Tr(a)=1$, $a\neq 1$. The six points $(1,0,a,1)$,$(1,1,a,1)$,$(a,0,1,1)$, $(a,1,1,1)$, $(1,a,a^2,1)$, $(a^2,a,1,1)$ give rise to a higgledy-piggledy set of $7$ planes in $\pg{5}{q}$.\end{itemize}
\end{thm}
\begin{proof}\begin{itemize}\item[(a)] Since $q\equiv 1\pmod{3}$, there is an irreducible polynomial of the form $\omega^3+\lambda=0$. Using Lemma \ref{juistevorm}, we find that the quadrics of the form \eqref{quadrics}  become
\begin{align} \lambda eX_0^2+\lambda fX_0X_1-dX_0X_2+dX_1^2+eX_1X_2+fX_2^2+X_3(uX_0+vX_1+wX_2+tX_3)=0.\end{align}
It is easy to check that the given six points are not coplanar. Furthermore, they are contained in 
%The points $P_{s,t}=(t^2,st,s^2-a,t)$, with $s\in \FF_q,t\in \FF_q^*$ are affine points of $\mathcal{E}$. six points $P_{s,t}$ with $s\in \{-1,0,1\}$ and $t\in \{1,-1\}$. 
 the elliptic quadric $\mathcal{E}$ with equation $X_0X_2-X_1^2-aX_3^2=0$, which meets $\pi$ in the conic $X_0X_2-X_1^2=0$.  Substituting the $6$ points into \eqref{quadrics} yields a system $\Xi$ of $6$ homogeneous equations in $d,e,f,u,v,w,t$ whose associated coefficient matrix is given by 
 $$\begin{bmatrix} a &\lambda&a^2&1&0&-a&1\\
 a &\lambda&a^2&-1&0&a&1\\
 a &\lambda+1-a&(1-a)^2+\lambda&1&1&1-a&1\\
  a &\lambda+a-1&(1-a)^2-\lambda&1&-1&1-a&1\\
   a &\lambda+1-a&(1-a)^2+\lambda&-1&-1&a-1&1\\
      a &\lambda+a-1&(1-a)^2-\lambda&-1&1&a-1&1
 \end{bmatrix}$$
 
 It can be checked that this matrix has full rank if and only if $a(1-a)(2a-1)\neq 0$. The statement follows from Proposition \ref{construction}.

 
\item[(b)] Now assume that $q$ is even and $\omega^3=\omega+1$. 
Using Lemma \ref{juistevorm}, we find that the equation for the quadrics \eqref{quadrics} now becomes
\begin{align} eX_0^2+(e+ f)X_0X_1+(d+f)X_0X_2+dX_1^2+eX_1X_2+fX_2^2\\ +X_3(uX_0+vX_1+wX_2+tX_3)=0.\label{qu2}\end{align}
The six given points are contained in the elliptic quadric $\mathcal{E}$ with equation $X_0X_2+X_1^2+X_1X_3+aX_3^2=0$, which meets $\pi$ in $X_0X_2+X_1^2=0$.
Again, these points are not coplanar, and expressing that those six points lie on an equation of the form \eqref{qu2} yields a system $\Xi$ in $d,e,f,u,v,w,t$ with coefficient matrix

$$\begin{bmatrix} a&1&a+a^2&1&0&a&1\\
1+a&a&1+a+a^2&1&1&a&1\\
a&a^2&a+1&a&0&1&1\\
1+a&a^2+a+1&1&a&1&1&1\\
0&1+a+a^3&a+a^2+a^4&1&a&a^2&1\\
0&a^4+a^3+a&a^3+a^2+1&a^2&a&1&1\end{bmatrix}$$
This matrix has full rank if and only if $a(1+a)\neq 0$. Hence, since $a\neq 0,1$, the statement follows from Proposition \ref{construction}.\qedhere
\end{itemize}
\end{proof}
%\textbf{Acknowledgements.}
%TODO

\bibliographystyle{plain}
\bibliography{main.bib}

Lins Denaux \& Jozefien D'haeseleer

Ghent University\\
Department of Mathematics: Analysis, Logic and Discrete Mathematics

Krijgslaan $281$ -- Building S$8$

$9000$ Ghent

BELGIUM

\texttt{e-mail: lins.denaux@ugent.be}\\
\texttt{e-mail: jozefien.dhaeseleer@ugent.be}

\texttt{website: }\url{https://users.ugent.be/~ldnaux}\\
\texttt{website: }\url{https://users.ugent.be/~jmdhaese}

\bigskip
Geertrui Van de Voorde

University of Canterbury
(Te Whare W$\overline{\mbox{a}}$nanga o Waitaha)\\
School of Mathematics and Statistics

Private Bag $4800$ -- Erskine Building

Christchurch $8140$

NEW ZEALAND

\texttt{e-mail: geertrui.vandevoorde@canterbury.ac.nz}

%\texttt{website: }\url{https://www.canterbury.ac.nz/engineering/contact-us/people/geertrui-van-de-voorde.html}

\end{document}