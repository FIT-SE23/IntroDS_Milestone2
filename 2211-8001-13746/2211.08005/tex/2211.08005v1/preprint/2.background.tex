\section{Related Work}
\label{sec:background}


\textbf{Distinguishing reality. }
\citet{kishino} elicited the reality–virtuality continuum, 
enabling the interpolation of elements of reality and elements of virtuality. 
We refer a \textit{reality} to the environment in which the user manifests, perceives, and interacts with objects.
\textit{Objects} are manifestations of information.
The \textit{view} refers to the user's point-of-view that allows them to perceive the reality at any given moment (e.g. first-person-view). 
An \textit{interface} is a tool that enables the user to interact with their reality (e.g. device screen, head-mounted display).
The virtual reality has been primarily used as a source of virtual objects, with the intention that these objects can be overlayed onto the physical reality at varying levels.
The digital reality, contrarily, is not a specially-designed reality for objects to be overlayed onto a physical reality (e.g. the digital reality is not geo-spatially organized). 
Given the level of immersion a human has in both physical and digital realities, 
unlike prior work in AR/MR/VR,
we aim to enable users to manipulate both their physical and digital reality, extensibly treating them as a single reality. 



\textbf{Manipulating physical reality. }
Rather than immersing in complete reality or virtuality, 
we blend between the two along the continuum.
Two general approaches to the manipulation of real/virtual objects in the physical reality are augmented reality and diminished reality \citep{mann1994}.
Augmented reality adds virtual information onto a reality.
NaviCam \citep{10.1145/215585.215639} was early work demonstrating the placement of messages on video screens.
VRCeption \citep{10.1145/3491102.3501821} enables users to dynamically transition along the reality-virtually continuum.
ScalAR \citep{10.1145/3491102.3517665} enables users to author virtual objects to be placed in their physical reality.
Diminished reality, on the other hand, removes real objects from reality. 
It erases physical objects through 
inpainting \citep{10.1109/ISMAR.2012.6402551},
approximation \citep{10.1145/3126594.3126601},
or multiple cameras \citep{10.1145/3173574.3173703}.
Software from digital interfaces can be ported to be used alongside AR/MR/VR systems (e.g. \citet{horizon}).
While this enables the usage of the software itself,
each reality is still compartmentalized.
For example, what one does on their desktop browsing sessions would play no effect on their physical world viewing experience. 



\textbf{Manipulating digital reality. }
The main challenges faced by frameworks that modify digital interfaces are:
interoperability between programs (apps, browsers) and OS, 
requiring escalation of privilege,
significant development/maintenance effort of modifications (e.g. patches break with version changes).
\textit{Code modifications}
make changes to source code,
either installation code to modify software before installation,
or run-time code to modify software during usage
(e.g. browser extensions for desktop/mobile, Cydia Substrate~\citep{cydia} for iOS, Xposed Framework~\citep{xp} for Android).
\textit{External modifications}
require installing a program that affects other programs (e.g. usage tracking with HabitLab~\citep{kovacs_thesis}).
\textit{Overlay modifications} render graphics on an overlay layer over an active interface instance
(e.g. occluding inappropriate text/images with models~\citep{greaseterminator, greasevision}). 




















    






























