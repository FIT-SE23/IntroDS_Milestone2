\newpage
\onecolumn
\appendix
\section{Examples of Minimal Prompt}
Here we provide Table~\ref{tab:task_formulate_example} for some examples of how to construct minimal prompted data according to \S~\ref{sec:unifydiscriminativetasks}.




\begin{table*}[htbp]
% \setlength{\tabcolsep}{0.6mm}
  \centering
    \resizebox{1.0\textwidth}{!}{\begin{tabular}{c|c|p{37em}|c}
    \toprule
    \multirow{2}[2]{*}{\textbf{Category}} & \multirow{2}[2]{*}{\textbf{Task Type}} & \multicolumn{1}{c|}{\multirow{2}[2]{*}{\textbf{Our Minmal Prompt}}} & \multirow{2}[2]{*}{\textbf{Label}} \\
          &       & \multicolumn{1}{c|}{} &  \\
    \midrule
    \multirow{4}[8]{*}{yes/no} & \multirow{2}[4]{*}{\shortstack{Paraphrase \\ Identification}} & John is Lily's husband. Lily is John's wife & 1 \\
\cmidrule{3-4}          &       & John is Lily's husband. Lily is John's mother. & 0 \\
\cmidrule{2-4}          & \multicolumn{1}{c|}{\multirow{4}[4]{*}{\shortstack{Natural \\ Language  \\ Inference}}} & \underline{Premise:} Dana Reeve, the widow of the actor Christopher Reeve, has died of lung cancer at age 44. \underline{Hypothesis:} Dana Reeve had an accident. & 1 \\
\cmidrule{3-4}          &       & \underline{Premise:} Dana Reeve, the widow of the actor Christopher Reeve, has died of lung cancer at age 44. \underline{Hypothesis:} Christopher Reeve had an accident. & 0 \\
    \midrule
    \multirow{10}[20]{*}{multi-choice} & \multirow{2}[4]{*}{\shortstack{Coreference \\ Resolution}} & Jane gives Joan candy because Joan was hungry. & 1 \\
\cmidrule{3-4}          &       & Jane gives Joan candy because Jane was hungry. & 0 \\
\cmidrule{2-4}          
& \multirow{2}[4]{*}{\shortstack{Question \\ Answer}} & The earth moves around the sun. What is the earch to the sun? Planet & 1 \\
\cmidrule{3-4}          &       & The earth moves around the sun. What is the earch to the sun? Satellite & 0 \\
\cmidrule{2-4}          & \multirow{2}[4]{*}{\shortstack{Topic \\ Classification}} & Open Source Apps Developer SugarCRM Releases Sugar.Sales 1.1. Science and technology & 1 \\
\cmidrule{3-4}          &       & Open Source Apps Developer SugarCRM Releases Sugar.Sales 1.1. Sports & 0 \\
\cmidrule{2-4}          & \multirow{2}[4]{*}{\shortstack{Sentence \\ Completion}} & A boy is running down a track. The boy lifts his body above the height of a pole. & 1 \\
\cmidrule{3-4}          &       & A boy is running down a track. The boy stands on his hands and springs. & 0 \\
\cmidrule{2-4}          & \multirow{2}[4]{*}{\shortstack{Sentiment \\ Classification}} & I really love this movie. Positive & 1 \\
\cmidrule{3-4}          &       & I don't like this movie. Negative & 1 \\
\bottomrule
    \end{tabular}}%
\caption{Examples of how we unify discriminative tasks. The underlined text represents additional words not present in raw inputs. Note that this is just our implementation of the UD formulation and there can be other ways of task formulation under the UD framework. Some tasks can either be yes/no tasks or multi-choice tasks, depending on how options are provided.}
\label{tab:task_formulate_example}
\end{table*}%





\section{Full Experiment Results}




\subsection{Evaluation on Big-Bench}\label{sec:bigbench}
Here we report the full results for 13 tasks in the Big-Bench \citet{bigbench}, which is also utilized in original T0 paper~\cite{T0-paper}. All the tasks from BIG-Bench are ensured unseen in our training set for the zero-shot setting. The results are displayed in Table ~\ref{tab:bigbench}, where UD outperforms T0 by 4-8 points on different scales.




\begin{table*}
\resizebox{\textwidth}{!}{%
    \begin{tabular}{l|ccccccccccccc|c}
    \toprule[1pt]
    \multirow{2}{*}{Model} 
        & \multirow{2}{*}{\shortstack{code \\ desc.}}
        & \multirow{2}{*}{\shortstack{conce\\-ptual}}
        & \multirow{2}{*}{\shortstack{known\\unknowns}}
        & \multirow{2}{*}{\shortstack{logic \\ grid}}
        & \multirow{2}{*}{\shortstack{logic \\ deduction}}
        & \multirow{2}{*}{\shortstack{miscon\\-ceptions}}
        & \multirow{2}{*}{\shortstack{novel\\concepts}}
        & \multirow{2}{*}{\shortstack{strate\\-gyqa}}
        & \multirow{2}{*}{\shortstack{wino\\-why}}
        & \multirow{2}{*}{\shortstack{syllo\\-gisms}}
        & \multirow{2}{*}{\shortstack{movie\\dialog}}
        & \multirow{2}{*}{\shortstack{lang\\-uage\_id}}
        & \multirow{2}{*}{\shortstack{vita\\-minc}} 
        & \multirow{2}{*}{Avg.}\\
    &&&&&&&&&&&&&&\\
    \midrule[1pt]
    UD-DeBERTaV3 & 76.7 & 64.1 & 76.1 & 39.9 & 54.9 & 50.2 & 50.0 & 59.9 & 45.8 & 50.4 & 57.7 & 13.3 & 61.5 & 53.9 \\
    \midrule[1pt]
    T0-Large$\star$ & 14.1 & 40.4 & \textbf{60.4} & \textbf{38.0} & \textbf{41.2} & 50.0 & 10.0 & \textbf{52.3} & \textbf{49.7} & 50.3 & 46.8 & 16.0 & 46.2 & 39.6 \\
    UD-Large & \textbf{51.7} & \textbf{54.4} & 47.8 & 33.4 & 34.6 & \textbf{50.2} & \textbf{26.5} & 47.0 & 45.7 & \textbf{50.6} & \textbf{51.7} & \textbf{16.3} & \textbf{55.8} & \textbf{43.5} \\
    
    \midrule[1pt]
    T0-XL$\star$ & 23.4 & 48.1 & 64.6 & \textbf{42.5} & \textbf{50.1} & \textbf{52.7} & 25.0 & 53.1 & \textbf{45.4} & 50.2 & 47.7 & 19.0 & 60.0 & 44.8 \\
    UD-XL & \textbf{53.3} & \textbf{73.8} & \textbf{65.2} & 37.2 & 37.8 & 48.0 & \textbf{35.3} & \textbf{53.1} & 45.3 & \textbf{50.4} & \textbf{50.1} & \textbf{22.9} & \textbf{63.7} & \textbf{48.9} \\
    \midrule[1pt]
    T0-XXL$\dagger$ & 36.7 & 62.5 & 63.0 & 39.6 & 55.4 & \textbf{52.5} & 15.6 & 52.7 & 47.4 & \textbf{51.8} & 53.8 & 20.7 & 64.7 & 47.4\\
    UD-XXL & 61.7 & 71.8 & 76.1 & 38.0 & 59.1 & 49.3 & \textbf{61.8} & 61.3 & 45.9 & 50.1 & 57.3 & 21.6 & 67.2 & 55.5 \\
    UD+-XXL & \textbf{63.3} & \textbf{82.5} & \textbf{84.8} & \textbf{39.2} & \textbf{67.5} & 49.3 & 58.8 & \textbf{64.2} & \textbf{47.5} & 50.4 & \textbf{57.9} & \textbf{27.3} & \textbf{70.2} & \textbf{58.7} \\
    \bottomrule[1pt]
    
    \end{tabular}%
    }
\caption{Zero-shot performance of Universal Discriminator and T0 on Big-Bench test tasks used in T0 paper. Results with $\dagger$ are reported by~\citeauthor{T0-paper}, and results with $\star$ are reproduced in our framework.}
\label{tab:bigbench}
\end{table*}

\subsection{Evaluation on BBH}\label{sec:bbh}
Here we report the full results for 22 discriminative tasks from BBH \citep{bbh}. For reference, we reproduce Flan-T5\citep{flant5}'s zero-shot performance on BBH tasks by evaluating their public checkpoints. All the tasks from BBH are ensured unseen in our training set for the zero-shot setting. The results are displayed in Table~\ref{tab:bbh_full}, where UD constantly performs better than T0 and Flan-T5 on all the scales even though Flan-T5 is trained on a much broader scope of tasks than UD is.

\begin{table*}
\resizebox{\textwidth}{!}{%

    \begin{tabular}{l|ccc|ccc|cccc}
    \toprule[1pt]
    Dataset & T0-Large & Flan-T5-Large & UD-Large & T0-XL & Flan-T5-XL & UD-XL & T0-XXL & Flan-T5-XXL & UD-XXL & UD+-XXL \\
    \midrule[1pt]
    boolean\_expression  & 48.4 & 49.6 & \textbf{64.0} & 47.6 & 54.8 & \textbf{68.4} & 46.4 & 56.8 & \textbf{68.4} & 66.0 \\
    causal\_judgement & 56.2 & 59.4 & \textbf{61.5} & 58.8 & 59.9 & \textbf{63.6} & 62.0 & 60.9 & \textbf{65.2} & 63.6 \\
    data\_understanding & 30.4 & 18.8 & \textbf{30.4} & 38.8 & 34.8 & \textbf{41.2} & \textbf{63.2} & 56.8 & 51.6 & 53.2  \\
    disambiguation\_qa & 54.4 & 34.8 & \textbf{68.4} & 61.2 & \textbf{66.8} & 65.2 & 64.4 & 66.8 & \textbf{67.2} & 66.8 \\
    formal\_fallacies & 54.4 & \textbf{55.6} & 50.4 & 52.4 & \textbf{54.0} & 46.4 & 52.0 & 55.2 & 54.0 & \textbf{58.8} \\
    geometric\_shapes & 0.0 & \textbf{21.6} & 9.6 & 0.0 & \textbf{20.0} & 9.6 & 11.2 & \textbf{31.2} & 9.6 & 9.6 \\
    hyperbaton & \textbf{72.0} & 59.6 & 71.2 & 52.4 & 58.8 & \textbf{66.8} & 63.2 & 70.8 & 68.0 & \textbf{82.0}\\
    logical\_deduction\_five\_objects & 34.8 & \textbf{40.0} & 32.8 & 38.8 & \textbf{48.0} & 39.2 & 46.4 & 53.6 & 58.4 & \textbf{65.2} \\
    logical\_deduction\_seven\_objects & 27.6 & \textbf{40.4} & 25.2 & 37.6 & \textbf{52.4} & 32.0 & 50.4 & 60.0 & 56.4 & \textbf{67.2}  \\
    logical\_deduction\_three\_objects & 49.2 & 37.6 & \textbf{60.4} & 62.8 & 64.8 & \textbf{69.2} & 65.6 & 74.4 & 80.8 & \textbf{83.2} \\
    movie\_recommendation & 51.4 & 55.0 & \textbf{60.4} & 55.0 & 47.4 & \textbf{69.6} & 61.0 & 38.5 & 73.2 & \textbf{78.8} \\
    navigate & 58.8 & 56.4 & \textbf{63.6} & 60.4 & 59.2 & 58.4 & 65.6 & 60.8 & 63.2 & 64.8 \\
    penguins\_in\_a\_table & 36.3 & 32.9 & \textbf{36.3} & 34.3 & \textbf{42.5} & 41.1 & 40.4 & 41.1 & 39.7 & \textbf{46.6} \\
    reasoning\_about\_colored\_objects & 39.2 & \textbf{40.4} & 36.4 & 41.6 & 47.2 & \textbf{54.4} & 56.8 & 61.6 & 57.2 & \textbf{63.2}\\
    ruin\_names & 23.0 & 22.6 & \textbf{44.4} & 21.8 & \textbf{33.5} & 24.4 & 17.8 & 34.7 & 35.6 & \textbf{68.8} \\
    snarks & 48.3 & 56.1 & \textbf{74.7} & 45.5 & 55.6 & \textbf{73.0} & 55.1 & 72.5 & 75.3 & \textbf{82.0} \\
    sports\_understanding & 53.2 & \textbf{55.6} & 54.8 & 47.6 & \textbf{52.4} & 51.6 & 52.8 & \textbf{60.0} & 57.6 & 56.0 \\
    temporal\_sequences & 13.2 & \textbf{25.2} & 23.6 & 24.8 & 22.4 & \textbf{63.2} & 14.8 & 28.8 & 43.2 & \textbf{60.8} \\
    tracking\_shuffled\_objects\_five\_objects & \textbf{12.8} & 12.4 & 12.0 & 12.8 & 12.0 & \textbf{13.2} & 12.0 & 15.2 & 12.4 & \textbf{20.0}  \\
    tracking\_shuffled\_objects\_seven\_objects & 7.6 & 8.4 & \textbf{9.6} & 8.8 & \textbf{9.2} & 8.4 & 8.0 & 13.2 & 8.4 & \textbf{14.0} \\
    tracking\_shuffled\_objects\_three\_objects & 33.2 & \textbf{33.6} & 31.2 & 33.6 & 32.8 & \textbf{34.8} & 29.6 & 24.4 & \textbf{33.6} & 20.8  \\
    web\_of\_lies & 51.2 & \textbf{52.4} & 51.2 & 51.2 & \textbf{52.4} & 47.6 & 50.8 & 50.0 & 50.4 & \textbf{56.8} \\
    \midrule[1pt]
    Avg.  & 38.9 & 39.5 & \textbf{44.2} & 40.4 & 44.6 & \textbf{47.3} & 45.0 & 49.4 & 51.3 & \textbf{56.7} \\
    \bottomrule[1pt]
    \end{tabular}%
}
  \caption{Zero-shot performance of Universal Discriminator, T0, and Flan-T5 on BBH test tasks \citep{bbh}.}
  \label{tab:bbh_full}%
\end{table*}%



\section{More Ablation Studies}

\subsection{Ablation on Base Models}\label{sec:base_models}
\begin{table*}[ht]
\setlength{\tabcolsep}{0.9mm}
\centering
\resizebox{\textwidth}{!}{%
    \begin{tabular}{l|l|ccccc|ccc|cc|c|c}
        \toprule[1pt]
        & \multirow{2}*{Base Model}
        & \multicolumn{5}{c|}{\textbf{Natural Language Inference}} & \multicolumn{3}{|c|}{\textbf{Sentence Completion}} & \multicolumn{2}{c|}{\textbf{Coreference}} & \multicolumn{1}{c|}{\textbf{WSD}} & \multirow{2}{*}{Avg.} \\
    & & RTE & CB & ANLI1 & ANLI2 & ANLI3 & COPA & Hella. & Story. & WSC & Wino. & WiC &  \\
    \midrule[1pt]
    \multirow{2}*{\shortstack{Encoder}}
    & DeBERTa-V3 (304M) 
        & 71.1
        & 76.8
        & 43.8
        & 41.3
        & 45.7
        & 96.0
        & 60.7
        & 97.4
        & 66.4
        & 83.6
        & 53.3
        & 66.9 \\
    & DeBERTa-V2 (1.5B) 
        & 77.6
        & 80.4
        & 43.2
        & 39.3
        & 44.8
        & 95.0
        & 67.2
        & 98.2
        & 74.0	& 82.1 & 56.0	& 68.9\\ \midrule
    \multirow{2}*{\shortstack{Enc-Dec}} & T5-Encoder (400M) 
        & 75.1	& 55.5	& 32.9	& 32.3	& 33.7	& 84.6	& 28.2	& 94.0	& 63.0	& 54.6	& 51.2	& 55.0 \\
    & T5-Encoder (1.5B)  & 79.7	& 68.9	& 43.1	& 38.5	& 42.3	& 94.1	& 31.5	& 97.5	& 68.8	& 61.3	& 54.1	& 61.8\\
    \midrule
    \multirow{1}*{\shortstack{Decoder}}
    & \multirow{1}*{GPT-XL (1.5B)}
        & \multirow{1}*{71.1}
        & \multirow{1}*{75.0}
        & \multirow{1}*{30.4}
        & \multirow{1}*{31.8}
        & \multirow{1}*{37.8}
        & \multirow{1}*{71.0}
        & \multirow{1}*{40.9}
        & \multirow{1}*{87.7}
        & \multirow{1}*{62.5}
        & \multirow{1}*{54.5}
        & \multirow{1}*{50.3}
        & \multirow{1}*{55.7}
    \\
    \bottomrule[1pt]
    \end{tabular}}
    \caption{Ablation study on different backbone models. We experiment with base models of different architectures and scales. ``Enc-Dec'' refers to models that are pretrained in an encoder-decoder manner.}
    \label{tab:ablationbasemodel}
\end{table*}

We also study the effects of using different backbone pretrained models. We experiment with three backbone models of different types, respectively the encoder part of an encoder-decoder model, an encoder model, and a decoder model. Specifically, we use the T5 encoder, DeBERTa \cite{debertav3}, and GPT \cite{radford2018gpt} respectively for these three types. It is noteworthy that though similar in architecture for both T5 encoder and DeBERTa, they are pretrained with different self-supervised language modeling tasks, which in fact leads to huge differences in zero-shot generalization, as we will show in Table~\ref{tab:ablationbasemodel}.











Results of different backbone models are presented in Table \ref{tab:ablationbasemodel}. 
Among all three types of backbone models, the encoder backbone models appear to be the most suitable type of backbone, where both encoder models of two scales respectively achieve the best and the second best results, outperforming all the others by more than 5 points.

Using the same number of parameters (i.e., 1.5B), both DeBERTa-V2 and T5-Encoder significantly outperform GPT-XL, which demonstrates that a bidirectional architecture works better than the unidirectional architecture for the discriminator formulation.
In addition, DeBERTa-V2 outperforms T5-Encoder by 7 points, implying that not only model architecture but also the self-supervised pretraining task determines the ability of UD discrimination. Models pretrained with masked language modeling tasks are more suitable for UD.

The impacts of the architecture and pretraining tasks of backbone models are even larger than the influence of scale, as we also observe that an encoder model with 300M parameters (i.e., DeBERTaV3) achieves much better performance than the T5 encoder and GPT-XL with 1.5B parameters.
















