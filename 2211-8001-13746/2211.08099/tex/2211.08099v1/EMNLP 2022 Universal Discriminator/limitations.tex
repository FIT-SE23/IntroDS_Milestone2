%\section{Limitations}
\begin{comment}
% UD:
% 1. 只能做discrim tasks,包括clf,multiple choiceqa, paraphrase identification, 尽管cover了绝大部分tasks
% 2. 一定程度上deprompt ,但不是no prompt,没有完全消除人工的依赖
Even though our proposed framework can unify a vast majority of tasks, our framework cannot handle generation tasks. Specifically, training on a large number of unified discriminative tasks can promote zero-shot task generalization on tasks like natural language inference and multi-choice sentence completion, but cannot handle generation tasks (e.g., summarization) directly, which is limited by the model architecture in nature. 
% This is mainly because we require the dataset to have choices or answer candidates so that our framework can figure out which one is the best.
However, as shown in Section \ref{sec:generalize}, UD demonstrates strong generalization and might be extended to handle generation tasks in future work.
% Second, although we relieve most efforts in designing various kinds of prompts and searching decent templates for each task, designing minimal prompt to properly concatenate input keywords and answer choices is still necessary which requires extra human efforts.
\end{comment}


