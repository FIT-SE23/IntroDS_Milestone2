% This must be in the first 5 lines to tell arXiv to use pdfLaTeX, which is strongly recommended.
\pdfoutput=1
% In particular, the hyperref package requires pdfLaTeX in order to break URLs across lines.

\documentclass[11pt]{article}

% Remove the "review" option to generate the final version.
% \usepackage[review]{emnlp2022}
\usepackage{emnlp2022}

% Standard package includes
\usepackage{times}
\usepackage{latexsym}

% For proper rendering and hyphenation of words containing Latin characters (including in bib files)
\usepackage[T1]{fontenc}
% For Vietnamese characters
% \usepackage[T5]{fontenc}
% See https://www.latex-project.org/help/documentation/encguide.pdf for other character sets

% This assumes your files are encoded as UTF8
\usepackage[utf8]{inputenc}

% This is not strictly necessary, and may be commented out,
% but it will improve the layout of the manuscript,
% and will typically save some space.
\usepackage{microtype}

% This is also not strictly necessary, and may be commented out.
% However, it will improve the aesthetics of text in
% the typewriter font.
\usepackage{inconsolata}

% If the title and author information does not fit in the area allocated, uncomment the following
%
%\setlength\titlebox{<dim>}
%
% and set <dim> to something 5cm or larger.

\usepackage{graphicx}
\usepackage{textcomp}
\usepackage{xcolor}
\usepackage{multirow}
\usepackage{subfigure}
\usepackage{longtable}
\usepackage{booktabs}
\usepackage{multirow}
\usepackage{tabularx}
\usepackage{comment}
\usepackage{hyperref}
\usepackage{amsmath}

\usepackage{ulem}
\usepackage{subfigure}


\newcommand{\std}{\scriptsize$\pm$}

% \newcommand{\solution}[0]{NAME~}
\newcommand{\method}[0]{universal discriminator~}

\newcommand{\mc}[1]{\mathcal{#1}}

\newcommand{\lzy}[1]{\textcolor{green}{lzy: #1}}
\newcommand{\xhk}[1]{\textcolor{orange}{xhk: #1}}
\newcommand\zy[1]{\textbf{ \textcolor{red}{zhilin: #1}}}
\newcommand\yn[1]{{\textcolor{red}{#1}}}
\newcommand\zj[1]{\textcolor{blue}{#1}}


\title{A Universal Discriminator for Zero-Shot Generalization}

% Author information can be set in various styles:
% For several authors from the same institution:
% \author{Author 1 \and ... \and Author n \\
%         Address line \\ ... \\ Address line}
% if the names do not fit well on one line use
%         Author 1 \\ {\bf Author 2} \\ ... \\ {\bf Author n} \\
% For authors from different institutions:
% \author{Author 1 \\ Address line \\  ... \\ Address line
%         \And  ... \And
%         Author n \\ Address line \\ ... \\ Address line}
% To start a seperate ``row'' of authors use \AND, as in
% \author{Author 1 \\ Address line \\  ... \\ Address line
%         \AND
%         Author 2 \\ Address line \\ ... \\ Address line \And
%         Author 3 \\ Address line \\ ... \\ Address line}



\author{Haike Xu$^{1}$, Zongyu Lin$^{1}$, Jing Zhou$^{1}$, Yanan Zheng$^{2}$\footnotemark[1], Zhilin Yang$^{134}$\footnotemark[1] \\
$^1$Institute for Interdisciplinary Information Sciences, Tsinghua University \\
$^2$Department of Computer Science and Technology, Tsinghua University \\
$^3$Shanghai Artificial Intelligence Laboratory, $^4$Shanghai Qi Zhi Institute \\
\texttt{haikexu@mit.edu}, 
\texttt{\{zyanan,zhiliny\}@tsinghua.edu.cn} \\}

\begin{document}
\maketitle
\renewcommand{\thefootnote}{\fnsymbol{footnote}}
\footnotetext[1]{Corresponding authors.}



\begin{abstract}
Generative modeling has been the dominant approach for large-scale pretraining and zero-shot generalization. In this work, we challenge this convention by showing that discriminative approaches perform substantially better than generative ones on a large number of NLP tasks. Technically, we train a single discriminator to predict whether a text sample comes from the true data distribution, similar to GANs. Since many NLP tasks can be formulated as selecting from a few options, we use this discriminator to predict the option with the highest probability. 
This simple formulation achieves state-of-the-art zero-shot results on the T0 benchmark, outperforming T0 by 16.0\%, 7.8\%, and 11.5\% respectively on different scales. In the finetuning setting, our approach also achieves new state-of-the-art results on a wide range of NLP tasks, with only 1/4 parameters of previous methods.
Meanwhile, our approach requires minimal prompting efforts, which largely improves robustness and is essential for real-world applications. Furthermore, we also jointly train a generalized UD in combination with generative tasks, which maintains its advantage on discriminative tasks and simultaneously works on generative tasks.
%\xhk{Furthermore, by extending to an encoder-decoder architecture for generative tasks and restricting the prediction on "yes"/"no" tokens for discriminative tasks, we jointly train a generalized UD which maintains its advantage on discriminative tasks and has simultaneously comparable performance on generative tasks.}
\footnotetext[2]{Our code will be available at \href{https://github.com/Rafa-zy/UD}{https://github.com/Rafa-zy/UD}.
}

% We also achieve new state-of-the-arts on a wide range of supervised NLP tasks, using only 1/4 parameters compared with the previous model.
% In the setting of finetuning, our approach also outperforms generative baselines on a wide range of tasks. 



% This paper proposes a novel universal discriminator for zero-shot generalization, which has substantially improved zero-shot performance by \textbf{\color{red} more than 6\% (even up to 16\%)} across a variety of model scales and test tasks, and has set the new state-of-the-art (SOTA) among previous zero-shot approaches.
% While previous methods unify NLP tasks into generative LM tasks using prompts, we for the first time propose a discriminative task formulation that fuses inputs and targets to construct conforming or contrastive information. It does not rely on handcrafted prompts and is general for all NLP tasks.
% Based on the formulation, we propose the universal discriminator by further pretraining an auto-encoding pretrained model with a multitask dataset.
% Experiments prove that the universal discriminator is not only high-performing on zero-shot tasks, but also achieves SOTA on a \textbf{\color{red} large number of} finetuned tasks across a variety of types with only \textbf{\color{red} less than 1/10} number of parameters.
% Given the high performance, insensitivity and universality on both zero-shot and standard finetuned settings, we humbly believe universal discriminator potentially can serve as an alternative to the previous prompt-based generative paradigm and a strong baseline for future research.
\end{abstract}

\section{Introduction}

Generative modeling has been the dominant approach for large-scale pretraining and zero-shot generalization~\cite{gpt3-paper,artetxe2021efficient,rae2021scaling}. 
Combined with prompts~\cite{gpt3-paper}, most of the natural language processing (NLP) tasks can be formulated into the fill-in-the-blank format and perform generative language modeling.
Based on the unified generative formulation, pretrained models such as GPT-3~\cite{gpt3-paper}, BERT~\cite{devlin2018bert,PET-paper}, T5~\cite{T5-paper}, can perform zero-shot inference on new tasks. 


More recent work~\cite{T0-paper} proposed to further pretrain a generative T5~\cite{T5-paper} with multitask prompted datasets and has substantially enhanced the performance of zero-shot generalization. 
In contrast, methods based on discriminative modeling~\cite{devlin2018bert} have not been able to achieve state-of-the-art performance on zero-shot learning. The adoption of discriminative approaches for zero-shot learning has been limited in the literature.


% Although there are a few works using discriminative modeling to perform zero-shot or few-shot learning, such as CLS finetuning using BERT or prompting using ELECTRA
% For example, BERT was CLS finetuned to perform zero-shot/few-shot learning, however, the zero-shot/few-shot performance are lagged far behind.

% \zy{Add a note: although BERT can be CLS finetuned (which is discriminative), but it is not the SOTA approach for zero-shot and few-shot learning.}

\begin{figure}%[htbp]
     \centering
     \includegraphics[width=1.05\linewidth]{figure/final_sota.png}
     \vspace{-15pt}
     \caption{Average zero-shot performance over 11 zero-shot tasks for our Universal Discriminator and T0~\cite{T0-paper}. Our universal discriminator significantly outperforms T0 across three different scales.}
     \label{fig:sota}
     \vspace{-15pt}
 \end{figure} 


In this work, we challenge the convention of zero-shot learning and propose to study and improve discriminative approaches. This is motivated by the fact that many NLP tasks can be framed as selecting from a few options; e.g., telling whether sentence A entails sentence B, or predicting which answer is correct for a given question. We call these tasks \textit{discriminative tasks}. As we will discuss in later sections, a significant portion of NLP tasks is in fact discriminative tasks. We hypothesize that discriminative approaches perform better for discriminative tasks.
% Despite the recent progress, it remains unknown how discriminative approaches perform in zero-shot generalization. Motivated by the fact that discriminative modeling learns to distinguish among options and goes better with discriminative tasks (e.g., telling whether sentence A entails sentence B, or telling which option correctly answer the question), we hypothesize that discriminative modeling would be better at zero-shot generalization, especially on discriminative tasks.

To verify the hypothesis, we propose the \textbf{universal discriminator (UD)}, which substantially improves zero-shot generalization over the previous generative state-of-the-art (SOTA)~\cite{T0-paper}, as Figure~\ref{fig:sota} shows.
The main idea is to train a single discriminator to predict whether a text sample comes from the true data distribution of natural language, similar to GANs \cite{goodfellow2014generative}. Given a set of training tasks with labeled data, we construct a dataset with positive and negative examples, where positive ones are in-distribution natural language samples and negative ones are out-of-distribution. There are two major types of discriminative tasks. The first type is tasks with multiple options, such as multi-choice question answering and news classification. We fill the options into the sentences and the ones with correct options are considered positive samples. The second type is tasks with yes/no options, which can be formulated as a binary discrimination problem itself. For example, natural language inference aims to predict whether a premise entails a hypothesis. In this case, we use a prompt to concatenate the premise $A$ and the hypothesis $B$ into a sentence ``Premise: $A$. Hypothesis: $B$.'' If entailment holds, this sample is treated as positive in-distribution samples and otherwise negative out-of-distribution ones.



% We define the true data distribution using multiple training tasks with labeled data. Specifically, since discriminative tasks can be formulated as selecting from a few options, samples with correct options form an empirical data distribution, while samples with incorrect options are considered out of distribution. In other words, our discriminator is trained to predict ``true'' for samples with correct options and ``false'' for incorrect ones. We use simple concatenation to minimize prompting efforts. For example, given an example (premise, hypothesis), a natural language inference task predicts whether the premise entails the hypothesis. We concatenate the premise and hypothesis, and assign the label ``true'' for entailment and ``false'' for non-entailment.


% First off, since many of the NLP tasks can be formulated as selecting from several options, we first reformulate the task data into natural text samples by concatenating different fields \zy{what are fields? undefined here. try using another word.}.
% For example, given an example of \zy{the} natural language inference task (\textit{Premise}, \textit{Hypothesis}, \textit{Label}), the natural text is reformulated as ``\textit{\{Premise\} || \{Hypothesis\}}'' labeled with \textit{\{Label\}}. \footnote{Here we use ``||'' to represents direct concatenation.} 
% Another example of topic classification task (\textit{Text}, \textit{Label}) where the \textit{Label} indicates the first option of \{Sports, Fashion, Politics\}, the corresponding natural texts are formulated as ``\textit{Text} || Sports'' labeled with 1, ``\textit{Text} || Fashion'' and ``\textit{Text} || Politics'' both labeled with 0.
% Secondly, we pretrain a pretrained model with reformulated multitask datasets to distinguish whether the text sample comes from the true data distribution. ~\footnote{An assumption is that negative-labeled text samples are artificially constructed thus do not come from the true data distribution, and vice versa.}

For the performance of zero-shot generalization, our approach achieves new state-of-the-art on the T0 benchmark, outperforming T0 by 16.0\%, 7.8\%, and 11.5\% respectively on different scales. 
UD also achieves state-of-the-art performance on a wide range of supervised NLP tasks, using only 1/4 parameters of previous methods.
Compared with the previous generative prompt-based methods, our universal discriminator requires minimal prompting, which is simple, robust, and applicable in real-world scenarios.

% By further scaling the number of tasks, our approach also sets the new state-of-the-art on \textbf{\color{red}[xxx]} tasks with less than 10\% of model parameters \zy{need to give a range} under the setting of standard finetuning.
% In the setting of finetuning, our approach also outperforms the generative baselines consistently across a wide range of tasks.


In addition, we also generalize UD to a larger scope of tasks, such that UD can perform discriminative and generative tasks at the same time. Specifically, we extend UD to the encoder-decoder architecture for training on generative tasks, and restrict the model's prediction on "yes"/"no" tokens for jointly training discriminative tasks. Results prove that generalized UD maintains UD's advantages on discriminative tasks and achieves comparable results on generative tasks (See \S~\ref{sec:generalizedud}). 
% We leave expanding UD to a broader range of generative tasks and achieve greater performance on generative tasks as our future work


% \xhk{I admit the limitation on generative tasks here as our future work.}

%\xhk{Although UD is designed for improving zero-shot performance for discriminative tasks, we can also combine this idea to train a generalized UD model which simultaneously solves both discriminative tasks and generative tasks, maintaining UD's advantage on discriminative tasks and get comparable results on generative tasks (See \S~\ref{sec:generalizedud}).}

% The universal discriminator provides a new perspective for zero-shot generalization---Compared with generating the true verbalizer that indicates task label with extensive prompt engineering, distinguishing between options with minimal prompting efforts is simple, robust, and high-performing, thus is more applicable in real-world scenarios. \zy{rewirte the above sentence, just focus on one point---minimal prompting}

\section{Related Work}

\subsection{Zero-Shot Generalization Using PLMs}
Pretrained language models (PLM) can transfer knowledge from training data to downstream tasks.
Prompting methods further narrow the gap between training data and downstream tasks. \citet{PET-paper} reformulate NLP tasks into cloze filling using prompts so that PLMs can conduct zero-shot inference by generating tokens given prompted inputs. \citet{meng2022generating} use PLMs to generate class-conditioned texts with the guidance of prompts without seeing any task-specific data.
Most recently, researchers have introduced natural language prompts to unify various kinds of tasks and propose a multi-task prompted training framework to achieve great zero-shot performance even faced with unseen downstream tasks (\citet{FLAN,T0-paper}).
However, zero-shot learning has been dominated by generative approaches.
% The success of this new paradigm motivates us to follow and iterate this line of research by proposing a universal discriminator to unify and train all tasks.


\subsection{Prompt-based and Prompt-free Methods in NLP}
Prompting is the method of reformatting NLP tasks using natural language templates to adapt to downstream tasks \cite{T5-paper,PET-paper}.
% With the help of prompts, LMs can directly predict the desired answer without training on task-specific data, making it possible to perform well under few-shot, even zero-shot setting. Also, prompts serve as a decent lubricant for multitask learning which facilitates the development of text-to-text pretrained models such as T5 (\cite{T5-paper}). 
To reduce the instability and labor costs brought by prompting, researchers have tried various approaches (\citet{ptuning-paper,he2021towards}) to learn continuous prompts. 

Recently, prompt-free methods are also being explored. \citet{mahabadi2022prompt} adopts task-specific adapters to learn task descriptions implicitly for few-shot learning with PLMs. 
It has also been indicated that using null prompts without task-specific templates can achieve decent performance compared with manually-designed prompts on various tasks (\citet{logan2021cutting}).

Our work further shows that minimal prompting performs better with our discriminative formulation in the multi-task zero-shot learning setting.

% liberate the time-cosuming manual prompt design lying in multi-task prompted pretraining (\citet{T0-paper,FLAN}) by adopting a universal task formulation and utilizing a universal discriminator to achieve better zero-shot generalization.

\subsection{Discriminative Models in NLP}

% Pretrained language models often adopt masked language modeling (MLM) as the self-supervised learning objective, e.g., BERT, RoBERTA (\citet{devlin2018bert,liu2019roberta}), where the input texts are corrupted by masked tokens and the models are trained to recover the original tokens. They performed well on a wide range of downstream tasks, yet require a large number of training samples and parameters to achieve decent generalizibility.

PLMs trained with masked language modeling (MLM) \cite{devlin2018bert,liu2019roberta} can be finetuned in a discriminative manner for downstream tasks. 
ELECTRA \cite{clark2020electra} trains a discriminator to detect whether a token has been replaced. WKLM \cite{xiong2019pretrained} employs an entity-centric approach for pretraining and predicts whether an entity has been replaced.
However, finetuning for these methods is usually based on one separate CLS head per task, which is not suitable for zero-shot generalization.

% To build a more sample-efficient pretraining framework, ELECTRA (\citet{clark2020electra}) changes the MLM objective to a replaced token detection (RTD) objective, where a generator replace several tokens in the training texts and a discriminator will predict whether each token is replaced or not. WKLM (\citet{xiong2019pretrained}) develops a entity-centric pretraining objective which detects and replaces the entity words with the entities of the same type in each document and train the model to predict whether each entity is replaced.

Recently, prompting has been combined with token-level discriminators based on ELECTRA for few-shot learning \cite{yao2022prompt,xia2022prompting}. While these are also discriminative approaches, there are a few key differences from our approach. First, these methods use ELECTRA to perform token-level discrimination, while we perform sequence-level discrimination, which is more flexible to tasks that have multi-word verbalizers or do not have verbalizers at all. Second, we unify all discriminative tasks into one single task with minimal prompting, which is convenient for zero-shot generalization and removes the need for designing task-specific prompts. Third, these methods are specific to ELECTRA-like pretraining, while our approach accepts arbitrary pretrained encoders. In our experiments, we will also make a direct comparison with these approaches to demonstrate our effectiveness.

% \xhk{Another concurrent work \cite{UniMC} shares a similar idea with our discriminative approach. In specific, given an example of a multiple-choice task, their method concatenates the tokens of input passage, question, and all the options together as their model's input text. Then they use their proposed option masked language modeling (O-MLM) and option prediction (OP) in the training phase. \zy{too many details} Although their method also entails a similar idea of the "discriminative approach", our UD method implements this idea in a completely different way: We are training UD to predict whether a text sample combined with every single option comes from the true data distribution of natural language. Our training method is more concise, more efficient in terms of the token length, and more beneficial for other potential settings \zy{not clear based on context} (see fine-tune Section~\ref{sec:ud_finetune}).} 

% \zj{Another concurrent work \cite{UniMC} shares a similar idea with our discriminative approach. In specific, given an example of a multiple-choice task, they concatenate all inputs and options together and then use option masked language modeling (O-MLM) and option prediction (OP) for training. Although their method also entails a similar idea of the ``discriminative approach'', our UD method implements this idea more concisely and efficiently. We are training UD to predict whether a text sample combined with every single option comes from the true data distribution of natural language. (See fine-tune Section~\ref{sec:ud_finetune} for details)} 

% \xhk{[haike: not sure if it is better to just stop here, or mention empirical performance comparison]}

% \xhk{It is hard to fairly compare our UD's empirical performance with UniMC because UniMC selects a set of multiple-choice training tasks to induce better zero-shot performance, whereas we restrict our UD using exactly the same training datasets in T0. Furthermore, UniMC is not tested on some tasks in the T0 benchmark and it is only implemented using small backbone models.}

% Recently, with the development and popularity of prompt-based approaches, researchers further adapt prompt-based few-shot learning to ELECTRA and achieves better performance on downstream tasks (\citet{yao2022prompt,xia2022prompting}).

% In our work, we borrow the similar idea from the discriminative pretrained language models and adapt to multi-task prompted training paradigm. The major difference is that discriminative models such as ELECTRA focuses on the pretraining stage, while our model puts emphasis on the multi-task finetuning stage and demonstrate that unifying all tasks (e.g., T0) into \xhk{discriminative tasks} classification problem can greatly boost the zero-shot task generalization while reducing the prompt-design efforts and the number of training samples.


%% to be discussed
% \subsection{Generative Adversarial Nets}


\begin{figure*}[t]
     \centering
     \includegraphics[width=0.8\linewidth]{figure/discrim_more.png}
     \vspace{-10pt}
     \caption{The discriminative tasks account for a large proportion among all tasks in the T0 benchmark\citep{T0-paper}. Also all the of test tasks of T0 are discriminative tasks.}
     \label{fig:discrim}
 \end{figure*} 

\input{figs/overview.tex}
%\vspace{-0.4cm}
\section{DeepFlow Overview}\label{sec:overview}

Figure~\ref{fig:overview} shows an overview of the \name framework. \name takes the following set of \textbf{inputs}: 
%
(1) \underline{System} design hierarchy (e.g., the number of accelerator nodes per device, the number of devices in the system, the network topology connecting nodes within a device and across the devices), 
(2) \underline{Architecture template} of each accelerator node which provides a high-level definition of its components and how those components fit together. The purpose of the template is to provide a blueprint for the accelerator without committing to any specific hardware parameters.
%A component definition (e.g., minimal compute units (MCU\footnote{Examples of what we regard as MCU includes SMU in older GPUs, Tensor Cores in newer GPUs or systolic array in TPUs}), memory hierarchy, network), specification of each component (e.g., flop rate for each MCU, MCU dimensions, number of MCUs sharing a set of register files, dataflow execution model, and characteristics and scope of different levels of memory hierarchy), 
(3) \underline{Technology} parameters for each hardware component (e.g. energy per flop), 
(4) \underline{Design budgets} for each hardware component (area, power, perimeter),  
(5) \underline{Machine learning model} specification in the form of a high-level compute graph, parameters of each compute node (kernel type, tensor dimensions), and
(6) \underline{Parallelism strategy} (data, model, kernel, and/or pipeline parallelism dimensions) which distributes the compute graph across the entire system. 
(7) \underline{Device mapping} strategy which defines mapping of parallel shards onto hardware nodes.
Given these inputs, \name predicts the end-to-end performance of one iteration (i.e., single batch) of the model and finds an optimal hardware-software-technology design point as \textbf{output}. 

DeepFlow is composed of two major components.
\underline{CrossFlow} which operates in a stand-alone mode and can predict performance for any input configuration; and a search and optimization engine (\underline{SOE}) which enables design space search. 
%To do so, \name breaks the problem into multiple phases.
%Each phase or building block of \name is described in details next.
\vspace{-0.1cm}
\subsection{CrossFlow Building Blocks}

\paragraph*{\em Micro-Architecture Generator Engine (AGE)}

AGE takes the following set of \textbf{inputs}:
(1) Design constraints (i.e the power, area and perimeter budget and breakdown across micro-architectural components such as cache, network, compute cores). 
This breakdown can be provided manually by users or automatically by the Search and Optimization Engine (SOE, Section~\ref{subsec:soe}).
%We also provide technology specifications such as 
%and their physical characteristics such as area/power per core under nominal operating conditions, SRAM/register characteristics. 
(2) Technology parameters such as energy per flop, energy per data bit transfer for each level of memory and network hierarchy, threshold and maximum gate voltage, integration substrate parameters such as bump/interconnect pitch. We provide a wide range of standard and future technology libraries as baseline. (3) Architecture template which is a blueprint of the underlying accelerator chip without committing to any specific hardware parameters. Given these input, AGE performs a frequency-voltage-area scaling optimization to generate the following \textbf{output} parameters such that design budgets for all component are met: 
(1) Compute throughput.
(2) Capacity for different levels of memory hierarchy.
(3) Bandwidth to each level of memory hierarchy.
(4) Inter-node as well as intra-node network bandwidth. 
These parameters are then utilized by the performance prediction engine (PPE) to estimate the execution time of each kernel.
%As mentioned previously, 
%The output of this stage is the input to performance engine to estimate the execution time of each kernel. Next, we describe the search and optimization engine (SOE) which feeds input values to AGE, if we want to use the model for architecture search.
%\vspace{-0.2cm}
\paragraph*{\em Compute Graph Transformation and Device Placement Engine (DPE)}
The parallelization strategy and device mapping are critical in deciding the overall execution time. Here, we first transform the model graph to a `super-graph' to reflect the parallelization strategy provided by the users manually, or SOE engine (Section~\ref{subsec:soe}) automatically. For example, to apply data parallelism, the model graph is replicated and appropriate edges are added to model the gradient exchange. After generating the transformed graph, DPE assigns the vertices of the transformed graph to the system nodes following a heuristic approach to minimize the communication overhead. %
%The details are presented in section~\ref{}.

%\vspace{-0.2cm}
\paragraph*{\em Performance Prediction Engine (PPE)}
%With the device mapping for all the vertices of the compute (super-)graph known, the next step is to calculate the overall execution time for a forward pass and/or a backward pass. 
We use hierarchical roofline modeling to predict the performance of each compute node. To calculate the overall end-to-end execution time, while respecting scheduling constraints (e.g. one kernel at a time per GPU, or prioritizing one kernel launch over another) we use event-driven simulation.%
%We explain the details of the PPE in section~\ref{}.
\subsection{Search and Optimization Engine (SOE)}\label{subsec:soe}
Co-optimizing micro-architectural parameters and the parallelization strategy that minimizes the overall end-to-end execution time requires navigating a large space of design parameters. 
Search and optimization engine (SOE) enables the automatic design space search and finds an 
%that meets the total power and area constraints, and simultaneously explores software parallelization strategies to find the 
optimal design point which meets the design constraints and minimizes the overall execution time.
%Because the hardware configuration space is very large, the search algorithm we designed 
SOE takes inspiration from ML-assisted search algorithms, in particular gradient decent search with momentum and builds on top of the CrossFlow modeling engine.
%The software parallelization design space is much smaller compared to the hardware design space and therefore we employ an exhaustive grid search. 

%Gradient search is an iterative process. In each step, SOE takes the predicted time from previous iteration as input to re-adjust the following parameter settings: (1) power, area and perimeter breakdown across different architectural components. (2) a parallelization strategy. These parameters will be fed back to CrossFlow to estimate the overall execution time. This process continues until convergence or user-specified number of steps. 
%The details of SOE's search algorithm are elaborated in Section~\ref{}. 
\vspace{-0.2cm}
\subsection{Parallelism Strategy Space}
\label{subsec:par_strategy}
There are a myriad of ways to parallelize a model across a large multi-node system. Exploring the parallelism space and finding the optimal strategy is critical to overall performance and system utilization. DeepFlow explores kernel, data and layer parallelism. It uniquely identifies each parallelism strategy by following notations: $\texttt{RC-\{KP1\}-\{KP2\}-d\{DP\}-p\{LP\}}$ or $\texttt{CR-\{KP1\}-d\{DP\}-p\{LP\}}$ depending on the choice of kernel parallelism.
RC (Row-Column) and CR (Column-Row) refer to different forms of kernel parallelism, i.e. distributed GEMM through inner-product or outer-product implementation.
%\begin{equation*}
%    \texttt{RC: R{KP1\}\_C\{KP2\}\_d\{DP\}\_p\{LP\}}
%\end{equation*}
%Where \texttt{RC} or \texttt{CR} refers to the type of kernel parallelism strategy, i.e. Row-Column or Column-Row,
%\texttt{N} refers to the number of parallel nodes,
\texttt{KP1} and \texttt{KP2} are the parameters of distributed GEMM. 
For Row-Column (\texttt{RC}) or inner-product, \texttt{KP1} and \texttt{KP2} would refer to the number of ways we shard the first matrix across rows and the second matrix across columns.
For Column-Row (\texttt{CR}) or outer-product, we would only need one parameter to specify the parallelization strategy; \texttt{KP1} will refer to the number of ways we cut the first matrix across columns and the second matrix across rows.
\texttt{DP} represents the number of model replicas and data shards assigned to each to exploit data parallelism.
\texttt{LP} is the number of ways we cut layers into stages to exploit pipeline parallelism.

\begin{comment}
\subsection{Modes of Operation}
\name has two modes of operation, standalone performance estimation mode and a architecture search mode.
\paragraph{Standalone Performance (SP) Estimation Mode}
Often ML practitioners or hardware designers want to estimate the performance of a model on a particular system configuration. For example, what is the cost optimal number of accelerators that one should deploy for distributed training? Or what is the estimated performance gain from choosing an accelerator with costlier HBM2E vs HBM2? To enable one to quickly answer such questions and to estimate performance under certain known system configurations, the tool can be run in the SP mode. 

In this mode, the description of the architecture of a scale-out system consisting of multiple accelerators, the architecture of the accelerator hardware themselves and the description of the neural network is taken as input, and fed into CrossFlow, which calculates the execution time of each training step. 

%In this mode, the description of the architecture of a scale-out system consisting of multiple accelerators, the architecture of the accelerator hardware themselves and the description of the neural network is taken as input. The tool calculates the execution time of each training step. 

%In this mode, the user 
%has the flexibility to use either just the \perfE or use \perfE alongside the AGE. While using just the \perfE  alone, the user needs to provide the architectural parameters of the tiles and the system. On the other hand, while using AGE  alongside \perfE, the user 
%needs to define the technology parameters and the hardware constraints i.e., the overall area and power breakdown among the different architectural components of the system. T

%In this mode, the tool generates the micro-architectural parameters of the accelerator chip using the AGE. It then runs the compute graph transformation and the device placement engine, and uses the \perfE to predict the execution time. 

\subsubsection{Architecture Search (AS) Mode}

The insatiable demand to run large models in the shortest possible time demands that we find the optimal hardware and software design points to train these models. From the hardware perspective, it is about finding the right micro-architecture as well as the overall system architecture of the distributed system. 
From the software perspective, it is about finding the right parallelization strategy. 
Often these decisions depend on each other, and so finding the optimal design points across the stack means 
navigating a large design space.

As one can imagine, the design space of the inputs to the tool is large and iterating over the entire design space is a tedious task. To efficiently search over the input space to find the optimal hardware constraints and parallelization strategy, the tool can be run in the AS mode. 
In this mode, the SOE module is used. The user will not need to provide the exact hardware parameters and the parallelization strategy. Only the architecture template and the initial compute graph will need to be provided as input to the tool. The tool then performs a search over the design space to find the optimal parameter settings that results in minimum training time. 
%We used gradient descent algorithm (details in Section~\ref{}) for this search.

%\subsection{Inputs and Outputs}

%\paragraph{SP-Mode}
%In this mode, the hardware 

%\paragraph{AS-Mode}

\end{comment}

%\vspace{-0.4cm}
\section{DeepFlow Overview}\label{sec:overview}

Figure~\ref{fig:overview} shows an overview of the \name framework. \name takes the following set of \textbf{inputs}: 
%
(1) \underline{System} design hierarchy (e.g., the number of accelerator nodes per device, the number of devices in the system, the network topology connecting nodes within a device and across the devices), 
(2) \underline{Architecture template} of each accelerator node which provides a high-level definition of its components and how those components fit together. The purpose of the template is to provide a blueprint for the accelerator without committing to any specific hardware parameters.
%A component definition (e.g., minimal compute units (MCU\footnote{Examples of what we regard as MCU includes SMU in older GPUs, Tensor Cores in newer GPUs or systolic array in TPUs}), memory hierarchy, network), specification of each component (e.g., flop rate for each MCU, MCU dimensions, number of MCUs sharing a set of register files, dataflow execution model, and characteristics and scope of different levels of memory hierarchy), 
(3) \underline{Technology} parameters for each hardware component (e.g. energy per flop), 
(4) \underline{Design budgets} for each hardware component (area, power, perimeter),  
(5) \underline{Machine learning model} specification in the form of a high-level compute graph, parameters of each compute node (kernel type, tensor dimensions), and
(6) \underline{Parallelism strategy} (data, model, kernel, and/or pipeline parallelism dimensions) which distributes the compute graph across the entire system. 
(7) \underline{Device mapping} strategy which defines mapping of parallel shards onto hardware nodes.
Given these inputs, \name predicts the end-to-end performance of one iteration (i.e., single batch) of the model and finds an optimal hardware-software-technology design point as \textbf{output}. 

DeepFlow is composed of two major components.
\underline{CrossFlow} which operates in a stand-alone mode and can predict performance for any input configuration; and a search and optimization engine (\underline{SOE}) which enables design space search. 
%To do so, \name breaks the problem into multiple phases.
%Each phase or building block of \name is described in details next.
\vspace{-0.1cm}
\subsection{CrossFlow Building Blocks}

\paragraph*{\em Micro-Architecture Generator Engine (AGE)}

AGE takes the following set of \textbf{inputs}:
(1) Design constraints (i.e the power, area and perimeter budget and breakdown across micro-architectural components such as cache, network, compute cores). 
This breakdown can be provided manually by users or automatically by the Search and Optimization Engine (SOE, Section~\ref{subsec:soe}).
%We also provide technology specifications such as 
%and their physical characteristics such as area/power per core under nominal operating conditions, SRAM/register characteristics. 
(2) Technology parameters such as energy per flop, energy per data bit transfer for each level of memory and network hierarchy, threshold and maximum gate voltage, integration substrate parameters such as bump/interconnect pitch. We provide a wide range of standard and future technology libraries as baseline. (3) Architecture template which is a blueprint of the underlying accelerator chip without committing to any specific hardware parameters. Given these input, AGE performs a frequency-voltage-area scaling optimization to generate the following \textbf{output} parameters such that design budgets for all component are met: 
(1) Compute throughput.
(2) Capacity for different levels of memory hierarchy.
(3) Bandwidth to each level of memory hierarchy.
(4) Inter-node as well as intra-node network bandwidth. 
These parameters are then utilized by the performance prediction engine (PPE) to estimate the execution time of each kernel.
%As mentioned previously, 
%The output of this stage is the input to performance engine to estimate the execution time of each kernel. Next, we describe the search and optimization engine (SOE) which feeds input values to AGE, if we want to use the model for architecture search.
%\vspace{-0.2cm}
\paragraph*{\em Compute Graph Transformation and Device Placement Engine (DPE)}
The parallelization strategy and device mapping are critical in deciding the overall execution time. Here, we first transform the model graph to a `super-graph' to reflect the parallelization strategy provided by the users manually, or SOE engine (Section~\ref{subsec:soe}) automatically. For example, to apply data parallelism, the model graph is replicated and appropriate edges are added to model the gradient exchange. After generating the transformed graph, DPE assigns the vertices of the transformed graph to the system nodes following a heuristic approach to minimize the communication overhead. %
%The details are presented in section~\ref{}.

%\vspace{-0.2cm}
\paragraph*{\em Performance Prediction Engine (PPE)}
%With the device mapping for all the vertices of the compute (super-)graph known, the next step is to calculate the overall execution time for a forward pass and/or a backward pass. 
We use hierarchical roofline modeling to predict the performance of each compute node. To calculate the overall end-to-end execution time, while respecting scheduling constraints (e.g. one kernel at a time per GPU, or prioritizing one kernel launch over another) we use event-driven simulation.%
%We explain the details of the PPE in section~\ref{}.
\subsection{Search and Optimization Engine (SOE)}\label{subsec:soe}
Co-optimizing micro-architectural parameters and the parallelization strategy that minimizes the overall end-to-end execution time requires navigating a large space of design parameters. 
Search and optimization engine (SOE) enables the automatic design space search and finds an 
%that meets the total power and area constraints, and simultaneously explores software parallelization strategies to find the 
optimal design point which meets the design constraints and minimizes the overall execution time.
%Because the hardware configuration space is very large, the search algorithm we designed 
SOE takes inspiration from ML-assisted search algorithms, in particular gradient decent search with momentum and builds on top of the CrossFlow modeling engine.
%The software parallelization design space is much smaller compared to the hardware design space and therefore we employ an exhaustive grid search. 

%Gradient search is an iterative process. In each step, SOE takes the predicted time from previous iteration as input to re-adjust the following parameter settings: (1) power, area and perimeter breakdown across different architectural components. (2) a parallelization strategy. These parameters will be fed back to CrossFlow to estimate the overall execution time. This process continues until convergence or user-specified number of steps. 
%The details of SOE's search algorithm are elaborated in Section~\ref{}. 
\vspace{-0.2cm}
\subsection{Parallelism Strategy Space}
\label{subsec:par_strategy}
There are a myriad of ways to parallelize a model across a large multi-node system. Exploring the parallelism space and finding the optimal strategy is critical to overall performance and system utilization. DeepFlow explores kernel, data and layer parallelism. It uniquely identifies each parallelism strategy by following notations: $\texttt{RC-\{KP1\}-\{KP2\}-d\{DP\}-p\{LP\}}$ or $\texttt{CR-\{KP1\}-d\{DP\}-p\{LP\}}$ depending on the choice of kernel parallelism.
RC (Row-Column) and CR (Column-Row) refer to different forms of kernel parallelism, i.e. distributed GEMM through inner-product or outer-product implementation.
%\begin{equation*}
%    \texttt{RC: R{KP1\}\_C\{KP2\}\_d\{DP\}\_p\{LP\}}
%\end{equation*}
%Where \texttt{RC} or \texttt{CR} refers to the type of kernel parallelism strategy, i.e. Row-Column or Column-Row,
%\texttt{N} refers to the number of parallel nodes,
\texttt{KP1} and \texttt{KP2} are the parameters of distributed GEMM. 
For Row-Column (\texttt{RC}) or inner-product, \texttt{KP1} and \texttt{KP2} would refer to the number of ways we shard the first matrix across rows and the second matrix across columns.
For Column-Row (\texttt{CR}) or outer-product, we would only need one parameter to specify the parallelization strategy; \texttt{KP1} will refer to the number of ways we cut the first matrix across columns and the second matrix across rows.
\texttt{DP} represents the number of model replicas and data shards assigned to each to exploit data parallelism.
\texttt{LP} is the number of ways we cut layers into stages to exploit pipeline parallelism.

\begin{comment}
\subsection{Modes of Operation}
\name has two modes of operation, standalone performance estimation mode and a architecture search mode.
\paragraph{Standalone Performance (SP) Estimation Mode}
Often ML practitioners or hardware designers want to estimate the performance of a model on a particular system configuration. For example, what is the cost optimal number of accelerators that one should deploy for distributed training? Or what is the estimated performance gain from choosing an accelerator with costlier HBM2E vs HBM2? To enable one to quickly answer such questions and to estimate performance under certain known system configurations, the tool can be run in the SP mode. 

In this mode, the description of the architecture of a scale-out system consisting of multiple accelerators, the architecture of the accelerator hardware themselves and the description of the neural network is taken as input, and fed into CrossFlow, which calculates the execution time of each training step. 

%In this mode, the description of the architecture of a scale-out system consisting of multiple accelerators, the architecture of the accelerator hardware themselves and the description of the neural network is taken as input. The tool calculates the execution time of each training step. 

%In this mode, the user 
%has the flexibility to use either just the \perfE or use \perfE alongside the AGE. While using just the \perfE  alone, the user needs to provide the architectural parameters of the tiles and the system. On the other hand, while using AGE  alongside \perfE, the user 
%needs to define the technology parameters and the hardware constraints i.e., the overall area and power breakdown among the different architectural components of the system. T

%In this mode, the tool generates the micro-architectural parameters of the accelerator chip using the AGE. It then runs the compute graph transformation and the device placement engine, and uses the \perfE to predict the execution time. 

\subsubsection{Architecture Search (AS) Mode}

The insatiable demand to run large models in the shortest possible time demands that we find the optimal hardware and software design points to train these models. From the hardware perspective, it is about finding the right micro-architecture as well as the overall system architecture of the distributed system. 
From the software perspective, it is about finding the right parallelization strategy. 
Often these decisions depend on each other, and so finding the optimal design points across the stack means 
navigating a large design space.

As one can imagine, the design space of the inputs to the tool is large and iterating over the entire design space is a tedious task. To efficiently search over the input space to find the optimal hardware constraints and parallelization strategy, the tool can be run in the AS mode. 
In this mode, the SOE module is used. The user will not need to provide the exact hardware parameters and the parallelization strategy. Only the architecture template and the initial compute graph will need to be provided as input to the tool. The tool then performs a search over the design space to find the optimal parameter settings that results in minimum training time. 
%We used gradient descent algorithm (details in Section~\ref{}) for this search.

%\subsection{Inputs and Outputs}

%\paragraph{SP-Mode}
%In this mode, the hardware 

%\paragraph{AS-Mode}

\end{comment}




\begin{table*}[htbp]
% \setlength{\tabcolsep}{0.6mm}
  \centering
    \resizebox{1.0\textwidth}{!}{\begin{tabular}{c|c|p{37em}|c}
    \toprule
    \multirow{2}[2]{*}{\textbf{Category}} & \multirow{2}[2]{*}{\textbf{Task Type}} & \multicolumn{1}{c|}{\multirow{2}[2]{*}{\textbf{Our Minmal Prompt}}} & \multirow{2}[2]{*}{\textbf{Label}} \\
          &       & \multicolumn{1}{c|}{} &  \\
    \midrule
    \multirow{4}[8]{*}{yes/no} & \multirow{2}[4]{*}{\shortstack{Paraphrase \\ Identification}} & John is Lily's husband. Lily is John's wife & 1 \\
\cmidrule{3-4}          &       & John is Lily's husband. Lily is John's mother. & 0 \\
\cmidrule{2-4}          & \multicolumn{1}{c|}{\multirow{4}[4]{*}{\shortstack{Natural \\ Language  \\ Inference}}} & \underline{Premise:} Dana Reeve, the widow of the actor Christopher Reeve, has died of lung cancer at age 44. \underline{Hypothesis:} Dana Reeve had an accident. & 1 \\
\cmidrule{3-4}          &       & \underline{Premise:} Dana Reeve, the widow of the actor Christopher Reeve, has died of lung cancer at age 44. \underline{Hypothesis:} Christopher Reeve had an accident. & 0 \\
    \midrule
    \multirow{10}[20]{*}{multi-choice} & \multirow{2}[4]{*}{\shortstack{Coreference \\ Resolution}} & Jane gives Joan candy because Joan was hungry. & 1 \\
\cmidrule{3-4}          &       & Jane gives Joan candy because Jane was hungry. & 0 \\
\cmidrule{2-4}          
& \multirow{2}[4]{*}{\shortstack{Question \\ Answer}} & The earth moves around the sun. What is the earch to the sun? Planet & 1 \\
\cmidrule{3-4}          &       & The earth moves around the sun. What is the earch to the sun? Satellite & 0 \\
\cmidrule{2-4}          & \multirow{2}[4]{*}{\shortstack{Topic \\ Classification}} & Open Source Apps Developer SugarCRM Releases Sugar.Sales 1.1. Science and technology & 1 \\
\cmidrule{3-4}          &       & Open Source Apps Developer SugarCRM Releases Sugar.Sales 1.1. Sports & 0 \\
\cmidrule{2-4}          & \multirow{2}[4]{*}{\shortstack{Sentence \\ Completion}} & A boy is running down a track. The boy lifts his body above the height of a pole. & 1 \\
\cmidrule{3-4}          &       & A boy is running down a track. The boy stands on his hands and springs. & 0 \\
\cmidrule{2-4}          & \multirow{2}[4]{*}{\shortstack{Sentiment \\ Classification}} & I really love this movie. Positive & 1 \\
\cmidrule{3-4}          &       & I don't like this movie. Negative & 1 \\
\bottomrule
    \end{tabular}}%
\caption{Examples of how we unify discriminative tasks. The underlined text represents additional words not present in raw inputs. Note that this is just our implementation of the UD formulation and there can be other ways of task formulation under the UD framework. Some tasks can either be yes/no tasks or multi-choice tasks, depending on how options are provided.}
\label{tab:task_formulate_example}
\end{table*}%






\section{Approach}

Previous works \citep{T0-paper,FLAN} have shown that prompted multi-task training can greatly improve zero-shot performance on unseen tasks. One intuitive reason behind the validity of this improvement is that all the NLP tasks share a common ability that allows LMs to solve unseen tasks based on the data from other training tasks. To test this idea and even enhance zero-shot generalization, a direct way is explicitly defining what this "common ability" is. Here, we define this "common ability" by designing a new general task of ``discriminating whether a text sample comes from the true data distribution of natural language''. 

We will first formulate the learning problem (\S~\ref{sec:formualtion}), and then define the concept \textit{discriminative tasks} (\S~\ref{sec:disc}), followed by describing how we transform discriminative tasks into our shared formulation.
In \S~\ref{sec:ud} and \S~\ref{sec:generalizedud}, we will study our UD, respectively on discriminative tasks and on a generalized scope of both discriminative and generative tasks.


\subsection{Multi-Task Training for Zero-Shot Generalization} \label{sec:formualtion}


Now we describe the learning problem we aim to solve in this work.
We adopt the same setting as in \citet{T0-paper}. The input to our problem is a set of training tasks with labeled data, and the goal is to train a model that generalizes to unseen test tasks. The training and test tasks are constrained to have distinct task types for the evaluation of cross-task-type generalization. A pretrained model is jointly trained on the set of training tasks and directly evaluated on the set of test tasks in a zero-shot manner.


% The difference between our multi-task training and theirs is that our training and evaluation processes are all in the UD format (which be described in section~\ref{sec:ud}) and then transform the result to each individual task. We believe that training and evaluating in this shared format can better induce zero-shot generalization performance across different tasks.

% \paragraph{Finetuning}
% We also study the finetuning paradigm. Specifically, after we train the model using multiple tasks, we follow previous work \cite{T5-paper} to further finetune the model on each individual task to obtain the best performance on these tasks. We use this setting to test the performance of our approach with abundant labels.


% whether our UD format training has advantage in the standard finetuning paradigm. In specific, we first perform a multi-tasking training on a pretrained T5 model, and then finetune each individual task, where all the training and evaluating processes are in UD format. We hypothesis that the shared UD format makes it possible for LM to learn from other tasks and thus improve the accuracy after finetuning.

\subsection{Discriminative Tasks} \label{sec:disc}

% \zy{given the definition of discriminative tasks, and then prove that a significant portion of NLP tasks are discriminative tasks.}

We use the term ``discriminative tasks'' to refer to tasks that can be framed as selecting from a few options. 
%\xhk{instead of generating an answer verbalizer by the model itself}

More concretely, there are two types of discriminative tasks. The first type is tasks with multiple options, such as multi-choice question answering and news classification. The problem can be framed as selecting the right option from multiple ones, where the options are either customized for each sample (e.g., multi-choice question answering) or shared within the task (e.g., news classification). The second type is tasks with yes/no options, such as paraphrase identification and natural language inference. Given a sample of these tasks, a model is asked to predict a yes/no (or true/false) answer. 
%\xhk{We separately create the second type of tasks because the yes/no token itself doesn't contain much information compared with customized choices. Empirical experiments suggest that there is a need to unify them in a different way.} 
% \sout{In this work, we will mainly study how to use discriminative approaches to optimize the performance of discriminative tasks.}

It is important to notice that discriminative tasks constitute a significantly large portion of modern NLP research tasks. For example, all of the test tasks of the T0 benchmark~\cite{T0-paper}, SuperGLUE~\cite{wang2019superglue}, GLUE~\cite{wang2019glue} are discriminative tasks. 
As shown in Figure \ref{fig:discrim}, discriminative tasks constitute up to 60+\% in the T0 multi-task benchmark.
Also note that our definition of discriminative tasks has a larger scope compared to the conventional notion of ``classification'' which usually refers to tasks with a non-customized, fixed set of labels. In contrast, discriminative tasks might have sample-customized options, e.g., multi-choice question answering and coreference resolution.

% For ``discriminative tasks", we mean those tasks with limited answer choices provided to LMs during the test phase. We use two observations of discriminative tasks compared with generative tasks in developing our method: First, rather than considering which is the answer for a task from scratch, discriminative tasks only require LMs to judge whether each choice is acceptable and then select the most acceptable one. Second, if a choice is not the correct answer for a given task, we can usually find some violation of real language usage in certain concatenation of the input and the wrong choice. The first observation makes it reasonable to train a specialized discriminator to do the ``true distribution" judgment for each choice separately, and the second observation implies the existence of universal rules behind such judgements across different tasks, motivating us to train a universal discriminator.


% \subsection{Unifying training data from most NLP classification tasks to \solution task}
\subsection{A Universal Discriminator}
\label{sec:ud}

% \zy{Moved from other places, need to rewrite and fit in} 

Given a text sample $x$, let $P(\text{true} | x)$ be the probability that $x$ is sampled from the true data distribution of natural language. We train a universal discriminator (UD), denoted as $D(x)$, to estimate the probability $P(\text{true} | x)$ for each text sample $x$. From another perspective of contrastive learning \cite{oord2018representation}, this problem can also be viewed as learning a partial order of the probability distribution. Specifically, for two text samples $x_1$ and $x_2$, if $P(\text{true} | x_1) > P(\text{true} | x_2)$, the UD is expected to predict $D(x_1) > D(x_2)$. This contrastive view is essential for tasks with multiple options, i.e., learning to select from a few options based on the partial order given by UD.


% Given a labeled dataset $\mc{D}=\{x_i,y_i\}$ where $x_i$ is text and $y_i\in[0,1]$ is the probability that $x_i$ comes from the true data distribution of natural language. Our goal is to train a universal discriminator (UD) which can provide a mapping $f(x)$ from a text to a probability that $x$ comes from the true data distribution of natural language. This tasks measure the ability to discriminate whether a text sample comes from the true data distribution of natural language or not.

Figure~\ref{fig:overview} compares the multi-task prompted formulation of T0 and the formulation of our UD.
In the following, we will show how we use this formulation of UD to unify and solve discriminative tasks.
% In the following, we will show how we do this and once we get such a discriminator, how it can solve all the discriminative tasks.

\subsubsection{Unifying Discriminative Tasks}
\label{sec:unifydiscriminativetasks}

We assume that a data example is considered ``correct'' if it follows the true data distribution of natural languages, while ``wrong" if it deviates much from the true data distribution. 
% \xhk{We assume that given a discriminative task example, a "correct" transformation \zy{not clear based on context, what is a transformation?} (see the following paragraphs for different transformation methods) of this example can be considered as coming from the true data distribution of natural language, while some other "wrong" transformation can't, where our proposed UD is aimed at predicting this.} \zy{language not formal enough}
Given this assumption, we claim that almost all discriminative tasks are equivalent to our defined task (i.e., estimating $P(\text{true} | x)$) above. Here, ``equivalent" has bi-directional meanings: on one hand, there exists a data transformation method such that one piece of training data from a discriminative task can be transformed into several pieces of UD's training data. 
On the other hand, there exists a data transformation method such that UD can solve a discriminative task by first predicting $D(\cdot)$ for a set of transformed samples and then using a mapping from UD's outputs to the original task's outputs.
% On the other hand, there exists a data transformation method such that UD can solve a discriminative task by first predicting $P(\text{true} | x)$ for a transformed input $x$ and then using a mapping from UD's outputs to the original task's outputs.
% On the other hand, there exists a data transformation method such that UD answers \zy{answers?} a piece of test data from a discriminative task by first predicting scores $D(\cdot)$ for a set of its transformed samples and then using a mapping from UD's outputs to the original task's outputs. \zy{simplify this sentence}


%\xhk{Given this assumption, we claim that almost all discriminative tasks are equivalent to our defined task (i.e., estimating $P(\text{true} | x)$) above. Here, ``equivalent" has bi-directional meanings: on one hand, for a piece of training data from any discriminative task, we can combine its input $x_{in}$ with each of its options from $\{c_i\}_{i=1}^{N_c}$ to get several pieces of UD's training data ($\{x_i=(x_{in},c_i)\}_{i=1}^{N_c}$). On the other hand, given a piece of test data $x_{in},\{c_i\}_{i=1}^{N_c}$ from any discriminative task, UD can answer it by first predicting scores $\{D((x_{in},c_i))\}_{i=1}^{N_c}$ for the set of combined samples between input and each option and use a mapping from the UD's outputs to the original task's outputs}


%Given this assumption, we claim that almost all discriminative tasks are equivalent to our defined task (i.e., estimating $P(\text{true} | x)$) above. Here, ``equivalent" has bi-directional meanings: on one hand, there exists a data transformation method such that a discriminative task's training data can be transformed into UD's training data. On the other hand, there exists a data transformation method such that UD can solve a discriminative task by first predicting $P(\text{true} | x)$ for a transformed input $x$ and then using a mapping from UD's outputs to the original task's outputs.

Based on the definition of discriminative tasks in \ref{sec:disc}, there are two main categories, multi-choice tasks and yes/no tasks. We will discuss each category in detail as follows (also see Table \ref{tab:task_formulate_example} for specifics).

% For all the discriminative tasks we have met, we design 3 unifying methods to transform each tasks's data to UD's iuput data, basing on the relationship of the given answer choices: parallel, opposite, or extent. Please refer to Table~\ref{tab:task_formulate_example} to see examples for each unifying method.

\paragraph{Multi-Choice Tasks}
For multi-choice tasks, we concatenate the text input $x_{in}$ with each choice $\{c_i\}_{i=1}^{N_c}$ to form samples. For example, for multi-choice question answering, we concatenate the given paragraph and question with each answer candidate. See Table \ref{tab:task_formulate_example} for more task formulation. During training, the concatenated samples with the correct choice are given label $1$ ("correct" transformation) for UD and the other incorrect ones are given label $0$ ("wrong" transformation). During testing, similarly, we concatenate the text input 
$x_{in}$ with each choice $\{c_i\}_{i=1}^{N_c}$ 
to form several samples 
$\{(x_{in},c_i)\}_{i=1}^{N_c}$ 
and ask UD for their $D(\cdot)$ scores. We then select the sample with the maximal $D(\cdot)$ score and output its corresponding choice.

%For multi-choice tasks, we concatenate each choice with the text inputs to form a sample. For example, for multi-choice question answering, we concatenate each candidate answer with the given paragraph and question. See Table \ref{tab:task_formulate_example} for more task formulation. During training, the concatenated samples with the correct choices are given label 1 for UD and the incorrect ones are given label 0. During test, for each sample, we select the concatenated sentence with the maximal UD probability and output its corresponding choice.

% \xhk{move to multi-choice paragraph}
% Some of the multi-choice tasks might be related to regression; i.e., a restaurant review classification task assigns 1 to 5 stars to an input text. In this case, we can set the groundtruth label as a probability (e.g., 0.25, 0.5, 0.75, 1.0, etc), which is compatible with the cross entropy loss. We note that other formulations are also possible.

% For tasks like Question Answering, Sentence Completion, the answer choices are several unrelated words or phrases. Our unifying method is to use minimal prompt to concatenate all the raw input keywords and each answer choice. In training phase, the concatenated sentence with correct answer is given label 1, and the other sentences are given label 0. In testing phase, we select the concatenated sentence with the maximal probability of true, and output its corresponding answer choice.

% Some of the multi-choice tasks might be related to regression; e.g., a restaurant review classification task assigns 1 to 5 stars to an input text. We design a slightly more accurate formulation for these tasks (see Appendix \ref{sec:extent}).

% \xhk{A special case of multi-choice tasks is extent measurement, e.g. the degree of sentiment, or the attitude of a given paragraph. We design a slightly more accurate unifying method in appendix~\ref{sec:extent}}

\paragraph{Tasks with Yes/No Choices}
% \xhk{One reviewer suggested how to use UD in test phase is unclear, so I rewrite this paragraph.}
For yes/no tasks, we directly treat the text input $x_{in}$ as a sample and assign its 0/1 label based on its yes/no label. During training, we use $x_{in}$ with its assigned 0/1 label as UD's training data. During testing, we first get the output of UD on $x_{in}$, $D(x_{in})$, and then output answer yes/no based on whether $D(x_{in})>0.5$\footnote{We note that more delicate threshold search might be possible, but we find it performs well using a constant 0.5.}. 

%We separately create a new method for tasks with Yes/No choices because here the answer tokens Yes/No themselves don't contain much information compared with customized choices. 
Empirical experiments suggest that unifying tasks with Yes/No choices in such a new way can produce better zero-shot performance than using the same method for Multi-Choice Tasks because the answer tokens here don't contain much information and thus the model cannot benefit from concatenation.

%For yes/no tasks, we use a more direct connection to UD. During training, we use the yes/no label as the UD label for each sample. During test, we perform binary classification using a threshold of 0.5\footnote{We note that more delicate threshold search might be possible, but we find it perform well using a constant 0.5.} based on the UD predictions.


% Some tasks are essentially binary interrogating questions, e.g. judging whether a hypothesis is implied by the premise, judging whether two sentences has the same meaning. Their answer choices are usually Yes/No, True/False. For these tasks, we assume that the answer for this tasks is an intrinsic nature of the given raw input, even without choice attached. Through minimal prompting, we get a concatenation of the raw input's keywords as the input for UD. In training phase, we assign it a label basing on its answer. In test phase, we do the binary selection by checking whether the probability of true outputed by UD is larger than 0.5. 

% \paragraph{Extent Answer Choices}
% Some tasks are asking for the location of the given input in an extent measurement, e.g. the degree of sentiment, or the attitude of a given paragraph. Usually, "positive" and "negative" are the two ends in the measurement. For these tasks, we use minimal prompt to concatenate the input with each of the two extreme extent verbalizers, e.g. "positive" and "negative“, and we assign it the label basing on the true location of the sentence in the extent measurement.

\paragraph{Minimal Prompting} A key principle we follow for task formulation is minimal prompting. From Table \ref{tab:task_formulate_example}, one can see that our prompts are minimal in the sense that they are mostly just concatenations. This is very different from T0 \cite{T0-paper} and other generative approaches \cite{gpt3-paper,PET-paper} that add lengthy task descriptions with different wordings into the prompts.

We argue that there are two major benefits of minimal prompting. First,  previous work \cite{liu2021gpt} has shown that zero-shot and few-shot performances are very sensitive to the prompts used for inference. Minimal prompting is more robust and requires less prompt engineering efforts at test time. This is especially important for true zero-shot real-world applications as there is no data available for choosing the right prompt. Second, as we will show in our experiments, UD performs much better with minimal prompts than lengthy descriptive prompts, while generative approaches do not work well with minimal prompts. This is also consistent with our motivation that UD unifies discriminative tasks so it does not rely much on descriptions for each task.

Note that it is also important to use minimal prompts to resolve ambiguity for yes/no tasks. For example, consider the natural language inference (NLI) task that predicts whether a premise $A$ entails a hypothesis $B$. Simply concatenating $A$ and $B$ is ambiguous, because the model cannot tell which is the premise. The model also is not aware that this is an NLI task. To resolve this kind of ambiguity, we use a minimal prompt ``Premise: A. Hypothesis: B.'' instead, as shown in Table \ref{tab:task_formulate_example}.

% \subsubsection{Minimal Prompting}

% \zy{Describe the idea of minimal prompting here}

% In the three unifying methods described above, we need a minimal prompting to concatenate different keywords of a task's raw input and each of its choices. The goal of our minimal prompting is to make the prompted sentence looks more like sampled from the true language distribution for correct answer and less like sampled from the true distribution for other wrong choices, but not to instructing what the task is, which is the key difference between our minimal prompt and the task descriptive prompts used in \citet{T0-paper}. We call it ``minimal'' because for many cases, only trivial concatenation with choice attached at the end is enough for out goal, and it is much simpler than the prompts used in \citet{T0-paper}. However, in some cases, such ``minimal" prompts are necessary because trivial concatenation may result some ambiguity for certain different tasks. For example, in NLI tasks, if we want to judge whether a hypothesis ``B`` is implied by a premise ``A", simple concatenation like ``A B" is ambiguous, because attaching a reason instead of a hypothesis also looks like a sentence sampled from true language distribution, but it should be given False for the NLI task. In this case, our minimal prompt should give ``premise: A Hypothesis: B" to clarify the meaning of each sentence component. Similar necessary case for minimal prompting usually happens for those tasks with opposite choices.

% \subsection{Using a \solution oracle to solve most NLP classification tasks}
% \subsection{Training and Evaluating Universal Discriminator}

% \subsection{Architecture}
\subsubsection{Architecture}

UD can use any pretrained encoder model as the backbone. In this work, we experiment with the T5 encoder and DeBERTa \cite{debertav3}. Since T5 is an encoder-decoder model, we only use the encoder part. For the T5 backbone, we perform mean pooling over the last-layer encoder features, followed by a dropout layer and a linear layer to predict a scalar logit. For the DeBERTa backbone, we use the last-layer feature of the first token, followed by a two-layer perceptron with dropout to also output a scalar logit. We train UD with the binary cross entropy loss.



\subsection{A Generalized Universal Discriminator}
\label{sec:generalizedud}

To further study how the discriminative approaches work in combination with generative tasks, we also propose to experiment with a generalized version of UD (denoted as generalized UD).



Different from the previous UD that only uses an encoder as the backbone model, the generalized UD employs an encoder-decoder architecture. In the following, we experiment with the T5 model.
Generalized UD takes both discriminative and generative tasks into consideration, and is jointly trained over both types of tasks at the same time.



For discriminative tasks, they are reformulated into binary classification tasks through minimal prompting, as is described in  \S~\ref{sec:unifydiscriminativetasks}. Specifically, it takes the minimal prompted texts into the encoder and uses the decoder to predict over \{``Yes'', ``No''\}.
In such cases, generalized UD is optimized with the binary cross-entropy loss.
For generative tasks, they take the form of ``input-and-target'' pairs. Generalized UD is fed with the textual inputs, and generates the targets through decoding.
For generative tasks, generalized UD is trained to optimize the cross-entropy loss.






% \zy{describe the architecture}

% \zy{Moved from other places, need to rewrite and fit in} 

% We inherit the multi-task training phase by \citep{T0-paper,FLAN}. During the training phase, we transform all training tasks' data into data for UD, and train a LM to solve the this task, i.e. we train a universal discriminator. Then, in the zero-shot testing phase, we transform each testing task's data into UD format and use UD's prediction to deduce the answer for the original testing task. Therefore, once we train a UD, we can use it to solve almost all the NLP tasks. 

\begin{comment}
Here we only give an example from COPA in table~\ref{tab:copa_raw_input}

The raw input is
\begin{table}[htbp]
    \centering
    \begin{tabularx}{0.5\textwidth}{ l|X }
        \toprule
        Input & The hamburger meat browned.  \\
        \midrule 
        Ask for & cause  \\
        \midrule
        Choices & The cook froze it., The cook grilled it.\\
        \midrule 
        Target & The cook grilled it.  \\
        \bottomrule 
    \end{tabularx}
    \caption{an example from COPA's raw data}
    \label{tab:copa_raw_input}%
\end{table}%

The way we test this task is that we first similarly unifying the data to the \solution data format, by concatenating its input, "ask for", and choice. Then we use our \solution model to predict the probability of True for each sentence. Finally, we select the choice with whose concatenated sentence has the highest probability of True.

Our input for \solution is

\begin{table}[htbp]
    \centering
    \begin{tabularx}{0.5\textwidth}{ X }
        \toprule
        Input \\
        \midrule
        The hamburger meat browned. cause The cook froze it. \\
        \midrule 
        The hamburger meat browned. cause The cook grilled it. \\
        \bottomrule 
    \end{tabularx}
    \caption{a unified COPA example}
    \label{tab:quarel_consistency}%
\end{table}%
\end{comment}


% \subsection{Discussion}

% % \zy{This section mainly discusses our advantages to generative modeling.}

% % \zy{Moved from other places, need to rewrite and fit in} 

% Similar to prompting, which makes all NLP tasks to "text generation task", here we make all NLP tasks to the UD format, which is a discriminative task. One key difference between our unifying method and prompting is that, prompting uses human task descriptive language to unify all tasks to "text generation task", but there is no evidence that the LM can truly understand the prompt's instruction \citep{()} and the choice of different prompts can significantly disturb the performance \citep{}. Therefore, prompts are just unifying the input and output format of all tasks rather than unifying their task nature. However, after our unifying method, we hypothesis that all tasks now share the same ability, more suitable for multi-task training, and can induce better zero-shot performance to unseen tasks. Besides, our unifying methods need minimal prompting, no task descriptive language is used, which significantly reduces performance's sensitivity on handcrafted prompts and save human efforts on designing such prompts.

% \xhk{move to here: why we want to train a universal discriminator for discriminative tasks? We use two observations of discriminative tasks compared with generative tasks in developing our method: First, rather than considering which is the answer for a task from scratch, discriminative tasks only require LMs to judge whether each choice is acceptable and then select the most acceptable one. Second, if a choice is not the correct answer for a given task, we can usually find some violation of real language usage in certain concatenation of the input and the wrong choice. The first observation makes it reasonable to train a specialized discriminator to do the ``true distribution" judgment for each choice separately, and the second observation implies the existence of universal rules behind such judgements across different tasks, motivating us to train a universal discriminator.}
% \section{Method}
\subsection{Problem Formulation}
A well-calibrated model is expected to output prediction confidence (e.g., the highest probability after softmax activation) comparable to or aligned with its task accuracy (i.e., empirical likelihood). For example, given 100 examples with the prediction confidence of 0.8 (or 80\%), we expect that 80 examples will be correctly classified. Following~\citet{icml17}, we estimate the calibration error by empirical approximations. Specifically, we partition all examples into $K$ bins of equal size according to their prediction confidences. Formally, for any $p\in[\ell_k,u_k)$, we define the empirical calibration error as:
\begin{equation}
\hat{\mathcal{E}}_k=\frac{1}{|\mathcal{B}_k|}\Big|\sum_{i\in\mathcal{B}_k}\big[\mathbbm{1}(\hat{y}_i=y_i)-\hat{p}_i\big]\Big|,
\end{equation}
where $y_i$, $\hat{y}_i$ and $\hat{p}_i$ are the true label, predicted label and confidence for $i$-th example, and $\mathcal{B}_k$ denotes the bin with prediction confidences bounded between $\ell_k$ and $u_k$.
To evaluate the calibration error of classifiers, we further adopt a weighted average of the calibration errors of all bins as the Expected Calibration Error (ECE)~\citep{DBLP:conf/aaai/NaeiniCH15}:
\begin{align}
    \textrm{ECE} =\sum_{k=1}^K\frac{|\mathcal{B}_k|}{n} \hat{\mathcal{E}}_{k},
    \label{eq:ece}
\end{align}
where $n$ is the example number and lower is better.
Note that the calibration goal is to minimize the calibration error without significantly sacrificing prediction accuracy. 


\subsection{Our Approach}

Generally, text classification models are optimized by Maximum Likelihood Estimation (MLE), which minimizes the cross-entropy loss between the predicted and actual probability over $k$ different classes.
To minimize the calibration error, we add a regularization term to the original cross-entropy loss as a multi-task setup.

Our intuition is that if the error of the model on example $i$ is more significant than its error on example $j$ (i.e., example $i$ is considered more difficult for the classifier), then the magnitude of attributions on example $i$ should not be greater than the magnitude of attributions on example $j$. Moreover, we penalize the magnitude of attributions with the model confidence~\cite{DBLP:conf/acl/XinTYL20}, as the high error examples also should not have high confidence. Compared to the prior post-calibration methods (e.g., temperature scaling learns a single parameter with a validation set to rescale all the logits), our method is more flexible and sufficient to calibrate the model during training.

% The magnitude of attributions is gathered by  $\ell_{2}$ normalization.
Formally, given a training set $\mathcal{D} =$ $\{(\boldsymbol{x}_{1},y_{1})$$,\cdots,$$(\boldsymbol{x}_{n},y_{n})\}$ where $\boldsymbol{x}_{i}$ is the embeddings of input tokens and $y_{i}$ is the one-hot vector corresponding to its true label, an attribution of the golden label for input $\boldsymbol{x}_i$ is a vector $\boldsymbol{
a}_i = (a_{i1},\cdots,a_{il})$, and $a_{ij}$ is defined as the attribution of $x_{ij}$ ($l$ is the length). Here, attention scores are taken as the self-attention weights induced from the start index to all other indices in the penultimate layer of the model; this excludes weights associated with any special tokens added. Then, the token attribution $a_{ij}$ is the normalized attention score~\cite{FRESH} scaled by the corresponding gradients $\nabla \alpha_{ij}= \frac{\partial \hat{y}}{\partial \alpha_{ij}}$~\cite{serrano-smith-2019-attention}. At last, our training minimizes the following loss: 
\vspace{-2mm}
\begin{equation}
\label{loss_function}
    \mathcal{L}_{CME} = \mathcal{L}_{classify} + \lambda \mathcal{L}_{calib},
\end{equation} where $\lambda$ is a weighted hyperparameter. The $L_{calib}$ is calculated as follows:
\vspace{-2mm}

\begin{align}
    \mathcal{L}_{calib} &= \sum_{1\leq i,j\leq n}\Psi_{i,j} \mathbbm{1}[e_i > e_j], \label{eqn:atten1}\\
    % \Psi_{i,j}=max\left\{ 0, t(\boldsymbol{x}_i) - t(\boldsymbol{x}_j)\right\}^{2}, \label{eqn:atten2} \\
     \Psi_{i,j} &= \max[0, t(\boldsymbol{x}_i) - t(\boldsymbol{x}_j)]^{2}, \label{eqn:atten2} \\
    t(\boldsymbol{x}_i) &=  \lVert{ a_{ij}}\rVert_2 * c_i, \label{eqn:atten3}
\end{align}
%  L_{2}\left ( a_{ij} \right )
where $e_i$ and $e_j$ are the error of example $i$ and example $j$, the confidence $c_i$ is estimated by the max probability of output~\cite{DBLP:conf/iclr/HendrycksG17}, with the L2 aggregation. The products could be further scaled by $\sqrt{l}$. 
In practice, strictly computing $L_{calib}$ for all example pairs is computationally prohibitive. Alternatively, we only consider examples from the mini-batch (similar lengths) of the current epoch. In other words, we consider all pairs where $e_i$ = 1 and $e_j$ = 0 where $e$ is calculated by using zero-one error function. The comparisons of example pairs can also be calculated from more history after every epoch or by splitting examples into groups, and we leave it to future work. 

\begin{algorithm}[t!]
 \small
\caption{{\small{Explanation-based Calibrated Training}}}\label{euclid}
 \textbf{Inputs} : Train set $\mathcal{D}$, Number of epochs $T$, Learning rate $\eta$, Optimizer $G$.
\\
\textbf{Output}: Calibrated Text Model $M$
\begin{algorithmic}[1]
%\Require
%\Require{}
%\Require{$\mathcal{Q}$: }
% \State {Let $\mathcal{Q}$ : Empirical Probability Matrix $\in \mathbb{R}^{B \times K}$}
% \State {Random initialization of $\Theta$}
\State Random Initialize $\thetav$.
\For{epoch $= 1 \ldots T$}
    \State{Split $\mathcal{D}$ into random mini-batches \{$b$\}.}
    \For{a batch $b$ from $\mathcal{D}$}{}
        \State{Backward model $M$ for $\nabla_{\thetav} \mathcal{L}_{classify}(\thetav,\mathcal{Y})$.}  
        %\State{Update $\Theta$ by SGD using Loss}
        \State{Calculate the attribution by scaled attention.}
        \State{Computes absolute value of attributions.}
        \State{Normalized it by applying \textrm{Softmax} function.}
        % \If{current step $\in \mathcal{S}$}
        %     \State{$\hat{p}$ = softmax($\Theta,\mathcal{D}$)}
        %     \State{$\mathcal{Q} \leftarrow CalEmpProb(\hat{p},B)$}
        % \EndIf
        \State{Calculate $\mathcal{L}_{CME}$ by Eqn.~\ref{loss_function},~\ref{eqn:atten1},~\ref{eqn:atten2},~\ref{eqn:atten3}.}
        \State{Optimize the model parameters $\thetav$ by G:}
        \State{\hspace*{\algorithmicindent}$\thetav \leftarrow  \thetav - \eta \nabla_{\thetav}\mathcal{L}_{CME}(\thetav,\mathcal{Y})$.}  
    \EndFor
    %\For{$x,y \in \mathcal{D}$}
    %\State{
    %\EndFor
\EndFor
\end{algorithmic}
\label{alg:alg}

\end{algorithm}

Full training details are shown in Algorithm~\ref{alg:alg}. To compute the gradient w.r.t the learnable weight independently, we retain the computation graph in the first back-propagation of classification loss. The model explanations are dynamically produced during training and then used to update the model parameters, which can be easily applied to most off-the-shelf neural networks. \footnote{Code is available here: \url{https://github.com/crazyofapple/CME-EMNLP2022/}}



\input{de-prompting}
% \newpage

\begin{table*}[t]
\setlength{\tabcolsep}{1.5mm}
\centering

\subtable[On 11 discriminative test tasks following the T0 benchmark.]{
\resizebox{\textwidth}{!}{%
    \begin{tabular}{l|l|c|ccccc|ccc|cc|c|c}
    \toprule[1pt]
    \multirow{2}{*}{Base Model} &
    \multirow{2}{*}{Method} &
    \multirow{2}{*}{\#Params} & 
    \multicolumn{5}{|c|}{\textbf{Natural Language Inference}} & \multicolumn{3}{|c|}{\textbf{Sentence Completion}} & \multicolumn{2}{c|}{\textbf{Coreference}} & \multicolumn{1}{c|}{\textbf{WSD}} 
    & \multirow{2}{*}{Avg.}\\
    & & & RTE & CB & ANLI1 & ANLI2 & ANLI3 & COPA & Hella. & Story. & WSC & Wino. & WiC &  \\
    \midrule[1pt]
    Decoder-only & GPT-3 & 175B 
        &63.5 &46.4
        &34.6 &	35.4&	34.5&	91.0&	78.9&	83.2&	65.4&	70.2&	- & -\\
    Decoder-only & GLaM & 137B 
        & 56.3	& 39.3	& 39.7	& 35.5	& 34.1	& 90.0	& 76.7	& 81.1	& 82.1	& 71.3	& 50.6 & 59.7\\
    MoE Decoder-only & GLaM & 64B 
        & 66.8	& 33.9	& 40.9	& 38.2	& 40.9	& 90.0	& 77.1	& 82.5	& 83.5	& 73.4	& 50.5 & 61.6\\
    Decoder-only & PaLM & 540B 
        & 72.9	& 51.8	& 48.0	& 44.2	& 45.7	& 93.0	& 83.4	& 84.6	& 89.1	& 81.1	& 59.1 & 68.5\\
    Decoder-only & FLAN & 137B 
        & 78.3	& 64.1	& 47.7	& 43.9	& 47.0	& 90.6	& 56.4	& 92.2	& 80.8	& 67.3 & - & -\\
    \midrule[1pt]
    \multirow{3}*{\shortstack{ELECTRA}}
    & PE-CLS & 335M
        & 60.2	& 57.4	& 34.1	& 34.4	& 36.4	& 92.7	& 44.1	& 96.0	& 62.8	& 56.3	& 50.7	& 56.8
        \\
    & PE-PROB & 335M
        & 54.0	& 49.2	& 32.3	& 33.3	& 33.5	& 81.9	& 36.7	& 89.5	& 64.3	& 50.7	& 50.9	& 52.4 \\
    & PE-REP & 335M
        & 69.0	& 61.3	& 36.1	& 35.0	& 39.4	& 91.2	& 47.0	& 96.8	& 70.0	& 56.2	& 51.1	& 58.5
        \\
    \midrule
    \multirow{1}*{\shortstack{DeBERTaV3}}
    & \multirow{1}*{{UD (ours)}} & 304M
        & \multirow{1}*{71.1}
        & \multirow{1}*{76.8}
        & \multirow{1}*{43.8}
        & \multirow{1}*{41.3}
        & \multirow{1}*{45.7}
        & \multirow{1}*{96.0}
        & \multirow{1}*{60.7}
        & \multirow{1}*{97.4}
        & \multirow{1}*{66.4}
        & \multirow{1}*{83.6}
        & \multirow{1}*{53.3}
        & \multirow{1}*{66.9}
    \\
    \midrule[1pt]
    \multirow{2}*{\shortstack{T5-Large}}
    & \multirow{1}*{T0 $\star$} & 800M
        & 75.1	& 55.5	& 32.9	& 32.3	& 33.7	& 84.6	& 28.2	& 94.0	& 63.0	& 54.6	& 51.2	& 55.0 \\


    & {UD (ours)} & 400M
        & \textbf{83.8}
        & \textbf{80.4}
        & \textbf{36.8}
        & \textbf{34.2}
        & \textbf{42.2}
        & \textbf{90.0}
        & \textbf{56.1}
        & \textbf{96.4}
        & \textbf{68.3}
        & \textbf{62.9}
        & \textbf{54.6}	
        & \textbf{64.1} \\
    \midrule[1pt]
    \multirow{3}*{\shortstack{T5-XL}}
    & \multirow{1}*{T0 $\dagger$} & 3B
        & 64.6 
        & 45.4
        & 33.8
        & 33.1
        & 33.3
        & 72.4
        & 27.3
        & 84.0
        & 65.1
        & 51.0
        & 50.7
        & 51.0 \\

    & \multirow{1}*{T0 $\star$} & 3B
    & \textbf{79.7}	& 68.9	& \textbf{43.1}	& \textbf{38.5}	& 42.3	& \textbf{94.1}	& 31.5	& 97.5	& 68.8	& 61.3	& \textbf{54.1}	& 61.8\\

 
    & {UD (ours)} & 1.5B
        & 78.7
        & \textbf{73.2}
        & 41.2
        & 36.3
        & \textbf{45.4}
        & 94.0
        & \textbf{70.1}
        & \textbf{97.9}
        & \textbf{72.1}
        & \textbf{70.6}
        & 53.0	
        & \textbf{66.6} \\
    \midrule[1pt]
    \multirow{4}*{\shortstack{T5-XXL}}
    & \multirow{1}*{T0 $\dagger$} & 11B
        & 80.8
        & 70.1
        & 43.6
        & 38.7
        & 41.3
        & 90.0
        & 33.6
        & 92.4
        & 61.5
        & 59.9
        & 56.6
        & 60.8 \\

    & \multirow{1}*{T0 $\star$} & 11B
    & \textbf{85.8}	& 73.3	& 47.3	& 42.0	& 46.1	& 94.4	& 31.5	& 98.4	& 62.8	& 72.8	& 56.0	& 64.6 \\

    & {UD (ours)} & 5.5B
    & 80.5	& 87.5	& 49.0	& 42.9 & 	48.8	& 95.0	& 77.4	& \textbf{98.6}	& 73.1	& 82.2	& 57.1	& 72.0 \\

    & {UD+ (ours)} & 5.5B
    & 82.0	& \textbf{89.3}	& \textbf{53.4} & \textbf{48.1} & \textbf{51.0} & \textbf{96.0} & \textbf{78.9} & 96.7	& \textbf{75.0}	& \textbf{86.4}	& \textbf{58.5}	& \textbf{74.1} \\
    \bottomrule[1pt]
\end{tabular}
}
\label{tab:maintable:top}
}


\subtable[On 13 discriminative BigBench tasks following the T0 benchmark]{
\resizebox{0.7\textwidth}{!}{%
    \begin{tabular}{l|cc|cc|ccc|}
    \toprule[1pt]
    \multirow{1}{*}{Model} 
        & \multirow{1}{*}{\shortstack{T0-Large}}
        & \multirow{1}{*}{\shortstack{UD-large}}
        & \multirow{1}{*}{\shortstack{T0-XL}}
        & \multirow{1}{*}{\shortstack{UD-XL}}
        & \multirow{1}{*}{\shortstack{T0-XXL}}
        & \multirow{1}{*}{\shortstack{UD-XXL}}
        & \multirow{1}{*}{\shortstack{UD+-XXL}}\\
    \midrule[1pt]
    BigBench (Avg.) & 39.6 & \textbf{43.5} & 44.8 & \textbf{48.9} & 47.4 & 55.5 & \textbf{58.7} \\
    \bottomrule[1pt]
    \end{tabular}%
    }
\label{tab:maintable:middle}
}

\subtable[On 22 discriminative BBH tasks]{
\resizebox{\textwidth}{!}{%
    \begin{tabular}{l|ccc|ccc|cccc|}
    \toprule[1pt]
    \multirow{1}{*}{Model} 
        & \multirow{1}{*}{\shortstack{T0-Large}}
        & \multirow{1}{*}{\shortstack{Flan-T5-Large}}
        & \multirow{1}{*}{\shortstack{UD-Large}}
        & \multirow{1}{*}{\shortstack{T0-XL}}
        & \multirow{1}{*}{\shortstack{Flan-T5-XL}}
        & \multirow{1}{*}{\shortstack{UD-XL}}
        & \multirow{1}{*}{\shortstack{T0-XXL}}
        & \multirow{1}{*}{\shortstack{Flan-T5-XXL}}
        & \multirow{1}{*}{\shortstack{UD-XXL}}
        & \multirow{1}{*}{\shortstack{UD+-XXL}}\\
    \midrule[1pt]
    BBH (Avg.) & 38.9 & 39.5 & \textbf{44.2} & 40.4 & 44.6 & \textbf{47.3} & 45.0 & 49.4 & 51.3 & \textbf{56.7} \\
    \bottomrule[1pt]
    \end{tabular}%
    }
\label{tab:maintable:bottom}
}
\caption{
Zero-shot performance of our UD and baselines.
Results in the first block are reported by previous work, respectively from GPT-3~\cite{gpt3-paper}, GLaM~\cite{glam}, PaLM~\cite{palm}, and FLAN~\cite{FLAN}.
Note that we provide these reported results for reference, and do not compare directly. Some of the reported tasks are evaluated on the test split, while we follow the better baseline method T0 to report on validation splits.
Results with $\dagger$ are reported by~\citeauthor{T0-paper}, and results with $\star$ are reproduced in our framework. We reproduced the three variants of prompting ELECTRA~\cite{xia2022prompting} under our setting, denoted as ``PE-CLS'', ``PE-PROB'', ``PE-REP''.
Results for Flan-T5-Large/Xl/XXL~\citep{flant5} are reproduced by testing zero-shot performance on their released checkpoints.
In the same group, T0 and Flan-T5 has 2x model parameters compared to UD. For abbreviation, we denote UD based on T5-XX as ``UD-XX'', e.g., UD-XL refers to UD based on the T5-XL model.
}
\label{tab:maintable}
\vspace{-0.7cm}
\end{table*}
% \begin{table}[t]
\setlength{\tabcolsep}{4.5mm}
\centering
    \resizebox{0.5\textwidth}{!}{%
    \begin{tabular}{lcccc}
        \toprule[1pt]
        % \textbf{Finetuned Task} & \textbf{Task Type} & \textbf{Metric} & \textbf{Eval Set} & \textbf{SOTA Reference} & \textbf{SOTA} & \textbf{Ours}  \\
        \textbf{Finetuned Task} & \textbf{T0} & \textbf{UD}  \\
        \midrule[1pt]
        MRPC & 90.5 & 89.7 (to be improve) \\
        QQP & 85.9 (to improve) & \textbf{91.6} \\
        PAWS & 95.1 & \textbf{97.2} \\
        WikiQA  & 96.1 & \textbf{96.5}\\
        CosmosQA & 88.4 & \textbf{90.7}\\
        DREAM & 90.5 & \textbf{91.6} \\
        QuAIL & 65.6 & \textbf{80.2} \\
        QuaRel & 88.2 & \textbf{95.3}\\
        QuaRTz & 94.1 & \textbf{94.5} \\
        SciQ & 97.6& \textbf{98.1}\\
        SocialIQA &  \textbf{82.2} & 81.7 \\
        WikiHop & & 58.6\\
        Amazon & \textbf{97.6}(to improve) & 97.3 \\
        IMDB & \textbf{96.9} & 96.7\\
        Rotten & 93.4 & \textbf{93.6} \\
        Yelp & 72.28 (first two prompts) & 68.1 (hard to improve)\\
        AGNews & 95.1 & \textbf{95.3} \\
        DBPedia & & \\
        TREC & 96.7 & \textbf{97.8}\\
        \bottomrule[1pt]
    \end{tabular}
    }
    \caption{Results on finetuned tasks for UD and the baseline T0. Both methods use T5-XXL as a base model. T0 has 2x model parameters compared to UD.}

    % compared with state-of-the-art results.}
    \label{tab:finetunedtasks}
\end{table}
\begin{table*}[!htp]
\setlength{\tabcolsep}{1.5mm}
\centering
\subtable[On 11 discriminative test tasks following the T0 benchmark.]{
\resizebox{\textwidth}{!}{%
    \begin{tabular}{l|ccccc|ccc|cc|c|c}
        \toprule[1pt]
        \multirow{2}*{Method}
        & \multicolumn{5}{c|}{\textbf{Natural Language Inference}} & \multicolumn{3}{c|}{\textbf{Sentence Completion}} & \multicolumn{2}{c|}{\textbf{Coreference}} & \multicolumn{1}{c|}{\textbf{WSD}} & \multirow{2}{*}{Avg.} \\
    & RTE & CB & ANLI1 & ANLI2 & ANLI3 & COPA & Hella. & Story. & WSC & Wino. & WiC &  \\
    \midrule[1pt]
    T0-XL %
        & \textbf{79.7}	& 68.9	& 43.1	& 38.5	& 42.3	& \textbf{94.1}	& 31.5	& \textbf{97.5}	& \textbf{68.8}	& 61.3	& \textbf{54.1}	& 61.8\\
    GenUD-XL %
        & 71.5	& \textbf{80.4}	& \textbf{43.1}	& \textbf{39.5}	& \textbf{42.6}	& 94.0	& \textbf{55.8}	& 96.7	& 63.5	& \textbf{75.5}	& 52.8	& \textbf{65.0}\\
    \bottomrule[1pt]
    \end{tabular}%
}
\label{tab:genud:top}
}
\subtable[On 13 discriminative Big-Bench tasks following the T0 benchmark.]{
\resizebox{\textwidth}{!}{%
    \begin{tabular}{l|ccccccccccccc|c}
    \toprule[1pt]
    \multirow{2}{*}{Model} 
        & \multirow{2}{*}{\shortstack{code \\ desc.}}
        & \multirow{2}{*}{\shortstack{conce\\-ptual}}
        & \multirow{2}{*}{\shortstack{known\\unknowns}}
        & \multirow{2}{*}{\shortstack{logic \\ grid}}
        & \multirow{2}{*}{\shortstack{logic \\ deduction}}
        & \multirow{2}{*}{\shortstack{miscon\\-ceptions}}
        & \multirow{2}{*}{\shortstack{novel\\concepts}}
        & \multirow{2}{*}{\shortstack{strate\\-gyqa}}
        & \multirow{2}{*}{\shortstack{wino\\-why}}
        & \multirow{2}{*}{\shortstack{syllo\\-gisms}}
        & \multirow{2}{*}{\shortstack{movie\\dialog}}
        & \multirow{2}{*}{\shortstack{lang\\-uage\_id}}
        & \multirow{2}{*}{\shortstack{vita\\-minc}} 
        & \multirow{2}{*}{Avg.} \\
    &&&&&&&&&&&&&&\\
    \midrule
    T0-XL & 23.4 & 48.1 & 64.6 & \textbf{42.5} & 50.1 & \textbf{52.7} & 25.0    & 53.1 & 45.4 & 50.2 & 47.7 & \textbf{19.0} & 60.0 & 44.8 \\
    GenUD-XL & \textbf{60.0} & \textbf{64.1} & \textbf{69.6} & 38.2 & \textbf{52.8}  & 48.9 & \textbf{44.1} & \textbf{57.1} & \textbf{46.5} & \textbf{50.4} & \textbf{50.9} & 15.5 & \textbf{66.8} & \textbf{48.9} \\
    \bottomrule[1pt]
    \end{tabular}%
\label{tab:genud:mid}
}
}




\subtable[On 15 generative tasks from Big-Bench]{
\resizebox{\textwidth}{!}{%
    \begin{tabular}{l|ccccccccccccccc|c}
    \toprule[1pt]
    \multirow{3}{*}{Model}
        & \multirow{3}{*}{\shortstack{auto \\ debugging}}
        & \multirow{3}{*}{\shortstack{simple \\ arith \\ -metic}}
        & \multirow{3}{*}{\shortstack{repeat\\copy \\ logic}}
        & \multirow{3}{*}{\shortstack{sufficient \\ information}}
        & \multirow{3}{*}{\shortstack{simple \\ text \\ editing}}
        & \multirow{3}{*}{\shortstack{scientific \\ press \\ release}}
        & \multirow{3}{*}{\shortstack{code\\ names}}     
        & \multirow{3}{*}{\shortstack{emoji\\movies}}
        & \multirow{3}{*}{\shortstack{penguins\\in a \\ table}}
        & \multirow{3}{*}{\shortstack{few \\ shot\\nlg}}
        & \multirow{3}{*}{\shortstack{operators}}
        & \multirow{3}{*}{\shortstack{tense}}
        & \multirow{3}{*}{\shortstack{geometric\\shapes}}
        & \multirow{3}{*}{\shortstack{chinese \\ remainder\\ theorem}}
        & \multirow{3}{*}{\shortstack{temporal\\sequences}}
        & \multirow{3}{*}{\shortstack{Avg.}}\\
    &&&&&&&&&&&&&&&&\\[1em]
    \midrule
    T0-XL & 11.2 & 6.7 & \textbf{25.8} & 33.8 & 7.5 & \textbf{6.7} & \textbf{44.8} & \textbf{8.7} & \textbf{11.4} & 17.4 & \textbf{10.5} & 80.7 & 0.0 & 0.0 & 14.0 & \textbf{18.6}\\
    GenUD-XL & \textbf{15.5} & 6.7 & 8.2 & \textbf{34.4} & \textbf{12.6} & 6.4 & 25.1 & 0.0 & 8.1 & \textbf{20.5} & 3.7 & \textbf{80.9} & 0.0 & 0.0 & \textbf{33.5}  & 17.0\\
    \bottomrule[1pt]
    \end{tabular}%
}
}

\caption{Zero-shot performance for generalized UD and T0 on discriminative and generative tasks. 
We select the top 15 uncommon generative tasks from BigBench basing on ascending order of data size. (We assume that datasets with smaller sizes are less common, and more suitable for zero-shot tests.) The metrics are respectively accuracy for discriminative tasks and ROUGE1 for generative tasks. ``GenUD'' denotes our generalized UD method.}
\label{tab:genud}
\end{table*}







\begin{table}[htbp]
\setlength{\tabcolsep}{1.5mm}
  \centering
\resizebox{0.35\textwidth}{!}{
    \begin{tabular}{lcc}
    \toprule
    \textbf{Dataset} & \textbf{SOTA} & \textbf{UD+-XXL} \\
    \midrule
    QQP     & \textbf{90.60}  & 90.44 \\
    DREAM     & 91.80  & \textbf{94.95} \\
    QuAIL   & 87.20  & \textbf{88.13} \\
    IMDB    & 97.30  & \textbf{97.44}  \\
    AgNews   & \textbf{95.58}  & 95.56  \\
    OBQA   & 87.20  & \textbf{89.20} \\
    STSB     & 92.30  & \textbf{92.90} \\
    CSQA    & \textbf{84.90}  & 84.68  \\
    SST-2     & 97.30  & \textbf{97.48} \\
    QNLI    & 96.50  & \textbf{96.56} \\
    AbductiveNLI &  89.80  & \textbf{93.20} \\
    VitaminC   & 91.10  & \textbf{92.62} \\
    MNLI  &  \textbf{92.10}  & 92.03  \\
    MCScript &  97.30  & \textbf{98.03} \\
    MCScript 2.0 &  97.90  & \textbf{98.01} \\
    AdversarialNLI (r3) &53.50  & \textbf{67.83 } \\
    COLA   & \textbf{71.50}  & 71.42  \\
    \midrule
    Avg.   & 89.05  & \textbf{90.62} \\
    \bottomrule
    \end{tabular}%
}
  \caption{Results on fully-supervised tasks for UD, which is based on the encoder of T5-xxl. Previous sota model \citep{ul2} has 4x model parameters compared to UD. }
  \label{tab:finetune}%
\vspace{-0.7cm}
\end{table}%



\section{Experiments}

\begin{table*}[t]
\setlength{\tabcolsep}{1.5mm}
\centering
\small
\resizebox{\textwidth}{!}{%
    \begin{tabular}{l|ccccc|ccc|cc|c|c}
        \toprule[1pt]
        & \multicolumn{5}{c|}{\textbf{Natural Language Inference}} & \multicolumn{3}{|c|}{\textbf{Sentence Completion}} & \multicolumn{2}{c|}{\textbf{Coreference}} & \multicolumn{1}{c|}{\textbf{WSD}} & \multirow{2}{*}{Avg.} \\
    & RTE & CB & ANLI1 & ANLI2 & ANLI3 & COPA & Hella. & Story. & WSC & Wino. & WiC &  \\
    \midrule[1pt]
    UD (Minimal)     & \textbf{83.75}
        & \textbf{80.36}
        & 36.80
        & \textbf{34.20}
        & \textbf{42.17}
        & \textbf{90.00}
        & \textbf{56.07}
        & \textbf{96.37}
        & \textbf{68.27}
        & \textbf{62.90}
        & \textbf{54.55}	
        & \textbf{64.13} \\
    UD (Instructive)    & 72.24 
        & 64.52 
        & \textbf{36.98} 
        & 33.40 
        & 39.73 
        & 85.31 
        & 45.15 
        & 96.01 
        & 65.38 
        & 53.94 
        & 50.94 
        & 58.51\\
    \midrule
    T0 (Minimal) & 61.56  & \textbf{57.81}  & 30.57  & 30.27  & 33.38  & 67.19  & \textbf{33.81}  & 66.56  & 60.94  & 52.81  & \textbf{51.72}  & 49.69  \\
    T0 (Instructive) & \textbf{75.05}	& 55.48	& \textbf{32.87}	& \textbf{32.29}	& \textbf{33.67}	& \textbf{84.59}	& 28.24	& \textbf{93.97}	& \textbf{62.98}	& \textbf{54.59}	& 51.16	& \textbf{54.99} \\

    \bottomrule[1pt]
    \end{tabular}}
    \caption{Zero-shot performance for UD and T0 respectively with instructive and minimal prompts. Instructive prompts are lengthy descriptions of tasks \citep{T0-paper}, while minimal prompts use a simple concatenation of input data.}
\label{tab:promptablatiion}
\end{table*}

\begin{table*}[ht]
\setlength{\tabcolsep}{0.9mm}
\centering
\resizebox{\textwidth}{!}{%
    \begin{tabular}{l|l|ccccc|ccc|cc|c|c}
        \toprule[1pt]
        & \multirow{2}*{Base Model}
        & \multicolumn{5}{c|}{\textbf{Natural Language Inference}} & \multicolumn{3}{|c|}{\textbf{Sentence Completion}} & \multicolumn{2}{c|}{\textbf{Coreference}} & \multicolumn{1}{c|}{\textbf{WSD}} & \multirow{2}{*}{Avg.} \\
    & & RTE & CB & ANLI1 & ANLI2 & ANLI3 & COPA & Hella. & Story. & WSC & Wino. & WiC &  \\
    \midrule[1pt]
    \multirow{2}*{\shortstack{Encoder}}
    & DeBERTa-V3 (304M) 
        & 71.1
        & 76.8
        & 43.8
        & 41.3
        & 45.7
        & 96.0
        & 60.7
        & 97.4
        & 66.4
        & 83.6
        & 53.3
        & 66.9 \\
    & DeBERTa-V2 (1.5B) 
        & 77.6
        & 80.4
        & 43.2
        & 39.3
        & 44.8
        & 95.0
        & 67.2
        & 98.2
        & 74.0	& 82.1 & 56.0	& 68.9\\ \midrule
    \multirow{2}*{\shortstack{Enc-Dec}} & T5-Encoder (400M) 
        & 75.1	& 55.5	& 32.9	& 32.3	& 33.7	& 84.6	& 28.2	& 94.0	& 63.0	& 54.6	& 51.2	& 55.0 \\
    & T5-Encoder (1.5B)  & 79.7	& 68.9	& 43.1	& 38.5	& 42.3	& 94.1	& 31.5	& 97.5	& 68.8	& 61.3	& 54.1	& 61.8\\
    \midrule
    \multirow{1}*{\shortstack{Decoder}}
    & \multirow{1}*{GPT-XL (1.5B)}
        & \multirow{1}*{71.1}
        & \multirow{1}*{75.0}
        & \multirow{1}*{30.4}
        & \multirow{1}*{31.8}
        & \multirow{1}*{37.8}
        & \multirow{1}*{71.0}
        & \multirow{1}*{40.9}
        & \multirow{1}*{87.7}
        & \multirow{1}*{62.5}
        & \multirow{1}*{54.5}
        & \multirow{1}*{50.3}
        & \multirow{1}*{55.7}
    \\
    \bottomrule[1pt]
    \end{tabular}}
    \caption{Ablation study on different backbone models. We experiment with base models of different architectures and scales. ``Enc-Dec'' refers to models that are pretrained in an encoder-decoder manner.}
    \label{tab:ablationbasemodel}
\end{table*}
\begin{table}
\centering
\setlength{\tabcolsep}{3.0mm}
\resizebox{0.5\textwidth}{!}{%
\begin{tabular}{l|c}
    \toprule[1pt]
    Setting & Accuracy \\
    \midrule[1pt]
    True Data vs Manually-Generated Data & 80.0 \\
    True Data vs Model-Generated Data & 74.4 \\
    \bottomrule[1pt]
    \end{tabular}%
    }
    \caption{
    The accuracy of UD discriminating real data and generated data. We feed UD with a real sample $x$ from the real-world data distribution, and a sample $x'$ from manual generation or model-based generation. 
    If UD assigns higher score to $x$ than $x'$ (i.e., $D(x)>D(x')$), it is considered an accurate prediction.
    }
  \label{tab:explain}%
\end{table}%



\subsection{Experimental Setup}\label{sec:setup}

We performed extensive experiments to validate the performance of the zero-shot generalization of our UD. We follow the same zero-shot setting as T0~\citep{T0-paper} by training on multi-task datasets and evaluating a held-out set of tasks that are never seen during training. 

\paragraph{Datasets}
The original T0 training set consists of 38 tasks of 8 different types.
% ~\footnote{We did not consider T0+ and T0++, since they are partially intersected with the test sets, making some test tasks unable to be evaluated under the zero-shot setting.}
There are in total 21/38 discriminative training tasks, with which we train the UD.
% ~\footnote{The original paper~\citep{T0-paper} claims 39 training datasets but releases a training set with 38 datasets (``common\_gen''  excluded). We directly start with the released data.}. 
% It consists of a majority of discriminative tasks and a small number of generative tasks.
% We train our \method with 21 discriminative tasks within.
% , which is around 55\% of the original T0 training data.\xhk{change to 21/38=0.55}
The evaluation set covers four types of tasks, including natural language inference (RTE~\citep{2005_RTE}, CB~\citep{de2019_CB}, ANLI/R1-R3~\citep{NieWDBWK20_ANLI}), coreference resolution (WSC~\citep{WSC2012}, Winogrande~\citep{SakaguchiBBC20_winogrande}), sentence completion (COPA~\citep{COPA2011}, StoryCloze~\citep{story_cloze}, Hellaswag~\citep{ZellersHBFC19_hellaswag}), and word sense disambiguation (WiC~\citep{wic-paper}).
Following T0, we use accuracy on the validation split as the evaluation metric.
For prompt-based baselines, we report the average accuracy over multiple prompts for each test task.
Besides, we also evaluate zero-shot performance on several BigBench~\cite{bigbench} tasks, which are also adopted by T0~\cite{T0-paper}.\footnote{The original T0 reported results on 14 BigBench tasks. We separately report the results of 13 discriminative tasks and the other generative task in the following.}


% \lzy{Noted that we also evaluate the zero-shot performance on a subset of Big-Bench Benchmark~\cite{bigbench} adopted by original T0 paper~\cite{T0-paper}.\footnote{The original T0 reported results on 14 BigBench tasks. In our work, we focus on 13 discriminative tasks, leaving improving performance of the only generative tasks for future exploration. \zy{No need to say that. We also have generation results. Just say we're gonna report generation result separately.}}}


\paragraph{Baselines}
We primarily compare our method with T0~\citep{T0-paper}, which is a generative approach.
% that shares the same goal as UD (i.e., zero-shot generalization), but uses a totally different framework (i.e., generative or discriminative) as well as input format (i.e., prompt or minimal prompt).
Another baseline is prompting ELECTRA~\cite{xia2022prompting} which is a recent work on discriminative modeling.
Since it was proposed in a different setting (i.e., a  few-shot setting or direct zero-shot inference without any finetuning), we reproduced their method under our multitask zero-shot setting for comparison.

For a fair comparison, we follow T0 to use the T5-V1.1-LM-Adapted~\citep{T5-paper} as the backbone model, and we experimented with three different scales, respectively 800M, 3B, and 11B. 
For UD, it only makes use of the encoder of T5-v1.1 and additionally replaces the output layer with a classification head.
Moreover, for direct comparison with \citet{xia2022prompting}, we use DeBERTaV3-Large \citep{debertav3} as the backbone model which shares the same bidirectional architecture and has a smaller number of parameters.

In addition, we also provide reported zero-shot results of several large language models (with hundreds of billions of parameters) for reference, including GPT-3~\cite{gpt3-paper}, GLaM~\cite{glam}, PaLM~\cite{palm}, and FLAN~\cite{FLAN}.


% we also experiment with another backbone DeBERTaV3-Large~\citep{debertav3} to achieve better zero-shot performance.

\paragraph{Training}
% We implemented both baselines and our method, and perform experiments with exactly the same environments.
During training, we truncate the input sequence to 256 tokens and use a batch size of 256. For optimization, we use the Adam optimizer with a fixed learning rate of 1e-5 and a dropout rate of 0.1. Each experiment is trained with 10, 8, and 5 epochs respectively for 800M, 3B, and 11B models.
% We perform checkpoint selection by directly using the final (fixed-epoch) checkpoint for evaluation.
% \xhk{by choosing the one with the maximal average zero-shot performance per xxx steps.}

% For data processing, similar to T0, we truncate any dataset with over MAX\_DATA\_SIZE to have MAX\_DATA\_SIZE / num\_prompts. 
% Different from ~\citet{T0-paper} that uses a value of 500k for MAX\_DATA\_SIZE, we use a value of 50k, which experimentally yields better zero-shot performance for the T0 baseline.
% The training data of UD are produced by \xhk{replacing different prompted data version from T0 training data with only one minimal prompted version}, which strictly guarantees all methods share same raw task data.




% \subsection{Main Results}
\subsection{Main Results on Zero-Shot Tasks}

\paragraph{UD Zero-Shot Results}
The main results are presented in Table~\ref{tab:maintable}.
We compare methods of similar scales. 
Results in Table \ref{tab:maintable:top} show that our UD substantially outperforms the T0 baseline on average by a large margin of around 9, 5, and 7 points respectively at Large, XL, and XXL scales.
Comparing the results of UD-T5-Large, UD-DeBERTaV3, and prompting ELECTRA, both variants of UD also substantially outperform prompting ELECTRA by more than 6 points.
% In addition, UD also demonstrates superior zero-shot ability compared with models with hundreds of billions of parameters (see results in the first block of Table~\ref{tab:maintable:top}).
On BIG-Bench datasets, results in Table \ref{tab:maintable:bottom} show that our UD outperforms the T0 baseline by a margin of around 4-8 points.
Overall, these results demonstrate the advantages of UD at every scale, and a broad range of tasks compared with baselines.

Another interesting finding is that the advantages of UD significantly increase along with scaling.
When scaling from Large-scale to XL-scale (i.e., around 3.75x of the parameters), the average performance improves by around 2 points. However, when scaling from XL-scale to XXL-scale (i.e., 3.6x of the parameters), the improvements of average zero-shot performance enlarge to 8 points.
Based on the observation, we hypothesize that UD can achieve even better performance of zero-shot generalization if further scaling to an even larger models, which we leave to future work.

% Results show that our \method substantially outperforms our baseline T0 on average zero-shot performance, by a large margin of around 8, 5, and 7 points respectively at the Large (800M), XL (3B), and XXL (11B) scales.
% \xhk{Our UD also substantially outperforms ELECTRA in the Large (800M) scale.} 

To further boost the zero-shot performance, we also train a new variant of UD at 11B scale by scaling to more training tasks, including the discriminative English tasks used in \citet{1600tasks}, and the discriminative English tasks used in \citet{ul2}. The new model is denoted as UD+.
UD+ achieves the highest average accuracy among all the zero-shot evaluation tests.

% \begin{comment}
% ul2 (CommonsenseQA \cite{commonsense_qa}, ),
% csqa2.json 9264
% glue_cola.json 8551
% glue_sst2.json 67349
% glue_stsb.json 5749
% mcscript.json 19462
% mcscript2.json 28382
% openbookqa.json 19828
% qasc.json 40670
% qasc_with_ir.json 40670
% race_high.json 249780
% race_middle.json 101684
% social_i_qa.json 100230
% super_glue_boolq.json 9427
% super_glue_multirc.json 27243
% ai2_science_elementary.json 2493
% ai2_science_middle.json 2424
% onestopqa_advanced.json 1296
% physical_iqa.json 33211
% protocol_comparison_harsht.json 3698
% reclor.json 3726
% ai2_arc_ARC_Easy.json 9002
% ai2_arc_ARC_Challenge.json 4476r,  
% \end{comment}
% \zy{what data?}

% \xhk{move bigbench result in appendix A.1 here. into Table 2}
% \yn{add bigbench analysis}

% , in the mean time still guaranteeing that there is no overlap between training tasks and the held-out tasks.

% \xhk{We also extend our training datasets (please refer to appendix~\ref{sec:ud_plus_data}) and train a model UD+. 
% UD+ achieves the highest average accuracy among all the zero-shot evaluation test, in the mean time still guaranteeing that there is no overlap between training tasks and the held-out tasks.}

% \yn{do we need to add the following?}
% \yn{
% Interestingly, we also have observed some findings on zero-shot performance along with scaling.
% For baseline T0, the zero-shot performance keeps improving on most of the datasets when scaling to larger-scale models.
% Exceptions are Hellaswag and WSC, where zero-shot performance on them are basically unchanged when scaling.
% For our \method, the performance of zero-shot generalization consistently improves with the model scale increasing on all sentence completion and coreference resolution tasks, and partial NLI tasks.
% Exceptions are that RTE, CB and WSC demonstrates a degradation on zero-shot performance when scaling from large to XL scale.
% This could be explained that 
% % {\color{red} xxxxx}
% % \xhk{I guess if we add minimal prompts, RTE and CB will improve for larger model...? so maybe we can remove this observation for now?}
% }


% \paragraph{Results on Finetuned Tasks}

% To evaluate the performance on finetuned tasks, we finetuned T0/UD respectively on each training task. This is similar to multi-task finetuning \cite{T5-paper}.
% % We use this experiment to test the effectiveness of UD with abundant labels. 
% We experimented with all the T0 discriminative training tasks.
% Table~\ref{tab:finetunedtasks} shows the finetuning results on T0 and UD at the 11B scale.
% We observe that UD outperforms T0 on \textbf{\color{red} xxx/19} of the considered finetuned tasks.
% To be specific, on topic classification tasks, paraphrase identification tasks, and multiple-choice QA tasks, UD shows the largest advantages against T0. These finetuning results demonstrate that UD does not only perform well in the zero-shot setting but also improves performance when abundant labels are available.



% \yn{add a new subsection of seq2seqUD}
\paragraph{Generalized UD Zero-Shot Results}

The zero-shot results of generalized UD on 11 T0 discriminative test tasks and on 13 Big-Bench tasks are respectively reported in Table~\ref{tab:gen_ud:top} and Table~\ref{tab:gen_ud:mid}.
In addition, to test how generalized UD performs on zero-shot generative tasks, we also select 4 generative tasks from Big-Bench for evaluation. Results are presented in Table~\ref{tab:gen_ud:bottom}.


Analyses are as follows.
(1) Comparing the results of generalized UD and T0, generalized UD still holds significant improvements on discriminative tasks.
(2) Comparing generalized UD with our previous UD (in Table~\ref{tab:maintable}), we observe there is a slight decrease in average performance, proving that adding generative tasks into training could have impacted a little bit, in trade for capability for handling generative tasks.
(3) On 4 generative zero-shot tasks, both generalized UD and T0 show comparable results.
(4) On 13 discriminative BigBench tasks, we observe that UD-Large outperforms T0-Large by 6.67\%, UD-XL outperforms T0-XL by over 4\%, and Generalized UD-XL outperforms T0-XL by over 6\%, further indicating the effectiveness of our proposed framework.


%\yn{recheck the analysis along with table data!}

% Comparing Generalized UD with methods in Table~\ref{tab:maintable} (i.e., UD and T0) of similar scales, we observe the zero-shot performance on discriminiative tasks slightly decrease but generally hold still, compared to UD (ours).

% It still significantly outperforms baseline T0 to a large degree.
% From Table~\ref{tab:gen_ud}, we shall observe, on generative tasks both generalized UD and T0 show comparable results.}









\subsection{SOTA Results on Finetuned Tasks}
\label{sec:ud_finetune}

To explore how UD performs on fully-supervised tasks, we finetuned UD for a wide range of downstream tasks and reported their results in Table \ref{tab:finetune}.
% To explore whether UD can help improve the performance in fully-supervised learning, we conduct experiments by finetuning each downstream task. 
For each finetuning experiment, the maximum training epoch is set to be 10.
We search a hyper-parameter space with learning rate in \{2e-5, 1e-5, 5e-6\}, batch size in \{32, 64, 128\}.
We select the best checkpoint using a validation set with early stopping.
% We set the maximum training epoch to 10, search the hyper-parameters (learning rate in \{2e-5, 1e-5, 5e-6\}, batch size in \{32, 64, 128\}) and select the best checkpoint based on the validation set with early stopping.

% Results are in Table \ref{tab:finetune}.
From results in Table \ref{tab:finetune}, we find that UD can achieve remarkable performance on most of the downstream tasks. 
We achieve state-of-the-art performance on 12 out of the 17 tasks we evaluated. The results also show that more challenging tasks (tasks that require more knowledge) will benefit more from the multi-task training period, especially some QA tasks.








\subsection{Ablation Study}

We have also conducted ablation studies to further explore how several factors affect the performance of zero-shot generalization. 

\subsubsection{Instructive Prompts vs Minimal Prompts}

UD employs minimal prompts that use simple concatenation, while previous approaches rely on lengthy instructive prompts to provide more detailed instructions \cite{T0-paper,FLAN,gpt3-paper}. 
Statistically, we count the average number of prompt words (excluding raw input) for both minimal and instructive prompts, and statistics are respectively $0.4$ versus $>10$.
% \xhk{A statistic comparison on the average number of the prompt word count (excluding raw input) is $0.4$ for minimal prompts versus $>10$ for previous instructive prompts.}  
We compare these two types of prompts in the following experiment.
We adopt the instructive prompts from T0 and apply them on UD without changing the discriminator formulation. To construct minimal prompts for T0, we remove all the instructive words similar to UD.

% It is an interesting question whether minimal prompts also play a role in the \method, 
%considering that concatenating task data with prompts theoretically indeed reduces all tasks into the original LM tasks, hence improving task generalization.
% considering that simple concatenation of task data's keywords with a minimal prompt is enough to unify it into the UD format.

% We compare the zero-shot performance when using prompt and minimal prompt for \method and prompt and prompt-free for T0. 



% To construct instructive prompts for UD, we adopt the instructive prompts from T0 we concatenated the prompted inputs and each target choice (verbalizer).~\footnote{Here we use the same prompts as T0.} The corresponding \method label is 1 when concatenating correct target choice and 0 otherwise.

% \xhk{To construct prompt-free inputs for T0, we directly remove all the prompt words, still letting the model to predict the target verbalizer.}


Results are shown in Table~\ref{tab:promptablatiion}. We observe that minimal prompts yield better performance for UD than instructive prompts. In contrast, for T0, instructive prompts perform much better than minimal prompts. These results are consistent with our motivation that UD tends to unify the tasks better with a shared discrimination formulation. As a result, task-specific instructions are not necessary and might hurt generalization performance. Generative approaches, on the other hand, rely on instructive prompts to better distinguish different tasks.


% \xhk{for our UD method, minimal prompt version has better accuracy, because under the unified UD task format, task descriptive language in prompt is no longer needed and may even increase the sentence complexity to be understood by LM. Additionally, UD's tasks is to discriminate between correct and wrong choices where prompts are identical phrases in each choice's concatenated sentence, so prompts actually play no role in the discriminating process. However, for T0, prompted version has better accuracy than the prompt-free version (note that prompt-free is the extreme and usual case for minimal prompting) because generative model's goal is to generate the correct verbalizer from the huge vocabulary, which can be efficiently narrowed by the existence of prompts. Therefore, we can conclude that minimal prompted format works well for discriminative models and prompted format works well for generative models.}



\subsubsection{Ablation on Base Models}

We also study the effects of using different backbone pretrained models. We experiment with three backbone models of different types, respectively the encoder part of an encoder-decoder model, an encoder model, and a decoder model. Specifically, we use the T5 encoder, DeBERTa \cite{debertav3}, and GPT \cite{radford2018gpt} respectively for these three types. It is noteworthy that though similar in architecture for both T5 encoder and DeBERTa, they are pretrained with different self-supervised language modeling tasks, which in fact leads to huge differences in zero-shot generalization, as we will show in Table~\ref{tab:ablationbasemodel}.
% We study the effect of different backbone pretrained models. We experiment with three types of backbone models---using the encoder part of an encoder-decoder model, using an encoder model, and using a decoder model. We use the T5 encoder, DeBERTa \cite{debertav3}, and GPT \cite{radford2018gpt} respectively for these three types.






% We study the effect of different types of models (discriminative vs. generative), or backbone models (auto-encoding vs. auto-regressive), on zero-shot generalization with \method. In addition to T5-Encoder, we also experiment the advanced DeBERTaV3-Large~\cite{debertav3} that has achieved new SOTA on a diverse set of tasks. We also implement GPT-XL for comparision.

% Results are shown in Table~\ref{tab:promptablatiion}.

% shows the results between discriminative and generative models with fixed prompted or not version. It can be observed \xhk{no matter we use promped data or minimal prompt/prompt-free data, our discriminative models always have better zero-shot generalization performance than generative models.}



Results of different backbone models are presented in Table \ref{tab:ablationbasemodel}. 
Among all three types of backbone models, the encoder backbone models appear to be the most suitable type of backbone, where both encoder models of two scales respectively achieve the best and the second best results, outperforming all the others by more than 5 points.

Using the same number of parameters (i.e., 1.5B), both DeBERTa-V2 and T5-Encoder significantly outperform GPT-XL, which demonstrates that a bidirectional architecture works better than the unidirectional architecture for the discriminator formulation.
In addition, DeBERTa-V2 outperforms T5-Encoder by 7 points, implying that not only model architecture but also the self-supervised pretraining task determines the ability of UD discrimination. Models pretrained with masked language modeling tasks are more suitable for UD.

The impacts of the architecture and pretraining tasks of backbone models are even larger than the influence of scale, as we also observe that an encoder model with 300M parameters (i.e., DeBERTaV3) achieves much better performance than the T5 encoder and GPT-XL with 1.5B parameters.

% Results are shown in Table \ref{tab:ablationbasemodel}. Using the same number of parameters, encoder backbone models (i.e., DeBERTa) substantially outperform the T5 encoder and the GPT decoder. This indicates that pretrained encoders are more suitable for our discriminator formulation. Interestingly, an encoder model with 300M parameters (i.e., DeBERTaV3) achieves much better performance than the T5 encoder and GPT-XL with 1.5B parameters.








% Table~\ref{tab:ablationbasemodel} shows the results for different discriminative models, where DeberTa consists of solely an encoder, T5-Encoder is the encoder part of the full T5 model, GPT-XL consists of an encoder and a decoder. We can observe that the encoder structure performs better for discriminative tasks.

% \subsection{What Contribute to the Zero-Shot Generalization of \method?}

\subsection{How Well UD Generalizes to a Broader Domain?} \label{sec:generalize}

In the previous sections, we have trained UD to solve the task of discriminating whether a text sample comes from the true data distribution of natural language. So far we have constrained the problem to supervised labeled tasks. However, this discrimination problem formulation is in fact general and can be applied to a broader domain of natural language. We conduct the following experiment to see how UD generalizes.


% In order to explore the mechanism of the universal discriminator and explain how it promotes zero-shot generalization. We conduct the following extensive experiment.

To test whether a model discriminates against the true data distribution, a straightforward way of verification is to compare the probability of real data with that of some generated, fake data. This form of verification is not specific to any downstream task and can be viewed as generalizing to a broader domain. Formally, given a text sample $x$, let $D(x)$ be the output of UD, which estimates the probability that $x$ is sampled from the true data distribution, i.e., $P(\text{true} | x)$. Given a true data sample $x$ and a generated data sample $x'$, we expect a well-trained UD to predict $D(x) > D(x')$.

% First, we assume that the essence of our universal discriminator $D$ is to learn whether the data are sampled from the real text distribution or not. A straightforward way to verify this key point is to compare the likelihood of the real data label given real data x computed as $D(x)=p(y=1|x)$ with the likelihood of the real data label given generated data $x'$ computed as $D(x’)=p(y=1|x’)$. 

Specifically, we randomly select 2,600 real data samples $x$ from the validation set of the T0 training data and generate the data $x’$ in two different ways: model-based generation and manual generation.

For a model-based generation, we utilize the T0-Large model with a paraphrase prefix ``Paraphrase the sentence:'' to generate data $x'$. It is expected that the generated samples $x'$ are similar to true samples $x$ to some extent but demonstrate some flaws that are unique to generated data. For a manual generation, we manually create some conflict or contradiction in the real sample $x$. Specifically, we manually attach wrong answers to the original data and obtain $x’$ , which is similar to what we have done in constructing negative samples in our main framework. 

We then use our \method based on T5-Encoder Large to compute the probability $D(x)$ and $D(x')$ for both real and generated data. As displayed in Table~\ref{tab:explain}, we find that the \method assigns a higher score for $x$ than $x'$ $80\%$ of the time for manually-generated data. When tested with model-generated data, UD assigns a high probability for real data in $74\%$ of the cases.
This is probably because manually generated data are more paradoxical and logically incoherent and thus are easier for UD to discriminate. Overall, these results demonstrate that the discrimination ability of UD is not limited to the downstream tasks on which it was trained, but is also generalizable to a broader domain of text data. This indicates a possibility of extending UD to other scenarios such as model pretraining and generation tasks.


% For model-based generation, we utilize two models which generate high-quality and low-quality data $x$. It should be noted that we hope the generated $x'$are similar to $x$ to some extent.
% First, we leverage a T5-small model [citation] to generate similar semantics to real data x by feeding the $x$ with the prefix  ‘paraphrase:’. Obviously, the generated $x'$ are bound to be far from real data distribution. Then, we utilize the T5-small model to finetune on quora for paraphrase identification task. Then we leverage the finetuned T5-small model to do the same paraphrase generation as before and yield $x’$ with relatively high quality. 

% For heuristic-based generation, we manually create some conflict or contradiction in the real data $x$. In detail, we randomly shuffle the words given each real data sample and get inconsistent data $x’$.

% After generating the data $x’$ from different approaches, we evaluate the likelihood of real data distribution given real data $x$ and generated data $x’$, which is formulated as $D(x)=p(y|x)$ and $D(x’)=p(y|x’)$ respectively. The results are shown in Table [reference] and the generated data examples are presented in Appendix [reference].





\section{Conclusion}
In this paper, we extend the idea of SynGEC \cite{zhang2022syngec} and propose the CSynGEC approach to enhance GEC models by exploiting tailored constituent-based syntax. Experimental results show that incorporating constituent-based syntax produced by a GEC-oriented constituency parser can effectively help GEC models. 
Furthermore, we attempt to combine dependency-based and constituent-based syntax from both intra-model and inter-model aspects, and find that simultaneously using two kinds of syntax leads to more obvious improvement.


% \newpage

% \section{Limitation}
Even though our generalized UD can get comparable performance on some generative tasks, generalized UD may not handle certain complex generation tasks very well (e.g., summarization) 
We leave expanding UD to solve a broader range of generative tasks and achieve greater performance advantage as our future work. 






% Entries for the entire Anthology, followed by custom entries
\bibliography{anthology}
\bibliographystyle{acl_natbib}

% \appendix

\section{Supplemental Tables}

%\section{Hyperparameters of Other Bandit Algorithms}
%\label{sec:bandit_hyperparams}
%Table~\ref{tab:hyperparams} lists the hyperparameters for bandit algorithms other than dBE.

\newcommand\topmidheader[2]{\multicolumn{#1}{c}{\textbf{#2}}\\%
                \addlinespace[1ex]}

\newcommand{\midheader}[2]{%
        \midrule\topmidheader{#1}{#2}}

\newcommand{\specialcell}[3][c]{% 
        \begin{tabular}[#1]{@{}#2@{}}#3\end{tabular}}%

\aptLtoX[graphic=no,type=env]{\begin{table}[htb]
  \centering
  \caption{Hyperparameters of bandit algorithms}
  \label{tab:hyperparams}
  \begin{tabular}{llc}
    \toprule
    Sign & Description & Value \\
    \multicolumn{3}{c}{\textbf{UCB1}}\\
    $c$ & Parameter to control the confidence level used in $\sqrt{c \cdot {\log{t}}/{N_t(arm)}}$ & 0.5  \\
    \multicolumn{3}{c}{\textbf{Thompson Sampling}}\\
    $p(\theta)$ & Prior Distribution & $\mathcal{B}(1, 1)$ \\
    \multicolumn{3}{c}{\textbf{discounted Thompson Sampling}}\\
    $\gamma$ & Discount factor & $1-10^{-8}$ \\
    \multicolumn{3}{c}{\textbf{discounted Thompson Samplingadaptive shrinking Thompson Sampling}}\\
    $M$ & Parameter to control memory usage in a data structure ADWIN2 \cite{ADWIN} & 10 \\
    $\delta$ & Parameter to control the confidence level in a data structure ADWIN2 & $1-10^{-7}$ \\
    \multicolumn{3}{c}{\textbf{EXP-IX}}\\
    $\eta_t$ & Parameter used for weights of arms & $\sqrt{\frac{2 \cdot \log{K}}{K \cdot t}}$ \\
    \addlinespace[1ex]
    $\gamma_t$ & Parameter used for loss estimates & $\frac{\eta_t}{2}$ \\
    \multicolumn{3}{c}{\textbf{EXP3++}}\\
    $\alpha$ & Constant used in calculating $\xi_t(a)$ & $3$ \\
    $\beta$ & Constant used in calculating $\xi_t(a)$ & $256$ \\
    \bottomrule
  \end{tabular}
\end{table}}{\begin{table}[htb]
  \centering
  \caption{Hyperparameters of bandit algorithms}
  \label{tab:hyperparams}
  \begin{tabular}{llc}
    \toprule
    Sign & Description & Value \\
    \midheader{3}{UCB1}
    $c$ & \specialcell{l}{Parameter to control the confidence \\ level used in $\sqrt{c \cdot {\log{t}}/{N_t(arm)}}$} & 0.5  \\
    \midheader{3}{Thompson Sampling}
    $p(\theta)$ & Prior Distribution & $\mathcal{B}(1, 1)$ \\
    \midheader{3}{discounted Thompson Sampling}
    $\gamma$ & Discount factor & $1-10^{-8}$ \\
    \midheader{3}{adaptive shrinking Thompson Sampling}
    $M$ & \specialcell{l}{Parameter to control memory usage \\ in a data structure ADWIN2 \cite{ADWIN}} & 10 \\
    $\delta$ & \specialcell{l}{ Parameter to control the confidence \\ level in a data structure ADWIN2} & $1-10^{-7}$ \\
    \midheader{3}{EXP-IX}
    $\eta_t$ & Parameter used for weights of arms & $\sqrt{\frac{2 \cdot \log{K}}{K \cdot t}}$ \\
    \addlinespace[1ex]
    $\gamma_t$ & Parameter used for loss estimates & $\frac{\eta_t}{2}$ \\
    \midheader{3}{EXP3++}
    $\alpha$ & Constant used in calculating $\xi_t(a)$ & $3$ \\
    $\beta$ & Constant used in calculating $\xi_t(a)$ & $256$ \\
    \bottomrule
  \end{tabular}
\end{table}}

\begin{table}[htb]
  \centering
  \caption{Commit IDs of the PUTs used in our vulnerability discovery and AFL++ used as the baseline.}
  \begin{tabular}{lc}
    \toprule
    Program & Commit \\
    \midrule

    AFL++ & 32a0d6ac315 (ver ++3.14c) \\
    Bloaty &  60209eb \\
    HarfBuzz & 77eeec5 \\
    libarchive & 86c9361 \\
       libxml2 & dea91c9 \\
    MuPDF & ef3d68d \\
   PHP & fdf0455f \\
    Poppler & 6d72d82 \\
    PROJ & 76dfefe \\
    QPDF &  3794f8e \\
    libtpm2 & bc3bb26 \\
    Wireshark  & 1fc621e \\
    Xpdf & N/A (ver 4.03) \\

    \bottomrule
  \end{tabular}
\label{tab:commit-ids}
\end{table}


\begin{table}[htb]
  \centering
  \caption{Initial and theoretical maximum values of code coverage of the PUTs in FuzzBench. 
           Initial values were investigated only in the PUTs used.}
  \begin{tabular}{lcc}
    \toprule
    PUT & Initial & Maximum \\
    \midrule

bloaty\_fuzz\_target & N/A & 83114 \\
curl\_curl\_fuzzer\_http & N/A & 78362 \\
freetype2-2017 & 1517 & 26262 \\
harfbuzz-1.3.2 & N/A & 12212 \\
jsoncpp\_jsoncpp\_fuzzer & N/A & 2114 \\
lcms-2017-03-21 & 149 & 7036 \\
libjpeg-turbo-07-2017 & N/A & 9384 \\
libpcap\_fuzz\_both & 2 & 7294 \\
libpng-1.2.56 & 138 & 3736 \\
libxml2-v2.9.2 & 258 & 67994 \\
libxslt\_xpath & N/A & 51456 \\
mbedtls\_fuzz\_dtlsclient & N/A & 12888 \\
openssl\_x509 & 6026 & 54116 \\
openthread-2019-12-23 & N/A & 19846 \\
php\_php-fuzz-parser & N/A & 215210 \\
proj4-2017-08-14 & 46 & 6534 \\
re2-2014-12-09 & 1 & 3982 \\
sqlite3\_ossfuzz & 4767 & 28766 \\
systemd\_fuzz-link-parser & N/A & 1798 \\
vorbis-2017-12-11 & 410 & 4082 \\
woff2-2016-05-06 & N/A & 5708 \\
zlib\_zlib\_uncompress\_fuzzer & N/A & 910 \\

    \bottomrule
  \end{tabular}
\label{tab:fuzzbench_max_cov}
\end{table}

\begin{table}[htb]
\centering
\caption{List of unique bugs found in the 7-day trial (manually triaged).}
\begin{minipage}{\columnwidth}

\centering
\begin{tabular}{lll}
\toprule

ID & PUT & Bug Type \\
\midrule
Bug-A & bloaty & NULL Pointer Deref \\
Bug-B & harfbuzz & Out-of-bounds Read \\
Bug-C & mupdf & Assertion Fail \\
Bug-D & mupdf & NULL pointer deref \\
Bug-E & xpdf & Stack Overflow \\
Bug-F & xpdf & NULL Pointer Deref \\
Bug-G \footnote{CVE-2022-24106 is issued.} & xpdf & Use of Uninitialized Value \\
Bug-H \footnote{CVE-2022-24107 is issued.} & xpdf & Integer Overflow \\
Bug-I & php & Use-After-Free \\
Bug-J & php & Use-After-Free \\
Bug-K & php & NULL Pointer Deref \\
Bug-L & php & Use-After-Free \\ 
Bug-M & php & NULL Pointer Deref \\
Bug-N & php & Assertion Fail \\
Bug-O & php & Use-After-Free \\
Bug-P & php & Use-After-Free \\
Bug-Q \footnote{CVE-2022-23308 is issued.} & libxml2 & Use-After-Free \\
\bottomrule
\end{tabular}

\label{tab:7d-bug}
\end{minipage}
\end{table}

\begin{table*}[htb]
  \centering
  \caption{List of the PUTs used in Section~\ref{sec:banditcomparison}. If the source code of a PUT was maintained in Git, the latest version at the time of the experiment in the master (or main) branch was used for the build. The `+' sign in a version indicates that the used source code is not the official release version of the source code.}
  \renewcommand\tabularxcolumn[1]{m{#1}}
  \renewcommand{\arraystretch}{1.2}
  \begin{tabularx}{\textwidth}{lXllXc}
    \toprule
    Project & Version & Commit ID & PUT & Format of Initial Seeds & Initial Edge Coverage \\
    \midrule
    Bloaty & v1.1+ & 60209eb & fuzz\_target & Executable (e.g., ELF, PE, Mach-O) & 4773\\
    libmpeg2 & N/A & 5432dc1 & mpeg2\_dec\_fuzzer & MPEG2 & 2428 \\
    PHP & 8.0+ & fdf0455f & php-fuzz-execute & PHP source code & 25241 \\
    HarfBuzz & 3.1.0 & 77eeec5 & hb-shape-fuzzer & Font (e.g., TrueType, OpenType) & 15298 \\
    Xpdf & 4.03 & N/A & fuzz\_pdfload & PDF & 4755 \\
    libtpm2 & N/A & bc3bb26 & tpm2\_execute\_command\_fuzzer & TPM command & 3884\\
    libyaml & v0.2.5+ & f8f760f & libyaml\_dumper\_fuzzer & YAML & 1310 \\
    libzip & 1.8.0+ & bff2eb9 & zip\_read\_fuzzer & ZIP & 805 \\
    libgit2 & v1.3.0+ & 50b4d53 & download\_refs\_fuzzer & Git packet & 3911 \\
    file & 5.41+ & fcbb5d8 & magic\_fuzzer & any (e.g., Zstd compressed file) & 1171 \\
%    MuPDF & 1.19.0+ & ef3d68d & pdf\_fuzzer & PDF & 16936 \\
%    libxml2 & 2.9.12+ & dea91c9 & xml & XML & 7027 \\
    \bottomrule
  \end{tabularx}
\label{tab:put_details}
\end{table*}

%\section{Full Results of Some Experiments}
%\label{sec:full_result}

%Table~\ref{tab:alg_cmp_all}, Figure \ref{fig:vis_bandits} and Figure \ref{fig:full_ablation_time_vs_cov} show the omitted results.

\begin{table*}[htb]
\centering
\caption{Median edge coverage obtained by AFL++ and 8 versions of \OurMethodName-AFL++ in 10 PUTs after 24 h. }

\begin{tabular}{lccccccccc}
\toprule

PUT & AFL++ & UCB1 & KLUCB & TS & dTS & dBE & ADS-TS & EXP3-IX & EXP3++ \\
\midrule

bloaty & \textit{1845.5} & 2198.5 & 2246.0 & 2232.5 & 2191.0 & 2292.0 & \textbf{2340.0} & 2181.5 & 2231.5 \\
harfbuzz & \textit{13497.5} & 14031.5 & 14247.5 & 14360.5 & \textbf{14374.0} & 14067.5 & 14149.0 & 13883.0 & 13891.0 \\
xpdf & \textit{3384.0} & 3494.0 & 3812.5 & \textbf{4618.5} & 4166.5 & 3791.5 & 3902.0 & 3860.0 & 3615.0 \\
libzip & \textit{267.5} & 272.0 & 274.0 & 268.0 & 268.5 & 271.5 & \textbf{276.0} & 271.5 & 268.0 \\
libgit2 & 898.0 & 888.5 & 890.5 & 906.5 & \textbf{916.0} & 884.0 & 914.0 & 899.5 & \textit{881.0} \\
php & \textit{9841.5} & 11861.0 & 13551.5 & \textbf{14324.0} & 14187.5 & 12657.5 & 13408.0 & 11423.5 & 11828.5 \\
libmpeg2 & \textit{1873.5} & 1900.5 & 1905.0 & 1905.5 & \textbf{1906.5} & 1903.0 & \textbf{1906.5} & 1897.0 & 1902.0 \\
tpm2 & \textit{281.5} & 299.5 & 313.0 & 317.0 & \textbf{317.5} & 305.0 & 311.0 & 298.5 & 291.0 \\
libyaml & 2811.5 & 2841.0 & \textbf{2841.5} & \textit{2800.5} & 2837.0 & 2827.5 & 2831.5 & 2828.0 & 2834.5 \\
file & 830.5 & 829.5 & 828.0 & 827.0 & 827.5 & 833.5 & \textbf{840.5} & 826.5 & \textit{826.0} \\

\bottomrule

\end{tabular}

\label{tab:alg_cmp_all}
\end{table*}

\begin{table*}[htb]
\centering
\caption{P-value of Mann-Whitney's U test (Holm-Bonferroni corrected) and Vargha-Delaney's $\hat{A}_{12}$ between AFL++ and the fuzzer in the column for the evaluation conducted in Section~\ref{subsec:eval-vs-existing}. If the p-value is bold, the difference is significant in the test ($p < 0.01$). The characters `L', `M', `S' and `N' in parentheses indicate that the effect size is large, medium, small, and none, respectively, according to \cite{A12}. The `+' sign means the fuzzer in the column is superior to AFL++ when compared by rank sum as well as $\hat{A}_{12}$, and the `-' sign means the opposite.}
\begin{tabular}{lllllllllllll}
 \toprule

  & \multicolumn{2}{c}{MOpt} & \multicolumn{2}{c}{CMFuzz} & \multicolumn{2}{c}{Karamcheti} & \multicolumn{2}{c}{\HavocMAB{}} & \multicolumn{2}{c}{SLOPT} \\
  \cmidrule(r){2-3}\cmidrule(r){4-5}\cmidrule(r){6-7} \cmidrule(r){8-9} \cmidrule(r){10-11}
  PUT & $p$ & $\hat{A}_{12}$ & $p$ & $\hat{A}_{12}$ & $p$ & $\hat{A}_{12}$ & $p$ & $\hat{A}_{12}$ & $p$ & $\hat{A}_{12}$ \\
\midrule

openssl\_x509 & \textbf{ < 0.001 } & 0.82 (+L) & \textbf{ 0.023 } & 0.71 (+L) & \textbf{ < 0.001 } & 0.92 (+L) & \textbf{ < 0.001 } & 0.82 (+L) & \textbf{ < 0.001 } & 0.91 (+L) \\
re2-2014-12-09 & \textbf{ < 0.001 } & 0.18 (-L) & > 0.1 & 0.37 (-S) & > 0.1 & 0.38 (-S) & > 0.1 & 0.47 (-N) & > 0.1 & 0.52 (+N) \\
proj4-2017-08-14 & \textbf{ < 0.001 } & 0.08 (-L) & \textbf{ < 0.001 } & 0.86 (+L) & \textbf{ < 0.001 } & 0.99 (+L) & > 0.1 & 0.54 (+N) & \textbf{ < 0.001 } & 0.92 (+L) \\
sqlite3\_ossfuzz & > 0.1 & 0.55 (+N) & \textbf{ < 0.001 } & 0.85 (+L) & \textbf{ < 0.001 } & 0.93 (+L) & 0.1 & 0.68 (+M) & \textbf{ < 0.001 } & 1.00 (+L) \\
libxml2-v2.9.2 & \textbf{ < 0.001 } & 0.08 (-L) & \textbf{ < 0.001 } & 0.93 (+L) & \textbf{ < 0.001 } & 0.98 (+L) & \textbf{ < 0.001 } & 0.97 (+L) & \textbf{ < 0.001 } & 0.84 (+L) \\
freetype2-2017 & \textbf{ < 0.001 } & 0.08 (-L) & 0.094 & 0.33 (-M) & > 0.1 & 0.54 (+N) & > 0.1 & 0.52 (+N) & \textbf{ < 0.001 } & 0.79 (+L) \\
libpcap\_fuzz\_both & > 0.1 & 0.57 (+S) & \textbf{ < 0.001 } & 0.79 (+L) & \textbf{ < 0.001 } & 0.80 (+L) & \textbf{ < 0.001 } & 0.87 (+L) & \textbf{ < 0.001 } & 0.81 (+L) \\
libpng-1.2.56 & > 0.1 & 0.42 (-S) & > 0.1 & 0.36 (-M) & > 0.1 & 0.49 (-N) & > 0.1 & 0.56 (+S) & 0.049 & 0.68 (+M) \\
lcms-2017-03-21 & > 0.1 & 0.45 (-N) & \textbf{ 0.037 } & 0.70 (+M) & \textbf{ < 0.001 } & 0.85 (+L) & > 0.1 & 0.37 (-S) & \textbf{ < 0.001 } & 0.88 (+L) \\
vorbis-2017-12-11 & > 0.1 & 0.39 (-S) & > 0.1 & 0.56 (+S) & \textbf{ < 0.001 } & 0.20 (-L) & > 0.1 & 0.62 (+S) & 0.092 & 0.65 (+M) \\

\bottomrule
\end{tabular}
\label{tab:statistics}
\end{table*}

\clearpage

\section{Algorithm Overview}

\begin{algorithm}[H]

\centering
\caption{Pseudocode of \OurMethodName{}}
\label{alg:slopt}

\begin{algorithmic}[0]

\Require{\mbox{}\\
    $initial\_seeds$ -- a set of initial test cases \\
    $program$ -- a PUT to be fuzzed
}

\Ensure{\mbox{}\\
    $queue$ -- a set of valuable test cases \\
    $crashes$ -- a set of test cases that trigger crashes
}

%\begin{adjustwidth}{-9pt}{}
%\setstretch{0.85}
\vspace{5pt}

\Function{RandomMutation}{$seed, instance_{mut}, instances_{bat}$}
\State $input$ $\gets$ \Call{CopyBytesFromSeed}{$seed$}
\State $mutation$ $\gets$ \Call{SelectArm}{$instance_{mut}$}
\State $idx$ $\gets$ \Call{GetGroupIndex}{$len(input)$}
\State $batch\_size$ $\gets$ \Call{SelectArm}{$instances_{bat}[idx][mutation]$}
\For{$i$ $\gets$ $1$ \textbf{to} $batch\_size$}
    \State $pos$ $\gets$ \Call{SelectPosition}{$input$}
    \State $input$ $\gets$ \Call{ApplyOperator}{$mutation, input, pos$}
\EndFor
\State \textbf{return} $input, mutation, batch\_size$
\EndFunction

%\end{adjustwidth}

%\vspace{-6pt}

%\begin{adjustwidth}{-9pt}{}
%\setstretch{0.85}

\vspace{5pt}

\Function{MutationFuzzing}{$initial\_seeds, program$}

\State $crashes$ $\gets$ $\varnothing$
\State $queue$ $\gets$ \Call{ConstructQueue}{$initial\_seeds$}
\State $instance_{mut}$ $\gets$ \Call{CreateBanditArms}{$number\_of\_mutations$}
\For{$i$ $\gets$ $1$ \textbf{to} $5$}
 \For{$j$ $\gets$ $1$ \textbf{to} $number\_of\_mutations$}
  \State $instances_{bat}[i][j]$ $\gets$ \Call{CreateBanditInstance}{$7$}
 \EndFor
\EndFor

\State

\While{ $\neg$ \Call{UserWantsStop}{\null}}
 \State $seed$ $\gets$ \Call{SelectSeed}{$queue$}
 \State $energy$ $\gets$ \Call{DecideEnergy}{$seed$}
 \For{$i$ $\gets$ $1$ \textbf{to} $energy$}
  \State $input, mutation, batch\_size$ 
  \State $\gets$ \Call{RandomMutation}{$seed, instance_{mut}, instances_{bat}$}
  \State $result$ $\gets$ \Call{ExecutePUT}{$program, input$}
  \State $b$ $\gets$ \Call{WasInputValuable}{$result$}
  \State \Call{RewardArm}{$mutation, b$}
  \State \Call{RewardArm}{$batch\_size, b$}
  \State \Call{SaveInputIfValuable}{$queue, input, result$}
  \State \Call{SaveInputIfCrash}{$crashes, input, result$}
 \EndFor
\EndWhile
\EndFunction

%\end{adjustwidth}

\end{algorithmic}
\end{algorithm}



\end{document}
