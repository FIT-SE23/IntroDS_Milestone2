\begin{figure*}[t]
     \centering
     \includegraphics[width=0.8\linewidth]{figure/discrim_more.png}
     \vspace{-10pt}
     \caption{The discriminative tasks account for a large proportion among all tasks in the T0 benchmark\citep{T0-paper}. Also all the of test tasks of T0 are discriminative tasks.}
     \label{fig:discrim}
 \end{figure*} 

\input{figs/overview.tex}
%\vspace{-0.4cm}
\section{DeepFlow Overview}\label{sec:overview}

Figure~\ref{fig:overview} shows an overview of the \name framework. \name takes the following set of \textbf{inputs}: 
%
(1) \underline{System} design hierarchy (e.g., the number of accelerator nodes per device, the number of devices in the system, the network topology connecting nodes within a device and across the devices), 
(2) \underline{Architecture template} of each accelerator node which provides a high-level definition of its components and how those components fit together. The purpose of the template is to provide a blueprint for the accelerator without committing to any specific hardware parameters.
%A component definition (e.g., minimal compute units (MCU\footnote{Examples of what we regard as MCU includes SMU in older GPUs, Tensor Cores in newer GPUs or systolic array in TPUs}), memory hierarchy, network), specification of each component (e.g., flop rate for each MCU, MCU dimensions, number of MCUs sharing a set of register files, dataflow execution model, and characteristics and scope of different levels of memory hierarchy), 
(3) \underline{Technology} parameters for each hardware component (e.g. energy per flop), 
(4) \underline{Design budgets} for each hardware component (area, power, perimeter),  
(5) \underline{Machine learning model} specification in the form of a high-level compute graph, parameters of each compute node (kernel type, tensor dimensions), and
(6) \underline{Parallelism strategy} (data, model, kernel, and/or pipeline parallelism dimensions) which distributes the compute graph across the entire system. 
(7) \underline{Device mapping} strategy which defines mapping of parallel shards onto hardware nodes.
Given these inputs, \name predicts the end-to-end performance of one iteration (i.e., single batch) of the model and finds an optimal hardware-software-technology design point as \textbf{output}. 

DeepFlow is composed of two major components.
\underline{CrossFlow} which operates in a stand-alone mode and can predict performance for any input configuration; and a search and optimization engine (\underline{SOE}) which enables design space search. 
%To do so, \name breaks the problem into multiple phases.
%Each phase or building block of \name is described in details next.
\vspace{-0.1cm}
\subsection{CrossFlow Building Blocks}

\paragraph*{\em Micro-Architecture Generator Engine (AGE)}

AGE takes the following set of \textbf{inputs}:
(1) Design constraints (i.e the power, area and perimeter budget and breakdown across micro-architectural components such as cache, network, compute cores). 
This breakdown can be provided manually by users or automatically by the Search and Optimization Engine (SOE, Section~\ref{subsec:soe}).
%We also provide technology specifications such as 
%and their physical characteristics such as area/power per core under nominal operating conditions, SRAM/register characteristics. 
(2) Technology parameters such as energy per flop, energy per data bit transfer for each level of memory and network hierarchy, threshold and maximum gate voltage, integration substrate parameters such as bump/interconnect pitch. We provide a wide range of standard and future technology libraries as baseline. (3) Architecture template which is a blueprint of the underlying accelerator chip without committing to any specific hardware parameters. Given these input, AGE performs a frequency-voltage-area scaling optimization to generate the following \textbf{output} parameters such that design budgets for all component are met: 
(1) Compute throughput.
(2) Capacity for different levels of memory hierarchy.
(3) Bandwidth to each level of memory hierarchy.
(4) Inter-node as well as intra-node network bandwidth. 
These parameters are then utilized by the performance prediction engine (PPE) to estimate the execution time of each kernel.
%As mentioned previously, 
%The output of this stage is the input to performance engine to estimate the execution time of each kernel. Next, we describe the search and optimization engine (SOE) which feeds input values to AGE, if we want to use the model for architecture search.
%\vspace{-0.2cm}
\paragraph*{\em Compute Graph Transformation and Device Placement Engine (DPE)}
The parallelization strategy and device mapping are critical in deciding the overall execution time. Here, we first transform the model graph to a `super-graph' to reflect the parallelization strategy provided by the users manually, or SOE engine (Section~\ref{subsec:soe}) automatically. For example, to apply data parallelism, the model graph is replicated and appropriate edges are added to model the gradient exchange. After generating the transformed graph, DPE assigns the vertices of the transformed graph to the system nodes following a heuristic approach to minimize the communication overhead. %
%The details are presented in section~\ref{}.

%\vspace{-0.2cm}
\paragraph*{\em Performance Prediction Engine (PPE)}
%With the device mapping for all the vertices of the compute (super-)graph known, the next step is to calculate the overall execution time for a forward pass and/or a backward pass. 
We use hierarchical roofline modeling to predict the performance of each compute node. To calculate the overall end-to-end execution time, while respecting scheduling constraints (e.g. one kernel at a time per GPU, or prioritizing one kernel launch over another) we use event-driven simulation.%
%We explain the details of the PPE in section~\ref{}.
\subsection{Search and Optimization Engine (SOE)}\label{subsec:soe}
Co-optimizing micro-architectural parameters and the parallelization strategy that minimizes the overall end-to-end execution time requires navigating a large space of design parameters. 
Search and optimization engine (SOE) enables the automatic design space search and finds an 
%that meets the total power and area constraints, and simultaneously explores software parallelization strategies to find the 
optimal design point which meets the design constraints and minimizes the overall execution time.
%Because the hardware configuration space is very large, the search algorithm we designed 
SOE takes inspiration from ML-assisted search algorithms, in particular gradient decent search with momentum and builds on top of the CrossFlow modeling engine.
%The software parallelization design space is much smaller compared to the hardware design space and therefore we employ an exhaustive grid search. 

%Gradient search is an iterative process. In each step, SOE takes the predicted time from previous iteration as input to re-adjust the following parameter settings: (1) power, area and perimeter breakdown across different architectural components. (2) a parallelization strategy. These parameters will be fed back to CrossFlow to estimate the overall execution time. This process continues until convergence or user-specified number of steps. 
%The details of SOE's search algorithm are elaborated in Section~\ref{}. 
\vspace{-0.2cm}
\subsection{Parallelism Strategy Space}
\label{subsec:par_strategy}
There are a myriad of ways to parallelize a model across a large multi-node system. Exploring the parallelism space and finding the optimal strategy is critical to overall performance and system utilization. DeepFlow explores kernel, data and layer parallelism. It uniquely identifies each parallelism strategy by following notations: $\texttt{RC-\{KP1\}-\{KP2\}-d\{DP\}-p\{LP\}}$ or $\texttt{CR-\{KP1\}-d\{DP\}-p\{LP\}}$ depending on the choice of kernel parallelism.
RC (Row-Column) and CR (Column-Row) refer to different forms of kernel parallelism, i.e. distributed GEMM through inner-product or outer-product implementation.
%\begin{equation*}
%    \texttt{RC: R{KP1\}\_C\{KP2\}\_d\{DP\}\_p\{LP\}}
%\end{equation*}
%Where \texttt{RC} or \texttt{CR} refers to the type of kernel parallelism strategy, i.e. Row-Column or Column-Row,
%\texttt{N} refers to the number of parallel nodes,
\texttt{KP1} and \texttt{KP2} are the parameters of distributed GEMM. 
For Row-Column (\texttt{RC}) or inner-product, \texttt{KP1} and \texttt{KP2} would refer to the number of ways we shard the first matrix across rows and the second matrix across columns.
For Column-Row (\texttt{CR}) or outer-product, we would only need one parameter to specify the parallelization strategy; \texttt{KP1} will refer to the number of ways we cut the first matrix across columns and the second matrix across rows.
\texttt{DP} represents the number of model replicas and data shards assigned to each to exploit data parallelism.
\texttt{LP} is the number of ways we cut layers into stages to exploit pipeline parallelism.

\begin{comment}
\subsection{Modes of Operation}
\name has two modes of operation, standalone performance estimation mode and a architecture search mode.
\paragraph{Standalone Performance (SP) Estimation Mode}
Often ML practitioners or hardware designers want to estimate the performance of a model on a particular system configuration. For example, what is the cost optimal number of accelerators that one should deploy for distributed training? Or what is the estimated performance gain from choosing an accelerator with costlier HBM2E vs HBM2? To enable one to quickly answer such questions and to estimate performance under certain known system configurations, the tool can be run in the SP mode. 

In this mode, the description of the architecture of a scale-out system consisting of multiple accelerators, the architecture of the accelerator hardware themselves and the description of the neural network is taken as input, and fed into CrossFlow, which calculates the execution time of each training step. 

%In this mode, the description of the architecture of a scale-out system consisting of multiple accelerators, the architecture of the accelerator hardware themselves and the description of the neural network is taken as input. The tool calculates the execution time of each training step. 

%In this mode, the user 
%has the flexibility to use either just the \perfE or use \perfE alongside the AGE. While using just the \perfE  alone, the user needs to provide the architectural parameters of the tiles and the system. On the other hand, while using AGE  alongside \perfE, the user 
%needs to define the technology parameters and the hardware constraints i.e., the overall area and power breakdown among the different architectural components of the system. T

%In this mode, the tool generates the micro-architectural parameters of the accelerator chip using the AGE. It then runs the compute graph transformation and the device placement engine, and uses the \perfE to predict the execution time. 

\subsubsection{Architecture Search (AS) Mode}

The insatiable demand to run large models in the shortest possible time demands that we find the optimal hardware and software design points to train these models. From the hardware perspective, it is about finding the right micro-architecture as well as the overall system architecture of the distributed system. 
From the software perspective, it is about finding the right parallelization strategy. 
Often these decisions depend on each other, and so finding the optimal design points across the stack means 
navigating a large design space.

As one can imagine, the design space of the inputs to the tool is large and iterating over the entire design space is a tedious task. To efficiently search over the input space to find the optimal hardware constraints and parallelization strategy, the tool can be run in the AS mode. 
In this mode, the SOE module is used. The user will not need to provide the exact hardware parameters and the parallelization strategy. Only the architecture template and the initial compute graph will need to be provided as input to the tool. The tool then performs a search over the design space to find the optimal parameter settings that results in minimum training time. 
%We used gradient descent algorithm (details in Section~\ref{}) for this search.

%\subsection{Inputs and Outputs}

%\paragraph{SP-Mode}
%In this mode, the hardware 

%\paragraph{AS-Mode}

\end{comment}

%\vspace{-0.4cm}
\section{DeepFlow Overview}\label{sec:overview}

Figure~\ref{fig:overview} shows an overview of the \name framework. \name takes the following set of \textbf{inputs}: 
%
(1) \underline{System} design hierarchy (e.g., the number of accelerator nodes per device, the number of devices in the system, the network topology connecting nodes within a device and across the devices), 
(2) \underline{Architecture template} of each accelerator node which provides a high-level definition of its components and how those components fit together. The purpose of the template is to provide a blueprint for the accelerator without committing to any specific hardware parameters.
%A component definition (e.g., minimal compute units (MCU\footnote{Examples of what we regard as MCU includes SMU in older GPUs, Tensor Cores in newer GPUs or systolic array in TPUs}), memory hierarchy, network), specification of each component (e.g., flop rate for each MCU, MCU dimensions, number of MCUs sharing a set of register files, dataflow execution model, and characteristics and scope of different levels of memory hierarchy), 
(3) \underline{Technology} parameters for each hardware component (e.g. energy per flop), 
(4) \underline{Design budgets} for each hardware component (area, power, perimeter),  
(5) \underline{Machine learning model} specification in the form of a high-level compute graph, parameters of each compute node (kernel type, tensor dimensions), and
(6) \underline{Parallelism strategy} (data, model, kernel, and/or pipeline parallelism dimensions) which distributes the compute graph across the entire system. 
(7) \underline{Device mapping} strategy which defines mapping of parallel shards onto hardware nodes.
Given these inputs, \name predicts the end-to-end performance of one iteration (i.e., single batch) of the model and finds an optimal hardware-software-technology design point as \textbf{output}. 

DeepFlow is composed of two major components.
\underline{CrossFlow} which operates in a stand-alone mode and can predict performance for any input configuration; and a search and optimization engine (\underline{SOE}) which enables design space search. 
%To do so, \name breaks the problem into multiple phases.
%Each phase or building block of \name is described in details next.
\vspace{-0.1cm}
\subsection{CrossFlow Building Blocks}

\paragraph*{\em Micro-Architecture Generator Engine (AGE)}

AGE takes the following set of \textbf{inputs}:
(1) Design constraints (i.e the power, area and perimeter budget and breakdown across micro-architectural components such as cache, network, compute cores). 
This breakdown can be provided manually by users or automatically by the Search and Optimization Engine (SOE, Section~\ref{subsec:soe}).
%We also provide technology specifications such as 
%and their physical characteristics such as area/power per core under nominal operating conditions, SRAM/register characteristics. 
(2) Technology parameters such as energy per flop, energy per data bit transfer for each level of memory and network hierarchy, threshold and maximum gate voltage, integration substrate parameters such as bump/interconnect pitch. We provide a wide range of standard and future technology libraries as baseline. (3) Architecture template which is a blueprint of the underlying accelerator chip without committing to any specific hardware parameters. Given these input, AGE performs a frequency-voltage-area scaling optimization to generate the following \textbf{output} parameters such that design budgets for all component are met: 
(1) Compute throughput.
(2) Capacity for different levels of memory hierarchy.
(3) Bandwidth to each level of memory hierarchy.
(4) Inter-node as well as intra-node network bandwidth. 
These parameters are then utilized by the performance prediction engine (PPE) to estimate the execution time of each kernel.
%As mentioned previously, 
%The output of this stage is the input to performance engine to estimate the execution time of each kernel. Next, we describe the search and optimization engine (SOE) which feeds input values to AGE, if we want to use the model for architecture search.
%\vspace{-0.2cm}
\paragraph*{\em Compute Graph Transformation and Device Placement Engine (DPE)}
The parallelization strategy and device mapping are critical in deciding the overall execution time. Here, we first transform the model graph to a `super-graph' to reflect the parallelization strategy provided by the users manually, or SOE engine (Section~\ref{subsec:soe}) automatically. For example, to apply data parallelism, the model graph is replicated and appropriate edges are added to model the gradient exchange. After generating the transformed graph, DPE assigns the vertices of the transformed graph to the system nodes following a heuristic approach to minimize the communication overhead. %
%The details are presented in section~\ref{}.

%\vspace{-0.2cm}
\paragraph*{\em Performance Prediction Engine (PPE)}
%With the device mapping for all the vertices of the compute (super-)graph known, the next step is to calculate the overall execution time for a forward pass and/or a backward pass. 
We use hierarchical roofline modeling to predict the performance of each compute node. To calculate the overall end-to-end execution time, while respecting scheduling constraints (e.g. one kernel at a time per GPU, or prioritizing one kernel launch over another) we use event-driven simulation.%
%We explain the details of the PPE in section~\ref{}.
\subsection{Search and Optimization Engine (SOE)}\label{subsec:soe}
Co-optimizing micro-architectural parameters and the parallelization strategy that minimizes the overall end-to-end execution time requires navigating a large space of design parameters. 
Search and optimization engine (SOE) enables the automatic design space search and finds an 
%that meets the total power and area constraints, and simultaneously explores software parallelization strategies to find the 
optimal design point which meets the design constraints and minimizes the overall execution time.
%Because the hardware configuration space is very large, the search algorithm we designed 
SOE takes inspiration from ML-assisted search algorithms, in particular gradient decent search with momentum and builds on top of the CrossFlow modeling engine.
%The software parallelization design space is much smaller compared to the hardware design space and therefore we employ an exhaustive grid search. 

%Gradient search is an iterative process. In each step, SOE takes the predicted time from previous iteration as input to re-adjust the following parameter settings: (1) power, area and perimeter breakdown across different architectural components. (2) a parallelization strategy. These parameters will be fed back to CrossFlow to estimate the overall execution time. This process continues until convergence or user-specified number of steps. 
%The details of SOE's search algorithm are elaborated in Section~\ref{}. 
\vspace{-0.2cm}
\subsection{Parallelism Strategy Space}
\label{subsec:par_strategy}
There are a myriad of ways to parallelize a model across a large multi-node system. Exploring the parallelism space and finding the optimal strategy is critical to overall performance and system utilization. DeepFlow explores kernel, data and layer parallelism. It uniquely identifies each parallelism strategy by following notations: $\texttt{RC-\{KP1\}-\{KP2\}-d\{DP\}-p\{LP\}}$ or $\texttt{CR-\{KP1\}-d\{DP\}-p\{LP\}}$ depending on the choice of kernel parallelism.
RC (Row-Column) and CR (Column-Row) refer to different forms of kernel parallelism, i.e. distributed GEMM through inner-product or outer-product implementation.
%\begin{equation*}
%    \texttt{RC: R{KP1\}\_C\{KP2\}\_d\{DP\}\_p\{LP\}}
%\end{equation*}
%Where \texttt{RC} or \texttt{CR} refers to the type of kernel parallelism strategy, i.e. Row-Column or Column-Row,
%\texttt{N} refers to the number of parallel nodes,
\texttt{KP1} and \texttt{KP2} are the parameters of distributed GEMM. 
For Row-Column (\texttt{RC}) or inner-product, \texttt{KP1} and \texttt{KP2} would refer to the number of ways we shard the first matrix across rows and the second matrix across columns.
For Column-Row (\texttt{CR}) or outer-product, we would only need one parameter to specify the parallelization strategy; \texttt{KP1} will refer to the number of ways we cut the first matrix across columns and the second matrix across rows.
\texttt{DP} represents the number of model replicas and data shards assigned to each to exploit data parallelism.
\texttt{LP} is the number of ways we cut layers into stages to exploit pipeline parallelism.

\begin{comment}
\subsection{Modes of Operation}
\name has two modes of operation, standalone performance estimation mode and a architecture search mode.
\paragraph{Standalone Performance (SP) Estimation Mode}
Often ML practitioners or hardware designers want to estimate the performance of a model on a particular system configuration. For example, what is the cost optimal number of accelerators that one should deploy for distributed training? Or what is the estimated performance gain from choosing an accelerator with costlier HBM2E vs HBM2? To enable one to quickly answer such questions and to estimate performance under certain known system configurations, the tool can be run in the SP mode. 

In this mode, the description of the architecture of a scale-out system consisting of multiple accelerators, the architecture of the accelerator hardware themselves and the description of the neural network is taken as input, and fed into CrossFlow, which calculates the execution time of each training step. 

%In this mode, the description of the architecture of a scale-out system consisting of multiple accelerators, the architecture of the accelerator hardware themselves and the description of the neural network is taken as input. The tool calculates the execution time of each training step. 

%In this mode, the user 
%has the flexibility to use either just the \perfE or use \perfE alongside the AGE. While using just the \perfE  alone, the user needs to provide the architectural parameters of the tiles and the system. On the other hand, while using AGE  alongside \perfE, the user 
%needs to define the technology parameters and the hardware constraints i.e., the overall area and power breakdown among the different architectural components of the system. T

%In this mode, the tool generates the micro-architectural parameters of the accelerator chip using the AGE. It then runs the compute graph transformation and the device placement engine, and uses the \perfE to predict the execution time. 

\subsubsection{Architecture Search (AS) Mode}

The insatiable demand to run large models in the shortest possible time demands that we find the optimal hardware and software design points to train these models. From the hardware perspective, it is about finding the right micro-architecture as well as the overall system architecture of the distributed system. 
From the software perspective, it is about finding the right parallelization strategy. 
Often these decisions depend on each other, and so finding the optimal design points across the stack means 
navigating a large design space.

As one can imagine, the design space of the inputs to the tool is large and iterating over the entire design space is a tedious task. To efficiently search over the input space to find the optimal hardware constraints and parallelization strategy, the tool can be run in the AS mode. 
In this mode, the SOE module is used. The user will not need to provide the exact hardware parameters and the parallelization strategy. Only the architecture template and the initial compute graph will need to be provided as input to the tool. The tool then performs a search over the design space to find the optimal parameter settings that results in minimum training time. 
%We used gradient descent algorithm (details in Section~\ref{}) for this search.

%\subsection{Inputs and Outputs}

%\paragraph{SP-Mode}
%In this mode, the hardware 

%\paragraph{AS-Mode}

\end{comment}

%\vspace{-0.4cm}
\section{DeepFlow Overview}\label{sec:overview}

Figure~\ref{fig:overview} shows an overview of the \name framework. \name takes the following set of \textbf{inputs}: 
%
(1) \underline{System} design hierarchy (e.g., the number of accelerator nodes per device, the number of devices in the system, the network topology connecting nodes within a device and across the devices), 
(2) \underline{Architecture template} of each accelerator node which provides a high-level definition of its components and how those components fit together. The purpose of the template is to provide a blueprint for the accelerator without committing to any specific hardware parameters.
%A component definition (e.g., minimal compute units (MCU\footnote{Examples of what we regard as MCU includes SMU in older GPUs, Tensor Cores in newer GPUs or systolic array in TPUs}), memory hierarchy, network), specification of each component (e.g., flop rate for each MCU, MCU dimensions, number of MCUs sharing a set of register files, dataflow execution model, and characteristics and scope of different levels of memory hierarchy), 
(3) \underline{Technology} parameters for each hardware component (e.g. energy per flop), 
(4) \underline{Design budgets} for each hardware component (area, power, perimeter),  
(5) \underline{Machine learning model} specification in the form of a high-level compute graph, parameters of each compute node (kernel type, tensor dimensions), and
(6) \underline{Parallelism strategy} (data, model, kernel, and/or pipeline parallelism dimensions) which distributes the compute graph across the entire system. 
(7) \underline{Device mapping} strategy which defines mapping of parallel shards onto hardware nodes.
Given these inputs, \name predicts the end-to-end performance of one iteration (i.e., single batch) of the model and finds an optimal hardware-software-technology design point as \textbf{output}. 

DeepFlow is composed of two major components.
\underline{CrossFlow} which operates in a stand-alone mode and can predict performance for any input configuration; and a search and optimization engine (\underline{SOE}) which enables design space search. 
%To do so, \name breaks the problem into multiple phases.
%Each phase or building block of \name is described in details next.
\vspace{-0.1cm}
\subsection{CrossFlow Building Blocks}

\paragraph*{\em Micro-Architecture Generator Engine (AGE)}

AGE takes the following set of \textbf{inputs}:
(1) Design constraints (i.e the power, area and perimeter budget and breakdown across micro-architectural components such as cache, network, compute cores). 
This breakdown can be provided manually by users or automatically by the Search and Optimization Engine (SOE, Section~\ref{subsec:soe}).
%We also provide technology specifications such as 
%and their physical characteristics such as area/power per core under nominal operating conditions, SRAM/register characteristics. 
(2) Technology parameters such as energy per flop, energy per data bit transfer for each level of memory and network hierarchy, threshold and maximum gate voltage, integration substrate parameters such as bump/interconnect pitch. We provide a wide range of standard and future technology libraries as baseline. (3) Architecture template which is a blueprint of the underlying accelerator chip without committing to any specific hardware parameters. Given these input, AGE performs a frequency-voltage-area scaling optimization to generate the following \textbf{output} parameters such that design budgets for all component are met: 
(1) Compute throughput.
(2) Capacity for different levels of memory hierarchy.
(3) Bandwidth to each level of memory hierarchy.
(4) Inter-node as well as intra-node network bandwidth. 
These parameters are then utilized by the performance prediction engine (PPE) to estimate the execution time of each kernel.
%As mentioned previously, 
%The output of this stage is the input to performance engine to estimate the execution time of each kernel. Next, we describe the search and optimization engine (SOE) which feeds input values to AGE, if we want to use the model for architecture search.
%\vspace{-0.2cm}
\paragraph*{\em Compute Graph Transformation and Device Placement Engine (DPE)}
The parallelization strategy and device mapping are critical in deciding the overall execution time. Here, we first transform the model graph to a `super-graph' to reflect the parallelization strategy provided by the users manually, or SOE engine (Section~\ref{subsec:soe}) automatically. For example, to apply data parallelism, the model graph is replicated and appropriate edges are added to model the gradient exchange. After generating the transformed graph, DPE assigns the vertices of the transformed graph to the system nodes following a heuristic approach to minimize the communication overhead. %
%The details are presented in section~\ref{}.

%\vspace{-0.2cm}
\paragraph*{\em Performance Prediction Engine (PPE)}
%With the device mapping for all the vertices of the compute (super-)graph known, the next step is to calculate the overall execution time for a forward pass and/or a backward pass. 
We use hierarchical roofline modeling to predict the performance of each compute node. To calculate the overall end-to-end execution time, while respecting scheduling constraints (e.g. one kernel at a time per GPU, or prioritizing one kernel launch over another) we use event-driven simulation.%
%We explain the details of the PPE in section~\ref{}.
\subsection{Search and Optimization Engine (SOE)}\label{subsec:soe}
Co-optimizing micro-architectural parameters and the parallelization strategy that minimizes the overall end-to-end execution time requires navigating a large space of design parameters. 
Search and optimization engine (SOE) enables the automatic design space search and finds an 
%that meets the total power and area constraints, and simultaneously explores software parallelization strategies to find the 
optimal design point which meets the design constraints and minimizes the overall execution time.
%Because the hardware configuration space is very large, the search algorithm we designed 
SOE takes inspiration from ML-assisted search algorithms, in particular gradient decent search with momentum and builds on top of the CrossFlow modeling engine.
%The software parallelization design space is much smaller compared to the hardware design space and therefore we employ an exhaustive grid search. 

%Gradient search is an iterative process. In each step, SOE takes the predicted time from previous iteration as input to re-adjust the following parameter settings: (1) power, area and perimeter breakdown across different architectural components. (2) a parallelization strategy. These parameters will be fed back to CrossFlow to estimate the overall execution time. This process continues until convergence or user-specified number of steps. 
%The details of SOE's search algorithm are elaborated in Section~\ref{}. 
\vspace{-0.2cm}
\subsection{Parallelism Strategy Space}
\label{subsec:par_strategy}
There are a myriad of ways to parallelize a model across a large multi-node system. Exploring the parallelism space and finding the optimal strategy is critical to overall performance and system utilization. DeepFlow explores kernel, data and layer parallelism. It uniquely identifies each parallelism strategy by following notations: $\texttt{RC-\{KP1\}-\{KP2\}-d\{DP\}-p\{LP\}}$ or $\texttt{CR-\{KP1\}-d\{DP\}-p\{LP\}}$ depending on the choice of kernel parallelism.
RC (Row-Column) and CR (Column-Row) refer to different forms of kernel parallelism, i.e. distributed GEMM through inner-product or outer-product implementation.
%\begin{equation*}
%    \texttt{RC: R{KP1\}\_C\{KP2\}\_d\{DP\}\_p\{LP\}}
%\end{equation*}
%Where \texttt{RC} or \texttt{CR} refers to the type of kernel parallelism strategy, i.e. Row-Column or Column-Row,
%\texttt{N} refers to the number of parallel nodes,
\texttt{KP1} and \texttt{KP2} are the parameters of distributed GEMM. 
For Row-Column (\texttt{RC}) or inner-product, \texttt{KP1} and \texttt{KP2} would refer to the number of ways we shard the first matrix across rows and the second matrix across columns.
For Column-Row (\texttt{CR}) or outer-product, we would only need one parameter to specify the parallelization strategy; \texttt{KP1} will refer to the number of ways we cut the first matrix across columns and the second matrix across rows.
\texttt{DP} represents the number of model replicas and data shards assigned to each to exploit data parallelism.
\texttt{LP} is the number of ways we cut layers into stages to exploit pipeline parallelism.

\begin{comment}
\subsection{Modes of Operation}
\name has two modes of operation, standalone performance estimation mode and a architecture search mode.
\paragraph{Standalone Performance (SP) Estimation Mode}
Often ML practitioners or hardware designers want to estimate the performance of a model on a particular system configuration. For example, what is the cost optimal number of accelerators that one should deploy for distributed training? Or what is the estimated performance gain from choosing an accelerator with costlier HBM2E vs HBM2? To enable one to quickly answer such questions and to estimate performance under certain known system configurations, the tool can be run in the SP mode. 

In this mode, the description of the architecture of a scale-out system consisting of multiple accelerators, the architecture of the accelerator hardware themselves and the description of the neural network is taken as input, and fed into CrossFlow, which calculates the execution time of each training step. 

%In this mode, the description of the architecture of a scale-out system consisting of multiple accelerators, the architecture of the accelerator hardware themselves and the description of the neural network is taken as input. The tool calculates the execution time of each training step. 

%In this mode, the user 
%has the flexibility to use either just the \perfE or use \perfE alongside the AGE. While using just the \perfE  alone, the user needs to provide the architectural parameters of the tiles and the system. On the other hand, while using AGE  alongside \perfE, the user 
%needs to define the technology parameters and the hardware constraints i.e., the overall area and power breakdown among the different architectural components of the system. T

%In this mode, the tool generates the micro-architectural parameters of the accelerator chip using the AGE. It then runs the compute graph transformation and the device placement engine, and uses the \perfE to predict the execution time. 

\subsubsection{Architecture Search (AS) Mode}

The insatiable demand to run large models in the shortest possible time demands that we find the optimal hardware and software design points to train these models. From the hardware perspective, it is about finding the right micro-architecture as well as the overall system architecture of the distributed system. 
From the software perspective, it is about finding the right parallelization strategy. 
Often these decisions depend on each other, and so finding the optimal design points across the stack means 
navigating a large design space.

As one can imagine, the design space of the inputs to the tool is large and iterating over the entire design space is a tedious task. To efficiently search over the input space to find the optimal hardware constraints and parallelization strategy, the tool can be run in the AS mode. 
In this mode, the SOE module is used. The user will not need to provide the exact hardware parameters and the parallelization strategy. Only the architecture template and the initial compute graph will need to be provided as input to the tool. The tool then performs a search over the design space to find the optimal parameter settings that results in minimum training time. 
%We used gradient descent algorithm (details in Section~\ref{}) for this search.

%\subsection{Inputs and Outputs}

%\paragraph{SP-Mode}
%In this mode, the hardware 

%\paragraph{AS-Mode}

\end{comment}




\begin{table*}[htbp]
% \setlength{\tabcolsep}{0.6mm}
  \centering
    \resizebox{1.0\textwidth}{!}{\begin{tabular}{c|c|p{37em}|c}
    \toprule
    \multirow{2}[2]{*}{\textbf{Category}} & \multirow{2}[2]{*}{\textbf{Task Type}} & \multicolumn{1}{c|}{\multirow{2}[2]{*}{\textbf{Our Minmal Prompt}}} & \multirow{2}[2]{*}{\textbf{Label}} \\
          &       & \multicolumn{1}{c|}{} &  \\
    \midrule
    \multirow{4}[8]{*}{yes/no} & \multirow{2}[4]{*}{\shortstack{Paraphrase \\ Identification}} & John is Lily's husband. Lily is John's wife & 1 \\
\cmidrule{3-4}          &       & John is Lily's husband. Lily is John's mother. & 0 \\
\cmidrule{2-4}          & \multicolumn{1}{c|}{\multirow{4}[4]{*}{\shortstack{Natural \\ Language  \\ Inference}}} & \underline{Premise:} Dana Reeve, the widow of the actor Christopher Reeve, has died of lung cancer at age 44. \underline{Hypothesis:} Dana Reeve had an accident. & 1 \\
\cmidrule{3-4}          &       & \underline{Premise:} Dana Reeve, the widow of the actor Christopher Reeve, has died of lung cancer at age 44. \underline{Hypothesis:} Christopher Reeve had an accident. & 0 \\
    \midrule
    \multirow{10}[20]{*}{multi-choice} & \multirow{2}[4]{*}{\shortstack{Coreference \\ Resolution}} & Jane gives Joan candy because Joan was hungry. & 1 \\
\cmidrule{3-4}          &       & Jane gives Joan candy because Jane was hungry. & 0 \\
\cmidrule{2-4}          
& \multirow{2}[4]{*}{\shortstack{Question \\ Answer}} & The earth moves around the sun. What is the earch to the sun? Planet & 1 \\
\cmidrule{3-4}          &       & The earth moves around the sun. What is the earch to the sun? Satellite & 0 \\
\cmidrule{2-4}          & \multirow{2}[4]{*}{\shortstack{Topic \\ Classification}} & Open Source Apps Developer SugarCRM Releases Sugar.Sales 1.1. Science and technology & 1 \\
\cmidrule{3-4}          &       & Open Source Apps Developer SugarCRM Releases Sugar.Sales 1.1. Sports & 0 \\
\cmidrule{2-4}          & \multirow{2}[4]{*}{\shortstack{Sentence \\ Completion}} & A boy is running down a track. The boy lifts his body above the height of a pole. & 1 \\
\cmidrule{3-4}          &       & A boy is running down a track. The boy stands on his hands and springs. & 0 \\
\cmidrule{2-4}          & \multirow{2}[4]{*}{\shortstack{Sentiment \\ Classification}} & I really love this movie. Positive & 1 \\
\cmidrule{3-4}          &       & I don't like this movie. Negative & 1 \\
\bottomrule
    \end{tabular}}%
\caption{Examples of how we unify discriminative tasks. The underlined text represents additional words not present in raw inputs. Note that this is just our implementation of the UD formulation and there can be other ways of task formulation under the UD framework. Some tasks can either be yes/no tasks or multi-choice tasks, depending on how options are provided.}
\label{tab:task_formulate_example}
\end{table*}%






\section{Approach}

Previous works \citep{T0-paper,FLAN} have shown that prompted multi-task training can greatly improve zero-shot performance on unseen tasks. One intuitive reason behind the validity of this improvement is that all the NLP tasks share a common ability that allows LMs to solve unseen tasks based on the data from other training tasks. To test this idea and even enhance zero-shot generalization, a direct way is explicitly defining what this "common ability" is. Here, we define this "common ability" by designing a new general task of ``discriminating whether a text sample comes from the true data distribution of natural language''. 

We will first formulate the learning problem (\S~\ref{sec:formualtion}), and then define the concept \textit{discriminative tasks} (\S~\ref{sec:disc}), followed by describing how we transform discriminative tasks into our shared formulation.
In \S~\ref{sec:ud} and \S~\ref{sec:generalizedud}, we will study our UD, respectively on discriminative tasks and on a generalized scope of both discriminative and generative tasks.


\subsection{Multi-Task Training for Zero-Shot Generalization} \label{sec:formualtion}


Now we describe the learning problem we aim to solve in this work.
We adopt the same setting as in \citet{T0-paper}. The input to our problem is a set of training tasks with labeled data, and the goal is to train a model that generalizes to unseen test tasks. The training and test tasks are constrained to have distinct task types for the evaluation of cross-task-type generalization. A pretrained model is jointly trained on the set of training tasks and directly evaluated on the set of test tasks in a zero-shot manner.


% The difference between our multi-task training and theirs is that our training and evaluation processes are all in the UD format (which be described in section~\ref{sec:ud}) and then transform the result to each individual task. We believe that training and evaluating in this shared format can better induce zero-shot generalization performance across different tasks.

% \paragraph{Finetuning}
% We also study the finetuning paradigm. Specifically, after we train the model using multiple tasks, we follow previous work \cite{T5-paper} to further finetune the model on each individual task to obtain the best performance on these tasks. We use this setting to test the performance of our approach with abundant labels.


% whether our UD format training has advantage in the standard finetuning paradigm. In specific, we first perform a multi-tasking training on a pretrained T5 model, and then finetune each individual task, where all the training and evaluating processes are in UD format. We hypothesis that the shared UD format makes it possible for LM to learn from other tasks and thus improve the accuracy after finetuning.

\subsection{Discriminative Tasks} \label{sec:disc}

% \zy{given the definition of discriminative tasks, and then prove that a significant portion of NLP tasks are discriminative tasks.}

We use the term ``discriminative tasks'' to refer to tasks that can be framed as selecting from a few options. 
%\xhk{instead of generating an answer verbalizer by the model itself}

More concretely, there are two types of discriminative tasks. The first type is tasks with multiple options, such as multi-choice question answering and news classification. The problem can be framed as selecting the right option from multiple ones, where the options are either customized for each sample (e.g., multi-choice question answering) or shared within the task (e.g., news classification). The second type is tasks with yes/no options, such as paraphrase identification and natural language inference. Given a sample of these tasks, a model is asked to predict a yes/no (or true/false) answer. 
%\xhk{We separately create the second type of tasks because the yes/no token itself doesn't contain much information compared with customized choices. Empirical experiments suggest that there is a need to unify them in a different way.} 
% \sout{In this work, we will mainly study how to use discriminative approaches to optimize the performance of discriminative tasks.}

It is important to notice that discriminative tasks constitute a significantly large portion of modern NLP research tasks. For example, all of the test tasks of the T0 benchmark~\cite{T0-paper}, SuperGLUE~\cite{wang2019superglue}, GLUE~\cite{wang2019glue} are discriminative tasks. 
As shown in Figure \ref{fig:discrim}, discriminative tasks constitute up to 60+\% in the T0 multi-task benchmark.
Also note that our definition of discriminative tasks has a larger scope compared to the conventional notion of ``classification'' which usually refers to tasks with a non-customized, fixed set of labels. In contrast, discriminative tasks might have sample-customized options, e.g., multi-choice question answering and coreference resolution.

% For ``discriminative tasks", we mean those tasks with limited answer choices provided to LMs during the test phase. We use two observations of discriminative tasks compared with generative tasks in developing our method: First, rather than considering which is the answer for a task from scratch, discriminative tasks only require LMs to judge whether each choice is acceptable and then select the most acceptable one. Second, if a choice is not the correct answer for a given task, we can usually find some violation of real language usage in certain concatenation of the input and the wrong choice. The first observation makes it reasonable to train a specialized discriminator to do the ``true distribution" judgment for each choice separately, and the second observation implies the existence of universal rules behind such judgements across different tasks, motivating us to train a universal discriminator.


% \subsection{Unifying training data from most NLP classification tasks to \solution task}
\subsection{A Universal Discriminator}
\label{sec:ud}

% \zy{Moved from other places, need to rewrite and fit in} 

Given a text sample $x$, let $P(\text{true} | x)$ be the probability that $x$ is sampled from the true data distribution of natural language. We train a universal discriminator (UD), denoted as $D(x)$, to estimate the probability $P(\text{true} | x)$ for each text sample $x$. From another perspective of contrastive learning \cite{oord2018representation}, this problem can also be viewed as learning a partial order of the probability distribution. Specifically, for two text samples $x_1$ and $x_2$, if $P(\text{true} | x_1) > P(\text{true} | x_2)$, the UD is expected to predict $D(x_1) > D(x_2)$. This contrastive view is essential for tasks with multiple options, i.e., learning to select from a few options based on the partial order given by UD.


% Given a labeled dataset $\mc{D}=\{x_i,y_i\}$ where $x_i$ is text and $y_i\in[0,1]$ is the probability that $x_i$ comes from the true data distribution of natural language. Our goal is to train a universal discriminator (UD) which can provide a mapping $f(x)$ from a text to a probability that $x$ comes from the true data distribution of natural language. This tasks measure the ability to discriminate whether a text sample comes from the true data distribution of natural language or not.

Figure~\ref{fig:overview} compares the multi-task prompted formulation of T0 and the formulation of our UD.
In the following, we will show how we use this formulation of UD to unify and solve discriminative tasks.
% In the following, we will show how we do this and once we get such a discriminator, how it can solve all the discriminative tasks.

\subsubsection{Unifying Discriminative Tasks}
\label{sec:unifydiscriminativetasks}

We assume that a data example is considered ``correct'' if it follows the true data distribution of natural languages, while ``wrong" if it deviates much from the true data distribution. 
% \xhk{We assume that given a discriminative task example, a "correct" transformation \zy{not clear based on context, what is a transformation?} (see the following paragraphs for different transformation methods) of this example can be considered as coming from the true data distribution of natural language, while some other "wrong" transformation can't, where our proposed UD is aimed at predicting this.} \zy{language not formal enough}
Given this assumption, we claim that almost all discriminative tasks are equivalent to our defined task (i.e., estimating $P(\text{true} | x)$) above. Here, ``equivalent" has bi-directional meanings: on one hand, there exists a data transformation method such that one piece of training data from a discriminative task can be transformed into several pieces of UD's training data. 
On the other hand, there exists a data transformation method such that UD can solve a discriminative task by first predicting $D(\cdot)$ for a set of transformed samples and then using a mapping from UD's outputs to the original task's outputs.
% On the other hand, there exists a data transformation method such that UD can solve a discriminative task by first predicting $P(\text{true} | x)$ for a transformed input $x$ and then using a mapping from UD's outputs to the original task's outputs.
% On the other hand, there exists a data transformation method such that UD answers \zy{answers?} a piece of test data from a discriminative task by first predicting scores $D(\cdot)$ for a set of its transformed samples and then using a mapping from UD's outputs to the original task's outputs. \zy{simplify this sentence}


%\xhk{Given this assumption, we claim that almost all discriminative tasks are equivalent to our defined task (i.e., estimating $P(\text{true} | x)$) above. Here, ``equivalent" has bi-directional meanings: on one hand, for a piece of training data from any discriminative task, we can combine its input $x_{in}$ with each of its options from $\{c_i\}_{i=1}^{N_c}$ to get several pieces of UD's training data ($\{x_i=(x_{in},c_i)\}_{i=1}^{N_c}$). On the other hand, given a piece of test data $x_{in},\{c_i\}_{i=1}^{N_c}$ from any discriminative task, UD can answer it by first predicting scores $\{D((x_{in},c_i))\}_{i=1}^{N_c}$ for the set of combined samples between input and each option and use a mapping from the UD's outputs to the original task's outputs}


%Given this assumption, we claim that almost all discriminative tasks are equivalent to our defined task (i.e., estimating $P(\text{true} | x)$) above. Here, ``equivalent" has bi-directional meanings: on one hand, there exists a data transformation method such that a discriminative task's training data can be transformed into UD's training data. On the other hand, there exists a data transformation method such that UD can solve a discriminative task by first predicting $P(\text{true} | x)$ for a transformed input $x$ and then using a mapping from UD's outputs to the original task's outputs.

Based on the definition of discriminative tasks in \ref{sec:disc}, there are two main categories, multi-choice tasks and yes/no tasks. We will discuss each category in detail as follows (also see Table \ref{tab:task_formulate_example} for specifics).

% For all the discriminative tasks we have met, we design 3 unifying methods to transform each tasks's data to UD's iuput data, basing on the relationship of the given answer choices: parallel, opposite, or extent. Please refer to Table~\ref{tab:task_formulate_example} to see examples for each unifying method.

\paragraph{Multi-Choice Tasks}
For multi-choice tasks, we concatenate the text input $x_{in}$ with each choice $\{c_i\}_{i=1}^{N_c}$ to form samples. For example, for multi-choice question answering, we concatenate the given paragraph and question with each answer candidate. See Table \ref{tab:task_formulate_example} for more task formulation. During training, the concatenated samples with the correct choice are given label $1$ ("correct" transformation) for UD and the other incorrect ones are given label $0$ ("wrong" transformation). During testing, similarly, we concatenate the text input 
$x_{in}$ with each choice $\{c_i\}_{i=1}^{N_c}$ 
to form several samples 
$\{(x_{in},c_i)\}_{i=1}^{N_c}$ 
and ask UD for their $D(\cdot)$ scores. We then select the sample with the maximal $D(\cdot)$ score and output its corresponding choice.

%For multi-choice tasks, we concatenate each choice with the text inputs to form a sample. For example, for multi-choice question answering, we concatenate each candidate answer with the given paragraph and question. See Table \ref{tab:task_formulate_example} for more task formulation. During training, the concatenated samples with the correct choices are given label 1 for UD and the incorrect ones are given label 0. During test, for each sample, we select the concatenated sentence with the maximal UD probability and output its corresponding choice.

% \xhk{move to multi-choice paragraph}
% Some of the multi-choice tasks might be related to regression; i.e., a restaurant review classification task assigns 1 to 5 stars to an input text. In this case, we can set the groundtruth label as a probability (e.g., 0.25, 0.5, 0.75, 1.0, etc), which is compatible with the cross entropy loss. We note that other formulations are also possible.

% For tasks like Question Answering, Sentence Completion, the answer choices are several unrelated words or phrases. Our unifying method is to use minimal prompt to concatenate all the raw input keywords and each answer choice. In training phase, the concatenated sentence with correct answer is given label 1, and the other sentences are given label 0. In testing phase, we select the concatenated sentence with the maximal probability of true, and output its corresponding answer choice.

% Some of the multi-choice tasks might be related to regression; e.g., a restaurant review classification task assigns 1 to 5 stars to an input text. We design a slightly more accurate formulation for these tasks (see Appendix \ref{sec:extent}).

% \xhk{A special case of multi-choice tasks is extent measurement, e.g. the degree of sentiment, or the attitude of a given paragraph. We design a slightly more accurate unifying method in appendix~\ref{sec:extent}}

\paragraph{Tasks with Yes/No Choices}
% \xhk{One reviewer suggested how to use UD in test phase is unclear, so I rewrite this paragraph.}
For yes/no tasks, we directly treat the text input $x_{in}$ as a sample and assign its 0/1 label based on its yes/no label. During training, we use $x_{in}$ with its assigned 0/1 label as UD's training data. During testing, we first get the output of UD on $x_{in}$, $D(x_{in})$, and then output answer yes/no based on whether $D(x_{in})>0.5$\footnote{We note that more delicate threshold search might be possible, but we find it performs well using a constant 0.5.}. 

%We separately create a new method for tasks with Yes/No choices because here the answer tokens Yes/No themselves don't contain much information compared with customized choices. 
Empirical experiments suggest that unifying tasks with Yes/No choices in such a new way can produce better zero-shot performance than using the same method for Multi-Choice Tasks because the answer tokens here don't contain much information and thus the model cannot benefit from concatenation.

%For yes/no tasks, we use a more direct connection to UD. During training, we use the yes/no label as the UD label for each sample. During test, we perform binary classification using a threshold of 0.5\footnote{We note that more delicate threshold search might be possible, but we find it perform well using a constant 0.5.} based on the UD predictions.


% Some tasks are essentially binary interrogating questions, e.g. judging whether a hypothesis is implied by the premise, judging whether two sentences has the same meaning. Their answer choices are usually Yes/No, True/False. For these tasks, we assume that the answer for this tasks is an intrinsic nature of the given raw input, even without choice attached. Through minimal prompting, we get a concatenation of the raw input's keywords as the input for UD. In training phase, we assign it a label basing on its answer. In test phase, we do the binary selection by checking whether the probability of true outputed by UD is larger than 0.5. 

% \paragraph{Extent Answer Choices}
% Some tasks are asking for the location of the given input in an extent measurement, e.g. the degree of sentiment, or the attitude of a given paragraph. Usually, "positive" and "negative" are the two ends in the measurement. For these tasks, we use minimal prompt to concatenate the input with each of the two extreme extent verbalizers, e.g. "positive" and "negative“, and we assign it the label basing on the true location of the sentence in the extent measurement.

\paragraph{Minimal Prompting} A key principle we follow for task formulation is minimal prompting. From Table \ref{tab:task_formulate_example}, one can see that our prompts are minimal in the sense that they are mostly just concatenations. This is very different from T0 \cite{T0-paper} and other generative approaches \cite{gpt3-paper,PET-paper} that add lengthy task descriptions with different wordings into the prompts.

We argue that there are two major benefits of minimal prompting. First,  previous work \cite{liu2021gpt} has shown that zero-shot and few-shot performances are very sensitive to the prompts used for inference. Minimal prompting is more robust and requires less prompt engineering efforts at test time. This is especially important for true zero-shot real-world applications as there is no data available for choosing the right prompt. Second, as we will show in our experiments, UD performs much better with minimal prompts than lengthy descriptive prompts, while generative approaches do not work well with minimal prompts. This is also consistent with our motivation that UD unifies discriminative tasks so it does not rely much on descriptions for each task.

Note that it is also important to use minimal prompts to resolve ambiguity for yes/no tasks. For example, consider the natural language inference (NLI) task that predicts whether a premise $A$ entails a hypothesis $B$. Simply concatenating $A$ and $B$ is ambiguous, because the model cannot tell which is the premise. The model also is not aware that this is an NLI task. To resolve this kind of ambiguity, we use a minimal prompt ``Premise: A. Hypothesis: B.'' instead, as shown in Table \ref{tab:task_formulate_example}.

% \subsubsection{Minimal Prompting}

% \zy{Describe the idea of minimal prompting here}

% In the three unifying methods described above, we need a minimal prompting to concatenate different keywords of a task's raw input and each of its choices. The goal of our minimal prompting is to make the prompted sentence looks more like sampled from the true language distribution for correct answer and less like sampled from the true distribution for other wrong choices, but not to instructing what the task is, which is the key difference between our minimal prompt and the task descriptive prompts used in \citet{T0-paper}. We call it ``minimal'' because for many cases, only trivial concatenation with choice attached at the end is enough for out goal, and it is much simpler than the prompts used in \citet{T0-paper}. However, in some cases, such ``minimal" prompts are necessary because trivial concatenation may result some ambiguity for certain different tasks. For example, in NLI tasks, if we want to judge whether a hypothesis ``B`` is implied by a premise ``A", simple concatenation like ``A B" is ambiguous, because attaching a reason instead of a hypothesis also looks like a sentence sampled from true language distribution, but it should be given False for the NLI task. In this case, our minimal prompt should give ``premise: A Hypothesis: B" to clarify the meaning of each sentence component. Similar necessary case for minimal prompting usually happens for those tasks with opposite choices.

% \subsection{Using a \solution oracle to solve most NLP classification tasks}
% \subsection{Training and Evaluating Universal Discriminator}

% \subsection{Architecture}
\subsubsection{Architecture}

UD can use any pretrained encoder model as the backbone. In this work, we experiment with the T5 encoder and DeBERTa \cite{debertav3}. Since T5 is an encoder-decoder model, we only use the encoder part. For the T5 backbone, we perform mean pooling over the last-layer encoder features, followed by a dropout layer and a linear layer to predict a scalar logit. For the DeBERTa backbone, we use the last-layer feature of the first token, followed by a two-layer perceptron with dropout to also output a scalar logit. We train UD with the binary cross entropy loss.



\subsection{A Generalized Universal Discriminator}
\label{sec:generalizedud}

To further study how the discriminative approaches work in combination with generative tasks, we also propose to experiment with a generalized version of UD (denoted as generalized UD).



Different from the previous UD that only uses an encoder as the backbone model, the generalized UD employs an encoder-decoder architecture. In the following, we experiment with the T5 model.
Generalized UD takes both discriminative and generative tasks into consideration, and is jointly trained over both types of tasks at the same time.



For discriminative tasks, they are reformulated into binary classification tasks through minimal prompting, as is described in  \S~\ref{sec:unifydiscriminativetasks}. Specifically, it takes the minimal prompted texts into the encoder and uses the decoder to predict over \{``Yes'', ``No''\}.
In such cases, generalized UD is optimized with the binary cross-entropy loss.
For generative tasks, they take the form of ``input-and-target'' pairs. Generalized UD is fed with the textual inputs, and generates the targets through decoding.
For generative tasks, generalized UD is trained to optimize the cross-entropy loss.






% \zy{describe the architecture}

% \zy{Moved from other places, need to rewrite and fit in} 

% We inherit the multi-task training phase by \citep{T0-paper,FLAN}. During the training phase, we transform all training tasks' data into data for UD, and train a LM to solve the this task, i.e. we train a universal discriminator. Then, in the zero-shot testing phase, we transform each testing task's data into UD format and use UD's prediction to deduce the answer for the original testing task. Therefore, once we train a UD, we can use it to solve almost all the NLP tasks. 

\begin{comment}
Here we only give an example from COPA in table~\ref{tab:copa_raw_input}

The raw input is
\begin{table}[htbp]
    \centering
    \begin{tabularx}{0.5\textwidth}{ l|X }
        \toprule
        Input & The hamburger meat browned.  \\
        \midrule 
        Ask for & cause  \\
        \midrule
        Choices & The cook froze it., The cook grilled it.\\
        \midrule 
        Target & The cook grilled it.  \\
        \bottomrule 
    \end{tabularx}
    \caption{an example from COPA's raw data}
    \label{tab:copa_raw_input}%
\end{table}%

The way we test this task is that we first similarly unifying the data to the \solution data format, by concatenating its input, "ask for", and choice. Then we use our \solution model to predict the probability of True for each sentence. Finally, we select the choice with whose concatenated sentence has the highest probability of True.

Our input for \solution is

\begin{table}[htbp]
    \centering
    \begin{tabularx}{0.5\textwidth}{ X }
        \toprule
        Input \\
        \midrule
        The hamburger meat browned. cause The cook froze it. \\
        \midrule 
        The hamburger meat browned. cause The cook grilled it. \\
        \bottomrule 
    \end{tabularx}
    \caption{a unified COPA example}
    \label{tab:quarel_consistency}%
\end{table}%
\end{comment}


% \subsection{Discussion}

% % \zy{This section mainly discusses our advantages to generative modeling.}

% % \zy{Moved from other places, need to rewrite and fit in} 

% Similar to prompting, which makes all NLP tasks to "text generation task", here we make all NLP tasks to the UD format, which is a discriminative task. One key difference between our unifying method and prompting is that, prompting uses human task descriptive language to unify all tasks to "text generation task", but there is no evidence that the LM can truly understand the prompt's instruction \citep{()} and the choice of different prompts can significantly disturb the performance \citep{}. Therefore, prompts are just unifying the input and output format of all tasks rather than unifying their task nature. However, after our unifying method, we hypothesis that all tasks now share the same ability, more suitable for multi-task training, and can induce better zero-shot performance to unseen tasks. Besides, our unifying methods need minimal prompting, no task descriptive language is used, which significantly reduces performance's sensitivity on handcrafted prompts and save human efforts on designing such prompts.

% \xhk{move to here: why we want to train a universal discriminator for discriminative tasks? We use two observations of discriminative tasks compared with generative tasks in developing our method: First, rather than considering which is the answer for a task from scratch, discriminative tasks only require LMs to judge whether each choice is acceptable and then select the most acceptable one. Second, if a choice is not the correct answer for a given task, we can usually find some violation of real language usage in certain concatenation of the input and the wrong choice. The first observation makes it reasonable to train a specialized discriminator to do the ``true distribution" judgment for each choice separately, and the second observation implies the existence of universal rules behind such judgements across different tasks, motivating us to train a universal discriminator.}