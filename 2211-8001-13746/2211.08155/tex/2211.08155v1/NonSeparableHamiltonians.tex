\documentclass[reprint,superscriptaddress,amsmath,amssymb,aps,pre]{revtex4-1}%{revtex4-2}

%\usepackage{lipsum}
\usepackage{amsmath}
\usepackage{amsthm}
\usepackage{graphicx} % Include figure files
\usepackage{dcolumn}  % Align table columns on decimal point
\usepackage{bm}       % bold math
\usepackage{hyperref} % add hypertext capabilities
\usepackage{xspace}
% \usepackage{minted}   % render computer code
% bibtex main.aux | pdflatex  --shell-escape main.tex 

\usepackage{color}
\usepackage[normalem]{ulem}

\newcommand{\VEC}[1]{\mbox{\boldmath${#1}$}}
\newcommand{\schr}{Schr\"odinger\xspace}
\newcommand{\ps}{phase space\xspace}
\newcommand{\ns}{non\-separable\xspace}

\newcommand{\e}{{\rm e}}
\newcommand{\ep}{\varepsilon}
\newcommand{\cat}{{\cal T}}
\newcommand{\ga}{\gamma}
\newcommand{\tr}{{\rm Tr}}
\newcommand{\be}{\begin{equation}}
\newcommand{\ee}{\end{equation}}
\newcommand{\lra}{\longrightarrow}
\newcommand{\nn}{\nonumber}
\newcommand{\hT}{\hat{\VEC T}}
\newcommand{\hV}{\hat{\VEC V}}
\newcommand{\hH}{\hat{\VEC H}}

\newcommand{\hx}{\hat{\VEC x}}
\newcommand{\hp}{\hat{\VEC p}}
\newcommand{\hU}{\hat{\VEC U}}

\newcommand{\htH}{\hat{\VEC H}_A}
\newcommand{\pa}{\partial}
%\def\oneone{\rlap 1\mkern4mu{\rm l}}
\def\t1{e_{_T}}
\def\v1{e_{_V}}
\def\tvf{e_{_{TV}}}
\def\ct{e_{_{TTV}}}
\def\cv{e_{_{VTV}}}

% \def\tt{e_{_{TTTTV}}}
% \def\tv{e_{_{VTTTV}}}
% \def\vt{e_{_{TTVTV}}}
%\def\vv{e_{_{VTVTV}}}
%\def\fr{_{FR}}
%\def\fft#1#2{{#1 \over #2}}
%\def\ft#1#2{{#1 \over #2}}
%\def\tH{{H_A}}

\newtheorem{theorem}{Theorem}

\begin{document}

\title{Exponential Unitary Integrators for Nonseparable Quantum Hamiltonians}

\author{Maximilian Ciric}
\email{max.ciric@outlook.com}
\affiliation{Department of Physics,~Astronomy~and~Mathematics,
    University~of Hertfordshire, Hatfield, AL10 9AB, UK}

\author{Denys I. Bondar}
\email{dbondar@tulane.edu}
\affiliation{Tulane University, New Orleans, LA 70118, USA}

\author{Ole Steuernagel}
\email{O.Steuernagel@gmail.com}
\affiliation{Department of Physics,~Astronomy~and~Mathematics,
    University~of Hertfordshire, Hatfield, AL10 9AB, UK}


\date{\today}

\begin{abstract}
  Quantum Hamiltonians containing nonsepa\-ra\-ble products of non-commuting operators, such as
  $\hx^m\hp^n$, are problematic for numerical studies using split-operator techniques since such
  products cannot be represented as a sum of separable terms, such as $T(\hp) + V(\hx)$. In the case
  of classical physics, Chin {[Phys. Rev. E {\bf 80}, 037701 (2009)]} developed a procedure to
  approximately represent nonsepa\-ra\-ble terms in terms of separable ones. We extend Chin's idea
  to quantum systems. We demonstrate our findings by numerically evolving the Wigner distribution of
  a Kerr-type oscillator whose Hamiltonian contains the nonsepa\-ra\-ble term
  $\hx^2 \hp^2+\hp^2 \hx^2$.  The general applicability of Chin's approach to any Hamiltonian of
  polynomial form is proven.
\end{abstract}

\maketitle

\section{\label{sec:introduction}Introduction}

Split-operator methods~\cite{Feit_JCP82} are popular across many domains of physics because they
combine the best of two worlds -- simplicity of implementation and preservation of physical
properties such as norm and energy. For the time evolution of Hamiltonian quantum systems (including
those with nonlinearities~\cite{Javanainen_JPA06}), unitary split-operator
integrators~\cite{Yoshida_PLA90} have emerged as reliable workhorses. However, such split-operator
me\-thods are currently restricted to Hamiltonians which are separable~\cite{Cabrera_PRA15}, i.e.
those that are a sum of two terms $T(\hp)$ and $V(\hx)$, each depending only on~$\hp$
and~$\hx$ respectively (throughout, we denote all quantum operators in bold face).

Other recent approaches treating nonsepa\-ra\-ble classical Hamiltonians exist, including one that
yields good long time behaviour for classical dynamics, but which does this at the resource-intensive
price of doubling up the \ps, see~\cite{Tao_PRE16} and references therein. In the case of quantum
systems this is too high a price to pay.

Classical evolution is governed by Poisson-brackets, whose commutators Chin's
method~\cite{Chin_PRE07} combines algebraically such that upon application of exponential
split-operator integrators it extends to the treatment of \ns Hamiltonians. Whilst such classical
evolution is \ps volume-conserving~\cite{Chin_AJP20} quantum evolution is not~\cite{Oliva_PhysA17},
and its evolution is governed by Moyal brackets. Therefore it remained unclear whether Chin's
approach can be adapted to quantum systems.

Here we show that Chin's approach can be extended to quantum systems, see
Sec.~\ref{sec:QuantumChin}, such as Kerr-oscillators, see Sec.~\ref{sec:Kerr}. In
Sec.~\ref{sec:ChinGeneral}, we generalize its application to (\ns) Hamiltonians composed of
polyno\-mials. We use Wigner's quantum \ps representation~\cite{Curtright_Treatise2014} and
investigate its numerical performance in Secs.~\ref{sec:WignerKerr} and~\ref{sec:Numerics} and then
conclude in Sec.~\ref{sec:conclusion}.

\section{Extending Chin's approach to quantum evolution}\label{sec:QuantumChin}

\subsection{\label{subsec:separable_to_nonseparable}From separable to \ns propa\-ga\-tors}

Separable Hamiltonians of the form $\hH = T(\hp) + V(\hx)$ allow for operator splitting. A state is propagated by a small
time-step, $\ep = -i \Delta t/\hbar$, with the unitary propaga\-tor~$\hU(\ep)$. $\hU$ can be split
into the approximate form~\cite{Chin_PRE07,Yoshida_PLA90}
\begin{equation}\label{eq:U_time_ev}
\hU(\ep) = {\rm e}^{\ep \hH} = {\rm e}^{\ep(\hT+\hV)} \approx \prod_{i=1}^N
  {\rm e}^{t_i\ep \hT}{\rm e}^{v_i\ep \hV} ={\rm e}^{\ep\htH},
\end{equation}
here $\hT = T(\hp)$ and $\hV = V(\hx)$. $\htH$ approximates~$\hH$; it can only
contain terms even in $\ep$ since odd terms in~$\htH$ would yield terms even in $\ep$ (which are real-valued) in the
exponential ${\rm e}^{\ep\htH}$, thus breaking unitarity.

Following~\cite{Chin_PRE07}, we will only consider symmetric factorisation schemes
for~(\ref{eq:U_time_ev}) such that the weighting coefficients are either $t_1=0$ and
$v_i=v_{N-i+1}$, $t_{i+1}=t_{N-i+1}$, or $v_N=0$ and $v_i=v_{N-i}$, $t_{i}=t_{N-i+1}$.

Then, according to the Baker-Campbell-Hausdorff formula~\cite{Yoshida_PLA90}, $\htH$ has the form
\begin{eqnarray}
&&\htH = \t1\hT+\v1\hV+\ep^2\ct[\hT\hT\hV]\nn\\
&&\qquad\quad+\ep^2\cv[\hV\hT\hV] +O(\ep^4) \; ,
\label{hopbk}
\end{eqnarray}
where condensed commutator brackets $[\hT\hT\hV]\equiv[\hT,[\hT,\hV]]$,
$[\hT\hV\hT\hV]\equiv[\hT,[\hV,[\hT,\hV]]]$, etc., are used. The coeffi\-ci\-ents $\t1$, $\cv$,
etc., are functions of $\{t_i\}$ and $\{v_j\}$.

By choosing $\{t_i\}$ and $\{v_j\}$ such that $\sum_i t_i = 0 = \sum_j v_j$, we impose that \be
\t1=\v1=0. \label{tvzero} \ee

If we also impose that %either
$\cv=0$, or~$\ct=0$, then the approximate
propa\-ga\-tor~(\ref{eq:U_time_ev}) codes for \ns Hamiltonians~$\htH$, either of the form
\be \label{eq:HTTV_HVVT} \htH \approx \hH_{TTV} \propto [\hT\hT\hV] {\rm, \;\; or}\;\; \htH \approx
\hH_{VTV} \propto [\hV\hT\hV].  \ee

To summarise, combined separable terms in Eq.~(\ref{eq:U_time_ev}) can emulate specific
nonsepa\-ra\-ble operator products~(\ref{eq:HTTV_HVVT}).

Chin showed in \cite{Chin_PRE07} that the specific symmetric, ninth order
product~(\ref{eq:U_time_ev}), with the coefficients $ v_0 = -2 ( v_1 + v_2)$, $ t_1 = -t_2$,
$v_2 = -\frac{1}{2}v_1$ and $v_1 = \frac{1}{t_2^2}$, enables us to impose removal of the third order
term $[\hV \hT \hV]$, resulting in
%\begin{subequations}
\begin{eqnarray}\label{eq:Exp9}
\hU_9(\ep) & \approx & e^{\ep v_2 \hV} e^{\ep t_2 \hT} e^{\ep v_1 \hV} e^{\ep t_1 \hT} e^{\ep v_0 \hV} e^{\ep t_1 \hT} e^{\ep v_1 \hV} e^{\ep t_2 \hT} e^{\ep v_2 \hV} \nonumber \\
  & \approx & \exp(\ep^3[\hT \hT \hV] + \ep^5 E_5 + \ep^7 E_7 + \dots) \ . 
\end{eqnarray}
%\end{subequations}
Here, $t_2$ is a free parameter that can be chosen to mini\-mise errors introduced through $E_5$;
following~\cite{Chin_PRE07}, we use $t_2 = -6^{\frac{1}{3}}$.

\subsection{\label{subsec:schrodingerPicture}Propagating a state in the \schr picture}

In split operator techniques, when a propagator~$\exp[\ep(T(\hp)+ V(\hx))]$, with a small time step
$\ep = -i \Delta t/\hbar$, is applied to state~$\psi(\hx,t)$, we end up applying the sequence of
maps
\begin{eqnarray}\label{eq:unitary_op_f}
\psi(&\hx,& t+\ep)  = \exp[\ep(\hT+ \hV)]
                             \;\cdot \psi(\hx, t) \nonumber\\
  & \approx &
              \mathcal{F}_{\hp \rightarrow \hx} \cdot \exp{ %\left
              [ \ep \hT %\right
              ] \cdot } \;
              \mathcal{F}_{\hx \rightarrow \hp}\cdot \exp{ %\left
              [ \ep \hV %\right
              ]}\cdot \psi(\hx, t) .
\end{eqnarray}
Here, $\cal F$ denotes fast Fourier transforms (and their inverses, in obvious notation) central to
the speedup and numerical stability associated with the use of split opera\-tor techniques.  To give
an example, expression~(\ref{eq:Exp9}) entails the application of~$\cal F$ at eight times per
step~$\ep$.

\section{\label{sec:Kerr}Product terms in Kerr oscillator Hamiltonian}

The single-mode Kerr oscillator, in its simplest form, has the energy of the harmonic oscillator
squared and is therefore ana\-ly\-tically fully solvable. Explicitly, its Hamiltonian has the form
\begin{equation}\label{eq:kerr_h}
  \hH_{\textrm{Kerr}} = \left ( \frac{\hp^2}{2} + \frac{\hx^2}{2} \right )^2
   = \frac{\hp^4}{4} +  \frac{[\hp^2,\hx^2]_+}{4} + \frac{\hx^4}{4} ,
 \end{equation}
 where we used the anti-commutator $[\hat{a},\hat{b}]_+ \equiv \hat{a}\hat{b} + \hat{b}\hat{a}$.
 The quantum Kerr effect comes about due to the self-interaction of photons in nonlinear
 media~\cite{Kirchmair_NAT13}. Its dynamics is non-trivial and periodic with a recurrence time of
 $\tau = \frac{\pi}{\hbar}$; its \ps current follows circles~\cite{Oliva_Kerr_18}.

 We now show that its nonsepa\-ra\-ble terms $[\hp^2, \hx^2]_+$ can be cast into the shape of~$\htH$
 in Eq.~(\ref{hopbk}).  To first order in the time step $\ep = -i \Delta t/\hbar$, the
 Moyal bracket~\cite{Cabrera_PRA15,Oliva_PhysA17} of quantum \ps dynamics agrees with the classical
 Poisson bracket~\cite{Chin_PRE07,Oliva_PhysA17}. We therefore have to hope that the
 commutator~$[\hT \hT \hV]$ in Eq.~(\ref{eq:Exp9}) behaves similarly to the classical Poisson
 bracket--based Lie operators ana\-lysed by Chin~\cite{Chin_PRE07}.

Following Ref.~\cite{Chin_PRE07}, we therefore try the ansatz of a second order polynomial for~$\hT$ and a fourth
order polynomial for~$ \hV$. The choices~$\hT=\frac{c_T}{2\sqrt{2}}\hp^2$ and
$ \hV = \frac{c_V}{12}\hx^4$ yield $[\hT \hT \hV] = \frac{c_T^2 c_V}{96} ( \hx^4 % \cdot
\hp^4+ \hp^4 %\cdot
\hx^4- 2 \; \hp^2% \cdot
\hx^4 %\cdot
\hp^2)$. With $c_T=1=c_V$ and using Heisenberg's commutation relation
$[\hp,\hx] =\frac{\hbar}{\rm i}$ this simplifies~\cite{Munoz_JPCS16} to
\begin{eqnarray}\label{eq:TTV}
 [\hT \hT \hV] = -\frac{\hbar^4}{4} - \frac{\hbar^2}{4} [\hp^2,\hx^2]_+, 
\end{eqnarray}
with a real-valued constant term which gives rise to a global phase that can be ignored or
subtracted out.

Alternatively, one can use $\hT=\frac{\hp^4}{12}$ and $\hV=\frac{\hx^2}{2\sqrt{2}}$ while
swapping $\hV \leftrightarrow \hT$ in expression~(\ref{eq:Exp9}) giving the same final
result~(\ref{eq:TTV}).

We have thus established that a propagator using the separable terms $\hT$ and $\hV$ in~$\hU$, in
Eq.~(\ref{eq:U_time_ev}), can gene\-rate a propa\-ga\-tor featuring the product term $[\hp^2,\hx^2]_+$,
which represents the \ns middle term of the Kerr Hamiltonian in Eq.~(\ref{eq:kerr_h}).


\section{\label{sec:WignerKerr}Propagation of mixed states using Wigner's \ps approach}

Instead of considering pure states in the \schr picture, as in
Sec.~\ref{subsec:schrodingerPicture}, we now study the time evolution of general mixed quantum
states $W(x,p,t)$, in Wigner's \ps representation. $W$'s dynamics is described by Moyal's equation of
motion~\cite{Moyal_MPCPS49}
\begin{equation}\label{eq:moyal_motion}
    \frac{\partial W}{\partial t} = \{\!\!\{ {H} , W \}\!\!\} = \frac{1}{\rm i \hbar} \hat{\cal G}[W] \; .
\end{equation}
Here, the Hamiltonian $H$, is given by the Wigner transform~\cite{Hancock_EJP04} of $\hH$, which in
the case of the Kerr Hamiltonian~(\ref{eq:kerr_h}) is
$ {H} = \left ( \frac{{p}^2}{2} + \frac{{x}^2}{2} \right )^2 - \frac{\hbar^2}{4}$.  The generator of
motion $\hat{\cal G}$ is the Lie superoperator associated with the Moyal bracket
\cite{Curtright_Treatise2014}, namely

\begin{align}\label{EqMoyalBraket}
    \{\!\!\{ f, g\}\!\!\} &\equiv \frac{f \star g - g \star f}{i\hbar} \notag\\
        &= \frac{2}{\hbar} f(x,p) \sin\left[\frac{\hbar}{2} \left( 
        \overleftarrow{\frac{\partial}{\partial x}} \overrightarrow{\frac{\partial}{\partial p}}
        - \overleftarrow{\frac{\partial}{\partial p}} \overrightarrow{\frac{\partial}{\partial x}}
    \right) \right] g(x,p),
\end{align}
where $\star$ denotes the Groenewold-Moyal product \cite{Groenewold_Phys46,Curtright_Treatise2014}
\begin{align}\label{EqMoyalStar}
    \star &\equiv \exp\left[\frac{i\hbar}{2} \left( 
        \overleftarrow{\frac{\partial}{\partial x}} \overrightarrow{\frac{\partial}{\partial p}}
        - \overleftarrow{\frac{\partial}{\partial p}} \overrightarrow{\frac{\partial}{\partial x}}
    \right) \right] \\
    &= \sum_{n=0}^{\infty} \frac{(i\hbar)^n}{2^n n!} \left( 
        \overleftarrow{\frac{\partial}{\partial x}} \overrightarrow{\frac{\partial}{\partial p}}
        - \overleftarrow{\frac{\partial}{\partial p}} \overrightarrow{\frac{\partial}{\partial x}}
    \right)^n
\end{align}
in which the arrows denote the `direction' of differentiation:
$f\overrightarrow{\frac{\partial}{\partial x}} g = g\overleftarrow{\frac{\partial}{\partial x}} f =
f \frac{\partial}{\partial x} g$.

%
%%
%


Taylor's expansion of Moyal's bracket~\eqref{EqMoyalBraket} yields
\begin{align}\label{EqClassicalLimitMoyal}
    \{\!\!\{ f, g\}\!\!\} = \{f, g\} + \mathcal{O}\left( \hbar^n [\mbox{derivatives}^{2n}]\Bigr\vert_{n \geq 2} \right) .
\end{align}
To lowest order, this gives us Poisson's bracket
$\{f, g\} = \frac{\partial f}{\partial x} \frac{\partial g}{\partial p} - \frac{\partial f}{\partial
  p} \frac{\partial g}{\partial x}$ of classical mechanics. We see that in Wigner's representation
the time evolution is formally similar to that in the classical case treated by
Chin~\cite{Chin_PRE07}.  Wigner's representation is additionally of interest, because it can be
treated efficiently numerically since Moyal's equation of motion~(\ref{eq:moyal_motion}) can be cast
into the form of a Schr\"odinger equation~\cite{Cabrera_PRA15,Kolaczek_2020}, see next
Sec.~\ref{sec:Numerics}.

Equation~\eqref{EqMoyalStar} connects the Wigner transform~\cite{Hancock_EJP04,Curtright_Treatise2014} of
non-commutative Hilbert space operator products $f(\hx, \hp) \cdot g(\hx, \hp)$ with
non-commutative $\star$-products $f(x,p) \star g(x,p)$ on \ps:
\begin{align}
    f(x,p) \star g(x,p) & \Longleftrightarrow f(\hx, \hp) \cdot g(\hx, \hp), \label{EqMoyalStartCorrespondance} \\
    \{\!\!\{ f(x,p), g(x,p) \}\!\!\} & \Longleftrightarrow
    [f(\hx, \hp), g(\hx, \hp)]. \label{EqMoaylCommutaorCorrespondence}
\\\nonumber
\end{align}


\section{\label{sec:Numerics}Numerical considerations}

In Wigner-Weyl transformed variables, we can give $\hat{\cal G}$ of Eq.~(\ref{eq:moyal_motion}) the
explicit form~\cite{Cabrera_PRA15}
\begin{eqnarray}\label{eq:generator_op}
  \hat{\cal G} & = & {H} \left ( \hat{x} - \frac{\hbar}{2}\hat{\theta} , \hat{p}
                     + \frac{\hbar}{2}\hat{\lambda} \right ) - {H} \left ( \hat{x}
                     + \frac{\hbar}{2}\hat{\theta} , \hat{p} - \frac{\hbar}{2}\hat{\lambda} \right ) \quad \\
               & \equiv & \hat{H}_{-,+} - \hat{H}_{+,-}  \quad ,  \label{eq:Hplusminus}
\end{eqnarray}
with the commutation relations \cite{Cabrera_PRA15,Bondar_PRL12}:
\begin{align}
\label{commutation-rels}
 {[} \hat{x} , \hat{p} {]_-} = 0,  \quad
 {[} \hat{x} , \hat{\lambda} {]_-} = i, \quad
 {[} \hat{p} , \hat{\theta} {]_-} = i, \quad
 {[} \hat{\lambda} , \hat{\theta} {]_-} = 0 \; ,
\end{align}
which span a suitable Wigner-Weyl `Hilbert \ps'~\cite{Cabrera_PRA15}. Hence,
\begin{eqnarray}\label{eq:EvolutionCompactlyWritten}
\hat{\cal U} & = & \exp \left ( \ep \left ( \hat{H}_{-,+} - \hat{H}_{+,-} \right ) \right ) \; ,
\end{eqnarray}
and for time-independent Hamiltonians
\begin{equation}\label{eq:moyal_time_ev}
    W(t) = \exp{(-i \frac{t-t_0}{\hbar} \hat{\cal G})}[W(t_0)]  = \hat{\cal U}[W(t_0)] \; .
\end{equation}

We emphasise that in choosing the $(x,\theta)$-represen\-tation~\cite{Cabrera_PRA15},
for Eq.~(\ref{commutation-rels}), using suitable Bopp operators~\cite{bopp1956mecanique}
\begin{align}\label{xThetaRepresentation}
  \hat{x} = x, \quad
  \hat{p} = i \frac{\partial}{\partial_\theta}, \quad
  \hat{\lambda} = -i \frac{\partial}{\partial _x}, \text{ and }\;
  \hat{\theta} = \theta , \quad 
\end{align}
Eq.~(\ref{eq:moyal_motion}) becomes Schr\"{o}dinger-like, making it possible to apply efficient
numerical propagation employing fast Fourier transform
methods~\cite{Cabrera_PRA15,Arnold_SIAM96,Thomann_INMPDE17,Kolaczek_2020}. This is very useful for
systems that cannot be modelled as pure states, such as in the presence of decoherence.

Using $\hat{P}_{\pm} \equiv \hat{p} \pm \frac{\hbar}{2} \hat \lambda$ and
$\hat{X}_{\pm} \equiv \hat{x} \pm \frac{\hbar}{2} \hat \theta$, we can express
$\hat{\cal U}$~(\ref{eq:EvolutionCompactlyWritten}) for the Kerr Hamiltonian~\eqref{eq:kerr_h} as
\begin{widetext}
\begin{subequations}
\begin{eqnarray}
  \hat{\cal U}_{\textrm{Kerr}} & = & \exp{\left[ \frac{\ep}{4} \left ( \hat{P_{+}}^4 - \hat{P_{-}}^4
                                     + \left [ \hat{P_{+}}^2\hat{X_{-}}^2 - \hat{P_{-}}^2\hat{X_{+}}^2
                                     + \hat{X_{-}}^2\hat{P_{+}}^2 - \hat{X_{+}}^2\hat{P_{-}}^2 \right ]
                                     + \hat{X_{-}}^4 - \hat{X_{+}}^4 \right ) \right]}
                                     \label{eq:U_KerrMod} \\
                               & = & \exp{\left [ \frac{\ep}{4} \left ( \hat{P_{+}}^4
                                     - \hat{P_{-}}^4 \right ) \right ]}
                                     \exp {\left[ \frac{\ep}{4} \left( [ \hat{P_{+}}^2,\hat{X_{-}}^2]_+
                                     - [ \hat{P_{-}}^2,\hat{X_{+}}^2 ]_+ \right) \right]}
                                     \exp {\left [ \frac{\ep}{4} \left ( \hat{X_{-}}^4 - \hat{X_{+}}^4\right ) \right ]}
                                     \; +{\cal O}(\ep^2) \;
                                     \label{eq:U_KerrMod2} \; .
\end{eqnarray}
\end{subequations}
%\end{widetext}
According to Eq.~(\ref{eq:TTV}), the appearance of anti-commutators in the middle exponential of
expression~(\ref{eq:U_KerrMod2}) allows us to express the contribution from the central product term
in the Kerr Hamiltonian~(\ref{eq:kerr_h}) as a \emph{single} product of form~(\ref{eq:Exp9}); for an
efficient implementation in Python see~\cite{Bondar_github}.
\end{widetext}


% \begin{figure}[t]
% \includegraphics[scale=0.465]{figures/logplot.png}
% \caption{Plot of $\log_{10} | E(t,dt) - E_0 |_{\textrm{max}}$ versus $\log_{10} ( dt )$, for initial
%   state~(\ref{eq:W0}) with coordinate displacement~$x_0=2$, establishes that the algorithm's energy
%   fluctuations $| E(t,dt) - E_0 |$ scale with~${\cal O}(dt)$ for the case of the Kerr
%   system~(\ref{eq:kerr_h}). Here, $E(t,dt)$ stands for numerically determined energy at time $t$
%   using time step~$dt$ and $E_0=E(0,dt)$. Blue represents numeri\-cal results and the red line is
%   a linear regression fit.}
% \label{fig:err_logplot_energies}
% \end{figure}

\subsection{\label{subsec:WignerKerrNumerics} Error Scaling for Kerr System}

In the following, we set $\hbar = 1$ and use  coherent states 
\begin{equation}\label{eq:W0}
    W \left ( x , p, t=0 \right ) = \frac{1}{\pi} \exp \left ( -(x-x_0)^2 - (p-p_0)^2 \right )
  \end{equation}
as initial states. For an example of their time evolution, see Fig.~\ref{fig:KerrIllustration}.
  
Using exponential propagators (whose action is time-reversible), we confirmed that Chin's approach
preserves the state's norm at machine precision.

We checked for energy and phase stability, varying the time step $dt$. 
% Surprisingly, energy and phase fluctuations scale with~${\cal O}(dt)$, see
% Fig.~\ref{fig:err_logplot_energies}.  This is surprising since in the case of the Kerr
% system~(\ref{eq:kerr_h}) the~${\cal O}(\ep^5)$-term in Eq.~(\ref{eq:Exp9}) is nonzero.
In the case of a classical system with similar structure Chin reports~\cite{Chin_PRE07}, in accord with Eq.~(\ref{eq:Exp9}),
scaling with order ${\cal O}(\ep^5)/{\cal O}(\ep^3) \sim {\cal O}(dt^{2/3})$; this is roughly what we observed
here, in the quantum Kerr case, as well.
% worse than the
% fortuitous %(and unexplained)
% ${\cal O}(dt)$-scaling for the quantum Kerr oscillator we report in
% Fig.~\ref{fig:err_logplot_energies}.


% We note in passing that numerical propagation can be more forgiving in the quantum case than the
% classical case since the formation of very fine features in \ps is suppressed in the quantum
% case~\cite{Zurek_NAT01,Oliva_Shear_19} (although we cannot see whether that might explains the
% fortuitous scaling observed here).

% We observed this scaling in a numerical implementation for which we made sure that the rest of the
% algorithm propagated with order ${\cal O}(dt^2)$ by using Strang-splitting~\cite{Strang_SIAM68} for
% the outer (bracketing) exponential terms in Eq.~(\ref{eq:U_KerrMod2}) representing the potential
% term~${\hx^4} /  {4}$ of~$\hH_{\textrm{Kerr}}$~(\ref{eq:kerr_h}).

The period of our Kerr system is $\tau = \pi$, see Fig.~\ref{fig:KerrIllustration}~(c). As a proxy
for phase drift, associated with this algorithm, we determine the wave function overlap at
recurrence time~$\tau$ and find that
$ | \langle W(0) | W(\pi) \rangle | -1 \sim {\cal O}(dt^{1.2})$, a scaling better than that of the
energy fluctuations. For this we could not find a quantitative explanation.

% \begin{figure}[t]
%    \begin{minipage}[h]{0.98\columnwidth}
%      \includegraphics[width=0.5\textwidth]{figures/9_Exp__WKerr__Nx128_Np128_X6_P6_T3.141500.png}
%      \put(-106,102){\rotatebox{0}{{\bf (a)} $\hU_9$}}
%      \includegraphics[width=0.5\textwidth]{figures/7_Exp__WKerr__Nx128_Np128_X6_P6_T3.141500.png}
%      \put(-100,111){\rotatebox{0}{{\bf (b)} $\hU_7(\ep)$}}
%      \caption{Kerr system~(\ref{eq:kerr_h}) evolves an initial coherent state~(\ref{eq:W0})
%        $W(x_0, p_0)=W(3/\sqrt{2}, 3/\sqrt{2})$ using timesteps $dt=0.0005$. At the revival
%        time~$\tau = \pi$ the initial state recurs with lower accuracy in the case of $\hU_7$ (right
%        panel) than $\hU_9$ (left panel).}
%     \label{fig:9Exp_vs_7Exp}
%    \end{minipage}
%   \end{figure}

\subsection{\label{subsec:7Exp} Modification of Chin's expression}

In the Kerr-case studied here, it is possible to modify Chin's expression~(\ref{eq:Exp9}) by
removing first and last terms

\begin{figure*}[t]
%   \begin{minipage}[h]{1.59\columnwidth}
   \begin{minipage}[h]{1.999\columnwidth}
     % \centering %    \includegraphics[width=0.03\textwidth]{figures/Fig_1__9Exp/ColorBar.pdf}
     \includegraphics[width=0.235\textwidth]{figures/Fig_1__9Exp/10473__t_1.047_128x128_R_3_Kerr__MaxsPropagator_9_Exp__2ndStrang_psi_2.121,2.121__dt0.0001.pdf}
     \put(-106,120){\rotatebox{0}{{\bf (a)} $\substack{t=1.047 \approx \tau/3 \\ (dt=0.0001)}$ }}     \includegraphics[width=0.235\textwidth]{figures/Fig_1__9Exp/20945__t_2.095_128x128_R_3_Kerr__MaxsPropagator_9_Exp__2ndStrang_psi_2.121,2.121__dt0.0001.pdf}
     \put(-106,120){\rotatebox{0}{{\bf (b)} $\substack{t=2.095 \approx \frac{2}{3} \tau \\ (dt=0.0001)}$ }} \includegraphics[width=0.235\textwidth]{figures/Fig_1__9Exp/31416__t_3.142_128x128_R_3_Kerr__MaxsPropagator_9_Exp__2ndStrang_psi_2.121,2.121__dt0.0001.pdf}
     \put(-106,120){\rotatebox{0}{{\bf (c)} $\substack{t=3.142 \approx \tau \\ (dt=0.0001)}$ }}
    \includegraphics[width=0.0475\textwidth]{figures/Fig_1__9Exp/ColorBar.pdf} \includegraphics[width=0.235\textwidth]{figures/Fig_1__9Exp/2094__t_3.141_128x128_R_3_Kerr__MaxsPropagator_9_Exp__2ndStrang_psi_2.121,2.121__dt0.0015.pdf}
     \put(-106,120){\rotatebox{0}{{\bf (d)} $\substack{t=3.142 \approx \tau \\ (dt=0.0015)}$ }}
     \\ \includegraphics[width=0.23\textwidth]{figures/Fig_1__7Exp/10473__t_1.047_128x128_R_3_Kerr__OlesPropagator_7_Exp__2ndStrang_psi_2.121,2.121__dt0.0001.pdf} \includegraphics[width=0.23\textwidth]{figures/Fig_1__7Exp/20945__t_2.095_128x128_R_3_Kerr__OlesPropagator_7_Exp__2ndStrang_psi_2.121,2.121__dt0.0001.pdf} \includegraphics[width=0.23\textwidth]{figures/Fig_1__7Exp/31416__t_3.142_128x128_R_3_Kerr__OlesPropagator_7_Exp__2ndStrang_psi_2.121,2.121__dt0.0001.pdf}
         \includegraphics[width=0.0475\textwidth]{figures/Fig_1__9Exp/ColorBar.pdf} \includegraphics[width=0.235\textwidth]{figures/Fig_1__7Exp/2094__t_3.141_128x128_R_3_Kerr__OlesPropagator_7_Exp__2ndStrang_psi_2.121,2.121__dt0.0015.pdf}
     % \includegraphics[width=0.3\textwidth]{figures/02080.png}
     % \put(-100,130){\rotatebox{0}{{\bf (b)} $t=2.08$}}
     % \includegraphics[width=0.3\textwidth]{figures/03120.png}
     % \put(-106,130){\rotatebox{0}{{\bf (c)} $t=3.13\approx \pi$}}
         \caption{Kerr system~(\ref{eq:kerr_h}) evolves an initial coherent state~(\ref{eq:W0}) with
           amplitude `3' [$W_0(x_0=3/\sqrt{2}, p_0=3/\sqrt{2})$] using timesteps $dt=0.0001$ in
           columns {\bf (a)} to {\bf (c)} and $dt=0.0015$ in column {\bf (d)}. We observe that the
           initial state recurs at the system's recurrence time~$\tau = \pi $~columns {\bf (c)} and
           {\bf (d)} and fractional revivals of the initial states with approximate three-fold
           symmetry~\cite{Oliva_Shear_19,Averbukh_PLA89,Robinett_PR04} form at times $\tau / 3$ and
           $2 \tau / 4$, see columns {\bf (a)} and {\bf (b)}, respectively.  The same parameters are
           used in top and bottom row, except that the top row employs the more accurate propagator
           $\hU_9$ of Eq.~(\ref{eq:Exp9}) whereas the bottom row uses $\hU_7$ of
           Eq.~(\ref{eq:Exp7}), the numerical errors become more pronounced with both, increasing
           propagation time and increasing timesteps.}
    \label{fig:KerrIllustration}
   \end{minipage}
  \end{figure*}
%
  yiel\-ding the approximation
\begin{subequations}
  \begin{eqnarray}\label{eq:Exp7}
  \hU_7(\ep) & = & e^{\ep t_2 \hT} e^{\ep v_1 \hV} e^{\ep t_1 \hT}
  e^{\ep v_0 \hV} e^{\ep t_1 \hT} e^{\ep v_1 \hV} e^{\ep t_2 \hT} \qquad \\
  & = & \hU_9(\ep) - \frac{\ep^3}{t_2^3} [\hV\hV\hT] + {\cal O}(\ep^5) .
  \end{eqnarray}
\end{subequations}

  We use the modified coefficients $ v_0 = -2 v_1$, $ t_1 = -t_2$, $v_1 =
\frac{1}{t_2^2}$, %enables us to remove the third order term $[\hV \hT \hV]$:
and again~$\hT=\frac{c_T}{2\sqrt{2}}\hp^2$ and $ \hV = \frac{c_V}{12}\hx^4$, with the
final result~\cite{Munoz_JPCS16}

\begin{equation}\label{eq:TTV7}
\begin{aligned}
  \hU_7(\ep) &= 1 + \ep^3\left( -\frac{\hbar^4}{4} - \frac{\hbar^2}{4} [\hp^2,\hx^2]_+
  + \frac{\hbar^2}{9 \sqrt{2} \, t_2^3} \hx^6 \right) \\&+ {\cal O}(\ep^5). \qquad
\end{aligned}
\end{equation}

This is similar to result~(\ref{eq:TTV}). We have to compensate for the unwanted term in $\hx^6$, by
subtracting $ \frac{\hbar^2}{9 \sqrt{2} \, t_2^3} \hx^6$ from potential terms in
Eqs.~(\ref{eq:unitary_op_f}) and~(\ref{eq:U_KerrMod2}), but gain the advantage of having to
numerically calculate fewer terms.

Whether a form like~(\ref{eq:Exp7}) that is as useful as~(\ref{eq:TTV7}) can be found in the general case, we do not know at this stage. We emphasize that $\hU_7$ has fewer product terms and runs a little faster but also performs worse than~$\hU_9$ of Eq.~(\ref{eq:Exp9}) in absolute terms, see Fig.~\ref{fig:KerrIllustration}. The errors in energy and phase both scale roughly with
${\cal O}(dt^{2/3})$, similarly to and worse than in the case of~$\hU_9$, respectively.

Further discussions of such questions is beyond the scope of this work.

\section{\label{sec:ChinGeneral}Chin's approach is general}

In order to show that Chin's approach is generally applicable, let us prove
\begin{theorem}\label{TheoremGenralDecomposition}
  Any polynomial $p(\hx, \hp)$ of $\hx$ and $\hp$ can be written as a finite linear combination of
  $[\hx^n\hx^n\hp^m] \equiv [\hx^n, [\hx^n, \hp^m]]$ and
  $[\hp^m\hx^n\hp^m] \equiv [\hp^m, [\hx^n, \hp^m]]$.
\end{theorem}
\begin{proof}
  Let us provide a constructive proof. A polynomial $\Pi(\hx, \hp)$ is mapped to $P(x,p)$ in \ps,
  according to Eq.~\eqref{EqMoyalStartCorrespondance}. Assume $x^N p^M$ is its leading term,
  namely, $P(x,p)$ is a polynomial of order $(N,M)$.
  
Via Eq.~\eqref{EqMoaylCommutaorCorrespondence}, a double commutator $[\hx^n \hx^n \hp^m]$ corresponds to  $\{\!\!\{ x^n, \{\!\!\{x^n, p^m \}\!\!\} \}\!\!\}$. In fact, the latter are polynomials because the Moyal bracket~\eqref{EqMoyalBraket} is obtained by differentiating its arguments. According to Eq.~\eqref{EqClassicalLimitMoyal}, the leading term of the polynomial $\{\!\!\{ x^n, p^m \}\!\!\}$ is $x^{n-1}p^{m-1}$; hence, the leading term of  $\{\!\!\{ x^n, \{\!\!\{x^n, p^m \}\!\!\} \}\!\!\}$ is $x^{2n-2} p^{m-2}$. Likewise, a double commutator $[\hp^m \hx^n \hp^m]$ corresponds to the polynomial $\{\!\!\{ p^m, \{\!\!\{x^n, p^m \}\!\!\} \}\!\!\}$ with the leading term of order $x^{n-2} p^{2m-2}$.

The set of polynomials 
\begin{align}
  \Big[ \{\!\!\{ x^n, \{\!\!\{x^n, p^m \}\!\!\} \}\!\!\}, \{\!\!\{ p^m, \{\!\!\{x^n, p^m \}\!\!\} \}\!\!\}
  \Big]_{n=1,2,\ldots, N + 2}^{m=1,2,\ldots, M+2}
\end{align}
is linearly independent and large enough to span the set of polynomials of order $(N,M)$, including $P(x,p)$.
\end{proof}

We observe that in the above proof all Moyal brackets $\{\!\!\{.,. \}\!\!\} $ can be substituted by Poisson brackets $\{.,.\} $ whilst leaving the argument intact: Chin's approach applies to general classical systems as well.

Theorem~\ref{TheoremGenralDecomposition} prescribes how any polynomial quantum Hamiltonian can be
decomposed into the Hamiltonians of the form \eqref{eq:HTTV_HVVT}. Hence, Chin's algorithm is very
general.

\section{\label{sec:conclusion}Conclusion}

We have shown that Chin's method~\cite{Chin_PRE07} for the
propa\-gation of classical \ns Hamiltonians can be adopted to quantum systems.
Chin's method is general, and therefore allows for the universal treatment of \ns
Hamiltonians using split-operator techniques.

% Specifically for Kerr systems we find that numerical
% errors can scale with order ${\cal O}(dt)$, which is better than suggested by theory.

\begin{acknowledgments}
D.I.B. was supported by by the W. M. Keck Foundation and Army Research Office (ARO) (grant W911NF-19-1-0377; program manager Dr.~James Joseph). The views and conclusions contained in this document are those of the authors and should not be interpreted as representing the official policies, either expressed or implied, of ARO or the U.S. Government. The U.S. Government is authorized to reproduce and distribute reprints for Government purposes notwithstanding any copyright notation herein.
\end{acknowledgments}

\bibliography{Ole_Bibliography}

% \newpage

% \section{Appendix}


% \begin{figure*}[h]
% %   \begin{minipage}[h]{1.59\columnwidth}
%    \begin{minipage}[h]{1.999\columnwidth}
% \includegraphics[scale=0.465]{figures/logplot.png}
% \caption{Plot of $\log_{10} | E(t,dt) - E_0 |_{\textrm{max}}$ versus $\log_{10} ( dt )$, for initial
%   state~(\ref{eq:W0}) with coordinate displacement~$x_0=2$, establishes that the algorithm's energy
%   fluctuations $| E(t,dt) - E_0 |$ scale with~${\cal O}(dt)$ for the case of the Kerr
%   system~(\ref{eq:kerr_h}). Here, $E(t,dt)$ stands for numerically determined energy at time $t$
%   using time step~$dt$ and $E_0=E(0,dt)$. Blue represents numeri\-cal results and the red line is
%   a linear regression fit.}
% % \label{fig:err_logplot_energies}
% % \end{figure}
%    \end{minipage}
%   \end{figure*}



\end{document}

%%% Local Variables:
%%% mode: latex
%%% TeX-master: t
%%% End:
