\pagenumbering{gobble}

\begin{widetext}
\newpage
\noindent{\bf Responses to REFEREE’S REPORTS and Resubmission of  EPJP-D-22-046653\\
Title: Exponential Unitary Integrators for Nonseparable Quantum Hamiltonians\\
Authors:  Maximilian \'Ciri\'c et al.
\\\\}
We would like to thank the reviewers for their efforts and reviews. They helped us to improve this
manuscript and catch several typos and passages that benefited from improvements. Thank you! We highlighted changes in \new{red} in the manuscript.
\\\\
\\
\underline{The first reviewer suggested that the paper could be published but suggested a few
  changes.}
\\\\

\underline{Specifically: The first reviewer critizised \emph{
    ``(In item 2. and) 1. The authors’ indiscriminated use of bold face letters is}}\\
\underline{very confusing and misleading."}
\\\\\\
This comment made us add the following clarifying paragraph on page 1, right column:
\begin{quote}\indent{\new{ Here, we deal with classical and quantum operators as well as their eigenvalues.
  Following~\cite{Cabrera_PRA15}, we therefore adopt the following notation. Throughout, bold
  lettering ($\bm x, \bm p$) refers to quantum scalars whereas regular lettering ($ x, p$) refers to
  their classical counterparts or functions, such as~$T$ or $V.$ Hats [($\hp$, $\hx$) vs. ($\hat p$,
  $\hat x$)] indicate their respective operators.}}
\end{quote}
We found that this is probably one of the most comprehensive ways of dealing with the multiple
meanings of the quantities we consider, moreover, we follow the convention previously established
in~\cite{Cabrera_PRA15}.
\\\\\\\\
\underline{The first reviewer also pointed out a typo and suggested to rephrase Eq.~(6) in item
  `5'.}
\\\\\\
We follow this advice and now write Eq.~(\ref{eq:unitary_op_f}) as
\new{\begin{eqnarray}%\label{eq:unitary_op_f}
\psi(&{\bm x},& t+\Delta t)  = \exp[\ep(T(\hp)+ V(\hx))]
                             \; \psi({\bm x}, t) \nonumber\\
  & \approx &  \exp(\ep T(\hp)) \exp(\ep V(\hx))
                             \; \psi({\bm x}, t) \nonumber\\
  & = &
              \mathcal{F}_{{\bm p} \rightarrow {\bm x}} \exp(\ep T({\bm p})) \;
              \mathcal{F}_{{\bm x} \rightarrow {\bm p}} \exp(\ep V({\bm x})) \psi({\bm x}, t) .
\end{eqnarray}}
\\\\
\underline{In response to the first reviewer's item `3' we now state that}
\\\\
{\color{red} since $\htH$ must be hermitian, it has to have an even-power expansion in $\ep$.  }
\\\\
\underline{In response to the first reviewer's item `4' we now use the suggested term}
\\\\
``product of nine exponentials''.
\\\\\\
\underline{In response to the first reviewer's items `6' and `8' we now make explicit, why we use the
  Wigner representation, namely:}
\begin{quote}
\new{We employ Wigner's representation for the following four reasons: firstly, many dissipative
  systems use coupling terms of product form, so Chin's approach allows us to avoid iterations such
  those as used in Eqs.(63) and~(64) of~\cite{Cabrera_PRA15}.  Secondly, the Wigner representation
  describes mixed systems which result from such dissipative couplings. Thirdly, it can be
  efficiently implemented (in Schr\"{o}dinger equation-like form, see below
  and~\cite{Cabrera_PRA15}). Finally, comparison of the quantum with Chin's classical description
  becomes transparent when using the Wigner representation since it describes $W$'s dynamics using
  Moyal brackets~\cite{Moyal_MPCPS49}, the quantum analog of Poisson brackets:}
\end{quote}
\newpage
\underline{Also in response the first reviewer's item `6' ``optional" suggestion:}
\\\\
We have implement numerical method for the Schr\"{o}dinger equation and found that it follows the same trends. Moreover, now we have added the github repository~\cite{Bondar_github_nonsep} (see the Data Availability Statement) with codes for the analysis of the Wigner and Schr\"{o}dinger propagator. Also we updated the discussion at the end of Section~\ref{subsec:WignerKerrNumerics}.
\\\\
\underline{The first reviewer, in item `7', wonders whether `closed orbits' might lead to special
  scaling.}
\\\\
We now checked additionally. We checked for half-orbits and found the same scaling, nothing special
about closed orbits in this system, it seems.

Also, see our additional material in response to the second reviewer's suggestion to report on $L^2$ and
$L^\infty$ scaling.

See the end of Section~\ref{subsec:WignerKerrNumerics}
\\\\

%  \begin{itemize} 

 \vspace{1.5cm}
%  \item 
%\end{itemize} 
\hrule

\vspace{0.5cm}
 
\underline{The second reviewer thinks that the paper can be published but suggested a few
  changes.}
\\\\

\underline{The second reviewer states that ``The product symbol in equation (1) is ambiguous}
\\\\
We therefore now added on page `1', the sentence ``Here, all operator-products are meant to be from
left to right: $\prod_{i=1}^N \hU_i = \hU_1 \hU_2 \cdots \hU_N$.''
\\\\\\
\underline{The second reviewer refers to reference 12 and a typo.}
\\\\
We corrected this.
\\\\\\
\underline{The second reviewer mentions typos in Eqs. (21) and (22).}
\\\\
We corrected these.
\\\\\\
\underline{The second reviewer mentions typos regarding the use of $dt$ vs $\Delta t$.}
\\\\
We corrected these.
\\\\\\
\underline{The second reviewer suggested to investigate the difference $W(0)-W(\pi)$ in $L^2$ and
  $L^\infty$ norms.}
\\\\
We added this material at the end of sub-section~\ref{subsec:WignerKerrNumerics} and in the appendix.
\\\\\\
\underline{The second reviewer suggested to tighten our terminology on `universality' of Chin's approach.}
\\\\
We now state that Chin's approach allows for the treatment of non-separable polynomial
Hamiltonians.

\vspace{1cm}
\newpage
Let us thank both reviewers for their many thoughtful suggestions. 
 \vspace{1cm}

We look forward to hearing from you.
\\\\\\
Yours, {Maximilian \'Ciri\'c}, {Denys I. Bondar} and Ole Steuernagel
\newpage
\end{widetext}

%%% Local Variables:
%%% mode: latex
%%% TeX-master: t
%%% End:
