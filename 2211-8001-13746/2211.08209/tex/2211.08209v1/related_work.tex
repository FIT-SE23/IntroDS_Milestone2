\section{Related work}
%
\label{sec_related_work}
We provide an overview of related work that focus either on exponential family learning or on unit-level counterfactual inference with unobserved confounding. 
%
%
%
We refer the reader to \cite{imbens2015causal, hernan2020causal} for counterfactual inference with no unobserved confounding (as well as closely related concepts of ignorability in statistics and selection on the observables in economics). Likewise, we refer the reader to \cite{Pearl2009, Pearl2016} for counterfactual inference when the underlying causal mechanism (i.e., the directed acyclic graph) is known unlike our work.
%
%


\paragraph{Exponential family learning.}  There has been extensive work on learning parameters of a single exponential family distribution from multiple samples (see \cite{ShahSW2021B} for an overview). Of particular interest are the works that focus on learning sparse Markov random fields with (a) discrete variables \cite{VuffrayML2022} and (b) continuous variables \cite{ShahSW2021A} which inspire our loss function. Recently, a few works \cite{KandirosDDGD2021, DaganDDA2021} have focused on learning Ising model, i.e., 
%
sparse Markov random fields with binary variables, with one sample. However, these works focus on special cases where either the dependencies between the variables or a specific subset of the parameters are already known.
%
%
%
%
%
%

\paragraph{Unit-level counterfactual inference.} 
%
For unit-level inference with unobserved confounding, prior work has largely focused on latent factor models, where the interventions and potential outcomes are assumed to be independent conditional on latent factors. These include popular frameworks of difference-in-differences \citep{bertrand2004much, angrist2009mostly}, synthetic controls \citep{abadie1, abadie2}, synthetic difference-in-differences~\cite{arkhangelsky2019synthetic}, and synthetic interventions \citep{agarwal2021synthetic}.  A recent work  \cite{dwivedi2022counterfactual} provides a latent factor model based approach for counterfactual inference in sequential experiments where the treatment mechanism is designed and known, and there is no confounding by definition. Notably these works allow only for finitely many interventions, and need multiple units to be simultaneously treated with the same interventions for a period of time (for their estimation strategies to work). Another key difference is that these works directly learn the outcomes, and not the distributions like we do.
%
%
%
%

%
%


%
%
%
%

%




%

