\chapter{The standard cosmological model}
\label{chap:cosmology}
%\addcontentsline{toc}{chapter}{Introduction}

%Throughout human history, the majesty and mysteriousness of the Universe have attracted scientists and philosophers. This curiosity led to the birth of \emph{Cosmology}, the branch of science that studies the Universe as a whole and attempts to answer the most important questions about its origin, structure, evolution, and ultimate destiny.
%To pursue this (ambitious) goal, cosmologists gather empirical evidences and measurements to create theoretical models that can explain real-world observations. Advances in experimental precision or observations of new physical phenomena then lead to improvements in the theoretical model that provide a more realistic description of the Universe.
%%%%%%%%%%%To answer the most critical questions about the origin, structure, evolution, and ultimate destiny of the Universe, cosmologists gather empirical evidence and measurements to create theoretical models that explain real-world observations. Advances in experimental precision or observations of new physical phenomena lead to improvements in the theoretical model that provide a more realistic description of the Universe. The evidence used to formulate the current model of the Universe comes mainly from astronomical observations, namely the observation and analysis of signals emitted from distant sources. %The nature of the signals can vary, and they can reveal a range of information. For example, the electromagnetic signals from a star or galaxy can be analysed to draw conclusions about its physical properties (such as temperature, composition, rotational speed, and relative motion), the distance from us, as well as the characteristics of the space(time) through which the signal has passed. Another notable example is gravitational waves, which are direct evidence of the existence of black holes and provide a new window to peer into regions inaccessible with electromagnetic signals. Furthermore, observing the same type of astronomical object at different stages helps understand the processes that led to its formation and evolution. Astronomical observations not only provide information about the current state of the Universe but can also reveal its history. This is because, due to the finite speed of light, there is a delay between the moment of observation and the emission of the signal, and this delay increases with the distance between the observer and the source. Therefore, observations on cosmological scales reveal an earlier state of the Universe. Another consequence of the finite speed of light is a maximum distance beyond which nothing can be observed because the signals have not yet reached us: this physical limit is the \emph{particle horizon} and marks the edge of our observable Universe. 
%%%%%%%%%%%we observe the state of sources at the moment of emission of the signal, and  showing the state of the Universe at early times. 

The increasing amount of data collected by experimental probes like SDSS, Planck, LSST, SKA and others\footnote{{\color{blue}{https://www.sdss.org}}, {\color{blue}{https://www.cosmos.esa.int/web/planck}}, {\color{blue}{https://www.lsst.org}}, {\color{blue}{http://skatelescope.org/}}} have produced a generally coherent picture of our Universe.
The general purpose of these experiments is to investigate the three main pieces of evidence in cosmology:
%The three most important observations in cosmology are:
%
\begin{itemize}
\item the \emph{cosmic microwave background} (\setwd{CMB}{acr:CMB}): the CMB is the electromagnetic radiation that pervades the Universe as homogeneous and isotropic background noise. It was discovered by A. Penzias and R. Wilson in 1965 \cite{Penzias:1965}, when they observed a thermal black body spectrum with a temperature of $ 3.5 \, {\rm K}$ across the entire sky.
Successive measurements \cite{COBE:1992syq, WMAP:2008lyn, planck2018param, planck2019anl, planck2020CMB} have revealed further details in the CMB structure, showing small fluctuations of $\Delta T/ T \sim 10^{-5}$ around the average temperature of $2.7 \, {\rm K}$. Accurate mapping of the small anisotropies in the temperature of the CMB is of fundamental importance to cosmology, as it provides a clue to the structure of the Universe at very early times, see Fig.~\ref{fig:CMB}. %Indeed, the CMB anisotropies correspond to regions of slightly different densities at very early times.
 
\item the \emph{large-scale structure} (\setwd{LSS}{acr:LSS}): observations in different wavelength ranges of electromagnetic radiation have revealed a hierarchical organisation of astrophysical objects: massive objects tend to form gravitationally bound structures such as galaxies and galaxy clusters, which organise themselves on cosmological scales into superclusters, filaments, and voids, forming the so-called \emph{cosmic web} \cite{carroll2017book}. Increasingly better galaxy surveys, like \cite{york2000sloan, colless20012df, mccracken2003virmos, vogt2005deep, beasley2002vlba}, have measured the distance and shape of cosmological structures with great precision and produced an accurate three-dimensional map of the LSS of the Universe, see Fig.~\ref{fig:LSS}. Furthermore, since these cosmological structures observed today are the result of the evolution of the tiny CMB anisotropies under the influence of gravity, we expect to find features of the CMB radiation anisotropy in the observed LSS.
% produced a precise map of the LSS of the Universe allowing us to infer the evolution of their clustering over cosmic time. Moreover, since these cosmological structures observed today are the results of the evolution of the tiny CMB anisotropies under the effects of gravity, we expect to find characteristics of the anisotropy of the CMB radiation in the observed LSS. %An important example are the \emph{baryonic acoustic oscillations} (BAO) \MG{write BAO from $12.3.6$ ellis}

\item the \emph{type Ia supernov\ae} (\setwd{SnIa}{acr:SnIa}): the distance-redshift relation is one of the landmarks of modern cosmology. It is also known as ``Hubble law'' and was first derived by G. Lema\^{i}tre in 1927, \cite{lemaitre1927univers}, as a linear relation between the distance of a galaxy $D$ and its recession velocity $v$ ($v = c z$ for small redshifts $z < 1$), i.e. $z= \frac{H_{\rm 0}}{c} D$. The same relation was measured by E. Hubble in 1929, \cite{hubble1988relation}, finding a value of the constant of proportionality $H_{\rm 0}=500 \, {\rm km/s/Mpc}$. This is considered the first evidence for the expansion of the Universe\footnote{This relationship is universal and does not depend on the location of observation. So if we make the same measurement from another galaxy, we get the same relation with the same constant $H_{\rm 0}$. From this, we can conclude that the Universe is expanding isotropically at the expansion rate $H_{\rm 0}$.% taking another not only the farther the galaxies are the more rapidly they recede from us but also that this relation is valid galaxies recede from us in all directions.
}. An accurate measurement of $H_{\rm 0}$ requires a precise estimate of the redshift and the distance of the source: the redshift is obtained directly from spectroscopic analysis, while the distance is derived using indirect methods, such as geometrical relations and/or physical properties of astronomical candles. %To this second category belongs the \emph{standard candles}, namely astronomical objects that have known absolute luminosity $L$, so that their distance is obtained by measuring the luminosity flux at the observer $F$ and using the relation $D=\sqrt{\frac{L}{4 \pi F}}$ (see Eq.\eqref{eq:D_lum_def}). 
Precise measurements of the distance-redshift relation in cosmology are performed using SnIa standard candles\footnote{Standard candles are astronomical objects whose absolute luminosity $L$ is known, so their distance is determined by measuring the luminosity flux at the observer $F$ and using the relation $D=\sqrt{\frac{L}{4 \pi F}}$ (see Eq.\eqref{eq:D_lum_def}). SnIa are explosions of white dwarf stars characterised by a precise relation between the brightness and the timescale of the explosion.},  see Fig.~\ref{fig:SnIa}, which led not only to a better estimate of the present-day expansion rate $H_{\rm 0}=73.48\pm 1.66\, {\rm km/s/Mpc}$ \cite{Riess:2018uxu}, but also to the first confirmation of the accelerated expansion of the Universe, \cite{Riess:1998, Perlmutter_1999}.
\end{itemize}

This chapter describes the basics of modern cosmology and how the information obtained from the CMB, LSS, and SnIa observations are merged to produce a theoretical model of the Universe.
%%%%%%%%%%%%%%%%%%%%%%%%%%%%%%%%%%%%%%%%%%%%%%%%%%%%%%%%%%%%%%%%%%%%%%%%%%%%%%%%%%%%%%%%%%%%%%%%%%%%%%%FIGURE
\begin{figure*}[ht]
    \centering
    \begin{subfigure}{0.49\linewidth}%{0.8\columnwidth}%0.40\textwidth
        \includegraphics[width=\linewidth]{pict/Planck_CMB.jpg}
       \caption{Credits: Planck {\color{blue}{https://www.esa.int/}}}
        \label{fig:CMB}
    %\end{subfigure}
   % \begin{subfigure}{0.25\linewidth}%{0.8\columnwidth}
        \includegraphics[width=\linewidth]{pict/lss-dev_sdss_org}
        \caption{Credits: SDSS {\color{blue}{https://www.sdss.org/}}}
        \label{fig:LSS}
    \end{subfigure}
    \begin{subfigure}{0.5\linewidth}%{0.8\columnwidth}
        \includegraphics[width=\linewidth]{pict/riess_SnIa}
        \caption{Credits: Riess et.al. 1998, \cite{Riess:1998}. }
        \label{fig:SnIa}
    \end{subfigure}
    \caption{Precise measurements of the CMB (Fig.~\ref{fig:CMB}), the LSS (Fig.~\ref{fig:LSS}), and the SnIa (Fig.~\ref{fig:SnIa}) have been used to constraint the standard model for cosmology.}\label{fig:evidences}
\end{figure*}
%%%%%%%%%%%%%%%%%%%%%%%%%%%%%%%%%%%%%%%%%%%%%%%%%%%%%%%%%%%%%%%%%%%%%%%%%%%%%%%%%%%%%%%%%%%%%%%%%%%%%%%

\section{Cosmological models}
As noted above, observations on cosmological scales suggest that the gravitational attraction of the primordial anisotropies observed in the CMB shaped the Universe's large-scale structure. In general, gravity is the dominant interaction responsible for the formation of structures at all scales (from planets to the LSS), and it represents the fundamental mechanism underlying the formation and evolution of the Universe. Therefore, a model of the Universe must be consistent with the laws of gravity.

\subsection{General relativity}
From a theoretical point of view, the gravitational interaction at the macroscopic level\footnote{Gravity is the weakest of the four fundamental interactions of nature and has a negligible influence on the behaviour of subatomic particles. However, there are events in the cosmos that involve strong gravitational effects at the quantum scale that can only be described by a theory of quantum gravity.} is described by \emph{general relativity} (\setwd{GR}{acr:GR}), which was proposed by A. Einstein in 1915, \cite{Einstein:1915gr}. 
The primary distinction between general relativity and Newtonian gravity is Einstein's interpretation of gravity as a geometric property of spacetime. 
This interpretation was supported by the \emph{equivalence principle}, which states that gravitational acceleration is the same as the acceleration of an inertially moving body and by the fact that there can be no absolute concept of inertia, but only the inertia of masses relative to each other. 
In this perspective, Einstein assumes that free particles move along geodesics in a four-dimensional Riemannian manifold $\mathcal{M}$ whose points represent physical locations in space and time. 
Locally, we can specify a reference frame and label the point in $\mathcal{M}$ by a coordinate system $\{ x^{\mu} \}$. However, the coordinate system $\{ x^{\mu} \}$ is not uniquely defined since the laws of physics must be independent with respect to this choice.
To manifest this ``gauge invariance'', general relativity is formulated using the invariant structures of tensors.
Indeed, at each point $x$ of the manifold $\calM$, we can introduce a tangent space\footnote{Defined as the real vector space that intuitively contains all the possible directions in which one can tangentially pass through $x$.} $T_{x}\calM$ and the cotangent space $T^*_{x}\calM$, being the dual space to $T_{x}\calM$, and define the type $(r, s)$ tensor $\bm{T}$ as the multilinear map 
$$\bm{T}: \underbrace{T^*_{x}\calM \times \cdots \times T^*_{x}\calM}_\text{r} \times \underbrace{T_{x}\calM \times \cdots \times T_{x}\calM}_\text{s} \, \to\, \mathbb{R}\, .$$
Although this invariant coordinate formulation fits well the purposes of general relativity, for practical calculations it is better to express tensors by their components: a type $(r, s)$ tensor may be written as
\begin{equation}
\bm{T} =T\UD{\mu_1 \cdots \mu_r}{\nu_1 \cdots \nu_s} \,  \dfrac{\partial}{\partial x^{\mu_1}} \otimes \cdots \otimes \dfrac{\partial}{\partial x^{\mu_r}} \otimes d x^{\nu_1} \otimes \cdots \otimes d x^{\nu_s}\, 
\end{equation}
where $\dfrac{\partial}{\partial x^{\mu_i}}$ is a basis for the $i$-th tangent space, with $i=1, 2, \cdots r$, and $d x^{\nu_j} $ a basis for the $j$-th cotangent space, with $j=1, 2, \cdots s$. In other words, the type $(r, s)$ tensor $\bm{T}$ associates r vectors $\dfrac{\partial}{\partial x^{\mu_1}}, \cdots , \dfrac{\partial}{\partial x^{\mu_r}}$ and s covectors $d x^{\nu_1} , \cdots, d x^{\nu_s}$ to a scalar $T\UD{\mu_1 \cdots \mu_r}{\nu_1 \cdots \nu_s}$. It is indeed evident that the tensor components $T\UD{\mu_1 \cdots \mu_r}{\nu_1 \cdots \nu_s}$ depend on the choice of coordinates, although the tensor itself $\bm{T}$ is independent. This freedom in the choice of coordinate system can be used to simplify the computation of the tensor components.

The geometry of Riemannian spacetime is encoded in the metric tensor $g_{\mu \nu}$, which is the core object of general relativity. Indeed, the metric is used to define coordinate invariants, such as the squared line element $ds^2=g_{\mu \nu}dx^{\mu}dx^{\nu}$, which expresses the measure of the proper distance between two arbitrarily close events in spacetime $x^{\mu}$ and $x^{\mu}+ dx^{\mu}$.
Moreover, the metric tensor is used to introduce essential structures like the \emph{covariant derivative} (or connection) $\bm{\nabla}$, which is the covariant generalization of the partial derivative that allows to derive and transport tensors along the manifold. The action of the covariant derivative $\nabla_{\sigma}$ on a type $(r, s)$ tensor is expressed as
\begin{equation}
\nabla_{\sigma}T\UD{\mu_1 \cdots \mu_r}{\nu_1 \cdots \nu_s} =\partial_{\sigma} T\UD{\mu_1 \cdots \mu_r}{\nu_1 \cdots \nu_s} + \Sigma_{i}\Gamma^{\mu_{i}}_{\sigma \lambda} T\UD{\mu_1 \cdots \lambda \cdots \mu_r}{\nu_1 \cdots \nu_s} - \Sigma_{j}\Gamma^{\lambda}_{\sigma \nu_{j}} T\UD{\mu_1 \cdots \mu_r}{\nu_1 \cdots \lambda \cdots \nu_s}\, ,
\end{equation}
with $\partial_{\mu}=\dfrac{\partial}{\partial x^{\mu}}$, and $\bm{\Gamma}$ being the \emph{Christoffel symbols} expressing the difference between the covariant and the partial derivative\footnote{It worth mentioning that $\bm{\Gamma}$ are not tensor quantities since they do not transform as tensor. However, the difference $C\UD{\rho}{\mu \nu}=\Gamma^{\rho}_{\mu \nu}-\Gamma^{\rho}_{\nu \mu}$ defines a tensor expressing the torsion of the spacetime. In general relativity we assume that the spacetime is torsion-less, implying that $\Gamma^{\rho}_{\mu \nu}$ are symmetric in the lower indices.} in terms of metric tensor derivatives
\begin{equation}
\Gamma^{\rho}_{\mu \nu}=\dfrac{1}{2} g^{\rho \lambda}\left( \partial_{\mu}g_{\lambda \nu}+\partial_{\nu}g_{\mu \lambda}-\partial_{\lambda}g_{\mu \nu}\right)\, . \label{eq:christoffel_symb}
\end{equation}
One application of the covariant derivative is the \emph{directional derivative} $t^{\sigma}\nabla_{\sigma} v^{\mu}$, namely the derivative of a vector field $v^{\mu}$ in the direction of a given vector $t^{\mu}$. If $t^{\mu}$ is the tangent vector to the curve $C$ in the manifold and $t^{\sigma}\nabla_{\sigma} v^{\mu}=0$ along all points of the curve, we say that the vector $v^{\mu}$ is \emph{parallel transported along} $C$. In the special case where the tangent vector $t^{\mu}$ is parallel transported along itself $t^{\sigma}\nabla_{\sigma} t^{\mu}=0$, we call the curve $C$ a \emph{geodesic}, which in general relativity represents the trajectory of a free particle. In special relativity, the trajectories of free particles are straight lines, but in general relativity, the structure of spacetime is curved by the presence of matter, and this ``bends'' the particle's trajectories.

The equations which make it possible to derive the metric tensor associated with a given distribution of matter are the Einstein's field equations%. These equations are ten nonlinear partial differential equations which require properly specified boundary conditions for their solution. What ‘properly specified’ might mean has been the subject of a considerable literature over the past decades
%The metric itself contains the physical meaning of the coordinates utilized to generate that solution. %Except inasmuch as we may have attempted to choose coordinates that exhibited some specific symmetry, the coordinates have no a priori significance.
%the curvature of the spacetime is directly related to the  presence of massive objects via the Einstein field equations
\begin{equation}
G_{\mu \nu}\equiv R_{\mu \nu}-\dfrac{1}{2} R g_{\mu \nu}=\dfrac{8 \pi G}{c^4}T_{\mu \nu}
%G_{\mu \nu}=\dfrac{8 \pi G}{c^4}T_{\mu \nu}
\, , \label{eq:einsteinEq}
\end{equation} 
where $G_{\mu \nu}$ is the Einstein tensor representing the curvature of the spacetime, $T_{\mu \nu}$ is the stress-energy tensor representing the energy and momentum of matter and radiation, while $G$ and $c$ are the gravitational constant and the speed of light, respectively. %Mathematically, Eqs.~\eqref{eq:einsteinEq} are ten nonlinear partial differential equations whose solutions give the components of the metric tensor $g_{\mu \nu}$, namely the fundamental quantity in general relativity representing the specetime.
The two quantities $R_{\mu \nu} \equiv R\UD{\sigma}{\mu \sigma \nu}$ and $R \equiv g^{\mu \nu} R_{\mu \nu}$ are the Ricci tensor and the Ricci scalar, and they are directly defined from the Riemann tensor
\begin{equation}
R\UD{\rho}{\mu \sigma \nu}=\partial_{\sigma}\Gamma^{\rho}_{\mu \nu} -\partial_{\nu}\Gamma^{\rho}_{\mu \sigma}+\Gamma^{\rho}_{\lambda \sigma}\Gamma^{\lambda}_{\mu \nu}-\Gamma^{\rho}_{\lambda \nu}\Gamma^{\lambda}_{\mu \sigma}\, .
\end{equation} 


\subsection{The FLRW class of models}
A valid cosmological model must be based on general relativity, and its spacetime must satisfy the Einstein equations~\eqref{eq:einsteinEq}. Thus, it is a matter of finding the correct stress-energy tensor to describe the correct distribution of energy and momentum in the Universe. 
According to cosmological observations, the Universe seems to be filled with an isotropic and homogeneous mass distribution on large scales, which can be represented as a perfect fluid characterised by an average energy density $\bar{\rho}$ and isotropic pressure $p$. 
%we can ideally assume that the Universe is filled with an isotropic and homogeneous mass distribution on large scales, which is represented as a perfect fluid characterised by a mean mass density $\bar{\rho}_{\rm m}$ and an isotropic pressure $p$.
The expression for stress-energy tensor of this cosmic fluid is written as
\begin{equation}
T^{\mu \nu}=\left(\bar{\rho} + \dfrac{p}{c^2} \right)u^{\mu}u^{\nu} + p g^{\mu \nu}\, , \label{eq:T_fluid}
\end{equation}
with $u^{\mu}=g^{\mu \nu}u_{\nu}$ the four-velocity vector field of the fluid. In addition, the tensor $T_{\mu \nu}$ satisfies the following conservation rules
\begin{equation}
\nabla_{\mu}T^{\mu \nu} \equiv \partial_{\mu}T^{\mu \nu} +\Gamma^{\mu}_{\mu \lambda}T^{\lambda \nu} + \Gamma^{\nu}_{\mu \lambda}T^{\mu \lambda}=  0\, ,\label{eq:T_conservation}
\end{equation}
whose components are the continuity equation for the energy density $ \bar{\rho}$ and the Euler equation for the fluid. 

Under the same assumptions of isotropy and homogeneity, we may derive the following form of the metric\footnote{It is customary to express the metric components by specifying the line element.}
\begin{equation}
ds^2= - c^2 dt^2+ a(t)^2\left[ \dfrac{d r^2}{1-k r^2}+ r^2 d\theta^2+r^2 \sin(\theta)^2 d\phi^2\right]\, , \label{eq:FLRW_metric}
\end{equation} 
where $(t, r, \theta, \phi)$ are coordinates, $a(t)$ is the scale factor, and $k$ is a constant expressing the spatial curvature: usually the coordinates are rescaled such that $k$ is set to $-1$, $1$, or $0$ for space of constant negative, positive, or null spatial curvature, respectively.  
%The metric in Eq.~\eqref{eq:FLRW_metric} is the \emph{Friedmann-Lema\^{i}tre-Robertson-Walker} metric (FLRW) which represents a class of cosmological models for homogeneous and isotropic universes that are expanding (or contracting) as dictated by the scale factor $a(t)$.
%Hence, the dynamics of the FLRW models is encoded in the expression of the scale factor which is found from the Einstein field equation, Eq.~\eqref{eq:einsteinEq},  
The metric in Eq.~\eqref{eq:FLRW_metric} is the \emph{Friedmann-Lema\^{i}tre-Robertson-Walker} metric (\setwd{FLRW}{acr:FLRW}) and it represents a class of cosmological models with the dynamics determined by the scale factor $a(t)$.
The time coordinate $t$ is known as the \emph{cosmic time}, and it is the proper time measured in a comoving frame with the observer, i.e. the frame in which the observer's position remains unchanged $(r, \theta, \phi)=\const$. In this frame the four-velocity of a fluid element is simply $u^{\mu}=(1, 0, 0, 0)$ and the stress-energy conservation, Eq.~\eqref{eq:T_conservation}, for the FLRW metric simply reduces to 
\begin{equation}
\dfrac{d}{d t}(\bar{\rho} a^3)=-\dfrac{p}{c^2} \dfrac{d}{d t}(a^3)\, .\label{eq:eos_1}
\end{equation}
This formula expresses the mass-energy conservation, relating the change of the energy density in a element volume $\dfrac{d}{d t}(\bar{\rho} a^3)$ to the pressure acting on that volume $p \dfrac{d}{d t}(a^3)$. In cosmology Eq.~\eqref{eq:eos_1} is usually written as
\begin{equation}
\dfrac{d \log(\bar{\rho})}{d t}=-3 (1+w) \dfrac{d \log(a)}{d t}\, ,\label{eq:FLRW_eos}
\end{equation}
with $w=p/(c^2 \bar{\rho})$. If $w$ is time independent, Eq.~\eqref{eq:FLRW_eos} has solution 
\begin{equation}
\bar{\rho}= \bar{\rho}_0 \left( \dfrac{a_0}{a}\right)^{3 (1+w)}\, ,\label{eq:eos_sol}
\end{equation} 
with $\bar{\rho}(t_{\rm 0})=\bar{\rho}_0$ and $a(t_0)=a_0$ the mass-energy density and the scale factor at present time $t_{\rm 0}$, respectively. The quantity $w=p/(c^2 \bar{\rho})$ gives the \emph{equation of state} (\setwd{EOS}{acr:EOS}) of the cosmic fluid, where some notable examples are:
\begin{itemize}
\item $w=1/3$ for radiation, which gives $\bar{\rho}_{r}\propto a^{- 4}$;
\item $w=0$ for (pressureless) matter, which gives $\bar{\rho}_{\rm m}\propto a^{- 3}$;
\item $w=-1$ for vacuum (dark) energy, which gives $\bar{\rho}_{\Lambda}= \const$.
\end{itemize} 
From this we can conclude that the mass-energy density is the sum of different species $\bar{\rho}(t)=\Sigma_{\rm x} \bar{\rho}_{\rm x}(t) $ and each species scales as a different power of the scale factor according to its EOS $w_{\rm x}$. In cosmology, it is usually assumed that the only species contributing to the mass-energy density are baryonic matter $\bar{\rho}_{\rm b}$, dark matter $\bar{\rho}_{\rm m}$, radiation $\bar{\rho}_{\rm r}$, and dark energy $\bar{\rho}_{\rm \Lambda}=\frac{\Lambda c^2}{8 \pi G}$, i.e. $\bar{\rho}=\bar{\rho}_{\rm b}+\bar{\rho}_{\rm m}+\bar{\rho}_{\rm r}+\bar{\rho}_{\rm \Lambda}$. Similarly, the species contributing to the cosmic pressure are the radiation pressure $p_{\rm r}$ and the dark energy pressure $p_{\rm \Lambda}=-\frac{\Lambda c^4}{8 \pi G}$, since the  (baryonic and dark) matter have vanishing\footnote{The baryonic and dark particles are non-relativistic and therefore their energy density is much larger than their pressure.} pressure $w=0$. More often we refer to ``dust'', or simply matter, to indicate both baryonic and dark matter components as the part of the perfect fluid that has positive mass density and vanishing pressure.
By defining the \emph{Hubble parameter} (also known as expansion rate) $H=\dfrac{1}{a}\dfrac{d a}{d t}$ and the \emph{critical density} as $\rho_{\rm c}(t)=\frac{3 H(t)^2}{8 \pi G}$, we express the mass-energy density by the dimensionless \emph{density parameter} $\Omega(t)=\frac{\bar{\rho}(t)}{\rho_{\rm c}(t)}$. This is particularly convenient in cosmology, since using Eq.~\eqref{eq:eos_sol} one can write the density parameter of each species $\Omega_{\rm x}(t)$ in terms of its value at the present time, $\Omega_{\rm x}(t_0)=\Omega_{\rm x_0}$, i.e.
\begin{equation}
\Omega_{\rm x}=\dfrac{8 \pi G}{3 H_0^2}\bar{\rho}_{\rm x}= \Omega_{\rm x_0} \left(\dfrac{a_0}{a}\right)^{3 (1+w_{\rm x})}\, ,\label{eq:Omega_Omegat0}
\end{equation}
and thus clearly separate the contributions of the different species at present time in the density parameter $\Omega(t)$ as
\begin{equation}
\Omega(t)= (\Omega_{\rm b_0}+\Omega_{\rm m_0})\left(\dfrac{a_0}{a(t)}\right)^{3} +\Omega_{\rm r_0} \left(\dfrac{a_0}{a(t)}\right)^{4} +\Omega_{\rm \Lambda_0}\, . \label{eq:Omega(t)}
\end{equation}

The Einstein field equations in Eq.~\eqref{eq:einsteinEq} for the FLRW metric give the \emph{Friedmann equations} 
\begin{align}
\dfrac{1}{a^2}\left(\dfrac{d a}{d t}\right)^2 & = \dfrac{8 \pi G}{3} \bar{\rho}- \dfrac{k c^2}{a^2} \label{eq:friedmann_1}\\
\dfrac{1}{a}\dfrac{d^2 a}{d t^2} & = -\dfrac{4 \pi G}{3} \left(\bar{\rho}+\dfrac{3 p}{c^2}\right)\, . \label{eq:friedmann_2}
\end{align}
%By defining the \emph{Hubble parameter} (also known as expansion rate) $H=\dfrac{1}{a}\dfrac{d a}{d t}$ and the \emph{critical density} as $\rho_{\rm c}(t)=\frac{3 H(t)^2}{8 \pi G}$, we express the mass-energy density via the dimension-less \emph{density parameter} $\Omega(t)=\frac{\bar{\rho}(t)}{\rho_{\rm c}(t)}$
The first Friedmann equation, Eq.~\eqref{eq:friedmann_1}, expressed in terms of the density parameter Eq.~\eqref{eq:Omega(t)} reads
\begin{equation}
H(t)^2 = H_0^2 \left[ \Omega_{\rm r_0} \left(\dfrac{a_0}{a(t)}\right)^{4} + (\Omega_{\rm b_0}+\Omega_{\rm m_0}) \left(\dfrac{a_0}{a(t)}\right)^{3} + \Omega_{\rm k_0} \left(\dfrac{a_0}{a(t)}\right)^{2}+ \Omega_{\rm \Lambda_0} \right] \, , \label{eq:Friedmann_edH}
\end{equation}
with $\Omega_{\rm k_0}= - \dfrac{k c^2}{H_0^2 a_0^2}$. %, and $H_0$ the value of the Hubble parameter at present time, also known as the Hubble constant. 
%%%%%%%%%%%%%%%%
%Finally, the Friedmann equation Eq.~\eqref{eq:friedmann_1} in terms of the various density parameters reads
%\begin{equation}
%H(t)^2 = H_0^2 \left[ \Omega_{\rm r_0} \left(\dfrac{a_0}{a(t)}\right)^{4} + (\Omega_{\rm b_0}+\Omega_{\rm m_0}) \left(\dfrac{a_0}{a(t)}\right)^{3} + \Omega_{\rm k_0} \left(\dfrac{a_0}{a(t)}\right)^{2}+ \Omega_{\rm \Lambda} \right] \, . \label{eq:Friedmann_edH}
%\end{equation}
Similarly, the second Friedmann equation, Eq.~\eqref{eq:friedmann_2}, in terms of density parameters reads
\begin{equation}
q\equiv -\dfrac{\dfrac{d H}{dt}+H^2}{H^2}=\dfrac{1}{2}\sum_{\rm x}  (1+3 w_{\rm x}) \Omega_{\rm x} \, , \label{eq:Friedmann_edQ}
\end{equation}
where $q$ is the \emph{deceleration parameter}, and we remind that $\Omega_{\rm x}(t)=\frac{8 \pi G}{3 H(t)^2}\bar{\rho}_{\rm x}(t)$ and $w_{\rm x}$ are the density parameter and the EOS parameter for the ${\rm x}^{th}$ specie at time $t$, respectively.

The peculiarity of the FLRW models is that they predict the beginning of the Universe from a singularity point at $t=0$, i.e. $a(0)=0$: this is known as \emph{Big Bang} and corresponds to the origin of the Universe ($13.8$ billion years, according to the standard cosmological model, \cite{ellis2012relativistic, planck2018param}) from extreme conditions of density, pressure, and temperature. From this extreme initial state, the Universe began its adiabatic expansion, becoming less dense and colder, allowing the formation of all elementary particles and gradually, electrons, photons, and baryons. One of the strongest evidence in favour of the Big Bang theory is the CMB, corresponding to the radiation relict formed $379000$ years after the Big Bang in the \emph{recombination epoch}, \cite{ellis2012relativistic}. Before that time, the Universe was indeed a plasma of electrons, protons, and nuclei, in which the photons were constantly scattered. Due to cosmic expansion, conditions in the era of recombination became favourable for the formation of atoms (mainly hydrogen and helium) until the photons decoupled from matter and began to move freely through the expanding Universe: this is observed today as the CMB thermal radiation. Thus, on the one hand, the CMB is an excellent source of information, as it provides a ``snapshot'' of the Universe as it was $379000$ years after the Big Bang, but on the other hand, it also constitutes a limit to cosmological observations with electromagnetic radiation. 

Some notable examples from the class of FLRW models are the \emph{de Sitter} and \emph{Einstein-de Sitter} (\setwd{EdS}{acr:EdS}) models, each representing a (spatially flat) Universe containing only dark energy and only dark matter, respectively. The de Sitter model represents an empty Universe (without matter) containing only dark energy, which determines the expansion rate $H \propto \sqrt{\Lambda}$. It is characterised by an exponentially growing scale factor $a(t)=e^{H t}$, which causes an accelerated expansion of the de Sitter Universe. Since no other mechanism opposes the accelerated expansion, at a certain point, any observer in a de Sitter Universe will start experiencing event horizons, beyond which it is impossible to see or perceive anything.
In contrast, the Einstein-de Sitter model represents a Universe containing only dust (pressureless matter) with a density of $\bar{\rho}_{\rm m}\propto H^2$. In this model, the distance between two comoving observers increases with $t^{2/3}$, but this expansion is balanced by gravitational attraction so that it tends asymptotically to zero as time approaches infinity.
Although these models do not explain current observations, they can be considered reasonable approximations for past epochs of the Universe, \cite{ellis2012relativistic}. Using Friedmann equation Eq.~\eqref{eq:Friedmann_edH}, we can distinguish the following epochs: at very early times, i.e. $a(t)$ small, the Universe was dominated by radiation, since $H^2 \sim H_0^2 \Omega_{\rm r_0} a^{-4}$. After that, followed a matter-dominated era (for a spatially flat Universe), with $H^2 \sim H_0^2 \Omega_{\rm m_0} a^{-3}$, which lasted until dark energy took over, leading to an accelerated expansion of the Universe, $q=\Omega_{\rm m_0}/2 -\Omega_{\rm \Lambda_0} <0$.

\section{The $\Lambda$CDM model}

The Friedmann equations Eqs.~\eqref{eq:Friedmann_edH}-\eqref{eq:Friedmann_edQ}, together with the EOS and Eq.~\eqref{eq:FLRW_eos}, completely define the dynamics and composition of the cosmological model. Therefore, to build a complete picture of the cosmological model that best fits our Universe, one must measure the values of the cosmological parameters $H_0$, $\Omega_{\rm r_0}$, $\Omega_{\rm b_0}$, $\Omega_{\rm m_0}$, $\Omega_{\rm k_0}$, and $\Omega_{\rm \Lambda_0}$ from the observations, and determine the dynamics by solving the Friedmann equations. 
The first important constraint on the total energy-density of the Universe is provided by the Friedmann equation Eq.~\eqref{eq:Friedmann_edH} evaluated at present time, i.e.
\begin{equation}
 \Omega_{\rm r_0} + \Omega_{\rm b_0} + \Omega_{\rm m_0} + \Omega_{\rm k_0} + \Omega_{\rm \Lambda_0} = 1 \, . \label{eq:components}
\end{equation}
Several cosmological parameters can be measured within the same observation, and the specific observation can constrain each parameter differently: their values are obtained by best-fitting the different measurements of the cosmological parameters by the CMB, SnIa, and LSS. The latest results, published in $2018$ by the Planck Collaboration \cite{planck2018param}, depict a cosmological model consistent with a spatially flat Universe $\Omega_{\rm k_0}= 0.001\, \pm 0.002$, and dominated by the dark sector, i.e. dark matter $\Omega_{\rm m_0}=0.315 \, \pm 0.007$ and dark energy $\Omega_{\rm \Lambda}=0.685\, \pm 0.0073$. The baryonic matter represents only few percent of all the energy content of the Universe, i.e. $\Omega_{\rm b_0}=0.022\, \pm 0.0001$, while the radiation component (intended as photons and massless neutrinos) have a negligible effect at present time $\Omega_{\rm r_0}\sim 10^{-5}$. The value of the Hubble constant\footnote{This value of $H_{\rm 0}$ is in $3.7\, \sigma$ tension with the local measurement of $H_{\rm 0}$ from SnIa, $H_{\rm 0}=73.48\, \pm 1.66\, {\rm km/s/Mpc}$ \cite{Riess:2018uxu}.}
 is $H_{\rm 0}=67.4\, \pm 0.5\, {\rm km/s/Mpc}$.
%They compared the measurements from CMB data, comparing it to not just multiple types of CMB power spectra, but also independent cosmological sensors like as BAOs, SN1a, and galaxy clusters.
This parametrization defines the standard cosmological model, also dubbed as the \setwd{$\Lambda$CDM}{acr:LCDM} model.

The dynamics of the $\Lambda$CDM model is prescribed by the Friedmann equations Eqs.~\eqref{eq:Friedmann_edH}-\eqref{eq:Friedmann_edQ}, with $ \Omega_{\rm m_0} + \Omega_{\rm \Lambda} = 1$ (here $\Omega_{\rm m_0}$ considers both baryonic and dark matter), and completely encoded in the scale factor. 
Before proceeding to find the expression for the scale factor, let us make some considerations about the coordinates. The FLRW line element, Eq.~\eqref{eq:FLRW_metric}, for the $\Lambda$CDM model simplifies to 
\begin{equation}
ds^2= -c^2 dt^2 + a(t)^2 \delta_{i j} dq^idq^j \, ,
\end{equation}
where $q^i$ are the spatial coordinates in flat space. As for time, we prefer to use the \emph{conformal time} $\eta$ coordinate, which is related to the cosmic time $t$ as $d t = a(\eta) d \eta$. The advantage is that in conformal time the metric further reduces to
\begin{equation}
ds^2=a(\eta)^2\left(-c^2 d\eta^2+ \delta_{i j} dq^idq^j \right)\, ,
\label{eq:conf_ds}
\end{equation} 
simplifying also the calculations. Moreover, the conformal time has the clear physical meaning of particle horizon $c\, \eta$, i.e. the maximum distance ideally travelled by a photon since the beginning of the Universe, \cite{dodelson2003modern, ellis2012relativistic}.
In conformal time $d t = a(\eta) d \eta$, the Hubble parameter transforms as
\begin{equation}
%H(t)=\dfrac{1}{a}\dfrac{d a}{dt}=\dfrac{1}{a}\dfrac{d a}{d\eta}\dfrac{d \eta}{dt}= \dfrac{1}{a}\dfrac{\dfrac{d a}{d\eta}}{a}=\dfrac{\mathcal{H}(\eta)}{a(\eta)}
H(t)=\dfrac{\mathcal{H}(\eta)}{a(\eta)}\, , \label{eq:H_to confH}
\end{equation}
where we have defined the conformal Hubble parameter $\mathcal{H}(\eta)=\frac{1}{a(\eta)}\frac{d a(\eta)}{d\eta}$.
The Friedmann equations~\eqref{eq:Friedmann_edH}-\eqref{eq:Friedmann_edQ} for $\Lambda$CDM in conformal time reads
\begin{align}
\mathcal{H}^2 & = \mathcal{H}_0^2 \left[ \dfrac{\Omega_{\rm m_0} }{a} + \Omega_{\rm \Lambda_0} a^2 \right] \label{eq:LCDM_friedmann_H}\\
\dot{\mathcal{H}} & = \mathcal{H}^2-\dfrac{3}{2} \dfrac{\mathcal{H}_0^2 \Omega_{\rm m_0}}{a}\, , \label{eq:LCDM_friedmann_q}
\end{align}
where dotted quantities indicates derivative with respect to conformal time, i.e. $\dot{\mathcal{H}}=\frac{d \mathcal{H}}{d \eta}$, and we have used the standard convention of setting the scale factor today to unit $a_0=1$. The expression for the density parameters in conformal time \cite{Villa:2015ppa} is 
\begin{align}
\Omega_{\rm m} & =\dfrac{8 \pi G a^2}{3 \mathcal{H}^2}\bar{\rho}_{\rm m}=\dfrac{\mathcal{H}^2_0 \Omega_{\rm m_0}}{a \mathcal{H}^2} \\
\Omega_{\rm \Lambda} & =\dfrac{a^2 c^2 \Lambda}{3 \mathcal{H}^2}=\dfrac{a^2 \mathcal{H}^2_0 \Omega_{\rm \Lambda_0}}{\mathcal{H}^2}\, .
\end{align}
The scale factor for the $\Lambda$CDM model can be explicitly found (using the results in \cite{gradshteyn2014table}) by solving Eq.~\eqref{eq:LCDM_friedmann_H}
\begin{equation}
a(\eta)=\frac{\sqrt[3]{\frac{\Omega_{\rm m_0}}{\Omega_{\rm \Lambda}}} \Big(1-{\rm cn}\left(\mathit{y} |\mathit{r} \right)\Big)}{(\sqrt{3}-1)+(\sqrt{3}+1) {\rm cn}\left(\mathit{y} |\mathit{r} \right)}\, ,\label{eq:a_LCDM}
\end{equation}
where ${\rm cn}(\mathit{y}|\mathit{r})$ is the Jacobi elliptic cosine function, with $\mathit{y}=\left(\sqrt[4]{3} \sqrt[6]{\Omega_{\rm \Lambda}} \sqrt[3]{\Omega_{\rm m_0}}\right) \mathcal{H}_0 \eta$, and $\mathit{r}=\sqrt{\frac{\sqrt{3}+2}{4}}$.

%\section{Structure formation}
The $\Lambda$CDM model presented so far is based on the fundamental assumptions of homogeneity and isotropy of the Universe. However, as has been noted several times, these properties are only satisfied on average and on very large scales, with the transition from clustered structures to a homogeneous distribution beginning on scales\footnote{This scale is also known as \emph{End of Greatness}.} of $\sim 100\, {\rm Mpc}$, \cite{Yadav:2010cc, Scrimgeour:2012wt, Sam:2020}. 
%The hierarchical organization of cosmological structures on smaller scales exhibit a distinct transition from an inhomogeneous, clustered distribution to a homogeneous distribution on larger scales.
%Therefore, we need to include in the cosmological model how the observable Universe's structure evolved from initial density fluctuations. 
%Therefore, the $\Lambda$CDM model needs to be extended to describe these deviations from homogeneity ... to describe how the observable Universe's structure evolved from initial density fluctuations.
%Therefore, while the Friedmann equations Eqs.~\eqref{eq:LCDM_friedmann_H}-\eqref{eq:LCDM_friedmann_q} describe the overall dynamics of the Universe, we still need to account for the development of structures like as galaxies and galaxy clusters, which are evident on smaller scales.
%Therefore, the $\Lambda$CDM model needs to be extended to describe the origin of these deviations from homogeneity and isotropy, and how they evolved into the LSS observed today.
So, while the $\Lambda$CDM model describes the overall dynamics of the Universe, we still need to account for the evolution of structures such as galaxies and galaxy clusters that are visible on smaller scales.

\subsection{Early fluctuations and cosmological perturbation theory}
%early times, linear PT
%\MG{DON'T LIKE MUCH: Although the first $3.7 \times 10^{5}$ years are not directly accessible via (electromagnetic) observations, the state of the Universe at the end of the recombination era is well known thanks to the detailed map of the CMB, \cite{COBE:1992syq, WMAP:2008lyn, planck2020CMB}.  %The CMB map's structures have been thoroughly examined.
The features of the CMB map have been thoroughly examined, revealing that in the first moments after the Big Bang, small temperature variations were generated by quantum fluctuations on microscopic scales generating the seeds for galaxies and clusters, \cite{COBE:1992syq, WMAP:2008lyn, planck2020CMB}. Thanks to the inflationary paradigm\footnote{\emph{Inflation} is an epoch of accelerated expansion in the early Universe, and it was introduced to explain the coherence of CMB anisotropies on angular scales larger than the apparent cosmological horizon at recombination.}, \cite{tsujikawa2003introductory}, the early evolution of perturbations is well described on cosmological scales by relativistic \emph{perturbation theory} (\setwd{PT}{acr:PT}), \cite{ kodama1984cosmological}. Indeed, the small primordial fluctuations can be conceived as tiny perturbations $\delta \rho$ over a homogeneous and isotropic density distribution $\bar{\rho}$, so that the real mass-energy density is expanded as $\rho = \bar{\rho}+\delta \rho$ up to linear order. Similarly, the real spacetime is modelled by a perturbed metric $g_{\mu \nu}=\bar{g}_{\mu \nu}+\delta g_{\mu \nu}$, where $\bar{g}_{\mu \nu}$ is the background metric and $\delta g_{\mu \nu}$ is the small linear perturbation. 
It is important to note that $\delta g_{\mu \nu}$ is not uniquely defined due to coordinate gauge freedom: the same physical perturbation can be described by a different tensor perturbation
\begin{equation}
\widetilde{\delta g_{\mu \nu}}=\delta g_{\mu \nu} + \mathcal{L}_{\xi} \bar{g}_{\mu \nu}\, , \label{eq:PT_gauge_trans}
\end{equation}
see e.g. \cite{wald2010general}.
The term $\mathcal{L}_{\xi} \bar{g}_{\mu \nu}\equiv \xi^{\sigma}\partial_{\sigma} \bar{g}_{\mu \nu} + \bar{g}_{\mu \nu} \partial_{\mu} \xi^{\sigma} + \bar{g}_{\mu \nu} \partial_{\nu} \xi^{\sigma}$ is the Lie derivative and it represents the action on $\bar{g}_{\mu \nu}$ of an ``infinitesimal diffeomorphism'' generated by the vector field $\xi^{\mu}$ (see e.g. App. C in \cite{wald2010general}). In other words, the form of the perturbed metric $g_{\mu \nu}$ depends\footnote{For a different approach see \cite{bardeen1980gauge, kodama1984cosmological}.} on the specific choice of the gauge, and the first-order transformation between two different gauge choices is given by Eq.~\eqref{eq:PT_gauge_trans} (see \cite{Villa:2015ppa} for gauge transformations of 3 different gauges up to second-order PT).

Considering linear perturbations (\setwd{$\rm Lin$}{acr:Lin}) over the flat FLRW background Eq.~\eqref{eq:conf_ds}, the most general form of the spacetime metric is \cite{Villa:2015ppa}
\begin{align}
g_{0 0}&=-a^2(1+2 \Psi) \nonumber \\
g_{0 i}&=a^2(\partial_i B +\omega_i) \label{eq:metric_PT}\\
g_{i j}&=a^2\{(1-2 \Phi)\delta_{i j}+2 D_{i j} E +\partial_{(i}F_{j)}+\chi_{i j}\}\, , \nonumber 
\end{align}
where $\Psi, \, \Phi,\, B,\, E$ are the scalar modes, $\omega_i,\, F_i$ are the transverse vector modes ($\partial^i \omega_i=\partial^i F_i=0$), and $\chi_{i j}$ is the transverse and tracefree tensor mode ($\partial^i \chi_{i j}=\chi\UD{i}{i}=0$). The operator $D_{i j}=\partial_i \partial_j - 1/3 \delta_{i j} \nabla^2$ is the traceless symmetric double gradient operator, see e.g. \cite{Bertschinger:1993xt}.
The expressions in Eq.~\eqref{eq:metric_PT} define the \emph{scalar-vector-tensor decomposition} of the metric tensor \cite{kodama1984cosmological, bardeen1980gauge}. %In this form, the first-order perturbation modes are decoupled\footnote{However, this property is not satisfied at higher-orders, where the modes are coupled, see e.g. \cite{Bertschinger:1993xt, Villa:2015ppa}.} in the first-order expansion of the Einstein equations. 
Usually, in the study of structure formation, the linear vector and tensor modes can be neglected, i.e. $\omega_i \sim F_i \sim \chi_{i j}\sim 0$ \cite{matarrese1998relativistic, Bartolo:2005kv}, simplifying the expression of Einstein equations. The reason is that linear vector modes are decaying in time and linear tensor modes, i.e. primordial gravitational waves, are decoupled from the other perturbation modes. Each gauge corresponds to a specific choice of $\Psi, \, \Phi,\, B,\, E$:
\begin{itemize}
	\item the \emph{Lagrangian frame} (or \emph{synchronous-comoving gauge}) is defined by choosing $B=\Psi=0$ and corresponds, in analogy with fluid dynamics, to the reference frame associated with the coordinates comoving with the cosmic flow. In this reference frame, the positions of the fluid particles do not evolve in time, see Fig.~\ref{fig:E_L_frames}.
%	\begin{equation}
%	ds^2=a^2[-d\eta^2 +\gamma_{i j} d q^i dq^j] \, . \label{eq:synchronous-comoving}
%	\end{equation}
	\item \emph{Eulerian gauges} are all gauges identified by the (spatial) choice $E=0$. They correspond to a frame associated with the observer measuring the matter stream, i.e. not comoving with the cosmic flow. The expression of $B$ fully specifies the gauge. For example, a common choice in PT is the \emph{Poisson gauge} identified by $B=0$,  \cite{Bertschinger:1993xt}. With this coordinate choice, the positions of the particle of the fluid evolve from their initial positions, see again Fig.~\ref{fig:E_L_frames}.  %so called since the time-time component of Einstein equations reduces to the Poisson equation, relating the matter density and the gravitational potential, \cite{Bertschinger:1993xt}. 
%	\begin{equation}
%	ds^2=a(\eta)^2\left\{-(1+2 \psi)d\eta^2 + 2 B_i d x^i d \eta + [(1- 2\phi)\delta_{i j}+2 \gamma_{i j}] dx^idx^j \right\} \, , \label{eq:Eulerian}
%	\end{equation}
%	with $\partial^{i} B_i=0,$ and $\partial^{i}\gamma_{i j}=\gamma\UD{i}{i}=0$ depending on the gauge choice. 
\end{itemize}
%A common choice in cosmological perturbation theory is to use the \emph{Poisson gauge}, so called since the time-time component of Einstein equations reduces to the Poisson equation, relating the matter density and the gravitational potential, \cite{Bertschinger:1993xt}. 
%%%%%%%%%%%%%%%%%%%%%%%%%%%%%%%%%%%%%%%%%%%%%%%%%%%%%%%%%%%%%%%%%%%%%%%%%%%%%%%%%%%%%%%%%%%%%%%%%%%%%%%FIGURE
\begin{figure}[ht]
    \centering
    \includegraphics[width=0.7\linewidth]{pict/frame2.pdf}
    \caption{Illustration of the evolution of a group of particles in an expanding Universe in the Eulerian (left) and Lagrangian (right) frames. At the initial time $t$, the particles have a certain position with respect to the uniform grid. Due to the expansion and the gravitational interaction, at a later time $t+\Delta t$ the particles are in a different position with respect to the grid in the Eulerian frame (left), with the new position being marked by the vector $x^{\mu}$. In the Lagrangian frame (right), on the other hand, the positions of the particles $q^{\mu}$ do not change with time.}\label{fig:E_L_frames}
\end{figure}
%%%%%%%%%%%%%%%%%%%%%%%%%%%%%%%%%%%%%%%%%%%%%%%%%%%%%%%%%%%%%%%%%%%%%%%%%%%%%%%%%%%%%%%%%%%%%%%%%%%%%%%

The form of the first-order line element in the Poisson gauge considering scalar perturbations only is \cite{Villa:2015ppa}
\begin{equation}
ds^2=a(\eta)^2\left[-\left(1+2 \dfrac{\Psi(\eta, q^i)}{c^2}\right) c^2d\eta^2 + \left(1- 2\dfrac{\Phi(\eta, q^i)}{c^2}\right)\delta_{i j} dq^idq^j \right]\, ,
\end{equation}
where the perturbations $\Psi(\eta, q^i)$ and $\Phi(\eta, q^i)$ are obtained by solving\footnote{The equations are solved order by order, with the zeroth-order resembling the Friedmann equations Eqs.~\eqref{eq:LCDM_friedmann_H}-\eqref{eq:LCDM_friedmann_q}.} the Einstein equations, Eq.~\eqref{eq:einsteinEq}, expanded up to linear order, see \cite{Bertschinger:1993xt}. 
% QUI The perturbations are ruled by the Einstein equations, Eq.~\eqref{eq:einsteinEq}, expanded up to linear order in $\phi$. The equations are solved order by order, with the zero-th order resembling the Friedmann equations Eqs.~\eqref{eq:LCDM_friedmann_H}-\eqref{eq:LCDM_friedmann_q}. 
Let us start by noting that $\Psi=\Phi=\phi$, as it follows from the trace-free part of the $(i, j)$ components.
At first order, the $(0,0)$ component returns the Poisson equation
\begin{equation}
\nabla^2 \phi(\eta, q^i)-\dfrac{3}{2}\dfrac{\stuff}{a}\delta_{\rm Lin}(\eta, q^i)=0\, , \label{eq:Poisson_eq}
\end{equation}
with $\delta_{\rm Lin}(\eta, q^i)$ the \emph{linear Newtonian density contrast} defined as $\delta_{\rm Lin}=\dfrac{\delta \rho}{\bar{\rho}}+ \calO(\delta \rho^2)$, and $\nabla^2=\partial^i \partial_i$. %\rho-\bar{\rho}
From the other components of Einstein equations, we obtain that the scalar perturbation decomposes\footnote{Actually, the scalar perturbation is composed by growing and decaying modes like $\phi(\eta, q^i)=g_{+}(\eta) \phi_{+}(q^i)+g_{-}(\eta) \phi_{\rm -}(q^i)$. However the decaying modes are quickly suppressed leaving only with growing modes.} into the \emph{present time gravitational potential} $\phi_{\rm 0}(q^i)$ and a time-dependent part containing the \emph{growth factor} $\mathcal{D}(\eta)$ 
\begin{equation}
\phi(\eta, q^i)=\dfrac{\mathcal{D}(\eta)}{a(\eta)} \phi_{\rm 0}(q^i)\, .
\end{equation}
The growth factor $\mathcal{D}(\eta)$ is the growing mode solution for the linear density contrast, i.e. $\delta_{\rm Lin}(\eta, q^i)=\mathcal{D}(\eta) \delta_{\rm Lin}(\eta_0, q^i)$, and it is found by solving the evolution equation for the first-order density contrast \cite{peebles1993principles}
\begin{equation} \label{eqforF}
\ddot{\delta}_{\rm Lin} +\mathcal{H}\dot{\delta}_{\rm Lin}-\frac{3}{2}\mathcal{H}_0^2\Omega_{m_0}\frac{\delta_{\rm Lin}}{a}=0\,.
\end{equation}
The analytical solution for $\mathcal{D}$ is given in \cite{Villa:2015ppa},
\begin{equation}
\mathcal{D} (\eta) = \frac{a}{\frac{5}{2}\Omega_{\rm{m 0}} } \sqrt{1+\frac{\Omega_{\rm{\Lambda}}}{\Omega_{\rm{m 0}}}a^3}\,  {}_2 F_{1} \left( \frac{3}{2}, \frac{5}{6}, \frac{11}{6}, -\frac{\Omega_{\rm{\Lambda 0}}}{\Omega_{\rm{m 0}}}a^3 \right)\,,
\label{eq:grow_mode}
\end{equation}
with ${}_2 F_{1} \left(a,b,c, y\right)$ being the Gaussian (or ordinary) hypergeometric function.
%\begin{equation}
%\ddot{\delta}+\mathcal{H}\dot{\delta}-\dfrac{3}{2}\dfrac{\stuff}{a}\delta=0\, ,
%\end{equation}
%with $\delta(\eta, x^i)=\mathcal{D}(\eta)\delta(\eta_0, x^i)$.
In conclusion at early times %the specetime metric is expressed as
%\begin{equation}
%ds^2=a(\eta)^2\left[-(1+2 \dfrac{\mathcal{D}}{a} \phi_{\rm 0})d\eta^2 + (1- 2 \dfrac{\mathcal{D}}{a} \phi_{\rm 0})\delta_{i j} dx^idx^j \right]\, ,
%\end{equation} 
%and 
the small inhomogeneities are described by the density contrast 
\begin{equation}
\delta_{\rm Lin} = \dfrac{2}{3 \stuff}\left( \mathcal{D} \nabla^2 \phi_0 - 3 \mathcal{H}\dot{\mathcal{D}}\phi_0 \right)\, .
\end{equation}
%with the second term in parenthesis being the relativistic corrections to the Newtonian density contrast (first term), see \cite{Villa:2015ppa}.

%The evolution of the inhomogeneities is well described by linear PT only when $\delta \ll 1$. Initially this is the case, but at later times the density fluctuation increases under the influence of gravity, leading to a nonlinear phase of the structure formation. This nonlinear growth of structure is observed on small scale and it is usually described using Newtonian dynamics.
The evolution of inhomogeneities can be modelled by linear PT only if $\delta \ll 1$. Initially, this is the case, but at later times the density fluctuations become larger under the influence of gravity, reaching values of $\delta \sim 10^2$ for filaments and $\delta \sim 10^6$ for galaxies. %This increase of the density fluctuations marks the transition to nonlinear dynamics for structure formation, which is usually treated by means of Newtonian gravity. % as it is motivated by the fact that the peculiar gravitational potential $\phi_0$ remains small (from galaxies to cosmic scales) despite that $\delta \gg 1$, with an amplitude of $\phi_0 /c^2 \sim 10^{-5}$.
The evolution of the gravitational instability that led from the early linear perturbations to the present-day inhomogeneities is the primary goal of the study of \emph{structure formation}.

 
\subsection{Analytical approaches to structure formation}

The equations of GR control gravitational instability, but some applications are well described by the \emph{Newtonian approximation}, namely by a weak-field and slow-motion limit of GR. These requirements are indeed satisfied on small scales, where the dimensionless peculiar gravitational potential $\phi_g/c^2$ remains small ( $\phi_g /c^2 \sim 10^{-5}$), and the peculiar velocity is never relativistic. In particular, for a fluctuation of proper scale $L$, the dimensionless peculiar gravitational potential is
\begin{equation}
\dfrac{\phi_g}{c^2} \sim \delta \left(\dfrac{L}{r_{\rm H}}\right)^2\, 
\end{equation}
with $r_{\rm H}= c H^{-1}$ the Hubble radius, implying that $\phi_g/c^2$ remains small even if $\delta \gg 1$. Usually, this legitimises the use of cosmological simulations based on Newtonian dynamics to describe the nonlinear structure growth on small scales. Formally, the Newtonian approach is obtained by perturbing only the time-time component of the FLRW metric Eq.~\eqref{eq:conf_ds} by $2\phi_g /c^2$
\begin{equation}
ds^2=a^2 \left[- \left( 1+2 \dfrac{\phi_g}{c^2} \right) c^2 d \eta^2 + \delta_{i j}d x^i dx^j \right]\, . \label{eq:generic_new}
\end{equation}
The Einstein equations give the Poisson equation Eq.~\eqref{eq:Poisson_eq} again, while the dynamics is described by the stress-energy conservation Eq.~\eqref{eq:T_conservation} with continuity and Euler equations. %It is then clear that the Newtonian approximation reproduces the non-relativistic limit of the linear PT.
Recently, cosmologists have begun to investigate cosmic dynamics beyond the Newtonian approximation and to search for measurable relativistic effects on cosmic scales. %, \cite{bruni2014computing, bertacca2015galaxy}. This is the case, for example, when the Newtonian approximated metric is used to compute the geodesics of relativistic particles\footnote{It is well known that the Newtonian estimate of the Rees-Sciama effect and the gravitational lensing effect differ by a factor of two.}, see e.g. \cite{schneider1992gravitational}.

Estimating the importance of relativistic corrections in structure formation is of paramount importance in cosmology (see e.g. \cite{bruni2014computing, bertacca2015galaxy, thomas2015fully, Barrera-Hinojosa:2020gnx} and refs. therein), and several approximation techniques have been developed to account for nonlinear GR effects in structure formation: % to this list belongs the \emph{post-Newtonian approach} (PN), which extends the Newtonian approximation by releasing the assumption of weak-field. Formally, it is obtained by expanding the GR equations in inverse powers of the speed of light, with the zero-order being the Newtonian limit. For the application of the PN approach to cosmological perturbations see \cite{tomita1988post, shibata1995post, carbone2005unified} and \cite{mater} for the formulations of PN cosmology in two different gauges.
to this list belongs the \emph{post-Newtonian approximation} (\setwd{PN}{acr:PN}), which we will encounter in Chapter~\ref{chap:nonlinearities}.
%The PN approach extends the Newtonian approximation by releasing the weak-field assumption. 
Formally, it is obtained by expanding the equations of GR in inverse powers of the speed of light, where the zero-order is the Newtonian limit. For the application of the PN approach to cosmological perturbations, see \cite{tomita1988post, shibata1995post, carbone2005unified} and \cite{mater} for the formulations of PN cosmology in two different gauges. In the following, we give a brief overview of the PN approximation as presented in \cite{mater}.

%In the Eulerian Poisson gauge, the post-Newtonian expansion of the Newtonian metric in Eq.~\eqref{eq:generic_new} reads
%\begin{equation}
%ds^2=a^2 \left[- \left( 1+2 \dfrac{\phi_g}{c^2}+\dfrac{V}{c^4} \right) c^2 d \eta^2 + \left( 1+2 \dfrac{\phi_g}{c^2} \right)\delta_{i j}d x^i dx^j \right]\, . \label{eq:generic_PN}
%\end{equation}
%The Eulerian PN metric can be directly inserted into the evolution equations and solve them at first-order in $1/c^2$ expansion: this is the Eulerian approach. In \cite{mater}, the 

The authors use the Lagrangian approach by relating the evolved (Eulerian) position $x^{\mu}$ and the initial (Lagrangian) position $q^{\mu}$ of the fluid particles with the transformation
\begin{align}
%x^0(\eta, q^j)&=\eta+\dfrac{1}{c^2}\calS^0(\eta, q^j)
x^{\mu}(\eta, q^j)&= q^{\mu} + \calS^{\mu}(\eta, q^j)\, .\label{eq:Euler_to_Lagran}
\end{align}
The vector $\calS^{\mu}(\eta, q^i)$ is the \emph{displacement vector} and represents the difference in matter flow induced by the inhomogeneities. In a homogeneous Universe, the comoving Eulerian coordinate $x^{\mu}$ matches the Lagrangian coordinate $q^{\mu}$. The presence of inhomogeneities locally alters the expansion as the perturbations grow with time. This is encoded by the displacement vector $\calS^{\mu}(\eta, q^i)$, which is the fundamental field describing the evolution of the inhomogeneities \cite{Catelan:1994kt, Catelan:1994ze}. Equivalently, the relation in Eq.~\eqref{eq:Euler_to_Lagran} can be expressed by the Jacobian of the transformation \cite{mater}
\begin{equation}
\mathcal{J}\UD{\mu}{\nu}=\dfrac{\partial x^{\mu}}{\partial q^{\nu}}=\delta\UD{\mu}{\nu}+\calS\UD{\mu}{\nu}\, ,\label{eq:Jacobian_EtoL}
\end{equation}
 where $\calS\UD{\mu}{\nu}=\dfrac{\partial \calS^{\mu}}{\partial q^{\nu}}$ is the deformation tensor. The expression of $\calS\UD{\mu}{\nu}$ is found perturbatively by searching for solutions of the trajectories $x^{\mu}$: this is the key point of the PN approach presented in \cite{mater}. Instead of perturbing over the density fluctuations and the velocity fields, as in the Eulerian approach, the perturbation is performed only in the trajectories. %while the perturbation is performed in the trajectories, all other quantities such us the density fluctuation are calculated exactly.
%instead of perturbing over density fluctuations and velocity fields, like in the Eulerian approach, the perturbation is performed in the trajectories.

%The Newtonian limit of the Jacobian $\mathcal{J}\UD{\mu}{\nu}$ has the form
%\begin{equation}
%\mathcal{J}\UD{\mu}{\nu}=\begin{pmatrix}
%1 && 0 \\
%0 && \mathcal{J}\UD{i}{\nu}
%\end{pmatrix}\, ,
%\end{equation}
%with $\mathcal{J}\UD{i}{\nu}=dfrac{\partial \calS^{i}}{\partial q^{\nu}}$.
%In synchronous-comoving gauge the Newtonian metric assumes the form
%\begin{equation}
%ds^2=a^2 \left[- c^2 d \eta^2 + \delta_{i j}\mathcal{J}\UD{i}{\mu}\mathcal{J}\UD{j}{\nu}d x^{\mu} dx^{\nu} \right]\, , \label{eq:generic_Lagrangian}
%\end{equation}
%and the evolution is determined by the (Newtonian) Lagrangian dynamics, \cite{Catelan:1994ze, Catelan:1994kt}.% Raychaudhuri equation, continuity equation and the momentum constraint.
%%%%%%%%%%%%%%%%%%%%%%%%%%%%%%%%%%%%%%%%%%
The post-Newtonian expression of the Jacobian $\mathcal{J}\UD{\mu}{\nu}$ has the form
\begin{equation}
\mathcal{J}\UD{\mu}{\nu}=\begin{pmatrix}
1+ \dfrac{1}{c} \dfrac{\partial \calS^0}{\partial \eta} && \dfrac{1}{c} \dfrac{\partial \calS^0}{\partial q^{j}} \\
 && \\
v^{i} && \mathcal{J}\UD{i}{j}
\end{pmatrix}\, ,\label{eq:jacobian_PN}
\end{equation}
with $v^i= \delta^{i j}\dfrac{\partial \calS^0}{\partial q^{j}}$ the peculiar velocity, and the spatial deformation $\mathcal{J}\UD{i}{j}=\dfrac{\partial \calS^{i}}{\partial q^{j}}$ being the Newtonian limit of the Jacobian Eq.~\eqref{eq:jacobian_PN}.
In synchronous-comoving gauge the post-Newtonian metric assumes the form
\begin{align}
ds^2=a^2 \left\{- c^2 d \eta^2 + \right.& \left. \gamma_{l k}d q^{l} dq^{k} \right\}=  \nonumber\\
a^2 \left\{- c^2 d \eta^2 + \right.& \left. \left[\left( 1+\dfrac{\chi}{c^2}\right)\delta_{i j}\mathcal{J}\UD{i}{l}\mathcal{J}\UD{j}{k} + \dfrac{1}{c^2} \pi_{l k}\right]d q^{l} dq^{k} \right\}\, . \label{eq:generic_Lagrangian}
\end{align}
The PN scalar and tensor modes $\chi$ and $\pi_{i j}$ are sourced by combinations of the peculiar gravitational field $\phi_{g}$ and the peculiar velocity gradient tensor $\theta\UD{i}{j}=1/2 \gamma^{i l}\dfrac{\partial \gamma_{l j}}{\partial \eta}$. In particular, they are found from 
\begin{align}
\chi = & 2 \mathcal{H}\mathcal{S}^{0}-2 \phi_{g}- \Upsilon \\
D^2 \pi_{i j } = & D_i D_j \Upsilon + \delta_{i j} D^2\Upsilon+2 (\theta\UD{k}{k}\theta_{i j}-\theta_{i k}\theta\UD{k}{j})\, ,
\end{align}
where $D_i$ is the covariant spatial derivative of $\gamma_{i j}$ in the Newtonian limit, while $\Upsilon$ and $\mathcal{S}^0$ are the solutions of
\begin{align}
D^2 \Upsilon = & -\dfrac{1}{2} \left[(\theta\UD{k}{k})^2-\theta\UD{i}{j}\theta\UD{j}{i}\right] \\
D^2 \mathcal{S}^0 = & \theta\UD{k}{k}\, .
\end{align}
%where $\chi$ and $\pi_{i j}$ are the PN scalar and tensor fields obtained from the lowest order in the $1/c^2$ expansion of the energy constraint and in the evolution equation, respectively. %The Jacobian $\mathcal{J}$ results from the Raychaudhuri equation\footnote{The Raychaudhuri equation expresses the evolution equation for the expansion of a volume element of the cosmological fluid.} for the velocity-gradient tensor $\theta_{i j}=\mathcal{J}_{i l}\frac{\partial}{\partial \eta}\mathcal{J}\UD{l}{j}$ (see \cite{Catelan:1994ze, Catelan:1994kt} for the dynamics in Lagrangian frame and \cite{mater, carbone2005unified, Villa:2011vt, Verde:2014} for discussions of this PN approach and its application). 
%Thus, the PN spatial meric $\gamma_{i j}$ is fully specified in the Lagrangian frame. 
Once that $\gamma_{i j}$ is known, one obtains the density contrast $\delta(\eta, q^i)$ from the exact expression of the continuity equation in Lagrangian frame
\begin{equation}
\delta(\eta, q^i)=\left(1+\delta_0(q^i)\right)\sqrt{\dfrac{\gamma_0(q^i)}{\gamma(\eta, q^i)}}-1\, , \label{eq:delta_in_pn}
\end{equation}
where $\gamma={\rm det}(\gamma_{i j})$, and $\delta_0(q^i)=\delta(\eta_0, q^i)$ and $\gamma_0(q^i)=\delta(\eta_0, q^i)$ are respectively the density contrast and the determinant of the spatial metric at present time.
Note that the expression of the density fluctuations in Lagrangian framework Eq.~\eqref{eq:delta_in_pn} is not expanded in $\gamma$. In other words, it is capable of mimicking the  nonlinear behaviour of structure formation\footnote{However, this perturbation technique is limited by the formation of caustic singularities.}.
%%%%%%%%%%%%%%%%%%%%%%%%%%%%%%%%%%%%%%%%%

The PN approximation is one of the many proposed methods to describe nonlinear GR effects in structure formation. 
Other perturbative approaches are: the \emph{post-Friedmann approximation} (see \cite{Milillo:2015cva, Rampf:2016wom} for a different approach, which adapts to cosmology the weak-field post-Minkowskian approximation and reproduces linear-order cosmological perturbation theory at their zeroth-order), the \emph{weak-field approximation}\footnote{The leading order of the last two approximation schemes were shown to be equivalent for a dust Universe in the Poisson gauge in \cite{kopp2014newton}, whereas \cite{mater, carbone2005unified} were constructed on purpose to include second-order perturbation theory at their PN order.} (see \cite{green2011new} for the development of the framework and \cite{Adamek:2013wja} for estimations with the use of N-body simulations for a plane-symmetric Universe), and, more recently, a \emph{two-parameters gauge-invariant approximation} (see \cite{Goldberg:2016lcq}).  
%too close to paper
%\MG{HEYLEY Thesis: Post-Friedmann expansion provides an approximation for GR that cap- tures both the small-scale nonlinear dynamics and the large-scale linear dy- namics (Milillo et al., 2015). In the Newtonian limit of this expansion, there is a non-zero vector potential, in addition to the usual scalar potential, in the metric tensor (Bruni, Thomas, and Wands, 2014). This encapsulates the frame-dragging effect, sourced by purely Newtonian terms, and there- fore can be calculated from nonlinear N-body simulations. Bruni, Thomas, and Wands (2014) performed the first calculation of the frame-dragging po- tential from a purely Newtonian simulation, showing it has small enough magnitude that N-body dynamics should be unaffected, however, could be measurable in weak-lensing cosmological surveys (see also Thomas, Bruni, and Wands, 2015b; Thomas, Bruni, and Wands, 2015a).}

\subsection{Inhomogeneous models}
Another analytical approach to structure formation is to search for exact (i.e. non-perturbative) inhomogeneous solutions to the Einstein equations. These inhomogeneous solutions %aim to relax some (possibly all) the symmetries of FLRW model. %have \emph{no spacetime symmetry}, in contrast to the maximally symmetric FLRW metric, which is characterised by translational and rotational symmetry as a consequence of the homogeneity and isotropy assumptions. However, to be suitable as a valid cosmological model, an inhomogeneous solution must contain a subclass of the non-vacuum and non-static FLRW metric as a limit (as defined in \cite{bolejko2011inhomogeneous}).
are not assumed to have the symmetries of the FLRW models. However, interesting classes of such solutions typically contain the FLRW models as a limit (in \cite{bolejko2011inhomogeneous} this was used to define suitable classes for representing inhomogeneous cosmological models).
Two noticeable examples are the \emph{Lema\^{i}tre-Tolman-Bondi} (\setwd{LTB}{acr:LTB}) and the \emph{Szekeres} models (see \cite{lemaitre1933univers, tolman1934effect, bondi1947spherically} and \cite{Szekeres:1974ct} for the original articles), which have been extensively studied in cosmology. The LTB is a spherically symmetric solution containing only dust that is inhomogeneously distributed along the radial direction, i.e. the matter is condensed into concentric shells (overdensities) separated by underdense regions. In this model, an observer at the centre of a local underdense region measures a local accelerated expansion caused by the large-scale inhomogeneities. This feature of the LTB models has been investigated as an alternative explanation for the SnIa observations without the need for dark energy, \cite{celerier1999we, alexander2009local}.
Although the LTB solution has been shown to be a valuable toy model\footnote{An interesting review of misleading concepts on LTB models can be found in~\cite{krasinski2012drift}, Sec. 4.} to test possible probes for inhomogeneities and anisotropies at late times, \cite{Quercellini:2010zr}, they cannot be considered as a realistic model of the Universe for its intrinsic symmetries.

A further improvement is represented by the Szekeres models, a class of exact solutions of the Einstein equations that includes both LTB and FLRW solutions as limits.
%%%%%%%%%%%%%%%%%%%%%%%%%%%%%%%%%% 
In his original paper \cite{Szekeres:1974ct}, Szekeres finds all solutions of the form
\begin{equation}
ds_{\rm Sz}^2=-c^2 d t^2 + e^{2 \alpha(t,q_{\rm 1},q_{\rm 2},q_{\rm 3})} {dq^{\rm 1}}^2 + e^{2 \beta(t,q_{\rm 1},q_{\rm 2},q_{\rm 3})} ({dq^{\rm 2}}^2+ {dq^{\rm 3}}^2)
\label{eq:Old_Szekeres_metric}
\end{equation}
Two distinct classes of spacetime metrics can be distinguished: class \textsc{I}, which are a generalization of the  Lema\^{i}tre-Bondi-Tolman model, and class \textsc{II}, which are a generalization of the Kantowski-Sachs and FLRW models. These original solutions were obtained for a pressureless matter (dust) and later extended by Barrow and Stein-Schabes in \cite{Barrow:1984zz} to include a cosmological constant $\Lambda$. 
M. Bruni and N. Meures presented a more recent formulation of the class-\textsc{II} solutions in \cite{Meures:2011ke} that will be later used in Chapter~\ref{chap:bigonlight}. This formulation distinguishes the contribution of inhomogeneities from the FLRW background and allows us to express the spacetime metric in a form more convenient for cosmological applications: the expression of the line element Eq.~\eqref{eq:Old_Szekeres_metric} for this Szekeres model\footnote{We choose here to use our notation instead that of \cite{Meures:2011ke}. The line element (\ref{eq:Szekeres_metric}) is different from the one presented in \cite{Meures:2011ke} because we use conformal time. Of course, this does not affect the results, since it can be easily shown that the two metrics are equivalent under a coordinate transformation.} is rewritten as
%%%%%%%%%%%%%%%%%%%%%%%%%%%%%%%%%%%%%%
\begin{equation}
ds_{\rm Sz}^2=a^2 \left[ -c^2  d \eta^2 + {dq^{\rm 1}}^2 + {dq^{\rm 2}}^2+ Z^2(\eta,q^{\rm 1},q^{\rm 2},q^{\rm 3}) {dq^{\rm 3}}^2 \right]\, .\label{eq:Szekeres_metric}
\end{equation}
As it is shown in \cite{Meures:2011ke}, thanks to the symmetry of the problem, the function $Z(\eta,q^{\rm 1},q^{\rm 2},q^{\rm 3})$ is decomposed as 
\begin{equation}\label{eq:sz_Z}
Z(\eta,q^{\rm 1},q^{\rm 2},q^{\rm 3})=F(\eta,q^{\rm 3})+A(q^{\rm 1},q^{\rm 2},q^{\rm 3})\,,
\end{equation}
where the function $F(\eta, q^{\rm 3})$ satisfies Newton's evolution equation for the first-order density contrast\footnote{This was implicitly shown in Sec.~$5$ of the Szekeres'original paper \cite{Szekeres:1974ct} and later by many other authors such as those of \cite{bonnor:1977pp}. However, it was Goode and Wainwright who explicitly recognized that the relativistic equations for the density fluctuations in Szekeres model are the same as in Newtonian gravity, \cite{Goode:1982pg}. They also provide a new formulation of Szekeres solutions which is much more useful in cosmology and in which the relation to the FLRW solution is clarified.}, Eq.~\eqref{eqforF}. Neglecting the decaying modes, it is therefore possible to factorize $F(\eta, q^{\rm 3})$ without loss of generality as\footnote{The time-dependent-only growing mode is denoted by $f_+$ in \cite{Meures:2011ke} and is given in Eq.~(11b) in a dimensionless time variable $\tau$. To reconcile $\cal D$ in \eqref{eq:grow_mode} and $f_+$, one must: first transform $f_+(\tau)$ into a conformal time $f_{+}(\eta)$ and then normalise so that $f_{+}(\eta_{\rm 0})=1$. The final result is $F(\eta, q^{\rm 3})$ as in \eqref{eq:F_Sz}.}
\begin{equation}
\label{eq:F_Sz}
F(\eta, q^{\rm 3})= {\cal D}(\eta) \beta_+(q^{\rm 3})\, ,
\end{equation}
where we remind that $\mathcal{D}$ is the growing mode solution for the density contrast, Eq.~\eqref{eq:grow_mode}. It follows that  $F(\eta, q^{\rm 3})$ coincides with the linear density contrast and more precisely we have $\delta_{\rm Lin}(\eta, q^{\rm 3})=- {\cal D}(\eta) \beta_+(q^{\rm 3})$ \footnote{The minus sign between $F$ and $\delta$ follows from the fact that in eq. ($A8$) of \cite{Meures:2011ke} the authors set, in full generality,  $\delta_{in}=-\frac{F_{in}}{F_{in}+A}$.}.

The purely spatial function $A(q^{\rm 1},q^{\rm 2},q^{\rm 3})$ sets the spatial distribution of the density contrast 
\begin{equation}
\delta_{\rm Sz}=-\dfrac{F}{F+A}\, . \label{eq:Sz_density}
\end{equation}
From Einstein equations follows that $A$ is decomposed as, \cite{Meures:2011ke}
\begin{equation}
A(q^{\rm 1}, q^{\rm 2}, q^{\rm 3})=1+\beta_{\rm +}(q^{\rm 3}) B \left\{\left[q^{\rm 1}+\omega(q^{\rm 3})\right]^2+\left[q^{\rm 2}+\gamma(q^{\rm 3})\right]^2\right\}\, .
\label{eq:Sz_A}
\end{equation}
with $\omega$ and $\gamma$ being two real functions which reduces to $\omega=\gamma=0$ for the special case of axial symmetry around $q^{\rm 3}$.
The term $B$ in Eq.~\eqref{eq:Sz_A} is a constant and is given by (see App. C in \cite{Grasso:2021zra})
\begin{equation}
B= \frac{5}{4} \stuff \frac{\cal D_{\rm in}}{a_{\rm in}}\, ,
\label{eq:link_on_B}
\end{equation}
where $\mathcal{D}_{\rm in}= a_{\rm in}$ for initial conditions set deeply in the matter-dominated era. As noted before, the function $\beta_{\rm +}$ is the part of $A$ which specifies the spatial distribution of the first-order density contrast, and it can be related to the peculiar gravitational potential $\phi_{\rm 0}$ via the cosmological Poisson equation Eq.~\eqref{eq:Poisson_eq}.

Other examples of exact inhomogeneous cosmological models are \emph{black-hole lattices} (see \cite{linquist1957dynamics, clifton2009archipelagian, clifton2015applications, bentivegna2018black} and \cite{bentivegna2012evolution, bentivegna2013evolution} for numerical investigations), \emph{plane symmetric models} or \emph{wall Universe} (see \cite{diDio2012back, adamek2014distance, Villa:2011vt} for an exact, numerical, and PN analysis of the wall Universe), and \emph{Swiss Cheese} models (\cite{einstein1945influence}).
 

\subsection{Numerical simulations}
Together with the latter two approaches, numerical simulations in cosmology have become a valuable tool for modelling nonlinear regimes in structure formation. 
The first generation of numerical codes in cosmology used the Newtonian approximation of GR to simulate systems of N self-gravitating identical objects. A pioneering application of these \emph{N-body simulations} was proposed by J. Peebles in 1970, \cite{peebles1970structure}, to model the formation of the Coma cluster.
The steep growth in computational power led to the development of more detailed N-body codes capable of simulating up to $10^{12}$ particles and producing a realistic structure of the cosmic web, \cite{vogelsberger2014introducing, nelson2021illustristng}.
%Moreover, N-body simulations were also used to input approximated GR equations as e.g. in \cite{Bruni:2013mua, Adamek:2014xba, Fidler:2017pnb}.
%%%%%%%%
A number of attempts have been made to incorporate relativistic corrections into N-body simulations using some of the perturbation methods listed above, see e.g.  \cite{bonvin2011galaxy, green2012newtonian, bertacca2014observed}, as well as N-body simulations have been used as an input for approximated GR equations, as in \cite{Bruni:2013mua, Adamek:2014xba, Fidler:2017pnb}.
 
Another approach to cosmological simulations is represented by \emph{full-GR numerical codes}, which use numerical methods to directly solve Einstein's equations (see \cite{loffler2012einstein, Bentivegna:2016stg, Mertens:2015ttp, Adamek:2016zes, macpherson2017,east2018comparing, Daverio:2019gql, barrera2020gramses} for the codes used in cosmology, and \cite{Adamek:2020jmr} for the comparison between them).
Numerical solutions of the Einstein equations were used early on to study the dynamics of strong-field gravitational systems, such as the study of the two-body problem for the two ends of a wormhole (Hahn and Lindquist in 1964, \cite{hahn1964two}) and the generation of pure gravitational waves (Eppley in 1977, \cite{eppley1977evolution}), where the highly nonlinear relativistic phenomena dominate. 
Since these early applications, continuous improvements in numerical techniques have shaped \emph{numerical relativity} and enabled many successes in the description of compact astrophysical objects \cite{Pretorius:2005gq, baker2006gravitational, campanelli2006accurate, buonanno2007inspiral, hinder2018eccentric, baiotti2008accurate, chaurasia2018gravitational} and in cosmological dynamics, \cite{Giblin:2015vwq, Bentivegna:2015flc, adamek2016general, macpherson2017, Macpherson:2018akp, Barrera-Hinojosa:2020gnx}.

To simulate full-GR dynamics in numerical relativity, the Einstein equations Eq.~\eqref{eq:einsteinEq} must be reformulated as an initial value problem, clearly separating the temporal and spatial dependence in the equations. This procedure is known as the \emph{$3+1$ splitting of spacetime} (or \emph{\setwd{ADM}{acr:ADM} formalism}) \cite{arnowitt1959, Smarr:1977uf}, and forms the common theoretical framework for most of the numerical codes mentioned so far. 
In what follows, we will discuss some of the key concepts of the $3+1$ approach to GR that we will encounter later in this thesis. For comprehensive references on the 3+1 formalism, see \cite{Alcubierre2008, baumgarte2010numerical, gourgoulhon20123+}.

Let us start by considering a manifold $(\mathcal{M}, g_{\mu \nu})$ which is globally foliated by a family of three-dimensional space-like hypersurfaces $\Sigma_t$. We also assume that the foliation is labelled by a monotonic function $t$ such that $t=\const$ on each slice, see Fig.~\ref{fig:slice}. In other words, each slice $\Sigma_t$ is identified as the level (hyper)surface of $t=\const$ and is characterised by the timelike  vector $\nabla^{\mu} t=g^{\mu \nu}\nabla_{\nu} t$ orthogonal to the hypersurface and such that $(\nabla^{\mu}t) (\nabla_{\mu} t)= - \alpha^{-2}$.
 In this view, the function $t$ can be interpreted\footnote{Note that $t$ will not necessary coincide with the proper time of any particular observer.} as a ``global time'' (synchronising all points on $\Sigma_t$), whose flow is represented by $\nabla^{\mu} t$. %(which synchronises all points on $\Sigma_t$).
The unit \emph{normal vector} to $\Sigma_t$ $n^{\mu}$ is given as $n^{\mu}=-\alpha \nabla^{\mu}t$ and represents the normalised time-like vector colinear to the time flow $\nabla^{\mu}t$. %The unit normal vector identifies all ``Eulerian observers'', i.e. all observers at rest in the slice. 
The \emph{lapse function} $\alpha$ gives the flow rate of the proper time  $\tau$ of an ``Eulerian observer'', i.e. an observer moving along the normal vector $n^{\mu}=\frac{d x^{\mu}}{d \tau}$, with respect to the global time $t$ \cite{Alcubierre2008}
\begin{equation}
d\tau=\alpha d t\, . \label{eq:lapse}
\end{equation}
This can be shown by computing the variation of the global time flow $\nabla^{\mu}t$ along $n^{\mu}$
\begin{equation}
n^{\mu}\nabla_{\mu}t=(-\alpha \nabla^{\mu}t)\nabla_{\mu}t=\dfrac{1}{\alpha}\, .
\end{equation}
%Therefore, $\alpha$ gives precisely the difference between the proper time of the Eulerian observer $\tau$ and the global time $t$.
%The unit normal vector identifies all ``Eulerian observers'', i.e. all observers at rest in the slice. 
%The difference between the proper time of the Eulerian observer $\tau$ and the global time $t$ is given by the scalar \emph{lapse} function $\alpha$
%%%%%%%%%%%%%%%%%%%%%%%%%%%%%%%%%%%%%%%%%%%%%%%%%%%%%%%%%%%%%%%%%%%%%%%%%%%%%%%%%%%%%%%%%%%%%%%%%%%%%%%FIGURE
\begin{figure}[ht]
    \centering
    \includegraphics[width=0.9\linewidth]{pict/sliceM.pdf}
    \caption{The constant-time hypersurfaces $\Sigma_{t}$ foliating the four-dimensional spacetime $\mathcal{M}, g_{\mu \nu}$, introduce a unit normal vector $n^{\mu}$ orthogonal to $\Sigma_{t}$. The coordinate flow, i.e. the lines of constant spatial coordinates, is given in terms of the lapse $\alpha$ and the shift $\beta^i$ gauge functions.}\label{fig:slice}
\end{figure}
%%%%%%%%%%%%%%%%%%%%%%%%%%%%%%%%%%%%%%%%%%%%%%%%%%%%%%%%%%%%%%%%%%%%%%%%%%%%%%%%%%%%%%%%%%%%%%%%%%%%%%%

The adjective ``Eulerian'' refers to the fact that it is the observer who measures how the points on the foliation evolve in time. %are at rest in the slice. 
In general, any other observer is called a ``coordinate'' observer and identified by the timelike vector field
\begin{equation}
t^{\mu}=\alpha n^{\mu}+\beta^{\mu}\, ,
\end{equation}
where $\beta^{\mu}$ is the \emph{shift vector} quantifying the displacement on $\Sigma_t$ of the coordinate observer $t^{\mu}$ with respect to the position of the Eulerian observer. The choice of $\alpha=1$ and $\beta^i=0$ for $t^{\mu}$ is equivalent to choosing a Lagrangian observer.
%, which quantifies the displacement on $\Sigma_t$ of the ``coordinate'' observer $t^{\mu}$ with respect to the Eulerian observer $n^{\mu}$.
The unit normal vector $n^{\mu}$, together with the spacetime metric $g_{\mu \nu}$ define the induced \emph{spatial metric} on each slice \cite{gourgoulhon20123+}
\begin{equation}
\gamma\UD{\mu}{\nu}=g\UD{\mu}{ \nu} + n^{\mu}n_{\nu}\, , \label{eq:spatial_metric}
\end{equation}
which is linked to the orthogonal projector tensor on $\Sigma_t$ as $\gamma\UD{\mu}{\nu}=g^{\mu \rho}\gamma_{\rho \nu}$.
In the adapted coordinate system $t^\mu$, the components of the normal vector and the metric $g_{\mu \nu}$ are written in terms of $(\alpha, \beta^i, \gamma_{i, j})$ as 
\begin{equation}
\begin{matrix}
n^{\mu}=(\dfrac{1}{\alpha},-\dfrac{\beta^i}{\alpha})\, , && n_{\mu}=(-\alpha,0)
\end{matrix}
\label{eq:nu_nd}
\end{equation}
and
\begin{equation}
\begin{matrix}
g_{\mu \nu}=\begin{pmatrix}
\beta_{i}\beta^i-\alpha^2 &~ \beta_i\\
\beta_j &~ \gamma_{i j}
\end{pmatrix}\, , && g^{\mu \nu}=\begin{pmatrix}
-\alpha^{-2} &~ \alpha^{-2}\beta^i\\
\alpha^{-2} \beta^j &~ \gamma^{i j}-\alpha^{-2} \beta^i \beta^j
\end{pmatrix}\, ,
\end{matrix}
\label{eq:g_3+1}
\end{equation}
where the Latin indices runs from $1$ to $3$, and $\beta_{i}=\gamma_{i j}\beta^j$, \cite{Alcubierre2008}. 

The projection of a generic (r, s) tensor ${}^{(4)} T^{\mu_{\rm 1} \cdots \mu_{\rm r}}_{ \ \ \nu_{\rm 1} \cdots \nu_{\rm s}}$ on the slice is obtained by using the projector $\gamma\UD{\mu}{\nu}$
\begin{equation}
{}^{(3)}T^{\mu_{\rm 1} \cdots \mu_{\rm r}}_{ \ \ \nu_{\rm 1} \cdots \nu_{\rm s}}={}^{(4)} T^{\rho_{\rm 1} \cdots \rho_{\rm r}}_{ \ \ \sigma_{\rm 1} \cdots \sigma_{\rm s}} \gamma\UD{\mu_{\rm 1}}{\rho_{\rm 1}}\cdots\gamma\UD{\mu_{\rm r}}{\rho_{\rm r}} \gamma\UD{\sigma_{\rm 1}}{\nu_{\rm 1}}\cdots\gamma\UD{\sigma_{\rm s}}{\nu_{\rm s}}\, ,
\label{eq:3Dproj}
\end{equation}  
where we have denoted as ${}^{(3)}T^{\mu_{\rm 1} \cdots \mu_{\rm r}}_{ \ \ \nu_{\rm 1} \cdots \nu_{\rm s}}$ the projected tensor, \cite{gourgoulhon20123+}. %\footnote{The notation ${}^{(3)}T$ to indicate projected quantities may result confusing to a less experienced user. Later, we will prefer to use a different, more explicit, notation to indicate the projected part of vectors.}
The very same projector is also used to define the operation of \emph{covariant derivative on the slice} $D_{\mu}=\gamma\UD{\sigma}{\mu}\nabla_{\sigma}$: for the $(r, s)$ tensor ${}^{(4)} T^{\mu_{\rm 1} \cdots \mu_{\rm r}}_{ \ \ \nu_{\rm 1} \cdots \nu_{\rm s}}$ (see e.g. \cite{gourgoulhon20123+})
\begin{equation}
D_{\lambda} T^{\mu_{\rm 1} \cdots \mu_{\rm r}}_{ \ \ \nu_{\rm 1} \cdots \nu_{\rm s}}= \gamma\UD{\mu_{\rm 1}}{\rho_{\rm 1}}\cdots\gamma\UD{\mu_{\rm r}}{\rho_{\rm r}} \gamma\UD{\sigma_{\rm 1}}{\nu_{\rm 1}}\cdots\gamma\UD{\sigma_{\rm s}}{\nu_{\rm s}}\, \gamma\UD{\epsilon}{\lambda}\nabla_{\epsilon} T^{\rho_{\rm 1} \cdots \rho_{\rm r}}_{ \ \ \sigma_{\rm 1} \cdots \sigma_{\rm s}} \, ,
\label{eq:3D_covD}
\end{equation}
which is written in terms of the 3D Christoffel symbol ${}^{(3)}\Gamma^{k}_{\ i j}=\frac{1}{2}\gamma^{k l}(\frac{\partial \gamma_{l j}}{\partial x^i}+\frac{\partial \gamma_{i l}}{\partial x^j}-\frac{\partial \gamma_{i j}}{\partial x^l})$.

With the introduction of the spacetime foliation, one must distinguish between the \emph{intrinsic curvature} of the hypersurface and an \emph{extrinsic curvature} in order to fully characterise the curvature of spacetime. The intrinsic curvature is the curvature of the hypersurface and is defined by the three-dimensional Riemann tensor, i.e. the Riemann tensor with respect to the spatial metric $\gamma_{i j}$
\begin{equation}
{}^{(3)}R\UD{k}{i s j}=\partial_{s} {}^{(3)}\Gamma^{k}_{i j} -\partial_{j} {}^{(3)}\Gamma^{k}_{i s}+ {}^{(3)}\Gamma^{k}_{l s} {}^{(3)}\Gamma^{l}_{i j}- {}^{(3)}\Gamma^{k}_{l j} {}^{(3)}\Gamma^{l}_{i s}\, .
\end{equation} 
On the other hand, the \emph{extrinsic curvature} $K_{\mu \nu}$ represents the curvature of the hypersurfaces with respect to the embedding higher-dimensional spacetime $\mathcal{M}$. It can be determined as the covariant variation of the normal vector $\nabla_{\mu} n_{\mu}$ along the slice $\Sigma_{t}$, namely
\begin{equation}
K_{\mu \nu}=-\gamma\UD{\sigma}{\mu}\gamma\UD{\rho}{\nu}\nabla_{\sigma}n_{\rho}\, .
\label{eq:K_def}
\end{equation}
In conclusion, after the introduction of the foliation, the geometry of spacetime $(\calM, \, g_{\mu \nu})$ is completely defined by the four quantities $(\alpha, \, \beta_{i}, \, \gamma_{i j}, \, K_{i j})$.

The projection of the Einstein equations Eq.~\eqref{eq:einsteinEq} decomposes the system into 3 relations \cite{Alcubierre2008, baumgarte2010numerical}:
\begin{itemize}
\item the \emph{Hamiltonian constraint}: \begin{equation}
{}^{(3)}R+(K\UD{i}{i})^2-K_{i j}K^{i j}- \frac{16 \pi G}{c^2} \rho=0\, ,
\end{equation}
\item the \emph{momentum constraint}: \begin{equation}
D_{j}K\UD{j}{i}- D_{i}K\UD{j}{j}- \frac{8 \pi G}{c^3} S_{i}=0\, ,
\end{equation}
\item the \emph{evolution equation}: \begin{align}
\left(\dfrac{\partial}{\partial t}- \mathcal{L}_{\beta}\right)K_{i j}=\alpha & \left[{}^{(3)}R_{i j}-2 K_{i l}K\UD{l}{j}+K_{i j}K\UD{l}{l} \right]- D_i D_j\alpha\\
-&\dfrac{8 \pi G}{c^4}\alpha \left[S_{i j} -\dfrac{1}{2}\gamma_{i j}(S\UD{i}{i} - \rho c^2) \right]\, ,
\end{align}
\end{itemize}
where we have defined the various components of the stress-energy tensor as $\rho c^2\equiv T_{\mu \nu}n^{\mu}n^{\nu}$, $S_{\mu}\equiv -\gamma\UD{\sigma}{\mu} n^{\rho}T_{\sigma \rho}$, and $S_{\mu \nu}\equiv \gamma\UD{\sigma}{\mu}\gamma\UD{\rho}{\nu} T_{\sigma \rho}$.
An additional evolution equation
\begin{equation}
\left(\dfrac{\partial}{\partial t}- \mathcal{L}_{\beta}\right)\gamma_{i j}=- 2 \alpha K_{i j}\, ,
\end{equation}
 is obtained from Eq.~\eqref{eq:K_def}.
The first two are constraints arising from the conservation of energy and momentum, while the last two indicate the evolution of the metric. As mentioned earlier, the covariant formulation of GR leaves us free to choose the gauge in which we perform the computation. The gauge choice in the ADM formalism is represented by a specific choice of the lapse $\alpha$ and the shift $\beta^i$. 
Despite the success of the ADM formalism in transforming the Einstein equations into an initial value problem, the equations obtained are weakly hyperbolic, which prevents the simulations from evolving over a long time and becoming rapidly unstable.
This problem is overcome by the \emph{BSSN formalism}, \cite{shibata1995evolution, baumgarte1998numerical}, which rewrites the ADM equations into a highly hyperbolic form and allows arbitrarily long and stable evolutions of the Einstein equations.

%\subsection{Light propagation in numerical relativity}
\subsection{Numerical relativity and cosmological observations: the state of the art}


The methods presented above provide a comprehensive general relativistic description of cosmic dynamics. 
Although they allow the inclusion of GR effects in the growth of structures, the key aspect is to describe and evaluate the nonlinear GR effects on cosmological observations.
This requires an equally accurate description of light propagation, which is necessary for tests and comparisons with real data.
These studies are still in the early stages and have been approached in a variety of ways. For example, the distance-redshift relation has been investigated using perturbative methods (see e.g. \cite{DiDio:2016DlPT} for calculations up to second order PT, and  \cite{Sanghai:2017yyn} for PN calculations within a class of cosmological models and the deviation from the homogeneous FLRW), with exact methods (see e.g. \cite{Fanizza:2013GLC} for exact calculations in the geodesic light-cone gauge and \cite{celerier1999we, alexander2009local} for calculations in LTB models), and in various cosmological simulations (see e.g. \cite{adamek2014distance, Adamek:2018rru, Macpherson:2021gbh}). 
Another important example is the estimation of the weak gravitational lensing effect, i.e. the phenomenon of light deflection in the presence of massive objects. Weak lensing on cosmic scales can be used to probe the presence of dark matter and gain insight into the constituents of the Universe, \cite{Stafford:2021uvk}. Furthermore, by comparing the statistical features of the distortion map from galaxy surveys with those obtained from theoretical models, weak lensing can be used to distinguish between different models of modified gravity, \cite{KiDS:2021opn,Euclid:2021icp, Ruan:2021rqv}.
Theoretical estimates of weak lensing observables in the post-Friedmann formalism are presented in \cite{Thomas:2014aga, Gressel:2019jxw}.
Weak lensing map and power spectrum have also been extracted from numerical simulations \cite{Giblin:2017ezj, Lepori:2020ifz}.
%ray tracing:  Barreira:2016wqo, 
%light-cone in N: Borzyszkowski:2017ayl, 
According to these initial findings, the codes used to simulate GR dynamics appear to be consistent with Newtonian simulations for predicting weak lensing observables \cite{Thomas:2014aga, Lepori:2020ifz}, although there is a shift in the luminosity distance statistics \cite{Adamek:2018rru}. Moreover, the PN approximation leads to results different from $\Lambda$CDM for certain cosmological models \cite{Sanghai:2017yyn}. However, some efforts still need to be made to adapt the truly GR numerical codes to (observational) cosmology.
%%%%%%%%
%\MG{Relativistic effects in the non-linear regime have recently become to be investigated with relativistic simulations in galaxy clustering and lensing observables, and the Hubble diagram, see e.g. \cite{Borzyszkowski:2017ayl, Zhu:2017jfl, Giblin:2017ezj, Breton:2018wzk, Adamek:2018rru, Beutler:2020evf, Lepori:2020ifz, Guandalin:2020snp, Lepori:2021lck} and refs. therein.} \MG{Add that light propagation and observables are computed using perturbative schemes (refer to papers where D lum is computed at first and second order) whose expression depends on the geometrical characteristics of the model and on the perturbation scheme  (and order) used!}

The central importance of estimating relativistic effects in cosmological observables requires a theoretical framework that comprehensively describes the propagation of light and all optical effects that result from its interaction with cosmological structures. In the next chapter, we give an overview of the theory of light propagation in geometric optics and introduce the theoretical foundations on which this work is based.
\endinput
