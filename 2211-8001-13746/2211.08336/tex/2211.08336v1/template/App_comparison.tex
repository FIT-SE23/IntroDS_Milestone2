

\chapter{The wall universe and the Szekeres model}
\MG{from non-linearities}
\label{par:crfSzekeres}

In this section we compare the Szekeres spacetime with the plane-parallel case considered in this work. \MG{USED IN CHAP1: CHANGE HERE! In his original paper \cite{Szekeres:1974ct}, Szekeres studied all the solutions to the Einstein field equation with irrational dust for line elements having the form
\begin{equation}
ds_{\rm Sz}^2=-d t^2 + e^{2 \alpha(t,q_{\rm 1},q_{\rm 2},q_{\rm 3})} dq_{\rm 1}^2 + e^{2 \beta(t,q_{\rm 1},q_{\rm 2},q_{\rm 3})} (dq_{\rm 2}^2+ dq_{\rm 3}^2)
\label{eq:Old_Szekeres_metric}
\end{equation}
Two different classes of solutions can be distinguished: class-\textsc{I} solutions are a generalization of the  Lema\^{i}tre-Bondi-Tolman model while class-\textsc{II} solutions are a generalization of Kantowski-Sachs and FLRW model. 
Subsequently, Barrow and Stein-Schabes \cite{Barrow:1984zz} generalized the Szekeres solutions by adding a cosmological constant $\Lambda$ to the dust. 
More recently, Bruni and Meures \cite{Meures:2011ke} presented a new formulation of the class-\textsc{II} Szekeres solutions in which the separation between inhomogeneities and the FLRW background is explicitly provided and thus the spacetime metric is presented in a more convenient form for cosmological applications. We compare our plane-parallel metric with their formulation. We begin by summarizing the results presented in \cite{Meures:2011ke}.

The authors focused their analysis on the Szekeres solutions which admit a flat FLRW background and such that the line element\footnote{We choose here to use  our notation instead that of that of  \cite{Meures:2011ke}. The line element (\ref{eq:Szekeres_metric}) is different from the one presented in \cite{Meures:2011ke} since we use conformal time and we have chosen a different axis of symmetry. Of course this does not affect any results, since it is easy to show that the two metrics are equivalent under a coordinate transformation.} can be written as
\begin{equation}
ds_{\rm Sz}^2=a^2 \left( -d \eta^2 + \gamma^{\rm{Sz}}_{\rm 11} dq_{\rm 1}^2 + \gamma^{\rm{Sz}}_{\rm 22} dq_{\rm 2}^2+ \gamma^{\rm{Sz}}_{\rm 33}dq_{\rm 3}^2 \right)
\end{equation}
where
\begin{equation} \label{eq:Szekeres_metric}
\begin{split}
\gamma^{\rm{Sz}}_{\rm 11} =& X^2(\eta,q_{\rm 1},q_{\rm 2},q_{\rm 3}) \\
\gamma^{\rm{Sz}}_{\rm 22}= & 1\\
\gamma^{\rm{Sz}}_{\rm 33}=& 1.
 \end{split}
\end{equation}
As it is shown in \cite{Meures:2011ke}, thanks to the symmetry of the problem, the function $X(\eta,q_{\rm 1},q_{\rm 2},q_{\rm 3})$ can be decomposed as 
\begin{equation}\label{eq:sz_X}
X(\eta,q_{\rm 1},q_{\rm 2},q_{\rm 3})=F(\eta,q_{\rm 1})+A(q_{\rm 1},q_{\rm 2},q_{\rm 3})\,,
\end{equation}
where the function $F(\eta, q_{\rm 1})$ satisfies the Newtonian evolution equation for the first-order density contrast\footnote{This was shown implicitly in Sec.~$5$ of the Szekeres' original paper \cite{Szekeres:1974ct} and subsequently by many other authors as those of \cite{bonnor:1977pp}. However, it was Goode and Wainwright who recognized explicitly that the relativistic equations for the density fluctuations in Szekeres model are the same as in Newtonian gravity, \cite{Goode:1982pg}. They also provide a new formulation of the Szekeres solutions, much more useful in cosmology, in which the relationship with the FLRW solution is clarified.}
\begin{equation} \label{eqforF}
\ddot{F} +\mathcal{H}\dot{F}-\frac{3}{2}\mathcal{H}_0^2\Omega_{m_0}\frac{F}{a}=0\,,
\end{equation}
which admits two linearly independent solutions, the growing and decaying modes, as is well known. Then $F(\eta, q_{\rm 1})$ coincides with the linear density contrast and more precisely we have $F(\eta, q_{\rm 1})= -\delta_{\rm Lin}(\eta, q_{\rm 1})$ \footnote{The minus sign between $F$ and $\delta$ follows from the fact that in eq. ($A8$) of \cite{Meures:2011ke} the authors set, in full generality,  $\delta_{in}=-\frac{F_{in}}{F_{in}+A}$.}. Neglecting the decaying modes it is possible, without loss of generality, to factorize $F(\eta, q_{\rm 1})$ as\footnote{The time-dependent-only growing mode is denoted by $f_+$ in \cite{Meures:2011ke} and it is given in a dimensionless time variable $\tau$ in Eq.~(11b). To match $\cal D$ in \eqref{eq:growth_fact} and  $f_+$ one needs to: first transform $f_+(\tau)$ to conformal time $f_{+}(\eta)$ and then normalise such that $f_{+}(\eta_{\rm 0})=1$. The final result is $F(\eta, q_{\rm 1})$ as in \eqref{eq:F_Sz}.}
\begin{equation}
\label{eq:F_Sz}
F(\eta, q_{\rm 1})= {\cal D}(\eta) \beta_+(q_{\rm 1})\, ,
\end{equation}
where $\mathcal{D}$ is the growing mode solution for the density contrast given by (see e.g. Eq.~($5.13$) in \cite{Villa:2015ppa}, where we have already normalized in order to have $\mathcal{D}_0=1$)
\begin{equation}
\mathcal{D} (\eta) = \frac{a}{\frac{5}{2}\Omega_{\rm{m 0}} } \sqrt{1+\frac{\Omega_{\rm{\Lambda 0}}}{\Omega_{\rm{m 0}}}a^3}\,  {}_2 F_{1} \left( \frac{3}{2}, \frac{5}{6}, \frac{11}{6}, -\frac{\Omega_{\rm{\Lambda 0}}}{\Omega_{\rm{m 0}}}a^3 \right)\,,
\label{eq:growth_fact}
\end{equation}
with ${}_2 F_{1} \left(a,b,c, x\right)$ being the Gaussian (or ordinary) hypergeometric function. }

On the other hand, our Newtonian plane-parallel metric is given by:
\begin{equation} \label{eq:app_metricNWT}
\begin{split}
\gamma^{\rm{N}}_{11} =& \left(1-\frac{2}{3}\frac{ \mathcal{D} \partial_{\rm q_{\rm 1}}^2 \phi_0}{ \stuff} \right)^2 \\
\gamma^{\rm{N}}_{22}= & 1\\
\gamma^{\rm{N}}_{33}=& 1.
 \end{split}
\end{equation}

We now investigate the link between \eqref{eq:app_metricNWT} and \eqref{eq:Szekeres_metric} by comparing the two forms of $\gamma_{\rm 11}$ and referring to Eqs.~\eqref{eq:sz_X} and \eqref{eq:F_Sz}. Let us start from~\eqref{eq:F_Sz}: to fix the time-independent function $\beta_{\rm +}(q_{\rm 1})$, one can take advantage from the fact that $\delta_{\rm Lin}(\eta, q_{\rm 1})=-{\cal D}(\eta) \beta_+(q_{\rm 1})$ and use the cosmological Poisson equation \eqref{eq:Poisson_eq} to find
\begin{equation}
\beta_+(q_{\rm 1}) = - \frac{2}{3}\frac{\partial_{\rm q_{\rm 1}}^2 \phi_{\rm 0}(q_{\rm 1})}{\stuff}\, .
\label{eq:beta+}
\end{equation}
At this point we have completely fixed $F(\eta, q_{\rm 1})$. Now, by looking at Eq.~\eqref{eq:sz_X} it is straightforward to conclude that the two metrics \eqref{eq:app_metricNWT} and \eqref{eq:Szekeres_metric} are fully equivalent if $A(q_{\rm 1},q_{\rm 2},q_{\rm 3})= 1$\footnote{To be more precise it would be enough to neglect the dependence on $(q_{\rm 2}, q_{\rm 3})$  in~\eqref{eq:sz_X} by imposing that $A(q_{\rm 1},q_{\rm 2},q_{\rm 3})\equiv A(q_{\rm 1})$, i.e. $$X(\eta, q_{\rm 1})= A(q_{\rm 1})+F(\eta, q_{\rm 1})\, .$$ 
However, if this were the case, it is easy to show that performing the coordinate transformation $\tilde{q_{\rm 1}}=\int A(q_{\rm 1}) d q_{\rm 1}$ and rescaling $\tilde{F}(\eta, \tilde{q}_{\rm 1})=\frac{F(\eta, q_{\rm 1})}{A(q_{\rm 1})}$, we obtain again $X(\eta, \tilde{q}_{\rm 1})= 1 + \tilde{F}(\eta, \tilde{q}_{\rm 1})$. }.
However, this cannot be the case, as we will now show. Let us start by noticing that planar symmetry implies that the metric components can depend only on the coordinates $(\eta, q_{\rm 1})$, while in the Szekeres symmetry the metric can depend in general on all spatial coordinates (and time). In~\cite{Meures:2011ke} this dependence is encoded in $A(q_{\rm 1},q_{\rm 2},q_{\rm 3})$ in Eq.~\eqref{eq:sz_X} and has the form
\begin{equation}
A= 1+ B \beta_{\rm +}(q_{\rm 1})\left[\left(q_{\rm2}+\gamma(q_{\rm 1})\right)^2+\left(q_{\rm 3}+\omega(q_{\rm 1})\right)^2 \right] \, ,
\label{eq:Sz_A}
\end{equation}
thus the only possibility to have $A=1$ is that $B=0$.
After simple manipulations of the Einstein equations together with Eq.~\eqref{eq:Sz_A} we find that 
\begin{equation}
2 B=\frac{3}{2}\frac{\stuff}{a} \mathcal{D} + \mathcal{H} \dot{\mathcal{D}} \, .
\end{equation}
In \cite{Villa:2015ppa} it is shown that the R.H.S. is constant and it is always different from zero. Indeed, from Eq.($5.52$) in \cite{Villa:2015ppa} we find
\begin{equation}
B= \frac{5}{4} \stuff \frac{\cal D_{\rm in}}{a_{\rm in}}\, ,
\label{eq:link_on_B}
\end{equation}
where $\mathcal{D}_{\rm in}= a_{\rm in}$ for Einstein-de Sitter initial conditions. We finally conclude that the two metrics \eqref{eq:Szekeres_metric} and \eqref{eq:app_metricNWT} cannot coincide.
\endinput
