\chapter{Code Tests}%tests
\MG{incollare da bigonlight}
\label{chap:test}

%%%%%%%%%%%%%%%%%%%%%%%%%%%%%%%%%%%%%%%%%%%%%%%%%%%%%%%%%%%%%%%%%%%%%%%%%%%%%%%%%%%%%%%%%%%%%%%%%%%%%%%%%%%%%%%

In this section we test the accuracy of \texttt{BiGONLight} within well-known cosmological models. The tests are performed by considering the following observables: redshift, angular diameter distance, parallax distance, and redshift drift. We compare the results obtained with three different procedures, by defining the estimator $\Delta O({\rm BGO, X})$
\begin{equation}
\Delta O({\rm BGO, X}) \equiv \dfrac{O^{\rm BGO}-O^{\rm X}}{O^{\rm X}}\, ,
\label{eq:deltaO_def}
\end{equation}
where $O^{\rm BGO}$ refers to Eqs.~\eqref{eq:redshift_def},~\eqref{eq:D_ang_BGO},~\eqref{eq:D_par_BGO}, and~\eqref{eq:z_DRIFT_BGO}. We consider the following three cases: (i) the $\Lambda$CDM model, where the specetime metric is the analytical input for {\tt BiGONLight} to compute $O^{\rm BGO}$ and for $O^{\rm X}$ we use the analytical well-known solutions for all the four observables, see Sec.~\ref{sec:LCDM}; (ii) the inhomogeneous Szekeres model \cite{Meures:Szekeres}, where the specetime metric is again the analytical input for {\tt BiGONLight} but to obtain $O^{\rm X}$ we solve numerically a specific differential equation for each observable, see Sec.~\ref{sec:Szekeres}; (iii) the Einstein-de Sitter model, where the input for {\tt BiGONLight} are the $3+1$ quantities coming from the {\tt Einstein Toolkit} simulation and $O^{\rm X}$ is obtained analytically, see Sec.~\ref{sec:ET}.

It worth noting that if $O^{\rm X}$ is obtained analytically, then $\mathrm{max}\left|\Delta O({\rm BGO, X})\right|$ represents the simulation error: in case (i) we have just the computational error from {\tt BiGONLight} whereas in case (iii) the final error in the observables is the combined effect of both the {\tt Einstein Toolkit} and {\tt BiGONLight} finite precision. On the other hand, if $O^{\rm X}$ is obtained numerically, as in case (ii), then $\Delta O({\rm BGO, X})$ gives only an estimation of the accuracy of the two methods used.

\section{Light propagation in homogeneous cosmologies: LCDM, EdS}
\label{sec:LCDM}
\MG{add EdS together with LCDM + fix intro}
The first group of tests regards the study of light propagation in the flat $\Lambda$CDM model. %This is an exact solution of the Einstein field equations representing an homogeneous and isotropic spacetime and the matter-energy content consists of irrotational dust of cold dark matter and a cosmological constant $\Lambda$. The line element is given by
%\begin{equation}
%ds^2=a(\eta)^2\left(-d\eta^2+ \delta_{i j} dx^idx^j \right)
%\label{eq:FLRW_ds}
%\end{equation} 
%where $\eta$ is the conformal time and $a(\eta)$ is the scale factor, which is the solution of Einstein equations and describes the dynamics of the model. The explicit result is found to be \cite{gradshteyn2014table}
%\begin{equation}
%a(\eta)=\frac{\sqrt[3]{\frac{\Omega_{\rm m_0}}{\Omega_{\rm \Lambda}}} \Big(1-{\rm cn}\left(\mathit{y} |\mathit{r} \right)\Big)}{(\sqrt{3}-1)+(\sqrt{3}+1) {\rm cn}\left(\mathit{y} |\mathit{r} \right)}\, ,\label{eq:a_LCDM}
%\end{equation}
%where ${\rm cn}(\mathit{y}|\mathit{r})$ is the Jacobi elliptic cosine function, with $\mathit{y}=\left(\sqrt[4]{3} \sqrt[6]{\Omega_{\rm \Lambda}} \sqrt[3]{\Omega_{\rm m_0}}\right) \mathcal{H}_0 \eta$, $\mathcal{H}_{\rm 0}$ being the Hubble parameter $\mathcal{H}=\frac{1}{a}\frac{d a}{d \eta}$ evaluated today, and $\mathit{r}=\sqrt{\frac{\sqrt{3}+2}{4}}$.

To test {\tt BiGONLight} we consider two classical observables, namely the redshift $z$ and the angular diameter distance $D_{\rm ang}$, and two interesting observables that are not yet measured in the cosmological context as they belong to the new research field named Real-time Cosmology, see Ref.~\cite{Quercellini:2010zr}. One is the parallax distance which exploits the motion of the Solar System with respect to the Cosmic Microwave Background frame providing a baseline of $78 \, {\rm AU}$ per year for the cosmic parallax\footnote{For an exhaustive definition of the parallax see e.g.  Ref.~\cite{rasanen}, in which the author distinguishes between the three different cases: one source observed by two  observers separated by spacelike interval (classic parallax), two sources observed by two  observers separated by spacelike interval and two sources observed by one observer at two different moments (cosmic parallax also known as position drift).}. Cosmic parallax was first proposed in $1986$ in Ref.~\cite{kardashev} and it is expected to be measured by the Gaia satellite,~\cite{refId0}, in the next few years. For discussions and forecasts about the measurements of the cosmological parallax distance we refer to Refs.~\cite{Ding:2009xs, Quartin:2009xr, Quercellini:2008ty, Quercellini:2010zr, rasanen, singal2015cosmological, Marcori:2018cwn, PhysRevLett.121.021101} and Refs. therein. The other is the redshift drift, i.e. the time variation of the redshift of a source. It was first derived for the FLRW models in Refs.~\cite{sandage, McVittie1962}. Since then, and particularly in recent years, a lot of work has been done to investigate the measurability of the redshift drift in cosmology and the information gained, see e.g. Refs.~\cite{Corasaniti:2007bg, Uzan:2007tc, Uzan:2008qp, Martinelli:2012vq,  Lazkoz:2017fvx}.

The analytical expressions in the flat FLRW cosmologies for the four observables that we consider are
\begin{align}
z^{\rm \Lambda CDM} =& \dfrac{a_0}{a}-1 \label{eq:z_FLRW}\\
D^{\rm \Lambda CDM}_{\rm ang}=&\dfrac{a_0}{1+z}\displaystyle\int^z_0 \dfrac{d z'}{ (1+z')\mathcal{H}(z')}%=\dfrac{1}{1+z} \displaystyle\int^z_0 \dfrac{d z'}{ H_0 E(z')}
 \label{eq:D_ang_FLRW}\\
\nonumber \\
D^{\rm \Lambda CDM}_{\rm par}=& \dfrac{a_0}{\mathcal{H}_{\rm 0}}\dfrac{\displaystyle\int^z_0 \dfrac{\mathcal{H}_{\rm 0} d z'}{ (1+z')\mathcal{H}(z')}}{1+\displaystyle\int^z_0 \dfrac{\mathcal{H}_{\rm 0} d z'}{ (1+z')\mathcal{H}(z')}}%=\dfrac{1}{H_{\rm 0}}\dfrac{\displaystyle\int^z_0 \dfrac{d z'}{ E(z')}}{1+\displaystyle\int^z_0 \dfrac{d z'}{ E(z')}}
 \label{eq:D_par_FLRW}\\
\nonumber  \\
\zeta^{\rm \Lambda CDM}=& \dfrac{\mathcal{H}_{\rm 0}}{a_{\rm 0}}\left(1-\dfrac{\mathcal{H}(z)}{\mathcal{H}_{\rm 0}}\right)%=H_{\rm 0}\left(1-\dfrac{E(z)}{1+z}\right)
\, , \label{eq:z_drift_FLRW}
\end{align}
where the Hubble parameter in the $\Lambda$CDM model is given by
\begin{equation}
\mathcal{H}(z)=\mathcal{H}_{\rm 0}\sqrt{\Omega_{\rm m_0}(1+z)+\Omega_{\rm \Lambda}(1+z)^{-2}}\, .
\label{eq:E(z)}
\end{equation}
We normalize the today scale factor $a_{\rm 0}=1$ and we take $\mathcal{H}_{\rm 0}=67.36\ {\rm km\, s^{-1}\, Mpc^{-1}}$, the matter parameter today $\Omega_{\rm m_0}=0.315$, and the cosmological constant parameter  $\Omega_{\Lambda}=0.685$ from  \cite{planck2018param}.
The integral in Eq.~\eqref{eq:D_ang_FLRW} and~\eqref{eq:D_par_FLRW} can be solved analytically and the results is, \cite{gradshteyn2014table}
\begin{equation}
\displaystyle\int^z_0 \dfrac{d z'}{ (1+z')\mathcal{H}(z')}=\dfrac{
{\rm F}\big[ \xi(z) |\mathit{r} \big] - {\rm F}\big[ \xi(0)|\mathit{r} \big]}{ (\Omega_{\rm m_0})^{\frac{1}{3}}(\Omega_{\rm \Lambda})^{\frac{1}{6}} 3^{\frac{1}{4}}}
\end{equation}
where ${\rm F}\big[ \xi(z) | \mathit{r} \big]$ the elliptic integral of the first kind, with arguments $\mathit{r}=\sqrt{\frac{2+\sqrt{3}}{4}}$ and
\begin{equation}
\xi(z)=\arccos \left( \frac{2\sqrt{3}}{1+\sqrt{3}+(1+z)\sqrt[3]{\frac{\Omega_{\rm m_0}}{\Omega_{\rm \Lambda}}}}-1 \right)\, .
\end{equation}

The fact that we have analytical expressions for the observables makes this model a perfect test-bed for the code.
We compare the results from {\tt BiGONLight} with the one in Eq.~\eqref{eq:z_FLRW}-\eqref{eq:z_drift_FLRW} by considering the variation
\begin{equation}
\Delta O({\rm BGO, \Lambda CDM}) \equiv \dfrac{O^{\rm BGO}-O^{\rm  \Lambda CDM}}{O^{\rm  \Lambda CDM}}\, .
\label{eq:deltaO_LCDM}
\end{equation}
As shown in Fig.~\ref{fig:LCDM}, the numerical implementation of the BGO method is in excellent agreement with the analytical formulas in $\Lambda$CDM, the variation $\Delta O$ being $10^{-22}\, \div \,10^{-32}$. The maximum value of $\Delta O$, of the order of $10^{-22}$, represents the numerical error over the observables and we reached such small values by using the precision control options {\tt WorkingPrecision, PrecisionGoal} and {\tt AccuracyGoal} implemented in {\tt Mathematica}.
%%%%%%%%%%%%%%%%%%%%%%%%%%%%%%%%%%%%%%%%%%%%%%%%%%%%%%%%%%%%%%%%%%%%%%%%%%%%%%%%%%%%%%%%%%%%%%%%%%%%%%%
\begin{figure}[ht]
    \centering
    \begin{subfigure}{0.49\linewidth}%{0.8\columnwidth}%0.40\textwidth
        \includegraphics[width=\linewidth]{figures/LCDM_z.pdf}
       \caption{Redshift}
        \label{fig:LCDM_z}
    \end{subfigure}
    \begin{subfigure}{0.49\linewidth}%{0.8\columnwidth}
        \includegraphics[width=\linewidth]{figures/LCDM_drift.pdf}
        \caption{Redshift drift}
        \label{fig:LCDM_drift}
    \end{subfigure}
    \begin{subfigure}{0.49\linewidth}%{0.8\columnwidth}
        \includegraphics[width=\linewidth]{figures/LCDM_Dang.pdf}
        \caption{Angular diameter distance}
        \label{fig:LCDM_Dang}
    \end{subfigure}
    \begin{subfigure}{0.49\linewidth}%{0.8\columnwidth}
        \includegraphics[width=\linewidth]{figures/LCDM_Dpar.pdf}
        \caption{Parallax distance}
        \label{fig:LCDM_Dpar}
    \end{subfigure}
    \caption{Variations in the $\Lambda$CDM model, Eq.~\eqref{eq:deltaO_LCDM}, for the redshift~\ref{fig:LCDM_z}, the redshift drift~\ref{fig:LCDM_drift}, the angular diameter distance~\ref{fig:LCDM_Dang}, and the parallax distance~\ref{fig:LCDM_Dpar}. The variable in the horizontal axis is the redshift in $\Lambda$CDM. The values for the cosmological parameters are taken from \cite{planck2018param}.}\label{fig:LCDM}
\end{figure}
%%%%%%%%%%%%%%%%%%%%%%%%%%%%%%%%%%%%%%%%%%%%%%%%%%%%%%%%%%%%%%%%%%%%%%%%%%%%%%%%%%%%%%%%%%%%%%%%%%%%%%%
%%%%%%%%%%%%%%%%%%%%%%%%%%%%%%%%%%%%%%%%%%%%%%%%%%%%%%%%%%%%%%%%%%%%%%%%%%%%%%%%%%%%%%%%%%%%%%%%%%%%%%%


\section{Light propagation in exact inhomogeneous cosmologies}
%\MG{scrivere breve intro}
%%%%%%%%%%%%%%%%%%%%%%%%%%%%%%%%%%%%%%%%%%%%%%%%%%%%%%%%%%%%%%%%%%%%%%%%%%%%%%%%%%%%%%%%%%%%%%%%%%%%%%%%%%
%\subsection{LTB} 
%\MG{scrivere dopo}
%%%%%%%%%%%%%%%%%%%%%%%%%%%%%%%%%%%%%%%%%%%%%%%%%%%%%%%%%%%%%%%%%%%%%%%%%%%%%%%%%%%%%%%%%%%%%%%%%%%%%%%%%%
%\section{The Szekeres model}
\label{sec:Szekeres}
\MG{USED IN CHAP 1: The second group of code tests is performed considering an inhomogeneous cosmological model which is an exact solution of Einstein equations, firstly presented in \cite{Szekeres:1974ct}. The line element for the Szekeres spacetime is:
\begin{equation}
ds^2=- dt^2+e^{2 \alpha} {dx^{\rm 1}}^2+e^{2 \beta} ({dx^{\rm 2}}^2+{dx^{\rm 3}}^2)
\label{eq:Sz_generic}
\end{equation}
with $\alpha$ and $\beta$ functions of the spacetime coordinates $(t, x^{\rm 1},x^{\rm 2},x^{\rm 3})$ that are determined by solving the Einstein equations. We can distinguish two different families of Szekeres models depending whether $\partial_{\rm x^{\rm 3}} \beta \neq 0$ or $\partial_{\rm x^{\rm 3}} \beta = 0$: the first case defines the ``class I'' family, which is a generalization of LTB models (with non-concentric shells), while the case $\partial_{\rm x^{\rm 3}} \beta = 0$ corresponds to a simultaneous generalization of the Friedmann and Kantowski-Sachs models and it defines the ``class II'' family. For a cosmological formulation of the Szekeres spacetimes see e.g. \cite{Goode:1982pg}.
For our tests, we will use a class II Szekeres model filled with dust and a positive cosmological constant as presented in~\cite{Meures:Szekeres}. For this model, the line element \eqref{eq:Sz_generic} is rewritten as:
\begin{equation}
ds^2=a(\eta)^2 \left[- d\eta^2 + {dx^{\rm 1}}^2+{dx^{\rm 2}}^2 +Z(\eta,x^{\rm 1}, x^{\rm 2}, x^{\rm 3})^2 {dx^{\rm 3}}^2\right]
\label{eq:Sz_line}
\end{equation}
and, for the special case of axial symmetry around $x^{\rm 3}$, we have the following form for the function $Z$:
\begin{equation}
Z(\eta,x^{\rm 1},x^{\rm 2},x^{\rm 3})=1+\beta_{\rm +}(x^{\rm 3}) \mathcal{D}(\eta)+\beta_{\rm +}(x^{\rm 3}) B \left({x^{\rm 1}}^2+{x^{\rm 2}}^2\right)\, .
\label{eq_met_func}
\end{equation}
The function $\mathcal{D}$ is the growing mode solution of the first-order Newtonian density contrast defined as $\delta=\dfrac{\rho -\rho_{\rm \Lambda CDM}}{\rho_{\rm \Lambda CDM}}$ and it is given by, see e.g. \cite{Villa:2015ppa},
\begin{equation}
\mathcal{D} (\eta) = \frac{a}{\frac{5}{2}\Omega_{\rm{m 0}} } \sqrt{1+\frac{\Omega_{\rm{\Lambda}}}{\Omega_{\rm{m 0}}}a^3}\,  {}_2 F_{1} \left( \frac{3}{2}, \frac{5}{6}, \frac{11}{6}, -\frac{\Omega_{\rm{\Lambda 0}}}{\Omega_{\rm{m 0}}}a^3 \right)\,,
\label{eq:grow_mode}
\end{equation}
with ${}_2 F_{1} \left(a,b,c, y\right)$ being the Gaussian (or ordinary) hypergeometric function. The term $B$ in Eq.~\eqref{eq_met_func}
 is constant and given by (see App. C in \cite{Grasso:2021zra})
\begin{equation}
B= \frac{5}{4} \stuff \frac{\cal D_{\rm in}}{a_{\rm in}}\, ,
\label{eq:link_on_B}
\end{equation}
where $\mathcal{D}_{\rm in}= a_{\rm in}$ for initial conditions set deeply in the matter-dominated era. The function $\beta_{\rm +}$ specifies the spatial distribution of the first-order density contrast and it can be related to the peculiar gravitational potential $\phi_{\rm 0}$ via the cosmological Poisson equation at present time
\begin{equation}
\beta_{\rm +}=-\dfrac{2}{3}\dfrac{\partial^2_{\rm x^{\rm 3}}\phi_{\rm 0}}{\stuff}\, .
\end{equation}}
For the tests, we will use a sinusoidal profile for the peculiar gravitational potential $\phi_{\rm 0}= \mathcal{A} \sin(\omega x^{\rm 3})$ with $\omega=\dfrac{2 \pi}{500\, \rm Mpc}$ and amplitude $\mathcal{A}$ such that $\delta_{\rm 0}=0.1$ for the density contrast today. 

In the following, we will present the tests over the redshift, the angular diameter distance and the redshift drift. Contrary to the $\Lambda$CDM case, here the observables are obtained using two numerical methods and the difference is expressed by
\begin{equation}
\Delta O({\rm BGO, Sz}) \equiv \dfrac{O^{\rm BGO}-O^{\rm  Sz}}{O^{\rm  Sz}}\, .
\label{eq:deltaO_Sz}
\end{equation}
All the three tests are done considering that the observer $\mathcal{O}$ is located at the origin of the reference frame, with coordinates $(\eta_0, 0,0,0)$, and she/he sees the light coming from a comoving distant source $\mathcal{S}$, with coordinates $(\eta, 0,0, x^3)$. The light beam is propagating along the $x^3$ axis, with tangent vector $\ell^{\mu}=(\ell^{0},0,0,\ell^{3})$.

The first observable we test is the redshift: for a photon travelling along the $x^{\rm 3}$-axis it is
\begin{equation}
1+z^{\rm Sz}=\dfrac{\left(a \ell^0\right)|_{\rm \mathcal{S}} }{ \left(a \ell^0\right)|_{\rm \calO}}\, ,
\label{eq:Szekeres_z}
\end{equation}
where $\ell^0$ is obtained by solving the geodesic equation
\begin{equation}
\dfrac{d \ell^0}{d \eta} = - \ell^0 \left(2 \mathcal{H} + \dfrac{\dot{Z}}{Z}\right)\, . \label{eq:l0_Szekeres}
\end{equation}
In the above expressions $Z$ is given in Eq.~\eqref{eq_met_func}, $a$ is the scale factor \eqref{eq:a_LCDM}, and $\mathcal{H}$ the Hubble parameter \eqref{eq:E(z)} of the $\Lambda CDM$ background.
The variation $\Delta z(\rm BGO, Sz)$ refers to the numerical solution of Eq.~\eqref{eq:l0_Szekeres} as opposite to the $3+1$ geodesic solved by {\tt BiGONLight}.

The second observable under analysis is the angular diameter distance $D_{\rm ang}$. The standard procedure to compute $D_{\rm ang}$ is solving the Sachs focusing equation 
\begin{equation}
\dfrac{d^2 D_{\rm ang}}{d \lambda^2}= -\left(|\sigma|^2 + \dfrac{1}{2 }R_{\mu \nu}\ell^{\mu}\ell^{\nu}\right)D_{\rm ang}\, ,
\end{equation}
where the initial conditions given at $\cal O$ (i.e. the focusing point) are $D_{\rm ang}|_{\cal O}=0 $ and $\left. \dfrac{d D_{\rm ang}}{d \lambda}\right|_{\cal O}=(\ell_{\sigma}u^{\sigma})_{\rm \calO}$, and $|\sigma|$ is the shear that in the Szekeres model Eqs.~\eqref{eq:Sz_line}-\eqref{eq_met_func} simply vanishes, as shown in~\cite{Meures:2011ke}.
Using the Einstein equations to express $R_{\mu \nu}\ell^{\mu}\ell^{\nu}$ in terms of the density contrast and using conformal time instead of the affine parameter, the focusing equation assumes the form
\begin{equation}
\ddot{D}^{\rm Sz}_{\rm ang}+\dfrac{\dot{\ell}^0}{\ell^0} \dot{D}^{\rm Sz}_{\rm ang}=-  \dfrac{3}{2}\dfrac{\mathcal{H}_{\rm 0}^2\Omega_{\rm m_0}}{a}(\delta+1) D_{\rm ang}^{\rm Sz}\, ,
\label{eq:focusing_eq}
\end{equation}
where the dots refers to derivatives respect to conformal time $\eta$ and the initial conditions at the observation point are $ \ D_{\rm ang} \left|_{\rm \calO}\right.=0$ and $ \dot{D}_{\rm ang} \left|_{\rm \calO}\right.=\dfrac{(\ell_{\sigma}u^{\sigma})_{\rm \calO}}{\ell^0_{\rm \calO}}$.
In synchronous-comoving gauge the density contrast along the geodesic comes directly from the continuity equation and reads
\begin{equation}
\delta= - \dfrac{\mathcal{D} \beta_{\rm +}}{Z}=\dfrac{\dfrac{2}{3}\dfrac{\partial^2_{\rm x^{\rm 3}}\phi_{\rm 0}}{\stuff}\mathcal{D}}{1-\dfrac{2}{3}\dfrac{\partial^2_{\rm x^{\rm 3}}\phi_{\rm 0}}{\stuff}\mathcal{D}-\dfrac{2}{3}\dfrac{\partial^2_{\rm x^{\rm 3}}\phi_{\rm 0}}{\stuff} B ({x^{\rm 1}}^2+{x^{\rm 2}}^2)}\, .
\label{eq:Sz_delta}
\end{equation}
In our estimator Eq.~\eqref{eq:deltaO_Sz} the angular diameter distance $D_{\rm ang}^{\rm Sz}$ is obtained by integrating  Eq.~\eqref{eq:focusing_eq}, whereas $D_{\rm ang}^{\rm BGO}$ is obtained from Eq.~\eqref{eq:D_ang_BGO} with {\tt BiGONLight}.

The last test concerns the calculation of the redshift drift, namely the secular variation of the redshift of the source. It was calculated for some inhomogeneous cosmological models, see e.g. \cite{quartin, yoo2011redshift, Mishra2012,balcerzak2013redshift, mishra2014redshift}, but to our knowledge there is no expression for the Szekeres model considered here, thus we give in the following a short derivation.
Let us consider that during the proper time lapse $\delta \tau_{\mathcal{O}}$ the spacetime coordinates of the observer change from $x^{\mu}_{\mathcal{O}}=(\eta_{\mathcal{O}}, 0,0,0)$ to $X^{\mu}_{\mathcal{O}}=(\Theta_{\mathcal{O}}, 0,0,0)$. Similarly, in the corresponding proper time lapse $\delta \tau_{\mathcal{S}}$, the spacetime coordinates of the source change from $x^{\mu}_{\mathcal{S}}=(\eta_{\mathcal{S}}, 0,0,x^3)$ to $X^{\mu}_{\mathcal{S}}=(\Theta_{\mathcal{S}},0,0,x^3)$. 
Note that in this gauge $\mathcal{O}$ and $\mathcal{S}$ are comoving, meaning that they both have fixed spatial positions (i.e. $\delta x^i_{\mathcal{O}} = \delta x^i_{\mathcal{S}}=0$), but the time changes differently at $\mathcal{O}$ and at $\mathcal{S}$ (i.e. $\Theta_\mathcal{O} - \eta_\mathcal{O} \neq \Theta_\mathcal{S} - \eta_\mathcal{S}$). The redshift and the conformal time of the source change according to\footnote{On the R.H.S. of Eqs.~\eqref{eq:Sz_Z(t)}-\eqref{eq:Sz_T(t)} we use the fact that the source location $x^i_{\mathcal{S}}$ has components on the $x^3$ axis only.}
\begin{align}
\mathcal{Z}(\Theta_{\mathcal{S}}, x^{i}_{\mathcal{S}})&=z(\eta_{\mathcal{S}} ,x^3_{\mathcal{S}})+\delta z(\eta_{\mathcal{S}} ,x^3_{\mathcal{S}})\label{eq:Sz_Z(t)}\\
\Theta(x^{i}_{\mathcal{S}})&=\eta(x^3_{\mathcal{S}})+\delta \eta(x^{3}_{\mathcal{S}})\, ,\label{eq:Sz_T(t)}
\end{align}
while the same quantities at the observer position $X^{\mu}_{\mathcal{O}}$ are $\mathcal{Z}(X^{\mu}_{\mathcal{O}})=z(x^{\mu}_{\mathcal{O}})=0 $ and $ \Theta(X^{i}_{\mathcal{O}})=\eta(x^{i}_{\mathcal{O}})+\delta \eta(x^{i}_{\mathcal{O}})$.
Our final aim is to compute the redshift drift, namely 
\begin{equation}
\zeta=\dfrac{\delta \ln(1+z)}{\delta \tau_{\mathcal{O}}}\, .
\label{eq:Sz_def_drift}
\end{equation}
Let us begin with the variation of the source redshift with respect to the observer proper time  $\frac{\delta z}{\delta \tau_{\mathcal{O}}}$, where it is better to obtain first $\frac{d \delta z}{d x^3}$ (Eq.~\eqref{eq:Sz_delz_dx1}) and $\frac{d \delta \tau}{d x^3}$ (Eq.~\eqref{eq:Sz_deltau_dx}) separately, and then combine them to get an ODE (Eq.~\eqref{eq:z_drift_Szekeres}), whose solution gives the redshift drift.
The starting point are the differentials $\frac{d \eta}{d x^3}$ and $\frac{d z}{d x^3}$. The first is simply given by the null condition $\ell^0=-Z \ell^3$, and reads
\begin{equation}
 \frac{d \eta}{d x^3}=-Z\, ,
\label{eq:Sz_deta_dx}
\end{equation}
while $\frac{d z}{d x^3}$ is obtained by differentiating Eq.~\eqref{eq:Szekeres_z}\footnote{From now on we drop the index $\mathcal{S}$, since  all the quantities are evaluated at the source.}
\begin{align}
\dfrac{d z}{d x^3}=\dfrac{1}{ \left(a \ell^0\right)|_{\rm \calO}}\left(\frac{d a}{d x^3} \ell^0 +\frac{d \ell^0 }{d x^3} a \right)|_{\rm \calS}&=\dfrac{\left(a \ell^0\right)|_{\rm \calS}}{ \left(a \ell^0\right)|_{\rm \calO}}\left(\frac{1}{a}\frac{d a}{d \eta}\frac{d \eta}{d x^3} +\frac{1}{\ell^0}\frac{d \ell^0 }{d \eta}\frac{d \eta}{d x^3} \right)|_{\rm \calS} \nonumber \\
&=(1+z)\left(\mathcal{H} Z + \dot{Z} \right)\, ,
\label{qe:Sz_dz_dx} 
\end{align}
where we used Eq.~\eqref{eq:Sz_deta_dx}, Eq.~\eqref{eq:l0_Szekeres}, and the fact that the redshift at the observer is fixed.
Now we use Eq.~\eqref{eq:Sz_deta_dx} to differentiate Eq.~\eqref{eq:Sz_T(t)}
\begin{align}
\frac{d \delta \eta}{d x^3}=\frac{d \Theta}{d x^3}-\frac{d \eta}{d x^3}=-Z(\Theta, x^3)+Z(\eta, x^3)&=-\left[Z(\eta, x^3)+\dot{Z}(\eta, x^3) \delta \eta\right]+Z(\eta, x^3) \nonumber \\
&=-\dot{Z} \delta \eta\, ,
\label{eq:Sz_deleta_dx}
\end{align} 
and similarly for the redshift we have
\begin{equation}
\dfrac{d \delta z}{d x^3}=\dfrac{d \mathcal{Z}}{d x^3}-\dfrac{d z}{d x^3}=(1+z)\left( Z \mathcal{H}+\dot{Z}\right)^{\cdot}\delta \eta +\left( Z \mathcal{H}+\dot{Z}\right)\delta z \, ,
\label{eq:Sz_delz_dx1}
\end{equation} 
where we keep first-order terms in $\delta \eta$ and $\delta z$ only.
Rearranging terms and using Eq.~\eqref{qe:Sz_dz_dx} again, we get
\begin{equation}
\dfrac{d }{d x^3}\left(\dfrac{\delta z}{1+z} \right)=\left( Z \mathcal{H}+\dot{Z}\right)^{\cdot}\delta \eta\, . \label{eq:Sz_delz_dx2}
\end{equation}
The final step is to express the variation of the conformal time $\delta \eta$ in terms of the proper time at the observer $\delta \tau_{\mathcal{O}}$. We use their relation, which is simply
\begin{equation}
\delta \tau=\sqrt{|g_{\mu \nu} \delta x^{\mu} \delta x^{\nu}|}=a \sqrt{|-\delta \eta^2 + Z^2 \delta {x^{\rm 3}}^2|}= a \delta \eta\, ,
\label{eq:Sz_tau_eta}
\end{equation}
since $\delta x^i=0$. We need the derivative with respect to $x^3$, which reads
\begin{equation}
\frac{d \delta \tau}{d x^3}=\frac{d (a \delta \eta)}{d x^3}=\frac{d a}{d x^3}\delta \eta +a \frac{d \delta \eta}{d x^3}=\left( \frac{1}{a} \frac{d a}{d \eta}\frac{d \eta }{d x^3} - \dot{Z}\right) a \delta \eta=-\left( \mathcal{H} Z + \dot{Z}\right) \delta \tau\, ,
\label{eq:Sz_deltau_dx}
\end{equation}
and, together with Eq.~\eqref{qe:Sz_dz_dx} and Eq.~\eqref{eq:Sz_tau_eta},   the solution of the last equality is
\begin{equation}
\delta \eta=\frac{1}{a}\dfrac{\delta \tau_{\mathcal{O}}}{1+z}\, . \label{Sz_eta}
\end{equation}
By changing $x^3$ to conformal time $\eta$ with the chain rule $\frac{d}{dx^3}=\frac{d\eta}{dx^3}\frac{d}{d\eta}=-Z\frac{d}{d\eta}$, we get from Eq.~\eqref{eq:Sz_delz_dx2} the ODE in $\eta$ for the redshift drift in the Szekeres model
\begin{equation}
\frac{d\zeta}{d \eta}\equiv\frac{d}{d \eta}\left(\frac{\delta \log(1+z)}{\delta \tau_{\mathcal{O}}}\right)=-\frac{1}{a(1+z)}\frac{\left(H Z+ \dot{Z}\right)^{\cdot}}{Z}\, .
\label{eq:z_drift_Szekeres}
\end{equation}
For our test, in the estimator Eq.~\eqref{eq:deltaO_Sz}, $\zeta^{\rm Sz}$ is the numerical solution of Eq.~\eqref{eq:z_drift_Szekeres} and $\zeta^{\rm BGO}$ is the expression in Eq.~\eqref{eq:z_DRIFT_BGO} obtained with {\tt BiGONLight}.
%%%%%%%%%%%%%%%%%%%%%%%%%%%%%%%%%%%%%%%%%%%%%%%%%%%%%%%%%%%%%%%%%%%%%%%%%%%%%%%%%%%%%%%%%%%%%%%%%%%%%%%
\begin{figure}[ht]
    \centering
    \begin{subfigure}{0.49\linewidth}
        \includegraphics[width=\linewidth]{figures/Szekeres_z.pdf}
       \caption{Redshift}
        \label{fig:Sz_z}
    \end{subfigure}
    \begin{subfigure}{0.49\linewidth}
        \includegraphics[width=\linewidth]{figures/Szekeres_drift.pdf}
        \caption{Redshift drift}
        \label{fig:Sz_drift}
    \end{subfigure}
    \begin{subfigure}{0.49\linewidth}
        \includegraphics[width=\linewidth]{figures/Szekeres_Dang.pdf}
        \caption{Angular diameter distance}
        \label{fig:Sz_Dang}
    \end{subfigure}
    \caption{Variations in the Szekeres model, Eq.~\eqref{eq:deltaO_Sz}, for the redshift~\ref{fig:Sz_z}, the redshift drift~\ref{fig:Sz_drift}, and the angular diameter distance~\ref{fig:Sz_Dang}. The variable in the horizontal axis is the redshift in the Szekeres model.}\label{fig:Sz}
\end{figure}

%%%%%%%%%%%%%%%%%%%%%%%%%%%%%%%%%%%%%%%%%%%%%%%%%%%%%%%%%%

As for the $\Lambda$CDM model, also for the Szekeres model we have a very good agreement between the observables calculated using  the standard procedure and the observables from \texttt{BiGONLight}. The smallness of all the variations $\Delta z (\rm BGO, Sz)$, $\Delta \zeta(\rm BGO, Sz)$ and $\Delta D_{\rm ang}(\rm BGO, Sz)$ shown in Fig.~\ref{fig:Sz} means that our code could be a reliable tool for light propagation in inhomogeneous cosmologies, represented here with the Szekeres model, which is computationally more complicated than the homogeneous $\Lambda$CDM case.

%%%%%%%%%%%%%%%%%%%%%%%%%%%%%%%%%%%%%%%%%%%%%%%%%%%%%%%%%%%%%%%%%%%%%%%%%%%%%%%%%%%%%%%%%%%%%%%%%%%%%%%
%%%%%%%%%%%%%%%%%%%%%%%%%%%%%%%%%%%%%%%%%%%%%%%%%%%%%%%%%%%%%%%%%%%%%%%%%%%%%%%%%%%%%%%%%%%%%%%%%%%%%%%

\subsection{A dust universe in Numerical Relativity}
\label{sec:ET}
The main application of the {\tt BiGONLight} package is the computation of observables from numerically simulated spacetimes. 
 For our test we choose to use the {\tt FLRWSolver}\footnote{{\color{blue}{\tt {https://github.com/hayleyjm/FLRWSolver\_public}}.}}, \cite{macpherson2017}, which is a module (or {\it thorn}) of the {\tt Einstein Toolkit} (ET), \cite{loffler2012einstein}: the ET is a collection of  open-source codes, called thorns, based on the {\tt Cactus} framework, \cite{goodale2002cactus}, which allows to solve the Einstein equations in the BSSN formulation of the 3+1 splitting, \cite{shibata1995evolution, baumgarte1998numerical}. The role of the {\tt FLRWSolver} is to provide the initial conditions in the form of linear perturbations around the Einstein-de Sitter (EdS) background, which are then evolved with the ET. Here, we limit ourself to the EdS background model and set perturbations to zero. In other words, we consider a FLRW model in which the Universe is flat and contains only cold dark matter.  The line element of the EdS model in conformal time is
\begin{equation}
ds^2={a^2_{\rm EdS}}(\eta) \left(-d\eta^2+ {dx^2}^2 + {dx^2}^2 + {dx^3}^2 \right)\, ,
\end{equation}
where $a_{\rm EdS}(\eta)=\eta^2$ is the scale factor. 
We carry out the simulation in a cubic domain $-L \le \{x^1, x^2, x^3 \}\le L$ with periodic boundary conditions and spatial resolution $\Delta x=\Delta y=\Delta z= \frac{L}{20}$, where $L$ is the simulation unit length in Mpc\footnote{The physical value is $L=268.11\, {\rm Mpc}$, as it is explained in \ref{apx:Units}.}. The initial data are given at $\eta_{\rm in}=L$ and such that $\gamma^{\rm in}_{i j}=\delta_{i j}$. % and $\rho_{\rm in}= \frac{3 \mathcal{H}_{\rm in}^2}{8 \pi}=\frac{3}{2 \pi L^2}$. 
 The simulation runs with the ET up to $\eta_0=33.2 L$, which corresponds to integrating from redshift $z = 1100$ to present time $z=0$, and we choose a fixed temporal resolution $\Delta \eta=\frac{L}{100}$ due to computational time convenience.
To give an estimation of the simulation error we define
\begin{equation}
\Delta a({\rm ET, EdS}) \equiv \dfrac{a^{\rm ET}-a^{\rm EdS}}{a^{\rm EdS}}\, ,
\label{eq:DeltaA}
\end{equation}
which is the variation between the analytical scale factor in EdS, i.e. $a^{\rm EdS}=\eta^2$ and the scale factor from the numerical simulation $a^{\rm ET}={\rm det}(\gamma_{i j})^{\frac{1}{6}}$.
%%%%%%%%%%%%%%%%%%%%%%%%%%%%%%%%%%%%%%%%%%%%%%%%%%%%%%%%%%%%%%%%%%%%%%%%%%%%%%%%%%%%%%%%%%%%%%%%%%%%%%%
\begin{figure}[ht]
    \centering
        \includegraphics[width=0.8\linewidth]{figures/ET_a_StoO_fullRange.pdf}
\caption{Scale factor in EdS: ET simulation precision. The variable on the horizontal axis is the conformal time in computational units (see \ref{apx:Units}).}
    \label{fig:scale_ET}
\end{figure}
%%%%%%%%%%%%%%%%%%%%%%%%%%%%%%%%%%%%%%%%%%%%%%%%%%%%%%%%%% 
The result, shown in Fig.~\ref{fig:scale_ET}, is of the order of $ 10^{-10}$ and this value is determined by the specific setting that we choose for the ET simulation.

The simulated EdS model constitutes the playground of our tests. We perform light propagation with {\tt BiGONLight} using forward integration in time with the method described in Sec.~\ref{sec:W3+1}. We start from the source $\mathcal{S}$, placed at redshift $z= 10$ in $x^{\mu}_{\calS}=(\eta_{\calS}, 0, 0, 0)$, and we end at the observer $\calO$. The emitted light moves along the diagonal of the cubic domain with initial tangent vector $\ell^{\mu}_{\calS}=(-1, -\frac{1}{2}, \frac{1}{2},\frac{\sqrt{2}}{2})$, until it reaches the observer at $x^{\mu}_{\calO}$. 
The numerical accuracy in the calculation of the observables is tested by means of the variation $\Delta O({\rm BGO, ET})$ defined as
\begin{equation}
\Delta O({\rm BGO, EdS}) \equiv \dfrac{O^{\rm BGO}-O^{\rm  EdS}}{O^{\rm  EdS}}\, ,
\label{eq:deltaO_ET}
\end{equation}
where $O^{\rm BGO}$ is computed numerically with {\tt BiGONLight} using as input the EdS model simulated with the ET and $O^{\rm EdS}$ is the analytical expression in the EdS model that reads
\begin{align}
D_{\rm ang}^{\rm EdS}&=\dfrac{2 a_0}{\mathcal{H}_{\rm 0}}\dfrac{\sqrt{1+z}-1}{(1+z)^{\frac{3}{2}}} \label{eq:ET_Dang}\, , \\
D_{\rm par}^{\rm EdS}&=\dfrac{a_0}{\mathcal{H}_{\rm 0}} \dfrac{\sqrt{1+z}-1}{\frac{3}{2}\sqrt{1+z}-1} \label{eq:ET_Dpar}\, , \\
\zeta^{\rm EdS}&=\dfrac{\mathcal{H}_{0}}{a_0} (1-\sqrt{1+z}) \label{eq:ET_drift} \, .
\end{align}
These are obtained by integrating Eqs.~\eqref{eq:D_ang_FLRW}-\eqref{eq:z_drift_FLRW} with $\Omega_{\rm m_0}=1$ and $\Omega_{\rm \Lambda}=0$.
%%%%%%%%%%%%%%%%%%%%%%%%%%%%%%%%%%%%%%%%%%%%%%%%%%%%%%%%%%%%%%%%%%%%%%%%%%%%%%%%%%%%%%%%%%%%%%%%%%%%%%%
\begin{figure}[ht]
    \centering
    \begin{subfigure}{0.49\linewidth}
        \includegraphics[width=\linewidth]{figures/ET_z_StoO.pdf}
       \caption{Redshift}
        \label{fig:ET_z}
    \end{subfigure}
    \begin{subfigure}{0.49\linewidth}
        \includegraphics[width=\linewidth]{figures/ET_drift_StoO.pdf}
        \caption{Redshift drift}
        \label{fig:ET_drift}
    \end{subfigure}
    \begin{subfigure}{0.49\linewidth}
        \includegraphics[width=\linewidth]{figures/ET_Dang_StoO.pdf}
        \caption{Angular diameter distance}
        \label{fig:ET_Dang}
    \end{subfigure}
    \begin{subfigure}{0.49\linewidth}
        \includegraphics[width=\linewidth]{figures/ET_Dpar_StoO.pdf}
        \caption{Parallax distance}
        \label{fig:ET_Dpar}
    \end{subfigure}
    \caption{Variations in the EdS model with ET, Eq.~\eqref{eq:deltaO_ET}, for the redshift~\ref{fig:ET_z}, the redshift drift~\ref{fig:ET_drift}, the angular diameter distance~\ref{fig:ET_Dang}, and the parallax distance~\ref{fig:ET_Dpar}. The variable in the horizontal axis is the redshift in EdS.}
    \label{fig:ET}
\end{figure}
%%%%%%%%%%%%%%%%%%%%%%%%%%%%%%%%%%%%%%%%%%%%%%%%%%%%%%%%%%
The results are shown in Fig.~\ref{fig:ET}. 
What we see is the error on $\Delta O({\tt BGO, EdS})$ which has in principle two contributions: one from the simulation of the EdS model and the other from the simulation of light propagation. We have already isolated the second contribution coming from {\tt BiGONLight} in the $\Lambda$CDM test. Indeed, since we use the analytical solution for $\Lambda$CDM both in the numerical and analytical computation for the observables, the error $\Delta O({\tt BGO, \Lambda CDM})$ we find in Fig.~\ref{fig:LCDM} is entirely due to the simulation of light propagation and is of the order of $10^{-22} \div 10^{-31}$.
On the other hand, in Fig.~\ref{fig:scale_ET} we see that the accuracy of the ET simulation we use is much larger, i.e. of the order of $10^{-10}$, which is of the same order of the one for $\Delta O({\tt BGO, EdS})$ in Fig.~\ref{fig:ET}. Therefore we can conclude that the final error on the observables we find in Fig.~\ref{fig:ET} is settled by the accuracy of the ET in simulating the EdS model. Let us finally remark that for this test we perform light propagation using forward integration, namely from the source $\calS$ to the observer $\calO$. We also repeated the computation of $\Delta O({\tt BGO, EdS})$ using backward integration, form $\calO$ to $\calS$, in {\tt BiGONLight} and we found the same results.
%%%%%%%%%%%%%%%%%%%%%%%%%%%%%%%%%%%%%%%%%%%%%%%%%%%%%%%%%%%%%%%%%%%%%%%
%%%%%%%%%%%%%%%%%%%%%%%%%%%%%%%%%%%%%%%%%%%%%%%%%%%%%%%%%%%%%%%%%%%%%%%
	

\endinput
