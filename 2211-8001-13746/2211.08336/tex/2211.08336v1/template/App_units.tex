\chapter{Units}
\MG{da bigonlight paper}
\label{apx:Units}

Throughout this article all quantities are expressed in geometric units, i.e. units defined by the relation $G = c = 1$, thus such that masses, time and lengths have the same unit of measurement. For instance, we can fix a unit of length\footnote{The same can be done by fixing a unit of time $T$ to define $[mass]=T \frac{c^3}{G}$ as unit of mass and $[length]=T c$ as unit of length, or fixing a unit of mass $M$ to define $[time]=M \frac{G}{c^3}$ as unit of time and $[length]=M \frac{G}{c^2}$ as unit of length.} $[length]=L$ and define $[mass]=\frac{c^2 L}{G}$ as unit of mass and $[time]=\frac{L}{c}$ as unit of time: in geometric units they all reduce to $[mass]=[time]=[length]=L$. Within this choice, every physical quantity $Q_{\rm phys}$ can be expressed as $Q_{\rm phys}=Q_{\rm comp} L^{\alpha}$, where $Q_{\rm comp}$ is dimensionless, $L$ is a arbitrary length to be chosen, and $\alpha$ is a certain exponent. This way of writing is particularly useful in numerical simulations, where all physical quantities are represented as dimensionless numbers and units are assigned when analysing the results.
Usually $L$ is fixed to be a length meaningful for the specific physical situation under consideration. For example, a common choice in numerical cosmological dynamics is to set $L$ equal to a characteristic length of the simulation (e.g. N-body and GR hydrodynamics), such as the side of the simulated box. 
In cosmology it is usually chosen to set $L$ equal to the conformal time in Mpc, i.e. $L=\eta^{\rm phys}$, thus $\eta^{\rm comp}=1$, at some special instant like e.g. at initial time or today. This choice is particularly convenient, since  conformal time is found by integrating the Friedmann equation and reads
 \begin{equation}
 %\dfrac{\dot{a}}{a}=H_0 a E(a)
 \eta =\dfrac{1}{H_0}  \mathlarger{\int}^{a}_{0} \dfrac{d \tilde{a}}{\tilde{a}^2E(\tilde{a})}\, ,
 \label{eq:UNITS_eta}
 \end{equation}
where $E(a)=\sqrt{\Omega_{\rm m_0}\left(\frac{a_0}{a}\right)^3+\Omega_{\rm \Lambda}}$ for a universe containing cold dark matter and a cosmological constant. 

For the $\Lambda$CDM model, Sec.~\ref{sec:LCDM}, and for the Szekeres model, Sec.~\ref{sec:Szekeres}, we choose $L=\eta_0$ and we normalize the scale factor to 1 today. This is the natural choice for this two cases since we studied light propagation backward in time.
By integrating Eq.~\eqref{eq:UNITS_eta} together with the normalization $a_0=1$ for the value of the today scale factor we have
\begin{equation}
 %\eta =\dfrac{ \mathlarger{\int}^{a_0}_{a} \dfrac{\tilde{a}^{-2}}{\sqrt{\tilde{a}^{-3}\Omega _{\rm m0}+ \Omega _{\Lambda }}} \, d \tilde{a} }{\mathcal{H}_0}
\eta^{\rm \Lambda CDM} = \dfrac{1}{\mathcal{H}_0  3^{\frac{1}{4}} \Omega_{\Lambda}^{\frac{1}{6}} \Omega_{\rm m_0}^{\frac{1}{3}}} F\left( \arccos\left(\frac{1+(1-\sqrt{3})\sqrt[3]{\frac{\Omega_{\Lambda}}{\Omega_{\rm m_0}}}a}{1+(1+\sqrt{3})\sqrt[3]{\frac{\Omega_{\Lambda}}{\Omega_{\rm m_0}}}a}\right);\frac{\sqrt{2+\sqrt{3}}}{2}\right)\, ,
\label{eq:UNITS_eta_LCDM}
 \end{equation}
with $F(x;y)$ being the elliptic integral of the first kind. Substituting the values of the cosmological parameters from \cite{planck2018param}, $\Omega_{\rm m_0}=0.315$ and $\Omega_{\rm \Lambda}=0.685$ and expressing the Hubble constant in Mpc, i.e. $\mathcal{H_{\rm 0}}=2.2469 \times 10^{-4}\, {\rm Mpc^{-1}}$, Eq.~\eqref{eq:UNITS_eta_LCDM} gives $L=\eta^{\rm \Lambda CDM}_0=\eta^{\rm Sz}_0=14.4152\,  \text{Gpc}$.

For the EdS model, Sec.~\ref{sec:ET}, we study light propagation forward in time and we decided to use the same conventions as the one implemented in the ET. Here, the simulation is carried out in a cubic domain volume of comoving side $2L$, which is initialised at initial time  $\eta_{\rm in}$ (and not today) and the scale factor is normalised to 1 at $\eta_{\rm in}$. Let us start by integrating \eqref{eq:UNITS_eta} for the EdS model, i.e. $\Omega_{\rm m_0}=1$ and $\Omega_{\rm \Lambda}=0$, which gives
\begin{equation}
\eta^{\rm EdS}=\dfrac{2}{\mathcal{H}_0}\sqrt{\dfrac{a}{a_0}}\, .
\label{eq:UNITS_eta_EdS}
\end{equation}
The value of $L$ is set by evaluating Eq.~\eqref{eq:UNITS_eta_EdS} at initial time and choosing $\eta^{\rm EdS}_{\rm in}=L$. Substituting $a_{\rm in}=1$ and $\sqrt{a_0}=33.2$ from the numerical simulation, and $\mathcal{H}_{0}=2.2469 \times 10^{-4}\, {\rm Mpc^{-1}}$ from \cite{planck2018param} in Eq.~\eqref{eq:UNITS_eta_EdS}, it follows that the value of $L$ is
\begin{equation}
L=268.11 \, {\rm Mpc}\, .
\end{equation}

We report for completeness the dimensions of all the main quantities in units of the characteristic length $L$
\begin{table}[ht!]
\centering
\begin{tabular}{|l|l|}
\hline
 \multicolumn{2}{|c|}{Physical quantities in units of $L$} \\
 \hline
Hubble constant & $\mathcal{H}_0^{\text{phys}}  = \mathcal{H}_0^{\text{comp}}L^{-1}$\\
Conformal time & $\eta_{\rm phys}=\eta_{\rm comp} L $\\
Spatial coordinates & $x_{\rm phys}^i=x_{\rm comp}^i L$\\
Gravitational potential & $\phi_0^{\rm phys}= \phi_0^{\rm comp} L^0$\\
Velocity field & $\nabla \phi_0^{\rm phys}= \nabla \phi_0^{\rm comp} L^{-1}$\\
Density field & $\nabla^2 \phi_0^{\rm phys}= \nabla^2 \phi_0^{\rm comp} L^{-2}$\\
Frequency & $\omega^{phys}=\omega^{comp} L^{-1}$\\
Angular diameter distance &  $D_{\rm ang}^{phys}=D_{\rm ang}^{comp} L$\\
Parallax distance &  $D_{\rm par}^{phys}=D_{\rm par}^{comp} L$\\
Redshift drift & $\zeta^{phys}=\zeta^{comp} L^{-1}$\\
Redshift $z$ & dimensionless\\
Scale factor $a(\eta)$ & dimensionless\\
Growing mode $\mathcal{D}(\eta)$ & dimensionless\\
Cosmological parameters $\Omega_i$ & dimensionless\\
\hline
\end{tabular}
\end{table}
\endinput
