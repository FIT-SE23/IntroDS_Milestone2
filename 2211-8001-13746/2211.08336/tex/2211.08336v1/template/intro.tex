\chapter{Introduction}
\label{chap:intro}
%\addcontentsline{toc}{chapter}{Introduction}

To answer the most critical questions about the origin, structure, evolution and ultimate destiny of the Universe, cosmologists gather empirical evidence and measurements to create theoretical models that explain real-world observations.  
The evidence used to formulate the current model of the Universe comes mainly from astronomical observations, namely the observation and analysis of signals emitted from distant sources and propagating at the speed of light. The nature of these light-like signals can vary, and they can reveal a range of information: for example, the electromagnetic signals from a star or galaxy can be analysed to draw conclusions about its physical properties (such as temperature, composition, rotational speed, and relative motion), the distance from us, as well as the characteristics of the space(time) through which the signal has passed. Another notable example is gravitational waves, which are direct evidence of the existence of black holes and provide a new window to peer into regions inaccessible with electromagnetic signals, \cite{abbott2016observation}. 
Advances in experimental precision or observations of new physical phenomena lead to improvements in the theoretical model that provides a deeper understanding of our Universe.

Today, this constant exchange between theoretical models and observations has led to a scientific revolution, ushering in the era of \emph{precision cosmology}, \cite{howlett2012cmb, jones2017precision}. The term ``precision'' refers to the targeted accuracy of $1\%$, which is aimed at both experimental observations and theoretical predictions: the next generation of galaxy surveys\footnote{See e.g. {\color{blue}{https://www.skatelescope.org}}, {\color{blue}{https://www.euclid-ec.org}}, {\color{blue}{https://www.lsst.org}}, {\color{blue}{http://litebird.jp/eng/}}, {\color{blue}{https://www.jpl.nasa.gov/missions/spherex}}.} will scan almost the whole extragalactic sky to unprecedented depth and resolution, producing the most detailed map of the Universe ever made. Moreover, this remarkable precision of the experimental data opens up the possibility of measuring small temporal variations in cosmological observables known as \emph{optical drift effects}, \cite{Quercellini:2010zr}. These ``real-time'' effects have the potential to provide new and crucial insights into the structure and evolution of the Universe, \cite{Quercellini:2008ty, Ding:2009xs, Quartin:2009xr,rasanen, PhysRevLett.121.021101}.
In terms of theoretical predictions, numerical simulations have made great strides in describing the formation of cosmic structures, ranging from large to small scales and accounting for relativistic effects, \cite{Giblin:2015vwq, Bentivegna:2015flc, adamek2016general, macpherson2017, Macpherson:2018akp, Barrera-Hinojosa:2020gnx}. In addition, cosmological simulations are used to study nonlinear relativistic effects in lensing observations, \cite{Giblin:2017ezj,Beutler:2020evf, Lepori:2020ifz}, and distance measurements, \cite{adamek2014distance, Adamek:2018rru, Macpherson:2021gbh}.

In this view, the fundamental problem is to describe how an observer perceives signals emitted by a distant object in any spacetime. The difficulty of such an analysis is that the observed quantities depend on the curvature of spacetime and the motion of the emitter and the observer \cite{rasanen, Korzynski:2018}. This problem is easily overcome in the new formulation of light propagation in geometric optics that we presented in \cite{Grasso:2018mei}.
All possible effects on light distortions caused by the curvature between the observer and the source are encoded in the bilocal geodesic operators, which are the fundamental quantities of our formalism. In this way, the effects on the light due to curvature and those caused by the kinematics of the observer and the source can be clearly distinguished, \cite{Grasso:2018mei}. 
Once the bilocal geodesic operators are computed, they can be combined with the source and observer motion to obtain all possible optical observables such as magnification, shear, angular distance, and the real-time observables (i.e., parallax, redshift drift and position drift). In this sense, the bilocal geodesic operators formalism provides a unified framework for studying light propagation and the calculation of optical observables, \cite{Grasso:2018mei, Grasso:2021iwq}.

\section{Main results}
\label{sec:mainRes}
This thesis is devoted to the presentation of {\tt BiGONLight}, {\tt Bi}local {\tt G}eodesic {\tt O}perators framework for {\tt N}umerical {\tt Light} propagation, the {\tt Wolfram} package\footnote{{\color{blue}{ https://github.com/MicGrasso/bigonlight.git.}}} I have developed for the study of light propagation in numerical simulations. 
The original results of this thesis were published in the two peer-reviewed articles \cite{Grasso:2021iwq, Grasso:2021zra}, and are summarised as follows:
\begin{description}
\item[The $3+1$ bilocal geodesic operators framework:] numerically generated spacetimes are evolved in full general-relativistic simulations using the ADM formalism, based on the $3+1$ splitting of the Einstein equations. To make {\tt BiGONLight} compatible with such computer-generated spacetimes, I have obtained the expressions of the parallel transport equations, the optical tidal matrix, and the geodesic deviation equations for the bilocal operators in 3+1 form. Once the geodesic connecting observer and emitter is found (as shown in \cite{Vincent:2012kn}), the above equations can be used to perform the parallel transport of a reference frame and obtain the bilocal geodesic operators along that geodesic. The computation is simplified by using the general matrix form of the optical tidal matrix and the bilocal operators projected into the semi-null frame, which I have obtained.
\item[BiGONLight:] the $3+1$ bilocal geodesic operators framework is encoded in the package as a collection of {\tt Mathematica} functions. These functions take as input the ADM quantities directly from a numerical simulation or provided by the user in analytical components to find the bilocal geodesic operators. The bilocal operators are the starting point to obtain all possible optical observables by combining them with the observer and emitter four-velocities and four-accelerations. The package leaves complete control to the user, who can choose the position of the source and the observer anywhere along the null geodesic with any four-velocities and four-accelerations. 
\item[From forward to backwards-integrated bilocal operators:] I have used the properties of the geodesic deviation equation to obtain the transformation between forward-integrated and backwards-integrated bilocal geodesic operators. Forward-integrated bilocal geodesics operators can be helpful in cosmological simulations to study the properties of light propagation on-the-fly with the simulation of spacetime.
However, observables are obtained using backwards-integrated bilocal geodesic operators since they express observations performed by the observer receiving the light emitted by a source in the past. The explicit transformations between these two methods enlarge the range of applicability of the package.
\item[Tests using three cosmological models:] the accuracy of the package is tested by computing redshift, angular diameter distance, parallax distance, and redshift drift in well-known cosmological models. Three different kinds of inputs are provided: analytical metric components of a homogeneous $\Lambda$CDM model, analytical metric components of the inhomogeneous Szekeres model (as presented in \cite{Meures:2011ke, Meures:2011gp}), numerical data of a uniform dust Universe (EdS) evolved with the {\tt Einstein Toolkit} and {\tt FLRWSolver}, \cite{loffler2012einstein, macpherson2017}.
\item[Isolating nonlinear effects of light propagation:] we present a detailed analysis of the different ways in which inhomogeneities contribute to nonlinearities in cosmological observables. In this study, I have applied {\tt BiGONLight} to compute observables calculated at different approximations in a plane-parallel inhomogeneous spacetime. The nonlinear effects are evaluated as the fractional difference between observables obtained at the three different approximations linear perturbation theory, Newtonian, and post-Newtonian approximations. The inhomogeneities are tuned by varying the model's free parameters, and their contributions to the observables are obtained by analysing the variations in the fractional differences.
\end{description}

\subsection{Structure of the thesis}
The outline of the thesis is the following: the basis of modern cosmology and the description of light propagation with the bilocal geodesic operators are described in Chapters~\ref{chap:cosmology} and~\ref{chap:BGO}, respectively. A detailed presentation of {\tt BiGONLight} and its code tests is given in Chapter~\ref{chap:bigonlight}. Chapter~\ref{chap:nonlinearities} presents the application of the package to study the nonlinear contributions to light propagation in the inhomogeneous plane-symmetric Universe. Finally, the conclusions and a detailed summary of the results of my research are addressed in Chapter~\ref{chap:conclusion}.

\section{Conventions and notations}
%\addcontentsline{toc}{section}{Conventions and notations}
This dissertation is a collection of articles, and every effort has been made to make the notation as consistent and clear as possible.
Throughout the text, we assume that the spacetime metric $g_{\mu \nu}$ has signature $(-,+,+,+)$. We also assume the Einstein summation notation $\sum_{\mu} a_{\mu} b^{\mu}\equiv a_{\mu} b^{\mu}$, with the range of the sum depending on the nature of the index $\mu$. 
Greek indices ($\alpha, \beta, ...$) run from 0 to 3, while Latin indices ($i,j, ...$) run from 1 to 3 and refer to spatial coordinates only. Latin indices ($A,B, ...$) run from 1 to 2. Tensors and bitensors expressed in a semi-null frame are denoted using boldface indices: Greek boldface indices ($\boldsymbol{\alpha}, \boldsymbol{\beta}, ...$) run from 0 to 3, Latin boldface indices ($\mathbf{a}, \mathbf{b}, ...$) run from 1 to 3 and capital Latin boldface indices ($\mathbf{A}, \mathbf{B}, ...$) run from 1 to 2. Boldface letters ($\mathbf{X}, \mathbf{Y}, ...$) are also used for vectors in quotient space $\mathcal{P}_{p}$, while objects in $\mathcal{Q}_{p}$ are denoted using square brackets ($[X],\, [Y],\, ...$). Objects defined in tangent spaces $T_{p}\mathcal{M}$ are denoted by standard letters.
%Latin tilded indices ($\tilde{a}, \tilde{b}, ...$), running from $0$ to $7$, denote indices for the components of the $8 \times 8$ BGO matrix $\mathcal{W}$. 
Overdotted quantities denote a derivative with respect to conformal time, i.e. $\dot{A}=\frac{d A}{d \eta}$. Quantities with a subscript $0$ are meant to be evaluated at present, whereas the subscript $``{\rm in}"$ indicates the initial time. Quantities with a subscript $\cal S$ (or $\cal O$) are meant to be evaluated at the source (observer) position. 
%Throughout the text, we assume $G = c = 1$. %$\calM$ be the spacetime with a Lorentzian metric $g$, of signature $(-,+,+,+)$.
A description on how to set physical units in {\tt BiGONLight} is presented in Chapter~\ref{chap:bigonlight} (appendix A in \cite{Grasso:2021iwq}).

\begin{table}
\centering
\begin{tabular}{ |p{3cm}||p{8cm}|p{3cm}|  }
 \hline
 \multicolumn{3}{|c|}{\cellcolor{gray!20}List of Acronyms} \\
 \hline
 Acronym & Signification & page\\
 \hline
\ref{acr:ADM}   & Arnowitt-Deser-Misner formalism &   \pageref{acr:ADM}\\ 
\ref{acr:BGO}   & bilocal geodesic operators &   \pageref{acr:BGO}\\
\ref{acr:CMB} & cosmic microwave background &  \pageref{acr:CMB}\\
\ref{acr:EdS}   & Einstein-de Sitter &   \pageref{acr:EdS}\\
\ref{acr:EOS}   & equation of state &   \pageref{acr:EOS}\\
\ref{acr:FLA}   & flat lightcone approximation &   \pageref{acr:FLA}\\
\ref{acr:FLRW}   & Friedmann-Lema\^{i}tre-Robertson-Walker &   \pageref{acr:FLRW}\\
\ref{acr:GDE}   & geodesic deviation equation &   \pageref{acr:GDE}\\
\ref{acr:GR}   & general relativity &   \pageref{acr:GR}\\
\ref{acr:hot}   & higher-order term &   \pageref{acr:hot}\\
\ref{acr:LCDM}   & $\Lambda$-cold dark matter &   \pageref{acr:LCDM}\\
\ref{acr:Lin}   & first-order in perturbation theory &   \pageref{acr:Lin}\\
\ref{acr:LSS}   & large-scale structure &   \pageref{acr:LSS}\\
\ref{acr:LTB}   & Lema\^{i}tre-Tolman-Bondi &   \pageref{acr:LTB}\\
\ref{acr:ODE}   & ordinary differential equations &   \pageref{acr:ODE}\\
\ref{acr:PN}   & post-Newtonian approximation &   \pageref{acr:PN}\\
\ref{acr:PT}   & cosmological perturbation theory &   \pageref{acr:PT}\\
\ref{acr:SNF}   & semi-null frame &   \pageref{acr:SNF}\\
\ref{acr:SnIa}   & type Ia supernov\ae candels &   \pageref{acr:SnIa}\\
 \hline
\end{tabular}
\end{table}


\endinput
