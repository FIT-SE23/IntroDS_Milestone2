\chapter{Application: Isolating non-linearities in light propagation in inhomogeneous spacetimes}%nonlinearities
\label{chap:nonlinearities}
\MG{intro da paper }
Upcoming galaxy surveys like Euclid, LSST, SKA and others\footnote{{\color{blue}{\tt {http://sci.esa.int/euclid/}}}, {\color{blue}{\tt {https://www.lsst.org}}}, {\color{blue}{\tt {http://skatelescope.org/}}}} mark the beginning of a new exciting era, dubbed \textit{precision cosmology}. The reason behind this name is twofold: on one side these future observations will map almost all the visible universe with the unprecedented precision of $1\%$ and on the other side cosmological modelling aim at the same precision target. 

In this view treating non-linearities, i.e. going beyond (linear) cosmological perturbation theory is of crucial importance and new approximation schemes were developed specifically or approximations used in other contexts were applied to cosmology. They include: the post-Newtonian (PN) approximation (see \cite{mater, carbone2005unified} for formulations of PN cosmology in two different gauges), the post-Friedmann approximation (see \cite{Milillo:2015cva, Rampf:2016wom} for a different approach, which adapts to cosmology the weak-field post-Minkowskian approximation and reproduces linear-order cosmological perturbation theory at their zeroth-order), the weak-field approximation\footnote{The leading order of the last two approximation schemes were shown to be equivalent for a dust universe in the Poisson gauge in \cite{kopp2014newton}, whereas \cite{mater, carbone2005unified} were constructed on purpose to include second-order perturbation theory at their PN order.} (see \cite{green2011new} for the development of the framework and \cite{Adamek:2013wja} for estimations with the use of Newtonian simulations for a plane-symmetric universe), and, more recently, a two-parameters gauge-invariant approximation (see \cite{Goldberg:2016lcq}). In addition, over the past few decades, numerical simulations have increasingly become a powerful tool in cosmology to model the growth of nonlinear structures. Since Newtonian dynamics seems to be a good approximation to describe late-time structure formation, the first generation of cosmological simulations adopted Newtonian gravity to simulate cosmological dynamics. Then, Newtonian simulations were used to feed approximate field equations coming from General Relativity (GR) as e.g. in \cite{Bruni:2013mua, Adamek:2014xba, Fidler:2017pnb}. Only recently we assist to a revolution in cosmological simulations with the birth of codes aiming at simulating fully general relativistic dynamics, \cite{loffler2012einstein, Bentivegna:2016stg, giblin2016departures, adamek2016general, macpherson2017, barrera2020gramses}: for the state of the art and the comparison among different codes, see \cite{Adamek:2020jmr}.

However, a sophisticated general relativistic (exact or approximated) description of cosmological dynamics is not the end of the story. The key point is how (much) it affects light propagation, the final aim being to characterize and (hopefully) measure nonlinear GR effects in the observables on cosmological scales or, at least, quantify their bias in observations. These studies are still in their infancy but they are addressed with several approaches most of which we briefly sketched above. A noncomprehensive list includes \cite{Thomas:2014aga, Barreira:2016wqo, Borzyszkowski:2017ayl, Sanghai:2017yyn, Giblin:2017ezj, Adamek:2018rru, Gressel:2019jxw, Lepori:2020ifz, Macpherson:2021gbh}.
Despite being too early to draw definitive conclusions, it seems that the codes that approximate GR dynamics are in agreement with Newtonian simulations for what concerns weak-lensing observables \cite{Thomas:2014aga, Lepori:2020ifz} but a modification in the statistics of the luminosity distance \cite{Adamek:2018rru} was found. In addition, the PN approximation for some models gives predictions different from $\Lambda$CDM \cite{Sanghai:2017yyn}. A bit of work is still needed to adapt to (observational) cosmology the truly GR numerical codes.

In this paper we examine the differences between linear and nonlinear light propagation. An accurate treatment of the problem would require to analyse light propagation in a realistic model of the universe. However, our aim is not to make general predictions, but rather to deeply investigate the various factors and effects on observables coming from nonlinearities. For this purpose, we decided to employ a toy-model of the universe in which light rays pass through a series of plane-symmetric perturbations around a Friedmann-Lema\^{i}tre-Robertson-Walker (FLRW) background. This model is known as plane-parallel or wall universe, and it was used in the past to study the back-reaction from the small-scale inhomogeneities \cite{Villa:2011vt, diDio2012back, adamek2014distance, Clifton:2019cep}.
We start by extending the results of \cite{Villa:2011vt} by providing the so-called Zel'dovich solution with a $\Lambda$CDM background. In this model, we compute the redshift and the angular diameter distance within three different approximation schemes: linear, Newtonian and post-Newtonian. In order to quantify and isolate nonlinear contributions, we present our results in terms of the relative differences between observables computed with these three different approximations (see Sec. \ref{sec:method} for details). We also analyse different aspects of nonlinearities, e.g. scale-dependence, non-Gaussianity, etc. Even if our modelling is very simple, we believe that this kind of analysis is representative of more general configurations. 

Besides, an important novelty of this work is that we make use of the new {\tt BiGONLight Mathematica} package to study light propagation in GR and compute observables numerically, \cite{Grasso:2021iwq}. Contrary to other software, {\tt BiGONLight} implements light propagation within the new Bilocal Geodesic Operator (BGO) framework, which is applicable to more general situations than the standard formalism and it is also suitable to construct new observables, \cite{Grasso:2018mei, Korzynski:2019oal}. This unique design makes the package adaptable to study various light propagation problems in numerical simulations.

We begin by presenting in Sec. \ref{sec:plane//} the plane-parallel toy-model as introduced in \cite{Villa:2011vt}. Then, in Sec. \ref{sec:lightprop} we briefly describe the BGO framework, pointing out to \cite{Grasso:2018mei} and \cite{Korzynski:2019oal} for further details. In Sec. \ref{sec:method}, we introduce the goals of our analysis and the method which led to the results presented in Sec. \ref{sec:results}. Finally, we address our conclusions in Sec. \ref{sec:concl}.

Notation: Greek indices ($\alpha, \beta, ...$) run from 0 to 3, while Latin indices ($i,j, ...$) run from 1 to 3 and refer to spatial coordinates only. Latin indices ($A,B, ...$) run from 1 to 2. Tensors and bitensors expressed in a seminull frame are denoted using boldface indices: Greek boldface indices ($\boldsymbol{\alpha}, \boldsymbol{\beta}, ...$) run from 0 to 3, Latin boldface indices ($\mathbf{a}, \mathbf{b}, ...$) run from 1 to 3 and capital Latin boldface indices ($\mathbf{A}, \mathbf{B}, ...$) run from 1 to 2. Latin tilded indices ($\tilde{a},\tilde{b}, ...$) run from 0 to 7 and refer to the indices of $8 \times 8$ matrices $\mathcal{W}$ and $\Omega$. A dot denotes derivative with respect to conformal time.
Quantities with a subscript 0 are meant to be evaluated at present, whereas the subscript $``{\rm in}"$ indicates the initial time. 
Similarly, we indicate with a subscript $\mathcal{O}$ ($\mathcal{S}$) quantities defined at the observer (source). An overbar indicates quantities evaluated in the $\Lambda$CDM model.
In this paper we use three different approximations and consequently three different notations: ``N'' for Newtonian, ``PN'' for post-Newtonian, ``Lin'' for first-order perturbation theory. We place these abbreviations up or down depending on convenience.

\section{The wall universe in three approximations}
\MG{paper + pippone N e PN}
\MG{aggiungere pippone Zel'dovic e come si ottiene N and PN}
\label{sec:plane//}
%%%%%%%%%%%%%%%%%%%%%%%%%%%%%%%%%%%%%%%%%%%%%%%%%%%%%%%%%%%%%%%%%%%%%%%
%%%%%%%%%%%%%%%%%%%%%%%%%%%%%%%%%%%%%%%%%%%%%%%%%%%%%%%%%%%%%%%%%%%%%%%
We consider a toy-model characterized by the choice of globally plane-parallel configurations, i.e. the case where the initial perturbation field depends on a single coordinate. The dynamics of this very simple universe consists of a collection of parallel planes that collapse along the direction of their normal to form a pancake. For the purposes of our work, we are not interested in a more realistic modeling of the Universe; rather our main aim is to estimate, isolate and compare purely nonlinear and non-Newtonian contributions in light propagation, e.g. in fundamental observables such as  redshift and angular diameter distance. 

We work in the synchronous-comoving gauge and leave to future work the gauge issue of every perturbation scheme that affects the observables as well as the estimate of the related contributions in other gauges. Despite gauge effects in the observables are known in standard cosmological perturbation theory (see Ref.~\cite{Yoo:2017svj} for a recent discussion of gauge invariance of cosmological observables up to second order), the issue is more delicate for non-standard approximations, such as those considered in this paper.

The starting point of our analysis is the results of Ref.~\cite{Villa:2011vt}: the authors started from the Newtonian background given by the well-known Zel'dovich approximation, \cite{zeldovich70a}, which, for plane-parallel perturbations in the Newtonian limit, represents an exact solution. They then obtained the exact analytical form for the PN metric, thereby providing the exact PN extension of the Zel'dovich solution. Let us remark how the Zel'dovich approximation is constructed: in its conformal version, it is an expansion around the three-dimensional spatial displacement vector of the CDM particles between the position comoving with the Hubble flow and the true position governed by perturbations. The peculiarity is the following: the solution for the displacement vector is strictly linear, as it is found from the linear Newtonian equations of motion. But all other dynamical quantities, such as the mass density, are written in terms of such a displacement vector, as if it was exact, i.e. from their nonperturbative definition. The same construction was first extended to the PN approximation of General Relativity, where the metric tensor also is a dynamical variable, in Ref.~\cite{mater} and specialized in the plane-parallel case in Ref.~\cite{Villa:2011vt}.
The Zel'dovich specific feature is evident in the form of the metric tensor \eqref{eq:metricNWT}, which is quadratic in the perturbations, and the density contrast in Eq.~\eqref{eq:density_N} for the Newtonian background and in Eq.~\eqref{eq:metricPN} and Eq.~\eqref{eq:density_PN} for the PN solution found in Ref.~\cite{Villa:2011vt}.

We provide here the $\Lambda$CDM extension of the PN metric found in Ref.~\cite{Villa:2011vt}, which was obtained for the Einstein-de Sitter background model, i.e. the dust-only universe. 
Starting from the line element
\beq
ds^2=a^2(\eta)\left\{ -c^2 d\eta^2 + \gamma^{\rm{PN}}_{11}(\eta, q_{\rm 1}) dq_{\rm 1}^2 +\gamma^{\rm{PN}}_{22}(\eta, q_{\rm 1}) dq_{\rm 2}^2+\gamma^{\rm{PN}}_{33}(\eta, q_{\rm 1}) dq_{\rm 3}^2\right\}
\eeq

we then obtain the conformal metric given by\footnote{We take this chance to point out a typo in Eq.~(4.37) of Ref.~\cite{Villa:2011vt}: in the first term of the second line of the expression for $\gamma_{11}$ the correct coefficient is $5/756$ instead of $5/576$.}
\begin{equation} \label{eq:metricPN}
\begin{split}
\gamma^{\rm{PN}}_{11} =& \left(1-\frac{2}{3}\frac{\partial_{q_1}^2 \phi_0}{ \stuff} \mathcal{D} \right)^2 +\\
 & +\frac{1}{c^2}\left[-\frac{10}{3} \phi_0 +(4 a_{\rm nl}-5)\frac{ 10}{9}\frac{(\partial_{\rm q_{\rm 1}} \phi_0)^2}{ \stuff}\mathcal{D}  +(a_{\rm nl}-1)\frac{40}{9}\frac{ \phi_0  \partial_{\rm q_{\rm 1}}^2 \phi_0}{ \stuff} \mathcal{D}  + \right.\\
& \left.  \left(\frac{41}{7}-4 a_{\rm nl}\right)\frac{20}{27} \frac{(\partial_{\rm q_{\rm 1}} \phi_0)^2  \partial_{\rm q_{\rm 1}}^2 \phi_0}{(\stuff)^2}\mathcal{D}^2 + \left(3-2 a_{\rm nl}\right) \frac{40}{27}\frac{\phi_0  (\partial_{\rm q_{\rm 1}}^2 \phi_0)^2}{ (\stuff)^2} \mathcal{D}^2 -\frac{80}{189}\frac{(\partial_{\rm q_{\rm 1}} \phi_0)^2 ( \partial_{\rm q_{\rm 1}}^2\phi_0)^2}{(\stuff)^3}\mathcal{D}^3 \right]\\
\gamma^{\rm{PN}}_{22}= & 1+\frac{1}{c^2}\left[ \frac{10}{9}\left( \frac{\mathcal{D} (\partial_{\rm q_{\rm 1}} \phi_0)^2 }{\stuff}-3\phi_0\right)\right]\\
\gamma^{\rm{PN}}_{33}=& 1+\frac{1}{c^2}\left[ \frac{10}{9}\left( \frac{\mathcal{D} (\partial_{\rm q_{\rm 1}} \phi_0)^2 }{\stuff}-3\phi_0\right)\right].
 \end{split}
\end{equation}
In the above expression $\eta$ is the conformal time, $a$ is the scale-factor encoding the evolution of the $\Lambda$CDM  background, $\mathcal{H}_0$, $\Omega_{\rm m_0}$, and $\phi_0$ are the (conformal) Hubble parameter $\mathcal{H} \equiv \dot{a}/a$, the matter (ordinary plus dark) densaity parameter, and the peculiar gravitational potential, respectively, all evaluated at present. The dot denotes differentiation with respect to conformal time. $\mathcal{D}$ is the growing mode solution of the first-order equation for the density contrast which is defined as 
\begin{equation}
\delta(\eta, q_{\rm 1})\equiv \frac{\rho(\eta, q_{\rm 1})}{\bar{\rho}(\eta)}-1,
\end{equation}
where $\bar{\rho}$ the $\Lambda$CDM  background matter density.
At first order in standard perturbation theory and without loss of generality, the space and time dependence of the expression of the growing density contrast can be factored out. In our 
one-dimensional case we have $\delta^{\rm Lin}(\eta, q_{\rm 1})=\mathcal{D}(\eta)\delta^{\rm Lin}_0(q_{\rm 1})$, where we fix the constant $\delta_0$ at the present time, and the growing mode $\mathcal{D}$ obeys the well-known equation
\begin{equation} \label{eqforD+}
\ddot{\cal D} +\mathcal{H}\dot{\cal D}-\frac{3}{2}\mathcal{H}_0^2\Omega_{m_0}\frac{\cal D}{a}=0\,.
\end{equation}
It is worth noticing that these quantities are all connected via the cosmological Poisson equation
\begin{equation}
\mathcal{D} \nabla^2 \phi_{\rm 0} -\frac{3}{2} \stuff  \delta_{\rm Lin} =0 \,.
\label{eq:Poisson_eq}
\end{equation}
Finally, we follow here the parametrization for primordial non-Gaussianity defined in Ref.~\cite{Bartolo:2005kv}: the number $a_{\rm nl}$ parametrizes local primordial non-Gaussianity of the gauge-invariant curvature perturbation of uniform density hypersurfaces. This is linked to the parametrization of the primordial gravitational potential by a simple relation between the respective parameters: $f_{\rm nl}= (5/3) (a_{\rm nl} -1)$.

The metric in~\eqref{eq:metricPN} corresponds to the most sophisticated approximation that we will use in this paper: although being referred to the 1D toy-model, it is fully nonlinear in the standard perturbative sense, i.e. it is not assumed that density perturbations are small. On the contrary, taking advantage of the Zel'dovich prescription, we calculate the density contrast nonperturbatively, see Eq.~\eqref{eq:density_PN} below.
The PN approximation extends standard perturbation theory including the leading-order corrections to the Newtonian treatment, which are the terms proportional to $1/c^2$. For the convergence of the PN expansion in the metric~\eqref{eq:metricPN}, see Ref.~\cite{Villa:2011vt}.  
We will compare light propagation in the spacetime described by~\eqref{eq:metricPN} with other two cases, that are both extended in ~\eqref{eq:metricPN}: the linear order of standard cosmological perturbation theory and the Newtonian approximation. The linear spacetime metric in the synchronous-comoving gauge is very well known and in 1D it reads
\begin{equation} \label{eq:metricIPT}
\begin{split}
\gamma^{\rm{Lin}}_{11} =& 1-\frac{4}{3}\frac{ \mathcal{D} \partial_{\rm q_{\rm 1}}^2 \phi_0}{ \stuff} -\frac{10}{3\,c^2}\phi_0\\
\gamma^{\rm{Lin}}_{22}= & 1-\frac{10}{3\,c^2}\phi_0\\
\gamma^{\rm{Lin}}_{33}=& 1-\frac{10}{3\,c^2}\phi_0.
 \end{split}
\end{equation}
This metric is the solution of the Einstein's equations expanded at first order around the FLRW background. Note however that the planar symmetry reduces the degrees of freedom to be only scalar (there are no vector or tensor mode in 1D, by construction) and confines the dynamical part in $\gamma^{\rm{Lin}}_{11}$ only, i.e. only in the direction of the perturbations, while in the other two directions we have just the (PN) initial conditions.
On the other hand, in the Newtonian approximation we have
\begin{equation} \label{eq:metricNWT} 
\begin{split}
\gamma^{\rm{N}}_{11} =& \left(1-\frac{2}{3}\frac{ \mathcal{D} \partial_{\rm q_{\rm 1}}^2 \phi_0}{ \stuff} \right)^2 \\
\gamma^{\rm{N}}_{22}= & 1\\
\gamma^{\rm{N}}_{33}=& 1.
 \end{split}
\end{equation}
This metric can be read off~\eqref{eq:metricPN} by discarding the PN corrections proportional to $1/c^2$. 

For completeness we report here the expressions of the density contrast in the three cases:
\begin{equation}
\delta_{\rm Lin}=\frac{2}{3}\frac{\mathcal{D} \partial^2_{\rm q_{\rm 1}} \phi_0}{\stuff}
\label{eq:density_lin}
\end{equation}
\begin{equation}
\delta_{\rm N}=\frac{\frac{2}{3}\frac{\mathcal{D} \partial^2_{\rm q_{\rm 1}} \phi_0}{\stuff}}{1-\frac{2}{3}\frac{\mathcal{D} \partial^2_{\rm q_{\rm 1}} \phi_0}{\stuff}} 
\label{eq:density_N}
\end{equation}

\begin{equation}
\begin{array}{l}
\delta_{\rm PN}= \frac{\frac{2}{3}\frac{\mathcal{D} \partial_{\rm q_{\rm 1}}^2 \phi_0}{\mathcal{H}_0^2 \Omega_{\rm m0}}}{\left(1-\frac{2}{3}\frac{\mathcal{D} \partial_{\rm q_{\rm 1}}^2 \phi_0}{\mathcal{H}_0^2 \Omega_{\rm m0}} \right)}+\frac{1}{c^2}\frac{1}{\left(1-\frac{2}{3}\frac{\mathcal{D} \partial_{\rm q_{\rm 1}}^2 \phi_0}{\mathcal{H}_0^2 \Omega_{\rm m0}} \right)^2}\left[\frac{5}{9}(3-4 a_{\rm nl})\frac{\mathcal{D}}{\mathcal{H}_0^2 \Omega_{\rm m0}}(\partial_{\rm {q_{\rm 1}}} \phi_{\rm 0})^2+\frac{20}{9}(2-a_{\rm nl})\frac{\mathcal{D}}{\mathcal{H}_0^2 \Omega_{\rm m0}}(\phi_{\rm 0} \partial^2_{\rm {q_{\rm 1}}} \phi_{\rm 0})\right. \\
 \left. +\frac{20}{21}\left(\frac{2}{3}\frac{\mathcal{D}}{\mathcal{H}_0^2 \Omega_{\rm m0}}\right)^2 \partial_{\rm q_{\rm 1}}^2 \phi_0 (\partial_{\rm q_{\rm 1}} \phi_0)^2 \right]
\end{array}
\label{eq:density_PN}
\end{equation}

Note that the Newtonian density contrast, according to the Zel'dovich approximation, is calculated exactly from the continuity equation in the synchronous-comoving gauge (see Eq.~\eqref{eq:delta_ex}) as the PN one, which is just expanded in powers of $1/c^2$.
\begin{figure}[h]
\includegraphics[width=\columnwidth]{figures/deltas.pdf}
\caption{ 
Density contrast at present (conformal) time $\eta_0$ in the three approximations $\delta_{Lin}$, $\delta_{N}$ and $\delta_{PN}$, as in Eqs.~\eqref{eq:density_lin},~\eqref{eq:density_N} and~\eqref{eq:density_PN} respectively. The plots are obtained setting up the potential as $\phi_{\rm 0} =\mathcal{I} \sin(\omega q_{\rm 1})$ with $\omega=\frac{2 \pi}{500 \, \rm Mpc} $ and amplitude $\mathcal{I}$ such that ${\rm max} \Big(\delta_{\rm PN}(\eta_{\rm 0}, q_{\rm 1})\Big)=0.1$. The values for $\Omega_{\rm m0}$, $\Omega_{\rm \Lambda}$, $f_{\rm nl}$ and $\mathcal{H}_{\rm 0}$ are taken from~\cite{planck2018param, planck2019anl}.}
        \label{fig:delta_profile}
\end{figure}
We take our initial conditions at $\eta_{\rm in}$, after the end of inflation and in the matter-dominated era, when linear theory around the Einstein-de Sitter model is still a good approximation. The explicit expression for the initial density contrast is thus
\begin{equation}
\delta_{\rm in}=\frac{2}{3}\frac{\mathcal{D}_{\rm in} \partial^2_{\rm q_{\rm 1}} \phi_0}{\stuff}\,
\label{eq:density_in}
\end{equation}
where $\mathcal{D}_{\rm in}\propto \eta^2_{\rm in}$ is the linear growing mode of the Einstein-de Sitter model. 
We model the profile of the gravitational potential at present as $\phi_{\rm 0} = \mathcal{I} \sin(\omega q_{\rm 1})$, with frequency $\omega=\frac{2 \pi}{500 \, \rm{Mpc}}$ and amplitude $\mathcal{I}$ such that ${\rm max} \Big(\delta_{\rm PN}(\eta_{\rm 0}, q_{\rm 1})\Big)=0.1$. We set the fiducial values of the cosmological parameters from~\cite{planck2018param,planck2019anl} with $\Omega_{\rm m0} =0.3153$, $\Omega_{\rm \Lambda}=0.6847$, $\mathcal{H}_{\rm 0}=67.36$ and $a_{\rm nl}=\frac{3}{5} f_{\rm nl}+1 = 0.46$. 
The three profiles~\eqref{eq:density_lin}, ~\eqref{eq:density_N} and ~\eqref{eq:density_PN} of the density contrast at the present time are shown in Fig.~\ref{fig:delta_profile}. The plot shows the amplitudes of the curves of the N and PN density contrast are shifted to higher values compared to the linear one, namely the N and PN corrections have the effect of increasing both the under- and the over-density peaks by the same amount $\approx4.15\times 10^{-3}$.
The differences in the evolution of the density contrast in the three approximations are more evident from Fig.~\ref{fig:density_variations}, in which we analyse the growth of an initial under-density, $\delta_{\rm in}<0$, over-density, $\delta_{\rm in}>0$, and the case of vanishing $\delta_{\rm in}$.
In Figs.~\ref{fig:delta_Lin_N} and  \ref{fig:delta_PN_N} we show the deviations $\left|\frac{\delta_{\rm Lin}-\delta_{\rm N}}{\delta_{\rm N}}\right|$ and $\left|\frac{\delta_{\rm PN}-\delta_{\rm N}}{\delta_{\rm N}}\right|$ respectively, at fixed position as a function of time for the two cases of initial over- and under-densities. 
In Fig.~\ref{fig:delta_Lin_N} we clearly see that the variation Lin vs N grows with time,
reaching $\approx 9\%$ at present, which is exactly the shift of $4.15\times 10^{-3}$ that we see in Fig.~\ref{fig:delta_profile}. The variation PN vs N grows with time as well, Fig.~\ref{fig:delta_PN_N}, but the value at present is 4 orders of magnitude less. Furthermore, in this figure we can also appreciate how the over-densities accrete faster than the under-densities, as one should expect. This is not visible in Fig.~\ref{fig:delta_Lin_N}, due to the fact that the difference between the linear and the Newtonian approximation is dominant.
%%%%%%%%%%%%%%%%%%%%%%%%%%%%%%%%%%%%%%%%%%%%%%%%%%%%%%%%%%%%%%%%%%%%%%%%%%%%%%%%%%%%%%%%%%%%%%%%%%%%%%%
\begin{figure*}[ht]
    \centering
    \begin{subfigure}{0.49\linewidth}%{0.8\columnwidth}%0.40\textwidth
        \includegraphics[width=\linewidth]{figures/delta_lin_N}
       \caption{$\left|\frac{\delta_{\rm Lin}-\delta_{\rm N}}{\delta_{\rm N}}\right|$}
        \label{fig:delta_Lin_N}
    \end{subfigure}
    \begin{subfigure}{0.49\linewidth}%{0.8\columnwidth}
        \includegraphics[width=\linewidth]{figures/delta_PN_N.pdf}
        \caption{$\left|\frac{\delta_{\rm PN}-\delta_{\rm N}}{\delta_{\rm N}}\right|$ }
        \label{fig:delta_PN_N}
    \end{subfigure}
    \\
    \begin{subfigure}{0.5\linewidth}%{0.8\columnwidth}
        \includegraphics[width=\linewidth]{figures/delta_homo.pdf}
        \caption{ Difference $\left|\delta_{\rm PN}-\delta_{\rm N}\right|$ and $\left|\delta_{\rm PN}-\delta_{\rm N}\right|$ for regions with $\delta_{\rm in} = 0$ }
        \label{fig:delta_homo}
    \end{subfigure}
    \caption{Comparison between $\delta_{\rm Lin}$, $\delta_{\rm N}$ and $\delta_{\rm PN}$ as functions of conformal time $\eta$. The comparisons  \ref{fig:delta_Lin_N} and \ref{fig:delta_PN_N} are evaluated for an initial over-density $\delta_{\rm in}>0$ and under-density $\delta_{\rm in}<0$, while \ref{fig:delta_homo} is plotted for $\delta_{\rm in}=0$. All the three approximations are obtained for $\phi_{\rm 0} =\mathcal{I} \sin(\omega q_{\rm 1})$ with $\omega=\frac{2 \pi}{500 \, \rm Mpc} $,  amplitude $\mathcal{I}$ such that ${\rm max} \Big(\delta_{\rm PN}(\eta_{\rm 0}, q_{\rm 1})\Big)=0.1$. and cosmological parameters taken from~\cite{planck2018param, planck2019anl}.}\label{fig:density_variations}
\end{figure*}
%%%%%%%%%%%%%%%%%%%%%%%%%%%%%%%%%%%%%%%%%%%%%%%%%%%%%%%%%%%%%%%%%%%%%%%%%%%%%%%%%%%%%%%%%%%%%%%%%%%%%%%
The case of $\delta_{\rm in}=0$ is presented in Fig.~\ref{fig:delta_homo}, where we plot the differences $\left|\delta_{\rm Lin}-\delta_{\rm N}\right|$ and $\left|\delta_{\rm PN}-\delta_{\rm N}\right|$. The reason why we took the difference instead of the variation (as was done in the over- and under-density cases) is to avoid the operation of dividing by zero, since we are considering regions with $\delta=0$.
We begin by noticing that for $\delta_{\rm in}=0$ both $\delta_{\rm Lin}$ and $\delta_{\rm N}$ vanish at any time, as evident from~\eqref{eq:density_lin} and~\eqref{eq:density_N}: both $\delta_{\rm Lin}$ and $\delta_{\rm N}$ are proportional to $\partial^2_{\rm q_{\rm 1}}\phi_{\rm 0}$ and they are exactly zero at the same value of $q_{\rm 1}$, in our toy model $\phi_{\rm 0}\propto \sin(\omega q_{\rm 1})$. Conversely, the term $\propto\partial_{\rm q_{\rm 1}}\phi_{\rm 0}$ in $\delta_{\rm PN}$~\eqref{eq:density_PN} implies that $\delta_{\rm PN}(\eta_{\rm in})\neq 0$ at the position where $\delta_{\rm in}=0$. Therefore, what is actually shown in Fig.~\ref{fig:delta_homo} is the evolution of the density contrast in the post-Newtonian approximation, whose absolute value increases up to $\approx 3\ \times 10^{-6}$. 
%%%%%%%%%%%%%%%%%%%%%%%%%%%%%%%%%%%%%%%%%%%%%%%%%%%%%%%%%%%%%%%%%%%%%%%%%%%%%%%%%%%%%%%%%%%%%%%%%%%%%%%

\section{Method}
\label{sec:method}
%%%%%%%%%%%%%%%%%%%%%%%%%%%%%%%%%%%%%%%%%%%%%%%%%%%%%%%%%%%%%%%%%%%%%%%
%%%%%%%%%%%%%%%%%%%%%%%%%%%%%%%%%%%%%%%%%%%%%%%%%%%%%%%%%%%%%%%%%%%%%%%
The core of our analysis is to estimate the magnitude of the nonlinear effects on light propagation, through the comparison of some cosmological observables calculated within different approximation schemes. In particular, we will compare the redshift $z$ and the angular diameter distance $D_{\rm ang}$ computed in the following three cases:
\begin{enumerate}
\item using the first-order expansion in standard cosmological perturbation theory of the plane-parallel metric \eqref{eq:metricIPT} and performing light propagation perturbatively, up to first order. We will denote as $O^{\rm Lin}$, the generic observable $O$ obtained in this way, which only includes effects linear in the perturbations; ($z^{\rm Lin}$ and $D_{\rm ang}^{\rm Lin}$ are derived in App. \ref{apx:linear_obs});
\item using the Newtonian part of the plane-parallel metric, namely the metric in Eq.~\eqref{eq:metricNWT}, and performing exact light propagation\footnote{The term ``exact'' refers to the fact that no perturbative approach is used when we derive and solve the equations describing the propagation of light and observables. In other words, even if the spacetime metric was obtained using some perturbation scheme, we use it as if it were exact for the entire procedure to calculate the observables, starting from the very beginning, i.e. the geodesic equation. We will discuss this approach further on in this section.} using numerical integration. The observables calculated in this way will be indicated as $O^{\rm N}$;
\item using the full PN plane-parallel metric \eqref{eq:metricPN} and performing exact light propagation via numerical integration. We denote the observables calculated with this method as $O^{\rm PN}$.
\end{enumerate}
The observables calculated with the last two methods, $O^{\rm N}$ and $O^{\rm PN}$, are obtained using \emph{\texttt{BiGONLight.m}} (\textbf{Bi}-local \textbf{G}eodesic \textbf{O}perators framework for \textbf{N}umerical \textbf{Light} propagation), a publicly available {\tt Mathematica} package ({\color{blue}{\tt {https://github.com/MicGrasso/bigonlight1.0}}}) developed to study light propagation in numerical simulations using the BGO framework. The package contains a collection of function definitions, including those to compute geodesics, parallel transported frames and solve the BGO's equation~\eqref{eq:GDE_for_BGO}. \texttt{BiGONLight.m} works as an independent package that, once is called by a {\tt Mathematica} notebook, can be used to compute numerically the BGO along the line of sight, given the spacetime metric, the four-velocities and accelerations of source and observer as inputs: a sample of the notebook we used for our analysis can be found in the repository folder \emph{Plane-parallel}. An exhaustive description of \texttt{BiGONLight} and several tests of the package are presented in \cite{Grasso:2021iwq}. Here we just report in App. \ref{apx:bigonlight} two case studies of code testing, the $\Lambda$CDM and the Szekeres model. 

Let us now comment about the fact that we use exact light propagation for the Newtonian and post-Newtonian observables, despite the fact that the respective spacetime metric is obtained with perturbative techniques. First, we notice that this method used to compute $O^{\rm PN}$ does not produce observables strictly of PN order: the observables $O^{\rm PN}$ will contain also some of higher than PN contributions, coming from the fact that we start from the PN metric \eqref{eq:metricPN} but we do not expand further the equations for light propagation or the expressions for the observables in powers of $1/c^2$ (we set $c=1$ everywhere). One would naively expect that the higher than PN terms are always subleading with respect to the PN ones, as in any well-defined perturbation scheme. The key point here is if this hierarchy, which starts at the level of metric perturbations, is preserved throughout the full calculation to the final results, especially in our case where the equations to compute the observables are fully non-linear. We find that this is indeed the case, as indicated in similar investigations in the literature. In order to show this explicitly and to give an estimate of the higher than PN corrections, we have compared the density contrast calculated strictly up to PN order $\delta_{\rm PN}$ \eqref{eq:density_PN} and the density contrast $\delta_{\rm ex}$ obtained from its exact expression from the continuity equation in synchronous-comoving gauge, i.e.
\begin{equation}
\delta_{\rm ex}(\eta,q_{\rm 1})= (\delta(\eta_{\rm in},q_{\rm 1})+1)\sqrt{\frac{|\gamma(\eta_{\rm in},q_{\rm 1})|}{|\gamma(\eta
,q_{\rm 1})|}}-1 \, ,
\label{eq:delta_ex}
\end{equation} 
where $|\gamma|$ is the short-hand notation for the determinant of the metric \eqref{eq:metricPN}, calculated here without expanding in powers of $1/c^2$.
%%%%%%%%%%%%%%%%%%%%%%%%%%%%%%%%%%%%%%%%%%%%%%%%%%%%%%%%%%%%%
\begin{figure}[h]
    \centering
    \begin{subfigure}[h]{0.49\textwidth}
        \includegraphics[width=1.01\textwidth]{figures/delta_PN_N.pdf}
       \caption{$\left|\frac{\delta_{\rm PN}-\delta_{\rm N}}{\delta_{\rm N}}\right|$}
    \end{subfigure}
    \!\!\! 
    \begin{subfigure}[h]{0.49\textwidth}
        \includegraphics[width=1.01\textwidth]{figures/delta_ex_PN.pdf}
        \caption{$\left|\frac{\delta_{\rm ex}-\delta_{\rm PN}}{\delta_{\rm PN}}\right|$ }
    \end{subfigure}
    \caption{Evolution of the variation $\delta_{\rm PN}$ vs $\delta_{\rm N}$ (a) and $\delta_{\rm ex}$ vs $\delta_{\rm PN}$ (b) for initial over-density $\delta_{\rm in}>0$ and under-density $\delta_{\rm in}<0$ regions for $k = 500\, \rm Mpc$. The variation $\delta_{\rm ex}$ vs $\delta_{\rm PN}$ (b) is 4 orders of magnitude smaller than the variation $\delta_{\rm PN}$ vs $\delta_{\rm N}$ (a).}\label{fig:over-under_delta_ex_VS_PN}
\end{figure}

\begin{figure}[h]
    \centering
    \begin{subfigure}[h]{0.49\textwidth}
        \includegraphics[width=1.01\textwidth]{figures/delta_homo_PN_N.pdf}
        \caption{$\left|\delta_{\rm PN}-\delta_{\rm N}\right|$ }
       
    \end{subfigure}
    \!\!\! 
    \begin{subfigure}[h]{0.49\textwidth}
        \includegraphics[width=1.01\textwidth]{figures/delta_homo_ex_PN.pdf}
        \caption{$\left|\delta_{\rm ex}-\delta_{\rm PN}\right|$ }
    \end{subfigure}
    \caption{Evolution of the variation $\delta_{\rm PN}$ vs $\delta_{\rm N}$ (a) and $\delta_{\rm ex}$ vs $\delta_{\rm PN}$ (b) for regions with $\delta_{\rm in}=0$ and $k = 500\, \rm Mpc$. The variation $\delta_{\rm ex}$ vs $\delta_{\rm PN}$ (b) is 6 orders of magnitude smaller than the variation $\delta_{\rm PN}$ vs $\delta_{\rm N}$ (a).}\label{fig:d=0_delta_ex_VS_PN}
\end{figure}
%%%%%%%%%%%%%%%%%%%%%%%%%%%%%%%%%%%%%%%%%%%%%%%%%%%%%%%%%%%%%
The plots in Fig.~\ref{fig:over-under_delta_ex_VS_PN} show that the variation between $\delta_{\rm PN}$ and $\delta_{\rm ex}$ is 4 orders smaller than the variation between $\delta_{\rm N}$ and $\delta_{\rm PN}$ for initial over- and under-dense regions and it is 6 orders smaller when we consider regions with vanishing initial density contrast, Fig.\ref{fig:d=0_delta_ex_VS_PN}. This is something we expected, since the impact of the corrections gets smaller and smaller with the increase of the order in the expansion and, more importantly, we were able to isolate and quantify the corrections coming from the higher than PN terms. This specific result holds for the density contrast but it is perfectly reasonable that this estimation is roughly valid for the observables too, even if the calculation to get them is different. We believe that the argument just presented validates our method of performing exact light propagation.

In order to compare the observables calculated within different approximations, we introduce the dimensionless variation $\Delta O$ for the generic observable $O$ calculated in the two approximations $\mathit{a}$ and $\mathit{b}$ defined as:
\begin{equation}
\Delta O (b, a)= \frac{O^{\mathit{b}}-O^{\mathit{a}}}{O^{\mathit{a}}}
\label{eq:variation}
\end{equation}
where $\mathit{a}$ and $\mathit{b}$ stand for $\rm \Lambda CDM$, $\rm Lin$, $\rm N$ or $\rm PN$, namely the $\rm \Lambda CDM$ background, the linear order in standard PT, Newtonian or post-Newtonian approximations, respectively.

Having introduced the general method we use for our analysis and we defined the key quantity for our comparisons, we have to specify the free functions and the parameters of the plane-parallel universe we are considering, of the $\rm \Lambda CDM$ background model and its perturbations. We recall that the evolution of the inhomogeneities in our model is governed by the growing mode solution $\mathcal{D}$ \eqref{eqforD+}, while the spatial part of the matter distribution is determined by the gravitational potential $\phi_{\rm 0}$, which is the only free function. We use a sinusoidal profile for the gravitational potential $\phi_{\rm 0}$ defined as:
\begin{equation}
\phi_{\rm 0}= \mathcal{I} \sin(\omega q_{\rm 1})
\label{eq:phi}
\end{equation}
where the frequency $\omega=2 \pi/k$ is determined from the scale of the inhomogeneities $k$, while the amplitude $\mathcal{I}$ is obtained from \eqref{eq:density_PN} for a certain value of the maximum of post-Newtonian density contrast today $\delta^{\rm max}_{0}$. The scale $k$ and the maximum of the density contrast $\delta^{\rm max}_{0}$ are linked by the matter power spectrum and we will repeat our analysis for different values of $(k, \delta^{\rm max}_{0})$ (this will be discussed in the Sec.~\ref{sec:results}). In Table \ref{tab:k-delta} we report the chosen  values for the scales and the corresponding maximum of the density contrast today.
\begin{table}\caption{\label{tab:k-delta}Values $(k, \, \delta^{\rm max}_{0})$ used in our analysis.}
%\begin{ruledtabular}
\begin{tabular}{lccccc}
$k\, (\rm Mpc)\, $ & $\, 500\, $ & $\, 300 \,$ & $\, 100 \,$ & $\, 50 \,$ & $\, 30 \,$\\
$\delta_{\rm 0}^{\rm max}$ & $0.1$ & $0.35$ & $1$ & $1.5$ & $1.8$\\
\end{tabular}
%\end{ruledtabular}
\end{table}
The cosmological parameters are set using the fiducial values from~\cite{planck2018param}, i.e. $\Omega_{\rm m0} =0.3153$, $\Omega_{\rm \Lambda}=0.6847$ and $\mathcal{H}_{\rm 0}=67.36$. For primordial non-Gaussianity we use the parameter $a_{\rm nl}$ introduced in \cite{Bartolo:2005kv}. It is linked to the parameter $f_{nl}$ by:
\begin{equation}
a_{\rm nl}=\frac{3}{5}f_{\rm nl}+1
\end{equation}
where $a_{\rm nl}=1$, i.e. $f_{\rm nl}=0$, correspond to the case of exact Gaussian fluctuations. The latest measurement of $f_{\rm nl}$ from the Planck collaboration \cite{planck2019anl} gives $a_{\rm nl}=0.46 \pm 3.06$ that will fix $a_{\rm nl}=0.46$ as the fiducial value for our analysis. However, since in our case we take deterministic initial conditions, $a_{\rm nl}$ merely represents an extra free parameter of our approach which tunes the post-Newtonian corrections\footnote{This is evident, since $a_{\rm nl}$ appears only in the post-Newtonian terms and some of them can be cancelled or dimmed with an appropriate choice of the $a_{\rm nl}$'s value.}. Given that $a_{\rm nl}$ has a lot of room to vary inside its confidence interval of $\pm 3.06$, we have also investigated how the comparison Newtonian vs post-Newtonian gets modified if we take different values of $a_{\rm nl}$ to calculate post-Newtonian observables $O^{\rm PN}$ (see Sec.~\ref{sec:results}).

The last things we need to specify  are  the observer and emitter positions and their kinematics. In our study we place the observer in a position with vanishing initial density contrast $\delta_{\rm in}=0$ and we will leave the analysis on how the comparison change when the observer is located in an initial overdensity or underdensity for future investigations. 
The geodesic equations and the BGO equations \eqref{eq:GDE_for_BGO} are solved giving the initial conditions at the observer position and they are integrated backwards in time up to redshift $z=10$.
The choice of analysing only sources at $z=10$ still leaves us the freedom in selecting the direction from which the light is coming. The difference between geodesics with different directions is mainly due to the way in which the geodesics cross the parallel planes with uniform density.  
To investigate this effect, we have considered two geodesics, one with direction normal to the planes and one with direction parallel to the bisect as represented in Fig. \ref{fig:directions}, considering in both cases the observer in a position with $\delta_{\rm in}=0$ and the gravitational potential \eqref{eq:phi} set such that $k=500 {\rm Mpc}$ and $\delta^{\rm max}_{\rm 0}=0.1$. For both geodesics we have analysed what are the effects of the direction on the variations Newtonian vs post-Newtonian for $\Delta z$ (Fig.~\ref{fig:delta_z_direz}) and $\Delta D_{\rm ang}$ (Fig.~\ref{fig:delta_D_direz}).
%%%%%%%%%%%%%%%%%%%%%%%%%%%%%%%%%%%%%%%%%%%%%%%%%%%%%%%%%%%%%
\begin{figure}[h]
    \centering
        \includegraphics[width=0.35\textwidth]{figures/dir.pdf}
       \caption{Graphic representation of the direction normal to the planes (blue) and the direction along the bisect (red). Two geodesics with these directions will intersect the uniform density planes with different angles. Therefore the matter distribution profiles along the geodesics are also different. }
        \label{fig:directions}
\end{figure}

\begin{figure}[!hb]
    \centering
    \begin{subfigure}[h]{0.49\textwidth}
        \includegraphics[width=1.01\textwidth]{figures/dirZ_N_PN.pdf}
       \caption{$\Delta z$}
        \label{fig:delta_z_direz}
    \end{subfigure}
    \!\!\! 
    \begin{subfigure}[h]{0.49\textwidth}
        \includegraphics[width=1.01\textwidth]{figures/dirD_N_PN.pdf}
        \caption{$\Delta D_{\rm ang}$ }
        \label{fig:delta_D_direz}
    \end{subfigure}
    \caption{$\Delta z(PN, N)$ \eqref{fig:delta_z_direz} and $\Delta D_{\rm ang}(PN, N)$ \eqref{fig:delta_D_direz} according to our definition \eqref{eq:variation} for the two geodesics with directions normal to the planes (blue lines) and parallel to the bisect (orange lines).}\label{fig:diff_directions}
\end{figure}

%%%%%%%%%%%%%%%%%%%%%%%%%%%%%%%%%%%%%%%%%%%%%%%%%%%%%%%%%%%%%
From the plots we can conclude that there are small differences in the comparison post-Newtonian vs Newtonian for geodesics with different directions. However, the change in the matter distribution on the geodesic induced by the different directions does not modify the magnitude of the variations too much, but only their shapes. In conclusion, when we consider geodesics along the normal, the effects of the non-linearities are somewhat smaller than for the geodesics along the bisect direction. Nevertheless, from now on, we will consider only geodesics directed along the bisect: this will not affect our conclusions because we will make all the comparisons using geodesics along the bisect in all the cases under study. 

For clarity, the following list summarizes the conditions we set for our analysis:
\begin{itemize}
\item If not specified, the observer $\mathcal{O}$ is placed in a position with initial vanishing density contrast $\delta_{\rm in}=0$.
\item The sources are at redshift $z=10$ and such that the observer receives the light with direction parallel to the bisect.
\item Our analysis is performed in synchronous comoving gauge implying that both emitter and observer are comoving with the cosmic flow. 
\item The primordial non-Gaussianity parameter $a_{\rm nl}$ is set using the fiducial value from Planck \cite{planck2019anl}, i.e. $a_{\rm nl}=0.46$. However, in Sec.~\ref{sec:results} we will also consider the case when $a_{\rm nl}$ is set equal to the extreme of its confidence interval.
\end{itemize}


 
%%%%%%%%%%%%%%%%%%%%%%%%%%%%%%%%%%%%%%%%%%%%%%%%%%%%%%%%%%%%%%%%%%%%%%%
%%%%%%%%%%%%%%%%%%%%%%%%%%%%%%%%%%%%%%%%%%%%%%%%%%%%%%%%%%%%%%%%%%%%%%%

\section{Results}
\label{sec:results}
%%%%%%%%%%%%%%%%%%%%%%%%%%%%%%%%%%%%%%%%%%%%%%%%%%%%%%%%%%%%%%%%%%%%%%%
%%%%%%%%%%%%%%%%%%%%%%%%%%%%%%%%%%%%%%%%%%%%%%%%%%%%%%%%%%%%%%%%%%%%%%%
In this section we present the results of our study that we plot in terms of the quantity
\begin{equation}
\Delta O (b, a)= \frac{O^{\mathit{b}}-O^{\mathit{a}}}{O^{\mathit{a}}}\, ,
\label{eq:variation_res}
\end{equation} 
where our observables $O$ are the redshift $z$ and the angular diameter distance $D_{\rm ang}$ and $a, b$ stand for the approximations used in turn.
Let us start with Figs. \ref{fig:dz_point_1} and \ref{fig:dD_point_1} in which we plot the variation between linear and Newtonian approximations, $\Delta z (Lin, N)$ and $\Delta D_{\rm ang} (Lin, N)$, and the PN corrections to the Newtonian approximation, $\Delta z (PN, N)$ and $\Delta D_{\rm ang} (PN, N)$ for three different scales, $k=30\, , 100\, , 300\, \rm Mpc$.
%%%%%%%%%%%%%%%%%%%%%%%%%%%%%%%%%%%%%%%%%%%%%%%%%%%%%%%%%%%%%%%%%%%%%%%%%%%%%%%%%%%%%%%%%%%%%%%%%%%%%%%
\begin{figure*}[ht]
    \centering
    \begin{subfigure}{0.49\linewidth}%{0.8\columnwidth}%0.40\textwidth
        \includegraphics[width=\linewidth]{figures/R1_z300.pdf}
       \caption{ k=300}
        \label{fig:dz_point_1_k300}
    \end{subfigure}
    \begin{subfigure}{0.49\linewidth}%{0.8\columnwidth}
        \includegraphics[width=\linewidth]{figures/R1_z100.pdf}
        \caption{ k=100 }
        \label{fig:dz_point_1_k100}
    \end{subfigure}
    \\
    \begin{subfigure}{0.5\linewidth}%{0.8\columnwidth}
        \includegraphics[width=\linewidth]{figures/R1_z30.pdf}
        \caption{ k=30 }
        \label{fig:dz_point_1_k30}
    \end{subfigure}
    \caption{Redshift variations, as defined in Eq.~\eqref{eq:variation_res}, Linear vs Newtonian (blue) and post-Newtonian vs Newtonian (orange) on three different scales $k=30\, , 100\, , 300\, \rm Mpc$. We see that $\Delta z (Lin, N) \sim 10^2 \, \Delta z (PN, N)$ on every scale $k$. The variable on the horizontal axis is the $\Lambda CDM$ redshift.}\label{fig:dz_point_1}
\end{figure*}
%%%%%%%%%%%%%%%%%%%%%%%%%%%%%%%%%%%%%%%%%%%%%%%%%%%%%%%%%%%%%%%%%%%%%%%%%%%%%%%%%%%%%%%%%%%%%%%%%%%%%%%
\begin{figure*}[ht]
    \centering
    \begin{subfigure}[h]{0.49\linewidth}%{0.8\columnwidth}
        \includegraphics[width=\linewidth]{figures/R1_D300.pdf}
       \caption{ k=300}
        \label{fig:dD_point_1_k300}
    \end{subfigure}
    \begin{subfigure}[h]{0.49\linewidth}%{0.8\columnwidth}
        \includegraphics[width=\linewidth]{figures/R1_D100.pdf}
        \caption{ k=100 }
        \label{fig:dD_point_1_k100}
    \end{subfigure}
    \begin{subfigure}[h]{0.49\linewidth}%{0.8\columnwidth}
        \includegraphics[width=\linewidth]{figures/R1_D30.pdf}
        \caption{ k=30 }
        \label{fig:dD_point_1_k30}
    \end{subfigure}
    \caption{Angular diameter distance variations, as defined in Eq.~\eqref{eq:variation_res}, Linear vs Newtonian (blue) and post-Newtonian vs Newtonian (orange) on three different scales $k=30\, , 100\, , 300\, \rm Mpc$. We see that $\Delta D_{ \rm ang} (Lin, N) \sim \Delta D_{\rm ang} (PN, N)$ on every scale $k$. The variable on the horizontal axis is the $\Lambda CDM$ redshift.}\label{fig:dD_point_1}
\end{figure*}
%%%%%%%%%%%%%%%%%%%%%%%%%%%%%%%%%%%%%%%%%%%%%%%%%%%%%%%%%%%%%%%%%%%%%%%%%%%%%%%%%%%%%%%%%%%%%%%%%%%%%%%
The main result here is that the variations behave differently for the redshift and for the angular diameter distance. Indeed, while for $z$ the post-Newtonian corrections are two orders of magnitudes smaller than the nonlinear Newtonian contributions with respect to linear theory, for $D_{\rm ang}$ the two corrections are of the same order. This can be clearly seen on $k=300 \, \rm Mpc$, Figs. \ref{fig:dz_point_1_k300} and \ref{fig:dD_point_1_k300}, and the same behavior also holds on smaller scales, Figs. \ref{fig:dz_point_1_k100} - \ref{fig:dz_point_1_k30} and \ref{fig:dD_point_1_k100} - \ref{fig:dD_point_1_k30}.
For $z\lesssim 2$ we have that $\Delta z(\rm PN, N)\sim 10^{-6}$ on $k=300\, \rm Mpc$ with oscillation dumped as the redshift increases. On the other hand, for the angular diameter distance $\Delta D_{ \rm ang} (Lin, N) \sim \Delta D_{ \rm ang} (PN, N) \sim 10^{-4}$ on $k=300\, \rm Mpc$ in the full redshift range $[0, \, 10]$. 

We dedicate a separate study, reported in Figs. \ref{fig:dz_point_2} and \ref{fig:dD_point_2}, to the change of the amplitude of all the variations with the scale. We consider inhomogeneities scales of $k= \, 500, \, 300, \, 100, \, 50,\, 30\, \rm Mpc$.
%%%%%%%%%%%%%%%%%%%%%%%%%%%%%%%%%%%%%%%%%%%%%%%%%%%%%%%%%%%%%%%%%%%%%%%%%%%%%%%%%%%%%%%%%%%%%%%%%%%%%%%
\begin{figure}[!ht]
    \centering
    \begin{subfigure}[h]{0.49\textwidth}
        \includegraphics[width=\textwidth]{figures/R2_zLN.pdf}
       \caption{$\Delta z (\rm Lin, N)$ }
        \label{fig:dz_LvsN_point2}
    \end{subfigure}
    \begin{subfigure}[h]{0.49\textwidth}
        \includegraphics[width=\textwidth]{figures/R2_zPNN.pdf}
        \caption{$\Delta z (\rm PN, N)$}
        \label{fig:dz_PNvsN_point2}
    \end{subfigure}
    \caption{Variations $\Delta z (\rm Lin, N)$ and $\Delta z (\rm PN, N)$, as defined in Eq.~\eqref{eq:variation_res}, on different scales in the range $[30, \, 500] \, \rm Mpc$. Both the variations show a maximum around $k=100 \, \rm Mpc$. The variable on the horizontal axis is the $\Lambda CDM$ redshift.}\label{fig:dz_point_2}
\end{figure}
%%%%%%%%%%%%%%%%%%%%%%%%%%%%%%%%%%%%%%%%%%%%%%%%%%%%%%%%%%%%%%%%%%%%%%%%%%%%%%%%%%%%%%%%%%%%%%%%%%%%%%%
\begin{figure}[!ht]
    \centering
    \begin{subfigure}[h]{0.49\textwidth}
        \includegraphics[width=\textwidth]{figures/R2_DLN.pdf}
       \caption{$\Delta D_{\rm ang} (\rm Lin, N)$}
        \label{fig:dD_LvsN_point2}
    \end{subfigure}
    \begin{subfigure}[h]{0.49\textwidth}
        \includegraphics[width=\textwidth]{figures/R2_DPNN.pdf}
        \caption{$\Delta D_{\rm ang} (\rm PN, N)$}
        \label{fig:dD_PNvsN_point2}
    \end{subfigure}
    \caption{Variations $\Delta D_{\rm ang} (\rm Lin, N)$ and $\Delta D_{\rm ang} (\rm PN, N)$, as defined in Eq.~\eqref{eq:variation_res}, on different scales in the range $[30, \, 500] \, \rm Mpc$. The amplitudes monotonically decrease as the scale $k$ becomes smaller. The variable on the horizontal axis is the $\Lambda CDM$ redshift.}\label{fig:dD_point_2}
\end{figure}
%%%%%%%%%%%%%%%%%%%%%%%%%%%%%%%%%%%%%%%%%%%%%%%%%%%%%%%%%%%%%%%%%%%%%%%%%%%%%%%%%%%%%%%%%%%%%%%%%%%%%%%
Fig. \ref{fig:dz_point_2} shows that both the variations $\Delta z(\rm Lin, \, N)$ and $\Delta z(\rm PN, \, N)$ increase from $k=500\, \rm Mpc$ to reach the maximum amplitude on $k=100 \, \rm Mpc$ and then decreases down to $k=30\, \rm Mpc$. In terms of amplitudes we have: $\Delta z(\rm Lin, \, N) \sim 10^{-4}$, with a maximum $\sim 10^{-3}$ around $k=100\, \rm Mpc$ and $\Delta z(\rm PN, \, N) \sim 10^{-7}$, with a maximum $\sim 10^{-6}$ around $k=100\, \rm Mpc$. We again note that the variations for the redshift are damped as $z$ increases. This is most evident for $\Delta z(\rm Lin, \, N)$.
The angular diameter distance shows in Fig. \ref{fig:dD_point_2} a different behavior: the amplitude of both variations $\Delta D_{\rm ang} (\rm Lin, N)$ and $\Delta D_{\rm ang} (\rm PN, N)$ decreases monotonically as the scale $k$ become smaller. Both the amplitudes start from $\Delta D_{\rm ang} \sim 10^{-4}$ on $k=500 \, \rm Mpc$ and decrease to $10^{-6}$ on $k=30 \, \rm Mpc$.

%%%%%%%%%%%%%%%%%%%%%%%%%%%%%%%%%%%%%%%%%%%%%%%%%%%%%%%%%%%%%%%%%%%%%%%%%%%%%%%%%%%%%%%%%%%%%%%%%%%%%%%
\begin{figure}[!ht]
    \centering
     \begin{subfigure}[h]{0.49\textwidth}
        \includegraphics[width=\textwidth]{figures/R3_zda.pdf}
        \caption{$\Delta z^{\rm PN} (a_{\rm nl_{\rm 1}},\, a_{\rm nl_{\rm 2}})$}
        \label{fig:zPN_a1_vs_a2}
    \end{subfigure}
   \begin{subfigure}[h]{0.49\textwidth}
        \includegraphics[width=\textwidth]{figures/R3_zaPNN.pdf}
       \caption{$\Delta z (\rm PN, \, N)$}
        \label{fig:dz_PNvsN_for_anl}
    \end{subfigure}
    \caption{The effect of varying primordial non-Gaussianity for the redshift: the variation in Eq.~\eqref{eq:DeltaO_anls},  \eqref{fig:zPN_a1_vs_a2}, and PN correction for different values of $a_{\rm nl}$, \eqref{fig:dz_PNvsN_for_anl}. We find that $\Delta z^{\rm PN} (a_{\rm nl_{\rm 1}}, a_{\rm nl_{\rm 2}}) \lesssim  \, \Delta z (PN, N)$. The variable on the horizontal axis is the $\Lambda CDM$ redshift.}\label{fig:dz_anl}
\end{figure}
%%%%%%%%%%%%%%%%%%%%%%%%%%%%%%%%%%%%%%%%%%%%%%%%%%%%%%%%%%%%%%%%%%%%%%%%%%%%%%%%%%%%%%%%%%%%%%%%%%%%%%%
\begin{figure}[!ht]
    \centering
    \begin{subfigure}[h]{0.49\textwidth}
        \includegraphics[width=\textwidth]{figures/R3_Dda.pdf}
        \caption{ $\Delta D_{\rm ang}^{\rm PN} (a_{\rm nl_{\rm 1}},\, a_{\rm nl_{\rm 2}})$}
        \label{fig:DPN_a1_vs_a2}
    \end{subfigure}
    \begin{subfigure}[h]{0.49\textwidth}
        \includegraphics[width=\textwidth]{figures/R3_DaPNN.pdf}
       \caption{$\Delta D_{\rm ang}(\rm PN, \, N)$}
        \label{fig:dD_PNvsN_for_anl}
    \end{subfigure}
    \caption{The effect of varying primordial non-Gaussianity for the angular diameter distance: the variation in Eq.~\eqref{eq:DeltaO_anls},  \eqref{fig:DPN_a1_vs_a2},  and PN correction for different values of $a_{\rm nl}$, \eqref{fig:dD_PNvsN_for_anl}. We find that $\Delta D_{\rm ang}^{\rm PN} (a_{\rm nl_{\rm 1}}, a_{\rm nl_{\rm 2}}) \sim 10^{-4}  \, \Delta D_{\rm ang} (PN, N)$. The variable on the horizontal axis is the $\Lambda CDM$ redshift.}\label{fig:dD_anl}
\end{figure}
%%%%%%%%%%%%%%%%%%%%%%%%%%%%%%%%%%%%%%%%%%%%%%%%%%%%%%%%%%%%%%%%%%%%%%%%%%%%%%%%%%%%%%%%%%%%%%%%%%%%%%%
As we mentioned in Sec. \ref{sec:method}, different values of the primordial non-Gaussianity parameter $a_{\rm nl}$ tune some of the post-Newtonian terms in \eqref{eq:metricPN}, e.g. a perfect Gaussian initial perturbation ($a_{\rm nl}=1$) cancels out the third term in the PN part of the metric \eqref{eq:metricPN}. We then decided to quantify how the PN observables change when we vary the values of $a_{\rm nl}$ inside the confidence interval measured by Planck, \cite{planck2019anl}. For this analysis, we choose $a_{\rm nl}= 1, 0.46, -2.6, 3.52$, corresponding to Gaussian perturbations ($a_{\rm nl}= 1$), Planck 2018 fiducial value ($a_{\rm nl}= 0.46$), and extremes of confidence interval ($a_{\rm nl}= -2.6, 3.52$). We start the discussion of our results by looking at Figs. \ref{fig:zPN_a1_vs_a2} and \ref{fig:DPN_a1_vs_a2}, in which we plot for the PN observables the quantity
\begin{equation}
\Delta O^{\rm PN} (a_{\rm nl_{\rm 1}}, a_{\rm nl_{\rm 2}})= \frac{O^{\rm{PN}}_{a_{\rm nl_{\rm 1}}}-O^{\rm{PN}}_{a_{\rm nl_{\rm 2}}}}{O^{\rm{PN}}_{a_{\rm nl_{\rm 2}}}}\, ,
\label{eq:DeltaO_anls}
\end{equation} 
where we fix $a_{\rm nl_{\rm 2}}$ to the Planck best-fit value and we vary $a_{\rm nl_{\rm 1}}$.
The effect is different for the redshift and for the angular diameter distance: the variation in $D_{\rm ang}^{\rm PN}$ is $\sim 10^{-9}$, two orders of magnitude smaller than the one in $z^{\rm PN}$. This very difference is evident when we plot the PN corrections for different values of $a_{\rm nl}$, see Figs. \ref{fig:dz_PNvsN_for_anl} and \ref{fig:dD_PNvsN_for_anl}. The effect of tuning primordial non-Gaussianity is roughly of the same order as the PN correction for the redshift and also changes its shape. On the contrary, $D_{\rm ang}$ is completely insensitive to the variation of the non-Gaussianity parameter, since $\Delta D_{\rm ang}^{\rm PN} (a_{\rm nl_{\rm 1}}, a_{\rm nl_{\rm 2}}) \sim 10^{-4}  \, \Delta D_{\rm ang} (PN, N)$.

To conclude our analysis, we isolate and quantify the contribution of the linear PN initial seed proportional to the gravitational potential, i.e. $\gamma_{i j}= - \frac{10}{3 c^2}\phi_{\rm 0} \delta_{i j}$ in the spacetime metric \eqref{eq:metricPN}. To do so, we define an hybrid spacetime metric, labelled with $\tilde{N}$, by adding the initial PN seed to the Newtonian metric~\eqref{eq:metricNWT}, i.e.
\begin{equation} \label{eq:metricNt} 
\begin{split}
\gamma^{\tilde{\rm{N}}}_{11} =& \left(1-\frac{2}{3}\frac{ \mathcal{D} \partial_{\rm q_{\rm 1}}^2 \phi_0}{ \stuff} \right)^2 - \frac{10}{3 c^2}\phi_{\rm 0} \\
\gamma^{\tilde{\rm{N}}}_{22}= & 1 - \frac{10}{3 c^2}\phi_{\rm 0}\\
\gamma^{\tilde{\rm{N}}}_{33}=& 1 - \frac{10}{3 c^2}\phi_{\rm 0}\, ,
 \end{split}
\end{equation}
and we have compared the angular diameter distance computed from the two have compared the angular diameter distance computed from the two approximations in Eq.~\eqref{eq:metricNWT} and in Eq.~\eqref{eq:metricNt}.
%%%%%%%%%%%%%%%%%%%%%%%%%%%%%%%%%%%%%%%%%%%%%%%%%%%%%%%%%%%%%%%%%%%%%%%%%%%%%%%%%%%%%%%%%%%%%%%%%%%%%%%
\begin{figure}[!ht]
    \centering
     \begin{subfigure}[h]{0.49\textwidth}
        \includegraphics[width=\textwidth]{figures/R4_DPNrN.pdf}
        \caption{Comparison PN vs N (blue) and PN vs $\tilde{N}$ (orange) for the angular diameter distance, as defined in Eq.~\eqref{eq:variation_res}, on $ k= 300 \, \rm Mpc$. The variation $\Delta D_{ \rm ang} (PN, \tilde{N}) \sim \, 10^{-6}$ is two orders of magnitude smaller than $\Delta D_{\rm ang} (PN, N)$. The variable on the horizontal axis is the $\Lambda CDM$ redshift.}
        \label{fig:dD_PNvsRN_k=300}
    \end{subfigure}
    \\
    \begin{subfigure}[h]{0.49\textwidth}
        \includegraphics[width=\textwidth]{figures/R4_Dlcdm.pdf}
       \caption{Comparison between the different approximations $\rm Lin$, N, $\tilde{N}$, PN and the $\Lambda CDM$ background for the angular diameter distance.}
        \label{fig:dD_PPvsLCDM_withRN_k=300}
    \end{subfigure}
    \caption{Results for the contribution of the initial seeds to the angular diameter distance. For the definition of $\tilde{N}$ see Eq. \eqref{eq:metricNt}.}\label{fig:point_3_LCDM}
\end{figure}
%%%%%%%%%%%%%%%%%%%%%%%%%%%%%%%%%%%%%%%%%%%%%%%%%%%%%%%%%%%%%%%%%%%%%%%%%%%%%%%%%%%%%%%%%%%%%%%%%%%%%%%
The inclusion of the initial seed in the modified Newtonian model is such that the PN variation is reduced by two orders of magnitude, see Fig. \ref{fig:dD_PNvsRN_k=300}. In other words the initial seed is the leading order of the post-Newtonian correction. The effect is even more evident when we consider the variations of each of the approximation $\rm Lin$, N, $\tilde{N}$, PN respect to the $\Lambda CDM$ background: we can clearly distinguish between the two approximations, observing that $\tilde{N}$ behaves as expected very close to the PN approximation.

\endinput
%%%%%%%%%%%%%%%%%%%%%%%%%%%%%%%%%%%%%%%%%%%%%%%%%%%%%%%%%%%%%%%%%%%%%%%%%%%%%%%%%%%%%%%%%%%%%%%%%%%%%%%
