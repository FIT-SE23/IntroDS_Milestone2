\section*{Abstract}    
\label{sec:abstract}
\addcontentsline{toc}{chapter}{Abstract}

%corrected by Mikolaj
With the advent of precision cosmology, our theoretical predictions must aspire to the same level of precision as achieved by experimental probes. In this context, numerical simulations including general relativistic effects represent the state-of-the-art method to describe the formation of structures. However, aside from a detailed description of the dynamics, it is necessary to have an equally accurate explanation of the effects of such structures on light propagation and modelling their impacts on measurable quantities. 

The investigation of relativistic effects in the most general way requires a unified treatment of light propagation in cosmology. This goal can be achieved with the new interpretation of the geodesic deviation equation in terms of the bilocal geodesic operators (BGO). The BGO formalism extends the standard formulation, providing a unified framework to describe all possible optical phenomena due to the interaction between light and spacetime curvature. 
	
In my dissertation, I present {\tt BiGONLight}, a {\tt Mathematica} package that applies the BGO formalism to study light propagation in numerical relativity. The package encodes the 3+1 bilocal geodesic operators framework as a collection of {\tt Mathematica} functions. The inputs are the spacetime metric plus the kinematics of the observer and the source in the form of the 3+1 quantities, which may come directly from a numerical simulation or can be provided by the user as analytical components. These data are then used for ray tracing and computing the BGO's in a completely general way, i.e. without relying on symmetries or specific coordinate choices. The primary purpose of the package is the computation of optical observables in arbitrary spacetimes. The uniform theoretical framework of the BGO formalism allows for the extraction of multiple observables within a single computation, while the {\tt Wolfram} language provides a flexible computational framework that makes the package highly adaptable to perform both numerical and analytical studies of light propagation. {\tt BiGONLight} is tested by computing the redshift, angular diameter distance, parallax distance, and redshift drift in well-known cosmological models. We use three different inputs for the metric: two analytical metrics, the homogeneous $\Lambda$CDM model and the inhomogeneous Szekeres model, and 3+1 quantities from a simulated dust Universe. The tests show an excellent agreement with known results.

The characteristics of {\tt BiGONLight} make it a suitable tool for studying the impact of inhomogeneities on light propagation. We investigate various sources of nonlinear general relativistic effects on light propagation induced by inhomogeneous cosmic structures. {\tt BiGONLight} is used to calculate observables computed at different approximations in a plane-parallel inhomogeneous spacetime. The nonlinear effects are evaluated as the fractional difference between the observables obtained at the three different approximations: linear perturbation theory, Newtonian, and post-Newtonian approximations. The inhomogeneities are tuned by varying the model’s free parameters, and their contributions to the observables are obtained by analysing the variations in the fractional differences. Using this method we estimate the Newtonian and post-Newtonian corrections to the linear observables and analyse how these corrections change as we vary the size and magnitude of the inhomogeneities. We also explain the role of the linear initial seed as the dominant post-Newtonian contribution and show that the remaining post-Newtonian nonlinear corrections are less than $1\%$, which is consistent with previous results in the literature.






\endinput

