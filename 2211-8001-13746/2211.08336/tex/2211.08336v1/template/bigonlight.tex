\chapter{The BiGONLight package}%bigonlight
\label{chap:bigonlight}
The BGO framework in Sec.~\ref{sec:BGO} offers a uniform theoretical approach to describe various optical phenomena due to light propagation in a curved spacetime. The only limits to the formalism are imposed by those of the GDE, which are valid for most of the observations in astronomy. Conversely, no assumptions are made on the details of the spacetime geometry, making the BGO the ideal machinery to study general and nonperturbative relations between gravitational lensing, parallax effects and drift effects all within the same framework. 
The most straightforward application of the BGO formalism is the computation of observables in astrophysics and cosmology, since the observables are easily given as combination of BGO and the source and the observer four-velocities and four-accelerations. This method is particularly useful in numerical relativity, when it is applied to extract observables from numerical relativistic simulations of the spacetime. In fact, it just requires of solving a number of linear ODE's using geometric data collected along a null geodesic. Numerical relativistic simulations are currently becoming one of the most important tools of theoretical cosmology, able to reproduce  ``virtual copies'' of the universe on which perform tests and measurement forecasts \cite{Bentivegna:2012ei, Yoo:2013yea, Bentivegna:2013jta, Yoo:2014boa, Bentivegna:2015flc, Giblin:2015vwq, Mertens:2015ttp, Adamek:2015eda, Adamek:2016zes, Macpherson:2016ict, Macpherson:2018akp}. In this prospective, the use of drift effects to constraint cosmological models has recently been considered by many authors \cite{quercellini,Quercellini:2008ty,Fontanini:2009qq,Quercellini:2009ni,Krasinski:2012ty, Krasinski:2011iw, Krasinski:2012nw, Krasinski:2010rc,rasanen,  KrasiNSki:2015nta}. However, while the null geodesic tracking is a fairly standard problem in numerical relativity \cite{Vincent:2012kn, Bentivegna:2016fls}, extracting the drift effects is not. It is, of course, possible to do it without the BGO, by doing ray tracing of a number of geodesics connecting observer and emitter at consecutive moments, but this procedure is rather cumbersome.

%%%%%%%%%%%%%%%%%%%%%%%%%%%%%%%%%%%%%%%%%%%%%%%%%%
In this chapter we present {\tt BiGONLight}\footnote{{\color{blue}{\tt {https://github.com/MicGrasso/bigonlight.git}}}.}, \textbf{Bi}local \textbf{G}eodesic \textbf{O}perators framework for \textbf{N}umerical \textbf{Light} propagation, a \texttt{Mathematica} package developed for extracting observables from numerically generated spacetimes using the BGO formalism. The principal aim of the package is to provide a unified procedure to calculate multiple observables in numerical relativity: this is guaranteed since, once that the BGO are determined from the output metric of a numerical simulation, a all set of observables are obtained within the same computation. In order to be compatible with the majority of the codes in Numerical Cosmology, {\tt BiGONLight} encodes the BGO formalism in $3+1$ form and it uses the \texttt{Mathematica} powerful symbolic algebra manipulation and precision control options. The concepts presented in this chapter are published in \cite{Grasso:2021iwq}.
%The paper is organized as follows: in Sec.~\ref{sec:light}, we give the fundamentals of light propagation and its formulation in terms of Bi-local Geodesic Operators. In Sec.~\ref{sec:3+1} we  present {\tt BiGONLight} and the equations to compute the BGO in $3+1$ form encoded into the package. The recipe to compute observables in numerical relativity using {\tt BiGONLight} and the expressions of the observables in terms of BGO are given in Sec.~\ref{sec:obs}. The last section of this paper, Sec.~\ref{sec:test}, is dedicated to the code tests, which are performed in the following three cosmological models: $\Lambda$CDM (see Sec.~\ref{sec:LCDM}), Szekeres (see Sec.~\ref{sec:Szekeres}), and a numerically evolved dust universe (see Sec.~\ref{sec:ET}). We draw our conclusions in Sec.~\ref{sec:concl}.\\

%\section{State of the art on light propagation in numerical relativity}

%The increasing amount of observational data we had in the last decades has triggered significant progress in the field of cosmology. The theoretical framework commonly used to interpret these data is the $\Lambda$CDM model, which assumes that the large scale geometry of the Universe is described by the spatially homogeneous and isotropic Friedmann-Lema\^{i}tre-Robertson-Walker (FLRW) model, with galaxies and clusters of galaxies producing just small metric perturbations. The dynamics of the FLRW model is governed by the Friedmann equations with an equation of state sourced mainly by two unknown substances: the dark energy, a mysterious energy responsible for the accelerated expansion of the universe, and the (cold) dark matter, an exotic type of matter interacting with radiation and ordinary matter only through gravity. \MG{If from one side the dynamics of the universe as a whole is described by the Friedmann equations, on the other side we still need to describe the formation of structures like galaxies and clusters of galaxies which are visible on smaller scales.} The current large scale structure of the universe is evolved from the primordial small fluctuations, which are visible in the CMB. These perturbations acted like initial seeds from which, under the influence of gravity, grew all other structures like galaxies and cluster of galaxies.
%The evolution of these cosmic seeds is described by general relativity and several methods have been developed to do that. \MG{At early times, these small inhomogeneities are well described by linear perturbation theory, i.e. their dynamics is described by the solution of the Einstein equation at linear order. However, GR is a nonlinear theory implying that nonlinear effects in the inhomogeneities dynamics may be relevant in the structure formation.}
%The role of the nonlinear General-Relativistic effects from these inhomogeneities has been discussed by many authors \cite{}, with particular emphasis in their contribute to the accelerated expansion of the universe.

%Therefore, treating non-linearities in the cosmological structure dynamics is of crucial importance and several approximation schemes were developed specifically or approximations used in other contexts were applied to cosmology. (PLACE A BRIEF EXPLANATION ON WHAT THEY DO) They include: the post-Newtonian (PN) approximation (see \cite{mater, carbone2005unified} for formulations of PN cosmology in two different gauges), the post-Friedmann approximation (see \cite{Milillo:2015cva, Rampf:2016wom} for a different approach, which adapts to cosmology the weak-field post-Minkowskian approximation and reproduces linear-order cosmological perturbation theory at their zeroth-order), the weak-field approximation\footnote{The leading order of the last two approximation schemes were shown to be equivalent for a dust universe in the Poisson gauge in \cite{kopp2014newton}, whereas \cite{mater, carbone2005unified} were constructed on purpose to include second-order perturbation theory at their PN order.} (see \cite{green2011new} for the development of the framework and \cite{Adamek:2013wja} for estimations with the use of Newtonian simulations for a plane-symmetric universe), and, more recently, a two-parameters gauge-invariant approximation (see \cite{Goldberg:2016lcq}). 
%In addition, over the past few decades, numerical simulations have increasingly become a powerful tool in cosmology to model the growth of nonlinear structures. Since Newtonian dynamics seems to be a good approximation to describe late-time structure formation, the first generation of cosmological simulations adopted Newtonian gravity to simulate cosmological dynamics. Then, Newtonian simulations were used to feed approximate field equations coming from General Relativity (GR) as e.g. in \cite{Bruni:2013mua, Adamek:2014xba, Fidler:2017pnb}. Only recently we assist to a revolution in cosmological simulations with the birth of codes aiming at simulating fully general relativistic dynamics, \cite{loffler2012einstein, Bentivegna:2016stg, mertens2016integration, adamek2016general, macpherson2017, barrera2020gramses}: for the state of the art and the comparison among different codes, see \cite{Adamek:2020jmr}.

%%However, a sophisticated general relativistic (exact or approximated) description of cosmological dynamics is not the end of the story. The key point is how (much) it affects light propagation, the final aim being to characterize and (hopefully) measure nonlinear GR effects in the observables on cosmological scales or, at least, quantify their bias in observations. These studies are still in their infancy but they are addressed with several approaches most of which we briefly sketched above. A noncomprehensive list includes \cite{Thomas:2014aga, Barreira:2016wqo, Borzyszkowski:2017ayl, Sanghai:2017yyn, Giblin:2017ezj, Adamek:2018rru, Gressel:2019jxw, Lepori:2020ifz, Macpherson:2021gbh}.
%%Despite being too early to draw definitive conclusions, it seems that the codes that approximate GR dynamics are in agreement with Newtonian simulations for what concerns weak-lensing observables \cite{Thomas:2014aga, Lepori:2020ifz} but a modification in the statistics of the luminosity distance \cite{Adamek:2018rru} was found. In addition, the PN approximation for some models gives predictions different from $\Lambda$CDM \cite{Sanghai:2017yyn}. A bit of work is still needed to adapt to (observational) cosmology the truly GR numerical codes. 

%\MG{The simulations represent a virtual laboratory where recreate copies of the universe, which are then used to perform tests and comparison with real observations. On this point it is fundamental the analysis of optical phenomena like gravitational lensing and the forecast of observables like e.g. the luminosity distance to test the distance-redshift relation, main indication of dark energy. }
%On the other hand relativistic effects in the non-linear regime have recently become to be investigated with relativistic simulations in galaxy clustering and lensing observables, and the Hubble diagram, see e.g. \cite{Borzyszkowski:2017ayl, Zhu:2017jfl, Giblin:2017ezj, Breton:2018wzk, Adamek:2018rru, Beutler:2020evf, Lepori:2020ifz, Guandalin:2020snp, Lepori:2021lck} and refs. therein. \MG{Add that light propagation and observables are computed using perturbative schemes (refer to papers where D lum is computed at first and second order) whose expression depends on the geometrical characteristics of the model and on the perturbation scheme  (and order) used!}

%%%%%%QUI
%%\MG{Several ways to include non-linearities: PT, PN, other perturbative schemes where inhomogeneites are considered small statistical fluctuations over a homogeneous background. On the other hand, numerical method can be used to solve einstein equation as an evolutionary problem starting from appropriate initial conditions. Firstly used in strong field regime to describe binary systems, GW emission, BH accretion, it was also used in cosmology to describe an ensamble of galaxies moving with newtonian dynamics over a homogeneous background. Take from nonlinearities paper! + ``In order to make the most from this revolution in observational astronomy and cosmology, the same accuracy is required in the theoretical predictions and interpretations of what we measure. This tough task requires much effort from two sides. At one hand the most recent progress on cosmological dynamics are represented by  general-relativistic simulations of cosmic structures with no assumed symmetries employing exact solutions of Einstein equations, \cite{loffler2012einstein,Giblin:2015vwq,Bentivegna:2016stg,macpherson2017,Clesse:2017jjp,East:2017qmk,Daverio:2019gql}, or approximated treatments, \cite{adamek2016general,barrera2020gramses}. The common ambitious aim is to have a description valid from large to small scales which accounts for relativistic effects. '' }. 



%\MG{In this panorama the new approach of BGO \cite{Grasso:2018mei} represents the perfect machinery to study light propagation in numerical relativity, providing a unified framework to extract multiple observables within the same calculation and this is easily adaptable to perform light propagation on-the-fly with a simulation of relativistic dynamics. }
%For this purpose, we develop {\tt BiGONLight} (\textbf{Bi}-local \textbf{G}eodesic \textbf{O}perator framework for \textbf{N}umerical \textbf{Light} propagation), a {\tt Mathematica} package which simulates light propagation in Numerical Relativity within the BGO formalism, described in Sec.~\ref{sec:BGO}. It is publicly available at the repository {\color{blue}{\tt {https://github.com/MicGrasso/bigonlight.git}}} and it works as an external library that can be called inside a {\tt Mathamatica} notebook.  %\texttt{Mathematica} provides a large variety of numerical methods that can be used and customized to adapt them to the particular problem, as well as several precision control options, which allows the user to set the  precision and accuracy of numerical result through the commands \texttt{WorkingPrecision}, \texttt{SetPrecision} and \texttt{SetAccuracy}.
%%  This is exploited in the functions {\tt SolveGeodesic[]}, {\tt SolveEnergy[]}, {\tt PTransportedFrame[]} and {\tt SolveBGO[]} in which the user can choose the numerical methods used to solve the system of ODE\footnote{The user can choose between three different methods: $``RK''$ a $4^{th}$ order Runge-Kutta method, $``A''$ which lets {\tt Mathematica} to decide what is the best method to use, and $``SS''$ denoting the Stiffness-Switching method, which allows to switch between implicit and explicit methods to resolve stiff problems.}. Another useful feature is the \texttt{Mathematica}'s precision control options, which allows the user to set the  precision and accuracy of numerical result through the commands \texttt{WorkingPrecision}, \texttt{SetPrecision} and \texttt{SetAccuracy}. 
%In the following we will show how to recast as an evolutionary problems the covariant equations in the BGO framework, and the procedure to obtain the BGO implemented in the code. %We dedicate a separate chapter, Chap.~\ref{cha:obs}, to the computation of the observables once the BGO are known. 

%%%%%%%%%%%%%%%%%%%%%%%%%%%%%%%%%%%%%%%%%%%%%%%%%%%%%%%%%%%%%%%%%%%%%%%%%%%%%%%%%%%%%%%%%%%%%%%%%%%%%%%%%%%%%%%


\section{The BGO framework in 3+1 form}
\label{sec:3+1}

\MG{USED IN CHAP. COSMOLOGY: To perform  simulations of full-GR dynamical systems in numerical relativity, one needs to separate the temporal and the spatial dependence in the Einstein field equation, to recast their expression in the form of an evolutionary problem. This procedure is known as $3+1$ splitting of the spacetime and it commonly used by the numerical codes in cosmology and astrophysics. To be compatible with the numerical output generated by these codes, in {\tt BiGONLight} we adopt the BGO framework expressed in the 3+1 form.
In the following we summarise some basic definitions of the 3+1 formalism and report the 3+1 version of all the equations for light propagation used to compute the BGO. For comprehensive references of the 3+1 formalism see e.g. \cite{Smarr:1977uf, Alcubierre2008, baumgarte2010numerical, gourgoulhon20123+}.

The procedure of splitting a four-dimensional spacetime $(\mathcal{M},g_{\mu \nu})$ into its $3+1$ form, the so-called ADM formalism, was introduced by Arnowitt, Deser and Misner in \cite{arnowitt1959}. It is constructed by foliating a four-dimensional manifold $\mathcal{M}$ with a family of 3D space-like hypersurfaces $\Sigma_t$, labelled by a monotonic function $t$, such that $t=\const$ on each slice. This space-like foliation defines the time-like vector field $n^{\mu}$, which is orthonormal to the slices and it can be regarded as the four-velocity of Eulerian observes. 
The geometry on each hypersurface is described by the following quantities:
\begin{itemize}
\item the induced metric $\gamma_{\mu \nu}$ is defined as the covariant form of the orthogonal projector onto the slices 
\begin{equation}
\gamma_{\mu \nu}=g_{\mu \nu} + n_{\mu}n_{\nu}
\end{equation}
and it is used to measure proper distances on $\Sigma_t$;
\item the covariant derivative on the slice
\begin{equation}
D_{\nu}V^{\mu}=\gamma\UD{\sigma}{\nu}\gamma\UD{\mu}{\rho}\nabla_{\sigma}V^{\rho} \, ,
\label{eq:3D_covD}
\end{equation}
which is written in terms of the 3D Christoffel symbol ${}^{(3)}\Gamma^{k}_{\ i j}=\frac{1}{2}\gamma^{k l}(\frac{\partial \gamma_{l j}}{\partial x^i}+\frac{\partial \gamma_{i l}}{\partial x^j}-\frac{\partial \gamma_{i j}}{\partial x^l})$, once a coordinate system ${x^i}$ on $\Sigma_{t}$ is introduced;
\item the extrinsic curvature $K_{\mu \nu}$ defined as
\begin{equation}
K_{\mu \nu}=-\gamma\UD{\sigma}{\mu}\gamma\UD{\rho}{\nu}\nabla_{\sigma}n_{\rho}\, ,
\label{eq:K_def}
\end{equation}
which represents the curvature of the 3D hypersurfaces with respect to the 4D embedding spacetime.
\end{itemize}
A natural choice for a coordinate system $x^{\mu}$ is the one adapted to the foliation: the corresponding reference frame is such that the three vectors $\partial^{\mu}_i=\left(\frac{\partial}{\partial x^i}\right)^{\mu}$ are tangent to the hypersurface while $\partial^{\mu}_0=\left(\frac{\partial}{\partial t}\right)^{\mu}$ is transverse to it. In particular, the time-like vector field $\left(\frac{\partial}{\partial t}\right)^{\mu}$ is tangent to a congruence of world-lines of coordinate observers and it is given by
\begin{equation}
\left(\dfrac{\partial}{\partial t}\right)^{\mu}=\alpha n^{\mu}+\beta^{\mu}\, ,
\end{equation}
where $\alpha$ is the lapse function, which measures the proper time of the Eulerian observers, and $\beta^{\mu}$ is the  shift vector, which quantifies the displacement on $\Sigma_t$ of the coordinate observer $\left(\frac{\partial}{\partial t}\right)^{\mu}$ with respect to the Eulerian observer $n^{\mu}$. 
The components of the normal vector and the metric $g_{\mu \nu}$ in the adapted coordinates are written with the lapse, the shift and the induced metric as
\begin{equation}
n^{\mu}=(\dfrac{1}{\alpha},-\dfrac{\beta^i}{\alpha})\, ,
\label{eq:nu_nd}
\end{equation}
and
\begin{equation}
g_{\mu \nu}=\begin{pmatrix}
\beta_{i}\beta^i-\alpha^2 &~ \beta_i\\
\beta_j &~ \gamma_{i j}
\end{pmatrix}\, ,
\label{eq:g_3+1}
\end{equation}
where the Latin indices runs from $1$ to $3$. A generic 4D tensor is projected on the slice as 
\begin{equation}
{}^{(3)}T^{\mu_{\rm 1} \cdots \mu_{\rm m}}_{ \ \ \nu_{\rm 1} \cdots \nu_{\rm n}}={}^{(4)} T^{\rho_{\rm 1} \cdots \rho_{\rm m}}_{ \ \ \sigma_{\rm 1} \cdots \sigma_{\rm n}} \gamma\UD{\mu_{\rm 1}}{\rho_{\rm 1}}\cdots\gamma\UD{\mu_{\rm m}}{\rho_{\rm m}} \gamma\UD{\sigma_{\rm 1}}{\nu_{\rm 1}}\cdots\gamma\UD{\sigma_{\rm n}}{\nu_{\rm n}}\, .
\label{eq:3Dproj}
\end{equation}  

}

The idea behind the design of our package is to create a tool which is adaptable to perform analytical as well as numerical studies of light propagation, using only few   inputs provided by the user.
The required input are the specetime metric and the source/observer kinematics, which can be provided in two different ways: (i) from the output of a relativistic numerical simulation of the spacetime dynamics, i.e. directly the ADM quantities ($\alpha, \, \beta^i,\, \gamma_{i j},\, K_{i j}$) generated by a numerical simulation, (ii) from the analytical expression of the metric $g_{\mu \nu}$ and the emitter/observer four-velocities ($u^{\mu}_{\mathcal{S}}$, $u^{\mu}_{\mathcal{O}}$) and four-accelerations ($w^{\mu}_{\mathcal{S}}$, $w^{\mu}_{\mathcal{O}}$) for some exact model. In this case, the ADM quantities are computed in {\tt BiGONLight} by the function {\tt ADM[]} accordingly to Eqs.~\eqref{eq:K_def},~\eqref{eq:nu_nd},~\eqref{eq:g_3+1}.
On top of that, \texttt{BiGONLight} contains other functions to calculate the Christoffel symbols, \texttt{Christoffel[]}, and the Riemann tensor, \texttt{Riemann[]}. 
Note that in both cases, the input metric is provided in form of components within a specific coordinate system and not in full tensorial form. The package is designed to work in any gauge and any coordinate system, leaving the choice to the user. 
Once we have summarized the basics, we will now describe in more details and step by step the procedure for obtaining the BGO in the 3+1 decomposition: we report the 3+1 version of the geodesic equation following \cite{Vincent:2012kn} and we derive all the 3+1 ingredients for the evolution equation of the BGO.
%%%%%%%%%%%%%%%%%%%%%%%%%%%%%%%%%%%%%%%%%%%%%%%%%%%

\subsection{Geodesic equation in $3+1$ decomposition}
The first code solving the $3+1$ geodesic equation was presented in \cite{hughes1994finding} and it was used to map the event horizon in numerical simulations with black holes (like heads-on collision of two black holes), while a more recent formulation of the $3+1$ geodesic equation was presented in~\cite{Vincent:2012kn}. Here, we briefly resume the procedure in~\cite{Vincent:2012kn}, since we will use their approach throughout all our calculations.

The null geodesics representing light rays connecting source and observer is obtained by solving 
\begin{equation}
\ell^{\sigma}\nabla_{\sigma} \ell^{\mu}=0
\label{eq:covar_geo}
\end{equation}
where $\ell^{\mu}$ is the tangent vector which obeys to the null condition
\begin{equation}
\ell^{\sigma}\ell_{\sigma}=0\, .
\label{eq:null_cond}
\end{equation}
It is $3+1$ decomposed as:
\begin{equation}
\ell^{\mu}=\calE(n^{\mu}+V^{\mu})\, ,
\label{eq:l_split}
\end{equation} 
where $\calE$ is defined by $\calE=-n^{\mu} \ell_{\mu}$ and $V^{\mu}n_{\mu}=0$. In other words, $\calE V^{\mu}$ is the component of $\ell^{\mu}$ tangent to $\Sigma_t$ and $\calE n_{\nu}$ is the orthogonal one. Substituting Eq.~\eqref{eq:l_split} in the geodesic equation~\eqref{eq:covar_geo}
after a long but straightforward calculation, see~\cite{Vincent:2012kn}, the geodesic equation decouples in two differential equations
\begin{equation}
\dfrac{d \calE}{d t}= \calE \left(\alpha K_{i k} V^j V^k - V^j \partial_j \alpha \right)\, ,
\label{eq:3+1_geodE}
\end{equation}
\begin{equation}
\left\{ \begin{array}{l}
\dfrac{d x^i}{d t}= \alpha V^i - \beta^i \\
 \\
\dfrac{d V^i}{d t}= \alpha V^j \left[V^i \left( \partial_j \log \alpha - K_{j k} V^k \right) +2K^i_{\ \ j} - {}^{(3)}\Gamma^i_{\ j k} V^k \right] - \gamma^{i j} \partial_j \alpha - V^j \partial_j \beta^i
\end{array}\right.
\label{eq:3+1_geodV}
\end{equation}
for the orthogonal and the tangent components, respectively.

In the \texttt{BiGONLight} package the two functions  \texttt{EnergyEquations[]} and \texttt{GeodesicEquations[]} give the differential equations~\eqref{eq:3+1_geodE} and~\eqref{eq:3+1_geodV}, respectively. Next, we need to specify the initial conditions. The function \texttt{InitialConditions[]} is specifically constructed to get the initial conditions for $V^i$ and $\calE$ consistent with the null condition~\eqref{eq:null_cond} and the decomposition~\eqref{eq:l_split}.
The ODE~\eqref{eq:3+1_geodE} and~\eqref{eq:3+1_geodV} are then solved by using two customized versions of the Mathematica \texttt{NDSolve[]} function: \texttt{SolveEnergy[]} and \texttt{SolveGeodesic[]}.

%%%%%%%%%%%%%%%%%%%%%%%%%%%%%%%%%%%%%%%%%%%%%%%%%%%

\subsection{Parallel transport equation in the $3+1$ decomposition}
The BGO map the changes of the deviations ($\delta x^{\mu}$, $\Delta \ell^{\mu}$) between the observer $\calO$ and the source $\cal S$ along the photon geodesic $\gamma$, see Eq.~\eqref{eq:compact_W(S,O)}. This is possible only if we introduce a frame parallel transported along $\gamma$, which allows us to compare quantities at the observation point and at source position. For our purposes, the definition and the parallel transport of the frame have to be split into the 3+1 form. 
Let us start with the parallel transport of a generic vector $e^{\mu}$ along a given geodesic with tangent $\ell^{\mu}$, which is governed by
\begin{equation}
\ell^{\alpha}\nabla_{\alpha}e^{\mu}=0\, .
\label{eq:parallel_transport}
\end{equation}
The 3+1 decomposition of $e^{\mu}$ reads
\begin{equation}
e^{\mu}=C n^{\mu}+E^{\mu}\, ,
\label{eq:e_split}
\end{equation}
where we have defined the orthogonal component $C n^{\mu}=-n_{\alpha}e^{\alpha} n^{\mu}$ and the tangent component $E^{\mu}=\gamma\UD{\mu}{\alpha} e^{\alpha}$. 
The parallel transport equation~\eqref{eq:parallel_transport} becomes
\begin{equation}
n^{\mu}(n^{\alpha}\nabla_{\alpha} C+V^{\alpha}\nabla_{\alpha}C)+C(n^{\alpha}\nabla_{\alpha}n^{\mu}+V^{\alpha}\nabla_{\alpha}n^{\mu})+n^{\alpha}\nabla_{\alpha}E^{\mu}+V^{\alpha}\nabla_{\alpha}E^{\mu}=0\, .
\label{eq:PT_interm_1}
\end{equation}
Now, we make use of some $3+1$ well-known relations: for the first term we have $n^{\mu}\nabla_{\mu}C=\frac{1}{\alpha} \mathcal{L}_{\alpha \vec{n}}C$, in the second bracket we substitute $n^{\alpha}\nabla_{\alpha}n^{\mu}=D^{\mu}\log\alpha$ and the definition of the extrinsic curvature $\nabla_{\alpha}n^{\mu}=-K\UD{\mu}{\alpha}$, and for the expansion of the last two terms we use the two identities:
\begin{align*}
V^{\alpha}\nabla_{\alpha}E^{\mu}&=V^{\alpha}D_{\alpha}E^{\mu}-K_{\alpha \beta}V^{\alpha}E^{\beta} n^{\mu}\\
 \\
n^{\nu}\nabla_{\nu}E^{\mu}&=\dfrac{1}{\alpha}\left( \mathcal{L}_{\alpha \vec{n}}E^{\mu}+E^{\nu}\nabla_{\nu}(\alpha n^{\mu})\right)=\dfrac{1}{\alpha} \mathcal{L}_{\alpha \vec{n}}E^{\mu} - E^{\nu}K\UD{\mu}{\nu}+E^{\nu}\partial_{\nu}(\log\alpha) \  n^{\mu}\, .
\end{align*}
We then re-write Eq.~\eqref{eq:PT_interm_1} split into two parts, the one proportional to $n^{\mu}$ and the other tangent to $\Sigma_t$ as
\begin{equation}
\left\{\begin{array}{l}
\dfrac{1}{\alpha} \mathcal{L}_{\alpha \vec{n}}C
+V^{\nu}\partial_{\nu}C + E^{\nu}\partial_{\nu}(\log\alpha) -K_{\nu \rho}V^{\nu}E^{\rho}=0\\
 \\
\dfrac{1}{\alpha} \mathcal{L}_{\alpha \vec{n}}E^{\mu} - E^{\nu}K\UD{\mu}{\nu} +V^{\nu}D_{\nu}E^{\mu}+ C\left(\gamma^{\mu \nu}D_{\nu}\log\alpha -K\UD{\mu}{\nu}V^{\nu}\right)  =0\, ,
\end{array}\right.
\label{eq:separated_PT}
\end{equation}
where both must vanish individually in order to satisfy the parallel transport condition~\eqref{eq:parallel_transport}. The last steps consist in expanding the Lie derivative
\begin{equation}
\mathcal{L}_{\alpha \vec{n}}=\dfrac{\partial}{\partial t}-\mathcal{L}_{ \vec{\beta}}
\end{equation}
and converting partial derivatives with respect to the time $t$ into total derivative via
\begin{equation}
\dfrac{\partial}{\partial t}=\dfrac{d }{d t}-\alpha V^j\partial_j + \beta ^j\partial_j \, .
\end{equation}
The final result for the $3+1$ parallel transport equation is
\begin{equation}
\left\{ \begin{array}{l}
\dfrac{1}{\alpha} \dfrac{d C}{d t} + E^{i}\partial_{i}\log\alpha -K_{i j}V^{i}E^{j}=0\\
\\
 \dfrac{1}{\alpha}\left(\dfrac{d E^{i}}{d t}+ E^j\partial_j \beta^i \right)+ \ ^{(3)}\Gamma^i_{\ j k} V^j E^{k} - K\UD{i}{ j}E^{j} +C\left( \gamma^{i j}D_{j}\log\alpha -K\UD{i}{j}V^{j}\right)  =0\, ,
\end{array} \right.
\label{eq:3+1PT}
\end{equation}
where we have only spatial indices $i, j=1,2,3$ since all the quantities lie on $\Sigma_t$.

The system of equations~\eqref{eq:3+1PT} has to be solved for each vector of the frame we choose. 
Following \cite{Grasso:2018mei}, we choose the semi-null frame (SNF) composed by the tetrad of vectors 
\begin{equation}
\phi\UD{\mu}{\boldsymbol{\alpha}}=(u^{\mu},\phi\UD{\mu}{\mathbf{1}},\phi\UD{\mu}{\mathbf{2}},\ell^{\mu})\, ,
\end{equation}
where $u^{\mu}$ is the matter four-velocity, $\ell^{\mu}$ is the tangent of the photon geodesic. The two vectors $\phi\UD{\mu}{\boldsymbol{A}}$ are  orthonormal to both $u^{\mu}$ and $\ell^{\mu}$, namely:
\begin{equation}
\left\{ \begin{array}{l}
%g_{\mu \nu} u^{\mu}u^{\nu}=-1\\
g_{\mu \nu} u^{\mu}\phi\UD{\mu}{\mathbf{A}}=0\\
g_{\mu \nu} \ell^{\mu}\phi\UD{\mu}{\mathbf{A}}=0\\
g_{\mu \nu} u^{\mu}\ell^{\nu}=Q\\
g_{\mu \nu} \phi\UD{\mu}{\mathbf{A}}\phi\UD{\mu}{\mathbf{B}}=\delta_{\mathbf{A} \mathbf{B}}\\
%g_{\mu \nu} \ell^{\mu}\ell^{\nu}=0\, ,
\end{array}\right.
\label{eq:SNF_def_rel}
\end{equation}
%\justify
with $Q$ a real number. Each vector of the SNF is then decomposed in $3+1$ form as
\begin{equation}
\phi^{\mu}_{\boldsymbol{\alpha}}=\Phi_{\boldsymbol{\alpha}}n^{\mu}+F^{\mu}_{\boldsymbol{\alpha}}\,
\label{eq:3+1vec}
\end{equation}
and parallelly propagated by solving Eq.~\eqref{eq:3+1PT} for its orthogonal and tangent components, i.e. $\Phi_{\boldsymbol{\alpha}} n^{\mu}=-n_{\sigma}\phi^{\sigma}_{\boldsymbol{\alpha}} n^{\mu}$ and $F^{\mu}_{\boldsymbol{\alpha}}=\gamma\UD{\mu}{\sigma} \phi^{\sigma}_{\boldsymbol{\alpha}}$. 

In {\tt BiGONLight} this is demanded to the function \texttt{PTransportedFrame[]}, which gives as output the components in~\eqref{eq:3+1vec} of the SNF parallel transported. The function \texttt{PTransportedFrame[]} uses \texttt{ParallelTransport[]} to obtain Eq.~\eqref{eq:3+1PT} for each vector of the frame and then solve them with a customised {\tt NDSolve[]} function.


\subsection{The optical tidal matrix and the GDE for the BGO}
\label{sec:W3+1}
So far the $3+1$ equations Eqs.~\eqref{eq:3+1_geodE}-\eqref{eq:3+1_geodV} and Eq.~\eqref{eq:3+1PT} we presented are valid for any type of geodesics and to parallelly transport any type of vectors along that geodesic. From now on, we will restrict the BGO formalism to the case of a SNF parallelly propagated along a null-like geodesic.
The equation for the BGO has then to be projected onto the SNF. The result is simply given by
\begin{equation}
\dfrac{d}{d \lambda} \mathcal{W}=\begin{pmatrix}
0 & \mathbb{1}\\
R\UDDD{\boldsymbol{\mu}}{\ell}{\ell}{\boldsymbol{\nu}} & 0
\end{pmatrix} \mathcal{W}\, ,
\label{eq:GDE_for_BGO_inframe}
\end{equation}
where the only formal difference with respect to Eq.~\eqref{eq:GDE_for_BGO} is that the covariant derivative along the photon geodesic $D/D\lambda$ reduces to the total derivative $d/d\lambda$ in the SNF.
In Eq.~\eqref{eq:GDE_for_BGO_inframe}, the optical tidal matrix is projected in the SNF and it is given by 
\beq
R\UDDD{\boldsymbol{\mu}}{\ell}{\ell}{\boldsymbol{\nu}}=h^{\boldsymbol{\mu} \boldsymbol{\omega}}R_{\boldsymbol{\omega} \ell \ell \boldsymbol{\nu}}=\phi^{\rho \boldsymbol{\mu}}R_{\rho \alpha \beta \sigma} \ell^{\alpha}\ell^{\beta} \phi\UD{\sigma}{\boldsymbol{\nu}}
\label{eq:RinSNF}
\eeq
where the inverse of the induced metric of the frame
\begin{equation}
 h^{\boldsymbol{\mu} \boldsymbol{\omega}}=\begin{pmatrix}
0 & 0 & 0 & \frac{1}{Q} \\
0 & 1 & 0 & 0 \\
0 & 0 & 1 & 0 \\
\frac{1}{Q} & 0 & 0 & \frac{1}{Q^2}
\end{pmatrix}\, 
\label{eq:frame_matrix}
\end{equation}
is used to raise the indices.
In general $R\UDDD{\boldsymbol{\mu}}{\ell}{\ell}{\boldsymbol{\nu}}$ is a $4 \times 4$ matrix with non trivial components. However, in the SNF it is easy to use the symmetries of the Riemann tensor to show that the components $R\UDDD{\boldsymbol{0}}{\ell}{\ell}{\boldsymbol{\nu}}=R\UDDD{\ell}{\ell}{\ell}{\boldsymbol{\nu}}$ and $R\UDDD{\boldsymbol{\mu}}{\ell}{\ell}{\boldsymbol{0}}=R\UDDD{\boldsymbol{\mu}}{\ell}{\ell}{\ell}$ vanish.
Now, let us use Eq.~\eqref{eq:l_split} and Eq.~\eqref{eq:3+1vec} in Eq.~\eqref{eq:RinSNF} to write the optical tidal matrix in terms of $3+1$ quantities
\beq
\begin{array}{l}
R\UDDD{\boldsymbol{\mu}}{\ell}{\ell}{\boldsymbol{\nu}}=(\Phi^{\boldsymbol{\mu}} n^{\rho} + F^{\boldsymbol{\mu} \rho})R_{\rho \alpha \beta \sigma}\mathcal{E}^2( n^{\alpha}n^{\beta}+n^{\alpha}V^{\beta}+V^{\alpha}n^{\beta}+V^{\alpha}V^{\beta})(\Phi_{\boldsymbol{\nu}} n^{\sigma} + F\UD{\sigma}{\boldsymbol{\nu}})
\end{array}
\eeq
After some tedious but straightforward calculations, we finally obtain
\begin{equation}
\begin{array}{lc}
 R\UDDD{\boldsymbol{\mu}}{\ell}{\ell}{\boldsymbol{\nu}}= & \calE^2 h^{\boldsymbol{\mu} \boldsymbol{\rho}} \left[ \mathcal{R}_{\beta \alpha}\left(\Phi_{\boldsymbol{\rho}} F\UD{\beta}{\boldsymbol{\nu}}V^{\alpha}+\Phi_{\boldsymbol{\nu}} F\UD{\beta}{\boldsymbol{\rho}}V^{\alpha}-\Phi_{\boldsymbol{\rho}}\Phi_{\boldsymbol{\nu}} V^{\beta}V^{\alpha}-F\UD{\alpha}{\boldsymbol{\rho}}F\UD{\beta}{\boldsymbol{\nu}}\right)+\right.\\
 & \mathcal{C}_{\sigma \beta \alpha}\left(\Phi_{\boldsymbol{\rho}} F\UD{\sigma}{\boldsymbol{\nu}}V^{\alpha}V^{\beta}+\Phi_{\boldsymbol{\nu}} F\UD{\sigma}{\boldsymbol{\rho}}V^{\alpha}V^{\beta}-F\UD{\alpha}{\boldsymbol{\rho}} F\UD{\sigma}{\boldsymbol{\nu}}V^{\beta}-F\UD{\alpha}{\boldsymbol{\nu}} F\UD{\sigma}{\boldsymbol{\rho}}V^{\beta} \right)+\\
 & \left.\mathcal{G}_{\omega \alpha \beta \sigma} F\UD{\omega}{\boldsymbol{\rho}}V^{\alpha}V^{\beta} F\UD{\sigma}{\boldsymbol{\nu}}\right]
\end{array}
\label{eq:3+1_opt}
\end{equation}
where
\begin{align}
 \mathcal{G}_{\mu \alpha \beta \nu}&=R_{\rho \delta \theta \sigma}\gamma\UD{\rho}{\mu}\gamma\UD{\delta}{\alpha}\gamma\UD{\theta}{\beta}\gamma\UD{\sigma}{\nu}= {}^{(3)}R_{\mu \alpha \beta \nu}+ K_{\mu \beta} K_{\alpha \nu}-K_{\mu \nu} K_{\beta \alpha} \label{eq:Gauss}\\
%\nonumber  \\
 \mathcal{C}_{\mu \alpha \beta}&=R_{\rho \delta \theta \sigma}n^{\rho}\gamma\UD{\delta}{\alpha}\gamma\UD{\theta}{\beta}\gamma\UD{\sigma}{\mu}= D_{\alpha}K_{\mu \beta}-D_{\mu}K_{\alpha \beta} \label{eq:Codazzi}\\
% \nonumber \\
 \mathcal{R}_{\mu \nu}&=R_{\rho \delta \theta \sigma}n^{\rho}\gamma\UD{\delta}{\nu}\gamma\UD{\theta}{\mu}n^{\sigma}= \mathcal{L}_{\mathbf{n}}K_{\nu \alpha}+\dfrac{1}{\alpha}D_{\nu}D_{\alpha}\alpha+ K\UD{\rho}{\alpha}K_{\nu \rho}
\label{eq:Ricci}
\end{align}
are the Gauss relation, the Codazzi relation and the Ricci relation, respectively (see e.g. \cite{baumgarte2010numerical}).
In \texttt{BiGONLight}, the three functions \texttt{GaussRelation[]}, \texttt{CodazziRelation[]}, \texttt{RicciRelation[]} compute the relations in Eqs.~\eqref{eq:Gauss}-\eqref{eq:Ricci} and the function \texttt{OpticalTidalMatrix[]} collects all previous results together in order to obtain the optical tidal matrix $R\UDDD{\boldsymbol{\mu}}{\ell}{\ell}{ \boldsymbol{\nu}} / \calE^2$ in Eq.~\eqref{eq:3+1_opt} as output.
We now have the equation for the BGO, Eq.~\eqref{eq:GDE_for_BGO_inframe}, projected onto the SNF and written in terms of $3+1$ quantities. To solve it, it is easier to change the derivation variable from the affine parameter $\lambda$ to the time $t$ according to
\begin{equation}
\dfrac{d }{d \lambda}=\dfrac{d \, t}{d \lambda}\dfrac{d }{d t}=\frac{ \mathcal{E}}{\alpha}\dfrac{d }{d t} \, ,
\end{equation}
where we used the time component of the tangent vector in Eq.~\eqref{eq:l_split}, i.e. $\ell^0=d \, t / d \lambda= \mathcal{E} / \alpha$.
The GDE for the BGO, Eq.~\eqref{eq:GDE_for_BGO_inframe}, decouples in two systems of first-order ODE, one for $(\WXX, \WLX)$ and the other for $(\WXL, \WLL)$, computed separately using the function \texttt{BGOequations[]}:
\begin{equation}
\left\{\begin{array}{l}
\dfrac{d \, \WXX{}\UD{\boldsymbol{\mu}}{\boldsymbol{\nu}}}{d t}=\dfrac{\alpha}{ \mathcal{E}}\, \WLX{}\UD{\boldsymbol{\mu}}{\boldsymbol{\nu}}\\
 \\
\dfrac{d \, \WLX{}\UD{\boldsymbol{\mu}}{\boldsymbol{\nu}}}{d t}=\dfrac{\alpha}{ \mathcal{E}}\, R\UDDD{\boldsymbol{\mu}}{\ell}{\ell}{\boldsymbol{\sigma}}\WXX{}\UD{\boldsymbol{\sigma}}{\boldsymbol{\nu}}
\end{array}\right.\ , \
\left\{\begin{array}{l}
\dfrac{d \, \WXL{}\UD{\boldsymbol{\mu}}{\boldsymbol{\nu}}}{d t}=\dfrac{\alpha}{ \mathcal{E}}\, \WLL{}\UD{\boldsymbol{\mu}}{\boldsymbol{\nu}}\\
 \\
\dfrac{d \, \WLL{}\UD{\boldsymbol{\mu}}{\boldsymbol{\nu}}}{d t}=\dfrac{\alpha}{ \mathcal{E}}\, R\UDDD{\boldsymbol{\mu}}{\ell}{\ell}{\boldsymbol{\sigma}}\WXL{}\UD{\boldsymbol{\sigma}}{\boldsymbol{\nu}}
\end{array}\right.
\label{eq:GDE_BGO_syst}
\end{equation}
with initial conditions:
\begin{equation}
\left\{\begin{array}{l}
\WXX{}\UD{\boldsymbol{\mu}}{\boldsymbol{\nu}}\left|_{\rm \calO}\right.=\delta\UD{\boldsymbol{\mu}}{\boldsymbol{\nu}}\\
\WXL{}\UD{\boldsymbol{\mu}}{\boldsymbol{\nu}}\left|_{\rm \calO}\right.=0\\
\WLX{}\UD{\boldsymbol{\mu}}{\boldsymbol{\nu}}\left|_{\rm \calO}\right.=0\\
\WLL{}\UD{\boldsymbol{\mu}}{\boldsymbol{\nu}}\left|_{\rm \calO}\right.=\delta\UD{\boldsymbol{\mu}}{\boldsymbol{\nu}}\\
\end{array}\right.
\label{eq:GDE_BGO_IC}
\end{equation}
The systems in Eq.~\eqref{eq:GDE_BGO_syst} are solved separately in {\tt BiGONLight} by \texttt{SolveBGO[]}. Getting the BGO with the procedure just described is one important part of {\tt BiGONLight}. Let us recall that the only inputs required are the spacetime metric, the observer four-velocity components and the initial and ending points.

With initial conditions in Eq.~\eqref{eq:GDE_BGO_IC}, Eq.~\eqref{eq:GDE_BGO_syst} gives the BGO $\mathcal{W}(p_\lambda,\mathcal{O})$ integrated backward in time from the observer. Similarly, if the initial conditions were set at the source $\calS$, the solution of Eq.~\eqref{eq:GDE_BGO_syst} gives the BGO $\mathcal{W}(p_\lambda,\mathcal{S})$ integrated forward in time. However, in this case, to be used in observables calculation we need to implement the transformations Eqs.~\eqref{eq:WXX_inverse}-\eqref{eq:WLL_inverse} to obtain the backward integrated BGO $\mathcal{W}(p_\lambda,\mathcal{O})$ from $\mathcal{W}(p_\lambda,\mathcal{S})$.
These relations are coded in the section ``Forward to backward transformation for $\mathcal{W}$ operators'' of each sample notebook in the repository {\color{blue}{\tt {https://github.com/MicGrasso/\\ 
bigonlight.git}}}.

%%%%%%%%%%%%%%%%%%%%%%%%%%%%%%%%%%%%%%%%%%%%%%%%%%%%%%%%%%%%%%%%%%%%%%%%%%%%%%%%%%%%%%%%%%%%%%%%%%%%%%%%%%%%%%%

\section{The recipe to calculate observables with BiGONLight}
\MG{incollare da bigonlight}
\label{sec:obs}
The recipe to obtain optical observables using \texttt{BiGONLight} can be summarized as follows. One needs to:
\begin{enumerate}
\item specify the spacetime metric $g_{\mu \nu}$ and the source $\mathcal{S}$ and observer $\mathcal{O}$ kinematics, namely four-velocity $u^{\mu}$ and four-acceleration $w^{\mu}$. They can be given already in $3+1$ components or as 4D quantities and the functions {\tt ADM[]} and {\tt Vsplit[]} will do the splitting of $g_{\mu \nu}$, and $u^{\mu}$ and $w^{\mu}$ respectively;
\item set up the initial photon geodesic using {\tt Vsplit[]} for the null tangent $\ell^{\mu}$, which gives its $3+1$ components $\calE$ and $V^{i}$. Alternatively one can give $\calE$ and a vector $V^{i}$ that has to be assigned by specifying the spatial direction $V^2$ and $V^3$ and use {\tt InitialConditions[]} to get $V^1$ from the null condition; \\
\item obtain the geodesic equations Eq.~\eqref{eq:3+1_geodE} and Eq.~\eqref{eq:3+1_geodV} from {\tt GeodesicEquations[]} and {\tt EnergyEquations[]}, and then solve them with {\tt SolveGeodesic[]} and {\tt SolveEnergy[]}\footnote{For the purposes of our paper, we need to solve only photon geodesics. However, the code can be used to trace any type of geodesics, namely time-like and space-like also, by specifying the appropriate initial tangent vector in the $3+1$ splitting with {\tt Vsplit[]}.};\\
\item set up the initial conditions for the SNF according to Eq.~\eqref{eq:SNF_def_rel}, directly in $3+1$ components or using {\tt SNF[]}, which is specifically designed to compute the SNF in $3+1$. Then {\tt PTransportedFrame[]} will give the SNF parallel transported along the light ray;\\
\item compute separately $R\UDDD{\boldsymbol{\mu}}{\ell}{\ell}{\boldsymbol{\nu}}$ projected into the SNF with {\tt OpticalTidalMatrix[]};\\
\item obtain the ODE system for the BGO in Eq.~\eqref{eq:GDE_BGO_syst} with {\tt BGOequations[]} and, together with the initial conditions in Eq.~\eqref{eq:GDE_BGO_IC}, solve it using {\tt SolveBGO[]} to finally find the full $\mathcal{W}$ matrix;\\
\end{enumerate}
Note that the components of the optical tidal matrix can have a very complicated expression, that may cause problems in solving the GDE \eqref{eq:GDE_BGO_syst}. This can be overcame by using an interpolated form of $R\UDDD{\boldsymbol{\mu}}{\ell}{\ell}{\boldsymbol{\nu}}$: the {\tt Mathematica Interpolation[]} function allows to use different methods and reach an excellent precision, see Sec.\ref{sec:test}.

Step (vi) is the starting point to compute the optical observables, which are all given by different combinations and/or functions of the $\mathcal{W}$ components. It is important to remark that all the observables in the BGO formalism are written in terms of $\mathcal{W}(\calS, \calO)$, namely the map computed from the observer to the source. As we already recalled, this is obtained directly by integrating the GDE backward in time. However in some cases, e.g. if the spacetime model comes from a numerical simulation, it may be more convenient to get the inverse BGO map $\mathcal{W}^{-1}(\calS, \calO)$ and then use the transformations Eqs.~\eqref{eq:WXX_inverse}-\eqref{eq:WLL_inverse} to obtain the $\mathcal{W}$ needed for the observables. 

In the next chapter we will use this recipe to test the accuracy of {\tt BiGONLight} in reproducing well-known results. The tests are conducted comparing a number of observables of interest in cosmology obtained by using two different method, i.e. with {\tt BiGONLight} and with the standard calculation present in the literature. The tests are conducted for several cosmological models and for each model we will consider four observables: the redshift $z$, the angular diameter distance $D_{\rm ang}$, the parallax distance $D_{\rm par}$ and the redshift drift $\zeta$, as they are defined in Sec.~\ref{sec:observables}.




%For different cosmological models we will compute a number of observables using {\tt BiGONLight} and we will compare them with the corresponding observable obtained by standard calculation. The observables that we will analyse are

%%%%%%%%%%%%%%%%%%%%%%%%%%%%%%%%%%%%%%%%%%%%%%%%%%%%%%%%%%%%%%%%%%%%%%%%%%%%%%%%%%%%%%%%%%%%%%%%%%%%%%%%%%%%%%%

\endinput


Here we list the four observables that we study in this paper. The redshift is simply given by its definition
\begin{equation}
1+z = \dfrac{\left(\ell_{\sigma } u^{\sigma}\right)|_{\cal S}}{\left(\ell_{\sigma } u^{\sigma}\right)|_{\cal O}}\, ,
\label{eq:redshift_def}
\end{equation}
where $\ell^{\sigma}$ is the tangent to the light ray, and $u^{\sigma}_{\mathcal{O}}$ and $u^{\sigma}_{\mathcal{S}}$ are the observer and source four-velocities. The same definition in $3+1$ splitting reads:
\begin{equation}
1+z=\frac{\mathcal{E}_{\mathcal{S}}}{\mathcal{E}_{\mathcal{O}}}\,\frac{1-\left(\gamma _{ij} V^i U^j\right)|_{\mathcal{S}}}{1-\left(\gamma _{ij} V^i U^j\right)|_{\mathcal{O}}}\left[\frac{1- \left(\gamma _{ij}U^i U^j\right)|_{\mathcal{S}}}{1-\left(\gamma _{ij}U^i U^j\right)|_{\mathcal{O}}}\right]^{\frac{1}{2}}\, .
\label{eq:redshift_def_3+1}
\end{equation}
The angular diameter distance is formally given by
\begin{equation}
D_{ang} = \left(\ell_{\sigma} u^{\sigma}\right)|_{\cal O} \left| \det \left(\WXL {}\UD{\bm A}{\bm B}\right) \right|^{\frac{1}{2}}\, ,
\label{eq:D_ang_BGO}
\end{equation}
where $\WXL {}\UD{\bm A}{\bm B}$ is the map between the physical size of the source and the angle subtended in the sky, as measured at the observer position, namely
\begin{equation}
\delta \theta^{\bm A}= {\left( \ell_{\sigma} u^{\sigma}\right)|_{\cal O}}^{-1} \left(\WXL {}\UD{\bm A}{\bm B}\right)^{-1} \delta x_{\mathcal{S}}^{\bm B}\, .
\end{equation}
Conversely, the parallax distance is related to the displacement of the observer position and the apparent angular shift of the source position, as measured from the observer
\begin{equation}
\delta \theta^{\bm A}= - {\left( \ell_{\sigma} u^{\sigma}\right)|_{\cal O}}^{-1} \left(\WXL {}\UD{\bm A}{\bm C}\right)^{-1} \WXX {}\UD{\bm C}{\bm B} \delta x_{\mathcal{O}}^{\bm B}\, .
\end{equation}
The expression for the parallax distance is
\begin{equation}
D_{par} =\left(\ell_{\sigma} u^{\sigma}\right)|_{\cal O} \frac{\left| \det \left(\WXL {}\UD{\bm A}{\bm B}\right) \right|^{\frac{1}{2}}}{\left| \det \left(\WXX {}\UD{\bm A}{\bm B}\right) \right|^{\frac{1}{2}}}\, .
\label{eq:D_par_BGO}
\end{equation}
The last observable that we consider in this paper is the redshift drift $\zeta$, given in terms of the BGO by, \cite{Julius}
\begin{equation}
\zeta\equiv\frac{\delta \log(1+z)}{\delta \tau_{\mathcal{O}}}= \Xi_{\rm Doppler}- \left( u_{\mathcal{O}} , \, \frac{u_{\mathcal{S}}}{1+z} \right) \boldsymbol{U} \begin{pmatrix} u_{\mathcal{O}}\\ \dfrac{u_{\mathcal{S}}}{1+z}\end{pmatrix}\, .
\label{eq:z_DRIFT_BGO}
\end{equation}
In the above expression $\tau_{\calO}$ is the proper time of the observer, the first term
\begin{equation}
\Xi_{\rm Doppler}=\left[\dfrac{1}{1+z}\dfrac{\left(\ell^{\bm \mu}w_{\bm \mu}\right)|_{\mathcal{S}}}{\left(\ell^{\bm \mu} u_{\bm \mu}\right)|_{\mathcal{S}}}-\dfrac{\left(\ell^{\bm \mu}w_{\bm \mu}\right)|_{\mathcal{O}}}{\left(\ell^{\bm \mu} u_{\bm \mu}\right)|_{\mathcal{O}}}\right]
\label{eq:doppler}
\end{equation}
represents the Doppler effect caused by the four-acceleration $w^{\sigma}$ of the observer and the source, and $U$ is an $8 \times 8$ matrix given by the following combinations of the BGO
\begin{equation}
U=\begin{pmatrix}
- \WXL^{-1}{}\UD{\bm \nu}{\bm \rho}\WXX{}\UD{\bm \rho}{\bm \sigma} & \WXL^{-1}{}\UD{\bm \nu}{\bm \rho} \\
\WLL{}\UD{\bm \mu}{\bm \nu}\WXL^{-1}{}\UD{\bm \nu}{\bm \rho}\WXX{}\UD{\bm \rho}{\bm \sigma}-\WLX{}\UD{\bm \mu}{\bm \sigma} & - \WLL{}\UD{\bm \mu}{\bm \nu} \WXL^{-1}{}\UD{\bm \nu}{\bm \rho}
\end{pmatrix} \, .
\end{equation}
Let us notice that, even if the BGO formalism is independent of the specific choice of the frame used, the observables are dependent on this choice, as evident by the explicit dependence on $u^{\mu}_{\mathcal{O}}$, $u^{\mu}_{\mathcal{S}}$, $w^{\mu}_{\mathcal{O}}$ and $w^{\mu}_{\mathcal{S}}$ in Eqs. \eqref{eq:D_ang_BGO}-\eqref{eq:doppler}. Indeed, it is possible to transform locally between two different frames using an appropriate Lorentz transformation, but this modifies the observables introducing special relativistic effects like Doppler effect or aberration.

The reader can find the derivation of the Eqs.~\eqref{eq:D_ang_BGO}-\eqref{eq:z_DRIFT_BGO} in \cite{Grasso:2018mei, Korzynski:2018, Julius}). All the expressions in Eqs.~\eqref{eq:D_ang_BGO}, \eqref{eq:D_par_BGO} and \eqref{eq:z_DRIFT_BGO} are new with respect to the standard approach, in the sense that these observables are expressed within a new, unified framework. However, while for $D_{\rm ang}$ and $D_{\rm par}$ there already exist analogous formulas, where instead of the BGO we have the magnification and the parallax matrix (see \cite{Korzynski:2019oal} for the comparison), it did not exist a general formula for the redshift drift: Eq.~\eqref{eq:z_DRIFT_BGO} looks the same for every spacetime model under consideration. Instead, in the standard approach the redshift drift is calculated by taking the derivative with respect to the time coordinate of the definition of the redshift, and this depends on the specific form of the metric tensor and null-geodesic, the latter depending in turns on the symmetries that one gives to the initial conditions. This means that the equations to get the redshift drift in the standard approach look different for each specific model and/or configuration of the light rays\footnote{To be more precise, Eq.~\eqref{eq:z_drift_FLRW} is valid for the FLRW model only as well as Eq.~\eqref{eq:z_drift_Szekeres} is valid for the Szekeres model and geodesics along the symmetry axis only. Instead, Eq.~\eqref{eq:z_DRIFT_BGO} looks the same in both cases.}.




%%%%%%%%%%%%%%%%%%%%%%%%%%%%%%%%%%%%%%%%%%%%%%%%%%%%%%%%%%%%%%%%%%%%%%%%%%%%%%%%%%%%%%%%%%%%%%%%%%%%%%%%%%%%%%%

PRENDERE SPUNTO?
Theory of General Relativity presented in chapter $2$ provide a description of gravitation, as due to the geometry of spacetime structure, where all physically possible spacetimes correspond to solutions of Einstein's field equation (\ref{EinsteinField_eq}). So, by solving the Einstein's equation for a specific metric tensor, we are able to obtain complete predictions for that particular gravitational system. Unfortunately, Einstein's field equation represents a complicated coupled system of nonlinear partial differential equations,  so it is very hard to find a solution analytically. 

Due to these difficulties, historically two different approaches have arisen trying to find a solution of the Einstein's equation: the first-one is the \emph{perturbative approach} and the second-one is the \emph{numerical approach}.

\section{The Perturbative Approach}
The known solutions, such as the FLRW (\ref{FLRW_eq}) and the Schwarzschild (\ref{Schwarzschildmet_eq}) metrics discussed in the previous chapter, tell us a great deal about cosmology and gravitational fields. However these solutions have a high degree of symmetry\footnote{This is a necessary condition if we want to resolve the Einstein's equation. The high degree of symmetry of the metrics allow us to reduce in some way the differential equations system.} and thus it is difficult to compare them with real systems. For example, the FLRW metric alone can not explain how small inhomogeneities in the matter distribution evolve over time or we can define what happens if we have two black holes one close to other using the simple Schwarzschild metric. 

However, if the deviation from a known exact solution $g_{\mu \nu}$ is small, it makes sense to look for an approximate solution $ \tilde{g}_{\mu \nu}=g_{\mu \nu}+h_{\mu \nu}$ linearizing Einstein's equation in the perturbation $h_{\mu \nu}$. This is exactly the technique used by Einstein one hundred years ago to find the existance of a gravitational radiation, that was recently observed experimentaly by the LIGO and VIRGO interferometers.

In general the perturbative approach can be summarize with the following procedure: let us consider the Einstein's field equation for a perturbed metric $\tilde{g}_{\mu \nu}$ in vacuum. $\tilde{g}$ describe a spacetime metric that differs from a known metric $g_{\mu \nu}$ linearly. If the deviation from the exact metric is small, we can suppose that the new tensor $\tilde{g}$:
\begin{enumerate}
\item it depends differentiably on a parameter $\lambda$,
\item it is true that $\tilde{g}(\lambda)|_{\lambda=0}=g$
\end{enumerate}

With the previous conditions, small $\lambda$ corresponds to small deviations from $g$ and knowing $\tilde{g}(\lambda)$ we can find an exact solution of the Einstein's field equation. Writing the void Einstein's equation in terms of $\tilde{g}(\lambda)$ as
$$\mathcal{E}[\tilde{g}(\lambda)]=0$$
and differentiating it with respect to $\lambda$ we can find that
\begin{equation}
\dfrac{d}{d \lambda} \left( \mathcal{E}[\tilde{g}(\lambda)] \right) |_{\lambda=0}=0
\label{linear_eq}
\end{equation}


Is easy to see that the last equation (\ref{linear_eq}) is linear in $\frac{d \tilde{g}}{d \lambda}|_{\lambda=0} \equiv h $ and thus we can try to solve it respect to $\gamma$, instead of the original Einstein's equation.
Solving equation (\ref{linear_eq}), finally we have that $\tilde{g}(\lambda)=g+\lambda h$ provide a good approximation to the real metric for sufficient small $\lambda$.

To derive the equation\footnote{A complete and rigorous derivation of the perturbed approach goes beyond our purposes but a more rigorous derivation can be found in reference \cite{wald2010general, misner1973gravitation}} (\ref{linear_eq}) we first need to understand how to compute the Ricci tensor $R_{\mu \nu}(\lambda)$, for the metric $\tilde{g}(\lambda)$, in terms of quantities related to the unperturbed metric $g_{\mu \nu}$. Then we can differentiate this expression with respect to $\lambda$ and imposing that $\lambda=0$ we find the searched equation.
Denoting with $\tilde{\nabla}_{\mu}$ the derivative operator associated with $\tilde{g}(\lambda)_{\mu \nu}$, and $\nabla_{\mu}$ the derivative operator associated with $g_{\mu \nu}$, the difference between the two derivative operators is determined by the quantity $C^{\sigma}_{\mu \nu}(\lambda)$ define\footnote{This is similar to the Christoffel symbols because it express the connection between the two metrics $\tilde{g}_{\mu \nu }(\lambda)$ and $\tilde{g}_{\mu \nu }(0)=g_{\mu \nu }$} as
$$C^{\sigma}_{\mu \nu}(\lambda)=\dfrac{1}{2}\tilde{g}^{\sigma \delta}(\lambda)(\nabla_{\mu}\tilde{g}_{\nu \delta }(\lambda)+\nabla_{\nu}\tilde{g}_{\mu \delta}(\lambda)-\nabla_{\delta}\tilde{g}_{\mu \nu }(\lambda))$$ 
The perturbed Ricci tensor $\tilde{R}_{\mu \nu}$ is then 
\begin{equation}
\tilde{R}_{\mu \nu}= 2[ \nabla_{\rho}C^{\rho}_{\mu \nu}-\nabla_{\mu}C^{\rho}_{\rho \nu}  + C^{\sigma}_{\nu \mu}C^{\rho}_{\rho \sigma}-C^{\sigma}_{\nu \rho}C^{\rho}_{\mu \sigma}]
\label{riccipert_eq}
\end{equation}
where we have used the Einstein's field equation in vacuum $R_{\mu \nu}=0$ for $g_{\mu \nu}$ exact and known solution.
we are seeking the equivalent equation to (\ref{linear_eq}), so we have to differentiate (\ref{riccipert_eq}) respect to $\lambda$. Thus we obtain
\begin{equation}
\dot{\tilde{R}}_{\mu \nu}=2[ \nabla_{\rho}\dot{C}^{\rho}_{\mu \nu}-\nabla_{\mu}\dot{C}^{\rho}_{\rho \nu}]
\label{dotriccipert_eq}
\end{equation}
where $\dot{C}^{\rho}_{\mu \nu}=\frac{d C^{\rho}_{\mu \nu}}{d \lambda}|_{\lambda}= \frac{1}{2}g^{\rho \sigma}(\nabla_{\mu} h_{\nu \sigma}+\nabla_{\nu} h_{\mu \sigma}-\nabla_{\sigma} h_{\mu \nu})$. Substituting this in equation (\ref{dotriccipert_eq}) we finally have 

\begin{equation}
\nabla_{\mu} \nabla_{\nu} h +\nabla^{\rho} \nabla_{\rho}h_{\mu \nu} - \nabla^{\rho} \nabla_{\nu} h_{\mu \rho}+ \nabla^{\rho} \nabla_{\mu} h_{\nu \rho}=0
\label{linearfin_eq}
\end{equation}

where $h=h^{\mu}_{\mu}=g^{\mu \nu}h_{\mu \nu}$. This is a \emph{linear} system of second order differential equations in the perturbation tensor $h_{\mu \nu}$ and thus it is simpler\footnote{We can simplify further this equation noting that exist a gauge freedom in the choice of  the perturbation $h$. Choosing the transverse traceless gauge $\nabla^{\mu}h_{\mu \nu}=0$ and $h=h^{\mu}_{\mu}=g^{\mu \nu}h_{\mu \nu}=0$ the previous equation becomes
$$\nabla^{\rho}\nabla_{\rho} h_{\mu \nu} - 2R^{\rho \ \ \sigma}_{\ \mu \nu }h_{\rho \sigma}=0$$} than the original Einstein's field equation in the perturbed metric $\tilde{g}_{\mu \nu}$.

Although perturbation theory provides a powerful tool for facing the problems related with the Einstein field equation, the procedure is not painless. In fact it is very difficult to estimate the error involved in replacing $\tilde{g}$ by its simplified linear expression $g+\lambda h$ or, in general, we have no way of verifying if there are spurious solutions of equation (\ref{linearfin_eq}). 

Perturbation theory plays an important role in the study of structure formations. As we saw in Chapter $2$, the Standard Model of Cosmology ($\Lambda CDM$) is based on the metric FLRW that describes a homogeneous and isotropic Universe. So, the question how from a homogeneous and isotropic universe, would be possible to develop local structures like galaxies (see discussion in Chapter $1$) arises. A road to find a solution outlined when we was able to see that the cosmic microwave background radiation (CMB) (which initially appeared homogeneous and isotropic) showed small differences in temperature. This fact suggested that the structures we observe today must have originated from primordial inhomogeneities or, as they are poetically called, `` the seeds of galaxies '' Fig. \ref{CMB_fig}.
\begin{figure}
\centering
\includegraphics[scale=0.6]{figure/CMB.png} 
\caption{The Cosmic Microwave Background radiation (CMB) was observed by Penzias and Wilson in $1964$ (figure A). Some notable experiments in the list are COBE, which first detected the temperature anisotropies of the CMB, and showed that it had a black body spectrum. DASI, which first detected the polarization signal. WMAP and the \emph{Planck spacecraft} (figure D, image credit: ESA), which have produced the highest resolution all-sky map. Figure B show the polarization of the signal due to the Earth rotation, while in figure C the polarization is subtracted but we can distinguish a concentration of the radiation due to the electromagnetic emission from the galactic disc.(image credit: NASA)}
\label{CMB_fig}
\end{figure}


\section{The Numerical Approach}

Unlike the perturbation method, the numerical approach tries to find an exact solution to the Einstein's equation using numerical methods to calculate the differential equations involved. The numerical approach is designed to solve exactly the gravitational field equation deriving a metric that was not initially known\footnote{Unlike perturbative approach, which begins with a metric already known and adds a perturbation.}. To achieve this purpose, the numerical formulation must be implemented by powerful computers capable of performing a many  operations in a short time.
Thanks to the rapid development of modern computers capability, numerical simulations play a major role in modern cosmology and astrophysics\footnote{Nowadays, supercomputers are capable of following the evolution of billions to trillions of particles in cosmological volumes; however, upcoming observational missions place higher and higher demands on the realism required by cosmological codes.}.

 The field that applies computational-science techniques to the relativistic Universe is called \emph{Numerical Relativity}. The amount of detail required to simulate the large-scale Universe is formidable: the main purpose of numerical relativists is to tame this task and build more and more realism into the simulations. 
 Numerical relativity is applied to many areas, such as cosmological models, critical phenomena, coalescence of black holes and neutron stars, etc. In any of these cases, Einstein's equations has to be rewrite like a Cauchy problem, splitting the spacetime into three dimensional space on one hand and in time on the other hand.
There exist several different approaches to set the Einstein field equations like a Cauchy problem; here we will present the \emph{$3+1$ formalism}, which allow us to split the spacetime into a three dimensional space on one hand and time on the other.  
 
Let us consider a spacetime $M$ with metric $g_{\mu \nu}$ and let us assume that the spacetime $M$ can be foliated into a family of non-intersecting spacelike $3-$surfaces $\Sigma_t$, which arise as the level surfaces of a scalar function $t$ that can be interpreted as a \emph{global time function} (\ref{gaugefig}). Such a foliation of spacetime into spatial hypersurfaces is also called \emph{synchronization}.
\begin{figure}[htbp]
  \centering
  \includegraphics[scale=0.4]{figure/slices.png}%
  \caption{Two adjacent  spacelike hipersurfaces. The figure show the definitions of the lapse function $\alpha$ and the shift vector $\beta^i$. It is clear that, setting a coordinate base on the hipersurfaces, the gauge functions are closely depended by that choice.}
  \label{gaugefig}
\end{figure}

The geometry of these hypersurfaces could be described by introducing the following quantities:
\begin{enumerate}
\item the three dimensional metric $\gamma_{ij}$ that measures proper distances within the hypersurfaces $$dl^2=\gamma_{ij}dx^i dx^j$$
\item the \emph{lapse function} of proper time $d\tau$ between two adjacent hypersurfaces measured by normal observers to the hypersurfaces (the so-called \emph{normal or Eulerian observers}): $d\tau= \alpha(t,\vec{x}) dt$
\item The relative velocity between the Eulerian observers and the lines of constant spatial coordinates, called \emph{shift vector} $\beta(t, \vec{x}) $: $x^i_{t+dt}=x^i_t- \beta(t, x^j) dt$
\end{enumerate}

Using these quantities, the usual line element $ds^2=g_{\mu \nu}dx^{\mu}dx^{\nu}$ take the following form

$$ds^2=(-\alpha^2+\beta_i \beta^i)dt^2+2 \beta_i dt dx^i + \gamma_{ij}dx^i dx^j$$

Notice that both the way in which we foliate the spacetime and also the way in which the spatial coordinate propagates from a hypersurface to another are not uniquely defined. This because the lapse functions $\alpha$ and shift vector $\beta^i$ are freely specifiable functions that carry information about the choice of coordinate system. For this reason  $\alpha$ and $\beta^i$ are known as the \emph{gauge functions}.

We can use the normal vector $n^{\mu}$ to the spatial hypersurfaces to define the gauge functions and the three dimensional metric $\gamma_{ij}$ in a more formal way. Let us start with the three dimensional metric $\gamma_{ij}$, which is simply defined as the metric induced on each hypersurface $\Sigma$. We can define the projection tensor 
$$P^{\mu}_{\nu}=\delta^{\mu}_{\nu} + n^{\mu} n_{\nu}$$
as the tensor that project an arbitrary spacetime vectors $v$ into the space of vectors tangent to the surface\footnote{Applying $P^{\mu}_{\nu}$ on the normal vector $n^{\nu}$ we simply get $$P^{\mu}_{\nu} n^{\nu}=\delta^{\mu}_{\nu} n^{\nu} + n^{\mu} n_{\nu} n^{\nu}=$$
$$=n^{\mu} + n^{\mu} (\vec{n} \cdot \vec{n})=n^{\mu} - n^{\mu} =0$$}.

 The corresponding tensor with one index lowered define the \emph{spatial metric} on the hypersurface 
$$\gamma_{\mu \nu}=g_{\mu \nu} + n_{\mu} n_{\nu}$$

Notice that, written in this way, it is a full four-dimensional tensor so the adjective ``spatial'' is incorrect. However in an adapted reference frame, the only non-zero components of this tensor are the components of the metric tensor in the surface $\gamma_{ij}$.

Let us consider now the lapse function $\alpha$: we can redefine it using the global time function as $\alpha=(- \nabla t \cdot \nabla t)^{-\frac{1}{2}}$. The unit normal vector can be defined in terms of $t$ and $\alpha$ as
\begin{equation}
n^{\mu}=- \alpha \nabla^{\mu} t
\label{alpha}
\end{equation}
where the minus sign is there to guarantee that $\vec{n}$ is future pointing.

Also for the shift vector we start considering the previous definition 
$$x^i_{t+dt}=x^i_t- \beta^i dt$$
Using the previous definition for $n^{\mu}$, we have

\begin{equation}
\beta^i=- \alpha (\vec{n} \cdot \nabla x^{i})=- \alpha (n^j \nabla_j x^{i})
\label{beta}
\end{equation}

The $\beta^i$ can be interpreted as the scalar components of a $4$-vector $\beta^{\mu}=(0, \beta^i)$, which is orthogonal to $\vec{n}$.

Using the quantities previously defined, we can construct a \emph{time vector} $\vec{t}$
\begin{equation}
t^{\mu}=\alpha n^{\mu} + \beta^{\mu}
\label{ti}
\end{equation}  
which is the tangent vector to the lines of constant spatial coordinates.
It is trivial to observe\footnote{From (\ref{ti}) we have $t^{\mu}n_{\mu}= \alpha n^{\mu} n_{\mu} + \beta^{\mu} n_{\mu}= - \alpha$. This, using the defining equation of $n^{\mu}$ (\ref{alpha}), imply that $\vec{t}$ satisfy the condition $t^{\mu} \nabla_{\mu}t=1$. We then find that the shift is nothing more than the projection of $\vec{t}$ onto the spatial hypersurface $\beta_{\mu}=\gamma_{\mu \nu}t^{\nu}$.} that the shift $\beta^i$ and the lapse $\alpha$ can be introduced in a completely coordinate-independent way 
$$t^{\mu}n_{\mu}= - \alpha \ \ \ \ \ \ \ \ \ \ \ \beta_{\mu}=\gamma_{\mu \nu}t^{\nu}$$
choosing a vector field $t^{\mu}$ which satisfy the condition $t^{\mu} \nabla_{\mu}t=1$. 
Notice that we have not a restriction on the type of the vector field $t^{\mu}$, so it could be a time-like, space-like or null-like vector field: a space-like $t^{\mu}$ vector field would correspond to a superluminal motion of the coordinate, but it is only an effect due to that particular choice of coordinate system and the coordinates can be chosen freely.

Now let us observe that the curvature of spacetime manifests itself on the hypersurfaces $\Sigma_i$ as the changing of the normal vector orientation when it is parallel transported. So the covariant derivative (along the surface) of the unit normal vector $\vec{n}$ projected onto the surface is a measure of how much the orientation of the surface changes from place to place, i.e. a measure of its curvature as seen from the outside. Thus we can define the \emph{Extrinsic Curvature} $K_{\mu \nu}$ as the tensor
\begin{equation}
K_{\mu \nu}=-P^{\alpha}_{\mu} \nabla_{\alpha} n_{\nu}=-(\nabla_{\mu} n_{\nu}+n^{\mu}n_{\alpha}\nabla_{\alpha} n_{\nu})
\label{Defextrinsic_eq}
\end{equation}

The extrinsic curvature tensor $K_{\mu \nu}$ is clearly a purely spatial tensor, in fact it only has components tangent to the hypersurface $n^{\mu} K_{\mu \nu}=n^{\nu} K_{\mu \nu}=0$, implying that $K^{00}=K^{0i}=0$. Thus we can consider only the spatial components of the tensor and we can see that it is a symmetric%\footnote{The symmetry of the extrinsic curvature tensor is not trivial: the ``spatiality'' of $K_{\mu \nu}$ is  guaranteed by the projection of $\nabla_{\mu} n_{\nu}$ on the hypersurface. This imply that the gradient of the normal vector is necessarily orthogonal to it  $n_{\mu}\nabla_{\nu} n_{\mu}=0$. However, in general the $\nabla_{\mu} n_{\nu}$ is not symmetric, so we have that $n_{\mu}\nabla_{\mu} n_{\nu} \neq 0$, unless the normal lines are geodesic. The reason for this is that $n^{\mu}$ is a unitary vector and thus in general is not equal to the gradient of the time function $t$, except when the lapse is unity (see Eq. \ref{alpha}). What we want to proof is that once we project onto the hypersurface, it turns out that $P^{\alpha}_{\mu} \nabla_{\alpha}n_{\nu}$ is indeed symmetric (so $K_{\mu \nu}$). In order to see this, consider the congruence of timelike geodesics orthogonal to the hypersurface $\Sigma$ with unit tangent vector $\vec{\xi}$. Near $\Sigma$ let us consider a new foliation of spacetime given by a time function $\tilde{t}$ such that $\xi_{\mu}=\nabla_{\mu}\tilde{t}$. Since $\xi_{\mu}$ is the gradient of a scalar function, then we will have $\nabla_{\mu}\xi_{\nu}=\nabla_{\nu}\xi_{\mu}$. Moreover, $\nabla_{\mu}\xi_{\nu}$ will be purely spatial without the need to project it since $\vec{\xi}$ is tangent to the timelike geodesics. Now, the vector field $\vec{n}$ will generally not coincide with $\vec{\xi}$ outside of $\Sigma$, but it will coincide within, so its derivatives along directions tangential to the hypersurface must be equal to those of $\vec{\xi}$ that is $$P^{\alpha}_{\mu}\nabla_{\alpha} n_{\nu}=P^{\alpha}_{\mu}\nabla_{\alpha} \xi_{\nu}=\nabla_{\mu} \xi_{\nu}=\nabla_{\nu} \xi_{\mu}=P^{\alpha}_{\nu}\nabla_{\alpha} \xi_{\mu}=P^{\alpha}_{\nu}\nabla_{\alpha} n_{\mu}$$} 
 tensor $K_{\mu \nu}=K_{\nu \mu}$. 
 
As we can see from equation (\ref{Defextrinsic_eq}), we have defined the extrinsic curvature as the parallel transport of the normal vector projected onto the hypersurface. However, it easy to see that $K_{\mu \nu}$ can be defined as the Lie derivative of the spatial metric along the normal direction
\begin{equation}
K_{\mu \nu}= -\dfrac{1}{2} \mathcal{L}_{\vec{n}}\gamma_{\mu \nu}
\label{Klie_eq}
\end{equation}
This came from the definition of the Lie derivative

$$\mathcal{L}_{\vec{n}}\gamma_{\mu \nu} = n^{\alpha} \nabla_{\alpha}\gamma_{\mu \nu} +\gamma_{\mu \alpha} \nabla_{\nu} n^{\alpha} + \gamma_{\nu \alpha} \nabla_{\mu} n^{\alpha}$$
$$=n^{\alpha} \nabla_{\alpha}(n_{\mu}n_{\nu}) +g_{\mu \alpha} \nabla_{\nu} n^{\alpha} + g_{\nu \alpha} \nabla_{\mu} n^{\alpha}$$
$$=n^{\alpha}n_{\mu} \nabla_{\alpha}n_{\nu} + n^{\alpha}n_{\nu} \nabla_{\alpha}n_{\mu} + \nabla_{\nu} n_{\mu} +  \nabla_{\mu} n^{\nu}$$
$$=(\gamma^{\alpha}_{\mu}-g^{\alpha}_{\mu}) \nabla_{\alpha}n_{\nu} + (\gamma^{\alpha}_{\nu}-g^{\alpha}_{\nu}) \nabla_{\alpha}n_{\mu} + \nabla_{\nu} n_{\mu} +  \nabla_{\mu} n^{\nu}$$
$$=\gamma^{\alpha}_{\mu}\nabla_{\alpha}n_{\nu} + \gamma^{\alpha}_{\nu}\nabla_{\alpha}n_{\mu}=-2K_{\mu \nu}$$

We must remember that our purpose is to write the Einstein's equation as a Cauchy problem. Therefore, let us try to use the quantities previously defined to write the gravitational field equation (\ref{EinsteinField_eq}) as a set of dynamical equations and boundary conditions.

In order to achieve this goal, let us see that, since $\vec{n}$ is normal to the hypersurface, it turns out that for any scalar function $\phi$ we have 
$$\mathcal{L}_{\vec{n}}\gamma_{\mu \nu}=\dfrac{1}{\phi}\mathcal{L}_{\phi\vec{n}}\gamma_{\mu \nu}$$  
Using this and taking as scalar function the lapse function $\alpha$, equation (\ref{Klie_eq}) become
$$K_{\mu \nu}=-\dfrac{1}{2 \alpha}\mathcal{L}_{\alpha\vec{n}}\gamma_{\mu \nu}=-\dfrac{1}{2 \alpha}(\mathcal{L}_{\vec{t}}-\mathcal{L}_{\vec{\beta}})\gamma_{\mu \nu}$$ 
where we used (\ref{ti}). 
Taking only the spatial components and noticing that in the adapted coordinate system $\mathcal{L}_{\vec{t}}= \partial_{t}$, we finally find
$$\partial_{t}\gamma_{i j} -\mathcal{L}_{\vec{\beta}} \gamma_{i j}=-2 \alpha K_{i j}$$
\begin{equation}
\implies \ \ \ \partial_{t}\gamma_{i j} =-2 \alpha K_{i j} + D_i \beta_j +  D_j \beta_i
\label{timegamma_eq}
\end{equation}
where $D_i$ is the \emph{$3$-dimensional covariant derivative}, which is defined as the projection of the $4$-dimensional covariant derivative $D_{\mu}= P^{\alpha}_{\mu} \nabla_{\alpha}$.

Equation (\ref{timegamma_eq}) is the evolution equation for the three-metric $\gamma_{ij}$: in order to close the system, we need the evolution equation for the extrinsic curvature. In order to find this equation, we need to rewrite the Einstein's equation in the $3+1$ formalism considering contractions of (\ref{EinsteinField_eq}) with the normal vector $\vec{n}$ and the projector tensor $P^{\mu}_{\nu}$. However, a complete and rigorous derivation of these quantities is beyond the scope of this thesis, so we will limit to state the final results (see  \cite{wald2010general} for more rigorous derivations).
The following are the fundamental relations on which rely all the derivation:
\begin{enumerate}
\item \emph{The Gauss-Codazzi equation}: \begin{equation}
P^{\mu}_{\alpha}P^{\nu}_{\beta}P^{\rho}_{\gamma}P^{\sigma}_{\delta}R_{\mu \nu \rho \sigma}=^{(3)}R_{\alpha \beta \gamma \delta}+K_{\alpha \gamma}K_{\beta \delta}-K_{\alpha \delta}K_{\beta \gamma}
\label{GaussCodaz_eq}
\end{equation}
where $^{(3)}R_{\alpha \beta \gamma \delta}$ is the \emph{intrinsic} $3$-dimensional Riemann tensor of the hypersurfice;
\item \emph{The Gauss-Mainardi equation}:\begin{equation}
P^{\mu}_{\alpha}P^{\nu}_{\beta}P^{\rho}_{\gamma}n^{\sigma}R_{\mu \nu \rho \sigma}=D_{\beta}K_{\alpha \gamma}-D_{\alpha}K_{\beta \gamma}
\label{GaussMainardi_eq}
\end{equation} 
\item The projection of the Riemann tensor contracted twice with $\vec{n}$:\begin{equation}
P^{\mu}_{\beta}P^{\nu}_{\gamma}n^{\rho}n^{\sigma}R_{\mu \nu \rho \sigma}=\mathcal{L}_{\vec{n}}K_{\beta \gamma}+K_{\beta \rho}K^{\rho}_{\gamma}+ \dfrac{1}{\alpha}D_{\beta}D_{\gamma} \alpha
\end{equation}
\end{enumerate}

Let us start considering the last relation, noticing that it involves both the lapse function $\alpha$ and the Lie derivative of the extrinsic curvature along the normal direction, which clearly correspond to evolution in time. Using equation (\ref{GaussCodaz_eq}), we find
$$\mathcal{L}_{\vec{t}}K_{\mu \nu}-\mathcal{L}_{\vec{\beta}}K_{\mu \nu}=-D_{\mu}D_{\nu}\alpha+\alpha(P^{\rho}_{\mu}P^{\sigma}_{\nu}R_{\rho \sigma}+^{(3)}R_{\mu \nu}+KK_{\mu \nu}-2K_{\mu \lambda}K^{\lambda}_{\nu})$$
Projecting the Einstein's equation and substituting into this (concentrating on the spatial components), we finally find
$$\partial_{t}K_{i j}-\mathcal{L}_{\vec{\beta}}K_{i j}=-D_{i}D_{j}\alpha+\alpha(^{(3)}R_{i j}+KK_{i j}-2K_{il}K^{l}_{j})+ 4\pi \alpha[\gamma_{ij}(S-\rho)-2S_{ij}]$$
where $S$ is the trace of the spatial stress tensor $S_{\mu \nu}=P^{\alpha}_{\mu}P^{\beta}_{\nu}T_{\alpha \beta}$, while $\rho=n^{\mu}n^{\nu}T_{\mu \nu}$. Expanding the Lie derivative along the shift vector we finally find:
\begin{equation}
\partial_{t}K_{i j}=\beta^{l} \partial_{l}K_{i j}+K_{li} \partial_{j}\beta^{l} + K_{lj} \partial_{i}\beta^{l} -D_{i}D_{j}\alpha+\alpha(^{(3)}R_{i j}+KK_{i j}-2K_{il}K^{l}_{j})+ 4\pi \alpha[\gamma_{ij}(S-\rho)-2S_{ij}]
\label{timeK_eq}
\end{equation}

These equations give the dynamical equations of the six independent components\footnote{Using the symmetry property of the extrinsic curvature tensor, it is easy to see that $K_{i j}$ has only six independent components.} of the extrinsic curvature $K_{ij}$. Summing this components to those of the space metric $\gamma_{ij}$ ($\gamma_{ij}$ is represented as a symmetric $3 \times 3$ matrix, thus it has six independent components), we have $12$ independent components. Since the Einstein equations can be written as a system of $20$ first-order differential equations,  we still need $8$ independent components to close the system. 
A help comes from the Gauss-Codazzi (\ref{GaussCodaz_eq}) and the Gauss-Mainardi (\ref{GaussMainardi_eq}) equations: we can combine the projected Einstein's field equations
$$P^{\mu \rho}P^{\nu \sigma}R_{\mu \nu \rho \sigma}=(g^{\mu \rho}+n^{\mu}n^{\rho})(g^{\nu \sigma}+n^{\nu}n^{\sigma})R_{\mu \nu \rho \sigma}$$
$$=R+2n^{\mu}n^{\nu}R_{\mu \nu}$$
$$=2 n^{\mu}n^{\nu}G_{\mu \nu}=2 n^{\mu}n^{\nu}(8 \pi T_{\mu \nu}) $$

with the Gauss-Codazzi relations $P^{\rho \mu}P^{\sigma \nu}R_{\rho \sigma \mu \nu}= ^{(3)}R+K^{2}-K_{\mu \nu}K^{\mu \nu}$, we then find
$$^{(3)}R+K^{2}-K_{\mu \nu}K^{\mu \nu}=P^{\rho \mu}P^{\sigma \nu}R_{\rho \sigma \mu \nu}=2 n^{\mu}n^{\nu}G_{\mu \nu}=16 \pi \rho$$
Summing up, through the Einstein equation, we have found
\begin{equation}
^{(3)}R+K^{2}-K_{\mu \nu}K^{\mu \nu}=16 \pi \rho
\label{const1_eq}
\end{equation}  
where $\rho$ corresponds to the local energy density.

Consider now the mixed contraction of the Einstein tensor
$$P^{\lambda \mu}n^{\nu}G_{\mu \nu}=P^{\lambda \mu}n^{\nu}R_{\mu \nu}$$
Using this result, the Gauss-Mainardi equations (\ref{GaussMainardi_eq}) then become
$$\gamma^{\lambda \mu}n^{\nu}G_{\mu \nu}=D^{\lambda}K-D_{\mu}K^{\lambda \mu}$$

Through the Einstein field equation, we find
\begin{equation}
D_{\mu}(K^{\lambda \mu}-\gamma^{\lambda \mu} K)=8 \pi j^{\lambda}
\label{const2_eq}
\end{equation}
where we have defined $j^{\mu}=-P^{\mu \nu}n^{\rho}T_{\nu \rho}$, which corresponds to the momentum density.

Notice that these four equations involve no explicit time derivatives, so they are not evolution equations but rather constraints that must be satisfied at all times. Moreover they are also completely independent of the gauge functions $\alpha$ and $\beta^{i}$, and this means that equations (\ref{const1_eq}) and (\ref{const2_eq}) are constraints that refer purely to a given hypersurface.

The presented $3+1$ decomposition of the space-time is known as the \emph{York formulation of the ADM Evolution Equations} (Arnowitt-Deser-Misner equations) \cite{BrumanteShapiro}:
\begin{equation}
\partial_{t} \gamma_{i j}=-2 \alpha K_{i j} + \mathcal{L}_{\beta}\gamma_{i j}
\label{ADMdin}
\end{equation}
$$ \partial_{t} K_{i j}= -D_i D_j \alpha + \alpha [R_{ij}-2 K^k_i K_{jk}+K K_{ij}] + \mathcal{L}_{\beta}K_{ij}$$

\begin{equation}
\mathcal{H}=R- K_{ij}K^{ij}+K^2=16 \pi \rho
\label{ADMconst}
\end{equation}
$$\mathcal{M}_i=D^jK_{ij}-D_i K= 8 \pi j_i$$
where the (\ref{ADMdin}) are the evolution equations for the spatial metric and the extrinsic curvature, while the (\ref{ADMconst}) are the \emph{Hamiltonian} and \emph{Momentum constraints} respectively. However, we have not obtained constraints on the choice of the \emph{gauge functions} lapse $\alpha(t, x^i)$ and shift $\beta^i(t, x^j)$. The freedom in specifying the \emph{gauge functions} derive from the original tensor nature of the Einstein equation. The freedom in choosing the gauge variables allows us to choose the coordinates in a way that simplifies the evolution equations for treating many of the most important physical and astrophysical problems.% carry information about the choice of coordinate system (Fig.\ref{gauge})
Let us go back to the Einstein equation: we said that it is a set of $10$ second-order differential equations of the metric tensor that is equivalent to a set of $20$ first-order differential equations of the metric components. In the  standard $ADM$ we have in total $16$ conditions: $12$ from the evolution equations (\ref{ADMdin}) plus $4$ conditions (\ref{ADMconst}). The $4$ missing equations represent the gauge freedom inherent to General Relativity and correspond to the freedom in specifying the gauge functions $\alpha$ and $\beta^i$. 

The metric and the extrinsic curvature $(\gamma_{\mu \nu}, K_{\mu \nu})$ can be considered as the equivalent of positions and velocities in classical mechanics - they measure the ``instantaneous'' state of the gravitational field, and form the fundamental variables in our initial value formulation. Moreover, we have derived the evolution equation for $\gamma_{ij}$ from considerations purely geometric while the evolution equation for $K_{ij}$ comes from the Einstein's equation. This means that the three-metric contain information about the kinematics of the gravitational system while the extrinsic curvature $K_{ij}$ contain information about the dynamics of the gravitational system.

%%%%%%%%%%%%%%%%%%%%%%%% PROBLEMI ADM %%%%%%%%%%%%%%%%%%%%%%%%%

Although the $ADM$ formulation has served our purpose of writing the equation of Einstein as a Cauchy problem just fine, there are some problems associated with this kind of $3+1$ formulation. The main issue related to the $ADM$ formulation, and also the main reason why it  was initially abandoned to split the Einstein equations, is that the $ADM$ evolution equations are \emph{weakly hyperbolic}.
To appreciate this statement we need a small digression: a large class of systems in physics can be written as a Cauchy problem, which evolution equations are written as
$$\partial_t \vec{f}(\vec{r}, t) + A \cdot \nabla \vec{f}(\vec{r}, t) = \vec{S}(\vec{r}, t)$$ 
where $\vec{f}(\vec{r}, t)$ is a vector field that express the state of the system, $\vec{S}(\vec{r}, t)$ is a source term and $A$ is a matrix of coefficients. The properties of the system are determined by the features of $A$ and $\vec{S}(\vec{r}, t)$, and in particular we have that 
\begin{enumerate}
\item if the elements of $A$ and $\vec{S}$ are constants, then the system is \emph{linear};
\item if the elements of $A$ and $\vec{S}$ are functions of the spacetime, then the system is \emph{linear} with variable coefficients;
\item if the elements of $A$ are functions of the vector field $A=A(\vec{f})$ then the system is \emph{non-linear}
\end{enumerate}

In order to correctly interpret the physical system, we must verify the \emph{well-posedness} of the Cauchy problem that describe the system. The notion of well-posedness of the Cauchy problem is closely related to the nature of the given system of differential equations, and in particular with the criteria of \emph{hyperbolicity}
\begin{definition}
\centering
\emph{ A system of differential equations is said to be \textsc{(strongly) hyperbolic} if the Cauchy problem is well posed.}
\end{definition}
The following theorem is valid:
\begin{theorem}
If the coefficient matrices $A$ of the equations system is a constant matrices then a necessary condition, in order that the system be strongly hyperbolic, is that:
\begin{enumerate}
\item $A$ has N real eigenvalues $\{ \lambda_1 \cdots \lambda_N \}$;
\item  $A$ is diagonalizable.
\end{enumerate}
\end{theorem}

Thanks to this theorem, we can define other ``hyperbolicity degrees'' to characterize our Cauchy problem:
\begin{definition}
\centering
\emph{ A system of differential equations is said to be \textsc{symmetric hyperbolic} if the coefficients matrix $A$ is symmetric $A=A^T$}
\end{definition}
\begin{definition}
\centering
\emph{ A system of differential equations is said to be \textsc{strictly hyperbolic} if the eigenvalues $\lambda_i$ of the coefficients matrix $A$ are real and distinct.}
\end{definition}
\begin{definition}
\centering
\emph{ A system of differential equations is said to be \textsc{weakly hyperbolic} if the coefficients matrix $A$ is not diagonalizable.}
\end{definition}

So it follows that a weakly hyperbolic system is not guaranteed to be well-posed and therefore the numerical solution of such equations led to the growth of unstable models.
Analyzing the $ADM$ equations, from the mixed derivatives in the Ricci tensor for the solution of the extrinsic curvature\footnote{The evolution equations for the extrinsic curvature is
$$\partial_t K_{ij}= D_i D_j \alpha + \alpha (R_{i j}+\cdots)$$
the term $R_{ij}$ contains $\gamma_l^i \gamma_m^j \partial_i \partial_j$, which led to diagonal second derivatives but also mixed second derivatives ($\partial_x \partial_y$, $\partial_x \partial_z$, $\partial_z \partial_y$).}, we can infer that (\ref{ADMdin}) are weak hyperbolic equations. 
Therefore, in order to continue, we can follow two paths: abandon the $ADM$ formulation, or modify it in such a way that the $ADM$ equations become hyperbolics.

We will choose the second approach, but first it is worth noting that the just presented York formulation of the ADM equation is different from the original ADM formulation. The original ADM formulation uses as dynamical variables the spatial metric $\gamma_{ij}$ and its canonical conjugate momentum $\pi_{ij}$, which are related to the extrinsic curvature as

$$K_{ij}=-\dfrac{1}{\sqrt{\gamma}}(\pi_{i j}- \dfrac{1}{2} \gamma_{i j} \pi)$$

Rewriting the original $ADM$ equations in terms of the extrinsic curvature, we notice that they are different from the formulation \textit{\'a la} York for a term proportional to the Hamiltonian constraint. So the evolution equations of two formulations are \emph{physically equivalent} (since they only differ by the addition of a term proportional to the Hamiltonian constraint), but they are not \emph{mathematically equivalent}.
This consideration take us to a fundamental observation: the evolution equations are highly non-unique since we can always add to them arbitrary multiples of the constraints. The different systems of evolution equations will still coincide in the physical solutions, but might differ dramatically in their mathematical properties.
From now on we will use the ADM equations \textit{\'a la} York, which we will call \emph{Standard ADM}.\\


The non-uniqueness of the $3 + 1$ decomposition allow us to define new formulations of evolution equations (\ref{ADMdin}), physically equivalents to the standard $ADM$ equations but with better stability properties for numerical simulations. 
In particular, we will present a specific formulation which is particularly adapts for numerical simulations: the \emph{Conformal ADM}, or (the more commonly used name) \emph{Baumgarte, Shapiro, Shibata, Nakamura, Oohara and Kojima formulation} ($BSSNOK$).

The starting point of this reformulation is the \emph{conformal rescaling of the spatial metric}
\begin{equation}
\tilde{\gamma}_{i j}= \psi^{-4} \gamma_{ij}
\label{gammaConf_eq}
\end{equation}
$\psi$ is a conformal factor that can be chosen in a number of different ways.
In particular the $BSSNOK$ formulation choose the conformal factor in such a way that the conformal metric $\tilde{\gamma}_{ij}$ has unit determinant during all the evolution, i. e. $\psi^4= \gamma^{\frac{1}{3}}$ with $\gamma$ the determinant of the spatial metric. With this choice we have that is
\begin{equation}
\psi= \gamma^{\frac{1}{12}}
\label{ConfFact}
\end{equation}
Now, using the evolution equation for the spatial metric (\ref{ADMdin}), we find that the evolution equation for the determinant of the metric is
$$\partial_t \gamma = \gamma (-2 \alpha K + 2 D_i \beta^i)=-2 \gamma(\alpha K - \partial_i \beta^i)+ \beta^i \partial_i \gamma$$
which, using the (\ref{ConfFact}), implies
$$\partial_t \psi =-\dfrac{1}{6} \psi(\alpha K - \partial_i \beta^i)+ \beta^i \partial_i \psi $$
Written in this form it is easy to see that, redefining the conformal factor $\psi$ as $\phi=\log \psi= \frac{1}{12} \log \gamma$ so that $\tilde{\gamma}_{ij}=e^{-4 \phi}\gamma_{ij}$, the previous equations are
\begin{equation}
\partial_t \phi =-\dfrac{1}{6} (\alpha K - \partial_i \beta^i)+ \beta^i \partial_i \phi 
\end{equation}  

Another feature of the $BSSNOK$ formulation is the separation of the extrinsic curvature into its trace $K$ and its tracefree part
\begin{equation}
A_{ij}=K_{ij}- \dfrac{1}{3} \gamma_{ij}K
\end{equation}
and the application of a conformal rescaling to the traceless extrinsic curvature
\begin{equation}
\tilde{A}_{ij}= \psi^{-4}A_{ij}=e^{-4 \phi}A_{ij}
\end{equation}
The traceless condition follows from
$$\tilde{A}_{ij}\gamma^{ij}=\tilde{A}^i_i= \psi^4(K^i_i - \dfrac{1}{3}\gamma_{ij}\gamma^{ij}K)=0$$

In addition to the previous quantities, the BSSNOK formalism also define the \emph{Cristoffel symbols of conformal space metric} $\tilde{\Gamma}^i_{jk}$

\begin{equation}
\tilde{\Gamma}^i_{jk}=\Gamma^i_{jk}-2(\delta^i_j \partial_k \phi + \delta^i_k \partial_j \phi - \gamma_{jk}\gamma^{il} \partial_l \phi)
\end{equation}

and the \emph{Gammas}, to separate the mixed derivatives:
\begin{equation}
\tilde{\Gamma}^i= \tilde{\gamma}^{jk} \tilde{\Gamma}^i_{jk}=\tilde{\gamma}^{ij} \tilde{\gamma}^{kl}\partial_l \tilde{\gamma}_{jk}
\end{equation}
this implies $\tilde{\Gamma}^i= 2\tilde{\gamma}^{ij} \partial_j \phi+ e^{4 \phi} \tilde{\Gamma}^i_{jk}$.

The system of evolution equations then takes the form

\begin{equation}
\dfrac{d}{dt} \tilde{\gamma}_{ij}= -2 \alpha \tilde{A}_{ij}
\end{equation}
\begin{equation}
\dfrac{d}{dt} \phi= -\dfrac{1}{6} \alpha K
\end{equation}
\begin{equation}
\dfrac{d}{dt} \tilde{A}_{ij}= e^{-4\phi}\left\{ -D_i D_j \alpha + \alpha R_{ij}+ 4 \pi \alpha [\gamma(S-\rho)-2 S_{ij}]\right\} + \alpha(K \tilde{A}_{ij}-2\tilde{A}_{ik}\tilde{A}^k_j)
\end{equation}
\begin{equation}
\dfrac{d}{dt} K= -D_i D^i \alpha + \alpha( \tilde{A}_{ij}\tilde{A}^{ij}+ \dfrac{1}{3}K^2)+ 4 \pi \alpha (S+\rho)
\end{equation}
where we have posed $\frac{d}{dt}= \partial_t - \mathcal{L}_{\vec{\beta}}$, $TF$ denotes the tracefree part of the expression inside the brackets and, in the evolution equation for $K$, we have used the Hamiltonian constraint in order to eliminate the Ricci scalar:
\begin{equation}
R=K_{ij}K^{ij}-K^2+ 16 \pi \rho= \tilde{A}_{ij}\tilde{A}^{ij}+ \dfrac{2}{3}K^2 + 16 \pi \rho
\end{equation}

In the evolution equations for $\tilde{A}_{ij}$ and $K$ there appear covariant derivatives of the lapse function with respect to the physical metric $\gamma_{ij}$. This can be calculated by using the Cristoffel symbols of conformal space metric previously defined, since $\tilde{\Gamma}^i= 2\tilde{\gamma}^{ij} \partial_j \phi+ e^{4 \phi} \tilde{\Gamma}^i_{jk}$.
From this we can infer the evolution equation for the $\tilde{\Gamma}^i$:
\begin{equation}
\dfrac{d}{dt}\tilde{\Gamma}^i= \tilde{\gamma}^{jk}\partial_j \partial_k \beta^i + \dfrac{1}{3}\tilde{\gamma}^{ij}\partial_j \partial_k \beta^k - 2 \tilde{A}^{ij} \partial_j \alpha
+ 2 \alpha(\tilde{\Gamma}^i_{jk}\tilde{A}^{jk}+ 6 \tilde{A}^{ij} \partial_j \phi - \dfrac{2}{3}\tilde{\gamma}^{ij} \partial_j K -8 \pi \tilde{j}^i) 
\end{equation}
These equations are those normally referred to as the $BSSNOK$ formulation and, when are compared with the $ADM$ equations, we have five more evolution equations to consider ($\partial_t \phi$, $\partial_t K$ and $\partial_t \tilde{\Gamma}^i$) but now the system is hyperbolic and indeed well behaved. Looking more closely to the degrees of freedom, we have $15$ variables (we have found that there are $17$ variables: $\phi$, $K$ and $\tilde{\Gamma}^i$ are $5$ variables, plus $12$ from $\tilde{\gamma}_{ij}$ and $\tilde{A}_{ij}$; however we have that $\tilde{\gamma}_{ij}$ and $\tilde{A}_{ij}$ are traceless or with a known trace, so we have $2$ conditions which bring the number of variables to $15$.)

The constraint equations are then
\begin{equation}
^{(3)}R-\tilde{A}_{ij}\tilde{A}^{ij}+\dfrac{2}{3}K^2=16 \pi \rho
\label{BSSNOKhamilton_eq}
\end{equation}
\begin{equation}
D_j(\tilde{A}^{ij}-\dfrac{2}{3}\psi^4 \tilde{\gamma}^{ij}K)=8 \pi j^i
\label{BSSNOKmomentum_eq}
\end{equation}
where $\rho$ and $j^i$ are the energy and momentum densities seen by the normal observers 
\begin{equation}
\rho=n^{\mu}n^{\nu}T_{\mu \nu}
\end{equation}
\begin{equation}
j^i=-\gamma^{i \mu}n^{\nu}T_{\mu \nu}
\end{equation}

The $BSSNOK$ formulation allows us to write six out of the ten Einstein field equations as the true evolution equations of the spacetime and the remaining four equations as the constraints that must be satisfied at all times. The existence of the constraints implies that, if we want to do a simulation, we have to first solve the initial condition to obtain the initial values of ${\gamma_{ij}, K_{ij}}$ and then we can evolve the system. Moreover, since they must be true during all the evolution, they are used as a measure of the quality of the solution.
Due to their importance, it is fundamental to learn how to handle the constraints, even if it is not a simple task to absolve. In fact (\ref{BSSNOKhamilton_eq}) and (\ref{BSSNOKmomentum_eq}) form a system of four coupled partial differential equations of elliptic type and in general they are difficult to solve. Still there are several procedures that can be used to solve these equations: here we will present the more common \emph{conformal thin-sandwich approach}.

The basic idea of the conformal thin-sandwich approach to construct initial data is to prescribe the conformal metric on each of two nearby spatial hypersurfaces (the ``thin-sandwich''), or equivalently the conformal metric and its time derivative on a given hypersurface. We then start with the same conformal decomposition $\tilde{\gamma}_{ij}= \psi^{-4} \gamma_{ij}$ and, defining $\tilde{u}_{ij}=\partial_t \tilde{\gamma}_{ij}$, we require that
$$\tilde{\gamma}^{ij}\tilde{u}_{ij}=0$$
which is equivalent of demanding that the volume element of the conformal metric remains momentarily fixed.
Now, considering the tracefree part of the evolution equation for the spatial metric, let us define\footnote{We use an algebraic result which states that 
\begin{theorem}
Any symmetric-tracefree tensor $S^{ij}$ can be split in
$$S^{ij}=S^{*ij}+(\mathbf{L} W)_{ij}$$
\end{theorem}
where $S^{*ij}$ is a symmetric, traceless and transverse ($D_jS{*ij}=0$) tensor.}

$$u_{ij}=\partial_t \gamma_{ij}-\dfrac{1}{3}\gamma_{ij}(\gamma_{kl}\partial_t \gamma_{kl})=-2 \alpha A_{ij}+ (\mathbf{L} \beta)_{ij}$$

where $\mathbf{L}$ is the operator associated with the spatial metric and defined as $(\mathbf{L} W)^{ij}=D^iW^j+D^j W^i - \frac{2}{3}\gamma^{ij}D_k W^k$. It is known as the \emph{conformal Killing form}\footnote{$\mathcal{L}_{\vec{W}}(\gamma^{-\frac{1}{3}}\gamma_{ij})=(\mathbf{L}W)_{ij}=0$, that is the conformal metric $\tilde{\gamma}_{ij}=\gamma^{-\frac{1}{3}}\gamma_{ij}$} associated with the vector $W^i$.
From $\tilde{\gamma}^{ij}\tilde{u}_{ij}=0$ follows that $$\partial_t \log \psi=\partial_t \log \gamma^{\frac{1}{12}} $$
so that one finds
$$\tilde{u}_{ij}= \partial_t (\psi^{-4}\gamma_{ij})=\psi^{-4} (\partial_t \gamma_{ij}- 4 \gamma_{ij}\partial_t \log \psi)$$
$$=\psi^{-4} \left( \partial_t \gamma_{ij}-\dfrac{1}{3}\gamma_{ij} \partial_t \log \gamma \right)=\psi^{-4} u_{ij}$$
Solving $A^{ij}$ using the expression for $u_{ij}$, we have
$$A^{ij}=\dfrac{\psi^{-4}}{2 \alpha} \left[ (\tilde{\mathbf{L}}\beta)^{ij}-\tilde{u}^{ij} \right]$$
Taking 
$$\tilde{A}^{ij}= \psi^{10}A^{ij}$$
and defining the \emph{conformal lapse} $\tilde{\alpha}=\psi^{-6} \alpha$, we find
$$\tilde{A}^{ij}=\dfrac{1}{2 \tilde{\alpha}} \left[ (\tilde{\mathbf{L}}\beta)^{ij}-\tilde{u}^{ij} \right]$$
The conformal lapse $\tilde{\alpha}=\psi^{-6} \alpha$, in the case when $\psi=\gamma^{\frac{1}{12}}$, corresponds  to a rescaling of the lapse $\tilde{\alpha}=\gamma^{\frac{1}{2}} \alpha$.
Using the new definition introduced, we can rewrite the constraints as
\begin{equation}
8 \tilde{D}^2 \psi - ^{(3)}\tilde{R} \psi +\psi^{-7}\tilde{A}_{ij}\tilde{A}^{ij}- \dfrac{2}{3}\psi^{5}K^2 + 16 \pi \psi^5 \rho=0
\label{thinHamiltonian_eq}
\end{equation}
\begin{equation}
\tilde{D}_j \left[ \dfrac{1}{2 \tilde{\alpha}}(\tilde{\mathbf{L}}\beta)^{ij} \right]-\tilde{D}_j \left[ \dfrac{1}{2 \tilde{\alpha}}\tilde{u}^{ij} \right]-\dfrac{2}{3}\psi^{6}\tilde{D}^i K -8 \pi \psi^{10}j^i=0
\label{thinMoment_eq}
\end{equation}

The Hamiltonian constraint (\ref{thinHamiltonian_eq}) and the Momentum constraint (\ref{thinMoment_eq}) in this form can be solved to obtain the  conformal factor $\psi$ and the shift vector $\beta^i$, given the conformal metric $\tilde{\gamma}_{ij}$, its time derivative $\tilde{u}^{ij}$, the trace of the extrinsic curvature $K$, the conformal lapse $\tilde{\alpha}$, matter densities $\tilde{\rho}$ and $\tilde{j}^i$. One then reconstructs the physical quantities:
$$\gamma_{ij}=\psi^4 \tilde{\gamma}_{ij}$$
$$K^{ij}=\psi^{-10}\tilde{A}^{ij}+ \dfrac{1}{3} \gamma^{ij}K$$
with $\tilde{A}^{ij}=\dfrac{1}{2 \tilde{\alpha}} \left[ (\tilde{\mathbf{L}}\beta)^{ij}-\tilde{u}^{ij} \right]$.\\

To better understand the advantage of using this approach to find the initial conditions, we will show how to apply the methods just described to the problem of finding initial data for a physical system of multiple black hole spacetimes. This system has the advantage of being a pure vacuum solution, so we will not have to worry about how to describe the matter content. Moreover, it is a solution of great interests in Numerical Relativity both historically, since  the first aim of numerical relativists was to study gravitational waves emitted by rotating binary black holes, and recently, since it is currently used to investigate inhomogeneous cosmological models (that will be analyzed in the following chapters).

Before of evolving the dynamical equations for $(\gamma_{ij}, K_{ij})$, we need to find the initial conditions, so we have to decide an initial time instant. Ideally, we consider the case of time-symmetric initial data, corresponding to a time when the black holes are momentarily static. In this conditions, we clearly have $K_{ij}=0$ as initial data for the extrinsic curvature. This imply that the momentum constraints (\ref{thinMoment_eq}) must be trivially satisfied.

Thus we only need to solve the Hamiltonian constraint (\ref{thinHamiltonian_eq}), which in this case reduces to
\begin{equation}
8 \tilde{D}^2 \psi - \tilde{R} \psi=0
\label{initialMultBH_eq}
\end{equation} 
because $K^{ij}=0$ and we are in vacuum $\rho=0$.
We can chose to simplify the (\ref{initialMultBH_eq}) further by choosing a flat conformal metric $\tilde{R} =0$, obtaining
$$D^2_{flat} \psi=0$$
with $D^2_{flat}$ the standard flat three-Laplace operator.In this way the boundary conditions correspond to an \emph{asymptotically flat spacetime} so that, far away of the gravitational field goes to zero and we have $\psi=1$. Choosing $\psi=1$ everywhere (and this is a solution of the previous boundary equation), we simply recovery the Minkowski spacetime as initial data.

Proceeding to the next approximation, we have
\begin{equation}
\psi=1+\dfrac{k}{r}
\end{equation}
with $k$ an arbitrary constant, and $\lim_{r \to \inf} \psi=1$. Using this conformal factor, the physical metric in spherical coordinates is
$$dl^2=\left( 1+\dfrac{k}{r} \right)^4 \left[ dr^2+r^2 d\Omega^2 \right]$$ 
This metric correspond to the spatial metric for a Schwarzschild black hole written in isotropic coordinates, with $k=M/2$, and thus it corresponds to the solution of the initial data problem for a single Schwarzschild black hole.
As the boundary equation is linear (it was written as a Laplace's equation), we can simply superpose solutions to obtain new solutions. We can build the conformal factor for a multiple $N$ black hole system as

\begin{equation}
\psi= 1+ \sum_{i=1}^{N} \dfrac{m_i}{2 |\vec{r}- \vec{r}_i|}
\label{conformalMultiBH_eq}
\end{equation}
where $\vec{r}_i$ are the positions of each black hole and $m_i$ are known as the \emph{bare masses}\footnote{The bare mass corresponds with the individual mass only in the case of a single black hole, for more than one black hole the definition of the individual masses is somewhat trickier. A commonly used definition is obtained by going to the asymptotically flat end associated with each hole and calculating the $ADM$ mass $M_{ADM}= \sum_i m_i$ there.}. This solution forms the basis of the so called \emph{puncture} of black holes, which we discuss deeply in  Appendix A.

\section{The Einstein Toolkit}

While the formalism of conformal decomposition may appear unnecessarily technical and perhaps confusing initially, this example demonstrates that it provides an extremely powerful tool for constructing solutions to Einstein's equations. In fact, once the formalism has been developed, it is much easier to derive the Schwarzschild solution by going through the above steps then deriving it from Einstein's equations directly. However, except for very idealized problems with special symmetries, such equations must be solved by numerical means and implemented by supercomputers.
This is only one of many examples demonstrating how important the computation is to study natural phenomena.
Also thanks to the rapid development of supercomputers capabilities and of graphics modelling techniques, numerical simulations play a major role in modern physics and in general in all science.

In the field of computational astrophysics, a plethora of codes have been developed to simulate astronomical systems such as galaxies formation and dynamics, merging of binary black holes or neutron stars, supernovae explosions, etc. For instance, a first archievement in this field was the \emph{Millennium Run} \cite{millenium}, which used more than $10$ billion particles to trace the evolution of the matter distribution in a cubic region of the Universe over $2$ billion light-years on a side (Fig. \ref{millenium}). 
\begin{figure}
\centering
\includegraphics[scale=0.17]{figure/millenium.jpg} 
\caption{The Millennium Run is based on a $N$-body code which simulated the evolution of about $10^{10}$ particles under the effects of gravity: Virgo scientists at Max Planck Society in Garching, Germany, have been able to recreate evolutionary histories both for the $20$ million galaxies which populate this enormous volume and for the supermassive black holes which occasionally power quasars at their hearts. By comparing such simulated data to large observational surveys, one can clarify the physical processes underlying the buildup of real galaxies and black holes.
}
\label{millenium}
\end{figure}

A more recent simulation is the \emph{Illustris simulation}: a surprising project in which it is tracked the expansion of the universe, the gravitational pull of matter onto itself, the motion or "hydrodynamics" of cosmic gas, as well as the formation of stars and black holes. These physical components and processes are all modeled starting from initial conditions resembling the very young universe $300,000$ years after the Big Bang and until the present day, spanning over $13.8 $ billion years of cosmic evolution. The simulated volume contains tens of thousands of galaxies captured in high-detail, covering a wide range of masses, rates of star formation, shapes, sizes, and with properties that agree well with the galaxy population observed in the real universe.

Therefore, it is clear that a well-tested and easy to use computational infrastructures would be a great advantage to researchers, which would be able to focus less on the computational aspect of such tasks, and spent more time on the actual physics. In recent years, several codes have been developed with this intent: some of them, such as $N$-body codes, use a typically non-relativistic approach simulating Newtonian gravity effects to which corrective terms are added; others rely on the perturbative approach described above, considering terms of first or second order in relativistic perturbations. Specifically, in this thesis we will focus on that class of totally relativistic codes (relativistic hydrodynamic codes or based on the BSSN formulation), and in particular we will present the \emph{Einstein Toolkit}\cite{loffler2012einstein}, a powerful open-source software platform of core computational tools for relativistic astrophysics and gravitational physics.

\begin{figure}
  \centering
  \includegraphics[width=80mm]{figure/einstool}%
  \qquad\qquad
  \includegraphics[width=90mm]{figure/cactus}
  \caption{Einstein Toolkit and Cactus logos}
  \label{log_fig}
\end{figure}

The Einstein Toolkit $(ET)$ (Fig. \ref{log_fig}) consists of quite a large number of components, of which most currently use the \textsf{Cactus} computational toolkit framework, which provides the basic modular infrastructure for numerical simulations. 
\textsf{Cactus} is a general framework for the development of portable, modular applications, originally developed for numerical relativity, and it is the outcome of many years of code development: its first version was released in $1995$ and over the years it has been generalized for use by scientists in other fields.

\textsf{Cactus} programs are split into two types of components: ``\emph{thorns}'' and ``\emph{flash}''. Flash are the core of the \textsf{Cactus} framework and they represent the \textsf{Cactus} structure that manages how the parts of the program must be carried out. Flash are typically independent of other parts of \textsf{Cactus} and acts as utility and service library. Thorns are separate modules which encapsulate the implementation of some functionality and they can be written in different languages like \textsf{C, C++, Fortran, OpenCL}. Typically they are developed independently and do not directly interact with each other, rather each of them interacts with the flesh which provides the ``glue'' between different thorns. 

Now let us introduce some of the arrangements provided with the \textsf{Cactus} framework, as well as  some of the more important tools.
First of all, \textsf{Cactus} does not provide executable files but it provides infrastructure to \emph{create} executable. So \textsf{Cactus} simulations require an executable to be compiled, which collect all the thorns and modules that will need during the running. This setup allows for quick adaptation to changes in the local environment, (e.g., updated libraries), but also for a simple way to tailor the source code to the users' needs. The executable has one mandatory argument: a \emph{parameter file}. This is a simple text file, containing key-value pairs of parameter names and the desired settings within the simulation. The executable collect and organize all thorns which must be used to model a sample of physical scenarios, while the parameter file is used to choose which of these should be realized.

\textsf{Cactus} separates physics code from infrastructure code; so a typical physics thorn will not care about memory management, parallelization, Input/Output (IO) or grid refinement but these tasks are left to a special thorn called \emph{driver}. Two driver thorns provided with the $ET$ are \textsf{PUGH} and \textsf{Carpet}: both set the the virtual domain of simulations with a Cartesian grid but while \textsf{PUGH} provide a uniform unigrid for the whole domain, \textsf{Carpet} provides a multigrid infrastructure which cover the simulation domain by a set of overlapping patches.

Usually we call \emph{Arrangement} a collections of thorns, which signaling a common task or have a common origin. This grouping is solely done for the benefit of a better overview for the user, in fact being part of one or another arrangement does not have any special meaning for a thorn.
Then, we can continue giving an overview on some of the basic \textsf{Cactus} and ET arrangements:
\begin{enumerate}
\item \emph{Core Arrangement} are all the thorns that provide infrastructure for basic utilities as boundary conditions, coordinates setting, I/O, symmetries (\textsf{CactusBase} and \textsf{CactusUtils}), numerical implementations (\textsf{CactusNumerical}) or external libraries as HDF$5$, FFTW and MPI (\textsf{ExternalLibraries}).
\item \textsf{EinsteinBase} thorns define and register basic variables within numerical relativity. Thorns of this arrangement make use or modify, for example the $ADM$ variables or the stress-energy tensor. Main thorns include \textsf{ADMBase}, which defines groups of grid functions for basic spacetime variables, such as the spatial metric or the extrinsic curvature, based on the $3 + 1$ $ADM$ construction, \textsf{HydroBase} which defines basic variables and grid functions for hydrodynamics evolution or \textsf{TmunuBase} which defines grid functions for stress-energy tensor (``right-hand-side'' of Einstein's equations).
\item \emph{EinsteinEvolve or McLachlan} arrangement contain several thorns related with the numerical evolution of Einstein's field equations. In particular \textsf{McLachlan} thorn is the ET's spacetime evolution code. It uses an accurate finite differencing scheme to discretize spacetime variables in the $BSSN$ form. It is designed to inter-operate through the \textsf{ADMBase}  and \textsf{TmunuBase} interface. Other important thorns are \textsf{GRHydro}, a general relativistic magneto-hydrodynamics (GRMHD) matter evolution, and \textsf{PunctureTracker}, which takes care of black hole tracking.
\item \emph{EinsteinAnalysis} arrangement which includes several thorns useful for analysis such as \textsf{AHFinderDirect}, to find black hole apparent horizons given $(\gamma_{i j}, K_{i j})$; \textsf{ADMAnalysis}, to calculate several quantities from the $ADM$ variables like the components of the Ricci tensor or the Ricci scalar; \textsf{ADMConstraints}, to calculate the $ADM$ constraints violation. 
\end{enumerate}

Now, to better understand how ET works, let us review the main steps we need to follow in order to produce a simulation.
Once we have downloaded the ET on the machine, the first step is having the required \emph{thornlist}\footnote{A thornlist is a text file containing the list of thorns to be downloaded together with the path and version controlling system used for each thorn.} for the configuration intended to be built. Typically, one will not be compiling all the thorns provided with the ET but it is preferable to build a thornlist that is as small as possible, including only the required thorns, but taking care of the (non-trivial) dependencies between thorns. 

In order to compile a \textsf{Cactus} simulation, we first need to create a configuration, specifying the machine features and the thornlist we want to use. The result of this operation is an executable file.

We should now be ready for running. For this, a \emph{parameter file} is needed, specifying which thorns to use within the simulation (not all compiled thorns need to be active), and which specific model parameters have been chosen.
Modifying the thorn activated in the parameter file or the values of the thorn arguments, we change the physical scenarios simulated. 

Therefore, we described core parts of the Einstein Toolkit as generally as possible to give a consistent overview of its capabilities and components. From what has been said, we can consider Einstein Toolkit as a virtual laboratory where we can implement experiments (the simulations) that would otherwise be unfeasible.\\


%In the following chapter we will apply some of the thorns discussed in this chapter, to study the evolution of a Black-Hole Lattice.


Nowadays, both the perturbative approach and the numerical approach are used with pretty good results to investigate the origin and the evolution of the large-scale inhomogeneities of the Universe. However, current and planned galaxy surveys will measure the structure of the Universe with unprecedented precision and we can only make progress in understating the physical laws that govern the Universe if theoretical predictions are made with matching or better accuracy than these measurements. Therefore, we need to improve the realism of our simulations to gain measurements more and more accurate to compare with the experiments.

In this context, the perturbation approach has insurmountable limits, due to its nature, that can only be ignored under certain conditions. On the other hand, the numerical approach to this field is relatively young, and it ``recently'' addressed to the degree of realism required for our purposes. With the intent to pursue this objective, in my thesis I studied the behavior of an inhomogeneous Universe using the numerical approach. In particular, I used the results of Numerical Relativity just treated to simulate the inhomogeneous system of blacks holes lattices\footnote{In particular, in a preliminary phase of this thesis, I riproduced the outcomes already obtained in a previous work by Bentivegna and Korzynski which, using numerical approaches, studied the evolution of closed Universes from initial data on a hypersurface at time-symmetry \cite{eloisaBHL}.}. 
I have choice this particular system for two main reasons:
\begin{enumerate}
\item Black holes, and in particular a system made of multiple black holes is a well-known system in Numerical Relativity (as we have seen before) \cite{mroue2012precessing, sperhake2014numerical}.
\item A discrete distribution of black holes well reproduce the real condition of the mass distribution in the Universe (local concentrations of matter separated by voids).
\end{enumerate}

In order to understand what are the principal features of the black hole lattice Universes, in the following chapter we will present the \emph{Lindquist-Wheeler black hole lattices}.