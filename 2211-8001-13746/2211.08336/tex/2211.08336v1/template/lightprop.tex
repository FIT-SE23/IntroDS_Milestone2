\chapter{The BGO formalism for light propagation}%light_prop
\label{chap:BGO}
Cosmologists and astronomers use electromagnetic and gravitational radiation as their major tools for studying the structure and development of the Universe.  
These ''light-like'' signals contain information about the emitting source as well as of the spacetime geometry, the latter derived from the effects induced by gravity.
In the near future, these effects will be measured with unprecedented precision over a wider range of scales and redshift by the next generation of galaxy surveys and CMB experiments\footnote{{\color{blue}{\tt {https://www.skatelescope.org}}}, {\color{blue}{\tt {https://www.euclid-ec.org}}}, {\color{blue}{\tt {https://www.lsst.org}}}, {\color{blue}{\tt {http://litebird.jp/eng/}}}, {\color{blue}{\tt {https://www.jpl.nasa.gov/missions/spherex}}}}. This revolution in cosmology marks also the beginning of the \emph{real-time cosmology} era, \cite{Quercellini:2010zr}, in which it will be possible to measure small temporal changes in cosmological observables, called \emph{optical drift effects}. These real-time effects can provide important and new information about the structure and evolution of the Universe.  
From the point of view of the basic theory of light propagation, a new approach was presented in \cite{Grasso:2018mei}. The key ingredients of this new formulation are the \emph{bilocal geodesic operators} which represent the map from the portion of spacetime occupied by the observer to that occupied by the source and provide the complete description of the distortion of the light rays in between. 


This chapter is divided into two parts: in the first part, we review the fundamental equations of light propagation in geometric optics, starting from Maxwell's equations in curved spacetimes. This is standard knowledge, see e.g. \cite{mtw, wald2010general, perlick-lrr}, and serves here as an introduction to the basic concepts of geometric optics in general relativity.
In the second part, we discuss the bilocal geodesic operators formulation of light propagation in geometric optics, based on the results presented in \cite{Grasso:2018mei}. This part provides the theoretical framework for the original results presented in Chapters~\ref{chap:bigonlight} and \ref{chap:nonlinearities}. %\cite{ PhysRev.166.1263, PhysRev.166.1272, Harte:2018wni}


\section{Light propagation in curved spacetime}
\label{sec:Maxwell_to_GeomOpt}

\emph{Light signals}, intended as radiation travelling at the speed $c=299792.5\, {\rm km/s}$, are governed by \emph{Maxwell's equations}
\begin{align}
F_{\mu \nu}&=\nabla_{\mu} A_{\nu}-\nabla_{\nu} A_{\mu} \label{eq:curvF}\\
\nabla_{\nu} F^{\mu \nu}&= 4 \pi J^{\mu} \label{eq:curvME1}\\
\nabla_{[\lambda}F_{\mu \nu]}&=0\, , \label{eq:curvME2}
\end{align}
with $J^{\mu}=(c \rho, \, J^i)$ the four-current of the charge density $\rho$ and current density $J^i$. $F^{\mu \nu}$ is the Faraday tensor (also known as field strength or electromagnetic tensor) defined as the field strength of the four-vector potential $A^{\mu}=(\phi / c, A^i)$
\begin{equation}
F_{\mu \nu}=\nabla_{\mu} A_{\nu}-\nabla_{\nu} A_{\mu}\, , \label{eq:F}
\end{equation}
whose components are the electric $F^{0 i}=-F^{i 0}=E^i$ and magnetic $F^{i j}=\varepsilon^{i j k}B^k$ fields. 
From the very definition of $F_{\mu \nu}$, Eq.~\eqref{eq:F}, we see that the electromagnetic tensor is invariant under the gauge transformation $A_\mu \to A_\mu + \nabla_\mu \chi$, with $\chi$ a scalar function. This constitutes the gauge freedom of electromagnetism. A convenient gauge choice is the Lorentz gauge condition $\nabla_{\mu} A^{\mu}=0$, in which Eq.~\eqref{eq:curvME1} takes the simpler form of a wave equation for the four-vector potential
\begin{equation}
\nabla_{\nu}\nabla^{\nu}A^{\mu}=R\UD{\mu}{\sigma }A^{\sigma}- 4 \pi J^{\mu}\, , \label{eq:curvelectromagn}
\end{equation}
where the Ricci tensor $R\UD{\mu}{\sigma}=R\DU{\sigma}{\mu}=R^{\nu \, \, \, \, \, \mu}_{\,\,  \sigma \nu}$ appear from the commutation of the two covariant derivatives as $\nabla_{\nu}\nabla^{\mu}A^{\nu}=\nabla^{\mu}\nabla_{\nu}A^{\nu}+R^{\nu \, \, \, \, \, \mu}_{\,\,  \sigma \nu}A^{\sigma}$.
The expression in Eq.~\eqref{eq:curvelectromagn} describes the dynamics of electromagnetic potential in a generic spacetime, (see e.g. \cite{mtw, wald2010general, perlick-lrr}).

In the absence of sources $J^{\mu}=0$ Eq.~\eqref{eq:curvelectromagn} gives the propagation equation for the electromagnetic radiation
\begin{equation}
\nabla_{\nu}\nabla^{\nu}A^{\mu} - R\UD{\mu}{\sigma }A^{\sigma}=0\, , \label{eq:curvER}
\end{equation}
which one has to solve to study light propagation in cosmology. %Given our goal of studying light propagation in cosmology, we are interested in finding solutions to this equation. 
However, this equation is too general, and needs to be ``adapted'' to apply to astronomical observations.
As described in \cite{mtw}, one can distinguish three characteristic lengths\footnote{These characteristic lengths are  evaluated in a local inertial frame, e.g. the one at rest respect a nearby galaxy.}:
\begin{enumerate}
\item the typical wavelength of the electromagnetic radiation $\lambda$,
\item the typical length over which the amplitude, the polarization and the wavelength vary $L$, 
\item the typical radius of curvature of the spacetime $\mathcal{R}$, defined such that $\mathcal{R}=\mathcal{O}\left( |R\UD{\mu}{\rho \sigma \nu}|^{-1/2}\right)$, with  $|R\UD{\mu}{\rho \sigma \nu}|$ denoting the magnitude of the typical component of the Riemann tensor.
\end{enumerate}
In the range of astrophysical observations, the typical electromagnetic wavelength extends\footnote{For gravitational waves the typical wavelength can be larger, as the one of the first gravitational waves detection $\lambda \sim 10^4 \, {\rm km}$, \cite{abbott2016observation}.} from $\sim 10 \, {\rm m}$ for radio waves emitted by active radio galaxies to the smaller wavelengths of visible, X-rays, and gamma-rays emissions. 
On the other hand, the typical curvature's radius $\mathcal{R}$ of the spacetime where these electromagnetic signals propagate is usually much larger. For instance on cosmological scales we have that\footnote{As an estimate of $|R\UD{\mu}{\rho \sigma \nu}|$ we use the Ricci scalar that in a flat FLRW metric gives $R=6\left(\dfrac{\dot{a}^2+a \ddot{a}}{a^2} \right) \sim \left(\dfrac{\dot{a}}{a}\right)^2=H^2$. In other cases, like e.g. in a Schwarzschild metric, the Ricci scalar is not a good indicator and we need to use a different estimator for $|R\UD{\mu}{\rho \sigma \nu}|$, like the Kretschmann scalar $|R^{\mu \rho \sigma \nu}R_{\mu \rho \sigma \nu}|$.} $\mathcal{R} \sim c / H$, with $H$ the Hubble parameter whose value depends on the cosmological era\footnote{At the recombination era (around $3.7 \times 10^5$ years after the Big Bang, the epoch at which the ionised plasma of electrons and protons first became bound forming neutral hydrogen atoms, which did not scatter the photons but allowed them to travel freely) the Hubble parameter was $H=1255\, \, {\frac{km}{s\, Mpc}}$ corresponding to $\mathcal{R} \sim 0.2 \, {\rm Mpc}$.}: at present time $H_0=67.4\, {\frac{km}{s\, Mpc}}$, see~\cite{planck2018param}, which gives $\mathcal{R} \sim 4.5 \, {\rm Gpc}$. Also on smaller scales the condition $\lambda \ll \mathcal{R}$ remain valid, for instance close to the surface of the Sun, one can calculate that the curvature radius is of the order of $ 10^8\, {\rm km}$. 
%at a distance of $6 \times 10^7\, {\rm km}$ form the Sun the typical curvature radius is of $\left(\dfrac{c^2 r^3}{G M}\right)^(1/2)\sim 4 \times 10^{11} \, {\rm km}$.
%Therefore, it is safe to assume that in general $\lambda \ll \mathcal{R}$
We need to go to very strong gravitational regimes, like close to the event horizon $r_s$ of a black hole where $\mathcal{R}\sim r_s$ to have the condition $\lambda \ll \mathcal{R}$ no longer valid for part of the electromagnetic spectrum\footnote{For instance, for a Sun-like black hole we have that $\mathcal{R}\sim r_s\sim 3\, {\rm km}$ which is smaller than the low frequency radio waves, or for a Earth-like black hole $\mathcal{R} \sim 8\, {\rm mm}$ which is smaller than micro waves $\lambda \sim 1\, {\rm cm}$.}.
Except for the last instance, we can treat light propagation within the so called \emph{geometric optics approximation}, which is valid whenever $\lambda$ is much smaller than each of the other scales involved, i.e.
\begin{align}
\lambda \ll L\, &\text{ and }\, \lambda \ll \mathcal{R}\, .
\end{align}
Within this approximation one can look for solutions of Eq.~\eqref{eq:curvER} in the form of a rapidly oscillating wave with a nearly constant amplitude, namely
\begin{equation}
A^{\mu}=C^{\mu} e^{i \theta} \, , \label{eq:em_ansaz}
\end{equation}
where the phase $\theta \propto \frac{2\pi}{\lambda}$ is a real function of the spacetime's position, while $C^{\mu}$ is in general a complex four-vector expressing the amplitude and polarization of the electromagnetic wave. % and such that $|\nabla^{\nu}\nabla_{\nu}C^{\mu}| \ll |\nabla_{\nu}C^{\mu}| \ll |C^{\mu}|$.
%Given this ansaz, assuming to reduce the wavelength $\lambda$ by keeping $L$ and $\mathcal{R}$ unchanged, the phase $\theta$ will get larger and larger but $C^{\mu}$ will not vary very much. Therefore, we can express the dependence on $\lambda$ in Eq.~\eqref{eq:em_ansaz} by introducing the parameter $\epsilon=\lambda / d $, where $d={\rm min}(L, \, \mathcal{R})$, and doing the expansion
Given this ansatz, we note that with $\lambda$ decreasing to zero, and $L$ and $\mathcal{R}$ fixed, the phase $\theta$ will get larger and larger, but $C^{\mu}$ will not vary very much. Therefore, we can express the dependence on $\lambda$ in Eq.~\eqref{eq:em_ansaz} by introducing the parameter $\epsilon=\lambda / d $, where $d={\rm min}(L, \, \mathcal{R})$, and expanding the solution in its powers 
\begin{equation}
A^{\mu}=(a^{\mu}+\epsilon b^{\mu}+\epsilon^2 c^{\mu}+\cdots) e^{i \frac{\theta}{\epsilon}} \, . \label{eq:em_ansaz_expansion}
\end{equation}
Note that $a^{\mu} e^{i \frac{\theta}{\epsilon}}$ is the leading order term and it constitutes the geometric optics approximation of our solution. The other higher-order terms (\setwd{h.o.t.}{acr:hot}) are all contained in $(\epsilon b^{\mu},\,  \epsilon^2 c^{\nu},\, \cdots)$, representing the post-geometric optics corrections (see e.g. \cite{ PhysRev.166.1263, PhysRev.166.1272, Harte:2018wni} for approaches beyond geometric optics).
Applying the ansatz in Eq.~\eqref{eq:em_ansaz_expansion}, the wave equation in geometric approximation gives %the conditions that the vector amplitude $C^{\mu}=(a^{\mu}+\epsilon b^{\mu}+\epsilon^2 c^{\mu}+\cdots)$ and the phase $\theta$ has to satisfy in geometric optics 
\begin{align}
\ell_{\mu} a^{\mu}&=0  \label{eq:geom_phase}\\
a^\mu \ell_{\nu} \ell^{\nu} & =0 \label{eq:geom1}\\
- b^\mu \ell_{\nu} \ell^{\nu}  + i \left( a^{\mu} \nabla_{\nu}\ell^{\nu} \right. & \left. + 2 \ell_{\nu} \nabla^{\nu} a^{\mu} \right) =0 \, .\label{eq:geom2}
\end{align}
The first relation, Eq.~\eqref{eq:geom_phase} is the orthogonality relation between the vector amplitude $a^{\mu}$ and $\ell^{\mu}=\nabla^{\mu} \theta$, the vector normal to the surfaces of constant phase.
%%%%%%%%%%%%%%%
The relation Eq.~\eqref{eq:geom1} expresses the fact that in the geometric optics approximation one can consider the electromagnetic signals as travelling along null-like geodesic, whose tangent vector $\ell^{\mu}$ satisfies the geodesic equation\footnote{The geodesic equation~\eqref{eq:geodesicEQ} is related to the condition Eq.~\eqref{eq:geom1} as $0=\nabla_{\mu}(\ell^{\nu}\ell_{\nu})=(\nabla_{\mu}\ell^{\nu})\, \ell_{\nu}+\ell^{\nu}(\nabla_{\mu}\ell_{\nu})= \ell_{\nu}(\nabla_{\mu}\nabla^{\nu}\theta)+\ell^{\nu}(\nabla_{\mu}\nabla_{\nu}\theta)= \ell_{\nu} (\nabla^{\nu}\nabla_{\mu}\theta)+\ell^{\nu}(\nabla_{\nu}\nabla_{\mu}\theta)= \ell_{\nu} (\nabla^{\nu}\ell_{\mu})+\ell^{\nu}(\nabla_{\nu}\ell_{\mu})=2 \ell^{\nu}\nabla_{\nu}\ell_{\mu}$% $\ell^{\nu}\nabla_{\nu}\ell_{\mu}=\ell^{\nu}\nabla_{\nu}(\nabla_{\mu}\theta)=\ell^{\nu}\nabla_{\mu}(\nabla_{\nu}\theta)=\ell^{\nu}\nabla_{\mu}\ell_{\nu}=\frac{1}{2}\nabla_{\mu}(\ell^{\nu}\ell_{\nu})=0$
, where we have used $\nabla_{\mu}(\nabla_{\nu}\theta)=\nabla_{\mu}(\partial_{\nu}\theta)=\partial_{\mu}\partial_{\nu}\theta-\Gamma^{\sigma}_{\mu \nu}\partial_{\sigma}\theta=\partial_{\nu}\partial_{\mu}\theta-\Gamma^{\sigma}_{\nu \mu}\partial_{\sigma}\theta=\nabla_{\nu}(\nabla_{\mu}\theta)$.} 
\begin{equation}
\ell^{\nu}\nabla_{\nu}\ell^{\mu}=0 \, . \label{eq:geodesicEQ}
\end{equation}
A generic geodesic can be represented as a parametric curve $\gamma(\lambda)$, where the parameter $\lambda$ spans the geodesic such that to a small variation of the parameter $d \lambda$ correspond a small displacement along the geodesic itself: 
$$d x^{\mu}= \ell^{\mu} \, d \lambda\, .$$
Therefore, the geodesic equation~\eqref{eq:geodesicEQ} can be expressed as the covariant derivative with respect to $\lambda$ as
\begin{equation}
\dfrac{D}{D \lambda}\ell^{\mu}=\dfrac{d^2 x^{\mu}}{d \lambda^2}+ \Gamma^{\mu}_{\sigma \rho} \dfrac{d x^{\sigma}}{d \lambda} \dfrac{d x^{\rho}}{d \lambda}=0 \, . \label{eq:geodesicEQ_par}
\end{equation}
It is worth noticing that Eq.~\eqref{eq:geodesicEQ_par} is satisfied if $\lambda$ is an affine parameter of the geodesic $\gamma(\lambda)$. However, the parametrisation of the geodesic is not unique, i.e. it is always possible to choose a different parametrisation $\tau$, such that $\gamma(\lambda) \to \gamma(\tau)$. In general the transformation introduces a new term in Eq.~\eqref{eq:geodesicEQ_par} proportional to the tangent vector, such that $\dfrac{D \ell^{\mu}}{D \tau}\propto \ell^{\mu}$: in this case $\tau$ is said a non-affine parameter. It is easy to show that the transformation 
\begin{equation}
\lambda \to A \cdot \lambda + B , \label{eq:affineparameter}
\end{equation}
with $A,\, B={\rm const} $, is the only possible transformation that leaves the Eq.~\eqref{eq:geodesicEQ_par} satisfied, namely it transforms an affine parameter into a new affine parameter (see problem 5 of Sec. 3 in~\cite{wald2010general}). 
Moreover, a different parametrisation of the geodesic $\gamma$ changes the value of the squared norm of the tangent vector. In general for time-like and space-like geodesics we prefer to use an affine parametrisation of the geodesic such that its tangent vector $k^{\mu}$ is normalised as $k^\mu k_\mu = \sigma$, with $\sigma=-1$ for time-like geodesics and $\sigma=1$ for space-like geodesics.
In the particular case of null geodesics we do not have a preferred, normalised parametrisation, since Eq.~\eqref{eq:geom1} holds, so we may always reparametrise $\gamma$ by an affine transformation, Eq.~\eqref{eq:affineparameter}. Then the null tangent vector $\ell^\mu$ transforms according to
 \bea
  \ell^\mu \to \frac{1}{A}\,\ell^\mu \label{eq:laffine}\, .
 \eea

The relation in Eq.~\eqref{eq:geom2} is the propagation equation for the vector amplitude: it is convenient to express $a^{\mu}$ as $a^{\mu}=a p^{\mu}$, where  $a=\sqrt{|a^{\mu}a_{\mu}|}$ is the scalar amplitude and $p^{\mu}=a^{\mu}/a$ is the polarization vector. Thus, Eq.~\eqref{eq:geom2} becomes
\begin{equation}
p^{\mu} \left(a  \nabla_{\nu}\ell^{\nu} + 2 \ell_{\nu} \nabla^{\nu} a \right) + 2 a \ell_{\nu} \nabla^{\nu} p^{\mu}=0\, . \label{eq:polartransp1}
\end{equation}
The term in parenthesis has an important physical meaning and it represents the flux conservation in geometric optics approximation. This can be easily proved by using the continuity equation $\nabla^{\mu}T_{\mu \nu}=0$ for the electromagnetic stress-energy tensor in the geometric optics approximation $T_{\mu \nu}= a^2 e^{2 i \theta} \ell_{\mu} \ell_{\nu}+ h.o.t.$. After some straightforward calculations, one get that at the leading order
\begin{equation}
0=\nabla^{\mu}T_{\mu \nu}=2 a \nabla^{\mu} a \,  \ell_{\mu} + a^2\nabla^{\mu}\ell_{\mu}=\nabla^{\mu}(a^2 \ell_{\mu})\, . \label{eq:photcons}
\end{equation}
The vector $a^2 \ell^{\mu}$ is the photon flux density and the volume integral $(8 \pi \hbar )^{-1}\int a^2 \ell^0 \sqrt{|-g|} d^3 x$ gives the number of photons (or geodesics) in the volume of integration on any $x^0 = {\rm const}$ hypersurface.
Implementing Eq.~\eqref{eq:photcons} in Eq.~\eqref{eq:polartransp1} we obtain the propagation equation for the polarization vector
\begin{equation}
\ell^{\nu} \nabla_{\nu} p^{\mu}=0\, , \label{eq:polartransp}
\end{equation}
or in other words, the polarization vector is parallel transported along the null geodesic.

To conclude, the geometric optics approximation can be summarised as follows:
\begin{itemize}
\item when an electromagnetic signal satisfies the conditions $\lambda \ll L\, \text{ and }\, \lambda \ll \mathcal{R}$ we can look for solutions to Eq.~\eqref{eq:curvER} of the form $A^{\mu}= a p^{\mu} e^{i \theta} + h.o.t.$;
\item in this approximation we can consider the photons as travelling along light rays (null geodesics), Eq.~\eqref{eq:geom1}, with the tangent vector $\ell_{\mu}=\nabla_{\mu} \theta$ being the normal to the surfaces of constant phase $\theta$;
\item the amplitude $a$ is governed by the evolution equation $$\ell^{\mu}\nabla_{\mu} a = -\frac{1}{2} a (\nabla_{\mu}\ell^{\mu})\, ,$$
which leads to the conservation of the photon number Eq.~\eqref{eq:photcons};
\item the polarization vector $p^{\mu}$ is perpendicular to the light rays, Eq.~\eqref{eq:geom_phase} and it is parallel transported along them Eq.~\eqref{eq:polartransp}.
\end{itemize}

\subsection{Geometric description of light beams}

%The geometric optics approximation provides a description of light propagation in terms of rays. In this view, the image of an astronomical object effectively represents the cross-sectional area of a light beam, namely the bunch of rays emitted by the object and focused at the observer.
%The behaviour of narrow light beams can be described using the geodesic deviation equation (GDE), i.e. the equation describing the tendency of nearby geodesics to converge or diverge from each other due to the curvature of the spacetime. \MG{In this section we will start by introducing the GDE from a geometric prospective and then we will adapt it to describe a typical situation in observational astronomy.}
%A light beam is represented geometrically as a smooth one-parameter family of null geodesics $\gamma_{\rm \tau}(\lambda)$, i.e. for each ${\rm \tau} \in \mathds{R}$ correspond a null geodesic of the light beam parametrised by the affine parameter $\lambda$, and such that the map $(\lambda, \tau) \to \gamma_{\rm \tau}(\lambda)$ is smooth and one-to-one\footnote{The fact that the map $(\lambda, \tau) \to \gamma_{\rm \tau}(\lambda)$ is one-to-one implies that the geodesic of the beam do not cross.}. 
%Defining $\Sigma$ as the two-dimensional submanifold spanned by the geodesics of the family $\gamma_{\rm \tau}(\lambda)$, one can introduce the coordinate base $(\ell^{\mu}, \xi^{\mu})$: the vector $\ell^{\mu}=\left(\frac{\partial}{\partial \lambda}\right)^{\mu}$ is tangent to the family of geodesics and it satisfies Eq.~\eqref{eq:geodesicEQ}). The vector  $\xi^{\mu}=\left(\frac{\partial}{\partial \tau}\right)^{\mu})$ is the deviation vector and it represents the displacement between two nearby geodesics.
%%%%%%%%%%%%%%%%%%%%

The geometric optics approximation provides a description of light propagation in terms of rays. In this view, the image of an astronomical object is effectively presented as the cross section of a light beam, namely the bunch of rays emitted by the object and focused at the observer. However, the curved spacetime between the emitting object and the observer can bend the geodesics and hence produce deformations in the cross section of the beam. The result is that the apparent position, size, shape, and luminosity of the emitter will appear modified to the observer. 
These effects are described using the geodesic deviation equation, which is the equation describing the tendency of nearby geodesics to converge or diverge from each other due to the curvature of the spacetime. 
In this section we will introduce the geodesic deviation equation from a geometric prospective, without any restriction to its application, and we postpone to the next section, Sec.~\ref{sec:BGO}, the implementation of the concepts introduced here to describe a typical situation in observational astronomy.

%%%%%%%%%%%%%%%%%%%%
As we just mentioned, the equation of geodesic deviation expresses the changes of the distance between a point $x^{\mu}(\lambda)$ on one geodesic $\gamma(\lambda)$ to a point $\tilde{x}^{\mu}(\lambda)$ on a nearby geodesic $\tilde{\gamma}(\lambda)$ at the same value\footnote{Here one has to assume that the two geodesics are labelled by the same affine parameter $\lambda$. However, since there is no unique way of relating the affine parameter on one geodesic to the affine parameter on another, we have a degeneracy in the separation vector definition. This will be clarified later.} of $\lambda$. The major assumption is that the two geodesics are close enough, so as we can define a deviation vector $\xi^{\mu}(\lambda)=\tilde{x}^{\mu}(\lambda) - x^{\mu}(\lambda)$ expressing the difference between the two points and such that the geodesic deviation equation is derived by expanding the geodesic equation for $\tilde{\gamma}(\lambda)$ up to linear order in $\xi^{\mu}$. Therefore, the geodesic deviation equation gives only the linear corrections in the deviation vector. Several authors have extended the geodesic deviation equation beyond the linear order, see e.g. \cite{Bazanski1, Puetzfeld:2015uxi, Vines:2014oba}, but the first-order is enough for the purposes of this work.

This way of deriving the geodesic deviation equation requires the introduction of several details which make the derivation difficult to follow. Instead, we decided to present the geodesic deviation equation as derived in~\cite{wald2010general}: on the one hand this derivation has the advantage of having a clear physical interpretation, but on the other hand it hides the perturbative nature of the geodesic deviation equation. Here, we will try to solve this question clarifying where the linearisation in the deviation vector takes place.
%%%%%%%%%%%%%%%%%%%%
Let us start by representing a light beam as a smooth one-parameter family of geodesics $\{\gamma_{\rm \tau}(\lambda)\}$, i.e. for each ${\rm \tau} \in \mathds{R}$ correspond a null geodesic $\gamma_{\tau}(\lambda)$ of the beam parametrised by the affine parameter $\lambda$, and such that the map $(\lambda, \tau) \to \{\gamma_{\rm \tau}(\lambda)\}$ is smooth. %and one-to-one\footnote{The fact that the map $(\lambda, \tau) \to \{\gamma_{\rm \tau}(\lambda)\}$ is one-to-one implies that the geodesics of the family do not cross.}. 
For our purposes we have considered light geodesics, but the derivation we will present is completely general and it does not depends on the nature of the geodesics.
Defining $\Sigma$ as the two-dimensional submanifold spanned by the geodesics of the family $\{\gamma_{\rm \tau}(\lambda)\}$, one can introduce the coordinate base $(\ell^{\mu}, \xi^{\mu})$: the vector $\ell^{\mu}=\frac{d x^{\mu}}{d \lambda}$ is tangent to the family of geodesics and it satisfies Eq.~\eqref{eq:geodesicEQ}). The vector  $\xi^{\mu}=\frac{ d x^{\mu}}{d \tau}$ is the deviation vector and it represents the infinitesimal displacement between two nearby geodesics, see Fig~\ref{fig:congruence}. %It is worth to note that the parametrization of the geodesics is not unique, i.e. it is always possible to transform $\lambda \to \tilde{\lambda}= a(\tau) \lambda +b(\tau)$, with $\tilde{\lambda}$ a new affine parameter for $\gamma_{\tau}$.
%%%%%%%%%%%%%%%%%%%%%%%%%%%%%%%%%%%%%%%%%%%%%%%%%%%%%%%%%%%%%%%%%%%%%%%%%%%%%%%%%%%%%%%%%%%%%%%%%%%%%%%FIGURE
\begin{figure}[ht]
    \centering
    \includegraphics[width=0.9\linewidth]{pict/congruence.pdf}
    \caption{The tangent vector to the geodesics $\ell^{\mu}$ and the deviation vector $\xi^{\mu}$, representing the infinitesimal displacement from a nearby geodesic, characterise the geodesics of the family $\{\gamma_{\rm \tau}(\lambda)\}$.}\label{fig:congruence}
\end{figure}
%%%%%%%%%%%%%%%%%%%%%%%%%%%%%%%%%%%%%%%%%%%%%%%%%%%%%%%%%%%%%%%%%%%%%%%%%%%%%%%%%%%%%%%%%%%%%%%%%%%%%%%
To evaluate the change of the vector field $\xi^{\mu}$, along the flow defined by the tangent vector to the geodesics $\ell^{\mu}$, let us calculate the Lie derivative of $\xi^{\mu}$ with respect to $\ell^{\mu}$ on $\Sigma$
\begin{equation}
\mathcal{L}_{\ell} \xi^{\mu} = \ell^{\nu}\nabla_{\nu} \xi^{\mu}-\xi^{\nu}\nabla_{\nu} \ell^{\mu}=\ell^{\nu}\partial_{\nu} \xi^{\mu}-\xi^{\nu}\partial_{\nu} \ell^{\mu}\, ,\label{eq:lie_gde}
\end{equation}
where last equality follows from the symmetry of the Christoffel symbols $\Gamma^{\lambda}_{\mu \nu}$.
Now, since $\ell^\mu=\frac{\partial x^{\mu}}{\partial \lambda}$ and $\xi^{\mu}=\frac{\partial x^{\mu}}{\partial \tau}$, it is easy to see that the Lie derivative vanishes, implying the commutation of the two vector fields
\begin{equation}
\ell^{\nu}\nabla_{\nu} \xi^{\mu}=\xi^{\nu}\nabla_{\nu} \ell^{\mu}\, . \label{eq:commutation_vector}
\end{equation}
%Lie derivative =0 means that parallel transport of xi along l is the same as the parallel transport of l along xi.

Let us remark that the perturbative nature of the geodesic deviation equation is already assumed when we use the Lie derivative. In fact, from the very definition of the Lie derivative we have
\begin{equation}
- \mathcal{L}_{\ell} \xi^{\mu}=\mathcal{L}_{\xi} \ell^{\mu}=\lim_{\Delta \tau \to 0} \dfrac{1}{\Delta \tau} \left[ \dfrac{\partial x^{\mu}_{0}}{\partial x^{\nu}_{\tau}}\ell^{\nu}(x_{\tau})-\ell^{\mu}(x_{0})\right]\, ,\label{eq:pro1}
\end{equation}
where $\xi^{\mu}$ points from $x^{\mu}_0 \in \gamma_0$ to a nearby point $x^{\mu}_{\tau} \in \gamma_{\tau}$.
%In fact, the Lie derivative $\mathcal{L}_{\ell} \xi^{\mu}$ is noting but the change up to linear order of the flow of $\xi$ along the vector field $\ell$.
%An intuitive reasoning of that can be obtained as follows: in Eq.~\eqref{eq:lie_gde} the Lie derivative $\mathcal{L}_{\ell} \xi^{\mu}$ is evaluated at $x^{\mu}_0 \in \gamma_0$, with $\xi^{\mu}$ the vector pointing from $x^{\mu}_0$ to a closer point $x^{\mu}_{\tau} \in \gamma_{\tau}$. From Eq.~\eqref{eq:commutation_vector} and the definition of the Lie derivative we have
%\begin{equation}
%- \mathcal{L}_{\ell} \xi^{\mu}=\mathcal{L}_{\xi} \ell^{\mu}=\lim_{\Delta \tau \to 0} \dfrac{1}{\Delta \tau} \left[ \dfrac{\partial x^{\mu}_{0}}{\partial x^{\nu}_{\tau}}\ell^{\nu}(x_{\tau})-\ell^{\mu}(x_{0})\right]\, .\label{eq:pro1}
%\end{equation}
We can now see that Eq.~\eqref{eq:commutation_vector}, which is the starting point of this derivation, expresses the vanishing of the linear order expansion of the flow of $\ell^{\mu}$ along the vector field $\xi^{\mu}$.
%The Lie derivative in Eq.~\eqref{eq:lie_gde} is obtained by expanding Eq.~\eqref{eq:pro1} up to linear order in $\Delta \tau$. 

We can also go back from the right hand side of Eq.~\eqref{eq:pro1} to the Lie derivative in Eq.~\eqref{eq:lie_gde} by expanding the first term as an infinitesimal coordinate transform $x^{\mu}_{\tau}$ to $x^{\mu}_0=x^{\mu}_{\tau}- \xi^{\mu} \Delta \tau + \calO(\Delta \tau^2)$ acting on $\ell^\mu$:
\begin{equation}
 \dfrac{\partial x^{\mu}_{0}}{\partial x^{\nu}_{\tau}}\ell^{\nu}(x_{\tau})= \ell^{\mu}(x_{\tau})-\ell^{\nu}(x_{\tau}) \partial_{\nu} \xi^{\mu}(x_0) \Delta \tau + \calO(\Delta \tau^2)\, .\label{eq:pro2}
\end{equation}
Expressing $\ell^{\mu}(x_{\tau})=\ell^{\mu}(x_0)+\xi^{\nu}(x_0)\partial_{\nu}\ell^{\mu}(x_0) \, \Delta \tau + \mathcal{O}(\Delta \tau^2)$ in Eq.~\eqref{eq:pro2}, we indeed obtain the previous expression of the Lie derivative
\begin{equation}
\mathcal{L}_{\xi} \ell^{\mu}= \dfrac{1}{\Delta \tau} \left[\ell^{\mu}(x_{0})+\xi^{\nu}\partial_{\nu}\ell^{\mu}\Delta \tau-\ell^{\nu}\partial_{\nu}\xi^{\mu}\Delta \tau+\calO(\Delta \tau^2)-\ell^{\mu}(x_{0})\right]=\xi^{\nu}\partial_{\nu}\ell^{\mu}-\ell^{\nu}\partial_{\nu}\xi^{\mu}\, .
\end{equation}


%The assumption here is that the two points are close enough such that we can write
%\begin{equation}
%x^{\mu}_{\tau}= x^{\mu}_0 + dx^{\mu}=x^{\mu}_0 + \xi^{\mu} d\tau+ \mathcal{O}(d\tau^2)\, . \label{eq:infinitytransf}
%\end{equation} 
%Now, let us add and subtract $\ell^\mu / d\tau$ in Eq.~\eqref{eq:lie_gde} to obtain
%\begin{equation}
%\mathcal{L}_{\ell} \xi^{\mu} = \dfrac{(\ell^{\mu} + \ell^{\nu}\partial_{\nu} \xi^{\mu}\, d\tau)-(\ell^{\mu} + \xi^{\nu}\partial_{\nu} \ell^{\mu}\, d\tau)}{d\tau}\, .
%\end{equation}
%In this form we can appreciate how the two terms in parenthesis represent expansions of the tangent vector up to first order in $d\tau$: the first term come from considering Eq.~\eqref{eq:infinitytransf} as an infinitesimal coordinate transformation from $x^{\mu}_0$  to $x^{\mu}_{\tau}$ and it represent the tangent vector in the new coordinate, i.e. $\tilde{\ell}^{\mu}(x_\tau)=\frac{\partial x^{\mu}_\tau}{\partial x^{\nu}_0}\ell^{\nu}(x_0)=\ell^{\mu}(x_0)+\ell^{\nu}(x_0)\partial_{\nu}\xi^{\mu} \, d\tau + \mathcal{O}(d\tau^2)$. The second term is the linear expansion of the tangent vector evaluated at $x^{\mu}_{\tau}$, i.e. $\ell^{\mu}(x_\tau)=\ell^{\mu}(x_0+dx)=\ell^{\mu}(x_0)+\xi^{\nu}(x_0)\partial_{\nu}\ell^{\mu} \, d\tau + \mathcal{O}(d\tau^2)$. In conclusion, the Lie derivative Eq.~\eqref{eq:lie_gde} contains only linear terms in $d \tau$, namely is valid for small displacements.

Let us move back on the derivation. Following the interpretation that $\xi^{\mu}$ represents the displacement between nearby geodesics, the left hand side of Eq.~\eqref{eq:commutation_vector} defines the relative velocity between geodesics. Similarly, the relative acceleration between the geodesics of the family is $w^{\mu}=\ell^{\rho}\nabla_{\rho} (\ell^{\nu}\nabla_{\nu} \xi^{\mu})$. From Eq.~\eqref{eq:commutation_vector} then we have%\footnote{The second line is simply the Leibniz rule. The third line replaces a double covariant derivative by the derivatives in the opposite order plus the Riemann tensor. In the fourth line we use Leibniz again (in the opposite order from usual), and then we cancel two identical terms and use the geodesic equation for $\ell^{\mu}$.} 
\begin{align}
w^{\mu}&=\ell^{\rho}\nabla_{\rho}(\xi^{\nu}\nabla_{\nu} \ell^{\mu}) \nonumber \\
 & = (\ell^{\rho}\nabla_{\rho}\xi^{\nu})(\nabla_{\nu} \ell^{\mu}) + \ell^{\rho}\xi^{\nu}\nabla_{\rho}\nabla_{\nu} \ell^{\mu} \nonumber \\
& = (\xi^{\rho}\nabla_{\rho}\ell^{\nu})(\nabla_{\nu} \ell^{\mu}) + \ell^{\rho}\xi^{\nu}(\nabla_{\nu}\nabla_{\rho} \ell^{\mu}+R\UD{\mu}{\sigma \rho \nu }\ell^{\sigma}) \nonumber \\
& = (\xi^{\rho}\nabla_{\rho}\ell^{\nu})(\nabla_{\nu} \ell^{\mu}) + \xi^{\nu}\nabla_{\nu}(\ell^{\rho}\nabla_{\rho} \ell^{\mu})-(\xi^{\nu}\nabla_{\nu}\ell^{\rho})(\nabla_{\rho} \ell^{\mu})+R\UD{\mu}{\sigma \rho \nu }\ell^{\sigma}\ell^{\rho}\xi^{\nu} \nonumber \\
& = R\UD{\mu}{\sigma \rho \nu }\ell^{\sigma}\ell^{\rho}\xi^{\nu}\, .
 \label{eq:acc_gde}
\end{align}
The result is the geodesic deviation equation (\setwd{GDE}{acr:GDE}), 
\begin{equation}
\ell^{\rho}\nabla_{\rho} (\ell^{\nu}\nabla_{\nu} \xi^{\mu})=R\UD{\mu}{\sigma \rho \nu }\ell^{\sigma}\ell^{\rho}\xi^{\nu}\, ,\label{eq:GDE} 
\end{equation}
which relates the relative acceleration between infinitesimally close geodesics with the spacetime curvature, \cite{wald2010general}. The GDE for timelike geodesics plays an important role in the foundation of General Relativity, since it can be used to characterize the spacetime curvature as the relative motion of free falling bodies. This was covered by many authors, see e.g. \cite{pirani, szekeres, Bazanski1, Bazanski2, Aleksandrov1979, CiufoliniDemianski, Vines:2014oba, Puetzfeld:2015uxi, Flanagan:2018yzh}). 
For analysis on the GDE for null geodesics see \cite{Bartelmann_2010, Clarkson:2016zzi, Clarkson:2016ccm, Korzynski:2018, Uzun:2018yes, Grasso:2018mei}.

\subsection{Properties of the GDE}
Let us discuss now two general properties of the  GDE solutions which hold irrespectively of the underlying geometry, as shown in \cite{Vines:2014oba, Korzynski:2018, Grasso:2018mei}.
The first property is derived by multiplying the GDE, Eq.~\eqref{eq:GDE}, by $\ell_{\mu}$ to obtain
\begin{equation}
\nabla_{\ell} \nabla_{\ell} \left(\ell_{\mu} \xi^\mu \right)=0\, ,
\label{eq:gde_prop_in}
\end{equation}
with $\nabla_{\ell}=\ell^{\sigma} \nabla_{\sigma}$, and we use the fact that $\ell^{\mu}$ is a solution of Eq.~\eqref{eq:geodesicEQ} and $\ell_{\mu} R\UD{\mu}{\alpha \beta \nu}\ell^{\alpha}\ell^{\beta}=0$ from the symmetry of the Riemann tensor. Then we have 
\bean
\xi^\mu \, \ell_\mu = C + D\,\lambda,
\eean
with $C,D = \const$. In this way we have defined 2 constants of motion for the GDE, namely the quantities %\footnote{It is worth to note that in the case that the geodesics of the family are all null, one has $D=\nabla_{\ell}\xi^\mu\, \ell_\mu=0$. This is derived multiplying by $\ell_{\mu}$ the relation in Eq.~\eqref{eq:commutation_vector} and use the fact that $\xi^{\nu}(\nabla_{\nu} \ell^\mu)\ell_\mu=\frac{1}{2}\xi^{\nu}\nabla_{\nu}( \ell^\mu \ell_\mu)=0$. For null geodesics last equality is always satisfied since follows from the fact that $\ell^\mu \ell_\mu=0$. For any other type of geodesics, where $\ell^\mu \ell_\mu=\epsilon$, one has to impose that the normalization of the tangent vector is the same along all the other geodesics of the family, i.e. $\nabla_{\xi} \epsilon=0$.}
\begin{align}
D = (\nabla_{\ell}\xi^\mu)\, \ell_\mu \label{eq:gde_prop1B}\\
C = \xi^\mu \, \ell_\mu - D\,\lambda\, , \label{eq:gde_prop1A}
\end{align}
are conserved along the geodesics.

The second property states that if $\xi^\mu(\lambda)$ is a solution of the GDE, then also $\tilde{\xi}^\mu =  \xi^\mu + \alpha(\lambda)\,\ell^\mu$ is  a solution. The form of the proportionality function $\alpha(\lambda)$ is easily obtained by inserting $\tilde{\xi}^\mu$ into Eq.~\eqref{eq:GDE} to find that $\alpha(\lambda)= (E + F\,\lambda)$, with $E,F = \const$. In other words, we have that
\bea
\tilde{\xi}^\mu =  \xi^\mu + (E + F\,\lambda)\,\ell^\mu \label{eq:gde_prop2}
\eea
is a solution of Eq.~\eqref{eq:GDE}.
This ``gauge freedom'' of adding terms proportional to $\ell^\mu$ to the solution of the GDE is a direct consequence of the freedom we have in choosing the affine parametrisation of a geodesic. In fact, as previously discussed, if $\lambda$ is the affine parameter of the geodesic $\gamma_0 \in \{\gamma_{\tau}\}$, then $A(\tau)\, \lambda + B(\tau)$ is the only possible form for the affine parameter of any of the other geodesics of the family $\{\gamma_{\tau}\}$.
Geometrically, $\tilde{\xi}^\mu$ corresponds to the same congruence of geodesics as $\xi^\mu$, but with a change of parametrisation of the geodesics around $\gamma_0$, see Fig.~\ref{fig:prop2}. 
%%%%%%%%%%%%%%%%%%%%%%%%%%%%%%%%%%%%%%%%%%%%%%%%%%%%%%%%%%%%%%%%%%%%%%%%%%%%%%%%%%%%%%%%%%%%%%%%%%%%%%%FIGURE
\begin{figure}[ht]
    \centering
    \includegraphics[width=0.9\linewidth]{pict/prop2.pdf}
    \caption{The two deviation vectors $\xi^{\mu}$ and $\tilde{\xi}^{\mu}=\xi^{\mu}+\alpha(\lambda)\ell^{\mu}$ identify the same geodesic $\gamma_{\tau}$. This invariance of the GDE is related to the gauge freedom one has in choosing the affine parameter. Indeed, the different deviation vector $\tilde{\xi}^{\mu}$ can be obtained by introducing the different affine parametrisation $\lambda'$ of the geodesic $\gamma_{\tau}$.
}\label{fig:prop2}
\end{figure}
%%%%%%%%%%%%%%%%%%%%%%%%%%%%%%%%%%%%%%%%%%%%%%%%%%%%%%%%%%%%%%%%%%%%%%%%%%%%%%%%%%%%%%%%%%%%%%%%%%%%%%%

Assuming that the character of the geodesics is conserved, i.e. the geodesics of the family are all of the same type, we can draw some conclusions regarding $C$ and $D$. Let us start by using Eq.~\eqref{eq:gde_prop1A} to write Eq.~\eqref{eq:gde_prop2} as
\begin{equation}
\tilde{C} + \lambda \tilde{D}=\tilde{\xi}^{\mu}\ell_{\mu}=C + \lambda D+(E+F \lambda)\ell^{\mu}\ell_{\mu}\, ,
\end{equation}
where $C + \lambda D=\xi^{\mu}\ell_{\mu}$, and the equality is satisfied for
\begin{align}
\tilde{C} &= C + E\, \ell^{\mu}\ell_{\mu} \\
\tilde{D} &= D + F\, \ell^{\mu}\ell_{\mu} \, .
\end{align}
We distinguish two cases: 
\begin{itemize} 
\item for a time-like or space-like family we have $\ell^\mu \ell_\mu=\epsilon \neq 0$, implying that it is always possible to choose a different reparametrisation of the geodesics around $\gamma_0$ such that the vectors $\xi^\mu$ and $\ell^{\mu}$ stay perpendicular along $\gamma_0$, namely
\begin{align}
\tilde{C} &= C + \dfrac{-C}{\ell^{\mu}\ell_{\mu}}\, \ell^{\mu}\ell_{\mu}=0 \label{eq:gdeRel1}\\
\tilde{D} &= D + \dfrac{-D}{\ell^{\mu}\ell_{\mu}}\, \ell^{\mu}\ell_{\mu}=0 \label{eq:gdeRel2}\, .
\end{align}
\item for a null family one has $\ell^\mu \ell_\mu=0$ for all geodesics, implying that for any choice of $E$ and $F$ we have $\tilde{C} = C$ and $\tilde{D} = D$.
\end{itemize} 
%%%%%%%%%%%%%%%%%%
%The freedom in choosing a proper affine reparametrization of the geodesics around $\gamma_0$ such that $D=C=0$, i.e. the vectors $\xi^\mu$ and $\ell^{\mu}$ stay perpendicular along $\gamma_0$;
We conclude that for a congruence of null geodesics there is no\footnote{Note that we are not saying that $\xi^\mu \ell_{\mu}\neq 0$ for null geodesics. This is possible and it represent a specific choice of the initial conditions for the null geodesics of the family. What we meant is that if $\xi^\mu \ell_{\mu}\neq 0$, then it is not possible to chose a proper affine reparametrisation to transform $\xi^\mu$ into a vector orthogonal to $\ell^\mu$.} affine reparametrisation of the geodesics around $\gamma_0$ which makes the deviation vector $\xi^\mu$ perpendicular to  $\ell^{\mu}$, \cite{Korzynski:2018, Grasso:2018mei}. In this sense the null families represent a special class of families of geodesics.

Let us proceed in our analysis by noting that Eq.~\eqref{eq:gde_prop1B} can be expressed as $D=\frac{1}{2}\xi^{\nu}\nabla_{\nu}(\ell^\mu \ell_\mu)$, which follows multiplying by $\ell_{\mu}$ the relation in Eq.~\eqref{eq:commutation_vector}, and using the equality\footnote{It is derived as follows: $\nabla_{\nu}( \ell^\mu \ell_\mu)=(\nabla_{\nu} \ell^\mu)\ell_\mu+(\nabla_{\nu} \ell_\mu)\ell^\mu=(\nabla_{\nu} \ell^\mu)\ell_\mu+[\nabla_{\nu}(g_{\mu \sigma} \ell^\sigma)]g^{\mu \rho} \ell_\rho$. Using $\nabla_{\nu}g_{\mu \sigma}=0$, we have
$\nabla_{\nu}( \ell^\mu \ell_\mu)=(\nabla_{\nu} \ell^\mu)\ell_\mu+(\nabla_{\nu} \ell^\sigma)\delta\DU{ \sigma}{ \rho} \ell_\rho=2(\nabla_{\nu} \ell^\mu)\ell_\mu=2(\nabla_{\nu} \ell_\mu)\ell^\mu$.} $\xi^{\nu}(\nabla_{\nu} \ell^\mu)\ell_\mu=\frac{1}{2}\xi^{\nu}\nabla_{\nu}(\ell^\mu \ell_\mu)$. Again we have different conclusions depending on the character of the family: 
\begin{itemize}
\item for a time-like or space-like family we have a preferred normalisation of the tangent vector $\ell^\mu \ell_\mu=\epsilon$, which in general may differ among the geodesics of the family. In other words, the value of $\ell^\mu \ell_\mu=\epsilon$ changes along $\xi^{\mu}$, implying that $D=\xi^{\nu}\nabla_{\nu}(\ell^\mu \ell_\mu)\neq 0$. To impose $D=0$ we need to perform a reparametrisation of the other geodesics of the family such that the normalisation of the tangent vector remains constant: this is precisely what we demanded in Eq.~\eqref{eq:gdeRel2}.
\item for a null family one has $\ell^\mu \ell_\mu=0$ for all geodesics, implying $D=0$ and $C = \xi^\mu \, \ell_\mu={\rm const}$ irrespectively of the parametrisation. 
\end{itemize} 
The condition $D=0$ for null geodesics leads to the \emph{flat lightcone approximation} (\setwd{FLA}{acr:FLA}) for the time of arrival of the electromagnetic signals\footnote{It will be clarified later when we introduce the semi-null frame.}, which is a direct consequence of the linearity of the GDE in $\xi^{\mu}$, \cite{Grasso:2018mei}. 
Indeed, if $\gamma_0$ is a null geodesic with tangent vector $\ell^{\mu}$ and $\gamma$ is a nearly displaced geodesic of the same family, with tangent vector $k^{\mu}=\ell^{\mu}+\nabla_{\ell} \xi^{\mu}$, the condition for $\gamma$ to remain null reads 
\begin{equation}
g_{\mu \nu}(\ell^{\mu}+ \nabla_{\ell} \xi^{\mu})(\ell^{\nu}+ \nabla_{\ell} \xi^{\nu})= 2 \ell_{\mu}\nabla_{\ell} \xi^{\mu}+\nabla_{\ell} \xi_{\mu}\nabla_{\ell} \xi^{\mu}=0\, , \label{eq:nulldisplaced}
\end{equation} 
where we have already removed the term $\ell_{\mu}\ell^{\mu}=0$. Since we are considering small displacements, the term $\nabla_{\ell} \xi_{\mu}\nabla_{\ell} \xi^{\mu}$ is quadratic in $\xi^{\mu}$ and it can be neglected. The null condition for $\gamma$ reduces to $\ell_{\mu}\nabla_{\ell} \xi^{\mu}=0$, that from Eq.~\eqref{eq:gde_prop1B} reads $D=0$. 


%%%%%%%%%%%%%%%%%%%%%%%%%%%%%%%%%%%%%%%%%%%%%%%%%%%%%%%%%%

\section{The bilocal geodesic operators}
\label{sec:BGO}
%%%%%%%%%%%%%%%%%%%%%%%%%%%%%%%%%%%%%%%%%%%%%%%%%%%%%%%%%%%%%%%%%%%%%%%%%%%%%%%%%%%%%%%%%%%%%%%%%%%%%%%%%%%%%%%

%\MG{physical representation per introdurre i BGO: geometric setup, scrivi gde in forma di deviazioni e poi come combinazione lineare di BGO}

%The general solution of the GDE around a null geodesics allows for studying the behaviour of the rays of light in a completely covariant manner. This is precisely the topic of the next section, in which we will propose a new formulation of the GDE in terms of bilocal operators.
%The GDE prescribes how the curvature of the spacetime modifies the relative distance between nearby geodesics, therefore its general solution around a null geodesics allows for studying the behaviour of the rays of light in a completely covariant manner. \MG{In this section we apply these concepts to describe a typical situation in observational astronomy where an observer receives the light from distant sources.}


In the geometric optics regime, the geodesic equation~\eqref{eq:geodesicEQ} and the GDE~\eqref{eq:GDE} are the two master equations governing light propagation in the presence of curvature.
%We have seen how electromagnetic radiation in cosmology and astrophysics can be treated within the geometric optics approximation, in which light propagation is ruled by the geodesic equation~\eqref{eq:geodesicEQ} and the GDE~\eqref{eq:GDE}. \MG{In this section we apply these concepts to describe a typical situation in observational astronomy where an observer receives the light from distant sources.}
In the following we apply these concepts to describe a typical situation in observational astronomy.

Consider the physical system consisting of an observer $\calO$ and a source $\mathcal{S}$ far apart and moving freely along their timelike worldlines. %Naming $N_{\mathcal{O}}$ and $N_{\mathcal{S}}$ the regions of the spacetime in which the observer and the source are moving, we assume that $N_{\mathcal{O}}$ and $N_{\mathcal{S}}$ are causally connected, so that any signal emitted by $\mathcal{S}$ is received by $\mathcal{O}$ at any later time. %We also assume that the typical length scales of $N_{\mathcal{O}}$ and $N_{\mathcal{S}}$ are small compared to the distance between them.
We denote the regions of spacetime where the observer and the source are moving $N_{\mathcal{O}}$ and $N_{\mathcal{S}}$, and we suppose that $N_{\mathcal{O}}$ and $N_{\mathcal{S}}$ are causally connected, meaning that any signal emitted by $\mathcal{S}$ is received by $\mathcal{O}$ at any later time.
If  $L$ is typical size of $N_\calO$ and $N_\calE$, it must be much smaller than the characteristic curvature scale of the spacetime $\mathcal{R}$, i.e. $L \ll \mathcal{R}$.
In this case, we may %introduce in both regions locally flat coordinate systems, centred at $\calO$ and $\calE$ respectively, in which the metric tensor is the flat Minkowski metric up to quadratic terms in $x^\mu$:
%\beq
%g_{\mu\nu} = \eta_{\mu\nu} + O(x^2).
%\eeq
%The additional terms are due to the local spacetime curvature and therefore scale like $(L/\mathcal{R})^2$. We may consider them negligibly small because the size of both regions $L$ is too small for any curvature effects to be directly detected by experiments performed within each region. 
%Therefore we
 effectively treat both $N_\calO$ and $N_\calE$ as flat, and use special relativity to describe the effects on light propagation in these regions. From a geometric prospective, the local flatness of the two regions allows us to identify points in $N_\calO$ and $N_\calE$ with points in the corresponding tangent spaces $T_{x_\calO} \calM $ and $ T_{x_\calS} \calM$, up to quadratic terms in $x^\mu$. %Stating the same in a more coordinate-invariant manner: with the curvature effects being negligible in regions of size $L$,  we may  simply identify the points in $N_\calO$ and $N_\calE$ with points in the tangent spaces $T_\calO \calM$ and $T_\calE \calM$ in the vicinity of $0$, using, for example, the exponential map. Under this identification, the physical spacetime metric agrees with the flat metric on the appropriate tangent space up to quadratic terms in the distance from 0.

\bfi
\includegraphics[width=0.9\textwidth]{pict/geometricsetup.pdf}
\caption{The observer $\calO$ and the source $\calS$ are free to move along their worldlines in the two small regions $N_\calO$ and $N_\calE$. The point $x^{\mu}_{\calS}$ on the $\calS$'s worldline is connected to $x^{\mu}_{\calO}$ on the $\calO$'s worldline by the null geodesic $\gamma_0$. $\gamma_0$ can be identified by the initial position and initial tangent vector at the observer $(x^{\mu}_{\calO}, \ell^{\mu}_{\calO})$. Since the observer and source are free to move, they will be connected at a later time by another geodesic $\gamma$ identified by the displacement vectors $(\delta x_\calO^\mu, \Delta \ell_\calO^\mu)$.}
\label{fig:geometricsetup}
\efi

Since $N_\calO$ and $N_\calE$ are causally connected, it is possible to find a \emph{fiducial null geodesic} $\gamma_0$  going from $\calE$ to $\calO$, i.e. such that $\gamma(\lambda_{\mathcal{S}})=x^{\mu}_{\mathcal{S}}$ and $\gamma(\lambda_{\mathcal{O}})=x^{\mu}_{\mathcal{O}}$ are the source's and observer's positions, respectively\footnote{Note that the two values $\lambda_{\mathcal{S}}$ and $\lambda_{\mathcal{O}}$, for which $\gamma(\lambda_{\mathcal{S}})=x^{\mu}_{\mathcal{S}}$ and $\gamma(\lambda_{\mathcal{O}})=x^{\mu}_{\mathcal{O}}$, may change if we consider a different affine parameter for $\gamma_0$. However, in geometric optics we are interested only in the question whether or not a null geodesic passes through a given event and what null direction it follows at that moment.}. The vector $\ell_\calO^\mu$ is the tangent vector to $\gamma_0$ at $\calO$ and $\ell_\calE^\mu$ is the corresponding tangent vector at $\calE$.
To simplify the notation we will denote $T_{x_\calO} \calM$ by $T_\calO \calM$ and $T_{x_\calS} \calM$ by $T_\calS \calM$. %QUII In this view, these relations are invariant with respect to reparametrization, because the value of the affine parameter  at which the null geodesic intersects a given point carries no physical meaning. Therefore we may identify all null geodesics which share the same path and consider affine reparametrizations (\ref{eq:affineparameter})  gauge transformations from the point of view of geometric optics. They should therefore leave all physical observables invariant.}. %For convenience we also assume that we have parametrised our geodesic backwards in time, i.e. $\lambda_\calO < \lambda_\calE$.
%Now, in Sec.~\ref{sec:Maxwell_to_GeomOpt} we have defined the light beam as all geodesics connecting $\calS$ and $\calO$. 
Now, we generalise the definition of light beam by considering all null geodesics connecting points from $N_\calO$ with $N_\calE$, which are contained in a four-dimensional tube around $\gamma_0$. We consider that this tube is sufficiently narrow such that we can use the first-order geodesic deviation equation, Eq.~\eqref{eq:GDE}, for describing the deviation between these geodesics.
Within this assumptions, the geodesics are uniquely specified by giving their initial points and initial tangent vectors in $N_\calO$ (or in $N_\calE$).
Alternatively, we can characterize the geodesics by their deviation from the fiducial null geodesic $\gamma_0$: we use this second method of  identification, see Figure \ref{fig:geometricsetup}, defining the \emph{initial displacement vector} $\delta x(\lambda_\calO) \in T_\calO \calM$ as the displacement between two nearby geodesics of the family $\gamma$ ad $\gamma_0$ at $N_{\calO}$
\bea
\delta x^{\mu}(\lambda_\calO)\equiv \delta x_\calO^\mu =y^{\mu}_\calO - x^{\mu}_\calO\, , \label{eq:deltaxOdef}
\eea
where $y^{\mu}_\calO=\gamma(\lambda_\calO)$ and $x^{\mu}_\calO= \gamma_0(\lambda_\calO)$.
Note that the displacement $\delta x^{\mu}_\calO$ can be in all spatial and temporal directions.
%To describe the displacement in the tangent vector one may be tempted of using $\delta \ell^{\mu}_{\calO}=\frac{d \delta x_\calO}{d \lambda}=\frac{d y_\calO}{d \lambda}-\ell_\calO$, where $\frac{d y_\calO}{d \lambda}$ is the tangent vector to $\gamma(\lambda_{\calO})$. However, $\delta \ell_\calO^\mu$ defined this way is not a proper vector, because it does not transform like a vector under general coordinate system transformations. As we know from elementary differential geometry, this is because we have subtracted here the components of vectors defined at two close but distinct points. Of course, this can be fixed by adding an appropriate term involving the Christoffel symbols: we define
The variation of $\delta x^{\mu}_\calO$ along the fiducial geodesic gives the \emph{initial direction deviation vector} $\Delta \ell_\calO^\mu$, defined as 
 \bea
\Delta \ell^\mu_\calO \equiv \left.\nabla_{\ell}\delta x^{\mu}(\lambda)\right|_\calO= \delta \ell_\calO^\mu + \Gamma\UD{\mu}{\nu\sigma}(x_\calO)\, \ell_\calO^\nu \,\delta x_\calO^\sigma\, , \label{eq:DeltalOdef}
\eea
where $\Delta \ell^\mu_\calO\equiv \Delta \ell^\mu(\lambda_\calO)$, $\delta \ell_\calO^\mu\equiv\delta \ell^\mu(\lambda_\calO)= \left.\frac{d \delta x^\mu}{d \lambda}\right|_\calO$, and $\Gamma\UD{\mu}{\nu\sigma}(x_\calO)$ are the Christoffel symbols at $\calO$.  
%The resulting expression $\Delta \ell_\calO^\mu$ parametrises the initial tangent vector equally well, but unlike the ``bare'' $\delta \ell_\calO^\mu$ it is a proper vector. From the geometric point of view, equation (\ref{eq:DeltalOdef}) yields the \emph{same} vector in $T_\calO \calM$ independently of the coordinate system chosen.
%The vector $\Delta \ell_\calO^\mu$ is the \emph{initial direction deviation vector}  and expresses the difference between $\frac{d y_\calO}{d \lambda}$ parallel transported from $y^{\mu}_{\calO}$ to $x^{\mu}_{\calO}$ and $\ell_\calO^\mu$. 
The  pair $(\delta x_\calO^\mu, \Delta \ell_\calO^\mu)$ labels all the geodesics in the vicinity of $\gamma_0$ and will be referred to as \emph{the displacement vectors}.
Let us notice that the choice of setting the initial displacement and direction deviation at $N_{\calO}$ is arbitrary. In the rest of the chapter we adopt this choice but in principle we could choose to parametrise the geodesics of the family by starting from $N_{\calE}$ giving $(\delta x_\calE^\mu, \Delta \ell_\calE^\mu)$. This second method is described  in \cite{Grasso:2021iwq} and it is one of the original results of this thesis.
%The choice of the particular value $\lambda_\calO$ of the affine parameter $\lambda$ at which we parametrise the initial displacement and direction deviation is arbitrary: in principle we could choose any point  and any value for that purpose. This choice is nevertheless consistent with the assumption that the geodesics are only linearly perturbed with respect to the fiducial one: we may expect that for slightly perturbed geodesics the endpoint given by $\lambda = \lambda_\calO$ will lie very close to the corresponding endpoint of $\gamma_0$, i.e. $\calO$, and therefore within $N_\calO$. Thus we can parametrise the position by a small displacement vector. The same reasoning applies to the other endpoint.


%%%%%%%%%%%%%%%%%%%%%%%%%%%%%%%%%%%%%%%%%%%%%%%%%%%%%%%%%%%%%%%%%%%%%%%%%%%%%%%%%%%%%%%%%%%%%%%%%%%%%%%%%%%%%%%%%%%%%%%%%%%%%%%%%%%%%%%%%%%%%%%%%%%%%%%%%%%%%%%
Since the geodesics are expected to be confined within the narrow tube all along $\gamma_0$, the initial displacement vectors must be small. In this case, their propagation from $\calO$ to $\calE$ is described by the GDE
\bea
\nabla_{\ell} \nabla_{\ell} \delta x^{\mu} - R\UD{\mu}{\alpha\beta\nu}\,\ell^\alpha\,\ell^\beta\,\delta x^\nu = 0\, , \label{eq:GDE_delta}
\eea
with the initial data
\bea
\left. \delta x^\mu(\lambda)\right|_\calO &=& \delta x_\calO^\mu  \label{eq:GDEID1}\\
\left. \nabla_{\ell} \delta x^\mu(\lambda)\right|_\calO &=& \Delta \ell_\calO^\mu. \label{eq:GDEID2}
\eea
The solution gives the displacements at the other end for $\lambda=\lambda_\calE$: $\delta x^\mu(\lambda_\calE)=\delta x_\calE^\mu$ and $\left. \nabla_{\ell} \delta x^\mu(\lambda)\right|_\calS=\Delta \ell_\calE^\mu $. The combination $R\UD{\mu}{\alpha\beta\nu}\,\ell^\alpha\,\ell^\beta$ is the \emph{optical tidal matrix} expressing the spacetime curvature along the line of sight $\gamma_0$.
%The condition for applicability of the GDE means that the gravitational lensing produces only a linear distortion of the image of all objects in $N_\calE$, which excludes the possibility of multiple imaging for light rays contained within the tube. 
The condition for applicability of the GDE excludes the possibility of multiple imaging for light rays contained within the tube. 

%%%%%%%%%%%%%%%%%%%%%%%%%%%%%%%%%%%%%%%%%%%%%%%%%%%%%%%%%%%%%%%%%%%%%%%%%%%%%%%%%%%%%%%%%%%%%%%%%%%%%%%%%%%%%%%%%%%%%%%%%%%%%%%%%%%%%%%%%%%%%%%%%%%%%%%
%Since the GDE is linear, the solution at $\lambda_\calE$ must be a linear function of the initial data at $\calO$:
Due to the linearity of the GDE, the solutions at $\lambda_{\calS}$ are given as linear combination of the initial conditions $(\delta x^{\mu}_{\calO}, \, \Delta \ell^{\mu}_{\calO})$ 
\bea
\delta x_\calE^\mu = { W_{XX} }\UD{\mu}{\nu}\,\delta x_\calO^\nu + { W_{XL} }\UD{\mu}{\nu}\,\Delta \ell_\calO^\nu \label{eq:positiondeviation1} \\
\Delta \ell_\calE^\mu = { W_{LX} }\UD{\mu}{\nu}\,\delta x_\calO^\nu + { W_{LL} }\UD{\mu}{\nu}\,\Delta \ell_\calO^\nu \label{eq:directiondeviation1},
\eea
with $\WXX$, $\WXL$, $\WLX$, $\WLL$ being bilocal operators  (also known as 2-point tesors \cite{SyngeBook} or bitensors \cite{Poisson2011, Vines:2014oba}), acting from $T_\calO \calM$ to $T_\calE \calM$. 
We refer to the four operators $W_{XX}$, $W_{XL}$, $W_{LL}$ and $W_{LX}$ as the \emph{bilocal geodesic operators} (\setwd{BGO}{acr:BGO}), \cite{Grasso:2018mei}. In the context of timelike geodesics the first two are the \emph{Jacobi propagators} $K$ and $H$ introduced in \cite{DeWittBrehme, Dixon2, Vines:2014oba}. Moreover, the BGO can also be related to the Synge's worldfunction \cite{SyngeBook, Dixon2, Vines:2014oba}. Recently the BGO defined along a timelike geodesic have  been used as a tool to study of the gravitational waves memory effect \cite{Flanagan:2018yzh}. Here, we will focus exclusively on the application of the BGO to describe null geodesics as presented in \cite{Grasso:2018mei}. The notation we introduced in Eqs.~\eqref{eq:positiondeviation1}-\eqref{eq:directiondeviation1} highlights their relation with the resolvent operator, or the Wro\'nski matrix \cite{Fleury:2014gha} for the GDE, 
\begin{equation}
\mathcal{W}=\begin{pmatrix}
\WXX{}\UD{\mu}{\nu} && \WXL{}\UD{\mu}{\sigma}\\
\WLX{}\UD{\rho}{\nu} && \WLL{}\UD{\rho}{\sigma}
\end{pmatrix}\, .
\end{equation}
$\mathcal{W} = \mathcal{W}(\calE, \calO)$ is the linear mapping between vector sums of two copies of the tangent space, i.e.
\beq
\mathcal{W}: T_\calO \calM \oplus T_\calO \calM \to T_\calE \calM \oplus T_\calE \calM\, ,
\eeq
defined by the relation 
\begin{equation}
\begin{pmatrix}
\delta x^{\mu}_{\calS}\\
\Delta \ell^{\rho}_{\calS}
\end{pmatrix}= \begin{pmatrix}
\WXX{}\UD{\mu}{\nu} && \WXL{}\UD{\mu}{\sigma}\\
\WLX{}\UD{\rho}{\nu} && \WLL{}\UD{\rho}{\sigma}
\end{pmatrix}\begin{pmatrix}
\delta x^{\nu}_{\calO}\\
\Delta \ell^{\sigma}_{\calO}
\end{pmatrix}=\mathcal{W}\begin{pmatrix}
\delta x^{\nu}_{\calO}\\
\Delta \ell^{\sigma}_{\calO}
\end{pmatrix}\, .
\label{eq:deviations_compact}
\end{equation}
It is also a symplectic mapping, as noted by Uzun \cite{Uzun:2018yes}, since in GR the ordinary differential equations (\setwd{ODE}{acr:ODE}) for null geodesics can be expressed as a Hamiltonian system, both in general and in the first-order perturbation theory \cite{Fleury:2014gha}. In contrast to other approaches, here we evaluate displacements in all four dimensions, including time. The Wro\'nski matrix formalism extends to the fully four-dimensional GDE without any problems, preserving its properties (as shown in \cite{Julius-spherical}).
%%%%%%%%%%%%%%%%%%%%

It follows easily from the geodesic deviation equation (\ref{eq:GDE_delta}) and from Eqs.~(\ref{eq:positiondeviation1})-(\ref{eq:directiondeviation1}) that the BGO can be expressed as solutions to the GDE along $\gamma_0$ and with initial data at $\lambda_\calO$. %Assume we fix a parallel-propagated frame $e_{\boldsymbol{\mu}}$ along the null geodesic.
Let us start by using the fact $\nabla_{\ell} \delta x^{\mu}(\lambda)=\Delta \ell^{\mu}(\lambda)$ to express the GDE~\eqref{eq:GDE_delta} as a system of two first-order ODE as
\begin{equation}
\left\{ \begin{matrix}
\nabla_{\ell} \delta x^{\mu}= \Delta \ell^{\mu}\\
\\
\nabla_{\ell} \Delta \ell^{\rho}= R\UD{\rho}{\alpha\beta\nu}\,\ell^\alpha\,\ell^\beta \delta x^{\nu}
\end{matrix}\right.\, , 
\end{equation}
with initial conditions
\begin{align}
\delta x^{\mu}(\lambda_{\mathcal{O}})&=\delta x^{\mu}_{\mathcal{O}}\\
\Delta \ell^{\rho}(\lambda_{\mathcal{O}})&= \Delta \ell^{\rho}_{\mathcal{O}}\, .
\end{align}
In a more compact form the system becomes
\begin{equation}
\nabla_{\ell} \begin{pmatrix}
\delta x^{\mu}\\
\Delta \ell^{\rho}
\end{pmatrix}= \begin{pmatrix}
0 && \delta\UD{\mu}{\sigma}\\
R\UDDD{\rho}{\ell}{\ell}{\nu} && 0
\end{pmatrix}\begin{pmatrix}
\delta x^{\nu}\\
\Delta \ell^{\sigma}
\end{pmatrix}\, ,
\label{eq:GDE_deviations}
\end{equation}
where we defined $R\UD{\rho}{\alpha\beta\nu}\,\ell^\alpha\,\ell^\beta=R\UD{\rho}{\ell \ell \nu}$.
Making use of Eq.~\eqref{eq:deviations_compact}, one obtains a matrix ODE for $\mathcal{W}$
\begin{equation}
\nabla_{\ell} \mathcal{W}= \begin{pmatrix}
0 && \mathbb{1} \\
R_{\ell \ell} && 0
\end{pmatrix} \cdot \mathcal{W}\, ,
\label{eq:GDE_W}
\end{equation}
where the dot indicates usual matrix product operation. The initial conditions for the BGO are
\begin{equation}
\mathcal{W}= \mathbb{1}_{8 \times 8}\, .
\end{equation}

Let us clarify that the action of the covariant derivative is intended as acting on bitensors \cite{Poisson2011}, i.e. 
\begin{equation}
\nabla_{\ell} \mathcal{W}= \begin{pmatrix}
\partial_{\ell} \WXX{}\UD{\mu}{\nu}+ \Gamma^{\mu}_{\alpha \beta}\WXX{}\UD{\alpha}{\nu}\ell^{\beta}  && \partial_{\ell} \WXL{}\UD{\mu}{\sigma}+ \Gamma^{\mu}_{\alpha \beta}\WXL{}\UD{\alpha}{\sigma}\ell^{\beta}  \\
\partial_{\ell} \WLX{}\UD{\rho}{\nu} + \Gamma^{\rho}_{\alpha \beta}\WLX{}\UD{\alpha}{\nu}\ell^{\beta}  && \partial_{\ell} \WLL{}\UD{\rho}{\sigma}+ \Gamma^{\rho}_{\alpha \beta}\WLL{}\UD{\alpha}{\sigma}\ell^{\beta}
\end{pmatrix}\, .
\end{equation}
%\begin{equation}
%\nabla_l \begin{pmatrix}
%\WXX{}\UD{\mu}{\nu} && \WXL{}\UD{\mu}{\sigma}\\
%\WLX{}\UD{\rho}{\nu} && \WLL{}\UD{\rho}{\sigma}
%\end{pmatrix}= \begin{pmatrix}
%0 && \delta\UD{\mu}{\alpha}\\
%R\UDDD{\rho}{\ell}{\ell}{\beta} && 0
%\end{pmatrix}\begin{pmatrix}
%\WXX{}\UD{\beta}{\nu} && \WXL{}\UD{\beta}{\sigma}\\
%\WLX{}\UD{\alpha}{\nu} && \WLL{}\UD{\alpha}{\sigma}
%\end{pmatrix}\, ,
%\label{eq:GDE_W}
%\end{equation}
%with initial conditions
%\begin{equation}
%\left\{ \begin{matrix}
%\WXX {}\UD{\mu}{\nu}(\lambda_{\calO})=\delta\UD{\mu}{\nu} \\
% \WXL {}\UD{\mu}{\sigma}(\lambda_{\calO})=0\\
%\WLX {}\UD{\rho}{\nu}(\lambda_{\calO})=0\\
% \WLL {}\UD{\rho}{\sigma}(\lambda_{\calO})=\delta\UD{\rho}{\sigma} \, .
%\end{matrix}\right. 
%\end{equation}

The relations in Eq.~\eqref{eq:GDE_W} are the evolution equations for the BGO along $\gamma_0$. They show that the BGO can be expressed as non-local functionals of the Riemann tensor along the line of sight.  
Even though the GDE and the matrix equations~\eqref{eq:GDE_W} are linear, the BGO are \emph{nonlinear} functionals of the curvature tensor along $\gamma_0$. 
This can be easily checked as follows: let us take two solutions of Eq.~\eqref{eq:GDE_W}, $\mathcal{W}^{(1)}$ and $\mathcal{W}^{(2)}$, each ones corresponding to two different optical tidal tensor functions,
$R^{(1)}_{\ell \ell}$ and $R^{(2)}_{\ell \ell}$
%\begin{equation}
\begin{align}
\nabla_{\ell} \mathcal{W}^{(1)}=& \begin{pmatrix}
0 && \mathbb{1} \\
R^{(1)}_{\ell \ell} && 0
\end{pmatrix} \mathcal{W}^{(1)} \\
\nabla_{\ell} \mathcal{W}^{(2)}=& \begin{pmatrix}
0 && \mathbb{1} \\
R^{(2)}_{\ell \ell} && 0
\end{pmatrix} \mathcal{W}^{(2)}\, .
\end{align}
%\end{equation}
Then let us see if the linear combination of these solutions $a \mathcal{W}^{(1)}+b \mathcal{W}^{(2)}$, with $a, \, b\, \in \mathbb{R}$, is also a solution of Eq.~\eqref{eq:GDE_W} for the same linear combination of the optical tidal tensor functions, i.e.
\begin{equation}
\nabla_{\ell} (a \mathcal{W}^{(1)}+ b \mathcal{W}^{(2)}) = \begin{pmatrix}
0 && \mathbb{1} \\
a R^{(1)}_{\ell \ell}+ b R^{(2)}_{\ell \ell} && 0
\end{pmatrix} (a \mathcal{W}^{(1)} + b \mathcal{W}^{(2)})\, .
\end{equation}
Rearranging the terms we finally get
\begin{align}
\nonumber a\left[\nabla_{\ell} \mathcal{W}^{(1)}- \begin{pmatrix}
0 && \mathbb{1} \\
a R^{(1)}_{\ell \ell} && 0
\end{pmatrix} \mathcal{W}^{(1)}\right]+ & b \left[\nabla_{\ell} \mathcal{W}^{(2)}- \begin{pmatrix}
0 && \mathbb{1} \\
b R^{(1)}_{\ell \ell} && 0
\end{pmatrix} \mathcal{W}^{(2)}\right] =\\
& b \begin{pmatrix}
0 && 0 \\
a R^{(1)}_{\ell \ell} && 0
\end{pmatrix}  \mathcal{W}^{(2)} +a \begin{pmatrix}
0 && 0 \\
 b R^{(2)}_{\ell \ell} && 0
\end{pmatrix}  \mathcal{W}^{(1)}\, , \label{eq:gde_nonlin}
\end{align}
from which it is easy to see that a linear combination of the solutions of~\eqref{eq:GDE_W} \emph{does not} satisfy the same equations proving the nonlinearity of the BGO with respect to the optical tidal tensor\footnote{In the case $a=b=1$, the left hand side of Eq~\eqref{eq:gde_nonlin} vanishes while the right hand side is in general non zero.}.
The nonlinearity of the BGO with respect to the curvature reflects the fact that, although they describe small deviations from the fiducial geodesics, the BGO captures all nonlinear effects of light bending combined along $\gamma_0$.
%Note also that although all four BGO are effectively functionals of the \emph{same} optical tidal tensor $R\UD{\mu}{\ell \ell \nu}(\lambda)$,  they are in fact \emph{different functionals}  and therefore without any further assumptions regarding the curvature their values for any pair of points $\calO$ and $\calE$ should be treated as completely independent from each other.

\subsection{Algebraic properties of the BGO.} 
From its very definition, the $\mathcal{W}$ matrix satisfies the following properties
\begin{align}
\mathcal{W}(\mathcal{O}, \mathcal{S})&=\mathcal{W}^{-1}(\mathcal{S}, \mathcal{O}) \label{eq:Winversion}\\
\mathcal{W}(\mathcal{S}, \mathcal{O})&=\mathcal{W}(\mathcal{S}, p_{\lambda}) \, \mathcal{W}(p_{\lambda}, \mathcal{O})\, ,
\label{eq:Wcomposition}
\end{align}
with $p_{\lambda}$ an arbitrary point on the fiducial geodesic $\gamma_0$, \cite{Grasso:2018mei}. We also have that $\mathcal{W}$ is a symplectic mapping, \cite{Uzun:2018yes}.

Moreover, the properties of the GDE \eqref{eq:gde_prop1A}-\eqref{eq:gde_prop2} can be immediately translated to corresponding properties of the BGO, which hold irrespective of the spacetime geometry or whether $\gamma_0$ is null or not, \cite{Grasso:2018mei}. 
From the first property, Eqs.~\eqref{eq:gde_prop1A} and~\eqref{eq:gde_prop1B}, we have that for any initial data $\delta x_\calO^\mu$ and $\Delta \ell_\calO^\mu$ the values of $C$ and $D$ need to remain equal in $\calO$ and $\calE$. This means that
\begin{align}
\ell_{\calO\,\mu}\,\Delta \ell_\calO^\mu &= \ell_{\calE\,\mu}\,\Delta \ell_\calE^\mu \label{eq:Dl_conserv}\\
\ell_{\calO\,\mu}\,\delta x_\calO^\mu - \lambda_\calO\,\ell_{\calO\,\mu}\,\Delta \ell_\calO^\mu &= \ell_{\calE\,\mu}\,\delta x_\calE^\mu - \lambda_\calE\,\ell_{\calE\,\mu}\,\Delta \ell_\calE^\mu\, . \label{eq:dx_conserv}
\end{align}
We make use of Eqs.(\ref{eq:positiondeviation1})-(\ref{eq:directiondeviation1}) in order to express $\delta x_\calE^\mu$ and $\Delta \ell_\calE^\mu$ by $\delta x_\calO^\mu$ and $\Delta \ell_\calO^\mu$.
The resulting equations are equivalent to the following 4 relations:
\bea
\ell_{\calE\,\mu}\,{ W_{XX} }\UD{\mu}{\nu} 
 &=& \ell_{\calO\,\nu} \label{eq:Wprop5} \\
\ell_{\calE\,\mu}\,{ W_{XL} }\UD{\mu}{\nu} &=& (\lambda_\calE - \lambda_\calO)\,\ell_{\calO\,\nu} \label{eq:Wprop7} \\
\ell_{\calE\,\mu}\,{ W_{LX} }\UD{\mu}{\nu} &=& 0 \label{eq:Wprop6} \\
\ell_{\calE\,\mu}\,{ W_{LL} }\UD{\mu}{\nu} &=& \ell_{\calO\,\mu} \,.  \label{eq:Wprop8}
\eea 
The ``inverted'' relations are obtained by considering the solution (\ref{eq:gde_prop2}) at $\calO$ and $\calE$: we have $\delta \tilde{x}_\calO^\mu = \delta x_\calO^\mu + (E + \lambda_\calO\, F)\,\ell_\calO^\mu$, 
$\Delta \tilde{\ell}_\calO^\mu =  \Delta \ell_\calO^\mu + F\,\ell_\calO^\mu$ and $\delta \tilde{x}_\calE^\mu =  \delta x_\calE^\mu + (E + \lambda_\calE\,F)\,\ell_\calE^\mu$,  $\Delta \tilde{\ell}_\calE^\mu = \Delta \ell_\calE^\mu + F\,\ell_\calE^\mu$. We substitute these equations to (\ref{eq:positiondeviation1})-(\ref{eq:directiondeviation1}) and 
assuming the resulting relations must hold for all $E$ and $F$ we get
\bea
{ W_{XX} }\UD{\mu}{\nu}\,\ell_\calO^\nu &=& \ell_\calE^\mu \label{eq:Wprop1}\\
{ W_{XL} }\UD{\mu}{\nu}\,\ell_\calO^\nu &=& (\lambda_\calE - \lambda_\calO)\,\ell_\calE^\mu \label{eq:Wprop3} \\
{ W_{LX} }\UD{\mu}{\nu}\,\ell_\calO^\nu &=& 0 \label{eq:Wprop2} \\
{ W_{LL} }\UD{\mu}{\nu}\,\ell_\calO^\nu &=& \ell_\calE^\mu \label{eq:Wprop4}\, .
\eea

Finally, let us note that two of the BGO undergo rescaling under the affine reparametrisations of the fiducial null geodesic $\gamma_0$. Namely, under the transformation Eq.~\eqref{eq:affineparameter} we have the following rescaling for the BGO\footnote{The derivation of these relations uses the fact that under the affine reparametrisation the tangent to the geodesic transforms as Eq.~\eqref{eq:laffine}. Then one can use the GDE, Eq.~\eqref{eq:GDE_W}, together with Eqs.~\eqref{eq:Wprop1}-\eqref{eq:Wprop4} to obtain the relations Eqs.~\eqref{eq:Wresc1}-\eqref{eq:Wresc4}.}
\begin{align}
 W_{XX} &\to \tilde{W}_{XX} = W_{XX} \label{eq:Wresc1}\\
 W_{XL} &\to  \tilde{W}_{XL} = A\cdot W_{XL} \label{eq:Wresc2}\\ 
 W_{LX} &\to  \tilde{W}_{LX} = \frac{1}{A} \cdot W_{LX} \label{eq:Wresc3}\\
 W_{LL} &\to  \tilde{W}_{LL} = W_{LL}\label{eq:Wresc4}\, .
\end{align}

\subsection{BGO for light propagation and the quotient space}
The considerations we have made so far are independent of the character of the geodesics we are considering, so they are valid for BGO describing the properties of all types of geodesics in the vicinity of $\gamma_0$. 
Since our goal is the application of the BGO to describe light propagation, from now on we will consider only families of null geodesics.
%let us specify the BGO to describe a congruence of null geodesics. 
%From Eqs.~\eqref{eq:Wprop5}-\eqref{eq:Wprop8} and~\eqref{eq:Wprop1}-\eqref{eq:Wprop4}, we have that the 
Let us start by noting that the BGO distinguish differently parametrised geodesics sharing the same path. However, from the point of view of geometric optics, an affine reparametrisation of null geodesics is just a gauge freedom, as we have already discussed in Eq.\eqref{eq:gde_prop2}. The change in the affine parameter of the geodesics around $\gamma_0$ transforms solutions of the GDE $\xi^{\mu}$ as $\tilde{\xi}^{\mu} = \xi^{\mu}+\alpha(\lambda) \ell^{\mu}$, with $\xi^{\mu}$ and $\tilde{\xi}^{\mu}$ pointing at the same displaced geodesic. Here, we want to isolate this gauge freedom in order to consider only geodesics having different path. In other words, we want to identify the initial data for which the position and directional deviations only differ by a multiple of  $\ell_\calO^\mu$
\bea
\left(\begin{array}{l}
\delta x_\calO^\mu\\
\Delta \ell_\calO^\mu
\end{array} \right) \sim \left(\begin{array}{l}
\delta x_\calO^\mu +  C_1 \ell_\calO^\mu \\
\Delta \ell_\calO^\mu +  C_2 \ell_\calO^\mu
\end{array} \right)\, , \label{eq:identification1}
\eea
for some non-vanishing constants $C_1$ and $C_2$. Substituting Eq.~\eqref{eq:identification1} in Eqs.~\eqref{eq:positiondeviation1}-\eqref{eq:directiondeviation1}, and making use of Eqs.~\eqref{eq:Wprop1}-\eqref{eq:Wprop4}, we obtain that adding this type of terms in $N_\calO$ leads to a similar change in the final data
\bea
\left(\begin{array}{l}
\delta x_\calE^\mu\\
\Delta \ell_\calE^\mu
\end{array} \right) \sim \left(\begin{array}{l}
\delta x_\calE^\mu +  D_1\,\ell_\calE^\mu \\
\Delta \ell_\calE^\mu +  D_2\,\ell_\calE^\mu,
\end{array} \right)\, , \label{eq:equiv3}
\eea
with constants $D_1$ and $D_2$ related to $C_1$ and $C_2$.
The relation $Y^{\mu} \sim X^{\mu}$ in Eqs.~\eqref{eq:identification1}-\eqref{eq:equiv3} defines equivalence classes $[X]$ in both $T_\calO \calM$ and $T_\calS \calM$. To remove this gauge freedom, one can consider the BGO as maps between vectors $[X]$ in the quotient spaces $\calQ_\calO = T_\calO \calM / \ell_\calO$ and $\calQ_\calE = T_\calE \calM / \ell_\calE$, see Fig.~\ref{fig:quotientspaces}.
The equivalence relation $Y^{\mu} \sim X^{\mu}$ effectively suppresses one dimension (the one along the tangent vector) in the tangent spaces in $N_{\calS}$ and $N_{\calO}$, leaving only three non-trivial directions in $\mathcal{Q}$. 

Let us consider the subspaces $\ell_{\calO}^\perp \subset T_\calO \calM$ and $\ell_{\calS}^\perp \subset T_\calS \calM$ consisting of all vectors orthogonal to $\ell_{\calO}^\mu$ and $\ell_{\calS}^\mu$, respectively. In analogy to $\calQ$, it is possible to define the two-dimensional \emph{perpendicular spaces} $\calP_\calO = \ell_\calO^\perp / \ell_\calO$ and $\calP_\calE = \ell_\calE^\perp / \ell_\calE$ as the subspaces orthogonal to $\ell_\calO^\mu$ and $\ell_\calE^\mu$, respectively \cite{Korzynski:2018, Grasso:2018mei}. %As explained in \cite{Korzynski:2018}, the quotient spaces $\calP_\calO$ and $\calP_\calE$ inherit a positive definite metric from the spacetime metric $g_{\mu \nu}$. This is shown noticing that the scalar product $g_{\mu \nu}\bm{X}^{\mu}\bm{Y}^{\nu}$ for two vectors in $\calP$ is invariant with respect the relation $\sim$, i.e.
%\begin{equation}
%g_{\mu \nu}(X^{\mu}+c_{\rm 1}\ell^{\mu}_{\calO})(Y^{\nu}+c_{\rm 2}\ell^{\nu}_{\calO})= g_{\mu \nu}X^{\mu}Y^{\nu}+c_{\rm 1}g_{\mu \nu}\ell_{\calO}^{\mu}Y^{\nu}+c_{\rm 2}g_{\mu \nu}X^{\mu}\ell_{\calO}^{\nu}+c_{\rm 1}c_{\rm 2}g_{\mu \nu}\ell_{\calO}^{\mu}\ell_{\calO}^{\nu}=g_{\mu \nu}X^{\mu}Y^{\nu}\, .
%\end{equation}
%% $\calP_\calO$ and $\calP_\calE$ inherit the metric from $T_{\calO} \mathcal{M}$ and $T_{\calS} \mathcal{M}$
%In other words, $\calQ$ is the space containing all geodesics parallel to $\gamma_0$, while $\calP$ contains the geodesics paralle to $\gamma_0$ and on the null hypersurface ortogonal to $\ell$.
The physical meaning of the quotient spaces $\calQ$ and $\calP$ will be clear once we introduce a reference frame.
%$\calP_\calO = \ell_\calO^\perp / \ell_\calO$ and $\calP_\calE = \ell_\calE^\perp / \ell_\calE$ \MG{This invariance leads to the following idea: instead of considering the optical operators as acting from one tangent space to another one we may consider them as mappings between the appropriate quotient spaces $\calQ_\calO = T_\calO \calM / \ell_\calO$ and $\calQ_\calE = T_\calE \calM / \ell_\calE$, see Figure \ref{fig:quotientspaces}. Namely, let $\calQ_\calE$ be the quotient of $T_\calE\calM$ by the equivalence relation $X^\mu \sim Y^\mu$ iff $X^\mu = Y^\mu + c\,\ell^\mu_\calE$ for any real number $c$.  $\calQ_\calO$ can be defined in an analogous way, with $\ell_\calE$ replaced by $\ell_\calO$ and $T_\calE\calM$ replaced by $ T_\calO \calM$. Within these 3-dimensional spaces we also consider the 2-dimensional subspaces orthogonal to $\ell_\calO^\mu$ and $\ell_\calE^\mu$ respectively, i.e. $\calP_\calO = \ell_\calO^\perp / \ell_\calO$ and $\calP_\calE = \ell_\calE^\perp / \ell_\calE$, see again Figure \ref{fig:quotientspaces}: they  will be referred to as the \emph{perpendicular spaces}. Unlike $\calQ_\calO$ and $\calQ_\calE$, the perpendicular spaces $\calP_\calO$ and $\calP_\calE$ inherit a positive definite metric $q$ from the Lorentzian spacetime metric $g$ \cite{Korzynski:2018}: let $\bm X$ and $\bm Y$ be two vectors in $\calP_\calE$ and let $X$ and $Y$ be any two corresponding vectors in $T_\calE \calM$, i.e. $\bm X = [X]$ and $\bm Y = [Y]$, where $[\cdot]$ denotes the linear operation of taking the equivalence class of a vector in $T_\calE \calM$  with respect to the relation $\sim$  defined above. The reader may check that the formula $q(\bm X,\bm Y) = g(X,Y)$ defines the same value of the scalar product of $\bm X$ and $\bm Y$ irrespective of the choice of $X$ and $Y$.
%The angles and distances calculated using $q$ correspond to the angles and distances measured by any observer on points projected down to the plane perpendicular to the direction of observation (the Sachs's screen space) along the null direction of light propagation \cite{Korzynski:2018}. This is a reformulation of the well-known Sachs shadow theorem \cite{sachs, perlick-lrr} in terms of the quotient spaces. The physical meaning of the quotient spaces $\calQ$ and $\calP$ will be clear once we introduce a reference frame.}
%%%%%%%%%%%This fact gives the geometry of the perpendicular spaces an explicitly observer-invariant meaning and we explore it later in order to separate the dependence of observables on the observer's and emitter's frame and on the spacetime geometry.}
\bfi
\centering
\includegraphics[width=\linewidth]{pict/quotientspaces.pdf}
\caption{Geometry of the quotient spaces $\calQ_\calO$ and $\calP_\calO$. Elements $[X] \in \calQ_\calO$ correspond to the vectors $X^{\mu}$ and $Y^{\mu}$ in $T_\calO \calM$ identified by the relation $Y^{\mu} \sim X^{\mu}$, i.e. such that $Y^{\mu}=X^{\mu}+c\ell^{\mu}_{\calO}$. Geometrically is the space containing all geodesics in $N_\calO$ parallel to $\gamma_0$. Elements $\bm{X} \in \calP_\calO$ corresponds to the vectors $X^{\mu}$ and $Y^{\mu}$ in $T_\calO \calM$, which are perpendicular to $\ell_\calO$ (i.e. $X^{\mu}\ell_\calO{}_{\mu}=0$ and $Y^{\mu}\ell_\calO{}_{\mu}=0$) and identified by $Y^{\mu} \sim X^{\mu}$. Geometrically, $\calP_\calO$ is the space containing the geodesics parallel to $\gamma_0$ and lying on the null hypersurface orthogonal to $\ell_\calO$.}
\label{fig:quotientspaces}
\efi

%The BGO are fundamental objects which are directly related to the GDE, so they can describe only effects on light propagation which are linear in the deviations $(\delta x^{\mu}, \Delta \ell^{\mu})$. This means that, for a family of null geodesics the condition that a displaced geodesic remain null is
%\begin{equation}
%g_{\mu \nu}(\ell^{\mu}+ \Delta \ell^{\mu})(\ell^{\nu}+ \Delta \ell^{\nu})=\ell_{\mu}\ell^{\mu} + 2 \ell_{\mu}\Delta \ell^{\mu}+\Delta \ell_{\mu}\Delta \ell^{\mu}=0\, .
%\end{equation}


\subsection{The semi-null frame and the observer's sky}

Up to this point, the GDE formulation in terms of bilocal operators we just presented is completely covariant, namely it was derived without invoking explicitly a coordinate system or a frame of reference. However, in order to relate the BGO to actual observable quantities, we need to introduce a reference frame. 
In the context of relativistic geometric optics, it is customary to introduce a frame that relates to the results of observations at $\calO$. The standard approach is to use the Sachs orthonormal frame, consisting of the observer four-velocity $u_\calO^\mu$, the direction vector $r^\mu$, being the direction from which the observer sees the light coming, and two perpendicular, spatial vectors $\phi_{\bm{A}}^\mu$ spanning the so called Sachs screen space \cite{perlick-lrr, sachs, ehlers-jordan-sachs}. However, here we use a different frame called the \emph{semi-null frame} (\setwd{SNF}{acr:SNF}) \cite{Grasso:2018mei}, in which the two quotient spaces $\calQ$ and $\calP$ have a simple physical interpretation. The SNF consists of $u_\calO^\mu$, the same two perpendicular, spatial vectors $\phi_{\bm{A}}^\mu$ and the null vector $\ell^\mu$ instead of $r^\mu$. It is not orthonormal and we can check that the products of the basis vectors read
\begin{align}
\nonumber \ell^\mu\,\ell_\mu &= 0 \\
\nonumber \phi_{\bm{A}}^\mu\,\ell_\mu &= 0 \\
\nonumber u_\calO^\mu\,\phi_{\bm{A}\,\mu} &= 0 \\
 \phi_{\bm{A}}^\mu\,\phi_{\bm{B}\,\mu} &= \delta_{\bm{A}\bm{B}} \label{eq:SNF_relations}\\
\nonumber u_\calO^\mu\,u_{\calO\,\mu} &= -1 \\
\nonumber \ell^\mu\,u_{\calO\,\mu} &= Q\, , 
\end{align}
where we have introduced the constant\footnote{Note that the sign of the constant $Q$ depends on the temporal orientation of $\ell^{\mu}$. The standard convention in cosmology is to consider the tangent vector past-oriented, from the observer $\calO$ to the source $\calS$, which cause to have $Q>0$. However, we will later consider the case when $\ell^{\mu}$ is future-oriented and in that case $Q<0$.} $Q$ for the product of $\ell^\mu$ and $u_\calO^\mu$.
We denote the frame indices by boldface letters: capital Latin indices $\bm{A}$, $\bm{B}, \ldots$,  running over the spatial components $\bm{1}$ and $\bm{2}$,
lower case Latin indices $\bm{i}$, $\bm{j}, \ldots$, running over $\bm{1}$, $\bm{2}$ and $\bm{3}$, and the boldface Greek indices $\bm{\mu}$, $\bm{\nu}, \ldots$,  running over all 4 dimensions from $\bm{0}$ to $\bm{3}$.
The associated coframe $\psi^{\bm{\alpha}}_{\mu}$ is composed of the tetrad of vectors $\psi^{\bm{\alpha}}_{\mu}=(\frac{\ell_{\mu}}{Q}, \phi^{\bm{A}}_{\mu}, \frac{u_{\mu}}{Q}+\frac{\ell_{\mu}}{Q^2})$. In the SNF the displacement vector $\delta x^{\mu}$ has components $\delta x^{\bm{\mu}}=\delta x^{\nu}\hat{\psi}^{\bm{\mu}}_{\nu}=(\frac{\delta x^{\nu}\ell_{\nu}}{Q}, \delta x^{\nu}\hat{\phi}^{\bm{A}}_{\nu}, \frac{\delta x^{\nu}\hat{u}_{\calO\, \nu}}{Q}+\frac{\delta x^{\nu}\ell_{\nu}}{Q^2})$, where hatted vectors are parallel transported along the fiducial geodesic $\gamma_0$.

In the SNF the presence of the quotient spaces have a natural explanation: the first three components of the GDE (\ref{eq:GDE_delta}) in the SNF\footnote{The optical tidal matrix in the SNF components is defined as $R\UD{\bm{\mu}}{\ell \ell \bm{\nu}}=\hat{\psi}^{\bm{\mu}}_{\rho}R\UD{\rho}{\ell \ell \sigma}\hat{\phi}^{\sigma}_{\bm{\nu}}$ and from the symmetries of the Riemann tensor follows that $R\UD{\bm{0}}{\ell \ell \bm{\nu}}=R\UD{\bm{\mu}}{\ell \ell \bm{3}}=0$.} 
\begin{align}
\frac{d^2 \delta x^{\bm{0}}}{ d \lambda^2} = & 0 \label{eq:gdeSNF0}\\
\frac{d^2 \delta x^{\bm{A}}}{ d \lambda^2} = & R\UD{\bm{A}}{\ell \ell \bm{0}}\delta x^{\bm{0}}+R\UD{\bm{A}}{\ell \ell \bm{B}}\delta x^{\bm{B}} \label{eq:gdeSNFA}\, ,
\end{align}
decouple from the fourth one, $\frac{d^2 \delta x^{\bm{3}}}{ d \lambda^2} = R\UD{\bm{3}}{\ell \ell \bm{0}}\delta x^{\bm{0}}+R\UD{\bm{3}}{\ell \ell \bm{B}}\delta x^{\bm{B}}$. % \MG{forming a subspace of solutions in $\calQ$}.
Moreover, from $\frac{d^2 \delta x^{\bm{0}}}{ d \lambda^2} = 0$ we have that $\delta x^{\bm{0}}=\frac{\delta x^{\mu}\ell_{\mu}}{Q}= \const $, according to Eq.~\eqref{eq:gde_prop1A} for null geodesics in the flat lightcones approximation, in which we have  $\frac{d \delta x^{\bm{0}}}{ d \lambda} \propto \ell_{\mu}\Delta \ell^{\mu}=0$. 

Geometrically, the condition $\ell_{\calO\,\mu}\,\delta x_{\calO}^\mu = \ell_{\calE\,\mu}\,\delta x_{\calE}^\mu $ defines foliations of $N_\calO$ and $N_\calE$ by families of null hypersurfaces, see Fig.~\ref{fig:foliations}, implying that the observers located on a leaf in $N_\calO$ can only perceive the events lying on the corresponding leaf in $N_\calE$, as explained in \cite{Grasso:2018mei}.
%It is also clear that while the null tangent vectors $l_\calO^\mu$ and $l_\calE^\mu$ are defined only up to a common rescaling according to (\ref{eq:laffine}), this ambiguity does not affect the two foliations or the correspondence between their leaves.
\bfi
\centering
\includegraphics[width=.9\textwidth]{pict/FLA_foliation.pdf}
\caption{The past light cone (blue) in $N_{\calO}$ degenerates to the flat null hypersurface (blue plane) in $N_{\calS}$. Similarly, the future light cone (orange) in $N_{\calS}$ degenerates to the flat null hypersurface (orange plane) in $N_{\calO}$. $\calO$ can observe only those events on the corresponding hypersurface in $N_{\calS}$.}%$N_{\calO}$ and $N_{\calS}$ are foliated by families of null hypersurfaces corresponding to the degenerate light cones centred at the opposite ends of $\gamma_0$. 
\label{fig:foliations}
\efi
The interpretation of the two null foliations is straightforward: at two ends of $\gamma_0$ the foliations $\delta x^{\mu}\ell_{\mu}= \const$  are the degenerate families of light cones centred at the opposite ends of $\gamma_0$. In other words, the past lightcone of the point $p$ in $N_\calO$ degenerate to a flat hypersurface in $N_\calE$ due to the large distance between the two regions and their small size. Similarly, the future light cone of any point on that null hypersurface will degenerate to the null hypersurface containing $p$ in $N_\calO$. These clarifies the name used for the condition $\ell_{\mu}\Delta \ell^{\mu}=0$. %Physically, the flatness of light cones means that the R\o mer delay is in a good approximation independent of the perpendicular displacements of the observer and the source because of the large distance between them.
In the special case $\delta x^{\bm{0}}=0$, Eqs.~\eqref{eq:gdeSNFA} decouple from Eq.~\eqref{eq:gdeSNF0}, and their solutions $(\delta x^{\bm{A}}, \Delta \ell^{\bm{A}})$ form a subspace of solutions in $\calP$.

Vectors expressed in the SNF have a very simple representation in the quotient spaces $\calQ_\calO = T_\calO \calM / \ell_\calO$ and $\calQ_\calS = T_\calS \calM / \ell_\calS$. In fact, for vectors $[X] \in \calQ$ we can ``forget'' about the fourth component $X^{\bm 3}$, namely $(X^{\bm 0}, X^{\bm 1}, X^{\bm 2}, X^{\bm 3}) \to (X^{\bm 0}, X^{\bm 1}, X^{\bm 2})$. Similarly, vectors in any perpendicular subspace $\calP$ along $\gamma_0$ have additionally vanishing first component, i.e. $(0,X^{\bm 1}, X^{\bm 2})$. 

The introduction of a frame is mandatory to perform measurements like the positions of celestial objects: the observer $\calO$ sees the source $\calS$ in the direction corresponding to the line of sight $\gamma_0$ as
\begin{equation}
r^\mu_0 = \frac{1}{\ell_\sigma\,u_\calO^\sigma}\,\ell^\mu + u_\calO^\mu\, . \label{eq:dir}
\end{equation}
The direction $r^\mu_0$ serves as a reference point on the observer screen, indeed in the SNF gives $r^{\bm{A}}_0=(0,0)$. Similarly, for the geodesic $\gamma$ emitted by a another source $\mathcal{S}'$ in $N_{\calS}$, the observer sees the light from the direction 
\begin{equation}
r^\mu=\frac{1}{k_\sigma\,u_\calO^\sigma}\,k^\mu + u_\calO^\mu\, ,
\end{equation}
where $k^{\mu}$ is the tangent vector of the geodesic. %, see Fig.~\eqref{fig:}. 
The position of $\mathcal{S}'$ is identified by the observer $\calO$ measuring the angle between $r^\mu_0$ and $r^\mu$, that for a source which lies close\footnote{The approximation is valid for $\delta\theta^{\bm A} \ll 1\, {\rm rad}$. The relation for larger angles requires the use of the standard trigonometric formulae.} to $r_0^\mu$ is simply
\bea
 \delta \theta^{\bm A} \approx r^{\bm A}\, . \label{eq:positionapprox}
\eea
Therefore, all objects in the observer's view can be conceived as being projected on an ideal sphere, representing the observer's sky, where the apparent positions of objects in the sky are determined from the transversal components of $r^\mu$ in the semi-null frame of the observer, denoted as $r^{\bm{A}}$.
%%%%%%%%%%%%%%%
The expression Eq.~\eqref{eq:dir} defines an observer-dependent mapping from the set of null tangent vectors ${\cal N}_\calO = \left\{ \ell \in T_\calO \calM | \ell^\mu\,\ell_\mu = 0, \ell^0 < 0 \right\}$ to the observer's sky of directions $\textrm{Dir}(u_\calO) = \left\{ r\in T_\calO \calM | r^\mu\,r_\mu = 1, u_\calO^\mu\,r_\mu = 0 \right\}$, i.e. the set of normalised, purely spatial vectors for the observer \cite{perlick, low, Korzynski:2018, Grasso:2018mei}.
%Because the space ${\cal N}_\calO$ lacks of a well-defined metric, it is not possible to calculate the angular distance between points on the sky. Instead, one must pass to the observer-dependent space $\textrm{Dir}(u_\calO)$, which however introduces the dependence of the observer's four-velocity. This is commonly referred to as the \emph{light aberration effect}, or \emph{stellar aberration} and may be explained by a relative tilt of the spheres of directions (and the whole simultaneity planes) for observers with different four-velocities, see \cite{liebscher1998} for a more detailed discussion involving the historical background.

The introduction of the observer's sky provides an observer-dependent method to identify the apparent position of sources. Now we will see how observations made by observers with different four-velocities and at different points in $N_\calO$ may be compared. This is not a simple task in a generic spacetime because the position on the sky is a vector in the observer-dependent space of directions. Let us split the task in two simpler problems:
\begin{enumerate}
  \item how do we compare position vectors at different points, 
  \item how do we compare directions on the sky measured by
observers boosted with respect to each other.% since the notion of a spatial vector is different for each of them. 
\end{enumerate}
The first issue is overcome by the assumed flatness of $N_\calO$, which allows to use the parallel propagation of the frame for $\calO$ to identify $T_\calO \calM$ as the tangent space at all points. %This is possible independently of the paths connecting points with $\calO$ we may choose for that purpose. 
Thus, we can introduce a parallel propagated SNF $\phi_{\bm{\alpha}}^\mu=(u_\calO^\mu, \phi_{\bm{A}}^\mu, \ell_\calO^\mu)$ from $\calO$ throughout the whole region $N_\calO$ to compare vector or tensor defined at different points.  
From now on all equations are expressed in this type of parallel transported SNF at $N_\calO$ (and a similar one at $N_\calE$). The tangent vector of $\gamma_0$ is $\ell^{\mu}_{\calO}$, while for the other null geodesics the tangent vector is simply $k_\calO^\mu = \ell^\mu_\calO + \Delta \ell_\calO^\mu$. Then, the direction vector $r^{\bm{\mu}}$ in the SNF of the observer $u_\calO^\mu$ become 
\begin{align}
 r^{\bm{A}} =r^{\mu}\phi_{\mu}^{\bm{A}}&=\left(\dfrac{\ell^\mu_\calO + \Delta \ell_\calO^\mu}{u_{\calO\,\sigma}\,\left(\ell_\calO^\sigma + \Delta \ell_\calO^\sigma\right)}+u^{\mu}_{\calO}\right)\phi_{\mu}^{\bm{A}}= \frac{\Delta \ell^{\bm{A}}_\calO}{u_{\calO\,\sigma}\,\left(\ell_\calO^\sigma + \Delta \ell_\calO^\sigma\right)} \nonumber \\
  & = \frac{\Delta \ell^{\bm{A}}_\calO}{u_{\calO\,\sigma}\,\ell_\calO^\sigma\left(1+\frac{u_{\calO\,\sigma}\,\Delta \ell_\calO^\sigma}{u_{\calO\,\sigma}\,\ell_\calO^\sigma}\right)}= \frac{\Delta \ell^{\bm{A}}_\calO}{u_{\calO\,\sigma}\,\ell_\calO^\sigma}\,\left(1 - \frac{\Delta \ell_\calO^{\bm{0}}}{u_{\calO\,\sigma}\,\ell_\calO^\sigma} + \Delta \ell_\calO^{\bm{3}}\right)^{-1}\, , \label{eq:position1}
\end{align}
where in the last equality we have used Eq.~\eqref{eq:SNF_relations} to express
\begin{equation}
u_{\calO\,\sigma}\,\Delta \ell_\calO^\sigma= u_{\calO\,\sigma}\,(\phi^{\sigma}_{\bm{\nu}}\Delta \ell_\calO^{\bm{\nu}})=- \Delta \ell_\calO^{\bm{0}}+u_{\calO\,\sigma}\,\ell_{\calO}^{\sigma}\Delta \ell_\calO^{\bm{3}}\, .
\end{equation}
Let us remark that we are in the regime of the first-order GDE, so we only need to consider linear terms in the displacement (and its derivatives). This is indeed the condition of the FLA, i.e. $\ell_{\calO\, \mu}\Delta\ell^{\mu}_{\calO}=0$, that we have considered in Eq.~\eqref{eq:nulldisplaced}. If we express this condition in the SNF we have $0=\ell_{\calO\, \mu}(\phi^{\mu}_{\bm{\nu}}\Delta\ell^{\bm{\nu}}_{\calO})=\ell_{\calO\, \mu}(u^{\mu}_{\calO}\Delta\ell^{\bm{0}}_{\calO})$, or in other words the FLA gives that $\Delta\ell^{\bm{0}}_{\calO}$ is an higher-order correction.
Now, we can expand Eq.~\eqref{eq:position1} in terms of the components of the direction deviation vector $\Delta \ell$, to obtain
\begin{equation}
r^{\bm{A}} = \frac{\Delta \ell^{\bm{A}}_\calO}{u_{\calO\,\sigma}\,\ell_\calO^\sigma}\,\left(1 - \Delta \ell_\calO^{\bm{3}}\right)+ h.o.t.=\frac{\Delta \ell^{\bm{A}}_\calO}{u_{\calO\,\sigma}\,\ell_\calO^\sigma} + h.o.t.\, \label{eq:direction_rA}
\end{equation}
From Eq.~\eqref{eq:positiondeviation1}, the direction deviation vector $\Delta \ell^{\bm{A}}_\calO$ is related to the BGO as
\begin{equation}
\Delta \ell^{\bm{A}}_\calO=(\WXL^{-1}){}\UD{\bm{A}}{\bm{\nu}}\,\left[ \delta x^{\bm{\nu}}_{\calS}-\WXX{}\UD{\bm{\nu}}{\bm{\sigma}}\delta x^{\bm{\sigma}}_{\calO} \right]\, .\label{eq:Dl_inBGO_with_Ps}
\end{equation}
Should be emphasised that the vector $\left[ \delta x^{\mu}_{\calS}-\WXX{}\UD{\mu}{\sigma}\delta x^{\sigma}_{\calO} \right]$ is an element of the perpendicular space $\ell^{\perp}_{\calS}$, in fact from Eq.~\eqref{eq:dx_conserv} for null geodesics in the FLA and Eq.~\eqref{eq:Wprop5} follow that \cite{Grasso:2018mei}
\begin{equation}
\ell_{\calS}{}_{\mu} \delta x^{\mu}_{\calS}-\ell_{\calS}{}_{\mu}\WXX{}\UD{\mu}{\sigma}\delta x^{\sigma}_{\calO}=\ell_{\calS}{}_{\mu} \delta x^{\mu}_{\calS}-\ell_{\calO}{}_{\sigma}\delta x^{\sigma}_{\calO}=0\, .
\end{equation}
Therefore, although $\delta x^{\mu}_{\calS}$ and $\WXX{}\UD{\mu}{\sigma}\delta x^{\sigma}_{\calO}$ are not necessarily orthogonal to $\ell^{\mu}_{\calS}$, the combination $\left[ \delta x_{\calS}-\WXX(\delta x_{\calO}) \right]^{\mu}$ certainly is\footnote{Since $\Delta \ell_\calO \in \calP_{\calO}$, this implies also that $\WXL{}\UD{\bm{A}}{\bm{B}} : \calP_{\calO} \to \calP_{\calS}$ is the operator mapping direction deviations in $\calP_{\calO}$ to images in $\calP_{\calS}$.}, and it can be pulled back to the quotient space $\calP_{\calS}$ to finally obtain
\begin{equation}
r^{\bm{A}} = \frac{(\WXL^{-1}){}\UD{\bm{A}}{\bm{B}}}{u_{\calO\,\sigma}\,\ell_\calO^\sigma}\,\left[ \delta x_{\calS}-\WXX(\delta x_{\calO}) \right]^{\bm{B}}\, .\label{eq:rA_wBGO}
\end{equation}
%where $\left[ \delta x_{\calS}-\WXX(\delta x_{\calO}) \right]^{\bm{B}}$ denotes the $\bm{B}$ components of $\left( \delta x^{\bm{\nu}}_{\calS}-\WXX{}\UD{\bm{\nu}}{\bm{\rho}} \delta x_{\calO}^{\bm{\rho}} \right)$. 
%r^{\bm{A}} = \frac{(\WXL^{-1}){}\UD{\bm{A}}{\bm{\nu}}}{u_{\calO\,\sigma}\,\ell_\calO^\sigma}\,\left( \delta x^{\bm{\nu}}_{\calS}-\WXX{}\UD{\bm{\nu}}{\bm{\rho}} \delta x_{\calO}^{\bm{\rho}} \right)
%r^{\bm{A}} = \frac{(\WXL^{-1}){}\UD{\bm{A}}{\bm{B}}}{u_{\calO\,\sigma}\,\ell_\calO^\sigma}\,\left( \delta x_{\calS}^{\bm{B}}-\WXX{}\UD{\bm{B}}{\bm{C}} \delta x_{\calO}^{\bm{C}} \right)


The second question is how we compare directions registered by another observer $\calO'$ with a different four-velocity $U^\mu_{\calO}$. As for $\calO$, we introduce a SNF $f_{\bm{\alpha}}^\mu=(U_{\calO}^\mu, f_{\bm{A}}^\mu, \ell_\calO^\mu)$ adapted to the observer $\calO'$ that is used to define directions on $U_{\calO}^\mu$'s sky. The relation between directions measured by the two observers is contained in the expression of the spatial vectors $f_{\bm{A}}^\mu$ in components of $\phi_{\bm{\alpha}}^\mu$
\begin{equation}
f_{\bm{A}}^\mu=\omega\UD{\bm{B}}{\bm{A}}\phi_{\bm{B}}^\mu+C_{\bm{A}}\,\ell_\calO^\mu\, ,\label{eq:boostRel}
\end{equation}
where the coefficients $\omega\UD{\bm{B}}{\bm{A}}$ and $C_{\bm{A}}$ are found using the relations for the SNF. In particular, from $f_{\bm{A}}^\mu (f_{\bm{B}})_\mu=\delta_{\bm{A}\,\bm{B}}$ follows that $\omega\UD{\bm{B}}{\bm{A}}\omega\UD{\bm{D}}{\bm{C}}\delta_{\bm{B}\,\bm{D}}=\delta_{\bm{A}\,\bm{C}}$. In the case that $f_{\bm{A}}^\mu$ and $\phi_{\bm{A}}^\mu$ have the same orientation, i.e. ${\rm det}\left(\omega\UD{\bm{B}}{\bm{A}}\right)>0$, then the matrix coefficient $\bm{\omega}\in SO(2)$. In conclusion, Eq.~\eqref{eq:boostRel} tell us that the two screen vectors are related by a rotation around the direction vector $r^{\mu}$ and possibly by a component along $\ell^{\mu}_{\calO}$.
We can always choose the spatial vectors $f_{\bm{A}}^\mu$ aligned along $\phi_{\bm{A}}^\mu$, such that $f_{\bm{A}}^\mu = \phi_{\bm{A}}^\mu + C_{\bm{A}}\,\ell_\calO^\mu$, with appropriate $C_{\bm{1}}$ and $C_{\bm{2}}$ \cite{Korzynski:2018}. The two pairs of vectors $f_{\bm{A}}^\mu $ and $ \phi_{\bm{A}}^\mu$ belong to the same equivalence classes in $\calP_{\calO}$. This way both $u_\calO^\mu$ and $U^\mu$ may use the fiducial null vector $\ell_\calO^\mu$ to provide the reference direction on their skies and the screen vectors $\phi_{\bm{A}}^\mu$ and $f_{\bm{A}}^\mu$ as the two perpendicular vectors on the celestial sphere. Now, the two spatial components of the direction vector $r^{\bm{A}}$ can be used to compare the registered directions on the sky between the two observers. 
In general, to any SNF $(u^\mu,\phi_{\bm{ A}}^\mu, \ell_\calO^\mu)$ we can take $\bm{ u} = [u]$, $\bm{\phi}_{\bm{ A}} = [\phi_{\bm{ A}}]$ to obtain a frame $(\bm{ u},\bm{\phi}_{\bm{ A}})$ in $\calQ_\calO$ and a frame $(\bm{\phi}_{\bm{ A}})$ in $\calP_\calO$. By parallel propagating the SNF and repeating this procedure we obtain similar parallel propagated frames $(\hat u^\mu,\hat \phi_{\bm{ A}}^\mu,\hat \ell_\calO^\mu)$, $(\hat {\bm{ u}},\hat {\bm{\phi}}_{\bm{ A}})$ and $(\hat{\bm{ \phi}}_{\bm{ A}})$, in $T_p \calM$, $\calQ_p$ and $\calP_p$, respectively.

%%%%%%%%%%%%%%%%%%%%%%%%%%%%%%%%%%%%%%%%%%%%%%%%%%%%%%%%%%%%%%%%%%%%%%%%%%%%%%%%%%%%%%%%%%%%%%%%%%%%%%%%%%%%%%%%%%%%%%%%%%%%%%%%%%%%%%%%%%%%%%%%%%%%%%%%%%%%%%%%%%

\section{Observables with the BGO: momentary observables and drift effects}
\label{sec:observables}
%%%%%%%FIX QUI
In this section we apply the machinery of the BGO to compute multiple observables within the same framework. We show the derivation of the angular diameter distance, the parallax distance, the position drift, and the redshift drift as functionals of the BGO, following the results in \cite{perlick-lrr, Korzynski:2018, Grasso:2018mei, Korzynski:2019oal}. We also recall the definition of the redshift and  the luminosity distance, as these are fundamental quantities in cosmology.

\subsection{The redshift}
The redshift is a dimensionless quantity which measures the relative difference in the light wavelength between the emission and the observation points. In our system, the photons travelling along $\gamma_0$ from the source $\calS$ to the observer $\calO$ will experience the redshift $z$
\begin{equation}
1+z = \dfrac{\left(\ell_{\sigma } u^{\sigma}\right)|_{\cal S}}{\left(\ell_{\sigma } u^{\sigma}\right)|_{\cal O}}\, ,
\label{eq:redshift_def}
\end{equation}
where $\ell^{\sigma}$ is the tangent to $\gamma_0$, and $u^{\sigma}_{\mathcal{O}}$ and $u^{\sigma}_{\mathcal{S}}$ are the observer and source four-velocities.
In practical applications we distinguish the following sources of the redshift: 
\begin{itemize}
\item when $\calS$ and $\calO$ are moving with respect to each other, we have \emph{the relativistic Doppler effect},
\item when $\calS$ is immersed in a different gravitational potential than $\calO$, i.e. when one end of $\gamma_0$ is in a region of the spacetime more curved then the other end, we have \emph{the gravitational redshift},
\item when the spacetime between $\calS$ and $\calO$ is expanding, we have  \emph{the cosmological redshift}.
\end{itemize}
In FLRW spacetimes, the cosmological redshift is directly related to the scale factor $1+z=a_0/a$ and can be used as an independent variable to express other quantities, such as distance measurements.


\subsection{The angular diameter distance}
In astronomy there are several method to measure the distance of faraway objects, each based on different techniques. The \emph{angular diameter distance} (or area distance\footnote{Actually, the angular diameter distance and the area distance have different definitions, as pointed out in \cite{perlick-lrr}. However, here we will consider the two as synonyms to the definition in Eq.~\eqref{eq:D_ang_def}.}) is a measure of distance based on the idea that the farther away an object is, the smaller it appears to be, and it is defined as
\begin{equation}
D_{ang} = \sqrt{\left|\dfrac{A_{\calS}}{ \Omega_{\calO}}\right|}\, ,
\label{eq:D_ang_def}
\end{equation}
with $A_{\calS}$ being the area of the cross-section $C$ of the emitting body measured in its own frame $\phi_{\calS}{}^{\mu}_{\bm{\nu}}$, and $ \Omega_{\calO}$ is the solid angle\footnote{Note that $A_{\calS}$ and $\Omega_{\calO}$ are signed quantities depended on the orientation. To remove this dependence we have introduced an absolute value in Eq.~\eqref{eq:D_ang_def}.} occupied by the image $I$ in the observer's celestial sphere, see Fig.~\ref{fig:magnificationmatrix}. %For simplicity, we require that the parallel transported spatial vectors $\hat{\phi}_{\calO}{}^{\mu}_{\bm{A}}$ are aligned to $\phi_{\calS}{}^{\mu}_{\bm{A}}$. 
\bfi
\centering
\includegraphics[width=0.9\textwidth]{pict/magnification.pdf}
\caption{On the null hypersurface $\ell_{\calS}{}_{\mu}\delta x^{\mu}_{\calS}=0$ the cross section $C$ of the source $\calS$ has area $A_{\calS}$. On the corresponding null hypersurface $\ell_{\calO}{}_{\mu} \delta x^{\mu}_{\calO}=0$, the observer $\calO$ measures the solid angle $\Omega_{\calO}$ occupied by the image $I$ of the source.}
\label{fig:magnificationmatrix}
\efi
Now, the area and the solid angle can be expressed in terms of the deviation vector $\delta x^{\mu}$ as
\begin{align}
A_{\calS}= & \int_{C} \delta x^{\bm{1}}_{\calS} \wedge \delta x^{\bm{2}}_{\calS} \nonumber \\
\Omega_{\calO} =& \int_{I} \delta \theta^{\bm{1}}_{\calO} \wedge \delta \theta^{\bm{2}}_{\calO}=\frac{1}{(u_{\calO\,\sigma}\,\ell_\calO^\sigma)^2}\left[{\rm det}\left(\WXL {}\UD{\bm A}{\bm B}\right)\right]^{-1}\,\int_{C} \delta x^{\bm{1}}_{\calS} \wedge \delta x^{\bm{2}}_{\calS}\, , \label{eq:area-angle_Dang}
\end{align}
where we have used Eq.~\eqref{eq:rA_wBGO} with $\delta x^{\mu}_{\calO}=0$ to express small angles $\delta \theta^{\bm{A}}\sim r^{\bm{A}}$.
Inserting Eqs.~\eqref{eq:area-angle_Dang} in the definition Eq.~\eqref{eq:D_ang_def} we obtain the angular diameter distance in terms of BGO as
\begin{equation}
D_{ang} = \left(\ell_{\sigma} u^{\sigma}\right)|_{\cal O} \left| \det \left(\WXL {}\UD{\bm A}{\bm B}\right) \right|^{\frac{1}{2}}\, .
\label{eq:D_ang_BGO}
\end{equation}
%\MG{A simple physical interpretation of the angular diameter distance can be explained as in Figure \ref{fig:triangles}.
%\bfi
%\centering
% \includegraphics[width=0.7\textwidth]{figures/Fig6-trianglesEmodif.pdf}
%\caption{A point-like luminous object at $\calE$ and two other point-like luminous objects nearby, forming a triangle on the cross-section plane in the direction of the propagation of light. The angular diameter distance between $\calO$ and $\calE$ is obtained as the square root of the ratio between the area of the triangle on the cross-section plane and the stereographic angle occupied by the triangle with vertices at the apparent positions of the 3 objects at the observer's sky.}
%\label{fig:triangles}
%\efi
%Assume we observe from the point $\calO$ the light emitted by three pointlike objects at the events $\calE$, $\calE_1$ and $\calE_2$, all lying on the same null hypersurface $l_{\calE\,\mu}\,\delta x_{\calE}^\mu = 0$.  Their projections from a triangle of area $A_\calE$ on the cross-sectional plane perpendicular to $l_\calE$ (this value is independent of the choice of the observer in $\calE$ due to the Sachs shadow theorem). On the other hand, their images form a triangle on the observer's sky, covering the stereographic angle of $\Omega_\calE$.  In a spacetime with strong lensing the image may be subject to a strong deformation, including a rescaling, shear, and rotation, but the angular diameter distance is defined simply by the ratio of $A_\calE$ and $\Omega_\calE$ via (\ref{eq:Dangdef1}).}

By direct comparison with the standard definition of the angular diameter distance, see e.g. \cite{perlick-lrr}, we have that the two-by-two submatrix of $\WXL$ is the well-known Jacobi operator $\mathcal{D}\UD{\bm A}{\bm B}$, namely the map between physical separations $\delta x^{\bm{A}}_{\calS}$ at the source position and direction deviations $ \Delta \ell^{\bm{A}}_{\calO}$ at the observer position.
%\begin{equation}
%\delta \theta^{\bm A}= {\left( \ell_{\sigma} u^{\sigma}\right)|_{\cal O}}^{-1} \left(\WXL {}\UD{\bm A}{\bm B}\right)^{-1} \delta x_{\mathcal{S}}^{\bm B}\, .
%\end{equation}

\subsection{The luminosity distance}
Similar to the angular diameter distance, the \emph{luminosity distance} is based on the idea that the farther away an object is located, the fainter appears its light. Indeed, if we consider an isotropic light emission, the energy flux of the light $F$ decreases with distance from the source $D$ according to the flux-luminosity relation $F= L / (4 \pi D^2)$. The luminosity distance is then defined as the ratio between the luminosity $L$ of the source and the flux measured at the observer
\begin{equation}
D_{\rm lum}=\sqrt{\dfrac{L}{4 \pi F}}\, . \label{eq:D_lum_def}
\end{equation}
As we did for $D_{\rm ang}$, also in this case we can express eq.~\eqref{eq:D_lum_def} in terms of geometrical quantities by inverting the role of observer and emitter and calculating the flux of photons. However, in this case we prefer to use the well-known result obtained by Etherington, see \cite{etherington, etherington2}, that relates the luminosity distance $D_{\rm lum}$ to the angular diameter distance $D_{\rm ang}$ via the \emph{distance duality relation}
\begin{equation}
D_{\rm lum}=(1+z)^2 D_{\rm ang}\, . \label{eq:etherington_rel}
\end{equation}


\subsection{The parallax and the parallax distance}
Another distance estimator in astronomy is the parallax distance, called this way because it takes into account the apparent change in position of a source on the celestial sphere when viewed from at least two different viewpoints, known as the parallax effect. Unlike angular diameter distance or luminosity distance, it does not require knowledge of the source's properties. For this reason, parallax is an attractive method of measuring distance. However, studying the parallax of distant objects is complex and requires accurate astrometric measurements \cite{Ding:2009xs, gaiadr2-parallaxes, Hobbs:2019arz}.
On top of that, the parallax has a straightforward interpretation only in a flat space and in non-relativistic context: its generalization to general relativity is more cumbersome, generating confusion on its interpretation \cite{mccrea, weinberg-letter, kasai, rosquist, rasanen, Marcori:2018cwn}. A covariant treatment of cosmic parallax was proposed by R\"{a}s\"{a}nen in \cite{rasanen}, where the author distinguishes different definitions of parallax by the distance between observation points $\delta x^{\mu}_{\calO}$. %Following \cite{rasanen, Grasso:2018mei}, we classify the following source of parallax:
%\begin{enumerate}
%\item for $\delta x^{\mu}_{\calO}$ spacelike 
%	\begin{itemize}
%	\item \emph{classic (or trigonometric) parallax}: a single source $\calS$ is seen simultaneously by two different observers $\calO$ and $\calO'$ in $N_{\calO}$;
%	\item \emph{relative parallax}: two sources $\calS$ and $\calS'$ in $N_{\calS}$ are seen simultaneously by two observers $\calO$ and $\calO'$ in $N_{\calO}$.% In this case we have also dependence on the sources' motion. 
%	\end{itemize}
%\item for $\delta x^{\mu}_{\calO}$ timelike 
%	\begin{itemize}
%	\item \emph{single worldline parallax}: a single source $\calS$ is observed by $\calO$ at two consecutive instants $\tau_{\calO}$ and $\tau_{\calO}+ \delta \tau_{\calO}$. Under certain assumptions it is possible to relate the single worldline parallax to the classic parallax;%%\footnote{In general, the observations performed at different events will register light emitted by $\calS$ in different moments along the source's worldline.  One may, however, devise a measurement in which many observers will deliberately measure the emitter's position at carefully chosen moments so that all of them register signals emitted exactly at the same moment $\calE$. This can be achieved by appropriate timing of the observations.}
%	\item \emph{position drift}: two different sources $\calS$ and $\calS'$ are observed by $\calO$ at two consecutive instants $\tau_{\calO}$ and $\tau_{\calO}+ \delta \tau_{\calO}$; 
%	\end{itemize}
%\end{enumerate}
%On top of that, we have also the case where the parallax of a source is generated by a change in the reference frame
In the following we will focus on two definitions of parallax: the classic parallax, that we use to define the (classical) parallax distance, and the position drift.

Let us examine the classic parallax of a source $\calS$ as it is seen by a number of observers in $N_{\calO}$. The observers are chosen such that they all perceive signals emitted exactly at the same moment by $\calS$. In other words, all observation points lie on the same null hypersurface $\delta x_\calO^\mu\,\ell_{\calO\,\mu} = \const$:
%%%%%%%%%%%%%%%%%%%%%%%%%%%%%%%%%%%%%%%%%%%%%%%%%%%%%%%%%%%%%
\bfi
\centering
\includegraphics[width=0.8\textwidth]{pict/parallax2.pdf}
\caption{The worldlines of two observers $\calO$ and $\calO'$ cross the same null hypersurface $\delta x_\calO^\mu\,\ell_{\calO\,\mu} = 0$ and observe the source $\calS$, which lies on the corresponding null hypersurface $\delta x_\calS^\mu\,\ell_{\calS\,\mu} = 0$ in $N_{\calS}$. $\calO$ and $\calO'$ are displaced by $\delta x^{\mu}_{\calO}$, and they perceive the source in the apparent positions $\sf s$ and $\sf s'$, respectively. The difference between $\sf s$ and $\sf s'$ gives the parallax angle $\delta \theta^{\mu}_{\calO}$. On the screen $\hat{\phi}^{\mu}_{\bm{A}}$, the displacement $\delta x^{\bm{A}}_{\calO}$ and the angular distance $\delta \theta^{\bm{A}}_{\calO}$ are related by the parallax matrix $\Pi\UD{\bm{A}}{\bm{B}}$.}
\label{fig:parallaxmatrix}
\efi
%%%%%%%%%%%%%%%%%%%%%%%%%%%%%%%%%%%%%%%%%%%%%%%
for simplicity we assume that the observers are comoving\footnote{This way we do not need to consider the aberration effects when comparing the results of their measurements.} and they all perform the measurement when their worldlines cross the null hypersurface $\delta x_\calO^\mu\,\ell_{\calO\,\mu} = 0$. Therefore, at the moment of observation we have $[\delta x_\calO] \in \calP_\calO$ for the equivalence class of their displacement vectors, see Fig.~\ref{fig:parallaxmatrix}. 

The difference in the apparent position of the source, as measured by the two observes $\calO$ and  $\calO'$, defines the classic parallax. Using Eq.~\eqref{eq:rA_wBGO} with $\delta x^{\mu}_{\calS}=0$ the classic parallax takes the form
\bea
\delta \theta_\calO^{\bm A} = - \dfrac{1}{\ell_{\calO\, \sigma} u^{\sigma}_{\calO}}\left(\WXL^{-1}\right){}\UD{\bm A}{\bm B} \, \WXX{}\UD{\bm B}{\bm C} \, \delta x_\calO^{\bm C}\, .\label{eq:parallax_dev}
\eea
%%%%%%%%%%%%
\bfi
\centering
 \includegraphics[width=0.8\textwidth]{pict/Dpar2.pdf}
\caption{The source $\calS$ is observed by $\calO$, $\calO_{\rm 1}$, and $\calO_{\rm 2}$ in $N_{\calO}$. The positions of the three observers form the blue triangle $T_{\rm 2}$ in the space perpendicular to $\ell_{\calO}$. Similarly, on the shared celestial sphere $S_{\calO}$, the apparent positions of the source $\sf s$, $\sf s_{\rm 1}$, and $\sf s_{\rm 2}$ form the orange triangle $T_{\rm 1}$. The parallax distance is defined as the ratio between the area of $T_{\rm 2}$ and the solid angle of $T_{\rm 1}$.}
\label{fig:Dpar}
\efi
%%%%%%%%%%%%%
The product
\begin{equation}
\Pi\UD{\bm A}{\bm B}= \dfrac{1}{\ell_{\calO\, \sigma} u^{\sigma}_{\calO}}\left(\WXL^{-1}\right){}\UD{\bm A}{\bm B}\, \WXX{}\UD{\bm B}{\bm C}\, ,\label{eq:parallax_matrix}
\end{equation}
defines the observer-dependent \emph{parallax matrix}, namely the map between perpendicular displacement on the observer's side $\delta x^{\bm{A}}_{\calO}$, and two-dimensional angles $\delta \theta^{\bm{A}}_{\calO}$ measuring the observed position on the sky in comparison with the position observed by $\calO'$ at $\calO$, see Fig.~\ref{fig:Dpar}.

In astronomy the parallax has been used to measure the distances to objects up to few kiloparsecs \cite{2001A&A...369..339P,  Riess:2014uga}. In the following we present the parallax distance formula in terms of BGO\footnote{As noted in  \cite{rasanen, Grasso:2018mei}, for curved spacetimes the trigonometric parallax angle depends on the direction of the baseline $\delta x^{\mu}_{\calO}$. To overcome this problem, we use the baseline-averaged definition of parallax distance as presented in \cite{Grasso:2018mei}.} by using the definition of the classic parallax discussed earlier. Let us consider an observer $\calO$ and two additional observers $\mathcal{O}_{1}$ and $\mathcal{O}_{2}$, comoving with $\calO$ and such that their displacement with respect to $\calO$ are on $\calP_\calO$. We also introduce a screen frame $[\hat{\phi}_{\bm{A}}] \in \calP_\calO$, which is parallel transported on $N_{\calO}$. The three observers define a triangle $T_2$ on the screen space perpendicular to $\ell_{\calO}$ with area $A_\calO= \int_{T_2} \delta x^{\bm{1}}_\calO \wedge \delta x^{\bm{2}}_\calO$. 
Now, the observers measure the apparent position of a source $\calS$ and, using the parallel transported frame, they combine their observations on a shared celestial sphere $S_\calO$, see Fig.~\ref{fig:Dpar}. The combined observation form a solid triangle $T_1$ on $S_\calO$. 
Denoting as $\Omega_\calO=\int_{T_1} \delta \theta^{\bm{1}}_\calO \wedge \delta \theta^{\bm{2}}_\calO$ the solid angle taken up by $T_1$, we define the parallax distance as
\begin{equation}
D_{par} = \sqrt{\left|\frac{A_\calO}{\Omega_\calO}\right|}\, . \label{eq:D_par_def}
\end{equation}
Using Eq.~\eqref{eq:parallax_dev} to express the solid angle, i.e. 
\begin{equation}
\Omega_\calO=\int_{T_1} \delta \theta^{\bm{1}}_\calO \wedge \delta \theta^{\bm{2}}_\calO= \dfrac{1}{(\ell_{\calO\, \sigma} u^{\sigma}_{\calO})^2}{\rm det}\left[\left(\WXL^{-1}\right){}\UD{\bm A}{\bm B} \WXX{}\UD{\bm B}{\bm C}\right] \,\int_{T_2} \delta x^{\bm{1}}_\calO \wedge \delta x^{\bm{2}}_\calO\, ,
\end{equation}
the expression for the parallax distance becomes
\begin{equation}
D_{par} =\left(\ell_{\sigma} u^{\sigma}\right)|_{\cal O} \frac{\left| \det \left(\WXL {}\UD{\bm A}{\bm B}\right) \right|^{\frac{1}{2}}}{\left| \det \left(\WXX {}\UD{\bm A}{\bm B}\right) \right|^{\frac{1}{2}}}\, .
\label{eq:D_par_BGO}
\end{equation}
%%%%%%%%%%%%%%%%%%%%%%%%%%%%%%%%%%%%%%%%%%%%%%%

\subsection{Position drift} 
Among the various definitions of parallax, there is also the case where a single observer measures the temporal changes in the position of the source in the sky. %with a given worldline identifies directions on the observer's spheres at different times. 
This momentary rate of change of the source's position in the observer's sky is the \emph{position drift} $\delta_{\calO} r^{\bm{A}}$ \cite{Korzynski:2018, Grasso:2018mei} and is one of the real-time measurements commonly referred to as \emph{drift effects}. In contrast to the classic parallax, the position drift depends on the four-velocities of both the observer and the emitter, involving also the observer's four-acceleration \cite{Korzynski:2018, Hellaby:2017soj, Marcori:2018cwn}.
A general formula for the position drift has already been presented in \cite{Korzynski:2018}, while a special case for spherically symmetric metrics was presented in \cite{Quercellini:2009ni, Quercellini:2010zr}. Here we present the general formula for the position drift in terms of BGO as derived in \cite{Grasso:2018mei}.

As was said many times, the position drift measures the temporal change of directions on the observer's sky as the observer moves along its worldline. This means that, contrary to the other observables considered so far, we are actually measuring changes in the direction vector as the observer crosses different null hypersurfaces $\delta x^{\sigma}\ell_{\sigma}=\const$. This implies that, having fixed a reference frame, we need to find a way to transport it along the observer's worldline in order to be able to measure the changes in the directions registered on the next null hypersurface. The choice we make is to use the Fermi-Walker transport, which reduces to the usual parallel transport if the $\calO$'s worldline is a geodesic. The Fermi-Walker transport of vectors in the observer's sphere $\textrm{Dir}(u_\calO)$ along the worldline defines our ``fixed directions on the sky'' \cite{Hellaby:2017soj, Korzynski:2018}. 
The Fermi-Walker derivative of the direction vector $r_0^{\mu}$ in Eq.~\eqref{eq:dir} is expressed as
\begin{equation}
\delta_{\calO} r^{\mu}= u_{\calO}^{\sigma}\nabla_{\sigma} r_0^{\mu} - u_{\calO}^{\mu} w_{\calO\, \sigma}r_0^{\sigma}+w_{\calO}^{\mu} u_{\calO\, \sigma}r_0^{\sigma}\, ,\label{eq:Fermi-Walker_der}
\end{equation}
where $u_{\calO}^{\mu}$ and $w_{\calO}^{\mu}$ are the observer's four-velocity and four-acceleration respectively. 
%From the expression of the direction vector $r_0^{\mu}$ Eq.~\eqref{eq:dir}
From Eq.~\eqref{eq:dir} follows that the last term in Eq.~\eqref{eq:Fermi-Walker_der} vanishes, since $u_{\calO\, \sigma}r^{\sigma}_0=0$, while the covariant derivative $u_{\calO}^{\sigma}\nabla_{\sigma} r_0^{\mu}$ is\footnote{The relation $\Delta \ell^{\mu}=\delta \tau_{\calO} u^{\sigma}_{\calO}\nabla_{\sigma}\ell^{\mu}_{\calO}$ follows from the definition $\Delta \ell^{\mu}_{\calO}=\ell^{\sigma}\nabla_{\sigma} \delta x^{\mu}_{\calO}$ by using Eqs.~\eqref{eq:commutation_vector} and expressing $\delta x^{\mu}_{\calO}=u^{\mu}_{\calO} \delta \tau_{\calO}$.}
\begin{equation}
u^{\sigma}_{\calO}\nabla_{\sigma} r^{\mu}_0=\dfrac{u^{\sigma}_{\calO}\nabla_{\sigma}\ell^{\mu}_{\calO}}{\left(\ell^{\rho}_{\calO}u_{\calO}{}_{\rho}\right)} +u^{\sigma}_{\calO}\nabla_{\sigma} u^{\mu}_{\calO}=\dfrac{\dfrac{\Delta \ell^{\mu}_{\calO}}{\delta \tau_{\calO}}}{\left(\ell^{\rho}_{\calO}u_{\calO}{}_{\rho}\right)} +w^{\mu}_{\calO}\, ,\label{eq:uDr}
\end{equation}
with $\delta \tau_{\calO}$ the observer's proper time. 
The pull-back to $\calP_{\calO}$ of Eq.~\eqref{eq:Fermi-Walker_der} defines the position drift $\delta_{\calO} r^{\bm{A}}$ measured with respect to inertially dragged fixed directions
\begin{equation}
\delta_{\calO} r^{\bm{A}}= \dfrac{\dfrac{\Delta \ell^{\bm{A}}_{\calO}}{\delta \tau_{\calO}}}{\left(\ell^{\rho}_{\calO}u_{\calO}{}_{\rho}\right)} +w^{\bm{A}}_{\calO}\, .\label{eq:position_drift_def}
\end{equation} 
%%%%%%%%%%%%%%%%%%%%%%%
The first term in Eq.~\eqref{eq:position_drift_def} can be obtained from Eq.~\eqref{eq:Dl_inBGO_with_Ps} as
\bea
\dfrac{\Delta \ell_{\calO}^{\bm A}}{\delta \tau_{\calO}} = \left(\WXL^{-1}\right){}\UD{\bm A}{\bm B}\,\left[\frac{1}{1+z}\,u_\calE - \WXX(u_\calO) \right]^{\bm{B}}\, , \label{eq:rA_in_tau}
\eea
where we expressed the two displacements as
\begin{align}
\delta x^{\mu}_{\calO} = & u^{\mu}_{\calO} \delta \tau_{\calO}\\
\delta x^{\mu}_{\calS} = & u^{\mu}_{\calS} \delta \tau_{\calS} \, , \label{eq:dx_udt}
\end{align}
and we used the relation\footnote{The relation is obtained from the condition $\delta x^{\mu}_{\calO}\ell_{\calO\, \mu}=\delta x^{\mu}_{\calS}\ell_{\calS\, \mu}$, by using Eq.~\eqref{eq:dx_udt} and the definition of redshift Eq.~\eqref{eq:redshift_def}.}
\begin{equation}
\delta \tau_{\calS}=\dfrac{\delta \tau_{\calO}}{1+z}\, , \label{eq:tE-tO_formula}
\end{equation}
between the proper time as measured at the observer $\delta \tau_{\calO}$ and the proper time as measured at the source $\delta \tau_{\calS}$, \cite{perlick, kermack_mccrea_whittaker_1934, Korzynski:2018}.
%%%%%%%%%%%%%%%%%%%%%%%%%%%%%%%%%%%%%%%%%%%%%%%%%%%%%%%
Combining (\ref{eq:position_drift_def}) and (\ref{eq:rA_in_tau}) yields
\bea
\delta_\calO r^{\bm A} = \dfrac{1}{\ell_{\calO\, \sigma} u^{\sigma}_{\calO}} \left(\WXL^{-1}\right){}\UD{\bm A}{\bm B}\,\left[\frac{1}{1+z}\,u_\calE - \WXX(u_\calO) \right]^{\bm{B}} + w_\calO^{\bm A}\, . \label{eq:positionDRIFT_BGO}
\eea
%\delta_\calO r^{\bm A} = \dfrac{1}{\ell_{\calO\, \sigma} u^{\sigma}_{\calO}} \left(\WXL{}\UD{\bm A}{\bm B}\right)^{-1}\,\left(\frac{1}{1+z}\,u^{\bm{B}}_\calE - \WXX{}\UD{\bm B}{\bm C}\,u_\calO^{\bm C} \right) + w_\calO^{\bm A}\, .
Note that the last term is the perpendicular component of the observer's four-acceleration. It corresponds to the special relativistic effect of the position drift due to the drift of the aberration \cite{rasanen,Korzynski:2018,Marcori:2018cwn}.  %Its influence on the drift of the positions of sources at cosmological distances has been recently discussed in \cite{Marcori:2018cwn}. 
For a longer discussion of the position drift formula and its physical and astrophysical consequences see \cite{Korzynski:2018, Grasso:2018mei}.


\subsection{The redshift drift formula}
The last observable under consideration is the redshift drift, a real time observable expressing the temporal changing of the redshift, due to cosmic expansion and proper motion of the source and the observer. The formulation of the redshift drift was firstly proposed by Sandage in 1962 \cite{sandage}, and later applied by A. Loeb \cite{loeb} as a tracer of the expansion of the Friedmann-Lema\^{i}tre-Robertson-Walker Universe.  Here, we will present the derivation of the general formula of the redshift drift in terms of BGO, \cite{Julius}.

Let us consider two consecutive measurements of the redshift $z$ as taken by the observer $\mathcal{O}$ at the two instants $\tau_{\mathcal{O}}$ and $\tau_{\mathcal{O}} + \delta \tau_{\mathcal{O}}$. In the time lapse $\delta \tau_{\mathcal{O}}$ the observer and the source are shifted along their worldlines by:
\begin{equation}
\begin{array}{l}
\delta x^{\mu}_{\mathcal{O}}=u^{\mu}_{\mathcal{O}} \delta \tau_{\mathcal{O}} \\
\delta x^{\mu}_{\mathcal{S}}=u^{\mu}_{\mathcal{S}} \delta \tau_{\mathcal{S}}=\dfrac{1}{1+z}u^{\mu}_{\mathcal{S}} \delta \tau_{\mathcal{O}}\, ,
\end{array}
\end{equation}
where we have used the relation between $\delta \tau_{\mathcal{S}}$ and $\delta \tau_{\mathcal{O}}$ in Eq.~\eqref{eq:tE-tO_formula}. 
Now, the redshift drift is obtained varying with respect to the observer proper time the definition of the redshift. For our convenience, let us take the logarithm of the redshift in Eq.~\eqref{eq:redshift_def}
\begin{equation}
\log(1+z)=\log(\left.\ell^{\mu}u_{\mu}\right|_{\mathcal{S}})-\log(\left.\ell^{\mu}u_{\mu}\right|_{\mathcal{O}})\, ,
\end{equation}
and do its variation
\begin{equation}
\delta \log(1+z)=\dfrac{(\Delta \ell^{\mu}_{\mathcal{S}}u_{\mathcal{S}}{}_{\mu}+\ell^{\mu}_{\mathcal{S}}\Delta u_{\mathcal{S}}{}_{\mu})}{\ell^{\mu}_{\mathcal{S}}u_{\mathcal{S}}{}_{\mu}}-\dfrac{(\Delta \ell^{\mu}_{\mathcal{O}}u_{\mathcal{O}}{}_{\mu} + \ell^{\mu}_{\mathcal{O}}\Delta u_{\mathcal{O}}{}_{\mu})}{\ell^{\mu}_{\mathcal{O}}u_{\mathcal{O}}{}_{\mu}}\, .
\label{eq:red_drift_intermediate1}
\end{equation}
The two terms
\begin{equation}
\begin{array}{l}
\Delta u^{\mu}_{\mathcal{O}}=w^{\mu}_{\mathcal{O}} \delta \tau_{\mathcal{O}}\, , \\
\Delta u^{\mu}_{\mathcal{S}}=w^{\mu}_{\mathcal{S}} \delta \tau_{\mathcal{S}}=\dfrac{1}{1+z}w^{\mu}_{\mathcal{S}} \delta \tau_{\mathcal{O}}
\end{array}\, ,
\end{equation}
define the four-acceleration of the observer $w_{\calO}$ and the emitter $w_{\calS}$.

The variation in Eq.~\eqref{eq:red_drift_intermediate1} can be reshuffled as
\begin{equation}
\begin{array}{l l}
\delta \log(1+z)= & \\
\left(\dfrac{1}{1+z}\dfrac{\ell^{\mu}_{\mathcal{S}}w_{\mathcal{S}}{}_{\mu}}{\ell^{\mu}_{\mathcal{S}}u_{\mathcal{S}}{}_{\mu}}-\dfrac{\ell^{\mu}_{\mathcal{O}}w_{\mathcal{O}}{}_{\mu}}{\ell^{\mu}_{\mathcal{O}}u_{\mathcal{O}}{}_{\mu}}\right)\delta \tau_{\mathcal{O}}+\left(\dfrac{\Delta \ell^{\mu}_{\mathcal{S}}u_{\mathcal{S}}{}_{\mu}}{\ell^{\mu}_{\mathcal{S}}u_{\mathcal{S}}{}_{\mu}}-\dfrac{\Delta \ell^{\mu}_{\mathcal{O}}u_{\mathcal{O}}{}_{\mu}}{\ell^{\mu}_{\mathcal{O}}u_{\mathcal{O}}{}_{\mu}}\right)
\end{array}\, .
\label{eq:red_drift_intermediate2}
\end{equation}
The first term is a special relativistic term representing the Doppler effect along the line of sight caused by the four-acceleration of the observer and the emitter 
\begin{equation}
\Xi_{\rm Doppler}=\left[\dfrac{1}{1+z}\dfrac{\left(\ell^{\mu}w_{ \mu}\right)|_{\mathcal{S}}}{\left(\ell^{ \mu} u_{ \mu}\right)|_{\mathcal{S}}}-\dfrac{\left(\ell^{ \mu}w_{ \mu}\right)|_{\mathcal{O}}}{\left(\ell^{ \mu} u_{ \mu}\right)|_{\mathcal{O}}}\right]
\label{eq:doppler}\, .
\end{equation}
The second term contains the effects of the spacetime curvature on the redshift drift and it can be expressed in terms of the BGO. Let us start by writing the second term in the matrix form:
\begin{equation}
\dfrac{\Delta \ell^{\mu}_{\mathcal{S}}u_{\mathcal{S}}{}_{\mu}}{\ell^{\mu}_{\mathcal{S}}u_{\mathcal{S}}{}_{\mu}}-\dfrac{\Delta \ell^{\mu}_{\mathcal{O}}u_{\mathcal{O}}{}_{\mu}}{\ell^{\mu}_{\mathcal{O}}u_{\mathcal{O}}{}_{\mu}}=-\left(\dfrac{u_{\mathcal{O}}{}_{\nu}}{\ell^{\mu}_{\mathcal{O}}u_{\mathcal{O}}{}_{\mu}}\ \  \dfrac{u_{\mathcal{S}}{}_{\mu}}{\ell^{\mu}_{\mathcal{S}}u_{\mathcal{S}}{}_{\mu}}\right) \cdot\left(\begin{matrix}
\Delta \ell^{\nu}_{\mathcal{O}}\\
-\Delta \ell^{\mu}_{\mathcal{S}}
\end{matrix} \right)\, .
\label{eq:second_term1}
\end{equation}
The vector $\left( \Delta \ell^{\nu}_{\mathcal{O}}\ \ -\Delta \ell^{\mu}_{\mathcal{S}} \right)^{\rm T}$ is expressed in terms of the BGO using  Eqs.~\eqref{eq:positiondeviation1}-\eqref{eq:directiondeviation1}
\begin{equation}
\left\{\begin{array}{l}
\Delta \ell^{\nu}_{\mathcal{O}}= \WXL^{-1}{}\UD{\nu}{\rho} \delta x^{\rho}_{\mathcal{S}}- \WXL^{-1}{}\UD{\nu}{\rho}\WXX{}\UD{\rho}{\sigma} \delta x^{\sigma}_{\mathcal{O}}\\
-\Delta \ell^{\mu}_{\mathcal{S}}= -\WLX{}\UD{\mu}{\rho} \delta x^{\rho}_{\mathcal{O}} - \WLL{}\UD{\mu}{\nu} \WXL^{-1}{}\UD{\nu}{\rho} \delta x^{\rho}_{\mathcal{S}} + \WLL{}\UD{\mu}{\nu}\WXL^{-1}{}\UD{\nu}{\rho}\WXX{}\UD{\rho}{\sigma} \delta x^{\sigma}_{\mathcal{O}}
\end{array}\right. \, ,
\end{equation}
from which we finally get
\begin{equation}
\begin{pmatrix}
\Delta \ell^{\nu}_{\mathcal{O}}\\
-\Delta \ell^{\mu}_{\mathcal{S}}
\end{pmatrix}=\begin{pmatrix}
- \WXL^{-1}{}\UD{\nu}{\rho}\WXX{}\UD{\rho}{\sigma} & \WXL^{-1}{}\UD{\nu}{\rho} \\
\WLL{}\UD{\mu}{\nu}\WXL^{-1}{}\UD{\nu}{\rho}\WXX{}\UD{\rho}{\sigma}-\WLX{}\UD{\mu}{\sigma} & - \WLL{}\UD{\mu}{\nu} \WXL^{-1}{}\UD{\nu}{\rho}
\end{pmatrix} \begin{pmatrix}
\delta x^{\sigma}_{\mathcal{O}}\\
\delta x^{\rho}_{\mathcal{S}}
\end{pmatrix}
\label{eq:U_matrix}
\end{equation}

Denoting
\begin{equation}
U=\begin{pmatrix}
- \WXL^{-1}{}\UD{\nu}{\rho}\WXX{}\UD{ \rho}{ \sigma} & \WXL^{-1}{}\UD{ \nu}{ \rho} \\
\WLL{}\UD{ \mu}{ \nu}\WXL^{-1}{}\UD{ \nu}{ \rho}\WXX{}\UD{ \rho}{ \sigma}-\WLX{}\UD{ \mu}{ \sigma} & - \WLL{}\UD{ \mu}{ \nu} \WXL^{-1}{}\UD{ \nu}{ \rho}
\end{pmatrix} \, 
\end{equation}
as the large $8 \times 8$ block matrix containing the BGO, Eq.~\eqref{eq:U_matrix} becomes
\begin{equation}
\begin{pmatrix}
\Delta \ell^{\nu}_{\mathcal{O}}\\
-\Delta \ell^{\mu}_{\mathcal{S}}
\end{pmatrix}=U \begin{pmatrix}
\delta x^{\sigma}_{\mathcal{O}}\\
\delta x^{\rho}_{\mathcal{S}}
\end{pmatrix}=U\begin{pmatrix}
 u^{\sigma}_{\mathcal{O}} \delta \tau_{\mathcal{O}}\\
\dfrac{\delta \tau_{\mathcal{O}}}{1+z} u^{\rho}_{\mathcal{S}}
\end{pmatrix}\, ,
\label{eq:U_matrix_2}
\end{equation}
and it can then inserted in Eq.~\eqref{eq:second_term1} that finally becomes
\begin{equation}
\begin{array}{l c}
-(\dfrac{u_{\mathcal{O}}{}_{\nu}}{\ell^{\mu}_{\mathcal{O}}u_{\mathcal{O}}{}_{\mu}}\ \  \dfrac{u_{\mathcal{S}}{}_{\mu}}{\ell^{\mu}_{\mathcal{S}}u_{\mathcal{S}}{}_{\mu}}) \cdot\left(\begin{matrix}
\Delta \ell^{\nu}_{\mathcal{O}}\\
-\Delta \ell^{\mu}_{\mathcal{S}}
\end{matrix} \right)= & \\
-\dfrac{\delta \tau_{\mathcal{O}}}{\ell^{\mu}_{\mathcal{O}}u_{\mathcal{O}}{}_{\mu}}(u_{\mathcal{O}}{}_{\nu}\ \  \dfrac{u_{\mathcal{S}}{}_{\mu}}{1+z}) \cdot U \cdot \begin{pmatrix}
 u^{\sigma}_{\mathcal{O}}\\
\dfrac{u^{\rho}_{\mathcal{S}}}{1+z} 
\end{pmatrix} \, ,  & \\
\end{array}
\end{equation}
where we invite the reader to notice that this derivation was made considering $U$ with upper-down indices distribution.
%The U matrix is applied on the 8-index vector $Y$, defined as 
%$$Y=\begin{pmatrix}
% u^{\sigma}_{\mathcal{O}}\\
%\dfrac{u^{\rho}_{\mathcal{S}}}{1+z} 
%\end{pmatrix}\, . $$

Finally, Eq.~\eqref{eq:red_drift_intermediate2} in terms of the new defined quantities gives the expression of the redshift drift $\frac{\delta \log(1+z)}{\delta \tau_{\mathcal{O}}} \equiv \zeta$ in terms of the BGO
\begin{equation}
\zeta= \Xi_{\rm Doppler}- (u_{\mathcal{O}}{}_{\nu}\ \  \dfrac{u_{\mathcal{S}}{}_{\mu}}{1+z}) \cdot U \cdot \begin{pmatrix}
 u^{\sigma}_{\mathcal{O}}\\
\dfrac{u^{\rho}_{\mathcal{S}}}{1+z} 
\end{pmatrix}\, .
\label{eq:z_DRIFT_BGO}
\end{equation}
%\zeta= \Xi_{\rm Doppler}- \left( u_{\mathcal{O}} \, \, \frac{u_{\mathcal{S}}}{1+z} \right) \cdot \boldsymbol{U} \cdot \begin{pmatrix} u_{\mathcal{O}}\\ \dfrac{u_{\mathcal{S}}}{1+z}\end{pmatrix}\, .
The expression Eq.~\eqref{eq:z_DRIFT_BGO} is completely general in the sense that it was derived from general considerations and without referring to a specific model. Of course, the specific expression of the BGO is dictated by the particular form of the spacetime in which they are calculated, but once the BGO are computed, they can be used to calculate the redshift drift with the formula above. 

%%%%%%%%%%%%%%%%%%%%%%%%%%%%%%%%%%%%

In conclusion, the BGO are fundamental objects describing multiple effects on light propagation in the regime of geometric optics. Let us notice that, although the BGO formalism is independent of the frame used, the observables depend on the emitter and observer kinematics, as shown by the explicit dependence of $u^{\mu}_{\mathcal{O}}$, $u^{\mu}_{\mathcal{S}}$, $w^{\mu}_{\mathcal{O}}$ and $w^{\mu}_{\mathcal{S}}$ in Eqs.~\eqref{eq:D_ang_BGO}-\eqref{eq:z_DRIFT_BGO}. Indeed, it is possible to apply the Lorentz transformations to change reference frame, but this modifies the observables introducing special relativistic effects such as the Doppler effect or aberration.
In this sense, the BGO provide a unified framework for computing all optical observables, such us those in Eqs.~\eqref{eq:D_ang_BGO},% Eqs.~\eqref{eq:etherington_rel},
 \eqref{eq:D_par_BGO}, \eqref{eq:positionDRIFT_BGO}, and \eqref{eq:z_DRIFT_BGO}. 
Moreover, while there are already analogous formulas for $D_{\rm ang}$ and $D_{\rm par}$, where instead of the BGO we have the magnification and the parallax matrix (see \cite{Korzynski:2019oal} for the comparison), there was no general formula for the position drift and the redshift drift: Eqs.~\eqref{eq:positionDRIFT_BGO} and \eqref{eq:z_DRIFT_BGO} look the same for each spacetime model considered. %The standard approach to compute the position and redshift drift use the specific symmetries of the metric tensor and null-geodesic, the latter depending in turns on the initial conditions. This means that the equations of the drift effects in the standard approach look different for each specific model and/or configuration of the light rays.

\endinput
	%%%%%%%%%%%%%%%%%%%%%%%%%%%%%%%%%%%%%%%%%%%%%%%%%%%%%%%%%%%%%%%%%%%%%%%%%%%%%%%%%%%%%%%%%%%%%%%%%%%%%%%%%%%%%%%%%%%%%%%%%%%%%%%%%%%%%%%%%%%%%%%%%%%%%%%%%%%%%%%%%%%%%%%				


%\section{Inverted integration direction with BGO}

%The usual procedure for studying light propagation consists in giving the initial conditions at $\mathcal{O}$ and by integrating the GDE backward in time, up to the source $\mathcal{S}$. The physical motivation is of course that every measurement is done from the observer position and this explains why all observables are expressed in terms of $\mathcal{W}(\mathcal{S},\mathcal{O})$.
%The BGO $\mathcal{W}(p_{\lambda},\mathcal{O})$ obtained by integrating Eq.~\eqref{eq:GDE_W} backwards connect the observer with an arbitrary point $p_{\lambda}$ on the geodesic, ending at the source  $p_{\lambda_{\mathcal{S}}}=\mathcal{S}$. 
%Although it is natural to study light propagation this way, there are circumstances in which it is more convenient to think forward in time, namely to integrate Eq.~\eqref{eq:GDE_W} from the source to $p_{\lambda}$, ending at the observer $p_{\lambda_{\mathcal{O}}}=\mathcal{O}$, and obtain $\mathcal{W}(p_{\lambda},\mathcal{S})$. To this second case belong all the simulations of cosmological dynamics, in which the Einstein equations are solved forward in time, with initial conditions given at the end of inflation. It follows that forward integration of light propagation, on-the-fly with the simulation of the spacetime dynamics, would be a cost-efficient and time-saving methodology instead of using the natural approach for light propagation in post-processing. 
%These two opposing procedures find a meeting point within the BGO framework, which provides a relatively easy way to transform from forward integrated $\mathcal{W}(p_{\lambda},\mathcal{S})$ to backward integrated $\mathcal{W}(p_{\lambda},\mathcal{O})$, which are the one used to compute observables. 
%We reserve this last section to show how to invert the integration order from $\calW(\calO, \calS)$ to $\calW(\calS, \calO)$, see \cite{Grasso:2021iwq}. 

%The transformation follows from the BGO properties and it is found by multiplying Eq.~\eqref{eq:Wcomposition} by $\calW^{-1} (\calS,p_{\lambda})$ from the left and using Eq.~\eqref{eq:Winversion} to obtain\footnote{In the last equality of Eq.~\eqref{eq:W(p,o)_from_W(p,s)} we used Eq.~\eqref{eq:Winversion}.}
%\begin{align}
%\calW (p_{\lambda},\calO)&=\calW^{-1} (\calS,p_{\lambda})\,  \calW (\calS,\calO) \nonumber \\
%&=\calW (p_{\lambda},\calS)\, \calW^{-1} (\calO, \calS) \, .
%\label{eq:W(p,o)_from_W(p,s)}
%\end{align}
%In order to be used, the Eq.~\eqref{eq:W(p,o)_from_W(p,s)} requires the computation of the inverse of a $8 \times 8$ matrix $ \calW^{-1} (\calO, \calS)$, which in general is not a simple task. Here is where we use the symplectic property of $\mathcal{W}$, i.e.
%\begin{equation}
%\calW^T{}\UD{\tilde{m}}{\tilde{a}} \Omega_{\tilde{m} \tilde{s}} \calW\UD{\tilde{s}}{\tilde{b}} =\Omega_{\tilde{a} \tilde{b}}\, ,
%\label{eq:propW_symplectic}
%\end{equation}
%where $\Omega$ is the $8 \times 8$ non-singular, skew-symmetric matrix 
%\begin{equation}
%\Omega_{\tilde{a} \tilde{b}}=\begin{pmatrix}
%0 & h_{\boldsymbol{\alpha} \boldsymbol{\beta}}\\
%-h_{\boldsymbol{\gamma} \boldsymbol{\delta}} & 0
%\end{pmatrix}\, ,
%\end{equation}
%with Latin tilded indices running from $0$ to $7$ ($\tilde{a}, \dots=0,1,\dots,7 $) and the Greek bold indices ($\boldsymbol{\alpha}=\bm{0},\bm{1},\dots,\bm{3}$) indicate the components in the SNF. 
%By inverting Eq.~\eqref{eq:propW_symplectic} we find
%\begin{align}
%\calW^{-1}&=\Omega^{-1 }\calW^T \Omega \nonumber\\
%&=\begin{pmatrix}
%0 & -h^{\boldsymbol{\alpha} \boldsymbol{\rho}}\\
%h^{\boldsymbol{\beta} \boldsymbol{\sigma}} & 0
%\end{pmatrix}\begin{pmatrix}
%\WXX{}\UD{\boldsymbol{\nu}}{\boldsymbol{\sigma}} & \WLX{}\UD{\boldsymbol{\mu}}{\boldsymbol{\sigma}}\\
%\WXL{}\UD{\boldsymbol{\nu}}{\boldsymbol{\rho}} & \WLL{}\UD{\boldsymbol{\mu}}{\boldsymbol{\rho}}
%\end{pmatrix}\begin{pmatrix}
%0 & h_{\boldsymbol{\nu} \boldsymbol{\gamma}}\\
%-h_{\boldsymbol{\mu} \boldsymbol{\delta}} & 0
%\end{pmatrix}\, .
%\label{eq:derive_W_inverse}
%\end{align}
%Finally, using Eq.~\eqref{eq:derive_W_inverse} in Eq.~\eqref{eq:W(p,o)_from_W(p,s)}, the  transformation from forward to backward BGO reads
%\begin{align}
%\WXX(p_{\lambda}, \calO){}\UD{\boldsymbol{\sigma}}{\boldsymbol{\nu}}&=  \WXX(p_{\lambda}, \calS){}\UD{\boldsymbol{\sigma}}{\boldsymbol{\alpha}} h^{\boldsymbol{\alpha} \boldsymbol{\rho}} \, \WLL^T( \calO,\calS){}\UD{\boldsymbol{\mu}}{\boldsymbol{\rho}} \, h_{\boldsymbol{\mu} \boldsymbol{\nu}}+ \nonumber\\
%&-\WXL(p_{\lambda}, \calS){}\UD{\boldsymbol{\sigma}}{\boldsymbol{\alpha}}h^{\boldsymbol{\alpha} \boldsymbol{\rho}} \, \WLX^T( \calO,\calS){}\UD{\boldsymbol{\mu}}{\boldsymbol{\rho}} \, h_{\boldsymbol{\mu} \boldsymbol{\nu}}\label{eq:WXX_inverse}\\
%\WXL(p_{\lambda}, \calO){}\UD{\boldsymbol{\sigma}}{\boldsymbol{\nu}}&= - \WXX(p_{\lambda}, \calS){}\UD{\boldsymbol{\sigma}}{\boldsymbol{\alpha}} h^{\boldsymbol{\alpha} \boldsymbol{\rho}} \, \WXL^T( \calO,\calS){}\UD{\boldsymbol{\mu}}{\boldsymbol{\rho}} \, h_{\boldsymbol{\mu} \boldsymbol{\nu}}+ \nonumber\\
%&+\WXL(p_{\lambda}, \calS){}\UD{\boldsymbol{\sigma}}{\boldsymbol{\alpha}}h^{\boldsymbol{\alpha} \boldsymbol{\rho}} \, \WXX^T( \calO,\calS){}\UD{\boldsymbol{\mu}}{\boldsymbol{\rho}} \, h_{\boldsymbol{\mu} \boldsymbol{\nu}}\label{eq:WXL_inverse}\\
%\WLX(p_{\lambda}, \calO){}\UD{\boldsymbol{\sigma}}{\boldsymbol{\nu}}&=  \WLX(p_{\lambda}, \calS){}\UD{\boldsymbol{\sigma}}{\boldsymbol{\alpha}} h^{\boldsymbol{\alpha} \boldsymbol{\rho}} \, \WLL^T( \calO,\calS){}\UD{\boldsymbol{\mu}}{\boldsymbol{\rho}} \, h_{\boldsymbol{\mu} \boldsymbol{\nu}}+ \nonumber\\
%&-\WLL(p_{\lambda}, \calS){}\UD{\boldsymbol{\sigma}}{\boldsymbol{\alpha}}h^{\boldsymbol{\alpha} \boldsymbol{\rho}} \, \WLX^T( \calO,\calS){}\UD{\boldsymbol{\mu}}{\boldsymbol{\rho}} \, h_{\boldsymbol{\mu} \boldsymbol{\nu}}\label{eq:WLX_inverse}\\
%\WLL(p_{\lambda}, \calO){}\UD{\boldsymbol{\sigma}}{\boldsymbol{\nu}}&= - \WLX(p_{\lambda}, \calS){}\UD{\boldsymbol{\sigma}}{\boldsymbol{\alpha}} h^{\boldsymbol{\alpha} \boldsymbol{\rho}} \, \WXL^T( \calO,\calS){}\UD{\boldsymbol{\mu}}{\boldsymbol{\rho}} \, h_{\boldsymbol{\mu} \boldsymbol{\nu}}+ \nonumber\\
%&+\WLL(p_{\lambda}, \calS){}\UD{\boldsymbol{\sigma}}{\boldsymbol{\alpha}}h^{\boldsymbol{\alpha} \boldsymbol{\rho}} \, \WXX^T( \calO,\calS){}\UD{\boldsymbol{\mu}}{\boldsymbol{\rho}} \, h_{\boldsymbol{\mu} \boldsymbol{\nu}}
%\label{eq:WLL_inverse}
%\end{align}


%%%%%%%%%%%%%%%%%%%%%%%%%%%%%%%%%%%%%%%%%%%%%%%%%%%%%%%%%%%%%%%%%%%%%%%%%%%%%%%%%%%%%%%%%%%%%%%%%%%%%%%%%%%%%%%%%%%%%%%%%%%%%%%%%%%%%%%%%%%%%%%

