% Chapter Template
%\part{Study of parameter space for shock formation and oscillation}


\chapter{Paper II: ``Isolating nonlinearities of light propagation in inhomogeneous cosmologies''} % Main chapter title
\label{chap:nonlinearities} % Change X to a consecutive number; for referencing this chapter elsewhere, use \ref{ChapterX}

This chapter presents the application of {\tt BiGONLight} to study the nonlinear contributions to light propagation in an inhomogeneous cosmological model. The ultimate goal of this work is to isolate and quantify how different sources of inhomogeneities contribute to nonlinearities in cosmological observables. To this end, instead of using a realistic model of the Universe, it is preferable to have a toy model, whose properties can be tuned easily by changing its parameters. %Therefore, instead of making forecast of observables, we study in detail the different variables and effects of the nonlinearities on the observables.%we try to draw general conclusions rather than real treatment of light propagation, which would require a realistic model of the Universe. 
The \emph{wall Universe}, in which the matter is condensed in a sequence of plane-symmetric perturbations around a homogeneous distribution (FLRW background), is the toy model employed in this investigation. The analytical expression of its metric in PN approximation is given in \cite{Villa:2011vt} and is used in this analysis at three different approximations: linear PT, Newtonian, and PN.
The forms of the metric at Newtonian and PN approximations are provided as input metric in {\tt BiGONLight} to compute numerically the observables at Newtonian and PN order, respectively. The observables at linear PT are obtained analytically from the expressions of the BGO at linear order.
The nonlinear contributions are determined as relative differences between observables computed within these three different approximations.

The study answers the following four questions: 
\begin{enumerate}
	\item what are the Newtonian and PN corrections to the linear PT observables?
	\item what is the impact of the size of inhomogeneities?
	\item how much do the free parameters of the model affect the comparison?
	\item how important are the nonlinear PN corrections?
\end{enumerate}
Although many other authors have already examined the first question  (see e.g. \cite{Dyer:1974, Barausse:2005nf, Bonvin:2006, Meures:2011gp, Umeh:2012pn, Macpherson:2021gbh}), the other three questions penetrate deeply into the origin of nonlinearities, whose contributions are precisely evaluated by {\tt BiGONLight}.\\


\textit{\textbf{Author's contribution}}\\
\newline
The work described in this Chapter was performed in collaboration with E. Villa, M. Korzy\'nski, and  S. Matarrese.
The idea of this analysis was proposed by E. Villa and later discussed with the other authors. The $\Lambda$CDM extension of the metric in \cite{Villa:2011vt} (originally formulated for an EdS background) have been obtained jointly by me and E. Villa (Eq.~$2$ in \cite{Grasso:2021zra}). I have performed the simulations for light propagation and the computations of the observables in the three approximations. 
E. Villa and I did the preliminary analysis on the comparisons of the observables, and the results were discussed with all the other authors in order to draw conclusions. 
The comparison of the metric in Eq.~($7$) in \cite{Grasso:2021zra} with the Szekeres metric in \cite{Meures:2011ke} was done jointly by E. Villa and me. We also derived the analytical expressions of the linear observables with the BGO, and we did the match with the known formulas in the literature (i.e. without BGO). The draft was written by E. Villa and me and jointly published by all authors.





\includepdf[pages=-]{paper/nonlinear.pdf}

% In this model, we compute the redshift and the angular diameter distance within three different approximation schemes: linear, Newtonian and post-Newtonian. In order to quantify and isolate nonlinear contributions, we present our results in terms of the relative differences between observables computed with these three different approx- imations (see Sec. IV for details). We also analyse different aspects of nonlinearities, e.g., scale-dependence, non- Gaussianity, etc. Even if our modeling is very simple, we believe that this kind of analysis is representative of more general configurations.