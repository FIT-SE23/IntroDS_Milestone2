\chapter{BGO at linear PT and linear observables} 
\label{apx:linear_obs}

\MG{da non-linearities}
\subsection{Solutions for the plane-parallel universe}

In this appendix we consider the flat FLRW background with linear perturbations in the synchronous-comoving gauge. We obtain the linearised evolution equations for the BGO, Eq.~\eqref{eq:1PT_BGO_syst}, the general expressions for their solutions, Eq.~\eqref{eq:1PT_BGO}, and the linear angular diameter distance $D_{\rm ang}^{\rm Lin}$ written in terms of the BGO, Eq.\eqref{eq:Dlin_lambda}.
We then specialise the general solutions to the $\Lambda$CDM background with perturbations at first order in standard cosmological perturbation theory and within our plane-parallel toy model and finally obtain the analytic expressions for $z^{\rm Lin}$ and $D_{\rm ang}^{\rm Lin}$ that we used in Section~\ref{sec:method}. 

The spacetime metric has the form
\begin{equation}
\tilde{g}_{\mu \nu}=a^2 g_{\mu \nu}
\end{equation}
and is expanded at first order as
\begin{equation}
\tilde{g}_{\mu \nu}= a^2 (\bar{g}_{\mu \nu}+\delta g_{\mu \nu})
\end{equation}
where $\bar{g}_{\mu \nu}$ is the conformal flat FLRW background, i.e. the Minkowski metric and $\delta g_{\mu \nu}$ represents the first-order scalar perturbations in the synchronous-comoving gauge, in general given by $\delta g_{\mu \nu}= {\rm Diag}(0, \delta g_{11}, \delta g_{22}, \delta g_{33})$.

The first observable that we study in this work is the redshift, defined as in Eq.~\eqref{eq:redshift_def}
\begin{equation}
1+z=\frac{\left.\tilde{g}_{\mu \nu}\tilde{\ell}^{\mu}\tilde{u}^{\nu}\right|_{\mathcal{S}}}{\left.\tilde{g}_{\mu \nu}\tilde{\ell}^{\mu}\tilde{u}^{\nu}\right|_{\mathcal{O}}}\,.
\label{eq:redshift_def_app}
\end{equation}
In the above expression ${u}^{\mu} $ is the four-velocity of the observer (source) and $\ell^{\mu}$ is the tangent vector to the photon geodesics. For our coordinates choice, all observers and sources are comoving with the cosmic flow with four-velocity given by 
\begin{equation}
\tilde{u}^{\mu}\equiv \frac{1}{a}\bar{u}^{\mu} =\frac{1}{a}\left(1 ,0, 0, 0\right)
\end{equation}
at all orders and the null tangent vector is expanded as
\begin{equation}
\tilde{\ell}^{\mu}=\frac{a^2_{\rm \mathcal{O}}}{a^2}\left(\bar{\ell}^{\mu}+\delta \ell^{\mu}\right)
\end{equation}
with $\bar{\ell}^{\mu}=\left(\bar{\ell}^{0}, \bar{\ell}^{i}\right)$
\footnote{The conformal tangent vector in the FLRW background $\bar{\ell}^{\mu}$ is constant. Note that usually the normalisation $\bar{\ell}^{0}\pm 1$ is used for the temporal component. Here, however, we leave it unnormalised.}. 
The linear redshift is then given by
 \begin{equation}
 1+z=\frac{a_{\mathcal{O}}}{a_{\mathcal{S}}}\left[ 1 +\frac{1}{\bar{\ell}^0} \left( \left.  \delta \ell^{0}\right|_{\mathcal{S}} - \left.  \delta \ell^{0}\right|_{\mathcal{O}} \right) \right]\, ,
 \label{eq:linear_z}
 \end{equation}
 where $\delta \ell^{0}$ is founded from the first-order geodesic equation
 \begin{equation}
 \frac{d \delta \ell^{\mu}}{d \lambda}= \frac{1}{2}\bar{g}^{\mu \sigma} \partial_{\sigma} \delta g_{\alpha \beta} \bar{\ell}^{\alpha}\bar{\ell}^{\beta}-\bar{g}^{\mu \sigma}\bar{\ell}^{\alpha} \partial_{\alpha} \delta g_{\sigma \beta}\bar{\ell}^{\beta}\,.
 \label{eq:lin_geod}
 \end{equation}
 
The second observable is the angular diameter distance $\tilde{D}_{\rm ang}$. However, it is more convenient to expand the conformal angular distance $D_{\rm ang}$ which we write here in terms of the BGO as (for a derivation see \cite{Grasso:2018mei} and \cite{Korzynski:2019oal})
\begin{equation}
D_{\rm ang}=\left. \ell^{\mu}u_{\mu}\right|_{\mathcal{O}}|\mathrm{det}(\WXL{}\UD{\boldsymbol{A}}{\boldsymbol{B}})|^{\frac{1}{2}}
\label{eq:dang_def_app_lin}
\end{equation}
and then obtain $\tilde{D}_{\rm ang}$ from the very well-known conformal transformation $\tilde{D}_{\rm ang}= \frac{a}{a_{\mathcal{O}}} D_{\rm ang}$, that we verified for Eq. \eqref{eq:dang_def_app_lin}.
The BGO $\WXL{}\UD{\boldsymbol{A}}{\boldsymbol{B}}$ are expanded as
\begin{equation}
\WXL{}\UD{\boldsymbol{A}}{\boldsymbol{B}}=\overline{\WXL}{}\UD{\boldsymbol{A}}{\boldsymbol{B}}+ \delta \WXL{}\UD{\boldsymbol{A}}{\boldsymbol{B}}\,,
\end{equation}
where $\overline{\WXL}$ and $\delta \WXL$ are found by solving the linearised GDE \eqref{eq:GDE_for_BGO} in conformal space
\begin{equation}
\frac{d}{d \lambda} \mathcal{W}=\begin{pmatrix}
0 & \mathbb{1}_{4 \times 4}\\
R_{\ell \ell} & 0
\end{pmatrix} \mathcal{W}
\end{equation} 
Notice that the optical tidal matrix in the frame is purely a first-order quantity - the conformal Riemann tensor vanishes in the background - and it is given by:
\begin{equation}
R\UDDD{\boldsymbol{\mu}}{\ell}{\ell}{\boldsymbol{\nu}}=\bar{\phi}^{\rho \boldsymbol{\mu}} \delta R_{\rho \alpha \beta \sigma}\bar{\ell}^{\alpha} \bar{\ell}^{\beta}
 \bar{\phi}\UD{\sigma}{\boldsymbol{\nu}}
\end{equation}
where $\delta R_{\rho \alpha \beta \sigma}$ is the first-order Riemann tensor and $\bar{\phi}\UD{\mu}{\boldsymbol{\alpha}}=(u^{\mu}, \bar{\phi}\UD{\mu}{\boldsymbol{A}}, \bar{\ell}^{\mu})$ is the background parallel transported frame along the background geodesic.

Let us start by solving the background GDE, which reads:
\begin{equation}
\left\{\begin{array}{l}
\frac{d \, \overline{\WXX}{}\UD{\boldsymbol{\mu}}{\boldsymbol{\nu}}}{d \lambda}=\overline{\WLX}{}\UD{\boldsymbol{\mu}}{\boldsymbol{\nu}}\\
\frac{d \, \overline{\WLX}{}\UD{\boldsymbol{\mu}}{\boldsymbol{\nu}}}{d \lambda}=0\\
\frac{d \, \overline{\WXL}{}\UD{\boldsymbol{\mu}}{\boldsymbol{\nu}}}{d \lambda}= \overline{\WLL}{}\UD{\boldsymbol{\mu}}{\boldsymbol{\nu}}\\
\frac{d \, \overline{\WLL}{}\UD{\boldsymbol{\mu}}{\boldsymbol{\nu}}}{d \lambda}=0
\end{array}\right.
\label{eq:background_BGO_syst}
\end{equation}
with initial conditions $\overline{\mathcal{W}}=\mathbb{1}_{8 \times 8}$. The solution is
\begin{equation}
\overline{\mathcal{W}}=\begin{pmatrix}
\delta\UD{\boldsymbol{\mu}}{\boldsymbol{\nu}} & (\lambda - \lambda_{\mathcal{O}})\delta\UD{\boldsymbol{\mu}}{\boldsymbol{\nu}}\\
0 & \delta\UD{\boldsymbol{\mu}}{\boldsymbol{\nu}}
\end{pmatrix}
\label{eq:BGO_background}
\end{equation}
Next we find the first-order BGO from:
\begin{equation}
\left\{\begin{array}{l}
\frac{d \, \delta \WXX{}\UD{\boldsymbol{\mu}}{\boldsymbol{\nu}}}{d \lambda}=\delta \WLX{}\UD{\boldsymbol{\mu}}{\boldsymbol{\nu}}\\
\frac{d \, \delta \WLX{}\UD{\boldsymbol{\mu}}{\boldsymbol{\nu}}}{d \lambda}=R\UDDD{\boldsymbol{\mu}}{\ell}{ \ell}{ \boldsymbol{\nu}} \\
\frac{d \, \delta \WXL{}\UD{\boldsymbol{\mu}}{\boldsymbol{\nu}}}{d \lambda}= \delta \WLL{}\UD{\boldsymbol{\mu}}{\boldsymbol{\nu}}\\
\frac{d \,  \delta \WLL {}\UD{\boldsymbol{\mu}}{\boldsymbol{\nu}}}{d \lambda}= (\lambda - \lambda_{\mathcal{O}})R\UDDD{\boldsymbol{\mu}}{\ell}{ \ell}{ \boldsymbol{\nu}} 
\end{array}\right.
\label{eq:1PT_BGO_syst}
\end{equation}
with initial conditions $\delta \mathcal{W}=\boldsymbol{0}_{8 \times 8}$, where we have replaced the background solutions \eqref{eq:BGO_background}.
We obtain
\begin{equation}
\left\{\begin{array}{l}
\delta \WXX{}\UD{\boldsymbol{\mu}}{ \boldsymbol{\nu}}=\int^{\lambda_{\mathcal{O}}}_{\lambda}\int^{\lambda_{\mathcal{O}}}_{\lambda'}R\UDDD{\boldsymbol{\mu}}{\ell}{ \ell}{ \boldsymbol{\nu}} d\lambda'd\lambda''\\
\delta \WXL{}\UD{\boldsymbol{\mu}}{ \boldsymbol{\nu}}= \int^{\lambda_{\mathcal{O}}}_{\lambda}(\lambda_{\mathcal{O}}-\lambda')(\lambda- \lambda')R\UDDD{\boldsymbol{\mu}}{\ell}{ \ell}{ \boldsymbol{\nu}} d\lambda'\\
\delta \WLX{}\UD{\boldsymbol{\mu}}{ \boldsymbol{\nu}}= -\int^{\lambda_{\mathcal{O}}}_{\lambda}R\UDDD{\boldsymbol{\mu}}{\ell}{ \ell}{ \boldsymbol{\nu}} d\lambda'  \\
\delta \WLL{}\UD{\boldsymbol{\mu}}{ \boldsymbol{\nu}}= \int^{\lambda_{\mathcal{O}}}_{\lambda}(\lambda_{\mathcal{O}}-\lambda')R\UDDD{\boldsymbol{\mu}}{\ell}{ \ell}{ \boldsymbol{\nu}} d\lambda'
\end{array}\right.\,.
\label{eq:1PT_BGO}
\end{equation}
In order to find $D_{\rm ang}^{\rm Lin}$ from the expansion of Eq.~\eqref{eq:dang_def_app_lin} we need the second of the above solutions and we recall that the expansion of the square root of the determinant is given by
\begin{equation}
\sqrt{\mathrm{det} W_{XL}}  = \sqrt{\mathrm{det} \left(\overline{W_{XL}}\right)}\left[ 1+\frac{1}{2}\mathrm{tr}\left(\overline{W_{XL}}^{-1} \delta W_{XL} \right) \right]
\end{equation}
Now, looking at Eqs.~\eqref{eq:BGO_background} and~\eqref{eq:1PT_BGO} we have that
\begin{equation}
\left\{\begin{array}{l}
\sqrt{\mathrm{det} \left(\overline{W_{XL}}\right)}=(\lambda-\lambda_{\cal O})\\
\mathrm{tr}\left(\overline{W_{XL}}^{-1} \delta W_{XL} \right)=\frac{\int^{\lambda_{\mathcal{O}}}_{\lambda}(\lambda_{\mathcal{O}}-\lambda')(\lambda- \lambda')R\UDDD{\boldsymbol{A}}{\ell}{ \ell}{ \boldsymbol{A}} d\lambda'}{(\lambda-\lambda_{\cal O})}
\end{array}\right.
\end{equation}
The final result for $D^{\rm Lin}_{\rm ang}$ is
\begin{equation}
\begin{array}{c}
D_{\rm ang}^{\rm Lin}=(\ell^{0}_{\mathcal{O}}+\delta \ell^0_{\mathcal{O}})(\lambda_{\mathcal{O}}-\lambda)\\
-\frac{\ell^{0}_{\mathcal{O}}}{2}\int^{\lambda_{\mathcal{O}}}_{\lambda}(\lambda_{\mathcal{O}}-\lambda')(\lambda- \lambda')\mathrm{tr}\left(R\UDDD{\boldsymbol{A}}{\ell}{ \ell}{ \boldsymbol{B}}\right) d\lambda'\,.
\end{array}
\label{eq:Dlin_lambda}
\end{equation}
It is important to stress that all the quantities are evaluated along the background geodesic, i.e. $\eta \equiv \bar{\eta}$ and $\bar{q}_{\rm 1}(\eta)\equiv \frac{\bar{\ell}^1}{\bar{\ell}^0}(\bar{\eta}_{\rm \mathcal{O}}-\bar{\eta})+\bar{q}_{\rm 1}(\bar{\eta}_{\rm \mathcal{O}})$

We checked that our result coincides with the standard result in the literature, e.g. \cite{di2016curvature}, by simply noting that the quantity $-\frac{1}{2}\mathrm{tr}\left(R\UDDD{\boldsymbol{A}}{\ell}{ \ell}{ \boldsymbol{B}}\right)$ is nothing more than the Ricci part of the optical tidal matrix $\mathcal{R}$, usually defined as
\begin{equation}
\mathcal{R}=\frac{1}{2}R_{\alpha \beta} \ell^{\alpha} \ell^{\beta}=-\frac{1}{2}R\UDDD{\mu}{\alpha}{\beta}{\mu} \ell^{\alpha} \ell^{\beta}\,,
\end{equation}
and evaluated at first order. 

We finally specialise the above results for our plane-parallel model. The linear perturbation of the spacetime metric around the flat $\Lambda$CDM background is
\begin{equation}
\delta g_{\mu \nu}=\begin{pmatrix}
0 & 0 & 0 & 0\\
0 & -\mathcal{F}-\frac{10}{3 c^2}\phi_{\rm 0} & 0 & 0\\
0 & 0 & -\frac{10}{3 c^2}\phi_{\rm 0} & 0\\
0& 0 & 0 & -\frac{10}{3 c^2}\phi_{\rm 0}
\end{pmatrix}\, ,
\label{eq:sync_gauge_quantities}
\end{equation}
where we define 
\begin{equation}
\mathcal{F}(\eta, q_{\rm 1})=\frac{4}{3}\frac{\partial^2_{\rm q_{\rm 1}}\phi_{\rm 0}(q_{\rm 1})}{\stuff}\mathcal{D}(\eta)
\end{equation}
A straightforward substitution gives for the redshift
 \begin{equation}
 \begin{array}{l}
 1+z^{\rm Lin}=\\
 \vspace*{0.5 cm} \frac{a_{\mathcal{O}}}{a_{\mathcal{S}}}\left[ 1 -\left(\frac{\bar{\ell}^1}{\bar{\ell}^0}\right)^2 \int^{\bar{\eta}_{\mathcal{O}}}_{\bar{\eta}_{\mathcal{S}}} \frac{2}{3}\frac{\partial_{\rm q_{\rm 1}} \phi_{\rm 0}(q_{\rm 1}(\bar{\eta}'))}{\stuff}\partial_{0}{\mathcal{D}}(\bar{\eta}')   d \bar{\eta}' \right]
 \end{array}
 \label{eq:zlin_plane-parallel_2}
 \end{equation}
and for the angular diameter distance
\begin{equation}
\tilde{D}^{\rm Lin}_{\rm ang}(\eta)=\frac{a}{a_{\mathcal{O}}}\left[(\bar{\eta}_{\rm \mathcal{O}}-\bar{\eta})+ \frac{\bar{\ell}^{1^2}}{2 \bar{\ell}^{0^2}}\int^{\eta_{\rm \mathcal{O}}}_{\eta} \int^{\eta_{\rm \mathcal{O}}}_{\eta'}  \dot{\mathcal{F}} d\eta'd\eta''+\int^{\eta_{\rm \mathcal{O}}}_{\eta}(\eta_{\rm \mathcal{O}}-\eta')(\eta-\eta') \frac{\mathcal{R}(\eta')}{\bar{\ell}^{0^2}}  d\eta'\right]
\label{eq:Dang_time_conf}
\end{equation}
The two last relations are those we use in section~\ref{sec:method} for our comparison.
\endinput
