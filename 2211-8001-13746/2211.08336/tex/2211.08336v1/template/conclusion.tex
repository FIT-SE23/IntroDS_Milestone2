\chapter{Summary}
\label{chap:conclusion}

This dissertation deals with the computation of optical observables in cosmological simulations using the new {\tt Wolfram} package {\tt BiGONLIght}.
Numerical simulations have become an increasingly important instrument in modern cosmology for reconstructing the Universe's large-scale structure. To test the validity of cosmological theories, it is essential to correctly simulate the interaction of light with these structures to determine the origin of nonlinear relativistic effects measured in observations. In the past, this has been done with various methods for gravitational lensing observations. With the possibility of more precise measurements on the one hand and the observation of new quantities on the other, a unique approach to light propagation is needed to keep pace with this revolution in observational cosmology.
The main feature of the package I have created is that it allows the direct implementation of the BGO formalism for computing multiple observables in a single calculation. This is possible because the BGO provide a unified framework for describing all possible optical effects caused by gravity on light propagation. Once computed along a geodesic, the BGO can be used to compute observables such as magnification, shear, and angular diameter distance, as well as new real-time observables such as parallax, redshift drift, and position drift resulting from temporal variations in the positions of sources and observers.


In the first paper, we introduce {\tt BiGONLIght} and show how it is applied to compute multiple observables in numerical relativity. In order to be compatible with most of the full-GR codes employed in numerical relativity, the package is designed to implement the BGO framework in the $3+1$ form. To this end, I express the parallel transport equation, the optical tidal matrix $R\UD{\bm{\mu}}{\ell \ell \bm{\nu}}$, and the evolution equation for the BGO in terms of the ADM quantities (Eqs.~($31$),~($39$), and~($44$) in \cite{Grasso:2021iwq}). These results, together with the transformations from forward to backward integrated BGO (see Eqs.~($49$)-($52$) in \cite{Grasso:2021iwq}), are my main theoretical contribution to this paper.
Together with the $3+1$ geodesic equation (presented in \cite{Vincent:2012kn}), these formulas are encoded in {\tt BiGONLIght} as {\tt Mathematica} functions.
%the expressions of these formulas are obtained by {\tt Mathematica} functions defined in {\tt BiGONLIght}. % and functions to solve numerically each of these ODEs. 
These functions take as input the ADM quantities (namely the spatial metric $\gamma_{i j}$, the extrinsic curvature $K_{i j}$, the lapse $\alpha$ and the shift $\beta^i$) and the $3+1$ components of the velocities and accelerations  of the observer $\calO$ and source $\calS$ to obtain the ODEs for computing the geodesics, to perform the parallel transport of a tetrad of vectors, and to compute the BGO. 
The user can provide the input as interpolated data from a numerical simulation or as analytical expressions of the components of the metric and the four-vectors (velocities and accelerations of $\calO$ and $\calS$). For this second case, I have included functions in the package that perform the $3+1$ splitting of the four-dimensional metric tensor and the four-vectors to obtain the ADM quantities. This hybrid design makes {\tt BiGONLIght} highly adaptable to study different types of problems in both numerical simulations and analytical (perturbation and/or exact) approaches. 

The solutions of the ODEs for the geodesics, parallel transport, and GDE for BGO are found numerically using {\tt BiGONLIght} functions that solve these ODEs within a certain numerical precision. The user sets the precision via the precision control options implemented in {\tt Mathematica}. 
The final output of {\tt BiGONLight} are the BGO $\mathcal{W}(\calS, \, \calO)$ computed along the geodesics from the observer $\calO$ to the source $\calS$. These can be used to obtain observables as described in Sec.~\ref{sec:observables}. Furthermore, the ready-to-use transformations from forward-integrated BGO $\mathcal{W}(\calO, \, \calS)$ to backward-integrated BGO $\mathcal{W}(\calS, \, \calO)$ make {\tt BiGONLight} potentially adaptable to perform light propagation on-the-fly with a simulation of spacetime, that is forward in time by construction. The package is currently designed to perform light propagation in post-processing rather than on-the-fly, with the advantage of processing inputs from a variety of numerical codes for cosmological dynamics.
The procedure for computing observables with {\tt BiGONLIght} is described in \cite{Grasso:2021iwq} and implemented in a set of example notebooks publicly available on the GitHub repository {\color{blue}{https://github.com/MicGrasso/bigonlight}} under the GPL-3.0 license. 
The release of {\tt BiGONLIght} is the main result of my research and the most important achievement of this thesis.


We test {\tt BiGONLIght} by computing observables in three different cosmological models: the two homogeneous $\Lambda$CDM and EdS models and the inhomogeneous Szekeres model. The spacetime metrics corresponding to the $\Lambda$CDM and the Szekeres model are provided analytically, while the metric for the EdS model is obtained from the numerical evolution of a homogeneous dust Universe performed with the {\tt Einstein Toolkit} and the {\tt FLRWSolver}.
In the homogeneous $\Lambda$CDM model, I have calculated the redshift, the angular diameter distance, the parallax distance, and the redshift drift using {\tt BiGONLIght} and I have compared these results with those obtained using analytical expressions from the literature. This test shows that {\tt BiGONLIght} can accurately reproduce the analytical results with a relative difference of the order of $10^{-22} \div 10^{-31}$.
In the inhomogeneous Szekeres model, the metric presented in \cite{Meures:2011ke, Meures:2011gp} is provided as an analytical input to {\tt BiGONLIght} to compute the redshift and the angular diameter distance. I compare these results with those obtained by numerically solving the equations derived in \cite{Meures:2011gp}. I also test the calculation of the redshift drift using {\tt BiGONLIght}. In this case, the comparison is made with the formula I derived for the specific structure of light propagation in the Szekeres model under consideration (Eq.~($96$) in \cite{Grasso:2021iwq}). %The relative difference of the observables obtained with {\tt BiGONLIght} and with other methods
These other code tests in the Szekeres metric also show a good agreement between the observables obtained with {\tt BiGONLIght} and those obtained with other methods, with a relative difference of the order of $\sim 10^{-22}$.
The last group of code tests is performed considering the homogeneous EdS model evolved numerically with the {\tt Einstein Toolkit} and {\tt FLRWSolver}. The data resulting from this simulation are passed as numerical input to {\tt BiGONLIght} and used to calculate the observables. These tests also differ from the other two groups, because in this case I have performed light propagation forward in time, namely from the source $\calS$ to the observer $\calO$, and obtained the forward integrated BGO $\mathcal{W}(\calO, \calS)$. Then I use the transformations in Eqs.~($49$)-($52$) in the paper \cite{Grasso:2021iwq} to obtain $\mathcal{W}(\calS, \calO)$ and compute observables. The tests on the observables (redshift, angular diameter distance, parallax distance, and redshift drift) show that the precision of the numerical simulation of the spacetime, which in our case is $\sim 10^{-10}$, determines the accuracy of the observables computed with {\tt BiGONLIght}. The very same results are obtained if $\mathcal{W}(\calS, \calO)$ are directly computed, as was done for the tests in $\Lambda$CDM and Szekeres models.

In the second paper, we show how {\tt BiGONLight} can be used to isolate and quantify various nonlinear contributions to light propagation. The analysis is performed in a toy model of the Universe, where the inhomogeneities in the density fluctuations form a sequence of plane-symmetric perturbations around a homogeneous $\Lambda$CDM background. The nonlinearities of light propagation are quantified by considering the relative difference of observables, defined as $\Delta O({\rm b, a})=(O^{\rm b}-O^{\rm a})/O^{\rm a}$, where ${\rm a,\, b}={\rm Lin, \, N,\, PN}$ denotes the following three different approximations: linear observables $O^{\rm Lin}$, obtained using standard first-order perturbation theory, Newtonian observables $O^{\rm N}$, obtained using the Newtonian approximation of the plane-parallel metric as the analytical input to {\tt BiGONLight}, and post-Newtonian observables $O^{\rm PN}$, obtained using the post-Newtonian approximation of the plane-parallel metric as the analytical input to {\tt BiGONLight}. The expressions of the plane-parallel metric in Newtonian and PN approximations are given in \cite{Villa:2011vt}, and we extended them providing the corresponding metrics with a $\Lambda$CDM background. These metrics fit our analysis well, as the terms from all three approximations are easily identifiable and can be used directly as input in the package to compute observables. For our analysis, we only consider the redshift $z$ and the angular diameter distance $D_{\rm ang}$ computed in the three different methods. After a preliminary analysis, we decided to fix the setting for light propagation as discussed in Sec.~IV in \cite{Grasso:2021zra}. 

The variations $\Delta O({\rm Lin, N})$ and $\Delta O({\rm PN, N})$ are calculated by varying the free parameters of the model. The gravitational potential $\phi_0(q^{\rm 1})$ is a free function that provides the spatial profile of the matter distribution. We consider a sinusoidal distribution $\phi_0(q^{\rm 1})=\mathcal{I} \sin(\frac{2 \pi}{k}q^{\rm 1})$, where $\mathcal{I}$ and $k$ denote the amplitude and scale of the inhomogeneities, respectively. We vary $(k, \,\mathcal{I})$ as described in Sec.~V and according to the list of values in Table~$1$ in \cite{Grasso:2021zra}. Another free parameter of the model is $a_{\rm nl}$, which is related to the primordial non-Gaussianity parameter $f_{\rm nl}$. It expresses the deviations from a Gaussian distribution of the primordial fluctuations and an estimate of its value is $a_{\rm nl}=0.46 \pm 3.06$, see \cite{planck2019anl}. In the PN metric \cite{Villa:2011vt}, a specific value of $a_{\rm nl}$ can modulate the effects of some of the PN terms. We compute $O^{\rm PN}$ for the four values of $a_{\rm nl}=0.46,\, 3.52,\,-2.6,\,1 $, corresponding to the reference value, the two extremes of the confidence interval, and for perfect Gaussian distribution. The other (cosmological) parameters are set using the values measured by Planck satellite \cite{planck2018param}.
We isolate the various sources of nonlinear corrections in the observables by analysing the dependence of $\Delta O$ on these freely specifiable quantities.

My original contributions to this paper are all simulations in the three approximations with {\tt BiGONLight} and calculations of $\Delta O$, for the various choices of the free parameters $(\mathcal{I}, \, k, \, a_{\rm nl})$.
These results have led to the following findings:%\pagebreak
%%%%%%%%%%%%%%%%
\begin{enumerate}[(i)]
\item We quantify the nonlinear corrections in the observables from Newtonian and PN approximations by computing $\Delta O({\rm Lin, N})$ and $\Delta O({\rm PN, N})$. In general, our results are consistent with similar results in the literature, as $\Delta z({\rm Lin, N})$ and $\Delta D_{\rm ang}({\rm Lin, N})$ are well below $1\%$. However, we note a different behaviour in the two observables. For the redshift, the Newtonian corrections contribute most to the nonlinearities, with $\Delta z({\rm Lin, N})\sim 10^{2}\Delta z({\rm PN, N})$. On the other hand, for the angular diameter distance we find that $\Delta D_{\rm ang}({\rm Lin, N})\sim \Delta D_{\rm ang}({\rm PN, N})$, i.e. $D_{\rm ang}^{\rm Lin}$ and $D_{\rm ang}^{\rm PN}$ have similar corrections with respect to $D_{\rm ang}^{\rm N}$.
\item We estimate the effects of the scale of perturbations $k$ by computing $\Delta O$ for different values of $k$. The observables at linear, Newtonian, and PN order are computed for $k=500\, {\rm Mpc}, \,300\, {\rm Mpc}, \,100\, {\rm Mpc}, \,50\, {\rm Mpc}, \,300\, {\rm Mpc}$. We find that the amplitude of $\Delta D_{\rm ang}$ decreases monotonically with the scale for both linear-Newtonian and PN-Newtonian comparisons. On the other hand, the amplitude for $\Delta z$ increases for $500 \, {\rm Mpc} \leq k \leq 100 \, {\rm Mpc}$ and decreases  for $100 \, {\rm Mpc} < k \leq 30  \, {\rm Mpc}$, with a maximum amplitude of $\Delta z$ for $k=100 \, {\rm Mpc}$.
\item We constrain the dependence on primordial non-Gaussianity in $\Delta O ({\rm PN, N})$ and \\
$\Delta O^{\rm PN}(a_{\rm nl_1}, a_{\rm nl_2})$ with $O^{\rm PN}$ computed with the four different values of $a_{\rm nl}$. In both cases, we found that the primordial non-Gaussianity parameter has negligible effects in our comparison, i.e. $\Delta O^{\rm PN}(a_{\rm nl_1}, a_{\rm nl_2}) \ll \Delta O ({\rm PN, N})$.
\item Finally, we examine the relative difference $\Delta O({\rm a, \Lambda CDM })$ of the observables in the three approximations ${\rm a} = {\rm Lin, \, N, \, PN}$ with respect to the observable for the $\Lambda$CDM background. This comparison for the angular diameter distance shows that the leading contribution to the PN corrections is the linear PN term $ - \frac{5}{3 c^2} \phi_0$, known as ``initial seeds''.
\end{enumerate}
%%%%%%%%%%%%%%%
The results in (iv) motivate the findings in (i) and (iii). In (i), the result $\Delta D_{\rm ang}({\rm Lin, N})\sim \Delta D_{\rm ang}({\rm PN, N})$ is related to the fact that the initial seed is only present in the linear metric and the PN metric (as the leading correction), but is absent in the Newtonian metric. On the other hand, in (iii), the dependence on $a_{\rm nl}$ is negligible since the primordial non-Gaussian parameter can only trigger the nonlinear PN terms, and these are subleading with respect to the linear PN initial seeds.

In conclusion, I present {\tt BiGONLight} in my thesis as a reliable numerical tool that can be easily adapted to perform numerical and analytical analyses of light propagation by computing multiple observables within a single calculation. It can be used to perform complex analyses, such as the one presented, with accuracy sufficient to constrain small nonlinear effects, such as those caused by nonlinear PN terms. 

%Future works are devoted to enlarge the regime of applicability of the package, by implementing the possibility of computing observables at different orders in perturbations 
\newpage
\endinput

\begin{enumerate}[(i)]
\item We quantify the nonlinear corrections in the observables from Newtonian and PN approximations by computing $\Delta O({\rm Lin, N})$ and $\Delta O({\rm PN, N})$. In general, our results are consistent with similar results in the literature, as $\Delta z({\rm Lin, N})$ and $\Delta D_{\rm ang}({\rm Lin, N})$ are well below $1\%$. However, we resolve a different behaviour for the two observables. For the redshift, the Newtonian corrections contribute most to the nonlinearities, with $\Delta z({\rm Lin, N})\sim 10^{2}\Delta z({\rm PN, N})$. Instead, for the angular diameter distance we have that $\Delta D_{\rm ang}({\rm Lin, N})\sim \Delta D_{\rm ang}({\rm PN, N})$, i.e. $D_{\rm ang}^{\rm Lin}$ and $D_{\rm ang}^{\rm PN}$ have similar corrections with respect to $D_{\rm ang}^{\rm N}$.
\item We estimate the effects of the scale of perturbations $k$ by computing $\Delta O$ for different values of $k$. The observables at linear, Newtonian, and PN order are computed for $k=500\, {\rm Mpc}, \,300\, {\rm Mpc}, \,100\, {\rm Mpc}, \,50\, {\rm Mpc}, \,300\, {\rm Mpc}, \,$. We find that the amplitude of $\Delta D_{\rm ang}$ decreases monotonically with the scale for both linear-Newtonian and PN-Newtonian comparisons. On the other hand, the amplitude for $\Delta z$ increases for $500 \, {\rm Mpc} \leq k \leq 100 \, {\rm Mpc}$ and decreases  for $100 \, {\rm Mpc} < k \leq 30  \, {\rm Mpc}$, with a maximum amplitude of $\Delta z$ for $k=100 \, {\rm Mpc}$.
\item We constrain the dependence on primordial non-Gaussianity in $\Delta O ({\rm PN, N})$ and \\
$\Delta O^{\rm PN}(a_{\rm nl_1}, a_{\rm nl_2})$ with $O^{\rm PN}$ computed with the four different values of $a_{\rm nl}$. In both cases, we found that the primordial non-Gaussianity parameter has negligible effects in our comparison, i.e. $\Delta O^{\rm PN}(a_{\rm nl_1}, a_{\rm nl_2}) \ll \Delta O ({\rm PN, N})$.
\item Finally, we examine the relative difference $\Delta O({\rm a, \Lambda CDM })$ of the observables in the three approximations ${\rm a} = {\rm Lin, \, N, \, PN}$ with respect to the observable for the $\Lambda$CDM background. This comparison for the angular diameter distance shows that the leading contribution to the PN corrections are the linear PN initial conditions $ - \frac{5}{3 c^2} \phi_0$, known as ``initial seeds''.
\end{enumerate}