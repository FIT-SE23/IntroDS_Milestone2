%\begin{center}
% \textbf{\Large{Streszczenie}}\\
\section*{Streszczenie} 
 \textbf{Polish translation of the abstract}\\
%\end{center}
\addcontentsline{toc}{chapter}{Streszczenie}
%\flushleft

Wraz z początkiem kosmologii precyzyjnej, przewidywania teoretyczne powinny zbliżać się do podobnego poziomu precyzji jak eksperymenty. W tym kontekście symulacje numeryczne uwzględniające poprawki związane z ogólną teorią względności stanowią najlepszą obecnie metodę opisu formowania struktury. Jednakże, oprócz dokładnego opisu dynamiki, niezbędne jest także równie dokładne wyjaśnienie oddziaływania tych struktur na propagację światła i precyzyjne modelowanie ich wpływu na wielkości mierzalne. 

Badanie wszyskich relatywistycznych efektów w najbardziej ogólnym sformułowaniu wy\-ma\-ga jednolitego podejścia do problemu propagacji propagacji światła w kosmologii. Można to osiągnąć dzięki nowej interpretacji równania dewiacji geodezyjnych w języku bilokalnych operatorów geodezyjnych (bilocal geodesic operators, BGO). Formalizm BGO jest rozszerzeniem standardowego opisu, wprowadzającym jednolity opis różnych zjawisk optycznych związanych z oddziaływaniem krzywizny czasoprzestrzeni na światło.

W tej rozprawie prezentuję {\tt BiGONLight}, pakiet w języku {\tt Mathematica} implementujący formalizm BGO do badania propagacji światła w numerycznej ogólnej teorii względności. Pakiet implementuje formalizm BGO dla danych w rozkładzie 3+1 jako kolekcję funkcji języka {\tt Mathematica}. Dane wejściowe stanowią metryka czasoprzestrzeni oraz kinematyka obserwatora i źródła, oba w rozkładzie 3+1, które mogą pochodzić bezpośrednio z numerycznej symulacji albo zostać dostarczone bezpośrednio przez użytkownika jako funkcje w jawnej postaci. Dane te służą do śledzenia promieni światła (ray-tracing) i obliczenia bilokalnych operatorów geodezyjnych w najogólniejszy możliwy sposób, bez korzystania z symetrii czasoprzestrzeni lub własności układu współrzędnych. Głównym zadaniem pa\-kie\-tu jest obliczanie obserwabli optycznych w dowolnej czasoprzestrzeni. Formalizm BGO pozwala na wyznaczenie wszystkich obserwabli podczas jednego obliczenia, a język {\tt Wolfram} dostarcza narzędzi numerycznych, dzięki czemu pakiet łatwo nadaje się do zarówno numerycznych, jak i analitycznych badań nad propagacją światła. Pakiet {\tt BiGONLight} został przetestowany przez obliczenie przesunięcia ku czerwieni, odległości kątowej, odległości paralaktycznej i dryfu przesunięcia ku czerwieni w prostych modelach kosmologicznych. Badania przeprowadzamy dla trzech przykładów: dwóch metryk podanych analitycznie, tzn. modelu jednorodnego $\Lambda$CDM bez perturbacji i niejednorodnego modelu z klasy Szekeresa, oraz dla symulowanego wszechświata z pyłem w rozkładzie 3+1. Testy pokazały bardzo dobrą zgodność z analitycznymi wzorami.

Dzięki wyżej wymienionym cechom pakiet {\tt BiGONLight} jest dobrym narzędziem do badania wpływu niejednorodności na propagację światła. Zbadaliśmy źródła efektów relatywistycznych w propagacji światła spowodowane przez niejednorodne struktury. Pakiet {\tt BiGONLight} został użyty do obliczenia obserwabli obliczonych w różnych przybliżeniach w niejednorodnym modelu typu plane-parallel. Efekty nieliniowe obliczone zostały jako względne różnice między trzema przybliżeniami: liniowy rachunkiem perturbacyjnym, przybliżeniem new\-to\-now\-skim oraz post-newtonowskim. Wielkość niejednorodności można regulować zmieniając wolne parametry modelu, a wpływ tych parametrów na obserwable otrzymany został przez a\-na\-li\-zę zmian względnych różnic. W ten sposób szacujemy wielkość newtonowskich oraz post-newtonowskich poprawek do liniowego rzędu rachunku zaburzeń i badamy jak zmieniają się one przy zmianie rozmiaru i amplitudy niejednorodności. Wyjaśniamy także rolę początkowej perturbacji metryki przez potencjał skalarny i pokazujemy, że nieliniowe poprawki są mniejsze niż $1\%$. Wyniki te są zgodne z wcześniejszymi badaniami na ten temat.

\endinput

