% Chapter Template
%\part{Study of parameter space for shock formation and oscillation}


\chapter{Paper I: ``{\tt BiGONLight}: light propagation with bilocal operators in Numerical Relativity''} % Main chapter title
\label{chap:bigonlight}

The chapter presents {\tt BiGONLight}, a {\tt Wolfram} package designed to implement the BGO framework in $3 + 1$ form to compute optical observables from numerically generated spacetimes. The package is completely general: it can simulate light propagation in geometric optics approximation in any spacetime, with no assumptions on the gauge or coordinate system used. 
It was specifically designed to be compatible with the full-GR codes in numerical relativity based on the ADM formalism. Nevertheless, it also takes advantage of Mathematica's symbolic algebra manipulation to compute the BGO from the analytical expression of the metric tensor.
The output of the package are the BGO, which are then combined with the observer $\calO$ and source $\calS$ four-velocities ($u^{\mu}_{\calO}$, $u^{\mu}_{\calS}$) and four-accelerations ($w^{\mu}_{\calO}$, $w^{\mu}_{\calS}$) to compute observables.

%{\tt BiGONLight} is introduced together with the theoretical framework of the BGO in the $3+1$ form and its implementation in the package.
%This work is conceptually divided into two parts. In the first part, {\tt BiGONLight} is introduced, presenting the $3+1$ formulation of the BGO framework and its implementation in the package. In the second part, the code is tested by calculating observables in three well-known cosmological models: the $\Lambda$CDM and the Szekeres (analytical) spacetimes, and a dust Universe obtained from numerical simulation.
This work is conceptually divided into two parts. In the first part, {\tt BiGONLight} is presented together with the theoretical formulation of the BGO framework in  $3+1$ form. In the second part, the code is tested by calculating observables in three well-known cosmological models: the $\Lambda$CDM and the Szekeres (analytical) spacetimes, and a dust Universe obtained from numerical simulation.\\

\textit{\textbf{Author's contribution}}\\
\newline
%{\tt BiGONLight} provides a unified approach to extract all possible observables within the same computation. 
{\tt BiGONLight} is my original contribution in \cite{Grasso:2021iwq} and my main achievement in this thesis. On the $13^{th}$ of July 2021, I released the stable version of the package (v1.0), which is publicly available on the GitHub repository {\color{blue}{https://github.com/MicGrasso/bigonlight}} under the GPL-3.0 license.
The {\tt BiGONLight} package is a collection of {\tt Wolfram} functions that can be used in a {\tt Mathematica} notebook to compute observables.
The procedure to compute observables with {\tt BiGONLight} can be summarised as follows:
\begin{enumerate}[(i)]
	\item the user provides the metric $g_{\mu \nu}$, and the observer $\calO$ and source $\calS$ kinematics as input. They can be given already in $3+1$ decomposition or as four-dimensional quantities. In the last instance, the user can use the functions {\tt ADM[]} and {\tt Vsplit[]} to perform the $3+1$ decomposition of $g_{\mu \nu}$ and of the vectors ($u^{\mu}$, $w^{\mu}$), respectively;
	\item set the initial conditions ($x^{\mu}_{\calO}$, $\ell^{\mu}_{\calO}$) for the photon's geodesic, provided as $3+1$ components or using {\tt Vsplit[]} and {\tt InitialConditions[]};
	\item obtain the expression of the geodesic equation in $3+1$ with the functions {\tt GeodesicEquations[]} and {\tt EnergyEquations[]}. The functions implement the $3+1$ geodesic equations obtained by Vincent et. al. in \cite{Vincent:2012kn};
	\item solve the geodesic equations numerically with {\tt SolveGeodesic[]} and {\tt SolveEnergy[]};
	\item set the initial conditions for the SNF, directly in ADM components or using {\tt SNF[]}. Then the function {\tt PTransportedFrame[]} computes the parallel transported SNF along the geodesic by solving the $3+1$ parallel transport equations Eq.~($31$) in \cite{Grasso:2021iwq};
	\item compute the optical tidal matrix projected into the SNF with the {\tt OpticalTidalMatrix[]} function, as expressed in Eq.~($39$) in \cite{Grasso:2021iwq};
	\item compute the expressions of the evolution equations for the BGO with {\tt BGOequations[]}, as in Eq.~($44$) in \cite{Grasso:2021iwq}, and solve them with {\tt SolveBGO[]} for obtaining $\mathcal{W}(\calS, \calO)$.
\end{enumerate} 
Step (vii) is the starting point for obtaining the observables by combining the BGO with the observer and source four-velocities and four-accelerations as discussed in Sec.~\ref{sec:observables}.

Other of my original contributions in \cite{Grasso:2021iwq} are the expressions of the parallel transport equation and the optical tidal matrix in terms of ADM quantities (Eq.~($31$) and Eq.~($39$) in \cite{Grasso:2021iwq}), and the explicit transformation relations from forward to backward integrated BGO (Eqs.~($49$)-($52$) in \cite{Grasso:2021iwq}). I also obtained the specific form of the optical tidal matrix and the BGO in the SNF, and I used these expressions to simplify the calculations of their components.

I have also performed the tests computing the redshift, the angular diameter distance, the parallax distance, and the redshift drift for two analytical spacetimes corresponding to the $\Lambda$CDM and the (axially-symmetric) Szekeres models (presented in \cite{Meures:2011ke}), and for a numerically evolved dust Universe (EdS). The tests were designed jointly by E. Villa and me. I have performed the numerical calculations of the observables with {\tt BiGONLight} and the comparisons with their analytical expressions in the $\Lambda$CDM and EdS models (Fig.~$2$ and Figs.~$4$-$5$ in \cite{Grasso:2021iwq}). The numerical evolution of the dust Universe was done by me with the {\tt Einstein Toolkit} and the {\tt FLRWSolver}, \cite{loffler2012einstein, macpherson2017}.

In the axially-symmetric Szekeres model we were considering, there are no analytical expressions for the observables: the redshift and the angular diameter distance can be obtained numerically as shown in \cite{Meures:2011gp} and the results compared with those obtained with {\tt BiGONLight} (Fig.~$3$ in \cite{Grasso:2021iwq}). Conversely, there was no known way to calculate the redshift drift. To this end, I derived an ODE whose solution gives the redshift drift for a geodesic along the axis of symmetry of the model (Eq.~($96$) in \cite{Grasso:2021iwq}). Finally, I have compared the redshift drift obtained from this result with the one obtained with the package. 
%Note that, we decided not to include the parallax distance in the tests for the Szekeres model, because its forecasted measure for inhomogeneous models are below the instrumental precision.
The results were discussed jointly by E. Villa and me, and published in \cite{Grasso:2021iwq}.





% and with arbitrary precision, thanks of Mathematica's accuracy and precision control features.
\includepdf[pages=-]{paper/bigonlight_pub.pdf}
%bigonlight_accepted.pdf
