% ****** Start of file apssamp.tex ******
%
%   This file is part of the APS files in the REVTeX 4.2 distribution.
%   Version 4.2a of REVTeX, December 2014
%
%   Copyright (c) 2014 The American Physical Society.
%
%   See the REVTeX 4 README file for restrictions and more information.
%
% TeX'ing this file requires that you have AMS-LaTeX 2.0 installed
% as well as the rest of the prerequisites for REVTeX 4.2
%
% See the REVTeX 4 README file
% It also requires running BibTeX. The commands are as follows:
%
%  1)  latex apssamp.tex
%  2)  bibtex apssamp
%  3)  latex apssamp.tex
%  4)  latex apssamp.tex
%
\documentclass[%
 reprint,
superscriptaddress,
%groupedaddress,
%unsortedaddress,
%runinaddress,
%frontmatterverbose, 
% preprint,
%preprintnumbers,
%nofootinbib,
% nobibnotes,
%bibnotes,
 amsmath,amssymb,
 aps,
%pra,
prb,
%rmp,
%prstab,
%prstper,
%floatfix,
]{revtex4-2}

\usepackage{color}
\usepackage{graphicx}% Include figure files
\usepackage{dcolumn}% Align table columns on decimal point
\usepackage{bm}% bold math
\usepackage{hyperref}% add hypertext capabilities
\usepackage[mathlines]{lineno}% Enable numbering of text and display math
%\linenumbers\relax % Commence numbering lines

%\usepackage[showframe,%Uncomment any one of the following lines to test 
%%scale=0.7, marginratio={1:1, 2:3}, ignoreall,% default settings
%%text={7in,10in},centering,
%%margin=1.5in,
%%total={6.5in,8.75in}, top=1.2in, left=0.9in, includefoot,
%%height=10in,a5paper,hmargin={3cm,0.8in},
%]{geometry}

\begin{document}

% \preprint{APS/123-QED}

\title{Intrinsic high-temperature quantum anomalous Hall effect and tunable magnetic topological phases in \textit{XY}Bi${}_2$Te${}_5$}% Force line breaks with \\

\author{Xin-Yi Tang}
\affiliation{%
 State Key Laboratory of Low Dimensional Quantum Physics, Department of Physics, Tsinghua University, Beijing 100084, China
}%

\author{Zhe Li}%
 \email{lizhe21@iphy.ac.cn}
\affiliation{%
 Bejing National Research Center for Condensed Matter Physics, and Institute of Physics, Chinese Academy of Sciences, Beijing 100190, China
}%


\author{Feng Xue}
\affiliation{%
 State Key Laboratory of Low Dimensional Quantum Physics, Department of Physics, Tsinghua University, Beijing 100084, China
}%
\affiliation{%
 Bejing Institute of Quantum Information Science, Beijing 100193, China
}%

\author{Pengfei Ji}
\affiliation{%
 State Key Laboratory of Low Dimensional Quantum Physics, Department of Physics, Tsinghua University, Beijing 100084, China
}%

\author{Zetao Zhang}
\affiliation{%
 State Key Laboratory of Low Dimensional Quantum Physics, Department of Physics, Tsinghua University, Beijing 100084, China
}%

\author{Xiao Feng}
\affiliation{%
 State Key Laboratory of Low Dimensional Quantum Physics, Department of Physics, Tsinghua University, Beijing 100084, China
}%
\affiliation{%
 Bejing Institute of Quantum Information Science, Beijing 100193, China
}%
\affiliation{%
 Frontier Science Center for Quantum Information, Beijing 100084, China
}%

\author{Yong Xu}
\affiliation{%
 State Key Laboratory of Low Dimensional Quantum Physics, Department of Physics, Tsinghua University, Beijing 100084, China
}%
\affiliation{%
 Frontier Science Center for Quantum Information, Beijing 100084, China
}%
\affiliation{%
Tencent Quantum Laboratory, Tencent, Shenzhen, Guangdong 518057, China
}%
\affiliation{%
RIKEN Center for Emergent Matter Science (CEMS), Wako, Saitama 351-0198, Japan
}%

\author{Quansheng Wu}
\affiliation{%
 Bejing National Research Center for Condensed Matter Physics, and Institute of Physics, Chinese Academy of Sciences, Beijing 100190, China
}%
\affiliation{%
 University of Chinese Academy of Sciences, Beijing 100049, China
}%

\author{Ke He}
 \email{kehe@tsinghua.edu.cn}
\affiliation{%
 State Key Laboratory of Low Dimensional Quantum Physics, Department of Physics, Tsinghua University, Beijing 100084, China
}%
\affiliation{%
 Bejing Institute of Quantum Information Science, Beijing 100193, China
}%
\affiliation{%
 Frontier Science Center for Quantum Information, Beijing 100084, China
}%

\date{\today}% It is always \today, today,
             %  but any date may be explicitly specified

\begin{abstract}
By first-principles calculations, we study the magnetic and topological properties of \textit{XY}Bi${}_2$Te${}_5$-family (\textit{X}, \textit{Y}=Mn, Ni, V, Eu) compounds. The strongly coupled double magnetic atom-layers can significantly enhance the magnetic ordering temperature while keeping the topologically non-trivial properties. Particularly, NiVBi${}_2$Te${}_5$ is found to be a magnetic Weyl semimetal in bulk and a Chern insulator in thin film with both the Curie temperature and magnetic gap well above 77 K. Ni${}_2$Bi${}_2$Te${}_5$ exhibits a dynamic axion state above 77 K with a magnetic ordering temperature up to room temperature. These results indicate an approach to realize high temperature QAHE and other topological quantum effects for practical applications.
\end{abstract}

\maketitle
% introduction
In the past few decades, topological states of materials has grown into a large branch of modern condensed matter physics \cite{hasan-RevModPhys.82.3045,qi-RevModPhys.83.1057,wsm-rev-RevModPhys.90.015001,Tokura2019}. Emergent topological phases, especially the ones connected with magnetism, like quantum anomalous Hall effect (QAHE) \cite{haldane-PhysRevLett.61.2015,yurui-doi:10.1126/science.1187485,qahe-doi:10.1126/science.1234414,apl-doi:10.1063/1.4935075,Chang2015}, axion insulators (AxIs) \cite{tft-PhysRevB.78.195424,daf-Li2010,wj-axi-PhysRevB.92.081107,nagaosa-axi-PhysRevB.92.085113,tokura-axi-Mogi2017,chang-axi-PhysRevLett.120.056801} and magnetic Weyl semimetals (WSMs) \cite{wan-wsm-PhysRevB.83.205101,Co3Sn2S2-doi:10.1126/science.aav2873,Co3Sn2S2-Liu2018}, have attracted lots of attention. QAHE has been experimentally achieved in Cr-doped topological insulator thin films initially in 2013 \cite{qahe-doi:10.1126/science.1234414}. However, the reliance on extremely low temperature hindered its applications. Many materials that may realize high temperature QAH insulators, magnetic WSMs or AxIs have been proposed, such as Co${}_3$Sn${}_2$S${}_2$ \cite{Co3Sn2S2-doi:10.1126/science.aav2873,Co3Sn2S2-Liu2018}, EuIn${}_2$As${}_2$ \cite{euin2as2-PhysRevLett.122.256402}, LiFeSe \cite{lifese-PhysRevLett.125.086401}, PdBr${}_3$-family materials \cite{pdbr3-PhysRevApplied.12.024063}, NiAsO${}_3$ and PdSbO${}_3$ \cite{qiao-PhysRevLett.129.036801}. Although some of them have been prepared and well-studied, quantized transport properties—the crucial mile stone—have not been achieved. Thin films of the intrinsic magnetic topological insulator MnBi${}_2$Te${}_4$ \cite{mbt-exp-Yan,mbt-exp-Otrokov2019} are a high temperature QAH system that has shown quantized transport properties in experiment. The magnetically induced surface state gap (briefly, magnetic gap) of MnBi${}_2$Te${}_4$ thin films can reach $\sim$50 meV \cite{ljh-maggap-PhysRevB.100.121103}, more than enough to support QAHE above 77 K. Zero-field quantization of the anomalous Hall resistance have been observed in odd septule-layer (SL) MnBi${}_2$Te${}_4$ thin flakes at 1.4 K \cite{mbt-exp-zyb-doi:10.1126/science.aax8156}, higher than that in magnetically doped (Bi,Sb)${}_2$Te${}_3$. Amazingly, under high magnetic field, nearly quantized Hall resistance has been observed in some MnBi${}_2$Te${}_4$ thin flake samples at a temperature as high as $\sim$40 K \cite{nsr-10.1093/nsr/nwaa089}. Other magnetic topological phases such as magnetic WSM and AxI can be realized in MnBi${}_2$Te${}_4$ with certain magnetic configurations and thicknesses \cite{mbt-sciadv-doi:10.1126/sciadv.aaw5685,mbt-otrokov-PhysRevLett.122.107202,mbt-wj-PhysRevLett.122.206401,mbt-exp-Liu2020,nsr-10.1093/nsr/nwaa089}. Obviously, MnBi${}_2$Te${}_4$ provides a solid base for the exploration of high-temperature QAHE and other related topological quantum effects.

\begin{figure*}[t]
    \centering
    \includegraphics[width=0.8\linewidth]{Fig1.png}
    \caption{Structural, electronic and magnetic features of \textit{XY}Bi${}_2$Te${}_5$. (a) Side view of 2-NL \textit{XY}Bi${}_2$Te${}_5$ lattice structure. We choose \textit{X}=Ni and \textit{Y}=V as an example. (b) Phonon spectrum of monolayer NiVBi${}_2$Te${}_5$. (c) $C_V$-$T$ curve of all ten \textit{XY}Bi${}_2$Te${}_5$ combinations obtained by Monte Carlo simulations and (d) corresponding Néel/Curie temperatures. The spoiling temperature of liquid nitrogen is marked as red dashed line and ``BT" is short for ``Bi${}_2$Te${}_5$". (e) Partial density of states distribution (without SOC) and (f) band structure along high symmetry lines (with SOC) using monolayer MnEuBi${}_2$Te${}_5$ and NiVBi${}_2$Te${}_5$ as examples. Magnetic-coupling-related $d$-orbital distributions are highlighted. }
    \label{fig1}
\end{figure*}

% The temperature to achieve the QAHE is determined by both the magnetic ordering temperature and the magnetic gap size. Since the magnetic gap size of MnBi${}_2$Te${}_4$ thin films is as large as $\sim$50 meV, the bottleneck is its magnetic ordering temperature which is only $\sim$25 K. {\color{blue}{Such a low temperature originates not only from the small exchange energy but also from magnetic fluctuations due to the two-dimensional nature of its magnetism. The nearly quantized Hall resistance observed in MnBi${}_2$Te${}_4$ flakes at a temperature higher than its Néel temperature can be attributed to the suppression of the magnetic fluctuations under an external magnetic field, which elevates the effective magnetic ordering temperature.}} Without external magnetic field, the most straightforward way to suppress the magnetic fluctuations \cite{mermin-PhysRevLett.17.1133} is to make the magnetic layers a little thicker and thus more three-dimensional. On the other hand, only adding one more magnetic atomic layer in each MnBi${}_2$Te${}_4$ SL may not destroy the topologically non-trivial properties.

The temperature to achieve the QAHE is determined by both the magnetic ordering temperature and the magnetic gap size. Since the magnetic gap size of MnBi${}_2$Te${}_4$ thin films is as large as $\sim$50 meV, the bottleneck is its magnetic ordering temperature which is only $\sim$25 K. Such a low temperature originates not only from the weak magnetic coupling strength but also from magnetic fluctuations due to the two-dimensional nature of its magnetism \cite{mermin-PhysRevLett.17.1133}. A straightforward way to solve the two problems is to make the strongly coupled magnetic atomic layer (Mn) a little thicker while not thick enough to destroy the topological properties. Actually, the Curie temperature of magnetic thin films is significantly elevated as the thickness increases to even two atomic layers \cite{fewlayers-PhysRevB.49.3962}. 

In fact, there exists such a material: Mn${}_2$Bi${}_2$Te${}_5$ which can be considered as MnBi${}_2$Te${}_4$ with an additional MnTe bilayer inserted in each SL. The material was predicted to be a possible carrier of either QAHE effect or topological magnetoelectric effect (TME) depending on its layer magnetization \cite{225cpl-Zhang_2020,225prb-PhysRevB.102.121107}. Mn${}_2$Bi${}_2$Te${}_5$ single crystals have recently been successfully prepared \cite{225exp-PhysRevB.104.054421}. However, as discussed below, the magnetic ground state of Mn${}_2$Bi${}_2$Te${}_5$ is near ferromagnetic-antiferromagnetic (FM-AFM) boundary, which leads to a very small exchange energy and thus a magnetic ordering temperature below 20 K. We can replace one of Mn atom layer in Mn${}_2$Bi${}_2$Te${}_5$ with other magnetic atoms, or even both, namely \textit{XY}Bi${}_2$Te${}_5$ where \textit{X} and \textit{Y} represent magnetic atoms, to implement the above idea. 

In this Letter, via systematic first-principles calculations, we demonstrate our findings on high-temperature topological phases in \textit{XY}Bi${}_2$Te${}_5$-family materials, where \textit{X}, \textit{Y}=Mn, Ni, V, Eu. We predict that NiVBi${}_2$Te${}_5$ is a FM WSM in the bulk phase and a Chern insulator in thin films that can show the QAHE well above 77 K. Furthermore, the bulk phase of Ni${}_2$Bi${}_2$Te${}_5$ is a dynamic axion insulator with the magnetic ordering temperature near room temperature.

% fig1

The unit cell of \textit{XY}Bi${}_2$Te${}_5$ consists of \textit{ABC}-stacking Te1-Bi1-Te2-\textit{X}-Te3-\textit{Y}-Te4-Bi2-Te5 nonuple atomic layers (NLs), with a van der Waals (vdW) vacuum gap between neighboring NLs. A schematic crystal structure is shown in Fig. \ref{fig1}(a). The space group is $P\bar{3}m1$ (No. 164) for the case \textit{X}=\textit{Y} while $P3m1$ (No. 156) otherwise, depending on whether inversion symmetry is preserved when magnetic moments are ignored.

The magnetic configurations of the ground state and the coupling strength depend on superexchange rules. In \textit{XY}Bi${}_2$Te${}_5$, three types of magnetic couplings determine the stable magnetic configurations, including couplings between \textit{X} and \textit{Y} atoms in the same NL (interatomic-layer couplings), between \textit{X} and \textit{Y} atoms located in neighboring NLs (interlayer couplings) and between \textit{X} or \textit{Y} atoms in the same atomic layer plane (intralayer couplings). All of these couplings are illustrated in Fig. \ref{fig1}(a). Since interlayer and intralayer couplings resemble those in MnBi${}_2$Te${}_4$-family materials\cite{mbt-sciadv-doi:10.1126/sciadv.aaw5685,mbt-otrokov-PhysRevLett.122.107202,mbt-wj-PhysRevLett.122.206401,lizhe-PhysRevB.102.081107}, we mainly need to understand the interatomic-layer couplings, which are determined by superexchange interactions. 

\begin{table*}[t]
\caption{\label{tab1}Interlayer and interatomic-layer exchange energy per unit cell of bulk \textit{XY}Bi${}_2$Te${}_5$ materials. Positive (negative) values stand for FM (AFM) ground state and red (black) terms stand for out-of-plane (in-plane) magnetic ground state. For Mn${}_2$Bi${}_2$Te${}_5$, all presented data is based on theoretical lattice constants.}
\begin{ruledtabular}
\begin{tabular}{crrrrrrrrrr}
\textit{X}-\textit{Y} & \color{red}Mn-Mn & \color{red}Ni-Ni & V-V & Eu-Eu & Mn-V & \color{red}Ni-V & \color{red}Mn-Ni & \color{red}Mn-Eu & V-Eu & \color{red}Ni-Eu \\
Interlayer (meV) & \color{red}-1.74 & \color{red}-10.82 & -0.68 & -0.11 & +1.30 & \color{red}+1.59 & \color{red}-4.14 & \color{red}+0.38 & -0.30 & \color{red}+0.80\\
Interatomic (meV) & \color{red}+1.36 & \color{red}-266.89 & -96.10 & -3.17 & +34.40 & \color{red}+84.46 & \color{red}-104.69 & \color{red}+11.68 & -13.43 & \color{red}+24.52
\end{tabular}
\end{ruledtabular}
\end{table*}

Table \ref{tab1} shows the calculated interlayer and interatomic-layer coupling strengths of \textit{XY}Bi${}_2$Te${}_5$-family materials. Similar to the MnBi${}_2$Te${}_4$-family materials \cite{lizhe-PhysRevB.102.081107}, the sign and relative strength of the magnetic couplings can also be understood with the Goodenough-Kanamori rules. Here, we focus on the interatomic-layer coupling which is expected to be much stronger than the interlayer one. The interatomic-layer coupling of \textit{XY}Bi${}_2$Te${}_5$ is contributed by two kinds of hopping channels of \textit{X}-Te-\textit{Y} bonds, one with the bond angle near $90^{\circ}$ ($ \theta_1 $  in Fig. \ref{fig1}(a)), and the other with the bond angle near $180^{\circ}$ ($ \theta_2 $  in Fig. \ref{fig1}(a)). The signs of the couplings contributed by the two channels (referred as $ \theta_1 $ and $ \theta_2 $ channels below, respectively) are opposite. For Ni${}_2$Bi${}_2$Te${}_5$ and V${}_2$Bi${}_2$Te${}_5$, $ \theta_2 $ is very close to $180^{\circ}$, meanwhile $ \theta_1 $ deviates several degrees from $90^{\circ}$. Therefore, the coupling is dominated by the $ \theta_2 $ channel which is AFM. The interatomic-layer coupling of NiVBi${}_2$Te${}_5$ is also dominated by the $ \theta_2 $ channel, which however gives a FM ground state since the number of $ 3d $  electron is above 5 in Ni and below 5 in V. In Mn${}_2$Bi${}_2$Te${}_5$, the interatomic-layer coupling is even smaller than the interlayer one. It is because the couplings via the $ \theta_1 $ and $ \theta_2 $ channel are largely cancelled with each other. It is to note that by slightly varying the lattice biaxial strength, on-site Coulomb repulsion $ U $  in GGA+$ U $  or the correlation functionals (Fig. S1 \cite{sm}), the interatomic-layer coupling of Mn${}_2$Bi${}_2$Te${}_5$ can be changed from FM to A-type AFM. This does not contradict previous work on Mn${}_2$Bi${}_2$Te${}_5$ \cite{225cpl-Zhang_2020,225prb-PhysRevB.102.121107}. Nontheless, whether the interatomic-layer coupling is FM or AFM in real Mn${}_2$Bi${}_2$Te${}_5$ materials has not been decided in experiment \cite{225exp-PhysRevB.104.054421}.

We estimate Néel/Curie temperature of \textit{XY}Bi${}_2$Te${}_5$ materials (bulk phase) with Monte Carlo simulations. As shown in Fig. \ref{fig1}(c) and Fig. \ref{fig1}(d), Ni${}_2$Bi${}_2$Te${}_5$, V${}_2$Bi${}_2$Te${}_5$, NiVBi${}_2$Te${}_5$, MnNiBi${}_2$Te${}_5$ and MnVBi${}_2$Te${}_5$ have Néel/Curie temperatures above 77 K, and Ni${}_2$Bi${}_2$Te${}_5$ even has a room-temperature Néel temperature. The result coincides with their strong interatomic-layer couplings (all above 30 meV per unit cell) shown in Table. \ref{tab1}.

It's worth mentioning that our calculations for Mn${}_2$Bi${}_2$Te${}_5$ (17 K) is close to the experimental result (20 K) \cite{225exp-PhysRevB.104.054421}. Besides, VEuBi${}_2$Te${}_5$ exhibits two critical temperatures, 90 K and 29 K, corresponding to those of the V and Eu atomic layers, respectively. It is due to the weak magnetic coupling strength between V and Eu atomic-layers.

One can see from Fig. \ref{fig1} and Table \ref{tab1} that the compounds including Ni or V atoms have higher Néel/Curie temperature. The partial density of states (DOS) and projected band structures of Mn, Eu, Ni, V are plotted in Fig. \ref{fig1}(e) and Fig. \ref{fig1}(f) respectively. Appreciably, the distributions of Ni $3d$ and V $3d$ orbitals locate near the Fermi level, which benefit the virtual hopping processes and greatly enhance the magnetic coupling strength \cite{book}. With unoccupied $e_g$ orbitals situating even nearer the Fermi level, Ni behaves much better than V. In contrast, the Mn $3d$ and Eu $5d$ orbitals locate quite far from the Fermi level, generating much weaker coupling strength.


% fig2

\begin{figure}
    \centering
    \includegraphics[width=1\linewidth]{Fig2.png}
    \caption{Topological features of NiVBi${}_2$Te${}_5$ in bulk and thin films. (a) Band structure of WSM NiVBi${}_2$Te${}_5$. Projected band details near one of the WPs are shown in the inset. (b) Surface states on the (100) plane, showing a Fermi arc connecting the two WPs. (c) The motions of the sum of WCCs on spheres surrounding each WP in the momentum space. (d) The evolution of band gap and Chern number with thickness of NiVBi${}_2$Te${}_5$. Chiral edge states of (e) 6-NL ($\mathcal{C}=1$) and (f) 20-NL ($\mathcal{C}=2$) slabs.}
    \label{fig2}
\end{figure}

Figure \ref{fig2} exhibits the topological characters of NiVBi${}_2$Te${}_5$, the \textit{XY}Bi${}_2$Te${}_5$ compound with FM ground state and the highest Curie temperature. Due to the inherent out-of-plane FM nature and Curie temperature up to 157 K (Fig. \ref{fig1}(d)), NiVBi${}_2$Te${}_5$ holds various fascinating topological states far above liquid-nitrogen temperature under zero field. It possesses a type-I WSM phase in bulk. The band structure is demonstrated in Fig. \ref{fig2}(a), and one pair of Weyl points (WPs) emerge along the $-A$-$\Gamma$-$A$ line, with one of them shown in the inset. Our surface state calculations confirm that one pair of WPs exist due to $\mathcal{T}$-symmetry breaking. Figure \ref{fig2}(b) clearly shows that a Fermi arc connects two WPs that are symmetrically located along the $-A$-$\Gamma$-$A$ on the (100) surface plane. Two WPs demonstrate opposite chirality by checking the motions of the sum of Wannier charge centers (WCCs) (Fig. \ref{fig2}(c)). Beyond intrinsic WSM phase in NiVBi${}_2$Te${}_5$, WSM can appear in some other superlattices if appropriate external magnetic field is applied to tune the magnetic configurations. See Part VII of Supplemental Material \cite{sm} for more discussions.

% Besides, by artificially changing the stacking configuration of NiVBi${}_2$Te${}_5$ and NiEuBi${}_2$Te${}_5$, a tunable WSM phase can be discovered in NiV-VNiBi${}_2$Te${}_5$ and NiEu-EuNiBi${}_2$Te${}_5$ superlattices, once the AFM interlayer coupling is flipped to FM one under external magnetic field (Fig. S6 \cite{sm}). Considering the weak interlayer bonding, it is possible to be experimentally realized via van der Waals epitaxy.

The intrinsic bulk WSM phase in NiVBi${}_2$Te${}_5$ indicates that its thin films can host Chern insulator phases with Chern number growing with increasing thickness \cite{HgCr2Se4-PhysRevLett.107.186806}. Figure. \ref{fig2}(d) shows the evolution of the gap size and the Chern number calculated by maximally localized Wannier function (MLWF) based tight-binding models. A NiVBi${}_2$Te${}_5$ film is a normal insulator below 6-NL, Chern insulator with $\mathcal{C}=1$ between 6-NL and 17-NL, and high Chern insulator ($\mathcal{C}\geqslant 2$) above 17-NL. The relatively short distance between the two WPs leads to slow growth of Chern number, compared to that of FM state MnBi${}_2$Te${}_4$ \cite{nsr-10.1093/nsr/nwaa089}. Naturally, the slab band gaps experience a close and reopen process every time the Chern number changes. In the Chern insulator region, which supports QAHE, the full gap reaches its maximum value of 16.5 meV in 10-NL slabs, corresponding to 157 K. Figures \ref{fig2}(e) and \ref{fig2}(f) demonstrate the calculated chiral edge states of 6-NL and 20-NL respectively, validating the Chern insulator phases with different Chern numbers.



% fig3

\begin{figure}
    \centering
    \includegraphics[width=1\linewidth]{Fig3.png}
    \caption{Intrinsic dynamic axion state in ground state Ni${}_2$Bi${}_2$Te${}_5$. (a, b) Band structures of bulk Ni${}_2$Bi${}_2$Te${}_5$ with and without SOC, respectively. (c) DOS of A-type AFM bulk Ni${}_2$Bi${}_2$Te${}_5$ with SOC, clearly illustrating its insulating behavior. (d) The evolution of band gap under increasing SOC strength. (e) Calculated surface states on the (001) and (1$\bar{1}$0) plane respectively. (f) Schematic illustration of the two surfaces. $ \boldsymbol{a} $, $ \boldsymbol{b} $, $ \boldsymbol{c} $ are the three lattice vectors.}
    \label{fig3}
\end{figure}

Another eminent compound in the family is Ni${}_2$Bi${}_2$Te${}_5$, which shows the same magnetic structures and lattice symmetries as AFM Mn${}_2$Bi${}_2$Te${}_5$. The $\mathcal{P}$-breaking, $\mathcal{T}$-breaking and $\mathcal{PT}$-conserving situation brings chances on achieving dynamic axion states \cite{daf-Li2010}, as reported in Mn${}_2$Bi${}_2$Te${}_5$ \cite{225cpl-Zhang_2020,225prb-PhysRevB.102.121107}. The Ni-Te-Ni superexchange interactions are much stronger than Mn-Te-Mn in Mn${}_2$Bi${}_2$Te${}_5$, bringing room-temperature AFM order. Figures \ref{fig3}(a) and \ref{fig3}(b) show the orbital-projected band structures of bulk Ni${}_2$Bi${}_2$Te${}_5$ with and without spin-orbit coupling (SOC) respectively. Band inversion between Bi $6p_z$ and Te $5p_z$ indicates the topologically non-trivial character. The calculated DOS (Fig. \ref{fig3}(c)) exhibits a full-gap in the whole system. By increasing the SOC strength of bulk Ni${}_2$Bi${}_2$Te${}_5$ from 70\% to 100\%, in which 100\% represents the realistic value, the band gap firstly decreases, reaching the minimum (52.5 meV) at 87\% and then reincreases, indicating a band inversion (Fig. \ref{fig3}(d) and Fig. S9 \cite{sm}). Surface states are gapped on all surfaces (Fig. \ref{fig3}(e)), and the minimum of them reaches approximately 7 meV ($\sim$87 K). The above calculations can be concluded in two aspects: a band inversion process without gap-closing point and gapped surface states on all surfaces. They are predicted to appear in materials preserving $\mathcal{PT}$ symmetry but breaking $\mathcal{P}$ and $\mathcal{T}$ symmetries individually, serving as the signatures on the emergence of dynamic axion fields \cite{pt-PhysRevB.101.081109}. 



% fig4

We further investigate bulk phase transitions in ten compounds of \textit{XY}Bi${}_2$Te${}_5$ under hydrostatic pressures from -1.0 GPa to +1.0 GPa with out-of-plane FM magnetization, where ambient pressure is labeled as 0.0 GPa. Three distinguishable types of pressure-induced topological phases are identified, including normal insulator, WSM and magnetic topological insulator. Additionally, the strong magnetic couplings in \textit{XY}Bi${}_2$Te${}_5$ are robust under external pressure. For example, the Curie temperature of NiVBi${}_2$Te${}_5$ even increases slightly from 157 K (0.0 GPa) to 166 K (+1.0 GPa) (Fig. S5). Since out-of-plane FM MnVBi${}_2$Te${}_5$ becomes topologically non-trival under moderate external pressure (above +0.24 GPa), its thin films are also expected to demonstrate high-temperature QAHE. More details are discussed in Part X of Supplemental Material \cite{sm}.

% conclusion
In conclusion, we systematically study the magnetic and topological properties in \textit{XY}Bi${}_2$Te${}_5$-family materials. Interatomic-layer exchange couplings play a crucial role in keeping magnetic order above 77 K (some compounds even above 150 K), while interlayer couplings and hybridization between top and bottom surfaces determine the band topology. We find NiVBi${}_2$Te${}_5$ is an emergent material with high Chern number and high-temperature ($\sim$157 K) QAHE state and Ni${}_2$Bi${}_2$Te${}_5$ as another candidate demonstrating room-temperature AFM order and dynamic axion states up to 87 K. Under external pressure or magnetic field, possible high-temperature WSM and QAHE phases can also be achieved.

\begin{acknowledgments}
We thank Boxuan Li for helpful discussions and Jiaheng Li for technical support. This work was supported by the National Natural Science Foundation of China (92065206).

X.-Y.T. and Z.L. contributed equally to this work.
\end{acknowledgments}

\nocite{*}

\bibliography{apssamp}% Produces the bibliography via BibTeX.

\end{document}
%
% ****** End of file apssamp.tex ******
