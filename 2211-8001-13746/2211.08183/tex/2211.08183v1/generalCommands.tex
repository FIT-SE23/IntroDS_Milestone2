% Additional packages

\usepackage[usenames,dvipsnames,svgnames,table]{xcolor}
\usepackage[subfigure]{tocloft}
\usepackage{changes}
\usepackage{xifthen}
\usepackage{wasysym}
\usepackage{algorithm}
\usepackage{algpseudocode}
%\usepackage{graphicx}
%\usepackage{subfigure}
\usepackage{amsopn}
\usepackage{bm}
\usepackage{multicol}
%\usepackage{enumitem}

% Figures path
\newcommand{\figuresPath}{}

% set figures path

\newcommand{\setFiguresPath}[1]
{
  \renewcommand{\figuresPath}{#1}
}

\makeatletter

\algnewcommand\NewParameterLine{%
	\newline \hspace*{\algorithmicindent} \hspace*{\algorithmicindent}
}

\algnewcommand\LeftComment[1]{%
	\Statex \vspace{0.5\baselineskip}\hspace{\ALG@thistlm}$\triangleright$  #1\hfill %
}

\algnewcommand\FirstLeftComment[1]{%
	\Statex \vspace{0.5\baselineskip}\hspace{\algorithmicindent}\hspace{\ALG@thistlm}$\triangleright$  #1\hfill %
}

\algnewcommand\FirstLeftCommentCont[1]{%
	\Statex \hspace{\algorithmicindent}\hspace{\ALG@thistlm}\phantom{$\triangleright$}  #1\hfill %
}
\algnewcommand\MultiLineState{%
	\Statex \hspace{\algorithmicindent}\hspace{\ALG@thistlm}%
}
\makeatother

\renewcommand{\algorithmicrequire}{\textbf{Input:}}
\renewcommand{\algorithmicensure}{\textbf{Output:}}

\newcommand{\Item}[2]
{\item #1{#2}}

\newcommand{\Not}
{\textbf{not \ }}

\newcommand{\AND}
{\textbf{and \ }}

\newcommand{\OR}
{\textbf{or \ }}

\newcommand{\code}[1]
{\texttt{#1}}

\renewcommand{\textproc}[1]
{\texttt{#1}}

\newcounter{exampleCounter}
\refstepcounter{exampleCounter}

\newenvironment{Example}[1]
{\textbf{\arabic{exampleCounter}. #1}.}
{\refstepcounter{exampleCounter}}

% Define \includeGraphics command in order to take into account the figures path.

% export figures to usedFigures directory

%\renewcommand{\includegraphics}[2][]
%{
%	\write18{cp figures/#2 usedFigures}
%}

%--My own definitions--

\newcommand{\C}
{
  @{}c@{}
}

\newcommand{\dotProduct}[3][]
{\ensuremath{\left( #2,#3 \right)}_{#1}\xspace}

\newcommand{\ds}
{
  \displaystyle
}

\newcommand{\define}[1]
{\emph{#1}}

\renewcommand{\vec}[1]{\mathbf{#1}}

\newcommand{\mat}[1]{\mathbf{#1}}
\newcommand{\Frac}[2]{\frac{\displaystyle #1}{\displaystyle #2}}

\providecommand{\url}[1]{\texttt{#1}}

\newcommand{\Rn}[1]
{
\ensuremath{\mathbb{R}^{#1}}
}

%\newcommand{\norm}[2][]
%{\ifthenelse{\isempty{#1}}
%	{\ensuremath{\left\|#2 \right\|}}
%	{\ensuremath{\left\|#2 \right\|_{#1}}}\xspace
%}

\newcommand{\norm}[2][]
{
	\ensuremath{\left\| #2\right\|
		\ifthenelse{\isempty{#1}}
		{}
		{_{{#1}}}}\xspace
}

\newcommand{\gradient}[1]
{\ensuremath{\nabla#1}\xspace
}

\newcommand{\hessian}[1]
{\ensuremath{\mat{H}#1}\xspace
}

\newcommand{\high}[1]{\textcolor{Sepia}{#1}}

\newcommand{\exponential}[1]
{\ensuremath{\textrm{e}^{#1}}\xspace}


\newcounter{todoListCounter}

\newcommand{\theTodoListCounter}
{
  \arabic{todoListCounter}
}

\newcommand{\done}
{\textcolor{OliveGreen}{$\CheckedBox$}}

\newcommand{\notDone}
{\textcolor{Sepia}{$\Square$}}

\newcommand{\doNot}
{\textcolor{BrickRed}{$\XBox$}}

\newcommand{\notSure}
{\textcolor{Sepia}{$\Square$\hspace{-0.6em}}\textcolor{Peach}{\textbf{?}}}


\newcommand{\task}[1][\notDone]
{\item[\Large #1]}

\newenvironment{todoList}[2][\notDone]
{
  \refstepcounter{todoListCounter}
  \newtodo{#1}{#2}
  \begin{itemize}
%  	\setlength{\itemsep}{0.0\baselineskip}

}
{
  \end{itemize}
  \vspace{-0.5\baselineskip}
  \rule{\linewidth}{1pt}\\
}

\newcommand{\listTodoName}
{ \ifthenelse{\theTodoListCounter >1}
  {
    There are \textcolor{Sepia}{\theTodoListCounter} todo's
  }
  {
    There is \textcolor{Sepia}{\theTodoListCounter} todo
  }
}

\newlistof{todo}{tmp}{\listTodoName}

\providecommand{\newtodo}[2]
{
  \refstepcounter{todo}
  \noindent \rule{\linewidth}{1pt} \\
  \textbf{\large \theTodoListCounter} \textcolor{Sepia}{ TODO}: {\sc #2} \hfill \raisebox{-1ex}{\Huge #1}\\
  \rule{\linewidth}{1pt} \\
  \vspace{-1.0\baselineskip}
  \addcontentsline{tmp}{todo}{{\large #1} \textcolor{Sepia}{TODO \numberline{\thetodo:}}\textcolor{black}{#2}}
}

\newcommand{\listOfTodo}
{
  \ifthenelse{\theTodoListCounter > 0}
  {
    \clearpage
    \listoftodo
  }
  {
  }
}

%\setremarkmarkup{\marginpar[#1]{\textcolor{Changes@Color#1}{#2}}}

\newcommand{\eval}[2]
{\ensuremath{#1\mathopen{}\left(#2\right)\mathclose{}}}

\newcommand{\Functional}[1]
{E_{#1}}

\newcommand{\format}[1]
{\texttt{#1}}

%\newtheorem{Rem}{Remark}
%\newtheorem{Proposition}{Proposition}
%\theoremstyle{definition}
%\newtheorem{Corollary}{Corollary}