\usepackage{xspace}

\algrenewcommand\algorithmicindent{1.0em}

\newcommand{\distortionFunction}
{\ensuremath{\eta\xspace}}

\newcommand{\qualityFunction}
{\ensuremath{\mathrm{q}\xspace}}

\newcommand{\weight}
{\ensuremath{\omega}\xspace}

\newcommand{\mapName}
{\ensuremath{\bm{\phi}^*\xspace}}

\newcommand{\multipliers}
{\ensuremath{\bm{\lambda}}\xspace}

\newcommand{\mapNameB}
{\ensuremath{\bm{\phi}\xspace}}

\newcommand{\elementName}
{\ensuremath{e\xspace}}

\newcommand{\domainName}
{\ensuremath{\Omega\xspace}}

\newcommand{\meshName}
{\ensuremath{\mathcal{M}\xspace}}

\newcommand{\nodeSetName}
{\ensuremath{\mathcal{V}\xspace}}

\newcommand{\pCoord}
{\ensuremath{\vec{x}}\xspace}

\newcommand{\iCoord}
{\ensuremath{\vec{y}}\xspace}

\newcommand{\mCoord}
{\ensuremath{\bm{\xi}}\xspace}

\newcommand{\uCoord}
{\ensuremath{\vec u}\xspace}

\newcommand{\tCoord}
{\ensuremath{t}\xspace}

\newcommand{\vCoord}
{\ensuremath{\vec p}\xspace}

\newcommand{\distortion}[2][]
{\ifthenelse{\isempty{#2}}
	{\ensuremath{\distortionFunction_{#1}}}
	{\ensuremath{\distortionFunction_{#1}(#2)}}\xspace
}

\newcommand{\modDistortion}[2][]
{\ifthenelse{\isempty{#2}}
	{\ensuremath{\distortionFunction_{0}}}
	{\ensuremath{\distortionFunction_{0}(#2)}}\xspace
}

\newcommand{\quality}[2][]
{\ifthenelse{\isempty{#2}}
	{\ensuremath{\qualityFunction_{#1}}}
	{\ensuremath{\qualityFunction_{#1}(#2)}}\xspace
}

\newcommand{\modQuality}[2][]
{\ifthenelse{\isempty{#2}}
	{\ensuremath{\qualityFunction_{#1}^*}}
	{\ensuremath{\qualityFunction_{#1}^*(#2)}}\xspace
}

\newcommand{\objectiveFunction}[2][]
{\ifthenelse{\isempty{#2}}
	{\ensuremath{f_{#1}}}
	{\ensuremath{f_{#1}(#2)}}\xspace
}

\newcommand{\h}[1]
{\ifthenelse{\isempty{#1}}
	{\ensuremath{\determinant_\delta}}
	{\ensuremath{\determinant_\delta(#1)}}\xspace
}

%\newcommand{\surface}[1]
%{\ifthenelse{\isempty{#1}}
%	{\ensuremath{\bm{\varphi}}}
%	{\ensuremath{\bm{\varphi}(#1)}}\xspace
%}
%
%\newcommand{\curve}[1]
%{\ifthenelse{\isempty{#1}}
%	{\ensuremath{\bm{\gamma}}}
%	{\ensuremath{\bm{\gamma}(#1)}}\xspace
%}

\newcommand{\jacobian}[1][]
{\ifthenelse{\isempty{#1}}
	{\ensuremath{\mat{S}}}
	{\ensuremath{\mat{S}(#1)}}\xspace
}

\newcommand{\map}[2][]
{\ensuremath{\mapName
\ifthenelse{\isempty{#2}}
	{{}}
	{_{#2}}
\ifthenelse{\isempty{#1}}
	{{}}
	{(#1)}}\xspace
}

\newcommand{\mapB}[2][]
{\ensuremath{\mapNameB
\ifthenelse{\isempty{#2}}
	{{}}
	{_{#2}}
\ifthenelse{\isempty{#1}}
	{{}}
	{(#1)}}\xspace
}

\newcommand{\mapD}[2][]
{\ensuremath{\mapNameB_h
    \ifthenelse{\isempty{#2}}
    {{}}
    {_{#2}}
    \ifthenelse{\isempty{#1}}
    {{}}
    {(#1)}}\xspace
}

\newcommand{\determinant}[1][]
{\ifthenelse{\isempty{#1}}
	{\ensuremath{\sigma}}
	{\ensuremath{\sigma(#1)}}\xspace
}

\newcommand{\element}[1][]
{\ifthenelse{\isempty{#1}}
	{\ensuremath{\elementName}}
	{\ensuremath{\elementName_{#1}}}\xspace
}

\newcommand{\nodeSet}[1]
{\ensuremath{\nodeSetName({#1})}\xspace}

\newcommand{\domain}[2][]
{\ensuremath{\domainName
	\ifthenelse{\isempty{#1}}
	{{}}
	{^{#1}}_{#2}}\xspace
}

\newcommand{\firstDerivative}[2]
{\ensuremath{\displaystyle \frac{\partial #1}{\partial #2}}\xspace}

\newcommand{\secondDerivative}[3]
{\ensuremath{\displaystyle \frac{\partial^2 #1}{\partial #2 \partial #3}}\xspace}

\newcommand{\Gradient}[3][]
{\ensuremath{
	\nabla
	\ifthenelse{\isempty{#1}}
		{}
		{_{#1}}
	#2
	\ifthenelse{\isempty{#3}}
		{}
		{(#3)}
}\xspace
}

\newcommand{\Jacobian}[2][]
{\ifthenelse{\isempty{#1}}
	{\ensuremath{\textrm{\textbf{D}}#2}}
	{\ensuremath{\textrm{\textbf{D}}#2(#1)}}\xspace
}

\newcommand{\mesh}[2][]
{\ensuremath{\meshName^{{#1}}_{{#2}}}\xspace}
%{\ifthenelse{\isempty{#1}}
%	{\ensuremath{\meshName}}
%	{\ensuremath{\meshName_{#1}}}\xspace
%}

\newcommand{\projection}[2]
{\eval{\Pi_{#1}}{#2}}

%\DeclareMathOperator*{\argmin}{argmin}
\DeclareMathOperator*{\argmin}{arg\,min}

\newcommand{\node}[1]
{\ifthenelse{\isempty{#1}}
	{\ensuremath{v}}
	{\ensuremath{\vec x_{#1}}}\xspace
}

\newcommand{\scalarProduct}[3][]
{\ensuremath{\langle #2, #3\rangle
\ifthenelse{\isempty{#1}}
	{}
	{_{#1}}}\xspace
}

\newcommand{\trace}
{\ensuremath{\boldsymbol T}\xspace}

\newcommand{\diff}{\ensuremath{\text{d}}\xspace}

\newcommand{\virtualSurface}{\ensuremath{\mathcal{S}}\xspace}
\newcommand{\virtualCurve}{\ensuremath{\mathcal{C}}\xspace}
\newcommand{\surface}{\ensuremath{S}\xspace}


\newcommand{\xNumNodes}{{n_N}}
\newcommand{\xNumNodesBound}{{m_N}}


