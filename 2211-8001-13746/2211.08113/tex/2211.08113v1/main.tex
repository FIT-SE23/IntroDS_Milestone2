%%%%%%%%%%%%%%%%%%%%%%%%%%%%%%%%%%%%%%%%%%%%%%%%%%%%%%%%%%%%%%%%%%%%%%
%%%%%%%%%%%%%%%%%%%%%%%%%%%%%% PREAMBLE %%%%%%%%%%%%%%%%%%%%%%%%%%%%%% 
\documentclass{article} 
%=====================================================================	   
%==================== Usepackage ====================%
\usepackage{amsmath}% ams mathematical formulas  
\usepackage{amssymb}% ams mathematical symbols 
\usepackage{bm}% bold math symbols
\usepackage{siunitx}% si units      
\usepackage{cite}% numeric citation
\usepackage{lmodern}% latin modern family of fonts
%================== End Usepackage ==================%
%=====================================================================	
%==================== Newcommand ====================%      
\newcommand{\eneqref}[1]{Eq. \eqref{#1}}
%================== End Newcommand ==================%
%=====================================================================	
%%%%%%%%%%%%%%%%%%%%%%%%%%%% END PREAMBLE %%%%%%%%%%%%%%%%%%%%%%%%%%%%
%%%%%%%%%%%%%%%%%%%%%%%%%%%%%%%%%%%%%%%%%%%%%%%%%%%%%%%%%%%%%%%%%%%%%%
%%%%%%%%%%%%%%%%%%%%%%%%%%%%%%% FORMAT %%%%%%%%%%%%%%%%%%%%%%%%%%%%%%%
%---------------------------------------------------------------------	  
%-------------------- Page Format -------------------%
\usepackage[a4paper]{geometry} \geometry{scale=0.8} 
%------------------ End Page Format -----------------%
%---------------------------------------------------------------------
%----------------- Paragraph Format -----------------%
\setlength {\parskip} {1em}
%--------------- End Paragraph Format ---------------%
%---------------------------------------------------------------------
%%%%%%%%%%%%%%%%%%%%%%%%%%%%% END FORMAT %%%%%%%%%%%%%%%%%%%%%%%%%%%%% 
%%%%%%%%%%%%%%%%%%%%%%%%%%%%%%%%%%%%%%%%%%%%%%%%%%%%%%%%%%%%%%%%%%%%%%

%%%%%%%%%%%%%%%%%%%%%%%%%%%%%%%%%%%%%%%%%%%%%%%%%%%%%%%%%%%%%%%%%%%%%%
%%%%%%%%%%%%%%%%%%%%%%%%%%%%%% DOCUMENT %%%%%%%%%%%%%%%%%%%%%%%%%%%%%%
\begin{document}	

\title{Rotating Equivalent Temperature and Solar Granulation}
\author{Chen Haibin, Wu Rong\thanks{wurong2@mail3.sysu.edu.cn}}
\date{}
\maketitle

\section*{\centering Abstract}  

The rotation of the fluid cell generates additional pressure and exhibits properties similar to temperature in isotrotropic expansion, affecting the convection criteria in the form of size, with small fluid cells in a natural convection state and large fluid cells in a forced convection state. This is verified in observational data on solar  granulation, which also infer that the critical size of the granule is negatively correlated with the value of the local average vortex. 

keyword: rotating equivalent temperature, solar granulation, convection criterion, rotating turbulent thermal convection

\section*{Introduction}

A large number of observations show that, There is a critical diameter in the granule, which can be divided into large and small granule according to the diameter. The number of large granules decreases rapidly with the increase of diameter, while the number of small granules monotonously increases or remains flat with the decrease of diameter. The average brightness, maximum brightness and fractal dimension of the two kinds of granules is different. The critical diameter of granule obtained by different physical quantities is different, but the range is close to $1 ''. 3  \sim 1 ''. 5$  \cite {1,2,3,4,5,6,7}.


In the study of rotating turbulent thermal convection and solar differential rotation \cite {8}, we found the effect of the rotating on criterion convection, which makes small-diameter fluid cells in natural convection state and large-diameter fluid cells in forced convection state. This theory can be further extended to solar granule.


\section{Thermal Properties of Rotating Fluid Cells}

\subsection{Additional pressure provided by rotation in a rotating fluid cell}

There is a pressure gradient along the radial direction in the fluid rotating uniformly without external force, which can be expressed as
\begin{equation}
	\frac{\mathrm{d} p}{\mathrm{d} R} = \rho \Omega^2 R   .
	\label{Eq.1}
\end{equation}
\\
The pressure distribution along the radial direction in the uniformly rotating fluid is
\begin{equation}
	p = p_O + \frac{1}{2} \rho \Omega^2 R^2   ,
	\label{Eq.2}
\end{equation}
\\
Where $p_O$ is the pressure of the rotation axis of the fluid. Suppose that a point $\mathrm {A}$ in the rotating fluid is $R$ away from the rotation axis, a point $\mathrm{B}$ in its neighborhood is $l$ away from the point $\mathrm{A}$, and the pressure of the point $\mathrm {A}$ and point $\mathrm {B}$ can be expressed as
\begin{equation}
	\left\{\begin{matrix} 
		& p_{\mathrm{A}} = p_O + \frac{1}{2} \rho \Omega^2 R^2   ; \hfill\hfill\\  
		& p_{\mathrm{B}} = p_O + \frac{1}{2} \rho \Omega^2 R^2 + \rho \Omega^2 R l \mathrm{cos} \left\langle \boldsymbol{R , l} \right\rangle  + \frac{1}{2} \rho \Omega_2 l_2  .\hfill\hfill\\
	\end{matrix}\right.  
	\label{Eq.3}
\end{equation}
\\
Then the pressure difference between the point $\mathrm {A}$ and the point $\mathrm{B}$ is
\begin{equation}
	\delta p = \rho \Omega^2 R l \mathrm{cos} \left\langle \boldsymbol{R , l} \right\rangle + \frac{1}{2} \rho \Omega^2 R^2   ,
	\label{Eq.4}
\end{equation}
\\
where, $\rho \Omega^2 R l \mathrm{cos} \left\langle \boldsymbol{R, l} \right\rangle$ is the pressure difference caused by the radial pressure gradient, and $\frac{1}{2} \rho \Omega^2 R^2$ is the second-order pressure difference caused by the pressure gradient unique to the rotating fluid. The second term is independent of the pressure gradient direction, but only related to the distance between two points. For fluid cells with the concept of size, the average pressure on the surface is generally greater than the average pressure inside. In some convection, it is necessary to consider the second-order pressure difference, which will lead to new characteristics of the rotating fluid.

Take an independent cylinder fluid cell with radius of $a$, height of $2 a$, and rotational speed of $\Omega$ from the rotating fluid as the research object. Since the fluid cell rotates, there is a pressure gradient along the radial direction in the cylinder, which is
\begin{equation}
	\frac{\mathrm{d} p}{\mathrm{d} r} = \rho \Omega^2 r   .
	\label{Eq.5}
\end{equation}
\\
The pressure distribution along the radial direction in the rotating cylinder obtained by integration is
\begin{equation}
	p = p_{o} + \frac{1}{2} \rho \Omega^2 r^2 ,
	\label{Eq.6}
\end{equation}
\\
where $p_{o}$ is the pressure on the rotation axis of the cylinder. Same temperature $T$, different rotational speed $\Omega$, $p_{o}$ is not the same; And $p_{o}$ is also different with same rotational speed $\Omega$ and different temperature $T$, . $p_{o}$ includes both temperature and rotational speed terms, so it is necessary to decouple the pressure generated by temperature and rotational speed . The average pressure on the cylinder surface can be expressed as
\begin{equation}
	\bar{p}_{S}  = \frac{\iint_{S} p \mathrm{d} S} {\iint_S \mathrm{d} S} = \bar{p}_{T} + \bar{p}_{\Omega} ,
	\label{Eq.7}
\end{equation}
\\
where, $\bar{p}_T$ only includes temperature term, $\bar{p}_\Omega$ only includes the rotational speed term. Thus, the average pressure on the cylinder surface is decoupled.

Let the initial rotational speed of the cylinder be zero. When the rotational speed changes, the pressure distribution and density distribution of the internal fluid will also change. Assuming that the gas is adiabatic, the relationship between pressure and density satisfies
\begin{equation}
	\rho = \bar{\rho} \left( \frac{p}{\bar{p}_{T}} \right) ^ \frac{1}{\gamma} ,
	\label{Eq.8}
\end{equation}
\\
where $\gamma$ is the specific heat ratio of gas, $\bar{\rho}$ is the average density when the cylinder speed is zero, $\bar{p}_{T}$ is the average surface pressure when the cylinder speed is zero. The mass can be obtained by integrating the density. Since the volume and mass of the gas remain unchanged before and after the speed change
\begin{equation}
	\int_0 ^a 2 \pi r a \rho \mathrm{d} r = \int_0 ^a 2 \pi r a \bar{\rho} \mathrm{d} r ,
	\label{Eq.9}
\end{equation}
\\
\eneqref{Eq.8} is substituted into \eneqref{Eq.9}, Tidied up
\begin{equation}
	\int_0 ^a \left( \frac{p}{\bar{p}_{T}} \right) ^ \frac{1}{\gamma} r \mathrm{d} r = \int_0 ^a r \mathrm{d} r .
	\label{Eq.10}
\end{equation}
\\
Within linear range, $\left( \frac{p}{\bar{p}_{T}} \right) ^ \frac{1}{\gamma} \approx 1 + \frac{1}{\gamma} \frac{p - \bar{p}_{T}}{\bar{p}_{T}}$, \eneqref{Eq.10} is simplified as
\begin{equation}
	\int_0 ^a \frac{1}{\gamma} \frac{p - \bar{p}_{T}}{\bar{p}_{T}} r \mathrm{d} r = 0 ,
	\label{Eq.11}
\end{equation}
\\
\eneqref{Eq.6} is substituted into \eneqref{Eq.11}, and the solution is
\begin{equation}
	p_{o} = \bar{p}_T - \frac{1}{4} \rho \Omega^2 a^2 ,
	\label{Eq.12}
\end{equation}
\\
It can be seen from the formula, $\bar{p}_T$ is the pressure of the cylinder at the radius of $\frac{\sqrt{2}}{2} a$. Within the linear range, the temperature and pressure at this suface are independent of the rotational speed, and the temperature at this suface $T$ can be used as the characteristic temperature of the cylinder. 
It can be seen from the calculation that the average pressure contributed by rotation to the cylinder bottom is zero, and the average pressure contributed to the cylinder side is $\frac{1}{4} \rho \Omega^2 a^2$ , then the average pressure provided by rotation to the cylinder surface is
\begin{equation}
	\bar{p}_\Omega = \frac{1}{6} \rho \Omega^2 a^2 ,
	\label{Eq.13}
\end{equation}
\\
In the derivation process, the adiabatic condition is introduced, but in fact, as long as the gas cell does not transport heat with the outside world and the pressure change is still in the linear range, the above equation is still valid.

For other shapes such as sphere, if the expansion is Isotropy, the work done by pressure is $\mathrm{d} W = \iint p \mathrm{d} S \mathrm{d} l$ , the total volume change is $\mathrm{d} V = \iint \mathrm{d} S \mathrm{d} l$ . imitation of thermal, $\mathrm{d} W = p \mathrm{d} V$ , so that $\bar{p}_\Omega = \frac{\mathrm{d} W_ \Omega}{\mathrm{d} V}$ , the expression of the additional pressure provided by rotation is similar to that of a cylinder, which can be written as
\begin{equation}
	\bar{p}_\Omega = k_\Omega \rho \Omega^2 a^2 .
	\label{Eq.14}
\end{equation}
\\
For Different shapes, The values of $k_\Omega$ are different.


\subsection{Rotating equivalent temperature}

During isotropic expansion, the pressure generated by the rotation of the fluid cell does work and causes the change of rotation energy. The relationship between the pressure and energy generated by the rotation is similar to the relationship between the pressure and heat energy in ideal gas. Thermology is a very mature discipline. Therefore, the rotation of the fluid cell can be compared with the thermal motion of molecules in the research.

According to the equation of state of ideal gas, the pressure provided by the thermal motion of molecules is $p = K \rho T$ , where $K = \frac{R_\mathrm{m}}{M_\mathrm{m}}$ , $R_\mathrm{m}$ is the molar gas constant, $M_\mathrm{m}$ is the molar mass of the gas. The pressure generated by rotation can also be expressed in this form, so that
\begin{equation}
	\bar{p}_\Omega = K \rho T_\Omega
	\label{Eq.15}
\end{equation}
\\
Then the equivalent temperature $T_\Omega$ of rotation cylinder  can be expressed as
\begin{equation}
	T_\Omega = \frac{1}{6} \frac{M_\mathrm{m}}{R_\mathrm{m}} \Omega^2 a^2 
	\label{Eq.16}
\end{equation}
\\
The rotational kinetic energy of a cylinder with a radius of $a$ , a height of $2 a$ , and a rotating speed of $\Omega$ can be obtained by integration
\begin{equation}
	E_\Omega = \frac{1}{2} \rho \Omega^2 a^5 
	\label{Eq.17}
\end{equation}
\\
It can also be expressed as
\begin{equation}
	E_\Omega = \frac{1}{4} M \Omega^2 a^2 
	\label{Eq.18}
\end{equation}
\\
Where $M$ is the total mass of the cylinder. Imitating the heat capacity, the change of rotational kinetic energy $\mathrm{d} E_\Omega$ caused by the change of rotational equivalent temperature  $\mathrm{d} T_\Omega$ of a rotating cylinder in the case of isotropic expansion is
\begin{equation}
	\mathrm{d} E_\Omega = \frac{M}{M_\mathrm{m}} \left( \frac{3}{2} R_\mathrm{m} \mathrm{d} T_\Omega \right)
	\label{Eq.19}
\end{equation}
\\
So the equivalent kinetic energy capacity of the rotating cylinder ${C_V}_\Omega$ is
\begin{equation}
	{C_V}_\Omega = \frac{3}{2} R_\mathrm{m}
	\label{Eq.20}
\end{equation}
\\
This is the same as the heat capacity of monatomic gas, indicating that they have similar properties. The isotropic expansion of a rotating fluid cell is very similar to the adiabatic expansion of an ideal gas, which can be regarded as an ideal gas expansion process with a degree of freedom of 3 and a specific heat ratio of $\frac{5}{3}$ .

Vorticity is a more convenient physical quantity than rotational speed in calculation. The relationship between the main vorticity of fluid cells $\omega$ and the rotating speed $\omega$ satisfies
\begin{equation}
	\omega = 2 \Omega
	\label{Eq.21}
\end{equation}
\\
Rotating equivalent temperature $T_\Omega$ can also be converted into vorticity equivalent temperature $T_\Omega$ , i.e
\begin{equation}
	T_\omega = T_\Omega = \frac{1}{24} \frac{M_{\mathrm{m}}}{R_{\mathrm{m}}} \omega^2 a^2
	\label{Eq.22}
\end{equation}

\subsection{Transient equilibrium of fluid cell pressure}

When the rotating equivalent temperature of the fluid cell is different from that of the environmental fluid, the rotational speed of the fluid cell is different from that of the environmental fluid, which will lead to the imbalance between the fluid cell and the environmental fluid, resulting in inertial waves, which will make the rotational speed of the fluid cell and the environmental fluid in the oscillate finally tend to be consistent. The characteristic time of the evolution of the inertial waves is the rotational period of the environmental fluid $T_0$ . When the evolution characteristic time of expansion process $t$ is far less than $T_0$ , the influence of inertial wave can be ignored when studying this process, and the local pressure can be unbalanced.

In the convection process, for small-scale spherical fluid cells, the acoustic wave propagation in the expansion process is much faster than the inertial wave. The balance of the total pressure on the surface is much faster than the balance of rotating speed. Therefore, the transient balance of fluid cell is marked by the balance of total pressure on surface .The evolution time $t $ to reach the balance is significantly greater than $\frac{a}{v_c}$ , where $a$ is the characteristic size of the fluid cell, $v_c$ is the sound speed, at this time, the pressure balance meets
\begin{equation}
	\iint_S p \mathrm{d} S = \iint_S p^* \mathrm{d} S
	\label{Eq.23}
\end{equation}
\\
Where, $S$ is the boundary area of the fluid cell, $p$ is the pressure of the fluid cell inside the boundary, and $p^*$ is the pressure of the environmental fluid outside the boundary. To simplify the calculation, we assume that the above relationship is also satisfied in the cylindrical fluid cell, that is, the average surface pressure of the cylindrical fluid cell $\bar{p}_S =\frac{\iint_S p \mathrm{d} S}{\iint_S \mathrm{d} S}$ and the average surface pressure of the environment $\bar{p}_S^* =\frac{\iint_S p^* \mathrm{d} S}{\iint_S \mathrm{d} S}$ is satisfied when the fluid cell is in transient balance
\begin{equation}
	\bar{p}_S = \bar{p}_S^*
	\label{Eq.24}
\end{equation}

\section{Convection Criterion and granule}

\subsection{Criterion of Rotating Turbulent Thermal Convection}

In the non rotating fluid model, when the temperature gradient is higher than Adiabatic gradient, thermal convection will occur, which is the convection criterion. When the average expansion process of the fluid is nearly isotropic, the gradient of the rotating equivalent temperature has the same effect as the temperature gradient, and will also affect the convection criterion.

Since the balance speed of  pressure of the fluid is fast,  the speed of heat conduction is slow, the expansion process of the disturbed fluid cell is approximate to adiabatic expansion, and the relationship between pressure and density meets $p=K_1 \rho^{\gamma}$ , where $K_1$ is a constant, and $\gamma$ is the specific heat ratio of the gas. Combined ideal gas equation of state $p = K_2 \rho T $ , where $K_2$ is a constant, so the temperature gradient determined is
\begin{equation}
	\frac{\mathrm{d} T}{\mathrm{d} l} = \left( 1 - \frac{1}{\gamma} \right) \frac{T}{p} \frac{\mathrm{d} p}{\mathrm{d} l}
	\label{Eq.25}
\end{equation}
Where $l$ is in the direction of gravity (centrifugal force), since this temperature gradient is derived under adiabatic conditions, it is also known as the adiabatic temperature gradient $\left( \frac{\mathrm{d} T}{\mathrm{d} l} \right) _\mathrm{ad}$ . When the actual temperature gradient $\left( \frac{\mathrm{d} T}{\mathrm{d} l} \right) _\mathrm{rd}$  satisfies the Schwarz convection criterion
\begin{equation}
    \left|  \frac{\mathrm{d} T}{\mathrm{d} l} \right| _\mathrm{rd} > \left| \frac{\mathrm{d} T}{\mathrm{d} l} \right| _\mathrm{ad}
	\label{Eq.26}
\end{equation}
\\
The force acting on the disturbed air cell will keep it moving away from its initial position, convection will occur.

There are two temperatures in the rotating turbulent thermal convection. The isotropic expansion of the rotating equivalent temperature can be regarded as a thermal process, with a specific heat ratio of $\frac{5}{3}$ . The gas in the solar convection zone exists as atoms or ions and can be regarded as a monoatomic gas with a specific heat ratio of $\frac{5}{3}$ . The equivalent adiabatic temperature gradient can be expressed as
\begin{equation}
	\left( \frac{\mathrm{d} T + \mathrm{d} T_\Omega}{\mathrm{d} l} \right) _\mathrm{ad} = \frac{2}{5} \frac{T}{\bar{p}_T} \frac{\mathrm{d} \bar{p}_T}{\mathrm{d} l} + \frac{2}{5} \frac{T_\Omega}{\bar{p}_\Omega} \frac{\mathrm{d} \bar{p}_\Omega}{\mathrm{d} l}
	\label{Eq.27}
\end{equation}
or
\begin{equation}
	\left( \frac{\mathrm{d} T + \mathrm{d} T_\Omega}{\mathrm{d} l} \right) _\mathrm{ad} = \frac{2}{3} \left( T + T_\Omega \right) \frac{\mathrm{d} \rho}{\rho \mathrm{d} l}
	\label{Eq.28}
\end{equation}
Obviously, The adiabatic gradient can be expressed more simply as a density gradient. The new convection criterion is
\begin{equation}
	\left| \frac{\mathrm{d} T + \mathrm{d} T_\Omega}{\mathrm{d} l} \right| _\mathrm{rd} > \left| \frac{\mathrm{d} T + \mathrm{d} T_\Omega}{\mathrm{d} l} \right| _\mathrm{ad}
	\label{Eq.29}
\end{equation}

\subsection{Scale form of convection criterion}

According to \eneqref{Eq.16}
\begin{equation}
	\frac{\mathrm{d} T_\Omega}{\mathrm{d} l} = 2 T_\Omega \left( \frac{\mathrm{d} \Omega}{\Omega \mathrm{d} l} + \frac{\mathrm{d} a}{a \mathrm{d} l} \right)
	\label{Eq.30}
\end{equation}
\\
The mass is conservative during the expansion of the fluid cell. There are
\begin{equation}
	\frac{\mathrm{d} a}{a \mathrm{d} l} = - \frac{1}{3} \frac{\mathrm{d} \rho}{\rho \mathrm{d} l}
	\label{Eq.31}
\end{equation}
\\
Substitutioning  \eneqref{Eq.30} and \eneqref{Eq.31} into \eneqref{Eq.28} ,we have
\begin{equation}
	\frac{\mathrm{d} T}{\mathrm{d} l} + 2 T_\Omega \frac{\mathrm{d} \Omega}{\Omega \mathrm{d} l} = \frac{2}{3} T \frac{\mathrm{d} \rho}{\rho \mathrm{d} l} + \frac{4}{3} T_\Omega \frac{\mathrm{d} \rho}{\rho \mathrm{d} l} 
	\label{Eq.32}
\end{equation}
\\
Tidied up
\begin{equation}
	2 T_\Omega \left( \frac{\mathrm{d} \Omega}{\Omega \mathrm{d} l} - \frac{2}{3} \frac{\mathrm{d} \rho}{\rho \mathrm{d} l} \right) = \frac{2}{3} T \frac{\mathrm{d} \rho}{\rho \mathrm{d} l} - \frac{\mathrm{d} T}{\mathrm{d} l} 
	\label{Eq.33}
\end{equation}
\\
Substituting the \eneqref{Eq.16} into the \eneqref{Eq.33} ,we have
\begin{equation}
	a^2 = \frac
	{T \left( \frac{2}{3} \frac{\mathrm{d} \rho}{\rho \mathrm{d} l} - \frac{\mathrm{d} T}{T \mathrm{d} l} \right)}
	{\frac{1}{3} \frac{M_\mathrm{m}}{R_\mathrm{m}} \Omega^2 \left( \frac{\mathrm{d} \Omega}{\Omega \mathrm{d} l} - \frac{2}{3} \frac{\mathrm{d} \rho}{\rho \mathrm{d} l} \right)}
	\label{Eq.34}
\end{equation}
\\
It can be seen that the convection criterion is related to the size of the fluid cell. Given the temperature $T$ , relative temperature gradient  $\frac{\mathrm{d} T}{T \mathrm{d} l}$ , relative density gradient $\frac{\mathrm{d} \rho}{\rho \mathrm{d} l}$ , rotational speed  $\Omega$ and relative speed gradient $\frac{\mathrm{d} \Omega}{\Omega \mathrm{d} l}$ of the ambient fluid, the size of the fluid cell of the convection criterion can be expressed as  $a_\mathrm{ad}$ . Then the condition of the fluid cell size for spontaneous thermal convection can be judged by the new convection criterion $\left| \frac{\mathrm{d} T + \mathrm{d} T_\Omega}{\mathrm{d} l} \right| _\mathrm{rd} > \left| \frac{\mathrm{d} T + \mathrm{d} T_\Omega}{\mathrm{d} l} \right| _\mathrm{ad}$ . Generally, $\frac{\mathrm{d} \Omega}{\Omega \mathrm{d} l}$ and $\frac{\mathrm{d} T}{T \mathrm{d} l}$ are identical symbol with the $\frac{\mathrm{d} \rho}{\rho \mathrm{d} l}$ . The discussion is as follows:

(1)When $\left| \frac{\mathrm{d} T}{T \mathrm{d} l} \right| < \left| \frac{2}{3} \frac{\mathrm{d} \rho}{\rho \mathrm{d} l} \right|$ and $\left| \frac{\mathrm{d} \Omega}{\Omega \mathrm{d} l} \right| < \left| \frac{2}{3} \frac{\mathrm{d} \rho}{\rho \mathrm{d} l} \right|$ , Natural convection cannot occur;

(2)when $\left| \frac{\mathrm{d} T}{T \mathrm{d} l} \right| > \left| \frac{2}{3} \frac{\mathrm{d} \rho}{\rho \mathrm{d} l} \right|$ and $\left| \frac{\mathrm{d} \Omega}{\Omega \mathrm{d} l} \right| < \left| \frac{2}{3} \frac{\mathrm{d} \rho}{\rho \mathrm{d} l} \right|$ , The size form of the convection criterion is $a < a_ \mathrm{ad}$ ;

(3)when $\left| \frac{\mathrm{d} T}{T \mathrm{d} l} \right| < \left| \frac{2}{3} \frac{\mathrm{d} \rho}{\rho \mathrm{d} l} \right|$ and $\left| \frac{\mathrm{d} \Omega}{\Omega \mathrm{d} l} \right| < \left| \frac{2}{3} \frac{\mathrm{d} \rho}{\rho \mathrm{d} l} \right|$, The size form of the convection criterion is $a > a_\mathrm{ad}$;

(4)when $\left| \frac{\mathrm{d} T}{T \mathrm{d} l} \right| > \left| \frac{2}{3} \frac{\mathrm{d} \rho}{\rho \mathrm{d} l} \right|$ and $\left| \frac{\mathrm{d} \Omega}{\Omega \mathrm{d} l} \right| < \left| \frac{2}{3} \frac{\mathrm{d} \rho}{\rho \mathrm{d} l} \right|$, Convection is driven by both rotating gradient and temperature gradient at the same time. The size of the fluid cells is unrestricted.

\subsection{Size Limitation of Granule}

In the solar convection zone, the granule exists in the range of 5 to 16 minutes, which is much smaller than the solar rotation period. So the fluid cell expansion is close to isotropy in Statistics . The thermal properties of  rotation can be fully represented.

Convection in the solar convection zone is driven by the temperature gradient, so there is $\left| \frac{\mathrm{d} T}{T \mathrm{d} l} \right| > \left| \frac{2}{3} \frac{\mathrm{d} \rho}{\rho \mathrm{d} l} \right|$ . Under the vorticity transport  of rotating turbulent thermal convection, the relative totation gradient $\frac{\mathrm{d} \Omega}{\Omega \mathrm{d} l}$ Tend to $\frac{2}{3} \frac{\mathrm{d} \rho}{\rho \mathrm{d} l}$ , while the relative rotation gradient $\frac{\mathrm{d} \Omega}{\Omega \mathrm{d} l}$ affected by viscosity Tend to zero. Therefore, the final relative rotation gradient is between both, $\left| \frac{\mathrm{d} T}{T \mathrm{d} l} \right| > \left| \frac{2}{3} \frac{\mathrm{d} \rho}{\rho \mathrm{d} l} \right|$ , which satisfies the second case of the discussion. The convection criterion has the form of size  $a < a_\mathrm{ad}$ .

When $a < a_\mathrm{ad}$ , thermal convection occurs spontaneously; When $a > a_\mathrm{ad}$ , the fluid cell will have an equilibrium position, the disturbed fluid cell will oscillate near the equilibrium position and decelerate due to dissipation of viscous. natural convection can stimulate forced convection in rotating turbulent thermal convection, so the critical size $a_\mathrm{ad}$ is not the upper limit of the size of the fluid cell. But while $a > a_\mathrm{ad}$ ,the number of fluids cell decreases sharply with the increase of $a$ . This is reflected in the size distribution characteristics of granule.A large number of observations have shown that there is a critical diameter with the granule, which can be divided into mini granule and lage granule according to the diameter. The number of granule decreases rapidly with the increase of diameter. The number of mini grains increases monotonously or remains flat with the decrease of diameter. There are differences on average brightness, maximum brightness and fractal dimension between the two types of granule.  The observed data are consistent with the theory in this paper

In the solar convection zone, the distributions of temperature $T$ and relative temperature gradient $\frac{\mathrm{d} T}{T \mathrm{d} l}$ , density $\rho$ and relative density gradient $\frac{\mathrm{d} \rho}{\rho \mathrm{d} l}$ vary little with latitude, but the rotational speed $\Omega$ and relative rotation gradient $\frac{\mathrm{d} \Omega}{\Omega \mathrm{d} l}$ have significant differences with latitude. For ease of calculation, the rotational speed $\Omega$ of the fluid cell can be replaced by vorticity $\omega$ , the critical size of the fluid cell is
\begin{equation}
	a_\mathrm{ad} = \sqrt{ \frac
	{T \left( \frac{2}{3} \frac{\mathrm{d} \rho}{\rho \mathrm{d} l} - \frac{\mathrm{d} T}{T \mathrm{d} l} \right)}
	{\frac{1}{12} \frac{M_\mathrm{m}}{R_\mathrm{m}} \omega^2 \left( \frac{\mathrm{d} \omega}{\omega \mathrm{d} l} - \frac{2}{3} \frac{\mathrm{d} \rho}{\rho \mathrm{d} l} \right)}}
	\label{Eq.35}
\end{equation}
\\
The solar has significant differential rotation, so the influence of rotational speed $\Omega$ with latitude cannot be neglecte. while the observation data of relative vorticity gradient $\frac{\mathrm{d} \omega}{\omega \mathrm{d} l}$ is less. If the influence of $\frac{\mathrm{d} \omega}{\omega \mathrm{d} l}$ with latitude is neglected, the distribution of critical size with latitude can be expressed as
\begin{equation}
	a_\mathrm{ad} \left( \varphi \right) \approx \frac{{a_\mathrm{ad}}_0 \omega_0}{\omega \left( \varphi \right)}  
	\label{Eq.36}
\end{equation} 
\\
The solar rotates faster near the equator and slower near the pole, $\omega \left( \varphi \right)$ decreases with increasing latitude, so $a_\mathrm{ad}$ may increase with latitude.


\section{Conclusion}

The rotation of a fluid cell generates additional pressure and exhibits temperature-like properties in isotropic expansion. It can be expressed by the equivalent temperature of rotation or vorticity. The rotational equivalent temperature can influence the convection criteria. When the values of vorticity gradient and temperature gradient are within certain ranges, the size of the fluid cell can determine whether it is in a state of natural convection. Under the temperature gradient and vorticity gradient in the solar, Small fluid cells are in a state of natural convection while large fluid cells are in a forced convection state. With the critical size as the boundary, the number, brightness and fractal latitude of fluid cell have different distribution characteristics. This is validated in the observed data of granulation, which also infers that the critical  size is negatively correlated with the local average vorticity.




\bibliographystyle{IEEEtran}
\bibliography{reference.bib}

 \end{document}
%%%%%%%%%%%%%%%%%%%%%%%%%%%% END DOCUMENT %%%%%%%%%%%%%%%%%%%%%%%%%%%%
%%%%%%%%%%%%%%%%%%%%%%%%%%%%%%%%%%%%%%%%%%%%%%%%%%%%%%%%%%%%%%%%%%%%%%
