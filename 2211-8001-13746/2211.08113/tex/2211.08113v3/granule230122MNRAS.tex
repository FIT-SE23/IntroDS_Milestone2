% mnras_template.tex 
%
% LaTeX template for creating an MNRAS paper
%
% v3.0 released 14 May 2015
% (version numbers match those of mnras.cls)
%
% Copyright (C) Royal Astronomical Society 2015
% Authors:
% Keith T. Smith (Royal Astronomical Society)

% Change log
%
% v3.0 May 2015
%    Renamed to match the new package name
%    Version number matches mnras.cls
%    A few minor tweaks to wording
% v1.0 September 2013
%    Beta testing only - never publicly released
%    First version: a simple (ish) template for creating an MNRAS paper

%%%%%%%%%%%%%%%%%%%%%%%%%%%%%%%%%%%%%%%%%%%%%%%%%%
% Basic setup. Most papers should leave these options alone.
\documentclass[fleqn,usenatbib]{mnras}

% MNRAS is set in Times font. If you don't have this installed (most LaTeX
% installations will be fine) or prefer the old Computer Modern fonts, comment
% out the following line
\usepackage{newtxtext,newtxmath}
% Depending on your LaTeX fonts installation, you might get better results with one of these:
%\usepackage{mathptmx}
%\usepackage{txfonts}

% Use vector fonts, so it zooms properly in on-screen viewing software
% Don't change these lines unless you know what you are doing
\usepackage[T1]{fontenc}

% Allow "Thomas van Noord" and "Simon de Laguarde" and alike to be sorted by "N" and "L" etc. in the bibliography.
% Write the name in the bibliography as "\VAN{Noord}{Van}{van} Noord, Thomas"
\DeclareRobustCommand{\VAN}[3]{#2}
\let\VANthebibliography\thebibliography
\def\thebibliography{\DeclareRobustCommand{\VAN}[3]{##3}\VANthebibliography}


%%%%% AUTHORS - PLACE YOUR OWN PACKAGES HERE %%%%%

% Only include extra packages if you really need them. Common packages are:
\usepackage{amsmath}% ams mathematical formulas  
%\usepackage{amssymb}% ams mathematical symbols 
\usepackage{esint} 
%\usepackage{bm}% bold math symbols
\usepackage{siunitx}% si units      
%\usepackage{cite}% numeric citation
\usepackage{lmodern}% latin modern family of fonts
%%%%%%%%%%%%%%%%%%%%%%%%%%%%%%%%%%%%%%%%%%%%%%%%%%

%%%%% AUTHORS - PLACE YOUR OWN COMMANDS HERE %%%%%

% Please keep new commands to a minimum, and use \newcommand not \def to avoid
% overwriting existing commands. Example:
%\newcommand{\pcm}{\,cm$^{-2}$}	% per cm-squared

%%%%%%%%%%%%%%%%%%%%%%%%%%%%%%%%%%%%%%%%%%%%%%%%%%

%%%%%%%%%%%%%%%%%%% TITLE PAGE %%%%%%%%%%%%%%%%%%%

% Title of the paper, and the short title which is used in the headers.
% Keep the title short and informative.
\title[Rotation Effects on Solar Granule]{How Dose Rotation Effect Convection Criterion and Solar Granule Size}

% The list of authors, and the short list which is used in the headers.
% If you need two or more lines of authors, add an extra line using \newauthor
\author[Haibin Chen et al.]{
	Haibin Chen,
	Rong Wu\thanks{wurong2@mail3.sysu.edu.cn}
	\\
	% List of institutions
    School of Aeronautics and Astronautics, Sun Yat-sen University, Xin Gang Road W, Guangzhou 510275, China\\
}

% These dates will be filled out by the publisher
\date{Accepted XXX. Received YYY; in original form ZZZ}

% Enter the current year, for the copyright statements etc.
\pubyear{2023}

% Don't change these lines
\begin{document}
	\label{firstpage}
	\pagerange{\pageref{firstpage}--\pageref{lastpage}}
	\maketitle
	
	% Abstract of the paper
	\begin{abstract}
		In this paper, based on the spherically symmetric expansion hypothesis of solar granule, considering the effects of rotation factor on the rotational energy and pressure of granules and making comparisons to molecular thermal motion and ideal gas law, a rotating equivalent temperature hypothesis is proposed and a new convection criterion associated with rotational speed is obtained. Through the analysis of the solar convection zone, it can be found that there is a critical size of granules and the rotation makes the granules with small diameter in natural convection state, while the granules with large diameter in forced convection state. This conclusion can be verified by the observation data of granulation.
%The expantion of solar granules is isotropic instatistical. Spherical symmetry expantion of granules will change rotation kinetic energy. There should be additional press provide by rotation and effect convection criterion. The effects of rotation can be expressed with equivalent temperature.The bigger the granules the stronger the effect. It makes the small granules in natural convection and large granules in forced convection distinguish by critical diameter.
	\end{abstract}
	
	% Select between one and six entries from the list of approved keywords.
	% Don't make up new ones.
	\begin{keywords}
		convection -- turbulence -- sun:granulation -- sun:interior
	\end{keywords}
	
	%%%%%%%%%%%%%%%%%%%%%%%%%%%%%%%%%%%%%%%%%%%%%%%%%%
	
	%%%%%%%%%%%%%%%%% BODY OF PAPER %%%%%%%%%%%%%%%%%%
	
	\section{Introduction}
	
	Granulation is the most obvious structure on the solar surface besides sunspots. 
	A large number of observations \citep{Bray1977,Roudier1986,Berrilli2002,Yu2011} show that, there is a critical diameter in granulation, which is the basis for dividing large granules and small granules. 
	The number of large granules decreases rapidly with the increase of diameter, while the number of small granules increases monotonously or remains flat with the decrease of diameter. 
	Hirzberger's research \citep{Hirzberger1997,Hirzberger1999} shows that there are differences in the average brightness, maximum brightness and fractal dimension of granules with the diameter of $1 ''. 37 $ as the boundary.
	
	Granulation is considered as evidence of the solar interior convection. 
	Schwarzschild criterion indicates the judging condition of the irrotational fluid convection driven by the temperature gradient under the adiabatic condition. 
	The phenomenon that the number of large granules decreases with the increase of diameter suggests that there may be some mechanism to inhibit convection, and the larger the granules, the stronger the inhibition.
	When studying the solar differential rotation \citep{Chen2022}, the author found that the rotational energy would change during the spherically symmetric expansion of the fluid cells in the rotating turbulent thermal convection, and the change of the rotational energy could be considered as the work of the additional pressure generated by the rotation. 
	The pressure caused by the rotation further effects the density, buoyancy and equilibrium state of granules, and changes the convection criterion. 
	
	\section{Thermal Properties of Rotating Solar Granule}
	
	\subsection{Spherical symmetrical expansion hypothesis of solar granule}
	
	Due to the high Reynolds number in the solar convection zone, random motion dominates the flow of solar granules and thermal convection and rotation can be supplemented. 
	At this point, a granule can be simplified as a "molecule" moving irregularly with a size and rotation, and can expand or be compressed as the external pressure changes.
	This assumption abandons the complex thermal convection models and turbulence models and is more advantageous to study some essential properties of rotating turbulent thermal convection.
	
	The solar rotation period $T_ 0$ is about 27 days, while the existence time of granules is about 6 minutes, so there is a gap of 4 or 5 orders of magnitude between them.
	In the process of a granule expansion, there may be a rotational speed difference between it and the ambient fluid. 
	According to the propagation law of the inertial wave of the rotating fluid, the rotational speed difference of the granules will be transmitted to the outside by the inertial wave with a period of $\frac{T_0}{2}$, which means that the rotational speed difference generated by the expansion of a granule during its existence time is stuck in the granule, and the expansion shape can be considered independent of the rotation. 
	In the absence of other effects, the expansion of granulation is statistically close to isotropic. For a single granule, it can be assumed that its expansion is spherically symmetrical.
	
	\subsection{Rotational Energy Change and Additional Pressure}
	
	Ignoring viscosity, the rotational energy of a granule changes during spherically symmetric expansion. 
	According to the conservation of angular momentum, the relationship between the rotational speed $\Omega$ and size $a$ of the granule is
	\begin{equation}\label{Eq.1}
		\Omega = \Omega_0 \left( \frac{a}{a_0} \right) ^{-2},
	\end{equation}
	where $\Omega_0$ and $a_0$ are the initial rotational speed and size, and the differential form of the above equation is
	\begin{equation}\label{Eq.2}
		\frac{\mathrm{d} \Omega}{\Omega} = -2 \frac{\mathrm{d} a}{a} .
	\end{equation}
	
	For example, the rotational kinetic energy of a cylindrical granule with a radius of $a$, a height of $2a$ and a rotational speed of $\Omega$ is
	\begin{equation}\label{Eq.3}
		E_\Omega = \frac{1}{2} \pi \rho \Omega^2 a^5   
	\end{equation}
	and can also be expressed as
	\begin{equation}\label{Eq.4}
		E_\Omega = \frac{1}{4} M \Omega^2 a^2   ,
	\end{equation}
	where $\rho$ and $M$ are the density and mass of the granule espectively. 
	It can be calculated that the change of the rotational energy of the granule during spherically symmetric expansion is
	\begin{equation}\label{Eq.5}
		\frac{\mathrm{d} E_\Omega}{E_\Omega} = -2 \frac{\mathrm{d} a}{a} .
	\end{equation}
	The change of rotational energy of the granule requires the rotation to provide additional pressure $p_\Omega$ to do work,
	that is, the work done by the additional pressure is
	\begin{equation}\label{Eq.6}
		\mathrm{d} E_\Omega = - \oiint p_\Omega \mathrm{d} l \mathrm{d} S ,
	\end{equation}
	and the total volume change of the granule is
	\begin{equation}\label{Eq.7}
		\mathrm{d} V = \oiint \mathrm{d} l \mathrm{d} S .
	\end{equation}
	Calculate the average pressure on the surface of the granule for the additional pressure provided by the rotation, letting
	\begin{equation}\label{Eq.8}
		\bar{p}_\Omega = - \frac{\mathrm{d} E_\Omega}{\mathrm{d} V} ,
	\end{equation}
	according to $\frac{\mathrm{d} V }{V} = 3 \frac{\mathrm{d} a }{a}$ and $V = 2 \pi a^3$, then there is
	\begin{equation}\label{Eq.9}
		\bar{p}_\Omega = \frac{1}{6} \rho \Omega^2 a^2.
	\end{equation}
	Similarly, for other shapes of granules, the additional pressure provided by rotation can be obtained as
	\begin{equation}\label{Eq.10}
		\bar{p}_\Omega = k_\Omega \rho \Omega^2 a^2,
	\end{equation}
	where $k_\Omega$ is a constant related to the shape of the granule. Different shapes, $k_\Omega$ has different values.
	
	Since the solar rotation period is several orders of magnitude longer than the existence time of the granules, the rotational speed difference between the granule and the ambient fluid is allowed to exist. The local pressure can be unbalanced, while the overall pressure is in transient equilibrium.
	.
	\subsection{Rotating equivalent temperature hypothesis}
	
	The rotation of the granule not only effects the pressure on the surface, but also increases the overall energy, which is similar to the relationship among temperature, pressure and molecular kinetic energy. It is probablely easier to illustrate the effects of the rotation by analogy with molecular thermal motion.
	
	According to ideal gas law, the pressure provided by molecular thermal motion is $p = \frac{R_\mathrm{m}}{M_\mathrm{m}} \rho T$ , where $R_\mathrm{m}$ is the molar gas constant and $M_\mathrm{m}$ is the molar mass of the gas. The additional pressure generated by rotation can also be expressed in this form, that is
	\begin{equation}\label{Eq.11}
		\bar{p}_\Omega = \frac{R_\mathrm{m}}{M_\mathrm{m}} \rho T_\Omega   ,
	\end{equation}
	where $T_\Omega$ is an equivalent temperature corresponding to the additional pressure of rotation, and rotating equivalent temperature $T_\Omega $ can be expressed as
	\begin{equation}\label{Eq.12}
		T_\Omega = \frac{1}{6} \frac{M_\mathrm{m}}{R_\mathrm{m}} k_\Omega \Omega^2 a^2   .
	\end{equation}
	
	Following the calculation of the specific heat capacity at constant volume in thermology, the change of rotational energy $\mathrm{d} E_\Omega$ of a rotating cylindrical granule caused by the change of rotating equivalent temperature $\mathrm{d} T_\Omega$ during spherically symmetric expansion is 
	\begin{equation}\label{Eq.13}
		\mathrm{d} E_\Omega = \frac{3}{2} \frac{M}{M_\mathrm{m}} R_\mathrm{m} \mathrm{d} T_\Omega   , 
	\end{equation}
	and the equivalent specific kinetic energy capacity at constant volume $C_{V_\Omega}$ of the rotating cylindrical granule is
	\begin{equation}\label{Eq.14}
		C_{V_\Omega} = \frac{3}{2} R_\mathrm{m}  , 
	\end{equation}
	which is the same as the specific heat capacity at constant volume of monoatomic gases, indicating that they have similar properties. 
	Under the rotating equivalent temperature hypothesis, the spherically symmetric expansion of a rotating granule is very similar to the adiabatic expansion of an ideal gas, which can be considered as the adiabatic expansion of an ideal gas with a degree of freedom of 3 and a specific heat ratio of $\frac{5}{3}$.
	
	\section{Convection Criteria Effected by Rotation}
	
	When the pressure of the some region is determined, the additional pressure generated by the rotation of the granules in the region influences the density of the granules, which effects the balance of the granules and changes the convection criterion.
	
	The Schwarzschild criterion for irrotational granule is
	\begin{equation}\label{Eq.15}
		\left| \frac{\mathrm{d} T}{\mathrm{d} l} \right| _{\mathrm{rd}} > \left| \frac{\mathrm{d} T}{\mathrm{d} l} \right| _{\mathrm{ad}} ,
	\end{equation}
	where $\left( \frac{\mathrm{d} T}{\mathrm{d} l} \right) _{\mathrm{rd}}$ is the real temperature gradient of the granule, $\left( \frac{\mathrm{d} T}{\mathrm{d} l} \right) _{\mathrm{ad}}$ is the adiabatic temperature gradient and can be expressed as
	\begin{equation}\label{Eq.16}
		\left( \frac{\mathrm{d} T}{\mathrm{d} l} \right) _{\mathrm{ad}} = \left( 1-\frac{1}{\gamma} \right) T \frac{\mathrm{d} p}{p \mathrm{d} l} ,
	\end{equation}
	or
	\begin{equation}\label{Eq.17}
		\left( \frac{\mathrm{d} T}{\mathrm{d} l} \right) _{\mathrm{ad}} = \left( \gamma - 1 \right) T \frac{\mathrm{d} \rho}{\rho \mathrm{d} l} .
	\end{equation}
	Considering the effect of rotating equivalent temperature, criterion of rotating turbulent thermal convection is
	\begin{equation}\label{Eq.18}
		\left| \frac{\mathrm{d} T  \, \text{+} \, \mathrm{d} T_\Omega}{\mathrm{d} l} \right| _{\mathrm{rd}} > \left| \frac{\mathrm{d} T  \, \text{+} \,  \mathrm{d} T_\Omega}{\mathrm{d} l} \right| _{\mathrm{ad}} ,
	\end{equation}
	in which the rotating equivalent adiabatic temperature gradient is
	\begin{equation}\label{Eq.19}
		\left( \frac{\mathrm{d} T_\Omega}{\mathrm{d} l} \right) _{\mathrm{ad}} = \frac{2}{3} T_\Omega \frac{\mathrm{d} \rho}{\rho \mathrm{d} l} = T_\Omega \left( 2 \frac{\mathrm{d} \Omega }{\Omega \mathrm{d} l} \, \text{+} \, 2 \frac{\mathrm{d} a }{a \mathrm{d} l} \right) .
	\end{equation}
	For a single granule, because of the mass conservation, $\frac{\mathrm{d} a }{a \mathrm{d} l} = - \frac{1}{3} \frac{\mathrm{d} \rho }{\rho \mathrm{d} l}$, then we get
	\begin{equation}\label{Eq.20}
		\frac{\mathrm{d} \Omega}{\Omega \mathrm{d} l} = \frac{2}{3} \frac{\mathrm{d} \rho }{\rho \mathrm{d} l} ,
	\end{equation}
	which is the same as the relationship between rotational speed and density of inviscid rotating granule in the process of isotropic expansion. So the rotating equivalent adiabatic temperature gradient can also be described by the adiabatic rotational speed gradient, that is, 
	\begin{equation}\label{Eq.21}
		\left( \frac{\mathrm{d} \Omega}{\Omega \mathrm{d} l} \right) _\mathrm{ad} = \frac{2}{3} \frac{\mathrm{d} \rho }{\rho \mathrm{d} l} .
	\end{equation}
	
	Since the two temperature gradients are different but interact with each other, the convection criterion does not need to strictly satisfy $\left| \frac{\mathrm{d} T}{\mathrm{d} l} \right| _{\mathrm{rd}} > \left| \frac{\mathrm{d} T}{\mathrm{d} l} \right| _{\mathrm{ad}}$ and $\left| \frac{\mathrm{d} T_\Omega}{\mathrm{d} l} \right| _{\mathrm{rd}} > \left| \frac{\mathrm{d} T_\Omega}{\mathrm{d} l} \right| _{\mathrm{ad}}$, that is, convection does not necessarily need to be driven simultaneously by both temperature gradient and rotational speed gradient.  When one of them is satisfied, the convection criterion ( \ref{Eq.18} ) can be met by limiting the size of the granules. According to the critical condition of convection criterion  ( \ref{Eq.18} ) , the critical size of the granules can be calculated as
	\begin{equation}\label{Eq.22}
		a^2_\mathrm{ad} = - \frac
		{T \left[ \frac{\mathrm{d} T}{T \mathrm{d} l} - \left( \gamma - 1 \right) \frac{\mathrm{d} \rho}{\rho \mathrm{d} l} \right]}
		{\frac{k_\Omega M_m}{2 R_m} \Omega^2 \left( \frac{\mathrm{d} \Omega}{\Omega \mathrm{d} l} - \frac{2}{3} \frac{\mathrm{d} \rho}{\rho \mathrm{d} l} \right)} .
	\end{equation}
	The discriminant of the new convection criterion can be discussed in several cases:
	
	Case 1: When $\left| \frac{\mathrm{d} T}{T \mathrm{d} l} \right| < \left| \left( \gamma - 1 \right) \frac{\mathrm{d} \rho}{\rho \mathrm{d} l} \right|$ and $\left| \frac{\mathrm{d} \Omega}{\Omega \mathrm{d} l} \right| < \left| \frac{2}{3} \frac{\mathrm{d} \rho}{\rho \mathrm{d} l} \right|$, natural convection can not occur;
	
	Case 2: When $\left| \frac{\mathrm{d} T}{T \mathrm{d} l} \right| > \left| \left( \gamma - 1 \right) \frac{\mathrm{d} \rho}{\rho \mathrm{d} l} \right|$ and $\left| \frac{\mathrm{d} \Omega}{\Omega \mathrm{d} l} \right| < \left| \frac{2}{3} \frac{\mathrm{d} \rho}{\rho \mathrm{d} l} \right|$, the convection is driven by the temperature gradient and suppressed by the rotational speed gradient, and the description of convection criterion by the granule size is $a < a_\mathrm{ad}$; 
	
	Case 3: When $\left| \frac{\mathrm{d} T}{T \mathrm{d} l} \right| < \left| \left( \gamma - 1 \right) \frac{\mathrm{d} \rho}{\rho \mathrm{d} l} \right|$ and $\left| \frac{\mathrm{d} \Omega}{\Omega \mathrm{d} l} \right| > \left| \frac{2}{3} \frac{\mathrm{d} \rho}{\rho \mathrm{d} l} \right|$, the convection is driven by the  the rotational speed gradient and suppressed by temperature gradient, and the description of convection criterion by the granule size is $a > a_\mathrm{ad}$; 
	
	Case 4: When $\left| \frac{\mathrm{d} T}{T \mathrm{d} l} \right| > \left| \left( \gamma - 1 \right) \frac{\mathrm{d} \rho}{\rho \mathrm{d} l} \right|$ and $\left| \frac{\mathrm{d} \Omega}{\Omega \mathrm{d} l} \right| > \left| \frac{2}{3} \frac{\mathrm{d} \rho}{\rho \mathrm{d} l} \right|$, the convection is driven by both temperature gradient and rotational speed gradient, and the granule size of the convection criterion is unlimited. 
	
	The deduction of the critical size of the granules and discriminant of convevtion criterion is valid when $\frac{\mathrm{d} T}{T \mathrm{d} l}$ and $\frac{\mathrm{d} \Omega}{\Omega \mathrm{d} l}$ have the same sign as $\frac{\mathrm{d} \rho}{\rho \mathrm{d} l}$. When there is a different sign among them, special cases need to be discussed specifically. 
	
	\section{Solar Granule Size Effected by Rotation}
	
	Thermal convection in the solar convection zone is driven by temperature gradient which meets $\left| \frac{\mathrm{d} T}{T \mathrm{d} l} \right| > \left| \left( \gamma - 1 \right) \frac{\mathrm{d} \rho}{\rho \mathrm{d} l} \right|$ radially, so the convection criterion is limited to Case 2 or Case 4. According to the observation, the rotational speed gradient in the solar convection zone is generally low, that is, most regions meet $\left| \frac{\mathrm{d} \Omega}{\Omega \mathrm{d} l} \right| < \left| \frac{2}{3} \frac{\mathrm{d} \rho}{\rho \mathrm{d} l} \right|$. At this time, the convection criterion meets  Case 2, the description by the granule size is $a < a_\mathrm{ad}$, three gradient terms are all negative signs.
	
	In Case 2, when $a < a_\mathrm{ad}$, thermal convection occurs spontaneously;
	when $a > a_\mathrm{ad}$, the granule will have an equilibrium position, and the disturbed granule will oscillate near the equilibrium position and decelerate under the dissipation of viscosity, heat and rotational energy. 
	In rotating turbulent thermal convection, natural convection of small granules can stimulate forced convection of large granules, so the critical size $a_\mathrm{ad}$ is not the upper limit of the size at which convection occurs, but the relative number of granules with $a > a_\mathrm{ad}$ decreases rapidly with the increase of $a$.
	This is consistent with the size distribution characteristics of granules, that is, with the critical size as the boundary, the number, brightness and fractal dimension of granules have different distribution characteristics, and the number of large granules decreases rapidly with the increase of diameter, while the number of small granules increases monotonously  or keeps flat with the decrease of diameter.
	
	The rotational speed gradient in part of the solar convection zone may rise under the influence of other factors, leading to $\left| \frac{\mathrm{d} \Omega}{\Omega \mathrm{d} l} \right| < \left| \frac{2}{3} \frac{\mathrm{d} \rho}{\rho \mathrm{d} l} \right|$. At this time, the convection criterion satisfies Case 4, where the size of the granules is unlimited and the convection is aggravated, which may correspond to some phenomena in the solar active region, such as Coronal mass ejection.
	
	The above discussions are based on the same sign of rotational speed gradient and density gradient, but in some regions of the solar convection zone, rotational speed gradient and density gradient are different signs. Here, rotational speed gradient inhibits convection, which corresponds to Case 2 of convection criterion.
	
	\section{Conclusions}
	
	The rotation of granules will generate additional pressure, and show the properties similar to temperature in the spherically symmetric expansion, which can be expressed by the rotating equivalent temperature. 
	The rotating equivalent temperature will effect the convection criterion, and the rotational speed gradient and temperature gradient will jointly effect the convection. The critical size of the granules can be used to determine whether the granules are in natural convection. 
	In the solar quiet region, the small granules are in natural convection, while the large granules are in forced convection.
	The observation data confirmed that the number of large granules decreases rapidly with the increase of diameter, while the number of small granules monotonously increases or remains flat with the decrease of diameter.
	
	%\section*{Acknowledgements}
	
	%
	%The Acknowledgements section is not numbered. Here you can thank helpful
	%colleagues, acknowledge funding agencies, telescopes and facilities used etc.
	%Try to keep it short.
	
	%%%%%%%%%%%%%%%%%%%%%%%%%%%%%%%%%%%%%%%%%%%%%%%%%%
	\section*{Data Availability}
	The data underlying this article are available in the article and in its online supplementary material.
	%
	% 
	%The inclusion of a Data Availability Statement is a requirement for articles published in MNRAS. Data Availability Statements provide a standardised format for readers to understand the availability of data underlying the research results described in the article. The statement may refer to original data generated in the course of the study or to third-party data analysed in the article. The statement should describe and provide means of access, where possible, by linking to the data or providing the required accession numbers for the relevant databases or DOIs.
	
	
	
	
	%%%%%%%%%%%%%%%%%%%% REFERENCES %%%%%%%%%%%%%%%%%%
	
	% The best way to enter references is to use BibTeX:
	
	\bibliographystyle{mnras}
	\bibliography{referencegranule} % if your bibtex file is called example.bib
	
	
	% Alternatively you could enter them by hand, like this:
	% This method is tedious and prone to error if you have lots of references
	%\begin{thebibliography}{99}
	%\bibitem[\protect\citeauthoryear{Author}{2012}]{Author2012}
	%Author A.~N., 2013, Journal of Improbable Astronomy, 1, 1
	%\bibitem[\protect\citeauthoryear{Others}{2013}]{Others2013}
	%Others S., 2012, Journal of Interesting Stuff, 17, 198
	%\end{thebibliography}
	
	%%%%%%%%%%%%%%%%%%%%%%%%%%%%%%%%%%%%%%%%%%%%%%%%%%
	
	%%%%%%%%%%%%%%%%% APPENDICES %%%%%%%%%%%%%%%%%%%%%
	
	%\appendix
	
	%\section{Some extra material}
	%
	%If you want to present additional material which would interrupt the flow of the main paper,
	%it can be placed in an Appendix which appears after the list of references.
	
	%%%%%%%%%%%%%%%%%%%%%%%%%%%%%%%%%%%%%%%%%%%%%%%%%%
	
	
	% Don't change these lines
	\bsp	% typesetting comment
	\label{lastpage}
\end{document}

% End of mnras_template.tex
