\chapter{Introduction}
\label{sec:intro}

Motion planning is the problem of finding valid paths, expressed as sequences of configurations, or trajectories, expressed as sequences of controls, which move a robot from a given start state to a desired goal state while avoiding obstacles.  It has applications in problems ranging from mobile robotics \citep{Barraquand_1991}, manipulation planning \citep{OM-2005}, graphics and animation \citep{KAAT-2008}, protein folding \citep{SA-2001} to crowd simulation \citep{bayazit2002better,sud_2008,Toll-2012} and multi-robot applications \citep{svestka-1995,Clark2001RandomizedMP}.  Variations of the motion planning problem can include dynamic constraints, which can be important in autonomous driving and aerial vehicles (Figure~\ref{fig:introduction}).  

\begin{figure}[t!]
    \centering
    \begin{subfigure}{\linewidth}
    \centering
    \includegraphics[width=5.0cm,height=5.0cm]{images/fig_1_manipulation_3.PNG}
    \includegraphics[width=5.0cm,height=5.0cm]{images/fig_1_grasping_3.PNG}
    \end{subfigure} \\
    \vspace{0.1cm}
    \begin{subfigure}{\linewidth}
    \centering
    % \includegraphics[width=0.49\linewidth]{images/fig1_omnirobot.jpeg}
    \includegraphics[width=5.0cm,height=5.0cm]{images/fig1_omnirobot_square.jpg}
    \includegraphics[width=5.0cm,height=5.0cm]{images/aerial_vehicle.png}
    \end{subfigure} \\
    \vspace{-.05in}
    \caption{\textbf{Example domains where machine learning has been integrated with sampling-based motion planning:}  a) grasping objects with robotic arms (image from \cite{chamzas2019using}),  b) dexterous manipulation with adaptive hands (image from \cite{9196564}),
     c) robot navigation of ground robots (image from \cite{omnirobot}),
     d) aerial vehicles (image from \cite{FPCFT-2017}).}   
    \vspace{-.25in}
    \label{fig:introduction}
\end{figure}

% \begin{figure}[t!]
%     \centering
%     \begin{subfigure}{\linewidth}
%     \includegraphics[width=0.49\linewidth]{images/fig_1_manipulation_3.PNG}
%     \includegraphics[width=0.49\linewidth]{images/fig_1_grasping_3.PNG}
%     \end{subfigure} \\
%     \begin{subfigure}{\linewidth}
%     % \includegraphics[width=0.49\linewidth]{images/fig1_omnirobot.jpeg}
%     \includegraphics[width=0.49\linewidth]{images/fig1_omnirobot_square.jpg}
%     \includegraphics[width=0.49\linewidth]{images/aerial_vehicle.png}
%     \end{subfigure} \\
%     \vspace{-.05in}
%     \caption{\textbf{Example domains where machine learning has been integrated with sampling-based motion planning:}  a) grasping objects with robotic arms (image from \citep{chamzas2019using}),  b) dexterous manipulation with adaptive hands (image from \citep{9196564}),
%      c) robot navigation of ground robots (image from \citep{omnirobot}),
%      d) aerial vehicles (image from \citep{FPCFT-2017}).}   
%     \vspace{-.2in}
%     \label{fig:introduction}
% \end{figure}

The motion planning problem is $\texttt{PSPACE-Complete}$ \citep{reif1979complexity,canny1988complexity,latombe1991robot}, and its complexity depends exponentially on the number of degrees of freedom of the robotic system.  This makes traditional, complete methods difficult to apply for problems with more than 4 or 5 degrees of freedom.  This has motivated work on developing sampling-based methods, which often use random samples to explore the underlying configuration space. Examples of popular Sampling-Based Motion Planners (\sbmps) include the Probabilistic Roadmap ({\tt PRM}) \citep{KSLO-1996}, the Rapidly-exploring Random Tree ({\tt RRT}) \citep{L-1998} and the Expansive Spaces \citep{hsu1997path} algorithms.  Sampling-based motion planners give up on the traditional notion of completeness and instead aim for probabilistic completeness, which means that they are guaranteed to eventually discover a solution if one exists but cannot confirm solution non-existence.  Progress in the field has also allowed the development of methods, such as {\tt RRT$^*$} and {\tt PRM$^*$} \citep{KF-2011}, which are also asymptotically optimal, i.e., they guarantee convergence to an optimal solution if one exists.

Beyond their algorithmic properties, these methods have proven quite effective in finding solutions for relatively high-dimensional challenges, where the traditional approaches do not scale. Their popularity also stems from the fact that they provide flexible frameworks, which are rather straightforward to implement and adapt for a large variety of robotic systems. Nevertheless, they may still face challenges as the complexity of the underlying planning problem increases:

\begin{myenum}
\item[1.] Some of the challenges relate to \emph{computational efficiency}, which may be hindered from the exploration of the underlying configuration space via sampling, especially when the key primitives of these planners, such as collision checking or forward propagation of the system's dynamics, are computationally expensive. 
\item[2.] Other issues relate to \emph{path quality}. Despite the progress in understanding the conditions for asymptotic optimality, convergence to high-quality solutions may be hindered in practice when naive exploration primitives are employed, such as the random sampling of controls. 
\item[3.] Furthermore, \sbmps\, like most motion planning methods, typically assume the \emph{availability of an accurate, complete model}. Traditional, engineered models may be inaccurate or unable to express all critical physical aspects of the problem or not predict how a dynamic environment may evolve. Furthermore, sensing constraints may introduce partial observability and uncertainty about the environment. These factors can limit the applicability of \sbmps. 
\end{myenum}

\noindent These challenges motivate the use of machine learning to improve the computational efficiency of \sbmps, accelerate their practical convergence to high-quality solutions, and provide access to accurate-enough, data-driven models, which adapt to varying environmental conditions and sensing input.

Machine learning enables the autonomous derivation of solutions to problems based on prior experience and data.  It promises to constantly improve performance by incorporating new data and identifying solutions that engineered approaches may not be able to achieve. This makes machine learning especially useful for an application like robotics, where a robotic agent must contend with an endless variety of tasks and environments.  
There are also many scenarios where learned agents can approximate costly computations.  In these cases, the learned agent can be trained to model the computations in a pre-processing step, then used during run-time in place of the computation or as a heuristic for it.  This is especially useful in robotics, where the robot must act and react to situations in real-time.         

Fundamentally, machine learning algorithms operate by building a model of observed data, that can predict and generalize to new examples. This data can come from various sources: an existing dataset, through human training or demonstration, or it can be accumulated from the results of previous predictions that the model has made. For model-based agents, learning is done by fitting the model's parameters to data, which can be done using methods such as regression or reinforcement learning.  During query time, the model is queried to give predictions based on patterns observed in the data.     

There is a large body of literature on the application of machine learning algorithms to improve the efficiency of robotic systems in general \citep{kbp-2013, kroemer2019review}. Recently, there has been a lot of attention on the progress of deep learning methods, which has resulted in many efforts to utilize the corresponding tools in robotics \citep{sunderhauf2018limits}. This survey focuses specifically on integrating machine learning tools to improve the efficiency, convergence, and applicability of \sbmps.

This survey covers a wide breadth of robotic applications, including, but not limited to, mobile robot navigation, manipulation planning, and planning for systems with dynamic constraints. In particular, this monograph first reviews the attempts to use machine learning to improve the performance of individual primitives used by \sbmps (Section~\ref{sec:learning-primitives}). It also studies a series of planners that use machine learning to adaptively select from a set of motion planning primitives. It then proceeds to study a series of integrated architectures that learn an end-to-end mapping of sensor inputs to robot trajectories or controls (Section~\ref{sec:integrated}). Finally, it studies how \sbmps\ can operate over learned models of robotic system that account for noise and uncertainty (Section~\ref{sec:planning_under_uncertainties}).

 

%However, to focus on applications to \sbmps, these applications are limited to problems in which the environment is known, and the robot's motions are assumed to be noiseless.

The survey concludes with a comparative discussion of the different approaches covered in Sections~\ref{sec:learning-primitives}~-~\ref{sec:planning_under_uncertainties}.  It evaluates these approaches in terms of their impact on computational efficiency of the planner, quality of the computed paths, and their overall applicability.  It then outlines the broad difficulties and limitations of these methods, as well as potential directions of future work.

% This survey also gives a comparative discussion of the approaches covers.  In this discussion it evaluates the impact of learned methods on computational efficiency of the planner as well as their impact on path quality.  It also discusses how learned models make SBMP methods easier to use.  It then discusses some of the difficulties and limitations facing existing methods.  Finally, it discusses potential future directions of work.


\begin{comment}
In order to focus on applications to SBMP, we will limit this survey to work addressing problems in which the environment is known and the motions of the robot are assumed to be noiseless.  We are aware of a great deal of work addressing issues such as sensor noise, localization (e.g. SLAM), noisy motions and/or unknown environments, however these are outside of the scope of this survey.  We also do not discuss  how machine learning has been applied to the areas of perception and task planning.  Due to space constants we will highlight a relevant subset of methods while only briefly mentioning other methods.
\end{comment}

    
