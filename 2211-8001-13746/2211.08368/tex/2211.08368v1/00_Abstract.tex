Sampling-based methods are widely adopted solutions for robot motion planning. The methods are straightforward to implement, effective in practice for many robotic systems. It is often possible to prove that they have desirable properties, such as probabilistic completeness and asymptotic optimality. Nevertheless, they still face challenges as the complexity of the underlying planning problem increases, especially under tight computation time constraints, which impact the quality of returned solutions or given inaccurate models. This has motivated machine learning to improve the computational efficiency and applicability of Sampling-Based Motion Planners (\sbmps). This survey reviews such integrative efforts and aims to provide a classification of the alternative directions that have been explored in the literature. It first discusses how learning has been used to enhance key components of \sbmps, such as node sampling, collision detection, distance or nearest neighbor computation, local planning, and termination conditions. Then, it highlights planners that use learning to adaptively select between different implementations of such primitives in response to the underlying problem's features.  It also covers emerging methods, which build complete machine learning pipelines that reflect the traditional structure of \sbmps. It also discusses how machine learning has been used to provide data-driven models of robots, which can then be used by a \sbmp. Finally, it provides a comparative discussion of the advantages and disadvantages of the approaches covered, and insights on possible future directions of research. An online version of this survey can be found at: \url{https://prx-kinodynamic.github.io/}