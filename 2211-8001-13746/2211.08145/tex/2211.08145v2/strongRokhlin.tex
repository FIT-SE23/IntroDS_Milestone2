\documentclass[11pt]{amsart}
\usepackage{amsfonts,amscd,latexsym,amsmath,amssymb,enumerate,verbatim,amsthm,epsfig,alltt,color}
\usepackage[shortlabels]{enumitem}


\newcommand{\Homeo}{\mathrm{Homeo}}
\newcommand{\Act}{\mathrm{Act}}

\newcommand{\Rat}{\mathbb{Q}}
\newcommand{\Rea}{\mathbb{R}}
\newcommand{\Nat}{\mathbb{N}}
\newcommand{\Int}{\mathbb{Z}}

\newcommand{\sep}{\mathrm{sep}}
\newcommand{\topo}{\mathrm{top}}
\newcommand{\Map}{\mathrm{Map}}
\newcommand{\Res}{\mathrm{Res}}


\newcommand{\FF}{\mathcal{F}}
\newcommand{\GG}{\mathcal{G}}
\newcommand{\XX}{\mathcal{X}}
\newcommand{\YY}{\mathcal{Y}}
\newcommand{\ZZ}{\mathcal{Z}}
\newcommand{\AAA}{\mathcal{A}}
\newcommand{\BB}{\mathcal{B}}
\newcommand{\NN}{\mathcal{N}}
\newcommand{\UU}{\mathcal{U}}
\newcommand{\SH}{\mathcal{S}}
\newcommand{\CC}{\mathcal{C}}
\newcommand{\DD}{\mathcal{D}}
\newcommand{\QQ}{\mathcal{Q}}
\newcommand{\PP}{\mathcal{P}}
\newcommand{\RR}{\mathcal{R}}
\newcommand{\WW}{\mathcal{W}}
\newcommand{\VV}{\mathcal{V}}
\newcommand{\SSS}{\mathcal{S}}
\newcommand{\EE}{\mathcal{E}}

\theoremstyle{plain}


\newtheorem{theorem}{\bf Theorem}[section]
\newtheorem{lemma}[theorem]{\bf Lemma}
\newtheorem{proposition}[theorem]{\bf Proposition}
\newtheorem{corollary}[theorem]{\bf Corollary}
\newtheorem{fact}[theorem]{\bf Fact}
\newtheorem{claim}[theorem]{\bf Claim}

\newtheorem{thmx}{Theorem}
\renewcommand{\thethmx}{\Alph{thmx}}

\theoremstyle{definition}
\newtheorem{definition}[theorem]{\bf Definition}
\newtheorem{example}[theorem]{\bf Example}
\newtheorem{remark}[theorem]{\bf Remark}
\newtheorem{refthm}{\bf Theorem}
\newtheorem{problem}[theorem]{\bf Problem}
\newtheorem{question}[theorem]{\bf Question}
\newtheorem{construction}[theorem]{\bf Construction}

\begin{document}
	\title[Strong topological Rokhlin property]{Strong topological Rokhlin property, shadowing, and symbolic dynamics of countable groups}
	\author{Michal Doucha}
	\address{Institute of Mathematics\\
		Czech Academy of Sciences\\
		\v Zitn\'a 25\\
		115 67 Praha 1\\
		Czechia}
	\email{doucha@math.cas.cz}
	\urladdr{https://users.math.cas.cz/~doucha/}
	\keywords{countable groups, Cantor dynamics, symbolic dynamics, subshifts of finite type, generic actions, shadowing}
	\thanks{The author was supported by by the GA\v{C}R project 22-07833K and RVO: 67985840.}
	
\begin{abstract}
A countable group $G$ has the strong topological Rokhlin property (STRP) if it admits a continuous action on the Cantor space with a comeager conjugacy class. We show that having the STRP is a symbolic dynamical property. We prove that a countable group $G$ has the STRP if and only if certain sofic subshifts over $G$ are dense in the space of subshifts. A sufficient condition is that isolated shifts over $G$ are dense in the space of all subshifts. 
		
We provide numerous applications including the proof that a group that decomposes as a free product of finite or cyclic groups has the STRP. We show that finitely generated nilpotent groups do not have the STRP unless they are virtually cyclic as well as many groups of the form $G_1\times G_2\times G_3$ where each factor is recursively presented. We show that a large class of non-finitely generated groups do not have the STRP, this includes any group with infinitely generated center and the Hall universal locally finite group.

We find a very strong connection between the STRP and shadowing, a.k.a. pseudo-orbit tracing property. We show that shadowing is generic for actions of a countable group $G$ if and only if $G$ has the STRP.

We also consider the STRP for totally disconnected locally compact groups and characterize pro-countable metrizable groups with the STRP. In particular, inverse limits of groups with the STRP have the STRP and so every pro-finite group has the STRP.
\end{abstract}
	
\maketitle

\setcounter{tocdepth}{1}
\tableofcontents

\section{Introduction}
This paper presents and explores certain connections between generic group actions on the Cantor space and the structure of subshifts of finite type and sofic subshifts over these groups.

Let us start with some history and motivation. One of the earliest occurencies of `genericity results' in measurable and topological dynamics was Halmos' paper \cite{Halm}, following the previous result of Oxtoby and Ulam in \cite{OxUl}, where he showed that ergodicity and weak mixing are generic properties among p.m.p. bijections of the standard probability space, removing the fears that such useful properties might actually be rare among general p.m.p. bijections (see \cite{AlpPra02} for the connections betwen Halmos's and Oxtoby-Ulam's papers). He also showed that the conjugacy class of any aperiodic p.m.p. bijection is dense, a result which is also attributed to Rokhlin and often identified with the Rokhlin lemma. This also explains why the name `Rokhlin' is attached to results of this kind, we refer to the survey \cite{GlaWei08}. We refer to monographs \cite{AlpPra} and \cite{AkHurKen} for more historical background and many results of this sort and to \cite{Hoch08} for some more recent developments in the topological case.

It is our aim here to investigate these problems, but not only for single invertible transformations, or in the language that we shall use, for actions of the group of integers, but for general countable group actions. This connects this area of research with combinatorial and geometric group theory and reveals sometimes surprising and beautiful differences between dynamical properties of geometrically different groups. We continue with more examples.

In topological dynamical category, Glasner and Weiss showed in \cite{GlaWei01} that there is an action of the integers on the Cantor space with a dense conjugacy class. This has been extended to actions of $\Int^d$ by Hochman \cite{Hoch12} and his proof works for any countable group, so density of conjugacy classes is not an interesting phenomenon in this case although interesting differences occur if we require a dense conjugacy class that is computable, see again \cite{Hoch12}. Glasner, Thouvenot, and Weiss in \cite{GlaThouWei} also showed that for any countable group $G$, a generic p.m.p. action has a dense conjugacy class (this is also an unpublished indepedently proved result of Hjorth).

It is then of particular interest whether in a given setting or given category there is an action which itself is generic, meaning its conjugacy class is comeager, thereby reducing the investigation of generic properties in that category to the properties of that particular action. This is known to be false for the integer actions on the standard probability space by del Junco \cite{delJu} and actually for actions of all countable amenable groups by Foreman and Weiss \cite{ForWei}. It was therefore of surprise when Kechris and Rosendal showed in \cite{KeRo} that there is a generic action of the integers on the Cantor space. This result sparked such an interest that it has been since then re-proved several times by different authors, see e.g. \cite{AkGlWei}, \cite{BeDa}, \cite{Kwia}. The present paper is another in the line. The paper \cite{BeDa} also contains an extensive investigation of the dynamical properties of this generic action, among other things it shows it has the shadowing property, an important notion in our investigation as well.

Another remarkable result of Hochman \cite{Hoch12} came few years later showing that this fails for $\Int^d$, for $d\geq 2$, i.e. these groups do not admit a generic action on the Cantor space. Hochman also attached the adjective `strong topological Rokhlin' to countable groups, as opposed to the previous usage when it was attached to topological groups or actions, to denote those countable groups that do admit a generic action on the Cantor space. Kwiatkowska in \cite{Kwia} later showed that free groups on finitely many generators do have this strong topological Rokhlin property in contrast to the result of Kechris and Rosendal from \cite{KeRo} that free groups on countably infinitely many generators do not.

The case of other countable groups has not been known and it is the aim of this paper to fill this gap. We show that the strong topological Rokhlin property, i.e. having a generic action on the Cantor space, is a property that is visible on the symbolic dynamical level. Indeed we show it is tightly connected with the structure of sofic subshifts over the corresponding group. Before continuing further, let us mention that symbolic dynamics is another area of dynamics traditionally reserved for actions of $\Int$ which has recently seen a tremendous progress in investigating more general group actions, where again substantial differences occur when the acting group varies. Early and by now classical results where these differences first occurred, between $\Int$ and $\Int^2$, are related to the domino problem and the existence of weakly and strongly aperiodic subshifts of finite type (see \cite{Berger} for these early results and see \cite{Cohen} for a recent breakthrough, where geometric group theory was beautifully blended into symbolic dynamics). Another important early occurrence is related to the Gottschalk surjunctivity conjecture (\cite{Gott}) which later gave rise to sofic groups (\cite{Grom}). We refer to \cite{CSCo-book} for general introduction into symbolic dynamics over countable groups.

We start the presentation of our results. The following is a characterization of groups with the strong topological Rokhlin property. For the sake of the next theorem we informally define a \emph{projectively isolated subshift} $X\subseteq A^G$, where $G$ is a countable group and $A$ a finite set with at least two elements, as a closed subshift such that there exists a subshift of finite type $Y\subseteq B^G$, for some finite $B$, and a factor map $\phi: Y\to X$ such that $\phi[Y']=X$ for every subshift $Y'\subseteq Y$ that is sufficiently close to $Y$ with respect to the Hausdorff distance. A precise definition is provided later as Definition~\ref{def:projisolated}. Also for two subshifts $X,Y\subseteq A^G$ and a finite set $F\subseteq G$ we write $X\subseteq_F Y$ to denote that $X\subseteq Y$ and moreover the $F$-patterns of $X$ coincide with the $F$-patterns of $Y$.

\begin{thmx}\label{thm:intro1}
Let $G$ be a countable group. Then $G$ has the strong topological Rokhlin property if and only if for every closed subshift $X\subseteq A^G$, for some finite set with at least two elements, and every $\varepsilon>0$ there is a projectively isolated subshift $X'\subseteq A^G$ whose Hausdorff distance to $X$ is at most $\varepsilon$.\bigskip

A sufficient condition that guarantees the strong topological Rokhlin property is that for every subshift of finite type $X\subseteq A^G$, for some finite $A$, and for every finite $F\subseteq G$ there are a subshift of finite type $Y\subseteq_F X$ and a finite set $E\subseteq G$ so that $Y$ is $\subseteq_E$-minimal.
\end{thmx}
We remark that there have been few similar results where the density of isolated points in some spaces was equivalent to genericity of certain actions, see \cite{GlKiMe} for actions of groups on countable sets (and another proof and related results in \cite{DouMa}) and \cite{KeLiPi} for representations of $C^*$-algebras and unitary representations of groups (see also \cite{DouMaVal} for another proof).\medskip

Theorem~\ref{thm:intro1} is then exploited to produce new examples and non-examples. From the positive side we show:
\begin{thmx}\label{thm:intro2}
Let $G=\bigstar_{i\leq n} G_i$ be a free product of the groups $(G_i)_{i\leq n}$, each of them being either finite or cyclic. Then $G$ has the strong topological Rokhlin property.
\end{thmx}

From the negative side we have more results and the following is a selection of some of them.
\begin{thmx}\label{thm:intro3}
Let $G$ be one of the following groups:
\begin{itemize}
	\item finitely generated infinite nilpotent group that is not virtually cyclic;
	\item $G_1\times G_2\times G_3$, where $G_i$, for $i\in\{1,2,3\}$ is finitely generated and recursively presented, and $G_1$ is indicable;
	\item an infinitely generated group such that for every finitely generated subgroup $H$ there is either $g\in G\setminus H$ that centralizes $H$, or has no relation with $H$ whatsoever.
\end{itemize}
Then $G$ does not have the strong topological Rokhlin property.
\end{thmx}

The next main result concerns the shadowing property, also known as the pseudo-orbit tracing property. Shadowing is by now one of the fundamental notions in dynamical systems, closely related to hyperbolicity and topological stability. We refer to the monograph \cite{Palm} for an introduction and historical background. Although it was originally defined for single homeomorphisms (in fact, diffeomorphisms) it makes perfect sense for general discrete group actions and it was for the first time defined in this generality in \cite{OsTi} and investigated since then in numerous publications, see e.g. \cite{ChKeo}, \cite{Mey}, \cite{BarGaRaLi}, \cite{LiChZh}. Genericity of shadowing has been also already extensively studied, see e.g. \cite{PilPla}, \cite{BrMeRa}, \cite{Ko07}, \cite{KoMaOp}, \cite{BeDa}, and references therein. In \cite{BeDa}, genericity of shadowing for homeomorphisms on the Cantor space has been derived as the property of the generic integer action. We continue in this line of research for more general groups and discover a very strong relation with the strong topological Rokhlin property; in fact, an equivalence between the STRP and genericity of shadowing. We show that actions with generic conjugacy class have shadowing, however if a group does not admit a generic action, then shadowing is a meager property for actions of such a group.

\begin{thmx}\label{thm:intro4}
Let $G$ be a countable group. Then shadowing is generic for continuous actions of $G$ on the Cantor space if and only if $G$ has the strong topological Rokhlin property.

In particular, if $G$ has the STRP, then the generic action has shadowing, and if $G$ does not have the STRP, then actions with shadowing form a dense and meager set.
\end{thmx}

We investigate several other genericity results. For instance we show that for a fixed countable amenable group, zero entropy is generic for actions on the Cantor space, building on previous work on genericity of zero entropy for subshifts from \cite{FriTam}.

Finally, we study generic actions of topological groups. There is a natural Polish space of actions on the Cantor space of an arbitrary locally compact second countable group. Since actions on the Cantor space make sense rather just for totally disconnected groups we focus on that case and provide a characterization of locally compact pro-countable metrizable groups that admit generic action on the Cantor space. Here we state again just a simplified version.

\begin{thmx}\label{thm:intro5}
Let $G$ be a locally compact pro-countable group that is an inverse limit of countable groups with the strong topological Rokhlin property. Then $G$ has the strong topological Rokhlin property itself. In particular, every pro-finite group has the strong topological Rokhlin property.
\end{thmx}

The paper is orgnaized as follows. Section~\ref{sect:Cantor} consists of preliminaries and basic results on group actions on the Cantor space and symbolic dynamics over general groups. Most of the definition, crucial for the rest of the paper, are also contained there. Section~\ref{sect:mainproofs} contains the proof of Theorem~\ref{thm:intro1}. Section~\ref{sect:freeproducts} introduces new notions and techniques that are jointly with Theorem~\ref{thm:intro1} necessary to prove Theorem~\ref{thm:intro2}. Section~\ref{sect:noSTRP} is focused on providing new non-examples of groups with the strong topological Rokhlin property and Section~\ref{sect:genericdynamics} contains results on genericity of entropy and shadowing of group actions on the Cantor space. Final Section~\ref{sect:locallycompactgroups} studies the strong topological Rokhlin property for totally disconnected locally compact groups.

Let us also mention that our standard monographs concerning symbolic dynamics over general groups, topological dynamics of general group actions, and (geometric) group theory respectively are \cite{CSCo-book}, \cite{KeLi-book}, and \cite{DruKap-book} respectively, to which we refer in case the reader finds any unexplained notion from these areas in the sequel.
\section{Cantor and symbolic dynamics}\label{sect:Cantor}

Let $\CC$ denote the Cantor space. Denote by $\Homeo(\CC)$ the topological group of all self-homeomorphisms of $\CC$ equipped with the uniform topology. Fixing a compatible metric $d$ on $\CC$ one can see that basic open neighborhoods of a homeomorphism $\phi\in\Homeo(\CC)$ are of the form $\{\psi\in\Homeo(\CC)\colon \forall x\in\CC\; (d(\phi(x),\psi(x))<\varepsilon)\}$, where $\varepsilon>0$ is a varying parameter.
\subsection{Spaces of actions and shifts}	
\subsubsection{Spaces of actions}
	\begin{definition}\label{def:spaceofactions}
For a countable group $G$, denote by $\Act_G(\CC)$ the Polish space of all continuous actions of $G$ on $\CC$. Formally, $\Act_G(\CC)$ is identified with the space of all homomorphisms from $G$ into the topological group $\Homeo(\CC)$, which can be further identified with a closed subset of $\Homeo(\CC)^G$ with the product topology.
	\end{definition}

\begin{lemma}\label{lem:basicopennbhds}
Let $G$ be a countable group and $\alpha\in\Act_G(\CC)$. The following sets form basic open neighborhoods of $\alpha$: \[\NN_\alpha^{F,\PP}:=\{\beta\in\Act_G(\CC)\colon \forall f\in F\;\forall x\in\CC\;\forall P\in\PP\;(\alpha(f)x\in P\Leftrightarrow \beta(f)x\in P)\},\] where $F\subseteq G$ is a finite subset and $\PP$ is a partition of $\CC$ into disjoint non-empty clopen sets.
\end{lemma}
\begin{proof}
Fix a countable group $G$ and an action $\alpha\in\Act_G(\CC)$. Fixing a compatible metric $d$ on $\CC$ it is clear that basic open neighborhoods of $\alpha$ are, by the definition of the product of uniform topologies on $\Homeo(\CC)^G$, of the form \[\NN_\alpha^{F,\varepsilon}:=\{\beta\in\Act_G(\CC)\colon \forall f\in F\forall x\in\CC\; (d(\alpha(f)x,\beta(f)x)<\varepsilon)\},\] where $F\subseteq G$ is a finite set and $\varepsilon>0$. So it is enough to show that for given $\varepsilon>0$ there is a clopen partition $\PP_\varepsilon$ of $\CC$ so that $\NN_\alpha^{F,\PP_\varepsilon}\subseteq \NN_\alpha^{F,\varepsilon}$, and conversely, given a clopen partition $\PP$ of $\CC$ that there is $\varepsilon_\PP>0$ so that $\NN_\alpha^{F,\varepsilon_\PP}\subseteq \NN_\alpha^{F,\PP}$. For the former, it suffices to take a clopen partition $\PP_\varepsilon$ whose elements have diameter less than $\varepsilon$, and for the latter it suffices to take $\varepsilon_\PP>0$ which is smaller than the Lebesgue number of $\PP$.
\end{proof}

Notice that for any countable group $G$ the group $\Homeo(\CC)$ naturally acts on $\Act_G(\CC)$ by conjugation, where for $\phi\in\Homeo(\CC)$ and $\alpha\in\Act_G(\CC)$ the action $\phi\alpha\phi^{-1}$ is naturally defined by \[\big(\phi\alpha\phi^{-1}\big)(g):=\phi\alpha(g)\phi^{-1},\] for $g\in G$.

Although informally defined already in the abstract, for the sake of formal soundness let us provide a precise definition of the strong topological Rokhlin property.
\begin{definition}
Let $G$ be a countable group. We say that $G$ has the \emph{strong topological Rokhlin property} if there exists $\alpha\in\Act_G(\CC)$ such that the set \[\{\phi\alpha\phi^{-1}\colon \phi\in\Homeo(\CC)\}\subseteq \Act_G(\CC)\] is comeager.
\end{definition}
We state the following important fact that any invariant subset in $\Act_G(\CC)$ with the Baire property is either meager, or comeager.
\begin{fact}\label{fact:0-1-law}
Let $G$ be a countable group and let $\AAA\subseteq \Act_G(\CC)$ be a subset with the Baire property that is closed under conjugation, i.e. for any $\phi\in\Homeo(\CC)$, $\phi\AAA\phi^{-1}=\AAA$. Then $\AAA$ is either meager, or comeager.
\end{fact}
\begin{proof}
We apply \cite[Theorem 8.46]{KechrisBook} with $G$ equal to $\Homeo(\CC)$ and $X$ equal to $\Act_G(\CC)$. We only need to check that the action of $\Homeo(\CC)$ on $\Act_G(\CC)$ is topologically transitive. This is equivalent with the existence of an element $\alpha\in\Act_G(\CC)$ with a dense conjugacy class. This is proved in \cite[Proposition 1.2]{Hoch12}  - notice that the proposition is stated only for $\Int^d$, however the first paragraph of the proof mentions it works for any countable group.
\end{proof}
\subsubsection{Subshits and their spaces}
\begin{definition}\label{def:spaceofsubshifts}
Let $A$ be a finite set with at least two elements and $G$ be a countable group. Consider the set $A^G$ equipped with the product topology with which it is either finite discrete if $G$ is finite, or homeomorphic to the Cantor space if $G$ is infinite. The group $G$ acts on $A^G$ by \emph{shift}, i.e. for $x\in A^G$ and $g,h\in G$ we have \[gx(h):=x(g^{-1}h).\] When we need a symbol for the shift action, we shall use the symbol $\sigma: G\curvearrowright A^G.$ Any closed and $G$-invariant subspace of $A^G$ will be called \emph{subshift}.

By $\SH_G(A)$, we shall denote the compact space of all subshifts of $A^G$ equipped with the Vietoris topology - or equivalently, with the topology induced by the Hausdorff metric coming from a compatible metric on $A^G$.

By $\SH_G(n)$ we denote the space $\SH_G(\{1,\ldots,n\})$. When there is no danger of confusion, we shall sometimes identify the spaces $\SH_G(A)$ and $\SH_G(n)$ for $|A|=n$.

Sometimes when we are given a subshift $X$ over a group $G$, however the finite set $A$ is not clear from the context, we shall denote it by $\alpha(X)$, i.e. $X\subseteq \alpha(X)^G$. We may occassionally call the finite set $A$ an \emph{alphabet}.
\end{definition}
In the sequel, we shall use the wording `non-trivial finite set' to emphasize that the set in question has at least two elements.

\begin{remark}
Since the spaces of subshifts $\SH_G(A)$ will play a major role in this paper, we feel obliged to provide few comments on them. They closely resemble the spaces of subgroups of a given countable group with the Chabauty topology (see \cite{Cha}) and surely any reader familiar with the latter will also feel comfortable with the former. We could not track the origin of when these spaces appeared for the first time and apparently at least for the group $\Int$, and also for $\Int^d$, they have been in a folklor use for some time. For more general groups the oldest references we could find were \cite{GaoJacSew} and \cite{FriTam}. Since then these spaces, for general groups, were for example spectacularly used, using actually a genericity argument, in characterizing strongly amenable countable groups in \cite{FriTamVF}.
\end{remark}
\begin{definition}
Let $G$ be a countable group and $A$ be a non-trivial finite set. If $F\subseteq G$ is a finite set, elements from $A^F$ will be called \emph{patterns}. If $X\subseteq A^G$ is a subshift, a pattern $p\in A^F$ is called \emph{forbidden} in $X$ if there is no $x\in X$ such that $x\upharpoonright F=p$; conversely, $p$ is called \emph{allowed} if there is $x\in X$ such that $x\upharpoonright F=p$.

By $X_F$ we denote the set of patterns from $A^F$ that are allowed in $X$, i.e. the set \[\{p\in A^F\colon \exists x\in X\; (x\upharpoonright F=p)\}.\]

For any pattern $p\in X_F$ we shall denote by $C_p(X)$ the clopen set \[\{x\in X\colon x\upharpoonright F=p\},\] that is, $\{C_p(X)\colon p\in X_F\}$ is a clopen partition of $X$. Sometimes, when the subshift $X$ is clear from the context, we may write just $C_p$ instead of $C_p(X)$.
\end{definition}

There is a convenient form of basic open neighborhoods of subshifts in $\SH_G(A)$. The proof of the following lemma is similar to the proof of Lemma~\ref{lem:basicopennbhds} and left to the reader.
\begin{lemma}
Let $G$ be a countable group, $A$ a non-trivial finite set, and $X\subseteq A^G$ a subshift. Basic open neighborhoods of $X$ in $\SH_G(A)$ are of the form \[\NN_X^F:=\{Y\subseteq A^G\colon Y_F=X_F\},\] where $F\subseteq G$ is a finite set.
\end{lemma}
Notice that the set of the form $\NN_X^F$ as above is actually clopen in $\SH_G(A)$.
\begin{construction}
Let $\PP=\{P_1,\ldots,P_n\}$ be a partition of a compact metrizable zero-dimensional space $X$ into disjoint non-empty clopen sets and let $\alpha: G\curvearrowright X$ be a continuous action of a countable group $G$ on $X$. We denote by $Q^\alpha_\PP$ the continuous $G$-equivariant map from $X$ to $\PP^G$, freely identified with $n^G$, which is defined as follows. For any $x\in X$, $g\in G$ and $i\leq n$ \[Q^\alpha_\PP(x)(g):= i\;\;\;\text{ if and only if }\;\;\; \alpha(g^{-1})x\in P_i.\]
\end{construction}

The verification that $Q^\alpha_\PP$ is continuous is straightforward, we only check here that it is $G$-equivariant which will also help the reader see why $\alpha(g^{-1})$ was used instead of $\alpha(g)$ in the definition above. Note that for $x\in \CC$ and $g,h\in G$ we have \[\begin{split}Q^\alpha_\PP\big(\alpha(h)x\big)(g)=i & \Leftrightarrow \alpha(g^{-1})\alpha(h)x\in P_i\Leftrightarrow \alpha(g^{-1}h)x\in P_i\\ & \Leftrightarrow Q^\alpha_\PP(x)(h^{-1}g)=i\Leftrightarrow h Q^\alpha_\PP(x)(g)=i.\end{split}\]

By $\QQ(\alpha,\PP)$ we shall denote the subshift of $n^G$, an element of $\SH_G(n)$, which is the image of $Q^\alpha_\PP$.

The following basic lemma, which is a generalization of the well-known Curtis-Hedlund-Lyndon theorem from symbolic dynamics, shows that every continuous equivariant map from a zero-dimensional compact metrizable space $X$, equipped with an action $\alpha: G\curvearrowright X$, onto a subshift is of the form $Q^\alpha_\PP$, for some clopen partition $\PP$ of $X$.

\begin{lemma}\label{lem:factoringonshift}
Let $G$ be a countable group acting continuously on a zero-dimensional compact metrizable space $X$. Let us denote the action by $\alpha$. Let $\phi: (X,\alpha)\rightarrow A^G$ be a continuous $G$-equivariant map into the shift space $A^G$, for some finite non-trivial $A$. Then there is a clopen partition $\PP$ of $X$ such that $\phi=Q^\alpha_\PP$.
\end{lemma}
\begin{proof}
Let $Y\subseteq A^G$ be the image of $X$ via $\phi$, which is a subshift. Enumerate $A$ as $\{a_1,\ldots,a_n\}$ and for $i\leq n$, set \[C(i):=\{x\in Y\colon x(1_G)=a_i\},\] which we can assume to be, without loss of generality, non-empty. Then $\PP':=\{C(i)\colon i\leq n\}$ is a clopen partition of $Y$ and for $i\leq n$ we set $P_i:=\phi^{-1}(C(i))$, which is a clopen subset of $X$, and \[\PP:=\{P_i\colon i\leq n\}\] is a clopen partition of $X$. Notice that for every $x\in X$ we have $\phi(x)(1_G)=a_i$ if and only if $\phi(x)\in C_i$, which is if and only if $x\in P_i$, i.e. $Q^\alpha_\PP(x)=i$. Thus in order to check that $\phi=Q^\alpha_\PP$ we pick $x\in X$ and $g\in G$, and we verify \[\phi(x)(g)=\big(g^{-1}\phi(x)\big)(1_G)=\phi(g^{-1}x)(1_G)=Q^\alpha_\PP(g^{-1}x)(1_G)=Q^\alpha_\PP(x)(g),\] where we have formally identified $A=\{a_1,\ldots,a_n\}$ with $\{1,\ldots,n\}$.
\end{proof}
The following standard result can be derived as a corollary.
\begin{corollary}[The Curtis-Hedlund-Lyndon theorem]
	Let $G$ be a countable group, $A$ and $B$ be two non-trivial finite sets, and $\phi:X\rightarrow Y$ be a continuous $G$-equivariant map between two subshifts $X\subseteq A^G$ and $Y\subseteq B^G$. Then there exist a finite set $F\subseteq G$ and a map $f:X_F\rightarrow B$ such that for every $x\in X$ and $g\in G$ \[\phi(x)(g)=f(g^{-1}x\upharpoonright F).\]
\end{corollary}
%\begin{proof}
%Fix $\phi$ and $X$, $Y$ as in the statement. By Lemma~\ref{lem:factoringonshift}, $\phi=Q^\sigma_\PP$, for the clopen partition %$\PP:=\{P_b\colon b\in B\}$, where $P_b:=\phi^{-1}(y\in Y\colon y(1_G)=b)$ for $b\in B$, of $X$, and where $\sigma$ denotes the shift action of %$G$ on $X$. There exists a finite set $F\subseteq G$ such that \[\big\{\{x\in X\colon x\upharpoonright F=p\}\colon p\in X_F\big\}\] refines %$\PP$. For $p\in X_F$ and $b\in B$ we set \[f(p)=b\;\text{ if and only if }\;\{x\in X\colon x\upharpoonright F=p\}\subseteq P_b.\]
%\end{proof}


We continue with two propositions connecting the spaces and topologies of $\Act_G(\CC)$ and $\SH_G(n)$ which will be instrumental in proving the main theorem.
\begin{proposition}\label{prop:Qcontinuity}
Let $G$ be a countable group and $\PP$ a partition of $\CC$ into disjoint non-empty clopen sets. Then the map \[\QQ(\cdot,\PP):\Act_G(\CC)\to \SH_G(\PP)\] defined as \[\alpha\to \QQ(\alpha,\PP)\] is continuous and onto.
\end{proposition}
\begin{proof}
We fix $G$ and $\PP$ as in the statement. Take $\alpha\in\Act_G(\CC)$ and set $X:=\QQ(\alpha,\PP)$. Let $U$ be an open neighborhood of $X$ which we may assume is of the form $\NN_X^F$, for some finite symmetric set $F\subseteq G$, which, without loss of generality, contains the unit. We shall find an open neighborhood $V$ of $\alpha$ so that $\QQ(\cdot,\PP)[V]\subseteq U$. We claim that $V:=\NN_\alpha^{F,\PP}$ is such a neighborhood. Indeed, pick $\beta\in V$ and set $Y:=\QQ(\beta,\PP)$. Let us show that $Y\in U=\NN_X^F$, i.e. $Y_F=X_F$. Let $p\in X_F$ be a pattern allowed in $X$, so there is $x\in X$ such that $x\upharpoonright F=p$. Let $z\in\CC$ be an arbitrary element such that $Q^\alpha_\PP(z)=x$. By the definition of the neighborhood $V=\NN_\alpha^{F,\PP}$ \[\forall f\in F\;\forall P\in\PP\; \big(\alpha(f)z\in P\Leftrightarrow \beta(f)z\in P\big).\] It follows that for $y:=Q^\beta_\PP(z)\in Y$ we have \[p=x\upharpoonright F=y\upharpoonright F,\] thus $p$ is allowed in $Y$ as well. We proved that $X_F\subseteq Y_F$, the other direction is proved symmetrically. This finishes the proof that $\QQ(\cdot,\PP)$ is continuous.

Let us show that it is surjective. Pick $X\subseteq \PP^G$ and and for each $P\in\PP$, let $\psi_P:C_{\{1_G,P\}}(X)\times\CC\rightarrow P$ be a homeomorphism. Here $C_{\{1_G,P\}}(X):=\{x\in X\colon x(1_G)=P\}$, which corresponds to the notation $C_p(X)$, where $p=\{1_G,P\}\in \PP^{\{1_G\}}$. We take the product of $C_{\{1_G,P\}}(X)$ with $\CC$ to ensure that it has no isolated points. 

Since $\{C_{\{1_G,P\}}(X)\times\CC\colon P\in\PP\}$ is a clopen partition of $X\times\CC$, it follows that $\psi=\coprod_{P\in\PP} \psi_P: X\times\CC\rightarrow\CC$ is a homeomorphism. We define $\gamma:=\psi\circ\big(\sigma\times\mathrm{Id}\big)\circ\psi^{-1}\in\Act_G(\CC)$, where $\sigma\times\mathrm{Id}$ is the action of $G$ on $X\times\CC$, which acts as the shift on the first coordinate and as the identity on the other. We claim that $\QQ(\gamma,\PP)=X$. This follows from two facts. First that for every $z\in\CC$ there are $x\in X$ and $y\in \CC$ such that $\psi(x,y)=z$, and conversely for every $x\in X$ and $y\in\CC$ there is $z\in\CC$ with $\psi(x,y):=z$. Second, that for every $x\in X$ and $y,z\in\CC$ such that $z=\psi(x,y)$, and for every $g\in G$ and $P\in\PP$ we have \[\begin{split}Q_\PP^\gamma(z)(g)=P& \Leftrightarrow \gamma(g^{-1})z\in P\\ &\Leftrightarrow \big(\sigma(g^{-1})\times\mathrm{Id}\big)(x,y)\in C_{\{1_G,P\}}(X)\times\CC\Leftrightarrow x(g)=P.\end{split}\]
\end{proof}

\begin{proposition}\label{prop:Qmap-nbhds}
Let $G$ be a countable group and $\PP'\preceq\PP$ two clopen partitions of $\CC$, one refining the other. 
\begin{enumerate}
	\item\label{it1-Qmap-nbhds} For every $\alpha\in\Act_G(\CC)$, finite set $F\subseteq G$ and $X\in \QQ(\cdot,\PP)[\NN_\alpha^{F,\PP}]$ we have \[\QQ(\cdot,\PP)[\NN_\alpha^{F,\PP}]=\NN_X^F.\]
	
	\item\label{it2-Qmap-nbhds} For every $\alpha\in\Act_G(\CC)$ we have $\QQ(\alpha,\PP)=\phi\big(\QQ(\alpha,\PP')\big)$, where $\phi:(\PP')^G\rightarrow \PP^G$ is the map induced by the inclusion map $\iota:\PP'\rightarrow\PP$, i.e. satisfying $P\subseteq \iota(P)$ for every $P\in\PP'$. 
\end{enumerate}
\end{proposition}
\begin{proof}
Let us fix $G$ and $\PP'\preceq\PP$ as in the statement.\medskip

We first prove \eqref{it1-Qmap-nbhds}. Fix additionally $\alpha$, $F\subseteq G$ and $X$ as in the statement. Since for every $Y\in\NN_X^F$ we have $\NN_Y^F=\NN_X^F$, we can without loss of generality assume that $X=\QQ(\alpha,\PP)$. We first prove that $\QQ(\cdot,\PP)[\NN_\alpha^{F,\PP}]\subseteq \NN_X^F$. Pick $\beta\in\NN_\alpha^{F,\PP}$ and set $Y:=\QQ(\beta,\PP)$. We need to prove that $X_F=Y_F$. Take some $p\in X_F$ and let $x\in\CC$ be such that for every $f\in F$ and $P\in\PP$ \[\alpha(f^{-1})x\in P\Leftrightarrow p(f)=P,\] thus $Q_\PP^\alpha(x)\upharpoonright F=p$. Since by definition for every $f\in F$ and $P\in\PP$ we have \[\alpha(f^{-1})x\in P\Leftrightarrow \beta(f^{-1})x\in P,\] it follows that $Q_\PP^\beta(x)\upharpoonright F=p$ as well. We showed that $X_F\subseteq Y_F$, the inclusion $Y_F\subseteq X_F$ is proved symetrically.

Now we prove the reverse inclusion $\NN_X^F\subseteq \QQ(\cdot,\PP)[\NN_\alpha^{F,\PP}]$. Take any $Y\in\NN_X^F$. We define a clopen partition $\PP'':=\{R_p\colon p\in X_F\}\preceq\PP$, where for $p\in X_F$, \[R_p:=\{x\in\CC\colon \alpha(f^{-1})(x)\in P\;\text{ if and only if }\; p(f)=P\}.\] For each $p\in X_F$, since $Y_F=X_F$, we have that $C_p(Y)\neq\emptyset$ and let $\psi_p:C_p(Y)\times\CC\rightarrow R_p$ be a fixed homeomorphism (we take the product of $C_p(Y)$ with $\CC$ to ensure it has no isolated points and it is therefore homeomorphic to $\CC$) and set $\psi:Y\times\CC\rightarrow \CC$ to be the homeomorphism $\coprod_{p\in C} \psi_p$. We define the action $\beta\in\Act_G(\CC)$ by setting for $g\in G$ and $x\in \CC$ \[\beta(g)(x):=\psi\circ\big(\sigma(g)\times\rm{Id}\big)\circ\psi^{-1}(x),\] where $\sigma\times\rm{Id}$ is an action of $G$ on $Y\times \CC$ which acts as the shift on the first coordinate and as the identity on the second.

We check that $\gamma\in \NN_\alpha^{F,\PP}$. Pick $x\in\CC$, $f\in F$, and $P\in\PP$. We need to verify that $\alpha(f^{-1})x\in P$ if and only if $\beta(f^{-1})x\in P$. Without loss of generality, assume that $\alpha(f^{-1})x\in P$ and it suffices now to verify that $\beta(f^{-1})x\in P$. Let $p\in X_F$ be such that $x\in R_p$. Then by definition, $p(f)=P$ and we verify that $\beta(f^{-1})x=\psi\circ (\sigma(f^{-1})\times\mathrm{Id})\circ\psi^{-1}(x)\in P$. We have $\psi^{-1}(x)\in C_p(Y)\times \CC$ and so $(\sigma(f^{-1})\times\mathrm{Id})\circ\psi^{-1}(x)\in C_{p'}(Y)\times\CC$, for $p'\in X_F$ where $p'(1_G)=P$. Consequently, $\beta(f^{-1})x\in R_{p'}$, thus, by the definition of $R_{p'}$, \[\beta(f^{-1})x=\alpha(1^{-1}_G)\big(\beta(f^{-1})x\big)\in p'(1_G)=P,\] which is what we were supposed to show.\medskip

We now prove \eqref{it2-Qmap-nbhds}. Fix $\alpha\in\Act_G(\CC)$ and consider the map $\phi:(\PP')^G\rightarrow\PP^G$ as in the statement. Set $X:=\QQ(\alpha,\PP')$ and $Y:=\QQ(\alpha,\PP)$. We need to show that $\phi[X]=Y$. Pick an arbitrary $z\in\CC$ and set $x:=Q_{\PP'}^\alpha(z)\in X$, $y:=Q_\PP^\alpha(z)$. Now let $g\in G$ and we check that $\phi(x)(g)=y(g)$, which will finish the proof. For $P\in\PP$ we have \[\begin{split}y(g)=P & \Leftrightarrow \alpha(g^{-1})z\in P\Leftrightarrow \exists P'\in\PP'\;\big( P'\subseteq P\wedge \alpha(g^{-1})z\in P'\big)\\ & \Leftrightarrow x(g)=P'\Leftrightarrow \phi(x)(g)=P,\end{split}\] which shows the equality.
\end{proof}

\subsection{Isolated subshifts, subshifts of finite type, and sofic subshifts}
\begin{definition}\label{def:SFT}
	Let $G$ be a countable group and $A$ be a finite set with at least two elements. A subshift $X\subseteq A^G$ is \emph{of finite type}, shortly an \emph{SFT}, if there exists a finite set $F\subseteq G$, called the \emph{defining window} of $X$, and a set $\FF\subseteq A^F$ of patterns such that for $x\in A^G$ we have \[x\in X\;\text{ if and only if }\; \forall g\in G\; (gx\upharpoonright F\in\FF).\]
	The set $\FF$ is called \emph{the set of allowed patterns} for $X$, while its complement $F^A\setminus \FF$ is called \emph{the set of forbidden patterns} for $X$.
	
	We also say that a subshift $X\subseteq A^G$ is \emph{sofic} if it is a factor of a subshift of finite type (possibly defined over different finite set, i.e. subshift of $B^G$ for some non-trivial finite $B$).
\end{definition}

Subshifts of finite type play also a prominent role in the topology of the space $\SH_G(A)$.
\begin{lemma}\label{lem:SFTnbhrds}
Let $G$ be a countable group and $A$ be a non-trivial finite set. Then the subshifts of finite type are dense in $\SH_G(A)$. Moreover, for every subshift of finite type $X\subseteq A^G$ there is a finite set $F\subseteq G$ so that all subshifts in the basic open neighborhood $\NN_X^F$ are subshifts of $X$.

In particular, every open set in $\SH_G(A)$ contains an open subset which has a subshift that is maximal with respect to inclusion, it is of finite type, and all other subshifts in the open set are its subshifts.
\end{lemma}
\begin{proof}
Fix $G$ and $A$ as in the statement. Let $\NN\subseteq \SH_G(A)$ be an open set and let $X\in\NN$ and $F\subseteq G$ be such that $\NN_X^F\subseteq \NN$. Let $Z\subseteq A^G$ be the subshift of finite type whose defining window is $F$ and the defining set of allowed patterns is $X_F$. Clearly, $X\subseteq Z$, in particular, $Z$ is non-empty, and by definition $Z\in\NN_X^F\subseteq \NN$. This shows that subshifts of finite type are dense. Conversely, suppose that $Z$ is a given subshift of finite type whose defining window is a finite subset $F\subseteq G$. Then for any finite $F'\supseteq F$ we have that for every $Z'\in\NN_X^{F'}$ necessarily $Z'\subseteq Z$ as $Z$ is the largest subshift $Z'$ with respect to inclusion with the property $Z_{F'}=Z'_{F'}$.

The `In particular' part of the statement is then an immediate consequence.
\end{proof}

\begin{lemma}\label{lem:SFTfactorofSFT}
Let $A$ be a finite set with at least two elements, $G$ a countable group and let $X\subseteq A^G$ be a subshift. Let $F\subseteq G$ be a finite subset. Then for the partition \[\PP:=\{C_p\subseteq X\colon p\in X_F\},\] the map $Q^\sigma_\PP$, where $\sigma$ is the shift action on $X$, is one-to-one.
\end{lemma}
\begin{proof}
The image is clearly a closed $G$-invariant subset of $\PP^G$, i.e. a subshift, so we only need to check the map is injective. This is however obvious since if $x\neq y\in X$ then there is $g\in G$ such that $\sigma(g)x$ and $\sigma(g)y$ lie in different elements of the partition $\PP$ and thus $Q^\sigma_\PP(x)(g)\neq Q^\sigma_\PP(y)(g)$.
\end{proof}

\begin{definition}\label{def:Rouzy}
Let $G$ be a countable group, $F\subseteq G\setminus\{1_G\}$ a finite subset, and $V$ a finite (vertex) set. We call the collection $\VV_F=(V,(E_f)_{f\in F})$ \emph{an $F$-Rauzy graph} if for each $f\in F$, $(V,E_F)$ is a directed graph with no sources and no sinks. That is, $E_f$ is a set of oriented edges between the vertices $V$ such that for every $v\in V$ there are at least one incoming and one outgoing edge to $v$, resp. from $v$.

Having $\VV_F$ as above we define a subshift $X_{\VV_F}\subseteq V^G$ as follows. For $x\in V^G$ we set \[x\in X_{\VV_F}\;\text{ if and only if }\; \forall g\in G\;\forall f\in F\;\big((x(g),x(gf))\in E_f\big).\]
It is clear that $X_{\VV_F}$ is of finite type.
\end{definition}

Notice that alternatively and equivalently one can require to have the edge set $E_f$ for all $f\in G$ with the requirement that for all but finitely many $f\in G$, $E_f$ is a complete directed graph on $V$.\medskip

Conversely, let us prove the well-known fact (at least in case $G=\Int$) that any SFT $X\subseteq A^G$, for any $G$ and $A$, is conjugate to an SFT of the form $X_{\VV_F}$ for some $F$-Rauzy graph.
\begin{proposition}\label{prop:SFTprop}
Let $G$ be a countable group and $A$ a finite set with at least two elements. Let $X\subseteq A^G$ be a subshift of finite type whose defining window is a finite set $F'\subseteq G$. Let $S$ be a finite symmetric set (not containing $1_G$) such that $\langle F'\rangle=\langle S\rangle\leq G$. Then there exists an $S$-Rauzy graph $\VV_S$, %for some finite $F\subseteq G\setminus\{1_f\}$, 
such that $X$ is conjugate to $X_{\VV_S}$.
\end{proposition}
\begin{proof}
Fix $G$, $A$, $X$, $F'$, and $S$ as in the statement. Since the finite symmetric set $S$ generates the same subgroup as the set $F'$ does, there exists a ball, denoted by $F$, with respect to the word metric induced by $S$, that contains $F'$. Without loss of generality, we can from now on assume that the defining window for $X$ is this bigger set $F$. Set $V:=X_F$. For every $s\in S$ and $p,p'\in V$ set \[(p,p')\in E_s\;\text{ if and only if }\;p\cup p'\cdot s\in X_{F\cup Fs},\] where $p'\cdot s$ is a pattern from $X_{Fs}$ defined by $p's(g)=p'(gs^{-1})$, for $g\in Fs$. The fact that $p\cup p'\cdot s\in X_{F\cup Fs}$ also includes the requirement that $p\cup p'\cdot s$ is a well-defined map, i.e. for $g\in F\cap Fs$ we have $p(g)=p'\cdot s(g)$.

It is now straightforward to verify that the map $T:X\rightarrow X_{\VV_F}$ defined for $x\in X$ and $g\in G$ by \[T(x)(g):=x\upharpoonright gF\] is an isomorphism, i.e. a $G$-equivariant continuous bijection. Indeed, clearly it is continuous and $G$-equivariant. That it is a bijection follows from the fact that the inverse to $T$ is the map $x\in X_{\VV_F}\to x'\in X$, where for $g\in G$ we have \[x'(g)=x(g)(1_G).\]
\end{proof}

A special subclass of subshifts of finite type will be of crucial importance.
\begin{definition}\label{def:isolatedsubshift}
	Let $G$ and $A$ be as in Definition~\ref{def:SFT}. A subshift $X\in\SH_G(A)$ is called \emph{isolated} if any of the following (easily verified to be) equivalent conditions hold:
	\begin{enumerate}
		\item $X$ is isolated in the topology of $\SH_G(A)$.
		\item\label{def:isolatedsubshift2} $X$ is of finite type and there exists a finite set $F\subseteq G$ such that there is no proper subshift $Y\subsetneq X$ satisfying $Y_F=X_F$.
	\end{enumerate}
\end{definition}

\begin{remark}
The requirement of being of finite type in Definition~\ref{def:isolatedsubshift} \eqref{def:isolatedsubshift2} is important since it is a straightforward application of Zorn's lemma that for every subshift $X\subseteq A^G$, for appropriate $G$ and $A$, and every finite $F\subseteq G$ there exists a subshift $Y\subseteq X$ that is minimal in inclusion with respect to the condition that $Y_F=X_F$.

Notice also that every minimal subshift of finite type is isolated.
\end{remark}
\begin{lemma}\label{lem:isolatedcharacterization}
Let $A$ be a finite set having at least two elements and $G$ a countable group. A subshift $X\subseteq A^G$ is isolated if and only if it is of finite type and there exists a partition $\PP$ of $X$ into disjoint non-empty clopen sets such that there is no proper non-empty subshift $Y\subseteq X$ which intersects every element of $\PP$.
\end{lemma}
\begin{proof}
If $X$ is isolated then by definition it is of finite type and there is a finite set $F\subseteq G$ so that $\NN_X^F=\{X\}$. Then the partition \[\PP:=\{C_p\subseteq X\colon p\in A^F\text{ is an allowed pattern in }X\}\] is as desired.

Conversely, suppose that $X\subseteq A^G$ is of finite type and has the property as in the statement with respect to a partition $\PP$. It is clear that then it has the same property with respect to any refinement $\PP'\preceq \PP$. Thus we can find a refinement $\PP'\preceq \PP$ which is of the form $\{C_p\colon p\in A^F\text{ is an allowed pattern in }X\}$, for some finite set $F\subseteq G$. It is clear that then $\NN_X^F=\{X\}$, so $X$ is isolated.
\end{proof}
The corollary is that being isolated is also a conjugacy invariant.
\begin{corollary}\label{cor:isolatedinvariant}
Let $A$ and $B$ be finite sets with at least two elements, and $G$ a countable group. Suppose that $X\subseteq A^G$ is isolated and $Y\subseteq B^G$ is a subshift such that there is a $G$-equivariant homeomorphism between $X$ and $Y$. Then $Y$ is isolated as  well.
\end{corollary}
\begin{proof}
Let $F\subseteq G$ be a finite set such that $\NN_X^F=\{X\}$. Let $\psi:X\rightarrow Y$ be a $G$-equivariant homeomorphism. Then $\PP:=\{\psi[C_p]\colon p\in X_F\}$ is a partition of $Y$ with the property that there is no proper non-empty subshift $Z\subseteq Y$ that has a non-empty intersection with every element of $\PP$. Indeed, if $Z$ were such a subshift, then it is straightforward to check $\psi^{-1}(Z)\in\NN_X^F$, yet $\psi^{-1}(Z)\neq X$, a contradiction that $X$ is isolated. So we are done by Lemma~\ref{lem:isolatedcharacterization} and the fact that being of finite type is a conjugacy invariant.
\end{proof}

\begin{lemma}\label{lem:CHL}
Let $A$ and $B$ be two finite sets with at least two elements and let $G$ be a countable group. Let $X\in\SH_G(A)$ and $Y\in\SH_G(B)$ be two shifts and $\phi:X\rightarrow Y$ a continuous $G$-equivariant map. Then there exists a finite set $F\subseteq G$ such that for every $Z\in\NN_X^F$, $\phi$ is defined on $Z$.
\end{lemma}
\begin{proof}
Fix $X$, $Y$ and $\phi:X\rightarrow Y$ as in the statement. By the Curtis-Hedlund-Lyndon theorem, $\phi$ is induced by some map $\phi_0: X_F\rightarrow B$, where $F\subseteq G$ is a finite subset, so that for every $x\in X$ and $g\in G$ we have \[\phi(x)(g):=\phi_0\big(g^{-1}x\upharpoonright F\big).\] Now if $Z\in\NN_X^F$ then for every $z\in Z$, $z\upharpoonright F\in X_F$, thus $\phi$ can be defined on $Z$ using $\phi_0$ exactly in the same way.
\end{proof}

\begin{definition}\label{def:projisolated}
Let $A$ be a finite set with at least two elements and $G$ be a countable group. A subshift $X\in\SH_G(A)$ is called \emph{projectively isolated} if there exist a subshift $Y\in\SH_G(B)$, for some finite set $B$, an open neighborhood $\NN_Y^F$, for some finite set $F\subseteq G$, and a map $\phi_0: Y_F\rightarrow A$ such that for every $Z\in\NN_Y^F$, the corresponding continuous $G$-equivariant map $\phi_Z:Z\rightarrow A^G$ satisfies \[\phi_Z[Z]=X.\]
We shall call any such map $\phi_Z$ an \emph{isolated factor map}.

If the subshift $Y$ can be taken to be isolated, i.e. if $X$ is a factor of an isolated subshift, then we call $X$ \emph{strongly projectively isolated}.
\end{definition}

We collect several basic observations and lemmas about projectively isolated subshifts.

\begin{lemma}
If $X\in\SH_G(A)$ is an isolated subshift, then it is also strongly projectively isolated, and a strongly projectively isolated subshift is projectively isolated.
\end{lemma}
\begin{proof}
Let $X$ be strongly projectively isolated, i.e. it is a factor of some isolated subshift $Y$. Using the notation of Definition~\ref{def:projisolated}, we take $\NN_X^F$ to be the neighborhood $\{Y\}$ and we take the identity as $\phi_X$.

If $X$ is isolated, then it is obviously a factor of an isolated subshift, thus strongly projectively isolated.
\end{proof}

\begin{lemma}
Every projectively isolated subshift is sofic. In particular, there are at most countably many projectively isolated subshifts - for each alphabet.
\end{lemma}
\begin{proof}
Suppose that $X$ is projectively isolated and find the corresponding $Y$, $\NN_Y^F$, and $\phi_0$ as in Definition~\ref{def:projisolated}. Since $\NN_Y^F$ contains a subshift $Z$ of finite type (by Lemma~\ref{lem:SFTnbhrds}) and since $\phi_Z[Z]=X$, we get that $X$ is a factor of a subshift of finite type, thus it is sofic.
\end{proof}

\begin{lemma}\label{lem:projshiftproperties}
Let $X$ be a projectively isolated subshift and let $Y$ be a subshift witnessing that $X$ is projectively isolated - as in Definition~\ref{def:projisolated}. Then every subshift that is a factor of $X$ is also projectively isolated and every subshift that factors onto $Y$ also witnesses that $X$ is projectively isolated.
\end{lemma}
\begin{proof}
The first assertion that factor subshifts of projectively isolated subshifts are themselves projectively isolated is obvious. We prove the second assertion. Let $Y$, $\NN_Y^F$ and $\phi_0: Y_F\rightarrow \alpha(X)$ be as in Definition~\ref{def:projisolated} and let $Y'$ be a subshift so that there is a factor map $P:Y'\rightarrow Y$. It is straightforward to see that there is a finite subset $E\subseteq G$ such that \begin{itemize}
	\item $P$ is induced by a finite map $P_0:Y'_E\rightarrow \alpha(Y)$;
	\item for every $Z\in\NN_{Y'}^E$ we have $P_Z[Z]\in\NN_Y^F$.
\end{itemize}
It is then clear that the map $\phi\circ P$, using Lemma~\ref{lem:CHL}, can be defined on any $Z\in\NN_{Y'}^E$ and we have $(\phi\circ P)_Z [Z]=X$.
\end{proof}

\begin{corollary}\label{cor:projisolatedconjugacy}
Being projectively isolated is a conjugacy invariant.
\end{corollary}
\begin{proof}
An isomorphism is a special case of a factor map, so we are done by Lemma~\ref{lem:projshiftproperties}.
\end{proof}
The content of the following lemma is that the map projecting onto a projectively isolated subshift is without loss of generality induced by a map between their respective alphabets.
\begin{lemma}\label{lem:standardformoffactormaps}
Let $X$ be a projectively isolated subshift and let $Y$ be a subshift factoring onto $X$ witnessing that. Then there exists a subshift $Y'$ isomorphic to $Y$ which witnesses that $X$ is projectively isolated via a map $P:Y'\rightarrow X$ which is defined by a map $P_0:\alpha(Y')\rightarrow \alpha(X)$.
\end{lemma}
\begin{proof}
Let $X$ and $Y$ be as in the statement and suppose that the projective isolatedness is witnessed by a map $\phi:Y\rightarrow X$ induced by a finite map $\phi_0:Y_F\rightarrow \alpha(X)$, where $F\subseteq G$ is finite. Consider the partition $\PP:=\{C_p\subseteq Y\colon p\in Y_F\}$ of $Y$ as in Lemma~\ref{lem:SFTfactorofSFT}. Using the notation of Lemma~\ref{lem:SFTfactorofSFT}, $\QQ_\PP^\sigma: Y\rightarrow \PP^G$ induces an isomorphism between $Y$ and its image denote $Y'$. It is then clear that the map $P:=\phi\circ (\QQ_\PP^\sigma)^{-1}: Y'\rightarrow X$ is induced by a map $P_0:\PP\rightarrow\alpha(X)$, finishing the proof.
\end{proof}
\begin{example}
Let us provide an example of a subshift that is strongly projectively isolated, however it is not isolated. Let $X\subseteq \{0,1\}^\Int$ be the shift \[\{x\in \{0,1\}^\Int\colon |x^{-1}(1)|\leq 1\}.\] $X$ cannot be isolated since it is not even a subshift of finite type. Let $Y\subseteq \{-1,0,1\}^\Int$ be a subshift of finite type whose defining window is an interval of length $2$ and the allowed patterns are $\{-1-1,-11,10,00\}$. Let $P:Y\rightarrow X$ be a factor map induced by the finite map $P_0:\{-1,0,1\}\to\{0,1\}$ defined by $P_0(-1)=P_0(0)=0$ and $P_0(1)=1$. One easily checks that $\NN_Y^{\{0,1\}}=\{Y\}$, i.e. $Y$ is isolated, thus $Y$ and $P$ witness that $X$ is projectively isolated.
\end{example}

We conclude this section with one more notion of a subshift, weaker than being sofic, yet still obtained given finite data.
\begin{definition}\label{def:effectiveshubshift}
Let $G$ be a finitely generated and recursively presented group and let $A$ be a non-trivial finite set. Fix a finite symmetric generating set $S\subseteq G$ for $G$. We say that a subshift $X\subseteq A^G$ is \emph{effective} if there exists an algorithm (formally, a Turing machine) that given a pattern $p\in A^F$, which is presented to the algorithm in the form $\{(w_i,a_i)\colon i\leq n\}$, where each $a_i\in A$ and $w_i$ is a word in letters from $S$, decides whether $p$ is allowed in $X$, or not.
\end{definition}
\begin{remark}
The notion of an effective subshift was introduced by Hochman for $\Int^d$-subshifts in \cite{Hoch09}. A general notion for general finitely generated groups was given in \cite{AuBarSab}. The definition above corresponds to their definition jointly with \cite[Proposition 2.1 and Lemma 3.3]{AuBarSab} for recursively presented groups. We will restrict to this case in this paper.
\end{remark}
The following follows from the definition.
\begin{lemma}\label{lem:effectivesubshiifts}
Let $1\to N\to G\to H\to 1$ be a short exact sequence of groups where $G$ and $H$ are finitely generated and recursively presented. We assume that $N\subseteq G$. Let $A$ be a non-trivial finite set, let $X\subseteq A^G$ be a subshift on which $N$ acts trivially, and let $X'\subseteq A^H$ be the corresponding subshift over $H$. Then $X$ is effective if and only if $X'$ is.
\end{lemma}
\subsection{Inverse limits}
Since we shall often use inverse limits of actions of groups on zero-dimensional compact metrizable spaces, we briefly describe their construction here. Let $G$ be a countable group acting continuously, for each $n\in\Nat$, on a zero-dimensional compact metrizable space $X_n$. Suppose that for each $n\geq m\geq k$ there are factor maps $\phi_n^m:X_n\rightarrow X_m$,$\phi_m^k: X_m\rightarrow X_k$, and $\phi_n^k: X_n\rightarrow X^k$ satisfying $\phi_n^k=\phi_m^k\circ\phi_n^m$ and $\phi_n^m=\mathrm{id}$ if $n=m$. Then the \emph{inverse limit} $\underset{n\to\infty}{\varprojlim} X_n$ of the sequence $(X_n)_{n\in\Nat}$, with respect to connecting maps $(\phi_n^m)_{n\geq m}$, is the set \[\{(x_n)_{n\in\Nat}\in \prod_{n\in\Nat} X_n\colon \forall n\geq m\;\big(\phi_n^m(x_n)=x_m\big)\}\] equipped with the diagonal action of $G$. Since $\underset{n\to\infty}{\varprojlim} X_n$ is clearly a $G$-invariant subset of $\prod_{n\in\Nat} X_n$, this is well-defined. Notice also that it is a closed subset of a countable product of zero-dimensional compact metrizable spaces, therefore it is itself zero-dimensional compact metrizable space. We shall also denote by $\phi^m$, for $m\in\Nat$, the factor map from $\underset{n\to\infty}{\varprojlim} X_n$ onto $X_m$.

The following fact will not be directly used, so we leave the proof for the reader.
\begin{fact}
Let $G$ be a countable group and fix $\alpha\in\Act_G(\CC)$ and some compatible metric $d$ on $\CC$. Let $(\PP_n)_{n\in\Nat}$ be a sequence of refining clopen partitions of $\CC$ such that $\lim_{n\to\infty} \max\{\mathrm{diam}_d(P)\colon P\in\PP_n\}=0$. Then $\alpha$ is an inverse limit of $(\QQ(\alpha,\PP_n))_{n\in\Nat}$.
\end{fact}
\section{Strong topological Rokhlin property}\label{sect:mainproofs}
This section is devoted to proving Theorem~\ref{thm:intro1} from Introduction. We restate it more precisely below.
\begin{theorem}\label{thm:mainRokhlin}
Let $G$ be a countable group. Then $G$ has the strong topological Rokhlin property if and only if for every $n\geq 2$, the set of projectively isolated subshifts in $\SH_G(n)$ is dense.
\end{theorem}

We prove the two implications separately. 
\subsection{Failure of the strong topological Rokhlin property}
\begin{proposition}\label{prop:noRokhlin}
Let $G$ be a countable group. If there exists $n\geq 2$ such that the projectively isolated subshifts are not dense in $\SH_G(n)$, then $G$ does not have the strong topological Rokhlin property. In fact, every conjugacy class in $\Act_G(\CC)$ is meager.
\end{proposition}
\begin{proof}
We fix a countable group $G$ without the STRP.

Let $n\geq 2$ be such that the set of projectively isolated subshifts is not dense in $\SH_G(n)$. That is, there is a non-empty open set $U\subseteq \SH_G(n)$ that contains no projectively isolated subshifts. For every $X\in U$ set \[\begin{split} A(X):= \{ & \alpha\in \Act_G(\CC)\colon\\ & \text{for no clopen partition } \PP=\{P_1,\ldots,P_n\}\text{ of }\CC,\; \QQ(\alpha,\PP)= X\}.\end{split}\]\bigskip

\noindent{\bf Claim.} \emph{For every $X\in U$, $A(X)$ is a dense $G_\delta$ set.}\bigskip

We prove the claim. Fix a clopen partition $\PP=\{P_1,\ldots,P_n\}$ and we show that the set \[\AAA^\PP_X:=\{\alpha\in\Act_G(\CC)\colon \QQ(\alpha,\PP)\neq X\}\] is open and dense. In order to show that it is open, we use Proposition~\ref{prop:Qcontinuity} telling us that the map $\QQ(\cdot,\PP)$ is continuous. It follows that \[\AAA^\PP_X=\QQ^{-1}(\cdot,\PP)\Big(\SH_g(n)\setminus\{X\}\Big)\] is open as a preimage of an open set.

We now show that $\AAA^\PP_X$ is dense. Fix some open set $V\subseteq \Act_G(\CC)$. We claim that there is $\alpha\in V$ such that $\QQ(\alpha,\PP)\neq X$. 

We may suppose that $V$ is of the form $\NN_\beta^{F,\PP'}$, for some $\beta\in\Act_G(\CC)$, finite symmetric $F\subseteq G$ containing the unit $1_G$, and a clopen partition $\PP'\preceq\PP$. By Proposition~\ref{prop:Qmap-nbhds}~\eqref{it1-Qmap-nbhds}, setting $Y:=\QQ(\beta,\PP')$, we have $\QQ(\cdot,\PP')[V]=\NN_Y^F$. Let $\phi:(\PP')^G\rightarrow\PP^G$ be the factor map induced by the inclusion map $\phi_0:\PP'\rightarrow\PP$ defined so that for every $P\in\PP'$ we have $P\subseteq \phi_0(P)$. Since $X$ is not projectively isolated, there exists $Y'\in\NN_Y^F$ such that $\phi[Y']\neq X$. Again by Proposition~\ref{prop:Qmap-nbhds}~\eqref{it1-Qmap-nbhds}, there is some $\gamma\in V$ satisfying $\QQ(\gamma,\PP')=Y'$. However, then by Proposition~\ref{prop:Qmap-nbhds}~\eqref{it2-Qmap-nbhds} we get \[\QQ(\gamma,\PP)=\phi\big(\QQ(\gamma,\PP')\big)=\phi[Y']\neq X,\] which finishes the proof that $\AAA_X^\PP$ is dense since $V$ was arbitrary.\medskip

Now we just notice that there are only countably many partitions $\PP$ of $\CC$ into disjoint non-empty clopen $n$-many sets, the set of such denoted by $\mathbb{P}_n$, and therefore \[A(X)=\bigcap_{\PP\in\mathbb{P}_n} \AAA^\PP_X,\] which is dense $G_\delta$ by the Baire category theorem.

This finishes the proof of the claim.$\qed$\bigskip

In order to reach a contradiction, suppose now that $G$ does have the strong topological Rokhlin property, i.e. there is $\alpha\in\Act_G(\CC)$ whose conjugacy class is comeager in $\Act_G(\CC)$. Since the conjugacy class is dense it has to intersect the open set \[\UU:=\QQ^{-1}(\cdot,\PP)\big(U\big),\] where $\PP:=\{P_1,\ldots,P_n\}$ is an arbitrary partition of $\CC$ into disjoint non-empty clopen $n$-many sets. Without loss of generality, we assume that $\alpha\in\UU$. Set $X:=\QQ(\alpha,\PP)\in U$. Since by the claim, $A(X)$ is dense $G_\delta$, it has to intersect the conjugacy class of $\alpha$. So there is some $\varphi\in\Homeo(\CC)$ such that \[\alpha':=\varphi \alpha \varphi^{-1}\in A(X).\] Set $\PP':=\{\varphi P_1,\ldots,\varphi P_n\}$. By definition, we have \[\QQ(\alpha',\PP')=\QQ(\alpha,\PP)=X,\] which contradicts that $\alpha'\in A(X)$. 

Applying Fact~\ref{fact:0-1-law}, we get that actually every conjugacy class is meager. This finishes the proof.
\end{proof}

\subsection{Establishing the strong topological Rokhlin property}\hfill\medskip

This subsection is devoted to the proof of the other implication of Theorem~\ref{thm:mainRokhlin}.\bigskip

We assume that for each $n\geq 2$, the set of projectively isolated subshifts in $\SH_n(G)$ is dense and we prove that $G$ has the strong topological Rokhlin property. The generic action $\alpha\in\Act_G(\CC)$ will be constructed as a projective limit $\underset{n\to\infty}{\varprojlim} X_n$ of any sequence $(X_n)_{n\in\Nat}$ of projectively isolated subshifts, with connnecting maps $\varphi_n^m: X_n\twoheadrightarrow X_m$ that are isolated factor maps, over alphabets of various sizes, which satisfies the following properties (the factor maps between projectively isolated subshifts are assumed to be isolated factor maps).

\begin{equation}\label{eq:universality}
\forall n\geq 2\;\forall Y\in\SH_G(n)\text{ projectively isolated }\exists m\in\Nat\;\exists\psi_m^Y: X_m\twoheadrightarrow Y.
\end{equation}
In words, this property says that for any projectively isolated subshift $Y$ there is an element $X_m$ from the sequence from which there is a factor map onto $Y$.

\begin{equation}\label{eq:cofinality}
\begin{split}
& \forall m\in\Nat\;\forall n\geq 2\;\forall Y\in\SH_G(n)\text{ projectively isolated }\forall \psi_Y^m: Y\twoheadrightarrow X_m\\ & \exists m'>m\;\exists \psi_{m'}^Y: X_{m'}\twoheadrightarrow Y\; (\psi_Y^m\circ \psi_{m'}^Y=\varphi_{m'}^m).
\end{split}\end{equation}
In words, this property says that for any projectively isolated subshift $Y$ that factors onto some element $X_m$ of the sequence, there is a further element $X_{m'}$ in the sequence that factors onto $Y$ such that the canonical factor map from $X_{m'}$ onto $X_m$ is equal to the composition of the factor maps from $X_{m'}$ onto $Y$ and then from $Y$ onto $X_m$.\bigskip

We need to show four things. 
\begin{enumerate}[(a)]
	\item The existence, i.e. that there exists such a sequence of projectively isolated subshifts.
	\item The uniqueness, i.e. that given two sequences with such properties, the resulting inverse limits are conjugate.
	\item The `Cantorness' of the action, that is, the inverse limit, which is by definition a continuous action of $G$ on some zero-dimensional compact metrizable space, is indeed a Cantor action, i.e. that the inverse limit as a topological space has no isolated points.
	\item The genericity, i.e. that the conjugacy class of such an inverse limit is dense $G_\delta$ in $\Act_G(\CC)$.\bigskip
\end{enumerate}

Although these four facts are naturally enumerated in the order as above, we shall prove (b) first, before proving (a).

At this point, we would like to inform the reader whose is familiar with Fra\"iss\'e theory that the generic action we aim to construct is in fact the Fra\"iss\'e limit of projectively isolated subshifts. However, since appealing to the general Fra\"iss\'e theorem would not simplify the proof much and instead we would need to introduce the general Fra\"iss\'e theory, below we provide a self-contained proof. Reader familiar with Fra\"iss\'e theory will notice that (b), the uniqueness, would be a consequence of the Fra\"iss\'e theorem. The existence, step (a), would be also a consequence of the general theorem provided we know that the set of projectively isolated subshifts satisfies the so-called joint embedding and amalgamation properties, which is what we are essentially proving in Claim 1 and Claim 2 respectively and what consists the main part of step (a). Steps (c) and (d) do not follow from the general theory. We refer to \cite{Kub} for the modern category-theoretic treatment of Fra\" iss\' e theory which may enlighten the construction below to the reader. There is also a version of Fra\" iss\' e theory specifically developed in the projective topological setting in \cite{IrwSol} that was applied in \cite{Kwia}.
  
\subsubsection{Step (b)}
\begin{lemma}\label{lem:MAINuniqueness}
Let $(X_n)_{n\in\Nat}$, resp. $(Y_m)_{m\in\Nat}$ be two sequences of projectively isolated subshifts, with connecting maps $\varphi_n^m:X_n\twoheadrightarrow X_m$, resp. $\psi_n^m: Y_n\twoheadrightarrow Y_m$, satisfying \eqref{eq:universality} and \eqref{eq:cofinality}. Then their inverse limits are conjugated.
\end{lemma}
\begin{proof}[Proof of Lemma~\ref{lem:MAINuniqueness}]
We fix the sequences as in the statement. We first use \eqref{eq:universality} of $(Y_m)_{m\in\Nat}$ to find $m_1\in\Nat$ and a factor map $f_1: Y_{m_1}\twoheadrightarrow X_1$. Then we use \eqref{eq:cofinality} of $(X_n)_{n\in\Nat}$ applied to the factor map $f_1: Y_{m_1}\twoheadrightarrow X_1$ to find $n_1>1$ and a factor map $g_1: X_{n_1}\twoheadrightarrow Y_{m_1}$. In the next step, we use \eqref{eq:cofinality} of $(Y_m)_{m\in\Nat}$, resp. of $(X_n)_{n\in\Nat}$ to find $m_2>m_1$, resp. $n_2>n_1$, and factor maps $f_2: Y_{m_2}\twoheadrightarrow X_{m_1}$, resp. $g_2: X_{n_2}\twoheadrightarrow Y_{m_2}$.

Continuing analogously, we obtain strictly increasing sequences $m_1<m_2<m_3<\ldots$ and $1<n_1<n_2<n_3<\ldots$ and factor maps $f_k: Y_{m_k}\twoheadrightarrow X_{n_{k-1}}$, resp. $f_k: X_{n_k}\twoheadrightarrow Y_{m_k}$, for each $k\geq 2$. Moreover, the maps have the property that for every $k\geq 2$ we have \[f_k\circ g_k: X_{n_k}\rightarrow X_{n_{k-1}}=\varphi_{n_k}^{n_{k-1}}\] and \[g_{k-1}\circ f_k: Y_{m_k}\rightarrow Y_{m_{k-1}}=\psi_{m_k}^{m_{k-1}}.\] It follows that the inverse limit of the sequence \[X_1\xleftarrow{f_1}Y_{m_1}\xleftarrow{g_1} X_{n_1}\xleftarrow{f_2}Y_{m_2}\xleftarrow{g_2}X_{n_2}\xleftarrow{g_3}\ldots\]

is equal (up to conjugacy) to both the inverse limits of the sequences
\[X_1\xleftarrow{\varphi_2^1}X_2\xleftarrow{\varphi_3^2}X_3\xleftarrow{\varphi_4^3}X_4\xleftarrow{\varphi_5^4}\ldots\]
and
\[Y_1\xleftarrow{\psi_2^1}Y_2\xleftarrow{\psi_3^2}Y_3\xleftarrow{\psi_4^3}Y_4\xleftarrow{\psi_5^4}\ldots\] which finishes the proof.
\end{proof}

Let us say that two inverse sequences of subshifts $(X_n)_n$ and $(Y_n)_n$ with connecting factor maps $(\phi_n^m:X_n\rightarrow X_m)_{n,m}$, resp. $(\psi_n^m:Y_n\rightarrow Y_m)_{n,m}$ are \emph{strongly isomorphic} if for each $n\in\Nat$ there is an isomorphism $f_n: X_n\rightarrow Y_n$ and moreover $\phi_n^{n-1}=f_{n-1}^{-1}\circ\psi_n^{n-1}\circ f_n$.
\begin{lemma}\label{lem:stronglyisomorphic}
Let $\beta\in\Act_G(\CC)$ be an inverse limit of an inverse sequence of subshifts $(X_n)_n$ with respect to connecting maps $(\phi_n^m)_{n\geq m}$. Then there exists a refining sequence of clopen partitions $\PP_1\succeq\PP_2\succeq\PP_3\succeq\ldots$ such that the inverse sequences of the two sequences of subshifts $(X_n)_n$ and $(\QQ(\beta,\PP_n))_n$ are strongly isomorphic.
\end{lemma}
\begin{proof}
We construct the partitions $(\PP_n)_n$ and the isomorphisms $(f_n)_n$ by induction. For the first step, the action $\beta$ factors by $\phi^1$ onto $X_1$ and $\phi^1$ is, by Lemma~\ref{lem:factoringonshift}, induced by some clopen partition which we shall denote by $\PP_1$. We may thus set $f_1$ to be the identity.

Now assume we have found a partition $\PP_{n-1}$ and an isomorphism $f_{n-1}: X_{n-1}\rightarrow \QQ(\beta,\PP_{n-1})$ with the requested properties. Consider the factor map $f_{n-1}\circ \phi_n^{n-1}: X_n\rightarrow\QQ(\beta,\PP_{n-1})$. By the Curtis-Hedlund-Lyndon theorem, it is induced by some finite map $P_0: (X_n)_F\rightarrow \PP_{n-1}$, where $F\subseteq G$ is finite. $(X_n)_F$ induces a clopen partition $\PP'_n:=\{C_p(X_n)\colon p\in (X_n)_F\}$ of $X_n$ and its preimage via the factor map $\phi^n:\beta\rightarrow X_n$ is a clopen partition $\PP_n$ of $\CC$. By Lemma~\ref{lem:SFTfactorofSFT}, $X_n$ and $\QQ(\beta,\PP_n)$ are isomorphic via the canonical isomorphism $f_n:X_n\rightarrow \QQ(\beta,\PP_n)$ induced by the finite canonical bijection $f'_n: (X_n)_F\rightarrow\PP_n$ which sends $p\in (X_n)_F$ to $P\in\PP_n$ if and only if \[P=(\phi^n)^{-1}(\{x\in X_n\colon x\upharpoonright F=p\}).\] It is straightforward to verify that $\PP_n$ and $f_n$ have the desired properties.
\end{proof}

There is another issue that we want to address here. Notice that the conditions \eqref{eq:universality} and \eqref{eq:cofinality} were formulated with respect to a fixed sequence. However, an action can be written as an inverse limit of two possibly different sequences. The next lemma shows that this is in fact not an issue at all.
\begin{lemma}\label{lem:independencyonsequence}
Let $\beta\in\Act_G(\CC)$ be an action that can be written both as an inverse limit of projectively isolated subshifts $(X_n)_{n\in\Nat}$ as well as the inverse limit of projectively isolated subshifts $(Y_n)_{n\in\Nat}$. 
\begin{enumerate}[(a)]
	\item\label{it:universality} Suppose that $\beta$ satisfies \eqref{eq:universality} with respect to $(X_n)_{n\in\Nat}$. Then it does with respect to $(Y_n)_{n\in\Nat}$ as well.
	\item\label{it:cofinality} Suppose that $\beta$ satisfies \eqref{eq:cofinality} with respect to $(X_n)_{n\in\Nat}$. Then it does with respect to $(Y_n)_{n\in\Nat}$ as well.
\end{enumerate}

\end{lemma}
\begin{proof}
We fix $\beta$, $(X_n)_{n\in\Nat}$ and $(Y_n)_{n\in\Nat}$ as in the statement. We assume that the connective factor maps between $X_n$ and $X_m$, resp. between $Y_n$ and $Y_m$, for $n\geq m$, are $\phi_n^m:X_n\rightarrow X_m$, resp. $\psi_n^m: Y_n\rightarrow Y_m$.

Let us first assume that $(X_n)_n$ and $(Y_n)_n$ are strongly isomorphic. Then it is easy to check both \ref{it:universality} and \ref{it:cofinality} by a simple diagram chasing. Therefore, applying Lemma~\ref{lem:stronglyisomorphic}, we may without loss of generality assume that there are two refining sequences of clopen partitions \[\PP_n\succeq\PP_2\succeq\PP_3\succeq\ldots\]
and
\[\PP'_1\succeq\PP'_2\succeq\PP'_3\succeq\ldots\] such that for every $n\in\Nat$,
\begin{itemize}
	\item  $\QQ(\beta,\PP_n)=X_n$ and $\QQ(\beta,\PP'_n)=Y_n$;
	\item the connectives maps $\phi_n^m$, resp. $\psi_n^m$, for $n\geq m$, are of the form $\phi_{\PP_n}^{\PP_m}$, resp. $\phi_{\PP'_n}^{\PP'_m}$.
\end{itemize}

Since the diameteres of the elements of both $\bigcup_n \PP_n$ and $\bigcup_n \PP'_n$ must tend to $0$, for every $n\in\Nat$ there are $m,m'\in\Nat$ such that $\PP'_{m'}\preceq\PP_n$ and $\PP_m\preceq\PP'_n$.\medskip
	
We start with \ref{it:universality}. We suppose that $\beta$ satisfies \eqref{eq:universality} with respect to $(X_n)_{n\in\Nat}$ and we show it does also with respect to $(Y_n)_{n\in\Nat}$.

Fix a projectively isolated subshift $Z$. Since $\beta$ satisfies \eqref{eq:universality} with respect to $(X_n)_n$ there exists $n\in\Nat$ and a factor map $\phi_n^Z:X_n\rightarrow Z$. However, as argued above, there is $m\in\Nat$ such that $\PP'_m\preceq \PP_n$, therefore the composition $\phi_n^Z\circ \phi_{\PP'_m}^{\PP_n}$ is the desired factor map from $Y_m$ onto $Z$.\medskip

Next we continue with \ref{it:cofinality}. Suppose that $Z$ is a projectively isolated subshift that factors onto $Y_m$ via some $\phi_Z^{Y_m}$, for some $m\in\Nat$. There exists $k\in\Nat$ such that $\PP_k\preceq\PP'_m$, thus there is a factor map $\phi_{\PP_k}^{\PP'_m}: X_k\rightarrow Y_m$. Applying Claim 2 from Proposition~\ref{prop:MAINexistence}, there exists a projectively isolated subshift $Z'$ and factor maps $\phi_{Z'}^Z:Z'\rightarrow Z$, $\phi_{Z'}^{X_k}: Z'\rightarrow X_k$ such that $\phi_{\PP_k}^{\PP'_m}\circ \phi_{Z'}^{X_k}=\phi_Z^{Y_m}\circ \phi_{Z'}^Z$. Since $\beta$ satisfies \eqref{eq:cofinality} with respect to $(X_n)_n$ there exists $l\in\Nat$ and a factor map $\phi_{X_l}^{Z'}$ such that $\phi_l^k=\phi_{Z'}^{X_k}\circ \phi_{X_l}^{Z'}$. Finally, there is $n\in\Nat$ such that $\PP'_n\preceq\PP_l$. We claim that the map \[\phi_{Y_n}^Z:=\phi_{Z'}^Z\circ \phi_{X_l}^{Z'}\circ \phi_{\PP'_n}^{\PP_k}: Y_k\rightarrow Z\] is as desired. The proof is done by a simple diagram chasing.
\end{proof}
\subsubsection{Step (a)}

\begin{proposition}\label{prop:MAINexistence}
There exists an inverse sequence $(X_n)_{n\in\Nat}$ of projectively isolated subshifts, with connecting maps $\varphi_n^m: X_n\twoheadrightarrow X_m$ being factor maps, which satisfies the properties \eqref{eq:universality} and \eqref{eq:cofinality}.
\end{proposition}
\begin{proof}[Proof of Proposition~\ref{prop:MAINexistence}]
We shall need two claims.\bigskip

\noindent{\bf Claim 1.} \emph{Let $X$ and $Y$ be two projectively isolated subshifts, possibly over different alphabets. Then there exists a projectively isolated subshift $Z$ which factors on both $X$ and $Y$.}\bigskip

We prove the claim. Let $X'$, resp. $Y'$ be projectively isolated subshifts such that there are a finite symmetric $F\subseteq G$ and factor maps $\psi_X$, resp. $\psi_Y$ such that for every $X''\in\NN_{X'}^F$, resp. every $Y''\in\NN_{Y'}^F$ \[\psi_X[X'']=X\;\text{ and }\;\psi_Y[Y'']=Y.\]

First we set $Z_0:=X'\times Y'\subseteq nm^G$ to be the product subshift. We denote by $P_X:Z_0\to X'$, resp. $P_Y: Z_0\to Y'$ the coordinate projection maps. Notice that for any $Z'\in\NN_{Z_0}^F$ we have $P_X[Z']\in\NN_{X'}^F$ and analogously $P_Y[Z']\in\NN_{Y'}^F$. Since projectively isolated shifts are dense, we may find a projectively isolated $Z\in\NN_{Z_0}^F$. Since, as noted above, $P_X[Z]\in\NN_{X'}^F$ and $P_Y[Z]\in\NN_{Y'}^F$, we get \[\psi_X\circ P_X[Z]=X\;\text{ and }\;\psi_Y\circ P_Y[Z]=Y,\] which finishes the proof of the claim.$\qed$.\bigskip

\noindent {\bf Claim 2.} \emph{Assume that we are given projectively isolated subshifts $V$, $X$ and $Y$, and factor maps $V_X: X\rightarrow V$ and $V_Y: Y\rightarrow V$. Then there exists a projectively isolated subshift $Z$ and factor maps $P_X:Z\rightarrow X$ and $P_Y:Z\rightarrow Y$ such that $V_X\circ P_X=V_Y\circ P_Y$.}\bigskip

We prove the claim. Let again $X'$, resp. $Y'$ be projectively isolated subshifts such that there are a finite symmetric $F\subseteq G$ and factor maps $\psi_X$, resp. $\psi_Y$ such that for every $X''\in\NN_{X'}^F$, resp. every $Y''\in\NN_{Y'}^F$ \[\psi_X[X'']=X\;\text{ and }\;\psi_Y[Y'']=Y.\] We moreover assume that $F$ is big enough so that the maps $\psi_X$, resp. $\psi_Y$ are induced by finite maps $\psi_X^0: \alpha(X')^F\to \alpha(X)$, resp. $\psi_Y^0: \alpha(Y')^F\to \alpha(Y)$. Analogously, the maps $V_X$, resp. $V_Y$ are induced by finite maps $V_X^0: \alpha(X)^F\to \alpha(V)$, resp. $V_Y^0: \alpha(Y)^F\to \alpha(V)$. In particular, it follows that the composition maps $V_X\circ \psi_X$, resp. $V_Y\circ\psi_Y$ are induced by finite maps $W_X^0:\alpha(X')^{F^2}\to \alpha(V)$, resp. $W_Y^0:\alpha(Y')^{F^2}\to \alpha(V)$.

First we set \[Z_0:=\{(x,y)\in X'\times Y'\colon V_X\circ\psi_X(x)=V_Y\circ\psi_Y(y)\},\] and we denote by $P_X$, resp. $P_Y$ the coordinate projections of $Z_0$ into $X'$, resp. $Y'$. We claim that for any $Z\in\NN_{Z_0}^{F^2}$ we have \begin{enumerate}
	\item $\psi_X\circ P_X[Z]=X$;
	\item $\psi_Y\circ P_Y[Z]=Y$;
	\item for any $z\in Z$ we have \[V_X\circ \psi_X\circ P_X(z)=V_Y\circ\psi_Y\circ P_Y(z).\]
\end{enumerate}

Fix $Z\in\NN_{Z_0}^{F^2}$. We first prove (1), (2) is symmetric. In order to prove $\psi_X\circ P_X[Z]=X$, and that even the map $\psi_X\circ P_X$ is defined on $Z$, it suffices to show that $P_X[Z]\in\NN_{X'}^F$. It is clear that $P_X[Z]_F\subseteq X'_F$, so we need to prove the other inclusion. Pick any finite pattern $p\in X'_F$, some $x\in X'$ such that $x\upharpoonright F=p$, and some $y\in Y'$ such that $V_X\circ\psi_X(x)=V_Y\circ\psi_Y(y)$. Then $(x,y)\upharpoonright F\in (Z_0)_F$, thus $(x,y)\upharpoonright F\in Z_F$, so for some $z\in Z$ we have $z\upharpoonright F=(x,y)\upharpoonright F$, so $P_X(z)\upharpoonright F=p$, and we are done.

We now show (3). Suppose that there is $z\in Z$ such that $V_X\circ\psi_X\circ P_X(z)\neq V_Y\circ\psi_Y\circ P_Y(z)$ and assume without loss of generality that $V_X\circ\psi_X\circ P_X(z)(1_G)\neq V_Y\circ\psi_Y\circ P_Y(z)(1_G)$. It follows that $W_X^0\circ P_X(z)\upharpoonright F^2\neq W_Y\circ P_Y(z)\upharpoonright F^2$. However, by definition, there must be $z_0\in Z_0$ such that $z_0\upharpoonright F^2=z\upharpoonright F^2$. Consequently, $V_X\circ\psi_X\circ P_X(z_0)(1_G)\neq V_Y\circ\psi_Y\circ P_Y(z_0)(1_G)$, so $V_X\circ\psi_X\circ P_X(z_0)\neq V_Y\circ\psi_Y\circ P_Y(z_0)$, which contradicts the definition of $Z_0$.

Now, using that projectively isolated subshifts are dense, it suffices to pick projectively isolated $Z\in\NN_{Z_0}^{F^2}$, and we are done. This finishes the proof of Claim 2.$\qed$.\bigskip

We can now finish the proof of the proposition. Let $(Z_n)_{n\in\Nat}$ be an enumeration of all projectively isolated subshifts from $\bigcup_{m\in\Nat} \SH_G(m)$. Let $(\psi_n: Y_n\rightarrow Y'_n)_{n\in\Nat}$ be an enumeration of all isolated factor maps between two projectively isolated subshifts, with an infinite repetition.

We recursively construct the inverse sequence \[X_1\xleftarrow{\varphi_2^1}X_2\xleftarrow{\varphi_3^2}X_3\xleftarrow{\varphi_4^3}X_4\xleftarrow{\varphi_5^4}\ldots.\] In the first odd step, we set $X_1:=Z_1$. In the first even step, we consider the factor map $\psi_1: Y_1\rightarrow Y'_1$. If $X_1$ is not equal (up to conjugacy) to $Y'_1$, we choose $X_2$ and $\varphi_2^1: X_2\rightarrow X_1$ arbitrarily so that we get an isolated factor map between two projectively isolated subshifts. If $X_1$ is equal (up to conjugacy) to $Y'_1$, then we set $X_2:=Y_1$ and $\varphi_2^1:=\psi_1$.

Suppose we have already constructed \[X_1\xleftarrow{\varphi_2^1}X_2\xleftarrow{\varphi_3^2}\ldots\xleftarrow{\varphi_{2m}^{2m-1}}X_{2m},\] for some $m\in\Nat$. We now choose $X_{2m+1}$ and $\varphi_{2m+1}^{2m}: X_{2m+1}\rightarrow X_{2m}$. We apply Claim 1 to obtain a projectively isolated subshift $Z$ which factors both on $X_{2m}$ and $Z_m$. We set $X_{2m+1}:=Z$ and we set $\varphi_{2m+1}^{2m}$ to be the factor map from $X_{2m+1}=Z$ onto $X_{2m}$.

Now we choose $X_{2m+2}$ and $\varphi_{2m+2}^{2m+1}$. We consider the factor map $\psi_m: Y_m\rightarrow Y'_m$. If $Y'_m$ is not equal to $X_k$, for any $k\leq 2m$, then we extend the sequence to $X_{2m+2}$ arbitrarily. So suppose that $Y'_m=X_k$ (up to conjugacy) for some $k\leq 2m$. Then we apply Claim 2 to the factor maps $\psi_m: Y_m\rightarrow X_k$ and $\varphi_{2m+1}^k:=\varphi_{k+1}^k\circ\ldots\circ\varphi_{2m+1}^{2m}$ to obtain a projectively isolated subshift $Z$ and factor maps $P_Y: Z\rightarrow Y_m$ and $P_X: Z\rightarrow X_{2m+1}$ such that $\psi_m\circ P_Y=\varphi_{2m+1}^k\circ P_X$. We set $X_{2m+2}:=Z$ and $\varphi_{2m+2}^{2m+1}:=P_X$. This finishes the description of the recursive construction.

We finish the proof by showing that the constructed inverse sequence \[X_1\xleftarrow{\varphi_2^1}X_2\xleftarrow{\varphi_3^2}X_3\xleftarrow{\varphi_4^3}X_4\xleftarrow{\varphi_5^4}\ldots\] is as desired, i.e. it satisfies conditions \eqref{eq:universality} and \eqref{eq:cofinality}. For \eqref{eq:universality}, let $Y$ be an arbitrary projectively isolated subshift - in $\SH_G(n)$, for some $n\geq 2$. Then $Y=Z_m$ for some $m\in\Nat$, and we have guaranteed during the recursion process that there is a factor map from $X_{2m+1}$ onto $Z_m=Y$.

For \eqref{eq:cofinality}, let $Y$ be some projectively isolated subshift and $\psi: Y\rightarrow X_n$ a factor map, for some $n\in\Nat$. Then $\psi=\psi_m$, for some $m>n$. The latter condition, $m>n$, can be achieved, since $(\psi_m)_{m\in\Nat}$ is an enumeration with an infinite repetition. Then we have guaranteed during the recursion process that there is a factor map $\phi: X_{2m+2}\rightarrow Y$ such that $\psi\circ \phi=\varphi_{2m+2}^n$. This finishes the proof.
\end{proof}

\subsubsection{Step (c)}
\begin{lemma}\label{lem:inverseisCantor}
The inverse limit of the sequence $(X_n)_{n\in\Nat}$ is, as a topological space, homeomorphic to the Cantor space.
\end{lemma}
\begin{proof}
Each of the subshifts $X_n$, $n\in\Nat$, is as a topological space a zero-dimensional compact metrizable space. The inverse limit is, as a topological space, therefore zero-dimensional compact metrizable as well. So in order to show that it is homeomorphic to $\CC$ we only need to show it has no isolated points. Denote by $\mathbb{X}$ the inverse limit as a topological space which we recall is equal to \[\{(x_n)_{n\in\Nat}\in\prod_{n\in\Nat} X_n\colon \forall n>m\; (\varphi_n^m(x_n)=x_m)\}\] with the topology inherited from the product topology of $\prod_{n\in\Nat} X_n$. Pick any $(x_n)_{n\in\Nat}\in\mathbb{X}$. In order to show that $(x_n)_{n\in\Nat}$ is not isolated it suffices to show that for every $m\in\Nat$ either $x_m$ is not isolated in $X_m$ or there exist $n>m$ and $y\neq x_n\in X_n$ such that $\varphi_n^m(y)=x_m$. Indeed, this follows from two basic observations. First, that for every $y\in X_n$, for $n\in\Nat$, there exists $(y_n)_{n\in\Nat}\in\mathbb{X}$ such that $y_n=y$. This is immediate since all the maps $\varphi_n^m$ are onto. Second, that if the picked point $(x_n)_{n\in\Nat}$ were isolated, then by the definition of the topology of $\mathbb{X}$ there would be $m\in\Nat$ such that $x_m$ would be isolated in $X_m$ and for every $n>m$ the only element $x\in X_n$ such that $\varphi_n^m(x)=x_m$ would be equal to $x_n$.

Fix $m\in\Nat$ and suppose that $x_m$ is isolated in $X_m$, otherwise we are done, and let us show that there are $n>m$ and $y\neq x_n\in X_n$ such that $\varphi_n^m(y)=x_m$. In order to show that, it in turn suffices to prove that there is an arbitrary projectively isolated subshift $Y$ that factors onto $X_m$, via some $\psi:Y\rightarrow X_m$, and there are $y\neq y'\in Y$ such that $\psi(y)=\psi(y')=x_m$. Indeed, this follows from \eqref{eq:cofinality} since then there is $n>m$ such that $X_n$ factors onto $Y$, via some $\psi': X_n\rightarrow Y$, and we have $\psi\circ\psi'=\varphi_n^m$.

Since $x_m$ is isolated in $X_m$ there exists a finite set $F\subseteq G$ such that $x_m$ is the unique point $x\in X_m$ satisfying \[x_m\upharpoonright F=x\upharpoonright F.\] Let $Z$ be an arbitrary projectively isolated subshift having at least two points.  Without loss of generality, the finite set $F$ is big enough so that 
\begin{itemize}
	\item $Z_F$ contains at least two patterns;
	\item the defining window of $\psi_{m+1}^m$ is $F$;
	\item $\psi_{m+1}^m[X]=X_m$, for every $X\in\NN_{X_{m+1}}^F$;
	\item  there is also a projectively isolated subshift $Z'$ and a factor map $\phi:Z'\rightarrow Z$, with defining window $F$, such that $\phi[Z'']=Z$, for every $Z''\in\NN_{Z'}^F$.
\end{itemize}

Now as in Claim 1 from the proof of Proposition~\ref{prop:MAINexistence}, we can obtain a projectively isolated subshift $Y\in \NN_{X_{m+1}\times Z'}^F$ that factors on both $X_m$ and $Z$, via $\psi:=\psi_{m+1}^m\circ P_1$, resp. $\phi\circ P_2$, where $P_1$ and $P_2$ are the two coordinate projections.

Set $p:=x_m\upharpoonright F^2$. Since $Z_F$ contains at least two patterns and the defining window of $\psi:Z'\rightarrow Z$ is $F$ there exist at least two different patterns $q\neq q'\in Z'_{F^2}$. Since $Y\in \NN_{X_{m+1}\times Z'}^F$, the patterns $p\cdot q\neq p\cdot q'$ are allowed in $Y$, thus let $(x,y)\neq (x',y')\in Y$ be two points such that $(x,y)\upharpoonright F=p\cdot q$ and $(x',y')\upharpoonright F=p\cdot q'$. It follows that \[\psi\big((x,y)\big)\upharpoonright F=\psi\big((x',y')\big)\upharpoonright F.\]

However, since the only element $x\in X_m$ with $x\upharpoonright F=x_m\upharpoonright F$ is $x_m$, we get \[\psi\big((x,y)\big)=\psi\big((x',y')\big)\] and we are done.
\end{proof}

\subsubsection{Step (d)}\hfill\bigskip

We shall now continue denoting the inverse limit of $(X_n)_{n\in\Nat}$ by $\mathbb{X}$, which we have just proved is homeomorphic, as a space, to $\CC$. The elements of $\mathbb{X}$ will be sometimes denoted by simple symbols such as $x$ or $y$, however we shall still use the notations $x_n$, resp. $y_n$, for $n\in \Nat$, which is the $n$-th coordinate of $x$, resp. of $y$.

The space $\mathbb{X}$ is equipped with a continuous action of $G$, which we shall denote by $\gamma$, as it is a closed $G$-invariant subspace of $\prod_{n\in\Nat} X_n$ which is equipped with the product  of the shift actions of $G$. We fix some homeomorphism $\phi: \mathbb{X}\rightarrow \CC$ and set \[\alpha:=\phi \gamma\phi^{-1}\in\Act_G(\CC),\] that is, for $g\in G$ and $x\in\CC$ we have \[\alpha(g)x=\varphi\Big(\gamma(g)\varphi^{-1}(x)\Big).\]

\begin{proposition}\label{prop:densityofgenericaction}
The conjugacy class of $\alpha$ is dense in $\Act_G(\CC)$.
\end{proposition}
\begin{proof}
Fix some basic open set in $\Act_G(\CC)$ which we may suppose to be of the form $\NN_\beta^{F,\PP}$, where $\beta\in\Act_G(\CC)$, $F\subseteq G$ is a finite set, which may assume to be symmetric and containing $1_G$, and $\PP=\{P_1,\ldots,P_n\}$ is a partition of $\CC$ into disjoint non-empty clopen sets. Set $Y:=\QQ(\beta,\PP)$ and notice that by Proposition~\ref{prop:Qmap-nbhds}~\eqref{it1-Qmap-nbhds}, we have that $\NN_Y^F=\QQ(\cdot,\PP)[\NN_\beta^{F,\PP}]$.

Let also $\RR=\{R_p\colon p\in Y_F\}$ be the clopen partition of $\CC$, where for $p\in Y_F$ \[R_p:=\{x\in\CC\colon \forall f\in F\;\forall P\in\PP\; \big(\beta(f^{-1})x\in P\Leftrightarrow p(f)=P\big)\}.\] In other words, $R_p=(Q_\beta^\PP)^{-1}(C_p(Y))$, for $p\in Y_F$. Clearly, $\RR\preceq\PP$.

Since projectively isolated subshifts are dense, there exists such a subshift $Z\in\NN_Y^F$. By \eqref{eq:universality}, there are $m\in\Nat$ and a factor map $\psi_m^Z:X_m\rightarrow Z$. Note that $\PP$ can be also viewed as a clopen partition of $Z$ identifying $P\in\PP$ with $\{z\in Z\colon z(1_G)=P\}$. Let $\PP':=(\psi_m^Z)^{-1}(\PP)=\{P'_1:=(\psi_m^Z)^{-1}(P_1),\ldots,P'_n:=(\psi_m^Z)^{-1}(P_n)\}$ and set 

\[\PP'':=\Big\{P''_i:=\{(x_n)_n\colon x_m\in P'_i\}\colon i\leq n\Big\},\] which is a clopen partition of $\mathbb{X}$.

Since $Z\in \NN_Y^F$ we have $Z_F=Y_F$ and so there exists a homeomorphism $\eta:\mathbb{X}\rightarrow \CC$ satisfying for every $p\in Z_F$ that \[\eta\big[\{x\in\mathbb{X}\colon \forall f\in F\;\forall i\leq n\;\big(\gamma(f^{-1})x\in P''_i\Leftrightarrow p(f)=P_i\big)\}\big]=R_p.\] We now set \[\alpha':=\eta\gamma\eta^{-1}\in\Act_G(\CC).\] Obviously, $\alpha$ and $\alpha'$ are conjugated, so we just need to show that $\alpha'\in \NN_\beta^{F,\PP}$.

Pick an arbitrary $x\in\CC$ and $f\in F$, and suppose that $\beta(f)x\in P_i$, for some $i\leq n$. We must show that $\alpha'(f)x\in P_i$ as well. Let $p\in Z_F$ be such that $x\in R_p$. In particular, we have $p(f^{-1})=P_i$. Then by the definition of $\eta$, for $v:=\eta^{-1}(x)$ we have $\gamma(f)v\in P''_i$, so  $\eta\big(\gamma(f)v\big)\in R_{p'}$, for some $p'\in Z_F$ so that $p'(1_G)=P_i$. So from the definition of $R_{p'}$ it follows that \[\alpha'(f)x=\eta(\gamma(f)v)=\beta(1^{-1}_G)\big(\eta(\gamma(f)v)\big)\in P_i.\]

\end{proof}

\begin{proposition}
The conjugacy class of $\alpha$ is $G_\delta$ in $\Act_G(\CC)$.
\end{proposition}

\begin{proof}
We shall define three $G_\delta$ sets $\GG_1$, $\GG_2$ and $\GG_3$, and show that the conjugacy class $[\alpha]$ of $\alpha$ is their intersection.

We recall that for every $n\geq 2$, by $\mathbb{P}_n$ we denote the set of all partitions of $\CC$ into disjoint non-empty clopen $n$-many sets and we set $\mathbb{P}:=\bigcup_{n\in\Nat} \mathbb{P}_n$. 

For any partitions $\PP'\preceq\PP$, denote by $\phi_{\PP'}^\PP$ the factor map from $(\PP')^G$ onto $\PP^G$ where for any $x\in(\PP')^G$ and $g\in G$ we have \[\phi_{\PP'}^\PP(x)(g)=P\in\PP\text{ if and only if }x(g)\subseteq P.\] Denote by $\ZZ$ the set of all projectively isolated subshifts from $\bigcup_{m\in\Nat} \SH_G(n)$. We note that for any finite alphabet $A$ we shall freely identify $\SH_G(A)$ with $\SH_G(|A|)$. We set \[\begin{split}\GG_1:=\big\{& \beta\in\Act_G(\CC)\colon \forall \PP\in\mathbb{P}\;\exists \PP'\in\mathbb{P}\;\exists X\in\SH_G(\PP') \exists F\subseteq_\mathrm{fin} G\\ & \big(\PP'\preceq \PP\wedge \forall Z,Z'\in\NN_X^F(\phi_{\PP'}^\PP[Z]=\phi_{\PP'}^\PP[Z'])\wedge\QQ(\beta,\PP')\in\NN_X^F\big) \big\}.\end{split}\]

\begin{fact}\label{fact:mainthm1}
The set $\GG_1$ is a $G_\delta$ set of all actions $\beta\in\Act_G(\CC)$ that are inverse limits of projectively isolated subshifts.
\end{fact}
\begin{proof}[Proof of Fact~\ref{fact:mainthm1}]
First we show that $\GG_1$ is $G_\delta$. For that, it suffices to show that for fixed partitions $\PP$ and $\PP'$ such that $\PP'\preceq\PP$, for fixed $X\in \SH_G(\PP')$ and its neighborhood $\NN_X^F$ with the property that $\phi_{\PP'}^\PP[Z]=\phi_{\PP'}^\PP[Z']$, for all $Z,Z'\in\NN_X^F$ (notice that this last condition does not depend on $\beta$), the set \[\{\beta\in\Act_G(\CC)\colon \QQ(\beta,\PP')\in \NN_X^F\}\] is open. This however follows since the map $\QQ(\cdot,\PP')$ is continuous by Proposition~\ref{prop:Qcontinuity}.

Now suppose that $\beta\in\GG_1$ and we show that it is an inverse limit of projectively isolated subshifts. Fix some compatible metric $d$ on $\CC$ and let $\PP_1$ be an arbitrary partition of $\CC$ into disjoint non-empty clopen subsets whose all elements have diameter less than $1/2$ with respect to $d$. Since $\beta\in\GG_1$ there exists a refinement $\PP_2\preceq\PP_1$ such that $\QQ(\beta,\PP_2)$ and the map $\phi_{\PP_2}^{\PP_1}: \QQ(\beta,\PP_2)\rightarrow \QQ(\beta,\PP_1)$ witnesses that $\QQ(\beta,\PP_1)$ is projectively isolated. By Lemma~\ref{lem:projshiftproperties}, we may refine the partition $\PP_2$ if necessary and without loss of generality may assume that all elements of $\PP_2$ have diameter less than $1/2^2$.

We repeat the argument with $\PP_2$ to obtain $\PP_3\preceq \PP_2$, whose elements we may assume have diameter less than $1/2^3$, such that $\QQ(\beta,\PP_3)$ and the map $\phi_{\PP_3}^{\PP_2}: \QQ(\beta,\PP_3)\rightarrow \QQ(\beta,\PP_2)$ witnesses that $\QQ(\beta,\PP_2)$ is projectively isolated. We continue analogously to obtain partition $\PP_n\preceq\PP_{n-1}$, whose elements have diameter less than $1/2^n$, for all $n\in\Nat$.

We claim that the inverse limit of \[\QQ(\beta,\PP_1)\xleftarrow{\phi_{\PP_2}^{\PP_1}}\QQ(\beta,\PP_2)\xleftarrow{\phi_{\PP_3}^{\PP_2}}\QQ(\beta,\PP_3)\xleftarrow{\phi_{\PP_4}^{\PP_3}}\ldots\] is equal to $\beta$. Indeed, it is clear that the map \[x\in\CC\mapsto \big(Q^\beta_{\PP_n}(x)\big)_{n\in\Nat}\] is a factor map onto the inverse limit. So it suffices to check that it is one-to-one. Pick $x\neq y\in\CC$. Since the diameters of the elements of the partitions $(\PP_n)_{n\in\Nat}$ tend to $0$, there exists $n\in\Nat$ such that $x$ and $y$ lie in different elements of the partition $\PP_n$, so $Q^\beta_{\PP_n}(x)\neq Q^\beta_{\PP_n}(y)$, and therefore the map above is one-to-one.\medskip

Now conversely assume that an action $\beta\in\Act_G(\CC)$ is an inverse limit \[Y_1\xleftarrow{\psi_2^1}Y_2\xleftarrow{\psi_3^2}Y_3\xleftarrow{\psi_4^3}\ldots\] of projectively isolated subshifts and we shall show that $\beta\in\GG_1$. Applying Lemma~\ref{lem:stronglyisomorphic}, we may assume that there exists a refining sequence of partitions $\PP_1\succeq\PP_2\succeq\PP_3\succeq\ldots$ of $\CC$ whose elements have diameters going to $0$ and such that for every $n\geq 1$, $\QQ(\beta,\PP_n)$ is a projectively isolated subshift of $\PP_n^G$ (using Corollary~\ref{cor:projisolatedconjugacy}).

We now verify that $\beta$ satisfies the condition from the definition of $\GG_1$. Let $\PP$ be an arbitrary partition of $\CC$. Since the diameters of the elements of the partitions $(\PP_n)_{n\geq 1}$ tend to $0$ we get that for a sufficiently large $n$, $\PP_n\preceq\PP$ (whose elements have diameter less than the Lebesgue number of $\PP$). By definition, $\QQ(\beta,\PP_n)$ is projectively isolated as witnessed by $\QQ(\beta,\PP_{n+1})$ and the map $\phi_{\PP_{n+1}}^{\PP_n}$. By Lemma~\ref{lem:projshiftproperties}, $\QQ(\beta,\PP)$ is projectively isolated as well, witnessed again by $\QQ(\beta,\PP_{n+1})$ and the map $\phi_{\PP_{n+1}}^\PP$. We can get then take $X$ to be $\QQ(\beta,\PP_{n+1})$, $\PP'$ to be $\PP_{n+1}$. These together with sufficiently large $F\subseteq G$ verify the condition from $\GG_1$.
\end{proof}
\end{proof}
 

 Next for any $Z\in\ZZ$ we shall denote by $\NN(Z)$ some open neighborhood of a subshift $Y$ such that there is a map $\phi_Z$ defined on all subshifts of $\NN(Z)$ and such that $\phi_Z[Y']=Z$, for all $Y'\in\NN(Z)$.
 
We set \[\GG_2:=\big\{\beta\in\Act_G(\CC)\colon \forall Z\in\ZZ\; \exists\PP\in\mathbb{P}\; \big(\QQ(\beta,\PP)\in\NN(Z)\big)\big\}.\] Again, it is easy to show that $\GG_2$ is $G_\delta$. Indeed, it suffices to notice that for a fixed $Z\in\ZZ$ and a partition $\PP$, the set $\{\beta\in\Act_G(\CC)\colon \QQ(\beta,\PP)\in\NN(Z)\}$ is open, which is clear as $\QQ(\cdot,\PP)$ is continuous. 
 
 \begin{fact}\label{fact:mainthm2}
 The set $\GG_1\cap\GG_2$ is a $G_\delta$ set of those actions of $G$ on $\CC$ that are inverse limits of isolated subshifts and that satisfy \eqref{eq:universality}.
 \end{fact}
\begin{proof}[Proof of Fact~\ref{fact:mainthm2}]
We have already noticed that it is $G_\delta$. Suppose that $\beta\in\GG_1\cap\GG_2$. By Fact~\ref{fact:mainthm1}, $\beta$ is an inverse limit of isolated subshifts. In fact, by the proof of Fact~\ref{fact:mainthm1} or applying Lemma~\ref{lem:stronglyisomorphic}, we may suppose that there is a sequence of refining partitions $\PP_1\succeq\PP_2\succeq\PP_3\succeq\ldots$ whose elements have diameters going to $0$ and such that for each $n\in\Nat$, $Y_n:=\QQ(\beta,\PP_n)$ is projectively isolated; that is, $\beta$ is an inverse limit of the sequence $(Y_n)_{n\in\Nat}$ with their canonical connecting maps.

To show that $\beta$ satisfies \eqref{eq:universality}, pick an arbitrary projectively isolated subshift $Z\in\ZZ$. Since $\beta\in\GG_2$, by definition there are a clopen partition $\PP$ and a map $\phi_Z$ between two subshifts such that \[\phi_Z[\QQ(\beta,\PP)]=Z.\] Since the diameters of the elements of the partitions $(\PP_n)_{n\in\Nat}$ there exists $n\in\Nat$ such that $\PP_n\preceq \PP$. Thus, since $\phi_{\PP_n}^\PP\circ Q_{\PP_n}^\beta=Q_\PP^\beta$, we get that the map $\phi_n^Y:=\phi_Z\circ \phi_{\PP_n}^\PP$ is a map from $Y_n$ onto $Z$ verifying \eqref{eq:universality}.\medskip

Conversely, suppose that $\beta\in\Act_G(\CC)$ is a continuous action of $G$ on $\CC$ that is an inverse limit of isolated subshifts $(Y_m)_{m\in\Nat}$ and that satisfies \eqref{eq:universality}. Applying Lemmas~\ref{lem:stronglyisomorphic} and~\ref{lem:independencyonsequence} we may suppose that there is a refining sequence of clopen partitions $\PP_1\succeq\PP_2\succeq\PP_3\succeq\ldots$ such that for each $n\in\Nat$, $\QQ(\beta,\PP_n)$ is projectively isolated and $\beta$ satisfies \eqref{eq:universality} with respect to the inverse sequence $(\QQ(\beta,\PP_n))_{n\in\Nat}$ and their canonical connecting isolated factor maps. Then choosing any $Z\in\ZZ$, by definition there are $m\in\Nat$ and an isolated factor map $\phi_m^Z:\QQ(\beta,\PP_m)\rightarrow Z$ defined on an open neighborhood $\NN(Z)$ of $\QQ(\beta,\PP_M)$. Since obviously $\QQ(\beta,\PP_m)\in\NN(Z)$, $\PP_m$ is the desired clopen partition $\PP$ witnessing that $\beta\in\GG_2$, finishing the proof.
\end{proof}

Finally, we set \[\begin{split}\GG_3:=&\Big\{ \beta\in\Act_G(\CC)\colon \forall X,Z\in\ZZ\;\forall \psi_X^Z: X\rightarrow Z\text{ factor map}\;\;\forall \PP\in\mathbb{P}\\ & \Big(\QQ(\beta,\PP)=Z\Rightarrow \exists \PP'\in\mathbb{P}\; \big(\QQ(\beta,\PP')\in\NN(X)\wedge\psi_X^Z\circ\phi_X\circ Q^\beta_{\PP'}=Q^\beta_\PP\big)\Big)\Big\} \end{split}\]

\begin{fact}\label{fact:mainthm3}
The set $\GG_1\cap\GG_3$ is a $G_\delta$ set of those actions of $G$ on $\CC$ that are inverse limits of isolated subshifts and satisfy \eqref{eq:cofinality}.
\end{fact}
\begin{proof}[Proof of Fact~\ref{fact:mainthm3}]
To show that $\GG_1\cap\GG_3$ is $G_\delta$ it is enough to check that $\GG_3$ is $G_\delta$. In order to do that, for fixed isolated subshifts $X,Z\in\ZZ$ such that $X$ factors onto $Z$ via some $\psi_X^Z$ (which we may suppose is defined on some open $\NN(Z)$ and for all $X'\in\NN(Z)$ we have $\phi_X^Z[X']=Z$), and for a fixed partition $\PP$ of $\CC$ we check that the set \[\begin{split}\AAA:=\big\{\beta\in\Act_G(\CC)\colon & \QQ(\beta,\PP)=Z\Rightarrow \exists \PP'\in\mathbb{P}\\ & \big(\QQ(\beta,\PP')\in\NN(X)\wedge\psi_X^Z\circ\phi_X\circ Q^\beta_{\PP'}=Q^\beta_\PP\big)\big\}\end{split}\] is $G_\delta$. However, this is a set of those $\beta\in \Act_G(\CC)$ such that either $\QQ(\beta,\PP)\neq Z$, which is open since $\SH_G(\PP)\setminus\{Z\}$ is open and $\QQ(\cdot,\PP)$ is continuous, or for which there is a partition $\PP'$ such that $\QQ(\beta,\PP')\in\NN(X)$ and $\psi_X^Z\circ\phi_X\circ Q^\beta_{\PP'}=Q^\beta_\PP$, so it suffices to check that for a fixed $\PP'$ the set of those $\beta\in\Act_G(\CC)$ such that $\QQ(\beta,\PP')\in\NN(X)$ and $\psi_X^Z\circ\phi_X\circ Q^\beta_{\PP'}=Q^\beta_\PP$ is open. The condition that $\QQ(\beta,\PP')\in\NN(X)$ is open as $\QQ(\cdot,\PP')$ is continuous. Furthermore, since $\psi_X^Z\circ\phi_X\circ Q^\beta_{\PP'}$ is by Lemma~\ref{lem:factoringonshift} given by some clopen partition $\PP''$, we can easily check that $\psi_X^Z\circ\phi_X\circ Q^\beta_{\PP'}=Q^\beta_\PP$ if and only if the partitions $\PP$ and $\PP''$ are equal, so the condition $\psi_X^Z\circ\phi_X\circ Q^\beta_{\PP'}=Q^\beta_\PP$ is open since for a fixed clopen partition $\RR$ the set $\{\beta\in\Act_G(\CC)\colon \psi_X^Z\circ\phi_X\circ Q^\beta_{\PP'}\text{ is induced by }\RR\}$ is open.

Thus we get that $\AAA$ is open.
\medskip

Suppose now that $\beta\in\GG_1\cap \GG_3$. By Fact~\ref{fact:mainthm1}, $\beta$ is an inverse limit of some sequence of isolated subshifts which we may again suppose to be given by some refining sequence of clopen partitions $\PP_1\succeq\PP_2\succeq\PP_3\succeq\ldots$. For each $n\in\Nat$, denote $\QQ(\beta,\PP_n)$ by $Y_n$. We verify that $\beta$ satisfies \eqref{eq:cofinality}. So suppose that $X$ is an isolated subshift and there is a factor map $\psi_X^m: X\rightarrow Y_m$, for some $m\in\Nat$. Since $\beta\in\GG_3$ and for the partition $\PP_m$ we have $\QQ(\beta,\PP_m)=Y_m$ we can find a partition $\PP'$ such that $\QQ(\beta,\PP')\in\NN(X)$ and $\psi_X^m\circ\phi_X\circ Q_{\PP'}^\beta=Q_{\PP_m}^\beta$. We can further find some $n\in\Nat$ such that $\PP_n\preceq\PP'$, so the map $\phi_X\circ\phi_{\PP_n}^{\PP'}:Y_n\rightarrow X$ is the desired map showing that $\beta$ satisfies \eqref{eq:cofinality}.\medskip

Conversely, suppose that $\beta\in\Act_G(\CC)$ is an action of $G$ on $\CC$ that is an inverse limit of some sequence of isolated subshifts and which satisfies \eqref{eq:cofinality}. By Fact~\ref{fact:mainthm1}, $\beta\in\GG_1$, so we verify that $\beta\in\GG_3$. Using Lemmas~\ref{lem:stronglyisomorphic} and~\ref{lem:independencyonsequence}, we can suppose that the inverse sequence of projectively isolated subshifts with which $\beta$ satisfies \eqref{eq:cofinality} is given by some refining sequence of clopen partitions $\PP_1\succeq\PP_2\succeq\PP_3\succeq\ldots$. For each $n\in\Nat$, denote $\QQ(\beta,\PP_n)$ by $Y_n$.

Let $X,Z\in\ZZ$, $\psi_X^Z:X\rightarrow Z$ and $\PP\in\mathbb{P}$ be as in the definition of $\GG_3$ and such that $\QQ(\beta,\PP)=Z$. We can find $m\in\Nat$ such that $\PP_m\preceq\PP$, so we get a factor map $\phi_{\PP_m}^\PP:Y_m\rightarrow Z$. Applying Claim 2 of Proposition~\ref{prop:MAINexistence} we can find a projectively isolated subshift $W$ and factor maps $\psi_W^X:W\rightarrow X$ and $\psi_W^{Y_m}:W\rightarrow Y_m$ such that $\psi_X^Z\circ\psi_W^X=\phi_{\PP_m}^\PP\circ\psi_W^{Y_m}$. We may suppose that $\psi_W^X$ is defined on some open neighborhood $\NN(X)$ such that for all $W'\in\NN(X)$ we have $\psi_W^X[W']=X$. Now we apply the condition \eqref{eq:cofinality} with respect to $W$, $Y_m$ and the map $\psi_W^{Y_m}$ to get $n\in\Nat$ and a map $\psi_{Y_n}^W: Y_n\rightarrow W$ such that $\psi_W^{Y_m}\circ\psi_{Y_n}^W=\phi_{\PP_n}^{\PP_m}$ (notice that $\phi_{\PP_n}^{\PP_m}$ is the connecting factor map from $Y_n$ onto $Y_m$). The factor map $\psi_{Y_n}^W\circ Q_{\PP_n}^\beta$ is induced by some partition $\PP'$, which we claim is as desired. We clearly have $\QQ(\beta,\PP')\in\NN(X)$. In the following we shall denote $\psi_W^X$ by $\phi_X$ to be consistent with the notation of the definition of $\GG_3$. It remains to verify that $Q^\beta_\PP=\psi_X^Z\circ\phi_X\circ Q^\beta_{\PP'}$. By a simple diagram chasing we get \[\begin{split}Q^\beta_\PP &=\phi_{\PP_m}^\PP\circ\phi_{\PP_n}^{\PP_m}\circ Q_{\PP_n}^\beta=\phi_{\PP_m}^\PP\circ \psi_W^{Y_m}\circ \psi_{Y_n}^W\circ Q_{\PP_n}^\beta\\ & =  \psi_X^Z\circ\psi_W^X\circ\psi_{Y_n}^W\circ Q_{\PP_n}^\beta = \psi_X^Z\circ\phi_X\circ Q_{\PP'}^\beta,\end{split}\] which finishes the proof of the fact and also of Proposition~\ref{prop:MAINexistence}.
\end{proof}

Theorem~\ref{thm:mainRokhlin} jointly with the recent paper \cite{PavSchmie} give yet another proof of the existence of a generic action of $\Int$ on the Cantor space. Indeed, in \cite[Theorem 3.6]{PavSchmie} they prove that isolated subshifts are dense in $\SH_\Int(n)$, for $n\geq 2$. Notice also that by Theorem~\ref{thm:mainRokhlin}, every finite group has the strong topological Rokhlin property since actually every subshift over a finite group is isolated.

\section{Free products with the strong topological Rokhlin property}\label{sect:freeproducts}
With this section we start with our applications of Theorem~\ref{thm:mainRokhlin}. Our goal is to prove Theorem~\ref{thm:intro2}. For this we need to introduce some new notions.
\begin{definition}\label{def:automaton}
	Let $A$ be a finite non-trivial set (thought of as a set of colors in this context) and let $G=\bigstar_{i\leq n} G_i$, where each for each $i\leq n$, $G_i$ is either finite or infinite cyclic. Let $S\subseteq G$ be a finite symmetric set consisting of the generator $g_i\in G_i$ and its inverse $g_i^{-1}$ if $G_i$ is infinite cyclic, and of $G_j\setminus\{1\}$ if $G_j$ is finite. A \emph{coloring automaton} is a map $\Omega: S\times A\rightarrow A$.
\end{definition}

\begin{construction}\label{con:automaton}
A coloring automaton $\Omega$ over $G$ as above and $A$ produces an element $x\in A^G$ which is uniquely defined by chosing the starting element $g\in G$ and the starting color $a\in A$, and then coloring the rest of the elements by the fixed rule given by the map $\Omega$. Moreover, the automaton keeps track on how it moves along the Cayley graph of $G$ with respect to $S$.

More formally, we proceed as follows. We define \[B:=A\times \{\leftarrow,\rightarrow,\emptyset\}^S.\] For any $b\in B$, we denote by $b_1$, resp. $b_2$ its projection to the first coordinate, which is an element of $A$, resp. its projection to the second coordinate, which is an element of $\{\leftarrow,\rightarrow,\emptyset\}^S$. Upon choosing the initial group element and color, we define an element $x\in A^G$ and an element $\tilde{x}\in B^G$. \bigskip

\noindent{\bf Step 1.} Pick any $g\in G$ and $a\in A$. We set $x(g)=a$. For every $s\in S$ we set \[x(gs):=\Omega(s,a),\] and in order to keep track of how the automaton moves, we set $\tilde{x}(g)_1=a$ and for each $s\in S$ we set $\tilde{x}(g)_2(s)=\rightarrow$. Then for each $s\in S$ we set \[\tilde{x}(gs)_1:=x(gs)\text{ and }\tilde{x}(gs)_2(s^{-1})=\leftarrow.\] Additionally, for every other $s'\neq s^{-1}\in S$ we set \[\tilde{x}(gs)_2(s'):=\begin{cases}
	\emptyset & \text{if }s,s'\text{ belong to the same finite group }G_j;\\
	\rightarrow & \text{otherwise.}
\end{cases}
\]\medskip

\noindent{\bf General Step 2.} Assume that the automaton has defined $x(h)$ and $\tilde{x}(h)$, for some $h\in G$, arriving to $h$ from some $h'$ in the direction of some $s'\in S$, i.e. $h=h's'$. Then we have two cases.
\begin{enumerate}
	\item The element $s'$ does not belong to any finite group $G_j$, i.e. it is the generator or its inverse of some infinite cyclic group $G_i$. Then for every $s\neq s'\in S$ we set \[x(hs):=\Omega(s,x(h)),\] \[\tilde{x}(hs)_1:=x(hs)\text{ and }\tilde{x}(hs)_2(s^{-1})=\leftarrow.\] Moreover, for every $s''\neq s^{-1}\in S$ we set \[\tilde{x}(hs)_2(s''):=\begin{cases}
		\emptyset & \text{if }s,s''\text{ belong to the same finite group }G_j;\\
		\rightarrow & \text{otherwise.}
	\end{cases}
	\]
	\item The element $s'$ belongs to some finite group $G_j$. Then for every $s\in G_j$, $x(hs)$ and $\tilde{x}(hs)$ have already been defined since $hs=h's''$ for $s''=s's\in G_j$. For $s\notin G_j$ we set as before \[x(hs):=\Omega(s,x(h)),\] \[\tilde{x}(hs)_1:=x(hs)\text{ and }\tilde{x}(hs)_2(s^{-1})=\leftarrow.\] Moreover, for every $s''\neq s\in S$ we set \[\tilde{x}(hs)_2(s''):=\begin{cases}
		\emptyset & \text{if }s,s''\text{ belong to the same finite group }G_j;\\
		\rightarrow & \text{otherwise.}
	\end{cases}
	\]
\end{enumerate}
\end{construction}
Let us denote by $X_\Omega$, resp. $\tilde{X}_\Omega$ the closure of the subset of $A^G$ of all those elements of $A^G$, resp. closure of the subset of $B^G$ of all those elements of $B^G$, produced by the coloring automaton by the procedure above. The coordinate projection $p:B\rightarrow A$ induces a factor map $P:B^G\rightarrow A^G$.

\begin{theorem}\label{thm:automataproduceprojisolatedshfts}
	Let $A$ be a non-trivial finite set, and $G$ and $\Omega$ be a group and an automaton as in Definition~\ref{def:automaton}. Then $\tilde{X}_\Omega$ is an isolated subshift and $X_\Omega$ is a strongly projectively isolated subshift which is witnessed by the isolated factor map $P$, i.e. $P[\tilde{X}_\Omega]=X_\Omega$.
\end{theorem}

\begin{proof}
Fix the notation as in the statement. We also follow the notation from Construction~\ref{con:automaton}; in particular, we set $B:=A\times \{\leftarrow,\rightarrow,\emptyset\}^S$. The map $P$ is therefore induced by the projection map $p:B\rightarrow A$. It is clear that $P[\tilde{X}_\Omega]=X_\Omega$, so we only need to prove that $\tilde{X}_\Omega$ is isolated.

First we notice that $\tilde{X}_\Omega$ is of finite type. We define a subshift of finite type $X\subseteq B^G$ and show that $X=\tilde{X}_\Omega$. We set the defining window to be $F:=S\cup\{1\}$. A pattern $p\in B^F$ is allowed if it is produced by the automaton $\Omega$, i.e. if and only if $p\in (\tilde{X}_\Omega)_F$. Clearly, $\tilde{X}_\Omega\subseteq X$. We show the converse. Pick $x\in X$. We distinguish two cases.\smallskip

\noindent{\bf Case 1.} We have that there is $g\in G$ such that $x(g)_2(s)=\rightarrow$ for all $s\in S$. Then it is straightforward to check that the rules imposed by the allowed patterns of $X$ force $x$ to correspond to an element produced by $\Omega$ when the starting vertex is $g\in G$ and the starting color is $x(g)_1$. It follows by definition that $x\in \tilde{X}_\Omega$.\smallskip

\noindent{\bf Case 2.} For every $g\in G$ there is one, and consequently by the rules imposed by allowed patterns \emph{exactly one}, $s\in S$ such that $x(g)_2(s)=\leftarrow$. Pick an arbitrary $g\in G$ and set $g_1:=g$. By the assumption there is unique $s\in S$ such that $x(g)_2(s)=\leftarrow$. Set $g_2:=gs$. Assuming that $g_{n-1}$, for $n\in\Nat$ has been defined, we set $g_n:=g_{n-1}s$, where $s$ is the unique element of $S$ such that $x(g_{n-1})_2(s)=\leftarrow$. This defines an infinite path $(g_n)_{n\in\Nat}$. This path is unique up to finite initial segment. Indeed, choosing a different starting element $h\in G$ there is a unique path $h_1=h,\ldots,h_n=g$ such that $h_i^{-1}h_i\in S$, for $1<i\leq n$, and such that either for each $i<n$ we have $x(h_i)_2(h_{i+1}^{-1}h_i)=\leftarrow$, or for each $i<n$ we have $x(h_i)_2(h_{i+1}^{-1}h_i)=\rightarrow$. Notice that such a path between $h$ and $g$ is unique although there may be more than one path $h'_1=h,\ldots,h'_m=g$ such that $(h'_i)^{-1}h'_i\in S$, for $1<i\leq m$.

Now for every $n\in\Nat$ we define an element $x_n\in \tilde{X}_\Omega$, which is an element of $B^G$ produced by $\Omega$ with the starting element $g_n$ and the starting color $x(g_n)_1$. It is again straightforward to check that $\lim_{n\to\infty} x_n=x$, so we get that $x\in\tilde{X}_\Omega$ as desired.\medskip

It remains to show that $\tilde{X}_\Omega=X$ is isolated. We claim that $\NN_X^F=\{X\}$. Indeed, this follows from the facts that
\begin{itemize}
	\item for every $X'\in\NN_X^F$ we have $X'\subseteq X$;
	\item every $X'\subseteq X$ that contains the pattern $p_a\in B^F$, which is defined as $x_a\upharpoonright F$, where $x_a\in\tilde{X}_\Omega$ is the element of $B^G$ produced by $\Omega$ with the starting element $1_G$ and the starting color $a$, must be equal to $X$. Indeed, $X=\tilde{X}_\Omega$ is by definition the closure of the set of configurations produced by $\Omega$, which is equal to the smallest subshift containing configurations produced by $\Omega$ with the starting element $1_G$ and all possible colors $a\in A$.
\end{itemize}
\end{proof}

\begin{lemma}\label{lem:nbhdisomorphism}
	Let $A$ and $B$ be non-trivial finite sets, $G$ be a countable group, and $X\subseteq A^G$ and $Y\subseteq B^G$ be two subshifts of finite type such that there is an isomorphism $\phi:X\rightarrow Y$. Then $\phi$ canonically induces a bijection $\Phi$ between an open neighborhood $\NN_X$ of $X$ and $\NN_Y$ so that for every $Z\in\NN_X$, $Z$ is isomorphic to $\Phi(Z)$.
\end{lemma}
\begin{proof}
	Let us fix $A$, $B$, $G$, $X$, $Y$, and $\phi:X\rightarrow Y$ as in the statement. We may suppose that there is a finite set $F\subseteq G$ such that $F$ is the defining window for $X$ and $\phi$ is determined by a finite map $f:X_F\rightarrow B$. Set $\NN_X:=\NN_X^F$ and for $Z\in\NN_X$ set \[\Phi(Z):=\phi[Z].\] Since each $Z\in\NN_X$ is a subshift of $X$ and $\phi:X\rightarrow Y$, clearly $\Phi(Z)$ is isomorphic to $Z$ for each such $Z$. Thus it remains to show that $\NN_Y:=\Phi[\NN_X]$ is an open neighborhood of $Y$. Clearly, it contains $Y$, so we show that $\NN_Y$ is open. Recall that $X_F$ induces a clopen partition \[\PP:=\{C_p(X)\colon p\in X_F\}\] of $X$. Set $\PP':=\phi(\PP)=\{\phi[C_p(X)]\colon p\in X_F\}$, which is a clopen partition of $Y$. It follows that $\NN_Y$ consists of those subshifts of $Y$ that have non-empty intersection with each element of $\PP'$. In order to show that this defines an open set we pick $Z\in\NN_Y$ and show that there is a finite set $E\subseteq G$ such that $\NN_Z^E\subseteq \NN_Y$. Since each clopen set in $Y$ is a finite union of basic clopen sets, there exists a finite set $E\subseteq G$ such that the partition \[\PP'':=\{C_p(Y)\colon p\in Y_E\},\] is a refinement of $\PP'$.
	
	We claim that $E$ is the desired set so that $\NN_Z^E\subseteq \NN_Y$. Indeed, pick $V\in\NN_Z^E$, so by definition $Z_E=V_E$. Since $Z\in\NN_Y$, for every $P\in\PP'$ there is at least one $p\in Y_E$ such that $p\in Z_E$ and $C_p(Y)\subseteq P$. Since $V_E=Z_E$, $V\cap C_p(Y)\neq \emptyset$, so $V\cap P\neq\emptyset$. Since $P\in\PP'$ was arbitrary, we are done.
\end{proof}
\begin{lemma}\label{lem:isoofSFTnbhds}
Let $G$ be a countable group, $F\subseteq G$ be a finite subset $A$ be a non-trivial finite set, and $X\subseteq A^G$ a subshift of finite type whose defining window is contained in $F$. Then there are a non-trivial finite set $B$, a subshift of finite type $Y\subseteq B^G$, and an isomorphism $\phi: X\rightarrow Y$ that induces a bijection $\Phi$ between the open neighborhoods $\NN_X^F$ and $\NN_Y^{\{1_G\}}$ so that for every $X'\in\NN_X^F$, $X'$ and $\Phi(X')$ are isomorphic.
\end{lemma}
\begin{proof}
Fix $G$, $F$, $A$, and $X$ as in the statement. We may suppose that $1_G\in F$. Set $B:=X_F$. Then by Lemma~\ref{lem:SFTfactorofSFT}, the identity map $f:X_F\rightarrow B$ induces a map $\phi:X\rightarrow B^G$ which is an isomorphism between $X$ and its image $\phi[X]\subseteq B^G$ which we denote by $Y$. By the proof of the preceding Lemma~\ref{lem:SFTnbhrds}, $\phi$ induces a bijection $\Phi$ between the neighborhood $\NN_X^F$ and the neighborhood $\NN_Y$ of $Y$ of the form $\{\phi[X']\colon X'\in\NN_X^F\}$, which is easily checked to be equal to $\NN_Y^{\{1_G\}}$.
\end{proof}
\begin{definition}
	Let $G$ and $H$ be countable groups, $A$ be a non-trivial finite set, and $X\subseteq A^G$, $Y\subseteq A^H$ subshifts. The \emph{free product} $X\ast Y$ of $X$ and $Y$ is a subshift of $A^{G\ast H}$ defined as \[\{x\in A^{G\ast H}\colon \forall g\in G\ast H\; (gx\upharpoonright G\in X\wedge gx\upharpoonright H\in Y)\}.\] Equivalently, if $P_X$, resp. $P_Y$ are the set of forbidden patterns of $X$, resp. of $Y$, then $X\ast Y$ is defined as a subshift of $A^{G\ast H}$ whose set of forbidden patterns is $P_X\cup P_Y$.
	
	In particular, if $X$ and $Y$ are subshifts of finite type, then so is $X\ast Y$.
\end{definition}


\begin{lemma}\label{lem:freeprodofshifts}
	Let $G$ and $H$ be countable groups, $A$ be a non-trivial finite set, and $X\subseteq A^{G\ast H}$ be a subshift of finite type. Then $X$ is isomorphic to a free product $V\ast W$, where $V\subseteq A^G$ and $W\subseteq A^H$ are subshifts of finite type.
\end{lemma}
\begin{proof}
	Fix $G$ $H$, $A$, and $X$ as in the statement. By Proposition~\ref{prop:SFTprop} we may assume that there are finite sets $S_G\subseteq G$ and $S_H\subseteq H$ so that for $x\in A^{G\ast H}$ we have $x\in X$ if and only if for every $g\in G\ast H$ and for each $s\in S_G\cup S_H$ \[gx\upharpoonright \{1,s\}\text{ is an allowed pattern}.\] The set of allowed patterns defined on the subsets $\{1,s\}$, where $s\in S_G$, resp. $s\in S_H$, will be denoted by $P_G$, resp. by $P_H$.
	
	We define $V\subseteq A^G$ to be the subshift of finite type, where for $x\in A^G$ we have $x\in V$ if and only if for every $g\in G$ and $s\in S_G$ \[gx\upharpoonright \{1,s\}\in P_G.\] We define a subshift $W\subseteq A^H$ of finite type analogously, with $S_H$ and $P_H$ instead of $S_G$ and $P_G$.
	
	We leave to the reader the straightforward verification that $X=V\ast W$.
\end{proof}

\begin{definition}
Let $G$ and $H$ be countable groups and let $A,B,C$ be non-trivial finite sets. Let $\phi_0:B\rightarrow A$ and $\psi_0:C\rightarrow A$ be maps and denote by $B\times_{\phi,\psi} C$ the restricted direct product $\{(b,c)\in B\times C\colon \phi_0(b)=\psi_0(c)\}$. Denote also by $p_B$, resp. $p_C$ the projection from $B\times C$ on the first, resp. the second coordinate, and by $P_B$, resp. $P_C$ the corresponding induced maps from $(B\times C)^G$ onto $B^G$, resp. from $(B\times C)^H$ onto $C^H$. Suppose that $X\subseteq B^G$ and $Y\subseteq C^H$ are subshifts. Then the \emph{restricted free product} $X\ast_{\phi,\psi} Y$ of $X$ and $Y$ with respect to the maps $\phi_0$ and $\psi_0$ is a subshift of $(B\times_{\phi,\psi}C)^{G\ast H}$ defined as \[\{x\in (B\times_{\phi,\psi}C)^{G\ast H}\colon \forall g\in G\ast H\;\big(P_B(gx\upharpoonright G)\in X\wedge P_C(gx\upharpoonright H)\in Y\big)\}.\]
\end{definition}

\begin{lemma}\label{lem:isooffreeprodofSFT}
Let $G$ and $H$ be countable groups and let $A,B,C$ be non-trivial finite sets. Let $\phi_0:B\rightarrow A$ and $\psi_0:C\rightarrow A$ be maps which induce maps $\phi:B^G\rightarrow A^G$, resp. $\psi:C^H\rightarrow A^H$. Suppose that $X\subseteq A^G$ and $X'\subseteq B^G$, resp. $Y\subseteq A^H$ and $Y'\subseteq C^H$ are subshifts such that $\phi$, resp. $\psi$ induces an isomorphism between $X'$ and $X$, resp. between $Y'$ and $Y$. Then the map $(\phi_0,\psi_0):B\times_{\phi,\psi}C\rightarrow A$ induces an isomorphism between $X'\ast_{\phi,\psi} Y'$ and $X\ast Y$.
\end{lemma}
\begin{proof}
Fix the notation as in the statement. The canonical map $(\phi_0,\psi_0):B\times_{\phi,\psi}C\rightarrow A$ induces a map $\eta:(B\times_{\phi,\psi}C)^{G\ast H}\rightarrow A^{G\ast H}$. We need to show that $\eta[X'\ast_{\phi,\psi} Y']=X\ast Y$ and $\eta\upharpoonright X'\ast_{\phi,\psi} Y'$ is one-to-one.

First we check that $\eta[X'\ast_{\phi,\psi} Y']\subseteq X\ast Y$. Pick $x\in X'\ast_{\phi,\psi} Y'$ and let us show that $\eta(x)\in X\ast Y$. We need to verify that for every $g\in G\ast H$ we have $\eta(gx)\upharpoonright G\in X$ and $\eta(gx)\upharpoonright H\in Y$. Fix $g\in G\ast H$. By definition, we have $P_B(gx\upharpoonright G)\in X'$ and $P_C(gx\upharpoonright H)\in Y'$. Notice however that we have \[\eta(gx)\upharpoonright G=\phi\big(P_B(gx\upharpoonright G)\big)\in\phi[X']=X.\] The case of $\eta(gx)\upharpoonright H$ is analogous.

Next we show injectivity and surjectivity of the map $\eta:X'\ast_{\phi,\psi} Y'\rightarrow X\ast Y$. Let us start with the injectivity. Pick $x\neq y\in X'\ast_{\phi,\psi} Y'$. Without loss of generality we can assume that $x(1)\neq y(1)$. It follows that either $P_B(x\upharpoonright G)\neq P_B(y\upharpoonright G)$ or $P_C(x\upharpoonright H)\neq P_C(x\upharpoonright H)$. Assume the former, the latter is treated analogously. Then \[\eta(x)\upharpoonright G=\phi\big(P_B(x\upharpoonright G)\big)\neq \phi\big( P_B(y\upharpoonright G)\big)=\eta(y)\upharpoonright G,\] since $\phi$ is injective. However, the inequality $\eta(x)\upharpoonright G\neq\eta(y)\upharpoonright G$ implies the inequality $\eta(x)
\neq \eta(y)$.

Finally, we show the surjectivity. First we notice that since $\phi$ and $\psi$ are isomorphisms, the maps $\phi^{-1}:X\rightarrow X'$, resp. $\psi^{-1}: Y\rightarrow Y'$ are also isomorphisms induced, by the Curtis-Hedlund-Lyndon theorem, by some finite maps $f_X: X_F\rightarrow B$, resp. $f_Y: Y_E\rightarrow C$, where $F\subseteq G$ and $E\subseteq H$ are finite sets. Let $y\in X\ast Y$. Let us define $x\in X'\ast_{\phi,\psi} Y'$ as follows. For $g\in G\ast H$ we set $x(g):=(b,c)$, where \[b:=f_X(gy\upharpoonright F),\; c:=f_Y(gy\upharpoonright E).\] Since $\phi_0\big(f_X(gy\upharpoonright F)\big)=y(g)=\psi_0\big(f_Y(gy\upharpoonright E)\big)$, we have that $(b,c)\in B\times_{\phi,\psi} C$. In order to check that $x\in X'\ast_{\phi,\psi} Y'$ we need to verify that for every $g\in G\ast H$ we have $P_B(gx\upharpoonright G)\in X'$ and $P_C(gx\upharpoonright H)\in Y'$. We verify the former, the latter is analogous, and without loss of generality, we assume that $g=1$. However, then \[P_B(x\upharpoonright G)=\phi^{-1}(y\upharpoonright G)\in \phi^{-1}[X]=X'.\]
It remains to check that $\eta(x)=y$, which is straightforward and left to the reader.
\end{proof}
\begin{definition}
Let $G$ be a countable group, $A$ and $B$ be non-trivial finite sets, and let $X\subseteq A^G$ and $Y\subseteq B^G$ be subshifts. Say that a continuous $G$-equivariant map $\phi:Y\rightarrow X$ is \emph{basic} if it is induced by a finite map $\phi_0:B\rightarrow A$.
\end{definition}
\begin{proposition}\label{prop:freeprodsautomata}
	Let $G$ and $H$ be countable groups of the form allowed in Definition~\ref{def:automaton}. Let $A$ be a non-trivial finite set and $X\subseteq A^G$, resp. $Y\subseteq A^H$ subshifts of finite type. Suppose that for every open neighborhood $\NN_X$ of $X$ in $\SH_G(A)$, resp. $\NN_Y$ of $Y$ in $\SH_G(A)$ there is a subshift $X'\in\NN_X$, resp. $Y'\in\NN_Y$ isomorphic via a basic isomorphism to a subshift produced by a coloring automaton. Then the same holds true for every open neighborhood of the free product $X\ast Y$.
\end{proposition}
\begin{proof}
Fix $G$, $H$, $A$, and $X$ and $Y$ as in the statement. Let $\NN$ be an open neighborhood of $X\ast Y$. Applying Lemma~\ref{lem:isoofSFTnbhds}, without loss of generality, we may assume that $\NN=\NN_{X\ast Y}^{\{1\}}$, i.e. for $Z\subseteq A^{G\ast H}$ we have $Z\in\NN$ if and only if $Z$ is \emph{fully colored}, i.e. for every $a\in A$ there is $z\in Z$ with $z(1)=a$. By the assumption there are $X'\subseteq X$ and $Y'\subseteq Y$ that are fully colored, i.e. for every $a\in A$ there are $x\in X'$, resp. $y\in Y'$ satisfying $x(1)=a=y(1)$, and moreover $X'$ and $Y'$ are isomorphic to subshifts $X''\subseteq B^G$, resp. $Y''\subseteq C^H$, for some non-trivial finite sets $B$ and $C$ that are produced by some coloring automata $\Omega_X$ and $\Omega_Y$ defined on finite symmetric generating sets $S_X\subseteq G$ and $S_Y\subseteq H$. Moreover, the isomorphisms $\phi:X''\rightarrow X'$, resp. $\psi:Y''\rightarrow Y'$ are basic, i,e. induced by finite (surjective) maps $\phi_0:B\rightarrow A$, resp. $\psi_0: C\rightarrow A$.

We define a new coloring automaton on $G\ast H$ with the set of colors $B\times_{\phi,\psi} C$ and with respect to the finite symmetric generating set $S:=S_X\cup S_Y$. For $(b,c)\in B\times_{\phi,\psi} C$ and $s\in S$ we set \[\Omega(s,(b,c)):=\begin{cases}
	(\Omega_X(s,b),c') & \text{if }s\in S_X\text{ so that }(\Omega_X(s,b),c')\in B\times_{\phi,\psi} C;\\
	(b',\Omega_Y(s,c)) & \text{if }s\in S_Y\text{ so that }	(b',\Omega_Y(s,c))\in B\times_{\phi,\psi} C.
\end{cases}\] In other words, if $s$ belongs to, say, $S_X$, $\Omega(s,(b,c))$ is $(b',c')$, where $b'$ is produced by the automaton $\Omega_X$ and $c'$ is arbitrary so that $(b',c')\in B\times_{\phi,\psi} C$.

Denote by $Z\subseteq (B\times_{\phi,\psi} C)^{G\ast H}$ the subshift produced by the coloring automaton $\Omega$. Clearly we have $Z\subseteq X''\ast_{\phi,\psi} Y''$. By Lemma~\ref{lem:isooffreeprodofSFT}, we have an isomorphism $\eta: X''\ast_{\phi,\psi} Y''\rightarrow X'\ast Y'$. Therefore $Z':=\eta[Z]\subseteq X'\ast Y'\subseteq X\ast Y$. Since $Z$ is fully colored, i.e. for every $(b,c)\in B\times_{\phi,\psi} C$ there is $z\in Z$ such that $z(1)=(b,c)$, also $Z'$ is fully colored. Therefore $Z\in\NN$.
\end{proof}

\begin{corollary}\label{cor:stabilityunderfreeprod-automata}
	Let $G$ and $H$ be countable groups such that for each $F\in\{G,H\}$ and every non-trivial finite set $A$ the set of those subshifts that are isomorphic to subshifts produced by coloring automata is dense in $\SH_F(A)$. Then the same is true for $F=G\ast H$ and any non-trivial finite set $A$.
	
	In particular, the set of strongly projectively isolated subshifts is dense in $\SH_F(A)$, for every non-trivial finite set $A$, and $F$ has the strong topological Rokhlin property.
\end{corollary}
\begin{proof}
Fix the groups $G$ and $H$ and a non-trivial finite set $A$. Let $\NN\subseteq \SH_{G\ast H}(A)$ be an open neighborhood. By Lemma~\ref{lem:SFTnbhrds}, we may suppose that $\NN$ is an open neighborhood of some subshift of finite type $X$, and applying further Lemmas~\ref{lem:freeprodofshifts} and ~\ref{lem:nbhdisomorphism} we may assume that $X$ is a free product of subshifts of finite type defined over $G$ and $H$ respectively. Then we just apply Proposition~\ref{prop:freeprodsautomata}. The `in particular' part follows then from Theorems~\ref{thm:automataproduceprojisolatedshfts} and~\ref{thm:mainRokhlin}.
\end{proof}

We have already remarked that in \cite[Theorem 3.6]{PavSchmie} they prove that isolated subshifts are dense in $\SH_\Int(n)$, for $n\geq 2$. In fact, they prove that subshifts that are isomorphic to subshifts of finite type given by Rouzy graphs with no middle cycles property are isolated and dense. Here a Rouzy graph $(V,E)$ (in our previous notation from Definition~\ref{def:Rouzy}, an $\{1\}$-Rouzy graph) has \emph{no middle cycles property} if it has no cycle that has both incoming and outgoing edge from outside of the cycle - we refer to \cite[Definition 3.1 and Lemma 3.3]{PavSchmie} for more details.
\begin{theorem}\label{thm:STRPfreeproducts}
	Let $(G_i)_{i\leq n}$ be a finite sequence of groups where each of them is either finite or cyclic. Then $\bigstar_{i\leq n} G_i$ has the strong topological Rokhlin property.
	
	In fact, the set of strongly projectively isolated subshifts is dense in the spaces of subshifts over $\bigstar_{i\leq n} G_i$.
\end{theorem}
\begin{proof}
In order to apply Corollary~\ref{cor:stabilityunderfreeprod-automata} it suffices to show that for any group $G$ that is either finite or infinite cyclic and every non-trivial finite set $A$ the set of those subshifts produced by coloring automata is dense in $\SH_G(A)$.\medskip

\noindent{\bf Case 1.} $G$ is finite. Fix a non-trivial finite set $A$ and a subshift $X\subseteq A^G$. Set $B:=X_G$ and define a subshift $Y\subseteq B^G$ isomorphic to $X$ via a map $\phi:Y\rightarrow X$ which induced by the map $\phi_0:B=X_G\rightarrow A$ defined by $p\in X_G\to p(1_G)$. Now we define an automaton $\Omega$ with respect to $S=G\setminus\{1_G\}$ and $B$ as the set of colors as follows. For $b\in B=X_G$ and $g\in G\setminus\{1_G\}$ we set \[\Omega(g,b):=g^{-1}b.\]
It is straighforward that the subshift produced by $\Omega$ is equal to $Y$.\medskip

\noindent{\bf Case 2.} $G$ is infinite cyclic.

Since $G$ is isomorphic to $\Int$, we shall use the additive notation for group operations. We fix a non-trivial finite set $A$ and a subshift of finite type $X\subseteq A^\Int$ and some nighborhood $\NN$ od $X$. By \cite[Theorem 3.6]{PavSchmie}, $\NN$ contains an NMC (no middle cycle) subshift $Y$, where $Y$ has NMC if there exists $n\in\Nat$ such that the Rouzy graph on $Y_{[0,n]}$ (i.e. the graph where $Y_{[0,n]}$ is the set of vertices and the set of edges is defined in the obvious way as in Proposition~\ref{prop:SFTprop}) has no middle cycles property. By increasing $n$ we may moreover assume that the graph, which we shall denote $(V,E)$, has no vertex that has both more than one incoming edges and more than one outgoing edges. We let $P:V\rightarrow\{0,1\}$ to be the characteristic function of the set of those vertices of $V$ that belong to a simple cycle.

We shall now define the automaton as follows, with respect to $S=\{-1,1\}$ and $V$ as the set of colors. Pick $v\in V$. We define \[\Omega(1,v):=\begin{cases}
	w & \text{if }P(v)=0\text{ and }w\text{ is arbitrary such that }(v,w)\in E;\\
	w & \text{if }P(v)=1,\; P(w)=1\text{ and }(v,w)\in E.
\end{cases}\]
In words, we set $\Omega(1,v)$ to be the unique successor of $v$ on a simple cycle provided that $v$ lies on a simple cycle, otherwise we set $\Omega(1,v)$ to be an arbitrary successor ov $v$ in the graph. We define $\Omega(-1,v)$ symmetrically.


It is clear that the subshift generated by $\Omega$ is a subshift $Y'$ of $Y_{[0,n]}^\Int$ isomorphic to $Y$ via a map $\phi:Y'\rightarrow Y$ induced by $\phi_0:Y_{[0,n]}\to A$ defined by  $p\in Y_{[0,n]}\to p(0)$.
\end{proof}

\section{Groups without the strong topological Rokhlin property}\label{sect:noSTRP}
In this section we initiate the production of negative examples, i.e. we find many new examples of groups without the strong topological Rokhlin property. We prove here Theorem~\ref{thm:intro3}. We start with one more characterization of the strong topological Rokhlin property that is useful especially for proving the negative results.

When $X$ is a subshift and $\AAA$ is a collection of closed subsets of $X$, not necessarily a partition, i.e. the sets need not be disjoint and cover $X$, we say that $X$ is \emph{$\AAA$-minimal} if there is no non-empty proper subshift $Y\subseteq X$ that intersects every element from $\AAA$.

\begin{theorem}\label{thm:2ndcharacterizationofSTRP}
A countable group $G$ has the strong topological Rokhlin property if and only if for every non-trivial finite set $A$, any sofic subshift $X\subseteq A^G$, which is a factor of a subshift of finite type $Z\subseteq B^G$, for some non-trivial finite $B$, via a factor map $\phi: Z\rightarrow X$, and any open neighborhood $\NN$ of $X$ in $\SH_G(A)$, there are 
\begin{itemize}
	\item a sofic subshift $Y\subseteq Z$ satisfying $\phi[Y]\in\NN$,
	\item a subshift of finite type $V\subseteq C^G$, for some non-trivial finite set $C$, factoring on $Y$ via some $\psi: V\rightarrow Y$,
	\item and a clopen partition $\PP$ of $V$  such that $\phi[Y]$ is $\AAA$-minimal, where $\AAA=\phi\circ\psi(\PP)$.
\end{itemize}
  
\end{theorem}
\begin{proof}
Fix a countable group $G$.\medskip

Suppose first that there are a non-trivial finite set $A$, a sofic subshift $X\subseteq A^G$, which is a factor of a subshift of finite type $Z\subseteq B^G$, for some non-trivial finite $B$, via a factor map $\phi: Z\rightarrow X$, and an open neighborhood $\NN$ of $X$ in $\SH_G(A)$ such that no sofic subshift $Y\subseteq Z$ satisfying $\phi[Y]\in\NN$ is $\AAA$-minimal for any $\AAA:=\phi\circ\psi[\PP]$, where $V\subseteq C^G$, for some non-trivial finite set $C$, factors onto $Y$ via $\psi:V\rightarrow Y$ and $\PP$ is a clopen partition of $V$.

The open neighborhood of $X$ is without loss of generality of the form $\NN_X^F$ for some finite set $F\subseteq G$. Each pattern $p\in X_F$ determines a basic clopen set $C_p\subseteq X$ so that $\RR:=\{C_p\colon p\in X_F\}$ is a clopen partition of $X$. The preimage of $\RR$ via $\phi$ is a clopen partition $\PP'$ of $Z$. We may refine $\PP'$ to a clopen partition $\PP''$ that consists of basic clopen sets $C_p(Z)$ indexed by patterns $p\in Z_E$, where $E\subseteq G$ is a finite set, and so that for every $Z'\in\NN_Z^E$ we have that $Z'\subseteq Z$ - the latter claim by Lemma~\ref{lem:SFTnbhrds}. We now claim that $\NN_Z^E$ contains no projectively isolated subshift. If we show it, applying Proposition~\ref{prop:noRokhlin}, we get that $G$ does not have the strong topological Rokhlin property.

Suppose on the contrary that $Z'\in\NN_Z^E$ is projectively isolated. In particular, it is a sofic subshift of $Z$. Let $V\subseteq C^G$ be a subshift, for some finite non-trivial set $C$, such that there are an open neighborhood $\NN$ of $V$ in $\SH_G(C)$ and a map $\psi$ defined on every subshift from $\NN$ so that for every $V'\in\NN$ we have $\psi[V']=Z'$. We may suppose that $V$ is of finite type and that $\NN=\NN_V^J$, for some finite set $J\subseteq G$, so that for every $V'\in\NN_V^J$ we have $V'\subseteq V$, again by applying Lemma~\ref{lem:SFTnbhrds}. Set \[Y:=\phi\circ\psi[V].\] Since $\psi[V]=Z'\in \NN_Z^E$ we get $\psi[V]\subseteq Z$, and thus $Y=\phi[Z']\subseteq\phi[Z]=X$. Second, since $\psi[V]\in\NN_Z^E$ we get that $\psi[V]$ intersects every clopen subset from $\PP'$, and thus $Y=\phi\circ\psi[V]$ intersects every clopen subset of $\RR$, so $Y\in\NN_X^F$. Set now \[\AAA:=\{\phi\circ\psi\big(C_p(V)\big)\colon p\in V_J\}.\] We have $Y=\bigcup \AAA$ and by the assumption, $Y$ is not $\AAA$-minimal. Therefore there is a proper subshift $Y'\subseteq Y$ that non-trivially intersects every set from $\AAA$. However then \[V':=(\phi\circ\psi)^{-1}(Y')\] is a proper subshift of $V$ that non-trivially intersects every set $C_p(V)$, for $p\in V_J$. In other words, $V'\in\NN_V^J$. But then by the assumption on the projective isolation of $Z'$ we get $\psi[V']=Z'$. Therefore \[Y'=\phi\circ\psi[V']=\phi[Z']=\phi\circ\psi[V]=Y,\] which is a contradiction as the inclusion $Y'\subseteq Y$ is strict.\bigskip

Now we prove the converse. Using Theorem~\ref{thm:mainRokhlin}, it suffices to show that projectively isolated subshifts are dense in $\SH_G(A)$, for any non-trivial finite set $A$. Fix such $A$ and an open set $\NN\subseteq\SH_G(A)$. We may suppose that $\NN$ is of the form $\NN_X^F$, where $X\subseteq A^G$ is a subshift of finite type, $F\subseteq G$ is a finite set, and moreover for every $X'\in\NN_X^F$ we have $X'\subseteq X$. Now we use the assumption. Notice that since $X$ is of finite type, we have $Z=X$ and $\phi=\mathrm{id}$ in the notation of the statement. By the assumption, there is a sofic subshift $Y\in\NN_X^F$ that is $\AAA$-minimal, where $\AAA$ is an image of a clopen partition $\PP$ of some subshift of finite type $V\subseteq C^G$, for some non-trivial finite set $C$, that factors onto $Y$. We claim that $Y$ is projectively isolated. Denote by $\psi$ the factor map from $V$ onto $Y$. We may refine the clopen partition $\PP$ to a clopen partition \[\PP':=\{C_p(V)\colon p\in V_E\},\] where $E\subseteq G$ is a finite set. We can assume that $E$ is big enough so that for every $V'\in\NN_V^E$ we have $V'\subseteq V$. The image of $\PP'$ is a family of closed sets $\AAA'$ that refines $\AAA$, i.e. for every $R'\in\AAA'$ there is $R\in\AAA$ such that $R'\subseteq R$. It is clear that since $Y$ is $\AAA$-minimal it is also $\AAA'$-minimal. In order to show that $Y$ is projectively isolated it is enough to show that for every $V'\in\NN_V^E$ we have $\phi[V']=Y$. Pick $V'\in\NN_V^E$. First of all notice that since $V'\subseteq V$, $\psi$ is indeed defined on $V'$. Second, since $V'\in\NN_V^E$, by definition $V'\cap C_p(V)\neq\emptyset$ for every $p\in V_E$. Thus $\phi[V']\cap\phi[C_p(V)]\neq\emptyset$, for every $p\in V_E$, so $V'$ intersects non-trivially every element of $\AAA'$. However, since $Y$ is $\AAA'$-minimal, it follows that $\phi[V']=Y$.
\end{proof}

\subsection{Finitely generated groups without the STRP}
We start with results that concern finitely generated groups. As we shall see it is much harder to disprove the strong topological Rokhlin property for finitely generated groups.
\begin{theorem}\label{thm:disjointunionpropersubshifts}
Let $G$ be a countable group. Suppose that for some non-trivial finite set $A$ there is a sofic subshit $X\subseteq A^G$, being a factor of a subshift of finite type $Z\subseteq B^G$, for some non-trivial finite set $B$, via a map $\phi: Z\rightarrow X$, such that for every effective subshift $Y\subseteq Z$, $\phi[Y]$ is a disjoint union of infinitely many proper subshifts. 

Suppose also that there is a short exact sequence of groups \[1\to N\to H\to G\to 1,\] where $H$ is finitely generated and recursively presented, and $N$ is finitely generated. Then $H$, and in particular also $G$, does not have the strong topological Rokhlin property.
\end{theorem}
\begin{proof}
Fix the groups $G$, $N$, and $H$, the non-trivial finite sets $A$ and $B$, and the subshifts $X\subseteq A^G$, $Z\subseteq B^G$, and the factor map $\phi: Z\rightarrow X$ - as in the statement. Notice first that since $H$ is finitely generated and recursively presented, and $N$ is finitely generated, it follows that $G$ is also finitely generated and recursively presented, so the use of effective subshifts over $G$ is in accordance with Definition~\ref{def:effectiveshubshift}.

 Let $V\subseteq A^H$ be the subshift \[\{v\in A^H\colon \forall g\in H\;\forall h\in N\; \big(v(g)=v(gh)\big)\},\] and $W\subseteq B^H$ be the subshift \[\{w\in B^H\colon \forall g\in H\;\forall h\in N\; \big(w(g)=w(gh)\big)\}.\]
Clearly, there are a 1-1 correspondence \[Y\subseteq A^G\to Y'\subseteq V\] between subshifts of $A^G$ and subshifts of $V$, and a 1-1 correspondence \[Y\subseteq B^H\to Y'\subseteq W.\] Let therefore $X'\subseteq V$ be the subshift of $V$, and so of $A^H$, that corresponds to $X$, and let $Z'\subseteq W$ be the subshift of $W$, and so of $B^H$, that corresponds to $Z$. Assuming without loss of generality, by applying Lemma~\ref{lem:factoringonshift}, that $\phi: Z\rightarrow X$ is induced by a finite map $f:B\rightarrow A$, the same map $f$ also induces a factor map $\phi': Z'\rightarrow X'$. We claim that $Z'$ is of finite type. To see this, let $S\subseteq N$ be a finite symmetric generating set of $N$. Let $\FF$ be the finite set of forbidden patterns for $Z$. We may suppose that they are all defined on a finite set $T\subseteq G$. For each $t\in T$ choose one $t'\in H$ such that $Q(t')=t$, where $Q:H\rightarrow G$ is the quotient map. For each pattern $p\in\PP$ define a patter $p':T'\rightarrow A$ by \[p'(t')=p(t)\;\;\text{ for }t'\in T',\] and denote by $\FF'$ the set of such patterns. Moreover, for every $s\in S$ and $a\neq b\in A$ define a pattern $p_{s,a,b}:\{1,s\}\to A$ by \[p_{s,a,b}(1)=a\text{ and }p_{s,a,b}(s)=b,\] and denote by $\FF''$ the set of patterns $\{p_{s,a,b}\colon s\in S, a\neq b\in A\}$. It is now straightforward to verify and left to the reader that the set $\FF'\cup\FF''$ of forbidden patterns defines the subshift $Z'$. It also follows that $X'$ as a factor of $Z'$ is sofic. Now we show that for no sofic subshift $Y'\subseteq Z'$, $\phi'[Y']$ is $\phi'[\AAA]$-minimal for $\AAA$ being an image of clopen partition of some subshift of finite type factoring onto $Y'$. Once this claim is shown, we apply Theorem~\ref{thm:2ndcharacterizationofSTRP} to conclude that $H$ does not have the strong topological Rokhlin property.

Let us therefore prove the claim. Suppose on the contrary that $Y'\subseteq Z'$ is a sofic subshift which is a factor of a subshift of finite type $V\subseteq C^H$, for some non-trivial finite set $C$, via some $\psi: V\rightarrow Y'$, and that there is a clopen partition $\PP$ of $V$ such that for $\AAA:=\psi[\PP]$ we have that $\phi'[Y']$ is $\phi'[\AAA]$-minimal. As in the proof of Theorem~\ref{thm:2ndcharacterizationofSTRP}, we may suppose that $\PP$ is of the form $\{C_p(V)\colon p\in V_F\}$, for some finite set $F\subseteq H$, and moreover for every $V'\in\NN_V^F$ we have $V'\subseteq V$. Suppose that $Y$, the subshift of $Z$ corresponding to $Y'$, is effective. We show how to finish the proof then. For each $D'\in\phi'[\AAA]$, which is a closed subset of $\phi'[Y']$, denote by $D$ the corresponding closed subset of $\phi[Y]$.

Since by the assumption $\phi[Y]$ is a disjoint union of infinitely many proper subshifts, for each $D$ corresponding to $D'\in\phi'[\AAA]$ we can find a proper subshift $Y_D\subseteq \phi[Y]$ with $Y_D\cap D\neq\emptyset$, and moreover that for $D_1\neq D_2$ where $D'_1\neq D'_2\in\phi'[\AAA]$ either $Y_{D_1}=Y_{D_2}$ or $Y_{D_1}\cap Y_{D_2}=\emptyset$. Also by the assumption there is a proper subshift $Y_0\subseteq \phi[Y]$ such that $Y_0\cap \bigcup_{D'\in\phi'[\AAA]} Y_D=\emptyset$. It follows that \[W:=\bigcup_{D'\in\phi'[\AAA]} Y_D\] is a proper subshift of $\phi[Y]$ that intersects non-trivially every set $D$ corresponding to $D'\in\phi'[\AAA]$. Thus the corresponding $W'\subseteq \phi'[Y']$ is a proper subshift of $\phi[Y']$ that intersects non-trivially every set from $\phi'[\AAA]$, thus $\phi'[Y]$ is not $\phi'[\AAA]$-minimal, a contradiction.

So it remains to show that $Y$ is effective. Notice however that $Y'$ is sofic, thus effective, so we finish by the application of Lemma~\ref{lem:effectivesubshiifts}.
\end{proof}

\begin{corollary}\label{cor:nilpotentnoSTRP}
No infinite finitely generated nilpotent group that is not virtually cyclic has the strong topological Rokhlin property.
\end{corollary}
\begin{proof}
Let $H$ be a finitely generated nilpotent group that is not virtually cyclic. Then $H$ has $\Int^2$ as a quotient and both $H$ and $\Int^2$ are finitely presented. In particular, they are finitely generated and recursively presented, and moreover the kernel of the quotient map from $H$ onto $\Int^2$ is finitely generated. So in order to apply Theorem~\ref{thm:disjointunionpropersubshifts} it is enough to show that $\Int^2$ satisfies the conditions imposed on $G$ in the statement of the theorem. This follows from the results in \cite{Hoch12}. The desired sofic subshift $X\subseteq A^{\Int^2}$ is the subshift $Z$ from \cite[Theorem 5.3]{Hoch12}. Notice that by the construction $X$ is a disjoint union of minimal subshifts. Let $Z$ be a subshift of finite type factoring on $X$ via some factor map $\phi$. To show that for every effective subshift $Y\subseteq Z$, $\phi[Y]\subseteq X$ is a disjoint union of infinitely many subshifts, notice first that $\phi[Y]$ is then itself effective. So by \cite[Lemma 5.5]{Hoch12} each minimal subsystem of $\phi[Y]$ is an accumulation point of other minimal subsystems of $\phi[Y]$. This finishes the proof.
\end{proof}

We now need few notations. First, if $H\leq G$ are a group and a subgroup, $A$ is a non-trivial finite set, and $X\subseteq A^G$ is a subshift, the \emph{$H$-projective subdynamics} of $X$ is the subshift \[\{x\upharpoonright H\colon x\in X\}\subseteq A^H.\]

Moreover, we say that an element $x\in A^G$ is \emph{Toeplitz} if for every $g\in G$ the set \[\{g^{-1}h\colon x(h)=x(g)\}\subseteq G\] is a finite-index subgroup. We refer the reader to \cite{Krie} for details on Toeplitz elements in subshifts over general countable groups.

Recall also that a group is \emph{indicable} if it admits an epimorphism onto $\Int$.\medskip

We shall also need the notion of \emph{(a non-trivial) Medvedev degree} (see \cite[Section 3.2]{Hoch12}). Let $X\subseteq 2^\Nat$ be a set. We say that it is \emph{effective} if its complement is a recursive set of cylinders (cf. with Defintion~\ref{def:effectiveshubshift}). We say that an effective set $X\subseteq 2^\Nat$ has \emph{non-trivial Medvedev degree} if there is no computable function computing an element of $X$. Let $X,Y\subseteq 2^\Nat$ be effective sets. If there is a computable function from $X$ into $Y$ and $Y$ has non-trivial Medvedev degree, then clearly so does $X$.

In the following result we use the simulation theorem of Barbieri from \cite{Bar19} where we `simulate' certain effective subshifts as projective subdynamics of sofic subshifts

\begin{theorem}
Let $G$ be a  finitely generated recursively presented indicable group. Let $H_1,H_2$ be finitely generated recursively presented groups. Then $G\times H_1\times H_2$ does not have the strong topological Rokhlin property. %as well as any co-extension of $G\times H_1\times H_2$ by a finitely generated group.
\end{theorem}
\begin{proof}
Fix $G$,  and $H_1$ and $H_2$ as in the statement. Fix also an epimorphism $f:G\twoheadrightarrow \Int$. We show that $G\times H_1\times H_1$ satisfies the requirement on the group $G$ from the statement of Theorem~\ref{thm:disjointunionpropersubshifts}. The conclusion will then follow by Theorem~\ref{thm:disjointunionpropersubshifts}.

Let $g\in G$ be an arbitrary element satisfying $f(g)=1\in\Int$. Notice that $G$ splits as a semi-direct product $\langle g\rangle \ltimes K$, where $K\subseteq G$ is the kernel of the epimorphism $f$ and $\langle g\rangle$ is infinite cyclic. We also fix a finite symmetric generating set $S$ of $G$ containing $g$. We now follow Hochman's argument from \cite[Section 5.2]{Hoch12} to code effectively closed subsets of $2^\Nat$ by effectively closed subshifts of $3^G$. We also correct a mistake that appeared in that coding in \cite[Section 5.2]{Hoch12} (notice that the map defined there cannot distinguish between the elements $10000\ldots$ and $01111\ldots$). Let $\omega\in 2^\Nat$, where again $2$ is here identified with the two-element set $\{1,2\}$. We define $z_\omega\in 3^G$ as follows. We choose an arithmetic progression $A_1\subseteq \Int$ of period $3$ passing through $0$ and we set $z_\omega(g^i)=\omega(1)$ for $i\in A_1$. We also set $z_\omega(g^i)=3$ for $i\in 1+A_1:=\{j+1\colon j\in A_1\}$. Next let $A_2$ be an arithmetic progression of period $9$ passing through $2$ and we set $z_\omega(g^i)=\omega(2)$ for $i\in A_2$ and $z_\omega(g^j)=3$ for $j\in 3+A_2$. Next we take an arithmetic progression $A_3$ of period $27$ passing through $8$ and set $z_\omega(g^i)=\omega(3)$ for $i\in A_3$, etc. Finally, we set $z_\omega(h)=z_\omega(g^i)$ if and only if $h=g^ik$, where $k\in K$.

Clearly, $z_\omega$ is a Toeplitz configuration, so \[Z_\omega:=\overline{G\cdot z_\omega}\subseteq 3^G\] is a minimal subshift by \cite[Corollary 3.4]{Krie}. Moreover, for every $z\in Z_\omega$ there is an algorithm that computes $\omega$ with $z$ as an input. Indeed, $z_\omega$ is invariant under the shift by the subgroup $K$, therefore so is $z$, and the algorithm only reads the data from $z\upharpoonright \{g^i\colon i\in\Int\}$. The element $\omega$ can then be recovered as in \cite[Section 3.1]{Bar19}.

Moreover, if $\Omega\subseteq 2^\Nat$ is an effectively closed subset, defining \[Z:=\bigcup_{\omega\in\Omega} Z_\omega,\] we claim that
\begin{itemize}
	\item the Medvedev degree of $Z$ is at least that of $\Omega$; in particular, if $\Omega$ has non-trivial Medvedev degree, then so does $Z$;
	\item $Z$ is an effectively closed subshift of $3^G$.
\end{itemize}
The argument from the paragraphs above show that $\Omega$ is computable from $Z$, so the Medvedev degree is at least the Medvedev degree of $\Omega$ and if $\Omega$ has non-trivial Medvedev degree, so does $Z$.

We clearly have that $Z$ is closed under the shift action of $G$, so we need to show it is effective, i.e. we need to show that there is an algorithm that given a finite pattern $p\in 3^F$, for some finite $F\subseteq G$, decides whether it is forbidden in $Z$. Given such a pattern $p\in 3^F$, the algorithm first checks whether $p$ is invariant under $f$, i.e. whether for $h\neq h'\in F$ such that $f(h)=f(h')$ we have $p(h)=p(h')$. This is verifiable by an algorithm and if $p$ is not $f$-invariant, then the algorithm considers $p$ to be forbidden. If on the other hand $p$ is $f$-invariant, the algorithm decides whether $p$ is a fragment of a coding of some $\omega\in 2^\Nat$ into $z_\omega$. If not, then $p$ is considered as forbidden. If yes, then the algorithm appeals to the algorithm computing that $\Omega$ is effectively closed to semi-decide whether $p$ is forbidden.\medskip

Now let $Z'\subseteq Z$ be an effectively closed subshift. We claim that it is a disjoint union of infinitely many proper subshifts. Since $Z$ is a disjoint union of minimal subshifts, so is $Z'$. Therefore we just need to show that $Z'$ is a union of infinitely many of them. This however follows from \cite[Lemma 5.4]{Hoch12} - notice it was proved there for $G=\Int^d$, but it works generally.

Now we apply \cite[Theorem 4.2]{Bar19} with $G$, $H_1$, $H_2$, and the effectively closed subshift $Z\subseteq 3^G$ to obtain a sofic subshift $Y\subseteq B^{G\times H_1\times H_2}$, for some non-trivial finite set $B$, such that the restriction of the shift action of $G$ to $Y$ is conjugate to $X$ and the $G$-projective subdynamics of $Y$ is equal to $Z$, and $H_1\times H_2$ acts trivially. Since $Y$ is sofic, let $X\subseteq D^{G\times H_1\times H_2}$, for some non-trivial finite set $D$, be a subshift of finite type that factors via some $\phi:X\twoheadrightarrow Y$ onto $Y$. In order to finish the proof, we need to show that for every effective subshift $X'\subseteq X$, $\phi[X']\subseteq Y$ is a disjoint union of infinitely many proper subshifts. Let $X'\subseteq X$ be effective. Then since symbolic factors of effective subshifts are effective, $\phi[X']\subseteq Y$ is effective as well. Since $G$-projective subdynamics is $Z$, subshifts of $Y$ are in $1-1$ correspondence with subshifts of $Z$, so $\phi[X']$ corresponds to an effective subshift of $Z$ which is by the argument above a disjoint union of infinitely many disjoint minimal subshifts.
\end{proof}


\subsection{STRP for groups that are not finitely generated}
In this section we investigate the strong topological Rokhlin property for groups that are not finitely generated. We have already mentioned that Kechris and Rosendal noticed in \cite{KeRo} that the free group on countably infnitely many generators does not have the strong topological Rokhlin property. It is plausible, although we cannot prove it at the moment, that finite generation is a necessary condition for having the strong topological Rokhlin property. One however cannot immediately dismiss the idea that a non-finitely generated group with the STRP could exist. Interesting recent observations of Barbieri \cite{Bar22} point out that the problem of existence of a strongly aperiodic subshift of finite type for groups that are not finitely generated also looks impossible on the first sight, however he shows that the `first sight' is wrong in this case.

Our most general result is the following.
\begin{theorem}\label{thm:nonFGgroups}
Let $G$ be a countable group. If for every finitely generated subgroup $H\leq G$ there exists $g\in G\setminus H$ such that one of the conditions below holds:
\begin{enumerate}
	\item $g$ centralizes $H$, i.e. $gh=hg$ for all $h\in H$;
	\item The subgroup $\langle g, H\rangle$ is equal to the free product $\langle g\rangle\ast H$;
\end{enumerate}
then $G$ does not have the strong topological Rokhlin property.
\end{theorem}
\begin{proof}
We shall show something much stronger. We show that for such $G$ and for any finite $A$ with at least two elements, the only projectively isolated subshifts in $\SH_G(A)$ are the singletons, i.e. the monochromatic configurations. An application of Proposition~\ref{prop:noRokhlin} then immediately gives that $G$ does not have the strong topological Rokhlin property.\medskip

Fix $G$ and $A$. Assume that $X\subseteq A^G$ is a projectively isolated subshift which is witnessed by some subshift of finite type $Y\subseteq B^G$, for some finite alphabet $B$, some open neighborhood $\NN$ of $Y$ and a factor map $\phi$ defined on every $Z\in\NN$ so that $\phi[Z]=X$. Applying Lemma~\ref{lem:standardformoffactormaps} and Proposition~\ref{prop:SFTprop} we may and will without loss of generality suppose that
\begin{itemize}
	\item $\phi$ is induced by a finite map $\phi_0:B\rightarrow A$;
	\item $\NN$ is of the form $\NN_Y^F$ for some finite $F$ and $Y$ is of the form $X_{\VV_F}$ for some $F$-Rouzy graph $\VV_F=(V,(E_f)_{f\in F})$. In particular, $B=V$.
\end{itemize}

Let $H:=\langle F\rangle\leq G$ be the subgroup of $G$ generated by $F$. By our assumption, there exists $g\in G\setminus H$ such that either $g$ centralizes $H$ or has no relations with $H$, i.e. $\langle g, H\rangle$ is equal to the free product $\langle g\rangle\ast H$.\medskip

Let us at first assume the latter. We define a new $F'$-Rouzy graph $\VV'_{F'}:=(V, (E'_f)_{f\in F'})$, where
\begin{itemize}
	\item $F':=F\cup\{g\}$;
	\item $E'_f=E_f$, for $f\in F$;
	\item $E'_g$ is $\{(v,v)\colon v\in V\}$, the directed graph on $V$ consisting of all loops.
\end{itemize}

Set $Z:=X_{\VV'_{F'}}$. It is immediate that $Z\in \NN$. We need to check that $Z$ is non-empty and that $\phi[Z]\neq X$.

Denote by $\bar H$ the subgroup $\langle g,H\rangle$. Let $p_1$ be the identity map on $H$ and let $p_2:\langle g\rangle:\rightarrow\{1_G\}$ be the map sending $g$ to the unit. Set \[p:=p_1\ast p_2: \bar H\rightarrow H.\]
Notice that every $h\in \bar H$ can be uniquely written as a word $h_1 h_2\ldots h_n$, where for $i<n$, $h_i\in H$ and $h_{i+1}\in\langle g\rangle$, or vice versa, and assuming for concreteness that e.g. $h_1\in H$ and $h_n\in \langle g\rangle$ we have in that case \[p(h_1 h_2\ldots h_n)=h_1 h_3\ldots h_{n-1}.\] 

To check that $Z$ is non-empty, pick any $y\in Y$. Let $(g_n)_{n\in\Nat}$ be an arbitrary set of left coset representatives for $\bar H$ in $G$. We define $z\in B^G$ as follows. Any $g\in G$ can be uniquely written as $g_n\cdot h$ for some $n\in\Nat$ and $h\in\bar H$ and we set \[z(g):=y(p(h)).\]
We claim that $z\in Y$ witnesses that $Z$ is non-empty since $z$ contains only allowed patterns as defined by $\VV'_{F'}$. This needs to be checked only on the left cosets of $\bar H$ since the forbidden patterns are determined by a finite subset of $\bar H$. We check it for the coset $\bar H$.

We need to check that for every $h\in\bar H$ and $f\in F'$, $(z(h),z(hf))$ is allowed. We do it by induction on the length of $h$ as a word $h_1 h_2\ldots h_n$, where for $i<n$, $h_i\in H$ and $h_{i+1}\in\langle g\rangle$, or vice versa. Suppose first that $n=1$. Either $h\in H$, or $h\in\langle g\rangle$. In the former case, if moreover $f\in F\subseteq H$, then $(z(h),z(hf))=(y(h),y(hf))$, and if $f=g$, then $z(h)=y(h)$ and $z(hf)=y(p(hf))=y(h)=z(h))$ which is again allowed. In the latter case, i.e. $h\in\langle g\rangle$ we have $z(h)=y(p(h))=z(1_G)$.  If $f\in H$ then arguing as before we have $(z(h),z(hf))=(y(1_G),y(f))$ which is allowed, or if $f=g$ we have $z(h)=z(hf)=y(1_G)$ which is again allowed. The general induction step is verified similarly and left to the reader.\medskip

Now we check that $\phi[Z]\subsetneq X$. Notice first that since for every $z\in Z$ and $h\in G$ we have $z(h)=z(hg)$ and since $\phi$ is induced by $\phi_0:B\rightarrow A$ we get $\phi(z)(h)=\phi(z)(hg)$. Thus in order to show that $\phi[Z]\subsetneq X$ it is enough to find $x\in X$ and $h\in G$ such that $x(h)\neq x(hg)$.

Denote by $(v_n)_{n\in\Nat}$ some left coset representatives for $H$ in $G$, where $v_1=1_G$ and $v_2=g$. For $y\in B^G$ we have $y\in Y$ if and only if there exist $(z_n)_{n\in\Nat}\subseteq Y$ such that for any $n\in\Nat$ \[\forall h\in H\; \big(y(v_n h)=z_n(h)\big).\] Find arbitrarily some $(z_n)_{n\in\Nat}\subseteq Y$ such that \[\phi_0\big(z_1(1_G)\big)\neq \phi_0\big(z_2(1_G)\big),\] and define $y\in B^G$ so that for any $n\in\Nat$ \[\forall h\in H\; \big(y(v_n h)=z_n(h)\big).\] Then by above, we have $y\in Y$, however by definition \[\phi(y)(1_G)=\phi_0\big(z_1(1_G)\big)\neq \phi_0\big(z_2(1_G)\big)=\phi(y)(g).\] Since $\phi(y)\in X$, this is the desired contradiction.\bigskip

Now we assume that there exists $g\in G\setminus H$ that centralizes $H$. We distinguish two cases.
\begin{enumerate}
	\item Either $\bar H:=\langle g,H\rangle$ is equal to the direct product $H\times \langle g\rangle$,
	\item or $g$ is a root of a non-trivial element of $H$, i.e. there exist $h\in H\setminus \{1_G\}$ and $n\geq 2$ such that $g^n=h$ (where $h$ must be in the center of $H$).
\end{enumerate}

In the first case, we can define $Z$ to be $X_{\VV'_{F'}}$, where $\VV'_{F'}$ is exactly the same as above and we leave to the reader to verify that again $Z\in\NN$, $Z$ is non-empty, and $\phi[Z]\subsetneq X$.

So we now consider the case that there are $h\in H\setminus \{1_G\}$ and $n\geq 2$ such that $g^n=h$. We also assume that $n$ is the minimal $m\geq 2$ such that $g^m\in H$. We then set $Z\subseteq Y$ to be the subshift of finite type where for any $z\in B^G$, in order to be in $Z$, we require that (obviously) $z\in Y$ and there are no $h\in G$ and $1\leq i<j<k\leq n-1$ such that \[z(hg^i)\neq z(hg^j)\neq z(hg^k).\]
It is straightforward to check that $Z\in\NN$ and $\phi[Z]\subsetneq X$, so we only show that $Z$ is non-empty. Pick any $y\in Y$ and let $(g_m)_{m\in\Nat}$ be some set of left coset representatives of $\bar H=\langle g,H\rangle$ in $G$. Then we define $z\in A^G$  by setting for any $m\in\Nat$, $h\in H$ and $i\leq n-1$ \[z(g_m h g^i)=y(h).\]  Clearly, $z\in Y$. Moreover, one can easily read off the definition that there are no $m\in\Nat$, $h\in H$ and $1\leq i<j<k\leq n-1$ such that $z(g_m h g^i)\neq z(g_m h g^j)\neq z(g_m h g^k)$, thus $z\in Z$ as desired. This finishes the proof.
\end{proof}

The following immediate corollary is another proof of the fact proved in \cite[2nd remark on page 331]{KeRo}.
\begin{corollary}
Let $G$ be the free group on countably many generators. Then $G$ does not have the strong topological Rokhlin property.
\end{corollary}
The next result also immediately follows from Theorem~\ref{thm:nonFGgroups}. It is perhaps not so surprising that the free abelian group on countably many generators does not have the strong topological Rokhlin property since already $\Int^d$, for $d\geq 2$, does not have it, however we get that even `one-dimensional' groups such as $\Rat$ do not have the strong topological Rokhlin property.
\begin{corollary}
Let $G$ be a group that contains a center that is not finitely generated. Then $G$ does not have the strong topological Rokhlin property. In particular, non-finitely generated abelian groups do not have the strong topological Rokhlin property.
\end{corollary}
One of the possible candidates for a non-finitely generated group that has the strong topological Rokhlin property could be the Hall universal locally finite group. Recall that the Hall group is defined as the unique countably infinite locally finite group which contains every finite group as a subgroup and where any two finitely generated, i.e. finite, subgroups are conjugated (see \cite{Hall}). This group itself is generic, in a sense, in the class of countable locally finite groups, so one might hope that this could be the natural candidate for a non-finitely generated group with the strong topological Rokhlin property.
\begin{corollary}
The Hall universal locally finite group does not have the strong topological Rokhlin property.
\end{corollary}
\begin{proof}
We apply Theorem~\ref{thm:nonFGgroups}. Denote by $G$ the Hall group and let $H\leq G$ be some finitely generated subgroup. We verify that condition (2) of Theorem~\ref{thm:nonFGgroups} is satisfied. Since $G$ contains every finite group as a subgroup the direct product $H\times \Int_2$ embeds via some monomorphism $\phi$ into $G$. Denote by $f\in\Int_2$ the non-trivial element of order $2$. Set $H':=\phi[H\times\{1\}]$. Since $H'\leq G$ is isomorphic to $H$ there exists $g\in G$ such that $gH'g^{-1}=H$. It follows that $g\phi\big(1,f)\big)g^{-1}$ is a non-trivial element commuting with $H$. This finishes the proof.
\end{proof}

\section{Dynamical properties of the generic actions}\label{sect:genericdynamics}
This section contains several other genericity results in Cantor dynamics of group actions. Most of them are consequences of Theorem~\ref{thm:mainRokhlin} and they reduce to investigating properties of the single generic action. However, this is not the case in general - see the results concerning entropy and shadowing below.  The main result proved here is Theorem~\ref{thm:intro4} from Introduction.

\subsection{Basic dynamics and entropy}
We start with a simple proposition.
\begin{proposition}
For no countable group with the strong topological Rokhlin property the generic action is transitive.
\end{proposition}
\begin{proof}
Let $G$ be a group with the STRP and let $\alpha\in\Act_G(\CC)$ be the generic action. Suppose it is transitive. By the proof of Theorem~\ref{thm:mainRokhlin}, $\alpha$ factors onto every projectively isolated subshift over $G$. Setting $X\subseteq 2^G$ to be e.g. the two element subshift consisting of the two fixed points, we see that $X$ is isolated and thus it is a factor of $\alpha$. Factors of transitive systems are transitive though, a contradiction.
\end{proof}

Our next focus is on topological entropy. Here we do not appeal to Theorem~\ref{thm:mainRokhlin} and follow and apply the results from \cite{FriTam}.
\begin{definition}
Let $G$ be a countable amenable group and let $(F_n)_{n\in\Nat}$ be a F\o lner sequence of strictly increasing finite subsets that cover $G$. Suppose that $G$ acts continuously on a compact metrizable space $X$ with a compatible metric $d$. For a finite set $F\subseteq G$ denote by $d_F$ the metric on $X$ defined for $x,y\in X$ by \[d_F(x,y):=\sup_{g\in F} d(gx,gy).\] 
The \emph{topological entropy}  $h_\topo(X,G)$ of the action is \[\sup_{\varepsilon>0} h_\sep(\varepsilon),\] where for $\varepsilon>0$ \[h_\sep(\varepsilon):=\limsup_{n\to\infty} \frac{1}{|F_n|}\log\sep(d_{F_n},\varepsilon),\] and $\sep(d_{F_n},\varepsilon)$ is the maximal size of an $\varepsilon$-separated subset of $X$ with respect to the metric $d_{F_n}$.

If $X\subseteq A^G$ is a subshift then in fact we have \[h_\topo(X,G)=\limsup_{n\to\infty}\frac{\log|X_{F_n}|}{|F_n|}.\]
\end{definition}
It is known that the value of entropy does not depend on the compatible metric $d$. We refer to \cite[Chapter 9.9]{KeLi-book} for more details on topological entropy of general countable group actions.

\begin{lemma}\label{lem:entropyinverselimit}
Let $G$ be a countable amenable group acting on a zero-dimensional compact metrizable space $X$ so that the action is an inverse limit of subshifts $(X_n)_{n\in\Nat}$. Then $h_\topo(X,G)=0$ if and only if for all $n\in\Nat$ we have $h_\topo(X_n,G)=0$.
\end{lemma}
\begin{proof}
Fix $G$, the action of $G$ on a space $X$, and the sequence $(X_n)_{n\in\Nat}$. The left-to-right implication is clear since entropy cannot increase under taking factors, see e.g. \cite[Remark 10.16]{KeLi-book}. So suppose that $h_\topo(X_n,G)=0$ for all $n\in\Nat$ and let us show that $h_\topo(X,G)=0$. For the rest of the proof, we fix some compatible metric on $X$ and the metric notions are considered with respect to it. Suppose on the contrary that $h_\topo(X,G)>\delta>0$. Fix a F\o lner sequence $(F_n)_{n\in\Nat}$ of $G$ that is strictly increasing and covers $G$, and we may moreover supppose it consists of symmetric sets. Then there exists $\varepsilon>0$ so that $h_\sep(\varepsilon)>\delta$, and so for an arbitrarily large $n$, and without loss of generality for all $n$, we have $\log\sep(d_{F_n},\varepsilon)/|F_n|>\delta$.

By Lemma~\ref{lem:factoringonshift}, for each $n\in\Nat$ the factor map $\phi_n:X\rightarrow X_n$ is induced by a clopen partition $\PP_n$ of $X$ so that $\phi_n=Q_{\PP_n}$. Since $X$ is an inverse limit of the sequence $(X_n)_{n\in\Nat}$ there exists $m\in\Nat$ so that $\max\{\mathrm{diam}(P)\colon P\in\PP_m\}<\varepsilon$. For every $n\in\Nat$ let $E_n\subseteq X$ be a subset of size $\sep(d_{F_n},\varepsilon)$ that is $\varepsilon$-separated with respect to $d_{F_n}$. Set $D_n:=\phi_m[E_n]\subseteq X_m$. Since for $x\neq y\in E_n$ there exists $g\in F_n$ so that $d(gx,gy)>\varepsilon$ we have that $gx$ and $gy$ lie in a different element of the partition $\PP_m$. It follows that $|(X_m)_{F_n}|\geq |D_n|=|E_n|$. However, then by definition we have \[0=h_\topo(X_m,G)=\limsup_{n\to\infty}\frac{\log|(X_m)_{F_n}|}{|F_n|}\geq \limsup_{n\to\infty}\frac{\log |E_n|}{|F_n|}>\delta,\] a contradiction.
\end{proof}

\begin{corollary}
Let $G$ be a countable amenable group that has the strong topological Rokhlin property. Then the entropy of the generic action is zero.
\end{corollary}
\begin{proof}
Let $G$ be a countable amenable group with the strong topological Rokhlin property and let $\alpha\in\Act_G(\CC)$ be the generic action. By the proof of Proposition~\ref{prop:MAINexistence}, $\alpha$ is an inverse limit of projectively isolated subshift. It is enough to show that every projectively isolated subshift has zero topological entropy since then we will be done by Lemma~\ref{lem:entropyinverselimit}. By \cite[Theorem 1.4]{FriTam}, for every non-trivial finite set $A$, the set $\{X\in\SH_G(A)\colon h_\topo(X)=0\}$ is comeager, in particular it is dense. Since for every projectively isolated subshift $Z$ there is an open set $\NN$ of subshifts all of whose factors onto $Z$, it follows that there is a zero-entropy subshift factoring onto $Z$. Since entropy cannot increase in factors, we are done.
\end{proof}

There are however many amenable groups that do not have the strong topological Rokhlin property, e.g. all finitely generated nilpotent groups that are not virtually cyclic by Corollary~\ref{cor:nilpotentnoSTRP}. In fact, we do not know any amenable group that is not virtually cyclic that has the strong topological Rokhlin property. We want to show that nevertheless, zero entropy is generic in $\Act_G(\CC)$ for any countable amenable group.

In what follows, we freely identify the spaces $\SH_G(n)$ and $\SH_G(A)$, when $|A|=n$. The following proposition is of independent interest and could be possibly applied in more situations.
\begin{proposition}\label{prop:comeagertransfer}
Let $G$ be a countable group and let $\EE$ be a property of subshifts over $G$ that is closed under taking factors and such that for each $n\in\Nat$ \[D_n:=\{X\in \SH_G(n)\colon X\text{ has }\EE\}\] is a dense $G_\delta$ set. Then the set \[\DD:=\{\alpha\in\Act_G(\CC)\colon \forall \PP\text{ clopen partition of }\CC\;\big(\QQ(\alpha,\PP)\in D_{|\PP|}\big)\}\] is dense $G_\delta$ in $\Act_G(\CC)$.
\end{proposition}
\begin{proof}
We fix the notation from the statement. Fix a clopen partition $\PP$. It suffices to show that the set \[\DD_\PP:=\{\alpha\in\Act_G(\CC)\colon \QQ(\alpha,\PP)\in D_{|\PP|}\}\] is dense $G_\delta$ as \[\DD=\bigcap_{\PP\text{ clopen partition}} \DD_\PP.\] By the continuity of the map $\QQ(\cdot,\PP)$, for every clopen partition $\PP$, $\DD_\PP$ is $G_\delta$, so we need to prove it is dense. Let $\NN\subseteq\Act_G(\CC)$ be an open set which we may suppose is of the form $\NN_\beta^{F,\PP'}$ for some $\beta\in\Act_G(\CC)$, finite set $F\subseteq G$, and a clopen partition $\PP'\preceq\PP$.

Set $Y:=\QQ(\beta,\PP')$ and notice that by Proposition~\ref{prop:Qmap-nbhds} we have $\QQ(\cdot,\PP')[\NN]=\NN_Y^F$. Since $D_{|\PP'|}$ is dense in $\SH_G(\PP')$ there exists $Y'\in \NN_Y^F\cap D_{|\PP'|}$. Let $\phi: (\PP')^G\rightarrow \PP^G$ be the factor map induced by the inclusion map $\phi_0:\PP'\rightarrow\PP$ defined so that for every $P\in\PP'$ we have $P\subseteq \phi_0(P)$. As $\EE$ is closed under taking factors, we have $\phi[Y']\in D_{|\PP|}$.  Again by Proposition~\ref{prop:Qmap-nbhds}~\eqref{it1-Qmap-nbhds}, there exists $\gamma\in\NN$ such that $\QQ(\gamma,\PP')=Y'$, and then, by Proposition~\ref{prop:Qmap-nbhds}~\eqref{it2-Qmap-nbhds}, we have \[\QQ(\gamma,\PP)=\phi\big(\QQ(\gamma,\PP')\big)\in D_{|\PP|},\] showing that $\DD_\PP$ is dense as $\NN$ was arbitrary. This finishes the proof.
\end{proof}
\begin{corollary}
Let $G$ be a countable amenable group. Then the set \[\{\alpha\in\Act_G(\CC)\colon h_\topo(\alpha)=0\}\] is dense $G_\delta$.
\end{corollary}
\begin{proof}
Fix a countable amenable group $G$ and let $A$ be a non-trivial finite set. By \cite[Theorem 1.4]{FriTam} the set of subshifts in $\SH_G(A)$ with zero entropy is comeager. However, an inspection of the proof in fact shows it is dense $G_\delta$. In fact, it is easy to see that the set of subshifts with zero entropy is $G_\delta$. Applying Proposition~\ref{prop:comeagertransfer} with $\EE$ being the property of having zero entropy, which is closed under taking factors, concludes the proof.
\end{proof}

\subsection{Shadowing} As mentioned in Introduction, shadowing has been originally defined for actions of $\Int$, however now the notion is available for any countable group (see \cite{OsTi} and \cite{ChKeo}).
\begin{definition}
Let $G$ be a countable group acting on a compact metrizable space $X$. Let $d$ be a compatible metric on $X$. For $\delta>0$ and finite set $S\subseteq G$, a $G$-indexed set $(x_g)_{g\in G}\subseteq X$ is called a \emph{$(\delta,S)$-pseudo-orbit} if for every $g\in G$ and $s\in S$ we have \[d(s\cdot x_g,x_{sg})<\delta.\]

We say that the action has the \emph{shadowing}, or \emph{the pseudo-orbit tracing property}, if for any $\varepsilon>0$ there are $\delta>0$ and finite set $S\subseteq G$ such that for any $(\delta,S)$-pseudo-orbit $(x_g)_{g\in G}\subseteq X$ there exists $x\in X$ whose orbit $\varepsilon$-traces the pseudo-orbit, i.e. for every $g\in G$ \[d(x_g,g\cdot x)<\varepsilon.\]
\end{definition}
It is straightforward to verify that shadowing does not depend on the choice of the compatible metric and moreover that if $G$ is finitely generated then the finite set $S$ from the definition can be always taken to be some fixed finite generating set of $G$.

The following definition has its origin in \cite{GoMe} where it turned out to be crucial for describing inverse limits with shadowing for actions of $\Int$. The version for general countable groups appeared in \cite{LiChZh}.

\begin{definition}
Let $G$ be a countable group. Let $(X_n)_n$ be an inverse system of subshifts over $G$, where the bonding maps $(\phi_n^m: X_n\rightarrow X_m)_{n\geq m\in\Nat}$ are not necessarily onto, however for every $n_0\in\Nat$ there exists $n\in\Nat$ such that for every $m\geq n$ we have $\phi_m^{n_0}[X_m]=\phi_n^{n_0}[X_n]$. Then we say that the inverse system satisfies the \emph{Mittag-Leffler condition}.
\end{definition}

The main results of \cite{GoMe} and \cite{LiChZh} will be applied in the following theorem.

\begin{theorem}\label{thm:genericshadowing}
Let $G$ be a finitely generated group with the strong topological Rokhlin property. Then the generic action of $G$ on the Cantor space has shadowing. In particular, shadowing is generic in $\Act_G(\CC)$.
\end{theorem}
\begin{proof}
Let $G$ be as in the statement and let $\alpha\in\Act_G(\CC)$ be the generic action. By the proof of Theorem~\ref{thm:mainRokhlin}, $\alpha$ is an inverse limit of a sequence $(X_n)_{n\in\Nat}$ of projectively isolated subshifts with respect to isolated factor maps $(\phi_n^m:X_n\to X_m)_{n\geq m\in\Nat}$.\medskip

We define a sequence $(Y_n)_{n\in\Nat}$ of subshifts of finite type. Let $Y_1$ be any subshift of $\alpha(X_1)^G$ that is of finite type and such that $X_1\subseteq Y_1$. Pick now $n\geq 2$. By definition, the isolated factor map $\phi_n^{n-1}: X_n\rightarrow X_{n-1}$ is defined on some neighborhood $\NN_{X_n}^{F_n}$, where $F_n\subseteq G$ is a finite set, such that for every $Y\in\NN_{X_n}^{F_n}$, we have $\phi_n^{n-1}[Y]=X_{n-1}$. Notice that $\NN_{X_n}^{F_n}$ contains a subshift of finite type $Y_n$ such that $X_n\subseteq Y_n$. Indeed, define $Y_n$ to be a subshift of $\alpha(X_n)^G$ that is of finite type, whose defining window is $F_n$ and with $(X_n)_{F_n}$ as the set of allowed patterns. By definition, $X_n\subseteq Y_n\in \NN_{X_n}^{F_n}$ and $\phi_n^{n-1}[Y_n]=X_{n-1}$.

We claim that the system $(Y_n)_{n\in\Nat}$ with bonding maps $(\phi_n^m)_{n\geq m\in\Nat}$ is an inverse system satisfying the Mittag-Leffler condition and the inverse limit is equal to $\alpha$. Pick $n_0\in\Nat$. Let $n:=n_0+1$. Then for any $m\geq n$ we have \[\phi_m^{n_0}[Y_m]=\phi_m^{n_0}[X_m]=\phi_n^{n_0}[X_n]=\phi_n^{n_0}[Y_n]=X_{n_0},\] witnessing the Mittag-Leffler condition. Finally, let us show that the inverse limits of the sequences $(X_n)_n$, resp. $(Y_n)_n$ are equal. Pick $(x_n)_n\in \underset{n\to\infty}{\varprojlim} X_n$. Since for every $n\in\Nat$, $X_n\subseteq Y_n$, clearly $(x_n)_n\in \underset{n\to\infty}{\varprojlim} Y_n$. Conversely, pick $(y_n)_n\in \underset{n\to\infty}{\varprojlim} Y_n$. We claim that for every $n\in\Nat$, $y_n\in X_n$, and thus $(y_n)_n\in \underset{n\to\infty}{\varprojlim} X_n$. Indeed, for any $n\in\Nat$ we have $\phi_{n+1}^n(y_{n+1})=y_n$. Since $y_{n+1}\in Y_{n+1}$ and $\phi_{n+1}^n[Y_{n+1}]=X_n$, we get $y_n\in X_n$.\medskip

Finally, we apply \cite[Theorem 1.2]{LiChZh} with $(Y_n)_n$ to get that $\alpha$ has shadowing.
\end{proof}


As a corollary of Theorems~\ref{thm:genericshadowing} and~\ref{thm:STRPfreeproducts} we get the following.
\begin{corollary}
Let $G=\bigstar_{i\leq n} G_i$, where for $i\leq n$, $G_i$ is finite or cyclic. Then shadowing is generic in $\Act_G(\CC)$.
\end{corollary}
The next step is to prove the converse, i.e. actions with shadowing form a meager set in $\Act_G(\CC)$ for groups $G$ without the strong topological Rokhlin property. First we reformulate the original definition of pseudo-orbit tracing property to an equivalent definition for zero-dimensional spaces that is more convenient for us as it employs clopen partitions instead of compatible metrics.
\begin{lemma}\label{lem:shadowinonCantor}
Let $G$ be a finitely generated group acting continuously on a zero-dimensional compact metrizable space $X$. Let $S\subseteq G$ be a finite symmetric generating set. Then the action has the pseudo-orbit tracing property if and only if for every clopen partition $\PP$ of $X$ there exists a refinement $\PP'\preceq \PP$ so that for every $(x_g)_{g\in G}\subseteq X$ satisfying that for all $g\in G$ and $s\in S$ we have $x_{sg}$ and $s x_g$ lie in the same element of the partition $\PP'$, there is $x\in X$ such that $gx$ and $x_g$ lie in the same element of $\PP$ for every $g\in G$.
\end{lemma}
\begin{proof}
The straightforward proof is similar to the proof of Lemma~\ref{lem:basicopennbhds} and left to the reader.
\end{proof}

By definition, every symbolic factor of a subshift of finite type is sofic. Since subshifts of finite type are precisely those subshifts with pseudo-orbit tracing property (this is easy to check, a formal proof can be found in \cite{ChKeo}), the following proposition is a generalization of this fact - that a factor of an SFT is sofic.
\begin{proposition}\label{prop:shadowingfactorsonsofic}
Let $G$ be a finitely generated group acting continuously on a zero-dimensional compact metrizable space $X$ so that the action has the pseudo-orbit tracing property. Then every symbolic factor of the action is sofic.
\end{proposition}
\begin{proof}
Fix the group $G$ and its finite symmetric generating set $S\subseteq G$ containing the unit. Fix also a zero-dimensional compact metrizable space $X$ on which $G$ acts continuously with the pseudo-orbit tracing property. Denote the action by $\alpha$. Let $\phi:\alpha\rightarrow Y$ be a factor map, where $Y\subseteq A^G$ is a subshift for some non-trivial finite set $A$. We show that $Y$ is sofic. By Lemma~\ref{lem:factoringonshift} there exists a clopen partition $\PP$ of $X$, with $|\PP|=|A|$, so that $\phi=Q_\PP^\alpha$. By the assumption and Lemma~\ref{lem:shadowinonCantor}, there exists a refinement $\PP'\preceq\PP$ so that for every $(x_g)_{g\in G}\subseteq X$ satisfying that for all $g\in G$ and $s\in S$ we have $x_{sg}$ and $s x_g$ lie in the same element of the partition $\PP'$, there is $x\in X$ such that $gx$ and $x_g$ lie in the same element of $\PP$ for every $g\in G$.

We define a subshift $Z\subseteq (\PP')^G$ of finite type with defining window $S$ as follows. A pattern $p\in (\PP')^S$ is allowed if and only if there exists $x\in X$ such that for every $s\in S$ and $P\in\PP'$ \[p(s)=P\Leftrightarrow s^{-1}x\in P.\] Let $f_0:\PP'\rightarrow \PP$ be the inclusion map, i.e. $P\subseteq f_0(P)$ for every $P\in\PP'$. It induces a continuous $G$-equivariant map $\psi:Z\rightarrow \PP^G$. We claim that $\psi[Z]=Y$. The proof of the claim will finish the proof of the proposition since $Y$ will then be a factor of a subshift of finite type, therefore sofic.\medskip

Let us first show that $Y\subseteq \psi[Z]$. Set $\phi':=Q_{\PP'}^\alpha$ and notice that
\begin{itemize}
	\item $\phi=\psi\circ\phi'$,
	\item $\phi'[X]\subseteq Z\subseteq (\PP')^G$ since by the definition of $Z$, we have $(\phi[X])_S=Z_S$.
\end{itemize}

Pick $y\in Y$ and $x\in X$ such that $\phi(x)=y$. For $z:=\phi'(x)\in \phi'[X]\subseteq Z$ we then have \[\psi(z)=\psi\circ\phi'(x)=\phi(x)=y.\]

Finally we show that $\psi[Z]\subseteq Y$. Pick $z\in Z$ and let us show that $\psi(z)\in Y$. For every $g\in G$, $p_g:=gz\upharpoonright S\in (\PP')^S$ is an allowed pattern of $Z$, so by the definition of $p_g$ there exists $x_g\in X$ such that for every $s\in S$ and $P\in\PP'$ \[p(s)=P\Leftrightarrow s^{-1}x_g\in P.\] It follows that $(x_g)_{g\in G}\subseteq X$ has the property that for every $g\in G$ and $s\in S$ we have $x_{sg}$ and $sx_g$ lie in the same element of the partition of $\PP'$. Then, applying the pseudo-orbit tracing property with respect to $\PP'$ and $\PP$, there exists $x\in X$ such that for every $g\in G$ we have that $gx$ and $x_g$ lie in the same element of $\PP$. We claim that $\phi(x)=\psi(z)$. Pick $g\in G$ and let us show that $\phi(x)(g)=\psi(z)(g)$. For $P\in\PP$ we have \[\begin{split}& \phi(x)(g)=P\Leftrightarrow g^{-1}x\in P\Leftrightarrow x_{g^{-1}}\in P\Leftrightarrow p_{g^{-1}}(1_G)\subseteq P\Leftrightarrow\\ & g^{-1}z(1_G)\subseteq P\Leftrightarrow z(g)\subseteq P\Leftrightarrow \psi(z)(g)=P.\end{split}\]
\end{proof}

\begin{proposition}\label{prop:NoSTRPnonsoficfactors}
Let $G$ be a countable group that does not have the strong topological Rokhlin property. Then the set \[\AAA:=\{\alpha\in\Act_G(\CC)\colon \exists \PP\text{ clopen partition }\;(\QQ(\alpha,\PP)\text{ is not sofic})\}\] is non-meager.
\end{proposition}
\begin{proof}
Fix $G$ without the strong topological Rokhlin property. By Proposition~\ref{prop:noRokhlin}, there exist a non-trivial finite set $A$ and an open set $\NN\subseteq\SH_G(A)$ that contains no projectively isolated subshift. Applying Lemma~\ref{lem:SFTnbhrds}, without loss of generality, we may assume that $\NN=\NN_X^F$, where $X\subseteq A^G$ is a subshift of finite type, $F\subseteq G$ is a finite subset, and for every $X'\in\NN$ we have $X'\subseteq X$. By Proposition~\ref{prop:Qcontinuity}, there exist $\alpha\in\Act_G(\CC)$ and a clopen partition $\PP$ of $\CC$ such that $\QQ(\alpha,\PP)=X$. Moreover, again by Proposition~\ref{prop:Qcontinuity}, by the continuity of $\QQ(\cdot,\PP)$ there exists an open neighborhood of $\alpha$, which we may suppose to be of the form $\NN_\alpha^{E,\PP'}$, for some finite set $E\subseteq G$ and a clopen partition $\PP'\preceq \PP$ such that $\QQ(\cdot,\PP)[\NN_\alpha^{E,\PP'}]\subseteq \NN_X^F$. In particular, for every $\beta\in\NN_\alpha^{E,\PP'}$ we have $\QQ(\beta,\PP)\subseteq X$.

Let $\{X_n\colon n\in\Nat\}$ be an enumeration of all sofic subshifts inside $\NN$. Notice that the set is indeed infinite since otherwise, as $\NN$ does not contain isolated points, $\NN$ would contain an open subset without any sofic subshift which is a contradiction as subshifts of finite type are dense - recall Lemma~\ref{lem:SFTnbhrds}. Let $m=|A|$ and for each $n\in\Nat$ set \[\begin{split}A_n:= & \{\alpha\in\Act_G(\CC)\colon\\ & \text{for no clopen partition }\RR=\{R_1,\ldots,R_m\}\text{ of }\CC,\; \QQ(\alpha,\RR)=X_n\}.\end{split}\]

By {\bf Claim} from Proposition~\ref{prop:noRokhlin}, $A_n$ is dense $G_\delta$.

Suppose now that $\AAA$ is meager, so $\Act_G(\CC)\setminus\AAA$ is comeager, and we reach a contradiction. There exists an action \[\gamma\in \NN_\alpha^{E,\PP'}\cap \big(\Act_G\setminus \AAA\big)\cap \bigcap_{n\in\Nat} A_n,\] since the set on the right-hand side is non-meager, so non-empty. Since $\gamma\in\NN_\alpha^{E,\PP'}$ we get that $\QQ(\gamma,\PP)\subseteq X$. Since $\gamma\notin\AAA$ we get that $Y:=\QQ(\gamma,\PP)\subseteq X$ is sofic. However, then $Y=X_n$ for some $n\in\Nat$. Since $\gamma\in A_n$ it follows that $\QQ(\gamma,\PP)\neq X_n$, a contradiction.
\end{proof}

\begin{theorem}
Let $G$ be a countable group that does not have the strong topological Rokhlin property. Then the set \[\SSS:=\{\alpha\in\Act_G(\CC)\colon \alpha\text{ has shadowing}\}\] is dense, but meager. In particular, shadowing is not generic for actions of $G$.
\end{theorem}
\begin{proof}
Fix $G$ as in the statement. Density of $\SSS$ follows from the fact that subshifts of finite type have shadowing (see e.g. \cite{ChKeo}) and actions $\alpha\in\Act_G(\CC)$ conjugate to subshifts of finite type are dense by Proposition~\ref{prop:Qcontinuity} and Lemma~\ref{lem:SFTnbhrds}. To get meagerness, applying Fact~\ref{fact:0-1-law}, it suffices to show that $\SSS$ is not comeager. To reach a contradiction, suppose that $\SSS$ is comeager. Then it has a non-empty intersection with the set $\AAA$ from the statement of Proposition~\ref{prop:NoSTRPnonsoficfactors}, so there is $\alpha\in \SSS\cap\AAA$. It follows that there is a clopen partition $\PP$ of $\CC$ such that $\QQ(\alpha,\PP)$ is not sofic. That is however in contradiction with Proposition~\ref{prop:shadowingfactorsonsofic}.
\end{proof}
The following is an immediate corollary (cf. with e.g. \cite{Ko07}).
\begin{corollary}
Generically, Cantor space actions of $\Int^d$, for $d\geq 2$, or more generally of finitely generated nilpotent groups that are not virtually cyclic, do not have shadowing.
\end{corollary}
\section{Actions of locally compact groups}\label{sect:locallycompactgroups}
Recall that given a locally compact second-countable Hausdorff topological group $G$ (equivalently, $G$ is locally compact and Polish) the set $\Act_G(\CC)$ of all continuous actions of $G$ on the Cantor space can be given a Polish topology generated by the basic open neighborhoods
\[\NN_\alpha^{K,\PP}:=\{\beta\in\Act_G(\CC)\colon \forall g\in K\;\forall x\in\CC\; \forall P\in\PP\; (\alpha(g)x\in P\Leftrightarrow \beta(g)x\in P)\},\]
where $\alpha\in\Act_G(X)$, $K\subseteq G$ is a compact subset, and $\PP$ is a clopen partition of $\CC$.

Clearly this definition agrees with Definition~\ref{def:spaceofactions} when $G$ is discrete.

It is therefore possible to extend the definition of having the strong topological Rokhlin property to locally compact Polish groups by again requiring that there is a continuous action of such a group on the Cantor space whose conjugacy class is generic in the space of all actions.\medskip

Given a locally compact group $G$, there is a short exact sequence \[1\to G_0\to G\to G/G_0\to 1,\] where $G_0$ is the connected component of the unit of $G$ and $G/G_0$ is totally disconnected. Whenever such $G$ acts continuously on the Cantor space, the action is trivial when restricted to $G_0$, since $\CC$ is totally disconnected, so the action factorizes through an action of $G/G_0$. From this it follows that if $G$ is a locally compact group, then $G$ has the strong topological Rokhlin property if and only if $G/G_0$ does. It is therefore reasonable to restrict to the case when $G$ is totally disconnected.

\begin{definition}
Let $G$ and $H$ be groups (possibly topological) acting continuously on compact spaces $X$ and $Y$ respectively. Denote the respective actions by $\alpha$ and $\beta$. A \emph{generalized factor map} from $\alpha$ onto $\beta$ consists of a surjective continuous map $\phi: X\rightarrow Y$ and an epimorphism $f: G\rightarrow H$ (which is continuous if $G$ and $H$ are topological) such that for every $x\in X$ and $g\in G$ \[\phi\big(\alpha(g)x\big)=\beta(f(g))\big(\phi(x)\big).\]

If there is no danger of confusion, by abusing the notation, we shall still call such (pairs of) maps factor maps.
\end{definition}

Let $G$ be a pro-countable group acting continuously on $\CC$ via an action denoted by $\alpha$ and let $\PP$ be a clopen partition of $\CC$. Let $N\trianglelefteq G$ be a clopen normal subgroup such that $\PP$ is $N$-invariant, meaning that for every $g\in N$ and $P\in\PP$ we have $\alpha(g)[P]\subseteq P$ (and thus $\alpha(g)[P]=P$).  Denote by $\phi$ the quotient map $G\to \Gamma$, where $\Gamma=G/N$ is countable discrete. We can the define a factor map $Q: \alpha\to \Gamma^\PP$ by setting for $x\in\CC$ and $gN\in\Gamma$ \[Q(x)(gN)=P\text{ if and only if }\alpha(g^{-1})x\in P.\] Since $\PP$ is $N$-invariant, this is well-defined. We leave to the reader the straightforward verification that $Q$ is continuous and that for $g\in G$ and $x\in\CC$ we have \[Q(\alpha(g)x)=gN\cdot Q(x).\] In case when $\PP$ and $N$ are not clear from the context we shall write $Q^{\PP,N}$ instead of simply $Q$. We shall also denote by $\QQ(\cdot,\PP,N)$ the corresponding continuous map from the clopen set $\{\alpha\in\Act_G(\CC)\colon \PP\text{ is }N\text{-invariant with respect to }\alpha\}$ to $\SH_{G/N}(\PP)$.

\begin{lemma}\label{lem:clopennormalsubgroup}
Let $G$ be a pro-countable group. Suppose that $G$ continuously acts on $\CC$ and let $\PP$ be a clopen partition of $\CC$. Then there is a clopen normal subgroup $N\trianglelefteq G$ such that $\PP$ is $N$-invariant.
\end{lemma}
\begin{proof}
Denote the action of $G$ on $\CC$ by $\alpha$. Pick $x\in\CC$ and let $P_X\in\PP$ be the element of the clopen partition so that $x\in P_x$. Since there is an open neighborhood basis of the unit of $G$ consisting of clopen normal subgroups, there is a clopen normal subgroup $N_x\trianglelefteq G$ and an open neighborhood $x\in U_x\subseteq P_x$ such that for every $g\in N_x$ and $y\in U_x$ we have $\alpha(g)y\in P_x$. Since $\CC=\bigcup_{x\in\CC} U_x$, by compactness there is a finite set $E\subseteq\CC$ such that $\CC=\bigcup_{x\in E} U_x$. Set \[N:=\bigcap_{x\in E} N_x.\] We claim that $N$ is as desired. $N$ is, as a finite intersection of clopen normal subgroups, a clopen normal subgroup. Pick $g\in N$ and $P\in\PP$ and let us prove that $\alpha(g)[P]\subseteq P$. Choose any $y\in P$. Then there is $x\in E$ such that $y\in U_x\subseteq P_x=P$ and moreover we have $g\in N_x$. Therefore $\alpha(g)y\in P_x=P$ and since $y\in P$ was arbitrary we get $\alpha(g)[P]\subseteq P$.
\end{proof}

\begin{proposition}
Let $G$ be a pro-countable group acting continuously on $\CC$. Then the action is an inverse limit of $(X_n)_{n\in\Nat}$, where for each $n\in\Nat$
\begin{itemize}
	\item $X_n$ is a subshift of $\Gamma_n^{A_n}$,
	\item $A_n$ is a non-trivial finite set;\,
	\item $\Gamma_n$ is a countable discrete group,
\end{itemize}
and $G$ is an inverse limit of $(\Gamma_n)_{n\in\Nat}$.
\end{proposition}
\begin{proof}
Fix $G$ and the action, which we shall denote by $\alpha$. Let $(\PP_n)_{n\in\Nat}$ be a sequence of refining clopen partitions of $\CC$ such that, with respect to some compatible metric $d$ on $\CC$, \[\lim_{n\to\infty}\max\{\mathrm{diam}_d(P)\colon P\in\PP_n\}\to 0.\] For each $n\in\Nat$ we apply Lemma~\ref{lem:clopennormalsubgroup} to get a clopen normal subgroup $N_n\trianglelefteq G$ such that $\PP_n$ is $N_n$-invariant. We may without loss of generality assume that $N_n\subseteq N_m$, for $n>m$, and that $\bigcap_{n\in\Nat} N_n=\{1\}$. For $n\in\Nat$ denote by $\Gamma_n$ the discrete group $G/N_n$ and let $Q_n$ be the factor map from $\alpha$ onto a subshift $X_n\subseteq \Gamma_n^{\PP_n}$ defined using $\PP_n$ and $N_n$. The spaces $(X_n)_{n\in\Nat}$ will be as desired, we need to define the connecting (generalized) factor maps between them.\medskip

We show that for each $n>m$ there is a canonical factor map $\omega_n^m:X_n\rightarrow X_m$. First we notice that there is an epimorphism $\Omega_n^m:\Gamma_n\rightarrow\Gamma_m$ coming from the short exact sequence \[1\to N_m/N_n\to \Gamma_n\to \Gamma_m\to 1.\] Denote the group $N_m/N_n$ simply by $N$ and we identify it with a normal subgroup of $\Gamma_n$ so that we have $\Gamma_n/N=\Gamma_m$. Second, since $\PP_n\preceq\PP_m$ there is a map $f:\PP_n\rightarrow \PP_m$ such that for $P\in\PP_n$ we have $P\subseteq f(P)$. Now we claim that for every $x\in X_n$ and every $g\in\Gamma_n$ and $h\in N$ we have \[f(x(g))=f(x(gh)).\] Indeed, pick any $z\in\CC$ such that $x=Q_n(z)$ and $g'\in G$, resp. $h'\in N_m$ such that $g'N_n=g$, resp. $h'N_n=h$. The equality $f(x(g))=f(x(gh))$ is, by definitions of $Q_n$ and $f$, equivalent with the statement that $\alpha((g')^{-1})z$ and $\alpha((g'h')^{-1})z$ lie in the same element of $\PP_m$. However, since $(g'h')(g')^{-1}\in N_m$ and $\PP_m$ is $N_m$-invariant, this statement follows.

Consequently, we can safely define $\omega_n^m:X_n\rightarrow X_m$ by setting for $x\in X_n$ and $gN\in\Gamma_m$, where $g\in\Gamma_n$ \[\omega_n^m(x)(gN)=f(x(g)).\]
It is easy to verify that this map moreover satisfies \[\forall z\in\CC\;\forall g\in\Gamma_m\; \big( Q_m(z)(g)=\omega_n^m\circ Q_m(z)(g).\big)\]
From the preceding formula one also gets that for every $n>m>k$ we have \[\omega_n^k=\omega_m^k\circ\omega_n^m.\] We leave to the reader the verification that $\alpha$ is the inverse limit of $(X_n)_{n\in\Nat}$ with respect to the system of connecting maps $(\omega_n^m)_{n>m\in\Nat}$.
\end{proof}

We extend the notion of a projectively isolated subshift so that it incorporates the notion of a generalized factor map.
\begin{definition}
Let $\Gamma$ be a countable group and $A$ be a non-trivial finite set. We say that a subshift $X\subseteq A^\Gamma$ is \emph{projectively isolated with respect to an epimorphism $f:\Lambda\rightarrow\Gamma$}, where $\Lambda$ is a countable group, if there exist a subshift $Y\subseteq B^\Lambda$, for some non-trivial finite set $B$, an open neighborhood $\NN_Y^F$, for some finite subset $F\subseteq \Lambda$, and a map $\phi_0:Y_F\rightarrow A$ such that for every $Z\in\NN_Y^F$ the map $\phi_0$ defines a generalized factor map $\phi_Z: Z\rightarrow X$ onto $X$ together with the epimorphism $f$.
\end{definition}

\begin{lemma}
Let $\Gamma$ and $\Gamma'$ be countable groups such that there is an epimorphism $\omega:\Gamma'\rightarrow\Gamma$ with finite (or more generally finitely generated) kernel $H$. Let $A$ be a non-trivial finite set and define a subshift (of finite type) \[X:=\{x\in A^{\Gamma'}\colon \forall g\in\Gamma'\;\forall h\in H\; (x(g)=x(gh))\}.\] The set of subshifts of $X$ is an open subset in $\SH_{\Gamma'}(A)$ and there is a canonical inclusion-preserving bijection $\Psi$ between subshifts of $X$ and subshifts of $A^\Gamma$ with $\Phi(X)=A^\Gamma$. Moreover, $\Phi$ is continuous. In particular, for every open subset $\NN\subseteq \SH_\Gamma(A)$, $\Phi^{-1}(\NN)$ is open in $\SH_{\Gamma'}(A)$.
\end{lemma}
\begin{proof}
Fix $\Gamma$, $\Gamma'$ and $A$ as in the statement. First we notice that subshifts of $X$ are an open set in $\SH_{\Gamma'}(A)$. For simplicity, assume that $H$ is finite since we shall later use only this case; however, the argument also works with a small change for $H$ finitely generated. We claim that the open neighborhood $\NN_X^H$ contains precisely subshifts of $X$. This follows from the observations that $\NN_X^H$ consists precisely of subshifts whose elements are constant on cosets of $H$, and all such subshifts are subshifts of $X$.

The existence and definition of $\Phi$ are obvious, we only show the continuity. Let $\NN\subseteq\SH_\Gamma(A)$ be open; we may suppose it is of the form $\NN_Y^F$ for some subshift $Y\subseteq A^\Gamma$ and a finite set $F\subseteq \Gamma$. Set $Y':=\Phi^{-1}(Y)$ and $F':=\omega^{-1}(F)\subseteq \Gamma'$, which is a finite subset. Then it is easy to check that \[\Phi^{-1}(\NN_Y^F)=\NN_{Y'}^{F'}\subseteq\SH_{\Gamma'}(A).\] The `In particular' part then follows.
\end{proof}
\begin{corollary}\label{cor:projisolatedwrepimorphisms}
Let $\omega_1:\Gamma''\rightarrow\Gamma'$ and $\omega_2: \Gamma'\rightarrow\Gamma$ be two epimorphisms between countable groups, where $\omega_1$ has a finite (or finitely generated) kernel. Suppose that $X\subseteq A^\Gamma$, for some non-trivial finite $A$, is projectively isolated with respect to $\omega_2$. Then it is also projectively isolated with respect to $\omega_2\circ\omega_1$.
\end{corollary}

\begin{theorem}
Let $G$ be a locally compact pro-countable second-countable group. Then $G$ has the strong topological Rokhlin property if and only if for some, equivalently for every, inverse sequence of countable groups $(\Gamma_n)_{n\in\Nat}$ with connecting epimorphisms $(\omega_n^m:\Gamma_n\rightarrow \Gamma_m)_{n>m}$ whose inverse limit is $G$ we have that for every $n\in\Nat$, non-trivial finite set $A$, and an open set $\UU\subseteq \SH_{\Gamma_n}(A)$ there are $m>n$ and a subshift $X\in\NN$ which is projectively isolated with respect to the epimorphism $\omega_m^n$.
\end{theorem}

\begin{proof}
Let $G$ be a locally compact pro-countable Polish group. Suppose first that the right-hand side condition is not satisfied and fix an inverse sequence of countable groups $(\Gamma_n)_{n\in\Nat}$ with connecting epimorphisms $(\omega_n^m:\Gamma_n\rightarrow \Gamma_m)_{n>m}$ whose inverse limit is $G$. Denote by $(N_n)_{n\in\Nat}$ the decreasing sequence of compact clopen normal subgroups with trivial interesection such that $G/N_n=\Gamma_n$ for every $n\in\Nat$. By assumption, there are $n\in\Nat$,  a non-trivial finite set $A$, and an open set $\UU\subseteq \SH_{\Gamma_n}(A)$ that contains no subshift that is projectively isolated with respect to $\omega_m^n$, for any $m>n$.

For every $X\in\UU$ define the set \[\begin{split}\AAA(X):=\{& \alpha\in\Act_G(\CC)\colon \forall \PP\text{ a clopen partition }\\ & \PP\text{ is not }N_n\text{-invariant with respect to }\alpha\text{ or }Q(\alpha,\PP,N_n)\neq X\}.\end{split}\] We claim that $A(X)$ is dense $G_\delta$. In order to prove that, it suffices to show that fixing a clopen partition $\PP$ and denoting by $\Act_G^{\PP,N_n}(\CC)$ the clopen set \[\{\alpha\in\Act_G(\CC)\colon \PP\text{ is }N_n\text{-invariant with respect to }\alpha\},\] the set \[\AAA(X,\PP):=\{\alpha\in\Act_G^{\PP,N_n}(\CC)\colon \QQ(\alpha,\PP,N_n)\neq X\}\] is open and dense in $\Act_G^{\PP,N_n}(\CC)$. Indeed, then we have \[\AAA(X)=\bigcap_{\PP\text{ clopen partition}} \Big(\AAA(X,\PP)\cup \big(\Act_G(\CC)\setminus\Act_G^{\PP,N_n}(\CC)\big)\Big),\] and so $\AAA(X)$ is dense $G_\delta$ by the Baire category theorem and since there are only countably many clopen partitions.

By continuity of $\QQ(\cdot,\PP,N_n)$, $\AAA(X,\PP)$ is open, we need to check that it is dense. Fix an open subset $\NN\subseteq \Act_G^{\PP,N_n}(\CC)$. Without loss of generality, we may assume that $\NN$ is of the form $\NN_\beta^{K,\PP'}$, where $\beta\in\Act_G^{\PP,N_n}$, $\PP'\preceq\PP$ and $K\subseteq G$ is a compact subset containing the clopen compact normal subgroup $N_n$. Applying Lemma~\ref{lem:clopennormalsubgroup}, we can find $m>n$ so that $\PP'$ is $N_m$-invariant. We may also assume that $K=F\cdot N_m$, where $F\subseteq G$ is a finite subset. We can now proceed as in the proof of Proposition~\ref{prop:noRokhlin}, so we only sketch how to continue. As in the proof of Proposition~\ref{prop:noRokhlin} we define a subshift of finite type $Y\subseteq (\PP')^{\Gamma_m}$ whose defining window is $F/N_m\subseteq \Gamma_m$ and its basic open neighborhood $\NN_Y^{F/N_m}$ so that for every $Y'\in\NN_Y^{F/N_m}$ we have $Y'\subseteq Y$. The inclusion map $\phi:\PP'\rightarrow \PP$ defined so that $P\subseteq\phi(P)$, for every $P\in\PP"$, induces a generalized factor map $\phi$ from $Y$ into $A^{\Gamma_n}$. Since $X$ is not projectively isolated with respect to $\omega_m^n$ there exists $Y'\in \NN_Y^{F/N_m}$ so that $\phi[Y']\neq X$. Then as in the proof of Proposition~\ref{prop:noRokhlin} we can define $\beta\in\Act_G(\CC)$ so that
\begin{itemize}
	\item $\beta\in\NN$;
	\item $\QQ(\beta,\PP,N_n)=\phi[Y']\neq X$.
\end{itemize}
 This finishes the proof of the density.

Now we conclude as in the proof of Proposition~\ref{prop:noRokhlin}. Set $k:=|A|$ and let $\PP$ be an arbitrary partition of $\CC$ into $k$-many non-empty clopen sets. The set \[\VV:=\{\beta\in\Act_G^{\PP,N_n}\colon \QQ(\beta,\PP,N_n)\in\UU\}\] is an open subset of $\Act_G(\CC)$. Therefore if there were a generic action of $G$, let us denote it by $\alpha$, some of its conjugates would lie in $\VV$. Without loss of generality, assume $\alpha\in\VV$ and set $X:=\QQ(\alpha,\PP,N_n)\subseteq \UU$. Then the conjugacy class of $\alpha$ intersects the dense $G_\delta$ set $\AAA(X)$ and we reach a contradiction as at the end of the proof of Proposition~\ref{prop:noRokhlin}.\bigskip

We now prove the other direction. We fix a sequence of countable groups $(\Gamma_n)_{n\in\Nat}$ such that $G$ is an inverse limit of this sequence. As before, we shall denote, for $n>m\in\Nat$, by $\omega_n^m:\Gamma_n\rightarrow \Gamma_m$ the connecting epimorphisms, and by $\Omega^n:G\rightarrow \Gamma_n$ the epimorphism from the inverse limit. We also denote by $N_n$ the kernel of $\Omega_n$ which is a clopen normal subgroup of $G$ and we may moreover, without loss of generality, assume that it is compact - as $G$ is locally compact and the groups $(N_n)_{n\in\Nat}$ form its neighborhood basis at the unit. The proof proceeds as the proof of Proposition~\ref{prop:MAINexistence}, so we satisfy ourselves just with a sketch. In the proof of Proposition~\ref{prop:MAINexistence} we used the Fra\"iss\' e theory implicitly; here in order to reduce the arguments we use it explicitly and assume the reader is familiar with its basics (we refer to \cite{Kub} for an introduction to a general category-theoretic framework).

Consider now the class $\DD$ of all subshifts $X\subseteq A^{\Gamma_n}$, where
\begin{itemize}
	\item $A$ is a non-trivial finite set;
	\item $n\in\Nat$, thus $\Gamma_n$ is an approximation of $G$ from the fixed inverse sequence;
	\item $X$ is a subshift that is projectively isolated with respect to the epimorphism $\omega_m^n:\Gamma_m\rightarrow\Gamma_n$, for some $m>n$.
\end{itemize}
Below we prove a claim, analogous to Claims 1 and 2 from the proof of Proposition~\ref{prop:MAINexistence}, showing that $\DD$ has the joint embedding and amalgamation properties.\medskip

\noindent{\bf Claim} \emph{$\DD$ has the joint embedding/projection  and amalgamation properties.}\medskip

Let $X,Y\in\DD$ and let $X'\in \NN_X\subseteq \SH_{\Gamma_n}(A)$, resp. $Y'\in \NN_Y\subseteq\SH_{\Gamma_m}(B)$ be projectively isolated subshifts with respect to epimorphisms $\omega_k^n$, resp. $\omega_l^m$. By Corollary~\ref{cor:projisolatedwrepimorphisms} we may assume that $n=m$ and $k=l$. Now we can proceed exactly as in Claim 1 from the proof of Proposition~\ref{prop:MAINexistence} to form a product subshift $X'\times Y'\subseteq \Gamma_n^{A\times B}$ and to find a projectively isolated subshift $Z$ with respect to some epimorphism in the neighborhood of $X'\times Y'$ that factors onto both $X$ and $Y$.

The amalgamation is proved analogously as in Claim 2 of the proof of Proposition~\ref{prop:MAINexistence} with the help of Corollary~\ref{cor:projisolatedwrepimorphisms} as in the proof of the joint embedding/projection above.
\bigskip

By the general Fra\" iss\' e theory, $\DD$ has a limit which is a generic inverse limit of systems $(X_n)_{n\in\Nat}$ from $\DD$.  Denote for each $k>l$ by $\phi_k^l:X_k\rightarrow X_l$ the connecting map which is a generalized factor map. For each $n\in\Nat$, $X_n\subseteq A_n^{\Gamma_{m_n}}$ for some $m_n\in\Nat$ such that the sequence $(m_n)_{n\in\Nat}$ is, without loss of generality, strictly increasing, and $A_n$ is a non-trivial finite set. The inverse limit is thus a continuous action of $\varprojlim_{n\to\infty} \Gamma_{m_n}$, which is equal to $G$, on a compact metrizable zero-dimensional space which is an inverse limit of $X_n$'s as topological spaces. This is verified to have no isolated points as in Lemma~\ref{lem:inverseisCantor}. Therefore we get a continuous action of $G$ on the Cantor space. 

In order to simplify the notation, we shall assume that $n_m=m$ for every $m\in\Nat$. This generic inverse limit also satisfies the following two properties that uniquely characterize it up to conjugacy- analogously as the generic inverse limit from the proof of Proposition~\ref{prop:MAINexistence} does:

\begin{equation}\label{eq:firstsimplified}
\begin{split}
& \forall n\geq 2\; \forall m,m'\in\Nat\;\forall Y\in\SH_{\Gamma_m}(n)\text{ projectively isolated with respect to }\omega_{m'}^m\\ & \exists m''\geq m'\;\exists \psi_{m''}^Y: X_{m''}\twoheadrightarrow Y
\end{split}
\end{equation}

\begin{equation}\label{eq:secondsimplified}
\begin{split}
& \forall m,m',m''\in\Nat\;\forall n\geq 2\;\forall Y\in\SH_{\Gamma'}(n)\text{ projectively isolated with respect to }\omega_{m''}^{m'}\\
& \forall\forall \psi_Y^m: Y\twoheadrightarrow X_m\;\exists k>m\;\exists\psi_k^Y: X_k\twoheadrightarrow Y\; (\psi_Y^m\circ \psi_k^Y=\phi_k^m)
\end{split}
\end{equation}

One then checks, as in Fact~\ref{fact:mainthm1} using $\GG_1$, that the set of those actions $\beta\in\Act_G(\CC)$ that are inverse limits of subshifts over the groups $(\Gamma_n)_{n\in\Nat}$ that projectively isolated with respect to epimorphisms of the type $\omega_m^n$ is a $G_\delta$ set. The set of such inverse limits satisfying \eqref{eq:firstsimplified}, resp. \eqref{eq:secondsimplified} is shown to be $G_\delta$ as in Fact~\ref{fact:mainthm2} using $\GG_2\cap\GG_1$, resp. as in Fact~\ref{fact:mainthm3} using $\GG_3\cap\GG_1$.

The density in $\Act_G(\CC)$ of the conjugacy class of the generic inverse limit is shown similarly as in Proposition~\ref{prop:densityofgenericaction}.
\end{proof}
\section{Remarks, problems, and questions}
It is our hope that this paper will stimulate further research in this area, even among researchers working in symbolic dynamics over general groups, and more applications of Theorem~\ref{thm:mainRokhlin} will be found. The following is the most general problem that we state and believe it is worth of attention.
\begin{problem}
Apply Theorem~\ref{thm:mainRokhlin} to a wider class of groups. That is, find more groups for which (strongly) projectively isolated subshifts are dense in the spaces of subshifts.
\end{problem}
We know very little about the permanence properties of the class of countable groups satisfying the STRP. We even do not know whether every virtually cyclic group satisfies the STRP. Inspired by the results from \cite{Cohen} we ask the following.
\begin{question}
Is the class of countable groups satisfying the STRP closed under commensurability? Under virtual isomorphism? Under quasi-isometry?
\end{question}
The following is based on the fact we do not know any projectively isolated subshift that is not strongly projectively isolated.
\begin{question}
Do there exist a group $G$ and a subshift $X\subseteq A^G$, for some non-trivial finite $A$, such that $X$ is projectively isolated, however not strongly projectively isolated?
\end{question}
We conclude with a question related to the results of Section~\ref{sect:noSTRP}.
\begin{question}
Does there exist a countable group that is not finitely generated, yet it still has the STRP?
\end{question}
\bigskip

\noindent{\bf Acknowledgements.} We would like to thank to Sebasti\' an Barbieri for explaining us the different notions of effective subshifts over general countable groups and to Alexander Kechris and Steve Alpern for several comments.
\bibliographystyle{siam}
\bibliography{references-Rokhlin}
\end{document}