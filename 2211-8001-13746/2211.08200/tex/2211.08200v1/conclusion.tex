\section{CONCLUSION}
\label{sec:conclusion}

In this paper, 
% we explore a novel problem of inferring the users' economic statuses with their mobility records generated with their cars. W
we study user socioeconomic status inference, and propose a novel socioeconomic-aware deep model called \texttt{DeepSEI} for the task. \texttt{DeepSEI} incorporates two neural networks, i.e., deep network and recurrent network. 
%
% In the model framework, we manage to utilize the features for the two networks from four points of views, 
We extract features from three aspects of users' mobility records, namely spatiality, temporality and activity, at the coarse and detailed levels, and feed them to the two networks.
%
We conduct the experiments on Geolife dataset, POI dataset and house price dataset, and the results show that the \texttt{DeepSEI} model achieves a noticeable improvement over the baselines. 
%
In the future, we plan to explore more mining and learning tasks based on users' mobility records, such as anomaly detection.

\smallskip
\noindent\textbf{Acknowledgments:}
This research/project is supported by the National Research Foundation, Singapore under its AI Singapore Programme (AISG Award No: AISG-PhD/2021-08-024[T]).
%
This research is also supported by the Ministry of Education, Singapore, under its Academic Research Fund (Tier 2 Awards MOE-T2EP20220-0011 and MOE-T2EP20221-0013). 
This research is also supported in part by the China-Singapore International Joint Research Institute (CSIJRI), Guangzhou, China (Award No. 206-A021002).
Any opinions, findings and conclusions or recommendations expressed in this material are those of the author(s) and do not reflect the views of National Research Foundation, Singapore and Ministry of Education, Singapore. This project is partially supported by HKU-SCF FinTech Academy and also Shenzhen Science and Technology Innovation Committee (SZ-HK-Macau Technology Research Programme, \#SGDX20210823103537030).

\if 0
\noindent \textbf{User Profiling.} We will investigate the possibility of inferring some of users’ profile attributes such as 
% home and work locations, 
income level, marital status, etc. from the mobility records data. 
% While it looks to be an easy task to infer some of the attributes such as home location, e.g., the location of the records during mid-night is likely to be the home location, it is not trivial for other types attributes. 
With some attributes of users provided as ground-truths, this project will train some supervised learning models such as neural networks and use the trained models for inferring these attributes for those users, which are not available/missing/outdated.

\noindent \textbf{Abnormal Driving Behavior Detection.} We will perform driving behavior anomaly detection. Specifically, we will develop techniques that could detect if the driving behavior of a user (places to visit, frequency of travel, etc.) has changed, i.e., it is different from expected. This would help to better prepare for potential changes of users’ loan payment behaviors or provide evidence that some special attention should be paid to the abnormal users and investigate possible causes of the changes, e.g., the driver of a car has changed from a customer to his/her family member, etc. We will explore some commonly used anomaly detection methods such as those deviation-based models~\cite{aggarwal2017introduction} for this task.
\fi

