\section{INTRODUCTION}
\label{sec:introduction}

With the rapid development of GPS devices and mobile technologies, recent years have witnessed an unprecedented growth in mobility data. This big amount of data has attracted many research efforts to acquire knowledge of human mobility behaviors. More specifically, extensive studies have been conducted on profiling users from mobility records. For example, it has been explored to infer users' demographic attributes from their check-ins~\cite{zhong2015you}, users' ethics and gender from their photo sharing data with geo tags~\cite{riederer2020location}, passengers' employment statuses from their smart card data~\cite{zhang2019deep, ding2019estimating}, and users' demographics from their trajectories~\cite{wu2019inferring}, etc.
% , and users' socioeconomic status from their trajectories~\cite{xu2018human}, etc. 
%
While these techniques are extensive and have some merits, there still exist some scenarios that have been overlooked and/or cannot be adequately solved by them. 
For example, in
% some FinTech applications 
some real-life applications such as car loans, 
% the mobility records that are available are not those based on smart phones as most existing studies assume, but those based on GPSs installed on their cars. In these scenarios, only when users have a movement, the mobility records reflect those of the users. 
% For these applications, 
quite many demographic attributes such as the age and gender of the users are already provided by users.
What is demanded for these applications is to infer the socioeconomic statuses of users, e.g., the prices of their living houses and whether they will pay their monthly loans on time, etc.
Yet these have been mostly overlooked by existing studies~\cite{zhong2015you,riederer2020location,zhang2019deep,ding2019estimating,wu2019inferring}.

In this paper, we aim to infer users' socioeconomic statuses from their mobility records.
%
This is motivated by two considerations. 
First, users' socioeconomic statuses are closely linked to where they live or work, both of which could be potentially reflected by their mobility records. Second, users' socioeconomic statuses can sometimes be disclosed by the places they visit, especially those they visit during weekends, and the patterns of their visits at these places, which again could be revealed by their mobility records.
%
Here, a user's socioeconomic status can refer to many different indicators, such as the price range of the user's living house~\cite{xu2018human,ding2019estimating}, the likelihood that the user will pay a car loan installment on time, or the user's income, etc. Constrained by the availability of datasets and privacy concerns, in this paper, we infer the home location of a user based on his/her mobility records (i.e., Geolife) and then crawl the house price data from the Web based on the home location as the proxy of the user's socioeconomic status. 
Since both the mobility records data and the house price data are publicly available, no privacy will be broken in this study.


% To this end, this paper will investigate techniques for (1) taking the GPS mobility records as inputs, (2) incorporating socioeconomic contexts for learning user socioeconomic status, (3) extracting various types of features from the inputted mobility records data and context data, and (4) inferring the users' socioeconomic statuses. Each of these elements makes it distinctive from existing studies.
%
Specifically, we propose a socioeconomic-aware deep model called \texttt{DeepSEI} for user socioeconomic status inference. 
In \texttt{DeepSEI}, it first preprocesses the users' mobility records data by filtering the noises, extracting the stay points, and inferring the activities behind the extracted stay points.
Then, it incorporates two networks, namely \emph{deep network} and \emph{recurrent network}, to capture users' activities data at a coarse level and at a detailed level, respectively, for this task. 
The deep network aims to capture some statistics based on users' mobility records (i.e., at a coarse level) and the recurrent network aims to capture the sequential patterns behind users' mobility records (i.e., at a detailed level).

The deep network takes as inputs three features of users' mobility records data, including spatiality diversity, temporality diversity and activity diversity. 
% The rationale of designing the features is that (1) 
Spatiality diversity captures the spatial information in the territory where users' daily activities are conducted. 
Temporality diversity captures the temporal regularity of users, which can potentially help to indicate their professions, e.g., self-employers tend to stay at home and only go out occasionally, while some users working at a government department would commute more regularly. 
Activity diversity reveals the diversity of movements among users' activity locations, which can reflect users' socioeconomic statuses as shown in~\cite{xu2018human,wu2019inferring}.

% The rationale of designing the features is that (1) spatial and (2) temporal features are two fundamental features to 

The recurrent network takes as inputs the sequences of activities of users, where each activity has spatial, temporal and semantic features. The spatial and temporal features indicate where and when the activities are conducted and the semantic features, e.g., working or shopping, indicate the activity types and provide the context for understanding users' daily routines.
%
% In the recurrent network, 
We adopt a hierarchical LSTM with two levels for the recurrent network.
% to capture the users' sequential activities, where each activity involves spatial, temporal and semantic features. 
% For the hierarchical LSTM, 
The activities within a day are modeled in the low-level LSTM and the activities within days are modeled in the high-level LSTM. The hierarchical LSTM brings two advantages. First, users' sequences of mobility records are generally long, and the hierarchical structure can reduce the length and alleviate the issue of degraded performance for handling long sequences. Second, users' mobility records are organized on a daily basis, and the two-level LSTMs preserve the users' periodic information. 
{\newadd{We note that other sequence encoder models, e.g., Transformer, are also applicable for the task, and we leave it a future work to explore these models.}}
%

% We train our model in a supervised manner by concatenating the outputs of the two networks, {\newadd{and training with the collected socioeconomic contexts (e.g., house price) as labels to infer their different socioeconomic classes by following many studies~\cite{xu2018human,ding2019estimating}}}.


% we propose an economic-aware deep model named \texttt{DeepLPP} for loan payment prediction. The \texttt{DeepLPP} model involves three components. The rationale behind the model is that we first preprocess the raw datasets they usually involve noises and extract meaningful semantics from mobility data, e.g., the stay points, which are potentially associated with some daily activities of a user; economic contexts such as house prices collected via web crawlers. Then, we prepare features that could be potentially used for the mining and learning tasks, e.g., loan payment prediction task targeted in this paper. Finally, we conduct the task based on two neural networks, namely \emph{Deep Network} and \emph{Recurrent Network}, which take extracted features as inputs and output an anomaly score between 0.0 and 1.0. {\CommentZheng{The rationale of designing the two networks is that}}
% %
% We introduce the features we explored in this paper, they are used to describe users' daily lives in four aspects, i.e., \emph{mobility}, \emph{temporality}, \emph{activity} and \emph{economy}, which are captured via \emph{Deep Network} and \emph{Recurrent Network}, respectively.

% In the deep network, we involve the following indicators for the task and capture them via a feedforward neural network, the indicators are (1) home and work location, (2) temporality diversity, (3) activity diversity and (4) monthly loan payment. {\CommentZheng{The rationale is that}}
% %
% Among these indicators, home and work locations are two important spatial locations where users stay the longest and commute regularly; temporality diversity and activity diversity are two entropy-based indicators, which are widely used to analyze human mobility patterns as suggested in the existing studies~\cite{xu2018human,pappalardo2015returners,song2010modelling}; monthly loan payment is an economic indicator that is associated with the task. Overall, these indicators capture the how regularly, broadly, frequently, and intensively a user would travel within a city.

% In the recurrent network, we tackle long sequences of mobility records and preserve users' periodic information with a hierarchical LSTM. The hierarchical LSTM is implemented with two levels, where the low level captures mobility records within one day and the high level is  for handling the records between days. We extract the features from mobility records and feed embedded features (i.e., vectors after an embedding layer) into the hierarchical LSTM, including (1) mobility embedding, (2) temporality embedding, (3) activity sequence embedding and (4) economic-aware region embedding. {\CommentZheng{The rationale is that}} 
% %
% Here, mobility and temporality embeddings are used to capture the spatial and temporal characteristics of users' daily travel; activity patterns are informative to users' economic statuses as studied in the human mobility research~\cite{de2013unique,xu2018human}; additionally, we introduce a new feature based on economic-aware regions in which users' mobility records are located, the regions are incorporated with economic contexts (i.e., house prices), and learned as vectors used for the model.

The novelty of the paper is two-fold. First, the problem setting is new. To our best knowledge, there are no existing studies that take GPS mobility records as inputs and infer users’ socioeconomic statuses. %(such as the prices of their living houses). 
Second, our method is distinctive from existing ones in that it explores a data-driven solution with two well-designed neural networks for the inference task.
% , which corresponds to the first of its kind.
%
% This paper will contribute significantly to the scientific literature with new techniques for inferring users’ socioeconomic statuses from their mobility data. 
% The results of this paper could be immediately applicable to the FinTech industry for assessing the user risk for various economic activities.
% The techniques developed in this paper can potentially be used in some real-life applications, e.g., assessing the user risk for various socioeconomic activities such as applying for car loans.
% , monitoring customers’ driving behaviors (and alerting for further investigation if necessary, e.g., when the driver of a car is changed from one person to another). %
In summary, we make the following contributions: 
\begin{itemize}[leftmargin=9pt]
    \item We study a novel problem of 
    inferring users' socioeconomic statuses based on their GPS mobility records. This problem is new and has practical applications in real life (e.g., risk assessment for car loan applications/managements).
    % for many FinTech industries.
    % of WeBank~\footnote{\url{https://www.webank.com}}
    % This problem is investigated based on a new data source and differs from the mobile phone data used in the previous studies. 
    % An improved understanding of the relationship between the users' mobility and economic statuses can benefit many real-world tasks, including loan payment prediction, user profiling and abnormal driving behavior detection. We study the first task in this paper, and latter two tasks are discussed in future work.
    \item We propose a novel learning framework called \texttt{DeepSEI} for the problem, which is a supervised deep learning model and incorporates two neural networks (i.e., deep network and recurrent network) to capture the features from three aspects of users' mobility records, i.e., spatiality, temporality and activity, at both coarse and detailed levels.
    
    \item We conduct the experiments on real-world GPS trajectory, POI and house price datasets, which are publicly available, and the results demonstrate our method's superior performance for the task, e.g., our method outperforms the best baseline by at least 15\% in terms of prediction accuracy. 
    % In addition, we have tested our model on real FinTech business, it achieved satisfactory results.
\end{itemize}

%The rest of paper is organized as follows. We review the literature in Section~\ref{sec:related}. We give the problem description in Section~\ref{sec:preliminary}. We present our \texttt{DeepSEI} model in Section~\ref{sec:method}, and report our experimental results in Section~\ref{sec:experiment}. We conclude this paper in Section~\ref{sec:conclusion}.
