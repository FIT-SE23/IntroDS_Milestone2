\section{RELATED WORK}
\label{sec:related}

% We review the literature in two aspects that are related to our study, i.e., (1) 
% % inferring socioeconomic status from human mobility data, 
% human mobility data analytics,
% and (2) mobility and temporality prediction.

\subsection{Human Mobility Analytics}
% Our work is related to the studies on inferring socioeconomic status from users' mobility records.
Earlier studies~\cite{hanson1981travel,kwan1999gender,limtanakool2006influence} examine the relationships between human travel behaviors with their profiles, including gender~\cite{kwan1999gender,limtanakool2006influence}, age~\cite{limtanakool2006influence,kwan1999gender}, race~\cite{kwan1999gender}, employment status and income~\cite{hanson1981travel}. For example, Hanson et al.~\cite{hanson1981travel} suggest that an individual's employment status and income have a positive impact on his/her travel frequency. Kwan et al.~\cite{kwan1999gender} reveal that men are more inclined to visit recreation places than women. The earlier studies make a great impact on the subsequent research; however, the works are conducted on travel survey data, which needs to be collected manually, and consequently the findings are mainly based on a small number of volunteers for a short period of time.

With the proliferation of communication techniques, mobile phone data becomes a new data source for conducting this type of research. Mobile phone data is collected from cellphone users, where the mobile phone records track a user's id, the user's location when he/she makes a phone call (the location is reported as the longitude and latitude), and the timestamp at which the phone call starts. With the mobile phone data, Xu et al.~\cite{xu2015understanding} present a home-based approach to analyze
human activities in Shenzhen. They find that people who live in the northern part of Shenzhen are generally with a small activity space around their homes, and people with a larger activity space mainly live in the southern part, where the economy is highly developed. Further, they propose an analytical framework for understanding the relationships between human mobility and socioeconomic status~\cite{xu2018human}. Specifically, they take two cities, Singapore and Boston, for case studies, and reveal an interesting finding that the richer tend to travel shorter in Singapore but longer in Boston. 
Blumenstock et al.~\cite{blumenstock2015predicting} predict individual socioeconomic status (e.g., poverty and wealth levels) with users' survey data collected from their historical mobile calls. 
%
In addition, Huang et al.~\cite{huang2016activity} explore possible factors that may influence individual daily activities, and the results demonstrate that socioeconomic status, urban spatial structure, work place and region geographical layout all play a critical role. Kelly et al.~\cite{kelly2013uncovering} use the location data collected from mobile sensors to identify some predictability patterns that can be linked to users' demographics such as age, gender and social meeting contacts, etc.
{\newadd{Ding et al.~\cite{ding2019estimating} estimate users’ socioeconomic statuses from their subway smart card data, which records users' pick-up and drop-off locations at subway stations for each trip.}}
%
Different from these studies, we propose to infer users' socioeconomic statuses with their GPS trajectories and develop a deep learning based model called \texttt{DeepSEI}.
% , which incorporates two neural networks, namely Deep Network and Recurrent Network, to capture the features covering three aspects of users' daily lives, namely spatiality, temporality and activity. 
% To the best of our knowledge, this is the first of its kind.

\subsection{Mobility and/or Temporality Prediction}
We review the existing studies regarding the prediction task, where mobility and temporality information is involved. For mobility prediction, with the proliferation of location-based services, it has been a hot research topic in recent years. Mobility prediction aims at predicting the next location for the user, while POI recommendation aims to predict the following several locations that the user will visit. 
Earlier studies~\cite{monreale2009wherenext, zhang2014splitter} for the next location prediction task are based on exacting the historical user mobility patterns. In recent years, many learning-based methods~\cite{chen2020context,ju2020interaction,feng2018deepmove} are proposed to model the users' mobility patterns in a data-driven manner. For example, DeepMove~\cite{feng2018deepmove} is an attentional recurrent neural network based model to capture the user's periodical patterns for his/her mobility prediction.
% , which is effective with two attention mechanisms. 
% One is to attend historical records with the users' current statuses, and the other one is to preserve the sequential information among historical records with the recurrent module.
%
%In addition, Wang et al.~\cite{wang2015regularity} propose a hybrid predictive model for mobility prediction, which incorporates both the regularity and conformity of human mobility into the model.
%
Chen et al.~\cite{chen2020context} propose a context-aware deep model called DeepJMT to jointly predict where and when a user will visit next, which considers both users' visit histories and the spatial and user contexts of the visits.
% and thus DeepJMT takes both user’s spatial and temporal patterns into the prediction. 
%~\cite{chiu2013stochastic}
For temporality prediction, many existing studies~\cite{xiao2017modeling,du2016recurrent} adopt temporal point process to model the time as a sequence of discrete random events, and then jointly predict the next event type and timestamp. %Du et al.~\cite{du2016recurrent} apply RNN to learn the representation of influences from the event history for prediction, and then develop an efficient stochastic gradient algorithm to speed up model training on millions of data points. Further, Xiao et al.~\cite{xiao2017modeling} synergically model the time series and event sequence for the prediction, where time series and event sequence are fed into two RNNs, and then the outputs of the two RNNs are concatenated to an embedding mapping layer to predict the next event type and timestamp. In addition, Yan et al.~\cite{yan2018improving} propose to improve the temporal point process with discriminative and adversarial learning, which aims to generate the temporal sequences that are close to the real distribution.
%
Our problem differs from these studies mainly in that we aim to predict users' socioeconomic statuses but not their mobility and/or temporality. 
% However, these studies usually develop techniques for predicting some spatial or temporal information of a moving object, e.g., what the next location is; when the next event will happen.

\if 0
\subsection{Spatial Outlier Detection}
Our task is related to an outlier detection task, i.e., it aims to detect overdue loan payment behaviors using the mobility records generated with the loaned vehicles. 

Spatial outlier detection task~\cite{chen2008detecting} aims to discover some objects with multiple attributes, whose non-spatial attributes are significantly different compared to their spatial neighbors. Classical methods for spatial outlier detection include density-based method~\cite{breunig2000lof}, kernel-based method~\cite{aggarwal2001outlier} and deviation-based method~\cite{aggarwal2017introduction}, where the common idea is to identify the data samples (called anomalies) that are significantly from others. In particular, Chen et al.~\cite{chen2008detecting} formulate the spatial outlier detection problem and develop a KNN-based method with Mahalanobis distance for the task. Further, Cai et al.~\cite{cai2013spatial} investigate a self-organizing map (SOM) method for the task, and achieve more effective results for high dimensional data. In addition, Kou et al.~\cite{kou2007spatial} and Liu et al.~\cite{liu2010spatial} explore the graph-based method for the task, where a graph is built based on the spatial attribute, and different techniques such as random walk~\cite{liu2010spatial} are developed for detecting the anomalies on the graph. In recent years, Zheng et al.~\cite{zheng2017contextual} further incorporate additional contextual attributes to the spatial objects, and apply metric learning for outlier detection.

In summary, the outlier definition in existing studies is quite different with ours. In our work, we detect anomalous loan payment behavior in a supervised manner, where we consider users' profile attributes from four aspects, i.e, mobility, temporality, activity and economy. 
\fi
