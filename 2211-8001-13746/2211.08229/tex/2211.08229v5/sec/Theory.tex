\newpage
~
\newpage
\begin{figure}[!t]
    \centering
    \includegraphics[width=0.9\textwidth]{figs/v2.pdf}
    \caption{Visual illustrations of (a) all possible $V_2$ that contain the trigger $e$. (b) $\Delta w$ and $\Delta h$ for left-right layout.}
    \label{v2}
\end{figure}

\section{Proof of Theorem 1}
\label{theorem1}
For simplicity, we prove the optimal locations of the reference object and trigger for left-right layout. The proof for bottom-top layout is similar. 

\noindent
\textbf{Computing $p_1(s)$ and $p_2(s)$}Given arbitrary $s \in (0, S]$, we aim to explicitly express the probabilities of $p_1(s)$ and $p_2(s)$.  
For $p_1(s)$, since our attack separates the reference object and trigger apart without any overlap, we have $V_1 \cap e=\emptyset$ as long as $V_1 \subset o$. Therefore, we have: 
$$p_1(s) = \text{Pr}\{(V_1 \subset o) \cap (V_1 \cap e=\emptyset)\} = \text{Pr}\{V_1 \subset o\}$$ 
Then, $p_1(s)$ can be computed as the ratio between the area of upper-left corners of $V_1$ such that $V_1 \subset o$ and that of all possible $V_1 \subset b$:
\begin{equation}
\begin{split}
p_1(s) &= \text{Pr}\{V_1 \subset o\} \\ 
&= 
\begin{cases}
\frac{(o_w-s)(o_h-s)}{(b_w-s)(b_h-s)}, \quad &s \in \mathcal{X}_1 \\
0, \quad &s \notin \mathcal{X}_1
\end{cases}
\end{split}
\label{p1}
\end{equation}
where $\mathcal{X}_1 = (0, \min\{o_w,o_h\}$]. We have $\mathcal{X}_1$ because $V_1$ should not exceed the size of $o$. 

Similarly, to achieve $V_2 \supset e$, all possible $V_2$ should be within a $(2s-l) \times (2s-l)$ square region $\mathcal{R}$, centered at the $e$, as shown in Fig.~\ref{v2}(a). 
Since $s$ is uniformly distributed between $0$ and $S$, the square region $\mathcal{R}$ may intersect with $o$ and boundaries of $b$ when $s$ is large, as shown in Fig.~\ref{v2}(b). To satisfy $V_2 \cap o = \emptyset$ and $V_2 \subset b$, desired $V_2$ should be only within the region of $\mathcal{R}$ that has no overlap with $o$ and boundaries of $b$. We assume the width and height of this region as $\Delta w$ and $\Delta h$. Given fixed $b_w$, $o_x$ and $e_x$, $\Delta w$ is a function of crop size $s$ and given fixed $b_h$ and $e_y$, $\Delta h$ is also a function of $s$. Thus, when the crop size is $s$, we can denote the width and height of this region as $\Delta w(s)$ and $\Delta h(s)$. Then, we follow the same procedure as $p_1(s)$ to obtain the probability $p_2(s)$ as:
\begin{equation}
\begin{split}
p_{2}(s) &= \text{Pr}\{(V_2 \supset e) \cap (V_2 \cap o=\emptyset)\} \\
& = 
\begin{cases}
\frac{(\Delta w(s)-s)(\Delta h(s)-s)}{(b_w-s)(b_h-s)}, \quad &s \in \mathcal{X}_2 \\
0, \quad &s \notin \mathcal{X}_2
\end{cases}
\end{split}
\label{p2}
\end{equation}
where $\mathcal{X}_2 = (l, \min\{b_w-(o_x+o_w),b_h\}]$. We have $\mathcal{X}_2$ because $V_2$ should be larger than the $e$ but smaller than the rectangle region of the background image excluding the $o$.

Recall that we are supposed to maximize the $p$ in Equation~\ref{totalp} with aforementioned forms of $p_1(s)$ and $p_2(s)$. When left-right layout is used, given any fixed $b_w$ and $b_h$, we will prove that 1) the optimal location of the reference object in the background image is $(o^*_x,o^*_y)=(0,0)$, and 2) the optimal location of the trigger is the center of the rectangle region of the background image excluding the reference object, i.e., $(e^*_x,e^*_y)=(\frac{b_w+o_w-l}{2}, \frac{b_h-l}{2})$.

\noindent
\textbf{Optimal location of the trigger:} Let's derive the optimal location $(e^*_x,e^*_y)$ of the trigger $e$ first. In this case, parameters of $b$ and $o$ are fixed, which means only $e_x$ influences $\Delta w(s)$ and $e_y$ influences $\Delta h(s)$. We denote the horizontal distance between $e$ and $o$ as $d_1$ and the horizontal distance between $e$ and the right boundary of $b$ as $d_2$. Then we have:
\begin{equation}
\begin{split}
& d_1=e_x-(o_x+o_w), \\
& d_2=b_w-(e_x+l),
\end{split}
\label{d12}
\end{equation}
where both $d_1$ and $d_2$ depend on $e_x$. Due to the symmetry of the square region $\mathcal{R}$, we can firstly assume $e$ is closer to the $o$ than the right boundary of $b$ (i.e., $d_1 \leq d_2$), as shown in Fig.~\ref{v2}(b). In this case, we express $\Delta w(s)$ as follows:
\begin{equation}
\Delta w(s)=
\begin{cases}
2s-l,  \quad &s \in (\min\{\mathcal{X}_2\},d_1+l] \\
d_1+s, \quad &s \in (d_1+l,d_2+l] \\
b_w-(o_x+o_w), \quad &s \in (d_2+l, \max\{\mathcal{X}_2\}]
\end{cases} 
\end{equation}

If there exists $e_x$ and $e_x^{\prime}$ such that $d_1 < d_1^{\prime}\leq d_2^{\prime}<d_2$, we can obtain $\Delta w^{\prime}(s)-\Delta w(s)$ as:
\begin{equation} 
\begin{split}
&\Delta w^{\prime}(s)-\Delta w(s)=\\
&\begin{cases}
0,  \quad &s \in (\min\{\mathcal{X}_2\},d_1+l] \\
s-(d_1+l), \quad &s \in (d_1+l,d_1^{\prime}+l] \\
d_1^{\prime}-d_1, \quad &s \in (d_1^{\prime}+l,d_2^{\prime}+l] \\
(d_2+l)-s, \quad &s \in (d_2^{\prime}+l,d_2+l] \\
0, \quad &s \in (d_2+l,\max\{\mathcal{X}_2\}]
\end{cases} 
\end{split}
\label{minus}
\end{equation}
We have $\Delta w(s) \leq \Delta w^{\prime}(s)$ holds for all $s$. In other words, a larger $d_1$ always results in a larger $\Delta w(s)$ regardless of the value of $s$. Since we know that $\Delta w(s)$ is positively correlated with $p$ and we have $d_1 \leq d_2$ by assumption, $d_1=d_2$ will achieve the optimal $\Delta w(s)$ for all $s$ and maximize the $p$. We should get the same optimal result (i.e., $d_1=d_2$) if we start by assuming $d_1 \geq d_2$. Therefore, according to Equation~\ref{d12}, we obtain $e_x^{*}$ as:
\begin{align}
e_x^{*} = \frac{b_w+o_x+o_w-l}{2}
\label{ex}
\end{align}
It is noted that we will derive the optimal location of the reference object $(o^*_{x},o^*_{y})=(0,0)$ for left-right layout. Therefore, we can further reduce the Equation~\ref{ex} as $e^{*}_x= \frac{b_w+o^*_x+o_w-l}{2} = \frac{b_w+o_w-l}{2}$.

Next, we denote the vertical distance between $e$ and the top boundary of $b$ as $d_3$ and the vertical distance between $e$ and the bottom boundary of $b$ as $d_4$:
\begin{equation}
\begin{split}
& d_3=e_y \\
& d_4=b_h-(e_y+l)
\end{split}
\label{d34}
\end{equation}
where both $d_3$ and $d_4$ depend on $e_y$. By assuming $d_3 \leq d_4$, we express $\Delta h(s)$ as follows:
\begin{equation}
\Delta h(s)=
\begin{cases}
2s-l,  \quad &s \in (\min\{\mathcal{X}_2\},d_3+l] \\
d_3+s,  \quad &s \in (d_3+l,d_4+l] \\
b_h,  \quad &s \in (d_4+l, \max\{\mathcal{X}_2\}]
\end{cases} 
\end{equation}
If there exists $e_y$ and $e_y^{\prime}$ such that $d_3<d_3^{\prime} \leq d_4^{\prime}<d_4$, similar to Equation~\ref{minus}, we can show that $\Delta h(s) \leq \Delta h^{\prime}(s)$ holds for all $s$.  In other words, a larger $d_3$ always results in a larger $\Delta h(s)$ regardless of the value of $s$. Since $\Delta h(s)$ is also positively correlated with $p$ and we have $d_3 \leq d_4$, we conclude that $d_3=d_4$ will maximize the $p$. Therefore, we obtain $e^*_y$ according to Equation~\ref{d34} as:
\begin{align}
e_y^{*} = \frac{b_h-l}{2}
\end{align}

\noindent
\textbf{Optimal location of the reference object:} Given ($e^*_x, e^*_y$), our next step is to derive the optimal location ($o^*_x, o^*_y$) of the reference object $o$ such that $p$ is maximized. Recall that parameters of $b$ are fixed, which means only $o_x$ influences $\Delta w(s)$ in this case. Assume there exists an $o^{\prime}_x>o_x$, which results in $\Delta w^{\prime\prime}(s)$. Under the optimal location of the trigger, we obtain $\Delta w^{\prime\prime}(s)-\Delta w(s)$ as:
\begin{equation}
\begin{split}
&\Delta w^{\prime\prime}(s)-\Delta w(s)=\\ 
&\begin{cases}
0,  \quad &s \in (\min\{\mathcal{X}_2\}, f(o^{\prime}_x)] \\
b_w-(o^{\prime}_x+o_w)-(2s-l),  \quad &s \in (f(o^{\prime}_x), f(o_x)] \\
o_x-o^{\prime}_x,  \quad &s \in (f(o_x),\max\{\mathcal{X}_2\}] \\
\end{cases} 
\end{split}
\end{equation}
where $f(o_x)=\frac{b_w-o_x-o_w+l}{2}$ indicates the smallest $s$ such that $V_2$ touches the $o$ and right boundary of $b$ under the input $o_x$. We show that if $o^{\prime}_x>o_x$, $\Delta w^{\prime\prime}(s) \leq \Delta w(s)$ holds for all $s$. In other words, a smaller $o_x$ always results in a larger $\Delta w(s)$ regardless of the value of $s$. Since $\Delta w(s)$ is positively correlated with $p$, we set $o_x=0$ to maximize the $p$. As for $o_y$, any $o_y \in [0, b_h-o_h]$ will lead to the same $p$. Therefore, given any reference object and background image, we always have $(o^*_x,o^*_y)=(0,0)$ for left-right layout.

\section{Proof of Theorem 2}
\label{theorem2}
For left-right layout, we aim to prove that for any $o$ and $e$, given any width of the background image $b_w > o_w$, the optimal height of the background image should be the height of the reference object, i.e., $b^*_h = o_h$. The proof of optimal width for bottom-top layout is similar.

Given the optimal locations of reference object $o$ and trigger $e$ in background image $b$, we obtain $\Delta h^*(s)$ and $\Delta w^*(s)$ as follows:
\begin{equation}
\begin{split}
& \Delta h^{*}(s)=
\begin{cases}
2s-l,\quad &s \in (\min\{\mathcal{X}_2\},\frac{b_h+l}{2}] \\
b_h,   \quad &s \in (\frac{b_h+l}{2},\max\{\mathcal{X}_2\}]
\end{cases}
\\
& \Delta w^{*}(s)=
\begin{cases}
2s-l,\quad &s \in (\min\{\mathcal{X}_2\},\frac{b_w-o_w+l}{2}] \\
b_w-o_w,\quad &s \in (\frac{b_w-o_w+l}{2},\max\{\mathcal{X}_2\}]
\end{cases}
\end{split}
\end{equation}
In this case, we derive the marginal probability of $p$ under the optimal locations of $o$ and $e$ as:
\begin{equation}
p_1p_2 =
\begin{cases}
\frac{(o_w-s)(o_h-s)(\Delta w^*(s)-s)(\Delta h^*(s)-s)}{(b_w-s)^2(b_h-s)^2}, &s\in \mathcal{X}\\
0, &s \notin \mathcal{X}
\end{cases}
\end{equation}
where $\mathcal{X}=\mathcal{X}_1 \cap \mathcal{X}_2 = (l,\min\{o_w,o_h,b_w-o_w\}]$. Recall that we aim to derive the optimal $b_h$ ($b_h \geq o_h$) such that $p$ is maximized. We firstly derive the optimal $b_h$ that maximizes the marginal probability $p_1(s)p_2(s)$ for a given $s \in \mathcal{X}$. We have:
\begin{equation}
\begin{split}
&\argmax_{b_h} p_1(s)p_2(s) = \argmax_{b_h} \frac{\Delta h^*(s)-s}{(b_h-s)^2} \\
&= \argmax_{b_h} [\log(\Delta h^*(s)-s) - 2 \log(b_h-s)]
\end{split}
\label{bh}
\end{equation}
Let's denote $g(b_h, s)=\log(\Delta h^*(s)-s) - 2 \log(b_h-s)$. We consider two scenarios:

\noindent
\textbf{(i).} If there exists $b_h$ and $b^{\prime}_h$ such that $\frac{b_h+l}{2} < \frac{b^{\prime}_h+l}{2} \leq \max\{\mathcal{X}\}$, we can obtain $g(b^{\prime}_h, s)-g(b_h, s)$ as:
\begin{equation}
\begin{split}
&g(b^{\prime}_h, s)-g(b_h, s)=\\
&\begin{cases}
\log \frac{(b_h-s)^2}{(b^{\prime}_h-s)^2}, \quad &s \in (\min\{\mathcal{X}\}, \frac{b_h+l}{2}]\\
\log \frac{(b_h-s)(s-l)}{(b^{\prime}_h-s)(b^{\prime}_h-s)}, \quad &s \in (\frac{b_h+l}{2}, \frac{b^{\prime}_h+l}{2}]\\
\log \frac{(b_h-s)}{(b^{\prime}_h-s)}, \quad &s \in (\frac{b^{\prime}_h+l}{2}, \max\{\mathcal{X}\}]
\end{cases}
\end{split}
\end{equation}
We show that if there exists $b_h$ and $b^{\prime}_h$ such that $\frac{b_h+l}{2} < \frac{b^{\prime}_h+l}{2} \leq \max\{\mathcal{X}\}$, $g(b^{\prime}_h, s) \leq g(b_h, s)$ holds for all $s$. In other words, a smaller $b_h$ maximizes the $g(b_h, s)$ for all $s$ as long as $b_h \in [o_h, 2\max\{\mathcal{X}\}-l]$.

\noindent
\textbf{(ii).} If there exists $b_h$ and $b^{\prime}_h$ such that $\frac{b^{\prime}_h+l}{2} > \frac{b_h+l}{2} > \max\{\mathcal{X}\}$, we can obtain $g(b^{\prime}_h, s)-g(b_h, s)$ as:
$$g(b^{\prime}_h, s)-g(b_h, s)=\log \frac{(b_h-s)^2}{(b^{\prime}_h-s)^2} <0$$
Therefore, a smaller $b_h$ also maximizes the $g(b_h, s)$ for all $s$ as long as $b_h \in (2\max\{\mathcal{X}\}-l, \infty)$. 

\noindent
Combining \textbf{(i)} and \textbf{(ii)}, we theoretically prove that $g(b_h, s)$ monotonically decreases for all $s \in \mathcal{X}$ as $b_h$ increases. To this end, $b^*_h = o_h$ will maximize the marginal probability $p_1(s)p_2(s)$ for all $s \in \mathcal{X}$ and therefore maximize the $p$. 
