\begin{abstract}
Contrastive learning (CL) pre-trains general-purpose encoders using an unlabeled pre-training dataset, which consists of images or image-text pairs. CL is vulnerable to {data poisoning based backdoor attacks (DPBAs)}, in which an attacker injects {poisoned inputs} into the pre-training dataset so the encoder is backdoored. 
However, existing DPBAs achieve limited effectiveness. In this work, we take the first step to analyze the limitations of existing backdoor attacks and propose new DPBAs called \textbf{{\name}} to CL. {\name} introduces a new attack strategy to create poisoned inputs and uses a theory-guided method to maximize attack effectiveness. 
Our experiments show that  {\name} substantially outperforms existing DPBAs. 
In particular, {\name} is the first DPBA that achieves \textbf{more than 90\%} attack success rates with only a few (3) reference images and a small poisoning ratio (0.5\%). 
Moreover, we also propose a defense, called {localized cropping}, to defend against DPBAs. Our results show that our defense can reduce the effectiveness
of DPBAs, but it sacrifices the utility of the encoder, highlighting the need for new defenses.
\end{abstract}
