\documentclass[aps,prl,10pt,twocolumn,superscriptaddress]{revtex4}
%\documentclass[aps,prl,10pt, twocolumn, superscriptaddress, nofootinbib]{revtex4}
\usepackage{mathrsfs, amssymb, amsmath}  
\usepackage{footmisc}
%\usepackage{bbm, bm, dsfont, yfonts, mathrsfs, dsserif}
\usepackage{latexsym}
\usepackage{natbib, comment}
\usepackage{url}
\usepackage{dcolumn}
\usepackage{multirow}
\usepackage{color}
\usepackage{soul}
\usepackage[normalem]{ulem}
\usepackage{amsfonts,amssymb,amsmath, txfonts}
\usepackage{graphicx,epsfig, cancel}
\usepackage{psfrag}
\usepackage{hyperref}
\hypersetup{colorlinks=true}
\usepackage{mathtools}
\usepackage{enumitem}
\usepackage{float}

\usepackage{epsfig}
\usepackage{amssymb}
\usepackage{amsfonts}
\usepackage{amsmath}
\usepackage{euscript}
\usepackage{verbatim}
\usepackage{latexsym}
\usepackage{graphicx}
\usepackage{caption, color}

%\usepackage{svg}
\usepackage{subcaption}

\usepackage[all]{xy}

\usepackage[dvipsnames]{xcolor}
\usepackage{xcolor}
\hypersetup{ linktoc=all,
    colorlinks, linkcolor={brightpink},
    citecolor={blue}, urlcolor={blue}
}

%%%%%%%%%%%%%%%%%%%%%%%%%%%%%%%%%%%%%%%%%%%%%%%%%%%%%%%%%%%%%%%%%%%%%%
\definecolor{rosy}{RGB}{230,235,252}
\definecolor{myframetitle}{RGB}{90,89,170}
\definecolor{myblocktitle}{RGB}{140,185,249}
\definecolor{mytitle}{RGB}{10,80,26}

\definecolor{darkgreen}{RGB}{27,130,45}
\definecolor{darkblue}{rgb}{0,0,0.3}
\definecolor{darkred}{rgb}{0.7,0,0}

\definecolor{light gray}{RGB}{220,220,220}
\definecolor{dark purple}{RGB}{108,0,217}
\definecolor{pink}{RGB}{190,20,100}
\definecolor{orang}{RGB}{193,63,0}
\definecolor{green}{RGB}{11,98,17}
\definecolor{darkpink}{RGB}{153,0,76}
\definecolor{bluegreen}{RGB}{0,102,102}
\definecolor{greenlagan}{RGB}{0,102,0}
\definecolor{redgreen}{RGB}{102,102,0}
\definecolor{Redgreen}{RGB}{153,76,0}
\definecolor{vividviolet}{rgb}{0.62, 0.0, 1.0}
\definecolor{amaranth}{rgb}{0.9, 0.17, 0.31}
\definecolor{palatinateblue}{rgb}{0.15, 0.23, 0.89}
\definecolor{brightpink}{rgb}{1.0, 0.0, 0.5}
\definecolor{cornflowerblue}{rgb}{0.39, 0.58, 0.93}
\definecolor{deepcarminepink}{rgb}{0.94, 0.19, 0.22}
\definecolor{radicalred}{rgb}{1.0, 0.21, 0.37}
%%%%%%%%%%%%%%%%%%%%%%%%%%%%%%%%%%%%%%%%%%%%%%%%%%%%%%%%%%%%%%%%%%%%%%%%%%%%%

\usepackage{listings}

%%%%%%Define some new commands and  macros
\newcommand{\beq}{\begin{equation}}
\newcommand{\eeq}{\end{equation}}
\newcommand{\bea}{\begin{eqnarray}}
\newcommand{\eea}{\end{eqnarray}}
\newcommand{\beas}{\begin{eqnarray*}}
\newcommand{\eeas}{\end{eqnarray*}}
\newcommand{\defi}{\stackrel{\rm def}{=}}
\newcommand{\non}{\nonumber}
\newcommand{\bquo}{\begin{quote}}
\newcommand{\enqu}{\end{quote}}
%%%%%%%%%%%%%%%%
\renewcommand{\(}{\begin{equation}}
\renewcommand{\)}{\end{equation}}
%%%%%%%%%%%%%%%%%%%%%%%%%%%%%%%%%% definitions
\def \eqn#1#2{\begin{equation}#2\label{#1}\end{equation}}

\def\e{\epsilon}
\def\IZ{{\mathbb Z}}
\def\IR{{\mathbb R}}
\def\IC{{\mathbb C}}
\def\IQ{{\mathbb Q}}
\def\de{\partial}
\def\Tr{ \hbox{\rm Tr}}
\def\H{ \hbox{\rm H}}
\def\HE{ \hbox{$\rm H^{even}$}}
\def\HO{ \hbox{$\rm H^{odd}$}}
\def\K{ \hbox{\rm K}}
\def\Im{ \hbox{\rm Im}}
\def\Ker{ \hbox{\rm Ker}}
\def\const{\hbox {\rm const.}}
\def\o{\over}
\def\im{\hbox{\rm Im}}
\def\re{\hbox{\rm Re}}
\def\bra{\langle}\def\ket{\rangle}
\def\Arg{\hbox {\rm Arg}}
\def\Re{\hbox {\rm Re}}
\def\Im{\hbox {\rm Im}}
\def\exo{\hbox {\rm exp}}
\def\diag{\hbox{\rm diag}}
\def\longvert{{\rule[-2mm]{0.1mm}{7mm}}\,}
\def\a{\alpha}
\def\dag{{}^{\dagger}}
\def\tq{{\widetilde q}}
\def\p{{}^{\prime}}
\def\W{W}
\def\N{{\cal N}}
\def\hsp{,\hspace{.7cm}}

\def\br{\nonumber}
\def\IZ{{\mathbb Z}}
\def\IR{{\mathbb R}}
\def\IC{{\mathbb C}}
\def\IQ{{\mathbb Q}}
\def\IP{{\mathbb P}}
\def \eqn#1#2{\begin{equation}#2\label{#1}\end{equation}}


\newcommand{\C}{\ensuremath{\mathbb C}}
\newcommand{\Z}{\ensuremath{\mathbb Z}}
\newcommand{\R}{\ensuremath{\mathbb R}}
\newcommand{\rp}{\ensuremath{\mathbb {RP}}}
\newcommand{\cp}{\ensuremath{\mathbb {CP}}}
\newcommand{\vac}{\ensuremath{|0\rangle}}
\newcommand{\vact}{\ensuremath{|00\rangle}                    }
%\newcommand{\oc}{\ensuremath{\overline{c}}}
\newcommand{\psizero}{\psi_{0}}
\newcommand{\phizero}{\phi_{0}}
\newcommand{\hzero}{h_{0}}
\newcommand{\psiin}{\psi_{\rh}}
\newcommand{\phiin}{\phi_{\rh}}
\newcommand{\hin}{h_{\rh}}
\newcommand{\rh}{r_{h}}
\newcommand{\rb}{r_{b}}
\newcommand{\psibnd}{\psi_{0}^{b}}
\newcommand{\psibndp}{\psi_{1}^{b}}
\newcommand{\phibnd}{\phi_{0}^{b}}
\newcommand{\phibndp}{\phi_{1}^{b}}
\newcommand{\gbnd}{g_{0}^{b}}
\newcommand{\hbnd}{h_{0}^{b}}
\newcommand{\zh}{z_{h}}
\newcommand{\zb}{z_{b}}
\newcommand{\man}{\mathcal{M}}
\newcommand{\hbr}{\bar{h}}
\newcommand{\tbr}{\bar{t}}

\newcommand\tcr{\textcolor{red}}
\newcommand\tcb{\textcolor{blue}}
\newcommand\tcg{\textcolor{green}}

\newcommand\snote[1]{\textcolor{red}{\bf [Sh:\,#1]}}






\def\red{\textcolor{red}}
\def\blue{\textcolor{blue}}
\def\green{\textcolor{green}}
%
%
%\def\jcap{JCAP}
%\def\lsim{\:\raisebox{-1.1ex}{$\stackrel{\textstyle<}{\sim}$}\:}
\def\gsim{\:\raisebox{-1.1ex}{$\stackrel{\textstyle>}{\sim}$}\:}
%\newcommand{\ba}{\begin{array}}
%\newcommand{\ea}{\end{array}}
%\newcommand{\be}{\begin{equation}}
%\newcommand{\ee}{\end{equation}}
%\newcommand{\bea}{\begin{eqnarray}}
%\newcommand{\eea}{\end{eqnarray}}
%\def\al{\alpha}
\def\weff{w_{\tiny{\text{eff}}}}
%\def\H0{\varmathbb{H0}}
%\def\H0{\mathbbmtt{H0}}
%\def\H0{\mathrsfs{H0}}
%\def\H0{\mathds{H0}}
%\def\H0{\mathbbb{H0}}
\def\H0{{\text{H}\hspace*{-2.05mm}\text{H} 0\hspace*{-1.35mm}0\ }}

%\def\mt{$\mu$-$\tau$}
%\def\mnuf{${\cal M}_{\nu f }$}
\def\lcdm{$\Lambda$CDM}
%
\def\be{\begin{equation}}
\def\ee{\end{equation}}
\def\beq{\begin{equation}}
\def\eeq{\end{equation}}
\def\bea{\begin{eqnarray}}
\def\eea{\end{eqnarray}}
\newcommand{\dd}{\textrm{d}}
\newcommand{\nn}{\nonumber \\ }
\renewcommand{\thefootnote}{\arabic{footnote}}


\begin{document}

%\title{Cosmic Fluid Tilt  as Growing Cosmological Hair}
%\title{Cosmic Fluid Tilt, an Instability in Cosmological Principle}
%\title{Cosmic Fluid Tilt, an Instability in Cosmological Principle}
\title{A Tilt Instability in the Cosmological Principle}
%\title{A Dipole Crack in   Cosmological Principle}
%\title{Dipole Cosmology}


%\begin{comment}

\author{Chethan Krishnan}\email{chethan.krishnan@gmail.com}
\affiliation{Center for High Energy Physics,\\
Indian Institute of Science, Bangalore 560012, India}

% ORCID: 0000-0003-0813-5156
\author{Ranjini Mondol}\email{ranjinim@iisc.ac.in }
\affiliation{Center for High Energy Physics,\\
Indian Institute of Science, Bangalore 560012, India}

\author{M. M. Sheikh-Jabbari}\email{jabbari@theory.ipm.ac.ir}
\affiliation{School of Physics, Institute for Research in Fundamental Sciences (IPM),\\ P.O.Box 19395-5531, Tehran, Iran}


\begin{abstract}
We show that the Friedmann-Lema\^{i}tre-Robertson-Walker (FLRW) framework has an instability towards the growth of fluid flow anisotropies, even if the Universe is accelerating. This flow (tilt) instability in the matter sector is invisible to Cosmic No-Hair Theorem-like arguments, which typically only flag shear anisotropies in the metric. We illustrate our claims in the setting of ``dipole cosmology'', the maximally Copernican generalization of FLRW that can accommodate a flow. Simple models are sufficient to show that the cosmic flow need not track the  shear, even in the presence of a positive cosmological constant.  We also emphasize that the growth of the tilt hair is fairly generic if the total equation of state $w(t) \rightarrow -1$ at late times (as it does in standard cosmology), irrespective of the precise model of dark energy.

%We show that the Friedmann-Lema\^{i}tre-Robertson-Walker (FLRW) framework has an instability towards the growth of fluid flow (tilt) anisotropies, even when the Universe is accelerating. This tilt instability in the matter sector is invisible to Cosmic No-Hair Theorem-like arguments, which typically only flag shear anisotropies in the metric. We illustrate our claims in the setting of ``dipole cosmology'', the maximally Copernican generalization of FLRW that can accommodate a flow. Simple models are sufficient to show that the cosmic flow need not track the  shear, even in the presence of a positive cosmological constant.  We also note that the growth of the tilt hair is fairly generic if the total equation of state $w(t) \rightarrow -1$ at late times (as it does in standard cosmology), irrespective of the precise model of vacuum energy.  

%Modern cosmology is based on Friedmann-Lema\^{i}tre-Robertson-Walker (FLRW) framework which describes a homogeneous and isotropic spacetime and matter distribution. Nonetheless, there are increasing evidence that we may be living in a Universe with a preferred cosmic flow direction associated with a non-kinematic cosmic dipole. We take the first steps in formulating ``dipole cosmology''  which accommodates such a cosmic dipole and flow. In the dipole cosmology framework with cosmic comoving time $t$, the background spacetime metric is specified by an over all Hubble parameter $H(t)$ and  cosmic anisotropy (shear) $\sigma(t)$, and the matter sector besides the energy density and pressure, is specified by the tilt $\beta(t)$ which parameterizes the cosmic flow. We study cosmic evolution for simple but typical cosmic fluids and show that the cosmic flow $\beta(t)$ does not trace the cosmic shear. In particular, for an accelerated expanding universe the flow (tilt) can be non-vanishing or  even growing while the anisotropy (shear) dies off.  As another important theoretical outcome, our analysis reveals that FLRW cosmology is unstable under tilt perturbations. Hence, the current dipole observations are theoretically expected and our dipole cosmology provides a viable cosmological framework. 



%Modern cosmology is based on Friedmann-Lema\^{i}tre-Robertson-Walker (FLRW) framework which describes a homogeneous and isotropic spacetime and matter distribution. Nonetheless, there are increasing evidence that we may be living in a Universe with a preferred cosmic flow direction associated with a non-kinematic cosmic dipole. We take the first steps in formulating ``dipole cosmology''  which accommodates such a cosmic dipole and flow. In the dipole cosmology framework with cosmic comoving time $t$, the background spacetime metric is specified by an over all Hubble parameter $H(t)$ and  cosmic anisotropy (shear) $\sigma(t)$, and the matter sector besides the energy density and pressure, is specified by the tilt $\beta(t)$ which parameterizes the cosmic flow. We study cosmic evolution for simple but typical cosmic fluids and show that the cosmic flow $\beta(t)$ does not trace the cosmic shear. In particular, for an accelerated expanding universe the flow (tilt) can be non-vanishing or  even growing while the anisotropy (shear) dies off.  As another important theoretical outcome, our analysis reveals that FLRW cosmology is unstable under tilt perturbations. Hence, the current dipole observations are theoretically expected and our dipole cosmology provides a viable cosmological framework. 


\end{abstract}

\maketitle











\setcounter{footnote}{0}

%%%%%%%%%%%%%%%%%%%%%%%%%%%%%%%%%%%%%%%%%%%%%%%%%%%%%%%%%%%%%%%%%%%%%%%%%%%%%%%%%%%%%%%%%%%%%%
%\tableofcontents 
%%%%%%%%%%%%%%%%%%%%%%%%%%%%%%%%%%%%%%%%%%%%%%%%%%%%%%%%%%%%%%%%%%%%%%%%%%%%%%%%%%%%%%%%%%%%%%

\section{Introduction and motivation }

Copernicus put forward the viewpoint that we are not privileged observers in the Universe \cite{Copernicus}. Copernican viewpoint has influenced physics and in particular cosmology ever since \cite{WeinbergOldBook, Ellis-Maartens-McCallum--Book}. After Hubble's discovery in 1929 \cite{Hubble-1929},  the Copernican principle has been dubbed  the ``cosmological principle''. It states that  Universe is homogeneous and isotropic on constant cosmic time slices and is formulated within the FLRW cosmology. The current concordance flat $\Lambda$ Cold Dark Matter (\lcdm) model, sometimes also called the standard model of cosmology, is a specific model within the FLRW framework. 

Flat \lcdm\ model culminates the cosmology today and  it rests upon cosmological data at various different redshifts. However, steadily improving precision in observations has created tensions within the model \cite{crisis, Intro0,Eleonora-et-al-review-1}. Many models, almost all within the FLRW framework, have been proposed to resolve the tensions, but none of them seem fully satisfactory \cite{Perivolaropoulos:2021jda, SNOWMASS-2022}. %.to address the tensions in a satisfactory way
This has promoted the idea  that the current cosmological tensions may be symptoms of a much deeper issue, a violation of the cosmological principle \cite{Upper-bound-H0, binning-fitting}. This viewpoint resonates with the growing concern that the cosmological principle may not pass all observational tests, see \cite{Beyond-FLRW-review} for a recent review.
%, from the Cosmic Microwave Background (CMB) data from Planck mission \cite{Planck-2018}, to supernovae (SN) standard candles \cite{SNe} and to large scale structure and baryon acoustic oscillation (BAO) data \cite{BAO}. According to the flat \lcdm\ model, the Universe is filled with dark energy (DE), modeled by the cosmological constant, and the pressureless matter (dust), including baryonic and dark matter. The background cosmology in flat \lcdm\ has two parameters in it, the Hubble expansion rate today $H_0$ and the percentage of DE in the total energy budget of the Universe today $\Omega_\Lambda$. 

%The Planck collaboration has determined these two parameters by percent level precision \cite{Planck-2018}. The same two parameters could be directly measured by local astrophysical/cosmological observations, without invoking any particular cosmological model, and only assuming homogeneity and isotropy \cite{crisis, Intro0,Eleonora-et-al-review-1}. While the precision of $\Omega_\Lambda$ direct measurements is still not competitive ($\sim 10\%$), the precision on $H_0$ measurement has been increasing since Hubble Space Telescope stated goal of $10\%$ in turn of the millennium to $\sim 1\%$ now \cite{crisis}. At $1\%$ precision the cosmological inferences of $H_0$ (in particular that of Planck collaboration) and the local measurements has reached $10\%$, $4-6\sigma$,   discrepancy \cite{Eleonora-combined-analysis}; the Hubble tension \cite{crisis}. Besides the Hubble tension there are other, still statistically less significant, tensions between observations and the Planck flat \lcdm\ model values \cite{Perivolaropoulos:2021jda}. Appearance of the tensions in the last decade has promoted checking various possible sources for systematic errors, and so far the tension seems to be persisting, and even growing. This has motivated several models beyond \lcdm\ to address or ease the tensions. Nearly all of these models are within the FLRW setting but none of them seem to resolve all the tensions and statistically perform much better than flat \lcdm, see \cite{SNOWMASS-2022} and references therein for a recent review and discussions. This situation has prompted speculations that these tensions are calling upon us to look beyond  FLRW setting for a resolution \cite{H0-running, Upper-bound-H0}.

%In a parallel line of developments, improving cosmological data, in particular the SN datasets \cite{SNe-dipole-anomaly,SNe-hemisphere-anomaly} and that of quasars (QSO) \cite{Subir-et-al-QSO,QSO-hemisphere-anomaly}, has strengthened the traces of breakdown of cosmic isotropy. In particular, evidence for the presence of a cosmic dipole has been noted in radio galaxies \cite{radio-dipole},  bulk flows \cite{Bulk-Flow} and the CMB data \cite{CMB-anomalies}.  See \cite{Beyond-FLRW-review} for a recent review and references. If these observations are substantiated, given  pivotal role of the cosmological principle in our current understanding of cosmology, it is imperative to formulate cosmological models which can accommodate such cosmological  dipole observations. 

The simplest (ie., the most Copernican) setting that can accommodate a cosmic flow is the ``dipole cosmology'' paradigm of \cite{KMS} that generalizes the FLRW framework. This is a tilted axially symmetric Bianchi V/VII$_h$ cosmology that falls into the broader rubric of tilted homogeneous models \cite{King, Ellis-lectures, Ellis-Maartens-McCallum--Book}. The ``tilt'' refers to a flow in the fluid that is {\em not} orthogonal to the homogeneous time slices. As such, dipole cosmology is a minimal setting within the tilted models for formulating a cosmic dipole. It has 2 functions of the time coordinate in the background metric -- an overall Hubble expansion rate and the metric anisotropy parametrized by shear. It also has 3 functions in the cosmic fluid sector -- energy density, pressure and the tilt. Part of our goal in this letter is to emphasize that this is an extremely simple and tractable framework that can accommodate a dipole flow, that generalizes the Friedmann equations while still being ODEs.

A second  motivation is that this class of models allow us to investigate certain stability properties of the (conventional) FLRW framework that is often not emphasized. Standard lore dictates that cosmic acceleration washes away the anisotropies. Statements of this kind are usually called Cosmic No-Hair Theorems \cite{Wald-cosmic-no-hair}, and they usually refer to {very fast (exponential) falloff of} the shear anisotropy in the metric. Dipole cosmology enables us to see that even while the shear anisotropies die down, the tilt anisotropies of an accelerating Universe need not. We will illustrate this here in the setting of fluids with constant equations of state together with a positive cosmological constant $\Lambda$. This is an extremely simple scenario that was not discussed in detail in \cite{KMS}, even though it was observed there that tilt growth is a feature that is easy to arrange, in particular in models where the total equation of state $w(t) \rightarrow -1$ at late times. 



%After presenting the basic setup we work out Einstein equations for these 5 parameters; there are 4 such equations, and the set is completed by the addition of an equation of state (EoS), as in the usual FLRW models. Then we analyse these equations and in particular, focus on  the evolution of cosmic shear and the tilt for a variety of models. 

Let us list a few of our main results. (1) Even when the  shear dies off at late times, the tilt or bulk flow can be relatively large or can even grow. (2) Even with a very small initial value for tilt and almost FLRW background, we can get a sizable bulk flow, {signalling a tilt instability in the FLRW cosmology.} (3) These features can arise even in an accelerating Universe (say, one with a positive cosmological constant). (4) Together with the cases considered in \cite{KMS}, the results of this paper show that these observations, while not absolutely generic, are quite easily realized (and generic indeed, in some regions of parameter/initial-condition space).  The punchline is that at late times, even if we are dealing with metrics which are homogeneous and almost isotropic, we still could have substantial bulk flows and cosmic dipoles. It seems likely that this is of significance for late time observational cosmology and  cosmic tensions, even though to make a precise statement, we will need more detailed model building within the dipole cosmology paradigm. 

Let us emphasize a simple but potentially confusing point. Dipole (tilt) anisotropy is an observer-dependent concept; even in a homogeneous nearly isotropic metric, non-trivial cosmic flows can result in perceived anisotropies/dipoles by standard observers living on those flows.  A detailed discussion of dipole cosmography will be presented elsewhere.

Our analysis is done using the dipole cosmology equations (presented below) that generalize the FLRW Friedmann equations. They can be viewed as keeping track of the non-linear evolution of tilt perturbations around the FLRW system. Interpreted this way, our result is a demonstration that FLRW cosmology is not stable against homogeneous, anisotropic tilt perturbations even when the Universe is accelerating. In this work we focus on the main results; more detailed analyses and discussions may be found in previous and upcoming works, in particular in \cite{KMS, upcoming-1, cosmo}. 


\section{Dipole Cosmology, the basic setup}\label{sec:2-basic-setup}
%Metric, stress tensor and the dipole cosmology equations}

The most general cosmological metric in 4 dimensions has 6 functions of space and time. Assuming a cosmic (comoving) time coordinate $t$, and spatial homogeneity on constant time slices, we remain with 3 functions of $t$. In our dipole cosmology besides homogeneity we also assume axisymmetry which removes one more function and hence the metric involves only 2 functions of $t$. The metric in the dipole cosmology setting takes the form 
\bea\label{DipoleMetric}
    ds^{2} = -dt^{2} + a^{2}(t)\left[ e^{4b(t)} dz^{2} +  e^{-2b(t)-2A_{0} z}\big(dx^{2}+dy^{2}\big) \right],
\eea
where $a(t)$ is the over all scale factor, $b(t)$ parameterizes the anisotropy and $A_0\neq 0$ is a constant of dimension of inverse length. While it may be set to 1 by a choice of units, we keep it for later convenience. See \cite{Ellis-Maartens-McCallum--Book, MacCallum, King, Stewart-Ellis, Ellis-lectures, Coley-2006} for background, and \cite{KMS, upcoming-1} for discussions. We may define the Hubble expansion rate $H(t)$ and the cosmic shear $\sigma (t)$ as usual
\begin{equation}\label{H-sigma}
H:=\frac{\dot{a}}{a},\qquad \sigma:=3\dot{b} 
\end{equation}
where \emph{dot} denotes derivative w.r.t. $t$. When $\sigma=0$ the metric reduces to an open FLRW universe. 

In this metric the $z$ direction is chosen to be along the cosmic flow (the tilt). To see this, we note that the in $(t,z, x,y)$ frame the energy momentum tensor of a perfect fluid with energy density $\rho$ and pressure $p$ is 
\begin{equation}\label{T-tilted}
\begin{split}
&T^{a}{}_{b} = T_{\text{iso}}{}^{a}{}_{b} + T_{\text{tilt}}{}^{a}{}_{b}\\
&T_{\text{iso}}{}^{a}{}_{b}=\text{diag}(-\rho, p,p,p)\\
&T_{\text{tilt}}{}^{a}{}_{b}=(\rho+p)\sinh\beta\left(\begin{array}{cccc}
 -\sinh\beta &  ae^{2b}  \cosh \beta & 0 & 0 \\
-\cosh \beta/(ae^{2b}) &  \sinh\beta  & 0 & 0 \\
 0 & 0 & 0 & 0 \\
 0 & 0 & 0 & 0\\
\end{array}
\right)
\end{split}
\end{equation}
where $T_{\text{iso}}{}^{a}{}_b$ is the energy momentum tensor of a usual isotropic perfect fluid. The off-diagonal terms in $T^a{}_b$ are a manifestation of the non-zero bulk flow along $z$ direction; $\beta$ parameterizes the tilt, the bulk flow. The important point in \eqref{T-tilted} is that when $\rho+p=0$, i.e. for a cosmological constant, $\beta$ drops out and we recover the usual FLRW setup. In other words, a cosmological constant can be viewed as having any value of the tilt.


\section{Evolution equations of dipole cosmology}

In our dipole cosmology we have 5 functions of $t$, $\rho(t), p(t), a(t), b(t), \beta(t)$ which are to be specified by Einstein's field equations
\bea\label{cosmoC}
R_{ab}-\frac12 Rg_{ab}= T_{ab}, 
\eea
or the continuity equations $\nabla^a T_{ab}=0$, yielding
\begin{subequations}\label{EoM-H-sigma}
\begin{align}
%\dot{H}+3H^2-2\frac{A_0^2}{a^2} e^{-4b}&=\frac12(\rho-p)+\frac13 (\rho+p)\sinh^2\beta+\Lambda\label{EoM-H-sigma-a} \\
%\dot{\sigma}+3H\sigma&= (\rho+p)\sinh^2\beta \label{shear-EoM}\\
H^2-\frac19\sigma^2-\frac{A_0^2}{a^2} e^{-4b}=\frac{\rho}{3}&+\frac13 (\rho+p)\sinh^2\beta%+\frac{\Lambda}{3} 
\label{EoM-H-sigma-c}\\
\dot{\sigma}+\sigma \big(3H -\frac{2A_0}{a}&\tanh\beta e^{-2b}\big)=0\label{EoM-sigma}\\
%\end{align}
%\end{subequations}
%\begin{subequations}\label{EoM-continuity}
%\begin{align}
\dot{\rho}+3H(\rho+p)=-(\rho+p&)\tanh\beta(\dot{\beta}-\frac{2A_0}{a} e^{-2b}) \label{Con1-1} \\
\dot{p}+H(\rho+p)= -(\rho+p&)\left( \frac23\sigma+\dot{\beta}\coth{\beta}\right). \label{Con2-1}
\end{align}
\end{subequations}
Moreover, we note that the above equations imply
\begin{equation}\label{EoM-H-sigma-d}\begin{split}
 \sigma &=\frac{1}{4A_0} a e^{2b}(\rho+p)\sinh2\beta
%\\
%&(\rho+p) a e^{2b}\sinh2\beta  = K\  \sigma 
\end{split}
\end{equation}
Note also that for $p=-\rho$, $\beta$ drops out of equations and the solution to the above equations for $\rho>0$ is a de Sitter geometry in an open Universe slicing. 

Among the above 4 equations,  \eqref{EoM-H-sigma-c} and \eqref{Con1-1} are generalization of  the 2 Friedmann equations of the usual FLRW cosmology, while \eqref{EoM-sigma} (or \eqref{EoM-H-sigma-d}) and \eqref{Con2-1} are new and govern dynamics of the shear and the tilt. To solve the above equations one needs to supplement them with another equation, which may be taken to be the equation of state $w(t):=d p/d\rho$.


\section{Examples of dipole cosmology models}\label{sec:DC-evolution}

In a previous paper \cite{KMS}, we considered scenarios where the $w(t)$ mentioned above had the feature that $w(t) \rightarrow -1$ at late times, so that the late Universe accelerates. Many examples of this type were considered and a general criterion was provided for when such models will lead to increasing tilt at late times. In particular, if the approach of $w(t)$ to $-1$ at late times is exponential and sufficiently fast, it was noted that the tilt would grow. It should be emphasized that this is $not$ a particularly stringent demand: indeed, it was pointed out that the effective (time-dependent) equation of state of the standard flat $\Lambda$CDM cosmology satisfies the demand! This indicates that the phenomenon of tilt growth is fairly generic  in the space of physically interesting models. 

%In usual FLRW models, we are free to fill the Universe with any combination of perfect fluids with any desired EoS, e.g. radiation, pressureless matter, dark energy, curvature and .... For the dipole cosmology such a freedom does not exist, because all these component must have the same tilt, otherwise the assumed axisymmetry of setup will be violated. One can see this more explicitly through analyses of our dynamical evolution \eqref{EoM-H-sigma}. To this end  take a generic $N$ component cosmic fluid $\rho=\sum_{i=1}^N \rho_i, p=\sum_{i=1}^N p_i$ and let each of them have EoS $w_i(t)$. Inserting these into  \eqref{EoM-H-sigma} one readily sees that $\dot\rho_i/(\rho_i+p_i)$ and $w_i$ should be independent of $i$. That is, we are only allowed to have one type of cosmic  fluid, defined through $w=w(t)$.  This argument, however, has an exception: as pointed out the cosmic fluid with $p=-\rho$, a ``cosmological constant'', is compatible with any tilt. Therefore, the most general fluid we can have in the dipole cosmology setup is a fluid with a given EoS $w$ plus a cosmological constant. {Note however that, here $w$ can be a generic, but given, function of time. Such a $w(t)$, as discussed in \cite{upcoming-1}, can represent a multi-component cosmic fluid composed of several perfect fluids of constant EoS $w_i$.}

In this paper, we will discuss another simple class of accelerating models that also exhibit late time tilt growth. These are models with a fluid with constant equation of state $w$ and  a cosmological constant. We will see that in these dipole $w$-$\Lambda$ models, tilt can increase at late times if $w$ is stiff enough. This will be taken as further evidence that tilt growth can be a fairly generic phenomenon in accelerating Universes. 

These $w$-$\Lambda$ models, as well as the general class of $w(t) \rightarrow -1$ models of \cite{KMS}, are presented to illustrate that our claims are fairly generically true. The specific models themselves are not of particular interest here.


\subsection{Dipole \texorpdfstring{$w$-$\Lambda$}{wLambda} models%: Dipole cosmologies with constant EoS and a cosmological constant 
}\label{sec:dipole-const-w-Eos}

As an illustrative example  we study models involving a fluid of constant EoS $w$ and a cosmological constant $\Lambda$. It follows from our earlier discussion that a $\Lambda$ can be incorporated into \eqref{EoM-H-sigma} by defining 
\be\label{w-Lambda-p-rho}
p=w\tilde{\rho}-\Lambda,\qquad \rho=\tilde\rho+\Lambda, \qquad -1< w\leq 1.
\ee
For a generic $w$, \eqref{EoM-H-sigma} implies
\begin{subequations}\label{const-w-dipole-1}
\begin{align}
&\hspace{-7mm}\frac{\ddot{a}}{a}=   \dot{H}+H^2=\frac23\Lambda-\frac{\tilde\rho}{6}(1+3w)-\frac29 \sigma^2-\frac{\tilde\rho}{3}(1+w)\sinh^2\beta\label{cosmic-acceleration-const-w}\\
&\hspace{-10mm}\dot{\beta}\big(\coth\beta-w \tanh \beta\big)=(3w-1)H-\frac23\sigma-\frac{2w A_0}{a(t)} e^{-2b}\tanh\beta \label{beta-growth-w-const}
\\
&\tilde\rho^{\frac{w}{1+w}} a e^{2b} \sinh\beta = C= const. \label{const-w-dipole-rho-X-beta}\\
&\sigma=\frac{C(1+w)}{2A_0}\tilde\rho^{\frac{1}{1+w}}\cosh\beta \label{sigma-w-dipole}
\end{align}
\end{subequations}
In our analysis we consider non-negative cosmological constant $\Lambda\geq 0$. From these equations we learn:
\begin{enumerate}
    \item As in the FLRW case, for $w\leq -1/3$ we get an accelerated expansion for any $\Lambda\geq 0$.
\item  As Universe expands $a(t)$ grows  and $\rho(t)$ drops.
\item The shear $\sigma(t)$ goes to zero (exponentially fast for accelerated expansion) and the universe isotropizes rapidly. 
\item Since $-1<w\leq 1$, $\coth\beta>1$ and $\tanh\beta <1$, the coefficient of $\dot\beta$ term is always positive. While the last term in \eqref{beta-growth-w-const} does not have a definite sign, it becomes insignificant at late times due to the expansion. Therefore,  for $w>1/3$ the sign of  $\dot\beta$  is positive and the tilt can grow \cite{Coley-2006}.
\item Cosmological constant $\Lambda$ does not explicitly appear in \eqref{beta-growth-w-const} and therefore it is plausible that $\beta$ growth only depends on  $w$ ($>1/3$) and not the value of $\Lambda$; $\beta$ growth can happen in accelerating/decelerating cosmologies. This is indeed what we have verified numerically. In fact, we find %as the plots in \ref{fig:ConstEoSbeta} and  \ref{fig:w-Lambda-dipole-model} 
that $\beta$ growth is faster in cases with $\Lambda$, {due to faster growth of $H$}.
\end{enumerate}
The above is also confirmed by  numerical evolution of the  equations for $\Lambda=0$ and $\Lambda>0$ cases. We plot the latter  %which have been respectively depicted in Figures \ref{fig:ConstEoSbeta} and
in FIG.~\ref{fig:w-Lambda-dipole-model}. 

\begin{figure}[htp]
    \centering
        \subfloat[\label{fig:wLambda-a}]{{\includegraphics[width=8.0 cm]{w-Lambda-H.jpeg}}}\\
%   \subfloat[\label{fig:wLambda-rho}]{{\includegraphics[width=8.5 cm]{w-Lambda-rho.jpeg}}}\\ 
   \subfloat[\label{fig:wLambda-sigma}]{{\includegraphics[width=8.5 cm]{ShearH.jpeg} }} \\ \subfloat[\label{fig:wLambda-beta}]{{\includegraphics[width=8.5 cm]{w-Lambda-beta.jpeg} }}  \\ 
 \subfloat[\label{fig:wLambda-beta-growth}]{{\includegraphics[width=8.5 cm]{ShearQ.jpeg} }}     %    \subfloat[\label{fig:aVar1}]{{\includegraphics[width=8.0 cm]{DbetaNConstant.jpeg} }}\\ 
%\subfloat[\label{fig:w1h}]{{\includegraphics[width=8.0 cm]{aNConstant.jpg} }}\hfill
\caption{Evolution of overall Hubble parameter $H(t)$,  $\sigma/H$, tilt $\beta$ and the dimensionless ratio $\sigma/({\beta}H)$ for dipole $w$-$\Lambda$ model. Initial values are  $a_{in} = 1, b_{in} = 0$, $\rho_{in} = 0.6$, $\beta_{in} = 10^{-4}$ and $\Lambda= 0.0109$. In these plots we have adopted the units in which  $A_0=1$. We have plotted 4 representative values of $w$;   $\beta$ grows for $w>1/3$ at late times. While all the 4 cases have essentially the same $H(t)$ they differ in evolution of $\beta$.}
    \label{fig:w-Lambda-dipole-model}
\end{figure}

\section{Discussion and outlook}

We have formulated and analysed dipole cosmology, which is constructed to accommodate a cosmic bulk flow in a minimal generalization of the FLRW framework. Our theoretical analyses is based upon Einstein field equations while allowing for a tilt in the energy momentum tensor of the cosmic fluid. It reveals a few important facts. The tilt can remain large and even grow in time, while geometry isotropises (cosmic shear dying off fast), illustrating that FLRW cosmology is unstable against tilt perturbations. This claim is true even in an accelerating Universe, in particular one containing a positive cosmological constant. It is in fact quite generic in Universes where $w(t) \rightarrow -1$ at late times. {We also uncovered another interesting feature in some classes of dipole cosmology models \cite{KMS}: while $\beta$ goes to zero at very late times, there could be intermediate stages where the tilt can grow}. While these results are not in the context of a realistic model, they may be relevant for various claims about late time flows and dipole anisotropies in the recent literature. 

 %Such titled perturbations can and does naturally appear in cosmology during course of evolution of the Universe. Such perturbations can lead to a generic dipole cosmology as we  constructed here. 

Our findings may seem at odds with the common lore, which is based on intuitions gained from Wald's celebrated cosmic no-hair theorem \cite{Wald-cosmic-no-hair} and the fact that inflation leaves us in a Universe in which shear is suppressed. The cosmic evolution afterward does not provide sources for the shear and it remains small in the course of cosmic history. Our analysis emphasizes that tilt and shear are two separate concepts, and our dynamical equations \eqref{EoM-H-sigma} reveal that $\sigma\ll 1$ does not imply $\beta\ll 1$ and in fact $\beta$ can be or become of order 1. These results are compatible with the usual lore, but they highlight the important distinction between having an (almost) isotropic and homogeneous metric and having a homogeneous and isotropic cosmology, which is usually taken to be synonymous. In particular, an observer living in a flowing fluid of galaxies will see a homogeneous Universe with a dipole anisotropy even if the shear ansiotropies in the metric are small \cite{cosmo}. Viewed in this light, the  CMB observations which are usually taken as observational confirmation of the usual FLRW cosmologies (modulo CMB anomalies \cite{Beyond-FLRW-review, CMB-anomalies}), primarily indicate isotropy of the background metric over which the light propagates and does not exclude bulk flows and cosmic dipoles. 

Our analysis prompts several theoretical  and observational questions, the most important of which being instability of the homogeneous and isotropic cosmologies to homogeneous tilt perturbations; even if one starts with an isotropic Universe after inflation, cosmic evolution may develop a sizable bulk flow or cosmic dipole. It may be interesting to consider models with mixtures of flowing fluids as a setting for model building in dipole cosmology. This will provide a very simple model building paradigm that generalizes the FLRW setting. In the present work, we loosely attributed the tilt $\beta(t)$ to bulk flows and to cosmic dipoles. But in order to properly connect to observations, one should make this attribution precise by developing cosmography in the dipole cosmology setting. At the observational level, it may be interesting to extend the analysis of current and future data to search for traces of a dipole or a bulk flow, see \cite{Beyond-FLRW-review} for a recent review. %Moreover, one needs to revisit the usual data analysis methods and software, and adapt them to dipole cosmology.  %and finally, try to propose observational setups  and data analysis dedicated to searches for cosmic dipoles. 

%Our results are indicative, but But it is sufficiently suggestive, that it invites the whole cosmology community, theoretical and observational alike, to revisit the FLRW framework. %While the hints from our current work are inconclusive, this may lead to a rewriting of the field.



\section*{Acknowledgments}


We thank Ehsan Ebrahimian, Rajeev Jain, Roya Mohayaee, Eoin O Colgain, Mohamed Rameez and Subir Sarkar for discussions. The work of MMShJ is supported in part by SarAmadan grant No ISEF/M/401332. 

%\appendix 

%\section{Field equations in terms of different variables}\label{sec:appendix}



\begin{thebibliography}{99}

\bibitem{Copernicus}
https://en.wikipedia.org/wiki/De\_revolutionibus\_orbium\_coelestium
%215 citations counted in INSPIRE as of 12 Jul 2022

%
%\bibitem{Strominger:2001pn}
%A.~Strominger,
%``The dS / CFT correspondence,''
%JHEP \textbf{10} (2001), 034
%doi:10.1088/1126-6708/2001/10/034
%[arXiv:hep-th/0106113 [hep-th]].
%1058 citations counted in INSPIRE as of 15 Jul 2022

%\cite{WeinbergOldBook}Ellis-Maartens-McCallum--Book
\bibitem{WeinbergOldBook}
S.~Weinberg,
``Gravitation and Cosmology: Principles and Applications of the General Theory of Relativity,'' John Wiley and Sons, 1972;
``Cosmology,'' Oxford University press, 2008.

%215 citations counted in INSPIRE as of 12 Jul 2022
\bibitem{Ellis-Maartens-McCallum--Book}
Ellis, G. F. R., Maartens, R. and MacCallum, M. A. H. (2012). Relativistic cosmology. Cambridge University Press. https://doi.org/10.1017/CBO9781139014403


\bibitem{Hubble-1929}
E. Hubble, ``A relation between distance and radial velocity among extra-galactic nebulae,'' PNAS USA 15 (1929) 168–173.


%\bibitem{Planck-2018}
%N.~Aghanim \textit{et al.} [Planck],
%``Planck 2018 results. VI. Cosmological parameters,''
%Astron. Astrophys. \textbf{641} (2020), A6
%[erratum: Astron. Astrophys. \textbf{652} (2021), C4]
%doi:10.1051/0004-6361/201833910
%[arXiv:1807.06209 [astro-ph.CO]].
%8165 citations counted in INSPIRE as of 21 Jul 2022



%\bibitem{SNe}
%D.~M.~Scolnic \textit{et al.} [Pan-STARRS1],
%``The Complete Light-curve Sample of Spectroscopically Confirmed SNe Ia from Pan-STARRS1 and Cosmological Constraints from the Combined Pantheon Sample,''
%Astrophys. J. \textbf{859} (2018) no.2, 101
%doi:10.3847/1538-4357/aab9bb
%[arXiv:1710.00845 [astro-ph.CO]].
%1356 citations counted in INSPIRE as of 29 Aug 2022

%\bibitem{BAO}
%J.~Magana, M.~H.~Amante, M.~A.~Garcia-Aspeitia and V.~Motta,
%``The Cardassian expansion revisited: constraints from updated Hubble parameter measurements and type Ia supernova data,''
%Mon. Not. Roy. Astron. Soc. \textbf{476} (2018) no.1, 1036-1049
%doi:10.1093/mnras/sty260
%[arXiv:1706.09848 [astro-ph.CO]].
%101 citations counted in INSPIRE as of 29 Aug 2022


%\bibitem{Intro1} 
%  A.~G.~Riess, S.~Casertano, W.~Yuan, L.~M.~Macri and D.~Scolnic,
%  ``Large Magellanic Cloud Cepheid Standards Provide a 1\% Foundation for the Determination of the Hubble Constant and Stronger Evidence for Physics beyond $\Lambda$CDM,''
%  Astrophys.\ J.\  {\bf 876}, no. 1, 85 (2019)
%  doi:10.3847/1538-4357/ab1422
%  [arXiv:1903.07603 [astro-ph.CO]].
  %%CITATION = doi:10.3847/1538-4357/ab1422;%%
  %290 citations counted in INSPIRE as of 10 Feb 2020
  
 %\cite{Verde:2019ivm}
\bibitem{crisis}
L.~Verde, T.~Treu and A.~G.~Riess,
``Tensions between the Early and the Late Universe,''
Nature Astron. \textbf{3}, 891
%doi:10.1038/s41550-019-0902-0
[arXiv:1907.10625 [astro-ph.CO]].
%613 citations counted in INSPIRE as of 15 Jul 2022

\bibitem{Intro0}
E.~Di Valentino, L.~A.~Anchordoqui, O.~Akarsu, Y.~Ali-Haimoud, L.~Amendola, N.~Arendse, M.~Asgari, M.~Ballardini, S.~Basilakos and E.~Battistelli, \textit{et al.}
``Cosmology intertwined II: The Hubble constant tension,''
Astropart. Phys. \textbf{131}, 102605 (2021)
%doi:10.1016/j.astropartphys.2021.102605
[arXiv:2008.11284 [astro-ph.CO]].
%144 citations counted in INSPIRE as of 30 Jan 2022  

 
\bibitem{Eleonora-et-al-review-1}
E.~Di Valentino, O.~Mena, S.~Pan, L.~Visinelli, W.~Yang, A.~Melchiorri, D.~F.~Mota, A.~G.~Riess and J.~Silk,
``In the realm of the Hubble tension\textemdash{}a review of solutions,''
Class. Quant. Grav. \textbf{38} (2021) no.15, 153001
%doi:10.1088/1361-6382/ac086d
[arXiv:2103.01183 [astro-ph.CO]].
%505 citations counted in INSPIRE as of 21 Jul 2022


%\bibitem{Eleonora-combined-analysis}
%E.~Di Valentino,
%``A combined analysis of the $H_0$ late time direct measurements and the impact on the Dark Energy sector,''
%Mon. Not. Roy. Astron. Soc. \textbf{502} (2021) no.2, 2065-2073
%doi:10.1093/mnras/stab187
%[arXiv:2011.00246 [astro-ph.CO]].
%54 citations counted in INSPIRE as of 21 Jul 2022

%\cite{Perivolaropoulos:2021jda}
\bibitem{Perivolaropoulos:2021jda}
L.~Perivolaropoulos and F.~Skara,
``Challenges for $\Lambda$CDM: An update,''
%doi:10.1016/j.newar.2022.101659
[arXiv:2105.05208 [astro-ph.CO]].
%178 citations counted in INSPIRE as of 21 Jul 2022


\bibitem{SNOWMASS-2022}
E.~Abdalla, G.~Franco Abell\'an, A.~Aboubrahim, A.~Agnello, O.~Akarsu, Y.~Akrami, G.~Alestas, D.~Aloni, L.~Amendola and L.~A.~Anchordoqui, \textit{et al.}
``Cosmology intertwined: A review of the particle physics, astrophysics, and cosmology associated with the cosmological tensions and anomalies,''
JHEAp \textbf{34} (2022), 49-211
doi:10.1016/j.jheap.2022.04.002
[arXiv:2203.06142 [astro-ph.CO]].
%64 citations counted in INSPIRE as of 21 Jul 2022


%\bibitem{H0-running}
%C.~Krishnan, E.~\'O.~Colg\'ain, M.~M.~Sheikh-Jabbari and T.~Yang,
%``Running Hubble Tension and a H0 Diagnostic,''
%Phys. Rev. D \textbf{103} (2021) no.10, 103509
%doi:10.1103/PhysRevD.103.103509
%[arXiv:2011.02858 [astro-ph.CO]].
%32 citations counted in INSPIRE as of 17 Jul 2022

%\cite{Krishnan:2022fzz}
%\bibitem{Krishnan:2022fzz}
%C.~Krishnan and R.~Mondol,
%``$H_0$ as a Universal FLRW Diagnostic,''
%[arXiv:2201.13384 [astro-ph.CO]].
%3 citations counted in INSPIRE as of 17 Jul 2022

%\cite{Krishnan:2021dyb}
\bibitem{Upper-bound-H0}
C.~Krishnan, R.~Mohayaee, E.~\'O.~Colg\'ain, M.~M.~Sheikh-Jabbari and L.~Yin,
``Does Hubble tension signal a breakdown in FLRW cosmology?,''
Class. Quant. Grav. \textbf{38} (2021) no.18, 184001
%doi:10.1088/1361-6382/ac1a81
[arXiv:2105.09790 [astro-ph.CO]];
%47 citations counted in INSPIRE as of 21 Jul 2022

C.~Krishnan, R.~Mohayaee, E.~\'O.~Colg\'ain, M.~M.~Sheikh-Jabbari and L.~Yin,
``Hints of FLRW breakdown from supernovae,''
Phys. Rev. D \textbf{105} (2022) no.6, 063514
%doi:10.1103/PhysRevD.105.063514
[arXiv:2106.02532 [astro-ph.CO]].
%37 citations counted in INSPIRE as of 29 Aug 2022











%\bibitem{Poulin:2018cxd}
%V.~Poulin, T.~L.~Smith, T.~Karwal and M.~Kamionkowski,
%``Early Dark Energy Can Resolve The Hubble Tension,''
%Phys. Rev. Lett. \textbf{122} (2019) no.22, 221301
%doi:10.1103/PhysRevLett.122.221301
%[arXiv:1811.04083 [astro-ph.CO]].
%441 citations counted in INSPIRE as of 15 Jul 2022 





\begin{comment}
\bibitem{SNe-dipole-anomaly}%SNe-hemisphere-anomaly}

J.~Colin, R.~Mohayaee, M.~Rameez and S.~Sarkar,
``Evidence for anisotropy of cosmic acceleration,''
Astron. Astrophys. \textbf{631} (2019), L13
%doi:10.1051/0004-6361/201936373
[arXiv:1808.04597 [astro-ph.CO]].
%109 citations counted in INSPIRE as of 21 Jul 2022

M.~Rameez and S.~Sarkar,
``Is there really a Hubble tension?,''
Class. Quant. Grav. \textbf{38} (2021) no.15, 154005
%doi:10.1088/1361-6382/ac0f39
[arXiv:1911.06456 [astro-ph.CO]].
%41 citations counted in INSPIRE as of 21 Jul 2022

R.~Mohayaee, M.~Rameez and S.~Sarkar,
``The impact of peculiar velocities on supernova cosmology,''
[arXiv:2003.10420 [astro-ph.CO]].
%17 citations counted in INSPIRE as of 21 Jul 2022

A.~K.~Singal,
``Peculiar motion of Solar system from the Hubble diagram of supernovae Ia and its implications for cosmology,''
[arXiv:2106.11968 [astro-ph.CO]].
%12 citations counted in INSPIRE as of 21 Jul 2022

N.~Horstmann, Y.~Pietschke and D.~J.~Schwarz,
``Inference of the cosmic rest-frame from supernovae Ia,''
[arXiv:2111.03055 [astro-ph.CO]].
%12 citations counted in INSPIRE as of 21 Jul 2022


\end{comment}

%\cite{Krishnan:2021jmh}
\bibitem{binning-fitting}

E.~\'O.~Colg\'ain, M.~M.~Sheikh-Jabbari, R.~Solomon, M.~G.~Dainotti and D.~Stojkovic,
``Putting Flat $\Lambda$CDM In The (Redshift) Bin,''
[arXiv:2206.11447 [astro-ph.CO]].
%10 citations counted in INSPIRE as of 14 Nov 2022

E.~\'O.~Colg\'ain, M.~M.~Sheikh-Jabbari and R.~Solomon,
``High Redshift $\Lambda$CDM Cosmology: To Bin or not to Bin?,''
[arXiv:2211.02129 [astro-ph.CO]].
%1 citations counted in INSPIRE as of 14 Nov 2022

\begin{comment}
\bibitem{Subir-et-al-QSO}
N.~J.~Secrest, S.~von Hausegger, M.~Rameez, R.~Mohayaee, S.~Sarkar and J.~Colin,
``A Test of the Cosmological Principle with Quasars,''
Astrophys. J. Lett. \textbf{908} (2021) no.2, L51
%doi:10.3847/2041-8213/abdd40
[arXiv:2009.14826 [astro-ph.CO]];

V.~V.~Makarov and N.~J.~Secrest,
``Testing the Cosmological Principle: Astrometric Limits on Systemic Motion of Quasars at Different Cosmological Epochs,''
Astrophys. J. Lett. \textbf{927} (2022) no.1, L4
%doi:10.3847/2041-8213/ac551d
[arXiv:2202.07536 [astro-ph.GA]];
%0 citations counted in INSPIRE as of 21 Jul 2022

%83 citations counted in INSPIRE as of 21 Jul 2022
%\bibitem{Secrest:2022uvx}
N.~Secrest, S.~von Hausegger, M.~Rameez, R.~Mohayaee and S.~Sarkar,
``A Challenge to the Standard Cosmological Model,''
[arXiv:2206.05624 [astro-ph.CO]].
%2 citations counted in INSPIRE as of 21 Jul 2022

A.~K.~Singal,
``Solar system peculiar motion from the Hubble diagram of quasars and testing the cosmological principle,''
Mon. Not. Roy. Astron. Soc. \textbf{511} (2022) no.2, 1819-1829
%doi:10.1093/mnras/stac144
[arXiv:2107.09390 [astro-ph.CO]].
%9 citations counted in INSPIRE as of 21 Jul 2022

\bibitem{QSO-hemisphere-anomaly}%Subir-et-al-QSO

O.~Luongo, M.~Muccino, E.~\'O.~Colg\'ain, M.~M.~Sheikh-Jabbari and L.~Yin,
``Larger H0 values in the CMB dipole direction,''
Phys. Rev. D \textbf{105} (2022) no.10, 103510
%doi:10.1103/PhysRevD.105.103510
[arXiv:2108.13228 [astro-ph.CO]];
%36 citations counted in INSPIRE as of 21 Jul 2022

E.~\'O.~Colg\'ain, M.~M.~Sheikh-Jabbari, R.~Solomon, G.~Bargiacchi, S.~Capozziello, M.~G.~Dainotti and D.~Stojkovic,
``Revealing Intrinsic Flat $\Lambda$CDM Biases with Standardizable Candles,''
[arXiv:2203.10558 [astro-ph.CO]].
%6 citations counted in INSPIRE as of 21 Jul 2022


\bibitem{radio-dipole}
C.~A.~P.~Bengaly, T.~M.~Siewert, D.~J.~Schwarz and R.~Maartens,
``Testing the standard model of cosmology with the SKA: the cosmic radio dipole,''
Mon. Not. Roy. Astron. Soc. \textbf{486} (2019) no.1, 1350-1357
%doi:10.1093/mnras/stz832
[arXiv:1810.04960 [astro-ph.CO]].
%15 citations counted in INSPIRE as of 21 Jul 2022

C.~A.~P.~Bengaly, R.~Maartens, N.~Randriamiarinarivo and A.~Baloyi,
``Testing the Cosmological Principle in the radio sky,''
JCAP \textbf{09} (2019), 025
%doi:10.1088/1475-7516/2019/09/025
[arXiv:1905.12378 [astro-ph.CO]].
%7 citations counted in INSPIRE as of 21 Jul 2022

\bibitem{Bulk-Flow}
A.~Kashlinsky, F.~Atrio-Barandela, D.~Kocevski and H.~Ebeling,
``A measurement of large-scale peculiar velocities of clusters of galaxies: results and cosmological implications,''
Astrophys. J. Lett. \textbf{686} (2009), L49-L52
%doi:10.1086/592947
[arXiv:0809.3734 [astro-ph]].
%240 citations counted in INSPIRE as of 21 Jul 2022

F.~Qin, D.~Parkinson, C.~Howlett and K.~Said,
``Cosmic Flow Measurement and Mock Sampling Algorithm of Cosmicflows-4 Tully\ensuremath{-}Fisher Catalog,''
Astrophys. J. \textbf{922} (2021) no.1, 59
%doi:10.3847/1538-4357/ac249d
[arXiv:2109.14808 [astro-ph.CO]].
%2 citations counted in INSPIRE as of 21 Jul 2022




\end{comment}

%\cite{Aluri:2022hzs}
\bibitem{Beyond-FLRW-review}
P.~K.~Aluri, P.~Cea, P.~Chingangbam, M.~C.~Chu, R.~G.~Clowes, D.~Hutsem\'ekers, J.~P.~Kochappan, A.~Krasi\'nski, A.~M.~Lopez and L.~Liu, \textit{et al.}
``Is the Observable Universe Consistent with the Cosmological Principle?,''
[arXiv:2207.05765 [astro-ph.CO]].
%0 citations counted in INSPIRE as of 21 Jul 2022


%\cite{Krishnan:2022qbv}
\bibitem{KMS}
C.~Krishnan, R.~Mondol and M.~M.~Sheikh-Jabbari,
``Dipole Cosmology: The Copernican Paradigm Beyond FLRW,''
[arXiv:2209.14918 [astro-ph.CO]].
%1 citations counted in INSPIRE as of 13 Nov 2022


\bibitem{King}
A.~R.~King and G.~F.~R.~Ellis,
``Tilted homogeneous cosmological models,''
Commun. Math. Phys. \textbf{31} (1973), 209-242
%doi:10.1007/BF01646266
%177 citations counted in INSPIRE as of 22 Jul 2022
%\cite{Ellis:1998ct}

\bibitem{Ellis-lectures}
G.~F.~R.~Ellis and H.~van Elst,
``Cosmological models: Cargese lectures 1998,''
NATO Sci. Ser. C \textbf{541} (1999), 1-116
%doi:10.1007/978-94-011-4455-1\_1
[arXiv:gr-qc/9812046 [gr-qc]].
%422 citations counted in INSPIRE as of 17 Jul 2022

%\cite{Wald:1983ky}
\bibitem{Wald-cosmic-no-hair}
R.~M.~Wald,
``Asymptotic behavior of homogeneous cosmological models in the presence of a positive cosmological constant,''
Phys. Rev. D \textbf{28} (1983), 2118-2120
%doi:10.1103/PhysRevD.28.2118
%656 citations counted in INSPIRE as of 20 Jul 2022

\bibitem{upcoming-1}
C.~Krishnan and R.~Mondol, \textit{Work in progress}.

\bibitem{cosmo}

E. Ebrahimian, C. Krishnan, R. Mondol, M.M. Sheikh-Jabbari, \textit{Work in progress}.

%\cite{Stewart:1967tz}
\bibitem{Stewart-Ellis}
J.~M.~Stewart and G.~F.~R.~Ellis,
``Solutions of Einstein's equations for a fluid which exhibit local rotational symmetry,''
J. Math. Phys. \textbf{9} (1968), 1072-1082
doi:10.1063/1.1664679
%128 citations counted in INSPIRE as of 17 Jul 2022

%\cite{MacCallum}
\bibitem{MacCallum}
G.~F.~R.~Ellis and M.~A.~H.~MacCallum,
``A Class of homogeneous cosmological models,''
Commun. Math. Phys. \textbf{12} (1969), 108-141
%doi:10.1007/BF01645908
%504 citations counted in INSPIRE as of 17 Jul 2022

\bibitem{Coley-2006}
A.~A.~Coley, S.~Hervik and W.~C.~Lim,
``Fluid observers and tilting cosmology,''
Class. Quant. Grav. \textbf{23} (2006), 3573-3591
%doi:10.1088/0264-9381/23/10/021
[arXiv:gr-qc/0605128 [gr-qc]].
%19 citations counted in INSPIRE as of 14 Nov 2022


%\cite{Schwarz:2015cma}
\bibitem{CMB-anomalies}
D.~J.~Schwarz, C.~J.~Copi, D.~Huterer and G.~D.~Starkman,
``CMB Anomalies after Planck,''
Class. Quant. Grav. \textbf{33} (2016) no.18, 184001
%doi:10.1088/0264-9381/33/18/184001
[arXiv:1510.07929 [astro-ph.CO]].
%231 citations counted in INSPIRE as of 21 Jul 2022






\end{thebibliography}
\end{document}



\bibitem{Ellis-Maartens}
G.~F.~R.~Ellis, H.~van Elst and R.~Maartens,
``General relativistic analysis of peculiar velocities,''
Class. Quant. Grav. \textbf{18} (2001), 5115-5124
%doi:10.1088/0264-9381/18/23/308
[arXiv:gr-qc/0105083 [gr-qc]].
%18 citations counted in INSPIRE as of 22 Jul 2022

