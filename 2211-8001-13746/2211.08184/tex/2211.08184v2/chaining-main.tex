\documentclass[11pt]{article}

\pdfoutput=1
\usepackage{fullpage}           % layout
\usepackage{amsfonts}           % \mathbb
\usepackage{amsthm, amssymb}             % proof
\usepackage{xspace}             % \xspace
\usepackage{graphicx}
\usepackage{adjustbox}          % \adjustbox 
\usepackage{multicol}            
\usepackage{url}
\usepackage{enumerate, enumitem}  %enumerate environment with optional argument
\usepackage{subcaption}
\usepackage[usenames]{xcolor} % for coordinating edits
\newcommand{\sout}[1]{\st{#1}}
%% \usepackage{hyperref}
\usepackage{amsmath, hyperref, nicefrac}
\usepackage[capitalize]{cleveref}
\usepackage{thm-restate}
\usepackage{parskip}
\usepackage{mathtools}
\usepackage{multirow}
\usepackage{algorithmic}
\usepackage{algorithm}
\usepackage{array}

\colorlet{darkgreen}{green!45!black}



\newcommand{\set}[1]{\{#1\}}
\newcommand{\etal}{et al.\xspace}
\newcommand{\eps}{\varepsilon}
\newcommand{\opt}{\text{OPT}}
\newcommand{\cost}{\text{cost}}
\newcommand{\calS}{\mathcal{S}}
\newcommand{\calP}{\mathcal{P}}
\newcommand{\calL}{\mathcal{L}}
\newcommand{\calR}{\mathcal{R}}
\newcommand{\calT}{\mathcal{T}}
\newcommand{\calE}{\mathcal{E}}

\newcommand{\by}{\mathbf{p}}

\newcommand{\rank}{\textbf{rank}}
\newcommand{\dist}{\text{dist}}
\newcommand{\R}{\mathbb{R}}
\newcommand{\E}{\mathbb{E}}
\newcommand{\pr}{\mathbb{P}}
%\newcommand{\calE}{\mathcal{E}}
\newcommand{\calF}{\mathcal{F}}
\newcommand{\calX}{\mathcal{X}}
\newcommand{\bI}{\bar{I}_i}
\newcommand{\cand}{\mathbb{C}}
\newcommand{\greedy}{\mathcal{A}}
\newcommand{\A}{\mathcal{A}}
\newcommand{\poly}{\text{poly}}
\newcommand{\alg}{\greedy}
\newcommand{\centers}{\mathcal{C}}
\newcommand{\coreset}{\Omega}
\newcommand{\offset}{F}
\newcommand{\weight}{f}
\newcommand{\inner}{R_I}
\newcommand{\out}{R_O}
\newcommand{\main}{R_M}
\newcommand{\size}{\Gamma}
\newcommand{\polylog}{\text{polylog}}


\DeclareMathOperator{\sign}{sign}
\DeclareMathOperator{\argmax}{argmax}
\DeclareMathOperator{\argmin}{argmin}

\newcommand{\valuedelta}{\frac{\log^2 1/\eps}{2^{O(z\log z)}\min(\eps^2, \eps^z)}\left(k \log |\cand| + \log \log (1/\eps) + \log(1/\pi)\right)}

\newtheorem{lemma}{Lemma}
\newtheorem{observation}{Observation}
\newtheorem{theorem}{Theorem}
%% \newtheorem{lemma}[theorem]{Lemma}
\newtheorem{corollary}[theorem]{Corollary}
\newtheorem{claim}[theorem]{Claim}
\newtheorem{definition}{Definition}
%% \newtheorem{axiom}[theorem]{Axiom}
\newtheorem{fact}{Fact}
%% \newtheorem{observation}[theorem]{Observation}
\newtheorem{proposition}[theorem]{Proposition}
\newtheorem{remark}{Remark}
\newtheorem{question}{Question}

\usepackage{wrapfig}
\newcommand{\erclogowrapped}[1]{%
\setlength\intextsep{0pt}%
\begin{wrapfigure}[3]{r}{#1*\real{1.1}}%
\includegraphics[width=#1]{LOGO_ERC-FLAG_EU_crop.jpg}%
\end{wrapfigure}%
}

\newcounter{sideremark}
\newcommand{\marrow}{\stepcounter{sideremark}\marginpar{$
\longleftarrow\scriptstyle\arabic{sideremark}$}}
\newcommand{\blue}[1]{{\color{blue}\bf #1}}
 \newcommand{\david}[1]{
 %  \ifdraft{
    \textsf{\blue{*** (David) \marrow #1 ***}}
 %  }
 %  \fi
 }

\newcommand{\red}[1]{{\color{red}\bf #1}}
 \newcommand{\vincent}[1]{
 %  \ifdraft{
    \textsf{\red{*** (Vincent) \marrow #1 ***}}
 %  }
 %  \fi
 }
 
 
 \newcommand{\green}[1]{{\color{green}\bf #1}}
 \newcommand{\chris}[1]{
 %  \ifdraft{
    \textsf{\green{*** (Chris) \marrow #1 ***}}
 %  }
 %  \fi
 }


\usepackage{amsmath}
\title{Improved Coresets for Euclidean $k$-Means\footnote{An extended abstract appeared at NeurIPS 2022.}}
\author{Vincent Cohen-Addad \and Kasper Green Larsen \and 
  David Saulpic
  \and
  Chris Schwiegelshohn \and Omar Ali Sheikh-Omar
}
\date{}


\begin{document}


\maketitle


\begin{abstract}
Given a set of $n$ points in $d$ dimensions, the Euclidean $k$-means problem (resp. the Euclidean $k$-median problem) consists of finding $k$ centers such that the sum of squared distances (resp. sum of distances) from every point to its closest center is minimized. The arguably most popular way of dealing with this problem in the big data setting is to first compress the data by computing a weighted subset known as a coreset and then run any algorithm on this subset. The guarantee of the coreset is that for any candidate solution, the ratio between coreset cost and the cost of the original instance is less than a $(1\pm \varepsilon)$ factor. The current state of the art  coreset size is $\tilde O(\min(k^{2} \cdot \varepsilon^{-2},k\cdot \varepsilon^{-4}))$ for Euclidean $k$-means and $\tilde O(\min(k^{2} \cdot \varepsilon^{-2},k\cdot \varepsilon^{-3}))$ for Euclidean $k$-median. The best known lower bound for both problems is $\Omega(k \varepsilon^{-2})$. In this paper, we improve the upper bounds $\tilde O(\min(k^{3/2} \cdot \varepsilon^{-2},k\cdot \varepsilon^{-4}))$ for $k$-means and $\tilde O(\min(k^{4/3} \cdot \varepsilon^{-2},k\cdot \varepsilon^{-3}))$ for $k$-median. In particular, ours is the first provable bound that breaks through the $k^2$ barrier while retaining an optimal dependency on $\varepsilon$. 
\end{abstract}


\section{Introduction}


Accurate estimates of posterior probabilities are crucial for neural networks in various Natural Language Processing (NLP) tasks~\cite{icml17,DBLP:conf/nips/Lakshminarayanan17}. For example, it would be helpful for humans if the models deployed in practice abstain or interact when they cannot make a decision with high confidence~\cite{DBLP:journals/jamia/JiangOKO12}. While Pre-trained Language Models (PLMs) have improved the performance of many NLP tasks~\cite{bert,roberta}, how to better avoid miscalibration is still an open research problem ~\cite{calibration_emnlp20,dan_roth_emnlp21}. 
\begin{table}[t!]
    \centering
    \begin{tabular}{l|p{0.65\columnwidth}}
    \hline

    %  Example 1: & It is \hlc[cyan!10]{a} \hlc[red!40]{warm} \hlc[red!60]{funny} \hlc[red!40]{engaging} \hlc[cyan!20]{film} . \\ \hline
     Positive & a fast \hlc[green!10]{funny} \hlc[green!40]{highly} \hlc[green!80]{enjoyable} movie.\\ \hline
    %  like a south of the border melrose place
     
     Negative & It's about \hlc[red!5]{following} your \hlc[green!10]{dreams} \hlc[red!10]{no} matter \hlc[red!5]{what} your \hlc[green!5]{parents} think.\\
    \hline
  \end{tabular}
    \caption{Two motivating examples with highlight explanations~\cite{SST}. The saturation of the colors signifies the magnitude. The confidence of the model should be easily recognized by looking at token attributions.}
    % \vspace{-4mm}
    \label{tab:example-m}
\end{table}
In this paper, we investigate if and how model explanations can help calibrate the model. 

Explanation methods have attracted considerable research interest in recent years for revealing the internal reasoning processes behind models~\cite{IG,Uncertainty_Aware_Attention,deeplift}. Token attribution scores generated by explanation methods represent the contribution to the prediction~\cite{diagnostic}. Intuitively, one can draw some insight for analyzing and debugging neural models from these scores if they are correctly attributed, as shown in Table~\ref{tab:example-m}. For example, when the model identifies a highly indicative pattern, the tokens involved would have high attribution scores for the predicted label and low attribution scores for other labels. Similarly, if the model has difficulty recognizing the inductive information of any class (i.e., the attribution scores are not high for any label), the model should not be highly confident. As such, the computed explanation of an instance could indicate the confidence of the model in its prediction to some extent.
 
Inspired by this, we propose a simple and effective method named \textbf{CME} that can be applied at training time and improve the performance of the confidence estimates. The estimated confidence measures how confident the model is for a specific example. Ideally, reasonable confidence estimates should have higher confidence for correctly classified examples resulting in higher attributions than incorrect ones. Hence, given an example pair during training with an inverse classification relationship, we regularize the classifier by comparing the wrong example's attribution magnitude and the correct example's attribution magnitude.

Our work is related to recent works on incorporating explanations into learning. Different from previous studies that leverage explanations to help users predict model decisions~\cite{DBLP:journals/corr/abs-2102-02201} or improve the accuracy~\cite{DBLP:conf/icml/RiegerSMY20}, we focus on answering the following question: \textit{are these explanations of black-box models useful for calibration?} If so, how should we exploit the predictive power of these explanations? Considering the model may be uninterpretable due to the nature of neural networks and limitations of explanation method~\cite{Fragile,DBLP:conf/nips/YehHSIR19}, a calibrated model by CME at least can output the unbiased confidence. Moreover, we exploit intrinsic explanation during training, which does not require designing heuristics~\cite{xiye1} and additional data augmentation~\cite{mixup21acl}.
% Are these explanations useful for calibrating the model?

We conduct extensive experiments using BERT~\cite{bert} and RoBERTa~\cite{roberta} to show the efficacy of our approach on three natural language understanding tasks (i.e., natural language inference, paraphrase detection, and commonsense reasoning) under In-Domain (ID) and Out-of-Domain (OD) settings. CME achieves the lowest expected calibration error without accuracy drops compared with strong SOTA methods, e.g.,~\citet{mixup21acl}. When combined with Temperature Scaling (TS)~\cite{icml17}, the expected calibration errors are further reduced as better calibrated posterior estimates under these two settings.


%\input{coreset}
\section{Preliminaries and Setup}
\label{sec:prelim}

First, we require the following basic notions. For a point $p\in \mathbb{R}^d$, we denote $\|p\|_2 = \sqrt{\sum_{i=1}^d p_i^2}$ to be the Euclidean norm of $p$ and $\|p\|_1 = \sum_{i=1}^d |p_i|$. The distinct number of points in a point set $P$ is denoted by $\|P\|_0$. Note that the true number of points $|P|$ may be larger than $\|P\|_0$ as different points may lie on the same coordinates.
Given a solution $\mathcal{S}$ consisting of at most $k$ centers, and any subset $P'\subset P$ we use $\cost(P',\calS):= \sum_{p\in P'} \cost(p,\calS) =\sum_{p\in P} \min_{s\in \calS} w_p\cost(p,s),$ where $\cost(p,s) = \|p-s\|^2$ for Euclidean $k$-means and $\cost(p,s) = \|p-s\|$ for Euclidean $k$-median and $w_p$ is a non-negative weight (in the basic case this simply $1$ whereas for the coreset it can be any non-negative number). 
To unify the notation, we will often write $\cost(p,s) = \|p-s\|^z$ where $z=1$ corresponds to $k$-median and $z=2$ corresponds to $k$-means.
We also denote by $v^{\calS} \in \mathbb{R}^{\|P\|_0}$ the cost vector associated with the point set $P$ and solution $\calS$, that is $v_p^{\calS} := w_p\cost(p,s)$. Note that $\|v^{\calS}\|_1 = \cost(P,\calS)$.
The classic coreset guarantee is to show that for any solution $\calS$ the designated coreset $\Omega$ satisfies
$$ |\cost(\Omega,\calS) - \cost(P,\calS)|\leq \varepsilon \cdot \cost(P,\calS).$$
We will later introduce an equivalent statement that uses cost vectors. It will also be convenient to consider coresets with an additive error $E$ which satisfy
$$ |\cost(\Omega,\calS) - \cost(P,\calS)|\leq \varepsilon \cdot \cost(P,\calS) + E.$$

\cite{CGSS22} showed that any coreset algorithm that works for instances with the following assumptions can be extended to general instances: 
\begin{description}
\item[Assumption 1:] $\|P\|_0 \in \poly(k,\varepsilon^{-1})$.
\item[Assumption 2:] $d\in O(\log(k/\varepsilon) \cdot \varepsilon^{-2})$.
\item[Assumption 3:] $w_p = 1$, for all $p\in P$. Note that this only applies to the weights of the original points; the coreset points will have different weights.
\item[Assumption 4:] There exists a solution $\greedy$ such that
\begin{enumerate}
%\item $\displaystyle \cost(P,\greedy) = O(1)\cdot \min_{\calS}\cost(P,\calS)$
\item $|\greedy| \in O(k)$.
\item For any two clusters $C_i$, $C_j$ induced by $\greedy$, $\cost(C_i,\greedy) \leq 2\cdot \cost(C_j,\greedy)$.
\item For any cluster $C_j$ induced by $\greedy$ and any two points $p,p'\in C_j$, $\cost(p,\greedy) \leq 2\cdot \cost(p',\greedy)$
\end{enumerate}
\end{description}

To keep this paper self contained, we will detail the validity of these assumptions at the end of this section.  

The sampling procedure is now very simple. Given that these aforementioned assumptions hold, we sample a points $p\in C_j$ with probability $\mathbb{P}_p :=\frac{1}{|C_j|}\cdot \frac{\cost(C_j,\greedy)}{\cost(P,\greedy)}$ and add it to the designated coreset $\Omega$. Furthermore, $p$ receives the weight $w_p = \frac{1}{\mathbb{P}_p}$. Overall, our basic cost estimator for any candidate solution $\calS$ is therefore
$$ \cost(\Omega,\calS):= \frac{1}{|\Omega|} \sum_{p\in \Omega} \cost(p,\calS) \cdot w_p.$$

It is routine to check that $\mathbb{E}[\cost(\Omega,\calS)] = \cost(P,\calS)$. The remainder of this section will be devoted to showing that $\Omega$ satisfies for all $\calS$
\begin{equation}
\label{eq:coresetguarantee}
|\cost(\Omega,\calS) - \cost(P,\calS)|\leq \frac{\varepsilon}{\log^2 \varepsilon^{-1}} \cdot \left( \cost(P,\calS) + \cost(P,\greedy)\right)
\end{equation}
Using the framework from \cite{CSS21}, this implies an $O(\varepsilon)$ coreset in general.

\subsection{Justification of the Assumptions}

To obtain the first assumption, we compute any coreset of size $\text{poly}(k,\varepsilon^{-1})$ in preprocessing. Constructions of these coresets are abundant in literature and any one would serve our needs.

To obtain the second assumption, we apply a terminal embedding on the coreset. A terminal embedding guarantees that for any point $p\in P$ and any point $q\in \mathbb{R}^d$, where $d$ is the dimension of the points of $P$, we have a mapping $f$ s.t.
$$ \|p-q\|^2 = (1\pm \varepsilon) \|f(p)-f(q)\|^2.$$ 
\cite{NaN18} showed that for any $n$-point set a terminal embedding of target dimension $\tilde{O}(\varepsilon^{-2}\log n )$ exists, which, combined with the first assumption, yields the desired target dimension.

To obtain the third assumption, we merely have to ensure that the weights of the coreset points are integers. A number of constructions satisfy this but a simple way of always enforcing this is to scale and round the weights (see Corollary 2 of \cite{CSS21}).

The fourth assumption follows from the preprocessing of \cite{CSS21}, see Sections 3.3 and 4.1 of that reference. Similarly, the same preprocessing, given that $\greedy$ is an $O(1)$-approximation, also shows that Eq \ref{eq:coresetguarantee} implies that the overall construction will be a coreset (subject to rescaling $\varepsilon$ by constant factors), see Section 4.2 of the aforementioned reference.
We must point out that a point set cannot always be decomposed into only sets that satisfy the aforementioned assumption. Nevertheless \cite{CGSS22} showed that every other case require only $\tilde{O}(k/\varepsilon^2)$ many sampled points (compared Lemmas 15 and 17 of that reference.)

Finally, we remark that these steps and assumptions immediately also apply to the $k$-median problem.


\section{Analysis}
\label{sec:main}

In this section we prove the following theorems.

\begin{theorem}
\label{thm:main}
For any set of points in $d$ dimensional Euclidean space, there exists a coreset for $k$-means clustering of size $\tilde{O}(k^{3/2} \varepsilon^{-2})$.
\end{theorem}

\begin{theorem}
\label{thm:main2}
For any set of points in $d$ dimensional Euclidean space, there exists a coreset for $k$-median clustering of size $\tilde{O}(k^{4/3} \varepsilon^{-2})$.
\end{theorem}

If not remarked upon, the analysis will holds for both problems.


We first describe the random process used to show concentration of the estimator. 


\subsection{Setting up the Chaining Analysis}

First, we observe that Eq.\ref{eq:coresetguarantee} is equivalent to showing 
$$\sup_{\calS} \frac{|\cost(\Omega,\calS) - \|v\|_1|}{\left( \cost(P,\calS) + \cost(P,\greedy)\right)}  \leq \frac{\varepsilon}{\log^{2}\varepsilon^{-1}}.$$
Our goal is to show that
$$\mathbb{E}_{\Omega} \left[\sup_{\calS} \frac{|\cost(\Omega,\calS) - \|v\|_1|}{\left( \cost(P,\calS) + \cost(P,\greedy)\right)}\right]  \leq \frac{\varepsilon}{\log^{2}\varepsilon^{-1}},$$
where $\mathbb{E}_{\Omega}$ is meant to denote the expectation over the randomness of $\Omega$. This implies that the desired guarantee holds with constant probability.

We now apply a standard symmetrization argument.
\begin{lemma}[Appendix B.3 of ~\cite{RudraW14}]
\label{lem:symmetrization}
Let $g_p$ be independent standard Gaussian random variables. Then.
$$\mathbb{E}_{\Omega}\underset{\calS}{\text{sup}} \left[\left\vert\frac{\frac{1}{|\Omega|}\sum_{p\in \Omega} \cost(p,\calS)\cdot w_p - \|v\|_1}{\left( \cost(P,\calS) + \cost(P,\greedy)\right)} \right\vert \right]  \leq  \sqrt{2\pi}\mathbb{E}_{\Omega}\mathbb{E}_{g}\underset{\calS}{\text{sup}} \left[\left\vert\frac{\frac{1}{|\Omega|}\sum_{p\in \Omega} \cost(p,\calS)\cdot w_p\cdot g_p}{\left( \cost(P,\calS) + \cost(P,\greedy)\right)}\right\vert\right].$$
\end{lemma}

It is therefore sufficient to show

\begin{equation}
\label{eq:main}
\mathbb{E}_{\Omega}\mathbb{E}_{g}\underset{\calS}{\text{sup}}\left[ \left\vert\frac{\frac{1}{|\Omega|}\sum_{p\in \Omega} \cost(p,\calS)\cdot w_p\cdot g_p}{\left( \cost(P,\calS) + \cost(P,\greedy)\right)} \right\vert\right]  \leq \frac{\varepsilon}{\sqrt{2\pi} \log^{2}\varepsilon^{-1}}.
\end{equation}

Let $z=1$ for Euclidean $k$-median and $2$ for Euclidean $k$-means.
We partition the clusters of any solution $\calS$ by type. We consider a cluster $C_j$ of type $T_i$ if for 
$$ 2^{i} \min_{p\in C_j} \cost(p,\greedy) \leq  \min_{p\in C_j} \min_{s\in \calS} \cost(p,\calS) \leq 2^{i+1} \min_{c\in \greedy} \cost(p,\greedy).$$
The number of clusters $C_j\in T_i$ are denoted by $k_i$.
If $C_j$ is of type $i\leq 3$, we say $C_j$ is of type $T_{small}$ and if $C_j$ is of type $i\geq \log \gamma \varepsilon^{-z}$, for a sufficiently large absolute constant $\gamma$, we say that $C_j$ is of type $T_{large}$.
Then, we show
\begin{eqnarray} 
\label{eq:small}
\mathbb{E}_{\Omega} \mathbb{E}_{g}  \left[\sup_{\calS} \left\vert\frac{\frac{1}{|\Omega|} \sum_{C_j \in T_{small}} \sum_{p\in C_j\cap \Omega}\cost(p,\calS)w_p\cdot g_p }{\left( \cost(P,\calS) + \cost(P,\greedy)\right)}\right\vert \right] \leq \frac{\varepsilon}{\sqrt{2\pi}\log^{3}\varepsilon^{-1}} \\
\label{eq:typei}
\mathbb{E}_{\Omega}\mathbb{E}_{g}  \left[\sup_{\calS} \left\vert\frac{\frac{1}{|\Omega|}\sum_{C_j \in T_{i}}\sum_{p\in C_j\cap \Omega}\cost(p,\calS)w_p\cdot g_p }{\left( \cost(P,\calS) + \cost(P,\greedy)\right)} \right\vert \right] \leq \frac{\varepsilon}{\sqrt{2\pi}\log^{3}\varepsilon^{-1}} \\
\label{eq:large}
\mathbb{E}_{\Omega}\mathbb{E}_{g} \left[\sup_{\calS} \left\vert \frac{\frac{1}{|\Omega|}\sum_{C_j \in T_{large}}\sum_{p\in C_j\cap \Omega}\cost(p,\calS)w_p\cdot g_p }{\left( \cost(P,\calS) + \cost(P,\greedy)\right)}\right\vert \right]  \leq \frac{\varepsilon}{\sqrt{2\pi}\log^{3}\varepsilon^{-1}}
\end{eqnarray}

Note that if Equation \ref{eq:typei} holds for $i \in \{3,\ldots,\log 1/\varepsilon\}$, this also implies Equation \ref{eq:main}, as the error from each type can only sum up in the worst case and there are at most $O(\log \varepsilon^{-1})$ many types.

The small and large types are comparatively simple to handle.

\begin{lemma}[Lemmas 15 and 16 of \cite{CGSS22}]
\label{lem:easy}
Let $|\Omega| \geq \kappa \frac{k}{\varepsilon^2} \log^{10}(k/\varepsilon)$ for some absolute constant $\kappa$. Then Equations \ref{eq:small} and \ref{eq:large} hold.
\end{lemma} 

Our main objective will be to prove the following lemma.

\begin{lemma}
\label{main:lemma}
Let $|\Omega| \geq \kappa_1 \frac{k^{1+ z/(z+2)}}{\varepsilon^2} \log^{10}(k/\varepsilon) \geq \kappa_2 \frac{k}{\varepsilon^2} \log^{10}(k/\varepsilon) \cdot \left(\frac{\min(k_i,2^{i}) \cdot 2^{i} k \cdot k_i }{(k+k_i\cdot 2^{i})^2}\right)$ for some absolute constants $\kappa_1$ and $\kappa_2$. Then Equation \ref{eq:typei} holds.
\end{lemma} 

 

%This now yields the desired bound on the coreset size.
%
%\begin{corollary}
%\label{cor:main}
%Let $|\Omega| \geq \kappa \frac{k^{1.5}}{\varepsilon^2} \log^9(k/\varepsilon)$ for some absolute constant $\kappa$. Then Equation \ref{eq:typei} holds.
%\end{corollary}
%\begin{proof}
%Suppose $k_i\leq \sqrt{k}$. Then $\left(\frac{\min(k_i,2^{i}) \cdot 2^{i} k \cdot k_i }{(k+k_i\cdot 2^{i})^2}\right) \leq \left(\frac{2^{i} k \cdot k_i^2}{k \cdot k_i \cdot 2^{i}}\right) \leq k_i \leq \sqrt{k}$. 
%
%Now suppose $k_i\geq \sqrt{k}$. Then $\left(\frac{\min(k_i,2^{i}) \cdot 2^{i} k \cdot k_i }{(k+k_i\cdot 2^{i})^2}\right) \leq \left(\frac{2^{2i} k \cdot k_i}{k_i^2 \cdot 2^{2i}}\right) \leq \frac{k}{k_i} \leq \sqrt{k}$.
%\end{proof}

Combining Lemma \ref{lem:easy} and Lemma \ref{main:lemma} then implies Theorem~\ref{thm:main}.

\subsection{Proof of Lemma \ref{main:lemma}}

The proof of Lemma \ref{main:lemma} mainly consists of defining a nested sequence of nets over cost vectors over which we apply a union bound. Roughly speaking, for any cost vector $v^{\calS}$, we aim to find an approximating cost vector $v'$ such that 
$$ |v_p^{\calS} - v_p'| \leq \varepsilon \cdot \sqrt{\cost(p,\calS)^{z-1}\cdot \cost(p,\greedy)^{3-z}}.$$
Thus, on closer inspection, we have an error proportionate to $\varepsilon\cdot \sqrt{\cost(p,\calS)\cdot \cost(p,\greedy)}$ for $k$-means and $\varepsilon\cdot \cost(p,\calS)$ for $k$-median.

This analysis differs from the terminal-embedding-based nets one used in \cite{CGSS22}, which aimed for an error of the order $\varepsilon \cdot \cost(p,\calS)$.

Suppose we have, for every $\varepsilon$, a suitable collection of approximating cost vectors $\mathbb{N}_{\log 1/\varepsilon}$ with this guarantee for any candidate $\calS$\footnote{The reason for indexing the net by $\mathbb{N}_{\log 1/\varepsilon}$ and not by $\mathbb{N}_{\varepsilon}$ is to conveniently sum over $\sum_{i=1}^{\infty} \log| N_{i}|$, rather than $\sum_{i=1}^{\infty} \log| N_{2^i}|$.}. Let $v^{\calS,\varepsilon}$ be the cost vector approximating $v^{\calS}$ in the net $\mathbb{N}_{\log 1/\varepsilon}$.  Then we can write
$$ v_p^{\calS} = \sum_{h=0}^{\infty} v_p^{\calS,2^{-(h+1)}} - v_p^{\calS,2^{-h}},$$
with $v_p^{\calS,1} = 0$.
Our goal is to now bound
\begin{eqnarray}
\nonumber
& & \mathbb{E}_{\Omega} \mathbb{E}_{g} \left[\sup_{\calS}  \left\vert\frac{\sum_{C_j \in T_{i}}\sum_{p\in C_j\cap \Omega}\cost(p,\calS)w_p\cdot g_p }{|\Omega|\cdot \left( \cost(P,\calS) + \cost(P,\greedy)\right)}  \right\vert \right]\\
\nonumber
&= & \mathbb{E}_{\Omega} \mathbb{E}_{g} \left[\sup_{v^{\calS}} \left\vert \frac{\sum_{h=0}^{\infty}\sum_{C_j \in T_{i}}\sum_{p\in C_j\cap \Omega}(v_p^{\calS,2^{-(h+1)}} - v_p^{\calS,2^{-h}})w_p\cdot g_p }{|\Omega|\cdot \left( \cost(P,\calS) + \cost(P,\greedy)\right)} \right\vert\right]  \\
\nonumber
&\leq &  \sum_{h=0}^{\infty} \mathbb{E}_{\Omega} \mathbb{E}_{g} \left[\sup_{v^{\calS,h+1}-v^{\calS,h}\in \mathbb{N}_{2^{-(h+1)}}\times \mathbb{N}_{2^{-h}}} \left\vert \frac{\sum_{C_j \in T_{i}}\sum_{p\in C_j\cap \Omega}(v_p^{\calS,2^{-(h+1)}} - v_p^{\calS,2^{-h}})w_p\cdot g_p }{|\Omega|\cdot \left( \cost(P,\calS) + \cost(P,\greedy)\right)} \right\vert \right]   \\
&=& \label{eq:base}
\mathbb{E}_{\Omega} \mathbb{E}_{g}  \left[  \sup_{v^{\calS,1}\in \mathbb{N}_{2^{-1}}} \left\vert \frac{\sum_{C_j \in T_{i}}\sum_{p\in C_j\cap \Omega}v_p^{\calS,2^{-1}}w_p\cdot g_p }{|\Omega|\cdot \left( \cost(P,\calS) + \cost(P,\greedy)\right)} \right\vert \right]   \\
\label{eq:telescopesmall}
&+& \sum_{h=1}^{\log \varepsilon^{-2}} \mathbb{E}_{\Omega} \mathbb{E}_{g}   \left[  \sup_{v^{\calS,h+1}-v^{\calS,h}\in \mathbb{N}_{h+1}\times \mathbb{N}_{h}}\left\vert \frac{\sum_{C_j \in T_{i}}\sum_{p\in C_j\cap \Omega}(v_p^{\calS,2^{-(h+1)}} - v_p^{\calS,2^{-h}})w_p\cdot g_p }{|\Omega|\cdot \left( \cost(P,\calS) + \cost(P,\greedy)\right)}  \right\vert \right] \\
\label{eq:telescopelarge}
&+& \sum_{h=\log \varepsilon^{-2}}^{\infty} \mathbb{E}_{\Omega} \mathbb{E}_{g}  \left[ \sup_{v^{\calS,h+1}-v^{\calS,h}\in \mathbb{N}_{h+1}\times \mathbb{N}_{h}}  \left\vert\frac{\sum_{C_j \in T_{i}}\sum_{p\in C_j\cap \Omega}(v_p^{\calS,2^{-(h+1)}} - v_p^{\calS,2^{-h}})w_p\cdot g_p }{|\Omega|\cdot \left( \cost(P,\calS) + \cost(P,\greedy)\right)} \right\vert  \right] 
\end{eqnarray}


We will bound Equations \ref{eq:base} and \ref{eq:telescopelarge} directly. For the $O(\log \varepsilon^{-1})$ equations in term \ref{eq:telescopesmall}, we prove a bound on each.
Thus, we aim for a bound of the order $O(\frac{\varepsilon}{\log^3 \varepsilon^{-1}})$; the overall bound then follows by summing up the errors and rescaling by constant factors.
Technically, bounding each of the terms in Equations \ref{eq:base}, \ref{eq:telescopesmall} and \ref{eq:telescopelarge} requires somewhat different arguments. 
For the sake of illustrating the key new ideas we first focus on Eq. \ref{eq:telescopesmall}. 

%In order to get a bound for Eq. \ref{eq:base}, we split the estimator into two parts as follows. First, let $q_j:=\frac{\sum_{p\in C_j}v_p^{\calS,2^{-1}}}{|C_j|}$. Now we consider
%\begin{eqnarray}
%\label{eq:firstest}
%& &\frac{1}{|\Omega|}\frac{\sum_{C_j \in T_{i}}\sum_{p\in C_j\cap \Omega}(v_p^{\calS,2^{-1}} - q_j)w_p}{\left( \cost(P,\calS) + \cost(P,\greedy)\right)}g_p \\
%\label{eq:secondest}
%& &+ \frac{1}{|\Omega|}\frac{\sum_{C_j \in T_{i}}\sum_{p\in C_j\cap \Omega}q_j \cdot w_p}{\left( \cost(P,\calS) + \cost(P,\greedy)\right)}g_p 
%\end{eqnarray}
%Thus, Equation~\ref{eq:base} becomes
%\begin{eqnarray}
%\nonumber
%& & \mathbb{E}_{\Omega} \mathbb{E}_{g}  \left[  \sup_{v^{\calS,1}\in \mathbb{N}_{2^{-1}}} \left\vert \frac{\sum_{C_j \in T_{i}}\sum_{p\in C_j\cap \Omega}  v_p^{\calS,2^{-1}}w_p}{|\Omega|\cdot \left( \cost(P,\calS) + \cost(P,\greedy)\right)}g_p \right\vert \right] \\
%\label{eq:base1}
%&=&\mathbb{E}_{\Omega} \mathbb{E}_{g}  \left[  \sup_{v^{\calS,1}\in \mathbb{N}_{2^{-1}}}\left\vert \frac{\sum_{C_j \in T_{i}}\sum_{p\in C_j\cap \Omega}|v_p^{\calS,2^{-1}}-q_j|w_p}{|\Omega|\cdot\left( \cost(P,\calS) + \cost(P,\greedy)\right)}g_p \right\vert \right] \\
%\label{eq:base2}
%&+& \mathbb{E}_{\Omega} \mathbb{E}_{g}  \left[  \sup_{v^{\calS,1}\in \mathbb{N}_{2^{-1}}} \left\vert\frac{\sum_{C_j \in T_{i}}\sum_{p\in C_j\cap \Omega}q_j\cdot w_p}{|\Omega|\cdot\left( \cost(P,\calS) + \cost(P,\greedy)\right)}g_p \right\vert \right]
%\end{eqnarray}
%
%Due to Assumption 4, we have $\cost(T_i,\calS) = O(1) \cdot k_i\cost(C_j,\calS)$, for any $C_j\in T_i$ Thus
%\begin{eqnarray}
%\nonumber
%& & \mathbb{E}_{\Omega} \mathbb{E}_{g}  \left[  \sup_{v^{\calS,1}\in \mathbb{N}_{2^{-1}}} \left\vert\frac{\sum_{C_j \in T_{i}}\sum_{p\in C_j\cap \Omega}q_j\cdot w_p}{|\Omega|\cdot\left( \cost(P,\calS) + \cost(P,\greedy)\right)}g_p \right\vert \right] \\
%\nonumber
%&\leq & \mathbb{E}_{\Omega} \mathbb{E}_{g}  \left[  \sup_{v^{\calS,1}\in \mathbb{N}_{2^{-1}}} \left\vert \frac{\sum_{C_j \in T_{i}}\sum_{p\in C_j\cap \Omega}q_j\cdot w_p}{|\Omega|\cdot  \cost(T_i,\calS) }g_p \right\vert \right]\\
%\label{eq:base3}
%&\leq & \mathbb{E}_{\Omega} \mathbb{E}_{g}  \left[  \sup_{v^{\calS,1}\in \mathbb{N}_{2^{-1}}} \left\vert \max_{C_j \in T_{i}}  \frac{\sum_{p\in C_j\cap \Omega}q_j\cdot w_p}{|\Omega|\cdot  \cost(C_j,\calS) }g_p \right\vert \right]
%\end{eqnarray}

The next section presents the nets for the cost vectors. The subsequent section  bounds the variance. The final section combines these results and completes the proof of Lemma \ref{main:lemma}.

\subsubsection*{Cost Vector Nets}

\begin{definition}\label{def:clusteringnets}
Let $I$ be a metric space, $P$ a set of points, $k$ a positive integer, and let $\alpha > 0$ be a precision parameters and let $\greedy$ be some solution with at most $k'$ centers.  Let $\mathbb{C}\subset I^k$ be a (potentially infinite) set of candidate $k$-clusterings. 
We say that a set of cost vectors $\mathbb{N}\subset \R^{|P|}$ is an $(\alpha,k)$-clustering net if for every $\calS\in \mathbb{C}$ there exists a vector $v' \in \mathbb{N}$ such that the following condition holds.
For all $p \in P$, 
$$|v^{\calS}_p - v'_p| \leq \alpha\cdot \sqrt{\cost(p,\calS)^{z-1}\cdot \cost(p,\greedy)^{3-z}}.$$
\end{definition}

These clustering nets have a substantially smaller error than those proposed in \cite{CGSS22}, which had an error of the order $\alpha\cdot \left(\cost(p,\calS) + \cost(p,\greedy)\right)$. 

Given a set of points $X$ in Euclidean space, an $\varepsilon$-net is a subset $S\subset X$ such that for every  $p\in X$ there exists a $q$ in $S$ with $\|p-q\|\leq \varepsilon$. Throughout this section, we will frequently use the fact that in $d$ dimensions, there exists an $\varepsilon$-net of cardinality $(1+2/\varepsilon)^d$ (see for example \cite{Pis99}).
%\begin{lemma}[\cite{Pis99}]
%\label{lem:Euclideannets}
%For unit Euclidean ball of dimension $d$ centered around the origin, there exists an $\varepsilon$-net of cardinality $(1+2/\varepsilon)^d$.
%\end{lemma}
Our main goal in this section is to prove the following lemma.

%\begin{lemma}[Compare Lemma 22 of \cite{CGSS22}]
%\label{lem:netsizelarge}
%Let $P$ be a set of points in $d$ dimensional Euclidean space, $k$ a positive integer and $\greedy$ be a candidate solution. 
%Define $\cand$ to be the set of possible candidate centers such that the clusters induced by $\greedy$ are of type $i$, with $3\leq i \leq \log 1/\varepsilon^2$.
%For all $\alpha \leq 1/2$, there exists an $(\alpha,k)$-clustering net $\mathbb{N}$ of $\cand$ with 
%$$|\mathbb{N}|\leq \exp\left(\gamma\cdot k\cdot d \cdot i\log(4/\alpha)\right),$$
%where $\gamma$ is an absolute constant.  
%\end{lemma}
%\begin{proof}
%The only difference to Lemma 22 of \cite{CGSS22} is that the nets are required to have an error of $\alpha\cdot \sqrt{\cost(p,\calS)\cost(p,\greedy)}$ rather than $\alpha\cdot \left(\cost(p,\calS)+ \cost(p,\greedy)\right)$. This can be done by rescaling $\varepsilon$ by $2^{-i}$, which in turn is absorbed by the constant $\gamma$ as $2^ \leq O(1)\cdot \varepsilon^{-2}$.
%\end{proof}
%
%The main new result is as follows.

\begin{lemma}
\label{lem:netsize}
Let $P$ be a set of points in $d$ dimensional Euclidean space, $k$ a positive integer, $\greedy$ be a candidate solution with $k_i$ clusters and $\gamma$ and absolute constant. 
Define $\cand$ to be the set of possible candidate centers such that the clusters induced by $\greedy$ are of type $i$, with $3\leq i \leq \log 1/\varepsilon^z$.
For all $\alpha \leq 1/2$, there exists an $(\alpha,k)$-clustering net $\mathbb{N}$ of $\cand$ with 
$$|\mathbb{N}|\leq \exp\left(\gamma \cdot k \cdot \log \|P\|_0  \cdot \min(k_i + \alpha^{-2}, \alpha^{-2}\cdot 2^i) \cdot i\log\frac{1}{\alpha})\right).$$
\end{lemma}
\begin{proof}[Proof]
We first show that given a set of vectors $P$ and any vector $s$, there always exists a small subset $U$ of $P$ such that all inner products between $p\in P$ and $s$ are preserved by the span of $U$.


\begin{lemma}\label{lem:innerproduct}
Let $P = \{p_1, ..., p_n\} \subseteq \R^d$ and let $s\in \mathbb{R}^d$. Then there exists $U \subseteq P$, with $|U| = O(\eps^{-2})$ and orthogonal basis $\Pi_U$, such that
\begin{equation}
    \label{eq:desire}
\forall p \in P, |p^T (I-\Pi_U\Pi_U^T) s| \leq \eps \|(I-\Pi_U\Pi_U^T) p\| \cdot \min_{p\in P} \|p-s\|
\end{equation}
\end{lemma}
\begin{proof}
Start with $U_0 = \text{argmin}_{p\in P} \|p-s\|$, and proceed in rounds. 
Note that $\|(I-\Pi_{U_0}\Pi_{U_0}^T)s\| \leq \|p-s\|$ for all $p\in P$.

In each round $i$, denote the current set of vectors $U_i$ with orthogonal basis $\Pi_{U_i}$. We add a vector $p_i$ if the following equation holds
$$|p^T (I-\Pi_{U_i}\Pi_{U_i}^T) s| \geq \eps \|(I-\Pi_{U_i}\Pi_{U_i}^T) p\| \cdot \|(I-\Pi_{U_0}\Pi_{U_0}^T)s\|.$$
We observe that if this equation holds for all $p\in P$, then Equation \ref{eq:desire} must also hold.
Note that $(I-\Pi_{U_i}\Pi_{U_i}^T)p$ is orthogonal to the span of $U_{i}$ of all previously added vectors.
Thus, due to the Pythagorean theorem, we have
$$\sum_{i}^{t} \left(\frac{(p^T (I-\Pi_{U_{i-1}}\Pi_{U_{i-1}}^T) s)}{\|(I-\Pi_{U_{i-1}}\Pi_{U_{i-1}}^T) p\| \cdot \|(I-U_0U_0^T)s\|}\right)^2 \geq t \cdot \varepsilon^2.$$
Therefore, after $t=\varepsilon^{-2}$ many rounds $(I-\Pi_U\Pi_U^T) s = 0,$
which implies that after at most $\varepsilon^{-2}$ rounds Eq. \ref{eq:desire} has to hold.
\end{proof}

With this lemma, we can prove our net bound.
Our objective is to generate a small set of cost vectors that satisfy the desired guarantee.
Throughout this proof, let  $\dist(p,\greedy)= \cost(p,\greedy)^{1/z}$ be the distance of $p$ to its center in $\greedy$.
We first define the cost vectors. For each subset $U$ of size $O(\min(\alpha^{-2}2^i,\alpha^{-2}+k_i)$, we consider the the subspace $\Pi_{U}$ spanned by $U$. In this subspace we consider $(\alpha/2^i)\cdot \dist(p,\greedy)$-nets of every ball centered around $\Pi_{U}p$ with radius $60\cdot 2^i/2 \cdot \dist(p,\greedy)$ for all $p\in P$. Such a net has size $\exp(\gamma\cdot \rank(U)i \log \alpha)$, for some constant $\gamma$ and there exist at most $\|P\|_0 \cdot \exp(\gamma\cdot |U| i \log \alpha)$ many such nets. Furthermore, there are at most ${\|P\|_0 \choose |U|} \leq \|P\|_0^{|U|}$ such subsets.

Now, for every point $p$, define an exponential sequence $\alpha^2 (1+\alpha/2^i)^j$ for $j\in \{0,\ldots \log 10 \cdot 2^i\}$. There exist at most $\|P\|_0$ such sequences and every such sequence consists of at most $O(\alpha^{-1}\cdot 2^i \cdot i)$ many values.
We combine every net point in ever ball of every subspace with all values in the exponential sequence to obtain the evaluation for a single candidate center. The overall number of candidate centers is therefore of the order $\|P\|_0^{|U|}\cdot \exp(\gamma\cdot |U| i \log \alpha)$, for a sufficiently large $\gamma$. The overall number of candidate cost vectors is now the number of $k$ subsets of candidate centers, i.e. $\|P\|_0^{k\cdot |U|}\cdot \exp(\gamma\cdot k\cdot |U| i \log \alpha)$. Combined with the bounds on $U$, this yields the desired size. What remains to be shown is that the thus constructed cost vectors are a $(\alpha,k)$-clustering net.

Here, we use that for any center $s$ in some candidate solution $\calS$
$$\|p-s\|^2 = \|\Pi_U(p-s)\|^2 + \|(I-\Pi_U\Pi_U^T)p\|^2 + \|(I-\Pi_U\Pi_U^T)s\|^2 - 2p^T(I-\Pi_U\Pi_U^T)s.$$
The nets for the span of $\Pi_U$ are so fine that the distance $\|\Pi_Us-s'\|^2$ is essentially negligible compared to the maximum error incurred by $2p^T(I-\Pi_U\Pi_U^T)s$, where $s'$ is the point in the span of $\Pi_U$ closest to $\Pi_Us$ and the same holds for the exponential sequence approximating the term $\|(I-\Pi_U\Pi_U^T)s\|^2$. Thus, the error is dominated by $2p^T(I-\Pi_U\Pi_U^T)s$. Now, we can assume that the input point closest to $s$ is included in $U$. Then $\min_{p\in P}\|p-s\| \leq O(1)\cdot 2^{i/z}\cdot \cost(p,\greedy)^{1/z}$ and $\|(I-\Pi_U\Pi_U^T)p\| \leq \cost(p,\calS)^{1/z} \leq O(1)\cdot 2^{i/z}\cost(p,\greedy)^{1/z}$.
If $\alpha^{-2}\cdot 2^i < k_i + \alpha^{-2}$, we have
$$|p^T(I-\Pi_U\Pi_U^T)s| \leq  \alpha \cdot 2^{-i/2} \cdot \|(I-\Pi_U\Pi_U^T) p\| \cdot \min_{p\in P}\|p-s\| \leq O(1)\cdot \alpha\cdot \cost(p,\calS)^{z-1}\cost(p,\greedy)^{3-z}$$
otherwise we have $\|(I-\Pi_U\Pi_U^T)p\| \leq \cost(p,\greedy)^{1/z}$ which implies
$$|p^T(I-\Pi_U\Pi_U^T)s| \leq  \alpha \cdot \|(I-\Pi_U\Pi_U^T) p\| \cdot \min_{p\in P}\|p-s\| \leq \alpha\cdot \cost(p,\calS)^{z-1}\cost(p,\greedy)^{3-z}.$$
Rescaling $\alpha$ by constant factors yields the claim.
%
%Let $\calS$ be a candidate solution and let $\calS'$ be the subset of $\calS$ serving the points of $P$.
%For a center $s\in \calS'$, let $U_1$ be the subset of vectors of $P$ given by Lemma \ref{lem:innerproduct} with $\varepsilon:= \alpha\cdot 2^{i/2}$ and let $U_2$ be the subset of vectors of $P$ given by Lemma \ref{lem:innerproduct} with $\varepsilon:= \alpha$ and let $U_2' = U_2 \cup \greedy$. Let $U = \begin{cases}U_1 &\text{if } rank(U_1) \leq rank(U_2') \\ U_2' & \text{else}\end{cases}$ and let $\Pi_{U}$ be the orthogonal matrix of the span of $U$. 
%
%Suppose, for every point $p$ in the span of $U$, we have are given an $(\alpha/2^i)\cdot \sqrt{\cost(p,\greedy)}$-net of the ball centered around $\Pi_{U}p$ with radius $60\cdot 2^i/2 \cdot \sqrt{\cost(p,\greedy)}$. Such a net has size $\exp(\gamma\cdot \rank(U)i \log \alpha)$, for some constant $\gamma$ and there exist at most $\|P\|_0 \cdot \exp(\gamma\cdot \rank(S)i \log \alpha)$ many such nets. 
%Denote by $s'$ the point in these nets closest to $s\Pi_{U}$.
%We now argue that
%\begin{eqnarray}
%\label{eq:netclose}
%\|Pi_Us-s'\| &\leq &\alpha \cdot \sqrt{\cost(p,\greedy)}  \text{ if } \|Pi_{U}(p-s)\|\leq 60\cdot 2^i/2 \sqrt{\cost(p,\greedy)} \\
%\nonumber
%\|Pi_Us-s'\| &\geq & 10 2^{i/2} \cdot \sqrt{\cost(p,\greedy)}  \text{ if } \|Pi_{U}(p-s)\|\geq 20\cdot 2^i/2 \sqrt{\cost(p,\greedy)} 
%\end{eqnarray}
%Let $p'\in P$ be the point with minimal $\cost(p,\greedy)$ satisfying the first condition. Then for any point $p$ that also satisfies this condition $\|\Pi_{U}s-s'\| \leq \alpha\cdot 2^{-i}2^{i/2} \cdot \sqrt{\cost(p',\greedy)}  \leq \alpha \cdot \sqrt{\cost(p',\greedy)} \leq \alpha \cdot \sqrt{\cost(p,\greedy)}$.
%
%Conversely, suppose $p$ satisfies the first condition then $\|Pi_{U}(p-s)\|\geq 20\cdot 2^i/2 \sqrt{\cost(p,\greedy)}$. Let $q$ be the point in the $(\alpha/2^i)$-net of the ball centered around $\Pi_{U}p$ with radius $20\cdot 2^i$ closest to $\Pi_Us$, which implies $\|q-\Pi_Up\| \geq 20\cdot 2^i/2 \cdot \sqrt{\cost(p,\greedy)} - (\alpha/2^i)\cdot \sqrt{\cost(p,\greedy)} \geq 10\cdot 2^i/2 \cdot \sqrt{\cost(p,\greedy)}$.
%
%Next, we consider a net over all values $\|(I-P_UP_U^T)s\|$. For every point $p$, define an exponential sequence $\alpha^2 (1+\alpha/2^i)^j$ for $j\in \{0,\ldots \log 10 2^i\}$. There exist at most $\|P\|_0$ such sequences and every such sequence consists of at most $O(\alpha^{-1}\cdot 2^i \cdot i)$ many values. Similar to the bound above, we will show that there exists an $s''$ in the union of sequences such that
%\begin{eqnarray}
%\label{eq:netsingle}
%\vert s''-\|(I-P_UP_U^T)s\| \vert &\leq &\alpha \cdot \sqrt{\cost(p,\greedy)}  \text{ if } \|(I-P_UP_U^T)s\|\leq 60\cdot 2^i/2 \sqrt{\cost(p,\greedy)} \\
%\nonumber
%\vert s'-\|(I-P_UP_U^T)s\| \vert &\geq & 10 2^{i/2} \cdot \sqrt{\cost(p,\greedy)}  \text{ if } \|(I-P_UP_U^T)s\|\geq 20\cdot 2^i/2 \sqrt{\cost(p,\greedy)}
%\end{eqnarray}
%This statement now holds due to the triangle inequality and by definition of the exponential sequence.
%
%We now conclude for a single center $s$. Assume that the point in $P$ closest to $s$ is included in $U$. Let $s'$ be the point in the span of $U$ satisfying Eq. \ref{eq:netclose} and let $s''$ be the element in the exponential sequence satisfying Eq. \ref{eq:netsingle}. We wish to show that for some absolute constant $\beta$
%\begin{eqnarray}
%\nonumber
%\vert \|p-s\|^2 &-& \left(\|\Pi_Up-s'\|^2 + \|(I-\Pi_U\Pi_U^T)p\|^2 + s'' \right) \vert \\
%\label{eq:netboth}
%&\leq &\beta \cdot \alpha \cdot \sqrt{\cost(p,\greedy)\|p-s\|^2}  \text{ if } \|p-s\| \leq 60\cdot 2^i/2 \sqrt{\cost(p,\greedy)} \\
%\nonumber
%\vert \|p-s\|^2 &-& \left(\|\Pi_Up-s'\|^2 + \|(I-\Pi_U\Pi_U^T)p\|^2 + s'' \right) \vert \\
%\nonumber
% &\geq & 10 2^{i/2} \cdot \sqrt{\cost(p,\greedy)}  \text{ if } \|p-s\| \geq 60\cdot 2^i/2 \sqrt{\cost(p,\greedy)}
%\end{eqnarray}
%First, we decompose $\|p-s\|^2$ as follows. We have
%\begin{eqnarray*}
%\|p-s\|^2 &=& \|\Pi_{U}(p-s)\|^2 + \|(I-\Pi_{U}\Pi_{U}^T)(p-s)\|^2 \\
%&\leq & \|\Pi_{U}(p-s)\|^2 + \|(I-\Pi_{U}\Pi_{U}^T)p\|^2 +\|(I-\Pi_{U}\Pi_{U}^T)s\|^2 - 2p^T(I-\Pi_{U}\Pi_{U}^T)s
%\end{eqnarray*}
%We first focus on the case that $\|p-s\| \leq 60\cdot 2^i/2 \sqrt{\cost(p,\greedy)}$. Since projection only decreases the norm, this implies that $\|\Pi_{U}(p-s)\|$ and $\|(I-\Pi_{U}\Pi_{U}^T)(p-s)\|$ are less than $60\cdot 2^i/2 \sqrt{\cost(p,\greedy)}$ as well. This implies that Equations \ref{eq:netclose} and \ref{eq:netboth} hold, which implies
%\begin{equation*}
%\vert \|p-s\|^2 - \left(\|\Pi_Up-s'\|^2 + \|(I-\Pi_U\Pi_U^T)p\|^2 + s'' \right) \vert \leq \alpha \cdot \sqrt{\cost(p,\greedy)} + |2p^T(I-\Pi_{U}\Pi_{U}^T)s|
%\end{equation*}
%Since the point closest to $s$ is contained in $U$, we have $\|(I-\Pi_U\Pi_U^T)s\| \leq \sqrt{\cost(p,s)}$. Suppose $|U| = \alpha^{-2} \cdot 2^i$. Due to the triangle inequality $\|(I-\Pi_U\Pi_U^T)p\| \leq 2 \|p-s\| \leq 4 \cdot 2^{i/2} \sqrt{\cost(p,\greedy)}$. Then Lemma \ref{lem:innerproduct} implies that $p^T(I-\Pi_{U}\Pi_{U}^T)s \leq \alpha\cdot 2^{-i/2} \|(I-\Pi_U\Pi_U^T)p\| |(I-\Pi_U\Pi_U^T)s\| \leq 2\alpha \sqrt{\cost(p,\greedy)\cost(p,s)}$.
%Now suppose $\alpha^{-2}+k_i$. In this case $\|(I-\Pi_U\Pi_U^T)p\| \leq \sqrt{\cost(p,\greedy)}$. Consequently, $p^T(I-\Pi_{U}\Pi_{U}^T)s \leq \alpha\cdot \|(I-\Pi_U\Pi_U^T)p\| |(I-\Pi_U\Pi_U^T)s\| \leq 2\alpha \sqrt{\cost(p,\greedy)\cost(p,s)}$.
%Thus, the desired difference holds.
%
%We now focus on the case that $\|p-s\| \geq 60\cdot 2^i/2 \sqrt{\cost(p,\greedy)}$. In this case, either $\|\Pi_{U}(p-s)\|$ or $\|(I-\Pi_{U}\Pi_{U}^T)(p-s)\|$ are at least $40\cdot 2^i/2 \sqrt{\cost(p,\greedy)}$. In the former case, Eq. \ref{eq:netclose} shows that we rule out $s$ as a viable center for $p$. In the latter case, we have due to Lemma \ref{lem:innerproduct} $\|(I-\Pi_{U}\Pi_{U}^T)(p-s)\|2= \|(I-\Pi_{U}\Pi_{U}^T)p\|^2 +\|(I-\Pi_{U}\Pi_{U}^T)s\|^2 - 2p^T(I-\Pi_{U}\Pi_{U}^T)s \geq (1-\alpha)\cdot \left(\|(I-\Pi_{U}\Pi_{U}^T)p\|^2 +\|(I-\Pi_{U}\Pi_{U}^T)s\|^2\right)$. Here, either $\|(I-\Pi_{U}\Pi_{U}^T)p\|$ or $\|(I-\Pi_{U}\Pi_{U}^T)s\|$ are now at least $20\cdot 2^i/2 \sqrt{\cost(p,\greedy)}$. In the former case, simply evaluating $\|(I-\Pi_{U}\Pi_{U}^T)p\|$ rules out $s$ as a viable center and in the latter case Eq. \ref{eq:netsingle} rules out $s$ as a viable center.
%
%We can now conclude overall. By assumption, all points were members of clusters of type $i$. Therefore, there exists at least one viable center for every point. Thus only the upper equation in \ref{eq:netboth} applies. The overall lemma now follows by rescaling $\alpha$.
\end{proof}

We also require an additional net that works for low dimensions.

\begin{lemma}[Compare Lemma 22 of \cite{CGSS22}]
\label{lem:netsizelarge}
Let $P$ be a set of points in $d$ dimensional Euclidean space, $k$ a positive integer and $\greedy$ be a candidate solution. 
Define $\cand$ to be the set of possible candidate centers such that the clusters induced by $\greedy$ are of type $i$, with $3\leq i \leq \log 1/\varepsilon^2$.
For all $\alpha \leq 1/2$, there exists an $(\alpha,k)$-clustering net $\mathbb{N}$ of $\cand$ with 
$$|\mathbb{N}|\leq \exp\left(\gamma\cdot k\cdot d \cdot i\log(4/\alpha)\right),$$
where $\gamma$ is an absolute constant.  
\end{lemma}
\begin{proof}
The only difference to Lemma 22 of \cite{CGSS22} is that the nets are required to have an error of $\alpha\cdot \sqrt{\cost(p,\calS)\cost(p,\greedy)}$ rather than $\alpha\cdot \left(\cost(p,\calS)+ \cost(p,\greedy)\right)$. This can be done by rescaling $\varepsilon$ by $2^{-i}$, which in turn is absorbed by the constant $\gamma$ as $2^i \leq O(1)\cdot \varepsilon^{-2}$.
\end{proof}



\subsubsection*{Bounding the Variance}
We now use the cost vectors to obtain an improved variance for the estimator
$$\frac{\sum_{C_j \in T_{i}}\sum_{p\in C_j\cap \Omega}(v_p^{\calS,2^{-(h+1)}} - v_p^{\calS,2^{-h}})w_p}{\left( \cost(P,\calS) + \cost(P,\greedy)\right)}g_p.$$
The bounds on variance for any random variable $\sum a_p g_p$ with standard Gaussians $g_p$ is Gaussian distributed with mean $0$ and variance $\sum a_p^2$.

Before we do this, we require an additional notion. Let $\calE$ denote the event that $\frac{1}{|\Omega|}\sum_{p\in C_j\cap \Omega} w_p = (1\pm \varepsilon) \cdot |C_j|$.
The following lemma bounds the probability of $\calE$ occurring.
\begin{lemma}
\label{lem:eventE} [Compare Lemma 19 of \cite{CGSS22}]
If Assumption 4 holds, then event $\calE$ holds with probability $1-k^{-2}$ if $|\Omega| > \kappa\cdot k \varepsilon^{-2}\log k$ for a sufficiently high absolute constant $\kappa$.
\end{lemma}

\begin{lemma}
\label{lem:variance}
Given Assumption 4, the variance of $\frac{\sum_{C_j \in T_{i}}\sum_{p\in C_j\cap \Omega}(v_p^{\calS,2^{-(h+1)}} - v_p^{\calS,2^{-h}})w_p\cdot g_p}{|\Omega|\cdot \left( \cost(P,\calS) + \cost(P,\greedy)\right)}$ is at most
\begin{eqnarray*}
\gamma \cdot \frac{2^{-2h}}{|\Omega|}  \cdot \frac{k \cdot k_i 2^{i(z-1)}}{(k+k_i\cdot 2^i)^2}& & \text{conditioned on event } \calE \\
\gamma \cdot \frac{2^{-2h}\cdot k}{|\Omega|} \cdot \frac{k \cdot k_i 2^{i(z-1)}}{(k+k_i\cdot 2^i)^2}& & \text{conditioned on event } \overline{\calE}
\end{eqnarray*} 
for an absolute constant $\gamma$.
\end{lemma}
\begin{proof}
We first observe that since the $g_p$ are standard normal Gaussians, the entire estimator is Gaussian distributed with variance
$$ \sum_{C_j \in T_{i}}\sum_{p\in C_j\cap \Omega}\frac{1}{|\Omega|^2}\left(\frac{(v_p^{\calS,2^{-(h+1)}} - v_p^{\calS,2^{-h}})w_p}{\left( \cost(P,\calS) + \cost(P,\greedy)\right)}\right)^2.$$
We have 
\begin{eqnarray*}
|v_p^{\calS,2^{-(h+1)}} - v_p^{\calS,2^{-h}}| &=& |v_p^{\calS,2^{-(h+1)}} - \cost(p,\calS) + \cost(p,\calS) - v_p^{\calS,2^{-h}}| \\
&\leq & 2\cdot 2^{-h}\cdot \sqrt{\cost(p,\calS)^{z-1}\cdot \cost(p,\greedy)^{3-z}}
\end{eqnarray*}
 due to Lemma \ref{lem:netsize}. Furthermore, by definition $w_p= \frac{\cost(P,\greedy)|C_j|}{\cost(C_j,\greedy)}$. Finally, by definition of type $i$, we have $\cost(p,\calS)\cdot |C_j| = O(1)\cdot \cost(C_j,\calS)$ and by Assumption 4 we have $\cost(p,\greedy)\cdot |C_j| = O(1)\cdot \cost(C_j,\greedy)$ for all $p\in C_j$. 
\begin{eqnarray*}
& &\sum_{C_j \in T_{i}}\sum_{p\in C_j\cap \Omega}\frac{1}{|\Omega|^2}\left(\frac{(v_p^{\calS,2^{-(h+1)}} - v_p^{\calS,2^{-h}})w_p}{\left( \cost(P,\calS) + \cost(P,\greedy)\right)}\right)^2 \\
&\leq & O(1)\cdot\sum_{C_j \in T_{i}}\sum_{p\in C_j\cap \Omega}\frac{1}{|\Omega|^2}\left(\frac{2^{-h}\cdot \sqrt{\cost(p,\calS)^{z-1}\cdot \cost(p,\greedy)^{3-z}} \cdot \cost(P,\greedy)|C_j|}{\cost(C_j,\greedy) \cdot \left( \cost(P,\calS) + \cost(P,\greedy)\right)}\right)^2 \\
&\leq &O(1)\cdot \sum_{C_j \in T_{i}}\sum_{p\in C_j\cap \Omega}\frac{1}{|\Omega|^2}\left(\frac{2^{-2h}\cdot \cost(C_j,\greedy)^{1-z}\cdot \cost(C_j,\calS)^{z-1}\cdot \cost(P,\greedy)^2}{ \left( \cost(P,\calS) + \cost(P,\greedy)\right)^2}\right)  
\end{eqnarray*}
Now, let $k_i$ be the number of clusters of type $i$. Then due to Assumption 4 $\cost(C_j,\calS) \cdot k_i\leq O(1)\cost(P,\calS)$, for all $C_j$ of type $i$. Finally, note that $\frac{\cost(P,\greedy)}{\cost(C_j,\greedy)}\leq O(1)\cdot k$, also due to Assumption 4. 
Combining this, we then have
\begin{eqnarray*}
& &\sum_{C_j \in T_{i}}\sum_{p\in C_j\cap \Omega}\left(\frac{(v_p^{\calS,2^{-(h+1)}} - v_p^{\calS,2^{-h}})w_p}{\left( \cost(P,\calS) + \cost(P,\greedy)\right)}\right)^2 \\
&\leq & O(1)\cdot\sum_{C_j \in T_{i}}\sum_{p\in C_j\cap \Omega}\left(\frac{2^{-2h}\cdot 2^{i(z-1)} k^2}{|\Omega|^2 \cdot \left( k+k_i \cdot 2^i\right)^2}\right)\\
&\leq & O(1)\cdot\sum_{C_j \in T_{i}}\sum_{p\in C_j\cap \Omega}\left(\frac{2^{-2h}\cdot k}{k_i \cdot |\Omega|^2 }\right)   \cdot \frac{k_i \cdot k \cdot 2^{i(z-1)}}{(k+k_i\cdot 2^i)^2}
\end{eqnarray*}
Assuming event $\calE$, this may now be bounded by $O(1)\cdot \frac{2^{-2h}}{|\Omega|}\cdot \frac{k \cdot k_i 2^{i(z-1)}}{(k+k_i\cdot 2^i)^2}$. If event $\calE$ does not hold, we may bound the term by $\frac{2^{-2h}\cdot k}{k_i \cdot |\Omega| } \cdot \frac{k \cdot k_i 2^{i(z-1)}}{(k+k_i\cdot 2^i)^2}\leq \frac{2^{-2h}\cdot k}{|\Omega| }  \frac{k \cdot k_i 2^{i(z-1)}}{(k+k_i\cdot 2^i)^2}$.
\end{proof}




%\begin{proof}
%We first observe that since the $g_p$ are standard normal Gaussians, the entire estimator is Gaussian distributed with variance
%$$ \sum_{C_j \in T_{i}}\sum_{p\in C_j\cap \Omega}\frac{1}{|\Omega|^2}\left(\frac{(v_p^{\calS,2^{-(h+1)}} - v_p^{\calS,2^{-h}})w_p}{\left( \cost(P,\calS) + \cost(P,\greedy)\right)}\right)^2.$$
%We have $(v_p^{\calS,2^{-(h+1)}} - v_p^{\calS,2^{-h}}) = (v_p^{\calS,2^{-(h+1)}} - \cost(p,\calS) + \cost(p,\calS) - v_p^{\calS,2^{-h}}) \leq 2\cdot 2^{-h}\cdot \sqrt{\cost(p,\greedy)\cost(p,\calS)}$ due to Lemma \ref{lem:netsize}. Furthermore, by definition $w_p= \frac{\cost(P,\greedy)|C_j|}{\cost(C_j,\greedy)}$. Finally, by definition of type $i$, we have $\cost(p,\calS)\cdot |C_j| = O(1)\cdot \cost(C_j,\calS)$ and by Assumption 4 we have $\cost(p,\greedy)\cdot |C_j| = O(1)\cdot \cost(C_j,\greedy)$ for all $p\in C_j$. 
%\begin{eqnarray*}
%& &\sum_{C_j \in T_{i}}\sum_{p\in C_j\cap \Omega}\frac{1}{|\Omega|^2}\left(\frac{(v_p^{\calS,2^{-(h+1)}} - v_p^{\calS,2^{-h}})w_p}{\left( \cost(P,\calS) + \cost(P,\greedy)\right)}\right)^2 \\
%&\leq & O(1)\cdot\sum_{C_j \in T_{i}}\sum_{p\in C_j\cap \Omega}\frac{1}{|\Omega|^2}\left(\frac{2^{-h}\cdot \sqrt{\cost(p,\greedy)\cost(p,\calS)} \cdot \cost(P,\greedy)|C_j|}{\cost(C_j,\greedy) \cdot \left( \cost(P,\calS) + \cost(P,\greedy)\right)}\right)^2 \\
%&\leq &O(1)\cdot \sum_{C_j \in T_{i}}\sum_{p\in C_j\cap \Omega}\frac{1}{|\Omega|^2}\left(\frac{2^{-2h}\cdot \cost(C_j,\greedy)\cdot \cost(C_j,\calS) \cost(P,\greedy)^2}{ \cost(C_j,\greedy)^2 \cdot \left( \cost(P,\calS) + \cost(P,\greedy)\right)^2}\right)  
%\end{eqnarray*}
%Now, let $k_i$ be the number of clusters of type $i$. Then due to Assumption 4 $\cost(C_j,\calS) \cdot k_i\leq O(1)\cost(P,\calS)$, for all $C_j$ of type $i$. Finally, note that $\frac{\cost(P,\greedy)}{\cost(C_j,\greedy)}\leq O(1)\cdot k$, also due to Assumption 4. 
%Combining this, we then have
%\begin{eqnarray*}
%& &\sum_{C_j \in T_{i}}\sum_{p\in C_j\cap \Omega}\left(\frac{(v_p^{\calS,2^{-(h+1)}} - v_p^{\calS,2^{-h}})w_p}{\left( \cost(P,\calS) + \cost(P,\greedy)\right)}\right)^2 \\
%&\leq & O(1)\cdot\sum_{C_j \in T_{i}}\sum_{p\in C_j\cap \Omega}\left(\frac{2^{-2h}\cdot \cost(P,\calS) \cost(P,\greedy) \cdot k}{k_i \cdot |\Omega|^2 \cdot \left( \cost(P,\calS) + \cost(P,\greedy)\right)^2}\right)\\
%&\leq & O(1)\cdot\sum_{C_j \in T_{i}}\sum_{p\in C_j\cap \Omega}\left(\frac{2^{-2h}\cdot k}{k_i \cdot |\Omega|^2 }\right)   \cdot \frac{k\cdot k_i\cdot 2^i}{(k+k_i\cdot 2^i)^2}
%\end{eqnarray*}
%Assuming event $\calE$, this may now be bounded by $O(1)\cdot \frac{2^{-2h}}{|\Omega|}\cdot \frac{k\cdot k_i\cdot 2^i}{(k+k_i\cdot 2^i)^2}$. If event $\calE$ does not hold, we may bound the term by $\frac{2^{-2h}\cdot k}{k_i \cdot |\Omega| } \cdot \frac{k\cdot k_i\cdot 2^i}{(k+k_i\cdot 2^i)^2}\leq \frac{2^{-2h}\cdot k}{|\Omega| }  \cdot \frac{k\cdot k_i\cdot 2^i}{(k+k_i\cdot 2^i)^2}$.
%\end{proof}
%
%For the variance of the estimator used for Eq. \ref{eq:base1}, we use the following lemma.
%
%\begin{lemma}
%\label{lem:varbase}
%If Assumption 4 holds, the variance of $\frac{1}{|\Omega|}\frac{\sum_{C_j \in T_{i}}\sum_{p\in C_j\cap \Omega}(v_p^{\calS,2^{-1}} - q_j)w_p}{\left( \cost(P,\calS) + \cost(P,\greedy)\right)}g_p$ is at most
%\begin{eqnarray*}
%\gamma \cdot \frac{1}{|\Omega|}  \cdot \frac{k\cdot k_i\cdot 2^i}{(k+k_i\cdot 2^i)^2}& & \text{conditioned on event } \calE \\
%\gamma \cdot \frac{k}{|\Omega|} \cdot \frac{k\cdot k_i\cdot 2^i}{(k+k_i\cdot 2^i)^2}& & \text{conditioned on event } \overline{\calE}
%\end{eqnarray*} 
%for an absolute constant $\gamma$.
%\end{lemma}
%\begin{proof}
%We will bound $|v_p^{\calS,2^{-1}} - q_j|$ for any point $p\in C_j$.
%Due to the triangle inequality and by Assumption 4 which states that all points have roughly equal distance to their center in $\greedy$, we have 
%$$|\sqrt{v_p^{\calS,2^{-1}}} - \sqrt{q_j}| \leq O(1) \cdot \sqrt{\cost(p,\greedy)}.$$
%Futhermore, again due to the triangle inequality, $C_j\in T_i$ with $i>3$ and Assumption 4, we have $(\sqrt{v_p^{\calS,2^{-1}}} + \sqrt{q_j}) = O(1) \sqrt{\cost(p,\calS}$. 
%Therefore
%\begin{eqnarray*}
%|v_p^{\calS,2^{-1}} - q_j|  &=& |\sqrt{v_p^{\calS,2^{-1}}} - \sqrt{q_j}| \cdot (\sqrt{v_p^{\calS,2^{-1}}} + \sqrt{q_j}) = O(1) \sqrt{\cost(p,\calS)\cost(p,\greedy)}
%\end{eqnarray*}
%Using this bound and the same steps as in Lemma \ref{lem:variance}, we then have
%\begin{eqnarray*}
%& &\sum_{C_j \in T_{i}}\sum_{p\in C_j\cap \Omega}\frac{1}{|\Omega|^2}\left(\frac{(v_p^{\calS,2^{-1}} - q_j)w_p}{\left( \cost(P,\calS) + \cost(P,\greedy)\right)}\right)^2 \\
%&\leq & O(1) \cdot \sum_{C_j \in T_{i}}\sum_{p\in C_j\cap \Omega}\frac{1}{|\Omega|^2}\frac{\cost(p,\calS)\cost(p,\greedy) \cost(P,\calS)^2 |C_j|^2}{\cost(C_j,\greedy)^2 \cdot\left( \cost(P,\calS) + \cost(P,\greedy)\right)^2} \\
%&\leq & O(1) \cdot \sum_{C_j \in T_{i}}\sum_{p\in C_j\cap \Omega}\left(\frac{k}{k_i \cdot |\Omega|^2 }\right) \cdot \frac{k\cdot k_i\cdot 2^i}{(k+k_i\cdot 2^i)^2}
%\end{eqnarray*}
%Conditioned on event $\calE$, this now becomes  $O(1)\cdot \frac{1}{|\Omega|}\cdot \frac{k\cdot k_i\cdot 2^i}{(k+k_i\cdot 2^i)^2}$ and similarly, if event $\calE$ does not hold, we have the bound $\frac{k}{|\Omega| }  \cdot \frac{k\cdot k_i\cdot 2^i}{(k+k_i\cdot 2^i)^2}$.
%\end{proof}
%
%We now focus on the variance of the estimator used for Eq. \ref{eq:base2}. Due to Assumption 4, we have $\cost(T_i,\calS) = O(1) \cdot k_i\cost(C_j,\calS)$, for any $C_j\in T_i$ Thus
%\begin{eqnarray}
%\nonumber
%& & \mathbb{E}_{\Omega} \mathbb{E}_{g}  \left[ \left\vert \sup_{v^{\calS,1}\in \mathbb{N}_{2^{-1}}} \frac{\sum_{C_j \in T_{i}}\sum_{p\in C_j\cap \Omega}q_j\cdot w_p}{|\Omega|\cdot\left( \cost(P,\calS) + \cost(P,\greedy)\right)}g_p \right\vert \right] \\
%\nonumber
%&\leq & \mathbb{E}_{\Omega} \mathbb{E}_{g}  \left[ \left\vert \sup_{v^{\calS,1}\in \mathbb{N}_{2^{-1}}} \frac{\sum_{C_j \in T_{i}}\sum_{p\in C_j\cap \Omega}q_j\cdot w_p}{|\Omega|\cdot  \cost(T_i,\calS) }g_p \right\vert \right]\\
%\label{eq:base3}
%&\leq & \max_{C_j \in T_{i}} \mathbb{E}_{\Omega} \mathbb{E}_{g}  \left[ \left\vert \sup_{v^{\calS,1}\in \mathbb{N}_{2^{-1}}} \frac{\sum_{p\in C_j\cap \Omega}q_j\cdot w_p}{|\Omega|\cdot  \cost(C_j,\calS) }g_p \right\vert \right]
%\end{eqnarray}
%
%We now obtain the following variance for the estimator use in Equation \ref{eq:base3}.
%
%\begin{lemma}
%\label{lem:varq}
%If Assumption 4 holds, the variance of $\frac{\sum_{p\in C_j\cap \Omega}q_j\cdot w_p}{|\Omega|\cdot  \cost(C_j,\calS) }g_p$, given that $C_j\in T_i$ with $i\in \{3,\ldots,\log \varepsilon^{-2}\}$ is at most
%\begin{eqnarray*}
%\gamma \cdot \frac{k}{|\Omega|}  & & \text{conditioned on event } \calE \\
%\gamma \cdot \frac{k^2}{|\Omega|} & & \text{conditioned on event } \overline{\calE}
%\end{eqnarray*} 
%for an absolute constant $\gamma$.
%\end{lemma}
%\begin{proof}
%Recall by Assumption 4 $\cost(P,\calS) = O(1) \cdot k \cdot \cost(C_j,\calS)$.
%We have
%\begin{eqnarray*}
%& &\sum_{p\in C_j\cap \Omega}\left(\frac{q_j\cdot w_p}{|\Omega|\cdot  \cost(C_j,\calS) }\right)^2 \\
%&=&\sum_{p\in C_j\cap \Omega}\left(\frac{q_j\cdot \cost(P,\greedy)\cdot |C_j|}{|\Omega|\cdot \cost(C_j,\greedy)\cdot  \cost(C_j,\calS) }\right)^2 \\
%&=&O(1) \cdot \sum_{p\in C_j\cap \Omega}\left(\frac{k}{|\Omega|}\right)^2
%\end{eqnarray*}
%Conditioned on event $\calE$, $|C_j \cap \Omega| = \frac{1}{k}\cdot |\Omega|$ and this now becomes  $O(1)\cdot \frac{k}{|\Omega|}$. Otherwise, we have the bound $\frac{k^2}{|\Omega| }$.
%\end{proof}

\subsubsection*{Completing the Proof for Eq. \ref{eq:telescopesmall}}

Throughout this section, we use the bound on the expected maximum of independent Gaussians.  
\begin{lemma}[Lemma 2.3 of \cite{massart2007}]
\label{lem:minichain}
Let $g_i\thicksim\mathcal{N}(0,\sigma_i^2)$, $i\in [n]$ be Gaussian random variables and suppose $\sigma_i\leq \sigma$ for all $i$. Then $ \mathbb{E}[\underset{i\in [n]}{\max} |g_i|] \leq 2\sigma\cdot \sqrt{2\ln n}.$
\end{lemma}

The number of cost vectors in $\mathbb{N}_{h+1}\times \mathbb{N}_{h}$ is at most $\exp\left(\gamma \cdot k \cdot \log \|P\|_0  \cdot \min(k_i + 2^{2h}, 2^{2h}\cdot 2^i) \cdot i\cdot h)\right)$ for some absolute constant $\gamma$ due to Lemma \ref{lem:netsize}.
With the bound on the variance (Lemma \ref{lem:variance} and conditioned on event $\calE$), we then have

\begin{eqnarray}
\nonumber
& &\sum_{h=1}^{\log \varepsilon^{-2}} \mathbb{E}_{\Omega} \mathbb{E}_{g}   \left[ \left.\sup_{v^{\calS,h+1}-v^{\calS,h}\in \mathbb{N}_{h+1}\times \mathbb{N}_{h}}  \left\vert\frac{\sum_{C_j \in T_{i}}\sum_{p\in C_j\cap \Omega}(v_p^{\calS,2^{-(h+1)}} - v_p^{\calS,2^{-h}})w_p\cdot g_p}{\left( \cost(P,\calS) + \cost(P,\greedy)\right)}  \right\vert \right\vert \calE\right] \\
\nonumber
&\leq & \sum_{h=1}^{\log \varepsilon^{-2}} \sqrt{\gamma \cdot k \cdot \log \|P\|_0  \cdot \min(k_i + 2^{2h}, 2^{2h}\cdot 2^i) \cdot i \cdot h \cdot \frac{2^{-2h}}{|\Omega|} \cdot \frac{k \cdot k_i 2^{i(z-1)}}{(k+k_i\cdot 2^i)^2}} \\
\label{eq:vareventE}
& \leq & 2 \sqrt{\gamma \cdot k \cdot \log \|P\|_0  \cdot \min(k_i , 2^i) \cdot i \cdot \log^3 \varepsilon^{-1} \cdot \frac{1}{|\Omega|} \cdot \frac{k \cdot k_i 2^{i(z-1)}}{(k+k_i\cdot 2^i)^2}}.
\end{eqnarray}

Conditioned on event $\calE$ not holding, we then have using a similar calculation

\begin{eqnarray}
\nonumber
& &\sum_{h=1}^{\log \varepsilon^{-2}} \mathbb{E}_{\Omega} \mathbb{E}_{g}   \left[ \left. \sup_{v^{\calS,h+1}-v^{\calS,h}\in \mathbb{N}_{h+1}\times \mathbb{N}_{h}}\left\vert \frac{\sum_{C_j \in T_{i}}\sum_{p\in C_j\cap \Omega}(v_p^{\calS,2^{-(h+1)}} - v_p^{\calS,2^{-h}})w_p\cdot g_p}{\left( \cost(P,\calS) + \cost(P,\greedy)\right)}  \right\vert \right\vert \overline{\calE}\right] \\
\nonumber
&\leq & \sum_{h=1}^{\log \varepsilon^{-2}} \sqrt{\gamma \cdot k \cdot \log \|P\|_0  \cdot \min(k_i + 2^{2h}, 2^{2h}\cdot 2^i) \cdot i \cdot h \cdot \frac{2^{-2h}\cdot k}{|\Omega|}  \cdot \frac{k \cdot k_i 2^{i(z-1)}}{(k+k_i\cdot 2^i)^2}} \\
\label{eq:vareventnotE}
& \leq & 2 \sqrt{\gamma \cdot k^2 \cdot \log \|P\|_0  \cdot \min(k_i , 2^i) \cdot i \cdot \log^3 \varepsilon^{-1} \cdot \frac{1}{|\Omega|} \cdot \frac{k \cdot k_i 2^{i(z-1)}}{(k+k_i\cdot 2^i)^2}}.
\end{eqnarray}

We have $\mathbb{P}[\overline{\calE}]\leq 1/k^2$ due to Lemma \ref{lem:eventE}.
Since $\|P\|_0 \leq \text{poly}(k,\varepsilon^{-1})$, $2^{i}\leq O(1)\cdot \varepsilon^{-2}$, we can combine Equations \ref{eq:vareventE} and \ref{eq:vareventnotE} with the law of total expectation to obtain

\begin{eqnarray}
\nonumber
& &\sum_{h=1}^{\log \varepsilon^{-2}} \mathbb{E}_{\Omega} \mathbb{E}_{g}   \left[  \sup_{v^{\calS,h+1}-v^{\calS,h}\in \mathbb{N}_{h+1}\times \mathbb{N}_{h}} \left\vert\frac{\sum_{C_j \in T_{i}}\sum_{p\in C_j\cap \Omega}(v_p^{\calS,2^{-(h+1)}} - v_p^{\calS,2^{-h}})w_p\cdot g_p}{\left( \cost(P,\calS) + \cost(P,\greedy)\right)}  \right\vert\right] \\
\nonumber
&\leq & 2 \sqrt{\gamma \cdot k \cdot \log \|P\|_0  \cdot \min(k_i , 2^i) \cdot i \cdot \log^3 \varepsilon^{-1} \cdot \frac{1}{|\Omega|} \cdot \frac{k \cdot k_i \cdot 2^{i(z-1)}}{(k+k_i\cdot 2^i)^2}}  \\
\nonumber
&  & + 2 \sqrt{\gamma \cdot k^2 \cdot \log \|P\|_0  \cdot \min(k_i , 2^i) \cdot i \cdot \log^3 \varepsilon^{-1} \cdot \frac{1}{|\Omega|} \cdot \frac{k \cdot k_i \cdot 2^{i(z-1)}}{(k+k_i\cdot 2^i)^2}} \cdot \frac{1}{k^2} \\
\label{eq:finalbound}
&\leq & 4 \sqrt{\gamma \cdot k \cdot \log \|P\|_0  \cdot \min(k_i , 2^i) \cdot i \cdot \log^3 \varepsilon^{-1} \cdot \frac{1}{|\Omega|} \cdot \frac{k \cdot k_i \cdot 2^{i(z-1)}}{(k+k_i\cdot 2^i)^2}} 
%\\ &\leq & O(1)\frac{\varepsilon^{-2}}{\log^3 \varepsilon^{-1}}.
\end{eqnarray}
Using a straightforward, but tedious calculation, we have $\min(k_i,2^i)\frac{k \cdot k_i \cdot 2^{i(z-1)}}{(k+k_i\cdot 2^i)^2} \in O(k^{z/(z+2)})$.
Specifically, if $\min(k_i,2^i) = k_i$, the term may be bounded by $\frac{k_i^2 \cdot k\cdot 2^{i(z-1)}}{k^{3-z}\cdot (k_i\cdot 2^i)^{z-1}}$. If $\min(k_i,2^i) = 2^i$, the term may be bounded by $\frac{k_i \cdot k\cdot 2^{i\cdot z}}{k^{2-z}\cdot (k_i\cdot 2^i)^{z}}$. Setting both terms to be equal, solving for $k_i$ yields $k_i = k^{(z+1)/(z+2)}$. Inserting that value of $k_i$ back into $\frac{k_i^2 \cdot k\cdot 2^{i(z-1)}}{k^{3-z}\cdot (k_i\cdot 2^i)^{z-1}}$ then yields the upper bound $k^{z/(z+2)}$.
Therefore, by our choice of $|\Omega|$, we can bound Eq. \ref{eq:finalbound} by $O(1)\frac{\varepsilon^{-2}}{\log^3 \varepsilon^{-1}}$.
%The types $i$ we consider here, satisfy $2^i\leq O(1)\cdot\varepsilon^{-2}$ and moreover by Assumption 1, $\|P\|_0 \in \text{poly}(k,\varepsilon^{-1})$. Thus, we have 
%
%\begin{eqnarray}
%\nonumber
%& &\sum_{h=1}^{\log \varepsilon^{-2}} \mathbb{E}_{\Omega} \mathbb{E}_{g}   \left[  \sup_{v^{\calS,h+1}-v^{\calS,h}\in \mathbb{N}_{h+1}\times \mathbb{N}_{h}}\left\vert \frac{\sum_{C_j \in T_{i}}\sum_{p\in C_j\cap \Omega}(v_p^{\calS,2^{-(h+1)}} - v_p^{\calS,2^{-h}})w_p}{\left( \cost(P,\calS) + \cost(P,\greedy)\right)}g_p  \right\vert\right] \\
%\label{eq:finalbound}
%&\leq & O(1) \cdot  \sqrt{k \cdot \log k  \cdot \min(k_i , 2^i) \cdot \log^5 \varepsilon^{-1} \cdot \frac{1}{|\Omega|} \cdot \frac{k\cdot k_i\cdot 2^i}{(k+k_i\cdot 2^i)^2}} 
%\end{eqnarray}

%To bound Equation \ref{eq:telescopelarge}, we proceed very similarly. Here, we use Lemma \ref{lem:netsizelarge} and Assumption 2 to show that the number of cost vectors in $\mathbb{N}_{h+1}\times \mathbb{N}_{h}$ is at most $\exp\left(\gamma \cdot k \cdot \log \|P\|_0  \cdot \varepsilon^{-2} \log h/\varepsilon)\right)$. The remaining analysis is almost entirely equal to that used to derive Eq. \ref{eq:finalbound} and we merely state the bound:
%
%\begin{eqnarray}
%\nonumber
%& &\sum_{\log \varepsilon^{-2}}^{\infty} \mathbb{E}_{\Omega} \mathbb{E}_{g}   \left[ \sup_{v^{\calS,h+1}-v^{\calS,h}\in \mathbb{N}_{h+1}\times \mathbb{N}_{h}}\left\vert  \frac{\sum_{C_j \in T_{i}}\sum_{p\in C_j\cap \Omega}(v_p^{\calS,2^{-(h+1)}} - v_p^{\calS,2^{-h}})w_p}{\left( \cost(P,\calS) + \cost(P,\greedy)\right)}g_p  \right\vert\right] \\
%\label{eq:finalboundlarge}
%&\leq & O(1) \cdot  \sqrt{k \cdot \log k  \cdot \min(k_i , 2^i) \cdot \log^5 \varepsilon^{-1} \cdot \frac{1}{|\Omega|} \cdot \frac{k\cdot k_i\cdot 2^i}{(k+k_i\cdot 2^i)^2}} 
%\end{eqnarray}

%For Equation \ref{eq:base} we focus on Equations \ref{eq:base1} and \ref{eq:base3}. For the former, we have $|\mathbb{N}_{2^{-1}}|\leq \exp\left(\gamma \cdot k \cdot \log \|P\|_0   \cdot \min(k_i, 2^i) \cdot i)\right)$. Thus, combined with Lemma \ref{lem:varbase} and conditioning on event $\calE$, we have
%\begin{eqnarray*}
%& &\mathbb{E}_{\Omega} \mathbb{E}_{g}  \left[ \sup_{v^{\calS,1}\in \mathbb{N}_{2^{-1}}}  \left\vert\frac{\sum_{C_j \in T_{i}}\sum_{p\in C_j\cap \Omega}|v_p^{\calS,2^{-1}}-q_j|w_p}{|\Omega|\cdot\left( \cost(P,\calS) + \cost(P,\greedy)\right)}g_p \right\vert \vert \calE \right] \\
%&=& O(1)\sqrt{\gamma \cdot k \cdot \log \|P\|_0   \cdot \min(k_i, 2^i) \cdot i) \cdot  \frac{1}{|\Omega|}  \cdot \frac{k\cdot k_i\cdot 2^i}{(k+k_i\cdot 2^i)^2}} 
%\end{eqnarray*}
%
%Similarly, not conditioning on event $\calE$ implies
%\begin{eqnarray*}
%& &\mathbb{E}_{\Omega} \mathbb{E}_{g}  \left[ \sup_{v^{\calS,1}\in \mathbb{N}_{2^{-1}}}  \left\vert \frac{\sum_{C_j \in T_{i}}\sum_{p\in C_j\cap \Omega}|v_p^{\calS,2^{-1}}-q_j|w_p}{|\Omega|\cdot\left( \cost(P,\calS) + \cost(P,\greedy)\right)}g_p \right\vert \vert \overline{\calE}\right] \\
%&leq & O(1)\sqrt{\gamma \cdot k \cdot \log \|P\|_0   \cdot \min(k_i, 2^i) \cdot i) \cdot  \frac{k^2}{|\Omega|}  \cdot \frac{k\cdot k_i\cdot 2^i}{(k+k_i\cdot 2^i)^2}} 
%\end{eqnarray*}
%
%The remaining analysis is now analogous to that used to derive Eq. \ref{eq:finalbound} which then yields
%\begin{eqnarray}
%\nonumber
%& &\mathbb{E}_{\Omega} \mathbb{E}_{g}  \left[\sup_{v^{\calS,1}\in \mathbb{N}_{2^{-1}}} \left\vert  \frac{\sum_{C_j \in T_{i}}\sum_{p\in C_j\cap \Omega}|v_p^{\calS,2^{-1}}-q_j|w_p}{|\Omega|\cdot\left( \cost(P,\calS) + \cost(P,\greedy)\right)}g_p \right\vert \right] \\
%\label{eq:finalboundbase}
%&\leq &O(1) \cdot  \sqrt{k \cdot \log k  \cdot \min(k_i , 2^i) \cdot \log^5 \varepsilon^{-1} \cdot \frac{1}{|\Omega|} \cdot \frac{k\cdot k_i\cdot 2^i}{(k+k_i\cdot 2^i)^2}} 
%\end{eqnarray}
%
%For the term in Equation \ref{eq:base3}, we note that $q_j$ is a scalar and the expectation is scales linearly when multiplying with scalars. Thus, for every cluster, we have a net of size $1$, which means we have an overall net of size $k$. We thus obtain 
%\begin{eqnarray*}
%& &\mathbb{E}_{\Omega} \mathbb{E}_{g}  \left[ \sup_{v^{\calS,1}\in \mathbb{N}_{2^{-1}}} \max_{C_j \in T_{i}} \left\vert  \frac{\sum_{p\in C_j\cap \Omega}q_j\cdot w_p}{|\Omega|\cdot  \cost(C_j,\calS) }g_p \right\vert \vert \calE\right] \\
%&=& O(1)\sqrt{ \log k \frac{k}{|\Omega|}  \cdot \frac{k\cdot k_i\cdot 2^i}{(k+k_i\cdot 2^i)^2}} 
%\end{eqnarray*}
%
%Similarly, not conditioning on event $\calE$ implies
%\begin{eqnarray*}
%& &\mathbb{E}_{\Omega} \mathbb{E}_{g}  \left[ \sup_{v^{\calS,1}\in \mathbb{N}_{2^{-1}}} \max_{C_j \in T_{i}}  \left\vert \frac{\sum_{p\in C_j\cap \Omega}q_j\cdot w_p}{|\Omega|\cdot  \cost(C_j,\calS) }g_p \right\vert \vert \overline{\calE}\right] \\
%&leq & O(1)\sqrt{\log k \frac{k^2}{|\Omega|}  \cdot \frac{k\cdot k_i\cdot 2^i}{(k+k_i\cdot 2^i)^2}} 
%\end{eqnarray*}
%
%Combining both terms now yields
%
%\begin{eqnarray}
%\nonumber
%& &\mathbb{E}_{\Omega} \mathbb{E}_{g}  \left[  \sup_{v^{\calS,1}\in \mathbb{N}_{2^{-1}}} \max_{C_j \in T_{i}}  \left\vert\frac{\sum_{p\in C_j\cap \Omega}q_j\cdot w_p}{|\Omega|\cdot  \cost(C_j,\calS) }g_p \right\vert \right] \\
%\label{eq:finalboundbaseq}
%&\leq &O(1) \cdot  \sqrt{\frac{k \log k}{|\Omega|}} 
%\end{eqnarray}
%
%Combining the bounds in Equations \ref{eq:finalbound}, \ref{eq:finalboundlarge}, \ref{eq:finalboundbase} and \ref{eq:finalboundbaseq} for the respective terms in Equations \ref{eq:telescopesmall}, \ref{eq:telescopelarge}, \ref{eq:base1} and \ref{eq:base3} now yields the claim.

\paragraph{Completing the Proof for Eq. \ref{eq:telescopelarge}}


Here, we use Lemma \ref{lem:netsizelarge} and Assumption 2 to show that the number of cost vectors in $\mathbb{N}_{h+1}\times \mathbb{N}_{h}$ is at most $\exp\left(\gamma \cdot k \cdot \log \|P\|_0  \cdot \varepsilon^{-2} \log h/\varepsilon)\right)$. Conditioned on event $\calE$, we therefore have

\begin{eqnarray*}
\nonumber
& &\sum_{\log \varepsilon^{-2}}^{\infty} \mathbb{E}_{\Omega} \mathbb{E}_{g}   \left[ \sup_{v^{\calS,h+1}-v^{\calS,h}\in \mathbb{N}_{h+1}\times \mathbb{N}_{h}} \left. \left\vert  \frac{\sum_{C_j \in T_{i}}\sum_{p\in C_j\cap \Omega}(v_p^{\calS,2^{-(h+1)}} - v_p^{\calS,2^{-h}})w_p}{\left( \cost(P,\calS) + \cost(P,\greedy)\right)}g_p  \right\vert \right\vert \calE\right] \\
&\leq & \sum_{\log \varepsilon^{-2}}^{\infty} O(1) \cdot  \sqrt{\gamma \cdot k \cdot \log \|P\|_0  \cdot \varepsilon^{-2} \cdot \log h/\varepsilon \cdot \frac{2^{-2h}}{|\Omega|} \cdot \frac{k\cdot k_i\cdot 2^{i(z-1)}}{(k+k_i\cdot 2^i)^2}} \\
&\leq & \sum_{1}^{\infty} O(1) \cdot  \sqrt{\gamma \cdot k \cdot \log \|P\|_0   \cdot \log h/\varepsilon \cdot \frac{2^{-2h}}{|\Omega|} \cdot \frac{k\cdot k_i\cdot 2^{i(z-1)}}{(k+k_i\cdot 2^i)^2}} \\
&\leq & O(1) \cdot  \sqrt{\gamma \cdot k \cdot \log \|P\|_0   \cdot \log 1/\varepsilon \cdot \frac{1}{|\Omega|} \cdot \frac{k\cdot k_i\cdot 2^{i(z-1)}}{(k+k_i\cdot 2^i)^2}} 
\end{eqnarray*}

Similarly, if $\calE$ does not hold, we have

\begin{eqnarray*}
\nonumber
& &\sum_{\log \varepsilon^{-2}}^{\infty} \mathbb{E}_{\Omega} \mathbb{E}_{g}   \left[ \sup_{v^{\calS,h+1}-v^{\calS,h}\in \mathbb{N}_{h+1}\times \mathbb{N}_{h}}\left. \left\vert  \frac{\sum_{C_j \in T_{i}}\sum_{p\in C_j\cap \Omega}(v_p^{\calS,2^{-(h+1)}} - v_p^{\calS,2^{-h}})w_p}{\left( \cost(P,\calS) + \cost(P,\greedy)\right)}g_p  \right\vert\right\vert \overline{\calE}\right] \\
&\leq & \sum_{\log \varepsilon^{-2}}^{\infty} O(1) \cdot  \sqrt{\gamma \cdot k \cdot \log \|P\|_0  \cdot \varepsilon^{-2} \cdot \log h/\varepsilon \cdot \frac{2^{-2h}k}{|\Omega|} \cdot \frac{k\cdot k_i\cdot 2^{i(z-1)}}{(k+k_i\cdot 2^i)^2}} \\
&\leq & \sum_{1}^{\infty} O(1) \cdot  \sqrt{\gamma \cdot k \cdot \log \|P\|_0   \cdot \log h/\varepsilon \cdot \frac{2^{-2h}k}{|\Omega|} \cdot \frac{k\cdot k_i\cdot 2^{i(z-1)}}{(k+k_i\cdot 2^i)^2}} \\
&\leq & O(1) \cdot  \sqrt{\gamma \cdot k \cdot \log \|P\|_0   \cdot \log 1/\varepsilon \cdot \frac{k}{|\Omega|} \cdot \frac{k\cdot k_i\cdot 2^{i(z-1)}}{(k+k_i\cdot 2^i)^2}} 
\end{eqnarray*}

We have $\mathbb{P}[\overline{\calE}]\leq 1/k^2$ due to Lemma \ref{lem:eventE}.
Since $\|P\|_0 \leq \text{poly}(k,\varepsilon^{-1})$, $2^{i}\leq O(1)\cdot \varepsilon^{-2}$ and by our choice of $|\Omega|$, we can combine the last two equations with the law of total expectation to obtain
\begin{eqnarray}
\nonumber
& &\sum_{\log \varepsilon^{-2}}^{\infty} \mathbb{E}_{\Omega} \mathbb{E}_{g}   \left[ \sup_{v^{\calS,h+1}-v^{\calS,h}\in \mathbb{N}_{h+1}\times \mathbb{N}_{h}}\left\vert  \frac{\sum_{C_j \in T_{i}}\sum_{p\in C_j\cap \Omega}(v_p^{\calS,2^{-(h+1)}} - v_p^{\calS,2^{-h}})w_p}{\left( \cost(P,\calS) + \cost(P,\greedy)\right)}g_p  \right\vert\right] \\
\label{eq:finalboundlarge}
&\leq & O(1) \cdot  \sqrt{k \cdot \log k  \cdot \min(k_i , 2^i) \cdot \log^5 \varepsilon^{-1} \cdot \frac{1}{|\Omega|} \cdot \frac{k\cdot k_i\cdot 2^{i(z-1)}}{(k+k_i\cdot 2^i)^2}}.
\end{eqnarray}

Observe that Eq. \ref{eq:finalboundlarge} and Eq. \ref{eq:finalbound} are essentially identical up to lower order terms.

\paragraph{Completing the Proof for Eq. \ref{eq:base}}

Here, we first split Eq. \ref{eq:base} into two estimators that will be easier to handle. We split the estimator into two parts as follows. First, let $q_j:=\frac{\sum_{p\in C_j}v_p^{\calS,2^{-1}}}{|C_j|}$. Now we consider
\begin{eqnarray}
\label{eq:firstest}
& &\frac{1}{|\Omega|}\frac{\sum_{C_j \in T_{i}}\sum_{p\in C_j\cap \Omega}(v_p^{\calS,2^{-1}} - q_j)w_p}{\left( \cost(P,\calS) + \cost(P,\greedy)\right)}g_p \\
\label{eq:secondest}
& &+ \frac{1}{|\Omega|}\frac{\sum_{C_j \in T_{i}}\sum_{p\in C_j\cap \Omega}q_j \cdot w_p}{\left( \cost(P,\calS) + \cost(P,\greedy)\right)}g_p 
\end{eqnarray}
Thus, Equation~\ref{eq:base} becomes
\begin{eqnarray}
\nonumber
& & \mathbb{E}_{\Omega} \mathbb{E}_{g}  \left[  \sup_{v^{\calS,1}\in \mathbb{N}_{2^{-1}}} \left\vert \frac{\sum_{C_j \in T_{i}}\sum_{p\in C_j\cap \Omega}  v_p^{\calS,2^{-1}}w_p}{|\Omega|\cdot \left( \cost(P,\calS) + \cost(P,\greedy)\right)}g_p \right\vert \right] \\
\label{eq:base1}
&=&\mathbb{E}_{\Omega} \mathbb{E}_{g}  \left[  \sup_{v^{\calS,1}\in \mathbb{N}_{2^{-1}}}\left\vert \frac{\sum_{C_j \in T_{i}}\sum_{p\in C_j\cap \Omega}|v_p^{\calS,2^{-1}}-q_j|w_p}{|\Omega|\cdot\left( \cost(P,\calS) + \cost(P,\greedy)\right)}g_p \right\vert \right] \\
\label{eq:base2}
&+& \mathbb{E}_{\Omega} \mathbb{E}_{g}  \left[  \sup_{v^{\calS,1}\in \mathbb{N}_{2^{-1}}} \left\vert\frac{\sum_{C_j \in T_{i}}\sum_{p\in C_j\cap \Omega}q_j\cdot w_p}{|\Omega|\cdot\left( \cost(P,\calS) + \cost(P,\greedy)\right)}g_p \right\vert \right]
\end{eqnarray}

For the variance of the estimator used for Eq. \ref{eq:base1}, we use the following lemma.

\begin{lemma}
\label{lem:varbase}
If Assumption 4 holds, the variance of $\frac{1}{|\Omega|}\frac{\sum_{C_j \in T_{i}}\sum_{p\in C_j\cap \Omega}(v_p^{\calS,2^{-1}} - q_j)w_p}{\left( \cost(P,\calS) + \cost(P,\greedy)\right)}g_p$ is at most
\begin{eqnarray*}
\gamma \cdot \frac{1}{|\Omega|}  \cdot \frac{k\cdot k_i\cdot 2^{i(z-1)}}{(k+k_i\cdot 2^i)^2}& & \text{conditioned on event } \calE \\
\gamma \cdot \frac{k}{|\Omega|} \cdot \frac{k\cdot k_i\cdot 2^{i(z-1)}}{(k+k_i\cdot 2^i)^2}& & \text{conditioned on event } \overline{\calE}
\end{eqnarray*} 
for an absolute constant $\gamma$.
\end{lemma}
\begin{proof}
The proof of this is very close to the proof of Lemma 9 from \cite{CSS21}. 
For $k$-median, this is a straightforward application of the triangle inequality.
For $k$-means, the analysis is slightly more involved and included for completeness. Thus, throughout this proof, we have $z=2$.

We will bound $|v_p^{\calS,2^{-1}} - q_j|$ for any point $p\in C_j$.
Due to the triangle inequality and by Assumption 4 which states that all points have roughly equal distance to their center in $\greedy$, we have 
$$|\sqrt{v_p^{\calS,2^{-1}}} - \sqrt{q_j}| \leq O(1) \cdot \sqrt{\cost(p,\greedy)}.$$
Futhermore, again due to the triangle inequality, $C_j\in T_i$ with $i>3$ and Assumption 4, we have $(\sqrt{v_p^{\calS,2^{-1}}} + \sqrt{q_j}) = O(1) \sqrt{\cost(p,\calS}$. 
Therefore
\begin{eqnarray*}
|v_p^{\calS,2^{-1}} - q_j|  &=& |\sqrt{v_p^{\calS,2^{-1}}} - \sqrt{q_j}| \cdot (\sqrt{v_p^{\calS,2^{-1}}} + \sqrt{q_j}) = O(1) \sqrt{\cost(p,\calS)\cost(p,\greedy)}
\end{eqnarray*}
Using this bound and the same steps as in Lemma \ref{lem:variance}, we then have
\begin{eqnarray*}
& &\sum_{C_j \in T_{i}}\sum_{p\in C_j\cap \Omega}\frac{1}{|\Omega|^2}\left(\frac{(v_p^{\calS,2^{-1}} - q_j)w_p}{\left( \cost(P,\calS) + \cost(P,\greedy)\right)}\right)^2 \\
&\leq & O(1) \cdot \sum_{C_j \in T_{i}}\sum_{p\in C_j\cap \Omega}\frac{1}{|\Omega|^2}\frac{\cost(p,\calS)\cost(p,\greedy) \cost(P,\calS)^2 |C_j|^2}{\cost(C_j,\greedy)^2 \cdot\left( \cost(P,\calS) + \cost(P,\greedy)\right)^2} \\
&\leq & O(1) \cdot \sum_{C_j \in T_{i}}\sum_{p\in C_j\cap \Omega}\left(\frac{k}{k_i \cdot |\Omega|^2 }\right) \cdot \frac{k\cdot k_i\cdot 2^i}{(k+k_i\cdot 2^i)^2}
\end{eqnarray*}
Conditioned on event $\calE$, this now becomes  $O(1)\cdot \frac{1}{|\Omega|}\cdot \frac{k\cdot k_i\cdot 2^i}{(k+k_i\cdot 2^i)^2}$ and similarly, if event $\calE$ does not hold, we have the bound $\frac{k}{|\Omega| }  \cdot \frac{k\cdot k_i\cdot 2^i}{(k+k_i\cdot 2^i)^2}$.
\end{proof}


We now focus on the variance of the estimator used for Eq. \ref{eq:base2}. Due to Assumption 4, we have $\cost(T_i,\calS) = O(1) \cdot k_i\cost(C_j,\calS)$, for any $C_j\in T_i$ Thus
\begin{eqnarray}
\nonumber
& & \mathbb{E}_{\Omega} \mathbb{E}_{g}  \left[ \left\vert \sup_{v^{\calS,1}\in \mathbb{N}_{2^{-1}}} \frac{\sum_{C_j \in T_{i}}\sum_{p\in C_j\cap \Omega}q_j\cdot w_p}{|\Omega|\cdot\left( \cost(P,\calS) + \cost(P,\greedy)\right)}g_p \right\vert \right] \\
\nonumber
&\leq & \mathbb{E}_{\Omega} \mathbb{E}_{g}  \left[ \left\vert \sup_{v^{\calS,1}\in \mathbb{N}_{2^{-1}}} \frac{\sum_{C_j \in T_{i}}\sum_{p\in C_j\cap \Omega}q_j\cdot w_p}{|\Omega|\cdot  \cost(T_i,\calS) }g_p \right\vert \right]\\
\label{eq:base3}
&\leq &  \mathbb{E}_{\Omega} \mathbb{E}_{g}  \left[ \left\vert \sup_{v^{\calS,1}\in \mathbb{N}_{2^{-1}}} \max_{C_j \in T_{i}}\frac{\sum_{p\in C_j\cap \Omega}q_j\cdot w_p}{|\Omega|\cdot  \cost(C_j,\calS) }g_p \right\vert \right]
\end{eqnarray}

We now obtain the following variance for the estimator used in Equation \ref{eq:base3}.

\begin{lemma}
\label{lem:varq}
If Assumption 4 holds, the variance of $\frac{\sum_{p\in C_j\cap \Omega}q_j\cdot w_p}{|\Omega|\cdot  \cost(C_j,\calS) }g_p$, given that $C_j\in T_i$ with $i\in \{3,\ldots,\log \varepsilon^{-2}\}$ is at most
\begin{eqnarray*}
\gamma \cdot \frac{k}{|\Omega|}  & & \text{conditioned on event } \calE \\
\gamma \cdot \frac{k^2}{|\Omega|} & & \text{conditioned on event } \overline{\calE}
\end{eqnarray*} 
for an absolute constant $\gamma$.
\end{lemma}
\begin{proof}
Recall by Assumption 4 $\cost(P,\calS) = O(1) \cdot k \cdot \cost(C_j,\calS)$.
We have
\begin{eqnarray*}
& &\sum_{p\in C_j\cap \Omega}\left(\frac{q_j\cdot w_p}{|\Omega|\cdot  \cost(C_j,\calS) }\right)^2 \\
&=&\sum_{p\in C_j\cap \Omega}\left(\frac{q_j\cdot \cost(P,\greedy)\cdot |C_j|}{|\Omega|\cdot \cost(C_j,\greedy)\cdot  \cost(C_j,\calS) }\right)^2 \\
&=&O(1) \cdot \sum_{p\in C_j\cap \Omega}\left(\frac{k}{|\Omega|}\right)^2
\end{eqnarray*}
Conditioned on event $\calE$, $|C_j \cap \Omega| = \frac{1}{k}\cdot |\Omega|$ and this now becomes  $O(1)\cdot \frac{k}{|\Omega|}$. Otherwise, we have the bound $\frac{k^2}{|\Omega| }$.
\end{proof}

We now bound Equations \ref{eq:base1} and \ref{eq:base3}. For the former, we have $|\mathbb{N}_{2^{-1}}|\leq \exp\left(\gamma \cdot k \cdot \log \|P\|_0   \cdot \min(k_i, 2^i) \cdot i)\right)$. Thus, combined with Lemma \ref{lem:varbase} and conditioning on event $\calE$, we have
\begin{eqnarray*}
& &\mathbb{E}_{\Omega} \mathbb{E}_{g}  \left[ \sup_{v^{\calS,1}\in \mathbb{N}_{2^{-1}}}  \left\vert\frac{\sum_{C_j \in T_{i}}\sum_{p\in C_j\cap \Omega}|v_p^{\calS,2^{-1}}-q_j|w_p}{|\Omega|\cdot\left( \cost(P,\calS) + \cost(P,\greedy)\right)}g_p \right\vert \vert \calE \right] \\
&=& O(1)\sqrt{\gamma \cdot k \cdot \log \|P\|_0   \cdot \min(k_i, 2^i) \cdot i) \cdot  \frac{1}{|\Omega|}  \cdot \frac{k\cdot k_i\cdot 2^{i(z-1)}}{(k+k_i\cdot 2^i)^2}} 
\end{eqnarray*}

Similarly, not conditioning on event $\calE$ implies
\begin{eqnarray*}
& &\mathbb{E}_{\Omega} \mathbb{E}_{g}  \left[ \sup_{v^{\calS,1}\in \mathbb{N}_{2^{-1}}}  \left\vert \frac{\sum_{C_j \in T_{i}}\sum_{p\in C_j\cap \Omega}|v_p^{\calS,2^{-1}}-q_j|w_p}{|\Omega|\cdot\left( \cost(P,\calS) + \cost(P,\greedy)\right)}g_p \right\vert \vert \overline{\calE}\right] \\
&\leq & O(1)\sqrt{\gamma \cdot k \cdot \log \|P\|_0   \cdot \min(k_i, 2^i) \cdot i) \cdot  \frac{k}{|\Omega|}  \cdot \frac{k\cdot k_i\cdot 2^{i(z-1)}}{(k+k_i\cdot 2^i)^2}} 
\end{eqnarray*}

We have $\mathbb{P}[\overline{\calE}]\leq 1/k^2$ due to Lemma \ref{lem:eventE}.
Plugging in $\|P\|_0 \leq \text{poly}(k,\varepsilon^{-1})$ and our choice of $|\Omega|$, we can combine the last two equations with the law of total expectation to obtain
\begin{eqnarray}
\nonumber
& &\mathbb{E}_{\Omega} \mathbb{E}_{g}  \left[\sup_{v^{\calS,1}\in \mathbb{N}_{2^{-1}}} \left\vert  \frac{\sum_{C_j \in T_{i}}\sum_{p\in C_j\cap \Omega}|v_p^{\calS,2^{-1}}-q_j|w_p}{|\Omega|\cdot\left( \cost(P,\calS) + \cost(P,\greedy)\right)}g_p \right\vert \right] \\
\label{eq:finalboundbase}
&\leq &O(1) \cdot  \sqrt{k \cdot \log k  \cdot \min(k_i , 2^i) \cdot \log^5 \varepsilon^{-1} \cdot \frac{1}{|\Omega|} \cdot \frac{k\cdot k_i\cdot 2^{i(z-1)}}{(k+k_i\cdot 2^i)^2}} 
\end{eqnarray}

For the term in Equation \ref{eq:base3}, we note that $\frac{q_j\cdot w_p}{\cost(C_j,\calS)}=\cost(P,\greedy)$. Thus, for every cluster, we have a net of size $1$, which means we have an overall net of size $k$. We thus obtain 
\begin{eqnarray*}
& &\mathbb{E}_{\Omega} \mathbb{E}_{g}  \left[ \sup_{v^{\calS,1}\in \mathbb{N}_{2^{-1}}} \max_{C_j \in T_{i}} \left\vert  \frac{\sum_{p\in C_j\cap \Omega}q_j\cdot w_p}{|\Omega|\cdot  \cost(C_j,\calS) }g_p \right\vert \vert \calE\right] \\
&\leq & O(1)\sqrt{ \log k \frac{k}{|\Omega|}  \cdot \frac{k\cdot k_i\cdot 2^i}{(k+k_i\cdot 2^i)^2}} 
\end{eqnarray*}

Similarly, conditioning on event $\overline{\calE}$ implies
\begin{eqnarray*}
& &\mathbb{E}_{\Omega} \mathbb{E}_{g}  \left[ \sup_{v^{\calS,1}\in \mathbb{N}_{2^{-1}}} \max_{C_j \in T_{i}}  \left\vert \frac{\sum_{p\in C_j\cap \Omega}q_j\cdot w_p}{|\Omega|\cdot  \cost(C_j,\calS) }g_p \right\vert \vert \overline{\calE}\right] \\
&\leq & O(1)\sqrt{\log k \frac{k^2}{|\Omega|}  \cdot \frac{k\cdot k_i\cdot 2^{i(z-1)}}{(k+k_i\cdot 2^i)^2}} 
\end{eqnarray*}

Combining both terms, using $\mathbb{P}[\overline{\calE}]\leq 1/k^2$ due to Lemma \ref{lem:eventE} and the law of total expectation, we obtain
\begin{eqnarray}
\nonumber
& &\mathbb{E}_{\Omega} \mathbb{E}_{g}  \left[  \sup_{v^{\calS,1}\in \mathbb{N}_{2^{-1}}} \max_{C_j \in T_{i}}  \left\vert\frac{\sum_{p\in C_j\cap \Omega}q_j\cdot w_p}{|\Omega|\cdot  \cost(C_j,\calS) }g_p \right\vert \right] \\
\label{eq:finalboundbaseq}
&\leq &O(1) \cdot  \sqrt{\frac{k \log k}{|\Omega|}} 
\end{eqnarray}

Combining the bounds in Equations \ref{eq:finalbound}, \ref{eq:finalboundlarge}, \ref{eq:finalboundbase} and \ref{eq:finalboundbaseq} for the respective terms in Equations \ref{eq:telescopesmall}, \ref{eq:telescopelarge}, \ref{eq:base1} and \ref{eq:base3} now yields the claim.


%\input{DP}

\section{Disclosure of Funding Acknowledgements}
Kapser Green Larsen was partially supported by the Independent Research Fund Denmark (DFF) under a Sapere Aude Research Leader grant No 9064-00068B. 

\erclogowrapped{5\baselineskip}David Saulpic has received funding from
the European Research Council (ERC) under the European Union's Horizon 2020
research and innovation programme (Grant agreement No.\ 101019564
``The Design of Modern Fully Dynamic Data Structures (MoDynStruct)''.

Chris Schwiegelshohn was partially supported by the Independent Research Fund Denmark (DFF) under a Sapere Aude Research Leader grant No 1051-00106B and the Innovation Fund Denmark under grant agreement No 0153-00233A. 

Omar Ali Sheikh-Omar was partially supported by the Innovation Fund Denmark under grant agreement No 0153-00233A.


\bibliographystyle{plain}
\bibliography{reference}
\newpage


%\appendix
%%
%%
%%\section{Analysis Details}
%
%
%\input{app_chaining}
%% \newpage

\begin{table*}[t]
\setlength{\tabcolsep}{1.5mm}
\centering

\subtable[On 11 discriminative test tasks following the T0 benchmark.]{
\resizebox{\textwidth}{!}{%
    \begin{tabular}{l|l|c|ccccc|ccc|cc|c|c}
    \toprule[1pt]
    \multirow{2}{*}{Base Model} &
    \multirow{2}{*}{Method} &
    \multirow{2}{*}{\#Params} & 
    \multicolumn{5}{|c|}{\textbf{Natural Language Inference}} & \multicolumn{3}{|c|}{\textbf{Sentence Completion}} & \multicolumn{2}{c|}{\textbf{Coreference}} & \multicolumn{1}{c|}{\textbf{WSD}} 
    & \multirow{2}{*}{Avg.}\\
    & & & RTE & CB & ANLI1 & ANLI2 & ANLI3 & COPA & Hella. & Story. & WSC & Wino. & WiC &  \\
    \midrule[1pt]
    Decoder-only & GPT-3 & 175B 
        &63.5 &46.4
        &34.6 &	35.4&	34.5&	91.0&	78.9&	83.2&	65.4&	70.2&	- & -\\
    Decoder-only & GLaM & 137B 
        & 56.3	& 39.3	& 39.7	& 35.5	& 34.1	& 90.0	& 76.7	& 81.1	& 82.1	& 71.3	& 50.6 & 59.7\\
    MoE Decoder-only & GLaM & 64B 
        & 66.8	& 33.9	& 40.9	& 38.2	& 40.9	& 90.0	& 77.1	& 82.5	& 83.5	& 73.4	& 50.5 & 61.6\\
    Decoder-only & PaLM & 540B 
        & 72.9	& 51.8	& 48.0	& 44.2	& 45.7	& 93.0	& 83.4	& 84.6	& 89.1	& 81.1	& 59.1 & 68.5\\
    Decoder-only & FLAN & 137B 
        & 78.3	& 64.1	& 47.7	& 43.9	& 47.0	& 90.6	& 56.4	& 92.2	& 80.8	& 67.3 & - & -\\
    \midrule[1pt]
    \multirow{3}*{\shortstack{ELECTRA}}
    & PE-CLS & 335M
        & 60.2	& 57.4	& 34.1	& 34.4	& 36.4	& 92.7	& 44.1	& 96.0	& 62.8	& 56.3	& 50.7	& 56.8
        \\
    & PE-PROB & 335M
        & 54.0	& 49.2	& 32.3	& 33.3	& 33.5	& 81.9	& 36.7	& 89.5	& 64.3	& 50.7	& 50.9	& 52.4 \\
    & PE-REP & 335M
        & 69.0	& 61.3	& 36.1	& 35.0	& 39.4	& 91.2	& 47.0	& 96.8	& 70.0	& 56.2	& 51.1	& 58.5
        \\
    \midrule
    \multirow{1}*{\shortstack{DeBERTaV3}}
    & \multirow{1}*{{UD (ours)}} & 304M
        & \multirow{1}*{71.1}
        & \multirow{1}*{76.8}
        & \multirow{1}*{43.8}
        & \multirow{1}*{41.3}
        & \multirow{1}*{45.7}
        & \multirow{1}*{96.0}
        & \multirow{1}*{60.7}
        & \multirow{1}*{97.4}
        & \multirow{1}*{66.4}
        & \multirow{1}*{83.6}
        & \multirow{1}*{53.3}
        & \multirow{1}*{66.9}
    \\
    \midrule[1pt]
    \multirow{2}*{\shortstack{T5-Large}}
    & \multirow{1}*{T0 $\star$} & 800M
        & 75.1	& 55.5	& 32.9	& 32.3	& 33.7	& 84.6	& 28.2	& 94.0	& 63.0	& 54.6	& 51.2	& 55.0 \\


    & {UD (ours)} & 400M
        & \textbf{83.8}
        & \textbf{80.4}
        & \textbf{36.8}
        & \textbf{34.2}
        & \textbf{42.2}
        & \textbf{90.0}
        & \textbf{56.1}
        & \textbf{96.4}
        & \textbf{68.3}
        & \textbf{62.9}
        & \textbf{54.6}	
        & \textbf{64.1} \\
    \midrule[1pt]
    \multirow{3}*{\shortstack{T5-XL}}
    & \multirow{1}*{T0 $\dagger$} & 3B
        & 64.6 
        & 45.4
        & 33.8
        & 33.1
        & 33.3
        & 72.4
        & 27.3
        & 84.0
        & 65.1
        & 51.0
        & 50.7
        & 51.0 \\

    & \multirow{1}*{T0 $\star$} & 3B
    & \textbf{79.7}	& 68.9	& \textbf{43.1}	& \textbf{38.5}	& 42.3	& \textbf{94.1}	& 31.5	& 97.5	& 68.8	& 61.3	& \textbf{54.1}	& 61.8\\

 
    & {UD (ours)} & 1.5B
        & 78.7
        & \textbf{73.2}
        & 41.2
        & 36.3
        & \textbf{45.4}
        & 94.0
        & \textbf{70.1}
        & \textbf{97.9}
        & \textbf{72.1}
        & \textbf{70.6}
        & 53.0	
        & \textbf{66.6} \\
    \midrule[1pt]
    \multirow{4}*{\shortstack{T5-XXL}}
    & \multirow{1}*{T0 $\dagger$} & 11B
        & 80.8
        & 70.1
        & 43.6
        & 38.7
        & 41.3
        & 90.0
        & 33.6
        & 92.4
        & 61.5
        & 59.9
        & 56.6
        & 60.8 \\

    & \multirow{1}*{T0 $\star$} & 11B
    & \textbf{85.8}	& 73.3	& 47.3	& 42.0	& 46.1	& 94.4	& 31.5	& 98.4	& 62.8	& 72.8	& 56.0	& 64.6 \\

    & {UD (ours)} & 5.5B
    & 80.5	& 87.5	& 49.0	& 42.9 & 	48.8	& 95.0	& 77.4	& \textbf{98.6}	& 73.1	& 82.2	& 57.1	& 72.0 \\

    & {UD+ (ours)} & 5.5B
    & 82.0	& \textbf{89.3}	& \textbf{53.4} & \textbf{48.1} & \textbf{51.0} & \textbf{96.0} & \textbf{78.9} & 96.7	& \textbf{75.0}	& \textbf{86.4}	& \textbf{58.5}	& \textbf{74.1} \\
    \bottomrule[1pt]
\end{tabular}
}
\label{tab:maintable:top}
}


\subtable[On 13 discriminative BigBench tasks following the T0 benchmark]{
\resizebox{0.7\textwidth}{!}{%
    \begin{tabular}{l|cc|cc|ccc|}
    \toprule[1pt]
    \multirow{1}{*}{Model} 
        & \multirow{1}{*}{\shortstack{T0-Large}}
        & \multirow{1}{*}{\shortstack{UD-large}}
        & \multirow{1}{*}{\shortstack{T0-XL}}
        & \multirow{1}{*}{\shortstack{UD-XL}}
        & \multirow{1}{*}{\shortstack{T0-XXL}}
        & \multirow{1}{*}{\shortstack{UD-XXL}}
        & \multirow{1}{*}{\shortstack{UD+-XXL}}\\
    \midrule[1pt]
    BigBench (Avg.) & 39.6 & \textbf{43.5} & 44.8 & \textbf{48.9} & 47.4 & 55.5 & \textbf{58.7} \\
    \bottomrule[1pt]
    \end{tabular}%
    }
\label{tab:maintable:middle}
}

\subtable[On 22 discriminative BBH tasks]{
\resizebox{\textwidth}{!}{%
    \begin{tabular}{l|ccc|ccc|cccc|}
    \toprule[1pt]
    \multirow{1}{*}{Model} 
        & \multirow{1}{*}{\shortstack{T0-Large}}
        & \multirow{1}{*}{\shortstack{Flan-T5-Large}}
        & \multirow{1}{*}{\shortstack{UD-Large}}
        & \multirow{1}{*}{\shortstack{T0-XL}}
        & \multirow{1}{*}{\shortstack{Flan-T5-XL}}
        & \multirow{1}{*}{\shortstack{UD-XL}}
        & \multirow{1}{*}{\shortstack{T0-XXL}}
        & \multirow{1}{*}{\shortstack{Flan-T5-XXL}}
        & \multirow{1}{*}{\shortstack{UD-XXL}}
        & \multirow{1}{*}{\shortstack{UD+-XXL}}\\
    \midrule[1pt]
    BBH (Avg.) & 38.9 & 39.5 & \textbf{44.2} & 40.4 & 44.6 & \textbf{47.3} & 45.0 & 49.4 & 51.3 & \textbf{56.7} \\
    \bottomrule[1pt]
    \end{tabular}%
    }
\label{tab:maintable:bottom}
}
\caption{
Zero-shot performance of our UD and baselines.
Results in the first block are reported by previous work, respectively from GPT-3~\cite{gpt3-paper}, GLaM~\cite{glam}, PaLM~\cite{palm}, and FLAN~\cite{FLAN}.
Note that we provide these reported results for reference, and do not compare directly. Some of the reported tasks are evaluated on the test split, while we follow the better baseline method T0 to report on validation splits.
Results with $\dagger$ are reported by~\citeauthor{T0-paper}, and results with $\star$ are reproduced in our framework. We reproduced the three variants of prompting ELECTRA~\cite{xia2022prompting} under our setting, denoted as ``PE-CLS'', ``PE-PROB'', ``PE-REP''.
Results for Flan-T5-Large/Xl/XXL~\citep{flant5} are reproduced by testing zero-shot performance on their released checkpoints.
In the same group, T0 and Flan-T5 has 2x model parameters compared to UD. For abbreviation, we denote UD based on T5-XX as ``UD-XX'', e.g., UD-XL refers to UD based on the T5-XL model.
}
\label{tab:maintable}
\vspace{-0.7cm}
\end{table*}
% \begin{table}[t]
\setlength{\tabcolsep}{4.5mm}
\centering
    \resizebox{0.5\textwidth}{!}{%
    \begin{tabular}{lcccc}
        \toprule[1pt]
        % \textbf{Finetuned Task} & \textbf{Task Type} & \textbf{Metric} & \textbf{Eval Set} & \textbf{SOTA Reference} & \textbf{SOTA} & \textbf{Ours}  \\
        \textbf{Finetuned Task} & \textbf{T0} & \textbf{UD}  \\
        \midrule[1pt]
        MRPC & 90.5 & 89.7 (to be improve) \\
        QQP & 85.9 (to improve) & \textbf{91.6} \\
        PAWS & 95.1 & \textbf{97.2} \\
        WikiQA  & 96.1 & \textbf{96.5}\\
        CosmosQA & 88.4 & \textbf{90.7}\\
        DREAM & 90.5 & \textbf{91.6} \\
        QuAIL & 65.6 & \textbf{80.2} \\
        QuaRel & 88.2 & \textbf{95.3}\\
        QuaRTz & 94.1 & \textbf{94.5} \\
        SciQ & 97.6& \textbf{98.1}\\
        SocialIQA &  \textbf{82.2} & 81.7 \\
        WikiHop & & 58.6\\
        Amazon & \textbf{97.6}(to improve) & 97.3 \\
        IMDB & \textbf{96.9} & 96.7\\
        Rotten & 93.4 & \textbf{93.6} \\
        Yelp & 72.28 (first two prompts) & 68.1 (hard to improve)\\
        AGNews & 95.1 & \textbf{95.3} \\
        DBPedia & & \\
        TREC & 96.7 & \textbf{97.8}\\
        \bottomrule[1pt]
    \end{tabular}
    }
    \caption{Results on finetuned tasks for UD and the baseline T0. Both methods use T5-XXL as a base model. T0 has 2x model parameters compared to UD.}

    % compared with state-of-the-art results.}
    \label{tab:finetunedtasks}
\end{table}
\begin{table*}[!htp]
\setlength{\tabcolsep}{1.5mm}
\centering
\subtable[On 11 discriminative test tasks following the T0 benchmark.]{
\resizebox{\textwidth}{!}{%
    \begin{tabular}{l|ccccc|ccc|cc|c|c}
        \toprule[1pt]
        \multirow{2}*{Method}
        & \multicolumn{5}{c|}{\textbf{Natural Language Inference}} & \multicolumn{3}{c|}{\textbf{Sentence Completion}} & \multicolumn{2}{c|}{\textbf{Coreference}} & \multicolumn{1}{c|}{\textbf{WSD}} & \multirow{2}{*}{Avg.} \\
    & RTE & CB & ANLI1 & ANLI2 & ANLI3 & COPA & Hella. & Story. & WSC & Wino. & WiC &  \\
    \midrule[1pt]
    T0-XL %
        & \textbf{79.7}	& 68.9	& 43.1	& 38.5	& 42.3	& \textbf{94.1}	& 31.5	& \textbf{97.5}	& \textbf{68.8}	& 61.3	& \textbf{54.1}	& 61.8\\
    GenUD-XL %
        & 71.5	& \textbf{80.4}	& \textbf{43.1}	& \textbf{39.5}	& \textbf{42.6}	& 94.0	& \textbf{55.8}	& 96.7	& 63.5	& \textbf{75.5}	& 52.8	& \textbf{65.0}\\
    \bottomrule[1pt]
    \end{tabular}%
}
\label{tab:genud:top}
}
\subtable[On 13 discriminative Big-Bench tasks following the T0 benchmark.]{
\resizebox{\textwidth}{!}{%
    \begin{tabular}{l|ccccccccccccc|c}
    \toprule[1pt]
    \multirow{2}{*}{Model} 
        & \multirow{2}{*}{\shortstack{code \\ desc.}}
        & \multirow{2}{*}{\shortstack{conce\\-ptual}}
        & \multirow{2}{*}{\shortstack{known\\unknowns}}
        & \multirow{2}{*}{\shortstack{logic \\ grid}}
        & \multirow{2}{*}{\shortstack{logic \\ deduction}}
        & \multirow{2}{*}{\shortstack{miscon\\-ceptions}}
        & \multirow{2}{*}{\shortstack{novel\\concepts}}
        & \multirow{2}{*}{\shortstack{strate\\-gyqa}}
        & \multirow{2}{*}{\shortstack{wino\\-why}}
        & \multirow{2}{*}{\shortstack{syllo\\-gisms}}
        & \multirow{2}{*}{\shortstack{movie\\dialog}}
        & \multirow{2}{*}{\shortstack{lang\\-uage\_id}}
        & \multirow{2}{*}{\shortstack{vita\\-minc}} 
        & \multirow{2}{*}{Avg.} \\
    &&&&&&&&&&&&&&\\
    \midrule
    T0-XL & 23.4 & 48.1 & 64.6 & \textbf{42.5} & 50.1 & \textbf{52.7} & 25.0    & 53.1 & 45.4 & 50.2 & 47.7 & \textbf{19.0} & 60.0 & 44.8 \\
    GenUD-XL & \textbf{60.0} & \textbf{64.1} & \textbf{69.6} & 38.2 & \textbf{52.8}  & 48.9 & \textbf{44.1} & \textbf{57.1} & \textbf{46.5} & \textbf{50.4} & \textbf{50.9} & 15.5 & \textbf{66.8} & \textbf{48.9} \\
    \bottomrule[1pt]
    \end{tabular}%
\label{tab:genud:mid}
}
}




\subtable[On 15 generative tasks from Big-Bench]{
\resizebox{\textwidth}{!}{%
    \begin{tabular}{l|ccccccccccccccc|c}
    \toprule[1pt]
    \multirow{3}{*}{Model}
        & \multirow{3}{*}{\shortstack{auto \\ debugging}}
        & \multirow{3}{*}{\shortstack{simple \\ arith \\ -metic}}
        & \multirow{3}{*}{\shortstack{repeat\\copy \\ logic}}
        & \multirow{3}{*}{\shortstack{sufficient \\ information}}
        & \multirow{3}{*}{\shortstack{simple \\ text \\ editing}}
        & \multirow{3}{*}{\shortstack{scientific \\ press \\ release}}
        & \multirow{3}{*}{\shortstack{code\\ names}}     
        & \multirow{3}{*}{\shortstack{emoji\\movies}}
        & \multirow{3}{*}{\shortstack{penguins\\in a \\ table}}
        & \multirow{3}{*}{\shortstack{few \\ shot\\nlg}}
        & \multirow{3}{*}{\shortstack{operators}}
        & \multirow{3}{*}{\shortstack{tense}}
        & \multirow{3}{*}{\shortstack{geometric\\shapes}}
        & \multirow{3}{*}{\shortstack{chinese \\ remainder\\ theorem}}
        & \multirow{3}{*}{\shortstack{temporal\\sequences}}
        & \multirow{3}{*}{\shortstack{Avg.}}\\
    &&&&&&&&&&&&&&&&\\[1em]
    \midrule
    T0-XL & 11.2 & 6.7 & \textbf{25.8} & 33.8 & 7.5 & \textbf{6.7} & \textbf{44.8} & \textbf{8.7} & \textbf{11.4} & 17.4 & \textbf{10.5} & 80.7 & 0.0 & 0.0 & 14.0 & \textbf{18.6}\\
    GenUD-XL & \textbf{15.5} & 6.7 & 8.2 & \textbf{34.4} & \textbf{12.6} & 6.4 & 25.1 & 0.0 & 8.1 & \textbf{20.5} & 3.7 & \textbf{80.9} & 0.0 & 0.0 & \textbf{33.5}  & 17.0\\
    \bottomrule[1pt]
    \end{tabular}%
}
}

\caption{Zero-shot performance for generalized UD and T0 on discriminative and generative tasks. 
We select the top 15 uncommon generative tasks from BigBench basing on ascending order of data size. (We assume that datasets with smaller sizes are less common, and more suitable for zero-shot tests.) The metrics are respectively accuracy for discriminative tasks and ROUGE1 for generative tasks. ``GenUD'' denotes our generalized UD method.}
\label{tab:genud}
\end{table*}







\begin{table}[htbp]
\setlength{\tabcolsep}{1.5mm}
  \centering
\resizebox{0.35\textwidth}{!}{
    \begin{tabular}{lcc}
    \toprule
    \textbf{Dataset} & \textbf{SOTA} & \textbf{UD+-XXL} \\
    \midrule
    QQP     & \textbf{90.60}  & 90.44 \\
    DREAM     & 91.80  & \textbf{94.95} \\
    QuAIL   & 87.20  & \textbf{88.13} \\
    IMDB    & 97.30  & \textbf{97.44}  \\
    AgNews   & \textbf{95.58}  & 95.56  \\
    OBQA   & 87.20  & \textbf{89.20} \\
    STSB     & 92.30  & \textbf{92.90} \\
    CSQA    & \textbf{84.90}  & 84.68  \\
    SST-2     & 97.30  & \textbf{97.48} \\
    QNLI    & 96.50  & \textbf{96.56} \\
    AbductiveNLI &  89.80  & \textbf{93.20} \\
    VitaminC   & 91.10  & \textbf{92.62} \\
    MNLI  &  \textbf{92.10}  & 92.03  \\
    MCScript &  97.30  & \textbf{98.03} \\
    MCScript 2.0 &  97.90  & \textbf{98.01} \\
    AdversarialNLI (r3) &53.50  & \textbf{67.83 } \\
    COLA   & \textbf{71.50}  & 71.42  \\
    \midrule
    Avg.   & 89.05  & \textbf{90.62} \\
    \bottomrule
    \end{tabular}%
}
  \caption{Results on fully-supervised tasks for UD, which is based on the encoder of T5-xxl. Previous sota model \citep{ul2} has 4x model parameters compared to UD. }
  \label{tab:finetune}%
\vspace{-0.7cm}
\end{table}%



\section{Experiments}

\begin{table*}[t]
\setlength{\tabcolsep}{1.5mm}
\centering
\small
\resizebox{\textwidth}{!}{%
    \begin{tabular}{l|ccccc|ccc|cc|c|c}
        \toprule[1pt]
        & \multicolumn{5}{c|}{\textbf{Natural Language Inference}} & \multicolumn{3}{|c|}{\textbf{Sentence Completion}} & \multicolumn{2}{c|}{\textbf{Coreference}} & \multicolumn{1}{c|}{\textbf{WSD}} & \multirow{2}{*}{Avg.} \\
    & RTE & CB & ANLI1 & ANLI2 & ANLI3 & COPA & Hella. & Story. & WSC & Wino. & WiC &  \\
    \midrule[1pt]
    UD (Minimal)     & \textbf{83.75}
        & \textbf{80.36}
        & 36.80
        & \textbf{34.20}
        & \textbf{42.17}
        & \textbf{90.00}
        & \textbf{56.07}
        & \textbf{96.37}
        & \textbf{68.27}
        & \textbf{62.90}
        & \textbf{54.55}	
        & \textbf{64.13} \\
    UD (Instructive)    & 72.24 
        & 64.52 
        & \textbf{36.98} 
        & 33.40 
        & 39.73 
        & 85.31 
        & 45.15 
        & 96.01 
        & 65.38 
        & 53.94 
        & 50.94 
        & 58.51\\
    \midrule
    T0 (Minimal) & 61.56  & \textbf{57.81}  & 30.57  & 30.27  & 33.38  & 67.19  & \textbf{33.81}  & 66.56  & 60.94  & 52.81  & \textbf{51.72}  & 49.69  \\
    T0 (Instructive) & \textbf{75.05}	& 55.48	& \textbf{32.87}	& \textbf{32.29}	& \textbf{33.67}	& \textbf{84.59}	& 28.24	& \textbf{93.97}	& \textbf{62.98}	& \textbf{54.59}	& 51.16	& \textbf{54.99} \\

    \bottomrule[1pt]
    \end{tabular}}
    \caption{Zero-shot performance for UD and T0 respectively with instructive and minimal prompts. Instructive prompts are lengthy descriptions of tasks \citep{T0-paper}, while minimal prompts use a simple concatenation of input data.}
\label{tab:promptablatiion}
\end{table*}

\begin{table*}[ht]
\setlength{\tabcolsep}{0.9mm}
\centering
\resizebox{\textwidth}{!}{%
    \begin{tabular}{l|l|ccccc|ccc|cc|c|c}
        \toprule[1pt]
        & \multirow{2}*{Base Model}
        & \multicolumn{5}{c|}{\textbf{Natural Language Inference}} & \multicolumn{3}{|c|}{\textbf{Sentence Completion}} & \multicolumn{2}{c|}{\textbf{Coreference}} & \multicolumn{1}{c|}{\textbf{WSD}} & \multirow{2}{*}{Avg.} \\
    & & RTE & CB & ANLI1 & ANLI2 & ANLI3 & COPA & Hella. & Story. & WSC & Wino. & WiC &  \\
    \midrule[1pt]
    \multirow{2}*{\shortstack{Encoder}}
    & DeBERTa-V3 (304M) 
        & 71.1
        & 76.8
        & 43.8
        & 41.3
        & 45.7
        & 96.0
        & 60.7
        & 97.4
        & 66.4
        & 83.6
        & 53.3
        & 66.9 \\
    & DeBERTa-V2 (1.5B) 
        & 77.6
        & 80.4
        & 43.2
        & 39.3
        & 44.8
        & 95.0
        & 67.2
        & 98.2
        & 74.0	& 82.1 & 56.0	& 68.9\\ \midrule
    \multirow{2}*{\shortstack{Enc-Dec}} & T5-Encoder (400M) 
        & 75.1	& 55.5	& 32.9	& 32.3	& 33.7	& 84.6	& 28.2	& 94.0	& 63.0	& 54.6	& 51.2	& 55.0 \\
    & T5-Encoder (1.5B)  & 79.7	& 68.9	& 43.1	& 38.5	& 42.3	& 94.1	& 31.5	& 97.5	& 68.8	& 61.3	& 54.1	& 61.8\\
    \midrule
    \multirow{1}*{\shortstack{Decoder}}
    & \multirow{1}*{GPT-XL (1.5B)}
        & \multirow{1}*{71.1}
        & \multirow{1}*{75.0}
        & \multirow{1}*{30.4}
        & \multirow{1}*{31.8}
        & \multirow{1}*{37.8}
        & \multirow{1}*{71.0}
        & \multirow{1}*{40.9}
        & \multirow{1}*{87.7}
        & \multirow{1}*{62.5}
        & \multirow{1}*{54.5}
        & \multirow{1}*{50.3}
        & \multirow{1}*{55.7}
    \\
    \bottomrule[1pt]
    \end{tabular}}
    \caption{Ablation study on different backbone models. We experiment with base models of different architectures and scales. ``Enc-Dec'' refers to models that are pretrained in an encoder-decoder manner.}
    \label{tab:ablationbasemodel}
\end{table*}
\begin{table}
\centering
\setlength{\tabcolsep}{3.0mm}
\resizebox{0.5\textwidth}{!}{%
\begin{tabular}{l|c}
    \toprule[1pt]
    Setting & Accuracy \\
    \midrule[1pt]
    True Data vs Manually-Generated Data & 80.0 \\
    True Data vs Model-Generated Data & 74.4 \\
    \bottomrule[1pt]
    \end{tabular}%
    }
    \caption{
    The accuracy of UD discriminating real data and generated data. We feed UD with a real sample $x$ from the real-world data distribution, and a sample $x'$ from manual generation or model-based generation. 
    If UD assigns higher score to $x$ than $x'$ (i.e., $D(x)>D(x')$), it is considered an accurate prediction.
    }
  \label{tab:explain}%
\end{table}%



\subsection{Experimental Setup}\label{sec:setup}

We performed extensive experiments to validate the performance of the zero-shot generalization of our UD. We follow the same zero-shot setting as T0~\citep{T0-paper} by training on multi-task datasets and evaluating a held-out set of tasks that are never seen during training. 

\paragraph{Datasets}
The original T0 training set consists of 38 tasks of 8 different types.
% ~\footnote{We did not consider T0+ and T0++, since they are partially intersected with the test sets, making some test tasks unable to be evaluated under the zero-shot setting.}
There are in total 21/38 discriminative training tasks, with which we train the UD.
% ~\footnote{The original paper~\citep{T0-paper} claims 39 training datasets but releases a training set with 38 datasets (``common\_gen''  excluded). We directly start with the released data.}. 
% It consists of a majority of discriminative tasks and a small number of generative tasks.
% We train our \method with 21 discriminative tasks within.
% , which is around 55\% of the original T0 training data.\xhk{change to 21/38=0.55}
The evaluation set covers four types of tasks, including natural language inference (RTE~\citep{2005_RTE}, CB~\citep{de2019_CB}, ANLI/R1-R3~\citep{NieWDBWK20_ANLI}), coreference resolution (WSC~\citep{WSC2012}, Winogrande~\citep{SakaguchiBBC20_winogrande}), sentence completion (COPA~\citep{COPA2011}, StoryCloze~\citep{story_cloze}, Hellaswag~\citep{ZellersHBFC19_hellaswag}), and word sense disambiguation (WiC~\citep{wic-paper}).
Following T0, we use accuracy on the validation split as the evaluation metric.
For prompt-based baselines, we report the average accuracy over multiple prompts for each test task.
Besides, we also evaluate zero-shot performance on several BigBench~\cite{bigbench} tasks, which are also adopted by T0~\cite{T0-paper}.\footnote{The original T0 reported results on 14 BigBench tasks. We separately report the results of 13 discriminative tasks and the other generative task in the following.}


% \lzy{Noted that we also evaluate the zero-shot performance on a subset of Big-Bench Benchmark~\cite{bigbench} adopted by original T0 paper~\cite{T0-paper}.\footnote{The original T0 reported results on 14 BigBench tasks. In our work, we focus on 13 discriminative tasks, leaving improving performance of the only generative tasks for future exploration. \zy{No need to say that. We also have generation results. Just say we're gonna report generation result separately.}}}


\paragraph{Baselines}
We primarily compare our method with T0~\citep{T0-paper}, which is a generative approach.
% that shares the same goal as UD (i.e., zero-shot generalization), but uses a totally different framework (i.e., generative or discriminative) as well as input format (i.e., prompt or minimal prompt).
Another baseline is prompting ELECTRA~\cite{xia2022prompting} which is a recent work on discriminative modeling.
Since it was proposed in a different setting (i.e., a  few-shot setting or direct zero-shot inference without any finetuning), we reproduced their method under our multitask zero-shot setting for comparison.

For a fair comparison, we follow T0 to use the T5-V1.1-LM-Adapted~\citep{T5-paper} as the backbone model, and we experimented with three different scales, respectively 800M, 3B, and 11B. 
For UD, it only makes use of the encoder of T5-v1.1 and additionally replaces the output layer with a classification head.
Moreover, for direct comparison with \citet{xia2022prompting}, we use DeBERTaV3-Large \citep{debertav3} as the backbone model which shares the same bidirectional architecture and has a smaller number of parameters.

In addition, we also provide reported zero-shot results of several large language models (with hundreds of billions of parameters) for reference, including GPT-3~\cite{gpt3-paper}, GLaM~\cite{glam}, PaLM~\cite{palm}, and FLAN~\cite{FLAN}.


% we also experiment with another backbone DeBERTaV3-Large~\citep{debertav3} to achieve better zero-shot performance.

\paragraph{Training}
% We implemented both baselines and our method, and perform experiments with exactly the same environments.
During training, we truncate the input sequence to 256 tokens and use a batch size of 256. For optimization, we use the Adam optimizer with a fixed learning rate of 1e-5 and a dropout rate of 0.1. Each experiment is trained with 10, 8, and 5 epochs respectively for 800M, 3B, and 11B models.
% We perform checkpoint selection by directly using the final (fixed-epoch) checkpoint for evaluation.
% \xhk{by choosing the one with the maximal average zero-shot performance per xxx steps.}

% For data processing, similar to T0, we truncate any dataset with over MAX\_DATA\_SIZE to have MAX\_DATA\_SIZE / num\_prompts. 
% Different from ~\citet{T0-paper} that uses a value of 500k for MAX\_DATA\_SIZE, we use a value of 50k, which experimentally yields better zero-shot performance for the T0 baseline.
% The training data of UD are produced by \xhk{replacing different prompted data version from T0 training data with only one minimal prompted version}, which strictly guarantees all methods share same raw task data.




% \subsection{Main Results}
\subsection{Main Results on Zero-Shot Tasks}

\paragraph{UD Zero-Shot Results}
The main results are presented in Table~\ref{tab:maintable}.
We compare methods of similar scales. 
Results in Table \ref{tab:maintable:top} show that our UD substantially outperforms the T0 baseline on average by a large margin of around 9, 5, and 7 points respectively at Large, XL, and XXL scales.
Comparing the results of UD-T5-Large, UD-DeBERTaV3, and prompting ELECTRA, both variants of UD also substantially outperform prompting ELECTRA by more than 6 points.
% In addition, UD also demonstrates superior zero-shot ability compared with models with hundreds of billions of parameters (see results in the first block of Table~\ref{tab:maintable:top}).
On BIG-Bench datasets, results in Table \ref{tab:maintable:bottom} show that our UD outperforms the T0 baseline by a margin of around 4-8 points.
Overall, these results demonstrate the advantages of UD at every scale, and a broad range of tasks compared with baselines.

Another interesting finding is that the advantages of UD significantly increase along with scaling.
When scaling from Large-scale to XL-scale (i.e., around 3.75x of the parameters), the average performance improves by around 2 points. However, when scaling from XL-scale to XXL-scale (i.e., 3.6x of the parameters), the improvements of average zero-shot performance enlarge to 8 points.
Based on the observation, we hypothesize that UD can achieve even better performance of zero-shot generalization if further scaling to an even larger models, which we leave to future work.

% Results show that our \method substantially outperforms our baseline T0 on average zero-shot performance, by a large margin of around 8, 5, and 7 points respectively at the Large (800M), XL (3B), and XXL (11B) scales.
% \xhk{Our UD also substantially outperforms ELECTRA in the Large (800M) scale.} 

To further boost the zero-shot performance, we also train a new variant of UD at 11B scale by scaling to more training tasks, including the discriminative English tasks used in \citet{1600tasks}, and the discriminative English tasks used in \citet{ul2}. The new model is denoted as UD+.
UD+ achieves the highest average accuracy among all the zero-shot evaluation tests.

% \begin{comment}
% ul2 (CommonsenseQA \cite{commonsense_qa}, ),
% csqa2.json 9264
% glue_cola.json 8551
% glue_sst2.json 67349
% glue_stsb.json 5749
% mcscript.json 19462
% mcscript2.json 28382
% openbookqa.json 19828
% qasc.json 40670
% qasc_with_ir.json 40670
% race_high.json 249780
% race_middle.json 101684
% social_i_qa.json 100230
% super_glue_boolq.json 9427
% super_glue_multirc.json 27243
% ai2_science_elementary.json 2493
% ai2_science_middle.json 2424
% onestopqa_advanced.json 1296
% physical_iqa.json 33211
% protocol_comparison_harsht.json 3698
% reclor.json 3726
% ai2_arc_ARC_Easy.json 9002
% ai2_arc_ARC_Challenge.json 4476r,  
% \end{comment}
% \zy{what data?}

% \xhk{move bigbench result in appendix A.1 here. into Table 2}
% \yn{add bigbench analysis}

% , in the mean time still guaranteeing that there is no overlap between training tasks and the held-out tasks.

% \xhk{We also extend our training datasets (please refer to appendix~\ref{sec:ud_plus_data}) and train a model UD+. 
% UD+ achieves the highest average accuracy among all the zero-shot evaluation test, in the mean time still guaranteeing that there is no overlap between training tasks and the held-out tasks.}

% \yn{do we need to add the following?}
% \yn{
% Interestingly, we also have observed some findings on zero-shot performance along with scaling.
% For baseline T0, the zero-shot performance keeps improving on most of the datasets when scaling to larger-scale models.
% Exceptions are Hellaswag and WSC, where zero-shot performance on them are basically unchanged when scaling.
% For our \method, the performance of zero-shot generalization consistently improves with the model scale increasing on all sentence completion and coreference resolution tasks, and partial NLI tasks.
% Exceptions are that RTE, CB and WSC demonstrates a degradation on zero-shot performance when scaling from large to XL scale.
% This could be explained that 
% % {\color{red} xxxxx}
% % \xhk{I guess if we add minimal prompts, RTE and CB will improve for larger model...? so maybe we can remove this observation for now?}
% }


% \paragraph{Results on Finetuned Tasks}

% To evaluate the performance on finetuned tasks, we finetuned T0/UD respectively on each training task. This is similar to multi-task finetuning \cite{T5-paper}.
% % We use this experiment to test the effectiveness of UD with abundant labels. 
% We experimented with all the T0 discriminative training tasks.
% Table~\ref{tab:finetunedtasks} shows the finetuning results on T0 and UD at the 11B scale.
% We observe that UD outperforms T0 on \textbf{\color{red} xxx/19} of the considered finetuned tasks.
% To be specific, on topic classification tasks, paraphrase identification tasks, and multiple-choice QA tasks, UD shows the largest advantages against T0. These finetuning results demonstrate that UD does not only perform well in the zero-shot setting but also improves performance when abundant labels are available.



% \yn{add a new subsection of seq2seqUD}
\paragraph{Generalized UD Zero-Shot Results}

The zero-shot results of generalized UD on 11 T0 discriminative test tasks and on 13 Big-Bench tasks are respectively reported in Table~\ref{tab:gen_ud:top} and Table~\ref{tab:gen_ud:mid}.
In addition, to test how generalized UD performs on zero-shot generative tasks, we also select 4 generative tasks from Big-Bench for evaluation. Results are presented in Table~\ref{tab:gen_ud:bottom}.


Analyses are as follows.
(1) Comparing the results of generalized UD and T0, generalized UD still holds significant improvements on discriminative tasks.
(2) Comparing generalized UD with our previous UD (in Table~\ref{tab:maintable}), we observe there is a slight decrease in average performance, proving that adding generative tasks into training could have impacted a little bit, in trade for capability for handling generative tasks.
(3) On 4 generative zero-shot tasks, both generalized UD and T0 show comparable results.
(4) On 13 discriminative BigBench tasks, we observe that UD-Large outperforms T0-Large by 6.67\%, UD-XL outperforms T0-XL by over 4\%, and Generalized UD-XL outperforms T0-XL by over 6\%, further indicating the effectiveness of our proposed framework.


%\yn{recheck the analysis along with table data!}

% Comparing Generalized UD with methods in Table~\ref{tab:maintable} (i.e., UD and T0) of similar scales, we observe the zero-shot performance on discriminiative tasks slightly decrease but generally hold still, compared to UD (ours).

% It still significantly outperforms baseline T0 to a large degree.
% From Table~\ref{tab:gen_ud}, we shall observe, on generative tasks both generalized UD and T0 show comparable results.}









\subsection{SOTA Results on Finetuned Tasks}
\label{sec:ud_finetune}

To explore how UD performs on fully-supervised tasks, we finetuned UD for a wide range of downstream tasks and reported their results in Table \ref{tab:finetune}.
% To explore whether UD can help improve the performance in fully-supervised learning, we conduct experiments by finetuning each downstream task. 
For each finetuning experiment, the maximum training epoch is set to be 10.
We search a hyper-parameter space with learning rate in \{2e-5, 1e-5, 5e-6\}, batch size in \{32, 64, 128\}.
We select the best checkpoint using a validation set with early stopping.
% We set the maximum training epoch to 10, search the hyper-parameters (learning rate in \{2e-5, 1e-5, 5e-6\}, batch size in \{32, 64, 128\}) and select the best checkpoint based on the validation set with early stopping.

% Results are in Table \ref{tab:finetune}.
From results in Table \ref{tab:finetune}, we find that UD can achieve remarkable performance on most of the downstream tasks. 
We achieve state-of-the-art performance on 12 out of the 17 tasks we evaluated. The results also show that more challenging tasks (tasks that require more knowledge) will benefit more from the multi-task training period, especially some QA tasks.








\subsection{Ablation Study}

We have also conducted ablation studies to further explore how several factors affect the performance of zero-shot generalization. 

\subsubsection{Instructive Prompts vs Minimal Prompts}

UD employs minimal prompts that use simple concatenation, while previous approaches rely on lengthy instructive prompts to provide more detailed instructions \cite{T0-paper,FLAN,gpt3-paper}. 
Statistically, we count the average number of prompt words (excluding raw input) for both minimal and instructive prompts, and statistics are respectively $0.4$ versus $>10$.
% \xhk{A statistic comparison on the average number of the prompt word count (excluding raw input) is $0.4$ for minimal prompts versus $>10$ for previous instructive prompts.}  
We compare these two types of prompts in the following experiment.
We adopt the instructive prompts from T0 and apply them on UD without changing the discriminator formulation. To construct minimal prompts for T0, we remove all the instructive words similar to UD.

% It is an interesting question whether minimal prompts also play a role in the \method, 
%considering that concatenating task data with prompts theoretically indeed reduces all tasks into the original LM tasks, hence improving task generalization.
% considering that simple concatenation of task data's keywords with a minimal prompt is enough to unify it into the UD format.

% We compare the zero-shot performance when using prompt and minimal prompt for \method and prompt and prompt-free for T0. 



% To construct instructive prompts for UD, we adopt the instructive prompts from T0 we concatenated the prompted inputs and each target choice (verbalizer).~\footnote{Here we use the same prompts as T0.} The corresponding \method label is 1 when concatenating correct target choice and 0 otherwise.

% \xhk{To construct prompt-free inputs for T0, we directly remove all the prompt words, still letting the model to predict the target verbalizer.}


Results are shown in Table~\ref{tab:promptablatiion}. We observe that minimal prompts yield better performance for UD than instructive prompts. In contrast, for T0, instructive prompts perform much better than minimal prompts. These results are consistent with our motivation that UD tends to unify the tasks better with a shared discrimination formulation. As a result, task-specific instructions are not necessary and might hurt generalization performance. Generative approaches, on the other hand, rely on instructive prompts to better distinguish different tasks.


% \xhk{for our UD method, minimal prompt version has better accuracy, because under the unified UD task format, task descriptive language in prompt is no longer needed and may even increase the sentence complexity to be understood by LM. Additionally, UD's tasks is to discriminate between correct and wrong choices where prompts are identical phrases in each choice's concatenated sentence, so prompts actually play no role in the discriminating process. However, for T0, prompted version has better accuracy than the prompt-free version (note that prompt-free is the extreme and usual case for minimal prompting) because generative model's goal is to generate the correct verbalizer from the huge vocabulary, which can be efficiently narrowed by the existence of prompts. Therefore, we can conclude that minimal prompted format works well for discriminative models and prompted format works well for generative models.}



\subsubsection{Ablation on Base Models}

We also study the effects of using different backbone pretrained models. We experiment with three backbone models of different types, respectively the encoder part of an encoder-decoder model, an encoder model, and a decoder model. Specifically, we use the T5 encoder, DeBERTa \cite{debertav3}, and GPT \cite{radford2018gpt} respectively for these three types. It is noteworthy that though similar in architecture for both T5 encoder and DeBERTa, they are pretrained with different self-supervised language modeling tasks, which in fact leads to huge differences in zero-shot generalization, as we will show in Table~\ref{tab:ablationbasemodel}.
% We study the effect of different backbone pretrained models. We experiment with three types of backbone models---using the encoder part of an encoder-decoder model, using an encoder model, and using a decoder model. We use the T5 encoder, DeBERTa \cite{debertav3}, and GPT \cite{radford2018gpt} respectively for these three types.






% We study the effect of different types of models (discriminative vs. generative), or backbone models (auto-encoding vs. auto-regressive), on zero-shot generalization with \method. In addition to T5-Encoder, we also experiment the advanced DeBERTaV3-Large~\cite{debertav3} that has achieved new SOTA on a diverse set of tasks. We also implement GPT-XL for comparision.

% Results are shown in Table~\ref{tab:promptablatiion}.

% shows the results between discriminative and generative models with fixed prompted or not version. It can be observed \xhk{no matter we use promped data or minimal prompt/prompt-free data, our discriminative models always have better zero-shot generalization performance than generative models.}



Results of different backbone models are presented in Table \ref{tab:ablationbasemodel}. 
Among all three types of backbone models, the encoder backbone models appear to be the most suitable type of backbone, where both encoder models of two scales respectively achieve the best and the second best results, outperforming all the others by more than 5 points.

Using the same number of parameters (i.e., 1.5B), both DeBERTa-V2 and T5-Encoder significantly outperform GPT-XL, which demonstrates that a bidirectional architecture works better than the unidirectional architecture for the discriminator formulation.
In addition, DeBERTa-V2 outperforms T5-Encoder by 7 points, implying that not only model architecture but also the self-supervised pretraining task determines the ability of UD discrimination. Models pretrained with masked language modeling tasks are more suitable for UD.

The impacts of the architecture and pretraining tasks of backbone models are even larger than the influence of scale, as we also observe that an encoder model with 300M parameters (i.e., DeBERTaV3) achieves much better performance than the T5 encoder and GPT-XL with 1.5B parameters.

% Results are shown in Table \ref{tab:ablationbasemodel}. Using the same number of parameters, encoder backbone models (i.e., DeBERTa) substantially outperform the T5 encoder and the GPT decoder. This indicates that pretrained encoders are more suitable for our discriminator formulation. Interestingly, an encoder model with 300M parameters (i.e., DeBERTaV3) achieves much better performance than the T5 encoder and GPT-XL with 1.5B parameters.








% Table~\ref{tab:ablationbasemodel} shows the results for different discriminative models, where DeberTa consists of solely an encoder, T5-Encoder is the encoder part of the full T5 model, GPT-XL consists of an encoder and a decoder. We can observe that the encoder structure performs better for discriminative tasks.

% \subsection{What Contribute to the Zero-Shot Generalization of \method?}

\subsection{How Well UD Generalizes to a Broader Domain?} \label{sec:generalize}

In the previous sections, we have trained UD to solve the task of discriminating whether a text sample comes from the true data distribution of natural language. So far we have constrained the problem to supervised labeled tasks. However, this discrimination problem formulation is in fact general and can be applied to a broader domain of natural language. We conduct the following experiment to see how UD generalizes.


% In order to explore the mechanism of the universal discriminator and explain how it promotes zero-shot generalization. We conduct the following extensive experiment.

To test whether a model discriminates against the true data distribution, a straightforward way of verification is to compare the probability of real data with that of some generated, fake data. This form of verification is not specific to any downstream task and can be viewed as generalizing to a broader domain. Formally, given a text sample $x$, let $D(x)$ be the output of UD, which estimates the probability that $x$ is sampled from the true data distribution, i.e., $P(\text{true} | x)$. Given a true data sample $x$ and a generated data sample $x'$, we expect a well-trained UD to predict $D(x) > D(x')$.

% First, we assume that the essence of our universal discriminator $D$ is to learn whether the data are sampled from the real text distribution or not. A straightforward way to verify this key point is to compare the likelihood of the real data label given real data x computed as $D(x)=p(y=1|x)$ with the likelihood of the real data label given generated data $x'$ computed as $D(x’)=p(y=1|x’)$. 

Specifically, we randomly select 2,600 real data samples $x$ from the validation set of the T0 training data and generate the data $x’$ in two different ways: model-based generation and manual generation.

For a model-based generation, we utilize the T0-Large model with a paraphrase prefix ``Paraphrase the sentence:'' to generate data $x'$. It is expected that the generated samples $x'$ are similar to true samples $x$ to some extent but demonstrate some flaws that are unique to generated data. For a manual generation, we manually create some conflict or contradiction in the real sample $x$. Specifically, we manually attach wrong answers to the original data and obtain $x’$ , which is similar to what we have done in constructing negative samples in our main framework. 

We then use our \method based on T5-Encoder Large to compute the probability $D(x)$ and $D(x')$ for both real and generated data. As displayed in Table~\ref{tab:explain}, we find that the \method assigns a higher score for $x$ than $x'$ $80\%$ of the time for manually-generated data. When tested with model-generated data, UD assigns a high probability for real data in $74\%$ of the cases.
This is probably because manually generated data are more paradoxical and logically incoherent and thus are easier for UD to discriminate. Overall, these results demonstrate that the discrimination ability of UD is not limited to the downstream tasks on which it was trained, but is also generalizable to a broader domain of text data. This indicates a possibility of extending UD to other scenarios such as model pretraining and generation tasks.


% For model-based generation, we utilize two models which generate high-quality and low-quality data $x$. It should be noted that we hope the generated $x'$are similar to $x$ to some extent.
% First, we leverage a T5-small model [citation] to generate similar semantics to real data x by feeding the $x$ with the prefix  ‘paraphrase:’. Obviously, the generated $x'$ are bound to be far from real data distribution. Then, we utilize the T5-small model to finetune on quora for paraphrase identification task. Then we leverage the finetuned T5-small model to do the same paraphrase generation as before and yield $x’$ with relatively high quality. 

% For heuristic-based generation, we manually create some conflict or contradiction in the real data $x$. In detail, we randomly shuffle the words given each real data sample and get inconsistent data $x’$.

% After generating the data $x’$ from different approaches, we evaluate the likelihood of real data distribution given real data $x$ and generated data $x’$, which is formulated as $D(x)=p(y|x)$ and $D(x’)=p(y|x’)$ respectively. The results are shown in Table [reference] and the generated data examples are presented in Appendix [reference].





%
%
%Optionally include extra information (complete proofs, additional experiments and plots) in the appendix.
%This section will often be part of the supplemental material.


\end{document}