\appendix
\section{Limits on Flavor Violating Couplings}
\label{sec:FlavourViolating}



For the sake of completeness a projection of the limits induced onto the flavor changing parameters of the Lagrangian in 
Eq.~(\ref{eq:General_L}), is presented here. The dimension five 
ALP-quark Lagrangian reads:
%%%
\beq
\mathcal{L}^a_\mathrm{FC}=\frac{\partial_\mu a}{2f_a} \bar{d}\gamma^\mu(C^{(d)}_V+C^{(d)}_A\gamma^5)d+ 
\frac{\partial_\mu a}{2f_a} \bar{u}\gamma^\mu(C^{(u)}_V+C^{(u)}_A\gamma^5)u.
\label{eq:General_flavor_violating}
\eeq
%%%
$V$-type couplings will induce parity conserving mesonic decays such as $\PS_I\to \PS_F\, a$ and $\VE_I\to \VE_F\, a$, while  
$A$-type one will enter in parity violating processes such as $\PS\to \VE\, a$. 

Two different scenarios may be considered: 1) flavor violation is induced by tree level parameters or 2) flavor violation is 
induced by effective couplings that emerge due to RG effects.  In the first scenario, the $M_I\to M_F \,a$ branching ratio, 
in the ALP massless limit, reads:
%%%
\beq
\mathrm{Br}(M_I\to M_F \,a) = \kappa\frac{M_I^3}{16\pi\Gamma_I}\frac{\left|\left( C^{(u,d)}_{A,V}\right)_{ij}\right|^2}
{4 f^2_a}F(m_a^2)^2\left(1-\frac{M^2_F}{M^2_I}\right)^3,
\eeq
%%%
where $(C^{(q)}_{A,V})_{ij}$ is the relevant coupling mediating the FC transition and $F(m_a^2)$ is the associated hadronic form factor. 
The parameter $\kappa$ is 1 (1/3) if $M_I$ is a pseudoscalar (vector) meson respectively. The bounds extracted from hadronic meson 
decays into an invisible ALP are collected in Tab.~\ref{tab:fv_constraints}, for the chosen value $f_a=1$ TeV. From $K^+ \to \pi^+ \,a$ 
decay one can test the $sd$-vector sector, $(C^{(d)}_V)_{sd}$. To bound the $sd$-axial sector, $(C^{(d)}_A)_{sd}$, one could use 
the $K^* \to \pi\,a$ decays. These decays, however, are not measured yet, and therefore such limits have to 
be expressed as function of a still unknown branching ratio (Br):
%
\beq
\Big|(C^{(d)}_A)_{sd}\Big|=6 \cdot 10^3\cdot\left[\mathrm{Br}(K^* \to \pi\,a)\right]^{1/2}\, \mathrm{GeV}^{-1}.\label{eq:a1}
\eeq 
%
The $B^+\to K^+ a$ and $B^+\to K^{+*} a$ channels provide exclusion limits on $(C^{(d)}_V)_{bs}$ and $(C^{(d)}_A)_{bs}$ respectively.
Finally $(C^{(d)}_V)_{bd}$ and $(C^{(d)}_A)_{bd}$ can be tested via $B\to \pi(\rho) a$ decays. Again, as the  $B\to \rho \cancel{E}$ 
branching ratio is not measured one can express the bound as 
%
\beq
\Big|(C^{(d)}_A)_{bd}\Big|=7.4 \cdot 10 ^{-4}\cdot \left[\mathrm{Br}(B\to \rho \,a)\right]^{1/2}\, \mathrm{GeV}^{-1}\,.\label{eq:a2}
\eeq 
%
The numerical difference between Eq.~(\ref{eq:a1}) and (\ref{eq:a2}) is due to the huge difference in 
 the mean life of the resonances.
In the up-quark sector only the $cu$ sector can be tested via 
$D^+\to \pi^+(\rho^+) a$, yet not measured. Therefore, the limit on $(C^{(u)}_A)_{cu}$ is once again expressed as: 
%
\beq
\Big|(C^{(u)}_A)_{cu}\Big|=4.5 \cdot 10 ^{-3}\cdot \left[\mathrm{Br}(D\to \rho \,a)\right]^{1/2}\, \mathrm{GeV}^{-1}.
\eeq 
%  
However, a limit on $(C^{(u)}_V)_{cu}$ can be extracted following Ref.~\cite{MartinCamalich:2020dfe} and using a recast of 
$D^+\to \tau^+ (\to \pi^+ \nu) \bar{\nu}$ giving $\mathrm{Br}(D^+\to\pi^+ a)<8\cdot 10^{-6}$. Processes involving top quark 
transitions are clearly not yet accessible.
%
\begin{table}[h!]
\centering
\begin{tabular}{| c | c | c | c |}\hline
	Vector 				&				 Limit 			& 		Axial				 & Limit  \\\hline
\hline
	$|(C^{(d)}_V)_{sd}|/f_a$		&	$2.5\cdot 10^{-12}$ GeV$^{-1}$ 	&	$|(C^{(d)}_A)_{sd}|/f_a$		&	n.a.						\\\hline
	$|(C^{(d)}_V)_{bs}|/f_a$		&	$9\cdot 10^{-9}$ GeV$^{-1}$ 		&	$|(C^{(d)}_A)_{bs}|/f_a$		&	$1.3\cdot 10^{-8}$ GeV$^{-1}$ \\\hline
	$|(C^{(d)}_V)_{bd}|/f_a$		&	$1\cdot 10^{-8}$ GeV$^{-1}$ 		&	$|(C^{(d)}_A)_{bd}|/f_a$		&	n.a. 						\\\hline
	$|(C^{(u)}_V)_{cu}|/f_a$		&	$2\cdot 10^{-8}$ GeV$^{-1}$ 		&	$|(C^{(u)}_A)_{cu}|/f_a$		&	n.a. 						\\\hline

\end{tabular}
\caption{Limits on flavor violating couplings in the scenario 1) for $m_a=0$ and $f_a=1$ TeV.}
\label{tab:fv_constraints}
\end{table}
%

In the second considered scenario one can set to 0 the off--diagonal parameters of Eq.~(\ref{eq:General_flavor_violating}) at the high 
scale $f_a$. This procedure does not get rid completely of flavor--violation in the ALP--sector, as it can be generated from the flavor 
conserving parameters via RG equations proportionally the SM flavor violation induced by the CMK matrix~\cite{Bauer:2020jbp}.
One can project limits on these effective couplings by equating the amplitude of a given process obtained from 
Eq.~(\ref{eq:General_flavor_violating}) to the amplitudes in Eq.~(\ref{loopAmplitudeK+}) and (\ref{loopAmplitudeVector}) for $V/A$ type 
couplings respectively. The off--diagonal entries of the matrices are to be considered not as tree level parameters, but as effective 
ones, induced by diagonal couplings. In principle there are four different matrices in the Lagrangian in 
Eq.~(\ref{eq:General_flavor_violating}), but this scenario does not distinguishes axial or vector off--diagonal elements. 
The effective flavor violating couplings read:
%
\beq
C^{(d/u)}_{ij}=\frac{G_F m_q^2}{2\sqrt{2}\pi^2}\sum_f c^{(f)}_{ij}=\frac{G_F m_q^2}{2\sqrt{2}\pi^2}\sum_f V_{fi}V_{fj}^*c_f \frac{x_f}{x_q}\ln\left(\frac{f_a^2}{m_f^2} \right)
\eeq 
%
where the sum over the $f$ runs on up-type quarks for $C^{(d)}$ and on down-type quarks for $C^{(u)},$ and $m_q$ is the mass of the heaviest 
quark running in the loop.  The limits shown in 
Fig.~\ref{final_summary}, recasted onto bounds on $C^{(d/u)}$, are reported in Tab.~\ref{tab:fv_constraints_2}, for $f_a=1$ TeV.
%
\begin{table}[h!]
\centering
\begin{tabular}{| c | c | c | c |}\hline
	$d$-type 				&				 Limit 			& 		$u$-type			 & Limit 						 \\\hline
\hline
	$|(C^{(d)})_{sd}|/f_a$		&	$3.8\cdot 10^{-12}$ GeV$^{-1}$ 	&	$|(C^{(u)})_{cu}|/f_a$		&	$3.2\cdot 10^{-12}$ GeV$^{-1}$  \\\hline
	$|(C^{(d)})_{bs}|/f_a$		&	$4.3\cdot 10^{-10}$ GeV$^{-1}$ 	&	$|(C^{(u)})_{tc}|/f_a$		&	$9.1\cdot 10^{-10}$ GeV$^{-1}$	  \\\hline
	$|(C^{(d)})_{bd}|/f_a$	&	$9.3\cdot 10^{-11}$ GeV$^{-1}$ 	&	$|(C^{(u)})_{tu}|/f_a$		&	$7.7\cdot 10^{-11}$ GeV$^{-1}$	   \\\hline
\end{tabular}
\caption{Limits on flavor violating couplings in the scenario 2) for $m_a=0$ and $f_a=1$ TeV.}
\label{tab:fv_constraints_2}
\end{table}
%
