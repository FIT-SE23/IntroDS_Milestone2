%%%%%%%%%%%%%%%%%%%%%%%%%%%%%%%%%%%%%%%%%%%%%%%%%%%%%%%%%%%%%%%%%%%%%%%%%%%%%%%%%%%%%%%%%%%%%%%%%%%%%%%%%%%%%%%%%%%%%%%%%%%%%%%%%%%%%%%%%%%%%%
%%                                                                                                                                          %%
%%                                                   SECT3                                                                            %%
%%                                                                                                                                          %%
%%%%%%%%%%%%%%%%%%%%%%%%%%%%%%%%%%%%%%%%%%%%%%%%%%%%%%%%%%%%%%%%%%%%%%%%%%%%%%%%%%%%%%%%%%%%%%%%%%%%%%%%%%%%%%%%%%%%%%%%%%%%%%%%%%%%%%%%%%%%%%

\section{Mesonic Decay amplitudes}
\label{Sec:mesonic}

In this section, the amplitudes for the hadronic and leptonic meson decays into an invisible ALP are derived. The calculation 
of the tree-level amplitudes is performed through the Brodsky--Lepage method, while the hadronization of penguin contributions will 
make use of form factors calculated via LQCD methods. 


\subsection{Factorization for $s$--channel processes}
\label{sec:char_tree_level}
%
\begin{figure}
\centering
\includegraphics[scale=0.32]{PICTURES/tree_level_c2}\hspace{.75 cm}\includegraphics[scale=0.32]{PICTURES/tree_level_c}
\caption{Tree-level s-channel of a charged $(\bar{q}Q)$--meson decaying into a charged $(\bar{Q}^\prime q^\prime)$ meson and an ALP. 
Diagrams where the ALP is emitted from the final state meson can be easily obtained.\label{fig:tree_s}}
\end{figure}
%
A typical s--channel hadronic meson decay process in ALP is shown in Fig.~\ref{fig:tree_s}. Looking at the picture the factorization 
naturally emerges: the amplitude is a product of two uncorrelated vector currents obtained by cutting the diagram along the weak boson 
leg connecting the hadronic external states. Let the initial mesonic state be constituted by the $\bar q Q$ quark pair and the final 
one by the $\bar Q^\prime q^\prime$ quarks. The resulting amplitude will be of the form
\beq
\bra{M_F}(\bar{Q}^\prime\Gamma^{(\mathrm{F})\mu} q^\prime )\ket{0}\bra{0}(\bar q \,\Gamma^{(\mathrm{I})}_\mu Q)\ket{M_I} \label{eq:decomposition},
\eeq
where $M_I$ and $M_F$ are the initial and final mesons and the index $I$ $(F)$ indicates that the ALP is emitted from the initial 
(final) meson quarks. The hadronic to vacuum matrix element is defined as
\beq
\begin{split}
\bra{0} & (\bar q \,\Gamma_\mu Q)\ket{M} = i f_M \int dx \Tr [\Psi_M (x) \Gamma_\mu]\,.
\end{split}
\label{eq:mesonic_hadronization_s}
\eeq
Note that Eqs.~(\ref{eq:wave_PS}), (\ref{eq:wave_V}) and (\ref{eq:mesonic_hadronization_s}) follow a slightly different 
notation with respect to the referred literature. In particular the functions $\phi(x)$ have been normalized to one, in such a way 
that in Eq.~(\ref{eq:mesonic_hadronization_s}) the mesonic decay constants can be explicitly factorized. Moreover, the color 
structure, being trivial in all our processes, has been already explicitly traced. If one of the two operator $\Gamma^{(\mathrm{I,F})}$ 
in Eq.~(\ref{eq:decomposition}) is either $\gamma^\mu\gamma^5$ or $\gamma^\mu$, the hadronization procedure given by 
Eq.~(\ref{eq:mesonic_hadronization_s}) reproduces the usual definitions
\bea
\bra{0}\bar{q}\,\gamma^\mu \,\gamma_5 \,Q \ket{\PS(P_\PS)} & = & i f_\PS P_\PS^\mu \,\label{eq:Mformfactors1} ,\\
\bra{0}\bar{q}\,\gamma^\mu \,Q \ket{\VE(P_\VE)} & = & f_\VE M_\VE\epsilon^\mu(P_\VE) \,,\label{eq:Mformfactors}
\eea
%
for pseudoscalar and vector mesons, respectively. Thanks to the decorrelation between final and initial state one can obtain 
leptonic and radiative decay contributions by simply replacing one of the hadronic currents in Eq.~(\ref{eq:decomposition}) with 
a leptonic one.

The full amplitude for an s-channel W-mediated hadronic tree level meson decay can be written as: 
%
\beq
\bra{M_F}(\bar{Q}^\prime\gamma^\mu P_L q^\prime )\ket{0}\bra{0}(\bar q \,\Gamma^{(\mathrm{I})}_\mu Q)\ket{M_I} + 
\bra{M_F}(\bar{Q}^\prime \Gamma^{(\mathrm{F})}_\mu q^\prime )\ket{0}\bra{0}(\bar q \gamma^\mu P_LQ)\ket{M_I}.
\label{eq:def_str}
\eeq
%
Note that the diagram where the ALP is emitted from the $W$ internal line automatically vanishes, being 
the $W^+W^-$--ALP coupling proportional to the fully antisymmetric 4D tensor. The initial and finale Dirac structures, 
$\Gamma_\mu^{(\mathrm{I,F})}$, can be extracted from the corresponding Feynman diagrams In Fig.~\ref{fig:tree_s} only the case of initial 
meson ALP emission is shown explicitly, with the final meson emission case obtainable straightforwardly. The initial and final 
hard-scattering amplitudes read respectively:
%
\beq
\begin{split}
\Gamma^{(\mathrm{I})}_\mu & =\frac{4 G_F}{\sqrt{2} f_a}V_{CKM} 
      \left(c_q m_q \frac{\gamma^5\slashed{k}\gamma_\mu P_L}{m_a^2-2 k\cdot P_q}- 
            c_Q m_Q \frac{\gamma_\mu P_L\slashed{k}\gamma^5}{m_a^2-2 k\cdot P_Q} \right) \\
\Gamma^{(\mathrm{F})}_\mu & =\frac{4 G_F}{\sqrt{2} f_a}V_{CKM} 
       \left(c_{Q^\prime} m_{Q^\prime}\frac{\gamma^5\slashed{k}\gamma_\mu P_L}{m_a^2+2 k\cdot P_{Q^\prime}}-
             c_{q^\prime} m_{q^\prime} \frac{\gamma_\mu P_L\slashed{k}\gamma^5}{m_a^2+2 k\cdot P_{q^\prime}} \right), 
\end{split}\label{eq:mesongammas}
\eeq
with $k^\mu$ the ALP 4--momentum. For example, the amplitude for the pseudoscalar-to-pseudoscalar decay, with the ALP emitted 
from the initial meson, arising from the second term in Eq.~(\ref{eq:def_str}), reads: 
\beq
\bra{\PS_F}(\bar{Q}^\prime\gamma^\mu P_L q^\prime )\ket{0}\bra{0} (\bar q \,\Gamma^{(\mathrm{I})}_\mu Q)\ket{\PS_I} =  
i\frac{f_{\PS_F}}{2}P_F^\mu\left(i f_{\PS_I} \int dx \Tr [\Psi_{\PS_I} (x) \Gamma^{(\mathrm{I})}_\mu]\right). 
\label{eq:example_hadronization}
\eeq
The case of vector mesons can be straightforwardly obtained by using Eqs.~(\ref{eq:wave_V}) 
and (\ref{eq:Mformfactors}) instead of Eqs.~(\ref{eq:wave_PS}) and (\ref{eq:Mformfactors1}) in the corresponding amplitudes. In the 
following subsections the explicit expression for the various hadronic and leptonic s-channels decays are presented.


\subsubsection{Mesonic decays}
\label{sec:meson_form}

Using Eqs.~(\ref{eq:mesonic_hadronization_s}), (\ref{eq:mesongammas}) and (\ref{eq:example_hadronization}) and defining
 $P_q=(1-x)P_I$, $P_Q=xP_I$, $P_{q^\prime}=(1-y)P_F$ and $P_{Q\prime}=y P_F$, the s--channel amplitudes 
for a pseudoscalar-to-pseudoscalar meson decay, exemplified by the $B\to K a$ decay, with the ALP radiated from the initial (I) 
or final (F) meson read:
%
\bea
\mathcal{A}^{(s)}_\mathrm{I} & = & \frac{G_F V_{CKM} f_I f_F(k\cdot P_F)}{\sqrt{2}f_a} M_I \nn \\
  & & \hspace{0.5cm} \int_0^1\! dx \,g_I(x) \left[\frac{c_qm_q\,  \theta(1-x-\delta^M_a)}{m_a^2-2k\cdot P_I(1-x)}-
      \frac{c_Qm_Q\,\theta(x-\delta^M_a)}{m_a^2-2k\cdot P_Ix}\right]\phi_I(x)
\label{eq:a_s_ISR} \\
  \nn \\ \nn \\
\mathcal{A}^{(s)}_\mathrm{F} & = & \frac{G_F V_{CKM} f_I f_F(k\cdot P_I)}{\sqrt{2}f_a} M_F \nn \\ 
  & & \hspace{0.5cm} \int_0^1\! dy\,g_F(y)\left[\frac{c_{Q'}m_{Q'}}{m_a^2+2k\cdot P_Fy}-
      \frac{c_{q'}m_{q'}}{m_a^2+2k\cdot P_F(1-y)}\right]\phi_F(y).
\label{eq:a_s_FSR}
\eea
%
In the integrals of Eq.~(\ref{eq:a_s_ISR}) an explicit cut-off, $\delta^M_a = m_a/(2M_I)$, has to be introduced for $m_a\neq 0$ to 
remove the unphysical singularities. A simplified analytical expression of the amplitudes of Eq.~(\ref{eq:a_s_ISR}--\ref{eq:a_s_FSR}) 
can be obtained by taking $m_a =0$ and considering the ``very heavy'' meson limit defined in Eq.~(\ref{veryheavy}):
%
\beq
\left|\mathcal{A}^{(s)}_\mathrm{I} \right|\!\approx\!\frac{G_F V_\mathrm{CKM}f_If_F}{2\sqrt{2}f_a}M_I^2g_H(c_q-c_Q)
\label{eq:happrox_1}
\eeq
\beq
\left|\mathcal{A}^{(s)}_\mathrm{F} \right|\!\approx\!\frac{G_F V_\mathrm{CKM}f_If_F}{2\sqrt{2}f_a}M_F^2g_H(c_{Q^\prime}-c_{q^\prime}),
\label{eq:happrox_2}
\eeq
%
where $g_H$ is assumed to be constant. This approximation clearly shows the $M^2_{I(F)}$ dependence of the ISR (FSR) amplitude. Therefore, 
in typical pseudoscalar-to-pseudoscalar meson decays the ALP is predominantly emitted form the initial meson. Moreover, as pointed out 
in~\cite{Guerrera:2021yss}, a parametric cancellation occurs in the universal ALP-fermion coupling scenario. This cancellation is still 
partially at work even when the full $\phi(x)$ is used and indicates a possible underestimation of the amplitudes when $c_{q (q')} = 
c_{Q (Q')}$ is chosen.

The s--channel amplitudes for a pseudoscalar-to-vector meson decay, exemplified by the $B\to K^* a$ decay, with the ALP 
radiated from the initial (I) and final (F) meson read:
%
\bea
\mathcal{B}^{(s)}_\mathrm{I} &= & i \frac{G_F V_{CKM} f_I f_F (k\cdot\epsilon(P_F))}{\sqrt{2}f_a} M_I M_F \nn \\ 
 & & \hspace{0.5cm} \int_0^1 \!dx \,g_I(x)\left[\frac{c_qm_q\, \theta(1-x-\delta^M_a)}{m_a^2-2k\cdot P_I(1-x)}-
  \frac{c_Qm_Q\,\theta(x-\delta^M_a)}{m_a^2-2k\cdot P_Ix}\right]\phi_I(x)
 \label{eq:b_s_ISR} \\
  \nn \\ \nn \\
\mathcal{B}^{(s)}_\mathrm{F} & = &i \frac{G_F V_{CKM} f_I f_F}{\sqrt{2}f_a} \epsilon^\alpha(P_F) P_F^\beta(k^\beta P_I^\alpha- 
  k^\alpha P_I^\beta)  \nn \\ 
 & & \hspace{0.5cm} \int_0^1 \!dy \left[\frac{c_{Q'}m_{Q'}}{m_a^2+2k\cdot P_Fy}\, - \, 
  \frac{c_{q'}m_{q'}}{m_a^2+2k\cdot P_F(1-y)}\right]\phi_F(y),
\label{eq:b_s_FSR}
\eea
%
where $\epsilon^\alpha$ is the polarization of the vector resonance. The results for vector-to-pseudoscalar meson decay read:
%
\bea
\mathcal{C}^{(s)}_\mathrm{I} 
&=& -i\frac{G_F V_{CKM} f_If_F}{\sqrt{2}f_a} \epsilon^\alpha(P_I)P_I^\beta (k^\alpha P_F^\beta- 
  k^\beta P_F^\alpha) \nn \\ 
& & \hspace{0.5cm} \int_0^1 \!dx \left[\frac{c_q m_q\, \theta(1-x-\delta^M_a)}{m_a^2-2k\cdot P_I(1-x)} - 
  \frac{c_Q m_Q\, \theta(x-\delta^M_a)}{m_a^2-2k\cdot P_Ix}\right]\phi_I(x). \label{eq:c_s_ISR} \\
  \nn \\ \nn \\
\mathcal{C}^{(s)}_\mathrm{F} 
&=& -i\frac{G_F V_{CKM}f_If_F(k\cdot\epsilon(P_F)) }{\sqrt{2}f_a} M_I M_F \nn \\ 
& & \hspace{0.5cm} \int_0^1 \! dy \, g_F(y) \left[\frac{c_{Q'}m_{Q'}}{m_a^2+2k\cdot P_Fy}-
    \frac{c_{q'}m_{q'}}{m_a^2+2k\cdot P_F(1-y)}\right]\,\phi_F(y).
 \label{eq:c_s_FSR}
\eea
%%%
%%
%
One can easily show that also in these cases, for $m_a=0$ and assuming the ``very heavy'' meson limit, simple expressions for the 
amplitudes of the $\mathcal{B}$ and $\mathcal{C}$--type decays can be recovered, exhibiting a meson $M^2_{I(F)}$ dependence and 
a parametric cancellation for a universal ALP-fermion coupling, similarly to the results of Eqs.~(\ref{eq:happrox_1}--\ref{eq:happrox_2}).

Finally the amplitudes for vector-to-vector meson decays read: 
%
%%
%%%
\bea
\mathcal{D}^{(s)}_\mathrm{I} &=& -i\frac{G_F V_{CKM} f_I f_F}{\sqrt{2}f_a} \epsilon^\alpha(P_I) \epsilon^\mu(P_F)  P_I^\beta M_F  \nn \\ 
& & \hspace{0.5cm} \int_0^1 \!dx \left[\frac{c_q m_q\,\theta(1-x-\delta^M_a)}{m_a^2-2k\cdot P_I(1-x)}(\varepsilon_{\alpha\beta\mu\rho}k^\rho 
    + i k^\beta g^{\alpha\mu} -i k^\alpha g^{\beta\mu}) \, +\, \right. \nn \\ 
& & \hspace{1.5cm} \,\,\,\,+ \left. \frac{c_Q m_Q\,\theta(x-\delta^M_a)}{m_a^2-2k\cdot P_Ix}(\varepsilon_{\alpha\beta\mu\rho}k^\rho + i k^\alpha g^{\beta\mu} 
    - i k^\beta g^{\alpha\mu} )\right]\phi_I(x) \label{eq:d_s_ISR} \\
\nn \\ \nn \\
\mathcal{D}^{(s)}_\mathrm{F} &=& -i\frac{G_F V_{CKM} f_I f_F}{\sqrt{2}f_a} \epsilon^\alpha(P_F) \epsilon^\mu(P_I) P_F^\beta M_I  \nn \\ 
& & \hspace{0.2cm} \int_0^1 \!dy \left[\frac{c_{Q'} m_{Q'}}{m_a^2+2k\cdot P_Fy}(\varepsilon_{\alpha\beta\mu\rho}k^\rho 
   + i k^\beta g^{\alpha\mu} -i k^\alpha g^{\beta\mu}) \, +\, \right. \nn \\
& &  \hspace{1.2cm} \,\,\,+ \left.\frac{c_{q'} m_{q'}}{m_a^2+2k\cdot P_F(1-y)}(\varepsilon_{\alpha\beta\mu\rho}k^\rho 
   + i k^\alpha g^{\beta\mu} -i k^\beta g^{\alpha\mu} )\right]\!\phi_F(y). \label{eq:d_s_FSR}
\eea
In the ``very heavy'' meson limit the vector-to-vector amplitudes read:
%
\bea
\left|\mathcal{D}^{(s)}_\mathrm{I}\right| &\approx & \frac{f_I f_F G_F M_I^2 V_\mathrm{CKM}}{4 f_a}
                        \sqrt{(c_Q-c_q)^2+4\frac{M_F^2}{M_I^2}(c_Q^2+c_q^2)}, \label{heavyDI} \\
\left|\mathcal{D}^{(s)}_\mathrm{F}\right| &\approx & \frac{f_I f_F G_F M_F^2 V_\mathrm{CKM}}{4 f_a}
                        \sqrt{(c_Q^\prime-c_q^\prime)^2+4\frac{M_I^2}{M_F^2}(c_Q^{\prime2}+c_q^{\prime2})}. \label{heavyDF}
\eea
%
The parametric cancellation in this case is not completely at work as the epsilon tensor in 
Eqs.~(\ref{eq:d_s_ISR}-\ref{eq:d_s_FSR}) introduces an extra term proportional to the sum of the couplings squared. Notice 
also that, in the universal ALP-fermion coupling scenario, both the ISR and FSR amplitudes get proportional to the product 
$M_I \, M_F$ and there is no clear suppression of the FSR ALP process with respect to the ISR one, contrary to what happens 
for all the processes described by the $\mathcal{A}, \mathcal{B}$ and $\mathcal{C}$ amplitudes.


\subsubsection{Leptonic decay}
\label{sec:semilept_form}

For completeness we report here, briefly, the semileptonic meson decays, $M \to \ell\, \nu_\ell a$, derived in details 
in~\cite{Gallo:2021ame}. In Fig.~\ref{fig:tree_charged_lept} the diagrams where the ALP is emitted from the meson are drawn. 
The diagrams where the ALP is emitted by the charged leptons follow straightforwardly. These amplitudes can be factorized as 
\beq
\bra{0}  (\bar q \,\Gamma_\mu^{(h)} Q)\ket{M}(\bar{\ell}\gamma^\mu P_L \nu) + 
\bra{0}  (\bar q \,\gamma^\mu P_L Q)\ket{M}(\bar{\ell}\,\Gamma_\mu^{(\ell)}\, \nu),
\eeq
with $\Gamma_\mu^{(h)}$ and $\Gamma_\mu^{(\ell)}$ the Dirac structures related to the hadronic and leptonic emission, respectively.
%
\begin{figure}[t]
\centering
\includegraphics[scale=0.17]{PICTURES/Lept1}\hspace{1.3 cm}\includegraphics[scale=0.17]{PICTURES/Lept2}
\caption{Tree level contributions to the $M \to \ell\, \nu_\ell \, a$ amplitude, with the ALP emitted from the $M$ 
meson. The diagram where the ALP is emitted from the charged lepton is straightforward.}
\label{fig:tree_charged_lept}
\end{figure}
%
Using the methods introduced in Eqs.~(\ref{eq:mesonic_hadronization_s}--\ref{eq:Mformfactors1}) one obtains 
%
\beq
\begin{split}
\mathcal{E}_h=\frac{4 i G_F V_\mathrm{CKM}}{\sqrt{2}}\frac{f_M}{f_a}\frac{M_M^2}{2k\cdot P_M}\, 
                    \Big[ c_Q \frac{m_Q}{M_M}  \Phi^{(Q)}_M(m_a^2)
                    %\\& \qquad\qquad\qquad\qquad\qquad\qquad\qquad\qquad
                      - c_q \frac{m_q}{M_M} \Phi^{(q)}_M (m_a^2) \Big]
                      \left(\bar{\ell} \, \slashed{k} \,P_L \, \nu_\ell \right).\label{eq:MMALP} 
\end{split}
\eeq
%
for the amplitudes associated to the hadronic ALP emission. The functions $\Phi^{(q,Q)}_M (m_a^2)$ contain the integrals over 
the quark momentum fraction and are given by:
\bea
\Phi^{(q)}_M (m_a^2) &=& \int^{1-\delta_M}_0 \frac{k \cdot P_M}{m_a^2-2\,(1-x)\,k\cdot P_M} \, \phi_M(x) \,g_M(x)\, dx \nn \\
\Phi^{(Q)}_M (m_a^2) &=& \int^1_{\delta_M} \frac{k \cdot P_M}{m_a^2-2\,x\,k\cdot P_M}     \, \phi_M(x) \,g_M(x)\, dx \,.
\label{eq:leptonic_integrals}
\eea
%The kinematical cutoff $\delta_M=m_a/(2 M_M)$ prevents the appearance of unphysical bare singularities. 
One can obtain a simple expression for the hadronic amplitude by employing the ``very heavy'' meson approximation discussed in 
Sec.~\ref{sec:meson_form}. For $m_a=0$ one has
%
\beq
\mathcal{E}_h\approx i G_F V_\mathrm{CKM}\frac{f_M}{\sqrt{2}f_a}\frac{M_M^2}{k\cdot P_M}(c_q-c_Q) \left(\bar{\ell} \, \slashed{k} \,P_L \, \nu_\ell \right).
\eeq
%
The leptonic decay amplitude for the lepton ALP--emission process can be easily obtained by using the definition of the meson 
form factors of Eq.~(\ref{eq:Mformfactors1}), giving
\bea
\mathcal{E}_\ell &=& - \frac{4\,i\, G_F}{\sqrt{2}} V_{qQ} \,\frac{f_M}{f_a}  \left[  c_\ell\,m_\ell\, 
\left(\bar{\ell} \, P_L \,\nu_\ell \right) - 
\frac{ c_\ell \, m^2_\ell}{m_a^2\,+\, 2\, k\cdot p_\ell} \left(\bar{\ell} \, \slashed{k}\, P_L \,\nu_\ell \right) \right] \,.
\label{eq:MLALP}
\eea
%
assuming vanishing neutrino masses.



\subsection{Factorization of $t$--channel processes}
\label{sec:neut_tree_level}

Following a similar approach to the one used in the previous subsection one can now study meson decays occurring through a 
$t$--channel $W$ exchange. In Fig.~\ref{fig:fig_t}, typical diagrams with the ALP emitted from the initial meson state are 
depicted. Once again, diagrams with the ALP emitted from the final state are obvious, while the diagram where the ALP is emitted 
from the $W$ internal line automatically vanishes. In this case the hadronic process can be written:
%
\beq
\bra{M_F}(\bar Q' \,\Gamma^{(Q)}_\mu \,Q)(\bar{q} \, \gamma^\mu P_L \,q' )\ket{M_I} + 
\bra{M_F}(\bar Q' \,\gamma^\mu P_L \, Q)(\bar{q}\, \Gamma^{(\bar q)}_\mu \,q')\ket{M_I}, \label{eq:t_decomposition}
\eeq
%
where $\Gamma^{(Q,\bar{q})}$ are the Feynman amplitudes corresponding to the ALP emission from a quark or an antiquark line. Note 
that for processes in the $t$--channel, there are no trivial mesonic currents representing either the initial or final meson, and 
then the full Brodsky--Lepage machinery is always required for calculating these amplitudes.
%%%
\begin{figure}
\centering
\includegraphics[scale=0.33]{PICTURES/tree_level_n}\hspace{.5 cm}\includegraphics[scale=0.33]{PICTURES/tree_level_n2}
\caption{Tree-level t-channel of a neutral $(\bar{q}Q)$--meson decaying into a neutral $(\bar{Q}^\prime q^\prime)$ meson and an ALP.
Diagrams where the ALP is emitted from the final state meson can be easily obtained. Similar diagrams can be depicted for the 
CP conjugate process.}\label{fig:fig_t}
\end{figure}
%%%
The tree--level hard scattering of the diagrams depicted in Fig.\ref{fig:fig_t}, gives: 
%
\bea
\Gamma^{(Q)}_\mu & = & \frac{4 G_F}{\sqrt{2} f_a} V_{CKM} \left(c_{Q'} m_{Q'} \frac{\gamma^5\slashed{k}\gamma_\mu P_L}{m_a^2+2 k\cdot P_{Q'}}
                  - c_Q m_Q \frac{\gamma_\mu P_L\slashed{k}\gamma^5}{m_a^2-2 k\cdot P_Q} \right)\label{eq:gamma_Q} \\
\Gamma^{(\bar q)}_\mu & = & \frac{4 G_F}{\sqrt{2} f_a}V_{CKM}\left(c_{q} m_{q} \frac{\gamma^5\slashed{k}\gamma_\mu P_L}{m_a^2-2 k\cdot P_{q}}
                  - c_{q'} m_{q'} \frac{\gamma_\mu P_L\slashed{k}\gamma^5}{m_a^2+2 k\cdot P_{q'}} \right). \label{eq:gamma_qbar}
\eea
%
Using the procedure described in Ref.~\cite{Szczepaniak:1990dt}, one obtains, for the $t$--channel factorization:
\bea
\bra{M_F}(\bar Q'\, \Gamma^{(Q)}_\mu \,Q)(\bar{q} \, \gamma^{\mu}P_L\, q' )\ket{M_I}\!\! &=&\!  - 
  \frac{f_{M_F}f_{M_I}}{\sqrt{2}} \! \int \! dx\, dy \,\Tr [ \Psi_{M_I}(x)\,\gamma^{\mu}P_L\,\Psi_{M_F}(y)\,\Gamma^{(Q)}_\mu]\,,\nn\\
\bra{M_F}(\bar Q'\, \gamma^{\mu}P_L\ \,Q)(\bar{q} \, \Gamma^{(\bar q)}_\mu\, q' )\ket{M_I}\!\! &=&\!  - 
  \frac{f_{M_F}f_{M_I}}{\sqrt{2}} \! \int \! dx\, dy \,\Tr [ \Psi_{M_I}(x)\,\Gamma^{(\bar{q})}_\mu\,\Psi_{M_F}(y)\,\gamma^{\mu}P_L]\,.\nn\\ \label{eq:mesonic_hadronization}
\eea 
%\bea
%\bra{M_F}(\bar Q'\, \Gamma^{(Q)}_\mu \,Q)(\bar{q} \, \Gamma^{(B) \mu}\, q' )\ket{M_I}\!\! &=&\!  - 
%  \frac{f_{M_F}f_{M_I}}{\sqrt{2}} \! \int \! dx\, dy \,\Tr [ \Psi_{M_I}(x)\,\Gamma^{(B)\mu}\,\Psi_{M_F}(y)\,\Gamma^{(A)}_\mu]\,.\nn\\
%& & \label{eq:mesonic_hadronization}
%\eea 
Substituting $\Gamma^{(Q,\bar{q})}_\mu$ according to Eqs.~(\ref{eq:gamma_Q}--\ref{eq:gamma_qbar}) one finds the hadronized amplitudes. 
It is then phenomenologically convenient to separate ISR and FSR amplitudes.
The pseudoscalar-to-pseudoscalar meson decay amplitudes read:
%
\bea 
\mathcal{A}^{(t)}_\mathrm{I} & = & \frac{G_F V_{CKM} f_I f_F (k\cdot P_F) }{2f_a} M_I \nn \\ 
 & & \hspace{0.5cm} \int_0^1 dx \, g_I(x)\left[\frac{c_Qm_Q\,\theta(x-\delta^M_a)}{m_a^2-2k\cdot P_Ix}-\frac{c_qm_q\,\theta(1-x-\delta^M_a)}{m_a^2-2k\cdot P_I(1-x)}\right]\phi_I(x)\, , 
 \label{eq:a_t_ISR} \\
   \nn \\ \nn \\
\mathcal{A}^{(t)}_\mathrm{F} & = & \frac{G_F V_{CKM} f_I f_F (k \cdot P_I) }{2f_a} M_F \nn \\ 
 & & \hspace{0.5cm} \int_0^1 dy \,g_F(y) \left[\frac{c_{q'}m_{q'}}{m_a^2+2k\cdot P_F(1-y)}-\frac{c_{Q'}m_{Q'}}{m_a^2+2k\cdot P_Fy}\right]\phi_F(y) \, , 
 \label{eq:a_t_FSR}
\eea
%
while the pseudoscalar-to-vector $t$--channel transitions read:
%
\bea
\mathcal{B}^{(t)}_\mathrm{I} &=& i\frac{G_F V_{CKM} f_I f_F(k\cdot\epsilon(P_F)) }{2f_a} M_I M_F \nn \\ 
& & \hspace{0.5cm} \int_0^1 dx  \, g_I(x) \left[\frac{c_{Q}m_{Q}\,\theta(x-\delta^M_a)}{m_a^2-2k\cdot P_I x}-\frac{c_{q}m_{q}\,\theta(1-x-\delta^M_a)}{m_a^2-2k\cdot P_I(1-x)}\right]\phi_F(x)\, , 
\label{eq:b_t_ISR} \\
  \nn \\ \nn \\
\mathcal{B}^{(t)}_\mathrm{F} &=& i\frac{G_F V_{CKM} f_I f_F}{2f_a} \epsilon^\alpha(P_F)P_F^\beta(k^\beta P_I^\alpha-k^\alpha P_I^\beta)  \nn \\ 
& & \hspace{0.5cm} \int_0^1 dy\Big[\frac{c_{Q'}m_{Q'}}{m_a^2+2k\cdot P_Fy} - \frac{c_{q'}m_{q'}}{m_a^2+2k\cdot P_F(1-y)}\Big] \phi_F(y)\, .
\label{eq:b_t_FSR}
\eea
%
Finally, the vector-to-pseudoscalar and vector-to-vector decays amplitudes are given by:
%
\bea
\mathcal{C}^{(t)}_\mathrm{I} &=& -i\frac{G_F V_{CKM} f_I f_F}{2f_a}  \epsilon^\alpha(P_I)P_I^\beta(k^\beta P_F^\alpha - k^\alpha P_F^\beta)   \nn \\ 
& & \hspace{0.5cm} \int_0^1 dx \Big[\frac{c_Q m_Q\,\theta(x-\delta^M_a)}{m_a^2-2k\cdot P_Ix} - \frac{c_q m_q\,\theta(1-x-\delta^M_a)}{m_a^2-2k\cdot P_I(1-x)}
\Big] \phi_I(x).\label{eq:c_t_ISR} \\
   \nn \\ \nn \\
\mathcal{C}^{(t)}_\mathrm{F} &=&- i\frac{G_FV_{CKM} f_I f_F (k\cdot \epsilon(P_F))}{2f_a} M_I M_F \nn \\ 
& & \hspace{0.5cm} \int_0^1 dy \, g_F(y) \Big[\frac{c_{Q'}m_{Q'}}{m_a^2+2k\cdot P_Fy}-\frac{c_{q'}m_{q'}}{m_a^2+2k\cdot P_F(1-y)}\Big]\phi_F(y)\, ,
\label{eq:c_t_FSR}
\eea
and by
%
\bea
\mathcal{D}^{(t)}_\mathrm{I} &=& -i\frac{G_F V_{CKM} f_I f_F}{2f_a} \epsilon^\alpha(P_I)\epsilon^\mu(P_F) P_I^\beta M_F  \nn  \\ 
& & \hspace{0.5cm} \int_0^1 dx \Big[\frac{c_Q m_Q\,\theta(x-\delta^M_a)}{m_a^2-2k\cdot P_Ix} 
(\varepsilon_{\alpha\beta\mu\rho} k^\rho - i k^\beta g^{\alpha\mu} + i k^\alpha g^{\beta\mu}) + \nn \\ 
& & \hspace{1.5cm}\,\,\,\, \frac{c_q m_q\,\theta(1-x-\delta^M_a)}{m_a^2-2k\cdot P_I(1-x)}
 (\varepsilon_{\alpha\beta\mu\rho} k^\rho - i k^\alpha g^{\beta\mu} +i k^\beta g^{\alpha\mu})\Big]\phi_I(x).\label{eq:d_t_ISR} \\ 
  \nn \\ \nn \\
\mathcal{D}^{(t)}_\mathrm{F} &=& -i\frac{G_F V_{CKM} f_I f_F}{2f_a}\epsilon^\alpha(P_I)  \epsilon^\mu(P_F)P_F^\beta M_I  \nn \\  
& & \hspace{0.5cm} \int_0^1 dy \Big[\frac{c_{Q'} m_{Q'}}{m_a^2+2k\cdot P_Fy}
 (\varepsilon_{\alpha\beta\mu\rho} k^\rho - i k^\beta g^{\alpha\mu} + i k^\mu g^{\alpha\beta}) + \nn \\ 
& & \hspace{1.5cm}\,\,\,\, \frac{c_{q'} m_{q'}}{m_a^2+2k\cdot P_F(1-y)}
 (\varepsilon_{\alpha\beta\mu\rho} k^\rho - i k^\mu g^{\alpha\beta} +i k^\beta g^{\alpha\mu})\Big] \phi_F(y).\label{eq:d_t_FSR}
\eea
%
It is interesting to note that the results derived here are very symmetric with the $s$-channel case, up to a sign and a 
factor of $\sqrt{2}$. The only exception is given by the vector-to-vector decays where a minus sign is present in parts of the
tensor structure w.r.t. the $s$-channel case.
One can apply the ``very heavy" approximation used in Sec.~\ref{sec:meson_form} to derive simple analytic results also for 
the $t$-channel amplitudes. Expressions similar to the ones shown in Eqs.~(\ref{eq:happrox_1}--\ref{eq:happrox_2}) and  
Eqs.~(\ref{heavyDI}--\ref{heavyDF}) can be obtained also in the $t$-channel case.






