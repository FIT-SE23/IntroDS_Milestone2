%%%%%%%%%%%%%%%%%%%%%%%%%%%%%%%%%%%%%%%%%%%%%%%%%%%%%%%%%%%%%%%%%%%%%%%%%%%%%%%%%%%%%%%%%%%%%%%%%%%%%%%%%%%%%%%%%%%%
%%                                                                                                                %%
%%                                                   SECT 3                                                  %%
%%                                                                                                                %%
%%%%%%%%%%%%%%%%%%%%%%%%%%%%%%%%%%%%%%%%%%%%%%%%%%%%%%%%%%%%%%%%%%%%%%%%%%%%%%%%%%%%%%%%%%%%%%%%%%%%%%%%%%%%%%%%%%%%
%


\subsection{Penguin Hadronization}
\label{subs:penguin}

Flavor changing neutral current meson decays in ALP will often receive the dominant contributions from the one--loop penguin 
diagrams \cite{Izaguirre:2016dfi,Bauer:2017ris,Gavela:2019wzg}, shown in Fig.~\ref{fig:figure_loop}. 
%
\begin{figure}[!th]\center
\includegraphics[scale=0.13]{PICTURES/Fig3a.jpg}\hspace{1 cm}\includegraphics[scale=0.13]{PICTURES/Fig3b.jpg}
\caption{Dominant one-loop penguin contributions to meson decay in ALP. \label{fig:figure_loop}}
\end{figure}
%
In this kind of processes, only one quark line participates actively to the ALP emission, with the other quark playing  
the role of spectator. 

The hadronic matrix element for pseudoscalar-to-pseudoscalar meson decays is mediated by a vector current customarily 
factorised as:
\bea
\bra{\PS_F}\bar{q}\gamma^\mu Q\ket{\PS_I}=f_+(k^2)(P_{I}+P_{F})^\mu+f_-(k^2)\,k^\mu
\label{eq:loop_formfactors}
\eea
with $k=P_{I}-P_{F}$. The $f_{+,-}(k^2)$ form factors can be obtained from LQCD calculations \cite{Carrasco:2016kpy,
Gubernari:2018wyi,Ball:1993tp,Ball:2004rg,Ball:2004ye,Wang:2008ci,Wang:2008xt,Issadykov:2015iba,Lubicz:2017syv,
FermilabLattice:2017wmk,Cooper:2020wnj}, and are transition specific. Using Eq.~(\ref{eq:loop_formfactors}), the penguin 
contribution to the $\mathcal{P}_I \to \mathcal{P}_F \,a$ decay amplitude, assuming flavor diagonal quarks-ALP couplings, reads:
\beq
\begin{split}
&{\mathcal G}_{\PS_I\to\PS_F} = \frac{G_F\, m^2_q}{2\sqrt{2}\pi^2}\frac{M^2_I}{f_a}\left(1-\frac{M^2_F}{M^2_I}\right)\!
\left[f_+(k^2) + \frac{k^2}{M^2_I-M^2_F} f_-(k^2)\right]\!\sum_{f}\! \!c^{(f)}_{ij}.
\label{loopAmplitudeK+}        
\end{split}              
\eeq
The coefficient $c^{(f)}_{ij}$ has been opportunely normalized to factorize out the heaviest quark mass running in the loop, 
here dubbed $m_q$, and is defined as:
\beq
c^{(f)}_{ij} = V_{fi}V_{fj}^* \left[3\,c_W \frac{g(x_f)}{x_q}-\frac{c_{f}\,x_f}{4\,x_q} \ln\left(\frac{f_a^2}{m_f^2}\right)\right] 
\qquad \mbox{with} \quad \left(x_f \equiv \frac{m_f^2}{m_W^2}\right)\,.
\label{ALPflavorViolfirst}
\eeq
The penguin contribution where the ALP emitted from the internal $W$ line is included here for completeness, even if in the 
following phenomenological analysis $c_W=0$ will be assumed\footnote{ The contribution due to weak boson-ALP coupling and the 
interplay between quark and weak boson ALP coupling has been considered for example in~\cite{Gavela:2019wzg}.}. 
One-loop diagrams, with the ALP emitted from the initial/final quarks are suppressed by at least an extra $m_{f}^2/m_W^2$ factor 
(being $m_f$ the mass of the initial/final quark) with respect to the penguin contributions, as they arise at third order in 
the external momenta expansion. In the case of the $K$ and $D$ meson they can be safely neglected even compared to the  
tree level contributions. For $B$ mesons, instead, these contributions are roughly of the same order of the tree-level ones. 
Therefore, the complete 1--loop renormalization should be performed to be able to extract information on the ALP-external fermion 
couplings. 

In the case of vector-to-pseudoscalar transitions the hadronic matrix element can be divided in terms of four independent form factors. 
See for example \cite{Ball:2004rg} for the general expression. For the specific type of decays considered here, the only surviving 
form factor is given by:
\beq
\bra{\PS}\bar{q}\gamma^\mu\gamma_5Q\ket{\VE} = 2\,i \,M_\VE A_0(k^2) \frac{\left(\epsilon(P_\VE)\cdot k \right) k^\mu}{k^2}.
\label{eq:v_to_ps_FF}
\eeq
For numerical evaluation the $A_0(k^2)$ results explicitly reported in~\cite{Ball:1993tp,Ball:2004rg,Wang:2008xt,Wang:2008ci,
Fu:2014pba,Bharucha:2015bzk,Issadykov:2015iba,Gubernari:2018wyi,McLean:2019sds,McLean:2019jll} has been used. For strange 
meson decays, when not available, the form factors have be chosen to be 1, assuming an exact $SU(3)$ flavor symmetry. Finally, 
following the conventions of \cite{Bharucha:2015bzk}, one obtains
\beq
\mathcal{G}_{\VE\to\PS} = i \frac{G_F m_q^2}{\sqrt{2}\pi^2f_a} M_\VE A_0 (k^2) 
    \left(\epsilon(P_\VE)\cdot k \right)\,\sum_{f} c^{(f)}_{ij} 
\label{loopAmplitudeVector}    
\eeq
with $m_q$ the mass of the heaviest quark running in the loop and the coefficient $c^{(f)}_{ij}$ defined previously in 
Eq.~(\ref{ALPflavorViolfirst}).
The case of vector-to-pseudoscalar can be obtained from Eq.~(\ref{loopAmplitudeVector}) by exchanging $M_\VE$ 
with $M_\PS$ and taking the conjugate of the amplitude.


