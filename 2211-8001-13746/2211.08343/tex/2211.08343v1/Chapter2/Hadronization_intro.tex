%%%%%%%%%%%%%%%%%%%%%%%%%%%%%%%%%%%%%%%%%%%%%%%%%%%%%%%%%%%%%%%%%%%%%%%%%%%%%%%%%%%%%%%%%%%%%%%%%%%%%%%%%%%%%%%%%%%%%%%%%%%%%%%%%%%%%%%%%%%%%%
%%                                                                                                                                          %%
%%                                                   SECT2                                                                            %%
%%                                                                                                                                          %%
%%%%%%%%%%%%%%%%%%%%%%%%%%%%%%%%%%%%%%%%%%%%%%%%%%%%%%%%%%%%%%%%%%%%%%%%%%%%%%%%%%%%%%%%%%%%%%%%%%%%%%%%%%%%%%%%%%%%%%%%%%%%%%%%%%%%%%%%%%%%%%

\section{Mesonic Hadronization}
\label{sec:hadronization}


Despite the fact that in their rest frame mesons and baryons are complex non-static objects, they interact with highly 
relativistic particles mostly through their valence quark content, justifying a simplified description of these non--perturbative 
states \cite{Lepage:1980fj,Brodsky:1981rp,Szczepaniak:1990dt,EFREMOV1980245,Sterman:1997sx}. In the case at hand, the relevant 
observation is that the final state particles, when recoiling back to back must be emitted in a highly relativistic state. Therefore, 
the strong quantum effects that bind the meson constituents appear highly time-dilated, and the partonic content looks frozen, 
to the light escaping particle. For relative speeds near the speed of light the two recoiling particles are in contact for 
a very short time, decreasing as $(1-v^2/c^2)^{1/2}$. The relevant interactions can then only happen on small time scales and 
distances, relatively to typical mesonic masses and sizes, where QCD is perturbative. As such, the short--distance and the 
long--distance dynamics will have practically no interference. This incoherence between soft and hard physics implies that each 
meson, during the entire interaction with a highly relativistic particle, can be approximated with its partonic structure  
allowing for a significant simplification. 
\begin{figure}[!t]
\center
\includegraphics[scale=0.23]{PICTURES/Pion_Picture}
\caption{The valence quark state of a pion interacting with an external current carrying momentum $\mathcal{Q}$. The valence 
couple must be localized in $1/\mathcal{Q}$ in the transverse direction while in the longitudinal one Lorentz contraction assures 
the partons to be close. The picture is taken from \cite{Sterman:1997sx}.}
\label{fig:pion_picture} 
\end{figure}
The long--time dynamic of the initial state is described by the valence quarks Distribution Amplitude (DA) function, $\phi_I(x)$, 
representing the probability of finding the valence quarks of the incoming meson with a certain meson momentum fraction $x$. On the 
other side, the short--distance interaction between the meson quarks and the ALP is described by the hard--scattering amplitude 
$\Gamma (x,y,\mathcal{Q},\mu)$. 
Eventually, at a later time, quarks reform the outgoing meson, again described via a DA function, $\phi_F(x)$. The total amplitude 
for the process at hand can then be written as a convolution of the three probabilities: 
%
\beq
\bra{M_F}\Gamma\ket{M_I}=\int dx \, dy\Big (\phi^*_F(y,\mu) \otimes \,\Gamma(x,y,\mathcal{Q},\mu)\,\otimes
                 \phi_I(x,\mu)\Big)\Big(1+\mathcal{O}(m_q^2/\mathcal{Q}^2)\Big)
\label{eq:amplitude_fact}
\eeq
%
where $m_q$ is the mass of the lightest quark of the meson and $\Gamma(x,y,\mathcal{Q},\mu)$ is to be expanded perturbatively. 
In Eq.~(\ref{eq:amplitude_fact}) $\mathcal{Q}$ is the exchanged momentum, $x,y$ indicate the fraction of momentum carried by 
the heaviest parton of the initial and final meson respectively and $\mu$ is the renormalization scale. A natural choice is 
$\mathcal{Q}=\mu$, making the perturbative calculation consistent as long as $\alpha_{s}(\mathcal{Q})$ is perturbative. The length 
associated to this momentum exchange, $b=1/\mathcal{Q}$, represents the localization of the valence quarks in the transverse plane, 
relative to the mesons motion as pictorically shown in Fig.~\ref{fig:pion_picture}. If partons are separated more than $b$ that 
particular state will not contribute to the amplitude. Three particle states, e.g. with an extra gluon, will be suppressed by extra 
$1/\mathcal{Q}$ factors, since the probability of finding more than the minimum number of particles bunched up in $1/\mathcal{Q}$ 
decreases as $\mathcal{Q}$ grows. For a simple hard gluon exchange between two fermionic currents the classical dimension 
of the hard scattering amplitude is (mass$)^{-2}$ and since the dependence has to come from external momenta the approximate 
form of the hard amplitude can be expressed as 
\bea
\frac{1}{(xy\,\mathcal{Q}^2)}+\frac{1}{((1-x)(1-y)\,\mathcal{Q}^2)}. \nn
\eea
The end--point values, $x,y\simeq 0,1$, can be problematic as they violate the localization assumption and generate unphysical 
singularities. Indeed, for these values the hard scattering function spreads out in transverse space and it will not be anymore 
concentrated around the $1/\mathcal{Q}$ region. The physical picture corresponds to the case of a fast parton--slow parton couple. 
The slow parton will have $E\approx \Lambda_{QCD}$ and its superposition with more complicated external states is not evidently 
suppressed and indicates a failure of the localization assumption. A very asymmetric, somewhat long--range configuration has to 
be superimposed with the soft external states to estimate the contribution at the end--points. It can be shown, however, that in 
the large momentum exchange limit \cite{Lepage:1980fj,PhysRevLett.24.181,Duncan:1979ny,Braun:1997kw} these contributions are 
typically suppressed by extra factors of $m_q/\mathcal{Q}$ and can be safely neglected as a first order approximation. 


%%%%%%%%%%%%%%%%%%%%%%%%%%%%%%%%%%%%%%%%%%%%%%%%%%%%%%%%%%%%%
%
\subsection{Distribution Amplitudes}
%
%%%%%%%%%%%%%%%%%%%%%%%%%%%%%%%%%%%%%%%%%%%%%%%%%%%%%%%%%%%%%


\label{sec:B_Ltech}

\begin{figure}[!t]
\includegraphics[scale=0.5]{PICTURES/D_DA}
\caption{The D meson distribution amplitude $\phi_D$ at $\mu=1$ GeV (from \cite{Wu:2013lga}). The dotted, the dashed, the dash-dot 
and the solid lines are for different values of the Gegenbauer momentum $B_D=0, 0.20, 0.40$ and $0.60$ respectively.}
\end{figure}

The simplest way to implement the factorization mechanism described in the previous section is via the theory of QCD exclusive 
processes, firstly developed by Brodsky and Lepage \cite{Lepage:1980fj,Brodsky:1981rp}. The calculation has two main ingredients, 
the momentum DA, $\phi_i$, introduced in Eq.~(\ref{eq:amplitude_fact}) and the hard scattering amplitude. Indeed, the localization 
assumption and the relativistic view discussed at the beginning of Sec.~\ref{sec:hadronization} can be used to build the 
probability distributions of the valence quarks in momentum space, thus recovering their DAs \cite{Radyushkin:1977gp,Huang:2013yya,Wu:2013lga,
Brodsky:1981rp,Lepage:1980fj,Chernyak:1981zz}. 

Let's consider for the moment the case of a light meson $M_L$ of total momentum $P_M$ and let's label with $(\mathbf{p_T},x P_M),$ 
the three--momenta of one of its partons. The 2--component vector $\mathbf{p_T}=(p_x,p_y)$ spans the plane transverse to the direction 
of motion of the meson, while $x$ is the fraction of the meson momentum carried by the considered parton. Imposing the bound--state 
valence quarks approximation (i.e. localization) amounts to taking the part of the probability distribution dependent on its 
transverse momentum as an harmonic oscillator solution, namely:
\beq
\psi_L(x,\mathbf{p}_T) = A_L G_L(x)\exp\left(-\frac{\mathbf{p}^2_T+m_q^2}{8\beta_L^2(1-x)x}\right),
\label{eq:pion_distribution}
\eeq
where $A_L$ is an normalization constant and $\beta_L$ is a mass scale regulating the spread in the transverse plane of $\psi_L$.   
The explicit form of the $\mathbf{p}_T$ dependent part follows from the assumption that in the transverse plane the valence 
quarks lay in a s-wave state. Such a claim is supported by the fact that one is assuming the two partons to be close, thus 
limiting the contributions from states with higher angular momentum. The $x$-dependent function $G_L(x)$ is described by a 
Gegenbauer polynomials expansion~\cite{functions:1953}:
%%
%%%
\beq
G_L(x)=\left(1+ B^{(2)}_L C^{3/2}_2(2x-1)+\dots \right).
\eeq
%%%
%%
%
The parameters $B^{(n)}_L$ regulate the longitudinal momentum distribution among the partons. By integrating over the 
transverse momentum $\mathbf{p}_T$ down to the scale $\mu$, i.e. up to distances $\sim 1/\mathcal{Q}$, one obtains the 
following expression for the DA function:
\beq
\begin{split}
\phi_L(x,\mu)\propto \beta_L A_L &\sqrt{x(1-x)}G_L(x,\mu) \\ \times & 
      \left(\mathrm{Erf}\left[\sqrt{\frac{m_q^2+\mu^2}{8\beta_L^2x(1-x)}}\right] - 
            \mathrm{Erf}\left[\sqrt{\frac{m_q^2}{8\beta_L^2x(1-x)}}\right]\right).
\end{split}
\eeq
where now the Gegenbauer polynomials get multiplied by scale dependent momenta $B^{(n)}_L(\mu)$, see for example 
Ref.~\cite{PETERLEPAGE1979359,Efremov:1979qk,Mikhailov:1991pt}.
The resulting DA, $\phi_{L}(x,\mu)$ describing the light meson's quark momenta distribution is typically approximated by the 
simmetric function \cite{Lepage:1980fj,Brodsky:1981rp,Szczepaniak:1990dt}:
\bea
 \phi\!_L\!(x) \propto  x(1-x) \,.
 \label{eq:WfunctionL}
\eea

Having worked through the computation for a light meson, it is instructive to consider also the case of a heavy one. Here, 
one has to substitute the argument of the exponential in Eq.(\ref{eq:pion_distribution}) accordingly:
\beq
\frac{\mathbf{p}^2_T+m_q^2}{8\beta_L^2(1-x)x}\to \frac{\mathbf{p}^2_T+m_q^2}{8\beta_H^2(1-x)}+\frac{\mathbf{p}^2_T+m_Q^2}{8\beta_H^2x},
\eeq
where $q$ and $Q$ are the light and heavy partons in the meson. The resulting DA, $\phi_{H}(x,\mu)$, describing the heavy meson's 
quark momenta distribution can be approximated by \cite{Szczepaniak:1990dt}:
\bea
\phi\!_H\!(x) \propto  \left[\frac{\xi^2}{1-x}+\frac{1}{x}-1\right]^{-2}\, ,
\label{eq:WfunctionH}
\eea
where $\xi$ is a small parameter of $O(m_q/m_Q)$, measuring the light/heavy parton asymmetry in the momentum distribution.
The DA functions in Eqs.~(\ref{eq:WfunctionL}) and (\ref{eq:WfunctionH}) are assumed as an ansatz for describing light and heavy 
meson \cite{EFREMOV1980245,Lepage:1980fj,MUELLER1981237} momentum distributions\footnote{A detailed discussion on how adapt this 
DA description to the Kaon sector can be found in \cite{Guerrera:2021yss}.}. To simplify analytical expressions, it is often useful 
to consider the ``very heavy'' meson limit \cite{Bhattacharya:2018msv} by defining the parton masses $m_q=\xi M_I$, $m_Q=(1-\xi)M_I$ 
and assuming the simplified expression 
\bea
\phi_{H}(x) \approx \delta (1-x -\xi) 
\label{veryheavy}
\eea
for the DA function. 


Finally one associates to mesons a spinorial representation through the Bethe-Salpeter wave function, $\Psi(x)$, that
carries the quantum numbers of the resonance~\cite{Lepage:1980fj,Szczepaniak:1990dt,Aditya:2012ay,Hazard:2016fnc}. Therefore, 
for pseudoscalar and vector meson one defines respectively:
%
\bea
\Psi_{\! \mathcal{P}} (x) &=& \frac{\phi(x)}{4}\gamma^5(\slashed{P}_\mathcal{P} + g_\mathcal{P}(x)\,M_\mathcal{P}), \label{eq:wave_PS} \\
\Psi_{\! \mathcal{V}}(x) &=& \frac{\phi(x)}{4}(\sigma^{\alpha\beta}P_{\mathcal{V}\beta}-i g_\mathcal{V}(x) \,M_\mathcal{V}\gamma^\alpha)\epsilon_\alpha(P_\mathcal{V}), \label{eq:wave_V}
\eea
%
The mass functions $g_{\mathcal{P,V}}(x)$ introduced in Eqs.~(\ref{eq:wave_PS}) and (\ref{eq:wave_V}), are typically assumed to be constant 
and respectively $g_H \approx 1$ and $g_L \ll 1$ for heavy or light mesons. 

With all the ingredients in hand, the hadronic S-matrix elements describing meson decays in ALP can be calculated.
