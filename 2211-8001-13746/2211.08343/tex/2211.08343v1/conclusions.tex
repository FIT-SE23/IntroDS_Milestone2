%%%%%%%%%%%%%%%%%%%%%%%%%%%%%%%%%%%%%%%%%%%%%%%%%%%%%%%%%%%%%%%%%%%%%%%%%%%%%%%%%%%%%%%%%%%%%%%%%%%%%%%%%%%%%%%%%%%%
%%                                                                                                                %%
%%                                                   CONCLUSIONS                                       %%
%%                                                                                                                %%
%%%%%%%%%%%%%%%%%%%%%%%%%%%%%%%%%%%%%%%%%%%%%%%%%%%%%%%%%%%%%%%%%%%%%%%%%%%%%%%%%%%%%%%%%%%%%%%%%%%%%%%%%%%%%%%%%%%%


\section{Conclusions} 


In this paper, bounds on non-universal flavor--diagonal ALP--fermion couplings are presented. These limits have been extracted 
from mesonic decays assuming an invisible ALP signature. Two large classes of processes are studied: i) hadronic meson decays 
$M_I\to M_F \,a$, with $M_I$ and $M_F$ pseudoscalar and/or vector mesons and ii) leptonic meson decays $M\to \ell\nu a$. 
Lattice QCD and Brodsky--Lepage method are used for calculating the hadronic matrix elements associated to local and bi--local 
operators respectively. In particular, a complete set of tree-level amplitudes intervening in mesonic ALP decays have been derived 
for the first time, allowing a general comparison between tree-level and penguin mediated processes. For example, hadronic $K$ 
and $B$ decays in ALP are both clearly top-penguin dominated. However, already from a quick analysis of Eq.~(\ref{eq:treeloopratio}) 
one can estimate the different level of parameter cross-contamination that one could expect in a general non universal flavor 
conserving framework: while can be sizable for the case of $K$ meson hadronic decays, as exemplified in Fig.~\ref{fig:sum_plota}, 
it is instead completely negligible for the $B$ sector as explained in the text and shown in Fig.~\ref{final_summary}. Conversely, 
hadronic $D$ meson decays are typically tree-level dominated but a large penguin contamination is however expected. 
Moreover, a complete analysis of the independent limits on the different ALP-charged lepton diagonal couplings 
has been presented. Despite being several order of magnitude less constrained that the quark counterpart, these bounds extend 
previous limits, obtained mainly from astrophysical data, to the KeV-GeV ALP mass range. Finally, for completeness in 
App.~\ref{sec:FlavourViolating} a recast of the bounds in term of general, non-diagonal, ALP-fermion couplings is presented.  

The work presented here can be easily extended to visible ALP decays. It is however phenomenologically much more complicated 
to use those data for performing model independent analysis, as each decay amplitude will depend on the combinations of 
products of two different ALP-fermion couplings. 
