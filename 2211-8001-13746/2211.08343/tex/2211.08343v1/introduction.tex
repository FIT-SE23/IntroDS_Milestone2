%%%%%%%%%%%%%%%%%%%%%%%%%%%%%%%%%%%%%%%%%%%%%%%%%%%%%%%%%%%%%%%%%%%%%%%%%%%%%%%%%%%%%%%%%%%%%%%%%%%%%%%%%%%%%%%%%%%%
%%                                                                                                                %%
%%                                                   INTRO                                                   %%
%%                                                                                                                %%
%%%%%%%%%%%%%%%%%%%%%%%%%%%%%%%%%%%%%%%%%%%%%%%%%%%%%%%%%%%%%%%%%%%%%%%%%%%%%%%%%%%%%%%%%%%%%%%%%%%%%%%%%%%%%%%%%%%%

\section{Introduction}


Light pseudoscalar particles are a common feature of many extensions of the Standard Model (SM) of particle physics. These can 
be naturally introduced in beyond SM (BSM) scenarios, following the QCD axion paradigm \cite{Peccei:1977hh,Peccei1977,Wilczek:1977pj,
Weinberg:1977ma}, as pseudo Nambu-Goldstone bosons (pGBs) of a global $U(1)_{PQ}$ symmetry, non--linearly realized, 
spontaneously broken at a scale $f_a\gg v,$ where $v$ is the Higgs VEV. The main difference between the QCD axion and an 
Axion Like Particle (ALP) lies in abandoning the requirement that the only explicit breaking of the $U(1)_{PQ}$ symmetry arises 
from non--perturbative QCD effects \cite{Weinberg:1977ma}, imposing the well known relation $m_a f_a\approx m_\pi f_\pi$.
Therefore, allowing the ALP mass $m_a$ and the PQ symmetry breaking scale $f_a$ to be independent parameters gives rise to an 
abundance of scenarios populated by scalar singlets under the SM group, not necessarily tied to the solution of the Strong 
CP problem \cite{PhysRevD.19.2227,CREWTHER1979123,Pospelov:2005pr,RevModPhys.82.557,PhysRevLett.97.131801}. 
Notable BSM theories that include light singlet scalar fields are: string theory models \cite{Cicoli:2013ana}, familons 
\cite{PhysRevLett.49.1549,Feng:1997tn}, flaxions \cite{Calibbi:2016hwq,Ema:2016ops} and relaxions \cite{Graham:2015cka}. 
ALPs, regardless the naturalness problem they are called to solve, represent compelling candidates for explaining DM abundance 
in our Universe \cite{Preskill:1982cy,Abbott:1982af,Dine:1982ah}. Whatever is the scenario considered, due to the pGB origin of 
the ALP, it may be plausible that the first hints of New Physics (NP) at the $f_a$ scale could be hidden in low--energy observables. 

The most general CP conserving effective Lagrangian, including operators up to dimension five \cite{Georgi:1986df}, and 
describing ALP interactions with SM fermions and gauge bosons, is given by:
\beq
\delta\mathcal{L}^a=-\frac{\partial_\mu a}{2f_a} \bar{f}\gamma^\mu(C_V+C_A\gamma^5)f - 
                    \frac{\partial_\mu a}{f_a}\sum_Xc_XX_{\mu\nu}^a\tilde{X}^{\mu\nu a},
\label{eq:General_L}
\eeq
where $C_V$ and $C_A$ are hermitian matrices in flavor space, $a$ is the ALP field, $f$ are the SM fermions and $X_{\mu\nu}^a$ 
indicates any SM gauge boson field strength, with $\tilde{X}^{\mu\nu a}\equiv X_{\alpha\beta}^a\epsilon^{\alpha\beta\mu\nu}/2$. 
Note that the flavor conserving diagonal vector couplings are identically zero up to a shift in the electroweak anomalous coupling.
Following most of the literature \cite{Aditya:2012ay,Gavela:2019wzg,Merlo:2019anv,MartinCamalich:2020dfe,Guerrera:2021yss,
Gallo:2021ame}, a low--energy CP and flavor conserving effective Lagrangian for ALP-fermion interactions can be introduced: 
\beq
\delta\mathcal{L}^{a}_{\mathrm{ferm}} = -\frac{\partial_\mu a}{2f_a}c_i\,\bar{f}_i\gamma^\mu\gamma^5 f_i = 
                          i \frac{a}{f_a}  m_{i} \,\bar{f}_i\gamma^5c_if_i.
\label{eq:lag_def_c}
\eeq
The index $i$ extends over all the fermions but the neutrinos, assumed to be massless, with $c_i$ real, but not universal, 
ALP-fermions couplings. With the Lagrangian of Eq.~(\ref{eq:lag_def_c}) all flavor-violating effects will be loop induced 
and CKM suppressed,\footnote{In more general frameworks flavor violating couplings can be introduced at tree level 
\cite{Feng:1997tn,Bauer:2020jbp,MartinCamalich:2020dfe}, and limits on these parameters can be simply recovered by removing 
the loop factors and the CKM suppression.} in the spirit of the Minimal Flavor Violation (MFV) ansatz \cite{DAmbrosio:2002vsn}. 

Most of the attention in the past has been devoted in constraining ALPs couplings with gauge bosons, mainly photons 
and gluons. In particular, very stringent bounds on $c_\gamma$ can be obtained from astrophysical searches: helioscopes 
\cite{PhysRevLett.51.1415,Zioutas:1998cc,Irastorza_2011,Irastorza:1567109}, haloscopes \cite{PhysRevLett.117.141801,
PhysRevLett.122.121802,PhysRevLett.118.091801,Semertzidis:2019gkj}, anomalies in stellar evolution \cite{Ayala:2014pea,
Straniero:316736} or helioseismology \cite{Vinyoles_2015}. ALP-fermions couplings can be studied in astro--particles/DM 
experiments like XENON \cite{XENON100:2014csq,PhysRevLett.118.261301} or CASPEr \cite{PhysRevX.4.021030} and ARIADNE 
\cite{PhysRevLett.113.161801} or using astrophysical data, like for example supernova $\gamma$--ray emission \cite{Payez_2015}. 
All these searches are, however, limited to very light ALP masses, rarely heavier than few hundreds of eV, and, moreover, they 
can only bound first generation ALP-fermion couplings. Therefore, terrestrial beam experiments are complementary in the 
exploration of the ALP couplings parameter space, and, among them, flavor factories are very likely the most promising ones. 

Flavor physics experiments have received more and more attention from the phenomenological community \cite{Izaguirre:2016dfi,
Gavela:2019wzg,MartinCamalich:2020dfe,Bauer:2021mvw,Guerrera:2021yss,Gallo:2021ame}. Strong limits on the ALP couplings in 
Eq.~(\ref{eq:General_L}) can be derived, for example, through the study of the $K\to\pi a$ decay. In large part of the literature 
a flavor universal ALP--fermion coupling, often dubbed $c_{a\Phi}$, is assumed. In this scenario, the $K\to\pi a$ amplitude is 
penguin dominated\footnote{It has to be recalled that this strong bound mainly arises from the top-penguin diagrams, due to 
the large top--mass enhancement.} and one can bounds $c_W$ and $c_{a\Phi}$ at the level of $10^{-3}$ for $f_a=1$ TeV 
\cite{Gavela:2019wzg}. %, barring a parametric cancellation for a given relative sign between the two couplings. 
However, several models have been introduced where large hierarchies between axion couplings \cite{Choi:2014rja,Kaplan:2015fuy,
Giudice:2016yja,Farina:2016tgd} are naturally produced. It is then of foremost phenomenological relevance to scan the ALP 
couplings parameter space following a less unbiased approach and to identify case by case which limits can be extracted from 
a given experiment on each independent ALP-fermion coupling. For example, in a non--universal ALP--fermion coupling scenario, 
the strongest limit from the $K \to \pi a$ decay is obviously obtained for the ALP-top coupling, $c_t \approx  c_{a\Phi} 
\lesssim 10^{-3}$ for $f_a=1$ TeV, due to the penguin top--mass enhancing. However, being the charm penguin contribution to the 
$K \to \pi a$ roughly $10$\% of the top one, an independent bound on the ALP-charm coupling, $c_c \lesssim 10^{-2}$, can be 
derived, assuming all the other ALP couplings vanishing. Moreover, as noticed by \cite{Guerrera:2021yss}, the $K\to\pi a$ decays 
can also be mediated by tree--level diagrams with a $W^{\pm}$--boson exchanged in the s--channel (t--channel) for charged (neutral) 
$K$ decays. These diagrams contribute at the 1\% level to the total decay amplitude and therefore one can extract independent 
limits on the ALP-lighter quark couplings, $c_{u,d,s} \lesssim 10^{-1}$ for $f_a=1$ TeV, for most of the kinematically allowed 
$m_a$ range. 

However, one of the main obstacles in calculating hadronic observables is to deal with the associated non--perturbative matrix 
element. In treating transitions mediated by local operators, like for example penguin contributions with heavy virtual particles 
in the loop, one can make use of the available Lattice QCD results \cite{Carrasco:2016kpy}. Conversely, to compute products of 
bi--local operators mediated by virtual light states, alternative methods, like for example the Brodsky--Lepage technique 
\cite{Lepage:1980fj,Brodsky:1981rp,Szczepaniak:1990dt}, have to be used \cite{Guerrera:2021yss}. Only when the calculation of 
all these different contributions is done explicitly, one can fairly compare the sensitivity reach on ALP-fermion couplings 
of the different mesonic decay channels.  

From the previous discussion one realizes immediately that, differently from what naively expected, present experiments can 
already provide a full set of constraints on the possible ALP--fermion flavor structures, therefore motivating a thorough study 
of mesonic ALPs rare decays. In this work two wide classes of mesonic decays in ALP are considered: $i)$ the mesonic decays 
$M_I\to M_F a$, with $M_I$ and $M_F$ being either a pseudoscalar or a vector meson and $ii)$ the leptonic meson decay 
processes, $M\to \ell\nu a$. 
In all these processes an ``invisible'' ALP is assumed, i.e. the ALP lifetime is sufficiently long for escaping the detector 
(typically $\tau_a \ge 100$ ps) or alternatively the ALP is mainly decaying in a, not better specified, invisible sector. 

The flagship process for ALP searches in flavor transitions is undoubtedly the $K\to \pi + \cancel{E}$ signature, studied 
at NA62 \cite{NA62:2020pwi,CortinaGil:2020fcx,CortinaGil:2021nts,CortinaGil:2021gga} and KOTO \cite{Ahn:2018mvc}, that arises 
from a $s\to d\, a$ transition at the quark level. Kaon physics is thus in the spotlight for probing ALP couplings in the 
KeV to hundreds of MeV ALP--mass region. $B$--factories have also a fundamental impact in limiting the ALP--fermion coupling 
parameter space. BaBar, Belle \cite{Masso:1995tw,Bevan:2014iga,Dolan:2017osp,Belle:2017oht,Kou:2018nap,CidVidal:2018blh,
deNiverville:2018hrc,Dattola:2021cmw} and LHCb \cite{Aaij:2015tna,Aaij:2016qsm} are carefully analyzing visible and invisible 
signatures of $b$--meson decays. For example, Belle experiment had conducted searches for $B\to M_F \bar\nu\nu$ for many 
different mesonic final states $M_F$, testing ALP masses up to a few GeV. Conversely, measurements of $D$ mesons decays with 
final state composed of a meson and missing energy are missing at the moment \cite{ParticleDataGroup:2020ssz}. Another class 
of interesting processes for probing ALP--fermion physics is represented by decays with a mono--$\gamma +\cancel{E}$ signature. 
Flavor conserving $\Upsilon$ resonant searches exploiting decays such as $\Upsilon(nS) \to \Upsilon(1S)\pi^+\pi^-$ can be 
used to directly probe $\Upsilon(1S) \to \gamma +\cancel{E}$ decays \cite{BaBar:2009gco}.

The paper is organized as follows: in Sec.~\ref{sec:hadronization} a general discussion on the hadronization techniques 
needed for calculating  pseudoscalar and vector meson decays in ALP is presented. 
In Sec.~\ref{Sec:mesonic} the mesonic decay amplitudes in ALPs needed for 
calculating hadronic and leptonic meson decays in ALP are derived in general. Many of these amplitudes have been 
calculated here for the first time. Useful phenomenological approximation are discussed for each channel.
Section~\ref{sec:invisible_bounds_ff} is devoted to describe the phenomenological impact 
of ALP emission in weak--induced meson decays, assuming an invisible ALP in the final state. All the relevant charged and neutral 
meson hadronic decays in ALP are estimated at tree and/or at one--loop (penguin) level. Derived bounds on ALP--fermion couplings 
from hadronic and leptonic meson decays are then discussed and a complete summary of the present situation is shown. 
Finally, for completeness, in Appendix \ref{sec:FlavourViolating} exclusion bounds on flavor changing ALP-fermion parameters 
are presented, for two different scenarios.


