%%%%%%%%%%%%%%%%%%%%%%%%%%%%%%%%%%%%%%%%%%%%%%%%%%%%%%%%%%%%%%%%%%%%%%%%%%%%%%%%%%%%%%%%%%%%%%%%%%%%%%%%%%%%%%%%%%%%
%%                                                                                                                %%
%%                                                   SECT 4                                                       %%
%%                                                                                                                %%
%%%%%%%%%%%%%%%%%%%%%%%%%%%%%%%%%%%%%%%%%%%%%%%%%%%%%%%%%%%%%%%%%%%%%%%%%%%%%%%%%%%%%%%%%%%%%%%%%%%%%%%%%%%%%%%%%%%%
\section{Phenomenology of Invisible ALP and Mesons}
\label{sec:invisible_bounds_ff}

From the meson decay amplitudes calculated in the previous section, stringent limits on the ALP-fermion Effective Lagrangian 
of Eq.~(\ref{eq:lag_def_c}) can now be derived. The two different classes of meson decays are going to be discussed separately. 
Hadronic meson decays in ALPs are expected in general, to be the most constraining processes to test ALP--quark couplings 
in the sub-GeV ALP mass range, mainly thanks to the high precision experimental results of the Kaon sector 
\cite{NA62:2020pwi,CortinaGil:2020fcx,CortinaGil:2021nts,CortinaGil:2021gga,Ahn:2018mvc}. Nonetheless, very promising results 
are expected from $B$--factories \cite{Masso:1995tw,Bevan:2014iga,Dolan:2017osp,Kou:2018nap,CidVidal:2018blh,deNiverville:2018hrc,
Belle:2017oht}, for ALP masses up to few GeV \cite{Dolan:2014ska,Izaguirre:2016dfi,Gavela:2019wzg,MartinCamalich:2020dfe}. 
Semileptonic decays, on the other hand are useful to test ALP--lepton couplings producing new bounds in the KeV--GeV range 
for the all the ALP-charged lepton couplings, whereas most of the limits on ALP-quark couplings are not competitive with the 
ones obtained from the hadronic decay channels.


\subsection{Hadronic final states}
\label{sec:true_pheno}
In Sec.~\ref{Sec:mesonic} the tree-level and one-loop (penguin) contributions to the hadronic meson decays into an invisible 
ALP, $M_I\to M_F \, a$, have been derived. In Tab.~\ref{tab:constraints} the amplitudes of several charged meson decays are 
collected. For definiteness, $m_a=0$, $f_a=1$ TeV and $c_f=\pm 1$ have been used. As noticed in Sec.~\ref{sec:char_tree_level} and 
\ref{sec:neut_tree_level}, accidental cancellation can occurs in the tree--level amplitudes, depending on the relative sign between 
$c_{Q^{(\prime )}}$ and $c_{q^{(\prime )}}$ for all the processes but vector-to-vector decays, see Eqs.~(\ref{eq:a_s_ISR}-\ref{eq:c_s_FSR}) 
and Eqs.~(\ref{eq:a_t_ISR}-\ref{eq:c_t_FSR}). To make evident the impact of this accidental cancellation, the tree-level results in 
Tab.~\ref{tab:constraints} has been shown with a $(min-max)$ interval, obtained by setting $c_{Q^{(\prime )}}/c_{q^{(\prime )}}= (+1,-1)$ 
respectively. The origin of this parametric cancellation has been proved analytically by \cite{Guerrera:2021yss} both in the ``very 
light'' and ``very heavy'' meson limit. In Sec.~\ref{sec:char_tree_level} the results for the ``very heavy'' meson limits have 
been explicitly shown for the $\mathcal{A}$-type decays. From the results of Tab.~\ref{tab:constraints}, one learns which is the 
effectiveness of this parametric cancellation, once the numerical integration is performed using the non-approximated heavy 
DA function of Eq.~(\ref{eq:WfunctionH}). Depending on the specific decay channel, the tree-level decay rate 
can change from one to two orders of magnitude. Notice that in Tab.~\ref{tab:constraints} no vector-to-vector decay is 
presented, being still experimentally marginal. 
%%%
\begin{table}[t!]
\centering
\begin{tabular}{|c|c|c|}\hline
Channel & Tree--Level & Penguin \\\hline
\hline
%$B_c^\pm \to B^\pm a$	           &	$ 2\times10^{-10}$ 	&	 $ 3\times 10^{-8}$				\\\hline
$B_c^\pm \to D_s^\pm a$		 &	$ (6-160)\times 10^{-11}$ 		& 	$ 2 \times10^{-6}$  			\\\hline
$B_c^\pm \to D^\pm a$	        	 &       $ (1-30)\times 10^{-11} $		 &	$ 3 \times10^{-7}$				\\\hline
$B_c^\pm \to K^{*\pm} a$         	 &        $ (2-70)\times 10^{-11} $		 &	n.a.							\\\hline
$B_c^\pm \to \rho^\pm a$           	 &         $ (4-100)\times 10^{-11} $ 		&	n.a.							\\\hline
$B_c^\pm \to K^\pm a$               	 &        $ (8-230)\times 10^{-12}$		&	n.a.							 \\\hline
$B_c^\pm \to \pi^\pm a$			 &	 $(3-85)\times10^{-11}$		&	n.a.							\\\hline\hline
$B^\pm \to D_s^\pm a$			 &	$ (5-30)\times 10^{-12}$ 		& 	n.a.							 	\\\hline
$B^\pm \to D^\pm a$	          	 &    $ (1-7)\times 10^{-12}$		 	&	n.a.					\\\hline
$B^\pm \to K^{*\pm} a$	          	 &        $ (1-7) \times 10^{-12}$		&	 $4\times 10^{-6}$		\\\hline
$B^\pm \to \rho^\pm a$	          	 &       $ (3-20)\times 10^{-12} $ 		&	  $4\times 10^{-7}$			\\\hline
$B^\pm \to K^\pm a$              	&       $(8-50)\times 10^{-13} $		&	 $2\times 10^{-6}$		 \\\hline
$B^\pm \to \pi^\pm a$			&	 $(3-20)\times10^{-12}$			&	 $3 \times 10^{-7}$			 	\\\hline\hline
%$D_s^\pm \to D^\pm a$		&	$1 \times 10^{-8} $				&	 $3\times10^{-9}$ 				\\\hline
$D_s^\pm \to K^{*\pm} a$		&        $ (1-60) \times 10^{-11}$ 		&	  $ 6 \times 10^{-12}$ 		\\\hline
$D_s^\pm \to \rho^\pm a$		&        $ (3-170) \times 10^{-11}$		&	n.a.							\\\hline
$D_s^\pm \to K^\pm a$	        	&        $ (6-300) \times 10^{-12} $ 		&	 $7\times 10^{-12}$ 			\\\hline
$D_s^\pm \to \pi^\pm a$            	&        $ (2-120) \times 10^{-11} $		&	n.a.						 \\\hline\hline
$D^\pm \to K^{*\pm} a$	 		&        $ (2-100) \times 10^{-12}$ 		&	n.a.							\\\hline
$D^\pm \to \rho^\pm a$	 		&        $ (7-290) \times 10^{-12}$ 		&	$3\times 10^{-12}$			\\\hline
$D^\pm \to K^\pm a$	    	        &        $ (1-50) \times 10^{-12} $ 		&	n.a.						\\\hline
$D^\pm \to \pi^\pm a$           		&        $ (5-200) \times 10^{-12}$  		&	$6\times 10^{-12}$		 \\\hline\hline
%$K^{*\pm} \to \rho^\pm a$       	&        $ (2-2) \times 10^{-8} $	 		&	$4\times 10^{-7}$		 \\\hline
$K^{*\pm} \to K^\pm a$            	&        $ (5-25)\times 10^{-13} $	 	&	$4\times 10^{-8}$		 \\\hline
$K^{*\pm} \to \pi^\pm a$          	&        $ (3-20) \times 10^{-12}$ 		&	$3\times 10^{-9}$	 \\\hline\hline
$\rho^{\pm} \to K^\pm a$            &        $ (8-25)\times 10^{-12} $	 	&	$2\times 10^{-9}$	 \\\hline
$\rho^{\pm} \to \pi^\pm a$         	&        $ (3-9) \times 10^{-11}$ 		&	$4\times 10^{-10}$	 \\\hline\hline
$K^\pm \to \pi^\pm a$               	&        $ (2- 10) \times 10^{-12} $ 		&	$5\times 10^{-10}$	 \\\hline
\end{tabular}
\caption{Tree-level and penguin contribution to the hadronic charged meson decay rates, calculated for $m_a=0$, $f_a=1$ TeV and 
$c_f=\pm 1$, expressed in GeV${}^{-1}$ units. The interval in the tree--level column is obtained by setting $c_Q/c_q =(1,-1)$.}
\label{tab:constraints}
\end{table}

It is also useful to note from Eqs.~(\ref{eq:happrox_1}) and (\ref{eq:happrox_2}) that the ratio between the tree-level 
ISR and FSR amplitudes is always independent of the particular nature of the decay ($s$ or $t$) and, with the only exception 
of the vector to vector meson decay, is given by:
\beq
R^T_{I/F} = \left| \frac{\mathcal{M}^{(s,t)}_{I}}{\mathcal{M}^{(s,t)}_{F}} \right| \simeq \left(\frac{M_I}{M_F}\right)^2,
\eeq
with the identity strictly holding in the ALP massless limit and assuming a ``very light'' or a ``very heavy'' DA function. 
Therefore, for all the corresponding processes, the decay amplitude is always dominated by the ISR ALP emission. 
%
%
%%
\begin{table}[t!]
\centering
\begin{tabular}{|c|c|c|}\hline
Channel & Tree--Level & Penguins \\\hline
\hline
%$B_s^0 \to B^0 a$	          		&	n.a.					&	 $ 1\times 10^{-6}$				\\\hline
%$B_s^0 \to\overline{B}^0 a$     	&	n.a.					& 	$ 1 \times10^{-9}$  				\\\hline
$B_s^0 \to D_s^0 a$	     		& 	n.a.						&	$ 4 \times10^{-7}$					\\\hline
$B_s^0 \to D^0 a$	     			& $  (7-70)\times10^{-12}$		&	n.a.							\\\hline
$B_s^0 \to K^{*0} a$               	&         n.a.					&	 $4 \times10^{-6} $	 	\\\hline
$B_s^0 \to\rho^0 a$	   		&   $ (4-50)\times 10^{-13} $	& 	n.a.						\\\hline
$B_s^0 \to K_L^0 a$                	&   	n.a.						&	$ 3 \times 10^{-7}$  			 \\\hline\hline
$B^0 \to  K^{*0}   a$			&   	n.a.						&	 $ 4 \times10^{-6} $	 			\\\hline
$B^0\to D^0 a$				&	$ (3-30)\times 10^{-11}$ 		&   	n.a.								\\\hline
$B^0 \to  \rho^0 a$	           		&       $ (2-20)\times 10^{-12}$ 		& 	$6 \times 10^{-7} $		\\\hline
$B^0\to K_L^{0} a$	         	  	&      n.a.						&	 $4\times 10^{-6}$			\\\hline
$B^0\to \pi^0 a$		           	&       $ (1-10)\times 10^{-12} $ 	&	  $ 5 \times 10^{-7}$		\\\hline\hline
$D^0\to K^{*0} a$		           	&        $ (7-300) \times 10^{-12}$	& 	n.a.							\\\hline
$D^0\to \rho^0 a$		           	&        $ (5-200) \times 10^{-12}$	&	 $4\times 10^{-12}$			 \\\hline
$D^0\to K_L^0 a$		           	&        $ (7-270) \times 10^{-13}$	&	 n.a.					 	\\\hline
$D^0\to \pi^0 a$		           	&        $ (2-100) \times 10^{-12}$	&	 $3\times 10^{-12}$			\\\hline\hline
%$K^{*0}\to \rho^0 a$		   	&        $ (5-6) \times 10^{-8}$		&	$5\times 10^{-7}$			\\\hline
$K^{*0}\to K^0 a$		           	&        $ (2-6) \times 10^{-12}$		&	$3\times 10^{-9}$			\\\hline
$K^{*0}\to \pi^0 a$				&        $  (1-2) \times 10^{-11}$		&	 $ 3\times 10^{-9}$				\\\hline\hline
$\rho^{0}\to K^0 a$				&        $ (1-3) \times 10^{-11}$ 		&	 $ 2\times 10^{-9}$				\\\hline
$\rho^{0}\to \pi^0 a$				&        $ (2-7) \times 10^{-11}$		&	 $ 3\times 10^{-9}$			\\\hline\hline
$K_L^0\to \pi^0 a$			        &        $ (4-20) \times 10^{-15}$		&	 $1\times 10^{-10}$		\\\hline
\end{tabular}
\caption{Tree-level and penguin contribution to the hadronic neutral meson decay rates, calculated for $m_a=0$, $f_a=1$ TeV and 
$c_f=\pm 1$, expressed in GeV${}^{-1}$ units. The interval in the tree--level column is obtained by setting $c_Q/c_q =(1,-1)$.}
\label{tab:constraints2}
\end{table}
%
%
%
Finally, in Tab.~\ref{tab:constraints}, both the tree-level and penguin contributions, when available, are presented. As a rule 
of thumb the tree-level vs one-loop amplitudes ratio can be estimated by: 
\bea
R_{T/L} = \left| \frac{\mathcal{M}_\mathrm{T}^{(s,t)}}{\mathcal{\mathcal{M}_\mathrm{L}}} 
\right| \approx 2\,\pi^2 \frac{f_I \, f_F}{m_f^2} 
\left|\frac{V^\mathrm{CKM}_\mathrm{T}}{V^\mathrm{CKM}_\mathrm{L}} \right|,
\label{eq:treeloopratio}
\eea
where $m_f$ is the mass of the heaviest quark running in the penguin and $f_{I,F}$ the initial and final meson decay constants. Notice that 
for most of the $D$ decays the tree-level contribution is comparable if not larger than the loop one, as clearly the $m^2_b$ 
penguin loop enhancement is not sufficient to compensate for the typical loop suppression factor. Conversely, for the $K$ and 
$B$ meson sector the tree/loop ratio looks really tiny thanks to the large $m_t^2$ penguin enhancement. Nevertheless, for the $K$ 
sector the tree-level diagrams may have a non negligible impact, as they depend on different, and often less constrained, 
ALP-fermion couplings, as discussed in \cite{Guerrera:2021yss}.


In Tab.~\ref{tab:constraints2} the decay rates of several neutral meson decays are collected. For definiteness, again $m_a=0$, 
$f_a=1$ TeV and $c_f=\pm 1$ have been used. The same formula discussed for the charged meson decays can be straightforwardly 
obtained also for the neutral meson case. In particular, penguin amplitudes typically dominate over tree-level ones, when 
available. The only exception being again represented by the $D$ meson sector, where tree-level amplitudes are at least one 
order of magnitude larger than penguin ones. 

The Kaon sector is the sector from which the most precise bounds on ALP-fermion couplings are obtained \cite{Izaguirre:2016dfi,
Gavela:2019wzg,Guerrera:2021yss}, thanks to very precise decay rate measurements. In particular NA62 at CERN, looking at 
$K^+\to \pi^+ \,a$, has collected $3\times 10^{16}$ p.o.t. in Run 1 and is aiming for $10^{18}$ p.o.t by the end of Run 2. 
Using the complete Run 1 dataset, the NA62 experiment established an upper limit on the $\mathcal{B}(K^+\to \pi^+\,a)$ for an 
invisible ALP at the level of $10^{-11}$ in the mass ranges of 0--110 MeV and 155--260 MeV \cite{CortinaGil:2021nts,
CortinaGil:2020fcx}. NA62 experiment has also established upper limits on $\mathcal{B}(K^+ \to\pi^+\,a) \lesssim 10^{-9}$ in 
the 110--115 MeV mass range, i.e. around of the $\pi_0$ mass, from a dedicated analysis based on the 10\% of the Run 1 minimum 
bias dataset \cite{NA62:2020pwi}. Measurements of the $K^0_L\to \pi^0 \bar \nu \nu$ decay naturally provide limits on the 
$\mathcal{B}\left( K^0_L\to \pi^0 a\right)$ branching ratio. KOTO experiment \cite{Ahn:2018mvc} has reported a limit on 
$\mathcal{B}(K^0_L \to \pi^0 \,a) \lesssim 2.4\times 10^{-9}$ at 90\% CL with the 2015 dataset, practically independent on the ALP 
mass up to the kinematical limit. 

%%%
\begin{figure}[ht!]
\centering \hspace{-1.2cm}
\includegraphics[scale=0.8]{PICTURES/MesonToMeson_a.pdf}
\caption{Limits on ALP-quark couplings from hadronic $K$ meson decays, as function of the ALP mass $m_a$ for $f_a=1$ TeV. 
The meaning of the different lines/regions is explained in the text.}
\label{fig:sum_plota}
\end{figure}
%%%


In Fig.~\ref{fig:sum_plota} a summary of the constraints on the quark couplings from $K$ hadronic decays into an invisible ALP 
are collected as function of the mass $m_a$ and for the chosen reference value $f_a=1$ TeV. The pink (middle) and red (lower) 
shaded areas show the bounds directly derived in Ref.~\cite{Guerrera:2021yss} from the $K^+\to\pi^+\,a$ decay rate. The pink area 
represents the limits projected onto the ALP-valence s- and u-quarks couplings, assuming all the other ALP-fermion coupling 
vanishing. Given that the tree-level amplitude is largely dominated by the ISR amplitude, it depends mainly on $c_s$ and $c_u$. 
The highlighted region has a meaning similar to the range displayed in Tab.~\ref{tab:constraints} and~\ref{tab:constraints2}, 
i.e. the boundaries of the region correspond to two the limiting cases $c_u=\pm c_s$. The single parameter limit on $c_{s(u)}$, 
obtained by setting $c_{u(s)}=0$, lies approximately in the middle of the allowed range. The red (lower) shaded area, instead, 
represents the bound on $c_t$ obtained by NA62 data assuming a non vanishing $c_t$ coupling and letting $c_s$ vary in the range 
$[-0.05,0.05]$. Therefore, despite the fact that the penguin contribution to the $K^+\to\pi^+\, a$ decay is two order of magnitude 
larger than the tree-level one, a contamination of the $c_t$ coupling of roughly one order of magnitude is still possible\footnote{See 
Ref.~\cite{Guerrera:2021yss} for a detailed discussion on the ALP-fermion couplings bounds from $K\to\pi\, a$ decays.}. The single 
parameter limit on $c_{t}$, obtained by setting all other $c_i=0$, lies approximately in the middle of the allowed range. The 
continuous middle black line shows the single parameter limit on $c_c$, coupling associated to the sub-dominant term weighting 
roughly 10\% of the total penguin contribution. The upper black dashed line represents the exclusion limit on the $c_t$ parameter, 
obtained from the KOTO $K^0_L\to\pi^0\nu\bar{\nu}$ measurements, while the lower black dashed one represents the exclusion limit on 
$c_t$ obtained using NA62 $K^+\to\pi^+\nu\bar{\nu}$ branching ratio for inferring a limit on the $K^0_L$ branching ratio through the 
Grossman-Nir bound. As the top penguin loop largely dominate the $K^0_L$ decay rate, see Tab.~\ref{tab:constraints2}, no appreciable 
contamination from the tree-level diagrams is expected. Finally the (upper) violet shaded area represents the limits projected 
onto the ALP-valence s- and d-quarks couplings from the KOTO $K^0_L$ measurement, assuming all the other ALP-fermion coupling 
vanishing and corresponding to the two limiting cases $c_d=\pm c_s$. Therefore, in Fig.~\ref{fig:sum_plota} an exhaustive set 
of bounds on ALP-quark couplings obtained from hadronic $K$ decays is collected. The single parameter limits obtained from 
hadronic $K$ decays are also reported in Fig.~\ref{fig:sum_plotb} as continuous colored lines, for comparison.

%%%
\begin{figure}[h!]
\centering \hspace{-1.2cm}
\includegraphics[scale=0.8]{PICTURES/MesonToMeson_b.pdf}
\caption{Limits on ALP-quark couplings from hadronic meson decays, as function of the ALP mass $m_a$ for $f_a=1$ TeV. The meaning 
of the different lines is explained in the text.}
\label{fig:sum_plotb}
\end{figure}
%%%


Hadronic $B$ meson decays in ALP are expected to provide additional interesting bounds on ALP-fermion couplings. Still not 
competing with the ones extracted from $K$ decays, these bounds nevertheless suffer from a smaller theoretical uncertainty 
and extend the $m_a$ range of constraints up to a few GeV. The first thing one can notice from Tab.~\ref{tab:constraints} and 
Tab.~\ref{tab:constraints2} is that, for the $B$ meson sector, the tree-level vs loop ratio is two/three order smaller compared 
to the $K$ meson ratio, mainly due to the larger CKM suppression factor. Therefore, one expects practically no contamination on the 
$c_t$ bounds from ALP-fermion couplings entering in the tree-level amplitudes, assuming perturbativity. The lower dashed black line 
in Fig.~\ref{fig:sum_plotb} is the $c_t$ exclusion bounds obtained from the recent $B\to K a$ Belle II data \cite{Dattola:2021cmw}. 
Limits on $B$ valence quark couplings are extremely weak due to the smallness of the tree-level amplitude and lie in the 
$(10^4-10^5)$ region and are not shown in figure\footnote{Tree level contributions of $B$ meson valence quark 
 may have possible interferences with initial/final state emission, as commented in Sec.~\ref{subs:penguin}.}.
 From Tab.~\ref{tab:constraints} one can identify another very promising channel, namely $B_c^+\to D_s^+\,a$. 
A future experimental limit on $\mathrm{Br}(B_c^+\to D_s^+ a)<10^{-5}$ would provide competitive bounds on $c_t$. Other 
theoretically interesting channels are hadronic $B_c$ decays into vector meson, like $B_c^\pm\to \rho (K^*)\, a$. These decays 
have no penguin contribution and a small $M_{\rho,K^*}/M_{B_c}$ mass ratio. Therefore a very clean extraction on ALP-$B_c$ 
valence quark coupling could be addressed. 

A complementary set of information can be potentially extracted from hadronic $D$ decays in ALP. Indeed, $D$--mesons penguins do 
not dominate anymore over tree-level amplitudes being $m_b/m_t$ suppressed compared to similar $B$ and $K$ meson decays, as shown 
numerically by Tab.~\ref{tab:constraints} and Tab.~\ref{tab:constraints2}. At present there are no experimental results measuring 
hadronic $D$ decays in ALPs, the only information originate from a recast of the charged $D\to \tau(\to \pi \nu)\bar{\nu}$ decay 
onto the $D^\pm\to \pi^\pm \, a$ branching ratio, as proposed by Ref.~\cite{MartinCamalich:2020dfe}. Even though the predicted 
signals are quite weak these channels can provide sensitivity on the $D^\pm$ valence quark couplings, $c_d$ and $c_c$ and 
eventually on $c_b$ through the dominant down-type penguin loop. The individual limits on these ALP-quark couplings, shown 
in Fig.~\ref{fig:sum_plotb} as dot-dashed lines, appear evidently above the perturbativity region once $f_a=1$ TeV is chosen. 

The single parameter limits obtained from hadronic $K$ meson decays are represented as continuous colored lines. Finally, no 
experimental data are at present available for $K^*$ and $\rho$ hadronic decays in ALP. The expected pattern is assumed to be, 
however, very similar to the $K$s one. 


