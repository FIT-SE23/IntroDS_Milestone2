%%%%%%%%%%%%%%%%%%%%%%%%%%%%%%%%%%%%%%%%%%%%%%%%%%%%%%%%%%%%%%%%%%%%%%%%%%%%%%%%%%%%%%%%%%%%%%%%%%%%%%%%%%%%%%%%%%%%
%%                                                                                                       %%
%%                                                   SECT 5                                        %%
%%                                                                                                        %%
%%%%%%%%%%%%%%%%%%%%%%%%%%%%%%%%%%%%%%%%%%%%%%%%%%%%%%%%%%%%%%%%%%%%%%%%%%%%%%%%%%%%%%%%%%%%%%%%%%%%%%%%%%%%%%%%%%%%

\subsection{Leptonic Final States}
\label{sec:Leptonic_Final_States}

Pseudoscalar leptonic decay can be used to constraint flavor–diagonal ALP-fermion couplings through the tree-level 
amplitudes of Eqs.~(\ref{eq:MMALP}) and (\ref{eq:MLALP}). In the case of invisible ALP, considered throughout this paper, the 
simplest approach is to saturate the 1-$\sigma$ experimental limits on the corresponding SM leptonic branching ratio adding the 
pseudoscalar meson three-body leptonic ALP decay to the two-body leptonic SM one, having the same missing energy signature. At 
present, in fact, there is not enough available experimental information on the charged lepton energy distribution such that 
one can obtain stricter bounds by characterizing two-body vs three-body decay spectrum\footnote{See \cite{Gallo:2021ame} 
for more details on three-body spectral analysis of meson leptonic decays in ALP.}.

Leptonic $B$ decays have been measured at Babar and Belle. The latest Belle data for electron, muon and tau channel can be found in 
\cite{Belle:2006tbq,Belle:2019iji,Belle:2012egh}, respectively. Charmed meson decays have been measured at BESS (see 
\cite{Eisenstein:2008aa,BESIII:2013iro,BESIII:2018hhz} for $D$ and~\cite{BESIII:2016cws,BESIII:2019vhn} for $D_s$ decays respectively) 
and at Belle \cite{Belle:2013isi}. Leptonic Kaon decays have been measured by KLOE and NA62~\cite{NA62:2012lny,KLOE:2007wlh,
ParticleDataGroup:2020ssz}. 

%%%%
\begin{table}[t]
\centering
\begin{tabular}{|c||c||c||c|}\hline
Channel							& $c_i$ u-type  			& 		$c_i$ d-type 	&	$c_i$ leptons		\\ \hline\hline
$B^\pm\!\! \to\! e^\pm \bar{\nu}_e$		&	360				&	250				&	2$\cdot 10^{6}$		\\\hline
$B^\pm\!\! \to\! \mu^\pm \bar{\nu}_\mu$	&	170				&	120				&	5$\cdot 10^{3}$			\\\hline
$B^\pm\! \!\to \!\tau^\pm \bar{\nu}_\tau$	&	2.6$\cdot 10^{3}$	&	2$\cdot 10^{3}$		&	5$\cdot 10^{3}$		\\\hline\hline
$D^\pm\!\! \to \!e^\pm \bar{\nu}_e$		&	170				&	175				&	5$\cdot 10^{4}$	 	 \\\hline
$D^\pm\! \!\to\! \mu^\pm \bar{\nu}_\mu$	&	250				&	270				&	3.5$\cdot 10^{3}$	\\\hline
$D^\pm\! \!\to \!\tau^\pm \bar{\nu}_\tau$	&	1.5$\cdot 10^{5}$	&	1.5$\cdot 10^{5}$	&	1.7$\cdot 10^{5}$	\\\hline\hline
$D_s^\pm\!\! \to\! e^\pm \bar{\nu}_e$		&	125				&	120				&	5$\cdot 10^{5}$		 \\\hline
$D_s^\pm\!\! \to\! \mu^\pm \bar{\nu}_\mu$	&	180				&	175				&	3.5$\cdot 10^{3}$	\\\hline
$D_s^\pm \!\! \to \!\tau^\pm \bar{\nu}_\tau$&	5$\cdot 10^{4}$		&	1$\cdot 10^{5}$		&	5$\cdot 10^{4}$	 	 \\\hline\hline
$K^\pm \!\! \to\! e^\pm \bar{\nu}_e$		&	4				&	6				&	4$\cdot 10^{3}$		\\\hline
$K^\pm \!\! \to \!\mu^\pm \bar{\nu}_\mu$	&  	600	 		 	&	800				&	3$\cdot 10^{3}$	\\\hline
\end{tabular}
\caption{Limits on single ALP-fermion coupling $c_i$ derived from pseudoscalar meson leptonic decays for $m_a=0$ and $f_a=1$ TeV.}
\label{tabfa}
\end{table}
%%%%
The derived bounds on single ALP-fermion couplings, $c_i$, for $m_a=0$ and $f_a=1$ TeV are shown in Tab.~\ref{tabfa}. As an example, 
the first row in Tab.~\ref{tabfa} should be read as follows: the ``up–quark'' column represents the limit on $c_u$ by setting $c_b = 
c_e = 0$, the ``down–quark'' column represents the limit on $c_b$ by setting $c_u = c_e = 0$, and finally the value in the ``lepton'' 
column is the limit on $c_e$ for $c_u = c_b = 0$. It is true in general that hadronic meson decays in ALP can provide by far the most 
stringent limits on a universal ALP-fermion coupling, bounding $c_{a\Phi} \lesssim 5 \times 10^{-4}$ from $K$ decays or $c_{a\Phi} 
\lesssim 8 \times 10^{-3}$ from $B$ decays, for $f_a=1$ TeV. However, both these limits come associated to the top enhanced penguin, 
and therefore in a non-universal ALP-fermion coupling scenario can be applied to bound solely $c_t$. On the contrary, leptonic and 
hadronic decays in ALP often provide similar constrains on single ALP-light-quark couplings, as can be seen comparing the results 
of Tab.~\ref{tabfa} with the ones summarized in Fig.~\ref{fig:sum_plotb}. 
While, for example, single parameter bounds on $c_{u,s}$ derived from charged $K$ hadronic decay are still one order of magnitude 
better than the ones obtained from charged $K$ leptonic channel, conversely $c_c$ bounds obtained from charged $D$ meson leptonic 
decay are of the same order, or even slightly better than the corresponding charged $D$ meson hadronic ones. Finally, $c_b$ bounds 
from charged $B$ meson leptonic decays are at least two order of magnitude better than the hadronic limits. Therefore, meson 
leptonic decays provide useful complementary information once independent bounds on all the ALP-quark couplings are needed. 

From Tab.~\ref{tabfa}, evidently emerges that limits on ALP-lepton coupling are very weak, due to the combination of low masses 
(i.e. the electron case) and/or not very precise experimental results (i.e. the $\tau$ channels). However, despite these 
feeble bounds, pseudoscalar meson leptonic decays in ALP provide undoubtedly the best available limits on the ALP–lepton sector 
for an $m_a$ in the KeV-GeV range. In Fig.~\ref{fig:semileptonic_summ}, for exemplification, all the limits on the muon 
coupling, $c_\mu$, are collected as function of the ALP mass $m_a$, for the chosen values $f_a=1$ TeV. From the plot one can 
easily discern a slower saturation of the kinematical limit compared to the corresponding hadronic decays. Already at values 
$m_a \approx M_M/2$ a strong reduction of the decay rate appears. One can easily understand from the corresponding Dalitz plot 
that this effect is associated to the tree-body nature of the leptonic meson decay in ALP. Therefore some additional caution 
should be used in this case to generalize the validity of $m_a=0$ results to higher mass values.


%%%
\begin{figure}[t!]
\center
\includegraphics[scale=0.4]{PICTURES/Semileptonic}
\caption{Summary of the limits extracted from the semileptonic class, $M\to\ell \nu a$, on the muonic coupling $c_\mu$ with $f_a$ fixed at 1 TeV.\label{fig:semileptonic_summ}}
\end{figure}
