%
\documentclass[preprint,floats,aps,12pt,superscriptaddress,nofootinbib,floatfix]{revtex4}
%\documentclass[preprint,floats,aps,12pt,superscriptaddress]{jfm}
%\documentclass[twocolumn,floats,aps,superscriptaddress]{revtex4}
%
\usepackage{natbib} 
\usepackage{times} 
\usepackage{amssymb,amsmath}
\usepackage{bbold}
\usepackage{csquotes}
%
%%%%%%%%%%%%%%%%%%%%%%%%%%%%%%%%%%%%%%%%%%%%%%%%%%%%%%%%%%%%%%%%%%%%
% Define bottom line of manuscript - to keep track of version and dates - 
\usepackage{prelim2e}\usepackage[none,bottom]{draftcopy}
\draftcopyName{preprint / }{1.2} %ADAPT TEXT
\renewcommand{\PrelimWords}{Preprint%
 -- contact: u.thiele@uni-muenster.de -- www.uwethiele.de} %ADAPT TEXT 
%%%%%%%%%%%%%%%%%%%%%%%%%%%%%%%%%%%%%%%%%%%%%%%%%%%%%%%%%%%%%%%%%%%%
% Page geometry
%
%\textwidth 16cm
%\textheight 23cm
%\topmargin -1cm
%\oddsidemargin 0cm
%\parindent 1.5cm
%\pagestyle{empty}
%\renewcommand{\baselinestretch}{1.0}
%
%%%%%%%%%%%%%%%%%%%%%%%%%%%%%%%%%%%%%%%%%%%%%%%%%%%%%%%%%%%%%%%%%%%%
% Graphics output / general latex pdflatex switch
\ifx\pdfoutput\undefined
% we are running LaTeX, not pdflatex
\usepackage[usenames,dvips]{color}
\usepackage{graphicx}     % Include figure files
%\bibliographystyle{$HOME/Home/Bibliography/Bibstyles/revtex} %$ %ADAPT DIRECTORY no natbib
\bibliographystyle{plainnat} % with natbib
\else
% we are running pdflatex, so convert .eps files to .pdf
\usepackage[usenames,dvipsnames]{color}
\usepackage{epstopdf}
\usepackage[pdftex]{graphicx}
\usepackage[pdftex]{hyperref}
\hypersetup{a4paper,
    pdftitle={Manuscript on XXX} %ADAPT TEXT 
    pdfauthor={et al. and Uwe Thiele},  %ADAPT TEXT 
%pdfsubject={Programmdokumentation},
pdfproducer={lateX},
pdfview=FitV,       % FitH
pdfstartview=FitB,
linkcolor=blue,     % links to same page
citecolor=blue,     % citations
urlcolor=red,      % links to URLs
breaklinks=true,    % links may be split onto 2 lines
colorlinks=true,
citebordercolor=0 0 0,  % color for \cite
filebordercolor=0 0 0,
linkbordercolor=0 0 0,
menubordercolor=0 0 0,
urlbordercolor=0 0 0,
pdfhighlight=/I,
pdfborder=0 0 0,   % no box around links
bookmarksopen=true,
bookmarksnumbered=true
}
%\bibliographystyle{$HOME/Home/Bibliography/Bibstyles/unsrturl}  %$
                                %%ADAPT DIRECTORY no natbib
\bibliographystyle{plainurl} % with natbib
%\bibliographystyle{plainnat} % with natbib
\fi
%
\ifx\pdfoutput\undefined
\DeclareGraphicsExtensions{.eps}
\else
\DeclareGraphicsExtensions{.jpg, .pdf, .tif, .png}
\fi
%
\graphicspath{{.},{./}} %$ %ADAPT DIRECTORY
%
%%%%%%%%%%%%%%%%%%%%%%%%%%%%%%%%%%%%%%%%%%%%%%%%%%%%%%%%%%%%%%%%%%%%
% TEXT HIGHLIGHTING FOR MULTI-AUTHOR EDITING
% you might need to adapt the color definitions
%%%%%%%%%%%%%%%%%%%%%%%%%%%%%%%%%%%%%%%%%%%%%%%%%%%%%%%%%%%%%%%%%%%%
\usepackage[usenames,dvipsnames]{color}
\usepackage[normalem]{ulem}
%\renewcommand{\outuwe}[1]{}
\newcommand{\outuwe}[1]{\textcolor{BrickRed}{\sout{#1}}}
% defining comment types: \bf for changes in text; \tt for comments;
\newcommand{\ttuwe}[1]{\texttt{\textcolor{BrickRed}{\textbf{uwe:} #1}}}
\newcommand{\bfuwe}[1]{\textbf{\textcolor{BrickRed}{#1}}}
\newcommand{\bflen}[1]{\textbf{\textcolor{OliveGreen}{#1}}}
%
\newcommand{\ttlen}[1]{\texttt{\textcolor{OliveGreen}{\textbf{xxx:} #1}}}
%\renewcommand{\outcoaut}[1]{}
\newcommand{\outcoaut}[1]{\textcolor{darkgreen}{\sout{#1}}}
\newcommand{\tobias}[1]{\texttt{\textcolor{Blue}{\textbf{tobias:} #1}}}
\newcommand{\bftobias}[1]{\textbf{\textcolor{Blue}{#1}}}
\newcommand{\tto}[1]{\texttt{\textcolor{Green}{\textit{ #1}}}}
%%%%%%%%%%%%%%%%%%%%%%%%%%%%%%%%%%%%%%%%%%%%%%%%%%%%%%%%%%%%%%%%%%%%
%
%
%%%%%%%%%%%%%%%%%%%%%%%%%%%%%%%%%%%%%%%%%%%%%%%%%%%%%%%%%%%%%%%%%%%%
% Switch on/off notes  / label printing
%%%%%%%%%%%%%%%%%%%%%%%%%%%%%%%%%%%%%%%%%%%%%%%%%%%%%%%%%%%%%%%%%%%%
\newcommand{\no}[1]{{\tt #1}\\}
%\newcommand{\no}[1]{}
%
% SWITCH LABEL PRINTING ON
%\newcommand{\mylab}[1]{\label{#1}{\color{blue}\it \hspace*{.5cm} #1}}
% SWITCH LABEL PRINTING OFF
%\newcommand{\mylab}[1]{\label{#1}}
% USE \mylab INSTEAD OF \label
%%%%%%%%%%%%%%%%%%%%%%%%%%%%%%%%%%%%%%%%%%%%%%%%%%%%%%%%%%%%%%%%%%%%
%
\renewcommand{\vec}[1]{\mathbf{#1}}
\newcommand{\vecg}[1]{\boldsymbol{#1}}
\newcommand{\tens}[1]{\mathbf{\underline{#1}}}
\newcommand{\tensg}[1]{\boldsymbol{\underline{#1}}}
\def\D{{\mathrm d}}
\def\E{{\mathrm e}}
\def\I{{\mathrm i}}
\newcommand{\tf}[1]{\textstyle\frac #1}
\DeclareMathOperator{\tr}{tr}
\let\Re\relax
\DeclareMathOperator{\Re}{Re}
\let\Im\relax
\DeclareMathOperator{\Im}{Im}

%
\begin{document}
%
%----------------------------------------------------------------%
\title{Nonreciprocity induces resonances in two-field Cahn-Hilliard model}
%
\author{Tobias Frohoff-H\"ulsmann}
\email{t\_froh01@uni-muenster.de}
%\homepage{}
\thanks{ORCID ID: 0000-0002-5589-9397 }
\affiliation{Institut f\"ur Theoretische Physik, Westf\"alische Wilhelms-Universit\"at M\"unster, Wilhelm-Klemm-Str.\ 9, 48149 M\"unster, Germany}
%
\author{Uwe Thiele}
\email{u.thiele@uni-muenster.de}
\homepage{http://www.uwethiele.de}
\thanks{ORCID ID: 0000-0001-7989-9271}
\affiliation{Institut f\"ur Theoretische Physik, Westf\"alische Wilhelms-Universit\"at M\"unster, Wilhelm-Klemm-Str.\ 9, 48149 M\"unster, Germany}
\affiliation{Center for Nonlinear Science (CeNoS), Westf{\"a}lische Wilhelms-Universit\"at M\"unster, Corrensstr.\ 2, 48149 M\"unster, Germany}
%\affiliation{Center for Multiscale Theory and Computation (CMTC), Westf{\"a}lische Wilhelms-Universit\"at, Corrensstr.\ 40, 48149 M\"unster, Germany}
%
\author{Len M. Pismen}
%\email{}
%\homepage{}
%\thanks{ORCID ID:}
\affiliation{Department of Chemical Engineering, Technion - Israel Institute of Technology, Haifa 32000, Israel}
%
\begin{abstract} We consider a non-reciprocically coupled two-field Cahn-Hilliard system that has been shown to allow for oscillatory behaviour, a suppression of coarsening as well as the existence of localised states. Here, after introducing the model we first briefly review the linear stability of homogeneous states and show that all instability thresholds are identical to the ones for a corresponding Turing system (i.e., a two-species reaction-diffusion system). Next, we discuss possible interactions of linear modes and analyse the specific case of a ``Hopf-Turing'' resonance by discussing corresponding amplitude equations in a weakly nonlinear approach.  The thereby obtained states are finally compared with fully nonlinear simulations for a specific conserved amended FitzHugh-Nagumo system. We conclude by a discussion of the limitations of the weakly nonlinear approach.  \end{abstract}
%
%\begin{keyword} 
%Sliding drops \sep Heterogeneous substrates \sep Pinning and depinning
%\pacs{
%68.15.+e, % Thin films: Liquid thin films
%47.20.Ky  % Fluid dynamics: Nonlinearity (including bifurcation theory)
%47.55.Dz  % Drops and bubbles 
%68.08.-p  % Liquid-solid interfaces
%}
%\end{keyword} 
%
\maketitle
%
%\received{6.5.2002}
%
%----------------------------------------------------------------%
%
%%%%%%%%%%%%%%%%%%%%%%%%%%%%%%%%%%%%%%%%%%%%%%%%%%%%%%%%%%%%%%%%%%%%%%%%%%%%%%%
\section{Introduction} \label{sec:intro}
%%%%%%%%%%%%%%%%%%%%%%%%%%%%%%%%%%%%%%%%%%%%%%%%%%%%%%%%%%%%%%%%%%%%%%%%%%%%%%%
%
%\ttuwe{some further literature: resonance without conservation: \cite{YDZE2002prl,YDZE2002jcp}}
% nonreciprocal arxiv S Klapp?

%\ttuwe{remark on naming: Actually, \cite{PiRu1999ijbc} (sec.~4.1) also calls the homogeneous instability a ``long-scale instability'' because ``the instability domain is localized around $k_c = 0$''. That is exacltly the argument that Len did not want to hear from Uwe/Tobias.\\
%We should also look into \cite{DeDB1993pre,DLDB1996pre} or sec.VI of \cite{DeW1999acp}, are their eq.~(6.2) and (6.3) not the amplitude equations for the Hopf-Turing resonance? 
%}
%
%\ttuwe{conclusion: resonances between Turing and wave instabilities \cite{YDZE2002jcp} ro between different Turing instabilities \cite{YDZE2002prl} may also be interesting in the conserved case.}

Breaking Newton's third law has recently become a cherished pastime for theoretical physicists and applied mathematicians alike \cite{Pope2020l,YoBM2020pnasusa,KMSY2020pre,BFMR2022prx}. This not only formally breaks the boring symmetry in particle-particle interactions, but has dire consequences for the system's behavior: particles are not anymore attracted or repelled by their common center of mass (that remains at rest) but instead may start a chasing race as one (the ``predator'') is attracted by the other one while the latter (the ``prey'') is repelled by the first one \cite{ChKo2014jrsi}. In this way, oscillations and persistent motion may not only emerge for particle-based models but also characterize collective behavior as described by continuum models, e.g., nonreciprocal Cahn-Hilliard models \cite{SaAG2020prx,YoBM2020pnasusa,FrWT2021pre}, thereby providing a ``generic route to traveling states'' \cite{YoBM2020pnasusa}.\footnote{For a discussion of genericity see \cite{FHKG2022arxiv}.}

Bifurcationally speaking, above a critical value of the nonreciprocal coupling, oscillatory and traveling states emerge via Hopf and drift (pitchfork and transcritical) bifurcations \cite{FrWT2021pre}. Remarkably, such interactions may also result in the transformation of a monotonic large-scale or long-wave (Cahn-Hilliard) instability typical for phase separation \cite{Lang1992} into a monotonic small-scale (or short-wave) Turing-like instability well known from reaction-diffusion systems \cite{Turi1952ptrslsbs}. In consequence, in Cahn-Hilliard models nonreciprocity may cause complete suppression or arrest of coarsening \cite{FrWT2021pre} as well as the emergence of localized states \cite{FrTh2021ijam} with a slanted homoclinic snaking typical for systems with a conservation law \cite{Knob2016ijam,HAGK2021ijam}.

The original one-field passive Cahn-Hilliard model describes phase separation of binary fluid phases or isotropic solids \cite{CaHi1958jcp,Cahn1965jcp}. Eminent examples of nonvariational one-field variants include the convective Cahn-Hilliard model (broken parity symmetry) \cite{WORD2003pd,TALT2020n} and models extended by nonequilibrium chemical potentials that describe motility-induced phase separation \cite{SBML2014prl,RaBZ2019epje}. Two-field passive Cahn-Hilliard models feature a reciprocal coupling between species and are, e.g., employed to study phase separation in ternary mixtures as driven by gradients in the corresponding chemical potentials \cite{Eyre1993sjam,Ma2000jpsj}. Nonequilibrium conditions are readily attained when two interacting number-conserving species are present. It is sufficient to make their interactions \emph{nonreciprocal} \cite{SaAG2020prx,FrWT2021pre}. Particles of one kind may be attracted by particles of another kind, while the latter may be repelled by the former. Relations of this kind naturally occur between predators and prey or between parasitic and cooperating bacteria or between catalytic particles with different phoretic response to the chemical produced by a particles of another type. The analysis to follow reveals both parallels and differences between symmetry-breaking bifurcations in conservative active systems and in reaction-diffusion systems with autocatalytic components; a combination of number-conserved and -nonconserved species has also been considered in the framework of arrested phase separation \cite{LiCa2020jsm}. Two-field Cahn-Hilliard models with additional reaction terms in both equations (i.e., with nonvariational nonmass-conserving couplings) are also widely studied, e.g.\ in \cite{OkOh2003pre,SATB2014c,ZwHJ2015pre}.

In this communication, we will  reconsider the nonreciprocally coupled Cahn-Hilliard model (section~\ref{sec:model}), and discuss it as a fully mass-conserving equivalent to the classical Turing two-species reaction-diffusion system (section~\ref{sec:linear}). Then, we show that, in consequence, resonances exist between the conserved equivalents of Hopf instability and Turing instability. This occurs in the vicinity of the codimension-2 point where these two linear instabilities occur simultaneously (section~\ref{sec:resonance}). We close with a conclusion and an outlook in section~\ref{sec:conc}.

%Nonreciprocal two-field Cahn-Hilliard models  described by coupled continuity equations are applied to describe a variety of soft matter and biophysical systems, for instance,  active colloids \ttlen{ref needed}. 
%====================================================================
 \section{Cahn-Hilliard System with non-reciprocal coupling}
 \label{sec:model}

A nonreciprocal Cahn-Hilliard (CH) model describes interactions between two species with concentrations $u(\mathbf{x},t)$ and $v(\mathbf{x},t)$ where the effective nonequilibrium chemical potential of each species depends asymmetrically on the concentration of the other species:
%
\begin{equation}
\mu_u = \frac{\delta{\cal F}}{\delta u}+\mu_{u}^\mathrm{nv},  \qquad  \mu_v = \frac{\delta{\cal F}}{\delta v}+\mu_{v}^\mathrm{nv},
 \label{91muas} \end{equation}
% 
where
%
\begin{equation}
  {\cal F}
  % = {\cal F}_u +{\cal F}_v
  = \int {\rm d}\mathbf{x}\left( \frac{\kappa_u}{2}\left|\nabla u \right|^2 + \frac{\kappa_v}{2}\left|\nabla v \right|^2 + \chi(u,v) \right)
  \label{eq:energy}
\end{equation}
is the free energy functional with the general local potential $\chi(u,v)$. The nonvariational part of the chemical potentials $\mu^\mathrm{nv}_{u,v}$ (assumed to depend on $u$ and $v$) cannot be obtained from a common functional, so that $\partial_v \mu_{u}^\mathrm{nv}\neq \partial_u \mu_{v}^\mathrm{nv}$.
%Note that we abbreviate partial derivatives $\partial /\partial u$ as $\partial_u$ etc. 
%
Introducing the chemical potentials into the conservation laws $\partial_t u = -\nabla\cdot\vec{j}_u$ with $\vec{j}_u=-\gamma_u\nabla \mu_u$ (and similar for $v$) leads, after rescaling time and length, to the nonreciprocally coupled CH equations  
%
\begin{equation}
\frac{\partial}{\partial t} u= -\nabla^2 \left[\nabla^2 u + f(u,v) \right], \qquad
\frac{\partial}{\partial t} v= - \nabla^2 \left[\sigma \nabla^2 v + g(u,v)\right],
  \label{91uwas}  \end{equation}
%
where $\sigma = \gamma \kappa$ is the product of the ratios of mobilities $\gamma=\gamma_v/\gamma_u$ and rigidities $\kappa=\kappa_v/\kappa_u$ of the two species. Based on the functional \eqref{eq:energy}, the local terms in \eqref{91uwas} are then
  $f = - (\frac{\partial}{\partial u} \chi +\mu_{u}^\mathrm{nv})$ and $g = - \gamma (\frac{\partial}{\partial v} \chi + \mu_{v}^\mathrm{nv})$.
%  $f(u,v) = -a_u- u^3 - \mu_{u}^\mathrm{nv}$ and $g(u,v) = -a_v- v^3 - \mu_{v}^\mathrm{nv}$, respectively.

Note that the mass-conserving system \eqref{91uwas} without the outer Laplace operator $-\nabla^2$ directly corresponds to a typical two-species reaction-diffusion (RD) system, i.e., a system without mass conservation \cite{Pismen2006}. In the corresponding RD system, the parameter $\sigma $ represents the ratio of diffusion constants, while $f$ and $g$ represent the reaction terms. 
%
Below we will use this equivalence to relate the linear stability of homogeneous states of a nonreciprocal CH system directly to the linear stability of such states in RD systems. Due to mass conservation, in the CH system any homogeneous state $(u,v)=(u_s,v_s)$ automatically corresponds to a steady state. In contrast, for an RD system this requires adjustments of the constant parts of $f$ and $g$. We consider the linear stability of $(u_s,v_s)$ and show that a nonreciprocal coupling does not only allow for the classical CH instability (i.e., conserved monotonic large-scale instability) but may also result in a Matthews-Cox (MC) instability \cite{MaCo2000non, Proc2001pla} (i.e., conserved monotonic small-scale instability) and a ``conserved Hopf'',  i.e., a wave instability on the longest available scale replacing homogeneous oscillations incompatible with conservation laws.
%In the following we use the naming conventions for instabilities as given in Table~\ref{tab:instab}.
A conserved traveling wave, i.e., a conserved oscillatory small-scale instability requires at least three nonreciprocally coupled Cahn-Hilliard equations.
% However, if one allows for mixed conserved and nonconserved dynamics, already a three-component RD system  with one conservation law allows for a conserved oscillatory small-scale instability (see Ref.~\cite{FrTh2023pta} for an overview of different linear instabilities in the presence of conservation laws). %\ttlen{do you mean two? if 3, has to be merged with the preceding} UWE: I do not understand the remark!

%. instability can be very similar. In that case the amplitude equations consists of left and right TW amplitudes coupled to one neutral mode. In your case they are coupled to three neutral modes. My suggestion: In analogy to RD networks with $N$ coupled RD Equations we also introduce the conserved version for $N$ equations, then say that we consider $N=2$ and here we could say that for a ``conserved traveling wave'' we need $N=3$ which corresponds to three nonreciprocally coupled Cahn-Hilliard equations.}
%\ttuwe{Tobias, I think we should cover this at least briefly - but could be in our other contribution, then we would add a reference here.}\tobias{yes it is part of our other contribution anyway}

\section{Linear stability: Relation between conserved and nonconserved dynamics}
\label{sec:linear}

\subsection{Classification of instabilities}

\begin{table}[t]
\begin{tabular}{c || c | c}
& nonconserved dynamics& conserved dynamics\\
\hline
\hline
small-scale, monotonic & Turing & Matthews-Cox~\\
\hline
homogeneous/large-scale, monotonic & Allen-Cahn & Cahn-Hilliard~\\
\hline
homogeneous/large-scale, oscillatory & Hopf\footnote{Also known as Poincaré-Andronov-Hopf bifurcation.} & conserved Hopf~\\
\hline
\end{tabular}
\caption{Naming convention of linear instabilities classified via their spatial (homogeneous/large-scale vs.\ small-scale) and temporal (monotonic vs.\ oscillatory) properties for the cases of nonconserved and conserved dynamics.}
\label{tab:instab}
\end{table}

Before presenting the linear analysis of the model~\eqref{91uwas}, we introduce in Table~\ref{tab:instab} our classification of instabilities relevant for the present work. In the literature, the Cross-Hohenberg classification~\cite{CrHo1993rmp} is often used. However, here it is not a good choice, because it does not clearly distinguish between the conserved dynamics, i.e.,~model~\eqref{91uwas}, and the nonconserved dynamics, i.e.,~the corresponding RD system (Eq.~\eqref{91uwas} without the leading $-\nabla^2$). In Table~\ref{tab:instab} we distinguish two main classes - conserved and nonconserved dynamics, each divided into three subclasses depending on the spatial and temporal character of the growing modes at and very close to onset.

First, if the imaginary part of the temporal eigenvalue is zero, the unstable mode grows monotonically, if not, it defines the temporal frequency of the oscillation. Second, the wavenumber encoding the spatial structure of the unstable mode at onset is either zero or finite. In the former case one has an homogeneous or large-scale instability, in the latter the wavenumber defines the characteristic length scale of a small-scale instability.\footnote{Small-scale and large-scale instability are also referred to as ``short-wave'' and ``long-wave'' instability, respectively. Alternatively, but much less frequently also ``short-scale'' and ``long-scale'' instability is used \cite{GoNP1994pf}.}
In the nonconserved case, the instability at $k=0$ always corresponds to a homogeneous (or global) mode as each point of the finite or infinite domain grows monotonically or oscillatory without any spatial modulation. Thus, we refer to it as a ``homogeneous instability''. Such an homogeneous behavior is, however, incompatible with a fully conserved dynamics.\footnote{This is the case for model~\eqref{91uwas} where all components are conserved. For systems with fewer conserved quantities than dynamically evolving fields, homogeneous oscillations compatible with the conservation law are possible.} Instead, the monotonic or oscillatory mode with the smallest wavenumber compatible with the boundary conditions is excited, e.g.~for periodic boundary conditions its wavelength equals the domain size. This is called a conserved Hopf (or oscillatory long-wave) instability. In the following, we address all linear instabilities by the names given in Table~\ref{tab:instab}.
%
% Google says:
% "large-scale": 565.000.000; "long-wave": 4.170.000; "long-scale": 1.300.000
% "large-scale instability": 44.200; "long-wave instability": 20.900; "long-scale instability": 723 (most of them by Golovin/Pismen & Frohoff...)
% "small-scale": 142.000.000; "short-wave": 5.960.000; "short-scale": 5.140.000
% "small-scale instability": 22.600; "short-wave instability": 20.800; "short-scale instability": 1.460
% Web of Science says:
% "large-scale": 67k; "long-wave": 3.000; "long-scale": 97 in title
% "large-scale": 500k; "long-wave": 13.000; "long-scale": 1000 in topic
%
\subsection{Instability thresholds}
\label{sec:model2}
%
Using the ansatz $(u,v)=(u_s,v_s) + \varepsilon (u_1,v_1)\,\exp(\lambda t+ \text{i} \vec{k}\cdot\vec{x})$ with $\varepsilon\ll1$, and abbreviating partial derivatives w.r.t.~$u$ and $v$ as subscripts, e.g.~$\frac{\partial}{\partial u} f = f_u$, the linearized equations~\eqref{91uwas} are
 \begin{equation} 
   \left(\tens{L}(k^2)-\lambda \mathbb{1}\right)\,\left(\begin{array}{c}u_1\\v_1 \end{array} \right)=\vec{0}
  \quad\mathrm{with}\quad  
   \tens{L}= k^2\left( \begin{array}{cc} f_u- k^2 & f_v \\  g_u & g_v-\sigma k^2
  \end{array} \right) = k^2 \,\widetilde{\tens{L}}
\label{91jac}
\end{equation} 
%
The derivatives in the jacobian $\tens{L}$ are computed at the homogeneous state $(u_s,v_s)$ which we do not need to specify. The eigenvalues are given by
\begin{equation}
 \lambda= k^2 \tilde\lambda \qquad\mathrm{with}\qquad  \tilde\lambda=\frac{\tr \widetilde{\tens{L}}}{2} \pm \sqrt{\frac{(\tr \widetilde{\tens{L}})^2}{4} - \det \widetilde{\tens{L}}}
 \label{eq:lambda}  \end{equation}
It is important to note that the expression \eqref{91jac} differs from the classical Turing problem of the linear stability of a two-component RD system \cite{Turi1952ptrslsbs,Pismen2006} solely by the factor $k^2$. This implies that $\det\tens{L}(k^2)$ has zeros wherever $\det\widetilde{\tens{L}}(k^2)$ does.  Moreover, since thresholds of symmetry-breaking instabilities correspond to zero crossings of maxima of $\mathrm{Re}\lambda(k^2)$ (i.e., where its derivative with respect to $k^2$ vanishes) whenever both, $\mathrm{Re}\tilde\lambda(k^2)=0$ and $\partial_{k^2}[\mathrm{Re}\tilde\lambda(k^2)]=0$,
also $\mathrm{Re}\lambda(k^2)=0$ and $\partial_{k^2}[\mathrm{Re}\lambda(k^2)]=\partial_{k^2}[k^2\mathrm{Re}\tilde\lambda(k^2)]=0$ implying identical instability thresholds of conserved and nonconserved case.\footnote{The determinant $\det\tens{L}(k^2)$ has, in addition to zeroes of  $\det\widetilde{\tens{L}}(k^2)$, a persistent zeros at $k=0$ than. This may be irrelevant for linear stability but important for weakly nonlinear analysis.}  Therefore, the stability diagrams for homogeneous states of model~\eqref{91uwas} and of the corresponding RD model are identical. The discussed equivalence directly implies that $\sigma$, i.e.~the product of the ratios of mobilities and rigidities in the nonreciprocal Cahn-Hilliard system (conserved case), takes the role of the ratio of diffusion constants in the corresponding RD system (nonconserved case). However, besides their zero crossings, the dispersion relations of conserved ($\lambda(k)$) and nonconserved ($\tilde\lambda(k)$) case are different, and distinctions between the two cases are important for nonlinear analysis and detection of secondary instabilities. %\ttlen{replacing the removed part}

The instability thresholds for a conserved  system can therefore be established by analyzing the eigenvalues $\tilde\lambda$ for the nonconserved case. The onset of all monotonic instabilities, i.e., with $\Re \tilde\lambda = \Im \tilde\lambda = 0$, is determined by $\det \widetilde{\tens{L}}=0$, i.e.
\begin{align}
 0 =f_u g_v - f_v g_u - k^2(\sigma f_u +g_v)+\sigma k^4 \equiv A-k^2B+\sigma k^4~
 \label{91Jas}  \end{align}
%
which gives the following wavenumbers of marginally stable modes 
%\ttlen{is my eq. 5 wrong?}
\begin{equation}\label{eq:kpm}
k_\pm^2 = \frac{B}{2\sigma}\left[ 1 \pm \sqrt{1- \frac{4 \sigma A}{B^2}}\right]\,.
\end{equation}
Eq.~\eqref{eq:kpm} can have zero, one, or two positive real solutions. In the latter two cases the band of wavenumbers corresponding to positive real eigenvalues is $\left[0,k_\pm\right]$ and $\left[k_-,k_+\right]$, respectively\footnote{Provided that $\text{tr}\widetilde{\tens{L}}$ is negative at the roots. For a positive trace the corresponding root belongs to the subdominant eigenvalue (``$-$''sign in Eq.~\eqref{eq:lambda}), and hence the dominant eigenvalue is positive and has no root at $k_\pm$. 
%\ttlen{not so, then Hopf is dominant}\tobias{I think we cannot say what is dominant since we are well beyond the onset of linear instability, see Fig 2 c)}
}.
%\tobias{not completely correct, since in the inverse Turing case $\left[k_-,k_+\right]$ is the band of stable modes, see Fig 2 c)}\ttuwe{difficult to explain in text, maybe a footnote that there is an additional condition?}\tobias{see footnote}
If $B<0$ only $k_-$ can be real and only if $A=f_u g_v-f_v g_u<0$. The onset occurs at $k_-=0$ for $A=0$, i.e.~
\begin{equation}\label{eq:CH}
f_u = g_v^{-1}f_v g_u\,.
\end{equation} 
This corresponds to an Allen-Cahn instability (Table~\ref{tab:instab}).
If $B>0$, both $k_+$ and $k_-$ are real if $\frac{B^2}{4 \sigma}\geq A\geq 0$. A Turing instability occurs if $k_+=k_-$, i.e.\ with critical wavenumber
\begin{equation}\label{eq:kT}
k_T^2 =k_\pm^2=\frac{B}{2 \sigma}=\frac{\sigma f_u + g_v }{2 \sigma}= \frac{g_v \pm \sqrt{-\sigma f_v g_u}}{\sigma}\,.
\end{equation} 
This instability appears at $A=\frac{B^2}{4 \sigma}$, i.e.\ at $\sigma f_u = g_v  \pm 2 \sqrt{-\sigma f_v g_u}$ if the trace
\begin{equation}\label{eq:trace}
\tr \widetilde{\tens L} = f_u + g_v -  k_T^2 (1+ \sigma)
\end{equation}
 is negative, otherwise it would correspond to a minimum of the dispersion relation instead of a maximum.
That is, for an RD system a Turing instability requires at least one species to be autocatalytic ($B>0$) and, additionally, $(1-  \sigma) (\sigma f_u - g_v)<0$, which is proven by inserting $k_T$~\eqref{eq:kT} into the trace~\eqref{eq:trace} that has to be negative.  Here we choose $u$ as the one autocatalytic species, i.e.~we assume $f_u>0, \, g_v<0$, so that a Turing instability only occurs\footnote{In the RD setting, it corresponds to the known requirement that the inhibitor diffuses faster than the activator. That is, changing the roles of $u$ and $v$ alters the condition to $\sigma<1$. If both, $f_u$ and $g_v$, are either positive or negative a Turing instability cannot occur.} if $  \sigma >1 $ at
\begin{equation}\label{eq:Turing}
\sigma f_u = g_v  + 2 \sqrt{-\sigma f_v g_u}> |g_v|\,.
\end{equation} 
This means that nonreciprocity via $f_v g_u<0$ is another necessary condition, i.e.~a reciprocal interaction always prevents a Turing instability.
When $A$ crosses zero for $B>0$, $k_-$ becomes again complex, i.e., then $k_+$ is the only remaining root of the dispersion relation. However, this does not correspond to an Allen-Cahn instability, instead the Turing band of unstable wavenumbers simply attaches to $k=0$ when $A\leq 0$.


The loci of all oscillatory instabilities, i.e., at onset with $\Re \tilde\lambda = 0$ and frequency $\Im \tilde\lambda = \tilde\omega \ne 0$, are determined by $\tr \widetilde{\tens{L}}=0$, i.e., if $f_u +   g_v - k^2 (1+   \sigma)=0$. This gives marginally stable modes with
  \begin{equation}\label{eq:ko}
k_\text{o}^2 = \frac{f_u +   g_v }{1+   \sigma}
\end{equation}
and oscillations first occur at $k_\text{o} = 0$, i.e.\ only a Hopf instability (Table~\ref{tab:instab}) is possible.\footnote{To also obtain a traveling wave, i.e., oscillatory small-scale instability one needs at least three species, see e.g.~\cite{BeGY2020c,TeKN2005pd}.}
Its threshold is at 
\begin{equation}\label{eq:Hopf}
f_u +   g_v =0
\end{equation}
 with $\tilde\omega_0=\det\widetilde{\tens{L}}(k=0)=A$.
%

Next, we compare monotonic and oscillatory instabilities by considering the transition from real to complex eigenvalues. Complex eigenvalues $\widetilde \lambda$ occur for $(\tr{\tens{\widetilde L}})^2 - 4 \det{\tens{\widetilde L }}<0$, i.e., if 
%\begin{equation}
%A\geq A_c = \frac{1}{4  } \left[ f_u -   g_v - k^2\left(1-   \sigma \right) \right]^2
% \end{equation}
%\begin{equation}
%  g_v +  k^2 (1-   \sigma) - 2 \sqrt{-  f_v g_u} <  f_u <   g_v + k^2 (1-   \sigma) + 2 \sqrt{-  f_v g_u}
%\label{eq:real-oscil-k}
%\end{equation}
%\ttuwe{maybe the general relation is preferable
\begin{equation}
  [f_u -   g_v - \left(1-   \sigma \right) k^2]^2 + 4 f_v g_u < 0
  \label{eq:real-oscil-k}
\end{equation}
Similar to the Turing instability, any oscillatory instability requires nonreciprocal interactions $f_v g_u<0$. 
Note that only this product of inter-species interaction enters the linear analysis of eigenvalues. In other words, if the interaction is only unidirectional, e.g.~$g_u=0$, then the dispersion relation is unaltered, although the eigenmode is still affected.

In consequence, for $k=0$ condition~\eqref{eq:real-oscil-k} can be reduced to
\begin{equation}
A> \frac{1}{4 } (f_u +   g_v)^2
\end{equation}
which is outside the parameter region where a large-scale monotonic instability is observed ($A<0$).

For $k=k_T$ \eqref{eq:kT} in~\eqref{eq:real-oscil-k} the eigenvalue is complex if
%\begin{equation}
% g_v - \frac{2 \sqrt{\sigma  }}{1+ \sigma  } 2 \sqrt{-\sigma f_v g_u} < \sigma f_u<  g_v + \frac{2 \sqrt{\sigma  }}{1+ \sigma  }  2 \sqrt{-\sigma f_v g_u} \,.
%\end{equation}
%\tobias{ or should we write
\begin{equation}
 g_v - \frac{4 \sigma}{1+ \sigma  } \sqrt{- f_v g_u} < \sigma f_u<  g_v + \frac{4 \sigma}{1+ \sigma  } \sqrt{- f_v g_u} \,.
\end{equation}
%\ttuwe{just for me
 % \begin{equation}
% \left[\frac{2\sigma}{1+   \sigma} f_u -   \frac{2}{1+   \sigma} g_v \right]^2 + f_v g_u =    
 % (1+ \sigma)^2(\sigma f_u -g_v)^2 + 16\sigma^2 f_v g_u < 0
%  \label{eq:real-oscil-k2}
%\end{equation}}
%First variant is for comparison with Turing onset....but maybe more confusing than helpful}
The onset of the Turing instability [cf.~\eqref{eq:Turing}] is outside of this interval if $\sigma   \ne 1$. Then, our previous considerations based on real eigenvalues applies for the Turing instability. It also applies for the Cahn-Hilliard instability. For the special case $\sigma=1$, complex eigenvalues occur independently of the wavenumber at $\sigma f_u = g_v  \pm 2 \sqrt{-\sigma f_v g_u}$. This is identical to the onset condition for the Turing instability [cf.~\eqref{eq:Turing}]. Furthermore, at this specific point one has $k_T=k_\text{o}$. Thus, the Turing instability is prohibited by complex eigenvalues if $\sigma=1$ as one would expect for a Turing system with equal diffusion constants.

\begin{figure}[t]
\includegraphics[width=0.7\textwidth]{{./fugv_instab_sigma1.5fvgu-0.54_paramFig2_version2}.pdf}~\\
\caption{Linear stability diagrams in the $(g_v,f_u)$-plane showing thresholds of Hopf/conserved Hopf (homogeneous/large-scale oscillatory, Eq.~\eqref{eq:Hopf}), Turing/Matthews-Cox (small-scale monotonic, Eq.~\eqref{eq:Turing}), and Allen-Cahn/Cahn-Hilliard (homogeneous/large-scale monotonic, Eq.~\eqref{eq:CH}) instabilities as blue, orange and green lines, respectively. Here, $\sigma=1.5>1$ and nonreciprocity $f_v g_u=-0.54$. The focus lies on the region where $u$ is autocatalytic ($f_u>0$) and $v$ is not ($g_v<0$). The linearly stable region is delimited by heavy solid lines. Thin solid lines indicate where further instabilities set in beyond the dominating one. The dashed green line indicates where the already unstable Turing band reaches $k_-=0$ (transition across orange line) or where the unstable complex eigenvalues near $k=0$ become real (transition across blue line). The square symbol marks the codimension-2 point [Eq.~\eqref{eq:a22cd2}] where Hopf/conserved Hopf and Turing/Matthews-Cox instabilities simultaneously occur. The cross symbol indicates the loci of the dispersion relation given in Fig.~\ref{fig:dispersion-relations}.
} % $f_v=-1$, $g_u=0.54$
\label{fig:stab-diagram} 
\end{figure}

 A codimension-2 point exists if Hopf ($f_u +   g_v=0$) and Turing ($\sigma f_u = g_v + 2 \sqrt{-\sigma f_v g_u}$) instability occur simultaneously, i.e.~if
\begin{equation}
g_v = -f_u = -2 \frac{\sqrt{-\sigma f_v  g_u }}{1+  \sigma}
\label{eq:a22cd2}
\end{equation}
and all aforementioned requirements are fulfilled, too.
%
A typical stability diagram in the ($g_v,f_u$)-plane at fixed  $f_v$, $g_u$ and $\sigma$ is given in Fig.~\ref{fig:stab-diagram}. 

Although onset conditions for linear instabilities for the two-species RD system and the corresponding nonreciprocal two-field CH model are identical, the respective dispersion relations are not. In particular, the frequency scales with $k^2$, i.e., $\omega_0=k^2 \tilde \omega_0$. Therefore, directly at onset the large-scale instability cannot be oscillatory as there $k_\text{o}=0$, and the ``conserved Hopf'' instability differs from the standard Hopf  instability at $f_u=-g_v$, and takes place only when a mode with the largest available wavelength becomes unstable.
% \footnote{For systems with at least two more order parameter fields than conservation laws, homogeneous oscillations are possible. Three-component RD systems with one conservation law are minimal models for exhibiting homogeneous oscillations in the presence of a conserved quantity~\cite{FrTh2023pta}}%\ttlen{these systems are not conserved, but have one conserved component } In a finite system with the domain size $L$,d boundary conditions select a few modes from the band(s) of unstable wavenumbers. For periodic boundary conditions, these unstable modes have wavenumbers $k_{L/n}=\frac{2\pi n}{L}$, i.e.~wavelengths $L/n$, where $n$ is an integer. A conserved Hopf instability occurs when the marginal mode $k_o$ passes the smallest permitted wavenumber, i.e.,~$k_L=\frac{2\pi}{L}$. Similarly, the onset of the Matthews-Cox instability is shifted unless $k_T=k_{L/n}$ exactly holds for some $n$. Otherwise, $k_-$ or $k_+$ have to cross one of the $k_{L/n}$ to trigger the corresponding Matthews-Cox instability. 

\begin{figure}
\includegraphics[width=0.6\textwidth]{./onset_resonanceFig2}~\\[-3ex]
\caption{(a) Dispersion relation $\mathrm{Re}\,\lambda(k)$ [Eq.~\eqref{eq:lambda}] at the parameter values where the Matthews-Cox instability has its onset (at $k_{L/2}=k_T$) and is resonant with the critical mode of the conserved Hopf instability ($k_o=k_L$), i.e.~$k_T= 2k_o$. Parameters are $\sigma=1.5$, $f_u=0.76$, $f_v=-1$, $g_u=0.54$ and $g_v=-0.66$ and correspond to the cross symbol in Fig.~\ref{fig:stab-diagram}. In a finite system this resonance is only realizable for a perfectly tuned domain size of $L=10 \pi$.}
\label{fig:dispersion-relations} 
\end{figure}

As an interesting consequence of this finite size effect, resonances occur in the vicinity of the codimension-2 point of the infinite system. In the most interesting case, an oscillatory mode (wave) with $k_L =k_\text{o}$
%  the maximal but finite characteristic length scale $L=\frac{2\pi}{k_\text{o}}$
and a monotonic mode of wavenumber $k_{L/n}=k_T$ where $n>1$ are simultaneously marginal. Fig.~\ref{fig:dispersion-relations} presents a corresponding dispersion relation  when $n=2$ corresponding to parameters marked by a cross symbol in Fig.~\ref{fig:stab-diagram}. The larger $n$ the closer the position moves on the orange line toward the codimension-2 point. For finite systems of domain size $L\neq2\pi n/k_T$ the monotonic relevant modes are rather related to $k_-$ or $k_+$ the limiting values of the band of unstable Matthew-Cox modes [cf.~\ref{eq:kpm}].
Close to the corresponding primary bifurcations the behavior can be analyzed with the help of a weakly nonlinear analysis\cite{PiRu1999csf,YDZE2002jcp}, as described in section~\ref{sec:resonance-weak} for a general system \eqref{91uwas}. Further above onset on may compare the general weakly nonlinear results fully nonlinear time simulations as done in section~\ref{sec:resonance-specific} for a specific conserved amended FitzHugh-Nagumo model.
%
%and $k_-\lesssim k_{L/n} \lesssim k_+$ have similar growth rates and subsequently interact in the nonlinear regime. This behavior can occur arbitrarily close to the 
%
%An example is given in Fig.~\ref{fig:dispersion-relations}~(a) where $k_L$ and $k_{L/2}$ belong to the oscillatory and monotonic mode, respectively, and have similar growth rates. Such resonances are widely discussed for RD systems, i.e., in the absence of conservation laws. See, for instance, Refs.~ 
%\ttuwe{this paper is actually about small-scale monotonic and small-scale oscillatory. Len, do you have other references?}\ttlen{Pismen and Rubinstein, 1999 is not even in the list}\tobias{I added it}
%  characteristic length scale $L/n = \frac{2\pi}{k_+}$ [or $= \frac{2\pi}{k_-}$ if $k_-$ crosses a permitted wavenumber first]. 
%
%
%Another resonance occurs for $\sigma<1$, given that $u$ is autocatalytic and $v$ is not. It follows then rom Eq.~\eqref{eq:real-oscil-k} that the transition from real to complex eigenvalues occurs at
% \begin{equation}
%k_{rc,\pm}^2=\frac{\pm 2\sqrt{-f_v g_u} - f_u +  g_v}{\sigma-1}\,.
%\label{eq:krc}
%\end{equation}
%Above the onset of the conserved Hopf instability, positive real eigenvalues occur in the band $\left[0,k_{rc}^+\right]$ while complex eigenvalues with positive real parts are found in the band $\left[k_{rc}^+, k_\text{o}\right]$ as shown in Fig.~\ref{fig:dispersion-relations}~(b). 
%So only \emph{waves} with a finite $k_{rc,+} < k<k_\text{o}$ are possible, and, as ``autocatalytic activity'' $f_u$ increases, the band narrows since $k_{rc,+}$ catches up with $k_\text{o}$ while both increase. 
%This allows shorter waves to get excited. \ttuwe{What I do not see is how to clearly discuss how $k_{rc}$ and $k_o$ go to zero when the stability threshold is approached from above.} Note that an example for the described scenario is given in Fig.~\ref{fig:dispersion-relations}~(a).
%In this case, resonances occur between stationary modes and oscillatory modes of smaller wavelength that have similar growth rates. In Fig.~\ref{fig:dispersion-relations}~(b) the relevant interacting modes, i.e.~the ones with the largest growth rates, are the stationary $k_{L/2}$ mode and the oscillatory $k_{L/3}$ and $k_{L/4}$ modes. Their interaction can result in a suppression of phase separation~\cite{FrWT2021pre}.  In contrast to the resonance described above [cf.~Fig.~\ref{fig:dispersion-relations}~(a)], here, the interaction occurs well beyond the onset of instability. Therefore, a weakly nonlinear approach can not be applied.
%
%\ttuwe{I think \ref{fig:dispersion-relations}~(c) and corresponding discussion should be dropped. It is complicated and the ``inverse Matthews-Cox instability'' is not really an instability as the system does not change stability. It is surely important for the bifurcation diagrams, but we do not want to analyse them here anyway.}\tobias{Agree - see new Fig 2 [old one is commented]}
%
%The linear stability behavior for $\sigma<1$ (needed for the second case) is summarized in Fig.~\ref{fig:dispersion-relations}~(c) that should be compared to Fig.~\ref{fig:stab-diagram} for $\sigma>1$. Additionally to the instability thresholds the dotted green line in panel~(c) marks the occurrence of CH modes beyond the onset of the conserved Hopf instability given by $k_{rc}^+$~\eqref{eq:krc}. The right cross symbol corresponds to the case shown in panel~(b).
%
%%For an increasing difference between autocatalytic activity $f_u$ and $g_v$ one crosses the dashed orange line indicating an inverse Matthews-Cox instability.
%%``Inverse'' here indicates that in contrast to the MC instability for $\sigma>1$ marked by the orange line in Fig.~\ref{fig:stab-diagram}, here, a finite wavenumber band $\left[k_-, k_+\right]$ stabilizes, cf.~Fig.~\ref{fig:dispersion-relations}~(c) [left cross in panel~(d)]. As this inverse instability only affects the subdominant eigenvalue it will be rather irrelevant for the nonlinear behavior. Increasing $f_u-g_v$ further, $k_{rc}^+$ finally approaches $k_\text{o}$, i.e.~all unstable modes become stationary. This corresponds to the crossing of the dashed green line in Fig.~\ref{fig:dispersion-relations}~(d).
%In the following we focus on the case depicted in Fig.~\ref{fig:dispersion-relations}~(a) and derive the corresponding leading order amplitude equations that can describe resonance.
% and does only apply if no additional Turing monotonic instability is present at larger $k$, i.e., roughly to the right of the thin orange line in Fig.~\ref{fig:stab-diagram}.\tobias{Why does is not apply in the presence of a Turing instability? Do you refer to the case when the Turing instab sets in first and we expect it to be dominant in the linear regime?}
% UWE: ACTUALLY, I DO NOT KNOW WHERE THIS STATEMENT COMES FROM, MAYBE LEN? I JUST REMOVED IT
%
%\ttuwe{Actually in case (b) we could also have a resonance if $k_L$ and $k_{L/2}$ belong to the monotonic and oscillatory band, respectively, and have similar growth rates.}
%\tobias{Is a resonance analysis possible? At least if we want to capture the resonance via a WNA this would not work, I suppose, since it's not sufficient to have similar growth rates but small growth rates, too, in order to truncate the expansion, right? Case (b) however occurs far  (order (1)) away from the bifurcation related to the monotonic mode of resonance, i.e. we do not know any scaling of its growth rate. Or am I wrong? In contrast to case (a) where both bifurcations occur simultaneously [or almost simultaneously]} \ttuwe{you are right: I was thinking about a simulation.}
%
%
\section{Wave--Turing Resonance - weakly nonlinear analysis}
\label{sec:resonance}
%\subsection{General weakly nonlinear analysis}
\label{sec:resonance-weak}
 We consider the resonant interaction between a Matthews-Cox and two conserved Hopf modes with the three wave vectors forming an isosceles triangle. In a finite system, they satisfy the condition $\vec{k}_{o1}-\vec{k}_{o2} + \vec{k}_\pm=0$ with $k_{o1}=k_{o2}\equiv k_o$. The corresponding interactions for a nonconserved system has been analyzed in \cite{PiRu1999csf}. 
%Here, the situation is not as clear-cut, since waves within a range of wavelengths are possible. \ttuwe{Len,please send pdf of corresponding section and give further indications what are further conditions as a triangle can always be constructed if $2 k_{o1}>k_\pm$.}
We consider the case when the resonance occurs close to the common onset of linear instability for a specific finite system, defined by the critical values of two parameters. We impose a small deviation in one of these parameters, let us say $\beta=\beta_c + \varepsilon$ with $|\varepsilon|\ll 1$. Then, both the growth rates and the amplitudes are small and, thus, by expanding them in powers of $\varepsilon$, they are treated through a weakly nonlinear approach.

As an ansatz, we write the two-component vector field $\vec u = (u,v)$ as
\begin{equation}\label{eq:ansatz}
\vec u = \vec u_s + \varepsilon \left[a_+(T) \vec u_+ e^{\text{i} \vec k_+ \cdot \vec x} +a_{o1}(T) \vec u_{o1} e^{\text{i} \left(\vec k_{o1} \cdot \vec x + \omega_{o1} t\right)} +a_{o2}(T) \vec u_{o2} e^{\text{i} \left(\vec k_{o2} \cdot \vec x + \omega_{o2} t\right)} + \text{c.c.}\right] + \mathcal{O}(\varepsilon^2)
\end{equation}
where $a_+(T)$, $a_{o1}(T)$, $a_{o2}(T)$ are the amplitudes that evolve on a large timescale $T=\varepsilon t$; $\vec u_+ \in \mathbb{R}$ and $\vec u_{o1}, \vec u_{o2}\in \mathbb{C}$ are the zero eigenvectors of the stationary and the two wave modes, respectively. The frequencies $\omega_{o1}$ and $\omega_{o2}$ are the imaginary parts of the eigenvalues at onset of instability of the wave modes. We consider an isotropic system, which implies that $\vec u_{o1} = \vec u_{o2} \equiv \vec{u}_o$ and $\omega_{o1}=\omega_{o2} \equiv \omega_o$.
After inserting Eq.~\eqref{eq:ansatz} into Eqs.~\eqref{91uwas}, the leading-order amplitude equations are obtained at $\mathcal{O}(\varepsilon^2)$ by applying solvability conditions, i.e. multiplying with corresponding adjoint eigenvectors $\vec{u}_+^\dagger$, $\vec{u}_o^\dagger$ [that are normalized to satisfy $\vec{u}_+^\dagger \vec{u}_+= \vec{u}_o^\dagger \vec{u}_o=1$] and projecting onto the extant Fourier modes. As a result, we obtain
\begin{equation}\label{eq:AE}
\begin{split}
\partial_T a_+ = \vec u_+^\dagger \cdot \frac{\partial \tens{L}(k_+^2)}{\partial \beta}\cdot\vec{u}_+ (\beta-\beta_c) a_+ +  \vec u_+^\dagger \cdot \left(\vec{\bar u}_o \cdot \tens{\tens{H}}(k_+^2) \cdot \vec u_o\right) \bar a_{o1}  a_{o2},~\\
\partial_T a_{o1} = \vec u_o^\dagger \cdot \frac{\partial \tens{L}(k_o^2)}{\partial \beta}\cdot\vec{u}_o (\beta-\beta_c) a_{o1} +  \vec u_o^\dagger \cdot \left(\vec{u}_+ \cdot \tens{\tens{H}}(k_o^2) \cdot \vec u_o\right) \bar a_{+}  a_{o2},~\\
\partial_T a_{o2} = \vec u_o^\dagger \cdot \frac{\partial \tens{L}(k_o^2)}{\partial \beta}\cdot\vec{u}_o (\beta-\beta_c) a_{o2} +  \vec u_o^\dagger \cdot \left(\vec{u}_+ \cdot\tens{\tens{H}}(k_o^2) \cdot \vec u_o\right) a_{+}  a_{o1},~\\
\end{split}
\end{equation}
where $\tens{\tens{H}}(k^2)= \nabla_{\vec{u}}\tens{L}(k^2)$ is the Hessian evaluated for the homogeneous steady state $\vec u= \vec u_s$ and at $\beta=\beta_c$. Note the symmetry $H_{ijk}=H_{ikj}$. In the index notation, the Hessian can be written as
\begin{equation}
H_{1ij}(k^2) = k^2 \widetilde H_{1ij} = k^2 f_{ij} \qquad H_{2ij}(k^2) = k^2 \widetilde H_{2ij} =k^2 g_{ij}
\end{equation}
where the indices $\left\{i,j\right\}=\left\{1,2\right\}$ correspond to partial derivatives $\left\{i,j\right\}=\left\{u,v\right\}$ on the right hand side.\footnote{In index notation the first equation of \eqref{eq:AE} reads
\begin{equation}
\begin{split}
\partial_T a_+ = \left(u_+^\dagger\right)_i \frac{\partial L_{ij}(k_+^2)}{\partial \beta}\left({u}_+\right)_j (\beta-\beta_c) a_+ + H_{ijk}(k_+^2) \left(u_+^\dagger\right)_i \left(\bar u_o\right)_j  \left(u_o\right)_k \bar a_{o1}  a_{o2}.
\end{split}
\end{equation}
}
The amplitude equations~\eqref{eq:AE} describe a special example of resonant interactions allowing for finite amplitude states in the absence of stabilizing cubic terms (cf.~Ref.~\cite{PiRu1999csf}).

The general form of the lowest-order resonant amplitude equations is the same as in the standard Turing, i.e.~nonconserved, case, but the coefficients of these equations, dependent on the eigenvectors and the Hessian, are, of course, different:
%
 \begin{align} 
 \dot a_+ &= k_+^2 (\mu_+ a_+ + \nu_+ \bar a_{o1} a_{o2} )~\nonumber\\
\dot a_{o1} & = k_o^2(\mu_o a_{o1} + \nu_o \bar a_{+} a_{o2} ), \quad 
\dot a_{o2}  =  k_o^2(\mu_o a_{o2} + \nu_o a_+  a_{o1})
\label{9ampl}\end{align}
%
where $\nu_+= \vec u_+^\dagger \cdot \left(\vec{\bar u}_o \cdot \tens{\tens{\widetilde H}}(k_+^2) \cdot \vec u_o\right)$ is real, while  $\nu_o =\vec u_o^\dagger \cdot \left(\vec{u}_+ \cdot \tens{\tens{\widetilde H}}(k_o^2) \cdot \vec u_o\right)$ is complex. The coefficient $\mu_+$ is real and since the imaginary part of $ \mu_o$ can be absorbed into the frequency of the wave modes (i.e.,~applying a transformation $a_{o1/2} \to a_{o1/2}e^{i k_o^2 \text{Im}\mu_o t}$ or equivalently $\omega_o \to \omega_o + k_o^2 \text{Im} \mu_o$), this parameter can be viewed as real, too. Since the interaction coefficients are generally distinct, the system lacks gradient structure, allowing, in principle, for persistent non-stationary behavior within the amplitude-equation representation leading to secondary oscillations on an extended scale. Including cubic terms is unnecessary close to the onset, since the amplitudes may remain finite in this system even without higher-order damping interactions.

Using the polar representation of the complex amplitudes, $a_+ = \rho_+\; \E^{\I \theta_+}, \; a_{o1} = \rho_1 \E^{\I \theta_1}, \; a_{o2} = \rho_2 \E^{\I \theta_2}$, (\ref{9ampl}) is written in terms of the three real positive amplitudes and three phases, but dynamics depends only on the single phase combination $\Theta = \theta_+ + \theta_1 - \theta_2$, so that (\ref{9ampl}) can be reduced to a system of four real equations. The complex parameter $\nu_o$ will be presented in a polar form $\nu_o = \nu\E^{\I\varphi}, \, |\varphi|<\pi$. Three out of the five real parameters of this system can be eliminated by rescaling the amplitudes and time. Choosing $\mu_+$ and $\nu_+$ of the same sign implies $\cos \Theta<0$, whereas $|\Theta|<\pi/2$ if the signs are opposite, because only then the system allows for stationary solutions.
% \tobias{The reasoning is from me. Is it true? Or could we even say non-divergent solutions?}\ttuwe{too compact for my taste, too difficult to reproduce}
For our purpose, we assume $\nu_+>0, \, \mu_+<0$, i.e.~the Turing mode is linearly weakly damped. Using $1/(k_+^2 |\mu_+|)$ as the time scale, $|\mu_+|k_+^2/(\nu k_o^2)$ as the scale of $\rho_+$, and $|\mu_+|k_+/(\sqrt{\nu \nu_+}k_o)$ as a scale of $\rho_{1/2}$ reduces (\ref{9ampl}) to the polar form containing just the two parameters $\mu = \mu_o k_o^2/(|\mu_+| k_+^2)$ and $\varphi$:
 %
\begin{eqnarray} 
\dot{\rho}_+ & = & -\rho_+ + \rho_1 \rho_2 \cos \Theta, \nonumber\\
\dot{\rho}_1 &=& \mu \rho_1 +  
     \rho_2 \rho_+ \,\cos (\Theta - \varphi), \quad 
\dot{\rho}_2 = \mu \rho_2 +  
     \rho_1 \rho_+ \,\cos (\Theta + \varphi), \label{1amplreal}    \\ 
\dot{\Theta} & = & - \frac{\rho_1 \rho_2}{\rho_+} \sin \Theta
  -  \rho_+ \left[ \frac{\rho_1 }{\rho_2} \sin (\Theta + \varphi) + 
     \frac{\rho_2}{\rho_1} \sin (\Theta - \varphi) \right]. \label{1amplim}
\end{eqnarray} 
%
 %\ttlen{Eqs.~\eqref{1amplreal}-\eqref{1amplim} obey the symmetry $\left(\varphi,\rho_+, \rho_1, \rho_2\right) \to \left(-\varphi,\rho_+, -\rho_1, -\rho_2\right)$.
% removed: amplitudes are positive!}
The stationary values of the amplitudes $\rho_+,\rho_{1,2}$ obtained by resolving (\ref{1amplreal}) are
%
\begin{equation}
{\rho}_+ = \frac{|\mu|}{[\cos(\Theta-\varphi)\cos(\Theta+\varphi)]^{1/2}}, \quad
{\rho}_{1,2} = \left[-\frac{\mu}{\cos(\Theta \pm\varphi)\cos \Theta}\right]^{1/2}. 
\label{1statrho}    \end{equation}
%
Plugging (\ref{1statrho}) in (\ref{1amplim}) brings the equation defining stationary values of $\Theta$ to the form 
%
\begin{equation} 
- \tan \Theta+\mu[ \tan (\Theta-\varphi)+ \tan (\Theta+\varphi)]= 0.
      \label{1eqth}  \end{equation} 
%
A trivial solution to \eqref{1eqth} is $\Theta =0$ which defines the symmetric stationary solution with $\rho_+ = \rho_1^2 = \rho_2^2 = |\mu|/\cos \varphi$.

An asymmetric stationary solution is obtained when \eqref{1eqth} is converted through a long chain of trigonometric transformation to a transparent implicit relation
%
\begin{equation} 
 (1-2\mu)\cos^2 \Theta= \sin^2\varphi.
      \label{1eqtha}  \end{equation} 
%
    Since $(1-2\mu)\cos^2 \Theta \leq 1- 2 \mu$ and
    % \tobias{why $\mu>0$???} \ttlen{because $ \cos^2\varphi>0$; try $\mu>0$ and see the same}
    the asymmetric solution is confined to the interval $0\leq \mu\leq\frac 12 \cos^2\varphi$. The existence limits correspond to the bifurcation from the trivial state at $\mu=0$ and the pitchfork bifurcation from the symmetric solution at $\mu=\frac 12 \cos^2\varphi$, respectively. An additional restriction comes from the requirement for the amplitudes given by \eqref{1statrho} to be real and positive. This requires $|\Theta\pm\varphi|<\pi/2$, combining to $|\varphi|>\pi/2$. Branches of solutions do not diverge if $|\Theta|<\varphi-\pi/2$. The acceptable interval of both angles, as well as the sign of $\mu$ would overturn if we had chosen $ \mu_+>0$
    % \tobias{I do not see that easily - What does it mean that $\mu$ would overturn? First, $\mu$ only depends on $|\mu_+|$.  For $\mu_+>0$ Eqs~(27)-(28) stay the same except the linear growth term in $\rho_+$ becomes positive, i.e. $\partial_t \rho_+ = \rho_+ + \rho_1 \rho_2 \cos \Theta$. I think what we mean is that  the existence interval for asymmetric stationary state is overturned i.e. $-\frac 12 \cos^2\varphi\leq \mu\leq 0$. Correct?}. \ttlen{remove this sentence}
% UWE GETTING DESPERATE: WHICH SENTENCE SHALL WE REMOVE???

%\begin{figure}[t]
%\centering
%\begin{tabular}{cc}
 %(a) & (b) \\
%  %\includegraphics[width=0.45\hsize]{f1wta}\hspace{0.025\hsize} &
%  %\includegraphics[width=0.475\hsize]{f1wtb}
  %\end{tabular}
%\caption{Weakly nonlinear description of the resonance. (a) The branch of stationary solutions at $ \varphi=\frac 34 \pi$ as a function of $\mu$. The inset shows in blue shading the domain of oscillatory instability. (b) Oscillations at $\mu=0.1, \,\varphi=\frac 34 \pi$. In both panels, the curves for $\rho_+,  \rho_1, \rho_2, \Theta$ are shown in blue, orange, green, and red, respectively. \ttuwe{lettering needs to be larger.}\tobias{point $\alpha=3\pi/4$, $\mu=0.1$ is not within blue shaded region - how do we obtain the blue region?}}.  
%  \label{f1wt} \end{figure}
% LEN SEEMS TO HAVE REMOVED THIS
    
\begin{figure}[t]
\centering
   \includegraphics[width=0.8\hsize]{WNA_plots}~\\
  \includegraphics[width=0.22\hsize]{osc_mu0.26_phi0.6}\hspace{0.025\hsize} 
  \includegraphics[width=0.22\hsize]{osc_mu0.08_phi0.68}\hspace{0.025\hsize} 
  \includegraphics[width=0.22\hsize]{osc_mu0.1_phi0.75}\hspace{0.025\hsize} 
   \includegraphics[width=0.22\hsize]{osc_mu0.1_phi0.9}
 \caption{Results of the weakly nonlinear analysis are presented: (a) Branches of stationary solutions at $ \varphi= 0.68 \pi$ as a function of $\mu$. (b) Phase diagram where the solid and dashed lines indicate pitchfork and Hopf bifurcations, respectively. The regions of prevailing stationary linearly stable symmetric and asymmetric states are indicated via the letters ``S'' and ``A'', respectively. (c)-(f) Amplitudes as a function of time for different types of  oscillatory long-time behavior. The parameter are (c) $\mu=0.26, \,\varphi=0.6 \pi$, (d) $\mu=0.08, \,\varphi=0.68 \pi$, (e) $\mu=0.1, \,\varphi=0.75 \pi$ and (f) $\mu=0.1, \,\varphi=0.9 \pi$, and  are indicated by cross symbols in panel~(b). In (a,c-f) the curves for $\rho_+,  \rho_1, \rho_2$, and $\Theta$ are shown as blue, orange, green, and red lines, respectively.}  
  \label{fig:f1wt} \end{figure}
    
Fig.~\ref{fig:f1wt}~(a) presents a bifurcation diagram showing the stationary solution branches given by ~\eqref{1statrho} for fixed $\varphi=0.68 \pi$ where the colored lines give the three different amplitudes and the phase as described in the caption. A linear stability analysis of the symmetric state gives $\left[\frac12 \cos^2 \varphi, \frac14\right]$ as the $\mu$-range of linear stability limited by the aforementioned pitchfork bifurcation on the left (circles symbol) and a Hopf bifurcation (diamond symbols) on the right hand border. For the asymmetric state, the determinant of the Jacobi matrix of \eqref{1amplim} is computed as
%
\begin{equation} 
 4\mu^2 \left\{ \mu[\sec^2(\Theta-\varphi)  +\sec^2(\Theta+\varphi) ]- \sec^2 \Theta \right\}.
       \label{1stabj}  \end{equation} 
%
The expression in brackets is transformed with the help of \eqref{1eqtha} to
%
\begin{equation} 
 \sec(\Theta-\varphi)  +\sec(\Theta+\varphi) \sin^2\varphi \tan^2 \Theta,
       \label{1stabj1}  \end{equation} 
%
which is evidently positive, i.e.~indicates stability with regard to monotonic perturbations, whenever the asymmetric stationary solution exists.

The locus of the Hopf bifurcation in the $(\varphi, \mu)$ plane is given implicitly by the relation %\tobias{Is this the trace of the Jacobian of the amplitude equations? I do not understand the reasoning for (in)stability here}
%
\begin{align} 
&( 4\mu-1)(1 - 5\mu + 8\mu^2) \sin^4 \varphi + 3(4 - 7\mu + 4\mu^2)( \cos^2 \varphi-2\mu)^2 \cr 
&+[3 + 8\mu(\mu - 2)(1 - 2\mu)]( \cos^2 \varphi-2\mu) \sin^2 \varphi = 0.  
     \label{1stabj2}  \end{align} 
%
   Take note that this is a \emph{secondary} supercritical %\tobias{How do we know?}
   bifurcation on top of the Hopf bifurcation creating the linearly growing wave modes involved in this planform.  
The discussed existence and linear stability in the $(\varphi,\mu)$-plane are summarized in Fig.~\ref{fig:f1wt}~(b). The pitchfork bifurcation indicated by the circle symbols in panel~(a) is given as the solid line that separates the regions ``A'' and ``S'' where the asymmetric and symmetric stationary solutions are linearly stable, respectively. The stability regions are further limited by the dashed lines that mark the loci of the Hopf bifurcations given by diamond symbols in (a). Examples of corresponding periodic orbits obtained in the description of the amplitude equations are shown in Figs.~\ref{fig:f1wt}~(c) to~(f) in the sequence of increasing $\phi$; their loci in the $(\varphi,\mu)$-plane are marked by crosses in Fig.~\ref{fig:f1wt}~(b).
%
\section{Specific modified FitzHugh--Nagumo system}%\label{sec:FHN}
\label{sec:resonance-specific}
%
Next we aim at identifying resonant behavior in the fully nonlinear regime. To do so we have to focus on a specific nonreciprocal CH system. Here, we employ a simple representative example obtained by choosing $f$ and $g$ in Eqs.~\eqref{91uwas} to be of third order in intraspecies interactions and linear in interspecies interaction. In particular, we use $f(u,v)=u-u^3-v$ and $g(u,v)=\alpha u-\beta v - v^3$. Correspondingly, $\chi(u,v) = -u^2/2+u^4/4+\beta v^2/(2\gamma) + v^4/(4\gamma) +(1-\alpha/\gamma)uv/2$ in \eqref{eq:energy} as well as $\mu_{u}^\mathrm{nv}=(1+\alpha/\gamma)v/2$ and $\mu_{v}^\mathrm{nv}=-(1+ \alpha/\gamma)u/2$. Both species have in general a nonzero mean density, i.e. $\int u~{\rm d}x=u_s $ and $\int v~{\rm d}x= v_s$ that act as an effective quadratic nonlinearity in $f$ and $g$, respectively.\footnote{In other words, if we introduce the shifted densities $u-u_s$ and $v-v_s$, the resulting nonlinear terms in the shifted densities include  quadratic nonlinearities.}

In the absence of the cubic nonlinearity in $g$,  our example represents a fully mass-conserving version of the standard FitzHugh--Nagumo model. It represents a simple example of \eqref{91uwas}, however, as explained below, does not feature the secondary Hopf instability given by Eq.~\eqref{1stabj2} and the corresponding oscillatory behavior. %indicated in Fig.~\ref{f1wt} is absent.
Therefore, we include the cubic nonlinearity in $g$ and obtain a conserved modified FitzHugh--Nagumo system that is identical to the recently considered nonreciprocal Cahn-Hilliard model.

For the homogeneous state $(u,v)=(u_s,0)$ with $u_s^2<1/\sqrt 3$ we have $f_u=1-3u_s^2,\, f_v= -1, \, g_v=-(\beta +3 v_s^2), \, g_u=\alpha, \, A=\alpha - (1-3 u_s^2) (\beta +3v_s^2), \, B=(1-3u_s^2)\sigma-(\beta + 3v_s^2)$. Imposing $3 u_s^2<1$ [$\beta+3v_s^2>0$] we choose $u$ [not $v$] to be autocatalytic ($f_u>0$) [($f_v<0$)]. If the coupling is nonreciprocal ($f_v g_u<0$), i.e.~for $\alpha>0$, the necessary conditions for the instabilities are
 \begin{align*}
\text{Cahn-Hilliard:~~}&A<0 \Rightarrow \alpha<(1-3u_s^2)(\beta + 3v_s^2)~\\
\text{Matthews-Cox:~~}&B>0 \, \, \wedge \, \, \sigma>1 \, \, \wedge \, \, \sigma f_u > g_v  + 2 \sqrt{-\sigma f_v g_u} ~\\
& \Rightarrow \sigma(1-3u_s^2-3v_s^2) >\beta>-\sigma(1-3u_s^2-3v_s^2) + 2 \sqrt{\sigma \alpha} \, \, \wedge \, \, \sigma>1 ~\\
\text{and conserved Hopf:~~}& A>\frac{1}{4} \left(f_u+  g_v\right)^2 \, \, \wedge \, \, f_u + g_v>0 ~\\
&\Rightarrow \beta< \text{min}\left\{-1 +3u_s^2 -3v_s^2 + 2\sqrt{\alpha},1 - 3u_s^2 -3v_s^2\right\} \, \, \wedge \, \, \alpha> \left(\frac{1-3u_s^2}{2}\right)^2.
 \end{align*}
%\tobias{Phase separation means crossing green line anywhere, dotted is okay, right?}
% \tobias{former conditions (I think Lens version): $\alpha<\beta, \,\kappa <\beta, \, \gamma<\beta$, respectively. 1st one identical, 2nd one wrong direction (< instead of >) and missing $\sigma>1$ and onset condition, 3rd refers only to the onset condition but then with wrong exponent for $\gamma$ and neglects necessary conditions for complex EV. Agree? Or do we want to give only some conditions for some reasons I don't see?}
%
The wavenumbers of monotonic and oscillatory marginal modes [cf.~\eqref{eq:kpm} and~\eqref{eq:ko}] are then given by
\begin{align}
k^2_\pm = & k_T^2 \left[1 \pm \sqrt{1- \frac{4 \sigma \left(\alpha-(1-3u_s^2)(\beta +3v_s^2) \right)}{\left((1-3u_s^2)\sigma-(\beta +3v_s^2)\right)^2}}\right]~\label{eq:kpm_FHN}\\
\text{and}\qquad k_\text{o}^2 = & \frac{1-3u_s^2-(\beta +3v_s^2) }{1+  \sigma}\,. \label{eq:ko_FHN}
\end{align}
where $k_T^2=\frac{(1-3u_s^2)\sigma-(\beta+3v_s^2)}{2\sigma}$ is the critical wavenumber at the onset of the Matthews-Cox instability.

We now consider a scenario where a marginal conserved Hopf mode ($k_o=k_L$) and a marginal Matthews-Cox mode ($k_\pm=k_{L/2}$) are resonant in a one-dimensional domain. For the considered specific
system this is achieved with $4 k_\text{o}^2 = k_+^2 = k_{L/2}^2 $. Using Eqs.~\eqref{eq:kpm_FHN} and \eqref{eq:ko_FHN} gives after simplification
%\begin{equation}
% 4 \frac{1-3u_s^2-(\beta +3v_s^2) }{1+  \sigma}=\frac{(1-3u_s^2)\sigma-(\beta+ 3v_s^2)}{2\sigma} \left[1 + \sqrt{1- \frac{4 \sigma \left(\alpha-(1-3u_s^2)(\beta+3v_s^2) \right)}{\left((1-3u_s^2)\sigma-(\beta+3v_s^2)\right)^2}}\right]=\frac{4 \pi^2}{L^2}
%  \label{eq:res1d}
%\end{equation}
%that can be reduced (in some parameter regions) to
the critical values
%
\begin{equation}\label{eq:crit_values}
\alpha_c =\left(1 -3 u_s^2-\frac{16 \pi^2}{L^2} \right) \left(1 - 3 u_s^2 - 
   \frac{4 \pi^2}{L^2} (1 - 3 \sigma)\right) ~\qquad \beta_c = 1 -  3 (u_s^2 +v_s^2) - \frac{4 \pi^2 }{L^2}(1 + \sigma)
\end{equation}
that define this codimension-2 point. The corresponding frequency of the marginal conserved Hopf mode is
\begin{equation}
\omega_o=\frac{8 \sqrt{3} \pi ^3 \sqrt{L^2 (\sigma -1) \left(3 u_s^2-1\right)+4 \pi ^2 (4 \sigma -1)}}{L^4}.
\end{equation}
%\ttuwe{This differs from Len's condition
%  \begin{equation}
%\frac{ 1-\beta\gamma }{2\sqrt{2}(1+\kappa\gamma)}=\frac{3}{8\kappa}\left[(\kappa -\beta)\left(1-
% \sqrt{1 + \frac{32}{9} \frac{\kappa(\beta-\alpha)}{(\kappa -\beta)^2}}\right)\right].
%% \label{91bif2}  \end{equation}
%as we get a different prefactor on the left and use the root of the dispersion relation instead of the maximum on the right. Len, pls double check as we have a different formulae here.}
To consider the weakly nonlinear regime in the vicinity of the codimension-2 point we set $\alpha=\alpha_c$ and $\beta=\beta_c + \varepsilon$.
The resulting relevant nonzero entries of $\frac{\partial \tens{L}(k^2)}{\partial \beta}$ and the Hessian $\tens{\tens{H}}$ in Eqs.~\eqref{eq:AE} are
\begin{equation}
\frac{\partial L_{22}(k^2)}{\partial \beta} = -k^2 \qquad
H_{111}(k^2) = k^2  f_{uu} = -k^2  6u_s \qquad
H_{222}(k^2) = k^2  g_{vv} = -k^2  6v_s\,.
\end{equation}
Note that for the special case of a trivial homogeneous state, i.e.~$u_s=v_s=0$, quadratic interactions are absent and the description of resonances via the leading order amplitude equation~\eqref{eq:AE} does not apply.

Incorporating the imaginary part of $\mu_o$ into the frequency, the coefficients in Eqs.~\eqref{9ampl} become\footnote{To compare the specific model to  the weakly nonlinear results in Fig.~\ref{fig:f1wt} [from Eqs.~\eqref{1amplreal} and \eqref{1amplim}] we further obtain $\mu = \mu_o k_o^2/(|\mu_+| k_+^2)$, and $\phi$ as the phase of the complex parameter $\nu_o$. As there is an additional free parameter in the original model we have further fixed the absolute value of $\nu_o$ to one.}
\begin{align}
\mu_+ =&\varepsilon  \frac{\left(L^2 \left(1-3 u_s^2\right)-16 \pi ^2\right)}{12 \pi ^2 (\sigma +1)} \quad \mu_o = -\frac{\varepsilon}{2}
~\\
 \nu_+ =& \frac{L^2 (v_s-u_s (3 u_s v_s+1))-16 \pi ^2 v_s}{2 \pi ^2 (\sigma +1)}  ~\\
 \nu_o =&\frac{3 L^2 u_s}{L^2 \left(3 u_s^2-1\right)+16 \pi ^2}-3 v_s+ \I \frac{12 \pi ^2 \left(L^2 \left(3 u_s^2-1\right)+4 \pi ^2\right) \left(L^2 \left(3 u_s^2 v_s+u_s-v_s\right)+16 \pi ^2 v_s\right)}{\left(L^2 \left(3 u_s^2-1\right)+16 \pi ^2\right) L^4 \omega _o}\,.
\end{align}
%\ttuwe{this somehow ends in 'mid-air': To compare with the weakly nonlinear figure we would need to relate (37-39) to the values on the axis of fig.3b.. $\mu = \mu_o k_o^2/(|\mu_+| k_+^2)$ seems possible but $\varphi$ is not explicitly given, right? Also the promise that we say why the cubic term is needed we do not keep as it is not clear which terms would go away without the cubic and how this affects the comparison with Fig.3.}

%Choosing $L=10 \pi$ and incorporating the imaginary part of $\mu_o$ into the frequency the coefficients in Eqs.~\eqref{9ampl} become
%\begin{align}
%k_+ =& 2k_o = \frac{2}{5}\quad \mu_+ =\frac{7 - 25 u_s^2}{1 + \sigma} \varepsilon \quad \mu_o = -\frac{\varepsilon}{2}
%~\\
% \nu_+ =& -\frac{50 u_s}{1+\sigma}  \quad \nu_1 =\frac{25 u_s}{25 u_s^2 -7}+ \I \frac{25 u_s \left( 25 u_s^2 -8\right)}{\left(25 u_s^2 -7\right)\sqrt{7 \sigma - 25 u_s^2 \left(\sigma-1\right) -8}}
%\end{align}
%As in the previous section we choose $\mu_o$ positive and $\mu_+$ negative, i.e. we decrease $\beta$ via $\varepsilon<0$ and find the condition $L>L_\text{min}=\frac{4\pi}{\sqrt{1-3u_s^2}}$. For some specific parameters we find two-frequency oscillation as predicted based on the weakly nonlinear analysis~(cf.~\ref{fig:oscillation}). 

\begin{figure}[t]
\centering
\includegraphics[width=0.4\hsize]{./modes_interaction_2a}
\includegraphics[width=0.55\hsize]{waterfall}
\caption{(a) Dispersion relation $\mathrm{Re}\,\lambda(k)$ [Eq.~\eqref{eq:lambda}] for a finite system size $L$ at parameter values where one has resonant marginal modes $k_\text{o}=k_L$ [Eq.~\eqref{eq:ko}] and $k_+=k_{L/2}$ [Eq.~\eqref{eq:kpm}] of the conserved Hopf and Matthews-Cox instabilities, respectively. (b) Space-time plot of $u(x,t)$ obtained by fully nonlinear time simulation of  the mass-conserving modified FitzHugh–Nagumo system. It shows two-frequency oscillatory behavior. The parameters are $\sigma \approx 1.19557>1$, $f_u[=1-3u_s^2]\approx 0.81867$, $f_v=-1$, $g_u\left[=\alpha\right]\approx  0.60739$ and $g_v\left[=-\beta-3v_s^2\right]\approx -0.73075$ where for the specific model $\beta = \beta_c + \varepsilon \approx 0.15016$ with $\varepsilon=-10^{-4}$, $u_s \approx 0.24585$ and $v_s \approx 0.43992$. The domain size is $L=10 \pi$. In the formulation of the amplitude equations the corresponding parameters are $\mu= 0.05$, and $\varphi = -0.65 \pi$.}\label{fig:oscillation}
\end{figure}
% and $\nu= 1$

%\begin{figure}[t]
%\centering
%\includegraphics[width=0.4\hsize]{./modes_interaction_2a_version2}
%\includegraphics[width=0.55\hsize]{waterfall_version2}
%\caption{\tobias{Fig4 Alternative, corresponds to a cross in Fig 3b and panel (d)}(a) Dispersion relation $\mathrm{Re}\,\lambda(k)$ [Eq.~\eqref{eq:lambda}] at parameter values where for a finite system of size $L$ the marginal modes $k_\text{o}=k_L$ [Eq.~\eqref{eq:ko}] and $k_+=k_{L/2}$ [Eq.~\eqref{eq:kpm}]  of the conserved Hopf and Matthews-Cox instabilities, respectively, are at resonance. The parameters are $\sigma \approx 1.28774>1$, $f_u[=1-3u_s^2]\approx 0.58895$, $f_v=-1$, $g_u\left[=\alpha\right]\approx  0.30176$ and $g_v\left[=-\beta-3v_s^2\right]\approx -0.49754$. (b) Space-time plot of $u(x,t)$ obtained by fully nonlinear time simulation. It shows two-frequency oscillatory behavior. The remaining parameters of the specific model are
%$\beta = \beta_c + \varepsilon \approx -2.32284$ with $\varepsilon=-10^{-4}$, $u_s \approx 0.37016$ and $v_s \approx 0.96960$. The domain size is $L=10 \pi$. In the formulation of the amplitude equations it corresponds to the parameters: $\mu= 0.08$, $\varphi = -0.68 \pi$ and $\nu= 1$.}\label{fig:oscillation}
%\end{figure}

We have performed direct time simulations of this model in the vicinity of the degenerate bifurcation where we expect resonance behavior. In particular, we choose $\mu_o>0$ and $\mu_+<0$ as in section~\ref{sec:resonance}, i.e.\ we decrease $\beta$ via $\varepsilon<0$. A typical result is given in
%and find the condition $L>L_\text{min}=\frac{4\pi}{\sqrt{1-3u_s^2}}$. UWE: CONDITION FOR WHAT
Fig.~\ref{fig:oscillation} where panel~(a) gives the dispersion relation at parameter values where $k_\text{o}=k_L$ and $k_+=k_{L/2}$ are resonant while panel~(b) shows a space-time plot of the corresponding time simulation. The latter indeed shows a two-frequency behavior analogous to the secondary oscillations found with the weakly nonlinear approach in section \ref{sec:resonance}.

However, performing time simulations at parameters of that correspond to regions ``A'' and ``S'' in Fig.~\ref{fig:f1wt}~(b) where asymmetric and symmetric stationary solutions of the amplitude equations \eqref{eq:AE} are respectively stable, we only encounter traveling waves (TW), standing waves (SW) (both with wavenumber $k_L$), or stationary Turing patterns (ST) with wavenumber $k_{L/2}$. Due to the employed ansatz these states are not captured by the amplitude equations.
% As such states also result outside these parameter regions, i.e. they do not emerge at the Hopf-boundaries [dashed lines in Fig 3b].
%This disagreement results from the limited ansatz \eqref{eq:AE} that does not capture the observed TW, SW and ST states.

%Some more info: 
%\begin{itemize}
%\item What happens for the FHN model, i.e. for $v_s=0$?: First, there is only one nonzero entry in the Hessian that defines the quadratic leading order interaction. Second, we do not find secondary oscillations, because this behavior is suppressed by other solution: We have three free parameters  $L$, $\sigma$ and $u_s$ to adjust the parameters $\mu$ and $\varphi$. We are looking for the regime where asymmetric stationary states exist which is defined by $|\varphi|>\pi/2$ and $0<\mu< \frac12 \cos^2 \varphi $. We can find this regime. However another requirement is that the wave ($k_L$) and Turing ($k_{L/2}$) are the marginal modes and any other mode is linearly damped. This is never the case. Instead we always find that the mode $k_{L/3}$ has a positive growth rate.
%\tobias{see mathematica script WNA\_nonCH\_short in Owncloud folder} Consequently, time simulations show a stationary pattern with wavenumber $k_{L/3}$.
%Our description of resonance is $2 k_o = k_+$ or $2 k_o = k_-$. I think Len would prefer to use $2 k_o = k_T$ with $\lambda(k_T)=0$. Within this formulation one could say that it is not possible to meet the resonance condition ``$2 k_o = k_T$ with $\lambda(k_T)=0$'' for  $|\varphi|>\pi/2$ and $0<\mu< \frac12 \cos^2 \varphi $. Btw I use the former resonance condition in mathematica because it is easier to do the calculations/reduce the conditions. }
%And it is less strict, we only have one equality [together with the inequality $\lambda(k_{L/3}<0$] while the latter one has two equalities. The reason for this difference is that the former one is good enough for finite system, while for infinite systems the latter one has to be fulfilled - maybe we can include that in the discussion of a finite system in section~\ref{sec:linear}?}
%\item Simulations for nonreciprocal CH Equation, i.e. $u_s\neq 0$, $v_s\neq 0$:
%Simulations within the regions ``A'' and ``S'' in Fig 3b where asymmetric and symmetric stationary solution are stable we only find either traveling waves (TW) or standing waves (SW) [both with wavenumber $k_L$] or stationary Turing pattern (ST) with wavenumber $k_{L/2}$. Note that we can also find these states outside of these parameter regions, i.e. they do not emerge at the Hopf-boundaries [dashed lines in Fig 3b]. Reason: The three observed states [TW, SW, ST] are not captured by the leading-order amplitude equations~\eqref{eq:AE}. Their description requires a stabilizing third order term, since the quadratic interaction drops out [since e.g. $\rho_1=\rho_2=0$ for ST]. However, these states do also emerge from the homogeneous branch at $\mu=0$. A leading-order description that considers all states that emerge from $\mu=0$ has to have both quadratic and cubic nonlinearities, this could e.g. be achieved for small quadratic interaction parameters. The whole situation reminds one of hexagons [=resonance states] vs.~stripes[=TW,SW,ST states].
%Neglecting TW, SW, ST states, i.e. using amplitude equation~\eqref{eq:AE}, one can not make any statements about the relative stability between resonance states and TW, SW, ST. Based on the numerical findings we predict that the symmetric and asymmetric stationary states are always unstable w.r.t.~ either SW, TW or ST, i.e. the linear stability found in the formulation of \eqref{eq:AE} is not valid.


%As in the previous section we choose $\mu_+$ to be negative, i.e.decreasing $\beta$ via $\varepsilon<0$. Translating into polar representation and rescaling time and amplitudes the final set of equations are Eqs.~\eqref{1amplreal}-\eqref{1amplim} where we choose $\varphi =0.65 \pi $ and $\mu = 0.1$ that determines the mean density $u_s \approx 0.274$ and $\sigma \approx 3.094$. Then according to Eqs.~\eqref{eq:crit_values} $\alpha_c \approx 0.340$ and $\beta_c \approx 0.479$. Furthermore we choose $\varepsilon = -10^{-4}$ which guarantees that the leading order amplitude equation is valid.
%The stationary values are



\section{Summary and outlook}
\label{sec:conc}

We have considered a nonreciprocically coupled two-field Cahn-Hilliard system that is known to allow for oscillatory behavior and a suppression of coarsening.  Here, we have reviewed the linear stability analysis of homogeneous states and have shown that for general intra- and interspecies interaction terms all instability thresholds of the fully mass-conserving Cahn-Hilliard system are identical to the ones for the corresponding (nonmass-conserving) reaction-diffusion system. Next, we have briefly highlighted the differences in the linear behavior of conserved and nonconserved model that then all occur beyond the instability onset. Focusing on the codimension-2 point where conserved Hopf and Matthew-Cox instabilities simultaneously occur, we have discussed possible interactions of linear modes. In particular, we have analyzed the specific case of a ``Hopf-Turing'' resonance. To do so we have first employed a weakly nonlinear approach to consider the amplitude equations close to the codimension-2 point. After discussing their solution behavior in the general case we have derived the coefficients of the amplitude equations for a specific nonreciprocical Cahn-Hilliard model (that corresponds to a conserved amended FitzHugh-Nagumo model).\footnote{Although a conserved version of the standard FitzHugh-Nagumo model shows the codimension-2 point it does allow to adjust parameters in such a way that the parameter ranges of the weakly nonlinear model corresponds to the interesting one that is shown in Fig.~\ref{fig:f1wt}~(b).} Finally, we have shown that fully nonlinear time simulations indeed show two-frequency behavior analogous to the secondary oscillations discussed in the framework of the weakly nonlinear theory.

However, we have also noted that not all behavior predicted by the amplitude equation is found in the fully nonlinear calculation. To better understand where weakly and fully nonlinear results agree and where they disagree the mapping of the respective parameter sets and resulting behavior should in the future be further scrutinized. A problem that needs further attention is that the parameter mapping is not one-to-one and itself highly nonlinear. This makes it, for instance, quite difficult to identify parameter ranges where certain states dominate for the nonlinear model with corresponding ranges in the weakly nonlinear description. The usage of continuation methods might allow to obtain bifurcation diagrams for the nonlinear model that could then be directly compared to the bifurcation diagram presented here in the weakly nonlinear case.

%compare the analytical bifurcation diagram with a diagram obtained by the full system. We expect new instabilities since beside the given branches in Fig.~\ref{f1wt}a) additional branches for standing wave, traveling wave and stationary Turing pattern emerge at $\mu=0$, too. Problem: the mapping of the parameters, we cannot simply change $\mu$ in terms of the original parameters while keeping the other parameters fixed. 

%This result is difficult to reproduce in the framework of weakly nonlinear analysis in view of the strong influence of damping due to cubic terms. \ttlen{adding more cubic terms was a mistake`} 



% \appendix

% \section{Linear stability for $\sigma<1$}

% \begin{figure}
% \includegraphics[width=0.45\textwidth]{./modes_interaction_2b}
% \includegraphics[width=0.45\textwidth]{./modes_interaction_2c}
% \caption{(a) Dispersion relation $\mathrm{Re}\,\lambda(k)$ [Eq.~\eqref{eq:lambda}] with indication of a possible mode interaction in the nonlinear regime found for $\sigma=0.6<1$ that is alternative to the one discussed in sections~\ref{} and~\ref{}. The band of unstable wavenumbers above onset of the conserved Hopf instability may contain both, monotonic modes at small wavenumbers and oscillatory modes at larger wavenumbers. At the chosen parameter values of $g_v\left[=-\beta\right]=-0.18$ and $f_u=1$ the modes at $k=k_{L/2}$ and $k=k_{L/4}$ have approximately the same growth rate, i.e., a nonlinear time simulation might show mode interaction. \ttuwe{should be checked}. The wavenumbers $k_{L/n}$ are given for domain size $L=40$. (b) Linear stability diagram similar to Fig.~\ref{fig:stab-diagram} but for the case $\sigma<1$. The cross symbol indicates parameters of panel~(a). The remaining parameters are $f_v=-1$ and $g_u\left[=\alpha\right]=0.3$. Parameters in brackets indicate parameters for the specific nonreciprocal Cahn-Hilliard example of Section~\ref{sec:FHN}\tobias{still need to be adapted to the nonreciprocal CH example}. \ttuwe{in (b) remove orange dotted line}}
% \label{fig:dispersion-relations-app} 
% \end{figure}




% 
%\begin{figure}
%%\includegraphics[width=0.7\textwidth]{{./Fig3}.pdf}~\\[-3ex]
%%%\includegraphics[width=0.45\textwidth]{{./f1f2_instab_kappa3.82}.pdf}
%%\hfill
%%%\includegraphics[width=0.45\textwidth]{./f1f2_instab_delta_neg}
% \caption{  Different Dispersion relations related to the Videos. 
% }
%\label{fig:dispersion-relation_videos} 
%\end{figure}
%\ttuwe{To get an overview of the potentially interesting behavior there are some videos in the Dropbox (Folder ``Videos''. The ones named 3a to 3f correspond to dispersion relations given in Fig.~3. Len, please have a look and tell us what you find interesting.\\
%%
%Short description of the videos related to the corresponding panels in Fig.~3:
%\begin{itemize}
%\item[a] After lengthy transient, a simple traveling wave with $k=k_L$ wins
%\item[b] Persistent mode interaction where oscillatory mode at $k_L$ has very small but positive growth rate (do not miss the small oscillations!)
%\item[c] After transient involving oscillating 4-peak state, ultimately traveling wave with $k=k_L$ wins
%\item[d] Same parameters as in c), but larger domain: Modulated traveling wave with $k=k_{L/2}$ wins [the oscillatory $k_{L/2}$ mode also has the maximal growth rate]
%\item[e] Large domain: Traveling wave with $k=k_{L/8}$ wins [does not corresponds to mode with maximal growth rate], no coarsening despite real eigenvalue band
%\item[f] Only real positive EVs. Nevertheless traveling wave with $k=k_{L/3}$ wins, i.e. suppression of coarsening (similar to what we have found in \cite{FrWT2021pre}). 
%\end{itemize}
%}
%
%\bftobias{
%Figs.~\ref{fig:TW}-\ref{fig:modTW} show some quantities over time.
%Panels~(a)-(b) show norms. Panels~(c)-(h) show absolute value squared of fourier coefficients.
%For this the fields are expanded in a fourier series, i.e.
%\begin{equation}
%u(x,t) = \sum_n \tilde u_n(t) e^{\text{i} k_{L/n} x}
%\end{equation}
%Considering $\tilde u_n \tilde u_{-n} = |\tilde u_n|^2 $ we can distinguish between different solution types.
%We assume $u_n$ consists of three contributions 
%\begin{equation}
%\tilde u_n =  u_S e^{\text{i} \varphi_S} +  u_L e^{\text{i} \varphi_L + \text{i} \omega t} +  u_R e^{\text{i} \varphi_R - \text{i} \omega t}
%\end{equation}
%where $u_S$, $u_R$ and $u_L$ are the real amplitudes of the stationary pattern and the left and right traveling waves, respectively.
%Then
%\begin{align}
%\tilde u_n \tilde u_{-n} = & u_S^2 + u_L^2 +  u_R^2 ~\nonumber~\\
%+ & 2  u_L  u_R\cos\left(2 \omega t + \varphi_L - \varphi_R \right) + 2 u_S u_L \cos\left(\omega t + \varphi_L - \varphi_S\right) +2 u_S u_R \cos\left(\omega t + \varphi_R - \varphi_S\right)\label{eq:un}
%\end{align}
%Based on Eq.~\eqref{eq:un} we can classify different patterns:
%\begin{itemize}
%\item For a stationary pattern ($u_L=u_R=0$), it is $|\tilde u_n|^2=u_S^2$.
%\item For a purely right traveling wave (i.e. $u_S= u_L=0$): $|\tilde u_n|^2=  u_R^2$
%\item For a standing wave (i.e. $u_S=0$, $u_L=u_R$): $|\tilde u_n|^2=  2 u_R^2 (1 + \cos\left(2 \omega t + \varphi_L - \varphi_R \right)) $
%\item For swinging mode, i.e. $u_S \neq 0$, $u_L=u_R$
%\item modulated right traveling wave, i.e.~$u_S\ne 0$, $u_R>u_L$
%%\tobias{I'm not sure how to distinguish between $u_S=0$ and $u_S\neq 0$ if $u_R \neq u_L$...}
%\end{itemize}
%Fig.~\ref{fig:TW} shows a traveling wave with dominant wavenumber $k_L$, since $|\tilde u_n|^2$ is constant for all $n$.~\\
%Fig.~\ref{fig:swinging_mode} shows the swinging mode (see also beta0,47.mp4).: The Turing mode $|\tilde u_2|^2$ is large compared to the other amplitudes of the wave mode and shows small oscillations. It is superposed with a standing wave  of wavenumber $k_L$. I.e. in contrast to the amplitude equation where the swinging mode corresponds to a constant amplitude of the Turing mode we see that it is not constant in the fully nonlinear regime. This is due to higher order effects that describe the coupling of the standing wave in $\tilde u_1$ to the behavior of $\tilde u_2$. I.e. the constant amplitude derived from the amplitude equation is not a symmetry/property of the whole branch of the swinging mode, it is just a leading order effect. However it is still true that $u_L=u_R$ since in panel~(d) the minima of the oscillation touches zero.
%In contrast to Fig.~\ref{fig:modTW} where we see an offset, i.e. $u_R>u_L$ and thus the solution begins to move. This modulated wave corresponds to a solution as in beta0,46.mp4. 
%I suppose that there is no secondary Hopf bifurcation despite the calculation based on the leading order amplitude equation. Since the swinging mode does already consist of an oscillation in the Turing mode the fully nonlinear regime shows a (drift)pitchfork bifurcation instead that breaks the $u_R=u_L$ symmetry. The corresponding branch of modulated traveling waves terminates in a secondary bifurcation on the traveling wave branch. The swinging mode connects the stationary Turing pattern with the purely standing wave. The described bifurcation scenario is illustrated in the sketch~\ref{fig:sketch}.
%\begin{figure}
%%\includegraphics[width=\textwidth]{{./bifdiagram}.pdf}~
%\caption{Sketch of predicted bifurcation diagram}\label{fig:sketch}
%\end{figure}
%}
%
%\begin{figure}
%%%\includegraphics[width=\textwidth]{{./TW}.pdf}~\caption{Traveling wave. Parameters: $\alpha=1.3483$, $\sigma=4.6$, $\bar u=0.05$, $\beta=0.4398$. Domain size $L=20$.}\label{fig:TW}
%%\includegraphics[width=\textwidth]{{./0.37results}.pdf}~\\
%%\includegraphics[width=\textwidth]{{./0.37results_FTend}.pdf}~\caption{Traveling wave. Parameters: $\alpha=1.25$, $\sigma=4.6$, $\bar u=0.1$, $\beta=0.37$. Domain size $L=20$.}\label{fig:TW}
%\end{figure}
%
%
%\begin{figure}
%%%\includegraphics[width=\textwidth]{{./SwingingM}.pdf}~
%%\includegraphics[width=\textwidth]{{./0.41results}.pdf}~\\
%%\includegraphics[width=\textwidth]{{./0.41results_FTend}.pdf}~
%\caption{Swinging mode. Parameters: $\alpha=1.25$, $\sigma=4.6$, $\bar u=0.1$, $\beta=0.41$. Domain size $L=20$.}\label{fig:swinging_mode}
%\end{figure}
%
%\begin{figure}
%%\includegraphics[width=\textwidth]{{./modTW}.pdf}~
%\caption{Modulated traveling wave. Parameters: $\alpha=1.3483$, $\sigma=4.6$, $\bar u=0.05$, $\beta=0.4398$. Domain size $L=20$.}\label{fig:modTW}
%\end{figure}
%\ttuwe{Len's remark: For Tobias: The FN system should be solved just above the point where a Turing mode and a wave mode twice its wavelength are excited in the box equal to the wave mode's wavelength. In a larger box there will also be a longer wave with a larger growth rate. Unfortunately, bifurcation analysis, which would map the system on that considered in the '99 paper, is too cumbersome. Therefore an interesting regime with secondary oscillations (as in the blue area in the figure) could be located by trial and error. More interesting dynamics can be found at higher deviations from the bifurcation point .}

%\clearpage
%
%ADAPT DIRECTORY
\bibliography{FrTP2023}
%
% TO PRODUCE PAPER bib FILE: CREATE aux FILE, THEN RUN
% bibtool -i $HOME/Home/Bibliography/uwelitall.bib -i $HOME/Home/Bibliography/books.bib -x paper.aux -o paper.bib


%\include{TJB04_fig}
%
%
\end{document}
%
%
ENDENDENDENDENDENDENDENDENDENDENDENDENDENDENDENDENDENDENDENDENDENDENDENDe
ENDENDENDENDENDENDENDENDENDENDENDENDENDENDENDENDENDENDENDENDENDENDENDENDe
ENDENDENDENDENDENDENDENDENDENDENDENDENDENDENDENDENDENDENDENDENDENDENDENDe
ENDENDENDENDENDENDENDENDENDENDENDENDENDENDENDENDENDENDENDENDENDENDENDENDe
ENDENDENDENDENDENDENDENDENDENDENDENDENDENDENDENDENDENDENDENDENDENDENDENDe
ENDENDENDENDENDENDENDENDENDENDENDENDENDENDENDENDENDENDENDENDENDENDENDENDe
ENDENDENDENDENDENDENDENDENDENDENDENDENDENDENDENDENDENDENDENDENDENDENDENDe
ENDENDENDENDENDENDENDENDENDENDENDENDENDENDENDENDENDENDENDENDENDENDENDENDe
%
\begin{figure}[tbh]
  \caption{}
\mylab{}
\end{figure}
%
%

