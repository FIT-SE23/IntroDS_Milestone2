 \documentclass[superscriptaddress,longbibliography,amsmath,amssymb, footinbib]{revtex4-2}
% \documentclass[%
%  reprint,
% %superscriptaddress,
% %groupedaddress,
% %unsortedaddress,
% %runinaddress,
% %frontmatterverbose, 
% %preprint,
% %preprintnumbers,
% %nofootinbib,
% %nobibnotes,
% %bibnotes,
%amsmath,amssymb,
%  aps,
% %pra,
% %prb,
% %rmp,
% %prstab,
% %prstper,
% %floatfix,
% ]{revtex4-2}
\usepackage{tikz}
\usepackage{pgfplots}
\usepackage{graphicx}% Include figure files
\usepackage{dcolumn}% Align table columns on decimal point
\usepackage{bm}% bold math
\usepackage{esint}
\usepackage{color}
\newcommand{\red}{\color{red}}
\newcommand{\blue}{\color{blue}}
\def\be{\begin{equation}}
\def\ee{\end{equation}}
\usepackage{hyperref}% add hypertext capabilities
%\usepackage[mathlines]{lineno}% Enable numbering of text and display math
%\linenumbers\relax % Commence numbering lines

%\usepackage[showframe,%Uncomment any one of the following lines to test 
%%scale=0.7, marginratio={1:1, 2:3}, ignoreall,% default settings
%%text={7in,10in},centering,
%%margin=1.5in,
%%total={6.5in,8.75in}, top=1.2in, left=0.9in, includefoot,
%%height=10in,a5paper,hmargin={3cm,0.8in},
%]{geometry}

\begin{document}

%\preprint{APS/123-QED}

\title{
On Landauer--B\"uttiker formalism from a quantum quench
}

\author{Oleksandr Gamayun}
\email[Correspondence to: ]{oleksandr.gamayun@fuw.edu.pl}
%\affiliation{Bogolyubov Institute for Theoretical Physics, 03143 Kyiv, Ukraine}
\affiliation{
Faculty of Physics, University of Warsaw, ul. Pasteura 5, 02-093 Warsaw, Poland
}%


\author{Yuri Zhuravlev}
\affiliation{Bogolyubov Institute for Theoretical Physics, 03143 Kyiv, Ukraine}

 
\author{Nikolai Iorgov}%
% \email{iorgov@bitp.kiev.ua}
\affiliation{Bogolyubov Institute for Theoretical Physics, 03143 Kyiv, Ukraine}
\affiliation{Kyiv Academic University, 03142 Kyiv, Ukraine}





\date{\today}% It is always \today, today,
             %  but any date may be explicitly specified

\begin{abstract}
We study one-dimensional transport. We describe the transient regime and full-counting statistics + bound states. 
\end{abstract}

%\keywords{Suggested keywords}%Use showkeys class option if keyword
                              %display desired
\maketitle

%\tableofcontents



\section{Introduction}

The Landauer--B\"uttiker formalism lies in the heart of mesoscopic physics \cite{Landauer_1957,doi:10.1080/14786437008238472,PhysRevLett.57.1761}. 
It directly allows one to express the conductance in terms of the transmission matrix, this way relating transport and quantum properties \cite{Landauer_1992,Imry_1999}. 
Historically, the substantiation of this formalism via linear response theory was connected with certain controversies ({\it cf} \cite{PhysRevLett.46.618,PhysRevB.23.6851} and \cite{PhysRevB.22.3519,PhysRevLett.47.972,PhysRevB.24.2978,PhysRevB.24.1151}). The original Landauer formulas proved to be sensitive to the proper formulation of the physical problem, in particular, to the proper definition of leads, electron reservoirs and  self-consistency of linear response (for review see \cite{Stone_1988}). 
The controversies were finally resolved by Buttiker in Ref. \cite{PhysRevLett.57.1761}, where the general formulas for multi-terminal mesoscopic conductance were obtained. 

Even though according to elementary theory of tunnelling the transmission probability is defined in a stationary setup there were a lot of attention 
related to the non-equilibrium approach to the transport \cite{Caroli_1971,PhysRevB.22.5887}. 
The powerful analytic approaches involving Keldysh Green's function techniques were developed in Refs. \cite{Stefanucci2004,PhysRevB.69.195318,KOHLER2005,Moskalets2011}, along with the powerful numerical methods \cite{Gaury2016,PhysRevB.93.134506,Kloss2021}, which 
allows one not only to describe creation of the asymptotic currents and address their properties beyond the linear response regime but also explore behavior of the generic time-dependent quantum transport. 

From the point of view of the one-dimensional integrable models the attention to similar problems was renewed in the context of the quantum quenches, which are specifically,
understood as the evolution of the isolated quantum system initialized in the highly non-equilibrium state created either via the rapid change of the Hamiltonian, or containing macroscopic 
spatial inhomogeneities \cite{Calabrese_2007,Sotiriadis2008,Polkovnikov2011,Calabrese_2016,Eisert_2015}. The latter is more pertinent to the quantum transport setup and is dubbed as the partition approach \cite{Caroli_1971,PhysRevB.69.195318}. 
The large time behavior of such systems can be described by the generalized hydrodynamics 
\cite{Bertini_2016,Castro_Alvaredo_2016}, which allows one to get analytic treatment of the non-equilibrium steady currents, 
describe anomalous diffusion, and address the correlation functions (for review see the special issue \cite{Bastianello_2022}.  

The transport in the transnational invariant systems of free fermions and their spin analogues attracted a lot of attention due to the possibility of obtaining analytic answers for 
the average number of particles and its variance \cite{antal1999transport,Antal_2008,lancaster2010quantum,Viti_2016} (see also a numerical study in
\cite{PhysRevA.90.023624}). 
Other aspects of the evolution of the bipartite system were studied in \cite{Perfetto2017,Jin2021}. 
More delicate observables such as Loschmidt echo and Full Counting Statistics (FCS) were addressed in \cite{Viti_2016,St_phan_2017,PhysRevLett.110.060602,Sasamoto}, where the connection to  random matrix theory was performed 
and the FCS was expressed in terms of Fredholm determinants. 
Other connections of one-dimensional fermions at equilibrium in an external potentials and random matrix theory are reviewed in \cite{Dean2019}.


The simplest case when translational invariance is broken by a local defect in many cases also allows for an analytic treatment. 
Among others we would like to emphasize research that studies entropy evolution \cite{eisler2009entanglement,eisler2012on_entanglement,Dubail_2017}, 
transport properties within the interacting resonant level model \cite{Bransch_del_2010,PhysRevB.82.205414,Bidzhiev_2017,Bidzhiev_2019}, as well as non-integrable Ising chain \cite{PhysRevB.99.180302}. 
The inclusion of the defect in the generalized hydrodynamic approach was performed in \cite{Bertini2016} and the peculiarities of the thermalization via the defect were discussed in  
 \cite{10.21468/SciPostPhys.12.2.060}. Ref. \cite{10.21468/SciPostPhys.6.1.004} deals with the exact evaluation of the current and charge distribution for the bipartite scenario 
 when the left part of the system is prepared in the fully decorrelated state (infinite temperature) and is connected via the defect with the empty right part. 
 Further, this type of quench was considered for the hopping defect for the arbitrary initial distributions in \cite{Gamayun2020}, where FCS, Loschmidt echo and the entanglement entropy were computed. 
 In \cite{Schehr2022} analytic answers for the particle and energy currents as well as the full density distribution, were obtained for the continuous system with a delta impurity.
 
In this paper we study the continuous bipartite system with the arbitrary defect. 
We consider a bipartite protocol, in which the "left" part of the system is filled with fermions 
up to some chemical potential or distributed according to some probability (to model, for instance, the thermal initial state). 
according to some distributions and the "right" part is empty. 



in more details the physical formulation that corresponds to the non-equilibrium initial setup. 
Namely, we consider two closed one-dimensional systems to which we refer as to "left" and "right". related by the junction that corresponds to the potential scattering. 
Initially the left system contains free fermions subjected to the local potential. 
The energy level are filled up to some chemical potential or distributed according to some probability (to model, for instance, the thermal initial state). 
The right part of the system is empty. At some moment two systems are brought in contact and the subsequent evolution is given by the modified potential. 



\section{General properties of scattering}

In this section we briefly remind some general notions of the one-dimensional scattering on the local potential $V(x)$. 
The locality means that the potential vanishes fast enough as $|x|\to \infty$. For all practical purposes we assume that the potential is nonzero only in the finite domain 
$|x|<\xi$. This way, for $|x|>\xi$ the wave functions that correspond to the energy $E=k^2$ are the plane waves $e^{\pm i k x}$. With the chosen units of mass the Hamiltonian reads 
\be\label{eigenvalue}
- \frac{d^2\Psi}{dx^2}+V(x)\Psi = k^2\Psi
\ee
For every real $k \neq 0$ there exists a two-dimensional space of solutions (that corresponds to $k$ and $-k$). 
The typical basis in this space can be conveniently described by the Jost states $\psi_k$, $\varphi_k$  defined by their asymptotic behavior, namely
\be
\psi_k(x) = e^{-ikx} + o(1),\,\,\,\,\, x\to +\infty
\ee
\be
\varphi_k(x)= e^{-ikx} + o(1),\,\,\,\,\, x\to -\infty.
\ee
For a real potential these states are connected to their complex conjugated counterparts as $\psi_{-k}(x) = \bar{\psi}_k(x)$,
$\varphi_{-k}(x) = \bar{\varphi}_k(x)$.
If additionally the potential is symmetric $V(x) = V(-x)$, then  $\psi_k(-x)$ and $\varphi_k(-x)$ are still eigenfunctions. Considering the asymptotic behavior one can conclude that in this  case $
    \psi_k(-x) = \bar{\varphi}_k(x)$. 
Using Eq. \eqref{eigenvalue} we can see that the Jost solutions satisfy the following integral equations
\be\label{psiint}
\psi_k(x) = e^{-ik x} - \int\limits_x^\infty \frac{\sin(k(x-y))}{k}V(y) \psi_k(y) dy,
\ee
\be\label{phiint}
\varphi_k(x) = e^{-ik x} + \int\limits^x_{-\infty} \frac{\sin(k(x-y))}{k}V(y) \varphi_k(y) dy.
\ee 
As both Jost solutions form a basis they are connected by the linear transformation, the transfer matrix, 
\be\label{transfer}
\left(
\begin{array}{c}
	\varphi_k(x) \\
	\bar{\varphi}_k(x)
\end{array}
\right) = 
\mathcal{T}(k)
\left(
\begin{array}{c}
	\psi_k(x) \\
	\bar{\psi}_k(x)
\end{array}
\right),\,\,\,\,\,\,\,\,\, \mathcal{T}(k) = \left(
\begin{array}{cc}
	a_k & b_k\\
	\bar{b}_k & \bar{a}_k
\end{array}
\right).
\ee
Note that for a real potential $a_{-k}=\bar{a}_k$, $b_{-k}=\bar{b}_k$, while for symmetric potential  $b_k$ is purely imaginary. 


Considering the Wronskian of the eigenvalue problem \eqref{eigenvalue} we conclude that the transfer matrix is unimodular
\begin{equation}\label{uni}
\det \mathcal{T}(k) =|a_k|^2-|b_k|^2=	1.
\end{equation}
The transfer matrix $\mathcal{T}$ can be repacked into the $S$-matrix \cite{Newton1982} as follows 
\be
S = \frac{1}{a_k}\left(
\begin{array}{cc}
	-\bar{b}_k & 1\\
	1 & b_k
\end{array}
\right).
\ee
The unimodularity condition \eqref{uni} means the unitarity for S-matrix $SS^+ =1$. 
The transmission and the reflection coefficients are defined as the squared absolute values of the off-diagonal and diagonal components of the S-matrix, respectively,
\begin{equation}\label{tran}
    T(E) = \frac{1}{|a_k|^2},\qquad R(E) = \frac{|b_k|^2}{|a_k|^2}. 
\end{equation}
Here we present them as the functions of energy $E = k^2$. The unitarity \eqref{uni} guarantees that $T(E) + R(E) = 1$. 

The coefficient $a_k$ can be analytically continued to the upper half plane where it might have zeroes that correspond to the bound states. They are purely imaginary $k=i\varkappa$ so the corresponding energy is negative $E = -\varkappa^2$. In fact the analytic properties allow one to present (see for instance \cite{Novikov})
\begin{equation}
    \label{a}
a_k = \prod\limits_{n=1}^{N} \frac{k-i\varkappa_n}{k+i\varkappa_n}
\exp\left(\frac{1}{2\pi i}\int\limits_{-\infty}^{\infty}\frac{\log (1+|b_q|^2)}{q-k-i0}dq\right)
\end{equation}
To describe the wave function of a bound state we can use either $\varphi_k(x)$ and $\bar{\psi}_k(x)$ as both these functions can be analytically continued to the upper half plane. In fact, it turns out that they are proportional $\varphi_{i\varkappa}(x) = b_\varkappa \bar{\psi}_{i\varkappa}$. Taking into account the definition of transfer matrix \eqref{transfer} this relation is hardly surprising and $b_{\varkappa}$ can be considered as an analytic continuation of the $b_k$, however, contrary to $a_k$  such continuation is not always possible, and the coefficient $b_\varkappa$ should be considered as additional scattering data.  

Finally, let us comment on the normalization conditions of the continuous spectrum. Similar to \cite{Novikov} we conclude that 
\begin{equation}
    \int\limits_{-\infty}^\infty dx \varphi_k(x) \bar{\psi}_q(x) = a_q \delta(k-q)
\end{equation}
Therefore the Green's function $G(x,y,t)$ defined as a solution of the Schrodinger equation in $x$ variable with the initial condition $G(x,y,t=0) = \delta(x-y)$, can be presented as 
\begin{equation}\label{Gsimple}
    G(x,y,t) = \int_C\frac{dk}{2\pi} \frac{\varphi_k(x) \bar{\psi}_k(y)}{a_k} e^{-itE_k}.
\end{equation}
The contour $C$ originally goes along the real line. We notice however that the integrand can be analytically continued in the upper half plane. Moreover, in this form we can easily take into account also contributions from the bound states. To do so the contour $C$ should run above all positions of zeroes of $a_k$ in the upper half plane. 
Below we re-derive this presentation using wave functions in the box (hard-wall boundary conditions), and demonstrate how to express full counting statistics via the scattering data and Jost solutions.  


\section{Quench protocol and hard-wall wave functions} 

The scattering states introduced in the previous section describe an infinite system. To correctly formulate transport problem we consider 
open (hard-wall) boundary conditions placed at $x=\pm R$, perform computations at finite $R$, and send $R\to\infty$ in the end on the computation. 
At the initial moment of time only the left part of the system $x<0$ is filled. Meaning that the single particle  wave functions  $\Lambda_k$  are non-zero only in the interval $x\in [-R,0]$, more formally
\begin{equation}\label{eq1}
    - \frac{d^2\Lambda_k}{dx^2}+V_0(x)\Lambda_k = k^2\Lambda_k,\qquad\qquad \Lambda_k(0) = \Lambda_k(-R) = 0.
\end{equation}
The post-quench  wave functions satisfies
\begin{equation}\label{eq2}
    - \frac{d^2\chi_k}{dx^2}+V(x)\chi_k = k^2\chi_k,\qquad\qquad \chi_k(-R) = \chi_k(R) = 0.
\end{equation}
The initial $N$-particle state of the system  $|{\rm in}\rangle$ is given in a Fock space by an ordered set of momenta $q_1<q_2<\dots< q_N$. Formally, it can be presented as a wedge product
\begin{equation}\label{vac}
    |{\rm in}\rangle  =  \Lambda_{q_1}\bigwedge \Lambda_{q_2} \dots \bigwedge\Lambda_{q_N},
\end{equation}
which in the coordinate space corresponds to a single Slater determinant. The case of the statistical ensemble in the $N\to \infty$ limit can be described by taking the typical distribution of $q_i$. 
To characterize  many body dynamics we consider full counting statistics (FCS). It can be written as
\begin{equation}
    \mathcal{F}(\lambda,t) = \langle {\rm in}| e^{itH} e^{\lambda N_R} e^{-itH} |{\rm in} \rangle = \langle {\rm in}| e^{\lambda \int\limits_0^t d\tau J(\tau)} |{\rm in} \rangle,
\end{equation}
where $N_R$ is number of particles in right part of the system and $J(\tau)$ is the current with the point $x=0$. Due to the free fermionic structure of the initial state \eqref{vac} the FSC can be presented as 
\begin{equation}
    \mathcal{F}(\lambda,t) = \det X_{ab},
\end{equation}
with indices $a$ and $b$ correspond to the momenta in the initial state $|{\rm in}\rangle$, and the matrix elements are
\begin{equation}
	X_{ab} =\delta_{ab}+ (e^\lambda-1)\sum_{k,p}  \frac{(\Lambda_a,\chi_k)(\chi_k, P_>\chi_p)(\chi_p,\Lambda_b)}{\sqrt{(\Lambda_a,\Lambda_a)}(\chi_k,\chi_k)(\chi_p,\chi_p)\sqrt{(\Lambda_b,\Lambda_b)}} e^{it(E_k-E_p)}.
	\label{Xab}
\end{equation}
Here $P_>$ is a projector on the right part of the system i.e. $x\in [0, R)$. Our goal is to present this expression in the thermodynamic limit such that the 
FCS can be written as the Fredholm determinant of some trace-class operator. Namely, we present 
\begin{equation}\label{kk}
 X_{ab} =\delta_{ab}+ \frac{\pi}{R} K(q_a,q_b)+ o(1/R)   
\end{equation}
so that FCS
in the thermodynamic limit $R\to \infty$ transforms into a Fredholm determinant 
\begin{equation}\label{Ftd}
    \mathcal{F} (\lambda,t) = \det X \to \det \left(1 + \rho \hat{K}\right)
\end{equation}
where $\rho$ in the density of the initial state and the operator $\hat{K}$ acts on the integrable functions via the convolution with the kernel $K(q,q')$.
{\color{red} We compute this kernel in Sec. ??, while the full answer is given in Eq. ?? }

In the rest of this section we give an explicit description of the hard-wall wave functions in terms of the Jost functions and scattering data. 
We start with $\chi_k$. Assuming that the range of the potential $\xi$ is much smaller than $R$ the wave function can be presented as 
\begin{eqnarray}\label{chikk}
    \chi_k(x) = {\rm Im} \left[e^{ikR}\psi_k(x)\right]
\end{eqnarray}
where $\psi_k$ is a Jost function that corresponds to the potential $V(x)$ (see Eq. \eqref{psiint}). 
This way the condition $\chi_k(R) = 0 $ is satisfied automatically, while for the large negative $x$ the behavior reads 
\begin{equation}
    \chi_k(x) =  {\rm Im} \left[e^{ikR}(\bar{a}_k e^{-ikx} -b_k e^{ikx})\right].
\end{equation}
Here the scattering data corresponds to the potential $V(x)$. Demanding $\chi_k(-R)=0$ will provide us with the spectrum condition, that can be resolved as 
\begin{equation}\label{sp23}
	e^{2i k R} = \frac{i {\rm Im}\,b_k+\sqrt{1+({\rm Re}\,b_k)^2}}{\bar{a}_k} \equiv e^{-2i\delta(k)}.
\end{equation}
Here we have introduced the scattering phase $\delta(k)$. We have to take into account two possible solution that corresponds to the two different branches of the square root. 
Therefore, in fact we have two different scattering phases. For both of them we have $\delta(k) = - \delta(-k)$, this fact plus that the energy $E_k = k^2$, limits us to the only positive solutions of Eq.~\eqref{sp23}. 

Let us also discuss the normalization of the wave function. 
To this end we notice that the $k$ derivative of the $\chi_k$ satisfies
\begin{equation}
\left(
-\partial_x^2+ V(x) - k^2
\right)\partial_k\chi_k = 2k \chi_k,\qquad 
\left(
-\partial_x^2+ V(x) - k^2
\right)\chi_k =0.
\end{equation}
So we can write 
\begin{equation}
2k (\chi_k,\chi_k) =  \int\limits_{-R}^{R} dx \left[
-\frac{d^2\partial_k\chi_k }{dx^2}\chi_k(x) + \partial_k\chi_k \frac{d^2\chi_k(x)}{dx^2}
\right] =  \left[
-\frac{d\partial_k\chi_k }{dx}\chi_k(x) + \partial_k\chi_k \frac{d\chi_k(x)}{dx}
\right] \Big|_{-R}^{R}\label{norm0}
\end{equation}
This allows us to present 
\begin{equation}\label{norm}
    (\chi_k,\chi_k) = ({\rm Re}\, b_k + \sqrt{1 + ({\rm Re}\, b_k)^2})\sqrt{1 + ({\rm Re}\, b_k)^2} (R + \delta'(k)).
\end{equation}
Here $\delta'(k)$ means the momentum derivative.
Similarly, we can describe the matrix elements $(\chi_k, P_>\chi_p)=\int\limits_{0}^R dx \chi_k(x) \chi_p(x) $ on the projector in Eq. \eqref{Xab}
\begin{multline}\label{chi2}
     (E_k-E_p)(\chi_k, P_>\chi_p) = \int\limits_{0}^R dx \left[\left(-\partial_x^2+ V(x) \right)\chi_k(x)\right] \chi_p(x) -
     \int\limits_{0}^R dx \chi_k(x)\left(-\partial_x^2+ V(x) \right)\chi_p(x)
      \\ =\int\limits_{0}^R dx \partial_x\left(
     -\chi_p(x) \partial_x\chi_k(x)+ \chi_k(x)\partial_x\chi_p(x)
     \right) = \chi_p(0) \partial_x\chi_k(0)-\chi_k(0) \partial_x\chi_p(0).
\end{multline}

To describe bound states that might be present in the system, one can argue that due to finite range of the potential the corresponding wave functions will be localized around $x=0$, and decay exponentially for large $x$. Therefore the boundary conditions are satisfied automatically with the exponential precision, and we may put
\begin{equation}
    \chi_k^{\rm bound} (x) \approx \varphi_{i\varkappa}(x),\qquad k = i\varkappa. 
\end{equation}
Its normalization can be found in the similar manner taking into account the identification  $\varphi_{i\varkappa}(x) = b_\varkappa \bar{\psi}_{i\varkappa}(x)$, discussed in the previous chapter. 
Indeed, using the fact that at $x\to+\infty$, the leading term over the momentum in the leading wave function behaves as $a'_{i\varkappa } e^{\varkappa x}$, we obtain 
\begin{equation}\label{NormBound}
    (\varphi_{i\varkappa},\varphi_{i\varkappa}) = i a'_{i\varkappa} b_\varkappa
 \end{equation}



Similarly we can find the pre-quench wave function $\Lambda_k$. In this case it is more convenient to use the Jost solution \eqref{phiint} on the potential $V_0$, which we denote as $\Phi_k(x)$. In this notation we propose the following formula
\begin{equation}\label{lambda1}
    \Lambda_k(x) = {\rm Im}\frac{\Phi_k(x)}{\Phi_k(0)}
\end{equation}
Notice that in this form the boundary condition $\Lambda_k(0)=0$ is satisfied automatically, while the condition $\Lambda_k(-R)=0$ defines spectrum
and the scattering phase $\eta(k)$
\begin{equation}\label{sp44}
    e^{2ikR} = \frac{\Phi_k(0)}{\bar{\Phi}_k(0)} \equiv e^{-2i\eta(k)}.
\end{equation}
Normalization now reads as 
\begin{equation}\label{LambdaOver}
    (\Lambda_k,\Lambda_k) =  \frac{R +\eta'(k)}{2|\Phi_k(0)|^2}.
\end{equation}
Finally, computation of the overlaps between pre- and postquench wavefunctions in \eqref{Xab}, can be avoided completely, and replaced by the corresponding 
overlaps with the Jost's functions. Namely, as it follows from Eq. \eqref{chi2} the time derivative of the $\eqref{Xab}$ can be expressed via 
the (conjugated) time evolution of the wave function {\blue $\Lambda_q(x,t)$}
defined as 
\begin{equation}\label{L0}
    \Lambda_q(y,t) \equiv \sum_k \frac{(\Lambda_q,\chi_k)\chi_k(y)}{(\chi_k,\chi_k)}e^{itE_k} = \int\limits_{-R}^0 dx \Lambda_q(x) G^*(x,y,t).
\end{equation}
Here we have used the following presentation of the Green's function 
\begin{equation}\label{Gstar}
    G^*(x,y,t) \equiv \sum\limits_k  \frac{\chi_k(x)\chi_k(y)}{(\chi_k,\chi_k)} e^{itE_k},
\end{equation}
The summation is taken over all spectral points \eqref{sp23}. We perform this summation explicitly in Appendix \eqref{appG} with the genuine discrete degrees of freedom, and taking the thermodynamic limit only the very end. The computation is straightforward but a bit tedious.
However, the obtained result can be easily explained heuristically. Namely, one can argue that in the thermodynamic limit instead of function \eqref{Gstar}
one can use \eqref{Gsimple}. 
This way,  we can find a presentation only with the Jost solutions introduced in the previous section
\begin{equation}
   \Lambda_q(y,t) =   \int_C \frac{dk}{2\pi} \frac{(\Lambda_q,\varphi_k)\bar{\psi}_k(y)}{a_k}e^{itE_k}.
\end{equation}
{\blue The integrals path $C$} similar to Eq. \eqref{Gsimple} runs from $-\infty$ to $+\infty$ in the upper half plane above all positions of zeroes of $a_k$. 
The overlap $(\Lambda_q,\varphi_k)$ can be computed using the same trick as in Eq. \eqref{norm0} and Eq. \eqref{chi2}. 
Indeed, if we introduce function 
\begin{equation}\label{Xiqk}
    \Xi_{q,k} =\Lambda_q'(0)\varphi_k(0)- \int\limits_{-\infty}^0 dx \Lambda_q(x) (V_0(x) - V(x))\varphi_k(x),
\end{equation}
we can present 
\begin{equation}\label{overlll}
    (E_k -E_q) \int\limits_{-R}^0 dx \Lambda_q(x) \varphi_k(x) = \Xi_{q,k} - \Lambda_q'(-R)\varphi_k(-R).
\end{equation}
Here we have used that due to {\blue finite range} of the potentials the lower limit of the integration in \eqref{Xiqk} can be either $-R$ or $-\infty$. 
Taking into that for $k\in C$ the last term vanishes exponentially $\varphi_k(-R)\sim e^{ikR}$, we finally present 
\begin{equation}\label{L1}
     \Lambda_q(y,t) =   \int_C \frac{dk}{2\pi} \frac{\Xi_{q,k} \bar{\psi}_k(y)}{(k^2-q^2)a_k}e^{itE_k}
\end{equation}
This the final answer in the thermodynamic limit. 
{\blue Notice that $\Xi_{q,k}$ is a regular function and can be continued from the discrete spectrum to upper half plane of the variable $k$.} 
The direct computation of $\Lambda_q(0,t)$ and its derivative in the finite system is given in Appendix \eqref{appF}. 
In the next section we use the presentation \eqref{L1} to compute kernel $K(q,q')$ in \eqref{kk} and \eqref{Ftd}. 


\section{Kernel} 

To compute the kernel $K(q,q')$ for the Fredholm determinant of the Full Counting Statistics \eqref{Ftd}, we start by considering its time derivative. 
Using explicit presentation \eqref{Xab} and \eqref{chi2}, along with the definition \eqref{L0}, we arrive at 
\begin{equation}\label{dK}
    \frac{dK(q,q')}{dt} = \frac{2i(e^\lambda-1)}{\pi}
    |\Phi_q(0)| \left( f^{(1)}_q(t) \bar{f}^{(0)}_{q'}(t) - f^{(0)}_q(t)\bar{f}^{(1)}_{q'}(t)\right)|\Phi_{q'}(0)|.
\end{equation}
where we have denoted 
\begin{equation}\label{fa2m}
f^{(\alpha)}_q (t) = \partial^\alpha_y \Lambda_q(y,t)\Big|_{y=0}=
\int\limits_C \frac{dk}{2\pi} 
	\frac{\Xi_{q,k}\partial_x^\alpha\bar\psi_k(0)}{a_k}
	\frac{e^{itk^2}}{k^2-q^2}, \qquad \alpha=0,1.    
\end{equation}

Using presentation \eqref{L1} we can directly integrate Eq. \eqref{dK}. However, in order to easier assess the long-time asymptotic we first identically transform $f^{(\alpha)}_q $ to highlight the most relevant terms as $t\to+\infty$. To do so we notice that the exponential $e^{itk^2}$ is decaying in the first and third  quadrants of complex plane of $k$ (see Fig. ???).
So we deform the controur $C$ into $C'$ by pulling it towards the real negative line and crossing it. 
By doing so we inevitably encircle all positions of the bound states and the pole $k=-q$. 
The obtained deformation reads 
\begin{equation}\label{faCp}
	f^{(\alpha)}_q(t) = i \frac{\Xi_{q,-q}\partial_x^\alpha\bar\psi_{-q}(0)}{a_{-q}} 	\frac{e^{itq^2}}{2q}+   \sum_{n=1}^{N^\mathrm{b}} 
	\frac{i \Xi_{q,i\varkappa_n}\partial_x^\alpha\bar\psi_{i\varkappa_n}(0)}{a'_{i\varkappa_n}}
	\frac{e^{-it\varkappa_n^2}}{\varkappa_n^2+q^2}
		+
	\int\limits_{C'} \frac{dk}{2\pi} 
	\frac{\Xi_{q,k}\partial_x^\alpha\bar\psi_k(0)}{a_k}
	\frac{e^{itk^2}}{k^2-q^2}.
\end{equation}
Notice that we can use \eqref{overlll} along with the asymptotic behavior $\Lambda_q'(-R)\sim -q e^{iqR}/\Phi_q(0)$ for large $R$ (see \eqref{lambda1}) to obtain 
\begin{equation}\label{Xiqmq}
    \Xi_{q,-q}=-\frac{q}{\Phi_q(0)}.
\end{equation}
The direct proof of this expression from the definition \eqref{Xiqk} is given in Appendix \eqref{pp}. 
Further we use the symmetry $k\to -k$ to fold the full contour $C'$ and consider integration only with $\mathrm{Re}\,k>0$, namely
\begin{equation}\label{faRe}
	f^{(\alpha)}_q(t) =  \sum_{n=1}^{N^\mathrm{b}} 
	 B_{n,q}^{(\alpha)} e^{-it\varkappa_n^2}
		+F^{(\alpha)}_q e^{itq^2} + 
	\int\limits_{0}^{\infty} \frac{dk}{\pi} 
	\Omega^{(\alpha)}_{q,k}
	\frac{e^{itk^2}}{(k+i0)^2-q^2},
\end{equation}
\begin{equation}\label{BFOm}
    B_{n,q}^{(\alpha)}=
    \frac{i \Xi_{q,i\varkappa_n}\partial_x^\alpha\bar \psi_{i\varkappa_n}(0)}
    {a'_{i\varkappa_n}(\varkappa_n^2+q^2)},
    \qquad
    F_q^{(\alpha)}=-i \frac{\partial_x^\alpha\psi_{q}(0)}{2\Phi_q(0)a_{-q}},
    \qquad
    \Omega^{(\alpha)}_{q,k}=
    \mathrm{Re}\,
	\frac{\Xi_{q,k}\partial_x^\alpha\bar\psi_k(0)}{a_k}.
\end{equation}
Such form of $f^{(\alpha)}_q(t)$ is convenient for large $t$ asymptotic analysis. The main contributions come from the poles corresponding to bound states of $H$ and from the pole at $k=-q$.
They give an oscillatory behaviour in $t$. The integral in Eq.~\eqref{faRe} can be estimated by the method of stationary phase with a saddle point at $k=0$ producing power-like decay in $t$ for large $t$. The exponent of this power law in $t$ depends on the behaviour of $\Omega^{(\alpha)}_{q,k}$ at $k=0$. In the case of generic potentials, $a_k$ has a first order pole at $k=0$  while 
$\Xi_{q,k}$ and $\partial_x^\alpha\psi_k(0)$ are regular at $k=0$. 
Therefore $\Omega^{(\alpha)}_{q,k}$ has at least first order zero at $k=0$ 
which implies power law $t^{-1}$ (or even faster) decaying of the integral 
in Eq.~\eqref{faRe} for large $t$.
For some special potentials (for example reflectionless potentials), $a_k$ is regular at $k=0$. For such potentials the integral decays
as $t^{-1/2}$:
\begin{equation}\label{intreg}
    \int\limits_{0}^{\infty} \frac{dk}{\pi} 
	\Omega^{(\alpha)}_{q,k}
	\frac{e^{itk^2}}{(k+i0)^2-q^2}=\frac{I^{(\alpha)}_{q}}{\sqrt{t}}+ 	O(t^{-1}),
\end{equation}
\begin{equation}\label{Iq}
I^{(\alpha)}_{q}=- \frac{\sqrt{\pi} e^{i\pi/4} \Xi_{q,0}\partial_x^\alpha\psi_0(0)}{2 a_0 q^2}.
\end{equation}
In what follows we will need the relation 
\begin{equation}\label{resreg}
    I^{(1)}_{q}\bar I^{(0)}_{q'}-I^{(0)}_{q}\bar I^{(1)}_{q'}=0.
\end{equation}
Note, Eq.~\eqref{Iq} shows a singular behaviour for small $q$. 
In fact there is no such a singularity because in asymptotic analysis of Eq.~\eqref{intreg} we assumed 
that a pole at $k=q$ is far from stationary point $k=0$. This assumption is incorrect for small 
$q\lesssim t^{-1/2}$ and we need a different asymptotic analysis of  Eq.~\eqref{intreg}.
Such analysis was done for the current and it was shown that the contribution of small $q$ changes only subleading behaviour of it. We believe that small $q$ do not change asymptotic behaviour of FCS too.

Using notations from the previous section we can explicitly write down the kernel of the FCS in 
Eq.~\eqref{Xab}. After presenting this kernel in the form $X_{ab} =\delta_{ab}+ \frac{\pi}{R} K(q_a,q_b)+ o(1/R)$ 
the full counting statistics in the thermodynamic limit $R\to \infty$ transforms into a Fredholm determinant 
\begin{equation}\label{FCSdet}
    \mathcal{F} (\lambda,t) = \det X \to \det \left(1 + \rho \hat{K}\right)
\end{equation}
where $\rho$ is the density of the initial state and the operator $\hat{K}$ acts on the integrable functions via the convolution with the kernel $K(q,q')$.
To compute this kernel we integrate in $t$  Eq.~\eqref{dK} with substituted Eq.~\eqref{faRe}.
This way the kernels can be written as 
\begin{equation}\label{Kqqp}
	K(q,q') =\frac{2i(e^\lambda-1)}{\pi}
	|\Phi_q(0)| \left( S_{qq'}(t)  + M_{qq'}(t)- \bar M_{q'q}(t) \right)|\Phi_{q'}(0)| 
\end{equation}
\begin{equation}
	S_{qq'}(t) = (F_q^{(1)} \bar F_{q'}^{(0)}-F_q^{(0)} \bar F_{q'}^{(1)})
	\frac{e^{it(E_q - E_{q'})}-1}{i(E_q - E_{q'})},
\end{equation}
\begin{equation}
     M_{qq'}(t)=X_{qq'}(t)+Y_{qq'}(t)-Y_{qq'}(0),
\end{equation}
\begin{equation}
    X_{qq'}(t)=\sum_{n=1}^{N^\mathrm{b}} (B_{nq'}^{(0)} F_{q}^{(1)}-B_{nq'}^{(1)} F_{q}^{(0)})
    \frac{e^{it(\varkappa_n^2+E_{q})}-1}{i(E_{q}+\varkappa_n^2)}\\
    +\sum_{m<n}^{N^\mathrm{b}} (B_{mq}^{(1)}B_{nq'}^{(0)}-B_{mq}^{(0)} B_{nq'}^{(1)})
    \frac{e^{it(\varkappa_n^2-\varkappa_m^2)}-1}{i(\varkappa_n^2-\varkappa_m^2)},
\end{equation}
\begin{multline}\label{Yqqp}
    Y_{qq'}(t) =
      \int_0^\infty \frac{dk}{\pi} \frac{e^{it(E_k-E_{q'})}}{i(E^+_k-E_{q'})} 
    \frac{ \Omega_{qk}^{(1)}\bar F_{q'}^{(0)}- \Omega_{qk}^{(0)}\bar F_{q'}^{(1)}}{E^+_k-E_q} 
   +\frac12 \int\limits_0^\infty \frac{dk}{\pi}\int\limits_0^\infty \frac{dp}{\pi} \frac{e^{it(E_k-E_p)}}{i(E^+_k-E^-_p)}
   \frac{\Omega_{qk}^{(1)} \Omega_{q'p}^{(0)}-\Omega_{qk}^{(0)} \Omega_{q'p}^{(1)}}
   {(E^+_k-E_q)(E^-_p-E_{q'})}\\
     +  \sum_{n=1}^{N^\mathrm{b}} 
        \int_0^\infty \frac{dk}{\pi} \frac{e^{it(E_k+\varkappa_n^2)}}{i(E^+_k+\varkappa_n^2)} 
    \frac{ \Omega_{qk}^{(1)} B_{nq'}^{(0)}- \Omega_{qk}^{(0)} B_{nq'}^{(1)}}{E^+_k-E_q}   .
\end{multline}
In the limit of large $t$ the integrals in $Y_{qq'}(t)$ have a power like decaying behaviour.  
It can be deduced using asymptotic behaviour of the integral in Eq.~\eqref{faRe} at large $t$
after substitution into Eq.~\eqref{dK} and following integration in $t$.
The most subtle analysis is needed for the case when $a_k$ is regular at $k=0$.
In this case Eq.~\eqref{intreg} leads potentially to a logarithmic growth for large $t$ in the double integral 
in Eq.~\eqref{Yqqp}, but due to the relation \eqref{resreg} the leading term in $t$ is canceled. Finally, in the case of regular $a_k$ at $k=0$, the double integral  in Eq.~\eqref{Yqqp} is decaying for large $t$ as  $t^{-1/2}$,  while in the generic case (there is a pole of $a_k$ at $k=0$) it is decaying is as $t^{-1}$.

Unfortunately we were not able to provide a strict analysis of large $t$ behaviour of FCS given by Eq.~\eqref{FCSdet}. 
However we can guess a form of its leading term reducing the problem to known results on Fredholm  determinant
of generalized sine-kernel [...]. In the next section (?) we consider numerical analysis of FCS for different potentials supporting our approximations. 
In the case of presence of bound states for potential $V(x)$ there is a finite-rank perturbation  $X(t)$ of a Fredholm determinant. For simplicity we consider potential $V(x)$ without bound states. 
We believe that for large time $t$ the main contribution to FCS \eqref{FCSdet} comes from $S_{qq'}$ of the kernel \eqref{Kqqp}.
We rewrite the term with $S_{qq'}$  as a generalized sine-kernel 
\begin{multline}
	S_{qq'}(t) = 2e^{it(E_q - E_{q'})/2}(F_q^{(1)} \bar F_{q'}^{(0)}-F_q^{(0)} \bar F_{q'}^{(1)})
	\frac{\sin t(E_q - E_{q'})/2}{E_q - E_{q'}}\approx 
	2 (F_q^{(1)} \bar F_{q}^{(0)}-F_q^{(0)} \bar F_{q}^{(1)})
	\frac{\sin t(E_q - E_{q'})/2}{E_q - E_{q'}}\\
	=-\frac{iq}{2|\Phi_q(0)|^2} T(E) \frac{\sin t(E_q - E_{q'})/2}{E_q - E_{q'}},
\end{multline}
where the approximation sign means large $t$ approximation of $	S_{qq'}(t)$ valid under substitution into Fredholm determinant,
and at the last step we used Eq.~\eqref{BFOm}, a formula for the current at $x=0$ of Jost solution $\psi_q(x)$ and Eq.~\eqref{tran} for the transmission coefficient
$T(E)$. Also we expect that $Y(t)$ gives subleading correction and $Y(0)$ corrects  the leading term coming from sine-kernel by a time-independent prefactor. 
Finally, we expect that FCS given by Eq.~\eqref{FCSdet} in the case of absence of bound states has the same large-time leading asymptotic behaviour 
(up to a prefactor $C(\lambda)$ independent of time) as Fredholm determinant of a generalized sine-kernel
\begin{equation}\label{Fsindet}
    \mathcal{F}(\lambda,t) \approx C(\lambda)  \det \left(1 + \frac{e^\lambda-1}{\pi}\rho(E)T(E)\frac{\sin \frac{t(E-E')}{2}}{E-E'} \right),
\end{equation}
where $\rho(E)$ is a distribution of the initial state, and also we changed the integration variable $q$, momentum of fermions, to $E=q^2$. 
Comparing the kernel of Fredholm determinant in Eq.~\eqref{Fsindet} with a standard form of Fredholm determinant of a generalized sine-kernel 
\begin{equation}\label{detS}
     \mathcal{S}(\lambda,t) = \det \left(1 + \frac{e^{2\pi i \nu(E)}-1}{\pi} \frac{\sin \frac{t(E-E')}{2}}{E-E'} \right)
\end{equation}
we identify 
\begin{equation}\label{nuE}
    \nu(E)  = \frac{1}{2\pi i } \log \left(1+ (e^\lambda-1) \rho(E) T(E) \right).
\end{equation}
Large $t$ asymptotic behaviour of Eq.~\eqref{detS} was found in  [...] and for $\nu(E)$ given by Eq.~\eqref{nuE} it leads to  
\begin{equation}
      \mathcal{F}(\lambda,t)  \approx \frac{ C(\lambda)  \tilde C(\lambda) }{t^{\nu(E_F)^2}} \exp\left(
      \frac{t}{2\pi }\int\limits_0^{E_F} \log (1 + (e^\lambda-1) \rho(E) T(E)) dE 
      \right).
\end{equation}
{\blue Note, $\nu(0)=0$.} 
The prefactor $\tilde C(\lambda)$ is given by
\begin{multline}
    \tilde C(\lambda)=\frac{G(1+\nu(E_F))G(1-\nu(E_F))}{E_F^{\nu(E_F)^2}}
    \exp \left(\nu(E_F)\int_0^{E_F} \frac{\nu(E_F)-\nu(E)}{E_F-E}dE\right)\\
    \times\exp \frac12\int_0^{E_F}\int_0^{E_F} dEdE'\frac{\nu'(E)\nu(E')-\nu(E)\nu'(E')}{E-E'},
\end{multline}
where $G(x)$ is the Barnes function.

{\blue It is similar to Levitov--Lesovik formula ...}

\subsection{FCS for perfect lead attachment}

In this subsection we consider a special case of quench setup  when $V_0(x) = V(x)$ for $x<0$. 
We call this situation 	the \textit{perfect lead attachment}. In this case due to the integral presentation \eqref{phiint} the corresponding Jost functions coincide $\varphi_q(x) =\Phi_q(x)$ for $x \le 0$. From  presentation \eqref{Xiqk} we observe the factorization
\begin{equation}\label{XiqkPLA}
	\Xi_{q,k} = \Lambda'_q(0) \varphi_k(0),
\end{equation}
which imply a similar factorization ${f}^{(\alpha)}_q(t)= \Lambda'_q(0) g^{(\alpha)}_q(t)$ for  ${f}^{(\alpha)}_q(t)$ given  by Eq.~\eqref{fa2m},
where 
\begin{equation}\label{fa22}
	g^{(\alpha)}_q(t) =  \int\limits_C \frac{dk}{2\pi} \omega_k^{(\alpha)}
		\frac{e^{itk^2}}{k^2-q^2} ,\qquad
		\omega_k^{(\alpha)} \equiv \frac {\varphi_{k}(0)\partial_x^\alpha\bar\psi_k(0)}{a_k}.
\end{equation}
Comparing Eq.~\eqref{XiqkPLA} at $k=-q$ with Eq.~\eqref{Xiqmq} we conclude that 
$\Lambda'_q(0)= -q/|\varphi_q(0)|^2 $. 
Therefore now the Eq.~\eqref{dK} reads
\begin{equation}\label{dKqqp}
 \frac{dK(q,q')}{dt} = \frac{2i(e^\lambda-1)qq'}{\pi |\varphi_q(0)| |\varphi_{q'}(0)|}
\left( g^{(1)}_q(t) \bar{g}^{(0)}_{q'}(t) - g^{(0)}_q(t)\bar{g}^{(1)}_{q'}(t)\right).
\end{equation}
Integrating in $t$ we can present the kernel $K(q,q')$ in the integrable form
\begin{equation}\label{Kqqp}
	K(q,q') = \frac{2(e^\lambda-1)qq'}{\pi|\varphi_q(0)| |\varphi_{q'}(0)|} \frac{g^{(1)}_q(t) \bar{g}^{(0)}_{q'}(t) - g^{(0)}_q(t)\bar{g}^{(1)}_{q'}(t)+
	\bar{D}_q(t)-D_{q'}(t)}{E_q - E_{q'}},
\end{equation}
where 
\begin{equation}
	D_{q}(t) =i \int\limits_0^t d\tau \int\limits_C \frac{dk}{2\pi} e^{i\tau k^2}  
	\left[ \omega_k^{(1)}\bar{g}^{(0)}_{q}(\tau)-\omega_k^{(0)}\bar{g}^{(1)}_{q}(\tau) \right] .
\end{equation}
To check correctness of Eq.~\eqref{Kqqp} we need to compare its derivative in $t$ with Eq.~\eqref{dKqqp} using 
\begin{equation}
	\frac{d}{dt}g^{(\alpha)}_q(t)  =iq^2 g^{(\alpha)}_q(t)+ i \int\limits_C \frac{dk}{2\pi} \omega_k^{(\alpha)}e^{i t k^2}.
\end{equation}
Also we have to check that $K(q,q')=0$ at $t=0$. It is correct due to equality $\mathfrak{f}^{(0)}_q(0)=0$ following from analyticity of $\omega_k^{(0)}$ in the upper half-plane of $k$.
The integrable form of kernel $K(q,q')$ allows one, in particular, to replace evaluation of the Fredholm determinants by a solution of the Riemann--Hilbert problem \cite{Deift_1997,Bogoliubov1997}.
This approach is especially useful for the asymptotic analysis at large time $t\to + \infty$.
In this case, however, if we follow the standard procedure outlined in \cite{Bogoliubov1997}, the corresponding jump matrix will have size $4\times4$.

{\blue 
The kernel simplifies in the $t\to \infty $ limit. For $\mathfrak{f}^{(\alpha)}_q(t)$ we can neglect the last integral in Eq. \eqref{fa22}, and to find the large time asymptotic of ${D}_{q}(t)$
we present it identically as
\begin{equation}
    {D}_{q}(t) = \int\limits_{C} \frac{dk}{2\pi} \int\limits_{C^*} \frac{dp}{2\pi}
    \frac{e^{it(k^2-p^2)}-1}{k^2-p^2} \frac{\bar{\omega}_p^{(0)}\omega_k^{(1)}-\bar{\omega}_p^{(1)}\omega_k^{(0)}}{p^2-q^2}\approx 
    -\int\limits_{C} \frac{dk}{2\pi} \int\limits_{C^*} \frac{dp}{2\pi}
    \frac{1}{k^2-p^2+i0} \frac{\bar{\omega}_p^{(0)}\omega_k^{(1)}-\bar{\omega}_p^{(1)}\omega_k^{(0)}}{p^2-q^2}
\end{equation}
}

\section{The current}

The full current $J(t)$ of the particles flowing {\blue through $x=0$  to the right part of the system} can be evaluated using connection with FCS given by Eq.~\eqref{Ftd} 
\begin{equation}\label{J}
    J(t) =\frac{d}{dt} \frac{d\mathcal{F}(\lambda,t)}{d\lambda}\Big|_{\lambda=0}  
    = \mathrm{Tr}\,\left(\rho \frac{d}{dt} \frac{d\hat K}{d\lambda}\Big|_{\lambda=0}\right) =
    -\int_0^\infty dq \rho(q) 
    \frac{4|\Phi_q(0)|^2 }{\pi}
      {\rm Im}\,  f^{(1)}_q(t) \bar{f}^{(0)}_{q}(t) ,
\end{equation}
where at the last step we used trace of Eq.~\eqref{dK} .
As was discussed after Eq.~\eqref{faRe}, the integral in Eq.~\eqref{faRe} may be dropped for the calculation of current for large $t$ since it does not give a leading contribution:
\begin{equation}\label{faap}
	f^{(\alpha)}_q(t) \approx F^{(\alpha)}_q e^{itq^2} + \sum_{n=1}^{N^\mathrm{b}} 
	 B_{n,q}^{(\alpha)} e^{-it\varkappa_n^2}.
\end{equation}
Substituting this expression into Eq.~\eqref{J} we obtain three type of contributions to the current
\begin{equation}
    J(t)\approx J_\mathrm{LB} + J^\mathrm{b} + \delta J,
\end{equation}
where $J_\mathrm{LB}$ comes from the first summand of \eqref{faap},
$J^\mathrm{b}$ comes from the summands corresponding to bound states and
$\delta J$ follows from mixing the first summand and the summands corresponding to bound states.

To calculate  $J_\mathrm{LB}$ we use ${\rm Im}\, \psi_q'(0)\bar{\psi}_q(0) = -q$
and Eq.~\eqref{tran}
\begin{equation}
    J_\mathrm{LB}=\int \limits_{0}^\infty \frac{dq }{\pi} \frac{q\rho(q)}{|a_q|^2} = 
    \int \frac{dE}{2\pi} \rho(E) T(E) .
\end{equation}
It is well-known Landauer--B\"uttiker formula for the current. 

The contribution of bound states to the current is
\begin{equation}
    J^\mathrm{b}=\sum_{m<n}
    A_{mn} \sin t(\varkappa_m^2-\varkappa_n^2),
\end{equation}
where 
\begin{equation}\label{Amn}
    A_{mn} =\frac{  4 \left(\bar\psi_{i\varkappa_n}'(0)\bar\psi_{i\varkappa_m}(0)-\bar\psi_{i\varkappa_m}'(0)\bar\psi_{i\varkappa_n}(0)\right) }{a'_{i\varkappa_m}a'_{i\varkappa_n}}
    \int_0^\infty \frac{dq}{\pi} \rho(q) 
    |\Phi_q(0)|^2 
    \frac{ \Xi_{q,i\varkappa_m}\Xi_{q,i\varkappa_n}}
    {(\varkappa_m^2+q^2)(\varkappa_n^2+q^2)}    .
\end{equation}
For an even potential $V(x)$, the bound states are either even functions with $\bar\psi_{i\varkappa_n}'(0)=0$ or odd functions with  
$\bar\psi_{i\varkappa_n}(0)=0$. Therefore, in this case, a nontrivial contribution to the current may arise only from pairs of odd-even states. Furthermore, in the case of perfect lead attachment, $V(x)=V_0(x)$, we have $\Xi_{q,i\varkappa_n}=0$ for odd bound states 
$\bar\psi_{i\varkappa_n}(x)$
and therefore there is no contribution at all to the current from bound states in the case of perfect lead attachment with an even potential. 

The integral in $q$ for $\delta J$ can be estimated by the
contribution at $q=0$ by the method of stationary phase and it can be shown that   $\delta J$ 
decays for large $t$ at least as $t^{-1/2}$ and therefore does not give a leading contribution to the current.

Finally the leading contribution to the current for large $t$ consists of
constant  Landauer--B\"uttiker current and an oscillating current
\begin{equation}\label{Jtot}
     J(t)\approx \int \frac{dE}{2\pi} \rho(E) T(E) +
     \sum_{m<n}
    A_{mn} \sin t(\varkappa_m^2-\varkappa_n^2).
\end{equation}




\section{Examples}

In this section we apply general results to three potentials: $V(x)=g\delta(x)$ and $V(x)=-\lambda(\lambda-1)\cosh^{-2} x$ with $V_0(x)=0$, 
and $V(x)=V_0(x)=g_1\delta(x-d_1)+g_2\delta(x-d_2)$.

\subsection{ $V(x)=g\delta(x)$, $V_0(x)=0$}

For the potential $V(x)=g\delta(x)$ the Jost solutions can be found using Eqs.~\eqref{psiint} and \eqref{phiint} 
\begin{equation}
    \psi_k(x) =  e^{-ikx} - \frac{g}{k} \theta(-x) \sin (kx),
\end{equation}
\begin{equation}
    \varphi_k(x) = e^{-ikx} +  \frac{g}{k} \theta(x)  \sin (kx),
\end{equation}
where $\theta(x)$ is Heaviside step function.
So the scattering data is
\begin{equation}\label{Tdelta}
    a_k = 1- \frac{g}{2ik},\qquad b_k = \frac{g}{2ik},\qquad 
    T(E)  = \frac{1}{|a_k|^2} = \frac{k^2}{k^2+g^2/4} = \frac{E}{E+g^2/4}.
\end{equation}
If $g<0$ there is also a bound state corresponding to zero of $a_k$ at 
$k=i\varkappa=-ig/2$
\begin{equation}
    \varphi_{i\varkappa}(x) = e^{-\varkappa |x|}, \qquad 
    \varkappa=|g|/2.
\end{equation}
The states corresponding to the initial potential $V_0(x)$ are
\begin{equation}
    \Phi_q(x)=e^{-iqx},\qquad \Lambda_q(x)=-\sin qx.
\end{equation}
It leads to $\Xi_{q,k}=\Lambda_q'(0)\varphi_k(0)=-q$.
Functions $f^{(\alpha)}_q(t)$ can be obtained without Green function and they equal
\begin{equation}
    f_q^{(1)}(t)=\frac{1}{2}qe^{itq^2},
\end{equation}
\begin{equation}
    f^{(0)}_q(t)=-\frac{1}{2}\frac{qe^{itq^2}}{iq+g/2}-\theta(-g)\frac{q\varkappa e^{-it\varkappa^2}}{\varkappa^2+q^2}+qE(q),
\end{equation}
where
\begin{equation}
	E(q) = \int\limits_0^\infty \frac{dp}{\pi} \frac{p^2 e^{itp^2}}{(p^2+\varkappa^2)((p+i0)^2-q^2)}=\frac{\varkappa \bar{f}_{\varkappa}(t)}{2(q^2+\varkappa^2)}
	- \frac{iq f_q(t)}{2(q^2+\varkappa^2)}.
\end{equation}
and
\begin{equation}
	f_q(t) = e^{itq^2}\left[1- {\rm Erf} \left(qe^{i\pi/4}\sqrt{t}\right)\right]
\end{equation}
Kernel of FCS up to $\rho(q)(e^\lambda-1)/\pi$ is given by the formula 
\begin{equation}
    2ie^{-it(q^2-q'^2)/2}\int_{0}^{t}d\tau\,\left\{f^{(1)}_{q}(\tau)\bar f^{(0)}_{q'}(\tau)-f^{(0)}_{q}(\tau)\bar f^{(1)}_{q'}(\tau)\right\}=X_0(q,q')+X_1(q,q'),
\end{equation}
where 
\begin{equation}
	X_0(q,q') =qq' \frac{2q}{\varkappa^2+q^2} \frac{\sin\left[t(q^2-q'^2)/2\right]}{q^2-q'^2},
\end{equation}
\begin{multline}
	X_1(q,q') = qq' \frac{\sin\left[t(q^2-q'^2)/2\right]}{q^2-q'^2} \left(\frac{q'}{\varkappa^2+q'^2}-\frac{q}{\varkappa^2+q^2}\right)
	-2q q'\mathrm{Im}\left(e^{-it(q^2+q'^2)/2}\frac{e(q)-e(q')}{q^2-q'^2}\right) \\
	+\frac{q q'}{(\varkappa^2+q^2)(\varkappa^2+q'^2)}\left\{\varkappa\mathrm{Re}\left(e^{it(q^2+q'^2)/2}f_{\varkappa}(t)\right)-\frac{g}{2}\cos\left[t(q^2-q'^2)/2\right]-2\theta(-g)\varkappa\cos \left[t(q^2+q'^2+2\varkappa^2)/2\right]\right\},
\end{multline}
and 
\begin{equation}
    e(q)=\frac{q f_q(t)}{2(q^2+\varkappa^2)}.
\end{equation}
Using that we obtain formula for FCS
\begin{equation}
	\mathcal{F} (\lambda,t) = \det \left(1 + \frac{e^\lambda-1}{\pi}\rho(q)X_0(q,q') +\frac{e^\lambda-1}{\pi}\rho(q)X_1(q,q')\right),
\end{equation}
\subsection{Two delta functions}
We consider the case when at the initial time $V_0=0$ on $[-R;0]$ and then a potential with two delta-functions on $[-R;R]$ turns on:  
	\begin{equation}\label{V2delta}
		V(x) = g_1 \delta(x-d_1)+g_2\delta(x-d_2),  
	\end{equation}
where we assume that $d_2>0>d_1$. The Jost solutions for this potential can be find using Eq.~\eqref{psiint} and Eq.~\eqref{phiint} 
	\begin{equation}
		\psi_k(x) = e^{-ik x} - \theta(d_1-x)\frac{\sin(k(x-d_1))}{k}g_1 \psi_k(d_1) - \theta(d_2-x)\frac{\sin(k(x-d_2))}{k}g_2 \psi_k(d_2),
	\end{equation}
	\begin{equation}
		\varphi_k(x) = e^{-ik x} 	+\theta(x-d_1)\frac{\sin(k(x-d_1))}{k}g_1 \varphi_k(d_1) +\theta(x-d_2)\frac{\sin(k(x-d_2))}{k}g_2 \varphi_k(d_2),
	\end{equation}
where 
	\begin{equation}
	\psi_k(d_1) = e^{-ikd_1 } \left(1 + \frac{g_2}{2ik} \right)  - \frac{g_2}{2ik}e^{ik(d_1-2d_2)}, \qquad 
		\psi_k(d_2)= e^{-ikd_2},
	\end{equation}
	\begin{equation}
		\varphi_k(d_1) = e^{-ikd_1}, \qquad 
		\varphi_k(d_2) =  e^{-ikd_2 } \left(1 -\frac{g_1}{2ik} \right)  +\frac{g_1}{2ik}e^{ik(d_2-2d_1)}.
	\end{equation}
The scattering data follows from Eq.~\eqref{transfer} 
	\begin{equation}\label{Tdelta2a}
		a_k = \frac{g_1 g_2 e^{-2 i k (d_1-d_2)}+(2 k+i g_1) (2 k+i g_2)}{4 k^2},
	\end{equation}
	\begin{equation}\label{Tdelta2b}
		b_k = \frac{g_2 e^{-2 i d_2 k} (g_1-2 i k)-g_1 e^{-2 i d_1 k} (g_2+2 i k)}{4 k^2}.
	\end{equation}

Another way to find scattering data for potential $V(x)$ with two delta-functions is to use a composition of transfer matrices corresponding to each 
of delta-functions. This approach works in more general case when we have  a disjoint potential $V(x)=V_1(x)+V_2(x)$ with
$V_1(x)=0$ for $x>x_1$ and $V_2(x)=0$ for $x<x_2$ for some  $x_1<x_2$. In this case we have 
	\begin{equation}
		\mathcal{T}=\mathcal{T}_1\mathcal{T}_2,
	\end{equation} 
where $\mathcal{T}_j$ is the transfer matrix for $V_j$ with corresponding Jost solutions $\psi_j$ and $\varphi_j$. It can be derived in the following way
\begin{equation}
	\begin{pmatrix}
		\varphi_1\\
		\bar{\varphi}_1
	\end{pmatrix}=	\mathcal{T}_1\begin{pmatrix}
	\psi_1\\
	\bar{\psi}_1
\end{pmatrix}=\mathcal{T}_1 \begin{pmatrix}
	\varphi_2\\
	\bar{\varphi}_2
\end{pmatrix}=\mathcal{T}_1\mathcal{T}_2\begin{pmatrix}
\psi_2\\
\bar{\psi}_2
\end{pmatrix}.
\end{equation}
Also we have to take into account that the transfer matrix $\tilde {\mathcal{T}}$ for the shifted potential $\tilde V(x)=V(x-d)$ is obtained from $\mathcal{T}$ 
by conjugation by a diagonal matrix
\begin{equation}
\tilde{\mathcal{T}}=\mathcal{T}(d)=\begin{pmatrix}
		a_k & b_k e^{-2ikd}\\
		\bar{b}_k e^{2ikd} & \bar{a}_k 
	\end{pmatrix}.
\end{equation}
Therefore, in particular, for the potential \eqref{V2delta} with two delta functions we have the transfer matrix
\begin{equation}
	\mathcal{T}=\mathcal{T}_{g_1}(d_1)\mathcal{T}_{g_2}(d_2), \qquad  
	\mathcal{T}_g(0)=\begin{pmatrix}
		1-\frac{g}{2ik} & \frac{g}{2ik}\\
		-\frac{g}{2ik} & 1+\frac{g}{2ik}
	\end{pmatrix},
\end{equation}
where we used Eq.~\eqref{Tdelta}. It gives the scattering data \eqref{Tdelta2a} and \eqref{Tdelta2b}.


The potential $V(x)$ can has one or two bound states depending on the values of parameters. For simplicity, let us consider the symmetric case of the potential, i.e.  
$g_1=g_2=g$, $d_2=-d_1=d/2$,
\begin{equation}\label{Vsym}
	V(x)=g\delta(x+d/2)+g\delta(x-d/2).
\end{equation} 
The bound states momenta follow from the relation $a_{i\varkappa}=0$ or explicitly
\begin{equation}\label{bound2d}
a_{i\varkappa}=	\frac{(u-1)^2 -e^{-u D}}{u^2}=0,
\end{equation}
where we introduced notations 
\begin{equation}
k=i\varkappa, \qquad
	u=2\varkappa/|g|, \qquad D=|g|d.
\end{equation}
The equation \eqref{bound2d} has two solutions for $D>2$ and one solution for $0\le D \le 2$. Note $a_k$ has a simple pole at $k=0$ if $D\ne 2$. The case $D=2$ corresponds to the 
situation when a bound state arises from the continuous spectrum and in this case $a_{k}$ is regular at $k=0$.
For nonsymmetric potential $V(x)$, the condition $D\gtrless 2$ is generalized to
\begin{equation}
	d_{2}-d_1 \gtrless \frac{1}{|g_1|}+\frac{1}{|g_2|}.
\end{equation}

In what follows we will need
\begin{equation}\label{Xi2delta}
	\Xi_{qk}=\Lambda'_q(0)\varphi_k(0)+\int_{-\infty}^{0}dx \Lambda_q(x)V(x)\varphi_k(x)=-q+g_1  e^{-ikd_1} \left(\frac{q}{k}\sin kd_1-\sin qd_1\right),
\end{equation}
where we used 
\begin{equation}
	\Lambda_q(x)=\mathrm{Im}\, \Phi_q(x)=-\sin qx.
\end{equation}

To compute current \eqref{J} we need to find $f^{(\alpha)}_q(t)$ given by Eq.~\eqref{faRe}. The most difficult part of computation is estimation of integrals
\begin{equation}
	I^{(\alpha)}_q(t)=\int\limits_{0}^{\infty} \frac{dk}{\pi} 
	\Omega^{(\alpha)}_{q,k}
	\frac{e^{itk^2}}{(k+i0)^2-q^2},
\end{equation}
where 
\begin{equation}
	\Omega^{(\alpha)}_{q,k}=
	\mathrm{Re}\,
	\frac{\Xi_{q,k}\partial_x^\alpha\bar\psi_k(0)}{a_k}.
\end{equation}
For symmetric potential \eqref{Vsym}  we have
\begin{equation}
	\Omega^{(0)}_{q,k}=\frac{2k^2(-q+g\cos kd/2\sin qd/2)}{g^2+2k^2+g^2\cos kd-2gk \sin kd},
\end{equation}
\begin{equation}
	\Omega^{(1)}_{q,k}=-\frac{2gk^3\sin kd/2\sin qd/2}{g^2+2k^2-g^2\cos kd+2gk \sin kd}.
\end{equation}
To find asymptotic behaviour of the integrals $I^{(\alpha)}_q(t)$ we need to know expansions of the integrands at $k=0$
\begin{equation}
	\Omega^{(0)}_{q,k}=\frac{k^2}{g^2}(-q+g\sin qd/2)+O(k^4), \quad \Omega^{(1)}_{q,k}=-\frac{2k^2gd\sin qd/2}{(2+gd)^2} + O(k^4).
\end{equation}
The formula for $\Omega^{(1)}_{q,k}$ is valid for $D=-gd\ne 2$. The asymptotic   behaviour of $\Omega^{(1)}_{q,k}$ for  $D=2$ and small $k$ is
\begin{equation}
	\Omega^{(1)}_{q,k}=\frac{4}{d^2}\sin \frac{qd}{2}+\frac{k^2}{18}\sin \frac{qd}{2}+O(k^4).
\end{equation} 
Therefore, the integrals have following decaying behavior for large $t$
\begin{equation}
	I^{(\alpha)}_q(t)\sim t^{-\frac{3}{2}}  \quad \mathrm{for} \quad D\ne 2,\quad  \mathrm{and} \quad I^{(\alpha)}_q(t)\sim t^{-\frac{3}{2}+\alpha} \quad \mathrm{for} \quad D=2.
\end{equation}
They will be neglected since they do not give a contribution to  the leading term of asymptotic current
 given by Eq.~\eqref{Jtot}. If the potential has two bound states than there is an oscillatory part of the current with the amplitude of oscillations
given by Eq.~\eqref{Amn}
\begin{equation}\label{A12d2}
A_{12} =\frac{  4 \left(\bar\psi_{i\varkappa_2}'(0)\bar\psi_{i\varkappa_1}(0)-\bar\psi_{i\varkappa_1}'(0)\bar\psi_{i\varkappa_2}(0)\right) }{a'_{i\varkappa_1}a'_{i\varkappa_2}}
\int_0^\infty \frac{dq}{\pi} \rho(q) 
|\Phi_q(0)|^2 
\frac{ \Xi_{q,i\varkappa_1}\Xi_{q,i\varkappa_2}}
{(\varkappa_1^2+q^2)(\varkappa_2^2+q^2)}    .
\end{equation}
For the symmetric potential \eqref{Vsym}, a simplified formula for Eq.~\eqref{A12d2} can be computed using Eq.~\eqref{Xi2delta} and
\begin{equation}
	a'_{i\varkappa_j}=\left.\frac{da}{dk}\right|_{k=i\varkappa_j}=-\frac{2i}{|g|}\frac{(u_j-1)(D(u_j-1)+2)}{u_j^2},
\end{equation}
\begin{equation}
	\bar\psi_{i\varkappa_1}(0)=2-2/u_1, \qquad  \bar\psi_{i\varkappa_2}(0)=0, \qquad 
	 \bar\psi_{i\varkappa_1}'(0)=0 , \qquad
	\bar\psi_{i\varkappa_2}'(0)=(1-u_2)|g|.
\end{equation}
Therefore for such potential we have
\begin{equation}
	A_{12}=	\frac{2u_1 u_2^2|g|^3}{(D(u_1-1)+2)(D(u_2-1)+2)}\int_0^\infty \frac{dq}{\pi} \rho(q)
	\frac{ \Xi_{q,i\varkappa_1}\Xi_{q,i\varkappa_2}}
	{(\varkappa_1^2+q^2)(\varkappa_2^2+q^2)}.
\end{equation}
Finally,  the leading contribution to the current for large $t$ consists of
constant  Landauer--B\"uttiker current and an oscillating current (if there are two bound states)
\begin{equation}\label{Jtot}
     J(t)= \int \frac{dE}{2\pi} \rho(E) T(E) +
         A_{12} \sin t(E_2-E_1)+O(t^{-1/2}),
\end{equation}
where $T(E)=|a_k|^{-2}$ is the transmission coefficient,  the energy of bound states $E_j=-\varkappa_j^2$ and $\varkappa_j$ are defined by the roots of Eq.~\eqref{bound2d}, 
the amplitude $A_{12}$ is given by Eq.~\eqref{A12d2}. 

.



 


\subsection{An example of reflectionless potential}

In this subsection we consider an example of perfect lead attachment, i.e. $V_0(x)=V(x)$, $x<0$, for the reflectionless potential 
\begin{equation}\label{Vcosh}
V(x)=-\frac{2}{\cosh^2 x}.
\end{equation}
The corresponding Jost solutions are 
\begin{equation}
    \psi_k(x)= e^{-ikx}\left(1+\frac{2i}{k-i}\frac{1}{e^{2x}+1}\right),
\end{equation}
\begin{equation}
    \varphi_k(x)=\bar\psi_k(-x)=e^{-ikx}\left(1-\frac{2i}{k+i}\frac{1}{e^{-2x}+1}\right)=\frac{k-i}{k+i} \psi_k(x)
\end{equation}
with  
\begin{equation}
    a_k=\frac{k-i}{k+i} , \qquad b_k=0.
\end{equation}
This potential has one bound state corresponding to the zero of $a_k$ at $k=i$:
\begin{equation}
   \chi_1^{\rm b}(x) = \varphi_{k=i}(x)=\frac{1}{2\cosh x}.
\end{equation}
Initial one-particle states are given by \eqref{lambda1}:
\begin{equation}
    \Lambda_q(x)=-\frac{q \sin qx +\tanh x\cos qx}{q}.
\end{equation}
Therefore $\Xi_{q,k}$ defined in \eqref{Xiqk} becomes
\begin{equation}
    \Xi_{q,k}= \Lambda_q'(0)\varphi_k(0) =-\frac{1+q^2}{q} \cdot \frac{k}{k+i}.
\end{equation}
We have 
\begin{equation}
    \frac{\Xi_{q,k}\bar{\psi}'_k(0)}{a_k}=-i\frac{1+q^2}{q} k.
\end{equation}
Therefore the integral in \eqref{faRe} for $\alpha=1$ does not give a contribution. Since for the the  bound state 
$(\chi_1^{\rm b})'(0)=0$,  there is no contribution of bound states   
to $f^{(1)}_q(t)$. Finally we obtain
\begin{equation}
    f^{(1)}_q(t)=-\frac{1+q^2}{2q}e^{itq^2}.
\end{equation}

Similarly we have 
\begin{equation}
    \frac{\Xi_{q,k}\bar{\psi}_k(0)}{a_k}=-\frac{1+q^2}{q} \frac{k^2}{1+k^2}, \qquad \frac{i\Xi_{q,k}\bar{\psi}_k(0)}{a'_k}=-\frac{k^2}{2q}.
\end{equation}
Therefore 
\begin{equation}
    f^{(0)}_q(t) =  \left.\frac{i\Xi_{q,k}\bar{\psi}_k(0)e^{itk^2}}{a'_k}\right|_{k=i}
    -i\frac{\psi_q(0)e^{itq^2}}{2\varphi_q(0) a_{-q}} +  \int\limits_0^\infty \frac{dk}{\pi} {\rm Re} \left[
    \frac{\Xi_{q,k}\bar{\psi}_k(0)}{a_k}
    \right] \frac{e^{itk^2}}{(k+i0)^2-q^2}
\end{equation}
becomes
\begin{equation}\label{GqCosh}
    f^{(0)}_q(t) =  \frac{e^{-it }}    {2q}
    +\frac{e^{itq^2}}{2i} 
    -\frac{1+q^2}{q} 
    \int\limits_0^\infty \frac{dk}{\pi}  \frac{k^2}{1+k^2} \frac{e^{itk^2}}{(k+i0)^2-q^2}
\end{equation}
\begin{equation}
    \approx \frac{e^{-it }}    {2q}
    +\frac{e^{itq^2}}{2i}+\frac{1+q^2}{q^3}\frac{e^{3\pi i/4}}{4\sqrt{\pi}} t^{-\frac32},\qquad t\to \infty,
\end{equation}
where the asymptotic behaviour is given for fixed $q>0$.
The current is given by formula \eqref{J}
\begin{equation}
    J (t)= 
    -\int_0^\infty dq \rho(q) 
    \frac{4|\varphi_q(0)|^2 }{\pi}
      {\rm Im}\,  f^{(1)}_q(t) \bar{f}^{(0)}_{q}(t) .
\end{equation}
Using expressions for $f^{(1)}_q(t)$ and $f^{(0)}_q(t)$ we obtain 
\begin{equation}
    J(t) =  \int_0^\infty \frac{dq}{\pi} \rho(q)\left(q+\sin(1+q^2)t+2(1+q^2)\mathrm{Im}\int\limits_0^\infty \frac{dk}{\pi}  \frac{k^2}{1+k^2} \frac{e^{it(k^2-q^2)}}{(k+i0)^2-q^2}\right).
\end{equation}
The integral in $q$ of the second term is decreasing as $1/\sqrt{t}$, which can be shown by the method of stationary phase.
The contribution of the third term is even smaller for large $t$.  So, making substitution from momenta to energy,  we obtain the Landauer--B\"uttiker current 
for reflectionless potential \eqref{Vcosh}
\begin{equation}
    J=\int_0^\infty \frac{dE}{2\pi} \rho(E)+O(t^{-\frac{1}{2}}).
\end{equation}

\subsubsection{P\"oschl--Teller potential}

Here we follows notations of [Flugge Problem 39].

Namely, we consider Shr\"odinger equation for P\"oschl--Teller potential
\begin{equation}
    u''+ \left(k^2 + \frac{\lambda(\lambda-1)}{\cosh(x)^2} \right)u=0.
\end{equation}
The generic solution reads 
\begin{equation}
    u = A \cosh^\lambda(x) F(a_+,a_-,1/2,-\sinh^2(x)) + B  \cosh^\lambda(x) \sinh(x) F(a_++1/2,a_-+1/2,3/2,-\sinh^2(x)) 
\end{equation}
with 
\begin{equation}
    a_\pm = \frac{\lambda \pm ik}{2} .
\end{equation}
Taking into account 15.8.2 from [https://dlmf.nist.gov/15.8], we present at for large $x$
\begin{multline}
    u(x) \approx A \cosh^\lambda(x)
    \left(
    \frac{\Gamma(1/2)\Gamma(a_--a_+)[\sinh^2(x)]^{-a_+}}{\Gamma(a_-)\Gamma(1/2-a_+)} +
    \frac{\Gamma(1/2)\Gamma(a_+-a_-)[\sinh^2(x)]^{-a_-}}{\Gamma(a_+)\Gamma(1/2-a_-)} 
    \right) \\ 
    +B \sinh(x) \cosh^\lambda(x)\left(
    \frac{\Gamma(3/2)\Gamma(a_--a_+)[\sinh^2(x)]^{-a_+-1/2}}{\Gamma(a_-+1/2)\Gamma(1-a_+)} +
    \frac{\Gamma(3/2)\Gamma(a_+-a_-)[\sinh^2(x)]^{-a_--1/2}}{\Gamma(a_++1/2)\Gamma(1-a_-)} 
    \right).
\end{multline}
This gives at $x\to + \infty$
\begin{multline}
    u(x) \approx A\sqrt{\pi}
    \left(
    \frac{\Gamma(a_--a_+)e^{-ikx+ik\ln 2}}{\Gamma(a_-)\Gamma(1/2-a_+)} +
    \frac{\Gamma(a_+-a_-)e^{ikx-ik\ln 2}}{\Gamma(a_+)\Gamma(1/2-a_-)} 
    \right) \\ 
    +\frac{ B \sqrt{\pi}}{2} \left(
    \frac{\Gamma(a_--a_+)e^{-ikx+ik\ln 2}}{\Gamma(a_-+1/2)\Gamma(1-a_+)} +
    \frac{\Gamma(a_+-a_-)e^{ikx-ik\ln 2}}{\Gamma(a_++1/2)\Gamma(1-a_-)} 
    \right)
\end{multline}
and $x\to - \infty$
\begin{multline}
	u(x) \approx A\sqrt{\pi}
	\left(
	\frac{\Gamma(a_--a_+)e^{ikx+ik\ln 2}}{\Gamma(a_-)\Gamma(1/2-a_+)} +
	\frac{\Gamma(a_+-a_-)e^{-ikx-ik\ln 2}}{\Gamma(a_+)\Gamma(1/2-a_-)} 
	\right)  \\ -\frac{ B \sqrt{\pi}}{2} \left(
	\frac{\Gamma(a_--a_+)e^{ikx+ik\ln 2}}{\Gamma(a_-+1/2)\Gamma(1-a_+)} +
	\frac{\Gamma(a_+-a_-)e^{-ikx-ik\ln 2}}{\Gamma(a_++1/2)\Gamma(1-a_-)} 
	\right).
\end{multline}
So for the Jost solution $\varphi_k(x)$ we have to choose the following constants
\begin{eqnarray}
    A^\varphi  = \frac{\cosh(2\pi k)-\cos(2\pi \lambda)}{(2\pi)^2\sqrt{\pi}2^{ik}}\Gamma(1-ik)\Gamma(2a_+)\Gamma(1-2a_-) \Gamma(a_-)\Gamma(1/2-a_+)\equiv \alpha(k),\\ B^\varphi = 
      2\frac{\cosh(2\pi k)-\cos(2\pi \lambda)}{(2\pi)^2\sqrt{\pi}2^{ik}}\Gamma(1-ik)\Gamma(2a_+)\Gamma(1-2a_-) \Gamma(1-a_+)\Gamma(1/2+a_-)\equiv \beta(k).
\end{eqnarray}
And similarly for $\psi_k(x)$
\begin{equation}
	A^\psi =  \alpha(-k),\qquad B^\psi = -\beta(-k).
\end{equation}
The transmission and reflection coefficients can be expressed as 
\begin{equation}
	a_k = \frac{2\alpha(k)\beta(k)}{\alpha(-k)\beta(k)-\alpha(k)\beta(-k)} = \frac{\Gamma (1-i k) \Gamma (-i k)}{\Gamma (1-i k-\lambda ) \Gamma (\lambda -i k)},\qquad 
	b_k = {\blue -} \frac{\alpha(-k)\beta(k)+\alpha(k)\beta(-k)}{\alpha(-k)\beta(k)-\alpha(k)\beta(-k)} = -i\frac{\sin(\pi \lambda)}{\sinh(\pi k)}.
\end{equation}
Therefore
\begin{equation}
	\frac{\varphi_k(0)\partial_x \psi_{-k}(0)}{a_k} = -\frac{\alpha(k)\beta(k)}{a_k} = ik,
\end{equation}
\begin{equation}
	\frac{\varphi_k(0)\psi_{-k}(0)}{a_k} =  \frac{\alpha(k)^2}{a_k}  = -\frac{i k \Gamma \left(\frac{1}{2} (1-i k-\lambda )\right) \Gamma \left(\frac{1}{2} (\lambda -i k)\right)}{2 \Gamma \left(1-\frac{i k}{2}-\frac{\lambda }{2}\right) \Gamma \left(\frac{1}{2} (1+\lambda -i k)\right)}.
\end{equation}
Note, for arbitrary even potential $V(-x)=V(x)$, the Jost solutions are related as $\psi_{-k}(x)=\varphi_k(-x)$.
Taking into account that the Wronskian $\varphi_k(x) \partial_x \psi_{-k}(x) - \psi_{-k}(x) \partial_x \varphi_k(x)$ does not depend on $x$ and calculating it at $x\to -\infty $ and $x=0$
we obtain the relation
\begin{equation}
	\frac{\varphi_k(0)\partial_x \psi_{-k}(0)}{a_k} = ik.
\end{equation}




\begin{acknowledgments}
The authors acknowledge support by the National Research Foundation of Ukraine grant 2020.02/0296.
Y.Z. and N.I. were partially supported by NAS of Ukraine (project No. 0122U000888).
O. G. also acknowledges support from the Polish National Agency for Academic
Exchange (NAWA) through the Grant No. PPN/ULM/2020/1/00247.

\end{acknowledgments}

\appendix

\section{Green's function calculation}
\label{appG}
In this appendix we compute the thermodynamic limit of the Green's function $G(x,y,t)$ defined as 
\begin{equation}
    G^*(x,y,t) \equiv \sum\limits_k  \frac{\chi_k(x)\chi_k(y)}{(\chi_k,\chi_k)} e^{itE_k},\qquad t\ge 0. 
\end{equation}
Here summation is taken over all solution of the spectrum condition \eqref{sp23}. 
For a moment we focus on the case when bound states are absent in the spectrum. 
Using notations for $\chi_k$ in Eq. \eqref{chikk}, the norm \eqref{norm} and the phase \eqref{sp23}.
We present for the one particular choice of the square root
\begin{equation}
    \frac{\chi_k(x)\chi_k(y)}{(\chi_k,\chi_k)}  = \frac{1}{2(R+\delta'(k))}{\rm Re} \frac{\varphi_k(x)\bar{\psi}_k(y)}{a_k} - \frac{{\rm Re}Z_k(x,y)}{2(R+\delta'(k))\sqrt{1+({\rm Re}b_k)^2}}
\end{equation}
with 
\begin{equation}
   Z_k(x,y) = \frac{\psi_k(x)\psi_k(y)}{\bar a_k} + \frac{{\rm Re} b_k}{\bar{a}_k} \bar\varphi_{k}(x) \psi_k(y) 
\end{equation}
To evaluate the sum over $k$ we first notice that the norm \eqref{norm} can be presented as a derivative of the spectrum condition \eqref{sp23} 
\begin{equation}
     (\chi_k,\chi_k) =({\rm Re} b_k + \sqrt{1 + ({\rm Re} b_k)^2})\sqrt{1 + ({\rm Re} b_k)^2}  \frac{\partial_k [e^{2ikR+2i\delta(k)}-1]}{2i}.
\end{equation}
Further we employ the residue theorem in the following form
 \begin{equation}
     \sum_k \frac{F(k)}{\partial_k S(k)} = \frac{1}{2\pi i} \oint_\gamma  dk \frac{F(k)}{S(k)} 
 \end{equation}
where summation is over all solution of the equation $S(k)=0$ and the contour $\gamma$ runs around these values only and avoids any singularities of the function $F(k)$.  
This way we identically present 
\begin{equation}
    G^*(x,y,t) = \oint_\gamma \frac{dk}{2\pi} \frac{e^{itE_k}}{e^{2ikR + 2i\delta_+(k)}-1}\left(
    {\rm Re} \frac{\varphi_k(x)\bar{\psi}_k(y)}{a_k} - \frac{{\rm Re}Z_k(x,y)}{\sqrt{1+({\rm Re}b_k)^2}}
    \right)+ \left( \delta_+ \to \delta_- \right),
\end{equation}
where by $\delta_\pm$ we mean terms that are obtained by the flip of the sign  $\sqrt{1+({\rm Re}b_k)^2}$, specifically for the solutions of \eqref{sp23}
\begin{equation}
     \frac{i {\rm Im}b_k+\sqrt{1+({\rm Re}b_k)^2}}{\bar{a}_k} \equiv e^{-2i\delta_{\rm}(k)}.
\end{equation}
The contour $\gamma$ encompasses all solutions of $e^{2ikR + 2i\delta(k)}=1$. We can present it as two contours below and above the real axes oriented in the positive and negative directions correspondingly. 
In the thermodynamic limit (with exponential accuracy) we notice that only the contour above the real line contributes therefore we can present 
\begin{equation}
     G(x,y,t) =  \int\limits_{0}^{\infty} \frac{dk}{\pi} e^{itE_k}{\rm Re} \frac{\varphi_k(x)\bar{\psi}_k(y)}{a_k} .
\end{equation}
Here we have taken into account that upon the summation $Z_k(x,y)$ terms cancel out. 
Identically we can present, 
\begin{equation}
    G(x,y,t) =  \int\limits_{-\infty}^{\infty} \frac{dk}{2\pi} e^{itE_k}\frac{\varphi_k(x)\bar{\psi}_k(y)}{a_k} 
\end{equation}
Notice that the the function that we integrate can be analytically continued to the upper half plane. 
This allows us to write the general answer in the case when bound states are present in the system 
\begin{equation}
    G(x,y,t) =  \int\limits_{C} \frac{dk}{2\pi} e^{itE_k}\frac{\varphi_k(x)\bar{\psi}_k(y)}{a_k} 
\end{equation}
where the contour lies in the upper half above all positions of the bound states and connects $-\infty$ and $+\infty$. 

\section{$f^{(0)}(t)$ evaluation}
\label{appF}
In this appendix we demonstrate how to rigorously evaluate $f_q^{(0)}(t)$ defined in \eqref{bFq}, namely  
\begin{equation}
    f^{(0)}_q (t) = \sum_k \frac{(\Lambda_q,\chi_k) \chi_k(0)}{(\chi_k,\chi_k)}e^{itE_k}
\end{equation}
The evaluation for $f^{(1)}(t)$ would be analogous. The main formal problem is that the overlap $(\Lambda_q,\chi_k)$ is singular on a real line, therefore the trick with the summation introduced in the appendix \eqref{appG} requires small modifications in the part choosing the integration contours. More precisely to describe the singularity we assume that without loss of generality the eigenvalues of $\Lambda_q$ and $\chi_k$ are different so the corresponding overlap could be found from 
\begin{equation}
    (k^2-q^2) (\Lambda_q,\chi_k) = \Lambda_q'(0)\chi_k(0)  - \int\limits_{-R}^0 dx \Lambda_q(x) (V_0(x) - V(x))\chi_k(x), 
\end{equation}
where we have used boundary conditions \eqref{eq1} and \eqref{eq2}. This way, we present, 
\begin{equation}
    (\Lambda_q,\chi_k) = \frac{{\rm Im} \left(e^{-i\delta(k)} \Xi^\psi_{q,k}\right)}{k^2-q^2},
\end{equation}
\begin{equation}
	\Xi^\psi_{q,k} =  \Lambda_q'(0)\psi_k(0)  - \int\limits_{-\infty}^0 dx \Lambda_q(x) (V_0(x) - V(x))\psi_k(x).
\end{equation}
Notice that here we have replaced the lower integration boundary from $-R$ to $-\infty$, which is possible due to the
finite range of the potential. 
Moreover, in this expression the dependence of the momenta $k$ and $q$ is smooth, so in particular the limit as $q\to k$ is well defined, contrary to the overall overlap, where special care has to be taken to the numerator. In particular, one can drop the quantization conditions for $k$ and consider a limit $k\to q$  
\begin{equation}
	\Xi^\psi_{q,q} =  \Lambda_q'(0)\psi_q(0)- \int\limits_{-\infty}^0 dx \Lambda_q(x) (V_0(x) - V(x))\psi_q(x). 
\end{equation}
To evaluate this expression we notice that for unrestricted momentum $k$ we can write 
\begin{equation}
    (k^2 -q^2) \int\limits_{-R}^0 dx \Lambda_q(x) \psi_k(x) =  \Lambda_q'(0)\psi_k(0)- \Lambda_q'(-R)\psi_k(-R)- \int\limits_{-\infty}^0 dx \Lambda_q(x) (V_0(x) - V(x))\psi_k(x),
\end{equation}
which in the limit $k\to q$ leads to 
\begin{equation}\label{ChiQQ}
    \Xi_{q,q} =  \Lambda_q'(-R)\psi_q(-R) = -\frac{q e^{iqR}}{\Phi_q(0)}\left(\bar a_q e^{iqR}- b_q e^{-iqR}\right)=- \frac{q}{\Phi_q(0)} (\bar{a}_q e^{-2i\eta(q)}- b_q).
\end{equation}
Here at the last step we have used Eq. \eqref{sp44}. 

With all these notations the function $f^{(0)}_q(t)$ can be presented as 
\begin{equation}\label{fg1}
    f^{(0)}_q(t) = \sum_k \frac{ {\rm Im} (e^{-i\delta(k)}\Xi_{q,k}){\rm Im} (e^{-i\delta(k)}\psi_k(0))}{(k^2-q^2)(\chi_k,\chi_k)}e^{itE_k}.
    % \label{fg2}
\end{equation}
We are going to evaluate the sum in Eq. \eqref{fg1} in the thermodynamic limit by presenting it as a contour integral in a way similar to appendix\eqref{appG}
\begin{equation}
    f^{(0)}_q(t) =  \oint_\gamma \frac{dk}{\pi} \frac{ {\rm Im} (e^{-i\delta(k)}\Xi_{q,k}){\rm Im} (e^{-i\delta(k)}\psi_k(0))}{(k^2-q^2)({\rm Re} b_k + \sqrt{1 + ({\rm Re} b_k)^2})\sqrt{1 + ({\rm Re} b_k)^2} }\frac{e^{itk^2}}{e^{2ikR+2i\delta(k)}-1}.
\end{equation}
Here contour $\gamma$ runs only around all positive solutions of the equation $e^{2ikR+2i\delta(k)}=1$. And summation over two branches  of the square root in \eqref{sp23}  $\delta=\delta_{\pm}$ is assumed.
The contour $\gamma$ can be deformed into two contours above and below real line. But contrary to \eqref{appG} we have to subtract
contribution from the point $k=q$, therefore we can present $f^{(\alpha)}_q(t)$ as 
\begin{equation}
    f^{(0)}_q(t) =\hat{f}^{(0)}_q(t) - f^{(0,+)}_q(t) + f^{(0,-)}_q(t)  
\end{equation}
where 
\begin{equation}
   \hat{f}^{(0)}_q(t) = - i \frac{ {\rm Im} (e^{-i\delta(q)}\Xi^\psi_{q,q}){\rm Im}\, (e^{-i\delta(q)}\psi_q(0))}{q({\rm Re}\, b_q + \sqrt{1 + ({\rm Re}\, b_q)^2})\sqrt{1 + ({\rm Re}\, b_q)^2} }\frac{e^{itq^2}}{e^{2i(\delta(q)-\eta(q))}-1}
\end{equation}
and 
\begin{equation}
     f^{(0,\pm)}_q(t) =  \int_0^\infty\frac{dk}{\pi} \frac{ {\rm Im} (e^{-i\delta(k)}\Xi_{q,k}){\rm Im} (e^{-i\delta(k)}\psi_k(0))}{((k\pm i0)^2-q^2)({\rm Re}\, b_k + \sqrt{1 + ({\rm Re}\, b_k)^2})\sqrt{1 + ({\rm Re}\, b_k)^2} }
    \frac{e^{itk^2}}{e^{2i(k\pm i0)R+2i\delta(k)}-1}.
\end{equation}
In the first equality we have used that point $q$ corresponds to the spectrum of the prequench spectrum \eqref{sp44}. 
So far these transformations are exact. Further we address the large system size limit $R\to \infty$.
In this limit $ f^{(0,-)}_q\to 0$ vanishes and the computation for $f^{(0,+)}_q(t) $ is identical to $G^*$ in the previous appendix \eqref{appG}. 
\begin{equation}
    f^{(0,+)}_q(t)  = -\int_0^\infty\frac{dk}{\pi} 
    \frac{  {\rm Re}\left[\Xi^\varphi_{q,k}\partial_x^\alpha\bar{\psi}_k(0)a^{-1}_k \right] e^{itk^2}}{(k+ i0)^2-q^2},
\end{equation}
To compute the residue contribution we, first, use Eq. \eqref{sp23} to present 
\begin{equation}
    \frac{1}{e^{2i(\delta(q)-\eta(q))}-1} = \frac{\sqrt{1+({\rm Re}\, b_q)^2}+i {\rm Im}\, b_q + a_q e^{2i\eta(q)}}{-2i{\rm Im} [a_q e^{2i\eta(q)}+b_q]},
\end{equation}
and then peform summation over all branches of the square root to obtain
\begin{equation}
       \hat{f}^{(0)}_q(t) = - \frac{1}{2q}  \frac{{\rm Re} \left[\Xi^\psi_{q,q}\partial_x^\alpha\psi_q(0)\bar{a}^{-1}_q\right]
       -(a_q e^{2i\eta(q)}-\bar{b}_q){\rm Re}\left[\Xi^\varphi_{q,q}\partial_x^\alpha\bar{\psi}_q(0)a^{-1}_q \right]}{{\rm Im} [a_q e^{2i\eta(q)}-\bar{b}_q]} e^{itq^2}.
       \end{equation}
Here  we have introduced
\begin{equation}\label{Xiphi}
    \Xi^\varphi_{q,k} \equiv a_k \Xi^\psi_{q,k}+ b_k\bar{\Xi}^\psi_{q,k} = \Lambda_q'(0)\varphi_k(0) -\int\limits_{-\infty}^0 dx \Lambda_q(x) (V_0(x) - V(x))\varphi_k(x),
    \end{equation}
which coincides with  $\Xi_{q,k}$ in (..). 
The diagonal component can be obtained from \eqref{ChiQQ}, 
\begin{equation}\label{tt12}
    \Xi^\varphi_{q,q} = - \frac{q}{\bar{\Phi}_q(0)},
\end{equation}
which allows us to significantly simplify expression for $\hat{f}^{(\alpha)}_q$
\begin{equation}\label{Fq}
    f^{(\alpha)}_q(t) =   \frac{\partial_x^\alpha\psi_q(0)e^{itq^2}}{2i \bar{a}_q \Phi_q(0)} +  \int\limits_0^\infty \frac{dk}{\pi} {\rm Re} \left[
    \frac{\bar{\Xi}^\varphi_{q,k}\partial_x^\alpha\psi_k(0)}{\bar{a}_k}
    \right] \frac{e^{itk^2}}{(k+i0)^2-q^2}.
\end{equation}
Notice that extending the  integration over $k$ to the negative values we can also present 
\begin{equation}\label{ffqqq}
    f^{(\alpha)}_q(t) =  \int\limits_{-\infty}^\infty \frac{dk}{2\pi} 
    \frac{\Xi^\varphi_{q,k}\partial_x^\alpha\bar\psi_k(0)}{a_k}
   \frac{e^{itk^2}}{(k+i0)^2-q^2}.
\end{equation} 

Now let us discuss on how to account for the bound states. As we discussed in the previous chapter the corresponding wave function as a Jost function analytically continued to the 
upper half plane and evaluated at the purely imaginary momenta $\chi_n^{\rm bound}(x) = \varphi_{i\varkappa_n}(x)$. 
The contributions from the bound states modify Eq. \eqref{ffqqq} as follows 
\begin{equation}\label{fa1}
	f^{(\alpha)}_q(t) = \sum\limits_{n=1}^{N^{\rm b}} \frac{(\Lambda_q,\varphi_{i\varkappa_n})\partial_x^\alpha\varphi_{i\varkappa_n}(0)}{(\varphi_{i\varkappa_n},\varphi_{i\varkappa_n})}e^{-it \varkappa_n^2}+ \int\limits_{-\infty}^\infty \frac{dk}{2\pi} 
	\frac{\Xi^\varphi_{q,k}\partial_x^\alpha\bar\psi_k(0)}{a_k}
	\frac{e^{itk^2}}{(k+i0)^2-q^2}
\end{equation}
Using the normalization Eq. \eqref{NormBound} and the relation $\varphi_{i\varkappa} = b_{\varkappa} \bar{\psi}_{i\varkappa}$, we see that we can present $f^{(\alpha)}_q$ in the following way 
\begin{equation}\label{fa2}
	f^{(\alpha)}_q(t) =  \int\limits_C \frac{dk}{2\pi} 
	\frac{\Xi^\varphi_{q,k}\partial_x^\alpha\bar\psi_k(0)}{a_k}
	\frac{e^{itk^2}}{k^2-q^2},
\end{equation}
where the contour $C$ runs from $-\infty$ to $+\infty$ and lies in the upper-half plane above all the zeroes of $a_k$, this way, in particular passing the poles $k=\pm q$ from the above. 
{\blue Notice that this presentation makes it obvious that $f^{(\alpha)}_q(0)=0$, so the current \eqref{curJ} vanishes in the initial moment of time.} 


\section{Additional information about scattering data}
\label{scatter}
To describe connection of the Jost solutions with scattering states it is useful to introduce $T$-matrix 
\be
T_{Qk} = \int \frac{dy}{\sqrt{2\pi}}e^{-iQy}V(y)\Psi_k^s(y)
\ee
Then we will have
\be
\Psi_k^s(x)  = \frac{e^{ikx}}{\sqrt{2\pi}}+ \int dQ \frac{e^{iQx}}{\sqrt{2\pi}}\frac{T_{Qk}}{k^2/2-Q^2/2+i0}
\ee
for all $k$. 
If we define 
\be
V_q = \int \frac{dx}{2\pi} e^{-i q x}V(x)
\ee
then
\be\label{TQk}
T_{Qk} = V_{Q+k} + \int dq \frac{T_{qk}}{k^2/2-q^2/2+i0}V_{Q+q}.
\ee
For $k>0$ the asymptotic are  
\be
\Psi_k^s(x\to +\infty) = \frac{e^{ikx}}{\sqrt{2\pi}}\left(1+\frac{2\pi}{ik}T_{k,k}\right),\,\,\,\,\,\,\,\,
\Psi_k^s(x\to -\infty) = \frac{e^{ikx}}{\sqrt{2\pi}} + \frac{e^{-ikx}}{\sqrt{2\pi}} \frac{2\pi}{ik}T_{-k,k}
\ee
so we may identify
\be
\Psi_k^s(x) = \frac{\psi(x,-k)}{a_k\sqrt{2\pi}},\,\,\,\,\,\,\,\,\, k>0
\ee
and 
\be\label{kg0}
a_k = \frac{1}{1+ \frac{2\pi}{ik}T_{k,k}},\,\,\,\,\,\,\, \bar{b}_k = - \frac{\frac{2\pi}{ik}T_{-k,k}}{1+ \frac{2\pi}{ik}T_{k,k}}
\ee
For $k<0$ we have 
\be 
\Psi_k^s(x\to +\infty) = \frac{e^{ikx}}{\sqrt{2\pi}} + \frac{e^{-ikx}}{\sqrt{2\pi}}\frac{2\pi}{(-ik)}T_{-k,k},\,\,\,\,\,\, 
\Psi_k^s(x\to -\infty) = \frac{e^{ikx}}{\sqrt{2\pi}} \left(1 +  \frac{2\pi}{(-ik)}T_{k,k} \right)
\ee
So we may identify 
\be\label{kl0}
a_{-k} = \frac{1}{1 +  \frac{2\pi}{(-ik)}T_{k,k}},\,\,\,\,\,\,\,\, b_{-k} = \frac{\frac{2\pi}{(-ik)}T_{-k,k}}{1 +  \frac{2\pi}{(-ik)}T_{k,k}}
\ee
In particular we see that for $k>0$
\be\label{tt1}
T_{k,k} = T_{-k,-k},\,\,\,\,\,\,\,\,\, T_{k,-k}\left(1-\frac{2\pi}{ik}\bar{T}_{k,k}\right) = \bar{T}_{-k,k}\left(1+ \frac{2\pi}{ik}T_{k,k}\right)
\ee
and unitarity for $k>0$ gives
\be\label{tt2}
|a_k|^2-|b_k|^2=1\Longrightarrow T_{k,k}-\bar{T}_{k,k} - \frac{2\pi}{ik} \left(T_{-k,k}\bar{T}_{-k,k}+T_{k,k}\bar{T}_{k,k}\right)=0
\ee
Note that both relations \eqref{tt1} and \eqref{tt2} follows from the general expression \eqref{unit}.
In total we have 
\begin{equation}\label{rel1}
\Psi^s_k(x) = \theta(k>0) \frac{\psi_{-k}(x)}{a_k\sqrt{2\pi}}+ \theta(k<0)\frac{\varphi_{-k}(x)}{\bar{a}_k\sqrt{2\pi}}
\end{equation}
We can invert this relation to obtain
\be
\psi_k = \theta(k>0) \sqrt{2\pi}(\Psi^s_{-k} -b_k\Psi_k^s)+ \theta(k<0) \sqrt{2\pi}a_{-k}\Psi^s_{-k}
\ee
\be
\varphi_k =\theta(k>0)\sqrt{2\pi}a_k\Psi^s_{-k} + \theta(k<0)\sqrt{2\pi} (\Psi^s_{-k}+b_{k}\Psi_k^{+})
\ee
From Eqs. \eqref{psiint}, \eqref{phiint} one can show that $a_k$ is an analytic function in the upperhalf plane and has simple zeroes at points $k_n=i\varkappa_n$ that correspond to the bound states. 
Taking into account its asymptotic behaviour $a_k \to 1$ as $|k|\to \infty,\,\,\, {\rm Im} k>0$ one can easily show that the most generic form for $a_k$ is

The exponent can be transformed as
\be
\frac{1}{2\pi i}\int\limits_{-\infty}^{\infty}\frac{\log (1+|b_q|^2)}{q-k-i0} = \log|a_k| + \frac{{\rm v.p.}}{2\pi i}\int\limits_{-\infty}^{\infty}\frac{\log (1+|b_q|^2)}{q-k}
\ee
Further we can write 
\be\label{phasea}
\frac{{\rm v.p.}}{2\pi i}\int\limits_{-\infty}^{\infty}\frac{\log (1+|b_q|^2)}{q-k} = 
\frac{1}{2\pi i}\int\limits_{0}^{\infty}\frac{dq}{q}\log \frac{1+|b_{q+k}|^2}{1+|b_{q-k}|^2}
\ee
This form in particular will help us to establish $k\to 0$ behaviour. 
Indeed, first we rewrite \eqref{psiint} for large negative $x$ as
\begin{equation}
    \psi_k(x) = e^{-ikx} \left(1+ \frac{1}{ik}\int\limits_{-\infty}^\infty e^{iky}V(y) \psi_k(y) dy \right)
    - \frac{e^{ikx}}{ik} \int\limits_{-\infty}^\infty e^{-iky}V(y) \psi_k(y) dy 
\end{equation}
This form together with the definition of the scattering data \eqref{transfer} allows us to conclude that 
\begin{equation} \label{scatD1}
    \bar{a}_k = 1+ \frac{1}{ik}\int\limits_{-\infty}^\infty e^{iky}V(y) \psi_k(y) dy ,\qquad 
    b_k = \frac{1}{ik} \int\limits_{-\infty}^\infty e^{-iky}V(y) \psi_k(y) dy 
\end{equation}
Similarly, we can obtain 
\be
a_k = 1-\frac{1}{ik}\int\limits_{-\infty}^\infty e^{iky}V(y)\varphi_k(y)dy,\qquad 
b_k = \frac{1}{ik}\int\limits^\infty_{-\infty} e^{-iky}V(y)\varphi_k(y)dy
\ee

Using expressions these expressions in the $k\to 0$ limit, we obtain
\be
b_k \overset{k\to 0}{\approx} \frac{iC}{k},\,\,\,\,\,\, a_k \overset{k\to 0}{\approx} -\frac{iC}{k}
\ee
Now using presentation \eqref{a} with \eqref{phasea} as $k\to 0$ we get
\be
\frac{1}{2\pi i}\int\limits_{0}^{\infty}\frac{dq}{q}\log \frac{1+|b_{q+k}|^2}{1+|b_{q-k}|^2} \approx \frac{1}{2\pi i}\int\limits_{0}^{\infty}\frac{dq}{q}\log \frac{(k-q)^2}{(k+q)^2} = 
{\rm sgn}(k) \frac{\pi i}{2	}.
\ee
Therefore,
\be\label{kto0}
a_k \overset{k\to 0}{=} i(-1)^N\frac{|C|}{k}.
\ee
So we conclude that $C$ is real and its sign is defined by the number of bound states
\be
\frac{C}{|C|} = (-1)^{N-1}.
\ee


\textbf{Symmetric potential.} 

\section{Evaluation of $\Xi^\varphi_{q,q}$}\label{pp}

In this section we prove result \eqref{tt12}. First we notice that from the integral presentation for the Jost solutions $\Phi_q$
\begin{equation}
	\Phi_q(x) = e^{-iq x} + \int\limits^x_{-\infty} \frac{\sin(q(x-y))}{q}V_0(y) \Phi_q(y) dy,
\end{equation}
One can immediately obtain 
\begin{equation}
	\Phi_q(0) = 1 - \int\limits^0_{-\infty} \frac{\sin(qy)}{q}V_0(y) \Phi_q(y) dy    
\end{equation}
\begin{equation}
	\Phi'_q(0) = -iq + \int\limits^0_{-\infty} \cos(qy)V_0(y) \Phi_q(y) dy    
\end{equation}
So 
\begin{equation}\label{f1}
	\Phi'_q(0) +iq \Phi_q(0) = \int\limits^0_{-\infty} e^{-iqy}V_0(y) \Phi_q(y) dy   
\end{equation}
and 
\begin{equation}\label{f2}
	\Phi'_q(0) -iq \Phi_q(0) +2iq = \int\limits^0_{-\infty} e^{iqy}V_0(y) \Phi_k(y) dy.
\end{equation}
Now let us consider the following integral that enters definition of the $\Xi^\varphi_{q,q}$ \eqref{chi22}.
\begin{equation}
	I_1 = \int\limits_{-\infty}^0 dx \Phi_q(x) (V_0(x) - V(x))\varphi_q(x),
\end{equation}
Using integral presentation for $\varphi_q(x)$ in the first term and for $\Phi_q(x)$ in the second we obtain
\begin{multline}
	I_{1} =  \int\limits_{-\infty}^0 dx \Phi_q(x) V_0(x)e^{-iqx}- \int\limits_{-\infty}^0 dx e^{-iqx} V(x)\varphi_q(x) \\
	+\int\limits_{-\infty}^0 dx \int\limits^x_{-\infty}dy \frac{\sin(q(x-y))}{q} \Phi_q(x) V_0(x)V(y)\varphi_q(y)- \int\limits_{-\infty}^0 dx \int\limits^x_{-\infty}dy \frac{\sin(q(x-y))}{q} \Phi_q(y) V_0(y)V(x)\varphi_q(x)
\end{multline}
Changing variables in the last two integrals 
\begin{equation}
	I_{1} =  \int\limits_{-\infty}^0 dx \Phi_q(x) V_0(x)e^{-iqx}- \int\limits_{-\infty}^0 dx e^{-iqx} V(x)\varphi_q(x)+ 
	\int\limits_{-\infty}^0 dx \int\limits^0_{-\infty}dy \frac{\sin(q(x-y))}{q} \Phi_q(x) V_0(x)V(y)\varphi_q(y)
\end{equation}
Substituting Eqs. \eqref{f1} and \eqref{f2} we obtain
\begin{equation}
	I_1 = \Phi_q'(0)\varphi_q(0) - \varphi_q'(0) \Phi_q(0)
\end{equation}
Similar we can compute 
\begin{equation}
	I_{2} =  \int\limits_{-\infty}^0 dx \Phi_q(x) V_0(x)\bar{\varphi}_q(x) - \int\limits_{-\infty}^0 dx \Phi_q(x) V(x))\bar{\varphi}_q(x),
\end{equation}
namely
\begin{equation}
	I_{2}=  \int\limits_{-\infty}^0 dx \Phi_q(x) V_0(x)e^{iqx}- \int\limits_{-\infty}^0 dx e^{-iqx} V(x)\bar\varphi_q(x)+ 
	\int\limits_{-\infty}^0 dx \int\limits^0_{-\infty}dy \frac{\sin(q(x-y))}{q} \Phi_q(x) V_0(x)V(y)\bar\varphi_q(y)
\end{equation}
which gives 
\begin{equation}
	I_2 = 2iq + \Phi_q'(0)\bar\varphi_q(0) - \bar\varphi_q'(0) \Phi_q(0)
\end{equation}
This way using \eqref{chi22} with \eqref{lambda1} we obtain
\begin{equation}
	\Xi^\varphi_{q,q} =- \frac{q}{\bar\Phi_q(0)}.
\end{equation}


One more approach for the derivation:

Let us start from 
\begin{equation}
	(q^2-k^2)\int_{-R}^0 dx \Lambda_q(x) \psi_k(x) =  \Lambda_q'(0)\psi_k(0) - \Lambda_q'(-R)\psi_k(-R)- \int\limits_{-R}^0 dx \Lambda_q(x) (V_0(x) - V(x))\psi_k(x),
\end{equation}
where $q$ is chosen to have $\Lambda_q(-R)=\Lambda_q(0)=0$ and $k$ is arbitrary.
For $k=q$ we have
\begin{equation}
\Lambda_q'(0)\psi_q(0) - \int\limits_{-R}^0 dx \Lambda_q(x) (V_0(x) - V(x))\psi_q(x)=\Lambda_q'(-R)\psi_q(-R).
\end{equation}
Therefore
\begin{equation}
	\Xi_{q,q} \equiv  \Lambda_q'(0)\psi_q(0)- \int\limits_{-\infty}^0 dx \Lambda_q(x) (V_0(x) - V(x))\psi_q(x)=
	\lim_{R\to\infty}\Lambda_q'(-R)\psi_q(-R). 
\end{equation}
Using for large $R$
\begin{equation}
    \Lambda_q'(-R)\approx-\frac{q e^{iqR}}{\Phi_q(0)}, \qquad
    \psi_q(-R)\approx \bar a_q e^{iqR}- b_q e^{-iqR}
\end{equation}
we have 
\begin{equation}
    \Xi_{q,q}=- \frac{q}{\Phi_q(0)} (\bar{a}_q e^{-2i\eta(q)}- b_q).
\end{equation}

Similarly,
\begin{equation}
    \Xi^\varphi_{q,q} \equiv  \Lambda_q'(0)\varphi_q(0) -\int\limits_{-\infty}^0 dx \Lambda_q(x) (V_0(x) - V(x))\varphi_q(x)
=	\lim_{R\to\infty}\Lambda_q'(-R)\varphi_q(-R). 
\end{equation}
Using for large $R$
\begin{equation}
    \Lambda_q'(-R)\approx-\frac{q e^{iqR}}{\Phi_q(0)}=-\frac{q e^{-iqR}}{\bar\Phi_q(0)}, \qquad
    \varphi_q(-R)\approx e^{iqR}
\end{equation}
we have 
\begin{equation}
	\Xi^\varphi_{q,q} =- \frac{q}{\bar\Phi_q(0)}.
\end{equation}

\section{OLD}

\begin{equation}
	K(q,q') =\frac{2i(e^\lambda-1)}{\pi}
	|\Phi_q(0)| \left( S_{qq'}(t)  + X_{qq'}(t)- X_{qq'}(0)
	 + Y_{qq'}(t)- Y_{qq'}(0)
	\right)|\Phi_{q'}(0)| 
\end{equation}
\begin{equation}
	S_{qq'}(t) = (F_q^{(1)} \bar F_{q'}^{(0)}-F_q^{(0)} \bar F_{q'}^{(1)})
	\frac{e^{it(E_q - E_{q'})}-1}{i(E_q - E_{q'})},
\end{equation}
\begin{multline}
    Y_{qq'}(t) =\int_0^\infty \frac{dk}{\pi} \frac{e^{it(E_k-E_{q'})}}{i(E^+_k-E_{q'})} 
    \frac{ \Omega_{qk}^{(1)}\bar F_{q'}^{(0)}- \Omega_{qk}^{(0)}\bar F_{q'}^{(1)}}{E^+_k-E_q} -
    \int_0^\infty \frac{dp}{\pi} \frac{e^{it(E_{q}-E_p)}}{i(E^-_p-E_{q})} 
    \frac{F_q^{(1)} \Omega_{q'p}^{(0)}- F_q^{(0)} \Omega_{q'p}^{(1)}}{E^-_p-E_{q'}}  \\ 
   + \int\limits_0^\infty \frac{dk}{\pi}\int\limits_0^\infty \frac{dp}{\pi} \frac{e^{it(E_k-E_p)}}{i(E^+_k-E^-_p)}
   \frac{ \Omega_{qk}^{(1)} \Omega_{q'p}^{(0)}-\Omega_{qk}^{(0)} \Omega_{q'p}^{(1)}}
   {(E^+_k-E_q)(E^-_p-E_{q'})},
\end{multline}
\begin{multline}
    X_{qq'}(t) =
    \sum_{n=1}^{N^\mathrm{b}} 
        \int_0^\infty \frac{dk}{\pi} \frac{e^{it(E_k+\varkappa_n^2)}}{i(E^+_k+\varkappa_n^2)} 
    \frac{ \Omega_{qk}^{(1)} B_{nq'}^{(0)}- \Omega_{qk}^{(0)} B_{nq'}^{(1)}}{E^+_k-E_q} -
    \int_0^\infty \frac{dp}{\pi} \frac{e^{-it(\varkappa_n^2+E_p)}}{i(E^-_p+\varkappa_n^2)} 
    \frac{B_{nq}^{(1)} \Omega_{q'p}^{(0)}- B_{nq}^{(0)} \Omega_{q'p}^{(1)}}{E^-_p-E_{q'}} \\
    + (B_{nq}^{(0)}\bar F_{q'}^{(1)}-B_{nq}^{(1)}\bar F_{q'}^{(0)})
    \frac{e^{-it(\varkappa_n^2+E_{q'})}}{i(E_{q'}+\varkappa_n^2)}
     +(B_{nq'}^{(0)} F_{q}^{(1)}-B_{nq'}^{(1)} F_{q}^{(0)})
    \frac{e^{it(\varkappa_n^2+E_{q})}}{i(E_{q}+\varkappa_n^2)}\\
    +\sum_{m\ne n} (B_{mq}^{(1)}B_{nq'}^{(0)}-B_{mq}^{(0)} B_{nq'}^{(1)})
    \frac{e^{it(\varkappa_n^2-\varkappa_m^2)}}{i(\varkappa_n^2-\varkappa_m^2)}.
\end{multline}
\begin{equation}
	\delta K(q,q') = f_q\frac{\nu_{q',q}/(2q)-\nu_{q',q'}/(2q')}{i(E_q - E_{q'})} +  \int\limits_0^\infty \frac{dk}{2\pi } 
	\frac{\nu_{q',k}( \omega_{q,k} +\omega_{q,-k}) }{2k (E_k^+ - E_q) (E_k^+ - E_{q'})}
\end{equation}







\nocite{*}

\bibliography{lb}


\end{document}


% is given by
% \begin{equation}
%     	A_{12}=	\frac{2u_1 u_2^2|g|^3}{(D(u_1-1)+2)(D(u_2-1)+2)}\int_0^\infty \frac{dq}{\pi} \rho(q)
% 	\frac{ \Xi_{q,i\varkappa_1}\Xi_{q,i\varkappa_2}}
% 	{(\varkappa_1^2+q^2)(\varkappa_2^2+q^2)},
% \end{equation}
% where $u_i = 2\varkappa_i/|g|$ and 
% \begin{equation}\label{Xi2delta1}
% 	\Xi_{qk}=-q-g  e^{ikd/2} \left(\frac{q}{k}\sin \frac{kd}{2}-\sin \frac{qd}{2}\right).
% \end{equation}

 % \section{Examples}

% In this section we apply general results to three potentials: $V(x)=g\delta(x)$ and $V(x)=g_1\delta(x-d_1)+g_2\delta(x-d_2)$ with $V_0(x)=0$, 
% and  $V(x)=-2\cosh^{-2} x$ with $V_0(x)=V(x)$, $x<0$.

% \subsection{ $V(x)=g\delta(x)$, $V_0(x)=0$}

% For the potential $V(x)=g\delta(x)$ the Jost solutions can be found using Eqs.~\eref{psiint} and \eref{phiint} 
% \begin{equation}
%     \psi_k(x) =  e^{-ikx} - \frac{g}{k} \theta(-x) \sin (kx),
% \end{equation}
% \begin{equation}
%     \varphi_k(x) = e^{-ikx} +  \frac{g}{k} \theta(x)  \sin (kx),
% \end{equation}
% where $\theta(x)$ is Heaviside step function.
% So the scattering data is
% \begin{equation}\label{Tdelta}
%     a_k = 1- \frac{g}{2ik},\qquad b_k = \frac{g}{2ik},\qquad 
%     T(E)  = \frac{1}{|a_k|^2} = \frac{k^2}{k^2+g^2/4} = \frac{E}{E+g^2/4}.
% \end{equation}
% {\blue We introduce $\varkappa=|g|/2$.}
% If $g<0$ there is also a bound state corresponding to zero of $a_k$ at 
% $k=i\varkappa$ with wave function
% \begin{equation}
%     \varphi_{i\varkappa}(x) = e^{-\varkappa |x|}.
% \end{equation}
% The states corresponding to the initial potential $V_0(x)$ are
% \begin{equation}
%     \Phi_q(x)=e^{-iqx},\qquad \Lambda_q(x)=-\sin qx.
% \end{equation}
% It leads to $\Xi_{q,k}=\Lambda_q'(0)\varphi_k(0)=-q$.
% Functions $f^{(\alpha)}_q(t)$ can be obtained without Green function and they equal
% \begin{equation}
%     f_q^{(1)}(t)=\frac{1}{2}qe^{itq^2},
% \end{equation}
% \begin{equation}
%     f^{(0)}_q(t)=-\frac{1}{2}\frac{qe^{itq^2}}{iq+g/2}-\theta(-g)\frac{q\varkappa e^{-it\varkappa^2}}{\varkappa^2+q^2}+qE(q),
% \end{equation}
% where
% \begin{equation}
% 	E(q) = \int\limits_0^\infty \frac{dp}{\pi} \frac{p^2 e^{itp^2}}{(p^2+\varkappa^2)((p+i0)^2-q^2)}=\frac{\varkappa \bar{h}_{\varkappa}(t)}{2(q^2+\varkappa^2)}
% 	- \frac{iq h_q(t)}{2(q^2+\varkappa^2)}.
% \end{equation}
% and
% \begin{equation}
% 	h_q(t) := e^{itq^2}\left[1- {\rm Erf} \left(qe^{i\pi/4}\sqrt{t}\right)\right].
% \end{equation}
% Kernel of FCS up to $\rho(q)(e^\lambda-1)/\pi$ is given by the formula 
% \begin{equation}
%     2ie^{-it(q^2-q'^2)/2}\int_{0}^{t}d\tau\,\left\{f^{(1)}_{q}(\tau)\bar f^{(0)}_{q'}(\tau)-f^{(0)}_{q}(\tau)\bar f^{(1)}_{q'}(\tau)\right\}=X_0(q,q')+X_1(q,q'),
% \end{equation}
% where 
% \begin{equation}
% 	X_0(q,q') =qq' \frac{2q}{\varkappa^2+q^2} \frac{\sin\left[t(q^2-q'^2)/2\right]}{q^2-q'^2},
% \end{equation}
% \begin{multline}
% 	X_1(q,q') = qq' \frac{\sin\left[t(q^2-q'^2)/2\right]}{q^2-q'^2} \left(\frac{q'}{\varkappa^2+q'^2}-\frac{q}{\varkappa^2+q^2}\right)\\
% 	-2q q'\mathrm{Im}\left(e^{-it(q^2+q'^2)/2}\frac{e(q)-e(q')}{q^2-q'^2}\right) 
% 	+\frac{q q'}{(\varkappa^2+q^2)(\varkappa^2+q'^2)}\left\{\varkappa\,\mathrm{Re}
% 	\left(e^{it(q^2+q'^2)/2}h_{\varkappa}(t)\right)\right. \\
% 	\left. 
% 	-\frac{g}{2}\cos\left[t(q^2-q'^2)/2\right]-2\theta(-g)\varkappa\cos \left[t(q^2+q'^2+2\varkappa^2)/2\right]\right\},
% \end{multline}
% and 
% \begin{equation}
%     e(q)=\frac{q h_q(t)}{2(q^2+\varkappa^2)}.
% \end{equation}
% Using that we obtain formula for FCS
% \begin{equation}
% 	\mathcal{F} (\lambda,t) = \det \left(1 + \frac{e^\lambda-1}{\pi}\rho(q)X_0(q,q') +\frac{e^\lambda-1}{\pi}\rho(q)X_1(q,q')\right),
% \end{equation}
% \subsection{Two delta functions}
% We consider the case when at the initial time $V_0=0$ on $[-R;0]$ and then a potential with two delta-functions on $[-R;R]$ turns on:  
% 	\begin{equation}\label{V2delta}
% 		V(x) = g_1 \delta(x-d_1)+g_2\delta(x-d_2),  
% 	\end{equation}
% where we assume that $d_2>0>d_1$. The Jost solutions for this potential can be find using ~\eref{psiint} and ~\eref{phiint} 
% 	\begin{equation}
% 		\psi_k(x) = e^{-ik x} - \theta(d_1-x)\frac{\sin(k(x-d_1))}{k}g_1 \psi_k(d_1) - \theta(d_2-x)\frac{\sin(k(x-d_2))}{k}g_2 \psi_k(d_2),
% 	\end{equation}
% 	\begin{equation}
% 		\varphi_k(x) = e^{-ik x} 	+\theta(x-d_1)\frac{\sin(k(x-d_1))}{k}g_1 \varphi_k(d_1) +\theta(x-d_2)\frac{\sin(k(x-d_2))}{k}g_2 \varphi_k(d_2),
% 	\end{equation}
% where 
% 	\begin{equation}
% 	\psi_k(d_1) = e^{-ikd_1 } \left(1 + \frac{g_2}{2ik} \right)  - \frac{g_2}{2ik}e^{ik(d_1-2d_2)}, \qquad 
% 		\psi_k(d_2)= e^{-ikd_2},
% 	\end{equation}
% 	\begin{equation}
% 		\varphi_k(d_1) = e^{-ikd_1}, \qquad 
% 		\varphi_k(d_2) =  e^{-ikd_2 } \left(1 -\frac{g_1}{2ik} \right)  +\frac{g_1}{2ik}e^{ik(d_2-2d_1)}.
% 	\end{equation}
% The scattering data follows from \eref{transfer} 
% 	\begin{equation}\label{Tdelta2a}
% 		a_k = \frac{g_1 g_2 e^{-2 i k (d_1-d_2)}+(2 k+i g_1) (2 k+i g_2)}{4 k^2},
% 	\end{equation}
% 	\begin{equation}\label{Tdelta2b}
% 		b_k = \frac{g_2 e^{-2 i d_2 k} (g_1-2 i k)-g_1 e^{-2 i d_1 k} (g_2+2 i k)}{4 k^2}.
% 	\end{equation}

% Another way to find scattering data for potential $V(x)$ with two delta-functions is to use a composition of transfer matrices corresponding to each 
% of delta-functions. This approach works in more general case when we have  a disjoint potential $V(x)=V_1(x)+V_2(x)$ with
% $V_1(x)=0$ for $x>x_1$ and $V_2(x)=0$ for $x<x_2$ for some  $x_1<x_2$. In this case we have 
% 	\begin{equation}
% 		\mathcal{T}=\mathcal{T}_1\mathcal{T}_2,
% 	\end{equation} 
% where $\mathcal{T}_j$ is the transfer matrix for $V_j$ with corresponding Jost solutions $\psi_j$ and $\varphi_j$. It can be derived in the following way
% \begin{equation}
% 	\begin{pmatrix}
% 		\varphi_1\\
% 		\bar{\varphi}_1
% 	\end{pmatrix}=	\mathcal{T}_1\begin{pmatrix}
% 	\psi_1\\
% 	\bar{\psi}_1
% \end{pmatrix}=\mathcal{T}_1 \begin{pmatrix}
% 	\varphi_2\\
% 	\bar{\varphi}_2
% \end{pmatrix}=\mathcal{T}_1\mathcal{T}_2\begin{pmatrix}
% \psi_2\\
% \bar{\psi}_2
% \end{pmatrix}.
% \end{equation}
% Also we have to take into account that the transfer matrix $\tilde {\mathcal{T}}$ for the shifted potential $\tilde V(x)=V(x-d)$ is obtained from $\mathcal{T}$ 
% by conjugation by a diagonal matrix
% \begin{equation}
% \tilde{\mathcal{T}}=\mathcal{T}(d)=\begin{pmatrix}
% 		a_k & b_k e^{-2ikd}\\
% 		\bar{b}_k e^{2ikd} & \bar{a}_k 
% 	\end{pmatrix}.
% \end{equation}
% Therefore, in particular, for the potential \eref{V2delta} with two delta functions we have the transfer matrix
% \begin{equation}
% 	\mathcal{T}=\mathcal{T}_{g_1}(d_1)\mathcal{T}_{g_2}(d_2), \qquad  
% 	\mathcal{T}_g(0)=\begin{pmatrix}
% 		1-\frac{g}{2ik} & \frac{g}{2ik}\\
% 		-\frac{g}{2ik} & 1+\frac{g}{2ik}
% 	\end{pmatrix},
% \end{equation}
% where we used ~\eref{Tdelta}. It gives the scattering data \eref{Tdelta2a} and \eref{Tdelta2b}.

% The potential $V(x)$ can has one or two bound states depending on the values of parameters. In what follows we will consider the symmetric case of the potential, i.e.  
% $g_1=g_2=g$, $d_2=-d_1=d/2$,
% \begin{equation}\label{Vsym}
% 	V(x)=g\delta(x+d/2)+g\delta(x-d/2).
% \end{equation} 
% The bound states momenta follow from the relation $a_{i\varkappa}=0$ or explicitly
% \begin{equation}\label{bound2d}
% a_{i\varkappa}=	\frac{(u-1)^2 -e^{-u D}}{u^2}=0,
% \end{equation}
% where we introduced notations 
% \begin{equation}
% k=i\varkappa, \qquad
% 	u=2\varkappa/|g|, \qquad D=|g|d.
% \end{equation}
% The equation \eref{bound2d} has two solutions for $D>2$ and one solution for $0\le D \le 2$. Note $a_k$ has a simple pole at $k=0$ if $D\ne 2$. The case $D=2$ corresponds to the 
% situation when a bound state arises from the continuous spectrum and in this case $a_{k}$ is regular at $k=0$.
% Note, for nonsymmetric potential $V(x)$, the condition $D\gtrless 2$ is generalized to
% \begin{equation}
% 	d_{2}-d_1 \gtrless \frac{1}{|g_1|}+\frac{1}{|g_2|}.
% \end{equation}

% In what follows we will need
% \begin{equation}\label{Xi2delta}
% 	\Xi_{qk}=\Lambda'_q(0)\varphi_k(0)+\int_{-\infty}^{0}dx \Lambda_q(x)V(x)\varphi_k(x)=-q-g  e^{ikd/2} \left(\frac{q}{k}\sin \frac{kd}{2}-\sin \frac{qd}{2}\right),
% \end{equation}
% where we used 
% \begin{equation}
% 	\Lambda_q(x)=\mathrm{Im}\, \Phi_q(x)=-\sin qx.
% \end{equation}

% To compute current \eref{J} we need to find $f^{(\alpha)}_q(t)$ given by ~\eref{faRe}:
% \begin{equation}
% 	f^{(1)}_q(t) =  
% 	 B_{2,q}^{(1)} e^{-it\varkappa_2^2}
% 		+F^{(1)}_q e^{itq^2} + I^{(1)}_q(t),
% \end{equation}
% \begin{equation}
% 	f^{(0)}_q(t) =  
% 	 B_{1,q}^{(0)} e^{-it\varkappa_1^2}
% 		+F^{(0)}_q e^{itq^2} + I^{(0)}_q(t),
% \end{equation}
% \begin{equation}
%     B_{n,q}^{(\alpha)}=
%     \frac{i \Xi_{q,i\varkappa_n}\partial_x^\alpha\bar \psi_{i\varkappa_n}(0)}
%     {a'_{i\varkappa_n}(\varkappa_n^2+q^2)},
%     \qquad
%     F_q^{(\alpha)}=-i \frac{\partial_x^\alpha\psi_{q}(0)}{2a_{-q}},
% \end{equation}
% \begin{equation}
% 	a'_{i\varkappa_j}=\left.\frac{da}{dk}\right|_{k=i\varkappa_j}=-\frac{2i}{|g|}\frac{(u_j-1)(D(u_j-1)+2)}{u_j^2},
% \end{equation}
% \begin{equation}
% 	\bar\psi_{i\varkappa_1}(0)=2-2/u_1, \qquad  \bar\psi_{i\varkappa_2}(0)=\partial_x \bar\psi_{i\varkappa_1}(0)=0, \qquad 
% 	\partial_x \bar\psi_{i\varkappa_2}(0)=(1-u_2)|g|.
% \end{equation}

% The most difficult part of computation is estimation of integrals
% \begin{equation}
% 	I^{(\alpha)}_q(t)=\int\limits_{0}^{\infty} \frac{dk}{\pi} 
% 	\Omega^{(\alpha)}_{q,k}
% 	\frac{e^{itk^2}}{(k+i0)^2-q^2},
% \end{equation}
% where 
% \begin{equation}
% 	\Omega^{(\alpha)}_{q,k}=
% 	\mathrm{Re}\,
% 	\frac{\Xi_{q,k}\partial_x^\alpha\bar\psi_k(0)}{a_k}.
% \end{equation}
% For symmetric potential \eref{Vsym}  we have
% \begin{equation}
% 	\Omega^{(0)}_{q,k}=\frac{2k^2(-q+g\cos kd/2\sin qd/2)}{g^2+2k^2+g^2\cos kd-2gk \sin kd},
% \end{equation}
% \begin{equation}
% 	\Omega^{(1)}_{q,k}=-\frac{2gk^3\sin kd/2\sin qd/2}{g^2+2k^2-g^2\cos kd+2gk \sin kd}.
% \end{equation}
% To find asymptotic behavior of the integrals $I^{(\alpha)}_q(t)$ we need to know expansions of the integrands at $k=0$
% \begin{equation}
% 	\Omega^{(0)}_{q,k}=\frac{k^2}{g^2}(-q+g\sin qd/2)+O(k^4), \quad \Omega^{(1)}_{q,k}=-\frac{2k^2gd\sin qd/2}{(2+gd)^2} + O(k^4).
% \end{equation}
% The formula for $\Omega^{(1)}_{q,k}$ is valid for $D=-gd\ne 2$. The asymptotic   behavior of $\Omega^{(1)}_{q,k}$ for  $D=2$ and small $k$ is
% \begin{equation}
% 	\Omega^{(1)}_{q,k}=\frac{4}{d^2}\sin \frac{qd}{2}+\frac{k^2}{18}\sin \frac{qd}{2}+O(k^4).
% \end{equation} 
% Therefore, the integrals have following decaying behavior for large $t$
% \begin{equation}
% 	I^{(\alpha)}_q(t)\sim t^{-\frac{3}{2}}  \quad \mathrm{for} \quad D\ne 2,\quad  \mathrm{and} \quad I^{(\alpha)}_q(t)\sim t^{-\frac{3}{2}+\alpha} \quad \mathrm{for} \quad D=2.
% \end{equation}
% They will be neglected since they do not give a contribution to  the leading term of asymptotic current
%  given by ~\eref{Jtot}. If the potential has two bound states than there is an oscillatory part of the current with the amplitude of oscillations
% given by ~\eref{Amn}
% \begin{equation}\label{A12d2}
% A_{12} =-\frac{4}{\pi} \int_0^\infty dq  \rho(q)  
% B^{(1)}_{2,q} B^{(0)}_{1,q}.
% \end{equation}
% Finally,  the leading contribution to the current for large $t$ consists of
% constant  Landauer--B\"uttiker current and an oscillating current (if there are two bound states)





 


% \subsection{An example of reflectionless potential}

% In this subsection we consider an example of perfect lead attachment, i.e. $V_0(x)=V(x)$, $x<0$, for the reflectionless potential 
% \begin{equation}\label{Vcosh}
% V(x)=-\frac{2}{\cosh^2 x}.
% \end{equation}
% The corresponding Jost solutions are 
% \begin{equation}
%     \psi_k(x)= e^{-ikx}\left(1+\frac{2i}{k-i}\frac{1}{e^{2x}+1}\right),
% \end{equation}
% \begin{equation}
%     \varphi_k(x)=\bar\psi_k(-x)=e^{-ikx}\left(1-\frac{2i}{k+i}\frac{1}{e^{-2x}+1}\right)=\frac{k-i}{k+i} \psi_k(x)
% \end{equation}
% with  
% \begin{equation}
%     a_k=\frac{k-i}{k+i} , \qquad b_k=0.
% \end{equation}
% This potential has one bound state corresponding to the zero of $a_k$ at $k=i$:
% \begin{equation}
%   \chi_1^{\rm b}(x) = \varphi_{k=i}(x)=\frac{1}{2\cosh x}.
% \end{equation}
% Initial one-particle states are given by \eref{lambda1}:
% \begin{equation}
%     \Lambda_q(x)=-\frac{q \sin qx +\tanh x\cos qx}{q}.
% \end{equation}
% Therefore $\Xi_{q,k}$ defined in \eref{Xiqk} becomes
% \begin{equation}
%     \Xi_{q,k}= \Lambda_q'(0)\varphi_k(0) =-\frac{1+q^2}{q} \cdot \frac{k}{k+i}.
% \end{equation}
% We have 
% \begin{equation}
%     \frac{\Xi_{q,k}\bar{\psi}'_k(0)}{a_k}=-i\frac{1+q^2}{q} k.
% \end{equation}
% Therefore the integral in \eref{faRe} for $\alpha=1$ does not give a contribution. Since for the the  bound state 
% $(\chi_1^{\rm b})'(0)=0$,  there is no contribution of bound states   
% to $f^{(1)}_q(t)$. Finally we obtain
% \begin{equation}
%     f^{(1)}_q(t)=-\frac{1+q^2}{2q}e^{itq^2}.
% \end{equation}

% Similarly we have 
% \begin{equation}
%     \frac{\Xi_{q,k}\bar{\psi}_k(0)}{a_k}=-\frac{1+q^2}{q} \frac{k^2}{1+k^2}, \qquad \frac{i\Xi_{q,k}\bar{\psi}_k(0)}{a'_k}=-\frac{k^2}{2q}.
% \end{equation}
% Therefore 
% \begin{equation}
%     f^{(0)}_q(t) =  \left.\frac{i\Xi_{q,k}\bar{\psi}_k(0)e^{itk^2}}{a'_k}\right|_{k=i}
%     -i\frac{\psi_q(0)e^{itq^2}}{2\varphi_q(0) a_{-q}} +  \int\limits_0^\infty \frac{dk}{\pi} {\rm Re} \left[
%     \frac{\Xi_{q,k}\bar{\psi}_k(0)}{a_k}
%     \right] \frac{e^{itk^2}}{(k+i0)^2-q^2}
% \end{equation}
% becomes
% \begin{equation}\label{GqCosh}
%     f^{(0)}_q(t) =  \frac{e^{-it }}    {2q}
%     +\frac{e^{itq^2}}{2i} 
%     -\frac{1+q^2}{q} 
%     \int\limits_0^\infty \frac{dk}{\pi}  \frac{k^2}{1+k^2} \frac{e^{itk^2}}{(k+i0)^2-q^2}
% \end{equation}
% \begin{equation}
%     \approx \frac{e^{-it }}    {2q}
%     +\frac{e^{itq^2}}{2i}+\frac{1+q^2}{q^3}\frac{e^{3\pi i/4}}{4\sqrt{\pi}} t^{-\frac32},\qquad t\to \infty,
% \end{equation}
% where the asymptotic behavior is given for fixed $q>0$.
% The current is given by formula \eref{J}
% \begin{equation}
%     J (t)= 
%     -\int_0^\infty dq \rho(q) 
%     \frac{4|\varphi_q(0)|^2 }{\pi}
%       {\rm Im}\,  f^{(1)}_q(t) \bar{f}^{(0)}_{q}(t) .
% \end{equation}
% Using expressions for $f^{(1)}_q(t)$ and $f^{(0)}_q(t)$ we obtain 
% \begin{equation}
%     J(t) =  \int_0^\infty \frac{dq}{\pi} \rho(q)\left(q+\sin(1+q^2)t+2(1+q^2)\mathrm{Im}\int\limits_0^\infty \frac{dk}{\pi}  \frac{k^2}{1+k^2} \frac{e^{it(k^2-q^2)}}{(k+i0)^2-q^2}\right).
% \end{equation}
% The integral in $q$ of the second term is decreasing as $1/\sqrt{t}$, which can be shown by the method of stationary phase.
% The contribution of the third term is even smaller for large $t$.  So, making substitution from momenta to energy,  we obtain the Landauer--B\"uttiker current 
% for reflectionless potential \eref{Vcosh}
% \begin{equation}
%     J=\int_0^\infty \frac{dE}{2\pi} \rho(E)+O(t^{-\frac{1}{2}}).
% \end{equation}
