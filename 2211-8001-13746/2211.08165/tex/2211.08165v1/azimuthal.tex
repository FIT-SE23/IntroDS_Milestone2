%!TEX root = ./main.tex
\subsection{Example: Azimuthal Projection of the Sphere}
Let us try the string relaxation scheme on a simple but nontrivial example, where we can still intuitively understand the metric.
We look at the so-called azimuthal projection of the unit sphere \cite{Snyder1997}.

The azimuthal projection maps the sphere onto a bounded disk.
We choose to parametrize the disk by polar coordinates $(r, \theta)$ with $r \in [0, \pi)$ and $\theta \in (-\pi, \pi]$.
The sphere is mapped to the disk by assigning the north pole to the point $r=0$ and map every meridian to the corresponding radius by setting the polar angle equal to the latitude, i.e. $\theta = \lambda$.
Every point along the radius will get the distance to the north pole measured along the longitudinal lines.
The parametrization, therefore, maps the north pole to the center of the disk and the south pole to the entire boundary.
Given parameter values $(r, \theta)$ we get a point on the unit sphere by \begin{align}
  x = \sin r \cos \theta && y = \sin r \sin \theta && z = \cos r.
\end{align}
Let us pull the standard Euclidean metric on the surface of the unit sphere back to our azimuthal parameter space. We get \begin{equation}
  g_\circ = \begin{bmatrix}
    1 & 0\\0 & \sin^2 r 
  \end{bmatrix}.
\end{equation}
A curve that is a geodesic in the azimuthal map with respect to $g_\circ$, will also be mapped to a geodesic on the unit sphere in the Euclidean sense. This is easy to understand and visualize.
%Let's try it.
We fix strings between various pairs of end-points on the disc and connect them with a Euclidean straight line in $(r, \theta)$-space.
In Fig.~\ref{fig:polar} we show four different examples.
Initial curves are displayed as dashed lines.
Afterwards, we let the strings relax by the relaxation scheme shown above.
The strings will reduce their Riemannian length with respect to the metric $g_\circ$ and converge to the solid lines.
The gray lines show some intermediate states during the transient.

We observe that all strings converge to sections of great circles when mapped to the sphere.
This is no surprise, as all geodesics on the unit sphere are sections of great circles.
Note that the pink string, which is constant in longitude $\theta$, stays straight on the disc, as this already represents a great circle.
The strings get pushed towards center in the inner region ($r < \nicefrac{\pi}{2}$) and towards the circumference in the outer region ($r > \nicefrac{\pi}{2}$.)
Also, the blue and the orange string have the same (Riemannian) length!
This becomes clear when looking at them on the sphere.

\begin{figure}
  \centering
  \subfloat[Parameter Space]{\def\svgwidth{.4\textwidth}\tiny\input{pdftex/polar.pdf_tex}}\hspace{1cm}
  \subfloat[Sphere]{\def\svgwidth{.4\textwidth}\tiny\input{pdftex/sphere1_noframe.pdf_tex}}
  \caption{Computing geodesics on the sphere by relaxation of strings in a 2D parameter space. Dashed lines show initial guesses and solid lines show converged curved. The thin black lines show intermediate states on the path to convergence.}
  \label{fig:polar}
\end{figure}
