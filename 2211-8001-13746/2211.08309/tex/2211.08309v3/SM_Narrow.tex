% Target: Phys Rev Letter
%
% https://journals.aps.org/prl/authors
%
%
% 
\documentclass[aps,prl,floatfix,superscriptaddress,showpacs,nobibnotes,twocolumn]{revtex4-2}
\usepackage{amsmath,amssymb,graphicx,psfrag}


%%%%%%%%%%%%%%%%%%%%%%%%%%%  coloring
\usepackage{xcolor}
\definecolor{later}{rgb}{0.7, 0.7, 0.7}
\newcommand{\later}[1]{{ \color{later} {#1} }}

%
\newcommand{\red}[1] {{ \color{red} #1}}
\newcommand{\green}[1] {{ \color{green} #1}}
\newcommand{\blue}[1] {{ \color{blue} #1}}
\newcommand{\magenta}[1] {{ \color{magenta} #1}}
%
\newcommand{\AK}[1] {{\color{red} #1}}
\newcommand{\DT}[1] {{\color{blue} #1}}
\newcommand{\YS}[1] {{\color{magenta} #1}}
%%%%%%%%%%%%%%%%%%%%%%%%%%%%%%%%%%%%%%%%%%%%%%%%%%%%%%%%%%%%%%%%%

%%%%%%%%%%%%%%%%%%%%%%%%%%%  new commands
% Peclet number
\newcommand{\Pe}{\textrm{Pe}}
% for angle brackets
\usepackage{textcomp}
\newcommand{\angb}[1]{\ifmmode\langle#1\rangle\else\textlangle#1\textrangle\fi}
%%%%%%%%%%%%%%%%%%%%%%%%%%%%%%%%%%%%%%%%%%%%%%%%%%%%%%%%%%%%%%%%%

%%%%%%%%%%%%%%%%%%%%%%%%%%%  section numbering
\setcounter{secnumdepth}{2}
\addtocounter{secnumdepth}{2}

\renewcommand{\thesection}{\Roman{section}}
\renewcommand{\thesubsection}{\Alph{subsection}}
\renewcommand{\thesubsubsection}{\arabic{subsubsection}}

%%%%%%%%%%%%%%%%%%%%%%%%%%%%%%%%%%%%%%%%%%%%%%%%%%%%%%%%%%%%%%%%%

%%%%%%%%%%%%%%%%%%%%%%%%%%%   for ref command of section label 
\makeatletter
\def\p@subsection{}
\def\p@subsubsection{}
\makeatother
%%%%%%%%%%%%%%%%%%%%%%%%%%%%%%%%%%%%%%%%%%%%%%%%%%%%%%%%%%%%%%%%%

%%% BEGIN DOCUMENT
\begin{document}

\title{ 
Supplemental Material: \\
Scaling laws for two-dimensional dendritic crystal growth in a narrow channel
}


\author{Y. Song}
%\email[]{songy3@ornl.gov}
\affiliation{Department of Physics and Center for Interdisciplinary Research
on Complex Systems, Northeastern University, Boston, MA 02115 USA}
\affiliation{
Materials Science Division, Lawrence Livermore National Laboratory, Livermore, CA 94550, USA
}


\author{D. Tourret}
%\email[]{damien.tourret@imdea.org}
\affiliation{IMDEA Materials Institute, Getafe, 28906 Madrid, Spain}

\author{A. Karma}
\email[Corresponding author: ]{a.karma@northeastern.edu}
\affiliation{Department of Physics and Center for Interdisciplinary Research
on Complex Systems, Northeastern University, Boston, MA 02115 USA}


% \pacs{ ????? } not necessary anymore
\date{\today}


\maketitle
%\tableofcontents 

%%%%%%%%%%%%%%%%%%%%%%%%%%%%%%%%%%%%% 

\section{Simulation models} 

We use two simulation techniques, namely phase-field (PF) and dendritic-needle-network (DNN) models, to simulate a dendrite growth in a channel. 


\subsection{Phase-field model } 

The phase-field (PF) method for solidification has yielded quantitative comparison to experiments in two (e.g.,~\cite{GonzalezCincaEtAl05,Echebarria10,TourretKarma15,Tourret17, Guo17}) and three dimensions (e.g.,~\cite{Bergeon13, Tourret15, Clarke17,Tourret17}). 
Here, we use the PF model of a dilute binary alloy within the thin interface limit~\cite{KarmaRappel98}, with a corrective anti-trapping solute current at the interface~\cite{Karma01,Echebarria04}. 
We use a preconditioned phase-field $\psi$ with $\varphi\equiv\tanh(\psi/\sqrt{2})$~\cite{Glasner01,TourretKarma15}, in place of the classical $\varphi$ ($\varphi = +1$ in the solid and -1 in the liquid).
In the PF model, the dimensionless supersaturation $U$ is defined as
\begin{equation}
\label{eqn:PF:U}
U \equiv \frac{1}{1-k} \left[ \frac{ 2 c/c_l^0}{ ( 1+k ) - (1 - k) \varphi } - 1 \right] \, , 
\end{equation}
where $c$ is a solute concentration, and
$k = c_s^0/c_l^0$ is the solute partition coefficient with the equilibrium solute concentrations on the solid $c_s^0$ and on the liquid side $c_l^0$ of a planar interface, respectively. 
After space and time are scaled with respect to the diffuse interface width $W$ and relaxation time $\tau_0$ at the temperature $T_0$~\cite{Echebarria04}, the time evolution of $\psi$ and $U$ follows 
\begin{align}
\label{eqn:phi}
a_s(\theta)^2 \frac{\partial\psi}{\partial t} &= %\nonumber\\
\vec\nabla \left[a_s(\theta)^2\right] \vec\nabla\psi  \nonumber \\
&+~ a_s(\theta)^2 \left[  \nabla^2\psi -\varphi\sqrt{2}|\vec\nabla\psi|^2  \right]  \nonumber\\
&+~\sum_{m=x,y}\left[ \partial_m\left( |\vec\nabla\psi|^2 a_s(\theta) \frac{\partial a_s(\theta)}{\partial(\partial_m\psi)} \right) \right] \nonumber\\
&+~ \sqrt 2 \left( \varphi  - \lambda_c (1-\varphi^2) U \right) \, ,
\end{align}
\begin{align}
\label{eqn:U}
&\Big( 1+k-(1-k)\varphi \Big) \frac{\partial U}{\partial t} = %\nonumber\\
{D}' \; \vec\nabla\cdot\left[ (1-\varphi)\vec\nabla U\right]  \nonumber\\
&\qquad +~\vec\nabla\cdot\left[ \Big(1+(1-k)U\Big) \frac{(1-\varphi^2)}{2} \frac{\partial\psi}{\partial t} \frac{\vec\nabla\varphi}{|\vec\nabla\varphi|} \right] \nonumber\\
&\qquad +~\Big[ 1+(1-k)U \Big]  \frac{ \partial \varphi}{\partial t} \, , 
\end{align}
with the dimensionless diffusivity 
\begin{align}
\label{eqn:diffusivity}
D' &= D \frac{\tau_0}{W} = a_1 a_2 \frac{W}{d_0} \, ,
\end{align}
and the coupling factor 
\begin{align}
\label{eqn:cfactor}
\lambda_c &= a_1 \frac{W}{d_0} \, .
\end{align}
The constants $a_1=5\sqrt{2}/8$ and $a_2=47/75$ are obtained from the thin interface limit with vanishing kinetic coefficient~\cite{KarmaRappel98}. 
A 2D fourfold anisotropy is imposed on the diffuse interface width and on the phase field relaxation time, using the standard form $a_s(\theta)=1+\varepsilon_4 \cos (4\theta)$, where $\varepsilon_4$ is the surface tension anisotropy strength, and $\theta$ is an angle between the interface normal and a fixed $x$ axis linked to the crystal orientation~\cite{Echebarria04, Echebarria10, TourretKarma15}. 
Equations are solved numerically using finite differences on a grid size $\Delta x$ and an explicit time stepping $\Delta t$.
Further details about the model were published elsewhere~\cite{Echebarria04,TourretKarma15}.


\subsection{Dendrite-needle-network model } 

The DNN model consists in solving the transport of the solute field $u$ in the liquid phase, e.g., the diffusion equation, interacting with a network of needle-like branches at equilibrium. 
The growth velocity of each dendritic tip is obtained by combining solute balances at two distinct length scales, i.e., using two relations~\cite{TourretKarma13} summarized below.

The DNN formulation used here considers infinitely sharp needle-like branches~\cite{TourretKarma13}.
At a low supersaturation $\Omega$, since $l_D\gg\rho$, one can write a solute balance at an intermediate scale between $\rho$ and $l_D$. 
At a scale $\gg\rho$, curvature effects can be neglected and a dendritic branch can be represented as a sharp line segment at fixed equilibrium concentration $u = 0$. 
On the other hand, at a scale $\ll l_D$, the solute field relaxes fast enough to assume a Laplacian field $u$ in the vicinity of the tip. 
The analytical solution to this problem~\cite{DerridaHakim92} exhibits a square-root singularity of the normal gradient of the field in the vicinity of the tip as
\begin{equation} 
\left. \frac{\partial u}{\partial y} \right|_{y=0} = \frac{\mathcal F}{\sqrt{d_0 (x_t-x)}} 
\end{equation}
when the position $x$ tends to the tip position $x_t$. 
The flux intensity factor $\mathcal F$ is a measure of the incoming flux toward the tip, defined as
\begin{equation}
\label{eqn:FIF_basic}
\mathcal{F} \equiv \lim_{x \to x_t} \left. \sqrt{d_0(x_t - x)} \, \frac{\partial u}{ \partial y } \right|_{y=0} \, .
\end{equation}
In 2D, $\mathcal F$ can be directly calculated with an integral along a contour $\Sigma$ around the tip similar to the J-integral commonly used in fracture mechanics~\cite{Rice68,BAmar02,TourretKarma13}  	
\begin{equation} 
\label{eqn:FIFcontour}
{\mathcal F}^2 = \frac{d_0}{2 \pi} \int_{\Sigma} \Big[ \Big ( (\partial_{{x}}u)^2 - (\partial_{{y}}u)^2 \Big )n_{{x}} 
+ 2\ \partial_{{x}}u\partial_{{y}}u\, n_{{y}} \Big] \mathrm{d} \Sigma \, .
\end{equation}
Further away from the tip, the interface position $y_i(t)$ is assumed to follow the solution of a 1D diffusion problem in the direction normal to the branch. 
Using the flux along the sharp needle ($y=0$) as an approximation for the flux at the interface ($y=y_i(t)$), we write
\begin{equation} 
\frac{\mathrm{d}y_i(t)}{\mathrm{d}t} \approx D\, \left.\frac{\partial u}{\partial y} \right|_{y=0} = D \frac{\mathcal F }{ \sqrt{d_0 (x_t-x)}} \, .
\end{equation}
Then, assuming that, at this scale with $x_t-x \ll D/V$, the tip moves in a quasi-steady regime, we can use the change of variable $x_t-x=Vt$, and obtain the relation linking $\rho$ and $V$~\cite{TourretKarma13}
\begin{equation}
\label{eqn:FIFcondition}
\rho V^2 = \frac{2 D^2 \mathcal{F}^2}{d_0} \, .
\end{equation}
This relation, obtained here for an infinitely sharp needle using an analytical solution to the Laplace equation and an analogy to fracture mechanics, may also be obtained for a parabolic tip shape using solely solute mass conservation considerations~\cite{TourretKarma16,Tourret19}.

The second relation, established at the scale of the dendritic tip radius $\rho$, is the well-established solvability condition~\cite{Langer87,Langer89,Barbieri89,BAmar93},
\begin{equation}
\label{eqn:Solvcondition}
\rho^2 V = \frac{2 D d_0}{ \sigma } \, ,
\end{equation}
where $\sigma$ is the tip selection parameter fixed by the interface anisotropy~\cite{Barbieri89}. 
This relation for the existence of a steady growing solution of a parabolic tip has been validated by PF simulations, which have shown that $\sigma$ reaches a constant value as soon as parabolic tips start emerging at the very early stage of dendritic growth~\cite{PlappKarma00}. 
Therefore, in the DNN model, Eq.~\eqref{eqn:Solvcondition} is used during both steady state and transient growth stages.

Combining Eqs.~\eqref{eqn:FIFcondition} and~\eqref{eqn:Solvcondition}, the growth dynamics of each needle tip is given by
\begin{align}
\label{eqn:FIF_radius}
\frac{\rho(t)}{d_0} &= \left( \frac{ 2 }{ \sigma^2 \mathcal{F}(t)^2} \right)^{1/3} \, ,
\\
\label{eqn:FIF_velocity}
\frac{V(t) d_0 }{D} &= \left( 2 \sigma \mathcal{F}(t)^4 \right)^{1/3} \, ,
\end{align}
where the transient evolution of the solute field surrounding the dendrite tip is captured by the time-dependent flux intensity factor $\mathcal{F}(t)$.

For the sake of numerics and generality, we scale time and space with respect to a theoretical stationary radius $\rho_s$ and velocity $V_s$ of a single free dendrite tip.
For a supersaturation $\Omega\ll 1$ and an infinitely thin needle-like dendrite, the two-dimensional Ivantsov solution~\cite{KurzFisher, Ivantsov47} may be approximated as $\Omega\approx\sqrt{\pi\Pe}$.
Combining the latter with solvability Eq.~\eqref{eqn:Solvcondition}, we obtain 
\begin{align}
\label{eqn:Steady_radius}
\frac{\rho_s}{d_0} &= \frac{ \pi }{ \sigma \Omega^2} \,, 
\\
\label{eqn:Steady_velocity}
\frac{ V_s d_0}{D } &= \frac{ 2 \sigma \Omega^4}{ \pi^2 } \, .
\end{align}
%
The resulting dimensionless tip radius $\tilde{\rho}\equiv\rho/\rho_s$ and velocity $\widetilde{V}\equiv V/V_s$ evolve with time $\tilde{t}\equiv t V_s/\rho_s$ as
\begin{align}
\label{eqn:scaledR}
\tilde{\rho}&=(2 \widetilde{D}^2 \widetilde{\mathcal{F}}^2 )^{-1/3} \, ,
\\
\label{eqn:scaledV}
\widetilde{V}&=(2 \widetilde{D}^2 \widetilde{\mathcal{F}}^2 )^{2/3} \, ,
\end{align}
where $\widetilde{D}=\pi/(2\Omega^2)$, and the scaled flux intensity factor $\widetilde{\mathcal{F}}$, following Eq.~\eqref{eqn:FIFcontour}, is 
\begin{equation} 
\label{eqn:FIFscaledContour}
{\widetilde{\mathcal F}}^2 = \frac{1}{2 \pi}\int_{\Sigma} \Big[ \Big ( (\partial_{{\tilde x}}u)^2 - (\partial_{{\tilde y}}u)^2 \Big )n_{{\tilde x}} + 2\ \partial_{{\tilde x}}u\partial_{{\tilde y}}u\, n_{{\tilde y}} \Big] \mathrm{d} \Sigma .
\end{equation}
Equations \eqref{eqn:scaledR}-\eqref{eqn:FIFscaledContour}, together with the diffusion equation 
\begin{equation} 
\label{eqn:scaledDiff}
\partial_{\tilde t}u = \widetilde D\nabla^2 u
\end{equation}
and the boundary conditions ($u=0$ on each needle and $u=\Omega$ in the liquid far away from the solid) constitute the summary of the 2D DNN model for isothermal growth.

DNN equations are solved numerically using finite differences on a grid size $\Delta x$ with an explicit time step $\Delta t$.
In the ``sharp'' DNN model used here, the needles do not thicken and remain sharp line segments.
Nonetheless, we can estimate the volume of a needle-like branch by integrating over time $\partial u/\partial y |_{y=0}$ along its sides. 
In discrete form, let a needle grow in the $x$ direction, with its tip located at $(i+r,j)$, where $i$ and $j$ are integer grid coordinates and $0 \le r < 1$ describes the progression of the tip position between successive grid points.
At a position $(i',j)$ along the needle length, with $0 < i' \ (= i+r) <i $, the incoming solute flux on the $y+$ side of the dendrite is 
\begin{equation} 
\label{eqn:income}
\vec{I} = - D \frac{ u(i',j) - u(i',j+1)}{ \Delta x } \, ,
\end{equation}
where $u(i',j)=0$ on the needle. 
Moreover, the incoming solute at the tip in the $x$ direction is 
\begin{equation} 
\vec{I}_{t} = D \frac{ u(i+1,j)}{ (1-r) \Delta x } \, .
\end{equation}
Thus, during a time step $\Delta t$, the dendrite volume increases by $\vec{I} \, \Delta t \Delta x$ on each side and $\vec{I}_{t} \Delta t  \Delta x$ at the tip.
The total volume of a dendrite is calculated by integrating these fluxes over time.


%%%%%%%%%%%%%%%%%%%%%%%%%%%%%%%%%%%%%%%%%%%%%%%%%%%%%%%%%%%%%%%%%%%%%%%%%% 

\section{Method}

We perform PF and DNN simulations using parameters listed in Table~\ref{tab:parameters}. 
We keep the same values for $k$, $\varepsilon_4$, and $\sigma$ in all simulations. 

%%%%%%%%%%%%%%%%%% table
\begin{table}[b!]
\caption{ Parameters used for PF and DNN simulations.}
\begin{center}
\begin{tabular}{ c r r }
\hline
\hline
Symbols & PF model  & DNN model
\\ 
\hline
$\Omega$ & 0.05, 0.2 & 0.05, 0.1, 0.2
\\
$k$ & 0.15 & 
\\
$\varepsilon_4$ & 0.03 &  
\\
$\sigma$ & 0.22 & 0.22
\\
$\Delta x$ & 1~($W$)=30, 400~($d_0$) & 1, 4~($\rho_s$) 
\\
\hline 
\hline
\end{tabular}
\end{center}
\label{tab:parameters}
\end{table}
%%%%%%%%%%%%%%%%%%

In PF simulations, a solid seed with $U=0$ is initialized within a liquid domain at $U=-\Omega$.  
The seed is shaped as a quarter of a circle, located at the bottom-left corner of the domain. 
Symmetric (i.e., no-flux) boundary conditions are applied on the four domain boundaries. 

In DNN simulations, the initial solute field is $u=\Omega$, except for $u=0$ on the needle. 
We use $\Delta x = \rho_s$ for the higher $\Omega=0.2$ and $\Delta x = 4 \rho_s$ for the other $\Omega=0.05$ and 0.1. 
Those simulations use symmetric boundary conditions in all directions.

%%%%%%%%%%%%%%%%%%

For PF simulations using a solute supersaturation $\Omega=0.05$, the grid spacing is $\Delta x = W = 0.07 \rho_s$ with $W = 400 d_0$, and the radius of the initial solid seed is $100 d_0 \simeq 0.0175 \rho_s$. 
The simulation domain is a channel of width $\widetilde{\Lambda}  =  4.2, 10.9$, or $44.5$ in the $y$ direction and length $2801.0\rho_s$ in the $x$ direction. 
The simulated time is $\tilde{t} = 110$. 
The corresponding DNN simulations are initialized with two needles of length 3$\rho_s$, pointing towards positive $x$ and $y$ directions, and rooted at the bottom left corner. 
They grow until $\tilde{t} = 1100$ within a domain of $N_x \times N_y = 8000\rho_s \times 25 \rho_s$, where $N_x$ and $N_y$ are the domain sizes in $x$ and $y$ directions, respectively. 

Next, in order to investigate the effect of the channel width on dendrite growth dynamics, we use a supersaturation $\Omega = 0.2$ for PF simulations. Those simulations use $\Delta x= W =30d_0$ with $\widetilde{\Lambda}$ varying from 13.1 to 214.8 for a channel of length 1075.46$\rho_s$. 
The seed radius is $100d_0\simeq0.28\rho_s$ and the simulated time is $\tilde{t} = 200$. 
DNN simulations are performed with a single needle located in the center of the channel. 
We explore different channel width of 61, 125, 157, 213, 253, 509, and 4093$\rho_s$ with the length 27069$\rho_s$ for $\Omega = 0.2$. 
For $\Omega = 0.1$, the widths are 192, 308, 628, 756, 884, 1268, and 2548$\rho_s$ and the length of 36088$\rho_s$. 
The widths are 116, 500, 2036, 2804, 3188, 4084, and 8180$\rho_s$ and the length of 40960$\rho_s$ for $\Omega = 0.05$. 
In all simulations, the needle initially has a length $3 \Delta x$, and grows on the $x$ direction until $\tilde{t} =25000$.

%%%%% 
\section{Result}
\subsection{Needle growth at $\Omega = 0.05$}

\begin{figure}[!b]
\centering
\includegraphics[width=\columnwidth]{SMFig01.pdf}
\caption{ 
Needle growth within ${\Lambda}/l_D = 4.5$ (red line) and 5.1 (blue line) at $\Omega = 0.05$. 
}
\label{fig:long}
\end{figure}

We showed that the needle within a channel shows two growth patterns in Fig. 3 in the main mainuscript. 
When the channel width is lower than the critical spacing $\Lambda < \Lambda_c = 5 l_D$, the needle growth follows $V \sim t^{-2/3}$. 
On the other hand, the needle reaches a steady state for a wide channel. 

For the low $\Omega = 0.05$ in Fig. 3(d), the final growth patterns near $\Lambda_c$ are not clear by $\tilde{t} = 25000$. 
Hence, we perform additional simulations with $\Lambda/\rho_s$ = 2804 and 3188 ($\Lambda/l_D=$ 4.5 and 5.1, respectively) with a longer time $\tilde{t} =50000$. The results are shown in Fig.~\ref{fig:long}. 
In this log-log plot, the needle within ${\Lambda}/l_D = 4.5$ (red line) decelerates continuously, and the deceleration rate increases at the end of the simulation. 
On the other hand, for ${\Lambda}/l_D = 5.1$ (blue line), the needle tends to approach a steady state velocity between 0.6 and 0.7 $V_s$ as we observed in the other simulations. 
Therefore, those needles at $\Omega = 0.05$ agree with the suggested critical spacing $\Lambda_c = 5 l_D$; A dendrite needle within a narrow channel ${\Lambda} < \Lambda_c$ decelerates while the needle approaches a steady state velocity when the channel width is larger than $5 l_D$


%%%%%%%%%%%%%%%%%%%%%%%%%%%%%%%%%%%%%%%%%%%%%%%%%%%%%%%%%%%%%%%%%%%%%%%%%%%%
%bibliography
\begin{thebibliography}{99}

% Side-branch growth in two-dimensional dendrites. II. Phase-field model
\bibitem{GonzalezCincaEtAl05} R. Gonz\'alez-Cinca, Y. Couder, and A. Hern\'andez-Machado Phys. Rev. E {\bf 71}, 051601 (2005).

%Onset of sidebranching in directional solidification
\bibitem{Echebarria10} B. Echebarria, A. Karma, and S. Gurevich, Phys. Rev. E {\bf 81}, 021608 (2010).

% Growth competition of columnar dendritic grains: a phase-field study
\bibitem{TourretKarma15} D. Tourret and A. Karma, Acta Mater. {\bf 82}, 64-83 (2015).

% Branching-induced grain boundary evolution during directional solidification of columnar dendritic grains
\bibitem{Guo17} C. Guo, J. Li, H. Yu, Z. Wang, X. Lin, J. Wang, Acta Mater. {\bf 136}, 148-163 (2017).

% Grain growth competition during thin-sample directional solidification of dendritic microstructures: A phase-field study 
\bibitem{Tourret17} D. Tourret, Y. Song, A.J. Clarke, A. Karma, Acta Mater. {\bf 122}, 220-235 (2017).

%Spatiotemporal Dynamics of Oscillatory Cellular Patterns in Three-Dimensional ?Directional Solidification
\bibitem{Bergeon13} N. Bergeon, D. Tourret, L. Chen, J.-M. Debierre, R. Gu\'erin, A. Ramirez, B. Billia, A. Karma and R. Trivedi, Phys. Rev. Lett. {\bf 110}, 226102 (2013).

% Oscillatory cellular patterns in three-dimensional directional solidification
\bibitem{Tourret15} D. Tourret, J.-M. Debierre, Y. Song, F. L. Mota, N. Bergeon, R. Gu\'erin, R. Trivedi, B. Billia, A. Karma, Phys. Rev. E {\bf 92}, 042401 (2015).

% Microstructure selection in thin-sample directional solidification of an Al-Cu alloy: In situ X-ray imaging and phase-field simulations
\bibitem{Clarke17} A. J. Clarke, D. Tourret, Y. Song, S. D. Imhoff, P. J. Gibbs, J. W. Gibbs, K. Fezzaa, A. Karma, Acta Mater. {\bf 129}, 203-216 (2017).

%Quantitative phase-field modeling of dendritic growth in two and three dimensions
\bibitem{KarmaRappel98} A. Karma and W. J. Rappel , Phys. Rev. E {\bf 57}, 4323-4349 (1998).

%Phase-Field Formulation for Quantitative Modeling of Alloy Solidification
\bibitem{Karma01} A. Karma, Phys. Rev. Lett. {\bf 87}, 115701 (2001).

% Quantitative phase-field model of alloy solidification
\bibitem{Echebarria04} B. Echebarria, R. Folch, A. Karma, and M. Plapp, Phys. Rev. E {\bf 70}, 061604 (2004).

%Nonlinear Preconditioning for Diffuse Interfaces
\bibitem{Glasner01} K. Glasner, J. Comp. Phys. {\bf 174}, 695-711 (2001).

% Multiscale dendritic needle network model of alloy solidification
\bibitem{TourretKarma13} D. Tourret and A. Karma, Acta Mater. {\bf 61}, 6474-6491 (2013).
% Needle models of Laplacian growth
\bibitem{DerridaHakim92} B. Derrida and V. Hakim, Phys. Rev. A {\bf 45}, 8759-8765 (1992).
% A Path Independent Integral and the Approximate Analysis of Strain Concentration by Notches and Cracks
\bibitem{Rice68} J. R. Rice, J. Appl. Mech. {\bf 35}, 379-386 (1968).
% Exact results with the J-integral applied to free-boundary flows
\bibitem{BAmar02} M. Ben Amar, J. R. Rice, J. Fluid Mech. {\bf 461}, 321-341 (2002).
% Three-dimensional dendritic needle network model for alloy solidification
\bibitem{TourretKarma16} D. Tourret, A. Karma, Acta Mater. {\bf 120}, 240-254 (2016).
% Multiscale dendritic needle network model of alloy solidification with fluid flow
\bibitem{Tourret19} D. Tourret, M.M. Francois, A.J. Clarke, Comput. Mater. Sci. {\bf 162}, 206-227 (2019)
% 
\bibitem{Langer87} J. S. Langer, Lectures on the Theory of Pattern Formation, Chance and Matter (Les Houches, Session XLVI), edited by Souletie J., Bannimenus J. and R. Stora, Amsterdam, North-Holland, p. 629 (1987).
% Dendrites, Viscous Fingers, and the Theory of Pattern Formation
\bibitem{Langer89} J. S. Langer, Science {\bf 243}, 1150-1156 (1989).
% Predictions of dendritic growth rates in the linearized solvability theory
\bibitem{Barbieri89} A. Barbieri and J. S. Langer, Rhys. Rev. A {\bf 39}, 5314-5325 (1989).
% Theory of Pattern Selection in Three-Dimensional Nonaxisymmetric Dendritic Growth
\bibitem{BAmar93} M. Ben Amar and E. Brener, Phys. Rev. Lett. {\bf 71}, 589-592 (1993)
% Multiscale Random-Walk Algorithm for Simulating Interfacial Pattern Formation
\bibitem{PlappKarma00} M. Plapp and A. Karma, Phys. Rev. Lett. {\bf 84}, 1740-1743 (2000)

% Kurz Text
\bibitem{KurzFisher} W. Kurz, D. J. Fisher, {\it Fundamentals of solidification}, 3rd ed. (Trans Tech, Aedermannsdorf, 1992).
%Temperature field around a spherical, cylindrical, and needle-shaped crystal, growing in a pre-cooled melt
\bibitem{Ivantsov47} G.P. Ivantsov, {\it Dokl. Akad. Nauk SSSR} {\bf 58}, 567-569 (1947).

%%%%%%% 



\end{thebibliography}

\end{document}









