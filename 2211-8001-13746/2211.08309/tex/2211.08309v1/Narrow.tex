% Target: Phys Rev Letter
%
% https://journals.aps.org/prl/authors
% https://journals.aps.org/authors/length-guide
% 
% word limit: 3,750 words ~ 4 pages
% 	(incl. any text in the body of the article, in a figure caption or table caption, & in a footnote or an endnote)
% 	https://journals.aps.org/authors/length-guide
% abstract: ≤ 600 characters
% 
%
% 

\documentclass[aps,prl,floatfix,superscriptaddress,showpacs,nobibnotes,twocolumn]{revtex4-2}
\usepackage{amsmath,amssymb,graphicx,psfrag}


%%%%%%%%%%%%%%%%%%%%%%%%%%%  coloring
\usepackage{xcolor}
\definecolor{later}{rgb}{0.7, 0.7, 0.7}
\newcommand{\later}[1]{{ \color{later} {#1} }}

%
\newcommand{\red}[1] {{ \color{red} #1}}
\newcommand{\green}[1] {{ \color{green} #1}}
\newcommand{\blue}[1] {{ \color{blue} #1}}
\newcommand{\magenta}[1] {{ \color{magenta} #1}}
%
\newcommand{\AK}[1] {{\color{red} #1}}
\newcommand{\DT}[1] {{\color{blue} #1}}
\newcommand{\YS}[1] {{\color{magenta} #1}}
%%%%%%%%%%%%%%%%%%%%%%%%%%%%%%%%%%%%%%%%%%%%%%%%%%%%%%%%%%%%%%%%%

%%%%%%%%%%%%%%%%%%%%%%%%%%%  new commands
% Peclet number
\newcommand{\Pe}{\textrm{Pe}}
% for angle brackets
\usepackage{textcomp}
\newcommand{\angb}[1]{\ifmmode\langle#1\rangle\else\textlangle#1\textrangle\fi}
%%%%%%%%%%%%%%%%%%%%%%%%%%%%%%%%%%%%%%%%%%%%%%%%%%%%%%%%%%%%%%%%%

%%%%%%%%%%%%%%%%%%%%%%%%%%%  section numbering
\setcounter{secnumdepth}{2}
\addtocounter{secnumdepth}{2}

\renewcommand{\thesection}{\Roman{section}}
\renewcommand{\thesubsection}{\Alph{subsection}}
\renewcommand{\thesubsubsection}{\arabic{subsubsection}}

%%%%%%%%%%%%%%%%%%%%%%%%%%%%%%%%%%%%%%%%%%%%%%%%%%%%%%%%%%%%%%%%%

%%%%%%%%%%%%%%%%%%%%%%%%%%%   for ref command of section label 
\makeatletter
\def\p@subsection{}
\def\p@subsubsection{}
\makeatother
%%%%%%%%%%%%%%%%%%%%%%%%%%%%%%%%%%%%%%%%%%%%%%%%%%%%%%%%%%%%%%%%%

%%% BEGIN DOCUMENT
\begin{document}

\title{ 
Scaling laws for two-dimensional dendritic crystal growth in a narrow channel
}


\author{Y. Song}
%\email[]{songy20@llnl.gov}
%\altaffiliation[Current address: ]{Lawrence Livermore National Laboratory, Livermore, California 94550, United States}
\affiliation{Department of Physics and Center for Interdisciplinary Research
on Complex Systems, Northeastern University, Boston, MA 02115 USA}
\affiliation{
Materials Science Division, Lawrence Livermore National Laboratory, Livermore, CA 94550, USA
}


\author{D. Tourret}
%\email[]{damien.tourret@imdea.org}
\affiliation{IMDEA Materials Institute, Getafe, 28906 Madrid, Spain}

\author{A. Karma}
\email[Corresponding author: ]{a.karma@northeastern.edu}
\affiliation{Department of Physics and Center for Interdisciplinary Research
on Complex Systems, Northeastern University, Boston, MA 02115 USA}


% \pacs{ ????? } not necessary anymore
\date{\today}

\begin{abstract}

% length limit: <= 600 characters
% currently, 576 characters (incl. space) 
We investigate analytically and computationally the dynamics of 2D needle crystal growth from the melt in a narrow channel. 
Our analytical theory predicts that, in the low supersaturation limit, the growth velocity $V$ decreases in time $t$ as a power law $V \sim t^{-2/3}$, which we validate by phase-field and dendritic-needle-network simulations. 
Simulations further reveal that, above a critical channel width $\Lambda \approx 5l_D$, where $l_D$ the diffusion length, needle crystals grow with a constant $V<V_s$, where $V_s$ is the free-growth needle crystal velocity, and approaches $V_s$ in the limit $\Lambda\gg l_D$. 

\end{abstract}

\maketitle
%\tableofcontents

%%%%%%%%%%%%%%%%%%%%%%%%%%%%%%%%%%%%% 

Solid-liquid interfaces exhibit different patterns depending on their growth conditions~\cite{Kurz19}. 
When unconstrained, i.e. growing freely in an infinite domain at a given driving force (e.g. undercooling, supersaturation), a dendritic crystal is known to reach a steady growth velocity. 
The growth velocity and tip morphology of such a free dendrite are given by combining solute and/or heat conservation \cite{Ivantsov47} with 
the solvability condition that incorporates the effect of the anisotropy of the interface excess free energy  of the solid-liquid interface (surface tension)~\cite{Langer87, Langer89, Barbieri89, BAmar93}.
When the dendrite is free of lateral interactions from neighboring crystal, secondary (or higher-order) branches can emerge due to thermal fluctuations~\cite{Pieters86, Barber87, Langer87PRA, Brener93, BrenerTemkin95}, which leads to the formation of complex microstructures. 
In contrast, the constrained growth of a dendrite within a channel reduces to the well-known Saffman-Taylor problem~\cite{KesslerKoplikLevine86, Langer89, BrenerEtAl88, CouderEtal86, Kupferman95, BrenerMuller93, BAarBrener95}.
The resulting crystal shape is a symmetric finger or an unstable widening cell above or below a critical undercooling, respectively~\cite{SabouriGhomi01, brenerKessler02, BrenerMuller93, Kupferman95}. 
The threshold between these morphologies are linked to the surface tension~\cite{brenerKessler02, BrenerMuller93, Kupferman95}. 

Theoretical and numerical investigations of 2D and 3D crystal growth in a channel~\cite{KesslerKoplikLevine86, Langer89, BrenerEtAl88, CouderEtal86, Kupferman95, BrenerMuller93, BAarBrener95, Xing15, KassnerEtAl10} have mostly focused on the dendrite tip morphology for a high undercooling ($\Delta$) or solute supersaturation ($\Omega$). 
Exploring lower undercoolings, two-dimensional phase-field studies~\cite{SabouriGhomi01} have shown that a dendrite in a channel slows down with a power-law of time below an undercooling $\Delta=0.5$, along with a widening of the tip. 
For a higher $\Delta$, the dendrite shows a steady state growth with a symmetric dendrite tip. 
The transition between the two growth regimes is close to the morphological threshold $\Delta=0.49$ of a symmetrical finger~\cite{brenerKessler02, brenerKessler02,Kupferman95}. 

While these studies have shown the existence of these two growth regimes (power-law deceleration and a steady state growth)~\cite{SabouriGhomi01, Xing15}, an analytical theory for the decelerating dynamics and the transition between regimes remains lacking. 
In this letter, we study the growth of a 2D binary alloy dendrite in a channel for a low solute supersaturation $\Omega$ --- readily generalizable to the growth of a pure element at a low undercooling $\Delta$. 
We derive an analytical law for the growth of the confined dendrite by combining the solvability condition~\cite{Langer87,Langer89,Barbieri89,BAmar93} and solute conservation considering a sharp needle-like crystal~\cite{TourretKarma13}. 
Our analytical derivation predicts that the dendrite slows down with time $t$ as $V \sim A t^{-2/3}$, where the prefactor $A$ depends directly and nonlinearly upon the channel width $\Lambda$. 
We verify the theory via two numerical approaches, namely using phase-field (PF) ~\cite{Karma01,Echebarria04} and the dendritic-needle-network (DNN) models ~\cite{TourretKarma13,TourretKarma16}. 

%%%%%%%%%%%%%%%%%%%%%%%%%%%%%%%%%%%%%%%%%%%%%%%%%%%%%% PF model

The PF approach is a powerful computational method to simulate complex interface dynamics in various phenomena~\cite{LQChen02,Boettinger02}. 
The method relies on a ``phase field", or order parameter, which has a given value within different phases or grains, and varies continuously through interfaces.
In solidification, taking into account anisotropic properties of solid-liquid interfaces, PF simulations can reproduce the evolution of complex anisotropic structures~\cite{Kobayashi93,Echebarria10}.
An asymptotic analysis permits the use of a diffuse interface much wider than the actual physical interface~\cite{KarmaRappel98}. 
For an alloy, the introduction of an ``anti-trapping'' term corrects the numerically-induced solute trapping due to the artificially wide interface, thus ensuring quantitative and computationally-efficient predictions~\cite{Karma01,Echebarria04}.
Combined with additional numerical techniques, such as parallelization and nonlinear preconditioning \cite{Glasner01}, resulting calculations provide quantitative simulations in two (2D) and three (3D) dimensions, at time and length scales directly comparable to experiments~\cite{Bergeon13, TourretKarma15, Clarke17, Tourret17}. 

The DNN model~\cite{TourretKarma13,TourretKarma16} was developed to simulate dendrite growth at low $\Omega$.
In this regime, PF simulations are challenging, due to the extreme separation of scale between the dendrite tip radius $\rho$ and the solute diffusion length $l_D \equiv D/V$, where $D$ is the diffusivity in the liquid and $V$ is the dendrite tip velocity. 
While PF typically requires a grid spacing much lower than the dendrite tip radius, DNN simulations remain accurate using a grid spacing close to, or even above, a dendrite tip radius~\cite{TourretKarma13, TourretKarma16}, which provides a great computational advantage. 

The DNN model consists of solving the transport of the solute field $u$ in the liquid phase (e.g., the diffusion equation), interacting with a hierarchical network of needle-like branches at equilibrium. 
The growth velocity of each dendritic tip is obtained by combining solute balances at two distinct length scales~\cite{TourretKarma13}.
At a scale $\gg\rho$, curvature effects can be neglected and a dendritic branch can be represented as a sharp line segment at fixed equilibrium concentration $u = 0$. 
On the other hand, at a scale $\ll l_D$, the solute field relaxes fast enough to assume a Laplacian $u$ field in the vicinity of the tip. 
The analytical solution to this problem~\cite{DerridaHakim92} exhibits a square-root singularity of the normal gradient of the field in the vicinity of the tip with a ``flux intensity factor'' $\mathcal F$ that is a measure of the incoming flux towards the dendrite tip. 
Using an analogy with fracture mechanics, namely with the stress field ahead of a crack tip submitted to anti-plane shear (mode III) loading, $\mathcal F$ can be calculated with the J-integral commonly used in fracture mechanics~\cite{Rice68,BAmar02,TourretKarma13}.
The resulting conservation equation reads
\begin{equation}
\label{eqn:FIFcondition}
\rho V^2 = 2 D^2 \mathcal{F}^2 / d_0 \, ,
\end{equation}
where $d_0$ is the capillary length. 
By combining the above equation and the solvability condition~\cite{Langer87,Langer89,Barbieri89,BAmar93},
\begin{equation}
\label{eqn:Solvcondition}
\rho^2 V = 2 D d_0 / \sigma \, ,
\end{equation}
where $\sigma$ is the tip selection parameter fixed by the interface anisotropy~\cite{Barbieri89}, the tip radius and growth velocity of a needle tip is given by~\cite{TourretKarma13}
\begin{align}
\label{eqn:FIF_radius}
\frac{\rho(t)}{d_0} &= \left[ \frac{ 2 }{ \sigma^2 \mathcal{F}(t)^2} \right]^{1/3} \, ,
\\
\label{eqn:FIF_velocity}
\frac{V(t) d_0 }{D} &= \left[ 2 \sigma \mathcal{F}(t)^4 \right]^{1/3} \, .
\end{align}
The transient evolution of the solute field around the dendrite tip is captured by the time-dependent $\mathcal{F}(t)$.

For a needle-like dendrite at $\Omega\ll 1$, the 2D Ivantsov solution~\cite{Ivantsov47} reduces to $\Omega\approx\sqrt{\pi\Pe}$, where $\Pe=\rho V/(2D)$ is the P\'eclet number.
Then, the steady radius $\rho_s$ and velocity $V_s$ of a free dendrite become~\cite{TourretKarma13}
\begin{align}
\label{eqn:Steady_radius}
\frac{\rho_s}{d_0} &= \frac{ \pi }{ \sigma \Omega^2} \,, 
\\
\label{eqn:Steady_velocity}
\frac{ V_s d_0}{D } &= \frac{ 2 \sigma \Omega^4}{ \pi^2 } \, .
\end{align}

%%%%%%%%%%%%%%%%%%%%%%%%%%%%%%%%%%%%%%%%%%%%%%%%%% analytical solution

Let a needle-like dendrite grow within a channel of width $\Lambda$. 
For $\Lambda \ll l_D$, the solute field surrounding the needle tip is essentially Laplacian with $\nabla^2 u=0$. 
Applying the J-integral around the tip in a channel yields that $\mathcal{F}(t)$ depends on the solute gradient, $\partial u/\partial x$, ahead of the tip in the needle growth direction $x$ as~\cite{Rice68,BAmar02,TourretKarma13}
\begin{equation}
\label{eqn:FIFsq}
\mathcal{F}(t)^2 = \frac{ \Lambda d_0}{ 2 \pi } \left( \frac{ \partial u }{ \partial x}\right)^2 \, .
\end{equation}
For $\Lambda \ll l_D$, the solute concentration profile can also be considered one-dimensional (1D), such that the gradient in front of the needle follows the Zener approximation~\cite{Zener49, Ivantsov47}
\begin{equation}
\frac{ \partial u }{ \partial x}  \approx \frac{ \Omega }{ \sqrt{\pi D t} } \, .
\end{equation} 
Combining these last two equations with Eq.~\eqref{eqn:FIF_velocity}, the needle tip velocity can be expressed as 
\begin{align}
\label{eqn:Vorigin}
V(t) = \left ( \frac{ \sigma \Lambda^2 }{2 \pi^4 } \frac{ D }{ d_0 }  \Omega^4 \right)^{1/3} t^{-2/3}  \, .
\end{align}
Normalizing time and space using theoretical steady-state values of $\rho_s$ (Eq.~\eqref{eqn:Steady_radius}) and $V_s$ (Eq.~\eqref{eqn:Steady_velocity}) leads to
\begin{align}
\label{eqn:Vnarrow}
\widetilde{V} (\tilde{t}) =  \frac{V(t)}{V_s} = \left( \frac{\widetilde{\Lambda}}{2 \pi \tilde{t} } \right)^{2/3}  \, ,
\end{align}
where $\tilde{t} = t V_s/\rho_s$ and $\widetilde{\Lambda} = \Lambda/\rho_s$.

%%%%%%%%%%%%%%%%%%%%%%%%%%%%%%%%%%%%%%%%%%%%%%%%%% results

%%%%%%%%%%%%%%%%%%%%%%%%%%%%%%%%%%%%%%%%%%%%%%%%%% Figure 1
\begin{figure}[!b]
\includegraphics[width=3.4 in]{Fig01.pdf}
\caption{
Calculated growth velocities (a) and morphological evolution (b)-(c) of a dendrite in a channel of width $\widetilde \Lambda=44.5$ at a solute supersaturation $\Omega=0.05$.
Both DNN (blue solid line) and PF (red solid line) results exhibit a power law close to the analytical theory, Eq.~\eqref{eqn:Vnarrow} (pink dashed line).
Morphologies in (b) and (c) show solid-liquid interface locations at $\tilde{t}=0.01$, 0.1, 1, 10, and 100 from the DNN and PF simulations, respectively. 
}
\label{fig:Narrow}
\end{figure}
%%%%%%%%%%%%%%%%%%%%%%%%%%%%%%%%%%%%%%%%%%%%%%%%%% Figure 1

In order to validate the scaling law \eqref{eqn:Vnarrow} for a dendrite growing in a narrow channel at low $\Omega$, we first performed both PF and DNN simulations at $\Omega=0.05$, whose results for $\widetilde{\Lambda} = 44.5$ are shown in Fig.~\ref{fig:Narrow}(a). 
Both PF (red solid line) and DNN (blue solid line) predictions follow Eq.~\eqref{eqn:Vnarrow} (pink dashed line). 
While the DNN needle effectively remains a sharp line interacting with the solute field, its virtual thickness can still be estimated by time integration of the fluxes along its sides (see details in SM and Ref.~\cite{TourretKarma13}). 
The resulting interface and its time evolution, shown at $\tilde{t} =  0.01, 0.1, 1, 10,$ and $100$ in Fig.~\ref{fig:Narrow}(b), agrees reasonably with the interface predicted by the PF method (Fig.~\ref{fig:Narrow}(c)). 

Because the DNN dendrite remains infinitely sharp, thus matching the underlying assumption of Eq.~\eqref{eqn:Vnarrow}, the DNN-predicted velocity has a prefactor closer to that of Eq.~\eqref{eqn:Vnarrow} than that from the PF simulation.
In the latter, the dendrite actually thickens, thus partially filling the channel and effectively increasing the solute gradient in the $x$ direction and yielding a higher velocity than that predicted for a sharp needle.
These results are consistent with previous PF studies~\cite{SabouriGhomi01, Xing15}, which showed that the power law deceleration observed at a relatively low P\'eclet number disappeared when the undercooling was increased and the dendrite was progressively filling up the channel.

In order to investigate the effect of the dendrite volume (area) on its growth dynamics, we performed additional PF simulations with channel widths $\widetilde{\Lambda} = 4.2$ and $10.9$. 
In Fig.~\ref{fig:TooNarrow}(a), $\widetilde{V}$ is scaled by $\widetilde{\Lambda}^{2/3}$, in order to represent all simulations on the same graph, comparing them with the expected $\widetilde{V}/\widetilde{\Lambda}^{2/3} = (2 \pi \tilde{t} )^{-2/3}$ (pink dashed line) from Eq.~\eqref{eqn:Vnarrow}. 
%
%%%%%%%%%%%%%%%%%%%%%%%%%%%%%%%%%%%%%%%%%%%%%%%%%% Figure 2
\begin{figure}[!h]
\includegraphics[width=3.4 in]{Fig02.pdf}
\caption{
PF-predicted growth velocities (a) and morphological evolutions (b-d) of dendrites for different channel widths $\widetilde \Lambda=44.5$ (red), $10.9$ (green), and $4.2$ (purple solid line), compared to the analytical law with $\widetilde V\sim \tilde{t}^{-2/3}$ (pink dashed line).
Panels (b), (c), and (d) show snapshots of the interface locations at $\tilde t=0.01$, 0.1, 1, 10, 100 for $\widetilde \Lambda=44.5$, 10.9 and 4.2, respectively. 
They are scaled with respect to $\Lambda/2$, i.e., with $x' = 2 x / \Lambda$ and $y' = 2 y / \Lambda$. 
}
\label{fig:TooNarrow}
\end{figure}
%%%%%%%%%%%%%%%%%%%%%%%%%%%%%%%%%%%%%%%%%%%%%%%%%% Figure 2
%
Solid lines show results with $\widetilde{\Lambda} \simeq$ 44.5 (red), 10.9 (green), and 4.2 (purple). 
A wider channel results in a longer initial transient time.
Nevertheless, once a power-law regime is reached ($\tilde{t}\geq1$) the widest channel leads to the best agreement with Eq.~\eqref{eqn:Vnarrow}.

Interface shapes at $\tilde{t} = 0.01$, 0.1, 1, 10, and $100$ are drawn in Fig.~\ref{fig:TooNarrow}(b)-(d) for $\widetilde{\Lambda} = 44.5$, 10.9, and 4.2, respectively, with space coordinates normalized by the channel half-width.
As expected, narrower channels (Fig.~\ref{fig:TooNarrow}(d)) get filled by the dendrite faster than wider channels. 
These results confirm that, for a given $\Omega$, the dendrite velocity deviates from the proposed growth law when $\Lambda$ decreases, as the assumption that $\Lambda\gg\rho$ is no longer valid.

Regardless of the channel width, dendrites slow down with an exponent close to the predicted $V \sim t^{-2/3}$ law.
Once the effect of confinement becomes predominant, as long as $\rho\ll\Lambda$, the constancy of $\rho^2 V$~\cite{Langer87,Langer89,Barbieri89,BAmar93} implies that $\rho$ increases with a corresponding $\rho \sim t^{1/3}$.
Hence, as $\rho$ increases, the problem approaches that of a 1D planar interface, known to follow $V\sim t^{-1/2}$~\cite{Zener49, Pelce88,BrenerMuller93}. 
Hence, even though dendrites in Fig.~\ref{fig:TooNarrow} are expected to retain a finger morphology rather than asymptotically tending toward a planar interface, the higher exponent exhibited by the thicker dendrites (i.e. narrower channel, e.g. Fig.~\ref{fig:TooNarrow}(d)) likely indicates an intermediate state between dendritic ($V\sim t^{-2/3}$) and planar ($V\sim t^{-1/2}$) growth.

If the channel is very wide, the resulting free dendrite should not be affected by the domain boundaries and thus reach a steady state velocity $\widetilde{V}=1$. 
Consequently, for a given supersaturation, we expect that there exists a critical channel width $\Lambda_c$ that delimits these two growth regimes.
We performed simulations with various $\Lambda$ at a given $\Omega$ in order to identify ${\Lambda}_c$, namely with $\Omega=0.2$ (PF) and $\Omega = 0.2$, 0.1, and 0.05 (DNN). 
The diffusive interaction between the dendrite and the channel boundaries is expected to be the key mechanism that determines whether the dendrite will decelerate or reach a steady state. 
Hence, results presented in Fig.~\ref{fig:TooWide} are scaled with respect to the diffusion length ${l}_D$. 
For $\Omega=0.2$, 0.1, and 0.05, the diffusion lengths of a free dendrite are ${l}_D / \rho_s = 39.3$, 157.1, and 628.3, respectively. 

%%%%%%%%%%%%%%%%%%%%%%%%%%%%%%%%%%%%%%%%%%%%%%%%%% Figure 3
\begin{figure*}[!ht]
\includegraphics[width=6.0 in]{Fig03.pdf}
\caption{
Growth velocities of a dendrite within different channel widths $\Lambda$. 
(a) shows PF results at $\Omega=0.2$; (b), (c), and (d) show DNN results at $\Omega=0.2$, 0.1, and 0.05, respectively. 
The channel widths are scaled with respect to the diffusion length ${l}_D / \rho_s = 39.3$, 157.1, and 628.3 for $\Omega=0.2$, 0.1, and 0.05, respectively.
Within a narrow channel, the dendrite is decelerated with the power law (red dashed lines). Otherwise, it approaches a steady state (blue solid lines).
}
\label{fig:TooWide}
\end{figure*}
%%%%%%%%%%%%%%%%%%%%%%%%%%%%%%%%%%%%%%%%%%%%%%%%%% Figure 3

In PF simulations (Fig.~\ref{fig:TooWide}(a)), growth velocities in the three narrowest channels, namely $\Lambda/l_D=$ 0.3, 0.7 and 1.4 (red dashed lines) decelerate towards the proposed power law. 
As observed earlier, as ${\Lambda}$ is wider, it takes a longer time to reach the $\tilde{t}^{-2/3}$ growth regime.  
In particular, at ${\Lambda}/l_D = 107.2$ (lightest red dashed line), the dendrite is still well within a transition regime towards the power law at the end of the simulation.
On the other hand, the dendrite velocity for ${\Lambda}/l_D = 5.5$ (blue solid line) does not show any inflexion during the simulated time, and seems to approach a steady-state velocity. 

DNN simulations allow us to explore a broader range of parameters, and show the similar results, namely with $\Omega=0.2$, 0.1, and 0.05. 
Results, shown in Fig.~\ref{fig:TooWide}(b)-(d), exhibit the same behavior as in PF simulations (Fig.~\ref{fig:TooWide}(a)).
Dendrites in narrower channels (red dashed lines) slow down towards the theoretical power law.
Meanwhile, dendrites in wider channels (blue solid lines) reach a constant growth velocity. 
At $\Omega=0.05$, the distinction between decelerating and steady regime is not obvious within the time range represented in Fig.~\ref{fig:TooWide}(d), but it is appears unambiguously looking at longer simulations (see Supplementary Material).

Both PF and DNN results exhibit two distinct growth behaviors: the power law deceleration at low $\Lambda$ and a steady state growth at higher $\Lambda$. 
The crossover between the two distinct growth dynamics occurs at $ 2.7 < \Lambda_c/l_D < 5.5$ in PF simulations and at
$4.8 < \Lambda_c/l_D < 5.1$ according to DNN results. 
We conclude that the critical channel width $\Lambda_c$ is approximately $5\, {l}_D$. 
Interestingly, this crossover at $\Lambda\approx5\, {l}_D$ is consistent with other results showing that the transition between quasi-2D confined growth and 3D growth regimes occurs at a sample height of approximately $5\, {l}_D$ in 3D DNN simulations~\cite{DNNCET}. 

One unexpected observation from Fig.~\ref{fig:TooWide} is that, at ${\Lambda}>{\Lambda}_c$, dendrites can reach a steady state velocity that is notably smaller than $V_s$.
In such cases, the channel boundaries still interact with the dendrite even if ${\Lambda}$ is wide enough for a dendrite to reach a steady state. 
In our simulations, the reached steady state velocity are as low as about $0.6 V_s$. 
The channel width necessary for the dendrite to stabilize at a steady velocity close to the free dendrite theoretical velocity $V_s$ can be as high as over a hundred times the diffusion length at relatively high $\Omega=0.2$, as seen in Fig.~\ref{fig:TooWide}(b).

%%%%%%%%%%%%%%%%%%%%%%%%%%%%%%%%%%%%%%%%%%%%%%%%%%%%%%%%%%%%%%%%%%

In summary, we developed an analytical theory for transient dendrite growth in a channel that predicts the power law deceleration $V \sim t^{-2/3}$. This  result adds to our theoretical understanding of transient dendrite growth, which has been so far limited to early stage growth in an unconstrained geometry (described by the scaling law $V \sim t^{-2/5}$) \cite{Almgren93}. Both PF and DNN simulations of dendritic growth within a narrow channel show that this power law holds as long as the channel width $\Lambda$ is larger than the dendrite tip radius $\rho$.
PF simulations show that, as the dendrite slows down, following solvability theory~\cite{Langer87,Langer89,Barbieri89,BAmar93}, its radius increases, the filling of the channel by the dendrite increases, and the velocity exponent rises above $-2/3$, thus getting closer to that expected for a planar interface ($V \sim t^{-1/2}$).
On the other hand, when $\Lambda$ is larger than a critical width $\Lambda_c \approx 5 l_D$, dendrites can reach a steady state with a constant velocity.
When $\Lambda$ is slightly higher than $\Lambda_c$, the dendrite can stabilize at a velocity lower than $V_s$. 
This velocity increases towards $V_s$ as $\Lambda$ increases.

This study remains to be extended to three dimensions.
PF studies revealed that 3D growth in a narrow channel also exhibits a power law deceleration~\cite{Xing15} with a gradual transition toward a steady growth velocity~\cite{Xing15, DNNCET}. 
The lack of known analytical solution for the diffusion field surrounding an infinitely sharp needle --- similar to that used in 2D in this Letter --- makes the current approach not readily applicable, such that an analytical description of the deceleration regime and the corresponding critical width in 3D remains missing. 



%%%%%%%%%%%%%%%%%%%%%%%%%%%%%%%%%%%%%%%%%%%%%%%%%%%%%%%%%%%%%%%%%%%%%%%%
\section*{ ACKNOWLEDGEMENT }

This research was primarily supported by NASA grants NNX16AB54G and 80NSSC19K0135.
D.T. acknowledges financial support from the Spanish Ministry of Science through the Ramon y Cajal grant RYC2019-028233-I.
Y.S. acknowledges that this work was partially performed under the auspices of the U.S. Department of Energy by Lawrence Livermore National Laboratory under Contract DE-AC52-07NA27344.
 

%%%%%%%%%%%%%%%%%%%%%%%%%%%%%%%%%%%%%%%%%%%%%%%%%%%%%%%%%%%%%%%%%%%%%%%%%%%%
%\newpage

%bibliography
\begin{thebibliography}{99}

% Progress in modelling solidification microstructures in metals and alloys: dendrites and cells from 1700 to 2000
\bibitem{Kurz19} W. Kurz, D.J. Fisher, R. Trivedi, Int. Mater. Reviews  64, 311-354 (2019).
%Temperature field around a spherical, cylindrical, and needle-shaped crystal, growing in a pre-cooled melt
\bibitem{Ivantsov47} G.P. Ivantsov, {\it Dokl. Akad. Nauk SSSR} {\bf 58}, 567-569 (1947).
% 
\bibitem{Langer87} J. S. Langer, Lectures on the Theory of Pattern Formation, Chance and Matter (Les Houches, Session XLVI), edited by Souletie J., Bannimenus J. and R. Stora, Amsterdam, North-Holland, p. 629 (1987).
% Dendrites, Viscous Fingers, and the Theory of Pattern Formation
\bibitem{Langer89} J. S. Langer, Science {\bf 243}, 1150-1156 (1989).
% Predictions of dendritic growth rates in the linearized solvability theory
\bibitem{Barbieri89} A. Barbieri and J. S. Langer, Rhys. Rev. A {\bf 39}, 5314-5325 (1989).
% Theory of Pattern Selection in Three-Dimensional Nonaxisymmetric Dendritic Growth
\bibitem{BAmar93} M. Ben Amar and E. Brener, Phys. Rev. Lett. {\bf 71}, 589-592 (1993)
% Noise-driven sidebranching in the boundary-layer model of dendritic solidification.
\bibitem{Pieters86} R. Pieters and J. S. Langer, Phys. Rev. Lett. {\bf 56} 1948-1951 (1986).
% Dynamics of dendritic sidebranching in the two-dimensional symmetric model of solidification; Dendritic sidebranching in the three-dimensional symmetric model in the presence of noise
\bibitem{Barber87} M. N. Barber, A. Barbieri and J. S. Langer, Phys. Rev. A {\bf 36}, 3340-3349 (1987).
% Dendritic sidebranching in the three-dimensional symmetric model in the presence of noise
\bibitem{Langer87PRA} J. S. Langer, Phys. Rev. A {\bf 36}, 3350-3358 (1987).
% Needle-Crystal Solution in Three-Dimensional Dendritic Growth
\bibitem{Brener93} E. Brener, Phys. Rev. Lett. {\bf 71}, 3653-3656 (1993).
% Noise-induced sidebranching in the three-dimensional nonaxisymmetric dendritic growth
\bibitem{BrenerTemkin95} E. Brener and D. Temkin, Phys. Rev. E {\bf 51}, 351-359 (1995).
% Dendrite growth in a channel
\bibitem{KesslerKoplikLevine86} D. A. Kessler, J. Koplik, and H. Levine, Phys. Rev. A {\bf 34}, 4980-4987 (1986).
%Growth of a needle-shaped crystal in a channel
\bibitem{BrenerEtAl88} E. A. Brener, M. B. Geilikman, and D. E. Temkin, Sov. Phys. JETP {\bf 67}, 1002-1009 (1988)
% Narrow fingers in the Saffman-Taylor instability
\bibitem{CouderEtal86} Y. Couder, N. Gerard, and M. Rabaud, Phys. Rev. A {\bf 34} 5175-5178 (1986).
%
\bibitem{BAarBrener95} M. Ben Amar and E. Brener, Phys. Rev. Lett. {\bf 75}, 561-564 (1995). 
%; Physica (Amsterdam) 98D, 128 (1996).
% Crystal growth in a channel: Numerical study of the one-sided model
\bibitem{BrenerMuller93}  E. Brener, H. M\"uller-Krumbhaar, Y. Saito, and D. Temkin, Phys. Rev. E {\bf 47}, 1151-1155 (1993).
% Coexistence of symmetric and parity-broken dendrites in a channel  
\bibitem{Kupferman95} R. Kupferman, D. A. Kessler, and E. Ben-Jacob, Physica (Amsterdam) 213A, 451-464 (1995).
% commment on "Solidification of a Supercooled Liquid in a Narrow channel"
\bibitem{brenerKessler02} E. A. Brener and D. A. Kessler, Phys. Rev. Lett. {\bf 88}, 149601 (2002).
% Solidification of a Supercooled Liquid in a Narrow channel 
\bibitem{SabouriGhomi01} M. Sabouri-Ghomi, N. Provatas, and M. Grant, Phys. Rev. Lett. {\bf 86}, 5084-5087 (2001).
%Phase-field modeling of growth pattern selections in three-dimensional channels
\bibitem{Xing15} H. Xing, P. P. Duan, X. L. Dong, C. L. Chen, L. F. Du and K. X. Jin, Philosophical Magazine {\bf 95},1184–1200 (2015).
%
\bibitem{KassnerEtAl10} K. Kassner, R. Gu\'erin, T. Ducousso, and J.-M. Debierre, Phys. Rev. E {\bf 82} 021606 (2010). 
% Multiscale dendritic needle network model of alloy solidification
\bibitem{TourretKarma13} D. Tourret and A. Karma, Acta Mater. {\bf 61}, 6474-6491 (2013).
%PHASE-FIELD MODELS FOR MICROSTRUCTURE EVOLUTION
\bibitem{LQChen02} L. Q. Chen, Ann. Rev. Mater. Res. {\bf 32}, 113-140 (2002).
%PHASE-FIELD SIMULATION OF SOLIDIFICATION
\bibitem{Boettinger02} W. J. Boettinger, J. A. Warren, C. Beckermann, and A. Karma, Ann. Rev. Mater. Res. {\bf 32}, 163-194 (2002).
% Modeling and numerical simulations of dendritic crystal growth 
\bibitem{Kobayashi93}  R. Kobayashi, Physica D {\bf 63}, 410-423 (1993).
%Onset of sidebranching in directional solidification
\bibitem{Echebarria10} B. Echebarria, A. Karma, and S. Gurevich, Phys. Rev. E {\bf 81}, 021608 (2010).
%Quantitative phase-field modeling of dendritic growth in two and three dimensions
\bibitem{KarmaRappel98} A. Karma and W. J. Rappel , Phys. Rev. E {\bf 57}, 4323-4349 (1998).
%Phase-Field Formulation for Quantitative Modeling of Alloy Solidification
\bibitem{Karma01} A. Karma, Phys. Rev. Lett. {\bf 87}, 115701 (2001).
% Quantitative phase-field model of alloy solidification
\bibitem{Echebarria04} B. Echebarria, R. Folch, A. Karma, and M. Plapp, Phys. Rev. E {\bf 70}, 061604 (2004).
%Nonlinear Preconditioning for Diffuse Interfaces
\bibitem{Glasner01} K. Glasner, J. Comp. Phys. {\bf 174}, 695-711 (2001).
%Spatiotemporal Dynamics of Oscillatory Cellular Patterns in Three-Dimensional ?Directional Solidification
\bibitem{Bergeon13} N. Bergeon, D. Tourret, L. Chen, J.-M. Debierre, R. Gu\'erin, A. Ramirez, B. Billia, A. Karma and R. Trivedi, Phys. Rev. Lett. {\bf 110}, 226102 (2013).
% Growth competition of columnar dendritic grains: a phase-field study
\bibitem{TourretKarma15} D. Tourret and A. Karma, Acta Mater. {\bf 82}, 64-83 (2015).
% Microstructure selection in thin-sample directional solidification of an Al-Cu alloy: In situ X-ray imaging and phase-field simulations
\bibitem{Clarke17} A. J. Clarke, D. Tourret, Y. Song, S. D. Imhoff, P. J. Gibbs, J. W. Gibbs, K. Fezzaa, A. Karma, Acta Mater. {\bf 129}, 203-216 (2017).
% Grain growth competition during thin-sample directional solidification of dendritic microstructures: A phase-field study 
\bibitem{Tourret17} D. Tourret, Y. Song, A.J. Clarke, A. Karma, Acta Mater. {\bf 122}, 220-235 (2017).
% Three-dimensional dendritic needle network model for alloy solidification
\bibitem{TourretKarma16} D. Tourret, A. Karma, Acta Mater. {\bf 120}, 240-254 (2016).
% DNN CET 
\bibitem{DNNCET} P.A. Geslin, C.-H. Chen, A. M. Tabrizi, A. Karma, Acta Mater. {\bf 202}, 42-54 (2021); C.-H. Chen, A. M. Tabrizi, P.A. Geslin, A. Karma, Acta Mater. {\bf 202}, 463-477 (2021).
%
\bibitem{DerridaHakim92} B. Derrida and V. Hakim, Phys. Rev. A {\bf 45}, 8759-8765 (1992).
% Exact results with the J-integral applied to free-boundary flows
\bibitem{BAmar02} M. Ben Amar, J. R. Rice, J. Fluid Mech. {\bf 461}, 321-341 (2002).
% Needle models of Laplacian growth
\bibitem{Rice68} J. R. Rice, J. Appl. Mech. {\bf 35}, 379-386 (1968).
% Theory of Growth of Spherical Precipitates from Solid Solution
\bibitem{Zener49} C. Zener, J. Appl. Phys. {\bf 20}, 950-953 (1949).
% Unsteady Crystal Growth in a Channel in the Small Peclet Number Limit
\bibitem{Pelce88}  P. Pelce, Europhys. Lett. {\bf 7}, 453-457 (1988).
%%%%%%%%%%%%%%% 
\bibitem{Almgren93} R. Almgren, W.S. Dai and V. Hakim, Phys. Rev. Lett. {\bf 71} 3461 (1993).


%%%%%%%%%%%%%%%%%%%%%%%%%%%%%%%%%%%%%%%%%%%%%%%%%%%%%%%%%%%%%%%%%%%%%%%%%%%%
%%%%%%%%%%%%%%%%%%%%%%%%%%%%%%%%%%%%%%%%%%%%%%%%%%%%%%%%%%%%%%%%%%%%%%%%%%%%
%%%%%%%%%%%%%%%%%%%%%%%%%%%%%%%%%%%%%%%%%%%%%%%%%%%%%%%%%%%%%%%%%%%%%%%%%%%%


\end{thebibliography}

\end{document}









