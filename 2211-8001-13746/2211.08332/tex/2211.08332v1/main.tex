% CVPR 2023 Paper Template
% based on the CVPR template provided by Ming-Ming Cheng (https://github.com/MCG-NKU/CVPR_Template)
% modified and extended by Stefan Roth (stefan.roth@NOSPAMtu-darmstadt.de)

\documentclass[10pt,twocolumn,letterpaper]{article}

%%%%%%%%% PAPER TYPE  - PLEASE UPDATE FOR FINAL VERSION
% \usepackage[review]{cvpr}      % To produce the REVIEW version
% \usepackage{cvpr}              % To produce the CAMERA-READY version
\usepackage[pagenumbers]{cvpr} % To force page numbers, e.g. for an arXiv version

% Include other packages here, before hyperref.
\usepackage{graphicx}
\usepackage{amsmath}
\usepackage{amssymb}
\usepackage{booktabs}


% It is strongly recommended to use hyperref, especially for the review version.
% hyperref with option pagebackref eases the reviewers' job.
% Please disable hyperref *only* if you encounter grave issues, e.g. with the
% file validation for the camera-ready version.
%
% If you comment hyperref and then uncomment it, you should delete
% ReviewTempalte.aux before re-running LaTeX.
% (Or just hit 'q' on the first LaTeX run, let it finish, and you
%  should be clear).
\usepackage[pagebackref,breaklinks,colorlinks]{hyperref}

% XX-packages
\usepackage{booktabs}
\usepackage{array}
\newcolumntype{L}[1]{>{\raggedright\let\newline\\\arraybackslash\hspace{0pt}}m{#1}}
\newcolumntype{C}[1]{>{\centering\let\newline\\\arraybackslash\hspace{0pt}}m{#1}}
\newcolumntype{R}[1]{>{\raggedleft\let\newline\\\arraybackslash\hspace{0pt}}m{#1}}
\usepackage{tablefootnote}
\usepackage[font=small]{caption}
\captionsetup[table]{skip=3pt}
\usepackage{subcaption}
\usepackage{multirow}
\usepackage{enumitem}
\usepackage{float}
\usepackage{xcolor}

% Correction Tools

\newcommand{\HS}[1]{{\color{red}{{[Humphrey \bf #1 ]}}}}
\newcommand{\AW}[1]{{\color{blue}{{[Atlas: \bf #1 ]}}}}
\newcommand{\EZ}[1]{{\color{gray}{{[Eric: \bf #1 ]}}}}
\newcommand{\KW}[1]{{\color{orange}{{[Kai: \bf #1 ]}}}}
\newcommand{\XX}[1]{{\color{cyan}{{[Xingqian: \bf #1 ]}}}}

% \newcommand{\HS}[1]{}
% \newcommand{\AW}[1]{}
% \newcommand{\EZ}[1]{}
% \newcommand{\KW}[1]{}
% \newcommand{\XX}[1]{}


% Support for easy cross-referencing
\usepackage[capitalize]{cleveref}
\crefname{section}{Sec.}{Secs.}
\Crefname{section}{Section}{Sections}
\Crefname{table}{Table}{Tables}
\crefname{table}{Tab.}{Tabs.}


%%%%%%%%% PAPER ID  - PLEASE UPDATE
\def\cvprPaperID{3929} % *** Enter the CVPR Paper ID here
\def\confName{CVPR}
\def\confYear{2023}


\begin{document}

%%%%%%%%% TITLE - PLEASE UPDATE
\title{Versatile Diffusion: Text, Images and Variations All in One Diffusion Model}

\author{
    Xingqian Xu\textsuperscript{1},
    Zhangyang Wang\textsuperscript{2,3},
    Eric Zhang\textsuperscript{1},
    Kai Wang\textsuperscript{1},
    Humphrey Shi\textsuperscript{1,3} \\
{\small \textsuperscript{1}SHI Labs @ UIUC \& U of Oregon, \textsuperscript{2}UT Austin, \textsuperscript{3}Picsart AI Research (PAIR)}\\
{\small \textbf{\url{https://github.com/SHI-Labs/Versatile-Diffusion}}}
}

\twocolumn[{
\maketitle
\begin{center}
    \captionsetup{type=figure}
    \includegraphics[width=0.99\textwidth]{figure/teaser.pdf}
    \captionof{figure}{Demo results of our Versatile Diffusion (\textbf{VD}) framework on three directly supported tasks (\ie Figures \textbf{a}, \textbf{b}, and \textbf{c}) and three derived applications (\ie Figure \textbf{d}, \textbf{e}, and \textbf{f}). As shown in the captions, the three supported tasks are text-to-image, image-variation, and image-to-text. Figure \textbf{d} is a demo of the disentanglement between image semantics and style. Figure \textbf{e} shows the dual-guided generated samples using both image and text as context. Figure \textbf{f} shows the idea of image-text-image (I2T2I) controllable editing that converts images to text, modifies the text, and then generates new images.}
    \label{fig:teasor}
    \vspace{0.4cm}
\end{center}
}]
\thispagestyle{empty}


%%%%%%%%% ABSTRACT
\begin{abstract}

\begin{abstract}
% Modern ConvNets
Since the recent success of Vision Transformers (ViTs), explorations toward ViT-style architectures have triggered the resurgence of ConvNets.
% Novel view: interaction
In this work, we explore the representation ability of modern ConvNets from a novel view of multi-order game-theoretic interaction, which reflects inter-variable interaction effects w.r.t.~contexts of different scales based on game theory.
% MogaNet
Within the modern ConvNet framework, we tailor the two feature mixers with conceptually simple yet effective depthwise convolutions to facilitate middle-order information across spatial and channel spaces respectively.
% Experiments illustration
In this light, a new family of pure ConvNet architecture, dubbed MogaNet, is proposed, which shows excellent scalability and attains competitive results among state-of-the-art models with more efficient use of parameters on ImageNet and multifarious typical vision benchmarks, including COCO object detection, ADE20K semantic segmentation, 2D\&3D human pose estimation, and video prediction.
% Highlight results
Typically, MogaNet hits 80.0\% and 87.8\% top-1 accuracy with 5.2M and 181M parameters on ImageNet, outperforming ParC-Net-S and ConvNeXt-L while saving 59\% FLOPs and 17M parameters.
% code (arxiv & final version)
The source code is available at \url{https://github.com/Westlake-AI/MogaNet}.
\vspace{-1.0em}


\end{abstract}


\end{abstract}

%%%%%%%%% BODY TEXT
\section{Introduction}\label{sec:intro}

\section{Introduction}
\label{sec:introduction}
Reliable, fast, and efficient data processing is crucial given the growing volumes of data in both industry and research.
These needs are often addressed by using distributed dataflow frameworks like Spark~\cite{Zaharia2010}, and Flink~\cite{Carbone2015}.
As these frameworks' handle parallelism, distribution, and fault tolerance, they make it easier for users to create scalable data-parallel programs.
The resulting applications can use a variety of compute clusters for data processing.

However, it is still difficult to choose and configure resources in a way that closely meets user-specific goals and constraints~\cite{RajanKCK16,cloudcomputingchallenges2018}.
Numerous strategies have been put forth to assist users, and they can be grouped into two categories:
Model-based techniques~\cite{MaoAMK16,RajanKCK16,ShahAKW19,AlSayehS19,KirchoffXMR19,ChenLLWZ21silhouette,ScheinertTZWAWK21,WillTSBK21,AlSayehMJPS22} often rely on access to historical performance data, however, historical workload execution data is not always available.
Search-based techniques~\cite{AlipourfardLCVY17,HsuNFM18,bilal2020finding,klimovic2018selecta,fekry2020accelerating,MendesCRG20,LiuXL20,SongZLSFDS21} conduct costly profiling runs prior to executing the actual workload utilizing all, or a fraction, of the input data to iteratively create performance models.

Often enough though, the optimized resource configuration is only relevant for the workload at hand. 
Information about the underlying infrastructure are solely obtained implicitly, i.e., by measuring the performance of the target workload in one execution context.
As a consequence, a thorough understanding of utilized resources and their capabilities is lacking and insights gained cannot be easily transferred to other contexts, for instance, when profiling new workloads with different resource demands. 
This requires repeated profiling overhead for reoccurring or comparable workloads that could be avoided, rendering current approaches less resource-efficient than they could be.

Addressing these limitations, we present \emph{Perona}, a novel approach to explicit and robust infrastructure fingerprinting. 
It motivates the usage of common sets and configurations of benchmarking tools to assess the full capabilities of target infrastructures and to make the obtained benchmarking metrics directly comparable.
This explicit fingerprinting operation transparently reveals the characteristics of resources and allows ranking them.
Perona discards irrelevant benchmarking metrics in a data-driven manner by learning a dense, low-dimensional representation of input metric vectors. 
With these, more sophisticated resource decisions can be made for big data analytics, e.g., with regard to scheduling or resource allocations.
To be able to assess a recent benchmark execution, our approach incorporates results of prior benchmark executions, which is particularly useful for detecting resource degradation. 

\emph{Contributions}. The contributions of this paper are:

\begin{itemize}
    \item A novel approach for incorporating infrastructure fingerprinting into model-based methods for optimized resource configuration of workloads through ranking of resources and detection of degrading resource behavior.
    \item A method for context-aware representation learning of benchmark metrics, thereby not only discarding insignificant features but also taking prior benchmark runs and corresponding machine metrics into account. 
    \item An openly available implementation\footnote{\url{https://github.com/dos-group/perona-infrastructure-fingerprinting}} of Perona which we evaluated with regard to common metrics and in interplay with resource configuration methods for distributed dataflows and scientific workflows. 
\end{itemize}

\emph{Outline}. \autoref{sec:related_work} discusses the related work.
\autoref{sec:approach} describes the three main aspects of our approach in detail. 
\autoref{sec:evaluation} presents the results of our evaluation.
\autoref{sec:conclusion} concludes the paper and gives an outlook on future work.

\section{Related Works}\label{sec:related}

\textbf{Multi-modalities} are unions of information with different forms, including but not limited to vision, text, audio, \etc~\cite{mm_book0, mm_survey0}. Early deep learning work led by Ngiam \etal ~\cite{mm_dl0} learned a fused representation for audio and video. The similar idea was also adopted across vision and text label~\cite{mm_dl0}, and across vision and language~\cite{mm_dl2}. A part of multimodal approaches focused on zero-shot learning, for instance, DiViSE~\cite{mm_zscls0} targeted mapping images on semantic space from which unseen category labels can be predicted. Socher \etal~\cite{mm_zscls1} trained a recognition model with similar ideas in which images were projected on the space of text corpus. \cite{mm_zscls2} shared the same design as DiViSE but was upgraded for a large and noisy dataset. Another set of works~\cite{mm_cls0, mm_cls1,  mm_cls2,  mm_cls3}, focused on increasing classification accuracy via multimodal training: in which~\cite{mm_cls0} and~\cite{mm_cls1} did a simple concatenation on multimodal embeddings; ~\cite{mm_cls2} proposed a gated unit to control the multimodal information flow in the network; ~\cite{mm_cls3} surveyed FastText~\cite{fasttext} with multiple fusion methods on text classification. Meanwhile, multimodal training was also wide-adopted in detection and segmentation~\cite{rcnn,maskrcnn, oneformer}% Started from R-CNN~\cite{rcnn}, series of works structured with shared backbone and multiple heads in order to train and predict class ids, detection boxes~\cite{rcnn, fpn} and object masks~\cite{maskrcnn, oneformer} 
in one shot. Another topic, VQA~\cite{mm_vqa0, mm_vqa1}, conducted cross-modal reasoning that transferred visual concepts into linguistic answers. Methods such as~\cite{mm_vqa2, mm_vqa3} extracted visual concepts into neural symbolics, and ~\cite{mm_vqa4, mm_vqa5} learned additional concept structures and hierarchies.

\textbf{Multimodal generative tasks} involve simultaneous representation learning and generation/synthesis~\cite{mm_survey1}, in which representation networks~\cite{ae, vae, gan, vqvae, wavenet, prnet} with contrastive loss~\cite{clip, cl, mm_cl0, mm_cl1, mm_cl2} played an essential role. Specifically, our model VD adopts VAEs~\cite{vae} and CLIP~\cite{clip} as the latent and context encoders, which are two critical modules for the network. VD also shares the common cross-modal concepts such as domain transfer~\cite{cgan, cyclegan} and joint representation learning~\cite{mm_dl1, mm_gm0, mm_gm1}.

\textbf{Diffusion models} (DM)~\cite{dm_early0, ddpm} consolidate large family of methods including VAEs~\cite{vae, vqvae, vqvae2}, Markov chains~\cite{mcm0, dm_early0, mcm1, mcm2}, and score matching models~\cite{scorem0, scorem1}, \etc. Differ from GAN-based\cite{gan, biggan, stylegan2} and flow-based models~\cite{flow0, flow1}, DM minimizes the lower-bounded likelihoods~\cite{ddpm, scorem0} in backward diffusion passes, rather than exact inverse in flow~\cite{flow0} or conduct adversarial training~\cite{gan}. Among the recent works, DDPM~\cite{ddpm} prompted $\epsilon$-prediction that established a connection between diffusion and score matching models via annealed Langevin dynamics sampling~\cite{dm_early1, scorem0}. DDPM also shows promising results on par with GANs in unconditional generation tasks. Another work, DDIM~\cite{ddim}, proposed an implicit generative model that yields deterministic samples from latent variables. Compared with DDPM, DDIM reduces the cost of sampling without losing quality. Regarding efficiency, FastDPM~\cite{dm_fast0} investigated continuous diffusion steps and generalized DDPM and DDIM with faster sampling schedules. Another work, ~\cite{dm_fast1}, replaced the original fixed sampling scheme with a learnable noise estimation that boosted both speed and quality. ~\cite{dm_fast2} introduced a hieratical structure with progressive increasing dimensions that expedite image generations for DM. Regarding quality, ~\cite{dm_beat_gan} compared GANs with DMs with exhaustive experiments and concluded that DMs outperformed GANs on many image generation tasks. Another work, VDM~\cite{dm_vdm}, introduced a family of DM models that reaches state-of-the-art performance on density estimation benchmarks. Diffwave~\cite{dm_diffwave} and WaveGrad~\cite{dm_wavegrad} show that DM also works well on audio. ~\cite{dm_improved_ddpm} improved DDPM with learnable noise scheduling and hybrid objective, achieving even better sampling quality. 
\cite{dm_morecontrol} introduced semantic diffusion guidance to allow image or language-conditioned synthesis with DDPM.

\textbf{Text-to-image generation}, nowadays a joint effort of multimodal and diffusion research, has drawn lots of attention. Among these recent works, GLIDE~\cite{glide} adopted pretrained language models and the cascaded diffusion structure for text-to-image generation. DALL-E2~\cite{dalle2}, a progressive version from DALL-E~\cite{dalle}, utilized CLIP model~\cite{clip} to generate text embedding and adopted the similar hieratical structure that made 256 text-guided images and then upscaled to 1024. Similarly, Imagen~\cite{imagen} explored multiple text encoders~\cite{bert, t5, clip} with conditional diffusion models and explores the trade-offs between content alignment and fidelity via various weight samplers. LDM~\cite{ldm} introduced a novel direction in which the model diffuses on VAE latent spaces instead of pixel spaces. Such design reduced the resource needed during inference time, and its latter version, SD, has proven to be equally effective in text-to-image generation.



\section{Method}\label{sec:method}
In the below discussions, we use the Partial Markov Decision Process (POMDP) formation  since this setting fits in most real-world reinforcement learning tasks where the full state is not obtainable and the agent is only given observations from its onboard sensors \citep{Igl2018DeepVR}. A POMDP is defined as: $\mathcal{M}=(\mathcal{X}, \mathcal{A}, \mathcal{T}, r, \gamma)$, where  $\mathcal{X}$ denotes the observation space, $\mathcal{A}$ is the action space, $\mathcal{T}$ is the environment transition dynamics, $r$ is the reward function and $\gamma$ denotes the discount factor.

\subsection{Reusable agent-environment interaction models}
% \begin{wrapfigure}{r}{0.2\textwidth}
%     \centering
%     \includegraphics[width=0.2\textwidth]{figs/609px-Plato's_allegory_of_the_cave.jpg}\\
%     {\footnotesize Plato's allegory of the cave~\citep{Plato}. \normalsize}
%     %\label{fig:decentcem-a-architecture}
%     \vspace{-3mm}
% \end{wrapfigure}

\paragraph{(1) Embedded agency: small agent, big world}
% In 1969, Herbert Simon brought us the ``Simon's ant'' parable in his book ``{The Science of the Artificial}''\citep{simon1969science} (chapter 3) explaining that the complex agent behaviour we observed may not attribute to how complex the agent is, instead it may result from the interactions between the agent and the complex world, even the agent itself like an ant, is quite simple. It reminds us of a common setting in intelligent organisms: big world, small agent.

Unlike the \textit{Bayesian world model} approach \citep{ha2018world} that aims to construct a big model for the agent as a base knowledge about the world, the \textit{Embedded agency}\citep{orseau2012space} approach takes the perspective on how the small agent perceives, controls and gradually grows based on its own experience. Unlike a \textit{Bayesian world model} where states and models are objective, an \textit{embedded agency} approach is subjective\citep{agent} that agent's past experience will shape its behaviors in new tasks. This idea of intelligence closely relates to Descarte's theory of representationalism and mind-body dualism, and to the modern theory of embodied embedded cognition\citep{haugeland} in the philosophy of mind research. The \textit{embedded agency} approach is appealing in that it promises a scalable learning architecture to build a world model based on the agent's experience in a self-improvement manner, however, it is also ambitious. 
This paper investigates the core idea of \textit{embedded agency} about how an agent's past experience can shape its new task learning in a pre-training and reusing paradigm.  

\paragraph{(2) On agent-environment boundary}
Conceptually, from the \textit{embedded agency} perspective, a generally reusable model in heterogeneous environments and tasks should be the agent itself, since the agent is the only shared component across domains. Practically, using the agent's past experience to extract the agent model is difficult since the agent-environment boundary is not the same as we think in a physical system, like a robot. \citep{Jiang2019OnVF} provide a detailed analysis of this problem. \citep{sutton2018reinforcement} in chapter 3.1 explains that the agent-environment boundary is often task-specific in different abstraction levels and the boundary could change across tasks and environments, which makes exacting a general agent model from its various task data difficult. 

Instead of extracting an agent model, we propose to extract a general agent-environment interaction model that is commonly shared across tasks and environments, and thus can be viewed as an embodiment of the agent. This proposal is explained below.

\paragraph{(3) Generally reusable agent-environment interaction model}
In pre-training, suppose the agent's past experiences $\mathcal{D}$ are generated from M different environments and collected using unlimited behaviour policies induced by unlimited reward functions. This will compose a set of different POMDPs: $\{\mathcal{M}_{i}\}_{i=1}^{M}$, where $\mathcal{M}_{i}=(\mathcal{X}_{i}, \mathcal{A}, \mathcal{T}_{i}, \gamma)$ is a reward-free POMDP that drops the task reward for simplicity. $\mathcal{A}$ without a subscript $i$ means the agent's action space is unchanged across environments since the embedded agency considers the same agent across environments and tasks. Let's assign a one-hot vector encoding $Y_{i}$ as the label for each environment domain $\mathcal{M}_{i}$, then we have a set of domain labels $\mathcal{Y}=\{Y_{i}\}_{i=1}^{M}$, the collected dataset for pre-training can be denoted as $\mathcal{D}=\{D_{Y_{i}}\}_{i=1}^{M}$, where $D_{Y_{i}}$ is the dataset collection in environment domain $Y_{i}$.

We use successor features (SF)\citep{barreto2017successor} to capture the agent-environment interaction model since SFs summarize the dynamics induced by the environment when following a behaviour policy. The first step of building a generally reusable agent-environment interaction model is to learn cross-domain transferable successor features in order to extract a general agent-environment interaction model. For simplicity, we consider a random uniform behaviour policy $\pi_{0}$ for all environment domains. Recall that, SF is defined as the expected cumulative features of $\phi$ by following $\pi_{0}$ starting at a specific state. The SF of $(x^{i}, a)$ in environment domain $Y_{i}$ is defined as:
\begin{equation}
\begin{split}
    \psi^{\pi_{0},i}(x^{{i}}, a)  ={ \mathbb{E}}^{\pi_{0}}_{(x, a, x') \sim \mathcal{T}_{i}} [\sum_{t'=t}^{\infty} \gamma^{t'-t} \phi_{x_{t+1}} |X_{t}=x^{{i}}, A_{t}=a].
\end{split}
\end{equation}
, which satisfies the Bellman Equation as below:
\begin{equation}
\begin{split}
     \psi^{\pi_{0},i}(x^{{i}}, a)  & = \phi_{x^{i}} + \gamma {\mathbb{E}}^{\pi_{0}}_{(x, a, x') \sim \mathcal{T}_{i}} [\psi^{\pi_{0},i}(x^{{i}}_{t+1}, a_{t+1})|X_{t}=x^{{i}}, A_{t}=a]
\end{split}
\label{eq:SF_domain}
\end{equation}
, and it can be learned by minimizing the temporal difference (TD) error:
% Therefore, the successor features in environment domain $Y_{i}$ following a behaviour policy $\pi_{0}$ can be learned by minimizing the temporal difference (TD) error:
\begin{equation}
\begin{split}
     \delta_{sf, Y_{i}}^{2} = \norm {\phi_{x^{i}_{t}} + \gamma \psi^{\pi_{0},i}(x^{{i}}_{t+1}, a_{t+1}) - \psi^{\pi_{0},i}(x^{{i}}_{t}, a_{t})}
\end{split}
\label{eq:sf_loss}
\end{equation}
Next, let's consider a set of domains $\mathcal{Y}=\{Y_{i}\}_{i=0}^{M}$. The goal is to learn a successor feature approximation function $f(.;\theta_{sf}): \mathcal{X}_{i} \times \mathcal{A} \rightarrow 
 \boldsymbol {\psi}$ with parameters $\theta_{sf}$ that is transferable across all the domains $\mathcal{Y}$. Inspired by~\citep{feng2019self}, we add two constraints to Eq. \ref{eq:sf_loss} in order to make the learned SF cross-domain transferable, using the mutual information definition $I(.)$:
 \begin{equation}
\begin{split}
   {min} &  \hspace{0.3cm} \mathcal{L}_{sf}  =\frac{1}{M} \sum_{i=1}^{M}\mathbb{E}_{(x, a, x') \sim \mathcal{D}_{Y_{i}}}[\delta_{sf, Y_{i}}^{2}]\\
   s.t. & \hspace{0.3cm}  I(\mathbf{\psi}^{\pi_{0}}, Y) < \epsilon_{u}; \hspace{0.3cm} I(\mathbf{\psi}^{\pi_{0},i}, {x^{i}}) > \epsilon_{l}, \hspace{0.3cm} \forall i \in \{1, ..., M\}
 \end{split}
 \label{eq:total_loss}
\end{equation}
In the first constraint, $\psi^{\pi_{0}}$ is the SF for an arbitrary environment domain, and Y is the corresponding domain label. It limits the mutual information between an SF and its domain label to a threshold $\epsilon_{u}$ in order to make the learned successor feature domain-invariant. The domain index $Y_{i}$ is dropped since it generally applies to all domains. The second constraint term maintains the mutual information between an SF and its input domain-specific observations\footnote{We use the domain index $Y_{i}$ since it requires to estimating a specific domain's marginal distribution $p^{i}(x)$ that $x^{i} \sim p^{i}(x)$ during training, as detailed in Appendix. \ref{sec:app_cross_domain}.} above a threshold $\epsilon_{l}$, in order to prevent the optimization from collapsing to a trivial solution, for example, a random feature space is also domain-invariant but does not include any useful information for task learning. Technical details about optimizing Eq. \ref{eq:total_loss} are shown in Sec. 3.2.

Note that, learning a cross-domain transferable successor feature representation \textit{will not solve} the out-of-distribution (O.O.D.) problem when reusing it in downstream tasks with unseen changes, so that directly plugin the pre-trained successor features, as a common approach in (\citep{barreto2017successor,barreto2020fast}), \textbf{will not make our pre-trained model generally reusable}. We propose two techniques to tackle this problem, \textit{(1) embodied set construction} that discretizes the successor features into prototype sets (Sec. 3.2);  \textit{(2) feature projection} and \textit{projected Bellman updates} to enable learning stability-plasticity (Sec. 3.3). Combined together, we make the pre-trained agent-environment interaction model generally reusable.

% $\mathcal{Y}=\{Y_{1}, ..., Y_{i}, ...\}$ are the domain labels encoded in a one-hot embedding vector.

\subsection{Pre-training: embodied set construction}
The pre-training process is proposed as an embodied set construction method that has two steps: 
\begin{wrapfigure}[16]{R}{0.43\textwidth} %<-- Wrapfigure covers 6 lines
   \begingroup
\removelatexerror% Nullify \@latex@error
\begin{algorithm}[H]
% \setlength{\belowcaptionskip}{-10pt}
	\SetAlgoLined
	\small
	\KwIn{Offline dataset $\mathcal{D}=\{D_{Y_{i}}\}_{i=1}^{M}$, $\psi^{\pi_{0}}(x, a;\theta_{sf})$, embodied set size $N$}
	\KwResult{Embodied set structure $\Omega^{e}$}
	%\textcolor[rgb]{0.14,0.36,0.73}{\textbf{Initialization}}\\
	Initialize an empty embodied set $\Omega^{e}=\{\}$\\
	Initialize an empty successor feature vector list $\mathbf{L}_{sf}=\{\}$\\
	$\mathcal{D} \leftarrow $ Shuffle $(\mathcal{D})$\\
	\For{each $(x, a, x') \in \mathcal{D}$}{
	    \tcp{Compute cross-domain transferable successor features}
	    $\mathbf{L}_{sf} \leftarrow $ Append $\psi^{\pi_{0}}(x, a;\theta_{sf})$  \\
	}
	\tcp{Constructing embodied agent state set}
	K-means clustering ($\mathbf{L}_{sf}, N$)\\
	$\Omega^{e}=\{\mathbf{e}_{i}\}_{i=1}^{N}$ =  cluster-centers as behavior prototypes  \\
	\caption{Embodied Set Construction}
\end{algorithm}
\endgroup
  \end{wrapfigure}


% The inputs are the agent's past experience in multiple environment domains. The output is a constructed embodied set, which will be reused to support downstream task learning. The pre-training has three steps as below.
\paragraph{(1) Learn cross-domain transferable successor features}
Let a neural network parameterized with $\theta_{sf}$ to approximate the SF representation: $\psi^{\pi_{0}}(x, a;\theta_{sf})$, where $(x, a)$ comes from an arbitrary environment domain. Our aim is to use loss Eq. \ref{eq:total_loss} to find the optimal $\theta_{sf}$. By adding the Lagrangian multipliers $\lambda_{u}$ and $\lambda_{l}$, the Lagrangian dual of Eq. \ref{eq:total_loss} is:
\begin{equation}
\begin{split}
  \underset{\theta_{sf}}{min}  &  \hspace{0.3cm} \mathcal{L}_{sf} + \lambda_{u} I(\mathbf{\psi}^{\pi_{0}}, {Y}) - \lambda_{l} \sum_{i=1}^{M} I(\mathbf{\psi}^{\pi_{0},i}, {x^{i}}) \\
 \end{split}
 \label{eq:total_loss2}
\end{equation}
Directly optimizing the mutual information (MI) terms of Eq. \ref{eq:total_loss2} in high-dimensional space is challenging, we provide tractable solutions by approximating the upper bound and lower bound, as detailed in Appendix. \ref{sec:app_cross_domain}. Optimizing Eq. \ref{eq:total_loss2} will result in  transferable $\psi^{\pi_{0}}(x, a;\theta_{sf})$.


\paragraph{(2) Discretize to behavior prototypes: construct an embodied set structure}
To facilitate downstream task learning with our proposed \textit{embodied feature projection} and \textit{projected Bellman updates} (Sec. 3.3), we discretize the learned cross-domain successor features in three steps: (i) Compute all the successor features using the learned $\psi^{\pi_{0}}(x, a;\theta_{sf})$ from $\mathcal{D}$ containing all the environment domain samples. (ii) Cluster all the successor features using mini-batch K means. (iii) Collect the center of each cluster to construct a set structure $\Omega^{e}$. This is summarized in Algorithm 1. 

We call $\Omega^{e}$ as \textit{embodied set} since it is the component shared across environments and tasks that can be viewed as the embodiment of the agent. And each $\mathbf{e}_{i} \in \Omega^{e}$ is called a prototype since it is the center of a cluster that represents a prototype agent-environment interaction behaviour. 

  

% \begingroup
% \removelatexerror% Nullify \@latex@error
% \begin{algorithm}[]
% % \setlength{\belowcaptionskip}{-10pt}
% 	\SetAlgoLined
% 	\small
% 	\KwIn{Offline dataset $\mathcal{D}=\{D_{Y_{i}}\}_{i=1}^{M}$, $\psi^{\pi_{0}}(x, a;\theta_{sf})$, designed embodied set size $K$}
% 	\KwResult{Embodied set structure $\Omega^{e}$}
% 	%\textcolor[rgb]{0.14,0.36,0.73}{\textbf{Initialization}}\\
% 	Initialize an empty embodied set $\Omega^{e}=\{\}$\\
% 	Initialize an empty successor feature vector list $\mathbf{L}_{sf}=\{\}$\\
% 	$\mathcal{D} \leftarrow $ Shuffle $(\mathcal{D})$\\
% 	\For{each $(x, a, x') \in \mathcal{D}$}{
% 	    \tcp{Compute cross-domain transferable successor features}
% 	    $\mathbf{L}_{sf} \leftarrow $ Append $\psi^{\pi_{0}}(x, a;\theta_{sf})$  \\
% 	}
% 	\tcp{Constructing embodied agent state set}
% 	k-means $\leftarrow$ Mini-batch K-means clustering ($\mathbf{L}_{sf}, K$)\\
       
% 	$\Omega^{e}=\{\mathbf{e}_{i}\}_{i=1}^{k}$ =  k-means.cluster-centers as prototypes  \\
% 	\caption{Embodied Set Construction}
% \end{algorithm}
% \endgroup

\subsection{Re-use: a backbone for general downstream task learning}

Let a downstream task denoted as $\mathcal{M}_{u}=(\mathcal{X}^{u}, \mathcal{A}, \mathcal{T}^{u}, r^{u}, \gamma)$, where the observation space $\mathcal{X}^{u}$, dynamics $\mathcal{T}^{u}$, and task objectives $r^{u}$ can be unseen in pre-training. The action space $\mathcal{A}$ remains the same since we assume the same agent learning a new task based on its past experience. 

Reusing the pre-trained model under the above heterogeneous settings is difficult due to out-of-distribution concerns. Traditional methods will not work. For example, in methods reusing the pre-trained representation \citep{shah2021rrl,kingma2013auto}, changes in the environment will cause non-stationarity in the observation space that directly plugs in the learned representation will make the learning even worse than learning from scratch. For another example, in methods reusing the pre-trained successor features to fast compute the value functions \citep{barreto2020fast}, changes in task objective or the environment will break the linear reward feature assumption.

We propose to avoid directly plugging in the pre-trained model, but to use the pre-trained model---embodied set $\Omega^{e}$ as a base structure for projection-based techniques to accelerate the downstream task learning by tackling learning stability and plasticity explicitly.

\paragraph{(1) Stability: Retain previous knowledge by embodied feature projection} 
We define a feature projection operator $\Pi_{\Omega^{e}}(x, a)$, which takes the input of $(x, a)$, localizes its projection on the constructed embodied set $\Omega^{e}$, and returns the localized feature vector $\mathbf{e}$:
\begin{equation}
    \Pi_{\Omega^{e}}(x, a)=\mathbf{e}, \hspace{0.2cm} \forall \mathbf{e}\in \Omega^{e}, \text{   find the smallest } \xi(x, a, \mathbf{e})
\end{equation}
, where $\xi(x, a, \mathbf{e}) = \norm{\psi^{\pi_{0}}(x,a;\theta_{sf}) - \mathbf{e}}$, or it can be any other distance metric functions. The feature projection operator $\Pi_{\Omega^{e}}(x, a)$ retains previous knowledge by always matching the new task experience with the closest prototype $\mathbf{e}$ in the embodied set.

Note that $\Pi_{\Omega^{e}}(x, a)$ also works for changed sensor modalities. For example, assuming a task requires the agent to understand textual commands, then adding an extra-textual observation $z$ will augment the observation to $[x, z]$. The feature projection operator $\Pi_{\Omega^{e}}(x, a)$ still applies here by using the unchanged sensory modality part $x$. We will show that in Sec. 5.1 (3).

\paragraph{(2) Plasticity: Adapt to changes by projected Bellman Updates}
We use the below projected Bellman updates to accelerate learning the unknown downstream task $\mathcal{M}_{u}=(\mathcal{X}^{u}, \mathcal{A}, \mathcal{T}^{u}, r^{u}, \gamma)$:
\begin{equation}
    Q^{ \pi}(x,a) = \mathbb{E}_{(x, a, x') \sim \mathcal{T}^{u}}[r^{u} + \gamma V^{ \pi}_{proj}(\Pi_{\Omega^{e}}(x',\underset{a'}{\argmax}Q^{ \pi}(x',a'))]
    \label{eq:bellman1}
\end{equation}
\begin{equation}
   V^{ \pi}_{proj}(\Pi_{\Omega^{e}}(x, a))=\mathbb{E}_{(x, a, x') \sim \mathcal{T}^{u}}[r^{u} + \gamma \underset{a'}{max} Q^{ \pi}(x', a')
   \label{eq:bellman2}
\end{equation}
We maintain two value functions---the task $Q^{ \pi}$ and a projected version $V^{ \pi}_{proj}$, to support each other's learning in a bidirectional improvement manner. The motivation is that the projected function $V^{ \pi}_{proj}$ can learn faster than $Q^{ \pi}$  since it is defined on the pre-trained set $\Omega^{e}$ that retains past experience, but is not accurate since $\Omega^{e}$ does not adapt to the new task setting. Meanwhile, the task Q-value function $Q^{ \pi}$ should be more accurate but will take a longer time if learned from scratch. Learning that alternates Bellman updates Eq. \ref{eq:bellman1} and \ref{eq:bellman2} will play a trade-off between retaining the previous knowledge or adapting to the new tasks.

\paragraph{(3) Reuse example: DQN-embodied}
While the above two techniques \textit{embodied feature projection} and \textit{projected Bellman updates} can generally apply to any RL methods in learning a new task, we use DQN as an example to show how to use them to accelerate downstream task learning. Assume we use a neural network parameterized with $\theta_{u}$ to approximate $Q^{ \pi}$, and another neural network parameterized with $\mathbf{w}_{u}$ to approximate $V^{ \pi}_{proj}$. According to Eq. \ref{eq:bellman1} and \ref{eq:bellman2}, we compute the target value for $Q^{ \pi}$ and $V^{ \pi}_{proj}$ at training iteration i as:
\begin{equation}
\begin{split}
   y_{i} = \mathbb{E}_{(x, a, x') \sim \mathcal{T}^{u}}[r^{u}+\gamma V^{ \pi}_{proj}(\Pi_{\Omega^{e}}(x',\underset{a'}{\argmax}Q(x',a';\theta_{u,i-1}));\mathbf{w}_{u, i-1})]
 \end{split}
 \label{eq:dqn1}
\end{equation}
\begin{equation}
     y_{proj, i} = \mathbb{E}_{(x, a, x') \sim \mathcal{T}^{u}}[r^{u} + \gamma \underset{a'}{max} Q^{ \pi}(x', a';\theta_{u, i-1})]
     \label{eq:dqn2}
\end{equation}
Then, the learning objectives formulated as an LMSE loss can be written as:
\begin{equation}
\begin{split}
  \mathcal{L}_{i}(\theta_{u,i}) = \mathbb{E}_{(x, a, x') \sim \mathcal{T}^{u}}[(y_{i} - Q^{\pi}(x, a;\theta_{u, i}))^{2}] 
 \end{split}
 \label{eq:dqn3}
\end{equation}
\begin{equation}
    \mathcal{L}_{i}(\mathbf{w}_{u,i}) = \mathbb{E}_{(x, a, x') \sim \mathcal{T}^{u}}[( y_{proj, i} - V^{ \pi}_{proj}(\Pi_{\Omega^{e}}(x, a));\mathbf{w}_{u,i})^{2}]
    \label{eq:dqn4}
\end{equation}
During training, we alternate learning $Q^{ \pi}$ and $V^{ \pi}_{proj}$. A full description of the proposed DQN-embodied algorithm is summarized in Appendix \ref{sec:dqn_embodied}.

% \begin{wrapfigure}[20]{R}{0.6\textwidth} %<-- Wrapfigure covers 6 lines
%    \begingroup
% \removelatexerror% Nullify \@latex@error
% \begin{algorithm}[H]
%  \setlength{\belowcaptionskip}{-10pt}
% 	\SetAlgoLined
% 	\small
% 	\KwIn{Pre-trained embodied set $\Omega^{e}$, feature projection operator $\Pi_{\Omega^{e}}(x,a)$}
% 	%\textcolor[rgb]{0.14,0.36,0.73}{\textbf{Initialization}}\\
% 	Initialize $Q^{ \pi}(.;\theta_{u})$, $V^{ \pi}_{proj}(.;\mathbf{w}_{u})$,  and replay buffer $\mathcal{D}$\\
	
% 	\For{i=1:N}{
% 	    \tcp{Replay buffer}
% 	    \For{t=0:T}{
% 	        $\epsilon$ greedy select action $a_{t}$ based on ${max}_{a} Q^{ \pi}(x_{t},a_{t};\theta_{u,i})$\\
% 	        Execute action $a_{t}$ in environment, observe $x_{t+1}, r_{t}$\\
% 	        Append transition sample $(x_{t}, a_{t}, r_{t}, x_{t+1})$ in $\mathcal{D}$\\
% 	        Randomly sample batch transitions $\mathcal{B}=\{(x, a, r, x')\}$ from $\mathcal{D}$\\
% 	        \tcp{Learn $Q^{ \pi}$}
% 	        Set $y_{i} = \mathbb{E}_{(x, a, x') \sim \mathcal{T}^{u}}[r^{u}+\gamma V^{ \pi}_{proj}(\Pi_{\Omega^{e}}(x',\underset{a'}{max}Q(x',a';\theta_{u,i-1}));\mathbf{w}_{u, i-1})]$\\
% 	        Perform gradient descent step on $\mathcal{L}_{i}(\theta_{u,i})=(y_{i} - Q^{ \pi}(x, a;\theta_{u, i}))^{2}$\\
% 	        \tcp{Learn $V^{ \pi}_{proj}$ }
% 	        Set $ y_{proj, i} = \mathbb{E}_{(x, a, x') \sim \mathcal{T}^{u}}[r^{u} + \gamma \underset{a'}{max} Q^{ \pi}(x', a';\theta_{u, i-1})]$\\
% 	        Perform gradient descent step on $\mathcal{L}_{i}(\mathbf{w}_{u,i})=( y_{proj, i} - V^{ \pi}_{proj}(\Pi_{\Omega^{e}}(x, a));\mathbf{w}_{u,i})^{2}$
% 	    }
% 	}
% 	\caption{DQN-embodied}
% \end{algorithm}
% \endgroup
%   \end{wrapfigure}



\section{Experiments}\label{sec:experiments}
\section{Experiments}
\label{sec:expriments}
To verify the effectiveness of our method, we conduct extensive experiments on ImageNet-1K (IN-1K)~\cite{cvpr2009imagenet} for image classification, COCO~\cite{2014MicrosoftCOCO} for object detection and instance segmentation, and ADE20K~\cite{Zhou2018ADE20k} for semantic segmentation. All experiments are implemented with PyTorch on Ubuntu workstations with NVIDIA A100 GPUs. \textbf{Bold} and \hl{gray} indicate the best performance and our models.

% % table: IN-1K Tiny (5M) & Small (25M)
% \begin{figure*}[t!]
% \vspace{-1.0em}
% \begin{minipage}{0.5\linewidth}
% \centering
%     \begin{table}[H]
    % \vspace{-0.25em}
    \setlength{\tabcolsep}{0.3mm}
    \centering
\resizebox{\linewidth}{!}{
\begin{tabular}{llccccc}
    \toprule
    Architecture                            & Date         & Type & Image   & Param. & FLOPs & Top-1     \\
                                            &              &      & Size    & (M)    & (G)   & Acc (\%)  \\ \hline
    ResNet-18                               & CVPR'2016    & C    & $224^2$ & 11.7   & 1.80  & 71.5      \\
    ShuffleNetV2~$2\times$                  & ECCV'2018    & C    & $224^2$ & 5.5    & 0.60  & 75.4      \\
    EfficientNet-B0                         & ICML'2019    & C    & $224^2$ & 5.3    & 0.39  & 77.1      \\
    % MobileNetV3~$1\times$                   & ICCV'2019    & C    & $224^2$ & 5.4    & 0.23  & 75.2      \\
    % RegNetY-400MF                          & CVPR'2020    & C    & $224^2$ & 5.3    & 0.40  & 74.1      \\
    RegNetY-800MF                           & CVPR'2020    & C    & $224^2$ & 6.3    & 0.80  & 76.3      \\
    DeiT-T$^\dag$                           & ICML'2021    & T    & $224^2$ & 5.7    & 1.08  & 74.1      \\
    % DeiT-2G                                & ICML'2021    & T    & $224^2$ & 13.2   & 1.90  & 75.1      \\
    PVT-T                                   & ICCV'2021    & T    & $224^2$ & 13.2   & 1.60  & 75.1      \\
    T2T-ViT-7                               & ICCV'2021    & T    & $224^2$ & 4.3    & 1.20  & 71.7      \\
    % T2T-ViT-12                             & ICCV'2021    & T    & $224^2$ & 6.9    & 1.80  & 76.5      \\
    ViT-C                                   & NIPS'2021    & T    & $224^2$ & 4.6    & 1.10  & 75.3      \\
    SReT-T$_{Distill}$                      & ECCV'2022    & T    & $224^2$ & 4.8    & 1.10  & 77.6      \\
    PiT-Ti                                  & ICCV'2021    & H    & $224^2$ & 4.9    & 0.70  & 74.6      \\
    LeViT-S                                 & ICCV'2021    & H    & $224^2$ & 7.8    & 0.31  & 76.6      \\
    CoaT-Lite-T                             & ICCV'2021    & H    & $224^2$ & 5.7    & 1.60  & 77.5      \\
    Swin-1G                                 & ICCV'2021    & H    & $224^2$ & 7.3    & 1.00  & 77.3      \\
    % Swin-2G                                 & ICCV'2021    & H    & $224^2$ & 12.8   & 2.00  & 79.3      \\
    % MobileViT-XS                            & ICLR'2022    & H    & $256^2$ & 2.3    & 1.73  & 74.8      \\
    MobileViT-S                             & ICLR'2022    & H    & $256^2$ & 5.6    & 4.02  & 78.4      \\
    % MobileFormer-151M                      & CVPR'2022    & H    & $224^2$ & 7.6    & 0.29  & 75.2      \\
    MobileFormer-294M                       & CVPR'2022    & H    & $224^2$ & 11.4   & 0.59  & 77.9      \\
    ConvNext-XT                             & CVPR'2022    & C    & $224^2$ & 7.4    & 0.60  & 77.5      \\
    VAN-B0                                  & CVMJ'2023   & C    & $224^2$ & 4.1    & 0.88  & 75.4      \\
    ParC-Net-S                              & ECCV'2022    & C    & $256^2$ & 5.0    & 3.48  & 78.6      \\
    \rowcolor{gray94}\bf{MogaNet-XT}        & Ours         & C    & $256^2$ & 3.0    & 1.04  & 77.2      \\
    \rowcolor{gray94}\bf{MogaNet-T}         & Ours         & C    & $224^2$ & 5.2    & 1.10  & 79.0      \\
    \rowcolor{gray94}\bf{MogaNet-T}$^\S$    & Ours         & C    & $256^2$ & 5.2    & 1.44  & \bf{80.0} \\
    \bottomrule
    \end{tabular}
    }
    \vspace{-1.0em}
    \caption{\textbf{IN-1K classification} with lightweight models. \small{$\S$} denotes the refined training scheme.
    % \small{$\dag$} and \small{$\S$} are RSB A2 and refined training schemes.
    }
    \label{tab:in1k_cls_tiny}
    % \vspace{-0.5em}
\end{table}


% \begin{table}[h]
%     \vspace{-0.5em}
%     \setlength{\tabcolsep}{0.7mm}
%     \centering
% \resizebox{\linewidth}{!}{
% \begin{tabular}{llcccc}
%     \toprule
%     Architecture                                     & Date         & Image   & Param. & FLOPs & Top-1     \\
%                                                      &              & Size    & (M)    & (G)   & Acc (\%)  \\ \hline
%     ResNet-18$^\dag$~\cite{he2016deep}               & CVPR'2016    & $224^2$ & 11.7   & 1.80  & 71.5      \\
%     ShuffleNetV2~$2\times$~\cite{eccv2018shufflenet} & ECCV'2018    & $224^2$ & 5.5    & 0.60  & 75.4      \\
%     EfficientNet-B0~\cite{icml2019efficientnet}      & ICML'2019    & $224^2$ & 5.3    & 0.39  & 77.1      \\
%     MobileNetV3~$1\times$~\cite{iccv2019mobilenetv3} & ICCV'2019    & $224^2$ & 5.4    & 0.23  & 75.2      \\
%     RegNetY-800M~\cite{cvpr2020regnet}               & CVPR'2020    & $224^2$ & 6.3    & 0.80  & 76.3      \\ \hline
%     DeiT-T~\cite{icml2021deit}                     & ICML'201     & $224^2$ & 5.7    & 1.08  & 72.2      \\
%     PVT-T~\cite{iccv2021PVT}                       & ICCV'2021    & $224^2$ & 13.2   & 1.60  & 75.1      \\
%     T2T-ViT-7~\cite{iccv2021t2t}                   & ICCV'2021    & $224^2$ & 4.3    & 1.20  & 71.7      \\
%     T2T-ViT-12~\cite{iccv2021t2t}                  & ICCV'2021    & $224^2$ & 6.9    & 1.80  & 76.5      \\
%     ViT-C~\cite{nips2021vitc}                      & NIPS'2021    & $224^2$ & 4.6    & 1.10  & 75.3      \\ \hline
%     PiT-Ti~\cite{iccv2021pit}                      & ICCV'2021    & $224^2$ & 4.9    & 0.70  & 74.6      \\
%     LeViT-S~\cite{iccv2021levit}                   & ICCV'2021    & $224^2$ & 7.8    & 0.31  & 76.6      \\
%     CoaT-Lite-T~\cite{iccv2021coat}                & ICCV'2021    & $224^2$ & 5.7    & 1.60  & 77.5      \\
%     MobileViT-XS~\cite{iclr2022mobilevit}          & ICLR'2022    & $256^2$ & 2.3    & 1.73  & 74.8      \\
%     MobileViT-S~\cite{iclr2022mobilevit}           & ICLR'2022    & $256^2$ & 5.6    & 4.02  & 78.4      \\
%     Mobile-Former-151M~\cite{cvpr2022MobileFormer} & CVPR'2022    & $224^2$ & 7.6    & 0.29  & 75.2      \\
%     Mobile-Former-294M~\cite{cvpr2022MobileFormer} & CVPR'2022    & $224^2$ & 11.4   & 0.59  & 77.9      \\ \hline
%     ConvNext-XT~\cite{cvpr2022convnext}            & CVPR'2022    & $224^2$ & 7.4    & 0.60  & 77.5      \\
%     VAN-B0~\cite{guo2022van}                       & arXiv'2022   & $224^2$ & 4.1    & 0.88  & 75.4      \\
%     ParC-Net-S~\cite{eccv2022edgeformer}           & ECCV'2022    & $256^2$ & 5.0    & 3.48  & 78.6      \\
%     \rowcolor{gray94}\bf{MogaNet-XT}               & Ours         & $224^2$ & 3.0    & 0.80  & 76.3      \\
%     \rowcolor{gray94}\bf{MogaNet-T}                & Ours         & $224^2$ & 5.2    & 1.10  & 79.0      \\
%     \rowcolor{gray94}\bf{MogaNet-T}                & Ours         & $256^2$ & 5.2    & 1.44  & \bf{79.6} \\
%     \bottomrule
%     \end{tabular}
%     }
%     \vspace{-0.5em}
%     \caption{ImageNet-1K classification performance of lightweight (around 5M Parameters) models.}
%     \label{tab:in1k_cls_tiny}
%     \vspace{-0.75em}
% \end{table}

% \end{minipage}
% \begin{minipage}{0.5\linewidth}
% \centering
%     \begin{table}[h]
    \vspace{-0.25em}
    \setlength{\tabcolsep}{0.8mm}
    \centering
\resizebox{\linewidth}{!}{
\begin{tabular}{llcccc}
    \toprule
    Architecture                                 & Date         & Image   & Param. & FLOPs & Top-1     \\
                                                 &              & Size    & (M)    & (G)   & Acc (\%)  \\ \hline
    ResNet-50$^\dag$~\cite{he2016deep}           & CVPR'2016    & $224^2$ & 26     & 4.1   & 80.4      \\
    EfficientNet-B4~\cite{icml2019efficientnet}  & ICML'2019    & $380^2$ & 19     & 4.2   & 82.9      \\
    RegNetY-4GF$^\dag$~\cite{cvpr2020regnet}     & CVPR'2020    & $224^2$ & 21     & 4.0   & 81.5      \\ \hline
    Deit-S~\cite{icml2021deit}                   & ICML'2021    & $224^2$ & 22     & 4.6   & 79.8      \\
    Swin-T~\cite{liu2021swin}                    & ICCV'2021    & $224^2$ & 28     & 4.5   & 81.3      \\
    T2T-ViT$_t$-14~\cite{iccv2021t2t}            & ICCV'2021    & $224^2$ & 22     & 6.1   & 81.7      \\
    CSWin-T~\cite{cvpr2022CSWin}                 & CVPR'2022    & $224^2$ & 23     & 4.3   & 82.8      \\
    SReT-S~\cite{eccv2022SReT}                   & ECCV'2022    & $224^2$ & 21     & 4.2   & 81.9      \\
    LITV2-S~\cite{nips2022hilo}                  & NIPS'2022    & $224^2$ & 28     & 3.7   & 82.0      \\ \hline
    CoaT-S~\cite{iccv2021coat}                   & ICCV'2021    & $224^2$ & 22     & 12.6  & 82.1      \\
    CoAtNet-0~\cite{nips2021coatnet}             & NIPS'2021    & $224^2$ & 25     & 4.2   & 82.7      \\
    ViTAE-S~\cite{nips2021vitae}                 & NIPS'2021    & $224^2$ & 24     & 5.6   & 82.0      \\
    UniFormer-S~\cite{iclr2022uniformer}         & ICLR'2022    & $224^2$ & 22     & 3.6   & 82.9      \\
    EfficientFormer-L3~\cite{nips2022EfficientFormer} & NIPS'2022    & $224^2$ & 31     & 3.9   & 82.4      \\ \hline
    ConvNeXt-T~\cite{cvpr2022convnext}           & CVPR'2022    & $224^2$ & 29     & 4.5   & 82.1      \\
    VAN-B2~\cite{guo2022van}                     & arXiv'2022   & $224^2$ & 27     & 5.0   & 82.8      \\
    SLaK-T~\cite{Liu2022SLak}                    & arXiv'2022   & $224^2$ & 30     & 5.0   & 82.5      \\
    HorNet-T$_{7\times 7}$~\cite{nips2022hornet} & NIPS'2022    & $224^2$ & 22     & 4.0   & 82.8      \\
    \rowcolor{gray94}\bf{MogaNet-S}              & Ours         & $224^2$ & 25     & 5.0   & \bf{83.4} \\
    \bottomrule
    \end{tabular}
    }
    \vspace{-0.5em}
    \caption{\textbf{ImageNet-1K classification} performance of small size (around 25M parameters) models.}
    \label{tab:in1k_cls_small}
    \vspace{-1.25em}
\end{table}

% \end{minipage}
% \vspace{-1.5em}
% \end{figure*}

\subsection{ImageNet Classification}
\label{sec:exp_in1k}
\paragraph{Settings.} 
For classification experiments on ImageNet-1K, we train MogaNet variants following the standard procedure \cite{icml2021deit, liu2021swin} for a fair comparison. Specifically, the models are trained for 300 epochs by AdamW~\cite{iclr2019AdamW} optimizer with $224^2$ or $256^2$ resolutions, a basic learning rate $lr$ = $1\times 10^{-3}$, 5 epochs warmup, and a Cosine scheduler~\cite{loshchilov2016sgdr}. See Appendix~\ref{app:in1k_settings} for implementation details.
We compare four typical architectures: (\romannumeral1) \textbf{Classical ConvNets} include ResNet, SENet, ShuffleNetV2, EfficientNet, MobileNetV3, and RegNet. (\romannumeral2) \textbf{Transformers} include DeiT, Swin, T2T-ViT, PVT, Focal, ViT-C, CSWin, SReT, and LiTV2. (\romannumeral3) \textbf{Hybrid architectures} of attention and convolution include PiT, LeViT, CoaT, BoTNet, ViTAE, Twins, CoAtNet, MobileViT, Uniformer, Mobile-Former, ParC-Net, EfficientFormer, and MaxViT. (\romannumeral4) \textbf{Modern ConvNets} include ConvNeXt, RepLKNet, FocalNet, VAN, SLak, and HorNet.
% We compare four types of popular network architectures: (\romannumeral1) \textbf{Classical CNN} includes ResNet~\cite{he2016deep}, SENet~\cite{hu2018squeeze}, ShuffleNetV2~\cite{eccv2018shufflenet}, EfficientNet~\cite{icml2019efficientnet}, MobileNetV3~\cite{iccv2019mobilenetv3}, and RegNet~\cite{cvpr2020regnet}. (\romannumeral2) \textbf{Transformer} includes DeiT~\cite{icml2021deit}, Swin~\cite{liu2021swin}, T2T-ViT~\cite{iccv2021t2t}, PVT~\cite{iccv2021PVT}, FocalNet~\cite{nips2021Focal}, ViT-C~\cite{nips2021vitc}, CSWin~\cite{cvpr2022CSWin}, SReT~\cite{eccv2022SReT}, and LiTV2~\cite{nips2022hilo}. (\romannumeral3) \textbf{Hybrid} Transformer and CNN architecture includes PiT~\cite{iccv2021pit}, LeViT~\cite{iccv2021levit}, CoaT~\cite{iccv2021coat}, BoTNet~\cite{cvpr2021botnet}, ViTAE~\cite{nips2021vitae}, Twins~\cite{nips2021Twins}, CoAtNet~\cite{nips2021coatnet}, MobileViT~\cite{iclr2022mobilevit}, Uniformer~\cite{iclr2022uniformer}, Mobile-Former~\cite{cvpr2022MobileFormer}, and ParC-Net~\cite{eccv2022edgeformer}. (\romannumeral3) \textbf{Post-ConvNet} includes ConvNeXt~\cite{cvpr2022convnext}, RepLKNet~\cite{cvpr2022replknet}, VAN~\cite{guo2022van}, SLak~\cite{Liu2022SLak}, and HorNet~\cite{nips2022hornet}.

% table: IN-1K Tiny (5M) & Small (25M)
\begin{table}[H]
    % \vspace{-0.25em}
    \setlength{\tabcolsep}{0.3mm}
    \centering
\resizebox{\linewidth}{!}{
\begin{tabular}{llccccc}
    \toprule
    Architecture                            & Date         & Type & Image   & Param. & FLOPs & Top-1     \\
                                            &              &      & Size    & (M)    & (G)   & Acc (\%)  \\ \hline
    ResNet-18                               & CVPR'2016    & C    & $224^2$ & 11.7   & 1.80  & 71.5      \\
    ShuffleNetV2~$2\times$                  & ECCV'2018    & C    & $224^2$ & 5.5    & 0.60  & 75.4      \\
    EfficientNet-B0                         & ICML'2019    & C    & $224^2$ & 5.3    & 0.39  & 77.1      \\
    % MobileNetV3~$1\times$                   & ICCV'2019    & C    & $224^2$ & 5.4    & 0.23  & 75.2      \\
    % RegNetY-400MF                          & CVPR'2020    & C    & $224^2$ & 5.3    & 0.40  & 74.1      \\
    RegNetY-800MF                           & CVPR'2020    & C    & $224^2$ & 6.3    & 0.80  & 76.3      \\
    DeiT-T$^\dag$                           & ICML'2021    & T    & $224^2$ & 5.7    & 1.08  & 74.1      \\
    % DeiT-2G                                & ICML'2021    & T    & $224^2$ & 13.2   & 1.90  & 75.1      \\
    PVT-T                                   & ICCV'2021    & T    & $224^2$ & 13.2   & 1.60  & 75.1      \\
    T2T-ViT-7                               & ICCV'2021    & T    & $224^2$ & 4.3    & 1.20  & 71.7      \\
    % T2T-ViT-12                             & ICCV'2021    & T    & $224^2$ & 6.9    & 1.80  & 76.5      \\
    ViT-C                                   & NIPS'2021    & T    & $224^2$ & 4.6    & 1.10  & 75.3      \\
    SReT-T$_{Distill}$                      & ECCV'2022    & T    & $224^2$ & 4.8    & 1.10  & 77.6      \\
    PiT-Ti                                  & ICCV'2021    & H    & $224^2$ & 4.9    & 0.70  & 74.6      \\
    LeViT-S                                 & ICCV'2021    & H    & $224^2$ & 7.8    & 0.31  & 76.6      \\
    CoaT-Lite-T                             & ICCV'2021    & H    & $224^2$ & 5.7    & 1.60  & 77.5      \\
    Swin-1G                                 & ICCV'2021    & H    & $224^2$ & 7.3    & 1.00  & 77.3      \\
    % Swin-2G                                 & ICCV'2021    & H    & $224^2$ & 12.8   & 2.00  & 79.3      \\
    % MobileViT-XS                            & ICLR'2022    & H    & $256^2$ & 2.3    & 1.73  & 74.8      \\
    MobileViT-S                             & ICLR'2022    & H    & $256^2$ & 5.6    & 4.02  & 78.4      \\
    % MobileFormer-151M                      & CVPR'2022    & H    & $224^2$ & 7.6    & 0.29  & 75.2      \\
    MobileFormer-294M                       & CVPR'2022    & H    & $224^2$ & 11.4   & 0.59  & 77.9      \\
    ConvNext-XT                             & CVPR'2022    & C    & $224^2$ & 7.4    & 0.60  & 77.5      \\
    VAN-B0                                  & CVMJ'2023   & C    & $224^2$ & 4.1    & 0.88  & 75.4      \\
    ParC-Net-S                              & ECCV'2022    & C    & $256^2$ & 5.0    & 3.48  & 78.6      \\
    \rowcolor{gray94}\bf{MogaNet-XT}        & Ours         & C    & $256^2$ & 3.0    & 1.04  & 77.2      \\
    \rowcolor{gray94}\bf{MogaNet-T}         & Ours         & C    & $224^2$ & 5.2    & 1.10  & 79.0      \\
    \rowcolor{gray94}\bf{MogaNet-T}$^\S$    & Ours         & C    & $256^2$ & 5.2    & 1.44  & \bf{80.0} \\
    \bottomrule
    \end{tabular}
    }
    \vspace{-1.0em}
    \caption{\textbf{IN-1K classification} with lightweight models. \small{$\S$} denotes the refined training scheme.
    % \small{$\dag$} and \small{$\S$} are RSB A2 and refined training schemes.
    }
    \label{tab:in1k_cls_tiny}
    % \vspace{-0.5em}
\end{table}


% \begin{table}[h]
%     \vspace{-0.5em}
%     \setlength{\tabcolsep}{0.7mm}
%     \centering
% \resizebox{\linewidth}{!}{
% \begin{tabular}{llcccc}
%     \toprule
%     Architecture                                     & Date         & Image   & Param. & FLOPs & Top-1     \\
%                                                      &              & Size    & (M)    & (G)   & Acc (\%)  \\ \hline
%     ResNet-18$^\dag$~\cite{he2016deep}               & CVPR'2016    & $224^2$ & 11.7   & 1.80  & 71.5      \\
%     ShuffleNetV2~$2\times$~\cite{eccv2018shufflenet} & ECCV'2018    & $224^2$ & 5.5    & 0.60  & 75.4      \\
%     EfficientNet-B0~\cite{icml2019efficientnet}      & ICML'2019    & $224^2$ & 5.3    & 0.39  & 77.1      \\
%     MobileNetV3~$1\times$~\cite{iccv2019mobilenetv3} & ICCV'2019    & $224^2$ & 5.4    & 0.23  & 75.2      \\
%     RegNetY-800M~\cite{cvpr2020regnet}               & CVPR'2020    & $224^2$ & 6.3    & 0.80  & 76.3      \\ \hline
%     DeiT-T~\cite{icml2021deit}                     & ICML'201     & $224^2$ & 5.7    & 1.08  & 72.2      \\
%     PVT-T~\cite{iccv2021PVT}                       & ICCV'2021    & $224^2$ & 13.2   & 1.60  & 75.1      \\
%     T2T-ViT-7~\cite{iccv2021t2t}                   & ICCV'2021    & $224^2$ & 4.3    & 1.20  & 71.7      \\
%     T2T-ViT-12~\cite{iccv2021t2t}                  & ICCV'2021    & $224^2$ & 6.9    & 1.80  & 76.5      \\
%     ViT-C~\cite{nips2021vitc}                      & NIPS'2021    & $224^2$ & 4.6    & 1.10  & 75.3      \\ \hline
%     PiT-Ti~\cite{iccv2021pit}                      & ICCV'2021    & $224^2$ & 4.9    & 0.70  & 74.6      \\
%     LeViT-S~\cite{iccv2021levit}                   & ICCV'2021    & $224^2$ & 7.8    & 0.31  & 76.6      \\
%     CoaT-Lite-T~\cite{iccv2021coat}                & ICCV'2021    & $224^2$ & 5.7    & 1.60  & 77.5      \\
%     MobileViT-XS~\cite{iclr2022mobilevit}          & ICLR'2022    & $256^2$ & 2.3    & 1.73  & 74.8      \\
%     MobileViT-S~\cite{iclr2022mobilevit}           & ICLR'2022    & $256^2$ & 5.6    & 4.02  & 78.4      \\
%     Mobile-Former-151M~\cite{cvpr2022MobileFormer} & CVPR'2022    & $224^2$ & 7.6    & 0.29  & 75.2      \\
%     Mobile-Former-294M~\cite{cvpr2022MobileFormer} & CVPR'2022    & $224^2$ & 11.4   & 0.59  & 77.9      \\ \hline
%     ConvNext-XT~\cite{cvpr2022convnext}            & CVPR'2022    & $224^2$ & 7.4    & 0.60  & 77.5      \\
%     VAN-B0~\cite{guo2022van}                       & arXiv'2022   & $224^2$ & 4.1    & 0.88  & 75.4      \\
%     ParC-Net-S~\cite{eccv2022edgeformer}           & ECCV'2022    & $256^2$ & 5.0    & 3.48  & 78.6      \\
%     \rowcolor{gray94}\bf{MogaNet-XT}               & Ours         & $224^2$ & 3.0    & 0.80  & 76.3      \\
%     \rowcolor{gray94}\bf{MogaNet-T}                & Ours         & $224^2$ & 5.2    & 1.10  & 79.0      \\
%     \rowcolor{gray94}\bf{MogaNet-T}                & Ours         & $256^2$ & 5.2    & 1.44  & \bf{79.6} \\
%     \bottomrule
%     \end{tabular}
%     }
%     \vspace{-0.5em}
%     \caption{ImageNet-1K classification performance of lightweight (around 5M Parameters) models.}
%     \label{tab:in1k_cls_tiny}
%     \vspace{-0.75em}
% \end{table}

\begin{table}[h]
    \vspace{-0.25em}
    \setlength{\tabcolsep}{0.8mm}
    \centering
\resizebox{\linewidth}{!}{
\begin{tabular}{llcccc}
    \toprule
    Architecture                                 & Date         & Image   & Param. & FLOPs & Top-1     \\
                                                 &              & Size    & (M)    & (G)   & Acc (\%)  \\ \hline
    ResNet-50$^\dag$~\cite{he2016deep}           & CVPR'2016    & $224^2$ & 26     & 4.1   & 80.4      \\
    EfficientNet-B4~\cite{icml2019efficientnet}  & ICML'2019    & $380^2$ & 19     & 4.2   & 82.9      \\
    RegNetY-4GF$^\dag$~\cite{cvpr2020regnet}     & CVPR'2020    & $224^2$ & 21     & 4.0   & 81.5      \\ \hline
    Deit-S~\cite{icml2021deit}                   & ICML'2021    & $224^2$ & 22     & 4.6   & 79.8      \\
    Swin-T~\cite{liu2021swin}                    & ICCV'2021    & $224^2$ & 28     & 4.5   & 81.3      \\
    T2T-ViT$_t$-14~\cite{iccv2021t2t}            & ICCV'2021    & $224^2$ & 22     & 6.1   & 81.7      \\
    CSWin-T~\cite{cvpr2022CSWin}                 & CVPR'2022    & $224^2$ & 23     & 4.3   & 82.8      \\
    SReT-S~\cite{eccv2022SReT}                   & ECCV'2022    & $224^2$ & 21     & 4.2   & 81.9      \\
    LITV2-S~\cite{nips2022hilo}                  & NIPS'2022    & $224^2$ & 28     & 3.7   & 82.0      \\ \hline
    CoaT-S~\cite{iccv2021coat}                   & ICCV'2021    & $224^2$ & 22     & 12.6  & 82.1      \\
    CoAtNet-0~\cite{nips2021coatnet}             & NIPS'2021    & $224^2$ & 25     & 4.2   & 82.7      \\
    ViTAE-S~\cite{nips2021vitae}                 & NIPS'2021    & $224^2$ & 24     & 5.6   & 82.0      \\
    UniFormer-S~\cite{iclr2022uniformer}         & ICLR'2022    & $224^2$ & 22     & 3.6   & 82.9      \\
    EfficientFormer-L3~\cite{nips2022EfficientFormer} & NIPS'2022    & $224^2$ & 31     & 3.9   & 82.4      \\ \hline
    ConvNeXt-T~\cite{cvpr2022convnext}           & CVPR'2022    & $224^2$ & 29     & 4.5   & 82.1      \\
    VAN-B2~\cite{guo2022van}                     & arXiv'2022   & $224^2$ & 27     & 5.0   & 82.8      \\
    SLaK-T~\cite{Liu2022SLak}                    & arXiv'2022   & $224^2$ & 30     & 5.0   & 82.5      \\
    HorNet-T$_{7\times 7}$~\cite{nips2022hornet} & NIPS'2022    & $224^2$ & 22     & 4.0   & 82.8      \\
    \rowcolor{gray94}\bf{MogaNet-S}              & Ours         & $224^2$ & 25     & 5.0   & \bf{83.4} \\
    \bottomrule
    \end{tabular}
    }
    \vspace{-0.5em}
    \caption{\textbf{ImageNet-1K classification} performance of small size (around 25M parameters) models.}
    \label{tab:in1k_cls_small}
    \vspace{-1.25em}
\end{table}


% table: IN-1K Base (40M) & Large (80M)
\begin{figure*}[t!]
\vspace{-1.5em}
\begin{minipage}{0.495\linewidth}
\centering
    \begin{table}[H]
    \vspace{-0.25em}
    \setlength{\tabcolsep}{0.9mm}
    \centering
\resizebox{\linewidth}{!}{
\begin{tabular}{llcccc}
    \toprule
    Architecture                                 & Date         & Image   & Param. & FLOPs & Top-1     \\
                                                 &              & Size    & (M)    & (G)   & Acc (\%)  \\ \hline
    ResNet-101$^\dag$~\cite{he2016deep}          & CVPR'2016    & $224^2$ & 45     & 7.9   & 81.5      \\
    EfficientNet-B6~\cite{icml2019efficientnet}  & ICML'2019    & $528^2$ & 43     & 19.0  & 84.0      \\
    RegNetY-8GF$^\dag$~\cite{cvpr2020regnet}     & CVPR'2020    & $224^2$ & 39     & 8.1   & 82.2      \\ \hline
    T2T-ViT-24~\cite{iccv2021t2t}                & ICCV'2021    & $224^2$ & 64     & 13.2  & 82.2      \\
    Swin-S~\cite{liu2021swin}                    & ICCV'2021    & $224^2$ & 50     & 8.7   & 83.0      \\
    Focal-S~\cite{nips2021Focal}                 & NIPS'2021    & $224^2$ & 51     & 9.1   & 83.6      \\
    CSWin-S~\cite{cvpr2022CSWin}                 & CVPR'2022    & $224^2$ & 35     & 6.9   & 83.6      \\
    LITV2-M~\cite{nips2022hilo}                  & NIPS'2022    & $224^2$ & 49     & 7.5   & 83.3      \\ \hline
    CoaT-M~\cite{iccv2021coat}                   & ICCV'2021    & $224^2$ & 45     & 9.8   & 83.6      \\
    Twins-SVT-B~\cite{nips2021Twins}             & NIPS'2021    & $224^2$ & 56     & 8.6   & 83.2      \\
    CoAtNet-1~\cite{nips2021coatnet}             & NIPS'2021    & $224^2$ & 42     & 8.4   & 83.3      \\
    UniFormer-B~\cite{iclr2022uniformer}         & ICLR'2022    & $224^2$ & 50     & 8.3   & 83.9      \\
    FAN-B-Hybrid~\cite{icml2022FAN}              & ICML'2022    & $224^2$ & 50     & 11.3  & 83.9      \\ \hline
    ConvNeXt-S~\cite{cvpr2022convnext}           & CVPR'2022    & $224^2$ & 50     & 8.7   & 83.1      \\
    FocalNet-S (LRF)~\cite{nips2022focalnet}     & NIPS'2022    & $224^2$ & 50     & 8.7   & 83.5      \\
    HorNet-S$_{7\times 7}$~\cite{nips2022hornet} & NIPS'2022    & $224^2$ & 50     & 8.8   & 84.0      \\
    VAN-B3~\cite{guo2022van}                     & arXiv'2022   & $224^2$ & 45     & 9.0   & 83.9      \\
    SLaK-S~\cite{Liu2022SLak}                    & ICLR'2023    & $224^2$ & 55     & 9.8   & 83.8      \\
    \rowcolor{gray94}\bf{MogaNet-B}              & Ours         & $224^2$ & 44     & 9.9   & \bf{84.3} \\
    \bottomrule
    \end{tabular}
    }
    \vspace{-0.5em}
    \caption{\textbf{ImageNet-1K classification} performance of medium size (around 45M parameters) models.}
    \label{tab:in1k_cls_base}
\end{table}

\end{minipage}
~\begin{minipage}{0.495\linewidth}
\centering
    \begin{table}[H]
    \vspace{-0.25em}
    \setlength{\tabcolsep}{0.9mm}
    \centering
\resizebox{\linewidth}{!}{
\begin{tabular}{llcccc}
    \toprule
Architecture                    & Date       & Image   & Param. & FLOPs & Top-1     \\
                                &            & Size    & (M)    & (G)   & Acc (\%)  \\ \hline
ResNet-152$^\dag$               & CVPR'2016  & $224^2$ & 60     & 11.6  & 82.0      \\
SE-ResNet-154$^\dag$            & CVPR'2018  & $224^2$ & 115    & 20.9  & 81.7      \\
RegNetY-16GF                    & CVPR'2020  & $224^2$ & 84     & 16.0  & 82.9      \\ \hline
DeiT-B                          & ICML'2021  & $224^2$ & 86     & 17.5  & 81.8      \\
Swin-B                          & ICCV'2021  & $224^2$ & 89     & 15.4  & 83.5      \\
Focal-B                         & NIPS'2021  & $224^2$ & 90     & 16.4  & 84.0      \\
CSWin-B                         & CVPR'2022  & $224^2$ & 78     & 15.0  & 84.2      \\
LITV2-B                         & NIPS'2022  & $224^2$ & 87     & 13.2  & 83.6      \\ \hline
BoTNet-T7                       & CVPR'2021  & $256^2$ & 79     & 19.3  & 84.2      \\
Twins-SVT-L                     & NIPS'2021  & $224^2$ & 99     & 15.1  & 83.7      \\
CoAtNet-2                       & NIPS'2021  & $224^2$ & 75     & 15.7  & 84.1      \\
FAN-B-Hybrid                    & ICML'2022  & $224^2$ & 77     & 16.9  & 84.3      \\ \hline
ConvNeXt-B                      & CVPR'2022  & $224^2$ & 89     & 15.4  & 83.8      \\
RepLKNet-31B                    & CVPR'2022  & $224^2$ & 79     & 15.3  & 83.5      \\
FocalNet-B (LRF)                & NIPS'2022  & $224^2$ & 89     & 15.4  & 83.9      \\
HorNet-B$_{7\times 7}$          & NIPS'2022  & $224^2$ & 87     & 15.6  & 84.3      \\
VAN-B4                          & CVMJ'2023  & $224^2$ & 60     & 12.2  & 84.2      \\
SLaK-B                          & ICLR'2023  & $224^2$ & 95     & 17.1  & 84.0      \\
\rowcolor{gray94}\bf{MogaNet-L} & Ours       & $224^2$ & 83     & 15.9  & \bf{84.7} \\
    \bottomrule
    \end{tabular}
    }
    \vspace{-0.5em}
    \caption{\textbf{ImageNet-1K classification} performance of large size (around 80M parameters) models.}
    \label{tab:in1k_cls_large}
\end{table}

\end{minipage}
\vspace{-1.25em}
\end{figure*}

\paragraph{Results.}
We compare the image classification performances of four widely adopted model sizes (around 5M, 25M, 45M, and 80M parameters).
As for lightweight models, Table~\ref{tab:in1k_cls_tiny} shows that MogaNet-XT/T significantly outperforms existing lightweight architectures. Using the default training settings, MogaNet-T achieves 79.0\% top-1 accuracy, which improves models with around 5M parameters by at least 1.1\% using $224^2$ resolutions, while outperforming the current best backbone ParC-Net-S by 1.0\% using $256^2$ resolutions. Meanwhile, MogaNet-XT also surpasses models with 3M parameters, \textit{e.g.,} +4.6\% and +1.5\% over T2T-ViT-7 and MobileViT-XS. Particularly, MogaNet-T$^{\S}$ achieves 80.0\% top-1 accuracy using $256^2$ resolutions and the refined settings, which adjusts $lr$ and replaces RandAugment~\cite{cubuk2020randaugment} with 3-Augment~\cite{eccv2022deit3} as detailed in Appendix~\ref{app:advanced_tiny}.
As for small-size models, Table~\ref{tab:in1k_cls_small} shows MogaNet-S achieves 83.4\% top-1 accuracy, which consistently outperforms Transformers, hybrid architectures, and ConvNets, \textit{e.g.,} +2.1\% and +1.2\% over Swin-T and ConvNeXt-T. 
As for 45M and 80M models, we summarize their performances in Table~\ref{tab:in1k_cls_base} and Table~\ref{tab:in1k_cls_large} and MogaNet-B/L still surpass the current state-of-the-art architectures, especially improving Swin-S/B and ConvNeXt-S/B by 1.2\%/ 1.1\% and 1.1\%/ 0.8\%. MogaNet also outperforms recently proposed modern ConvNets, \textit{e.g.,} +0.9\% over RepLKNet-31B and +0.2\%/ 0.3\% over HorNet-S/B$_{7\times 7}$.


\subsection{Dense Prediction Tasks}
\label{sec:exp_det_seg}
\paragraph{Object detection and segmentation on COCO.}
We evaluate MogaNet for object detection and segmentation tasks on the COCO dataset using Mask-RCNN~\cite{2017iccvmaskrcnn} as the detector. Following the training and evaluation settings in \cite{liu2021swin}, we fine-tune the models with AdamW optimizer for $1\times$ training schedule (12-epoch) on the COCO~\textit{train2017} and evaluate on the COCO~\textit{val2017}. We adopt MMDetection~\cite{mmdetection} as the codebase and measure the performance by the box mAP (AP$^{bb}$) and mask mAP (AP$^{mk}$). Refer to Appendix~\ref{app:coco_settings} for more details. Table~\ref{tab:coco} shows that models with MogaNet-T/S/B significantly outperform all previous backbones. Specifically, MogaNet-T gains 3.6\% AP$^{bb}$ and 4.6\% AP$^{mk}$ over ResNet-18; MogaNet-S outperforms Swin-T (Transformers) by 3.9\% AP$^{bb}$ and 2.7\% AP$^{mk}$, and surpasses UniFormer-S (hybrid) by 0.5\% AP$^{bb}$; MogaNet-B outperforms Swin-T and LITV2-M (Transformer) by 2.9\% AP$^{bb}$ and 1.2\% AP$^{mk}$ respectively.

\vspace{-1.0em}
\paragraph{Semantic segmentation on ADE20K.}
We then evaluate MogaNet for semantic segmentation tasks on the ADE20K dataset using Semantic FPN~\cite{cvpr2019semanticFPN} and UperNet~\cite{eccv2018upernet} following the evaluation schemes in \cite{liu2021swin, yu2022metaformer}. All experiments are implemented on MMSegmentation~\cite{mmseg2020} codebase, and the performance is measured by mIoU (single scale). Based on Semantic FPN, the models are fine-tuned for 80K iterations by the AdamW optimizer. In Table~\ref{tab:ade20k}, MogaNet-S consistently outperforms previous architectures, \textit{e.g.,} +6.6\% over Swin-T (Transformer), +1.5\% over Uniformer-S (hybrid). Based on UperNet, the models are fine-tuned 160K by AdamW optimizer. In Table~\ref{tab:ade20k}, the models with MogaNet-S improves backbones of Transformers (+3.1\% over Swin-T), hybrid architectures (+1.6\% over UniFormer-S), and modern ConvNets (+1.1\% over HorNet-T$_{7\times 7}$. Refer to Appendix~\ref{app:ade20k_settings} for more details.

% figure (interaction) & table (ablation)
\begin{figure}[hb]
\vspace{-1.25em}
\centering
\begin{minipage}{0.38\linewidth}
    \vspace{-1.25em}
    \centering
    \begin{table}[H]
    \vspace{-0.5em}
    \setlength{\tabcolsep}{0.7mm}
    \centering
\resizebox{\linewidth}{!}{
    \begin{tabular}{l|c}
    \toprule
Modules                       & Top-1     \\
                              & Acc (\%)  \\ \hline
ConvNeXt-T                    & 82.1      \\
Baseline                      & 82.2      \\ \hline
\rowcolor{gray94}Moga Block   & \bf{83.4} \\
$- \mathrm{FD}(\cdot)$        & 83.2      \\
$-$Multi-$\mathrm{DW}(\cdot)$ & 83.1      \\
$- \mathrm{Moga}(\cdot)$      & 82.7      \\
$- \mathrm{CA}(\cdot)$        & 82.9      \\
    \bottomrule
    \end{tabular}
    }
    \vspace{-0.5em}
    % \caption{\textbf{Ablation of the designed modules on ImageNet-1K}.
    % }
    \label{tab:ablation_small}
    \vspace{-1.0em}
\end{table}

    \vspace{3pt}
    % \vspace{-0.25em}
\end{minipage}
~\begin{minipage}{0.59\linewidth}
    \centering
    \includegraphics[width=1.0\linewidth,trim= 4 0 0 0,clip]{Figs/fig_ablation_interaction.pdf}
    \vspace{-2.25em}
\end{minipage}
    \caption{
    \textbf{Ablation of the proposed modules on ImageNet-1K.} \textbf{Left}: the table verifies each proposed module based on the baseline of MogaNet-S. \textbf{Right}: the figure plots distributions of the interaction strength $J^{(m)}$ and verifies that $\mathrm{Miga}(\cdot)$ contributes the most to learning multi-order interactions and better performance.
    }
    \label{fig:ablation_interaction}
\vspace{-1.5em}
\end{figure}

% figure: gradcam
\begin{figure}[hb]
    \vspace{-0.75em}
    \centering
    \includegraphics[width=1.0\linewidth,trim= 4 0 0 0,clip]{Figs/fig_analysis_gradcam.pdf}
    \vspace{-1.75em}
    \caption{
    \textbf{Grad-CAM activation maps of models trained on ImageNet-1K.} MogaNet-S shows similar activation maps as local attention architectures (Swin-T), which are located on the semantic targets. Unlike the results of previous ConvNets, which might activate some irrelevant parts, the activation maps of MogaNet-S are more gathered. See more visualizations in Appendix~\ref{app:gradcam}.
    }
    \label{fig:analysis_gradcam}
    \vspace{-1.25em}
\end{figure}

% table: COCO & ADE20K
\begin{figure*}[t!]
\vspace{-1.5em}
\begin{minipage}{0.555\linewidth}
\centering
    \begin{table}[t]
    \vspace{-0.25em}
    \setlength{\tabcolsep}{0.4mm}
    \centering
\resizebox{\linewidth}{!}{
\begin{tabular}{lllcccc}
    \toprule
    Architecture                                 & Data      & Method        & Param. & FLOPs & AP$^{b}$  & AP$^{m}$  \\
                                                 &           &               & (M)    & (G)   & (\%)      & (\%)      \\ \hline
    ResNet-101~\cite{he2016deep}                 & CVPR'2016 & RetinaNet     & 57     & 315   & 38.5      & -         \\
    PVT-S~\cite{iccv2021PVT}                     & ICCV'2021 & RetinaNet     & 34     & 226   & 40.4      & -         \\
    CMT-S~\cite{guo2021cmt}                      & CVPR'2022 & RetinaNet     & 45     & 231   & 44.3      & -         \\
    \rowcolor{gray94}\bf{MogaNet-S}              & Ours      & RetinaNet     & 35     & 253   & \bf{45.8} & -         \\ \hline
    RegNet-1.6G~\cite{cvpr2020regnet}            & CVPR'2020 & Mask R-CNN    & 29     & 204   & 38.9      & 35.7      \\
    PVT-T~\cite{iccv2021PVT}                     & ICCV'2021 & Mask R-CNN    & 33     & 208   & 36.7      & 35.1      \\
    \rowcolor{gray94}\bf{MogaNet-T}              & Ours      & Mask R-CNN    & 25     & 192   & \bf{42.6} & \bf{39.1} \\ \hline
    Swin-T~\cite{liu2021swin}                    & ICCV'2021 & Mask R-CNN    & 48     & 264   & 42.2      & 39.1      \\
    Uniformer-S~\cite{iclr2022uniformer}         & ICLR'2022 & Mask R-CNN    & 41     & 269   & 45.6      & 41.6      \\
    ConvNeXt-T~\cite{cvpr2022convnext}           & CVPR'2022 & Mask R-CNN    & 48     & 262   & 44.2      & 40.1      \\
    PVTV2-B2~\cite{cvmj2022PVTv2}                & CVMJ'2022 & Mask R-CNN    & 45     & 309   & 45.3      & 41.2      \\
    LITV2-S~\cite{nips2022hilo}                  & NIPS'2022 & Mask R-CNN    & 47     & 261   & 44.9      & 40.8      \\
    FocalNet-T~\cite{nips2022focalnet}           & NIPS'2022 & Mask R-CNN    & 49     & 267   & 45.9      & 41.3      \\
    \rowcolor{gray94}\bf{MogaNet-S}              & Ours      & Mask R-CNN    & 45     & 272   & \bf{46.7} & \bf{42.2} \\ \hline
    Swin-S~\cite{liu2021swin}                    & ICCV'2021 & Mask R-CNN    & 69     & 354   & 44.8      & 40.9      \\
    Focal-S~\cite{nips2021Focal}                 & NIPS'2021 & Mask R-CNN    & 71     & 401   & 47.4      & 42.8      \\
    ConvNeXt-S~\cite{cvpr2022convnext}           & CVPR'2022 & Mask R-CNN    & 70     & 348   & 45.4      & 41.8      \\
    HorNet-B$_{7\times 7}$~\cite{nips2022hornet} & NIPS'2022 & Mask R-CNN    & 68     & 322   & 47.4      & 42.3      \\
    \rowcolor{gray94}\bf{MogaNet-B}              & Ours      & Mask R-CNN    & 63     & 373   & \bf{47.9} & \bf{43.2} \\ \hline
    Swin-L$^\ddag$~\cite{liu2021swin}            & ICCV'2021 & Cascade Mask  & 253    & 1382  & 53.9      & 46.7      \\
    ConvNeXt-L$^\ddag$~\cite{cvpr2022convnext}   & CVPR'2022 & Cascade Mask  & 255    & 1354  & 54.8      & 47.6      \\
    RepLKNet-31L$^\ddag$~\cite{cvpr2022replknet} & CVPR'2022 & Cascade Mask  & 229    & 1321  & 53.9      & 46.5      \\
    HorNet-L$^\ddag$~\cite{nips2022hornet}       & NIPS'2022 & Cascade Mask  & 259    & 1399  & 56.0      & 48.6      \\
    \rowcolor{gray94}\bf{MogaNet-XL}$^\ddag$     & Ours      & Cascade Mask  & 238    & 1355  & \bf{56.2} & \bf{48.8} \\
    \bottomrule
    \end{tabular}
    }
    \vspace{-0.5em}
    \caption{\textbf{Object detection and instance segmentation} with RetinaNet ($1\times$), Mask R-CNN ($1\times$), and Cascade Mask R-CNN (multi-scale $3\times$) on COCO \textit{val2017}. $^\ddag$ indicates using ImageNet-21K pre-trained models. The FLOPs are measured at resolution $800\times 1280$.}
    \vspace{-1.0em}
    \label{tab:coco}
\end{table}

\end{minipage}
\begin{minipage}{0.45\linewidth}
\centering
    \begin{table}[t]
    \vspace{-0.25em}
    \setlength{\tabcolsep}{0.9mm}
    \centering
\resizebox{\linewidth}{!}{
\begin{tabular}{c|llcccc}
    \toprule
Method           & Architecture                                 & Date                   & Crop                      & Param.                & FLOPs                  & mIoU$^{ss}$                 \\
                 &                                              &                        & size                      & (M)                   & (G)                    & (\%)                        \\ \hline
                 & ResNet50~\cite{he2016deep}                   & CVPR'2016              & 512$^2$                   & 29                    & 183                    & 36.7                        \\
                 & PVT-S~\cite{iccv2021PVT}                     & ICCV'2021              & 512$^2$                   & 28                    & 161                    & 39.8                        \\
\small{Semantic} & Twins-S~\cite{nips2021Twins}                 & NIPS'2021              & 512$^2$                   & 28                    & 162                    & 44.3                        \\
FPN              & Swin-T~\cite{liu2021swin}                    & ICCV'2021              & 512$^2$                   & 32                    & 182                    & 41.5                        \\
(80K)            & Uniformer-S~\cite{iclr2022uniformer}         & ICLR'2022              & 512$^2$                   & 25                    & 247                    & 46.6                        \\
                 & LITV2-S~\cite{nips2022hilo}                  & NIPS'2022              & 512$^2$                   & 31                    & 179                    & 44.3                        \\
                 % & VAN-B2~\cite{guo2022van}                     & arXiv'2022             & 512$^2$                   & 30                    & 164                    & 46.7                        \\
                 & \cellcolor{gray94}\bf{MogaNet-S}             & \cellcolor{gray94}Ours & \cellcolor{gray94}512$^2$ & \cellcolor{gray94}29  & \cellcolor{gray94}189  & \cellcolor{gray94}\bf{47.7} \\ \hline
                 & DeiT-S~\cite{icml2021deit}                   & ICML'2021              & 512$^2$                   & 52                    & 1099                   & 44.0                        \\
                 & Swin-T~\cite{liu2021swin}                    & ICCV'2021              & 512$^2$                   & 60                    & 945                    & 46.1                        \\
                 & ConvNeXt-T~\cite{cvpr2022convnext}           & CVPR'2022              & 512$^2$                   & 60                    & 939                    & 46.7                        \\
                 & Twins-S~\cite{nips2021Twins}                 & NIPS'2021              & 512$^2$                   & 54                    & 901                    & 46.2                        \\
                 & UniFormer-S~\cite{iclr2022uniformer}         & ICLR'2022              & 512$^2$                   & 52                    & 1008                   & 47.6                        \\
                 & HorNet-T$_{7\times 7}$~\cite{nips2022hornet} & NIPS'2022              & 512$^2$                   & 52                    & 926                    & 48.1                        \\
                 & \cellcolor{gray94}\bf{MogaNet-S}             & \cellcolor{gray94}Ours & \cellcolor{gray94}512$^2$ & \cellcolor{gray94}55  & \cellcolor{gray94}946  & \cellcolor{gray94}\bf{49.2} \\ \cline{2-7} 
                 & Swin-S~\cite{liu2021swin}                    & ICCV'2021              & 512$^2$                   & 81                    & 1038                   & 48.1                        \\
                 & ConvNeXt-S~\cite{cvpr2022convnext}           & CVPR'2022              & 512$^2$                   & 82                    & 1027                   & 48.7                        \\
UperNet          & SLaK-S~\cite{Liu2022SLak}                    & ICLR'2023              & 512$^2$                   & 91                    & 1028                   & 49.4                        \\
(160K)           & \cellcolor{gray94}\bf{MogaNet-B}             & \cellcolor{gray94}Ours & \cellcolor{gray94}512$^2$ & \cellcolor{gray94}74  & \cellcolor{gray94}1050 & \cellcolor{gray94}\bf{50.1} \\ \cline{2-7} 
                 & Swin-B~\cite{liu2021swin}                    & ICCV'2021              & 512$^2$                   & 121                   & 1188                   & 49.7                        \\
                 & ConvNeXt-B~\cite{cvpr2022convnext}           & CVPR'2022              & 512$^2$                   & 122                   & 1170                   & 49.1                        \\
                 & RepLKNet-31B~\cite{cvpr2022replknet}         & CVPR'2022              & 512$^2$                   & 112                   & 1170                   & 49.9                        \\
                 & SLaK-B~\cite{Liu2022SLak}                    & ICLR'2023              & 512$^2$                   & 135                   & 1185                   & 50.2                        \\
                 & \cellcolor{gray94}\bf{MogaNet-L}             & \cellcolor{gray94}Ours & \cellcolor{gray94}512$^2$ & \cellcolor{gray94}113 & \cellcolor{gray94}1176 & \cellcolor{gray94}\bf{50.9} \\ \cline{2-7} 
                 & Swin-L$^\ddag$~\cite{liu2021swin}            & ICCV'2021              & 640$^2$                   & 234                   & 2468                   & 52.1                        \\
                 & ConvNeXt-L$^\ddag$~\cite{cvpr2022convnext}   & CVPR'2022              & 640$^2$                   & 245                   & 2458                   & 53.7                        \\
                 & RepLKNet-31L$^\ddag$~\cite{cvpr2022replknet} & CVPR'2022              & 640$^2$                   & 207                   & 2404                   & 52.4                        \\
                 & \cellcolor{gray94}\bf{MogaNet-XL}$^\ddag$    & \cellcolor{gray94}Ours & \cellcolor{gray94}640$^2$ & \cellcolor{gray94}214 & \cellcolor{gray94}2451 & \cellcolor{gray94}\bf{54.0} \\
    \bottomrule
    \end{tabular}
    }
    \vspace{-0.5em}
    \caption{\textbf{Semantic segmentation} with semantic FPN (80K) and UperNet (160K) on ADE20K validation set. $^\ddag$ indicates using IN-21K pre-trained models. The FLOPs are measured at $512\times 2048$ or $640\times 2560$ resolutions.}
    \vspace{-1.0em}
    \label{tab:ade20k}
\end{table}

% Semantic FPN
% ResNet-50 80k
% PVT-S 40k
% PVT.V2-S 40k
% Swin-T 80k
% Twins-S 80k
% Poolformer-M36 40k
% Uniformer-S 80k
% LIT.V2 80k
% VAN-B2 40k
% MogaNet-S 80k

\end{minipage}
\vspace{-1.5em}
\end{figure*}

\subsection{Ablation and Analysis}
\label{sec:exp_ablation}
We first ablate the spatial aggregation module, including \textbf{$\mathrm{FD}(\cdot)$} and $\mathrm{Moga}(\cdot)$, which contains the \textbf{gating branch} and the context branch with \textbf{multi-order DWConv layers}, and the \textbf{channel aggregation} module $\mathrm{CA}(\cdot)$.
% As verified in Table~\ref{tab:ablation}, the proposed modules yield +2.4\% performance gain to the baselines.
As verified in Table~\ref{tab:ablation} and Figure~\ref{fig:ablation_interaction} (left), all proposed modules yield improvements with a few costs. Appendix~\ref{app:ablation} provides more ablation studies.
Furthermore, we empirically verify the multi-order interactions in Figure~\ref{app:ablation_multiorder} (right) and visualize class activation maps (CAM) by Grad-CAM~\cite{cvpr2017grad} in comparison to existing models in Figure~\ref{fig:analysis_gradcam}.


\section{Conclusion}
% for ICLR submit
\if\submission\submissionFinal
    \vspace{-0.5em}
    \section{Conclusion}
    \label{sec:conclusion}
    \vspace{-0.25em}
    This paper introduces a new modern ConvNet architecture, named MogaNet, through the lens of multi-order game-theoretic interaction.
    Built upon the modern ConvNet framework, we present a compact Moga Block and channel aggregation module to force the network to emphasize the expressive but inherently overlooked interactions across spatial and channel perspectives.
    Extensive experiments verify the consistent superiority of MogaNet in terms of both performance and efficiency compared to popular ConvNets, ViTs, and hybrid architectures on various vision benchmarks.
% 
\else
% for arXiv
% \if\submission\submissionarXiv
    \section{Conclusion}
    \label{sec:conclusion}
    In this paper, we introduce a new modern ConvNet architecture named MogaNet through the lens of multi-order game-theoretic interaction.
    % We demonstrate that using a spatial aggregation block and a channel aggregation block results in stronger feature interactions of intermediate complexities efficiently, boosting the performance of ConvNet architecture substantially on diverse vision scenarios.
    Built upon the modern ConvNet framework, we present a compact Moga Block and channel aggregation module to \pl{force the network to emphasize the expressive but inherently overlooked interactions} across spatial and channel spaces.
    %
    Extensive experiments demonstrate the consistent superiority of MogaNet in terms of both accuracy and computational efficiency compared to popular ConvNets, ViTs, and hybrid architectures on various vision benchmarks.
    We hope our work can prompt people to perceive the importance of multi-order interaction in representation learning and to facilitate the development of efficient deep architecture design.
%
\fi


\section*{Acknowledgement}
This work was supported by the National Key R\&D Program of China (No. 2022ZD0115100), the National Natural Science Foundation of China Project (No. U21A20427), and Project (No. WU2022A009) from the Center of Synthetic Biology and Integrated Bioengineering of Westlake University.
This work was done when Zedong Wang and Zhiyuan Chen interned at Westlake University. We thank the AI Station of Westlake University for the support of GPUs. We also thank Mengzhao Chen, Zhangyang Gao, Jianzhu Guo, Fang Wu, and all anonymous reviewers for polishing the writing of the manuscript.



%%%%%%%%% REFERENCES
{\small
\bibliographystyle{ieee_fullname}
\bibliography{egbib}
}

\end{document}
