\begin{figure*}[t]
    \centering
    \begin{subfigure}[b]{0.92\textwidth}
        \centering
        \includegraphics[width=\textwidth]{figure/qcompare1.pdf}
        \caption{Text-to-Image performance.}
        \vspace{0.2cm}
        \label{fig:qcompare_t2i}
    \end{subfigure}
    
    \begin{subfigure}[b]{0.92\textwidth}
        \centering
        \includegraphics[width=\textwidth]{figure/qcompare2.pdf}
        \caption{Image-Variation performance.}
        \vspace{0.2cm}
        \label{fig:qcompare_i2i}
    \end{subfigure}
    \begin{subfigure}[b]{0.92\textwidth}
        \centering
        \includegraphics[width=\textwidth]{figure/qcompare3-small.pdf}
        \caption{Image-to-Text performance.}
        \vspace{0.1cm}
        \label{fig:qcompare_i2t}
    \end{subfigure}
    
    \caption{
        These figures show the qualitative comparison between our VD models and prior works, from which we conclude that VD performs well on all three tasks. In text-to-image and image-variation, VD captures semantics from the input context more accurately. In image-to-text, VD generates more creative sentences and has a better chance to describe images with more details.
    }
    \label{fig:qcompare}
\end{figure*}
