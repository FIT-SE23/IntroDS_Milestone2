\begin{figure}
    \centering
    \footnotesize
    
    \begin{tabular}{c}
    \includegraphics[width=0.47\textwidth,height=2.5cm]
        {imgs/skip/image/004932.png} \\
        Front-view Image \\
    \end{tabular}
    
	\begin{tabular}{c@{}c@{}c}
        \includegraphics[width=0.19\textwidth,height=2.5cm]
        {imgs/skip/refine/1.png} &
		\includegraphics[width=0.14\textwidth,height=2.5cm]
        {imgs/skip/refine/2.png} &
		\includegraphics[width=0.14\textwidth,height=2.5cm]
        {imgs/skip/refine/3.png} \\
        $i=0$ & $i=1$ & $i=2$ \\
        \includegraphics[width=0.19\textwidth,height=2.5cm]
        {imgs/skip/refine/4.png} &
		\includegraphics[width=0.14\textwidth,height=2.5cm]
        {imgs/skip/refine/5.png} &
		\includegraphics[width=0.14\textwidth,height=2.5cm]
        {imgs/skip/new/004932.png} \\
        $i=3$ & $i=4$ & $i=5$
        
	\end{tabular}
	\caption{We visualize the segmentation maps generated from top-view features at the $i$-th scale (i.e., $i \in \{0,...,5\}$). In addition, we highlight the pixels that are discrepancies between our estimation and the ground truth. The green and red pixels represent False Negatives and False Positives, respectively.}
	\label{fig:re}
\end{figure}