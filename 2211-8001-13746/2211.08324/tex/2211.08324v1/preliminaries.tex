\section{Preliminaries, Augmentation LP, and Background}
\label{sec:prelim}

Let $G=(V,E)$ be a graph with edge capacities $x: E
\rightarrow \mathbb{R}_+$, and let $S,T$ be two disjoint vertex
subsets. We say that $x$ supports a flow of value $f$ between $A$ and
$B$ if, in the graph $G'$ obtained by shrinking $S$ to $s$ and $T$ to
$t$, the max $s$-$t$ flow is at least $f$.


\subsection{LP Relaxation for Augmentation}
Recall the definition of the $(p,q)$-Flex-SNDP problem: given a graph
$G = (V, E)$ with edge costs $c(e)$, a partition $\calU \uplus \calS$
on the edge set, and terminal pairs $(s_i, t_i) \in V \times V$, $i
\in [k]$, find the cheapest subgraph such that each terminal pair
$(s_i, t_i)$ is $(p,q)$-flex-connected. Equivalently, any cut $\delta(S)$
separating a terminal pair must have at least $p$ safe edges or at least $p +
q$ total edges. One can verify the feasibility of a solution in 
$n^{O(q)}$-time: for each subset of $q$ unsafe edges, remove them
and test for $p$-connectivity between terminal pairs.
We employ the augmentation methodology. Suppose we are given a partial solution $H
\subseteq E$ that satisfies $(p,q-1)$-flex-connectivity for the given
terminal pairs. We call a cut $S$ \emph{violated} with respect to $H$
if $|S \cap \{s_i,t_i\}| = 1$ for some $i$, and $|\delta_H(S) \cap \calU| < p$
  and $|\delta_H(S)| = p+q-1$. Note that any cut separating a
terminal pair in $H$ has at least $p$ safe edges or at least $p+q-1$ total
edges. Let $\calC = \{S \subseteq V:
S \text{ is violated}\}$.  We
can augment $H$ to obtain a feasible solution to $(p,q)$-Flex-SNDP
instance by covering all cuts in $\calC$.  This naturally leads to a cut-based
LP relaxation for the augmentation problem with variables $x_e \in
[0,1]$ for $e \in E \setminus H$:

\begin{align*}
    \min \sum_{e \in E \setminus H} c(e)x_e& & \text{\bf Augment-LP} \\
    \text{subject to } \sum_{e \in \delta_{E \setminus H}(S)} x_e &\geq 1 &\forall S \in \calC  \\
    x_e \in [0, 1] & & e \in E \setminus H
\end{align*}

\begin{claim}
  Augment-LP admits an $n^{O(q)}$-time separation oracle and hence can be solved in polynomial time for each fixed $q$.
\end{claim}

Recall that $(p, 0)$-Flex-SNDP is equivalent to EC-SNDP with all
terminal pairs having requirement $p$. We can obtain a
$2$-approximate feasible solution. For $\ell=0$ to $q-1$ we solve an
augmentation problem in each stage to go from $(p,\ell)$ to $(p,\ell+1)$
flex-connectivity. Thus an $\alpha$-approximation for the augmentation
problem implies an overall $(2 + q\alpha)$-approximation for $(p,
q)$-Flex-SNDP.

\begin{remark}
There is an LP relaxation for $(p,q)$-Flex-SNDP with the property that
a feasible fractional solution to it is also feasible for Augment-LP
for each stage \cite{ChekuriJ22}. Thus, proving an integrality gap
bound for Augment-LP gives upper bounds on the integrality gap of the
LP for $(p,q)$-Flex-SNDP.
\end{remark}


\subsection{\racke Tree Embeddings}
The results in this paper use \racke's capacity-based probabilistic
tree embeddings. We borrow the notation from~\cite{ChenLLZ22}. Given $G = (V, E)$ with capacity $x: E \to \R^+$
on the edges, a capacitated tree embedding of $G$ is a tree $\calT$,
along with two mapping functions $\calM_1: V(\calT) \rightarrow V(G)$
and $\calM_2: E(\calT) \rightarrow 2^{E(G)}$ that satisfy some
conditions. $\calM_1$ maps each vertex in $\calT$ to a vertex in $G$,
and has the additional property that it gives a one-to-one mapping
between the leaves of $\calT$ and the vertices of $G$.  $\calM_2$ maps
each edge $(a,b) \in E(\calT)$ to a path in $G$ between $\calM_1(a)$
and $\calM_1(b)$.  For notational convenience we view the two mappings
as a combined mapping $\calM$. For a vertex $u \in V(G)$ we use
$\calM^{-1}(u)$ to denote the leaf in $\calT$ that is mapped to $u$ by
$\calM_1$. For an edge $e \in E(G)$ we use $\calM^{-1}(e) = \{f \in
E(\calT)\mid e \in \calM_2(f)\}$. It is sometimes convenient to
view a subset $S \subseteq V(G)$ both as vertices in $G$ and
also corresponding leaves of $\calT$.

The mapping $\calM$ induces a capacity function $y: E(\calT)
\rightarrow \mathbb{R}_+$ as follows. Consider $f=(a,b) \in
E(\calT)$. $\calT - f$ induces a partition $(A,B)$ of $V(T)$ which in
turn induces a partition/cut $(A',B')$ of $V(G)$ via the mapping
$\calM$: $A'$ is the set of vertices in $G$ that correspond to the
leaves in $A$ and similarly $B'$. We then set $y(f) = \sum_{e \in
  \delta(A')} x(e)$, in other words $y(f)$ is the capacity of cut
$(A',B')$ in $G$. The mapping also induces loads on the edges of
$G$. For each edge $e \in G$, we let $\load(e) = \sum_{f \in E(\calT):
  e \in M(f)} y(f)$.  The relative load or \emph{congestion} of $e$ is
$\rload(e) = \load(e)/x(e)$.  The congestion of $G$ with respect to a
tree embedding $(\calT, \calM)$ is defined as $\max_{e \in E(G)}
\rload(e)$. Given a probabilistic distribution $\calD$ on trees
embeddings of $(G,x)$ we let
$$\beta_{\calD} = \max_{e \in E(G)}  \E_{(\calT,\calM) \sim \calD} \rload(e)$$
denote the maximum expected congestion.
\racke showed the following fundamental result on probabilistic embeddings of a capacitated graph into trees.

\begin{theorem}[\cite{Racke08}]
\label{thm:racke}
Given a graph $G$ and $x: E(G) \to \R^+$, there exists a probability
distribution $\calD$ on tree embeddings such that $\beta_{\calD} =
O(\log |V(G)|)$.  All trees in the support of $\calD$ have height at
most $O(\log(nC))$, where $C$ is the ratio of the largest to smallest
capacity in $x$. Moreover, there is randomized polynomial-time
algorithm that can sample a tree from the distribution $\calD$.
\end{theorem}

In the rest of the paper we use $\beta$ to denote the guarantee
provided by the preceding theorem where $\beta =
O(\log n)$ for a graph on $n$ nodes.

\paragraph{Implication for flows:} The original motivation for capacitated tree embeddings
is oblivious routing of multicommodity flows. A multicommodity flow
instance in a graph $G=(V,E)$ is specified by a demand matrix $D: V
\times V \rightarrow \mathbb{R}_+$ and the goal is to simultaneously
route $D(u,v)$ amount of flow between $u$ and $v$ for each vertex pair
$(u,v)$. We say that $D$ is routable in $G$ with congestion $\alpha$
if there is a feasible multicommodity flow in $G$ that satisfies all
demands such that the total flow on each edge $e$ is at most $\alpha
\cdot x(e)$ (it is routable if $\alpha \le 1$). It can be seen that
given a tree $(\calT,\calM)$ in $\calD$, any multicommodity flow that
can be routed in $G$ with capacities $x$ can also be routed in $\calT$
with capacities given by $y$ with congestion $1$ --- this is because
cut-condition is necessary for routing and in trees it is also
sufficient. Moreover, any routable multicommodity flow in $\calT$ with
demands only between leaves can be routed in $G$ with congestion
$\max_{e \in E(G)} \rload(e)$. The mapping of the routing in $\calT$
to $G$ is simple and follows the paths given by $\calM$. The
implication of this connection is the following corollary where we use
$\maxflow_H^z(A,B)$ to denote the maxflow between two disjoint vertex
subsets $A,B$ in a capacitated graph $H$ with capacities given by $z:
E(H) \rightarrow \mathbb{R}_+$.

\begin{corollary}
\label{cor:racketreeflow}
Let $\calD$ be the distribution guaranteed in Theorem~\ref{thm:racke}.
Let $A, B \in V(G)$ be two disjoint sets. Then (i) for any tree
$(\calT,\calM)$ in $\calD$, $\maxflow_G^x(A, B) \leq
\maxflow_\calT^y(\calM^-(A), \calM^-(B))$ and (ii) $\frac 1 \beta
\E_{(\calT,\calM) \sim \calD}[\maxflow_\calT^y(\calM^-(A),
  \calM^-(B))] \leq \maxflow_G^x(A, B)$.
\end{corollary}

\subsection{Group Steiner Tree, Set Connectivity and Tree Rounding}
The group Steiner tree problem was introduced in \cite{ReichW89} and
studied in approximation by Garg, Konjevod and Ravi
\cite{GargKR98}. The input is an edge-weighted graph $G=(V,E)$, a root
vertex $r \in V$, and $k$ groups $S_1,S_2,\ldots,S_k$ where each $S_i
\subseteq V$. The goal is to find a min-weight subgraph $H$ of $G$
such there is a path in $H$ from $r$ to each group $S_i$ (that is, to
some vertex in $S_i$). The approximability of this problem has
attracted substantial attention.  Garg et al.\ \cite{GargKR98}
described a randomized algorithm to round a fractional solution to a
cut-based LP relaxation when $G$ is a tree --- it achieves a $O(\log n
\log k)$-approximation. This has been shown to be essentially tight
from both an integrality gap and a hardness point of view
\cite{HalperinKKSW07,HalperinK03}. Their algorithm also yields an
$O(\log^2 n \log k)$-approximation in general graphs via
embeddings into tree metrics \cite{Bartal98,FRT03}. Better approximation in quasi-polynomial time
are known \cite{ChekuriP05,GrandoniLL19,GhugeN22}.

Set Connectivity is a generalization of group Steiner tree problem. Here we are given pairs of sets
$(S_1,T_1),(S_2,T_2),\ldots,(S_k,T_k)$ and the goal is to find a
min-cost subgraph $H$ such that there is an $(S_i,T_i)$ path in $H$
for each $i$.
Chekuri et al.\ \cite{ChekuriEGS11} obtained a
poly-logarithmic approximation and integrality gap by generalizing the
ideas from group Steiner tree.
Chalermsook, Grandoni and Laekhanukit \cite{CGL15} studied Survivable
Set Connectivity problem, motivated by earlier work in
\cite{GuptaKR10}. Here each pair $(S_i,T_i)$ has a connectivity
requirement $r_i$ which implies that one seeks $r_i$ edge-disjoint
paths between $S_i$ and $T_i$ in the chosen subgraph $H$; \cite{CGL15}
obtained a bicriteria-approximation via \racke tree and group Steiner tree rounding.
The recent work of Chen et al \cite{ChenLLZ22} uses related but more sophisticated ideas
to obtain the first true approximation for this problem. They refer to
the problem as Group Connectivity problem and obtain an $O(r^3 \log
r \log^7 n)$-approximation where $r = \max_i r_i$ connectivity
requirement (see \cite{ChenLLZ22} for more precise bounds).

\paragraph{Oblivious tree rounding:} The rounding algorithm for Set Connectivity
in trees given in \cite{ChekuriEGS11} establishes a poly-logarithmic
integrality gap, however, the rounding is not \emph{oblivious} to the
pairs. In \cite{CGL15} a randomized oblivious algorithm based on the
group Steiner tree rounding from \cite{GargKR98} is described. This is
useful since the sets to be connected during the course of their
algorithm are implicitly generated. We encapsulate their result in the
following lemma. The tree rounding algorithm in \cite{CGL15,ChenLLZ22}
is phrased slightly differently since they combine aspects of group
Steiner rounding and the congestion mapping that comes from \racke
trees. We separate these two explicitly to make the idea more
transparent. We refer to the algorithm from the lemma below as TreeRounding.

\begin{lemma}[\cite{CGL15,ChenLLZ22}]
\label{lem:setconnectivity-tree-rounding}
Consider an instance of Set Connectivity on an $n$-node tree $T=(V,E)$
with height $h$ and let $x: E \rightarrow [0,1]$. Suppose $A, B
\subseteq V$ are disjoint sets and suppose $K \subseteq E$ such that
$x$ restricted to $K$ supports a flow of $f \le 1$ between $A$ and
$B$. There is a randomized algorithm that is oblivious to $A, B, K$
(hence depends only on $x$ and value $f$) that outputs a subset $E' \subseteq E$
such that (i) The probability that $E' \cap K$ connects $A$ to $B$ is
at least a fixed constant $\phi$ and (ii) For any edge $e \in E$, the
probability that $e \in E'$ is $\min\{1,O(\frac{1}{f} h \log^2 n) x(e)\}$.
\end{lemma}