\section{Introduction}
The Survivable Network Design Problem (SNDP) is an important problem in
combinatorial optimization that generalizes many well-known problems
related to connectivity, and is also motivated by practical problems
related to the design of fault-tolerant networks. The input to this
problem is an undirected graph $G=(V,E)$ with non-negative edge costs $c: E \to \R_+$ and an integer function $r: V
\times V \to \Z^+$ which specifies a connectivity requirement for each
node pair $(u,v)$. The goal is to find a minimum-cost subgraph $H$ of
$G$ such that $H$ has $r(u,v)$ connectivity for each pair $(u,v)$. Our
focus in this paper is on edge-connectivity requirements; the resulting
problem is referred to as EC-SNDP. VC-SNDP refers to the problem in
which each pair $(u,v)$ requires $r(u,v)$ vertex connectivity. 
EC-SNDP contains as special cases classical problems such
as $s$-$t$ shortest path, minimum spanning tree (MST), minimum
$k$-edge-connected subgraph ($k$-ECSS), Steiner tree, Steiner forest
and several others. It is NP-Hard and APX-Hard to approximate.
Jain's seminal $2$-approximation for EC-SNDP via iterated rounding \cite{Jain01}
is the currently the best known approximation ratio.

In this paper we are interested in a new network design model
suggested by Adjiashvili \cite{Adjiashvili13} for which there are
several recent developments
\cite{AdjiashviliHM20,AdjiashviliHM22,AdjiashviliHMS20,BoydCHI22,ChekuriJ22,BansalCGI22}.
In this model, the edge set $E$ is partitioned to \emph{safe} edges
$\calS$ and \emph{unsafe} edges $\calU$. Vertices $s, t \in V$ are
$(p,q)$-flex-connected
\footnote{We follow the terminology from our
recent work \cite{ChekuriJ22} that is influenced by
\cite{AdjiashviliHM22,BoydCHI22}.}
if $s$ and $t$ are
$p$-edge-connected after deleting any subset of at most $q$ unsafe
edges. The Flex-SNDP problem is the following: the input is a graph
$G=(V,E)$ with edge costs $c:E \rightarrow \mathbb{R}_+$, a partition
$\calU \uplus \calS$ of the edge set, and functions $p, q: V \times V
\to \Z^+$. The goal is to find a min-cost subgraph $H$ of $G$ such
that each $u, v \in V$ is $(p(u, v), q(u, v))$-flex-connected in
$H$. We denote by $(p,q)$-Flex-SNDP the special case where for each
vertex pair $u,v$, either $p(u,v) = q(u,v) = 0$ or $p(u,v) = p$,
$q(u,v) = q$. Note that if all edges are safe, i.e. $E = \calS$, then
$(p, q)$-flex-connectivity is the same as $p$-connectivity, and if all
edges are unsafe, i.e. $E = \calU$, then $(p, q)$-flex-connectivity is
the same as $(p+q)$-connectivity. Flex-SNDP thus generalizes EC-SNDP.

The work so far in flexible connectivity has been on two special
cases. The first is the spanning case, which requires
$(p,q)$-flex-connectivity for all pairs of vertices. This is the
$(p,q)$-FGC problem \cite{BoydCHI22}. The other is when
the requirement is for a single pair $(s,t)$
\cite{Adjiashvili13,AdjiashviliHMS20}.  Following \cite{ChekuriJ22} we
refer to this as $(p,q)$-Flex-ST. For $(p,q)$-FGC,  \cite{BoydCHI22}
obtained an $O(q \log n)$-approximation, and constant factor
approximations have been developed for small values of $p,q$
\cite{BoydCHI22,BansalCGI22,ChekuriJ22}. For $(p,q)$-Flex-ST,
the only non-trivial approximations known are for $(1,q)$-Flex-ST and
$(p,1)$-Flex-ST \cite{AdjiashviliHMS20} and $(2,2)$-Flex-ST
\cite{ChekuriJ22}. In fact, no non-trivial approximation is known even for
$(3,2)$-Flex-ST or $(2,3)$-Flex-ST. No non-trivial result is known for
$(2,2)$-Flex-SNDP. We refer the reader to \cite{BoydCHI22,ChekuriJ22,BansalCGI22} for
a more detailed description of existing results.


Adjiashvili et al.\ \cite{AdjiashviliHMS20} show that when $p$ is part of the input and
large, $(p,1)$-Flex-ST is NP-Hard to approximate to almost polynomial
factors. Thus $(p,q)$-Flex-SNDP is a harder problem than EC-SNDP. In
our earlier paper \cite{ChekuriJ22} we raised the following
question: does $(p,q)$-Flex-SNDP admit an approximation ratio of the
form $f(p,q)$ or $f(p,q)\text{polylog}(n)$ for for all fixed $p,q$
where $f$ is some integer valued function? \cite{ChekuriJ22}
formulated an LP relaxation which can be solved in $n^{O(q)}$-time and
a corresponding question on its integrality gap was also implicitly
raised. The known techniques for EC-SNDP and related problems rely
on the requirement function satisfying structural
properties such as skew-supermodularity and uncrossability. These properties are crucial in
primal-dual and iterated rounding based algorithms \cite{GoemansGPSTW94,Jain01}. Recent
work on flexible connectivity network design
\cite{BoydCHI22,BansalCGI22,ChekuriJ22} extended some of these ideas
in non-trivial and interesting ways to the special cases that we
mentioned. However, the requirement function for flexible connectivity
is not as well-behaved (see \cite{ChekuriJ22} for some examples) and
it seems challenging to obtain any non-trivial algorithm for say
$(2,2)$-Flex-SNDP or $(3,2)$-Flex-ST.

\paragraph{Contribution:} In this paper we take a substantially different approach for
$(p,q)$-Flex-SNDP, motivated by a very recent work of Chen,
Laekhanukit, Liao, and Zhang \cite{ChenLLZ22}. They developed a new
algorithmic approach for survivable network design to tackle a
generalization of EC-SNDP to the group/set connectivity setting.
We use their framework to obtain the following theorem.

\begin{theorem}
\label{thm:main}
There is a randomized algorithm that yields an $O(q(p+q)^3 \log^7
n)$-approximation for $(p,q)$-Flex-SNDP and runs in expected
$n^{O(q)}$-time. The approximation is based on an LP relaxation for the problem.
\end{theorem}

\begin{remark}
  The algorithm for $(p,q)$-Flex-SNDP easily extends to the setting where
  the maximum connectivity requirement is dominated by $(p,q)$. 
\end{remark}

The preceding theorem sheds light on the approximability of the
problem --- as discussed, this has been challenging via past
techniques.  It suggests that there may be an $f(p,q)$-approximation
for $(p,q)$-Flex-SNDP via the LP relaxation. It also showcases
the generality of the approach in \cite{ChenLLZ22} which
is likely to have further impact in network design. For instance,
the algorithm and analysis extend to the Set Connectivity version of
flexible connectivity problem.


\subsection{Technical Overview}
We give a brief technical overview of the algorithm. We follow an
augmentation approach for $(p,q)$-Flex-SNDP following recent work
\cite{BoydCHI22,BansalCGI22,ChekuriJ22}. The idea is to start with a
subgraph $H_0$ that satisfies $(p,0)$-flex-connectivity for the given
instance (which can be solved via reduction to EC-SNDP) and
iteratively increase, in $q$ stages, to obtain a subgraph $H_q$ that
satisfies $(p,q)$-flex-connectivity. In stage $\ell$ the goal is to go
from $(p,\ell)$-flex-connectivity to
$(p,\ell+1)$-flex-connectivity. Call a set $S \subset V$ deficient in
stage $\ell$ if it separates some terminal pair and has the following
property: $|\delta_{H_{\ell}}(S)| = p+\ell$ and $|\delta_{H_{\ell}}(S)
\cap \calS| < p$ (has less than $p$ safe edges).  It is necessary and
sufficient to cover all the deficient cuts by any edge in
$E(G)-E(H_\ell)$ to increase to $(p,\ell+1)$-flex-connectivity. Thus,
the augmentation problem can ignore the distinction between safe and
unsafe edges.  The family of deficient cuts in this augmentation
problem, unfortunately, does not have nice properties such as
uncrossability except in some special cases. Instead, we rely on the
framework of \cite{ChenLLZ22} --- we start with a fractional solution
to a natural cut covering relaxation that can be solved in $n^{O(q)}$
time, and round it via their approach. The algorithm in
\cite{ChenLLZ22} consists of three main ingredients. The first is to
use the fractional solution $x$ to define a capacitated graph $G'$ in
a clever way. The second is to consider a probabilistic approximation
of $G'$ via capacitated trees that approximate the flow properties of
$G'$ as defined in the seminal work of \racke \cite{Racke08}. The
third is a dependent randomized rounding procedure on trees for the
group Steiner tree problem due to Garg, Konjevod and Ravi
\cite{GargKR98}; this rounding has been generalized to the Set
Connectivity problem that we will need \cite{ChekuriEGS11,CGL15}. The
overall algorithm of \cite{ChenLLZ22} is simple at a high-level. After
it sets up the graph $G'$, it repeatedly samples a tree from the
\racke tree distribution induced by $G'$, and does a randomized
oblivious Set Connectivity rounding on the sampled tree. We use the
same algorithm with relevant changes to the definition of capacities
of $G'$ that are tailored to flexible connectivity. The analysis in
\cite{ChenLLZ22} shows that the rounding procedure covers the
deficient cuts in only a polylogarithmic number of rounds, via a
clever argument. We adapt their analysis to show that it also works
for flexible connectivity.  An important comment is the following: the
algorithm in \cite{ChenLLZ22} was designed to address group/set
connectivity where the complexity comes from large groups (otherwise
it can be reduced to EC-SNDP). Even though we are solving a non-group
problem in $(p,q)$-Flex-SNDP, the algorithm and analysis shows that
ideas from set connectivity implicitly arise when one tries to connect
different components in the process of covering deficient cuts.


\paragraph{Organization:} Section~\ref{sec:prelim} sets up the relevant background on
the LP relaxation, \racke tree embeddings, group Steiner tree, Set Connectivity and
the tree rounding algorithm for them. Section~\ref{sec:algo} describes the rounding algorithm.
Section~\ref{sec:analysis} analyses the correctness and approximation ratio of the algorithm.
