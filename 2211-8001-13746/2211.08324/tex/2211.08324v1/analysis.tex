\section{Analysis}
\label{sec:analysis}
We will assume, following earlier discussion, that $\LG = \emptyset$
and focus on the case when the algorithm proceeds to the TreeRounding
step. Let $H$ denote the set of edges that satisfies
$(p,q)$-flex-connectivity for the given pairs. {\bf Augment-LP} is a
cut covering LP. Consider any violated cut $S$ with respect to $H$;
$S$ is violated because $S$ separates a pair $(s_i,t_i)$ and
$\delta_H(S)$ has exactly $(p+q)$ edges, of which at most $p-1$ are
safe. Let $F= \delta_H(S)$. We call $F$ a violating edge set. There
are at most $\binom {|H|} {p+q}$ violating edge sets, and since $|H|
\le n^2$, this is upper bounded by $O(n^{2(p+q)})$.
We say that a set of edges $H' \subseteq E \setminus H$ is a feasible
augmentation for violating edge set $F$ if for each pair $(s_i,t_i)$,
there is a path from $s_i$ to $t_i$ in the graph $(H\cup H')\setminus
F$. The following is a simple observation.

\begin{claim}
$H' \subseteq E \setminus H$ is a feasible solution to the
  augmentation problem iff for each violating edge set $F$, $H'$ is a
  feasible augmentation for $F$.
\end{claim}


The preceding observations allows us to focus on a fixed violating
edge set $F$, and ensuring that the algorithm outputs a set $H'$ that
is a feasible augmentation for $F$ with high probability. We observe
that the algorithm is oblivious to $F$. Thus, if we obtain a high
probability bound for a fixed $F$, since there are $O(n^{2(p+q)})$
violating edge sets, we can use the union bound to argue that $H'$ is
feasible solution for \emph{all} violating edge sets.  For the
remainder of this section, until we do the final cost analysis, we
work with a fixed violating edge set $F$.

Consider a tree $(\calT,\calM, y)$ in the \racke distribution for the
graph $G$ with capacities $\tilde x$. We let $\calM^{-1}(F)$ denote
the set of all tree edges corresponding to edges in $F$,
i.e. $\calM^{-1}(F) = \cup_{e \in F} \calM^{-1}(e)$.  We call $(\calT,
\calM, y)$ \emph{good} with respect to $F$ if $y(\calM^{-1}(F)) \leq
\frac 1 2$; equivalently, $F$ blocks a flow of at most $\frac 1 2$ in
$\calT$.

\begin{lemma}
\label{lemma:good_tree}
For a violating edge set $F$, a randomly sampled R\"{a}cke tree $(\calT, \calM, y)$ is good with respect to $F$ with probability at least $\frac 1 2$.
\end{lemma}
\begin{proof}
For each $e \in F$, $\tilde x_e =\frac 1 {4(p+q)\beta}$. Since the expected congestion of each edge is at most $\beta$, $\E[\load(e)] \leq \beta \tilde x_e \le \frac 1 {4(p+q)}$ for each $e \in F$. Note that $y(\calM^{-1}(F)) = \sum_{e \in F} \load(e)$, hence by linearity of expectation, $\E[y(\calM^{-1}(F)] = \sum_{e \in F} \E[\load(e)] \leq |F|\frac 1 {4(p+q)} = \frac 1 4$. Applying Markov's inequality to $y(\calM^{-1}(F))$ proves the lemma.
\end{proof}

Given the preceding lemma, a natural approach is to sample a good tree $\calT$ and hope that $\calT \setminus M^{-1}(F)$ still has good flow between each terminal pair. However, since we rounded down all edges in $\LG \cup H$, it is possible that $\calM^{-1}(F)$ contains an edge whose removal would disconnect a terminal pair in $\calT$, even if $\calT$ is good. See \cite{ChenLLZ22} for a more detailed discussion and example.

We note that our goal is to find a set of
edges $H' \subseteq E$ such that each $s_i$ to $t_i$ has a path in
$(H' \cup H) \setminus F$; these paths must exist in the original
graph, even if they do not exist in the tree. Therefore, instead of
looking directly at paths between $s_i$ and $t_i$ in $\calT$, we focus
on obtaining paths through components that are already connected in
$(V(G), H \setminus F)$. The rest of the argument is to show that
sufficiently many iterations of TreeRounding on any good tree $\calT$
for $F$ will yield a feasible set $H'$ for $F$.

\subsection{Shattered Components, Set Connectivity and Rounding}
Let $\Q_F$ be the set of connected components in the subgraph induced by $H \setminus F$. We use vertex subsets to denote components.
Let $\calT$ be a good tree for $F$. We say that a connected component $Q \in \Q_F$ is \emph{shattered} if it is disconnected in $\calT \setminus \calM^{-1}(F)$, else we call it \emph{intact}. For each $i \in [k]$, let $Q_{s_i} \in \Q_F$ be the component containing $s_i$, and $Q_{t_i} \in \Q_F$ be the component containing $t_i$. Note that $Q_{s_i}$ may be the same as $Q_{t_i}$ for some $i$, but if $F$ is a violating edge set then there is at least one $i$ such that $Q_{s_i} \neq Q_{t_i}$. Now, we define a Set Connectivity instance that is induced by $F$ and $\calT$. Consider two disjoint vertex subsets $A,B \subset V$.
We say that $(A,B)$ partitions the set of shattered components if each shattered component $Q$ is fully contained in $A$ or fully contained in $B$. 
Formally let 
$$Z_F = \{(A \cup Q_{s_i}, B \cup Q_{t_i}): (A, B)
\text{ partitions the shattered components}, i \in [k]\}.$$ In other
words, $Z_F$ is set of all partitions of shattered components that
separate some pair $(s_i,t_i)$.  Since the leaves of $\calT$ are in
one to one correspondence with $V$ we can view $Z_F$ as inducing a Set
Connectivity instance in $\calT$; technically we need to consider the
pairs $\{(\calM^{-1}(A),\calM^{-1}(B)) \mid (A,B) \in Z_F\}$; however,
for simplicity we conflate the leaves of $\calT$ with
$V$.  We claim that it suffices to find a feasible solution that
connects the pairs defined by $Z_F$ in the tree $\calT$.

\begin{claim}
\label{claim:shattered_suffices}
Let $E' \subseteq \calT \setminus \calM^{-1}(F)$. Suppose there exists a path in $E' \subseteq \calT \setminus \calM^{-1}(F)$ connecting $A$ to $B$ for all $(A, B) \in Z_F$. Then, there is an $s_i$-$t_i$ path for each $i \in [k]$ in $(\calM(E') \cup H) \setminus F$. 
\end{claim}
\begin{proof}
Let $E' \subseteq \calT \setminus \calM^{-1}(F)$ such that there is a path from $A$ to $B$ in $E'$ for each $(A, B) \in Z_F$. Assume for the sake of contradiction that $\exists i \in [k]$ such that $(s_i, t_i)$ are disconnected in $(\calM(E') \cup H) \setminus F$. Then, there must be some cut $S$ such that $\delta_{(\calM(E') \cup H) \setminus F}(S) = \emptyset$ and $|S \cap \{s_i, t_i\}| = 1$.

We observe that no component $Q \in \Q_F$ can cross $S$ since each $Q$
is connected in $H\setminus F$. Assume without loss of generality that
$s_i \in S$. Then, let $A = Q_{s_i} \cup \{Q \in \Q_F: Q \text{ is
  shattered }, Q \subseteq S\}$, and $B = Q_{t_i} \cup \{Q \in \Q_F: Q
\text{ is shattered }, Q \subseteq \overline S\}$. Clearly, $A
\subseteq S$, $B \subseteq \overline S$. Furthermore, $(A, B) \in
Z_F$. By
assumption, there is a path $P$ in $E'$ between $A$ and $B$. Since $E'
\cap \calM^{-1}(F) = \emptyset$, $\calM(E')$ cannot contain any edges
in $F$. Therefore, $\calM(P)$ contains a path that crosses $S$ which
implies that $|\delta_{\calM(E')}(S)| = |\delta_{\calM(E') \setminus
  F}(S)| \geq 1$, contradicting the assumption on $S$.
\end{proof}

We now argue that $(\calT, \calM, y)$ routes sufficient flow for each pair in $Z_F$ without using the edges in $\calM^{-1}(F)$; in other words $y$ is fractional solution (modulo a scaling factor) to the Set Connectivity instance $Z_F$ in the graph/forest $\calT \setminus \calM^{-1}(F)$. We can then appeal to TreeRounding lemma to argue that it will connect the pairs in $Z_F$ without using any edges in $F$.

\begin{lemma}
\label{lem:flowforeachpair}
Let $(A, B) \in Z_F$. Let
$S \subset V_{\calT}$ such that $A \subseteq S$ and $B \subseteq V_{\calT} \setminus S$. Then $y(\delta_{\calT \setminus \calM^{-1}(F)}(S)) \geq \frac 1 {4(p+q)\beta}$.
\end{lemma}
\begin{proof}
Let $S$ be a vertex set of $\calT$ that separates $A$ from $B$. First,
suppose there exists a component $Q \in \Q_F$ such that $Q$ crosses
$S$, i.e. $S \cap Q \neq \emptyset$ and $\overline S \cap Q \neq
\emptyset$. Since $(A, B)$ partitions the set of shattered components,
$Q$ must be intact in $\calT$. Let $u$ be a leaf in $Q \cap S$ and $v$
be a leaf in $Q \cap \overline S$.  Since $Q$ is intact in $T$ the
unique path connecting $u$ to $v$ in $\calT$ crosses $S$ and let $e$
be an edge on this path that crosses $S$. It suffices to show that
$y(e) \ge \frac 1 {4(p+q)\beta}$. This follow from properties of the
\racke tree. Since $u$ and $v$ are connected in $G'$ with a path using
only edges in $\LG \cup H$ each of which has a capacity of $\frac 1
{4(p+q)\beta}$, $u$-$v$ maxflow in $G'$ is at least $\frac 1
{4(p+q)\beta}$. From Corollary~\ref{cor:racketreeflow},
for any tree $\calT$, the $u$-$v$
maxflow in $\calT$ with capacities $y$ must be at least $\frac 1
{4(p+q)\beta}$. This in particular implies that $y(e) \ge \frac 1
{4(p+q)\beta}$ for every edge $e$ on the unique path from $u$ to $v$
in $\calT$.

 We can now restrict attention to the case that no connected component
 of $\Q_F$ crosses $S$. Consider $S'$ be the set of leaves in $S$ and
 consider the cut $(S',V\setminus S')$ in $G$. It follows that
 $(S',V-S')$ partitions the connected components in $\Q_F$ and
 $\delta_{H-F}(S') = \emptyset$. Since $(A,B) \in Z_F$ there is a pair
 $(s_i,t_i)$ such that $Q_{s_i} \in S'$ and $Q_{t_i} \in V\setminus
 S'$. Thus $(S',V\setminus S')$ is a violated cut with $F$ as its
 witness. Since $x$ is a feasible solution to {\bf Augment-LP} it follows
 that $x(\delta_{E \setminus H}(S')) \ge 1$. Recall that we assumed
 that $\LG = \emptyset$, and hence all
 edges in $\delta_{E \setminus H)}(S')$ are in $\SM$.
 Therefore, $x(\delta_{E \setminus H}(S')) = \tilde x(\delta_{E \setminus H}(S')) \ge 1$.

 The \racke tree property guarantees that $y(\delta_{\calT}(S)) \ge \tilde x(\delta_{G'}(S')) \ge 1$ (via Corollary~\ref{cor:racketreeflow}). 
 We note that 
 $$y(\delta_{\calT \setminus \calM^{-1}(F)}(S)) \ge y(\delta_{\calT}(S)) - y(\calM^-(F)) \ge 1 - 1/2 \ge 1/2.$$
 where we used the fact that $y(\calM^-(F)) \le 1/2$ since $\calT$ is good for $F$.
 Thus in both cases we verify the desired bound.
\end{proof}


\paragraph{Bounding $Z_F$:} A second crucial property is a bound on $|Z_F|$,
the number of pairs in the Set Connectivity instance induced by $F$ and a good tree $\calT$ for $F$. 

\begin{lemma}
\label{lem:boundnumpairs}
For a good tree $\calT$, $|Z_F| \leq 2^{2(p+q)\beta} k$.
\end{lemma}
\begin{proof}
Let $\ell$ be the number of shattered components and let them be
$Q_1,\ldots,Q_\ell$. For each $Q_i$ pick a pair of vertices $u_i,v_i$
that are in separate components of $\calT - \calM^{-1}(F)$. Let $A =
\{u_1,\ldots,u_\ell\}$ and $B = \{v_1,v_2,\ldots,v_\ell\}$. Since the
paths connecting $u_i,v_i$ are in different connected components of
$H\setminus F$, it follows that the $(A,B)$-maxflow in $H \setminus F$
is at least $\ell$. In the graph $G'$ obtained by scaling down the
capacity of edges of $H$, the maxflow is at least $\frac \ell
{4(p+q)\beta}$ which implies that it is at least this quantity in
$\calT$. Since $\calT$ is good, the total decrease of flow can be at
most $y(\calM^{-1}(F)) \le \frac 1 2$. By construction there is no
flow between $A$ and $B$ in $\calT - \calM^{-1}(F)$ which implies that
$\frac \ell {4(p+q)\beta} \le 1/2 \Rightarrow \ell \leq 2(p+q)\beta$.
Each pair in $Z_F$ corresponds to a subset of shattered components and
a demand pair $(s_i,t_i)$, and hence $|Z_F|\leq 2^\ell k \le
2^{2(p+q)\beta} k$.
\end{proof}

\subsection{Correctness and Cost}
Now we analyze the correctness and cost of the algorithms output.

\begin{lemma}
\label{claim:successprobforgoodtree}
Suppose $\calT$ is good for a violating edge set $F$. Then after $t$
rounds of TreeRounding with flow parameter $\frac 1 {4(p+q)\beta}$,
the probability that $H'$ is \emph{not} a feasible augmentation for
$F$ is at most $(1-\phi)^t |Z_F| \le 1/4$.
\end{lemma}
\begin{proof}
  Suppose $\calT$ is good for $F$. Let $(A,B) \in Z_F$. From
  Lemma~\ref{lem:flowforeachpair} the flow for $(A,B)$ in $\calT -
  \calM^{-1}(F)$ is at least $\frac 1 {4(p+q)\beta}$. From
  Lemma~\ref{lem:setconnectivity-tree-rounding}, with probability at
  least $\phi$, the pair $(A,B)$ is connected via a path in $\calT -
  \calM^{-1}(F)$. If all pairs are connected, then via
  Claim~\ref{claim:shattered_suffices}, $H'$ is a feasible
  augmentation for $F$. Thus, $H'$ is not a feasible augmentation if
  for some $(A,B) \in Z_F$ the TreeRounding does not succeed after $t$
  rounds. The probability of this, via the union bound over the pairs
  in $Z_F$, is at most $(1-\phi)^t |Z_F|$. From
  Lemma~\ref{lem:boundnumpairs}, $|Z_F| \le 2^{2(p+q)\beta}k$. Consider $t
  = \frac 1 \phi \log(4k \cdot 2^{2\beta(p+q)}) = O((p+q)\log n)$, since $\beta = O(\log n)$. Then, $(1-\phi)^t |Z_F| = 2^{2(p+q)\beta} k (1 -
  \phi)^t \leq 2^{2(p+q)\beta} k e^{-\phi t} \leq \frac 1 4$.
\end{proof}

\begin{lemma}
\label{lem:correctness}
The algorithm outputs a solution $H'$ such that $H \cup H'$ is a
feasible augmentation to the given instance with probability at least
$\frac 1 2$.
\end{lemma}
\begin{proof}
For a fixed $F$ the probability that a sampled tree is good is at
least $1/2$.  By Claim~\ref{claim:successprobforgoodtree}, conditioned
on the sampled tree being good for $F$, $t$ iterations of TreeRounding
fail to augment $F$ with probability at most $1/4$. Thus the probability that
all $t'$ iterations of sampling trees fail is $(1-3/8)^{t'}$. There
are at most $n^{2(p+q)}$ violating edge sets $F$. Consider $t' = \frac 8 3
\log (2n^{2(p+q)}) = O((p+q)\log n)$. By applying the union bound over
all violating edge sets $F$, the probability of the algorithm failing
is at most $n^{2(p+q)}(1 - 3/8)^{t'} \leq n^{2(p+q)}e^{-3t'/8} \leq
\frac 1 2$. Therefore, the output of the algorithm is a feasible
augmentation for all violating edge sets with probability at least
$\frac 1 2$.
\end{proof}


Now we analyze the expected cost of the edges output by the algorithm for augmentation
with respect to $\lpopt$, the cost of the fractional solution.

\begin{lemma}
\label{lem:costanalysis}
The total expected cost of the algorithm is $O((p+q)^3\log^7 n) \cdot \lpopt$.
\end{lemma}
\begin{proof}
Fix an edge $e \in \SM$ with fractional value $x_e$. Consider one
outer iteration of the algorithm in which it picks a random tree
$\calT$ from the \racke tree distribution and then runs $t$ iterations
of TreeRounding with flow parameter $\alpha = \frac 1
{4(p+q)\beta}$. Via Lemma~\ref{lem:setconnectivity-tree-rounding}, the
probability of an edge $f \in \calT$ being chosen is at most
$O(\frac{1}{\alpha} h \log^2n) y(f)$. Thus the expected cost for $e$
for one round of TreeRounding is $O(\frac{1}{\alpha} h \log^2n)
\sum_{f \in \calM^{-1}(e)} y(f) = O(\frac{1}{\alpha} h \log^2n)
\load(e)$. By the \racke distribution property, $\E_{\calT} [\load(e)]
\le \beta x_e$. By linearity of expectation, since there are a total
of $t \cdot t'$ iterations of TreeRounding, the total expected cost is
at most ($t \cdot t')\cdot O(\frac{1}{\alpha} h \log^2 n \beta) \sum_{e \in
  E} c(e) x_e$. By the analysis in Section~\ref{sec:algo}, $h = O(\log
n)$, and $\beta = O(\log n)$. Substituting in the values of $t$ and $t'$ stated
in Lemmas \ref{claim:successprobforgoodtree} and \ref{lem:correctness}, the total expected cost is at most $O((p+q)^3 \log^7 n) \cdot \lpopt$.
\end{proof}


Combining the correctness and cost analysis we obtain the following.
\begin{theorem}
  There is a randomized $O((p+q)^3\log^7 n)$ approximation for the augmentation problem
  via {\bf Augment-LP}.
\end{theorem}

Starting with a solution for $(p,0)$-flex-connectivity, and using $q$ augmentation iterations,
we obtain an $O(q(p+q)^3\log^7 n)$-approximate solution for the given instance of
$(p,q)$-Flex-SNDP, proving
Theorem~\ref{thm:main}.
