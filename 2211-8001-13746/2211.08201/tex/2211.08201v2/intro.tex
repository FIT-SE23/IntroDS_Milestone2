 
% E-commerce and logistic needs
In the recent years, the trade of goods over the internet, commonly referred as \emph{e-commerce} has consistently increased its influence, quickly taking over physical commerce thanks to its convenience to the consumer and large variety of options available. Companies involved in this business model have experience a massive growth, but must also deal with a large number of challenges in order to offer their services.

% Fulfillment centers
Fulfillment centers are the most critical part of the logistical infrastructures, massive warehouses tasked with the storage and handling of a large number of goods at the same time. The efforts to automatize these centers must deal with the challenge of coordinating hundreds of robots to transport items inside the warehouse efficiently.

    % Literature
    
% Basic Pathfinding
As computers became widely available to researches in the second half of the past century, complex problems such as the Shortest Path Problem (SPP) were approached with considerable success. Algorithms such as Bellman-Ford (\cite{bellman_routing_1958}), Dijkstra (\cite{dijkstra_note_1959}) and A* (\cite{hart_formal_1968}) exploited different properties of the problem to minimize the computational cost of solving the SPP.

% Multiagent Pathfinding
However, these methods only tackle single agent problems, limiting the range of applications. More recent effort have build upon previous methods to tackle several agents, algorithms such as Cooperative A* (\cite{silver_cooperative_2005}), M* (\cite{wagner_m_2011}) and Conflict Based Search (\cite{sharon_conflict-based_2015}) offer different possibilites in the tradeoff between computational cost and optimality for `multiagent pathfinding' (MAPF).

% Lifelong delivery
These algorithms are still unsuitable to deal with the environment of a common warehouse, as they only deal with one-shot problems, where each agent only has to move from one position to another. For a real environment, a robot is expected to move continuously, picking and delivering items. Such issue has been addressed recently (\cite{ma_lifelong_2017}, \cite{li_lifelong_2021})...

% Reinforcement Learning Intro
In parallel, the development of Reinforcement Learning (RL) has brought a collection of tools and frameworks to deal with complex problems. The generality of the framework provided by RL has encouraged its application in many different fields...
\cite{bertsekas2019reinforcement}

% Multiagent RL
Recent efforts in the RL learning fields have focused on tackling problems where several agents are involved. These problems deal with a control space that increases exponentially with the number of agents, rendering many problems of interest unfeasible for even powerful computation centers.

% Rollout applied to multiagent RL
Among the approaches developed to deal with this issue, a considerable success has been achieved with the use of multiagent rollout (\cite{bertsekas2021rollout}). By dealing with one agent at a time, the control space is dramatically decreases... 