\documentclass{article}

% if you need to pass options to natbib, use, e.g.:
     \PassOptionsToPackage{numbers, compress}{natbib}
% before loading tackling_climate_workshop_style

% ready for submission
 \usepackage[final]{tackling_climate_workshop_style}

% to compile a preprint version, e.g., for submission to arXiv, add add the
% [preprint] option:
%     \usepackage[preprint]{tackling_climate_workshop_style}

% to compile a camera-ready version, add the [final] option, e.g.:
%     \usepackage[final]{tackling_climate_workshop_style}

% to avoid loading the natbib package, add option nonatbib:
%     \usepackage[nonatbib]{tackling_climate_workshop_style}

\usepackage[utf8]{inputenc} % allow utf-8 input
\usepackage[T1]{fontenc}    % use 8-bit T1 fonts
\usepackage{hyperref}       % hyperlinks
\usepackage{url}            % simple URL typesetting
\usepackage{booktabs}       % professional-quality tables
\usepackage{amsfonts}       % blackboard math symbols
\usepackage{nicefrac}       % compact symbols for 1/2, etc.
\usepackage{microtype}      % microtypography
\usepackage[pdftex]{graphicx}
\usepackage{xcolor} % colors in text only
\usepackage{caption}
\usepackage{subcaption}
\usepackage{wrapfig}
\usepackage{comment}
\usepackage[symbol]{footmisc}
\renewcommand{\thefootnote}{\fnsymbol{footnote}}


\bibliographystyle{unsrtnat}

\title{Identification of medical devices using machine learning on distribution feeder data for informing power outage response}

% The \author macro works with any number of authors. There are two commands
% used to separate the names and addresses of multiple authors: \And and \AND.
%
% Using \And between authors leaves it to LaTeX to determine where to break the
% lines. Using \AND forces a line break at that point. So, if LaTeX puts 3 of 4
% authors names on the first line, and the last on the second line, try using
% \AND instead of \And before the third author name.

\author{
  Paraskevi Kourtza* \\
  University of Edinburgh \\
  Edinburgh, UK \\
  \texttt{s1265437@ed.ac.uk} \\
  \And
  Maitreyee Marathe* \\
  University of Wisconsin-Madison \\
  Madison, USA \\
  \texttt{mmarathe@wisc.edu} \\
  \And 
  Anuj Shetty* \\
  Stanford University\\
  Stanford, USA \\
  \texttt{anuj42@stanford.edu} \\
  \And
  Diego Kiedanski \\
  Yale University \\
  New Haven, USA \\
  \texttt{diego.kiedanski@yale.edu} \\
}

\begin{document}

\maketitle

\footnotetext{* These authors contributed equally.}

\begin{abstract}

Power outages caused by extreme weather events due to climate change have doubled in the United States in the last two decades. Outages pose severe health risks to over 4.4 million individuals dependent on in-home medical devices. Data on the number of such individuals residing in a given area is limited. This study proposes a load disaggregation model to predict the number of medical devices behind an electric distribution feeder. This data can be used to inform planning and response. The proposed solution serves as a measure for climate change adaptation.

\end{abstract}

\section{Problem and motivation}

Over 4.4 million people in the U.S. rely on electricity-dependent in-home medical devices and services \citep{noauthor_hhs_nodate}. Extreme weather events that cause power outages can lead to increased mortality rates \citep{kishore_mortality_2018, issa_deaths_2018} in such individuals or to additional stress for hospitals, shelters, and emergency services \citep{higgs_power_2009, nakayama_effect_2014, greenwald_emergency_2004}. Unfortunately, there have been several disaster emergencies that corroborate this (see Appendix \ref{appendix a-extreme events} for examples). Energy resilience for home healthcare is a largely unexplored problem and needs proactive approaches to meet the needs of this vulnerable community \citep{marathe_energy_homehealthcare}.

Climate change is driving more intense and frequent extreme weather events, which put pressure on an aging power grid infrastructure. Power outages tied to extreme weather have doubled across the U.S. in the last two decades, and the frequency and length of outages are at their highest \citep{brown_storms_2022}. Also, the lack of comprehensive data on medically fragile individuals hinders effective planning and response.

We propose performing load disaggregation on electric distribution feeder data to identify the number of in-home medical devices downstream of a feeder. Our solution does not depend on precise location or number of individuals, therefore respecting personal privacy. We also propose creating a dataset of power data measurements for in-home medical devices. Our method can be used to estimate the number of medically fragile individuals in an area and improve preparedness for power outages.

%It is necessary to have a comprehensive dataset of the number of such medically fragile individuals and their locations for effective resource planning and response to power outages and extreme weather events. We propose load disaggregation on electric distribution feeder data to identify the number of in-home medical devices present downstream of the feeder. This will not reveal the precise location or number of individuals, protecting their privacy, but will offer an estimate of the number of medical devices present in the area served by the feeder. This data can be useful during extreme weather event induced power outages and other emergencies, and is an approach for climate change adaptation.

\section{Related approaches and datasets}

Traditional methods to find the number and locations of such individuals rely on private patient data, either on lists by healthcare providers or electric utility medical baseline programs (e.g., from PG\&E \citep{noauthor_medical_nodate}), where customers can self-register if they use in-home medical devices.  However, this is relevant only for countries with centralized healthcare and it can lead to incomplete and inaccurate data, as shown by a 2018 survey on New York City residents \citep{dominianni_power_2018}.

The HHS emPOWER map \citep{noauthor_hhs_nodate} has improved on utility baseline programs by using Medicare insurance data to estimate the number of people who rely on in-home medical devices at zip code level. The map covers Medicare beneficiaries but it is estimated that millions more rely on electricity-dependent medical technology in the U.S., including over 180,000 children \citep{shapiro_home_2019}. The proposed study aims to build upon and expand the emPOWER map.


We found only one method in the literature that uses energy consumption data to address a facet of this problem. \citet{bean_keeping_2020} assume prior knowledge of the houses with in-home medical devices, and use smart meter voltage data to identify their supply phase. Our approach is based on energy disaggregation at the feeder level. Load disaggregation has been reviewed in \citep{angelis_nilm_2022, kelly_neural_2015} and most related work addresses disaggregation at the household level. \citet{ledva_real-time_2018} perform feeder-level disaggregation, using real-time feeder values to separate the demand of air conditioners and of other loads. Such non-intrusive load monitoring (NILM) techniques have also recently grown in popularity across different use cases, such as estimating solar PV generation for a given feeder \citep{vrettos_pv_2019}. However, these methods rely on real-time feedback for forecasting, which does not apply to our problem. Further discussion on related approaches can be found in Appendix \ref{appendix:related-approaches}


%\citet{bean_keeping_2020} assume prior knowledge of which houses have in-home medical devices and use smart meter voltage data at 15-minute resolution. They use an unsupervised learning algorithm to cluster houses with correlated voltage time series and identify the phase houses with medical devices are on. With this information, network operators can minimize the likelihood of loss of power to in-home medical device users

\section{Proposed methodology}
The input for the proposed model is power data from an electric distribution system feeder. As shown in Figure  \ref{fig:distribution_system}, a distribution substation has many feeders, and each feeder has multiple customers downstream. The predictor of the model is power data measured at the feeder, and the predictand is the number of medical devices present downstream of that feeder.

\begin{wrapfigure}{r}{0.45\textwidth}
  \begin{center}
  \vspace{-8mm}
  \includegraphics[width = 0.4\textwidth]{figures/distribution-feeder.jpg}
  \caption{Distribution system}
  \vspace{-3mm}
  \label{fig:distribution_system}
  \end{center}
\end{wrapfigure}


We propose partnering with 10-20 households using in-home medical devices and gaining consent to monitor their device usage. This range is large enough for a proof of concept, while being small enough to be feasible to locate within a feeder and onboard for our data collection. However, this may change at the time of implementation. Monitoring would be through smart meters at the medical device level, with a sampling rate of the order of 1Hz. These measurements would be used mainly to record whether each device was on or off for each timestamp, given some threshold. We will get a time series of the number of such devices running at a time to be used as ground truth data for training, and this would not be required at the time of model inference. The set of in-home medical devices is vast \citep{unitedhealthcare}, and we propose starting with a single device as a proof of concept. We propose choosing the in-home ventilator \citep{vyairemedical} as the device of interest, since it is used for pediatric patients as well as older adults. \citep{King921}.

\begin{figure}[htbp]
  \centering
  \vspace{-5mm}
  \includegraphics[width=0.7\linewidth]{figures/Dataset_generation.png}
  %\fbox{\rule[-.5cm]{0cm}{4cm} \rule[-.5cm]{4cm}{0cm}}
  \caption{Dataset generation}
  \label{fig:dataset_generation}
\end{figure}

As shown in Figure \ref{fig:dataset_generation}, for the training input, we would also need to partner with the distribution company to sample the feeder current with an Analog to Digital card at a sampling rate of 10-20 kHz. We assume we can identify and monitor all medical devices of the selected type served by the feeder. This data collection would be carried out over the course of 1 week. This duration is chosen based on other standard high-frequency datasets like BLUED \citep{anderson2012blued}.

% Our model has a multivariate input consisting of 1) the power time series and 2) high-frequency features to help distinguish the medical device type.
Let $i_t, v_t$ denote the instantaneous current and voltage measurement\footnote{Either single or multiphase.}. To transform a time-series problem into a tabular format compatible with traditional neural architectures, we will consider a rolling window of size $W$ (e.g., 5 seconds): $I^W_t = [i_{t-W+1}, \dots, i_t]$ and $V^W_t = [v_{t-W+1}, \dots, v_t]$. 
The final dataset $X$ will be a collection of vectors $x_t$ where each $x_t$ is itself a collection of features obtained from $I^W_t$ and $V^W_t$, i.e.,

\begin{equation}
x_t = (f^1(I^W_t, V^W_t), \dots,  f^L(I^W_t, V^W_t)))
\label{eq:input}
\end{equation}

with $f^j\colon \mathbb{R}^{W} \times \mathbb{R}^{W} \to \mathbb{R}$. 
To obtain the feature functions $f^j$, we will test the medical device in the lab in all its different operational modes. We would then pick the most essential features to distinguish this device's signature from other device types found in the PLAID dataset \citep{medico_voltage_2020}, as used in \citep{marchesoni-acland_end--end_2020}. These high-frequency features include the form factor, the phase shift between voltage and current, etc., similar to \citep{marchesoni-acland_end--end_2020}.

% These would be calculated over rolling time windows and then fed into the convolution layers of the model, 

% \begin{figure}[h!]
%     \centering
%     \includegraphics[width=0.3\linewidth]{figures/VI_trajectory_example_for_different_devices1.png}
%     \hspace{0.1cm}
%     \includegraphics[width=0.3\linewidth]{figures/VI_trajectory_example_for_different_devices2.png}
%     \hspace{0.1cm}
%     \includegraphics[width=0.3\linewidth]{figures/VI_trajectory_example_for_different_devices3.png}
%     \caption{VI trajectory example for different devices \cite{marchesoni-acland_end--end_2020}}
%     \label{fig:VI_trajectory}
% \end{figure}



% \begin{equation}
% % x_t = (p_{0,0}, \dots, p_{0, M}, \dots, p_{M, N}, ff, \dots), \ , y_t = (?)
% % x_t = (p_t, f_{1,t}, \dots, f_{N,t} )
% X = (x_0, \dots, x_{T-1}), \ , x_t = (f^1(I_t, V_t), \dots,  f^L(I_t, V_t))), \ , I_t = [i_t, \dots, i_{t-w}], \ V_t = [v_t, \dots, v_{t-w}]
% \label{eq:input}
% \end{equation}
% \begin{equation}
% % f_{i,t} = f(i_{t-w/2}, v_{t-w/2}, \dots, i_{t+w/2}, v_{t+w/2})
% \label{eq:feature}
% \end{equation}
% \begin{equation}
% y_t = \sum\limits_{j=1}^{N}d_{j,t}
% \label{eq:output}
% \end{equation}

% The input to the model $x_t$ is given by equation \ref{eq:input} where {\color{blue}$p_t$ is ?} and $f_{i,t}$ is a high frequency feature calculated over a window $w$ around time $t$ as given in equation \ref{eq:feature}. 

For every vector $x_t$, its corresponding target $y_t$ will correspond to the number of medical devices running within the time window $[t-W+1, t]$.

% The output $y_t$ is given by equation \ref{eq:output}, where $N$ is the number of houses and $d_{j,t}$ is the number of medical devices running in house $j$ at time $t$. 
The proposed data pipeline and model is shown in Figure \ref{fig:model_block_diagram}, and is based on the neural network model used in \citep{marchesoni-acland_end--end_2020}. We use the Mean Absolute Error metric $
MAE = \frac{1}{N} \sum\limits_t | \hat{y_t} - y_t|$

\begin{figure}[h!]
  \centering
  \vspace{-5mm}
  \includegraphics[width=0.6\linewidth]{figures/Model_block_diagram.png}
  %\fbox{\rule[-.5cm]{0cm}{4cm} \rule[-.5cm]{4cm}{0cm}}
  \caption{Proposed model block diagram}
  \vspace{-5mm}
  \label{fig:model_block_diagram}
\end{figure}

\begin{comment}
\section{Discussion and deployment}
{\color{blue} Technical limitations (can be put into the Challenges and Limitations appendix):}\\
The model will be trained to predict the number of devices belonging to a set of common in-home medical devices. Therefore, if none of the devices an individual owns belong to this set, the individual will not be accounted for. This will limit the transferability of this model to a case where devices other than those considered need to be identified. The presented approach aims at extending the load disaggregation framework for feeder-level data. Although feeder level disaggregation has been implemented, for instance to identify air conditioning loads \citep{ledva_real-time_2018},  it remains to be seen if this can be successfully implemented for medical devices, particularly if the power signatures of other devices are similar to the target medical devices.
\end{comment}

\section{Impact and conclusions}

The contributions of this study would be (1) a load disaggregation model to predict the number of medical devices downstream of a feeder, (2) a quantitative approach to estimate the medically fragile population in an area while preserving location privacy of such individuals, (3) a dataset of high frequency power data measurements for in-home medical devices. The main challenges and limitations of this approach are discussed in Appendix \ref{appendix:challenges-limitations}.

The estimated data on medical devices in an area can be used by multiple entities to inform planning and response to extreme weather events and power outages, thereby functioning as a vehicle for climate change adaptation. It can be used by electric utilities to plan public safety power shut-offs \citep{noauthor_utility_nodate}, the capacity and locations of community charging stations in outage-prone areas, and for priority restoration after an outage. Home healthcare agencies can use this data to plan for supplies such as oxygen tanks. Emergency management services can use the data for public outreach and distribution of emergency kits before an extreme-weather event. The resulting data from the study can be used to complement the existing emPOWER dataset, which has already proven useful in adaptation to extreme weather events \citep{noauthor_story_2021, noauthor_story_2021-1, noauthor_story_2021-2, noauthor_story_2021-3}. 

There are multiple stakeholders in this space, namely medically fragile households, home healthcare agencies and hospitals, durable medical equipment providers, electric utilities, and emergency management services. Partnerships with these stakeholders are critical for the successful formulation and implementation of the proposed solution. Energy resilience for home healthcare is a largely unexplored problem, and the proposed model aims to address gaps at this intersection. 



\section*{Acknowledgements}
The authors would like to thank the organizers of the Climate Change AI Summer School 2022 for facilitating the initial work for this proposal.


% Use unnumbered first-level headings for the acknowledgments. All acknowledgments
% go at the end of the paper before the list of references. Moreover, you are required to declare 
% funding (financial activities supporting the submitted work) and competing interests (related financial activities outside the submitted work). 
% More information about this disclosure can be found at: \url{https://neurips.cc/Conferences/2020/PaperInformation/FundingDisclosure}.

% Do {\bf not} include this section in the anonymized submission, only in the final paper. You can use the \texttt{ack} environment provided in the style file to automatically hide this section in the anonymized submission.



\medskip

\small

\bibliography{references}

\appendix
\section{Examples of extreme weather events and their impact} \label{appendix a-extreme events}
One-third of 4654 additional deaths in Puerto Rico during the three months following Hurricane Maria in 2017 can be attributed to health complications due to outage-related problems, such as failure of in-home medical devices \citep{kishore_mortality_2018}. Over 15\% of the deaths after Hurricane Irma in 2017 were due to worsening pre-existing medical conditions because of power outages \citep{issa_deaths_2018}. After Hurricane Gustav, 20\% to 40\% of the 1400 people who came into medical shelters depended on medical equipment \citep{higgs_power_2009}. After the 2011 earthquakes in Japan, there was an influx of medical device-dependent individuals into hospitals \citep{nakayama_effect_2014}. In the 24 hours after the 2003 blackouts in New York, 22\% of hospital admits were people relying on in-home medical devices \citep{greenwald_emergency_2004}. 

\section{More on related approaches}
\label{appendix:related-approaches}
\citet{pandey_structured_2019} perform energy disaggregation at the household level. More specifically, they build upon Powerlet-based energy disaggregation (PED) \citep{elhamifar_energy_2015}, which “\textit{captures the different power consumption patterns of each appliance as representatives (used as dictionary atoms) and then estimates a combination of these representatives that best approximates the observed aggregated power consumption.}” To overcome the limitation of co-occurrence in PED, they use the information on operation modes of the different devices and distinguish between similar co-occurring devices. Devices are divided into subsets, forming a binary tree, until only one device is left. Decomposing the home energy consumption becomes a recursive task, starting from the root node. However, disaggregating every other device at the feeder level would be inefficient for our problem, as we only want to account for medical devices.

\citet{ledva_real-time_2018} perform feeder-level disaggregation, using real-time feeder measurements to separate the demand of air conditioners (AC) and the demand of other loads (OL). Their implementation is based on an online learning model, Dynamic Fixed Share (DFS), and models created from historic building-level and device-level data. The historical data have a sampling frequency of one minute and are constructed using Pecan St Dataport data \citep{noauthor_dataport_nodate} for residential appliances and data from two buildings in California. Device-level demand estimates can also be obtained with Non-Intrusive Load Monitoring (NILM) algorithms. The DFS model uses weighted predictions from a bank of models to separately predict AC and OL demand. These two predicted demands can be added and compared to the true feeder demand, allowing for real-time improvements to the weighted predictions.

\citet{marchesoni-acland_end--end_2020} construct an end-to-end NILM system using high frequency data and neural networks. They use the PLAID dataset \citep{medico_voltage_2020} to select the most important high-frequency features, and train artificial neural networks (ANNs) on the UK-DALE \citep{kelly_uk-dale_2015} dataset to disaggregate the household-level signal. They constructed a high-frequency meter to be attached at the household level for 2 houses in Uruguay and smart meters at the device level, sampling every 1 minute, for evaluation. This approach and the successful results seem the most relevant to our problem. The ‘rectangles’ network in the paper predicts the beginning and end of the appliance  activation and the power consumed. We instead predict the number of running medical devices of a fixed type rather than disaggregating the exact signal.

\citet{bucci_cnn_2021} focus on detecting and classifying change of state events for different appliances from the aggregate current signal. They decompose the derivative of the RMS current using the Short Term Fourier Transform, which allows identification of each event based on its spectral information.

\section{Challenges and limitations}
\label{appendix:challenges-limitations}
The model predicts the number of medical devices downstream of a feeder. If this has to be mapped to the number of medically fragile individuals, further work in determining the number of devices per person is necessary, which may depend on data such as underlying conditions, geographical location, and income level of the neighborhood. Additionally, some common non-medical devices can be medically critical for some individuals. For example, for a diabetic person, the refrigerator is a critical medical device since it is necessary to store insulin.

The model will be trained to predict the number of devices belonging to a set of common in-home medical devices. Therefore, if none of the devices an individual owns belong to this set, the individual will not be accounted for. This will limit the transferability of this model to a case where devices other than those considered need to be identified.

The presented approach aims at extending the load disaggregation framework for feeder-level data. Although feeder level disaggregation has been implemented, for instance to identify air conditioning loads \citep{ledva_real-time_2018},  it remains to be seen if this can be successfully implemented for medical devices, particularly if the power signatures of other devices are similar to the target medical devices.

\begin{comment}

\section{TEMP: Deployment, Discussion, Impact}
The proposed model will predict the number of medical devices downstream of a feeder. This data can be used by multiple entities. It can be used by electric utilities to inform their power outage response and planning. Utilities must de-energize certain power lines for maintenance or mitigate wildfire risk like the public safety power shut-offs in California \cite{noauthor_utility_nodate}. They use different metrics to decide which lines to de-energize, and data on medically fragile individuals can be one of the metrics used to inform their decision. This data can also be used by utilities to plan the capacity and locations of community charging stations in affected areas. After a power outage, utilities can use data from the proposed model to prioritize restoration of areas with medically fragile households.

Home healthcare and durable medical equipment providers can use this data to plan for supplies such as oxygen tanks, marine batteries, and trucks. Emergency management services can use the data for public outreach before an expected extreme-weather event and to distribute emergency kits.

The resulting database from the study can be used to add to the existing emPOWER dataset. The emPOWER dataset has already proven to be useful in adaptation to different extreme weather events such as wildfires in Los Angeles County, California \citep{noauthor_story_2021}, earthquakes in Puerto Rico \citep{noauthor_story_2021-1}, flooding and windstorms in Broome County, New York \citep{noauthor_story_2021-2}, and multiple threats in Arizona \citep{noauthor_story_2021-3}.


It is important to note some of the limitations of the presented approach. The model predicts the number of medical devices downstream of a feeder. If this has to be mapped to the number of medically fragile individuals, further work in determining the number of devices a person is likely to use is necessary, which may depend on data such as underlying conditions, geographical location, and income level of the neighborhood. The model will be trained to predict the number of devices belonging to a set of common in-home medical devices. So if none of the devices an individual owns is dependent belong in this subset, the individual will not be accounted for. This will limit the transferability of this model to a case where devices other than those considered need to be identified.

Additionally, some common non-medical devices can be medically critical for some individuals. For example, for a diabetic person, the refrigerator is a critical medical device since it is necessary to store insulin. The presented approach aims at extending the load disaggregation framework for feeder-level data. Although feeder level disaggregation has been implemented, for instance, to identify air conditioning loads \cite{ledva_real-time_2018},  it remains to be seen if this can be successfully implemented for medical devices, particularly if the power signatures of other devices are similar to the target medical devices.

There are multiple stakeholders in this space, and partnerships are critical not just for generating training data and model evaluation but also during model formulation to ensure that the model generates and presents data in the most useful form. Stakeholders and partners include medically fragile households, home healthcare agencies and hospitals, durable medical equipment manufacturers and providers, electric utilities, emergency management services, and city and state agencies. Energy resilience for home healthcare is a largely unexplored problem, and the proposed model aims to address gaps in this intersection. 


\end{comment}


\end{document}