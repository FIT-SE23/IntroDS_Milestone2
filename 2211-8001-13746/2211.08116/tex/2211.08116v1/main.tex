\errorcontextlines999



\documentclass[sigconf]{acmart}
\settopmatter{printacmref=false}

\usepackage[utf8]{inputenc}
\usepackage{url}
\usepackage{multirow}
\usepackage{nicefrac}
\usepackage{mathell}
\usepackage{tilde}% modifies tilde in mathmode, use ~sin like \sin with any letter sequence
\usepackage{algorithm}
\usepackage{algpseudocode}
\usepackage{subcaption}
\usepackage{graphicx}

\let\xxx\epsilon\let\epsilon\varepsilon\let\varepsilon\xxx
\let\xxx\phi\let\phi\varphi\let\varphi\xxx

\def\hat#1{#1'}
\let\Tilde\widetilde
\newcommand{\mpara}[1]{\par\noindent{\bf #1\@.}}






\title{W-Trace: Robust and Effective Watermarking for GPS Trajectories}
\setcopyright{none}
\author[Rajjat Dadwal, Thorben Funke, Michael Nüsken, Elena Demidova]{Rajjat Dadwal$^{1}$, Thorben Funke$^{1}$, Michael Nüsken$^{2}$, Elena Demidova$^{3}$}
\affiliation{%
  \institution{
  $^{1}$L3S Research Center, Leibniz University Hannover, Hannover\country{Germany} \\
  $^{2}$Bonn-Aachen International Center for Information Technology, Bonn\country{Germany}\\
  $^{3}$Data Science and Intelligent Systems Group (DSIS), University of Bonn, Bonn\country{Germany}
  }
  }
\email{dadwal@L3S.de, tfunke@L3S.de, nuesken@bit.uni-bonn.de, elena.demidova@cs.uni-bonn.de}

\newcommand\blfootnote[1]{%
  \begingroup
  \renewcommand\thefootnote{}\footnote{#1}%
  \addtocounter{footnote}{-1}%
  \endgroup
}

\newcommand{\approach}{\textit{W-Trace}}
\newcommand{\nnumber}{\mathbb{N}}
\newcommand{\real}{\mathbb{R}}
\newcommand{\trajectoriesData}{\mathbf{D}}
\newcommand{\Trajectory}{T}
\newcommand{\subTrajectory}{t}
\renewcommand\footnotetextcopyrightpermission[1]{}
\AtBeginDocument{
  \providecommand\BibTeX{{%
    \normalfont B\kern-0.5em{\scshape i\kern-0.25em b}\kern-0.8em\TeX}}}

\setcopyright{None}

\hyphenation{ana-ly-tics po-pu-la-ri-ty know-led-ge re-fe-ren-ce fle-xib-le se-cond he-te-ro-ge-neous se-ve-ral existen-ce fa-ci-li-tate has-Be-gin-Time-Stamp his-to-ri-cal con-tained-in-Place cha-ra-cte-ris-tics pro-per-ty ori-gi-na-te de-ve-lo-ped re-le-van-ce ap-proach-es ap-proach-es
pro-per-ties ma-nual-ly ex-pe-ri-men-tal ge-ne-ra-ted mo-de-led mi-ni-mum ma-xi-mum ge-ne-ra-te ge-ne-ra-ting ge-ne-ra-ted ge-ne-ra-ti-on co-ve-ra-ge va-rie-ty me-thods exist-ing Fi-gu-re va-lu-es fa-shion di-gi-tal vi-si-ting ori-gi-nat-ing mi-li-ta-ry 
in-di-vi-du-al re-gu-lar pre-di-ca-tes pre-di-ca-te avai-la-bi-li-ty sig-ni-fi-cant-ly di-gi-tal re-le-vant
po-pu-lar ope-ra-ting
si-mi-lar
pro-ba-bi-li-ty
in-te-rac-tion op-tional-ly lin-kers
ge-ne-ra-tes
com-ple-xi-ty mo-dels sig-ni-fi-cant si-mi-la-ri-ty ne-ces-sa-ri-ly ca-te-go-ry
eva-lua-tion ma-nual fa-ci-li-ta-ted
in-te-rac-ti-ve de-di-ca-ted exam-ple
par-ti-cu-lar-ly nor-ma-li-ze ave-ra-ge usa-bi-li-ty
equi-va-lent ca-te-go-ries 
co-occur-ren-ce
mathe-ma-ti-cal
si-mi-la-ri-ty
mo-di-fied
veri-fi-able ve-ri-fi-ca-tion mo-di-fi-ca-tion thres-hold uti-li-ty re-ve-nue ad-ver-sa-rial owner-ship ge-ne-ra-li-za-bi-li-ty
}

\begin{document}
\begin{abstract}
With the rise of data-driven methods for traffic forecasting, accident prediction, and profiling driving behavior, personal GPS trajectory data has become an essential asset for businesses and emerging data markets. However, as personal data, GPS trajectories require protection. 
Especially by data breaches, verification of GPS data ownership is a challenging problem. 
%
Watermarking facilitates data ownership verification by encoding provenance information into the data. 
%
GPS trajectory watermarking is particularly challenging due to the spatio-temporal data properties and easiness of data modification; as a result, existing methods embed only minimal provenance information and lack robustness. 
%
In this paper, we propose \approach{} -- a novel GPS trajectory watermarking method based on Fourier transformation. 
%
We demonstrate the effectiveness and robustness of \approach{} 
on two real-world GPS trajectory datasets.

%
\end{abstract}
\begin{CCSXML}
<ccs2012>
<concept>
<concept_id>10002978</concept_id>
<concept_desc>Security and privacy</concept_desc>
<concept_significance>500</concept_significance>
</concept>
<concept>
<concept_id>10002951.10003227.10003236</concept_id>
<concept_desc>Information systems~Spatial-temporal systems</concept_desc>
<concept_significance>500</concept_significance>
</concept>
</ccs2012>
\end{CCSXML}

\ccsdesc[500]{Information systems~Spatial-temporal systems}
\ccsdesc[500]{Security and privacy}

\keywords{GPS trajectory, Watermarking, Data provenance, Data protection}
\maketitle
\pagestyle{plain}
\blfootnote{\textcopyright Rajjat Dadwal, Thorben Funke, Michael Nüsken and Elena Demidova, 2022. This is the author's version of the work. It is posted here for your personal use. Not for redistribution. The definitive version was published in the proceedings of 
The 30th International Conference on Advances in Geographic Information Systems, SIGSPATIAL 2022, \url{https://doi.org/10.1145/3557915.3561474}.\\}
\section{Introduction}
\label{sec:introduction}
%
Personal GPS trajectory data are adopted in various critical domains, including data-driven urban traffic management, mobility, communication, and health. 
%
However, GPS trajectory data encode sensitive personal information such as user addresses, visited locations, and routes. 
%
Sharing and trading personal GPS trajectory data, even based on user consent, can occasionally result in data breaches and user privacy loss \citep{chow2011trajectory}. 
%
\begin{figure}[h!]
   \centering
     \includegraphics[width=1\linewidth,height=4.6cm]{wtrace.pdf}
    \caption{An example \approach{} application scenario. Watermarked GPS trajectory data is modified and re-distributed by an adversary. 
    \approach{} enables data provenance verification.
    Map data: \textcopyright OpenStreetMap contributors, ODbL.
    }
    \label{fig:approach3k}
\end{figure}

Figure \ref{fig:approach3k} illustrates an example application scenario in which GPS trajectory data, initially shared according to the user's consent, is obtained by an adversary due to a data breach, modified to obscure the data origin, and illegally re-distributed on the market.
%
Whereas the modification makes it challenging to claim the data ownership and to identify the misuse, sensitive personal information, such as user routes and driving patterns encoded in the trajectory, remains visible. 
%
The risk of data breaches necessitates the development of effective and robust provenance information embedding methods for personal GPS trajectory data to facilitate data provenance verification. 

Digital watermarking refers to methods that embed provenance information (so-called watermarks) into noise-tolerant data.
%
Watermarking has been extensively studied in the media domain to protect images, videos, and audio files~\citep{bhat2011new,DBLP:journals/ijst/El-WahabEAE21}. 
%
In contrast, only a few initial approaches target watermarking of personal GPS trajectories~\citep{pan2019trajguard,jin2005watermarking}.
%
Watermarking GPS trajectories poses several challenges and is an inherently difficult task. 
The strength of a watermark is subject to a trade-off. 
On the one hand, a watermark should be robust, i.e., strong enough not to be removed by an adversary. 
On the other hand, a watermark should, at the same time, be weak, such that the watermarked data is still usable in the downstream applications.
%
In addition to this general challenge for digital watermarking, GPS trajectories are, with their non-uniform sampling rate and positional inaccuracy, inherently susceptible to different modifications than media data, such as removal/addition of points or re-sampling along the path.  
%
State-of-the-art watermarking methods in the trajectory domain either lack robustness \citep{jin2005watermarking} or are ineffective, i.e., they embed only a small amount of data \citep{pan2019trajguard}. 
%

In this paper, we propose \approach{} -- a novel, robust and effective watermarking method for personal GPS trajectories.
%
\approach{} represents two-dimensional trajectory coordinates as complex numbers and adopts Discrete Fourier Transform (DFT) to enable effective watermark embedding in the frequency domain. To the best of our knowledge, we are the first to propose a DFT-based watermarking scheme for GPS trajectories.
%
We confirm the effectiveness and robustness of our approach by considering a comprehensive set of attacks, i.e., adversarial trajectory modifications, including noise addition, point replacement, and size modifications.
%
We conduct an extensive evaluation using two real-world GPS trajectory datasets.
%
We demonstrate that under the majority of considered attacks, \approach{} retains the watermark in 100\% cases.
%
We make our algorithm and data processing pipeline available as open source\footnote{Software: \url{https://github.com/Rajjat/watermarkingTrajectory}}.  

\section{Definitions \&{} Problem Formulation}
\label{sec:preliminaries}
%
In this section, we introduce the definitions and the problem formulation, which we tackle with the proposed \approach{} approach. 
\begin{definition}[Trajectory]
    A trajectory $T$ is a list of GPS coordinates ordered by the corresponding timestamps:
   %  
  \[
       T=[ (p_j,t_j)], \mbox{with $t_j$ < $t_{j+1}$ for all $j$},
      \]
   %
    where $p_j = (a_j, b_j)$ is the two-dimensional position with latitude $a_j$ and longitude $b_j$ and $t_j$ is the timestamp of that position.
   Trajectory size, $\operatorname{size}(T)$, denotes the number of timestamps included in $T$. 
\end{definition}
%
A watermark is a signal embedded into the trajectory to enable verification of the trajectory origin. 
In this work, we represent watermarks as integer vectors.
\begin{definition}[Watermark]
A watermark $w \in \mathbb{Z}^{m}$ is an integer vector with the dimensionality $m$.
\end{definition}
The dimensionality $m$ of the watermark corresponds to the size of the (sub-)trajectory in which the watermark is embedded.

\textbf{Watermark verification} confirms if a given original watermark is embedded into the data and requires both the extracted watermark and the original watermark to be verified.
 
When the watermarking process  modifies a trajectory $T$ into $\Tilde{T}$,  
$\Tilde{T}$ needs to maintain usability for real-world applications. 
We make that intuition precise by defining a modification threshold.

\begin{definition}[Modification threshold]
    \sloppy%
    A modification threshold~$\sigma$ bounds a distance $D$ for trajectories.
    Given a modification threshold~$\sigma$, we consider $\Tilde{T}$ a $\sigma$-modification of $T$ if the spatial distance between these two trajectories is at most~$\sigma$.
    Formally:
   \begin{equation} 
   %
       \operatorname{D}(T,\Tilde{T}) \leq \sigma.
       \label{eq:mod-threshold}
     % 
    \end{equation}
\end{definition}
%
In our experiments, we work with $\sigma=$10~meters, which reflects the typical inaccuracy of GPS sensors~\citep{bevly2004global}.

Our goal is to watermark GPS trajectories such that the watermarked trajectory remains usable for downstream applications and the watermark can be verified effectively, even if the watermarked trajectory is modified. 
%
Formally, given a watermark embedding procedure $~EMB$, the respective watermarking verification procedure $~VER$, and a watermark $w$, 
we aim that a trajectory $T$ and its corresponding watermarked trajectory $\Tilde{T}=~EMB(T, w)$ obtained after applying watermarking are within the predefined modification threshold~$\sigma$. 
Moreover, we aim that the verification of $w$ with $~VER$ is possible, even if $\hat{\Tilde T}$ is modified from $\Tilde{T}$ within a modification threshold~$\sigma$.
Hence, we want to ensure that the verification $~VER(\hat{\Tilde T}, T, W)$ returns true, if $\hat{\Tilde T}$ is a $\sigma$-modification of $\Tilde{T}$. 

\section{The \approach{} Approach}
\label{sec:approach}

This section presents our proposed watermark embedding and verification method \approach{}. 

\subsection{Watermark Embedding}
\label{sec:watermark addition}
%
Watermark embedding aims to incorporate a watermark into a given GPS trajectory. 
%
We consider a trajectory $T$ of size $n$. 
%
We associate each GPS point $(a_j, b_j)$ with a complex number, 
\begin{align}
    c_{j} = a_{j} + i b_{j}, \label{eq:complex}
\end{align}
where $i$ is the imaginary unit.
We split the transformed trajectory into multiple sub-trajectories of equal size.
%
Next, we apply a Discrete Fourier Transform (DFT) to each sub-trajectory, where we use the Fast Fourier Transform (FFT) \citep{nussbaumer1981fast} algorithm for efficiency.
%
DFT retrieves a frequency domain representation of the input and results in a sequence of complex numbers of the same length as the input. 
%
We feed the list of positions $c = (c_{j})_{k\le j<l}$ from the sub-trajectory spanning the indices $k$ to $l$, represented as complex numbers, into the FFT algorithm. 
%
The resulting frequency representations we then represent via amplitudes $\alpha$ and phase angles $\phi$:
\begin{equation} 
    \label{eq:FFT}
    \alpha,\phi \gets ~FFT(c).
\end{equation}
%
Then, for a sub-trajectory, the watermark~$w$ with strength $s$ is inserted in the amplitude~$\alpha$:
%
%
\begin{equation} 
    \label{eq:watermark}
    \Tilde{\alpha} = \alpha+s \cdot w.
\end{equation}
%
A design decision of our method is to represent the watermark~$w$ as a vector of $1$, $-1$, and $0$ values
of the same size as each sub-trajectory. 
%
This watermark is chosen and stored by the user; the watermark may be the same for each sub-trajectory or vary. In our experiments, we generate the watermarks randomly.
%
The higher the watermark strength $s$, the more we modify the trajectory by inserting the watermark.
%
In our experiments, we use $s=0.0003$.
We split each trajectory into sub-trajectories of size 16. 
     In each sub-trajectory, we embed a watermark with $10$ non-zero dimensions.
%
Once the watermarks are inserted in the amplitude of each sub-trajectory, the next step is to apply an inverse FFT (IFFT) to obtain the watermarked sub-trajectory.
%
We take the watermarked amplitude $\Tilde{\alpha}$ with the original phase~$\phi$ and form a complex number $t_{j} \gets \Tilde{\alpha}_{j} ~exp( i \phi_{j} )$.
Applying the inverse FFT to the vector $t$, we obtain the watermarked trajectory
~$\Tilde{c}$. We abbreviate this as follows:
\begin{equation} 
    \label{eq:watermarkinserted}
   \Tilde{c}=(\Tilde{a},\Tilde{b}) \gets ~IFFT( \Tilde{\alpha}, \phi ).
\end{equation}

%
\subsection{Watermark Extraction \& Verification}
\label{sec:watermark-extraction}

Watermark verification aims to verify if the specific watermark $w$ is embedded in the given trajectory $\hat{\Tilde T}$. This process includes four steps: selection of a candidate trajectory, trajectory size alignment, watermark extraction, and watermark correlation.
%

\textbf{Candidate selection.} As input, the watermark verification process requires 
the trajectory $\hat{\Tilde T}$ to be verified, 
the original trajectory $T$, 
the watermark $w$ and the watermark strength parameter $s$ 
adopted in the watermark embedding process. 
%
As the candidate original trajectory $T$, we select the closest user trajectory based on the minimum haversine distance to $\hat{\Tilde T}$.

\textbf{Trajectory size alignment.} Our watermark verification process requires $T$ and $\hat{\Tilde T}$ to be of the same size.
%
If $size(\hat{\Tilde T})>size(T)$, i.e. the trajectory size increased, we filter the coordinates from $\hat{\Tilde T}$ based on the minimum haversine distance to the candidate trajectory $T$.
%
If the trajectory size of $\hat{\Tilde T}$ is smaller than $size(T)$, we fill the positions in $\hat{\Tilde T}$ with a re-sampling of the closest point (regarding the haversine distance) to obtain the same size.
%

\textbf{Watermark extraction}. 
   %
The watermark extraction process in \approach{} is non-blind, 
i.e., requiring the original data,
and is the reverse of the watermark insertion process.
We split $\hat{\Tilde T}$ into sub-trajectories of equal size and apply DFT to calculate the amplitude $\hat{\alpha}$.
We retrieve the watermark with:
%

\begin{equation} 
    \label{eq:watermark1}
    w'=\frac{\hat{\alpha}-\alpha}{s},
\end{equation}
%
where $\alpha$ is the amplitude of the candidate trajectory $T$ and $s$ is the watermark strength.
%

\textbf{Watermark correlation.} The next step to verify the watermark is to compute the correlation between the extracted watermark $\hat{w}$ and the original watermark $w$ of each sub-trajectory.
%
We adopt Normalized Cross-Correlation (NCC) -- a widely used watermark verification measure \citep{DBLP:journals/ijst/El-WahabEAE21}. %
NCC can successfully verify the watermarks in GPS trajectories, as demonstrated by our experiments.  
%
NCC of two watermarks, $w$ and $w'$, is computed as:
\begin{gather} 
    \label{eq:watermark2} %\gets
    \operatorname{NCC}(w,w') = \frac{\sum_{i}w_{i}   w'_{i}}{\sqrt{\sum_{i}w_{i}^2}\sqrt{\sum_{i}{w'_{i}}^2}}. 
\end{gather}
The value of NCC lies between $-1$ and $1$. NCC value $1$ indicates that two vectors are highly correlated,
whereas $0$ and $-1$ indicate no correlation and negative correlation, respectively. %
Finally, an average NCC score for all sub-trajectories of a given trajectory is calculated, and the verification is successful if this value is higher than the acceptance threshold $\tau$. We adopt $\tau>0.85$ based on \cite{pan2019trajguard}.
%
\section{Threat Model: Attacks on Trajectories}
%
\label{sec:attacks}

Digital watermarking is subject to adversarial attacks. 
%
The available knowledge limits the adversary's ability to prevent watermark verification. 
%
This paper assumes that an adversary has limited access, namely, knows the watermarked trajectory and the watermarking algorithm. In contrast, the original GPS data and the specific watermark embedded into the data remain unknown. 
%
An adversary with limited knowledge cannot remove the watermark directly. Instead, the adversary can attempt heuristic trajectory modifications to prevent watermark verification. 
%
We refer to such modifications as attacks on trajectories.
%


To quantify the utility of the trajectory modified in the adversarial settings for real-world applications, we follow the same principle as we introduced for the trajectory watermarking and apply a modification threshold $\sigma$:
\begin{align*}
\hat{\Tilde T}=AT(\Tilde{T}, \theta), \quad s.t.\ D(\Tilde{T}, \hat{\Tilde T})\leq \sigma.
\end{align*}
 Here, $AT(\cdot)$ is the attack function,
  $\Tilde{T}$ is the watermarked trajectory, 
 $\theta$ represents the specific attack parameter, $D(\cdot)$ is the distance metric, 
 $\hat{\Tilde T}$ is the modified watermarked trajectory, and $\sigma$ is the modification threshold
limiting the effects of the possible attacks on trajectories. 

In this paper, we focus on the attacks discussed in the literature in the contexts of trajectory watermarking~\citep{pan2019trajguard}, trajectory similarity measures~\citep{su2020survey} and the more general perspective of cryptography~\citep{halder2010watermarking}.
%
In particular, we consider four different attack types: 
noise additive attacks, point replacement attacks, size modification attacks, and the combination of these types, the hybrid attack.
% 


\mpara{Noise Additive Attacks}
%
In noise additive attacks, noise is inserted into trajectory coordinates. 
%
\begin{enumerate}
%
\item \textbf{Additive Gaussian White Noise (AGWN)} 
In this attack, for each position in the trajectory, a random sample from a normal distribution is drawn and added to the GPS position.


\item \textbf{Additive Signal to Noise Ratio (ASNR)}  
This attack is similar to the previous attack, but we scale the noise to achieve a selected signal-to-noise ratio (SNR).

\item \textbf{Additive Outliers with SNR (AOSNR)} 
%
We randomly select points with the probability $\theta =(p_{\text{AOSNR}})$, and then add scaled noise to these positions. 

%
\item \textbf{Double Embedding Attack (DEA)}
In the double embedding attack, an adversary attempts to remove the original watermark by embedding a different watermark with the same approach as the original watermark. 
\end{enumerate}

\mpara{Point Replacement Attacks}
Point replacement attacks remove specific trajectory elements and replace them with information based on the adjacent points. 
% 
\begin{enumerate}
\sloppy
\item \textbf{Replace Random Points (RRP)}
Points are selected with the probability $\theta=(p_{\text{RRP}})$,
and then those selected points are replaced with their respective previous points.
%
\item \textbf{Replace Random Points with Path (RRPP)}
%
replaces each point with the probability $\theta=(p_{\text{RRPP}})$. The replaced value is a convex combination of the remaining adjacent points.

\item \textbf{Replace Non-Skeleton Points with Path (RNSPP)}
In this attack, we use the Ramer–Douglas–Peucker (RDP) algorithm.
%
The points removed by the RDP algorithm are replaced with a convex combination of the adjacent points.
\end{enumerate}
%

\mpara{Size Modification Attacks}
%
In size modification attacks, the trajectory size is modified either by cropping or interpolation.
%
\begin{enumerate}
\item \textbf{Linear Interpolation Attack (LIA)} Additional points are inserted at random positions in the trajectory by linear interpolation, increasing the trajectory size. 
\item \textbf{Cropping Attack (CA)} Cropping attack removes selected points from the trajectory, decreasing the trajectory size.
\end{enumerate}


\mpara{Hybrid Attacks}
An adversary can combine several attacks on the same trajectory. We exemplify a hybrid attack as a sequence of a cropping attack (CA) followed by additive Gaussian white noise (AGWN) and replace random points (RRP). 

\section{Evaluation}
\label{sec:evaluation}

We aim to evaluate the effectiveness and robustness of \approach{} regarding the threat model.
%
In this section, we describe the experimental setup and results.

In this section we experimentally evaluate our method -- herein termed NIFM for Neural Integration-free Flow Maps -- for both 2D and 3D time-varying vector fields, comparing against various baselines that accelerate flow map computation in different ways. \new{A requirement that is common to all baselines is access to samples of the flow map. Unless otherwise stated (c.f. Sec.~\ref{subsec:error}), the methods against which we compare NIFM are based on flow maps generated via $4^{th}$ order Runge-Kutta integration (RK4), with step size set to half of the temporal voxel size. We also use this very integration scheme to generate ground-truth flow map samples for the purposes of evaluation.} In Table~\ref{tab:datasets} we list the datasets used for comparison purposes. Further, all reported computational timings are based on a system with 12-core CPU AMD Ryzen 9 3900X, 16GB RAM, and GPU NVIDIA GeForce RTX 2080 Ti with 12GB memory.

We consider the flow map super resolution technique proposed by Jakob et al.~\cite{jakob2020fluid}, wherein we train a convolutional neural network (CNN) model using the 2D fluid flow dataset provided by the authors. To train the CNN we generate 16x downsampled flow maps along with their corresponding high-resolution ground truth flow maps, varying start times and time span of the integration, to permit model generalization for arbitrary start time/duration.

Additionally, we compare our method with the deep learning based Lagrangian interpolation technique proposed by Han et al.~\cite{han2021exploratory}. This technique uses an encoder-decoder network and is most similar to ours in terms of the input data the model expects, and the output of the model. We train the model on flow map samples computed by, first, generating seeds sampled uniformly at random in space and time, and secondly, integrating for varying small time spans. This flow map sampling technique is intended to resemble the Lagrangian short generation scheme proposed by the authors. We made a minor modification to the network by removing the ReLU activation function used in the output layer, allowing the model to output negative values. Further, we compare our method with a SIREN~\cite{sitzmann2020implicit} that tacks time span on as an additional coordinate, along with particle space-time coordinates (c.f. Fig.~\ref{fig:illustrative_network}(a)). We train the SIREN with the same data used to train the encoder-decoder model. Note that we could use a hybrid grid-MLP model~\cite{muller2022instant,weiss2021fast} in lieu of a standard coordinate-based MLP, but for 3D unsteady flows this would require storage of a 5D grid, which is not feasible.

We also compare our method against the recent work by Li et al.~\cite{li2022efficient}, where the authors showed an improvement over prior work in efficiently interpolating Lagrangian representation to obtain new trajectories. \new{Note that the representation of flow in our datasets is Eulerian, whereas Li et al. works with particle-based data, thus, requiring a conversion from the former to the latter. For a fair comparison, we convert the Eulerian representation into a Lagrangian one by first placing $n_s$ number of seeds in the domain uniformly at random, where $n_s$ is the spatial resolution of the vector field data, and integrate these seed points via RK4. The temporal frequency with which we store particle positions is set as the temporal resolution of the field. Furthermore, the Lagrangian representation is limited to the temporal duration on which we are evaluating, to have a better distribution of particles throughout the domain}.

Last, we compare our method with the streakline vector field (SVF) work of Weinkauf et al.~\cite{weinkauf2010streak}. \new{Specifically, the SVF is first precomputed by estimating flow map derivatives, computed via RK4, and then at runtime streaklines are generated by integrating the SVF. We view this as a fair comparison to our technique in that both approaches incur a precomputation cost, and thus we aim to compare the computation and storage requirement for the representations, as well as the accuracy and computation efficiency for generating streaklines.}

\begin{table}[]
\caption{We list all datasets and their respective sizes used in experiments.}
\label{tab:datasets}
\centering
\scalebox{0.8}{
% \begin{tabular}{|c|c|}
% \hline
% Dataset       & Res (t,x,y(,z)) \\ \hline
% Double Gyre   & 500x400x200     \\ \hline
% Cylinder      & 1001x400x50     \\ \hline
% Boussinesq    & 2001x450x150    \\ \hline
% Tornado       & 50x128x128x128  \\ \hline
% Scalar Flow   & 151x100x178x100 \\ \hline
% Half-Cylinder & 151x640x240x80  \\ \hline                 
% \end{tabular}}
% \end{table}
\begin{tabular}{cc}
\hline
Dataset       & Res {[}t,x,y(,z){]} \\ \hline
Double Gyre   & 500x400x200         \\
Cylinder      & 1001x400x50         \\
Boussinesq    & 2001x450x150        \\
Fluid Simulation & 1001x512x512     \\
Tornado       & 50x128x128x128      \\
Scalar Flow   & 151x100x178x100     \\
Half-Cylinder & 151x640x240x80   
\end{tabular}}
\end{table}

\begin{figure*}[t]
\centering
\includegraphics[width=1\linewidth]{figures/quantitative.pdf}
\caption{We show the quantitative evaluation of flow map approximation methods across different datasets, and across different time spans, beginning at start times for which flow features have largely resolved.} 
\label{fig:quantitative}
\end{figure*}


\subsection{Implementation details}
We first describe the details of our network architecture, followed by details on optimization.

\textbf{Network architecture settings}
The design of $f_{\nu}$ and $f_{\tau}$ rely on parameter settings related to the multi-level feature grid, as well as the MLP. The feature grids for $f_{\nu}$ and $f_{\tau}$ are of identical design, where we use a 4-level feature grid, and each level is of a different spatial resolution. Specifically, for a given axis of resolution $w$ at level $l$, we set the resolution at the next level to be $w^{s \cdot l}$, with resolution scaling factor $s$ set to 1.65, following the guidance of M{\"u}ller et al.~\cite{muller2022instant}. Each grid stores $8$-dimensional feature vectors at its nodes, and thus the resulting concatenated feature is $32$-dimensional. We employ 2 and 1-layer MLPs for $f_{\nu}$ and $f_{\tau}$, respectively, along with activation $\sigma_{\tau}$ chosen to be a Swish activation~\cite{hayou2018selection}. Experimentally we found Swish to outperform other more standard activations for INRs, e.g. ReLU, sin, consistent with findings in AutoInt~\cite{lindell2021autoint}. We control for the size of the network by a compression ratio, expressed as the ratio of the vector field size to the network size. We adjust the spatial resolution of the feature grids to best match a provided compression ratio, but leave the MLPs unchanged as they comprise a tiny portion of the model. Last, we use a 3-layer MLP with $64$ layer width for the residual network. Unless otherwise specified, we use a compression ratio of $10$ for all 2D datsets, and customize compression ratios for 3D as appropriate.
 
\begin{table}[]
\caption{We report the preprocessing times for different methods across 2D unsteady flows, along with corresponding timings for FTLE computation, varying time span and image resolution.}
\label{tab:time}
\centering
\scalebox{0.7}{%
    \begin{tabular}{|c|c|c|c|c|c|c|c|}
    \hline
    Dataset                      & FTLE res           & $\tau$ & \begin{tabular}[c]{@{}c@{}}Inference \\time(s)\end{tabular} & \begin{tabular}[c]{@{}c@{}}Preprocessing \\time(min)\end{tabular} & CR & \begin{tabular}[c]{@{}c@{}}Storage \\(MB)\end{tabular} &method       \\ \hline
    \multirow{4}{*}{Fluid Sim} & \multirow{4}{*}{512x512}  & \multirow{4}{*}{7}   & 21.161                                                 & -              & -       &   2003 & GT \\ \cline{4-8} 
                                      &                           &                      & \textbf{0.585}                                & 48.01             & 10     &   189   & NIFM         \\ \cline{4-8} 
                                      &                           &                      & 2.010                                        & 63.33              & 1       &              & Siren   \\ \cline{4-8} 
                                      &                           &                      & 4.701                                          & 1104.60           & -       &              & FSR \\ \hline
    \multirow{4}{*}{Cylinder}         & \multirow{4}{*}{1200x150}  & \multirow{4}{*}{1}   & 1.853                                         & -                & -         &     153     & GT \\ \cline{4-8} 
                                      &                           &                      &  \textbf{0.055}                              & 33.50              & 10         &     16     & NIFM         \\ \cline{4-8}
                                      &                           &                      & 0.554                                        & 74.26              & -           &          & ED   \\ \cline{4-8} 
                                      &                           &                      & 0.324                                        & 41.76              & 1            &        & Siren   \\ \cline{4-8} 
                                      &                           &                      & 29.94                                              & 0.04         & -             &       & Spline    \\ \hline
    \multirow{4}{*}{Boussinesq}       & \multirow{4}{*}{450x1350} & \multirow{4}{*}{0.5} & 2.220                                          & -                 & -            &  1030  & GT \\ \cline{4-8} 
                                      &                           &                      &  \textbf{0.079}                               & 37.25             & 10            &      97    & NIFM         \\ \cline{4-8} 
                                      &                           &                      & 0.938                                        & 122.89              & -             &        & ED   \\ \cline{4-8} 
                                      &                           &                      & 0.621                                       & 63.28                & 1       &             & Siren   \\ \cline{4-8} 
                                      &                           &                      & 91.57                                             & 0.15            & -       &       & Spline    \\ \hline
    \multirow{4}{*}{\begin{tabular}[c]{@{}l@{}}Double Gyre\end{tabular}}      & \multirow{4}{*}{1200x600}  & \multirow{4}{*}{10}  & 34.453                      & -        & -   &  611   & GT \\ \cline{4-8} 
                                      &                           &                      &  \textbf{1.020}                               & 34.80                 & 10        &      29     & NIFM         \\ \cline{4-8}
                                      &                           &                      & 6.278                                        & 40.70                  & -         &         & ED   \\ \cline{4-8}
                                      &                           &                      & 1.689                                        & 19.84                 & 1       &    & Siren   \\ \cline{4-8} 
                                      &                           &                      & 252.63                                             & 0.29            & -        &          & Spline    \\ \hline
    \end{tabular}%
}
\end{table}
 
\textbf{Optimization details}
For both phases of optimization we use Adam~\cite{kingma2015adam}, where we take a total of $40,000$ optimization steps and decay the learning rate every $8,000$ steps. Specific to optimization phase, in fitting to the vector field we use a learning rate of $0.02$, while for flow map optimization we use a learning rate of $0.01$ -- fitting the flow map derivative to the vector field is quite stable, and benefits from larger learning rates. In optimizing for the flow map, we have the choice of leaving the instantaneous velocity portion of the network frozen, or fine-tuning its weights to compensate for the remainder of the network. Although we find that both give results of comparable accuracy, in some occasions we found that fine-tuning can mitigate small grid-based artifacts in the output when leaving these weights frozen, and hence we fine-tune this portion of the network, using a learning rate of $0.0008$.

Recall that our method supports a maximum time span $\tau_{max}$ on which to sample during optimization. Though in principle we could optimize for the full time span of a given dataset, we find that performance can suffer, especially for datasets exhibiting complex temporal dynamics. Thus, as a compromise we set a limit on $\tau_{max}$ during optimization, and at inference time, for any target $\tau > \tau_{max}$ we take multiple steps with our network until reaching the desired span $\tau$. Specifically, for all 2D datasets, expressed in terms of grid units we set $\tau_{max} = 48$ unless otherwise specified. For 3D datasets we customize $\tau_{max}$ based on grid resolution, and complexity of the flows.

\begin{figure*}[t]
    \centering
    \includegraphics[width=1\linewidth]{figures/fluid_sim-compressed.pdf}
    \caption{We compare FTLE (top row) and integration error (bottom row) for two Fluid Simulation datasets (Re 16 and Re 101.6) across different baselines. The left column corresponds to particles integrated beginning at $t_0 = 0$ for duration $\tau=7$, while the right column corresponds to particles integrated starting at $t_0 = 2$ and $\tau = 7$.} 
    \label{fig:ftle_fluid}
\end{figure*}

\begin{figure*}[t]
    \centering
    \includegraphics[width=1\linewidth]{figures/dg-compressed.pdf}
    \caption{We compare FTLE (top row) and integration error (bottom row) for different baselines for the Double Gyre dataset. Particles are integrated from $t_0=0$ for a time-span $\tau=10$.} 
    \label{fig:ftle_dg}
\end{figure*}

\subsection{2D unsteady flow}

We first conduct experimental comparisons for various 2D time-varying flow fields. Specifically, we evaluate different techniques by computing the error in flow map approximations over varying seed points (spatial position and starting time) that have been integrated for varying time spans. We express error as the averaged Euclidean distance between the ground-truth flow map output, and the approximation scheme's output, normalized by the domain's bounding-box diagonal length. In Fig.~\ref{fig:quantitative} we present quantitative results comparing our method against different baselines, and in Table~\ref{tab:time} we report inference and preprocessing times. Specifically, for the pathline interpolation approach of Li et al.~\cite{li2022efficient}, preprocessing refers to the time required to fit B-splines, while for Jakob et al.~\cite{jakob2020fluid} this refers to the time required to optimize the CNN for super resolution. For all remaining methods, preprocessing refers to the time required for optimizing to an individual flow field.

\begin{figure}[!t]
    \centering
    \includegraphics[width=1.0\linewidth]{figures/cy_h.pdf}
    \caption{We show the FTLE (top of each pair) and error maps (bottom of each pair) for the flow over cylinder dataset generated by integrating particles starting at $t_0=18$ for a time-span $\tau=1$.} 
    \label{fig:ftle_cy}
\end{figure}

\begin{figure}[!t]
    \centering
    \includegraphics[width=1.0\linewidth]{figures/bo.pdf}
    \caption{We show the FTLE (top of each pair) and error maps (bottom of each pair) for the Boussinesq dataset generated by integrating particles starting at $t_0=11.3$ for a time-span $\tau=0.5$.} 
    \label{fig:ftle_bo}
\end{figure}


In comparing the fluid simulation flows of varying Reynolds numbers, we find that our method sees consistent improvement in accuracy over SIREN and super resolution, while achieving faster inference times. We note that the super resolution approach requires optimizing a CNN over a collection of flow maps just once, and thus can generalize to low-resolution flow maps at inference time, albeit restricted to flows resembling those observed during training. Our method is limited to just a single dataset at a time, but nevertheless, our training times scale well in terms of standard INRs (e.g. SIREN), while exhibiting faster inference and more accurate flow map approximations. Qualititative results for the fluid simulation flows are shown in Fig.~\ref{fig:ftle_fluid} in the form of the FTLE -- \new{computed using the method of Haller~\cite{haller2001lagrangian}} -- and color-encoded flow map errors. For high Reynolds number flows, we see that the super resolution method can fail to adapt to the rate at which particles separate, as indicated by the color shift, while also blurring out detailed ridges in the FTLE.  Our method, however, excels in capturing FTLE ridges, while remaining efficient to compute, since the super resolution method still requires computing a low-resolution flow map as input to a (otherwise highly efficient) CNN. Recall that our method employs a compression ratio of $10$ for all 2D experiments, which limits the grid resolution, and thus might limit the details we can reproduce in the flow map. However, from these results, we see that the coarser feature grid resolution does not limit the spatial resolution of the FTLE.


\begin{figure*}[t]
\centering
\includegraphics[width=1\linewidth]{figures/streaklines_new.pdf}
\caption{We compare our method's ability to compute streaklines against the streakline vector field technique~\cite{weinkauf2010streak}, which only necessitates integrating a derived vector field. Qualitatively and quantitatively we find that our method produces comparable results, where we show varying step sizes used for evaluating the flow map.}
\label{fig:streaklines}
\end{figure*}

In comparing our method to other baselines (c.f. Fig.~\ref{fig:quantitative}) for Double Gyre, Cylinder, and Boussinesq, we find that our method obtains higher accuracy in relation to other techniques. Prior INR methods such as the encoder-decoder architecture of Han et al.~\cite{han2021exploratory}, or a pure coordinate-based approach~\cite{sitzmann2020implicit} poorly generalize. We find that for small step sizes, the performance of these methods in fact steeply declines, as numerical error accumulates with the more steps taken. We attribute this to the basic limitations of the network architectures employed, failing to address the properties (identity mapping, instantaneous velocity) we target in our network design. The inability to generalize in these methods is further demonstrated qualitatively for Figs.~\ref{fig:ftle_dg} - \ref{fig:ftle_bo}. \new{Pathline interpolation~\cite{li2022efficient} is notable in its small precomputation cost. Nevertheless, the method is less accurate in preserving the flow map, while incurring a high computation cost at runtime.}

We additionally evaluate our technique both quantitatively and qualitatively for the computation of streaklines. In Fig.~\ref{fig:streaklines} we show streaklines for the Cylinder dataset. We compare our method with SVF~\cite{weinkauf2010streak}. We can see that both the techniques are able to capture the vortices of the dataset faithfully, and are visually indistinguishable from the ground truth streaklines. Quantitatively both the techniques consistently incur low streakline error staying within the margin of $10^{-3}$ magnitude (relative to the bounding box diagonal). Interestingly, we find that both methods have comparable inference time as well, as reported in Table~\ref{tab:streaklines}, despite the fact the streakline vector field evaluates its field fewer times than our neural flow map, since we must take multiple steps for sufficiently long time spans. However, an advantage of our method lies in data parallelism; we can evaluate the flow map over varying space/time/duration in a single batch, whereas integrating the streakline vector field is, by necessity, a sequential process. We further note that SVF precomputation is quite expensive, both in terms of speed and storage space. In Table~\ref{tab:streaklines} we can see that the computation of the entire 4D SVF has very large storage requirements (160GB), whereas our method is in proportion to the size of the vector field (77MB). We note that while our technique can be easily scaled to 3D datasets, SVF preprocessing for 3D unsteady flows is infeasible in practice, necessitating a 5D grid for storage.


\begin{table}[!t]
\caption{We report storage requirements, preprocessing time and inference time for computing streaklines on the Cylinder dataset, comparing our method against the streakline vector field technique~\cite{weinkauf2010streak}.}
\label{tab:streaklines}
\centering
\scalebox{0.9}{
\begin{tabular}{|c|c|c|c|}
\hline
Method  & \begin{tabular}[c]{@{}c@{}}Preprocessing\\ Time\\ (min)\end{tabular} & \begin{tabular}[c]{@{}c@{}}Inference\\ Time\\ (sec)\end{tabular} & \begin{tabular}[c]{@{}c@{}}Storage\\ \end{tabular} \\ \hline
Ground Truth & NA  & 21.391 & 160.20 MB \\  \hline
SVF & 130.407 & 1.204 & 160.36 GB \\ \hline
NIFM (16 grid steps) & \multirow{2}{*}{40.060} & 0.952 & \multirow{2}{*}{77.20 MB}\\ \cline{1-1} \cline{3-3}
NIFM (24 grid steps) & &0.671 & \\ \hline
\end{tabular}}
\end{table}

\begin{table}[!t]
\caption{We report the processing times as well the FTLE computation times for different method across different 3D unsteady flow datasets.}
\label{tab:3d_ftle_times}
\centering
\scalebox{0.8}{
\begin{tabular}{|c|c|c|c|c|c|c|}
\hline
Dataset                                        & FTLE res                     & $\tau$                                     & \multicolumn{1}{l|}{\begin{tabular}[c]{@{}l@{}}Inference\\ times (s)\end{tabular}} & \multicolumn{1}{l|}{\begin{tabular}[c]{@{}l@{}}Processing\\ times(m)\end{tabular}} & CR & \multicolumn{1}{l|}{Method} \\ \hline
\multicolumn{1}{|c|}{\multirow{4}{*}{Tornado}} & \multirow{4}{*}{128x128x128} & \multirow{4}{*}{50}                     & 27.16                                                                              & -                                                                                 & -  & GT                          \\ \cline{4-7} 
\multicolumn{1}{|c|}{}                         &                              &                                         & \textbf{3.60}                                                                               & 35.55                                                                       & 10        & NIFM                        \\ \cline{4-7} 
\multicolumn{1}{|c|}{}                         &                              &                                         & 14.14                                                                              & 93.21                                                                           & 10   & SIREN                       \\ \cline{4-7} 
\multicolumn{1}{|c|}{}                         &                              &                                         & 286.29                                                                             & 0.87
& - & Spline                      \\ \hline
\multirow{4}{*}{Scalar Flow}                   & \multirow{4}{*}{100x178x100} & \multirow{4}{*}{2.5}                    & 81.72                                                                              & -                                                                              & -     & GT                          \\ \cline{4-7} 
                                               &                              &                                         & \textbf{2.55}                                                                               & 41.66                                                                          & 10     & NIFM                        \\ \cline{4-7} 
                                               &                              &                                         & 21.48                                                                              & 95.57                                                                      & 10         & SIREN                       \\ \cline{4-7} 
                                               &                              &                                         & 291.39                                                                             & 0.81                                                                        & -        & Spline                      \\ \hline
\multirow{3}{*}{Half-Cylinder}                 & \multirow{3}{*}{640x240x80}  & \multicolumn{1}{c|}{\multirow{3}{*}{2}} & 137.41                                                                             & -                                                                                & -   & GT                          \\ \cline{4-7} 
                                               &                              & \multicolumn{1}{c|}{}                   & \textbf{3.82}                                                                               & 45.56                                                                       & 40        & NIFM                        \\ \cline{4-7} 
                                               &                              & \multicolumn{1}{c|}{}                   & 53.52                                                                              & 103.13                                                                        & 40      & SIREN                       \\ \hline
\end{tabular}}
\end{table}

\begin{figure*}[t]
    \centering
    \includegraphics[width=0.9\linewidth]{figures/3d_results.pdf}
    \caption{We compare, both qualitatively (volume rendering of FTLE field) and quantitatively (flow map evaluation), our method with standard coordinate-based networks~\cite{sitzmann2020implicit} as well as pathline interpolation techniques~\cite{li2022efficient} for modeling the flow map in 3D unsteady flows. We find our method is quantitatively an improvement over other methods, and qualitatively our method contains fewer visual artifacts.}
    \label{fig:ftle_scalar}
\end{figure*}

\begin{figure*}[t]
    \centering
    \includegraphics[width=0.9\linewidth]{figures/half_cylinder_v2.pdf}
    \caption{In this figure, we compare our method both quantitatively and qualitatively against SIREN for the Half-Cylinder dataset. We find that our method is able to scale reasonably well to this large dataset, whereas, the SIREN fails to learn meaningful flow maps as can be seen from the FTLE.}
    \label{fig:ftle_half_cylinder}
\end{figure*}

\subsection{3D unsteady flow}

We next evaluate our method on a set of 3D unsteady flows, comparing our method with a SIREN-based flow map~\cite{sitzmann2020implicit} as well as the B-spline pathline interpolation technique~\cite{li2022efficient}. We first compare to the Tornado and Scalar Flow datasets, where we set the $\tau_{max}$ to $8$ and $24$, respectively, to match the temporal complexity in the flows. Fig.~\ref{fig:ftle_scalar} shows qualitative results, via volume-rendering of the FTLE, as well as quantitative results. Our method is an improvement, if not comparable, to prior methods, but we obtain significant gains in inference time, as reported in Table~\ref{tab:3d_ftle_times}. We further compare to the Half Cylinder dataset, a large-scale unsteady flow dataset that cannot be readily stored in memory. We found the pathline interpolation method~\cite{li2022efficient} failed to fit to the data, and thus we limit our comparison to SIREN, please see Fig.~\ref{fig:ftle_half_cylinder}. In this experiment we set $\tau_{max} = 8$ and the compression ratio to $40$ to compensate for the larger data size. We find our method captures turbulent features in the wake of the half-cylinder object ($Re=320$), whereas SIREN faces difficulties in accurately modeling the data. Notably, for this dataset we find our training scheme scales well (c.f. Table~\ref{tab:3d_ftle_times}) relative to the 2D unsteady flow datasets, whereas SIREN's increase in model size leads to slower training times.

\vspace{-.9em}

\new{
\subsection{Error analysis: numerical integration}
\label{subsec:error}
Our method can be viewed as a novel technique for integrating a vector field, and thus, it is worth asking: how does our method compare to conventional numerical integration schemes? To help answer this question, we compare NIFM to existing numerical schemes, namely Euler and RK4, evaluated under varying step sizes. For the purpose of evaluation we use the Sine Ridge dataset provided by Kuhn et al.~\cite{kuhn2012benchmark} - as this is a steady flow we adapt our method accordingly. The dataset has an analytically-defined flow map that allows us to compute the flow map error across different schemes. In Fig.~\ref{fig:analytical}, we show the FTLE (first row) and the flow map error (second row) for Euler, RK4, and NIFM. The FTLE is computed for a duration $\tau=1.2$ with step size set to 30, where a single step amounts to 0.01 in the physical domain. We can see that NIFM best captures the FTLE, while maintaining low error in the flow map, in contrast with Euler and RK4. This provides evidence that our method is not merely a fixed linear (e.g. Euler), or higher-order (e.g. RK4) integration scheme, but rather adapts to the features of the data. We further show quantitative results for duration 0.6 and 1.2, again varying the step size. We can see that while NIFM has a consistent performance across all step sizes, the flow map error increases significantly for both RK4 and Euler with increasing step size.
}
\begin{figure}[!t]
    \centering
    \includegraphics[width=1.0\linewidth]{figures/analytical_example.pdf}
    \caption{We compare NIFM to Euler and RK4 integration schemes, showing FTLE (top), flow map error (middle), and quantitative evaluation (bottom). We find NIFM performs consistently well across step sizes as compared to Euler and RK4 which can become numerically unstable.} 
    \label{fig:analytical}
\end{figure}

\subsection{Ablation: compression and supervision}

Last, we run model ablations to study the effects of various design choices. Due to space limitations we limit ablation to compression, as well as the role of supervision in learning flow maps. Further experiments regarding the architecture choices (number of levels in the multiresolution grid) and optimization scheme (number of steps to take, c.f. Eq.~\ref{eq:steps}) are detailed in the appendix.

In Fig.~\ref{fig:grid_artifact} we show the results of our model, for the FTLE of the Boussinesq, optimized under varying compression ratios. In this experiment we specifically wish to study how compression might impart visual artifacts in derived quantities of the flow map approximation, as a higher level of compression results in coarser feature grids. Indeed, we find that lower levels of compression lead to fewer grid-like artifacts in the resulting FTLE when taking a smaller steps, e.g. in this setting, a step size of 48 grid units in time amounts to an evaluation of the model just 3 times per position. We further report inference times for the smallest and largest level of compressions, and as expected, a larger number of steps requires longer inference times (e.g. more feedforward passes with the network). Interestingly, we find the inference time is fairly consistent across these compression ratios, indicating that the increased resolution of the grid has a negligible impact on this matter. As detailed in the appendix, we also find that the flow map accuracy takes just a small hit in performance across compression ratios, indicating that flow map accuracy might not be predictive of visual artifacts in derived quantities. Nevertheless, as shown in the figure, training times come at a cost with smaller compression ratios. We thus see natural trade-offs in the (1) flow map quality, (2) inference time (hinging on step size), and (3) training time.

\begin{figure}[!t]
    \centering
    \includegraphics[width=0.9\linewidth]{figures/grid-artifact.pdf}
    \caption{We qualitatively compare our model under varying compression ratios, showing the effect of compression on the step size taken by our model to produce the FTLE for the Boussinesq flow.} 
    \label{fig:grid_artifact}
\end{figure}

\begin{figure}[!t]
    \centering
    \includegraphics[width=1\linewidth]{figures/flowmap_vs_self.pdf}
    \caption{For the Boussinesq flow we compare our self-consistency criterion with that of directly supervising on flow samples, finding that our method produces comparable, if not improved, flow map approximations, without ever accessing the ground-truth flow map.} 
    \label{fig:self_consistency_exp}
\end{figure}

Our choice to learn flow maps via a self-supervisory signal is in contrast with how numerous visualization techniques interpolate~\cite{chandler2014interpolation,li2022efficient}, or build models~\cite{han2021exploratory} given samples of the flow map, e.g. typically as densely-sampled pathlines. Therefore we ask: is our self-consistency criterion an inferior objective to directly supervising on flow map samples? To this end, we have gathered a large collection of flow map samples, and modified our objective (Eq.~\ref{eq:composition-detail}) to accept the ground-truth flow map, and its corresponding derivative at the output position. We optimize for Boussinesq, using 20M and 50M flow map samples, and compare with our proposed objective, please see Fig.~\ref{fig:self_consistency_exp} for the results. We find that our method is able to learn comparable, if not better, flow map approximations, without ever observing flow map samples. In particular, at 50M samples we find that flow map supervision starts to become competitive with our method. Although supervising an on even larger number of samples might be more beneficial, clearly the data requirement starts to become prohibitively expensive, both for integrating the flow field, as well as storage requirements. In contrast, our method avoids these issues by requiring the vector field as the only supervision.


\mpara{Datasets}
We use two real-world trajectory datasets for evaluating the proposed watermarking method. 
We randomly selected $1100$ trajectories of size $256$ from each dataset.
%
\begin{enumerate}
\sloppy
\item \textbf{German Dataset} is provided by a proprietary data provider. 
%
The dataset contains trajectory data of vehicles from two German federal states: Saxony and Lower Saxony, in September 2019. 
The average sampling rate is 12 times per minute. 
%
\item \textbf{Porto Dataset} 
%
contains variable size trajectories generated by 442 taxis from July 1, 2013, to June 30, 2014, in Porto, Portugal \citep{porto}.
%
The sampling rate is four times per minute. 
%
\end{enumerate}

\mpara{Baselines}
%
We adopt state-of-the-art watermarking methods from the audio domain and GPS trajectories domain.

%
\begin{enumerate}
\sloppy
\item \textbf{IMF Watermarking \citep{DBLP:journals/ijst/El-WahabEAE21}} is a non-blind technique used in watermarking audio signals.
%
Each trajectory is represented as a signal (latitude/longitude vs. time) and decomposed into multiple parts using Empirical Mode Decomposition (EMD).
%

\item \textbf{TrajGuard \citep{pan2019trajguard}} watermarks a GPS trajectory using a geometric transformation based on a blind scheme, i.e., it does not require the original data for the extraction.
%
TrajGuard partitions the trajectory into multiple parts and then distributes the 
watermark into all the sub-trajectories.
%

\item \textbf{SVD Watermarking ~\citep{bhat2011new}} is based on a blind audio watermarking scheme. 
%
This method uses Singular Value Decomposition (SVD) and quantization index modulation. 
%
\end{enumerate}

\mpara{Evaluation Metrics}
%
To assess the watermark verification effectiveness and robustness, i.e., the ability to correctly recognize a watermark in modified trajectory data, we adopt \textbf{recognition rate}. 
%
Recognition rate is the ratio of the number of correctly identified watermarked trajectories (true positives, $TP$) to the total number of watermarked trajectories:
$~Recognition~rate=TP/(TP+FN)$, where $FN$ is the number of false negatives, i.e., unrecognized watermarked trajectories. 
%
Following \cite{pan2019trajguard}, we accept the watermark to be successfully verified if the average watermark correlation between the noised trajectory and watermarked trajectory is higher than the acceptance threshold, i.e., $\tau > 85\%$.


\mpara{Evaluation Results}
\approach{} approach is effective and robust against all the considered attacks in both datasets, as shown in Table~{\ref{tab:example}}.
%
The average recognition rate of \approach{} is around 99\% in both datasets, confirming the effectiveness, robustness, and generalizability of \approach{}. 
%
%
Baseline methods demonstrate varying performance against some attacks across the two datasets.
%
For example, TrajGuard does not perform well in multiple attacks, especially on the Porto dataset. 
%
This is because the Porto dataset is spatially denser than the German dataset, 
making TrajGuard more vulnerable to attacks \citep{pan2019trajguard}.
%
Furthermore, TrajGuard embeds a smaller amount of watermark information, leading to a lower recognition rate. 
%
IMF watermarking failed to detect the watermark in the German dataset, whereas this method works well for the Porto dataset.
%
The German dataset covers a large geographical area, including two German federal states, whereas the Porto dataset is limited to one city.
%
A denser spatial area of the Porto dataset leads to a better decomposition and makes the verification process more effective.
%
Regarding the SVD watermarking, we observe that the DEA attack destroys the quantization-based watermark detection process. 
%
In summary, in contrast to the baselines, \approach{} is more robust against the considered attacks and less dependent on data sparsity.  

%
\section*{Acknowledgments} This work is partially funded by the Federal Ministry for Economic Affairs and Climate Action (BMWK), Germany, under  ``CampaNeo'' (01MD19007B), and ``d-E-mand'' (01ME19009B), the European Commission (EU H2020) under ``smashHit'' (871477), the German Research Foundation under ``WorldKG'' (424985896), and by the B-IT foundation and the state of North Rhine-Westphalia (Germany).

\balance
 
\bibliographystyle{ACM-Reference-Format}
\bibliography{ref}
\end{document}

