\IEEEPARstart{P}{recise} and accurate yield forecasting is a key component in Fresh Produce (FP) Supply Chain Management (FSCM), since it plays a critical role in price negotiations, logistics, and scheduling. In particular accurate yield estimates are required a minimum of 3 weeks ahead (in the strawberry domain) which we call the horizon (Figure \ref{fig:ppp}), so that adequate time can be given to bidding, labour timetabling, logistics, and procurement. However, forecasting FP is incredibly difficult especially over a 3-week horizon where any number of variabilities can exist such as environmental fluctuations. Often the quantities of fresh produce we seek to deal with make it impractical to expect climate-controlled greenhouse conditions, meaning there is an element of weather forecasting that is required however we do not expressly aim to forecast weather in this work as this is a separate and highly complex problem of its own. Instead, we show how good yield forecasting can be and improve upon current practices while allowing for future works to delve specifically into weather forecasting.

Yield forecasting is difficult in particular due to the in-availability of data with which to forecast, this data being mostly non-existent, or incredibly difficult to attain. We believe the reasons why the data is unavailable is because of the difficulty of data collection, the perceived sensitivity with which this data is held, and the lack of clear benefits to the digital collection of such data. We also see resistance to the positive dynamic impetus of modernisation requiring a departure from growers' previous fixed practices.

FP optimisation is of global strategic importance since horticulture and agriculture are some of the biggest producers of greenhouse gasses, such that there can be a significant benefit to optimising production or minimising waste. In the UK our government has committed to reducing greenhouse gasses to net 0 by 2050, and agriculture has been expressly named as a key contributor of greenhouse gasses in the United Nations Climate Change Conference 2021 (COP21). Inaccurate forecasting or more specifically under/ over estimation leads to food waste and destruction costs or importing of FP from abroad. Assuming the cause of this discrepancy/ variability is adverse weather conditions, then those same weather conditions will have affected geographically approximate growing sites. In the UK climate discrepancies usually mean fruit must be imported from abroad, given our size, to meet any given procurement contract, as all the neighbouring growing sites will have suffered the same adverse environmental conditions and thus under-production.

\begin{figure}[t!]
  \centering
  \includegraphics[width=1\columnwidth]{content/img/drawio/past-present-premonition.pdf}
  \caption{Past (purple-pink), present (blue) and premonition (yellow) timelines/ windows overlayed on a depiction / rough reference of strawberry yields through the years of 2020 and 2021 along with temperature. Depicting the point of prediction relative to (at the seam of) horizon and history.}
  \label{fig:ppp}
\end{figure}

% \begin{figure}[t!]
%   \centering
%   \includegraphics[width=1\columnwidth]{content/img/drawio/premonition.drawio}
%   \caption{Premonition network multi-timeline conceptual architecture.}
%   \label{fig:premonition}
% \end{figure}

Other works (outlined with more detail in Section \ref{sec:literature}) have sought to solve the lack of data availability in agriculture using satellite/ remote-sensing data, using various machine learning, statistical, and some deep learning techniques. In this paper we show how we can collect data at some scale but with local/ high granularity, including fruit images, weather conditions, and irrigation data locally. Here we shall focus specifically on strawberry yields of strawberry tabletop and how we can predict them. We exemplify this approach at our \anon{Riseholme} strawberry tabletop/ polytunnel growing site and employ this data to create accurate forecasts with this 3-week horizon/ window to meet the needs of the bidding and procurement process. We do all this in collaboration with \anon{Berry Gardens Growers (BGG)}, one of the UK's largest soft, and stone fruit producers, and with their direction on industry standards to keep as close to the typical expectations as reasonably possible. We also have fortnightly visits by agronomists to ensure we are growing the strawberries satisfactorily.

We use this data in various neural network architectures in Section \ref{sec:methodology} and evaluate their performance in Section \ref{sec:discussion}, since the literature would suggest that deep learning approaches are the most performant even for FP. Of these new architectures, we showcase our Premonition Network which seeks to improve upon current tabular/ sequence prediction approaches using all three forms of context, the past, the present, and the premonition of the future. We use the past to learn the overarching distribution, we use the present to set some scale and granularity, and we use the premonition for variability from the standard distribution.