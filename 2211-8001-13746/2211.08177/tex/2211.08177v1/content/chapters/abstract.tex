Yield forecasting is a critical first step necessary for yield optimisation, with important consequences for the broader food supply chain, procurement, price-negotiation, logistics, and supply. However yield forecasting is notoriously difficult, and oft-inaccurate. Premonition Net is a multi-timeline, time sequence ingesting approach towards processing the past, the present, and premonitions of the future. We show how this structure combined with transformers attains critical yield forecasting proficiency towards improving food security, lowering prices, and reducing waste. We find data availability to be a continued difficulty however using our premonition network and our own collected data we attain yield forecasts 3 weeks ahead with a a testing set RMSE loss of ~0.08 across our latest season.
% Take for instance the past layer, is responsible for understanding the historic distributions of the data in regular intervals, like each year. The present is responsible for contextualising the past, say if this year production capabilities have expanded significantly, that the present layer can inform the merging layer(s) of this increase in scale. The premonition layer is responsible for variability, like adverse weather conditions expected in the next week. The premonition layer is responsible for informing the merge layer of the variability expected in the future, and thus it can calculate some expected deviation from the historic distribution that has been contextualised, and varied.