There are relatively few works in strawberry yield prediction using deep learning, instead the majority focus on statistical machine learning, and almost none that refer to privacy considerations \autocite{hopf2022development, VANDERVELDE2019203, bouras2021cereal, paudel2021machine, zhu2022deep, bali2022emerging, jafari2020, gastli2021, maskey2019}. However, several papers have stressed that a lack of data availability (\autocite{pearson2019distributed, durrant2021might, durrant2022role}), or more specifically a high expense of acquisition which significantly hinders the smooth application of state-fo-the-art neural networks towards the creation of powerful forecasting models \autocite{nassar22, jafari2020, gastli2021, chen2019, maskey2019}. Many of the aforementioned papers largely choose to tackle this lack of data by using satellite imagery although in some cases they use the California strawberry commission data paired with the California strawberry commission irrigation management information system (CIMIS). Unfortunately the data mentioned in these papers is behind multiple walls, and the CIMIS data is currently unavailable from the original source, so while we were able to find an excerpt of the CIMIS data elsewhere we were unable to find the full dataset making it very difficult to compare to.

Many different proposals for methods of predicting / forecasting yield (generically) exist, some using classical machine learning (e.g. \autocite{paudel2021machine}) others such as those by Nassar \autocite{nassar22} use neural networks in their specific case a mixture of CNN, LSTMs, GRUs and some attention heads. However all emphasise the need for better forecasting systems as demand increases and supply decreases due to global factors such as (but not limited to) COVID-19 and the Russia-Ukraine war. Current yield forecasting methods are highly archaic, often times they can be as simple as forecasting the average of the last few years' yields, or simple linear models based on heat hours. \autocite{paudel2021machine} One such example is the European Commission's MARS crop forecasting system (MCYFS) which has purportedly seen no improvement in its forecasting performance since 2006 and uses no machine learning. \autocite{paudel2021machine} Lastly the work by Paudel \autocite{paudel2021machine} shows that machine learning can already at the very least match (at the start of the season) or beat existing large-scale traditional crop yield forecasting systems such as the aforementioned MCYFS system.

The MCYFS system from 2006 to 2015 has a median MAE of 0.379, 0.368, 0.570 in soft wheat durum wheat and grain maize \autocite{VANDERVELDE2019203}. The most performant forecasts for this system appear to be sunflower yields at 0.162 MAE. However the assessment carried out by van der Velde does not state over what period these yield predictions are made specifically whether that be a few weeks, days or months ahead making this also a difficult comparison to make. It is also apparent that forecasting is becoming increasingly difficult with the higher degree of variability in climate conditions as the performance of this largely static forecasting system seems to be in slow decline \autocite{VANDERVELDE2019203}.

As more modern dynamic techniques are still only just beginning to be used in literature towards strawberry tabletop forecasting we look towards the application of these much more modern techniques, in particular deep learning / neural networks. However, as previously stated data is incredibly difficult to attain in this domain. Nassar \autocite{nassar22} appears to show how the compound deep learning models outperform standalone deep learning models and traditional machine learning models. Nevertheless, as with much work in this space, it is difficult to garner any concrete comparable statistics. From one of their diagrams (14) we believe we can see their most performant model to produce an MAE loss of roughly 0.14 or 14\% MAPE. They call this model Attention-ConvLSTM2D. While we do not have access to the same data as they have, we have seen even simple GRU models attain similar performance in our strawberry tabletop. However, we believe we can improve this performance on our own data by means of attention as their paper would also suggest, but instead of standalone attention heads we intend to use a much more complex and performant transformer model.

For the sake of completion, we also take a look at an adjacent work (strawberry counting including flowers) by Yang Chen \autocite{chen2019} which seeks to count strawberries. Clearly, the number of flowers currently available on the strawberries will be directly linked to the outcome of strawberry fruits since it is these pollinated flowers that will attempt to become fruits. Chen uses their own self-collected dataset in Florida using drones or UAVs to capture images of the fruiting strawberries. Using these drones and Faster R-CNN based on ResNet-50 \autocite{rcnn, resnet}.

Transformers as proposed by Vaswani et al \autocite{vaswani2017attention} are state-of-the-art neural network components for sequence-to-sequence problems. Strawberry yield prediction is such a problem thus we are keen to implement and use them in this scenario, having used other methods to varying degrees of success in the past \autocite{ukras2020Pipeline, FHEoNNG}. We also note that in contrast to our previous techniques transformers and their attention heads can help focus the neural network into parts of the data that are most important thus reducing the need for quite as much data compared to equivalently complex neural networks.

In short yield forecasting is essential for improving on food security, and sustainable development \autocite{zhu2022deep}. Yield estimation is difficult due to a lack of data availability and thus a lack of research using modern data-hungry techniques in this domain \autocite{nassar22, jafari2020, gastli2021, chen2019, maskey2019}. Most attempt to solve this data shortfall by using remote sensing, or by using a select few difficult-to-attain datasets like the california commissions data \autocite{zhu2022deep, jafari2020, nassar22}. Few works have applied modern deep learning / neural networks successfully to agriculture, and especially strawberries, the majority use either old neural network forms or don't use neural networks at all.