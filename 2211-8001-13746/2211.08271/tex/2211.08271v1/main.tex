\documentclass[12pt]{article}
\usepackage[utf8]{inputenc}
\usepackage{graphicx}% Include figure files
\usepackage{amsmath}% For math environment
\usepackage{dcolumn}% Align table columns on decimal point
\usepackage{bm}% bold math
\usepackage{xcolor}% colored writing
\usepackage{units}
\usepackage{siunitx}
\sisetup{separate-uncertainty=true}
\sisetup{range-units = single}
%\usepackage{scicite}
\usepackage{times}
\usepackage{amssymb}
\usepackage{verbatim}

%Bibliography
\usepackage[citestyle=numeric,style=nature,maxcitenames=3,maxnames=3,backend=biber,date=year,doi=false,url=false]{biblatex}
\input{bibfiles}
\DeclareUnicodeCharacter{03B3}{$\gamma$}
\DeclareUnicodeCharacter{03B4}{$\delta$}
\DeclareUnicodeCharacter{2212}{$-$}
\DeclareUnicodeCharacter{226A}{$\ll$}
\DeclareUnicodeCharacter{03BE}{$\xi$}


% The preamble here sets up a lot of new/revised commands and
% environments.  It's annoying, but please do *not* try to strip these
% out into a separate .sty file (which could lead to the loss of some
% information when we convert the file to other formats).  Instead, keep
% them in the preamble of your main LaTeX source file.

\usepackage{booktabs}

\newcommand{\differential}{\mathrm{d}}
%\newcommand{\bar}{\mathrm{bar}}


% The following parameters seem to provide a reasonable page setup.

\topmargin 0.0cm
\oddsidemargin 0.2cm
\textwidth 16cm 
\textheight 21cm
\footskip 1.0cm


\title{Multiple-flux-quanta vortices and quasiparticle states in the two-band superconductor Pb}
%\author{Thomas Gozlinski,$^{1\dagger\ast}$ Qili Li,$^{1\dagger}$ Rolf Heid,$^5$ Ryohei Nemoto,$^{2}$ Roland Willa,$^3$\\Toyo Kazu Yamada,$^{2,4}$ Jörg Schmalian,$^{3,5}$ \& Wulf Wulfhekel$^{1,5}$}

 \author
{Thomas Gozlinski,$^{1\ast\mathsection}$ Qili Li,$^{1\ast}$ Rolf Heid,$^5$ Ryohei Nemoto,$^{2}$ Roland Willa,$^3$\\Toyo Kazu Yamada,$^{2,4}$ Jörg Schmalian,$^{3,5}$ and Wulf Wulfhekel$^{1,5}$\\
\\
\normalsize{$^{1}$Physikalisches Institut, Karlsruhe Institute of Technology,}\\
\normalsize{Wolfgang-Gaede-Str.1, 76131 Karlsruhe, Germany}\\
\normalsize{$^{2}$Department of Materials Science, Chiba University,}\\
\normalsize{1-33 Yayoi-cho, Inage-ku, Chiba 263-8522, Japan}\\
\normalsize{$^{3}$Institute for Theory of Condensed Matter, Karlsruhe Institute of Technology,}\\
\normalsize{Wolfgang-Gaede-Str.1, 76131 Karlsruhe, Germany}\\
\normalsize{$^{4}$Molecular Chirality Research Centre, Chiba University,}\\
\normalsize{1-33 Yayoi-cho, Inage-ku, Chiba 263-8522, Japan}\\
\normalsize{$^{5}$Institute of Quantum Materials and Technologies, Karlsruhe Institute of Technology,}\\
\normalsize{Hermann-von-Helmholtz-Platz 1, 76344 Eggenstein-Leopoldshafen, Germany}\\
\\
\normalsize{$^\ast$T.G. and Q.L. contributed equally to this work.}\\
\normalsize{$^\mathsection$E-mail:  thomas.gozlinski@kit.edu.}
}

\date{\today}% It is always \today, today,
             %  but any date may be explicitly specified

\begin{document}

\baselineskip24pt

% Make the title.

\maketitle 

\begin{quote} \bf 
  Superconductors are classified by their behaviour in a magnetic field into type-I, which transition from a superconducting Meissner to a normal metallic state at the critical field and type-II, which have an additional Shubnikov phase of an Abrikosov lattice consisting of vortices with one magnetic flux quantum, each. For type-I superconductors of finite lateral dimensions, the transition to the normal state is, however, known to occur locally in form of domains in the so called intermediate Landau state. 
  Although bulk lead (Pb) is classified as a prototypical type-I superconductor, we surprisingly observe single-flux-quantum and multiple-flux-quanta vortices in the intermediate state using $\SI{}{\milli\kelvin}$ scanning tunneling microscopy. These vortices have a diameter of at least $\approx \SI{100}{\nano\meter}$ and grow in size with the number of confined flux quanta $\Phi_{0} = h/2e$. %changed to SI definition
  By probing the quasiparticle local density of states (LDOS) inside the vortices and comparison with simulations, we identify the Caroli-de Gennes-Matricon states of the two superconducting bands of Pb and use them to determine the winding number of the vortices.
%"Star" shaped LDOS variations in the vortices reflect the Fermi velocity anisotropy of the two lowest bands. Anomalous single-flux vortices emerge as a consequence of the weak coupling between the two superconducting bands.
\end{quote}

\section*{\label{sec:level1}Introduction}
The classical solutions in Ginzburg-Landau theory (GL) allow a thermodynamic classification of superconductors into type-I and type-II. Decisive for their behaviour in magnetic field is the interface energy between the superconducting and normal phase driven by the ratio of the London magnetic penetration depth $\lambda_\mathrm{L}$ and the superconducting coherence length $\xi$. For Ginzburg-Landau parameters $\kappa=\lambda_\mathrm{L}/\xi <1/\sqrt{2}$, type-I behaviour and for $>1/\sqrt{2}$ type-II behaviour was predicted \cite{ginzburg_theory_1950}. A type-I superconductor is characterized by a positive interface energy and an attractive vortex-vortex interaction favouring an intermediate state with large normal domains \cite{landau_theory_1937}. A type-II superconductor is characterized by a negative interface energy and a repulsive vortex-vortex interaction that favours an Abrikosov lattice of single-flux-quantum vortices in the mixed phase \cite{Abrikosov1957}. At the Bogomol'nyi point, $\kappa=1/\sqrt{2}$, which marks the phase transition between the two types of superconductors, vortices behave like non-interacting particles and the vortex configuration is infinitely degenerate \cite{bogomolnyi_stability_1976, jacobs_interaction_1979, weinberg_multivortex_1979, taubes_arbitraryn-vortex_1980}. This degeneracy is predicted to be lifted below $T_c$ and a transitional region in the $(\kappa,T)$ plane opens up, in which the superconductor cannot be categorized into either of the types described above \cite{lukyanchuk_theory_2001, lukyanchuk_rayleigh_2015, vagov_superconductivity_2016}. This regime is expected to host interesting flux patterns like multiple-flux-quanta (or giant) vortices \cite{efanov_exact_1997}. The crucial parameter $\kappa$ can be tuned toward the type-II regime by either an increase of $\lambda_\mathrm{L}$ or reduction of the effective $\xi$. The former is typically realized in thin films below a critical thickness \cite{cody_magnetic_1968, dolan_critical_1973}, the latter is realized by incorporation of impurities (for relevant experiments see Ref.~\cite{lukyanchuk_theory_2001} and references within) or epitaxial growth of the superconductor on a substrate that enhances interface scattering, e.g. Pb/Si(111) \cite{cren_ultimate_2009}. Alternatively, in multiple band superconductors the width of the transitional region was suggested to increase due to differences in the properties of the superconducting bands \cite{vagov_superconductivity_2016} potentially allowing to enter the transitional region in pure bulk samples. In this respect, the multiple band superconductor Pb, which is closest to the Bogomol'nyi point of all elemental superconductors, is a good candidate to study this transitional phase at temperatures well below $T_c$.

Inside the normal-core region of vortices, electronic bound states that lie within the superconducting gap are localized. These in-gap states were first theoretically studied by Caroli, de Gennes and Matricon (CdGM) in 1964 \cite{Caroli1964}. In 1989, Hess et al. confirmed that CdGM states exist in single-flux-quantum vortices by scanning tunneling microscopy (STM) \cite{Hess1989}. For vortices with multiple flux quanta, although studied in thin films with electron holography \cite{Hasegawa1991} and scanning Hall probe microscopy \cite{Ge2013}, their predicted bound states still lack experimental verification \cite{Tanaka1993,Virtanen1999,Virtanen2000,Tanaka2002}. Attempts to study these states in Pb by means of STM/STS have so far been limited to thin islands on Si(111) which are considered dirty limit type-II superconductors \cite{cren_ultimate_2009,cren_vortex_2011}. In this limit the CdGM states are considerably smeared out due to impurity scattering effects resulting in a plateau of differential conductance in the vortex core \cite{Silaev2013,yamane_impurity_2013,masaki_impurity_2015}. 


The energy spacing of discrete CdGM states differing by one $\hbar$ in their orbital angular momentum $\mu$ is of the order of $ \Delta^2/E_F$ and the discrete states form a quasicontinuum or branch of CdGM states for most superconducting materials. 
The CdGM states with low $\mu$ are confined closer to the vortex core than the ones with high $\mu$ thus leading to a translation of the relation between energy and angular momentum $E(\mu)$ to the relation in real space of higher energy states being located further away from the vortex center \cite{gygi_self-consistent_1991}. Strictly speaking, this is only true for a circular screening current, i.e. an isotropic Fermi velocity. In case of a $m$-flux quanta vortex, $m$ individual CdGM branches exist \cite{volovik_topological_2019, Volovik1993}.
STM allows to measure the variation of the local density of states (LDOS) inside the vortex and thus to determine the winding number of the vortex using a topological index theorem: The number of zero-energy crossings of CdGM state branches with varying angular momentum is directly related to the vorticity \cite{volovik_topological_2019, Volovik1993}. This theorem translates to real space and 
the number of zero-energy crossings of CdGM state branches with varying radius from the vortex core is related to the vorticity. 
For bands with anisotropic Fermi velocity, however, this theorem is not straight forward applicable since the radial symmetry is removed and a radial dependent measurement does not ensure crossing all diabolic points \cite{Volovik1993}.

Here, we show that stable single-flux-quantum and multiple-flux-quanta vortices exist in bulk single crystal Pb(111) at $\SI{45}{\milli\kelvin}$. Utilising STM, we measure the quasiparticle LDOS within these vortices.

%We therefore choose to simulate the quasiparticle trajectories within the quasiclassical Eilenberger theory including the Fermi velocity of each band obtained from DFT calculations and compare the obtained LDOS maps to our experimental results.

%Besides, the spatial shape of the zero bias LDOS in and around the vortex exhibits  a "star" shape which was first seen by Hess et al. in $\mathrm{2}$H-NbSe$_\mathrm{2}$ \cite{Hess1990}. It reflects the crystal symmetry and we can rule out other vortices as the origin for this shape due to the  absence of a vortex lattice in our case. With increasing sample bias (independent of sign) the star's arms continuously split into two parallel rays and eventually disappear in the continuum of scattering quasiparticle states as the energy approaches the gap edge $\Delta$. In the semi-classical framework this splitting can be understood as the energy dependent trajectories of CdGM states.

\section*{Results and discussions}

\subsection*{Two-band gaps and intermediate state}

After several cycles of sputtering and annealing, we obtained a clean Pb(111) surface with wide terraces and monoatomic steps, as shown in Fig. \ref{fig:intermediate_state}(a). Upon zero-field cooling the Pb(111) sample to $\SI{45}{\milli\kelvin}$ it enters its superconducting state below $T_c\approx \SI{7.2}{\kelvin}$ \cite{poole_handbook_2000}. Due to our low electronic temperature of less than $\SI{100}{\milli\kelvin}$ \cite{Balashov2018} we are able to resolve the two gaps \cite{Ruby2015} in the density of states by scanning tunnelling spectroscopy, even with a normal conducting tip. Fig. \ref{fig:intermediate_state}(c) shows the differential conductance in the superconducting state as black dots, including a temperature broadened two-gap BCS fit in orange. We determine the superconducting gaps to be $\Delta_1 = \SI{1.26(2)}{\milli e\volt}$ (smaller gap) and $\Delta_2 = \SI{1.40(2)}{\milli e\volt}$ (larger gap) in good agreement with previous measurement of the difference of the two gaps \cite{Ruby2015}. The intensity difference of the two coherence peaks has previously been attributed to the $k$-dependent tunnelling matrix elements and the larger gap has been assigned to the tubular Fermi surface sheet \cite{Ruby2015}. This is in contrast to Bogoliubov-de Gennes (BdG) based Korringa-Kohn-Rostoker (KKR) calculations \cite{Saunderson2020}, which deduced an opposite band-to-gap assignment. As will be discussed below, our study of the CdGM states in the vortices confirms the band-to-gap assignment of Saunderson et al. \cite{Saunderson2020}. We will from here on index the bands and Fermi surfaces according to their superconducting gap, i.e. the tubular Fermi surface responsible for $\Delta_1$ as Fermi surface (FS) 1 and the compact Fermi surface responsible for $\Delta_2$ as Fermi surface (FS) 2 (Fig. \ref{fig:intermediate_state}(c)).

After applying a perpendicular magnetic field of $B = \SI{85}{\milli\tesla}$, which is above the critical field $\mu_0 H_c \sim \SI{80}{\milli\tesla}$ \cite{chanin_critical-field_1972}, magnetic flux enters the sample from the sides and completely destroys superconductivity. Upon decreasing the field again below $H_c$, the Landau intermediate state is reached. It is detected by recording the differential conductance at the coherence peak of $\Delta_2$ while ramping the field down. Once a jump to the superconducting state below the tip is detected, the ramp is stopped. This ensures that one typically finds both, superconducting and normal conducting, areas in the scan range of the STM of $1.4\times \SI{1.4}{\micro\meter^2}$. The intermediate state is characterized by large normal and superconducting domains. The shapes and sizes of these domains in the intermediate state of lead have been extensively studied by magneto-optical methods revealing the strong dependencies
on temperature, sample shape and magnetic protocol \cite{prozorov_topological_2005, prozorov_equilibrium_2007, prozorov_suprafroth_2008, prozorov_dynamic_2009}.

A typical domain wall in the intermediate state is shown in the d$I$/d$U$ map in Fig. \ref{fig:intermediate_state}(b). At a tunnelling bias of $U_t=\SI{1.3}{\milli\volt}$ the normal conducting domains show up as areas of low conductance (purple) and the superconducting domains as areas of high conductance (green/yellow). A cross-sectional line scan across the domain wall, as in Fig. \ref{fig:intermediate_state}(d), shows how both gaps change from zero to their maximum on the length scale of the coherence length. The local recovery of superconductivity agrees well with reported coherence lengths of $\xi=\SI{87}{\nano\metre}$ \cite{poole_handbook_2000}. For a detailed analysis of the coherence length $\xi_{1,2}$ of the two bands measured inside vortices, we refer to the next section. Note that atomic step edges of the surface cause a contrast in d$I$/d$U$ as typically found in STM experiments. 


\subsection*{Single-flux-quantum vortices}

Using the detection method described in the previous section we are also able to find isolated, round normal conducting domains (appearance at $eU=\Delta_2$) of $\approx \SI{100}{\nano\metre}$ in diameter. An example is shown in Fig. \ref{fig:single_flux_vortex}(h). As will be shown later, these are vortices in the superconductor with integer number of flux quanta.
The finding of such small normal conducting domains is surprising considering that the domain wall energy in type-I superconductors is positive and the system thus tries to maximize its domain size. Even more surprising is the fact, that we found this shape after ramping the field down from the critical field. Magneto-optical measurements of cylindrical shaped intermediate state lead samples at $\SI{4.5}{\kelvin}$ reveal that normal domains are only tubular upon increasing magnetic field; after ramping down from the critical field the preferred structure is laminar \cite{prozorov_equilibrium_2007}. A deciding factor for the  intermediate state domain structure on a microscopic scale could be the effect of flux branching \cite{landau_intermediate_1938, landau_theory_1943, conti_branched_2015}. Since the overall domain structure in our experiment, however, consists of domains of various shapes and sizes, we argue that this finding is only consistent with circumstances under which vortices are non-interacting, i.e. a superconductor in the transitional phase (close to the Bogomol'nyi point) at temperatures well below $T_c$. 
The role of pinning of these vortices at bulk defects below the surface remains unclear. Repeating our magnetic protocol several times, the vortices may appear at similar positions suggestive of some kind of pinning. However, we also find the vortices to be mobile when varying the magnetic field (see next section).

%and our test experiment at $\SI{4.3}{\kelvin}$ indicates that the finding of these structures is not impossible but less likely at temperatures close to $T_c$ (see Fig. \ref{fig:4K_vortex}). move to end of paper

To determine the amount of flux carried by the small normal domains, we record differential conductance (d$I$/d$U$) maps at sub-gap energies, which essentially show the LDOS of quasiparticle bound states (CdGM states). At zero bias voltage, we find a threefold symmetric state in form of a star with a maximum in the star's centre (Fig. \ref{fig:single_flux_vortex}(a)) inside the normal domain. The quasiparticle density of states stretches over $\SI{100}{\nano\metre}$ in the $\langle 2\bar{1}\bar{1}\rangle$ directions (blue/green). 
Additional three weak arms (dark blue) along the $\langle 1\bar{2}1\rangle$ directions are visible. With increasing energy (independent of sign of bias voltage) the star's arms split into two with increasing splitting distance, while the central peak splits nearly isotropically to a ring shape (Fig. \ref{fig:single_flux_vortex}(b-f)). For $E\lesssim \Delta_1$ the strong arms are still visible and the ring reaches its maximal size (Fig. \ref{fig:single_flux_vortex}(g)). For $E\sim \Delta_2$ the vortex shows as a relatively round area of low conductance with $\approx \SI{100}{\nano\metre}$ in diameter (Fig. \ref{fig:single_flux_vortex}(h)). We simulated the quasiparticle trajectories inside a vortex carrying one flux quantum within the quasiclassical Eilenberger theory including the Fermi velocity of each band obtained from DFT calculations and compare the obtained LDOS maps to our experimental results (for details, see Methods).  
Right panels in Figure \ref{fig:single_flux_vortex}(a,e) display the solutions reproducing the star shape for the compact Fermi surface 2 and the ring like structure of the tubular Fermi surface 1. 
The simulations confirm that the observed states are the signature of a vortex in Pb(111) containing a single flux quantum.
The ring shaped states, related to the tubular Fermi surface (band 1) of nearly isotropic Fermi velocity, show the expected behaviour. At $U=0$, it creates a sharp maximum in the center of the vortex that splits into a ring of increasing diameter upon variation of the voltage away from $U=0$. For the compact Fermi surface (band 2) with large flat structures in the Fermi surface and anisotropic Fermi velocity, a star shaped structure is predicted whose arms split into two when going away from $U=0$.

%The isotropically splitting ZBP in the vortex core can be understood in the CdGM framework of Gygi and Schlüter \cite{gygi_self-consistent_1991}: The energy spacing of discrete CdGM states is $\sim \Delta^2/E_F$, so below our instrumental resolution, and we thus have a quasicontinuum of CdGM states with different orbital quantum number. We call this quasicontinuum a CdGM state branch. The CdGM states with low angular momentum are confined closer to the vortex core than the ones with high angular momentum number and so we see the relation between energy and angular momentum translated into real space. Strictly speaking, this is only true for a circular screening current and isotropic Fermi velocity. Hence, we expect these quasiparticles to live in the superconducting band of lead which has a compact Fermi surface sheet in the 1. Brillouin zone, that is because the projection of this Fermi surface sheet onto the (111) SBZ is almost circular. And indeed, our DFT calculations confirm that also the Fermi velocity is almost isotropic in the (111) plane. This leads us to the star shape.

Star shaped CdGM states have first been found by Hess et al. \cite{Hess1990} in 2H-NbSe$_2$, which due to the crystal symmetry have sixfold rotational symmetry. Pb crystallizes in the fcc structure and belongs to the point group $Fm\overline{3}m$. The electronic structure therefore only carries a threefold rotational symmetry about the [111] axis. Due to the discrete rotational symmetry, the angular momentum is not a good quantum number and the states mix. In general, the mixing results in states with different lateral confinement for different angles. At zero bias, the low angular momentum states carry the largest weight and lead to a maximum in the center of the vortex. At higher bias, states of larger angular momentum become increasingly important leading to a movement of the maxima away from the center, i.e. the star splits.  The large variation in quasiparticle localization depending on the angle from $\sim \SI{10}{\nano\metre}$ to over $\SI{100}{\nano\metre}$ agrees with the strong anisotropy of the 3D Fermi velocity in the compact band. 

% this section can be either here or in the supplementary
In fact, star shaped vortices with long arms have been recently observed in the Abrikosov lattice of La(0001) \cite{kim_anisotropic_2021}, owing to the large anisotropy of the responsible band's in-plane Fermi velocity. 
In Pb with its two bands, one may expect that the individual gaps will close independently when approaching the vortex center. Fig. \ref{fig:angle_meas_normal} displays the evolution of the gaps, i.e. the energy of the coherence peaks which appear as bright lines in the second derivative of d$I$/d$U$ with respect to $U$, as function of the distance from the vortex center and direction. For large distances from the center, the gaps $\Delta_1$ and $\Delta_2$ decrease towards the center in parallel with roughly the same length scale $\xi_1\sim \xi_2 \sim \SI{45}{\nano\metre}$. In the whole range, $\Delta_2$ follows a simple tanh function. At 50 nm, however, $\Delta_1$ crosses $\Delta_2$ and stays larger than $\Delta_2$. Thus, for the band 1, the gap size deviate from the tanh function near the center and decreases on a shorter length scale. This observation confirms the shrinking of the core known as the Kramer-Pesch effect \cite{kramer_core_1974}. It was quantified by numerical calculations by Gygi and Schlüter \cite{gygi_self-consistent_1991} for the vortex core size of type-II superconductors.
The slope of $\Delta_1$ near the core corresponds to a core size $\xi^{(c)}_1 = \Delta_1(\infty)\left[\lim_{r\rightarrow 0} \frac{\mathrm{d}\Delta_1(r)}{\mathrm{d}r}\right]^{-1} $of only $ \sim \SI{10}{\nano\metre}$. 
In our self-consistent calculation of the pair-potential $\Delta(r)$ for an isotropic vortex in the quasiclassical theory (see Supplementary Material), this Kramer-Pesch shrinking effect is also present and leads to substantial deviation from a tanh function with one universal $\xi$. 
Note, that theory in the clean limit, however, predicts a core shrinking proportional to $T/T_c$ when lowering the temperature. At very low temperatures $T\ll T_c$ the slope of the order parameter d$\Delta$/d$r$ at the vortex centre is even predicted to become infinite, which would show as a jump of $\Delta(r)$ at $r=0$ that is smoothed out over the distance $\xi T/T_c$ \cite{volovik_vortex_1993}. 
Experimentally, we do not find this extreme shrinking. 
%The shrinking is, however, limited by impurity scattering in the dirty limit \cite{kramer_local_1974, kramer_core_1974}, i.e. the  coherence length is much larger than the electron mean free path. 
The "squeezing" of low angular momentum states in the vortex core and thus the Kramer-Pesch effect is absent for the star, i.e. CdGM states of band 2%and surprisingly also for the CdGM states of band 1 in the anomalous vortex (see Fig. \ref{fig:anomalous_vortex})
. Consequently, $\xi^{(c)}_1$ deviates from $\xi^{(c)}_2$ which indicates that intraband coupling dominates over interband coupling, i.e. the two bands are sufficiently decoupled from each other, despite the gap sizes being not too different \cite{ichioka_locking_2017}. 

\subsection*{Anomalous single-flux-quantum vortices}

Besides the regular vortex situation, we find vortices, in which the two sets of CdGM states are laterally displaced. Figure \ref{fig:anomalous_vortex} shows such an anomalous vortex.
Both the ring centre and the star centre of the CdGM states are independently movable by a change in magnetic field and their relative displacement can be manipulated into different configurations, even back to the normal configuration from Figure \ref{fig:single_flux_vortex} (see Supplementary Material). We explain this by two effects. First, a change in magnetic field laterally moves the vortex and with it, the two sets of CdGM states. Second, a change in magnetic field can lead to a tilting or bending of the flux lines away from a normal direction to the surface. This leads to a breaking of the cylinder symmetry and can displace the two sets of CdGM states relative to each other. The individual sets of CdGM states behave as those of the regular vortices, except for their relative displacement (see Figure \ref{fig:anomalous_vortex} (a-h)). 

Interestingly, the lateral displacement allows an independent probing of the state sets and thus, an independent identification of their bands. By looking at single bias spectra at the star centre (black) and the ring centre (red) in Figure \ref{fig:anomalous_vortex}(i) we find that the amplitude of the peak at zero bias is three times larger for the ring than for the star, which is supported by our separate band simulations from earlier and can be explained by the larger lateral confinement of low angular momentum states in band 1 compared to band 2. Figures \ref{fig:anomalous_vortex}(j,k) reveal maxima in the LDOS along the cross-sections marked in the right panel of (i) in the form of heat maps of the differential conductance's second derivative with respect to bias voltage. It becomes apparent that $\Delta_2$ (cyan line) closes entirely whereas $\Delta_1$ (red line) does not completely close in the star's centre (black circle). Instead, $\Delta_1$ (red line) closes about $\SI{20}{\nano\metre}$ away from the star centre (red circle). Consequently, the superconducting gap $\Delta_1$ is linked to the ring and $\Delta_2$ to the star. The fact that the quasiparticles of $\Delta_1$ and $\Delta_2$ can be independently displaced with respect to each other supports our argument of a low interband coupling between the two superconducting bands.



\subsection*{Multiple-flux-quanta vortices}

We also observed larger vortices in the experiments. Figure \ref{fig:multi_flux_vortex} displays two examples. (a-c) shows a vortex with two flux quanta. Our semiclassical calculations indicate that for each flux quantum and each band, a branch of CdGM states is present. As a result, for the vortex with two flux quanta, the band 2 causes a structure with two arms per direction at zero bias (see Fig. \ref{fig:multi_flux_vortex}a) that individually split into two arms with bias voltage (see Fig. \ref{fig:multi_flux_vortex}b) just as in the single-flux-quantum vortex. An analogous behaviour is observed in the vortex with three flux quanta shown in Fig. \ref{fig:multi_flux_vortex}(d-f). Both vortices are larger than the single flux quantum vortex (compare Fig. \ref{fig:single_flux_vortex}h with Fig. \ref{fig:multi_flux_vortex}(c,f)). Further, they deviate from a round shape and laterally grow with the number of flux quanta. The number of flux quanta in the vortices does not seem to be limited. We observed giant vortices with over 10 flux quanta (see Supplementary Material). For band 1, the problem of multiple-flux-quanta vortices is similar to that of a single band superconductor with a spherical Fermi surface and has been studied in detail by Volovik {\it et al.} \cite{Volovik1993}. In essence, an $m$-flux quanta vortex results in $m$ branches of circular CdGM states of different radii. Due to symmetry, at zero bias and for odd $m$, a central spot is formed by one branch of the CdGM states and the other states form pairs of increasing ring diameters. The CdGM states at energies away from the Fermi level evolve by changing the radii. Thus, the central CdGM states turns from a spot of zero radius to a ring of finite radius, and the pairs of CdGM states split in their radius.
For an even $m$, no central spot is present at the Fermi energy but only pairs of CdGM states with distinct radii exist, that again split when moving away from the Fermi energy. This is the essence of the topological index number theory of Volovik.
In our case, we find a central spot of high differential conductance at zero bias voltage for the vortices with $m=1$ and $m=3$, while there is no central spot for the vortex with $m=2$ in agreement with the topological index number theory. When going away from the Fermi energy, the ring-shaped and spot-like CdGM states overlap with the star shaped states such that their distinction becomes impractical. Tunneling spectra at selected locations inside the vortex showing the very same effects are shown in Figure \ref{fig:multi_flux_vortex}(g,h).   



At last, we tested whether vortices would also be present at temperatures much closer to $T_c$. Our measurements at $\SI{4.3}{\kelvin}$ revealed that it is substantially harder to trap a vortex in our scan frame, but we managed to in a single case (see Supplementary Material).

\section*{Conclusions}
In conclusion, we report the observation of multiple-flux-quanta vortices in a traditional type-I bulk superconductor, i.e. single crystal Pb(111), by low-temperature scanning tunneling microscopy. 
We demonstrate a robust determination method for the winding number of the vortex by usage of a topological index theorem which relates the number of flux quanta and CdGM state branches.
The spatial anisotropy of the quasiparticle states inside the vortex reflects the crystal symmetry, and its shape is governed by the anisotropic Fermi velocity in the superconducting bands. An influence of neighboring flux lines can be ruled out due to the absence of an ordered flux pattern. In addition, we could show how CdGM states from two weakly coupled bands interact with each other in a single flux line. The emergence of such vortices in the intermediate state of a type-I superconductor is very surprising and motivates further studies on this subject.

\section*{Methods}
\paragraph{Experimental details}
The experiments were performed with a home-built scanning tunneling microscope with dilution refrigeration, which can reach a base temperature of $\SI{25}{\milli\kelvin}$ in a magnetic field of up to $\SI{7.5}{\tesla}$ \cite{Balashov2018}. 
%Nanonis by SPECS Zurich GmbH is used as the real-time control unit operating motors and converting analog signals. 
In our setup, the bias voltage $U_t$ is applied between sample and common machine ground so that a positive bias voltage probes the unoccupied states of the sample. The STM chamber is kept at a base pressure of $\SI{2e-10}{\milli\bar}$. The single crystal Pb(111) (miscut angle: $\pm 0.1^\circ$,  purity: $99.999\%$) has been purchased from MaTecK GmbH. It has cylindrical (hat) shape with a diameter of $\SI{8}{\milli\metre}$ and a thickness of $\SI{2}{\milli\metre}$. At a base pressure of $\SI{1e-10}{\milli\bar}$ the Pb crystal was prepared in cycles of sputtering with Ar$^+$ ion of $\SI{3}{\kilo\electronvolt}$ and subsequent annealing at $190^{\circ}$C and directly transferred into the STM in-situ. A tungsten tip was prepared by high-temperature flashing and soft dipping into a Au(111) surface in order to avoid picking up Pb atoms. The measurements (except for the one at $\SI{4.3}{\kelvin}$%, shown in Fig. \ref{fig:4K_vortex}
) were all performed below $\SI{45}{\milli\kelvin}$. After zero-field cooling the Pb crystal, the vortices were formed by ramping the perpendicular magnetic field from $\SI{0}{\milli\tesla}$ to $\SI{85}{\milli\tesla}$ and back down to a constant value below the critical field. Note that $\SI{80}{\milli \tesla}$ is the critical magnetic field for Pb \cite{chanin_critical-field_1972}. 
%Due to the switching experiments at this machine the magnet would regularly be used up to $\pm \SI{400}{\milli \tesla}$. By performing a Hall probe measurement, we can estimate the uncertainty of the magnetic field by hysteresis effects in the magnet to $\delta B = \SI{4.5}{\milli\tesla}$. The sample plate, on which the Pb crystal is mounted, is made out of molybdenum. Molybdenum is a type I superconductor, which in its purest form has a critical temperature $T_c\sim 900-958\,\SI{}{\milli\kelvin}$ \cite{mallon_measurements_1967} and a critical field of $H_c=\SI{99.7}{G}$ \cite{hein_critical_1963}. Thus, even for an almost iron-free molybdenum sample holder, lead would be the stronger superconductor and mainly responsible for flux trapping. 
The differential conductance was measured using a Lock-in amplifier at a frequency of $3.4-3.6\,\SI{}{kHz}$ and AC peak amplitude $U_\mathrm{ac}^\mathrm{PK}$ between $10$ and $\SI{100}{\micro\volt}$ (for details, see Supplementary Material). d$I$/d$U$ maps at sub-gap energies were recorded in a multi-pass configuration: In the ``record" phase the tip records the $z$-profile at the feedback condition (constant tunnelling current of $I_t$ at $eU_t>\Delta$) and in the ``play" phase the $z$-profile is repeated at a different bias voltage. In order to increase the signal for sub-gap energies an offset is often added to the $z$-profile bringing the tip closer to the surface. This record and play phase alternation is performed line for line until the entire area has been scanned.

\paragraph{Calculation details}
In order to obtain the simulated LDOS for vortices containing an arbitrary number of flux quanta, we used the Riccati parametrisation of the quasiclassical 3D Eilenberger equations as proposed in Ref. \cite{schopohl_transformation_1998} and numerically solved the one-dimensional differential equations under appropriate boundary conditions (for details, see Supplementary Material) for each band separately. Motivated by our experimental findings, we assumed a radial symmetric local pair potential $\Delta(r)$ in the plane perpendicular to a vortex line with s-wave symmetry that vanishes in the vortex centre. We used the ratio $\Delta_2(\infty)/\Delta_1(\infty)$ obtained from the experiment. We respected the broken translation symmetry at the crystal surface by a work function term. The magnetic vector potential was set to zero for all calculations shown in the main text. An inclusion of a magnetic vector potential of appropriate form only yielded small quantitative deviations from the zero-field case (see Supplementary Material), yet drastically increased the required computation time, which is why we refrained from it for the LDOS maps. 

Density functional calculations of the electronic structure of Pb
were carried out in the framework of the mixed-basis pseudopotential
method \cite{meyer97}. The electron-ion interaction was
represented by norm-conserving relativistic pseudopotentials \cite{vande85}. Spin-orbit coupling was incorporated within the pseudopotential scheme via Kleinman's formulation \cite{kleinman80} and was consistently taken into account in the charge self-consistency cycle using a spinor representation of the wave functions. Further details of the spin-orbit coupling implementation within the mixed-basis pseudopotential method can be found in a previous publication \cite{heid10}. For higher accuracy, 5$d$ semicore states were included in the valence space.  The deep $d$ potential is efficiently treated by the mixed-basis approach, where valence states are
expanded in a combination of plane waves and local functions.  Here,
local functions of $d$ type at the Pb sites were combined with plane
waves up to a kinetic energy of 20~Ry.  Brillouin-zone integration
was performed by sampling a 32$\times$32$\times$32 $k$-point mesh (corresponding to 2992 $k$ points in the irreducible part of the Brillouin zone) in conjunction with a Gaussian broadening of 0.2~eV. The
exchange-correlation functional was represented by the local-density
approximation in the parameterization of Perdew-Wang \cite{perdew92}.

This DFT technique was applied to obtain Fermi surface properties entering the Eilenberger equations.
Band energies were calculated on fine radial grids for a cylindrical coordinate system taking the [111] direction as the z-axis, to determine Fermi momenta $k_F$ for each of the two relevant bands.
At each $k_F$, 3-dimensional Fermi velocities $v_F$ were then calculated taking numerical derivatives of band energies around this point.
The optimized lattice parameter $a=4.89$~\AA\, was used throughout.

\section*{Author contributions}
T.G. and Q.L. conducted the experiments, performed the quasiclassical (Eilenberger) calculations and analysed the data. T.G. was responsible for presentation of the data. R.H. performed the DFT calculations. R.N. and T.K.Y. performed initial experiments and provided the crystal. R.W. and J.S. gave theoretical support. W.W. headed the study. T.G. and W.W. wrote the manuscript including input from all authors.
\section*{Acknowledgments}
We acknowledge funding from the Deutsche Forschungsgemeinschaft (DFG) with grant Wu 394/12-1 and discussions with A. Ustinov and S. Rex. R.H. acknowledges support by the state of Baden-Württemberg through bwHPC. This work was supported by JSPS KAKENHI under the grant number 17K19023.

%\bibliography{main}

%\bibliographystyle{Science}

\printbibliography

\clearpage

%%%%%%%%%%%%%%%%%%%%%%%%%%%%%%%%%%%%%%%%%%%%%%
%%%%%%%%%%%%%%%%%%% Appendix %%%%%%%%%%%%%%%%%
%%%%%%%%%%%%%%%%%%%%%%%%%%%%%%%%%%%%%%%%%%%%%%


%%%%%%%%%%%%%%%%%%%%%%%%%%%%%%%%%%%%%%%%%%%%%%
%%%%%%%%%%%%%%%%%%% Figures %%%%%%%%%%%%%%%%%%
%%%%%%%%%%%%%%%%%%%%%%%%%%%%%%%%%%%%%%%%%%%%%%
\clearpage

\begin{figure}
    \centering
    \includegraphics[width=\textwidth]{Intermediate_state_theory_v3.pdf}
    \caption{\textbf{Superconducting properties and intermediate state.} (a) Topographic scan image of the Pb(111) surface after the cleaning procedure. (b) d$I$/d$U$ map at $U_t=\SI{1.3}{\milli e\volt}$ showing a typical domain wall in the intermediate state at $B=\SI{23}{\milli\tesla}$. Right overlay: Corresponding topographic scan image. (c) Differential conductance in the superconducting state (or superconducting domain) (black points) including a two-gap BCS fit (orange) with $\Delta_1$ being the smaller and $\Delta_2$ being the larger gap. Inset: 3D models of Fermi surface sheets 1 and 2 (from Ref. \cite{choy_database_2000}) responsible for superconducting gaps $\Delta_1$ and $\Delta_2$ respectively. (d) d$I$/d$U$ spectra along a cross-section from normal conducting to superconducting domain. The area is marked in white in (a) and the profile direction is indicated by a red arrow. Individual spectra are offset with respect to each other by $\SI{0.7}{\micro S}$ The spectra were locally averaged over a straight part of the domain boundary and recorded in distance increments of $\Delta d=\SI{9.19}{\nano\metre}$.}
    \label{fig:intermediate_state}
\end{figure}

\begin{figure}
    \centering
    \includegraphics[width=\textwidth]{Single_flux_vortex_theory3d_v2.pdf}
    \caption{\textbf{Single-flux-quantum vortex signature.} (a-h) d$I/$d$U$ maps (left panel) and simulated LDOS of bands 1 and 2 (right panels) of a single-flux-quantum vortex at different bias voltages displaying the quasiparticle density at the surface. (a) The inset shows a 2D-FFT filtered topographic image of the Pb(111) lattice. Bulk crystal directions (red/yellow arrows) have been determined from glide planes.}
    \label{fig:single_flux_vortex}
\end{figure}

\begin{figure}
    \centering
    \includegraphics[width=\textwidth]{Angle_meas_normal.pdf}
    \caption{\textbf{Coherence lengths in the normal single-flux-quantum vortex.} Angle dependent radial measurement of the vortex core states from the vortex in Fig. \ref{fig:single_flux_vortex}. Displayed is the second derivative of the differential conductance $-\partial^2 \sigma /\partial U^2$ in order to highlight maxima in the LDOS. The point distance of single spectra is $\Delta d=\SI{2.57}{\nano\metre}$. Marked are the opening of $\Delta_1$ (red line), the opening of $\Delta_2$ (cyan line) and the CdGM states of band 2 (cyan dashed line). The inset shows the direction of the y-axis.}
    \label{fig:angle_meas_normal}
\end{figure}

\begin{figure}
    \centering
    \includegraphics[width=\textwidth]{Anomalous_vortex_v2.pdf}
    \caption{\textbf{Anomalous vortex signature.} (a-h) d$I/$d$U$ maps at different bias voltage for an anomalous vortex like Fig. \ref{fig:single_flux_vortex}. (i) Right panel: Enlarged zero bias d$I$/d$U$ map of the anomalous vortex including the position of line spectra (white arrows) and single spectra locations (red/black circle). Left Panel: Single d$I$/d$U$ spectra in the star centre (black) and ring centre (red) revealing ZBPs of different amplitude. (j,k) Heat maps of the second derivative of the differential conductance $-\partial^2 \sigma/\partial U^2$ along the cross-sections marked in (i).
    Red and cyan lines follow $\Delta_1$ and $\Delta_2$ respectively.
    }
    \label{fig:anomalous_vortex}
\end{figure}

\begin{figure}
    \centering
    \includegraphics[width=\textwidth]{Multi_flux_vortex_theory3d.pdf}
    \caption{\textbf{Multiple-flux-quanta vortex signature.} d$I$/d$U$ maps (left panel) and simulated LDOS of the bands 1 and 2 (right panels) of an $m=2$ (a-c) and $m=3$ (d-f) vortex. (g,h) Bias spectroscopies (bottom panel) recorded with the tip at specific locations marked in the zero bias maps (top panel).}
    \label{fig:multi_flux_vortex}
\end{figure}

\end{document}