% ****** Start of file apssamp.tex ******
%
%   This file is part of the APS files in the REVTeX 4.2 distribution.
%   Version 4.2a of REVTeX, December 2014
%
%   Copyright (c) 2014 The American Physical Society.
%
%   See the REVTeX 4 README file for restrictions and more information.
%
% TeX'ing this file requires that you have AMS-LaTeX 2.0 installed
% as well as the rest of the prerequisites for REVTeX 4.2
%
% See the REVTeX 4 README file
% It also requires running BibTeX. The commands are as follows:
%
%  1)  latex apssamp.tex
%  2)  bibtex apssamp
%  3)  latex apssamp.tex
%  4)  latex apssamp.tex
%
\documentclass[%reprint,
%superscriptaddress,
%groupedaddress,
%unsortedaddress,
%runinaddress,
%frontmatterverbose, 
preprint,
%preprintnumbers,
%nofootinbib,
%nobibnotes,
%bibnotes,
 amsmath,amssymb,
 aps,prx,superscriptaddress,
 longbibliography
%pra,
%prb,
%rmp,
%prstab,
%prstper,
%floatfix,
]{revtex4-2}
%\documentclass[12pt]{article}
\usepackage[utf8]{inputenc}
\usepackage{graphicx}% Include figure files
\usepackage{amsmath}% For math environment
\usepackage{dcolumn}% Align table columns on decimal point
\usepackage{bm}% bold math
\usepackage{xcolor}% colored writing
\usepackage{siunitx}
\sisetup{separate-uncertainty=true}
\sisetup{range-units = single}
\usepackage{times}
\usepackage{amssymb}
\usepackage{verbatim}
\usepackage[hidelinks]{hyperref}
\hypersetup{
  colorlinks   = true, %Colours links instead of ugly boxes
  urlcolor     = blue, %Colour for external hyperlinks
  linkcolor    = blue, %Colour of internal links
  citecolor   = blue %Colour of citations
}
%Bibliography
%\usepackage[citestyle=numeric,style=natbib,maxcitenames=10,maxnames=10,backend=biber,date=year,doi=false,url=false]{biblatex}
%\input{bibfiles}
\DeclareUnicodeCharacter{03B3}{$\gamma$}
\DeclareUnicodeCharacter{03B4}{$\delta$}
\DeclareUnicodeCharacter{2212}{$-$}
\DeclareUnicodeCharacter{226A}{$\ll$}
\DeclareUnicodeCharacter{03BE}{$\xi$}

\usepackage{booktabs}

\newcommand{\differential}{\mathrm{d}}
\DeclareSIUnit\bar{bar}

\begin{document}

\title{Supplementary Material to ``Identification of multiple-flux-quanta vortices by core states in the two-band superconductor Pb''}

\author{Thomas Gozlinski}
\email{thomas.gozlinski@kit.edu}
\thanks{These authors contributed equally to this work.}
\affiliation{Physikalisches Institut, Karlsruhe Institute of Technology,
Wolfgang-Gaede-Str.1, 76131 Karlsruhe, Germany}
\author{Qili Li}
\thanks{These authors contributed equally to this work.}
\affiliation{Physikalisches Institut, Karlsruhe Institute of Technology,
Wolfgang-Gaede-Str.1, 76131 Karlsruhe, Germany}
\author{Rolf Heid}
\affiliation{Institute for Quantum Materials and Technologies, Karlsruhe Institute of Technology, Hermann-von-Helmholtz-Platz 1, 76344 Eggenstein-Leopoldshafen, Germany}
\author{Ryohei Nemoto}
\affiliation{Department of Materials Science, Chiba University,
1-33 Yayoi-cho, Inage-ku, Chiba 263-8522, Japan}
\author{Roland Willa}
\affiliation{Institute for Theory of Condensed Matter, Karlsruhe Institute of Technology,
Wolfgang-Gaede-Str.1, 76131 Karlsruhe, Germany}
\author{Toyo Kazu Yamada}
\affiliation{Department of Materials Science, Chiba University,
1-33 Yayoi-cho, Inage-ku, Chiba 263-8522, Japan}
\affiliation{Molecular Chirality Research Centre, Chiba University,1-33 Yayoi-cho, Inage-ku, Chiba 263-8522, Japan}
\author{Jörg Schmalian}
\affiliation{Institute for Quantum Materials and Technologies, Karlsruhe Institute of Technology, Hermann-von-Helmholtz-Platz 1, 76344 Eggenstein-Leopoldshafen, Germany}
\affiliation{Institute for Theory of Condensed Matter, Karlsruhe Institute of Technology,
Wolfgang-Gaede-Str.1, 76131 Karlsruhe, Germany}
\author{Wulf Wulfhekel}
\affiliation{Physikalisches Institut, Karlsruhe Institute of Technology,
Wolfgang-Gaede-Str.1, 76131 Karlsruhe, Germany}
\affiliation{Institute for Quantum Materials and Technologies, Karlsruhe Institute of Technology, Hermann-von-Helmholtz-Platz 1, 76344 Eggenstein-Leopoldshafen, Germany}


%\collaboration{
%$^\ast$T.G. and Q.L. contributed equally to this work.\\
%$^\mathsection$E-mail:  thomas.gozlinski@kit.edu.
%}

\date{\today}% It is always \today, today,
             %  but any date may be explicitly specified

% Make the title.

\maketitle 

\clearpage

%Name Figures in this appendix S1,S2,..
\renewcommand{\thefigure}{S\arabic{figure}}
\setcounter{figure}{0} 
\renewcommand{\thetable}{S\arabic{table}}
\setcounter{table}{0} 

%%%%%%%%%%%%%%%%%%%%%%%%%%%%%%%%%%%%%%%%%%%%%%
%%%%%%%%%%%%%%%% Supplementary %%%%%%%%%%%%%%%
%%%%%%%%%%%%%%%%%%%%%%%%%%%%%%%%%%%%%%%%%%%%%%

\section*{Contents}
\noindent Extended Methods\\
Supplementary Notes 1-5\\

\section*{Extended methods}


\paragraph{STM measurement conditions}
Tab. \ref{tab:stm_conditions} summarizes the most important STM/STS measurement parameters used to obtain the figures in the main text and supplementary material.

\begin{table}[hpb]
    \centering
    \caption{\textbf{Measurement parameters.} $I_t$ denotes the feedback condition of the tunnelling current at the bias voltage $U_t$. $U_\mathrm{ac}^\mathrm{PK}$ is the peak AC voltage amplitude added via the Lock-in amplifier. In multi-pass d$I$/d$U$ maps $z_\mathrm{off}$ is the distance the tip is brought closer to the surface compared to the feedback condition. $B$ is the magnetic flux density in vacuum and $T$ is the temperature.}
    \setlength\extrarowheight{-3pt}
    \begin{tabular}{l c| c c c c c c}
        \toprule
         Fig.& & $I_t(\SI{}{\nano\ampere})$ & $U_t(\SI{}{\milli\volt})$ & $U_\mathrm{ac}^\mathrm{PK}(\SI{}{\micro\volt})$ &  $z_\mathrm{off}(\SI{}{\pico\metre})$ & $B(\SI{}{\milli\tesla})$ & $T(\SI{}{\milli\kelvin})$\\ \hline
         %\ref{fig:intermediate_state}
         1& A & $0.2$ & $1$ & - & - & - & $<45$\\
         & B & $0.05$ & $1.3$ & $50$ & - & $23$ & $<45$\\
         & C & $1$ & $3$ & $20$ & - & $0$ & $39$\\
         & D & $2$ & $3$ & $20$ & - & $23$ & $48$\\\hline
         %\ref{fig:single_flux_vortex}
         2& A-G & $0.5$ & $1.4$ & $50$ & $20$ & $0$ & $<45$\\
         & H & $0.5$ & $1.4$ & $50$ & - & $0$ & $<45$\\\hline
         %\ref{fig:angle_meas_normal}
         3& & $0.5$ & $3$ & $20$ & - & $0$ & $<45$\\\hline
         %\ref{fig:anomalous_vortex}
         4& A-G & $1$ & $1.4$ & $50$ & $20$ & $19$ & $<45$\\
         & H & $1$ & $1.4$ & $50$ & - & $19$ & $<45$\\
         & I & $1$ & $3$ & $20$ & - & $19$ & $40$\\
         & J & $1$ & $1.5$ & $10$ & - & $19$ & $40$\\
         & K & $1$ & $2$ & $20$ & - & $19$ & $40$\\\hline
         %\ref{fig:multi_flux_vortex}
         5& A-B & $0.5$ & $1.4$ & $50$ & $20$ & $0$ & $<45$\\
         & C & $0.5$ & $1.4$ & $50$ & - & $0$ & $<45$\\
         & D-E & $1$ & $1.4$ & $50$ & $20$ & $33$ & $<45$\\
         & F & $1$ & $1.4$ & $50$ & - & $33$ & $<45$\\
         & G & $1$ & $3$ & $20$ & - & $0$ & $<45$\\
         & H & $1$ & $3$ & $20$ & - & $33$ & $<45$\\\hline
         \ref{fig:anomolous_b_field_dep}& A-D & $1$ & $1.4$ & $50$ & - & IND$^a$& $<45$\\\hline
         \ref{fig:anomolous_to_normal}& A-B & $0.2$ & $1.4$ & $50$ & - & IND & $<45$\\
         & C-D & $1$ & $1.4$ & $20$ & - & IND & $<45$\\\hline
         \ref{fig:giant_flux_vortex}& A-B & $1$ & $3$ & $50$ & $20$ & $0$ & $<45$\\\hline
         \ref{fig:vortex_neg_Bfield_or_bias}& A-B & $0.5$ & $1.4$ & $50$ & - & IND & $36$\\
         & C-D & $1$ & $1.4$ & $20$ & - & IND & $42$\\\hline
         \ref{fig:4K_vortex}& A & $0.1$ & $1.8$ & $50$ & - & $0$ & $4300$\\
         & B & $0.1$ & $1.8$ & $50$ & $20$ & $0$ & $4300$\\
         & C & $0.1$ & $6$ & $100$ & - & $0$ & $4300$\\
    \end{tabular}
    \label{tab:stm_conditions}
    \\\footnotesize{$^a$ This parameter is indicated in the sub-figure itself.} 
\end{table}

\paragraph{3D Eilenberger calculations}
We follow Ref. \cite{schopohl_transformation_1998} in notation and describe how we solved the quasiclassical Eilenberger equations \cite{eilenberger_transformation_1968, larkin_quasiclassical_1969}
\begin{align}
    -\hbar \bm{v}_F \nabla \hat{g}(\bm{r};\bm{p}_F, i \epsilon_n) = 
    \left[
    \begin{pmatrix}
        i \epsilon_n + \bm{v}_F e\bm{A}(\bm{r}) & -\Delta (\bm{r},\bm{p}_F) \\
        \Delta^\dagger (\bm{r},\bm{p}_F) & -i \epsilon_n - \bm{v}_F e\bm{A}(\bm{r})
    \end{pmatrix},
     \hat{g}(\bm{r};\bm{p}_F, i \epsilon_n)
    \right],
    \label{eq:3DEilenberger/Eilenberger transport eq}
\end{align}
that hold for $k_F\xi\gg 1$.
The quasiclassical Green's function propagator
\begin{align}
    \hat{g} = \begin{pmatrix}
       g(\bm{r},\bm{p}_F,i \epsilon_n) & f(\bm{r},\bm{p}_F,i \epsilon_n)\\
       -f^\ast(\bm{r},-\bm{p}_F,i \epsilon_n) & g^\ast(\bm{r},-\bm{p}_F,i \epsilon_n)
    \end{pmatrix},
\end{align}
depends on spatial coordinate $\bm{r}$, crystal momentum $\bm{p}_F=\hbar \bm{k}_F$ and energy $\epsilon_n$, and must satisfy the normalization condition $\hat{g}^2 = \hat{1}$. $g$ and $f$ are normal and anomalous quasiclassical Green's function propagators and $\epsilon_n$ are fermionic Matsubara frequencies. Incorporated in this formalism is the self-consistent calculation of pair potential $\Delta(\bm{r})$ and $\bm{A}(\bm{r})$, which is essential for the description of a superconductor hosting vortices as these two fields are dependent on each other. Although this approach is numerically much more feasible than direct diagonalization of a BdG Hamiltonian when treating inhomegeneties, it is not necessary to solve the problem completely self-consistently in this work. The reason for that is, that our experimental results clearly show, that even though the sub-gap states in the vortex display highly anisotropic behaviour, the recovery of $\Delta(\bm{r})$ is isotropic in the plane around the vortex core. The local pair potential is assumed to have s-wave symmetry and was therefore modeled by 
\begin{align}
    \Delta(\bm{r},\bm{p}_F) = \Delta(\bm{r}) = \left(\Delta_0\tanh \frac{r}{\xi} + \Theta(z) W\tanh \frac{z}{a}\right)\left(\frac{x+i y}{r}\right)^m, 
\end{align}
with $\Theta$ being a Heaviside step function, $\Delta_0$ the maximum gap size, $\xi$ the coherence length, $a$ the lattice constant, $m$ the winding number of the vortex and $W$ the work function. The second term ensures, that quasiparticles travelling to the surface are decaying into the vacuum with the right damping factor. Since an isotropic $\Delta$ implies an isotropic in-plane current density, the magnetic field profile around the vortex was described by a vector potential of the form \cite{carneiro_vortex_2000}
\begin{align}
    \bm{A} &= A(r,z) \hat{\bm{e}}_\varphi \,,\\
    A(r,z) &= \frac{m\Phi_0}{2\pi\lambda^2}\int\limits_0^\infty \mathrm{d}k \frac{J_1(kr)}{k^2+\lambda^{-2}}S(k,z) \,,\label{eq:app2/Vector_potential}\\
    S(k,z) &= 
    \begin{cases}
        \frac{\kappa}{k+\kappa}\mathrm{e}^{-kz} & z>0\\
        1-\frac{k}{k+\kappa}\mathrm{e}^{\kappa z} & z\leq 0 \,,
    \end{cases}
\end{align}
that is cylinder symmetric in the bulk and deviates from the bulk value near and above the surface. Here, $\kappa = \sqrt{k^2+\lambda^{-2}}$, $\lambda$ is the magnetic penetration depth which was chosen to be $\lambda=\xi/\sqrt{2}$ and $J_1(x)$ is a Bessel function of first order.

Nils Schopohl and Kazumi Maki \cite{schopohl_quasiparticle_1995} could show that the Eilenberger equations can always be solved along a characteristic line and that the solution is universal for each point along this line. The line simply has to be parallel to the Fermi velocity vector $\bm{v}_F$. Points on this line are then characterized by the variable $X$ and two \textit{impact parameters} $Y$ and $Z$. 
\begin{align}
    \bm{r}(X) &= X \hat{\bm{u}} + Y \hat{\bm{v}} + Z \hat{\bm{w}},\\
    &= x \hat{\bm{x}} + y \hat{\bm{y}} + z \hat{\bm{z}} 
\end{align}
Transformation from the mobile frame of reference $(\hat{\bm{u}},\hat{\bm{v}},\hat{\bm{w}})$, where $\hat{\bm{u}}\parallel \bm{v}_F$, to the fixed coordinate system $(\hat{\bm{x}},\hat{\bm{y}},\hat{\bm{z}})$ is done by chaining rotation matrices using Euler angles:
\begin{align}
    \begin{pmatrix}
        v_x\\ v_y\\ v_z
    \end{pmatrix}&=
    R_{zyz}
    \begin{pmatrix}
        v_X\\ v_Y\\v_Z
    \end{pmatrix}\\
    R_{zyz} &= R_z(\eta) R_y(\chi - \pi/2) R_z(\psi) \nonumber\\
    &= 
    \begin{pmatrix}
        \cos\eta & -\sin\eta & 0\\
        \sin\eta & \cos\eta & 0 \\
        0 & 0 & 1
    \end{pmatrix}
    \begin{pmatrix}
        \sin\chi & 0 & \cos\chi\\
        0 & 1 & 0\\
        -\cos\chi & 0 & \sin\chi 
    \end{pmatrix}
     \begin{pmatrix}
        \cos\psi & -\sin\psi & 0\\
        \sin\psi & \cos\psi & 0 \\
        0 & 0 & 1
    \end{pmatrix}.
\end{align}
In order to align the axis $\hat{\bm{u}}$ of the mobile frame with the axis $\hat{\bm{x}}$ of the static coordinate system, no rotation about $\hat{\bm{w}}$ is needed and thus $\psi=0$. The other two rotational angles are defined by the components of the Fermi velocity vector as follows:
\begin{align}
    \eta &= \arccos \frac{v_z}{|\bm{v}|} \,,\\
    \chi &=
    \begin{cases}
        \arctan \frac{v_y}{v_x}\,, & v_x>0\\
        \arctan \frac{v_y}{v_x}+\pi\,, & v_x<0\\
        \text{sgn}(v_y)\frac{\pi}{2}\,, & v_x=0
    \end{cases}.
\end{align}
Using this transformation, the vector potential and gap parameter can be expressed in the variables of the mobile frame: $A(\bm{r}(X,Y,Z))$ and $\Delta(\bm{r}(X,Y,Z))$. On a characteristic line defined by the impact parameters $Y_p$ and $Z_p$, the functions from Eq.~\eqref{eq:3DEilenberger/Eilenberger transport eq} become \cite{schopohl_transformation_1998}
\begin{align}
    \Delta(X) &= \Delta(\bm{r}(X,Y_p,Z_p)) \,,\\
    i \Tilde{\epsilon}_n(X) &= i \epsilon_n + \bm{v}_F e\bm{A}(\bm{r}(X,Y_p,Z_p)) \,,
    \label{eq:app2/New_Matsubara_frequencies}\\
    \hat{g}(X) &= \hat{g}(\bm{r}(X,Y_p,Z_p),\bm{p}_F,i \epsilon_n).
\end{align}
The parametrisation of the Eilenberger equations on this 1D line is called the \textit{Riccati parametrisation}. Eq.~\eqref{eq:3DEilenberger/Eilenberger transport eq} formally reduces to the solution of two initial value problems, where the differential equations are scalar and of first order. The Eilenberger propagator is parametrised in terms of two scalar complex functions $a(X)$ and $b(X)$:
\begin{align}
    \hat{g}(X) = 
    \frac{1}{1+a(X)b(X)}
    \begin{pmatrix}
       1-a(X)b(X) & 2 i a(X)\\
       -2i b(X) & -1+a(X)b(X)
    \end{pmatrix}.
\end{align}
The differential equations that need to be numerically solved for $a(X)$ and $b(X)$ are \cite{schopohl_quasiparticle_1995}
\begin{align}
    \hbar v_F a^\prime(X)+[2\Tilde{\epsilon}_n+\Delta^\dagger(X)a(X)]a(X)-\Delta(X)&=0 \,,\\
    \hbar v_F b^\prime(X)-[2\Tilde{\epsilon}_n+\Delta(X)b(X)]b(X)+\Delta^\dagger(X)&=0
\end{align}
with boundary conditions
\begin{align}
    a(-\infty)&=\frac{\Delta(-\infty)}{\epsilon_n+\sqrt{\epsilon^2_n+|\Delta(-\infty)|^2}} \,,\\
    b(+\infty)&=\frac{\Delta^\dagger(+\infty)}{\epsilon_n+\sqrt{\epsilon^2_n+|\Delta(+\infty)|^2}}.
\end{align}
In order to solve them at a certain energy $E<\Delta_0$, the analytical continuation $i\epsilon_n \rightarrow E + i 0^+$ was used.
Finally, the local density of states at $\bm{r}$ was obtained from the real part of $g(\bm{r}(X))$, i.e.
\begin{equation}
    \mathcal{N}(\bm{p}_F) = \mathcal{N}_0(\bm{p}_F)\mathrm{Re}\left(\frac{1-a(X)b(X)}{1+a(X)b(X)}\right).
\end{equation}
In order to obtain the total local density of states one still needs an integration over the Fermi surface to calculate all trajectories that are present due to the various Fermi velocity vectors that exist, plus an integration over the impact factors $Y$ in order to account for trajectories that do not traverse the vortex centre. An integration over $Z$ is redundant since the solution in the bulk is the same for every $Z$. Near the surface this is strictly not the case anymore but the effect is small and with a large enough variety of Fermi velocity vectors (which we have) these trajectories are not missed. 

\paragraph{Influence of vector potential}
The inclusion of a non-zero vector potential $\bm{A}$ in the calculations increases the average time needed to solve the Eilenberger equations because the Matsubara frequencies in Eq.~\eqref{eq:app2/New_Matsubara_frequencies} gain a position dependent term that requires a coordinate transformation between the two reference frames. Therefore, the simulations shown in the main text are performed without vector potential. It is already visible from Eq.~\eqref{eq:app2/New_Matsubara_frequencies} that in a calculation where $\bm{A}$ and $\Delta$ are not solved self-consistently, $\bm{A}$ only enters the equation as an effective energy term. With a vector potential in the azimuthal direction like in Eq.~\eqref{eq:app2/Vector_potential} the scalar product with the Fermi velocity is only expected to yield significant contribution for large impact parameters (for $Y_p=0$, $\bm{v}_F$ is perpendicular to $\bm{A}$). That means trajectories with increasing impact parameter have large LDOS already for smaller distances than in the field free case. The splitting star arms in the LDOS maps should be squeezed to smaller distances from the core. In fact, this is what is seen in the calculations with vector potential at higher energies, as shown in Fig.~\ref{fig:app2/Vector Potential Influence}. This proves that even though there are quantitative differences to the case without vector potential, in the general characteristics, the LDOS patterns remain unchanged.
\begin{figure}
    \centering
    \includegraphics[scale=1]{Comparison_A0_vs_A1.pdf}
    \caption[Influence of vector potential]{\textbf{Influence of Vector Potential}: The LDOS obtained from solutions of the 3D Eilenberger equations for a single-flux-quantum vortex at energy $eU=\SI{0.8}{\milli e\volt}$ without (a) and with vector potential (b), as formulated in Eq.~\eqref{eq:app2/Vector_potential}, show only quantitative differences. With non-zero vector field, the star arms are still split, yet the CdGM states at this energy are squeezed into a smaller area around the vortex core.}
    \label{fig:app2/Vector Potential Influence}
\end{figure}

% \paragraph{DFT calculations}

% Density functional calculations of the electronic structure of Pb
% were carried out in the framework of the mixed-basis pseudopotential
% method \cite{meyer97}. The electron-ion interaction was
% represented by norm-conserving relativistic pseudopotentials \cite{vande85}. Spin-orbit coupling was incorporated within the pseudopotential scheme via Kleinman's formulation \cite{kleinman80} and was consistently taken into account in the charge self-consistency cycle using a spinor representation of the wave functions. Further details of the spin-orbit coupling implementation within the mixed-basis pseudopotential method can be found in a previous publication \cite{heid10}. For higher accuracy, 5$d$ semicore states were included in the valence space.  The deep $d$ potential is efficiently treated by the mixed-basis approach, where valence states are
% expanded in a combination of plane waves and local functions.  Here,
% local functions of $d$ type at the Pb sites were combined with plane
% waves up to a kinetic energy of 20~Ry.  Brillouin-zone integration
% was performed by sampling a 32$\times$32$\times$32 $k$-point mesh (corresponding to 2992 $k$ points in the irreducible part of the Brillouin zone) in conjunction with a Gaussian broadening of 0.2~eV. The
% exchange-correlation functional was represented by the local-density
% approximation in the parameterization of Perdew-Wang \cite{perdew92}.

% This DFT technique was applied to obtain Fermi surface properties entering the Eilenberger equations.
% Band energies were calculated on fine radial grids for a cylindrical coordinate system taking the [111] direction as the z-axis, to determine Fermi momenta $k_F$ for each of the two relevant bands.
% At each $k_F$, 3-dimensional Fermi velocities $v_F$ were then calculated taking numerical derivatives of band energies around this point.
% The optimized lattice parameter $a=4.89$~\AA\, was used throughout.

%DFT calculations: \tg{Rolf..} Band calculations incl. SO-interaction and %relativistic corrections. $k_z$ integration $\rightarrow$ effective %2D-projection of $v_F$ and DOS.


\clearpage

\section*{Supplementary Note 1: Anomalous vortices in varying magnetic fields}
The direction and magnitude of spatial displacement between the sets of CdGM state branches from the two superconducting bands found in anomalous vortices could not be tied to any crystal direction and appears to be almost random. By slowly varying the static magnetic field, the displacement can be modified, as shown in the d$I$/d$U$ maps of the same anomalous vortex in Fig.~\ref{fig:anomolous_b_field_dep} at different fields. We do, however, see an additional blue stripe in the zero-bias d$I$/d$U$ maps of anomalous vortices that appears on the same side of the star pattern as the ring state. For clarity, the ring state is enclosed by a white-dashed circle and the borders of the mentioned stipe are highlighted by white-dashed contour lines in Fig.~\ref{fig:anomolous_b_field_dep}. These stripes might hint at the direction in which the magnetic field lines are tilted beneath the surface.

As Fig.~\ref{fig:anomolous_to_normal} demonstrates, a vortex core can be moved by a slow change in magnetic field and the sets of CdGM states from the two superconducting bands are displaced independently. Consequently, we are able to transform an anomalous vortex back to a normal vortex proving that the two types are not inherently different and that the anomalous type may well be explained by flux line tilting.

\begin{figure}[hpb]
    \centering
    \includegraphics{Anomalous_B_field_dep_w_markers.pdf}
    \caption{\textbf{Manipulation of anomalous vortex pattern.} Zero bias d$I$/d$U$ maps of an anomalous vortex at different magnetic fields. The relative positional shift between the star centre and ring centre (red) is manifold and not tied to crystal directions. The position of the ring centre is marked by a white-dashed circle. The bright stripe mentioned in the main text is marked at its borders by white dashed lines.}
    \label{fig:anomolous_b_field_dep}
\end{figure}

\clearpage

\begin{figure}[hpt]
    \centering
    \includegraphics{Anomalous_to_normal_v2.pdf}
    \caption{\textbf{Manipulation from anomalous to normal vortex.} (a, c) Topographic images of the corresponding quasiparticle patterns in (b,d) showing the same location on the surface. (b,d) Zero bias d$I$/d$U$ map of the anomalous vortex before (b) and after the magnetic field ramp (d). After the decrease of magnetic field, the anomalous vortex in (b) moved and transformed into a normal type (d). In the process, both sets of CdGM states moved.}
    \label{fig:anomolous_to_normal}
\end{figure}

\clearpage

\section*{Supplementary Note 2: Giant vortex}
In a single case, a giant vortex containing $m>10$ flux quanta was found near a large sputtering defect. The zero-bias d$I$/d$U$ map of this vortex, shown in Fig.~\ref{fig:giant_flux_vortex}(a), features more than 10 arms of CdGM states stretching in each $\langle 2\bar{1}\bar{1} \rangle$ direction, which again split into pairs at larger bias (b). Unfortunately this particular giant vortex was partly outside the scan frame of the fine motion piezo tube, which prevented us from a definite determination of $m$.

\begin{figure}[hpb]
    \centering
    \includegraphics[width=0.5\textwidth]{Giant_flux_vortex.pdf}
    \caption{\textbf{Giant vortex.} d$I$/d$U$ maps of a giant vortex containing $m>10$ flux quanta. It shows more than 10 arms at zero bias (a) that individually split in two at sub-gap energies away from the Fermi level (b). This vortex is located at the edge of a large sputtering defect.}
    \label{fig:giant_flux_vortex}
\end{figure}

\clearpage

\section*{Supplementary Note 3: Particle-hole and time-reversal symmetry}
We find that the differential conductance maps show no qualitative change (compared to the figures in the main text) when the electric field in the tunnelling junction is reversed (inverse bias voltage sign $U\leftrightarrow -U$) or the magnetic field direction is reversed ($B\hat{\bm{e}}_z \leftrightarrow -B\hat{\bm{e}}_z$). As a point of proof, Fig.~\ref{fig:vortex_neg_Bfield_or_bias}(a,b) shows a normal vortex stabilized at $\SI{-14}{\milli\tesla}$ (after saturation at $\SI{-85}{\milli\tesla}$) for $U\geq \SI{0}{\milli\volt}$ and (c,d) shows an anomalous vortex stabilized at $\SI{18}{\milli\tesla}$ for $U\leq \SI{0}{\milli\volt}$. The invariance of the LDOS maps under reversal of the electric field demonstrates the particle-hole symmetry of the CdGM states and justifies their treatment as excitations of a BCS ground state, i.e. describing their dynamics by a mean-field BdG Hamiltonian or Eilenberger's quasiclassical Green's function propagators. The invariance of the LDOS maps under reversal of the magnetic field is the logical consequence of time-reversal symmetry of the CdGM states.

\begin{figure}[hpb]
    \centering
    \includegraphics[width=\textwidth]{Vortex_neg_Bfield_or_Bias.pdf}
    \caption{\textbf{Reversal of magnetic and electric field.} Applying the magnetic field or electric field in the opposite direction has no effect on the LDOS pattern inside the vortex. (a-b) d$I$/d$U$ maps of a normal single-flux-quantum vortex in reversed magnetic field. (c-d) d$I$/d$U$ maps of an anomalous vortex showing that reversing the sample bias yields indentical LDOS patterns.}    \label{fig:vortex_neg_Bfield_or_bias}
\end{figure}

\clearpage

\section*{Supplementary Note 4: Vortex at $\SI{4.3}{\kelvin}$}
At $\SI{4.3}{\kelvin}$ the stabilization of single vortices within the STM scan frame was substantially harder. Upon repeating the magnetic protocol in small steps seven times, we only found a vortex within our scan frame in a single case. The vortex signature is shown in Fig.~\ref{fig:4K_vortex}. At $U=\SI{1.8}{\milli\volt}\approx \Delta/e$ (a), it appears as a round depression in differential conductance that is slightly smaller than at base temperature which is to be expected due to the temperature dependence of the coherence length $\xi(T)=\xi_0\sqrt{1-T/T_c}$. At zero bias voltage (b), the CdGM states are significantly smeared out. However, a conductance maximum is still found in the centre of the vortex (c) indicating that it is a vortex with odd winding number and, considering its size and shape, most likely $m=1$. A distinction between $\Delta_1$ and $\Delta_2$ is not possible anymore at this temperature due to the temperature broadening of the d$I$/d$U$ spectra (c).

\begin{figure}[hpb]
    \centering
    \includegraphics[width=0.75\textwidth]{4K_vortex.pdf}
    \caption{\textbf{Vortex at higher temperature.} (a-b) d$I$/d$U$ maps of a vortex at $T=\SI{4.3}{\kelvin}$ and $B=\SI{0}{\milli\tesla}$. (c) Bias spectroscopies far away from the vortex (black squares) and in the centre of the vortex (blue triangles) at $\SI{4.3}{\kelvin}$. The red line shows a fit of Dynes form to the differential conductivity in the superconducting phase with a gap size of $\Delta=\SI{1.24}{\milli e\volt}$.}
    \label{fig:4K_vortex}
\end{figure}

\clearpage

\section*{Supplementary Note 5: Self-consistent calculation of $\Delta$}
We followed Ref. \cite{Silaev2013} and solved the Eilenberger equations in 2D self-consistently for $T=0.1/7.2\,T_c$ in the clean limit, i.e. we refined the pair potential using the self-consistency equation \cite{eilenberger_transformation_1968, Silaev2013}:
\begin{equation}
    \Delta(\bm{r})=2\pi T \Lambda \sum_{n=0}^{N_c} S_F^{-1}\oint_\mathrm{FS}f(i\epsilon_n,\bm{r},\bm{v}_F(\bm{k}))\mathrm{d}^2k
\end{equation}
with 
\begin{equation}
    \Lambda = \left[\log(T/T_c)+\sum\limits_{n=0}^{N_c}\frac{1}{n+1/2}\right]^{-1}
    .
\end{equation}
$S_F$ is the Fermi surface area, $\epsilon_n=2(n+1)\pi T$ the fermionic Matsubara frequencies and $\Lambda$ is the coupling constant. We choose $N_c$ such that $\epsilon_{N_c}=5\,T_c$. Setting the vector potential $\bm{A}=0$ and assuming an isotropic $\bm{v}_F$ we reach convergence %at $\sum_{\bm{r}}({\Delta}^{(j+1)}-{\Delta}^{(j)})<0.1\sum_{\bm{r}}{\Delta}$
for $\Delta$ with an accuracy of $10^{-4}\,T_c$. The result for $\Delta(r)$ is displayed in Fig.~\ref{fig:self-consistent gap} along with the function $\Delta(r)=\Delta_0 \tanh{r/\xi_0}$. For a distance from the vortex centre of $r<2\xi_0$ the recovery of $\Delta(r)$ significantly deviates from the function $\Delta_0\tanh{r/\xi_0}$. The slope of $\Delta(r)$ close to $r=0$ is roughly four times as steep as expected from a simple tanh behaviour with universal $\xi_0$. This leads to a core size $\xi^\mathrm{(c)}=\Delta(\infty)\left[\lim_{r\rightarrow 0} \frac{\mathrm{d\Delta(r)}}{\mathrm{d}r}\right]^{-1}$ that is roughly $\xi_0/4$.

\begin{figure}[hpb]
    \centering
    \includegraphics[width=0.75\textwidth]{Self-consistent_gap_100mK.pdf}
    \caption{\textbf{Kramer-Pesch effect}: The self-consistent calculation of the pair potential $\Delta$ for a vortex with isotropic Fermi velocity at $T=0.1/7.2\,T_c$ exhibits a shrinking of its core size according to the Kramer-Pesch effect, i.e. a steeper recovery of $\Delta$ close to the vortex centre that does not follow $\tanh{r/\xi_0}$.}
    \label{fig:self-consistent gap}
\end{figure}

\clearpage

%\begin{figure}
%    \centering
%    \includegraphics[width=\textwidth]{Calculations.pdf}
%    \caption{(a-f) LDOS maps of isotropic Fermi surface for flux number $m=1,2,3$ at Fermi energy and $0.3*delta$ above Fermi energy, respectively. (g-l) LDOS maps of six-fold symmetry Fermi surface for flux number $m=1,2,3$ at Fermi energy and $0.3*delta$ above Fermi energy, respectively.}  
%    \label{fig:calculations}
%\end{figure}

%\begin{figure}
%    \centering
%    \includegraphics[width=\textwidth]{Gap_closing_anomalous.pdf}
%    \caption{Local band gap closings in the anomalous vortex of Fig.~\ref{fig:anomalous_vortex}. (a) Enlarged zero bias d$I$/d$U$ map of the anomalous vortex including the position of line spectra (white arrows) and single spectra locations (red/black circle). (b) Single spectra in the star centre (black) and ring centre (red) revealing ZBPs. The magnitude of the ZBP in the ring centre is greater than in the star centre by a factor of three.}
%    \label{fig:gap_closing_anomalous}
%\end{figure}

\bibliography{supp}

\end{document}