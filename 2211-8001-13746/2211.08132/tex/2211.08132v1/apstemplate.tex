\documentclass[11pt, a4paper]{article}
\usepackage{styleBuding}
\usepackage{amsmath}
\usepackage{bm}
\usepackage{amssymb}
\usepackage{listings}
\usepackage{multirow}
\usepackage{makecell}
\usepackage{array}
\usepackage{subfigure}
\usepackage{siunitx} 
\usepackage{booktabs}

\DeclareMathOperator{\sgn}{sgn}
\usepackage{csquotes}
\newcommand{\tabincell}[2]{\begin{tabular}{@{}#1@{}}#2\end{tabular}}  
\newcommand{\comment}[1]{\textcolor{red}{#1}}
\newcommand{\rom}[1]{\uppercase\expandafter{\romannumeral #1\relax}}

\title{Reconstructing masses for semi-invisibly decaying particles pair-produced at lepton colliders}
\author[1,2]{Jin Min Yang}
\author[3]{,~Yang Zhang}
\author[1]{,~Pengxuan Zhu}
\author[1,2]{,~Rui Zhu}

\emailAdd{jmyang@itp.ac.cn}
\emailAdd{zhangyangphy@zzu.edu.cn}
\emailAdd{zhupx99@icloud.com}
\emailAdd{zhurui@itp.ac.cn}
\affiliation[1]{CAS Key Laboratory of Theoretical Physics, Institute of Theoretical Physics, Chinese Academy of Sciences, Beijing 100190, P. R. China}
\affiliation[2]{School of Physical Sciences, University of Chinese Academy of Sciences, Beijing 100049, P. R. China}
\affiliation[3]{School of Physics, Zhengzhou University, Zhengzhou 450000, P. R. China}

\abstract{
We present a set of Lorentz invariant kinematic variables for reconstructing masses of the particles in events at lepton colliders, in which a pair of identical particles are produced with subsequent decay into visible Standard Model (SM) particles plus invisible particles, i.e., $e^+ e^- \to {\bf P}{\bf P}$ followed by ${\bf P}\to {\bf V}+{\bf I}$ with $\bf V$ being a visible SM particle or a group of visible SM particles while  $\bf I$ being invisible. We define three variables $m_{\rm RC}^{\rm min}$, $m_{\rm RC}^{\rm max}$ and $m_{\rm LSP}^{\rm max}$, with the former two representing the minimal and maximum masses of $\bf P$ which can be inferred from kinematics, and $m_{\rm LSP}^{\rm max}$ being the upper limit on the mass of $\bf I$. The calculation procedure and analytical formulas of these variables are given. Then we discuss their distribution properties and demonstrate how these variables can help to search for various SM processes and the smuon pair production in supersymmetry with the $2\mu + E_{\rm miss}$ final state. As an example, the prospect of detecting the smuon pair production at CEPC with $\sqrt{s}=240~{\rm GeV}$ is numerically investigated, which shows that the discovery (detection) potential of smuon mass can reach up to 122 GeV (126 GeV), corresponding to a production cross section at order $(0.1)~{\rm fb}$, significantly better than the previous results.   
}


\begin{document}
\maketitle
\section{Introduction} 
To account for dark matter (DM), almost all the extensions of the Standard Model (SM) contain additional particles charged under some new symmetry and/or possess an exact parity. In these theories, the lightest particle is stable and serves as the DM candidate. These theories can be tested at colliders via pair production of new particles which decay into missing/invisible energy in detectors. Reconstructing the masses of such produced new particles are challenging since the energy carried away by the invisible particles can not be directly measured. 
 
\par The difficulty of reconstructing the events containing two or more invisible particles arises from our ignorance of momentum distribution between these missing particles. Specifically, the masses of the missing particles and their `parent' particles are unknown. An important question is what model-independent information about these particle masses can be deduced from this kind of events.  At hadron colliders, inspired by this question, the $m_{\rm T2}$ variable was developed and widely used~\cite{Lester:1999tx, Barr:2003rg, Cheng:2008hk}.

\par Currently, various lepton colliders like ILC~\cite{ Behnke:2013xla, Bambade:2019fyw, ILCInternationalDevelopmentTeam:2022izu}, FCC-ee~\cite{FCC:2018byv, Agapov:2022bhm} or CEPC~\cite{CEPCStudyGroup:2018ghi, Ruan:2018yrh, CEPCPhysicsStudyGroup:2022uwl} are being considered as machines for precise measurements. The most striking difference between a lepton collider and a hadron collider is that the former has a certain collision energy. On the one hand, this feature enables us to infer the energy and momentum of a particle in an event from other particles and collision energy. For example, we can use the Higgs-strahlung process $e^+ e^- \to Z^{\star} \to ZH$, which is the major Higgs production mechanism at the lepton collider, to precisely measure the Higgs mass $m_H$ independent of its decay modes by using the mass recoil to the $Z$-boson. This recoil mass technique has been investigated for both leptonic and hadronic $Z$-boson decays in determining the Higgs boson properties~\cite{Li:2010wu,Haddad:2014fma,CEPCStudyGroup:2018ghi} which show significant advantages of a lepton collider in precision measurements. On the other hand, the excellent energy and momentum resolutions allow for determination of missing energy and momentum $p^{\mu}_{\rm miss}$ with a good precision~\cite{Ruan:2018yrh}. 

\begin{figure}[t]
	\centering
	\subfigure[Fully invisible decaying state ${\bf P}_b$ recoiled by fully visible decaying state ${\bf P}_a$ ]{\includegraphics[width=0.45\textwidth]{fig1b.jpg}}
	\hspace{0.04\textwidth}
	\subfigure[Semi-invisible decaying states  ${\bf P}_a$ and  ${\bf P}_b$]{\includegraphics[width=0.45\textwidth]{fig1a.jpg}}
	\caption{\label{fig1} The decay trees of a pair produced  $\bf PP$  with subsequent decays into two visible states and two invisible states. The event topologies can be classified into two types: (a) one state ${\bf P}_b$ with a fully invisible decay  $  {\bf P}_b \to {\bf I}_{b1} {\bf I}_{b2}$ can be recoiled by another state ${\bf P}_a$ with a fully visible decay ${\bf P}_a \to {\bf V}_{a1} {\bf V}_{a2}$; (b) both particles ${\bf P}_a$ and ${\bf P}_b$ have semi-invisible decays, ${\bf P}_a \to {\bf V}_a {\bf I}_a$ and ${\bf P}_b \to {\bf V}_b {\bf I}_b$. }
\end{figure} 

\par For an event of process $e^+ e^- \to {\bf P}_a {\bf P}_b$ with subsequent decay into final states containing two visible systems and two invisible systems, two topologies are of interest:
\begin{itemize} 
\item[(i)] The first topology is displayed in Fig.~\ref{fig1}(a), where one parent state has a fully visible decay and the other one has a fully invisible  decay. In this case for the particle ${\bf P}_{b}$ we can use the recoil mass variable to determine its mass, which has been studied in the literature.  

\item[(ii)]  The second topology is depicted in Fig.~\ref{fig1}(b), where the parent particles have semi-invisible decays, ${\bf P}_a \to {\bf V}_a {\bf I}_a$ and ${\bf P}_b \to {\bf V}_b {\bf I}_b$. In this case, motivated by the high performance of future lepton colliders, a set of kinematic observables can be introduced to reconstruct the unknown mass information of the particles, which has not been presented in the literature and serves as the aim of this work.  
\end{itemize} 

\par This work is organized as follows. In Sec.~\ref{sec:rc}, focusing on the  topology in Fig.~\ref{fig1}(b), we derive these variables from kinematic constraints and also provide a fast algorithm for numerical calculations. In Sec.~\ref{sec:propty}, we discuss their properties from the distributions of various SM processes and the smuon pair production process in supersymmetry with a final state of di-muon plus missing energy. As an application,  in Sec.~\ref{sec:app} we show that the newly introduced variables can indeed improve the signal-to-background ratio significantly for the smuon pair production at CEPC. Finally, a brief summary is provided in Sec.~\ref{sec:sum}.

\section{\label{sec:rc}Reconstructing the semi-invisible decaying events}
The concept of decay trees was introduced in the recursive jigsaw reconstruction technique~\cite{Jackson:2016mfb, Jackson:2017gcy}. Each node in a decay tree is not only described as the collection of the four-momentum vectors of the relevant states, but also represents a reference frame to each intermediate combination of them. In a lepton collider, we can get the following from the detector:
\begin{itemize}
	\item $p^{\mu, \rm Lab}_{{\bf V}_a}$: the four-momentum of ${\bf V}_a$ in laboratory frame.
	\item $p^{\mu, \rm Lab}_{{\bf V}_b}$: the four-momentum of ${\bf V}_b$ in laboratory frame.
	\item $p^{\mu, \rm Lab}_{\rm miss}$: the missing four-momentum, which is obtained as  
		\begin{equation}
			p^{\mu, \rm Lab}_{\rm miss} = (\sqrt{s}, 0, 0, 0) - \sum_{i}^{n} p^{\mu, \rm Lab}_{i}, 
		\end{equation} 
		with $\sqrt{s}$ being the collision energy and $i$ being the index of all visible particles in the final states. In the events studied in this work, it is assumed as the vector sum of the four-momentums of the two invisible states 
		\begin{equation}\label{eq:pmiss}
			p^{\mu, \rm Lab}_{\rm miss} = p^{\mu, \rm Lab}_{{\bf I}_a} + p^{\mu, \rm Lab}_{{\bf I}_b}.
		\end{equation}
	\item $p^{\mu, \rm Lab}_{\rm ISR}$: the four-momentum of anything else in the event. While `ISR' usually refers to initial state radiation, in view of event reconstruction the $\rm ISR$ system is actually the collection of all visible particles which are not assigned into the $\bf V$-systems. 
\end{itemize}

%%%fig.2 
\begin{figure}[th]
	\centering
	\includegraphics[width=0.7\textwidth]{fig2_momentumspace.pdf}
	\caption{\label{fig2:momspace} The three-momentum vectors of semi-invisibly decaying particles shown in the $\bf PP$ frame.}
\end{figure}

\par For the semi-invisible decay event, the four-momentum of `parent' particles are given by 
\begin{equation}\label{eq:ppappb}
	p^\mu_{{\bf P}_a} = p^\mu_{{\bf V}_a} + p^\mu_{{\bf I}_a}, \quad
	p^\mu_{{\bf P}_b} = p^\mu_{{\bf V}_b} + p^\mu_{{\bf I}_b}. 
\end{equation}
In most cases, ${\bf P}_a$ and ${\bf P}_b$ are same particles with opposite charges, and the invisible states ${\bf I}_a$ and ${\bf I}_b$ are the DM candidates. So it is reasonable to assume 
\begin{equation}\label{eq:massconstraint}
	m_{\bf P}^2 = \left| p^{\mu}_{{\bf P}_a} \right|^2 = \left| p^\mu_{{\bf P}_b} \right|^2, \quad	m_{\bf I}^2 = \left| p^\mu_{{\bf I}_a} \right|^2= \left|p^\mu_{{\bf I}_b} \right|^2. 
\end{equation}
Given the constraints and assumptions in Eqs.~(\ref{eq:pmiss}), (\ref{eq:ppappb}) and (\ref{eq:massconstraint}), there remains two unknown degrees of freedom in each event. For the Lorentz four-vectors reconstructed, one can always boost the four-momentum vectors from one reference frame to another, and this will not affect the reconstruction of the particle masses. So firstly, we boost all vectors from the laboratory frame into the center-of-mass frame of the ${\bf PP}$ system. In the ${\bf PP}$ frame, according to the definition of the $\rm ISR$ system, one can write down 
\begin{equation}\label{eq:pmuPP}
\begin{split}
p^{\mu, \rm PP}_{{\bf I}_a} &= \left\{ \frac{1}{2}\left(- E_{{\bf V}_a}^{\rm PP} + E^{\rm PP}_{{\bf V}_b} + E^{\rm PP}_{\rm miss} \right), \vec{p}_{{\bf I}_a}^{\rm PP}\right\}, \\ 
p^{\mu, \rm PP}_{{\bf P}_a} &= \left\{ \frac{1}{2}\left(E_{{\bf V}_a}^{\rm PP} + E^{\rm PP}_{{\bf V}_b} + E^{\rm PP}_{\rm miss} \right), \vec{p}_{{\bf V}_a}^{\rm PP} + \vec{p}_{{\bf I}_a}^{\rm PP}\right\}, \\
p^{\mu, \rm PP}_{{\bf I}_b} 
&= \left\{ \frac{1}{2}\left(E_{{\bf V}_a}^{\rm PP} - E^{\rm PP}_{{\bf V}_b} + E^{\rm PP}_{\rm miss} \right), \vec{p}_{\rm miss}^{\rm PP} - \vec{p}_{{\bf I}_a}^{\rm PP}\right\}, \\ 
p^{\mu, \rm PP}_{{\bf P}_b} &= \left\{ \frac{1}{2}\left(E_{{\bf V}_a}^{\rm PP} + E^{\rm PP}_{{\bf V}_b} + E^{\rm PP}_{\rm miss} \right), \vec{p}_{{\bf V}_b}^{\rm PP} + \vec{p}_{\rm miss}^{\rm PP} - \vec{p}_{{\bf I}_a}^{\rm PP}\right\}. \end{split}
\end{equation} 
Here, the energies of ${\bf P}_a$ and ${\bf P}_b$ are set to be same in the $\bf PP$ frame, which is equivalent to the assumption of $m_{{\bf P}_a} = m_{{\bf P}_b}$. 

\par In Fig. \ref{fig2:momspace} we plot the momentums $\vec{p}_{{\bf V}_a}^{\rm PP}$, $\vec{p}_{{\bf I}_a}^{\rm PP}$, $\vec{p}_{{\bf V}_b}^{\rm PP}$ and $\vec{p}_{{\bf I}_b} ^{\rm PP}$ in the $\bf PP$ frame, and they form a visible plane (green) and an invisible plane (pink), which are defined by two closed three-momentum triangles
\begin{equation}
	\vec{p}^{\rm PP}_{{\bf V}_a} + \vec{p}^{\rm PP}_{{\bf V}_b} + \vec{p}^{\rm PP}_{\rm miss} = \vec{0}, \quad 
	\vec{p}^{\rm PP}_{{\bf I}_a} + \vec{p}^{\rm PP}_{{\bf I}_b} - \vec{p}^{\rm PP}_{\rm miss} = \vec{0}. 
\end{equation} 
The angle between the two planes is denoted as $\phi$, and the vertex opposite side $\vec{p}_{\rm miss}^{\rm PP}$ in the visible (invisible) triangle is marked as point $\bf P$ ($\bf I$). Accordingly, $\vec{p}_{{\bf P}_a}^{\rm PP}$ is the vector $\bf P$$\bf I$. The position of point $\bf I$ can not be fully reconstructed, however, it must satisfy the following constraints
\begin{itemize}
	\item $m_{{\bf I}_a} \geq 0$. The magnitude of $\vec{p}_{{\bf I}_a}^{\rm PP} $ must be smaller than $E_{{\bf I}_a} = \left(- E_{{\bf V}_a}^{\rm PP} + E^{\rm PP}_{{\bf V}_b} + E^{\rm PP}_{\rm miss} \right)\big/ 2$. Therefore, as shown in Fig.~\ref{fig2:momspace}, the point $\bf I$ is located in a sphere, where the spherical shell implies $m_{{\bf I}_a} = 0$.  
	\item $m_{{\bf I}_b} \geq 0$. Similar to the constraint $m_{{\bf I}_a} \geq 0$, but corresponding to another sphere with radius $E_{{\bf I}_b} = \left(E_{{\bf V}_a}^{\rm PP} - E^{\rm PP}_{{\bf V}_b} + E^{\rm PP}_{\rm miss} \right)\big/ 2$. 
	\item $m_{{\bf I}_a} = m_{{\bf I}_b}= m_{\bf I} $. Put this assumption into Eqs.~(\ref{eq:massconstraint}) and (\ref{eq:pmuPP}), one can get that $\vec{p}_{{\bf I}_a}^{\rm PP}$ satisfies 
	\begin{equation}\label{eq:mIa=mIb}
		\vec{p}_{{\bf I}_a}^{\rm PP}  \cdot \vec{p}_{\rm miss}^{\rm PP} - D = 0,
	\end{equation}
	where $D$ is determined by 
	\begin{equation}
		D = \frac{1}{2}\left( {E_{{\bf V}_a}^{\rm PP}}^2 - {E_{{\bf V}_b}^{\rm PP}}^2 + E_{\bf PP}^{\rm PP}(E_{{\bf V}_b}^{\rm PP} - E_{{\bf V}_a}^{\rm PP} ) + \left| \vec{p}_{\rm miss}^{\rm PP}\right|^2 \right),
	\end{equation}
	with $E_{\bf PP} = E_{{\bf V}_a}^{\rm PP} + E_{{\bf V}_b}^{\rm PP} + E_{\rm miss}^{\rm PP}$ being the total energy of all objects in the $\bf PP$ system. Eq.~(\ref{eq:mIa=mIb}) implies that the point $\bf I$ is located in a plane vertical to $\vec{p}_{\rm miss}^{\rm PP}$ (the light blue plane in Fig.~\ref{fig2:momspace}). $D$ determines the position of plane $m_{{\bf I}_a} = m_{{\bf I}_b}$, on which the intersection circle of spheres $m_{{\bf I}_a} = 0$ and $m_{{\bf I}_b} = 0$ locates. 
\end{itemize}
As shown in Fig~\ref{fig2:momspace}, so far we can conclude that the point $\bf I$ must be located on a flat round disc, which can be fully reconstructed. The vector $\vec{p}_{\rm miss}^{\rm PP}$ goes through the disc center $\bf O$ and is vertical to the disc, and the circle corresponding to the disc edge describes the equation $m_{\bf I} = 0$. 

\par There are two unknown degrees of freedom: the angle $\phi$ and $r$, where $r$ is the distance between the point $\bf I$ and disc center $\bf O$. The mass $m_{\bf I}$ can be determined by the magnitude of $\vec{p}_{{\bf I}_a}^{\rm PP}$. So, due to the rotational symmetry, the points on the circle of a given $r$ correspond to the same values of $m_{\bf I}$ reconstructed. Similarly, the mass $m_{\bf P}$ can be determined by the magnitude of $\vec{p}_{{\bf P}_a}^{\rm PP}$, i.e., the length of vector $\bf P$$\bf I$. The maximum value of $\left|\vec{p}_{{\bf P}_a}^{\rm PP} \right|$ can be obtained when $\bf I$ locates in the position of $\phi = \pi$ with a maximum $r$ (point $\bf B$ in Fig.~\ref{fig2:momspace}). If the projection of $\bf P$ on the $m_{{\bf I}_a} = m_{{\bf I}_b}$ plane (point $\bf C$) is inside the disc, the minimal $\left|\vec{p}_{{\bf P}_a}^{\rm PP} \right|$ value is corresponding to the vector $\bf P$$\bf C$. Else if $\bf C$ is outside the disc, i.e., the case in Fig.~\ref{fig2:momspace}, the minimal $\left|\vec{p}_{{\bf P}_a}^{\rm PP} \right|$ corresponds to the point of $\phi = 0$ with a maximum $r$ (point A in Fig.~\ref{fig2:momspace}). 

\par For a given set $(\phi, r)$, we can obtain a set of masses $(m_{\bf P}, m_{\bf I})$. Therefore, the geometric properties of the disc imply the kinetic information of the event, which can be represented as three `reconstructed-mass' variables: 
\begin{itemize}
	\item $m_{\rm LSP}^{\rm max}$: the value of $m_{\bf I}$ by choosing  center $\bf O$, corresponding to the  maximum value of $m_{\bf I}$ reconstructed.
	\item $m_{\rm RC}^{\rm min}$: the minimal value of $m_{\bf P}$ reconstructed by setting $\left|\vec{p}_{{\bf P}_a}^{\rm PP} \right| = \left| {\bf P}{\bf B} \right|$. 
	\item $m_{\rm RC}^{\rm max}$: the maximal value of $m_{\bf P}$ reconstructed by choosing the minimal $\left|\vec{p}_{{\bf P}_a}^{\rm PP} \right|$ value.    
\end{itemize}
The points $\bf O, A, B$ and $\bf C$ are on a straight line, i.e., the intersection of the visible plane and the $m_{{\bf I}_a} = m_{{\bf I}_b}$ plane. So the calculation of these three variables is a plane geometry problem. Finally, we can get \footnote{Ref.~\cite{Chen:2021omv} proposed a `reconstructed-mass' variable $m_{\rm reconst}$ similar to $m_{\rm RC}^{\rm min}$. Here our proposed variables are more general and we provide the analytical expressions.}
\begin{equation}\begin{split}
	m_{\rm LSP}^{\rm max} &= 
		\sqrt{ 
			E_{{\bf I}_a}^2 - \frac{1}{4\left| \vec{p}_{\rm miss}^{\rm PP} \right|^2 }
			\left( \left| \vec{p}_{\rm miss}^{\rm PP} \right|^2 + E_{{\bf I}_a}^2 - E_{{\bf I}_b}^2 \right)^2
		}, \\
	m_{\rm RC}^{\rm min} &= \sqrt{\frac{E_{\bf PP}^2}{4} - \left| {\bf PC} \right|^2 - \left(\left| {\bf OC} \right| + \left| {\bf OA} \right|  \right)^2  },
\end{split}\end{equation}
where
\small
\begin{equation}\label{eq:trilength}\begin{split}
	\left| {\bf PC} \right| = \frac{1}{2\left| \vec{p}_{\rm miss}^{\rm PP} \right| }
	\Big(
	 &	- E_{{\bf I}_a}^2 + E_{{\bf I}_b}^2 
		+ \left| \vec{p}_{{\bf V}_a}^{\rm PP} \right|^2 
		- \left| \vec{p}_{{\bf V}_b}^{\rm PP} \right|^2 
		\Big), \\
	\left| {\bf OC} \right| = \frac{1}{2\left| \vec{p}_{\rm miss}^{\rm PP} \right| } \bigg(
	 & 	- \left| \vec{p}_{\rm miss}^{\rm PP} \right|^4 
		- \left(
			\left| \vec{p}_{{\bf V}_a}^{\rm PP} \right|^2 
			- \left| \vec{p}_{{\bf V}_b}^{\rm PP} \right|^2 
		\right)^2 
 		+2 \left| \vec{p}_{\rm miss}^{\rm PP} \right|^2 
		\left(
			\left| \vec{p}_{{\bf V}_a}^{\rm PP} \right|^2 
			+ \left| \vec{p}_{{\bf V}_b}^{\rm PP} \right|^2
		\right) \bigg)^{1/2}, \\
	\left| {\bf OA} \right| = \frac{1}{2\left| \vec{p}_{\rm miss}^{\rm PP} \right| } \bigg(
	 & 	- E_{{\bf I}_a}^4 
		- \left(
			E_{{\bf I}_b}^2
			- \left| \vec{p}_{\rm miss}^{\rm PP} \right|^2 
		\right)^2 
		+2 E_{{\bf I}_a}^2 
		\left(
			E_{{\bf I}_b}^2 
			+ \left| \vec{p}_{\rm miss}^{\rm PP} \right|^2
		\right) \bigg)^{1/2}. 
\end{split}\end{equation}
\normalsize
For $m_{\rm RC}^{\rm max}$, it depends on the location of point $\bf C$. If $\bf C$ is inside the disc, i.e., $\left|\bf OC \right| < \left|\bf OA \right|$, we have
\begin{equation}
	m_{\rm RC}^{\rm max} = \sqrt{\frac{E_{\bf PP}^2}{4} - \left| {\bf PC} \right|^2 }, 
\end{equation}
otherwise, we have  
\begin{equation}
	m_{\rm RC}^{\rm max} = \sqrt{\frac{E_{\bf PP}^2}{4} - \left| {\bf PC} \right|^2 - \left(\left| {\bf OC} \right| - \left| {\bf OA} \right|  \right)^2  }. 
\end{equation}

\section{\label{sec:propty}Numerical analysis}
%%%fig.3 
\begin{figure}[t]
	\centering
	\makebox[\textwidth][c]{
	\includegraphics[width=0.40\linewidth]{zz2d2.pdf}\hspace{-0.2cm}
	\includegraphics[width=0.40\linewidth]{ww2d2.pdf}\hspace{-0.2cm}
	\includegraphics[width=0.40\linewidth]{tau2d2.pdf}
	}
	\makebox[\textwidth][c]{
	\includegraphics[width=0.40\linewidth]{zz2d1.pdf}\hspace{-0.2cm}
	\includegraphics[width=0.40\linewidth]{ww2d1.pdf}\hspace{-0.2cm}
	\includegraphics[width=0.40 \linewidth]{tau2d1.pdf} 
	}
	\caption{\label{fig:sm_distribution} The distributions of $m_{\rm RC}^{\rm min}$ versus $m_{\rm RC}^{\rm max}$ (top) and $m_{\rm LSP}^{\rm max}$ versus $m_{\rm Recoil}$  (bottom)  for simulated events of $ZZ$ (left),  $W^+W^-$ (center) and  $\tau\tau$ (right) decaying into di-muon plus multi-neutrinos. One-dimensional marginal distribution of each axis is plotted as histogram. Each estimated quantity is normalized by the true value event-by-event.}
\end{figure}
\par For illustration and comparison, we apply our approach to three SM pair-production processes $ZZ$, $WW$  and $\tau\tau$ with subsequent decay into final states of muon pair plus neutrinos. 
We generate each process at $e^+e^-$ collision of $\sqrt{s} = 240~{\rm GeV}$ with \textsc{MadGraph5aMC@NLO}\cite{Alwall:2014hca,Frederix:2018nkq} and \textsc{PYTHIA8}~\cite{Bierlich:2022pfr}, and plot their distributions in $m_{\rm RC}^{\rm min} - m_{\rm RC}^{\rm max}$ plane and $m_{\rm LSP}^{\rm max}-m_{\rm Recoil}$ plane in Fig.~\ref{fig:sm_distribution}. Here the recoil mass variable $m_{\rm Recoil}$ is defined as 
\begin{equation}
	m_{\rm Recoil} = \sqrt{s - 2 E_{\mu\mu} \cdot \sqrt{s} + m_{\mu\mu}^2},
\end{equation}
where $E_{\mu\mu}$ and $m_{\mu\mu}$ are respectively the total energy and invariant mass of the muon pair.
For comparison, the smuon pair production process in supersymmetry with subsequent decay into the final states of $\mu\mu\tilde{\chi}_1^0\tilde{\chi}_1^0$ is also investigated, which is shown in Fig.~\ref{fig:smuon_distribution}. 
%%%fig.4 
\begin{figure}[t]
	\centering
	\makebox[\textwidth][c]{
	\includegraphics[width=0.40\linewidth]{smuon2d2.pdf}\hspace{-0.2cm}
	\includegraphics[width=0.40\linewidth]{smuon100_0_2d2.pdf}\hspace{-0.2cm}
	\includegraphics[width=0.40\linewidth]{smuon100_40_2d2.pdf}
	}
	\makebox[\textwidth][c]{
	\includegraphics[width=0.40\linewidth]{smuon2d1.pdf}\hspace{-0.2cm}
	\includegraphics[width=0.40\linewidth]{smuon100_0_2d1.pdf}\hspace{-0.2cm}
	\includegraphics[width=0.40\linewidth]{smuon100_40_2d1.pdf} 
	}
	\caption{\label{fig:smuon_distribution}Similar to Fig.~\ref{fig:sm_distribution}, but for smuon pair production process with  $m_{\tilde{\mu}} = 50~{\rm GeV}, m_{\tilde{\chi}_1^0} = 0~{\rm GeV}$ (left),  $m_{\tilde{\mu}} = 100~{\rm GeV}, m_{\tilde{\chi}_1^0} = 0~{\rm GeV}$ (center), and $m_{\tilde{\mu}} = 100~{\rm GeV}, m_{\tilde{\chi}_1^0} = 40~{\rm GeV}$ (right) .}
\end{figure}
From their definitions and distributions, the variables $m_{\rm LSP}^{\rm max}$, $m_{\rm RC}^{\rm min}$ and $m_{\rm RC}^{\rm max}$ have the following properties:
\begin{itemize}
	\item One can easily find and also easy to prove  
		\begin{equation}\label{eq:reo_lsp}
			m_{\rm Recoil} > 2~ m_{\rm LSP}^{\rm max}.
		\end{equation}
	This is reasonable since $m_{\rm Recoil}$ can be treated as the invariant mass of $p_{\rm miss}^{\mu}$, and we split $p_{\rm miss}^{\mu}$ into two invisible particles ${\bf I}_a$ and ${\bf I}_b$. 
	
 \item For the decay topology of $ZZ$ event as depicted in Fig.~\ref{fig1}(a), the recoil mass variable $m_{\rm Recoil}$ can reconstruct the $Z$-mass readily. In this case we can also try to apply our reconstructed-mass variables. As shown in Fig.\ref{fig:sm_distribution}, the distribution of $m_{\rm LSP}^{\rm max}$ has a peak at $m_Z/2$. The distributions of $m_{\rm RC}^{\rm max}$ and $m_{\rm RC}^{\rm min}$ show a linear correlation between them. The underlying reason is that the shape of the visible triangle, formed by $\vec{p}^{\rm PP}_{{\bf V}_a}$, $\vec{p}^{\rm PP}_{{\bf V}_b} $ and $\vec{p}^{\rm PP}_{\rm miss}$, is similar for each event.
 
	\item For the events of semi-invisibly decaying $WW$, $\tau\tau$ and $\tilde{\mu}\tilde{\mu}$ processes, the observables $m_{\rm LSP}^{\rm max}$, $m_{\rm RC}^{\rm min}$ and $m_{\rm RC}^{\rm max}$ reflect the masses of particles involved directly, 
		\begin{equation}
			\begin{split}
			0 \leq & m_{\bf I} \leq m_{\rm LSP}^{\rm max}, \\
			0 \leq m_{\rm RC}^{\rm min} \leq & m_{\bf P} \leq m_{\rm RC}^{\rm max} \leq \sqrt{s}\big/2 . 
		\end{split}
		\end{equation}
		Note that for a small number of events we even have $m_{\rm RC}^{\rm min} > m_{\bf P}$ or $m_{\rm RC}^{\rm max} < m_{\bf P}$ (most of the dark blue area in Fig.\ref{fig:sm_distribution}). This is because the muons lose energy through photon radiation, and we use the so-called bare-level muons (i.e. after photon radiation) in the numerical calculation. This is why we have small tails beyond end-points.
  
	\item There is an unobvious relation 
		\begin{equation}\label{eq:maxs_relation}
		m_{\rm LSP}^{\rm max} < m_{\rm RC}^{\rm max}.
		\end{equation} 
		It may be hard to prove this relation rigorously, but we find that it is always true numerically.
  
	\item For massless {\bf I}, like the $WW$ event and $\tilde{\mu}\tilde{\mu}$ event with massless $\tilde{\chi}_1^0$, both $m_{\rm RC}^{\rm min}$ and $m_{\rm RC}^{\rm max}$ have a significant peak at $m_{\bf P}$ and also an obvious truncation at $m_{\bf P}$,  which is similar to the $m_{\rm T}$ variable for $W \to \ell \nu$ events~\cite{Smith:1983aa,CDF:2004sns,CDF:2022hxs} and the $m_{\rm T2}$ variable for $t\bar{t}$ events at hadron colliders. At a lepton collider, the $W$-boson mass is extracted via $\ell\nu \ell^\prime \nu^\prime$ channel by fitting the energy spectrum of the charged leptons, referred to as the pseudo-mass method~\cite{Straessner:2004pw, OPAL:2002hhr}. These peaks in $m_{\rm RC}^{\rm min}$ and $m_{\rm RC}^{\rm max}$ distributions in Fig.~\ref{fig:sm_distribution} imply that the $W$-boson mass can be `directly' read out from full-leptonic decay events.
 
	\item For massive {\bf I}, shown by the point $m_{\tilde{\mu}} = 100~{\rm GeV}, m_{\tilde{\chi}_1^0} = 40~{\rm GeV}$ in Fig.~\ref{fig:smuon_distribution}, the range $(m_{\rm RC}^{\rm min}, m_{\rm RC}^{\rm max})$ gets bigger as $m_{\tilde{\chi}_1^0}$ increases. Another notable change compared with massless {\bf I} is that the peaks at $m_{\bf P}$ disappear. Specifically, in this case most part of the $m_{\rm RC}^{\rm min}$ distribution is submerged in the $WW$ background. Relatively speaking, the $m_{\rm RC}^{\rm max}$ distribution is less sensitive to the variation of $m_{\bf I}$. So the cut for $m_{\rm RC}^{\rm max}$ can greatly suppress the background events in searches for heavy new particles. A cut $m_{\rm RC}^{\rm max} \geq 115~{\rm GeV}$, for example, will filter out more than 90\% $WW$ events, but can retain almost all the $\tilde{\mu}\tilde{\mu}$ events with $m_{\tilde{\mu}} \geq 115~{\rm GeV}$. From  Eq.~(\ref{eq:maxs_relation}), we find that $m_{\rm RC}^{\rm max} - m_{\rm LSP}^{\rm max}$ can approximately represent the mass splitting between $\bf P$ and $\bf I$ for a very massive $\bf P$. So in the proceeding section, we use $m_{\rm RC}^{\rm max} - m_{\rm LSP}^{\rm max}$ in the signal region (SR) definition for a larger signal-background ratio. 
\end{itemize}

%%% fig.5
\begin{figure}[t]
	\makebox[\textwidth][c]{
	\includegraphics[width=0.40\linewidth]{smuon100_80_2d1.pdf}\hspace{-0.2cm}
	\includegraphics[width=0.40\linewidth]{smuon100_80_2d2.pdf}\hspace{-0.2cm}
	\includegraphics[width=0.40\linewidth]{smuon100_80_2d3.pdf}
	}\vspace{-0.3cm}
	\caption{\label{fig:mrcs}The distributions of  $m_{\rm RC}^{\rm min}$ versus $m_{\rm RC}^{\rm max}$ (left), $m_{\rm RC}^{\rm min}(40~{\rm GeV})$ versus $m_{\rm RC}^{\rm max}(40~{\rm GeV})$ (center), and  $m_{\rm RC}^{\rm min}(80~{\rm GeV})$ versus $m_{\rm RC}^{\rm max}(80~{\rm GeV})$ (right), for $m_{\tilde{\mu}} = 100~{\rm GeV}$, $m_{\tilde{\chi}_1^0} = 80~{\rm GeV}$. One-dimensional marginal distribution of each axis is plotted as histogram. Each of the estimated quantities is normalized by the true value event-by-event.}
\end{figure}

\par The definitions of $m_{\rm RC}^{\rm min}$ and $m_{\rm RC}^{\rm max}$ assume that the invisible state $\bf I$ is massless. In case that we happen to know the mass of $\bf I$ or we have some physical motivation for a particular value of $m_{\bf I}$, we can further define a new pair of variables $m_{\rm RC}^{\rm min}(m_{\rm inv})$ and $m_{\rm RC}^{\rm max}(m_{\rm inv})$ with an input parameter $m_{\rm inv}$ representing the guessed mass of {\bf I}. Their calculation procedures are same as for $m_{\rm RC}^{\rm min}$ and $m_{\rm RC}^{\rm max}$, except that $E_{{\bf I}_a}^2$ and $E_{{\bf I}_b}^2$ in Eq.~(\ref{eq:trilength}) need to be respectively replaced by $(E_{{\bf I}_a}^2 - m_{\rm inv}^2)$ and $(E_{{\bf I}_b}^2 - m_{\rm inv}^2)$. However, for a specific event, one should note that not any value of $m_{\rm inv}$ is always allowed. For example, if the chosen value of $m_{\rm inv}$ happens to be larger than $m_{\rm LSP}^{\rm max}$, there will be no real solutions for $m_{\rm RC}^{\rm min}(m_{\rm inv})$ and $m_{\rm RC}^{\rm max}(m_{\rm inv})$.  

\par We plot the distributions of $\tilde{\mu}\tilde{\mu}$ events with $m_{\tilde{\mu}} = 100~{\rm GeV}$ and $m_{\tilde{\chi}_1^0} = 80~{\rm GeV}$ in $m_{\rm RC}^{\rm min}(m_{\rm inv})-m_{\rm RC}^{\rm max}(m_{\rm inv})$ plane with specific choices $m_{\rm inv} = 0, 40, 80~{\rm GeV}$ in Fig.~\ref{fig:mrcs}. From this figure we can find  
	\begin{equation}
		m_{\rm inv} \leq m_{\rm RC}^{\rm min}\left(m_{\rm inv}\right) \leq m_{\bf P} \leq m_{\rm RC}^{\rm min}\left(m_{\rm inv}\right) \leq \sqrt{s}\big/2.
	\end{equation}
And for a given event, as $m_{\rm inv}$ increases, the value of $m_{\rm RC}^{\rm min}\left(m_{\rm inv}\right)$ increases while $m_{\rm RC}^{\rm max}\left(m_{\rm inv}\right)$ decreases. If $m_{\rm inv}$ happens to be $m_{\bf I}$, the end-points and the peaks at $m_{\rm P}$ in the $m_{\rm RC}^{\rm min}\left(m_{\rm inv}\right)$ distribution appear again.  

\section{\label{sec:app}Application to smuon pair production at CEPC}
To demonstrate the efficacy of our constructed observables in the preceding section, we examine the process of smuon pair production at CEPC with $\sqrt{s} = 240~{\rm GeV}$. This channel has been investigated recently in ~\cite{Yuan:2022ykg, CEPCPhysicsStudyGroup:2022uwl}. In these studies, the events with an opposite-sign muon pair plus significant recoil mass in the final states are selected, and the signal regions are defined using the azimuth angles $\Delta\phi(\mu^\pm, {\rm Recoil})$ and $\Delta\phi(\mu^+, \mu^-)$, the cone sizes $\Delta R(\mu^\pm, {\rm Recoil})$ and $\Delta R(\mu^+, \mu^-)$, the muon energies $E_\mu^{\pm}$, the sum of the transverse momentum of two muons ${\rm sum}P_{\rm T}$, the invariant mass $m_{\rm \mu\mu}$ and the $m_{\rm Recoil}$ variable. Their results showed that the detection (discovery) limit of smuon mass can reach to 117 GeV (116 GeV), corresponding to a theoretical cross section of $17.0~{\rm fb}$ ($11.1~{\rm fb}$). In the following, we improve the search by using our reconstructed mass variables. 

\subsection{Simplified model setting}
%% fig.6 
\begin{figure}[h]
	\centering
	\includegraphics[width=0.5\linewidth]{smuon_feyndiag.pdf}
	\caption{\label{fig:smuon_prod} Feynman diagram of smuon pair production at CEPC.}
\end{figure}

\par In low energy supersymmetry (for some recent reviews, see, e.g., \cite{Wang:2022rfd,Baer:2020kwz,Yang:2022qyz}) , smuon pair production $e^+ e^- \to \tilde{\mu}^+ \tilde{\mu}^{-}$ at an $e^+ e^-$ collider occurs only via $\gamma$ or $Z$ exchange in the $s$-channel. As shown in Fig.~\ref{fig:smuon_prod}, the simplified model in the searching channel $ e^+ e^- \to \mu^+ \tilde{\chi}_1^0 \mu^- \tilde{\chi}_1^0 $ depends on the particle spectrum. Here $\tilde{\mu}$ contains both the left-handed $\tilde{\mu}_L$ and right-handed $\tilde{\mu}_R$, and their masses are degenerated which are labeled by $m_{\tilde{\mu}}$. The lightest neutralino $\tilde{\chi}_1^0$ is assumed to be a nearly pure bino, and the branching ratio ${\rm Br}(\tilde{\mu} \to \mu \tilde{\chi}_1^0)$ is set to 1.

\par In most cases, the cross section can be evaluated using narrow width approximation 
\begin{equation}
	\sigma(\tilde{\mu}\tilde{\mu} \to  \mu^+ \tilde{\chi}_1^0 \mu^- \tilde{\chi}_1^0) = \sigma(\tilde{\mu} \tilde{\mu}) \times \left({\rm BR}(\tilde{\mu} \to \mu \tilde{\chi}_1^0)\right)^2 .
\end{equation}
At the CEPC, the electron and positron beams are unpolarized, and the cross section is written as~\cite{deCarlos:1995cx, Moortgat-Pick:2005jsx}
\begin{eqnarray}
	\sigma(\tilde{\mu}\tilde{\mu}) &=& \frac{(1-4m_{\tilde{\mu}}^2/s)^{3/2}}{24\pi s} \Bigg[e^4 + \frac{e^2g^2}{2c_{W}^2} \left(\frac{1}{2}- 2 s_W^2 \right)^2 \frac{s(s - m_Z^2)}{(s-m_Z^2)^2 + \Gamma_Z^2 m_Z^2} \nonumber \\
	&& + \frac{g^4}{c_W^4}\left( \frac{1}{8}-\frac{1}{2} s_W^2 + s_W^4\right)^2 \frac{s^2}{(s-m_Z^2)^2 + \Gamma_Z^2 m_Z^2} \Bigg], 
\end{eqnarray}
where $e$ is the electromagnetic coupling constant, $g$ is the weak force coupling constant, $s_W \equiv \sin{\theta_W}$ and $c_W \equiv \cos{\theta_W}$ with $\theta_W$ being the Weinberg angle, $m_Z$ is the $Z$-boson mass and $\Gamma_Z$ is the $Z$-boson width. For $m_{\tilde{\mu}} > \sqrt{s}/2$ or the parameter region of narrow width approximation invalid~\cite{Berdine:2007uv}, for example the compressed region $m_{\tilde{\chi}_1^0} \lesssim m_{\tilde{\mu}}$ and $m_{\tilde{\mu}} \lesssim \sqrt{s}/2$, the cross section is well described by $2\to 3$ process 
\begin{equation}\label{eq:xsect23}
	\sigma(\tilde{\mu}\tilde{\mu} \to  \mu^+ \tilde{\chi}_1^0 \mu^- \tilde{\chi}_1^0)= \sigma(\tilde{\mu} \mu \tilde{\chi}_1^0) \times {\rm BR}(\tilde{\mu} \to \mu \tilde{\chi}_1^0),
\end{equation}
where one smuon is produced off-shell and thus the smuon width should be correctly taken into account\footnote{A similar study of probing slepton like particle $S$ via the off-shell $2\to 3$ process can be found in Ref.~\cite{Liu:2021mhn}. }. 

\par In practice, we use the cross section of $2\to3$ process in Eq.~(\ref{eq:xsect23}) in the whole simplified model parameter space. The cross sections of both signal and background are calculated by the package \textsc{Madgraph5aMC@NLO}~\cite{Alwall:2008pm}, and the smuon width is evaluated by the package \textsc{SUSY-HIT}~\cite{Djouadi:2006bz}. The package \textsc{Pythia8} is adopted for parton shower. The detector simulation, event reconstruction and analysis are performed in the package \textsc{Rivet-3.1.6}~\cite{Buckley:2010ar}. 

\subsection{Event selection }
\par The SM background involves all processes that contain one opposite-sign (OS) muon pair and large missing momentum. The first dominated process is the direct $\tau^+\tau^-$ production, with $\tau$ leptons decay into muons. The second major background comes from the processes involving weak bosons $W$/$Z$ and/or Higgs boson $h$ in the intermediate states. We separate this background into three classes: double-resonance background, single-resonance background and zero-resonance background, as defined in ~\cite{Cao:2018ywk}.  

\par The events containing exactly one OS muon pair and no reconstructed jet objects are selected, the energy of both muons is required larger than 1 GeV and $|\eta| < 3.0$. A lower cut on the invariant mass of the recoil system, $m_{\rm Recoil} > 1~{\rm GeV}$, is used to reject direct $\mu\mu$ production and some other SM processes without large recoil mass. After this pre-selection, $m_{\rm RC}^{\rm min}$, $m_{\rm RC}^{\rm max}$, $m_{\rm LSP}^{\rm max}$ and $m_{\rm RC}^{\rm max}- m_{\rm LSP}^{\rm max}$ distributions for the SM background and smuon pair production process are shown in Fig.~\ref{fig:cut}. 
%%% fig.7 
\begin{figure}
	\centering
	\makebox[\textwidth][c]{
	\includegraphics[width=0.3\linewidth]{cutmRCmin.pdf}\hspace{-0.1cm}
	\includegraphics[width=0.3\linewidth]{cutmRCmax.pdf}\hspace{-0.1cm}
	\includegraphics[width=0.3\linewidth]{cutmLSPmax.pdf}\hspace{-0.1cm}
	\includegraphics[width=0.3\linewidth]{cutdMRC.pdf}
	}
	\caption{\label{fig:cut}Distributions of variables after pre-selection criteria for SM backgrounds and smuon pair production process with $m_{\tilde{\mu}} = 115~{\rm GeV}$ and $m_{\tilde{\chi}_1^0} = 0, 40, 80, 110~{\rm GeV}$. The error bands indicate the Monte-Carlo statistical uncertainties.}
\end{figure}

%%%%table 1
\begin{table}[h]
\centering
\begin{tabular}{r|rr}
\hline
\hline
SR 							& SRH-01 		& SRH-02 		\\ \hline
$E_{\mu^\pm}$ [GeV] 			& - ~~ 		& $\in (50, 70)$ 		\\ 
$m_{\rm RC}^{\rm max}$ [GeV] 	& $> 110$ 	& $> 117$ 		\\ 
$m_{\rm RC}^{\rm min}$ [GeV] 	& $> 85$  	& $> 95$ 			\\ 
\hline\hline
\end{tabular}
\caption{\label{tab:sr-high} Summary of selection requirements for signal category ``SRH''.}
\end{table}

\par The signal regions (SRs) are optimized for smuon discovery by the kinematic selection criteria. These 12 SRs are not orthogonal, which can be classified into three SR categories SRH, SRM and SRL to cover the whole simplified model parameter space. 
\begin{itemize}
	\item For all SRs, events with the invariant mass $m_{\mu\mu}$ and/or the recoil mass $m_{\rm Recoil}$ in $Z$-window of $10~{\rm GeV}$ is vetoed to suppress the processes with $Z$-boson resonance. The cut $m_{\rm RC}^{\rm max} > 110~{\rm GeV}$  is used to improve the statistical significance of a higher smuon mass without losing signals. 
 
	\item The SRH category targets to the region with $m_{\tilde{\chi}_1^0} < 40~{\rm GeV}$ and contains two bins, which are summarized in Table~\ref{tab:sr-high}. SRH-02 is a subset of SRH-01, which targets to the off-shell smuon production region. The cuts on the $m_{\rm RC}^{\rm min}$ distribution is aiming to suppress the $WW$ background. 
 
	\item In the SRM category, five SRs are designed for the region $m_{\tilde{\chi}_1^0}\in (40, 90)~{\rm GeV}$, which are summarized in Table~\ref{tab:sr-mid}. The value of $m_{\rm LSP}^{\rm max}$ is required in a narrow interval for a specific $m_{\tilde{\chi}_1^0}$ value. SRM-01 is designed for $m_{\tilde{\chi}_1^0} \sim 40~{\rm GeV}$, as shown in Fig.~\ref{fig:cut}, $m_{\rm RC}^{\rm min} > 85~{\rm GeV}$ can preserve a large part of the signal events. While for $m_{\tilde{\chi}_1^0} \gtrsim 55~{\rm GeV}$, the $m_{\rm RC}^{\rm min}$ distribution of the signal is drowned in $WW$ and $W\mu\nu$ background, so $m_{\rm RC}^{\rm min}(40)$, $m_{\rm RC}^{\rm min}(70)$ and $m_{\rm RC}^{\rm min}(80)$ are used instead. 
 
	\item The SRL category is designed for the region $m_{\tilde{\chi}_1^0} > 80~{\rm GeV}$. A variable $m_{\rm RC}^{\rm max} - m_{\rm LSP}^{\rm max}$ is used for representing the mass splitting between smuon and LSP $\Delta m ({\tilde{\mu}}, {\tilde{\chi}_1^0})$. In order to improve the signal sensitivities, five SRs are defined for $\Delta m ({\tilde{\mu}}, {\tilde{\chi}_1^0})$ in different bins, which are summarized in Table~\ref{tab:sr-low}.
\end{itemize}

%%%%table 2
\begin{table}[h]
\centering
%\resizebox{\linewidth}{!}{
\begin{tabular}{r|rrrrr}
\hline
\hline
SR 							& SRM-01 			& SRM-02 			& SRM-03	& SRM-04 	& SRM-05 \\ \hline
$m_{\rm RC}^{\rm max}$ [GeV] 	& \multicolumn{5}{c}{$> 110~{\rm GeV}$} 		\\ 
$m_{\rm LSP}^{\rm max}$ [GeV]  	& $\in (40, 60)$  & $\in (50, 70)$ 	& $\in (60, 80)$ & $\in (70, 85)$ & $\in (80, 95)$	\\ 
$m_{\rm RC}^{\rm min}$ [GeV]  	& $> 85$  & - 	& -  & - & -		\\ 
$m_{\rm RC}^{\rm max}(40)$ [GeV]  	& $> 110$  & $> 110$ & - & - & -		\\ 
$m_{\rm RC}^{\rm min}(40)$ [GeV]  	& -  & $> 100$ 	& $> 95$  & - & -		\\ 
$m_{\rm RC}^{\rm min}(70)$ [GeV]  	& -  & - & - & $> 100$ & -		\\ 
$m_{\rm RC}^{\rm min}(80)$ [GeV]  	& -  & - &   & - & $> 105$	\\ 
\hline\hline
\end{tabular}
%}
\caption{\label{tab:sr-mid} Summary of selection requirements for signal category ``SRM''.}
\end{table}

%%%%table 3
\begin{table}[h]
%\resizebox{\linewidth}{!}{
\centering
\begin{tabular}{r|rrrrr}
\hline
\hline
SR 		& SRL-01 	& SRL-02 	& SRL-03	& SRL-04 	& SRL-05 \\ \hline
$m_{\rm RC}^{\rm max}$ [GeV] 	& \multicolumn{5}{c}{$> 110~{\rm GeV}$} 		\\ 
$m_{\rm RC}^{\rm max} - m_{\rm LSP}^{\rm max}$ [GeV]  	& $\in (35, 50)$  & $\in (25, 40)$ 	& $\in (15, 30)$ & $\in (0, 20)$ & $\in (0, 10)$		\\ 
$E_{\mu^\pm}$ [GeV]  	& $\in (34, 44)$  & $\in (28, 37)$	& $\in (22, 28)$  & $\in (15, 18)$ & -		\\ 
\hline\hline
\end{tabular}
%}
\caption{\label{tab:sr-low} Summary of selection requirements for signal category ``SRL''.}
\end{table}
 
\par In most of SRs, the dominated backgrounds are $WW$ and $W\mu\nu$ events produced through weak interaction, which enforce spin correlation among the beam particles and final state particles. Since smuon is a scalar particle, the decay products of smuon distribute isotropically in space and lead to a flat angular distribution. Motivated by these angular properties, Ref.~\cite{Cao:2018ywk} used the observable $\cos{\theta_{\mu^\pm}}$ to suppress the backgrounds, where $\theta_{\mu^\pm}$ is the polar angle of the muon $\mu^{\pm}$ with respect to the direction of the $e^+$ beam. And in ~\cite{Yuan:2022ykg}, the authors used azimuth angles and cone size variables from similar motivations. However, to illustrate the power of our reconstructed mass variables, we do not use any angular variable.   

\subsection{Results}
%%% fig.8 
\begin{figure}[th]
	\centering
	\includegraphics[width=0.7\linewidth]{SRs.pdf}
	\caption{\label{fig:SRs}The expected SM background and signal yields in the SRs. The statistical uncertainties in the background prediction are included in the uncertainty bands.}
\end{figure}
%%% fig.9
\begin{figure}[th]
	\centering
	\includegraphics[width=0.8\linewidth]{excl.pdf}
	\caption{\label{fig:res} Projected exclusion and discovery reach for direct smuon production. The shading colors are coded with the cross sections. Contours labeled by ``arXiv:2203.10580'' shows our recast result of Ref.~\cite{Yuan:2022ykg} for comparison. }
\end{figure}

The expected event number of the SM background and several signal benchmark points in binned SRs are shown in Fig.~\ref{fig:SRs}. The cross section for $m_{\tilde{\mu}}=115~{\rm GeV}$ is $23.9~{\rm fb}$. From Fig.~\ref{fig:SRs}, one can find that for the signals the acceptance rate of the most sensitive SR is around 10\%. While for the SM background, the SM events in some bins, such as SRH-02 and SRL-05, are suppressed below one thousand.   So this application shows that our reconstructed mass variables are quite useful in smuon pair production. 

\par As shown in Fig.~\ref{fig:res}, we calculate the expected $2\sigma$ and $5\sigma$ sensitivities for direct smuon production signals in the plane of $m_{\tilde{\mu}}$ versus $m_{\tilde{\chi}_1^0}$, where the shading color represents the signal cross section in Eq.~(\ref{eq:xsect23}). Since the ISR and beamstrahlung effect and the object reconstruction efficiency of detector are not well considered, for comparison we also recast the SRs in Ref.~\cite{Yuan:2022ykg} and reproduce the exclusion and the discovery regions. From Fig.~\ref{fig:res}, we can see that the detection (discovery) limit for a smuon can reach up to 126 GeV (122 GeV) for a massless $\tilde{\chi}_1^0$, corresponding to a smuon production cross section of $0.16~{\rm fb}$ ($0.34~{\rm fb}$).  Both exclusion and discovery limits of smuon mass can cross $\sqrt{s}\big/2$, which is very impressing. This means that the detection limit of the semi-invisible process at the CEPC can be pushed to $\mathcal{O}(0.1~{\rm fb})$,  more than an order of magnitude lower than the limit obtained in ~\cite{Yuan:2022ykg}. 

\section{\label{sec:sum}Summary}	
\par In summary, we introduced a new approach to search for semi-invisible decaying particles pair-produced at lepton colliders. In view of event reconstruction, the physical phase space of unknown degrees of freedom of invisible states forms a round disc in the center-of-mass frame. We found that the geometry property of the disc can be translated into a set of variables to reconstruct the mass information. These variables, $m_{\rm RC}^{\rm min}$, $m_{\rm RC}^{\rm max}$ and $m_{\rm LSP}^{\rm max}$, are built in a Lorentz-invariant way. In prospect, the variables can determine the $W$-boson mass via its full-leptonic decay events, and can also examine the detailed distributions for various new physics models containing DM particles to enhance the signal significance. As an example, we  focused on the signal of smuon pair production with final states containing an OS muon pair and large missing energy. Our detailed Monte Carlo simulation results showed that the detection limit for the smuon pair production cross section can reach to $0.179~{\rm fb}$. Although the primary focus of this work is the solely di-muon final states, our approach can be readily applied to any final states. We can simply assign to the $\bf V$-system the reconstructed particles consistent with the expected decays of parent particles, with additional visible states associated to the ISR system. Each $\bf V$ and ISR system may contain more than one final state object. For this case, the spirit of our approach remains unchanged and the additional handles of object identification and multiplicity can prove valuable for measurement and/or for further reducing backgrounds. The code of this work is available in \href{https://github.com/Buding820/mRC-variables}{GitHub page}. 

\section*{Acknowledgments}
The authors would like to thank Prof. Gang Li, Prof. Feng Lv and Prof. Xuai Zhuang for helpful discussion of event generation and background estimation. 
This work was supported by the National Natural Science Foundation of China 
(NNSFC) under grant Nos. 11821505, 12075300  and 12105248,  
by the Key Research Project of Henan Education Department for colleges and universities under grant number 21A140025,
by Peng-Huan-Wu Theoretical Physics Innovation Center (12047503),
by the CAS Center for Excellence in Particle Physics (CCEPP), 
and by the Key Research Program of the Chinese Academy of Sciences, Grant NO. XDPB15.

\bibliography{references.bib}
\bibliographystyle{CitationStyle}
\end{document}
