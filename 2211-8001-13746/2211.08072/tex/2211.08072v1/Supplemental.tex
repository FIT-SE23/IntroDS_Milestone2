\documentclass[
reprint,
superscriptaddress,
amsmath,amssymb,
aps,
onecolumn,
floatfix,
longbibliography
]{revtex4-2}

\usepackage{graphicx}% Include figure files
\usepackage{dcolumn}% Align table columns on decimal point
\usepackage{bm}% bold math
\usepackage{graphicx}
\usepackage{currfile}
%\usepackage{dcolumn}% 
\usepackage{multirow}
\usepackage{chemformula}
\usepackage{bm,soul}
\usepackage{color}
%\usepackage{hyperref}
%\usepackage[all]{hypcap}
\usepackage{gensymb}
\usepackage{amsmath}
\usepackage{amssymb}
\usepackage{dcolumn}
\usepackage{comment}

\newcommand{\magenta}[1]{{\color{magenta} #1}}

\newcommand{\Ang}{\AA$^{-1}$}
\newcommand{\etal}{\textit{et al.}}
\newcommand{\NI}{\noindent}
\newcommand{\NL}{$\newline$}
\newcommand{\ueV}{$\mu$eV}
\newcommand{\XE}{\ensuremath{\chi''(E)}}
\newcommand{\XQE}{\ensuremath{\chi''(Q,E)}}
\newcommand{\SQE}{\ensuremath{S(Q,E)}}
\newcommand{\minus}{\scalebox{0.75}[1.0]{$-$}}

\newcommand{\uSR}{$\mu$SR}
\newcommand{\uB}{\ensuremath{\mu_\textrm{B}}}
\newcommand{\herb}{ZnCu$_3$(OH)$_{6}$Cl$_{2}$}
\newcommand{\barlo}{Cu$_4$(OH)$_{6}$FBr}
\newcommand{\znbarlo}{ZnCu$_3$(OH)$_{6}$FBr}

\newcommand{\clari}{Cu$_4$(OH)$_{6}$FCl}
\newcommand{\clarid}{Cu$_4$(OD)$_{6}$FCl}
\newcommand{\znclari}{ZnCu$_3$(OH)$_{6}$FCl}
\newcommand{\znclarid}{ZnCu$_3$(OD)$_{6}$FCl}
\newcommand{\znclariddis}{Zn$_{0.74}$Cu$_{3.26}$(OD)$_{6}$FCl}
\newcommand{\xclari}{Zn$_x$Cu$_{4-x}$(OH)$_{6}$FCl}
\newcommand{\xclarid}{Zn$_x$Cu$_{4-x}$(OD)$_{6}$FCl}

\renewcommand*{\thesection}{\Roman{section}}
\renewcommand{\thefigure}{S\arabic{figure}}
\renewcommand{\thetable}{S\arabic{table}}

% % %   SUPPLEMENTAL   % % % 
\begin{document}

\title{Supplemental Material for \\ ``Magnetically ordered and kagome quantum spin liquid states in the Zn-doped claringbullite series''}

\author{M.~Georgopoulou} 
\affiliation{Institut Laue-Langevin, CS 20156, 38042 Grenoble Cedex 9, France}
\affiliation{Department of Chemistry, University College London, 20 Gordon Street, London, WC1H 0AJ, United Kingdom}

\author{B.~F\aa k}
\email{fak@ill.fr}
\affiliation{Institut Laue-Langevin, CS 20156, 38042 Grenoble Cedex 9, France}

\author{D. Boldrin} 
\affiliation{SUPA, School of Physics and Astronomy, University of Glasgow, Glasgow, G12 8QQ, United Kingdom}

\author{J. R. Stewart} 
\affiliation{ISIS Neutron and Muon Facility, Rutherford Appleton Laboratory, Science and Technology Facilities Council, Didcot OX11 0QX, UK}

\author{C. Ritter} 
\affiliation{Institut Laue-Langevin, CS 20156, 38042 Grenoble Cedex 9, France}

\author{E. Suard} 
\affiliation{Institut Laue-Langevin, CS 20156, 38042 Grenoble Cedex 9, France}

\author{J.~Ollivier} 
\affiliation{Institut Laue-Langevin, CS 20156, 38042 Grenoble Cedex 9, France}

\author{A.~S.~Wills} 
\affiliation{Department of Chemistry, University College London, 20 Gordon Street, London, WC1H 0AJ, United Kingdom}

\maketitle

\section{Sample synthesis}
Claringbullite, \ch{Cu4(OD)6FCl}, was synthesised by combining \ch{Cu2(OH)2CO3} (Sigma, 0.275~g, 1.25~mmol), \ch{CuCl2 * }2\ch{H2O} (Sigma, 0.215~g, 1.26~mmol), \ch{NH4F} (Alfa Aesar, 0.093~g, 2.50~mmol), \ch{HCl} (Sigma, 0.05~g, 37\% w.t. solution) and \ch{D2O} (10~mL) in a 15~mL Teflon-lined steel autoclave.

For Zn-claringbullite, with nominal stoichiometry \ch{ZnCu_3(OD)6FCl}, \ch{Cu2(OH)2CO3} (Aldrich, 0.250~g, 1.13~mmol), \ch{ZnCl2 * }$x$\ch{H2O} (Alfa, 0.190~g, 0.910~mmol), \ch{NH4F} (Alfa Aesar, 0.048~g, 1.29~mmol), \ch{HCl} (Sigma, 0.1~g, 37\% w.t.\ solution) and \ch{D2O} (10~mL) were placed in a 15~mL Teflon-lined steel autoclave. 

For both syntheses, the autoclaves were held at 200~$^\circ$C for 24~h and oven cooled to room temperature over 1~h. The products were washed {\textit{via}} centrifugation with \ch{D2O} (3 x 10~mL) and dried in an oven at 60~$^\circ$C for 4~h. Each synthesis produced approximately 0.2~g of material that were combined into a $\sim5$~g batch for Cu4 and $\sim7$~g batch for ZnCu3.

\section{Neutron diffraction}

Neutron diffraction measurements were carried out on D2B for the deuterated claringbullite and Zn-doped claringbullite samples. The diffraction data measured at $T=1.5$~K and Rietveld refinements  are shown in Figs.~\ref{Cu4RietveldRefinement} and \ref{ZnCu3RietveldRefinement}, with the refined atomic parameters given in Tables~\ref{Cu4RietveldRefinementResults} and \ref{ZnCu3RietveldRefinementResults} for claringbullite and Zn-doped claringbullite, respectively. In the case of Zn-claringbullite, the Cu kagome site occupation refined close to unity and was subsequently fixed to 1. To estimate the site disorder at the interlayer site, a Cu atom was placed on the $6h$ site and its isotropic displacement parameter was fixed to that previously determined for Zn-doped barlowite at $T=1.5$~K using neutron scattering data \cite{Tustain2020}. The Zn $2d$ and Cu $6h$ site occupations were refined such that their sum was equal to the nominal stoichiometry of a divalent metal ion. The possibility of the interlayer Cu occupying the $12j$ site was explored, but there was no evidence from our refinements for Cu to occupy the $12j$ instead of the $6h$ site as previously proposed for Zn-doped barlowite \cite{Smaha2020}.

\begin{figure}[!h]
\centering
\includegraphics[width=0.9\columnwidth]{Figures/FigS1Cu4D2B1p5K.png}
\caption{Claringbullite, \clarid, D2B data collected at $T = 1.5$~K (black) with $\lambda = 1.595226$~\AA. Rietveld refinement (red) in the Pnma space group with 72 variables and goodness-of-fit parameters $\chi^2 = 2.69$ and $R_\mathrm{wp} = 4.17$. The difference between the experimental data and the fit is shown in grey and the peak positions are in blue.}
\label{Cu4RietveldRefinement}
\end{figure}

\begin{table}[hb!]
    \centering
    \begin{tabular}{p{2.5cm} p{2.5cm} p{2.5cm} p{2.5cm} p{2.5cm} p{2.5cm}} \hline \hline
    \multicolumn{6}{c}{Lattice parameters} \\ \hline
    $a$ (\AA) & $b$ (\AA) &  $c$ (\AA) & $\alpha$ (\degree) & $\beta$ (\degree) & $\gamma$ (\degree) \\ \hline
     11.5359(9) & 9.1510(7) &  6.6848(5) & 90 & 90 & 90 \\ \hline
     \multicolumn{6}{c}{Atomic parameters} \\ \hline 
     Atom  & Wyckoff site & $x$ & $y$ & $z$ & $U_\mathrm{iso}$ (\AA$^2$) \\ \hline
 Cu1 &  $4a$ &  0 & 0 & 0  & 0.00341(11) \\
 Cu2 & $8d$ & 0.2492(4) & 0.5121(3)  & 0.2470(5)  & 0.00238(57)  \\
 Cu3 & $4c$ & 0.1870(6) & 1/4 & 0.0592(6) & 0.00381(65)  \\
 F & $4c$ & 0.5007(11)  & 1/4 & 0.0041(13) & 0.00876(57)  \\
  Cl &  $4c$ &  0.3304(5) &  1/4 &  0.5049(6) & 0.00616(37)  \\
 O1 & $8d$ & 0.2961(5) & 0.0921(6)  & 0.00062(67)  & 0.0054(12)  \\
 O2   & $8d$ & 0.1022(5)  & 0.0919(5)  & 0.1989(7) &  0.00118(90) \\
 O3  & $8d$ & 0.4003(5)  & 0.5889(5)  & 0.3000(7)  & 0.00062(87)  \\
 D1  & $8d$ & 0.3765(7)  & 0.1342(7)  & 1.0030(8)  & 0.0150(11)  \\
 D2   & $8d$ & 0.0611(5)  & 0.1294(5)  & 0.3161(7) &  0.0139(11)  \\
 D3  & $8d$ & 0.4401(5)  & 0.6381(6) &  0.1912(7)  & 0.0173(11)  \\
    \hline \hline
   
\end{tabular}
    \caption{Claringbullite, \clarid. Lattice parameters, atomic positions and isotropic thermal parameters from the Rietveld refinement in the $Pnma$ space group using data collected on D2B ($\lambda=1.595226$~\AA) at $T=1.5$~K. All sites are fully occupied.}
    \label{Cu4RietveldRefinementResults}
\end{table}


\begin{figure}[!h]
\centering
\includegraphics[width=0.9\columnwidth]{Figures/FigS2ZnCu3D2B1p5K.png}
\caption{Zn-claringbullite. Neutron diffraction data collected on D2B at $T = 1.5$~K (black) with $\lambda = 1.595226$~\AA. Rietveld refinement (red) in the $P6_3/mmc$ space group with 48 variables and goodness-of-fit parameters $\chi^2 = 2.28$ and $R_\mathrm{wp} = 4.60$. The D site refined to 0.96 showing good deuteration of the sample and was set to unity for the final refinement. The occupancy of the interlayer position refined to $\sim74$\% Zn occupation of the $2d$ site and $\sim8.5\%$ Cu occupation of the $6h$ site (see Table \ref{ZnCu3RietveldRefinementResults}). The difference between the experimental data and the fit is shown in grey and the peak positions are in blue. }
\label{ZnCu3RietveldRefinement}
\end{figure}

\begin{table}[hb!]
    \centering
    \begin{tabular}{p{2.2cm} p{2.2cm} p{2.2cm} p{2.2cm} p{2.2cm} p{2.2cm} p{2.2cm} } \hline \hline
    \multicolumn{7}{c}{Lattice parameters} \\ \hline
    $a$ (\AA) & $b$ (\AA) &  $c$ (\AA) & $\alpha$ (\degree) & $\beta$ (\degree) & $\gamma$ (\degree) & \\ \hline
     6.65918(6) & 6.65918(6) &  9.17288(9) & 90 & 90 & 120 \\ \hline
    \multicolumn{7}{c}{Atomic parameters} \\ \hline
     Atom  & Wyckoff site & $x$ & $y$ & $z$ & Occupation & $U_\mathrm{iso}$ (\AA$^2$) \\ \hline
  Cu &  $6g$ &  0.5 & 0 & 0  & 1 & 0.0078 \\
 Zn & $2d$ & 1/3 & 2/3  & 3/4  & 0.74(2) & 0.0244  \\
  Cu & $6h$ & 0.3086(10) & 0.6172(19) & 3/4  & 0.085(8) & 0.0023 \cite{Tustain2020}  \\
 F & $2b$ & 0  & 0 & 3/4 & 1 & 0.0064  \\
 Cl &  $2c$ &  2/3 &  1/3 &  3/4 & 1 & 0.0075 \\
 O & $12k$ & 0.20184(10) & 0.79816(10) & 0.90838(11) & 1 & 0.0038  \\
 D  & $12k$ & 0.12414(9) & 0.87586(9) & 0.86584(11) & 1 & 0.0151 \\
    \hline 
    \multicolumn{7}{c}{Anisotropic displacement parameters (\AA$^2$)} \\ \hline 
  Atom &  U$_{11}$ & U$_{22}$ & U$_{33}$ & U$_{12}$ & U$_{13}$ & U$_{23}$ \\ \hline 
Cu & 0.0007(3) & 0.0007(3) & 0.0129(5) & 0.00034(15) & -0.0014(2) & -0.0027(4) \\
Zn & 0.016(3) & 0.016(3) & 0.0004(19) & 0.0081(13) & 0 & 0 \\
Cl & 0.0068(5) & 0.0068(5) & 0.0082(9) & 0.0034(2) & 0 & 0 \\
F & 0.0040(7) & 0.0040(7) & 0.0133(13) & 0.0020(3) & 0 & 0 \\
O & 0.0016(4) & 0.0016(4) & 0.0089(5) & 0.0009(5) &  -0.0010(2) & 0.0010(2)\\ 
D & 0.0132(5) & 0.0132(5) & 0.0217(6) & 0.0081(5) & -0.0007(2) & 0.0007(2) \\ \hline \hline
   
\end{tabular}
    \caption{Zn-claringbullite. Atomic positions and displacement parameters from Rietveld refinement in the $P6_3/mmc$ space group using D2B data ($\lambda=1.595226$~\AA) collected at $T=1.5$\,K. The D site refined to 0.96 showing good deuteration of the sample and was set to unity for the final refinement. The anisotropic displacements were refined for all atoms except the Cu on the $6h$ site. The refined stoichiometry is \znclariddis.}
    \label{ZnCu3RietveldRefinementResults}
\end{table}


\clearpage

\section{Bulk magnetometry}
DC susceptibility measurements for claringbullite and Zn-claringbullite were carried out on a SQUID MPMS XL Quantum Design at the Institut N\'eel and a SQUID Quantum Design MPMS3 at the University of Glasgow. 
Magnetization measured as a function of temperature for claringbullite is shown in Fig.~\ref{FigCu4Susceptibility} and hysteresis loops at various temperatures are shown in Fig.~\ref{FigCu4MvsH}. Similarly for Zn-claringbullite, magnetization as a function of temperature and field are shown in Fig.~\ref{FigZnCu3Magnetometry}.

\begin{figure}[!h]
\centering
\includegraphics[width=0.9\columnwidth, trim=4 4 4 4,clip]{Figures/FigS3Cu4Susceptibility.png}
   
\caption{Claringbullite. {\bf a.} Field-cooled (red) and zero-field-cooled (black) magnetic susceptibility data collected in a field of 1000~G. There is a transition at $T_\mathrm{N}$~=~17~K. {\bf b.} First derivative of the susceptibility to more clearly show the transition at 17~K. {\bf c.} Inverse susceptibility, $\chi^{-1}$, as a function of temperature, $T$, of field-cooled and zero-field-cooled data collected in 1000~G with a linear Curie-Weiss fit (black) between 150~K and 320~K that gives $\theta_\mathrm{W}=-136(3)$~K. {\bf d.} Effective magnetic moment, $\mu_{\mathrm{eff}}$, as a function of temperature, $T$, calculated using $\mu_{\mathrm{eff}}=\sqrt{8\chi T} $.}
\label{FigCu4Susceptibility}
\end{figure}

\begin{figure}[!h]
\centering
\includegraphics[width=0.9\columnwidth, trim=4 4 4 4,clip]{Figures/FigS4Cu4MvsH.png}
   
\caption{Claringbullite. Magnetisation, $M$, as a function of field, $H$. \textbf{a.} Data between $T=2$~K and 50~K show a hysteresis loop opening at $T\leq 15$~K. \textbf{b.} At $T=2$~K the spontaneous moment is 0.017~\uB~Cu$^{-1}$ with a coercivity of 0.05~T.}
\label{FigCu4MvsH}
\end{figure}

\begin{figure}[!h]
\centering
\includegraphics[width=0.9\columnwidth]{Figures/FigS5ZnCu3Susceptibility.png}
   
\caption{Zn-claringbullite. \textbf{a.} Field-cooled (red) and zero-field-cooled (black) magnetic susceptibility data collected in a field of 1000 G. There is no transition down to 2\,K. \textbf{b.} Inverse susceptibility, ${\chi}^{-1}$, of field-cooled and zero-field-cooled data collected in 1000 G with a linear Curie-Weiss fit (black) between 150\,K and 320\,K that gives $\theta_\mathrm{W} = -206(1)$\,K. \textbf{c.} Effective magnetic moment, $\mu_{\textrm{eff}}$, as a function of temperature, $T$, calculated using $\mu_{\textrm{eff}}=\sqrt{8\chi T}$. \textbf{d.} Magnetisation, $M$, as a function of field, $H$, at $T=2$\,K showing a hysteresis loop with a spontaneous moment of $\sim2\times10^{-4}$~~\uB~Cu$^{-1}$ and a coercivity of $\sim3\times10^{-3}$\,T.}
\label{FigZnCu3Magnetometry}
\end{figure}

\clearpage

\section{Magnetic structure}
The magnetic structure of \clarid\ was refined using representation analysis in the space group $Pnma$ (no.~62) using the program SARAh \cite{Wills2000ASARAh} and the refinement is shown in Fig. \ref{FigCu4MagStrRefinement}. The basis vectors for $\Gamma_7$ are given in Tables \ref{cu4cbull_cu1_BVs}-\ref{cu4cbull_cu3_BVs}. The notation for the irreducible representations corresponds to that of Kovalev.

\begin{figure}[!hb]
\centering
\includegraphics[width=0.9\columnwidth]{Figures/FigS6Cu4MagStrRefinement.png}
   
\caption{Claringbullite. Magnetic structure refinement (red) with the IR $\Gamma_7$ using temperature subtracted data collected on D20 (blue). The magnetic Bragg peak positions are shown in green and the difference plot in black. The grey regions arise from the subtraction of nuclear peaks and were excluded from the refinements. The refined components of the magnetic moments are shown in Table \ref{cu4cbull_refined_moments}. }
\label{FigCu4MagStrRefinement}
\end{figure}

\begin{table*}[!h]
    \centering
    \begin{tabular}{c c c c c c} \hline \hline
     Atom number & Coordinates & {Basis vector} & {m$_a$} & m$_b$ &{m$_c$} \\ \hline
     Atom 1 & (0,~0,~0) & $\psi_1$ &  1 & 0 & 0 \\
    & & $\psi_2$ & 0 &  1 &  0  \\
     & & $\psi_3$ & 0 & 0 & 1  \\ 
      Atom 2 & (1/2,~1/2,~1/2) & $\psi_1$ & \minus 1 & 0 & 0 \\ 
      & & $\psi_2$ & 0 & 1 &  0  \\
     & & $\psi_3$ & 0 & 0 & 1  \\
     Atom 3 & (0,~1/2,~0) & $\psi_1$ &  1 & 0 & 0 \\
    & & $\psi_2$ & 0 & \minus 1 &  0  \\
     & & $\psi_3$ & 0 & 0 & 1  \\ 
     Atom 4 & (1/2,~0,~1/2) & $\psi_1$ & \minus 1 & 0 & 0 \\
    & & $\psi_2$ & 0 & \minus 1 &  0  \\
     & & $\psi_3$ & 0 & 0 & 1  \\ 
     
     \hline \hline
   
\end{tabular}
    \caption{The basis vectors of the $\Gamma_7$ irreducible representation of the space group $Pnma$ with $\mathbf{k}=(0,~0,~0)$ for the Cu1 $4a$ site. m$_a$, m$_b$ and m$_c$ represent the components with respect to the crystallographic axes.}
    \label{cu4cbull_cu1_BVs}
\end{table*}

\begin{table*}[!h]
    \centering
    \begin{tabular}{c c c c c c} \hline \hline
      Atom number & Coordinates & {Basis vector} & {m$_a$} & m$_b$ &{m$_c$} \\ \hline
       Atom 1 & (0.247,~0.496,~0.249) & $\psi_1$ & 1 & 0 & 0 \\ 
      & & $\psi_2$ & 0 & 1 &  0  \\
     & & $\psi_3$ & 0 & 0 & 1  \\
     Atom 2 & (0.747,~0.004,~0.251) & $\psi_1$ & \minus 1 & 0 & 0 \\
    & & $\psi_2$ & 0 & 1 &  0  \\
     & & $\psi_3$ & 0 & 0 & 1  \\ 
     Atom 3 & (0.753,~0.996,~0.751) & $\psi_1$ & 1 & 0 & 0 \\
    & & $\psi_2$ & 0 & \minus 1 &  0  \\
     & & $\psi_3$ & 0 & 0 & 1  \\ 
     Atom 4 & (0.253,~0.504,~0.743) & $\psi_1$ & \minus 1 & 0 & 0 \\
    & & $\psi_2$ & 0 & \minus 1 &  0  \\
     & & $\psi_3$ & 0 & 0 & 1 \\ 
     Atom 5 & (0.753,~0.504,~0.751) & $\psi_1$ &  1 & 0 & 0 \\
    & & $\psi_2$ & 0 & 1 &  0  \\
     & & $\psi_3$ & 0 & 0 & 1  \\
     Atom 6 & (0.253,~0.996,~0.749) & $\psi_1$ & \minus 1 & 0 & 0 \\
    & & $\psi_2$ & 0 & 1 &  0  \\
     & & $\psi_3$ & 0 & 0 & 1  \\
     Atom 7 & (0.247,~0.004,~0.249) & $\psi_1$ & 1 & 0 & 0 \\
    & & $\psi_2$ & 0 & \minus 1 &  0  \\
     & & $\psi_3$ & 0 & 0 & 1  \\
      Atom 8 & (0.747,~0.496,~0.251) & $\psi_1$ & \minus 1 & 0 & 0 \\
    & & $\psi_2$ & 0 & \minus 1 &  0  \\
     & & $\psi_3$ & 0 & 0 & 1  \\
       \hline \hline
   
\end{tabular}
    \caption{The basis vectors of the $\Gamma_7$ irreducible representation of the space group $Pnma$ with $\mathbf{k}=(0,~0,~0)$ for the Cu2 $8d$ site. m$_a$, m$_b$ and m$_c$ represent the components with respect to the crystallographic axes.}
    \label{cu4cbull_cu2_BVs}
\end{table*}

\begin{table*}[!htb]
    \centering
    \begin{tabular}{c c c c c c} \hline \hline
    Atom number & Coordinates & {Basis vector} & {m$_a$} & m$_b$ &{m$_c$} \\ \hline
       Atom 1 & (0.191,~1/4,~0.051) & $\psi_1$ & 1 & 0 & 0 \\ 
      & & $\psi_2$ & 0 & 0 &  1  \\
     Atom 2 & (0.691,~1/4,~0.449) & $\psi_1$ & \minus 1 & 0 & 0 \\
    & & $\psi_2$ & 0 & 0 &  1  \\
     Atom 3 & (0.809,~3/4,~0.949) & $\psi_1$ & 1 & 0 & 0 \\
    & & $\psi_2$ & 0 &0 & 1  \\
     Atom 4 & (0.309,~3/4,~0.551) & $\psi_1$ & \minus 1 & 0 & 0 \\
    & & $\psi_2$ & 0 & 0 &  1  \\
      \hline \hline
   
\end{tabular}
    \caption{The basis vectors of the $\Gamma_7$ irreducible representation of the space group $Pnma$ with $\mathbf{k}=(0,~0,~0)$ for the Cu3 $4c$ site. m$_a$, m$_b$ and m$_c$ represent the components with respect to the crystallographic axes.}
    \label{cu4cbull_cu3_BVs}
\end{table*}


\begin{table*}[!htb]
    \centering
    \begin{tabular}{c c c c c c} \hline \hline
     Atom & Wyckoff site & Coordinates & {m$_a$} ($\mu_\mathrm{B}$) & m$_b$ ($\mu_\mathrm{B}$) &{m$_c$} ($\mu_\mathrm{B}$) \\\hline
     Cu1 & $4a$ &  (0,~0,~0) &   0.266 & 0 & 0\\
             & & (1/2,~1/2,~1/2)  & \minus0.266 & 0 & 0\\
             & & (0,~1/2,~0) & 0.266 & 0 & 0\\
             & & (1/2,~0,~1/2) & \minus0.266 & 0 & 0\\
    \hline
    Cu2 & $8d$ &(0.247,~0.496,~0.249) & 0.353 & 0.120 & 0 \\
        &  & (0.747,~0.004,~0.251) & \minus0.353 & 0.120 & 0 \\
        & & (0.753,~0.996,~0.751)  & 0.353 & \minus0.120 & 0 \\
        & & (0.253,~0.504,~0.743) & \minus0.353 & \minus0.120 & 0 \\
       & &  (0.753,~0.504,~0.751) & 0.353 & 0.120 & 0 \\
       & &  (0.253,~0.996,~0.749) & \minus0.353 & 0.120 & 0 \\
       & &  (0.247,~0.004,~0.249) & 0.353 & \minus0.120 & 0 \\
       & & (0.747,~0.496,~0.251)  & \minus0.353 & \minus0.120 & 0 \\

    \hline
    Cu3 &  $4c$ & (0.191,~1/4,~0.051) & 0.491 & - &  0.153 \\
           &  & (0.691,~1/4,~0.449) & \minus0.491 & - &  0.153 \\
           & &  (0.809,~3/4,~0.949) & 0.491 & - &  0.153 \\
          &  & (0.309,~3/4,~0.551) & \minus0.491 & - &  0.153 \\
       \hline \hline
   
\end{tabular}
    \caption{The refined components of the magnetic moments for the equivalent positions of the three Cu sites (Cu1, Cu2 and Cu3) along the $a$, $b$ and $c$ crystallographic directions of the $Pnma$ unit cell. }
    \label{cu4cbull_refined_moments}
\end{table*}


\clearpage
\section{Spin wave model of claringbullite compatible with inelastic neutron scattering data}
Powder averaged spin wave excitations were calculated in SpinW \cite{Toth2015LinearStructures} for \clarid\ using the exchange interactions given in Table~I in the main text and the magnetic structure shown in Fig.~2 in the main text. These are compared to the experimental inelastic neutron scattering data collected on PANTHER and IN5 at the ILL in Fig.~\ref{FigSpinWModels}.

% Figure SpinW models for Cu4
\begin{figure}[!hb]
\centering
\includegraphics[width=0.6\columnwidth, trim=3 1 10 7,clip]{Figures/FigS7Cu4Spinw.png}
   
\caption{Claringbullite calculated spin wave spectra ({\bf left.}) using the exchange interactions in Table~I in the main text, compared to the experimental data ({\bf right.}). The directions of the Dzyaloshinskii-Moriya interactions are as described in the main text. The dashed white lines are the kinematic windows for the corresponding incident neutron energies. $S_E(Q)$ and $S_Q(E)$ scans of the calculated (blue) compared to the experimental (red) data are shown at the {\bf bottom.}}
\label{FigSpinWModels}
\end{figure}

\begin{thebibliography}{4}%
\makeatletter
\providecommand \@ifxundefined [1]{%
 \@ifx{#1\undefined}
}%
\providecommand \@ifnum [1]{%
 \ifnum #1\expandafter \@firstoftwo
 \else \expandafter \@secondoftwo
 \fi
}%
\providecommand \@ifx [1]{%
 \ifx #1\expandafter \@firstoftwo
 \else \expandafter \@secondoftwo
 \fi
}%
\providecommand \natexlab [1]{#1}%
\providecommand \enquote  [1]{``#1''}%
\providecommand \bibnamefont  [1]{#1}%
\providecommand \bibfnamefont [1]{#1}%
\providecommand \citenamefont [1]{#1}%
\providecommand \href@noop [0]{\@secondoftwo}%
\providecommand \href [0]{\begingroup \@sanitize@url \@href}%
\providecommand \@href[1]{\@@startlink{#1}\@@href}%
\providecommand \@@href[1]{\endgroup#1\@@endlink}%
\providecommand \@sanitize@url [0]{\catcode `\\12\catcode `\$12\catcode
  `\&12\catcode `\#12\catcode `\^12\catcode `\_12\catcode `\%12\relax}%
\providecommand \@@startlink[1]{}%
\providecommand \@@endlink[0]{}%
\providecommand \url  [0]{\begingroup\@sanitize@url \@url }%
\providecommand \@url [1]{\endgroup\@href {#1}{\urlprefix }}%
\providecommand \urlprefix  [0]{URL }%
\providecommand \Eprint [0]{\href }%
\providecommand \doibase [0]{https://doi.org/}%
\providecommand \selectlanguage [0]{\@gobble}%
\providecommand \bibinfo  [0]{\@secondoftwo}%
\providecommand \bibfield  [0]{\@secondoftwo}%
\providecommand \translation [1]{[#1]}%
\providecommand \BibitemOpen [0]{}%
\providecommand \bibitemStop [0]{}%
\providecommand \bibitemNoStop [0]{.\EOS\space}%
\providecommand \EOS [0]{\spacefactor3000\relax}%
\providecommand \BibitemShut  [1]{\csname bibitem#1\endcsname}%
\let\auto@bib@innerbib\@empty
%</preamble>
\bibitem [{\citenamefont {Tustain}\ \emph {et~al.}(2020)\citenamefont
  {Tustain}, \citenamefont {Ward-O'Brien}, \citenamefont {Bert}, \citenamefont
  {Han}, \citenamefont {Luetkens}, \citenamefont {Lancaster}, \citenamefont
  {Huddart}, \citenamefont {Baker},\ and\ \citenamefont {Clark}}]{Tustain2020}%
  \BibitemOpen
  \bibfield  {author} {\bibinfo {author} {\bibfnamefont {K.}~\bibnamefont
  {Tustain}}, \bibinfo {author} {\bibfnamefont {B.}~\bibnamefont
  {Ward-O'Brien}}, \bibinfo {author} {\bibfnamefont {F.}~\bibnamefont {Bert}},
  \bibinfo {author} {\bibfnamefont {T.-H.}\ \bibnamefont {Han}}, \bibinfo
  {author} {\bibfnamefont {H.}~\bibnamefont {Luetkens}}, \bibinfo {author}
  {\bibfnamefont {T.}~\bibnamefont {Lancaster}}, \bibinfo {author}
  {\bibfnamefont {B.~M.}\ \bibnamefont {Huddart}}, \bibinfo {author}
  {\bibfnamefont {P.~J.}\ \bibnamefont {Baker}},\ and\ \bibinfo {author}
  {\bibfnamefont {L.}~\bibnamefont {Clark}},\ }\bibfield  {title} {\bibinfo
  {title} {{From magnetic order to quantum disorder in the Zn-barlowite series
  of S = 1/2 kagom{\'{e}} antiferromagnets}},\ }\href
  {https://doi.org/10.1038/s41535-020-00276-4} {\bibfield  {journal} {\bibinfo
  {journal} {npj Quantum Mater.}\ }\textbf {\bibinfo {volume} {5}},\ \bibinfo
  {pages} {74} (\bibinfo {year} {2020})}\BibitemShut {NoStop}%
\bibitem [{\citenamefont {Smaha}\ \emph {et~al.}(2020)\citenamefont {Smaha},
  \citenamefont {He}, \citenamefont {Jiang}, \citenamefont {Wen}, \citenamefont
  {Jiang}, \citenamefont {Sheckelton}, \citenamefont {Titus}, \citenamefont
  {Wang}, \citenamefont {Chen}, \citenamefont {Teat}, \citenamefont {Aczel},
  \citenamefont {Zhao}, \citenamefont {Xu}, \citenamefont {Lynn}, \citenamefont
  {Jiang},\ and\ \citenamefont {Lee}}]{Smaha2020}%
  \BibitemOpen
  \bibfield  {author} {\bibinfo {author} {\bibfnamefont {R.~W.}\ \bibnamefont
  {Smaha}}, \bibinfo {author} {\bibfnamefont {W.}~\bibnamefont {He}}, \bibinfo
  {author} {\bibfnamefont {J.~M.}\ \bibnamefont {Jiang}}, \bibinfo {author}
  {\bibfnamefont {J.}~\bibnamefont {Wen}}, \bibinfo {author} {\bibfnamefont
  {Y.-F.}\ \bibnamefont {Jiang}}, \bibinfo {author} {\bibfnamefont {J.~P.}\
  \bibnamefont {Sheckelton}}, \bibinfo {author} {\bibfnamefont {C.~J.}\
  \bibnamefont {Titus}}, \bibinfo {author} {\bibfnamefont {S.~G.}\ \bibnamefont
  {Wang}}, \bibinfo {author} {\bibfnamefont {Y.-S.}\ \bibnamefont {Chen}},
  \bibinfo {author} {\bibfnamefont {S.~J.}\ \bibnamefont {Teat}}, \bibinfo
  {author} {\bibfnamefont {A.~A.}\ \bibnamefont {Aczel}}, \bibinfo {author}
  {\bibfnamefont {Y.}~\bibnamefont {Zhao}}, \bibinfo {author} {\bibfnamefont
  {G.}~\bibnamefont {Xu}}, \bibinfo {author} {\bibfnamefont {J.~W.}\
  \bibnamefont {Lynn}}, \bibinfo {author} {\bibfnamefont {H.-C.}\ \bibnamefont
  {Jiang}},\ and\ \bibinfo {author} {\bibfnamefont {Y.~S.}\ \bibnamefont
  {Lee}},\ }\bibfield  {title} {\bibinfo {title} {{Materializing rival ground
  states in the barlowite family of kagome magnets: quantum spin liquid, spin
  ordered, and valence bond crystal states}},\ }\href
  {https://doi.org/10.1038/s41535-020-0222-8} {\bibfield  {journal} {\bibinfo
  {journal} {npj Quantum Mater.}\ }\textbf {\bibinfo {volume} {5}},\ \bibinfo
  {pages} {23} (\bibinfo {year} {2020})}\BibitemShut {NoStop}%
\bibitem [{\citenamefont {Wills}(2000)}]{Wills2000ASARAh}%
  \BibitemOpen
  \bibfield  {author} {\bibinfo {author} {\bibfnamefont {A.~S.}\ \bibnamefont
  {Wills}},\ }\bibfield  {title} {\bibinfo {title} {{A new protocol for the
  determination of magnetic structures using simulated annealing and
  representational analysis (SARAh)}},\ }\href
  {https://doi.org/10.1016/S0921-4526(99)01722-6} {\bibfield  {journal}
  {\bibinfo  {journal} {Physica B}\ }\textbf {\bibinfo {volume} {278}},\
  \bibinfo {pages} {680} (\bibinfo {year} {2000})}\BibitemShut {NoStop}%
\bibitem [{\citenamefont {Toth}\ and\ \citenamefont
  {Lake}(2015)}]{Toth2015LinearStructures}%
  \BibitemOpen
  \bibfield  {author} {\bibinfo {author} {\bibfnamefont {S.}~\bibnamefont
  {Toth}}\ and\ \bibinfo {author} {\bibfnamefont {B.}~\bibnamefont {Lake}},\
  }\bibfield  {title} {\bibinfo {title} {{Linear spin wave theory for single-Q
  incommensurate magnetic structures}},\ }\href
  {https://doi.org/http://dx.doi.org/10.1088/0953-8984/27/16/166002} {\bibfield
   {journal} {\bibinfo  {journal} {J. Phys. Condens. Matter}\ }\textbf
  {\bibinfo {volume} {27}},\ \bibinfo {pages} {166002} (\bibinfo {year}
  {2015})}\BibitemShut {NoStop}%
\end{thebibliography}%

\end{document}
