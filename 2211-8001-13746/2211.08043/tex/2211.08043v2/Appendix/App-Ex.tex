%----------------------------------------------------------------------
%%% APP: EX
%----------------------------------------------------------------------
% !TEX root = ./Main.tex
In this appendix, we provide some computational details that were left out of the main text to streamline our presentation.

\begin{example}[name=Hellinger distance,continues=ex:Hell]
We proceed to compute the Taylor expansion of $\fixmap$ near $\sol = -1$ for the shifted operator $\vecfield(\point) = \point+1$.
Indeed, in this case, the fixed point operator $\fixmap$ is given by
%, for $\point \in (-1, 1)$,
\begin{align}
\label{eq:app-fixmap}
\fixmap(\point)
	= \proxof{\point}{-\step \vecfield(\point)}
	&= \proxof{\point}{-\step (\point+1)}
	\notag\\
	&= \frac{\point - \step(\point+1)\sqrt{1-\point^{2}}}{\sqrt{1-\point^{2} + (\point -\step(\point+1)\sqrt{1-\point^{2}})^{2}}}
	\notag\\
	&= \frac{\fixmapalt(\point)}{\sqrt{1-\point^{2} +\fixmapalt(\point)^{2}}},
\end{align}
with $\fixmapalt(\point) = \point - \step(\point+1)\sqrt{1-\point^{2}}$.
Now, the behavior of $\fixmapalt$ near $\sol = -1$ can be approximated as
\begin{align}
\label{eq:app-fixmapalt}
\fixmapalt(\point)
	&=\point - \step(\point+1)^{3/2}(1-\point)^{1/2}
	\notag\\
	&= -1 + (\point+1) - \step(\point+1)^{3/2}(2 - (\point+1))^{1/2}
	\notag\\
%	&= -1 + (\point+1) - \sqrt 2 \step(\point+1)^{3/2}\left(1 - \tfrac{1}{2}\parens*{\point+1}\right)^{1/2}
%	\notag\\
	&= -1 + (\point+1) - \sqrt 2 \step(\point+1)^{3/2}\left(1 - \tfrac{1}{4}\parens*{\point+1} + o(\point+1)\right)
	\notag\\
	&= -1 + (\point+1) - \sqrt 2 \step(\point+1)^{3/2} + \tfrac{\sqrt 2 \step}{4}(\point+1)^{5/2} + o \parens*{(\point+1)^{5/2}}\,.
\end{align}
%Another Taylor expansion allows us to express $\fixmapalt(\point)^{2}$ as
Another Taylor expansion then yields
\begin{align}
\fixmapalt(\point)^{2}
%	&= (-\fixmap(\point))^{2}
%	\notag\\
	&= \parens*{1 - (\point+1) + \sqrt 2 \step(\point+1)^{3/2} - \tfrac{\sqrt 2 \step}{4}(\point+1)^{5/2} + o \parens*{(\point+1)^{5/2}}}^{2}
	\notag\\
	&= 1 - 2(\point+1) + 2\sqrt 2 \step(\point+1)^{3/2} + (\point+1)^{2} - \tfrac{\sqrt 2 \step}{2}(\point+1)^{5/2} + o \parens*{(\point+1)^{5/2}}
\end{align}
so the denominator of \cref{eq:app-fixmap} becomes
\begin{align}
\sqrt{1-\point^{2} +\fixmapalt(\point)^{2}}
	&= \parens*{(\point+1)(2 - (\point+1) + \fixmapalt(\point)^{2}}^{2}
	\notag\\
	&= \parens*{1 + 2\sqrt 2 \step(\point+1)^{3/2} - \tfrac{\sqrt 2 \step}{2}(\point+1)^{5/2} + o \parens*{(\point+1)^{5/2}}}^{2}
	\notag\\
	&= 1 + \sqrt 2 \step(\point+1)^{3/2} - \tfrac{\sqrt 2 \step}{4}(\point+1)^{5/2} + o \parens*{(\point+1)^{5/2}}\,.
\end{align}
Thus, plugging this expansion and \cref{eq:app-fixmapalt} into \cref{eq:app-fixmap} gives
\begin{align}
\fixmap(\point)
	&= \frac
		{-1 + (\point+1) - \sqrt 2 \step(\point+1)^{3/2} + \tfrac{\sqrt 2 \step}{4}(\point+1)^{5/2} + o \parens*{(\point+1)^{5/2}}}
	{1 + \sqrt 2 \step(\point+1)^{3/2} - \tfrac{\sqrt 2 \step}{4}(\point+1)^{5/2} + o \parens*{(\point+1)^{5/2}}}
	\notag\\
	&= \parens*{
 -1 + (\point+1) - \sqrt 2 \step(\point+1)^{3/2} + \tfrac{\sqrt 2 \step}{4}(\point+1)^{5/2} + o \parens*{(\point+1)^{5/2}}
 }
	\notag\\
	&\qquad\times \parens*{
1 - \sqrt 2 \step(\point+1)^{\half[3]} + \frac{\sqrt 2 \step}{4}(\point+1)^{\half[5]} + o \parens*{(\point+1)^{\half[5]}}
 }
	\notag\\
&=
		-1
		+ (\point+1)
		- 2\sqrt{2}\step (\point+1)^{5/2} + o\parens*{(\point+1)^{5/2}}\,,
\end{align}
which gives our assertion when $\sol = -1$.
\hfill
\endenv
\end{example}

\begin{example}[name=Three-dimensional simplex,continues=ex:simplex-2d]
We conclude our treatment of the simplex by showing that $\state_{2,\run} \sim \state_{2,\run} / \state_{3,\run} = \Omega(1/\run)$ if $\slack_{2} = 0$ but $\slack_{1} > 0$.
To begin with, we have $\vecfield_{2}(\curr) = \state_{2,\run} = o(1)$ so, arguing as in the first part of the example, we readily get
\begin{equation}
\frac{\state_{1,\run+1}}{\state_{2,\run+1}}
	= \frac{\state_{1,\run}}{\state_{2,\run}}
		\exp(-\step\slack_{1} + o(1)),
\end{equation}
so $\state_{1,\run} / \state_{2,\run}$ converges to $0$ at a geometric rate.
Accordingly, the quantity %of interest 
$\state_{3,\run} / \state_{2,\run}$ %can be 
is bounded as
\begin{align}
\frac{\state_{3,\run+1}}{\state_{2,\run+1}}
	&= \frac{\state_{3,\run}}{\state_{2,\run}}
		\exp \parens*{\step \vecfield_{2}(\curr) - \step \vecfield_{3}(\curr)}
	%\notag\\
	= \frac{\state_{3,\run}}{\state_{2,\run}}
		\exp \parens*{\step \state_{2,\run} - \step (\state_{3,\run} - 1)}
	\notag\\
    &= \frac{\state_{3,\run}}{\state_{2,\run}}
		\exp \parens*{2\step \state_{2,\run} + \step \state_{1,\run}}
	%\notag\\
	\leq
		\frac{\state_{3,\run}}{\state_{2,\run}}
		\exp \parens*{2\step \frac{\state_{2,\run}}{\state_{3,\run}} + \step \state_{1,\run}}
\end{align}
%Let us check that
%	\begin{align}
%	\frac{\state_{2,\run}}{\state_{3,\run}} =\Omega\parens*{\frac{1}{\run}}\,.
%	\end{align}
%
% First, using that $(\state_{1,\run})_{\run=\running}$ goes to zero geometrically, we actually show that the ratio $\frac{\state_{1,\run}}{\state_{2,\run}}$ still goes to zero as $\run$ goes to infinity.
%	Indeed, for $\run \geq \start$, 
% \begin{align}
%		\frac{\state_{1,\run+1}}{\state_{2,\run+1}}&=
%		\frac{\state_{1,\run}}{\state_{2,\run}}
%		\exp \parens*{
%			-\step \vecfield_{1}(\curr)
%			+ \step \vecfield_{2}(\curr)
%		} \\
%		&=
%		\frac{\state_{1,\run}}{\state_{2,\run}}
%		\exp \parens*{
%			-\step \slack_1
%			+o(1)
%		}
%		&&\text{ for } \run = \running
%	\end{align}
% since $\vecfield_{1}(\curr) = \state_{1,\run} + \slack_1 = \slack_1 + o(1)$ and $\vecfield_{2}(\curr) = \state_{2,\run} = o(1)$.
% Then, from this one deduces that $\frac{\state_{1,\run}}{\state_{2,\run}}$ goes to zero, and even geometrically fast.
% With this at hand, we can now come back to the quantity of interest $\frac{\state_{2,\run}}{\state_{3,\run}}$. The multiplicative weight update again gives us the recursion
%	\begin{align}
%	\frac{\state_{3,\run+1}}{\state_{2,\run+1}}
%		&=
%		\frac{\state_{3,\run}}{\state_{2,\run}}
%		\exp \parens*{
%			+ \step \vecfield_{2}(\curr)
%			- \step \vecfield_{3}(\curr)
%		} \\
%		&=
%		\frac{\state_{3,\run}}{\state_{2,\run}}
%		\exp \parens*{
%			+ \step \state_{2,\run}
%			- \step (\state_{3,\run} - 1)
%		} \\
%		&=
%		\frac{\state_{3,\run}}{\state_{2,\run}}
%		\exp \parens*{
%			+ 2\step \state_{2,\run}
%			+ \step \state_{1,\run}
%		} \\
%		&\leq
%		\frac{\state_{3,\run}}{\state_{2,\run}}
%		\exp \parens*{
%			+2\step \frac{\state_{2,\run}}{\state_{3,\run}}
%			+ \step \state_{1,\run} 
%		} 
%	\end{align}
%	where we used that $\state_{1,\run} + \state_{2,\run} + \state_{3,\run} = 1$.
Now, since both $\frac{\state_{2,\run}}{\state_{3,\run}}$ and $\state_{1,\run}$ go to zero, 
\begin{align}
\frac{\state_{3,\run+1}}{\state_{2,\run+1}}
	&\leq \frac{\state_{3,\run}}{\state_{2,\run}}
		\parens*{1
			+ 2\step \frac{\state_{3,\run}}{\state_{2,\run}}
			+ \step \state_{1,\run} 
			+ o\parens*{2\frac{\state_{3,\run}}{\state_{2,\run}}
			+ \state_{1,\run}}}
	\notag\\
%&=
%		\frac{\state_{3,\run}}{\state_{2,\run}}
%			+ 2\step
%			+ \frac{\step \state_{1,\run} \state_{3,\run}}{\state_{2,\run}}
%			+ o\parens*{2 + \frac{\step \state_{1,\run} \state_{3,\run}}{\state_{2,\run}}}
%	\notag\\
&=
		\frac{\state_{3,\run}}{\state_{2,\run}}
			+2\step
			+o \parens*{1}\,.
\end{align}
since $\state_{1,\run}/\state_{2,\run}$ vanishes as $\run\to\infty$.
Hence, after telescoping, we conclude that $\frac{\state_{3,\run}}{\state_{2,\run}}
	\leq 2 \step \run + o(\run),$ which in turn shows that $\state_{2,\run} \sim \state_{2,\run} / \state_{3,\run} = \Omega(1/\run)$, as claimed.
\hfill
\endenv
\end{example}

\WAdelete{
\begin{remark}[name=Connection to the Bregman exponent,continues=rem:examples]
We proceed to show here that, in \cref{asm:ker-power} of \cref{asm:ker}, the Bregman exponent at the solution satisfies $\legsol \leq (1 + \kernelexp)/2$.
Indeed, under our stated assumptions, $\hker$ is actually differentiable throughout $(0,\infty)$ so, whenever $0 \leq \pointalt \leq \point$, we have
\begin{equation}
\label{eq:bregexp}
\hker(\pointalt) - \hker(\point)
	= -\int_{\pointalt}^{\point} \hker'(u) \dd u.
\end{equation}
Moreover, thanks to \cref{asm:ker}\ref{asm:ker-power} with $\kernelexp < 1$, the integral,
	$%\begin{equation}
		\int_0^\point \hker'
	$%\end{equation}
	is well-defined.
Then, letting $\pointalt\to0^{+}$ in \eqref{eq:bregexp},
we get
%\begin{equation}
\(
\hker(0) - \hker(\point)
	\leq -\int_{0}^{\point} \hker'(u) \dd u.
\)
%\end{equation}
On the other hand, \cref{asm:ker}\ref{asm:ker-power} implies that both $-\int_0^\point \hker'(u) \dd u$ and $\hker'(\point)(0 -\point)$ are bounded by $\bigoh(\point^{1-\kernelexp})$ so
%\begin{equation}
\(
\hker(0) - \hker(\point) -\hker'(\point)(0 - \point)
	= \bigoh(\point^{1-\kernelexp}).
\)
%\end{equation}
Since $\acts \neq \varnothing$ and $\hker'$ is locally Lipschitz continuous, a similar argument as in the proof of \cref{lem:Leg-proxdom} ultimately yields
\begin{align}
\breg(\sol,\point)
	&= \sum_{\coord \in \actcoords} \bigoh\parens[\big]{\point_\coord^{1-\kernelexp}}
		+ \sum_{\coord \notin \actcoords} \bigoh\parens[\big]{(\sol_\coord-\point_\coord)^{2}}
%	\notag\\
	= \bigoh \parens*{\norm{\sol - \point}^{1 - \kernelexp}}
\end{align}
%where we used the equivalence between norms, \eg between the Euclidean norm and the norm $\norm{\cdot}$.
which shows that the Legendre exponent of $\hreg$ at $\sol$ is at most $(1+\kernelexp)/2$, as claimed.
\endenv
\end{remark}
}
