%**********************************************************************
%***    BREGMAN RATES
%**********************************************************************

%!TEX TS-program =  pdflatex
%!TEX encoding =  UTF-8 Unicode
\synctex=1


%**********************************************************************
%***    1. DOCUMENT CLASS
%**********************************************************************
\documentclass[reqno,letterpaper]{amsart}
\usepackage[margin=1.6in,bottom=1.5in]{geometry}		% for tighter margins (80 CPL)


%**********************************************************************
%***    2.	CORE PACKAGES
%**********************************************************************

%----------------------------------------------------------------------
%% Basic math tools
%----------------------------------------------------------------------
\usepackage{amsmath}		% for AMS macros
\usepackage{amssymb}		% for AMS symbols
\usepackage{amsfonts}		% for AMS fonts
\usepackage{amsthm}		% for theorems
\usepackage[foot]{amsaddr}		% for author footnotes

\usepackage{mathtools}		% for advanced math

\mathtoolsset{%
%above-intertext-sep = -1ex		% for equation spacing
%below-intertext-sep = -2ex		% for equation spacing
%showonlyrefs,	% to show only referenced equations
}


%----------------------------------------------------------------------
%% Encoding (beware of conflicts)
%----------------------------------------------------------------------
\usepackage[utf8]{inputenc}		% for source encoding
\usepackage[T1]{fontenc}		% for font encoding


%----------------------------------------------------------------------
%% Math alphabets
%----------------------------------------------------------------------
\usepackage[%		% for math font selection
cal=cm,
%scr=euler,
%frak=euler
]
{mathalfa}


%**********************************************************************
%***    3. FONTS
%**********************************************************************

%----------------------------------------------------------------------
%% Blackboard bold
%----------------------------------------------------------------------
\usepackage{dsfont}		% for blackboard bold font
%\let\mathbb=\mathds


%----------------------------------------------------------------------
%% Sans serif
%----------------------------------------------------------------------
%\usepackage{cabin}		% sans serif
%\usepackage{sourcesanspro}


%----------------------------------------------------------------------
%% Typewriter
%----------------------------------------------------------------------
%\usepackage[ttdefault,scale=.95]{AnonymousPro}
%\usepackage{libertinus}
%\usepackage{lmodern}
%\renewcommand*\ttdefault{lmvtt}
%\renewcommand*\ttdefault{txtt}
%\usepackage[scale=1.05]{inconsolata}


%----------------------------------------------------------------------
%% Libertine
%----------------------------------------------------------------------
\usepackage[sf,mono=false]{libertine}
%\usepackage[libertine,libaltvw,cmintegrals,varbb]{newtxmath}



%**********************************************************************
%***    4. ANCILLARY PACKAGES
%**********************************************************************

%----------------------------------------------------------------------
%% Acronyms
%----------------------------------------------------------------------
\usepackage{acronym}		% for acronyms
\renewcommand*{\aclabelfont}[1]{\acsfont{#1}}		% for acronym label font
\newcommand{\acli}[1]{\emph{\acl{#1}}}		% for italicized acro
\newcommand{\aclip}[1]{\emph{\aclp{#1}}}		% for italicized acro (plural)
\newcommand{\acdef}[1]{\emph{\acl{#1}} \textup{(\acs{#1})}\acused{#1}}		% for acro def
\newcommand{\acdefp}[1]{\emph{\aclp{#1}} \textup{(\acsp{#1})}\acused{#1}}	% for acro def (plural)

\newcommand{\Acli}[1]{\emph{\Acl{#1}}}		% for italicized acro
\newcommand{\Aclip}[1]{\emph{\Aclp{#1}}}		% for italicized acro (plural)
\newcommand{\Acdef}[1]{\emph{\Acl{#1}} \textup{(\acs{#1})}\acused{#1}}		% for acro def
\newcommand{\Acdefp}[1]{\emph{\Aclp{#1}} \textup{(\acsp{#1})}\acused{#1}}	% for acro def (plural)


%----------------------------------------------------------------------
%% Captions
%----------------------------------------------------------------------
\usepackage[labelfont={bf,small},labelsep=colon,font=small]{caption}	% for caption control
\usepackage{subcaption}		% for subfigures
 

%----------------------------------------------------------------------
%% Colors
%----------------------------------------------------------------------
\usepackage[dvipsnames,svgnames]{xcolor}		% for color

\colorlet{MidnightBlue}{black!50!blue}
\colorlet{RoyalBlue}{black!20!blue}
\colorlet{OliveGreen}{black!60!green}
%\colorlet{Maroon}{black!30!red}
\colorlet{MyRed}{FireBrick!50!Crimson}
\colorlet{MyBlue}{DodgerBlue!60!black}
\colorlet{MyGreen}{DarkGreen!85!black}
\colorlet{MyLightBlue}{DodgerBlue!15}
\colorlet{MyLightGreen}{MyGreen!20}
\colorlet{LinkColor}{MediumBlue}
\colorlet{PrimalColor}{MyBlue}
\colorlet{PrimalFill}{MyLightBlue}
\colorlet{DualColor}{MyRed}


%----------------------------------------------------------------------
%% Document layout
%----------------------------------------------------------------------
\newcommand{\afterhead}{.\;}		% for changing headings
\newcommand{\ackperiod}{}		% for AMS bug in acknowledgments
\newcommand{\para}[1]{\medskip\paragraph{\textbf{#1\afterhead}}}
\newcommand{\itempar}[1]{\paragraph{\raisebox{1pt}{\footnotesize$\blacktriangleright$}\;\textbf{#1\afterhead}}}


%----------------------------------------------------------------------
%% Extra symbols
%----------------------------------------------------------------------
\usepackage{cancel}		% for cancelling terms
\usepackage{latexsym}		% for symbols

\usepackage{pifont}		% for dingbats
\newcommand{\cmark}{\ding{51}}		% for checkmark
\newcommand{\xmark}{\ding{55}}		% for x-cross
\newcommand{\attn}{{\color{MyRed}\ding{70}}}		% for attention


%----------------------------------------------------------------------
%% Figures and graphics
%----------------------------------------------------------------------
\usepackage{tikz}		% for figures
\usetikzlibrary{calc,patterns}		% for basic tikz figures
\usepackage{pgfplots}


%----------------------------------------------------------------------
%% Lists and tables
%----------------------------------------------------------------------
\usepackage{array}		% for flexible tables
\usepackage{booktabs}		% for better tables
\usepackage[inline,shortlabels]{enumitem}		% for lists
\setenumerate{itemsep=\smallskipamount,topsep=\smallskipamount,left=0pt}
\setitemize{itemsep=\smallskipamount,topsep=\smallskipamount,left=\parindent}
%\usepackage{multirow}


%----------------------------------------------------------------------
%% Microtypography
%----------------------------------------------------------------------
\usepackage[kerning=true]{microtype}		% for microtypography


%----------------------------------------------------------------------
%% Various
%----------------------------------------------------------------------
\usepackage{tabto}		% for spacing
\usepackage{xspace}		% for flexible spaces



%**********************************************************************
%***    5. REFERENCES
%**********************************************************************

%----------------------------------------------------------------------
%% Bibliographic citations
%----------------------------------------------------------------------
\usepackage[authoryear,sort&compress]{natbib}		% for citations
\def\bibfont{\footnotesize}
\def\bibsep{\smallskipamount}
\def\bibhang{24pt}
%\def\BIBand{and}
%\def\newblock{\ }
\bibpunct[, ]{[}{]}{,}{n}{,}{,}

%\newcommand{\citef}[2][]{\citeauthor{#2} \cite[#1]{#2}}
%\let\tempcite\cite
%\newcommand{\citefull}[2][]{\citeauthor{#2} \tempcite[#1]{#2}}
%\renewcommand*{\cite}{\citefull}		% change default behavior of cite


%----------------------------------------------------------------------
%% Hyperlinks
%----------------------------------------------------------------------
\usepackage{hyperref}
\hypersetup{
colorlinks=true,
linktocpage=true,
%pdfstartpage=1,
pdfstartview=FitH,
breaklinks=true,
pdfpagemode=UseNone,
pageanchor=true,
pdfpagemode=UseOutlines,
plainpages=false,
bookmarksnumbered,
bookmarksopen=false,
bookmarksopenlevel=1,
hypertexnames=true,
pdfhighlight=/O,
%hyperfootnotes=true,
%nesting=true,
%frenchlinks,
urlcolor=LinkColor,linkcolor=LinkColor,citecolor=LinkColor,	% for on-screen
%urlcolor=black, linkcolor=black, citecolor=black, %pagecolor=black,	% for printing
%pagecolor=RoyalBlue,
pdftitle={},
pdfauthor={},
pdfsubject={},
pdfkeywords={},
pdfcreator={pdfLaTeX},
pdfproducer={LaTeX with hyperref}
}

\newcommand{\EMAIL}[1]{\email{\href{mailto:#1}{#1}}}
\newcommand{\URLADDR}[1]{\urladdr{\href{#1}{#1}}}


%----------------------------------------------------------------------
%% Cleverefs (must go after hyperref)
%----------------------------------------------------------------------
\numberwithin{equation}{section}		% for cleveref (when using cleveref and hyperref)
\usepackage[sort&compress,capitalize,nameinlink]{cleveref}		% for cleveref formatting
%\crefname{algorithm}{Alg.}{Algs.}
\crefname{algo}{Algorithm}{Algorithms}
\crefname{assumption}{Assumption}{Assumptions}
\crefname{case}{Case}{Cases}
\newcommand{\crefrangeconjunction}{\textendash}		% for cleveref conjunctions
\newcommand{\creflastconjunction}{, and~}
%\creflabelformat{equation}{#2\textup{(#1)}#3}
%\crefrangeformat{equation}{\upshape(#3#1#4)\textendash(#5#2#6)}
%\creflabelformat{part}{(#2#1#3)}


%----------------------------------------------------------------------
%% Only referenced equations
%----------------------------------------------------------------------
%\usepackage{autonum}		% for citing referenced only / must go last



%**********************************************************************
%***    6. ENVIRONMENTS
%**********************************************************************

%----------------------------------------------------------------------
%% Algorithms
%----------------------------------------------------------------------
\usepackage{algorithm}		% for algorithm environments
\usepackage{algpseudocode}		% for algorithm macros
\renewcommand{\algorithmiccomment}[1]{\hfill\textsf{\#} #1}		% for algorithm comments


%----------------------------------------------------------------------
%% Theorem tools
%----------------------------------------------------------------------
\usepackage{thmtools}		% for theorem tools
\usepackage{thm-restate}		% for restating theorems


%----------------------------------------------------------------------
%% Theorem-like
%----------------------------------------------------------------------
\theoremstyle{plain}
\newtheorem{theorem}{Theorem}		% for theorems
\newtheorem{corollary}{Corollary}		% for corollaries
\newtheorem{lemma}{Lemma}		% for lemmas
\newtheorem{proposition}{Proposition}		% for propositions

\newtheorem{conjecture}{Conjecture}		% for conjectures
\newtheorem{claim}{Claim}		% for claims

\newtheorem*{corollary*}{Corollary}		% for corollaries (unnumbered)


%----------------------------------------------------------------------
%% Definition-like
%----------------------------------------------------------------------
\theoremstyle{definition}
\newtheorem{definition}{Definition}		% for definitions
\newtheorem{notation}{Notation}		% for notations
\newtheorem{assumption}{Assumption}		% for assumptions
\newtheorem{example}{Example}		% for examples

\newtheorem*{definition*}{Definition}		% for definitions (unnumbered)
\newtheorem*{assumption*}{Assumptions}		% for assumptions (unnumbered)
\newtheorem*{example*}{Example}		% for examples (unnumbered)

%\makeatletter		% for custom tags
%\newcommand{\asmtag}[1]{% \asmtag{<tag>}
%  \let\oldtheassumption\theassumption% Store \theassumption
%  \renewcommand{\theassumption}{#1}% Redefine it to a fixed value
%  \g@addto@macro\endassumption{% At \end{assumption}, ...
%    \addtocounter{assumption}{-1}% ...restore assumption counter value and...
%    \global\let\theassumption\oldtheassumption}% ...restore \theassumption
%  }
%\makeatother


%----------------------------------------------------------------------
%% Remark-like
%----------------------------------------------------------------------
\theoremstyle{remark}
\newtheorem{remark}{Remark}		% for remarks

\newtheorem*{remark*}{Remark}		% for remarks (unnumbered)


%----------------------------------------------------------------------
%% Environment end
%----------------------------------------------------------------------
\def\endenv{{\small$\blacklozenge$}}


%----------------------------------------------------------------------
%% Proofs
%----------------------------------------------------------------------
%\newenvironment{Proof}[1][Proof]{\begin{proof}[#1]}{\end{proof}}		% for redefining proofs
\renewcommand{\qedsymbol}{{\small$\blacksquare$}}		% for qed symbol
%\smartqed		% flush right qed marks, e.g. at end of proof

\newcounter{proofpart}
\newenvironment{proofpart}[1]
{\vspace{3pt}
\refstepcounter{proofpart}%
\par\textit{Part~\arabic{proofpart}:~#1}.\,}
{\smallskip}


%----------------------------------------------------------------------
%% Numbering
%----------------------------------------------------------------------
%\numberwithin{equation}{section}		% for equation numbering (beware of cleveref)
%\numberwithin{theorem}{section}		% for theorem numbering
%\numberwithin{definition}{section}		% for definition numbering
%\numberwithin{remark}{section}		% for remark numbering
\numberwithin{example}{section}		% for example numbering



%**********************************************************************
%***    7. EDITING
%**********************************************************************
\usepackage[showdeletions,suppress]{color-edits}		% for editing macros / use [suppress] for final
\usepackage[normalem]{ulem}		% for strikeout text

\newcommand{\needref}{{\color{red}\upshape\textbf{[??]}}\xspace}	% for missing refs

%\newcommand{\debug}[1]{{\color{MyRed}#1}}		% for macro coloring
\newcommand{\debug}[1]{#1}		% for removing macro coloring

\newcommand{\revise}[1]{{\color{MyBlue}#1}}		% for revision markup
\newcommand{\remove}[1]{{\color{gray}#1}}		% for removal markup
\newcommand{\ok}[1]{{\color{MyGreen}#1}}		% for revision markup
\newcommand{\lookout}[1]{{\bfseries\color{MyRed}[#1]}}		% for attention

\def\beginrev{\color{MyBlue}}		% for revision markup
\def\beginrmv{\color{gray}}		% for removal markup
\def\beginok{\color{MyGreen}}		% for ok markup
\def\endedit{\color{black}}		% for ending markup



%**********************************************************************
%***    MACROS: GENERAL
%**********************************************************************
\newcommand{\newmacro}[2]{\newcommand{#1}{\debug{#2}}}		% for shorthand definitions
\newcommand{\newop}[2]{\DeclareMathOperator{#1}{\debug{#2}}}		% for shorthand definitions


%----------------------------------------------------------------------
%% Delimiters
%----------------------------------------------------------------------
\DeclarePairedDelimiter{\braces}{\{}{\}}		% for braces
\DeclarePairedDelimiter{\bracks}{[}{]}		% for brackets
\DeclarePairedDelimiter{\parens}{(}{)}		% for parentheses

\DeclarePairedDelimiter{\abs}{\lvert}{\rvert}		% for absolute value
\DeclarePairedDelimiter{\ceil}{\lceil}{\rceil}		% for ceiling
\DeclarePairedDelimiter{\floor}{\lfloor}{\rfloor}		% for floor
\DeclarePairedDelimiter{\clip}{[}{]}		% for clipping
\DeclarePairedDelimiter{\negpart}{[}{]_{-}}		% for negative part
\DeclarePairedDelimiter{\pospart}{[}{]_{+}}		% for positive part

\DeclarePairedDelimiterX{\setdef}[2]{\{}{\}}{#1:#2}		% for set builder notation
\DeclarePairedDelimiterXPP{\exclude}[1]{\mathopen{}\setminus}{\{}{\}}{}{#1}


%----------------------------------------------------------------------
%% Number fields
%----------------------------------------------------------------------
\newcommand{\F}{\mathbb{F}}		% generic field
\newcommand{\N}{\mathbb{N}}		% for naturals
\newcommand{\Z}{\mathbb{Z}}		% for integers
\newcommand{\Q}{\mathbb{Q}}		% for rationals
\newcommand{\R}{\mathbb{R}}		% for reals
\newcommand{\C}{\mathbb{C}}		% for complex numbers (may clash)


%----------------------------------------------------------------------
%% Operators
%----------------------------------------------------------------------
\DeclareMathOperator*{\argmax}{arg\,max}		% for argmax
\DeclareMathOperator*{\argmin}{arg\,min}		% for argmin
\DeclareMathOperator*{\intersect}{\bigcap}		% for intersections
\DeclareMathOperator*{\union}{\bigcup}		% for unions

\DeclareMathOperator{\aff}{aff}		% for affine hull
\DeclareMathOperator{\lspan}{span}		% for linear span
\DeclareMathOperator{\bd}{bd}		% for boundary
\DeclareMathOperator{\bigoh}{\mathcal{O}}		% for Landau O
\DeclarePairedDelimiterXPP{\bigof}[1]{\mathcal{O}}{(}{)}{}{#1}		% for Landau O
\DeclareMathOperator{\card}{card}		% for cardinality
\DeclareMathOperator{\cl}{cl}		% for closure
\DeclareMathOperator{\conv}{conv}		% for convex hull (but see also \simplex)
\DeclareMathOperator{\crit}{crit}		% for critical set
\DeclareMathOperator{\diag}{diag}		% for diagonal matrices
\DeclareMathOperator{\diam}{diam}		% for diameter
\DeclareMathOperator{\dist}{dist}		% for distance
\DeclareMathOperator{\dom}{dom}		% for domain
\DeclareMathOperator{\eig}{eig}		% for eigenvalues
\DeclareMathOperator{\ess}{ess}		% for essential
\DeclareMathOperator{\grad}{\nabla}		% for gradient
\DeclareMathOperator{\Hess}{Hess}		% for Hessian
\DeclareMathOperator{\ind}{ind}		% for index
\DeclareMathOperator{\im}{im}		% for image
\DeclareMathOperator{\intr}{int}		% for interior
\DeclareMathOperator{\Jac}{Jac}		% for Jacobian
\DeclareMathOperator{\one}{\mathds{1}}		% for indicator
\DeclareMathOperator{\proj}{pr}		% for projection
\DeclareMathOperator{\prox}{prox}		% for prox
\DeclareMathOperator{\rank}{rank}		% for rank
\DeclareMathOperator{\relint}{ri}		% for relative interior
\DeclareMathOperator{\sign}{sgn}		% for sign
\DeclareMathOperator{\supp}{supp}		% for support
\DeclareMathOperator{\Sym}{Sym}		% for symmetric
\DeclareMathOperator{\tr}{tr}		% for trace
\DeclareMathOperator{\unif}{unif}		% for uniform distribution
\DeclareMathOperator{\vol}{vol}		% for volume


%----------------------------------------------------------------------
%% Text and formatting
%----------------------------------------------------------------------
\newcommand{\cf}{cf.\xspace}		% for consistency
\newcommand{\eg}{e.g.,\xspace}		% for consistency
\newcommand{\ie}{i.e.,\xspace}		% for consistency
\newcommand{\vs}{vs.\xspace}		% for consistency
\newcommand{\viz}{viz.\xspace}		% for consistency

\newcommand{\textbrac}[1]{\textup[#1\textup]}		% for upshape brackets
\newcommand{\textpar}[1]{\textup(#1\textup)}		% for upshape parentheses

\newcommand{\dis}{\displaystyle}		% for forcing display style
\newcommand{\txs}{\textstyle}		% for forcing inline style


%----------------------------------------------------------------------
%% Various
%----------------------------------------------------------------------
\newcommand{\alt}[1]{#1'}		% for variant version
\newcommand{\altalt}[1]{#1''}		% for second variant
\newcommand{\avg}[1]{\bar#1}		% for averaging (X by default)

\newmacro{\dd}{\:d}		% for integration
%\newcommand{\ddt}[1]{\frac{d#1}{dt}}		% for Leibniz
\newcommand{\ddt}{\frac{d}{dt}}		% for Leibniz
\newcommand{\del}{\partial}		% for derivatives
\newcommand{\eps}{\varepsilon}		% for better epsilon
\newcommand{\pd}{\partial}		% for derivatives
\newcommand{\wilde}{\widetilde}		% for wide tildes

\newcommand{\insum}{\sum\nolimits}		% for compact sums
\newcommand{\inprod}{\prod\nolimits}		% for compact products

\newmacro{\const}{c}		% for generic constant
\newmacro{\Const}{\rho}		% for generic constant
%\newmacro{\coef}{\lambda}		% for generic coefficient -> see below
\newmacro{\coefalt}{\mu}		% for generic coefficient
\usepackage{xparse}
\NewDocumentCommand{\coef}{O{\lambda}}{\debug{#1}}
\newmacro{\param}{\theta}		% for parameter
\newmacro{\params}{\Theta}		% for set of parameters

\newmacro{\pexp}{p}		% for first exponent
\newmacro{\qexp}{q}		% for second exponent
\newmacro{\rexp}{r}		% for third exponent

\newmacro{\radius}{r}



%**********************************************************************
%***    MACROS: SPECIFIC
%**********************************************************************

%----------------------------------------------------------------------
%% Algorithms (indexing)
%----------------------------------------------------------------------
\newmacro{\beforestart}{0}		% for before start index
\newmacro{\start}{1}		% for start index
\newmacro{\afterstart}{2}		% for second index
\newmacro{\running}{\start,\afterstart,\dotsc}		% for running
\newmacro{\halfrunning}{1,3/2,2\dotsc}		% for running

\newmacro{\run}{t}		% for main sequence index
\newmacro{\runalt}{s}		% for variant index
\newmacro{\runaltalt}{\tau}		% for second variant
\newmacro{\nRuns}{T}		% for total number of runs
\newmacro{\runs}{\mathcal{\nRuns}}		% for set of indices


%----------------------------------------------------------------------
%% Algorithms (states and recursions)
%----------------------------------------------------------------------
\newmacro{\state}{x}		% for main iterate
\newmacro{\statealt}{y}		% for variant state
\newmacro{\statealtalt}{z}		% for second variant

\newcommand{\new}[1][\point]{#1^{+}}		% for new iterate (x by default)

\newcommand{\beforeinit}[1][\state]{\debug{#1}_{\beforestart}}		% for zeroth iterate (X by default)
\newcommand{\init}[1][\state]{\debug{#1}_{\start}}		% for initial iterate (X by default)
\newcommand{\afterinit}[1][\state]{\debug{#1}_{\afterstart}}		% for second iterate (X by default)

\newcommand{\beforeiter}[1][\state]{\debug{#1}_{\runalt-1}}		% for before running iterate (X by default)
\newcommand{\iter}[1][\state]{\debug{#1}_{\runalt}}		% for running iterate (X by default)
\newcommand{\iterlead}[1][\state]{\debug{#1}_{\runalt+1/2}}		% for running iterate (X by default)
\newcommand{\afteriter}[1][\state]{\debug{#1}_{\runalt+1}}		% for after running iterate (X by default)

\newcommand{\beforeprev}[1][\state]{\debug{#1}_{\run-2}}		% for previous iterate (X by default)
\newcommand{\prev}[1][\state]{\debug{#1}_{\run-1}}		% for previous iterate (X by default)
\newcommand{\curr}[1][\state]{\debug{#1}_{\run}}		% for current iterate (X by default)
\renewcommand{\next}[1][\state]{\debug{#1}_{\run+1}}		% for next iterate (X by default)

\newcommand{\beforelead}[1][\state]{\debug{#1}_{\run-1/2}}		% for prev lead iterate (X by default)
\newcommand{\lead}[1][\state]{\debug{#1}_{\run+1/2}}		% for lead iterate (X by default)
\newcommand{\afterlead}[1][\state]{\debug{#1}_{\run+3/2}}		% for next lead iterate (X by default)

\newcommand{\beforelast}[1][\state]{\debug{#1}_{\nRuns-1}}		% for before last iterate (X by default)
\newcommand{\last}[1][\state]{\debug{#1}_{\nRuns}}		% for last iterate (X by default)
\newcommand{\afterlast}[1][\state]{\debug{#1}_{\nRuns+1}}		% for after last iterate (X by default)


%----------------------------------------------------------------------
%% Dynamical systems
%----------------------------------------------------------------------
\newmacro{\tstart}{0}		% for time start
\renewcommand{\time}{\debug{t}}		% for continuous time
\newmacro{\timealt}{s}		% for dummy continuous time
\newmacro{\horizon}{T}		% for horizon

\newmacro{\traj}{x}		% for trajectory
\newmacro{\trajalt}{y}		% for variant trajectory
\newmacro{\trajaltalt}{z}		% for second variant

\newmacro{\flowmap}{\Theta}		% for (semi)flow map
\DeclarePairedDelimiterXPP{\flowof}[2]{\flowmap_{#1}}{(}{)}{}{#2}		% for flow


%----------------------------------------------------------------------
%% Game theory
%----------------------------------------------------------------------
\newop{\Nash}{NE}		% for Nash equilibria
\newop{\CE}{CE}		% for correlated equilibria
\newop{\CCE}{CCE}		% for Hannan set
\newop{\NI}{NI}		% for Nikaido-Isoda function

\newop{\brep}{br}		% for best responses
\newop{\reg}{Reg}		% for regret
\newop{\preg}{\overline{Reg}}		% for pseudo-regret
\newop{\val}{val}		% for value function

\newcommand{\strat}{\point}		% for mixed strategy
\newcommand{\stratalt}{\pointalt}		% for variant strategy
\newcommand{\strats}{\points}		% for set of mixed strategies
\newcommand{\intstrats}{\intpoints}		% for set of interior strategies
\newcommand{\eq}{\sol}		% for Nash equilibrium
\newcommand{\eqs}{\sols}		% for set of Nash equilibria

\newmacro{\play}{i}		% for player index
\newmacro{\playalt}{j}		% for variant player index
\newmacro{\playaltlalt}{k}		% for second variant
\newmacro{\nPlayers}{N}		% for number of players
\newmacro{\players}{\mathcal{\nPlayers}}		% for set of players

\newmacro{\pure}{\alpha}		% for pure strategy
\newmacro{\purealt}{\beta}		% for variant pure strategy
\newmacro{\purealtalt}{\gamma}		% for second variant
\newmacro{\nPures}{A}		% for number of pure strategies
\newmacro{\pures}{\mathcal{\nPures}}		% for set of pure strategies

\newmacro{\loss}{\ell}		% for loss function
\newmacro{\pay}{u}		% for payoff function
\newmacro{\payv}{v}		% for payoff vector
\newmacro{\pot}{f}		% for potential function

\newmacro{\game}{\mathcal{G}}		% for game
\newmacro{\gamefull}{\game(\players,\points,\pay)}		% for full game

\newmacro{\fingame}{\Gamma}		% for finite game
\newmacro{\fingamefull}{\Gamma(\players,\pures,\pay)}		% for full finite game


%----------------------------------------------------------------------
%% Geometry
%----------------------------------------------------------------------
\newmacro{\gmat}{g}		% for metric tensor
\newmacro{\gdist}{\dist_{\gmat}}
\newmacro{\mfld}{M}		% for manifold
\newmacro{\form}{\omega}		% for generic form

\newmacro{\tvec}{z}		% for tangent vector
\newmacro{\uvec}{u}		% for unit vector

\newmacro{\ball}{\basin}		% for ball
\newmacro{\sphere}{\mathbb{S}}		% for sphere


%----------------------------------------------------------------------
%% Graph theory
%----------------------------------------------------------------------
\newmacro{\graph}{\mathcal{G}}
\newmacro{\vertices}{\mathcal{V}}
\newmacro{\edges}{\mathcal{E}}


%----------------------------------------------------------------------
%% Linear algebra (matrices)
%----------------------------------------------------------------------
\newmacro{\mat}{A}		% for generic matrix
\newmacro{\matalt}{c}		% for generic matrix
\newmacro{\hmat}{H}		% for Hessian matrix

\newop{\row}{row}		% for row space
\newop{\col}{col}		% for row space

\newmacro{\ones}{\mathbf{1}}		% for matrix of ones
\newmacro{\eye}{I}		% for identity matrix
\newmacro{\zer}{\mathbf{0}}		% for zero matrix

\newcommand{\mg}{\succ}		% for positive-definite
\newcommand{\mgeq}{\succcurlyeq}		% for positive-semidefinite
\newcommand{\ml}{\prec}		% for negative-definite
\newcommand{\mleq}{\preccurlyeq}		% for negative-semidefinite


%----------------------------------------------------------------------
%% Linear algebra (norms)
%----------------------------------------------------------------------
\DeclarePairedDelimiter{\norm}{\lVert}{\rVert}		% for norm
\DeclarePairedDelimiterXPP{\dnorm}[1]{}{\lVert}{\rVert}{_{\ast}}{#1}		% for dual norm
%\newcommand{\dnorm}[1]{\norm{#1}_{\ast}}		% for dual norm

\DeclarePairedDelimiterXPP{\onenorm}[1]{}{\lVert}{\rVert}{_{1}}{#1}		% for dual norm
\DeclarePairedDelimiterXPP{\twonorm}[1]{}{\lVert}{\rVert}{_{2}}{#1}		% for dual norm
\DeclarePairedDelimiterXPP{\supnorm}[1]{}{\lVert}{\rVert}{_{\infty}}{#1}		% for dual norm


%----------------------------------------------------------------------
%% Linear algebra (pairings)
%----------------------------------------------------------------------
\DeclarePairedDelimiter{\bra}{\langle}{\rvert}		% for bras
\DeclarePairedDelimiter{\ket}{\lvert}{\rangle}		% for kets
\DeclarePairedDelimiterX{\braket}[2]{\langle}{\rangle}{#1,#2}		% for brakets
%\DeclarePairedDelimiterX{\braket}[2]{\langle}{\rangle}{#1\mathopen{}\delimsize\vert\mathopen{}#2}


%----------------------------------------------------------------------
%% Linear algebra (vector spaces)
%----------------------------------------------------------------------
\newmacro{\vecspace}{\mathcal{V}}		% for generic vector space
\newmacro{\subspace}{\mathcal{W}}		% for vector subspace

\newmacro{\coord}{i}		% for coordinate index
\newmacro{\coordalt}{j}		% for variant coordinate
\newmacro{\coordaltalt}{k}		% for second variant
\newmacro{\nCoords}{n}		% for number of coordinates
\newmacro{\dims}{\nCoords}		% for dimension
\newmacro{\vdim}{\nCoords}		% for dimension (legacy alias)

\newmacro{\pvec}{z}		% for primal vector
\newmacro{\pvecalt}{r}		% for primal vector
\newmacro{\bvec}{e}		% for basis vector
\newmacro{\bvecs}{\mathcal{E}}		% for basis vectors
\newmacro{\cvec}{b}     % for column vector
\newmacro{\cvecalt}{d}     % for column vector

%----------------------------------------------------------------------
%% Linear algebra (vector space duality)
%----------------------------------------------------------------------
\newmacro{\pspace}{\mathcal{V}}		% for primal space
\newmacro{\dspace}{\pspace^{\ast}}		% for dual space

\newmacro{\dvec}{\dpoint}		% for dual vector
\newmacro{\dbvec}{\eps}		% for dual basis vectors

\newmacro{\dpoint}{y}		% for generic dual point
\newmacro{\dpointalt}{\alt\dpoint}		% for variant dual point
\newmacro{\dpointaltalt}{\altalt\dpoint}		% for second variant
\newmacro{\dpoints}{\mathcal{Y}}		% for set of dual points

\newmacro{\dstate}{Y}		% for dual state
\newmacro{\dbase}{v}		% for dual base


%----------------------------------------------------------------------
%% Logic and set theory
%----------------------------------------------------------------------
\newcommand{\defeq}{\coloneqq}		% for direct definition
\newcommand{\eqdef}{\eqqcolon}		% for reverse definition

\newcommand{\from}{\colon}		% for function definition
\newcommand{\too}{\rightrightarrows}		% for correspondences
\newcommand{\injects}{\hookrightarrow}		% for injections
\newcommand{\surjects}{\twoheadrightarrow}		% for surjections
\newcommand{\comp}[1]{#1^{\mathtt{c}}}		% for complement


%----------------------------------------------------------------------
%% Optimization (basics)
%----------------------------------------------------------------------
\newop{\Opt}{Opt}		% for value of problem
\newop{\Sol}{Sol}		% for solution of problem
\newop{\gap}{Gap}		% for gap function
\newop{\orcl}{Or}		% for oracle

\newmacro{\tfun}{f}		% for test function
\newmacro{\obj}{f}		% for objective function
\newmacro{\objalt}{g}		% for variant objective (smooth etc.)
\newmacro{\sobj}{F}		% for stochastic objective

\newmacro{\gvec}{g}		% for gradient vector
\newmacro{\oper}{A}		% for operator
\newmacro{\vecfield}{g}		% for vector field (selection etc.)
%\newmacro{\payfield}{v}		% for payoff field (selection etc.)

\newcommand{\sol}[1][\point]{#1^{\ast}}		% for solution point (x by default)
\newcommand{\sols}{\sol[\points]}		% for set of solutions
\newmacro{\solvec}{\vecfield(\sol)}		% for vector at a solution
\newmacro{\solpay}{\eq[\payv]}		% for vector at a solution

\newcommand{\test}[1][\point]{\hat#1}		% for test point (x by default)

\newmacro{\signal}{g}		% for signal
\newmacro{\step}{\gamma}		% for step-size
\newmacro{\learn}{\eta}		% for learning rate

\newmacro{\vbound}{G}		% for vector bound
\newmacro{\lips}{L}		% for Lipschitz modulus
\newmacro{\strong}{\mu}		% for strong convexity modulus
\newmacro{\smooth}{\beta}		% for strong smoothness modulus


%----------------------------------------------------------------------
%% Optimization (convex analysis)
%----------------------------------------------------------------------
\newop{\cone}{cone}
%\newop{\tspace}{T_{\points}}		% for tangent space
%\newop{\tcone}{TC_{\points}}		% for tangent cone
%\newop{\dcone}{\tcone^{\ast}}		% for dual cone
%\newop{\ncone}{NC_{\points}}		% for normal cone
%\newop{\pcone}{PC_{\points}}		% for polar cone
\newop{\tspace}{T}		% for tangent space
\newop{\tcone}{TC}		% for tangent cone
\newop{\dcone}{DC}		% for dual cone
\newop{\ncone}{NC}		% for normal cone
\newop{\pcone}{PC}		% for polar cone
\newop{\hull}{\Delta}		% for simplices

\newmacro{\cvx}{\mathcal{C}}		% for generic convex set
\newmacro{\subd}{\partial}		% for subdifferential
\newcommand{\subsel}{\nabla}		% for subdifferential


%----------------------------------------------------------------------
%% Optimization (min-max)
%----------------------------------------------------------------------
\newmacro{\minmax}{\mathcal{L}}		% for minmax objective

\newmacro{\minvar}{\point_{1}}		% for minimization variable
\newmacro{\minvaralt}{\alt\minvar}		% for variant minvar
\newmacro{\minvars}{\points_{1}}		% for set of minvars
\newmacro{\minsol}{\sol[\minvar]}		% for min solution

\newmacro{\maxvar}{\point_{2}}		% for maximization variable
\newmacro{\maxvaaltr}{\alt\maxvar}		% for variant maxvar
\newmacro{\maxvars}{\points_{2}}		% for set of maxvars
\newmacro{\maxsol}{\sol[\maxvar]}		% for max solution


%----------------------------------------------------------------------
%% Optimization (mirror descent)
%----------------------------------------------------------------------
\newop{\Eucl}{\Pi}		% for Euclidean projection
\newop{\logit}{\Lambda}		% for logit map
\newop{\dkl}{KL}		% for Kullback Leibler

\newmacro{\hreg}{h}		% for regularizer
\newmacro{\hconj}{\hreg^{\ast}}		% for regularizer
\newmacro{\breg}{D}		% for Bregman divergence
\newmacro{\mprox}{P}		% for Bregman prox-mapping
\newmacro{\mirror}{Q}		% for mirror map
\newmacro{\fench}{F}		% for Fenchel coupling
\newmacro{\hstr}{K}		% for strong convexity constant
\newmacro{\depth}{H}		% for regularizer depth
\newmacro{\proxdom}{\points_{\hreg}}		% for prox-domain
\newmacro{\zone}{\mathbb{D}}		% for Bregman zone
\newmacro{\hker}{\theta} % For the kernel of the Bregman divergence

\DeclarePairedDelimiterXPP{\proxof}[2]{\mprox_{#1}}{(}{)}{}{#2}		% for Bregman prox step
%\DeclarePairedDelimiterXPP{\proxof}[2]{\pmap}{(}{)}{}{#1,#2}		% for Bregman prox step

\newcommand{\lazy}[1]{#1^{\textup{lazy}}}		% for lazy iterate
\newcommand{\eager}[1]{#1^{\textup{eager}}}		% for eager iterate

%----------------------------------------------------------------------
%% Points and sets
%----------------------------------------------------------------------
\newmacro{\point}{x}		% for generic point
\newmacro{\pointalt}{\alt\point}		% for variant point
\newmacro{\pointaltalt}{\altalt\point}		% for second variant
\newmacro{\points}{\mathcal{X}}		% for set of points
\newmacro{\intpoints}{\relint\points}		%for point set interior

\newmacro{\base}{p}		% for reference point
\newmacro{\basealt}{q}		% for variant reference point
\newmacro{\basealtalt}{u}		% for second variant

\newmacro{\open}{\mathcal{U}}		% for open sets
\newmacro{\closed}{\mathcal{C}}		% for closed sets
\newmacro{\cpt}{\mathcal{K}}		% for compact sets
\newmacro{\nhd}{\mathcal{U}}		% for neighborhoods


%----------------------------------------------------------------------
%% Probability
%----------------------------------------------------------------------
\newop{\ex}{\mathbb{E}}		% for expectations
\newop{\prob}{\mathbb{P}}		% for probability
\newop{\Var}{Var}		% for variance
\newop{\simplex}{\hull}		% for simplices

\providecommand\given{}		% empty command for conditionals

\DeclarePairedDelimiterXPP{\exof}[1]{\ex}{[}{]}{}{%		% for conditional expectations
\renewcommand\given{\nonscript\,\delimsize\vert\nonscript\,\mathopen{}} #1}

\DeclarePairedDelimiterXPP{\probof}[1]{\prob}{(}{)}{}{%		% for conditional probabilities
\renewcommand\given{\nonscript\:\delimsize\vert\nonscript\:\mathopen{}} #1}

\DeclarePairedDelimiterXPP{\oneof}[1]{\one}{\{}{\}}{}{%		% for conditional expectations
\renewcommand\given{\nonscript\,\delimsize\vert\nonscript\,\mathopen{}} #1}

\newmacro{\sample}{\omega}		% for sample
\newmacro{\samples}{\Omega}		% for set of samples

\newmacro{\filter}{\mathcal{F}}		% for filtration
\newmacro{\probspace}{(\samples,\filter,\prob)}		% for probability space

\newcommand{\as}{\debug{\textpar{a.s.}}\xspace}		% for almost surely
\newmacro{\event}{E}       % for event
\newmacro{\eventalt}{H}       % for variant event
\newmacro{\mean}{\mu}		% for mean of distribution
\newmacro{\sdev}{\sigma}		% for mean of distribution
\newmacro{\variance}{\sdev^{2}}		% for mean of distribution


%----------------------------------------------------------------------
%% Stochastic approximation
%----------------------------------------------------------------------
\newcommand{\est}[1]{\hat #1}		% for estimates

\newmacro{\proper}{\tau}		% for proper time
\newcommand{\apt}[2][]{\state_{#1}(#2)}		% for APT (X by default)

\newmacro{\error}{Z}		% for error
\newmacro{\noise}{U}		% for noise
\newmacro{\bias}{b}		% for bias
\newmacro{\brown}{W}		% for Wiener process

\newmacro{\serror}{\theta}		% for scalar error
\newmacro{\snoise}{\xi}		% for scalar noise
\newmacro{\sbias}{\psi}		% for scalar bias

\newmacro{\sbound}{M}		% for signal bound
\newmacro{\bbound}{B}		% for bias bound
\newmacro{\noisepar}{\sdev}		% for noise parameter
\newmacro{\noisevar}{\variance}		% for noise variance


%----------------------------------------------------------------------
%% Legendre
%----------------------------------------------------------------------
\newmacro{\leg}{\alpha}		% Legendre exponent
\newcommand{\expleg}[1][{\sol[\legexp]}]{\frac{#1}{1-{#1}}}
\newmacro{\legexp}{\beta}		% for ``inverse'' Legendre exponent
\newmacro{\legsol}{\sol[\legexp]}		% Legendre exponent at a solution
\newmacro{\legconst}{\kappa}		% for Legendre constant
\newmacro{\legnhd}{\mathcal{U}}		% the nhd of the Legendre exp
\DeclarePairedDelimiterXPP{\legof}[1]{\legexp_{\hreg}}{(}{)}{}{#1}
\newmacro{\bregexp}{\alpha} % Legacy
\newmacro{\bregcst}{M} % Legacy
\newmacro{\kernelcst}{C}
\newmacro{\kernelexp}{p}
\newmacro{\bregbdedcst}{R}


%----------------------------------------------------------------------
%% Polyhedra
%----------------------------------------------------------------------
\newmacro{\constr}{j}		% for constraints
\newmacro{\nConstr}{m}		% for number of constraints

\newmacro{\slack}{\nu}		% for slackness multipliers

\newmacro{\actcoords}{\mathcal{A}}
%\newmacro{\actcoords}{\sol[\act]}
\newmacro{\flatcoords}{\actcoords_{\flat}}
\newmacro{\sharpcoords}{\actcoords_{\sharp}}

\newmacro{\drift}{\slack^{\ast}}
\newmacro{\drifteff}{\slack_{\mathrm{eff}}}
\newcommand{\acts}{\actcoords}		% shortcut
\newcommand{\sharps}{\sharpcoords}		% shortcut
\newcommand{\flats}{\flatcoords}		% shortcut

\newmacro{\coords}{\mathcal{I}}
%\newmacro{\coordsalt}{J}
\newmacro{\polycst}{P}
\newcommand{\orth}[1]{{#1}^{\bot}}
\newmacro{\intcst}{\sol[\coef]}
\newmacro{\intcstalt}{\widetilde{\coef}}



%----------------------------------------------------------------------
%% Sequences
%----------------------------------------------------------------------

\newmacro{\expstep}{\eta}
%\newmacro{\shift}[1][\run]{#1_0}
\newmacro{\cst}{q} % Constants for sequences
\newmacro{\cstalt}{q'} 
\newmacro{\seq}{u} % To be used in conjunction with prev, lead, current ie \curr[\seq]
\newmacro{\seqalt}{b} 



%**********************************************************************
%***    MACROS: AUTHOR-SPECIFIC
%**********************************************************************

%----------------------------------------------------------------------
%% WA MACROS
%----------------------------------------------------------------------
\addauthor[Waïss]{WA}{DarkGreen}
\newcommand{\WA}{\WAmargincomment}
%\newcommand{\WA}[1]{\hfill \break\colorbox{green!20!white}{\parbox{\textwidth}{\textbf{\footnotesize WA:}{~ \footnotesize #1}}}}
\newcommand{\half}[1][1]{{\frac{#1}{2}}}


%----------------------------------------------------------------------
%% FI MACROS
%----------------------------------------------------------------------
\addauthor[Franck]{FI}{Purple}
\newcommand{\FI}{\FImargincomment}
%\newcommand{\FI}[1]{\hfill \break\colorbox{Orange!20!white}{\parbox{\textwidth}{\textbf{\footnotesize FI:}{~ \footnotesize #1}}}}


%----------------------------------------------------------------------
%% JM MACROS
%----------------------------------------------------------------------
\addauthor[Jérôme]{JM}{DarkRed}
\newcommand{\JM}{\JMmargincomment}
%\newcommand{\JM}[1]{\hfill \break\colorbox{red!20!white}{\parbox{\textwidth}{\textbf{\footnotesize JM:}{~ \footnotesize #1}}}}


%----------------------------------------------------------------------
%% PM MACROS
%----------------------------------------------------------------------
\addauthor[Pan]{PM}{MediumBlue}
\newcommand{\PM}{\PMmargincomment}

\newmacro{\region}{\mathcal{R}}

\newcommand{\negspace}{\!\!\!}		% for short negative space
\newmacro{\fn}{f}		% for generic function
\newmacro{\fixmap}{F}		% for fixed-point iteration
\newmacro{\fixmapalt}{G}		% for fixed-point iteration
\newmacro{\gold}{\varphi}		% for golden ratio
\newmacro{\silver}{\Phi}		% for conjugate golden ratio
\newmacro{\energy}{E}		% for energy
\newmacro{\Lyap}{L}		% for energy
\newmacro{\diff}{\alpha} 

\newmacro{\thres}{\delta}		% for confidence level
\newmacro{\basin}{\mathcal{B}}		% for basin of attraction
\newmacro{\inhd}{\init[\nhd]}

\newmacro{\seed}{\theta}		% for seed
\newmacro{\seeds}{\Theta}		% for seed space
\newmacro{\pdist}{P}		% for seed law
\newmacro{\history}{\mathcal{H}}		% for filtrations



%**********************************************************************
%***    MAIN DOCUMENT BEGINS HERE
%**********************************************************************
\begin{document}


%**********************************************************************
%***    FRONT MATTER AND METADATA
%**********************************************************************

%----------------------------------------------------------------------
%%% TITLE & AUTHORS
%----------------------------------------------------------------------
\title
[Sharp convergence rates for Bregman proximal methods]
{The rate of convergence of Bregman proximal methods:\\
Local geometry vs. regularity vs. sharpness}		% for long title


%-------------------------------------------------------------------
\author
[W.~Azizian]
{Waïss Azizian$^{\ast,\sharp}$}
\address{$^{\ast}$\,%
Univ. Grenoble Alpes, LJK, Grenoble 38000, France.}
\address{$^{\sharp}$\,%
DI, ENS, Univ.~PSL, 75005, Paris, France.}
%-------------------------------------------------------------------
\author
[F.~Iutzeler]
{Franck Iutzeler$^{\ast}$}
%-------------------------------------------------------------------
\author
[J.~Malick]
{\\Jérôme Malick$^{\diamond}$}
\address{$^{\diamond}$\,%
Univ. Grenoble Alpes, CNRS, Grenoble INP, LJK, 38000 Grenoble, France.}
%-------------------------------------------------------------------
\author
[P.~Mertikopoulos]
{Panayotis Mertikopoulos$^{\S}$}
\address{$^{\S}$\,%
Univ. Grenoble Alpes, CNRS, Inria, Grenoble INP, LIG, 38000 Grenoble, France.}
%-------------------------------------------------------------------


\thanks{The authors are grateful to J.~Bolte for many fruitful discussions.}


%----------------------------------------------------------------------
%%% KEYWORDS
%----------------------------------------------------------------------
\subjclass[2020]{%
Primary 65K15, 90C33;
secondary 68Q25, 68Q32.}

\keywords{%
Legendre exponent;
optimistic mirror descent;
variational inequalities}


%----------------------------------------------------------------------
%%% ACRONYMS
%----------------------------------------------------------------------
\newacro{LHS}{left-hand side}
\newacro{RHS}{right-hand side}
\newacro{iid}[i.i.d.]{independent and identically distributed}
\newacro{lsc}[l.s.c.]{lower semi-continuous}
\newacro{NE}{Nash equilibrium}
\newacroplural{NE}[NE]{Nash equilibria}

\newacro{ABP}{abstract Bregman proximal}
\newacro{BP}{Bregman proximal}
\newacro{BPM}{Bregman proximal method}

\newacro{DGF}{distance-generating function}
\newacro{EG}{extra-gradient}
\newacro{MP}{mirror-prox}
\newacro{MD}{mirror descent}
\newacro{OMD}{optimistic mirror descent}
\newacro{OMWU}{optimistic multiplicative weights update}
\newacro{PMP}{past mirror-prox}
\newacro{AMP}{abstract mirror-prox}
\newacro{MPT}{mirror-prox template}

\newacro{VI}{variational inequality}
\newacroplural{VI}[VIs]{variational inequalities}
\newacro{VIP}{variational inequality problem}
\newacro{KKT}{Karush\textendash Kuhn\textendash Tucker}
\newacro{FOS}{first-order stationary}
\newacro{SOO}{second-order optimality}
\newacro{SOS}{second-order sufficient}
\newacro{DGF}{distance-generating function}
\newacro{SFO}{stochastic first-order oracle}



%----------------------------------------------------------------------
%%% ABSTRACT
%----------------------------------------------------------------------
\begin{abstract}
\section*{Abstract}    
\label{sec:abstract}
\addcontentsline{toc}{chapter}{Abstract}

%corrected by Mikolaj
With the advent of precision cosmology, our theoretical predictions must aspire to the same level of precision as achieved by experimental probes. In this context, numerical simulations including general relativistic effects represent the state-of-the-art method to describe the formation of structures. However, aside from a detailed description of the dynamics, it is necessary to have an equally accurate explanation of the effects of such structures on light propagation and modelling their impacts on measurable quantities. 

The investigation of relativistic effects in the most general way requires a unified treatment of light propagation in cosmology. This goal can be achieved with the new interpretation of the geodesic deviation equation in terms of the bilocal geodesic operators (BGO). The BGO formalism extends the standard formulation, providing a unified framework to describe all possible optical phenomena due to the interaction between light and spacetime curvature. 
	
In my dissertation, I present {\tt BiGONLight}, a {\tt Mathematica} package that applies the BGO formalism to study light propagation in numerical relativity. The package encodes the 3+1 bilocal geodesic operators framework as a collection of {\tt Mathematica} functions. The inputs are the spacetime metric plus the kinematics of the observer and the source in the form of the 3+1 quantities, which may come directly from a numerical simulation or can be provided by the user as analytical components. These data are then used for ray tracing and computing the BGO's in a completely general way, i.e. without relying on symmetries or specific coordinate choices. The primary purpose of the package is the computation of optical observables in arbitrary spacetimes. The uniform theoretical framework of the BGO formalism allows for the extraction of multiple observables within a single computation, while the {\tt Wolfram} language provides a flexible computational framework that makes the package highly adaptable to perform both numerical and analytical studies of light propagation. {\tt BiGONLight} is tested by computing the redshift, angular diameter distance, parallax distance, and redshift drift in well-known cosmological models. We use three different inputs for the metric: two analytical metrics, the homogeneous $\Lambda$CDM model and the inhomogeneous Szekeres model, and 3+1 quantities from a simulated dust Universe. The tests show an excellent agreement with known results.

The characteristics of {\tt BiGONLight} make it a suitable tool for studying the impact of inhomogeneities on light propagation. We investigate various sources of nonlinear general relativistic effects on light propagation induced by inhomogeneous cosmic structures. {\tt BiGONLight} is used to calculate observables computed at different approximations in a plane-parallel inhomogeneous spacetime. The nonlinear effects are evaluated as the fractional difference between the observables obtained at the three different approximations: linear perturbation theory, Newtonian, and post-Newtonian approximations. The inhomogeneities are tuned by varying the model’s free parameters, and their contributions to the observables are obtained by analysing the variations in the fractional differences. Using this method we estimate the Newtonian and post-Newtonian corrections to the linear observables and analyse how these corrections change as we vary the size and magnitude of the inhomogeneities. We also explain the role of the linear initial seed as the dominant post-Newtonian contribution and show that the remaining post-Newtonian nonlinear corrections are less than $1\%$, which is consistent with previous results in the literature.






\endinput


\end{abstract}
\maketitle



%**********************************************************************
%***    BODY TEXT
%**********************************************************************
\allowdisplaybreaks		% for breaking long displays
\acresetall		% for resetting acros


%----------------------------------------------------------------------
%%% INTRODUCTION
%----------------------------------------------------------------------
\section{Introduction}
\label{sec:introduction}
%----------------------------------------------------------------------
%%% INTRODUCTION
%----------------------------------------------------------------------
% !TEX root = ../Main.tex


\Acp{BPM} have a long and rich history in optimization, going back at least to the introduction of \acl{MD} by Nemirovski \& Yudin \citep{NY83}.
In plain terms, \acp{BPM} are first-order (constrained) optimization algorithms that forego Euclidean projections in favor of a more sophisticated ``prox-mapping'' that minimizes a certain distance-like functional known as the Bregman divergence \citep{NY83,CT93,Bre67,Kiw97}.
When this Bregman divergence is the Euclidean distance squared, one recovers the standard projection-based methods;
other than that, depending on the problem's feasible region, different Bregman setups lead to a diverse collection of algorithms,
from exponentiated gradient descent on the simplex \citep{NY83,BecTeb03,ACBFS02},
to matrix multiplicative weights on the positive-semidefinite cone \cite{TRW05,KSST12},
variants of Karmarkar's affine scaling algorithm for linear programs \cite{VMF86},
etc.

One of the most appealing features of \acp{BPM} is that they achieve almost dimension-free convergence rates in problems with a convex structure and a favorable geometry \textendash\ such as the $L^{1}$ ball, the spectraplex, second-order cones, etc. \cite{Bub15,Nes09,BecTeb03}.
This is owed to a delicate interplay between the algorithms' non-Euclidean update scheme and the global geometry of the problem's domain.
However, these (almost) dimension-free guarantees also come with some strings attached:
they do not concern the sequence of iterates generated by the method, but only its time average
\revise{(or, through the same, ``regret-based'' analysis, the method's ``best iterate'')};
in this way, the best guarantee that can be achieved after $\run$ iterations is $\bigoh(1/\run)$.

In terms of oracle complexity, this is sufficient for problems that are not strongly convex\,/\,strongly monotone, but if one targets finer, geometric convergence rates,
\revise{the inherent averaging involved in regret-based guarantees is hard to compensate.}
And, on the other extreme, if the problem is not convex\,/\,monotone to begin with, iterate averaging does not provide any quantifiable benefits whatsoever, so it becomes crucial to study the actual trajectory of the method.


%----------------------------------------------------------------------
%%% CONTRIBS
%----------------------------------------------------------------------
\para{Our contributions}

Our paper seeks to quantify the last-iterate convergence rate of \aclp{BPM} as a function of the Bregman divergence defining the method and the local geometry that it induces.
To treat this question in as general a manner as possible, we focus on \ac{VI} problems of the form
\begin{equation}
\label{eq:VI}
\tag{VI}
\text{Find $\sol\in\points$ such that}
	\;\;
	\braket{\vecfield(\sol)}{\point - \sol}
	\geq 0
	\;\;
	\text{for all $\point\in\points$},
\end{equation}
where $\points$ is a closed convex subset of a finite-dimensional normed space $\pspace$, and $\vecfield \from \points \to \dspace$ is a (possibly non-monotone) single-valued operator on $\points$ with values in $\dspace$, the dual of $\pspace$.
This problem is a staple of many areas of mathematical programming, game theory and data science, as it provides a template for ``optimization beyond minimization'' \textendash\ \ie for problems where finding an optimal solution does not necessarily involve minimizing a loss function.
In particular, in addition to standard minimization problems \textendash\ which are recovered when $\vecfield = \nabla\obj$ for some smooth function $\obj$ \textendash\ the general formulation \eqref{eq:VI} includes saddle-point problems, games, complementarity problems, etc.;
for an introduction, see \cite{FP03} and references therein.

In this broad context, we examine the rate of convergence of a wide class of \aclp{BPM} to local solutions of \eqref{eq:VI} that satisfy a \acl{SOS} condition.
Specifically, the class of algorithms we consider includes as special cases
\begin{enumerate*}
[(\itshape i\hspace*{1pt}\upshape)]
\item
the original \acf{MD} algorithm of \cite{NY83};
\item
the \acf{MP} method of Nemirovski \cite{Nem04} \textendash\ which has the same update structure as the Bregman-based algorithm of \cite{AT05} and contains as a special case the \acf{EG} algorithm of \cite{Kor76};
\item
the so-called \acf{OMD} method of \cite{RS13-NIPS} \textendash\ itself a Bregman analogue of the modified Arrow-Hurwicz algorithm of \cite{Pop80};
\end{enumerate*}
etc.

Our first finding is a crisp characterization of last-iterate convergence rate of \acp{BPM} in terms of the local geometry induced by the underlying Bregman function near a given solution of \eqref{eq:VI}.
We make this dependence precise via the notion of the \emph{Legendre exponent}, a regularity measure for Bregman methods due to \cite{AIMM21}, which can roughly be described as the logarithmic ratio of the volume of a Euclidean ball to that of a Bregman ball of the same radius.
For example, Euclidean methods have a Legendre exponent of $\legexp = 0$ and they converge at a linear rate;
entropic methods have a Legendre exponent of $\legexp = 1/2$ at boundary points, and they converge at a rate of $\bigoh(\run^{-1})$;
more generally,
as we show in \cref{thm:general}, methods with a Legendre exponent $\legexp>0$ converge at a rate of $\bigoh(\run^{1-1/\legexp})$.
\PM{We need to fix this: the $1-1/\legexp$ exponent is not consistent with the $\bigoh(1/\run)$ expression.}
\WA{I don't see the issues, yes this expression is not well-defined for $\legexp = 0$ but this is normal, the two situations differ radically.}
The Euclidean regime ($\legexp = 0$) is perfectly aligned with existing results for the geometric last-iterate convergence rate of the \ac{EG} algorithm and its variants \citep{GBVV+19,Mal15,HIMM19,MOP20}.
By contrast, the Legendre regime ($\legexp > 0$) indicates a significant drop in the algorithm's last-iterate convergence speed, even though ergodic convergence rates \cite{Nes04} and results for bilinear games \cite{WLZL21} might suggest otherwise.

Subsequently, motivated by applications to game theory and linear programming, we take a closer look at the convergence rate of \acp{BPM} across the constraints that are active at a solution $\sol$ of \eqref{eq:VI} depending on the position of $\vecfield(\sol)$ relative to said constraints. 
This analysis reveals that Bregman proximal methods have a particularly fine structure:
along \emph{sharp directions} (\ie constraints along which $\vecfield(\sol)$ is strictly inward-pointing), \acp{BPM} converge
\begin{enumerate*}
[(\itshape i\hspace*{1pt}\upshape)]
\item
at a rate of $\bigoh(1/\run^{1/(2\legexp-1)})$ if $1/2 < \legexp < 1$;
\item
at a \emph{geometric rate} if $0 < \legexp \leq 1/2$ (\eg for entropic methods);
and
\item
in a \emph{finite} number of iterations if $\legexp=0$
\end{enumerate*}
(\cf \cref{thm:sharp}).
Thus, even though the estimates of \cref{thm:general} are, in general tight, the actual convergence rate of a Bregman method along different coordinates\,/\,constraints could be starkly different \textendash\ and, in fact, dramatically faster if the solution under study is itself sharp.

The closest antecedent of our work is the conference paper \cite{AIMM21} where the Legendre exponent was introduced to analyze the convergence of \ac{OMD} in \emph{stochastic} \ac{VI} problems (without considering sharp directions and/or faster identification rates).
The stochastic and deterministic settings are obviously very different, both in the challenges involved as well as the rates obtained, so there is no overlap in our analysis and results.
Other than that, we are not aware of any comparable results in the literature concerning the radically different convergence landscape of \acp{BPM} along active and inactive constraints.


%----------------------------------------------------------------------
%%% SETUP
%----------------------------------------------------------------------
\section{Problem setup and preliminaries}
\label{sec:setup}
%----------------------------------------------------------------------
%% SETUP
%----------------------------------------------------------------------
% !TEX root = ../Main.tex


In the rest of our paper
$\pspace$ will denote an $\nCoords$-dimensional real space with norm $\norm{\cdot}$
and
$\points$ will be a closed convex subset thereof.
We will also write
$\dpoints \defeq \dspace$ for the dual of $\pspace$,
$\braket{\dpoint}{\point}$ for the canonical pairing between $\dpoint\in\dpoints$ and $\point\in\pspace$,
and
$\dnorm{\dpoint} \defeq \max \setdef{\braket{\dpoint}{\point}}{\norm{\point}\leq 1}$ for the induced dual norm on $\dpoints$.


%----------------------------------------------------------------------
%% ASSUMPTIONS
%----------------------------------------------------------------------
\subsection{Blanket assumptions}

Throughout the sequel, we will make the following assumptions for the defining vector field $\vecfield\from\points\to\dpoints$ of \eqref{eq:VI} and the solution $\sol\in\points$ under study:

\begin{assumption}
[Lipschitz continuity]
\label{asm:Lipschitz}
The vector field $\vecfield$ is \emph{$\lips$-Lipschitz continuous}, \ie
\begin{equation}
\label{eq:Lipschitz}
\tag{LC}
\dnorm{\vecfield(\pointalt) - \vecfield(\point)}
	\leq \lips \norm{\pointalt - \point}
	\quad
	\text{for all $\point,\pointalt\in\points$}.
\end{equation}
\end{assumption}

\begin{assumption}
[Second-order sufficiency]
\label{asm:strong}
There exists
a convex neighborhood $\basin$ of $\sol$ in $\points$
and
a positive constant $\strong > 0$
such that
\begin{equation}
\label{eq:strong}
\tag{SOS}
\braket{\vecfield(\point) - \vecfield(\sol)}{\point - \sol}
	\geq \strong \norm{\point - \sol}^{2}
	\quad
	\text{for all $\point\in\basin$}.
\end{equation}
\end{assumption}

In general, \Cref{asm:strong} guarantees that $\sol$ is the unique solution of \eqref{eq:VI} in $\basin$;
we illustrate this in two special cases of interest:
\begin{enumerate}
\item
\emph{Minimization problems:}
suppose that $\vecfield = \nabla\obj$ for some Lipschitz smooth objective function $\obj$ on $\points$.
Then, \cref{asm:strong} implies that $\obj$ grows (at least) quadratically along every ray emanating from $\sol$, \ie $\obj(\point) - \obj(\sol) \geq \braket{\subsel\obj(\sol)}{\point - \sol} + (\strong/2) \norm{\point - \sol}^{2} = \Omega(\norm{\point-\sol}^{2})$ for all $\point\in\basin$, implying in particular that $\sol$ is an isolated minimizer of $\obj$.
\item
\emph{Min-max problems:}
suppose that $\points$ factorizes as $\points = \minvars\times\maxvars$ for suitable factor sets $\minvars$, and $\maxvars$,
let $\minmax\from\points\to\R$ be a smooth function on $\points$,
and
write $\vecfield = (\nabla_{\minvar}\minmax,-\nabla_{\maxvar}\minmax)$ for the min-max gradient of $\minmax$ (with respect to $\minvar\in\minvars$ and $\maxvar\in\maxvars$ respectively).
Then, any solution $\sol = (\minsol,\maxsol)$ of \eqref{eq:VI} that satisfies \cref{asm:strong} enjoys the local growth bounds $\minmax(\minvar,\maxsol) - \minmax(\minsol,\maxsol) = \Omega(\norm{\minvar - \minsol}^{2})$ and $\minmax(\minsol,\maxsol) - \minmax(\minsol,\maxvar) = \Omega(\norm{\maxvar - \maxsol}^{2})$, implying in turn that $\sol$ is an isolated, hyperbolic saddle-point of $\minmax$.
\end{enumerate}
More examples satisfying \eqref{eq:strong} include strict \aclp{NE} in finite games \cite{FT91}, deterministic Nash policies in (generic) stochastic games \cite{SV15}, etc.
Overall, \cref{asm:Lipschitz,asm:strong} apply to a very wide range of problems, so we will treat them as blanket assumptions throughout.


%----------------------------------------------------------------------
%% METHOD
%----------------------------------------------------------------------
\subsection{\acl{BP} methods}
\label{subsec:BP_methods}

As we discussed in the introduction, the main algorithmic template that we will examine for solving \eqref{eq:VI} is a general class of first-order algorithms known as \acdefp{BPM}.
The defining ingredient of this class is the notion of \emph{Bregman regularizer}, which we define below as follows:

\begin{definition}
[Bregman regularizers and related notions]
\label{def:Bregman}
A proper, \acl{lsc}, strictly convex function $\hreg\from\pspace\to\R\cup\{\infty\}$ is a \emph{Bregman regularizer} on $\points$ if
the following are true
\begin{enumerate}
\item
$\hreg$ is supported on $\points$, \ie $\dom\hreg = \points$.
\item
The subdifferential of $\hreg$ admits a \emph{continuous selection}, \ie there exists a continuous mapping $\subsel\hreg\from\dom\subd\hreg\to\dpoints$ such that $\subsel\hreg(\point) \in \subd\hreg(\point)$ for all $\point\in\dom\subd\hreg$.
\item
$\hreg$ is \emph{locally strongly convex} relative to $\norm{\cdot}$, \ie for any compact set $\cpt \subseteq \dom \hreg$, we have
%there is some $\hstr > 0$ such that
%for all $\point\in \cpt \cap \dom\subd\hreg, \pointalt\in \cpt$, we have
\begin{equation}
\label{eq:hstr}
\hreg(\pointalt)
	\geq \hreg(\point)
		+ \braket{\subsel\hreg(\point)}{\pointalt - \point}
		+ \tfrac{\hstr}{2} \norm{\pointalt - \point}^{2}
%	\quad
%	\text{for all $\point\in \cpt \cap \dom\subd\hreg, \pointalt\in \cpt$}.
\end{equation}
for some $\hstr>0$ and for all $\point\in \cpt \cap \dom\subd\hreg$, $\pointalt\in \cpt$. 
\end{enumerate}
\noindent
%\WA{I don't know why there is so much space here now...}
%\PM{Welcome to the intricacies of \TeX :-)}
%\PM{The problem is that the item above was wrapped in a ``revise'' macro, which interferes with the equation wrapper (it adds a stop, and hence a linebreak).
%This is one of the reasons that I had introduced the beginrev and endedit commands (which don't have this problem and also don't interfere with Sync\TeX\ on longer tracts of text).
%I should have explained this, sorry\dots}
The set $\proxdom \defeq \dom\subd\hreg$ will be referred to as the \emph{prox-domain} of $\hreg$.
In addition, we also define the \emph{Bregman divergence} of $\hreg$ as
\begin{alignat}{2}
\label{eq:Breg}
\breg(\base,\point)
	&= \hreg(\base)
		- \hreg(\point)
		- \braket{\subsel\hreg(\point)}{\base - \point}
	&\qquad
	&\text{for all $\point\in\proxdom$, $\base\in\points$}
\intertext{and the induced \emph{prox-mapping} as}
\label{eq:prox}
\proxof{\point}{\dvec}
	&= \argmin\nolimits_{\pointalt\in\points} \{ \braket{\dvec}{\point - \pointalt} + \breg(\pointalt,\point) \}
	&\qquad
	&\text{for all $\point\in\proxdom$, $\dvec\in\dpoints$}.
\end{alignat}
\end{definition}

\begin{remark}
\label{rem:Bregman}
Examples of Bregman regularizers are given in \cref{sec:examples}, where we also take an in-depth look at their properties.
For our analysis and results, we will assume for convenience that $\hreg$ is $1$-strongly convex in a suitable neighborhood $\zone$ of $\sol$ which will be understood from the context;
this can always be achieved by rescaling $\hreg$, so there is no loss of generality.
To avoid technicalities, we will also tacitly assume that $\proxof{\point}{\dpoint}$ is well-defined whenever it is invoked (this is always the case if, for example, $\hreg$ is coercive or $\nabla\hreg$ is invertible).
\end{remark}

Given a Bregman regularizer on $\points$, the general class of \acdefp{BPM} that we will consider is defined via the generic recursion
\begin{equation}
\label{eq:BPM}
\tag{BPM}
\begin{aligned}
\lead
	= \proxof{\curr}{-\curr[\step]\curr[\signal]}
%	\\
	\qquad
\next
	= \proxof{\curr}{-\curr[\step]\lead[\signal]}
\end{aligned}
\end{equation}
where
\begin{enumerate*}
[(\itshape i\hspace*{1pt}\upshape)]
\item
$\run=\running$ denotes the method's iteration counter;
\item
$\curr[\step] > 0$ is a (non-increasing) step-size sequence;
\item
$\curr[\signal]$ and $\lead[\signal]$ are sequences of ``oracle signals'' that we discuss in detail below.
\end{enumerate*}
In terms of vocabulary, the iterates $\curr$, $\run=\running$, will be referred to as the ``\emph{base states}'' of the method, while the ``half-iterates'' $\lead$, $\run=\running$, will be referred to as the method's ``\emph{leading states}''.
Finally, in terms of initialization, we will take for convenience $\init = \state_{1/2}$.

Now, regarding the sequence of oracle signals $\curr[\signal]$ and $\lead[\signal]$ defining \eqref{eq:BPM}, we will assume throughout that
\begin{equation}
\label{eq:signal-lead}
\lead[\signal]
	= \vecfield(\lead)
	\qquad
	\text{for all $\run=\running$}
\end{equation}
\ie \eqref{eq:BPM} generates a new base state $\next$ by taking a Bregman proximal step from $\curr$ with oracle input from the leading state $\lead$.
By contrast, the leading state itself can be generated in a number of different ways from $\curr$, depending on the definition of $\curr[\signal]$:

\begin{assumption}
\label{asm:signal-base}
For all $\run=\running$, the oracle signal $\curr[\signal]$ is of the form:
\begin{equation}
\label{eq:signal-base}
\curr[\signal]
	= \coef[a] \vecfield(\curr)
		+ \coef[b] \vecfield(\beforelead)
\end{equation}
for some $\coef[a],\coef[b] \in [0,1]$ with
$\coef[a] + \coef[b] \leq 1$
and
$\coef[a] + \coef[b] = 1$ if $\coef[b]>0$.\footnote{Note that the requirement ``$\coef[a] + \coef[b] = 1$ if $\coef[b] > 0$'' is only intended to ease notation and does not lead to a loss in generality:
if $\coef[b] > 0$, we can always rescale $\curr[\step]$ by $\coef[a] + \coef[b]$ so the condition $\coef[a] + \coef[b] = 1$ is satisfied automatically.}
\end{assumption}

For concreteness, we illustrate below three archetypal Bregman methods that serve as the backbone of the above framework:

\begin{enumerate}
\setlength{\itemsep}{0pt}
\item
\Acli{MD}:
following \cite{NY83,BecTeb03,NJLS09}, the \acli{MD} algorithm proceeds recursively as $\new = \proxof{\point}{-\step\vecfield(\point)}$, so it can be recovered from \eqref{eq:BPM} by taking
\begin{alignat}{3}
\label{eq:MD}
\tag{MD}
\coef[a] = 0, \coef[b] = 0
	&\qquad
	\text{or, equivalently}
	&\qquad
\curr[\signal]
	&= 0
	&\quad
	\text{for all $\run=\running$}
\intertext{%
\item
\Acli{MP}:
following \cite{Nem04,JNT11}, the \acli{MP} algorithm corresponds to the choice}
\label{eq:MP}
\tag{MP}
\coef[a] = 1, \coef[b] = 0
	&\qquad
	\text{or, equivalently}
	&\qquad
\curr[\signal]
	&= \vecfield(\curr)
	&\quad
	\text{for all $\run=\running$}
\intertext{%
\item
\Acli{OMD}:
originally due to \cite{Pop80} (in the Euclidean case) and \cite{CYLM+12,RS13-NIPS} (for the general case), the \acli{OMD} algorithm is obtained by setting}
\label{eq:OMD}
\tag{OMD}
\coef[a] = 0, \coef[b] = 1
	&\qquad
	\text{or, equivalently}
	&\qquad
\curr[\signal]
	&= \vecfield(\beforelead)
	&\quad
	\text{for all $\run=\running$}
\end{alignat}
\end{enumerate}
These three algorithms are the most widely studied Bregman methods in the literature, so we will use them as running examples throughout.


%----------------------------------------------------------------------
%%% EXAMPLES
%----------------------------------------------------------------------
\section{Motivating examples}
\label{sec:examples}
%----------------------------------------------------------------------
%%% EXAMPLES
%----------------------------------------------------------------------
% !TEX root = ../Main.tex


We now proceed to take a closer look at some commonly used Bregman regularizers (and the induced prox-mappings) with the goal of determining the rate of convergence of the associated Bregman method.
For concreteness, we focus on one-dimensional problems where $\points$ is \revise{the closed interval $[0,\infty)$ or $[-1, 1]$} and $\vecfield$ is the affine vector field
\begin{equation}
\label{eq:simple}
\vecfield(\point)
	= \point - \sol,
	\quad
	\point\in\R,
\end{equation}
for different choices of $\sol\in\R$
\revise{(typically a boundary point of $\points$)}.
To streamline our presentation, we will only examine the \acl{MD} recursion \eqref{eq:MD} with constant step-size schedules $\curr[\step] \equiv \step$ for some $\step>0$.
In this case, we obtain the scheme
\begin{equation}
\label{eq:MD-generic}
\next
	= \fixmap(\curr)
	\quad
	\text{with}
	\quad
\fixmap(\point)
	= \proxof{\point}{-\step\vecfield(\point)},
\end{equation}
and we will examine the convergence speed of $\curr$ by analyzing the behavior of $\fixmap$ near $\sol$.
\revise{To illustrate the spectrum of different behaviors that arise near the boundary of $\points$, we will focus primarily on cases where $\sol$ is a boundary point.}
\smallskip

%----------------------------------------------------------------------
%% Euclidean begins here

\begin{example}
[Euclidean regularization]
\label{ex:Eucl}
We begin with the quadratic regularizer $\hreg(\point) = \point^{2}/2$ for $\point\in\points = [0,\infty)$.
In this case, noting that $\hreg'(\point) = \point$, we have:
\begin{flalign}
\label{eq:mirror-Eucl}
\begin{alignedat}{3}
\quad
	a)\;\;
	&\text{Prox-domain:}
	&\qquad
&\proxdom
	= \points
%	\hspace{20em}
	&\\
\quad
	b)\;\;
	&\text{Bregman divergence:}
	&\qquad
&\breg(\base,\point)
	= (\base-\point)^{2}/2
	&\\
\quad
	c)\;\;
	&\text{Prox-mapping:}
	&\qquad
&\proxof{\point}{\dvec}
	= \pospart{\point+\dvec}
	&
\end{alignedat}
&&
\end{flalign}
Consider now the case $\sol=0$, \ie $\vecfield(\point) = \point$.
Then, for $\step\in(0,1)$, the update \eqref{eq:MD-generic} becomes
\begin{equation}
\label{eq:MD-Eucl}
\fixmap(\point)
	= \point - \step\point
	= (1-\step) \point
    \quad
    \text{\revise{for all $\point\geq0$}}
\end{equation}
\ie $\fixmap$ is contracting.
We thus conclude that $\curr$ converges to $\sol=0$ at a geometric rate, \viz
\begin{equation*}
\tag*{\endenv}
\breg(\sol,\curr)
	= \tfrac{1}{2} \curr^{2}
	= \Theta\parens[\big]{(1-\step)^{2\run}}
	\;\;
	\text{or, in absolute value,}
	\;\;
\abs{\curr - \sol}
	= \Theta\parens{(1-\step)^{\run}}.
\end{equation*}
\end{example}

%% Euclidean ends here
%----------------------------------------------------------------------


%----------------------------------------------------------------------
%% Entropy begins here

\begin{example}
[Entropic regularization]
\label{ex:ent}
Another popular choice when $\points=[0,\infty)$ is the entropic regularizer $\hreg(\point) = \point\log\point$ \cite{BecTeb03,SS11,BBT17}.
In this case, we have $\hreg'(\point) = 1 + \log\point$, and hence:
\begin{flalign}
\label{eq:mirror-ent}
\begin{alignedat}{3}
\quad
	a)\;\;
	&\text{Prox-domain:}
	&\qquad
&\proxdom
	= \relint\points
	= (0,\infty)
%	\hspace{15em}
	&\\
\quad
	b)\;\;
	&\text{Bregman divergence:}
	&\qquad
&\breg(\base,\point)
	= \base \log(\base/\point) + \point - \base
	&\\
\quad
	c)\;\;
	&\text{Prox-mapping:}
	&\qquad
&\proxof{\point}{\dvec}
	= \point \exp(\dvec).
	&
\end{alignedat}
&&
\end{flalign}
Now, taking $\vecfield(\point) = \point$ as in the previous example, the update rule \eqref{eq:MD-generic} becomes
\begin{equation}
\label{eq:MD-ent}
\fixmap(\point)
	= \point \exp(-\step\point)
	= \point(1 - \step\point  + o(\point))
	= \point - \step\point^2  + o(\point^2)
	\quad
	\text{as $\point\to0$}.
\end{equation}
In contrast to \eqref{eq:MD-Eucl}, we now have $\fixmap(\point) \sim \point$ instead of $(1-\step)\point$, so $\fixmap$ is no longer a contraction.
Instead, the iterates of \eqref{eq:MD-ent} may be analyzed by means of the following lemma:
\begin{restatable}{lemma}{basicnum}
\label{lem:basicnum}
Suppose that $\fn\from\R_+\to\R_+$ admits the asymptotic expansion
\begin{equation}
\fn(\point)
	= \point
		- \coef\point^{1+\rexp}
		+ o(\point^{1+\rexp})
	\quad
	\text{as $\point\to0$}
\end{equation}
for positive constants $\coef,\rexp>0$.
Then, for $\init[\seq] > 0$ small enough, the sequence $\next[\seq] = \fn(\curr[\seq])$, $\run=\running$, converges to $0$ at a rate of $\curr[\seq] \sim (\coef\rexp\run)^{-1/\rexp}$.
\end{restatable}


Thanks to this lemma (which we prove \refapp{app:aux}), we readily conclude that $\curr$ converges to $0$ at a rate of
$\breg(\sol,\curr)
	= \curr
	= \abs{\curr - \sol}
	\sim 1/(\step\run).$
\hfill
\endenv
\end{example}

%% Entropy ends here
%----------------------------------------------------------------------


%----------------------------------------------------------------------
%% Fractional begins here

\begin{example}
[Fractional power]
\label{ex:frac}
Take $\points = [0,\infty)$ and $\vecfield(\point) = \point$ as in \cref{ex:Eucl,ex:ent} above.
Then, for a given $\qexp>0$, $\qexp\neq1$, the \emph{fractional power} regularizer \textendash\ or \emph{Tsallis entropy} \textendash\ on $\points$ is defined as $\hreg(\point) = [\qexp(1-\qexp)]^{-1} (\point - \point^{\qexp})$ \citep{Tsa88,ABB04,MS16}.
For this choice of regularizer, we have $\hreg'(\point) = (1 - \qexp\point^{\qexp-1}) / [\qexp(1-\qexp)]$, and a series of direct calculations gives:%
\footnote{Strictly speaking, the expression we provide for $\proxof{\point}{\dvec}$ is only valid when $\dvec < \point^{\qexp-1}/(1-\qexp)$.
\revise{The reason for this is that the} prox-mapping $\proxof{\point}{\dvec}$ is not well-defined for all values of $\dvec$;
this detail is not important in the calculations that follow, so we disregard it for now.}
\begin{flalign}
\label{eq:mirror-frac}
\begin{alignedat}{3}
\quad
	a)\;\;
	&\text{Prox-domain:}
	&\qquad
&\text{$\proxdom = (0,\infty)$ if $\qexp\in(0,1)$ and $\proxdom = [0,\infty)$ if $\qexp > 1$}
%	\hspace{3em}
	&\\
\quad
	b)\;\;
	&\text{Bregman divergence:}
	&\qquad
&\breg(\base,\point)
	= \frac{\point^{\qexp} - \base^{\qexp}}{\qexp(1-\qexp)}
		- \point^{\qexp-1} \frac{\point - \base}{1-\qexp}
	&\\
\quad
	c)\;\;
	&\text{Prox-mapping:}
	&\qquad
&\proxof{\point}{\dvec}
	= \bracks[\big]{\point^{\qexp-1} - (1-\qexp) \dvec}^{\frac{1}{\qexp-1}}
	\quad
	\text{for $\qexp\in(0,1)$}.
	&
\end{alignedat}
&&
\end{flalign}
Now, when applied to $\vecfield(\point) = \point$, the fractional power variant of \eqref{eq:MD-generic} for $\qexp\in(0,1)$ gives
\begin{equation}
\label{eq:MD-frac}
\fixmap(\point)
	= \point \, \bracks{1 + \step(1-\qexp)\point^{2-\qexp}}^{1/(\qexp-1)}
	= \point - \step\point^{3-\qexp} + o(\point^{3-\qexp})
	\quad
	\text{as $\point\to0$}.
\end{equation}
Hence, by \cref{lem:basicnum}, we conclude that $\curr$ converges to $0$ at a rate of
\begin{equation}
\tag*{\endenv}
\breg(\sol,\curr)
	= \Theta\parens[\big]{\run^{-\qexp/(2-\qexp)}}
	\;\;
	\text{or, in absolute value,}
	\;\;
\abs{\curr - \sol}
	= \Theta\parens[\big]{\run^{-1/(2-\qexp)}}.
\end{equation}
\end{example}

%% Fractional ends here
%----------------------------------------------------------------------


%----------------------------------------------------------------------
%% Hellinger begins here

\begin{example}
[Hellinger distance]
\label{ex:Hell}
Our last example concerns the Hellinger regularizer $\hreg(\point) = -\sqrt{1-\point^{2}}$ on $\points = [-1,1]$.
Since $\hreg'(\point) = \point / \sqrt{1-\point^{2}}$, we readily obtain the following:
\begin{flalign}
\label{eq:mirror-Hell}
\begin{alignedat}{3}
\quad
	a)\;\;
	&\text{Prox-domain:}
	&\qquad
&\proxdom = \relint\points = (-1,1)
%	\hspace{15em}
	&\\
\quad
	b)\;\;
	&\text{Bregman divergence:}
	&\qquad
&\breg(\base,\point)
	= \frac{1 - \base\point - \sqrt{(1-\base^{2})(1-\point^{2})}}{\sqrt{1-\point^{2}}}
	&\\
\quad
	c)\;\;
	&\text{Prox-mapping:}
	&\qquad
&\proxof{\point}{\dvec}
	= \frac{\point + \dvec\sqrt{1-\point^{2}}}{\sqrt{1-\point^{2} + (\point + \dvec\sqrt{1-\point^{2}})^{2}}}.
	&
\end{alignedat}
&&
\end{flalign}
In this case, taking $\vecfield(\point) = \point$ as per the previous examples, yields
\begin{equation}
\fixmap(\point)
	= \frac{\point - \step\point\sqrt{1-\point^{2}}}{\sqrt{1-\point^{2} + (\point - \step\point\sqrt{1-\point^{2}})^{2}}}
	\sim \point - \step\point
	\quad
	\text{as $\point\to0$},
\end{equation}
\ie $\curr$ converges to $\sol = 0$ at a geometric rate, as in \cref{ex:Eucl}.
On the other hand, if we consider the shifted operator $\vecfield(\point) = \point+1$, a somewhat tedious calculation (which we detail \revise{\refapp{app:ex}}) gives the following Taylor expansion near $\sol = -1$:
\begin{equation}
\fixmap(\point)
	= \sol
		+ (\point - \sol)
		- 2\sqrt{2}\step (\point - \sol)^{5/2} + o\parens*{(\point-\sol)^{5/2}}.
\end{equation}
Hence, by \cref{lem:basicnum}, we conclude that $\curr$ converges to $\sol = -1$ at a rate of
\begin{equation}
\tag*{\endenv}
\breg(\sol,\curr)
	= \Theta\parens{\run^{-1/3}}
	\;\;
	\text{or, in absolute value,}
	\;\;
\abs{\curr - \sol}
	= \Theta(\run^{-2/3}).
\end{equation}
\end{example}

%% Hellinger ends here
%----------------------------------------------------------------------


%----------------------------------------------------------------------
%% Rate figure begins here

\begin{figure}
\centering
\begin{subfigure}[b]{.45\linewidth}
\resizebox{\textwidth}{!}{%----------------------------------------------------------------------
%%% EXAMPLES
%----------------------------------------------------------------------
% !TEX root = ../Main.tex


We now proceed to take a closer look at some commonly used Bregman regularizers (and the induced prox-mappings) with the goal of determining the rate of convergence of the associated Bregman method.
For concreteness, we focus on one-dimensional problems where $\points$ is \revise{the closed interval $[0,\infty)$ or $[-1, 1]$} and $\vecfield$ is the affine vector field
\begin{equation}
\label{eq:simple}
\vecfield(\point)
	= \point - \sol,
	\quad
	\point\in\R,
\end{equation}
for different choices of $\sol\in\R$
\revise{(typically a boundary point of $\points$)}.
To streamline our presentation, we will only examine the \acl{MD} recursion \eqref{eq:MD} with constant step-size schedules $\curr[\step] \equiv \step$ for some $\step>0$.
In this case, we obtain the scheme
\begin{equation}
\label{eq:MD-generic}
\next
	= \fixmap(\curr)
	\quad
	\text{with}
	\quad
\fixmap(\point)
	= \proxof{\point}{-\step\vecfield(\point)},
\end{equation}
and we will examine the convergence speed of $\curr$ by analyzing the behavior of $\fixmap$ near $\sol$.
\revise{To illustrate the spectrum of different behaviors that arise near the boundary of $\points$, we will focus primarily on cases where $\sol$ is a boundary point.}
\smallskip

%----------------------------------------------------------------------
%% Euclidean begins here

\begin{example}
[Euclidean regularization]
\label{ex:Eucl}
We begin with the quadratic regularizer $\hreg(\point) = \point^{2}/2$ for $\point\in\points = [0,\infty)$.
In this case, noting that $\hreg'(\point) = \point$, we have:
\begin{flalign}
\label{eq:mirror-Eucl}
\begin{alignedat}{3}
\quad
	a)\;\;
	&\text{Prox-domain:}
	&\qquad
&\proxdom
	= \points
%	\hspace{20em}
	&\\
\quad
	b)\;\;
	&\text{Bregman divergence:}
	&\qquad
&\breg(\base,\point)
	= (\base-\point)^{2}/2
	&\\
\quad
	c)\;\;
	&\text{Prox-mapping:}
	&\qquad
&\proxof{\point}{\dvec}
	= \pospart{\point+\dvec}
	&
\end{alignedat}
&&
\end{flalign}
Consider now the case $\sol=0$, \ie $\vecfield(\point) = \point$.
Then, for $\step\in(0,1)$, the update \eqref{eq:MD-generic} becomes
\begin{equation}
\label{eq:MD-Eucl}
\fixmap(\point)
	= \point - \step\point
	= (1-\step) \point
    \quad
    \text{\revise{for all $\point\geq0$}}
\end{equation}
\ie $\fixmap$ is contracting.
We thus conclude that $\curr$ converges to $\sol=0$ at a geometric rate, \viz
\begin{equation*}
\tag*{\endenv}
\breg(\sol,\curr)
	= \tfrac{1}{2} \curr^{2}
	= \Theta\parens[\big]{(1-\step)^{2\run}}
	\;\;
	\text{or, in absolute value,}
	\;\;
\abs{\curr - \sol}
	= \Theta\parens{(1-\step)^{\run}}.
\end{equation*}
\end{example}

%% Euclidean ends here
%----------------------------------------------------------------------


%----------------------------------------------------------------------
%% Entropy begins here

\begin{example}
[Entropic regularization]
\label{ex:ent}
Another popular choice when $\points=[0,\infty)$ is the entropic regularizer $\hreg(\point) = \point\log\point$ \cite{BecTeb03,SS11,BBT17}.
In this case, we have $\hreg'(\point) = 1 + \log\point$, and hence:
\begin{flalign}
\label{eq:mirror-ent}
\begin{alignedat}{3}
\quad
	a)\;\;
	&\text{Prox-domain:}
	&\qquad
&\proxdom
	= \relint\points
	= (0,\infty)
%	\hspace{15em}
	&\\
\quad
	b)\;\;
	&\text{Bregman divergence:}
	&\qquad
&\breg(\base,\point)
	= \base \log(\base/\point) + \point - \base
	&\\
\quad
	c)\;\;
	&\text{Prox-mapping:}
	&\qquad
&\proxof{\point}{\dvec}
	= \point \exp(\dvec).
	&
\end{alignedat}
&&
\end{flalign}
Now, taking $\vecfield(\point) = \point$ as in the previous example, the update rule \eqref{eq:MD-generic} becomes
\begin{equation}
\label{eq:MD-ent}
\fixmap(\point)
	= \point \exp(-\step\point)
	= \point(1 - \step\point  + o(\point))
	= \point - \step\point^2  + o(\point^2)
	\quad
	\text{as $\point\to0$}.
\end{equation}
In contrast to \eqref{eq:MD-Eucl}, we now have $\fixmap(\point) \sim \point$ instead of $(1-\step)\point$, so $\fixmap$ is no longer a contraction.
Instead, the iterates of \eqref{eq:MD-ent} may be analyzed by means of the following lemma:
\begin{restatable}{lemma}{basicnum}
\label{lem:basicnum}
Suppose that $\fn\from\R_+\to\R_+$ admits the asymptotic expansion
\begin{equation}
\fn(\point)
	= \point
		- \coef\point^{1+\rexp}
		+ o(\point^{1+\rexp})
	\quad
	\text{as $\point\to0$}
\end{equation}
for positive constants $\coef,\rexp>0$.
Then, for $\init[\seq] > 0$ small enough, the sequence $\next[\seq] = \fn(\curr[\seq])$, $\run=\running$, converges to $0$ at a rate of $\curr[\seq] \sim (\coef\rexp\run)^{-1/\rexp}$.
\end{restatable}


Thanks to this lemma (which we prove \refapp{app:aux}), we readily conclude that $\curr$ converges to $0$ at a rate of
$\breg(\sol,\curr)
	= \curr
	= \abs{\curr - \sol}
	\sim 1/(\step\run).$
\hfill
\endenv
\end{example}

%% Entropy ends here
%----------------------------------------------------------------------


%----------------------------------------------------------------------
%% Fractional begins here

\begin{example}
[Fractional power]
\label{ex:frac}
Take $\points = [0,\infty)$ and $\vecfield(\point) = \point$ as in \cref{ex:Eucl,ex:ent} above.
Then, for a given $\qexp>0$, $\qexp\neq1$, the \emph{fractional power} regularizer \textendash\ or \emph{Tsallis entropy} \textendash\ on $\points$ is defined as $\hreg(\point) = [\qexp(1-\qexp)]^{-1} (\point - \point^{\qexp})$ \citep{Tsa88,ABB04,MS16}.
For this choice of regularizer, we have $\hreg'(\point) = (1 - \qexp\point^{\qexp-1}) / [\qexp(1-\qexp)]$, and a series of direct calculations gives:%
\footnote{Strictly speaking, the expression we provide for $\proxof{\point}{\dvec}$ is only valid when $\dvec < \point^{\qexp-1}/(1-\qexp)$.
\revise{The reason for this is that the} prox-mapping $\proxof{\point}{\dvec}$ is not well-defined for all values of $\dvec$;
this detail is not important in the calculations that follow, so we disregard it for now.}
\begin{flalign}
\label{eq:mirror-frac}
\begin{alignedat}{3}
\quad
	a)\;\;
	&\text{Prox-domain:}
	&\qquad
&\text{$\proxdom = (0,\infty)$ if $\qexp\in(0,1)$ and $\proxdom = [0,\infty)$ if $\qexp > 1$}
%	\hspace{3em}
	&\\
\quad
	b)\;\;
	&\text{Bregman divergence:}
	&\qquad
&\breg(\base,\point)
	= \frac{\point^{\qexp} - \base^{\qexp}}{\qexp(1-\qexp)}
		- \point^{\qexp-1} \frac{\point - \base}{1-\qexp}
	&\\
\quad
	c)\;\;
	&\text{Prox-mapping:}
	&\qquad
&\proxof{\point}{\dvec}
	= \bracks[\big]{\point^{\qexp-1} - (1-\qexp) \dvec}^{\frac{1}{\qexp-1}}
	\quad
	\text{for $\qexp\in(0,1)$}.
	&
\end{alignedat}
&&
\end{flalign}
Now, when applied to $\vecfield(\point) = \point$, the fractional power variant of \eqref{eq:MD-generic} for $\qexp\in(0,1)$ gives
\begin{equation}
\label{eq:MD-frac}
\fixmap(\point)
	= \point \, \bracks{1 + \step(1-\qexp)\point^{2-\qexp}}^{1/(\qexp-1)}
	= \point - \step\point^{3-\qexp} + o(\point^{3-\qexp})
	\quad
	\text{as $\point\to0$}.
\end{equation}
Hence, by \cref{lem:basicnum}, we conclude that $\curr$ converges to $0$ at a rate of
\begin{equation}
\tag*{\endenv}
\breg(\sol,\curr)
	= \Theta\parens[\big]{\run^{-\qexp/(2-\qexp)}}
	\;\;
	\text{or, in absolute value,}
	\;\;
\abs{\curr - \sol}
	= \Theta\parens[\big]{\run^{-1/(2-\qexp)}}.
\end{equation}
\end{example}

%% Fractional ends here
%----------------------------------------------------------------------


%----------------------------------------------------------------------
%% Hellinger begins here

\begin{example}
[Hellinger distance]
\label{ex:Hell}
Our last example concerns the Hellinger regularizer $\hreg(\point) = -\sqrt{1-\point^{2}}$ on $\points = [-1,1]$.
Since $\hreg'(\point) = \point / \sqrt{1-\point^{2}}$, we readily obtain the following:
\begin{flalign}
\label{eq:mirror-Hell}
\begin{alignedat}{3}
\quad
	a)\;\;
	&\text{Prox-domain:}
	&\qquad
&\proxdom = \relint\points = (-1,1)
%	\hspace{15em}
	&\\
\quad
	b)\;\;
	&\text{Bregman divergence:}
	&\qquad
&\breg(\base,\point)
	= \frac{1 - \base\point - \sqrt{(1-\base^{2})(1-\point^{2})}}{\sqrt{1-\point^{2}}}
	&\\
\quad
	c)\;\;
	&\text{Prox-mapping:}
	&\qquad
&\proxof{\point}{\dvec}
	= \frac{\point + \dvec\sqrt{1-\point^{2}}}{\sqrt{1-\point^{2} + (\point + \dvec\sqrt{1-\point^{2}})^{2}}}.
	&
\end{alignedat}
&&
\end{flalign}
In this case, taking $\vecfield(\point) = \point$ as per the previous examples, yields
\begin{equation}
\fixmap(\point)
	= \frac{\point - \step\point\sqrt{1-\point^{2}}}{\sqrt{1-\point^{2} + (\point - \step\point\sqrt{1-\point^{2}})^{2}}}
	\sim \point - \step\point
	\quad
	\text{as $\point\to0$},
\end{equation}
\ie $\curr$ converges to $\sol = 0$ at a geometric rate, as in \cref{ex:Eucl}.
On the other hand, if we consider the shifted operator $\vecfield(\point) = \point+1$, a somewhat tedious calculation (which we detail \revise{\refapp{app:ex}}) gives the following Taylor expansion near $\sol = -1$:
\begin{equation}
\fixmap(\point)
	= \sol
		+ (\point - \sol)
		- 2\sqrt{2}\step (\point - \sol)^{5/2} + o\parens*{(\point-\sol)^{5/2}}.
\end{equation}
Hence, by \cref{lem:basicnum}, we conclude that $\curr$ converges to $\sol = -1$ at a rate of
\begin{equation}
\tag*{\endenv}
\breg(\sol,\curr)
	= \Theta\parens{\run^{-1/3}}
	\;\;
	\text{or, in absolute value,}
	\;\;
\abs{\curr - \sol}
	= \Theta(\run^{-2/3}).
\end{equation}
\end{example}

%% Hellinger ends here
%----------------------------------------------------------------------


%----------------------------------------------------------------------
%% Rate figure begins here

\begin{figure}
\centering
\begin{subfigure}[b]{.45\linewidth}
\resizebox{\textwidth}{!}{%----------------------------------------------------------------------
%%% EXAMPLES
%----------------------------------------------------------------------
% !TEX root = ../Main.tex


We now proceed to take a closer look at some commonly used Bregman regularizers (and the induced prox-mappings) with the goal of determining the rate of convergence of the associated Bregman method.
For concreteness, we focus on one-dimensional problems where $\points$ is \revise{the closed interval $[0,\infty)$ or $[-1, 1]$} and $\vecfield$ is the affine vector field
\begin{equation}
\label{eq:simple}
\vecfield(\point)
	= \point - \sol,
	\quad
	\point\in\R,
\end{equation}
for different choices of $\sol\in\R$
\revise{(typically a boundary point of $\points$)}.
To streamline our presentation, we will only examine the \acl{MD} recursion \eqref{eq:MD} with constant step-size schedules $\curr[\step] \equiv \step$ for some $\step>0$.
In this case, we obtain the scheme
\begin{equation}
\label{eq:MD-generic}
\next
	= \fixmap(\curr)
	\quad
	\text{with}
	\quad
\fixmap(\point)
	= \proxof{\point}{-\step\vecfield(\point)},
\end{equation}
and we will examine the convergence speed of $\curr$ by analyzing the behavior of $\fixmap$ near $\sol$.
\revise{To illustrate the spectrum of different behaviors that arise near the boundary of $\points$, we will focus primarily on cases where $\sol$ is a boundary point.}
\smallskip

%----------------------------------------------------------------------
%% Euclidean begins here

\begin{example}
[Euclidean regularization]
\label{ex:Eucl}
We begin with the quadratic regularizer $\hreg(\point) = \point^{2}/2$ for $\point\in\points = [0,\infty)$.
In this case, noting that $\hreg'(\point) = \point$, we have:
\begin{flalign}
\label{eq:mirror-Eucl}
\begin{alignedat}{3}
\quad
	a)\;\;
	&\text{Prox-domain:}
	&\qquad
&\proxdom
	= \points
%	\hspace{20em}
	&\\
\quad
	b)\;\;
	&\text{Bregman divergence:}
	&\qquad
&\breg(\base,\point)
	= (\base-\point)^{2}/2
	&\\
\quad
	c)\;\;
	&\text{Prox-mapping:}
	&\qquad
&\proxof{\point}{\dvec}
	= \pospart{\point+\dvec}
	&
\end{alignedat}
&&
\end{flalign}
Consider now the case $\sol=0$, \ie $\vecfield(\point) = \point$.
Then, for $\step\in(0,1)$, the update \eqref{eq:MD-generic} becomes
\begin{equation}
\label{eq:MD-Eucl}
\fixmap(\point)
	= \point - \step\point
	= (1-\step) \point
    \quad
    \text{\revise{for all $\point\geq0$}}
\end{equation}
\ie $\fixmap$ is contracting.
We thus conclude that $\curr$ converges to $\sol=0$ at a geometric rate, \viz
\begin{equation*}
\tag*{\endenv}
\breg(\sol,\curr)
	= \tfrac{1}{2} \curr^{2}
	= \Theta\parens[\big]{(1-\step)^{2\run}}
	\;\;
	\text{or, in absolute value,}
	\;\;
\abs{\curr - \sol}
	= \Theta\parens{(1-\step)^{\run}}.
\end{equation*}
\end{example}

%% Euclidean ends here
%----------------------------------------------------------------------


%----------------------------------------------------------------------
%% Entropy begins here

\begin{example}
[Entropic regularization]
\label{ex:ent}
Another popular choice when $\points=[0,\infty)$ is the entropic regularizer $\hreg(\point) = \point\log\point$ \cite{BecTeb03,SS11,BBT17}.
In this case, we have $\hreg'(\point) = 1 + \log\point$, and hence:
\begin{flalign}
\label{eq:mirror-ent}
\begin{alignedat}{3}
\quad
	a)\;\;
	&\text{Prox-domain:}
	&\qquad
&\proxdom
	= \relint\points
	= (0,\infty)
%	\hspace{15em}
	&\\
\quad
	b)\;\;
	&\text{Bregman divergence:}
	&\qquad
&\breg(\base,\point)
	= \base \log(\base/\point) + \point - \base
	&\\
\quad
	c)\;\;
	&\text{Prox-mapping:}
	&\qquad
&\proxof{\point}{\dvec}
	= \point \exp(\dvec).
	&
\end{alignedat}
&&
\end{flalign}
Now, taking $\vecfield(\point) = \point$ as in the previous example, the update rule \eqref{eq:MD-generic} becomes
\begin{equation}
\label{eq:MD-ent}
\fixmap(\point)
	= \point \exp(-\step\point)
	= \point(1 - \step\point  + o(\point))
	= \point - \step\point^2  + o(\point^2)
	\quad
	\text{as $\point\to0$}.
\end{equation}
In contrast to \eqref{eq:MD-Eucl}, we now have $\fixmap(\point) \sim \point$ instead of $(1-\step)\point$, so $\fixmap$ is no longer a contraction.
Instead, the iterates of \eqref{eq:MD-ent} may be analyzed by means of the following lemma:
\begin{restatable}{lemma}{basicnum}
\label{lem:basicnum}
Suppose that $\fn\from\R_+\to\R_+$ admits the asymptotic expansion
\begin{equation}
\fn(\point)
	= \point
		- \coef\point^{1+\rexp}
		+ o(\point^{1+\rexp})
	\quad
	\text{as $\point\to0$}
\end{equation}
for positive constants $\coef,\rexp>0$.
Then, for $\init[\seq] > 0$ small enough, the sequence $\next[\seq] = \fn(\curr[\seq])$, $\run=\running$, converges to $0$ at a rate of $\curr[\seq] \sim (\coef\rexp\run)^{-1/\rexp}$.
\end{restatable}


Thanks to this lemma (which we prove \refapp{app:aux}), we readily conclude that $\curr$ converges to $0$ at a rate of
$\breg(\sol,\curr)
	= \curr
	= \abs{\curr - \sol}
	\sim 1/(\step\run).$
\hfill
\endenv
\end{example}

%% Entropy ends here
%----------------------------------------------------------------------


%----------------------------------------------------------------------
%% Fractional begins here

\begin{example}
[Fractional power]
\label{ex:frac}
Take $\points = [0,\infty)$ and $\vecfield(\point) = \point$ as in \cref{ex:Eucl,ex:ent} above.
Then, for a given $\qexp>0$, $\qexp\neq1$, the \emph{fractional power} regularizer \textendash\ or \emph{Tsallis entropy} \textendash\ on $\points$ is defined as $\hreg(\point) = [\qexp(1-\qexp)]^{-1} (\point - \point^{\qexp})$ \citep{Tsa88,ABB04,MS16}.
For this choice of regularizer, we have $\hreg'(\point) = (1 - \qexp\point^{\qexp-1}) / [\qexp(1-\qexp)]$, and a series of direct calculations gives:%
\footnote{Strictly speaking, the expression we provide for $\proxof{\point}{\dvec}$ is only valid when $\dvec < \point^{\qexp-1}/(1-\qexp)$.
\revise{The reason for this is that the} prox-mapping $\proxof{\point}{\dvec}$ is not well-defined for all values of $\dvec$;
this detail is not important in the calculations that follow, so we disregard it for now.}
\begin{flalign}
\label{eq:mirror-frac}
\begin{alignedat}{3}
\quad
	a)\;\;
	&\text{Prox-domain:}
	&\qquad
&\text{$\proxdom = (0,\infty)$ if $\qexp\in(0,1)$ and $\proxdom = [0,\infty)$ if $\qexp > 1$}
%	\hspace{3em}
	&\\
\quad
	b)\;\;
	&\text{Bregman divergence:}
	&\qquad
&\breg(\base,\point)
	= \frac{\point^{\qexp} - \base^{\qexp}}{\qexp(1-\qexp)}
		- \point^{\qexp-1} \frac{\point - \base}{1-\qexp}
	&\\
\quad
	c)\;\;
	&\text{Prox-mapping:}
	&\qquad
&\proxof{\point}{\dvec}
	= \bracks[\big]{\point^{\qexp-1} - (1-\qexp) \dvec}^{\frac{1}{\qexp-1}}
	\quad
	\text{for $\qexp\in(0,1)$}.
	&
\end{alignedat}
&&
\end{flalign}
Now, when applied to $\vecfield(\point) = \point$, the fractional power variant of \eqref{eq:MD-generic} for $\qexp\in(0,1)$ gives
\begin{equation}
\label{eq:MD-frac}
\fixmap(\point)
	= \point \, \bracks{1 + \step(1-\qexp)\point^{2-\qexp}}^{1/(\qexp-1)}
	= \point - \step\point^{3-\qexp} + o(\point^{3-\qexp})
	\quad
	\text{as $\point\to0$}.
\end{equation}
Hence, by \cref{lem:basicnum}, we conclude that $\curr$ converges to $0$ at a rate of
\begin{equation}
\tag*{\endenv}
\breg(\sol,\curr)
	= \Theta\parens[\big]{\run^{-\qexp/(2-\qexp)}}
	\;\;
	\text{or, in absolute value,}
	\;\;
\abs{\curr - \sol}
	= \Theta\parens[\big]{\run^{-1/(2-\qexp)}}.
\end{equation}
\end{example}

%% Fractional ends here
%----------------------------------------------------------------------


%----------------------------------------------------------------------
%% Hellinger begins here

\begin{example}
[Hellinger distance]
\label{ex:Hell}
Our last example concerns the Hellinger regularizer $\hreg(\point) = -\sqrt{1-\point^{2}}$ on $\points = [-1,1]$.
Since $\hreg'(\point) = \point / \sqrt{1-\point^{2}}$, we readily obtain the following:
\begin{flalign}
\label{eq:mirror-Hell}
\begin{alignedat}{3}
\quad
	a)\;\;
	&\text{Prox-domain:}
	&\qquad
&\proxdom = \relint\points = (-1,1)
%	\hspace{15em}
	&\\
\quad
	b)\;\;
	&\text{Bregman divergence:}
	&\qquad
&\breg(\base,\point)
	= \frac{1 - \base\point - \sqrt{(1-\base^{2})(1-\point^{2})}}{\sqrt{1-\point^{2}}}
	&\\
\quad
	c)\;\;
	&\text{Prox-mapping:}
	&\qquad
&\proxof{\point}{\dvec}
	= \frac{\point + \dvec\sqrt{1-\point^{2}}}{\sqrt{1-\point^{2} + (\point + \dvec\sqrt{1-\point^{2}})^{2}}}.
	&
\end{alignedat}
&&
\end{flalign}
In this case, taking $\vecfield(\point) = \point$ as per the previous examples, yields
\begin{equation}
\fixmap(\point)
	= \frac{\point - \step\point\sqrt{1-\point^{2}}}{\sqrt{1-\point^{2} + (\point - \step\point\sqrt{1-\point^{2}})^{2}}}
	\sim \point - \step\point
	\quad
	\text{as $\point\to0$},
\end{equation}
\ie $\curr$ converges to $\sol = 0$ at a geometric rate, as in \cref{ex:Eucl}.
On the other hand, if we consider the shifted operator $\vecfield(\point) = \point+1$, a somewhat tedious calculation (which we detail \revise{\refapp{app:ex}}) gives the following Taylor expansion near $\sol = -1$:
\begin{equation}
\fixmap(\point)
	= \sol
		+ (\point - \sol)
		- 2\sqrt{2}\step (\point - \sol)^{5/2} + o\parens*{(\point-\sol)^{5/2}}.
\end{equation}
Hence, by \cref{lem:basicnum}, we conclude that $\curr$ converges to $\sol = -1$ at a rate of
\begin{equation}
\tag*{\endenv}
\breg(\sol,\curr)
	= \Theta\parens{\run^{-1/3}}
	\;\;
	\text{or, in absolute value,}
	\;\;
\abs{\curr - \sol}
	= \Theta(\run^{-2/3}).
\end{equation}
\end{example}

%% Hellinger ends here
%----------------------------------------------------------------------


%----------------------------------------------------------------------
%% Rate figure begins here

\begin{figure}
\centering
\begin{subfigure}[b]{.45\linewidth}
\resizebox{\textwidth}{!}{\input{Figures/Examples.tikz}}
\end{subfigure}
\hfill
\begin{subfigure}[b]{.45\linewidth}
\resizebox{\textwidth}{!}{\input{Figures/Examples_log.tikz}}
\end{subfigure}
\caption{The rate of convergence of \eqref{eq:MD} in \crefrange{ex:Eucl}{ex:Hell}.
The Euclidean and shifted Hellinger regularizers lead to a geometric rate %of convergence 
(see left figure);
all other examples converge at a polynomial rate.}
\label{fig:examples}
\end{figure}

%% Rate figure ends here
%----------------------------------------------------------------------


Albeit one-dimensional, the above examples provide a representative view of the geometry of Bregman proximal methods near a solution.
Specifically, they show that the divergence induced by a given regularizer may exhibit a very different behavior at the boundary of\;$\points$:
when $\sol$ is a boundary point, $\breg(\sol,\point)$ grows
as $\Theta(\norm{\point - \sol}^{2})$ in the Euclidean case,
as $\Theta(\norm{\point - \sol})$ for the negative entropy,
and, more generally,
as $\Theta(\norm{\point-\sol}^{\qexp})$ for the $\qexp$-th power regularizer.
As a result, when used as a measure of convergence, it is important to rescale the Bregman  divergence in order to avoid inflating \textendash\ or \emph{deflating} \textendash\ an algorithm's rate of convergence. 

Nonetheless, even if we take this rescaling into account, different instances of \eqref{eq:MD} may lead to completely different rates of convergence.
Specifically, in terms of absolute values (or norms), we observe a
geometric rate in the Euclidean and shifted Hellinger cases,
a rate of $\Theta(1/\run)$ for the negative entropy,
and
a rate of $\Theta(1/\run^{1/(2-\qexp)})$ for the $\qexp$-th power regularizer (\cf \cref{fig:examples} above).
This is due to the different first-order behavior of the iterative update map $\point\gets\fixmap(\point)$ that underlies \eqref{eq:MD}, which is itself intimately related to the growth rate of the Bregman divergence near a solution $\sol$ of \eqref{eq:VI}.
We make this relation precise in the next section.}
\end{subfigure}
\hfill
\begin{subfigure}[b]{.45\linewidth}
\resizebox{\textwidth}{!}{\input{Figures/Examples_log.tikz}}
\end{subfigure}
\caption{The rate of convergence of \eqref{eq:MD} in \crefrange{ex:Eucl}{ex:Hell}.
The Euclidean and shifted Hellinger regularizers lead to a geometric rate %of convergence 
(see left figure);
all other examples converge at a polynomial rate.}
\label{fig:examples}
\end{figure}

%% Rate figure ends here
%----------------------------------------------------------------------


Albeit one-dimensional, the above examples provide a representative view of the geometry of Bregman proximal methods near a solution.
Specifically, they show that the divergence induced by a given regularizer may exhibit a very different behavior at the boundary of\;$\points$:
when $\sol$ is a boundary point, $\breg(\sol,\point)$ grows
as $\Theta(\norm{\point - \sol}^{2})$ in the Euclidean case,
as $\Theta(\norm{\point - \sol})$ for the negative entropy,
and, more generally,
as $\Theta(\norm{\point-\sol}^{\qexp})$ for the $\qexp$-th power regularizer.
As a result, when used as a measure of convergence, it is important to rescale the Bregman  divergence in order to avoid inflating \textendash\ or \emph{deflating} \textendash\ an algorithm's rate of convergence. 

Nonetheless, even if we take this rescaling into account, different instances of \eqref{eq:MD} may lead to completely different rates of convergence.
Specifically, in terms of absolute values (or norms), we observe a
geometric rate in the Euclidean and shifted Hellinger cases,
a rate of $\Theta(1/\run)$ for the negative entropy,
and
a rate of $\Theta(1/\run^{1/(2-\qexp)})$ for the $\qexp$-th power regularizer (\cf \cref{fig:examples} above).
This is due to the different first-order behavior of the iterative update map $\point\gets\fixmap(\point)$ that underlies \eqref{eq:MD}, which is itself intimately related to the growth rate of the Bregman divergence near a solution $\sol$ of \eqref{eq:VI}.
We make this relation precise in the next section.}
\end{subfigure}
\hfill
\begin{subfigure}[b]{.45\linewidth}
\resizebox{\textwidth}{!}{\input{Figures/Examples_log.tikz}}
\end{subfigure}
\caption{The rate of convergence of \eqref{eq:MD} in \crefrange{ex:Eucl}{ex:Hell}.
The Euclidean and shifted Hellinger regularizers lead to a geometric rate %of convergence 
(see left figure);
all other examples converge at a polynomial rate.}
\label{fig:examples}
\end{figure}

%% Rate figure ends here
%----------------------------------------------------------------------


Albeit one-dimensional, the above examples provide a representative view of the geometry of Bregman proximal methods near a solution.
Specifically, they show that the divergence induced by a given regularizer may exhibit a very different behavior at the boundary of\;$\points$:
when $\sol$ is a boundary point, $\breg(\sol,\point)$ grows
as $\Theta(\norm{\point - \sol}^{2})$ in the Euclidean case,
as $\Theta(\norm{\point - \sol})$ for the negative entropy,
and, more generally,
as $\Theta(\norm{\point-\sol}^{\qexp})$ for the $\qexp$-th power regularizer.
As a result, when used as a measure of convergence, it is important to rescale the Bregman  divergence in order to avoid inflating \textendash\ or \emph{deflating} \textendash\ an algorithm's rate of convergence. 

Nonetheless, even if we take this rescaling into account, different instances of \eqref{eq:MD} may lead to completely different rates of convergence.
Specifically, in terms of absolute values (or norms), we observe a
geometric rate in the Euclidean and shifted Hellinger cases,
a rate of $\Theta(1/\run)$ for the negative entropy,
and
a rate of $\Theta(1/\run^{1/(2-\qexp)})$ for the $\qexp$-th power regularizer (\cf \cref{fig:examples} above).
This is due to the different first-order behavior of the iterative update map $\point\gets\fixmap(\point)$ that underlies \eqref{eq:MD}, which is itself intimately related to the growth rate of the Bregman divergence near a solution $\sol$ of \eqref{eq:VI}.
We make this relation precise in the next section.


%----------------------------------------------------------------------
%%% GENERAL
%----------------------------------------------------------------------
\section{The Legendre exponent and convergence rate analysis}
\label{sec:general}
%----------------------------------------------------------------------
%%% GENERAL
%----------------------------------------------------------------------
% !TEX root = ../Main.tex


Our goal in this section is to provide a precise link between the geometry induced by a Bregman regularizer near a solution and the convergence rate of the associated Bregman proximal method.
The key notion in this regard is that of the \emph{Legendre exponent}, which we define and discuss in detail below.


%----------------------------------------------------------------------
%%% Legendre
%----------------------------------------------------------------------
\subsection{The Legendre exponent}
\label{sec:Legendre}

Our starting point is the observation that, without loss of generality, the local strong convexity requirement for $\hreg$ can be expressed as
\begin{equation}
\label{eq:Breg-lower}
\breg(\base,\point)
	\geq \tfrac{1}{2} \norm{\base - \point}^{2}
	\quad
	\text{\revise{for all $\point\in\proxdom$ sufficiently close to $\base$}}.
\end{equation}
Qualitatively, this means that the convergence topology induced by the Bregman divergence of $\hreg$ on $\points$ is \emph{at least as fine} as the ambient norm topology:
if a sequence $\curr[\point]\in\proxdom$, $\run=\running$, converges to $\base\in\points$ in the Bregman sense ($\breg(\base,\curr[\point]) \to 0$), it also converges in the ambient norm topology ($\norm{\curr[\point] - \base}\to0$).
On the other hand, from a quantitative standpoint, the rate of this convergence could be quite different:
as we saw in the previous section, the reverse inequality $\breg(\base,\point) = \bigoh(\norm{\base-\point}^{2})$ may fail to hold, in which case $\sqrt{\breg(\base,\curr[\point])}$ and $\norm{\point - \curr[\point]}$ would exhibit a different asymptotic behavior.

To quantify this gap, we use the notion of the \emph{Legendre exponent}, as introduced in \cite{AIMM21}.

\begin{definition}
\label{def:Legendre}
Let $\hreg$ be a Bregman regularizer on $\points$.
The \emph{Legendre exponent} of $\hreg$ at $\base\in\points$ is defined as
\begin{equation}
\label{eq:Legendre}
\legof{\base}
	\defeq \inf\setdef*{\legexp\in[0,1]}{\limsup_{\point\to\base} \frac{\sqrt{\breg(\base,\point)}}{\norm{\point-\base}^{1-\legexp}} < \infty}
\end{equation}
and we say that $\hreg$ is \emph{tight} at $\base$ if the infimum is attained in \eqref{eq:Legendre}, \ie if $\legof{\base}$ is the minimal $\legexp\in[0,1]$ such that
\begin{equation}
\label{eq:Breg-local}
\breg(\base,\point)
	= \bigof[\big]{\norm{\base - \point}^{2(1-\legexp)}}
	\quad
	\text{for $\point$ near $\base$}.
\end{equation}
\end{definition}

Informally, the Legendre exponent measures the deficit in relative size between ordinary ``norm neighborhoods'' in $\points$ and the corresponding ``Bregman neighborhoods'' induced by the sublevel sets of the Bregman divergence.
Specifically,
\begin{enumerate*}
[(\itshape i\hspace*{.5pt}\upshape)]
\item
the case $\legof{\base} = 0$ corresponds to the ``norm-like'' behavior $\breg(\base,\point) = \Theta(\norm{\base-\point}^{2})$;
\item
any other value $\legof{\base} \in (0,1)$ indicates a different limiting behavior for $\breg(\base,\point)$ as $\point\to\base$;
and, finally,
\item
when $\legof{\base} = 1$ we may have $\limsup_{\point\to\base} \breg(\base,\point) > 0$.
\end{enumerate*}
In this last case, the ambient norm topology is \emph{strictly coarser} than the Bregman topology in the sense that $\breg(\base,\curr)$ may remain bounded away from zero even if $\curr\to\base$ as $\run\to\infty$;
we provide an example of such behavior below \textendash\ and see also \cite{AIMM21,pauwels2023nature} for further discussion.

\begin{example}
[Non-compatible topologies]
Let $\points = \setdef{\point\in\R^{\nCoords}}{\twonorm{\point} \leq 1}$ be the unit Euclidean ball in $\R^{\nCoords}$ and consider the $\nCoords$-dimensional Hellinger regularizer $\hreg(\point) = -\sqrt{1 - \twonorm{\point}^{2}}$.
Then, for all $\base$ on the boundary of $\points$ and all $\point \in \proxdom = \intr\points$, we readily get
\begin{equation}
\label{eq:Breg-Hellinger}
\breg(\base,\point)
	= \frac{1 - \braket{\base}{\point}}{\sqrt{1 - \twonorm{\point}^{2}}}.
\end{equation}
If $\nCoords\geq2$, the limit $\lim_{\point\to\base} \breg(\base,\point)$ may not exist,
a fact which has the following counterintuitive consequences:
\begin{enumerate*}
[(\itshape i\hspace*{1pt}\upshape)]
\item
the ``Hellinger ball'' $\ball_{\radius}^{\hreg}(\base) \defeq \setdef{\point\in\proxdom}{\breg(\base,\point) \leq \radius^{2}/2}$ is \emph{not closed} in the Euclidean topology;
and
\item
the ``Hellinger center'' $\base$ of $\ball_{\radius}^{\hreg}(\base)$ actually belongs to the Euclidean boundary of $\ball_{\radius}^{\hreg}(\base)$.
\end{enumerate*}
As a result, for all $\nCoords\geq2$, it is straightforward to construct a sequence $\curr[\point]$ with $\twonorm{\curr[\point] - \base} \to 0$, but which remains at \emph{constant} Hellinger divergence relative to $\base$.
\footnote{For instance, if $\nCoords = 2$, the point $\point_{u} = (1-u,\sqrt{2u(1-u)})$ converges to $\base = (1,0)$ as $u\to0^{+}$, even though $\breg\left(\base,\point_{u}\right) = 1$ for all $u\in(0,1)$.
Crucially, if $\nCoords=1$, this phenomenon does not occur, \cf \cref{ex:Hell}.}
\hfill
\endenv
\end{example}
%----------------------------------------------------------------------


For illustration purposes, we compute below the Legendre exponent for each of the running examples of \cref{sec:examples} (see also \cref{tab:rates}):
\begin{enumerate}
\item
\emph{Quadratic regularization} (\cref{ex:Eucl}):
Since $\breg(\base,\point) = \parens{\base - \point}^{2}/2$ for all $\base,\point\in\points$, we have $\legof{\base} = 0$ for all $\base\in\points$.

\item
\emph{Negative entropy} (\cref{ex:ent}):
For $\base=0$, \cref{eq:mirror-ent} gives $\breg(0,\point) = \point$, so $\legof{0} = 1/2$.
Otherwise, for all $\base \in \proxdom = (0,\infty)$, a Taylor expansion with Lagrange remainder yields $\breg(\base,\point) = \bigoh(\parens{\base - \point}^{2})$, so $\legof{\base} = 0$ for all $\base\in(0,\infty)$.

\item
\emph{Tsallis entropy} (\cref{ex:frac}):
For $\base=0$, \cref{eq:mirror-frac} gives $\breg(0,\point) = \point^{\qexp}/\qexp$, so $\legof{0} = \max\{0,1-\qexp/2\}$.
Otherwise, for all $\base \in \proxdom = (0,\infty)$, a Taylor expansion yields $\breg(\base,\point) = \bigoh(\parens{\base - \point}^{2})$, so $\legof{\base} = 0$ in this case.

\item
\emph{Hellinger regularizer} (\cref{ex:Hell}):
For $\base = \pm1$, \cref{eq:mirror-Hell} readily gives $\breg(\pm1,\point) = \sqrt{(1 \mp \point)/(1 \pm \point)} = \Theta\parens{\abs{\point\mp1}^{1/2}}$, so $\legof{\pm1} = 1-1/4 = 3/4$.
Instead, if $\base\in(-1,1)$ a Taylor expansion again yields $\breg(\base,\point) = \bigoh(\parens{\base-\point}^{2})$, so $\legof{\base} = 0$ in this case.
\end{enumerate}
\smallskip

A common pattern that emerges above is that $\legof{\base}=0$ whenever $\base$ is an interior point.
We make this observation precise in \refinapp{lem:Leg-proxdom}{app:aux}, where we show more generally that $\legof{\base} = 0$ whenever $\nabla\hreg$ is (locally) Lipschitz continuous in a neighborhood of $\base$ in $\points$.%


%----------------------------------------------------------------------
%%% Legendre
%----------------------------------------------------------------------
\subsection{Convergence rate analysis}
\label{sec:rate-general}

We are now in a position to state our first general result for the convergence rate of \eqref{eq:BPM}.
To do so, we will make the blanket assumption that $\hreg$ is tight at $\sol$ with Legendre exponent $\legsol \defeq \legof{\sol}$.
In particular, this means that there exists a neighborhood $\legnhd$ of $\sol$ in $\points$ and a positive constant $\legconst>0$ such that
\begin{equation}
\label{eq:Breg-upper}
\breg(\sol,\point)
	\leq \frac{\legconst}{2} \norm{\point - \sol}^{2(1-\legsol)}
	\quad
	\text{for all $\point\in\legnhd$}.
\end{equation}
To ligthen notation, we will also assume that $\hreg$ is $1$-strongly convex on $\nhd$ (\cf \cref{rem:Bregman} and the beginning of \cref{sec:Legendre}).
We then have the following result.

\begin{theorem}
\label{thm:general}
Suppose that \cref{asm:Lipschitz,asm:strong,asm:signal-base} hold and \eqref{eq:BPM} is run with a constant step-size $\curr[\step] \equiv \step$, $\run = \running$, such that
\begin{equation}
\label{eq:step}
\step
	\leq \frac{1}{2\gold\lips}
	\quad
	\text{and}
	\quad
\step (1-\coef[a]-\coef[b])^{2}
	\leq \frac{\strong}{8\lips^{2}}
\end{equation}
where $\gold = (\sqrt{5}+1)/2$ is the golden ratio.
If $\init$ is initialized sufficiently close to $\sol$,
the iterates $\curr$ of \eqref{eq:BPM}
enjoys the bound
\begin{equation}
\label{eq:rate}
\breg(\sol,\curr)
	\leq \breg(\sol,\init) \cdot
	\begin{cases*}
		\parens*{1 - \frac{\strong\step}{2\legconst}}^{\run - 1}
			&\quad
			if $\legsol = 0$,
			\\[\medskipamount] 
		\bracks*{1 + \Const \strong \step(\run - 1)}^{1-1/\legsol}
			&\quad
			if $\legsol\in(0,1)$,
	\end{cases*}
\end{equation}
where
%\begin{equation}
\(
\Const
= \expleg
\max\braces[\big]{2\legconst^{\frac{1}{1 - \legsol}}\breg(\sol,\init)^{-\expleg}, 2^{\expleg}}^{-1}.
\)
%\end{equation}
\end{theorem}

Before moving on to the proof of \cref{thm:general}, some remarks and corollaries are in order (see also \cref{tab:rates} for an explicit illustration of the derived rates for \crefrange{ex:Eucl}{ex:Hell}):
\smallskip


%----------------------------------------------------------------------
%% Table of rates begins here

\begin{table}[tbp]
\footnotesize
\centering
\renewcommand{\arraystretch}{1.25}
%----------------------------------------------------------------------
%%% RATES
%----------------------------------------------------------------------
% !TEX root = ../Main.tex


\begin{tabular}{lcccc}
\toprule
	&\textbf{Domain ($\points$)}
	&\textbf{Regularizer ($\hreg$)}
	&\textbf{Legendre Exponent ($\legof{\base}$)}
	&\textbf{Convergence Rate}
	\\
\midrule
\scshape{Euclidean}	
	&arbitrary
	&$\point^{2}/2$
	&$0$
	&Linear %$\exp(-\bigoh(\run))$
	\\
\scshape{Entropic}
	&$[0,\infty)$
	&$\point\log\point$
	&$1/2$
	&$\bigoh(1/\run)$
	\\
\scshape{Tsallis}
	&$[0,\infty)$
%	&$\frac{\point - \point^{\qexp}}{\qexp(1-\qexp)}$
	&$[\qexp(1-\qexp)]^{-1} (\point - \point^{\qexp})$
	&$\max\braces{0,1-\qexp/2}$
	&$\bigoh(1/\run^{\qexp/(2-\qexp)})$
	\\
\scshape{Hellinger}
	&$[-1,1]$
	&$-\sqrt{1-\point^{2}}$
	&$3/4$
	&$\bigoh(1/\run^{1/3})$
	\\
\bottomrule
\end{tabular}

\smallskip
\caption{Summary of the Legendre exponents for the $1$-dimensional examples of \Cref{sec:setup} at a boundary point $\base$ of $\points \subset \R$, and the associated convergence rates in terms of the Bregman divergence $\breg(\sol,\curr)$}.
\label{tab:rates}
\end{table}

%% Table of rates ends here
%----------------------------------------------------------------------


%----------------------------------------------------------------------
\setcounter{remark}{0}
%----------------------------------------------------------------------
\begin{remark}
The first point of note is the sharp drop in the convergence rate of \eqref{eq:BPM} from geometric, when $\legsol=0$, to a power law when $\legsol>0$.
As we saw in \cref{sec:examples}, this drop is unavoidable, even when $\points$ is $1$-dimensional and $\vecfield$ is affine;
in fact, the calculations of \cref{sec:examples} show that the rates provided by \cref{thm:general} are, in general, unimprovable.
\hfill
\endenv
\end{remark}
%----------------------------------------------------------------------


%----------------------------------------------------------------------
\begin{remark}
We should also note that the guarantees of \cref{thm:general} are %all 
stated in terms of the Bregman divergence, not the ambient norm.
Since $\breg(\sol,\curr) = \Omega(\norm{\curr - \sol}^{2})$, these bounds can be restated in terms of $\norm{\curr-\sol}$, but this conversion is not without loss of information:
if the bound $\breg(\sol,\curr) = \Omega(\norm{\curr - \sol}^{2})$ is not tight, the actual rate in terms of the norm may be significantly different.
This phenomenon was already observed in the $1$-dimensional examples of \cref{sec:examples} where $\breg(\sol,\curr) = \Theta(\norm{\curr-\sol}^{2(1-\legsol)})$, in which case \cref{thm:general} gives
\begin{equation}
\norm{\curr-\sol}
	= \bigof[\big]{\run^{-1/(2\legsol)}}
\end{equation}
whenever $\legsol>0$ (see also \cref{tab:rates}).
In general however, the Bregman divergence may grow at different rates along different rays emanating from $\sol$, so it is not always possible to translate a Bregman-based bound to a norm-based bound (or vice versa).
This analysis requires a much closer look at the geometric structure of $\points$, depending on which constraints are active at $\sol$;
we examine this issue at depth in \cref{sec:sharp}.
\hfill
\endenv
\end{remark}
%----------------------------------------------------------------------


%----------------------------------------------------------------------
\begin{remark}
\label{rem:variable}
We should also note that, even though \cref{thm:general} is stated for a constant step-size, our proof allows for a variable step-size $\curr[\step]$, provided that the step-size conditions \eqref{eq:step} are satisfied.
In this case, the bounds \eqref{eq:rate} becomes
\begin{equation*}
%\tag*{\endenv}
% \label{eq:rate-Eucl-var}
\breg(\sol,\curr)
	\leq \breg(\sol,\init)
	\!\cdot\! 
	\begin{cases*}
	\prod_{\runalt=\start}^{\run-1}\parens*{1 - \frac{\strong\iter[\step]}{2\legconst}}
		&\text{if $\legsol=0$},
	\\
	\bracks[\big]{1 + \Const \strong \sum_{\runalt=\start}^{\run-1} \iter[\step]}^{1-1/\legsol}
		&\text{if $\legsol \in (0, 1)$)}.
		\;
	\end{cases*}
\end{equation*}
\end{remark}
%\hfill
%\endenv
%----------------------------------------------------------------------


%----------------------------------------------------------------------
%%% PROOF
%----------------------------------------------------------------------
\subsection{Proof of \cref{thm:general}}
\label{sec:proof-general}

We now proceed to the proof of \cref{thm:general}, beginning with a series of intermediate results tailored to the update structure of \eqref{eq:BPM}.
The first of these lemmas relates the Bregman divergence before and after a prox-step modulo an element of the polar cone $\pcone(\base) \defeq \setdef{\dbase\in\dpoints}{\braket{\dbase}{\point - \base} \leq 0 \; \text{for all $\point\in\points$}}$ of $\points$ at the reference point $\base$.

\begin{lemma}
\label{lem:onestep}
Let $\new = \proxof{\point}{\dvec}$ for some $\point\in \nhd \cap \proxdom$, $\dvec\in\dpoints$, such that $\new\in\nhd$.
Then, for all $\base\in\points$ and all 
$\dbase\in\pcone(\base)$, we have:
\begin{subequations}
\begin{align}
\breg(\base,\new)
	&\leq \breg(\base,\point)
		+ \braket{\dvec - \dbase}{\new - \base}
		- \breg(\new,\point)
		\\
	&\leq \breg(\base,\point)
		+ \braket{\dvec - \dbase}{\point - \base}
		+ \tfrac{1}{2} \dnorm{\dvec - \dbase}^{2}
		\,,
\end{align}
\end{subequations}
\end{lemma}

The next lemma extends \cref{lem:onestep} to emulate the two-step structure of \eqref{eq:BPM}:

\begin{lemma}
\label{lem:twostep}
\revise{Let $\new_{i} = \proxof{\point}{\dvec_{i}}$ for some $\point\in \nhd \cap \proxdom$, $\dvec_{i}\in\dpoints$, such that $\new_i \in \nhd$, $i=1,2$.}
Then, for all $\base\in\points$ and all $\dbase\in\pcone(\base)$, we have:
\begin{equation}
\breg(\base,\new_{2})
	\leq \breg(\base,\point)
		+ \braket{\dvec_{2} - \dbase}{\new_{1} - \base}
		+ \tfrac{1}{2} \dnorm{\dvec_{2} - \dvec_{1} - \dbase}^{2}
		- \tfrac{1}{2} \norm{\new_{1} - \point}^{2}.
\end{equation}
\end{lemma}

Versions of the above inequalities already exist in the literature, see \eg\cite[Lem.~4]{JNT11}, \cite[Prop.~B.4]{MLZF+19}.
The novelty in \cref{lem:onestep,lem:twostep} is the extra term involving the polar vector $\dbase\in\pcone(\base)$;
this term plays an important role in the sequel, so we provide a complete proof \refapp{app:aux}.

With these preliminaries in hand, we proceed to derive two further inequalities that play a pivotal role in the analysis of \eqref{eq:BPM}.
The first is an immediate corollary of \cref{lem:twostep}:

\begin{corollary}
\label{cor:template}
Let $\sol$ be a solution of \eqref{eq:VI}
Then, for all $\coef[c]\geq0$ and all $\run=\running$ \revise{such that $\curr, \lead \in \nhd$} the iterates of \eqref{eq:BPM} satisfy the template inequality
\begin{align}
\label{eq:template}
\breg(\sol,\next)
	\leq \breg(\sol, \curr)
		&- \curr[\step] \braket{\lead[\signal] - \coef[c]\solvec}{\lead - \sol}
	\notag\\
		&+ \tfrac{1}{2} \curr[\step]^{2} \dnorm{\lead[\signal] - \curr[\signal] - \coef[c]\solvec}^{2}
		- \tfrac{1}{2} \norm{\lead - \curr}^{2}.
\end{align}
\end{corollary}

\begin{proof}
Since $\sol$ is a solution of \eqref{eq:VI}, we have $\solvec \in -\pcone(\sol)$.
\cref{eq:template} then follows by invoking \cref{lem:twostep} with $\point \gets \curr$, $\base\gets\sol$, $\dbase \gets - \coef[c]\curr[\step]\solvec \in \pcone(\sol)$ and $(\dvec_{1},\dvec_{2}) \gets (-\curr[\step]\curr[\signal],-\curr[\step]\lead[\signal])$.
\end{proof}

The second inequality that we derive provides an ``energy function'' for \eqref{eq:BPM}, namely
\begin{equation}
\label{eq:energy}
\curr[\energy]
	= \curr[\breg] + \curr[\pot]
\end{equation}
where $\curr[\breg]
	= \breg(\sol,\curr)$
	and
$\curr[\pot] = \prev[\step]^{2} \dnorm{(\coef[a]+\coef[b]) \beforelead[\signal] - \prev[\signal]}^{2}$ (by convention, we take $\init[\pot] = 0$).
The lemma below outlines the Lyapunov properties of $\curr[\energy]$.

\begin{proposition}
\label{prop:energy}
Suppose that \cref{asm:Lipschitz,asm:signal-base} hold and \eqref{eq:BPM} is run with a step-size such that
\begin{equation}
\label{eq:step-energy}
\coef \curr[\step] + 4\curr[\step]^{2}\lips^{2}
	\leq 1
	\quad
	\text{for some $\coef\geq0$ and all $\run=\running$}
\end{equation}
Then the iterates $\curr$ of \eqref{eq:BPM} satisfy \revise{for $\run \geq \start$ such that $\curr,\lead \in \nhd$},
\begin{align}
\label{eq:energy-bound}
%\breg(\sol,\next) + \next[\pot]
%	\leq \breg(\sol,\curr)
%		+ (1 - \coef \curr[\step]) \curr[\pot]
\next[\energy]
	\leq \curr[\energy]
		- \coef\curr[\step] \curr[\pot]
	&- \curr[\step] \braket{\vecfield(\lead) - \solvec}{\lead - \sol}
	\notag\\
	&- \curr[\step] (\coef[a] + \coef[b]) \braket{\solvec}{\lead - \sol}
		- \tfrac{1}{2} \norm{\lead - \curr}^{2}
	\notag\\
	&+ \curr[\step]^{2} (1-\coef[a]-\coef[b])^{2} \lips^{2} \norm{\lead- \sol}^{2}
	\notag\\
	&+ 2\curr[\step]^{2}(\coef[a] + \coef[b])^{2} \lips^{2} \norm{\lead - \curr}^{2}.
\end{align}
\end{proposition}

\begin{proof}
Let $\coef[c] = 1 - \coef[a] - \coef[b]$ so $\coef[c] \geq 0$ by \cref{asm:signal-base}.
\cref{cor:template} then yields
\begin{align}
\label{eq:energy-1}
\next[\breg]
	\leq \curr[\breg]
	&- \curr[\step] \braket{\lead[\signal] - \solvec}{\lead - \sol}
	\notag\\
	&- \curr[\step] (\coef[a] + \coef[b]) \braket{\solvec}{\lead - \sol}
		- \tfrac{1}{2} \norm{\lead - \curr}^{2}
	\notag\\
	&+ \frac{\curr[\step]^{2}}{2} \dnorm{\lead[\signal] - \curr[\signal] - \coef[c]\solvec}^{2}.
\end{align}
Since $\lead[\signal] = (\coef[a] + \coef[b])\lead[\signal] + \coef[c]\lead[\signal]$, the last term above may be bounded as
\begin{align}
\label{eq:energy-2}
\tfrac{1}{2} \curr[\step]^{2} \dnorm{\lead[\signal] - \curr[\signal] - \coef[c]\solvec}^{2}
	&\leq \curr[\step]^{2}\coef[c]^{2} \dnorm{\lead[\signal] - \solvec}^{2}
		+ \curr[\step]^{2}\dnorm{(\coef[a]+\coef[b])\lead[\signal] - \curr[\signal]}^{2}
	\notag\\
	&\leq \curr[\step]^{2}\coef[c]^{2}\lips^{2} \dnorm{\lead- \sol}^{2}
		+ \curr[\step]^{2}\dnorm{(\coef[a]+\coef[b])\lead[\signal] - \curr[\signal]}^{2}
	\notag\\
	&= \curr[\step]^{2} (1-\coef[a]-\coef[b])^{2}\lips^{2} \norm{\lead- \sol}^{2} + \next[\pot]\,,
\end{align}
where we used \cref{asm:Lipschitz} in the second line and the definition \eqref{eq:energy} of $\curr[\pot]$ in the last one.
Thus, combining \cref{eq:energy-1,eq:energy-2} and comparing to \eqref{eq:energy-bound}, it suffices to show that
\begin{equation}
\label{eq:potbound}
2\next[\pot]
	\leq (1-\coef\curr[\step]) \curr[\pot]
		+ 4\curr[\step]^{2}(\coef[a] + \coef[b])^{2} \lips^{2} \norm{\lead - \curr}^{2}
	\quad
	\text{for all $\run=\running$}
\end{equation}
We consider two distinct cases for this below.

\para{Case 1: $\run=\start$}
By the definition \eqref{eq:energy} of $\curr[\pot]$ and \cref{eq:signal-base,eq:signal-lead}, we have:
\begin{align*}
\afterinit[\pot]
	= \init[\step]^{2} \dnorm{(\coef[a]+\coef[b])\signal_{3/2} - \init[\signal]}^{2}
	&= \init[\step]^{2} (\coef[a]+\coef[b])^{2} \dnorm{\vecfield(\state_{3/2}) - \vecfield(\init)}^{2}
	\notag\\
	&\leq \init[\step]^{2} (\coef[a]+\coef[b])^{2} \lips^{2} \norm{\state_{3/2} - \init}^{2},
\end{align*}
where we used the initialization assumption $\init = \state_{1/2}$ in the second equality and the Lipschitz continuity of $\vecfield$ in the last one.
Since $\init[\pot]=0$ by construction, our claim is immediate.

\para{Case 2: $\run>\start$}
By Young's inequality and the Lipschitz continuity of $\vecfield$, we readily obtain
\begin{align*}
\next[\pot]
	&= \curr[\step]^{2} \dnorm{(\coef[a]+\coef[b])\lead[\signal] - \curr[\signal]}^{2}
	\notag\\
	&= \curr[\step]^{2} \dnorm[\big]{
			(\coef[a]+\coef[b]) \bracks{\vecfield(\lead) - \vecfield(\curr)}
			+ \coef[b] \bracks{\vecfield(\curr) - \vecfield(\beforelead)}}^{2}
	\notag\\
	&\leq 2\curr[\step]^{2} (\coef[a] + \coef[b])^{2} \dnorm{\vecfield(\lead) - \vecfield(\curr)}^{2}
		+ 2\curr[\step]^{2} \coef[b]^{2} \dnorm{\vecfield(\curr) - \vecfield(\beforelead)}^{2}
	\notag\\
	&\leq 2\curr[\step]^{2} (\coef[a] + \coef[b])^{2} \lips^{2} \norm{\lead - \curr}^{2}
		+ 2\curr[\step]^{2} \coef[b]^{2} \lips^{2} \norm{\curr - \beforelead}^{2}
	\notag\\
	&\leq 2\curr[\step]^{2} (\coef[a] + \coef[b])^{2} \lips^{2} \norm{\lead - \curr}^{2}
		+ 2\curr[\step]^{2} \coef[b]^{2} \lips^{2} \prev[\step]^{2} \dnorm{\beforelead[\signal] - \prev[\signal]}^{2}
\end{align*}
where, in the last line, we used \cref{lem:proxlip} to bound the difference $\curr - \beforelead$ as
\begin{equation}
\norm{\curr - \beforelead}
	= \norm{\proxof{\prev}{-\prev[\step]\beforelead[\signal]} - \proxof{\prev}{-\prev[\step]\prev[\signal]}}
	\leq \prev[\step] \dnorm{\beforelead[\signal] - \prev[\signal]}.
\end{equation}
Finally, by \cref{asm:signal-base}, we have $\coef[c] = 0$ whenever $\coef[b] > 0$, so
$\coef[b]^{2} \prev[\step]^{2}\dnorm{\beforelead[\signal] - \prev[\signal]}^{2}
	= \coef[b]^{2} \prev[\step]^{2} \dnorm{(1 - \coef[c])\beforelead[\signal] - \prev[\signal]}^{2} = \coef[b]^{2}\curr[\pot]$ for all $\run>\start$.
Hence, putting everything together, we get
\begin{equation}
\next[\pot]
	\leq 2\curr[\step]^{2} (\coef[a] + \coef[b])^{2} \lips^{2} \norm{\lead - \curr}^{2}
		+ 2\curr[\step]^{2} \coef[b]^{2} \lips^{2} \curr[\pot].
\end{equation}
\cref{eq:potbound} then follows by the requirement \eqref{eq:step-energy}, which implies that $2\curr[\step]^{2} \lips^{2} \leq (1 - \coef\curr[\step])/2$.
\end{proof}

Moving forward, since $\sol$ is a solution of \eqref{eq:VI},
the first line of \eqref{eq:energy-bound} yields a negative $\bigoh(\curr[\step])$ contribution to $\curr[\energy]$, 
whereas the third and fourth lines collectively represent a subleading $\bigoh(\curr[\step]^{2})$ ``error term''. 
This decomposition would suffice for the analysis of \eqref{eq:BPM} if the coupling term $\braket{\solvec}{\lead - \sol}$ did not incur an additional $\bigoh(\curr[\step])$ positive contribution to $\next[\energy]$.
This error term is difficult to control but if $\sol$ satisfies \eqref{eq:strong}, we have the following bound.

\begin{lemma}
\label{lem:strong-bound}
Suppose that \cref{asm:strong} holds.
Then, for all $\point\in\points$, $\pointalt\in\basin$ and all $\coef[c]\in[0,1]$, we have:
\begin{equation}
\label{eq:strong-bound}
\braket{\vecfield(\pointalt) - \coef[c] \solvec}{\pointalt - \sol}
	\geq \tfrac{1}{2} \strong \norm{\point - \sol}^{2}
		- \strong \norm{\pointalt - \point}^{2}.
	\end{equation}
\end{lemma}

\begin{proof}
Since $\sol$ is a solution of \eqref{eq:VI} and $\coef[c]\in[0,1]$, we have $(1-\coef[c]) \braket{\solvec}{\pointalt - \sol} \geq 0$ for all $\pointalt\in\points$.
Hence, by \cref{asm:strong}, we get
\begin{equation}
\braket{\vecfield(\pointalt) - \coef[c]\solvec}{\pointalt - \sol}
	\geq \braket{\vecfield(\pointalt) - \solvec}{\pointalt - \sol}
	\geq \strong \norm{\pointalt -\sol}^{2}
\end{equation}
and our assertion follows from the basic bound $\norm{\point - \sol}^{2} \leq 2\norm{\point - \pointalt}^{2} + 2\norm{\pointalt - \sol}^{2}$.
\end{proof}

With this ancillary estimate in hand, we may finally sharpen \cref{prop:energy} to obtain a bona fide energy inequality for solutions satisfying \eqref{eq:strong}:

\begin{proposition}
\label{prop:energy-strong}
Suppose that \cref{asm:Lipschitz,asm:strong,asm:signal-base} hold and \eqref{eq:BPM} is run with\;$\curr[\step]$\;such\;that%
\begin{equation}
\label{eq:step-energy-strong}
2\strong \curr[\step] + 4\curr[\step]^{2}\lips^{2}
	\leq 1
	\quad
	\text{and}
	\quad
(1-\coef[a]-\coef[b])^{2}\curr[\step]
	\leq \frac{\strong}{8\lips^{2}}
	\quad
	\text{for all $\run=\running$}
\end{equation}
Then, for all $\run \geq \start$ such that \revise{$\curr,\lead \in \nhd$} and $\lead \in \basin$, we have
\begin{equation}
\label{eq:energy-strong}
\next[\energy]
	\leq \curr[\energy]
		- \strong \curr[\step] \curr[\pot]
		- \tfrac{1}{4} \strong\curr[\step] \norm{\curr - \sol}^{2}.
\end{equation}
\end{proposition}

\begin{proof}
Assume that $\lead\in\basin$ and set $\coef[c] = 1 - \coef[a] - \coef[b]$.
Then, invoking \cref{lem:strong-bound} with $\point \gets \curr$ and $\pointalt \gets \lead$, we get
\begin{flalign}
\MoveEqLeft
\braket{\vecfield(\lead) - \solvec}{\lead - \sol}
	+ (\coef[a] + \coef[b]) \braket{\solvec}{\lead-\sol}
	\notag\\
	&\geq \tfrac{1}{2} \strong \norm{\curr - \sol}^{2}
		- \strong \norm{\lead - \curr}^{2}.
\end{flalign}
Thus, taking $\coef \gets \strong$ in \cref{prop:energy} (in terms of step-size conditions, the first part of \eqref{eq:step-energy-strong} implies \eqref{eq:step-energy}) and combining with the above, the bound \eqref{eq:energy-bound} becomes
\begin{align}
\label{eq:strong-bound-energy-strong-proof}
\next[\energy]
	\leq \curr[\energy]
		- \strong \curr[\step] \curr[\pot]
		&- \tfrac{1}{2} \strong\curr[\step] \norm{\curr - \sol}^{2}
		+ \curr[\step]^{2} \coef[c]^{2} \lips^{2} \norm{\lead- \sol}^{2}
	\notag\\
	&- \tfrac{1}{2} \parens[\big]{1 - 4 \curr[\step]^{2} (\coef[a] + \coef[b])^{2} \lips^{2} - 2\strong \curr[\step]} \norm{\lead - \curr}^{2}.
\end{align}
Hence, writing $\norm{\lead - \sol}^{2} \leq 2\norm{\lead-\curr}^{2} + 2\norm{\curr-\sol}^{2}$ and rearranging, we obtain
\begin{align}
\label{eq:energy-strong-proof}
\next[\energy]
	\leq \curr[\energy]
		- \strong \curr[\step] \curr[\pot]
		&- \tfrac{1}{2} \parens[\big]{\strong \curr[\step] - 4 \curr[\step]^{2}\coef[c]^{2}\lips^{2}}
		\, \norm{\curr - \sol}^{2}
	\notag\\
	&- \tfrac{1}{2} \parens[\big]{1 - 4 \curr[\step]^{2} \parens{(\coef[a] + \coef[b])^{2} + \coef[c]^{2}} \lips^{2} - 2\strong \curr[\step]}
		\, \norm{\lead - \curr}^{2}.
\end{align}
Since $\coef[a],\coef[b],\coef[c] \geq 0$ and $\coef[a]+\coef[b]+\coef[c]=1$, we also have $(\coef[a]+\coef[b])^{2} + \coef[c]^{2} \leq 1$, so the step-size assumption \eqref{eq:step-energy-strong} guarantees that the last term in \eqref{eq:energy-strong-proof} is nonpositive.
Likewise, the second part of \eqref{eq:step-energy-strong} gives $\strong\curr[\step] - 4\curr[\step]^{2}\coef[c]^{2}\lips^{2} \geq \tfrac{1}{2}\strong\curr[\step]$, so the energy inequality \eqref{eq:energy-strong} follows and our proof is complete.
\end{proof}

We finally have all the required building blocks in place to prove \cref{thm:general}.

\begin{proof}[Proof of \cref{thm:general}]
Our proof strategy consists of the following basic steps:
\begin{enumerate}
\item
	We first show that, if the step-size of \eqref{eq:BPM} satisfies \eqref{eq:step} and $\init$ is initialized sufficiently close to $\sol$, \revise{the base and leading state sequences $\curr$ and $\lead$, $\run=\running$, both remain within the neighborhood $\legnhd\cap\basin$} of $\sol$ where \eqref{eq:strong}, \eqref{eq:Breg-lower}, and \eqref{eq:Breg-upper} all hold.

\item
By virtue of this stability result, the energy inequality \eqref{eq:energy-strong} and the definition of the Legendre exponent allow us to express $\curr[\breg] = \breg(\sol,\curr)$ as $\next[\breg] \leq \curr[\breg] - \bigof[\big]{\curr[\breg]^{1/(1-\legsol)}}$ up to an error term that vanishes at a geometric rate.
The rates \eqref{eq:rate} are then derived by analyzing this recursive inequality for $\legsol = 0$ and $\legsol>0$ respectively.
\end{enumerate}
%The formal proof is given below.
We now proceed to detail the two steps outlined above.

\para{Step 1: Stability}
Take $\radius > 0$ such that $\ball_{\radius}^\points(\sol) \defeq \{ \point\in\points : \norm{\point - \sol}
\leq \radius \} \subset \revise{\basin \cap \nhd}$
\revise{and such that $\proxof{\point}{- \step \vecfield(\pointalt)}$ belongs to $\nhd$ for all $\point\in\ball_{\radius}^\points(\sol) \cap \proxdom$, $\pointalt \in \ball_{\radius}^\points(\sol)$, and all admissible step-sizes $\step$.
This is indeed possible by the continuity of the prox-mapping (see \refinapp{lem:proxlip}{app:aux}) and of $\vecfield$.}
Assume further that $\state_{1/2} = \init \revise{\in \ball_\radius^\points(\sol)}$ is such that $\breg(\sol,\state_{1/2}) = \breg(\sol,\init) \leq (1 - \coef) \radius^{2}/4$, where $\coef\in(0,1)$ is a constant to be determined later.
that
\begin{equation}
\label{eq:stability}
\revise{\max\braces{\norm{\beforelead - \sol}, \norm{\curr - \sol}}}
	\leq \radius
	\quad
	\text{and}
	\quad
\curr[\energy]
	\leq \prev[\energy],
\end{equation}
which will show in particular that $\lead\in\basin\cap\nhd$ for all $\run\geq\start$.
Indeed:

\begin{itemize}
\addtolength{\itemsep}{\smallskipamount}
\item
	\revise{For the base case ($\run=\start$), we have $\state_{1/2} = \state_1 \in \ball_{\radius}^\points(\sol)$}
and $\init[\energy] = \beforeinit[\energy]$ by construction, so there is nothing to show.
\item
For the induction step, assume \eqref{eq:stability} holds.
Then, \revise{since $\curr \in \ball_{\radius}^\points(\sol) \subset \nhd$,}
\eqref{eq:Breg-lower} yields
\begin{equation}
\label{eq:curr-bound}
\tfrac{1}{2} \norm{\curr - \sol}^{2}
	\leq \curr[\breg]
	\leq \curr[\energy]
	\leq \init[\energy]
	= \breg(\sol,\init)
\end{equation}
\revise{
Moreover,  both $\curr$ and $\beforelead$ are in $\ball_{\radius}^\points(\sol)$ so that, by construction, $\lead$ is still in $\nhd$.}
Now, to show that $\lead \in \ball_{\radius}^\points(\sol)$, \cref{lem:onestep} with
$\base\gets\sol$, $\point \gets \curr$, $\dvec \gets -\curr[\step]\curr[\signal]$ and $\dbase\gets - (\coef[a] + \coef[b])\curr[\step]\solvec$ gives
\begin{alignat}{2}
\lead[\breg]
	&\leq \curr[\breg]
		&&- \curr[\step] \braket{\curr[\signal] - (\coef[a]+\coef[b])\solvec}{\lead - \sol}
	\notag\\
	&\leq \curr[\breg]
		&&- \coef[a] \curr[\step] \braket{\vecfield(\curr) - \solvec}{\lead - \sol}
	\notag\\
	&
		&&- \coef[b] \curr[\step] \braket{\vecfield(\beforelead) - \solvec}{\lead - \sol}
\end{alignat}
and hence, by Young's inequality and \eqref{eq:Breg-lower}, we get
\begin{align}
\tfrac{1}{2} \norm{\lead - \sol}^{2}
	\leq \curr[\breg]
		&+ \curr[\step]^{2} \coef[a] \dnorm{\vecfield(\curr) - \solvec}^{2}
		+ \curr[\step]^{2}\coef[b] \dnorm{\vecfield(\beforelead) - \solvec}^{2}
	\notag\\
		&+ \tfrac{1}{4} (\coef[a]+\coef[b]) \norm{\lead - \sol}^{2}\,.
\end{align}
Since $ \coef[a]+\coef[b] \leq 1$, using \cref{asm:Lipschitz} and rearranging gives
\begin{align}
\label{eq:checkme}
\norm{\lead - \sol}^{2}
	&\leq 4 \curr[\breg]
		+ 4\curr[\step]^{2} \lips^{2}
			\max\braces{\norm{\curr - \sol}^{2}, \norm{\beforelead-\sol}^{2}}
	\notag\\
	&\leq (1-\coef) \radius^{2}
		+ 4\curr[\step]^{2} \lips^{2} \radius^{2}
\end{align}
where we used the fact that $\norm{\beforelead - \sol}^{2} \leq \radius^{2}$
and
$\norm{\curr - \sol}^{2} \leq 2\curr[\breg] \leq \frac{1}{2} (1-\coef)\radius^{2}$
(by the inductive hypothesis and \eqref{eq:curr-bound} respectively).
Thus, with $2\curr[\step]\lips \leq 1/\gold < 1$ by assumption, choosing $\coef = 1/\gold^{2}$ gives $\norm{\lead - \sol}^{2} \leq \radius^{2}$, which completes the first part of the induction.
Finally, for the second part,
\revise{we have $\next\in\nhd$ because $\curr,\lead$ have been shown to be in $\ball_{\radius}^\points(\sol)$}
and our step-size assumption gives
\begin{equation}
2\strong \curr[\step] + 4 \curr[\step]^{2}\lips^{2}
	\leq 2 \curr[\step]\lips + 4 \curr[\step]^{2}\lips^{2}
	\leq 1/\gold + 1/\gold^{2}
	= 1.
\end{equation}
Thus, since $\lead\in\basin$, \cref{prop:energy-strong} readily gives
\begin{equation}
\label{eq:descent}
\next[\energy]
	\leq \curr[\energy]
		- \strong \curr[\step] \curr[\pot]
		- \tfrac{1}{4} \strong \curr[\step] \norm{\curr - \sol}^{2}
	\leq \curr[\energy],
\end{equation}
and the induction is complete.
\end{itemize}

\para{Step 2: Convergence rate analysis}
From \eqref{eq:descent} and the local Legendre bound \eqref{eq:Breg-upper}, we get
\begin{equation}
\label{eq:descent-Leg}
\next[\energy]
	\leq \curr[\energy]
		- \strong \curr[\step] \curr[\pot]
		- \frac{\strong \curr[\step]}{2^{1-\leg} \legconst^{1+\leg}} \curr[\breg]^{1+\leg}
  \qquad \text{with $\leg = \legsol/(1 - \legsol)$.}
\end{equation}
We now distinguish two cases, depending on whether $\legsol=0$ or $\legsol>0$.

\begin{enumerate}
[left=1em,label={\bfseries Case \arabic*:}]
\item
If $\legsol = 0$, we have $\leg=0$ by definition and $\legconst\geq1$ by \eqref{eq:Breg-lower}.
\cref{eq:descent-Leg} then gives
\begin{equation}
\next[\energy]
	\leq \curr[\energy]
		- \frac{\strong\curr[\step]}{2\legconst} \curr[\breg]
		- \strong \curr[\step] \curr[\pot]
	\leq \parens*{1 - \frac{\strong\curr[\step]}{2\legconst}} \curr[\energy]
\end{equation}
so the case $\legexp=0$ of \eqref{eq:rate} follows immediately by setting $\curr[\step] \equiv \step$ for all $\run$.

\item
If $\legsol > 0$, then $\leg > 0$ too, so we will proceed by rewriting all terms in \cref{eq:descent-Leg} in terms of $\curr[\energy]$.
To that end, we have:
\begin{align}
\next[\energy]
	&\leq \curr[\energy]
		- \strong \curr[\step] \curr[\pot]
		- \frac{\strong \curr[\step]}{2^{1-\leg} \legconst^{1+\leg}} \curr[\breg]^{1+\leg}
	\notag\\
	&\leq \curr[\energy]
		- \frac{\strong \curr[\step]}{\breg(\sol,\init)^{\leg}} \curr[\pot]^{1+\leg}
		- \frac{\strong \curr[\step]}{2^{1-\leg} \legconst^{1+\leg}} \curr[\breg]^{1+\leg}
	\notag\\
	&\leq \curr[\energy]
		- \frac{\strong\curr[\step]}{\max(2^{1-\leg}\legconst^{1+\leg}, \breg(\sol,\init)^\leg)}
			\bracks*{\breg(\sol, \curr)^{1+\leg}+\curr[\pot]^{1+\leg}}
	\notag\\
	&\leq \curr[\energy]
		- \frac{\strong\curr[\step]}{\max(2\legconst^{1+\leg},2^{\leg}\breg(\sol,\init)^\leg)}
			\curr[\energy]^{1+\leg}
\end{align}
where,
in the second line, we used \eqref{eq:stability} to get $\curr[\pot] \leq \breg(\sol,\curr) + \curr[\pot] \leq \breg(\sol,\init)$,
and,
in the last line, we used the convexity of $\point^{1+\leg}$.
The case $\legexp\in(0,1)$ of \eqref{eq:rate} then follows from Lemma 6 of \cite[p.~46]{Pol87} (recreated \revise{as }\refinapp{lem:Polyak}{app:aux}).
\end{enumerate}
\end{proof}


%----------------------------------------------------------------------
%%% LINEAR
%----------------------------------------------------------------------
\section{Finer results for linearly constrained problems}
\label{sec:sharp}
%----------------------------------------------------------------------
%%% SHARP
%----------------------------------------------------------------------
% !TEX root = ./Main.tex


Motivated by applications to game theory and linear programming, our goal in this section will be to take a closer look at the convergence rate of \eqref{eq:AMP} for different solution configurations that may arise in practice \textendash\ and, in particular, in linearly constrained problems.
To that end, we begin by revisiting the examples of \cref{sec:examples}.


%----------------------------------------------------------------------
%%% EXAMPLES REDUX
%----------------------------------------------------------------------
\subsection{Motivating examples, redux}
\label{sec:examples-sharp}

A common feature of \crefrange{ex:Eucl}{ex:Hell} is that the problem's defining vector field vanishes at the solution point under scrutiny.
In the series of examples below, we examine the rate of convergence achieved when this is not the case.


%----------------------------------------------------------------------
%% Euclidean begins here

\begin{example}[Euclidean regularization]
\label{ex:Eucl-sharp}
Consider again the quadratic regularizer of \cref{ex:Eucl} over $\points = [0,\infty)$, but with $\vecfield(\point) = \point + 1$.
The solution of \eqref{eq:VI} is still $\sol = 0$ but the update \eqref{eq:MD-generic} now becomes
\begin{equation}
\label{eq:MD-Eucl-sharp}
\fixmap(\point)
	= \pospart{\point - \step(\point + 1)}
	= \pospart{(1-\step) \point -\step}\,.
\end{equation}
Since $\fixmap(\point) = 0$ for all sufficiently small $\point>0$, we readily conclude that $\curr$ converges to $\sol$ in a \emph{finite} number of iterations.
\endenv
\end{example}

%% Euclidean ends here
%----------------------------------------------------------------------


%----------------------------------------------------------------------
%% Entropy begins here

\begin{example}[Entropic regularization]
\label{ex:ent-sharp}
Under the entropic regularizer of \cref{ex:ent}, and taking again $\vecfield(\point) = \point+1$, the update rule \eqref{eq:MD-generic} becomes
\begin{equation}
\label{eq:MD-ent-sharp}
\fixmap(\point)
	= \point \exp(-\step(\point + 1))
	= \point e^{- \step} + o(\point)
	\sim \point e^{-\step}
%	\quad
%	\text{for small $\point > 0$}.
%	\text{as $\point\to0$}
\end{equation}
\ie $\fixmap$ is a contraction for small $\point > 0$.
Hence, in contrast to \cref{ex:ent}, $\curr$ converges to $0$ at a geometric rate, even though the problem's solution lies on the boundary of $\points$.
\endenv
\end{example}

%% Entropy ends here
%----------------------------------------------------------------------


%----------------------------------------------------------------------
%% Fractional begins here

\begin{example}[Fractional power]
\label{ex:frac-sharp}
Finally, consider the fractional power regularizer of \cref{ex:frac}, again with $\vecfield(\point) = \point + 1$.
Then, for $\qexp \in (0,1)$, the update rule \eqref{eq:MD-generic} gives
\begin{align}
\fixmap(\point)
	&= \bracks{\point^{\qexp -1} + \step(1-\qexp)(\point + 1)}^{1/(\qexp-1)}
	% &= \point \bracks{1 + \step(1-\qexp)\point^{1-\qexp} + o(\point^{1-\qexp})}^{1/(\qexp-1)} \\
% 	&= \point \bracks{1 - \step\point^{1-\qexp} + o(\point^{1-\qexp})}
% 	\sim \point - \step\point^{2-\qexp}
%	\notag\\
	= \point - \step\point^{2-\qexp} + o(\point^{2-\qexp})
%	\quad
%	\text{as $\point\to0$}.
%	\text{for small $\point$}.
\end{align}
for small $\point>0$.
Thus, by \cref{lem:basicnum}, we conclude that $\curr$ converges to $0$ as $\abs{\curr - \sol} = \Theta\parens[\big]{\run^{-1/(1-\qexp)}}$ and hence $\breg(\sol,\curr) = \Theta\parens[\big]{\run^{-\qexp/(1-\qexp)}}$, which is again faster than the rate predicted by \cref{thm:general}.
\endenv
\end{example}

%% Fractional ends here
%----------------------------------------------------------------------


\Crefrange{ex:Eucl-sharp}{ex:frac-sharp} already show that the convergence rate of \eqref{eq:AMP} can change drastically depending on whether $\vecfield(\sol)$ is zero or not.
In the example below, we examine in more detail the behavior of the individual coordinates of $\curr$ as a function of the position of $\vecfield(\sol)$ relative to $\points$.


%----------------------------------------------------------------------
%% Simplex begins here

\begin{example}[Higher-dimensional simplices]
\label{ex:simplex-2d}
Consider the canonical two-di\-men\-sional simplex $\points = \setdef{(\point_{1}, \point_{2}, \point_3)\in \R_{+}^{3}}{\point_{1} + \point_{2} + \point_{3} = 1}$ of $\R^{3}$
%\PM{It's a 2d-simplex embedded in 3d-space, not a 3d-simplex, so I changed things a bit here ;-)}
equipped with the entropic regularizer
$\hreg(\point) = \sum_{\coord=1}^{3} \point_{\coord} \log\point_{\coord}$.
%with induced divergence function
%$\breg(\base,\point) = \sum_{\coord=1}^{3} \base_{\coord} \log(\base_{\coord}/\point_{\coord})$.
Consider also the vector field $\vecfield(\point) = \point - \base$ with $\base=(-\slack_{1}, -\slack_{2},1)$ for some $\slack_{1}, \slack_{2} \geq 0$, so the solution of \eqref{eq:VI} is $\sol = (0,0,1)$, an extreme point of $\points$.
%(see also \cref{fig:simplex} for a range of possible configurations).

Since the Legendre exponent of $\hreg$ at $\sol$ is easily seen to be $\legof{\sol} = 1/2$, \cref{thm:general} would indicate a rate of convergence of $\breg(\sol,\curr) = \bigoh(1/\run)$ or, in terms of norms, $\norm{\curr - \sol} = \bigoh(1/\run)$.
However, this rate can be very pessimistic if, for example, $\slack_{1} > 0$.
Indeed, in this case, since $\curr$ converges to $\sol = (0,0,1)$, the relevant coordinates of $\vecfield(\curr)$ will evolve as $\vecfield_{1}(\curr) = \state_{1,\run} + \slack_{1} = \slack_{1} + o(1)$ and $\vecfield_{3}(\curr) = \state_{3,\run} - 1 = o(1)$.
Accordingly, since entropic regularization on the simplex leads to the exponential weights update \cite{BecTeb03}
\begin{equation}
\label{eq:EW}
\state_{\coord,\run+1}
	\propto \state_{\coord,\run} \exp\parens*{-\step \vecfield_{\coord}(\curr)}
	\quad
	\text{for all $\run \geq \start$, $\coord = 1,2,3$},
\end{equation}
the fact that $\lim_{\run\to\infty}\state_{3,\run} = 1$ readily yields
\begin{align}
\state_{1,\run+1}
	\sim \frac{\state_{1,\run+1}}{\state_{3,\run+1}}
	&= \frac{\state_{1,\run}}{\state_{3,\run}}
		\exp\parens*{-\step \vecfield_{1}(\curr) + \step \vecfield_{3}(\curr)}
%	\\
	= \frac{\state_{1,\run}}{\state_{3,\run}}
		\exp\parens*{-\step \slack_{1} + o(1)}
%& & \text{ for } \run = \running
\end{align}
\ie $\state_{1,\run}$ converges to $0$ at a \emph{geometric} rate whenever $\slack_{1} > 0$.

By symmetry, the argument above yields the same rate for $\state_{2,\run}$ if $\slack_{2} > 0$.
However, as we show in \cref{app:ex}, if $\slack_{2} = 0$, we would have $\state_{2,\run} = \Theta(1/\run)$ no matter the value of $\slack_{1}$ (and likewise for the rate of $\state_{1,\run}$ if $\slack_{1}=0$).
In other words, the rate provided by \cref{thm:general} is tight for the coordinate $\coord \in \{1,2\}$ with a vanishing drift coefficient $\slack_{\coord}$, but not otherwise;
we will devote the rest of this section to deriving a formal statement (and proof) of the general principle underlying this observation.
%If $\slack_{2}$ was positive, we would have the same result for the second coordinate, but if it null, one can check that the convergence speed of $\state_{2,\run}$ does not improve.
%Indeed, with $\slack_{2} = 0$, one can show that
%\begin{align}
%\state_{2,\run+1}
%	\sim	\frac{\state_{2,\run+1}}{(\next)_3} =\Omega\parens*{\frac{1}{\run}}\,,
%\end{align}
%see \cref{app:ex} for details.
\endenv
\end{example}

%% Simplex begins here
%----------------------------------------------------------------------


%----------------------------------------------------------------------
%%% LINEAR SETUP
%----------------------------------------------------------------------
\subsection{Linearly constrained problems}

For concreteness, we will focus in what follows on linearly constrained problems, which is where the structural configurations outlined in the previous examples are more prominent.
To simplify the presentation and the analysis, we will identify $\pspace$ with $\R^{\nCoords}$ endowed with the Euclidean scalar product $\inner{\cdot}{\cdot}$, and we will not distinguish between primal and dual vectors (meaning in particular that the distinction between normal and polar cones will be likewise blurred).
%\FI{From now on, only normal cones, no dual/polar !}

Formally, we will consider polyhedral domains written in normal form as
\begin{equation}
\label{eq:polyhedron}
\points
	= \setdef{\point \in \R_{+}^{\nCoords}}{\mat \point = \cvec}
\end{equation}
for some matrix $\mat\in\R^{\nConstr\times\nCoords}$ and $\cvec\in\R^{\nCoords}$.%
\footnote{Inequality constraints of the form $\mat\point\leq\cvec$ can also be accommodated in \eqref{eq:polyhedron} by introducing the associated slack variables $s = \cvec - \mat\point \geq 0$.
Even though this leads to a more verbose presentation of $\points$, the form \eqref{eq:polyhedron} is much more convenient in terms of notational overhead, so we will stick with the equality formulation throughout.}
Moreover, %to avoid trivialities, 
we will further assume that $\points$ admits a \emph{Slater point}, \ie there exists some $\point\in\points$
such that $\point_{\coord} > 0$ for all $\coord=1,\dotsc,\nCoords$.
This setup is particularly flexible, as it allows us to identify the \emph{active} constraints at $\point\in\points$ with the zero components of $\point$.

Elaborating further on this, since $\inner{\vecfield(\sol)}{\point-\sol} \geq 0$ for all $\point\in\points$ and any solution $\sol$ of \eqref{eq:VI}, we directly infer that $-\solvec$ is an element of the normal cone $\ncone(\sol)$ to $\points$ at $\sol$.
In our polyhedral setting, $\ncone(\sol)$ admits an especially simple representation as
\begin{equation}
\label{eq:ncone-sharp}
\ncone(\sol)
	= \row(\mat)
		- \setdef{(\slack_{1},\dotsc,\slack_{\nCoords})\in\R_{+}^{\nCoords}}{\slack_{\coord}=0 \text{ whenever } \sol_{\coord} = 0}
%	= \setdef[\bigg]{-\sum_{\coord\in\actcoords} \slack_{\coord} \bvec_{\coord}}{\slack_{\coord}\geq0, \coord\in\actcoords}
%		+ \row(\mat)
%	= \left\{-\sum_{\coord \in \actcoords} \slack_{\coord} \bvec_{\coord} : (\slack_{\coord})_{\coord \in \actcoords} \in \R_+^{\actcoords}\right\} + \row(\mat) %\\
		%\relint \ncone(\sol) &= \left\{-\sum_{\coord \in \actcoords} \slack_{\coord} \bvec_{\coord} : (\slack_{\coord})_{\coord \in \actcoords} \in (\R_+^*)^{\actcoords}\right\} + \row(\mat)\,,
\end{equation}
%or, more compactly, as $\ncone(\sol) = \row(\mat) - \R_{+}^{\actcoords}$,
where $\row(\mat) = (\ker\mat)^{\perp} \subseteq \R^{\nCoords}$ denotes the row space of $\mat$ \cite[Ex.~5.2.6]{HUL01}.
%the normal cone to $\points$ at $\sol$ can be expressed more compactly as $\ncone(\sol) = \row(\mat) - \R_{+}^{\actcoords}$, with $\R^{\actcoords}$ treated here as a subspace of $\R^{\nCoords}$.
As a result, we see that $\sol$ is a solution of \eqref{eq:VI} if and only if $\vecfield(\sol)$ can be written in the form
\begin{equation}
\label{eq:slacks}
\solvec - \sum_{\coord\in\actcoords} \slack_{\coord}\bvec_{\coord}
	\in \row(\mat)
\end{equation}
for an ensemble of non-negative \emph{slackness coefficients} $\slack_{\coord} \geq 0$, $\coord\in\actcoords$, where
\begin{equation}
\label{eq:actcoords}
\actcoords
	\equiv \actcoords(\sol)
	= \setdef{\coord}{\sol_{\coord} = 0}
\end{equation}
denotes the set of inequality constraints of \eqref{eq:polyhedron} that are active at $\sol$.
In view of all this, we will distinguish the following solution configurations:

\begin{definition}[Sharpness]
\label{def:sharp}
Let $\sol\in\points$ be a solution of \eqref{eq:VI} with associated slackness coefficients $\slack_{\coord}$, $\coord\in\actcoords$, as per \eqref{eq:slacks}.
The set of \emph{sharp} \textpar{$\sharp$} and \emph{flat} \textpar{$\flat$} directions at $\sol$ are respectively defined as
\begin{equation}
\label{eq:sharp}
\sharps
	= \setdef{\coord\in\actcoords}{\slack_{\coord} > 0}
	\qquad
	\text{and}
	\qquad
\flats
%	= \actcoords\setminus\sharps
	= \setdef{\coord\in\actcoords}{\slack_{\coord} = 0},
\end{equation}
and we say that $\vecfield$ is \emph{sharp} at $\sol$ if $\sharps = \actcoords$ (or, equivalently, if $\flats = \varnothing$).
The \emph{sharpness} of $\vecfield$ at $\sol$ is then defined as
\begin{equation}
\label{eq:drift}
\drift
	= \min\nolimits_{\coord\in\sharps} \slack_{\coord}.
\end{equation}
%and we will say that $\sol$ is itself \emph{sharp} if
%$\lspan(\sharps) + \row(\mat) = \points$.
%$\sharps = \{1,\dotsc,\nCoords\}$ (\ie $\sol=0$ and $\sharps = \actcoords$).\FI{Added this, delete if too redundent}
\end{definition}

The terminology ``sharp'' and ``flat'' alludes to the case where $\vecfield$ is a gradient field, and is best illustrated by an example.
To wit, let $\obj(\point_{1},\point_{2}) = \point_{1} + \tfrac{1}{2}(\point_{2}-1)^{2}$ for $\point_{1},\point_{2}\geq0$, so $\obj$ admits a (unique) global minimizer at $\sol = (0,1)$.
Applying \cref{def:sharp} to $\vecfield = \nabla\obj$, we readily get $\sharps = \{1\}$ and $\flats=\{2\}$, reflecting the fact that $\obj(\point_{1},1)$ exhibits a sharp minimum at $0$ along $\point_{1}$ whereas the landscape of $\obj(0,\point_{2})$ is flat to first-order around $1$ along $\point_{2}$.


%----------------------------------------------------------------------
%%% RATES
%----------------------------------------------------------------------
\subsection{Convergence rate analysis}
\label{sec:rate-sharp}

We are now in a position to state and prove our refinement of \cref{thm:general} for linearly constrained problems.
To that end, following \citet{ABB04}, we will assume in the rest of this section that \eqref{eq:AMP} is run with a Bregman regularizer $\hreg$ that is adapted to the polyhedral structure of $\points$ as per the definition below:

\begin{definition}
\label{def:decomposable}
Let $\points$ be a polyhedral domain of the general form \eqref{eq:polyhedron}, and let $\hker\from\R_{+}\to\R$ be a continuous function such that
\begin{enumerate*}
[\itshape a\upshape)]
\item
$\hker''(\point)$ exists and is positive for all $\point>0$;
and
\item
$\hker''$ is locally Lipschitz on $(0,\infty)$.
\end{enumerate*}
Then, a Bregman regularizer $\hreg$ on $\points$ is said to be \emph{decomposable with kernel $\hker$} if
\begin{equation}
\label{eq:decomposable}
\hreg(\point)
	= \sum_{\coord=1}^{\nCoords} \hker(\point_{\coord})
	\quad
	\text{for all $\point\in\points$}.
\end{equation}
\end{definition}

%\begin{assumption}[Polyhedral Bregman regularizers]\label{asm:decomposable}
%	Let $\hker$ be a proper, \ac{lsc}, convex $\R\to\R\cup\{\infty\}$ function with domain $\dom \hker = [0, +\infty)$ and $\hker'$ be a continuous selection of subgradients. We define
%\begin{equation}
%\hreg(\point)
%	\defeq \sum_{\coord=1}^{\nCoords} \hker(\point_{\coord}) \text{ if } \mat \point = \cvec \text{ and } + \infty \text{ elsewhere}\,
%	\end{equation}
%	so that $\points = \dom \hreg$.
%	Moreover, we assume that $\hreg$ is 1 strongly convex w.r.t.~some norm $\|.\|$.
%\end{assumption}

In addition to facilitating calculations, the notion of decomposability will further allow us to describe the convergence rate of the iterates of \eqref{eq:AMP} near the boundary of $\points$ in finer detail.
%and ultimately explain the differences observed in \crefrange{ex:Eucl-sharp}{ex:simplex-2d}.
In fact, as it turns out, the speed of convergence along a given direction will actually be determined by the behavior of the derivative of the Bregman kernel $\hker$ near $0$.

%\PMedit
{
In this regard, there are two distinct regimes to consider.
First, if $\lim_{\point\to0^{+}}\hker'(\point) = -\infty$, it is straightforward to see that $\dom\subd\hreg = \relint\points$ so, by \cref{lem:mirror}, the iterates $\curr$ of \eqref{eq:AMP} will remain in $\relint\points$ for all $\run$;
in this case $\hreg$ is essentially smooth \textendash\ or \emph{Legendre} \textendash\ in the sense of \citet[Chap.~26]{Roc70}, and we will refer to it as \emph{steep}.
Otherwise, if $\hker'(0)$ exists and is finite, $\curr$ may reach the boundary of $\points$ in a finite number of iterations;
we will refer to this case as \emph{non-steep}.
The key difference between these two regimes is that, in the non-steep case, the algorithm may achieve convergence in a finite number of steps (at least along certain directions).
On the other hand, even though finite-time convergence is not possible in the steep regime, the algorithm's rate of convergence may still depend on the boundary behavior of $\hker$.
To illustrate this, we will consider the following concrete cases:
}

\begin{assumption}
\label{asm:ker}
Let $\hker\from\R_{+}\to\R$ be a kernel function as per \cref{def:decomposable}.
%Then, as $\point\to0^{+}$, one of the following holds:
Then $\hker'$ exhibits one of the following behaviors as $\point\to0^{+}$:
\begin{enumerate}
[(\itshape a\upshape)]
\item
\label[case]{asm:ker-Eucl}
\emph{Euclidean-like:}
	\tabto{7em}
	$\liminf_{\point\to0^{+}} \hker'(\point) > -\infty$.
% \PM{An internal remark for posterity:
% since $\hker'$ is increasing (by \cref{def:decomposable}), the above implies that $\lim_{\point\to0^{+}}\hker'(\point)$ also exists, and is hence equal to $\hker'(0)$.}
% \WA{Curiosity question on the internal remark: why couldn't this limit be > $\hker'(0)$?}
% \PM{I was thinking because $\hker$ is differentiable in the interior $+$ mean value theorem.}
% \WA{Indeed}

\item
\label[case]{asm:ker-log}
\emph{Entropy-like:}
	\tabto{7em}
	$\liminf_{\point\to0^{+}} \bracks{\hker'(\point) + \log\point} > -\infty$.
\item
\label[case]{asm:ker-power}
\emph{Power-like:}
	\tabto{7em}
	$\liminf_{\point\to0^{+}} \point^{\kernelexp} \hker'(\point) > -\infty$ for some $\kernelexp\in(0,1)$.
\end{enumerate}
\end{assumption}

%----------------------------------------------------------------------
\begin{remark*}
\Crefrange{asm:ker-Eucl}{asm:ker-power} above respectively mean that $\abs{\hker'(\point)}$ grows as $\bigoh(1)$, $\bigoh(\abs{\log\point})$ or $\bigoh(1/\point^{\kernelexp})$ as $\point\to0^{+}$.
Clearly, we have \ref{asm:ker-Eucl}$\implies$\ref{asm:ker-log}$\implies$\ref{asm:ker-power} so these cases are not exclusive;
nonetheless, to avoid overloading the presentation, when we say that \ref{asm:ker-log} holds, we will tacitly imply that \labelcref{asm:ker-Eucl} does not also hold at the same time \textendash\ and likewise for \ref{asm:ker-power}.
%Under this convention, \cref{asm:ker}\ref{asm:ker-Eucl} corresponds to non-steep regularizers, whereas \cref{asm:ker-log,asm:ker-power} are both steep.
\endenv
\end{remark*}
%----------------------------------------------------------------------


%\begin{assumption}\label{asm:ker}
% The behavior of $\hker$ near $0$ either satisfies:
%	\begin{enumerate}[(a)]
%			\item There exists $\kernelcst \in \R$, $\kernelexp > 0$, $\thres > 0$ such that, for $ \point \in \dom \partial \hker \cap [0, \thres)$,
%\begin{equation}
%	\log \point - \kernelcst \le \hker'(\point)\,.
%\end{equation}\label{asm:ker grad log}
%		\item There exists $\kernelcst > 0$, $\kernelexp > 0$, $\thres > 0$ such that, for $ \point \in \dom \partial \hker \cap [0, \thres)$,
%\begin{equation}
%	-\kernelcst\point^{-\kernelexp} \le \hker'(\point)
%\end{equation} \label{asm:ker grad sublinear}
%	\item $\hker'$ is defined at 0 and it is continuous on its whole domain $\proxdom = [0, +\infty)$.
%\label{asm:ker grad continuous}
%\WA{If we wanted to be able to say that this case implies that $\legexp = 0$, we would need to replace "continuous" by "locally Lipschitz".}
%	\end{enumerate}
%\end{assumption}
%
%With this notion at hand, we can now state our main result of this section: the coordinates of Mirror-prox type methods converge linearly, sublinearly, or even in finite time depending on the steepness of the Bregman regularizer as characterized by the above assumption.

With all this in hand, we proceed to show that, in linearly constrained problems, \eqref{eq:AMP} converges along sharp directions at $\sol$ at an accelerated rate relative to \cref{thm:general}:
sublinear rates may become linear, and linear rates transform to convergence in finite time.

\begin{theorem}
\label{thm:sharp}
%Let $\points$ be a polyhedral domain of the general form \eqref{eq:polyhedron}, and assume that \eqref{eq:AMP} is run with a decomposable regularizer as per \cref{def:decomposable}.
Suppose that \eqref{eq:AMP} is run in a polyhedral domain with a decomposable regularizer as per \cref{def:decomposable}.
Suppose further that \cref{asm:Lipschitz,asm:strong,asm:signal-base,asm:ker} hold, and that the method's step-size and initialization satisfy the requirements of \cref{thm:general}.
Then, for all $\coord\in\sharps$, we have:
\smallskip
\begin{subequations}
\label{eq:rate-sharp}
\begin{enumerate}
%[left=1em,label={\bfseries Case (\alph*):}]
[(\itshape a\upshape)]
\item
Under \cref{asm:ker}\ref{asm:ker-Eucl}, there exists some $\nRuns \geq \start$ such that:
\begin{align}
\label{eq:rate-sharp-Eucl}
\state_{\coord,\run}
	&= 0
	\quad
	\text{for all $\run\geq\nRuns$}
\shortintertext{%
\item
Under \cref{asm:ker}\ref{asm:ker-log}:}
\label{eq:rate-sharp-log}
\state_{\coord,\run}
	&= \bigoh\parens[\big]{\exp(-\step\drifteff\run/2)}
\shortintertext{%
\item
Under \cref{asm:ker}\ref{asm:ker-power}:}
\label{eq:rate-sharp-power}
\state_{\coord,\run}
	&= \bigoh\parens[\big]{(\step\drifteff\run/2)^{-1/\kernelexp}}
\end{align}
\end{enumerate}
\end{subequations}
where
\begin{equation}
\label{eq:drift-eff}
\drifteff
	= \begin{cases}
		\drift
			&\quad
			\text{if $\vecfield$ is sharp at $\sol$ \textpar{\ie $\flats=\varnothing$}},
			\\
		\drift/\varrho
			&\quad
			\text{otherwise},
	\end{cases}
\end{equation}
and $\varrho \equiv \varrho(\mat,\cvec,\sol) \geq 1$ is a positive constant that depends only on $\points$ and $\sol$.
\end{theorem}

\cref{thm:sharp} is the main result of this section so, before proving it, some remarks are in order.
We begin with the observation that, if the sharp directions at $\sol$ suffice to characterize it, the coordinate-wise guarantees \eqref{eq:rate-sharp} must extend to the full space.
This is always so if $\sol$ is an extreme point of $\points$ and $\vecfield$ is sharp at $\sol$, in which case we will say that $\sol$ is itself \emph{sharp}.
We then have the following immediate corollary of \cref{thm:sharp}:
%\begin{equation}
%\label{eq:sol-sharp}
%\lspan(\sharps) + \row(\mat)
%	= \pspace
%\end{equation}
%\WAedit{To better understand \cref{thm:sharp}, one can look at the particular case of $\sol$ being an extremal point and $\vecfield$ sharp at $\sol$.}


%----------------------------------------------------------------------
%% Simplex figure begins here

\begin{figure}
\centering
\resizebox{!}{.27\textwidth}{\input{Figures/Sol-mixed.tex}}
\hfill
\resizebox{!}{.27\textwidth}{%----------------------------------------------------------------------
%%% NASH
%----------------------------------------------------------------------
% !TEX root = ../Main.tex
%
%
\begin{tikzpicture}
[scale=.75,
>=stealth,
edgestyle/.style={-, line width=.75pt},
vecstyle/.style = {->, line width=.75pt},
nodestyle/.style={circle, fill=black,inner sep = .5pt},
plotstyle/.style={color=DarkGreen!80!Cyan,thick}]

% Local definitions
\def\unit{1.25}
\def\costhirty{0.8660256}
\def\tanthirty{0.57735026919}
\def\upangle{60}
\def\downangle{0}
%\def\diffangle{0}
%\def\midangle{100}
\def\conescale{1.75}
\small

% Coordinates
\coordinate (O) at (0,-\unit/6);

\coordinate (P1) at (-2*\unit*\costhirty,-\unit) {};
\coordinate (P2) at (2*\unit*\costhirty,-\unit) {};
\coordinate (P3) at (0,2*\unit) {};

\coordinate (sol) at ($(P3)$);
\coordinate (X) at (1.4*\unit,0.1*\unit);

%\coordinate (PP1) at ($(O)+1.1*(P1)-1.1*(O)$);
%\coordinate (PP2) at ($(O)+1.1*(P2)-1.1*(O)$);
%\coordinate (PP3) at ($(O)+1.1*(P3)-1.1*(O)$);

% Drawing
\fill [PrimalFill] (P1) -- (P2) -- (P3) -- cycle;

\draw [edgestyle, thick, DualColor] (sol.center) -- ++(\upangle+90:\conescale*\unit) node [midway,below left] {$\ncone(\sol)$};
\fill [DualColor!5] (sol)++(\upangle+90:\conescale*\unit) -- (sol) --++(\downangle+30:\conescale*\unit);
\draw [edgestyle, thick, DualColor] (sol.center) -- ++(\upangle+90:\conescale*\unit) node {};
\draw [edgestyle, thick, DualColor] (sol.center) -- ++(\downangle+30:\conescale*\unit);

\node (X) at (X) [PrimalColor] {$\points$};

\draw [->, gray!50] (O) to (P1);
\draw [->, gray!50] (O) to (P2);
\draw [->, gray!50] (O) to (P3);

%\draw [-stealth, gray!75] (P1) to (PP1);
%\draw [-stealth, gray!75] (P2) to (PP2);
%\draw [-stealth, gray!75] (P3) to (PP3);

\draw [edgestyle,PrimalColor] (P1) -- (P2) -- (P3) -- (P1);

\node [nodestyle, label = right:$\sol$] (sol) at (sol) {.};
%\draw [vecstyle] (sol.center) -- ($(sol)+(1/2,1)$) node [near end,left,black] {$-\solvec$};
\draw [vecstyle] (sol.center) -- ($(sol)+(-\unit,\tanthirty*\unit)$) node [very near end,right,black] {\;$-\solvec$};

\end{tikzpicture}}
\hfill
\resizebox{!}{.27\textwidth}{\input{Figures/Sol-strict.tex}}
\hspace{2em}
\caption{Different solution configurations:
a non-extreme solution where $\vecfield$ is sharp (left),
an extreme solution where $\vecfield$ is not sharp (center),
and
a sharp solution (\ie an extreme solution where $\vecfield$ is sharp; right).}
\label{fig:simplex}
\vspace{-2ex}
\end{figure}

%% Simplex figure ends here
%----------------------------------------------------------------------


\begin{corollary}
\label{cor:sharp}
%\WAedit{Under the condition that $\lspan \setdef{\bvec_\coord}{\coord \in \sharps} + \row(\mat) = \pspace$, we have $\norm{\curr - \sol} \to 0$ at a rate}
\PMedit{If $\sol$ is sharp, we have $\norm{\curr - \sol} \to 0$ at a rate given by}
\eqref{eq:rate-sharp-Eucl}, \eqref{eq:rate-sharp-log}, or \eqref{eq:rate-sharp-power},
%\cref{eq:rate-sharp-Eucl,eq:rate-sharp-log,eq:rate-sharp-power},
depending respectively on whether \cref{asm:ker-Eucl}, \labelcref{asm:ker-log}, or \labelcref{asm:ker-power} of \cref{asm:ker} holds.
\end{corollary}

\begin{proof}
First, note that $\sol$ is sharp if and only if $\lspan\setdef{\bvec_{\coord}}{\coord\in\sharps} + \row(\mat) = \R^{\nCoords}$:
indeed, since $\points$ is a polyhedron, $\sol$ is extreme if and only if $\ncone(\sol)$ has nonempty topological interior, and this, combined with \eqref{eq:ncone-sharp} and the fact that $\sharps = \actcoords$ (since $\vecfield$ is sharp at $\sol$), proves our assertion.
We thus conclude that, for all $\coord = 1,\dotsc,\nCoords$, there exist $\coef_{\coord\coordalt} \in \R$, $\coordalt\in\sharps$, such that $\bvec_{\coord} - \sum_{\coordalt\in\sharps} \coef_{\coord\coordalt} \bvec_{\coordalt} \in \row(\mat)$,
and hence, for all $\run = \running$, we have $\state_{\coord,\run} - \sol[\state_{\coord}] = \sum_{\coordalt \in \sharps} \coef_{\coord\coordalt} \state_{\coordalt,\run}$.
Our claim then follows from \cref{thm:sharp} and the fact that all norms are equivalent on $\R^{\nCoords}$.
\end{proof}

We continue with a series of observations elaborating further on \cref{thm:sharp}.

%----------------------------------------------------------------------
\setcounter{remark}{0}
%----------------------------------------------------------------------
\begin{remark}
[Examples]
\label{rem:examples}
In our series of running examples, the guarantees of \cref{thm:sharp} are as follows:
\begin{enumerate}
\item
\emph{Euclidean regularization} (\cref{ex:Eucl}):
With $\hker(\point) = \point^{2}/2$, this regularizer satisfies \cref{asm:ker}\ref{asm:ker-Eucl} because $\hker'$ is defined and continuous on all of $\R_{+}$, so we get convergence along the sharp directions of $\sol$ in a finite number of steps.
\item
\emph{Negative entropy} (\cref{ex:ent}):
The corresponding kernel is $\hker(\point) = \point \log \point$ for $\point \geq 0$.
Since $\hker'(\point) = \log \point + 1$ on $(0,\infty)$, $\hker$ satifies \cref{asm:ker}\ref{asm:ker-log}, so the algorithm's rate of convergence along sharp directions is geometric.
\item
\emph{Tsallis entropy} (\cref{ex:frac}):
For $\qexp < 1$, the kernel $\hker(\point) =
[\qexp(1-\qexp)]^{-1} (\point - \point^{\qexp})$ is differentiable on $(0, \infty)$.
Since $\hker'(\point) = [\qexp(1-\qexp)]^{-1} (1 - \qexp \point^{\qexp-1})$, this kernel satisfies \cref{asm:ker}\ref{asm:ker-power} with $\kernelexp = 1-\qexp$, leading to an $\bigoh(1/\run^{1/(1-\qexp)})$ rate of convergence along sharp directions.
It is also worth noting here that the Legendre exponent at $\sol$ is upper bounded by $\kernelexp$ as $\legsol \leq (1 + \kernelexp)/2$;
we defer the details of this calculation to \cref{app:ex}.
\item
\emph{Hellinger regularizer} (\cref{ex:Hell}):
The Hellinger kernel is given by $\hker(\point) = -\sqrt{1-\point^{2}}$, so $\hker'(\point) = \point/(1-\point^{2})$ for all $\point\in(-1,1)$.
The behavior of this kernel would then correspond to \cref{asm:ker}\ref{asm:ker-power} with $\kernelexp = 1/2$.
\end{enumerate}
To facilitate comparisons with \cref{thm:general}, we juxtapose the corresponding rates in \cref{tab:rates-sharp}.
% Indeed, let $\nhd$ be the neighborhood of $\base$ given by \cref{asm:ker} \ref{asm:ker grad sublinear}.
% At the cost of reducing $\nhd$, we can assume that $\nhd \cap \bd \dom \hker \subseteq \{\base\}$.
% Now, we can safely take some $\point \in \nhd \cap \dom \partial \hker$, with $\point \neq \base$ and it is guaranteed that $[\point, \base) \in \intr\dom \hker $.
\endenv
\end{remark}
%----------------------------------------------------------------------


%----------------------------------------------------------------------
%% Table of rates begins here

\begin{table}[tbp]
\footnotesize
\renewcommand{\arraystretch}{1.25}
%\setlength{\tabcolsep}{1em}
%----------------------------------------------------------------------
%%% RATES
%----------------------------------------------------------------------
% !TEX root = ../Main.tex

\begin{tabular}{lcccc}
\toprule
%	&\textbf{Domain ($\points$)}
	&\textbf{Bregman Kernel ($\hker$)}
	&\textbf{Generic rate (\cref{thm:general})}
	&\textbf{Sharp rate (\cref{thm:sharp})}
	\\
%    &($\points$)
%    &($\hker$)
%    &\textbf{from \cref{thm:general}}
%    &\textbf{of active constraints}
%    \\
\midrule
\scshape{Euclidean}	
%	&arbitrary
	&$\point^{2}/2$
	%&$0$
    %&$o(1)$
    &Linear
	&Finite time
	\\
\scshape{Entropic}
%	&$[0,\infty)$
	&$\point\log\point$
	%&$1/2$
    %&$\Theta\parens*{\log \frac{1}{\point}}$
	&$\bigoh(1/\run)$
	&Linear %$\exp(-\Omega(\run))$
	\\
\scshape{Tsallis}
%	&$[0,\infty)$
%	&$\frac{\point - \point^{\qexp}}{\qexp(1-\qexp)}$
	&$[\qexp(1-\qexp)]^{-1} (\point - \point^{\qexp})$
    %&$1 - \frac{\qexp}{2}$
	%&$\Theta\parens*{\frac{1}{\point^{1-\qexp}}}$
	&$\bigoh(1/\run^{\qexp/(2-\qexp)})$
	&$\bigoh(1/\run^{1/(1-\qexp)})$
	\\
\scshape{Hellinger}
%	&$[-1,1]$
	&$-\sqrt{1-\point^{2}}$
	%&$3/4$
    %&$\Theta\parens*{\frac{1}{\sqrt{(\point + 1)(1-\point)}}}$
    &$\bigoh(1/\run^{1/3})$
	&$\bigoh(1/\run^{2})$
	\\
\bottomrule
\end{tabular}

\smallskip
\caption{Summary of the accelerated rates of convergence observed along sharp directions as a function of the underlying Bregman kernel (\cf \cref{def:decomposable}).
The Euclidean, entropic and Tsallis kernels are the prototypical examples of \cref{asm:ker-Eucl,asm:ker-log,asm:ker-power} of \cref{asm:ker};
%(\cf \crefrange{ex:Eucl-sharp}{ex:frac-sharp});
to avoid trivialities, we only consider the behavior of $\hker$ at the boundary of its domain.}
\label{tab:rates-sharp}
\vspace{-2ex}
\end{table}

%% Table of rates ends here
%----------------------------------------------------------------------


%----------------------------------------------------------------------
\begin{remark}
[Solution configurations]
By construction (and the fact that $\points$ admits a Slater point), it is straightforward to verify that $\vecfield$ is sharp at $\sol$ if and only if $\vecfield(\sol) \in -\relint(\ncone(\sol))$;
likewise, $\sol$ is itself sharp if and only if $\vecfield(\sol) \in -\intr(\ncone(\sol))$.
%\PMedit
{
As we noted in the proof of \cref{cor:sharp}, the latter condition is equivalent to asking that $\lspan\setdef{\bvec_{\coord}}{\coord\in\sharps} + \row(\mat) = \R^{\nCoords}$, a condition which describes precisely the informal requirement that the sharp directions at $\sol$ suffice to characterize it.
By contrast, if $\vecfield$ is sharp at some non-extreme point $\sol$, there exists some (nonzero) $\tvec\in\tcone(\sol)$ such that $\braket{\vecfield(\sol)}{\tvec} = 0$, indicating that the accelerated rates of \cref{thm:sharp} cannot be active along the residual direction $\pvec$.
}
We illustrate these distinct solution configurations in \cref{fig:simplex}.

\endenv
\end{remark}
%----------------------------------------------------------------------


%----------------------------------------------------------------------
\begin{remark}[Alternative polyhedral representations]
Even though every polyhedral domain can be represented in normal form by means of \eqref{eq:polyhedron} \textendash\ possibly up to introducing a set of slack variables to account for constraints of the form $\mat\point \leq \cvec$ \textendash\ some polyhedra can be represented more succinctly as
%This theorem could be generalized, mainly at the cost of more intricate settings and notations, to the following class of polyhedra
\begin{equation}
\label{eq:polyhedron-alt}
\points
	= \setdef{\point\in\R^{\nCoords}}{\text{$\mat\point = \cvec$ and $\inner{\matalt_{\constr}}{\point} \in \region_{\constr}, \constr=1,\dotsc,\nConstr$}}
\end{equation}
for an ensemble of vectors $\matalt_{\constr} \in \R^{\nCoords}$ and intervals $\region_{\constr} \subseteq \R$ for each of the problem's inequality constraints $\constr=1,\dotsc,\nConstr$.
At the cost of heavier notation, \cref{thm:sharp} can be extended to this setting by considering decomposable Bregman regularizers of the form $\hreg(\point) = \sum_{\constr=1}^{\nConstr} \hker_{\constr}(\inner{\matalt_{\constr}}{\point})$, $\point\in\points$, where each $\hker_{\constr}$ is a suitable Bregman kernel on $\region_{\constr}$.
Mutatis mutandis, the sets of active, sharp and flat constraints can then be defined as in \cref{def:sharp}, and the guarantees of \cref{thm:sharp} would apply to the constraint excess variables $\chi_{\constr,\run} = \inner{\matalt_{\constr}}{\curr}$, $\constr=1,\dotsc,\nConstr$.
%	\begin{align}
%		\points = \{\point \in \pspace: \inner{\matalt_{\coord}}{\point} \in \dom \hker_{\coord}, \coord=1,\dots,\nConstr, \mat \point = \cvec\}\,,
%	\end{align}
%	with
%	\begin{align}
%		\hreg(\point) \defeq \sum_{\coord = 1}^{\nCoords} \hker_{\coord}\parens*{\inner{\matalt_{\coord}}{\point}} + \delta_{\mat \point = \cvec}\,.
%	\end{align}
%	However, the guarantees of the \cref{thm:sharp} would be on the quantities,
%	\begin{align}
%		|\inner{\matalt_{\coord}}{\point - \sol}| \text{ for } \coord \in \sharps\,.
%	\end{align}
\endenv
\end{remark}
%----------------------------------------------------------------------


%----------------------------------------------------------------------
\begin{remark}[Tightness and the structure of $\points$]
It is also important to note that the dependence of $\drifteff$ on the structure of $\points$ in the second branch of \eqref{eq:drift-eff} cannot be lifted.
To see this, let
\begin{equation}
\points
	= \setdef{\point\in\R_{+}^{2}}{\point_{1} = \eps\point_{2}}
\end{equation}
\ie $\mat = \bracks{1 \; -\eps}$ and $\ker\mat$ is spanned by the vector $\pvec = (\eps,1)$.
Then, if we take $\vecfield(\point) = \point - \base$ with $\base_{1} \leq 0$ and $\base_{2} \geq 0$ (so that the origin is a solution), and we equip $\points$ with the Bregman regularizer induced by the entropic kernel $\hker(\point) = \point\log\point$, a straightforward calculation shows that the iterates of \eqref{eq:MD} satisfy the recursion
\begin{equation}
\eps \hker'(\state_{1,\run+1}) + \hker'(\state_{2,\run+1})
	= \bracks[\big]{\eps \hker'(\state_{1,\run}) + \hker'(\state_{2,\run})}
		+ \step \parens[\big]{\eps\state_{1,\run} + \state_{2,\run}}
		- \step \inner{\base}{\pvec}.
\end{equation}
Thus, letting $\curr[\chi] = \state_{2,\run} = \state_{1,\run} / \eps$, the above can be rewritten as
\begin{equation}
\eps \hker'(\eps\next[\chi]) + \hker'(\next[\chi])
	= \bracks[\big]{\eps \hker'(\eps\curr[\chi]) + \hker'(\curr[\chi])}
		+ \step (\eps^{2} + 1) \curr[\chi]
		- \step \inner{\base}{\pvec}
\end{equation}
and hence, with $\hker'(\point) = 1 + \log\point$, we finally get
\begin{equation}
\next[\chi]
	= \curr[\chi] \exp\parens*{
			- \step \frac{\eps^{2} + 1}{\eps + 1} \curr[\chi]
			+ \step \frac{\inner{\base}{\pvec}}{\eps + 1}
		}.
\end{equation}
Now, if $\inner{\base}{\pvec} < 0$,
we readily infer that $\curr[\chi] = \state_{2,\run}$ converges linearly to $0$
%as predicted by \cref{thm:sharp}. More exactly, we have that, up to a constant
at a rate of
\begin{equation}
\curr[\chi]
	\sim \exp\parens*{- \frac{\abs{\inner{\base}{\pvec}}}{\eps + 1} \step\run}
\end{equation}
as predicted by \cref{thm:sharp}.
In particular, if $\vecfield(0) = - \base = (\drift, \drift)$
with $\drift > 0$, the iterates of \eqref{eq:MD} converge geometrically to zero with exponent $\step \drift$, which matches the estimate of \cref{thm:sharp} up to a factor of $1/2$ in the exponent.
On the contrary, if $\vecfield(0) = -\base = (\drift, 0)$, the term $\abs{\inner{-\base}{\pvec}} = \eps\drift$ depends on the linear structure of $\points$ and can be arbitrarily bad as $\eps$ goes to zero.
This illustrates why one cannot do away with the dependence on the linear structure of $\points$ when $\vecfield$ is not sharp at $\sol$.
\endenv
\end{remark}
%----------------------------------------------------------------------


%----------------------------------------------------------------------
%%% TECHNICAL PROOFS
%----------------------------------------------------------------------
\subsection{Proof of \cref{thm:sharp}}
\label{sec:proof-sharp}

We now proceed to the proof of \cref{thm:sharp}, beginning with two helper lemmas tailored to the polyhedral structure of $\points$.
%Before starting the proof of the result, we introduce two helpful lemmas linking the steepness of $\hreg$ to the one of $\hker$, and comparing active constraints.
The first is a book-keeping result regarding the subdifferentiability of $\hreg$.

\begin{lemma}
\label{lem:dh}
Let $\hreg$ be a decomposable regularizer on $\points$ with kernel $\hker$ as per \cref{def:decomposable}.
Then the domain of subdifferentiability of $\hreg$ is $\proxdom = \setdef{\point \in \points}{\point_{\coord} \in \dom \subd \hker \text{ for all } \coord = 1,\dots,\nCoords}$ and a continuous selection of $\subd\hreg$ is given by the expression
\begin{equation}
\label{eq:dh}
\nabla \hreg(\point)
	= \sum_{\coord = 1}^{\nCoords} \hker'(\point_{\coord})\bvec_{\coord}
	\quad
	\text{for all $\point\in\proxdom$}.
\end{equation}
%defines a continuous selection of subgradients of $\hreg$ on $\proxdom = \{\point \in \points: \point_{\coord} \in \dom \subd \hker \text{ for } \coord = 1,\dots,\nCoords\}$.
\end{lemma}

\begin{proof}
%This is a consequence of the fact that the subdifferential of a sum of functions is the sum of their subdifferentials if the the relative interiors of their domains intersect
See \citet[Thm~23.8]{Roc70}, whose qualification condition is satisfied thanks to the fact that $\points$ is polyhedral.
\end{proof}

%As a consequence, $\hreg$ is steep if and only if $\hker$ is steep:
%	\begin{equation}
%		\dom \partial \hreg = \relint \points \iff \dom \partial \hker = \intr \dom \hker = (0,+\infty)\,.
%	\end{equation}

The second ingredient we will need is a separation result in the spirit of Farkas' lemma.

\begin{restatable}{lemma}{separation}
\label{lem:separation}
%Suppose that $\points$ is of the general polyhedral form \eqref{eq:polyhedron}.
Let $\points$ be a polyhedral domain of the general form \eqref{eq:polyhedron}.
Then, for all $\sol\in\points$, there exists $\polycst = \polycst(\mat, \cvec, \sol) \geq 1$ such that, for all $\coords \subseteq \actcoords \equiv \actcoords(\sol)$, at least one of the following holds:
\begin{enumerate}
[(\itshape a\upshape)]
\item
\label[case]{itm:inactive}
$\coords \neq \varnothing$ and there exists $\coord \in \actcoords\setminus\coords$ such that
%\begin{align}
%\forall \point \in \points,\,
\(
\point_{\coord}
	\leq \polycst \max\setdef{\point_{\coordalt}}{\coordalt \in \coords}
%	\quad
%	\text{for all $\point\in\points$}.
%\end{align}
\)
for all $\point\in\points$.
\item
\label[case]{itm:active}
There exists $\pvec \in \ker\mat$ such that $\norm{\pvec} \leq \polycst$, $\pvec_{\coord} = 0$ if $\coord \in \coords$ and $\polycst \geq \pvec_{\coord} \geq 1$ if $\coord \in \actcoords \setminus \coords$.
\end{enumerate}
\end{restatable}

The proof of \cref{lem:separation} is based on Farkas' lemma so we relegate it to \cref{app:aux} and instead proceed to use it to prove our main result for linearly constrained problems.

\begin{proof}[Proof of \cref{thm:sharp}]
We will consider two main cases, depending on whether $\lim_{\point\to0^{+}}\hreg'(\point) = -\infty$ (the steep case) or $\lim_{\point\to0^{+}}\hreg'(\point) > -\infty$ (the non-steep case).
The steep regime will cover \cref{asm:ker-log,asm:ker-power} of \cref{asm:ker}, whereas the non-steep regime will account for \cref{asm:ker-Eucl}.

\para{Case 1: the steep regime}
We begin by noting that, without loss of generality, \cref{asm:ker-log,asm:ker-power} respectively imply that there exist $\kernelcst\in\R$ and $\thres>0$ such that, for all $\point\in(0,\thres)$, we have:
\begin{subequations}
\label{eq:ker-growth}
\begin{flalign}
\qquad
\text{Under \cref{asm:ker}\ref{asm:ker-log}:}
	&\;\;\;
	\text{$\hker'(\point) \geq \log\point - \kernelcst$}
	&&
	\\
\text{Under \cref{asm:ker}\ref{asm:ker-power}:}
	&\;\;\;
	\text{$\hker'(\point) \geq -\kernelcst\point^{-\kernelexp}$}
	&&
\end{flalign}
\end{subequations}
With this in mind, let $\radius > 0$ be sufficiently small so that the relative neighborhood $\ball_{\radius} \defeq \setdef{\point\in\points}{\norm{\point - \sol} \leq \radius}$ of $\sol$ in $\points$ satisfies:
\begin{itemize}
\item
$\ball_{\radius} \subseteq \nhd \cap \basin$ with $\basin$ and $\nhd$ defined by \eqref{eq:strong} and \eqref{eq:Breg-nhd} respectively.
\item
If $\point \in \ball_{\radius}$ then $\point_{\coord} < \thres$ for all $\coord \in \actcoords$.
%where $\thres$ is given by \cref{asm:ker grad sublinear} or \cref{asm:ker grad log} of \cref{asm:ker},
\item
If $\point \in \ball_{\radius}$, then $\dnorm{\vecfield(\point) - \vecfield(\sol)} \leq \drift/(2\polycst)$ with $\drift$ given by \eqref{eq:drift} and $\polycst$ given by \cref{lem:separation}.

%\PM{We have not defined $\bregbdedcst$, so maybe it would be better to define $\bregbdedcst$ as the supremum of (\dots) over $\ball_{\radius}$ when we actually need it?}
%\WA{Indeed, implemented}
\end{itemize}

%To proceed, 
Recall that Step 1 of the proof of \cref{thm:general} implies that the iterates $\curr$ and half-iterates $\lead$ of \eqref{eq:AMP} will remain in $\ball_{\radius}$ for all $\run$ %=\running$ 
if $\init$ is initialized sufficiently close to~$\sol$.
Accordingly, with this stability guarantee in hand, we will construct sets $\coords_{\sharp} \subseteq \sharps$, $\coords_{\flat} \subseteq \flats$ such that, for all $\coord \in \coords \defeq \coords_{\sharp} \cup \coords_{\flat}$,% we have
\begin{align}
\label{eq:propCoord}
\state_{\coord,\run}
	\leq \polycst^{\abs{\coords}}
		\cdot \begin{cases}
		\exp\parens*{\kernelcst + \polycst \dnorm{\grad \hreg(\init)} + \polycst \bregbdedcst - \step \drifteff(\run - 1)/2}
			&\quad
			\text{under \cref{asm:ker}\ref{asm:ker-log},}
			\\
		\kernelcst \pospart*{\step\drifteff(\run - 1)/2 - \polycst \dnorm{\grad \hreg(\init)} - \polycst \bregbdedcst}^{-\frac{1}{\kernelexp}}
			&\quad
			\text{under \cref{asm:ker}\ref{asm:ker-power}},
		\end{cases}
\end{align}
where
{$\bregbdedcst \defeq \sup_{\point \in \ball_{\radius}} \dnorm*{ \sum_{\coord \notin \actcoords} \hker'(\point_{\coord}) \bvec_\coord}$, $\polycst \geq 1$ is the constant given by \cref{lem:separation} and
$\drifteff$ is defined as in \eqref{eq:drift-eff} with $\varrho = \polycst\abs{\actcoords}$.
%	\begin{align}
%		\drifteff \defeq
%		\begin{cases}
%			\drift & \text{if the sharpness assumption \cref{assumption:sharpness} holds,} \\
%			\frac{\drift}{\polycst \abs{\actcoords}} & \text{otherwise.}
%		\end{cases}
%	\end{align}
}
%\WA{For reference: this is $< + \infty$ because this concerns only inactive coordinates and because $\hker$ is continuous on the interior of its domain}
%\PM{Eh, whatevs\dots I say, forget about it, if they ask, we'll answer\dots}
%\PM{There was a definition here for $\drift$ which clashed with the original definition in the theorem (basically whether we use $\polycst$ or $\polycst\abs{\actcoords}$ in the denominator;
%can you please check the commented source file?}
%\WA{This was because my constant $\polycst$ differed in the proof and in the theorem, this is explicit now, \ie we take $\varrho \defeq \polycst \abs{\actcoords}$, but this can be removed at the expense of an addtionnal term in the statement of the theorem.}
%\PM{Does the current sentence suit you? If yes, we can view the issue as resolved?}
%\WA{Yep!}
%\PM{This would also be a better place to define $\bregbdedcst$ I think\dots}
%\WA{Done}
%where $\kernelcst$ is as in \cref{asm:ker} and
%	\begin{align}
%		\drifteff \defeq
%		\begin{cases}
%			\drift & \text{if the sharpness assumption \cref{assumption:sharpness} holds,} \\
%			\frac{\drift}{\polycst \abs{\actcoords}} & \text{otherwise.}
%		\end{cases}
%	\end{align}


We will proceed inductively, starting with $\coords_{\sharp} = \coords_{\flat} = \varnothing$, in which case the stated property holds trivially.
For the inductive step, if \eqref{eq:propCoord} holds for $\coords_{\sharp} \subsetneq \sharps$ and $\coords_{\flat} \subseteq \flats$, we will show that there is some index $\coordalt \in \actcoords \setminus \coords$ such that \eqref{eq:propCoord} still holds for $\coords \cup \{\coordalt\}$.
By iterating this procedure, since the number of active constraints is finite, we will reach %a point where
$\coords_{\sharp} = \sharps$ and the result of the theorem will follow.

To carry all this out, assume that $\coords_{\sharp} \subsetneq \sharps$, $\coords_{\flat} \subseteq \flats$, and apply \cref{lem:separation} to $\coords = \coords_{\sharp} \cup \coords_{\flat}$.
If the first case of \cref{lem:separation} holds, then $\coords \neq \varnothing$ and there exists $\coord \in \actcoords$ such that
\begin{equation}
\state_{\coord,\run}
	\leq \polycst \max\nolimits_{\coordalt\in\coords} \state_{\coordalt,\run}
\end{equation}
so \eqref{eq:propCoord} still holds when $\coord$ is appended to $\coords_{\sharp}$ or $\coords_{\flat}$ (depending on whether it belongs to $\sharps$ or $\flats$).
Otherwise, the second case of \cref{lem:separation} holds and there exists some $\pvec \in \ker\mat$ with $\norm{\pvec} \leq \polycst$,
$\pvec_{\coord} = 0$ if $\coord \in \coords$,
and
$\polycst \geq \pvec_{\coord} \geq 1$ if $\coord \in \actcoords \setminus \coords$.
Since $\hreg$ is steep, this means that $\next = \proxof{\curr}{-\step \lead[\signal]}$ belongs to $\proxdom = \relint \points$ for all $\run=\running$, so the normal cone $\ncone(\next)$ to $\points$ at $\next$ will be the affine hull of $\points$, \ie $\ncone(\next) = \row(\mat)$.
Hence, \cref{lem:mirror} guarantees that
\begin{equation}
\label{eq:proof-sharp-mirror-step}
\subsel\hreg(\next)
	- \subsel\hreg(\curr)
	+ \step \lead[\signal]
	\in \row(\mat)
\end{equation}
so, telescoping over $\runalt = \start$ to $\run-1$, we get
\begin{equation}
\subsel\hreg(\curr)
	- \subsel\hreg(\init)
	+ \step \sum_{\runalt = \start}^{\run-1} \iterlead[\signal]
	\in \row(\mat).
\end{equation}
Taking the scalar product with $\pvec \in \ker\mat$ then yields
\begin{equation}
\sum_{\coord=1}^{\nCoords} \hker'(\state_{\coord,\run}) \pvec_{\coord}
	= \inner{\subsel\hreg(\init)}{\pvec}
		- \step \sum_{\runalt=\start}^{\run - 1} \inner{\iterlead[\signal]}{\pvec}
\end{equation}
so, after rearranging and invoking \eqref{eq:slacks} to write $\inner{\vecfield(\sol)}{\pvec} = \sum_{\coord \in \actcoords} \slack_{\coord} \pvec_{\coord}$, we get
\begin{align}
\sum_{\coord \in \actcoords \setminus \coords} \hker'(\state_{\coord,\run}) \pvec_{\coord}
	+ \sum_{\coord \in \coords} \hker'(\state_{\coord,\run}) \pvec_{\coord}
	&= \inner{\subsel\hreg(\init)}{\pvec}
		- \step \sum_{\runalt=\start}^{\run - 1} \sum_{\coord \in \actcoords} \slack_{\coord} \pvec_{\coord}
	\notag\\
	&+\step \sum_{\runalt = \start}^{\run - 1} \inner{\vecfield(\sol) - \iterlead[\signal]}{\pvec}
		- \sum_{\coord \notin \actcoords} \hker'(\state_{\coord,\run}) \pvec_{\coord}.
\end{align}
Now, by the properties we used to construct $\pvec$,
%($\norm{\pvec} \leq \polycst$, $\pvec_{\coord} = 0$ if $\coord \in \coords$ and $\polycst \geq \pvec_{\coord} \geq 1$ if $\coord \in \actcoords \setminus \coords$ as stated above),
we further have
%\PM{Gents, if I'm not mistaken $\sum_{\coord \notin \actcoords} \hker'(\state_{\coord,\run})$ is not a vector ;-)
%But anyway, to avoid lugging it around, maybe we give it a name and stop writing the sum everywhere?}
%\WA{Fixed. As for giving it a name, for me we introduc enough notations already, but as you prefer.}
\begin{align}
\label{eq:proof-sharp-template-ineq}
\sum_{\mathclap{\coord\in\actcoords \setminus \coords}} \hker'(\state_{\coord,\run}) \pvec_{\coord}
	& \leq \polycst \dnorm{\subsel\hreg(\init)}
		- \step\parens{\run-1} \sum_{\mathclap{\coord \in \actcoords \setminus \coords}} \slack_{\coord} \pvec_{\coord}
	\notag\\
	&\qquad
		+ \step\sum_{\runalt=\start}^{\run - 1} \polycst \dnorm{\vecfield(\sol) - \iterlead[\signal]}
		+ \polycst \dnorm*{\sum_{\coord \notin \actcoords} \hker'(\state_{\coord,\run}) \bvec_\coord}
	\notag\\
	& \leq \polycst \dnorm{\subsel\hreg(\init)}
		- \step (\run - 1) \sum_{\mathclap{\coord \in \actcoords \setminus \coords}} \slack_{\coord} \pvec_{\coord}
		+ \tfrac{1}{2} \step \drift (\run - 1)
		+ \polycst \bregbdedcst.
\end{align}
where the second inequality follows from how we chose $\ball_{\radius}$ at the beginning of the proof.

We conclude our analysis by distinguishing whether $\vecfield$ is sharp at $\sol$ (\ie if $\flats = \varnothing$ or not).
\begin{enumerate}
[left=0em,label={\bfseries Case \arabic*:}]
\item
If $\flats=\varnothing$, we have $\actcoords \setminus \coords = \sharps \setminus \coords_{\sharp}$ and $\slack_{\coord} \geq \drift$ for all $\coord \in \actcoords \setminus \coords$, so \eqref{eq:proof-sharp-template-ineq} gives
\begin{equation}
\sum_{\coord \in \actcoords \setminus \coords} \hker'(\state_{\coord,\run}) \pvec_{\coord} + \step (\run - 1) \drift \pvec_{\coord}
	\leq \polycst \dnorm{\subsel\hreg(\init)}
		+ \tfrac{1}{2} \step\drift(\run-1)
		+ \polycst \bregbdedcst.
\end{equation}
Choosing the coordinate $\coordalt \in \actcoords \setminus \coords$ which corresponds to the smallest term in the sum on the \ac{LHS}, we obtain that
\begin{align}
	\parens*{\hker'(\state_{\coordalt,\run}) + \step (\run - 1) \drift} (\abs{\actcoords} \setminus \coords) \pvec_{\coordalt}
	\leq
	\polycst \dnorm{\subsel\hreg(\init)}
	+ \tfrac{1}{2} \step\drift(\run - 1)
	+ \polycst \bregbdedcst
\end{align}
and noting that $\abs{\actcoords\setminus \coords} \pvec_{\coordalt} \geq 1$ yields
%\PM{Could you check this line to verify everything is aok? [I did a stupid change at some point and I'm not sure if I reverted things properly]}
%\WA{Checked}
%\WA{There is a very small case that we are brushing under the carpet here. This reasoning is valid only if $\hker'(\state_{\coordalt,\run}) + \step (\run - 1) \drift$ is non-negative. If it's negative, \cref{eq:proof-sharp-final-ineq1} still holds (trivially) though.}
\begin{align}
 \label{eq:proof-sharp-final-ineq1}
	\hker'(\state_{\coordalt,\run})
	\leq
	\polycst \dnorm{\subsel\hreg(\init)}
	-\tfrac{1}{2} \step\drift(\run - 1)
	+ \polycst \bregbdedcst\,.
\end{align}

\item
If $\flats \neq \varnothing$, then, since $\coords_{\sharp} \subsetneq \sharps$, the intersection of $\actcoords \setminus \coords$ and $\sharps$ is not empty so that, $\sum_{\coord \in \actcoords \setminus \coords} \slack_{\coord} \pvec_{\coord} \geq \sum_{\coord \in \actcoords \setminus \coords} \slack_{\coord} \geq \drift$ and the inequality \eqref{eq:proof-sharp-template-ineq} above becomes,
\begin{align}
	\sum_{\coord \in \actcoords \setminus \coords} \hker'(\state_{\coord,\run}) \pvec_{\coord}
	\leq
	\polycst \dnorm{\subsel\hreg(\init)}
	- \tfrac{1}{2} \step\drift(\run - 1)
	+ \polycst \bregbdedcst\,.
\end{align}

Now, choosing $\coordalt$ to be the coordinate in $\actcoords \setminus \coords$ which minimizes the \ac{LHS} and bounding $\abs{\actcoords \setminus \coords}$ by $1$ and $\abs{\actcoords}$, we get that
\begin{align}
\hker'(\state_{\coordalt,\run}) \pvec_{\coordalt}
	\leq
	\polycst \dnorm{\subsel\hreg(\init)}
	-\frac{\step \drift}{2\abs{\actcoords}} (\run - 1)
	+ \polycst \bregbdedcst\,.
\end{align}
Dividing both sides by $\pvec_{\coordalt}$ and using that it lies between $1$ and $\polycst$ gives
\begin{align}
 \label{eq:proof-sharp-final-ineq2}
	\hker'(\state_{\coordalt,\run})
	\leq
	\polycst \dnorm{\subsel\hreg(\init)}
	-\frac{\step \drift}{2 \polycst \abs{\actcoords}} (\run - 1)
	+ \polycst \bregbdedcst\,
\end{align}
\end{enumerate}
Therefore, we have shown that, in both cases, there exists $\coordalt \in \actcoords \setminus \coords$ such that \eqref{eq:proof-sharp-final-ineq2} holds, since $\polycst \abs{\actcoords} \geq 1$.
%\PM{Don't understand this sentence \textendash\ or, rather, I think I understand what you mean, but some polishing is in order.}
Therefore, combining this inequality with \eqref{eq:ker-growth} we conclude that \eqref{eq:propCoord} holds for $\coordalt$,
and 
%\PM{The bound was restated here (commented in the source), not sure why\dots was this intentional?}
%\WA{It's because it was the bound of \eqref{eq:propCoord} \emph{without} the term $\polycst^{\abs{\coords}}$, and it implies \eqref{eq:propCoord} since $\polycst \geq 1$.}
%\begin{align}
%\state_{\coordalt,\run}
%	\leq \begin{cases}
%		\exp\parens*{\kernelcst + \polycst \dnorm{\grad \hreg(\init)} + \polycst \bregbdedcst - \frac{\step \drifteff}{2}(\run - 1)}
%			&\quad
%			\text{ if \cref{asm:ker grad log} of \cref{asm:ker} holds,} \\
%		\kernelcst \pospart*{
%			\frac{\step \drifteff }{2}(\run - 1)
%			- \polycst \dnorm{\grad \hreg(\init)} - \polycst \bregbdedcst
%		}^{-\frac{1}{\kernelexp}}
%		& \text{ if \cref{asm:ker grad sublinear} of \cref{asm:ker} holds.}
%	\end{cases}
%\end{align}
since $\polycst \geq 1$,
%\PM{There's an issue here with what we've actually defined the constant to be.}
we conclude that we can augment $\coords_{\sharp}$ or $\coords_{\flat}$ by $\coordalt$, depending on whether it belongs to $\sharps$ or $\flats$.%respectively.
%\WA{Implicitely, I mean that ``the bound \eqref{eq:propCoord} still holds for the other coordinates that we had already put in $\coords$'', is it clear?}
%\PM{Yep.}
%\WA{Thx}

\para{Case 2: the non-steep regime}
The proof borrows the structure of the first case, though it is more straightforward.

	Take $\radius > 0$ small enough such that $\ball_{\radius} \defeq \{ \point\in\points : \norm{\point - \sol} \leq \radius \}$ satisfies
	\begin{itemize}
		\item $\ball_{\radius}$ is included in $\nhd \cap \basin$,
		\item if $\point, \pointalt \in \ball_{\radius}$ then, $\dnorm{\subsel\hreg (\point) - \subsel\hreg(\pointalt)} \leq \frac{\step \drift}{3 \polycst}$ where $\polycst$ is given by \cref{lem:separation}. This is possible since $\subsel\hreg$ is continuous at $\sol$.
%\WA{The radius $r$ depends on $\step$, this is not very nice but it is the simplest way. The precise way would be to use the uniform continuity of $\subsel\hreg$ near $\sol$...}
		\item if $\point \in \ball_{\radius}$, then $\dnorm{\vecfield(\point) - \vecfield(\sol)} \leq \frac{\drift}{3 \polycst}$.
 \item No other constraint $\point_{\coord} = 0$ with $\coord \notin \actcoords$ becomes active in $\ball_{\radius}$.
	\end{itemize}
	As we have seen in the stability part of the proof of \cref{thm:general}, if \eqref{eq:AMP} is started close enough to~$\sol$, then all the iterates $\curr$ and the half-iterates $\lead$ for $\runalt=\running$ are contained in $\ball_{\radius}$.


 %In this case, we will show that taking $\nRuns = \afterstart$ is enough to get the guarantee of the theorem.
 %\FI{??}\WA{Yep, I don't know either}
 As above, fix some $\run \geq \nRuns$.
 We will build sets $\coords_{\sharp} \subseteq \sharps$, $\coords_{\flat} \subseteq \flats$ with the property that that
 \begin{align}
 \label{eq:propertynonsteep}
 \text{ for all } \coord \in \coords_{\sharp} \cup \coords_{\flat}, \state_{\coord,\run} = 0 \,.
 \end{align}

	Starting with $\coords_{\sharp} = \coords_{\flat} = \varnothing$, \cref{eq:propertynonsteep} is trivially verified.
	Now, take $\coords_{\sharp} \subsetneq \sharps$, $\coords_{\flat} \subseteq \flats$ which satisfy the desired property and, as before, apply \cref{lem:separation} with $\coords \defeq \coords_{\sharp} \cup \coords_{\flat}$.
If the first case of \cref{lem:separation} holds, then $\coords \neq \varnothing$ and there exists $\coordalt \in \actcoords$ such that,
\begin{align}
	\state_{\coordalt,\run} & \leq \polycst \max\parens*{\state_{\coord,\run} : \coord \in \coords}\,,
\end{align}
which yields the result by adding $\coord$ to $\coords_{\sharp}$ or $\coords_{\flat}$ depending whether it belongs to $\sharps$ or $\flats$ .
Otherwise, if the second case holds, there is some $\pvec \in \ker\mat$ such that $\norm{\pvec} \leq \polycst$, $\pvec_{\coord} = 0$ if $\coord \in \coords$ and $\polycst \geq \pvec_{\coord} \geq 1$ if $\coord \in \actcoords \setminus \coords$.
For the sake of contradiction, assume that for all $\coord \in \actcoords \setminus \coords$, $\state_{\coord,\run} > 0$. Showing that this results in a contradiction will give us an additional coordinate $\coordalt \in \actcoords \setminus \coords$ for which $\state_{\coordalt,\run} = 0$ that we will then add to $\coords_{\sharp}$ or $\coords_{\flat}$ as in the first case.

 Now, let us determine the normal cone at $\curr$. %Thanks to this hypothesis and 
 Since $\curr$ belongs to $\ball_\radius^{\points}(\sol)$, no other constraint other than the ones corresponding to $\coords$ can become active, and these constraints are actually active by the definition of $\coords$ and \cref{eq:propertynonsteep}. Hence, the normal cone at $\curr$ (see \cref{eq:ncone-sharp}) becomes%\FI{W: Say why we have $ \coords$ and not $\actcoords$}\WA{Too much now or is it ok?}
\begin{align}
	\ncone(\curr) &= \left\{-\sum_{\coord \in \coords} \slack_{\coord} \bvec_{\coord} : (\slack_{\coord})_{\coord \in \coords} \in (\R_+)^{\coords}\right\} + \row(\mat)\,,
\end{align}
and taking a scalar product between the last inclusion of \cref{lem:mirror} and $\pvec$, we get that,
\begin{align}
	\inner*{ \subsel\hreg(\curr) - \subsel\hreg(\prev) - \step \beforelead[\signal]}{\pvec} = 0\,.
\end{align}
This means that,
\begin{align}
	\step\inner{\vecfield(\sol)}{\pvec} &=
	\inner*{ \subsel\hreg(\curr) - \subsel\hreg(\prev)}{\pvec} + \step\inner{\vecfield(\sol) - \step \beforelead[\signal]}{\pvec}
	\leq
	\frac{2\step \drift}{3}\label{eq:proof-sharp-eucl-final-ineq}\,
\end{align}
 where the last inequality comes from our definition of $\ball_\radius^{\points}(\sol)$.
However, by \eqref{eq:slacks} and the properties of $\pvec$, we also have %\FI{W: Say why we have $\actcoords \setminus \coords$ and not $\actcoords$}\WA{Is it enough?}
\begin{align}
	\inner{\vecfield(\sol)}{\pvec} &=
	\sum_{\coord \in \actcoords \setminus \coords} \slack_{\coord} \pvec_{\coord} \geq \drift\,
\end{align}
%since $\coords_{\sharp} \subsetneq \sharps$.
which is in contradiction with \eqref{eq:proof-sharp-eucl-final-ineq}.
We may therefore iteratively add coordinates of $\actcoords$ for which $ \state_{\coord,\run} = 0$, which completes the induction and our proof.
\end{proof}



%----------------------------------------------------------------------
%%% DISCUSSION
%----------------------------------------------------------------------
\section{Concluding remarks}
\label{sec:discussion}
%----------------------------------------------------------------------
%%% DISCUSSION
%----------------------------------------------------------------------
% !TEX root = ../Main.tex


Our results indicate that Euclidean regularization leads to faster trajectory convergence rates near \ac{SOS} solutions.
While this does not contradict the analysis of \cite{Nem04} \textendash\ which concerns the method's ergodic average and advocates the use of non-Euclidean regularizers in domains with a favorable geometry \textendash\ it \emph{does} run contrary to its spirit.
We attribute the source of this discrepancy
%(at least in the non-sharp case) 
to the fact that Lipschitz continuity and second-order sufficiency are both norm-based conditions, so it is plausible to expect that norm-based regularizers would lead to better results.
This raises the question of what the corresponding rate analysis would give in the case of Bregman-based variants of \eqref{eq:Lipschitz} and \eqref{eq:strong}, \eg as in the recent works of \cite{BDX11,BBT17,LFN18,ABM19,ABM20,AM21,ABM21}.
We defer this analysis to future work.


%----------------------------------------------------------------------
%%% ACKNOWLEDGMENTS
%----------------------------------------------------------------------
\section*{Acknowledgments}
\begingroup
\small
%----------------------------------------------------------------------
%%% THANKS
%----------------------------------------------------------------------
% !TEX root = ./Main.tex
%
%
The authors gratefully acknowledge financial support by
the French National Research Agency (ANR) in the framework of
the ``Investissements d'avenir'' program (ANR-15-IDEX-02),
the LabEx PERSYVAL (ANR-11-LABX-0025-01),
MIAI@Grenoble Alpes (ANR-19-P3IA-0003),
and the bilateral ANR-NRF grant ALIAS (ANR-19-CE48-0018-01).
Part of this work was done while the first author was visiting the Simons Institute for the Theory of Computing.
\endgroup



%**********************************************************************
%***    APPENDICES
%**********************************************************************
\numberwithin{lemma}{section}		% for numbering  in the appendix
\numberwithin{proposition}{section}		% for numbering  in the appendix
\numberwithin{equation}{section}		% for numbering in the appendix
\appendix


%----------------------------------------------------------------------
%%% APP: AUXILIARY
%----------------------------------------------------------------------
\section{Auxiliary results}
\label{app:aux}
%----------------------------------------------------------------------
%%% APP: AUX
%----------------------------------------------------------------------
% !TEX root = ../Main.tex
%
%
We provide here a series of basic properties, helper lemmas and auxiliary results that we use repeatedly in our paper.

%----------------------------------------------------------------------
%%% NUMERICAL SEQUENCES
%----------------------------------------------------------------------
\subsection{Lemmas on numerical sequences}
The first two results %we provide 
concern numerical sequences.

\begin{lemma}
%[\citeauthor{Pol87}, \citeyear{Pol87}, p.~46]
%[\citeauthor{Pol87}]
\label{lem:Polyak}
Consider two sequences of %non-negative 
real numbers $\curr[\seq], \curr[\diff] \geq 0$, $\run=\running$, such that
\begin{equation}
\next[\seq]
	\leq \curr[\seq]
		- \curr[\diff] \curr[\seq]^{1+\rexp}
	\quad
	\text{for some $\rexp>0$ and all $\run=\running$}
\end{equation}
Then, for all $\run=\running$, we have:%
%\PM{Shouldn't the sum go to $\run-1$ instead of $\run$?
%I changed the $\run$ to $\run+1$ preemptively, but please check\dots}
%\WA{Checked}
\begin{equation}
\next[\seq]
	\leq \frac{\init[\seq]}{\parens*{1 + \rexp \init[\seq]^{\rexp} \sum_{\runalt=\start}^{\run} \iter[\diff]}^{1/\rexp}}.
\end{equation}
\end{lemma}

\begin{proof}
See \cite[p.~46, Lem.~6]{Pol87}.
\end{proof}

The second result that we prove here is \cref{lem:basicnum}, a slight variant of the above lemma, which we restate below for convenience.

\basicnum*

%\begin{lemma}
%\label{lem:basicnum-app}
%Suppose that $\fn\from\R_+\to\R_+$ admits the asymptotic expansion
%\begin{equation}
%\fn(\point)
%	= \point
%		- \coef\point^{1+\rexp}
%		+ o(\point^{1+\rexp})
%	\quad
%	\text{as $\point\to0$}
%\end{equation}
%for positive constants $\coef,\rexp>0$.
%Then, for $\init[\seq] > 0$ small enough, the sequence $\next[\seq] = \fn(\curr[\seq])$, $\run=\running$, converges to $0$ at a rate of $\curr[\seq] \sim (\coef\rexp\run)^{-1/\rexp}$. % for some $\const>0$.
%\end{lemma}

\begin{proof}
By the assumption on $\fn$, there exists some $\eps > 0$ such that
%for $\point \in [0, \eps]$,
\begin{equation}
\point - 2\coef \point^{1+\rexp}
	\leq \fn(\point)
	\leq \point - \half[\coef] \point^{1+\rexp}
	\quad
	\text{for all $\point \in [0,\eps]$}.
\end{equation}
%As a first consequence, 
Note first that, if $\init[\seq] \leq \eps$, \cref{lem:Polyak} readily implies that $\curr[\seq]$ converges to $0$ and that $\curr[\seq] \leq \eps$ for all\;$\run$.
%\PM{A minor point, but how does \cref{lem:Polyak} imply anything about positivity?
%Should the order of the two phrases here be inverted?}
%\WA{Indeed, the non-negativeness comes by definition of the sequence, I removed it}
Moreover, if $\eps$ is small enough so that $1 -2\coef \eps^\rexp > 0$ and $\init[\seq]$ is positive, this implies that all $\curr[\seq]$, for $\run = \running$, are positive.
In particular, %it is then valid to 
we consider the sequence $\curr[\seq]^{-\rexp}$, $\run = \running$, for which we\;get
\begin{align}
\next[\seq]^{-\rexp} - \curr[\seq]^{-\rexp}
	&= \bracks[\big]{\curr[\seq] - \coef \curr[\seq]^{1+\rexp} + o\parens[\big]{\curr[\seq]^{1+\rexp}}}^{-\rexp}
		- \curr[\seq]^{-\rexp}
	\notag\\
	&= \curr[\seq]^{-\rexp}(1 - \coef \curr[\seq]^{\rexp} + o\left(\curr[\seq]^{\rexp}\right))^{-\rexp}
		- \curr[\seq]^{-\rexp}
%	\notag\\
	= \rexp \coef + o\left(1\right)\,.
\end{align}
Hence, $\curr[\seq]^{-\rexp} \sim \rexp \coef \run$ which gives the result.
\end{proof}


%----------------------------------------------------------------------
%%% BREGMAN DIV
%----------------------------------------------------------------------
\subsection{Properties of Bregman divergences and the induced prox-mappings}
We recall some basic properties of the Bregman divergence and the induced prox-mapping.
Variants of these properties are fairly well known in the literature, so we omit their proofs and we refer the interested reader to \cite{BecTeb03,JNT11,MZ19,MLZF+19,DMSV23,MHC24} and references therein for a more detailed discussion.
In particular, \cref{lem:mirror,lem:threepoint,lem:proxlip} correspond to Lemmas B.1, B.2 and B.4(a) of \cite{MLZF+19}, respectively.

\begin{restatable}{lemma}{mirror}
\label{lem:mirror}
Let $\hreg$ be a Bregman regularizer on $\points$ and let $\subsel\hreg$ be a continuous selection of $\subd\hreg$.
Then, for all $\point\in\proxdom$, $\new\in\points$ and $\dvec\in\dpoints$, we have:
\begin{subequations}
\begin{flalign}
\quad
	a)\;\;
	&\subd\hreg(\point)
	= \subsel\hreg(\point)
		+ \pcone(\point)
	\\
\quad
	b)\;\;
	&\new
	= \proxof{\point}{\dvec}
	\iff
\subsel\hreg(\point) + \dvec
	\in \subd\hreg(\new)
	\iff
\subsel\hreg(\new) - \subsel\hreg(\point)
	\in \dvec - \pcone(\new)
\end{flalign}
\end{subequations}
where $\pcone(\point) = \setdef{\dvec\in\dpoints}{\braket{\dvec}{\pointalt - \point} \leq 0 \; \text{for all $\pointalt\in\points$}}$ denotes the polar cone to $\points$ at $\point$.
%In particular, \eqref{eq:AMP} is well-posed:
%$\new = \proxof{\point}{\dvec}$ implies that $\new\in\proxdom$.
\end{restatable}

\begin{restatable}[$3$-point identity]{lemma}{threepoint}
\label{lem:threepoint}
%Let $\hreg$ be a Bregman regularizer on $\points$.
For all $\base\in\points$ and all $\point,\new\in\proxdom$, we have:
\begin{equation}
\label{eq:threepoint}
\breg(\base,\new)
	= \breg(\base,\point)
		+ \breg(\point,\new)
		+ \braket{\subsel\hreg(\new) - \subsel\hreg(\point)}{\point - \base}
\end{equation}
\end{restatable}

\begin{restatable}[Non-expansiveness]{lemma}{proxlip}
\label{lem:proxlip}
For all $\point\in\proxdom$ and all $\dvec,\new[\dvec]\in\dpoints$ we have:
\begin{equation}
	\norm{\proxof{\point}{\new[\dvec]} - \proxof{\point}{\dvec}}
	\leq \dnorm{\new[\dvec] - \dvec}
\end{equation}
\end{restatable}


The next two results that we provide consider the evolution of the Bregman divergence before and after a prox step (or two);
they are both adapted from \cite[Proposition B.3]{MLZF+19}, with the added proviso that $\hreg$ is assumed to be $1$-strongly convex on $\nhdalt$ (as per \cref{rem:Bregman}).

\begin{lemma}
\label{lem:onestep-app}
Let $\new = \proxof{\point}{\dvec}$ for $\point\in\proxdom$, $\dvec\in\dpoints$ such that $\new$ is still in $\nhdalt$.
Then, for all $\base\in\points$, %and all 
$\dbase\in\pcone(\base)$, we have:\!\!
\begin{subequations}
\begin{align}
\breg(\base,\new)
	&\leq \breg(\base,\point)
		+ \braket{\dvec - \dbase}{\new - \base}
		- \breg(\new,\point)
		\\
	&\leq \breg(\base,\point)
		+ \braket{\dvec - \dbase}{\point - \base}
		+ \tfrac{1}{2} \dnorm{\dvec - \dbase}^{2}\,.
\end{align}
\end{subequations}
\end{lemma}

\begin{proof}
Our proof follows \cite[Proposition B.3]{MLZF+19}, but with a slight modification to account for the extra term %involving 
with $\dbase\in\pcone(\base)$.
The first step is to invoke the three-point identity \eqref{eq:threepoint} to write
\begin{equation}
\breg(\base,\point)
	= \breg(\base, \new)
		+ \breg(\new, \point)
		+ \braket{\subsel\hreg(\point) - \subsel\hreg(\new)}{\new - \base}.
\end{equation}
Then, after rearranging to isolate $\breg(\base,\new)$, we get
\begin{align}
\breg(\base,\new)
	&= \breg(\base, \point)
		- \breg(\new, \point)
		- \braket{\subsel\hreg(\point) - \subsel\hreg(\new)}{\new - \base}
	\notag\\
	&\leq \breg(\base,\point)
		- \breg(\new, \point)
		+ \braket{\dvec}{\new - \base}
\end{align}
where the inequality in the last line follows from \cref{lem:mirror}.
Hence, given that $\braket{\dbase}{\new - \base} \leq 0$ by the fact that $\dbase\in\pcone(\base)$, we readily obtain
\begin{equation}
\label{eq:onestep-proof}
\breg(\base, \new)
	\leq  \breg(\base, \point)
		- \breg(\new, \point)
		+ \braket{\dvec - \dbase}{\new - \base}\,.
\end{equation}

For the second inequality of the lemma, note that 
\begin{align}
\braket{\dvec - \dbase}{\new - \base}
	&= \braket{\dvec - \dbase}{\point - \base}
		+ \braket{\dvec - \dbase}{\new - \point}
	\notag\\
	&\leq \braket{\dvec - \dbase}{\point - \base}
		+ \tfrac{1}{2} \dnorm{\dvec - \dbase}^{2}
		+ \tfrac{1}{2} \norm{\new - \point}^{2}
	\notag\\
	&\leq \braket{\dvec - \dbase}{\point - \base}
		+ \tfrac{1}{2} \dnorm{\dvec - \dbase}^{2}
		+ \breg(\new,\point)
\end{align}
where the penultimate inequality follows directly from Young's inequality and the last one from \eqref{eq:Breg-lower} and the fact that $\point$ and $\new$ both lie in $\nhdalt$.
Our assertion is then obtained by combining this last bound with \eqref{eq:onestep-proof}.
\end{proof}

\begin{lemma}
\label{lem:twostep-app}
Let $\new_{i} = \proxof{\point}{\dvec_{i}}$ for some $\point\in \nhdalt \cap \proxdom$ and $\dvec_{i}\in\dpoints$ such that $\new_i \in \nhdalt$, $i=1,2$.
Then, for all $\base\in\points$ and all $\dbase\in\pcone(\base)$, we have:
\begin{equation}
\breg(\base,\new_{2})
	\leq \breg(\base,\point)
		+ \braket{\dvec_{2} - \dbase}{\new_{1} - \base}
		+ \tfrac{1}{2} \dnorm{\dvec_{2} - \dvec_{1} - \dbase}^{2}
		- \tfrac{1}{2} \norm{\new_{1} - \point}^{2}.
\end{equation}
\end{lemma}

\begin{proof}
Our proof follows \citep[Proposotion B.4]{MLZF+19}, again with a slight modification to account for the extra terms with %involving 
$\dbase\in\pcone(\base)$.
Specifically, applying \cref{lem:onestep} with $\new_{2} = \mprox_\point(\dpoint_{2})$ and $\dbase \in \pcone(\base)$ gives
\begin{align}
\label{eq:twostep-proof}
\breg(\base,\new_{2})
	&\leq \breg(\base,\point)
		+ \braket{\dpoint_{2} - \dbase}{\new_{2} - \base}
		- \breg(\new_{2}, \point)
	\notag\\
	&\leq \breg(\base, \point)
		+ \braket{\dpoint_{2} - \dbase}{\new_{1} - \base}
		+ \braket{\dpoint_{2} - \dbase}{\new_{2} - \new_{1}}
		- \breg(\new_{2},\point)
\end{align}
To lower bound $\breg(\new_{2}, \point)$, we use again \cref{lem:onestep} with $\base \gets \new_{2}$ and $\new_{1} = \mprox_\point(\dpoint_{1})$.
This readily gives
\begin{equation}
\breg(\new_{2},\new_{1})
	\leq \breg(\new_{2},\point)
		+ \braket{\dpoint_{1}}{\new_{1} - \new_{2}}
		- \breg(\new_{1},\point)
\end{equation}
and hence,
%\begin{equation}
%\breg(\new_{2},\point)
%	\geq \breg(\new_{2},\new_{1})
%		+ \braket{\dpoint_{1}}{\new_{2} - \new_{1}}
%		+ \breg(\new_{1},\point).
%\end{equation}
after rearranging the above to isolate $\breg(\new_{2},\point)$ and substituting the resulting bound in \eqref{eq:twostep-proof}, we get
\begin{equation}
\breg(\base,\new_{2})
	\leq \breg(\base,\point)
		+ \braket{\dpoint_{2} - \dbase}{\new_{1} - \base}
		+ \braket{\dpoint_{2} - \dpoint_{1} - \dbase}{\new_{2} - \new_{1}} 
		- \breg(\new_{2},\new_{1})
		- \breg(\new_{1}, \point).
\end{equation}
Thus, by Young's inequality and the local strong convexity of $\hreg$, we finally obtain
%To complete the proof, we use Young's inequality and twice the strong convexity of $\hreg$ \cref{lem: strg cvx h} to get,
\begin{align}
\breg(\base,\new_{2})
	&\leq \breg(\base, \point)
		+ \braket{\dpoint_{2} - \dbase}{\new_{1} - \base}
		+ \tfrac{1}{2}\dnorm{\dpoint_{2} - \dpoint_{1} - \dbase}^{2}
	\notag\\
	&\qquad
		+ \tfrac{1}{2}\norm{\new_{2} - \new_{1}}^{2}
		- \tfrac{1}{2}\norm{\new_{2}-\new_{1}}^{2}
		- \tfrac{1}{2}\norm{\new_{1}-\point}^{2}
	\notag\\
	&\leq \breg(\base,\point)
		+ \braket{\dpoint_{2} - \dbase}{\new_{1} - \base}
			+ \tfrac{1}{2}\dnorm{\dpoint_{2} - \dpoint_{1} - \dbase}^{2}
			- \tfrac{1}{2}\norm{\new_{1}-\point}^{2}
\end{align}
and our proof is complete.
\end{proof}





%----------------------------------------------------------------------
%% SEPARATION
%----------------------------------------------------------------------
\subsection{Legendre exponent for interior points}

We now proceed to provide a more formal footint to our discussion in \Cref{sec:Legendre} regarding the fact that $\legof{\base}=0$ whenever $\base$ is an interior point.
The formal statement is as follows.

\begin{lemma}
\label{lem:Leg-proxdom}
Suppose that $\subsel\hreg$ is locally Lipschitz continuous.
Then $\legof{\base} = 0$ for all $\base \in \proxdom$;
in particular, $\legof{\base} = 0$ whenever $\base\in\relint\points$.
\end{lemma}

\begin{proof}
Fix some $\base \in \proxdom$ and suppose that $\subsel\hreg$ is locally Lipschitz continuous.
%selection of $\subd\hreg$.
Then there exists a neighborhood $\legnhd$ of $\base$ in $\points$ and some $\legconst > 0$ such that
\begin{equation}
\label{eq:hLip}
\dnorm{\nabla \hreg(\base) - \nabla \hreg(\point)}
	\leq \legconst \norm{\base - \point}
	\quad
	\text{for all $\point \in \legnhd \cap \proxdom$}.
\end{equation}
Now, since $\nabla \hreg(\base) \in \subd \hreg(\base)$,
we also have
\begin{align}
\breg(\base,\point)
	= \hreg(\base) - \hreg(\point) - \braket{\nabla \hreg(\point)}{\base - \point}
%	\notag\\
	&\leq \braket{\nabla \hreg(\base) - \nabla \hreg(\point)}{\base - \point}
	\notag\\
	&\leq \dnorm{\nabla \hreg(\base) - \nabla \hreg(\point)} \norm{\base - \point}
%	\notag\\
	\leq \legconst \norm{\base - \point}^{2}
\end{align}
for all $\point\in\legnhd \cap \proxdom$.
This shows that \eqref{eq:Breg-local} holds with $\legexp=0$, \ie $\legof{\base}=0$.
\end{proof}




%----------------------------------------------------------------------
%%% SEPARATION
%----------------------------------------------------------------------
\subsection{A separation result}

We now proceed to prove \cref{lem:separation}, which we restate below for convenience:

\separation*

%\begin{lemma}
%\label{lem:separation-app}
%%Suppose that $\points$ is of the general polyhedral form \eqref{eq:polyhedron}.
%Let $\points$ be a polyhedral domain of the %general 
%form \eqref{eq:polyhedron}.
%Then, for all $\sol\in\points$, there exists $\polycst = \polycst(\mat, \cvec, \sol) \geq 1$ such that, for all $\coords \subseteq \actcoords \equiv \actcoords(\sol)$, at least one of the following holds:
%\begin{enumerate}
%[(\itshape a\upshape)]
%\item
%\label[case]{itm:inactive}
%$\coords \neq \varnothing$ and there exists $\coord \in \actcoords\setminus\coords$ such that
%%\begin{align}
%%\forall \point \in \points,\,
%\(
%\point_{\coord}
%	\leq \polycst \max\setdef{\point_{\coordalt}}{\coordalt \in \coords}
%%	\quad
%%	\text{for all $\point\in\points$}.
%%\end{align}
%\)
%for all $\point\in\points$.
%\item
%\label[case]{itm:active}
%There exists $\pvec \in \ker\mat$ such that $\norm{\pvec} \leq \polycst$, $\pvec_{\coord} = 0$ if $\coord \in \coords$ and $\polycst \geq \pvec_{\coord} \geq 1$ if $\coord \in \actcoords \setminus \coords$.
%\end{enumerate}
%\end{lemma}


\begin{proof}
Our claim is trivial if $\coords = \actcoords$, so we will focus exclusively on the case $\coords \subsetneq \actcoords$.
The stated constant $\polycst = \polycst(\mat, \cvec, \sol)$ will then be obtained as the maximum of $1$ and the constants we obtain for each possible $\coords \subsetneq \actcoords$.
 
The proof consists in discussing whether there exists 
 $(\coef_\coord)_{\coord \in \actcoords\setminus\coords} \in (\R_{+})^{\actcoords\setminus \coords}$ \emph{not all zero} and $(\coefalt_\coord)_{\coord \in \coords} \in \R^{\coords}$ such that the inclusion
%\PM{I do not understand this paragraph (and the implications in the rest of the proof are not very clear either). Didn't touch.}
%\WA{I tried to rewrite a bit, let me know if it's clearer...}
\begin{align}
\points
	\subseteq \left\{\point \in \R^\nCoords : \sum_{\coord \in \actcoords \setminus\coords} \coef_\coord \point_\coord =\sum_{\coord \in \coords}\coefalt_\coord \point_\coord \right\} \label{eq:inclusion}
\end{align}
holds.
\Cref*{itm:inactive} considers the case when such coefficients exist, while \cref*{itm:active} considers when this is not possible.


\para{\cref*{itm:inactive}}
Assume that there exists $(\coef_\coord)_{\coord \in \actcoords\setminus\coords} \in (\R_{+})^{\actcoords\setminus \coords}$ \emph{not all zero} and $(\coefalt_\coord)_{\coord \in \coords} \in \R^{\coords}$ such that \eqref{eq:inclusion} holds.
In this case, $\coords$ must be non-empty since otherwise $\points$ would be reduced to $\{0\}$ (see the first inclusion), violating the definition \eqref{eq:polyhedron} of $\points$.
In addition, there is some $\coord \in \actcoords \setminus \coords$ such that $\coef_\coord > 0$ and thus we have
\begin{align}
 \forall \point \in \points,\,
\point_\coord \leq \frac{\max\left(|\coef_\coordalt|: \coordalt \in \coords\right)}{\coef_\coord}\max(\point_\coordalt : \coordalt \in \coords)\,
\end{align}
which corresponds to the first case of the lemma.
   
   
\para{\cref*{itm:active}}
%Otherwise, 
For all $(\coef_\coord)_{\coord \in \actcoords\setminus\coords} \in (\R_{+})^{\actcoords\setminus \coords}$ \emph{not all zero} and $(\coefalt_\coord)_{\coord \in \coords} \in \R^{\coords}$, \eqref{eq:inclusion} does not hold.
To interpret this situation, we use the fact that $\points$ is of the general polyhedral form \eqref{eq:polyhedron} so $\aff \points = \sol + \ker \mat$  and $\sol$ always satisfies $\sum_{\coord \in \actcoords \setminus\coords} \coef_\coord \sol_\coord =\sum_{\coord \in \coords}\coefalt_\coord \sol_\coord = 0$ so that
\begin{align}
%\points \subset \left\{\point \in \R^\nCoords : \sum_{\coord \in \actcoords \setminus\coords} \coef_\coord \point_\coord =\sum_{\coord \in \coords}\coefalt_\coord \point_\coord \right\}
\eqref{eq:inclusion}
	&\iff \aff \points \subset\!\left\{\point \in \R^\nCoords\!:\!\!\sum_{\coord \in \actcoords \setminus\coords} \coef_\coord \point_\coord =\sum_{\coord \in \coords}\coefalt_\coord \point_\coord \right\}
	\notag\\
	&\iff \ker \mat \subset\!\left\{\point \in \R^\nCoords\!:\!\!\sum_{\coord \in \actcoords \setminus\coords} \coef_\coord \point_\coord =\sum_{\coord \in \coords}\coefalt_\coord \point_\coord \right\}
	\notag\\
	&\iff \sum_{\coord \in \actcoords \setminus \coords} \coef_\coord \bvec_\coord - \sum_{\coord \in \coords} \coefalt_\coord \bvec_\coord \in \row(\mat).
%		= \im \mat^{\top}.
%&\iff \forall s \in \ker \mat,  \sum_{\coord \in \actcoords \setminus \coords} \coef_\coord s_\coord - \sum_{\coord \in \coords} \coefalt_\coord s_\coord = 0.
\end{align}
Therefore, the fact that  \eqref{eq:inclusion} does not hold for all $(\coef_\coord)_{\coord \in \actcoords\setminus\coords} \in (\R_{+})^{\actcoords\setminus \coords}$ \emph{not all zero} and $(\coefalt_\coord)_{\coord \in \coords} \in \R^{\coords}$ means that,
the system,
\begin{align}
\sum_{\coord \in \actcoords \setminus \coords} \coef_\coord \bvec_\coord = \sum_{\coord \in \coords} \coefalt_\coord \bvec_\coord + \mat^{\top} \pvecalt\,,
\end{align}
with variables
$
(\coef_\coord)_{\coord \in \actcoords \setminus \coords} \in (\R_{+})^{\actcoords \setminus \coords} \text{not all zero, } (\coefalt_\coord)_{\coord \in\coords} \in \R^{\coords}, \pvecalt \in \R^{\nConstr},
$ has no solution.
Hence, by Motzkin's theorem on the alternative (see \eg \cite[\S1.4.2]{BN01})\footnote{With the notations of \cite[\S1.4.2]{BN01}, the lines of the matrix $S$ are made of the $\bvec_\coord$ for $\coord \in \actcoords \setminus \coords$ and the lines of the matrix $N$ are the $\bvec_\coord$ for $\coord \in \coords$, $-\bvec_\coord$ for $\coord \in \coords$, the lines of $\mat$ and their opposite.}, this means that the system
\begin{equation}
\begin{cases}
	\pvec_{\coord} > 0
		&\quad
		\text{for $\coord \in \actcoords \setminus \coords$}
		\\
	\pvec_{\coord} = 0
		&\quad
		\text{for $\coord \in \actcoords$}
		\\
	\mat\pvec = 0
\end{cases}
\end{equation}
admits a solution $\pvec \in \R^{\nCoords}$.
Rescaling %this solution 
$\pvec$ and setting $\polycst$ to $\max(\norm{\pvec}, \norm{\pvec}_\infty)$ then gives the second case.
\end{proof}



%----------------------------------------------------------------------
%%% APP: EX
%----------------------------------------------------------------------
\section{Omitted calculations}
\label{app:ex}
%----------------------------------------------------------------------
%%% APP: EX
%----------------------------------------------------------------------
% !TEX root = ./Main.tex
In this appendix, we provide some computational details that were left out of the main text to streamline our presentation.

\begin{example}[name=Hellinger distance,continues=ex:Hell]
We proceed to compute the Taylor expansion of $\fixmap$ near $\sol = -1$ for the shifted operator $\vecfield(\point) = \point+1$.
Indeed, in this case, the fixed point operator $\fixmap$ is given by
%, for $\point \in (-1, 1)$,
\begin{align}
\label{eq:app-fixmap}
\fixmap(\point)
	= \proxof{\point}{-\step \vecfield(\point)}
	&= \proxof{\point}{-\step (\point+1)}
	\notag\\
	&= \frac{\point - \step(\point+1)\sqrt{1-\point^{2}}}{\sqrt{1-\point^{2} + (\point -\step(\point+1)\sqrt{1-\point^{2}})^{2}}}
	\notag\\
	&= \frac{\fixmapalt(\point)}{\sqrt{1-\point^{2} +\fixmapalt(\point)^{2}}},
\end{align}
with $\fixmapalt(\point) = \point - \step(\point+1)\sqrt{1-\point^{2}}$.
Now, the behavior of $\fixmapalt$ near $\sol = -1$ can be approximated as
\begin{align}
\label{eq:app-fixmapalt}
\fixmapalt(\point)
	&=\point - \step(\point+1)^{3/2}(1-\point)^{1/2}
	\notag\\
	&= -1 + (\point+1) - \step(\point+1)^{3/2}(2 - (\point+1))^{1/2}
	\notag\\
%	&= -1 + (\point+1) - \sqrt 2 \step(\point+1)^{3/2}\left(1 - \tfrac{1}{2}\parens*{\point+1}\right)^{1/2}
%	\notag\\
	&= -1 + (\point+1) - \sqrt 2 \step(\point+1)^{3/2}\left(1 - \tfrac{1}{4}\parens*{\point+1} + o(\point+1)\right)
	\notag\\
	&= -1 + (\point+1) - \sqrt 2 \step(\point+1)^{3/2} + \tfrac{\sqrt 2 \step}{4}(\point+1)^{5/2} + o \parens*{(\point+1)^{5/2}}\,.
\end{align}
%Another Taylor expansion allows us to express $\fixmapalt(\point)^{2}$ as
Another Taylor expansion then yields
\begin{align}
\fixmapalt(\point)^{2}
%	&= (-\fixmap(\point))^{2}
%	\notag\\
	&= \parens*{1 - (\point+1) + \sqrt 2 \step(\point+1)^{3/2} - \tfrac{\sqrt 2 \step}{4}(\point+1)^{5/2} + o \parens*{(\point+1)^{5/2}}}^{2}
	\notag\\
	&= 1 - 2(\point+1) + 2\sqrt 2 \step(\point+1)^{3/2} + (\point+1)^{2} - \tfrac{\sqrt 2 \step}{2}(\point+1)^{5/2} + o \parens*{(\point+1)^{5/2}}
\end{align}
so the denominator of \cref{eq:app-fixmap} becomes
\begin{align}
\sqrt{1-\point^{2} +\fixmapalt(\point)^{2}}
	&= \parens*{(\point+1)(2 - (\point+1) + \fixmapalt(\point)^{2}}^{2}
	\notag\\
	&= \parens*{1 + 2\sqrt 2 \step(\point+1)^{3/2} - \tfrac{\sqrt 2 \step}{2}(\point+1)^{5/2} + o \parens*{(\point+1)^{5/2}}}^{2}
	\notag\\
	&= 1 + \sqrt 2 \step(\point+1)^{3/2} - \tfrac{\sqrt 2 \step}{4}(\point+1)^{5/2} + o \parens*{(\point+1)^{5/2}}\,.
\end{align}
Thus, plugging this expansion and \cref{eq:app-fixmapalt} into \cref{eq:app-fixmap} gives
\begin{align}
\fixmap(\point)
	&= \frac
		{-1 + (\point+1) - \sqrt 2 \step(\point+1)^{3/2} + \tfrac{\sqrt 2 \step}{4}(\point+1)^{5/2} + o \parens*{(\point+1)^{5/2}}}
	{1 + \sqrt 2 \step(\point+1)^{3/2} - \tfrac{\sqrt 2 \step}{4}(\point+1)^{5/2} + o \parens*{(\point+1)^{5/2}}}
	\notag\\
	&= \parens*{
 -1 + (\point+1) - \sqrt 2 \step(\point+1)^{3/2} + \tfrac{\sqrt 2 \step}{4}(\point+1)^{5/2} + o \parens*{(\point+1)^{5/2}}
 }
	\notag\\
	&\qquad\times \parens*{
1 - \sqrt 2 \step(\point+1)^{\half[3]} + \frac{\sqrt 2 \step}{4}(\point+1)^{\half[5]} + o \parens*{(\point+1)^{\half[5]}}
 }
	\notag\\
&=
		-1
		+ (\point+1)
		- 2\sqrt{2}\step (\point+1)^{5/2} + o\parens*{(\point+1)^{5/2}}\,,
\end{align}
which gives our assertion when $\sol = -1$.
\hfill
\endenv
\end{example}

\begin{example}[name=Three-dimensional simplex,continues=ex:simplex-2d]
We conclude our treatment of the simplex by showing that $\state_{2,\run} \sim \state_{2,\run} / \state_{3,\run} = \Omega(1/\run)$ if $\slack_{2} = 0$ but $\slack_{1} > 0$.
To begin with, we have $\vecfield_{2}(\curr) = \state_{2,\run} = o(1)$ so, arguing as in the first part of the example, we readily get
\begin{equation}
\frac{\state_{1,\run+1}}{\state_{2,\run+1}}
	= \frac{\state_{1,\run}}{\state_{2,\run}}
		\exp(-\step\slack_{1} + o(1)),
\end{equation}
so $\state_{1,\run} / \state_{2,\run}$ converges to $0$ at a geometric rate.
Accordingly, the quantity %of interest 
$\state_{3,\run} / \state_{2,\run}$ %can be 
is bounded as
\begin{align}
\frac{\state_{3,\run+1}}{\state_{2,\run+1}}
	&= \frac{\state_{3,\run}}{\state_{2,\run}}
		\exp \parens*{\step \vecfield_{2}(\curr) - \step \vecfield_{3}(\curr)}
	%\notag\\
	= \frac{\state_{3,\run}}{\state_{2,\run}}
		\exp \parens*{\step \state_{2,\run} - \step (\state_{3,\run} - 1)}
	\notag\\
    &= \frac{\state_{3,\run}}{\state_{2,\run}}
		\exp \parens*{2\step \state_{2,\run} + \step \state_{1,\run}}
	%\notag\\
	\leq
		\frac{\state_{3,\run}}{\state_{2,\run}}
		\exp \parens*{2\step \frac{\state_{2,\run}}{\state_{3,\run}} + \step \state_{1,\run}}
\end{align}
%Let us check that
%	\begin{align}
%	\frac{\state_{2,\run}}{\state_{3,\run}} =\Omega\parens*{\frac{1}{\run}}\,.
%	\end{align}
%
% First, using that $(\state_{1,\run})_{\run=\running}$ goes to zero geometrically, we actually show that the ratio $\frac{\state_{1,\run}}{\state_{2,\run}}$ still goes to zero as $\run$ goes to infinity.
%	Indeed, for $\run \geq \start$, 
% \begin{align}
%		\frac{\state_{1,\run+1}}{\state_{2,\run+1}}&=
%		\frac{\state_{1,\run}}{\state_{2,\run}}
%		\exp \parens*{
%			-\step \vecfield_{1}(\curr)
%			+ \step \vecfield_{2}(\curr)
%		} \\
%		&=
%		\frac{\state_{1,\run}}{\state_{2,\run}}
%		\exp \parens*{
%			-\step \slack_1
%			+o(1)
%		}
%		&&\text{ for } \run = \running
%	\end{align}
% since $\vecfield_{1}(\curr) = \state_{1,\run} + \slack_1 = \slack_1 + o(1)$ and $\vecfield_{2}(\curr) = \state_{2,\run} = o(1)$.
% Then, from this one deduces that $\frac{\state_{1,\run}}{\state_{2,\run}}$ goes to zero, and even geometrically fast.
% With this at hand, we can now come back to the quantity of interest $\frac{\state_{2,\run}}{\state_{3,\run}}$. The multiplicative weight update again gives us the recursion
%	\begin{align}
%	\frac{\state_{3,\run+1}}{\state_{2,\run+1}}
%		&=
%		\frac{\state_{3,\run}}{\state_{2,\run}}
%		\exp \parens*{
%			+ \step \vecfield_{2}(\curr)
%			- \step \vecfield_{3}(\curr)
%		} \\
%		&=
%		\frac{\state_{3,\run}}{\state_{2,\run}}
%		\exp \parens*{
%			+ \step \state_{2,\run}
%			- \step (\state_{3,\run} - 1)
%		} \\
%		&=
%		\frac{\state_{3,\run}}{\state_{2,\run}}
%		\exp \parens*{
%			+ 2\step \state_{2,\run}
%			+ \step \state_{1,\run}
%		} \\
%		&\leq
%		\frac{\state_{3,\run}}{\state_{2,\run}}
%		\exp \parens*{
%			+2\step \frac{\state_{2,\run}}{\state_{3,\run}}
%			+ \step \state_{1,\run} 
%		} 
%	\end{align}
%	where we used that $\state_{1,\run} + \state_{2,\run} + \state_{3,\run} = 1$.
Now, since both $\frac{\state_{2,\run}}{\state_{3,\run}}$ and $\state_{1,\run}$ go to zero, 
\begin{align}
\frac{\state_{3,\run+1}}{\state_{2,\run+1}}
	&\leq \frac{\state_{3,\run}}{\state_{2,\run}}
		\parens*{1
			+ 2\step \frac{\state_{3,\run}}{\state_{2,\run}}
			+ \step \state_{1,\run} 
			+ o\parens*{2\frac{\state_{3,\run}}{\state_{2,\run}}
			+ \state_{1,\run}}}
	\notag\\
%&=
%		\frac{\state_{3,\run}}{\state_{2,\run}}
%			+ 2\step
%			+ \frac{\step \state_{1,\run} \state_{3,\run}}{\state_{2,\run}}
%			+ o\parens*{2 + \frac{\step \state_{1,\run} \state_{3,\run}}{\state_{2,\run}}}
%	\notag\\
&=
		\frac{\state_{3,\run}}{\state_{2,\run}}
			+2\step
			+o \parens*{1}\,.
\end{align}
since $\state_{1,\run}/\state_{2,\run}$ vanishes as $\run\to\infty$.
Hence, after telescoping, we conclude that $\frac{\state_{3,\run}}{\state_{2,\run}}
	\leq 2 \step \run + o(\run),$ which in turn shows that $\state_{2,\run} \sim \state_{2,\run} / \state_{3,\run} = \Omega(1/\run)$, as claimed.
\hfill
\endenv
\end{example}

\WAdelete{
\begin{remark}[name=Connection to the Bregman exponent,continues=rem:examples]
We proceed to show here that, in \cref{asm:ker-power} of \cref{asm:ker}, the Bregman exponent at the solution satisfies $\legsol \leq (1 + \kernelexp)/2$.
Indeed, under our stated assumptions, $\hker$ is actually differentiable throughout $(0,\infty)$ so, whenever $0 \leq \pointalt \leq \point$, we have
\begin{equation}
\label{eq:bregexp}
\hker(\pointalt) - \hker(\point)
	= -\int_{\pointalt}^{\point} \hker'(u) \dd u.
\end{equation}
Moreover, thanks to \cref{asm:ker}\ref{asm:ker-power} with $\kernelexp < 1$, the integral,
	$%\begin{equation}
		\int_0^\point \hker'
	$%\end{equation}
	is well-defined.
Then, letting $\pointalt\to0^{+}$ in \eqref{eq:bregexp},
we get
%\begin{equation}
\(
\hker(0) - \hker(\point)
	\leq -\int_{0}^{\point} \hker'(u) \dd u.
\)
%\end{equation}
On the other hand, \cref{asm:ker}\ref{asm:ker-power} implies that both $-\int_0^\point \hker'(u) \dd u$ and $\hker'(\point)(0 -\point)$ are bounded by $\bigoh(\point^{1-\kernelexp})$ so
%\begin{equation}
\(
\hker(0) - \hker(\point) -\hker'(\point)(0 - \point)
	= \bigoh(\point^{1-\kernelexp}).
\)
%\end{equation}
Since $\acts \neq \varnothing$ and $\hker'$ is locally Lipschitz continuous, a similar argument as in the proof of \cref{lem:Leg-proxdom} ultimately yields
\begin{align}
\breg(\sol,\point)
	&= \sum_{\coord \in \actcoords} \bigoh\parens[\big]{\point_\coord^{1-\kernelexp}}
		+ \sum_{\coord \notin \actcoords} \bigoh\parens[\big]{(\sol_\coord-\point_\coord)^{2}}
%	\notag\\
	= \bigoh \parens*{\norm{\sol - \point}^{1 - \kernelexp}}
\end{align}
%where we used the equivalence between norms, \eg between the Euclidean norm and the norm $\norm{\cdot}$.
which shows that the Legendre exponent of $\hreg$ at $\sol$ is at most $(1+\kernelexp)/2$, as claimed.
\endenv
\end{remark}
}




%**********************************************************************
%***    BIBLIOGRAPHY
%**********************************************************************
\bibliographystyle{icml}
\bibliography{bibtex/IEEEabrv,bibtex/Bibliography-PM}


\end{document}