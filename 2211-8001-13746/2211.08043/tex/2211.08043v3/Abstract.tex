%----------------------------------------------------------------------
%%% ABSTRACT
%----------------------------------------------------------------------
% !TEX root = ./Main.tex
%
%
We examine the last-iterate convergence rate of \aclp{BPM} \textendash\ from \acl{MD} to \acl{MP} and its optimistic variants \textendash\ as a function of the local geometry induced by the prox-mapping defining the method.
For generality, we focus on local solutions of constrained, non-monotone \aclp{VI}, and we show that the convergence rate of a given method depends sharply on its associated \emph{Legendre exponent}, a notion that measures the growth rate of the underlying Bregman function (Euclidean, entropic, or other) near a solution.
In particular, we show that boundary solutions exhibit a stark separation of regimes between methods with a zero and non-zero Legendre exponent:
the former converge at a linear rate, while the latter converge, in general, sublinearly.
This dichotomy becomes even more pronounced in linearly constrained problems where methods with entropic regularization achieve a linear convergence rate along sharp directions, compared to convergence in a finite number of steps under Euclidean regularization.