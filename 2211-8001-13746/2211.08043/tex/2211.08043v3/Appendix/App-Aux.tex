%----------------------------------------------------------------------
%%% APP: AUX
%----------------------------------------------------------------------
% !TEX root = ../Main.tex
%
%
We provide here a series of basic properties, helper lemmas and auxiliary results that we use repeatedly in our paper.

%----------------------------------------------------------------------
%%% NUMERICAL SEQUENCES
%----------------------------------------------------------------------
\subsection{Lemmas on numerical sequences}
The first two results %we provide 
concern numerical sequences.

\begin{lemma}
%[\citeauthor{Pol87}, \citeyear{Pol87}, p.~46]
%[\citeauthor{Pol87}]
\label{lem:Polyak}
Consider two sequences of %non-negative 
real numbers $\curr[\seq], \curr[\diff] \geq 0$, $\run=\running$, such that
\begin{equation}
\next[\seq]
	\leq \curr[\seq]
		- \curr[\diff] \curr[\seq]^{1+\rexp}
	\quad
	\text{for some $\rexp>0$ and all $\run=\running$}
\end{equation}
Then, for all $\run=\running$, we have:%
%\PM{Shouldn't the sum go to $\run-1$ instead of $\run$?
%I changed the $\run$ to $\run+1$ preemptively, but please check\dots}
%\WA{Checked}
\begin{equation}
\next[\seq]
	\leq \frac{\init[\seq]}{\parens*{1 + \rexp \init[\seq]^{\rexp} \sum_{\runalt=\start}^{\run} \iter[\diff]}^{1/\rexp}}.
\end{equation}
\end{lemma}

\begin{proof}
See \cite[p.~46, Lem.~6]{Pol87}.
\end{proof}

The second result that we prove here is \cref{lem:basicnum}, a slight variant of the above lemma, which we restate below for convenience.

\basicnum*

%\begin{lemma}
%\label{lem:basicnum-app}
%Suppose that $\fn\from\R_+\to\R_+$ admits the asymptotic expansion
%\begin{equation}
%\fn(\point)
%	= \point
%		- \coef\point^{1+\rexp}
%		+ o(\point^{1+\rexp})
%	\quad
%	\text{as $\point\to0$}
%\end{equation}
%for positive constants $\coef,\rexp>0$.
%Then, for $\init[\seq] > 0$ small enough, the sequence $\next[\seq] = \fn(\curr[\seq])$, $\run=\running$, converges to $0$ at a rate of $\curr[\seq] \sim (\coef\rexp\run)^{-1/\rexp}$. % for some $\const>0$.
%\end{lemma}

\begin{proof}
By the assumption on $\fn$, there exists some $\eps > 0$ such that
%for $\point \in [0, \eps]$,
\begin{equation}
\point - 2\coef \point^{1+\rexp}
	\leq \fn(\point)
	\leq \point - \half[\coef] \point^{1+\rexp}
	\quad
	\text{for all $\point \in [0,\eps]$}.
\end{equation}
%As a first consequence, 
Note first that, if $\init[\seq] \leq \eps$, \cref{lem:Polyak} readily implies that $\curr[\seq]$ converges to $0$ and that $\curr[\seq] \leq \eps$ for all\;$\run$.
%\PM{A minor point, but how does \cref{lem:Polyak} imply anything about positivity?
%Should the order of the two phrases here be inverted?}
%\WA{Indeed, the non-negativeness comes by definition of the sequence, I removed it}
Moreover, if $\eps$ is small enough so that $1 -2\coef \eps^\rexp > 0$ and $\init[\seq]$ is positive, this implies that all $\curr[\seq]$, for $\run = \running$, are positive.
In particular, %it is then valid to 
we consider the sequence $\curr[\seq]^{-\rexp}$, $\run = \running$, for which we\;get
\begin{align}
\next[\seq]^{-\rexp} - \curr[\seq]^{-\rexp}
	&= \bracks[\big]{\curr[\seq] - \coef \curr[\seq]^{1+\rexp} + o\parens[\big]{\curr[\seq]^{1+\rexp}}}^{-\rexp}
		- \curr[\seq]^{-\rexp}
	\notag\\
	&= \curr[\seq]^{-\rexp}(1 - \coef \curr[\seq]^{\rexp} + o\left(\curr[\seq]^{\rexp}\right))^{-\rexp}
		- \curr[\seq]^{-\rexp}
%	\notag\\
	= \rexp \coef + o\left(1\right)\,.
\end{align}
Hence, $\curr[\seq]^{-\rexp} \sim \rexp \coef \run$ which gives the result.
\end{proof}


%----------------------------------------------------------------------
%%% BREGMAN DIV
%----------------------------------------------------------------------
\subsection{Properties of Bregman divergences and the induced prox-mappings}
We recall some basic properties of the Bregman divergence and the induced prox-mapping.
Variants of these properties are fairly well known in the literature, so we omit their proofs and we refer the interested reader to \cite{BecTeb03,JNT11,MZ19,MLZF+19,DMSV23,MHC24} and references therein for a more detailed discussion.
In particular, \cref{lem:mirror,lem:threepoint,lem:proxlip} correspond to Lemmas B.1, B.2 and B.4(a) of \cite{MLZF+19}, respectively.

\begin{restatable}{lemma}{mirror}
\label{lem:mirror}
Let $\hreg$ be a Bregman regularizer on $\points$ and let $\subsel\hreg$ be a continuous selection of $\subd\hreg$.
Then, for all $\point\in\proxdom$, $\new\in\points$ and $\dvec\in\dpoints$, we have:
\begin{subequations}
\begin{flalign}
\quad
	a)\;\;
	&\subd\hreg(\point)
	= \subsel\hreg(\point)
		+ \pcone(\point)
	\\
\quad
	b)\;\;
	&\new
	= \proxof{\point}{\dvec}
	\iff
\subsel\hreg(\point) + \dvec
	\in \subd\hreg(\new)
	\iff
\subsel\hreg(\new) - \subsel\hreg(\point)
	\in \dvec - \pcone(\new)
\end{flalign}
\end{subequations}
where $\pcone(\point) = \setdef{\dvec\in\dpoints}{\braket{\dvec}{\pointalt - \point} \leq 0 \; \text{for all $\pointalt\in\points$}}$ denotes the polar cone to $\points$ at $\point$.
%In particular, \eqref{eq:AMP} is well-posed:
%$\new = \proxof{\point}{\dvec}$ implies that $\new\in\proxdom$.
\end{restatable}

\begin{restatable}[$3$-point identity]{lemma}{threepoint}
\label{lem:threepoint}
%Let $\hreg$ be a Bregman regularizer on $\points$.
For all $\base\in\points$ and all $\point,\new\in\proxdom$, we have:
\begin{equation}
\label{eq:threepoint}
\breg(\base,\new)
	= \breg(\base,\point)
		+ \breg(\point,\new)
		+ \braket{\subsel\hreg(\new) - \subsel\hreg(\point)}{\point - \base}
\end{equation}
\end{restatable}

\begin{restatable}[Non-expansiveness]{lemma}{proxlip}
\label{lem:proxlip}
For all $\point\in\proxdom$ and all $\dvec,\new[\dvec]\in\dpoints$ we have:
\begin{equation}
	\norm{\proxof{\point}{\new[\dvec]} - \proxof{\point}{\dvec}}
	\leq \dnorm{\new[\dvec] - \dvec}
\end{equation}
\end{restatable}


The next two results that we provide consider the evolution of the Bregman divergence before and after a prox step (or two);
they are both adapted from \cite[Proposition B.3]{MLZF+19}, with the added proviso that $\hreg$ is assumed to be $1$-strongly convex on $\nhdalt$ (as per \cref{rem:Bregman}).

\begin{lemma}
\label{lem:onestep-app}
Let $\new = \proxof{\point}{\dvec}$ for $\point\in\proxdom$, $\dvec\in\dpoints$ such that $\new$ is still in $\nhdalt$.
Then, for all $\base\in\points$, %and all 
$\dbase\in\pcone(\base)$, we have:\!\!
\begin{subequations}
\begin{align}
\breg(\base,\new)
	&\leq \breg(\base,\point)
		+ \braket{\dvec - \dbase}{\new - \base}
		- \breg(\new,\point)
		\\
	&\leq \breg(\base,\point)
		+ \braket{\dvec - \dbase}{\point - \base}
		+ \tfrac{1}{2} \dnorm{\dvec - \dbase}^{2}\,.
\end{align}
\end{subequations}
\end{lemma}

\begin{proof}
Our proof follows \cite[Proposition B.3]{MLZF+19}, but with a slight modification to account for the extra term %involving 
with $\dbase\in\pcone(\base)$.
The first step is to invoke the three-point identity \eqref{eq:threepoint} to write
\begin{equation}
\breg(\base,\point)
	= \breg(\base, \new)
		+ \breg(\new, \point)
		+ \braket{\subsel\hreg(\point) - \subsel\hreg(\new)}{\new - \base}.
\end{equation}
Then, after rearranging to isolate $\breg(\base,\new)$, we get
\begin{align}
\breg(\base,\new)
	&= \breg(\base, \point)
		- \breg(\new, \point)
		- \braket{\subsel\hreg(\point) - \subsel\hreg(\new)}{\new - \base}
	\notag\\
	&\leq \breg(\base,\point)
		- \breg(\new, \point)
		+ \braket{\dvec}{\new - \base}
\end{align}
where the inequality in the last line follows from \cref{lem:mirror}.
Hence, given that $\braket{\dbase}{\new - \base} \leq 0$ by the fact that $\dbase\in\pcone(\base)$, we readily obtain
\begin{equation}
\label{eq:onestep-proof}
\breg(\base, \new)
	\leq  \breg(\base, \point)
		- \breg(\new, \point)
		+ \braket{\dvec - \dbase}{\new - \base}\,.
\end{equation}

For the second inequality of the lemma, note that 
\begin{align}
\braket{\dvec - \dbase}{\new - \base}
	&= \braket{\dvec - \dbase}{\point - \base}
		+ \braket{\dvec - \dbase}{\new - \point}
	\notag\\
	&\leq \braket{\dvec - \dbase}{\point - \base}
		+ \tfrac{1}{2} \dnorm{\dvec - \dbase}^{2}
		+ \tfrac{1}{2} \norm{\new - \point}^{2}
	\notag\\
	&\leq \braket{\dvec - \dbase}{\point - \base}
		+ \tfrac{1}{2} \dnorm{\dvec - \dbase}^{2}
		+ \breg(\new,\point)
\end{align}
where the penultimate inequality follows directly from Young's inequality and the last one from \eqref{eq:Breg-lower} and the fact that $\point$ and $\new$ both lie in $\nhdalt$.
Our assertion is then obtained by combining this last bound with \eqref{eq:onestep-proof}.
\end{proof}

\begin{lemma}
\label{lem:twostep-app}
Let $\new_{i} = \proxof{\point}{\dvec_{i}}$ for some $\point\in \nhdalt \cap \proxdom$ and $\dvec_{i}\in\dpoints$ such that $\new_i \in \nhdalt$, $i=1,2$.
Then, for all $\base\in\points$ and all $\dbase\in\pcone(\base)$, we have:
\begin{equation}
\breg(\base,\new_{2})
	\leq \breg(\base,\point)
		+ \braket{\dvec_{2} - \dbase}{\new_{1} - \base}
		+ \tfrac{1}{2} \dnorm{\dvec_{2} - \dvec_{1} - \dbase}^{2}
		- \tfrac{1}{2} \norm{\new_{1} - \point}^{2}.
\end{equation}
\end{lemma}

\begin{proof}
Our proof follows \citep[Proposotion B.4]{MLZF+19}, again with a slight modification to account for the extra terms with %involving 
$\dbase\in\pcone(\base)$.
Specifically, applying \cref{lem:onestep} with $\new_{2} = \mprox_\point(\dpoint_{2})$ and $\dbase \in \pcone(\base)$ gives
\begin{align}
\label{eq:twostep-proof}
\breg(\base,\new_{2})
	&\leq \breg(\base,\point)
		+ \braket{\dpoint_{2} - \dbase}{\new_{2} - \base}
		- \breg(\new_{2}, \point)
	\notag\\
	&\leq \breg(\base, \point)
		+ \braket{\dpoint_{2} - \dbase}{\new_{1} - \base}
		+ \braket{\dpoint_{2} - \dbase}{\new_{2} - \new_{1}}
		- \breg(\new_{2},\point)
\end{align}
To lower bound $\breg(\new_{2}, \point)$, we use again \cref{lem:onestep} with $\base \gets \new_{2}$ and $\new_{1} = \mprox_\point(\dpoint_{1})$.
This readily gives
\begin{equation}
\breg(\new_{2},\new_{1})
	\leq \breg(\new_{2},\point)
		+ \braket{\dpoint_{1}}{\new_{1} - \new_{2}}
		- \breg(\new_{1},\point)
\end{equation}
and hence,
%\begin{equation}
%\breg(\new_{2},\point)
%	\geq \breg(\new_{2},\new_{1})
%		+ \braket{\dpoint_{1}}{\new_{2} - \new_{1}}
%		+ \breg(\new_{1},\point).
%\end{equation}
after rearranging the above to isolate $\breg(\new_{2},\point)$ and substituting the resulting bound in \eqref{eq:twostep-proof}, we get
\begin{equation}
\breg(\base,\new_{2})
	\leq \breg(\base,\point)
		+ \braket{\dpoint_{2} - \dbase}{\new_{1} - \base}
		+ \braket{\dpoint_{2} - \dpoint_{1} - \dbase}{\new_{2} - \new_{1}} 
		- \breg(\new_{2},\new_{1})
		- \breg(\new_{1}, \point).
\end{equation}
Thus, by Young's inequality and the local strong convexity of $\hreg$, we finally obtain
%To complete the proof, we use Young's inequality and twice the strong convexity of $\hreg$ \cref{lem: strg cvx h} to get,
\begin{align}
\breg(\base,\new_{2})
	&\leq \breg(\base, \point)
		+ \braket{\dpoint_{2} - \dbase}{\new_{1} - \base}
		+ \tfrac{1}{2}\dnorm{\dpoint_{2} - \dpoint_{1} - \dbase}^{2}
	\notag\\
	&\qquad
		+ \tfrac{1}{2}\norm{\new_{2} - \new_{1}}^{2}
		- \tfrac{1}{2}\norm{\new_{2}-\new_{1}}^{2}
		- \tfrac{1}{2}\norm{\new_{1}-\point}^{2}
	\notag\\
	&\leq \breg(\base,\point)
		+ \braket{\dpoint_{2} - \dbase}{\new_{1} - \base}
			+ \tfrac{1}{2}\dnorm{\dpoint_{2} - \dpoint_{1} - \dbase}^{2}
			- \tfrac{1}{2}\norm{\new_{1}-\point}^{2}
\end{align}
and our proof is complete.
\end{proof}





%----------------------------------------------------------------------
%% SEPARATION
%----------------------------------------------------------------------
\subsection{Legendre exponent for interior points}

We now proceed to provide a more formal footint to our discussion in \Cref{sec:Legendre} regarding the fact that $\legof{\base}=0$ whenever $\base$ is an interior point.
The formal statement is as follows.

\begin{lemma}
\label{lem:Leg-proxdom}
Suppose that $\subsel\hreg$ is locally Lipschitz continuous.
Then $\legof{\base} = 0$ for all $\base \in \proxdom$;
in particular, $\legof{\base} = 0$ whenever $\base\in\relint\points$.
\end{lemma}

\begin{proof}
Fix some $\base \in \proxdom$ and suppose that $\subsel\hreg$ is locally Lipschitz continuous.
%selection of $\subd\hreg$.
Then there exists a neighborhood $\legnhd$ of $\base$ in $\points$ and some $\legconst > 0$ such that
\begin{equation}
\label{eq:hLip}
\dnorm{\nabla \hreg(\base) - \nabla \hreg(\point)}
	\leq \legconst \norm{\base - \point}
	\quad
	\text{for all $\point \in \legnhd \cap \proxdom$}.
\end{equation}
Now, since $\nabla \hreg(\base) \in \subd \hreg(\base)$,
we also have
\begin{align}
\breg(\base,\point)
	= \hreg(\base) - \hreg(\point) - \braket{\nabla \hreg(\point)}{\base - \point}
%	\notag\\
	&\leq \braket{\nabla \hreg(\base) - \nabla \hreg(\point)}{\base - \point}
	\notag\\
	&\leq \dnorm{\nabla \hreg(\base) - \nabla \hreg(\point)} \norm{\base - \point}
%	\notag\\
	\leq \legconst \norm{\base - \point}^{2}
\end{align}
for all $\point\in\legnhd \cap \proxdom$.
This shows that \eqref{eq:Breg-local} holds with $\legexp=0$, \ie $\legof{\base}=0$.
\end{proof}




%----------------------------------------------------------------------
%%% SEPARATION
%----------------------------------------------------------------------
\subsection{A separation result}

We now proceed to prove \cref{lem:separation}, which we restate below for convenience:

\separation*

%\begin{lemma}
%\label{lem:separation-app}
%%Suppose that $\points$ is of the general polyhedral form \eqref{eq:polyhedron}.
%Let $\points$ be a polyhedral domain of the %general 
%form \eqref{eq:polyhedron}.
%Then, for all $\sol\in\points$, there exists $\polycst = \polycst(\mat, \cvec, \sol) \geq 1$ such that, for all $\coords \subseteq \actcoords \equiv \actcoords(\sol)$, at least one of the following holds:
%\begin{enumerate}
%[(\itshape a\upshape)]
%\item
%\label[case]{itm:inactive}
%$\coords \neq \varnothing$ and there exists $\coord \in \actcoords\setminus\coords$ such that
%%\begin{align}
%%\forall \point \in \points,\,
%\(
%\point_{\coord}
%	\leq \polycst \max\setdef{\point_{\coordalt}}{\coordalt \in \coords}
%%	\quad
%%	\text{for all $\point\in\points$}.
%%\end{align}
%\)
%for all $\point\in\points$.
%\item
%\label[case]{itm:active}
%There exists $\pvec \in \ker\mat$ such that $\norm{\pvec} \leq \polycst$, $\pvec_{\coord} = 0$ if $\coord \in \coords$ and $\polycst \geq \pvec_{\coord} \geq 1$ if $\coord \in \actcoords \setminus \coords$.
%\end{enumerate}
%\end{lemma}


\begin{proof}
Our claim is trivial if $\coords = \actcoords$, so we will focus exclusively on the case $\coords \subsetneq \actcoords$.
The stated constant $\polycst = \polycst(\mat, \cvec, \sol)$ will then be obtained as the maximum of $1$ and the constants we obtain for each possible $\coords \subsetneq \actcoords$.
 
The proof consists in discussing whether there exists 
 $(\coef_\coord)_{\coord \in \actcoords\setminus\coords} \in (\R_{+})^{\actcoords\setminus \coords}$ \emph{not all zero} and $(\coefalt_\coord)_{\coord \in \coords} \in \R^{\coords}$ such that the inclusion
%\PM{I do not understand this paragraph (and the implications in the rest of the proof are not very clear either). Didn't touch.}
%\WA{I tried to rewrite a bit, let me know if it's clearer...}
\begin{align}
\points
	\subseteq \left\{\point \in \R^\nCoords : \sum_{\coord \in \actcoords \setminus\coords} \coef_\coord \point_\coord =\sum_{\coord \in \coords}\coefalt_\coord \point_\coord \right\} \label{eq:inclusion}
\end{align}
holds.
\Cref*{itm:inactive} considers the case when such coefficients exist, while \cref*{itm:active} considers when this is not possible.


\para{\cref*{itm:inactive}}
Assume that there exists $(\coef_\coord)_{\coord \in \actcoords\setminus\coords} \in (\R_{+})^{\actcoords\setminus \coords}$ \emph{not all zero} and $(\coefalt_\coord)_{\coord \in \coords} \in \R^{\coords}$ such that \eqref{eq:inclusion} holds.
In this case, $\coords$ must be non-empty since otherwise $\points$ would be reduced to $\{0\}$ (see the first inclusion), violating the definition \eqref{eq:polyhedron} of $\points$.
In addition, there is some $\coord \in \actcoords \setminus \coords$ such that $\coef_\coord > 0$ and thus we have
\begin{align}
 \forall \point \in \points,\,
\point_\coord \leq \frac{\max\left(|\coef_\coordalt|: \coordalt \in \coords\right)}{\coef_\coord}\max(\point_\coordalt : \coordalt \in \coords)\,
\end{align}
which corresponds to the first case of the lemma.
   
   
\para{\cref*{itm:active}}
%Otherwise, 
For all $(\coef_\coord)_{\coord \in \actcoords\setminus\coords} \in (\R_{+})^{\actcoords\setminus \coords}$ \emph{not all zero} and $(\coefalt_\coord)_{\coord \in \coords} \in \R^{\coords}$, \eqref{eq:inclusion} does not hold.
To interpret this situation, we use the fact that $\points$ is of the general polyhedral form \eqref{eq:polyhedron} so $\aff \points = \sol + \ker \mat$  and $\sol$ always satisfies $\sum_{\coord \in \actcoords \setminus\coords} \coef_\coord \sol_\coord =\sum_{\coord \in \coords}\coefalt_\coord \sol_\coord = 0$ so that
\begin{align}
%\points \subset \left\{\point \in \R^\nCoords : \sum_{\coord \in \actcoords \setminus\coords} \coef_\coord \point_\coord =\sum_{\coord \in \coords}\coefalt_\coord \point_\coord \right\}
\eqref{eq:inclusion}
	&\iff \aff \points \subset\!\left\{\point \in \R^\nCoords\!:\!\!\sum_{\coord \in \actcoords \setminus\coords} \coef_\coord \point_\coord =\sum_{\coord \in \coords}\coefalt_\coord \point_\coord \right\}
	\notag\\
	&\iff \ker \mat \subset\!\left\{\point \in \R^\nCoords\!:\!\!\sum_{\coord \in \actcoords \setminus\coords} \coef_\coord \point_\coord =\sum_{\coord \in \coords}\coefalt_\coord \point_\coord \right\}
	\notag\\
	&\iff \sum_{\coord \in \actcoords \setminus \coords} \coef_\coord \bvec_\coord - \sum_{\coord \in \coords} \coefalt_\coord \bvec_\coord \in \row(\mat).
%		= \im \mat^{\top}.
%&\iff \forall s \in \ker \mat,  \sum_{\coord \in \actcoords \setminus \coords} \coef_\coord s_\coord - \sum_{\coord \in \coords} \coefalt_\coord s_\coord = 0.
\end{align}
Therefore, the fact that  \eqref{eq:inclusion} does not hold for all $(\coef_\coord)_{\coord \in \actcoords\setminus\coords} \in (\R_{+})^{\actcoords\setminus \coords}$ \emph{not all zero} and $(\coefalt_\coord)_{\coord \in \coords} \in \R^{\coords}$ means that,
the system,
\begin{align}
\sum_{\coord \in \actcoords \setminus \coords} \coef_\coord \bvec_\coord = \sum_{\coord \in \coords} \coefalt_\coord \bvec_\coord + \mat^{\top} \pvecalt\,,
\end{align}
with variables
$
(\coef_\coord)_{\coord \in \actcoords \setminus \coords} \in (\R_{+})^{\actcoords \setminus \coords} \text{not all zero, } (\coefalt_\coord)_{\coord \in\coords} \in \R^{\coords}, \pvecalt \in \R^{\nConstr},
$ has no solution.
Hence, by Motzkin's theorem on the alternative (see \eg \cite[\S1.4.2]{BN01})\footnote{With the notations of \cite[\S1.4.2]{BN01}, the lines of the matrix $S$ are made of the $\bvec_\coord$ for $\coord \in \actcoords \setminus \coords$ and the lines of the matrix $N$ are the $\bvec_\coord$ for $\coord \in \coords$, $-\bvec_\coord$ for $\coord \in \coords$, the lines of $\mat$ and their opposite.}, this means that the system
\begin{equation}
\begin{cases}
	\pvec_{\coord} > 0
		&\quad
		\text{for $\coord \in \actcoords \setminus \coords$}
		\\
	\pvec_{\coord} = 0
		&\quad
		\text{for $\coord \in \actcoords$}
		\\
	\mat\pvec = 0
\end{cases}
\end{equation}
admits a solution $\pvec \in \R^{\nCoords}$.
Rescaling %this solution 
$\pvec$ and setting $\polycst$ to $\max(\norm{\pvec}, \norm{\pvec}_\infty)$ then gives the second case.
\end{proof}
