%----------------------------------------------------------------------
%%% DISCUSSION
%----------------------------------------------------------------------
% !TEX root = ../Main.tex


Our results indicate that Euclidean regularization leads to faster trajectory convergence rates near \ac{SOS} solutions.
While this does not contradict the analysis of \cite{Nem04} \textendash\ which concerns the method's ergodic average and advocates the use of non-Euclidean regularizers in domains with a favorable geometry \textendash\ it \emph{does} run contrary to its spirit.
We attribute the source of this discrepancy
%(at least in the non-sharp case) 
to the fact that Lipschitz continuity and second-order sufficiency are both norm-based conditions, so it is plausible to expect that norm-based regularizers would lead to better results.
This raises the question of what the corresponding rate analysis would give in the case of Bregman-based variants of \eqref{eq:Lipschitz} and \eqref{eq:strong}, \eg as in the recent works of \cite{BDX11,BBT17,LFN18,ABM19,ABM20,AM21,ABM21}.
We defer this analysis to future work.